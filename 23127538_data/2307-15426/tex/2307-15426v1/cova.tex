%Format: LaTeX  
%%% Text vom 19.05.2022
\documentclass[pagesize,12pt,a4paper]{scrartcl}
%\documentclass[12pt,a4paper]{book}
%\usepackage{scrpage2}
%\usepackage{german}
\usepackage{amsmath}
\usepackage{amsthm}
\usepackage{amssymb}
\usepackage{amsxtra}
\usepackage[squaren]{SIunits}
\usepackage{graphicx}
\usepackage{braket}
%\usepackage{yhmath}
\usepackage{capt-of}
\usepackage{wasysym}
%\usepackage{showkeys}
%\usepackage{rotating}
\usepackage{makeidx}
\usepackage{inputenc}
\usepackage[T1]{fontenc}
\usepackage{url}
\usepackage{eepic}
\usepackage{mathrsfs}
\usepackage{comment}
%\usepackage{microtype}
%\usepackage{ellipsis}
\usepackage{mparhack}
%\usepackage{eurofont}
%\usepackage[activate]{pdfcprot}
%\usepackage{showkeys}
%\usepackage[bookmarks]{hyperref}
\hyphenation{time-di-la-tion equi-tem-po-ral know-ledge pro-duct equi-va-ri-ant every-where re-pre-sen-ta-tion re-pre-sen-ta-tions 
va-cu-um se-pa-ra-tions se-pa-ra-tion ste-reo-gra-phic pro-ducts pro-duct re-pre-sent Ha-mil-tonian de-nu-mer-able li-ne-ar
ne-ver-the-less par-ti-cu-lar he-li-ci-ty or-tho-go-nal ope-ra-tors ope-ra-tor gen-er-a-tors do-mi-nant trans-ver-sal to-po-lo-gy
four-mo-men-tum four-ve-lo-ci-ty}
\def\out{\text{out}}
\def\inn{\text{in}}
\newcommand{\A}{\mathtt E} 
\newcommand{\C}{\mathtt C}
\newcommand{\K}{\mathtt K} 
\newcommand{\ir}{\mathrm{i}}
\newcommand{\e}{\mathrm{e}}
\newcommand{\evol}{\mathfrak{e}}
\newcommand{\eins}{{\mathbf 1}}
\newcommand{\g}{\mathfrak{g}}
\newcommand{\phys}{\text{phys}}
\renewcommand{\jmath}{j}
\newcommand{\longpage}{\enlargethispage{1\baselineskip}}
\newcommand*\idx[1]{\index{#1}#1}
\providecommand{\dual}{\shortmid}
\providecommand{\abs}[1]{\lvert#1\rvert}
\providecommand{\norm}[1]{\lVert#1\rVert}
\newcommand{\clearemptypage}{\newpage{\pagestyle{empty}\clearpage}}
\DeclareMathOperator*{\ran}{Ran\,}
\DeclareMathOperator{\modulo}{mod\,}
\DeclareMathOperator{\s}{\mathrm s}
\DeclareMathOperator{\bv}{\mathrm b}
\DeclareMathOperator{\vol}{vol}
\DeclareMathOperator{\gh}{gh}
\DeclareMathOperator{\ch}{ch}
\DeclareMathOperator{\sh}{sh}
\DeclareMathOperator{\tr}{tr}
\DeclareMathOperator{\im}{im}
\DeclareMathOperator{\str}{str}
\DeclareMathOperator{\sign}{sgn}
\DeclareMathOperator{\artanh}{artanh}
\DeclareMathOperator{\arccot}{arccot}
\DeclareMathOperator{\rot}{rot}
\DeclareMathOperator{\grad}{grad}
\DeclareMathOperator{\divper}{div}
\DeclareMathOperator{\Ad}{Ad}
\DeclareMathOperator{\ad}{ad}
\DeclareMathOperator{\ind}{ind}
\DeclareMathOperator{\dv}{d}
\DeclareMathOperator{\d3x}{\!d\!^{\,3}\! x\,}
\DeclareMathOperator{\dfx}{\!d\!^{\,4}\! x\,}
\DeclareMathOperator{\id}{id}
\DeclareMathOperator{\plus}{\hat +}
\DeclareMathOperator{\T}{T}
\DeclareMathOperator{\N}{N}
\DeclareMathOperator{\supp}{supp}
\DeclareMathOperator{\dt}{\tilde d\!}
\DeclareMathOperator*{\Sym}{Sym}
\DeclareMathOperator*{\slim}{s-lim\,}
\DeclareMathOperator*{\diag}{diag}
\ifx\KOMAScript\undefined%
  \DeclareRobustCommand{\KOMAScript}{\textsf{K\kern.05em O\kern.05em%
      M\kern.05em A\kern.1em-\kern.1em Script}}
\fi
\newlength{\help}
\setlength{\help}{\textwidth}
\addtolength{\help}{-3in}
\newlength{\minuslaenge}
\settowidth{\minuslaenge}{$-$}
\setcounter{tocdepth}{2}
\setcounter{secnumdepth}{1}

\newtheoremstyle{note}% name
  {3pt}%      Space above
  {3pt}%      Space below
  {\rmshape}%         Body font
  {}%         Indent amount (empty = no indent, \parindent = para indent)
  {\bfseries}% Thm head font
  {:}%        Punctuation after thm head
  {.5em}%     Space after thm head: " " = normal interword space;
        %       \newline = linebreak
  {}%         Thm head spec (can be left empty, meaning `normal')

\theoremstyle{note}


\makeatletter

 \def\vec#1{\ensuremath{\mathchoice
                     {\mbox{\boldmath$\displaystyle\mathbf{#1}$}}
                     {\mbox{\boldmath$\textstyle\mathbf{#1}$}}
                     {\mbox{\boldmath$\scriptstyle\mathbf{#1}$}}
                     {\mbox{\boldmath$\scriptscriptstyle\mathbf{#1}$}}}}%
\makeatother




%\setlength{\unitlength}{1mm}
%%%%%%%
%\selectlanguage{english}
%\special{papersize=210mm,297mm hsize=596 vsize=842}

\begin{document}


  \title{Relativistic Covariance of Scattering}
  \author{Norbert Dragon\\
          Institut f\"ur Theoretische Physik\\
          Leibniz Universit\"at Hannover 
%\\
%orcid 0000-0002-3809-524X
}
\date{}

\maketitle

\begin{abstract} 

We analyze relativistic quantum scattering in the Schr\"odinger picture. The suggestive requirement
of translational invariance and conservation of the four-mo\-men\-tum, that the interacting Hamiltonian
commute with the four-momentum $P$ of free particles, is shown to imply the absence of interactions.

The relaxed requirement, that the interacting Hamiltonian $H'$ commute with the four-velocity $U= P/M$, $M=\sqrt{P^2}$,
allows Poincar\'e covariant interactions just as in the nonrelativistic case.
If the $S$-matrix is Lorentz invariant, it still commutes with the four-momentum $P$ though $H'$  does not.



Shifted observers, whose translations are generated by the four-velocity~$U$, just see a shifted superposition
of near-mass-degenerate states with unchanged relative phases, while the four-mo\-men\-tum generates 
oscillated superpositions with changed relative phases.

\end{abstract}

\newpage

%\mainmatter

\section{Introduction}

Despite the phenomenal agreement of the standard model with observed physics, 
the mathematical existence of relativistic scattering is still unknown.

In the quantum case and in the classical case relativistic scattering seems excluded by Haag's \cite{haag} and Leutwyler's \cite{leutwyler}  no-go theorems. 

In textbooks \cite{weinberg} one finds requirements for an interacting representation of Lorentz transformations 
but not their solution nor any proof of existence.

Assuming that one can switch on the interaction with a function $g:\mathbb R^4\mapsto [0,1]$, such that
$g(x)=0$ in a neighbourhood of $x$  means no interaction and $g(x)=1$ fully switched on interaction, 
and that the $S$-matrix is a series in $g$,  
Bogoliubov \cite{bogoliubov} shows that each unitary, perturbative, relativistic and causal $S$-matrix
is the time ordered exponential 
\begin{equation}
\label{svong}
S[g] =\T \exp{\,\ir\!\int\!\dv^{\,4}\!\! x \,\mathscr L_{\text{int}}(x, g(x))}\ ,
%\  \T\mathscr L_{\text{int}}(x,g(x))^\star = \T\mathscr L_{\text{int}}(x,g(x))\ ,
\end{equation}
where at each point $x$ the time ordered interaction Lagrangian $\T \mathscr L_{\text{int}}(x,g(x))$
is  a hermitian, scalar operator which depends on the intensity function $g$ and 
which is local 
\begin{equation}
\label{llocal}
\phantom{\,} [\T\mathscr L_{\text{int}}(x,g(x)),\T\mathscr L_{\text{int}}(y,g(y))] =0 \text{\qquad if $x-y$ is spacelike}\ .
\end{equation}
In the standard model $\T \mathscr L_{\text{int}}(x)$ is a normal ordered polynomial in the free fields at~$x$ 
which create and annihilate the elementary particles. 
But despite its fundamental role the convergence of the series is unknown as is the mathematical existence of relativistic scattering.

Mathematically well defined references on scattering theory \cite{reed3} restrict their discussion to nonrelativistic scattering
and specialize mainly to scattering by potentials which depend on the distance of the two incident particles.
Such an interaction is manifestly invariant under Galilei transformations, which govern nonrelativistic motion.

Using the Schr\"odinger picture we recapitulate the analysis of Reed and Simon in the relativistic case and find that the innocent looking
requirement of translational invariance, that the interacting Hamiltonian $H'$ commute with the generators $P^m$
of the translations of free particles, implies $H'=H$ and excludes scattering.

Each representation of the Poincar\'e group on many-particle states, however, is reducible and allows
the weaker invariance requirement that $H'$ commute with the four-velocity $U^m=P^m/M$, $M=\sqrt{P^2}$, 
which generates the translations of observers. Though $H'$ does not and must not commute with $P^m$, 
the resulting $S$-matrix does.

Using center coordinates we map relativistic scattering to the nonrelativistic case, thereby establishing its mathematical existence.
This shifts the requirement of locality (\ref{llocal}) into the focus of further investigations.

The usefulness of the Schr\"odinger picture is shown by the approximate factorization
of the scattering probability into the cross section and the integrated luminosity of the incident particles.
The latter is proportional to the spacetime overlap of the incident Schr\"odinger wavepacket and is basic to position 
measurements with light, which is plagued by the nonexistence of a position operator.

Notation: Let $T_{a}: x \mapsto x + a$ denote a translation in $\mathbb R^4$, $T_\Lambda: x \mapsto \Lambda x$ a Lorentz transformation, 
and $T_{a,\Lambda}=T_a T_\Lambda \in \mathfrak P$ a Poincar\'e transformation.
We denote by $U_{a,\Lambda}$ its unitary representation in a Hilbert space $\mathcal H_1$ of one-particle states.

\section{Free and Interacting Motion}


Two-particle states are spanned by products of one-particle states and naturally transform under
the Poincar\'e group by the product representation $U_{a,\Lambda}\otimes U_{a,\Lambda}$. 
Applied to two-particles states  $\Psi_2:(i,j,p_1,p_2)\mapsto \Psi^{ij}(p_1,p_2)$
the generators of translations, the momentum operators $P^m$,
satisfy the Leibniz rule and preserve the individual four-momenta, $p_1$ and $p_2$,
\emph{separately}, 
\begin{equation}
\label{p0free}
(P^m\, \Psi)^{ij}(p_1,p_2) = (p_1^m+p_2^m) \, \Psi^{ij}(p_1,p_2)\ .
\end{equation}
So the time evolution $\Psi(t)=\e^{-\ir H t}\Psi(0)$, generated by $H = P^0$, %$\Psi(t)=U(t)\Psi(0)$, 
is \emph{free}.

An interacting time evolution must not map products of one particle states to the product of the freely evolved factors
but has to change the relative motion and the individual momenta of the scattering many-particle states. 

Let the Hamiltonian $H'$ generate the unitary one-parameter group of an interacting time evolution in the Hilbert space $\mathcal H$ of many-particle states
\begin{equation}
U'(t)=\e^{-\ir H' t}\ ,\ U'(t+s)=U'(t)U'(s)\ , \ t,s \in \mathbb R\ ,
\end{equation}
with worldlines $\Gamma=\set{(t, U'(t)\Psi), t\in \mathbb R}$ in quantum spacetime $\mathbb R \times \mathcal H$. 
At early times $U'(t)\Psi$ consists of distant particles, elementary or composite,  moving freely 
before they come near enough to interact.
The final state is considered sufficiently late such that the scattered particles
%after forming bound states (hadrons, nuclei, ions, atoms, molecules) which screen long range forces, 
have separated, their mutual interactions have become
negligible and the particles move again freely. 
%Strictly speaking, this basic concept becomes problematic with gravity 
%which cannot be shielded and does not average out in the mean.

% Figure environment removed



To abstract from the inessential, one would like to consider the limits $\Psi_{\pm}$
of $U'(t)\Psi$ for $t\rightarrow \pm \infty$. 
\label{onepgroup}
But such limits of a unitary, nontrivial one-parameter group do not exist: 
$U'(s+t)=U'(s)U'(t)$ implies $\Psi_{\pm} =U'(s)\Psi_{\pm}$ and $U'(t)\Psi-\Psi_{\pm}=U'(t)(\Psi - \Psi_\pm)$.
As $U'(t)$ is unitary the norm of this difference  is time independent 
and vanishes in the limit only if it vanishes for all times.
i.e. only if $\Psi$ does \emph{not} move \cite{reed3}. 

A unitary, nontrivial group of motion has no limit.

The interacting evolution of scattering states $\Psi$ can at best approach the free evolution by $U(t)=\e^{-\ir H t}$
of asymptotic states $\Psi_\pm$ such that 
$U'(t)\Psi-U(t)\Psi_\pm$ or equivalently $U^{\prime -1}(t) U(t)\Psi_\pm $ converge for $t\rightarrow \pm \infty$. 
%\footnote{Less restrictive is to ask for $U'(t_\Psi'(t))\Psi'-U(t)\Psi$ to converge, where $\lim_{t\rightarrow \pm \infty} t_\Psi' - t = c_{\Psi}^\pm$.}
There have to exist the strong limits
\begin{equation}
\Omega_+ = \slim_{t\rightarrow \infty}\Omega(t)\ ,\ 
\Omega_- = \slim_{t\rightarrow -\infty}\Omega(t)\ ,\ 
\Omega(t) =\e^{\ir H'\,t}\,\e^{-\ir H\, t}\ ,\ 
\end{equation}
that the interacting path through each scattering state $\Psi$, which is orthogonal to all bound states of $H'$, 
has a past and a future free asymptote through states $\Psi_\inn$ and $\Psi_\out$ with
\begin{equation}
\label{waveop}
\lim_{t\rightarrow \infty }\Omega(t)\,\Psi_\out=  \Omega_+ \Psi_\out = \Psi\ ,\  
\lim_{t\rightarrow -\infty }\Omega(t)\,\Psi_\inn= \Omega_- \Psi_\inn =\Psi\ .
\end{equation}

The strong limit\index{strong limit}\index{$\slim$} of $\Omega(t)$ for $t\rightarrow \infty$ 
demands that for each $\varepsilon > 0$ and for each scattering state $\Psi$ %,which is orthogonal to all bound states of $H'$, 
there is a time~$T$ such that $\norm{(\Omega (t)-\Omega_+)\Psi}< \varepsilon \norm{\Psi}$ for all $t > T$.
This is a weaker condition than the uniform limit that $T$ be independent of~$\Psi$. 

To ask even stronger for the uniform limit of $\Omega(t)$ would require too much because at each time $T$ there are states 
which have not reached or left the interaction region. 

$\Omega_+$ and $\Omega_-$ are the generalized \idx{wave operators} or M\o ller operators.\index{Moller@M\o ller operator}

In more detail we write $\Omega_\pm(H',H)$ to display the involved Hamiltonians.  
One has  %$\Omega_{\pm}(H,H')= \Omega_\pm(a\,H,a\,H')$ for all positive $a$, 
$\Omega_{\pm}(H',H)^\star= \Omega_\pm(H,H')$ and
$\Omega_{\pm}(A,B)\,\Omega_{\pm}(B,C)=\Omega_{\pm}(A,C)$ \cite{reed3}.
% which makes one wonder whether one can switch on
% the coupling such that \mbox{$\Omega_\pm(H,H'(\lambda)) = \e^{\ir \lambda L_{\pm}}$}?


%Lorentz transformations $\Lambda$ map states $\Psi$ to states $U_\Lambda \Psi$ and paths $\Gamma:t\rightarrow \Psi(t)$ to paths $U_\Lambda \Gamma: t \mapsto U_\Lambda \Psi(t)$.
%The interaction is relativistic if the relations between the interacting paths $\Gamma$ through $\Psi$ and the initial states $\Psi_\inn$ and $\Psi_\out$ of their asymptotes are Lorentz invariant,
%\begin{equation}
%\label{lorentzmoell}
%U_\Lambda \Omega_\pm U_\Lambda{}^{-1}=\Omega_\pm\ .
%\end{equation}

By construction the wave operators $\Omega_{\pm}$ \emph{do not} commute with free time translations but intertwine
$H'$ unitarily with its corresponding $H$,
\begin{equation}
\label{moellerint}
\Omega_{\pm} = \slim_{t \rightarrow \pm\infty}\e^{\ir\, H'\,(t+a)}\e^{-\ir\, H\, (t+a)}= 
\e^{\ir \,H' a } \Omega_\pm \e^{-\ir\,H\,a}\ ,\quad
\e^{\ir \,H'\, a }\Omega_{\pm} = \Omega_{\pm} \e^{\ir \,H\, a} \ .
\end{equation}
%So one can exchange in figure \ref{fig:scattering} the slice $t=0$ in $\mathbb R \times \mathcal H$ by any slice $t=a$. 
Differentiation at $a=0$ shows
\begin{equation}
\label{scomH}
H' \Omega_\pm\Psi = \Omega_\pm H\Psi\ ,\ H'= \Omega_\pm H \Omega_\pm^{-1}\ ,
\end{equation}
for all smooth scattering states $\Psi$. On these states $H'$ is unitarily equivalent to $H$. 

But $H'$ must not commute with $H$, otherwise it commutes with $\Omega(t)$ and equals~$H$ if $\Omega_\pm$ exists. 
Thus, implementing translational invariance one must not require  $H'$ to commute with the unitary representation $U_a= \e^{\ir P a}$
of translations. 

\section{Cross Section and Luminosity}

We employ the Schr\"odinger picture and view time evolutions as worldlines $\set{(t,\Psi(t))}$ in quantum spacetime $\mathbb R \times \mathcal H$.
Interaction makes the worldlines of many-particle states depart from the free time evolution. 
$\Psi_\inn$ and $\Psi_\out$ are not states with momenta which are all directed towards or away from a scattering region. Rather 
they are the \emph{initial}  states of the future or past asymptotes which in the long run will automatically develop this property.
They only have to be many-particle states in the continuous spectrum of~$M'$.
That a state lies on an interacting trajectory is not a \emph{property} of the state but a \emph{relation} of the state and the path.
Such a relation does not contradict the additional relation that the same state also lies on a free trajectory.
By themselves states have no time evolution and are neither interacting nor free, they are just states
and determine the probabilities of the results of all measurements. 

Similarly in classical mechanics particles may traverse Kepler ellipses or straight lines, but this does not make the points of these curves elliptic
or straight. Nitpicking as the remark may seem, it spares the vain endeavours to construct interacting fields
or the futile considerations what an interacting Lorentz boost should be. Such denominations are widespread but misleading:
to be interacting is a property not of states but of time evolutions. Scattering theory  compares different time evolutions.
% in the same Hilbert space.



By the basic assumption of quantum theory the probability for a result $a_i$ (which for simplicity we take to be labeled
by some discrete index $i$) to occur if the state~$\Psi$ is measured with a perfect apparatus~$A$, is given by 
\begin{equation}
\label{qmbasic}
w(i,A,\Psi) = |\Braket{\Lambda_i | \Psi}|^2\ ,
\end{equation}
where $\Lambda_i$ is the state which yields $a_i$ with certainty.

In the \idx{Heisenberg picture} not the states evolve in the course of time $t$ but the operators which represent the measuring devices
and enter (\ref{qmbasic}) by their eigenvectors~$\Lambda$, 
\begin{equation}
\Braket{\Lambda(0)|\vphantom{U(t)^{\star}} U(t) \Psi(0)}_{\text{Schr\"odinger}}=\Braket{U(t)^{\star}\Lambda(0)|  \Psi(0)}_{\text{Heisenberg}}\ .
\end{equation}

The Heisenberg picture is invertibly related to the \idx{Schr\"odinger picture} and in this sense equivalent. 
But to ascribe the motion of \emph{several} particles, which move relative to each other and scatter, to the measuring devices
is as counterintuitive and misleading as the Ptolemaic system which describes the orbits of the planets 
in highly unsuitable, though admissible, coordinates in which the earth does not rotate  and does not orbit the sun. 
Try e.g. to understand the simple notion of the spacetime overlap of
colliding wave packets (\ref{lumin1}) in the Heisenberg picture. 

%If one restricts in the Heisenberg picture
%the attention to operators and neglects the states on which they act one tends to miss the notion of a strong limit and to require a norm limit
%which cannot exist.
%\longpage
% The Ptolemaic system is a memorable misconception which was shared throughout the
% scientific community and was handed down from generation to generation hindering scientific progress for two millennia.

The Schr\"odinger picture does not rule out to consider time independent states, such as $\Psi_\inn$ or $\Psi_\out$, 
the initial states of the asymptotes of interacting paths,
nor does it preclude time dependent operators such as free fields. 
They are used to construct a local, relativistic $S$-matrix.
Time dependent fields do \emph{not} have to represent measuring devices in the Heisenberg picture,
notwithstanding axiomatic systems which call them \lq observables\rq .


The scattering matrix or $S$-matrix\index{Smatrix@$S$-matrix} is the map 
\begin{equation}
\label{sinout}
S:\Psi_\inn \mapsto  \Psi_\out\ ,\ 
S = \Omega_+^{\star}\Omega_-
= \slim_{t,\,t' \rightarrow \infty}\,\e^{\ir H t}\e^{-\ir H' (t+t')}\e^{\ir H t'}\ .
\end{equation}
Its matrix elements are scalar products of out- and in-states, 
$\braket{\Phi_\inn | \Psi_\out}=\braket{\Phi_\inn |S \Psi_\inn }$. %$=\braket{\Psi_\out | S^{-1} \Phi_\out}=\braket{S \Psi_\out | \Phi_\out}$. 

By construction and by (\ref{moellerint}) the $S$-matrix commutes with temporal translations
\begin{equation}
\label{stemp}
S = \e^{\ir \,H\, a }\,S\,\e^{-\ir \,H\, a }\ ,\ [H, S] = 0\ .
\end{equation}
If the $S$-matrix commutes with Lorentztransformations, $U_\Lambda S\, U_\Lambda{}^{-1}=S$,
then $S$ commutes not only with $H=P^0$ but with all momenta $P^m$ and with all %Poincar\'e transformations 
$U_{a,\Lambda}$. Conservation of $P^m$ does not require $H'$ to commute with $P^0$.

The relativistic $S$-matrix \emph{in the momentum basis}, not the basis independent $S$-matrix by itself, contains the experimentally 
available information about the interacting particles. 
Consider the transfer matrix $T := \ir(S - 1)$. Applied to an incoming two-particle state $\Psi$, omitting spin indices, 
using the short hand $q=(q_1,\dots q_n)$
%and $p(q)=\sum_{i=1}^n q_i$ 
and the scalar product of one-particle states on mass shells
$\mathcal M_m= \set{p:p^0 = \sqrt{m^2 + \vec p^2}, \vec p \in \mathbb R^3 }\subset \mathbb R^4$,
\begin{equation}
\braket{\Phi|\Psi}=\int\!\!\dt p\, \Phi(p)^*\, \Psi(p)\ ,\   
\dt p = \frac{\dv^3 p}{(2 \pi)^3\, 2 \sqrt{m^2 + \vec p^3}} \ ,
\end{equation}
its reduced kernel $\Braket {q | \mathfrak T | p_1,p_2}$ is defined by 
\begin{equation}
\label{tmatrix}
\ir (S-\eins)\Psi(q) = 
(2\pi)^4\!\! \int\! \tilde \dv p_1\, \tilde \dv p_2\,
\delta^4(p_1+p_2 - \sum_{i=1}^n q_i)
\Braket{q | \mathfrak T | p_1,p_2 }\Psi(p_1,p_2)\,.
\end{equation}
It determines the partial cross sections for the production of $n$ particles with momenta~$q$ in some domain $\Delta_n$ (denoting $\dt q_1\dots \dt q_n$ by $\dt^{\, n}\! q$)
\begin{equation} 
\label{sigmavons}
\sigma_{(p_1,p_2) \rightarrow \Delta_n}= \frac{(2\pi)^4}{4\sqrt{(p_1\cdot p_2)^2 - m_1^2m_2^2}}\int_{\Delta_n}\!\! \tilde \dv^n\! q\,
\delta^4(p_1+p_2 - \sum_{i=1}^n q_i)\, \bigl| \Braket{q | \mathfrak T | p_1,p_2 }\bigr|^2\ .
\end{equation}

%\longpage

We give a simple proof of this well-known basic relation of quantum scattering theory to observable physics,
which has the virtue to also determine the luminosity, which is basic to our optical perception of the world.

Recall that the wave function $\Phi=(S-1)\Psi$ is smooth if $\Psi$ is smooth: 
the relativistic $S$-matrix commutes with Poincar\'e transformations and maps the domain of the algebra of the generators,
rapidly decreasing smooth wave functions \cite{schmuedgen}, to itself. 
%We remind the reader that $\int\!\dv^n x\, f(x)\, \delta^k(\phi(x))$, $k\le n$, is defined
%to be the integral $\dv^{(n-k)}x\, f(x)_{|_{\phi=0}}/|\det \frac{\partial \phi}{\partial x}_{|_{\phi=0}}| $ over the submanifold $\phi(x)=0$.
The $\delta^4$-function in (\ref{tmatrix}) is not a singularity but reduces the integral on $\mathcal M_1\times \mathcal M_2$ to the compact sub\-ma\-ni\-fold 
$ p_1 +  p_2 = \sum_{i=1}^n q_i$, the phase space of the reaction.
Similarly
\begin{equation}
\bigl((S^\star -1)\chi\bigr)(p_1,p_2)=\ir(2\pi)^4\!\! \int\! \tilde \dv^n q \,
\delta^4(p_1+p_2 - \sum_{i=1}^n q_i)\,
\Braket{ q| \mathfrak T |p_1,p_2  }^*\chi(q)
\end{equation}
is smooth if $\chi$ is smooth.

By the generalization of (\ref{qmbasic}) to results in a continuum, 
the integral $\int_{\Delta_n}\! \dt^n q\, \Phi^*(q)\,\Phi(q)$ is the probability to find after the scattering $n$ particles with momenta $q=(q_1,\dots q_n)$ 
in the domain $\Delta_n$. Inserting (\ref{tmatrix}) yields 4 momentum integrations with a product of $\delta^4$-functions of different variables
\begin{gather}
\nonumber
\delta^4(p_1 + p_2 - \sum_{i=1}^n q_i)\,\delta^4(p'_1 + p'_2 - \sum_{i=1}^n q_i)=\delta^4(p_1 + p_2 - \sum_{i=1}^n q_i)\,\delta^4(p'_1 + p'_2 - p_1 -p_2)\\
=\delta^4(p_1 + p_2 - \sum_{i=1}^n q_i)\,\frac{1}{(2\pi)^4}\int\!\dv^4 x\ \e^{\ir\, (p'_1 + p'_2 - p_1 -p_2)\,x}\ .
\end{gather}

So the second $\delta^4$-function can be exchanged by the spacetime integral over the products at $x$ of plane waves for each of the 
integration variables $(p_1,p_2,p'_1,p'_2)$. More precisely this applies if multiplied with smooth test functions, which is why we remarked 
that $\Psi$ and $(S-1)\Psi$ are smooth
and that the $\delta$-functions only reduce integrations to integrals over submanifolds.

In scattering of distinguishable particles the incoming state is a product of 
momentum wave packets,  $\Psi(p_1,p_2)=\Psi_1(p_1)\Psi_2(p_2)$, with support %\footnote{The \idx{support}  $\supp(g)$ of a measurable function 
%$g$ with domain~$\mathcal D$ is the smallest closed subset of~$\mathcal D$ such that 
%$g = 0$ almost everywhere outside $\supp(g)$.} 
contained in small neighbourhoods around ${\bar p}_1$ and ${\bar p}_2$.
For small enough neighbourhood the smooth $\mathfrak T$-function does not vary appreciably. So we extract
\begin{equation}
\begin{gathered}
\delta^4(p_1+p_2 - \sum_{i=1}^n q_i)\,\Braket {q | \mathfrak T | p_1,p_2}\Braket {q | \mathfrak T | p'_1,p'_2}^*/\sqrt{p_1^0\,p_1^{\prime\,0}\,p_2^0\,p_2^{\prime\,0}}\\
\sim 
\delta^4(\bar p_1+\bar p_2 - \sum_{i=1}^n q_i)\, |\!\Braket {q | \mathfrak T | \bar p_1,\bar p_2}|^2/(\bar p_1^0 \bar p_2^0)
\end{gathered}
\end{equation}
as if constant from the $(p_1,p_2,p'_1,p'_2)$-integrations
of the wave packets. Each of these $p$-integrations is of the form 
\begin{equation}
\tilde \Psi(x) = \sqrt{2} \int\!\dt p\,\sqrt{p^0}\, \Psi(p)\,\e^{-\ir p\, x}
\end{equation}
or its complex conjugate and yields in the Schr\"odinger picture for massive particles the corresponding freely propagating position wave function at $x\in \mathbb R^{1,3}$, 
the remaining integration variable.  Dropping the symbol $\ \bar{}\ $  we find the momentum $q$ in the domain~$\Delta_n$ 
(which must not overlap with the beam) with probability
\begin{gather}
\label{wahrstreu}
w_{(p_1,p_2) \rightarrow \Delta_n}=\\
\nonumber
\frac{(2\pi)^4}{4 p^0_1 p^0_2 }\,\int_{\Delta_n}\! \dt^n q\, \delta^4(p_1 + p_2 - \sum_{i=1}^n q_i)\ |\Braket {q | \mathfrak T | p_1,p_2}|^2\,
\int\!\dv^4 x\, |\tilde \Psi_1(x)|^2\, |\tilde \Psi_2(x)|^2\ .
\end{gather}
It factorizes into the cross section (\ref{sigmavons}) times the integrated \idx{luminosity} $L$
\begin{equation}
\label{lumin1}
\begin{gathered}
w_{(p_1,p_2) \rightarrow \Delta_n}= \sigma_{(p_1,p_2) \rightarrow \Delta_n} \,L\ ,\\ 
L =  \frac{\sqrt{(p_1\cdot p_2)^2 - m_1^2m_2^2}}{p_1^0 p_2^0} \int\!\!\dv\!^{\,4}\! x\,  |\tilde \Psi_1(t,\vec x)|^2\,|\tilde \Psi_2(t,\vec x)|^2\ .
\end{gathered}
\end{equation}
Even if photons are massless and do not allow for a generator $\vec X$ of translations of spatial momentum, such that strictly speaking they do not have a well-defined
position wave function, we take the integrated luminosity (\ref{lumin1}) to define macroscopic position measurement: to detect an object you shine light on it and register the 
reflected light. The other way round: a beam of light becomes visible if traversing mist. 

\begin{comment}
The total cross section $\sigma= \braket{\Phi|\Phi}/L$ is the complete sum of partial cross sections over all distinguishable results of the scattering. 
It depends on $\Psi_1$ and $\Psi_2$ only via $p_1$ and $p_2$ (and their spins which we do not display) and  is characteristic 
of the interaction of the particles, independent of other details of the states $\Psi_1$ and $\Psi_2$.
%That the probability $w_{(p_1,p_2) \rightarrow \Delta_n}$ is proportional to the spacetime overlap of the involved wave packets shows  macroscopic locality of scattering
%of massive particles.
\end{comment}
%\longpage

During the overlap of wave packets we neglect their spreading which occurs because they are superposed of momenta 
near $p_1$ and $p_2$. Then in fixed target scattering the density $|\tilde \Psi_1|^2(t,\vec x)=\rho_1(\vec x)$ is time independent and
the impinging wave packet %, a bunch, 
is rigidly shifted  with velocity $\vec v\ne 0$, $|\tilde \Psi_2|^2(t,\vec x)=\rho_2(\vec x- \vec v t)$. We employ coordinates  $\vec x =(x, x_\perp)$
parallel and perpendicular to $\vec v=(v,0,0)$ 
and denote the area densities obtained by integrating the volume densities along the beam by
\begin{equation}
\hat \rho(x_\perp)= \int\!\dv\! x\ \rho(x,x_\perp)\ .
\end{equation}
With these specifications the spacetime integral in (\ref{lumin1}) yields
\begin{equation}
\label{lumin}
\int\!\! \dv\!^{\,2}x_\perp\! \int\!\! \dv x\ \rho_1(x,x_\perp)\! \int\!\!\dv t\, \rho_2(x-vt,x_\perp)=
%\int\!\! \dv\!^{\,2}x_\perp\! \dv x\ \rho_1(x,x_\perp)\, \hat \rho_2(x_\perp) / v \\=
\int\!\! \dv\!^{\,2}x_\perp\, \hat \rho_1(x_\perp)\, \hat \rho_2(x_\perp)\, \frac{1}{v}\ .
\end{equation}

The beam overlaps the target, $\supp \hat \rho_1 \subset \supp \hat \rho_2$, else $\Psi_1$ is not a target in the beam. Moreover, within $\supp \rho_1$ 
the density of the beam $\hat \rho_2(x_\perp)=\bar \rho_2$ has its average value, at least after taking the mean of measurements with the target randomly positioned
in the beam. The remaining integral $\int\! \dv\!^{\,2}x_\perp\, \hat \rho_1(x_\perp)=1$ is the
number of targets. With $p_1=(m_1,0,0,0)$, $(p_1\cdot p_2)^2-m_1^2m_2^2= m_1^2 \vec p_2^2$ and $v= |\vec p_2| / p_2^0$ 
the integrated luminosity  for fixed target scattering turns out to be the mean area density of the beam, the inverse of the size $A$ of its transversal section,
\begin{equation}
L_{\text{fixed target}}=\bar \rho_2=1/A\ .
\end{equation}
The particle is scattered with the same probability $w=\sigma / A$
with which a randomly positioned point in the beam hits a fixed area 
of size $\sigma$ in the beam. This confirms that $\sigma_{(p_1,p_2)\rightarrow \Delta_n}$ (\ref{sigmavons}) 
is the partial cross section of the target.

\begin{comment}
It is Lorentz invariant, $\sigma_{(p_1,p_2)\rightarrow \Delta_n}=\sigma_{(\Lambda p_1,\Lambda p_2)\rightarrow \Lambda \Delta_n}$, if $S$ is relativistic. % (\ref{stemp}).
% and commutes with Lorentz transformations $U_\Lambda$.
The total cross section depends only on the particle species, the spin of $\Psi_\inn=\Psi_1\otimes \Psi_2$ and on the invariant mass squared $(p_1 + p_2)^2$.

In colliding beam experiments one does not know the area densities $\hat \rho_1$ and $\hat \rho_2$ of both beams which determine the luminosity by (\ref{lumin}).
One has to measure~$L$ by varying the overlap of the colliding wave packets.

The measured cross sections of conveniently chosen processes at some momenta are invertible functions $\lambda_{\text{phys}}(\lambda)$
of the parameters $\lambda$ of the theoretical $S$-matrix
which predicts the cross sections for the other momenta. If one calculates the $S$-matrix order by order in perturbation theory
then one has to adjust (renormalize) the parameters~ $\lambda$ in each order such that $\lambda_{\text{phys}}$ retain their values.
Only their relations to the cross sections at other momenta receive measurable quantum corrections.

%Comparing theory and measurement the present standard model of the interacting, fundamental particles 
%(quarks, electrons, neutrinos, gauge bosons and Higgs scalar) with about twenty parameters (masses and couplings) has been derived in agreement with all ex\-peri\-men\-tal 
%evidence of a century of high energy physics. 

%However, the structures of hadrons as bound states of quarks still 
%remain to be understood in detail as do the values of the parameters,  the incorporation of gravity in a quantum 
%theory with only finitely many parameters and gravity on galactic and cosmic scales, which indicates
%otherwise unobservable dark matter and dark energy.
\end{comment}

\section{Center Variables}

The total momentum $P^m$ of an $n$-particle state $\Psi(p_1,\dots p_n)$ defines its invariant mass~$M$,
\begin{equation}
P^m\,\Psi(p_1,\dots p_n)=(\sum_{i=1}^n p^m_i)\Psi(p_1,\dots p_n)\ ,\ M^2 = P^2\ ,
\end{equation}
which has a purely continuous spectrum
\footnote{For $n\ge 2$ there are no eigenstates of $M$, as $P^2\Psi = m^2 \Psi$ restricts the support of $\Psi$
in the product of  mass shells \mbox{$\mathcal M_{1}\times \dots \times \mathcal M_n$} 
to a submanifold with vanishing $3n$-dimensional measure.
This continuous spectrum of~$M$ distinguishes many-particle states from one-particle states. % which are eigenstates.
}  with positive energies 
and allows to factorize $P^m$ as four-velocity $U^m$ times $M$
\begin{equation}
\label{ucenter}
P^m = U^m\,M\ ,\ U^2 = 1\ , \ [U^m, M]=0\ .
\end{equation}

%\longpage

To separate the motion of the center from the relative motion of the scattering particles, we change the variables of the wave function $\Psi$ from the momenta 
$(p_1,\dots p_n)$, $n\ge 2$, to the constrained center variables $(u,q)$ where 
\begin{equation}
u^m=\frac{\sum_i p^m_i}{\sqrt{(\sum_j p_j)^2}}\ ,\ u^2 = 1\ ,
\end{equation}
is the four-velocity of the center. It is well-defined unless all momenta are lightlike and colinear,
a Lorentz invariant sub\-mani\-fold $S^2\times \mathbb R^n$ which is outside the domain of scattering theory.
To obtain $q=(q_1, \dots q_{n})\in \mathbb R^{3n}$, the relative momenta at rest,  we decompose each momentum $p_i$ into parts which are parallel and orthogonal to~$u$,
\begin{equation}
p_i = p_{i\,\parallel} + p_{i\,\perp}\ ,\ p_{i\,\parallel} = (p_i\cdot u)\, u\ ,\ p_{i\,\perp} = p_i - p_{i\,\parallel}\ ,
\end{equation}
and boost each $p_{i\,\perp}$ by the inverse of the Lorentz boost
\begin{equation}
\label{lorp}
L_u = \begin{pmatrix}
\sqrt{1+\vec u^2}&  \vec  u^{\T}\\
 \vec   u &\quad  \eins+ \frac{ \vec  u\, \vec  u^{\T}}{1+\sqrt{1+\vec u^2}}\ ,
\end{pmatrix}
\end{equation}
which maps $\underline u= (1,0,0,0)$ to the four-velocity $u = (\sqrt{1+\vec u^2,} \vec u)$, to
\begin{equation}
q_i = (L_u)^{-1}p_{i\,\perp}\ .
\end{equation}
Its $0$-component vanishes, $0= u \cdot p_{i\,\perp} = (L_u^{-1} u)\cdot (L_u^{-1} p_{i\,\perp}) = \underline u \cdot q_i = q_i^0$: 
each \mbox{$q_i=(0,\vec q_i)$} lies in~$\mathbb R^3$. By definition,  $\sum_i p_i = \sum_i p_{i\parallel}$, so  the $q_i$
are constrained,
\begin{equation}
0=\sum_{i=1}^n q_i\ .
\end{equation}

Because of $m_i{}^2=(p_{i\,\parallel} + p_{i\,\perp})^2= (p_i\cdot u)^2 + q_{i}^2$, 
one has $(p_i\cdot u)^2=m_i^2 + \vec q_i^2 $ 
and the momenta $p_i$ in terms of the constrained center variables are
\begin{equation}
\label{cent}
p_i(u,q) = \sqrt{m_i^2 + \vec q_i^2}\,  u + L_u\,  q_i = \sqrt{m_i^2 + \vec q_i^2}
\begin{pmatrix}
\sqrt{1+\vec u^2}\\
\vec u
\end{pmatrix}
+
\begin{pmatrix}
\vec u \cdot \vec q_i\\
\vec q_i + \frac{(\vec u \cdot \vec q_i)\,\vec u}{1 + \sqrt{1 + \vec u^2}}
\end{pmatrix}\ .
\end{equation}
By $\sum_i p_i = \sum_i (p_i\cdot u)\, u = M \,u$ the invariant mass $M$ is the energy in the rest system
\begin{equation}
\label{M}
(M \Psi)(u,q) =M(q)\, \Psi(u,q)\ ,\  M(q)=\sum_{i=1}^n \sqrt{m_i^2 + \vec q_i^2}\ge \sum_i m_i\ .
\end{equation}

The momenta $p_i$ and %the four-velocity 
$u$  Lorentz transform as four-vectors, $u \mapsto \Lambda u$, 
while for given $u$ the relative momenta~$q_i$ are Wigner rotated by 
$W(\Lambda, u)=L_{\Lambda u}^{-1} \Lambda L_u\in $SO$(3)$.

The constraint $\sum_i q_i = 0$ complicates $M(q)$. Solving it by $q_n=-\sum_{i=1}^{n-1} q_i$,
the mass $M(q)=\sum'\sqrt{m_i^2+ \vec q_i^2}+ \sqrt{m_n^2 +(\sum' \vec q_i)^2}$
depends for $n\ge 3$ not only on $\vec q_i^2$ but also on so called \idx{Hughes-Eckart term}s $\vec q_i\cdot \vec q_j$. 

Wavefunctions of the relative momenta $(q_1,\dots q_{n-1})$ together with the spins of the $n$ particles constitute a representation space of rotations SO$(3)$ or SU$(2)$. 
It decomposes into a sum $\sum_s \mathbb C^{2s+1}\otimes \mathcal I_s$ of multiplets on which the representation acts by 
multiplication with unitary spin-$s$ matrices with skew hermitian generators $\Gamma_{ij}$ leaving pointwise invariant the Hilbert space $\mathcal I_s$ of functions 
$f(r)$ of $d_n$ rotation invariant variables $r$ ($d_2=1$, $d_n=3(n-2)$ for $n > 2$). 
Each of these spin-$s$ multiplets of SO$(3)$ induces a representation $U_{a,\Lambda}$
of the Poincar\'e group $\mathfrak P$ in the space $\mathcal H_s \otimes \mathcal I_s $ of wave functions $\Psi(u,r)$
of the center's four-velocity~$u$, $u^2 = 1$, and of the invariants~$r$.  
The generators of $U_{a,\e^\omega}=\e^{\ir a P}\e^{-\ir \omega^{mn}M_{mn}/2}$, $M_{mn}= - M_{nm}$, act on these states by \cite{dragon}
\begin{gather}
\label{factoru}
P^m = U^m\, M\ ,\ \bigl(U^m \Psi\bigr)(u,r) = u^m\, \Psi(u,r)\ ,\  \bigl(M \Psi\bigr)(u,r) = M(r) \, \Psi(u,r)\ ,\\
\label{gencont}
\begin{aligned}
\bigl(-\ir M_{ij}\Psi\bigr)(u,r) &= -\bigl(u^i \partial_{u^j} - u^j \partial_{u^i}\bigr)\Psi(u,r) + \Gamma_{ij}\Psi(u,r)\ ,\\
\bigl(-\ir M_{0i}\Psi\bigr)(u,r) &= \sqrt{1+\vec u^2}\,\partial_{u^i}\Psi(u,r) + \Gamma_{ij}\frac{u^j}{1+\sqrt{1 +\vec u^2}}\Psi(u,r)\ .
\end{aligned}
\end{gather}
The states $\Psi$ are smooth not only as a function of $u$ but within open neighbourhoods also of the variables $r$, 
%At the boundaries of $r$-domains the states are continuous  
if they  are smooth functions of $(p_1\dots p_n)$ on each mass shell
%$\mathbb R^3$ or $S^2\times \mathbb R$ 
as is required for~$\Psi$ to be 
in the domain of the generators which act by the product rule.

For a two-particle system and an observer at rest the Hamiltonian $ P^0=U^0\,M $  is
\begin{equation}
\label{qvonz}
H=\sqrt{1+\vec u^2}\bigl(\sqrt{m_1^2 + \vec q^2} + \sqrt{m_2^2 + \vec q^2}\bigr)=
\sqrt{1+\vec u^2}\bigl(m_1+m_2  + \frac{\vec z^2}{2\mu}\bigr)
\end{equation}
where $1/\mu = 1/m_1+1/m_2$. The %monotonous 
function $\vec q^2(\vec z^2)$ exists by the implicit function theorem.
Explicitly it is given by 
\begin{equation}
\label{pvonm}
\vec{q}^{\, 2} = 
\frac{1}{4}(m^2 -2(m_1^2 + m_2^2) +  (m_1^2 - m_2^2)^2/m^2 )\ ,\ 
m^2(\vec z^2)=\bigl(m_1+m_2 + \vec z^2/(2\mu)\bigr)^2\ .
\end{equation}
%\begin{equation}
%4\, \vec q^2(\vec z^2) = \bigl (m_1 + m_2 + \frac{\vec z^2}{2 \mu}\bigr)^2 - 2 \bigl ( m_1^2 + m_2^2 \bigr )+ 
%\bigl ( \frac{ m_1^2 - m_2^2 }{m_1 + m_2 + \frac{\vec z^2}{2 \mu}}\bigr )^2\ .
%\end{equation}
To quadratic order in the velocities, $H$ is the nonrelativistic energy of the center of mass and of the relative motion 
confirming that the center variables generalize the center of mass coordinates to relativistic motion. However, the free relativistic Hamiltonian is
not the sum of the Hamiltonians of the center and the relative motion but their product. 
%In appropriate coordinates the
%relativistic and nonrelativistic Hamiltonian of the free relative two-body motion are the same.

Let the states $\Psi$, which the standard observer~$\mathcal O$ measures with devices~$A$, 
be related by the unitary representation $V_{a,\Lambda}$ of $\mathfrak P$
to the states $\Psi_{a,\Lambda}$, which Poincar\'e transformed \idx{observer}s $\mathcal O_{a,\Lambda}$ measure with the same results  %(\ref{qmbasic})
with their devices $A_{a,\Lambda}$,
\begin{equation}
\label{obspoin}
A_{a,\Lambda}=V_{a,\Lambda}\, A\, V_{a,\Lambda}{}^{-1}\ ,\   \Psi_{a,\Lambda}=V_{a,\Lambda}\Psi\ ,\ 
w(i,A,\Psi)=w(i,A_{a,\Lambda}, \Psi_{a,\Lambda})\ .
\end{equation}

To satisfy the Poincar\'e algebra, the generators $\hat P^m = U^m\,\hat M$ of translations $V_{a,\eins}$ simply have to employ some hermitian, positive  $\hat M$ 
which commutes with the four-velocity $U^m$ and with the Lorentz generators $M_{mn}$.
However, only if $\hat M$ \index{$\mathfrak P, \mathcal O_{a,\Lambda}, T_{a,\Lambda}$} is a multiple of $1$, 
do the observers agree on all Poincar\'e invariant measurements with devices $1\otimes \hat A$, which only act on the invariant arguments $r$,
\begin{equation}
V_{a,\e^{\omega}}= \e^{\ir\, U\cdot a} \e^{-\ir \,\omega^{mn} M_{mn}/2}\ .
\end{equation}
This is the unique (up to the scale of $a$), maximal de\-ge\-ne\-rate representation of $\mathfrak P$ on many particle states.
These transformations $V_{a,\Lambda}$ correspond one-to-one to the observers~$\mathcal O_{a,\Lambda}$. As their translation is generated by
the four-velocity $U$, a superposition of nearly mass degenerate particles is seen by  translated observers
as the same superposition multiplied with a common, mass independent phase rather than an oscillated superposition with changed relative phases.

%With the reducible representation of $\mathfrak P$ on many-particle states 
%one has to distinguish different generators of translations: the free momentum $P^m=\sum_i P_{(i)}^m$,
%the interacting momentum $P^{\prime m}$ and the four-velocity $U^m$ which translates states as seen by translated observers.

%That shifted observers see states shifted by $V_{a,\eins}=\e^{\ir U a}$
%applies already to superpositions of one-particle states with different masses,
%e.g. neutrinos. Shifted by $\e^{\ir\, P\cdot a}$ they show Rabi oscillations, because $\e^{\ir\, P\cdot a}$ shifts both by a common multiple of their
%unequal Compton wavelengths. But shifted observers see both particles shifted by a common 
%   length in {\sc si} units which is independent of the particle masses. 

For the interaction $H'$ to be Poincar\'e covariant it has to commute with the translations $V_{a,\eins}$
of observers. Hence it commutes with $U^m$. 
And it has to Lorentz transform as $0$-component $H' = P^{\prime\,0}$ of a four-vector,
\begin{equation}
\label{comvu}
[U^m, H']=0\ ,\  V_{a,\Lambda}\,P^{\prime\,0}\,V_{a,\Lambda}{}^{-1} = \bigl( \Lambda^{-1}\bigr){}^0{}_m P^{\prime m}\ .
\end{equation}
But if $H'$ commutes with the hermitian, positive operator $U^0=\sqrt{1+\vec U^2}$ then 
both have a common spectral resolution and $M'=H'/U^0$ is a well-defined hermitian and Lorentz invariant operator,
\begin{equation}
\label{condition}
H'= U^0\, M'\ ,\ P^{\prime\, m}= U^m\, M'\ ,\ 
[M', U^m]=0\ ,\ [M', M_{mn}] = 0\ .
\end{equation}
Hence the M\o ller operators are Lorentz invariant,
\begin{equation}
\begin{gathered}
\lim_{t\rightarrow \pm \infty }\bigl(\e^{\ir\, u^0\,  M'\, t}\e^{-\ir\, u^0\,  M\, t}\Psi\bigr)(u,r)=
\lim_{t'\rightarrow \pm \infty }\bigl(\e^{\ir\, M' \,t'}\e^{-\ir\,  M\, t'}\Psi\bigr)(u,r)\ ,\\
\Omega_{\pm}(H',H)=\Omega_\pm(M',M)\ .
\end{gathered}
\end{equation}
So the $S$-matrix commutes with Lorentz transformations. By construction it commutes with $H=P^0$ (\ref{stemp}), 
hence it commutes with all $U_{a,\Lambda}$ and not only with $V_{a,\Lambda}$,
\begin{equation}
U_{a,\Lambda}\,S= S\,U_{a,\Lambda}\ .
\end{equation}

To commute with $U^m$ is a weaker restriction  of~$H'$ than to commute with $P^m$. The latter restriction excludes
scattering, as $\Omega(t) = \e^{\ir (H-H')t}$ has a limit only if $H=H'$. 

Therefore, in a basis of $\mathcal I_s$ in which~$M$ acts multiplicatively, $(Mf)(r)= m(r) f(r)$, the interacting mass $M'$ must not be multiplicative. 
For example $M'= M + V$ can be a sum with a potential $V(\vec x)$ where the position operator $\vec x$
constitutes Heisenberg pairs $[x^i,z^j]=\ir \delta^{ij}$ with the relative momentum $\vec z$ defined in (\ref{qvonz}).
If $V$ is spherically symmetric then the $S$-matrix commutes not only with rotations but also with boosts as they
act for each~$u$ by Wigner rotations of $\vec z$ and~$\vec x$. 

In nonrelativistic theory \cite{reed3} spherically symmetric potentials with a nontrivial $S$-matrix are known.
So also nontrivial relativistic scattering exists mathematically and not only as perturbation series with unknown convergence properties.
The relation of $M'$ and the time ordered interaction Lagrangian (\ref{llocal}) remains to be analyzed.

%The scattering states are orthogonal to the eigenstates of $M'$. In the space of scattering states $M$
%and $M'$ have a continuous spectrum and are unitarily equivalent.

%Free motion  of a bound state does not occur at vanishing coupling, the state stays bound and couples its constituents with fixed, nonvanishing coupling and binding energy.
%Free motion of a bound state occurs in sufficient distance to other particles.
\begin{comment}
As long as the momentum transfer is small as compared to the binding energy
one expects composite particles to scatter similarly to elementary particles.
In deep inelastic scattering (large momentum transfer) the binding should become negligible and the momentum wave functions of bound states 
should manifest themselves as probability distributions of mainly longitudinal momentum of partons -- 
longitudinal, because boosts tilt by aberration momentum distributions
towards the direction of the boost.

%Massless states e.g. gravitons cannot be bound states as $M' \Psi = 0$ implies vanishing momentum $P^{\prime \,m} = M' U^m$. 

The null result of de\-di\-cated searches for single quarks
shows that the space of $\inn$-$\out$-states is only a subspace of the many-particle Hilbert space in which the local fields
create and annihilate single quarks and gluons. Experimentally, one can prepare only colorless states with three (or more) quarks and quark-antiquark pairs.
This is attributed to the dynamical reason that the forces between quarks do not decrease with distance.
Colored factors of many-particle states decay by quark exchange or pair creation into products of baryons and mesons. Quarks and gluons are confined to hadrons.
\end{comment}
%There the partial derivatives $\frac{\partial p}{\partial q, u }$ constitute the product matrix
%\begin{equation}
%J = \eins_{3\times 3} \otimes \tilde J  \ ,\ 
%\tilde J =
%\begin{pmatrix}
%1 &\cdots & &\sqrt{m_1^2 + \vec q_1^2}\\
%& \ddots & & \vdots \\
%  &  & \phantom{-}1 & \sqrt{m_{n-1}^2 + \vec q_{n-1}^2}\\
%-1& \cdots & -1 &  \sqrt{m_{n}^2 + \vec q_{n}^2}
%\end{pmatrix}\ ,
%\end{equation}
%with a factor $\eins_{3\times 3}$ from the three components of the involved vectors. The determinant of $\tilde J$ easily follows by adding its 
%first $n-1$ rows
%to the last row which produces an upper diagonal matrix $J'$ with $J'_{ij}=1$ if $i=j < n$ and
%$J'_{nn}=m(q)$,
%\begin{equation}
%m(q) =\sum_{i=1}^n \sqrt{m_i^2 + \vec q_i^2}\quad , \quad \sum_{i=1}^n \vec q_i = 0\ .
%\end{equation}
%Its  determinant 
%$\det J'=\det \tilde J= m(q)$ is the product  of its diagonal elements,  which  cubed 
%(because of the factor $\eins_{3\times 3}$) yields $\det J$ , so
%\begin{equation}
%\det J = m(q)^3
%\end{equation}
%and

%\longpage

In the Hamiltonian description of \emph{classical} relativistic systems one can choose the generators of spatial translations and of rotations 
to coincide in the free and in the interacting case \cite{peres} 
\begin{equation}
\label{pint}
P^{\prime\,i}\stackrel{?}{=} P^i\ ,\  M'_{ij}=M_{ij}\ ,\ i,j\in \set{1,2,3}\ .
\end{equation}
In a quantum system these relations are suggested if one employs \lq pictures\rq\ for the interacting and free time evolution which coincide
at some time. As rotations and spatial translations are time independent and coincide at some time, their generators should agree at all times in the different pictures.
But these relations exclude interaction: Poincar\'e covariance requires the interacting Hamiltonian $H'$ to commute with the four-velocity $U^m$  (\ref{comvu}),
hence $P^{\prime\, m}= M'\,U^m$ and $P^{m}= M\,U^m$.  But then (\ref{pint}) implies $M'=M$ and $P^{\prime\,m}=P^m$ and
excludes scattering. 

Moreover, the condition (\ref{pint}) is measurably wrong as shown by the atomic weights of isotopes.
Their rate of spatial momentum transferred by their support to prevent free fall, their weight,  depends on the binding. 

Using (\ref{pint}) and exploiting ingeniously  the ana\-ly\-ti\-ci\-ty of Lorentz transformations, \index{Haag's theorem}
Rudolf Haag proved  \cite{haag} the absence of interaction and $S=1$ in axiomatic quantum field theory: % \cite{wightman}:
the Wightman distributions, the vacuum expectation values of local fields, coincide at all times with the ones of free fields if the free and interacting fields coincide at an initial time.

Omitting all conditions the theorem is abbreviated to the statement that
the interaction picture only exists if there is no interaction. But it only excludes all theories based on the Wightman axioms.
They allow to reconstruct the fields and their transformations from the Wightman distributions up to unitary equi\-va\-len\-ce \cite{wightman}. 
But then, in case that there are no bound states, the distributions cannot distinguish the free from the interacting evolution as both differ only by unitary 
transformations~$\Omega_\pm$~(\ref{scomH}).
%In contrast, we conceive interaction as a property of the time evolution of many-particle states.

Haag's theorem is a no-go result of scattering theory as are the facts that a unitary group of motion has no limit, that $\Omega_\pm$  require the strong limit not the
norm limit and that on scattering states~$H$ and~$H'$ are unitarily equivalent but must not commute.

Haag's theorem is ignored by physicists who calculate successfully 
scattering amplitudes with Feynman graphs and Poincar\'e invariant rules to extract finite parts of products of free fields.
These calculations need no interacting fields. Strictly speaking 
$S = \T \exp\, \ir \int \dv^4\! x\, \mathcal L_{\text{int}}(x)$ is not a series in an algebra of operator valued distributions
as insinuated by the Wightman axioms.
Rather, the matrix elements of~$S$ are recursively  defined finite parts of integrals of products of distributions.

In Feynman graphs the term \lq interacting field\rq\  denotes an argument of
the time order~$\T$. But time order does \emph{not act on} operators but on a graded commutative algebra and \emph{yields} operator valued distributions.
The field's equation of motion \cite{zimmermann} in the abbreviated notation $\braket{\,X}:= \braket{\Omega | X \Omega }$
\begin{equation}
\label{varderop}
\braket{\,\T\,\e^{\ir \int\!\! \mathcal L_{\text{int}}}\,\frac{\hat \partial \mathcal L}{\hat \partial \phi}(x)\,\phi(x_1)\dots \phi(x_n)}
= \braket{\,\T\,\e^{\ir \int\!\! \mathcal L_{\text{int}}}\, \ir \frac{\delta }{\delta \phi(x)}\,\bigl(\phi(x_1)\dots \phi(x_n)\bigr)}
\end{equation}
contains the derivative ${\delta }/{\delta \phi(x)}$ confirming: an interacting field in a Feynman graph is an operation in a graded commutative algebra,
not an operator in Hilbert space.


\section{Position Measurement with Light}

Massless particles \emph{do not allow} a position operator $\vec X$, which generates translations of spatial momentum,
\begin{equation}
\label{transmom}
\bigl(\e^{\ir \vec b \cdot \vec X}\Psi\bigr)(\vec p)=\Psi(\vec p-\vec b)\ .
\end{equation}
It enlarges the algebra of the Poincar\'e generators by Heisenberg partners $X^j$ of the spatial momenta,
\begin{equation}
\ [P^i, P^j]=0=[X^i, X^j]\ ,\ [P^i, X^j] = -\ir\, \delta^{ij}\ ,  \ i,j\in\set{1,\dots D-1}\ . 
\end{equation}
Together with $P^0=\sqrt{\vec P^2}$ this algebra contains for $D > 2$
\begin{equation}
%P^0=\sqrt{\vec P^2}\ ,\
\sum_{j=1}^{D-1} [X^j,[X^j, P^0]]=-\frac{D-2}{|\vec P|}
\end{equation}
all powers of  $1/|\vec P|$. To be in the domain of this algebra, the wave functions have 
to decrease near $\vec p = 0$ faster than any power of $|\vec p|$. 

As the domain of the generators is invariant under the group which they generate \cite{schmuedgen} also all
$(\e^{\ir \, \vec b\,\vec X}\Psi)(\vec p)=\Psi(\vec p- \vec b)$ have to vanish at $\vec p=0$ for all~$\vec b$, 
thus $\Psi(\vec b)=0$ everywhere: 
the algebra of $\sqrt{\vec P^2}$, $\vec P$, the translations $\e^{\ir \vec b\cdot \vec X}$ of $\vec P$ and their generators $\vec X$
has no domain.

%The domain of the algebra of $\vec X$ and $\vec P$ are the smooth and rapidly decreasing functions $f:\mathbb R^{D-1}\rightarrow \mathbb C$, but unless $f(0)=0$ 
%the function $P^0 f: \vec p \mapsto |\vec p| f(\vec p) $ is not in this domain. The domain of the polynomial algebra  in
%$P^0,\vec P, \vec X$ vanishes. They do not generate an algebra.

%For a contradiction among smooth functions, a contradiction in one point is sufficient.
%Contrary to the rule of thumb, a point in a continuum can be essential
%though a generic state in Hilbert space determines its wave functions only almost everywhere. 
%Not only integrals are important in quantum physics but also smooth functions of orbits  and fixpoints.  

Different from massive particles the momentum spectrum of  massless particles contains a Lorentz fixed point, 
$p = 0$. There the function $p^0=\sqrt{\vec p^2}$ of $\mathbb R^{D-1}$  is only continuous but not smooth. 
This single, distinguished 
point is sufficient to spoil the translation invariance of spatial momentum. %in $\mathbb R^{D-1}$.
It prevents $P^0$ to enlarge the algebra of $\vec P$, the translations $\e^{\ir \vec b\cdot \vec X}$ and its generators~$\vec X$.
All attempts \cite{hawton, newton, pryce, wightman} to construct such generators for massless particles fail.

The position of a state  cannot be identified with the argument $x$
of the field $\Phi(x)$ which creates and annihilates the particle.  
By the Reeh-Schlieder theorem \cite{reeh} the operators
$\Phi_f = \int\! \dv^4 x \,f(x) \Phi(x) $ with support of $f$ contained in a fixed open set $\mathcal U$
create out of the vacuum a dense subspace of one-particle states. So their position cannot be restricted to $\mathcal U$.

\begin{comment}
The proposal, to use the Fourier transformed (with respect to the spatial momentum) momentum wave function as position wave function,
does not work because $\Psi$ is a section. The Fourier transformation of the local section $\Psi_N$ is not locally related to the one of $\Psi_S$.
\end{comment}

That there is no position operator for massless particles disappoints expectations, because we see the world and reconstruct the position of all objects by light
which we receive as flow of massless quanta. But we do not see a distant photon. Rather we see massive objects,
using (\ref{lumin}),  by the 
currents of photons which they emit or scatter and which are annihilated in our retina.

%\longpage

\section{Conclusions}
The correct covariance requirement allows relativistic scattering. 
It differs from the requirements used in Haag's and Leutwyler's no-go theorems.

The factorization of the scattering probability into cross section times luminosity
holds only in the approximation that in momentum space the colliding wave packets are narrow
as compared to scales on which scattering amplitudes vary appreciably and are in addition localized 
in spacetime precisely enough to define their overlap. Neither the limit of sharp wave packets 
in momentum space nor in spacetime exist. It remains conceptually dubious what scattering in strongly curved spacetime is.

The Hamiltonians of the relative motion and the motion of the center do not decompose into a sum but are a product.
Bound states are eigenstates of the interacting invariant mass. Its relation to the local interaction 
Lagrangian still needs clarification.





    %\section*{Acknowledgements}
%Norbert Dragon thanks Gleb Arutyunov, Arthur Hebecker and Hermann Nicolai for helpful e-mail correspondence and
%Wilfried Buchm\"uller, Stefan Theisen and Sergei Kuzenko for extended, clarifying discussions.
%%on questions related to the subject of this paper.

\begin{comment}
\section*{Declarations}
\begin{itemize}
\item Funding\\
Not applicable
\item Conflict of interest/Competing interests\\ 
There is no conflicting or competing interest.
\item Ethics approval\\
Not applicable
\item Availability of data and materials\\
Not applicable
\item Code availability\\
Not applicable
\item Authors' contributions\\
The article is original work of the authors, sources are cited.
\end{itemize}
\end{comment}

%\backmatter


\begin{thebibliography}{99}
%\longpage
%\bibitem{arutyunov}Gleb 
%Arutyunov, \emph{Lectures on String Theory,} 2009,\\ 
%\url{https://www2.physik.uni-muenchen.de/lehre/vorlesungen/wise_19_20/TD1_-String-Theory-I/arutyunov_notes.pdf}
%\bibitem{bahns}Dorothea Bahns, Katarzyna Rejzner and Jochen Zahn, \emph{The effective theory of strings,} \emph{Comm. Math. Phys.} 327 (2014) 779--814\\ 
%\url{https://arxiv.org/abs/1204.6263}
\bibitem{bogoliubov}Nikolay N. Bogoliubov and Dmitry V. Shirkov,
\emph{Introduction to the Theory of Quantized Fields,} John Wiley, New York, 1959
%\bibitem{bose}Samir K. Bose and R. Parker, \emph{Zero-Mass Representation of Poincar\'e Group and Conformal Invariance,} \emph{J. Math. Phys.} 10 (1969) 812--813
%\bibitem{fischler}Sidney Coleman, cited by Willy Fischler, Igor Klebanov, Joseph Polchinski and Leonard Susskind, 
%\emph{Quantummechanics of the Googolplexus,} \emph{Nucl. Phys.} B327 (1989) 157--177
%\bibitem{dimock}Jonathan Dimock, \emph{Locality in Free String Field Theory-II}, \emph{Annales Henri Poincar\'e} 3 (2002) 613,  
%\url{http://arxiv.org/abs/math-ph/0102027}
%\bibitem{dirac}Paul A. M. Dirac, \emph{Generalized Hamiltonian Dynamics, Canadian Journal of Mathematics} 2 (1950) 129--148
%\bibitem{dixmier}Jacques Dixmier and Paul Malliavin, \emph{Factorisations de fonctions et de vecteurs ind\'efiniment diff\'erentiables,}
%\emph{Bull. Sci. Math.} 102 (1978) 305--330
\bibitem{dragon}Norbert Dragon, \emph{Geometry and Quantum Features of Special Relativity}, Springer Nature Switzerland, Cham, in preparation
%\bibitem{ek}Bengt Ek and Bengt Nagel, \emph{Differentiable vectors and sharp momentum states of helicity representations of the Poincar\'e group,
%J. Math. Phys.} 25 (1984) 1662--1670 
%\bibitem{fronsdal}Mosh\'e Flato, Christian Fronsdal and Daniel Sternheimer, \emph{Difficulties with massless particles?,} \emph{Comm. Math. Phys.} 90 (1983) 563--573 
%\bibitem{witten}Michael B. Green, John H. Schwarz and Edward  Witten, \emph{Superstring Theory,} \emph{Cambridge University Press,} 1987
%\bibitem{goddard}Peter Goddard, Jeffrey Goldstone, Claudio Rebbi and Charles Thorne, \emph{Quantum Dynamics of a Massless Relativistic String, Nucl. Phys.} B56 (1973) 109-135
%\bibitem{grundling}Hendrik Grundling and Charles Angas Hurst, \emph{The operator quantization of the open bosonic string: Field algebra,} 
%\emph{Comm. Math. Phys.} 156 (1993) 473
\bibitem{haag}Rudolf Haag, \emph{On Quantum Field Theory,} Dan. Mat. Fys. Medd., 29 (1955) 12
\bibitem{hawton}Margaret Hawton, \emph{Position Operator with Commuting Components,} Phys. Rev. A 59 (1999) 954 - 959
%\bibitem{hanson}Andrew J. Hanson, Tullio Regge and Claudio Teitelboim, \emph{Constrained Hamiltonian Systems,} \emph{Accademia Nazionale dei Lincei,} Roma, 1976
%\bibitem{thooft}Gerard \rq t Hooft, \emph{Introduction to String Theory,} \\ \url{http://www.phys.uu.nl/~thooft/lectures/stringnotes.pdf}
%\bibitem{jorjadze}George Jorjadze, Jan Plefka and Jonas Pollok, \emph{Bosonic string quantization in a static gauge,}  
%\emph{Journal of Physics A: Mathematical and Theoretical} 45 (2012) 485\,401
\bibitem{leutwyler}Heinrich Leutwyler, \emph{A No-Interaction Theorem in Classical Relativistic Hamiltonian Particle Mechanics,} Il Nuovo Cimento 37 (1965) 556--567
%\bibitem{lomont}John S. Lomont and Harry E. Moses, 
%\emph{Simple Realizations of the Infinitesimal Generators of the Proper Orthochronous Imhomogeneous Lorentz Group for Mass Zero,}
%\emph{J. Math. Phys.} 3 (1962) 405--408
\bibitem{zimmermann}John H. Lowenstein, \emph{\textsc{bphz} Renormalization,} in
Giorgio Velo and Arthur S. Wightman (Eds.), \emph{Renormalization Theory,}
{\textsc{nato} Science Series~C 23, Springer,  Berlin} (1976) 95--160
%\bibitem{mcfarlane} Alan J. Macfarlane, On the Restricted Lorentz Group and Groups Homomorphically Related to It, J. Math. Phys. 3 (1962) 1116--1129
%\bibitem{mackey}George W. Mackey, Induced Representations of Locally Compact Groups I,
%Annals of Ma\-the\-matics, 55 (1952) 101--139;
%Induced Representations of Locally Compact Groups II,
%Annals of Mathematics, 58 (1953) 193--221;  
%Induced Representations of Groups and Quantum Mechanics, W. A. Benjamin, New York, 1968
%\bibitem{neumann}John von Neumann, Die Eindeutigkeit der Schr\"odingerschen Operatoren, Mathematische Annalen 104 (1931)  570--578
\bibitem{newton}Theodore Duddell Newton and Eugene Paul Wigner, \emph{Localized States for Elementary Systems,}  Rev. Mod. Phys. 21 (1949) 400--406
%\bibitem{oppermann}Florian Oppermann, Die Poincar\'e-symmetrische Saite, Masterarbeit, Institut f�r Theoretische Physik, Leibniz Universit�t Hannover, 2015 
%unpublished
\bibitem{pryce}Maurice H. L. Pryce, \emph{The mass-centre in the restricted theory of relativity and its connexion with the quantum theory of elementary particles,}
%Maurice Henry Lecorney Pryce \emph{The mass-centre in the restricted theory of relativity and its connexion with the quantum theory of elementary particles}
Proc. R. Soc. London, Ser. A 195 (1948) 62
\bibitem{peres}Asher Peres, \emph {Relativistic Dynamics with Noncanonical Positions,} Physical Review Letters, 27 (1971) 1666 --1668
%\bibitem{reed1}Michael Reed and Barry Simon, Methods of Modern Mathematical Physics, Volume~1 Functional Analysis,
%Academic Press, London, 1980
\bibitem{reed3}Michael Reed and Barry Simon, \emph{Methods of Modern Mathematical Physics,~Volume~3 Scattering Theory,}
Academic Press, London, 1980
\bibitem{reeh}Helmut Reeh and Siegfried Schlieder, \emph{Bemerkungen zur Unit\"ar\"aquivalenz von Lorentz\-in\-varianten Feldern,} Il Nuovo Cimento 22 (1961) 1051--1068
%\bibitem{scherk}Joel Scherk, \emph{An Introduction to the Theory of Dual Models and Strings,} \emph{Rev. Mod. Phys.} 47 (1975) 123--164
\bibitem{schmuedgen}Konrad Schm\"udgen, \emph{Unbounded Operator Algebras and Representation Theory,} Birkh\"auser, Basel, 1990, Chapter 10\\ % Example 10.2.14
%\bibitem{schmuedgen1}
Konrad Schm\"udgen, \emph{An Invitation to Unbounded Representations of $\star$-Algebras on Hilbert Space,} Springer Nature Switzerland, Cham, 2020 % Example 10.2.14
%\bibitem{skagerstam}Bo-Sture Skagerstam, Localization of Massless Spinning Particles and the Berry phase,
%Contribution to the Festschrift of John R. Klauder on Occasion of his 60\textsuperscript{th} Birthday, 
%\url{https://arxiv.org/abs/hep-th/9210054}
%\bibitem{stone}Marshall H. Stone, Linear Transformations in Hilbert Space, III. Operational Methods and Group Theory,
%Proceedings of the National Academy of Sciences of the United States of America, 16 (1930) 172--175
%\bibitem{tong}David Tong, Lectures on String Theory, \url{https://arxiv.org/abs/0908.0333v3}
\bibitem{weinberg}Steven Weinberg, \emph{The Quantum Theory of Fields,} Cambridge University Press, 1995
%\bibitem{west}Peter West, Introduction to Strings and Branes, Cambride University Press, 2012
%\bibitem{witten}Michael B. Green, John H. Schwarz, Edward  Witten, Superstring Theory, Cambridge University Press, 1987
\bibitem{wightman}Arthur S. Wightman, \emph{On the Localizability of Quantum Mechanical Systems,} Rev. Mod. Phys. 34 (1962) 845--872
\end{thebibliography}
\end{document}
