\documentclass
[aps,prb,twocolumn,floatfix,english,showpacs,10pt]{revtex4-2}%
\usepackage{graphicx}
\usepackage{booktabs}
\usepackage{amsmath}
\usepackage{physics}
\usepackage{amssymb}
\usepackage{colordvi}
\usepackage{verbatim}
\usepackage{xcolor}
\usepackage{mathrsfs}
\usepackage{epsfig}
\usepackage{lipsum}
\usepackage{amsfonts}
\usepackage{makecell}
\usepackage[unicode=true, breaklinks=false, pdfborder={0 0 1}, backref=false,
colorlinks=true, linkcolor=blue, urlcolor=blue, citecolor=blue]{hyperref}%
\setcounter{MaxMatrixCols}{30}
%TCIDATA{OutputFilter=latex2.dll}
%TCIDATA{Version=5.50.0.2960}
%TCIDATA{Codepage=1252}
%TCIDATA{LastRevised=Friday, April 22, 2022 09:16:37}
%TCIDATA{<META NAME="GraphicsSave" CONTENT="32">}
%TCIDATA{<META NAME="SaveForMode" CONTENT="1">}
%TCIDATA{BibliographyScheme=Manual}
%TCIDATA{Language=American English}
%BeginMSIPreambleData
\providecommand{\U}[1]{\protect\rule{.1in}{.1in}}
%EndMSIPreambleData
\setcitestyle{numbers,square}



\begin{document}
\title{Gating ferromagnetic resonance by superconductors via modulated reflection of magnetization-radiated electric fields}

\author{Xi-Han Zhou}

\affiliation{School of Physics, Huazhong University of Science and Technology, Wuhan 430074, China}


\author{Tao Yu}
\email{taoyuphy@hust.edu.cn}
\affiliation{School of Physics, Huazhong University of Science and Technology, Wuhan 430074, China}


\date{\today }


\begin{abstract}
We predict that ferromagnetic resonance in \textit{insulating} magnetic film with inplane magnetization radiates electric fields polarized along the magnetization with opposite amplitudes at two sides of the magnetic insulator, which can be modulated strongly by adjacent superconductors. With a single superconductor adjacent to the magnetic insulator this radiated electric field is totally reflected with a $\pi$-phase shift, which thereby  vanishes at the superconductor side and causes no influence on the ferromagnetic resonance. When the magnetic insulator is sandwiched by two superconductors, this  reflection becomes back and forth, so the electric field exists at both superconductors that drives the Meissner supercurrent, which in turn shifts efficiently the ferromagnetic resonance. We predict an ultrastrong coupling between magnons in the yttrium iron garnet and Cooper pairs in NbN with the frequency shift achieving tens of percent of the bare ferromagnetic resonance.  
\end{abstract}



\maketitle

\section{Introduction}

``Magnonics" exploits magnetic excitations, i.e., spin waves or their quanta, magnons, as potential information carriers for spin transport in insulators with low-energy consumption~\cite{Lenk,Chumak,Grundler,Demidov,Brataas,Barman,Yu_chirality}.
Interaction between magnons and Cooper pairs in heterostructures composed of magnets and superconductors may modulate the transport of spin information~\cite{spintronics_1,superconductor_gating_theory,superconductor_gating_exp,silaev,Bobkova,FI/s_system,Wei_Han,Linder,Belzig,similar_theory}, strongly enhance the magnon-photon interaction~\cite{Janssonn_1,Janssonn_2,spintronics_2,strong_coupling_3,Silaev_ultrastrong_coupling,ultrastrong_in_press}, and lead to the emergence of triplet Cooper pairing~\cite{Bergeret,study_1,study_2,Banerjee,Balatsky}, which 
is potential to bring unprecedented functionalities in spintronics~\cite{Bergeret,study_1,study_2}, quantum information~\cite{Q_information,Q_information_1,Q_information_2,Q_information_4,Q_information_5}, and topological quantum computation~\cite{topological}. In this heterostructure, the hybridized quantum states and distribution of macroscopic electromagnetic fields govern its properties.
For example, the ``ultrastrong coupling"~\cite{strong_coupling} with the coupling strength close to the ferromagnetic resonance (FMR) frequency unveils the importance of the dipolar interaction in the  superconductor(S)$|$metallic ferromagnet(F)$|$superconductor(S) heterostructure~\cite{spintronics_2,strong_coupling_3}, where the photon mode with a large mode density is localized in the nano-scale between two superconductors~\cite{Swihart}.


The importance of the dipolar interaction  also manifests in the superconductor gating effect on magnons~\cite{superconductor_gate,superconductor_gate_1,superconductor_gate_2,superconductor_gate_3,similar_theory,superconductor_gating_theory,superconductor_gating_exp}, in which the frequency of magnons with finite wave number~\cite{spin_wave,spin_wave_1,spin_wave_2,spin_wave_3} can be shifted  up to tens of GHz, as recently predicted~\cite{superconductor_gating_theory,similar_theory} and observed~\cite{superconductor_gating_exp} in the superconductor(S)$|$ferromagnet(F) insulator heterostructure.
The stray electric field of magnons drives the supercurrent in the adjacent superconductor which in turn generates the Oersted magnetic field that affects the low-frequency magnetization dynamics. This gating effect favors the spin diode~\cite{Yu_chirality,Yu_chirality_1} and magnon trap~\cite{Chumak_trap,magnon_trap,magnon_trap_1} in proper gating configurations. The FMR frequency in this S$|$F bilayer is not affected, however.


% Figure environment removed

On the other hand, the FMR of the metallic ferromagnet sandwiched by two superconductors  was shifted up to 50~mT in the resonant field when the thickness of two superconductor layers is larger than the London's penetration depth, as observed in several recent experiments~\cite{CPL_exp,PRA_exp,experiment}. Above the superconducting transition temperature, the FMR frequency recovers to the  Kittel mode~\cite{kittel_mode}, which may be exploited to realize the magnetic logic gate through a phase transition in the superconductor.  This phenomenon may be related to the frequency splitting induced by spin-triplet superconducting state~\cite{CPL_exp},  Meissner screening~\cite{experiment}, and giant demagnetization effects~\cite{Gient_de,silaev}.  It appears that this modulation could be absent for the FMR in the ferromagnetic insulators, however, which has not been reported yet~\cite{CPL_exp,experiment,Gient_de,silaev}.
The experiment ~\cite{CPL_exp} showed that inserting a thin insulator layer  in the heterostructures composed of a metallic ferromagnet sandwiched by two superconductors completely suppresses the shift of FMR. This raises the issue of whether the FMR can be gated or not in magnetic insulators by adjacent superconductors in proper configurations.   



In this work, we study this issue by going beyond the quasi-static approximation for magnetostatic modes~\cite{Rezende} and demonstrate that although the stray magnetic field of Kittel magnon with uniform magnetization precession is vanishingly small outside of the in-plane magnetized ferromagnetic insulating film, the radiated electric field is significant with opposite amplitudes at two sides of the magnetic film and polarization parallel to the magnetization direction. This distribution of the radiated electric field is sensitive to the adjacent superconductors due to the total reflection, as illustrated in Fig.~\ref{electric_field} for snapshots of the distribution of electric fields in different heterostructure configurations.  
The electric field is opposite at two sides of a single thin ferromagnetic insulator [Fig.~\ref{electric_field}(a)]; contra-intuitively, in the S$|$F bilayer this electric field is suppressed to vanish at the superconductor side  [Fig.~\ref{electric_field}(b)]; nevertheless, when sandwiched by two superconductors, the electric field is neither shifted to vanish nor screened completely, as plotted in Figs.~\ref{electric_field}(c) and (d) for symmetric and asymmetric configurations. These features are well understood by our mechanism of modulated reflection of magnetization-induced electric fields by superconductors, which predicts the absence of FMR shift in ferromagnetic insulator$|$superconductor heterostructure and the ultrastrong modulation of FMR, shifted up to tens of percent of the bare frequency when the ferromagnetic insulator is sandwiched by two thin superconductors. 

 



This paper is organized as follows. We address the model and general formalism in Sec.~\ref{model_formalism}. In Sec.~\ref{single_ferromagnet}, \ref{bilayer}, and \ref{trilayer}, we analyze the distribution of the electric fields from FMR of a single ferromagnetic insulator, S$|$F bilayer, and S$|$F$|$S heterostructure, respectively, and address the ultrastrong interaction between the FMR and supercurrent. We conclude and discuss in Sec.~\ref{conclusion}. 





 

\section{Model and general formalism}
\label{model_formalism}

We consider a heterostructure composed of a ferromagnetic insulating film of thickness $2d_F\sim O(50~{\rm nm})$ with inplane magnetization sandwiched by two thin superconductor layers with thickness $d_S\lesssim \lambda$ and $d_S'\lesssim \lambda$, respectively, as illustrated in Fig.~\ref{model2}. Here $\lambda\sim O(100~{\rm nm})$ is London's penetration depth of conventional superconductors. In the ferromagnetic insulators, the dynamics of magnetization ${\bf M}=M_x\hat{\bf x}+M_y\hat{\bf y}+M_0\hat{\bf z}$, where $M_0$ is the saturated magnetization, is phenomenologically governed by the Landau-Lifshitz equation~\cite{LLequation}
\begin{align}
    \partial {\bf M}/\partial t=-\mu_0\gamma {\bf M}\times{\bf H}_{\text{eff}},
\end{align}
where $\mu_0$ is the vacuum permeability and $-\gamma$ is the electron gyromagnetic ratio. The magnetization precesses around the effective magnetic field ${\bf H}_{\text{eff}}= {\bf H}_{\text{app}}+{\bf H}_d+{\bf H}_s$ that contains the external static field ${\bf H}_{\text{app}}=H_0 \hat{\bf z}$, the dipolar magnetic field ${\bf H}_d$ generated by the magnetic charge $-\nabla\cdot{\bf M}$,  and the Oersted magnetic field ${\bf H}_s$ from the superconductor that needs a self-consistent treatment with Maxwell's equation~\cite{superconductor_gating_theory}.  At low frequencies, it is sufficient to express the stray magnetic field as~\cite{LLequation,Rezende}
\begin{align}
        {\bf H}_{d,\beta}({\bf M})=\dfrac{1}{4\pi}\partial_\beta\partial_\alpha\int d{\bf r'}\dfrac{M_\alpha(\bf r')}{|\bf r-r'|}.
        \label{H(M)}
\end{align}
The exchange interaction plays no role in the FMR since the gradient of ${\bf M}$ vanishes for the uniform precession.

% Figure environment removed

On the other hand, the oscillating magnetic induction ${\bf B}=\mu_0({\bf M}+{\bf H})$ of frequency $\omega$ governs the radiation of electric fields inside and outside the ferromagnetic insulator according to~\cite{Jackson}
\begin{align}
\nabla\times {\bf E}=i\omega{\bf B},&&
    \nabla\times {\bf H}={\bf J}_s-i\omega\varepsilon_0 {\bf E},
\label{Maxwell_electric_field}
\end{align}
where $\varepsilon_0$ is the vacuum permittivity. When coupled with superconductors, this electric field drives the supercurrent ${\bf J}_s$ via London's equation~\cite{London_equation}  
\begin{align}
     \dfrac{\partial {\bf J}_s}{\partial t}= \dfrac{1}{\mu_0\lambda^2}{\bf E},&&
     \nabla\times {\bf J}_s=-\dfrac{1}{\mu_0\lambda^2}{\bf B}.
     \label{London}
\end{align} 
Here London's penetration depth at different temperatures $T<T_c$ follows the relation~\cite{London_equation}
\begin{align}
    \lambda(T)=\sqrt{\frac{m_e}{\mu_0n_ee^2}}\left(1-\left(\frac{T}{T_c}\right)^4\right)^{-1/2},
\end{align}
where $m_e$ is the electron mass and $n_e$ is the electron density. 


At low frequencies, we may apply the quasi-static approximation $\nabla\times {\bf B}=\mu_0{\bf J}_s$ in superconductors. Taking the curl of Eq.~(\ref{Maxwell_electric_field}) and substituting Eq.~(\ref{London}) into it, the electric field inside the superconductor obeys
    \begin{align}
    \nabla^2 {\bf E}-{\bf E}/\lambda^2=0.
    \label{electric_field_in_s}
\end{align}
On the other hand, taking the curl of  $\nabla\times {\bf B}=\mu_0
{\bf J}_s$ and combining with Eq.~(\ref{London}), the magnetic induction inside the superconductor obeys 
$\nabla^2 {\bf B}-{\bf B}/\lambda^2=0$.
    

The driven supercurrent then affects the magnetization dynamics.
From Eq.~(\ref{London}), the electric field drives supercurrent inside the superconductor, which then generates the vector potential. With the uniform magnetization precession, the system is translational invariant in the $y$-$z$ plane, so the supercurrent only depends on $x$ and as it for the vector potential~\cite{Jackson} 
\begin{align}
  {\bf A}(x)=\dfrac{\mu_0}{4\pi}\int d{\bf r'}\dfrac{{{\bf J}_s}(x')}{|{\bf r}-{\bf r}'|}.
  \label{vector_potential}
\end{align}
Accordingly, the Oersted magnetic field ${\bf H}_s=(1/\mu_0)\nabla\times {\bf A}$ only contains the $y$-component ${H}_y=-\partial_xA_z(x)/\mu_0$, which drives the magnetization. 

 

The boundary condition describes the fields at the interfaces~\cite{Jackson}. For the magnetic induction and field, ${\bf B}_\perp$ and ${\bf H}_\parallel$ are continuous at the boundaries. 
 Since there is no surface current or  charge accumulation, the electric field $\bf E$ is continuous at interfaces. 






\section{Single thin ferromagnetic insulator}

\label{single_ferromagnet}

We start with a single insulating ferromagnetic film to address the significant radiated electric fields from the uniform magnetization precession. For a single ferromagnetic insulator of thickness $2d_F$  biased by a static magnetic field ${\bf H}_{\rm app}=H_0 \hat{\bf z}$,  the magnetization $\bf M$ for the FMR is uniform inside the ferromagnetic layer by the constant demagnetization factor $N_{xx}=-1$. Since the magnetic film is sufficiently thin, we stick to the uniform precession throughout this work.  The opposite magnetic charges at the two surfaces of the film generate opposite magnetic field outside, which results in vanished stray magnetic field ${\bf H}_d=0$ outside the ferromagnetic layer, as also calculated from Eq.~(\ref{H(M)});  inside the ferromagnet, ${\bf H}_d=\{-M_x,0,0\}$ and ${\bf B}=\{0,\mu_0 M_y,\mu_0(H_0+M_0)\}$, in which  only the $y$-component of $\bf B$  oscillates with frequency $\omega$ that can radiate the electric field.

\subsection{Full solution}

Here we go beyond the quasi-static approximation and solve the radiated electric field.
According to Eq.~(\ref{Maxwell_electric_field}), the oscillating electromagnetic field is the source for radiating microwaves in space.  Taking  the curl of the first equation in Eq.~(\ref{Maxwell_electric_field}), the electric field obeys
\begin{align}
    \nabla^2{\bf E}+\varepsilon_0\mu_0\omega^2{\bf E}=-i\omega\mu_0\nabla\times {\bf M},
    \label{single_electric_field}
\end{align}
which has the solution
\begin{align}
    {\bf E}({\bf r})=\dfrac{i\mu_0 \omega}{4\pi}\int \dfrac{[\nabla'\times {\bf M}({\bf r'})]e^{i k|{\bf r-r'}|}}{|{\bf r-r'}|}d{\bf r'},
\end{align}
where $k=\omega\sqrt{\mu_0 \varepsilon_0}$ is the wave number of microwaves. Since only the $x$ and $y$ components of ${\bf M}$ oscillate with frequency $\omega$ and  $\bf M$ is uniform inside the ferromagnetic layer,  $(\nabla\times {\bf M})_{x,y}=0$ in all space, leading to $E_x=E_y=0$ and
\begin{align}
    E_z(x)=\dfrac{i\mu_0 \omega}{4\pi}\int \dfrac{[\partial_{x'}M_y({\bf r'})]e^{i k|{\bf r-r'}|}}{|{\bf r-r'}|}d{\bf r'}.
\end{align}
Using Weyl identity~\cite{Yu_chirality}
\begin{align}
\frac{e^{ik|{\bf r}-{\bf r'}|}}{|{\bf r}-{\bf r}'|}=\int dk_z'dk_y'\frac{ie^{i k_z'(z-z')+i k_y'(y-y')}e^{i\sqrt{k^2-k_z'^2-k_y'^2} |x-x'|}}{2 \pi\sqrt{k^2-k_z'^2-k_y'^2}},
\label{Weyl_identity}
\end{align}
we obtain the electric field 
\begin{align}
      E_z=\dfrac{-\mu_0 \omega M_y}{2 k}\begin{cases}
          e^{ik(x+d_F)}-e^{-ik(x-d_F)},  & -d_F<x<d_F \\
      e^{ik(x+d_F)}-e^{ik(x-d_F)},  &     x>d_F\\
       e^{-ik(x+d_F)}-e^{-ik(x-d_F)},  &     x<-d_F
    \end{cases}.
    \label{full_solution_single_layer}
\end{align}
From Eq.~(\ref{Maxwell_electric_field}), we find the magnetic induction $B_x=0$, $B_z=\mu_0(H_0+M_0)$ is static, and $B_y=-\partial_xE_z/(i\omega)$ follows
\begin{align}
      B_y=\dfrac{\mu_0  M_y}{2}\begin{cases}
    e^{ik(x+d_F)}+e^{-ik(x-d_F)},  & -d_F<x<d_F \\
      e^{ik(x+d_F)}-e^{ik(x-d_F)},  &     x>d_F\\
       -e^{-ik(x+d_F)}+e^{-ik(x-d_F)},  &     x<-d_F
    \end{cases}.
\end{align}


We are interested in the field near the ferromagnet with a distance $\sim \lambda$.
In ferromagnetic insulators, $\omega\sim100$~GHz~\cite{superconductor_gating_theory}, and $\lambda\sim100$~nm for conventional superconductors, so $k\lambda\sim 3\times10^{-5}\ll 1$. 
When $kx\rightarrow0$, we have
\begin{align}
      E_z(x)=\begin{cases}
          -i\mu_0 \omega M_yx,  & -d_F<x<d_F \\
      -i\mu_0 \omega M_yd_F,  &     x>d_F\\
       i\mu_0 \omega M_yd_F,  &     x<-d_F
    \end{cases}, 
    \label{electric_field_quasistatic}
\end{align}
as plotted in Fig.~\ref{electric_field}(a) for a snapshot. The magnetic induction 
\begin{align}
      B_y(x)=\begin{cases}
         \mu_0  M_y,  & -d_F<x<d_F \\
      0,  &     x>d_F\\
      0,  &     x<-d_F
    \end{cases},
\end{align}
recovers to the results from quasi-static approximation~\cite{Rezende} with vanished magnetic field $H_y$ outside of the ferromagnet.


\subsection{Quasi-static approximation}


The above analysis implies that when focusing on the near-field limit, we may apply the quasi-static approximation that sets $\nabla\times {\bf H}=0$ in Eq.~(\ref{Maxwell_electric_field}). 
 When focusing on the FMR case, $\bf E$ is translation invariant in the $y$-$z$ plane. \textit{i.e.,} $\partial_z E_x =0$. Taking the $y$-component of Eq.~(\ref{electric_field}), the oscillation of $B_y$ only generate $E_z$ parallel to the magnetization:
\begin{align}
-\partial_xE_z=i\omega \mu_0 M_y.
\label{electric_field_in_f}
\end{align}
 Integrating along $x$ across the ferromagnet yields
\begin{align}
    E_z(x)=-i\omega \mu_0 M_y(x+d_F)+E_z(x=-d_F).
\end{align}
 Thereby, $E_z$ depends linearly on $x$ inside the ferromagnet. Outside the ferromagnet, 
 \begin{align}
    E_z(x)=-2i\omega \mu_0 M_yd_F+E_z(x=-d_F)
\end{align}
is uniform, which is consistent with the vanished magnetic field $H_{y|\text{outside}}=0$ in the quasi-static approximation.
According to the symmetry, $E_z(x=0)=0$, 
so the electric field
is exactly the same as Eq.~(\ref{electric_field_quasistatic}).
  








\section{S$|$F heterostructure}
\label{bilayer}

We consider the S$|$F heterostructure composed of  a ferromagnetic film of thickness $2d_F$ and a superconductor of thickness $d_S$, as shown in Fig.~\ref{Total_reflection}. We demonstrate the adjacent superconductors modulate strongly the radiated electric field which explains the absence of the FMR shift in this configuration~\cite{superconductor_gating_exp,CPL_exp}.

% Figure environment removed

\subsection{Full solution}

 
Inside the ferromagnet, since $\nabla\times{\bf M}=0$ for uniform ${\bf M}$, Eq.~(\ref{single_electric_field}) has the solution $
E_z(x)=E_1 e^{ik x}+E_1'e^{-i k x}$. Inside the superconductor, according to Eqs.~(\ref{electric_field}) and (\ref{London}), the electric field  obeys 
\begin{align}
    \partial_x^2E_z+(\varepsilon_0\mu_0\omega^2-1/\lambda^2)E_z=0,
\end{align}
which has the solution
$E_z(x)=E_2 e^{ik^\prime x}+E_2'e^{-i k^\prime x}$,
where $k^\prime=\sqrt{\varepsilon_0\mu_0\omega^2-1/\lambda^2}$. Out of the heterostructure, the electric fields $E_3e^{ikx}$ and $E_4e^{-ikx}$ are radiated. These radiated electric fields are illustrated in Fig.~\ref{Total_reflection}. 




The amplitudes $\{E_1,E_1',E_2,E_2',E_3,E_4\}$ are governed by the boundary conditions, i.e., $E_z$ and $H_y$ are continuous at interfaces. The continuous $E_z$ at interfaces requests 
\begin{align}
    &E_1e^{ik d_F}+E_1^\prime e^{-i k d_F}= E_2e^{ik^\prime d_F}+E_2^\prime e^{-i k^\prime d_F},\nonumber\\
    &E_2e^{ik^\prime (d_F+d_S)}+E_2^\prime e^{-i k^\prime (d_F+d_S)}=E_3e^{i k (d_F+d_S)},\nonumber\\
    &E_1e^{-ik d_F}+E_1^\prime e^{i k d_F}=E_4e^{i kd_F}.
    \label{electric_boundary}
\end{align}
In the superconductors, $H_y=-1/(i\omega\mu_0)\partial_xE_z$, while in the ferromagnet, $H_y=-1/(i\omega\mu_0)\partial_xE_z-M_y$, so the continuous $H_y$ at interfaces leads to
\begin{align}
    &{k}(E_1e^{ik d_F}-E_1^\prime e^{-i k d_F})+\omega\mu_0M_y\nonumber\\
    &=k^\prime(E_2e^{ik^\prime d_F}-E_2^\prime e^{-i k^\prime d_F}),\nonumber\\
     &{k^\prime}(E_2e^{ik^\prime (d_F+d_S)}-E_2^\prime e^{-i k^\prime (d_F+d_S)})={k}E_3e^{i k (d_F+d_S)},\nonumber\\
    & k(E_1e^{-ik d_F}-E_1^\prime e^{i k d_F})+\omega\mu_0M_y=-kE_4e^{i kd_F}.
    \label{magnetic_boundary}
\end{align}



Combining Eqs.~(\ref{electric_boundary}) and (\ref{magnetic_boundary}), we obtain all the amplitudes. In the ferromagnetic insulator,  
\begin{align}
    &E_z(-d_F<x<d_F) \nonumber\\
    &={\cal R}E_0e^{-ik(x-d_F)}-\frac{\omega\mu_0M_y}{2k}\left(e^{ik(x+d_F)}-e^{-ik(x-d_F)}\right),
    \label{E_bilayer_in_f}
\end{align}
where the amplitude 
\begin{align}
    E_0=-\frac{\omega\mu_0M_y}{2k}\left(e^{2ik(x+d_F)}-1\right),
\end{align}
 and 
\begin{align}
    {\cal R}=\frac{e^{ik^\prime d_S}(k^2-k^{\prime2})+e^{-ik^\prime d_S}(k^{\prime2}-k^2)}{e^{ik^\prime d_S}(k-k^{\prime})^2-e^{-ik^\prime d_S}(k+k^{\prime})^2}
\end{align}
is the reflection coefficient of the electric field at the F$|$S interface.

When $d_S=0$, ${\cal R}=0$, the solution (\ref{E_bilayer_in_f}) recovers the single layer case (\ref{full_solution_single_layer}). On the other hand, even with a small $d_S\ll \lambda$, since $|k|\ll |k'|$ when $\omega\sim 100$~GHz, ${\cal R}\rightarrow -1$ implies the total reflection of the electric fields at the F$|$S interface even with ultrathin conventional superconductor layer. As shown below, this implies the absence of FMR shift in all the available experiments~\cite{superconductor_gating_exp,CPL_exp}.


Inside the superconductor, 
\begin{align}
    &E_z(d_F<x<d_F+d_S)=\frac{2kE_0}{e^{ik^\prime d_S}(k-k^{\prime})^2-e^{-ik^\prime d_S}(k+k^{\prime})^2}\nonumber\\
    &\times\left((k-k^\prime)e^{-ik^\prime(x-d_F+d_S)}-(k+k^\prime)e^{ik^\prime(x-d_F-d_S)}\right),
\end{align}
which is indeed very weak since $|k|\ll |k'|$.
Out of the heterostructure, 
\begin{align}
    &E_z(x>d_F+d_S)=\frac{-4kk^\prime E_0e^{ik(x-d_F-d_S)}}{e^{ik^\prime d_S}(k-k^{\prime})^2-e^{-ik^\prime d_S}(k+k^{\prime})^2},\nonumber \\
    &E_z(x<-d_F)= {\cal R}E_0e^{-ik(x-d_F)}\nonumber\\
    &-\frac{\omega\mu_0M_y}{2k}(e^{-ik(x+d_F)}-e^{-ik(x-d_F)}).
    \end{align}
      
      
With low frequencies and near the heterostructure, $kx\rightarrow0$, $kd_F\rightarrow 0$, and $kd_S\rightarrow 0$, so the electric fields
\begin{align}
    E_z(x)=\begin{cases}
        0,& x>d_F\\
        -i\omega\mu_0M_y(x-d_F), &-d_F<x<d_F\\
        2i\omega\mu_0M_yd_F, &x<-d_F
    \end{cases},
    \label{full_solution_bilayer}
\end{align}
which is illustrated  in Fig.~\ref{electric_field}(b) for a snapshot.
The electric field vanishes in the superconductor due to the total reflection with a $\pi$-phase shift ${\cal R}=-1$ that generates no supercurrent and thereby leads to no modulation on the FMR.




\subsection{Quasi-static approximation}
\label{quasi_static_bilayer}


The full solution clearly shows the 
absence of electric fields at the superconductor side of the S$|$F heterostructure, which can be well understood within the quasi-static approximation $\nabla\times{\bf H}=0$ or ${\bf J}_s$.  
Assuming $E_z(x=d_F)=\tilde{E}_0$ at the F$|$S interface, according to Eq.~(\ref{electric_field_in_s}) the electric field in the adjacent superconductor
\begin{align}
    E_z(x)=\tilde{E}_0\dfrac{\cosh{((x-d_S-d_F)/\lambda)}}{\cosh{(d_S/\lambda)}}
\end{align}
drives the supercurrents. For a thin superconducting film of thickness $O(\lambda)$, we are allowed to take an average of the supercurrent $J_{s,z}=[J_{s,z}(x=d_F)+J_{s,z}(x=d_F+d_S)]/2$, and from the first equation of Eq.~(\ref{London})
\begin{align}
    J_{s,z}=\dfrac{i}{\mu_0\omega\lambda^2}\tilde{E}_0\dfrac{1+\cosh(d_S/\lambda)}{2\cosh(d_S/\lambda)}.
\end{align}
The supercurrents generate the vector potential (\ref{vector_potential})
and the Oersted magnetic field according to ${H}_y=-\partial_xA_z/\mu_0$. Taking $k=0$ at low frequencies in the Weyl identity (\ref{Weyl_identity}), i.e.,~\cite{Yu_chirality} 
\begin{align} 	\dfrac{1}{|{\bf r}-{\bf r}'|}=\int dk_y'dk_z'\dfrac{e^{i k_y'(y-y')+i k_z'(z-z')}e^{-\sqrt{k_y'^2+k_z'^2} |x-x'|}}{2 \pi\sqrt{k_y'^2+k_z'^2}},
\label{Weyl_identity_2}
\end{align}
we obtain the Oersted magnetic field generated by the supercurrents
\begin{align}
	H_{s,y}(x)=\left\{
	\begin{array}{cc}
	d_S{{ J}}_{s,z}/2,	& x>d_F+d_S \\
	-d_S{{ J}}_{s,z}/2,	&~ x<d_F 
	\end{array}
	\right. .\label{hy-F|S}
\end{align}
However, constant $H_{s,y}$ independent of $x$ should vanish \textit{out of the heterostructure}  within the quasi-static approximation since a constant magnetic field renders the radiated electric field divergent, which requests $J_{s,z}=0$ when $d_S\ne 0$ and $E_z(x>d_F)=0$. Since the electric field is continuous at interfaces, $E_z(x=d_F)=\tilde{E}_0=0$ and according to Eq.~(\ref{electric_field_in_f}) $E_z(x=-d_F)=2id_F\omega \mu_0M_y$. These simple calculations thereby capture precisely the key physics of the full solution (\ref{full_solution_bilayer}).



\section{S$|$F$|$S heterostructure}
\label{trilayer}

Further, we consider the S$|$F$|$S heterostructure as illustrated in  Fig.~\ref{model2} composed of the  ferromagnetic insulator of thickness $2d_F$ and two adjacent superconductor films of thickness $d_S$ and $d_S'$, respectively.
In comparison to that of the S$|$F bilayer, the distribution of the electric field in S$|$F$|$S heterostructure changes much due to its back-and-forth reflection by the superconductors, as addressed in this section.

\subsection{Full solution}


Similar to the S$|$F heterostructure, inside the ferromagnet, $E_z(x)=E_1 e^{ikx}+E_1^\prime e^{-ikx}$; in the superconductor ``1'', $E_z(x)=E_2e^{ik'x}+E_2'e^{-ik'x}$; and in the superconductor ``2'', $E_z(x)=E_3e^{ik'x}+E_3'e^{-ik'x}$. Out of the heterostructure, the electric fields $E_4e^{ikx}$ and $E_5e^{-ikx}$ are radiated. These electric fields are illustrated in Fig.~\ref{SFS_reflection}.

% Figure environment removed

The amplitudes $\{E_1,E_1',E_2,E_2',E_3,E_3',E_4,E_5\}$ are governed by the boundary conditions. The continuous $E_z$ at interfaces requests
\begin{align}
    &E_1e^{ikd_F}+E_1'e^{-ikd_F}=E_2e^{ik'd_F}+E_2'e^{-ik'd_F},\nonumber\\
    &E_1e^{-ikd_F}+E_1'e^{ikd_F}=E_3e^{-ik'd_F}+E_3'e^{ik'd_F},\nonumber\\
    &E_2e^{ik'(d_F+d_S)}+E_2'e^{-ik'(d_F+d_S)}=E_4e^{ik(d_F+d_S)},\nonumber\\
    &E_3e^{-ik'(d_F+d_S')}+E_3'e^{ik'(d_F+d_S')}=E_5e^{ik(d_F+d_S')},
    \label{electric_boundary_SFS}
\end{align}
and the continuous $H_y$ at interfaces leads to 
\begin{align}
    &k(E_1e^{ikd_F}-E_1'e^{-ikd_F})+\omega\mu_0M_y=k'(E_2e^{ik'd_F}-E_2'e^{-ik'd_F}),\nonumber\\
    &k(E_1e^{-ikd_F}-E_1'e^{ikd_F})+\omega\mu_0M_y=k'(E_3e^{-ik'd_F}-E_3'e^{ik'd_F}),\nonumber\\
    &k'(E_2e^{ik'(d_F+d_S)}-E_2'e^{-ik'(d_F+d_S)})=kE_4e^{ik(d_F+d_S)},\nonumber\\
   & k'(E_3e^{-ik'(d_F+d_S')}-E_3'e^{ik'(d_F+d_S')})=-kE_5e^{ik(d_F+d_S')}.
   \label{magnetic_boundary_SFS}
\end{align}

Combining Eqs.~(\ref{electric_boundary_SFS}) and (\ref{magnetic_boundary_SFS}), we obtain the electric-field distribution, referring to Appendix~\ref{appendix} for the general solution. In particular, when $d_S=d_S'$, in the ferromagnetic film,
\begin{align}
    &E_z(-d_F<x<d_F)\nonumber\\
    &=\frac{-\omega\mu_0M_y\sinh{(ikx)}}{k\cosh{(ikd_F)}-k'f(u)\sinh{(ikd_F)}},
\end{align}
where $u=-[(k+k')/(k-k')]\exp(-2ik'd_S)$ and 
\begin{align}
    f(u)=\frac{u-1}{u+1}=\frac{k'\sinh{(ik'd_S)}-k\cosh{(ik'd_S)}}{k\sinh{(ik'd_S)}-k'\cosh{(ik'd_S)}}.
\end{align}
In the superconductor ``1", 
\begin{align}
    &E_z(d_F<x<d_F+d_S)\nonumber\\
    &=\frac{-\omega\mu_0M_y(ue^{ik'(x-d_F)}+e^{-ik'(x-d_F)})}{k(1+u)\coth(ikd_F)-k'(u-1)},
\end{align}
and in the superconductor ``2",
\begin{align}
    &E_z(-d_F-d_S<x<-d_F)\nonumber\\
    &=\frac{\omega\mu_0M_y(ue^{-ik'(x+d_F)}+e^{ik'(x+d_F)})}{k(1+u)\coth(ikd_F)-k'(u-1)}.
\end{align}
They both exist, and $E_z(x=-d_F)$ and $E_z(x=d_F)$ are opposite. 
Out of the heterostructure,
\begin{align}
    &E_z(x>d_F+d_S)\nonumber\\
    &=\frac{-\omega\mu_0M_y(ue^{ik'd_S}+e^{-ik'd_S})}{k(1+u)\coth(ikd_F)-k'(u-1)}e^{ikx},\nonumber\\
    &E_z(x<-d_F-d_S)\nonumber\\
    &=\frac{\omega\mu_0M_y(ue^{ik'd_S}+e^{-ik'd_S})}{k(1+u)\coth(ikd_F)-k'(u-1)}e^{-ikx}.
\end{align}

We illustrate in Fig.~\ref{electric_distribution} the distribution of the electric fields ${\rm Re}(E_z/(i \omega\mu_0M_yd_F))$ in the symmetric $d_S'=d_S=30$~nm and asymmetric $d_S'=2d_S=60$~nm S$|$F$|$S heterostructure, respectively, in the near-field limit. For  NbN with electron density $n_e=1.65\times10^{28}$/m$^3$~\cite{cooper_pair_density} and $T_c=6.5$~K, the London penetration depth $\lambda(T=0)=42.0$~nm and $\lambda(T=0.5T_c)=43.4$~nm. The fields are opposite at the two superconductors in the symmetric heterostructure but are skewed when $d_S\ne d_S'$. These fields carrying energy are radiated out in the far zone~\cite{Jackson}. When the superconductors are sufficiently thick $\{d_S,d_S'\}\gg \lambda$, these electric fields are confined between them, which corresponds to an excellent waveguide with small size~\cite{Swihart}. 

% Figure environment removed


\subsection{Quasi-static approximation}

As justified, the quasi-static approximation $\nabla\times {\bf H}=0$ or ${\bf J}_s$ is allowed when solving the electric fields \textit{near} the heterostructure~\cite{Jackson}.
In the FMR case, the radiated electric field is uniform in the $y$-$z$ plane, so from $\nabla\times {\bf E}=i \omega {\bf B}$, the $x$-component  $B_x=H_{d,x}+M_x=0$ generates no electric field outside the magnet. On the other hand, in the linear response regime $B_z=\mu_0 (H_0+M_z)$ is static, so only $B_y=\mu_0 M_y$ in the magnet radiates the time-dependent electric field according to
 $-\partial_xE_z=i\omega \mu_0 (M_y+H_{s,y})$.  Integrating along $x$ across the ferromagnet yields the net electric field at the interfaces obeying
\begin{align}
E_z(x=d_F)-E_z(x=-d_F)=-2d_F i \omega \mu_0 (M_y+H_{s,y}).
\label{electric-field}
\end{align}  
Out of the heterostructure, from the $z$-component of $\nabla\times {\bf H}=0$, $H_y|_{\rm outside}$ is a constant, which can be proved to vanish as in Sec.~\ref{quasi_static_bilayer}. 


In the quasi-static approximation, the  electric field in the superconductors $``1"$ and $``2"$ obeys Eq.~(\ref{electric_field_in_s}). 
From the boundary conditions with  continuous $E_z$
and $H_y$ at interfaces and $H_y|_{\text{outside}}=0$, the electric field in the superconductors reads
 \begin{align}
 &E_z(d_F<x<d_F+d_S)\nonumber\\
 &=E_z(x=d_F)\dfrac{\cosh((x-d_S-d_F)/\lambda)}{\cosh(d_S/\lambda)},\nonumber\\
 &E_z(-d_F-d_S<x<-d_F)\nonumber\\
 &=E_z(x=-d_F)\dfrac{\cosh((x+d_S'+d_F)/\lambda)}{\cosh(d_S'/\lambda)},
 \end{align}  
 which drive the supercurrents in the adjacent superconductors.  For thin superconducting films of thickness $O({\lambda})$, we are allowed to take an average of the supercurrents ${\bf J}^{(1)}_s=\left[{\bf J}_s(x=d_F)+{\bf J}_s(x=d_F+d_S)\right]/2 $ and $ {\bf J}^{(2)}_s=\left[{\bf J}_s(x=-d_F)+{\bf J}_s(x=-d_F-d_S')\right]/2 $, i.e.,
\begin{align}
 &{{ J}}_{s,z}^{(1)}=\dfrac{i}{\omega \mu_0\lambda^2}E_z(x=d_F)\dfrac{1+\cosh(d_S/\lambda)}{2\cosh(d_S/\lambda)},\nonumber\\
 &{{ J}}^{(2)}_{s,z}=\dfrac{i}{\omega \mu_0\lambda^2}E_z(x=-d_F)\dfrac{1+\cosh(d_S'/\lambda)}{2\cosh(d_S'/\lambda)}.\label{current}
\end{align}  


The supercurrents generate the vector potential (\ref{vector_potential}) and 
the Oersted magnetic field according to ${H}_{s,y}=-\partial_xA_z/\mu_0$.
Using the Weyl identity (\ref{Weyl_identity_2}) we obtain
\begin{align}
	{H_{s,y}}(x)=\left\{
	\begin{array}{cc}
	\left(d_S{{ J}}^{(1)}_{s,z}+d_S'{{ J}}^{(2)}_{s,z}\right)/2,	& x>d_F+d_S \\
	\left(-d_S{{ J}}^{(1)}_{s,z}+d_S'{{ J}}^{(2)}_{s,z}\right)/2,	&~ -d_F<x<d_F \\
\left(-d_S{{ J}}^{(1)}_{s,z}-d_S'{{ J}}^{(2)}_{s,z}\right)/2,	&~~ x<-d_F-d_S'
	\end{array}
	\right. .
\end{align}
 $H_{s,y}|_{\text{outside}}=0 $ requests
\begin{align}
	d_S{{ J}}^{(1)}_{s,z}+d_S'{{ J}}^{(2)}_{s,z}=0
 \label{out_H},
\end{align}  
so the Oersted magnetic field inside the ferromagnetic slab is reduced to 
\begin{align}
{H}_{s,y}(-d_F<x<d_F)=d_S'{J}_{s,z}^{(2)}=-d_SJ_{s,z}^{(1)}.\label{in_H}
\end{align} 
Thereby, when $d_S=d_S'$, the supercurrents are opposite in the two superconductors.
When $d_S'\rightarrow 0$, $H_{s,y}$ vanishes in the magnet.


Substituting Eqs.~(\ref{current}) and (\ref{electric-field}) into (\ref{out_H}), we obtain the electric field at the surface of the ferromagnetic film:
   \begin{align}
   \nonumber
&E_z(x=-d_F)=i \mu_0\omega d_Sd_F (M_y+H_{s,y}) \dfrac{\cosh(d_S/\lambda)+1}
{\cosh(d_S/\lambda)}\\
&\times \left(
\dfrac{d_S(\cosh(d_S/\lambda)+1)}{2\cosh(d_S/\lambda)}+\dfrac{d_S'(\cosh(d_S'/\lambda)+1)}{2\cosh(d_S'/\lambda)}
\right)^{-1}.
\end{align}  
Substituting it into Eq.~(\ref{in_H}), the Oersted magnetic field in the ferromagnetic film
\begin{align}
{H}_{s,y}(-d_F<x<d_F)&= -M_y\dfrac{ d_Fd_S'd_S G(d_S,d_S',\lambda)}{\lambda^2+d_Fd_S'd_S G(d_S,d_S',\lambda)}  , 
\label{H_y}                     
\end{align} 
where 
    \begin{align}
    \nonumber
&G(d_S,d_S',\lambda) =\dfrac{(\cosh(d_S/\lambda)+1)}{\cosh(d_S/\lambda)}\dfrac{(\cosh(d_S'/\lambda)+1)}{\cosh(d_S'/\lambda)}\\
&\times\left(
\dfrac{d_S(\cosh(d_S/\lambda)+1)}{\cosh(d_S/\lambda)}+\dfrac{d_S'(\cosh(d_S'/\lambda)+1)}{\cosh(d_S'/\lambda)}\right)^{-1}.
\end{align} 
These results capture precisely the key physics of the full solution and are convenient for the calculation of the interaction between Kittel magnon and Cooper pairs.


\subsection{Ultrastrong interaction between Kittel magnon and Cooper pairs}

Above we address that the dynamics of magnetization ${\bf M}$  generates $H_{s,y}$ via the backaction of superconductors, which, in turn, drives ${\bf M}$ in the ferromagnet, imposing a self-consistent problem that is solved by combining the Landau-Lifshitz and Maxwell's equations.


In the linear regime, the Landau-Lifshitz equation
\begin{align}
    &-i\omega M_x+\mu_0\gamma M_yH_0=\mu_0\gamma M_0H_{s,y},\nonumber\\
    &i\omega M_y+\mu_0\gamma M_x H_0=\mu_0\gamma M_0H_{d,x}.\label{LLG-equation}
		\end{align}
Substituting $B_x=M_x+H_{d,x}=0$ into Eq.~(\ref{LLG-equation}),  $M_y$ relates to $H_{s,y}$ via 
\begin{align}
M_y=\dfrac{\mu_0^2\gamma^2M_0(H_0+M_0)}{\mu_0^2\gamma^2H_0(H_0+M_0)-\omega^2}H_{s,y}.\label{M_y}
\end{align}
    When $d_S'\rightarrow0$, $H_{s,y}=0$ according to Eq.~(\ref{H_y}), and the FMR frequency recovers the Kittel formula $\tilde{\omega}_{\rm K}=\mu_0\gamma\sqrt{H_0(H_0+M_0)}$~\cite{kittel_mode}. With finite $d_S$ and $d_S'$, the FMR frequency is self-consistently solved via combining Eqs.~(\ref{H_y}) and (\ref{M_y}), leading to the modified FMR frequency 
\begin{align}
	&\omega_{\rm K}=\mu_0 \gamma\nonumber
 \\ &\times\sqrt{\frac{\lambda^2 H_0(H_0+M_0)+d_Sd_S'd_FG(d_S,d_S',\lambda)(H_0+M_0)^2}{d_Sd_S'd_FG(d_S,d_S',\lambda)+\lambda^2}}.
\end{align}
In particular, when $ d_S=d_S' $, 
  \begin{align}
\omega_{\rm K}&= {\mu_0 \gamma}\left(\dfrac{2\lambda^2 \cosh{(d_S/\lambda)}H_0(H_0+M_0)}
{d_Sd_F\left(\cosh{(d_S/\lambda)}+1\right)+2\lambda^2\cosh{(d_S/\lambda)}}\right.\nonumber\\
&\left.+\dfrac{d_Sd_F(\cosh{(d_S/\lambda)}+1)(H_0+M_0)^2}{d_Sd_F\left(\cosh{(d_S/\lambda)}+1\right)+2\lambda^2\cosh{(d_S/\lambda)}}\right)^{1/2}.
\label{FMR_shifted}
\end{align}  
Approaching $T_c$, $\lambda\rightarrow \infty$, $\cosh(d_S/\lambda)\rightarrow 1$, so the FMR frequency (\ref{FMR_shifted}) recovers the Kittel formula $\omega_{\rm K}\rightarrow \tilde{\omega}_{\rm K}$; otherwise $T<T_c$, it is shifted.



To show the FMR shift, we assume an oscillating magnetic field 
$\tilde{H}e^{-i\omega_0 t}\hat{\bf y}$ of frequency $\omega_0$ applied along the $\bf \hat{y}$-direction (the associated microwave electric field is  along the normal $\hat{\bf x}$-direction). 
The wavelength of this microwave is much larger than the thickness of the heterostructure, so it can be treated as uniform across the heterostructure thickness. 
It can penetrate the superconductor easily when $\{d_S,d_S'\}\sim \lambda$. With the wave vector$\parallel \hat{\bf z}$ parallel to the film, it only excites $\bf M$ in the ferromagnet but does not drive the superconductor. 


Including the external pump field $\tilde{H}e^{-i \omega_0 t}\hat{\bf y}$ and $H_{s,y}\hat{\bf y}$ from the superconductor (\ref{H_y}) into ${\bf H}_{\rm eff}$ and incorporating the Gilbert damping $\alpha_G$, the linearized Landau-Lifshitz-Gilbert equation reads 
\begin{align}
	-i\omega_0 M_x+\mu_0 \gamma M_y H_0&=\mu_0 \gamma M_0H_{\text{eff},y}+i \alpha_G \omega_0 M_y,\nonumber\\
	\mu_0 \gamma H_0 M_x+i\omega_0 M_y &=\mu_0 \gamma M_0 H_{\text{eff},x}+i\alpha_G \omega_0 M_x,
\end{align}
from which we solve with $\alpha_G\ll 1$
\begin{align}
M_y&=\dfrac{\mu_0^2\gamma^2M_0(H_0+M_0)}{\omega_{\rm K}^2-\omega_0^2-i \Gamma}\tilde{H},
	\nonumber\\
	M_x&=-i M_y\left[ \dfrac{\omega_0}{\mu_0\gamma(H_0+M_0)}+\dfrac{i \alpha_G \omega_0^2}{(\mu_0\gamma(H_0+M_0))^2}\right],
\end{align}
where 
\begin{align}
	\Gamma&=\dfrac{\alpha_G \omega_0(\mu_0^2	\gamma^2(H_0+M_0)^2+
	\omega_0^2)}{\mu_0\gamma(H_0+M_0)}.
\end{align}
The Oersted field in the thin magnetic film induced by the supercurrent
\begin{align}
	H_{s,y}=&-\tilde{H}\dfrac{\mu_0^2\gamma^2M_0(H_0+M_0)}{\omega_{\rm K}^2-\omega_0^2-i \Gamma}\dfrac{d_Fd_Sd_S'G(d_S,d_S',\lambda)}{\lambda^2+d_Sd_Fd_S'G(d_S,d_S',\lambda)}.
\end{align} 
The average supercurrent density in (one of) the thin superconductor
\begin{align}
J_s^{(1)}=\tilde{H}\dfrac{\mu_0^2\gamma^2M_0(H_0+M_0)}{\omega_{\rm K}^2-\omega_0^2-i \Gamma}\dfrac{d_Fd_S'G(d_S,d_S',\lambda)}{\lambda^2+d_Sd_Fd_S'G(d_S,d_S',\lambda)}.
\end{align}
The average electric field $E_z$ in (one of) 
 the superconductors
\begin{align}
    E^{(1)}_z=-i \tilde{H} \dfrac{\mu_0^3\gamma^2M_0(H_0+M_0)}{\omega_{\rm K}^2-\omega_0^2-i \Gamma}\dfrac{\omega_0 \lambda^2d_Fd_S'G(d_S,d_S',\lambda)}{\lambda^2+d_Sd_Fd_S'G(d_S,d_S',\lambda)}.
\end{align}


Here we illustrate the numerical results considering a yttrium iron garnet (YIG) film of thickness $2d_F=60$~nm sandwiched by two NbN superconductors of thickness $d_S=d_S'=30$~nm.
Insulating EuS thin magnetic film~\cite{EuS_thin_film,thin_film} is also a possible candidate to test our prediction. For YIG, $\mu_0M_0=0.2$~T and $\alpha_G=5\times 10^{-4}$~\cite{magnon_conductivity,YIG_parameter}. We use $\lambda(T=0.5T_c)=43.4$~nm for  NbN~\cite{cooper_pair_density} .  
We take the bias field $\mu_0H_0=0.05$~T and the excitation field $\mu_0\tilde{H}=0.01$~mT.
Figure~\ref{shift_results} shows the radiated electric field in (one of) the superconductors and the excited amplitudes of $\bf M$ as a function of the excitation frequency $\omega_0$. The frequency shift is $2\pi\times 1.6$~GHz, comparable to half of the bare FMR frequency $\tilde{\omega}_{\rm K}=2\pi\times 3.2$~GHz, corresponding to the decrease of the resonant magnetic field as large as 350~mT. This demonstrates the potential to achieve ultrastrong interaction between magnons and Cooper pairs even with magnetic insulators.




% Figure environment removed



\section{Conclusion and discussion}
\label{conclusion}

Magnetic insulators are ideal candidates for long-range spin transport~\cite{Lenk,Chumak,Grundler,Demidov,Brataas,Barman,Yu_chirality}, strong coupling between magnons and microwaves~\cite{Q_information_4}, and quantum information processing~\cite{Q_information,Q_information_1,Q_information_5}, gating which by superconductors may bring new control dimensions.
In comparison to metallic magnets, the mutual proximity effect may differ between magnetic insulators and superconductors, which may be helpful to distinguish different competitive mechanisms~\cite{Bergeret} in the future studies. Our model system differs from the \textit{metallic} ferromagnets since there are no electric currents flowing in the insulators that, if large, may affect the field distribution via radiation. Our theory  may apply to the antiferromagnet as well, in which case we need to replace the characterized frequency with terahertz ones.
  

    
In conclusion, we analyze the interaction between the Kittel magnons in insulating magnetic film and Cooper pairs in superconductors mediated by the radiated electric fields from the magnetization dynamics. Via highlighting the role of the total reflection of the electric fields at the ferromagnet-superconductor interface that are solved beyond the quasi-static approximation, we provide a comprehensive understanding of the absence of the FMR shift in the F$|$S heterostructure and predict its existence in the S$|$F$|$S heterostructure with the Meissner screening. The coupling between magnons and Cooper pairs is ultrastrong with the frequency shift achieving tens of percent of the bare FMR frequency, which may bring superior advantage in information processing in on-chip magnonics and quantum magnonics.  






\begin{acknowledgments}
We gratefully acknowledge Prof.~Guang Yang and Prof.~Lihui Bai 
for many inspiring discussions.
This work is financially supported by the National Natural Science Foundation of China, and the startup grant of Huazhong University of Science and Technology (Grants  No.~3004012185 and 3004012198). 
\end{acknowledgments}

\begin{appendix}
   

\section{General solution of $E_z$ in S$|$F$|$S heterostructure}
 \label{appendix}

Here we list the general solution of $E_z(x)$ in the S$|$F$|$S heterostructure when $d_S\ne d_S'$ in Fig.~\ref{SFS_reflection}. Inside the ferromagnet,
\begin{align}
    &E_z(-d_F<x<d_F)\nonumber\\
    &=\frac{-\omega\mu_0M_y(Ge^{ikx}+e^{-ikx})}{k(Ge^{ikd_F}-e^{-ikd_F})-k'f(u)(Ge^{ikd_F}+e^{-ikd_F}) },
\end{align}
    where 
    \begin{align}
        G=-\frac{-2k\sinh(ikd_F)+k'(f(u)e^{-ikd_F}+f(u')e^{ikd_F})  
        }{-2k\sinh(ikd_F)+k'(f(u)e^{ikd_F}+f(u')e^{-ikd_F})},
    \end{align}
    and $u'=-[(k+k')/(k-k')]\exp(-2ik'd_S')$.
In the superconductor ``1'',
\begin{align}
    E_z(d_F<x<d_F+d_S)=\frac{ue^{ik'(x-d_F)}+e^{-ik'(x-d_F)}}{1+u}\nonumber\\
    \times \frac{-\omega\mu_0M_y(Ge^{ikd_F}+e^{-ikd_F})}{
    k(Ge^{ikd_F}-e^{-ikd_F})-k'f(u)(Ge^{ikd_F}+e^{-ikd_F})
    }.
\end{align}
In the superconductor ``2'',
\begin{align}
    &E_z(-d_F-d_S'<x<-d_F)=\frac{e^{ik'(x+d_F)}+u'e^{-ik'(x+d_F)}}{1+u'}\nonumber\\
    &\times \frac{-\omega\mu_0M_y(Ge^{-ikd_F}+e^{ikd_F})}{
    k(Ge^{ikd_F}-e^{-ikd_F})-k'f(u)(Ge^{ikd_F}+e^{-ikd_F})
    }.
\end{align}
Out of the heterostructure,
\begin{align}
    &E_z(x>d_F+d_S)=\frac{ue^{ik'd_S}+e^{-ik'd_S}}{1+u}\nonumber\\
    &\times \frac{-\omega\mu_0M_y(Ge^{ikd_F}+e^{-ikd_F})e^{ik(x-d_F-d_S)}}{
    k(Ge^{ikd_F}-e^{-ikd_F})-k'f(u)(Ge^{ikd_F}+e^{-ikd_F})
    },\nonumber\end{align}
    \begin{align}
        & E_z(x<-d_F-d_S')=\frac{e^{-ik'd_S'}+u'e^{ik'd_S'}}{1+u'}\nonumber\\
    &\times \frac{-\omega\mu_0M_y(Ge^{-ikd_F}+e^{ikd_F})e^{-ik(x+d_F+d_S)}}{
    k(Ge^{ikd_F}-e^{-ikd_F})-k'f(u)(Ge^{ikd_F}+e^{-ikd_F})
    }. 
    \end{align}
  



\end{appendix}


\begin{thebibliography}{99}

%\textcolor{blue}{(some citation:metal affect YIG damping~\cite{metal_yig,metal_yig_1}, magnon conductivity~\cite{magnon_conductivity}, metal gate effect~\cite{metal_gate_1,metal_gate_2,metal_gate_3,metal_gate_4,metal_gate_5})}

%magnonic
\bibitem{Lenk} B. Lenk, H. Ulrichs, F. Garbs, and M. M\"{u}nzenberg, The building blocks of magnonics, Phys. Rep. \textbf{507}, 107 (2011).



\bibitem{Chumak} A. V. Chumak, V. I. Vasyuchka, A. A. Serga, and B. Hillebrands, Magnon spintronics, Nat. Phys. \textbf{11}, 453 (2015).


\bibitem{Grundler} D. Grundler, Nanomagnonics around the corner, Nat. Nanotechnol. \textbf{11}, 407 (2016).

\bibitem{Demidov} V. E. Demidov, S. Urazhdin, G. de Loubens, O. Klein, V. Cros, A. Anane, and S. O. Demokritov, Magnetization oscillations and waves driven by pure spin currents, Phys. Rep. \textbf{673}, 1 (2017).

\bibitem{Brataas} A. Brataas, B. van Wees, O. Klein, G. de Loubens, and M. Viret, Spin Insulatronics, Phys. Rep. \textbf{885}, 1 (2020).

\bibitem{Barman} Barman \textit{et al.}, The 2021 Magnonics Roadmap, J. Phys. Condens. Matter \textbf{33}, 413001 (2021).

\bibitem{Yu_chirality} T. Yu, Z. C. Luo, and G. E. W. Bauer, Chirality as generalized spin–orbit interaction in spintronics, Phys. Rep. \textbf{1009}, 1 (2023).





%........superconductor gating effect.............

\bibitem{spintronics_1} O. V. Dobrovolskity, R. Sachser, T. Br\"acher,  T. B\"ottcher, V. V. Kruglyak, R. V. Vovk, V. A. Shklovskij,  M. Huth, B. Hillebrands, and A. V. Chumak, Magnon-fluxon interaction in a ferromagnet/superconductor heterostructure, Nat. Phys. \textbf{15}, 477 (2019).

\bibitem{Wei_Han} Y. Yao, R. Cai, T. Yu, Y. Ma, W. Xing, Y. Ji, X.-
C. Xie, S.-H. Yang, and W. Han, Giant oscillatory Gilbert damping in superconductor/ferromagnet/superconductor junctions, Sci. Adv. \textbf{7}, eabh3686 (2021).

\bibitem{Linder} L. G. Johnsen, H. T. Simensen, A. Brataas, and J. Linder, Magnon Spin Current Induced by Triplet Cooper Pair Supercurrents, Phys. Rev. Lett. \textbf{127}, 207001 (2021).

\bibitem{superconductor_gating_theory} T. Yu and G. E. W. Bauer, Efficient Gating of Magnons by Proximity Superconductors, Phys. Rev. Lett. \textbf{129}, 117201 (2022).

\bibitem{similar_theory} M. A. Kuznetsov and A. A. Fraerman, Temperature-sensitive spin-wave nonreciprocity induced by interlayer dipolar coupling in ferromagnet/paramagnet and ferromagnet/superconductor hybrid systems, Phys. Rev. B \textbf{105}, 214401 (2022).

\bibitem{silaev} M. Silaev, Anderson-Higgs Mass of Magnons in Superconductor-Ferromagnet-Superconductor Systems, Phys. Rev. Appl. \textbf{18}, L061004 (2022).


\bibitem{Belzig} I. V. Bobkova, A. M. Bobkov, A. Kamra, and W. Belzig, Magnon-cooparons in magnet-superconductor hybrids, Commun. Mater. \textbf{3}, 95 (2022).

\bibitem{Bobkova} A. M. Bobkov, S. A. Sorokin, and I. V. Bobkova, Phys. Rev. B \textbf{107}, 174521 (2023).





\bibitem{FI/s_system} A. S. Ianovskaia, A. M. Bobkov, and I. V. Bobkova, Magnon influence on the superconducting DOS in FI/S bilayers, arXiv:2307. 03954.

\bibitem{superconductor_gating_exp} M. Borst, P. H. Vree, A. Lowther, A. Teepe, S. Kurdi, I. Bertelli, B. G. Simon, Y. M. Blanter, and T. van der Sar, Observation and control of hybrid spin-wave–Meissner-current transport modes, arXiv:2307.07581.





%..magnon-phonton.............

\bibitem{Janssonn_1} A. T. G. Janss\o{}nn, H. T. Simensen, A. Kamra, A. Brataas, and S. H. Jacobsen, Macroscale nonlocal transfer of superconducting signatures to a ferromagnet in a cavity, Phys. Rev. B \textbf{102}, 180506(R) (2020).


\bibitem{spintronics_2} I. A. Golovchanskiy, N. N. Abramov, V. S. Stolyarov, M. Weides, V. V. Ryazanov, A. A. Golubov, A. V. Ustinov, and  M. Y. Kupriyanov, Ultrastrong photon-to-magnon coupling in multilayered heterostructures involving superconducting coherence via ferromagnetic layers, Sci. Adv. \textbf{7}, eabe8638 (2021).





\bibitem{strong_coupling_3} I. A. Golovchanskiy, N. N. Abramov, V. S. Stolyarov, A. A. Golubov, M. Y. Kupriyanov, V. V. Ryazanov, and A. V. Ustinov, Approaching Deep-Strong On-Chip Photon-To-Magnon Coupling, Phys. Rev. Appl. \textbf{16}, 034029 (2021).

\bibitem{Silaev_ultrastrong_coupling} M. Silaev, Ultrastrong magnon-photon coupling, squeezed vacuum, and entanglement in superconductor/ferromagnet nanostructures, Phys. Rev. B \textbf{107}, L180503 (2023).


\bibitem{Janssonn_2} A. T. G. Janss\o{}nn, H. G. Hugdal, A. Brataas, and S. H. Jacobsen, Cavity-mediated superconductor–ferromagnetic-insulator coupling, Phys. Rev. B \textbf{107}, 035147 (2023).

\bibitem{ultrastrong_in_press}
A. Ghirri, C. Bonizzoni, M. Maksutoglu, A. Mercurio, O. D. Stefano, S. Savasta, and M. Affronte, Ultrastrong magnon-photon coupling achieved by magnetic films in contact with superconducting resonators, Phys. Rev. Appl.  (2023).

%triplet_Cooper_pairing................



\bibitem{study_1} M. Eschrig, Spin-polarized supercurrents for spintronics: a review of current progress, Rep. Prog. Phys. \textbf{78}, 104501 (2015).

\bibitem{study_2} J. Linder and J. W. A. Robinson, Superconducting spintronics, Nat. Phys. \textbf{11}, 307 (2015).

\bibitem{Bergeret} F. S. Bergeret, M. Silaev, P. Virtanen, and T. T. Heikkil\"a, Colloquium: Nonequilibrium effects in superconductors with a spin-splitting field,
Rev. Mod. Phys. \textbf{90}, 041001 (2018).

\bibitem{Balatsky} J. Linder and A. V. Balatsky, Odd-frequency superconductivity, Rev. Mod. Phys. \textbf{91}, 045005 (2019).

\bibitem{Banerjee} M. Amundsen, J. Linder, J. W. A. Robinson, I. \v{Z}uti\'{c}, and N. Banerjee, Colloquium: Spin-orbit effects in superconducting hybrid structures, 	arXiv: 2210.03549.

%.....quantum-information..............
\bibitem{Q_information} Y. Tabuchi, S. Ishino, A. Noguchi, T. Ishikawa,   R. Yamazaki,  K. Usami, and  Y. Nakamura, Coherent coupling between a ferromagnetic magnon and a superconducting qubit, Science \textbf{349}, 405 (2015).

\bibitem{Q_information_1} D. L. Quirion, S. P. Wolski, Y. Tabuchi, S. Kono, K. Usami, and Y. Nakamura, Entanglement-based single-shot detection of a single magnon with a superconducting qubit, Science \textbf{ 367}, 425 (2020).

\bibitem{Q_information_2} T. Yu, M. Claassen, D. M. Kennes, and M. A. Sentef, Optical manipulation of domains in chiral topological superconductors, Phys. Rev. Research \textbf{3}, 013253 (2021).




\bibitem{Q_information_4} B. Z. Rameshti, S. V. Kusminskiy, J. A. Haigh, K. Usami, D. Lachance-Quirion, Y. Nakamura, C. -M. Hu, H. X. Tang, G. E. W. Bauer, and Y. M. Blanter, Cavity magnonics, Phys. Rep. \textbf{979}, 1 (2022).

\bibitem{Q_information_5} D. Xu, X. -K. Gu, H. -K. Li, Y. -C. Weng, Y. -P. Wang, J. Li, H. Wang, S. -Y. Zhu, and J. Q. You, Quantum Control of a Single Magnon in a Macroscopic Spin System, Phys. Rev. Lett. \textbf{130}, 193603 (2023).



\bibitem{topological} K. M. D. Hals, M. Schecter, and M. S. Rudner, Composite Topological Excitations in Ferromagnet-Superconductor Heterostructures,
Phys. Rev. Lett. \textbf{117}, 017001 (2016).

\bibitem{strong_coupling} A. F. Kockum, A. Miranowicz, S. D. Liberato, S. Savasta, and  F. Nori, Ultrastrong coupling between light and matter, Nat. Rev. Phys. \textbf{1}, 19 (2019).


\bibitem{Swihart} J. C. Swihart, Field Solution for a Thin‐Film Superconducting Strip Transmission Line, J. Appl. Phys. \textbf{32}, 461 (1961).




\bibitem{superconductor_gate} I. A. Golovchanskiy, N. N. Abramov, V. S. Stolyarov, V. V. Bolginov, V. V. Ryazanov, A. A. Golubov, and A. V.
Ustinov,  Ferromagnet/Superconductor Hybridization for Magnonic Applications,  Adv. Funct. Mater. \textbf{28}, 1802375 (2018).

\bibitem{superconductor_gate_1} I. A. Golovchanskiy, N. N. Abramov, V. S. Stolyarov, V. V. Ryazanov, A. A. Golubov, and A. V. Ustinov, Modified dispersion law for spin waves coupled to a superconductor, J. Appl. Phys.
\textbf{124}, 233903 (2018).

\bibitem{superconductor_gate_2}  I. A. Golovchanskiy, N. N. Abramov, V. S. Stolyarov, P. S. Dzhumaev, O. V. Emelyanova, A. A. Golubov, V. V. Ryazanov, and A. V. Ustinov, Ferromagnet/Superconductor Hybrid Magnonic Metamaterials, Adv. Sci. \textbf{6}, 1900435 (2019).

\bibitem{superconductor_gate_3} I. A. Golovchanskiy, N. N. Abramov, V. S. Stolyarov, A. A. Golubov, V. V. Ryazanov, and A. V. Ustinov, Nonlinear spin waves in ferromagnetic/superconductor hybrids, J. Appl. Phys. \textbf{127}, 093903 (2020).


%........spin wave................

\bibitem{spin_wave}  Seshadri, Surface magnetostatic modes of a ferrite slab, Proc. IEEE \textbf{58}, 506 (1970).

\bibitem{spin_wave_1} C. Bayer, J. Jorzick, B. Hillebrands, S. O. Demokritov, R. Kouba, R. Bozinoski, A. N. Slavin, K. Y. Guslienko, D. V.
Berkov, N. L. Gorn, and M. P. Kostylev, Spin-wave excitations in finite rectangular elements of ${\mathrm{Ni}}_{80}{\mathrm{Fe}}_{20}$, Phys. Rev. B \textbf{72},
064427 (2005).

\bibitem{spin_wave_2} T. Yu, C. P. Liu, H. M. Yu, Y. M. Blanter, and G. E. W.
Bauer, Chiral excitation of spin waves in ferromagnetic films by magnetic nanowire gratings, Phys. Rev. B \textbf{99}, 134424 (2019).

\bibitem{spin_wave_3}  T. Yu, Y. M. Blanter, and G. E. W. Bauer, Chiral Pumping of Spin Waves, Phys. Rev. Lett.
\textbf{123}, 247202 (2019).

%\bibitem{spin_wave_4} J. L. Chen, T. Yu, C. P. Liu, T. Liu, M. Madami, K. Shen, J. Y. Zhang, S. Tu, M. S. Alam, K. Xia, M. Z. Wu, G. Gubbiotti, Y. M. Blanter, G. E. W. Bauer, and H. M. Yu, Excitation of unidirectional exchange spin waves by a nanoscale magnetic grating, Phys. Rev. B \textbf{100}, 104427 (2019).


%spin-diode+magnon-trap.................


\bibitem{Yu_chirality_1} T. Yu, J. Zou, B. Zeng, J. W. Rao, and K. Xia, Non-Hermitian Topological Magnonics, 	arXiv:2306.04348.

\bibitem{Chumak_trap} A. V. Chumak, A. A. Serga, and B. Hillebrands, Nat. Commun.
\textbf{5}, 4700 (2014).


\bibitem{magnon_trap} T. Yu, H. Wang, M. A. Sentef, H. Yu, and G. E. W. Bauer, Magnon trap by chiral spin pumping, Phys. Rev. B \textbf{102}, 054429 (2020).

\bibitem{magnon_trap_1} O. A. Santos and B. J. van Wees, Magnon confinement in an all-on-chip YIG cavity resonator using hybrid YIG/Py magnon barriers, arXiv:2306.14029.



%.....FMR-experimrnt+explan...........
\bibitem{CPL_exp}L. -L. Li, Y. -L. Zhao, X. -X. Zhang, and Y. Sun, Possible evidence for spin-transfer torque induced by spin-triplet supercurrents, Chin. Phys. Lett. \textbf{35}, 077401 (2018).



\bibitem{PRA_exp} I. A. Golovchanskiy, N. N. Abramov, V. S. Stolyarov, V. I. Chichkov, M. Silaev, I. V. Shchetinin, A. A. Golubov, V. V. Ryazanov, A. V. Ustinov, and M. Y. Kupriyanov, Magnetization Dynamics in Proximity-Coupled Superconductor-Ferromagnet-Superconductor Multilayers, Phys. Rev. Appl. \textbf{14}, 024086 (2020).

\bibitem{experiment} K. -R. Jeon, C. Ciccarelli, H. Kurebayashi, L. F. Cohen, X. Montiel, M. Eschrig, T. Wagner, S. Komori, A. Srivastava, J. W. A. Robinson, and M. G. Blamire, Effect of Meissner Screening and Trapped Magnetic Flux on Magnetization Dynamics in Thick $\mathrm{Nb}/{\mathrm{Ni}}_{80}{\mathrm{Fe}}_{20}/\mathrm{Nb}$ Trilayers, Phys. Rev. Appl. \textbf{11}, 014061 (2019).

\bibitem{kittel_mode} C. Kittel, On the Theory of Ferromagnetic Resonance Absorption, Phys. Rev. \textbf{73}, 155 (1948).

\bibitem{Gient_de} S. V. Mironov and A. I. Buzdin, Giant demagnetization effects induced by superconducting
films, Appl. Phys. Lett. \textbf{119}, 102601 (2021).



\bibitem{Rezende} S. M. Rezende, \textit{Fundamentals of Magnonics} (Springer,
Cham, 2020).






%............metal gate.............

%\bibitem{metal_gate_1}  M. L. Sokolovskyy, J. W. Klos, S. Mamica, and M. Krawczyk, Calculation of the spin-wave spectra in planar magnonic crystals with metallic overlayers, J. Appl. Phys. \textbf{111}, 07C515 (2012).

%\bibitem{metal_gate_2} M. Mruczkiewicz, M. Krawczyk, G. Gubbiotti, S. Tacchi, Y. A. Filimonov, D. V. Kalyabin, I. V. Lisenkov, and S. A. Nikitov, Nonreciprocity of spin waves in metallized magnonic crystal, New J. Phys. \textbf{15}, 113023 (2013).

%\bibitem{metal_gate_3} M. Mruczkiewicz and M. Krawczyk, Nonreciprocal dispersion of spin waves in ferromagnetic thin films covered with a finite-conductivity metal, J. Appl. Phys. \textbf{115}, 113909 (2014).

%\bibitem{metal_gate_4} M. Mruczkiewicz, E. S. Pavlov, S. L. Vysotsky, M. Krawczyk, Y. A. Filimonov, and S. A. Nikitov, Observation of magnonic band gaps in magnonic crystals with nonreciprocal dispersion relation, Phys. Rev. B \textbf{90}, 174416 (2014).

%\bibitem{metal_gate_5} M. Mruczkiewicz, P. Graczyk, P. Lupo, A. Adeyeye, G. Gubbiotti, and M. Krawczyk, Spin-wave nonreciprocity and magnonic band structure in a thin permalloy film induced by dynamical coupling with an array of Ni stripes, Phys. Rev. B \textbf{96}, 104411 (2017).


%\bibitem{chiral_radiation} T. Yu and G. E. W. Bauer, Chiral Coupling to Magnetodipolar Radiation, Top. Appl. Phys. \textbf{138}, 1 (2021). 

\bibitem{LLequation} L. D. Landau and E. M. Lifshitz, \textit{Electrodynamics of Continuous Media}, 2nd ed. (Butterworth-Heinenann, Oxford, U.K., 1984).



\bibitem{Jackson} J. D. Jackson, \textit{Classical Electrodynamics} (Wiley, New York, 1998).

\bibitem{London_equation} J. R. Schrieffer, \textit{Theory of Superconductivity}, (W. A. Benjimin, New York, 1964).

\bibitem{cooper_pair_density} S. P. Chockalingam, M. Chand, J. Jesudasan, V. Tripathi, and P. Raychaudhuri, Superconducting properties and Hall effect of epitaxial NbN thin films, Phys. Rev. B \textbf{77}, 214503 (2008).

%.........ultrathin YIG film.......
\bibitem{EuS_thin_film} O. W. Dietrich, A. J. Henderson, Jr., and H. Meyer, Spin-wave analysis of specific heat and magnetization in Euo and Eus, Phys. Rev. B \textbf{12}, 2844 (1975).

\bibitem{thin_film} Y. Hou, F. Nichele, H. Chi, A. Lodesani, Y. Wu, M. F. Ritter, D. Z. Haxell, M. Davydova, S. Ili\'{c}, O. Glezakou-Elbert, A. Varambally, F. S. Bergeret, A. Kamra, L. Fu, P. A. Lee, and J. S. Moodera
, Ubiquitous Superconducting Diode Effect in Superconductor Thin Films, Phys. Rev. Lett. \textbf{131}, 027001 (2023).



\bibitem{magnon_conductivity} X. Y. Wei, O. A. Santos, C. H. S. Lusero, G. E. W. Bauer, J. B. Youssef, and B. J. van Wees, Giant magnon spin conductivity in ultrathin yttrium iron garnet films, Nat. Mater. \textbf{21}, 1352 (2022).


\bibitem{YIG_parameter} S. Knauer, K. Dav\'{i}dkov\'{a}, D. Schmoll, R. O. Serha, A. Voronov,
Q. Wang, R. Verba, O. V. Dobrovolskiy, M. Lindner, T. Reimann,
C. Dubs, M. Urb\'{a}nek, and A. V. Chumak, Propagating spin-wave spectroscopy in a liquid-phase epitaxial nanometer-thick YIG film at millikelvin temperatures, J. Appl. Phys. \textbf{133}, 143905 (2023).







\end{thebibliography}

\end{document}
