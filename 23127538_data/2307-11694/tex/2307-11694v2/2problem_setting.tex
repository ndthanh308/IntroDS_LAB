
\section{Background and Problem Setting} \label{sec:problem_setting}

In the last few decades, combination therapy has emerged as an effective method to target genetically unstable diseases \citep{mokhtari2017combination}, with dramatic success in treating HIV \citep{moore1999natural} and more recently HCV\citep{liang2013current}. %
Unlike HIV and HCV which encode only 10-15 proteins \citep{frankel1998hiv, dubuisson2007hepatitis}, cancer is radically more complex. Since cancer has an unstable genome, combination therapy is often considered necessary \citep{mokhtari2017combination} and is commonly used in practice, with varying degrees of success.

Generally, drugs work by affecting cellular pathways--chain interactions of molecules which lead to changes in a cell. %
In \emph{drug synergy prediction}, our goal is to predict whether combining drugs will have positive or negative outcomes in the complex system of these interacting pathways. Generally, synergy lab experiments are conducted in cell lines, which are a population of cells from a multi-cellular organism (for example, human lung cancer cells). In this work, we also investigate \emph{inverse design} of drug molecules. Traditionally, the idea behind inverse design of molecules is to predict or retrieve a molecular structure which has some desired chemical property or protein target \citep{sanchez2018inverse}. In our work, we seek to explore inverse design at a higher level-- the ``interactome'' of drug interactions in complex cellular pathways.

\paragraph{General Problem Formulation} Given $k$ input drugs $d^1, d^2, \ldots, d^k \in \mathcal{D}$  along with a cell line $c \in \mathcal{C}$, the goal of drug synergy prediction is to predict a synergy value $y$ for the interactions between the drugs in the given cell line. In existing datasets, only the pairwise $k=2$ setting is considered. Thus, we focus our experiments on pairwise drug synergy, the most commonly researched setting, but our methods can naturally be extended to n-ary synergies. This problem can be considered as either a regression ($y \in \mathbb{R}$) or a binary classification problem (synergistic (True) or not (False); $y \in [0,1]$). Synergy data comes from a dataset of tuples $(d^1, d^2, c, y) \in \mathfrak{D}$. 

\paragraph{Few-Shot In-Context Setting} We also consider the few-shot setting in our formulation, which has applications for predicting synergies when there is scarce training data such as in tumor-specific synergy prediction, uncommon cancer tissues, or newly introduced single-agent therapies. %
In the few-shot setting, we assume there are $n$ synergy tuples available which contain an unknown entity $h$ (unknown cell line $c^h$ or unknown drug $d^h$). Define these tuples as $x_i := (d^1, d^2, c, y)_i$ for $i \in [1..n]$ where one of $d^1$, $d^2$, or $c$ is the unknown $h$. Each $x_i$ can then be used for training in addition to the existing training data. In our proposed method SynerGPT, we don't use these tuples $x_i$ in training-- rather, we use them as the prompt for in-context learning. Here, we are particularly interested in synergy prediction based on extremely small datasets (e.g. tested synergies from a patient's specific cancer cells), which makes traditional supervised approaches less effective. In section \ref{method:training_strategy}, we detail our training strategies for in-context learning with unknown $h$ from limited examples. 

% Figure environment removed

\vspace{-.3cm}
\paragraph{Inverse Drug Design from Drug Synergy Context} %
We propose a new task where the goal is to predict the structure of a molecule given a context of drug synergy tuples (e.g., we might be given 20 synergy tuples). We train a model to predict the structure of some unknown drug $d^h$ from its synergy relations with other drugs. This has two important uses. First, this may enable scientists to predict new molecules which have desirable synergies or similar synergies to existing drugs, which is a novel way to consider drug discovery. This can potentially enable the design of drugs that synergize specifically to target a given patient's unique cancer cells. Secondly, this can support explainability of the synergy prediction model as a function of the context it is fed, by ``visualizing'' SynerGPT's understanding of the unknown drug given the context. Section \ref{sec:inverse_design} shows that we can observe the structure of the drug evolving towards the ground truth as more context examples are given. As this is a novel and difficult problem; we initially frame it as a retrieval task, effectively constraining the output space, though from an implementation perspective it is trivial to instead predict structures by using a pretrained generative model for molecules \citep{jin2020hierarchical} with no architectural differences, as both the retrieval and generation of drug structures requires generating a latent vector. 


    
   

