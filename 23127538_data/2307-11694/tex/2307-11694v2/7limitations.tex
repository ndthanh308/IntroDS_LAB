
\section*{Limitations} \label{limitations}
While we are able to achieve strong performance without additional cellular or drug data, our approach is very much a black box akin to most deep learning methods. To address this, we propose the task of inverse design from drug synergy examples, which allows the visualization of the model's structural understanding as it gains more information. While this is a useful step, we do recognize that further research on mechanistic explainability would be valuable. We hope our contribution on synergy-based inverse design can inspire further work on explainability and that SynerGPT’s predictions can be useful inspiration for clinical researchers. We would also like to note that regardless of using deep learning models, pharmaceutical researchers are in many cases unable to explain the mechanisms of many important drugs on their own (e.g., Modafinil, Metformin, general anesthetics) \citep{stahl2020prescriber, rena2013molecular, brown2011general} --- let alone explain their interactions with each other. These drugs are prescribed to hundreds of millions of patients. Recent studies \citep{lin2019off} suggest that many purported protein drug targets may not be the actual target at all. Important progress with life-saving modern drugs can be made with limited visibility into underlying mechanisms, yet certainly improved mechanistic understanding would be highly useful.

While we show that strong performance is possible without features, future work will still likely want to integrate external database features into drug synergy prediction; however, they will likely need to be integrated in a more thoughtful manner in order to ensure an actual benefit. 



It would also likely be interesting for future work to investigate the internal connections language models are learning and what it might mean for understanding the fundamental biology of how cellular pathways interact. It is also worth noting that designing molecules using drug synergy tuples is a somewhat atypical task, so there may exist a wall in terms of the information content inherent in the context. 
While we do analysis by separating model performance into different tissue types in this work (as done in multiple prior studies), we note that for future research it is likely too limiting and simplistic to separate cell lines into tissues types.
