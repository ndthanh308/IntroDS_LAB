

\section{Context Optimization}
\label{appendix:optimization}


\begin{table}%
\centering
\begin{subtable}{0.75\textwidth}
\resizebox{\columnwidth}{!}{
\begin{tabular}{ c|c|c|c|c }

 \multicolumn{1}{c}{} & \multicolumn{2}{c}{\underline{Unknown Drug}} & \multicolumn{2}{c}{\underline{Unknown Cell Line}} \\ 
 \multicolumn{1}{c}{Strategy} & \multicolumn{1}{c}{ROC-AUC} & PR-AUC & ROC-AUC & \multicolumn{1}{c}{PR-AUC} \\
\thickhline
 Typical Unknown-First & 79.2 & 63.8 & 85.2 & 74.9 \\
 \hline
 Best Unknown-First & 80.8 & 66.4 & 85.6 & 75.7 \\
 \hline
 Error Reduction & 75.4 & 59.0 & 84.9 & 74.5 \\
 \hline
 Genetic Algorithm & \textbf{81.5} & \textbf{66.9} & \textbf{86.1} & \textbf{76.5} \\

\end{tabular}}
\end{subtable}
\caption{Test-set performance of different context optimization methods applied to the Unknown-First strategy. Note that the same model parameters are used in all cases and only the input context is changed. Considering Unknown-First as the distribution being sampled from, the genetic algorithm solution has a z-score of 4.02 indicating $p < 0.0001$.}
\label{tab:full_optimization}
\end{table} 

\subsection{Genetic Algorithm}
In the case of the genetic algorithm, each context bank synergy tuple $x \in \mathfrak{D}^c$ is considered as a gene which can be selected by the algorithm. Given $p$ ``unknown'' drugs or cell lines, each has $n$ slots for context examples in its prompt, which makes for $np$ total genes. We also enforce that each $x$ contains the relevant unknown drug $d^h$ or cell line $c^h$. We disallow each context example from being selected multiple times; the reasons for this is two-fold. First, in early experiments we found that if we use the same example for the entire context (e.g. 20 repeats of $x$), then the model performs poorly. This is likely because the model is not trained on duplicate input, so it is trying to make meaningless connections between the same $x$. Second, repeating $x$ in the context provides no new information to the model. Although we enforce this constraint, in practice without it the model will likely do the same thing on its own. 

For the genetic algorithm in context optimization, we use a population of 8 for 50 epochs. We use steady-state parent selection with 4 parents, single-point cross-over, 10\% gene mutation, and elitism. Each example in the context bank is considered a gene and we disallow repeated genes. This results in 351 evaluations on the validation set. 

\subsection{Error Reduction for Context Optimization} \label{appendix:error_reduction}
Using the Unknown-First strategy, we sample a context for some heldout tuple in the validation set. We then calculate the absolute error $\epsilon$ for the heldout tuple. For each context example $x^c$ in the heldout tuple's input context $P^n$, we store $\epsilon$ and the relevant heldout entity, $h$. After some number (for fairness we use the same number of times as the genetic algorithm evaluates ROC-AUC on the validation set--351) of epochs on the validation set, we calculate a mean error $\hat{\epsilon}_{h}(x^c)$ for that context example $x^c$. Finally, for each heldout drug or cell $h$, we select the $n$ context examples $x^c_i$ with the lowest $\hat{\epsilon}_{h}(x^c_i)$. 

As shown in Table \ref{tab:full_optimization}, this strategy produces poor performance. This indicates that simply selecting all of the most individually informative context examples is not useful. Rather, there is a more complex, non-linear interaction between examples which is informative to the model. This is intuitive, because the interaction between cellular pathways in complex and still not well understood. The ability for the context to be optimized by a genetic algorithm but not error reduction indicates that data collection strategies which emphasize diversity may be important to consider for constructing new drug synergy datasets. 
