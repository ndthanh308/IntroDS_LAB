

\section{Conclusions and Future Work}

As demonstrated by HIV, HCV, and now cancer, combination therapy is a critical option for disease treatment. Yet, difficulties arise in regards to understanding drug-drug interactions and patient-specific genetic differences. To tackle this, we show that encoder-only language models are effective for drug synergy prediction. We then build on these results by proposing SynerGPT, a decoder model with a novel training strategy for in-context learning which can produce strong results for few-shot drug synergy prediction. We additionally show that the model context can be optimized using non-linear black-box approaches, which has exciting implications for the design of a standardized drug synergy testing panel for creating patient-specific synergy datasets. Finally, we explore a novel task of inverse design using desired drug synergy tuples. Performance on this challenging task is low for unknown drugs; nonetheless, it shows promise for future work that may enable personalized drug discovery. %

\paragraph{Limitations}
While we are able to achieve strong performance without additional cellular or drug data, our approach is very much a black box akin to most deep learning methods. Future work will still likely want to integrate external database features. However, they will likely need to be integrated in a more thoughtful manner in order to ensure an actual benefit. It would also likely be interesting for future work to investigate the internal connections language models are learning and what it might mean for understanding the fundamental biology of how cellular pathways interact. It is also worth noting that designing molecules using drug synergy tuples is a somewhat atypical task, so there may exist a wall in terms of the information content inherent in the context. 
While we do analysis by separating model performance into different tissue types in this work (as done in multiple prior studies), we note that for future research it is likely too limiting and simplistic to separate cell lines into tissues types.

