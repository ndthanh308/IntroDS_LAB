
\section{Introduction}


Drug combination therapy is a standard practice for diseases including cancer \cite{mokhtari2017combination} and HIV. 
It is based on identifying multiple single agent therapies that, when used together, lead to synergistic effects. %
Predicting such combinatorial synergies is challenging, especially given the wide range of multiple different mutations as well as different genetic backgrounds typically found in different patients' cancer cells \cite{mroz2017challenges}. 
Many drug combinations can also cause increased toxicity \cite{zapata2020risk, juurlink2004rates} in a manner that may depend on specific patient backgrounds \cite{o2009cancer}, 
adding further complexity to the problem. 
To enable the safest and most effective implementation of combination therapy in cancer care, it is thus important to \emph{personalize} the prediction of drug synergies.%


Since the number of drug combinations scales exponentially, differentiating between synergistic and antagonistic pairings is very expensive to test in large quantities in laboratory conditions. Thus, considerable interest has recently grown in using machine learning for predicting synergistic and antagonistic effects between pairs of drugs in silico \cite{liu2020drugcombdb, preuer2018deepsynergy, rozemberczki2022moomin}. These approaches are typically not evaluated in the few-shot setting, where only a few training examples are given, which is particularly relevant in the personalized setting described above, and more generally for cancer tissue types for which there is limited training data for synergy learning models.
Additionally, these efforts use a variety of features to categorize the drugs, from molecular fingerprints \cite{preuer2018deepsynergy} to protein interactions \cite{yang2021graphsynergy}. Obtaining these features often requires integrating external knowledge sources (e.g., from drug databases), which often results in findings being restricted to the limited subsets of drugs for which this information is available and also requires specialized engineering in model design. Finally, it is unclear if these external sources are actually needed for current models.%

In this work, we address these limitations by exploring the ability of transformer language models (LMs) to learn drug synergy relations. We devise approaches that leverage transformers (1) \emph{without} any external knowledge required to be integrated into the model (i.e., no protein interaction networks or patient cell line features); (2) in the few-shot setting with an in-context learning approach that can generalize to novel unseen drugs and patient cell lines; and (3) for designing novel synergistic drug structures in the context of a specific patient's data. 

\paragraph{Transformer LMs are Strong Drug Synergy Learners---Even Without Textual Representations} First, we consider drug synergy prediction using transformer language models without enriching drugs/cells with information from external knowledge bases. We find these ``feature-less'' models are able to achieve results that are better or competitive in comparison to knowledge-enhanced state-of-art drug synergy models (e.g. BERT models achieve 84.1\% ROC-AUC to GraphSynergy's 83.4\%) %
Furthermore, in contrast to recent work that uses textual representations pre-trained on scientific corpora \cite{nadkarni2021scientific}, we discover an intriguing counter-intuitive finding: using \emph{randomized} (i.e. uninformative) tokens instead of drug/cell names is able to rival models that use textual names of those entities, suggesting that external information coming from pre-training on scientific corpora has negligible impact on current models in this setting. These findings motivate us to explore the power of transformer models without external information, and to study generalization beyond memorization capacity by evaluating on drugs/cells unseen during training.  %

\paragraph{SynerGPT: A New In-Context Drug Synergy Setting \& Model}  We take inspiration from recent work \cite{garg2022can} that showed how a GPT model architecture can be trained to ``in-context learn'' function classes such as linear functions (e.g., linear regression/classification) and neural networks. We pre-train a GPT model from scratch on known drug synergies---using no textual corpora---and explore its ability to generalize in the few-shot setting to drugs and patient cell lines \emph{unseen} during training. We find that our model, dubbed \texttt{SynerGPT}, is able to achieve strong competitive results without any external knowledge sources. In particular, we introduce a new setting of \emph{In-Context Learning for Drug Synergy} (ICL-DS). In-Context Learning (ICL) \cite{dong2022survey} has emerged as a powerful paradigm for few-shot learning \cite{brown2020language}. In ICL, trained model parameters are never explicitly updated after pre-training, and adaptation to each task is done on the fly given contextual examples. This is particularly appealing in settings where it is prohibitively costly to perform parameter updates for each incoming new task and context (e.g., for each new patient in a hospital setting). We devise novel pre-training approaches for ICL-DS, including strategies for optimizing the language model prompt selection with a genetic algorithm. Prompts comprise specific combinations of drugs tested for synergy on specific patient cell lines; optimizing prompt selection in this setting has potential implications for the design of a standardized assay panel of drugs and cells to be tested for %
a patient's particular tumor. While specific patient data at this level is not readily available, we re-purpose existing drug combination data to lay the foundations for formalizing and studying our approaches from a machine learning perspective.

\paragraph{Designing New Molecules to be Synergistic in the Context of a Specific Patient} Finally, in our third major contribution we propose an additional new task of \emph{Inverse Synergistic Drug Structure Design} (ISDSD): using a GPT transformer model for \emph{generating} or retrieving drug molecules that are synergistic in the context of a specific cancer patient's information (i.e., molecules that are synergistic with other drugs administered to a patient with specific cancer cells). This approach may in the future provide a new methodology for personalized drug candidate discovery.



