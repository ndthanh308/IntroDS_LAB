
\begin{abstract}
    This paper introduces a new method of discretization that collocates both endpoints of the domain and enables the complete convergence of the costate variables associated with the Hamilton boundary-value problem.
    This is achieved through the inclusion of an \emph{exceptional sample} to the roots of the Legendre-Lobatto polynomial, thus promoting the associated differentiation matrix to be full-rank.
    We study the location of the new sample such that the differentiation matrix is the most robust to perturbations and we prove that this location is also the choice that mitigates the Runge phenomenon associated with polynomial interpolation.
    Two benchmark problems are successfully implemented in support of our theoretical findings.
    The new method is observed to converge exponentially with the number of discretization points used.
    \par\vskip\baselineskip\noindent
    \textbf{Keywords:}
    {\small Polynomial interpolation; Lobatto Polynomial; Optimal control; Pseudospectral method; Costate convergence; Trajectory optimization. }
\end{abstract}
