
% Guidelines for the Conclusion.
%
% The Conclusion should contain:
%   * the authors' final thoughts about the study findings
%   * how the authors relate to the study aim
%   * how the findings will be of relevance to future research
%   * what recommendations can be made basis the author's research.
%
% Tips:
%   * The final thoughts and concluding statements are usually written in the present tense.
%   * if the author makes recommendations about the usage of the study findings as well as scope for further research, these are usually presented in the future tense.


\section{Conclusion}
In this paper, we have developed a discretization method that enables algebraic differentiation and integration over the domain $\closedinterval{-1}{1}$.
As discretization points, the new method suggests using the roots of the Lobatto polynomial of degree $N$ as well as the root of the Legendre polynomial of degree $N-1$ that is nearest zero, the latter denoted as \emph{exceptional sample}.
We have shown that the resulting differentiation matrix exhibits important beneficial properties that classify it as \emph{inversion-ready}.
We focused on the numerical solution of optimal control problems and we saw that the new method exhibits comparable performance with respect to the alternative methods.

The application of the new method to trajectory optimization is of particular interest, as it enables the direct attainment of the costate variables at both endpoints of the domain without sacrificing convergence behaviour.


