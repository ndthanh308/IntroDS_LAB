
\subsection{Convergence behaviour for an analytic problem}

In order to test the convergence rate of the new method, we shall consider a problem for which the solution is known. In this context we shall take a scalar nonlinear initial value problem from \cite{Garg2011} and \cite{Garg2010}, which is stated as follows.
\begin{prob} \label{prob:convergence}
    Determine $x(t)$ and $u(t)$, $\forall t \in \closedinterval{0}{2}$ such that
    \begin{equation}
        -x(2)
    \end{equation}
    is minimized, subject to the differential constraint
    \begin{equation}
        \dot{x}(t) = \tfrac{5}{2}( x(t) u(t) - x(t) - u^2(t) ) \,,
    \end{equation}
    and the initial condition $x(0) = 1$.
\end{prob}

The solution to Problem \ref{prob:convergence} is as follows, see \cite{Garg2011,Garg2010}:
\begin{align}
    x^*(t) &= 4/( 1 + 3 \exp{\tfrac{5t}{2}} )\\
    u^*(t) &= x^*(t)/2 \\
    \lambda^*(t) &= - \frac{\exp{\bigl(2 \ln(1 + 3 \exp{(\tfrac{5t}{2})}) - \tfrac{5t}{2}\bigr)} }{6 + 9 \exp(5) + \exp(-5) } 
\end{align}

We now show the achieved precision of the new method with respect to the known solution for many degrees of polynomial interpolation $N$ and we compare the result with the alternative Gauss-Lagrange methods, namely, the Gauss method \cite{Benson2004,Rao2010}, the Radau method \cite{Garg2011,Patterson2014} and the standard Lobatto method \cite{Ross2003,Fahroo2008}.

Figures \ref{fig:convergence-state}, \ref{fig:convergence-control}, and \ref{fig:convergence-costate} show the maximum absolute value of the errors obtained for state, control and costate, respectively, in relation to the known solution \emph{at the points where differential constraints are enforced}. These quantities are computed as follows
\begin{align}
    E_x &= \max_{k \in \setC}{\lvert \hat{x}_k - x^*(t_k) \rvert} \\
    E_u &= \max_{k \in \setC}{\lvert \hat{u}_k - u^*(t_k) \rvert} \\
    E_\lambda &= \max_{k \in \setC}{\lvert \hat{\lambda}_k - \lambda^*(t_k) \rvert} \,,
\end{align}
where $\hat{x}_k$, $\hat{u}_k$, and $\hat{\lambda}_k$ are, respectively, the discrete solutions of state, control and costate obtained with each method.
% Figure environment removed
% Figure environment removed
% Figure environment removed

As shown in previous research, see \cite{Garg2011,Garg2010}, the standard Lobatto method, which employs a square differentiation matrix, is, in general, the most inaccurate.
In addition, note that the IPOPT solver failed to reach optimality when using the standard Lobatto method for $N=\{30,34,35\}$. The absence of the respective points is evident in Fig. \ref{fig:convergence-costate}.
In contrast, we note that the new method exhibits identical precision as the method of Radau with regards to the state and control solutions.

% Finally, with respect to the costate solution, the new method exhibits comparable precision with the Radau method.
% has virtually the same accuracy as the methods of Gauss and Radau with regards to the state and control solutions.

