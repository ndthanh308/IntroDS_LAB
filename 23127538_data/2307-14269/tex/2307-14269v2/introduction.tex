
\section{Introduction}
% Introduce the problematic. Contextualize.
Gauss-Lagrange methods, also known as pseudospectral methods, are a group of numerical methods that have seen great implementation rates over the past decades, specially applied to trajectory optimization problems which are often analytically intractable \cite{Betts1998,Rao2009,Conway2011,Rao2014a,Malyuta2021}. 

These methods are a subset of \emph{direct methods}, which enable the explicit enforcement of differential and algebraic constraints without the need to derive the costate equation associated with the Hamilton boundary-value problem.
As a result, they are practical and suitable for general purpose implementations.
In addition, the properties of the Gaussian quadrature associated with these methods provide a good trade-off between accuracy and a sparse discretization of the domain, as these methods exhibit \emph{exponential convergence rates}, see, \emph{e.g.}, \cite{Garg2011}.

The usual approach taken with these methods for trajectory optimization is to transcribe the optimal control problem into a nonlinear programming problem (NLP) and use an NLP solver to obtain a vectorized solution.
Having obtained such solution, it is then trivial to extract the individual optimal control variables.
In this context, three of the most popular Gauss-Lagrange methods are the Legendre-Lobatto method, the Legendre-Gauss method, and the Legendre-Radau method.

% Ross and Fahroo achievements
The Legendre-Lobatto method developed in \cite{Elnagar1995,Fahroo2001,Ross2001,Ross2003,Ross2006} is attractive because it enables the collocation of the endpoints of the domain. However, the solutions obtained with this method typically exhibit lack of convergence of the costates, see, \emph{e.g.}, \cite{Garg2011}, and as a result the accuracy of the primal variables is reduced even in a converged solution.
It was shown in \cite{Garg2011} that the associated differentiation matrix is rank-deficient and this is the leading cause for the systematic failure of this method.

In \cite{Benson2004,Rao2010} the authors propose the Legendre-Gauss method, which guarantees the convergence of the costate variables at the expense of not collocating the endpoints of the domain. In addition, the method also requires the enforcement of one quadrature constraint per state variable in any given problem, which complicates implementation.

% Rao Garg and Patterson
The Legendre-Radau method introduced in \cite{Garg2011} preserves convergence of the costate variables, however, only one of the domain endpoints is collocated, therefore, control and costate variables are not obtained at the opposing endpoint.

Alternatively, in \cite{Liu2014a} the authors employ a Hermite interpolation scheme which enables a well conditioned problem with costate convergence guarantees.
However, the method still lacks the complete collocation of the terminal endpoint. As a result, the costate variables at this point are not obtained directly.

% In this paper
In this paper, we develop a method that enables the collocation of both endpoints of the domain and that guarantees convergence of the costate variables at every point.

By using the nodes of the Legendre-Lobatto method and including an additional discretization point at which differential constraints are not enforced, hereinafter referred to as \emph{exceptional sample}, we are able to systematically generate an appropriate differentiation matrix.
This is achieved by adhering to a reasonable qualification metric for the location of the exceptional sample and subsequently deriving a global minimizer.

As proof of concept, the new method is implemented for two classic optimal control problems for which the standard Legendre-Lobatto method has been shown to fail in the past. The precision of the new method is compared with that of the Legendre-Gauss method and of the Legendre-Radau method with a benchmark problem for which there exists an analytic solution.


