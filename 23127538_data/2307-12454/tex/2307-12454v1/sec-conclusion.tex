
We introduced the logical system $\CFP$ by extending $\IFP$ \cite{IFP}
with two propositional operators $\rt{A}{B}$ and $\Set(A)$,
and developed a method for
%\mps{\HT{satisfying?}}
extracting nondeterministic and concurrent 
programs that are provably total and  satisfy
their specifications. 

% \mps{\HT{(Rest-intro) is constructively valid under such interpretation?}
% \UB{Ah, sorry. Corrected it}}
While $\IFP$ already imports classical logic through nc-axioms that
need only be true classically, in $\CFP$ the access to classical logic
is considerably widened through the rule (Conc-lem) which,
when interpreting $\rt{A}{B}$ as $A \to B$ and identifying $\Set(A)$ with $A$,
is constructively 
%(but not classically)
invalid but has nontrivial 
%concurrent 
nondeterministic
computational content.


We applied our system to extract a 
%concurrent 
concurrent
translation from infinite Gray 
code to the signed digit representation, thus demonstrating that this approach
not only is about program extraction `in principle' but can be used to
solve nontrivial 
concurrent
%{nondeterministic}
computation problems through 
program extraction.
%

After an overview of related work, 
we conclude with some 
ideas
%directions 
for follow-up research.

\subsection{Related work}
\label{sub-related}
%
The CSL 2016 paper \cite{BergerCSL16} 
%\cite{BergerCSL16} 
is an early attempt to capture concurrency via program extraction 
and can be seen as the starting point of our work. Our main advances, compared 
to that paper, are that
it is formalized as a logic for concurrent execution of partial programs
by a globally angelic choice operator which is formalized by introducing a new connective $B|_A$, and
 that we are able to express bounded nondeterminism with complete
 control of the number of threads while
\cite{BergerCSL16} 
%% the 2016 paper
 modelled nondeterminism with
countably infinite branching, which is unsuitable 
or an overkill for most applications. Furthermore, our approach has a typing discipline,
a sound and complete small-step reduction,  and has %discovered
 the ability to switch 
between global and local nondeterminism (see Sect.~\ref{sub-local} below).
%\mps{\HT{Do not want to refer to something not in the main text.  Do you have any other idea?}}


As for the study of angelic nondeterminism,  it is not easy to develop a denotational 
semantics as we noted in Section \ref{sec-ang}, and
it has been mainly studied from the operational point of view,
e.g.,  notions of equivalence or refinement of processes and associated proof methods, which are all fundamental for correctness and termination
\cite{LassenMoran99,MoranSandsCarlsson2003,Lassen2006,sabel_schmidt-schauss_2008,CarayolHirschkoffSangiori2005,Levy07}.
%
Regarding 
%more 
imperative languages, 
Hoare logic and its extensions have been applied to nondeterminism and proving 
totality from the very beginning (\cite{Apt2019FiftyYO} is a good survey on this subject).
\cite{Mamouras15} studies angelic nondeterminism with an extension of Hoare Logic.

There are many logical approaches to concurrency.
%% An example of a successful approach to this problem is based
An example is an approach based
on extensions of Reynolds' separation logic~\cite{Reynolds:2002} to  
the concurrent and higher-order setting~\cite{OHearn07,Brookes07,Jungetal18}.
%
Logics for session types
and process calculi ~\cite{Wadler14a,CairesPfenningToninho16,Kouzapasetal16}
form another approach
that is oriented more towards the 
formulae-as-types/proofs-as-programs~\cite{Howard80,Wadler14} or rather
proofs-as-processes paradigm~\cite{Abramsky94}.
%
All these approaches provide highly specialized logics and expression languages
that are able to model and reason about concurrent programs with a fine control 
of memory and access management and complex communication patterns.
%


\subsection{Modelling locally angelic choice} 
\label{sub-local}
%
We remarked earlier that our interpretation of $\Amb$ corresponds to
\emph{globally} angelic choice. Surprisingly, \emph{locally} angelic choice
can be modelled by a slight modification of the restriction and 
%concurrency
the total concurrency
operators: We simply replace $A$ by the logically equivalent formula
$A \lor \False$, more precisely, we set
%
$\rtp{A}{B} \eqdef \rt{A}{(B\lor\False)}$ and
$\Set'(A) \eqdef \Set(A\lor\False)$.
%
Then the proof rules in 
Sect.~\ref{sec-cfp}
%Sects.~\ref{sec-partial} and~\ref{sec-conc}
with $\rt{}{}$ and $\Set$ replaced by $\rtp{}{}$ and
$\Set'$, respectively but without the 
strictness
%% productivity 
condition, are theorems of $\CFP$. 
%
To see that the operator $\Set'$ indeed corresponds to locally angelic choice
it is best to compare the realizers of the rule (Conc-mp) for $\Set$ and $\Set'$.
Assume $A$, $B$ are non-Harrop and $f$ is a realizer of $A \to B$.
Then, 
%
if $\Amb(a,b)$ realizes $\Set(A)$, 
then $\Amb(\strictapp{f}{a}, \strictapp{f}{b})$ realizes $\Set(B)$.
This means that to choose, say, the left argument of $\Amb$ as a result,
$a$ must terminate and so must the ambient (global) computation 
$\strictapp{f}{a}$.
%
On the other hand, the program extracted from the proof of (Conc-mp) for
$\Set'$ takes a realizer $\Amb(a,b)$ of $\Set'(A)$ and returns 
$\Amb(\strictapp{(\aup \circ f \circ \adown)}{a}, \strictapp{(\aup \circ f \circ \adown)}{b})$
as realizer of $\Set'(B)$, %$\Amb'(B)$, 
where $\aup$ and $\adown$ are the realizers of $B \to (B\lor\False)$ and 
$(A \lor\False) \to A$, 
%% respectively, 
namely,
$\aup \eqdef \lambda a.\, \Left(a)$ and
$\adown \eqdef \lambda c.\, \caseof{c}\{\Left(a) \to a\}$.
%% \begin{align*}
%% \aup &= \lambda a.\, \Left(a) \\
%% \adown &= \lambda c.\, \caseof{c}\{\Left(a) \to a\}
%% \end{align*}
%
Now, to choose the left argument of $\Amb$, 
%as a result,
it is enough for $a$ to terminate since the non-strict operation $\aup$
will immediately produce a w.h.n.f. %terminate 
%with a result of the form $\Left(\ldots)$, 
without invoking the ambient computation.
%
%
% \mps{\HT{If space allows, how about this? Is it too long and precise for conclusion?} 
% \UB{Not sure whether the sentence about $\Just$ and $\Jus$ is corect. I commented it out}}
%\mps{\UB{Commented out last paragraph.}}
By redefining realizers of $\rt{A}{B}$ and $\Set(A)$ as realizers of 
$\rtp{A}{B}$ and $\Set'(A)$ and 
%realizability interpretation of the rules of $\CFP$ as
%those of proofs of the corresponding rules for $\rest'$ and $\Set'$,
the realizers of the rules of $\CFP$ as those extracted from the proofs 
of the corresponding rules for $\rtp{}{}$ and $\Set'$,
we have another realizability interpretation of CFP that models
locally angelic choice.  

\subsection{Markov's principle with restriction}
\label{sub-markov}
%
So far, (Rest-intro) is the only rule that derives a restriction in a 
non-trivial way. 
%Are there other such rules? Indeed there are.
However, there are other such rules,
for example
%
\begin{center}
%\AxiomC{$\Dec(P)$}
\AxiomC{$\forall x \in \NN (P(x) \lor \neg P(x))$}
\RightLabel{Rest-Markov}
\UnaryInfC{$\rt{\exists x \in\NN\,P(x)}{\exists x \in\NN\,P(x)}$}
            \DisplayProof \ \ \ \ 
          \end{center}
%
%It is easy to see that 
If $P(x)$ is Harrop, then (Rest-Markov) 
is realized by minimization.
%
More precisely, if $f $ realizes $\forall x \in \NN (P(x) \lor \neg P(x))$,
then $\min(f)$ realizes the formula  
$\rt{\exists x \in\NN\, P(x)}{\exists x \in\NN\,P(x)}$,
where $\min(f)$ computes the least $k \in \NN$ such that $f\, k = \Left$
if such $k$ exists, and does not terminate, otherwise. 
%
One might expect as conclusion of (Rest-Markov) the formula
$\rt{(\neg\neg\exists x \in\NN\,P(x))}{\exists x \in\NN\,P(x)}$.  
However, because of (Rest-stab) (which is realized by the identity), 
this wouldn't make a difference.
%
The rule (Rest-Markov) can be used, for example, to prove that
Harrop predicates that are recursively enumerable (re) and 
have re complements are decidable. 
%have am re complement, are in fact decidable. 
From the proof one can extract a program 
%that decides the given predicate 
that 
%% for a given natural number 
concurrently searches for evidence of membership in the predicate and
%evidence for membership in the complement of the predicate.
its complement.
%


\subsection{Further directions for research}
\label{sub-further}
%
%\mps{
%\UB{I updated the reference to the Gauss paper.}
%}
The undecidability of equality of real numbers, which is 
%the crucial point in 
at the heart of
our case study on infinite Gray code, is also a critical point in 
Gaussian elimination where one needs to find a non-zero entry in a non-singular
matrix. As shown in~\cite{BergerSeisenbergerSpreenTsuiki22}, 
our approach makes it possible to search for such `pivot elements'
in a concurrent way. %(see the file \texttt{Gauss.hs} in~\cite{HaskellGray}).
% for experiments in this direction).
%
A further promising research direction is to extend the work on coinductive
presentations of compact sets in~\cite{Spreen20} to the concurrent setting.
%


%%% Local Variables:
%%% mode: latex
%%% TeX-master: "main"
%%% End:
