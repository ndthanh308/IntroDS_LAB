We show that the realizability interpretation 
of $\CFP$ is sound in 
the sense that from every $\CFP$ derivation one can extract 
a program realizing the proven formula (Theorem~\ref{thm-soundnessI}). 
%
In Section~\ref{sub-partial} we show the realizability
of the new proof rules and hence the soundness of the extracted programs.
In Section~\ref{subsec:soundness} we show that,
for a class of `admissible' formulas, the data of an 
extracted (closed) program $M$ (i.e., the elements of $\ddata(\val{M})$) realize
the formula obtained by deleting the operator $\Set$ 
and replacing restriction by implication.
Recall that, 
by the Computational Adequacy Theorems \ref{thm:data} and ~\ref{thm:dataconv}, 
the elements of $\ddata(\val{M})$ are exactly the data computed by
the operational semantics of $M$. 

Throughout this section, proofs take place in the system $\RCFP$
(Section~\ref{sub-RCFP})
which is based on classical logic. Classical logic is needed 
since for 
the rules (Rest-intro), (Rest-stab), and (Conc-lem), 
the verification that the extracted 
programs are realizers uses the law of excluded middle.
Note the relation `$d\in\ddata(a)$', which was defined on the meta-level 
in Section~\ref{sub-denot}, can be formalized in $\RCFP$.
This also holds for the rank function and the predicates $\Data$ and $\regD$ 
defined in Section~\ref{subsec:soundness}.
Since we always argue within $\RCFP$ we will refrain
from using semantic brackets from now on. 
For example, we will
write $\ddata(M)$ instead of $\ddata(\val{M})$,  
and the above statement about admissible formulas  
stands for the $\RCFP$ formula
%
$\rea(A)(M) \to \ddata(M) \subseteq \rea(A^-)$
%
where $A^-$ is obtained from $A$ by deleting $\Set$ and
replacing restriction by implication.
%
In fact, Theorem~\ref{thm-faithfulness} shows (for closed $A$) the more 
general formula $\forall a\,(\rea(A)(a) \to \ddata(a) \subseteq \rea(A^-))$.





\subsection{Soundness of extracted programs}
\label{sub-partial}
\label{sub-conc}   
%

The realizers of the proof rules for restriction and concurrency
in Table~\ref{table-infrule}
are depicted in Table~\ref{table-infrule-sound}.
% 
Proofs of the correctness of the realizers are given in Lemma~\ref{lem-restrict}.
%
We use the 
following typable programs:
%
\begin{eqnarray*}
\leftright
%\UB{\leftright\,b} 
&\eqdef& 
  \lambda b.\,  
\caseof{b}  \{\Left(\_) \to \Left; \Right(\_) \to \Right\}
:
\ftyp{(\rho+\sigma)}{(1+1)}\\
\mapamb
&\eqdef& 
\lambda f.\,\lambda c.\ 
\caseof{c}\,\{\Amb(a,b)\to\Amb(\strictapp{f}{a},\strictapp{f}{b})\}
:
\ftyp{(\ftyp{\rho}{\sigma})}{\ftyp{\Am(\rho)}{\Am(\sigma)}}\\
\seq 
&\eqdef& 
\lambda a.\,\lambda b.\,\strictapp{(\lambda c.\ b)}{a}
:
\ftyp{\rho}{\ftyp{\sigma}{\sigma}}
\quad \hbox{(used infix)}
\end{eqnarray*}
%
Here, $\rho$, $\sigma$ range over regular types, and for $\mapamb$
it is required in addition that $\rho$ and $\sigma$ are determined.


\begin{table}
\medbreak
\noindent
\fbox{
\begin{minipage}{\textwidth}
\medskip
\[
\infer[{\hbox{\begin{tabular}{l}Rest-intro
($A, B_0, B_1$ Harrop)\end{tabular}}
}]{
        \ire{(\leftright\, b)}{\rt{A}{(B_0 \vee B_1)}}
}{
\ire{b}{(A \to (B_0 \vee B_1))} \ \ \     \reah({\neg A \to B_0 \wedge B_1})
}
\]
%
\vspace{0.2em}
%
\[
\begin{array}{ll}
\infer[{ \hbox{Rest-return\ }}]{   \ire{a}{\rt{A}{B}}
}{
  \ire{a}{B}
} 
&
\infer[{
        \hbox{Rest-bind}
        \hbox{\begin{tabular}{l}
                $c=\strictapp{f}{a}$ if $B$ non-Harrop\\
                $c=a\,\seq\,f$ if $B$ Harrop
             \end{tabular}}
       }
]{\ire{c}{\rt{A}{B'}}
}{
\ire{a}{ \rt{A}{B}}\ \ \          \ire{f}{( B \to (\rt{A}{B'}))}
}
\end{array}
\]
\vspace{0.2em}
\[
\begin{array}{ll}
  \infer[{ \hbox{Rest-antimon}}]{
    \ire{a}{\rt{A'}{B}}
    }{
      \ire{}{(A' \to A)} \ \ \ \ire{a}{\rt{A}{B}  }
}\ \ \ \ \ \ \ &
  \infer[{ \hbox{Rest-mp}}]{
    \ire{b}{B}
}{
\ire{b}{\rt{A}{B}} \ \ \    \ire{}{A}
}
\end{array}
\]
\[
\begin{array}{ll}
  \infer[{ \hbox{Rest-efq}}]{
  \ire{\bot}{\rt{\False}{B}}
}{
}
\ \ \ \ \ \ \ \ &
\infer[{\hbox{Rest-stab}}]{
    \ire{b}{\rt{\neg\neg A}{B}}
    }{
    \ire{b}{\rt{A}{B}}
}

%
\end{array}
\]
%
\vspace{0.2em}
%
\[
  \infer[{ \hbox{Conc-lem}}]{
  \ire{\Amb(a,b)}{\Set(B)}
}{ 
\ire{a}{\rt{A}{B}}   \ \ \ \     \ire{b}{\rt{\neg A}{B}}
}
%
\qquad
%
  \infer[{ \hbox{Conc-return}}]{\ 
  \ire{\Amb(a,\bot)}{\Set(B)}
}{
\ire{a}{B}
}
%
\]
%
\vspace{0.2em}
%
\[
  \infer[{
            \hbox{Conc-mp}
            \hbox{\begin{tabular}{l}
               $d=\mapamb\ f\ c$ if $A$ non-Harrop\\
               $d=\Amb(f, \bot)$ if $A$ Harrop
                  \end{tabular}}
         }
]
{\
\ire{d}{\Set(B)}
}{
  \ire{f}{(A\to B)}\ \ \  \ire{c}{\Set(A) }
}\ \ \ \ \ 
\]
\smallskip
\end{minipage}
}
\medbreak
\caption{Realizers of the inference rules for $\rt{A}{B}$ and $\Set{B}$ with non-Harrop conclusions\label{table-infrule-sound}.}
\end{table}



\begin{lem}
\label{lem-restrict}
%
The rules for restriction and concurrency are realizable.
\end{lem}
%
\begin{proof}\quad\\


  \noindent\emph{Rest-intro}:
  \ \raisebox{-0.2cm}{$\infer[\hbox{($A, B_0, B_1$ Harrop)}]{
          \ire{(\leftright\,b)}{\rt{A}{(B_0 \vee B_1)}}
  }{
  \ire{b}{(A \to (B_0 \vee B_1))} \ \ \     \reah({\neg A \to B_0 \wedge B_1}) 
  }
  $}.\medbreak
  
  $\ire{b}{(A \to (B_0 \vee B_1))}$ means 
  $b:\tau(B_0\lor B_1) \land(\reah(A) \to \ire{b}{(B_0 \vee B_1)})$,
  and
  %
  $\reah(\neg A \to B_0 \wedge B_1)$ means
  $\neg \reah(A) \to \reah(B_0) \wedge \reah(B_1)$.
  We claim that $\rt{A}{(B_0 \vee B_1)}$ 
  is realized by   $\leftright\,b$.
    %
  First, the image of $\leftright$ is $\{\bot,\Left,\Right\}$ 
  which is a subset of 
  $\tau(B_0\lor B_1) = (\one+\one)_{\bot} = \{\bot,\Left(\bot),\Right(\bot),\Left,\Right\}$,
  so $\leftright\,b : \tau(B_0\lor B_1)$.
  %
  Assume %$\re\,A$, that is, 
  $\reah(A)$. 
  Then $b$ realizes $B_0 \lor B_1$. 
  Hence $b\in\{\Left,\Right\}$ and therefore $\leftright\,b = b \ne \bot$.
  %
  Now assume $\defined{\leftright\,b}$.
  We do a (classical) case analysis on whether $\reah(A)$ holds.
  If $\reah(A)$, then $\ire{b}{(B_0 \vee B_1)}$.
  Hence $b\in\{\Left,\Right\}$ and therefore $\leftright\,b = b$,
  so $\ire{(\leftright\,b)}{(B_0 \lor B_1)}$.
  If $\neg \reah(A)$, then $\reah(B_0)$ and $\reah(B_1)$. Hence,
  $\Left$ and $\Right$ both realize $B_0 \lor B_1$.
  Since $\defined{\leftright\,b}$,
  $\leftright\, b \in \{\Left,\Right\}$ and 
  therefore $\ire{(\leftright\,b)}{(B_0 \lor B_1)}$
  %
  \footnote{Note that $\leftright\,b$ cannot be replaced by $b$ since if $\neg\reah(A)$, $\reah(B_0)$, $\reah(B_1)$ hold and $b = \Left(\bot)$, then both premises of Rest-intro in Table~\ref{table-infrule-sound} hold, but $b$ does not realize $\rt{A}{(B_0 \vee B_1)}$.}.

  \medbreak\noindent
  \emph{Rest-return}:\ \raisebox{-0.2cm}{$\infer{  
   \ire{a}{\rt{A}{B}}
  }{
    \ire{a}{B}
  }$}. \medbreak
  
  Since $B$ is strict, $\ire{a}{B}$ implies $a \ne \bot$.
  Therefore, clearly $\ire{a}{\rt{A}{B}}$.
  
  \medbreak
  \noindent
  \emph{Rest-bind}:\ \raisebox{-0.2cm}{
  \infer{
        \ire{(\strictapp{f}{a})}{\rt{A}{B'}}
  }{
  \ire{a}{ \rt{A}{B}}\ \ \          \ire{f}{( B \to (\rt{A}{B'}))}
  }
  } 
%$B$ non-Harrop.
\medbreak
  
  If $B$ is non-Harrop, then 
  we have $\forall c\,(\ire{c}{B} \to \ire{(f\,c)}{\rt{A}{B'})})$.
  If $\re\,A$ then $\defined{a}$ and  $\ire{a}{B}$, and therefore $f\, a$ 
  realizes $\rt{A}{B'}$.
  Therefore $\defined{f\, a}$ because $\re\,A$.
  Note that $\strictapp{f}{a} = f\, a$ because $\defined{a}$.
  If $\defined{\strictapp{f}{a}}$, then $\defined{a}$.
  Since $\defined{a}$ and $\ire{a}{\rt{A}{B}}$,  we have $\ire{a}{B}$.
  Therefore,  $\strictapp{f}{a} = f\, a$ realizes $\rt{A}{B'}$.
  If $B$ is Harrop,  
    then
  $a\,\seq\,f$ 
 realizes $\rt{A}{B'}$ with a similar argument.
  
  \medbreak
  \noindent \emph{Rest-antimon}: \ \raisebox{-0.2cm}{
  $\infer{
    \ire{a}{\rt{A'}{B}}
      }{
        \ire{}{(A' \to A)} \ \ \ \ire{a}{\rt{A}{B}  }}$}. 
\medbreak

  
  Clearly, $\ire{a}{\rt{A'}{B}}$
  since $\re\,A'$ implies $\re\,A$.
  
  \medbreak
  \noindent \emph{Rest-mp}: \ \raisebox{-0.2cm}{
  $  \infer{
      \ire{b}{B}
  }{
  \ire{b}{\rt{A}{B}} \ \ \    \ire{}{A}
  }
  $}. 
\medbreak
  
  Clear from the definition of $\ire{b}{\rt{A}{B}}$.
  
  \medbreak
  \noindent \emph{Rest-efq}: \ {
  $\ire{\bot}{\rt{\False}{B}} $} for strict $B$.\medbreak
  
  Clear.
  
  \medbreak
  \noindent \emph{Rest-stab}: \ \raisebox{-0.2cm}{
  $
  \infer{
      \ire{b}{\rt{\neg\neg A}{B}}
      }{
      \ire{b}{\rt{A}{B}}}
  $}.  \medbreak
  
We use classical logic.
%
Clearly, $\re\,(\neg\neg A)$ is equivalent to  
$\neg\neg (\re\, A)$, and hence, by classical logic, equivalent to $\re\, A$.
Therefore, premise and conclusion 
are equivalent.

  
  \medbreak
  \noindent \emph{Conc-lem}: \ \raisebox{-0.2cm}{
  $
    \infer{
    \ire{\Amb(a,b)}{\Set(B)}
  }{
  \ire{a}{\rt{A}{B}}   \ \ \ \     \ire{b}{\rt{\neg A}{B}}
  }$}. 
\medbreak
  
  By classical logic
  $\re\,A$, or $\neg(\re\,A)$ i.e.\ $\re\,(\neg A)$. In the first case
  $a \ne \bot$  and in the second case $b \ne \bot$. Further, if $a \ne \bot$,
  then $a$ is a realizer of 
  $B$ since  $a$ realizes $\rt{A}{B}$. Similarly for $b$.
  
  \medbreak
  \noindent \emph{Conc-return}: \ \raisebox{-0.2cm}{
  $  \infer{\   \ire{\Amb(a,\bot)}{\Set(B)}
  }{
  \ire{a}{B}
  }$}. \medbreak
  
  Clear.
  
  \medbreak
  \noindent \emph{Conc-mp}: \ \raisebox{-0.2cm}{
  $  \infer{\
  \ire{(\mapamb\,f\,c)}{\Set(B)}
  }{
    \ire{f}{(A\to B)}\ \ \  \ire{c}{\Set(A) }
  }
  $}
$A$ non-Harrop
\medbreak
  
  We show that $\mapamb\,f\,c$ realizes $\Set(B)$.
  %
  If $\ire{\Amb(a, b)}{\Set(A)}$, then 
  $\defined{a} \lor \defined{b}$.  If $\defined{a}$, then
  $\strictapp{f}{a} = f\ a$ and $\ire{a}{A}$, therefore
  $\ire{(\strictapp{f}{a})}{B}$.  Since $B$ is strict, we have
  $\defined{\strictapp{f}{a}}$.   In the same way, 
  if $\defined{b}$ then $\defined{\strictapp{f}{b}}$.
  Therefore, we have $\defined{\strictapp{f}{a}} \lor \defined{\strictapp{f}{b}}$.  If $\defined{\strictapp{f}{a}}$, then $\defined{a}$ and thus
  $\ire{(\strictapp{f}{a})}{B}$ as we have observed.
  If  $A$ is Harrop,  then, since clearly $\re{A}$,
  it is realized by $\Amb(f,\bot)$.
  %
  \end{proof}
  %
%
\begin{thm}[Soundness Theorem I]
\label{thm-soundnessI}
%
From a 
$\CFP$ proof 
of a 
%
formula $A$ from a set of axioms
one can extract a 
program $M$ 
such that 
%
$\vdash M : \tau(A)$
and
$\RCFP$ proves $\ire{M}{A}$ 
from the same axioms. 
%

More generally, let $\Gamma$ be a set of Harrop formulas and
$\Delta$ a set of non-Harrop formulas.
%
Then, from a $\CFP(\ax)$ proof of 
a formula $A$ 
from the assumptions $\Gamma,\Delta$
one can extract a program $M$ with $\FV(M) \subseteq \vec u$ 
such that $\vec u : \tau(\Delta) \vdash M: \tau(A)$ 
%
and $\ire{M}{A}$ is 
provable in $\RCFP(\ax)$
from the assumptions $\reah(\Gamma)$,
$\ire{\vec u}{\Delta}$. 
%
\end{thm}
%
\begin{proof} %
Induction on $\CFP$ proofs. 
Realizability of the rules of $\IFP$ was shown in ~\cite{IFP}.
The rules for restriction and concurrency are taken care of
in Lemma~\ref{lem-restrict}.
%
The type correctness of the extracted programs for 
Rest-intro, 
Rest-bind and Conc-mp follows from the typings
of $\leftright$, $\mapamb$, and $\seq$.
%
\end{proof}
%
\begin{rem}
From the proof of the soundeness of the rule rest-intro in 
Lemma~\ref{lem-restrict} one can see that the restriction of that rule to
Harrop formulas is essential. Therefore, the concept of Harropness 
is not only a matter of optimizing program extraction, 
but also necessary for obtaining soundness in the first place.
\end{rem}


\begin{lem}\label{class-orelim-sleep}
In $\CFP$, one can prove all well-formed instances of the following formulas:
%
\begin{enumerate}
  \item[(1)] $(B\to B') \to \rt{A}{B} \to \rt{A}{B'}$  (Rest-mon)\medbreak
  \item[(2)] ${\rt{A_0}{B} \to \rt{A_1}{B} \to \neg\neg(A_0\lor A_1) \to \Set(B)}$
  (Class-orelim)\medbreak
  \item[(3)]
${\rt{A_0}{B_0} \to \rt{A_1}{B_1} \to \neg\neg(A_0\lor A_1) \to \Set(B_0\lor B_1)}$\medbreak
\item[(4)]
$A  \to {\rt{\neg A}{B}} $\medbreak
\item[(5)]
$(A \lor B) \to {\rt{\neg A}{B}} $\medbreak
\item[(6)]
$\rt{A_0}{(A \lor B)} \to {\rt{A_0\land\neg A}{B}} $\medbreak
%
\end{enumerate}
The extracted programs and their types 
for the case that $B,B',B_0,B_1,A$ are non-Harrop are:
\begin{enumerate}
  \item[(1)]
  $  \lambda f.\,\lambda b.\, \strictapp{f}{b}\ : \ 
  \ftyp{(\ftyp{\tau(B)}{\tau(B')})}{\ftyp{\tau(B)}{\tau(B')}}$, \smallskip

  \item[(2)]
  $\lambda a.\, \lambda b.\, \Amb(a,b)\ : \ 
  \ftyp{\tau(B)}{\ftyp{\tau(B)}{\Am(\tau(B))}}$, \smallskip
  
 \item[(3)]
 $  \lambda a.\, \lambda b.\, \Amb(\strictapp{\Left}{a},\strictapp{\Right}{b})\  : \ 
 \ftyp{\tau(B_0)}{\ftyp{\tau(B_1)}{\Am(\tau(B_0) + \tau(B_1))}}$,\smallskip
   
  \item[(4)]
  $\lambda a.\, \bot\ : \ 
  \ftyp{\tau(A)}{\tau(B)}$, \smallskip
    
  \item[(5)]
  $\lambda a.\, \caseof{a} \{\Left(\_) \to \bot;\Right(b) \to b\}\ : \ 
  \ftyp{(\tau(A) + \tau(B))}{\tau(B)}$, \smallskip
    
    \item[(6)] The same as (5).
  \end{enumerate}
\end{lem}

\begin{proof}
(1) This is an immediate consequence of (Rest-bind) and (Rest-return).

(2)  Assume $\rt{A_0}{B}$, $\rt{A_1}{B}$, and $\neg\neg(A_0\lor A_1)$.
By the second assumption and the rule (Rest-stab) we have $\rt{\neg\neg A_1}{B}$.
Since, by the third assumption, we have $\neg A_0 \to \neg\neg A_1$, 
$\rt{\neg A_0}{B}$ follows by (Rest-antimon). 
Together with the first assumption and
the rule (Conc-lem), we get $\Set(B)$.

  (3) Since  $B_i \to B_0\lor B_1$ ($i=0,1$), %$A_i \to A_0\lor A_1$ ($i=0,1$),
   this is an immediate consequence of
 (Rest-mon) and (Class-orelim), i.e.\ (1) and (2).

(4) Immediate, from (Rest-efq) and (Rest-antimon).

(5)
Suppose that $A \lor B$.  If $A$, then $\rt{\neg A}{B}$ by \HT{(4).}
%
 If $B$, then $\rt{\neg A}{B}$ by (Rest-return).

(6)
Suppose $\rt{A_0}{(A \lor B)}$. 
By (Rest-antimon), $\rt{A_0 \land \neg A}{(A \lor B)}$.
By (Rest-antimon) and $(A \lor B) \to \rt{\neg A}{B}$, we have 
$(A \lor B) \to \rt{A_0 \land \neg A}{B}$.
Therefore, by (Rest-bind), $\rt{A_0 \land \neg A}{B}$.
\end{proof}

\begin{example}
  \label{ex-dprime}
Continuing Example~\ref{ex-d}, we modify $\D(x)$ to 
$$\D'(x) \eqdef  \rt{x\neq 0}{(x\leq 0 \lor x\geq 0)}\,.$$
%
A realizer of $\D'(x)$, which has type $\bool$, may or may not terminate
(non-termination only occurs when $x = 0$).
However, in case of termination,  the result is guaranteed to 
realize $x\le 0 \lor x\ge 0$.
Note that, a realizer of $\D(x)$ also has type $\bool$ and may or may not terminate,
but there is no guarantee that
it realizes $x\le 0 \lor x\ge 0$ when it does terminate.
Nevertheless, 
%
$\D \subseteq \D'$ follows from (Rest-intro) 
(since $\D(x)$ is $x\ne 0\to x\le 0 \lor x\ge 0$, and 
$\neg x\ne 0\to x\le 0 \land x\ge 0$ is provable using stability of equality) 
and is realized by $\leftright$.
%
$\D' \subseteq \D$ holds by the rule Rest-mp.
\end{example}
%
\begin{example}
\label{example-ConsSD}
%
This 
builds on Examples~\ref{ex-d} and~\ref{ex-dprime}
and will be used in 
Section~\ref{sec-gray}.
Let $\tent(x) = 1-2|x|$ and consider the predicates
$\E(x) \eqdef  \D(x) \land \D(\tent(x))$ and
%
\[
\ConSD(x)  \eqdef  \Set((x\leq 0 \lor x\geq 0) \lor |x| \leq 1/2).
\]
We show $\E \subseteq \ConSD$:
%
From $\E(x)$ and Example~\ref{ex-dprime} we get $\D'(x)$ and $\D'(\tent(x))$
which unfolds to $\rt{x\neq 0}{(x\le 0  \lor x \ge 0 )}$ and 
$\rt{|x|\neq 1/2}{(|x| \geq 1/2  \lor |x| \leq 1/2 )}$. 
By Lemma \ref{class-orelim-sleep}~(6),
$\rt{|x| < 1/2}{(|x| \leq 1/2)}$. 
%
Since $\neg\neg((x \ne 0) \lor |x| < 1/2)$, we have $\ConSD(x)$ 
by  Lemma \ref{class-orelim-sleep}~(3).
 Moreover, $\tau(\E) = \bool\times \bool$ and 
$\tau(\ConSD) = \Am(\tri)$  where $\tri \eqdef \bool + \one$.
The extracted realizer of $\E \subseteq \ConSD$ is
%
\begin{align*}
&\conSD \eqdef \lambda c.\, 
   \caseof{c} \{ \\
& \hspace*{7em} \Pair(a, b) \to  
                  \Amb(\strictapp{\Left}{(\leftright\, a)},\\
& \hspace*{15.7em}        \Right {\downarrow} 
                         (\caseof{b} \{\Left(\_) \to \bot;\\
& \hspace*{23.9em}     %\Right(\_) \to \Nil\}))\}   
\Right(c) \to c\}))\}   
\end{align*}
%\mps{\HT{removed leftright from $\strictapp{\Left}{(\leftright\ a)}$ }}
%
   of type $\tau(\E \subseteq \ConSD) =  \bool\times \bool \to \Am(\tri)$.
%
Explanation of this program:
$a$ is $\Left$ or $\Right$ depending on whether $x \leq 0$ or $x \geq 0$
but may also be $\bot$ if $x = 0$.
$b$ is $\Left$ or $\Right$ depending on whether 
$|x| \geq 1/2$ or $|x| \leq 1/2$ 
%$|x| \leq 1/2$ or $|x| \geq 1/2$ 
but may also be $\bot$ if $|x| = 1/2$.  Since $x = 0$ and $x = 1/2$ do not
happen simultaneously, by evaluating $a$ and $b$ concurrently, we obtain one of them
from which we can determine one of the cases $x \leq 0$, $x \geq 0$, or $|x| \leq 1/2$.
\end{example}



\subsection{Soundness of extracted data}
\label{subsec:soundness}
%
In $\CFP$, we have a second Soundness Theorem 
which ensures the correctness
of all computations, i.e.\ fair reduction paths, 
of an extracted program $M$. 
%
More precisely, we show in Theorem~\ref{thm-soundnessII} that 
if  $M$ realizes a $\CFP$ formula $A$ satisfying a certain syntactic 
admissibility condition (see below), 
then all $d\in\ddata(M)$ realize the formula $A^-$ 
(recall from the introduction to Section~\ref{sec-pe} that $A^-$ is 
obtained from $A$ by deleting all concurrency operators $\Set$ 
and replacing every restriction by the corresponding implication).
%
Hence, by Computational Adequacy, all computations of $M$ 
converge to a realizer of $A^-$ (Theorem~\ref{thm-pe}).
Since $A^-$ is an $\IFP$ formula, this shows that 
the realizability interpretation of $\CFP$ is faithful to
that of
$\IFP$.
%
The proof of 
this result
requires some preparation.

Since the domain $D$ is algebraic (i.e.\ every element of $D$ is the 
directed supremum of compact elements), 
subdomains of $D$ can be characterized as follows. 
%
\begin{lem}
\label{lem-subdom}
A subset $X\subseteq D$ is a subdomain of $D$ iff the following two conditions hold:
%
\begin{enumerate}
\item[(i)] 
For all $a\in D$, $a$ is in $X$ iff all compact approximations are in $X$.
\item[(ii)] If $a_0,b_0\in X$ are compact and consistent (in $D$), 
then $a_0\sqcup b_0\in X$.
\end{enumerate}
\end{lem}
%
\begin{proof}
The easy proof is omitted.
\end{proof}


We define coinductively two subsets of $D$. The first, $\Data$, disallows
the constructor $\Amb$ altogether, the second, $\regD$, disallows immediate
nestings of $\Amb$ and can be seen as a semantic counterpart to the syntactic
regularity property of types. 
We call elements of $\regD$ \emph{regular}.
%
\begin{eqnarray*}
\Data &\eqnu& 
  (
   \{\Nil\} \cup 
   (\Data\times \Data) \cup 
   (\Data+\Data) \cup 
   (\ftyp{D}{D}))_{\bot} \\
\regD &\eqnu& 
  (
   \{\Nil\} \cup 
   (\regD\times \regD) \cup 
   (\regD+\regD) \cup 
   (\ftyp{D}{D}) \cup\\
 &&\hspace{0.5em}  \Amb(\regD\setminus\Amb(D,D),\regD\setminus\Amb(D,D))
  )_{\bot} 
\end{eqnarray*}
%
%
Clearly, $\Data\subseteq\regD\subseteq D$.
%


\begin{lem}
\label{lem-regular}
%
\begin{enumerate}
%
\item\label{lem-regular-sub}
$\Data$ and $\regD$ are subdomains of $D$. %
%
\item\label{lem-regular-data}
$\rho\subseteq\Data$ for every type $\rho$ without $\Am$
under the assumption that $\alpha\subseteq\Data$ for every 
free type variable of $\rho$.
%
\item\label{lem-regular-reg}
$\rho\subseteq\regD$ for every regular type $\rho$
under the assumption that $\alpha\subseteq\regD$ for every 
free type variable of $\rho$.
%
\item\label{lem-regular-rea}
If all free predicate variables of the well-formed 
predicate $P$ are strictly positive, then 
$\rea(P)\subseteq\adummy{\regD}$, 
under the assumption that 
$\reali{X}\subseteq\adummy{\regD}$ and 
$\alpha_X\subseteq\regD$ for every 
free predicate variable $X$ of $P$.
%
\end{enumerate}
\end{lem}

\begin{proof}
The proof of 
%
Part~(\ref{lem-regular-sub}) 
is easy using the characterization of subdomains 
in Lemma~\ref{lem-subdom}. We skip details.

%Part~(2) 
Part~(\ref{lem-regular-data}) 
is proven by structural induction on $\rho$.
%
For a type variable, this holds by assumption. 
For $\one, \rho\times\sigma,\rho+\sigma, \ftyp{\rho}{\sigma}$ it suffices to observe that
$\bot$, $\Nil$ and $\Fun(f)$ are data (for arbitrary $f\in[D\to D]$),
and that the constructors $\Pair,\Left,\Right$ preserve the property $\Data$.
%
To show that $\tfix{\alpha}{\rho}\subseteq\Data$ it suffices to show that
$\rho\subseteq \Data$ under the assumption that $\alpha\equiv\Data$. But this holds by 
the structural induction hypothesis.

%
Part~(\ref{lem-regular-reg}) 
is proven again by structural induction on $\rho$,
using the fact that for a determined type $\rho$ the non-bottom elements of 
$\tval{\rho}{\zeta}$ are not of the form  $\Amb(a, b)$.
%

We prove 
%
Part~(\ref{lem-regular-rea}) 
first with a strengthening of the assumption $\alpha_X\subseteq\regD$
to $\alpha_X\equiv\regD$ 
(keeping the assumption $\reali{X}\subseteq\adummy{\regD}$ unchanged).  
Under this strengthened assumption, $\reali{X}\subseteq\adummy{\alpha_X}$ and therefore
$\rea(P)\subseteq\adummy{\tau(P)}$, by Lemma~\ref{lem-realizers-typed}.
But $\tau(P)\subseteq \regD$, by 
%
Part~(\ref{lem-regular-reg}). 
To prove 
%
Part~(\ref{lem-regular-rea}) 
in general, we observe that, since all free predicate variables
are s.p. in $P$,  $\rea(P)$ depends monotonically on $\alpha_X$.
Therefore, if $\alpha_X\subseteq\regD$, 
the inclusion $\rea(P)\subseteq \adummy{\regD}$ continues to hold.
This proves 
%Part (4) 
Part~(\ref{lem-regular-rea}) 
provided the assumption $\alpha_X\equiv\regD$ is consistent, 
that is, a type $\alpha_X$ satisfying it actually does exist.
But, by 
%
Part~(\ref{lem-regular-sub}), 
$\regD$ is a subdomain of $D$ and hence a possible
value for $\alpha_X$.
\end{proof}

\begin{lem}\label{lem-data-nonempty}
\begin{enumerate}
%
\item\label{lem-data-nonempty-ne} 
$\ddata(a)$ is nonempty for all $a\in D$.
%
\item\label{lem-data-nonempty-sing} 
If $a\in\Data$, then $\ddata(a) = \{a\}$.
%
\item\label{lem-data-nonempty-data} 
If $a\in\regD$, then $\ddata(a) \subseteq \Data$.
%
\end{enumerate}
\end{lem}
\begin{proof}

%
(\ref{lem-data-nonempty-ne}) 
For every nonempty finite increasing sequence 
$\vec a,a_n = a_0,\ldots,a_{n-1},a_n$ of compact
elements $a_i\in D$ ($i\le n$) we define $\cd{\vec a,a_n}\in D$ by recursion on 
the rank of $a_n$: 
%
\begin{itemize}
\item[] If $a_n\in\{\bot,\Nil,\Fun(f)\}$, then $\cd{\vec a,a_n} \eqdef a_n$.
\item[] If $a_n=\Amb(b_n,c_n)$, then let 
$\vec b = b_0,\ldots,b_{n-1}$ and $\vec c = c_0,\ldots,c_{n-1}$
such that $a_i = \Amb(b_i,c_i)$ if $a_i\neq\bot$, and $b_i=c_i=\bot$ %,
 otherwise.
Now set 
\[ \cd{\vec a,a_n} \eqdef \left\{
\begin{array}{ll}
   \bot            & \hbox{if }b_n=c_n=\bot\\ 
    \cd{\vec b,b_n} & \hbox{if $b_n\neq\bot$ and, for all $i<n$, $b_i=\bot$ implies $c_i=\bot$}\\
    
   \cd{\vec c,c_n} & \hbox{otherwise}
\end{array}
\right.
\]
\item[] If $a_n=\Left(b_n)$, then let 
$\vec b = b_0,\ldots,b_{n-1}$ such that $a_i = \Left(b_i)$ if $a_i\neq\bot$, 
and $b_i=\bot$, otherwise.
Set $\cd{\vec a,a_n} \eqdef \Left(\cd{\vec b,b_n})$.
The cases that $a_n$ begins with $\Right$ or $\Pair$ are similar.
\end{itemize}
%
It is easy to see that 
if $a_n\dle a_{n+1}$, then $\cd{\vec a,a_n}\dle\cd{\vec a,a_n,a_{n+1}}$.

Now let $a\in D$. Since $D$ is algebraic with a countable base, 
there exists an increasing sequence of compact elements, $(a_n)_{n\in\NN}$,  
that has $a$ as its supremum. 
We show that for
$d \eqdef \bigsqcup\{\cd{a_0,...a_{n-1},a_n}\mid n\in\NN\}$, 
we have $d\in\ddata(a)$. 
To this end, we define the relation $P(a,d)$ %`$d\in P(a)$' 
as `$a$ is the supremum of an increasing sequence 
$(a_n)_{n\in\NN}$ of compact elements such that 
$d = \bigsqcup\{\cd{a_0,\ldots,a_{n-1},a_n}\mid n\in\NN\}$'
and prove:
%
\paragraph{Claim} 
%
If $P(a,d)$ then $d\in\ddata(a)$, for all $d,a\in D$.

We prove the Claim by coinduction: Assume 
$P(a,d)$. %$d\in P(a)$.
We have to show that the right-hand side of the definition of 
$\ddata$ in Section \ref{sub-denot}
%`$d\in\ddata$'
holds if $\ddata$ is replaced by $P$.
%
Let $a$ be the supremum of the increasing sequence $(a_n)_{n\in\NN}$ 
of compact elements.
%

\emph{Case $a=\Amb(b,c)$}.
%
Then, for some $m$, $a_k=\bot$ for all $k<m$ and $a_k=\Amb(b_k,c_k)$
for all $k\ge m$. Set $b_k=c_k=\bot$ for $k<m$.
%
If $b=c=\bot$ then $\cd{a_0,\ldots,a_{k-1},a_k}=\bot$
%$\cd{a_0,\ldots,a_{n-1},a_n}=\bot$
 for all $k$ 
and therefore $d=\bot$ which is correct.
%
If $b$ and $c$ are not both $\bot$, then there is a least $n\ge m$ such that 
$a_n =\Amb(b_n,c_n)$ and
$b_n,c_n$ are not both $\bot$. 
%
If $b_n\neq\bot$, then 
$\cd{a_0,\ldots,a_{k-1},a_k}= \cd{b_0,\ldots,b_{k-1},b_k}$ for all $k\ge n$.
Since for $k<n$, $\cd{a_0,\ldots,a_{k-1},a_k}= \bot=\cd{b_0,\ldots,b_{k-1},b_k}$,
$P(b, d)$ holds
%$d\in P(b)$ 
and we are done. 
%
If $b_n=\bot$, then $c_n\neq\bot$ and, with a similar argument,  
%
$\cd{a_0,\ldots,a_{k-1},a_k}= \cd{c_0,\ldots,c_{k-1},c_k}$ for all $k \geq n$
which implies that $P(c,d)$ holds and we are done again. 


The other cases are easy.

\medskip

%
(\ref{lem-data-nonempty-sing})
To show that $a\in\Data$ implies $\{a\}\subseteq\ddata(a)$, 
one sets $P(a,b) \eqdef a=b\in\Data$ and shows 
$P\subseteq\ddata$ by coinduction.

To show that  $a\in\Data$ implies $\ddata(a)\subseteq \{a\}$, 
one proves that for all compact $a_0\in D$,
if $a\in\Data$ and $d\in\ddata(a)$, then $a_0\dle a$ iff $a_0\dle d$. 
The proof can be done by induction
on $\rk(a_0)$. 

\medskip

%
(\ref{lem-data-nonempty-data})
One shows $\forall d\,(\exists a\in\regD\,(d\in\ddata(a)) \to d\in\Data)$
by coinduction.
%
\end{proof}
%
\begin{rem}
The proof of (\ref{lem-data-nonempty-ne})  
is constructive since for every compact approximating sequence 
of $a$, a compact approximating sequence of some element in $\ddata(a)$ is constructed. 
In particular, if $a$ is computable, then $\ddata(a)$ contains a computable element. 
However, there can be no computable function $f:D\to D$ such that $f(a)\in\ddata(a)$ 
for all $a\in D$, since such a function cannot even be monotone: 
We would necessarily have $f(\Amb(0,\bot))=0$ and $f(\Amb(\bot,1))=1$, hence
$f$ would map the consistent inputs $\Amb(0,\bot)$ and $\Amb(\bot,1)$ 
(`consistent' meaning `having a supremum') to the inconsistent outputs $0$ and $1$,
which is impossible for a monotone function.
\end{rem}
%
\begin{lem}
\label{lem-data-compact}
\begin{enumerate}
%
\item\label{lem-data-compact-elem}
If $a_0$ is compact (in $D$), 
then all elements of $\ddata(a_0)$ are compact.
%
\item\label{lem-data-compact-approx}
If $a_0$ is a compact approximation of $a$,
then for every $d_0\in\ddata(a_0)$ there exists 
some $d\in\ddata(a)$ such that $d_0\dle d$.
%
\item\label{lem-data-compact-reg}
If $a$ is regular and $d\in\ddata(a)$, 
then for every compact approximation $d_0$ of 
$d$ there exists a compact approximation $a_0$ of $a$ 
such that $d_0\in\ddata(a_0)$.
%
\item\label{lem-data-compact-sup}
If $a,b,c$ are compact such that $c = a\sqcup b$,
then for every $w\in\ddata(c)$, $w\in\ddata(a)\cup\ddata(b)$ or 
$w = u \sqcup v$ for some $u\in\ddata(a)$ and $v\in\ddata(b)$.
\end{enumerate}
\end{lem}

\begin{proof}
%
(\ref{lem-data-compact-elem})
Easy induction on $\rk(a_0)$.
\medskip

%
(\ref{lem-data-compact-approx})
%
Induction on $\rk(a_0)$.
%
If $a_0\in\{\bot,\Amb(\bot,\bot)\}$, then $d_0=\bot$ 
and we can take $d$ to be any element of 
$\ddata(a)$ (which is nonempty, as shown in 
Lemma~\ref{lem-data-nonempty}~(\ref{lem-data-nonempty-ne})).
%
If $a_0=\Amb(b_0,c_0)$ where, w.l.o.g., $b_0\neq\bot$ and $d_0\in\ddata(b_0)$, then,
since $a=\Amb(b,c)$ with $b_0\dle b$, by i.h.\ there is some $d\in\ddata(b)$ 
(hence also $d\in\ddata(a)$) with $d_0\dle d$.
%
The other cases are easy.
\medskip

%(3) 
(\ref{lem-data-compact-reg})
%
Induction on $\rk(d_0)$. 
The most complicated case is $a = \Amb(b,c)$ where $b\neq\bot$ and $d\in\ddata(b)$. 
Since $a$ is regular, $b$ starts with one of the 
constructors $\Fun,\Left,\Right,\Pair$.
W.l.o.g.\ assume $b=\Pair(b^1,b^2)$. Then $d=\Pair(d^1,d^2)$ with $d^i\in\ddata(b^i)$
and $d_0=\Pair(d^1_0,d^2_0)$ with $d^i_0\dle d^i$. 
By i.h.\ there are compact approximations
$b^i_0$ of $b^i$ such that $d^i_0\in\ddata(b^i_0)$. 
Hence $\Pair(d^1_0,d^2_0)\in\ddata(\Pair(b^1_0,b^2_0))$. 
Hence, $a_0 \eqdef \Amb(\Pair(b^1_0,b^2_0),\bot)$ is a compact approximation of $a$
with $d_0\in\ddata(a_0)$.
\medskip

%
(\ref{lem-data-compact-sup})
%
Induction on $\rk(c)$.
%
We assume that $a,b,c,w$ are all different from $\bot$, otherwise, the solution is easy.

Case $c=\Amb(c^1,c^2)$. 
Then $a=\Amb(a^1,a^2)$ and $b=\Amb(b^1,b^2)$ with $c^i = a^i\sqcup b^i$,
and, w.l.o.g., $c^1\neq\bot$ and $w\in\ddata(c^1)$.
By i.h., there are two cases: 
1. $w\in\ddata(a^1)\cup\ddata(b^1)$, say $w\in\ddata(a^1)$. 
Then $a^1\neq\bot$ and $w\in\ddata(a)$.
2. $w = u \sqcup v$ for some $u\in\ddata(a^1)$ and $v\in\ddata(b^1)$.
If $a^1=\bot$ or $b^1=\bot$, say $a^1=\bot$, then $u=\bot$ and $w=v$ and $w\in\ddata(b)$.
If $a^1$ and $b^1$ are both different from $\bot$, then $u\in\ddata(a)$ and $v\in\ddata(b)$ 
and we are done as well.

The other cases are easy.
\end{proof}

%
For the Faithfulness Theorem (Theorem~\ref{thm-faithfulness} below) to hold, 
an admissibility condition
must be imposed on $\CFP$ formulas which we describe now.


A CFP formula is a \emph{functional implication} if it is of the form 
$A \to B$ where $A$ and $B$ are both non-Harrop. 
%

A CFP expression is \emph{quasi-closed} if it contains no free predicate variables. 


A $\CFP$ expression is \emph{admissible} if it is quasi-closed, 
the concurrency operator and restriction occur only at s.p.~positions,
every restriction $\rt{A}{B}$ has a Harrop premise $A$,
and every occurrence of a functional implication that is not 
part of a Harrop expression, is part of a quasi-closed subexpression 
without $\Set$ and restriction.\footnote{The definition of admissibility in~\cite{CFPesop} is similar but disallows restrictions.}


For example, the expressions
$\Set(\mu(\lambda X\,\lambda x\,(x=0 \lor \forall y\,(\NN(y)\to X(f(x,y))))))$
and $\mu(\lambda X\,\lambda x\,\Set(\forall y\,(\NN(y)\to X(f(x,y)))))$
are admissible\footnote{$\mu(\lambda X\,\lambda x\,\Set(\forall y\,(\NN(y)\to X(f(x,y)))))$ contains the functional implication $\NN(y)\to X(f(x,y))$, but only as part of a Harrop expression (namely itself).}. 
On the other hand, the expression
$\mu(\lambda X\,\lambda x\,\Set(x=0 \lor \forall y\,(\NN(y)\to X(f(x,y)))))$
is not since it contains 
the functional implication $\NN(y)\to X(f(x,y))$ 
which is not contained in a Harrop expression and the minimal %only 
quasi-closed
subexpression it is contained in is
$\lambda X\,\lambda x\,\Set(x=0 \lor \forall y\,(\NN(y)\to X(f(x,y)))))$
which does contain $\Set$.

Further examples of admissible expressions are the predicate
$\ConSD$ from Example~\ref{example-ConsSD}, 
$\lambda x\,\Set((x\leq 0 \lor x\geq 0) \lor |x| \leq 1/2)$,
as well as the coinductive predicate $\myC_2$ from Section~\ref{sec-gray},
$\nu(\lambda X\,\lambda x\,(|x|\le 1 \land \Set(\exists\, d (\SD(d) \land X(2x-d)))))$. 

In the proof of the Faithfulness Theorem below, we will frequently use the 
fact that $\RCFP$ derivations are closed under substitution of predicates and 
respect extensional equality of predicates (recall from Section~\ref{sub-IFP}
that $P \equiv Q$ stands for $P \subseteq Q \land Q \subseteq P$):
%
\begin{lem}
\label{lem-respect}
%
\begin{enumerate}
%
\item\label{lem-respect-subst}
If $\RCFP$ proves $\Gamma\vdash A$, then $\RCFP$ proves $\Gamma[P/X] \vdash A[P/X]$.
%
\item\label{lem-respect-equiv}
$\RCFP$ proves $P\equiv Q \vdash A[P/X] \equiv A[Q/X]$.
%
\item\label{lem-respect-hyp}
$\RCFP$ proves $\Gamma\vdash A[P/X]$ if and only 
if $\RCFP$ proves $\Gamma, P\equiv X\vdash A$, provided $X$ is not free in $P,\Gamma$.
\end{enumerate}
\end{lem}
%
\begin{proof}
%
(\ref{lem-respect-subst})
and 
%
(\ref{lem-respect-equiv})
 can be easily proven by induction on $A$.
%
(\ref{lem-respect-hyp})
is an immediate consequence of (\ref{lem-respect-subst}) and (\ref{lem-respect-equiv}).
\end{proof}
%
\begin{thm}[Faithfulness]
\label{thm-faithfulness}
%
If $a\in D$ realizes an admissible well-formed $\CFP$ formula $A$, 
then all $d\in\ddata(a)$ realize $A^-$.

Formally, $\RCFP$ proves 
$\forall\vec x\,\forall a\,(\rea(A)(a) \to \ddata(a)\subseteq \rea(A^-))$
where $\vec x$ are the free object variables of $A$.
%
\end{thm}
%
\begin{proof}
%
The proof is accomplished through a series of definitions and claims. 


We call a $\CFP$ expression \emph{parametrically admissible (p-admissible)}
if it satisfies the same conditions as an admissible expression except
that it is no longer required to be quasi-closed but s.p.\ occurrences 
of free
predicate variables are allowed.
Hence, a $\CFP$ expression is p-admissible iff 
%
the concurrency operator $\Set$, restriction, 
and free predicate variables occur only at s.p. positions,
every restriction has a Harrop premise,
and every occurrence of a functional implication is part of a quasi-closed
subexpression without $\Set$ or restriction.


A \emph{p-admissible type} is a type that contains type variables 
and $\Am$ only at s.p.\ positions and every function type is
part of a closed subtype without $\Am$.

Clearly, if $P$ is p-admissible, then $\tau(P)$ is p-admissible
(easy structural induction on $P$).


We call an $\RCFP$ predicate $P$ whose last argument place 
ranges over $D$ \emph{regular}
if $P(\vec x,a)$ implies that $a$ is regular.
More precisely, ``$P$ is regular'' stands for the formula 
$\forall\vec x\forall a\,(P(\vec x,a) \to \regD(a))$.



For any $\RCFP$ predicate $P$ whose last argument place ranges over $D$
we define a regular predicate $P'$ of the same arity by
%
\[P'(\vec x,a) \eqdef                   a\in\regD \land 
                  \forall d\in\ddata(a)\, P(\vec x,d)\,.\]
%

We also define  
%
\[\ddata(P)(\vec x,d) \eqdef \exists a\in D\,(P(\vec x,a) \land d\in\ddata(a)).\]
%
Clearly, if $P$ is regular, then $P\subseteq Q'$ iff $\ddata(P)\subseteq Q$.

For the special case of a subdomain $\alpha$, considered as the unary predicate 
$\lambda a\,.\,a:\alpha$, 
$\ddata(\alpha) \equiv \bigcup\{\ddata(a)\mid a:\alpha\}$, 
and if $\alpha$ is regular, then
$\alpha\subseteq\beta'$ iff $\ddata(\alpha)\subseteq\beta$. 




\paragraph{Claim 1.}
%
If $\alpha$ is a subdomain of $D$, 
then so is $\alpha'$.



\paragraph{Proof of Claim~1.}
%
Let $\alpha$ be a subdomain of $D$. 
We show that $\alpha'$ satisfies the 
characterizing properties (i) and (ii) of subdomains from Lemma~\ref{lem-subdom}.

(i) Let $a\in\alpha'$ and $a_0$ a compact approximation of $a$. We have to show that
$a_0\in\alpha'$. Hence assume $d_0\in\ddata(a_0)$. 
Then, by 
%
Lemma~\ref{lem-data-compact}~(\ref{lem-data-compact-approx}), 
there exists some $d\in\ddata(a)$ such that $d_0\dle d$.
Since $a\in\alpha'$, $d\in\alpha$. Hence $d_0\in\alpha$, since, being a subdomain, 
$\alpha$ is downward closed.
%
Conversely, assume all compact approximations of $a$ are in $\alpha'$ 
and let $d\in\ddata(a)$.
We have to show that $d\in\alpha$. Since $\alpha$ is a subdomain, 
it suffices to show that all
compact approximations of $d$ are in $\alpha$. 
Hence let $d_0$ be a compact approximation of $d$.
Then, by 
%
Lemma~\ref{lem-data-compact}~(\ref{lem-data-compact-reg}), 
(by 
%
Lemma~\ref{lem-regular}~(\ref{lem-regular-sub}), 
$a$ is regular since all its compact approximations are), 
there is  some compact approximation $a_0$ of $a$ 
such that $d_0\in\ddata(a_0)$. Since, by assumption, $a_0\in\alpha'$, 
it follows $d_0\in\alpha$.

(ii) Let $a_0,b_0$ be compact and consistent elements of $\alpha'$. 
We have to show that 
$c_0 \eqdef a_0\sqcup b_0 \in \alpha'$. Hence let $d_0\in\ddata(c_0)$.
By 
%
Lemma~\ref{lem-data-compact}~(\ref{lem-data-compact-sup}), 
$d_0\in\ddata(a_0)\cup\ddata(b_0)$, 
in which case $d_0\in\alpha$ since $a_0,b_0\in\alpha'$,
or there are $a_1\in\ddata(a_0)$ and 
$b_1\in\ddata(b_0)$ such that $d_0 = a_1 \sqcup b_1$, in which case, 
since $a_1,b_1\in\alpha$ and $\alpha$ is a subdomain, $d_0\in\alpha$ as well.
This completes the proof of Claim~1.
 
%
For any type $\rho$ and $i=1,2$ let $\variant{\rho}{i}$ be obtained 
from $\rho$ by replacing 
each free type variable $\alpha$ by the fresh type variable $\alpha_i$.
%
Further, let $\rho^-$ be obtained from $\rho$ by deleting all occurrences of $\Am$.


\paragraph{Claim 2.}
%
For every p-admissible type $\rho$,
$\variant{\rho}{1}\subseteq \variant{\rho^-}{2}'$ 
under the assumption that 
$\alpha_i\subseteq\regD$ ($i=1,2$) 
and
$\alpha_1\subseteq\alpha_2'$
for every free type variable $\alpha$ of $\rho$.

\paragraph{Proof of Claim 2.}
%
Induction on $\rho$. 

If $\rho$ is closed and doesn't contain $\Am$ 
(by p-admissibility, this includes the case that $\rho$ is a function type),
then $\variant{\rho}{1} = \variant{\rho^-}{2} = \rho$ and, 
by 
%
Lemma~\ref{lem-regular}~(\ref{lem-regular-data})
$\rho\subseteq\Data$.
Hence 
$\ddata(\rho)\subseteq\rho$, by 
%
Lemma~\ref{lem-data-nonempty}~(\ref{lem-data-nonempty-sing}),  
and we are done.

Now assume that the above case does not apply.

If $\rho$ is a type variable $\alpha$, then 
$\variant{\rho}{1} = \alpha_1 \subseteq \alpha_2' = \variant{\rho^-}{2}'$.

If $\rho = \Am(\sigma)$, then $\rho^- = \sigma^-$ and, by the i.h.,\ 
$\variant{\sigma}{1} \subseteq \variant{\sigma^-}{2}'$. 
%
Let $a:\variant{\rho}{1}$. Then either $a=\bot$, 
in which case $\ddata(a) = \{\bot\} \subseteq\variant{\sigma^-}{2}$,
or else $a=\Amb(b,c)$ with $b,c:\variant{\sigma}{1}$, in which case 
%
$\ddata(a) \subseteq \ddata(b) \cup \ddata(c) \cup\{\bot\} \subseteq \variant{\sigma^-}{2}$.

If $\rho=\tfix{\alpha}{\sigma}$, then we have, up to bound renaming, 
$\variant{\rho}{1} = \tfix{\alpha_1}{\variant{\sigma}{1}}$ and
$\variant{\rho^-}{2} = \tfix{\alpha_2}{\variant{\sigma^-}{2}}$.
Since $\variant{\rho^-}{2}$ is a subdomain and hence, by Claim~1, 
so is $\variant{\rho^-}{2}'$, we can set
\begin{enumerate}
\item[] (1) $\alpha_1 \equiv \variant{\rho^-}{2}'$ 
\end{enumerate}
%
and achieve our goal (of proving $\variant{\rho}{1}\subseteq \variant{\rho^-}{2}'$) 
by proving $\variant{\sigma}{1} \subseteq \alpha_1$
(to be precise, by ``we can set'' we mean that we construct the formal proof 
with the extra assumption (1)).
Setting further
\begin{enumerate}
\item[] (2) $\alpha_2 \equiv \variant{\rho^-}{2}$
\end{enumerate}
%
we have $\alpha_1\subseteq\alpha_2'$ and therefore, by the induction hypothesis,
($\variant{\rho^-}{2}'$ and $\variant{\rho^-}{2}$ are both regular subdomains,
and $\sigma$ is p-admissible, 
since otherwise we would be in the first case),
$\variant{\sigma}{1}\subseteq \variant{\sigma^-}{2}'$. 
We further have $\variant{\sigma^-}{2} \equiv \variant{\rho^-}{2}$, by (2).
%
Therefore,
%
\[ \variant{\sigma}{1}\subseteq\variant{\sigma^-}{2}' \equiv  
\variant{\rho^-}{2}' \equiv \alpha_1.\]
%
The cases that $\rho$ is a product or a sum are easy.
This completes the proof of Claim~2.
\medskip

For any $\CFP$ predicate $P$ and $i=1,2$,
let $\variant{\rea(P)}{i}$ be obtained from $\rea(P)$ 
by replacing 
each free predicate variable $\reali{X}$ by
the fresh predicate variable $\reali{X}_i$
and
each free type variable $\alpha_X$
by the fresh type variable $(\alpha_X)_i$. 

\paragraph{Claim 3.}
%
If $P$ is p-admissible, then
$\variant{\rea(P)}{1}\subseteq \variant{\rea(P^-)}{2}'$ 
under the assumption that 
$(\alpha_X)_i\subseteq\regD$ 
($i=1,2$),
and furthermore $(\alpha_X)_1\subseteq(\alpha_X)_2'$ 
and $\reali{X}_1 \subseteq \reali{X}_2'$, 
for every free predicate variable $X$ of $P$.
%

Note that the Faithfulness Theorem is a special case of 
Claim~3:
%
If $a\in D$ realizes the admissible formula $A$, 
then, by 
Claim~3, 
$\rea(A^-)'(a)$ and therefore
all $d\in\ddata(a)$ realize $A^-$.
%
Therefore, the proof of Claim~3 completes the whole proof of the theorem.


\paragraph{Proof of Claim 3.}
%
Structural induction on p-admissible expressions $P$.

If $P$ is a Harrop predicate, then, by p-admissibility, 
it contains neither free predicate variables,
nor $\Set$, nor restriction, and therefore 
$\variant{\rea(P)}{1}=\variant{\rea(P^-)}{2} = \reah(P)$.
Hence, it suffices to show
%
$\forall a\,((a =\Nil\land\reah(P)) \to \forall d\in\ddata(a)\,(d=\Nil \land\reah(P)))$, 
%
which is a triviality since $\ddata(\Nil)=\{\Nil\}$.

From now on we assume that $P$ is non-Harrop. 

If $P$ is quasi-closed and contains neither $\Set$ nor restriction 
(because $P$ is p-admissible, this includes the case that $P$ 
is a functional implication), then $\rea(P)$ is quasi-closed and $P = P^-$.
Therefore, $\variant{\rea(P)}{1}= \variant{\rea(P^-)}{2} = \rea(P)$.
By 
%
Lemma~\ref{lem-regular}~(\ref{lem-regular-data}) 
and 
Lemma~\ref{lem-realizers-typed}, 
$\rea(P)\subseteq\Data$. Hence $\ddata(\rea(P)) = \rea(P)$ 
by 
%
Lemma~\ref{lem-data-nonempty}~(\ref{lem-data-nonempty-sing}),
and we are done.

If $P$ is a predicate variable $X$, then we have to prove
%
$\reali{X}_1\subseteq\reali{X}_2'$ which holds by the assumption.

From now on we assume that none of the above cases applies.

If $P$ is 
$A \to B$, then $A$ has  no free predicate variables
and $B$ is non-Harrop.
%
Since the case of a functional implication is excluded, $A$ must be Harrop.
Assume $\variant{\rea(A\to B)}{1}(b)$, that is, $b:\variant{\tau(B)}{1}$ and 
$\reah(A) \to \variant{\rea(B)}{1}(b)$.
We have to show $\variant{\rea(A\to B^-)}{2}'(b)$, that is,  
$b\in\regD$ and 
$\forall d\in\ddata(b)\,\variant{\rea(A\to B^-)}{2}(d)$, that is, for all
$d\in\ddata(b)$, $d : \variant{\tau(B^-)}{2}$ and 
$\reah(A) \to \variant{\rea(B^-)}{2}(d)$.
%
$b\in\regD$ holds since, 
by Claim~2, 
$\variant{\tau(B)}{1}\subseteq\variant{\tau(B)^-}{2}'\subseteq\regD$, 
and $\tau(B^-)=\tau(B)^-$.
%
Let $d\in\ddata(b)$.
$d : \variant{\tau(B^-)}{2}$ holds,
since, by 
Claim~2,  
$b:\variant{\tau(B^-)}{2}'$ 
(to apply Claim~2 we need the assumptions $\alpha_1\subseteq\alpha_2'$).
Assume $\reah(A)$. Then $\variant{\rea(B)}{1}(b)$ and, 
by the structural induction hypothesis (clearly, $B$ is again p-admissible), 
$\variant{\rea(B^-)}{2}(d)$.


If $P$ is $\rt{A}{B}$ where $A$ is a Harrop formula, then $P^-$ is $A \to B^-$.
Assume $\variant{\rea(P)}{1}(b)$. 
Then, clearly, $\variant{\rea(A\to B)}{1}(b)$. 
Hence, the rest of the proof is exactly as in the previous case
(even for the case that $B$ is Harrop since, as one easily checks, 
the above proof for $A\to B$ is also valid if $B$ is Harrop).

If $P$ is $\Set(B)$, then 
$\variant{\rea(P^-)}{2} = \variant{\rea(B^-)}{2}$ and,  
since $B$ is again p-admissible, by the induction hypothesis,
$\variant{\rea(B)}{1} \subseteq \variant{\rea(B^-)}{2}'$
Assume $\variant{\rea(P)}{1}(c)$ and $d\in\ddata(c)$.
Then $c = \Amb(a,b)$ with $a,b : \variant{\tau(B)}{1}$ and 
\[(a \neq\bot \land d\in\ddata(a) \land \variant{\rea(B)}{1}(a))\lor 
  (b \neq\bot \land d\in\ddata(b) \land \variant{\rea(B)}{1}(b)).\]
%
In either case, it follows $\variant{\rea(B^-)}{2}(d)$.


If $P$ is $\mu\,(\lambda X\,.\,Q)$, 
then $\rea(P)$ is 
$\mu(\lambda\reali{X}\,.\,\rea(Q)[\tfix{\alpha_X}{\tau(Q)}/\alpha_X])$
whereas $\rea(P^-)$ is 
$\mu(\lambda\reali{X}\,.\,\rea(Q^-)[\tfix{\alpha_X}{\tau(Q^-)}/\alpha_X])$.
Hence, setting 
(see Lemma~\ref{lem-respect}~(\ref{lem-respect-hyp}) for a justification)
\begin{itemize}
\item[] $(\alpha_X)_1 \equiv\variant{(\tfix{\alpha_X}{\tau(Q)})}{1}$
\item[] $(\alpha_X)_2 \equiv\variant{(\tfix{\alpha_X}{\tau(Q^-)})}{2}$
\end{itemize} 
we have
$\variant{\rea(P)}{1} \equiv \mu(\lambda\reali{X}_1\,.\,\variant{\rea(Q)}{1})$,
and 
$\variant{\rea(P^-)}{2} \equiv \mu(\lambda\reali{X}_2\,.\,\variant{\rea(Q^-)}{2})$. 
Therefore, we can prove the assertion of Claim~3
(which is $\variant{\rea(P)}{1} \subseteq \variant{\rea(P^-)}{2}'$)
by s.p.\ induction (second rule for $\mu$ in Table~\ref{table-proof-ifp}) i.e.,
it suffices to show $\variant{\rea(Q)}{1} \subseteq \variant{\rea(P^-)}{2}'$
under the assumption 
%
\begin{itemize}
\item[] $\reali{X}_1 \equiv \variant{\rea(P^-)}{2}'$.
\end{itemize}
Setting further 
\begin{itemize}
\item[] $\reali{X}_2 \equiv \variant{\rea(P^-)}{2}$
\end{itemize} 
we have $\reali{X}_1 \subseteq (\reali{X}_2)'$, trivially, and furthermore,  
by Claim~2,  
$(\alpha_X)_1 \subseteq ((\alpha_X)_2)'$ since 
$(\tfix{\alpha_X}{\tau(Q)})^-=\tfix{\alpha_X}{\tau(Q^-)}$.
Therefore, by the structural induction hypothesis
($Q$ is p-admissible since otherwise we would be in the second case),
$\variant{\rea(Q)}{1} \subseteq \variant{\rea(Q^-)}{2}'$. 
By the closure axiom, 
$\variant{\rea(Q^-)}{2}\subseteq \reali{X}_2$. 
Hence, $\variant{\rea(Q^-)}{2}' \subseteq (\reali{X}_2)'$
by the monotonicity of the operation $\cdot'$.
But $(\reali{X}_2)' \equiv \variant{\rea(P^-)}{2}'$.

If $P$ is $\nu\,(\lambda X\,.\,Q)$, then we work with $\ddata(\cdot)$ 
instead of $\cdot'$
using the earlier mentioned fact that 
if $Y \subseteq \regD$, then
$Y \subseteq Z'$ iff $\ddata(Y) \subseteq Z$.
%
Since the condition $\reali{Y}_1 \subseteq\reali{Y}_2'$ implies that 
$\reali{Y}_1\subseteq\adummy{\regD}$ for all free predicate variables $Y$ of $P$,
and these variables are all s.p., 
%
Lemma~\ref{lem-regular}~(\ref{lem-regular-rea}) 
yields that 
$\variant{\rea(P)}{1} \subseteq\regD$. 
Therefore,
the assertion to be proven is equivalent to 
$\ddata(\variant{\rea(P)}{1}) \subseteq \variant{\rea(P^-)}{2}$.
%
%
$\rea(P)$ is $\nu(\lambda\reali{X}\,.\,\rea(Q)[\tfix{\alpha_X}{\tau(Q)}/\alpha_X])$
and $\rea(P^-)$ is 
$\nu(\lambda\reali{X}\,.\,\rea(Q^-)[\tfix{\alpha_X}{\tau(Q^-)}/\alpha_X])$.
%
This means that, setting $(\alpha_X)_i$ as before, we have that
$\variant{\rea(P)}{1} \equiv \nu(\lambda\reali{X}_1\,.\,\variant{\rea(Q)}{1})$,
and 
$\variant{\rea(P^-)}{2} \equiv \nu(\lambda\reali{X}_2\,.\,\variant{\rea(Q^-)}{2})$. 
%
Therefore, 
$\ddata(\variant{\rea(P)}{1}) \subseteq \variant{\rea(P^-)}{2}$ can be proven
by 
s.p.\ coinduction, that is, we show 
$\ddata(\variant{\rea(P)}{1}) \subseteq \variant{\rea(Q^-)}{2}$
under the assumption 
%
\begin{itemize}
\item[] $\reali{X}_2 \equiv \ddata(\variant{\rea(P)}{1})$.
\end{itemize}
%
Setting 
%
\begin{itemize}
\item[] $\reali{X}_1 \equiv \variant{\rea(P)}{1}$
\end{itemize} 
we have 
%
$\ddata(\reali{X}_1) \subseteq \reali{X_2}$, 
hence $\reali{X}_1 \subseteq \reali{X_2}'$, 
since $\variant{\rea(P)}{1} \subseteq\regD$.
%
Therefore, by the structural induction hypothesis,
$\variant{\rea(Q)}{1} \subseteq \variant{\rea(Q^-)}{2}'$,  i.e.\ 
$\ddata(\variant{\rea(Q)}{1}) \subseteq \variant{\rea(Q^-)}{2}$.
%
Finally, by the coclosure axiom, 
$\reali{X}_1 \subseteq \variant{\rea(Q)}{1}$ and therefore,
by the monotonicity of the operation $\ddata(\cdot)$, we get
$\ddata(\variant{\rea(P)}{1}) \subseteq \ddata(\variant{\rea(Q)}{1})$.

In all other cases (conjunction, disjunction, quantifiers), the 
induction hypothesis applies in a straightforward way.


%
\end{proof}
\bigskip

Theorems~\ref{thm-soundnessI} and~\ref{thm-faithfulness} imply:
%
 \begin{thm}[Soundness Theorem II]
 \label{thm-soundnessII}
 %
 From a $\CFP$ proof of a well-formed admissible formula $A$ from a set of axioms 
one can extract a program $M$ such that
$\vdash M :\tau(A)$ and
$\RCFP$ proves, from the same axioms,
that all $d\in\ddata(\val{M})$ realize $A^-$,
that is, $\RCFP$ proves the formula 
%
$\forall d\in\ddata(M)\,\ire{d}{A^-}$.
%
\end{thm}
%


Theorems~\ref{thm-soundnessII} and~\ref{thm:data}, together with 
classical soundness yield:
%
\begin{thm}[Program Extraction]
  \label{thm-pe}
 %
 From a $\CFP$ proof of a well-formed admissible 
formula $A$ from a set of axioms
one can extract a program $M$ such that $\vdash M:\tau(A)$ and for any 
computation 
$M =  M_0 \newprintp M_1 \newprintp  \ldots$,
the limit, $\sqcup_{i \in \NN} (M_i)_{D}$, realizes  $A^-$ in every  
model of the axioms.

\end{thm}

 
%
\begin{rem}
%
The theorems above can be generalized by dropping in the definition of 
admissibility the condition that restrictions must have Harrop premises.
The definition of $A^-$ must then be modified by replacing $\rt{A}{B}$, where
$A$ is non-Harrop, by $\neg\neg A \to B$ (instead of $A\to B$). 
Since, by the rules (rest-antimon) and
(rest-stab), the formulas $\rt{A}{B}$ and $\rt{\neg\neg A}{B}$ are equivalent 
and, moreover, have the same realizers, 
the proof of Theorem~\ref{thm-faithfulness} requires
only minimal changes and the Theorems~\ref{thm-soundnessII} and~\ref{thm-pe}
are unchanged.
 
%
\end{rem}

