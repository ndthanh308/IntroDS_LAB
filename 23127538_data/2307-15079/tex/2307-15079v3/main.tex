% Version 1.2 of SN LaTeX, November 2022
%
% See section 11 of the User Manual for version history 
%
%%%%%%%%%%%%%%%%%%%%%%%%%%%%%%%%%%%%%%%%%%%%%%%%%%%%%%%%%%%%%%%%%%%%%%
%%                                                                 %%
%% Please do not use \input{...} to include other tex files.       %%
%% Submit your LaTeX manuscript as one .tex document.              %%
%%                                                                 %%
%% All additional figures and files should be attached             %%
%% separately and not embedded in the \TeX\ document itself.       %%
%%                                                                 %%
%%%%%%%%%%%%%%%%%%%%%%%%%%%%%%%%%%%%%%%%%%%%%%%%%%%%%%%%%%%%%%%%%%%%%

%%\documentclass[referee,sn-basic]{sn-jnl}% referee option is meant for double line spacing

%%=======================================================%%
%% to print line numbers in the margin use lineno option %%
%%=======================================================%%

%%\documentclass[lineno,sn-basic]{sn-jnl}% Basic Springer Nature Reference Style/Chemistry Reference Style

%%======================================================%%
%% to compile with pdflatex/xelatex use pdflatex option %%
%%======================================================%%

%%\documentclass[pdflatex,sn-basic]{sn-jnl}% Basic Springer Nature Reference Style/Chemistry Reference Style


%%Note: the following reference styles support Namedate and Numbered referencing. By default the style follows the most common style. To switch between the options you can add or remove “Numbered” in the optional parenthesis. 
%%The option is available for: sn-basic.bst, sn-vancouver.bst, sn-chicago.bst, sn-mathphys.bst. %  
 
%%\documentclass[sn-nature]{sn-jnl}% Style for submissions to Nature Portfolio journals
%%\documentclass[sn-basic]{sn-jnl}% Basic Springer Nature Reference Style/Chemistry Reference Style
\documentclass[sn-mathphys,Numbered]{sn-jnl}% Math and Physical Sciences Reference Style
%%\documentclass[sn-aps]{sn-jnl}% American Physical Society (APS) Reference Style
%%\documentclass[sn-vancouver,Numbered]{sn-jnl}% Vancouver Reference Style
%%\documentclass[sn-apa]{sn-jnl}% APA Reference Style 
%%\documentclass[sn-chicago]{sn-jnl}% Chicago-based Humanities Reference Style
%%\documentclass[default]{sn-jnl}% Default
%%\documentclass[default,iicol]{sn-jnl}% Default with double column layout

%%%% Standard Packages
%%<additional latex packages if required can be included here>

\usepackage{graphicx}%
\usepackage{multirow}%
\usepackage{amsmath,amssymb,amsfonts}%
\usepackage{amsthm}%
\usepackage{mathrsfs}%
\usepackage[title]{appendix}%
\usepackage{xcolor}%
\usepackage{textcomp}%
\usepackage{manyfoot}%
\usepackage{booktabs}%
\usepackage{algorithm}%
\usepackage{algorithmicx}%
\usepackage{algpseudocode}%
\usepackage{listings}%
\usepackage{tabularx}%
\usepackage{float}%
\usepackage{lineno}
% \linenumbers
%%%%

%%%%%=============================================================================%%%%
%%%%  Remarks: This template is provided to aid authors with the preparation
%%%%  of original research articles intended for submission to journals published 
%%%%  by Springer Nature. The guidance has been prepared in partnership with 
%%%%  production teams to conform to Springer Nature technical requirements. 
%%%%  Editorial and presentation requirements differ among journal portfolios and 
%%%%  research disciplines. You may find sections in this template are irrelevant 
%%%%  to your work and are empowered to omit any such section if allowed by the 
%%%%  journal you intend to submit to. The submission guidelines and policies 
%%%%  of the journal take precedence. A detailed User Manual is available in the 
%%%%  template package for technical guidance.
%%%%%=============================================================================%%%%

%\jyear{2021}%

%% as per the requirement new theorem styles can be included as shown below
% \theoremstyle{thmstyleone}%
% \newtheorem{theorem}{Theorem}%  meant for continuous numbers
% %%\newtheorem{theorem}{Theorem}[section]% meant for sectionwise numbers
% %% optional argument [theorem] produces theorem numbering sequence instead of independent numbers for Proposition
% \newtheorem{proposition}[theorem]{Proposition}% 
% %%\newtheorem{proposition}{Proposition}% to get separate numbers for theorem and proposition etc.

% \theoremstyle{thmstyletwo}%
% \newtheorem{example}{Example}%
% \newtheorem{remark}{Remark}%

% \theoremstyle{thmstylethree}%
% \newtheorem{definition}{Definition}%

\raggedbottom
%%\unnumbered% uncomment this for unnumbered level heads

\begin{document}

\title[Article Title]{Heterogeneous Vulnerability of Zero-Carbon Power Grids under Climate-Technological Changes}

%Identify the vulnerability of a carbon-free electric grid under climatic and technological changes
%%=============================================================%%
%% Prefix	-> \pfx{Dr}
%% GivenName	-> \fnm{Joergen W.}
%% Particle	-> \spfx{van der} -> surname prefix
%% FamilyName	-> \sur{Ploeg}
%% Suffix	-> \sfx{IV}
%% NatureName	-> \tanm{Poet Laureate} -> Title after name
%% Degrees	-> \dgr{MSc, PhD}
%% \author*[1,2]{\pfx{Dr} \fnm{Joergen W.} \spfx{van der} \sur{Ploeg} \sfx{IV} \tanm{Poet Laureate} 
%%                 \dgr{MSc, PhD}}\email{iauthor@gmail.com}
%%=============================================================%%

\author*[1]{\fnm{M. Vivienne } \sur{Liu}}\email{ml2589@cornell.edu}
\author[2]{\fnm{Vivek} \sur{Srikrishnan}}\email{vs498@cornell.edu}
% \equalcont{These authors contributed equally to this work.}
\author[2]{\fnm{Kenji} \sur{Doering}}\email{kmd266@cornell.edu}
\author[3]{\fnm{Elnaz} \sur{Kabir}}\email{ekabir@tamu.edu}
\author[2]{\fnm{Scott} \sur{Steinschneider}}\email{ss3378@cornell.edu}

\author[1,2]{\fnm{C. Lindsay} \sur{Andersion}}\email{cla28@cornell.edu}
% \equalcont{These authors contributed equally to this work.}

\affil*[1]{\orgdiv{Systems Engineering}, \orgname{Cornell University}, \orgaddress{\city{Ithaca}, \postcode{14853}, \state{NY}, \country{USA}}}

\affil[2]{\orgdiv{Department of Biological and Environmental Engineering}, \orgname{Cornell University}, \orgaddress{ \city{Ithaca}, \postcode{14853}, \state{NY}, \country{USA}}}

\affil[3]{\orgdiv{Department of Engineering Technology \& Industrial Distribution}, \orgname{Texas A\&M University}, \orgaddress{ \city{College Station}, \postcode{77843}, \state{TX}, \country{USA}}}

% \affil[3]{\orgdiv{Department}, \orgname{Organization}, \orgaddress{\street{Street}, \city{City}, \postcode{610101}, \state{State}, \country{Country}}}

%%==================================%%
%% sample for unstructured abstract %%
%%==================================%%

\abstract{The transition to decarbonized energy systems has become a priority globally to mitigate carbon emissions and, therefore, climate change. However, the vulnerabilities of zero-carbon power grids under climatic and technological changes have not been thoroughly examined. In this study, we focus on modeling the zero-carbon grid using a dataset that captures diverse future climatic-technological scenarios, with New York State as a case study. By accurately representing the topology and operational constraints of the power grid, we identify spatiotemporal heterogeneity in vulnerabilities arising from the interplay of renewable resource availability, high load, and severe transmission line congestion. Our findings reveal a need for 61-105\% more firm, zero-emission capacity to ensure system reliability. Merely increasing wind and solar capacity is ineffective in improving reliability due to transmission congestion and spatiotemporal variations in vulnerabilities. This underscores the importance of considering spatiotemporal dynamics and operational constraints when making decisions regarding additional investments in renewable resources.}


\keywords{Energy Systems Modeling, Power Grid Vulnerability, Spatial-Temporal Analysis, Zero-Emission Scenario}



\maketitle

\section{Introduction}\label{introduction} 

In light of the Intergovernmental Panel on Climate Change's recommendation~\cite{bashmakov2022climate} to achieve net-zero emissions by 2050, countries worldwide are formulating and implementing clean energy transition policies. Despite variations across countries~\cite{sun2020review}, these policies aim to electrify the transportation and heating sectors while simultaneously transitioning from fossil fuel-based electricity to a mix of renewable resources such as wind, solar, hydro, geothermal, and various forms of storage. 

Changes in climate and technology are likely to have profound impacts on both the supply and demand sides of energy systems~\cite{yalew2020impacts}. Unlike fully dispatchable fossil fuel generators, the output of variable renewable resources such as wind, hydro, and solar depend on precipitation, temperature, wind speed, and solar irradiation, rendering them more sensitive to climate variability and change~\cite{crook2011climate,schaeffer2012energy,yalew2020impacts}. Consequently, the availability of diverse renewable energy sources is subject to a range of stressors, from short-term intense fluctuations—typical of solar energy—to milder, long-term variations, as observed in large-scale hydroelectric systems

On the demand side, the electric load profile is heavily influenced by the same weather variables, as well as by human behaviors. Furthermore, the planned electrification of heating and transportation is expected to impact energy demand profiles, both in terms of magnitude~\cite{tarroja2018translating} and shape (e.g., from summer peak to winter peak~\cite{NYISOphaseI2019, NYtrends2022}). As such, there is a strong dependence between the supply and demand sides of energy systems, driven by latent weather variables~\cite{doering2018summer, ElnazK2023} and climatic-technological changes. Identifying and analyzing the vulnerabilities that can arise from these complex interactions, which have been largely overlooked in previous studies, are essential to ensure zero-carbon system reliability~\cite{perera2020quantifying, tarroja2018translating}.

Integrated Assessment Models (IAMs) such as EnergyPLAN~\cite{lund2021energyplan} and REMix~\cite{scholz2012renewable} have been used to model the interaction between electricity supply and demand and to evaluate the potential impacts of different strategies for energy decarbonization on global and regional scales~\cite{keppo2021exploring}. The common objective of these studies is to determine the least cost approach to reach emission reduction targets under alternative long-term climate scenarios~\cite{luderer2022impact, keppo2021exploring}. Although IAMs incorporate a wide range of models for various sectors, including energy supply, demand, transportation, land use, and agriculture, they often oversimplify the energy system by employing low spatial resolution models for renewable resource modeling~\cite{brazil_dranka2018planning, Portuguese_fernandes2014renewable, Macedonia_cosic2012100}. Furthermore, most analyses do not adequately represent the power grid topology and operational constraints~\cite{scholz2017application}. As a result, the proposed strategies may not reliably achieve their targets~\cite{liu2023spatiotemporal}. 

In this study, we investigate the reliability of power systems in the context of the transformation towards a low-carbon energy sector over an ensemble of 22-year scenarios. Our framework centers on the power system, integrating the impacts of both climatic and technological changes on the supply side, power transmission structure, and demand side. We use the New York State (NYS) power system to illustrate the potential vulnerabilities of a zero-carbon power system and discuss various technology options to improve the reliability of the system given the vulnerabilities identified.


\section{Main Text}
\subsection{NYS Power Grid and CLCPA}\label{NYSandCLCPA} 

In order to achieve the goals of the Climate Leadership and Community Protection Act (CLCPA), the NYS Climate Action Council developed a scoping plan that outlines the specific details of the plan to decarbonize NYS~\cite{scopingplan}. In this section, we introduce the NYS power grid and the CLCPA.

The NYS power grid has 11 load zones (Fig.~\ref{fig: NYoverview}) with spatially imbalanced supply and demand patterns, resulting in congested transmission lines that limit the transportation of energy from upstate (zones A-E) to downstate (zones F-K). NYS also relies heavily on hydropower, which is subject to seasonal and inter-annual fluctuations that are highly sensitive to climate change~\cite{cherry2005impacts}. To decarbonize the NYS grid, the CLCPA stands as one of the world's most ambitious climate-energy initiatives, targeting 70\% renewable energy generation by 2030, zero-emission electricity by 2040, and a minimum 85\% reduction in greenhouse gas emissions from 1990 levels by 2050. To reach these targets, onshore and offshore wind, utility solar, behind-the-meter (BTM) solar, transmission upgrades, new High Voltage Direct Current (HVDC) transmission lines, and electrification of multiple sectors are planned to be integrated into the current NYS grid in the next few decades~\cite{scopingplan} (see Supplementary Note 1 for more details about the NYS grid and the CLCPA). The NYS power grid and the ambitious CLCPA present an interesting case study for identifying vulnerabilities in future energy systems. 

The analysis in this study is based on scenarios proposed in~\cite{scopingplan}, which outline the detailed procedures and milestones defined in the CLCPA, coupled with an extensive dataset with high spatiotemporal resolution to characterize weather and load dynamics for 22 years.
While we primarily focus on NYS, the findings of this study provide valuable insights for stakeholders and policymakers on the general vulnerabilities of future carbon-free and renewable-dependent power grids under ambitious climate-energy policies, making them highly relevant and applicable to other regions.

% Figure environment removed

\subsection{Zero-carbon energy system modeling}\label{framework}
Our approach integrates the interactive effects of weather and climatic-technological factors on the supply, transmission, and demand components of the energy systems, as shown in Fig.~\ref{fig: Framework}. We employ a coherent set of input weather variables to all modules in the energy system including wind output, solar output, hydro output, dynamic transmission line rating, baseline load, and electrification of buildings and transportation (Fig.~\ref{fig: Framework}(a); see Methods for details). Specifically, we use 22-year reanalysis weather data from MERRA2~\cite{MERRA2} at 120 locations to capture spatiotemporal co-variability, seasonal patterns, and climatic variability between load and renewable outputs (Fig.~\ref{fig: Loadwindsolar}). 

% Figure environment removed

% Figure environment removed





Given the lack of consensus on the impact of climate change on wind speeds and solar radiation~\cite{yalew2020impacts}, we focus on the potential effects of temperature increase. Proposed technical mechanisms to mitigate climate change, such as the integration of renewable resources, the expansion of transmission lines, and the electrification of buildings and transportation, are represented based on information from the CLCPA Scoping Plan~\cite{scopingplan} and the Reliability Needs Assessment (RNA) report~\cite{Assessment2020} from the New York Independent System Operator (NYISO). To account for deep uncertainties, which refers to an inability to specify a consensus probability distribution for the uncertainty of interest, we use the Latin Hypercube Sampling (LHS)~\cite{loh1996latin} method to generate 300 combinations (Methods) of a wide range of alternative States of the World (SOWs)~\cite{walker2010addressing}. The SOWs include scenarios of varying renewable integration levels, electrification levels, and temperature increases. Each sampled combination of deeply uncertain parameters is used to adjust the reanalysis weather data to represent the warming impacts on the time series output of each module along with renewable integration and electrification levels in the power system. The result is 300-by-22 years of time-series data with a high spatial resolution derived for evaluation of power grid reliability as illustrated in Fig.~\ref{fig: Framework}(b).


We employ the Optimal Power Flow (OPF) framework to determine the optimal dispatch of resources, aiming to minimize power shortages, also known as load shedding (Fig.~\ref{fig: Framework}(c)). A linearized DC-OPF formulation is used to reduce computational complexity while ensuring that constraints on power flow and nodal-level power balance are maintained (Methods). By adopting a Multi-Criteria Decision Analysis (MCDA) approach to quantify the vulnerabilities and identify different failure mechanisms, we can explore potential problems to improve systems reliability~\cite{baumann2019review}. In~\cite{scopingplan}, it is emphasized that a fully decarbonized grid is contingent on a firm, zero-carbon resource, which refers to emission-free dispatchable resources (see Supplemental Note 1 for more details). Throughout this paper, we adopt the term ``firm, zero-emission capacity (FZEC)" to refer to the additional resource capacity required to uphold power grid reliability. In Table~\ref{tab: metricdef}, we define three evaluation metrics to 1) identify the potential vulnerabilities caused by different underlying failure mechanisms and 2) determine the sufficient FZEC needed to improve overall system reliability. 

\begin{table}[h]
\centering
\caption {\label{tab: metricdef} Definition of the evaluation metrics}

\begin{tabular}{| m{7em} | m{4.5cm}| m{4.5cm} | }

\hline
 Metric & Definition  & Implication to Reliability \\   \hline
Load Shedding Quantity & the total amount of load shedding during a specific duration & the total amount of additional energy required to maintain system reliability during a specific horizon\\
 \hline
Load Shedding Hours (Duration) & the number of hours with energy deficiency within a fixed duration & the duration of FZEC needed to maintain system reliability\\
 \hline
Maximum Load shedding (Intensity) & maximum instantaneous quantity of load shedding during a specific duration & the FZEC needed to maintain system reliability\\
\hline

\end{tabular}
\end{table}



\subsection{Spatiotemporal heterogeneity of the future system vulnerability
}\label{baselineresult}

To provide a baseline assessment of the reliability of the system, we perform a detailed evaluation that spans a 22-year simulation period, focusing on the inherent complexities of the power system. The initial analysis, shown in Fig.~\ref{fig: baseresult}, excludes the deep uncertainties that arise from climatic and technological changes to provide a reliable baseline. These results consider the changes in generation technology and load profile described in the CLCPA Scoping Plan, without considering the impacts of climatic-technological changes on system performance.

% Figure environment removed



The results in Fig.~\ref{fig: baseresult} demonstrate that the vulnerability of the future power grid exhibits spatiotemporal heterogeneity under the three evaluation metrics. Panels \textbf{(a)-(c)} highlight that the future system is more vulnerable in winter than in summer, while spring and fall exhibit much less load shedding overall. It is important to understand that the vulnerability patterns observed in winter and summer stem from different underlying mechanisms.

 
During winter, in load-centered zones J and K, load shedding intensity is the primary vulnerability, averaging 8549 MW and 6638 MW maximum load shedding over 22 years, which means that approximately 59\% and 89\% of load cannot be served during these severe energy deficits due to transmission congestion, respectively. Conversely, the generation-centered area (upstate zones A-E) exhibits a lower overall energy deficit. For example, zone D does not experience energy shortages throughout the 22-year simulation period, benefiting from its abundant and relatively stable hydropower supply. The load shedding primarily happens in zone C, due to relatively more congested interfaces (e.g., interface B-C) during cold days with wind droughts (wind drought refers to an extended period of very low wind speeds). This vulnerability can be attributed to two key factors:
\begin{itemize}
    \item The operational constraints on power flow pose frequent congestion (e.g., interfaces E-G and E-F), thereby impeding the transmission of needed energy from upstate to downstate. A different analysis that compares the vulnerability with and without key operational constraints reveals that the simplified representation of energy systems often used in these analyses is prone to underestimate system vulnerabilities. However, certain types of vulnerability can also be overestimated (see Supplementary Note 7 for detailed analysis).
    \item The co-variability between load and renewable outputs (e.g., heating demand caused by extremely cold days coincides with low renewable availability caused by wind droughts) makes the load-centered area more vulnerable and sensitive to weather conditions.
\end{itemize}

In general, under winter conditions, the amount of load shedding is the result of both load shedding intensity (at the load center) and duration (across the whole state), generally driven by high heating demand co-occurring with low renewable availability. The primary conditions are spatially different in intensity and duration (see Supplementary Note 9 for details and time-series illustrations).


During summer, the overall state-level energy deficit is much lower compared to winter, which primarily manifests in load centers. The load shedding is caused by the limited wind availability in summer (Fig.~\ref{fig: Loadwindsolar}), which leads to increased transmission line congestion (the dynamic rating modeling could exacerbate the congestion; see Supplementary Note 11 for details). The load shedding is particularly persistent after sunset when load centers heavily rely on stored energy and imports from upstate (see Supplementary Note 9 for detailed analysis). The total load shedding quantity is primarily influenced by high temperatures and wind droughts, with energy demand being particularly sensitive to extreme heat in densely populated areas. Additionally, hydro droughts can play a significant role, especially during the summer when hydropower contributes a larger portion of the supply due to reduced wind availability.


As described in Table~\ref{tab: metricdef}, the metrics measure vulnerability in a way that indicates the additional capacity needed to address power shortages, which in turn provides an estimate of the duration and capacity of the FZEC required. In addition, the spatial heterogeneity of vulnerability requires a spatially-disaggregated estimation of FZEC as load shedding is always accompanied by nearby upstream transmission line congestion, shown in all failure mechanisms. Congestion prevents the transfer of power from upstream zones to where it is demanded. As a result, although the maximum FZEC requirement identified without zonal analysis is 27 GW over the 22-year analysis period (Fig.~\ref{fig: baseresult}(b)), the actual need is likely as high as 37 GW when zonal requirements are included. Recall that in~\cite{scopingplan} 18-23 GW FZEC is estimated to ensure grid reliability. The more detailed study presented here indicates that the FZEC need is 61-105\% more than the scoping plan estimate. These findings underscore the importance of modeling energy systems with spatiotemporal co-variability and incorporating grid topology and operational constraints.



\subsection{Climatic and Technological Changes Exacerbate Vulnerabilities}

Building on the baseline analysis, the sensitivity of the findings can be analyzed under the impact of deeply uncertain climatic-technological factors (see Supplementary Note 5). These factors include temperature increase, capacity for renewable resources, and electrification rates for buildings and transportation. Using LHS, 300 combinations of these factors are analyzed to re-evaluate the 22-year analysis, enabling the understanding of system vulnerabilities and performance under a broad range of conditions.


To explore which factors are most significant for the vulnerability of the system, we first set a threshold based on information from~\cite{scopingplan} for each criterion to indicate violations as a binary variable. Then, we implement the Gradient Boosted Tree (GBT)~\cite{drucker1995boosting} method and use the importance scores of the inputs to rank the significance of the climatic-technological factors (see Methods), as shown in Fig.~\ref{fig: Factorranking}. It is obvious that temperature increase has crucial impacts on the energy system for both summer and winter. Therefore, as we count the number of violations for each evaluation metric over the 22-year horizon, we sort the scenarios in ascending order of temperature increase as shown in Fig.~\ref{fig: DU_result}. The sorted scenarios exhibit an increase in both violations of load shedding quantity and duration in summer while decreasing all types of vulnerability in winter. This shift highlights the changing dynamics of the system under climate change. Specifically, in winter, the overall vulnerability decreases as temperature increases, resulting from less extreme cold and reduced heating load, which is consistent with previous European studies~\cite{bloomfield2021quantifying}. It is worth noting that our analysis likely underestimates winter vulnerability due to the simplified representation of temperature increase as a step change across the year, and so we neglect increases in cold-weather extremes associated with climate change~\cite{wcd-3-1311-2022}.

% Figure environment removed



% Figure environment removed


In summer, the vulnerability increase is most notable in shedding event duration (Fig.~\ref{fig: DU_result}), while maximum load shedding is less significant. This results from the coinciding increase in summer cooling load with longer daylight hours and increased solar resources, resulting in smaller but more prolonged shortage events during daytime hours (see Supplementary Note 10 for a comparison between intensive temperature increase and no temperature increase). After sunset, cooling demand decreases, alleviating the need for additional generation. Consequently, maximum load shedding exceeding the threshold in summer is observed only in scenarios with temperature increases near $5^\circ$ C, combined with underbuilt wind and/or solar capacity. 

Finally, it is important to note that load shedding is almost always associated with nearby transmission line congestion that limits the effective use of available renewable resources. Therefore, increasing wind, solar, and battery capacity or decreasing electrification rates cannot fully compensate for the shortages caused by temperature increases. This supports our assertion that analyses of decarbonization strategies that neglect spatiotemporal co-variability and operational constraints are prone to underestimating system vulnerability. 




\section{Discussion and Conclusion} \label{conclusion}


This study shows that the spatiotemporal heterogeneity of the NYS power system leads to substantial vulnerabilities under the proposed decarbonization plan outlined in \cite{scopingplan}. Possible approaches to address these vulnerabilities include increased deployment of long-duration battery storage, and/or development of complementary flexible resources, such as green hydrogen. In general, total curtailed renewable energy in the state is sufficient to cover the total load shedding quantity (assuming a 75\% renewable conversion efficiency), except in 17 scenarios where the planned wind and solar capacities are installed at less than 80\% of planned. Consequently, seasonal storage solutions such as long-duration batteries and hydrogen exhibit great promise for addressing these challenges.

% Figure environment removed

While the topic of green hydrogen development is outside the scope of this paper, the concept involves producing hydrogen through electrolysis powered by the surplus renewable generation that cannot be directly consumed or stored. While the technology for hydrogen storage and transport is not yet fully mature, hydrogen offers the advantage of efficient use of curtailed renewable energy. This is particularly beneficial when renewable curtailment coincides with transmission line congestion, requiring excess renewable energy to be consumed locally. By locating the electrolyzer in areas with abundant renewable resources, hydrogen can be produced and then transported to high-demand regions that have limited \emph{local} renewable resources, without exacerbating transmission line congestion.  This offers a potential advantage over long-duration batteries, which are still subject to frequent transmission line congestion (Fig.~\ref{fig: hydrogenvsbatt}). In future work, a comprehensive analysis that compares hydrogen and long-duration batteries would be valuable to assess the cost, efficiency, and respective contributions to system reliability.

The interaction between climatic and technological impacts poses significant challenges for future carbon-free power systems. Exploring alternative scenarios and determining cost-effective pathways to decarbonize energy systems is crucial for combating climate change while ensuring reliability. While this study focuses on NYS, its ambitious climate policy and spatial heterogeneity allow us to identify broader insights applicable to other regions undergoing energy-system transitions.

Our findings show that identifying spatiotemporal heterogeneity in grid vulnerabilities requires consideration of grid topology, system constraints, and stressor co-variability over longer-term simulations. In the case study, vulnerability worsens during winter with heating and transportation electrification but shifts back to summer with a severe temperature increase. We also investigate the critical role of transmission congestion, coupled with various driving factors like extreme heat or wind drought, on load shedding intensity, duration, and renewable energy curtailment.

Our discussion provides valuable insights for policymakers and decision-makers, highlighting the importance of deploying FZEC of various technologies effectively to enhance system reliability. This information can aid in making informed decisions and shaping effective policies for a more resilient and reliable power grid in the face of climate-induced challenges.



\section{Methods} \label{methods}
\subsection{Modeling the zero-carbon energy system}
Like most zero-carbon energy systems, the envisioned NYS grid will rely heavily on newly installed wind and solar power, with existing hydro and nuclear sources, as the primary sources of energy. It is assumed that all fossil fuel generators will be retired by 2040. Our baseline zero-carbon configuration is based on the plan proposed by the Climate Action Council of NYS, which aims to achieve an 85\% reduction in economy-wide greenhouse gas emissions by 2050 compared to 1990 levels~\cite{CLCPA}. Detailed information regarding zonal capacity for wind, solar, and storage, as well as electrification assumptions, can be found in Supplementary Note 1.
 
To represent the zero-carbon configuration, we have adopted the grid topology and parameters from the baseline representation developed in~\cite{liu2022open} using 2019 data. These representations have been modified to reface the changes that are planned for the transition. The modeling details for each component of the energy system are provided below.

\subsubsection{Wind and Solar}
The capacity of wind and solar sites in the zero-carbon energy system is initially determined based on the information in~\cite{scopingplan}. Specifically, the system incorporates 11.6 GW of land-based wind, 14.7 GW of offshore wind, 13.6 GW of behind-the-meter (BTM) solar, and 51.2 GW of utility solar. The capacity allocations for different types of renewable resources are summarized in Table S2, and these zonal capacities are further distributed to bus-level (wind and utility solar on PV buses only; BTM solar on any type of buses) based on the potential wind and solar sites identified by the National Renewable Energy Laboratory (NREL). In this allocation, utility solar and wind generators are modeled as semi-dispatchable units, allowing them to generate power at any level below their forecasted limits. Conversely, BTM solar is modeled as a negative load to offset local energy needs.


To determine the bus-level renewable outputs for wind and solar resources, the Wind Integration National Dataset (WIND) Toolkit (WTK)~\cite{WTK} and Solar Integration National Dataset (SIND)~\cite{SIND} provided by NREL are employed. These toolkits enable the conversion of weather data spanning from 1998 to 2019 into unit power output values. As the WIND and SIND datasets only cover a limited time span, we use the reanalysis MERRA2~\cite{MERRA2} dataset to regenerate the wind and solar outputs for 22 years at different locations. By doing so, we preserve the spatial-temporal co-variability between the renewable outputs and other components in the zero-carbon grid that also take the MERRA2 dataset as inputs. 
 
The MERRA2 dataset offers a spatial resolution of 4 km × 4 km grids, and the nearest corresponding point is matched for each wind and solar site. To mitigate computational complexity, a subset of representative wind and solar sites is selected using the methodology outlined in~\cite{doering2022evaluating}. It is important to note that instead of computing exact power outputs, unit power output values are derived. This approach allows for flexibility in alternative capacity allocation during subsequent analyses.
 
For wind resources, the MERRA2 data is first bias corrected to the NREL WTK data, which is a more accurate dataset but has a limited time span. Then, the bias-corrected MERRA 2 data is used to generate the hourly unit power output as described in~\cite{doering2022evaluating}. On the other hand, the unit power output for solar resources is determined based on the bias-corrected temperature and solar radiation data from MERRA2. It is important to emphasize that other spatial and temporal modules, such as load, hydro, and dynamic transmission line ratings, are all modeled using the same MERRA2 data to maintain spatial-temporal correlation throughout the analysis.

\subsubsection{Hydro}
Given the abundance of hydropower resources in upstate NY, it is crucial to model the variability of hydropower availability for different times of the year under different climatic scenarios. In addition, water regulation practices mandate hydropower generation levels for each quarter month, particularly for larger hydropower plants, which depend on the availability of water resources. It is unrealistic to disregard available hydropower when there is excess wind and solar energy or to overuse non-existent hydropower when wind and solar have low outputs. As a result, distinct strategies are adopted for small and large hydro plants.
 
For the two major hydro plants in NYS, one located in zone A and the other in zone D, historical quarter monthly time series data for precipitation and average temperature are used as suggested by~\cite{semmendinger2022establishing}. These data are employed to predict the net basin supplies into the Great Lakes and ice conditions on the St. Lawrence River using an LSTM model~\cite{hochreiter1997long}. The predictions are then converted into quarter-monthly available hydropower. The power outputs serve as a basis for regulating the hydropower generation by enforcing the constraint that the aggregated quarter monthly dispatch from hydropower must match the available hydropower (see formulation in Supplementary Note 6). Implementing this constraint eliminates the curtailment of hydropower and prevents the unrealistic overuse of hydropower in real-world scenarios. 

On the other hand, smaller hydro plants rely heavily on the available river stream flow and are thus modeled as non-dispatchable generators, represented by the negative load. Average monthly capacity factors are calculated to account for the seasonal variation in stream flows, assuming a perfect correlation of power outputs among smaller hydro plants.

\subsubsection{Load}
The electricity demand profile in our analysis consists of three main components: baseline load, which represents the current load profile; the electrified load from residential and commercial buildings; and charging load from electric vehicles (EVs). To capture the relationship between the weather data and each load component, we leverage artificial neural networks (ANN). We present a comparison between the baseline load profile and the post-electrification load profile in Figure S11. This comparison illustrates the shift of peak demand from summer to winter due to widespread electrification.

\bmhead{Baseline load} To capture the intricate relationship between various factors influencing the load profiles, we leverage the zonal hourly load data from NYISO spanning from 2002 to 2019. Using this dataset, we train an artificial neural network (ANN) for each zone. The ANNs input temperature, load of the previous day, hour of the day, and day of the week, to generate the 24-hour load profile for a given day, as suggested by~\cite{satish2004effect, doering2022evaluating}. By incorporating the MERRA2 temperature data aggregated for each zone, the trained ANN models generate accurate hourly baseline zonal load profiles. These profiles are then allocated to individual buses to maintain the ratio of loads in the original model outlined in~\cite{liu2022open}.
 
\bmhead{Electrified building load}  We employ two advanced tools developed by NREL, namely ResStock~\cite{reyna2021us} and ComStock~\cite{parker2023comstock}, to accurately model energy consumption and potential energy savings associated with building upgrades in the residential and commercial sectors across the United States.
 
The ResStock tool enables us to simulate energy usage for five distinct types of residential buildings, including mobile homes, single-family attached and detached houses, and multi-family buildings with 2-4 or 5-plus units. ComStock simulates energy use for 14 different types of commercial buildings, including small office, medium office, large office, retail, strip mall, warehouse, primary school, secondary school, full-service restaurant, quick-service restaurant, small hotel, large hotel, hospital, and outpatient.
 
By leveraging these tools and the end-use load profile data~\cite{pigman2022end}, we estimate the energy savings potential achieved through different building upgrades. It is important to note that the energy savings are positive for fossil fuel usage, indicating a reduction, while they are negative for electricity, signifying increased electricity demand resulting from electrification. These state-level electricity savings, obtained for each building type, are available on an hourly basis for the Typical Meteorological Year, version 3 (TMY3)~\cite{wilcox2008users}.
 
To establish the connection between weather data and building load, we fit an ANN model for each building type. These models enable the prediction of electrified load based on input weather variables. By using the same MERRA2 data, our approach preserves the spatiotemporal co-variability between the electrified load and other weather-dependent modules, ensuring accurate predictions of the electrified load for different counties and years.
 
To properly scale the predicted electrified load, we consider the distribution of different building types within each county. Subsequently, the county-level loads are aggregated to the nearest bus in the power grid, ensuring alignment with the spatial representation of the power system (a comprehensive overview of our framework is provided in Supplementary Note 2). 
 
\bmhead{EV load} To accurately model the electric vehicle (EV) load, we employ the EVI-Pro Lite~\cite{wood2022evi} tool developed by NREL. The tool allows us to simulate the EV load profiles considering various factors that can influence their shape such as charging access, charging preferences, and the distribution of public and private charging stations.
 
In our study, we make certain assumptions regarding EV charging behavior because the transmission system is not sensitive to the assumptions of the EV charging load (primarily because the EV load is relatively small). We assume home charging is the preferred method, with Level 2 (L2) charging stations dominating the charging infrastructure. Additionally, since our primary focus is identifying vulnerabilities in the planned zero-carbon energy system, we assume minimal charging delays, namely no demand response in EV charging.



 
The simulation of EV load is conducted at the county level, taking into account the existing light-duty vehicles in each county. This allows us to capture the geographical distribution of EVs and their corresponding impact on the electricity grid. Further details regarding the selected parameters and a sample EV load profile can be found in Supplementary Note 3. 

\subsubsection{Battery}
As suggested in~\cite{scopingplan}, 19.8 GW of 8-hour duration lithium batteries is included in the model with a round trip efficiency of 85\%. The zonal capacity can be found in Table S1. The pumped hydro facility in zone E is modeled as a 12-hour battery with 1170 MW of capacity.
 
\subsubsection{Transmission Lines}
Transmission line constraints play a crucial role in the reliable operation of the power grid, particularly at the interfaces between load zones. In our study, we incorporate these constraints by including the transmission line upgrades outlined in the RNA report~\cite{Assessment2020}. In addition, we go beyond static ratings and introduce dynamic line ratings to account for the potential influence of climate change, particularly in relation to extreme heat events. This analysis also includes the new HVDC lines stated for construction in 2027.

\bmhead{Dynamic rating of transmission lines} The dynamic rating of transmission lines is modeled by the temperature-ampacity relationship described in~\cite{bartos2016impacts}. The ampacity is a function of ambient temperature, solar radiation, wind speed, and material parameters of the transmission cables. As the different cable models have a very small effect on the primary results, typical cable models identified in~\cite{bartos2016impacts} are used with MERRA2 data to derive the dynamic transmission line rating. Detailed model summary and parameter choice can be found in Supplementary Note 4.
 
\bmhead{New high voltage direct current transmission lines} The spatial imbalance between supply and demand in the NYS grid leads to significant congestion in the transmission interfaces connecting upstate and downstate regions, as discussed in~\cite{liu2023spatiotemporal}. To address this issue, NYS has contracted two new HVDC transmission lines: Clean Path New York and Champlain-Hudson. The Clean Path New York line, starting from the Fraser Substation in Delaware County, has a capacity of 1300 MW. The Champlain-Hudson line originates from Quebec and delivers hydropower with a capacity of 1250 MW. In our modeling, these HVDC lines are represented as sets of dummy generators to mimic the controllable power flow. They have the same output magnitude but opposite signs, enabling flexible control over the power flow on these lines as required. By simulating the operation and impact of these planned HVDC lines, we understand their impact on transmission capacity and facilitation of the smooth transfer of electricity across the grid.
 

\subsection{Alternative Scenarios Design}
The alternative SOWs design in our study aims to include a wide range of potential scenarios that acknowledge uncertainties arising from both climatic and technological factors.
 
To consider the impact of climate factors, we focus on temperature increase and design the scenarios as described in~\cite{semmendinger2022establishing}. To obtain projections for climate change, we use the CMIP6 climate model projection database~\cite{eyring2016overview}, which includes data from 46 distinct General Circulation Models (GCMs) and four different emission scenarios (SSP1-2.6, SSP2-4.5, SSP3-7.0, and SSP5-8.5). These projections are downscaled to the Great Lakes basin using the delta change method~\cite{maraun2016bias}. Monthly adjustments are made to historical precipitation data from 1952-2019, incorporating change factors derived from each GCM under each emission scenario. The projections reveal a warmer and wetter future, with temperature increases ranging from +1°C to +5.7°C  and precipitation changes between -3\% and +14\%  compared to historical averages. In total, 159 scenarios are developed. Note that not all the GCMs have the four emission scenarios and two GCMs do not provide future projections for the period of interest. These 159 scenarios represent different combinations of average temperature increase and corresponding changes to hydropower availability at a quarter-monthly time resolution.
 
Technological factors are represented by the electrification rate of buildings and electric vehicles (EVs) in NYS. We assume electrification rates range from 0.7 (representing a slower-than-expected rate) to 1.05 (indicating a slight over-electrification, potentially due to newly constructed buildings or vehicles). The chosen range is somewhat arbitrary, as the primary aim is to conduct a sensitivity analysis on the electrification rate's impact on overall system reliability. However, our findings reveal that the electrification rate is not a significant driving factor, as discussed in the main text. Additionally, we consider underbuilt/overbuilt scenarios for wind, solar, and storage units by scaling their capacities by factors ranging from 0.6 to 1.4.
 
Combining the climatic and technological factors leads to a six-dimensional sampling space. To generate a comprehensive set of combinations, we employ the Latin Hypercube sampling (LHS) technique, which ensures an even sampling from each dimension of the uncertainty space. This method outperforms random sampling approaches by avoiding clustered samples and reducing sampling variance~\cite{mckay2000comparison}. By generating 300 samples, our distributed sample set provides a better approximation of the underlying distribution within the high-dimensional sampling space than random sampling. Given that the goal of this study is to perform a sensitivity analysis to identify the failure mechanisms under different combinations of future uncertainties, 300 samples are considered sufficient to balance representation with computational requirements. 

\subsection{Enhanced DC-OPF Formulation}
The method employed in this study involves an enhanced formulation of the traditional DC-OPF (Optimal Power Flow) problem. The DC-OPF formulation is a linearized approximation that aims to optimize generator dispatch to meet demand at the lowest total cost. Since we aim to assess system reliability, we introduce a slack variable representing load shedding at each bus and minimize overall load shedding in the system.
 
In addition to the standard DC-OPF constraints, we incorporate additional constraints to account for HVDC lines, quarter-monthly hydro requirements, and battery state transitions. Furthermore, time-varying estimations for the maximum hourly outputs given the MERRA2 weather dataset are assigned to capture the upper limits of wind and solar generators, as well as upper and lower limits for transmission line capacities.
 
Given that the problem is solved at an hourly time-step for a full year length and this process is repeated for 22 years and 300 scenarios, we treat the renewable outputs calculated by historical weather data (with temperature adjustments for different climatic scenarios) as perfect forecasts to manage computation complexity. The full formulation of the problem can be found in Supplementary Note 6. We acknowledge that the uncertainties of renewable resources pose increased risks and vulnerabilities for the energy system. Future works focusing on designing accurate forecasting methodology and advanced control algorithms to manage the operation of the zero-carbon grid are extremely valuable but are out of the scope of this study.

\subsection{Gradient Boosting Tree and the Importance Score}
Gradient Boosting Trees combine multiple decision trees to make accurate predictions by iteratively building a series of weak learners, each of which corrects the errors of the previous model, resulting in a strong and highly predictive model capable of handling complex datasets. In this study, the GBT is used to map the climatic-technological factors (as features) to the violation indicators (as labels). The violation threshold is defined based on the information from as follows: the scoping plan identified 17722 MW of FZEC and claimed that 25GW of 100-hour long duration battery can ensure system reliability. We use 17722 MW as the threshold for Maximum Load Shedding, 100 hours as the threshold for load shedding hours, and 208 GWh as the threshold for load shedding quantity of a week, assuming 1/12th of the energy will be used for a quarter month from the 100-hour 25GW battery. 

The GBT method is able to capture the non-linear relationships in the feature space and uncover the most significant input factor~\cite{reed2022addressing}. The importance score quantifies the contribution of each feature in the model's decision-making process and is therefore calculated to help identify the most influential variables for the target variable. 



\bmhead{Acknowledgments}

This work was supported by the National Institute for Agriculture and Food (NIFA) and National Science Foundation (NSF) INFEWS Project No. NYC-121574 (grant no. 2019-67019-30122).


% \bmhead{Author contributions}

\bmhead{Competing interests}
The authors declare no competing interests.

% \bmhead{Additional information}
% Supplementary information is available for this paper at 

\bibliography{sn-bibliography}

\end{document}
