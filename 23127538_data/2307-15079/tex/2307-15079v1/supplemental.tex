\documentclass[9pt]{osa-supplemental-document}
\setboolean{shortarticle}{false}
\usepackage{gensymb}
\usepackage{float}
\usepackage{mdframed}
\usepackage{multicol}
\usepackage{multirow}
\title{Heterogeneous Vulnerability of Zero-Carbon Power Grids under Climate-Technological Changes: Supplementary Information}
\author{} %leave this blank
%% DO NOT ADD AUTHOR INFORMATION HERE; IT WILL BE ADDED DURING PRODUCTION

\begin{abstract}

\end{abstract}

\setboolean{displaycopyright}{false} %copyright statement should not display in the  supplemental document

\begin{document}

\maketitle



\section{Detailed NYS grid features and the CLCPA plan}

The upstate zones A-E had 92\% of zero-emission generation with a combination of hydro, nuclear, and wind in 2021, whereas downstate zones (F-K) had 8\% of zero-emission generation~\cite{nys2tail}. As the CLCPA mandates zero-emission by 2040, the Scoping Plan~\cite{scopingplan} outlined wind, solar, and storage unit allocation for different regions in NYS. Figure~\ref{NYScapacity} shows the planned capacity for different resources and the post-electrification load distribution. With the rich land availability in upstate and zone F, the majority of wind and solar are allocated to these regions. The darker color in zone J and K for the wind map are offshore winds. The majority of hydro is in zones A and D, contributing to approximately 80\% of the annual hydro generation. The zonal capacities for different resources are summarized in Table~\ref{tab:zonalcap2050}. This plan further exacerbates the unbalance and presses more pressure on the transmission interfaces between load zones. To account for the transmission expansion plan from NYISO, we use the interface limits from the Reliability Needs Assessment (RNA) report to update transmission limits from~\cite{Assessment2020}. The upper and lower limits can be found in Table~\ref{tab:interface2050}. Additionally, there are two High Voltage Direct Current (HVDC) lines in the works: the Champlain Hudson Power Express, with a capacity of 1250 MW, and the New York Clean Path, with a capacity of 1300 MW, scheduled for deployment in 2026 and 2027, respectively. Figures~\ref{CHPE} and~\ref{CPNY} depict the routes of these new HVDC lines.

% Figure environment removed


\begin{table}[H]
\centering
\caption {\label{tab:interface2050} Interface flow limits (MW)}
\begin{tabular}{lll}
\hline
\hline
 Interface & Lower Bound  & Upper Bound \\   \hline
 A-B   &  -2,200 & 2,200\\
 B-C   &  -1,600 & 1,500\\
 C-E   &  -5,650 & 5,650\\
 D-E   &  -1,600 & 2,650\\
 E-F   &  -3,925 & 3,925\\
 E-G   &  -1,600 & 2,300\\
 F-G   &  -5,400 & 5,400\\
 G-H   &  -7,375 & 7,375\\
 H-I   &  -8,450 & 8,450\\
 I-J   &  -4,350 & 4,350\\
 I-K   &  -515 & 1,293\\
 Total East   &  -3,400 & 5,600\\
 NY-NE   &  -1,700 & 1,300\\
 NY-IESO   &  -2,000 & 1,650\\
 NY-PJM   &  -900 & 500\\
\hline
\hline
\end{tabular}
\end{table}



\begin{table}[H]
\centering
\caption {\label{tab:zonalcap2050} Zonal capacity of renewable resources}
\resizebox{\columnwidth}{!}{
\begin{tabular}{llllllllllll}
\hline
\hline
Renewable Resource  & A & B & C & D & E & F & G & H & I & J & K\\   \hline
Land-Based Wind & 2692 &	390	&1923 &	1935 &	1821 & 1864	& 606 & 303	& 0	& 0	& 121\\
Offshore Wind & 0&	0&	0&	0	&0	&0&	0	&0	&0	&8250&	6488\\
BTM Solar  & 1297&	402&	1098	&127&	1240	&2154	&2270	&202	&299&	1676	&2883\\
Utility Solar  & 14440&	1648	&9006&	0	&5698	&15647&	3353	&0	&0&	0	&1441\\
Hydro  &2460	&63.8&	109.4&	909.8&	376.3&	269.6	&75.8&	0&0&0	&0\\
Battery  & 2479&	10	&2538&	2562&	892&	4727&	150	&140	&140&	4263	&1924\\
\hline
\hline
\end{tabular}}
\end{table}

% Figure environment removed

% Figure environment removed
\section{Building electrification module}


The data regarding electrification is sourced from the National Renewable Energy Laboratory (NREL), specifically the ResStock~\cite{reyna2021us} and ComStock~\cite{parker2023comstock} toolkits. These toolkits provide information on energy consumption and potential energy savings resulting from various upgrades for both residential and commercial buildings. The state-level data provides upgrades simulations regards to fossil fuel usage and electrified load. Additionally, the weather data used for building simulation corresponds to a typical meteorological year and is also provided.

Simulating building energy consumption requires significant computational resources. To overcome this challenge, we employ an emulator to project electricity savings (which are mostly negative, namely, electrified load) for two specific upgrades: "Whole-home electrification, high efficiency" (referred to as "upgrade 8") for residential buildings, and "DOAS HP Minisplits" (referred to as "upgrade 3") for commercial buildings. These upgrades primarily focus on electrifying the heating and cooling load, making them suitable for the focus of this study.

Firstly, we establish a function called $F_1(other fuel)$ using an Artificial Neural Network (ANN). This function maps the usage of other fuel types, such as natural gas and oil, to electricity savings at the state level. This choice is driven by the limitation that energy-saving data is only available at the aggregated state level. The assumption here is that the conversion from other energy sources to electricity remains consistent for each individual building type within the state. Consequently, county-level data on fossil fuel usage can be fed into the function to determine the electricity load for individual building types for each county. By leveraging the distribution of building types within each county, we can calculate the electrified load for every county.

Next, we model the relationship between the weather data of each county and the electrified load. This enables the joint modeling of the co-variability among the electrified load, baseline load, and renewable outputs using latent weather data. For each county, we employ another ANN approximation called $F_2(Weather data)$ to predict the total electrified load. Subsequently, the county-level load is aggregated to the power grid model's buses. The process outlined in Figure~\ref{fig: buildingeletrification} illustrates the framework for residential buildings, while for commercial buildings, the process remains the same, with the building types being replaced by the 14 commercial building types.

% Figure environment removed

\section{EV module}
The light-duty vehicle data is processed from the Vehicle, Snowmobile, and Boat Registrations~\cite{NYVehicle} dataset by filtering out the number of light-duty fleets for each county summarized in Table~\ref{tab: EV_county}. The median distance travel for each county is a required input for the EVI-Pro Lite model and is estimated based on the population density of 
each county~\cite{NYpopuden} using the population density to daily VMT from~\cite{wood2022evi}.The other parameters used for the EVI-Pro Lite model are summarized in Table~\ref{tab: EVIproparams}. A sample EV load profile for a week is shown in Fig~\ref{fig: evprofile}. 
\begin{table}[H]
\centering
\caption {\label{tab: EV_county} Light duty vehicle fleet size and median travel distance for each county }
\begin{tabular}{lll}
\hline
\hline
county &	fleet size	& median dvmt \\   \hline
Bronx&	26320&	14 \\ 
Kings&	60920	&14\\ 
New York&	25890	&14\\ 
Queens&	84524	&16\\ 
Richmond&	29686	&17\\ 
Albany&	15800&	22\\ 
Allegany	&1513&	31\\ 
Broome&	10253	&26\\ 
Cattaraugus&	2691	&31\\ 
Cayuga&	2818	&26\\ 
Chautauqua&	4679	&26\\ 
Chemung&	3998	&26\\ 
Chenango	&1847&	31\\ 
Clinton&	3632&	31\\ 
Columbia	&3298&	31\\ 
Cortland	&1597&	31\\ 
Delaware	&1904	&31\\ 
Dutchess	&15384&	26\\ 
Erie&	40570	&22\\ 
Essex	&1568	&31\\ 
Franklin&	1949	&31\\ 
Fulton&	2332&	26\\ 
Genesee&	2376	&26\\ 
Greene	&2107&	31\\ 
Hamilton	&276&	31\\ 
Herkimer	&2196	&31\\ 
Jefferson&	4378&	31\\ 
Lewis&	913	&31\\ 
Livingston	&2240	&26\\ 
Madison&	2691	&26\\ 
Monroe	&36962&	20\\ 
Montgomery&	2048	&26\\ 
Nassau&	96201&	17\\ 

\hline
\hline
\end{tabular}
\end{table}


\begin{table}[H]
\centering
\caption {\label{tab: DUparameters} Table \ref{tab: EV_county} continued }
\begin{tabular}{lll}
\hline
\hline
county &	fleet size	& mean dvmt \\   \hline
Niagara&	7823&	26\\ 
Oneida&	9393	&26\\ 
Onondaga	&21959&	22\\ 
Ontario	&5688&	26\\ 
Orange&	22821&	26\\ 
Orleans	&1138&	26\\ 
Oswego	&4344&	26\\ 
Otsego	&2421&	31\\ 
Putnam&	5739	&26\\ 
Rensselaer&	7317	&26\\ 
Rockland&	22377	&20\\ 
St Lawrence	&3544	&31\\ 
Saratoga&	14484&26\\ 
Schenectady	&9035&	22\\ 
Schoharie&	1235&	31\\ 
Schuyler	&845&	31\\ 
Seneca&	1305	&26\\ 
Steuben&	3783&	31\\ 
Suffolk	&94765	&20\\ 
Sullivan&	3885&	31\\ 
Tioga	&2234	&31\\ 
Tompkins	&3699&	26\\ 
Ulster&	8954	&26\\ 
Warren	&3476&	31\\ 
Washington	&2553	&31\\ 
Wayne&	3875	&26\\ 
Westchester	&57581&	18\\ 
Wyoming&	1407&	31\\ 
Yates	&922&	31\\ 


\hline
\hline
\end{tabular}
\end{table}

\begin{table}[H]
\centering
\caption {\label{tab: EVIproparams} EVI Lite Pro parameters }
\begin{tabular}{lll}
\hline
\hline
EVI pro params & Modeling choice \\ \hline
pev\_type & PHEV50 \\
pev\_dist	& EQUAL \\
class\_dist	& Sedan \\
home\_access\_dist	& HA100 \\
home\_power\_dist	& Equal \\
work\_power\_dist	& MostL2 \\
pref\_dist	& Home100 \\
res\_charging	& min\_delay \\
work\_charging & min\_delay \\

\hline
\hline
\end{tabular}
\end{table}


% Figure environment removed

\section{Dynamic rating formulation and result} \label{sec: dynamicrating}

The carrying capacity of electric power cables decreases as ambient air temperatures rise. Bartos et al., \cite{bartos2016impacts} estimate the impacts of rising air temperatures on electric transmission ampacity across the United States. They estimate the climate-attributable capacity reductions of transmission lines by constructing a thermal balance model, which estimates the rated ampacity of transmission lines based on cable properties and meteorological forcings. In the following, we explain this approach in detail. 

Assuming steady-state conditions and no conduction, the energy balance formula is:
\begin{align}
& q_j + q_s = q_c + q_r, \label{reducedEnergyBalance}
\end{align}
where $q_j$ is the resistive heating of the conductor ($W/m^{-1}$), $q_s$ is the radiative heat transfer from the sun to the conductor ($W/m^{-1}$), and $q_c$ and $q_r$, respectively, denote the convective and radiative heat losses from the conductor to surroundings ($W/m^{-1}$). Equation \ref{reducedEnergyBalance} implies that the total heat gain from the electrical current flowing
through the conductor and from the solar radiation striking the top half of the surface of the conductor equals the total heat loss due to convection and radiation. The heat gain due to electrical current ($q_j$) is also a function of the transferred current and the resistance of the conductor.
\begin{align}
& q_j = I^2 . R(T_{cond}), \label{electricalLoadHeatGain}
\end{align}
where $I$ is the current transferred through the conductor, and $R(T_{cond})$ is the resistance of the conductor at the given conductor temperature $T_{cond}$. Then, rearranging the heat balance formula \ref{reducedEnergyBalance} yields the maximum allowable current (the rated ampacity):
\begin{align}
& I = \sqrt{\frac{q_c+q_r-q_s}{R(T_{cond})}}, \label{ratedAmpacity}
\end{align}
In the following, we explain how to compute each term of the equation \ref{ratedAmpacity}. The convective heat loss per unit length, $q_c$, is calculated as:
\begin{align}
& q_c = \Bar{h}.\pi.D.(T_{cond}-T_{amb}), \label{convectiveHeatLoss}
\end{align}
where $D$ and $T_{amb}$, respectively, denote the conductor diameter (m) and the ambient air temperature (K). $\Bar{h}$ is the average heat transfer coefficient (W/m2-K) which is a function of wind speed, cable diameter, and some constants.

\begin{align}
& \Bar{h} = 0.3+\frac{0.62.(V.D/\nu)^{1/2}.Pr^{1/3}}{\left( 1+(\frac{0.4}{Pr})^{2/3}\right)^{1/4}}\left( 1+(\frac{V.D/\nu}{282000})^{5/8}\right)^{4/5}.k/D, \label{convectiveHeatLoss}
\end{align}

$V$ is the wind speed, $\nu$ is the dynamic viscosity of air ($m^2/s$), $Pr$ is the Prandtl number, and $k$ is the thermal conductivity of air (W/m-K). Table \ref{Table1} specifies the $(\nu,k, Pr)$ parameters under several air temperature scenarios. 

The radiation heat loss, $q_r$, and the solar heat gain, $q_s$, per unit length is computed as:
\begin{align}
& q_r = \epsilon.\sigma.\pi.D.(T_{cond}^4-T_{amb}^4), \label{radiationHeatLoss} \\
& q_s = \delta.D.a_s, \label{sunHeatGain}
\end{align}
where $\epsilon$ is the emissivity of the conductor surface, $\sigma$ is the Stefan–Boltzmann constant ($5.670e-8 W/m^2-K^4$), $\delta$ is the incident solar radiation, and $a_s$ is the solar absorptivity of the conductor surface. For a stranded aluminum conductor, it is assumed that $\epsilon = 0.7$ and $a_s = 0.9$. $\delta$ is the fixed value of $1000 W/m^2$ to represent full sun conditions.

Finally, in the expanded form, the rated ampacity of an overhead conductor can be expressed in terms of meteorological variables and cable properties:

\begin{align}
& I = \sqrt{\frac{\Bar{h}.\pi.D.(T_{cond}-T_{amb}) + \epsilon.\sigma.\pi.D.(T_{cond}^4-T_{amb}^4) - \delta.D.a_s}{R(T_{cond})}}.\label{ExpandedRatedAmpacity}
\end{align}

% What are the values of other parameters, including conductor diameter (D), conductor temperature ($T_{cond}$), wind speed (V), incident solar radiation ($\delta$), 

\begin{table}[H]
\centering
\caption{Air properties as a function of ambient air temperature.}
\label{Table1}
\begin{tabular}{|l|l|l|l|}
\hline
\begin{tabular}[c]{@{}l@{}}Temperature \\ (K)\end{tabular} & \begin{tabular}[c]{@{}l@{}}Dynamic Viscosity \\ (m\textasciicircum{}2/s)e-6\end{tabular} & \begin{tabular}[c]{@{}l@{}}Thermal Conductivity\\ (W/m-K)e-3\end{tabular} & Prandtl Number \\ \hline
200                                                        & 7.59                                                                                     & 18.1                                                                      & 0.737          \\
250                                                        & 11.44                                                                                    & 22.3                                                                      & 0.720          \\
300                                                        & 15.89                                                                                    & 26.3                                                                      & 0.707          \\
350                                                        & 20.92                                                                                    & 30.0                                                                      & 0.700          \\
400                                                        & 26.41                                                                                    & 33.8                                                                      & 0.690          \\
450                                                        & 32.39                                                                                    & 37.3                                                                      & 0.686          \\
500                                                        & 38.79                                                                                    & 40.7                                                                      & 0.684          \\
550                                                        & 45.57                                                                                    & 43.9                                                                      & 0.683          \\
600                                                        & 52.69                                                                                    & 46.9                                                                      & 0.689          \\ \hline
\end{tabular}
\end{table}


As per Equation~\ref{ExpandedRatedAmpacity}, the ampacity of transmission lines depends on several factors, including ambient temperature, solar radiation, wind speed, and specific parameters related to the cable models and conductor voltage classes. In \cite{bartos2016impacts}, it was found that the choice of cable models had minimal impact on the primary results. Therefore, we adopt the representative cable models proposed in this study, which are outlined in Table\ref{tab:cablemodel}. To determine the ambient temperature, solar radiation, and wind speed for each transmission line, we consider the spatial MERRA2 data and choose the three closest data points. The minimum dynamic rating among these three data points is selected to represent the worst-case scenario.

\begin{table}[H]
\centering
\caption{Characteristic cable models for conductor voltage classes}
\label{tab: cablemodel}
\begin{tabular}{|c|l|c|llc|}
\hline
\multirow{2}{*}{Nominal voltage (kV)} & \multirow{2}{*}{Model cable} & \multirow{2}{*}{Diameter (cm)} & \multicolumn{3}{l|}{AC Resistance (Ohms/km)}                    \\ \cline{4-6} 
                                      &                              &                                & \multicolumn{1}{l|}{25 C}  & \multicolumn{1}{l|}{50 C}  & 75 C  \\ \hline
500                                   & 3 x 954 kcm ACSR Cardinal    & 3.04                           & \multicolumn{1}{l|}{0.061} & \multicolumn{1}{l|}{0.067} & 0.073 \\
345                                   & 2 x 954 kcm ACSR Cardinal    & 3.04                           & \multicolumn{1}{l|}{0.061} & \multicolumn{1}{l|}{0.067} & 0.073 \\
230                                   & 1 x 1351 kcm ACSR Martin     & 3.62                           & \multicolumn{1}{l|}{0.044} & \multicolumn{1}{l|}{0.048} & 0.052 \\
115                                   & 1 x 795 kcm ACSR Condor      & 2.77                           & \multicolumn{1}{l|}{0.073} & \multicolumn{1}{l|}{0.080} & 0.087 \\
69                                    & 1 x 336 kcm ACSR Linnet      & 1.83                           & \multicolumn{1}{l|}{0.170} & \multicolumn{1}{l|}{0.186} & 0.203 \\ \hline
\end{tabular}
\end{table}

As the static rating for the interfaces of the NYS system is provided by NYISO, we assume that the static rating is set with the standard industry protocol with wind speed equal to $0.61m/s$, solar radiation equal to $1000 W/m^2$, conductor temperature equal to $75 /degree C$ and ambient temperature equal to $75 /degree C$. Then if we denote the interface flow calculated by the above condition for each line in the interfaces as $\Bar{L_{normal}}$, the interface limits given by the NYISO as $L_{nyiso}$ and the dynamic rating calculated for each hour based on Equation~\ref{ExpandedRatedAmpacity} as $L_s$, then the dynamic limits can be calculated by $\dfrac{L_s}{L_{nyiso}}L_{normal}$

\section{Deeply uncertain parameters}

The deeply uncertain parameters for the climatic and technological factors are summarized in Table~\ref{tab: DUparameters}

\begin{table}[H]
\centering
\caption {\label{tab: DUparameters} Climatic and technological factors }
\begin{tabular}{lll}
\hline
\hline
 Parameter & Lower Bound  & Upper Bound \\   \hline
Temperature increase & 0.95 & 5.64 \\
  Building electrification rate (zone A-E) & 0.7 & 1.05\\
  Building electrification rate (zone F-J) & 0.7 & 1.05\\
  Building electrification rate (zone J-K) & 0.7 & 1.05\\
  EV electrification rate (zone A-E) & 0.7 & 1.05\\
  EV electrification rate (zone F-J) & 0.7 & 1.05\\
  EV electrification rate (zone J-K) & 0.7 & 1.05\\
Wind capacity scaling factor & 0.6 & 1.4 \\
Solar capacity scaling factor & 0.6 & 1.4 \\
Battery capacity scaling factor & 0.6 & 1.4 \\
\hline
\hline
\end{tabular}
\end{table}




\section{Enhanced DCOPF formulation} \label{sec: dcopf}
\subsection*{Nomenclature}
\begin{mdframed}
\textbf{Sets and Indexes}
\item[$\mathcal{T}$] length of the planning horizon
\item[$\mathcal{Q}_q$] a set of time interval in a quarter month $q$
% \item[$H_n$] total number of generators in the system
\item[$\mathcal{B}$]  a set of buses in the system
\item[$\mathcal{L}$]  a set of transmission lines in the system
% \item[$L_n$]  total number of branches in the system
% \item[$S_n$]  total number of storage units in the system
% \item[$If_n$]  total number of interfaces in the system
\item[$\mathcal{C}$] a set of storage units
\item[$\mathcal{H}$] a set of large hydro generators
\item[$\mathcal{W}$] a set of wind generators
\item[$\mathcal{S}$] a set of solar generators
\item[$\mathcal{G}$] a set of nuclear generators
\item[$\mathcal{H}_{b}$]  a set of large hydro generators connected to bus $b$
\item[$\mathcal{W}_{b}$]  a set of wind generators connected to bus $b$
\item[$\mathcal{S}_{b}$]  a set of solar generators connected to bus $b$
\item[$\mathcal{G}_{b}$]  a set of nuclear generators connected to bus $b$
\item[$\mathcal{I}_b$]  a set of lines flow into bus $b$
\item[$\mathcal{O}_b$]  a set of lines flow out of bus $b$
\item[$\mathcal{C}_b$]  a set of storage units connected to bus $b$
\item[$\mathcal{IF}_i$]  a set of lines in zonal interface $i$
\item[$b \in \mathcal{B}$]  a bus in the system
\item[$t \in \mathcal{T}$]  a time interval
\item[$g \in \mathcal{H}\cup\mathcal{W}\cup\mathcal{S}\cup\mathcal{G}$]  a generator in the system
\item[$l \in \mathcal{L}$]  a transmission line in the system 
\item[$q \in \mathcal{Q}$]  a quarter month in the year \\


\textbf{Parameters} 
\item[$\overline{R}_g / \underline{R}_g$] upper/lower ramp rate limit of generator $g \in\mathcal{H}\cup\mathcal{G} $
\item[$\overline{P}_g / \underline{P}_g$]  generation upper/lower bound of generator $g \in\mathcal{H}\cup\mathcal{G} $
\item[$\overline{P}_{g,t} / \underline{P}_{g,t}$]  generation upper/lower bound of generator $g \in\mathcal{W}\cup\mathcal{S} $
\item[$\overline{L_{l,t}} / \underline{L_{l,t}}$] upper/lower bound of transmission line $l \in \mathcal{L}$ at time $t$
\item[$\overline{L_{IF_{i},t}} / \underline{L_{IF_{i},t}}$]  upper/lower bound of interface flow $i \in \mathcal{IF}$ at time $t$
\item[$D_{b,t}^{base}$]  baseline demand for bus $b \in \mathcal{B}$ in hour $t$ 
\item[$D_{b,t}^{bldg}$]  electrified building demand on bus $b\in \mathcal{B}$ in hour $t$ 
\item[$D_{b,t}^{ev}$]  EV charging demand on bus $b\in \mathcal{B}$ in hour $t$ 
\item[$D_{b,t}^{shydro}$] output of small hydro plants on bus $b\in \mathcal{B}$ at time $t$
\item[$D_{b,t}^{btm}$] output of behind the meter solar plants on bus $b\in \mathcal{B}$ at time $t$
\item[$\eta_{s}$]  round-trip efficiency of the storage unit at bus $s \in \mathcal{C}$ 
\item[$SOC_{s}$]  storage size for storage unit $s \in \mathcal{C}$ 
\item[$\Delta_{s}$]  charging/discharging capacity of storage unit $s \in \mathcal{C}$ 
\item[$H_{g,q}$] quarter monthly hydro power availability for hydro plant $g \in \mathcal{H}$ for quarter month $q$\\



\textbf{Variables}  
\item[$p_{g,t}$] generation of generator $g$ in hour $t$ 
\item[$e_{l,t}$] power flow of branch $l$ in hour $t$ 
\item[$\theta_{b,t}$] phase angle of bus $b$ in hour $t$ 
\item[$\delta_{s,t}^+, \delta_{s,t}^-$] charge/discharge power of storage unit $s$ in hour $t$ 
\item[$soc_{s,t}$] amount of stored energy in the storage unit $s$ at hour $t$ 
\item[$\mu_{b,t}$] the amount of load shedding on bus $b$ at hour $t$


\end{mdframed}


\begin{equation}\label{eq: obj}
\begin{aligned}
Min  \sum_{t=1}^{\mathcal{T}}(\sum_{b \in \mathcal{B}}(\mu_{b,t})+\lambda\sum_{s \in \mathcal{C}_b}(\delta_{s,t}^+ + \delta_{s,t}^-))
\end{aligned}
\end{equation}

\begin{equation} \label{eq: eq2}
    \begin{aligned}  
\sum_{g \in \mathcal{H}\cup\mathcal{W}\cup\mathcal{S}\cup\mathcal{G}} p_{g,t} + \sum_{l \in \mathcal{I}_b}e_{l,t} + \sum_{s \in \mathcal{C}_{b}}\delta_{s,t}^- =\sum_{l \in \mathcal{O}_b}e_{l,t} + D_{b,t}^{base}+ D_{b,t}^{bldg}+ D_{b,t}^{ev} - D_{b,t}^{btm} - D_{b,t}^{shdyro}\\ + \sum_{s \in \mathcal{C}_{b}}\delta_{s,t}^+ 
\quad \forall t \in \mathcal{T},  b \in \mathcal{B}
    \end{aligned}
\end{equation}


\begin{gather}
\vspace{-20pt}
\underline{P}_g \leq p_{g,t}\leq  \overline{P}_g \quad \forall t \in \mathcal{T} ,  \quad g \in \mathcal{H}\cup\mathcal{G} \label{eq: eq3}\\
\underline{P}_{g,t} \leq p_{g,t}\leq  \overline{P}_{g,t} \quad\forall t \in \mathcal{T} ,  \quad g \in \mathcal{W}\cup\mathcal{S} \label{eq: eq4}\\
\underline{R}_{g} \leq p_{g,t}  - p_{g,t-1}  \leq \overline{R}_g \quad \forall t \in \mathcal{T} , \quad g \in \mathcal{H}\cup\mathcal{G} \label{eq: eq5}\\
\underline{L} \leq e_{l,t} \leq \overline{L}  \quad \forall t \in \mathcal{T} , \quad l \in \mathcal{L} \label{eq: eq6} \\
 -\pi \leq \theta_{b,t} \leq \pi \quad \forall t \in \mathcal{T} ,\quad b \in \mathcal{B} \label{eq: eq7} \\
e_{l,t} = B_l (\theta_{b,t} - \theta_{b^\prime,t})  \quad \forall t \in \mathcal{T} , \quad l \in \mathcal{L}  \label{eq: eq8}\\
soc_{s,t+1} = soc_{s,t} + \dfrac{1}{\sqrt{\eta_s}}\delta_{s,t}^+ - \sqrt{\eta_s}\delta_{s,t}^- \quad \forall t \in \mathcal{T} , \quad s \in \mathcal{C} \label{eq: eq9}  \\
0 \leq soc_{s,t} \leq \overline{SOC_{s}}  \quad \forall t \in \mathcal{T} ,  \quad s \in \mathcal{C}  \label{eq: eq10}\\
0 \leq \delta_{s,t}^- \leq \Delta_{s}  \quad \forall t \in \mathcal{T} , \quad s \in \mathcal{C}  \label{eq: eq11}\\
0 \leq \delta_{s,t}^+ \leq  \Delta_{s}  \quad \forall t \in \mathcal{T}, \quad  s \in \mathcal{C}   \label{eq: eq12}\\
\underline{L_{IF_i,t}} \leq \sum_{l \in IF_i}e_{l,t} \leq \overline{L_{IF_i,t}}  \quad \forall t \in \mathcal{T} , \quad i \in \mathcal{IF}  \label{eq: eq13} \\
\sum_{t \in \mathcal{Q}_q}p_{g,t} = H_{g,q}  \quad  \forall q \in \mathcal{Q}, \quad g \in \mathcal{H} \label{eq: eq14}
\end{gather}

Equation~\ref{eq: obj} is the objective function for the enhanced DC-OPF problem. The first term represents load shedding for the whole system. The second term prevents the battery from charging and discharging simultaneously by assigning $\lambda = 0.01$, which is decided after the experiments to balance the optimality of the first term while limiting the simultaneous charging and discharging. Equation~\ref{eq: eq2} is the nodal load balancing constraint including the battery charging/discharging, the electrified load, and non-dispatchable resources modeled as negative load. Equation~\ref{eq: eq3} is the generation upper and lower limits constraint for dispatchable large hydro and nuclear generators. Equation~\ref{eq: eq4} is the generation upper and lower limits constraint for semi-dispatchable wind and solar generators. Noticing that the upper and lower bounds are modeled separately for dispatchable and semi-dispatchable generators. The upper limits for semi-dispatchable constraints are the predicted output calculated given the MERRA2 data. Equation~\ref{eq: eq5} is the ramping constraint for dispatchable generators. Equation~\ref{eq: eq6} is the transmission line limit constraint. Equation~\ref{eq: eq7} is the phase angle constraint. Equation~\ref{eq: eq8} describes the relationship between line flows with phase angle. Equation~\ref{eq: eq9}-\ref{eq: eq12} models the battery state transition, battery state of charge limits, and battery discharging and charging limits, respectively. Equation~\ref{eq: eq13} models the interface flow limits, noticing that the upper and lower limits are calculated by the method outlined in Section~\ref{sec: dynamicrating}. Finally, Equation~\ref{eq: eq14} enforces that the large hydro plants have to dispatch a certain amount of energy due to water regulations requirements.  



\section{Comparison for the impact of different modeling constraints}
In Section~\ref{sec: dcopf}, we introduced extended DC-OPF constraints that capture the complexities of real-world scenarios more accurately. In this section, we aim to examine the impact of transmission constraints, specifically phase angle constraints and transmission capacity limits. This analysis is crucial as it highlights the impacts of including system operation constraints in multi-year, multi-scenario studies, which is one of the key contributions of this research. To evaluate the role of these constraints, we compare two cases in terms of load shedding quantity, load shedding hours, and maximum load shedding, as depicted in Figures~\ref{fig: yearlylsq},\ref{fig: yearlylsf}, and\ref{fig: yearlylsm}, respectively:

\begin{itemize}
    \item Case 1: Standard DC-OPF formulation without transmission constraints. (Namely, the formulation in Section~\ref{sec: dcopf}  without ~\ref{eq: eq6},~\ref{eq: eq7},~\ref{eq: eq8}, ~\ref{eq: eq14}. constraint~\ref{eq: eq13} has a constant upper and lower bound.)
    \item Case 2:Standard DC-OPF formulation (with transmission constraints. (Namely, the formulation in Section~\ref{sec: dcopf}  without constraint~\ref{eq: eq14}. constraint~\ref{eq: eq13} has a constant upper and lower bound.)
    % \item Case 3: Standard DC-OPF with quarter monthly hydro constraint. (Namely, the formulation in Section~\ref{sec: dcopf}  with constraint~\ref{eq: eq13} has a constant upper and lower bound.)
    % \item Case 4: Standard DC-OPF with dynamic transmission line rating on interfaces. (Namely, the formulation in Section~\ref{sec: dcopf}  without  constraint~\ref{eq: eq14}. )
\end{itemize}



% Figure environment removed

% Figure environment removed

% Figure environment removed

The findings highlight the importance of considering system operation constraints. Neglecting these constraints can lead to a significant underestimation of system vulnerabilities, particularly when evaluating load shedding quantity and load shedding hours. This aligns with the expected outcome, as disregarding transmission line constraints and the power flow relationship with phase angle essentially eliminates the grid's topological structure, allowing power to be delivered anywhere in the grid.

However, the analysis of the maximum load shedding quantity in an hour reveals a mixture of scenarios. In the absence of transmission constraints, certain years exhibit higher maximum load shedding during a specific hour. Initially, this may seem counter-intuitive, but it can be attributed to the underlying assumption of "no transmission constraints" modeling. By assuming that power can be transmitted to any location at any time, the reliance on energy storage batteries diminishes, resulting in less energy being stored. Consequently, during periods of significant renewable shortage and negligible stored energy, the maximum load shedding amount can be substantial.

Figure~\ref{fig: wonet} illustrates five consecutive days, which includes the hour with the maximum observed load shedding in year 14 without transmission constraints. In contrast, Figure~\ref{fig: withnet} portrays the same five days, but with transmission constraints considered. As depicted in Figure~\ref{fig: wonet}, there is less battery charging during hours 35-40 and 80-86, resulting in a smaller reserved power for discharge during subsequent energy shortage events. Consequently, the load shedding quantity during hours 87-97 is significantly greater compared to the scenario with transmission constraints (Figure~\ref{fig: withnet}).



% Figure environment removed

% Figure environment removed

%\section{Decision tree and factor ranking}


% \section{Wind/Solar pattern for different seasons}


% The load, wind, and solar time-series data over the 22-year span are presented in Figure~\ref{fig: Loadwindsolar}. The red lines highlight the maximum values and the blue lines emphasize the minimum values. It is evident that over the 22-year horizon, the load would peak in winter due to the electrification of heating. Wind power availability is generally low in winter and solar availability is low in winter. Even though wind and solar compensate for each other seasonally~\cite{ElnazK2023}, daytime-only solar availability still causes energy deficiency in summer, especially during nighttime. 

% % Figure environment removed

\section{Load profile before and after electrification} \label{sec: electrificationcomparison}
The load profile before and after electrification is shown in Figure~\ref{fig: Loadeletrification}. Our results aligned with the projection in~\cite{NYISOphaseI2019} that by 2050, the peak demand shifts from summer to winter. 
% Figure environment removed



\section{Daily downstate curtailment in summer and winter, baseline analysis}
As explained in the main text, winter and summer exhibit distinct patterns. Figures~\ref{fig: allstatelssummer} and~\ref{fig: allstatelswinter} illustrate a typical summer week and a typical winter week with load shedding, respectively. During the summer week, there is significantly more solar availability in terms of intensity and duration. However, wind availability remains low throughout the week. Consequently, load shedding primarily occurs after sunset. It is worth noting that curtailment of wind and solar energy is observed during midday, indicating that transmission line congestion hampers the efficient utilization of renewable energy. A closer examination of the zonal load shedding pattern depicted in Figure~\ref{fig: zonallssummer} reveals that load shedding exclusively transpires in the downstate zones (G-K). This outcome aligns with expectations since downstate zones heavily depend on wind availability and thus face greater vulnerability during summer nights under prolonged wind droughts.

% Figure environment removed

% Figure environment removed

% Figure environment removed

% Figure environment removed
In contrast to summer, winter exhibits a different vulnerability pattern. With the electrification of the heating load (refer to Section~\ref{sec: electrificationcomparison}), the overall load profile substantially increases. Consequently, load shedding occurs when there is high demand coupled with wind drought, even during the middle of the day when solar power is available. Load shedding is not limited to downstate zones but also affects upstate zones due to lower temperatures during cold waves, as shown in Figure~\ref{fig: zonallswinter}.


\section{Summer load shedding comparison: baseline vs high temperature}

Figure~\ref{fig: S300summerweek} presents the same typical summer week as depicted in Figure~\ref{fig: zonallssummer} but with an extreme temperature rise of 5.64 \textdegree C. The overall demand profile experiences an increase due to the elevated temperatures. As a result, Load shedding occurs not only during nighttime but also throughout the daytime. In comparison to the significant hourly load shedding observed in the baseline winter scenario, the summer load shedding in the severe temperature increase scenario is characterized by lower magnitudes per hour. However, the load shedding is more consistent across all hours throughout the wind drought week.

% Figure environment removed



% Bibliography
\bibliography{sample}

%Manual citation list
%\begin{thebibliography}{1}
%\bibitem{Zhang:14}
%Y.~Zhang, S.~Qiao, L.~Sun, Q.~W. Shi, W.~Huang, %L.~Li, and Z.~Yang,
 % \enquote{Photoinduced active terahertz metamaterials with nanostructured
  %vanadium dioxide film deposited by sol-gel method,} Opt. Express \textbf{22},
  %11070--11078 (2014).
%\end{thebibliography}

\end{document}