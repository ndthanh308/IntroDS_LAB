\documentclass[11pt]{article}
\usepackage{amsmath,amsfonts,amssymb,amsthm}
\usepackage{mathtools}
\usepackage{dsfont}
\usepackage{enumerate}
\usepackage{esint}
\allowdisplaybreaks[4]

\newtheorem{Theorem}{Theorem}[section]
\newtheorem{Proposition}[Theorem]{Proposition}
\newtheorem{Lemma}[Theorem]{Lemma}
\newtheorem{Corollary}[Theorem]{Corollary}
\newtheorem{Definition}{Definition}[section]
\newtheorem{Remark}{Remark}[section]

\usepackage[colorlinks=true,linkcolor=blue,citecolor=green]{hyperref}

\usepackage[margin=2cm]{geometry}
%\usepackage{natbib}
%\setlength{\bibsep}{0pt plus 0.3ex}
\newlength{\bibitemsep}\setlength{\bibitemsep}{.0\baselineskip plus .05\baselineskip minus .05\baselineskip}
\newlength{\bibparskip}\setlength{\bibparskip}{0pt}
\let\oldthebibliography\thebibliography
\renewcommand\thebibliography[1]{%
  \oldthebibliography{#1}%
  \setlength{\parskip}{\bibitemsep}%
  \setlength{\itemsep}{\bibparskip}%
}

%\usepackage{setspace}
%\textwidth 160mm
%\textheight 230mm
%\hoffset -0.8in
%\parindent 5mm
%\voffset -1.5cm
%\renewcommand\baselinestretch{1.3}
%\renewcommand{\theequation}{\arabic{section}.\arabic{equation}}
%\renewcommand{\thefootnote}{\fnsymbol{footnote}}
%\makeatletter
%\newcommand{\rmnum}[1]{\romannumeral #1}
%\newcommand{\Rmnum}[1]{\expandafter\@slowromancap\romannumeral #1@}
%\makeatother

\numberwithin{equation}{section}

\begin{document}

\title{\Large\bf The gyrokinetic limit for the Plasma-Charge model in $\mathbb{R}^2$}
\author{Jingpeng Wu\thanks{School of Mathematics and Statistics, Huazhong University of Science and Technology; Institute of Artificial Intelligence, Huazhong University of Science and Technology, Wuhan, 430074, China (jpwu\_postdoc@hust.edu.cn).}}
\date{}
\maketitle

\begin{abstract}
In this article, we investigate the gyrokinetic limit for the two-dimensional Vlasov-Poisson system with point charges. We show that the solution converges to a measure-valued solution of the Euler equation with a defect measure, which extends the results in \cite{Mio19} to the case of multi-point charges and removes the small condition $\sup_{0<\varepsilon<1}\|f_{\varepsilon}^0\|_{L^1}<1$.

\noindent{\bf Keywords:} Vlasov equations; large data; gyrokinetic limit; Euler equation; measure-valued solution; Dirac mass

\noindent{\bf MR Subject Classification:} 35Q83; 82D10; 35B40; 35A24.
\end{abstract}
% MSC2020
% 35Q83  	Vlasov equations
% 78A35  	Motion of charged particles
% 82D10  	Statistical mechanics of plasmas
% 35B40  	Asymptotic behavior of solutions to PDEs
% 35A24  	Methods of ordinary differential equations applied to PDEs


\section{Introduction and main results}

\subsection{Introduction}
In this paper, we investigate the gyrokinetic limit for the  Vlasov-Poisson system with $N$ point charges, so called the Plasma-Charge model, which is described by the asymptotic behaviour of solutions to the following equations as $\varepsilon$ tends to zero:
\begin{equation}\label{eq-VP-point}\tag{$\mathbf{E}^{\varepsilon}$}
\left\{
\begin{split}
&\partial_tf_{\varepsilon}+\frac{v}{\varepsilon}\cdot\nabla_xf_{\varepsilon}+\left(\frac{v^{\perp}}{\varepsilon^2}+\frac{E_{\varepsilon}+\gamma F_{\varepsilon}}{\varepsilon}\right)\cdot\nabla_vf_{\varepsilon}=0,\\
&E_{\varepsilon}(t,x)=\int_{\mathbb{R}^2}\frac{x-y}{\vert x-y\vert^2}\rho_{\varepsilon}(t,y)\,\mathrm{d}y,\quad\rho_{\varepsilon}=\int_{\mathbb{R}^2}f_{\varepsilon}\,\mathrm{d}v,\\
&\dot{\xi}_{\alpha,\varepsilon}(t)=\frac{\eta_{\alpha,\varepsilon}(t)}{\varepsilon},\quad\dot{\eta}_{\alpha,\varepsilon}(t)=\gamma\left(\frac{\eta_{\alpha,\varepsilon}^{\perp}(t)}{\varepsilon^2}+\frac{E_{\varepsilon}(t,\xi_{\alpha,\varepsilon})}{\varepsilon}\right)+\frac{1}{\varepsilon}F_{\varepsilon}(t,\xi_{\alpha,\varepsilon}(t)),\\
&F_{\varepsilon}(t,x)=\sum_{\alpha}\frac{x-\xi_{\alpha,\varepsilon}(t)}{\vert x-\xi_{\alpha,\varepsilon}(t)\vert^2}\text{ if }x\notin\{\xi_{\alpha,\varepsilon}(t)\},\quad F_{\varepsilon}(t,\xi_{\alpha,\varepsilon}(t))=\sum_{\beta:\beta\ne\alpha}\frac{\xi_{\alpha,\varepsilon}(t)-\xi_{\beta,\varepsilon}(t)}{\vert\xi_{\alpha,\varepsilon}(t)-\xi_{\beta,\varepsilon}(t)\vert^2},\\
&(f_{\varepsilon},\xi_{\alpha,\varepsilon},\eta_{\alpha,\varepsilon})\vert_{t=0}=(f_{\varepsilon}^{0},\xi_{\alpha,\varepsilon}^0,\eta_{\alpha,\varepsilon}^0),\quad\alpha=1,2,\dots,N.
\end{split}\right.
\end{equation}
Here $f_{\varepsilon}=f_{\varepsilon}(t,x,v)$ is the  phase space density of the plasma particles at time $t\ge 0$, located at $x\in\mathbb{R}^2$  with  velocity  $v\in\mathbb{R}^2$. The $\alpha$-th point charge is located at  $\xi_{\alpha,\varepsilon}(t)$ with velocity $\eta_{\alpha,\varepsilon}(t)$. All the particles are submitted to the self-consistent electric field $E_{\varepsilon}(t,x)$, to the singular field $F_{\varepsilon}(t,x)$ induced by the point charges and to a large external  magnetic field, orthogonal to the plane, which is represented by the terms  $v^{\perp}/\varepsilon^2$ or  $\eta_{\alpha,\varepsilon}^{\perp}/\varepsilon^2$, here $(v_{1},v_{2})^{\perp}=(-v_{2},v_{1})$. The real number $\gamma\gtrless 0$ corresponds to a repulsive or attractive interaction between the plasma and point charges. In this paper, we consider the repulsive case and set $\gamma=1$ for convenience.

The system \eqref{eq-VP-point} is related to the following measure-valued Euler equation with a defect measure $\nu \in L^{\infty}(\mathbb{R}_+,\mathcal{M}(\mathbb{R}^2,\mathbb{R}^{2\times 2}))$:
\begin{equation}\label{eq-mv-E-defect}\tag{$\mathbf{E}_{\nu}^0$}
\left\{
\begin{split}
&\partial_t(\rho+\bar{\delta})+E_{\rho+\bar{\delta}}^{\perp}\cdot\nabla(\rho+\bar{\delta})=\nabla^2:\nu,\\
&E_{\rho+\bar{\delta}}=\frac{x}{\vert x\vert^2}*(\rho+\bar{\delta}),\quad\bar{\delta}(t,x)=\sum_{\alpha}\delta_{\xi_{\alpha}(t)}(x),
\end{split}\right.
\end{equation}
and the vortex-wave system:
\begin{equation}\label{eq-Vortex-Wave}\tag{$\mathbf{E}_0^0$}
\left\{
\begin{split}
&\partial_t\rho+(E_{\rho}+F)^{\perp}\cdot\nabla\rho=0,\quad E_{\rho}=\frac{x}{\vert x\vert^2}*\rho,\\
&\dot{\xi}_{\alpha}(t)=\big(E_{\rho}(t,\xi_{\alpha}(t))+F(t,\xi_{\alpha}(t))\big)^{\perp},\\
&F(t,x)=\sum_{\alpha}\frac{x-\xi_{\alpha}(t)}{\vert x-\xi_{\alpha}(t)\vert^2}\text{ if }x\notin\{\xi_{\alpha}(t)\},\quad F(t,\xi_{\alpha}(t))=\sum_{\beta:\beta\ne\alpha}\frac{\xi_{\alpha}(t)-\xi_{\beta}(t)}{\vert\xi_{\alpha}(t)-\xi_{\beta}(t)\vert^2}.
\end{split}\right.
\end{equation}
Without the point charges, the vortex-wave system reduces to the vorticity formulation of the $2D$ imcompressible Euler equation.

The classical Vlasov-Poisson system, \eqref{eq-VP-point} without charge nor magnetic field, has been widely studied, see e.g., \cite{Ars73} for weak solutions, \cite{UO78} for classical solutions in two dimensions, \cite{Pfa92,LP91} for classical solutions in three dimensions, \cite{Loe06,Mio16CMP} for the uniqueness of weak solutions. 

The Plasma-Charge model, \eqref{eq-VP-point} without magnetic field was firstly introduced by Caprino and Marchioro \cite{CM10}, in which the existence and uniqueness of the classical solution have been established  for the repulsive case $\gamma>0$ and for initial data being vacuum around point charges and satisfying some integrable conditions as in Theorem~\ref{thm-main} below. For three dimensional case, the well known Lagrangian method of \cite{Pfa92} and the Eulerian method of \cite{LP91} were developed in \cite{MMP11} and \cite{DMS15} respectively. The analysis in \cite{CM10,MMP11} relies on the bound of a pointwise energy function defined as $\bar{h}=\frac{1}{2}\vert v-\eta_{\alpha}\vert^2+\gamma G_d(x-\xi_{\alpha})$ with $G_d$ the Green function of $-\Delta$ in $\mathbb{R}^d$, which enables to handle the strictly positive distance between the plasma and the point charges, and the singular field $F_{\varepsilon}$ is always regular. Such mechanism is not applicable to the attractive case $\gamma<0$, which makes it more difficult to study. Nevertheless, the existence of classical solutions in $\mathbb{R}^2$ for $\gamma<0$ was proved in \cite{CMMP12}, based on some special properties of the logarithmic potential $G_2(x)=-\frac{1}{2\pi}\ln\vert x\vert$, and the uniqueness remains unsolved.

When adding an external magnetic field, the existence and uniqueness of classical solutions  to \eqref{eq-VP-point} with or without the point charges for $\gamma>0$, might be easily to prove by adapting the methods in \cite{UO78,Pfa92,CM10,MMP11} for two or three dimensions. A more interesting topic for experts in this area is the asymptotic behaviour of solutions to \eqref{eq-VP-point} as $\varepsilon$ tends to zero, which has been found to be a mechanism of deriving fluid equations \eqref{eq-mv-E-defect}, \eqref{eq-Vortex-Wave} from kinetic equations \eqref{eq-VP-point}, always called the gyrokinetic limit or the guiding center approximation in plasma physics. The case without point charge has been studied by Golse and Saint-Raymond \cite{GS99} in the periodic setting, which established the convergence of \eqref{eq-VP-point} to the incompressible Euler equation with a defect measure \eqref{eq-mv-E-defect}, see also Brenier's work \cite{Bre00} in a different scaling. Later Saint-Raymond \cite{Sai02} proved that the defect measure is not involved in the Euler equation and moreover it is equal to zero for sufficiently regular initial data. The same kind of convergence results of \cite{GS99,Sai02} are obtained in \cite{Mio16} by different techniques. Besides the already mentioned articles, there is a wide literature devoted to various asymptotical regimes for linear or nonlinear Vlasov-like equations with strong external magnetic field, leading to different nonlinear equations (see e.g., \cite{FS98,FS00,Sai00,FS01,GS03,BOS09,FR16}).

Recently Miot \cite{Mio19} investigated the gyrokinetic limit of \eqref{eq-VP-point} with a point charge for $\gamma>0$ and proved that the solution converges to a measure-valued  solution of the Euler equation with a defect measure \eqref{eq-mv-E-defect}. Furthermore, the limiting equation exactly yields the vortex-wave system \eqref{eq-Vortex-Wave} if the defect measure vanishes and under more regularity assumptions on $\rho, \xi$. The vortex-wave system \eqref{eq-Vortex-Wave} was introduced by Marchioro and Pulvirenti \cite{MP91}, then further analyzed in \cite{LM09,CLMN16}.

The aim of this article is to extend the results in \cite{Mio19} to the system \eqref{eq-VP-point} with $N>1$. Such extension seems non-trivial. Indeed, there are many works that have been only proven for the case with a single point charge, see e.g., \cite{DMS15,XZ21,AW21,PW21,PWY22,HW22}. To the best of the author's knowledge, it is unknown whether these results can be extended to the case with multi-point charges, and the difficulties posed by multi-point charges are varied. In the fluid theory, the generalization of the study on the fluid models coupled with single object to multi objects also seems to present complex difficulties, see e.g., for the vortex-wave system, the extension of \cite{LM09} to \cite{LM21}, and for the fluid-rigid body system, \cite{GLS14,GLS16,GMS18} to \cite{GS19}.

Very recently, the author with a collaborator \cite{WZ23} extended the moment propagation of \cite{DMS15} to \eqref{eq-VP-point} with $N>1$ in three dimensions. The main idea in \cite{WZ23} is to find a suitable pointwise energy functional, in whose derivative the principal singular terms cancel, and analogous thought has already manifested in \cite{LM21}. The approach in \cite{WZ23} can be applied to estimate the time integral of the electric field along the trajectories of the point charges, see Sec.~\ref{sec-est-point-singular-field}. A more crucial issue for the gyrokinetic limit lies in achieving continuity or almost continuity of the the trajectories of the point charges. It will be done by an iterative process with a non-uniform time period as the step size to prove the continuity on small but uniform time interval, see Sec.~\ref{sec-time-regularity}.

%This is the second work in the author's series of works on the study of the interaction between the plasma system and multi-point charges, following \cite{WZ23}. 

\subsection{Notations and Definitions}

\begin{itemize}
\item When without specifying, C always denotes a positive constant depending only on the constants $N,T,K_0,K_1$ given in Theorem~\ref{thm-main}.
\item For $A,B\in\mathbb{R}^{m\times n}$, $A:B=\sum_{1\le i\le m,\,1\le j\le n} A_{ij}B_{ij}$.
\item $\delta_{z}$ denotes the Dirac measure on $\mathbb{R}^2$, concentrated at $\{z\}$. 
\item A space of functions from domain $X$ to range $Y$ denotes as the form $L(X,Y)$, or $L(X)$ if $Y=\mathbb{R}$, or $L_+(X)$ if $Y=\mathbb{R}_+$. The norms of Lebesgue spaces $L^p(\Omega)$ denote $\|\cdot\|_{L^p(\Omega)}$ or $\|\cdot\|_{p}$ when $\Omega$ is clear from the context. $\mathcal{M}$ denotes the space of bounded real Radon measures. $C_0$ denotes the completion of $C_c$ with respect to $\|\cdot\|_{\infty}$. We always equip the vague topology i.e.~weak-* topology $\sigma(\mathcal{M},C_0)$ on $\mathcal{M}$. 
\item When the index set $\mathcal{I}$ and the index $i\in\mathcal{I}$ are clear, we denote $\{a_i\}$ as the set $\{a_i\colon i\in\mathcal{I}\}$.
\item We define the following summation signs for short:
\begin{equation*}
\sum_{\alpha}=\sum_{1\le\alpha\le N},\quad\sum_{\beta\colon\beta\ne\alpha}=\sum_{1\le\beta\le N,\beta\ne\alpha},\quad\sum_{\alpha\ne\beta}=\sum_{1\le\alpha\le N}\sum_{\beta\colon\beta\ne\alpha}.
\end{equation*}
\end{itemize}

\begin{Definition}[Energy and moment of \eqref{eq-VP-point}]
The energy associated to \eqref{eq-VP-point} is defined as
\begin{align*}
\mathcal{E}(f,\{\xi_{\alpha},\eta_{\alpha}\})=&\frac{1}{2}\iint_{\mathbb{R}^2\times\mathbb{R}^2}\vert v\vert^2f\,\mathrm{d}x\,\mathrm{d}v+\frac{1}{2}\sum_{\alpha}\vert\eta_{\alpha}\vert^2-\frac{1}{2}\iint_{\mathbb{R}^2\times\mathbb{R}^2}\ln\vert x-y\vert\rho(x)\rho(y)\,\mathrm{d}x\,\mathrm{d}y\\
&-\sum_{\alpha}\int_{\mathbb{R}^2}\ln\vert x-\xi_{\alpha}\vert\rho(x)\,\mathrm{d}x-\frac{1}{2}\sum_{\alpha\ne\beta}\ln\vert \xi_{\alpha}-\xi_{\beta}\vert
\end{align*}
and the momentum is defined as
\begin{equation*}
\mathcal{I}(f,\{\xi_{\alpha},\eta_{\alpha}\})=\int_{\mathbb{R}^2}(\vert x+\varepsilon v^{\perp}\vert^2-\varepsilon^2\vert v\vert^2)f\,\mathrm{d}x\,\mathrm{d}v+\sum_{\alpha}(\vert \xi_{\alpha}+\varepsilon\eta_{\alpha}^{\perp}\vert^2-\varepsilon^2\vert\eta_{\alpha}\vert^2).
\end{equation*}
\end{Definition}


Since the solutions to \eqref{eq-mv-E-defect} are measure-valued, we adopt the formulation introduced in \cite{Del91,Sch95,Pou02,Mio19} to clarify the nonlinear term $E_{\rho+\bar{\delta}}^{\perp}\cdot\nabla(\rho+\bar{\delta})=\nabla\cdot \big(E_{\rho+\bar{\delta}}^{\perp}(\rho+\bar{\delta})\big)$.
\begin{Definition}[]
Let $\rho,\mu\in \mathcal{M}_{+}(\mathbb{R}^{2})$. For all $\Phi\in C_{c}^{\infty}(\mathbb{R}^{2})$, the symmetric quadratic form is defined  by
\begin{equation*}
\mathcal{H}_{\Phi}[\rho,\mu]=\frac{1}{2}\iint_{\mathbb{R}^4}H_{\Phi}(x,y)\,\rho(\mathrm{d}x)\,\mu(\mathrm{d}y),
\end{equation*}
where
\begin{equation*}
H_{\Phi}(x,y)=\frac{(x-y)^{\perp}}{\vert x-y\vert^2}\cdot\big(\nabla\Phi(x)-\nabla\Phi(y)\big)\,\text{ if } x\ne y,\quad H_{\Phi}(x,x)=0.
\end{equation*}

The notation $\nabla\cdot(E_{\rho}^{\perp}\rho)$ then is a distribution defined by
\begin{equation}\label{eq-symmetric-form}
\langle\nabla\cdot(E_{\rho}^{\perp}\rho),\Phi\rangle_{\mathcal{D}'(\mathbb{R}^2),\mathcal{D}(\mathbb{R}^2)}:=-\mathcal{H}_{\Phi}[\rho,\rho].
\end{equation}
\end{Definition}

It is obvious that $H_{\Phi}$ is  well defined and bounded on $\mathbb{R}^{2}\times \mathbb{R}^{2}$, vanishes at infinity, and is continuous outside the diagonal $\{(x,x)\vert x\in\mathbb{R}^{2}\}$. If $\rho\in L^{p}(\mathbb{R}^2)$ for $p>2$ such that $E_{\rho}\rho$ is locally integrable, it is obvious that the distribution $\nabla\cdot(E_{\rho}^{\perp}\rho)$ defined above is equivalent to the distributional gradient of $E_{\rho}^{\perp}\rho$.

\begin{Definition}[Weak solutions to \eqref{eq-mv-E-defect}]
Let $\rho,\bar{\delta}\in L^{\infty}(\mathbb{R}_+,\mathcal{M}_{+}(\mathbb{R}^{2}))$, $\nu\in L^{\infty}(\mathbb{R}_+,\mathcal{M}(\mathbb{R}^2,\mathbb{R}^{2\times 2}))$. $(\rho,\bar{\delta})$ is said to be a weak solution to \eqref{eq-mv-E-defect} with a defect measure $\nu$ if: for all $\Phi\in C_c^{\infty}(\mathbb{R}_+\times\mathbb{R}^2)$
\begin{equation}\label{eq-weak-solution-mv-Euler-defect}
\left\{
\begin{split}
&\int_{\mathbb{R}^2}\Phi(t,x)\,(\rho(t,\mathrm{d}x)+\bar{\delta}(t,\mathrm{d}x))
-\int_{\mathbb{R}^2}\Phi(0,x)\,(\rho(0,\mathrm{d}x)+\bar{\delta}(0,\mathrm{d}x))\\
&=\int_0^t\int_{\mathbb{R}^2}\partial_{t}\Phi(s,x)\,(\rho(s,\mathrm{d}x)+\bar{\delta}(s,\mathrm{d}x))\,\mathrm{d}s+\int_0^t\mathcal{H}_{\Phi(s)}[\rho(s)+\bar{\delta}(s),\rho(s)+\bar{\delta}(s)]\,\mathrm{d}s\\
&\quad+\int_0^t\int_{\mathbb{R}^2}\nabla_x^2\Phi(s,x):\,\nu(s,\mathrm{d}x)\,\mathrm{d}s.
\end{split}\right.
\end{equation}
\end{Definition}

When there is no defect measure, assuming additional regularity on $\rho,\xi_{\alpha}$, the equation \eqref{eq-mv-E-defect} will reduce to the vortex-wave system. The proof is similar to \cite[Thm.1.6]{Mio19}, we put it in the appendix.
\begin{Proposition}\label{prn-E0nu-to-E00}
Let $(\rho,\{\xi_{\alpha}\})$ be an accumulation point given by Theorem~\ref{thm-main} and such that $\nu$  vanishes. If moreover $\rho \in L^{\infty}_{\rm{loc}}(\mathbb{R}_{+},L^{p}(\mathbb{R}^{2}))$ for some $p>2$ and  $\{\xi_{\alpha}\}\subset C^{1}(\mathbb{R}_{+},\mathbb{R}^{2})$, then  $(\rho,\{\xi_{\alpha}\})$  satisfies the vortex-wave system \eqref{eq-Vortex-Wave} in the distributional sense.
\end{Proposition}


\subsection{Main results}

Our main results can now be stated as follows.
\begin{Theorem}\label{thm-main}
Let $N>1$ and  $T>0$. Let $f_{\varepsilon}^{0}\ge 0$ and $(f_{\varepsilon}^{0},\{\xi_{\alpha,\varepsilon}^{0},\eta_{\alpha,\varepsilon}^{0}\})$ satisfy the following assumptions: for each $0<\varepsilon<1$, there exist constants $R_{\varepsilon},\sigma_{\varepsilon}>0$ depending only on $\varepsilon$ such that
\begin{equation}\label{eq-initial-assumptions}
\left\{\begin{split}
&\operatorname{supp}f_{\varepsilon}^{0}\subset\{(x,v)\in B(0,R_{\varepsilon}):\min_{\alpha}\vert x-\xi_{\alpha,\varepsilon}^0\vert\ge \sigma_{\varepsilon}\},\\
&K_0:=\sup_{0<\varepsilon<1}\Big(\int_{\mathbb{R}^2}(1+\vert x\vert^2)f_{\varepsilon}^0\,\mathrm{d}x\,\mathrm{d}v+\sum_{\alpha}\vert\xi_{\alpha,\varepsilon}^0\vert\Big)<\infty,\\
&K_1:=\left\vert\sup_{0<\varepsilon<1}\mathcal{E}(f_{\varepsilon}^0,\{\xi_{\alpha,\varepsilon}^0,\eta_{\alpha,\varepsilon}^0\})\right\vert<\infty,\quad\lim_{\varepsilon\to0}\varepsilon^2\| f_{\varepsilon}^0\|_{\infty}=0.
\end{split}\right.
\end{equation}
Let $(f_{\varepsilon},\{\xi_{\alpha,\varepsilon}\})$  denote the corresponding global weak solution of \eqref{eq-VP-point}.

Then there exists a subsequence  $\varepsilon_{n}\rightarrow 0$ as  $n\rightarrow+\infty$ such  that
\begin{itemize}
\item $\rho_{\varepsilon_{n}}\to\rho$ weak-* in $L^{\infty}(\mathbb{R}_{+},\mathcal{M}(\mathbb{R}^2))$, $\xi_{\alpha,\varepsilon_n}\to\xi_{\alpha}$ in $C^{r}([0,T],\mathbb{R}^2)$, $\forall r\in[0,1/4)$.

\item $\rho\in L^{\infty}(\mathbb{R}_{+},H^{-1}(\mathbb{R}^2))\cap C^{1/2}([0,T],W^{-2,1}(\mathbb{R}^2))$, $\xi_{\alpha}\in C^{1/2}([0,T],\mathbb{R}^2)$.

\item There  exists  $\mu_{0}=\mu_{0}(t,x,\omega)\in L^{\infty}(\mathbb{R}_{+},\mathcal{M}_{+}(\mathbb{R}^2\times\mathbb{S}^1))$  and  $\{\mathbf{M}^{\alpha}\}\subset L^{\infty}(\mathbb{R}_{+},\mathbb{R}^{2\times 2})$ such that $(\rho,\sum_{\alpha}\delta_{\xi_{\alpha}})$ is a weak solution to \eqref{eq-mv-E-defect} with defect measure $\nu$ defined by
\begin{equation}\label{eq-defect-measure}
\nu(t,x):=\int_{\mathbb{S}^1}\omega^{\perp}\otimes \omega \,\mu_{0}(t,x,\mathrm{d}\omega)+\sum_{\alpha} \mathbf{M}^{\alpha}(t)\delta_{\xi_{\alpha}(t)}.
\end{equation}
\end{itemize}
\end{Theorem}

\begin{Remark}
Note that we do not specify the convergence of $f_{\varepsilon}^{0}$, hence the initial condition of the limit equation is defined by the time continuity in distributional sense of the limit points.
\end{Remark}




\section{Preliminary estimates}

\subsection{ First dynamical estimates}

The characteristic associated to \eqref{eq-VP-point} is the solution of the following ODEs:
\begin{equation}
\left\{
\begin{split}
&\frac{\mathrm{d}}{\mathrm{d}s}X_{\varepsilon}(s,t,x,v)=\frac{V_{\varepsilon}(s,t,x,v)}{\varepsilon},\\
&\frac{\mathrm{d}}{\mathrm{d}s}V_{\varepsilon}(s,t,x,v)=\frac{V_{\varepsilon}^{\perp}(s,t,x,v)}{\varepsilon^2}+\frac{(E_{\varepsilon}+F_{\varepsilon})(s,X_{\varepsilon}(s,t,x,v))}{\varepsilon},\\
&(X_{\varepsilon},V_{\varepsilon})(t,t,x,v)=(x,v)\in\operatorname{supp}f_{\varepsilon}(t).
\end{split}\right.
\end{equation}
With the help of the characteristic, the solution to the system \eqref{eq-VP-point} can be represented by
\begin{equation*}
f_{\varepsilon}(t,x,v)=f_{\varepsilon}^0(X_{\varepsilon}(0,t,x,v),V_{\varepsilon}(0,t,x,v)).
\end{equation*}
It is well known that the characteristic $(X_{\varepsilon},V_{\varepsilon})(s,t,x,v)$ is a measure-preserving flow, which means, for any measurable function $g$, there holds:
\begin{equation*}
\int_{\mathbb{R}^2\times \mathbb{R}^2}g(x,v)\,\mathrm{d}x\,\mathrm{d}v=\int_{\mathbb{R}^2\times \mathbb{R}^2}g(X_{\varepsilon}(s,\tau,x,v),V_{\varepsilon}(s,\tau,x,v))\,\mathrm{d}x\,\mathrm{d}v.
\end{equation*}
In particular, we have the $L^p$ norms of $f_{\varepsilon}$ are conserved:
\begin{equation}\label{eq-Lp-conse}
\|f_{\varepsilon}(t)\|_{p}=\|f_{\varepsilon}^0\|_{p},\,~~\forall~1\le p\le \infty.
\end{equation}

Moreover, the energy and the moment associated to the system \eqref{eq-VP-point} are conserved:
\begin{equation}\label{eq-energy-moment-conse}
\begin{split}
\mathcal{E}(f_{\varepsilon},\{\xi_{\alpha,\varepsilon},\eta_{\alpha,\varepsilon}\})=\mathcal{E}(f_{\varepsilon}^0,\{\xi_{\alpha,\varepsilon}^0,\eta_{\alpha,\varepsilon}^0\}),\\
\mathcal{I}(f_{\varepsilon},\{\xi_{\alpha,\varepsilon},\eta_{\alpha,\varepsilon}\})=\mathcal{I}(f_{\varepsilon}^0,\{\xi_{\alpha,\varepsilon}^0,\eta_{\alpha,\varepsilon}^0\}).
\end{split}
\end{equation}
Rigorous proofs of them are standard, see \cite{Mio16,Mio19}, we omit for the sake of simplicity.

Thanks to the conservation laws \eqref{eq-Lp-conse} and \eqref{eq-energy-moment-conse}, we have the following a priori estimates.
\begin{Proposition}\label{prn-priori}
There exists a constant $C$ depending only on $K_0,K_1$ such that
\begin{align}
&\sup_{t\ge 0,\varepsilon>0}\iint_{\mathbb{R}^2\times\mathbb{R}^2}(\vert x\vert^2+\vert v\vert^2)f_{\varepsilon}(t,x,v)\,\mathrm{d}x\,\mathrm{d}v+\sum_{\alpha}(\vert\xi_{\alpha,\varepsilon}(t)\vert+\vert\eta_{\alpha,\varepsilon}(t)\vert)\le C,\label{eq-priori-1}\\
&\sup_{t\ge 0}\|\rho_{\varepsilon}(t)\|_{2}\le C\| f_{\varepsilon}^0\|_{\infty}^{1/2},\label{eq-priori-2}\\
&\sup_{t\ge 0,\varepsilon>0}\sum_{\alpha}\int_{\mathbb{R}^2}\big\vert\ln\vert x-\xi_{\alpha,\varepsilon}(t)\vert\big\vert\rho_{\varepsilon}(t,x)\,\mathrm{d}x\le C,\label{eq-priori-3}\\
&\sup_{t\ge 0,\varepsilon>0}\iint_{\mathbb{R}^2\times\mathbb{R}^2}\big\vert\ln\vert x-y\vert\big\vert\rho_{\varepsilon}(t,x)\rho_{\varepsilon}(t,y)\,\mathrm{d}x\,\mathrm{d}y\le C\label{eq-priori-4}.
\end{align}
And the non-concentration property holds:
\begin{equation}\label{eq-estimate-non-concentration}
\sup\limits_{t\in\mathbb{R}_{+}}\sup\limits_{0<\varepsilon<1}\sup\limits_{x_{0}\in\mathbb{R}^2}\sup\limits_{0<r<1/2}\vert\ln r\vert^{1/2}\int_{B(x_{0},r)}\rho_{\varepsilon}(t,x)\,\mathrm{d}x\le C.
\end{equation}
Moreover, there exists a constant $M>0$ depending only on $K_0,K_1$ such that
\begin{equation}\label{eq-priori-positive-distance}
\min_{t\ge 0,\varepsilon>0,\alpha\ne\beta}\vert\xi_{\alpha,\varepsilon}(t)-\xi_{\beta,\varepsilon}(t)\vert\ge M.
\end{equation}
\end{Proposition}
\begin{proof}
\eqref{eq-priori-1}-\eqref{eq-estimate-non-concentration} can be proved following lines by lines in \cite{Mio19}. We deduce \eqref{eq-priori-positive-distance} as follows: 
\begin{align*}
&\sum_{\alpha\ne\beta}\ln_{-}\vert \xi_{\alpha,\varepsilon}(t)-\xi_{\beta,\varepsilon}(t)\vert\\
&\le 2\mathcal{E}(f_{\varepsilon},\{\xi_{\alpha,\varepsilon},\eta_{\alpha,\varepsilon}\})+\iint_{\mathbb{R}^2\times\mathbb{R}^2}\ln_{+}\vert x-y\vert\rho_{\varepsilon}(t,x)\rho_{\varepsilon}(t,y)\,\mathrm{d}x\,\mathrm{d}y\\
&\quad+\sum_{\alpha}\int_{\mathbb{R}^2}\ln_{+}\vert x-\xi_{\alpha,\varepsilon}(t)\vert\rho_{\varepsilon}(t,x)\,\mathrm{d}x
+\sum_{\alpha\ne\beta}\ln_{+}\vert \xi_{\alpha,\varepsilon}(t)-\xi_{\beta,\varepsilon}(t)\vert\\
&\le 2\mathcal{E}(f_{\varepsilon}^0,\{\xi_{\alpha,\varepsilon}^0,\eta_{\alpha,\varepsilon}^0\})+\iint_{\mathbb{R}^2\times\mathbb{R}^2}\vert x-y\vert\rho_{\varepsilon}(t,x)\rho_{\varepsilon}(t,y)\,\mathrm{d}x\,\mathrm{d}y\\
&\quad+\sum_{\alpha}\int_{\mathbb{R}^2}\vert x-\xi_{\alpha,\varepsilon}(t)\vert\rho_{\varepsilon}(t,x)\,\mathrm{d}x
+\sum_{\alpha\ne\beta}\vert \xi_{\alpha,\varepsilon}(t)-\xi_{\beta,\varepsilon}(t)\vert.
\end{align*}
According to \eqref{eq-priori-1}-\eqref{eq-estimate-non-concentration}, the right hand side of the last inequality above can be controlled by a constant $\tilde{M}>0$ depending only on $K_0,K_1$, hence  we obtain
$\vert\xi_{\alpha,\varepsilon}(t)-\xi_{\beta,\varepsilon}(t)\vert\ge e^{-\tilde{M}}=:M$, the proof is completed.
\end{proof}

By the weak Young's inequality and the interpolation inequality in $L^p$, we have
\begin{Lemma}\label{lem-field-rho-interpo}
For  any $ p\in(1,2)$, we have
\begin{equation*}
\|E_{\varepsilon}(t)\|_{\frac{2p}{2-p}}\le C\|\rho_{\varepsilon}(t)\|_p\le C\|\rho_{\varepsilon}(t)\|_1^{\frac{2-p}{p}}\|\rho_{\varepsilon}(t)\|_2^{\frac{2p-2}{p}}.
\end{equation*}
\end{Lemma}

\subsection{$L_t^{\infty}H_x^{-1}$-bound of the densities}\label{subsec-LH-1-bound-rho}

We decompose $\rho_{\varepsilon}$ into a zero-mean part and a smooth part. Let $\bar{\rho}_{\varepsilon}(x)=\|f_{\varepsilon}^0\|_{1}\psi(x)$, where $\psi\in C_c^{\infty}(\mathbb{R}^2,\mathbb{R}_+)$ with $\operatorname{supp}\psi\subset B(0,1)$ and $\|\psi\|_{1}=1$. Denote 
\begin{equation*}
\widetilde{\rho}_{\varepsilon}=\rho_{\varepsilon}-\bar{\rho}_{\varepsilon},\quad\bar{E}_{\varepsilon}=\frac{x}{\vert x\vert^2}*\bar{\rho}_{\varepsilon},\quad\widetilde{E}_{\varepsilon}=\frac{x}{\vert x\vert^2}*\widetilde{\rho}_{\varepsilon}.
\end{equation*}
Notice $\|\bar{\rho}_{\varepsilon}\|_{L^1\cap L^{\infty}}\le (1+\|\psi\|_{\infty})\|f_{\varepsilon}^0\|_{1}$.

Recall the classical result from potential theory, see e.g., \cite[Lem.3]{GNPS05}.
\begin{Lemma}\label{lem-2D-potential}
Let $\rho\in L^2(\mathbb{R}^2)$ be such that
\begin{equation*}
\int_{\mathbb{R}^2}(1+\vert x\vert)\vert\rho(x)\vert\,\mathrm{d}x<\infty,\quad\int_{\mathbb{R}^2}\rho(x)\,\mathrm{d}x=0.
\end{equation*}
Consider the potential $U_{\rho}=-\ln\vert x\vert*\rho$. Then $U_{\rho}\in C_0(\mathbb{R}^2)$ and $\nabla U_{\rho}\in L^2(\mathbb{R}^2)$. In particular, the identity holds:
\begin{equation*}
2\pi\int_{\mathbb{R}^2}\rho(x)U_{\rho}(x)\,\mathrm{d}x=\int_{\mathbb{R}^2}\vert\nabla U_{\rho}(x)\vert^2\,\mathrm{d}x.
\end{equation*}
\end{Lemma}


Follow the technique in \cite{Mio16}, we have the uniform bound of the zero-mean parts.
\begin{Proposition}\label{prn-H-1-bound-density}
$\widetilde{\rho}_{\varepsilon}$ is uniformly bounded in $L^{\infty}(\mathbb{R}_+,H^{-1}(\mathbb{R}^2))$ and
\begin{align*}
\sup_{t\ge0,\varepsilon>0}\int_{\mathbb{R}^2}\vert\widetilde{E}_{\varepsilon}\vert^2\,\mathrm{d}x\le C.
\end{align*}
\end{Proposition}
\begin{proof}
Notice $\widetilde{\rho}_{\varepsilon}$ satisfies the assumption in Lemma~\ref{lem-2D-potential} and
\begin{align*}
\frac{1}{2\pi}\int_{\mathbb{R}^2}\vert\widetilde{E}_{\varepsilon}\vert^2\,\mathrm{d}x&=-\iint_{\mathbb{R}^2\times\mathbb{R}^2}\ln\vert x-y\vert\rho_{\varepsilon}(x)\rho_{\varepsilon}(y)\,\mathrm{d}x\,\mathrm{d}y-\iint_{\mathbb{R}^2\times\mathbb{R}^2}\ln\vert x-y\vert\bar{\rho}_{\varepsilon}(x)\bar{\rho}_{\varepsilon}(y)\,\mathrm{d}x\,\mathrm{d}y\\
&\quad+2\iint_{\mathbb{R}^2\times\mathbb{R}^2}\ln\vert x-y\vert\rho_{\varepsilon}(x)\bar{\rho}_{\varepsilon}(y)\,\mathrm{d}x\,\mathrm{d}y\\
&\le\iint_{\mathbb{R}^2\times\mathbb{R}^2}\big\vert\ln\vert x-y\vert\big\vert\rho_{\varepsilon}(x)\rho_{\varepsilon}(y)+\|f_{\varepsilon}^0\|_{1}^2\|\psi\|_{\infty}^2\iint_{B(0,1)\times B(0,1)}\big\vert\ln\vert x-y\vert\big\vert\\
&\quad+2\iint_{\mathbb{R}^2\times\mathbb{R}^2}\ln_+\vert x-y\vert\rho_{\varepsilon}(x)\bar{\rho}_{\varepsilon}(y).
\end{align*}
Notice $\ln_+\vert x-y\vert\le \vert x\vert+\vert y\vert$ and $\ln\vert x-y\vert$ is locally integrable in $\mathbb{R}^4$. By \eqref{eq-priori-1} and \eqref{eq-priori-4} we have
\begin{align*}
\sup_{t\ge0,\varepsilon>0}\int_{\mathbb{R}^2}\vert\widetilde{E}_{\varepsilon}\vert^2\,\mathrm{d}x\le C,
\end{align*}
which implies that $\widetilde{\rho}_{\varepsilon}$ is uniformly bounded in $L^{\infty}(\mathbb{R}_+,H^{-1}(\mathbb{R}^2))$. The proof is complete.
\end{proof}

\begin{Remark}
Notice by the H-L-S theorem, $\bar{E}_{\varepsilon}\in L_{\rm w}^2\cap L^{q}\not\subset L^2(\mathbb{R}^2)$ for any $2<q<\infty$. Hence the proposition above does not imply $E_{\varepsilon}\in L^2$.
\end{Remark}

\section{Estimates for the point charges}\label{sec-est-point-singular-field}
In this section we focus on estimates for the point charges. Firstly, we define the pointwise energy function as
\begin{equation*}
h_{\varepsilon}(t,x,v)=\frac{\vert v\vert^2}{2}+\sum_{\alpha}\big(\vert x-\xi_{\alpha,\varepsilon}(t)\vert-\ln\vert x-\xi_{\alpha,\varepsilon}(t)\vert\big)+K,
\end{equation*}
where $K>1$ is a constant. Notice $h_{\varepsilon}(t,x,v)\ge \vert v\vert^2/2$. We denote
\begin{equation*}
H_{k,\varepsilon}(t)=\sup_{0\le s\le t}\widetilde{H}_{k,\varepsilon}(s),\quad \widetilde{H}_{k,\varepsilon}(t)=\iint_{\mathbb{R}^2\times\mathbb{R}^2} h_{\varepsilon}^{k/2}f_{\varepsilon}(t,x,v)\,\mathrm{d}x\,\mathrm{d}v.
\end{equation*}


We have the following kinetic interpolation inequality, which is a trivial extension of \cite[Lem.3.1]{GS99}.
\begin{Lemma}\label{lem-interpo-Hk}
For all $0\le k\le l$, we have
\begin{equation*}
\Big\|\int_{\mathbb{R}^2} h_{\varepsilon}^{k/2}f_{\varepsilon}(t,\cdot,v)\,\mathrm{d}v\Big\|_{\frac{l+2}{k+2}}\le C \|f_{\varepsilon}^0\|_{\infty}^{\frac{l-k}{l+2}}H_{l,\varepsilon}(t)^{\frac{k+2}{l+2}}.
\end{equation*}
In particular, $H_{k,\varepsilon}(t)\le C'$ for all $0\le k\le 2$, where $C$ depends only on $k,l$; $C'$ depends only on $K_0,K_1$.
\end{Lemma}

For the sake of simplicity, we will use the following shorthand in the sequel of this section:
\begin{align*}
&(X_{\varepsilon},V_{\varepsilon})(s)=(X_{\varepsilon},V_{\varepsilon})(s,0,x,v),\\
&\mathbf{h}_{\varepsilon}(s)=h_{\varepsilon}(s,X_{\varepsilon}(s),V_{\varepsilon}(s)),
\end{align*}
where $(x,v)$ belongs to the support of $f_{\varepsilon}^0$. We adopt the approach established recently in \cite{WZ23} to obtain the following proposition. The proof is in the appendix.

\begin{Proposition}\label{prn-estimate-Lk}
We define $k$-th order singular moment by
\begin{equation*}
L_{k,\varepsilon}(t):=\sum_{\alpha}\int_0^t\iint_{\mathbb{R}^2\times\mathbb{R}^2} \frac{\mathbf{h}_{\varepsilon}(s)^{k/2}f_{\varepsilon}^0}{\vert X_{\varepsilon}(s)-\xi_{\alpha,\varepsilon}(s)\vert}\,\mathrm{d}x\,\mathrm{d}v\,\mathrm{d}s.
\end{equation*}
Then  for all $0\le k<l$ we have
\begin{align*}
&\int_0^t\iint_{\mathbb{R}^2\times\mathbb{R}^2}\frac{\mathbf{h}_{\varepsilon}(s)^{k/2}f_{\varepsilon}^0}{\vert X_{\varepsilon}(s)-\xi_{\alpha,\varepsilon}(s)\vert}\,\mathrm{d}x\,\mathrm{d}v\,\mathrm{d}s\\
&\le C(\varepsilon+\varepsilon^{-1}t) H_{k+1,\varepsilon}(t)+C\|f_{\varepsilon}^0\|_{\infty}^{\frac{l-k}{l+2}}H_{l,\varepsilon}(t)^{\frac{k+2}{l+2}}\int_0^t\|E_{\varepsilon}(s)\|_{\frac{l+2}{l-k}}\,\mathrm{d}s\\
&\qquad+CL_{k-1,\varepsilon}(t)+C\big(L_{0,\varepsilon}(t)+t\big)H_{k,\varepsilon}(t),
\end{align*}
where $C$ depends only on $N,K_0,K_1$ and $l,k$.
\end{Proposition}

Now we have the main result in the subsection. 
\begin{Corollary}\label{coro-key-L0}
For all $t\ge 0$, there holds
\begin{equation}\label{eq-coro-key-L0-1}
L_{0,\varepsilon}(t)\le C(\varepsilon^{-1}t+\varepsilon),
\end{equation}
where $C$ depends only on $N,K_0,K_1$.

In particular, there exists $t_0>0$ depending only on $N,K_0,K_1$ such that for all $0<t<t_0$, there holds
\begin{equation}\label{eq-coro-key-L0-2}
L_{0,\varepsilon}(t)\le Ct^{1/2}\varepsilon^{-1}.
\end{equation}
\end{Corollary}
\begin{proof}
Let $R,\sigma>0$ to be determined later, we split $\mathbb{R}^2\times\mathbb{R}^2$ into three parts $A_1,A_2,A_3$ as
\begin{align*}
A_1:=&\{(x,v):\vert X_{\varepsilon}(s)-\xi_{\alpha,\varepsilon}(s)\vert >\sigma\},\\
A_2:=&\{(x,v):\mathbf{h}_{\varepsilon}(s)>R\},\\
A_3:=&\{(x,v):\vert X_{\varepsilon}(s)-\xi_{\alpha,\varepsilon}(s)\vert \le\sigma,\mathbf{h}_{\varepsilon}(s)\le R\}.
\end{align*}
Then we have
\begin{align*}
&\int_0^t\iint_{\mathbb{R}^2\times\mathbb{R}^2} \frac{f_{\varepsilon}^0(x,v)}{\vert X_{\varepsilon}(s)-\xi_{\alpha,\varepsilon}(s)\vert}\,\mathrm{d}x\,\mathrm{d}v\,\mathrm{d}s\\
&=\int_0^t\iint_{A_1}+\int_0^t\iint_{A_2}+\int_0^t\iint_{A_3}\frac{f_{\varepsilon}^0(x,v)}{\vert X_{\varepsilon}(s)-\xi_{\alpha,\varepsilon}(s)\vert}\,\mathrm{d}x\,\mathrm{d}v\,\mathrm{d}s\\
&\le \sigma^{-1}\|f_{\varepsilon}^0\|_1t+R^{-1/2}\int_0^t\iint_{\mathbb{R}^2\times\mathbb{R}^2} \frac{\mathbf{h}_{\varepsilon}(s)^{1/2}f_{\varepsilon}^0(x,v)}{\vert X_{\varepsilon}(s)-\xi_{\alpha,\varepsilon}(s)\vert}\,\mathrm{d}x\,\mathrm{d}v\,\mathrm{d}s\\
&\quad+\int_0^t\iint_{\vert x-\xi_{\alpha,\varepsilon}(s)\vert \le\sigma,h_{\varepsilon}(s,x,v)\le R} \frac{f_{\varepsilon}(s,x,v)}{\vert x-\xi_{\alpha,\varepsilon}(s)\vert}\,\mathrm{d}x\,\mathrm{d}v\,\mathrm{d}s,
\end{align*}
Since $\vert v\vert \le 2\sqrt{h}_{\varepsilon}(s,x,v)$ and the $L^p$-norms are conserved \eqref{eq-Lp-conse}, we have
\begin{align*}
&\int_0^t\iint_{\vert x-\xi_{\alpha,\varepsilon}(s)\vert \le\sigma,h_{\varepsilon}(s,x,v)\le R} \frac{f_{\varepsilon}(s,x,v)}{\vert x-\xi_{\alpha,\varepsilon}(s)\vert}\,\mathrm{d}x\,\mathrm{d}v\,\mathrm{d}s\\
&\le \|f_{\varepsilon}^0\|_{\infty}\int_0^t\iint_{\vert x-\xi_{\alpha,\varepsilon}(s)\vert \le\sigma,\vert v\vert \le 2\sqrt{R}} \frac{1}{\vert x-\xi_{\alpha,\varepsilon}(s)\vert}\,\mathrm{d}x\,\mathrm{d}v\,\mathrm{d}s\\
&\le 8\pi^2\|f_{\varepsilon}^0\|_{\infty}R\sigma t.
\end{align*}
Hence we have
\begin{equation}\label{eq-2.10}
L_{0,\varepsilon}(t)\le N\sigma^{-1}t+NR^{-1/2}L_{1,\varepsilon}(t)+8\pi^2N\|f_{\varepsilon}^0\|_{\infty}R\sigma t.
\end{equation}
Take $k=1$, $l=2$ in Proposition~\ref{prn-estimate-Lk}, we have
\begin{align*}
L_{1,\varepsilon}(t)&\le C\Big((\varepsilon+t\varepsilon^{-1}) H_{2,\varepsilon}(t)+\|f_{\varepsilon}^0\|_{\infty}^{\frac{1}{4}}H_{2,\varepsilon}(t)^{\frac{3}{4}}\int_0^t\|E_{\varepsilon}(s)\|_{4}\,\mathrm{d}s\nonumber\\
&\qquad+L_{0,\varepsilon}(t)+\big(L_{0,\varepsilon}(t)+t\big)H_{1,\varepsilon}(t)\Big).
\end{align*}
By Lemma~\ref{lem-field-rho-interpo} and \eqref{eq-priori-2}, we have
\begin{align*}
\|E_{\varepsilon}(s)\|_{4}\le C\|\rho_{\varepsilon}(s)\|_1^{\frac{1}{2}}\|\rho_{\varepsilon}(s)\|_2^{\frac{1}{2}}\le C\|f_{\varepsilon}^{0}\|_{\infty}^{\frac{1}{4}},
\end{align*}
combining with $H_{2,\varepsilon}(t)\le C$ by Lemma~\ref{lem-interpo-Hk}, we obtain
\begin{align}\label{eq-2.11}
L_{1,\varepsilon}(t)\le\tilde{C}\Big(t\varepsilon^{-1}+\varepsilon+L_{0,\varepsilon}(t)+\|f_{\varepsilon}^0\|_{\infty}^{\frac{1}{2}}t\Big).
\end{align}
for some $\tilde{C}>1$.

Now take $R=\max\{(2N\tilde{C})^2,t^{-1}\}$ and insert \eqref{eq-2.11} into \eqref{eq-2.10}, we have
\begin{align*}
L_{0,\varepsilon}(t)&\le N\sigma^{-1}t+\min\{1/2,\tilde{C}Nt^{1/2}\}\Big(t\varepsilon^{-1}+\varepsilon+L_{0,\varepsilon}(t)+\|f_{\varepsilon}^0\|_{\infty}^{\frac{1}{2}}t\Big)\\
&\quad+8\pi^2N\|f_{\varepsilon}^0\|_{\infty}\max\{(2N\tilde{C})^2,t^{-1}\}\sigma t.
\end{align*}
Take $\sigma=\varepsilon t^{1/2}$ and recall the fact $\|f_{\varepsilon}^0\|_{\infty}\le C\varepsilon^{-2}$ by the assumptions \eqref{eq-initial-assumptions},  we obtain
\begin{align*}
L_{0,\varepsilon}(t)&\le 2N\varepsilon^{-1}t^{1/2}+\min\{1,2\tilde{C}Nt^{1/2}\}\Big(\varepsilon+(C+1)\varepsilon^{-1}t\Big)\\
&\quad+16\pi^2NC\varepsilon^{-1}\max\{(2N\tilde{C})^2,t^{-1}\}t^{3/2}.
\end{align*}
Let $t_0=(2N\tilde{C})^{-2}$, we obtain \eqref{eq-coro-key-L0-2}.

Similarly, take $R=\max\{(2N\tilde{C})^2,1\}$,  $\sigma=\varepsilon$ and insert \eqref{eq-2.11} into \eqref{eq-2.10}, we obtain \eqref{eq-coro-key-L0-1}.
\end{proof}

\section{Time regularity} \label{sec-time-regularity}


According to the ODE of $\eta_{\alpha,\varepsilon}$ in \eqref{eq-VP-point}, a consequence of \eqref{eq-coro-key-L0-2} combing with the boundedness of $\eta_{\alpha,\varepsilon}$ by \eqref{eq-priori-1} is that for all $0<s<t<T$
\begin{equation}\label{eq-eta-continuity}
\vert\eta_{\alpha,\varepsilon}(t)-\eta_{\alpha,\varepsilon}(s)\vert\le C(t-s)^{1/2}\varepsilon^{-2}.
\end{equation}



\subsection{Weak formulation}

\begin{Lemma}\label{lem-weak-form-rho}
$\int vf_{\varepsilon}\,\mathrm{d}v$ is bounded in $C^{1/4}(\mathbb{R}_+,W^{-1,1}(\mathbb{R}^2))$ and the following equation holds in the distributional sense:
\begin{equation*}
\partial_t\rho_{\varepsilon}+\nabla_x\cdot\Big((E_{\varepsilon}+F_{\varepsilon})^{\perp}\rho_{\varepsilon}\Big)
=\nabla_x\cdot\Big(\nabla_x \cdot\int_{\mathbb{R}^2}v \otimes v f_{\varepsilon}\,\mathrm{d}v\Big)^{\perp}
+\varepsilon\partial_t\nabla_x\cdot\int_{\mathbb{R}^2}v^{\perp}f_{\varepsilon}\,\mathrm{d}v.
\end{equation*}
\end{Lemma}
\begin{proof}
From Corollary~\ref{coro-key-L0}, $F_{\varepsilon}\rho_{\varepsilon}\in L_{t,x,v}^1$ for each $\varepsilon>0$. Hence testing \eqref{eq-VP-point} and applying the dominated convergence theorem, we can obtain that the following equations hold in the distributional sense (see \cite[Lem.3.2]{GS99} for details):
\begin{align*}
&\partial_t\rho_{\varepsilon}+\varepsilon^{-1}\nabla_x\cdot\int vf_{\varepsilon}\,\mathrm{d}v=0,\label{eq-}\\
&\varepsilon\partial_t\int vf_{\varepsilon}\,\mathrm{d}v+\nabla_x\cdot\int_{\mathbb{R}^2}v \otimes v f_{\varepsilon}\,\mathrm{d}v-(E_{\varepsilon}+F_{\varepsilon})\rho_{\varepsilon}-\varepsilon^{-1}\int v^{\perp}f_{\varepsilon}\,\mathrm{d}v=0,
\end{align*}
which implies the equation in the lemma. Moreover, the second equation implies: for all $0\le s<t<T$ with $t-s<t_0$
\begin{align*}
&\varepsilon\left\|\int vf_{\varepsilon}(t)\,\mathrm{d}v-\int vf_{\varepsilon}(s)\,\mathrm{d}v\right\|_{W^{-1,1}(\mathbb{R}^2)}\\
&\le C(t-s)\sup_{\tau\ge 0}\left[\varepsilon^{-1}\iint(1+\vert v\vert^2)f_{\varepsilon}(\tau)\,\mathrm{d}v\,\mathrm{d}x+\|\widetilde{E}_{\varepsilon}(\tau)\|_{2}\|\rho_{\varepsilon}(\tau)\|_{2}+\|\bar{E}_{\varepsilon}(\tau)\|_{\infty}\|\rho_{\varepsilon}(\tau)\|_{1}\right]\\
&\quad+C\varepsilon^{-1}(t-s)^{1/2}\\
&\le C\varepsilon^{-1}(t-s)^{1/2}.
\end{align*}
where we have used \eqref{eq-coro-key-L0-2} and the bounds in Sec.~\ref{subsec-LH-1-bound-rho}. On the other hand
\begin{equation*}
\varepsilon\left\|\int vf_{\varepsilon}(t)\,\mathrm{d}v-\int vf_{\varepsilon}(s)\,\mathrm{d}v\right\|_{W^{-1,1}(\mathbb{R}^2)}\le C\varepsilon\sup_{\tau\ge 0}\iint\vert v\vert f_{\varepsilon}(\tau)\,\mathrm{d}v\,\mathrm{d}x\le C\varepsilon.
\end{equation*}
Hence
\begin{equation*}
\varepsilon\left\|\int vf_{\varepsilon}(t)\,\mathrm{d}v-\int vf_{\varepsilon}(s)\,\mathrm{d}v\right\|_{W^{-1,1}(\mathbb{R}^2)}\le C(t-s)^{1/4}.
\end{equation*}
\end{proof}


\begin{Proposition}\label{prn-weak-form}
Let $\Phi\in C_{c}^{\infty}(\mathbb{R}_{+}\times\mathbb{R}^2)$, for all $t\ge 0$ we have
\begin{align*}
&\int_{\mathbb{R}^2}\Phi(t,x)\big(\rho_{\varepsilon}(t,x)\,\mathrm{d}x+\bar{\delta}_{\varepsilon}(t,\mathrm{d}x)\big)-\int_{\mathbb{R}^2}\Phi(0,x)\big(\rho_{\varepsilon}(0,x)\,\mathrm{d}x+\bar{\delta}_{\varepsilon}(0,\mathrm{d}x)\big)\\
&=\int_0^t\int_{\mathbb{R}^2}\partial_t\Phi(s,x)\big(\rho_{\varepsilon}(s,x)\,\mathrm{d}x+\bar{\delta}_{\varepsilon}(s,\mathrm{d}x)\big)\,\mathrm{d}s+\int_0^t\mathcal{H}_{\Phi(s,\cdot)}[\rho_{\varepsilon}(s,\cdot)+\bar{\delta}_{\varepsilon}(s,\cdot),\rho_{\varepsilon}(s,\cdot)+\bar{\delta}_{\varepsilon}(s,\cdot)]\,\mathrm{d}s\\
&\quad+\int_0^t\int_{\mathbb{R}^2}\nabla_x^2\Phi(s,x):\int_{\mathbb{R}^2}v^{\perp}\otimes vf_{\varepsilon}(s,x,v)\,\mathrm{d}v\,\mathrm{d}s\,\mathrm{d}x\\
&\quad+\sum_{\alpha}\int_0^t\nabla_x^2\Phi(s,\xi_{\alpha,\varepsilon}(s)):\eta_{\alpha,\varepsilon}^{\perp}\otimes\eta_{\alpha,\varepsilon}(s)\,\mathrm{d}s+R_{\varepsilon}(t)
\end{align*}
where $\bar{\delta}_{\varepsilon}(t,x)=\sum_{\alpha}\delta_{\xi_{\alpha,\varepsilon}(t)}(x)$. The reminder $R_{\varepsilon}$ satisfies the following estimates: for all $0\le s<t\le T$
\begin{equation}
%&\vert R_{\varepsilon}(t)\vert \le C\|\Phi\|_{\dot{W}^{2,\infty}(\mathbb{R}_{+}\times\mathbb{R}^2)}(t+1)\varepsilon,\label{eq-R-varepsilon-est-1}\\
\vert R_{\varepsilon}(t)-R_{\varepsilon}(s)\vert\le C\|\Phi\|_{\dot{W}^{2,\infty}(\mathbb{R}_{+}\times\mathbb{R}^2)}\min\{\varepsilon,(t-s)^{1/4}\}.\label{eq-R-varepsilon-est-2}
\end{equation}

%Moreover, if $\Phi$ is independent of $t$, then the estimate on $R_{\varepsilon}(t)$ can be improved as:
%\begin{equation*}\vert R_{\varepsilon}(t)\vert \le C\|\nabla_x\Phi\|_{\infty}(t+1)\varepsilon.
%\end{equation*}
\end{Proposition}
\begin{proof}
Apply the equation in Lemma~\ref{lem-weak-form-rho} with the test function $\Phi$. After symmetrizing the term $\nabla_x\cdot(E^{\perp}_{\varepsilon}\rho_{\varepsilon})$ as \eqref{eq-symmetric-form}, we obtain
\begin{align}
&\int_{\mathbb{R}^2}\Phi(t,x)\rho_{\varepsilon}(t,x)\,\mathrm{d}x-\int_{\mathbb{R}^2}\Phi(0,x)\rho_{\varepsilon}(0,x)\,\mathrm{d}x-\int_0^t\int_{\mathbb{R}^2}\partial_t\Phi(s,x)\rho_{\varepsilon}(s,x)\,\mathrm{d}s\,\mathrm{d}x\nonumber\\
&=\int_0^t\mathcal{H}_{\Phi(s,\cdot)}[\rho_{\varepsilon}(s),\rho_{\varepsilon}(s)]\,\mathrm{d}s+\int_0^t\int_{\mathbb{R}^2}\nabla_x\Phi(s,x)\cdot F_{\varepsilon}^{\perp}\rho_{\varepsilon}(s,x)\,\mathrm{d}s\,\mathrm{d}x\nonumber\\
&\quad+\int_0^t\int_{\mathbb{R}^2}\nabla_x^2\Phi(s,x):\int_{\mathbb{R}^2}v^{\perp}\otimes vf_{\varepsilon}(s,x,v)\,\mathrm{d}v\,\mathrm{d}s\,\mathrm{d}x
+I_{\varepsilon}^1(t),\label{eq-pf-prn-weak-form-1}
\end{align}
where
\begin{align*}
I_{\varepsilon}^1(t)&=\varepsilon\int_0^t\int_{\mathbb{R}^2}\partial_t\nabla_x\Phi(s,x)\cdot \int_{\mathbb{R}^2}v^{\perp}f_{\varepsilon}(s,x,v)\,\mathrm{d}v\,\mathrm{d}x\,\mathrm{d}s\\
&\quad-\varepsilon\int_{\mathbb{R}^2}\nabla_x\Phi(t,x)\cdot \int_{\mathbb{R}^2}v^{\perp}f_{\varepsilon}(t,x,v)\,\mathrm{d}v\,\mathrm{d}x
+\varepsilon\int_{\mathbb{R}^2}\nabla_x\Phi(0,x)\cdot \int_{\mathbb{R}^2}v^{\perp}f_{\varepsilon}^0(x,v)\,\mathrm{d}v\,\mathrm{d}x,
\end{align*}
by \eqref{eq-priori-1} and the continuity of $\int v^{\perp}f_{\varepsilon}\,\mathrm{d}v$ in Lemma~\ref{lem-weak-form-rho}, we have for all $0\le s<t\le T$
\begin{equation}\label{eq-prn-weak-form-R-1-1}
\vert I_{\varepsilon}^1(t)-I_{\varepsilon}^1(s)\vert\le C\|\Phi\|_{\dot{W}^{2,\infty}(\mathbb{R}_{+}\times\mathbb{R}^2)}\varepsilon(t-s)^{1/4}.
\end{equation}


Next by simple calculation, we have
\begin{align*}
&\Phi(t,\xi_{\alpha,\varepsilon}(t))-\Phi(0,\xi_{\alpha,\varepsilon}(0))\\
&=\int_0^t\partial_t\Phi(s,\xi_{\alpha,\varepsilon}(s))\,\mathrm{d}s+\varepsilon^{-1}\int_0^t\eta_{\alpha,\varepsilon}(s)\cdot\nabla_x\Phi(s,\xi_{\alpha,\varepsilon}(s))\,\mathrm{d}s
\end{align*}
and
\begin{align*}
&\varepsilon\nabla_x\Phi(t,\xi_{\alpha,\varepsilon}(t))\cdot\eta_{\alpha,\varepsilon}^{\perp}(t)-\varepsilon\nabla_x\Phi(0,\xi_{\alpha,\varepsilon}(0))\cdot\eta_{\alpha,\varepsilon}^{\perp}(0)\\
&=\varepsilon\int_0^t\partial_t\nabla_x\Phi(s,\xi_{\alpha,\varepsilon}(s))\cdot\eta_{\alpha,\varepsilon}^{\perp}(s)\,\mathrm{d}s+\int_0^t\nabla_x^2\Phi(s,\xi_{\alpha,\varepsilon}(s)):\eta_{\alpha,\varepsilon}^{\perp}\otimes\eta_{\alpha,\varepsilon}(s)\,\mathrm{d}s\\
&\quad+\int_0^t\nabla_x\Phi(s,\xi_{\alpha,\varepsilon}(s))\cdot\Big(-\frac{\eta_{\alpha,\varepsilon}(s)}{\varepsilon}+E^{\perp}_{\varepsilon}(s,\xi_{\alpha,\varepsilon}(s))+\sum_{\beta:\beta\ne\alpha}\frac{(\xi_{\alpha,\varepsilon}(s)-\xi_{\beta,\varepsilon}(s))^{\perp}}{\vert\xi_{\alpha,\varepsilon}(s)-\xi_{\beta,\varepsilon}(s)\vert^2}\Big)\,\mathrm{d}s.
\end{align*}
Adding the two equations above, we obtain
\begin{align}
&\Phi(t,\xi_{\alpha,\varepsilon}(t))-\Phi(0,\xi_{\alpha,\varepsilon}(0))\nonumber\\
&=\int_0^t\partial_t\Phi(s,\xi_{\alpha,\varepsilon}(s))\,\mathrm{d}s+\int_0^t\nabla_x^2\Phi(s,\xi_{\alpha,\varepsilon}(s)):\eta_{\alpha,\varepsilon}^{\perp}\otimes\eta_{\alpha,\varepsilon}(s)\,\mathrm{d}s\nonumber\\
&\quad+\int_0^t\nabla_x\Phi(s,\xi_{\alpha,\varepsilon}(s))\cdot\left(E^{\perp}_{\varepsilon}(s,\xi_{\alpha,\varepsilon}(s))+\sum_{\beta:\beta\ne\alpha}\frac{(\xi_{\alpha,\varepsilon}(s)-\xi_{\beta,\varepsilon}(s))^{\perp}}{\vert\xi_{\alpha,\varepsilon}(s)-\xi_{\beta,\varepsilon}(s)\vert^2}\right)\,\mathrm{d}s+I_{\varepsilon}^2(t),\label{eq-pf-prn-weak-form-2}
\end{align}
where
\begin{align*}
I_{\varepsilon}^2(t)=&\varepsilon\nabla_x\Phi(0,\xi_{\alpha,\varepsilon}(0))\cdot\eta_{\alpha,\varepsilon}^{\perp}(0)-\varepsilon\nabla_x\Phi(t,\xi_{\alpha,\varepsilon}(t))\cdot\eta_{\alpha,\varepsilon}^{\perp}(t)\\
&+\varepsilon\int_0^t\partial_t\nabla_x\Phi(s,\xi_{\alpha,\varepsilon}(s))\cdot\eta_{\alpha,\varepsilon}^{\perp}(s)\,\mathrm{d}s.
\end{align*}
by the energy conservation \eqref{eq-energy-moment-conse} and the assumptions \eqref{eq-initial-assumptions} we know that
\begin{equation}\label{eq-prn-weak-form-R-2-1}
\vert I_{\varepsilon}^2(t)\vert\le C\varepsilon(t\|\partial_t\nabla_x\Phi(s,x)\|_{\infty}+\|\nabla_x\Phi(t,x)\|_{\infty}).
\end{equation}
Recall \eqref{eq-eta-continuity}, we have  for all $0\le s<t\le T$
\begin{equation}\label{eq-prn-weak-form-R-2-2}
\vert I_{\varepsilon}^2(t)-I_{\varepsilon}^2(s)\vert\le C\|\Phi\|_{\dot{W}^{2,\infty}(\mathbb{R}_{+}\times\mathbb{R}^2)}\varepsilon^{-1}(t-s)^{1/2}.
\end{equation}

Moreover, we observe that
\begin{align*}
&\int_0^t\int_{\mathbb{R}^2}\nabla_x\Phi(s,x)F_{\varepsilon}^{\perp}(s,x)\rho_{\varepsilon}(s,x)\,\mathrm{d}s\,\mathrm{d}x+\sum_{\alpha}\int_0^t\nabla_x\Phi(s,\xi_{\alpha,\varepsilon}(s))\cdot E^{\perp}_{\varepsilon}(s,\xi_{\alpha,\varepsilon}(s))\,\mathrm{d}s\\
&=\sum_{\alpha}\int_0^t\int_{\mathbb{R}^2}\big(\nabla_x\Phi(s,x)-\nabla_x\Phi(s,\xi_{\alpha,\varepsilon}(s))\big)\cdot\frac{(x-\xi_{\alpha,\varepsilon}(s))^{\perp}}{\vert x-\xi_{\alpha,\varepsilon}(s)\vert^2}\rho_{\varepsilon}(s,x)\,\mathrm{d}x\,\mathrm{d}s\\
&=2\int_0^t\mathcal{H}_{\Phi(s)}[\rho_{\varepsilon}(s),\bar{\delta}_{\varepsilon}(s)]\,\mathrm{d}s.
\end{align*}
Similarly
\begin{equation*}
\sum_{\alpha}\int_0^t\nabla_x\Phi(s,\xi_{\alpha,\varepsilon}(s))\cdot\sum_{\beta:\beta\ne\alpha}\frac{(\xi_{\alpha,\varepsilon}(s)-\xi_{\beta,\varepsilon}(s))^{\perp}}{\vert\xi_{\alpha,\varepsilon}(s)-\xi_{\beta,\varepsilon}(s)\vert^2}\,\mathrm{d}s=\frac{1}{2}\sum_{\alpha\ne\beta}\int_0^tH_{\Phi(s)}\big(\xi_{\alpha,\varepsilon}(s),\xi_{\beta,\varepsilon}(s)\big)\,\mathrm{d}s.
\end{equation*}
And notice that
\begin{align*}
&\mathcal{H}_{\Phi(s)}[\rho_{\varepsilon}(s)+\bar{\delta}_{\varepsilon}(s),\rho_{\varepsilon}(s)+\bar{\delta}_{\varepsilon}(s)]\\
&=\mathcal{H}_{\Phi(s)}[\rho_{\varepsilon}(s),\rho_{\varepsilon}(s)]+2\mathcal{H}_{\Phi(s)}[\rho_{\varepsilon}(s),\bar{\delta}_{\varepsilon}(s)]+\frac{1}{2}\sum_{\alpha\ne\beta}H_{\Phi(s)}\big(\xi_{\alpha,\varepsilon}(s),\xi_{\beta,\varepsilon}(s)\big).
\end{align*}
By the above observation, the weak formulation follows by adding \eqref{eq-pf-prn-weak-form-1} and the sum in $\alpha$ of \eqref{eq-pf-prn-weak-form-2}. The estimates on the remainder term $R_{\varepsilon}=I_{\varepsilon}^1+I_{\varepsilon}^2$ follow from \eqref{eq-prn-weak-form-R-1-1}, \eqref{eq-prn-weak-form-R-2-1}, \eqref{eq-prn-weak-form-R-2-2}.
\end{proof}

\subsection{Time equi-continuity}


We first obtain the following lemma by the arguments of \cite[Cor.2.11]{Mio19}.
\begin{Lemma}\label{lem-min-continuity-point-charge}
Let $T>0$. There exists  $\widetilde{K}_{0}>2$ and $\widetilde{\varepsilon}_0=\widetilde{K}_{0}^{-4}$ depending only on $N,T,K_0,K_1$, such that for all $0<\varepsilon <\widetilde{\varepsilon}_0$ and for all  $0\le s < t\le T$ with $t-s\le\widetilde{\varepsilon}_0$
\begin{equation*}
\max_{\alpha}\min_{\beta}\vert\xi_{\alpha,\varepsilon}(t)-\xi_{\beta,\varepsilon}(s)\vert\le\widetilde{K}_{0}\min\{\sqrt{t-s+\varepsilon},(t-s)^{1/4}\}.
\end{equation*}
\end{Lemma}
\begin{proof}
We only prove $\max_{\alpha}\min_{\beta}\vert\xi_{\alpha,\varepsilon}(t)-\xi_{\beta,\varepsilon}(s)\vert\le\widetilde{K}_{0}\sqrt{t-s+\varepsilon}$. A similar routine can be used to prove $\max_{\alpha}\min_{\beta}\vert\xi_{\alpha,\varepsilon}(t)-\xi_{\beta,\varepsilon}(s)\vert\le\widetilde{K}_{0}(t-s)^{1/4}$.

By Proposition~\ref{prn-weak-form} we have for all  $\Phi\in C_{c}^{\infty}(\mathbb{R}^2)$ and for all  $0\le s < t\le T$
\begin{align*}
&\int_{\mathbb{R}^2}\Phi(x)\big(\rho_{\varepsilon}(t,x)\,\mathrm{d}x+\bar{\delta}_{\varepsilon}(t,\mathrm{d}x)\big)-\int_{\mathbb{R}^2}\Phi(x)\big(\rho_{\varepsilon}(s,x)\,\mathrm{d}x+\bar{\delta}_{\varepsilon}(s,\mathrm{d}x)\big)\\
&=\int_s^t\mathcal{H}_{\Phi}[\rho_{\varepsilon}(\tau,\cdot)+\bar{\delta}_{\varepsilon}(\tau,\cdot),\rho_{\varepsilon}(\tau,\cdot)+\bar{\delta}_{\varepsilon}(\tau,\cdot)]\,\mathrm{d}\tau+\int_s^t\iint_{\mathbb{R}^4}\nabla_x^2\Phi(x):v^{\perp}\otimes vf_{\varepsilon}(\tau,x,v)\,\mathrm{d}v\,\mathrm{d}x\,\mathrm{d}\tau\\
&\quad+\sum_{\alpha}\int_s^t\nabla_x^2\Phi(\xi_{\alpha,\varepsilon}(\tau)):\eta_{\alpha,\varepsilon}^{\perp}\otimes\eta_{\alpha,\varepsilon}(\tau)\,\mathrm{d}\tau+R_{\varepsilon}(t)-R_{\varepsilon}(s).
\end{align*}
Since $\vert H_{\Phi}(x,y)\vert\le C\|\nabla_x^2\Phi\|_{\infty}$, and by Proposition~\ref{prn-priori} it follows that
\begin{align}
&\Big\vert\int_{\mathbb{R}^2}\Phi(x)\rho_{\varepsilon}(t,x)\,\mathrm{d}x+\sum_{\alpha}\Phi(\xi_{\alpha,\varepsilon}(t))-\int_{\mathbb{R}^2}\Phi(x)\rho_{\varepsilon}(s,x)\,\mathrm{d}x-\sum_{\alpha}\Phi(\xi_{\alpha,\varepsilon}(s))\Big\vert\nonumber\\
&\le C\|\nabla_x^2\Phi\|_{\infty}(t-s)\sup_{\tau\in[s,t]}\Big(\|\rho_{\varepsilon}(\tau)\|^{2}_{1}+\|\rho_{\varepsilon}(\tau)\|_{1}+\iint_{\mathbb{R}^2}\vert v\vert^{2}f_{\varepsilon}(\tau)\,\mathrm{d}x\,\mathrm{d}v+\sum_{\alpha}\vert\eta_{\alpha,\varepsilon}(\tau)\vert^{2}\Big)\nonumber\\
&\quad+C\|\Phi\|_{\dot{W}^{2,\infty}(\mathbb{R}_{+}\times\mathbb{R}^2)}\min\{\varepsilon,(t-s)^{1/4}\}\nonumber\\
&\le C\|\Phi\|_{\dot{W}^{2,\infty}}(t-s+\varepsilon).\label{eq-lem-min-continuity-point-charge-1}
\end{align}

By contradiction, we assume there exist $\varepsilon,s,t$ and some $\alpha$ satisfying $\min_{\beta}\vert\xi_{\alpha,\varepsilon}(t)-\xi_{\beta,\varepsilon}(s)\vert> \widetilde{K}_{0}\sqrt{t-s+\varepsilon}$. We set
\begin{equation*}
\Phi(x)=\chi\left(\frac{2(x-\xi_{\alpha,\varepsilon}(t))}{\widetilde{K}_0\sqrt{t-s+\varepsilon}}\right),
\end{equation*}
where $\chi$ is a cut-off function such that $\chi=1$ on $B(0,1)$ and $\chi$ vanishes on $B(0,2)^c$. In particular, since $\widetilde{K}_{0}\sqrt{t-s+\varepsilon}<1$, we have
\begin{equation*}
\sum_{\alpha}\Phi(\xi_{\alpha,\varepsilon}(t))\ge 1,\quad\sum_{\alpha}\Phi(\xi_{\alpha,\varepsilon}(s))=0,\quad\|\Phi\|_{W^{2,\infty}}\le C\widetilde{K}_0^{-2}(t-s+\varepsilon)^{-1}.
\end{equation*}

In view of \eqref{eq-lem-min-continuity-point-charge-1} and using \eqref{eq-estimate-non-concentration}, we get
\begin{align*}
1&\le 2\sup_{\tau\in[0,T]}\int_{B(\xi_{\alpha,\varepsilon}(\tau),2\widetilde{K}_0\sqrt{t-s+\varepsilon})}\rho_{\varepsilon}(\tau,x)\,\mathrm{d}x+C\widetilde{K}_0^{-2}\\
&\le 2\sup_{\tau\in[0,T]}\int_{B(\xi_{\alpha,\varepsilon}(\tau),4\widetilde{K}_0^{-1})}\rho_{\varepsilon}(\tau,x)\,\mathrm{d}x+C\widetilde{K}_0^{-2}\\
&\le C\left(\vert\ln(4\widetilde{K}_0^{-1})\vert^{-1/2}+\widetilde{K}_0^{-3}\right)\le\frac{1}{2}
\end{align*}
if we choose $\widetilde{K}_0$ sufficiently large. This yields a contradiction and the lemma follows.
\end{proof}

Now we prove the crucial lemma in this section.
\begin{Lemma}\label{lem-small-time-continuity-point-charge}
Choose $\widetilde{K}_{0}> 1$ large enough in Lemma~\ref{lem-min-continuity-point-charge} and let $\widetilde{\varepsilon}_0=\widetilde{K}_{0}^{-8}$. Then for all $0<\varepsilon <\widetilde{\varepsilon}_0$ and for all $0\le s < t\le T$ with $t-s\le\widetilde{K}_{0}^{-4}$
\begin{equation*}
\sum_{\alpha}\vert\xi_{\alpha,\varepsilon}(t)-\xi_{\alpha,\varepsilon}(s)\vert\le\widetilde{K}_{0}\min\{\sqrt{t-s+\varepsilon},(t-s)^{1/4}\}.
\end{equation*}
\end{Lemma}
\begin{proof}
We only prove $\sum_{\alpha}\vert\xi_{\alpha,\varepsilon}(t)-\xi_{\alpha,\varepsilon}(s)\vert\le\widetilde{K}_{0}\sqrt{t-s+\varepsilon}$. Firstly, recall the equation of $\xi_{\alpha,\varepsilon}$ in \eqref{eq-VP-point} and the uniform bound of $\vert\eta_{\alpha,\varepsilon}\vert$ in \eqref{eq-priori-1}, we have for all $\alpha$ and $0\le s<t\le T$ with $t-s\le \widetilde{K}_0^{-4}\varepsilon$
\begin{equation*}
\vert\xi_{\alpha,\varepsilon}(t)-\xi_{\alpha,\varepsilon}(s)\vert\le \frac{\widetilde{C}(t-s)}{\varepsilon}\le \widetilde{C}\widetilde{K}_0^{-4}.
\end{equation*}
It is reasonable to choose $\widetilde{K}_0,\widetilde{\varepsilon}_0$ in Lemma~\ref{lem-min-continuity-point-charge} satisfying 
\begin{equation}\label{eq-choice-tilde-K0}
\min\{4\widetilde{C}\widetilde{K}_0^{-4},8\widetilde{K}_0^{-1}\}<M,\quad\widetilde{\varepsilon}_0=\widetilde{K}_{0}^{-8}.
\end{equation}
Then for $t-s\le \widetilde{K}_0^{-4}\varepsilon$
\begin{equation*}
\min_{\beta}\vert\xi_{\alpha,\varepsilon}(t)-\xi_{\beta,\varepsilon}(s)\vert\le\widetilde{K}_{0}\sqrt{t-s+\varepsilon}<M/2.
\end{equation*}
Recall \eqref{eq-priori-positive-distance}, it implies for all $0<\varepsilon<\widetilde{\varepsilon}_0$ and $0\le s<t\le T$ with $t-s\le \widetilde{K}_0^{-4}\varepsilon$
\begin{equation*}
\vert\xi_{\alpha,\varepsilon}(t)-\xi_{\alpha,\varepsilon}(s)\vert\le\widetilde{K}_{0}\sqrt{t-s+\varepsilon}.
\end{equation*}

Now we iterate this estimate in four steps:

{\bf Step 1:} Let $t-s\le \widetilde{K}_0^{-4}\varepsilon^{3/4}$, $\Delta_{\tau}:=\widetilde{K}_0^{-4}\varepsilon$, we have by \eqref{eq-choice-tilde-K0}
\begin{equation*}
\vert\xi_{\alpha,\varepsilon}(t)-\xi_{\alpha,\varepsilon}(s)\vert\le\left\lceil\frac{t-s}{\Delta_{\tau}}\right\rceil\widetilde{K}_{0}\sqrt{\Delta_{\tau}+\varepsilon}\le 2\widetilde{K}_{0}\varepsilon^{1/4}\sqrt{\widetilde{K}_0^{-4}+1}\le 4\widetilde{K}_{0}^{-1}< M/2.
\end{equation*}
By \eqref{eq-priori-positive-distance} and  Lemma~\ref{lem-min-continuity-point-charge}, we have for all $0<\varepsilon<\widetilde{\varepsilon}_0$ and $0\le s<t\le T$ with $t-s\le \widetilde{K}_0^{-4}\varepsilon^{3/4}$
\begin{equation*}
\vert\xi_{\alpha,\varepsilon}(t)-\xi_{\alpha,\varepsilon}(s)\vert\le\widetilde{K}_{0}\sqrt{t-s+\varepsilon}.
\end{equation*}

{\bf Step 2:} Similarly, let $t-s\le \widetilde{K}_0^{-4}\varepsilon^{1/2}$, $\Delta_{\tau}:=\widetilde{K}_0^{-4}\varepsilon^{3/4}$, we have by \eqref{eq-choice-tilde-K0}
\begin{equation*}
\vert\xi_{\alpha,\varepsilon}(t)-\xi_{\alpha,\varepsilon}(s)\vert\le 2\widetilde{K}_{0}\sqrt{\widetilde{K}_0^{-4}\varepsilon^{1/4}+\varepsilon^{1/2}}\le 4\widetilde{K}_{0}^{-1}< M/2.
\end{equation*}
By \eqref{eq-priori-positive-distance} and  Lemma~\ref{lem-min-continuity-point-charge}, we have for all $0<\varepsilon<\widetilde{\varepsilon}_0$ and $0\le s<t\le T$ with $t-s\le \widetilde{K}_0^{-4}\varepsilon^{1/2}$
\begin{equation*}
\vert\xi_{\alpha,\varepsilon}(t)-\xi_{\alpha,\varepsilon}(s)\vert\le\widetilde{K}_{0}\sqrt{t-s+\varepsilon}.
\end{equation*}

{\bf Step 3:} Again, let $t-s\le \widetilde{K}_0^{-4}\varepsilon^{1/4}$, $\Delta_{\tau}:=\widetilde{K}_0^{-4}\varepsilon^{1/2}$, we have by \eqref{eq-choice-tilde-K0}
\begin{equation*}
\vert\xi_{\alpha,\varepsilon}(t)-\xi_{\alpha,\varepsilon}(s)\vert\le 2\widetilde{K}_{0}\sqrt{\widetilde{K}_0^{-4}+\varepsilon^{1/2}}\le 4\widetilde{K}_{0}^{-1}< M/2.
\end{equation*}
By \eqref{eq-priori-positive-distance} and  Lemma~\ref{lem-min-continuity-point-charge}, we have for all $0<\varepsilon<\widetilde{\varepsilon}_0$ and $0\le s<t\le T$ with $t-s\le \widetilde{K}_0^{-4}\varepsilon^{1/4}$
\begin{equation*}
\vert\xi_{\alpha,\varepsilon}(t)-\xi_{\alpha,\varepsilon}(s)\vert\le\widetilde{K}_{0}\sqrt{t-s+\varepsilon}.
\end{equation*}

{\bf Step 4:} Now assume $k\ge 2$, such that for all $0<\varepsilon<\widetilde{\varepsilon}_0$ and $0\le s<t\le T$ with $t-s\le \widetilde{K}_0^{-4}\varepsilon^{2^{-k}}$
\begin{equation*}
\vert\xi_{\alpha,\varepsilon}(t)-\xi_{\alpha,\varepsilon}(s)\vert\le\widetilde{K}_{0}\sqrt{t-s+\varepsilon}.
\end{equation*}
Let $t-s\le \widetilde{K}_0^{-4}\varepsilon^{2^{-(k+1)}}$, $\Delta_{\tau}:=\widetilde{K}_0^{-4}\varepsilon^{2^{-k}}$, we have by \eqref{eq-choice-tilde-K0}
\begin{equation*}
\vert\xi_{\alpha,\varepsilon}(t)-\xi_{\alpha,\varepsilon}(s)\vert\le 2\widetilde{K}_{0}\varepsilon^{-2^{-(k+1)}}\sqrt{\widetilde{K}_0^{-4}\varepsilon^{2^{-k}}+\varepsilon}\le 4\widetilde{K}_{0}^{-1}< M/2.
\end{equation*}
By \eqref{eq-priori-positive-distance} and  Lemma~\ref{lem-min-continuity-point-charge}, we have for all $0<\varepsilon<\widetilde{\varepsilon}_0$ and $0\le s<t\le T$ with $t-s\le \widetilde{K}_0^{-4}\varepsilon^{2^{-(k+1)}}$
\begin{equation*}
\vert\xi_{\alpha,\varepsilon}(t)-\xi_{\alpha,\varepsilon}(s)\vert\le\widetilde{K}_{0}\sqrt{t-s+\varepsilon}.
\end{equation*}
By induction, the statement holds for all $k\ge 2$. Hence take $k\to\infty$, we have
for all $0<\varepsilon<\widetilde{\varepsilon}_0$ and $0\le s<t\le T$ with $t-s\le \widetilde{K}_0^{-4}$
\begin{equation*}
\vert\xi_{\alpha,\varepsilon}(t)-\xi_{\alpha,\varepsilon}(s)\vert\le\widetilde{K}_{0}\sqrt{t-s+\varepsilon}.
\end{equation*}
The proof is complete.
\end{proof}


\begin{Proposition}\label{prn-continuity-point-charge}
Let $T>0$. There exists $\widetilde{K}> 1$ and $\varepsilon_{0}>0$ depending only on $N,T,K_0,K_1$, such that for all $0< \varepsilon < \varepsilon_{0}$ and for all  $0\le s < t\le T$
\begin{equation*}
\|\rho_{\varepsilon}(t)-\rho_{\varepsilon}(s)\|_{\dot{W}^{-2,1}}+\sum_{\alpha}\vert\xi_{\alpha,\varepsilon}(t)-\xi_{\alpha,\varepsilon}(s)\vert\le\widetilde{K}\min\{\sqrt{t-s+\varepsilon},(t-s)^{1/4}\}.
\end{equation*}
\end{Proposition}

\begin{proof}
It follows from \eqref{eq-lem-min-continuity-point-charge-1} that
\begin{align*}
&\Big\vert\int_{\mathbb{R}^2}\Phi(x)\rho_{\varepsilon}(t,x)\,\mathrm{d}x-\int_{\mathbb{R}^2}\Phi(x)\rho_{\varepsilon}(s,x)\,\mathrm{d}x\Big\vert\\
&\le\sum_{\alpha}\vert\Phi(\xi_{\alpha,\varepsilon}(t))-\Phi(\xi_{\alpha,\varepsilon}(s))\vert+C\|\Phi\|_{\dot{W}^{2,\infty}}(t-s+\varepsilon)\\
&\le\|\Phi\|_{\dot{W}^{2,\infty}}\left(\sum_{\alpha}\vert\xi_{\alpha,\varepsilon}(t)-\xi_{\alpha,\varepsilon}(s)\vert+C(t-s+\varepsilon)\right).
\end{align*}
Applying Lemma~\ref{lem-small-time-continuity-point-charge} and notice $\xi_{\alpha,\varepsilon}$ is uniformly bounded by \eqref{eq-priori-1}, the conclusion follows.
\end{proof}



\section{Proof of Theorem~\ref{thm-main}}

We define the radial moment density as
\begin{equation*}
m_{k,\varepsilon}(t,x,\omega)=\int_{0}^{\infty}r^kf_{\varepsilon}(t,x,r\omega)\,\mathrm{d}r.
\end{equation*}
Notice the term in Proposition~\ref{prn-weak-form}
\begin{equation*}
\int_0^t\int_{\mathbb{R}^2}\nabla_x^2\Phi(s,x):\int_{\mathbb{R}^2}v^{\perp}\otimes vf_{\varepsilon}(s,x,v)\,\mathrm{d}v\,\mathrm{d}s\,\mathrm{d}x=\int_0^t\int_{\mathbb{R}^2}\int_{\mathbb{S}^1}\widetilde{\Phi}(s,x,\omega)m_{3,\varepsilon}(s,x,\omega)\,\mathrm{d}\omega\,\mathrm{d}s\,\mathrm{d}x,
\end{equation*}
where $\widetilde{\Phi}(s,x,\omega)=\nabla_x^2\Phi(s,x):\omega^{\perp}\otimes \omega\in C_c^{\infty}(\mathbb{R}_+\times\mathbb{R}^2\times\mathbb{S}^1)$.

Now we obtain the limit points by the compactness arguments stated in \cite{GS99,Mio19}.

{\bf Convergence of $f_{\varepsilon}$:} By the $L^1$-bound \eqref{eq-Lp-conse}, there exist a subsequence such that $f_{\varepsilon_n}\to f$ weak-* in $L^{\infty}(\mathbb{R}_{+},\mathcal{M}(\mathbb{R}^2\times\mathbb{R}^2))$. Moreover, $\nabla_{v}\cdot(v^{\perp}f)=0$ in the distributional sense by the lemmas in \cite[Lem.3.3]{GS99}, \cite[Lem.2.16]{Mio19}, which implies
\begin{equation*}
\int_0^t\int_{\mathbb{R}^2}\int_{\mathbb{S}^1}\widetilde{\Phi}(s,x,\omega)\,m_{3}(s,\mathrm{d}x,\mathrm{d}\omega)\,\mathrm{d}s=\int_0^t\int_{\mathbb{R}^2}\nabla_x^2\Phi(s,x):\int_{\mathbb{R}^2}v^{\perp}\otimes v\,f(s,\mathrm{d}x,\mathrm{d}v)\,\mathrm{d}s\equiv0,
\end{equation*}
where the measure $m_3(t)$ is defined by
\begin{equation*}
\iint_{\mathbb{R}^2\times\mathbb{S}^1}\Psi(x,\omega)\,m_{3}(t,\mathrm{d}x,\mathrm{d}\omega)=\iint_{\mathbb{R}^2\times\mathbb{R}^2}\Psi(x,v/\vert v\vert)\vert v\vert^2\,f(t,\mathrm{d}x,\mathrm{d}v)\quad\forall\Psi\in C_0(\mathbb{R}^2\times\mathbb{S}^1).
\end{equation*}

{\bf Convergence of $\rho_{\varepsilon}$:} Notice by the uniform bound of $\iint\vert v\vert^2f_{\varepsilon}$ in \eqref{eq-priori-1}, we have $\rho_{\varepsilon_n}\to \rho=\int_{\mathbb{R}^2}\,f(\mathrm{d}v)$ weak-* in $L^{\infty}(\mathbb{R}_{+},\mathcal{M}(\mathbb{R}^2))$. Recall Proposition~\ref{prn-continuity-point-charge}, $\rho\in C^{1/2}([0,T],W^{-2,1}(\mathbb{R}^2))$ for all $T>0$.

{\bf Convergence of $\xi_{\alpha,\varepsilon}$:} Recall Proposition~\ref{prn-continuity-point-charge} again and by the Arzel\`a-Ascoli theorem, there exists a subsequence, still denoted as $\xi_{\alpha,\varepsilon_n}$, such that $\xi_{\alpha,\varepsilon_n}\to\xi_{\alpha}$ in $C^{r}([0,T],\mathbb{R}^2)$, $\forall r\in[0,1/4)$ and $\xi_{\alpha}\in C^{1/2}([0,T],\mathbb{R}^2)$, $\forall T>0$, which implies
\begin{equation*}
\bar{\delta}_{\varepsilon_n}=\sum_{\alpha}\delta_{\xi_{\alpha,\varepsilon_n}}\to\bar{\delta}=\sum_{\alpha}\delta_{\xi_{\alpha}}\text{ weak-* in }L^{\infty}(\mathbb{R}_{+},\mathcal{M}(\mathbb{R}^2)).
\end{equation*}

{\bf Convergence of $\eta_{\alpha,\varepsilon}$:} Notice by \eqref{eq-priori-1} again, $\vert\eta_{\alpha,\varepsilon_n}\vert^2$ is uniformly bounded in $L^{\infty}(\mathbb{R}_{+})$. Hence there exists a subsequence such that
\begin{equation*}
\eta_{\alpha,\varepsilon_n}^{\perp}\otimes\eta_{\alpha,\varepsilon_n}\to \mathbf{M}^{\alpha}\text{ weak-* in }L^{\infty}(\mathbb{R}_{+},\mathbb{R}^{2\times 2}),
\end{equation*}
where $\mathbf{M}^{\alpha}$ satisfies $\mathbf{M}_{11}^{\alpha}=-\mathbf{M}_{22}^{\alpha}$ and $\mathbf{M}_{12}^{\alpha}=-\mathbf{M}_{21}^{\alpha}$.

{\bf The existence of defect measure:} Notice by \eqref{eq-priori-1}
\begin{equation*}
\sup_{t\ge0,\varepsilon>0}\int_{\mathbb{R}^2}\int_{\mathbb{S}^1}m_{3,\varepsilon}(s,x,\omega)\,\mathrm{d}\omega\,\mathrm{d}x=\sup_{t\ge0,\varepsilon>0}\iint_{\mathbb{R}^2\times\mathbb{R}^2}\vert v\vert^2f_{\varepsilon}(s,x,v)\,\mathrm{d}v\,\mathrm{d}x\le C,
\end{equation*}
there exist a subsequence, still denoted as $m_{3,\varepsilon_n}$, such that $m_{3,\varepsilon_n}\to \mu$ weak-* in $L^{\infty}(\mathbb{R}_{+},\mathcal{M}(\mathbb{R}^2\times\mathbb{S}^1))$. Denote $\mu_{0}:=\mu-m_3$, then $\mu_{0}\in L^{\infty}(\mathbb{R}_{+},\mathcal{M}_+(\mathbb{R}^2\times\mathbb{S}^1))$ by the arguments of \cite[p.802, Proof of Thm.A]{GS99}.

The compactness statements above imply that 
\begin{align*}
&\int_{\mathbb{R}^2}\Phi(t,x)\big(\rho_{\varepsilon_n}(t,x)\,\mathrm{d}x+\bar{\delta}_{\varepsilon_n}(t,\mathrm{d}x)\big)-\int_{\mathbb{R}^2}\Phi(0,x)\big(\rho_{\varepsilon_n}(0,x)\,\mathrm{d}x+\bar{\delta}_{\varepsilon_n}(0,\mathrm{d}x)\big)\\
&-\int_0^t\int_{\mathbb{R}^2}\partial_t\Phi(s,x)\big(\rho_{\varepsilon_n}(s,x)\,\mathrm{d}x+\bar{\delta}_{\varepsilon_n}(s,\mathrm{d}x)\big)\,\mathrm{d}s-\int_0^t\int_{\mathbb{R}^2}\nabla_x^2\Phi(s,x):\int_{\mathbb{R}^2}v^{\perp}\otimes vf_{\varepsilon_n}(s,x,v)\,\mathrm{d}v\,\mathrm{d}s\,\mathrm{d}x\\
&\quad-\sum_{\alpha}\int_0^t\nabla_x^2\Phi(s,\xi_{\alpha,\varepsilon}(s)):\eta_{\alpha,\varepsilon}^{\perp}\otimes\eta_{\alpha,\varepsilon}(s)\,\mathrm{d}s-R_{\varepsilon_n}(t)
\end{align*}
converges to
\begin{align*}
&\int_{\mathbb{R}^2}\Phi(t,x)\big(\rho(t,\mathrm{d}x)+\bar{\delta}(t,\mathrm{d}x)\big)-\int_{\mathbb{R}^2}\Phi(0,x)\big(\rho(0,\mathrm{d}x)+\bar{\delta}(0,\mathrm{d}x)\big)\\
&-\int_0^t\int_{\mathbb{R}^2}\partial_t\Phi(s,x)\big(\rho(s,\mathrm{d}x)+\bar{\delta}(s,\mathrm{d}x)\big)\,\mathrm{d}s-\int_0^t\int_{\mathbb{R}^2}\nabla_x^2\Phi(s,x):\int_{\mathbb{S}^1}\omega^{\perp}\otimes\omega\,\mu_0(s,\mathrm{d}x,\mathrm{d}\omega)\,\mathrm{d}s\\
&\quad-\sum_{\alpha}\int_0^t\nabla_x^2\Phi(s,\xi_{\alpha}(s)):\mathbf{M}^{\alpha}(s)\,\mathrm{d}s.
\end{align*}

Finally, we prove
\begin{align*}
\int_0^t\mathcal{H}_{\Phi(s,\cdot)}[\rho_{\varepsilon_n}(s,\cdot)+\bar{\delta}_{\varepsilon_n}(s,\cdot),\rho_{\varepsilon_n}(s,\cdot)+\bar{\delta}_{\varepsilon_n}(s,\cdot)]\,\mathrm{d}s
\end{align*}
converges to
\begin{align*}
\int_0^t\mathcal{H}_{\Phi(s,\cdot)}[\rho(s,\cdot)+\bar{\delta}(s,\cdot),\rho(s,\cdot)+\bar{\delta}(s,\cdot)]\,\mathrm{d}s.
\end{align*}
by the following lemma, which has been established in \cite{Del91,Sch95,Mio19}.
\begin{Lemma}\label{lem-convergence-quadratic-form}
Let $\mu_n^1,\mu_n^2$ be sequences in $L^{\infty}(\mathbb{R}_+,\mathcal{M}(\mathbb{R}^2))$ such that $\mu_n^1,\mu_n^2\to\mu^1,\mu^2$ weak-* in $L^{\infty}(\mathbb{R}_+,\mathcal{M}(\mathbb{R}^2))$, respectively. Assume $\mu_n^1,\mu_n^2$ are equicontinuous in time with values in some negative Sobolev space $W^{-s,q}(\mathbb{R}^2)$ and satisfy the non-concentration condition 
\begin{align*}
&\lim_{\epsilon\to 0^+}\sup_{0\le t\le T}\sup_{\vert x_0\vert\le R_1}\sup_n\mu_n^1(t,B(x_0,\epsilon R_2))=0\quad\forall T,R_1,R_2>0,\\
&\iint_{\vert x-y\vert\le R}H_{\Phi}(x,y)\mu_n^2(\mathrm{d}x)\mu_n^2(\mathrm{d}y)=0\quad\forall\Phi\in C_c^{\infty}(\mathbb{R}_+\times\mathbb{R}^2),\text{ for some }R>0.
\end{align*}

Then $\int_0^t\mathcal{H}_{\Phi(s)}[\mu_n^1(s)+\mu_n^2(s),\mu_n^1(s)+\mu_n^2(s)]\,\mathrm{d}s$ converges to $\int_0^t\mathcal{H}_{\Phi(s)}[\mu^1(s)+\mu^2(s),\mu^1(s)+\mu^2(s)]\,\mathrm{d}s$ for all $t\ge 0$.
\end{Lemma}

Let $\mu_n^1=\rho_{\varepsilon_n}$, $\mu_n^2=\bar{\delta}_{\varepsilon_n}$, then the time continuity holds by Proposition~\ref{prn-continuity-point-charge}. The non-concentration condition holds by \eqref{eq-estimate-non-concentration} and \eqref{eq-priori-positive-distance}. Hence the convergence stated above holds and the theorem is proved.

\begin{proof}[Proof of Lemma~\ref{lem-convergence-quadratic-form}]
It is proved directly by following the arguments between \cite[Lem.3.2]{Sch95} and \cite[Thm.3.3]{Sch95}.
\end{proof}

\section{Appendix}

\subsection{Proof of Proposition~\ref{prn-E0nu-to-E00}}

Let $\psi\in C_c^{\infty}(\mathbb{R}_{+},[0,1])$ such  that $\psi$ vanishes on  $[0,1]$ and  $\psi=1$ on $[2,+\infty)$ and set $\psi_{\epsilon}= \psi(\cdot/\epsilon)$, which converges to  $1$ almost  everywhere. Let $\Phi$ be a test function and define
\begin{align*}
&\Phi_{\epsilon}(t,x)=\Phi(t,x)\prod_{\alpha}\psi_{\epsilon}(\vert x-\xi_{\alpha}(t)\vert),\\
&\Phi_{\epsilon}^{\alpha}(t,x)=\Phi(t,x)\prod_{\beta:\beta\ne\alpha}\psi_{\epsilon}(\vert x-\xi_{\alpha}(t)\vert).
\end{align*}
Then it can be proved without difficulty by the fact that the estimate \eqref{eq-priori-positive-distance} holds for $\{\xi_{\alpha}\}$: for $\epsilon$ small enough, we have
\begin{align*}
&\Phi_{\epsilon}(t,\xi_{\alpha}(t))=0, \partial_{t}\Phi_{\epsilon}(t,\xi_{\alpha}(t))=0, \nabla\Phi_{\epsilon}(t,\xi_{\alpha}(t))=0\text{ for all }1\le\alpha\le N,\\ 
&\Phi_{\epsilon}^{\alpha}(t,\xi_{\beta}(t))=0, \partial_{t}\Phi_{\epsilon}^{\alpha}(t,\xi_{\beta}(t))=0, \nabla\Phi_{\epsilon}^{\alpha}(t,\xi_{\beta}(t))=0\text{ for all }1\le\beta\ne\alpha\le N.
\end{align*}

%Indeed
%\begin{equation*}
%\nabla\Phi_{\epsilon}(t,x)=\nabla\Phi(t,x)\prod_{\alpha}\psi_{\epsilon}(\vert x-\xi_{\alpha}(t)\vert)+\Phi(t,x)\nabla\Big(\prod_{\alpha}\psi_{\epsilon}(\vert x-\xi_{\alpha}(t)\vert))\Big),
%\end{equation*}
%where
%\begin{align*}
%\nabla\Big(\prod_{\alpha}\psi_{\epsilon}(\vert x-\xi_{\alpha}(t)\vert)\Big)
%&=\frac{1}{\epsilon}\sum_{\alpha}\Big[\frac{x-\xi_{\alpha}(t)}{\vert x-\xi_{\alpha}(t)\vert}\psi_{\epsilon}'(\vert x-\xi_{\alpha}(t)\vert)\prod_{\beta:\beta\ne\alpha}\psi_{\epsilon}(\vert x-\xi_{\beta}(t)\vert)\Big]\\
%&\le\frac{1}{\epsilon}\sum_{\alpha}\Big[\frac{x-\xi_{\alpha}(t)}{\vert x-\xi_{\alpha}(t)\vert}\psi_{\epsilon}'(\vert x-\xi_{\alpha}(t)\vert)\Big].
%\end{align*}
%Since the estimate \eqref{eq-priori-positive-distance} holds for $\{\xi_{\alpha}\}$, we have $\vert\xi_{\alpha}(t)-\xi_{\beta}(t)\vert\ge M>2\epsilon$ for $\epsilon$ small enough, hence $\nabla\Phi_{\epsilon}(t,\xi_{\alpha}(t))=0$.

Notice $E\in L^{\infty}_{\rm{loc}}(L^{\infty})$ and $\frac{\rho}{\vert x-\xi_{\alpha}\vert}\in L^{1}_{\rm{loc}}$ by the assumption $\rho\in L^{\infty}_{\rm{loc}}L^{p}$ for $p>2$. Hence take $\Phi_{\epsilon}$ as test functions in \eqref{eq-weak-solution-mv-Euler-defect} with $\nu=0$, we obtain
\begin{align*}
&\frac{\mathrm{d}}{\mathrm{d}t}\int_{\mathbb{R}^2}\Phi_{\epsilon}(t,x)\rho(t,x)\,\mathrm{d}x
\\
&=\int_{\mathbb{R}^2}\partial_{t}\Phi_{\epsilon}(t,x)\rho(t,x)\,\mathrm{d}x+\int_{\mathbb{R}^2}\left(E^{\perp}(t,x)+\sum_{\alpha}\frac{(x-\xi_{\alpha})^{\perp}}{\vert x-\xi_{\alpha}\vert^2}\right)\cdot\nabla\Phi_{\epsilon}(t,x)\rho(t,x)\,\mathrm{d}x.
\end{align*}
We claim that the equation above converges to the first equation in \eqref{eq-Vortex-Wave} as $\epsilon\to0$.



A similar routine can prove that
\begin{align*}
&\frac{\mathrm{d}}{\mathrm{d}t}\int_{\mathbb{R}^2}\Phi_{\epsilon}^{\alpha}(t,x)\rho(t,x)\,\mathrm{d}x+\frac{\mathrm{d}}{\mathrm{d}t}\Phi_{\epsilon}^{\alpha}(t,\xi_{\alpha}(t))
\\
&=\int_{\mathbb{R}^2}\partial_{t}\Phi_{\epsilon}^{\alpha}(t,x)\rho(t,x)\,\mathrm{d}x+\partial_{t}\Phi_{\epsilon}^{\alpha}(t,\xi_{\alpha}(t))+\int_{\mathbb{R}^2}\left(E^{\perp}(t,x)+\sum_{\alpha}\frac{(x-\xi_{\alpha})^{\perp}}{\vert x-\xi_{\alpha}\vert^2}\right)\cdot\nabla\Phi_{\epsilon}^{\alpha}(t,x)\rho(t,x)\,\mathrm{d}x\\
&\quad+E^{\perp}(t,\xi_{\alpha}(t))\cdot\nabla\Phi_{\epsilon}^{\alpha}(t,\xi_{\alpha}(t))+\sum_{\alpha\ne\beta}\frac{(\xi_{\alpha}-\xi_{\beta})^{\perp}}{\vert\xi_{\alpha}-\xi_{\beta}\vert^2}\cdot\nabla\Phi_{\epsilon}^{\alpha}(t,\xi_{\alpha}(t))
\end{align*}
converges to
\begin{align*}
&\frac{\mathrm{d}}{\mathrm{d}t}\int_{\mathbb{R}^2}\Phi(t,x)\rho(t,x)\,\mathrm{d}x+\frac{\mathrm{d}}{\mathrm{d}t}\Phi(t,\xi_{\alpha}(t))
\\
&=\int_{\mathbb{R}^2}\partial_{t}\Phi(t,x)\rho(t,x)\,\mathrm{d}x+\partial_{t}\Phi(t,\xi_{\alpha}(t))+\int_{\mathbb{R}^2}\left(E^{\perp}(t,x)+\sum_{\alpha}\frac{(x-\xi_{\alpha})^{\perp}}{\vert x-\xi_{\alpha}\vert^2}\right)\cdot\nabla\Phi(t,x)\rho(t,x)\,\mathrm{d}x\\
&\quad+E^{\perp}(t,\xi_{\alpha}(t))\cdot\nabla\Phi(t,\xi_{\alpha}(t))+\sum_{\alpha\ne\beta}\frac{(\xi_{\alpha}-\xi_{\beta})^{\perp}}{\vert\xi_{\alpha}-\xi_{\beta}\vert^2}\cdot\nabla\Phi(t,\xi_{\alpha}(t)),
\end{align*}
which minus the first equation in \eqref{eq-Vortex-Wave} to obtain the second equation in \eqref{eq-Vortex-Wave}.

{\bf Proof of the claim.} By Lebesgue's dominated convergence theorem, $\int\Phi_{\epsilon}(t,x)\rho(t,x)\,\mathrm{d}x$ tends to  $\int\Phi(t,x)\rho(t,x)\,\mathrm{d}x$ as $\epsilon\to 0$.

By direct computation, we have
\begin{align*}
\int_{\mathbb{R}^2}\partial_{t}\Phi_{\epsilon}(t,x)\rho(t,x)\,\mathrm{d}x&=\int_{\mathbb{R}^2}\partial_{t}\Phi(t,x)\prod_{\alpha}\psi_{\epsilon}(\vert x-\xi_{\alpha}\vert)\rho(t,x)\,\mathrm{d}x\\
&\quad+\int_{\mathbb{R}^2}\frac{\Phi(t,x)}{\epsilon}\sum_{\alpha}\Big[\frac{\xi_{\alpha}-x}{\vert \xi_{\alpha}-x\vert}\cdot\dot{\xi}_{\alpha}\psi'(\vert x-\xi_{\alpha}\vert/\epsilon)\prod_{\beta:\beta\ne\alpha}\psi_{\epsilon}(\vert x-\xi_{\beta}\vert)\Big]\rho(t,x)\,\mathrm{d}x\\
&=:I_{\epsilon}^1+I_{\epsilon}^2.
\end{align*}
It is obvious that $I_{\epsilon}^1$ converges to $\int\partial_{t}\Phi(t,x)\rho(t,x)\,\mathrm{d}x$  as $\epsilon\to 0$. For $I_{\epsilon}^2$, since by the assumption $\|\dot{\xi}_{\alpha}\|_{\infty}\le C$, we have by the H\"older inequality
\begin{equation*}
\vert I_{\epsilon}^2\vert\le\frac{C}{\epsilon}\sum_{\alpha}\int_{\vert x-\xi_{\alpha}(t)\vert\le2\epsilon} \vert \rho(t,x)\vert\,\mathrm{d}x\le\frac{C}{\epsilon}\|\rho(t)\|_{p}\epsilon^{2-\frac{2}{p}}.
\end{equation*}

Now we prove the convergence of the nonlinear term. Notice
\begin{align*}
&\int_{\mathbb{R}^2}\Big(E^{\perp}(t,x)+\sum_{\alpha}\frac{(x-\xi_{\alpha}(t))^{\perp}}{\vert x-\xi_{\alpha}(t)\vert^{2}}\Big)\cdot\nabla\Phi_{\epsilon}(t,x)\rho(t,x)\,\mathrm{d}x\\
&=\int_{\mathbb{R}^2}\Big(E^{\perp}(t,x)+\sum_{\alpha}\frac{(x-\xi_{\alpha}(t))^{\perp}}{|x-\xi_{\alpha}(t)|^{2}}\Big)\cdot\nabla\Phi(t,x)\prod_{\alpha}\psi_{\epsilon}(\vert x-\xi_{\alpha}(t)\vert)\rho(t,x)\,\mathrm{d}x\\
&+\int_{\mathbb{R}^2}E^{\perp}(t,x)\cdot\nabla\Big(\prod_{\alpha}\psi_{\epsilon}(\vert x-\xi_{\alpha}(t)\vert)\Big)\Phi(t,x)\rho(t,x)\,\mathrm{d}x\\
&+\int_{\mathbb{R}^2}\Big(\sum_{\alpha}\frac{(x-\xi_{\alpha}(t))^{\perp}}{\vert x-\xi_{\alpha}(t)\vert^{2}}\Big)\cdot\nabla\Big(\prod_{\alpha}\psi_{\epsilon}(\vert x-\xi_{\alpha}(t)\vert)\Big)\Phi(t,x)\rho(t,x)\,\mathrm{d}x\\
&=:I_{\epsilon}^3+I_{\epsilon}^4+I_{\epsilon}^5.
\end{align*}
On the one hand, the dominated convergence  theorem implies that  $I_{\epsilon}^3$ converges to
\begin{align*}
\int_{\mathbb{R}^2}\Big(E^{\perp}(t,x)+\sum_{\alpha}\frac{(x-\xi_{\alpha}(t))^{\perp}}{\vert x-\xi_{\alpha}(t)\vert^{2}}\Big)\cdot\nabla\Phi(t,x)\rho(t,x)\,\mathrm{d}x
\end{align*}
as $\epsilon\to 0$. On the other hand, we have by  H\"older inequality
\begin{align*}
\vert I_{\epsilon}^4\vert
&\le\frac{C}{\epsilon}\sum_{\alpha}\int_{\vert x-\xi_{\alpha}(t)\vert\le2\epsilon}\vert E(t,x)\vert \vert\rho(t,x)\vert\,\mathrm{d}x\\
&\le\frac{C}{\epsilon}\|E(t)\|_{\infty}\|\rho(t)\|_{p}\epsilon^{2-\frac{2}{p}}.
\end{align*}

Notice by the identity $a^{\perp}\cdot a =0$, we have
\begin{align*}
I_{\epsilon}^5=\int_{\mathbb{R}^2}\frac{1}{\epsilon}\sum_{\alpha\ne\beta}\Big[\frac{(x-\xi_{\alpha})^{\perp}}{\vert x-\xi_{\alpha}\vert^{2}}\cdot\frac{x-\xi_{\beta}}{\vert x-\xi_{\beta}\vert}\psi'(\vert x-\xi_{\beta}\vert/\epsilon)\prod_{\gamma:\gamma\ne\beta}\psi_{\epsilon}(\vert x-\xi_{\gamma}\vert)\Big]\Phi\rho(t,x)\,\mathrm{d}x.
\end{align*}
When $\vert x-\xi_{\beta}(t)\vert\le2\epsilon$, from $\vert\xi_{\alpha}(t)-\xi_{\beta}(t)\vert\ge M$, we have $\vert x-\xi_{\alpha}(t)\vert\ge M-2\epsilon$ for $\alpha\ne\beta$, therefore we have
\begin{align*}
\vert I_{\epsilon}^5\vert&\le\frac{C}{\epsilon}\sum_{\beta}\int_{\vert x-\xi_{\beta}(t)\vert\le2\epsilon} \vert \rho(t,x)\vert\,\mathrm{d}x\\
&\le\frac{C}{\epsilon}\|\rho(t)\|_{p}\epsilon^{2-\frac{2}{p}}.
\end{align*}
Since $1-2/p>0$, we have that $I_{\epsilon}^i$ with $i=2,4,5$ vanish in the limit $\epsilon\to 0$. Therefore, we have proved that $\rho$ satisfies the first equation of \eqref{eq-Vortex-Wave} in the sense of distributions.


\subsection{Proof of Proposition~\ref{prn-estimate-Lk}}
Firstly, we prove two estimates along the characteristic.
\begin{Lemma}\label{lem-estimate-character}
Assume $\vert X_{\varepsilon}(s)-\xi_{\alpha,\varepsilon}(s)\vert\le\frac{M}{2}$. Then we have
\begin{align*}
\frac{1}{\vert X_{\varepsilon}(s)-\xi_{\alpha,\varepsilon}(s)\vert}&\le\varepsilon^2\frac{\mathrm{d}^{2}}{\mathrm{d}s^{2}}\vert X_{\varepsilon}(s)-\xi_{\alpha,\varepsilon}(s)\vert+\varepsilon^{-1}\vert V_{\varepsilon}(s)-\eta_{\alpha,\varepsilon}(s)\vert\\
&\quad+\vert E_{\varepsilon}(s,X_{\varepsilon}(s))\vert+\vert E_{\varepsilon}(s,\xi_{\alpha,\varepsilon}(s))\vert+\frac{3N}{M}.
\end{align*}
\end{Lemma}
\begin{proof}
By direct calculation, we have
\begin{equation*}
\frac{\mathrm{d}}{\mathrm{d}s}\vert X_{\varepsilon}(s)-\xi_{\alpha,\varepsilon}(s)\vert=\frac{X_{\varepsilon}(s)-\xi_{\alpha,\varepsilon}(s)}{\vert X_{\varepsilon}(s)-\xi_{\alpha,\varepsilon}(s)\vert}\cdot\frac{V_{\varepsilon}(s)-\eta_{\alpha,\varepsilon}(s)}{\varepsilon},
\end{equation*}
and
\begin{align*}
\frac{\mathrm{d}^{2}}{\mathrm{d}s^{2}}\vert X_{\varepsilon}(s)-\xi_{\alpha,\varepsilon}(s)\vert&=\frac{\vert V_{\varepsilon}(s)-\eta_{\alpha,\varepsilon}(s)\vert^{2}}{\varepsilon^{2}\vert X_{\varepsilon}(s)-\xi_{\alpha,\varepsilon}(s)\vert}-\frac{[( X_{\varepsilon}(s)-\xi_{\alpha,\varepsilon}(s))\cdot(V_{\varepsilon}(s)-\eta_{\alpha,\varepsilon}(s))]^{2}}{\varepsilon^{2}\vert X_{\varepsilon}(s)-\xi_{\alpha,\varepsilon}(s)\vert^{3}}\nonumber\\
&\quad+\frac{ X_{\varepsilon}(s)-\xi_{\alpha,\varepsilon}(s)}{\varepsilon^2\vert X_{\varepsilon}(s)-\xi_{\alpha,\varepsilon}(s)\vert}\cdot\Big[\frac{V_{\varepsilon}^{\perp}(s)-\eta_{\alpha,\varepsilon}^{\perp}(s)}{\varepsilon}+(E_{\varepsilon}+F_{\varepsilon})(s,X_{\varepsilon}(s))\nonumber\\
&\qquad-E_{\varepsilon}(s,\xi_{\alpha,\varepsilon}(s))-\sum_{\beta:\beta\ne\alpha}\frac{\xi_{\alpha,\varepsilon}(s)-\xi_{\beta,\varepsilon}(s)}{\vert\xi_{\alpha,\varepsilon}(s)-\xi_{\beta,\varepsilon}(s)\vert^2}\Big]\nonumber\\
&\ge-\varepsilon^{-3}\vert V_{\varepsilon}(s)-\eta_{\alpha,\varepsilon}(s)\vert-\varepsilon^{-2}\vert E_{\varepsilon}(s,X_{\varepsilon}(s))\vert-\varepsilon^{-2}\vert E_{\varepsilon}(s,\xi_{\alpha,\varepsilon}(s))\vert\nonumber\\
&\quad-\sum_{\beta:\beta\ne\alpha}\frac{\varepsilon^{-2}}{\vert\xi_{\alpha,\varepsilon}(s)-\xi_{\beta,\varepsilon}(s)\vert}+\varepsilon^{-2}\frac{X_{\varepsilon}(s)-\xi_{\alpha,\varepsilon}(s)}{\vert X_{\varepsilon}(s)-\xi_{\alpha,\varepsilon}(s)\vert}\cdot F_{\varepsilon}(s,X_{\varepsilon}(s)),
\end{align*}
By Proposition~\ref{prn-priori}, we have
\begin{align*}
&\frac{X_{\varepsilon}(s)-\xi_{\alpha,\varepsilon}(s)}{\vert X_{\varepsilon}(s)-\xi_{\alpha,\varepsilon}(s)\vert}\cdot F_{\varepsilon}(s,X_{\varepsilon}(s))\\
&\le\varepsilon^2\frac{\mathrm{d}^{2}}{\mathrm{d}s^{2}}\vert X_{\varepsilon}(s)-\xi_{\alpha,\varepsilon}(s)\vert+\varepsilon^{-1}\vert V_{\varepsilon}(s)-\eta_{\alpha,\varepsilon}(s)\vert+\vert E_{\varepsilon}(s,X_{\varepsilon}(s))\vert+\vert E_{\varepsilon}(s,\xi_{\alpha,\varepsilon}(s))\vert+\frac{N}{M},
\end{align*}
which implies
\begin{align*}
\frac{1}{\vert X_{\varepsilon}(s)-\xi_{\alpha,\varepsilon}(s)\vert}&\le\sum_{\beta:\beta\ne\alpha}\frac{1}{\vert X_{\varepsilon}(s)-\xi_{\beta,\varepsilon}(s)\vert}+\varepsilon^2\frac{\mathrm{d}^{2}}{\mathrm{d}s^{2}}\vert X_{\varepsilon}(s)-\xi_{\alpha,\varepsilon}(s)\vert+\varepsilon^{-1}\vert V_{\varepsilon}(s)-\eta_{\alpha,\varepsilon}(s)\vert\\
&\quad+\vert E_{\varepsilon}(s,X_{\varepsilon}(s))\vert+\vert E_{\varepsilon}(s,\xi_{\alpha,\varepsilon}(s))\vert+\frac{N}{M}.
\end{align*}
By \eqref{eq-priori-positive-distance} and the assumption in the lemma, we have $\vert X_{\varepsilon}(s)-\xi_{\beta,\varepsilon}(s)\vert\ge M/2$ if $\beta\ne\alpha$, then the lemma follows.
\end{proof}
\begin{Lemma}\label{lem-estimate-pointwise-energy}
We have
\begin{align*}
\Big\vert\frac{\mathrm{d}}{\mathrm{d}s}\mathbf{h}_{\varepsilon}(s)\Big\vert
&\le C\varepsilon^{-1}\left(\sqrt{\mathbf{h}_{\varepsilon}(s)}(\vert E_{\varepsilon}(s,X_{\varepsilon}(s))\vert+1)+\sum_{\alpha}\frac{1}{\vert X_{\varepsilon}-\xi_{\alpha,\varepsilon}\vert}\right).
\end{align*}
\end{Lemma}
\begin{proof}
Differentiating $\mathbf{h}_{\varepsilon}(s)$ to obtain
\begin{align*}
\frac{\mathrm{d}}{\mathrm{d}s}\mathbf{h}_{\varepsilon}(s)&=\varepsilon^{-1}V_{\varepsilon}\cdot(E_{\varepsilon}+F_{\varepsilon})(s,X_{\varepsilon})+\sum_{\alpha}\left(\frac{X_{\varepsilon}-\xi_{\alpha,\varepsilon}}{\vert X_{\varepsilon}-\xi_{\alpha,\varepsilon}\vert}-\frac{X_{\varepsilon}-\xi_{\alpha,\varepsilon}}{\vert X_{\varepsilon}-\xi_{\alpha,\varepsilon}\vert^2}\right)\cdot\frac{V_{\varepsilon}-\eta_{\alpha,\varepsilon}}{\varepsilon}\\
&=\varepsilon^{-1}V_{\varepsilon}\cdot E_{\varepsilon}(s,X_{\varepsilon})+\sum_{\alpha}\left(\frac{X_{\varepsilon}-\xi_{\alpha,\varepsilon}}{\vert X_{\varepsilon}-\xi_{\alpha,\varepsilon}\vert}\cdot\frac{V_{\varepsilon}-\eta_{\alpha,\varepsilon}}{\varepsilon}+\frac{X_{\varepsilon}-\xi_{\alpha,\varepsilon}}{\vert X_{\varepsilon}-\xi_{\alpha,\varepsilon}\vert^2}\cdot\frac{\eta_{\alpha,\varepsilon}}{\varepsilon}\right).
\end{align*}
The lemma follows from the fact $\vert\eta_{\alpha,\varepsilon}(s)\vert\le C$ by \eqref{eq-priori-1} with $C$ depending only on $K_0,K_1$.
\end{proof}


\begin{proof}[Proof of Proposition~\ref{prn-estimate-Lk}]
We split the integral domains in $L_{k,\varepsilon}(t)$ into two parts to obtain
\begin{align*}
&\int_0^t\iint_{\mathbb{R}^2\times\mathbb{R}^2}\frac{\mathbf{h}_{\varepsilon}(s)^{k/2}f_{\varepsilon}^0}{\vert X_{\varepsilon}(s)-\xi_{\alpha,\varepsilon}(s)\vert}\,\mathrm{d}x\,\mathrm{d}v\,\mathrm{d}s\\
&\le\int_0^t\iint_{\vert X_{\varepsilon}(s)-\xi_{\alpha,\varepsilon}(s)\vert\le\frac{M}{2}}+\int_0^t\iint_{\vert X_{\varepsilon}(s)-\xi_{\alpha,\varepsilon}(s)\vert>\frac{M}{2}}\frac{\mathbf{h}_{\varepsilon}(s)^{k/2}f_{\varepsilon}^0}{\vert X_{\varepsilon}(s)-\xi_{\alpha,\varepsilon}(s)\vert}\,\mathrm{d}x\,\mathrm{d}v\,\mathrm{d}s\\
&\le\int_0^t\iint_{\vert X_{\varepsilon}(s)-\xi_{\alpha,\varepsilon}(s)\vert\le\frac{M}{2}}\frac{\mathbf{h}_{\varepsilon}(s)^{k/2}f_{\varepsilon}^0}{\vert X_{\varepsilon}(s)-\xi_{\alpha,\varepsilon}(s)\vert}\,\mathrm{d}x\,\mathrm{d}v\,\mathrm{d}s+\frac{2}{M}H_{k,\varepsilon}(t)t.
\end{align*}


By Lemma~\ref{lem-estimate-character}, we have
\begin{align*}
&\int_0^t\iint_{\vert X_{\varepsilon}(s)-\xi_{\alpha,\varepsilon}(s)\vert\le\frac{M}{2}}\frac{\mathbf{h}_{\varepsilon}(s)^{k/2}f_{\varepsilon}^0}{\vert X_{\varepsilon}(s)-\xi_{\alpha,\varepsilon}(s)\vert}\,\mathrm{d}x\,\mathrm{d}v\,\mathrm{d}s\\
&\le\int_0^t\iint_{\mathbb{R}^2\times\mathbb{R}^2}\mathbf{h}_{\varepsilon}(s)^{k/2}f_{\varepsilon}^0\Big(\varepsilon^2\frac{\mathrm{d}^{2}}{\mathrm{d}s^{2}}\vert X_{\varepsilon}(s)-\xi_{\alpha,\varepsilon}(s)\vert+\varepsilon^{-1}\vert V_{\varepsilon}(s)-\eta_{\alpha,\varepsilon}(s)\vert\\
&\qquad\qquad\qquad\qquad\qquad+\vert E_{\varepsilon}(s,X_{\varepsilon}(s))\vert+\vert E_{\varepsilon}(s,\xi_{\alpha,\varepsilon}(s))\vert+\frac{3N}{M}\Big)\,\mathrm{d}x\,\mathrm{d}v\,\mathrm{d}s\\
&=:I_{\varepsilon}^1(t)+I_{\varepsilon}^2(t)+I_{\varepsilon}^3(t)+I_{\varepsilon}^4(t)+\frac{3N}{M}H_{k,\varepsilon}(t)t.
\end{align*}
Now we estimate each term above. 

{\bf Estimate on $I_{\varepsilon}^1(t)$:} Firstly, using integration by parts in time, we obtain
\begin{align*}
I_{\varepsilon}^1(s)&=\varepsilon^2\iint_{\mathbb{R}^2\times\mathbb{R}^2} f_{\varepsilon}^0\Big(\int_0^t\mathbf{h}_{\varepsilon}(s)^{k/2}\frac{\mathrm{d}^2}{\mathrm{d}s^2}\vert X_{\varepsilon}(s)-\xi_{\alpha,\varepsilon}(s)\vert \,\mathrm{d}s\Big)\,\mathrm{d}x\,\mathrm{d}v\\
&=\varepsilon^2\iint_{\mathbb{R}^2\times\mathbb{R}^2} f_{\varepsilon}^0\Big\{\Big[\mathbf{h}_{\varepsilon}(s)^{k/2}\frac{\mathrm{d}}{\mathrm{d}s}\vert X_{\varepsilon}(s)-\xi_{\alpha,\varepsilon}(s)\vert \Big]\Big\vert^t_0\\
&\qquad\qquad\qquad-\int_0^t\frac{\mathrm{d}}{\mathrm{d}s}\mathbf{h}_{\varepsilon}(s)^{k/2}\frac{\mathrm{d}}{\mathrm{d}s}\vert X_{\varepsilon}(s)-\xi_{\alpha,\varepsilon}(s)\vert \,\mathrm{d}s\Big\}\,\mathrm{d}x\,\mathrm{d}v,
\end{align*}
by direct calculation, we have
\begin{equation*}
\frac{\mathrm{d}}{\mathrm{d}s}\vert X_{\varepsilon}(s)-\xi_{\alpha,\varepsilon}(s)\vert=\frac{X_{\varepsilon}(s)-\xi_{\alpha,\varepsilon}(s)}{\vert X_{\varepsilon}(s)-\xi_{\alpha,\varepsilon}(s)\vert }\cdot\frac{V_{\varepsilon}(s)-\eta_{\alpha,\varepsilon}(s)}{\varepsilon}\le C\varepsilon^{-1}\mathbf{h}_{\varepsilon}(s)^{1/2},
\end{equation*}
combining with Lemma~\ref{lem-estimate-pointwise-energy}, we have
\begin{equation*}
\Big[\mathbf{h}_{\varepsilon}(s)^{k/2}\frac{\mathrm{d}}{\mathrm{d}s}\vert X_{\varepsilon}(s)-\xi_{\alpha,\varepsilon}(s)\vert
\Big]\Big\vert^t_0\le C\varepsilon^{-1}(\mathbf{h}_{\varepsilon}(t)^{(k+1)/2}+\mathbf{h}_{\varepsilon}(0)^{(k+1)/2})
\end{equation*}
and
\begin{align*}
&\int_0^t\frac{\mathrm{d}}{\mathrm{d}s}\mathbf{h}_{\varepsilon}(s)^{k/2}\frac{\mathrm{d}}{\mathrm{d}s}\vert X_{\varepsilon}(s)-\xi_{\alpha,\varepsilon}(s)\vert \,\mathrm{d}s\\
&\le C\varepsilon^{-2}\int_0^t\mathbf{h}_{\varepsilon}(s)^{\frac{k-1}{2}}
\left(\sqrt{\mathbf{h}_{\varepsilon}(s)}\vert E_{\varepsilon}(s,X_{\varepsilon}(s))\vert+\sum_{\alpha}\frac{1}{\vert X_{\varepsilon}(s)-\xi_{\alpha,\varepsilon}(s)\vert}\right)\,\mathrm{d}s\\
&\le C\varepsilon^{-2}\int_0^t\mathbf{h}_{\varepsilon}(s)^{\frac{k}{2}}\vert E_{\varepsilon}(s,X_{\varepsilon}(s))\vert+\sum_{\alpha}\frac{\mathbf{h}_{\varepsilon}(s)^{\frac{k-1}{2}}
}{\vert X_{\varepsilon}(s)-\xi_{\alpha,\varepsilon}(s)\vert}\,\mathrm{d}s.
\end{align*}
Hence we have
\begin{align*}
I_{\varepsilon}^1(s)&\le C\varepsilon\iint_{\mathbb{R}^2\times\mathbb{R}^2} f_{\varepsilon}^0\Big(\mathbf{h}_{\varepsilon}(t)^{(k+1)/2}+\mathbf{h}_{\varepsilon}(0)^{(k+1)/2}\Big)\,\mathrm{d}x\,\mathrm{d}v\\
&\quad+C\iint_{\mathbb{R}^2\times\mathbb{R}^2} f_{\varepsilon}^0\left\{\int_0^t\mathbf{h}_{\varepsilon}(s)^{\frac{k}{2}}\vert E_{\varepsilon}(s,X_{\varepsilon}(s))\vert+\sum_{\alpha}\frac{\mathbf{h}_{\varepsilon}(s)^{\frac{k-1}{2}}}{\vert X_{\varepsilon}(s)-\xi_{\alpha,\varepsilon}(s)\vert}\,\mathrm{d}s\right\}\,\mathrm{d}x\,\mathrm{d}v\\
&\le C\Big(\varepsilon H_{k+1,\varepsilon}(t)+\int_0^t\iint_{\mathbb{R}^2\times\mathbb{R}^2} \mathbf{h}(s)^{k/2}f_{\varepsilon}^0\vert E_{\varepsilon}(s,X_{\varepsilon}(s))\vert \,\mathrm{d}x\,\mathrm{d}v\,\mathrm{d}s\\
&\quad+\sum_{\alpha}\int_0^t\iint_{\mathbb{R}^2\times\mathbb{R}^2}\frac{\mathbf{h}_{\varepsilon}(s)^{(k-1)/2}f_{\varepsilon}^0}{\vert X_{\varepsilon}(s)-\xi_{\alpha,\varepsilon}(s)\vert}\,\mathrm{d}x\,\mathrm{d}v\,\mathrm{d}s\Big)\\
&\le C\left(\varepsilon H_{k+1,\varepsilon}(t)+\|f_{\varepsilon}^0\|_{\infty}^{\frac{l-k}{l+2}}H_{l,\varepsilon}(t)^{\frac{k+2}{l+2}}\int_0^t\|E_{\varepsilon}(s)\|_{\frac{l+2}{l-k}}\,\mathrm{d}s+L_{k-1,\varepsilon}(t)\right),
\end{align*}
where we have used the H\"older inequality and Lemma~\ref{lem-interpo-Hk}.


{\bf Estimate on $I_{\varepsilon}^2(t)$:} By definition, we have $\vert V_{\varepsilon}(s)-\eta_{\alpha,\varepsilon}(s)\vert\le C\sqrt{\mathbf{h}_{\varepsilon}(s)}$. Hence
\begin{equation*}
I_{\varepsilon}^2=\varepsilon^{-1}\int_0^t\iint_{\mathbb{R}^2\times\mathbb{R}^2} \mathbf{h}_{\varepsilon}(s)^{k/2}f_{\varepsilon}^0\vert V_{\varepsilon}(s)-\eta_{\alpha,\varepsilon}(s)\vert\,\mathrm{d}x\,\mathrm{d}v\,\mathrm{d}s\le C\varepsilon^{-1}H_{k+1,\varepsilon}(t)t.
\end{equation*}

{\bf Estimate on $I_{\varepsilon}^3(s)$:} By  H\"older inequality and Lemma~\ref{lem-interpo-Hk} we have
\begin{align*}
I_{\varepsilon}^3&=\int_0^t\iint_{\mathbb{R}^2\times\mathbb{R}^2} \mathbf{h}_{\varepsilon}(s)^{k/2}f_{\varepsilon}^0\vert E_{\varepsilon}(s,X_{\varepsilon}(s))\vert\,\mathrm{d}x\,\mathrm{d}v\,\mathrm{d}s\\
&\le\int_0^t\Big\|\int_{\mathbb{R}^2}h_{\varepsilon}^{k/2}f_{\varepsilon}(t,x,v)\,\mathrm{d}v\Big\|_{\frac{l+2}{k+2}}\|E_{\varepsilon}(s)\|_{\frac{l+2}{l-k}}\,\mathrm{d}s\\
&\le C\|f_{\varepsilon}^0\|_{\infty}^{\frac{l-k}{l+2}}H_{l,\varepsilon}(t)^{\frac{k+2}{l+2}}\int_0^t\|E_{\varepsilon}(s)\|_{\frac{l+2}{l-k}}\,\mathrm{d}s.
\end{align*}

{\bf Estimate on $I_{\varepsilon}^4(t)$:}  We have
\begin{align*}
I_{\varepsilon}^4&=\int_0^t\iint_{\mathbb{R}^2\times\mathbb{R}^2} \mathbf{h}_{\varepsilon}(s)^{k/2}f_{\varepsilon}^0\vert E_{\varepsilon}(s,\xi_{\alpha,\varepsilon}(s))\vert\,\mathrm{d}x\,\mathrm{d}v\,\mathrm{d}s\\
&\le CH_{k,\varepsilon}(t)\int_0^t\vert E_{\varepsilon}(s,\xi_{\alpha,\varepsilon}(s))\vert \,\mathrm{d}s\le CH_{k,\varepsilon}(t)L_{0,\varepsilon}(t).
\end{align*}

Combining the estimates on $I_{\varepsilon}^i(t)$ for $i=1,2,3,4$, we have
\begin{align*}
&\int_0^t\iint_{\mathbb{R}^2\times\mathbb{R}^2}\frac{\mathbf{h}_{\varepsilon}(s)^{k/2}f_{\varepsilon}^0}{\vert X_{\varepsilon}(s)-\xi_{\alpha,\varepsilon}(s)\vert}\,\mathrm{d}x\,\mathrm{d}v\,\mathrm{d}s\\
&\le C\left(\varepsilon H_{k+1,\varepsilon}(t)+\|f_{\varepsilon}^0\|_{\infty}^{\frac{l-k}{l+2}}H_{l,\varepsilon}(t)^{\frac{k+2}{l+2}}\int_0^t\|E_{\varepsilon}(s)\|_{\frac{l+2}{l-k}}\,\mathrm{d}s+L_{k-1,\varepsilon}(t)\right)\\
&\qquad+C\varepsilon^{-1}H_{k+1,\varepsilon}(t)t+CH_{k,\varepsilon}(t)L_{0,\varepsilon}(t)+\frac{3N+2}{M}H_{k,\varepsilon}(t)t,
\end{align*}
the proof is complete.
\end{proof}



%\section*{Acknowledgements}

%\nocite{*}
\bibliographystyle{abbrv}
\bibliography{ref}
\end{document}









