%
\section{Introduction}
Reversible computing studies computations which can proceed both in
the standard, forward direction, and backward, going back to past
states. Reversible computation has attracted interest due to its
applications in areas as different as low-power computing~\cite{Landauer61},
simulation~\cite{CarothersPF99}, robotics~\cite{LaursenSE15},
biological modelling~\cite{CardelliL11,PhillipsUY12} and
debugging~\cite{microsoft,LaneseNPV18b}. 

There is widespread agreement in the literature about what properties
characterise reversible computation in the classical sequential (hence deterministic)
%  A main result of this line of research
%  is that the}
  a notion of reversibility %most
    suited for concurrent systems %is
    called
\emph{causal-consistent reversibility} (other notions were also used later on, e.g., to model biological systems~\cite{PhillipsUY12}). According to
an informal account of causal-consistent reversibility, any action can be undone provided that its 
consequences\footnote{By consequence we mean any subsequent transition which could not be permuted with $t$ while preserving the resulting state.}
%  done before $t$, \iu{ not sure about what follows} or if fired before would lead to a different outcome.}}, 
if any, are undone beforehand.
Following~\cite{DK04} this account is formalised using
% Causal-consistent reversibility equates computations which
% Most of the approaches above consider the so called causal-consistent
% reversibility~\cite{DK04,LMT14}, centred on
the notion of causal
equivalent traces: two traces are causal equivalent if and only if they only
differ for swapping independent actions, and inserting or removing
pairs of an action and its reverse. According to~\cite[Section~3]{DK04}
\begin{quote}
Backtracking an event is possible when and only when a causally
equivalent trace would have brought this event as the last one
\end{quote}
which is then formalised as the so called causal
consistency (CC)~\cite[Theorem 1]{DK04}, stating that coinitial computations are
causal equivalent if and only if they are cofinal.
Our new proof of CC (Proposition~\ref{prop:PL WF CC})
shows that it holds in essentially any reversible formalism satisfying the
Loop Lemma (roughly, any action can be undone) and the Parabolic Lemma (roughly, any computation is equivalent to a backward computation followed by a forward one),
and we believe that CC is insufficient on its own to capture the informal notion.

A formalisation closer to the informal statement above is provided
in~\cite[Corollary 22]{LaneseNPV18}, stating that a forward transition
$t$ can be undone after a derivation if and only if all the consequences of~$t$, if any,
are undone beforehand. We are not aware of other discussions trying to
formalise such a notion, except for~\cite{PU15}, in the
setting of reversible event structures.
In~\cite{PU15}, a reversible event structure is \emph{cause-respecting}
if an event cannot be reversed until all events it has caused have also been reversed;
it is \emph{causal} if it is cause-respecting and a reversible event can be reversed
if all events it has caused have been reversed~\cite[Definition~3.34]{PU15}.


We provide (Section~\ref{sec:CSCL}) a novel definition of the idea above, composed by:
\begin{description}
\item[Causal Safety (CS):]
an action cannot be reversed until any actions caused by it have been reversed;
\item[Causal Liveness (CL):]
we should allow actions to reverse in any order compatible with CS, not necessarily 
the exact inverse of the forward order.
\end{description}
We shall see that CC does not capture the same property as CS+CL
(Examples~\ref{ex:prerev not CSi},~\ref{ex:prerev not CL} and~\ref{ex:CS CL not CC}), and that there are slightly
different versions of CS and CL, which can all be proved under
a small set of reasonable assumptions.

The main aim of this paper is to take an abstract model, namely
labelled transition systems with independence equipped with reverse
transitions (Section~\ref{sec:LTSIs}), and to show that the properties
above (as well as others) can be derived from a small set of simple
axioms (Sections~\ref{sec:basic},~\ref{sec:events},~\ref{sec:CSCL} and~\ref{sec:coinitial}). This is in sharp contrast with the large part of
works in the literature, which consider specific frameworks such as
CCS~\cite{DK04}, CCS with broadcast~\cite{Mez18}, CCB~\cite{KU18},
$\pi$-calculus~\cite{CristescuKV13}, higher-order
$\pi$~\cite{LaneseMS16}, Klaim~\cite{GiachinoLMT17}, Petri
nets~\cite{MMU19}, $\mu$Oz~\cite{LienhardtLMS12} and
Erlang~\cite{LaneseNPV18}, and all give similar but formally unrelated
proofs of the same main results. Such proofs will become instances of
our general results.
%
%\todo{could add more examples - see our Concur 2019 introduction}
%
More precisely, our axioms will:
\begin{itemize}
\item
exclude behaviours which are not compatible with causal-consistent
reversibility (as we will discuss shortly);
\item
allow us to derive the main properties of reversible calculi which have been studied in the literature, 
such as CC (Proposition~\ref{prop:PL WF CC});
\item
hold for a number of reversible calculi which have been proposed, such
as RCCS~\cite{DK04} and reversible Erlang~\cite{LaneseNPV18} (Section~\ref{sec:casestudies}).
\end{itemize}
Thus, when defining a new reversible formalism, one just has to check
whether the axioms hold, and get for free the proofs of the most
relevant properties. Notably, the axioms are normally easier to prove
than the properties, hence the assessment of a reversible calculus
gets much simpler.

As a reference, Table~\ref{t:list} lists the axioms and properties used in this paper.
{\small 
\begin{table}[t!]
  \begin{center}
    \begin{tabular}{|c|c|c|c|c|} % <-- Alignments: 1st column left, 2nd middle and 3rd right, with vertical lines in between
      \hline
      \textbf{Acronym} & \textbf{Name} & \textbf{Defined in} & \textbf{Proved in} & \textbf{Using}\\
      \hline
      SP & Square Property & Def.~\ref{def:basic} & Axiom & -\\
      BTI & Backward Transitions are Independent & Def.~\ref{def:basic} & Axiom & - \\
      WF & Well-Founded & Def.~\ref{def:basic} & Axiom & -\\
      PCI & Propagation of Coinitial Independence & Def.~\ref{def:PCI} & Axiom & implied by LG or CLG \\ \hdashline
      IRE & Independence Respects Events & Def.~\ref{def:IRE} & Axiom & implied by LG \\
%      AC & Acyclic & & Axiom & -\\
%      PI & Propagation of Independence & Def.~\ref{def:pi} & Axiom & -\\
      CIRE & Coinitial Independence Respects Events & Def.~\ref{def:CIRE} & Axiom & implied by IRE or CLG \\
      BFCIRE & Backward-Forward CIRE & Def.~\ref{def:BFCIRE} & Axiom & implied by CIRE\\

      IEC & Independence of Events is Coinitial & Def.~\ref{def:IEC} & Axiom & - \\ \hdashline
      %      FD & Forward Diamond & & ? & -\\
      CLG & Coinitial Label-Generated & Def.~\ref{def:CLG} & Str.\ Ax. & - \\
      LG & Label-Generated & Def.~\ref{def:LG} & Str.\ Ax. & - \\
      IC & Independence is Coinitial & Def.~\ref{def:coinitial LTSI} & Str.\ Ax. & implied by CLG \\
      \hline
      PL & Parabolic Lemma & Def.~\ref{def:PL} & Prop.~\ref{prop:PL} & BTI, SP\\
      CC & Causal Consistency & Def.~\ref{def:cc} & Prop.~\ref{prop:PL WF CC} & WF, PL\\
      UT & Unique Transition & Def.~\ref{def:ut} & Cor.~\ref{cor:ut} & CC\\
      BLD & Backward Label Determinism & Def.~\ref{def:BD} & Prop.~\ref{prop:BD}  & SP, BTI, PCI\\ % SP, BTI, WF, PCI\\
      ID & Independence of Diamonds & Def.~\ref{def:ID} & Prop.~\ref{prop:ID} & BTI, PCI\\
      NRE & No Repeated Events & Def.~\ref{def:NRE} & Prop.~\ref{prop:NRE}  & Pre-rev.\\ %SP, BTI, WF, PCI\\
      RPI & Reversing Preserves Independence & Def.~\ref{def:rpi} & Prop.~\ref{prop:RPI}  & SP, PCI, IRE, IEC\\ % \todo{or LG} True but ...\\
      CS$\indt$ & Causal Safety & Def.~\ref{def:safe live} & Thm.~\ref{thm:CS} & Pre-rev., IRE\\ %SP, BTI, WF, PCI, IRE \\
      CL$\indt$ & Causal Liveness & Def.~\ref{def:safe live} & Thm.~\ref{thm:CL} & Pre-rev., IRE\\ %SP, BTI, WF, PCI, IRE \\
      ECh & Event Coherence & Def.~\ref{def:ECh} & Prop.~\ref{prop:ECh} & Pre-rev., (IRE or IEC) \\
      CS$\ci$ & coinitial Causal Safety & Def.~\ref{def:coind safe live} & Thm.~\ref{thm:CS coind} & Pre-rev.\\ %SP, BTI, WF, PCI \\
      CL$\ci$ & coinitial Causal Liveness & Def.~\ref{def:coind safe live} & Thm.~\ref{thm:CL coind} & Pre-rev., BFCIRE\\ %SP, BTI, WF, PCI, CIRE \\
      CS$_<$ & ordered Causal Safety & Def.~\ref{def:safe live <} & Prop.~\ref{prop:CS CL coind <} & Pre-rev.\\ %SP, BTI, WF, PCI\\
      CL$_<$ & ordered Causal Liveness & Def.~\ref{def:safe live <} & Prop.~\ref{prop:CS CL coind <} & Pre-rev., BFCIRE\\ %SP, BTI, WF, PCI, CIRE \\
%      \todo{FCL$_<$} & Forward ordered Causal Liveness & Def.~\ref{def:fwd live <} & Prop.~\ref{prop:FCL CL <} &  CL$_<$ \\
%      \todo{EIT} & Elimination of Inverse Transitions & Def.~\ref{def:EIT} & Prop.~\ref{prop:CLci EIT} & CL$\ci$ \\
%      RED & Reverse Event Determinism & Def.~\ref{def:RED} & Prop.~\ref{prop:NRE RED poly}  & SP, BTI, WF, PCI, NRE \\
%      EFP & Equivalent Forward-only Path & Def.~\ref{def:fcc efp} & Prop.~\ref{prop:EFP FCC CC} & CCI\\
%      FCC & Forward-only Causal Consistency & Def.~\ref{def:fcc efp} & Prop.~\ref{prop:EFP FCC CC} & EFP\\
%      ERT & Elimination of Reverse Transition & & & \\
%      RD & Reverse Diamond & Def.~\ref{def:RD} & Prop.~\ref{prop:RD} & BTI, (R)SP\\
      \hline
    \end{tabular}
  \end{center}
    \caption{Axioms and properties for causal reversibility.
`Str.\ Ax.' abbreviates `Structural Axiom' and `Pre-rev.'  abbreviates `Pre-reversible', namely SP, BTI, WF, PCI (cf.~Def.~\ref{def:prerev}). We call statements in the bottom part of the table, namely from PL to CL$_<$, properties. }
    \label{t:list}  
\end{table}
}

%% We show the relationships between our axioms and properties in Figure~\ref{fig:diagaxioms1}.
%% % Figure environment removed
%% Figure~\ref{fig:prerevdiag1} shows the axioms and properties which are implied by pre-reversibility, which is the most liberal setting where our constructions can be defined.
%% % Figure environment removed

In order to understand which kinds of behaviours are incompatible with
a causal-consistent reversible setting, consider the following CCS processes and their transitions as in Figure~\ref{fig:ccs1}:
% Figure environment removed

\begin{description}
\item[$a.\nil \tran a \nil$, $b.\nil \tran b \nil$:] from state $\nil$ one does not know whether to go back to $a.\nil$ or to $b.\nil$;
\item[$a.\nil + b.\nil \tran a \nil$, $a.\nil + b.\nil \tran b \nil$:] as above, but starting from the same process, hence showing that it is not enough to remember the initial configuration;
\item[$P \tran a P$ where $P \bydef a.P$:]  in state $P$ one does not know whether action $a$ has been performed, and, if 
it has been performed, then how many times. Due to this lack of information, one could go back an arbitrary number of times from 
$P$ to $P$.
%Such behaviour may be allowed in general but, because it goes against the idea that a state models a process reachable after a finite computation, we do not permit it.}
In this work, we do not permit an arbitrary number of backward moves, because it goes against the idea that a state models a process reachable after a finite computation. 



  
\Comment{
\iu{from state $P$ one can go back an arbitrary number of times to $P$}, which is against the idea that a state models a process reachable after a finite computation.
\todo{BLEND}
\il{in state $P$ one has no information about whether action $a$ has been performed, and in case how many times, hence one does not know whether it makes sense to undo $a$.}
}  
\end{description}
We remark that all such behaviours are perfectly reasonable in CCS,
and they are dealt with in the reversible setting by adding history information
about past actions. For example, in the first case one could remember the
initial state, in the second case both the initial state and the
action taken, and in the last case the number of iterations that have
been performed.

The paper is organised as follows. The next section introduces labelled transition systems with
independence (LTSIs). Three basic axioms for reversibility (SP, BTI and WF) are defined in Section~\ref{sec:basic}, 
and are used to prove the Parabolic Lemma and Causal Consistency.
Events are defined in Section~\ref{sec:events},
where another basic axiom (PCI) is formulated.
In Section~\ref{sec:CSCL} we discuss and define CS
and CL properties, and introduce three further basic axioms (IRE, CIRE and IEC)
that are used to prove them.
We consider three versions of CS and CL: those based on independence 
of transitions, on independence of events, and on ordering of events, and we study their relationships. 
We also show that axioms SP, BTI, WF, PCI and IEC, together with any one of IRE, CIRE and BFCIRE, are independent of each other.
Section~\ref{sec:coinitial} considers two structured forms of independence, namely independence defined on coinitial transitions only, and independence defined on labels only.
%defines coinitial LTSIs, namely LTSIs where independence is defined on coinitial transitions only, and considers their relationships with general LTSIs. 
Eight case studies of reversible formalisms are presented in Section~\ref{sec:casestudies}, where we
demonstrate that our basic axioms are very effective in proving the main reversibility properties.
%for the formalisms.
Section~\ref{sec:related} discusses relations with other works in the literature.
The final section contains concluding remarks and suggests potential future work.

This paper is an extended version of~\cite{LanesePU20}. The paper has been fully restructured, and now includes a number of additional or refined results. Beyond this, it includes full proofs of our results, as well as additional case studies, examples and explanations. We remark that the preliminary results in~\cite{LanesePU20} have already been exploited in~\cite{LaneseM20,AubertM21,BocchiLMY22,LamiLSCF22,BAGOSSY20223,Aub22,BM22}, which can be seen as further case studies for our approach.

%*** Not sure about what to do with the material in the rest of the section (IL) ***
%
%\todo{The intention here is to motivate the rival versions of CS/CL
%using either ordering or independence}
%Let us consider a simple example.
%Suppose that a system performs action $a$ followed by action $b$,
%to get to state $Q$, and then $a$ is reversed.
%Following the causal approach, we deduce that $a$ did not cause $b$,
%since $b$ did not have to be reversed in order to reverse $a$.
%Alternatively, we can deduce that $a$ and $b$ are concurrent
%(note that both $a$ and $b$ can be reversed at state $Q$),
%so that we could have performed $b$ followed by $a$ to reach $Q$.
%This gives us two possible forms of causal safety.
%We shall see that they are equivalent provided that certain axioms hold.
%Similar considerations apply to causal liveness.

% \todo{What is our story for independence versus concurrency?}
% 
% Consider a square with sides $a$ and $b$. We could use it to represent the forward behaviour
% of $a.b.\nil + b.a.\nil$.
% %and of $a.\nil \mid b.\nil$.
% When we want to work with reverse behaviour as well, then the square is not sufficient for
% representing both behaviours in $a.b.\nil + b.a.\nil$: we would need to know which sides we took
% so that we reverse correctly by backtracking on them. Without that knowledge, when we have performed
% $a$ and $b$, we are allowed to reverse any of the cofinal transitions in any order.
% So, we treat them as independent wrt
% to reversing, and thus also as independent when they ($a$ and $b$) are initially enabled. We can express this
% by adding independence $\ind$ between coinitial and cofinal $a$ and $b$ transitions in the square, giving
% a mutex square. Such decorated
% square represents two exclusive behaviours $ab$ and $ba$ which, when completed, can be freely undone. Consequently,
% an $a,b$ square without independence in any corner represents potentally $a.b.\nil + b.a.\nil$, where we can
% reverse actions incorrectly.
% 
% Concurrency additionally requires that $a$ and $b$ are also independent while taking place, namely in traces
% $ab$ and $ba$. This would allow them to be performed along each other and not only one after the other
% (in any ordeer), as would possible when they are in mutual exclusion. We represent concurrency between
% $a$ and $b$ by decorating not only the initial and final corners of the square but also the other
% two corners with independence $\ind$, giving us a concurrent square.
% 
% CC works well for (ladders of) mutex or concurrent squares. CS and CL on the other hand require
% the notion of event, which is based on concurrent squares. I think that CC and all our versions of
% CS and CL hold for mutex and for concurrent squares. However, if we add $\ind$ to another corner
% of the mutex square, then CC and CS still hold for all our versions. However, CL does not hold if we use
% Definition 4.11 but CL holds if we use 4.23: please check.
% 
%end

%\todo{remainder copied from Concur submission introduction - needs updating
%to remove local vs global and diamond equivalence, etc.}
%The aim of this paper is to produce a theory of causal-consistent
%reversibility which abstracts from the specific formalism. More
%precisely, we consider a generic reversible labelled transition
%relation (ltr---notably, among the works above, \cite{DK04,PU07} use
%an ltr, while
%\cite{LaneseMS16,GiachinoLMT17,LaneseNPV18} exploit a reduction
%semantics, which can be seen as an ltr where all the labels do
%coincide) and we consider a few axioms (originally presented
%in~\cite{PU07a}) that they may satisfy.  We then show which axioms are
%needed to ensure properties of interest such as the parabolic
%lemma. In this way, given a calculus one can just check the axioms and
%get for free all the relevant results. Notably, axioms should be
%simpler to check than the properties. This is indeed the case since
%axioms refer to a small number of related transitions---they
%are \emph{local} properties---while most of the properties of interest
%refer to arbitrary computations---they are \emph{global} properties.
%Our main contributions are as follows.
%We define a variant of causal equivalence
%which we call \emph{diamond equivalence}.
%This dispenses with the independence relation and instead focuses
%exclusively on the diamonds of transitions induced by concurrent actions.
%We show that the parabolic lemma and the causal consistency property
%are derivable from three basic axioms (Definition~\ref{def:pre-prime}),
%which are local apart from a well-foundedness condition.
%
%We give conditions on an abstract model with an independence relation
%under which causal equivalence coincides with diamond equivalence
%(Proposition~\ref{prop:ceqt ceqtind}),
%and show that these conditions are satisfied by RCCS (Proposition~\ref{prop:RCCS SL ID}).  
%
%We separate causal-consistent reversibility into two parts,
%a \emph{causal safety} condition saying that an action cannot be reversed
%if some of its effects have not been reversed,
%and a \emph{causal liveness} condition stating that an action can be reversed
%if all its effects have been reversed (Definition~\ref{def:safe live}).
%We show that these two conditions are not derivable from the basic axioms
%(Example~\ref{ex:pre-prime not CS}).
%Hence proofs of the parabolic lemma and the causal consistency property
%are not sufficient on their own,
%since these properties are consequences of the basic axioms.
%In the literature a separate proof of causal-consistent reversibility 
%has been given in~\cite{LaneseNPV18};
%we are not aware of other instances of such proofs.
%
%However if we add two further axioms to ensure that our ltrs are
%\emph{prime}~\cite{PU07a},
%we show that we can now derive causal safety and liveness
%(Propositions~\ref{prop:safety} and~\ref{prop:liveness}).
%These further axioms are non-local,
%but local versions exist which are equivalent
%(Proposition~\ref{prop:local prime}, taken from~\cite{PU07a}).
%Thus we have found a route using a small number of local axioms
%(plus well-foundedness) to establish the parabolic lemma and the causal consistency property and furthermore causal safety and liveness.
%


