%
\section{Basic Properties}\label{sec:basic}
In this section we show that most of the properties in the reversibility literature (see, e.g.,
\cite{DK04,PU07,LaneseMS16,LaneseNPV18}), in particular the Parabolic Lemma and
Causal Consistency, can be proved under minimal assumptions on the
combined LTSI under analysis.

We formalise the minimal assumptions using three axioms, described below.
\begin{definition}[Basic axioms]\label{def:basic}
  %Let $\mc L = (\Proc,\Lab,\tran{},\ind)$ be a combined LTSI.
  We say an LTSI %$\mc L$
  satisfies:
  \begin{description}
\item[Square property (SP)]\!\!: if whenever $t:P \tran \alpha Q$, $u:P
    \tran \beta R$ with $t \ind u$ then there are cofinal transitions
    $u': Q \tran \beta S$ and $t':R \tran \alpha S$;%\vspace{1pt}
\item[Backward transitions are independent (BTI)]\!\!: if whenever $t:P \rtran{a} Q$ and $t': P \rtran{b} Q'$ 
     and $t \neq t'$ then $t \ind t'$;%\vspace{3pt}
\item[Well-founded (WF)]\!\!: if there is no infinite reverse
    computation, i.e.\ we do not have $P_i$ (not necessarily distinct)
    such that $P_{i+1} \tran {a_i} P_i$ for all $i = 0,1,\ldots$.
  \end{description}
\end{definition}
WF can alternatively be formulated using backward transitions,
but the current formulation makes
sense also in non-reversible calculi (e.g., CCS), which can be used as a comparison.
Let us discuss the intuition behind these axioms. SP takes its
name from the Square Lemma, where it is proved for concrete calculi and
languages~in~\cite{DK04,LaneseMS16,LaneseNPV18}, and
captures the idea that independent transitions can be executed in any
order, that is they form commuting diamonds. SP can be seen as a
sanity check on the chosen notion of independence. BTI
generalises the key notion of backward determinism used in sequential
reversibility (see, e.g., \cite{Pin87} for finite state
automata and \cite{YokoyamaG07} for the imperative
language Janus) to a concurrent setting.  Backward determinism can be
spelled as ``two coinitial backward transitions do coincide''. This can be generalised to
``two coinitial backward transitions
are independent''.
We will show in Proposition~\ref{prop:sequential} that the two definitions are equivalent  when no transitions are independent, 
which is the common setting in sequential computing.
Note that BTI and SP together rule out examples $a.\nil \tran a \nil$, $b.\nil \tran b \nil$ as well as $a.\nil + b.\nil \tran a \nil$, $a.\nil + b.\nil \tran b \nil$ from the Introduction.
Finally, WF means that we consider systems which have a
finite past. That is, we consider systems starting from some initial
state and then moving forward and back.
WF rules out example $P \tran a P$ where $P = a.P$ from the Introduction.

Axioms SP and BTI are related to properties which are part of the definition
of (occurrence) transition systems with independence in~\cite[Definitions~3.7, 4.1]{SNW96}.
WF was used as an axiom in~\cite{PU07a}.

%\begin{definition}[Backward transitions are independent (BTI)]\label{def:bti}
%  If $t:P \rtran{a} Q$ and $t': P \rtran{b} Q'$ and $t \neq t'$ then $t \ind t'$. 
%\end{definition}
%
%A main property of independent transitions is that they form commuting
%diamonds.
%
%\begin{definition}[Square property (SP)]\label{def:sp}
%Whenever $t:P \tran \alpha Q$, $u:P \tran \beta R$ with $t \ind u$
%then there are cofinal transitions $u': Q \tran \beta S$ and $t':R
%\tran \alpha S$.
%\end{definition}
%The property above takes the name of Square Lemma in the different
%instances in the literature, when it is proved for concrete calculi
%and languages.
%
%\begin{definition}\label{def:wfut}
%  An LTS $(\Proc,\Lab,\tran{})$ satisfies:
%{\bf WF} (well-founded) if there is no infinite reverse computation,
%i.e.\ we do not have $P_i$ (not necessarily distinct)
%such that $P_{i+1} \tran {a_i} P_i$ for all $i = 0,1,\ldots$
%\end{definition}
%This gives our three basic axioms: BTI, SP and WF.

Using the minimal assumptions above we can prove relevant results
from the literature. As a preliminary step, we define causal equivalence,
equating computations differing only for swaps of independent
transitions and simplification of a transition
with its reverse.

\begin{definition}[Causal equivalence, cf.~{\cite[Definition 9]{DK04}}]\label{def:ceqt}
  %Let $(\Proc,\Lab,\tran{},\ind)$ be an
  Consider an LTSI satisfying SP.
Let $\ceqt$ be the smallest equivalence relation on paths closed under composition and satisfying:
\begin{enumerate}
\item
(swap)  
if $t:P \tran \alpha Q$, $u:P \tran \beta R$ are independent,
and $u': Q \tran \beta S$, $t':R \tran \alpha S$ (which exist by SP)
then $tu' \ceqt ut'$;
\item
(cancellation)
$t \rev t \ceqt \es$ \quad and \quad $\rev t t \ceqt \es$.
\end{enumerate}
\end{definition}

We first consider the Parabolic Lemma \cite[Lemma
  10]{DK04}, which states that each path is causal equivalent to a
backward path followed by a forward path.

\begin{definition}\label{def:PL}
  {\bf Parabolic Lemma property (PL)}: for any path $r$ there are forward-only paths $s,s'$ such that 
$r \ceqt \rev s s'$ and $\len s + \len {s'} \leq \len r$.
\end{definition}

\begin{proposition}\label{prop:PL}
Suppose an LTSI satisfies BTI and SP.  Then PL holds.
\end{proposition}

\begin{proof}
Suppose BTI and SP hold.
Define a function on paths as follows:
$d(r)$ is the number of pairs of forward transitions $(t,u)$
such that $t$ occurs in any position to the left of $\rev u$ in $r$.
%$(a,b)$ such that $a$ occurs to the left of $\rev b$ in $s$.
We say $r$ is parabolic iff $d(r) = 0$. We have to show that each path is causal equivalent to a parabolic one.

Suppose $d(r) > 0$.
We show that there is $s \ceqt r$ with $\len {s} \leq \len r$ and $d(s) < d(r)$.
Since $d(r) > 0$, we have $r = s_1 t \rev u s_2$ with
$s_1:P \tran {\sigma_1} R$,
$t:R \tran a S$, $\rev u:S \tran {\rev b} T$ and $s_2:T \tran {\sigma_2} Q$.
If $t = u$, then we obtain $r = s_1 t \rev u s_2 \ceqt s_1 s_2$.
Clearly $r \ceqt s_1s_2$ with $\len {s_1s_2} < \len r$ and $d(s_1s_2) < d(r)$.
%
So suppose $t \neq u$.
By BTI we have $\rev t \ind \rev u$.
By SP there are $S'$ and transitions $u':S' \tran b R$, $t':S' \tran a T$.
See Figure~\ref{fig:pl}.
% Figure environment removed
Then $\rev t \, \rev{u'} \ceqt \rev u \, \rev{t'}$.
Hence, $r = s_1 t \rev u s_2
\ceqt s_1 t \rev u \, \rev{t'} t' s_2
\ceqt s_1 t \rev t \, \rev{u'} t' s_2
\ceqt s_1 \rev{u'} t' s_2 = s$ as required.
Given that $\len {s_1 \rev {u'} t' s_2} = \len r$ and $d(s_1 \rev {u'} t' s_2) = d(r)-1$ the thesis follows.
\end{proof}
The proof of Proposition~\ref{prop:PL} is very similar to that
of~\cite[Lemma~10]{DK04} except that in the latter BTI is shown directly as part of the proof.

A corollary of PL is that if a process is reachable
from an irreversible process, then it is also forwards reachable from it. In other words,
making a system reversible does not introduce new reachable states
but only allows one to explore forwards-reachable states in a different order. 
This is relevant, e.g., in reversible debugging of concurrent
systems~\cite{GiachinoLM14,LaneseNPV18}, where one wants to find bugs that actually 
occur in forward-only computations.
\begin{corollary}\label{freachable}
  Suppose an LTSI satisfies PL. If a process $P$ is reachable
  from some irreversible process $Q$, then it is also forward
  reachable from $Q$.
\end{corollary}
\begin{proof}
  By hypothesis, there is some path $r: Q \ptran{} P$. Thanks to PL,
  there are forward-only paths $s,s'$ such that $\rev s s': Q \ptran{}
  P$.  Since $Q$ is irreversible, $s = \es$, hence $s': Q \ptran{}
  P$ as desired.
\end{proof}
When WF and PL hold, each process is reachable from a unique irreversible process.
\begin{proposition}\label{prop:unique irrev}
Suppose an LTSI satisfies WF and PL.
For any process $P$ there is a unique irreversible process $I$ such that $P$ is reachable from $I$.
\end{proposition}
\begin{proof}
Let $P$ be any process.
We use WF to deduce that there is an irreversible process $I$ such that $P$ is (forward) reachable from $I$ via some path $r$.
Suppose now that $I'$ is irreversible  and there is a path $r'$ from $I'$ to $P$.
Then $r'\rev r: I' \ptran{} I$.
By PL there are forward-only paths $s,s'$ such that $\rev s s': I' \ptran{} I$.
But since $I$ and $I'$ are irreversible, both $s = \es$ and $s' = \es$.
Hence $I' = I$ as required.
\end{proof}

We now move to causal consistency~\cite[Theorem 1]{DK04}.

\begin{definition}\label{def:cc}
{\bf Causal Consistency (CC)}: if $r$ and $s$ are coinitial and cofinal paths then $r \ceqt s$.
\end{definition}

Essentially, causal consistency states that history information allows
one to distinguish computations which are not causal equivalent.
Indeed, if two computations are cofinal, that is they reach the same
final state (which includes the stored history information) then they
need to be causal equivalent.

Causal consistency frequently includes the other direction, namely that
coinitial causal equivalent computations are cofinal, meaning that
there is no way to distinguish causal equivalent computations. This
second direction follows easily from the definition of causal equivalence.

Notably, our proof of CC below is very much shorter than existing
proofs, such as the one of \cite[Theorem 1]{DK04} for RCCS and the one
of \cite[Theorem 21]{LaneseNPV18} for reversible Erlang.

\begin{proposition}\label{prop:PL WF CC}
Suppose an LTSI satisfies WF and PL. 
Then CC holds.
\end{proposition}
\begin{proof}
Let $r:P \ptran \rho Q$ and $r':P \ptran {\rho'} Q$.
Using WF, let $I,s$ be such that $s:I \ptran \sigma P$, $I \in \Irr$.
Now $sr\rev{sr'}$ is a path from $I$ to $I$,
and so by PL there are $r_1,r_2$ forward-only such that $\rev{r_1}r_2 \ceqt sr\rev{sr'}$.
But $I \in \Irr$ and so $r_1 = \es$ and $r_2 = \es$.
Thus $\es \ceqt sr\rev{sr'}$, so that $sr \ceqt sr'$ and (by composing with \rev{s} on the left) $r \ceqt r'$
as required.
\end{proof}

Causal equivalent computations are strongly related
%quite close
in terms of the number
of transitions with a given label they contain.

\begin{proposition}\label{prop:count ceqt}
If $r \ceqt s$ then for any action $a$ the number of $a$-transitions in $r$ is the same as in $s$, where we count reverse transitions negatively.
\end{proposition}
\begin{proof}
  Straightforward, by induction on the derivation of $r \ceqt s$.
\end{proof}
\begin{remark}\label{rem:count fwd ceqt}
One consequence of 
Proposition~\ref{prop:count ceqt} is that if $r \ceqt s$ and $r$ and $s$
are both forward-only, then $\len r = \len s$.
\end{remark}

Causal consistency implies the unique transition property.

%\todo{Check if better to use also for both backwards}
%\todo{Yes, cfr. Lemma~\ref{lem:non-degenerate}}
\begin{definition}\label{def:ut}
  %An LTSI $(\Proc,\Lab,\tran{},\ind)$ satisfies
  \textbf{Unique transition (UT)}:
  if either $P \tran a Q$ and $P \tran b Q$ or $P \rtran a Q$ and $P \rtran b Q$ then $a = b$.
\end{definition}

%\begin{restatable}{corollary}{UT}\label{cor:ut}
\begin{corollary}\label{cor:ut}
  If an LTSI satisfies CC then it satisfies UT.
\end{corollary}
%\end{restatable}
\begin{proof}
  Since $P \tran a Q$ and $P \tran b Q$ are coinitial and cofinal then
  they are causal equivalent. By Proposition~\ref{prop:count ceqt} the
  counting of actions  %events 
  should be the same, hence $a=b$.
\end{proof}
UT was shown in the forward-only setting of occurrence TSIs in~\cite[Corollary~4.4]{SNW96};
it was taken as an axiom in~\cite{PU07a}. 
% We note that all our axioms are new apart from WF (and UT which is not central to our axiomatic approach).
% When we introduce further axioms later on, we shall comment if they relate to previousely proposed 
% axioms.
\begin{example}[PL alone does not imply WF or CC]\label{ex:PL not CC}
Consider the LTSI with states $P_i $ for $i = 0,1,\ldots$ and
transitions $t_i:P_{i+1} \tran a P_i$, $u_i:P_{i+1} \tran b P_i$ with $a \neq b$ and $\rev{t_i} \ind \rev{u_i}$.
BTI and SP hold.
Hence PL holds by Proposition~\ref{prop:PL}.
However clearly WF fails.
Also $t_i $ and $u_i$ are coinitial and cofinal,
% but we see that $t_i \ceqt u_i$ does not hold using
% Proposition~\ref{prop:count ceqt} and $a\neq b$.
% Hence CC fails.
and $a \neq b$, so that UT fails, and hence CC fails using Corollary~\ref{cor:ut}.
Note that the $ab$ diamonds here have the same side states so are degenerate (cf.~Lemma~\ref{lem:non-degenerate}).\finex
\end{example}
We have seen that SP is assumed when defining causal equivalence $\ceqt$.
%On the assumption of
Assuming SP, we give a diagram (Figure~\ref{fig:basic2}) to show implications between the remaining two axioms presented so far (BTI, WF) and the two main properties introduced so far (PL, CC).
We remark that the implications shown are strict (reverse implication does not hold).
%solid arrow head is strict implication
%open arrow head means it is not known if the implication is strict.
%% % Figure environment removed
%% \todo{Perhaps too many properties in Figure~\ref{fig:basic1}.
%% I put them all in to see what it looked like.
%% Probably best to omit UT altogether as in Figure~\ref{fig:basic2}.
%% Questions remain about whether implications are strict, etc.}
% Figure environment removed
We provide below counterexamples showing strictness of implications:
\begin{example}[SP, WF and CC do not imply PL]\label{ex:notPL}
  Consider the LTSI with states $P,Q,R$ and transitions $t:P\tran a R$, $u:Q \tran b R$,
  with an empty independence relation.
Then clearly BTI and PL fail.
However SP, WF and CC (and therefore UT) hold.

For CC, note that % the LTS is a path graph and
we can use cancellation to reduce each path to a unique shortest normal form
with respect to $\ceqt$.
There are various cases to check, depending on the initial and final states of the path, both ranging over $P,Q,R$.
Let $r:R \ptran\rho R$ be any path from $R$ to $R$.
If $r$ is non-empty, it must be of the form either $r = \rev t t r'$ or $r = \rev u u r''$.
We can use cancellation to get either $r \ceqt r'$ or $r \ceqt r''$.
Iterating the argument we see that $r \ceqt \es$. 
Now let $r:P \ptran\rho R$ be any path from $P$ to $R$.
Then $r = tr'$ where $r'$ is a path from $R$ to $R$.
Hence $r \ceqt t$.
Now let $r:P \ptran\rho P$ be any path from $P$ to $P$.
Then $r = tr'\rev t$ where $r'$ is a path from $R$ to $R$.
Hence $r \ceqt t\rev t \ceqt \es$.
Next let $r:P \ptran\rho Q$ be any path from $P$ to $Q$.
Then $r = tr'\rev u$ where $r'$ is a path from $R$ to $R$.
Hence $r \ceqt t\rev u$.
The remaining cases are similar.\finex
\end{example}
\begin{example}[SP, WF, PL and CC do not imply BTI]\label{ex:notBTI}
Consider the LTSI with states $P,Q,R,S$ and transitions $t:P\tran a Q$, $u:P \tran b R$,
$t':R \tran a S$ and $u':Q \tran b S$, with $t \ind u$.
Then BTI fails for $\rev{t'}$ and $\rev{u'}$.
However SP, WF and PL 
hold, and therefore CC also holds.

We show PL. %\todo{perhaps can be shortened}.
As in the proof of Proposition~\ref{prop:PL}, for a path $r$ let
$d(r)$ be the number of pairs of forward transitions $(t,u)$
such that $t$ occurs to the left of $\rev u$ in $r$.
Then $r$ is parabolic iff $d(r) = 0$.

Suppose $d(r) > 0$.
We show that there is $s \ceqt r$ with $\len {s} \leq \len r$ and $d(s) < d(r)$.
Since $d(r) > 0$, we have $r = s_1 t'' \rev{u''} s_2$.
If $t'' = u''$, then we can use cancellation as in the proof of Proposition~\ref{prop:PL}. 
So suppose $t'' \neq u''$.
Since the target of $t''$ must be the same as the source of $u''$,
the only possibilities are
$t'' = t'$, $u'' = u'$ or dually $t'' = u'$, $u'' = t'$.
We consider $t'' = t'$, $u'' = u'$;
the other case is similar.
So $r = s_1 t' \rev{u'} s_2$.
Since $t \ind u$ we have $tu' \ceqt ut'$.
Hence $\rev u t u' \rev{u'} \ceqt \rev u u t' \rev{u'}$,
and so $\rev u t \ceqt t' \rev{u'}$.
So $r \ceqt s = s_1 \rev u t s_2$ and $d(s) = d(r)-1$, $\len s = \len r$.\finex
\end{example}
\begin{example}[SP and WF do not imply CC (or PL)]\label{ex:WFnotCC}
Consider the LTSI of Example~\ref{ex:notBTI},
but without $t \ind u$.
Clearly SP and WF hold.
However CC fails, since there are paths $tu'$ and $ut'$ from $P$ to $S$,
but $tu' \not\ceqt ut'$.
To see this, imagine that the four transitions of the diamond correspond to
rotations around the centre of the diamond
(see Figure~\ref{fig:WFnotCC}).
% Figure environment removed
Measuring anti-clockwise rotation in radians
we see that $t$ and $u'$ each give a rotation of $-\pi/2$,
while $u$ and $t'$ each yield $+\pi/2$.
Let us define the rotation of a path to be the sum of the rotations of its transitions.
Path $tu'$ has rotation $-\pi$ while $ut'$ has $+\pi$.
Since there are no independent transitions,
the only operation of causal equivalence we can perform is to use
$t\rev t \ceqt \es$.
This clearly preserves the rotation of a path.
Hence $tu' \not\ceqt ut'$ as required.

PL does not hold either, otherwise CC would follow from Proposition~\ref{prop:PL WF CC}.\finex
\end{example}
\begin{example}[SP, BTI and CC do not imply WF]\label{ex:notWF}
Consider the LTSI with states $P_i $ for $i = 0,1,\ldots$ and
transitions $t_i:P_{i+1} \tran a P_i$.
Clearly WF does not hold.
However SP, BTI (and hence PL) hold; also CC (and hence UT) hold, noting that any
path is causally equivalent to a path which is entirely forward or entirely reverse.\finex
\end{example}
