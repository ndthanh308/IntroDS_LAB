\section{Related Work}\label{sec:related}
%\todo{To be completed.  Discuss~\cite{SNW96} and~\cite{PU07a}}
Causal Consistency (CC), Parabolic Lemma (PL) and informal versions of Causal Safety and Liveness (CS, CL), 
the main general properties of reversible computation considered in this paper, were
proposed by Danos and Krivine~\cite{DK04}. Since then, many reversible process calculi or formalisms 
have been developed as we have described in the Introduction. Most of them 
use memories to save information lost when computing forwards, which can be easily retrieved 
when computing in reverse. Concurrency relation between coinitial transitions is typically defined 
in terms of structural conditions on the
memories of the transitions. In order to show that reversibility 
is well-behaved, PL and then CC is proved. In contrast, CS and CL (in any of the variants we considered), or properties close to them, 
have not been widely considered.
% and we have now shown that CC alone is too weak to guarantee well-behaved reversibility.

Information needed for undoing of computation in a process calculus can be saved differently.  An alternative
method was proposed for reversing a process calculus given by
a general format of SOS rules in~\cite{PU06,PU07}.
When applied to CCS it produces CCSK, where reversible processes maintain 
their syntax as they compute, and executed actions are marked with \emph{communication keys}. 
When computation reverses, keys are removed, thus returning processes to their original form. 
This approach has a drawback in that it is not easy 
to define a concurrency relation purely on transition labels. As a result, proving CC in 
the traditional way is not straightforward. Hence, slightly different properties are proved 
to show that the resulting reversible calculi are
well-behaved. The main property is Reverse Diamond (RD):
if $Q \tran a P$, $R \tran b P$ and $Q \neq R$, then there is $S$ such that $S \tran a R$ and
$S \tran b Q$. In our setting, RD can be proved from the Loop Lemma, BTI and SP. It is worth noting that PL can be shown for CCSK mainly using RD~\cite[Lemma 5.9]{PU07}.
Moreover, a form of CC for forward computation is shown~\cite[Proposition 5.15]{PU07}: 
two forward computations from the same start to the same 
endpoint are \emph{homotopic}~\cite{vG96}, meaning that one computation can be transformed into the other by swapping 
adjacent transitions in commuting diamonds. In effect, concurrency is represented as commuting diamonds in 
the LTSs for reversible calculi obtained by applying the method in~\cite{PU06,PU07}.

A more abstract approach to defining desirable properties for reversibility was taken
in~\cite{PU07a}. General LTSs were considered instead of LTSs for specific reversible calculi, and two 
sets of axioms were proposed. The first set inherited RD and Forward Diamond (FD) from \cite {PU06,PU07},
and also included WF, UT and Event Determinism (ED)~\cite{SNW96,vG96}:
if $P \tran a Q$ and $P \tran a R$, and $(P,a,Q) \sqeqt (P,a,R)$, then $Q = R$.
%\todo{
ED is not a consequence of our basic axioms. Consider the LTS~\cite[Fig. 1]{PU07a}, and add coinitial independence 
using BTI and PCI.  The resulting LTSI is pre-reversible and satisfies CLG, yet it fails ED. 
%}
LTSs satisfying the
five axioms above are called \emph{prime} LTSs and are shown
to correspond to prime event structures.
Several interesting properties were proved for prime LTSs, including
RED (event determinism for backward transitions, which follows from BLD in our setting) and
NRE which we also consider here. 
The second set of axioms aimed at providing local versions of FD, ED and RED. 
%Combining the results of this paper (for LTSIs without independence) \todo{IVAN: results of this paper cannot be applied without independence} with those in ~\cite{PU07a}, 
%we can deduce CC since RD and WF imply PL. 
%Also, we can show that if WF, UT, RD as well as RED and NRE hold, then CS$_<$ and CL$_<$ follow. \todo{IVAN: how?}

%% \todo{IVAN: to integrate or redundant with the above?}
%% \iu{
%% There
%% is an alternative set of basic axioms for reversibility that does not involve an independence relation 
%% and which is sufficient to show PL, CC, CS$_<$ and CL$_<$. There axioms are WF, UT, RED and NRE, 
%% which we have considered in this paper, and additionally a crucial \emph{Reverse Diamond} (RD) axiom: 
%% if $Q \tran a P$, $R \tran b P$ and $Q \neq R$, then there is $S$ such that $S \tran a R$ and
%% $S \tran b Q$. WF, UT and RD are needed to show PL and CC. We can prove CS$\indt$ additionally with NRE,
%% and CL$\indt$ with both NRE and RED. 

%% Do we list Props 2.10, 2.11, 4.17 and 4.18 with their proofs from CONCUR 2019 paper here?
%% }


As we have mentioned in the Introduction, a combined causal safety and liveness property
has been formulated in~\cite[Corollary 22]{LaneseNPV18}.
%as follows: a forward transition
%$t$ can be undone after a derivation iff all its consequences, if any, are undone beforehand. 
A form of causal safety and liveness properties has been defined in the setting 
of reversible event structures in~\cite{PU13,PU15}.
A reversible event structure is called \emph{cause-respecting}
if an event cannot be reversed until all events it has caused have also been reversed, and 
it is \emph{causal} if it is cause-respecting and a reversible event can be reversed
if all events it has caused have been reversed~\cite[Definition~3.34]{PU15}. Causal reversible prime event structures
are considered in~\cite{MelgrattiMPPU2020} as well, where it is shown that they correspond precisely
to reversible occurrence nets.

%In other related work,
%we may particularly mention
Another related work is~\cite{DanosKS07},
which like ours takes an abstract view, though based on
category theory. However, its results concern irreversible actions,
and do not provide insights in our setting, where all actions are
reversible. The only other work which takes a general perspective
is~\cite{BernadetL16}, which concentrates on how to derive a
reversible extension of a given formalism. However, proofs concern a
limited number of properties (essentially our CC), and hold only 
for extensions built using the technique proposed there. 
An approach similar to that in~\cite{PU07,BernadetL16} is taken in~\cite{LaneseM20}, which focuses on
systems modelled using reduction semantics. In order to prove
properties of the reversible systems they build they use our theory
(taken from the conference version of the present
paper~\cite{LanesePU20}), hence this can be taken as an additional
case study for our results.
Finally,~\cite{EKM19} presents a number of properties such as, for example,
backward confluence, which arise in the context of reversing
of multiple transitions at the same time (called a step) in Place/Transition nets.
