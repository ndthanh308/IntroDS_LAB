%

\section{Causal Safety and Causal Liveness}\label{sec:CSCL}
In the literature, causal consistent reversibility is frequently
informally described by saying that ``a transition can be undone if
and only if each of its consequences, if any, has been undone''
% we explained this better in the introduction
%
(see, e.g., \cite{LaneseNPV18}).
%, which is also the only paper we are aware of where such a property is formalised). 
In this section we study this
property, where the two implications will be referred to as \emph{causal
  safety} and \emph{causal liveness}. We provide three different formalisations %versions
of
such properties, based on
independence of transitions (Section~\ref{sub:indtra}),
independence of events (Section~\ref{sub:indev}), and
ordering of events (Section~\ref{sub:ord}),
and study their relationships.
% In order to define such properties we need the concept of event.
In Figure~\ref{fig:diagprerev1} we show the relationships
between the various axioms and properties we shall study
in this section and Section~\ref{sec:coinitial}.
% Figure environment removed
\subsection{CS and CL via independence of transitions}\label{sub:indtra}
%% \todo{Ivan's points from emails:

%% - in the definition of CS/CL and variants we require independence between the
%%   reverse of the first transition and all the transitions $t$ in the path $r$ such that $\cte(r,[t]) \neq 0$

%% - the reason of why considering the reverse of the first transition was explained by myself in a previous mail

%% - the reason to consider $\neq 0$ instead of > 0 is to take into account also
%%   backward transitions in $r$

%% our FOSSACS development was mainly focused on coinitial independence, and in particular SP triggers when independence is available on coinitial transitions.
%%     Causal safety and liveness, as well as CC, are mainly focused on consecutive transitions, and would require sideways diamond more than our SP. Now, in our framework (without using RPI) the only way to trigger diamonds on consecutive transitions is by considering the reverse of the first transition, so to get a pair of coinitial transitions.
%%     For this reason I think it makes a lot of sense to have as hypothesis that the reverse of the first transition is independent on the others.}

We first define causal safety and liveness using the independence relation.
% The idea is that perhaps we do not need to use a causality relation here.
\begin{definition}\label{def:safe live}
Let $\mc L$ be an LTSI. % pre-reversible LTSI.
% (do we need to assume any of the axioms hold here?).
\begin{enumerate}
\item
We say that $\mc L$ is \emph{causally safe (CS$\indt$)} if whenever
$t_0:P \tran a Q$, $r:Q \ptran \rho R$, $\cte(r,[t_0]) = 0$ and $t_0\op:S \tran a R$
with $t_0 \sqeqt t_0\op$,
then $\rev{t_0} \ind t$ for all $t$ in $r$ such that $\cte(r,[t]) > 0$.
%
% we are wondering if we could write $\cte(r,[t]) \neq 0$ instead of $\cte(r,[t]) > 0$ here.
%
\item
We say that $\mc L$ is \emph{causally live (CL$\indt$)} if whenever
$t_0:P \tran a Q$, $r:Q \ptran \rho R$ and $\cte(r,[t_0]) = 0$ and
$\rev{t_0} \ind t$, for all $t$ in $r$ such that $\cte(r,[t]) > 0$,
then we have
$t_0\op:S \tran a R$ with $t_0 \sqeqt t_0\op$.
\end{enumerate}
\end{definition}
% \todo{Previously $\cte(r,[t])$ was only defined for forward $t$;
% now we would have to specify that $t$ is forward.
% If we allow reverse $t$ it is a bit strange to require $\cte(r,[t])>0$
% rather than $\cte(r,[t]) \neq 0$.
% See Remark~\ref{rem:rev count}.}
Properties CS$\indt$ and CL$\indt$ both consider a (forward) transition $t_0:P \tran a Q$  
followed by a path $r$ where the number of occurrences in $r$
of transitions that belong to the same event as $t_0$ is zero.
CS$\indt$ states that if after path $r$ a transition $t_0\op$ can be undone, where
$t_0$ and $t_0\op$ belong to the same event, then the reverse of $t_0$ 
is independent of all transitions $t$ where the number of occurrences in $r$ of 
the event of $t$ is positive.
%
Dually, CL$\indt$ requires that if the reverse of $t_0$ is independent of all transitions 
whose events have a positive number of occurrences in $r$, then it can be undone.

% \todo{ The paragraph above is a rewritten version of the next paragraph. It may be easier to follow.
% 
% In words, both CS$\indt$ and CL$\indt$ consider a (forward) transition $t_0:P \tran a Q$ performed
% before some path $r$ whose count of transitions in $[t_0]$ is
% $0$. CS$\indt$ states that if after path $r$ transition $t_0$ can be undone,
% that is the reverse of a transition in the same event is enabled, then
% the reverse of $t_0$ is independent of all transitions belonging to events which
% have in $r$ a positive number of occurrences. Dually, CL$\indt$ requires that
% if the reverse of $t_0$ is independent of all transitions belonging to events
% which have in $r$ a positive number of occurrences then it can be
% undone.}

\begin{remark}\label{rem:equal0notneeded}
In the definition of CS$\indt$ the condition that $\cte(r,[t_0]) = 0$
can be deduced from the other conditions using
% Propositions~\ref{prop:regeqzero} and~\ref{prop:NRE},
Lemma~\ref{lem:cte zero},
provided that the LTSI is pre-reversible.
\end{remark}


We use the reverse of $t_0$ when considering independence from $t$
because our axioms BTI, SP and PCI focus on \emph{coinitial} independence
rather than independence of consecutive transitions in a trace.
%one transition following another in a trace.
Take the simplest case where $r$ is a single transition $t:Q \tran b R$.
First assume $\rev{t_0} \ind t$;
note that this is coinitial independence.
We can use SP and PCI to get $t_0\op:S \tran a R$ with $t_0 \sqeqt t_0\op$,
which is an example of causal liveness.
Conversely, if we assume $t_0\op:S \tran a R$ with $t_0 \sqeqt t_0\op$,
we can use BTI, SP, BLD and PCI to get a diamond with $\rev{t_0} \ind t$,
which is an example of causal safety.

Note that in the discussion above to prove causal safety we need to consider also the case $r=t \rev t t$. Since $[\rev t]$ has a negative number of occurrences, we only need to show that $\rev{t_0} \ind t$, which can be proved as above. However, if we replaced the condition $\cte(r,[t]) > 0$ with $\cte(r,[t]) \neq 0$, we would also need to show  $\rev{t_0} \ind \rev t$, which does not follow from the axioms above. Intuitively, requiring $\rev{t_0} \ind \rev t$ would make little sense, since all the occurrences of $\rev t$ could be simplified with corresponding occurrences of $t$. This is why we decided to require $\cte(r,[t]) > 0$.

%% \todo{The above discussion might seem to suggest that causal safety and liveness
%% hold in a single diamond with coinitial independence.
%% However this is not the case for CS$\indt$, since with $t:Q \tran b R$
%% we can have $r = t \rev t t$, so that we can deduce $\rev {t_0} \ind \rev t$,
%% which is not coinitial independence.
%% See Proposition~\ref{prop:IC CSi}.
%% Therefore I wonder if we should return to $t_0 \ind t$ rather than
%% $\rev{t_0} \ind t$ in the definitions of CS$\indt$ and CL$\indt$.
%% The forward version is the more obvious definition, and the motivation for the
%% reverse version is weaker than previously thought.
%% Furthermore, I think we can show that the forward version of CS$\indt$
%% implies CIRE without using RPI.
%% See revised proof of Proposition~\ref{prop:CSindt RPI CIRE}.
%% Theorem~\ref{thm:CL} does use $\rev{t_0}$ in an essential way.
%% So we would have to add RPI as an assumption.
%% Still, we could avoid the work showing that CS$\indt$ implies CL$\ci$,
%% since that is now immediate via CIRE.
%% }

We have seen in the last two paragraphs that existing axioms are sufficient to show
CS$\indt$ and CL$\indt$ in the case where trace $r$ consists of
a single transition. However, existing axioms are not enough
for general $r$, as we will show in Examples~\ref{ex:prerev not CSi} and~\ref{ex:prerev not CL}.
%To show CS$\indt$ and CL$\indt$
Thus, we introduce
% We may wish to require
the following
axiom, which states that independence does not depend on the choice
of the representative inside an event.
\begin{definition}\label{def:IRE}
  {\bf Independence respects events (IRE)}:
Whenever $t \sqeqt t' \ind u$ we have $t \ind u$.
\end{definition}
IRE is one of the conditions in the definition of transition systems with
independence~\cite[Definition~3.7]{SNW96}.

IRE allows us to relate coinitial independence on events and independence on transitions.
\begin{lemma}\label{lem:coind IRE}
Assume an LTSI satisfies IRE.
If $[t] \coind [u]$ then $t \ind u$.
\end{lemma}
\begin{proof}
Immediate.
\end{proof}

Together with the axioms for pre-reversibility,
IRE is enough to show both CS$\indt$ and CL$\indt$.

% \begin{lemma}\label{lem:ind path}
% \todo{remove as no longer needed - use Lemma~\ref{lem:ladder} instead.}
% Suppose a pre-reversible LTSI satisfies IRE.
% Suppose also that $(P,a,Q) \sqeqt (R,a,S)$.
% Then there is a path $r$ from $Q$ to $S$ such that $(P,a,Q) \ind t$
% and $(P,a,Q) \ind \rev t$
% for all $t$ in $r$.
% \todo{Needs to change from $(P,a,Q)$ to the reverse transition.  Better to name $(P,a,Q)$?}
% \end{lemma}
%% \begin{proof}
%% Suppose that $(P,a,Q) \sqeqt (R,a,S)$.
%% Then there is a ladder of diamonds connecting $(P,a,Q)$ to $(R,a,S)$,
%% giving a path $r$ from $Q$ to $S$.
%% Let us show that $(P,a,Q) \ind t$ for all $t$ in $r$.
%% Let us consider any diamond in the ladder, with transitions
%% $P' \tran a Q'$, $t:Q' \tran \beta S'$, $t':P' \tran \beta R'$, $R' \tran a S'$, where  
%% $t$ is a transition in $r$ and $P' \tran a Q'$ is in the same event as $P \tran a Q$.
%% By definition of event $P' \tran a Q' \ind t' \sqeqt t$.
%% By IRE, $P \tran a Q \ind t$ as desired.
%% Furthermore, $(R',a,S') \ind \rev{t'}$, so that by IRE $(P,a,Q) \ind \rev t$. 
%% % also get $\rev {(P,a,Q)} \ind t$, $\rev {(P,a,Q)} \ind \rev t$.  
%% \end{proof}



% \begin{definition}\label{def:NRE fwd}
% An LTSI satisfies No Repeated Events (NRE) if in any forward-only path there cannot occur two different transitions $t$ and $t'$ such that $t \sqeqt t'$.
% \todo{rival duplicate definition Definition~\ref{def:NRE} - needs fixing}
% \end{definition}
% \begin{proposition}\label{prop:nre}
% Suppose that an LTSI satisfies CC and IRE.
% Then it also satisfies NRE.
% \todo{strengthened in Proposition~\ref{prop:CIRE NRE}}
% \end{proposition}
% \begin{proof}
% Let $r$ be a forward-only path, and suppose for a contradiction that it contains
% $t:P \tran a Q$ followed later by $t':P' \tran a Q'$ where $t \sqeqt t'$.
% Let $r'$ be the portion of $r$ from $Q$ to $P'$.
% % We can suppose that $\cte(r',[t]) = 0$.
% By Lemma~\ref{lem:ind path} there is a path $s$ from $Q$ to $Q'$ such that
% for all $u$ in $s$
% we have $t \ind u$.
% By CC, $s \ceqt r't'$.
% By Lemma~\ref{lemma:cccount}, $\cte(s,[t']) > 0$, since $r'$ is forward-only.
% Hence there is $u$ in $s$ such that $u \sqeqt t' \sqeqt t$.
% But then $t \ind u \sqeqt t$, and so $t \ind t$ by IRE, contradicting $\ind$ being
% irreflexive.
% \end{proof}

%\begin{restatable}{theorem}{CS}\label{thm:CS}
\begin{theorem}\label{thm:CS}
Let a pre-reversible LTSI satisfy IRE.
Then it satisfies CS$\indt$.
\end{theorem}
%\end{restatable}
%% \begin{proof}[Old Proof to be removed]
%% Suppose $P \tran a Q$, $r:Q \ptran \rho R$ and $S \tran a R$
%% with $(P,a,Q) \sqeqt (S,a,R)$.
%% By Lemma~\ref{lem:ind path} there is a path $s$ from $Q$ to $R$
%% such that $(P,a,Q) \ind u$ 
%% %\todo{and $(P,a,Q) \ind \rev u$}
%% and $(P,a,Q) \ind \rev u$ for all $u$ in $s$.
%% By CC, $r \ceqt s$.
%% %
%% Suppose $t$ in $r$ is such that $\cte(r,[t]) \neq 0$.
%% \todo{I think the original proof may have considered $\cte(r,[t]) < 0$,
%% though this is now excluded from Definition~\ref{def:safe live}.
%% If $\cte(r,[t]) < 0$, then since $t$ must be forward by the definition of $\cte(r,[t])$,
%% we would be considering $\rev t$ in $r$ rather than $t$ in $r$.}
%% If $\cte(r,[t]) > 0$, \todo{what if $\cte(r,[t]) < 0$?}
%% then $\cte(s,[t]) > 0$, thanks to Lemma~\ref{lemma:cccount}.
%% But then there is $u$ in $s$ such that $u \sqeqt t$.
%% We have $(P,a,Q) \ind u$ and so $(P,a,Q) \ind t$, using IRE.
%% \end{proof}
%% \todo{We can show the stronger form of CS$\indt$ with $\cte(r,[t]) \neq 0$
%% rather than $\cte(r,[t]) > 0$:
\begin{proof}
% \todo{NB Assumption $\cte(r,[t_0]) = 0$ not used as expected from Remark~\ref{rem:equal0notneeded}.}
Suppose $t_0:P \tran a Q$, $r:Q \ptran \rho R$ and $t_0\op:S \tran a R$
with $t_0 \sqeqt t_0\op$.
By Lemma~\ref{lem:ladder} there is a path $s$ from $Q$ to $R$
such that for all $u$ in $s$ we have $[t_0] \coind [u]$.
We deduce by Lemmas~\ref{lem:coind rev} and~\ref{lem:coind IRE}
that for all $u$ in $s$ we have 
$\rev{t_0} \ind u$. %and $\rev{t_0} \ind \rev u$
%\todo{last item and Lemma~\ref{lem:coind IRE} not needed? Last item not needed but Lemma~\ref{lem:coind IRE} needed. }
% By [modified] Lemma~\ref{lem:ind path} there is a path $s$ from $Q$ to $R$
% such that $\rev{t_0} \ind u$ 
% and $\rev{t_0} \ind \rev u$ for all $u$ in $s$.
By CC, $r \ceqt s$.

Take $t$ in $r$ such that $\cte(r,[t]) > 0$.
% If $\cte(r,[t]) > 0$, 
Then $\cte(s,[t]) > 0$, thanks to Lemma~\ref{lemma:cccount}.
But then there is $u$ in $s$ such that $u \sqeqt t$.
We have $\rev{t_0} \ind u$ and so $\rev{t_0} \ind t$, using IRE, as desired.
% If $\cte(r,[t]) < 0$, 
% then $\cte(s,[t]) < 0$, again by Lemma~\ref{lemma:cccount}.
% But then there is $u$ in $s$ such that $\rev u \sqeqt t$. %\todo{do we need Lemma~\ref{lem:coind rev}? Seems not}.
% We have $\rev{t_0} \ind \rev u$ and so $\rev{t_0} \ind t$ as desired, again using IRE. 
\end{proof}

%In order to prove CL we need an auxiliary result.
%
%\begin{lemma}\label{lemma:norm}
%  Let $r$ be a path. There is a path $r' \ceqtind r$ such that for each
%  event $e$, it is not the case that both $e$ and $\rev e$ occur.
%\end{lemma}
%\begin{proof}
%\end{proof}
% As a consequence, if an event $e$ occurs, then $\cte(r',e)>0$.
%As a consequence, if a transition $t$ occurs, then $\cte(r',[t])>0$.


%\begin{restatable}{theorem}{CL}\label{thm:CL}
% \todo{Theorem~\ref{thm:CL} is false by Example~\ref{ex:IRE2}.
% Need to add e.g. RPI (as we had originally).}\todo{Previous comment is outdated, right?}
\begin{theorem}\label{thm:CL}
  Let a pre-reversible LTSI satisfy IRE.
  Then it satisfies CL$\indt$.  
\end{theorem}
%\end{restatable}
% \begin{proof}
  % Suppose $t_0:P \tran a Q$, $r:Q \ptran \rho R$ and $\cte(r,[t_0]) =
  % 0$ and $t_0 \ind t$, for all $t$ such that $\cte(r,[t]) > 0$.  We
  % have to show that there is $S \tran a R$ with $(P,a,Q) \sqeqt
  % (S,a,R)$.
% 
  % Thanks to the parabolic lemma, there is $S$ such that $b: P
  % \ptran{\rho_b} S$ and $f: S \ptran{\rho_f} R$, with $b$ backward and
  % $f$ forward. Since $\cte(r,[t_0]) = 0$, thanks to
  % Lemma~\ref{lemma:cccount} $\cte(b;f,[t_0])=1$. As a consequence,
  % there is at least a transition $t'_0:P' \tran a Q' \in [t_0]$ in
  % $f$.  If we can show that $t'_0 \ind t''$ for each transition $t''$
  % following $t'_0$ in $f$, then the thesis will follow by commuting
  % $t'_0$ with all following transitions using SP.
% 
  % We have two cases. If $\cte(b,[t'']) = 0$ then $\cte(b;f,[t]) > 0$,
  % hence by hypothesis $t_0 \ind t''$ and by IRE $t'_0 \ind t''$ as desired.
% 
  % If $\cte(b,[t'']) < 0$, by looking at the proof of PL we see that
  % $t_0$ and a backward transition in $[\rev t'']$ have been commuted,
  % hence they are independent. By IRE and RPI, $t'_0 \ind t''$ as desired.
  % 
  % If $\cte(b,[t'']) < 0$, by looking at the proof of PL we see that
  % $t_0$ and a backward transition in $[\rev t'']$ have been commuted,
  % hence they are independent. By IRE and RPI, $t'_0 \ind t''$ as desired.
% 
  % This concludes the proof.
% \end{proof}
\begin{proof}
  Suppose $t_0:P \tran a Q$, $r:Q \ptran \rho R$ and $\cte(r,[t_0]) =
  0$ and $t_0 \ind t$, for all $t$ in $r$ such that $\cte(r,[t]) > 0$.  We
  have to show that there is $t_0\op:S \tran a R$ with $t_0 \sqeqt
  t_0\op$.

  Thanks to PL, there is $T$ such that $b: P
  \ptran{\rho_b} T$ and $f: T \ptran{\rho_f} R$, with $b$ backward and
  $f$ forward.
  By CC, $t_0r \ceqt bf$.
  Since $\cte(r,[t_0]) = 0$, thanks to
  Lemma~\ref{lemma:cccount} we have $\cte(bf,[t_0])=1$. As a consequence,
  there is a transition $t'_0:P' \tran a Q' \in [t_0]$ in
  $f$.
This $t'_0$ is in fact the unique transition in $[t_0]$ belonging to $f$ by Proposition~\ref{prop:NRE}.
  Let $f'$ be the portion of $f$ from $Q'$ to $R$.
  If we can show that $\rev{t'_0} \ind t''$ for each transition $t''$ in $f'$,
  then the thesis will follow by commuting
  $t'_0$ with all such transitions using SP and IRE.

  By Lemma~\ref{lem:ladder} there is a path $s$ from $Q$ to $Q'$ such that $[t_0] \coind [u]$
  for all $u$ in $s$.
  By CC, $r \ceqt sf'$.
  Take any $t''$ in $f'$.
  By Lemma~\ref{lemma:cccount}, $\cte(r,[t'']) = \cte(s,[t'']) + \cte(f',[t''])$.
  % If $\cte(s,[t'']) > 0$ then there is $u$ in $s$ such that $u \sqeqt t''$,
  % and so $t_0 \ind t''$ using IRE.
  If $\cte(s,[t'']) < 0$ then there is $u$ in $s$ such that $u \sqeqt \rev{t''}$.
Now $[t_0] \coind [u] = [\rev{t''}]$.
Therefore $[\rev{t_0}] \coind [t'']$ by Lemma~\ref{lem:coind rev}, and $\rev{t_0} \ind t''$ by Lemma~\ref{lem:coind IRE}.
  Suppose instead $\cte(s,[t'']) \geq 0$.
  Since $\cte(f',[t'']) > 0$, we have $\cte(r,[t'']) > 0$. 
  So there is $u$ in $r$ such that $u \sqeqt t''$,
  and by hypothesis $\rev{t_0} \ind u$, so that $\rev{t'_0} \ind t''$ using IRE.
\end{proof}
% \todo{Suppose we modify CL$\indt$
% by replacing $t_0 \ind t$ by $\rev{t_0} \ind t$ where $t_0:P \tran a Q$.
% I think that the above proof shows modified CL$\indt$ using IRE.
% Similarly, using a modified version of Lemma~\ref{lem:ind path}
% we can use IRE to show a version of CS$\indt$ proving $\rev{t_0} \ind t$ rather than $t_0 \ind t$
% (or of course both).
% }
 
%\begin{proof}[Relies on Lemma~\ref{lemma:norm} still to be proven]
%  
%  Suppose $t_0:P \tran a Q$, $r:Q \ptran \rho R$ and $\cte(r,[t_0]) = 0$ and
%$t_0 \ind t$, for all $t$ such that $\cte(r,[t]) > 0$.
%We have to show that there is
%$S \tran a R$ with $(P,a,Q) \sqeqt (S,a,R)$.
%
%Thanks to Lemma~\ref{lemma:norm} we have that there exists $r'
%\ceqtind r$ with $\cte(r',[t_0]) = 0$ (from $\cte(r,[t_0]) = 0$
%using Lemma~\ref{lemma:ccicount}). Furthermore, $t_0 \ind t$, for
%all $t$ in $r'$. Indeed, for all $t$ in $r'$ we have $\cte(r',[t]) >
%0$, and from Lemma~\ref{lemma:ccicount} we also have $\cte(r,[t]) >
%0$, hence $t_0 \ind t$ thanks to IRE.
%
%The proof is by induction on $\len{r'}$. If $\len{r'}=0$ then the thesis
%trivially holds by selecting $(S,a,R) = (P,a,Q)$.
%
%If $\len r'>0$ let $r'=t'r''$. From the above, we have $t_0 \ind t'$.
%Hence thanks to RPI and MSP there is a path $t'_1 t_1 r''$ with $t_1
%\sqeqt t_0$.
%\todo{Definition~\ref{sqeqt} as currently formulated also requires $Q \neq R$ or $P \neq S$.  See proof of Lemma~\ref{lemma:ccicount}}.
%Note that $\len{r''}<\len{r'}$. Also,
%$\cte(r'',[t_0]) = 0$. Finally, thanks to IRE $t_1 \ind t$ for all $t$ in $r''$%.
%Hence, by inductive hypothesis there
%is $S \tran a R$ with $t_1 \sqeqt (S,a,R)$. The thesis follows by
%transitivity of $\sqeqt$.
%\end{proof}

% \begin{proof}[Rough sketch]
% Suppose $P \tran a Q$, $r:Q \ptran \rho R$ and $\cte(r,[P,a,Q]) = 0$ and
% $(P,a,Q) \ind t$, for all $t$ such that $\cte(r,[t]) > 0$.
% Let $e = [P,a,Q]$.
% By PL we have forward-only $r_1,r_2$ such that $r \ceqt \rev{r_1}r_2$.
% % and no new events are introduced.
% Let $T$ be the source of both $r_1$ and $r_2$.
% Suppose that no $e$-transition occurs in either $r_1$ or $r_2$.
% We can use BTI and SP to create a ladder with $e$-transitions as rungs
% finishing with $T' \tran a T$.
% Now we can create a further ladder going along $r_2$.
% Each transition $u$ that we encounter is either from an event we saw in $r_1$,
% in which case we use IRE and RPI, or else it is new, in which case we know
% that $(P,a,Q) \ind t$ since $\cte(r,[t]) > 0$.
% We finish with the desired transition 
% $S \tran a R$ with $(P,a,Q) \sqeqt (S,a,R)$.
% 
% Now suppose that an $e$-transition occurs in either $r_1$ or $r_2$.
% We know $\cte(r,e) = 0$, so that $\cte(r_1,e) = \cte(r_2,e) > 0$.
% Start at the last $e$-transition $t:(P',a,Q')$ in $r_2$.
% We want to switch $t$ with each of the remaining
% transitions $u$ of $r_2$, which we can do using SP if we know that
% they are independent.
% Since $(P,a,Q) \sqeqt (P',a',Q')$, by Lemma~\ref{lem:ind path} there is a path $s$ from $Q$ to $Q'$
% such that $e$-transitions are independent of every $t'$ in $s$ (use of IRE).
% If a remaining $u$ satisfies 
% $\cte(r,[u]) > 0$ then we know that $(P,a,Q) \ind u$,
% so that $t \ind u$ (using IRE).
% Suppose that $\cte(r,[u]) \leq 0$.
% Then there is at least one $[u]$-transition in $r_1$.
% Using CC we see that $\cte(s,[u]) < 0)$ and so $t \ind \rev u$,
% so that $t \ind u$ using RPI.
% \end{proof}

%% CS$\indt$ and CL$\indt$ are not derivable from CC;
%% we give an example LTSI which satisfies CC but not CS$\indt$ and not CL$\indt$.
%% \todo{remove this next example?}
%% \begin{example}[CC does not imply CS$\indt$ or CL$\indt$]\label{ex:CC not CL}
%% Consider the LTS in Figure~\ref{fig:repeated1}.
%% % Figure environment removed
%% Independence is mostly coinitial and given by closing under BTI and PCI.
%% Additionally we make the leftmost $a$-transition independent with all $b$-transitions.
%% Note that all $a$-transitions belong to the same event,
%% and all $b$-transitions belong to the same event.
%% Also SP and WF hold, so that the LTSI is pre-reversible \todo{this is false as PCI fails}, and
%% CC holds.
%% However IRE does not hold.
%% % Also CS seems to hold using Definition~\ref{def:coinitial safe live}.
%% Furthermore CS$\indt$ fails using Definition~\ref{def:safe live}.
%% Indeed, consider any path $\ptran {ba\rev b}$ from the leftmost state. %start.
%% CS$\indt$ would imply that the first $b$ is independent with the $a$ but this is not the case
%% (we do however have $\rev b \ind a$).
%% % Or we can consider path $abab\rev a$ going along the top.
%% % CS would imply that the first $a$ is independent with the second $a$,
%% % which is again not the case.

%% Also CL$\indt$ fails using Definition~\ref{def:safe live}.
%% Indeed, consider any path $\ptran {abb}$ from the leftmost state. %start.
%% Since the leftmost $a$-transition is independent with all $b$-transitions,
%% we should be able to reverse $a$ at the end of the path, but this is not possible.
%% % Also IRE and NRE fail.
%% % \todo{The example seems correct but makes me wonder whether we have the right definitions
%% % of CS and CL.}
%% \end{example}
%% \todo{Example~\ref{ex:CC not CL} shows that the stipulation of IRE cannot be omitted in the statements of Theorems~\ref{thm:CS} and~\ref{thm:CL}.}

We now give examples of LTSIs which are pre-reversible and where CS$\indt$ and CL$\indt$ fail.
\begin{example}\label{ex:prerev not CSi}
Consider the LTSI shown in Figure~\ref{fig:notIRE} including the dashed transitions.
We add coinitial independence as generated by BTI and PCI.
BTI gives $(Q',\rev b,Q) \ind (Q',\rev a,P')$ and  $(R,\rev c,Q') \ind (R,\rev a,S)$.
Assuming $t_0:P \tran a Q$ and $t_0\op:S \tran a R$, PCI gives three additional independence pairs for each of
 the two diamonds:  $(Q, b,Q') \ind \rev{t_0}$, $t_0 \ind (P,b,P')$ and $(P', \rev{b}, P) \ind (P', a, Q')$  for the diamond with the source $P$, and  $(Q', c,R) \ind (Q', \rev{a}, P')$, $(P',a,Q') \ind (P',c,S)$ and $(S,\rev{c}, P') \ind t_0\op $ for the other diamond.
The LTSI is pre-reversible.
However CS$\indt$ fails.
% \todo{can deduce this from Proposition~\ref{prop:IC CSi}}
Transition $t_0$ is followed by a path $Q \ptran {bc} R$ and the transition $t_0\op$
satisfies $t_0 \sqeqt t_0\op$.
If CS$\indt$ held we could deduce that $\rev{t_0} \ind (Q',c,R)$,
which is not the case.
Similarly, we see that IRE fails, since
$\rev{t_0} \sqeqt (Q',\rev a ,P') \ind (Q',c,R)$ but not $\rev{t_0} \ind (Q',c,R)$.
Note, however, that CL$\indt$ holds, since only transitions inside the same diamond are independent, and transitions on one side of the diamond are undone by the corresponding transition on the opposite side.
% \todo{Do we need to say more?  The path $r$ can be arbitrarily long,
% though of course essentially just a single transition. IVAN: I believe we say enough}
\finex
% Figure environment removed
\end{example}

\begin{example}\label{ex:prerev not CL}
Consider the LTSI shown in Figure~\ref{fig:notIRE} excluding the dashed transitions.
We add coinitial independence as given by BTI and PCI, similarly to the previous example.
We also add $(Q,\rev a ,P) \ind (Q',c,R)$.
The LTSI is pre-reversible.
However CL$\indt$ fails.
We have $t_0:P \tran a Q$, $Q \ptran {bc} R$ and $\rev{t_0} \ind (Q,b,Q')$, $\rev{t_0} \ind (Q',c,R)$.
Clearly CL$\indt$ fails, since we cannot reverse the $a$-transition at $R$. 
% If CL$\indt$ held we could deduce that there is a transition
% $S \tran a R$ with $t_0 \sqeqt (S,a,R)$.
IRE fails since $(Q',\rev a ,P') \sqeqt \rev{t_0} \ind (Q',c,R)$
but not $(Q',\rev a ,P') \ind (Q',c,R)$.
Note, however, that CS$\indt$ holds since the only way to undo transitions is with transitions on the opposite side of the same diamond, and the path connecting them is another transition of the same diamond. Hence, the condition on independence holds, thanks to BTI and PCI.
% \todo{Actually CS$\indt$ fails, since we can have longer paths like $r = b\rev b b$ or $r = b \rev a \rev b a b$ instead of just $r = b$.
% To get CS$\indt$ to hold we must add $\rev a \ind b$, $\rev b \ind a$, $\rev a \ind \rev b$, for the $a$ and $b$ transitions.}
\finex
\end{example}
Examples~\ref{ex:prerev not CSi} and~\ref{ex:prerev not CL} show that the stipulation of IRE cannot be omitted in the statements of Theorems~\ref{thm:CS} and~\ref{thm:CL}, respectively.
These examples also show that we cannot deduce CS$\indt$ or CL$\indt$ from CC, nor one from the other.
\begin{example}[CS$\indt$ and CL$\indt$ do not imply CC]\label{ex:CS CL not CC}
Consider the LTSI with states $P,Q,R,S$ and transitions $t:P\tran a Q$, $u:P \tran b R$,
$t':R \tran {a'} S$ and $u':Q \tran {b'} S$, with empty independence relation.
This is essentially the same as Example~\ref{ex:WFnotCC},
except that we have disambiguated the transition labels, to reflect that
%since
the four transitions
form four different events.
Then CC does not hold, but we claim that both CS$\indt$ and CL$\indt$ hold.

CS$\indt$: There are four possible cases to check, depending on the initial forward transition.
Consider first $t:P \tran a Q$ and some $r:Q \ptran\rho Q'$, $P' \tran a Q'$, where
$\cte(r,[t]) = 0$ and $(P,a,Q) \sqeqt (P',a,Q')$. 
Clearly $P' = P$ and $Q' = Q$.
To verify CS$\indt$ in this case, it is enough to show that $\cte(r,[u]) = \cte(r,[t']) = \cte(r,[u']) = 0$.
Since $r$ is a circuit, it enters each state as often as it leaves it.
Furthermore, since $\cte(r,[t]) = 0$, $r$ enters $Q$ from $P$ as often as it leaves $Q$ towards $P$.
Hence $r$ must enter $Q$ from $S$ as often as it leaves $Q$ towards $S$,
meaning that $\cte(r,[u']) = 0$.
We can similarly deduce that $\cte(r,[t']) = 0$ and $\cte(r,[u]) = 0$.
% We also need to consider
% $u':Q \tran {b'} S$ and some $r:S \ptran\rho S'$, $Q' \tran {b'} S'$, where
% $\cte(r,u') = 0$ and $(Q,b',S) \sqeqt (Q',b',S')$. 
% Clearly $Q' = Q$ and $S' = S$.
% To verify CS in this case, it is enough to show that $\cte(r,t) = \cte(r,u) = \cte(r,u') = 0$.
% The argument is similar to the previous case.
The remaining three cases with initial transitions $u$, $t'$ and $u'$ are similar to the case for $t$.
% \todo{note that we have verified the strong form of CS$\indt$.}

CL$\indt$: Again there are four cases to check, depending on the initial forward transition.
Consider first $t:P \tran a Q$ and some $r:Q \ptran\rho Q'$ where
$\cte(r,[t]) = 0$ and for all $t''$ in $r$ we have %$\cte(r,[t'']) = 0$
$\cte(r,[t'']) \leq 0$ (indeed, if $\cte(r,[t'']) > 0$ we would require $\rev{t} \ind t''$, which is false since the independence relation is empty, hence the condition for CL$\indt$ would hold trivially). However, if $\cte(r,[t'']) < 0$ then
%\todo{there is $t'''$ in $r$ with $t''' \sqeqt \rev{t''}$, so that $t''' = \rev{t''}$, and }
there is $t'''$ in $r$ with $[t'''] = [\rev{t''}]$ (in this example actually $t''' = \rev{t''}$) and 
$\cte(r,[\rev{t''}]) > 0$, but, for the same reason as above, we cannot have $\cte(r,[\rev{t''}]) > 0$  since the independence relation is empty. Hence for each $t''$ we have $\cte(r,[t'']) = 0$, which implies $Q'=Q$, since the net rotation (cfr.\ Figure~\ref{fig:WFnotCC}) of each transition is zero, and so the net rotation of $r$ is zero. The thesis follows trivially.
%% We must show that $Q' = Q$.
%% We consider the four transitions to correspond to rotations around the centre of the diamond,
%% as in Figure~\ref{fig:WFnotCC}.
%% The rotation of path~$r$ is the sum of the rotations of the transitions followed.
%% Since for all $t''$ in $r$ we have $\cte(r,[t'']) \leq 0$, we see that each $t''$ contributes a net \il{non-positive} %zero
%% rotation, so that $r$ has a net
%% %zero
%% \il{non-positive}
%% rotation \il{as well}.
%% \il{Notice that the only way to have a negative rotation starting from $Q$ is to traverse $t$ backwards more times than forwards, but this is impossible since $\cte(r,[t]) = 0$. Hence, the net rotation of $r$ needs to be $0$, implying that $Q' = Q$.}
%% \todo{[Needs refining]
%% Suppose that $r$ has a change of direction of rotation.
%% It must be of the form $r = r_1t_1 \rev{t_1} r_2$.
%% We can cancel $t_1 \rev{t_1}$ to obtain $r' = r_1r_2$ which is a path from $Q$ to $Q'$.
%% Clearly $\cte(r',e) = \cte(r,e)$ for all events $e$, including $[t_1]$.
%% By iterating this procedure we can eliminate all changes of rotation from $r$.
%% Now it is clear that for all $t''$ in $r$ we have $\cte(r,[t'']) \geq 0$.
%% Combining, we see that for all $t''$ in $r$ we have $\cte(r,[t'']) = 0$.
%% }
% Since $\cte(r,[t]) = 0$, $r$ enters $Q$ from $P$ as often as it leaves $Q$ towards $P$.
% Hence $r$ leaves $Q$ towards $S$ at least as often as it enters $Q$ from $S$,
% so that $\cte(r,[u']) \geq 0$.
% Hence $\cte(r,[u']) = 0$ and $Q' = Q$ \todo{I think this argument needs completing}.
The remaining three cases with initial transitions $u$, $t'$ and $u'$ are similar to the case for~$t$.\finex
\end{example}

The next axiom states that independence is fully determined by its restriction to coinitial transitions. It is related to axiom (E) of~\cite[page 325]{SNW96},
but here we allow reverse as well as forward transitions.
%
\begin{definition}{\bf Independence of events is coinitial (IEC)}\label{def:IEC}:
if $t_1 \ind t_2$ then $[t_1] \coind [t_2]$.
% there are $t'_1 \sqeqt t_1$, $t'_2 \sqeqt t_2$ such that $t'_1$ and $t'_2$ are
% coinitial and $t'_1 \ind t_2'$.
\end{definition}

Thanks to previous axioms, independence behaves well
w.r.t.~reversing.
\begin{definition}{\bf Reversing preserves independence (RPI)}\label{def:rpi}:
  if $t \ind t'$ then $\rev t \ind t'$.
\end{definition}
%
%\begin{restatable}{proposition}{RPI}\label{prop:RPI}
 \begin{proposition}\label{prop:RPI}
If an LTSI satisfies SP, PCI, IRE, IEC then it also satisfies RPI.
\end{proposition}
%\end{restatable}
\begin{proof}
Suppose $t \ind u$.
We must show $\rev t \ind u$.
By IEC we have $t' \sqeqt t$, $u' \sqeqt u$ such that $t' \ind u'$ and $t',u'$ are coinitial.
% Complete the square to get $t''$ and $u''$ using SP.
By SP there is a diamond $t',u',t'',u''$ with $t' \sqeqt t''$, $u' \sqeqt u''$.
Then $\rev{t'} \ind u''$ using PCI.
Then $\rev t \sqeqt \rev{t'} \ind u'' \sqeqt u$
and so by IRE $\rev t \ind u$ as required.
\end{proof}
%% \todo{IVAN: I believe the proof below does not require WF either, since Lemma 4.15 requires pre-reversible in the statement but it seems not used in the proof, hence we can weaken the precondition}
%% \todo{
%% \begin{proof}[Alternative proof assuming pre-reversible, IEC, IRE]
%% Suppose $t \ind u$.
%% We must show $\rev t \ind u$.
%% By IEC we have $[t] \coind [u]$.
%% By Lemma~\ref{lem:coind rev} we get $[\rev t] \coind [u]$.
%% Finally by Lemma~\ref{lem:coind IRE} we obtain $\rev t \ind u$.
%% \end{proof}
We can use IEC or IRE to show that transitions which are part of the same event cannot be independent.
\begin{definition}{\bf Event coherence (ECh)}\label{def:ECh}:
if $t \sqeqt t'$ then $t \notind t'$.
\end{definition}
\begin{proposition}\label{prop:ECh}
If a pre-reversible LTSI satisfies either IRE or IEC then it also satisfies ECh.
\end{proposition}
\begin{proof}
Assume for a contradiction that $t \sqeqt t'$ and $t \ind t'$.
First suppose that IRE holds.
We deduce $t \ind t$, contradicting irreflexivity of $\ind$.
Now suppose that IEC holds.
Then $[t] \coind [t']$, and so $[t] \coind [t]$, contradicting irreflexivity of $\coind$ (Proposition~\ref{prop:coind irref}).
\end{proof}


% \todo{
% \begin{remark}\label{rem:rev count}
% Note that if $u$ in $r$ is such that $\cte(r,[u]) < 0$
% then there is $u'$ in $r$ such that $u' \sqeqt \rev u$ and $\cte(r,[u']) > 0$.
% Hence the condition $\cte(r,[t])>0$ in Definition~\ref{def:safe live}
% can be replaced by $\cte(r,[t]) \neq 0$ to yield equivalent definitions of CS$\indt$, CL$\indt$,
% on the additional assumption of RPI.
% \end{remark}}

All the axioms that we have introduced so far are independent,
i.e.\ none is derivable from the remaining axioms.

%We first show some examples related to independence of IRE and IEC.
The next example shows that IRE is not implied by other axioms.
\begin{example}\label{ex:CLG CSi}
Let $t:P \tran a Q$, $u:P \tran b R$,
$u':Q \tran b S$, $t':R \tran a S$,
with $t \ind u$, $\rev u \ind t'$, $\rev{t'} \ind \rev{u'}$, $u' \ind \rev t$, namely we have independence at all corners of the diamond.
Here we have two forward events, labelled with $a$ and $b$ respectively.
We have $t' \sqeqt t \ind u$ but not $t' \ind u$, so that IRE fails.
% \todo{can deduce this from Proposition~\ref{prop:IC CSi},
% which also shows that CS$\indt$ fails.
% This example has the same properties as Example~\ref{ex:prerev not CSi}.}
However axioms SP, BTI, WF, PCI and IEC hold.\finex
\end{example}
The next example shows that IEC is not implied by other axioms.
\begin{example}\label{ex:IRE1}
Let $t:P \tran a Q$, $u:R \tran b S$,
where all states are distinct,
and let $t \ind u$.
Then IEC fails;
however axioms SP, BTI, WF, PCI and IRE hold.\finex
\end{example}
The counterexample above remains valid also if $Q=R$, as shown below.
\begin{example}\label{ex:IRE2}
Let $t:P \tran a Q$, $u:Q \tran b S$,
and let $t \ind u$.
Then IEC fails;
however axioms SP, BTI, WF, PCI and IRE hold.\finex
% Note also that CL$\indt$ fails,
% since $a$ cannot be reversed at $R$,
 % showing that for pre-reversible LTSIs IRE is not sufficient to show CL$\indt$.
% \todo{But CL$\indt$ holds for the revised definition.}
\end{example}

We can now prove the independence result.
%\begin{restatable}{proposition}{independent}\label{prop:independent}
\begin{proposition}\label{prop:ind}
The axioms 
SP, BTI, WF, PCI, IRE and IEC are independent of each other.
% and do not imply ED.
% \todo{can omit ED}
\end{proposition}
%\end{restatable}
\begin{proof}
For each of the six axioms we give an LTSI which satisfies the other five axioms but not the axiom itself.
In each case it is straightforward to check that the remaining axioms hold.

{\bf SP:} Let $t:P \tran a Q$ and $u:P \tran b R$ with $t \ind u$.

{\bf BTI:} Let $P \tran a R$ and $Q \tran b R$ with an empty independence relation
(Example~\ref{ex:notPL}).

{\bf WF:} Let $P_{i+1} \tran a P_i$ for $i = 0,1,\ldots$ with an empty independence relation.

{\bf PCI:} Let $t:P \tran a Q$, $u:P \tran b R$,
$u':Q \tran b S$, $t':R \tran a S$,
with $\rev{t'} \ind \rev{u'}$.

{\bf IRE:} See Example~\ref{ex:CLG CSi}.

{\bf IEC:} See Example~\ref{ex:IRE1} or Example~\ref{ex:IRE2}.
% As noted elsewhere, the diagram for LED with independence between all forward and reverse $a$ and all forward and reverse $b$ transitions satisfies all six axioms but not ED.
\end{proof}


\subsection{CS and CL via independent events}\label{sub:indev}

We now introduce a second version of causal safety and liveness,
which uses independence like CS$\indt$ and CL$\indt$,
but on events rather than on transitions. More precisely, we use coinitial independence $\coind$.

%We can give a second formulation of causal safety and liveness using $\coind$:
\begin{definition}\label{def:coind safe live}
Let $\mc L = (\Proc,\Lab,\tran{},\ind)$ be an LTSI. % pre-reversible LTSI.
\begin{enumerate}
\item
We say that $\mc L$ is \emph{coinitially causally safe} (CS$\ci$) if whenever
$t_0:P \tran a Q$, $r:Q \ptran \rho R$, $\cte(r,[t_0]) = 0$ and $t_0\op:S \tran a R$
with $t_0 \sqeqt t_0\op$,
then $[\rev{t_0}] \coind e$ for all 
events $e$ such that $\cte(r,e) > 0$.
\item
We say that $\mc L$ is \emph{coinitially causally live} (CL$\ci$) if whenever
$t_0:P \tran a Q$, $r:Q \ptran \rho R$ and $\cte(r,[t_0]) = 0$ and
$[\rev{t_0}] \coind e$, for all 
events $e$ such that $\cte(r,e) > 0$,
%\todo{why only forward events? What if in between there is just a backward transition 
%not independent from e?  Also, it seems the proofs consider all events.}  
%		\iu{
%IU: It may be related to Lemma 4.27 and the comment just below it, 
%which says it is sufficient to consider forward events only, but I might be wrong.}
then we have
$t_0\op:S \tran a R$ with $t_0 \sqeqt t_0\op$.
\end{enumerate}
\end{definition}
Note that in Definition~\ref{def:coind safe live} we operate at the level of events,
rather than at the level of transitions as in Definition~\ref{def:safe live}.
Also note that we could replace $[\rev{t_0}] \coind e$ by
$[t_0] \coind e$ using Lemma~\ref{lem:coind rev}.
We have used the former for compatibility with Definition~\ref{def:safe live}.
% \begin{lemma}[Ladder Lemma]\label{lem:ladder}
%
%\begin{restatable}{theorem}{CScoind}\label{thm:CS coind}
\begin{theorem}\label{thm:CS coind}
If an LTSI is pre-reversible then it satisfies CS$\ci$.
\end{theorem}
%\end{restatable}
%
\begin{proof}
Suppose $t_0:P \tran a Q$, $r:Q \ptran \rho R$
%\todo{NB $\cte(r,[t_0]) = 0$ not needed for proof}
and $t_0\op:S \tran a R$
with $t_0 \sqeqt t_0\op$.
By Lemma~\ref{lem:ladder} there is a path $s$ from $Q$ to $R$
such that for all $u$ in $s$ we have
$[t_0] \coind [u]$.
By CC, $r \ceqt s$.

%\todo{Revised proof:
Suppose that $e$ is an event and $\cte(r,e) > 0$.
Then $\cte(s,e) > 0$, thanks to Lemma~\ref{lemma:cccount}.
Hence there is $u$ in $s$ such that $[u] = e$. % or $[u] = \rev e$.
Since $[t_0] \coind [u]$,
also $[t_0] \coind e$.
% either $[t_0] \coind e$ or $[t_0] \coind \rev e$. 
% In both cases we can deduce
Hence $[\rev{t_0}] \coind e$ using
Lemma~\ref{lem:coind rev}. 
%}
%% Suppose that $e$ is a forward event and $\cte(r,e) > 0$.
%% Then $\cte(s,e) > 0$, thanks to Lemma~\ref{lemma:cccount}.
%% Hence there is $u$ in $s$ such that $[u] = e$.
%% By Lemma~\ref{lem:ladder} we have $[t_0] \coind [u]$.
%% Hence $[t_0] \coind e$. 
%% Suppose that $e$ is a forward event and $\cte(r,e) < 0$.
%% Then $\cte(s,e) < 0$, thanks to Lemma~\ref{lemma:cccount}.
%% Hence there is $u$ in $s$ such that $[\rev u] = e$,
%% or equivalently $[u] = \rev e$.
%% By Lemma~\ref{lem:ladder} we have $[t_0] \coind [u]$.
%% Hence $[t_0] \coind \rev e$,
%% and $[t_0] \coind e$, using Lemma~\ref{lem:coind rev}.
%% \todo{The proof would show a form of CS$\ci$ where we talk of all events $e$
%% rather than all forward events $e$.}
%\todo{It looks like we could generalise the proof of CS
%to allow $P \tran\alpha Q$ where $\alpha$ can be forward or backward.}
\end{proof}
%

We now introduce a weaker version of axiom IRE (Definition~\ref{def:IRE}).
\begin{definition}{\bf Coinitial IRE (CIRE)}\label{def:CIRE}:
% if $t' \sqeqt t \ind u \sqeqt u'$ and $t',u'$ are coinitial then $t' \ind u'$.
if $[t] \coind [u]$ and $t,u$ are coinitial then $t \ind u$.
\end{definition}
It is easy to see that IRE implies CIRE.
By considering Example~\ref{ex:CLG CSi}
we see that an LTSI can be pre-reversible and satisfy CIRE (and IEC) but not IRE.
Also, CIRE is not sufficient to ensure ECh (Definition~\ref{def:ECh}) holds, as shown by the next example.
% \todo{postponed since CIRE was not defined at the previous position}
\begin{example}\label{ex:CSi+RPI CLi}
Let $t:P \tran a Q$, $u:P \tran b R$,
$u':Q \tran b S$, $t':R \tran a S$.
We add independence between all pairs of distinct transitions
drawn from $t,u,t',u'$.
We furthermore add those independent pairs derived from closing under RPI.
We see that the LTSI is pre-reversible.  It satisfies CIRE and RPI,
but not ECh, since $t \sqeqt t'$ and also $t \ind t'$.
\finex
\end{example}

The next example shows that notions of CS/CL based on independence on transitions and on coinitial independence of events are not equivalent.
%\todo{IVAN: the next example leaves the door open for some implications, do they hold? Or do we have counterexamples?
%[Iain: Assume pre-reversible and consider CS$\indt$, CL$\indt$ and CL$\ci$.
%I think CS$\indt$ implies CL$\ci$ (Conjecture~\ref{prop:CSi CL<}).
%Example~\ref{ex:prerev not CSi}:
%CLG holds, and hence CIRE, CL$\indt$, CL$\ci$.
%Shows that CL$\indt$+CL$\ci$ does not imply CS$\indt$.
%Can also use Example~\ref{ex:halfcube} for this.
%Example~\ref{ex:prerev not CL}:
%CIRE also holds, and hence CL$\ci$.
%Also CS$\indt$ holds, I think.
%Shows that CS$\indt$(+CL$\ci$) does not imply CL$\indt$.
%Example~\ref{ex:IC CLi}:
%CL$\indt$ holds but not CS$\indt$, CL$\ci$.]}
% Question: can we find an example where CS$\indt$ holds but not CL$\ci$?
% Question: can we find an example where CS$\indt$+CL$\indt$ holds but not CL$\ci$?
\begin{example}\label{ex:IC CLi}
% \todo{May have to be omitted for space reasons,
% but this example is less pathological
% than the ones with repeated events.}
Consider the LTSI in Figure~\ref{fig:IC CLi}.
% Figure environment removed
Independence is given by closing under BTI and PCI. Clearly WF and SP hold; hence the LTSI is pre-reversible and satisfies CS$\ci$.
There are three events, labelled $a,b,c$, which are all independent of each other.
Furthermore IEC holds, but not CIRE
(noting that the leftmost $b$ and $c$ transitions are coinitial but not independent, while the corresponding events are coinitially independent thanks to the rightmost square).
Also CL$\ci$ fails: consider $P \tran a Q \tran b R$,
where $a$ cannot be reversed at $R$ even though $[Q \tran{\rev{a}} P] \coind [Q \tran b R]$.
Differently from CS$\ci$, CS$\indt$ fails:
% \todo{can deduce this from Proposition~\ref{prop:IC CSi}}
e.g., from the leftmost corner one can do $bac\rev b$, reversing $b$, but the inverse of the first $b$-transition is not independent with the $c$-transition.
Differently from CL$\ci$, CL$\indt$ holds:
the only state at which any event that has occurred cannot be
immediately reversed is $R$.
So we can restrict attention to instances of $P' \tran a Q'$, $r:Q' \ptran \rho R$.
Furthermore $r$ must finish with either $Q \tran b R$ or the $c$ transition to $R$.
These two transitions are not independent with any inverse $a$ transition.
Hence CL$\indt$ holds in these cases vacuously.\finex
\end{example}
\begin{proposition}\label{prop:CSindt RPI CIRE}
Let $\mc L$ be a pre-reversible LTSI.
If $\mc L$ satisfies CS$\indt$ and RPI then $\mc L$ also satisfies CIRE.
\end{proposition}
\begin{proof}
%% Assume that $\mc L$ satisfies CS$\indt$ and RPI.
%% Suppose that $t,u$ are coinitial transitions such that $[t] \coind [u]$.
%% We must show that $t \ind u$.
%% Since $[t] \coind [u]$, there are coinitial $t',u'$ such that
%% $t \sqeqt t' \ind u' \sqeqt u$.
%% By SP we can complete a square containing $t',u'$ and two further cofinal transitions
%% $t'' \sqeqt t'$ and $u'' \sqeqt u'$ both with target $R$.

%% Suppose first that $t:P \tran a Q$ and $t':P' \tran a Q'$ are forward.
%% By Lemma~\ref{lem:ladder} there is a path $r:Q \ptran \rho Q'$.
%% % such that $[t] \coind [u_1]$ for all $u_1$ in $r$.
%% Let $s' = \rev t u \rev u t r$ (a path from $Q$ to $Q'$),
%% and consider the path
%% % $s = \rev t u \rev u t r u''$
%% $s = s'u''$
%% from $Q$ to $R$.
%% We see that $\cte(s,[t])= 0$,
%% using Lemma~\ref{lem:cte zero} applied to $t,t''$ and~$s$.
%% Hence CS$\indt$ applies to $t$ together with $s$ and $t''$.
%% We deduce that
%% $\rev t \ind u_1$ for all $u_1$ in $s$ such that $\cte(s,[u_1]) \neq 0$.
%% We see that $\cte(ts',[u]) = 0$ using
%% using Lemma~\ref{lem:cte zero} applied to $\rev u,\rev{u''}$ and $ts'$.
%% Noting that $\und{[t]} \neq \und{[u]}$
%% by %irreflexivity of $\coind$
%% Proposition~\ref{prop:coind und},
%% we obtain $\cte(s,[u]) = 1$
%% and so $\rev t \ind u$.
%% We deduce $t \ind u$ using RPI.

%% Suppose instead that
%% $t:P \tran {\rev a} Q$ and $t':P' \tran {\rev a} Q'$ are backward.
%% By Lemma~\ref{lem:ladder} applied to $\rev t, \rev {t'}$
%% there is a path $r:P \ptran \rho P'$.
%% % such that $[\rev t] \coind [u_1]$ for all $u_1$ in $r$.
%% Consider the path $s = u \rev u ru'$.
%% We see that $\cte(s,[\rev t])= 0$,
%% using Lemma~\ref{lem:cte zero} applied to $\rev t, \rev {t''}$ and~$s$.
%% Hence
%% CS$\indt$ applies to $\rev t$ together with $s$ and $\rev {t''}$.
%% We deduce that
%% $t \ind u_1$ for all $u_1$ in $s$ such that $\cte(s,[u_1]) \neq 0$.
%% We see that $\cte(u \rev u r,[u]) = 0$ using
%% using Lemma~\ref{lem:cte zero} applied to $\rev u,\rev{u'}$ and $u \rev u r$.
%% Clearly $\cte(s,[u]) = 1$ and so $t \ind u$ as required.

%\todo{Revised proof:
Assume that $\mc L$ satisfies CS$\indt$.
Suppose that $t,u$ are coinitial transitions such that $[t] \coind [u]$.
We must show that $t \ind u$.
We can suppose that at least one of $t$ and $u$ is forward;
otherwise we can obtain $t \ind u$ from BTI.
Without loss of generality, suppose that $t:P \tran a Q$ is forward.
Since $[t] \coind [u]$, there are coinitial $t':P' \tran a Q'$ and $u'$
such that $t \sqeqt t' \ind u' \sqeqt u$.
By SP we can complete a square containing $t',u'$ and two further transitions
$t'' \sqeqt t'$ and $u'' \sqeqt u'$ both with the same target~$R$.

By Lemma~\ref{lem:ladder} there is a path $s:Q \ptran \rho Q'$.
Let $r' = \rev t u \rev u t s$ (a path from $Q$ to $Q'$),
and consider the path
$r = r'u''$
from $Q$ to $R$.
We see that $\cte(r,[t])= 0$,
using Lemma~\ref{lem:cte zero} applied to $t,t''$ and~$r$.
Hence CS$\indt$ applies to $t$ together with $r$ and $t''$.
We deduce that
$\rev t \ind u_1$ for all $u_1$ in $r$ such that $\cte(r,[u_1]) > 0$.
We see that $\cte(tr',[u]) = 0$ using
Lemma~\ref{lem:cte zero} applied to $\rev u,\rev{u''}$ and $tr'$.
Noting that $\und{[t]} \neq \und{[u]}$
by %irreflexivity of $\coind$
Proposition~\ref{prop:coind und},
we obtain $\cte(r,[u]) = 1$
and so $\rev t \ind u$.
We deduce $t \ind u$ using RPI.
\end{proof}
We cannot omit the assumption of RPI in Proposition~\ref{prop:CSindt RPI CIRE},
in view of the following example.
\begin{example}\label{ex:halfcube mod}
Consider the `half cube' LTSI with transitions $a,b,c$ in Figure~\ref{fig:halfcube}.
% Figure environment removed
We add independence as given by BTI and PCI, and also between all pairs of transitions $t,u$ where at least one of $t,u$ is backward,
and $t \not\sqeqt u$, $t \not\sqeqt \rev u$. Clearly RPI does not hold. 
The LTSI is pre-reversible, and IEC holds.
CIRE does not hold; note that the $a$ and $b$-events are independent,
but after performing $c$ there are coinitial $a$ and $b$-transitions
which are not independent.
Both CL$\ci$ and CL$\indt$ hold: %\todo{[can deduce CL$\indt$ from CL$\ci$+IEC] not really needed, in case we need to postpone this example}:
note that at any state, all events that have occurred can be reversed
immediately.
%IC \todo{Definition~\ref{def:coinitial LTSI}} holds---all independence is coinitial.
We have ensured that CS$\indt$ holds, since all independence deducible from CS$\indt$ must involve a backward transition $\rev{t_0}$ and a transition $u$ such that $t_0 \not\sqeqt u$ and
$t_0 \not\sqeqt \rev u$.\finex
\end{example}

%\todo{IVAN: are we sure of the result below? It seems to me (2) should follow from CC. Also, if it holds, do we want to include this result in the paper?
%[Iain: (2) fails for Example~\ref{ex:IC CLi} and so does not follow from CC.]}
%\todo{
We can characterise CIRE as being equivalent to coinitial transitions
with a common derivative process being independent.
\begin{proposition}\label{prop:char CIRE}
Let $\mc L$ be a pre-reversible LTSI.  The following are equivalent:
\begin{enumerate}
\item\label{item:CIRE}
$\mc L$ satisfies CIRE;
\item\label{item:CDI}
If $t:P \tran \alpha Q$, $r:Q \ptran \rho S$ and 
$u:P \tran \beta R$, $s:R \ptran \sigma S$ where 
$\und\alpha \neq \und\beta$ and
$\cte(r,[t]) = \cte(s,[u]) = 0$ then
$t \ind u$.
\end{enumerate}
\end{proposition}
\begin{proof}
Assume (\ref{item:CIRE}).
Let $t:P \tran \alpha Q$, $r:Q \ptran \rho S$ and 
$u:P \tran \beta R$, $s:R \ptran \sigma S$ where 
$\und\alpha \neq \und\beta$ and
$\cte(r,[t]) = \cte(s,[u]) = 0$.
We must show $t \ind u$.
Since the LTSI is pre-reversible, polychotomy holds for events $[t]$ and $[u]$
(Proposition~\ref{prop:poly}).
We can exclude $[t] = [u]$ since
$\und\alpha \neq \und\beta$.
There is a rooted path $r_0$ from some irreversible $I$ to $P$.
Since NRE holds (Proposition~\ref{prop:NRE}),
$\cte(r_0,[t]) = \cte(r_0,[u])$.
By considering the paths $r_0t $ and $r_0u$ we deduce that neither $[u] < [t]$ nor $[t] < [u]$ hold.
By CC applied to $tr$ and $us$ we see that $\cte(r,[u]) = 1$.
Hence $r_0tr$ is a rooted path with $\cte(r_0tr,[t]) = \cte(r_0tr,[u]) = 1$,
so that we can exclude $[t] \cf [u]$.
By polychotomy we conclude that $[t] \coind [u]$.
Then $t \ind u$ by CIRE.

Assume (\ref{item:CDI}).
Let $[t] \coind [u]$ where
$t:P \tran \alpha Q$ and $u:P \tran \beta R$ are coinitial.
We must show $t \ind u$.
First note that $\und\alpha \neq \und\beta$
by Proposition~\ref{prop:coind und}.
We have $t \sqeqt t' \ind u' \sqeqt u$ where
$t':P' \tran \alpha Q'$ and $u':P' \tran \beta R'$ are coinitial.
By SP we have
$t'':R' \tran \alpha S$ and $u'':Q' \tran \beta S$.
By Lemma~\ref{lem:ladder} we have
$r':Q \ptran\rho Q'$ such that for all $u_1$ in $r'$ we have $[t] \coind [u_1]$,
and $s':R \ptran\sigma R'$ such that for all $u_2$ in $s'$ we have $[u] \coind [u_2]$.
Let $r = r'u''$ and $s = s't''$.
We have $\cte(r,[t]) = \cte(s,[u]) = 0$ using Lemma~\ref{lem:cte zero}.
Hence $t \ind u$ as required, using the hypothesis.
\end{proof}

Notably, in the proof of (\ref{item:CIRE}) $\Rightarrow$ (\ref{item:CDI}),
CIRE is only used in the last step. Hence, the result could be rephrased by stating that any pre-reversible LTSI satisfies (\ref{item:CDI}),
with a conclusion of $[t] \coind [u]$ rather than $t \ind u$.

The independence result in Proposition~\ref{prop:ind} holds also if we replace IRE by CIRE.
\begin{proposition}\label{prop:ind CIRE}
The axioms 
SP, BTI, WF, PCI, CIRE and IEC are independent of each other.
\end{proposition}
\begin{proof}
For each of the six axioms we need to give an LTSI which satisfies the other five axioms but not the axiom itself.
Since IRE implies CIRE, for all axioms apart from CIRE we can reuse the examples given in the proof of Proposition~\ref{prop:ind}.
Example~\ref{ex:IC CLi} provides an LTSI where CIRE fails and the remaining five axioms hold.
\end{proof}

We can distinguish three mutually exclusive cases for CIRE
(Definition~\ref{def:CIRE}):
\begin{description}
\item
[forward case:] both transitions are forward;
\item
[backward-forward case:] one transition is backward, one is forward;
\item
[backward case:] both transitions are backward (implied by BTI).
\end{description}
The second case is particularly relevant for the characterisation of CL$\ci$; hence we state it as a separate axiom.
\begin{definition}\label{def:BFCIRE}
{\bf Backward-Forward CIRE (BFCIRE)}:
if $t:P \tran a Q$ and $u:Q \tran b R$ and $[\rev t] \coind [u]$ then $\rev t \ind u$.
\end{definition}
Thus BFCIRE is just CIRE specialised to the case where one of the
coinitial transitions is backward and one is forward. It has some similarity with one of the properties of transition systems with independence in \cite{NW95} and \cite[Definition 4.1]{SNW96}, and Sideways Diamond properties in \cite{PU07a,Aub22}. However, all of these properties state that if two consecutive forward transitions are independent then they are two sides of a commuting diamond.   
%\todo{[Add citations for Sideways Diamond in the literature.]}

Analogously to what was done in Theorem~\ref{thm:CL} for CL$\indt$, we give below conditions for ensuring CL$\ci$.
Notably, here BFCIRE is necessary and sufficient,
while for CL$\indt$ we required IRE, which was sufficient but not necessary.
%\todo{IVAN: do we have an example showing that CIRE would not be enough before? Anything else meaningful to say in the comparison?
%[Iain:
%Example~\ref{ex:CLG}:
%CLG holds, and hence CIRE.
%Shows that CIRE does not imply CS$\indt$.
%Example~\ref{ex:prerev not CL}:
%CIRE also holds.
%Shows that CIRE does not imply CL$\indt$.
%]}

%\begin{restatable}{theorem}{CLcoind}\label{thm:CL coind}
\begin{theorem}\label{thm:CL coind}
Let $\mc L$ be a pre-reversible LTSI.  Then the following are equivalent:
\begin{enumerate}
\item\label{item:L BFCIRE}
$\mc L$ satisfies BFCIRE;
\item\label{item:L CLci}
$\mc L$ satisfies CL$\ci$.
\end{enumerate}
\end{theorem}
%\end{restatable}
\begin{proof}
Assume (\ref{item:L BFCIRE}).
  Suppose $t_0:P \tran a Q$, $r:Q \ptran \rho R$ and $\cte(r,[t_0]) =
  0$ and $[\rev{t_0}] \coind e$, for all $e$ such that $\cte(r,e) > 0$.  We
  have to show that there is $t_0\op:S \tran a R$ with $t_0 \sqeqt
  t_0\op$.

  Thanks to PL, there is $T$ such that $b: P
  \ptran{\rho_b} T$ and $f: T \ptran{\rho_f} R$, with $b$ backward and
  $f$ forward.
  By CC, $t_0r \ceqt bf$.
  Since $\cte(r,[t_0]) = 0$, thanks to
  Lemma~\ref{lemma:cccount} $\cte(bf,[t_0])=1$. As a consequence,
  there is a transition $t'_0:P' \tran a Q' \in [t_0]$ in
  $f$ (which is unique by Proposition~\ref{prop:NRE}).
  Let $f'$ be the portion of $f$ from $Q'$ to $R$.

  If we can show that $[\rev{t_0}] \coind [t'']$ for each transition $t''$ in $f'$,
  then the thesis will follow by commuting
  $t'_0$ with all such transitions using SP and BFCIRE.

  By Lemma~\ref{lem:ladder} there is a path $s$ from $Q$ to $Q'$ such that $[t_0] \ind [u]$  % \todo{and $t_0 \ind \rev u$}
  for all $u$ in $s$.
  By CC, $r \ceqt sf'$.
  Take any $t''$ in $f'$.
  By Lemma~\ref{lemma:cccount}, $\cte(r,[t'']) = \cte(s,[t'']) + \cte(f',[t''])$.
  % If $\cte(s,[t'']) > 0$ then there is $u$ in $s$ such that $u \sqeqt t''$,
  % and so $t_0 \ind t''$ using IRE.
  If $\cte(s,[t'']) < 0$ then there is $u$ in $s$ such that $u \sqeqt \rev{t''}$.
  Now $[t_0] \coind [u]$, and so $[t_0] \coind [t'']$ using Lemma~\ref{lem:coind rev}.
  So suppose $\cte(s,[t'']) \geq 0$.
  Since $\cte(f',[t'']) > 0$, we have $\cte(r,[t'']) > 0$. 
  So there is $u$ in $r$ such that $u \sqeqt t''$,
  and by hypothesis $[\rev{t_0}] \coind [u]$, so that $[\rev{t_0}] \coind [t'']$.

Assume (\ref{item:L CLci}). 
Suppose that $t_0:P \tran a Q$ and $u:Q \tran b R$ and $[\rev {t_0}] \coind [u]$.
Clearly $\cte(u,[t_0]) = 0$.
By CL$\ci$ we have $t_0^{\dagger}:S \tran a R$ with $t_0 \sqeqt t_0^{\dagger}$.
Using BTI and SP we can complete a square starting with $\rev u$ and
$\rev{t_0^{\dagger}}$.
Using BLD this square must include $t_0$.
Using PCI we see that $\rev{t_0} \ind u$ as required.
\end{proof}
CL$\ci$ (and BFCIRE) do not imply CIRE,
as shown by Example~\ref{ex:halfcube mod}.
\begin{lemma}\label{lem:CSi BFCIRE}
Let a pre-reversible LTSI satisfy CS$\indt$.  Then it satisfies BFCIRE.
\end{lemma}
\begin{proof}
Suppose $t_0:P \tran a Q$ and $u:Q \tran b R$ and $[\rev {t_0}] \coind [u]$.
We must show that $\rev {t_0} \ind u$.

By Lemma~\ref{lem:coind rev} $[t_0] \coind [u]$ and so there are coinitial
$t_0':P' \tran a Q'$ and $u':P' \tran b R'$ with
$t_0 \sqeqt t'_0 \ind u' \sqeqt u$. 
Using SP we can complete a square with $t_0^{\dagger}:R' \tran a S'$ and $u'':Q' \tran b S'$.
By Lemma~\ref{lem:ladder} applied to $u$ and $u''$ we have a path $s$ from $R$ to $S'$.
Let $r = us$.
Then $\cte(r,[t_0]) = 0$ using Lemma~\ref{lem:cte zero}.
Also $\cte(s,[u]) = 0$ using Lemma~\ref{lem:cte zero},
so that $\cte(r,[u]) > 0$.
By CS$\indt$ applied to $t_0,t_0^{\dagger}$ and $r$
we deduce $\rev{t_0} \ind u$ as required.
\end{proof}

Perhaps surprisingly,
we can now relate safety with independence of transitions to liveness with independence of events.
\begin{proposition}\label{prop:CSi CLci}
Let a pre-reversible LTSI satisfy CS$\indt$.
Then it satisfies CL$\ci$.  
\end{proposition}
\begin{proof}
By Lemma~\ref{lem:CSi BFCIRE} and Theorem~\ref{thm:CL coind}.
\end{proof}
% If SP, BTI, WF, PCI, CIRE hold then NRE holds and therefore polychotomy holds.
%\todo{Omit the next sentence: }The stipulation of CIRE cannot be omitted from the statement of Theorem~\ref{thm:CL coind} in view of Example~\ref{ex:IC CLi}.

CL$\ci$ (and BFCIRE) do not imply CS$\indt$,
as shown by the next example.
\begin{example}\label{ex:halfcube}
Consider the `half cube' LTSI with transitions $a,b,c$ in Figure~\ref{fig:halfcube}.
We add independence as given by BTI and PCI.
The LTSI is pre-reversible.
As in Example~\ref{ex:halfcube mod}, CIRE does not hold
%; note that the $a$ and $b$-events are independent,
%but after performing $c$ there are coinitial $a$ and $b$-transitions
%which are not independent.
while both CL$\ci$ (hence BFCIRE) and CL$\indt$ hold.  %\todo{[can deduce CL$\indt$ from CL$\ci$+IEC] not really needed, in case we need to postpone this example}:
%note that at any state, all events that have occurred can be reversed
%immediately.
%IC \todo{Definition~\ref{def:coinitial LTSI}} holds---
All pairs of independent transitions are coinitial.
CS$\indt$ however does not hold:
% \todo{can deduce this from Proposition~\ref{prop:IC CSi}}
consider $t_0:P \tran c Q$, $r:Q \ptran {\rev a b} R$, $S \tran c R$---here we do not have $\rev {t_0} \ind (Q',b,R)$.\finex
\end{example}

%\begin{restatable}{proposition}{correspondence}\label{prop:correspondence}
\begin{proposition}\label{prop:correspondence}
Let $\mc L$ be a pre-reversible LTSI satisfying IEC.
If $\mc L$ satisfies CL$\ci$ then $\mc L$ satisfies CL$\indt$.
\end{proposition}
%\end{restatable}
\begin{proof}
Immediate from the definitions.
\end{proof}
%% \todo{IVAN:
%%   Dually
%% \begin{proposition}\label{prop:correspondence}
%% Let $\mc L$ be a pre-reversible LTSI satisfying IEC.
%% If $\mc L$ satisfies CS$\indt$ then $\mc L$ satisfies CS$\ci$.
%% \end{proposition}
%% %\end{restatable}
%% \begin{proof}
%% Immediate from the definitions.
%% \end{proof}
%% [Iain: but CS$\ci$ follows just from pre-reversible.]}

We next give an example where CC holds but not CS$\ci$ (and not PCI).
\begin{example}\label{ex:CC not CS}
%  \todo{IVAN: try to highlight the relevant items with colors or something similar}
  Consider the cube with transitions $a,b,c$ on the left in Figure~\ref{fig:diamonds},
where the forward direction is from left to right.
% Figure environment removed
We add independence as given by BTI.  So SP, BTI, WF hold, but not PCI.
Consider the bold path from the leftmost end:
%From the start
we have an $a$-transition followed by a path $r = bc$ followed by $\rev a$.
For CS$\ci$ to hold, we want $\rev a$ to be the reverse of the same event as the first $a$.
They are connected by a ladder with sides $cb$.
We add independence for all corners on the two faces of the ladder ($ac$ and $ab$). Transitions
$\rev b$ and $\rev c$ at $P$ are independent (by BTI) so we obtain $\rev b\rev c \ceqt \rev c \rev b$, where 
$\rev b\rev c$ is dashed and $\rev c \rev b$ is bold. Since $\ceqt$ is closed under composition, we get
$bc \ceqt cb$.
However the bold $b$ is a different event from the event of the top $b$s since the bold-dashed $bc$ face does not have independence at each corner.
Therefore we do not get $[a] \coind [b]$ for the bold $a$ and bold $b$, and CS$\ci$ fails. However, we note that we do have $[a] \coind [b]$ for the bold $a$ and the dashed $b$ since $a$ and $b$ at $Q$ are independent.
%\todo{Iain: I find this very hard to follow - perhaps some colours would help?}\finex
%\todo{Clearly this example is a bit artificial.
%Does the example show that our definition of event is reasonable or not?} 
\end{example}
%% \begin{example}%\label{ex:CC not CS}
%%   \todo{Example~\ref{ex:CC not CS} and diagram rephrased.}
%%     Consider the cube with twelve transitions on the left in Figure~\ref{fig:diamonds1},
%% where the forward direction is from left to right.
%% The four transitions $a_1,a_2,a_3,a_4$ have the same label $a$;
%% similarly for the $b_i$ and $c_i$ transitions.
%% % Figure environment removed
%% We add independence as given by BTI.  So SP, BTI, WF hold, but not PCI.
%%  From the start we have $a_1$ followed by a path $r = b_1c_1$ followed by $\rev {a_2}$.
%% % For CS$\ci$ to hold, we want
%% Let us make $\rev {a_2}$ into the reverse of the same event as $a_1$.
%% They are connected by a ladder with rungs $a_1,a_3,a_2$ and sides $c_2,b_2$ and $c_3,b_3$.
%% We add independence for all corners on the two faces of the ladder. % ($ac$ and $ab$).
%% Now we have $a_1 \sqeqt a_3 \sqeqt a_2$;
%% also $c_2 \sqeqt c_3$ and $b_2 \sqeqt b_3$.
%% Furthermore $[a_1] \coind [b_2]$ since $a_3 \ind b_3$.
%% Then we get $b_1c_1 \ceqt c_2b_2$ (independence at a single corner is enough).
%% However $b_1$ is a different event from $b_2$ since the $b_1c_2$ face does not have independence at each corner.
%% Therefore we do not get $[a_1] \coind [b_1]$, and CS$\ci$ fails.\finex
%% %\todo{Clearly this example is a bit artificial.
%% %Does the example show that our definition of event is reasonable or not?} 
%% \end{example}
%
We next give an example where CS$\ci$ and CL$\ci$ hold 
but not CC.
%\todo{I think the counterexample below holds using the same reasoning also for plain CS and CL, am I right?
%}

%\iu{
%IU: I think CS$\indt$ fails in the RHS diagram: $P_2 \tran a Q_1$ and $Q_1 \tran c Q_0$ are not independent 
%but we can reverse $a$ at $Q_0$. CL$\indt$ seems to hold there.}
%\todo{Iain: I agree.}

%\todo{CL$<$ trivially holds since there are no rooted paths, but CS$<$ does not for the same reason, hence this is not a counterexample for these.
%}

%\iu{
%IU: I am not sure why we consider rooted paths; isn't the property supposed to hold for all paths? 
%I think both CS$_<$ and CL$_<$ hold for RHS diagram in Fig.~2.}
%\todo{Iain: Since there are no rooted paths, the ordering relation on events is empty.
%Hence CS$_<$ holds trivially.
%However CL$_<$ fails: consider a path $ab$.  Then $a \not < b$,
%but $a$ cannot be reversed after $b$.}
\begin{example}\label{ex:CS+CL not CC}
Consider the LTSI with
$Q_i \tran b P_i$, $P_{i+1} \tran c P_i$, $Q_{i+1} \tran c Q_i$,
$P_{i+1} \tran a Q_i$ for $i = 0,1,\ldots$. This is shown on the right in Figure~\ref{fig:diamonds}.
Clearly WF does not hold.
% However AC and UT hold.
We add coinitial independence to make BTI and PCI hold.
Then also SP and CIRE hold.
% I think we can add independence so that
% BTI, SP, RPI, PCI and IRE hold. 
However, CC fails since, for example
$P_1 \tran a Q_0 \tran b P_0$ and $P_1 \tran c P_0$ are coinitial and cofinal but not causally equivalent.
Note that there are just three events $a,b,c$ with $a \coind c$,
$b \coind c$ but not $a \coind b$.
% NRE does not hold, but 
CS$\ci$ and CL$\ci$ hold. Indeed, $c$ is independent from every other action, and it can always be undone, while $a$ and $b$ are independent from $c$ only and they can be undone after any path composed by $c$ and no others.
In more detail, if we have a path $a r \rev a$ with $\cte(r,a) = 0$ then $\cte(r,b) = 0$,
and if we have a path $b r \rev b$ with $\cte(r,b) = 0$ then $\cte(r,a) = 0$.\finex
%, and $b$ is concurrent to $c$ only and can be undone after any path composed by $c$ and no others. 
%\todo{diagram might help}
\end{example}

The independence result in Proposition~\ref{prop:ind CIRE} holds also if we replace CIRE by BFCIRE.
\begin{proposition}\label{prop:ind BFCIRE}
The axioms 
SP, BTI, WF, PCI, BFCIRE and IEC are independent of each other.
\end{proposition}
\begin{proof}
%-short version-
%The proof is essentially the same as for Proposition~\ref{prop:ind CIRE},
%using the same examples and
%noting that CIRE implies BFCIRE, and that BFCIRE (equivalent to CL$\ci$) fails
%in Example~\ref{ex:IC CLi}.
%
%[longer version]
For each of the six axioms we need to give an LTSI which satisfies the other five axioms but not the axiom itself.
Since CIRE implies BFCIRE, for all axioms apart from BFCIRE we can reuse the examples given in the proofs of Proposition~\ref{prop:ind CIRE}
(and of Proposition~\ref{prop:ind}).
Example~\ref{ex:IC CLi} provides an LTSI where BFCIRE (equivalent to CL$\ci$) fails and the remaining five axioms hold.
\end{proof}

%\subsection{EIT - to be omitted}
%\input {EIT.tex}

\subsection{CS and CL via ordering of forward events}\label{sub:ord}
% \todo{`Event' changed to `forward event' throughout Section~\ref{sub:ord}.}
We now give definitions of causal safety and causal liveness 
%CS/CL 
using ordering on forward events.
To this end, we exploit the causality relation $\leq$ on such events (see Definition~\ref{def:ordering}).
\begin{definition}\label{def:safe live <}
Let $\mc L = (\Proc,\Lab,\tran{},\ind)$ be an LTSI.
\begin{enumerate}
\item
We say that $\mc L$ is \emph{ordered causally safe (CS$_<$)} if whenever
$t_0:P \tran a Q$, $r:Q \ptran \rho R$, $\cte(r,[t_0]) = 0$ and $t_0\op:S \tran a R$
with $t_0 \sqeqt t_0\op$,
then $[t_0] \not < e'$ for all forward events $e'$ such that $\cte(r,e') > 0$.
\item
We say that $\mc L$ is \emph{ordered causally live (CL$_<$)} if whenever
$t_0:P \tran a Q$, $r:Q \ptran \rho R$ and $\cte(r,[t_0]) = 0$ and
$[t_0] \not < e'$ for all forward events $e'$ such that $\cte(r,e') > 0$
then we have
$t_0\op:S \tran a R$ with $t_0 \sqeqt t_0\op$. 
%\todo{[Iain: might mean adjusting notation in statement and proof of
%  Lemma~\ref{lem:coind <} and proof of Proposition~\ref{prop:CS CL coind <}.]
%  IVAN: changed $e$ back to $e'$ given Iain observation}
\end{enumerate}
\end{definition}
%\todo{IVAN: Since here we consider only forward events, and require a positive number of occurrences, we have no way to check backward events with a positive number of occurrences. Should we check their complement?}
%\todo{IL: should we use $[t]$ instead of $e'$ in the definition so to have a more direct correspondence?
%}

%\iu{
%IU: I think we discussed this before and decided to keep this expression as it corresponds very closely to 
%$[P,a,Q] \coind e$ for CS$\ci$ and $(P,a,Q) \ind t$ for CS$\indt$.}
The only difference between CS$_<$ and CS$\indt$
(Definition~\ref{def:safe live})
is that the former ensures $[t_0] \not < [t]$ instead of $\rev{t_0} \ind t$ for all transitions $t$ such that $[t]$ has a positive number of occurrences in $r$. Similarly for CL. Notably, we do not require  $[\rev{t_0}] \not < [t]$ since $<$ is defined on forward events and $t_0$ is forward.

%% We postpone stating sufficient conditions for CS$_<$ and CL$_<$ until we have
%% discussed the relationship between independence, causation and conflict on events
%% (see the next subsection).


%% \subsection{Polychotomy}
%% In this section we relate our three versions of causal safety and liveness,
%% with the help of what we call \emph{polychotomy},
%% which states that if events do not cause each other and are not in conflict,
%% then they must be independent.

%% \begin{example}\label{ex:pre-prime not CS}
%% \todo{remove?}
%% Consider the LTSI in Figure~\ref{fig:repeated2}.
%% % Figure environment removed
%% % It satisfies pre-prime and FD but not CS (or SD, CL).
%% % We have a path $abc\rev a$ from $P$ to $S$, but
%% % (the event labelled with) $a$ causes $c$.
%% % Thus being pre-prime is not enough to guarantee CS.
%% We add independence to make BTI and PCI hold.
%% Both SP and WF hold, and so the LTSI is pre-reversible. Thus, CC holds as well.
%% There are three events, labelled with $a,b,c$.
%% Clearly NRE fails for both $a$ and $b$.
%% We see that $a < c$ but also $a \coind c$, so that polychotomy fails.
%% CS$\ci$ holds by Theorem~\ref{thm:CS coind}.
%% However CS$_<$ fails:
%% consider the transition $P \tran a Q$ together with the path $r:Q \ptran {bc} R$ and
%% $S \tran a R$, and note that $a < c$.\finex
%% \end{example}
It may seem that the definition above does not take into account backward events that may occur in $r$, but the next lemma shows that such events are necessarily independent from $[t_0]$.
This allows us to connect ordered safety and liveness
with %coinitial
safety and liveness based on independence of events.
%\begin{restatable}{lemma}{coindless}\label{lem:coind <}
\begin{lemma}\label{lem:coind <}
Suppose that an LTSI is pre-reversible.
Suppose $t_0:P \tran a Q$, $e = [t_0]$, $r:Q \ptran \rho R$ and $\cte(r,e) = 0$.
Let $e'$ be a forward event:
\begin{enumerate}
\item\label{item:greater}
if $\cte(r,e') > 0$ then exactly one of $e \coind e'$ and $e < e'$ holds;
\item\label{item:less}
if $\cte(r,e') < 0$ then $e \coind e'$.
\end{enumerate}
\end{lemma}
%\end{restatable}
\begin{proof}
We know that polychotomy holds by Proposition~\ref{prop:poly}.
Also NRE holds by Proposition~\ref{prop:NRE}.
Suppose $t_0:P \tran a Q$, $e = [t_0]$, $r:Q \ptran \rho R$ and $\cte(r,e) = 0$ and
$\cte(r,e') \neq 0$ where $e'$ is a forward event.
We first note that $e \neq e'$, since $\cte(r,e) = 0$ and $\cte(r,e') \neq 0$.
By WF, there is a rooted path $s$ from some irreversible $I$ to $P$.
% Using PL, the path $s$ from $I$ to $P$ can be taken to be forward-only.
\begin{enumerate}
\item
Suppose first that $\cte(r,e') > 0$.
Since $\cte(st_0r,e) > 0$ and $\cte(st_0r,e') > 0$ we do not have $e \cf e'$.
Furthermore, if $e' < e$ then we must have $\cte(s,e') > 0$,
so that $\cte(st_0r,e') > 1$, contradicting NRE.
Then the result follows by polychotomy.

\item
Now suppose that $\cte(r,e') < 0$.
By Proposition~\ref{prop:regeqzero} we must have $\cte(s,e') > 0$.
We deduce that $e \not< e'$.
Since $\cte(st_0,e) > 0$ and $\cte(st_0,e') > 0$ we do not have $e \cf e'$.
Furthermore $\cte(st_0r,e) > 0$ and $\cte(st_0r,e') = 0$ (since $\cte(st_0,e') = 1$ combining $\cte(st_0,e') > 0$ shown above and NRE).
Hence $e' \not< e$.
By polychotomy, $e \coind e'$.
\qedhere
\end{enumerate}
\end{proof}
%\todo{NB We could restate as two cases depending on whether
%$\cte(r,e') > 0$ or $\cte(r,e') < 0$.
%This would help with the proof of Proposition~\ref{prop:CS CL coind <},
%where we have to reach into the proof.}


%\begin{restatable}{proposition}{CSCLcoind}\label{prop:CS CL coind <}
\begin{proposition}\label{prop:CS CL coind <}
Suppose that an LTSI $\mc L$ is pre-reversible.
Then
\begin{enumerate}
\item
% $\mc L$ satisfies CS$\ci$ iff $\mc L$ satisfies CS$_<$.
$\mc L$ satisfies CS$_<$.
%\todo{Mismatch in definitions as to whether $\cte(r,e') < 0$ is allowed, but proof works.}
\item
$\mc L$ satisfies CL$\ci$ iff $\mc L$ satisfies CL$_<$.
\end{enumerate}
\end{proposition}
%\end{restatable}
\begin{proof}
\begin{enumerate}
\item
% $\mc L$ satisfies CS$\ci$ iff $\mc L$ satisfies CS$_<$.
% Suppose that CS$\ci$ holds.
We know CS$\ci$ holds by Theorem~\ref{thm:CS coind}.
Assume that $t_0:P \tran a Q$, $e = [t_0]$, $r:Q \ptran \rho R$, $\cte(r,e) = 0$ and $t_0\op:S \tran a R$
with $t_0 \sqeqt t_0\op$.
Take any forward $e'$ such that $\cte(r,e') > 0$.
By Lemma~\ref{lem:coind <} we know that exactly one of $e \coind e'$ or $e < e'$ holds.
By CS$\ci$ we have $e \coind e'$, and therefore $e \not< e'$ as required.

% Conversely, suppose that CS$_<$ holds.
% We know CS$\ci$ holds by Theorem~\ref{thm:CS coind} (without the use of NRE).
\item
% $\mc L$ satisfies CL$\ci$ iff $\mc L$ satisfies CL$_<$.
  Suppose that CL$\ci$ holds.
Assume that
$P \tran a Q$, $e = [t_0]$, $r:Q \ptran \rho R$ and $\cte(r,e) = 0$ and
$e \not < e'$ for all 
forward $e'$ such that $\cte(r,e') > 0$.
Let event $e'$ be such that $\cte(r,e') > 0$.
Suppose first that $e'$ is forward.
By assumption $e \not < e'$.
So by Lemma~\ref{lem:coind <}(\ref{item:greater}) we obtain $e \coind e'$.
Suppose instead that $e'$ is reverse, so that $\rev {e'}$ is forward,
and $\cte(r,\rev{e'}) < 0$.
By Lemma~\ref{lem:coind <}(\ref{item:less}) we obtain $e \coind \rev{e'}$,
and hence $e \coind e'$ using Lemma~\ref{lem:coind rev}.
We deduce that $e \coind e'$ for all $e'$ such that $\cte(r,e') > 0$.
Hence by CL$\ci$ we have
$t_0\op:S \tran a R$ with $t_0 \sqeqt t_0\op$.

Conversely, suppose that CL$_<$ holds.
Assume that
$P \tran a Q$, $e = [t_0]$, $r:Q \ptran \rho R$ and $\cte(r,e) = 0$ and
$e \coind e'$ for all 
$e'$ such that $\cte(r,e') > 0$.
By Lemma~\ref{lem:coind <}(\ref{item:greater}) we know that $e \not < e'$ for all forward $e'$ such that $\cte(r,e') > 0$.
Hence by CL$_<$ we have
$t_0\op:S \tran a R$ with $t_0 \sqeqt t_0\op$.
\qedhere
\end{enumerate}
\end{proof}


%% Property RED below is also related to NRE and polychotomy.
%% % \begin{definition}\label{def:ED RED}
%% \begin{definition}\label{def:RED}
%% % An LTSI satisfies \emph{event determinism (ED)} if whenever
%% % $t,t'$ are forward coinitial transitions and $t \sqeqt t'$ then $t = t'$.
%% \todo{Remove?}
%% An LTSI satisfies \textbf{Reverse Event Determinism (RED)} if whenever
%% $t,t'$ are backward coinitial transitions and $t \sqeqt t'$ then $t = t'$.
%% \end{definition}
%% % \begin{proposition}\label{prop:RED}
%% % If an LTSI satisfies SP, BTI, WF, PCI, ED then it satisfies RED.
%% % \end{proposition}
%% % \begin{proof}
%% % Immediate using BTI and SP.
%% % \end{proof}
%% % \begin{proposition}\label{prop:RED}
%% % If an LTSI satisfies SP, BTI, WF, PCI, NRE then it satisfies RED.
%% % \end{proposition}
%% % \begin{proof}
%% % Suppose that $t:P \tran{\rev a} Q$, $u:P \tran{\rev a} Q'$
%% % are backward coinitial transitions and $t \sqeqt u$.
%% % Suppose for a contradiction that $t \neq u$.
%% % By BTI we have $t \ind u$.
%% % We can use SP to complete a diamond with transitions $t' \sqeqt t$,
%% % $u' \sqeqt u$.
%% % All four transitions belong to the same reverse event.
%% % We get a forward path containing two occurrences of the same event,
%% % contradicting NRE.
%% % \end{proof}
%% %
%% %\begin{restatable}{proposition}{NRERED}\label{prop:NRE RED poly}
%% \begin{proposition}\label{prop:NRE RED poly}
%% \todo{Remove}
%% If an LTSI $\mc L$ is pre-reversible
%% then the following are equivalent:
%% \begin{enumerate}
%% \item $\mc L$ satisfies NRE;  
%% \item  $\mc L$ satisfies RED;
%% \item independence $\coind$ is irreflexive on events; and  
%% \item polychotomy holds.
%% \end{enumerate}
%% \end{proposition}
%% %\end{restatable}
%% \begin{proof}
%% Suppose that NRE holds; we show RED.
%% Suppose that $t:P \tran{\rev a} Q$, $u:P \tran{\rev a} Q'$
%% are backward coinitial transitions and $t \sqeqt u$.
%% Suppose for a contradiction that $t \neq u$.
%% By BTI we have $t \ind u$.
%% We can use SP and PCI to complete a diamond with transitions $t' \sqeqt t$,
%% $u' \sqeqt u$.
%% All four transitions belong to the same reverse event.
%% We get a forward path containing two occurrences of the same event,
%% contradicting NRE.

%% Suppose that RED holds; we show that $\coind$ is irreflexive.
%% Suppose for a contradiction that $e \coind e$ \todo{for some event $e$}.
%% Then there are coinitial transitions $t,u \in e$ such that $t \ind u$.
%% We can use SP to complete a square with $t' \sqeqt t$ and $u' \sqeqt u$.
%% This square is non-degenerate by Lemma~\ref{lem:non-degenerate}.
%% All transitions in the square belong to the same event.
%% Hence there are two distinct reverse coinitial transitions from the same event,
%% contradicting RED.

%% % Suppose that NRE holds.
%% % Now suppose for a contradiction that we have an event $e$ such that $e \coind e$.
%% % This means that there are coinitial transitions $t,u \in e$
%% % such that $t \ind u$.
%% % We can use SP to complete a square with $t' \sqeqt t$ and $u' \sqeqt u$.
%% % We get a forward path containing two occurrences of the same event,
%% % contradicting NRE.

%% Suppose that $\coind$ is irreflexive; we show NRE.
%% Let $r$ be a rooted path from $I$ to $R$, and suppose for a contradiction that
%% $\cte(r,e) > 1$.
%% Using PL we can obtain a forward-only path $r'$ from $I$ to $R$ with $r \ceqt r'$.
%% By Lemma~\ref{lemma:cccount}, $\cte(r',e) > 1$.
%% Suppose $r'$ contains
%% $t:P \tran a Q$ followed later by $t':P' \tran a Q'$ where $t, t' \in e$.
%% Let $r''$ be the portion of $r'$ from $Q$ to $P'$.
%% By Lemma~\ref{lem:ladder} there is a path $s$ from $Q$ to $Q'$ such that for all $u$ in $s$
%% % we have $t \sqeqt t'' \ind u' \sqeqt u$ (some $t'',u'$).
%% we have $[t] \coind [u]$.
%% By CC, $s \ceqt r''t'$.
%% By Lemma~\ref{lemma:cccount}, $\cte(s,[t']) > 0$, since $r''$ is forward-only.
%% Hence there is $u$ in $s$ such that $u \sqeqt t' \sqeqt t$.
%% % But then $t \sqeqt t'' \ind u' \sqeqt t$,
%% % where $t'',u'$ are as above.
%% But then $[t] \coind [u] = [t]$,
%% contradicting our assumption that $\coind$ is irreflexive.

%% Suppose that NRE holds.
%% Then polychotomy holds by Proposition~\ref{prop:poly}.

%% Suppose that polychotomy holds.
%% Then since $e = e'$ and $e \coind e'$ are mutually exclusive,
%% $\coind$ must be irreflexive.
%% \end{proof}

%% We now give a criterion to show NRE.
%% %\begin{restatable}{proposition}{CIRENRE}\label{prop:CIRE NRE}
%% \begin{proposition}\label{prop:CIRE NRE}
%% \todo{Remove}
%% Suppose that a pre-reversible LTSI satisfies CIRE.
%% Then it also satisfies NRE.
%% \end{proposition}
%% %\end{restatable}
%% % \begin{proof}[Superseded by alternative proof]
%% % Let $r$ be a rooted path from $I$ to $R$, and suppose for a contradiction that
%% % $\cte(r,e) > 1$.
%% % Using PL we can obtain a forward-only path $r'$ from $I$ to $R$ with $r \ceqt r'$.
%% % By Lemma~\ref{lemma:cccount}, $\cte(r',e) > 1$.
%% % Suppose $r'$ contains
%% % $t:P \tran a Q$ followed later by $t':P' \tran a Q'$ where $t, t' \in e$.
%% % Let $r''$ be the portion of $r'$ from $Q$ to $P'$.
%% % By Lemma~\ref{lem:ladder} there is a path $s$ from $Q$ to $Q'$ such that for
%% % all $u$ in $s$
%% % % we have $t \sqeqt t'' \ind u' \sqeqt u$ (some $t'',u'$).
%% % we have $[t] \coind [u]$.
%% % By CC, $s \ceqt r''t'$.
%% % By Lemma~\ref{lemma:cccount}, $\cte(s,[t']) > 0$, since $r''$ is forward-only.
%% % Hence there is $u$ in $s$ such that $u \sqeqt t' \sqeqt t$.
%% % % But then $t \sqeqt t'' \ind u' \sqeqt t$,
%% % % where $t'',u'$ are as above.
%% % But then $[t] \coind [u] = [t]$.
%% % Hence $t \ind t$ by CIRE, contradicting $\ind$ being
%% % irreflexive.
%% % \end{proof}
%% \begin{proof}% [Alternative proof]
%% By Proposition~\ref{prop:NRE RED poly} it is enough to show that $\coind$ is irreflexive.
%% Suppose that $e \coind e$ for some event $e$.
%% Take any transition $t \in e$.
%% Then $t$ is coinitial with itself, and so by CIRE we have $t \ind t$,
%% which contradicts irreflexivity of $\ind$.
%% \end{proof}

%% The opposite implication does not hold.
%% \begin{example}\label{ex:NREnotCL revisited}
%% \todo{
%% % \todo{May have to be omitted for space reasons,
%% % but this example is less pathological
%% % than the ones with repeated events.}
%% Consider the pre-reversible LTSI of Example~\ref{ex:IC CLi} shown in
%Figure~\ref{fig:IC CLi}.
%% There are three events, labelled $a,b,c$, which are all independent of each other.
%% We see that NRE holds (hence CS$_<$ holds as well) but, as previously observed, not CIRE.
%% We previously observed that CL$\ci$ fails;
%% so does CL$_<$ in a similar fashion: consider $P \tran a Q \tran b R$,
%% where $a$ cannot be reversed at $R$, even though $a \not< b$.\finex
%% }
%% \end{example}


% \begin{example}\label{ex:LED}
% Consider the LTSI in Figure~\ref{fig:LED1}.
% % Figure environment removed
% Independence is given by closing under BTI and PCI.
% There are just two events, labelled with $a$ and $b$ respectively,
% which are concurrent.
% We see that axioms SP, BTI, WF, PCI, CIRE hold.
% Therefore also NRE and RED hold.
% However ED fails.
% \end{example}
% \begin{remark}
% Note that CL does not necessarily hold if the initial transition is backward rather than forward.
% Consider the LTSI in Example~\ref{ex:LED} where ED fails.
% Add a $c$ transition starting at $P$.
% Then if we perform $c$ followed by $r = \rev b b$,
% we have $\cte(r,b) = 0$,
% but we get to state $Q$ where $c$ is impossible.
% \todo{If we assume ED can we prove CL for reverse transitions?}
% \end{remark}

%\subsection{FCL$_<$ - to be omitted}
%\input {FCL.tex}

%\subsection{Comparing the different forms of CS/CL}\label{sub:comparing}

%% \todo{Table~\ref{tab:implications} can now be dropped as all open implications resolved.}
%% \begin{table}[t!]
%% \begin{center}
%% \begin{tabular}{|c|c|p{0.5\linewidth}|}
%% \hline
%% From & To & Comments \\
%% \hline
%% LG+IEC & none & Ex.~\ref{ex:LG+IEC} \\
%% CLG & none & Ex.~\ref{ex:prerev not CSi} \\
%% LG & none & Ex.~\ref{ex:LG} \\
%% IRE+IEC & none & Ex.~\ref{ex:IRE+IEC} \\
%% IC+CIRE & none & Ex.~\ref{ex:IC+CIRE} \\
%% CIRE+IEC & none & Ex.~\ref{ex:IRE+IEC} and Ex.~\ref{ex:IC+CIRE} \\
%% IC+CL$\ci$ & none & Ex.~\ref{ex:halfcube} \\
%% IRE & none & Ex.~\ref{ex:IRE1} or Ex.~\ref{ex:IRE2} \\
%% IEC+CL$\ci$ & none & Ex.~\ref{ex:halfcube} and Ex.~\ref{ex:IRE+IEC} \\
%% IC & none & Ex.~\ref{ex:IC} \\
%% CS$\indt$ & \todo{none} & Ex.~\ref{ex:prerev not CL}, Ex.~\ref{ex:CSi+RPI CLi} and \todo{Ex.~\ref{ex:halfcube mod}} \\
%% CIRE & none & Ex.~\ref{ex:prerev not CL}, Ex.~\ref{ex:CSi+RPI CLi} and Ex.~\ref{ex:IC+CIRE}  \\
%% CL$\indt$ & none & Ex.~\ref{ex:CSi+RPI CLi} and Ex.~\ref{ex:IC CLi} \todo{not needed: Ex.~\ref{ex:IRE1} or Ex.~\ref{ex:IRE2}} \\
%% CL$\ci$ & none & Ex.~\ref{ex:prerev not CL}, Ex.~\ref{ex:CSi+RPI CLi} and Ex.~\ref{ex:halfcube} \\
%% IEC & none & Ex.~\ref{ex:IC} and Ex.~\ref{ex:IRE+IEC} \\
%% \todo{ECh} & none & Ex.~\ref{ex:prerev not CL} and Ex.~\ref{ex:IC} \\
%% \hdashline
%% EIT & \todo{none} & Ex.~\ref{ex:prerev not CL}, Ex.~\ref{ex:CSi+RPI CLi} and Ex.~\ref{ex:halfcube} \\
%% CS$\indt$+RPI & CL$\indt$ & Ex.~\ref{ex:CSi+RPI CLi}\\
%% IRE+RPI & none & Ex.~\ref{ex:LG} and Ex.~\ref{ex:IRE+IEC}  \\
%% RPI & (none) & could use Ex.~\ref{ex:IC} closed under RPI, and Ex.~\ref{ex:CSi+RPI CLi}\\
%% \hline
%% \end{tabular}
%% \end{center}
%% \caption{Open implications - for our information.}
%% \label{tab:implications}  
%% \end{table}
\subsection{Implications between the different formalisations of CS/CL}\label{sub:comparing}

% Figure environment removed
%[Iain: no, 
%  and~\ref{ex:IC+CIRE}.

We have introduced three different formalisations of causal safety and liveness.  The implications between them, assuming pre-reversibility holds, %the different versions
are shown in Figure~\ref{fig:simpleCSCL}.

As can be seen in Table~\ref{t:list}, only two causal safety properties, namely CS$\ci$ and CS$_<$, hold for
pre-reversible LTSIs.  The causal liveness versions of these properties, namely CL$\ci$ and CL$_<$, additionally require BFCIRE. %CIRE. 
%We also know that its restricted version
Actually, BFCIRE is equivalent to both CL$\ci$ and CL$_<$. The last two properties,
CS$\indt$ and CL$\indt$, which are defined over general independence of transitions, require  IRE. No other implications hold beyond those shown. Counterexamples for lack of other implications
%between some causal properties
in Figure~\ref{fig:simpleCSCL} are pointed to in Figure~\ref{fig:diagprerev1}.

We postpone discussion of which particular version of CS or CL is most relevant in a specific setting until Section~\ref{subsec:comparison}, after we have introduced some structural axioms to better relate them.

\Comment{
\todo{We have introduced three different formalisations of causal safety and liveness.  The implications between the different versions are shown in Figure~\ref{fig:simpleCSCL}, assuming pre-reversibility holds.
The axioms IRE and CIRE which can be used to show CS$\indt$ and the three forms of CL are also shown.
We postpone discussion of which particular version of CS or CL is most relevant in different settings until Section~\ref{subsec:comparison}, after we have introduced some structural axioms.}

%\todo{IU: I am not sure if it is the right place for this subsection. It could appear after section 5}
As can be seen from Table~\ref{t:list}, only two causal safety properties, namely CS$\ci$ and CS$_<$, hold for
pre-reversible LTSI. \il{Instead, we required CIRE (or BFCIRE) to prove CL$\ci$ and CL$_<$, and IRE to prove CS$\indt$ and CL$\indt$.

These are needed: we show that pre-reversible is not enough for
CS$\indt$ in Example~\ref{ex:prerev not CSi}, for CL$\indt$ in
Example~\ref{ex:prerev not CL}, and for CL$\ci$ in Example~\ref{ex:IC
  CLi}. Thanks to Proposition~\ref{prop:CS CL coind <} it is not
enough for CL$_<$ either (indeed in Example~\ref{ex:IC CLi}, $P \tran
a Q \tran b R$, $a \not < b$ but $a$ cannot be reversed at $R$).

Also, CIRE holds in Example~\ref{ex:prerev not CL}, hence CIRE would
not be enough for CL$\indt$. Similarly, CIRE would not be enough for CS$\indt$
in view of Example~\ref{ex:prerev not CSi}. Indeed, CIRE trivially holds since all pairs of coinitial transitions are independent.

Hence, we can divide the notions in three layers, which require stronger and stronger conditions to be proved (remember that we stick to pre-reversible LTSIs, which is the most basic setting where these notions can be defined):

CS$\ci$, CS$_<$ $\nRightarrow$ CL$\ci$, CL$_<$ $\nRightarrow$ CS$\indt$, CL$\indt$

Note that this does not mean that notions which require stronger
conditions imply the less demanding, however of course notions which
require weaker conditions cannot imply more demanding ones. However,
trivially, all the notions imply CS$\ci$ and CS$_<$. Also, thanks to
Proposition~\ref{prop:CS CL coind <} CL$\ci$ and CL$_<$ are
equivalent.  At the contrary, CS$\indt$ and CL$\indt$ are not
comparable. Indeed, in Example~\ref{ex:prerev not CSi} CL$\indt$ holds but
CS$\indt$ fails. Dually, in Example~\ref{ex:prerev not CL} CS$\indt$
holds but CL$\indt$ fails. Hence we can refine the comparison above into:

CS$\ci$, CS$_<$ $\nRightarrow$ CL$\ci$ $\Leftrightarrow$ CL$_<$ $\nRightarrow$ CS$\indt$ $\nLeftrightarrow$ CL$\indt$

which is graphically represented in Figure~\ref{fig:simpleCSCL}.
% Figure environment removed
%[Iain: no, 
%  and~\ref{ex:IC+CIRE}.
}

\todo{TO FINISH}




%By remembering that IRE implies CIRE

%% Also, one can notice that, \il{as discussed in Example~\ref{ex:IC CLi},} in Figure~\ref{fig:IC CLi} CS$_<$ holds
%% but not CL$_<$, CL$\ci$. By adding $(P,a,Q) \ind (Q,b,R)$ we get that
%% CL fails as well. By taking the same diagram with reversed arrows and
%% not adding $(P,a,Q) \ind (Q,b,R)$ we get a diagram where CS$_<$ holds
%% but CS$\indt$ fails since we can undo $b$ after $ac$.
%% \todo{The inverse diagram is not pre-reversible since the last $c$ and $b$ need to form a square and do not.}

%% Hence we can rewrite the sequence above as:

%% CS$\ci$ $\nRightarrow$ CS$_<$ $\nRightarrow$ CL$_<$, CL$\ci$ $\leq$ CS$\indt$, CL$\indt$

%CS and CL are not derivable from CC;
%we give an example LTSI which satisfies CC but not CS and not CL.
%\begin{example}[]\label{}
%Consider the LTS in Figure~\ref{fig:repeated1}.
%% % Figure environment removed

%% \todo{IVAN: drop next Example? Or move to extra?}
%% 	\iu{IU: In the LTSI in Figure~\ref{fig:noED}, independence is given by closing under BTI and PCI. 
%% There are just two events $a$ and $b$, and the property Event Determinism (that we do not consider here) 
%% fails. Also, IRE fails.  I think that CS$_<$, CL$_<$, CS$_{ci}$ and CL$_{ci}$ hold. 
%% However, CS$\indt$ fails as the initial $a$ and the following $b$s are not independent \todo{IVAN: not true any more with new def of CS$\indt$}. 
%% 	But CL$\indt$ holds.\vspace{0.5cm}
%% }
%% \todo{Iain: The LTSI in Figure~\ref{fig:noED} satisfies CIRE+IEC but not IRE.
%% I think it has the same properties as Example~\ref{ex:CLG CSi}.}

\todo{I would merge the paragraph below with the one at the end of the next section. All in all, it seems the general idea is: "Use whatever it is easiest to define in your setting"}        
\iu{IU: What follows is an initial draft of a discussion about which of the three pairs 
	a user might wish to adopt for her reversible formalism.


One may ask which of the three versions of causal safety and liveness properties should be adopted for
a given reversible formalism. This depends mainly on whether or not a suitable independence relation 
can be easily defined, and then whether or not this relation is for coinitial transitions. The analysis 
of our case studies and other reversible formalisms shows that, regardless of independence, 
the notions of events and a causal ordering on events are universal. In such settings, 
it is natural to assume that events do not occur
multiple times during individual computations, namely NRE holds. We are not aware of any formalism 
for concurrent computation where NRE fails. We have shown that pre-reversible LTSIs for formalisms
with coinitial independence, where NRE also holds,
satisfy CS$\ci$ and CL$\ci$ \todo{The latter requires CIRE unless I missed something}. We conclude that in such settings, CS$\ci$ and CL$\ci$ are 
the weakest desirable properties, as we do not need any further, potentially useful axioms to \il{hold}. 
If we can additionally guarantee CIRE, then the finer properties CS$_<$ and CL$_<$ will
be more desirable. 

When it is more natural to define independence on general transitions, as for Petri or occurrence
nets, then IRE is a desirable property to have. Consequently, we can use CS$\indt$ and CL$\indt$, which hold for
pre-reversible LTSIs with IRE. Alternatively, we can work purely with events and use CS$_<$ and CL$_<$.

If an independence relation is not easily defined, but there are well understood notions of events
and \il{causal ordering} on events, then CS$_<$ and CL$_<$ are probably the preferred properties.}
}
