\section{Case Studies}\label{sec:casestudies}
We look at whether our axioms hold in various reversible formalisms.
Given that we consider a high number of formalisms, we do not provide full background on them, but refer for it to the original papers. Also, we sometimes repeat similar observations for different formalisms, so to make it possible to browse them out of order, to find information on a specific formalism of interest. 
Remarkably, all the works below provide proofs of the Loop Lemma.

\subsection{RCCS}\label{sec:rccs}
We consider here the semantics of RCCS in~\cite{DK04}, and restrict the attention to coherent processes~\cite[Definition 2]{DK04}. 
In RCCS, transitions $P \tran {\mu:\zeta} Q$ and $P \tran {\mu':\zeta'} Q'$
are concurrent if $\mu \cap \mu' = \emptyset$ \cite[Definition~7]{DK04}.
This allows us to define coinitial independence as
% $P \tran {\mu:\zeta} Q \ind P \tran {\mu':\zeta'} Q'$
% iff $\mu \cap \mu' = \emptyset$.
$t \ind u$ iff $t$ and $u$ are concurrent.
We now argue that the resulting coinitial LTSI is pre-reversible and
also satisfies CIRE. SP was shown in~\cite[Lemma 8]{DK04}.
BTI was shown in the proof of~\cite[Lemma 10]{DK04}.
WF is straightforward, noting that backward transitions decrease memory size.
Hence, we obtain a very much simplified proof of CC.
For PCI and CIRE we note that CLG holds
%independence is defined on the underlying labels
and thus Proposition~\ref{prop:CLG} applies.
% We can also obtain CS$_<$ and CL$_<$ by Proposition~\ref{prop:CS CL coind <}.
Therefore CS$\ci$ and CL$\ci$ hold.
Using Proposition~\ref{prop:gen coinit},
we can get an LTSI with general independence satisfying IRE and IEC,
and therefore CS$\indt$ and CL$\indt$.
% If we then apply $g$ to $\ind$ (Definition~\ref{def:gen coinit}), we can get a general pre-reversible
% LTSI which satisfies IRE and IEC by Proposition~\ref{prop:gen coinit}. Finally, we  get CS and CL which
% coincide with CS$_<$ and CL$_<$ respectively;
This is the first time these causal properties have been proved for RCCS.

\Comment{
In RCCS, transitions $P \tran {\mu:\zeta} Q$ and $P \tran {\mu':\zeta'} Q'$
are concurrent if $\mu \cap \mu' = \emptyset$.
We can generalise to non-coinitial transitions:
\begin{definition}[Independence for RCCS]\label{def:ind RCCS}
We say that two memories $m$ and $m'$ are coherent~\cite[Def.~1]{DK04}
iff they have a common initial portion followed by a fork on different
branches.

Then $P \tran {\mu:\zeta} Q \ind P' \tran {\mu':\zeta'} Q'$ iff
$\forall m \in \mu.\forall m' \in \mu'$ we have that $m$ and $m'$ are
coherent.
\end{definition}
If we restrict the attention to processes reachable by a process with
empty memories (that is, a CCS process), as done in~\cite{DK04}, then
all different memories in a process $P$ are pairwise
coherent~\cite[Last lines of Section 2]{DK04}.

Hence, by considering coinitial transitions, we have that $P \tran
{\mu:\zeta} Q \ind P \tran {\mu':\zeta'} Q'$ iff $\mu \cap \mu' =
\emptyset$, matching the definition of concurrency
in~\cite[Def.~7]{DK04}.

Let RCCS$\ind$ be RCCS with general independence.
We get an LTSI $\mc L = (\Proc,\Lab,\tran{},\ind)$.
\begin{conjecture}\label{conj:RCCSi}
$\mc L$ satisfies SP, BTI, WF, PCI, IRE, IEC and ED.
\todo{ED not defined - might be best to remove}
\end{conjecture}
\begin{proof}
SP was shown in~\cite[Lemma 8]{DK04}.
BTI was shown in the proof of~\cite[Lemma 10]{DK04}.
% PI and RPI follows directly from the definition of independence.
WF is straightforward noting that forward transitions increase memory size.
PCI and IRE look straightforward, since independence is defined on labels.
Rough idea for IEC:
if the memories are $m_1\cell{1}m$ and $m_2\cell{2}m$ then starting at
$m_1\cell{1}m$ carry out the $m_2$ transitions to get to a state where
both events are enabled.
As far as ED is concerned, it might be enough to show LED.
\end{proof}

% \todo{Material moved from Section~\ref{sec:coinitial}:
% Question: does RCCS satisfy IEC?
% Suppose that we define two transitions to be independent along the lines of
% Definition~\ref{def:ind RCCS}.
% Then the two transitions might be entirely unrelated.
% We can add to the definition of independence the requirement that there is a path from
% one source process to the other (which amounts to having a common irreversible ancestor).
% Rough idea for IEC: if the memories are $m_1\cell{1}m$ and $m_2\cell{2}m$ then starting at
% $m_1\cell{1}m$ carry out the $m_2$ transitions to get to a state where both events are enabled.
% }

\begin{conjecture}\label{conj:RCCS ind conc}
Let $\co$ be the concurrency relation on coinitial transitions in RCCS
as in~\cite{DK04}.
Let $g$ be the mapping of Definition~\ref{def:gen coinit}.
Then ${\ind} = g(\co)$, where $\ind$ is as in Definition~\ref{def:ind RCCS}.
\end{conjecture}
\begin{proof}
\todo{To be supplied. Essentially the same as showing that RCCS$\ind$ satisfies IEC.}
\end{proof}


For any simple CCS process $P$ in RCCS$\ind$, let $\Proc_P$ be the states
which are forwards reachable from $P$ using $\ftran{}$.
We note that $\Proc_P$ is closed under reverse transitions $\rtran{}$,
since $P$ is irreversible and RCCS$\ind$ satisfies PL.
\begin{conjecture}\label{conj:RCCSi oTSI}
$\mc L_P = (\Proc_P,P,\Lab,\ftran{},\ind)$ is an oTSI with initial state $P$
which also satisfies property (E).
\end{conjecture}
\begin{proof}
Immediate from Conjecture~\ref{conj:RCCSi}.
\end{proof}
It follows from~\cite[Cor~4.28]{SNW96} that $\mc L_P$ is equivalent to a
labelled prime event structure.

\todo{Perhaps we do not want to consider ED in the present work,
partly since that increases the overlap with ~\cite{PU07a}.}

Once SP and BTI are shown (already done in~\cite{DK04})
the remaining axioms WF, PCI and CIRE are straightforward to show,
noting that concurrency is defined on transition labels
and using Proposition~\ref{prop:underlying}.
We obtain a very much simplified proof of CC,
plus we show CS and CL for the first time.

We can use the method of Remark~\ref{rem:coinit to gen} to obtain a general
LTSI satisfying WF, SP, BTI, PCI, IRE, IEC.

\begin{remark}
The proof of CC in~\cite{DK04} uses EFP as a lemma~\cite[Lemma 11]{DK04}.
In our approach this becomes a simple consequence of CC.
\end{remark}
%end of Comment
}


  \subsection{CCSK}
The first notion of independence for CCSK~\cite{PU07} was given in~\cite{Aub22}. It is based on the proved transition system approach where transition labels
contain information about derivation of transitions. This information can be used to work out whether transitions are in conflict, causally dependent, or concurrent. Two forms of independence are defined in~\cite{Aub22}: general independence (called composable concurrency) and coinitial independence (called coinitial concurrency). CC is then obtained using our axiomatic approach (following~\cite{LanesePU20}, the conference version of the present paper) by showing SP \cite[Theorem~3]{Aub22}, BTI \cite[Lemma~6]{Aub22} and WF \cite[Lemma~7]{Aub22}.

%\iu{Some adjustment needed here as a result of discussions with Iain and Clément.}
Since coinitial independence is defined on labels, we can
%set $I(a,b)$, for labels $a$ and $b$, to hold if $a\smile b$~\cite[Definition 11]{Aub22} and
deduce that the %generated
LTSI is CLG. Hence, by Proposition~\ref{prop:CLG}, PCI and CIRE hold. This allows us to obtain CS$\ci$ and CL$\ci$.
Using Proposition~\ref{prop:gen coinit},
we can get an LTSI with general independence which satisfies IRE and IEC, which gives us CS$\indt$ and CL$\indt$ as well. 
As for RCCS, this is the first time such causal properties have been 
proved for CCSK.

\Comment{
Alternatively, coinitial independence can be defined for CCSK original labels as follows. We refer to~\cite{PU06,PU07} for the description of CCSK syntax and semantics.
\iu{Coinitial CCSK transitions $P\tran{\alpha[m]} Q$ and $P\tran {\beta[n]} R$ (forward or reverse)
% with $\alpha[m]\neq \beta[n]$, 
are \emph{independent} if and only if one of the conditions below holds:
\begin{enumerate}
\item $P\equiv U\Par V$, and $U\tran{\alpha[m]} U'$ with $V\tran {\beta[n]} V'$, 
	where $Q\equiv U'\Par V$ and $R\equiv U\Par V'$; 
\item $P\equiv c[k].U$ with $k\neq m,n$, and $U\tran{\alpha[m]} U'$ and  $U\tran {\beta[n]} U''$ 
	are independent, where $Q\equiv c[k].U'$ and $R\equiv c[k].U''$;
\item $P\equiv U+V$, and $U\tran{\alpha[m]} U'$ and $U\tran {\beta[n]} U''$ 
	are independent, where $Q\equiv U'+V$ and $R\equiv U''+V$;
\item $P\equiv U\setminus c$, and $U\tran{\alpha[m]} U'$ and $U\tran {\beta[n]} U''$ are independent, 
        where $\alpha, \beta\neq c$, $Q\equiv U' \setminus c$ and $R\equiv U''\setminus c$.
\end{enumerate}
This allows us to define \il{an} LTSI for CCSK with $\ind$ being this independence relation. Note that 
$a\Par a \tran{a[m]}  a[m]\Par a$ and $a\Par a \tran{a[n]}  a\Par a[n]$  are independent, but although $a.a$ has the
same initial transitions $a.a\tran{a[m]} a[m].a$ and $a.a\tran{a[n]} a[n].a$, 
they are not independent as 
they do not originate from different sides of a parallel composition. 

BTI, SP and PCI follow by induction on the structure of CCSK processes. 
For WF we note that when CCSK processes 
compute, their structure remains the same modulo addition (or removal) of a key or a pair of keys 
during each transition. Starting from an irreversible process (standard process in CCSK terminology), any derivative process will only 
have finitely many keys, hence WF is satisfied. As a result, the LTSI for CCSK is pre-reversible, 
and we obtain PL and CC by applying our axiomatic approach. 

In CCSK, unlike for RCCS, independence cannot be defined purely on the underlying labels, so we cannot use
Proposition~\ref{prop:CLG} to obtain CIRE. Instead, we can prove it by structural induction. 
This would  give us CS$_<$ and CL$_<$. Finally, using Proposition~\ref{prop:gen coinit},
we can obtain a notion of general independence which satisfies IRE and IEC,
and therefore CS$\indt$ and CL$\indt$. 
}
}

\subsection{HO$\pi$}\label{sec:hopi}
We consider here the uncontrolled reversible semantics for HO$\pi$~\cite{LaneseMS16}. 
We restrict our attention to reachable
processes, called there consistent.
The semantics is a reduction semantics; hence there are no labels (or, equivalently, all
the labels coincide). To have more informative labels we
can consider the transitions defined in~\cite[Section~3.1]{LaneseMS16},
where labels contain the memory created or consumed by the transition
  (they also contain a flag distinguishing backward from forward transitions, but this plays no role in the definition of the concurrency relation discussed below, hence we can safely drop it). 
%and a flag denoting whether the transition is forward or backward.
The notion of independence would be given by the concurrency relation on coinitial
transitions~\cite[Definition 9]{LaneseMS16}.
%We remark that the Loop Lemma holds~\cite[Lemma 6]{LaneseMS16}.
All pre-reversible LTSI axioms hold, as well as CIRE. 
Specifically, SP is proved in~\cite[Lemma 9]{LaneseMS16}. BTI holds since distinct memories have disjoint
sets of keys~\cite[Definition 3 and Lemma 3]{LaneseMS16} and by
the definition of concurrency~\cite[Definition 9]{LaneseMS16}.
WF holds as each backward step consumes a memory, which are a finite number to start with.
Finally, PCI and CIRE hold since CLG holds for the LTSI with annotated labels
%the notion of concurrency is defined on the
%annotated labels
and using our Proposition~\ref{prop:CLG}.
%
\Comment{
\begin{description}
\item[SP:] proved in~\cite[Lemma 9]{LaneseMS16};
\item[BTI:] since distinct memories have disjoint
  sets of keys~\cite[Definition 3 and Lemma 3]{LaneseMS16} and by 
  the definition of concurrency~\cite[Definition 9]{LaneseMS16};
% \item[PI:] does not apply since the notion of concurrency only considers coinitial transitions;
% \item[RPI:] does not apply since the notion of concurrency only considers coinitial transitions;
\item[WF:] since each backward step consumes a memory;
%\item[UT:] holds trivially under the reduction semantics, holds also
%  under the annotated semantics, since the label coincides with the memory;
\item[PCI, CIRE:] since the notion of concurrency is defined on the
  annotated labels and using our Proposition~\ref{prop:underlying}. 
%\item[CIRE:] since the notion of concurrency is defined on the
%  annotated labels.
%\item[IRE:] does not apply since the notion of concurrency only
  %considers coinitial transitions.
\end{description}
% Since SP, BTI, WF, and PCI hold, we obtain a very much simplified proof of CC. 
% We also get CS$_<$ and CL$_<$ and, applying the mapping $g$ from Section~5, we obtain
% a pre-reversible LTSI satisfying IRE and IEC.  This gives us CS and CL.
%
}

As a result we obtain a very much simplified proof of CC.
Moreover, using PCI and CIRE, we get the CS$\ci$ and CL$\ci$ safety and liveness properties and, 
applying mapping $g$ from Section~\ref{sec:coinitial}, we get a general 
pre-reversible LTSI satisfying IRE and IEC, so that CS$\indt$ and CL$\indt$ are satisfied. This is the first time
that causal properties have been shown for HO$\pi$.

\subsection{R$\pi$}\label{sec:pi}
We consider the (uncontrolled) reversible semantics for
$\pi$-calculus defined in~\cite{CristescuKV13}. We restrict the
attention to reachable processes. The semantics is an LTS
semantics.
Independence is given as concurrency which is defined for consecutive transitions~\cite[Definition
  4.1]{CristescuKV13}.  CC holds~\cite[Theorem~4.5]{CristescuKV13}.

Our results are not directly applicable to R$\pi$,
% since concurrency is defined for
% consecutive transitions, rather than coinitial transitions or pairs of transitions
% in general, so that none of .
since SP holds up to label equivalence of transitions on opposite sides
of the diamond,
rather than equality of labels as in our approach.
We would need to extend axiom SP and the definition of causal equivalence to allow for label equivalence in order to directly handle R$\pi$ using our axiomatic method.

We can however apply our theory to an LTSI obtained by considering labels up-to the equivalence relation $=_\lambda$~\cite[just before Lemma 4.3]{CristescuKV13}, which intuitively avoids to observe when a name is being extruded.
Notice that the Loop Lemma holds in this new LTSI as well.
However, the concurrency relation is given on consecutive transitions, and the same for their SP. Nevertheless, we can define independence as follows: $t \ind_\pi u$ iff $t$ and $u$ are coinitial and $t$ and $\rev u$ are concurrent. Notice that since $t$ and $u$ are coinitial then $t$ and $\rev u$ are consecutive. 
\begin{lemma}
  $\ind_\pi$ is symmetric.
\end{lemma}
\begin{proof}
  We have to show that $t$ and $\rev u$ are concurrent iff $\rev t$ and $u$ are concurrent. Since concurrency is defined as the complement of structural causality and contextual causality~\cite[Definition 4.1]{CristescuKV13}, it is enough to prove that $t$ and $\rev u$ are structural or contextual causal iff $\rev t$ and $u$ are. For structural causality, it follows from the definition~\cite[Definition 4.1]{CristescuKV13}. For contextual causality, it follows from~\cite[Proposition 4.2]{CristescuKV13}.
\end{proof}

With this definition of independence SP holds~\cite[Lemma 4.3]{CristescuKV13}. WF
holds as well since each backward step consumes at least a memory.
BTI has been proved as part of the proof of PL in~\cite[Lemma
  14]{CristescuPhD}.  As a result we obtain a proof of CC much simpler
than the one in~\cite[Theorem 11]{CristescuPhD} (note that causal
equivalence in~\cite[Definition 4.4]{CristescuKV13} is formalised
up-to $=_\lambda$ as well).

Independence is coinitial by construction.
We have to prove PCI and CIRE. Unfortunately, we cannot exploit CLG, since it does not hold, as is clear from the definition of structural cause~\cite[Definition 4.1]{CristescuKV13}, one of the ingredients of the concurrency relation. Thus we need to go for a direct proof. 
%\todo{CLG does not hold, presumably - could point this out.}

\begin{lemma}
  CIRE holds in the LTSI for R$\pi$.
\end{lemma}
\begin{proof}
  Concurrency is defined as the complement of structural causality and
  contextual causality~\cite[Definition
    4.1]{CristescuKV13}. Contextual causality is defined on
  labels~\cite[Proposition 4.2]{CristescuKV13}. Structural causality
  depends on whether the $i$ components of the two labels occur in the
  same memory in a specific relation~\cite[Definition
    2.2]{CristescuKV13}. However, one can notice that $i$ can only
  occur in the memory of one of the threads participating to the
  action (see~\cite[Table 1]{CristescuKV13}), which are the same in
  transitions in the same event. The thesis follows.
\end{proof}
\begin{lemma}
  PCI holds in the LTSI for R$\pi$.
\end{lemma}
\begin{proof}
  Similar to the one above.
\end{proof}
Using PCI and CIRE, we get the CS$\ci$ and CL$\ci$ safety and liveness properties. 
Applying mapping $g$ from Section~\ref{sec:coinitial}, we get a general 
pre-reversible LTSI satisfying IRE and IEC, so that CS$\indt$ and CL$\indt$ are satisfied.
Notice that the notion of independence is not influenced by the abstraction on labels; hence the results can be reflected on the original LTSI of R$\pi$.

% \begin{description}
% \item[SP:]
% \item[BTI:] 
% \item[PI:] 
% \item[RPI:] 
% \item[WF:] trivial, since each backward step consumes at least a memory;
% \item[PCI:] 
% \item[CIRE:] 
% \item[IRE:] 
% \end{description}


\subsection{Reversible internal $\pi$-calculus with extrusion histories}
The reversible internal $\pi$-calculus $\pi$IH~\cite{GPY21} is based on the work of Hildebrandt \emph{et al.}~\cite{HJN19},
which uses extrusion histories and locations to define a stable non-interleaving early operational semantics for the $\pi$-calculus.
Locations and extrusion histories are used to define independence of actions.
This notion of independence differs from the ones considered in the other case studies in that it allows actions with conflicting causes to be independent.
Despite this major difference, it is shown in~\cite{GPY21} that nearly all our (non-structural) axioms are satisfied
(SP, BTI, WF, PCI, IRE); the only exception is that IEC fails,
because a process can have independent transitions with conflicting causes without having a single state where equivalent transitions can both be performed.
We use IEC to show RPI (Proposition~\ref{prop:RPI}).
However RPI is shown in~\cite{GPY21}  for $\pi$IH without the need for IEC,
using the fact that independence is defined on transition labels.
In fact, LG holds for $\pi$IH, from which we can deduce PCI, IRE and RPI by Proposition~\ref{prop:LG}.
It follows that all the properties listed in Table~\ref{t:list} hold for $\pi$IH, with the exception of IEC, IC and CLG.


\subsection{Reversible Erlang}\label{sec:erlang}
We consider the uncontrolled reversible (reduction) semantics for Erlang
in~\cite{LaneseNPV18}. We restrict our attention to reachable
processes. 
%The semantics is a reduction semantics; hence reductions have no labels.
%(or, equivalently, all labels are the same).
In order to
have more informative labels we can consider the annotations defined
in~\cite[Section 4.1]{LaneseNPV18}. We can then define coinitial transitions to be independent
iff they are concurrent~\cite[Definition 12]{LaneseNPV18}.  %We remark that the
%The Loop Lemma holds~\cite[Lemma 11]{LaneseNPV18}.

We next discuss the validity of our axioms in reversible Erlang.
SP is proved in~\cite[Lemma 13]{LaneseNPV18} and BTI is trivial from the definition 
of concurrency~\cite[Definition 12]{LaneseNPV18}.  WF holds since the pair of non-negative integers 
(total number of elements in history, total number of messages queued) ordered under
lexicographic order decreases at each backward
step. Intuitively, each step but the ones derived using the rule for reverse sched 
(see~\cite[Figure~11]{LaneseNPV18}) consumes an item of memory, and each step derived using 
rule reverse sched removes a message from a process queue. Finally, PCI and CIRE hold since CLG holds for the LTSI with annotated labels,
%the notion of concurrency is defined on the annotated labels,
and by Proposition~\ref{prop:CLG}.
%
\Comment{
\begin{description}
\item[SP:] proved in~\cite[Lemma 13]{LaneseNPV18};
\item[BTI:] trivial from the definition of concurrency~\cite[Definition 12]{LaneseNPV18};
% \item[PI:] does not apply since the notion of concurrency only considers coinitial transitions;
% \item[RPI:] does not apply since the notion of concurrency only considers coinitial transitions;
\item[WF:] trivial, since the pairs of integers (total number of
  elements in memories, total number of messages queued) ordered under
  lexicographic order are always positive and decrease at each backward
  step. Intuitively, each step but the ones derived using the rule for reverse sched (see~\cite[Fig.~11]{LaneseNPV18}) consumes an item of memory, and each step derived using rule reverse sched removes a message from a process queue;
%\item[UT:] holds trivially under the reduction semantics, holds also
%  under the annotated semantics. Intuitively, the label can be deduced
%  from the history item, but for Sched, where it can be deduced from
%  the state of the queue;
\item[PCI, CIRE:] hold, since the notion of concurrency is defined on the
  annotated labels, and by Proposition~\ref{prop:underlying}.
%\item[CIRE:] holds, since the notion of concurrency is defined on the
%  annotated labels.
%\item[IRE:] does not apply since the notion of concurrency only
% considers coinitial transitions.
\end{description}
%SP is already shown~\cite[Lemma~13]{LaneseNPV18}. The remaining axioms BTI, WF, PCI and CIRE are 
%straightforward to show, noting that concurrency is defined on transition labels
%and using Proposition~\ref{prop:underlying}. 
%
Since SP, BTI and WF hold, we obtain a very much simplified proof of CC.
Moreover, using PCI and CIRE, we get the CS$_<$ and CL$_<$ safety and liveness properties and, applying mapping $g$ from Section~\ref{sec:coinitial}, we get a general 
pre-reversible LTSI satisfying IRE and IEC.  This in turn will satisfy \il{CS$\indt$ and CL$\indt$.}
%
%plus we show CS and CL in strengthened \todo{can we say that?} versions compared to~\cite[Corollary~22]{LaneseNPV18}).
%We can use the method of Remark~\ref{rem:coinit to gen} to obtain a general
%LTSI satisfying SP, BTI, WF, PCI, IRE and IEC.
}

Since this setting is very similar to the one of HO$\pi$
(both calculi have a reduction semantics and a coinitial notion of independence defined on enriched labels),
we get the same results as for
HO$\pi$ (described in Section~\ref{sec:hopi}), including CC, and causal safety and liveness.
%CS$\indt$ and CL$\indt$.



\subsection{Reversible occurrence nets}
We consider occurrence nets which are the result of unfolding Place/Transition nets, and their reversible versions~\cite{MMU19,MMU20,MelgrattiMPPU2020}. 
Reversible occurrence nets are occurrence nets 
(1-safe and with no backward conflicts)
extended with a backward (reverse in the terminology of~\cite{MMU20}) transition name $\overleftarrow{{\sf t}}$ for each forward transition name ${\sf t}$. We write $t, u$ (note the $italic$ font) for forward or backward transition names, and $\overleftarrow{t}, \overleftarrow{u}$ for their backward or forward duals. We use ``transition name'' to mean forward or backward transition name.
%
They give rise to an LTS where states 
are pairs $(N,m)$ with $N$ a net and $m$ a marking. A computation that represents firing a (forward or backward)
transition name $t$ in $(N,m)$ and resulting in $(N,m')$ is given by a firing relation $(N,m)\tran{t} (N,m')$~\footnote{
We use ``transition names'' in this subsection to name the members of the set of transitions which, together with the set of places,
are part of the definition of Place/Transition nets or occurrence nets. This distinguishes them from our transitions, which are called firings in Place/Transition nets and occurrence nets.}. 
%
Independence is the concurrency relation $\co$ which is defined between arbitrary firings as follows:
two firings are concurrent if their transition names are concurrent, that is when they are not in conflict 
and do not cause each other~\cite[Section 3]{MMU19,MMU20}. The last two notions are defined in terms 
of conditions on pre- and postset relations on transition names.
Hence, we get an LTSI with general independence. Note that transition names are unique. 
%However, we can equally well obtain an LTSI with coinitial independence
%using properties of coinitial concurrent transition names~\cite[Lemma~3.3]{MMU20}.

Properties SP and PL are shown as~\cite[Lemma~4.3]{MMU20} and \cite[Lemma~4.4]{MMU20}, respectively. 
Then CC is proved (over several pages) as~\cite[Theorem~4.6]{MMU20} using SP and PL.
%These proofs are over four pages long~\cite{MMU20}. 
The causal safety and causal liveness properties are not considered in \cite{MMU19,MMU20}.
However, a form of such properties is discussed in~\cite{MelgrattiMPPU2020} in the setting of reversible prime event structures; we discuss this point in Section~\ref{sec:related}.

We can obtain %\il{CS$\indt$ and CL$\indt$,} 
causal safety and causal liveness properties, as well as PL and CC, for reversible occurrence nets using our axiomatic approach.
The following lemma will be helpful.
\begin{lemma}\label{lem:nocausation}
Let $t$ and $u$ be enabled and coinitial (forward or backward) transition names. Then $t$ does not cause $u$. If additionally  $t$ and $u$ are backward, 
then they are not in conflict.
\end{lemma}
\begin{proof}
 Assume for contradiction that $t$ causes $u$. 
So there is a place, say $a$, in the preset of  $u$ such that $t$ causes $a$. Since $u$ is enabled there is a token in $a$. 
Also, since $t$ is enabled, after it fires a second token will arrive in $a$, thus
contradicting the 1-safe property of occurrence nets.

Let $t$ and $u$  be $\overleftarrow{{\sf t}}$ and 
  $\overleftarrow{{\sf u}}$ respectively.  Assume for contradiction that  they are in conflict. This means that they share a place, say $a$, in their presets. Hence, ${\sf t}$ and ${\sf u}$ share $a$ in their postsets, which contradicts the no backwards conflict property of occurrence nets. 
\end{proof}

%\iu{
We can now combine Lemma~\ref{lem:nocausation} with the conditions in~\cite[Lemma~3.3]{MMU20} of when enabled and coinitial $t$ and $u$ are concurrent.

\begin{lemma}\label{lem:on-concurrency}
Let $t$ and $u$ be enabled and coinitial (forward or backward) transition names. Then
$t\co u$ iff $t$ and $u$ are backward or they are not in an immediate conflict.
\end{lemma}
%\todo{Previous version:$t\co u$
%  $t\co u$ iff $t$ and $u$ are not in an immediate conflict if at least one of them is a forward transition name.}\\
As a consequence, BTI holds.


% With SP holding, BTI follows from~\cite[Lemma 3.3]{MMU20}. 
\begin{lemma}
  BTI holds in the LTSI for reversible occurrence nets.
\end{lemma}
%\begin{proof}
%  Assume enabled coinitial firings with reverse transition names $\overleftarrow{{\sf t}}$ and $\overleftarrow{{\sf u}}$. The firings are concurrent iff $\overleftarrow{{\sf t}}$ and $\overleftarrow{{\sf u}}$ are concurrent, which is when they are not in conflict and they do not cause each other: this is given by Lemma~\ref{lem:nocausation}. 
%\end{proof}

WF holds because there are no forward cycles of firings in occurrence nets, hence 
no infinite reverse paths. This gives us PL and CC.
Next, we prove PCI. 
\begin{lemma}
PCI holds in the LTSI for reversible occurrence nets.
\end{lemma}
\begin{proof}
Consider enabled coinitial  firings $\phi_1, \phi_2$ with transition names $t, u$ respectively, and assume $\phi_1\co \phi_2$. Hence $t\co u$. We get a commuting diamond by SP, where the opposite sides have the same transition names. Since $t\co u$, we have  $\overleftarrow{t}\co u$ by~\cite[Lemma 3.4]{MMU20}, so PCI holds. 
\end{proof}
This gives us a pre-reversible LTSI, and thus CS$\ci$ and CS$_<$ hold.

Given a pair of enabled coinitial concurrent  transition names we get a commuting diamond by SP, and the pairs of  coinitial transition names in all corners of the diamond are concurrent. Events can then be defined on firings in such diamonds as in Definition~\ref{def:sqeqt}, and we can show IRE.
\begin{lemma}\label{lem:IRE-occnets}
  IRE holds in the LTSI for reversible occurrence nets.
\end{lemma}
\begin{proof}
Let $\phi_1, \phi_2$ be firings with $t, u$ respectively, and let $\phi_1\co \phi_2$. 
This means that $t\co u$. Since any $\phi_1'$ equivalent to $\phi$ has the same transition name $t$, $t\co u$ gives us
$\phi_1'\co \phi_2$.
\end{proof}

Since IRE implies CIRE we obtain CL$\ci$ (or CL$_<$). We also have CS$\indt$ and CL$\indt$ as IRE holds.

An alternative proof strategy would be to show CLG first, but we believe this approach leads to more complex technicalities, and we would still need to prove IRE, hence we have preferred the approach above.
%}
%% Alternatively, we can express independence between coinitial firings with transition names $t$ and $u$ purely in terms of a relation on their underlying versions. If $t$ and $u$ are forward ${\sf t}$ and {\sf u}, then we let $I({\sf t},{\sf u})$ iff ${\sf t} \co {\sf u}$. This is equivalent to ${\sf t}$ and {\sf u} not being in an immediate conflict~\cite[Lemma~3.3]{MMU20}. 
%% If $t$ and $u$ are $\overleftarrow{{\sf t}}$ and ${\sf u}$ respectively, then 
%% $\overleftarrow{{\sf t}} \co {\sf u}$ iff ${\sf t}$ and ${\sf u}$ by Lemma~3.4 in \cite{MMU20}, where {\sf t} is the underlying version
%% of $\overleftarrow{{\sf t}}$.  We can then let $I({\sf t},{\sf u})$ iff ${\sf t} \co {\sf u}$. In terms of causal dependence and conflict relations, since they do not cause each other by Lemma~\ref{lem:nocausation},   $\overleftarrow{{\sf t}} \co {\sf u}$ is equivalent to requiring that {\sf t} and {\sf u} are not in an immediate conflict~\cite[Lemma~3.3]{MMU20}. Lastly, when $t$ are $u$ are 
%% $\overleftarrow{{\sf t}}$ and $\overleftarrow{{\sf u}}$, respectively, then $t \co u$ by Lemma~\ref{lem:nocausation}. We then get
%% $\overleftarrow{ t} \co \overleftarrow{u}$, which is ${\sf t} \co {\sf u}$, by SP and PCI. So, we let $I({\sf t},{\sf u})$ iff
%% ${\sf t} \co {\sf u}$. Hence, CLG holds and we obtain PCI, CIRE and IC by Proposition~\ref{prop:CLG}. We can then have Lemma~\ref{lem:IRE-occnets}.
%%   }
%. \todo{IVAN: Future work?}

\Comment{
%
% Some questions and material commented out following a meeting on 6 June
%

%
\todo{It is clear that applying the map $c$ to general 
independence $co$ gives the coinitial independence portion of $co$. Since there is no definition of events, thus no results on events in~\cite{MMU20}, some work is needed  to show that applying $g$ to the coinitial portion of $co$ (obtained by applying $c$) gives back the full global independence $co$.
Iain: I think this might be false.  Consider net $O_1$ in Figure 2 of~\cite{MMU20}.
With both initial places marked as shown, the two firings are coinitial and concurrent.  But if we consider the two firings got with only one initial place marked these are concurrent (I think).
But they cannot be got from the coinitial firings by a ladder (using our definition of event). }

  
%\Comment{ 
We can obtain %\il{CS$\indt$ and CL$\indt$,} 
causal safety and causal liveness properties, as well as PL and CC, for reversible occurrence nets using our axiomatic 
approach. With SP holding,  
%SP is proved as~\cite[Lemma 4.3]{MMU20}. 
BTI follows from~\cite[Lemma 3.3]{MMU20}. 
\todo{Iain: Lemma 3.3 says that reverse coinitial transitions are concurrent iff they do not cause each other,
while BTI says they are always concurrent.
So this is a strengthening of Lemma 3.3.
Consider transitions $t_1 < t_2 < t_3$.
With tokens immediately after $t_1$ and $t_3$ both of these can fire in reverse.
However they are not concurrent.
But this is not an occurrence net?}
WF holds because there are no forward cycles of firings in occurrence nets, hence 
no infinite reverse paths. This gives us PL and CC.
In order to have causal safety and causal liveness properties we first need to prove PCI. 
\todo{Iain: Why doesn't LG hold?  It seems that concurrency is defined
on the transitions rather than the firings. Perhaps the reason is that reverse
transitions are handled differently from forward transitions.
But it would still be the case that IRE holds?}
%
Consider two coinitial concurrent firings 
in a commuting diagram \todo{diamond?}. We can show that the firings on the consecutive sides of this diamond are 
also concurrent \todo{is this clear?}. Then we obtain PCI by~\cite[Lemma 3.4]{MMU20}. This gives us a pre-reversible LTSI and CS$\ci$ and CS$_<$ hold.  In order to get CL$\ci$ (or CL$_<$) we need to show CIRE, and IRE is required for CS$\indt$ and CL$\indt$.

%$\preS{t}$ and $\postS{u}$

\Comment{\todo{why bring in definitions of event?}
The equivalence relation $\sim$ on firings in commuting diamonds of firings can be defined as in Definition~\ref{def:sqeqt simp}, thus giving the notion of events as firings in the same equivalence class (Definitions~\ref{def:sqeqt}). We can then have $\coind$ as in Definition~\ref{def:coind events}, and we can show CIRE using SP, PCI and the definition when firings are concurrent.
}
%
%
%
%\Comment{
Since there is no backwards conflict in occurrence nets transition names have unique causal histories.
Firings for transition name $t$ (and its causal history) 
are equivalent if there is a ladder of commuting diamonds connecting them.  This gives the notion 
of event as the firings in the same equivalence class (for a transition name and its causal history). 
Consider %forward 
coinitial  concurrent firings, with presets of $t$ and $u$ contained in $m_1$ and $m_2$ respectively:
\begin{equation}
(N,m_1\oplus m_2 \oplus m_3)\tran{t}(N,m_1'\oplus m_2 \oplus m_3)
\quad 
(N,m_1\oplus m_2 \oplus m_3)\tran{u}(N,m_1\oplus m_2' \oplus m_3).
\end{equation}
They give rise by SP to these cofinal firings which make up a commuting diamond:
\begin{equation}
(N,m_1\oplus m_2' \oplus m_3)\tran{t}(N,m_1' \oplus m_2' \oplus m_3)
\quad 
(N,m_1' \oplus m_2 \oplus m_3)\tran{u}(N,m_1' \oplus m_2' \oplus m_3).
\end{equation}
Generally, firings that are equivalent to those in (1), namely those that can be connected by a ladder of commuting diamonds, have the following form, for some markings $m_1^\dagger, m_2^\dagger,  m_3^\dagger$ and $m_3^{\dagger\dagger}$: 
\begin{equation}
(N,m_1\oplus m_2^\dagger \oplus m_3^\dagger) \tran{t}(N,m_1'\oplus m_2^\dagger \oplus m_3^\dagger)
\quad 
(N,m_1^\dagger \oplus m_2 \oplus m_3^{\dagger\dagger})\tran{u}(N,m_1^\dagger\oplus m_2' \oplus m_3^{\dagger\dagger})
\end{equation}

To show CIRE we assume that the events of the firings in (3) are coinitially concurrent and that the firings are coinitial. Then we show that the firings are concurrent.

 Since the firings are coinitial  we  deduce that $m_1^\dagger = m_1$, $m_2^\dagger = m_2$, and $m_3^\dagger = 
m_3^{\dagger\dagger}$. So we have
\begin{equation}
(N,m_1\oplus m_2 \oplus m_3^\dagger) \tran{t}(N,m_1'\oplus m_2 \oplus m_3^\dagger) \quad 
(N,m_1 \oplus m_2 \oplus m_3^{\dagger})\tran{u}(N,m_1\oplus m_2' \oplus m_3^{\dagger}).
\end{equation}
%Note that various markings may overlap, for example $m_1$ and $m_2$. However, 
Since  the events of the firings are coinitially concurrent we deduce that there are equivalent coinitial concurrent firings for $t$ and $u$. Assume wlog that they are the firings in (1). In order to show that the firings in (4) are concurrent, we consider three cases: $t$ and $u$ are forward, $t$ is forward and $u$ is backward, and  $t$ and $u$ are backward.

In the first case,  $t$ and $u$ in (1) being concurrent means  that they are not in an immediate conflict, so their presets do not overlap: $ m_1 \cap m_2= \emptyset$. Hence, by~\cite[Lemma 3.3]{MMU20}
the firings in (4) are also concurrent.

The second case is $t$ is forward and $u$ is backward in (1). By PCI  applied to the commuting diamond with the firings (1) we obtain that
\begin{equation}
(N,m_1\oplus m_2' \oplus m_3)\tran{t}(N,m_1' \oplus m_2' \oplus m_3)
\quad 
(N,m_1 \oplus m_2' \oplus m_3)\tran{\rev u}(N,m_1 \oplus m_2 \oplus m_3)
\end{equation}
are forward and concurrent, hence $ m_1 \cap m_2'= \emptyset$. So we can deduce that 
\begin{equation}
(N,m_1\oplus m_2' \oplus m_3^\dagger) \tran{t}(N,m_1'\oplus m_2' \oplus m_3^\dagger) \quad 
(N,m_1 \oplus m_2' \oplus m_3^{\dagger})\tran{\rev u}(N,m_1\oplus m_2 \oplus m_3^{\dagger})
\end{equation}
are concurrent,  and that there is a commuting diamond with the two firings in (4),
$ (N,m_1\oplus m_2' \oplus m_3^\dagger) \tran{t}(N,m_1'\oplus m_2' \oplus m_3^\dagger)$ and
$(N,m_1'\oplus m_2 \oplus m_3^{\dagger})\tran{u}(N,m_1'\oplus m_2' \oplus m_3^{\dagger})$.
Since firings in (6) are concurrent the firings in (4) are also concurrent by PCI. 

In the last case the firings in (1) are backwards and since they are concurrent they do not cause each 
other~(\cite[Lemma 3.3]{MMU20}). This means the postset of $t$ does not overlap with the preset of $u$, namely
$m_1'\cap m_2=\emptyset$, and the other way round: $m_2'\cap m_1=\emptyset$. Hence, the firings in (4) are also concurrent.    


%}
%
%
%
\todo{We could try to show IRE but it would require a lot of work.  Or we could use the mapping $g$ from Section~\ref{sec:coinitial}.
Since CIRE holds we get IRE and IEC by Proposition~\ref{prop:gen coinit}.
However, it remains to be checked if $co$ coincides with $g$ applied to the coinitial portion of $co$. Showing this amounts to proving IRE.

Another way to obtain IRE is to enrich labels of firings. Since there is no backward conflict transition names have unique causal histories. Let $H(t)$ stand for (an encoding of) the causal history of $t$. We can have several identical transition names in an occurrence net resulting from unfolding of a Place/Transition net but they have their different causal histories. If we use firings with labels that combine a transition name with its causal history, then we can obtain LG: firings with labels $t, H(t)$ and $u, H(u)$ are defined to be independent iff $t$ and $u$ (with histories $H(t)$ and $H(u)$ respectively) are concurrent.  }

\Comment{
%
% Commented in May
%
The equivalence relation $\sim$ on firings in commuting diamonds of firings can be defined as in Definition~\ref{def:sqeqt simp}, thus giving the notion of events as firings in the same equivalence class (Definitions~\ref{def:sqeqt}). We can then have $\coind$ as in Definition~\ref{def:CIRE}.
Firings with transition $t$ are equivalent if there is a ladder of commuting diamonds connecting them. Consider forward coinitial and concurrent firings 
\begin{equation}
(N,m_1\oplus m_2 \oplus m_3)\tran{t}(N,m_1'\oplus m_2 \oplus m_3)
\quad 
(N,m_1\oplus m_2 \oplus m_3)\tran{u}(N,m_1\oplus m_2' \oplus m_3).
\end{equation}
By SP, we obtain these cofinal firings which make up a commuting diamond:
\begin{equation}
(N,m_1\oplus m_2' \oplus m_3)\tran{t}(N,m_1' \oplus m_2' \oplus m_3)
\quad 
(N,m_1' \oplus m_2 \oplus m_3)\tran{u}(N,m_1' \oplus m_2' \oplus m_3).
\end{equation}
For presets of $t$ and $u$, namely
$\preS{t}$ and $\preS{u}$,
 we have $\preS{t}\subseteq m_1$ and $\preS{u}\subseteq m_2$. Moreover, $ m_1 \cap m_2= \emptyset$ since the firings are concurrent.
Generally, firings that are equivalent to those in (1), namely those that can be connected by a ladder of commuting diamonds, have the following form, for some $m_1^\dagger, m_2^\dagger,  m_3^\dagger$ and $m_3^{\dagger\dagger}$: 
\begin{equation}
(N,m_1\oplus m_2^\dagger \oplus m_3^\dagger) \tran{t}(N,m_1'\oplus m_2^\dagger \oplus m_3^\dagger)
\quad 
(N,m_1^\dagger \oplus m_2 \oplus m_3^{\dagger\dagger})\tran{u}(N,m_1^\dagger\oplus m_2' \oplus m_3^{\dagger\dagger})
\end{equation}


%where none of $(m_1, m_2^\dagger)$ and $(m_1^\dagger, m_2)$ overlap. 
%$\coind$
To show CIRE assume that the events of the firings in (3) are related by $\coind$, written as $[t] \coind [u]$, and that the firings are coinitial.

 By SP, we obtain these cofinal firings which make up a commuting diamond:
\begin{equation}
(N,m_1\oplus m_2^\dagger \oplus m_3^\dagger) \tran{t}(N,m_1'\oplus m_2^\dagger \oplus m_3^\dagger)
\quad 
(N,m_1^\dagger \oplus m_2 \oplus m_3^{\dagger\dagger})\tran{u}(N,m_1^\dagger\oplus m_2' \oplus m_3^{\dagger\dagger})
\end{equation}


 and both are forward. The last implies that $m_1^\dagger = m_1$, $m_2^\dagger = m_2$, and $m_3^\dagger = 
m_3^{\dagger\dagger}$. So we have
$(N,m_1\oplus m_2 \oplus m_3^\dagger) \tran{t}(N,m_1'\oplus m_2 \oplus m_3^\dagger)$ and 
$(N,m_1 \oplus m_2 \oplus m_3^{\dagger})\tran{u}(N,m_1\oplus m_2' \oplus m_3^{\dagger})$.
%Note that various markings may overlap, for example $m_1$ and $m_2$. However, 
Since  $[t] \coind [u]$ we deduce that there are equivalent coinitial firings for $t$ and $u$, for some $m_3'$, 
$$(N,m_1\oplus m_2 \oplus m_3')\tran{t}(N,m_1'\oplus m_2 \oplus m_3')
\quad 
(N,m_1\oplus m_2 \oplus m_3')\tran{u}(N,m_1\oplus m_2' \oplus m_3'),
$$
which are concurrent, implying  $ m_1 \cap m_2= \emptyset$. Hence, the firings in (3) are concurrent.

The second case is $t$ is forward and $u$ is backward (and coinitial). By PCI  we have that
\begin{equation}
(N,m_1\oplus m_2' \oplus m_3)\tran{t}(N,m_1' \oplus m_2' \oplus m_3)
\quad 
(N,m_1 \oplus m_2' \oplus m_3)\tran{\rev u}(N,m_1 \oplus m_2 \oplus m_3)
\end{equation}
are concurrent. In a commuting diamond for the firings in (3), and have that the firings corresponding to those in (4) are concurrent. ince they are forward we can prove that  
}
%}
\Comment{Events can be defined on firings in such commuting diagrams as in our Definitions~\ref{def:sqeqt} and~\ref{def:sqeqt simp}, and then IRE holds as 
the concurrency relation preserves such events.\il{*** maybe rewrite the previous sentence?***}
}
}
\subsection{Reversible sequential systems}
%\il{
In \emph{sequential systems} there is no concurrency.
Hence, in this section, we represent them as LTSIs where the independence
relation, modelling concurrency, is empty.
This is for instance the case for Janus programs~\cite{YokoyamaG07} or CCSK processes without parallel composition.
%}
%% In this section a \emph{sequential system} is defined to be an LTSI where the independence
%% relation is the empty relation, which
%% is for instance the case for Janus programs~\cite{YokoyamaG07} or CCSK processes without parallel composition.
%% \todo{Omit?:This holds in particular when
%% selecting the concurrency relation as independence relation.}
In this setting,
% because the independence relation is empty,
SP, PCI, IRE and IEC hold trivially. 
Moreover, BTI is equivalent to backward determinism, which is
the main condition required for reversibility in a sequential setting (see, e.g., Janus~\cite{YokoyamaG07}).

\begin{definition}[Backward determinism]
An LTSI is backward deterministic iff $P \ftran{a} Q$ and $P'
\ftran{a'} Q$ imply $P = P'$ and $a=a'$.
\end{definition}

\begin{proposition}\label{prop:sequential}
A sequential system satisfies BTI iff it is backward deterministic.
\end{proposition}
\begin{proof}
For the left to right implication, assume towards a contradiction that
the system satisfies BTI but it is not backward deterministic. Then
there are $P \ftran{a} Q$ and $P' \ftran{a'} Q$ with $P \neq P'$ or $a
\neq a'$. By the Loop Lemma we have the reverse transitions, which are
coinitial and backwards, hence by BTI they need to be independent,
what is a contradiction since the independence relation is empty.

For the right to left implication, take two backward coinitial
transitions $t,t'$. By applying the Loop Lemma there exist $\rev t,
\rev t'$. One can notice that $\rev t,\rev t'$ satisfy the hypothesis
of backward determinism. Hence, $\rev t=\rev t'$ and $t=t'$. Hence BTI
trivially holds.
\end{proof}

WF does not hold in general and needs to be assumed.

If we assume WF then all our results hold, but they all become trivial
or almost trivial.  E.g., all events are singletons. Also, all the notions of causal liveness coincide, and they state that the last
transition can always be undone, but this is just one direction of the
Loop Lemma. Similarly, all the notions of causal safety do coincide, and they require that only the last transition can
be undone.
