% \documentclass[acmsmall,screen,review=false]{acmart}
\documentclass[11pt,a4paper]{amsart}
% These settings adjust the size of the page 
\setlength{\textwidth}{6in}
% \setlength{\textheight}{8.5in}
\setlength{\textheight}{8.8in}
\setlength{\oddsidemargin}{0in}
\setlength{\evensidemargin}{0in}
\setlength{\topmargin}{0.25in}
\usepackage{amsaddr}


% \documentclass[
	% final, %
	% british, %
	% orivec,envcountsect,envcountsame,runningheads, %
	% a4paper %
% ]{llncs}
% \acmJournal{TOCL}

\usepackage{centernot}
\usepackage{epsfig}
\usepackage{etoolbox} % for iftoggle
\usepackage{thmtools}
\usepackage{thm-restate}
%\usepackage{amssymb}
\usepackage{url}
\usepackage{arydshln}
\usepackage{psfrag} % does not work with pdflatex
\usepackage{marvosym}
\usepackage{amsmath} % for xrightarrow
\usepackage{amsthm}
\usepackage{hyperref}
%\usepackage{pstool}
\usepackage{orcidlink}


%

\newcommand\calF{\mathcal{F}}
\newcommand\calG{\mathcal{G}}
\newcommand\calM{\mathcal{M}}
\newcommand\calV{\mathcal{V}}
\newcommand\calU{\mathcal{U}}
\newcommand\calW{\mathcal{W}}
\newcommand\calP{\mathcal{P}}
\newcommand\calD{\mathbb{D}}
%%%%%%%%%%%%%%%%%
%% macros introduced by Luke 
\newcommand\mydef[1]{{\bf\em #1}}
%%%%%%%%%%%%%%%%%

\newcommand{\numviparams}{{| \lambda |}}
\newcommand{\scoreaccvars}[1]{s_1^{#1}, \ldots, s_{\numviparams}^{#1}}
\newcommand{\scoreaccvar}[2]{s_{#1}^{#2}}
\newcommand{\isdeterm}[1]{\text{Deterministic}({#1})}


\newcommand{\expect}[1]{\mathbb{E}\left[{#1}\right]}
\newcommand{\var}[1]{\mathbb{V}\left[ {#1} \right]}
\newcommand{\expectdist}[2]{\mathbb{E}_{#1}\left[ {#2} \right]}
\newcommand{\vardist}[2]{\mathbb{V}_{#1}\left[ {#2} \right]}
\newcommand{\cov}[2]{\mathbb{C}\text{ov}[{#1}][{#2}]}
\newcommand{\covv}[1]{\mathbb{C}\text{ov}[{#1}]}
\newcommand{\corr}[1]{\mathbb{C}\text{orr}[{#1}]}

\newcommand{\fix}[1]{\mathit{fix}\left({#1}\right)}
\newcommand{\sbr}[1]{\left\llbracket {#1} \right\rrbracket}
\newcommand{\ctxtype}[3]{{#1} \cong_\text{ctx} {#2} : {#3}}
\newcommand{\bigstep}[3]{{#1} \Downarrow_{#2} {#3}}


% PCF types
\newcommand{\bool}{\mathit{bool}}
\newcommand{\nat}{\mathit{nat}}

\newcommand{\ctx}[1]{\mathcal{C}\left[ {#1}\right] }
\newcommand{\pcft}[1]{\text{PCF}_{#1}}

\newcommand{\nfl}{\mathbb{N}_\bot}
\newcommand{\bfl}{\mathbb{B}_\bot}

% PCF constructs
\newcommand{\succc}[1]{\mathbf{succ}({#1})}
\newcommand{\succcn}[2]{\mathbf{succ}^{#1}({#2})}
\newcommand{\zero}{\mathbf{0}}
\newcommand{\zerotest}[1]{\mathbf{zero}\left({#1}\right)}
\newcommand{\pred}[1]{\mathbf{pred}\left( {#1} \right)}
\newcommand{\predn}[2]{\mathbf{pred}^{#1}\left( {#2} \right)}
\def\solvable{\#}

\newcommand{\true}{\mathbf{true}}
\newcommand{\false}{\mathbf{false}}
\newcommand{\pcffix}[1]{\mathbf{fix}\left({#1}\right)}
\newcommand{\pcffn}[3]{\mathbf{fn}~{#1}:{#2}\mathpunct{.}{#3}}
\newcommand{\pairtype}[2]{{#1} * {#2}}
\newcommand{\pairexp}[2]{\mathbf{pair}({#1}, {#2})}
\newcommand{\leftexp}[1]{\mathbf{left}({#1})}
\newcommand{\rightexp}[1]{\mathbf{right}({#1})}

\newcommand{\RationalPos}{\mathbb{Q}^{+}}

\newcommand{\meas}[1]{\mathbb{M}\left( {#1} \right) }
\newcommand{\integ}[1]{\sbr{#1}_I}

\newcommand{\notbigstep}[2]{{#1}~\cancel{\Downarrow}_{#2}}
\newcommand{\subtrace}[3]{{#1}^{{#2} \ldots {#3}}}
\newcommand{\supp}[1]{\textsf{supp}\left({#1}\right)}
\newcommand{\dom}[1]{\textsf{Dom}\left({#1}\right)}
\newcommand{\suppk}[2]{\textsf{Supp}^{#1}\left({#2}\right)}
\newcommand{\tracespace}{\bigcup_{n \in \mathbb{N}}[0, 1]^n}
\newcommand{\generictracespace}{\mathbb{T}}
\newcommand{\nnreals}{\mathbb{R}_{\geq 0}}
\newcommand{\posreals}{\mathbb{R}_{> 0}}
\newcommand{\reals}{\mathbb{R}}

\newcommand{\unrollkM}[2]{\textsf{unroll}_{#1}\left({#2}\right)}
\newcommand{\nphmcint}[5]{\Psi_\textsf{NP}\left({#1}, {#2}, {#3}, {#4}, {#5}\right)}

%SPCF constructs
\newcommand{\spcfvalues}{\Lambda^0_v}

\newcommand{\prevalueM}[1]{\textsf{value}^{-1}_{#1}(\spcfvalues{})}
\newcommand{\num}[1]{\underline{#1}}

% \theoremstyle{definition}
% \newtheorem{thm}{Theorem}
% \newtheorem{lem}{Lemma}
% \newtheorem{defn}{Definition}
% \newtheorem{conj}{Conjecture}
% \newtheorem{prop}{Proposition}

%\theoremstyle{definition}
%\newtheorem{defn}{Definition}[section]
%\newtheorem{example}[defn]{Example}
%
%
%\theoremstyle{plain}
%\newtheorem{thm}{Theorem}[section]
%\newtheorem{lem}[thm]{Lemma}
%\newtheorem{cor}[thm]{Corollary}
%\newtheorem{conj}[thm]{Conjecture}
%\newtheorem{prop}[thm]{Proposition}
%\newtheorem{remark}[thm]{Remark}

%% Proofs
%\let\oldproof\proof
%\renewcommand{\proof}{\color{blue}\oldproof}


\definecolor{codegreen}{rgb}{0,0.6,0}
\definecolor{codegray}{rgb}{0.5,0.5,0.5}
\definecolor{codepurple}{rgb}{0.58,0,0.82}
\definecolor{backcolour}{rgb}{0.95,0.95,0.92}

\lstdefinestyle{myStyle}{
    belowcaptionskip=1\baselineskip,
    breaklines=true,
    frame=none,
    basicstyle=\footnotesize\ttfamily,
    keywordstyle=\bfseries\color{green!40!black},
    commentstyle=\itshape\color{purple!40!black},
    identifierstyle=\color{blue},
    backgroundcolor=\color{gray!10!white},
    %backgroundcolor=\color{backcolour}, 
    numberstyle=\tiny\color{codegray},
    stringstyle=\color{codepurple},
    breakatwhitespace=false,                          
    keepspaces=true,                 
    numbers=left,       
    numbersep=5pt,                  
    showspaces=false,                
    showstringspaces=false,
    showtabs=false,                  
    tabsize=2,
}

% argmin/argmax
\DeclareMathOperator*{\argmax}{arg\,max}
\DeclareMathOperator*{\argmin}{arg\,min}

% Concatenation of lists
\newcommand\doubleplus{+\kern-1.3ex+\kern0.8ex}

% Program configurations
\newcommand{\tuple}[1]{\ensuremath{\langle #1 \rangle}}
% Rule based definitions
\newcommand{\Rule}[4][]{\ensuremath{\inferrule*[lab={\hypertarget{#2}{(\TirName{#2})}},#1]{#3}{#4}}}

% Calligraphic symbols
\newcommand{\calI}{{\mathcal I}} 
\newcommand{\calT}{{\mathcal T}}

%  Macro for new Y operator.
\newcommand{\yBounded}[3]{\mu^{#1}_{#2}\rvert_{#3}}

%%%%%%%%%%%%%%%%%
 
%%%%%%%%%%%%%%%%%

\newcommand{\expv}{\mathbb{E}}

\newcommand{\combTr}[2]{\left[\begin{matrix}
		#1\\
		#2
	\end{matrix} \right]}

\newcommand{\exType}[2]{\left\{\begin{matrix}
		#1\\
		#2
	\end{matrix} \right\}}
\newcommand{\myint}[1]{ [#1]}
\newcommand{\Uniform}{\ensuremath{\mathrm{Uniform}}}
\newcommand{\Normal}{\ensuremath{\mathrm{normal}}}
\DeclareMathOperator{\abs}{abs}
\DeclareMathOperator{\pdf}{pdf}

\newcommand{\intConf}[1]{\lceil#1\rceil}
\newcommand{\tr}{\boldsymbol{t}}

\newcommand{\sample}{\tt{sample}}
%\newcommand{\fix}{\texttt{fix}}
%\newcommand{\num}[1]{\underline{#1}}
\newcommand{\myif}{\texttt{if}}
\newcommand{\mylet}{\texttt{let} \, }
\newcommand{\myin}{\, \texttt{in} \,}
\newcommand{\mythen}{\, \texttt{then} \,}
\newcommand{\myelse}{\, \texttt{else} \,}
\newcommand{\score}{\tt{score}}
\newcommand{\tick}{\tt{tick}}

\newcommand{\term}{\tt{term}}
\newcommand{\pv}{\mathbf{v}}
\newcommand{\rv}{\mathbf{r}}

\newcommand{\interval}{\mathfrak{I}}

\newcommand{\typeReal}{\textbf{\textsf{R}}}

\newcommand{\symbolInt}{\myint{\cdot}}

\newcommand{\LambdaInterval}{\Lambda_{\interval}}
\newcommand{\LambdaSymbolic}{\Lambda_{\text{sym}}}

\newcommand{\toIntervalTerm}[1]{#1^{2\interval}}

%Others
\newcommand{\Sset}{\mathbb{S}}
\newcommand{\Iset}{\mathbb{I}}
\newcommand{\Rset}{\mathbb{R}}
\newcommand{\Nset}{\mathbb{N}}
\newcommand{\Zset}{\mathbb{Z}}

\newcommand{\Term}{\mathbb{T}}
\newcommand{\prob}{\mathbb{P}}
\newcommand{\expt}{\mathbb{E}}


\newcommand{\Leb}{\tt{Leb}}
\newcommand{\Red}{\tt{Red}}
\newcommand{\cost}{\text{cost}}

%\newcommand{\intervalab}[2]{\underline{[#1,#2]}}
\newcommand{\intervalab}{\underline{[a,b]}}
\newcommand{\interI}{\mathcal{I}}
\newcommand{\trans}{\mathcal{T}}

\newcommand{\iv}{\mathbb{I}}

% Programming language constructs
\newcommand{\lit}[1]{\underline{#1}}
\newcommand{\letIn}[1]{\mathsf{let}\,{#1}\,\mathsf{in}\,}
\newcommand{\fixLam}[2]{\mu {#1} {#2}.}
\newcommand{\ifElse}[3]{\mathsf{if} (#1 \le \num{0}) \, {#2} \,\mathsf{else}\, {#3}}

%%Basic notions
\newcommand{\pspace}{(\Omega,\mathcal{F},\probm)}
\newcommand{\probm}{\mathbb{P}}
\newcommand{\condexpv}[2]{{\expt}{\left[{#1} \mid {#2}\right]}}

\newcommand{\stdConf}[1]{(#1)}
%\newcommand{\intConf}[1]{\lceil#1\rceil}
%\newcommand{\intConf}[1]{(#1)}
%\newcommand{\symConf}[1]{\langle\!\langle  #1 \rangle\!\rangle}
%\newcommand\symPath[1]{(#1)}
\newcommand{\symPath}[1]{\langle\!\langle  #1 \rangle\!\rangle}
\newcommand\symConf[1]{(#1)}

\newcommand{\ifSimple}[3]{\mathsf{if}(#1, #2, #3)}
%\newcommand{\ifElse}[3]{\mathsf{if} (#1 \le 0) \, \allowbreak {#2} \, \allowbreak \mathsf{else}\, {#3}}
%\newcommand{\ifElse}[3]{\ifSimple{#1}{#2}{#3}}

%\newcommand{\trace}{\mathsf{s}}
%
%\newcommand\defn[1]{{\bf \em #1}}
\newcommand{\traces}{\mathbb{T}}
%
%\newcommand{\stdConf}[1]{(#1)}
%%\newcommand{\intConf}[1]{\lceil#1\rceil}
%\newcommand{\intConf}[1]{(#1)}
%%\newcommand{\symConf}[1]{\langle\!\langle  #1 \rangle\!\rangle}
%%\newcommand\symPath[1]{(#1)}
%\newcommand{\symPath}[1]{\langle\!\langle  #1 \rangle\!\rangle}
%\newcommand\symConf[1]{(#1)}

\newcommand{\valueSem}[1]{\mathsf{val}_{#1}} % value (semantics)
\newcommand{\weightSem}[1]{\mathsf{wt}_{#1}} % weight (semantics)
\newcommand{\measureSem}[1]{\llbracket #1 \rrbracket}
\newcommand{\posterior}{\mathsf{posterior}}


%%%%%%%%%
% 
%%%%%%%%
\newcommand{\loc}{\ell}
\newcommand{\locs}{\mathit{L}}
\newcommand{\blocs}{\mathit{L}_{\mathrm{b}}}

\newcommand{\iflocs}{\mathit{L}_{\mathrm{if}}}
\newcommand{\looplocs}{\mathit{L}_{\mathrm{while}}}

\newcommand{\alocs}{\mathit{L}_{\mathrm{a}}}
\newcommand{\wlocs}{\mathit{L}_{\mathrm{w}}}
\newcommand{\rlocs}{\mathit{L}_{\mathrm{r}}}
\newcommand{\Alocs}[1]{\mathit{L}_{\mathrm{A}}^{\mathsf{#1}}}
\newcommand{\Dlocs}{\mathit{L}_{\mathrm{nd}}}
\newcommand{\transitions}{{\rightarrow}}

%%% 
\newcommand{\plocs}{\mathit{L}_{\mathrm{p}}}
\newcommand{\tlocs}{\mathit{L}_{\mathrm{t}}}

\newcommand{\lin}{\loc_\mathrm{init}}
\newcommand{\lout}{\loc_\mathrm{out}}
\newcommand{\val}[1]{\mbox{\sl Val}_{#1}}

\newcommand{\pvars}{V_\mathrm{p}}
\newcommand{\rvars}{V_{\mathrm{r}}}
\newcommand{\pre}{\mathrm{pre}}

\newcommand{\sle}{\sqsubseteq}
\newcommand{\sge}{\sqsupseteq}

\newcommand{\lfp}{\mathrm{lfp}}
\newcommand{\gfp}{\mathrm{gfp}}

\newcommand{\rdvarjdis}{\mathcal D}
\newcommand{\sampset}{\textit{supp}}

\newcommand{\upd}{\mbox{\sl upd}}
\newcommand{\wet}{\mbox{\sl wt}}
\newcommand{\transset}{\mathfrak T}
\newcommand{\valin}{\pv_{\mathrm{init}}}
\newcommand{\ret}{\mbox{\sl ret}}

\newcommand{\win}{w_{\mathrm{init}}}

\newcommand{\sampdpd}{\overline{\Upsilon}}

\newcommand{\outmap}{\text{O}}
\newcommand{\sat}[1]{\langle #1 \rangle}
\newcommand{\monoid}{\mbox{\sl Monoid}}
\newcommand{\handelmanformat}{(\dagger)}

\newcommand{\trunc}{\mathcal{B}}

\newcommand{\ewt}{\mbox{\sl ewt}}
\newcommand{\statemap}{\text{St}}

\newcommand{\valrd}{{\mathbf{r}}}
\newcommand{\frmloc}{\ell^{\mathrm{src}}}
\newcommand{\toloc}{\ell^{\mathrm{dst}}}

\newcommand{\monomials}{\mathbf{M}}
\newcommand{\preS}[1]{\ensuremath{{}^\bullet{#1}}}
\newcommand{\postS}[1]{\ensuremath{#1^\bullet}}


\newtoggle{tech_report}


\toggletrue{tech_report}
%\togglefalse{tech_report}
	
\iftoggle{tech_report}{
% tech report version ...
\newcommand{\TechReportFlag}{}
}{
%conference version ...
}

%%%%%% To display ORCID Logo with link, Please add below definition and copy the ORCID_Color.eps in the manuscript package %%%%%

\makeatletter
% \RequirePackage[bookmarks,unicode,colorlinks=true]{hyperref}%
   % \def\@citecolor{blue}%
   % \def\@urlcolor{blue}%
   % \def\@linkcolor{blue}%
% \def\UrlFont{\rmfamily}
% \def\orcidID#1{\smash{\href{http://orcid.org/#1}{\protect\raisebox{-1.25pt}{\protect% Figure removed}}}}
% \makeatother

% amsart definitions:
\theoremstyle{plain}% default
\newtheorem{proposition}[theorem]{Proposition}
\newtheorem{lemma}[theorem]{Lemma}
\newtheorem{corollary}[theorem]{Corollary}

\theoremstyle{definition}
\newtheorem{definition}[theorem]{Definition}

\theoremstyle{remark}
\newtheorem{example}[theorem]{Example}

\begin{document}

% \mainmatter

\title{An Axiomatic Theory for Reversible Computation}


%\thanks{The authors acknowledge partial support from COST Action IC1405 on Reversible Computation - Extending Horizons of Computing.}}
% \titlerunning{Dummy short title}%optional, please use if title is longer than one line
\author{Ivan Lanese\,
\orcidlink{0000-0003-2527-9995}}
  \address{Focus Team, University of Bologna/INRIA, Italy}
    \email{ivan.lanese@gmail.com}

\author{Iain Phillips\,
\orcidlink{0000-0001-5013-5876}}
  \address{Imperial College London, England}
  \email{i.phillips@imperial.ac.uk}

 \author{Irek Ulidowski\,
\orcidlink{0000-0002-3834-2036}}
  \address{University of Leicester, England}
  \address{AGH University of Science and Technology, Krak\'{o}w, Poland}
  \email{i.ulidowski@leicester.ac.uk}

% \authorrunning{I. Lanese, I.\,C.\,C. Phillips and I. Ulidowski}
\renewcommand{\shortauthors}{Lanese, Phillips and Ulidowski}

  

%\keywords{Reversible Computation, Labelled Transition System with Independence, Causal Safety, Causal Liveness}
%Concurrent systems}


\begin{abstract}
Undoing computations of a concurrent system is beneficial in many
situations, e.g., in reversible debugging of multi-threaded programs
and in recovery from errors due to optimistic execution in parallel
discrete event simulation. A number of approaches have been proposed
for how to reverse formal models of concurrent computation including
process calculi such as CCS, languages like Erlang, prime event
structures and occurrence nets.  However it has not been settled what
properties a reversible system should enjoy, nor how the various
properties that have been suggested, such as the parabolic lemma and
the causal-consistency property, are related.  We contribute to a
solution to these issues by using a generic labelled transition system
equipped with a relation capturing whether transitions are independent
to explore the implications between these properties.  In particular,
we show how they are derivable from a set of axioms.  Our intention is
that when establishing properties of some formalism it will be easier
to verify the axioms rather than proving properties such as the
parabolic lemma directly.  We also introduce two new notions related
to causal consistent reversibility, namely causal liveness and causal
safety, stating, respectively, that an action can be undone if and
only if it is independent from all the following ones.  We show that
both causal liveness and causal safety are derivable from our axioms.

% We also characterise causal-consistent reversibility by two new
% properties: causal safety and causal liveness. We investigate several
% ways to define these properties, and compare them.  We show that these
% properties are not derivable from the basic axioms above, but require
% additional axioms. As a result, they are not consequences of the
% causal-consistency property, as erroneously hinted at in the literature.

%Hence proofs of the parabolic lemma and the causal
%consistency property are not sufficient on their own, since they are
%consequences of the basic axioms.  However, if we add an additional
%axiom, we can then derive causal safety and liveness.

\end{abstract}
%\keywords{Reversible Computation \and Labelled Transition System with Independence \and Causal Safety \and Causal Liveness}
% \begin{CCSXML}
% <ccs2012>
% <concept>
% <concept_id>10003752.10003753.10003761</concept_id>
% <concept_desc>Theory of computation~Concurrency</concept_desc>
% <concept_significance>500</concept_significance>
% </concept>
% <concept>
% <concept_id>10003752.10003753.10003761.10003764</concept_id>
% <concept_desc>Theory of computation~Process calculi</concept_desc>
% <concept_significance>300</concept_significance>
% </concept>
% <concept>
% <concept_id>10010147.10011777.10011014</concept_id>
% <concept_desc>Computing methodologies~Concurrent programming languages</concept_desc>
% <concept_significance>300</concept_significance>
% </concept>
% </ccs2012>
% \end{CCSXML}
% 
% \ccsdesc[500]{Theory of computation~Concurrency}
% \ccsdesc[300]{Theory of computation~Process calculi}
% \ccsdesc[300]{Computing methodologies~Concurrent programming languages}

\keywords{Reversible Computation, Labelled Transition System with Independence, Causal Consistency, Causal Safety, Causal Liveness}
  %Concurrent systems}
\maketitle

\section{Introduction}
Current quantum hardware is unable to carry out universal quantum computations due to the buildup of errors that occur during the computation. 
The magnitude of the individual error is currently above the value that the Threshold Theorem requires in order to kick-start quantum error correction and fault-tolerant quantum computation~\cite[Section 10.6]{nielsen_chuang_2010}. 
Although the experimentally achieved fidelity rates are promising and the error bounds are inching closer to the required threshold, we will have to work for the foreseeable future with quantum hardware with errors that build-up during the computation.  This implies that we can only do a limited number of steps before the output of the computation has become completely uncorrelated with the intended one.

For fault-tolerant quantum computing, we repeat four steps: 
1) We apply a number of single and two-qubit quantum gates, in parallel whenever possible; 
2) We perform a syndrome measurement on a subset of the qubits; 
3) We perform fast classical computations to determine which errors have occurred and how to correct them; 
and, 4) We apply correction terms based on the classical computations.
We then repeat these four steps with a next sequence of gates. 
These four steps are essential to fault-tolerant quantum computing. 


The starting point of this work is to use the four steps outlined above, not to carry out error correction and fault-tolerant computation, but to enhance short, constant-depth, {\em uncorrected} quantum circuits that perform single qubit gates and {\em nearest-neighbor} two qubit gates. 
Since in the long run we will have to implement error-correction and fault-tolerant computation anyhow, and this is done by such a four-step process, why not make other use of this architecture? Moreover, on some of the quantum hardware platforms, these operations are already in place.
Embracing this idea we naturally arrive at the question: what is the computational power of \textit{low-depth} quantum-classical circuits organized as in the four steps outlined above? 
We thus investigate circuits that execute a small, ideally constant, number of stages, where at each stage we may apply, in parallel, single qubit gates and {\em nearest-neighbor} two qubit gates, followed by measurements, followed by low-depth classical computations of which the outcome can control quantum gates in later stages. 
It is not clear, at first, whether such circuits, especially with constant depth, can do anything remotely useful. 
But we will see that this is indeed the case: many quantum computations can be done by such circuits in constant depth. 
By parallelizing quantum computations in this way, we improve the overall computational capabilities of these circuits, as we do not incur errors on qubits that are idle, simply because qubits are not idle for a very long time. 
Furthermore, reducing the depth of quantum circuits, at the cost of increasing width, allows the circuit to be run faster even if errors occur.

The first usage of such a four-step layout, not to do error correction, but to perform computations, can be found in the paradigm of measurement-based quantum computing~\cite{gottesman1999demonstrating,raussendorf2001one,jozsa2006introduction,clark2007generalised}: 
A universal form of quantum computing where a quantum state is prepared and operations are performed by measuring qubits in different bases, depending on previous measurements and intermediate measurements.

\citeauthor{PhamSvore2013} were the first to formalize the four-step protocol for performing computations~\cite{PhamSvore2013}. They included specific hardware topologies by considering two-dimensional graphs for imposing constraints on qubit interactions. In their model, they develop circuits for particularly useful multi-qubit gates, including specifying costs in the width, number of qubits, depth, number of concurrent time steps, size, and total number of non-Identity operations.
As a result, they find an algorithm that factors integers in polylogarithmic depth.
\citeauthor{Browne:2011} showed that the main tool in the work by \citeauthor{PhamSvore2013}, the fan-out gate, can also be replaced by additional log-depth classical computations in the measurement-based quantum computing setting~\cite{Browne:2011}.

More recently, \citeauthor{Cirac:2021} introduced a scheme to implement unitary operations involving quantum circuits combined with Local Operations and Classical Communication ($\mathsf{LOCC}$) channels: $\mathsf{LOCC}$-assisted quantum circuits~\cite{Cirac:2021}. Similarly to the four-step scheme we just described, they allow for a short depth circuit to be run on the qubits, followed by one round of $\mathsf{LOCC}$, in which ancilla qubits are measured and local unitaries are applied based on the measurement outcomes. They show that in this model any 1D transitionally invariant matrix-product state (MPS) with fixed bond dimension is in the same phase of matter as the trivial state. Similar ideas can be found in~\cite{TVV_NonAbelianTopologicalOrder_2022, tantivasadakarn2021long}.

In this work, we introduce a new model, called \textit{Local Alternating Quantum-Classical Computations} ($\LAQCC$). In this model we alternate between running quantum circuits (constrained by locality), ending in the measurement of a subset of qubits, and fast classical computations based on the measurement results. The outcome of the classical computations are then used to control future quantum circuits. We allow for flexibility in this model, by giving different constraints to the power of both the quantum circuits and the classical circuits as well as the number of alternations between them. 
Most attention will be given to $\LAQCC$ containing quantum circuits of constant depth, classical circuits of logarithmic depth and at most a constant number of alternations between them. 
Any circuit constructed in this model is considered to be of constant depth. 
We restrict ourselves to logarithmic depth classical computations, as this is the first natural and non-trivial extension beyond constant-depth classical computations. 
Constant-depth classical computations do however also have an equivalent constant-depth quantum implementation.

The definition of $\LAQCC$ sharpens the original definition of \citeauthor{PhamSvore2013} by adding constraints to the intermediate classical computations. This allows us to bound the power of $\LAQCC$ from above. 

The main result of \citeauthor{Cirac:2021}, that 1D translational invariant MPS with fixed bond dimension can be prepared by $\mathsf{LOCC}$-assisted circuits, relies on local symmetries of the MPS. These symmetries allow them to prepare local states (on a constant number of qubits) and glue them together by doing one round of the appropriate entangling measurement and corrections, after which they run a round of local unitaries to get the desired result. This general scheme for preparing states that exhibit an MPS description with the appropriate local symmetries requires only geometrically local unitaries and one round of measurement and corrections an therefore is accessible in $\LAQCC$. Studying different local symmetries, known as Symmetry Protected Topological (SPT) phases of matter, to find measurement-based constant depth circuits for states is a broad ongoing field of research~\cite{TVV_NonAbelianTopologicalOrder_2022, tantivasadakarn2021long, smith2023deterministic}. 
All these schemes have a $\LAQCC$ implementation.

%$\LAQCC$-circuits also exist for general schemes of preparing local states, based on the local tensors, and gluing them together using one round of entangled measurement and corrections, based on the local symmetry. 
%The main result of \citeauthor{Cirac:2021}, that 1D translational invariant MPS with fixed bond dimension can be prepared by $\mathsf{LOCC}$-assisted circuits, relies heavily on local symmetries of the MPS and as a result also has an equivalent $\LAQCC$ implementation. 
%The corrections applied after the measurement round are local unitaries depending on the local symmetries of the MPS. 

 

%This general scheme of preparing local states, based on the local tensors, and gluing it together by doing one round of entangled measurement and corrections, based on the local symmetry, is accessible in $\LAQCC$.
Note however that \citeauthor{Cirac:2021} also suggest a circuit for the $W$-state.
This circuit uses sequentially and dependent measurement-based corrections of the ancilla qubits. 
These dependent measurements translate to sequential alternations between the quantum and classical circuits and therefore increase the total depth to linear depth, exceeding the constant-depth constraints imposed by $\LAQCC$-circuits. 

We study the power of the $\LAQCC$ model with respect to state preparation, showing that even with only constant quantum-depth and logarithmic classical depth it remains possible to prepare states with long-range entanglement.
Another surprising result is that it is unlikely that $\LAQCC$ circuits are classically simulatable. We show that any instantaneous quantum polynomial-time (IQP) circuit~\cite{Bremner2010,Shepherd2009} has an $\LAQCC$ implementation.
Classical simulation of IQP circuits implies the collapse of the polynomial hierarchy to the third level, which is not believed to be true~\cite{Bremner2017}. Therefore, we expect that $\LAQCC$ circuits are unlikely to be classically simulatable. We bound the power of $\LAQCC$ by showing that it is contained in $\QNC^1$, the class of polynomial-size, log-depth circuits.

Next, we also study the power that intermediate classical calculations can add to quantum computations, by considering a new model that alternates between polynomially many polynomial-depth quantum circuits and unbounded classical computations
We study this model by doing a complexity theoretical analysis, where we draw inspiration from the notions of complexity given by \citeauthor{RosenthalYuen:2022}, \citeauthor{MetgerYuen:2023}, and \citeauthor{Aaronson:2004}.
All three complexity notions are based on the notion of state preparation, instead of more traditional definition of complexity such as the decidability of a computational problem. 
The first two consider classes based on sequences of quantum states preparable by a polynomial-sized quantum circuit, where the circuits are uniformly generated by a computational class, for instance, the class $\mathsf{PSPACE}$, which results in the complexity class $\mathsf{StatePSPACE}$~\cite{RosenthalYuen:2022,MetgerYuen:2023}.
The third notion considers a relative complexity, where the complexity is measured between two given states, and is measured by the number of gates, from a given gate-set, required to transform one state in another state~\cite{Aaronson:2004}. 
For our definition of state preparation complexity, we drop the uniformity constraint from~\cite{RosenthalYuen:2022,MetgerYuen:2023} and define a class as $\mathsf{StateX}$, which refers to states preparable by circuits of type $\mathsf{X}$. 
As an example, if $\mathsf{X} = \QNC^0$, this results in the class $\mathsf{StateQNC^0}$, which is the set of states preparable from the $\ket{0}^n$ state by poly-size constant-depth circuits. 
This notion is similar to the relative complexity from~\cite{Aaronson:2004}, where one state is the  $\ket{0}^n$ state and instead of counting the number of gates we consider the set of states preparable by a fixed number of gates. Using this notion of complexity we show that any state preparable by an $\LAQCC^*$ circuit is also preparable by a $\mathsf{PostQPoly}$ circuit, the class of circuits of polynomial depth with an additional post-selection gate. 

All Clifford circuits have a constant-depth $\LAQCC$ implementation, implying that any stabilizer state can be implemented by a constant-depth $\LAQCC$ circuit, see Section~\ref{sec:clifford_circuits} for a proof of this statement. 
Efficient circuits for stabilizer states have been known already through measurement-based quantum computing. Therefore this paper focuses on the preparation of non-stabilizer states, and as a surprising result we find novel constant-depth protocols for four very natural classes of non-stabilizer states.
Despite the extensive research into these four classes of non-stabilizer states and the many applications of them, no efficient constant- or low-depth state preparation protocols are known yet. We specifically consider these four classes as they are all often used as initial states in other algorithms.

The first state is a uniform superposition over an arbitrary number of states. 
This state finds applications in many quantum algorithms, as they often start with a uniform superposition over multiple states. 
This superposition is often achieved by applying Hadamard gates to every qubit due to its simplicity to prepare. 
Yet, the analysis of many algorithms, such as Shor's algorithm~\cite{Shor:1997}, would benefit from a different initial superposition. 
The circuit to prepare the uniform superposition over an arbitrary number of states uses an exact version of Grover search as a subroutine, that turns a probabilistic circuit, with a known constant probability of success, into a deterministic circuit. 
We use the circuit for preparing a uniform superposition over an arbitrary number of states as a subroutine in the next two quantum state preparation protocols. 

The second state is the $W$-state, the uniform superposition over all computational basis states of Hamming-weight~$1$, a natural long-ranged entangled state that displays a fundamentally nonequivalent type of entanglement from the Greenberger–Horne–Zeilinger state~\cite{WState:2000}, for which $\LAQCC$-type constant-depth circuits were previously known~\cite{PhamSvore2013, Cirac:2021}. 
The $W$-state is often used as benchmark for new quantum hardware~\cite{Haffner2005,Neeley2010,GarciaPerez:2021}. 
A novel way to prepare the $W$-state therefore gives a new way to benchmark different quantum devices with each other. 
A circuit for preparing the $W$-state was given in~\cite{Cirac:2021}, but this implementation requires sequentially alternating measurements followed by local unitaries, which in the $\LAQCC$ model is not considered to be of constant depth. 
We improve this protocol by giving an $\LAQCC$ implementation of the $W$-state, based on a compress-uncompress method that links the one-hot and binary encoding of integers.

The third state considered is the Dicke state, a generalization of the $W$-state, a superposition over all computational basis states with Hamming-weight $k$~\cite{Dicke:1954}. 
Dicke states have relevance in various practical settings.
For instance, for quantum game theory~\cite{zdemir2007}, quantum storage~\cite{Bacon_Compress:2006,Plesch:2010}, quantum error correction~\cite{ouyang2014permutation}, quantum metrology~\cite{toth2012multipartite}, and quantum networking~\cite{prevedel2009experimental}. 
Dicke states have been used as a starting state for variational optimization algorithms, most notably Quantum Alternating Operator Ansatz (QAOA)~\cite{Hadfield2019}, to find solutions to problems such as Maximum k-vertex Cover~\cite{Brandhofer2022,cook2020quantum}.
The ground states of physical Hamiltonians describing one-dimensional chains tend to show a resemblance to Dicke states such as states resulting from the Bethe ansatz, making them an ideal starting state when investigating the ground state behavior of these Hamiltonians~\cite{TDL_BetheAnsatzDerivation:2010,B_ExcitedStateQuantumPhaseTransitions:2013,DickeTransitions:2021}. 
For instance, the algorithm by \citeauthor{van2021preparing}, who give an algorithm to prepare the Bethe ansatz eigenstates of the spin-1/2 XXZ spin chain, starts by first preparing a Dicke state~\cite{van2021preparing}. 
A Dicke-state preparation protocol based on the compress-uncompress methodology used in the $W$-state furthermore finds applications in entanglement distillation, where the entanglement of a large state is concentrated on only a few qubits. 
Efficient deterministic circuits for preparing Dicke states have been proposed by \citeauthor{bartschi2019deterministic}~\cite{bartschi2019deterministic, bartschi2022deterministic_short_depth}. 
They provide a quantum circuit of depth $\mathO(k \log(\frac{n}{k}))$, allowing arbitrary connectivity, to prepare a Dicke state, which they conjecture to be optimal when $k$ is constant. 
In this work, we provide a constant-depth $\LAQCC$ circuit below their conjectured bound already for constant $k$. 
However, this does not directly disprove their conjecture, as we allow for intermediate measurements and classical computations. 
More significantly, we even construct constant-depth $\LAQCC$ circuits for $k = \mathO(\sqrt{n})$ greatly improving their bound.
This construction extends the compress-uncompress method for the $W$-state combined with additional subroutines. 

We continue with a log-depth state preparation protocol for the Dicke-state for arbitrary $k$. 
This protocol implements an efficient transformation between the factoradic number representation and the combinatorial number representation of a positive integer. 
The combinatorial number representation relates directly to the Dicke state. 
The provided efficient transformation between number representation systems might be of independent interest. 

We conclude by modifying our protocol for preparing a Dicke-state to a protocol that prepares quantum many-body scar states in constant-depth. 
These states have low entanglement and longer coherence times than states with similar energy density.
These characteristics make many-body scar states interesting to analyze and relevant within physics.
Many-body scar states appear for instance in the AKLT model~\cite{AKLT:1987,MRBAR:2018,MRB:2018} and different spin models~\cite{SI:2019,MOBFR:2020}.
Known methods for preparing these states have polynomial-depth~\cite{Gustafson:2023}, whereas our circuit has constant depth. 

% We conclude by studying the power that intermediate classical calculations can add to quantum computations. 
% In this study, we define a new model that relaxes constant-depth quantum circuits to polynomial depth quantum circuits, log-depth classical calculations to unbounded classical computations and a constant number of alternations to a polynomial number of alternations. 
% We call this model $\LAQCC^*$. 
% We study this model by doing a complexity theoretical analysis, where we draw inspiration from the notions of complexity given by \citeauthor{RosenthalYuen:2022}, \citeauthor{MetgerYuen:2023}, and \citeauthor{Aaronson:2004}.
% All three complexity notions are based on the notion of state preparation, instead of more traditional definition of complexity such as the decidability of a computational problem. 
% The first two consider classes based on sequences of quantum states preparable by a polynomial-sized quantum circuit, where the circuits are uniformly generated by a computational class, for instance, the class $\mathsf{PSPACE}$, which results in the complexity class $\mathsf{StatePSPACE}$~\cite{RosenthalYuen:2022,MetgerYuen:2023}.
% The third notion considers a relative complexity, where the complexity is measured between two given states, and is measured by the number of gates, from a given gate-set, required to transform one state in another state~\cite{Aaronson:2004}. 
% For our definition of state preparation complexity, we drop the uniformity constraint from~\cite{RosenthalYuen:2022,MetgerYuen:2023} and define a class as $\mathsf{StateX}$, which refers to states preparable by circuits of type $\mathsf{X}$. 
% As an example, if $\mathsf{X} = \QNC^0$, this results in the class $\mathsf{StateQNC^0}$, which is the set of states preparable from the $\ket{0}^n$ state by poly-size constant-depth circuits. 
% This notion is similar to the relative complexity from~\cite{Aaronson:2004}, where one state is the  $\ket{0}^n$ state and instead of counting the number of gates we consider the set of states preparable by a fixed number of gates. Using this notion of complexity we show that any state preparable by an $\LAQCC^*$ circuit is also preparable by a $\mathsf{PostQPoly}$ circuit, the class of circuits of polynomial depth with an additional post-selection gate. 

\paragraph{Summary of results}
\begin{itemize}
    \item We give a new definition of a computational model that captures the power of the four step process: applying a constant number of layers of one- and two-qubit gates; performing a syndrome measurement; perform a fast classical computation determining corrections; apply corrections. We call this model \emph{Local Alternating Quantum Classical Computations}, or $\LAQCC$ for short. In this model we bound the allowed quantum operations, intermediate classical calculations, and number of rounds separately. In Section~\ref{sec:LAQCC_model} we define this model and give a list of operations based on results from literature contained in this computational model. In some of these operations we explicitly use that we allow for multiple, but at most constant, rounds  of corrections.
    \item  We show show that there exist $\LAQCC$ circuits that can not be weakly simulated in Section~\ref{sec:IQP_in_LAQCC}. We further show that for every $\LAQCC$ circuit there exists a $\QNC^1$ circuit simulating it perfectly, in Section~\ref{sec:LAQCC_in_QNC1}.
    \item We introduce a new type computational complexity for preparing states and show that the extension of $\LAQCC$ where we allow a polynomial number of rounds and unbounded classical computation, is contained in $\mathsf{PostQPoly}$, the class of polynomial circuits with post-selection, in Section~\ref{sec:Complexity results}.
    \item We show a protocol to prepare the uniform superposition state of size $q$ in $\LAQCC$ using $\mathO(\ceil{\log_2(q)}^2)$ qubits in Section~\ref{sec:superposition_modulo_q}. 
    \item We show a protocol to prepare the $W_n$ state in $\LAQCC$ using $\mathO(n\log(n))$ qubits in Section~\ref{sec:W_state_in_LAQCC}.
    \item We show two ways of preparing the Dicke-$(n,k)$ state. The first method is in $\LAQCC$, works up to $k = \mathO(\sqrt{n})$, uses $\mathO(n^2\log(n))$ qubits, and is found in Section~\ref{sec:dicke:small_k}. The second method is in $\LAQCC\text{-}\mathsf{LOG}$ (an extension of $\LAQCC$ allowing for logarithmic number of alterations instead of constant), works for any $k$, uses $\mathO(\text{poly}(n))$ qubits, and is found in Section~\ref{sec:Dicke_in_LAQCC_LOG}. 
    \item We extend on our $\LAQCC$ method of generating Dicke-$(n,k)$ states for $k = \mathO(\sqrt{n})$ and show a protocol to generate many-body scar states for a particular Hamiltonian in $\LAQCC$ (Section~\ref{sec:many_body_scar}). 
\end{itemize}
Summarized in a table, we provide the following state generation protocols:
\begin{table}[htb]
\centering
\begin{tabular}{l|l|l|l}
\textbf{State description} & \textbf{Width} & \textbf{Depth} & \textbf{Implementation}\\
\hline 
Uniform superposition mod $q$: $\frac{1}{\sqrt{q}} \sum_{i = 0}^{q-1}\ket{i}$ & $\mathO(\ceil{\log^2 q})$ & $\mathO(1)$ & Section~\ref{sec:superposition_modulo_q}\\

$W$-state: $\frac{1}{\sqrt{n}}\sum_{i = 0}^{n-1}\ket{e_i}$ & $\mathO(n \log n)$ & $\mathO(1)$ & Section~\ref{sec:W_state_in_LAQCC}\\

Dicke-$(n,k)$, $k = \mathO(\sqrt{n})$: $\binom{n}{k}^{-1/2}\sum_{x \in \{0,1\}^n: |x| = k} \ket{x}$ &  $\mathO(n^2\log n)$ & $\mathO(1)$ 
&Section~\ref{sec:dicke:small_k}\\

Dicke-$(n,k)$: $\binom{n}{k}^{-1/2}\sum_{x \in \{0,1\}^n: |x| = k} \ket{x}$ & $\mathO(\text{poly}(n))$ & $\mathO(\log n)$ &Section~\ref{sec:Dicke_in_LAQCC_LOG}\\

QMBS: $\ket{S_k} = \frac{1}{k! \sqrt{\mathcal N(n,k)}}(Q^\dagger)^k \ket{\Omega}$ &  $\mathO(n^2\log n)$ & $\mathO(1)$  &  Section~\ref{sec:many_body_scar}
\end{tabular}
\caption{Summary of state preparation protocols given in this paper.}
\label{tab:sate_prep}
\end{table}
In the entry for the quantum many-body scar state $Q$ denotes the raising operator and $\mathcal N(n,k)=\binom{n-k-1}{k}$. 
Section~\ref{sec:many_body_scar} will provide more details on the variables and the implementation. 

\paragraph{Organization of the paper}
\noindent We first introduce relevant preliminaries in Section~\ref{sec:preliminaries}. 
In Section~\ref{sec:LAQCC_model} we formally define the class of Local Alternating Quantum-Classical Computations ($\LAQCC$). We also show that any Clifford circuit can be implemented in constant depth $\LAQCC$ (a result based on a result from measurement-based quantum computing~\cite{jozsa2006introduction}). 
This result allows us to give many useful multi-qubit gates and routines in Section~\ref{sec:gates_created_in_LAQCC}. 
Beyond that we show that constant depth $\LAQCC$ circuits are contained in $\QNC^1$ and that any $\mathsf{IQP}$ circuit has an $\LAQCC$ implementation.
We conclude this section with an analysis of a more powerful instantiation of $\LAQCC$ and show an inclusion with respect to the class $\mathsf{PostQPoly}$, which is the class of circuits of polynomial depth with one additional post-selection gate. 
In Section~\ref{sec:state_prep_in_LAQCC} we give $\LAQCC$ circuit implementations for preparing the uniform superposition over an arbitrary number of states, the $W$-state and the Dicke state up to $k = \mathO(\sqrt{n})$. We furthermore give a log-depth circuit implementation for preparing the Dicke state for any $k$. We conclude by showing a $\LAQCC$ circuit for generating many body scar states of a particular type of Hamiltonian.



%
\section{Labelled Transition Systems with Independence}\label{sec:LTSIs}
We want to study reversibility in a setting as general as possible.
Thus, we base on the core of the notion of \emph{labelled
  transition system with independence} (LTSI)~\cite[Definition~3.7]{SNW96}. However,
while~\cite{SNW96} requires a number of axioms on LTSI, we take the
basic definition and explore what can be done by adding or not adding
various axioms. Also, we extend LTSI with reverse transitions, since we
study reversible systems. We define first labelled transition
systems (LTSs).

We consider the LTS of the entire set of processes in a calculus,
rather than the transition graph of a particular process and its derivatives, hence we do not fix an initial state.

\begin{definition}\label{def:lts}
A {\em labelled transition system (LTS)\/} 
is a structure \mbox{$(\Proc,\Lab,\tran{})$}, where $\Proc$ is the set of states (or processes),
$\Lab$ is the set of action labels and ${\tran{}}\subseteq \Proc \times \Lab \times \Proc$
is a {\em transition relation\/}. 
\end{definition}
We let $P,Q,\ldots$ range over processes,
$a,b,c,\ldots$ range over labels,
and $t,u,v,\ldots$ range over transitions.
We can write $t:P \tran a Q$ to denote that $t = (P,a,Q)$.
We call $a$-transition a transition with label $a$.
    
\begin{definition}[LTS with independence]\label{def:ltsi}
We say that $(\Proc,\Lab,\tran{},\ind)$ is an \emph{LTS with
  independence} (LTSI) if $(\Proc,\Lab,\tran{})$ is an LTS and $\ind$
is an irreflexive symmetric binary relation on transitions.
\end{definition}
In many cases (see Section~\ref{sec:casestudies}), the notion of
independence coincides with the notion of concurrency. However, this
is not always the case. Indeed, concurrency implies that transitions
are independent since they happen in different processses, but
transitions taken by the same process can be independent as
well. Think, for instance, of a reactive process that may react in any
order to two events arriving at the same time, and the final result
does not depend on the order of reactions.

%We remark that independence may or may not coincide with concurrency.
%We will not speak about concurrency in this paper, however one can
%notice that in all the instances any reasonable notion of concurrency
%implies the notion of independence, while the opposite is not
%necessarily true.

We shall assume that all transitions are reversible,
so that the Loop Lemma~\cite[Lemma 6]{DK04} holds.
% A more general treatment would allow for some transitions to be irreversible.
This does not hold in models of reversibility with control mechanisms~\cite{LaneseMS12}
such as irreversible actions~\cite{DK05} or a rollback operator~\cite{LMSS11}.
Nevertheless,
when showing properties of models with controlled reversibility it has
proved sensible to first consider the underlying models where all transitions
are reversible, and then study how control mechanisms change the
picture~\cite{GiachinoLMT17,LaneseNPV18}.
The present work helps with the first step.
\begin{definition}[Reverse and combined LTS]\label{def:reverse transition}
Given an LTS 
$(\Proc,\Lab,\ftran{})$,
let the \emph{reverse LTS} be $(\Proc,\Lab,\rtran{})$,
where $P \rtran a Q$ iff $Q \ftran a P$.
It is convenient to combine the two LTSs (forward and reverse):
let the reverse labels be
$\rev\Lab = \{\rev a : a \in \Lab\}$,
and define the combined LTS to be
${\tran{}}\subseteq \Proc \times (\Lab \cup \rev\Lab) \times \Proc$
by $P \tran a Q$ iff $P \ftran a Q$ and $P \tran{\rev a} Q$ iff $P \rtran a Q$.
\end{definition}
We stipulate that the union $\Lab \cup \rev\Lab$ is disjoint.
We let $\alpha,\ldots$ range over $\Lab \cup \rev\Lab$.
For $\alpha \in \Lab \cup \rev\Lab$, the \emph{underlying} action label
$\und\alpha$ is defined as
$\und{a} = a$ and $\und{\rev a} = a$.
%if $\alpha \in \rev\Lab$.
%
%where $\alpha = a$ if $\alpha \in \Lab$
%and $\alpha = \rev a$ if $\alpha \in \rev\Lab$.
Let $\rev{\rev a} = a$ for $a \in \Lab$.  Given $t:P \tran \alpha Q$, let $\rev t:Q \tran {\rev\alpha} P$ be the transition which reverses $t$.
We define a labelling function $\lab$ from transitions to $\Lab\cup \rev\Lab$ by setting
$\lab((P,\alpha,Q)) = \alpha$.


We let $\rho,\sigma,\ldots$ range over finite sequences $\alpha_1 \ldots \alpha_n$,
with $\es$ representing the empty sequence. %starting and ending at $P$.
%with $\es_P$ representing the empty sequence starting and ending at $P$.
%We shall write $\es$ when $P$ is understood.
Given an LTS, a \emph{path} is a sequence of forward or reverse transitions
of the form
$P_0 \tran{\alpha_1} P_1 \cdots \tran{\alpha_n} P_n$.
We let $r,s,\ldots$ range over paths.
We may write $r:P \ptran \rho Q$ where the intermediate states are understood.
On occasion we may refer to a path simply by its sequence of labels $\rho$.
The concatenation of paths $r$ and $s$ is written $rs$.
Given a path $r:P \ptran \rho Q$, the inverse path is $\rev r:Q \ptran {\rev \rho} P$
where $\rev\es = \es$ and $\rev{\alpha\rho} = \rev \rho \; \rev \alpha$.
The length of a path $r$ (notated $\len r$) is the number of transitions in the path.
Paths $r:P \ptran \rho Q$ and $R \ptran \sigma S$ are
\emph{coinitial} if $P = R$ and \emph{cofinal} if $Q = S$.
We say that a path is \emph{forward-only} if it contains no reverse transitions;
similarly a path is \emph{backward-only} if it contains no forward transitions.
Sometimes we let $f,\ldots$ and $b,\ldots$ range over forward-only and backward-only paths, respectively;
it will be clear from the context whether $b$ represents an action label or a path.

Let $(\Proc,\Lab,\tran{})$ be an LTS. The irreversible processes in $(\Proc,\Lab,\tran{})$ are
$\Irr = \{P \in \Proc: P \not \rtran{}\}$.
A \emph{rooted path} is a
path $r:P \ptran \rho Q$ such that $P \in \Irr$.

In the following we consider LTSIs obtained by adding a notion of
independence to combined LTSs as above. We call the result a
\emph{combined LTSI}.

\begin{remark}
From now on, unless stated otherwise, we consider a combined LTSI $\mc L = (\Proc,\Lab,\tran{},\ind)$. We will refer to it simply as an LTSI.     
\end{remark}


%
\section{Basic Properties}\label{sec:basic}
In this section we show that most of the properties in the reversibility literature (see, e.g.,
\cite{DK04,PU07,LaneseMS16,LaneseNPV18}), in particular the Parabolic Lemma and
Causal Consistency, can be proved under minimal assumptions on the
combined LTSI under analysis.

We formalise the minimal assumptions using three axioms, described below.
\begin{definition}[Basic axioms]\label{def:basic}
  %Let $\mc L = (\Proc,\Lab,\tran{},\ind)$ be a combined LTSI.
  We say an LTSI %$\mc L$
  satisfies:
  \begin{description}
\item[Square property (SP)]\!\!: if whenever $t:P \tran \alpha Q$, $u:P
    \tran \beta R$ with $t \ind u$ then there are cofinal transitions
    $u': Q \tran \beta S$ and $t':R \tran \alpha S$;%\vspace{1pt}
\item[Backward transitions are independent (BTI)]\!\!: if whenever $t:P \rtran{a} Q$ and $t': P \rtran{b} Q'$ 
     and $t \neq t'$ then $t \ind t'$;%\vspace{3pt}
\item[Well-founded (WF)]\!\!: if there is no infinite reverse
    computation, i.e.\ we do not have $P_i$ (not necessarily distinct)
    such that $P_{i+1} \tran {a_i} P_i$ for all $i = 0,1,\ldots$.
  \end{description}
\end{definition}
WF can alternatively be formulated using backward transitions,
but the current formulation makes
sense also in non-reversible calculi (e.g., CCS), which can be used as a comparison.
Let us discuss the intuition behind these axioms. SP takes its
name from the Square Lemma, where it is proved for concrete calculi and
languages~in~\cite{DK04,LaneseMS16,LaneseNPV18}, and
captures the idea that independent transitions can be executed in any
order, that is they form commuting diamonds. SP can be seen as a
sanity check on the chosen notion of independence. BTI
generalises the key notion of backward determinism used in sequential
reversibility (see, e.g., \cite{Pin87} for finite state
automata and \cite{YokoyamaG07} for the imperative
language Janus) to a concurrent setting.  Backward determinism can be
spelled as ``two coinitial backward transitions do coincide''. This can be generalised to
``two coinitial backward transitions
are independent''.
We will show in Proposition~\ref{prop:sequential} that the two definitions are equivalent  when no transitions are independent, 
which is the common setting in sequential computing.
Note that BTI and SP together rule out examples $a.\nil \tran a \nil$, $b.\nil \tran b \nil$ as well as $a.\nil + b.\nil \tran a \nil$, $a.\nil + b.\nil \tran b \nil$ from the Introduction.
Finally, WF means that we consider systems which have a
finite past. That is, we consider systems starting from some initial
state and then moving forward and back.
WF rules out example $P \tran a P$ where $P = a.P$ from the Introduction.

Axioms SP and BTI are related to properties which are part of the definition
of (occurrence) transition systems with independence in~\cite[Definitions~3.7, 4.1]{SNW96}.
WF was used as an axiom in~\cite{PU07a}.

%\begin{definition}[Backward transitions are independent (BTI)]\label{def:bti}
%  If $t:P \rtran{a} Q$ and $t': P \rtran{b} Q'$ and $t \neq t'$ then $t \ind t'$. 
%\end{definition}
%
%A main property of independent transitions is that they form commuting
%diamonds.
%
%\begin{definition}[Square property (SP)]\label{def:sp}
%Whenever $t:P \tran \alpha Q$, $u:P \tran \beta R$ with $t \ind u$
%then there are cofinal transitions $u': Q \tran \beta S$ and $t':R
%\tran \alpha S$.
%\end{definition}
%The property above takes the name of Square Lemma in the different
%instances in the literature, when it is proved for concrete calculi
%and languages.
%
%\begin{definition}\label{def:wfut}
%  An LTS $(\Proc,\Lab,\tran{})$ satisfies:
%{\bf WF} (well-founded) if there is no infinite reverse computation,
%i.e.\ we do not have $P_i$ (not necessarily distinct)
%such that $P_{i+1} \tran {a_i} P_i$ for all $i = 0,1,\ldots$
%\end{definition}
%This gives our three basic axioms: BTI, SP and WF.

Using the minimal assumptions above we can prove relevant results
from the literature. As a preliminary step, we define causal equivalence,
equating computations differing only for swaps of independent
transitions and simplification of a transition
with its reverse.

\begin{definition}[Causal equivalence, cf.~{\cite[Definition 9]{DK04}}]\label{def:ceqt}
  %Let $(\Proc,\Lab,\tran{},\ind)$ be an
  Consider an LTSI satisfying SP.
Let $\ceqt$ be the smallest equivalence relation on paths closed under composition and satisfying:
\begin{enumerate}
\item
(swap)  
if $t:P \tran \alpha Q$, $u:P \tran \beta R$ are independent,
and $u': Q \tran \beta S$, $t':R \tran \alpha S$ (which exist by SP)
then $tu' \ceqt ut'$;
\item
(cancellation)
$t \rev t \ceqt \es$ \quad and \quad $\rev t t \ceqt \es$.
\end{enumerate}
\end{definition}

We first consider the Parabolic Lemma \cite[Lemma
  10]{DK04}, which states that each path is causal equivalent to a
backward path followed by a forward path.

\begin{definition}\label{def:PL}
  {\bf Parabolic Lemma property (PL)}: for any path $r$ there are forward-only paths $s,s'$ such that 
$r \ceqt \rev s s'$ and $\len s + \len {s'} \leq \len r$.
\end{definition}

\begin{proposition}\label{prop:PL}
Suppose an LTSI satisfies BTI and SP.  Then PL holds.
\end{proposition}

\begin{proof}
Suppose BTI and SP hold.
Define a function on paths as follows:
$d(r)$ is the number of pairs of forward transitions $(t,u)$
such that $t$ occurs in any position to the left of $\rev u$ in $r$.
%$(a,b)$ such that $a$ occurs to the left of $\rev b$ in $s$.
We say $r$ is parabolic iff $d(r) = 0$. We have to show that each path is causal equivalent to a parabolic one.

Suppose $d(r) > 0$.
We show that there is $s \ceqt r$ with $\len {s} \leq \len r$ and $d(s) < d(r)$.
Since $d(r) > 0$, we have $r = s_1 t \rev u s_2$ with
$s_1:P \tran {\sigma_1} R$,
$t:R \tran a S$, $\rev u:S \tran {\rev b} T$ and $s_2:T \tran {\sigma_2} Q$.
If $t = u$, then we obtain $r = s_1 t \rev u s_2 \ceqt s_1 s_2$.
Clearly $r \ceqt s_1s_2$ with $\len {s_1s_2} < \len r$ and $d(s_1s_2) < d(r)$.
%
So suppose $t \neq u$.
By BTI we have $\rev t \ind \rev u$.
By SP there are $S'$ and transitions $u':S' \tran b R$, $t':S' \tran a T$.
See Figure~\ref{fig:pl}.
% Figure environment removed
Then $\rev t \, \rev{u'} \ceqt \rev u \, \rev{t'}$.
Hence, $r = s_1 t \rev u s_2
\ceqt s_1 t \rev u \, \rev{t'} t' s_2
\ceqt s_1 t \rev t \, \rev{u'} t' s_2
\ceqt s_1 \rev{u'} t' s_2 = s$ as required.
Given that $\len {s_1 \rev {u'} t' s_2} = \len r$ and $d(s_1 \rev {u'} t' s_2) = d(r)-1$ the thesis follows.
\end{proof}
The proof of Proposition~\ref{prop:PL} is very similar to that
of~\cite[Lemma~10]{DK04} except that in the latter BTI is shown directly as part of the proof.

A corollary of PL is that if a process is reachable
from an irreversible process, then it is also forwards reachable from it. In other words,
making a system reversible does not introduce new reachable states
but only allows one to explore forwards-reachable states in a different order. 
This is relevant, e.g., in reversible debugging of concurrent
systems~\cite{GiachinoLM14,LaneseNPV18}, where one wants to find bugs that actually 
occur in forward-only computations.
\begin{corollary}\label{freachable}
  Suppose an LTSI satisfies PL. If a process $P$ is reachable
  from some irreversible process $Q$, then it is also forward
  reachable from $Q$.
\end{corollary}
\begin{proof}
  By hypothesis, there is some path $r: Q \ptran{} P$. Thanks to PL,
  there are forward-only paths $s,s'$ such that $\rev s s': Q \ptran{}
  P$.  Since $Q$ is irreversible, $s = \es$, hence $s': Q \ptran{}
  P$ as desired.
\end{proof}
When WF and PL hold, each process is reachable from a unique irreversible process.
\begin{proposition}\label{prop:unique irrev}
Suppose an LTSI satisfies WF and PL.
For any process $P$ there is a unique irreversible process $I$ such that $P$ is reachable from $I$.
\end{proposition}
\begin{proof}
Let $P$ be any process.
We use WF to deduce that there is an irreversible process $I$ such that $P$ is (forward) reachable from $I$ via some path $r$.
Suppose now that $I'$ is irreversible  and there is a path $r'$ from $I'$ to $P$.
Then $r'\rev r: I' \ptran{} I$.
By PL there are forward-only paths $s,s'$ such that $\rev s s': I' \ptran{} I$.
But since $I$ and $I'$ are irreversible, both $s = \es$ and $s' = \es$.
Hence $I' = I$ as required.
\end{proof}

We now move to causal consistency~\cite[Theorem 1]{DK04}.

\begin{definition}\label{def:cc}
{\bf Causal Consistency (CC)}: if $r$ and $s$ are coinitial and cofinal paths then $r \ceqt s$.
\end{definition}

Essentially, causal consistency states that history information allows
one to distinguish computations which are not causal equivalent.
Indeed, if two computations are cofinal, that is they reach the same
final state (which includes the stored history information) then they
need to be causal equivalent.

Causal consistency frequently includes the other direction, namely that
coinitial causal equivalent computations are cofinal, meaning that
there is no way to distinguish causal equivalent computations. This
second direction follows easily from the definition of causal equivalence.

Notably, our proof of CC below is very much shorter than existing
proofs, such as the one of \cite[Theorem 1]{DK04} for RCCS and the one
of \cite[Theorem 21]{LaneseNPV18} for reversible Erlang.

\begin{proposition}\label{prop:PL WF CC}
Suppose an LTSI satisfies WF and PL. 
Then CC holds.
\end{proposition}
\begin{proof}
Let $r:P \ptran \rho Q$ and $r':P \ptran {\rho'} Q$.
Using WF, let $I,s$ be such that $s:I \ptran \sigma P$, $I \in \Irr$.
Now $sr\rev{sr'}$ is a path from $I$ to $I$,
and so by PL there are $r_1,r_2$ forward-only such that $\rev{r_1}r_2 \ceqt sr\rev{sr'}$.
But $I \in \Irr$ and so $r_1 = \es$ and $r_2 = \es$.
Thus $\es \ceqt sr\rev{sr'}$, so that $sr \ceqt sr'$ and (by composing with \rev{s} on the left) $r \ceqt r'$
as required.
\end{proof}

Causal equivalent computations are strongly related
%quite close
in terms of the number
of transitions with a given label they contain.

\begin{proposition}\label{prop:count ceqt}
If $r \ceqt s$ then for any action $a$ the number of $a$-transitions in $r$ is the same as in $s$, where we count reverse transitions negatively.
\end{proposition}
\begin{proof}
  Straightforward, by induction on the derivation of $r \ceqt s$.
\end{proof}
\begin{remark}\label{rem:count fwd ceqt}
One consequence of 
Proposition~\ref{prop:count ceqt} is that if $r \ceqt s$ and $r$ and $s$
are both forward-only, then $\len r = \len s$.
\end{remark}

Causal consistency implies the unique transition property.

%\todo{Check if better to use also for both backwards}
%\todo{Yes, cfr. Lemma~\ref{lem:non-degenerate}}
\begin{definition}\label{def:ut}
  %An LTSI $(\Proc,\Lab,\tran{},\ind)$ satisfies
  \textbf{Unique transition (UT)}:
  if either $P \tran a Q$ and $P \tran b Q$ or $P \rtran a Q$ and $P \rtran b Q$ then $a = b$.
\end{definition}

%\begin{restatable}{corollary}{UT}\label{cor:ut}
\begin{corollary}\label{cor:ut}
  If an LTSI satisfies CC then it satisfies UT.
\end{corollary}
%\end{restatable}
\begin{proof}
  Since $P \tran a Q$ and $P \tran b Q$ are coinitial and cofinal then
  they are causal equivalent. By Proposition~\ref{prop:count ceqt} the
  counting of actions  %events 
  should be the same, hence $a=b$.
\end{proof}
UT was shown in the forward-only setting of occurrence TSIs in~\cite[Corollary~4.4]{SNW96};
it was taken as an axiom in~\cite{PU07a}. 
% We note that all our axioms are new apart from WF (and UT which is not central to our axiomatic approach).
% When we introduce further axioms later on, we shall comment if they relate to previousely proposed 
% axioms.
\begin{example}[PL alone does not imply WF or CC]\label{ex:PL not CC}
Consider the LTSI with states $P_i $ for $i = 0,1,\ldots$ and
transitions $t_i:P_{i+1} \tran a P_i$, $u_i:P_{i+1} \tran b P_i$ with $a \neq b$ and $\rev{t_i} \ind \rev{u_i}$.
BTI and SP hold.
Hence PL holds by Proposition~\ref{prop:PL}.
However clearly WF fails.
Also $t_i $ and $u_i$ are coinitial and cofinal,
% but we see that $t_i \ceqt u_i$ does not hold using
% Proposition~\ref{prop:count ceqt} and $a\neq b$.
% Hence CC fails.
and $a \neq b$, so that UT fails, and hence CC fails using Corollary~\ref{cor:ut}.
Note that the $ab$ diamonds here have the same side states so are degenerate (cf.~Lemma~\ref{lem:non-degenerate}).\finex
\end{example}
We have seen that SP is assumed when defining causal equivalence $\ceqt$.
%On the assumption of
Assuming SP, we give a diagram (Figure~\ref{fig:basic2}) to show implications between the remaining two axioms presented so far (BTI, WF) and the two main properties introduced so far (PL, CC).
We remark that the implications shown are strict (reverse implication does not hold).
%solid arrow head is strict implication
%open arrow head means it is not known if the implication is strict.
%% % Figure environment removed
%% \todo{Perhaps too many properties in Figure~\ref{fig:basic1}.
%% I put them all in to see what it looked like.
%% Probably best to omit UT altogether as in Figure~\ref{fig:basic2}.
%% Questions remain about whether implications are strict, etc.}
% Figure environment removed
We provide below counterexamples showing strictness of implications:
\begin{example}[SP, WF and CC do not imply PL]\label{ex:notPL}
  Consider the LTSI with states $P,Q,R$ and transitions $t:P\tran a R$, $u:Q \tran b R$,
  with an empty independence relation.
Then clearly BTI and PL fail.
However SP, WF and CC (and therefore UT) hold.

For CC, note that % the LTS is a path graph and
we can use cancellation to reduce each path to a unique shortest normal form
with respect to $\ceqt$.
There are various cases to check, depending on the initial and final states of the path, both ranging over $P,Q,R$.
Let $r:R \ptran\rho R$ be any path from $R$ to $R$.
If $r$ is non-empty, it must be of the form either $r = \rev t t r'$ or $r = \rev u u r''$.
We can use cancellation to get either $r \ceqt r'$ or $r \ceqt r''$.
Iterating the argument we see that $r \ceqt \es$. 
Now let $r:P \ptran\rho R$ be any path from $P$ to $R$.
Then $r = tr'$ where $r'$ is a path from $R$ to $R$.
Hence $r \ceqt t$.
Now let $r:P \ptran\rho P$ be any path from $P$ to $P$.
Then $r = tr'\rev t$ where $r'$ is a path from $R$ to $R$.
Hence $r \ceqt t\rev t \ceqt \es$.
Next let $r:P \ptran\rho Q$ be any path from $P$ to $Q$.
Then $r = tr'\rev u$ where $r'$ is a path from $R$ to $R$.
Hence $r \ceqt t\rev u$.
The remaining cases are similar.\finex
\end{example}
\begin{example}[SP, WF, PL and CC do not imply BTI]\label{ex:notBTI}
Consider the LTSI with states $P,Q,R,S$ and transitions $t:P\tran a Q$, $u:P \tran b R$,
$t':R \tran a S$ and $u':Q \tran b S$, with $t \ind u$.
Then BTI fails for $\rev{t'}$ and $\rev{u'}$.
However SP, WF and PL 
hold, and therefore CC also holds.

We show PL. %\todo{perhaps can be shortened}.
As in the proof of Proposition~\ref{prop:PL}, for a path $r$ let
$d(r)$ be the number of pairs of forward transitions $(t,u)$
such that $t$ occurs to the left of $\rev u$ in $r$.
Then $r$ is parabolic iff $d(r) = 0$.

Suppose $d(r) > 0$.
We show that there is $s \ceqt r$ with $\len {s} \leq \len r$ and $d(s) < d(r)$.
Since $d(r) > 0$, we have $r = s_1 t'' \rev{u''} s_2$.
If $t'' = u''$, then we can use cancellation as in the proof of Proposition~\ref{prop:PL}. 
So suppose $t'' \neq u''$.
Since the target of $t''$ must be the same as the source of $u''$,
the only possibilities are
$t'' = t'$, $u'' = u'$ or dually $t'' = u'$, $u'' = t'$.
We consider $t'' = t'$, $u'' = u'$;
the other case is similar.
So $r = s_1 t' \rev{u'} s_2$.
Since $t \ind u$ we have $tu' \ceqt ut'$.
Hence $\rev u t u' \rev{u'} \ceqt \rev u u t' \rev{u'}$,
and so $\rev u t \ceqt t' \rev{u'}$.
So $r \ceqt s = s_1 \rev u t s_2$ and $d(s) = d(r)-1$, $\len s = \len r$.\finex
\end{example}
\begin{example}[SP and WF do not imply CC (or PL)]\label{ex:WFnotCC}
Consider the LTSI of Example~\ref{ex:notBTI},
but without $t \ind u$.
Clearly SP and WF hold.
However CC fails, since there are paths $tu'$ and $ut'$ from $P$ to $S$,
but $tu' \not\ceqt ut'$.
To see this, imagine that the four transitions of the diamond correspond to
rotations around the centre of the diamond
(see Figure~\ref{fig:WFnotCC}).
% Figure environment removed
Measuring anti-clockwise rotation in radians
we see that $t$ and $u'$ each give a rotation of $-\pi/2$,
while $u$ and $t'$ each yield $+\pi/2$.
Let us define the rotation of a path to be the sum of the rotations of its transitions.
Path $tu'$ has rotation $-\pi$ while $ut'$ has $+\pi$.
Since there are no independent transitions,
the only operation of causal equivalence we can perform is to use
$t\rev t \ceqt \es$.
This clearly preserves the rotation of a path.
Hence $tu' \not\ceqt ut'$ as required.

PL does not hold either, otherwise CC would follow from Proposition~\ref{prop:PL WF CC}.\finex
\end{example}
\begin{example}[SP, BTI and CC do not imply WF]\label{ex:notWF}
Consider the LTSI with states $P_i $ for $i = 0,1,\ldots$ and
transitions $t_i:P_{i+1} \tran a P_i$.
Clearly WF does not hold.
However SP, BTI (and hence PL) hold; also CC (and hence UT) hold, noting that any
path is causally equivalent to a path which is entirely forward or entirely reverse.\finex
\end{example}


\section{Events}\label{sec:events}
In order to define and study causal safety and liveness (Section~\ref{sec:CSCL}),
we first need the concept of event.

%Definition of $\sqeqt$ with rotational symmetry and coinitial independence:

\begin{definition}[Event, general definition]\label{def:sqeqt}
  %Let $(\Proc,\Lab,\tran{},\ind)$ be an LTSI.
Consider an LTSI.  
Let $\sqeqt$ be the smallest equivalence relation satisfying:
if $t:P \tran \alpha Q$, $u:P \tran \beta R$,
$u':Q \tran \beta S$, $t':R \tran \alpha S$,
and $t \ind u$, $\rev u \ind t'$, $\rev{t'} \ind \rev{u'}$, $u' \ind \rev t$,
and
\begin{itemize}
\item
$Q \neq R$ if $\alpha$ and $\beta$ are both forwards or both backwards;
\item
$P \neq S$ otherwise;
\end{itemize}
then $t \sqeqt t'$.
The equivalence classes of transitions, written $[t]$ or $[P,\alpha,Q]$, are the \emph{events}.
We say that an event is \emph{forward} if it is the equivalence class of a forward transition;
similarly for \emph{reverse} events.
Given an event $e = [t]$ we let $\rev e = [\rev t]$.
Also, we let $\und{e}=e$ if $e$ is forward, $\und{e}=\rev e$ if $e$ is backward. 
\end{definition}
Intuitively, events are the equivalence classes generated by equating transitions on the opposite sides of commuting squares.
Events are introduced as a derived notion in an LTS with independence in~\cite{SNW96},
in the context of forward-only computation.
We have changed their definition by using coinitial independence at all corners
of the diamond,
yielding rotational symmetry.
This reflects our view that forward and backward transitions have equal status. 

The labelling function $\lab$ can be extended to $\,{\tran{}}/\sqeqt$ since the label does not depend on the choice of the representative inside the equivalence class.

%  to $\Lab$ by setting
%$\lab([P,\alpha,Q]) = \alpha$.

\subsection{Pre-reversible LTSIs}

Our definition can be simplified if the LTSI, and independence in
particular, are well-behaved. Thus, we now add a further axiom related
to independence. This leads us to pre-reversible LTSIs.

\begin{definition}\label{def:PCI}
  {\bf Propagation of coinitial independence (PCI)}\footnote{PCI was called CPI (coinitial propagation of independence) in~\cite{LanesePU20}; we changed the terminology following a suggestion from Marco Bernardo to better match the intuition.}:
  if $t:P
    \tran \alpha Q$, $u:P \tran \beta R$, $u': Q \tran \beta S$ and
    $t':R \tran \alpha S$ with $t \ind u$, then $u' \ind \rev t$.
\end{definition}


% \begin{definition}[Independence axioms]\label{def:indep}
  % We say that a combined LTSI $\mc L = (\Proc,\Lab,\tran{},\ind)$ satisfies:
  % \begin{description}
  % \item[Reversing preserves independence (RPI)] if $t \ind t'$ then
    % $\rev t \ind t'$;
% \todo{RPI can be defined later and derived rather than being an axiom}
  % \item[Coinitial propagation of independence (PCI)] if whenever $t:P
    % \tran \alpha Q$, $u:P \tran \beta R$, $u': Q \tran \beta S$ and
    % $t':R \tran \alpha S$ with $t \ind u$, we have $u' \ind \rev t$.
  % \end{description}
% \end{definition}
% RPI fixes the interplay between independence and reversing, while
PCI
states that independence is a property of commuting diamonds more than
of their specific pairs of edges. Indeed,
% together with RPI,
it allows
independence to propagate around a commuting diamond.

\begin{definition}[Pre-reversible LTSI]\label{def:prerev}
If an LTSI satisfies axioms SP, BTI, WF and PCI,
we say that it is \emph{pre-reversible}.
\end{definition}
The name `pre-reversible' indicates that we expect to require further axioms,
but the present four are enough to ensure that LTSIs are well-behaved,
with events compatible with causal equivalence (cfr.~Lemma~\ref{lemma:cccount}). Pre-reversible axioms are separated from further
axioms by a dashed line in Table~\ref{t:list}.

A first consequence of PCI is that coinitial transitions with mutually inverse labels are not independent.
\begin{lemma}\label{lemma:revnotind}
Suppose that an LTSI satisfies PCI. 
If $t: P \tran\alpha Q$ and $u: P \tran{\rev\alpha} R$ are coinitial transitions with mutually inverse labels,
then $t \centernot\ind u$.
\end{lemma}
\begin{proof}
Suppose that $t: P \tran\alpha Q$ and $u: P \tran{\rev\alpha} R$
are independent.
Consider the degenerate diamond with two copies of $P$ and transitions $t,u,\rev t, \rev u$.
By applying PCI we deduce $\rev t \ind \rev t$, which contradicts irreflexivity of $\ind$.
\end{proof}
Additionally, we cannot have two different coinitial backward transitions with the same label.
%\todo{Two defs of backward det, one should change name, maybe Backward Label Determinism?}
\begin{definition}\label{def:BD}
{\bf Backward label determinism (BLD):
if}
$t: P \rtran a Q$ and $u: P \rtran a R$ are coinitial backward transitions
with the same label then $t = u$.
\end{definition}
\begin{proposition}\label{prop:BD}
Suppose that an LTSI satisfies SP, BTI and PCI. Then it satisfies BLD.
\end{proposition}
\begin{proof}
Suppose $t: P \rtran a Q$ and $u: P \rtran a R$.
Then if $t \neq u$ we have $t \ind u$ by BTI.
We can complete a diamond with $Q \rtran a S$, $R \rtran a S$ by SP.
But then $Q \tran a P$ and $Q \rtran a S$ are independent by PCI.
This is a contradiction of Lemma~\ref{lemma:revnotind}.
\end{proof}
A consequence of Lemma~\ref{lemma:revnotind} is that an LTSI
satisfying BTI and PCI cannot include a diamond
$P \tran a Q \tran a S$, $P \tran a R \tran a S$
where all four transitions have the same label.
This can be seen as ruling out \emph{autoconcurrency}~\cite{Bed91}.
%Note that the transitions coming from different threads in a CCSK~\cite{PU07} process such as $a \Par a$ will be distinguished \todo{actually the key could be the same, I believe in this case we have a conflict on the key?} using keys:
%$a \Par a \tran {a \key m } a \key m \Par a$.

The following non-degeneracy property was shown for occurrence transition systems with independence in~\cite[page~312]{SNW96}, which considers forward transitions only.
We have to cope with backward as well as forward transitions.
%
%\begin{restatable}{lemma}{nondegenerate}\label{lem:non-degenerate}
\begin{lemma}\label{lem:non-degenerate}
Suppose that an LTSI is pre-reversible.
If we have a diamond
$t:P \tran \alpha Q$, $u:P \tran \beta R$ with $t \ind u$
together with cofinal transitions $u': Q \tran \beta S$ and $t': R \tran\alpha S$,
then the diamond is \emph{non-degenerate},
meaning that $P,Q,R,S$ are distinct states.
\end{lemma}
%\end{restatable}
%
\begin{proof}
We note that CC holds; hence UT holds thanks to Corollary~\ref{cor:ut}.
By WF we see that $P \neq Q \neq S \neq R \neq P$.
It remains to show $Q \neq R$ and $P \neq S$.

Suppose $Q = R$.
By $t \ind u$ we know $t \neq u$.  So $\alpha \neq \beta$.
But if $\alpha$ and $\beta$ are both forward or both backward this is impossible by UT.
If one is forward and the other is backward then this is impossible by WF.
Hence $Q \neq R$.

Suppose $P = S$.
If $\alpha$ and $\beta$ are both forward or both backward this is impossible by WF.
If one is forward and the other is backward then by UT this implies that
$\alpha = \rev\beta$.
Then $t \centernot\ind u$ by Lemma~\ref{lemma:revnotind},
which is a contradiction.
% Hence $u' = \rev t$\il{, but this is not possible because of Lemma~\ref{lemma:revnotind}.}
%By PCI we have $\rev t \ind u'$,
%which contradicts irreflexivity of $\ind$.
Hence $P \neq S$.
\end{proof}

If an LTSI is pre-reversible then by
Lemma~\ref{lem:non-degenerate} and the use of PCI
we can simplify the statement of Definition~\ref{def:sqeqt}
to:

\begin{definition}[Event, simplified definition]\label{def:sqeqt simp}
  %Let $(\Proc,\Lab,\tran{},\ind)$ be
  Consider a pre-reversible LTSI.
Let $\sqeqt$ be the smallest equivalence relation satisfying:
if $t:P \tran \alpha Q$, $u:P \tran \beta R$,
$u':Q \tran \beta S$, $t':R \tran \alpha S$,
and $t \ind u$,
then $t \sqeqt t'$.
\end{definition}

We are now able to show independence of diamonds (ID), which can be seen as
dual of SP.

\begin{definition}\label{def:ID}
%  [Independence of Diamonds (ID)]
% Let $\mc L = (\Proc,\Lab,\tran{},\ind)$ be an LTSI.
% Then $\mc L$ satisfies the
%
% Could we do without stating $(\Proc,\Lab,\tran{},\ind)$ here? We have defined PCI without stating it.
%
%An LTSI $(\Proc,\Lab,\tran{},\ind)$ satisfies the
%An LTSI satisfies
{\bf Independence of Diamonds (ID)}: if we have a diamond
$t:P \tran \alpha Q$, $u:P \tran \beta R$,
$u': Q \tran \beta S$ and $t':R \tran \alpha S$,
with %$Q \neq R$
\begin{itemize}
\item
$Q \neq R$ if $\alpha$ and $\beta$ are both forwards or both backwards;
\item
$P \neq S$ otherwise;
\end{itemize}
then $t \ind u$.
\end{definition}
%\begin{restatable}{proposition}{ID}\label{prop:ID}
\begin{proposition}\label{prop:ID}
    If an LTSI satisfies BTI and PCI then it satisfies ID.
\end{proposition}
%\end{restatable}
\begin{proof}
Suppose we have a diamond
$t:P \tran \alpha Q$, $u:P \tran \beta R$,
$u': Q \tran \beta S$ and $t':R \tran \alpha S$,
with %$Q \neq R$
\begin{itemize}
\item
$Q \neq R$ if $\alpha$ and $\beta$ are both forwards or both backwards;
\item
$P \neq S$ otherwise.
\end{itemize}
We must show $t \ind u$.
There are various cases, depending on whether $\alpha$ and $\beta$ are forwards or backwards.
If they are both forwards, then $Q \neq R$.  Hence $\rev{t'} \neq \rev{u'}$
and by BTI we have $\rev{t'} \ind \rev{u'}$.
By PCI, $u' \ind \rev t$ and again by PCI $t \ind u$ as required. 
Other cases are similar.
\end{proof}
In the proof of the above proposition it must be the case that
$\und\alpha \neq \und\beta$, or else we get a contradiction using
Lemma~\ref{lemma:revnotind}.
  
%\todo{If prereversible then by BD (uses BTI, SP, PCI) we must have
%$\und\alpha \neq \und\beta$ in the above.}


% \todo{This does not make much sense if Proposition~\ref{prop:count ceqt} is in the appendix - it was needed for Example~\ref{ex:PL not CC}, but now we are using UT instead there.}
% We can now refine Proposition~\ref{prop:count ceqt} from actions to events.
\subsection{Counting occurrences of events}
We now consider the interaction between events and causal equivalence.
We need some notation first.

\begin{definition}\label{def:count events}
Let $r$ be a path in an LTSI $\mc L$ and let $e$ be an event of $\mc L$.
Let $\cte(r,e)$ be the number of occurrences of transitions $t$ in $r$
such that $t \in e$, minus the number of occurrences of transitions $t$ in $r$ such that $t \in \rev e$.
We define $\cte(r,e)$ by induction on the length of $r$ as follows:
\begin{equation*}
\begin{split}
\cte(\es,e) & = 0 \\
\cte(tr,e) & =
\begin{cases}
\cte(r,e)+1 & \text{if } [t] = e \\
\cte(r,e)-1 & \text{if } [t] = \rev e \\
\cte(r,e) & \text{otherwise}
\end{cases}
\end{split}
\end{equation*}
\end{definition}
%\todo{In Definition~\ref{def:sqeqt} event meant forward event - now changed.
%So $\cte(r,e)$ was only defined for forward events;
%now it is defined for reverse events also.
%Note that Lemma~\ref{lemma:cccount} holds for reverse events also.}

We now show that $\cte(r,e)$ is invariant under causal equivalence.

%\begin{restatable}{lemma}{cccount}\label{lemma:cccount}
\begin{lemma}\label{lemma:cccount}
Let $\mc L$ be a pre-reversible LTSI.
  Let $r \ceqt s$. Then for each event $e$ we have that $\cte(r,e) = \cte(s,e)$.
\end{lemma}
%\end{restatable}
\begin{proof}
  We prove the thesis for $r$ and $s$ being derived by a single
  application of the axioms; the thesis will follow since equality is
  an equivalence relation.

  If $r=r_1tu'r_2$ and $s=r_1ut'r_2$ then we have by definition that
  $t \ind u$.
Hence,
  $[t]=[t']$ and $[u]=[u']$ using Definition~\ref{def:sqeqt simp}. The thesis follows.

  If $r=r_1t\rev tr_2$ and $s=r_1r_2$ (the other case is analogous)
  then the contribution of $t$ and $\rev t$ to $\cte(r,[t])$ (as well
  as to $\cte(r,e)$ for $t \notin e$) is $0$; hence the thesis
  follows.
\end{proof}
Lemma~\ref{lemma:cccount} generalises what was shown for the forward-only setting
in~\cite[Corollary~4.3]{SNW96}.
%
%\begin{restatable}{proposition}{regeqzero}\label{prop:re>0}
\begin{proposition}\label{prop:regeqzero}
If an LTSI is pre-reversible,
then for any rooted path $r$ and any forward event $e$ we have $\cte(r,e) \geq 0$.
\end{proposition}
%\end{restatable}
%\Comment{
\begin{proof}
Let $r$ be a rooted path.
Using PL (Proposition~\ref{prop:PL}), we obtain a coinitial and cofinal forward-only path $s$ such that $s \ceqt r$.  Let $e$ be any forward event.  Clearly $\cte(s,e) \geq 0$.
Hence $\cte(r,e) \geq 0$ by Lemma~\ref{lemma:cccount}.
\end{proof}
%}
%
We can lift independence from transitions to events.

\begin{definition}[Coinitially independent events]\label{def:coind events}
Let events $e,e'$ be \emph{coinitially independent},
written $e \coind e'$, iff there are coinitial transitions $t,t'$ such that
$[t] = e$, $[t'] = e'$ and $t \ind t'$.
\end{definition}
\begin{lemma}\label{lem:coind rev}
Assume an LTSI is pre-reversible. If $e \coind e'$ then we have also
$\rev e \coind e'$.
\end{lemma}
\begin{proof}
Suppose that $e \coind e'$.
Then there are coinitial $t, u$ such that $[t] = e$, $[u] = e'$ and $t \ind u$.
Use SP to complete a diamond with transitions $t' \sqeqt t$, $u' \sqeqt u$.
By PCI we have $\rev t \ind u'$.
Hence $\rev e \coind e'$ as required.
\end{proof}
Thus in pre-reversible LTSIs, $\coind$ is fully determined just considering forward events.
By Lemma~\ref{lem:coind rev},
if we know $e \coind e'$ then we know $\und e \coind \und{e'}$.
% \todo{IVAN: now und defined on events as well (Def. 5.1)}
\begin{proposition}\label{prop:coind irref}
Assume an LTSI is pre-reversible.  Then $\coind$ is irreflexive.
\end{proposition}
\begin{proof}
Suppose for a contradiction that $e \coind e$ for some event $e$.
By Lemma~\ref{lem:coind rev},
we can assume that $e$ is forward.
Then there are coinitial transitions $t,u \in e$ such that $t \ind u$.
We can use SP to complete a square with $t' \sqeqt t$ and $u' \sqeqt u$.
This square is non-degenerate by Lemma~\ref{lem:non-degenerate}.
% All transitions in the square belong to the same event.
% Hence there are two distinct reverse coinitial transitions from the same event,
% contradicting RED.
But now $\rev{t'}$ and $\rev{u'}$ are two distinct coinitial backward transitions with the same label, contradicting BLD (Proposition~\ref{prop:BD}).
\end{proof}
We can slightly strengthen the previous result as follows:
\begin{proposition}\label{prop:coind und}
Assume an LTSI is pre-reversible.
If $t:P \tran\alpha Q$ and $u:R \tran\beta S$ with $[t] \coind [u]$
then $\und\alpha \neq \und\beta$.
\end{proposition}
\begin{proof}
Similar to the proof of Proposition~\ref{prop:coind irref}.
\end{proof}

In pre-reversible LTSIs each event can occur at most once in a rooted path.
\begin{definition}\label{def:NRE}
{\bf No repeated events (NRE)}: for
any rooted path $r$ and any forward event $e$ we have $\cte(r,e) \leq 1$.
\end{definition}
In order to prove NRE we need the following lemmas.
\begin{lemma}[Ladder Lemma]\label{lem:ladder}
Assume an LTSI is pre-reversible.
Suppose that $t:P \tran \alpha Q$ and $t':P' \tran \alpha Q'$ with $t \sqeqt t'$.
Then there is a path $s$ from $Q$ to $Q'$ such that for all $u$ in $s$
% we have $t \sqeqt t'' \ind u' \sqeqt u$ (for some $t'',u'$),
% which we may write $t \sqeqt \ind \sqeqt u$.
we have $[t] \coind [u]$.
\end{lemma}
\begin{proof}
By the definition of $\sqeqt$ there is a ladder of diamonds connecting $t$ to $t'$.
This gives a path $s$ from $Q$ to $Q'$.
Take any $u$ in $s$, and consider the diamond containing $u$.
Let $u'$ be on the opposite side from $u$, so that $u' \sqeqt u$,
and let $t''$ be the rung nearest to $t$, so that $t \sqeqt t''$.
We have $t'' \ind u'$.  Hence result.
\end{proof}
\begin{lemma}\label{lem:cte zero}
Let $\mc L$ be a pre-reversible LTSI.
Suppose $t: P \tran\alpha Q$ and $t': P' \tran\alpha Q'$
with $t \sqeqt t'$, and suppose $r$ is a path from $Q$ to $Q'$.
Then $\cte(r,[t]) = 0$.
\end{lemma}
\begin{proof}
By Lemma~\ref{lem:ladder} there is a path $s$ from $Q$ to $Q'$
such that for all $u$ in $s$ we have $[t] \coind [u]$.
Let $\lab(t)=\alpha$ and $\lab(u)=\beta$.
By Proposition~\ref{prop:coind und} we have  $\und\alpha \neq \und\beta$.
%\todo{$\lab$ defined on events rather than transitions - might be good to define $\lab$ on transitions}
%Since $\coind$ is irreflexive (Proposition~\ref{prop:coind irref}),
%for all $u$ in $s$ we have $[t] \neq [u]$.
%Before we could have had $[\rev t] in r$
Hence $\cte(s,[t]) = 0$,
and by Lemma~\ref{lemma:cccount} $\cte(r,[t]) = 0$ as required.
\end{proof}

\begin{proposition}\label{prop:NRE}
If an LTSI is pre-reversible then it satisfies NRE.
\end{proposition}
\begin{proof}
Let $e$ be a forward event and $r$ be a rooted path from $I$ to $R$, and suppose for a contradiction that
$\cte(r,e) > 1$.
Using PL we can obtain a forward-only path $r'$ from $I$ to $R$ with $r \ceqt r'$.
By Lemma~\ref{lemma:cccount}, $\cte(r',e) > 1$.
Suppose $r'$ contains
$t:P \tran a Q$ followed later by $t':P' \tran a Q'$ where $t, t' \in e$.
Let $r''$ be the portion of $r'$ from $Q$ to $P'$.
% By Lemma~\ref{lem:ladder} there is a path $s$ from $Q$ to $Q'$ such that for all $u$ in $s$
% % we have $t \sqeqt t'' \ind u' \sqeqt u$ (some $t'',u'$).
% we have $[t] \coind [u]$.
% By CC, $s \ceqt r''t'$.
% By Lemma~\ref{lemma:cccount}, $\cte(s,[t']) > 0$, since $r''$ is forward-only.
% Hence there is $u$ in $s$ such that $u \sqeqt t' \sqeqt t$.
% % But then $t \sqeqt t'' \ind u' \sqeqt t$,
% % where $t'',u'$ are as above.
% But then $[t] \coind [u] = [t]$,
% contradicting
% % our assumption that $\coind$ is irreflexive.
% Proposition~\ref{prop:coind irref}.
By Lemma~\ref{lem:cte zero} applied to $t,t'$ and path $r''t'$
we have $\cte(r''t',[t]) = 0$.
This is a contradiction since $r''$ is forward-only.
\end{proof}
NRE was shown in the forward-only setting of occurrence transition systems with independence in~\cite[Corollary~4.6]{SNW96}.
It was also shown in the reversible setting without independence
in~\cite[Proposition~2.10]{PU07a}.
% \il{**** We cannot use Figure~\ref{fig:repeated1} any more, look for another example or to prove the thesis ****}
% \il{In the proposition above, one would expect} $0\leq \cte(r,e) \leq 1$, but this is not guaranteed  in pre-reversible
% LTSIs as Figure~\ref{fig:repeated1} at page \pageref{fig:repeated1} shows paths with repeated events $a$ and $b$. We shall later on 
% add one more axiom (CIRE, Definition~\ref{def:CIRE}) that would ensure that rooted paths have no repeated events
% (see Definition~\ref{def:NRE} and Proposition~\ref{prop:CIRE NRE}).

\begin{example}\label{ex:repeated}
Consider the LTSI in Figure~\ref{fig:repeated}.
% Figure environment removed
Independence holds only between coinitial transitions and is given by closing under BTI and propagating
independence around the corners of diamonds as in PCI whenever possible.
Note however that PCI does not hold, since we have coinitial independent $a$ and $\rev a$-transitions,
contradicting Lemma~\ref{lemma:revnotind}.
As well as BTI, axioms SP and WF hold, so that CC holds.
All $a$-transitions belong to the same event,
and all $b$-transitions belong to the same event.
We have rooted paths where the same event is repeated,
contradicting NRE.
Note also that BLD fails and that $\coind$ is reflexive.\finex
\end{example}

\subsection{Polychotomy}
We now show what we call \emph{polychotomy},
which states that if forward events do not cause each other and are not in conflict,
then they must be independent.
This will help us to relate the different notions of causal safety and liveness (Section~\ref{sec:CSCL}).
We first define causality and conflict relations on forward events.
\begin{definition}[Causality relation on forward events]\label{def:ordering}
  Let $\mc L$
  %= (\Proc,\Lab,\tran{},\ind)$
  be an LTSI.
Let $e,e'$ be forward events of $\mc L$.
Let $e \leq e'$ iff
for all rooted paths $r$, if $\cte(r,e') > 0$ then $\cte(r,e) > 0$.
As usual $e < e'$ means $e \leq e'$ and $e \neq e'$.
If $e < e'$ we say that
% $e'$ is an \emph{effect} of $e$ and
$e$ is a \emph{cause} of $e'$.
%The notions of $<$, cause and effect
%apply to transitions as well:
%Let $\pi=sts't's''$ be a path with transitions $t,t'$. We say that $t<t'$ if $[t]<[t']$. Also
%$t'$ is an effect of $t$ (or
%$t$ causes $t'$) if {\bf every} rooted path that contains $t'_1\in[t']$ also contains
%$t_1\in [t]$.
\end{definition}
As expected, the causality relation is a partial ordering (i.e., a reflexive, transitive and antisymmetric relation).
\begin{lemma}\label{lem:po}
If an LTSI is pre-reversible then
$\leq$ is a partial ordering on events.
\end{lemma}
\begin{proof}
Reflexivity and transitivity are immediate.
For antisymmetry, suppose that $e_1 \leq e_2$ and $e_2 \leq e_1$,
where $e_1,e_2$ are forward events.
Then for all rooted $r$, $\cte(r,e_1) >0$ iff $\cte(r,e_2) >0$.
Since the LTSI is pre-reversible, by Proposition~\ref{prop:regeqzero},
for all rooted $r$, $\cte(r,e_1) \geq 0$ and $\cte(r,e_2) \geq 0$.
Let $r$ be a shortest rooted path such that $\cte(r,e_1) >0$.
We can use WF to show that $r$ must exist.
Then $\cte(r,e_2) > 0$.
Also $r = r't$, where $\cte(r',e_1) = 0$ (otherwise $r$ would not be a shortest path) and so $\cte(r',e_2) = 0$.
We see that both $[t] = e_1$ and $[t] = e_2$,
showing that $e_1 = e_2$ as required.
\end{proof}
In~\cite{vGV97,PU07a}, orderings on forward events have been defined using forward-only rooted paths;
in fact, the definitions coincide for pre-reversible LTSIs.
% satisfying SP, BTI, WF and PCI.
\begin{definition}[\cite{vGV97,PU07a}]\label{def:ordering fwd}
  Let $\mc L$ % = (\Proc,\Lab,\tran{},\ind)$
  be an LTSI.
Let $e,e'$ be forward events of $\mc L$.
Let $e \leqf e'$ iff
for all rooted forward-only paths $r$,
if  $\cte(r,e') >0$
%$r$ contains a representative of $e'$
then $\cte(r,e) >0$.
%$r$ also contains a representative of $e$.
\end{definition}
\begin{lemma}\label{lem:ordering}
For any LTSI, and any forward events $e,e'$,
$e \leq e'$ implies $e \leqf e'$.
If an LTSI is pre-reversible then
$e \leqf e'$ implies $e \leq e'$.
\end{lemma}
\begin{proof}
Straightforward using PL and Lemma~\ref{lemma:cccount}.
\end{proof}
% We start by defining a conflict relation on events.
\begin{definition}\label{def:conflict}
Two forward events $e,e'$ are in \emph{conflict}, written $e \cf e'$,
if there is no rooted path $r$ such that $\cte(r,e) >0$ and $\cte(r,e') > 0$.
\end{definition}
Much as for orderings, conflict on events has been defined previously
using forward-only rooted paths~\cite{vGV97,PU07a};
in fact, the definitions coincide for pre-reversible LTSIs.
We omit the details.

We can now introduce the main result of this section.
\begin{definition}[Polychotomy]\label{def:poly}
% Suppose that WF, SP, PL, CC, PCI, NRE hold.
Let $\mc L$ be a pre-reversible LTSI.
We say that $\mc L$ satisfies \emph{polychotomy} if whenever
$e,e'$ are \emph{forward} events, then exactly one of the following holds: %\\
\begin{enumerate}
\item $e = e'$;\quad%\quad
\item $e < e'$;\quad%\quad %(in every rooted forward-only path containing $t' \in e'$ there is an earlier $t \in e$)
\item $e' < e$;\quad%\quad
\item $e \cf e'$; or\quad %\quad %(no rooted forward-only path contains both $t \in e$ and $t' \in e'$)
\item $e \coind e'$. %(there are \emph{coinitial} $t \in e$ and $t' \in e'$ such that $t \ind t'$)
\end{enumerate}
\end{definition}
%
%\begin{restatable}[Polychotomy]{lemma}{poly}\label{prop:poly}
\begin{proposition}[Polychotomy]\label{prop:poly}
Assume an LTSI is pre-reversible.
Then polychotomy holds.
% Let $e,e'$ be \emph{forward} events.  Then exactly one of the following holds:
% \begin{enumerate}
% \item
% $e = e'$
% \item
% $e < e'$ %(in every rooted forward-only path containing $t' \in e'$ there is an earlier $t \in e$)
% \item
% $e' < e$
% \item
% $e$ and $e'$ are in conflict %(no rooted forward-only path contains both $t \in e$ and $t' \in e'$)
% \item
% $e \coind e'$ %(there are \emph{coinitial} $t \in e$ and $t' \in e'$ such that $t \ind t'$)
% \end{enumerate}
% \todo{If true, this conjecture implies Conjecture~\ref{conj:poly coinitial}.}
\end{proposition}
%\end{restatable}
\begin{proof}
%Let event $e,e'$ have labels $a,b$ respectively ($a$ and $b$ may or may not be equal).
Consider two forward events $e$ and $e'$ which may or may not be equal.

We first check mutual exclusivity.
Suppose $e = e'$.
Then $e < e$ is impossible by definition of $<$.
%WF \todo{$e < e$ implies $e \neq e$ by definition - no need for WF}.
Also $e$ cannot be in conflict with itself (we can use WF to show that there is at least one rooted path).
Finally, $e \coind e$ is impossible by Proposition~\ref{prop:coind irref}.
% implies that there are two distinct coinitial transitions $t_1,t_2 \in e$
% which are independent.
% By SP we can complete the square to get a forward-only path with two occurrences of $e$,
% which is impossible by NRE \todo{uses WF?}\todo{Ivan: don't think so}.
From now on we assume $e \neq e'$.

Next suppose $e < e'$.
We can rule out $e' < e$ using %WF.
Lemma~\ref{lem:po}.

Using Lemma~\ref{lem:ordering}, we know that $e \ltf e'$, hence there must be some rooted forward-only path with $e$ followed by $e'$ (WF ensures at least one rooted path exists),
and so $e$ and $e'$ are not in conflict.
Finally $e \coind e'$ implies that there are two coinitial transitions $t \in e$, $t' \in e'$
which are independent.
Using SP to complete the square we see that $e < e'$ is impossible by NRE,
which holds by Proposition~\ref{prop:NRE}.

Similarly we see that $e' < e$ implies that $e$ and $e'$ are not in conflict and not independent.

Next suppose that $e \cf e'$.
If $e \coind e'$
then there are two coinitial transitions $t \in e$, $t' \in e'$
which are independent.
Using SP to complete the square and WF we see that we have a rooted forward-only path
containing occurrences of both $e$ and $e'$ contradicting them being in conflict.

Suppose that none of (1)-(4) hold.
We must show (5).
Since $e,e'$ do not conflict, there is a
%forward-only path
rooted path $r$ starting at some irreversible $I$ such that $\cte(r,e) > 0$ and $\cte(r,e') > 0$. If more than one such path exists, choose one of minimal length.
 W.l.o.g.~suppose that $r$ finishes with $t' \in e'$ at $P$.
Since not $e < e'$, using Lemma~\ref{lem:ordering} also $e \ltf e'$ does not hold; hence there is another forward-only path~$r'$ from some irreversible $I'$
finishing with $t'' \in e'$ at $Q$
such that $\cte(r',e) = 0$.
% and not containing $a$.
% There is a ladder of diamonds connecting the two $b$s creating a path $s$ from $Q$ to $P$.
By Lemma~\ref{lem:ladder} there is a path $s$ from $Q$ to $P$
such that $e' \coind [u]$ for every $u$ in $s$.
Using Proposition~\ref{prop:unique irrev} we deduce that $I' = I$.
By CC $r \ceqt r's$ and so by Lemma~\ref{lemma:cccount} $\cte(s,e) > 0$ and $s$ must contain $u \in e$,
% We then deduce that there is a diamond of $a$ and $b$ in the ladder,
yielding $e \coind e'$ as required.
\end{proof}


%

\section{Causal Safety and Causal Liveness}\label{sec:CSCL}
In the literature, causal consistent reversibility is frequently
informally described by saying that ``a transition can be undone if
and only if each of its consequences, if any, has been undone''
% we explained this better in the introduction
%
(see, e.g., \cite{LaneseNPV18}).
%, which is also the only paper we are aware of where such a property is formalised). 
In this section we study this
property, where the two implications will be referred to as \emph{causal
  safety} and \emph{causal liveness}. We provide three different formalisations %versions
of
such properties, based on
independence of transitions (Section~\ref{sub:indtra}),
independence of events (Section~\ref{sub:indev}), and
ordering of events (Section~\ref{sub:ord}),
and study their relationships.
% In order to define such properties we need the concept of event.
In Figure~\ref{fig:diagprerev1} we show the relationships
between the various axioms and properties we shall study
in this section and Section~\ref{sec:coinitial}.
% Figure environment removed
\subsection{CS and CL via independence of transitions}\label{sub:indtra}
%% \todo{Ivan's points from emails:

%% - in the definition of CS/CL and variants we require independence between the
%%   reverse of the first transition and all the transitions $t$ in the path $r$ such that $\cte(r,[t]) \neq 0$

%% - the reason of why considering the reverse of the first transition was explained by myself in a previous mail

%% - the reason to consider $\neq 0$ instead of > 0 is to take into account also
%%   backward transitions in $r$

%% our FOSSACS development was mainly focused on coinitial independence, and in particular SP triggers when independence is available on coinitial transitions.
%%     Causal safety and liveness, as well as CC, are mainly focused on consecutive transitions, and would require sideways diamond more than our SP. Now, in our framework (without using RPI) the only way to trigger diamonds on consecutive transitions is by considering the reverse of the first transition, so to get a pair of coinitial transitions.
%%     For this reason I think it makes a lot of sense to have as hypothesis that the reverse of the first transition is independent on the others.}

We first define causal safety and liveness using the independence relation.
% The idea is that perhaps we do not need to use a causality relation here.
\begin{definition}\label{def:safe live}
Let $\mc L$ be an LTSI. % pre-reversible LTSI.
% (do we need to assume any of the axioms hold here?).
\begin{enumerate}
\item
We say that $\mc L$ is \emph{causally safe (CS$\indt$)} if whenever
$t_0:P \tran a Q$, $r:Q \ptran \rho R$, $\cte(r,[t_0]) = 0$ and $t_0\op:S \tran a R$
with $t_0 \sqeqt t_0\op$,
then $\rev{t_0} \ind t$ for all $t$ in $r$ such that $\cte(r,[t]) > 0$.
%
% we are wondering if we could write $\cte(r,[t]) \neq 0$ instead of $\cte(r,[t]) > 0$ here.
%
\item
We say that $\mc L$ is \emph{causally live (CL$\indt$)} if whenever
$t_0:P \tran a Q$, $r:Q \ptran \rho R$ and $\cte(r,[t_0]) = 0$ and
$\rev{t_0} \ind t$, for all $t$ in $r$ such that $\cte(r,[t]) > 0$,
then we have
$t_0\op:S \tran a R$ with $t_0 \sqeqt t_0\op$.
\end{enumerate}
\end{definition}
% \todo{Previously $\cte(r,[t])$ was only defined for forward $t$;
% now we would have to specify that $t$ is forward.
% If we allow reverse $t$ it is a bit strange to require $\cte(r,[t])>0$
% rather than $\cte(r,[t]) \neq 0$.
% See Remark~\ref{rem:rev count}.}
Properties CS$\indt$ and CL$\indt$ both consider a (forward) transition $t_0:P \tran a Q$  
followed by a path $r$ where the number of occurrences in $r$
of transitions that belong to the same event as $t_0$ is zero.
CS$\indt$ states that if after path $r$ a transition $t_0\op$ can be undone, where
$t_0$ and $t_0\op$ belong to the same event, then the reverse of $t_0$ 
is independent of all transitions $t$ where the number of occurrences in $r$ of 
the event of $t$ is positive.
%
Dually, CL$\indt$ requires that if the reverse of $t_0$ is independent of all transitions 
whose events have a positive number of occurrences in $r$, then it can be undone.

% \todo{ The paragraph above is a rewritten version of the next paragraph. It may be easier to follow.
% 
% In words, both CS$\indt$ and CL$\indt$ consider a (forward) transition $t_0:P \tran a Q$ performed
% before some path $r$ whose count of transitions in $[t_0]$ is
% $0$. CS$\indt$ states that if after path $r$ transition $t_0$ can be undone,
% that is the reverse of a transition in the same event is enabled, then
% the reverse of $t_0$ is independent of all transitions belonging to events which
% have in $r$ a positive number of occurrences. Dually, CL$\indt$ requires that
% if the reverse of $t_0$ is independent of all transitions belonging to events
% which have in $r$ a positive number of occurrences then it can be
% undone.}

\begin{remark}\label{rem:equal0notneeded}
In the definition of CS$\indt$ the condition that $\cte(r,[t_0]) = 0$
can be deduced from the other conditions using
% Propositions~\ref{prop:regeqzero} and~\ref{prop:NRE},
Lemma~\ref{lem:cte zero},
provided that the LTSI is pre-reversible.
\end{remark}


We use the reverse of $t_0$ when considering independence from $t$
because our axioms BTI, SP and PCI focus on \emph{coinitial} independence
rather than independence of consecutive transitions in a trace.
%one transition following another in a trace.
Take the simplest case where $r$ is a single transition $t:Q \tran b R$.
First assume $\rev{t_0} \ind t$;
note that this is coinitial independence.
We can use SP and PCI to get $t_0\op:S \tran a R$ with $t_0 \sqeqt t_0\op$,
which is an example of causal liveness.
Conversely, if we assume $t_0\op:S \tran a R$ with $t_0 \sqeqt t_0\op$,
we can use BTI, SP, BLD and PCI to get a diamond with $\rev{t_0} \ind t$,
which is an example of causal safety.

Note that in the discussion above to prove causal safety we need to consider also the case $r=t \rev t t$. Since $[\rev t]$ has a negative number of occurrences, we only need to show that $\rev{t_0} \ind t$, which can be proved as above. However, if we replaced the condition $\cte(r,[t]) > 0$ with $\cte(r,[t]) \neq 0$, we would also need to show  $\rev{t_0} \ind \rev t$, which does not follow from the axioms above. Intuitively, requiring $\rev{t_0} \ind \rev t$ would make little sense, since all the occurrences of $\rev t$ could be simplified with corresponding occurrences of $t$. This is why we decided to require $\cte(r,[t]) > 0$.

%% \todo{The above discussion might seem to suggest that causal safety and liveness
%% hold in a single diamond with coinitial independence.
%% However this is not the case for CS$\indt$, since with $t:Q \tran b R$
%% we can have $r = t \rev t t$, so that we can deduce $\rev {t_0} \ind \rev t$,
%% which is not coinitial independence.
%% See Proposition~\ref{prop:IC CSi}.
%% Therefore I wonder if we should return to $t_0 \ind t$ rather than
%% $\rev{t_0} \ind t$ in the definitions of CS$\indt$ and CL$\indt$.
%% The forward version is the more obvious definition, and the motivation for the
%% reverse version is weaker than previously thought.
%% Furthermore, I think we can show that the forward version of CS$\indt$
%% implies CIRE without using RPI.
%% See revised proof of Proposition~\ref{prop:CSindt RPI CIRE}.
%% Theorem~\ref{thm:CL} does use $\rev{t_0}$ in an essential way.
%% So we would have to add RPI as an assumption.
%% Still, we could avoid the work showing that CS$\indt$ implies CL$\ci$,
%% since that is now immediate via CIRE.
%% }

We have seen in the last two paragraphs that existing axioms are sufficient to show
CS$\indt$ and CL$\indt$ in the case where trace $r$ consists of
a single transition. However, existing axioms are not enough
for general $r$, as we will show in Examples~\ref{ex:prerev not CSi} and~\ref{ex:prerev not CL}.
%To show CS$\indt$ and CL$\indt$
Thus, we introduce
% We may wish to require
the following
axiom, which states that independence does not depend on the choice
of the representative inside an event.
\begin{definition}\label{def:IRE}
  {\bf Independence respects events (IRE)}:
Whenever $t \sqeqt t' \ind u$ we have $t \ind u$.
\end{definition}
IRE is one of the conditions in the definition of transition systems with
independence~\cite[Definition~3.7]{SNW96}.

IRE allows us to relate coinitial independence on events and independence on transitions.
\begin{lemma}\label{lem:coind IRE}
Assume an LTSI satisfies IRE.
If $[t] \coind [u]$ then $t \ind u$.
\end{lemma}
\begin{proof}
Immediate.
\end{proof}

Together with the axioms for pre-reversibility,
IRE is enough to show both CS$\indt$ and CL$\indt$.

% \begin{lemma}\label{lem:ind path}
% \todo{remove as no longer needed - use Lemma~\ref{lem:ladder} instead.}
% Suppose a pre-reversible LTSI satisfies IRE.
% Suppose also that $(P,a,Q) \sqeqt (R,a,S)$.
% Then there is a path $r$ from $Q$ to $S$ such that $(P,a,Q) \ind t$
% and $(P,a,Q) \ind \rev t$
% for all $t$ in $r$.
% \todo{Needs to change from $(P,a,Q)$ to the reverse transition.  Better to name $(P,a,Q)$?}
% \end{lemma}
%% \begin{proof}
%% Suppose that $(P,a,Q) \sqeqt (R,a,S)$.
%% Then there is a ladder of diamonds connecting $(P,a,Q)$ to $(R,a,S)$,
%% giving a path $r$ from $Q$ to $S$.
%% Let us show that $(P,a,Q) \ind t$ for all $t$ in $r$.
%% Let us consider any diamond in the ladder, with transitions
%% $P' \tran a Q'$, $t:Q' \tran \beta S'$, $t':P' \tran \beta R'$, $R' \tran a S'$, where  
%% $t$ is a transition in $r$ and $P' \tran a Q'$ is in the same event as $P \tran a Q$.
%% By definition of event $P' \tran a Q' \ind t' \sqeqt t$.
%% By IRE, $P \tran a Q \ind t$ as desired.
%% Furthermore, $(R',a,S') \ind \rev{t'}$, so that by IRE $(P,a,Q) \ind \rev t$. 
%% % also get $\rev {(P,a,Q)} \ind t$, $\rev {(P,a,Q)} \ind \rev t$.  
%% \end{proof}



% \begin{definition}\label{def:NRE fwd}
% An LTSI satisfies No Repeated Events (NRE) if in any forward-only path there cannot occur two different transitions $t$ and $t'$ such that $t \sqeqt t'$.
% \todo{rival duplicate definition Definition~\ref{def:NRE} - needs fixing}
% \end{definition}
% \begin{proposition}\label{prop:nre}
% Suppose that an LTSI satisfies CC and IRE.
% Then it also satisfies NRE.
% \todo{strengthened in Proposition~\ref{prop:CIRE NRE}}
% \end{proposition}
% \begin{proof}
% Let $r$ be a forward-only path, and suppose for a contradiction that it contains
% $t:P \tran a Q$ followed later by $t':P' \tran a Q'$ where $t \sqeqt t'$.
% Let $r'$ be the portion of $r$ from $Q$ to $P'$.
% % We can suppose that $\cte(r',[t]) = 0$.
% By Lemma~\ref{lem:ind path} there is a path $s$ from $Q$ to $Q'$ such that
% for all $u$ in $s$
% we have $t \ind u$.
% By CC, $s \ceqt r't'$.
% By Lemma~\ref{lemma:cccount}, $\cte(s,[t']) > 0$, since $r'$ is forward-only.
% Hence there is $u$ in $s$ such that $u \sqeqt t' \sqeqt t$.
% But then $t \ind u \sqeqt t$, and so $t \ind t$ by IRE, contradicting $\ind$ being
% irreflexive.
% \end{proof}

%\begin{restatable}{theorem}{CS}\label{thm:CS}
\begin{theorem}\label{thm:CS}
Let a pre-reversible LTSI satisfy IRE.
Then it satisfies CS$\indt$.
\end{theorem}
%\end{restatable}
%% \begin{proof}[Old Proof to be removed]
%% Suppose $P \tran a Q$, $r:Q \ptran \rho R$ and $S \tran a R$
%% with $(P,a,Q) \sqeqt (S,a,R)$.
%% By Lemma~\ref{lem:ind path} there is a path $s$ from $Q$ to $R$
%% such that $(P,a,Q) \ind u$ 
%% %\todo{and $(P,a,Q) \ind \rev u$}
%% and $(P,a,Q) \ind \rev u$ for all $u$ in $s$.
%% By CC, $r \ceqt s$.
%% %
%% Suppose $t$ in $r$ is such that $\cte(r,[t]) \neq 0$.
%% \todo{I think the original proof may have considered $\cte(r,[t]) < 0$,
%% though this is now excluded from Definition~\ref{def:safe live}.
%% If $\cte(r,[t]) < 0$, then since $t$ must be forward by the definition of $\cte(r,[t])$,
%% we would be considering $\rev t$ in $r$ rather than $t$ in $r$.}
%% If $\cte(r,[t]) > 0$, \todo{what if $\cte(r,[t]) < 0$?}
%% then $\cte(s,[t]) > 0$, thanks to Lemma~\ref{lemma:cccount}.
%% But then there is $u$ in $s$ such that $u \sqeqt t$.
%% We have $(P,a,Q) \ind u$ and so $(P,a,Q) \ind t$, using IRE.
%% \end{proof}
%% \todo{We can show the stronger form of CS$\indt$ with $\cte(r,[t]) \neq 0$
%% rather than $\cte(r,[t]) > 0$:
\begin{proof}
% \todo{NB Assumption $\cte(r,[t_0]) = 0$ not used as expected from Remark~\ref{rem:equal0notneeded}.}
Suppose $t_0:P \tran a Q$, $r:Q \ptran \rho R$ and $t_0\op:S \tran a R$
with $t_0 \sqeqt t_0\op$.
By Lemma~\ref{lem:ladder} there is a path $s$ from $Q$ to $R$
such that for all $u$ in $s$ we have $[t_0] \coind [u]$.
We deduce by Lemmas~\ref{lem:coind rev} and~\ref{lem:coind IRE}
that for all $u$ in $s$ we have 
$\rev{t_0} \ind u$. %and $\rev{t_0} \ind \rev u$
%\todo{last item and Lemma~\ref{lem:coind IRE} not needed? Last item not needed but Lemma~\ref{lem:coind IRE} needed. }
% By [modified] Lemma~\ref{lem:ind path} there is a path $s$ from $Q$ to $R$
% such that $\rev{t_0} \ind u$ 
% and $\rev{t_0} \ind \rev u$ for all $u$ in $s$.
By CC, $r \ceqt s$.

Take $t$ in $r$ such that $\cte(r,[t]) > 0$.
% If $\cte(r,[t]) > 0$, 
Then $\cte(s,[t]) > 0$, thanks to Lemma~\ref{lemma:cccount}.
But then there is $u$ in $s$ such that $u \sqeqt t$.
We have $\rev{t_0} \ind u$ and so $\rev{t_0} \ind t$, using IRE, as desired.
% If $\cte(r,[t]) < 0$, 
% then $\cte(s,[t]) < 0$, again by Lemma~\ref{lemma:cccount}.
% But then there is $u$ in $s$ such that $\rev u \sqeqt t$. %\todo{do we need Lemma~\ref{lem:coind rev}? Seems not}.
% We have $\rev{t_0} \ind \rev u$ and so $\rev{t_0} \ind t$ as desired, again using IRE. 
\end{proof}

%In order to prove CL we need an auxiliary result.
%
%\begin{lemma}\label{lemma:norm}
%  Let $r$ be a path. There is a path $r' \ceqtind r$ such that for each
%  event $e$, it is not the case that both $e$ and $\rev e$ occur.
%\end{lemma}
%\begin{proof}
%\end{proof}
% As a consequence, if an event $e$ occurs, then $\cte(r',e)>0$.
%As a consequence, if a transition $t$ occurs, then $\cte(r',[t])>0$.


%\begin{restatable}{theorem}{CL}\label{thm:CL}
% \todo{Theorem~\ref{thm:CL} is false by Example~\ref{ex:IRE2}.
% Need to add e.g. RPI (as we had originally).}\todo{Previous comment is outdated, right?}
\begin{theorem}\label{thm:CL}
  Let a pre-reversible LTSI satisfy IRE.
  Then it satisfies CL$\indt$.  
\end{theorem}
%\end{restatable}
% \begin{proof}
  % Suppose $t_0:P \tran a Q$, $r:Q \ptran \rho R$ and $\cte(r,[t_0]) =
  % 0$ and $t_0 \ind t$, for all $t$ such that $\cte(r,[t]) > 0$.  We
  % have to show that there is $S \tran a R$ with $(P,a,Q) \sqeqt
  % (S,a,R)$.
% 
  % Thanks to the parabolic lemma, there is $S$ such that $b: P
  % \ptran{\rho_b} S$ and $f: S \ptran{\rho_f} R$, with $b$ backward and
  % $f$ forward. Since $\cte(r,[t_0]) = 0$, thanks to
  % Lemma~\ref{lemma:cccount} $\cte(b;f,[t_0])=1$. As a consequence,
  % there is at least a transition $t'_0:P' \tran a Q' \in [t_0]$ in
  % $f$.  If we can show that $t'_0 \ind t''$ for each transition $t''$
  % following $t'_0$ in $f$, then the thesis will follow by commuting
  % $t'_0$ with all following transitions using SP.
% 
  % We have two cases. If $\cte(b,[t'']) = 0$ then $\cte(b;f,[t]) > 0$,
  % hence by hypothesis $t_0 \ind t''$ and by IRE $t'_0 \ind t''$ as desired.
% 
  % If $\cte(b,[t'']) < 0$, by looking at the proof of PL we see that
  % $t_0$ and a backward transition in $[\rev t'']$ have been commuted,
  % hence they are independent. By IRE and RPI, $t'_0 \ind t''$ as desired.
  % 
  % If $\cte(b,[t'']) < 0$, by looking at the proof of PL we see that
  % $t_0$ and a backward transition in $[\rev t'']$ have been commuted,
  % hence they are independent. By IRE and RPI, $t'_0 \ind t''$ as desired.
% 
  % This concludes the proof.
% \end{proof}
\begin{proof}
  Suppose $t_0:P \tran a Q$, $r:Q \ptran \rho R$ and $\cte(r,[t_0]) =
  0$ and $t_0 \ind t$, for all $t$ in $r$ such that $\cte(r,[t]) > 0$.  We
  have to show that there is $t_0\op:S \tran a R$ with $t_0 \sqeqt
  t_0\op$.

  Thanks to PL, there is $T$ such that $b: P
  \ptran{\rho_b} T$ and $f: T \ptran{\rho_f} R$, with $b$ backward and
  $f$ forward.
  By CC, $t_0r \ceqt bf$.
  Since $\cte(r,[t_0]) = 0$, thanks to
  Lemma~\ref{lemma:cccount} we have $\cte(bf,[t_0])=1$. As a consequence,
  there is a transition $t'_0:P' \tran a Q' \in [t_0]$ in
  $f$.
This $t'_0$ is in fact the unique transition in $[t_0]$ belonging to $f$ by Proposition~\ref{prop:NRE}.
  Let $f'$ be the portion of $f$ from $Q'$ to $R$.
  If we can show that $\rev{t'_0} \ind t''$ for each transition $t''$ in $f'$,
  then the thesis will follow by commuting
  $t'_0$ with all such transitions using SP and IRE.

  By Lemma~\ref{lem:ladder} there is a path $s$ from $Q$ to $Q'$ such that $[t_0] \coind [u]$
  for all $u$ in $s$.
  By CC, $r \ceqt sf'$.
  Take any $t''$ in $f'$.
  By Lemma~\ref{lemma:cccount}, $\cte(r,[t'']) = \cte(s,[t'']) + \cte(f',[t''])$.
  % If $\cte(s,[t'']) > 0$ then there is $u$ in $s$ such that $u \sqeqt t''$,
  % and so $t_0 \ind t''$ using IRE.
  If $\cte(s,[t'']) < 0$ then there is $u$ in $s$ such that $u \sqeqt \rev{t''}$.
Now $[t_0] \coind [u] = [\rev{t''}]$.
Therefore $[\rev{t_0}] \coind [t'']$ by Lemma~\ref{lem:coind rev}, and $\rev{t_0} \ind t''$ by Lemma~\ref{lem:coind IRE}.
  Suppose instead $\cte(s,[t'']) \geq 0$.
  Since $\cte(f',[t'']) > 0$, we have $\cte(r,[t'']) > 0$. 
  So there is $u$ in $r$ such that $u \sqeqt t''$,
  and by hypothesis $\rev{t_0} \ind u$, so that $\rev{t'_0} \ind t''$ using IRE.
\end{proof}
% \todo{Suppose we modify CL$\indt$
% by replacing $t_0 \ind t$ by $\rev{t_0} \ind t$ where $t_0:P \tran a Q$.
% I think that the above proof shows modified CL$\indt$ using IRE.
% Similarly, using a modified version of Lemma~\ref{lem:ind path}
% we can use IRE to show a version of CS$\indt$ proving $\rev{t_0} \ind t$ rather than $t_0 \ind t$
% (or of course both).
% }
 
%\begin{proof}[Relies on Lemma~\ref{lemma:norm} still to be proven]
%  
%  Suppose $t_0:P \tran a Q$, $r:Q \ptran \rho R$ and $\cte(r,[t_0]) = 0$ and
%$t_0 \ind t$, for all $t$ such that $\cte(r,[t]) > 0$.
%We have to show that there is
%$S \tran a R$ with $(P,a,Q) \sqeqt (S,a,R)$.
%
%Thanks to Lemma~\ref{lemma:norm} we have that there exists $r'
%\ceqtind r$ with $\cte(r',[t_0]) = 0$ (from $\cte(r,[t_0]) = 0$
%using Lemma~\ref{lemma:ccicount}). Furthermore, $t_0 \ind t$, for
%all $t$ in $r'$. Indeed, for all $t$ in $r'$ we have $\cte(r',[t]) >
%0$, and from Lemma~\ref{lemma:ccicount} we also have $\cte(r,[t]) >
%0$, hence $t_0 \ind t$ thanks to IRE.
%
%The proof is by induction on $\len{r'}$. If $\len{r'}=0$ then the thesis
%trivially holds by selecting $(S,a,R) = (P,a,Q)$.
%
%If $\len r'>0$ let $r'=t'r''$. From the above, we have $t_0 \ind t'$.
%Hence thanks to RPI and MSP there is a path $t'_1 t_1 r''$ with $t_1
%\sqeqt t_0$.
%\todo{Definition~\ref{sqeqt} as currently formulated also requires $Q \neq R$ or $P \neq S$.  See proof of Lemma~\ref{lemma:ccicount}}.
%Note that $\len{r''}<\len{r'}$. Also,
%$\cte(r'',[t_0]) = 0$. Finally, thanks to IRE $t_1 \ind t$ for all $t$ in $r''$%.
%Hence, by inductive hypothesis there
%is $S \tran a R$ with $t_1 \sqeqt (S,a,R)$. The thesis follows by
%transitivity of $\sqeqt$.
%\end{proof}

% \begin{proof}[Rough sketch]
% Suppose $P \tran a Q$, $r:Q \ptran \rho R$ and $\cte(r,[P,a,Q]) = 0$ and
% $(P,a,Q) \ind t$, for all $t$ such that $\cte(r,[t]) > 0$.
% Let $e = [P,a,Q]$.
% By PL we have forward-only $r_1,r_2$ such that $r \ceqt \rev{r_1}r_2$.
% % and no new events are introduced.
% Let $T$ be the source of both $r_1$ and $r_2$.
% Suppose that no $e$-transition occurs in either $r_1$ or $r_2$.
% We can use BTI and SP to create a ladder with $e$-transitions as rungs
% finishing with $T' \tran a T$.
% Now we can create a further ladder going along $r_2$.
% Each transition $u$ that we encounter is either from an event we saw in $r_1$,
% in which case we use IRE and RPI, or else it is new, in which case we know
% that $(P,a,Q) \ind t$ since $\cte(r,[t]) > 0$.
% We finish with the desired transition 
% $S \tran a R$ with $(P,a,Q) \sqeqt (S,a,R)$.
% 
% Now suppose that an $e$-transition occurs in either $r_1$ or $r_2$.
% We know $\cte(r,e) = 0$, so that $\cte(r_1,e) = \cte(r_2,e) > 0$.
% Start at the last $e$-transition $t:(P',a,Q')$ in $r_2$.
% We want to switch $t$ with each of the remaining
% transitions $u$ of $r_2$, which we can do using SP if we know that
% they are independent.
% Since $(P,a,Q) \sqeqt (P',a',Q')$, by Lemma~\ref{lem:ind path} there is a path $s$ from $Q$ to $Q'$
% such that $e$-transitions are independent of every $t'$ in $s$ (use of IRE).
% If a remaining $u$ satisfies 
% $\cte(r,[u]) > 0$ then we know that $(P,a,Q) \ind u$,
% so that $t \ind u$ (using IRE).
% Suppose that $\cte(r,[u]) \leq 0$.
% Then there is at least one $[u]$-transition in $r_1$.
% Using CC we see that $\cte(s,[u]) < 0)$ and so $t \ind \rev u$,
% so that $t \ind u$ using RPI.
% \end{proof}

%% CS$\indt$ and CL$\indt$ are not derivable from CC;
%% we give an example LTSI which satisfies CC but not CS$\indt$ and not CL$\indt$.
%% \todo{remove this next example?}
%% \begin{example}[CC does not imply CS$\indt$ or CL$\indt$]\label{ex:CC not CL}
%% Consider the LTS in Figure~\ref{fig:repeated1}.
%% % Figure environment removed
%% Independence is mostly coinitial and given by closing under BTI and PCI.
%% Additionally we make the leftmost $a$-transition independent with all $b$-transitions.
%% Note that all $a$-transitions belong to the same event,
%% and all $b$-transitions belong to the same event.
%% Also SP and WF hold, so that the LTSI is pre-reversible \todo{this is false as PCI fails}, and
%% CC holds.
%% However IRE does not hold.
%% % Also CS seems to hold using Definition~\ref{def:coinitial safe live}.
%% Furthermore CS$\indt$ fails using Definition~\ref{def:safe live}.
%% Indeed, consider any path $\ptran {ba\rev b}$ from the leftmost state. %start.
%% CS$\indt$ would imply that the first $b$ is independent with the $a$ but this is not the case
%% (we do however have $\rev b \ind a$).
%% % Or we can consider path $abab\rev a$ going along the top.
%% % CS would imply that the first $a$ is independent with the second $a$,
%% % which is again not the case.

%% Also CL$\indt$ fails using Definition~\ref{def:safe live}.
%% Indeed, consider any path $\ptran {abb}$ from the leftmost state. %start.
%% Since the leftmost $a$-transition is independent with all $b$-transitions,
%% we should be able to reverse $a$ at the end of the path, but this is not possible.
%% % Also IRE and NRE fail.
%% % \todo{The example seems correct but makes me wonder whether we have the right definitions
%% % of CS and CL.}
%% \end{example}
%% \todo{Example~\ref{ex:CC not CL} shows that the stipulation of IRE cannot be omitted in the statements of Theorems~\ref{thm:CS} and~\ref{thm:CL}.}

We now give examples of LTSIs which are pre-reversible and where CS$\indt$ and CL$\indt$ fail.
\begin{example}\label{ex:prerev not CSi}
Consider the LTSI shown in Figure~\ref{fig:notIRE} including the dashed transitions.
We add coinitial independence as generated by BTI and PCI.
BTI gives $(Q',\rev b,Q) \ind (Q',\rev a,P')$ and  $(R,\rev c,Q') \ind (R,\rev a,S)$.
Assuming $t_0:P \tran a Q$ and $t_0\op:S \tran a R$, PCI gives three additional independence pairs for each of
 the two diamonds:  $(Q, b,Q') \ind \rev{t_0}$, $t_0 \ind (P,b,P')$ and $(P', \rev{b}, P) \ind (P', a, Q')$  for the diamond with the source $P$, and  $(Q', c,R) \ind (Q', \rev{a}, P')$, $(P',a,Q') \ind (P',c,S)$ and $(S,\rev{c}, P') \ind t_0\op $ for the other diamond.
The LTSI is pre-reversible.
However CS$\indt$ fails.
% \todo{can deduce this from Proposition~\ref{prop:IC CSi}}
Transition $t_0$ is followed by a path $Q \ptran {bc} R$ and the transition $t_0\op$
satisfies $t_0 \sqeqt t_0\op$.
If CS$\indt$ held we could deduce that $\rev{t_0} \ind (Q',c,R)$,
which is not the case.
Similarly, we see that IRE fails, since
$\rev{t_0} \sqeqt (Q',\rev a ,P') \ind (Q',c,R)$ but not $\rev{t_0} \ind (Q',c,R)$.
Note, however, that CL$\indt$ holds, since only transitions inside the same diamond are independent, and transitions on one side of the diamond are undone by the corresponding transition on the opposite side.
% \todo{Do we need to say more?  The path $r$ can be arbitrarily long,
% though of course essentially just a single transition. IVAN: I believe we say enough}
\finex
% Figure environment removed
\end{example}

\begin{example}\label{ex:prerev not CL}
Consider the LTSI shown in Figure~\ref{fig:notIRE} excluding the dashed transitions.
We add coinitial independence as given by BTI and PCI, similarly to the previous example.
We also add $(Q,\rev a ,P) \ind (Q',c,R)$.
The LTSI is pre-reversible.
However CL$\indt$ fails.
We have $t_0:P \tran a Q$, $Q \ptran {bc} R$ and $\rev{t_0} \ind (Q,b,Q')$, $\rev{t_0} \ind (Q',c,R)$.
Clearly CL$\indt$ fails, since we cannot reverse the $a$-transition at $R$. 
% If CL$\indt$ held we could deduce that there is a transition
% $S \tran a R$ with $t_0 \sqeqt (S,a,R)$.
IRE fails since $(Q',\rev a ,P') \sqeqt \rev{t_0} \ind (Q',c,R)$
but not $(Q',\rev a ,P') \ind (Q',c,R)$.
Note, however, that CS$\indt$ holds since the only way to undo transitions is with transitions on the opposite side of the same diamond, and the path connecting them is another transition of the same diamond. Hence, the condition on independence holds, thanks to BTI and PCI.
% \todo{Actually CS$\indt$ fails, since we can have longer paths like $r = b\rev b b$ or $r = b \rev a \rev b a b$ instead of just $r = b$.
% To get CS$\indt$ to hold we must add $\rev a \ind b$, $\rev b \ind a$, $\rev a \ind \rev b$, for the $a$ and $b$ transitions.}
\finex
\end{example}
Examples~\ref{ex:prerev not CSi} and~\ref{ex:prerev not CL} show that the stipulation of IRE cannot be omitted in the statements of Theorems~\ref{thm:CS} and~\ref{thm:CL}, respectively.
These examples also show that we cannot deduce CS$\indt$ or CL$\indt$ from CC, nor one from the other.
\begin{example}[CS$\indt$ and CL$\indt$ do not imply CC]\label{ex:CS CL not CC}
Consider the LTSI with states $P,Q,R,S$ and transitions $t:P\tran a Q$, $u:P \tran b R$,
$t':R \tran {a'} S$ and $u':Q \tran {b'} S$, with empty independence relation.
This is essentially the same as Example~\ref{ex:WFnotCC},
except that we have disambiguated the transition labels, to reflect that
%since
the four transitions
form four different events.
Then CC does not hold, but we claim that both CS$\indt$ and CL$\indt$ hold.

CS$\indt$: There are four possible cases to check, depending on the initial forward transition.
Consider first $t:P \tran a Q$ and some $r:Q \ptran\rho Q'$, $P' \tran a Q'$, where
$\cte(r,[t]) = 0$ and $(P,a,Q) \sqeqt (P',a,Q')$. 
Clearly $P' = P$ and $Q' = Q$.
To verify CS$\indt$ in this case, it is enough to show that $\cte(r,[u]) = \cte(r,[t']) = \cte(r,[u']) = 0$.
Since $r$ is a circuit, it enters each state as often as it leaves it.
Furthermore, since $\cte(r,[t]) = 0$, $r$ enters $Q$ from $P$ as often as it leaves $Q$ towards $P$.
Hence $r$ must enter $Q$ from $S$ as often as it leaves $Q$ towards $S$,
meaning that $\cte(r,[u']) = 0$.
We can similarly deduce that $\cte(r,[t']) = 0$ and $\cte(r,[u]) = 0$.
% We also need to consider
% $u':Q \tran {b'} S$ and some $r:S \ptran\rho S'$, $Q' \tran {b'} S'$, where
% $\cte(r,u') = 0$ and $(Q,b',S) \sqeqt (Q',b',S')$. 
% Clearly $Q' = Q$ and $S' = S$.
% To verify CS in this case, it is enough to show that $\cte(r,t) = \cte(r,u) = \cte(r,u') = 0$.
% The argument is similar to the previous case.
The remaining three cases with initial transitions $u$, $t'$ and $u'$ are similar to the case for $t$.
% \todo{note that we have verified the strong form of CS$\indt$.}

CL$\indt$: Again there are four cases to check, depending on the initial forward transition.
Consider first $t:P \tran a Q$ and some $r:Q \ptran\rho Q'$ where
$\cte(r,[t]) = 0$ and for all $t''$ in $r$ we have %$\cte(r,[t'']) = 0$
$\cte(r,[t'']) \leq 0$ (indeed, if $\cte(r,[t'']) > 0$ we would require $\rev{t} \ind t''$, which is false since the independence relation is empty, hence the condition for CL$\indt$ would hold trivially). However, if $\cte(r,[t'']) < 0$ then
%\todo{there is $t'''$ in $r$ with $t''' \sqeqt \rev{t''}$, so that $t''' = \rev{t''}$, and }
there is $t'''$ in $r$ with $[t'''] = [\rev{t''}]$ (in this example actually $t''' = \rev{t''}$) and 
$\cte(r,[\rev{t''}]) > 0$, but, for the same reason as above, we cannot have $\cte(r,[\rev{t''}]) > 0$  since the independence relation is empty. Hence for each $t''$ we have $\cte(r,[t'']) = 0$, which implies $Q'=Q$, since the net rotation (cfr.\ Figure~\ref{fig:WFnotCC}) of each transition is zero, and so the net rotation of $r$ is zero. The thesis follows trivially.
%% We must show that $Q' = Q$.
%% We consider the four transitions to correspond to rotations around the centre of the diamond,
%% as in Figure~\ref{fig:WFnotCC}.
%% The rotation of path~$r$ is the sum of the rotations of the transitions followed.
%% Since for all $t''$ in $r$ we have $\cte(r,[t'']) \leq 0$, we see that each $t''$ contributes a net \il{non-positive} %zero
%% rotation, so that $r$ has a net
%% %zero
%% \il{non-positive}
%% rotation \il{as well}.
%% \il{Notice that the only way to have a negative rotation starting from $Q$ is to traverse $t$ backwards more times than forwards, but this is impossible since $\cte(r,[t]) = 0$. Hence, the net rotation of $r$ needs to be $0$, implying that $Q' = Q$.}
%% \todo{[Needs refining]
%% Suppose that $r$ has a change of direction of rotation.
%% It must be of the form $r = r_1t_1 \rev{t_1} r_2$.
%% We can cancel $t_1 \rev{t_1}$ to obtain $r' = r_1r_2$ which is a path from $Q$ to $Q'$.
%% Clearly $\cte(r',e) = \cte(r,e)$ for all events $e$, including $[t_1]$.
%% By iterating this procedure we can eliminate all changes of rotation from $r$.
%% Now it is clear that for all $t''$ in $r$ we have $\cte(r,[t'']) \geq 0$.
%% Combining, we see that for all $t''$ in $r$ we have $\cte(r,[t'']) = 0$.
%% }
% Since $\cte(r,[t]) = 0$, $r$ enters $Q$ from $P$ as often as it leaves $Q$ towards $P$.
% Hence $r$ leaves $Q$ towards $S$ at least as often as it enters $Q$ from $S$,
% so that $\cte(r,[u']) \geq 0$.
% Hence $\cte(r,[u']) = 0$ and $Q' = Q$ \todo{I think this argument needs completing}.
The remaining three cases with initial transitions $u$, $t'$ and $u'$ are similar to the case for~$t$.\finex
\end{example}

The next axiom states that independence is fully determined by its restriction to coinitial transitions. It is related to axiom (E) of~\cite[page 325]{SNW96},
but here we allow reverse as well as forward transitions.
%
\begin{definition}{\bf Independence of events is coinitial (IEC)}\label{def:IEC}:
if $t_1 \ind t_2$ then $[t_1] \coind [t_2]$.
% there are $t'_1 \sqeqt t_1$, $t'_2 \sqeqt t_2$ such that $t'_1$ and $t'_2$ are
% coinitial and $t'_1 \ind t_2'$.
\end{definition}

Thanks to previous axioms, independence behaves well
w.r.t.~reversing.
\begin{definition}{\bf Reversing preserves independence (RPI)}\label{def:rpi}:
  if $t \ind t'$ then $\rev t \ind t'$.
\end{definition}
%
%\begin{restatable}{proposition}{RPI}\label{prop:RPI}
 \begin{proposition}\label{prop:RPI}
If an LTSI satisfies SP, PCI, IRE, IEC then it also satisfies RPI.
\end{proposition}
%\end{restatable}
\begin{proof}
Suppose $t \ind u$.
We must show $\rev t \ind u$.
By IEC we have $t' \sqeqt t$, $u' \sqeqt u$ such that $t' \ind u'$ and $t',u'$ are coinitial.
% Complete the square to get $t''$ and $u''$ using SP.
By SP there is a diamond $t',u',t'',u''$ with $t' \sqeqt t''$, $u' \sqeqt u''$.
Then $\rev{t'} \ind u''$ using PCI.
Then $\rev t \sqeqt \rev{t'} \ind u'' \sqeqt u$
and so by IRE $\rev t \ind u$ as required.
\end{proof}
%% \todo{IVAN: I believe the proof below does not require WF either, since Lemma 4.15 requires pre-reversible in the statement but it seems not used in the proof, hence we can weaken the precondition}
%% \todo{
%% \begin{proof}[Alternative proof assuming pre-reversible, IEC, IRE]
%% Suppose $t \ind u$.
%% We must show $\rev t \ind u$.
%% By IEC we have $[t] \coind [u]$.
%% By Lemma~\ref{lem:coind rev} we get $[\rev t] \coind [u]$.
%% Finally by Lemma~\ref{lem:coind IRE} we obtain $\rev t \ind u$.
%% \end{proof}
We can use IEC or IRE to show that transitions which are part of the same event cannot be independent.
\begin{definition}{\bf Event coherence (ECh)}\label{def:ECh}:
if $t \sqeqt t'$ then $t \notind t'$.
\end{definition}
\begin{proposition}\label{prop:ECh}
If a pre-reversible LTSI satisfies either IRE or IEC then it also satisfies ECh.
\end{proposition}
\begin{proof}
Assume for a contradiction that $t \sqeqt t'$ and $t \ind t'$.
First suppose that IRE holds.
We deduce $t \ind t$, contradicting irreflexivity of $\ind$.
Now suppose that IEC holds.
Then $[t] \coind [t']$, and so $[t] \coind [t]$, contradicting irreflexivity of $\coind$ (Proposition~\ref{prop:coind irref}).
\end{proof}


% \todo{
% \begin{remark}\label{rem:rev count}
% Note that if $u$ in $r$ is such that $\cte(r,[u]) < 0$
% then there is $u'$ in $r$ such that $u' \sqeqt \rev u$ and $\cte(r,[u']) > 0$.
% Hence the condition $\cte(r,[t])>0$ in Definition~\ref{def:safe live}
% can be replaced by $\cte(r,[t]) \neq 0$ to yield equivalent definitions of CS$\indt$, CL$\indt$,
% on the additional assumption of RPI.
% \end{remark}}

All the axioms that we have introduced so far are independent,
i.e.\ none is derivable from the remaining axioms.

%We first show some examples related to independence of IRE and IEC.
The next example shows that IRE is not implied by other axioms.
\begin{example}\label{ex:CLG CSi}
Let $t:P \tran a Q$, $u:P \tran b R$,
$u':Q \tran b S$, $t':R \tran a S$,
with $t \ind u$, $\rev u \ind t'$, $\rev{t'} \ind \rev{u'}$, $u' \ind \rev t$, namely we have independence at all corners of the diamond.
Here we have two forward events, labelled with $a$ and $b$ respectively.
We have $t' \sqeqt t \ind u$ but not $t' \ind u$, so that IRE fails.
% \todo{can deduce this from Proposition~\ref{prop:IC CSi},
% which also shows that CS$\indt$ fails.
% This example has the same properties as Example~\ref{ex:prerev not CSi}.}
However axioms SP, BTI, WF, PCI and IEC hold.\finex
\end{example}
The next example shows that IEC is not implied by other axioms.
\begin{example}\label{ex:IRE1}
Let $t:P \tran a Q$, $u:R \tran b S$,
where all states are distinct,
and let $t \ind u$.
Then IEC fails;
however axioms SP, BTI, WF, PCI and IRE hold.\finex
\end{example}
The counterexample above remains valid also if $Q=R$, as shown below.
\begin{example}\label{ex:IRE2}
Let $t:P \tran a Q$, $u:Q \tran b S$,
and let $t \ind u$.
Then IEC fails;
however axioms SP, BTI, WF, PCI and IRE hold.\finex
% Note also that CL$\indt$ fails,
% since $a$ cannot be reversed at $R$,
 % showing that for pre-reversible LTSIs IRE is not sufficient to show CL$\indt$.
% \todo{But CL$\indt$ holds for the revised definition.}
\end{example}

We can now prove the independence result.
%\begin{restatable}{proposition}{independent}\label{prop:independent}
\begin{proposition}\label{prop:ind}
The axioms 
SP, BTI, WF, PCI, IRE and IEC are independent of each other.
% and do not imply ED.
% \todo{can omit ED}
\end{proposition}
%\end{restatable}
\begin{proof}
For each of the six axioms we give an LTSI which satisfies the other five axioms but not the axiom itself.
In each case it is straightforward to check that the remaining axioms hold.

{\bf SP:} Let $t:P \tran a Q$ and $u:P \tran b R$ with $t \ind u$.

{\bf BTI:} Let $P \tran a R$ and $Q \tran b R$ with an empty independence relation
(Example~\ref{ex:notPL}).

{\bf WF:} Let $P_{i+1} \tran a P_i$ for $i = 0,1,\ldots$ with an empty independence relation.

{\bf PCI:} Let $t:P \tran a Q$, $u:P \tran b R$,
$u':Q \tran b S$, $t':R \tran a S$,
with $\rev{t'} \ind \rev{u'}$.

{\bf IRE:} See Example~\ref{ex:CLG CSi}.

{\bf IEC:} See Example~\ref{ex:IRE1} or Example~\ref{ex:IRE2}.
% As noted elsewhere, the diagram for LED with independence between all forward and reverse $a$ and all forward and reverse $b$ transitions satisfies all six axioms but not ED.
\end{proof}


\subsection{CS and CL via independent events}\label{sub:indev}

We now introduce a second version of causal safety and liveness,
which uses independence like CS$\indt$ and CL$\indt$,
but on events rather than on transitions. More precisely, we use coinitial independence $\coind$.

%We can give a second formulation of causal safety and liveness using $\coind$:
\begin{definition}\label{def:coind safe live}
Let $\mc L = (\Proc,\Lab,\tran{},\ind)$ be an LTSI. % pre-reversible LTSI.
\begin{enumerate}
\item
We say that $\mc L$ is \emph{coinitially causally safe} (CS$\ci$) if whenever
$t_0:P \tran a Q$, $r:Q \ptran \rho R$, $\cte(r,[t_0]) = 0$ and $t_0\op:S \tran a R$
with $t_0 \sqeqt t_0\op$,
then $[\rev{t_0}] \coind e$ for all 
events $e$ such that $\cte(r,e) > 0$.
\item
We say that $\mc L$ is \emph{coinitially causally live} (CL$\ci$) if whenever
$t_0:P \tran a Q$, $r:Q \ptran \rho R$ and $\cte(r,[t_0]) = 0$ and
$[\rev{t_0}] \coind e$, for all 
events $e$ such that $\cte(r,e) > 0$,
%\todo{why only forward events? What if in between there is just a backward transition 
%not independent from e?  Also, it seems the proofs consider all events.}  
%		\iu{
%IU: It may be related to Lemma 4.27 and the comment just below it, 
%which says it is sufficient to consider forward events only, but I might be wrong.}
then we have
$t_0\op:S \tran a R$ with $t_0 \sqeqt t_0\op$.
\end{enumerate}
\end{definition}
Note that in Definition~\ref{def:coind safe live} we operate at the level of events,
rather than at the level of transitions as in Definition~\ref{def:safe live}.
Also note that we could replace $[\rev{t_0}] \coind e$ by
$[t_0] \coind e$ using Lemma~\ref{lem:coind rev}.
We have used the former for compatibility with Definition~\ref{def:safe live}.
% \begin{lemma}[Ladder Lemma]\label{lem:ladder}
%
%\begin{restatable}{theorem}{CScoind}\label{thm:CS coind}
\begin{theorem}\label{thm:CS coind}
If an LTSI is pre-reversible then it satisfies CS$\ci$.
\end{theorem}
%\end{restatable}
%
\begin{proof}
Suppose $t_0:P \tran a Q$, $r:Q \ptran \rho R$
%\todo{NB $\cte(r,[t_0]) = 0$ not needed for proof}
and $t_0\op:S \tran a R$
with $t_0 \sqeqt t_0\op$.
By Lemma~\ref{lem:ladder} there is a path $s$ from $Q$ to $R$
such that for all $u$ in $s$ we have
$[t_0] \coind [u]$.
By CC, $r \ceqt s$.

%\todo{Revised proof:
Suppose that $e$ is an event and $\cte(r,e) > 0$.
Then $\cte(s,e) > 0$, thanks to Lemma~\ref{lemma:cccount}.
Hence there is $u$ in $s$ such that $[u] = e$. % or $[u] = \rev e$.
Since $[t_0] \coind [u]$,
also $[t_0] \coind e$.
% either $[t_0] \coind e$ or $[t_0] \coind \rev e$. 
% In both cases we can deduce
Hence $[\rev{t_0}] \coind e$ using
Lemma~\ref{lem:coind rev}. 
%}
%% Suppose that $e$ is a forward event and $\cte(r,e) > 0$.
%% Then $\cte(s,e) > 0$, thanks to Lemma~\ref{lemma:cccount}.
%% Hence there is $u$ in $s$ such that $[u] = e$.
%% By Lemma~\ref{lem:ladder} we have $[t_0] \coind [u]$.
%% Hence $[t_0] \coind e$. 
%% Suppose that $e$ is a forward event and $\cte(r,e) < 0$.
%% Then $\cte(s,e) < 0$, thanks to Lemma~\ref{lemma:cccount}.
%% Hence there is $u$ in $s$ such that $[\rev u] = e$,
%% or equivalently $[u] = \rev e$.
%% By Lemma~\ref{lem:ladder} we have $[t_0] \coind [u]$.
%% Hence $[t_0] \coind \rev e$,
%% and $[t_0] \coind e$, using Lemma~\ref{lem:coind rev}.
%% \todo{The proof would show a form of CS$\ci$ where we talk of all events $e$
%% rather than all forward events $e$.}
%\todo{It looks like we could generalise the proof of CS
%to allow $P \tran\alpha Q$ where $\alpha$ can be forward or backward.}
\end{proof}
%

We now introduce a weaker version of axiom IRE (Definition~\ref{def:IRE}).
\begin{definition}{\bf Coinitial IRE (CIRE)}\label{def:CIRE}:
% if $t' \sqeqt t \ind u \sqeqt u'$ and $t',u'$ are coinitial then $t' \ind u'$.
if $[t] \coind [u]$ and $t,u$ are coinitial then $t \ind u$.
\end{definition}
It is easy to see that IRE implies CIRE.
By considering Example~\ref{ex:CLG CSi}
we see that an LTSI can be pre-reversible and satisfy CIRE (and IEC) but not IRE.
Also, CIRE is not sufficient to ensure ECh (Definition~\ref{def:ECh}) holds, as shown by the next example.
% \todo{postponed since CIRE was not defined at the previous position}
\begin{example}\label{ex:CSi+RPI CLi}
Let $t:P \tran a Q$, $u:P \tran b R$,
$u':Q \tran b S$, $t':R \tran a S$.
We add independence between all pairs of distinct transitions
drawn from $t,u,t',u'$.
We furthermore add those independent pairs derived from closing under RPI.
We see that the LTSI is pre-reversible.  It satisfies CIRE and RPI,
but not ECh, since $t \sqeqt t'$ and also $t \ind t'$.
\finex
\end{example}

The next example shows that notions of CS/CL based on independence on transitions and on coinitial independence of events are not equivalent.
%\todo{IVAN: the next example leaves the door open for some implications, do they hold? Or do we have counterexamples?
%[Iain: Assume pre-reversible and consider CS$\indt$, CL$\indt$ and CL$\ci$.
%I think CS$\indt$ implies CL$\ci$ (Conjecture~\ref{prop:CSi CL<}).
%Example~\ref{ex:prerev not CSi}:
%CLG holds, and hence CIRE, CL$\indt$, CL$\ci$.
%Shows that CL$\indt$+CL$\ci$ does not imply CS$\indt$.
%Can also use Example~\ref{ex:halfcube} for this.
%Example~\ref{ex:prerev not CL}:
%CIRE also holds, and hence CL$\ci$.
%Also CS$\indt$ holds, I think.
%Shows that CS$\indt$(+CL$\ci$) does not imply CL$\indt$.
%Example~\ref{ex:IC CLi}:
%CL$\indt$ holds but not CS$\indt$, CL$\ci$.]}
% Question: can we find an example where CS$\indt$ holds but not CL$\ci$?
% Question: can we find an example where CS$\indt$+CL$\indt$ holds but not CL$\ci$?
\begin{example}\label{ex:IC CLi}
% \todo{May have to be omitted for space reasons,
% but this example is less pathological
% than the ones with repeated events.}
Consider the LTSI in Figure~\ref{fig:IC CLi}.
% Figure environment removed
Independence is given by closing under BTI and PCI. Clearly WF and SP hold; hence the LTSI is pre-reversible and satisfies CS$\ci$.
There are three events, labelled $a,b,c$, which are all independent of each other.
Furthermore IEC holds, but not CIRE
(noting that the leftmost $b$ and $c$ transitions are coinitial but not independent, while the corresponding events are coinitially independent thanks to the rightmost square).
Also CL$\ci$ fails: consider $P \tran a Q \tran b R$,
where $a$ cannot be reversed at $R$ even though $[Q \tran{\rev{a}} P] \coind [Q \tran b R]$.
Differently from CS$\ci$, CS$\indt$ fails:
% \todo{can deduce this from Proposition~\ref{prop:IC CSi}}
e.g., from the leftmost corner one can do $bac\rev b$, reversing $b$, but the inverse of the first $b$-transition is not independent with the $c$-transition.
Differently from CL$\ci$, CL$\indt$ holds:
the only state at which any event that has occurred cannot be
immediately reversed is $R$.
So we can restrict attention to instances of $P' \tran a Q'$, $r:Q' \ptran \rho R$.
Furthermore $r$ must finish with either $Q \tran b R$ or the $c$ transition to $R$.
These two transitions are not independent with any inverse $a$ transition.
Hence CL$\indt$ holds in these cases vacuously.\finex
\end{example}
\begin{proposition}\label{prop:CSindt RPI CIRE}
Let $\mc L$ be a pre-reversible LTSI.
If $\mc L$ satisfies CS$\indt$ and RPI then $\mc L$ also satisfies CIRE.
\end{proposition}
\begin{proof}
%% Assume that $\mc L$ satisfies CS$\indt$ and RPI.
%% Suppose that $t,u$ are coinitial transitions such that $[t] \coind [u]$.
%% We must show that $t \ind u$.
%% Since $[t] \coind [u]$, there are coinitial $t',u'$ such that
%% $t \sqeqt t' \ind u' \sqeqt u$.
%% By SP we can complete a square containing $t',u'$ and two further cofinal transitions
%% $t'' \sqeqt t'$ and $u'' \sqeqt u'$ both with target $R$.

%% Suppose first that $t:P \tran a Q$ and $t':P' \tran a Q'$ are forward.
%% By Lemma~\ref{lem:ladder} there is a path $r:Q \ptran \rho Q'$.
%% % such that $[t] \coind [u_1]$ for all $u_1$ in $r$.
%% Let $s' = \rev t u \rev u t r$ (a path from $Q$ to $Q'$),
%% and consider the path
%% % $s = \rev t u \rev u t r u''$
%% $s = s'u''$
%% from $Q$ to $R$.
%% We see that $\cte(s,[t])= 0$,
%% using Lemma~\ref{lem:cte zero} applied to $t,t''$ and~$s$.
%% Hence CS$\indt$ applies to $t$ together with $s$ and $t''$.
%% We deduce that
%% $\rev t \ind u_1$ for all $u_1$ in $s$ such that $\cte(s,[u_1]) \neq 0$.
%% We see that $\cte(ts',[u]) = 0$ using
%% using Lemma~\ref{lem:cte zero} applied to $\rev u,\rev{u''}$ and $ts'$.
%% Noting that $\und{[t]} \neq \und{[u]}$
%% by %irreflexivity of $\coind$
%% Proposition~\ref{prop:coind und},
%% we obtain $\cte(s,[u]) = 1$
%% and so $\rev t \ind u$.
%% We deduce $t \ind u$ using RPI.

%% Suppose instead that
%% $t:P \tran {\rev a} Q$ and $t':P' \tran {\rev a} Q'$ are backward.
%% By Lemma~\ref{lem:ladder} applied to $\rev t, \rev {t'}$
%% there is a path $r:P \ptran \rho P'$.
%% % such that $[\rev t] \coind [u_1]$ for all $u_1$ in $r$.
%% Consider the path $s = u \rev u ru'$.
%% We see that $\cte(s,[\rev t])= 0$,
%% using Lemma~\ref{lem:cte zero} applied to $\rev t, \rev {t''}$ and~$s$.
%% Hence
%% CS$\indt$ applies to $\rev t$ together with $s$ and $\rev {t''}$.
%% We deduce that
%% $t \ind u_1$ for all $u_1$ in $s$ such that $\cte(s,[u_1]) \neq 0$.
%% We see that $\cte(u \rev u r,[u]) = 0$ using
%% using Lemma~\ref{lem:cte zero} applied to $\rev u,\rev{u'}$ and $u \rev u r$.
%% Clearly $\cte(s,[u]) = 1$ and so $t \ind u$ as required.

%\todo{Revised proof:
Assume that $\mc L$ satisfies CS$\indt$.
Suppose that $t,u$ are coinitial transitions such that $[t] \coind [u]$.
We must show that $t \ind u$.
We can suppose that at least one of $t$ and $u$ is forward;
otherwise we can obtain $t \ind u$ from BTI.
Without loss of generality, suppose that $t:P \tran a Q$ is forward.
Since $[t] \coind [u]$, there are coinitial $t':P' \tran a Q'$ and $u'$
such that $t \sqeqt t' \ind u' \sqeqt u$.
By SP we can complete a square containing $t',u'$ and two further transitions
$t'' \sqeqt t'$ and $u'' \sqeqt u'$ both with the same target~$R$.

By Lemma~\ref{lem:ladder} there is a path $s:Q \ptran \rho Q'$.
Let $r' = \rev t u \rev u t s$ (a path from $Q$ to $Q'$),
and consider the path
$r = r'u''$
from $Q$ to $R$.
We see that $\cte(r,[t])= 0$,
using Lemma~\ref{lem:cte zero} applied to $t,t''$ and~$r$.
Hence CS$\indt$ applies to $t$ together with $r$ and $t''$.
We deduce that
$\rev t \ind u_1$ for all $u_1$ in $r$ such that $\cte(r,[u_1]) > 0$.
We see that $\cte(tr',[u]) = 0$ using
Lemma~\ref{lem:cte zero} applied to $\rev u,\rev{u''}$ and $tr'$.
Noting that $\und{[t]} \neq \und{[u]}$
by %irreflexivity of $\coind$
Proposition~\ref{prop:coind und},
we obtain $\cte(r,[u]) = 1$
and so $\rev t \ind u$.
We deduce $t \ind u$ using RPI.
\end{proof}
We cannot omit the assumption of RPI in Proposition~\ref{prop:CSindt RPI CIRE},
in view of the following example.
\begin{example}\label{ex:halfcube mod}
Consider the `half cube' LTSI with transitions $a,b,c$ in Figure~\ref{fig:halfcube}.
% Figure environment removed
We add independence as given by BTI and PCI, and also between all pairs of transitions $t,u$ where at least one of $t,u$ is backward,
and $t \not\sqeqt u$, $t \not\sqeqt \rev u$. Clearly RPI does not hold. 
The LTSI is pre-reversible, and IEC holds.
CIRE does not hold; note that the $a$ and $b$-events are independent,
but after performing $c$ there are coinitial $a$ and $b$-transitions
which are not independent.
Both CL$\ci$ and CL$\indt$ hold: %\todo{[can deduce CL$\indt$ from CL$\ci$+IEC] not really needed, in case we need to postpone this example}:
note that at any state, all events that have occurred can be reversed
immediately.
%IC \todo{Definition~\ref{def:coinitial LTSI}} holds---all independence is coinitial.
We have ensured that CS$\indt$ holds, since all independence deducible from CS$\indt$ must involve a backward transition $\rev{t_0}$ and a transition $u$ such that $t_0 \not\sqeqt u$ and
$t_0 \not\sqeqt \rev u$.\finex
\end{example}

%\todo{IVAN: are we sure of the result below? It seems to me (2) should follow from CC. Also, if it holds, do we want to include this result in the paper?
%[Iain: (2) fails for Example~\ref{ex:IC CLi} and so does not follow from CC.]}
%\todo{
We can characterise CIRE as being equivalent to coinitial transitions
with a common derivative process being independent.
\begin{proposition}\label{prop:char CIRE}
Let $\mc L$ be a pre-reversible LTSI.  The following are equivalent:
\begin{enumerate}
\item\label{item:CIRE}
$\mc L$ satisfies CIRE;
\item\label{item:CDI}
If $t:P \tran \alpha Q$, $r:Q \ptran \rho S$ and 
$u:P \tran \beta R$, $s:R \ptran \sigma S$ where 
$\und\alpha \neq \und\beta$ and
$\cte(r,[t]) = \cte(s,[u]) = 0$ then
$t \ind u$.
\end{enumerate}
\end{proposition}
\begin{proof}
Assume (\ref{item:CIRE}).
Let $t:P \tran \alpha Q$, $r:Q \ptran \rho S$ and 
$u:P \tran \beta R$, $s:R \ptran \sigma S$ where 
$\und\alpha \neq \und\beta$ and
$\cte(r,[t]) = \cte(s,[u]) = 0$.
We must show $t \ind u$.
Since the LTSI is pre-reversible, polychotomy holds for events $[t]$ and $[u]$
(Proposition~\ref{prop:poly}).
We can exclude $[t] = [u]$ since
$\und\alpha \neq \und\beta$.
There is a rooted path $r_0$ from some irreversible $I$ to $P$.
Since NRE holds (Proposition~\ref{prop:NRE}),
$\cte(r_0,[t]) = \cte(r_0,[u])$.
By considering the paths $r_0t $ and $r_0u$ we deduce that neither $[u] < [t]$ nor $[t] < [u]$ hold.
By CC applied to $tr$ and $us$ we see that $\cte(r,[u]) = 1$.
Hence $r_0tr$ is a rooted path with $\cte(r_0tr,[t]) = \cte(r_0tr,[u]) = 1$,
so that we can exclude $[t] \cf [u]$.
By polychotomy we conclude that $[t] \coind [u]$.
Then $t \ind u$ by CIRE.

Assume (\ref{item:CDI}).
Let $[t] \coind [u]$ where
$t:P \tran \alpha Q$ and $u:P \tran \beta R$ are coinitial.
We must show $t \ind u$.
First note that $\und\alpha \neq \und\beta$
by Proposition~\ref{prop:coind und}.
We have $t \sqeqt t' \ind u' \sqeqt u$ where
$t':P' \tran \alpha Q'$ and $u':P' \tran \beta R'$ are coinitial.
By SP we have
$t'':R' \tran \alpha S$ and $u'':Q' \tran \beta S$.
By Lemma~\ref{lem:ladder} we have
$r':Q \ptran\rho Q'$ such that for all $u_1$ in $r'$ we have $[t] \coind [u_1]$,
and $s':R \ptran\sigma R'$ such that for all $u_2$ in $s'$ we have $[u] \coind [u_2]$.
Let $r = r'u''$ and $s = s't''$.
We have $\cte(r,[t]) = \cte(s,[u]) = 0$ using Lemma~\ref{lem:cte zero}.
Hence $t \ind u$ as required, using the hypothesis.
\end{proof}

Notably, in the proof of (\ref{item:CIRE}) $\Rightarrow$ (\ref{item:CDI}),
CIRE is only used in the last step. Hence, the result could be rephrased by stating that any pre-reversible LTSI satisfies (\ref{item:CDI}),
with a conclusion of $[t] \coind [u]$ rather than $t \ind u$.

The independence result in Proposition~\ref{prop:ind} holds also if we replace IRE by CIRE.
\begin{proposition}\label{prop:ind CIRE}
The axioms 
SP, BTI, WF, PCI, CIRE and IEC are independent of each other.
\end{proposition}
\begin{proof}
For each of the six axioms we need to give an LTSI which satisfies the other five axioms but not the axiom itself.
Since IRE implies CIRE, for all axioms apart from CIRE we can reuse the examples given in the proof of Proposition~\ref{prop:ind}.
Example~\ref{ex:IC CLi} provides an LTSI where CIRE fails and the remaining five axioms hold.
\end{proof}

We can distinguish three mutually exclusive cases for CIRE
(Definition~\ref{def:CIRE}):
\begin{description}
\item
[forward case:] both transitions are forward;
\item
[backward-forward case:] one transition is backward, one is forward;
\item
[backward case:] both transitions are backward (implied by BTI).
\end{description}
The second case is particularly relevant for the characterisation of CL$\ci$; hence we state it as a separate axiom.
\begin{definition}\label{def:BFCIRE}
{\bf Backward-Forward CIRE (BFCIRE)}:
if $t:P \tran a Q$ and $u:Q \tran b R$ and $[\rev t] \coind [u]$ then $\rev t \ind u$.
\end{definition}
Thus BFCIRE is just CIRE specialised to the case where one of the
coinitial transitions is backward and one is forward. It has some similarity with one of the properties of transition systems with independence in \cite{NW95} and \cite[Definition 4.1]{SNW96}, and Sideways Diamond properties in \cite{PU07a,Aub22}. However, all of these properties state that if two consecutive forward transitions are independent then they are two sides of a commuting diamond.   
%\todo{[Add citations for Sideways Diamond in the literature.]}

Analogously to what was done in Theorem~\ref{thm:CL} for CL$\indt$, we give below conditions for ensuring CL$\ci$.
Notably, here BFCIRE is necessary and sufficient,
while for CL$\indt$ we required IRE, which was sufficient but not necessary.
%\todo{IVAN: do we have an example showing that CIRE would not be enough before? Anything else meaningful to say in the comparison?
%[Iain:
%Example~\ref{ex:CLG}:
%CLG holds, and hence CIRE.
%Shows that CIRE does not imply CS$\indt$.
%Example~\ref{ex:prerev not CL}:
%CIRE also holds.
%Shows that CIRE does not imply CL$\indt$.
%]}

%\begin{restatable}{theorem}{CLcoind}\label{thm:CL coind}
\begin{theorem}\label{thm:CL coind}
Let $\mc L$ be a pre-reversible LTSI.  Then the following are equivalent:
\begin{enumerate}
\item\label{item:L BFCIRE}
$\mc L$ satisfies BFCIRE;
\item\label{item:L CLci}
$\mc L$ satisfies CL$\ci$.
\end{enumerate}
\end{theorem}
%\end{restatable}
\begin{proof}
Assume (\ref{item:L BFCIRE}).
  Suppose $t_0:P \tran a Q$, $r:Q \ptran \rho R$ and $\cte(r,[t_0]) =
  0$ and $[\rev{t_0}] \coind e$, for all $e$ such that $\cte(r,e) > 0$.  We
  have to show that there is $t_0\op:S \tran a R$ with $t_0 \sqeqt
  t_0\op$.

  Thanks to PL, there is $T$ such that $b: P
  \ptran{\rho_b} T$ and $f: T \ptran{\rho_f} R$, with $b$ backward and
  $f$ forward.
  By CC, $t_0r \ceqt bf$.
  Since $\cte(r,[t_0]) = 0$, thanks to
  Lemma~\ref{lemma:cccount} $\cte(bf,[t_0])=1$. As a consequence,
  there is a transition $t'_0:P' \tran a Q' \in [t_0]$ in
  $f$ (which is unique by Proposition~\ref{prop:NRE}).
  Let $f'$ be the portion of $f$ from $Q'$ to $R$.

  If we can show that $[\rev{t_0}] \coind [t'']$ for each transition $t''$ in $f'$,
  then the thesis will follow by commuting
  $t'_0$ with all such transitions using SP and BFCIRE.

  By Lemma~\ref{lem:ladder} there is a path $s$ from $Q$ to $Q'$ such that $[t_0] \ind [u]$  % \todo{and $t_0 \ind \rev u$}
  for all $u$ in $s$.
  By CC, $r \ceqt sf'$.
  Take any $t''$ in $f'$.
  By Lemma~\ref{lemma:cccount}, $\cte(r,[t'']) = \cte(s,[t'']) + \cte(f',[t''])$.
  % If $\cte(s,[t'']) > 0$ then there is $u$ in $s$ such that $u \sqeqt t''$,
  % and so $t_0 \ind t''$ using IRE.
  If $\cte(s,[t'']) < 0$ then there is $u$ in $s$ such that $u \sqeqt \rev{t''}$.
  Now $[t_0] \coind [u]$, and so $[t_0] \coind [t'']$ using Lemma~\ref{lem:coind rev}.
  So suppose $\cte(s,[t'']) \geq 0$.
  Since $\cte(f',[t'']) > 0$, we have $\cte(r,[t'']) > 0$. 
  So there is $u$ in $r$ such that $u \sqeqt t''$,
  and by hypothesis $[\rev{t_0}] \coind [u]$, so that $[\rev{t_0}] \coind [t'']$.

Assume (\ref{item:L CLci}). 
Suppose that $t_0:P \tran a Q$ and $u:Q \tran b R$ and $[\rev {t_0}] \coind [u]$.
Clearly $\cte(u,[t_0]) = 0$.
By CL$\ci$ we have $t_0^{\dagger}:S \tran a R$ with $t_0 \sqeqt t_0^{\dagger}$.
Using BTI and SP we can complete a square starting with $\rev u$ and
$\rev{t_0^{\dagger}}$.
Using BLD this square must include $t_0$.
Using PCI we see that $\rev{t_0} \ind u$ as required.
\end{proof}
CL$\ci$ (and BFCIRE) do not imply CIRE,
as shown by Example~\ref{ex:halfcube mod}.
\begin{lemma}\label{lem:CSi BFCIRE}
Let a pre-reversible LTSI satisfy CS$\indt$.  Then it satisfies BFCIRE.
\end{lemma}
\begin{proof}
Suppose $t_0:P \tran a Q$ and $u:Q \tran b R$ and $[\rev {t_0}] \coind [u]$.
We must show that $\rev {t_0} \ind u$.

By Lemma~\ref{lem:coind rev} $[t_0] \coind [u]$ and so there are coinitial
$t_0':P' \tran a Q'$ and $u':P' \tran b R'$ with
$t_0 \sqeqt t'_0 \ind u' \sqeqt u$. 
Using SP we can complete a square with $t_0^{\dagger}:R' \tran a S'$ and $u'':Q' \tran b S'$.
By Lemma~\ref{lem:ladder} applied to $u$ and $u''$ we have a path $s$ from $R$ to $S'$.
Let $r = us$.
Then $\cte(r,[t_0]) = 0$ using Lemma~\ref{lem:cte zero}.
Also $\cte(s,[u]) = 0$ using Lemma~\ref{lem:cte zero},
so that $\cte(r,[u]) > 0$.
By CS$\indt$ applied to $t_0,t_0^{\dagger}$ and $r$
we deduce $\rev{t_0} \ind u$ as required.
\end{proof}

Perhaps surprisingly,
we can now relate safety with independence of transitions to liveness with independence of events.
\begin{proposition}\label{prop:CSi CLci}
Let a pre-reversible LTSI satisfy CS$\indt$.
Then it satisfies CL$\ci$.  
\end{proposition}
\begin{proof}
By Lemma~\ref{lem:CSi BFCIRE} and Theorem~\ref{thm:CL coind}.
\end{proof}
% If SP, BTI, WF, PCI, CIRE hold then NRE holds and therefore polychotomy holds.
%\todo{Omit the next sentence: }The stipulation of CIRE cannot be omitted from the statement of Theorem~\ref{thm:CL coind} in view of Example~\ref{ex:IC CLi}.

CL$\ci$ (and BFCIRE) do not imply CS$\indt$,
as shown by the next example.
\begin{example}\label{ex:halfcube}
Consider the `half cube' LTSI with transitions $a,b,c$ in Figure~\ref{fig:halfcube}.
We add independence as given by BTI and PCI.
The LTSI is pre-reversible.
As in Example~\ref{ex:halfcube mod}, CIRE does not hold
%; note that the $a$ and $b$-events are independent,
%but after performing $c$ there are coinitial $a$ and $b$-transitions
%which are not independent.
while both CL$\ci$ (hence BFCIRE) and CL$\indt$ hold.  %\todo{[can deduce CL$\indt$ from CL$\ci$+IEC] not really needed, in case we need to postpone this example}:
%note that at any state, all events that have occurred can be reversed
%immediately.
%IC \todo{Definition~\ref{def:coinitial LTSI}} holds---
All pairs of independent transitions are coinitial.
CS$\indt$ however does not hold:
% \todo{can deduce this from Proposition~\ref{prop:IC CSi}}
consider $t_0:P \tran c Q$, $r:Q \ptran {\rev a b} R$, $S \tran c R$---here we do not have $\rev {t_0} \ind (Q',b,R)$.\finex
\end{example}

%\begin{restatable}{proposition}{correspondence}\label{prop:correspondence}
\begin{proposition}\label{prop:correspondence}
Let $\mc L$ be a pre-reversible LTSI satisfying IEC.
If $\mc L$ satisfies CL$\ci$ then $\mc L$ satisfies CL$\indt$.
\end{proposition}
%\end{restatable}
\begin{proof}
Immediate from the definitions.
\end{proof}
%% \todo{IVAN:
%%   Dually
%% \begin{proposition}\label{prop:correspondence}
%% Let $\mc L$ be a pre-reversible LTSI satisfying IEC.
%% If $\mc L$ satisfies CS$\indt$ then $\mc L$ satisfies CS$\ci$.
%% \end{proposition}
%% %\end{restatable}
%% \begin{proof}
%% Immediate from the definitions.
%% \end{proof}
%% [Iain: but CS$\ci$ follows just from pre-reversible.]}

We next give an example where CC holds but not CS$\ci$ (and not PCI).
\begin{example}\label{ex:CC not CS}
%  \todo{IVAN: try to highlight the relevant items with colors or something similar}
  Consider the cube with transitions $a,b,c$ on the left in Figure~\ref{fig:diamonds},
where the forward direction is from left to right.
% Figure environment removed
We add independence as given by BTI.  So SP, BTI, WF hold, but not PCI.
Consider the bold path from the leftmost end:
%From the start
we have an $a$-transition followed by a path $r = bc$ followed by $\rev a$.
For CS$\ci$ to hold, we want $\rev a$ to be the reverse of the same event as the first $a$.
They are connected by a ladder with sides $cb$.
We add independence for all corners on the two faces of the ladder ($ac$ and $ab$). Transitions
$\rev b$ and $\rev c$ at $P$ are independent (by BTI) so we obtain $\rev b\rev c \ceqt \rev c \rev b$, where 
$\rev b\rev c$ is dashed and $\rev c \rev b$ is bold. Since $\ceqt$ is closed under composition, we get
$bc \ceqt cb$.
However the bold $b$ is a different event from the event of the top $b$s since the bold-dashed $bc$ face does not have independence at each corner.
Therefore we do not get $[a] \coind [b]$ for the bold $a$ and bold $b$, and CS$\ci$ fails. However, we note that we do have $[a] \coind [b]$ for the bold $a$ and the dashed $b$ since $a$ and $b$ at $Q$ are independent.
%\todo{Iain: I find this very hard to follow - perhaps some colours would help?}\finex
%\todo{Clearly this example is a bit artificial.
%Does the example show that our definition of event is reasonable or not?} 
\end{example}
%% \begin{example}%\label{ex:CC not CS}
%%   \todo{Example~\ref{ex:CC not CS} and diagram rephrased.}
%%     Consider the cube with twelve transitions on the left in Figure~\ref{fig:diamonds1},
%% where the forward direction is from left to right.
%% The four transitions $a_1,a_2,a_3,a_4$ have the same label $a$;
%% similarly for the $b_i$ and $c_i$ transitions.
%% % Figure environment removed
%% We add independence as given by BTI.  So SP, BTI, WF hold, but not PCI.
%%  From the start we have $a_1$ followed by a path $r = b_1c_1$ followed by $\rev {a_2}$.
%% % For CS$\ci$ to hold, we want
%% Let us make $\rev {a_2}$ into the reverse of the same event as $a_1$.
%% They are connected by a ladder with rungs $a_1,a_3,a_2$ and sides $c_2,b_2$ and $c_3,b_3$.
%% We add independence for all corners on the two faces of the ladder. % ($ac$ and $ab$).
%% Now we have $a_1 \sqeqt a_3 \sqeqt a_2$;
%% also $c_2 \sqeqt c_3$ and $b_2 \sqeqt b_3$.
%% Furthermore $[a_1] \coind [b_2]$ since $a_3 \ind b_3$.
%% Then we get $b_1c_1 \ceqt c_2b_2$ (independence at a single corner is enough).
%% However $b_1$ is a different event from $b_2$ since the $b_1c_2$ face does not have independence at each corner.
%% Therefore we do not get $[a_1] \coind [b_1]$, and CS$\ci$ fails.\finex
%% %\todo{Clearly this example is a bit artificial.
%% %Does the example show that our definition of event is reasonable or not?} 
%% \end{example}
%
We next give an example where CS$\ci$ and CL$\ci$ hold 
but not CC.
%\todo{I think the counterexample below holds using the same reasoning also for plain CS and CL, am I right?
%}

%\iu{
%IU: I think CS$\indt$ fails in the RHS diagram: $P_2 \tran a Q_1$ and $Q_1 \tran c Q_0$ are not independent 
%but we can reverse $a$ at $Q_0$. CL$\indt$ seems to hold there.}
%\todo{Iain: I agree.}

%\todo{CL$<$ trivially holds since there are no rooted paths, but CS$<$ does not for the same reason, hence this is not a counterexample for these.
%}

%\iu{
%IU: I am not sure why we consider rooted paths; isn't the property supposed to hold for all paths? 
%I think both CS$_<$ and CL$_<$ hold for RHS diagram in Fig.~2.}
%\todo{Iain: Since there are no rooted paths, the ordering relation on events is empty.
%Hence CS$_<$ holds trivially.
%However CL$_<$ fails: consider a path $ab$.  Then $a \not < b$,
%but $a$ cannot be reversed after $b$.}
\begin{example}\label{ex:CS+CL not CC}
Consider the LTSI with
$Q_i \tran b P_i$, $P_{i+1} \tran c P_i$, $Q_{i+1} \tran c Q_i$,
$P_{i+1} \tran a Q_i$ for $i = 0,1,\ldots$. This is shown on the right in Figure~\ref{fig:diamonds}.
Clearly WF does not hold.
% However AC and UT hold.
We add coinitial independence to make BTI and PCI hold.
Then also SP and CIRE hold.
% I think we can add independence so that
% BTI, SP, RPI, PCI and IRE hold. 
However, CC fails since, for example
$P_1 \tran a Q_0 \tran b P_0$ and $P_1 \tran c P_0$ are coinitial and cofinal but not causally equivalent.
Note that there are just three events $a,b,c$ with $a \coind c$,
$b \coind c$ but not $a \coind b$.
% NRE does not hold, but 
CS$\ci$ and CL$\ci$ hold. Indeed, $c$ is independent from every other action, and it can always be undone, while $a$ and $b$ are independent from $c$ only and they can be undone after any path composed by $c$ and no others.
In more detail, if we have a path $a r \rev a$ with $\cte(r,a) = 0$ then $\cte(r,b) = 0$,
and if we have a path $b r \rev b$ with $\cte(r,b) = 0$ then $\cte(r,a) = 0$.\finex
%, and $b$ is concurrent to $c$ only and can be undone after any path composed by $c$ and no others. 
%\todo{diagram might help}
\end{example}

The independence result in Proposition~\ref{prop:ind CIRE} holds also if we replace CIRE by BFCIRE.
\begin{proposition}\label{prop:ind BFCIRE}
The axioms 
SP, BTI, WF, PCI, BFCIRE and IEC are independent of each other.
\end{proposition}
\begin{proof}
%-short version-
%The proof is essentially the same as for Proposition~\ref{prop:ind CIRE},
%using the same examples and
%noting that CIRE implies BFCIRE, and that BFCIRE (equivalent to CL$\ci$) fails
%in Example~\ref{ex:IC CLi}.
%
%[longer version]
For each of the six axioms we need to give an LTSI which satisfies the other five axioms but not the axiom itself.
Since CIRE implies BFCIRE, for all axioms apart from BFCIRE we can reuse the examples given in the proofs of Proposition~\ref{prop:ind CIRE}
(and of Proposition~\ref{prop:ind}).
Example~\ref{ex:IC CLi} provides an LTSI where BFCIRE (equivalent to CL$\ci$) fails and the remaining five axioms hold.
\end{proof}

%\subsection{EIT - to be omitted}
%\input {EIT.tex}

\subsection{CS and CL via ordering of forward events}\label{sub:ord}
% \todo{`Event' changed to `forward event' throughout Section~\ref{sub:ord}.}
We now give definitions of causal safety and causal liveness 
%CS/CL 
using ordering on forward events.
To this end, we exploit the causality relation $\leq$ on such events (see Definition~\ref{def:ordering}).
\begin{definition}\label{def:safe live <}
Let $\mc L = (\Proc,\Lab,\tran{},\ind)$ be an LTSI.
\begin{enumerate}
\item
We say that $\mc L$ is \emph{ordered causally safe (CS$_<$)} if whenever
$t_0:P \tran a Q$, $r:Q \ptran \rho R$, $\cte(r,[t_0]) = 0$ and $t_0\op:S \tran a R$
with $t_0 \sqeqt t_0\op$,
then $[t_0] \not < e'$ for all forward events $e'$ such that $\cte(r,e') > 0$.
\item
We say that $\mc L$ is \emph{ordered causally live (CL$_<$)} if whenever
$t_0:P \tran a Q$, $r:Q \ptran \rho R$ and $\cte(r,[t_0]) = 0$ and
$[t_0] \not < e'$ for all forward events $e'$ such that $\cte(r,e') > 0$
then we have
$t_0\op:S \tran a R$ with $t_0 \sqeqt t_0\op$. 
%\todo{[Iain: might mean adjusting notation in statement and proof of
%  Lemma~\ref{lem:coind <} and proof of Proposition~\ref{prop:CS CL coind <}.]
%  IVAN: changed $e$ back to $e'$ given Iain observation}
\end{enumerate}
\end{definition}
%\todo{IVAN: Since here we consider only forward events, and require a positive number of occurrences, we have no way to check backward events with a positive number of occurrences. Should we check their complement?}
%\todo{IL: should we use $[t]$ instead of $e'$ in the definition so to have a more direct correspondence?
%}

%\iu{
%IU: I think we discussed this before and decided to keep this expression as it corresponds very closely to 
%$[P,a,Q] \coind e$ for CS$\ci$ and $(P,a,Q) \ind t$ for CS$\indt$.}
The only difference between CS$_<$ and CS$\indt$
(Definition~\ref{def:safe live})
is that the former ensures $[t_0] \not < [t]$ instead of $\rev{t_0} \ind t$ for all transitions $t$ such that $[t]$ has a positive number of occurrences in $r$. Similarly for CL. Notably, we do not require  $[\rev{t_0}] \not < [t]$ since $<$ is defined on forward events and $t_0$ is forward.

%% We postpone stating sufficient conditions for CS$_<$ and CL$_<$ until we have
%% discussed the relationship between independence, causation and conflict on events
%% (see the next subsection).


%% \subsection{Polychotomy}
%% In this section we relate our three versions of causal safety and liveness,
%% with the help of what we call \emph{polychotomy},
%% which states that if events do not cause each other and are not in conflict,
%% then they must be independent.

%% \begin{example}\label{ex:pre-prime not CS}
%% \todo{remove?}
%% Consider the LTSI in Figure~\ref{fig:repeated2}.
%% % Figure environment removed
%% % It satisfies pre-prime and FD but not CS (or SD, CL).
%% % We have a path $abc\rev a$ from $P$ to $S$, but
%% % (the event labelled with) $a$ causes $c$.
%% % Thus being pre-prime is not enough to guarantee CS.
%% We add independence to make BTI and PCI hold.
%% Both SP and WF hold, and so the LTSI is pre-reversible. Thus, CC holds as well.
%% There are three events, labelled with $a,b,c$.
%% Clearly NRE fails for both $a$ and $b$.
%% We see that $a < c$ but also $a \coind c$, so that polychotomy fails.
%% CS$\ci$ holds by Theorem~\ref{thm:CS coind}.
%% However CS$_<$ fails:
%% consider the transition $P \tran a Q$ together with the path $r:Q \ptran {bc} R$ and
%% $S \tran a R$, and note that $a < c$.\finex
%% \end{example}
It may seem that the definition above does not take into account backward events that may occur in $r$, but the next lemma shows that such events are necessarily independent from $[t_0]$.
This allows us to connect ordered safety and liveness
with %coinitial
safety and liveness based on independence of events.
%\begin{restatable}{lemma}{coindless}\label{lem:coind <}
\begin{lemma}\label{lem:coind <}
Suppose that an LTSI is pre-reversible.
Suppose $t_0:P \tran a Q$, $e = [t_0]$, $r:Q \ptran \rho R$ and $\cte(r,e) = 0$.
Let $e'$ be a forward event:
\begin{enumerate}
\item\label{item:greater}
if $\cte(r,e') > 0$ then exactly one of $e \coind e'$ and $e < e'$ holds;
\item\label{item:less}
if $\cte(r,e') < 0$ then $e \coind e'$.
\end{enumerate}
\end{lemma}
%\end{restatable}
\begin{proof}
We know that polychotomy holds by Proposition~\ref{prop:poly}.
Also NRE holds by Proposition~\ref{prop:NRE}.
Suppose $t_0:P \tran a Q$, $e = [t_0]$, $r:Q \ptran \rho R$ and $\cte(r,e) = 0$ and
$\cte(r,e') \neq 0$ where $e'$ is a forward event.
We first note that $e \neq e'$, since $\cte(r,e) = 0$ and $\cte(r,e') \neq 0$.
By WF, there is a rooted path $s$ from some irreversible $I$ to $P$.
% Using PL, the path $s$ from $I$ to $P$ can be taken to be forward-only.
\begin{enumerate}
\item
Suppose first that $\cte(r,e') > 0$.
Since $\cte(st_0r,e) > 0$ and $\cte(st_0r,e') > 0$ we do not have $e \cf e'$.
Furthermore, if $e' < e$ then we must have $\cte(s,e') > 0$,
so that $\cte(st_0r,e') > 1$, contradicting NRE.
Then the result follows by polychotomy.

\item
Now suppose that $\cte(r,e') < 0$.
By Proposition~\ref{prop:regeqzero} we must have $\cte(s,e') > 0$.
We deduce that $e \not< e'$.
Since $\cte(st_0,e) > 0$ and $\cte(st_0,e') > 0$ we do not have $e \cf e'$.
Furthermore $\cte(st_0r,e) > 0$ and $\cte(st_0r,e') = 0$ (since $\cte(st_0,e') = 1$ combining $\cte(st_0,e') > 0$ shown above and NRE).
Hence $e' \not< e$.
By polychotomy, $e \coind e'$.
\qedhere
\end{enumerate}
\end{proof}
%\todo{NB We could restate as two cases depending on whether
%$\cte(r,e') > 0$ or $\cte(r,e') < 0$.
%This would help with the proof of Proposition~\ref{prop:CS CL coind <},
%where we have to reach into the proof.}


%\begin{restatable}{proposition}{CSCLcoind}\label{prop:CS CL coind <}
\begin{proposition}\label{prop:CS CL coind <}
Suppose that an LTSI $\mc L$ is pre-reversible.
Then
\begin{enumerate}
\item
% $\mc L$ satisfies CS$\ci$ iff $\mc L$ satisfies CS$_<$.
$\mc L$ satisfies CS$_<$.
%\todo{Mismatch in definitions as to whether $\cte(r,e') < 0$ is allowed, but proof works.}
\item
$\mc L$ satisfies CL$\ci$ iff $\mc L$ satisfies CL$_<$.
\end{enumerate}
\end{proposition}
%\end{restatable}
\begin{proof}
\begin{enumerate}
\item
% $\mc L$ satisfies CS$\ci$ iff $\mc L$ satisfies CS$_<$.
% Suppose that CS$\ci$ holds.
We know CS$\ci$ holds by Theorem~\ref{thm:CS coind}.
Assume that $t_0:P \tran a Q$, $e = [t_0]$, $r:Q \ptran \rho R$, $\cte(r,e) = 0$ and $t_0\op:S \tran a R$
with $t_0 \sqeqt t_0\op$.
Take any forward $e'$ such that $\cte(r,e') > 0$.
By Lemma~\ref{lem:coind <} we know that exactly one of $e \coind e'$ or $e < e'$ holds.
By CS$\ci$ we have $e \coind e'$, and therefore $e \not< e'$ as required.

% Conversely, suppose that CS$_<$ holds.
% We know CS$\ci$ holds by Theorem~\ref{thm:CS coind} (without the use of NRE).
\item
% $\mc L$ satisfies CL$\ci$ iff $\mc L$ satisfies CL$_<$.
  Suppose that CL$\ci$ holds.
Assume that
$P \tran a Q$, $e = [t_0]$, $r:Q \ptran \rho R$ and $\cte(r,e) = 0$ and
$e \not < e'$ for all 
forward $e'$ such that $\cte(r,e') > 0$.
Let event $e'$ be such that $\cte(r,e') > 0$.
Suppose first that $e'$ is forward.
By assumption $e \not < e'$.
So by Lemma~\ref{lem:coind <}(\ref{item:greater}) we obtain $e \coind e'$.
Suppose instead that $e'$ is reverse, so that $\rev {e'}$ is forward,
and $\cte(r,\rev{e'}) < 0$.
By Lemma~\ref{lem:coind <}(\ref{item:less}) we obtain $e \coind \rev{e'}$,
and hence $e \coind e'$ using Lemma~\ref{lem:coind rev}.
We deduce that $e \coind e'$ for all $e'$ such that $\cte(r,e') > 0$.
Hence by CL$\ci$ we have
$t_0\op:S \tran a R$ with $t_0 \sqeqt t_0\op$.

Conversely, suppose that CL$_<$ holds.
Assume that
$P \tran a Q$, $e = [t_0]$, $r:Q \ptran \rho R$ and $\cte(r,e) = 0$ and
$e \coind e'$ for all 
$e'$ such that $\cte(r,e') > 0$.
By Lemma~\ref{lem:coind <}(\ref{item:greater}) we know that $e \not < e'$ for all forward $e'$ such that $\cte(r,e') > 0$.
Hence by CL$_<$ we have
$t_0\op:S \tran a R$ with $t_0 \sqeqt t_0\op$.
\qedhere
\end{enumerate}
\end{proof}


%% Property RED below is also related to NRE and polychotomy.
%% % \begin{definition}\label{def:ED RED}
%% \begin{definition}\label{def:RED}
%% % An LTSI satisfies \emph{event determinism (ED)} if whenever
%% % $t,t'$ are forward coinitial transitions and $t \sqeqt t'$ then $t = t'$.
%% \todo{Remove?}
%% An LTSI satisfies \textbf{Reverse Event Determinism (RED)} if whenever
%% $t,t'$ are backward coinitial transitions and $t \sqeqt t'$ then $t = t'$.
%% \end{definition}
%% % \begin{proposition}\label{prop:RED}
%% % If an LTSI satisfies SP, BTI, WF, PCI, ED then it satisfies RED.
%% % \end{proposition}
%% % \begin{proof}
%% % Immediate using BTI and SP.
%% % \end{proof}
%% % \begin{proposition}\label{prop:RED}
%% % If an LTSI satisfies SP, BTI, WF, PCI, NRE then it satisfies RED.
%% % \end{proposition}
%% % \begin{proof}
%% % Suppose that $t:P \tran{\rev a} Q$, $u:P \tran{\rev a} Q'$
%% % are backward coinitial transitions and $t \sqeqt u$.
%% % Suppose for a contradiction that $t \neq u$.
%% % By BTI we have $t \ind u$.
%% % We can use SP to complete a diamond with transitions $t' \sqeqt t$,
%% % $u' \sqeqt u$.
%% % All four transitions belong to the same reverse event.
%% % We get a forward path containing two occurrences of the same event,
%% % contradicting NRE.
%% % \end{proof}
%% %
%% %\begin{restatable}{proposition}{NRERED}\label{prop:NRE RED poly}
%% \begin{proposition}\label{prop:NRE RED poly}
%% \todo{Remove}
%% If an LTSI $\mc L$ is pre-reversible
%% then the following are equivalent:
%% \begin{enumerate}
%% \item $\mc L$ satisfies NRE;  
%% \item  $\mc L$ satisfies RED;
%% \item independence $\coind$ is irreflexive on events; and  
%% \item polychotomy holds.
%% \end{enumerate}
%% \end{proposition}
%% %\end{restatable}
%% \begin{proof}
%% Suppose that NRE holds; we show RED.
%% Suppose that $t:P \tran{\rev a} Q$, $u:P \tran{\rev a} Q'$
%% are backward coinitial transitions and $t \sqeqt u$.
%% Suppose for a contradiction that $t \neq u$.
%% By BTI we have $t \ind u$.
%% We can use SP and PCI to complete a diamond with transitions $t' \sqeqt t$,
%% $u' \sqeqt u$.
%% All four transitions belong to the same reverse event.
%% We get a forward path containing two occurrences of the same event,
%% contradicting NRE.

%% Suppose that RED holds; we show that $\coind$ is irreflexive.
%% Suppose for a contradiction that $e \coind e$ \todo{for some event $e$}.
%% Then there are coinitial transitions $t,u \in e$ such that $t \ind u$.
%% We can use SP to complete a square with $t' \sqeqt t$ and $u' \sqeqt u$.
%% This square is non-degenerate by Lemma~\ref{lem:non-degenerate}.
%% All transitions in the square belong to the same event.
%% Hence there are two distinct reverse coinitial transitions from the same event,
%% contradicting RED.

%% % Suppose that NRE holds.
%% % Now suppose for a contradiction that we have an event $e$ such that $e \coind e$.
%% % This means that there are coinitial transitions $t,u \in e$
%% % such that $t \ind u$.
%% % We can use SP to complete a square with $t' \sqeqt t$ and $u' \sqeqt u$.
%% % We get a forward path containing two occurrences of the same event,
%% % contradicting NRE.

%% Suppose that $\coind$ is irreflexive; we show NRE.
%% Let $r$ be a rooted path from $I$ to $R$, and suppose for a contradiction that
%% $\cte(r,e) > 1$.
%% Using PL we can obtain a forward-only path $r'$ from $I$ to $R$ with $r \ceqt r'$.
%% By Lemma~\ref{lemma:cccount}, $\cte(r',e) > 1$.
%% Suppose $r'$ contains
%% $t:P \tran a Q$ followed later by $t':P' \tran a Q'$ where $t, t' \in e$.
%% Let $r''$ be the portion of $r'$ from $Q$ to $P'$.
%% By Lemma~\ref{lem:ladder} there is a path $s$ from $Q$ to $Q'$ such that for all $u$ in $s$
%% % we have $t \sqeqt t'' \ind u' \sqeqt u$ (some $t'',u'$).
%% we have $[t] \coind [u]$.
%% By CC, $s \ceqt r''t'$.
%% By Lemma~\ref{lemma:cccount}, $\cte(s,[t']) > 0$, since $r''$ is forward-only.
%% Hence there is $u$ in $s$ such that $u \sqeqt t' \sqeqt t$.
%% % But then $t \sqeqt t'' \ind u' \sqeqt t$,
%% % where $t'',u'$ are as above.
%% But then $[t] \coind [u] = [t]$,
%% contradicting our assumption that $\coind$ is irreflexive.

%% Suppose that NRE holds.
%% Then polychotomy holds by Proposition~\ref{prop:poly}.

%% Suppose that polychotomy holds.
%% Then since $e = e'$ and $e \coind e'$ are mutually exclusive,
%% $\coind$ must be irreflexive.
%% \end{proof}

%% We now give a criterion to show NRE.
%% %\begin{restatable}{proposition}{CIRENRE}\label{prop:CIRE NRE}
%% \begin{proposition}\label{prop:CIRE NRE}
%% \todo{Remove}
%% Suppose that a pre-reversible LTSI satisfies CIRE.
%% Then it also satisfies NRE.
%% \end{proposition}
%% %\end{restatable}
%% % \begin{proof}[Superseded by alternative proof]
%% % Let $r$ be a rooted path from $I$ to $R$, and suppose for a contradiction that
%% % $\cte(r,e) > 1$.
%% % Using PL we can obtain a forward-only path $r'$ from $I$ to $R$ with $r \ceqt r'$.
%% % By Lemma~\ref{lemma:cccount}, $\cte(r',e) > 1$.
%% % Suppose $r'$ contains
%% % $t:P \tran a Q$ followed later by $t':P' \tran a Q'$ where $t, t' \in e$.
%% % Let $r''$ be the portion of $r'$ from $Q$ to $P'$.
%% % By Lemma~\ref{lem:ladder} there is a path $s$ from $Q$ to $Q'$ such that for
%% % all $u$ in $s$
%% % % we have $t \sqeqt t'' \ind u' \sqeqt u$ (some $t'',u'$).
%% % we have $[t] \coind [u]$.
%% % By CC, $s \ceqt r''t'$.
%% % By Lemma~\ref{lemma:cccount}, $\cte(s,[t']) > 0$, since $r''$ is forward-only.
%% % Hence there is $u$ in $s$ such that $u \sqeqt t' \sqeqt t$.
%% % % But then $t \sqeqt t'' \ind u' \sqeqt t$,
%% % % where $t'',u'$ are as above.
%% % But then $[t] \coind [u] = [t]$.
%% % Hence $t \ind t$ by CIRE, contradicting $\ind$ being
%% % irreflexive.
%% % \end{proof}
%% \begin{proof}% [Alternative proof]
%% By Proposition~\ref{prop:NRE RED poly} it is enough to show that $\coind$ is irreflexive.
%% Suppose that $e \coind e$ for some event $e$.
%% Take any transition $t \in e$.
%% Then $t$ is coinitial with itself, and so by CIRE we have $t \ind t$,
%% which contradicts irreflexivity of $\ind$.
%% \end{proof}

%% The opposite implication does not hold.
%% \begin{example}\label{ex:NREnotCL revisited}
%% \todo{
%% % \todo{May have to be omitted for space reasons,
%% % but this example is less pathological
%% % than the ones with repeated events.}
%% Consider the pre-reversible LTSI of Example~\ref{ex:IC CLi} shown in
%Figure~\ref{fig:IC CLi}.
%% There are three events, labelled $a,b,c$, which are all independent of each other.
%% We see that NRE holds (hence CS$_<$ holds as well) but, as previously observed, not CIRE.
%% We previously observed that CL$\ci$ fails;
%% so does CL$_<$ in a similar fashion: consider $P \tran a Q \tran b R$,
%% where $a$ cannot be reversed at $R$, even though $a \not< b$.\finex
%% }
%% \end{example}


% \begin{example}\label{ex:LED}
% Consider the LTSI in Figure~\ref{fig:LED1}.
% % Figure environment removed
% Independence is given by closing under BTI and PCI.
% There are just two events, labelled with $a$ and $b$ respectively,
% which are concurrent.
% We see that axioms SP, BTI, WF, PCI, CIRE hold.
% Therefore also NRE and RED hold.
% However ED fails.
% \end{example}
% \begin{remark}
% Note that CL does not necessarily hold if the initial transition is backward rather than forward.
% Consider the LTSI in Example~\ref{ex:LED} where ED fails.
% Add a $c$ transition starting at $P$.
% Then if we perform $c$ followed by $r = \rev b b$,
% we have $\cte(r,b) = 0$,
% but we get to state $Q$ where $c$ is impossible.
% \todo{If we assume ED can we prove CL for reverse transitions?}
% \end{remark}

%\subsection{FCL$_<$ - to be omitted}
%\input {FCL.tex}

%\subsection{Comparing the different forms of CS/CL}\label{sub:comparing}

%% \todo{Table~\ref{tab:implications} can now be dropped as all open implications resolved.}
%% \begin{table}[t!]
%% \begin{center}
%% \begin{tabular}{|c|c|p{0.5\linewidth}|}
%% \hline
%% From & To & Comments \\
%% \hline
%% LG+IEC & none & Ex.~\ref{ex:LG+IEC} \\
%% CLG & none & Ex.~\ref{ex:prerev not CSi} \\
%% LG & none & Ex.~\ref{ex:LG} \\
%% IRE+IEC & none & Ex.~\ref{ex:IRE+IEC} \\
%% IC+CIRE & none & Ex.~\ref{ex:IC+CIRE} \\
%% CIRE+IEC & none & Ex.~\ref{ex:IRE+IEC} and Ex.~\ref{ex:IC+CIRE} \\
%% IC+CL$\ci$ & none & Ex.~\ref{ex:halfcube} \\
%% IRE & none & Ex.~\ref{ex:IRE1} or Ex.~\ref{ex:IRE2} \\
%% IEC+CL$\ci$ & none & Ex.~\ref{ex:halfcube} and Ex.~\ref{ex:IRE+IEC} \\
%% IC & none & Ex.~\ref{ex:IC} \\
%% CS$\indt$ & \todo{none} & Ex.~\ref{ex:prerev not CL}, Ex.~\ref{ex:CSi+RPI CLi} and \todo{Ex.~\ref{ex:halfcube mod}} \\
%% CIRE & none & Ex.~\ref{ex:prerev not CL}, Ex.~\ref{ex:CSi+RPI CLi} and Ex.~\ref{ex:IC+CIRE}  \\
%% CL$\indt$ & none & Ex.~\ref{ex:CSi+RPI CLi} and Ex.~\ref{ex:IC CLi} \todo{not needed: Ex.~\ref{ex:IRE1} or Ex.~\ref{ex:IRE2}} \\
%% CL$\ci$ & none & Ex.~\ref{ex:prerev not CL}, Ex.~\ref{ex:CSi+RPI CLi} and Ex.~\ref{ex:halfcube} \\
%% IEC & none & Ex.~\ref{ex:IC} and Ex.~\ref{ex:IRE+IEC} \\
%% \todo{ECh} & none & Ex.~\ref{ex:prerev not CL} and Ex.~\ref{ex:IC} \\
%% \hdashline
%% EIT & \todo{none} & Ex.~\ref{ex:prerev not CL}, Ex.~\ref{ex:CSi+RPI CLi} and Ex.~\ref{ex:halfcube} \\
%% CS$\indt$+RPI & CL$\indt$ & Ex.~\ref{ex:CSi+RPI CLi}\\
%% IRE+RPI & none & Ex.~\ref{ex:LG} and Ex.~\ref{ex:IRE+IEC}  \\
%% RPI & (none) & could use Ex.~\ref{ex:IC} closed under RPI, and Ex.~\ref{ex:CSi+RPI CLi}\\
%% \hline
%% \end{tabular}
%% \end{center}
%% \caption{Open implications - for our information.}
%% \label{tab:implications}  
%% \end{table}
\subsection{Implications between the different formalisations of CS/CL}\label{sub:comparing}

% Figure environment removed
%[Iain: no, 
%  and~\ref{ex:IC+CIRE}.

We have introduced three different formalisations of causal safety and liveness.  The implications between them, assuming pre-reversibility holds, %the different versions
are shown in Figure~\ref{fig:simpleCSCL}.

As can be seen in Table~\ref{t:list}, only two causal safety properties, namely CS$\ci$ and CS$_<$, hold for
pre-reversible LTSIs.  The causal liveness versions of these properties, namely CL$\ci$ and CL$_<$, additionally require BFCIRE. %CIRE. 
%We also know that its restricted version
Actually, BFCIRE is equivalent to both CL$\ci$ and CL$_<$. The last two properties,
CS$\indt$ and CL$\indt$, which are defined over general independence of transitions, require  IRE. No other implications hold beyond those shown. Counterexamples for lack of other implications
%between some causal properties
in Figure~\ref{fig:simpleCSCL} are pointed to in Figure~\ref{fig:diagprerev1}.

We postpone discussion of which particular version of CS or CL is most relevant in a specific setting until Section~\ref{subsec:comparison}, after we have introduced some structural axioms to better relate them.

\Comment{
\todo{We have introduced three different formalisations of causal safety and liveness.  The implications between the different versions are shown in Figure~\ref{fig:simpleCSCL}, assuming pre-reversibility holds.
The axioms IRE and CIRE which can be used to show CS$\indt$ and the three forms of CL are also shown.
We postpone discussion of which particular version of CS or CL is most relevant in different settings until Section~\ref{subsec:comparison}, after we have introduced some structural axioms.}

%\todo{IU: I am not sure if it is the right place for this subsection. It could appear after section 5}
As can be seen from Table~\ref{t:list}, only two causal safety properties, namely CS$\ci$ and CS$_<$, hold for
pre-reversible LTSI. \il{Instead, we required CIRE (or BFCIRE) to prove CL$\ci$ and CL$_<$, and IRE to prove CS$\indt$ and CL$\indt$.

These are needed: we show that pre-reversible is not enough for
CS$\indt$ in Example~\ref{ex:prerev not CSi}, for CL$\indt$ in
Example~\ref{ex:prerev not CL}, and for CL$\ci$ in Example~\ref{ex:IC
  CLi}. Thanks to Proposition~\ref{prop:CS CL coind <} it is not
enough for CL$_<$ either (indeed in Example~\ref{ex:IC CLi}, $P \tran
a Q \tran b R$, $a \not < b$ but $a$ cannot be reversed at $R$).

Also, CIRE holds in Example~\ref{ex:prerev not CL}, hence CIRE would
not be enough for CL$\indt$. Similarly, CIRE would not be enough for CS$\indt$
in view of Example~\ref{ex:prerev not CSi}. Indeed, CIRE trivially holds since all pairs of coinitial transitions are independent.

Hence, we can divide the notions in three layers, which require stronger and stronger conditions to be proved (remember that we stick to pre-reversible LTSIs, which is the most basic setting where these notions can be defined):

CS$\ci$, CS$_<$ $\nRightarrow$ CL$\ci$, CL$_<$ $\nRightarrow$ CS$\indt$, CL$\indt$

Note that this does not mean that notions which require stronger
conditions imply the less demanding, however of course notions which
require weaker conditions cannot imply more demanding ones. However,
trivially, all the notions imply CS$\ci$ and CS$_<$. Also, thanks to
Proposition~\ref{prop:CS CL coind <} CL$\ci$ and CL$_<$ are
equivalent.  At the contrary, CS$\indt$ and CL$\indt$ are not
comparable. Indeed, in Example~\ref{ex:prerev not CSi} CL$\indt$ holds but
CS$\indt$ fails. Dually, in Example~\ref{ex:prerev not CL} CS$\indt$
holds but CL$\indt$ fails. Hence we can refine the comparison above into:

CS$\ci$, CS$_<$ $\nRightarrow$ CL$\ci$ $\Leftrightarrow$ CL$_<$ $\nRightarrow$ CS$\indt$ $\nLeftrightarrow$ CL$\indt$

which is graphically represented in Figure~\ref{fig:simpleCSCL}.
% Figure environment removed
%[Iain: no, 
%  and~\ref{ex:IC+CIRE}.
}

\todo{TO FINISH}




%By remembering that IRE implies CIRE

%% Also, one can notice that, \il{as discussed in Example~\ref{ex:IC CLi},} in Figure~\ref{fig:IC CLi} CS$_<$ holds
%% but not CL$_<$, CL$\ci$. By adding $(P,a,Q) \ind (Q,b,R)$ we get that
%% CL fails as well. By taking the same diagram with reversed arrows and
%% not adding $(P,a,Q) \ind (Q,b,R)$ we get a diagram where CS$_<$ holds
%% but CS$\indt$ fails since we can undo $b$ after $ac$.
%% \todo{The inverse diagram is not pre-reversible since the last $c$ and $b$ need to form a square and do not.}

%% Hence we can rewrite the sequence above as:

%% CS$\ci$ $\nRightarrow$ CS$_<$ $\nRightarrow$ CL$_<$, CL$\ci$ $\leq$ CS$\indt$, CL$\indt$

%CS and CL are not derivable from CC;
%we give an example LTSI which satisfies CC but not CS and not CL.
%\begin{example}[]\label{}
%Consider the LTS in Figure~\ref{fig:repeated1}.
%% % Figure environment removed

%% \todo{IVAN: drop next Example? Or move to extra?}
%% 	\iu{IU: In the LTSI in Figure~\ref{fig:noED}, independence is given by closing under BTI and PCI. 
%% There are just two events $a$ and $b$, and the property Event Determinism (that we do not consider here) 
%% fails. Also, IRE fails.  I think that CS$_<$, CL$_<$, CS$_{ci}$ and CL$_{ci}$ hold. 
%% However, CS$\indt$ fails as the initial $a$ and the following $b$s are not independent \todo{IVAN: not true any more with new def of CS$\indt$}. 
%% 	But CL$\indt$ holds.\vspace{0.5cm}
%% }
%% \todo{Iain: The LTSI in Figure~\ref{fig:noED} satisfies CIRE+IEC but not IRE.
%% I think it has the same properties as Example~\ref{ex:CLG CSi}.}

\todo{I would merge the paragraph below with the one at the end of the next section. All in all, it seems the general idea is: "Use whatever it is easiest to define in your setting"}        
\iu{IU: What follows is an initial draft of a discussion about which of the three pairs 
	a user might wish to adopt for her reversible formalism.


One may ask which of the three versions of causal safety and liveness properties should be adopted for
a given reversible formalism. This depends mainly on whether or not a suitable independence relation 
can be easily defined, and then whether or not this relation is for coinitial transitions. The analysis 
of our case studies and other reversible formalisms shows that, regardless of independence, 
the notions of events and a causal ordering on events are universal. In such settings, 
it is natural to assume that events do not occur
multiple times during individual computations, namely NRE holds. We are not aware of any formalism 
for concurrent computation where NRE fails. We have shown that pre-reversible LTSIs for formalisms
with coinitial independence, where NRE also holds,
satisfy CS$\ci$ and CL$\ci$ \todo{The latter requires CIRE unless I missed something}. We conclude that in such settings, CS$\ci$ and CL$\ci$ are 
the weakest desirable properties, as we do not need any further, potentially useful axioms to \il{hold}. 
If we can additionally guarantee CIRE, then the finer properties CS$_<$ and CL$_<$ will
be more desirable. 

When it is more natural to define independence on general transitions, as for Petri or occurrence
nets, then IRE is a desirable property to have. Consequently, we can use CS$\indt$ and CL$\indt$, which hold for
pre-reversible LTSIs with IRE. Alternatively, we can work purely with events and use CS$_<$ and CL$_<$.

If an independence relation is not easily defined, but there are well understood notions of events
and \il{causal ordering} on events, then CS$_<$ and CL$_<$ are probably the preferred properties.}
}


\section{Structured notions of independence}\label{sec:coinitial}
In this section we consider two structured notions of independence, namely independence defined on coinitial transitions only and independence determined by labels only.
To this end, we introduce `structural axioms' in Definitions~\ref{def:coinitial LTSI},~\ref{def:CLG} and~\ref{def:LG}.  These have a different status from the axioms already introduced: rather than expressing fundamental properties that are desirable in LTSIs,
they are properties that hold in various reversible formalisms %calculi
(as we shall see in Section~\ref{sec:casestudies}), are easy to verify,
and can be used to derive other
%particular
axioms in a generic fashion.

\subsection{Coinitial independence}
In this section we discuss coinitial LTSIs, defined as follows,
and their relationship with LTSIs in general.

\begin{definition}\label{def:coinitial LTSI}
  {\bf Independence is coinitial (IC)}:
  for all transitions $t,u$,
if $t \ind u$ then $t$ and $u$ are coinitial.
\end{definition}

We say that an LTSI $\mc L$ is coinitial if it satisfies IC.
We also say that its independence relation $\ind$ is coinitial.

Coinitial independence is of interest since in many cases it is easier
to define independence only on coinitial transitions. Indeed,
coinitial independence arises, e.g., from the notions of concurrency
in~\cite[Definition 7]{DK04} for RCCS and in~\cite[Definition 5]{LaneseNPV18} for Core Erlang.

The next example satisfies IC and all and only the properties in Figure~\ref{fig:diagprerev1} implied by it. In particular, it shows that IC does not imply CL$\indt$, CL$\ci$, or
CS$\indt$ (this last follows from Proposition~\ref{prop:CSi CLci}).
\begin{example}\label{ex:IC}
Consider the LTSI in Figure~\ref{fig:IC}.
% Figure environment removed
All independence is coinitial as generated by BTI and PCI,
and the LTSI is pre-reversible.
%[Would be good to use the tool to check that this is pre-reversible.]
%\il{[IL: Done, it is pre-reversible, see exICwithtool.png]}
There are three events, which we denote by $e_a,e_b,e_c$,
with labels $a,b,c$, respectively. 
%Moreover, there are reverse events,
%for example the event $\rev{e_b}$ for the transitions with label $b$.
% Note that $e_a \coind e_b$, $e_b \coind e_c$ and $e_c \coind e_a$.
CL$\indt$ fails:
let $t: P \tran a Q$ and let $r$ from $Q$ to $R$ be $\rev b b$ (dashed transitions).
We have $\cte(r,e_b) = 0$;
however $a$ cannot be reversed at $R$, as CL$\indt$ would yield.
Also CS$\indt$ fails:
let $t: P \tran a Q$ and let $r'$ be $\rev c\, \rev b$ from $Q$ to $S$ (bold transitions).
After $r'$, $\rev a$ is possible.
However $\rev t$ is not independent with the $\rev b$ transition,
as CS$\indt$ would yield.
%EIT fails: start at the state reached by performing the leftmost $b$ transition.
%We can reach $R$ by $\rev b cab$ and also by $ac$.
%However we cannot reach $R$ by performing $ca$, as EIT would yield.
Also CL$\ci$ fails:
let $t_0$ be the $a$ following the leftmost $b$, and let $r''$ be the $c$
transition with target $R$.  We have $e_a \coind e_c$.
However $a$ cannot be reversed at $R$, as CL$\ci$ would yield.
\finex
\end{example}

Coinitial independence is inconsistent with the axiom IRE,
showing that IRE is only appropriate for the setting of general,
rather than coinitial independence:
\begin{proposition}\label{prop:IC IRE}
Let a pre-reversible LTSI have a non-empty independence relation,
and satisfy IC.  Then IRE does not hold.
\end{proposition}
\begin{proof}
Suppose for a contradiction that IRE holds.
Since the independence relation is non-empty and IC holds,
we have $t \ind u$ with $t,u$ coinitial.
By SP and PCI we can complete a diamond with $t'\sqeqt t$, $u' \sqeqt u$.
Since $t' \sqeqt t \ind u$ we deduce by IRE that $t' \ind u$.
However $t'$ and $u$ are not coinitial,
contradicting IC.
\end{proof}

We define
a mapping~$\mathrel{c}$ restricting general independence to coinitial transitions and
a mapping~$\mathrel{g}$ extending independence along events.
% so to satisfy IRE and IEC.
\begin{definition}\label{def:gen coinit}
  Given an LTSI %$(\Proc,\Lab,\tran{},\ind)$
  $\mc L$, define
$t \mathrel{g(\ind)} u$ iff \mbox{$t \sqeqt t' \ind u' \sqeqt u$}
for some $t',u'$.
Furthermore, define $t \mathrel{c(\ind)} u$
iff $t \ind u$ and $t,u$ are coinitial.
\end{definition}

  We extend $\mathrel{c}$ and $\mathrel{g}$ to LTSIs $(\Proc,\Lab,\tran{},\ind)$:
  they behave as the identity of the first three components, and as expected on the fourth. Similarly, we write $\mathrel{c(\sqeqt)}$ and $\mathrel{g(\sqeqt)}$ for the equivalence relations in $\mathrel{c(\mc L)}$ and $\mathrel{g(\mc L)}$, respectively.
  
We now show that $c$ and $g$ play well with events.
\begin{lemma}
 Given an LTSI $\mc L$, ${\sqeqt} = {c(\sqeqt)}$.
\end{lemma}
\begin{proof}
Follows by noticing that the definition of event only exploits independence on coinitial transitions.
\end{proof}

\begin{lemma}
 Given an LTSI $\mc L$, $t \sqeqt u$ implies $t \mathrel{g(\sqeqt)} u$.
\end{lemma}
\begin{proof}
By definition of $\sqeqt$, noticing that ${\ind} \subseteq {g(\ind)}$.
\end{proof}

\begin{lemma}
 Given a pre-reversible LTSI $\mc L$, $t \mathrel{g(\sqeqt)} u$ implies $t \sqeqt u$.
\end{lemma}
\begin{proof}
By definition of $\sqeqt$, we have $t \, g(\sqeqt) \, u$ if there is a chain of commuting squares connecting $t$ and $u$. Thanks to ID (which holds in pre-reversible LTSIs) all such squares are commuting squares in $\mc L$, hence $t \sqeqt u$ as desired.
\end{proof}

We can now study the impact of $c$ and $g$ on the axioms satisfied by the LTSI to which they are applied.
%
%\begin{restatable}{proposition}{gencinit}\label{prop:gen coinit}
\begin{proposition}\label{prop:gen coinit}
Let $\mc L = (\Proc,\Lab,\tran{},\ind)$ be a pre-reversible LTSI. %Then
\begin{enumerate}
\item
if $\mc L$ is coinitial and satisfies CIRE then $c(g(\ind)) = {\ind}$;
\item
if $\mc L$ satisfies IRE and IEC then $g(c(\ind)) = {\ind}$;
\item\label{item:gprop}
If $\mc L$ is coinitial and satisfies CIRE then
$g(\mc L)$
is a pre-reversible LTSI and satisfies IRE and IEC.
\item
if $\mc L$ satisfies IRE then
$c(\mc L)$
is a pre-reversible coinitial LTSI and satisfies CIRE.
\end{enumerate}
\end{proposition}
%\end{restatable}
\begin{proof}
\begin{enumerate}
\item
Clearly ${\ind} \subseteq c(g(\ind))$.
For the converse,
suppose $t \sqeqt t' \ind u' \sqeqt u$ and $t',u'$ are coinitial and $t,u$ are coinitial.
Then $t \ind u$ by CIRE.
\item
Suppose $t \ind u$.
By IEC we have $t \sqeqt t' \ind u' \sqeqt u$ with $t',u'$ coinitial.
Hence $t \mathrel{g(c(\ind))} u$.
Conversely, suppose $t \mathrel{g(c(\ind))} u$.
Then $t \ind u$ by IRE.
\item
%Suppose that $t \ind u$.  Then by SP there is a diamond with $t',u'$ and by PCI we have $\rev t \ind u'$, $\rev {u'} \ind \rev {t'}$ and $t' \ind \rev{u'}$.
%Furthermore $t \sqeqt t'$ and $u \sqeqt u'$.
Suppose $t \mathrel{g(\ind)} u$ and $t,u$ are coinitial.
Then by CIRE $t \ind u$.  So we can use SP for $\ind$ to complete the diamond.
Hence SP holds for $\mc L'$.

Clearly PCI holds for
$g(\mc L)$
since $g(\ind)$ and $\ind$ agree on coinitial transitions by CIRE.

For IRE, suppose $t' \sqeqt t \mathrel{g(\ind)} u \sqeqt u'$.  Then clearly $t' \mathrel{g(\ind)} u'$.

Finally, for IEC suppose $t \mathrel{g(\ind)} u$.  Then
$t \sqeqt t' \ind u' \sqeqt u$ with $t',u'$ coinitial, which is exactly what is needed
for IEC.
\item
Immediate.
\qedhere
\end{enumerate}
\end{proof}
Thanks to Proposition~\ref{prop:gen coinit}, we can extend
a coinitial pre-reversible LTSI satisfying CIRE in a canonical way to
a pre-reversible LTSI satisfying IRE and IEC.

Note that $g(\mc L)$ satisfies IRE (and hence ECh) by construction, since $t \mathrel{g(\ind)} u \sqeqt t'$ implies $t \mathrel{g(\ind)} t'$. Conditions in Proposition~\ref{prop:gen coinit}, item~(\ref{item:gprop}) are only needed for the other properties.

%will hold for the LTSI \il{$g(\mc L)$} generated from coinitial $\mc
%L$ using $g$ in Proposition~\ref{prop:gen coinit}, since


\subsection{Label-generated independence}
In some reversible calculi (such as RCCS) independence of coinitial transitions is defined purely
by reference to the labels.

\begin{definition}\label{def:CLG}
%Let $\mc L = (\Proc,\Lab,\tran{},\ind)$ be an LTSI.
  %Then $\mc L$ is
  {\bf Coinitial label-generated (CLG)}: if there is an irreflexive binary relation $I$ on $\Lab$ such that for any
transitions $t:P \tran\alpha Q$ and $u:P \tran\beta R$
we have $t \ind u$ iff $t$ and $u$ are coinitial and $I(a,b)$, where $a$ and $b$ are the underlying labels
$a = \und\alpha$, $b = \und\beta$.
\end{definition}

If this is the case then %it is a simple matter to verify
the axioms IC, PCI and CIRE hold by construction.
\begin{proposition}\label{prop:CLG}
%[replaces Proposition~\ref{prop:underlying}]
If an LTSI is CLG then it satisfies IC, PCI and CIRE.
\end{proposition}
\begin{proof}
Straightforward, noting for PCI and CIRE that labels on opposite sides of a diamond of transitions must be equal.
\end{proof}


%% \begin{proposition}\label{prop:underlying}
%%   %  Let $\mc L = (\Proc,\Lab,\tran{},\ind)$ be
%% \todo{Replace with the one below?}
%% \il{Consider} a coinitial LTSI.
%% Suppose that $I$ is an irreflexive binary relation on $\Lab$ such that for any
%% coinitial transitions $t:P \tran\alpha Q$ and $u:P \tran\beta R$
%% we have $t \ind u$ iff $I(a,b)$, where $a$ and $b$ are the underlying labels
%% $a = \und\alpha$, $b = \und\beta$.
%% Then $\mc L$ satisfies PCI and CIRE.
%% \end{proposition}
%% \begin{proof}
%% Straightforward, noting that labels on opposite sides of a diamond of transitions must be equal.
%% \end{proof}
Note that $I$ must be irreflexive, since $\ind$ is irreflexive by definition. Even more, we already have seen that for a pre-reversible LTSI there cannot be independent coinitial transitions $t$, $u$ with the same underlying label (as a consequence of Lemma~\ref{lemma:revnotind} and BLD).


\begin{definition}\label{def:LG}
  {\bf Label-generated (LG)}: if there is an irreflexive binary relation $I$ on $\Lab$ such that for any
transitions $t:P \tran\alpha Q$ and $u:R \tran\beta S$
we have $t \ind u$ iff $I(a,b)$, where $a$ and $b$ are the underlying labels
$a = \und\alpha$, $b = \und\beta$.
\end{definition}
\begin{proposition}\label{prop:LG}
If an LTSI is LG then it satisfies PCI, IRE and RPI.
\end{proposition}
\begin{proof}
Straightforward.
\end{proof}
Note that LG does not imply IEC, in view of the following example.
\begin{example}\label{ex:LG}
Consider the LTSI with two transitions $t:P \tran a Q$ and $u:R \tran b S$, where all states are distinct (as in Example~\ref{ex:IRE1}) and $a \neq b$.
Let independence be generated by the relation $I = \{(a,b)\}$.
Then LG holds, but not IEC, since $t \ind u$ but not $[t] \coind [u]$.\finex
\end{example}

However, LG is compatible with IEC, in view of the following example.
\begin{example}\label{ex:LG+IEC}
Let $t:P \tran a Q$, $u:P \tran b R$,
$u':Q \tran b S$, $t':R \tran a S$,
where all states are distinct and $a \neq b$.
Let independence be generated by the relation $I = \{(a,b)\}$.
Then both LG and IEC hold.
% Here we have two forward events, labelled with $a$ and $b$ respectively.
% Axioms SP, BTI, WF, PCI, IRE and IEC hold.
However IC fails.
\finex
\end{example}

All the axioms and properties we have considered previously are closed under
%taking
disjoint unions of LTSIs, defined as follows.
\begin{definition}[Disjoint union of LTSIs]
Take two LTSIs $(\Proc_1,\Lab_1,\tran{}_1,\ind_1)$ and $(\Proc_2,\Lab_2,\tran{}_2,\ind_2)$. Their disjoint union is $(\Proc_1 \cup \Proc_2,\Lab_1 \cup \Lab_2,\tran{}_1\cup\tran{}_2,\ind_1\cup\ind_2)$ provided that $\Proc_1 \cap \Proc_2 = \emptyset$, undefined otherwise.
\end{definition}
However LG and CLG are not necessarily closed under disjoint unions of LTSIs,
in view of the following examples.
% \todo{examples are useful for the diagram of implications} 
\begin{example}\label{ex:IRE+IEC}
Take the disjoint union of the LTSI of Example~\ref{ex:LG+IEC}
together with a further transition
$T \tran a U$ with an empty generator relation
(this component satisfies LG).
Then LG fails; however IEC and IRE still hold.
\finex
\end{example}
\begin{example}\label{ex:IC+CIRE}
Take the disjoint union of the LTSI of
%Example~\ref{ex:CLG CSi}
Example~\ref{ex:prerev not CSi}
(which satisfies CLG)
together with further transitions $T \tran a U$ and $T \tran b V$
with an empty generator relation
(this component satisfies CLG).
Then CLG fails; however IC and CIRE still hold.
% along with CS$\indt$.
\finex
\end{example}

The mapping $g$ converts an LTSI satisfying CLG into one satisfying LG+IEC.
The mapping $c$ converts an LTSI satisfying LG into one satisfying CLG.
Note that there is an alternative way to convert an LTSI satisfying CLG into one satisfying LG:
simply use the relation $I$ applied to any pair of transitions.
This will in general create more independent transitions than using $g$,
and so the result may not satisfy IEC.



% \begin{remark}\label{rem:coinit to gen}
% As a particular case of this,
% suppose that $\mc L = (\Proc,\Lab,\tran{},\ind)$ is a coinitial LTSI satisfying SP, BTI and WF,
% and suppose that there is a relation $I$ such that $t \ind u$ iff $I(a,b)$,
% where $a,b$ are the underlying labels of $t,u$.
% Then by Proposition~\ref{prop:underlying}, $\mc L$ satisfies PCI and CIRE.
% Let $\mc L' = (\Proc,\Lab,\tran{},g(\ind))$
% where $g(\ind)$ is as in Definition~\ref{def:gen coinit}.
% By Proposition~\ref{prop:gen coinit}, $\mc L'$ is pre-reversible and satisfies IRE and IEC.
% Note that if $t \mathrel{g(\ind)} u$ then $I(a,b)$, where $a,b$ are the respective underlying labels of $t,u$.
% However the converse does not necessarily hold;
% we also require that $t \sqeqt t'$, $u \sqeqt u'$ where $t',u'$ are coinitial.
% \begin{remark}\label{rem:CS CL coinit gen}
%\todo{add references to results used?}
%
%As a consequence of Proposition~\ref{prop:gen coinit} and results from Section~\ref{sec:further},
\subsection{Relating different forms of CS/CL}\label{subsec:comparison}
We now  discuss the relationships between different forms of CS/CL 
and consider which ones to work with in particular reversible settings. 
The starting point is how independence is or can be defined in such settings,
and whether it is general or coinitial. We explain how structural axioms and results of this section, 
together with our axioms, can be used to arrive at the most appropriate causal safety and liveness 
properties for such reversible settings. 
%beyond a basic comparison in Section~\ref{sub:comparing}. 

We can sometimes move between
LTSIs satisfying CS$\ci$ and CL$\ci$ (or equivalently  CS$_<$ and CL$_<$), all defined in terms of coinitial independence, and LTSIs satisfying CS$\indt$ and
CL$\indt$, which are based on general independence, using mappings $c$ and $g$.
Thus, if we have a coinitial pre-reversible LTSI $\mc L$ satisfying CIRE then CS$\ci$ and CL$\ci$
hold (using Theorems~\ref{thm:CS coind} and~\ref{thm:CL coind}, respectively).  
The LTSI $g(\mc L)$ is pre-reversible and satisfies IRE and IEC 
by Proposition~\ref{prop:gen coinit}. This will satisfy CS$\indt$ and CL$\indt$ as a result of 
applying Theorems~\ref{thm:CS} and~\ref{thm:CL}, respectively.
It will also satisfy CS$\ci$ and CL$\ci$. 
%
Conversely, if we have a general pre-reversible LTSI $\mc L'$ satisfying IRE then CS$\indt$ and CL$\indt$
hold by Theorems~\ref{thm:CS} and~\ref{thm:CL}, respectively. The LTSI $c(\mc L')$ 
is a coinitial pre-reversible LTSI satisfying CIRE.
This will satisfy CS$\ci$  and CL$\ci$.

Intuitively, one can think of coinitial independence as a compact way
of representing general independence (provided that this is
well-behaved, in that it satisfies IRE and IEC), and $c$ and $g$ as
ways of moving between the two representations (Proposition~\ref{prop:gen coinit}). 
CS$\indt$ and CL$\indt$ work on the general
representation only, since they check independence between transitions that
may be far apart. The other two forms of CS/CL can instead work with
both the representations, and they are equivalent (Figure~\ref{fig:simpleCSCL}). 
Moreover, once we have LTSI with general independence we can work immediately with 
CS$\indt$ and CL$\indt$. On the other hand, when independence is coinitial, we
need to instantiate the notion of event, and understand whether events are causally dependent or coinitial independent,
%develop the notion of events
% \todo{Iain: not sure about this distinction}
before we can use the other two notions of CS/CL.
The choice between CS$_<$/CL$_<$ and CS$\ci$/CL$\ci$ depends on whether independence 
or ordering is more easily or naturally defined on events. 

In some process calculi and programming languages, as can be seen in the next section,  
independence can be defined in terms of transition labels, which gives us structural axioms 
CLG and LG. So, to show CS/CL we tend to show CLG (RCCS, CCSK, HO$\pi$, Erlang) 
or we prove CIRE (R$\pi$, reversible occurrence nets) and then use $g$. 
Alternatively, we show LG ($\pi$IH).

Note that whether or not CLG/LG can be applied to a reversible formalism may depend on the level of abstraction adopted in the transition labels.


\Comment{
We provide here some insights on the relations between different forms
of CS/CL, beyond the comparison in
Section~\ref{sub:comparing}. Indeed, we can sometimes move between
LTSIs satisfying \todo{CS$\ci$ and CL$\ci$ (or equivalently  CS$_<$ and CL$_<$) } and LTSIs satisfying CS$\indt$ and
CL$\indt$, using mappings $c$ and $g$.

\todo{Thus, if we have a coinitial pre-reversible LTSI $\mc L$ satisfying CIRE then CS$\ci$ and CL$\ci$
hold (using Theorems~\ref{thm:CS coind} and~\ref{thm:CL coind}, respectively).  
The LTSI $g(\mc L)$ is pre-reversible and satisfies IRE and IEC 
by Proposition~\ref{prop:gen coinit}. This will satisfy CS$\indt$ and CL$\indt$ as a result of 
applying Theorems~\ref{thm:CS} and~\ref{thm:CL}, respectively.
It will also satisfy CS$\ci$ and CL$\ci$. 
%
Conversely, if we have a general pre-reversible LTSI $\mc L'$ satisfying IRE then CS$\indt$ and CL$\indt$
hold by Theorems~\ref{thm:CS} and~\ref{thm:CL}, respectively. The LTSI $c(\mc L')$ 
is a coinitial pre-reversible LTSI satisfying CIRE.
This will satisfy CS$\ci$  and CL$\ci$.}

Intuitively, one can think as coinitial independence as a compact way
to represent general independence (provided that this is
well-behaved, in that it satisfies IRE and IEC), and $c$ and $g$ as
ways of moving between the two representations (as shown in Proposition~\ref{prop:gen coinit}). CS$\indt$ and
CL$\indt$ can be thought of as notions that work on the general
representation only, since they check independence between transitions that
may be far away. The other two forms of CS/CL can instead work with
both the representations, and they are equivalent as shown in
Figure~\ref{fig:simpleCSCL}.
\il{On the other side CS$\indt$ and CL$\indt$
directly rely on independence, hence they are \todo{easier} to work with.  Thus,
one can decide to work either with CS$\indt$ and CL$\indt$ using a
general notion of independence, possibly generated from a coinitial
one using $g$, or on a more compact coinitial independence,
but using the two more \todo{demanding} notions of CS/CL. The choice between
CS$_<$/CL$_<$ and CS$\ci$/CL$\ci$ depends on whether independence or
ordering is more easily defined on events.  }

\todo{In the case studies, to show CS/CL we tend to show CLG (RCCS, CCSK, Ho$\pi$, Erlang) or CIRE (R$\pi$, reversible occurrence nets) and then use $g$. Or show LG ($\pi$IH).

Note that whether CLG/LG can be applied to a reversible formalism may depend on the level of abstraction adopted in the transition labels.}
}

% \end{remark}
%
% Of course CS and CL will not hold, as all independence is coinitial.
% \end{remark}

% The further axioms RPI, PI and IRE rely on independence being defined for
% non-coinitial transitions, unlike the basic axioms BTI, SP, WF.
% However some of the consequences of the further axioms can be formulated
% with independence only defined for coinitial transitions, e.g. FD, RED, NRE.
% So we might want to present axioms for the two scenarios (coinitial or general),
% with the possibility that the more general scenario has easier proofs
% or simpler axioms.
% The coinitial scenario corresponds to the development for most reversible calculi
% in the literature.

% We would like examples to show that CC does not imply CS or CL.
% \begin{example}\label{ex:coinitial CC not CL}
% Consider the ltr in Figure~\ref{fig:coinitial repeated1}.
% % Figure environment removed
% CC holds (with SP, BTI, WF, PCI).
% Also CS$\ci$ seems to hold using Definition~\ref{def:coind safe live}.
% However CL$\ci$ fails using Definition~\ref{def:coind safe live}.
% Indeed, consider a path $\ptran {abb}$ from the start (either possibility).
% Since $[a] \ind [b]$ according to Definition~\ref{def:coind events},
% we should be able to reverse $a$, but this is not possible.
% Also CIRE and NRE fail.
% \end{example}



\section{Case Studies}\label{sec:casestudies}
We look at whether our axioms hold in various reversible formalisms.
Given that we consider a high number of formalisms, we do not provide full background on them, but refer for it to the original papers. Also, we sometimes repeat similar observations for different formalisms, so to make it possible to browse them out of order, to find information on a specific formalism of interest. 
Remarkably, all the works below provide proofs of the Loop Lemma.

\subsection{RCCS}\label{sec:rccs}
We consider here the semantics of RCCS in~\cite{DK04}, and restrict the attention to coherent processes~\cite[Definition 2]{DK04}. 
In RCCS, transitions $P \tran {\mu:\zeta} Q$ and $P \tran {\mu':\zeta'} Q'$
are concurrent if $\mu \cap \mu' = \emptyset$ \cite[Definition~7]{DK04}.
This allows us to define coinitial independence as
% $P \tran {\mu:\zeta} Q \ind P \tran {\mu':\zeta'} Q'$
% iff $\mu \cap \mu' = \emptyset$.
$t \ind u$ iff $t$ and $u$ are concurrent.
We now argue that the resulting coinitial LTSI is pre-reversible and
also satisfies CIRE. SP was shown in~\cite[Lemma 8]{DK04}.
BTI was shown in the proof of~\cite[Lemma 10]{DK04}.
WF is straightforward, noting that backward transitions decrease memory size.
Hence, we obtain a very much simplified proof of CC.
For PCI and CIRE we note that CLG holds
%independence is defined on the underlying labels
and thus Proposition~\ref{prop:CLG} applies.
% We can also obtain CS$_<$ and CL$_<$ by Proposition~\ref{prop:CS CL coind <}.
Therefore CS$\ci$ and CL$\ci$ hold.
Using Proposition~\ref{prop:gen coinit},
we can get an LTSI with general independence satisfying IRE and IEC,
and therefore CS$\indt$ and CL$\indt$.
% If we then apply $g$ to $\ind$ (Definition~\ref{def:gen coinit}), we can get a general pre-reversible
% LTSI which satisfies IRE and IEC by Proposition~\ref{prop:gen coinit}. Finally, we  get CS and CL which
% coincide with CS$_<$ and CL$_<$ respectively;
This is the first time these causal properties have been proved for RCCS.

\Comment{
In RCCS, transitions $P \tran {\mu:\zeta} Q$ and $P \tran {\mu':\zeta'} Q'$
are concurrent if $\mu \cap \mu' = \emptyset$.
We can generalise to non-coinitial transitions:
\begin{definition}[Independence for RCCS]\label{def:ind RCCS}
We say that two memories $m$ and $m'$ are coherent~\cite[Def.~1]{DK04}
iff they have a common initial portion followed by a fork on different
branches.

Then $P \tran {\mu:\zeta} Q \ind P' \tran {\mu':\zeta'} Q'$ iff
$\forall m \in \mu.\forall m' \in \mu'$ we have that $m$ and $m'$ are
coherent.
\end{definition}
If we restrict the attention to processes reachable by a process with
empty memories (that is, a CCS process), as done in~\cite{DK04}, then
all different memories in a process $P$ are pairwise
coherent~\cite[Last lines of Section 2]{DK04}.

Hence, by considering coinitial transitions, we have that $P \tran
{\mu:\zeta} Q \ind P \tran {\mu':\zeta'} Q'$ iff $\mu \cap \mu' =
\emptyset$, matching the definition of concurrency
in~\cite[Def.~7]{DK04}.

Let RCCS$\ind$ be RCCS with general independence.
We get an LTSI $\mc L = (\Proc,\Lab,\tran{},\ind)$.
\begin{conjecture}\label{conj:RCCSi}
$\mc L$ satisfies SP, BTI, WF, PCI, IRE, IEC and ED.
\todo{ED not defined - might be best to remove}
\end{conjecture}
\begin{proof}
SP was shown in~\cite[Lemma 8]{DK04}.
BTI was shown in the proof of~\cite[Lemma 10]{DK04}.
% PI and RPI follows directly from the definition of independence.
WF is straightforward noting that forward transitions increase memory size.
PCI and IRE look straightforward, since independence is defined on labels.
Rough idea for IEC:
if the memories are $m_1\cell{1}m$ and $m_2\cell{2}m$ then starting at
$m_1\cell{1}m$ carry out the $m_2$ transitions to get to a state where
both events are enabled.
As far as ED is concerned, it might be enough to show LED.
\end{proof}

% \todo{Material moved from Section~\ref{sec:coinitial}:
% Question: does RCCS satisfy IEC?
% Suppose that we define two transitions to be independent along the lines of
% Definition~\ref{def:ind RCCS}.
% Then the two transitions might be entirely unrelated.
% We can add to the definition of independence the requirement that there is a path from
% one source process to the other (which amounts to having a common irreversible ancestor).
% Rough idea for IEC: if the memories are $m_1\cell{1}m$ and $m_2\cell{2}m$ then starting at
% $m_1\cell{1}m$ carry out the $m_2$ transitions to get to a state where both events are enabled.
% }

\begin{conjecture}\label{conj:RCCS ind conc}
Let $\co$ be the concurrency relation on coinitial transitions in RCCS
as in~\cite{DK04}.
Let $g$ be the mapping of Definition~\ref{def:gen coinit}.
Then ${\ind} = g(\co)$, where $\ind$ is as in Definition~\ref{def:ind RCCS}.
\end{conjecture}
\begin{proof}
\todo{To be supplied. Essentially the same as showing that RCCS$\ind$ satisfies IEC.}
\end{proof}


For any simple CCS process $P$ in RCCS$\ind$, let $\Proc_P$ be the states
which are forwards reachable from $P$ using $\ftran{}$.
We note that $\Proc_P$ is closed under reverse transitions $\rtran{}$,
since $P$ is irreversible and RCCS$\ind$ satisfies PL.
\begin{conjecture}\label{conj:RCCSi oTSI}
$\mc L_P = (\Proc_P,P,\Lab,\ftran{},\ind)$ is an oTSI with initial state $P$
which also satisfies property (E).
\end{conjecture}
\begin{proof}
Immediate from Conjecture~\ref{conj:RCCSi}.
\end{proof}
It follows from~\cite[Cor~4.28]{SNW96} that $\mc L_P$ is equivalent to a
labelled prime event structure.

\todo{Perhaps we do not want to consider ED in the present work,
partly since that increases the overlap with ~\cite{PU07a}.}

Once SP and BTI are shown (already done in~\cite{DK04})
the remaining axioms WF, PCI and CIRE are straightforward to show,
noting that concurrency is defined on transition labels
and using Proposition~\ref{prop:underlying}.
We obtain a very much simplified proof of CC,
plus we show CS and CL for the first time.

We can use the method of Remark~\ref{rem:coinit to gen} to obtain a general
LTSI satisfying WF, SP, BTI, PCI, IRE, IEC.

\begin{remark}
The proof of CC in~\cite{DK04} uses EFP as a lemma~\cite[Lemma 11]{DK04}.
In our approach this becomes a simple consequence of CC.
\end{remark}
%end of Comment
}


  \subsection{CCSK}
The first notion of independence for CCSK~\cite{PU07} was given in~\cite{Aub22}. It is based on the proved transition system approach where transition labels
contain information about derivation of transitions. This information can be used to work out whether transitions are in conflict, causally dependent, or concurrent. Two forms of independence are defined in~\cite{Aub22}: general independence (called composable concurrency) and coinitial independence (called coinitial concurrency). CC is then obtained using our axiomatic approach (following~\cite{LanesePU20}, the conference version of the present paper) by showing SP \cite[Theorem~3]{Aub22}, BTI \cite[Lemma~6]{Aub22} and WF \cite[Lemma~7]{Aub22}.

%\iu{Some adjustment needed here as a result of discussions with Iain and Clément.}
Since coinitial independence is defined on labels, we can
%set $I(a,b)$, for labels $a$ and $b$, to hold if $a\smile b$~\cite[Definition 11]{Aub22} and
deduce that the %generated
LTSI is CLG. Hence, by Proposition~\ref{prop:CLG}, PCI and CIRE hold. This allows us to obtain CS$\ci$ and CL$\ci$.
Using Proposition~\ref{prop:gen coinit},
we can get an LTSI with general independence which satisfies IRE and IEC, which gives us CS$\indt$ and CL$\indt$ as well. 
As for RCCS, this is the first time such causal properties have been 
proved for CCSK.

\Comment{
Alternatively, coinitial independence can be defined for CCSK original labels as follows. We refer to~\cite{PU06,PU07} for the description of CCSK syntax and semantics.
\iu{Coinitial CCSK transitions $P\tran{\alpha[m]} Q$ and $P\tran {\beta[n]} R$ (forward or reverse)
% with $\alpha[m]\neq \beta[n]$, 
are \emph{independent} if and only if one of the conditions below holds:
\begin{enumerate}
\item $P\equiv U\Par V$, and $U\tran{\alpha[m]} U'$ with $V\tran {\beta[n]} V'$, 
	where $Q\equiv U'\Par V$ and $R\equiv U\Par V'$; 
\item $P\equiv c[k].U$ with $k\neq m,n$, and $U\tran{\alpha[m]} U'$ and  $U\tran {\beta[n]} U''$ 
	are independent, where $Q\equiv c[k].U'$ and $R\equiv c[k].U''$;
\item $P\equiv U+V$, and $U\tran{\alpha[m]} U'$ and $U\tran {\beta[n]} U''$ 
	are independent, where $Q\equiv U'+V$ and $R\equiv U''+V$;
\item $P\equiv U\setminus c$, and $U\tran{\alpha[m]} U'$ and $U\tran {\beta[n]} U''$ are independent, 
        where $\alpha, \beta\neq c$, $Q\equiv U' \setminus c$ and $R\equiv U''\setminus c$.
\end{enumerate}
This allows us to define \il{an} LTSI for CCSK with $\ind$ being this independence relation. Note that 
$a\Par a \tran{a[m]}  a[m]\Par a$ and $a\Par a \tran{a[n]}  a\Par a[n]$  are independent, but although $a.a$ has the
same initial transitions $a.a\tran{a[m]} a[m].a$ and $a.a\tran{a[n]} a[n].a$, 
they are not independent as 
they do not originate from different sides of a parallel composition. 

BTI, SP and PCI follow by induction on the structure of CCSK processes. 
For WF we note that when CCSK processes 
compute, their structure remains the same modulo addition (or removal) of a key or a pair of keys 
during each transition. Starting from an irreversible process (standard process in CCSK terminology), any derivative process will only 
have finitely many keys, hence WF is satisfied. As a result, the LTSI for CCSK is pre-reversible, 
and we obtain PL and CC by applying our axiomatic approach. 

In CCSK, unlike for RCCS, independence cannot be defined purely on the underlying labels, so we cannot use
Proposition~\ref{prop:CLG} to obtain CIRE. Instead, we can prove it by structural induction. 
This would  give us CS$_<$ and CL$_<$. Finally, using Proposition~\ref{prop:gen coinit},
we can obtain a notion of general independence which satisfies IRE and IEC,
and therefore CS$\indt$ and CL$\indt$. 
}
}

\subsection{HO$\pi$}\label{sec:hopi}
We consider here the uncontrolled reversible semantics for HO$\pi$~\cite{LaneseMS16}. 
We restrict our attention to reachable
processes, called there consistent.
The semantics is a reduction semantics; hence there are no labels (or, equivalently, all
the labels coincide). To have more informative labels we
can consider the transitions defined in~\cite[Section~3.1]{LaneseMS16},
where labels contain the memory created or consumed by the transition
  (they also contain a flag distinguishing backward from forward transitions, but this plays no role in the definition of the concurrency relation discussed below, hence we can safely drop it). 
%and a flag denoting whether the transition is forward or backward.
The notion of independence would be given by the concurrency relation on coinitial
transitions~\cite[Definition 9]{LaneseMS16}.
%We remark that the Loop Lemma holds~\cite[Lemma 6]{LaneseMS16}.
All pre-reversible LTSI axioms hold, as well as CIRE. 
Specifically, SP is proved in~\cite[Lemma 9]{LaneseMS16}. BTI holds since distinct memories have disjoint
sets of keys~\cite[Definition 3 and Lemma 3]{LaneseMS16} and by
the definition of concurrency~\cite[Definition 9]{LaneseMS16}.
WF holds as each backward step consumes a memory, which are a finite number to start with.
Finally, PCI and CIRE hold since CLG holds for the LTSI with annotated labels
%the notion of concurrency is defined on the
%annotated labels
and using our Proposition~\ref{prop:CLG}.
%
\Comment{
\begin{description}
\item[SP:] proved in~\cite[Lemma 9]{LaneseMS16};
\item[BTI:] since distinct memories have disjoint
  sets of keys~\cite[Definition 3 and Lemma 3]{LaneseMS16} and by 
  the definition of concurrency~\cite[Definition 9]{LaneseMS16};
% \item[PI:] does not apply since the notion of concurrency only considers coinitial transitions;
% \item[RPI:] does not apply since the notion of concurrency only considers coinitial transitions;
\item[WF:] since each backward step consumes a memory;
%\item[UT:] holds trivially under the reduction semantics, holds also
%  under the annotated semantics, since the label coincides with the memory;
\item[PCI, CIRE:] since the notion of concurrency is defined on the
  annotated labels and using our Proposition~\ref{prop:underlying}. 
%\item[CIRE:] since the notion of concurrency is defined on the
%  annotated labels.
%\item[IRE:] does not apply since the notion of concurrency only
  %considers coinitial transitions.
\end{description}
% Since SP, BTI, WF, and PCI hold, we obtain a very much simplified proof of CC. 
% We also get CS$_<$ and CL$_<$ and, applying the mapping $g$ from Section~5, we obtain
% a pre-reversible LTSI satisfying IRE and IEC.  This gives us CS and CL.
%
}

As a result we obtain a very much simplified proof of CC.
Moreover, using PCI and CIRE, we get the CS$\ci$ and CL$\ci$ safety and liveness properties and, 
applying mapping $g$ from Section~\ref{sec:coinitial}, we get a general 
pre-reversible LTSI satisfying IRE and IEC, so that CS$\indt$ and CL$\indt$ are satisfied. This is the first time
that causal properties have been shown for HO$\pi$.

\subsection{R$\pi$}\label{sec:pi}
We consider the (uncontrolled) reversible semantics for
$\pi$-calculus defined in~\cite{CristescuKV13}. We restrict the
attention to reachable processes. The semantics is an LTS
semantics.
Independence is given as concurrency which is defined for consecutive transitions~\cite[Definition
  4.1]{CristescuKV13}.  CC holds~\cite[Theorem~4.5]{CristescuKV13}.

Our results are not directly applicable to R$\pi$,
% since concurrency is defined for
% consecutive transitions, rather than coinitial transitions or pairs of transitions
% in general, so that none of .
since SP holds up to label equivalence of transitions on opposite sides
of the diamond,
rather than equality of labels as in our approach.
We would need to extend axiom SP and the definition of causal equivalence to allow for label equivalence in order to directly handle R$\pi$ using our axiomatic method.

We can however apply our theory to an LTSI obtained by considering labels up-to the equivalence relation $=_\lambda$~\cite[just before Lemma 4.3]{CristescuKV13}, which intuitively avoids to observe when a name is being extruded.
Notice that the Loop Lemma holds in this new LTSI as well.
However, the concurrency relation is given on consecutive transitions, and the same for their SP. Nevertheless, we can define independence as follows: $t \ind_\pi u$ iff $t$ and $u$ are coinitial and $t$ and $\rev u$ are concurrent. Notice that since $t$ and $u$ are coinitial then $t$ and $\rev u$ are consecutive. 
\begin{lemma}
  $\ind_\pi$ is symmetric.
\end{lemma}
\begin{proof}
  We have to show that $t$ and $\rev u$ are concurrent iff $\rev t$ and $u$ are concurrent. Since concurrency is defined as the complement of structural causality and contextual causality~\cite[Definition 4.1]{CristescuKV13}, it is enough to prove that $t$ and $\rev u$ are structural or contextual causal iff $\rev t$ and $u$ are. For structural causality, it follows from the definition~\cite[Definition 4.1]{CristescuKV13}. For contextual causality, it follows from~\cite[Proposition 4.2]{CristescuKV13}.
\end{proof}

With this definition of independence SP holds~\cite[Lemma 4.3]{CristescuKV13}. WF
holds as well since each backward step consumes at least a memory.
BTI has been proved as part of the proof of PL in~\cite[Lemma
  14]{CristescuPhD}.  As a result we obtain a proof of CC much simpler
than the one in~\cite[Theorem 11]{CristescuPhD} (note that causal
equivalence in~\cite[Definition 4.4]{CristescuKV13} is formalised
up-to $=_\lambda$ as well).

Independence is coinitial by construction.
We have to prove PCI and CIRE. Unfortunately, we cannot exploit CLG, since it does not hold, as is clear from the definition of structural cause~\cite[Definition 4.1]{CristescuKV13}, one of the ingredients of the concurrency relation. Thus we need to go for a direct proof. 
%\todo{CLG does not hold, presumably - could point this out.}

\begin{lemma}
  CIRE holds in the LTSI for R$\pi$.
\end{lemma}
\begin{proof}
  Concurrency is defined as the complement of structural causality and
  contextual causality~\cite[Definition
    4.1]{CristescuKV13}. Contextual causality is defined on
  labels~\cite[Proposition 4.2]{CristescuKV13}. Structural causality
  depends on whether the $i$ components of the two labels occur in the
  same memory in a specific relation~\cite[Definition
    2.2]{CristescuKV13}. However, one can notice that $i$ can only
  occur in the memory of one of the threads participating to the
  action (see~\cite[Table 1]{CristescuKV13}), which are the same in
  transitions in the same event. The thesis follows.
\end{proof}
\begin{lemma}
  PCI holds in the LTSI for R$\pi$.
\end{lemma}
\begin{proof}
  Similar to the one above.
\end{proof}
Using PCI and CIRE, we get the CS$\ci$ and CL$\ci$ safety and liveness properties. 
Applying mapping $g$ from Section~\ref{sec:coinitial}, we get a general 
pre-reversible LTSI satisfying IRE and IEC, so that CS$\indt$ and CL$\indt$ are satisfied.
Notice that the notion of independence is not influenced by the abstraction on labels; hence the results can be reflected on the original LTSI of R$\pi$.

% \begin{description}
% \item[SP:]
% \item[BTI:] 
% \item[PI:] 
% \item[RPI:] 
% \item[WF:] trivial, since each backward step consumes at least a memory;
% \item[PCI:] 
% \item[CIRE:] 
% \item[IRE:] 
% \end{description}


\subsection{Reversible internal $\pi$-calculus with extrusion histories}
The reversible internal $\pi$-calculus $\pi$IH~\cite{GPY21} is based on the work of Hildebrandt \emph{et al.}~\cite{HJN19},
which uses extrusion histories and locations to define a stable non-interleaving early operational semantics for the $\pi$-calculus.
Locations and extrusion histories are used to define independence of actions.
This notion of independence differs from the ones considered in the other case studies in that it allows actions with conflicting causes to be independent.
Despite this major difference, it is shown in~\cite{GPY21} that nearly all our (non-structural) axioms are satisfied
(SP, BTI, WF, PCI, IRE); the only exception is that IEC fails,
because a process can have independent transitions with conflicting causes without having a single state where equivalent transitions can both be performed.
We use IEC to show RPI (Proposition~\ref{prop:RPI}).
However RPI is shown in~\cite{GPY21}  for $\pi$IH without the need for IEC,
using the fact that independence is defined on transition labels.
In fact, LG holds for $\pi$IH, from which we can deduce PCI, IRE and RPI by Proposition~\ref{prop:LG}.
It follows that all the properties listed in Table~\ref{t:list} hold for $\pi$IH, with the exception of IEC, IC and CLG.


\subsection{Reversible Erlang}\label{sec:erlang}
We consider the uncontrolled reversible (reduction) semantics for Erlang
in~\cite{LaneseNPV18}. We restrict our attention to reachable
processes. 
%The semantics is a reduction semantics; hence reductions have no labels.
%(or, equivalently, all labels are the same).
In order to
have more informative labels we can consider the annotations defined
in~\cite[Section 4.1]{LaneseNPV18}. We can then define coinitial transitions to be independent
iff they are concurrent~\cite[Definition 12]{LaneseNPV18}.  %We remark that the
%The Loop Lemma holds~\cite[Lemma 11]{LaneseNPV18}.

We next discuss the validity of our axioms in reversible Erlang.
SP is proved in~\cite[Lemma 13]{LaneseNPV18} and BTI is trivial from the definition 
of concurrency~\cite[Definition 12]{LaneseNPV18}.  WF holds since the pair of non-negative integers 
(total number of elements in history, total number of messages queued) ordered under
lexicographic order decreases at each backward
step. Intuitively, each step but the ones derived using the rule for reverse sched 
(see~\cite[Figure~11]{LaneseNPV18}) consumes an item of memory, and each step derived using 
rule reverse sched removes a message from a process queue. Finally, PCI and CIRE hold since CLG holds for the LTSI with annotated labels,
%the notion of concurrency is defined on the annotated labels,
and by Proposition~\ref{prop:CLG}.
%
\Comment{
\begin{description}
\item[SP:] proved in~\cite[Lemma 13]{LaneseNPV18};
\item[BTI:] trivial from the definition of concurrency~\cite[Definition 12]{LaneseNPV18};
% \item[PI:] does not apply since the notion of concurrency only considers coinitial transitions;
% \item[RPI:] does not apply since the notion of concurrency only considers coinitial transitions;
\item[WF:] trivial, since the pairs of integers (total number of
  elements in memories, total number of messages queued) ordered under
  lexicographic order are always positive and decrease at each backward
  step. Intuitively, each step but the ones derived using the rule for reverse sched (see~\cite[Fig.~11]{LaneseNPV18}) consumes an item of memory, and each step derived using rule reverse sched removes a message from a process queue;
%\item[UT:] holds trivially under the reduction semantics, holds also
%  under the annotated semantics. Intuitively, the label can be deduced
%  from the history item, but for Sched, where it can be deduced from
%  the state of the queue;
\item[PCI, CIRE:] hold, since the notion of concurrency is defined on the
  annotated labels, and by Proposition~\ref{prop:underlying}.
%\item[CIRE:] holds, since the notion of concurrency is defined on the
%  annotated labels.
%\item[IRE:] does not apply since the notion of concurrency only
% considers coinitial transitions.
\end{description}
%SP is already shown~\cite[Lemma~13]{LaneseNPV18}. The remaining axioms BTI, WF, PCI and CIRE are 
%straightforward to show, noting that concurrency is defined on transition labels
%and using Proposition~\ref{prop:underlying}. 
%
Since SP, BTI and WF hold, we obtain a very much simplified proof of CC.
Moreover, using PCI and CIRE, we get the CS$_<$ and CL$_<$ safety and liveness properties and, applying mapping $g$ from Section~\ref{sec:coinitial}, we get a general 
pre-reversible LTSI satisfying IRE and IEC.  This in turn will satisfy \il{CS$\indt$ and CL$\indt$.}
%
%plus we show CS and CL in strengthened \todo{can we say that?} versions compared to~\cite[Corollary~22]{LaneseNPV18}).
%We can use the method of Remark~\ref{rem:coinit to gen} to obtain a general
%LTSI satisfying SP, BTI, WF, PCI, IRE and IEC.
}

Since this setting is very similar to the one of HO$\pi$
(both calculi have a reduction semantics and a coinitial notion of independence defined on enriched labels),
we get the same results as for
HO$\pi$ (described in Section~\ref{sec:hopi}), including CC, and causal safety and liveness.
%CS$\indt$ and CL$\indt$.



\subsection{Reversible occurrence nets}
We consider occurrence nets which are the result of unfolding Place/Transition nets, and their reversible versions~\cite{MMU19,MMU20,MelgrattiMPPU2020}. 
Reversible occurrence nets are occurrence nets 
(1-safe and with no backward conflicts)
extended with a backward (reverse in the terminology of~\cite{MMU20}) transition name $\overleftarrow{{\sf t}}$ for each forward transition name ${\sf t}$. We write $t, u$ (note the $italic$ font) for forward or backward transition names, and $\overleftarrow{t}, \overleftarrow{u}$ for their backward or forward duals. We use ``transition name'' to mean forward or backward transition name.
%
They give rise to an LTS where states 
are pairs $(N,m)$ with $N$ a net and $m$ a marking. A computation that represents firing a (forward or backward)
transition name $t$ in $(N,m)$ and resulting in $(N,m')$ is given by a firing relation $(N,m)\tran{t} (N,m')$~\footnote{
We use ``transition names'' in this subsection to name the members of the set of transitions which, together with the set of places,
are part of the definition of Place/Transition nets or occurrence nets. This distinguishes them from our transitions, which are called firings in Place/Transition nets and occurrence nets.}. 
%
Independence is the concurrency relation $\co$ which is defined between arbitrary firings as follows:
two firings are concurrent if their transition names are concurrent, that is when they are not in conflict 
and do not cause each other~\cite[Section 3]{MMU19,MMU20}. The last two notions are defined in terms 
of conditions on pre- and postset relations on transition names.
Hence, we get an LTSI with general independence. Note that transition names are unique. 
%However, we can equally well obtain an LTSI with coinitial independence
%using properties of coinitial concurrent transition names~\cite[Lemma~3.3]{MMU20}.

Properties SP and PL are shown as~\cite[Lemma~4.3]{MMU20} and \cite[Lemma~4.4]{MMU20}, respectively. 
Then CC is proved (over several pages) as~\cite[Theorem~4.6]{MMU20} using SP and PL.
%These proofs are over four pages long~\cite{MMU20}. 
The causal safety and causal liveness properties are not considered in \cite{MMU19,MMU20}.
However, a form of such properties is discussed in~\cite{MelgrattiMPPU2020} in the setting of reversible prime event structures; we discuss this point in Section~\ref{sec:related}.

We can obtain %\il{CS$\indt$ and CL$\indt$,} 
causal safety and causal liveness properties, as well as PL and CC, for reversible occurrence nets using our axiomatic approach.
The following lemma will be helpful.
\begin{lemma}\label{lem:nocausation}
Let $t$ and $u$ be enabled and coinitial (forward or backward) transition names. Then $t$ does not cause $u$. If additionally  $t$ and $u$ are backward, 
then they are not in conflict.
\end{lemma}
\begin{proof}
 Assume for contradiction that $t$ causes $u$. 
So there is a place, say $a$, in the preset of  $u$ such that $t$ causes $a$. Since $u$ is enabled there is a token in $a$. 
Also, since $t$ is enabled, after it fires a second token will arrive in $a$, thus
contradicting the 1-safe property of occurrence nets.

Let $t$ and $u$  be $\overleftarrow{{\sf t}}$ and 
  $\overleftarrow{{\sf u}}$ respectively.  Assume for contradiction that  they are in conflict. This means that they share a place, say $a$, in their presets. Hence, ${\sf t}$ and ${\sf u}$ share $a$ in their postsets, which contradicts the no backwards conflict property of occurrence nets. 
\end{proof}

%\iu{
We can now combine Lemma~\ref{lem:nocausation} with the conditions in~\cite[Lemma~3.3]{MMU20} of when enabled and coinitial $t$ and $u$ are concurrent.

\begin{lemma}\label{lem:on-concurrency}
Let $t$ and $u$ be enabled and coinitial (forward or backward) transition names. Then
$t\co u$ iff $t$ and $u$ are backward or they are not in an immediate conflict.
\end{lemma}
%\todo{Previous version:$t\co u$
%  $t\co u$ iff $t$ and $u$ are not in an immediate conflict if at least one of them is a forward transition name.}\\
As a consequence, BTI holds.


% With SP holding, BTI follows from~\cite[Lemma 3.3]{MMU20}. 
\begin{lemma}
  BTI holds in the LTSI for reversible occurrence nets.
\end{lemma}
%\begin{proof}
%  Assume enabled coinitial firings with reverse transition names $\overleftarrow{{\sf t}}$ and $\overleftarrow{{\sf u}}$. The firings are concurrent iff $\overleftarrow{{\sf t}}$ and $\overleftarrow{{\sf u}}$ are concurrent, which is when they are not in conflict and they do not cause each other: this is given by Lemma~\ref{lem:nocausation}. 
%\end{proof}

WF holds because there are no forward cycles of firings in occurrence nets, hence 
no infinite reverse paths. This gives us PL and CC.
Next, we prove PCI. 
\begin{lemma}
PCI holds in the LTSI for reversible occurrence nets.
\end{lemma}
\begin{proof}
Consider enabled coinitial  firings $\phi_1, \phi_2$ with transition names $t, u$ respectively, and assume $\phi_1\co \phi_2$. Hence $t\co u$. We get a commuting diamond by SP, where the opposite sides have the same transition names. Since $t\co u$, we have  $\overleftarrow{t}\co u$ by~\cite[Lemma 3.4]{MMU20}, so PCI holds. 
\end{proof}
This gives us a pre-reversible LTSI, and thus CS$\ci$ and CS$_<$ hold.

Given a pair of enabled coinitial concurrent  transition names we get a commuting diamond by SP, and the pairs of  coinitial transition names in all corners of the diamond are concurrent. Events can then be defined on firings in such diamonds as in Definition~\ref{def:sqeqt}, and we can show IRE.
\begin{lemma}\label{lem:IRE-occnets}
  IRE holds in the LTSI for reversible occurrence nets.
\end{lemma}
\begin{proof}
Let $\phi_1, \phi_2$ be firings with $t, u$ respectively, and let $\phi_1\co \phi_2$. 
This means that $t\co u$. Since any $\phi_1'$ equivalent to $\phi$ has the same transition name $t$, $t\co u$ gives us
$\phi_1'\co \phi_2$.
\end{proof}

Since IRE implies CIRE we obtain CL$\ci$ (or CL$_<$). We also have CS$\indt$ and CL$\indt$ as IRE holds.

An alternative proof strategy would be to show CLG first, but we believe this approach leads to more complex technicalities, and we would still need to prove IRE, hence we have preferred the approach above.
%}
%% Alternatively, we can express independence between coinitial firings with transition names $t$ and $u$ purely in terms of a relation on their underlying versions. If $t$ and $u$ are forward ${\sf t}$ and {\sf u}, then we let $I({\sf t},{\sf u})$ iff ${\sf t} \co {\sf u}$. This is equivalent to ${\sf t}$ and {\sf u} not being in an immediate conflict~\cite[Lemma~3.3]{MMU20}. 
%% If $t$ and $u$ are $\overleftarrow{{\sf t}}$ and ${\sf u}$ respectively, then 
%% $\overleftarrow{{\sf t}} \co {\sf u}$ iff ${\sf t}$ and ${\sf u}$ by Lemma~3.4 in \cite{MMU20}, where {\sf t} is the underlying version
%% of $\overleftarrow{{\sf t}}$.  We can then let $I({\sf t},{\sf u})$ iff ${\sf t} \co {\sf u}$. In terms of causal dependence and conflict relations, since they do not cause each other by Lemma~\ref{lem:nocausation},   $\overleftarrow{{\sf t}} \co {\sf u}$ is equivalent to requiring that {\sf t} and {\sf u} are not in an immediate conflict~\cite[Lemma~3.3]{MMU20}. Lastly, when $t$ are $u$ are 
%% $\overleftarrow{{\sf t}}$ and $\overleftarrow{{\sf u}}$, respectively, then $t \co u$ by Lemma~\ref{lem:nocausation}. We then get
%% $\overleftarrow{ t} \co \overleftarrow{u}$, which is ${\sf t} \co {\sf u}$, by SP and PCI. So, we let $I({\sf t},{\sf u})$ iff
%% ${\sf t} \co {\sf u}$. Hence, CLG holds and we obtain PCI, CIRE and IC by Proposition~\ref{prop:CLG}. We can then have Lemma~\ref{lem:IRE-occnets}.
%%   }
%. \todo{IVAN: Future work?}

\Comment{
%
% Some questions and material commented out following a meeting on 6 June
%

%
\todo{It is clear that applying the map $c$ to general 
independence $co$ gives the coinitial independence portion of $co$. Since there is no definition of events, thus no results on events in~\cite{MMU20}, some work is needed  to show that applying $g$ to the coinitial portion of $co$ (obtained by applying $c$) gives back the full global independence $co$.
Iain: I think this might be false.  Consider net $O_1$ in Figure 2 of~\cite{MMU20}.
With both initial places marked as shown, the two firings are coinitial and concurrent.  But if we consider the two firings got with only one initial place marked these are concurrent (I think).
But they cannot be got from the coinitial firings by a ladder (using our definition of event). }

  
%\Comment{ 
We can obtain %\il{CS$\indt$ and CL$\indt$,} 
causal safety and causal liveness properties, as well as PL and CC, for reversible occurrence nets using our axiomatic 
approach. With SP holding,  
%SP is proved as~\cite[Lemma 4.3]{MMU20}. 
BTI follows from~\cite[Lemma 3.3]{MMU20}. 
\todo{Iain: Lemma 3.3 says that reverse coinitial transitions are concurrent iff they do not cause each other,
while BTI says they are always concurrent.
So this is a strengthening of Lemma 3.3.
Consider transitions $t_1 < t_2 < t_3$.
With tokens immediately after $t_1$ and $t_3$ both of these can fire in reverse.
However they are not concurrent.
But this is not an occurrence net?}
WF holds because there are no forward cycles of firings in occurrence nets, hence 
no infinite reverse paths. This gives us PL and CC.
In order to have causal safety and causal liveness properties we first need to prove PCI. 
\todo{Iain: Why doesn't LG hold?  It seems that concurrency is defined
on the transitions rather than the firings. Perhaps the reason is that reverse
transitions are handled differently from forward transitions.
But it would still be the case that IRE holds?}
%
Consider two coinitial concurrent firings 
in a commuting diagram \todo{diamond?}. We can show that the firings on the consecutive sides of this diamond are 
also concurrent \todo{is this clear?}. Then we obtain PCI by~\cite[Lemma 3.4]{MMU20}. This gives us a pre-reversible LTSI and CS$\ci$ and CS$_<$ hold.  In order to get CL$\ci$ (or CL$_<$) we need to show CIRE, and IRE is required for CS$\indt$ and CL$\indt$.

%$\preS{t}$ and $\postS{u}$

\Comment{\todo{why bring in definitions of event?}
The equivalence relation $\sim$ on firings in commuting diamonds of firings can be defined as in Definition~\ref{def:sqeqt simp}, thus giving the notion of events as firings in the same equivalence class (Definitions~\ref{def:sqeqt}). We can then have $\coind$ as in Definition~\ref{def:coind events}, and we can show CIRE using SP, PCI and the definition when firings are concurrent.
}
%
%
%
%\Comment{
Since there is no backwards conflict in occurrence nets transition names have unique causal histories.
Firings for transition name $t$ (and its causal history) 
are equivalent if there is a ladder of commuting diamonds connecting them.  This gives the notion 
of event as the firings in the same equivalence class (for a transition name and its causal history). 
Consider %forward 
coinitial  concurrent firings, with presets of $t$ and $u$ contained in $m_1$ and $m_2$ respectively:
\begin{equation}
(N,m_1\oplus m_2 \oplus m_3)\tran{t}(N,m_1'\oplus m_2 \oplus m_3)
\quad 
(N,m_1\oplus m_2 \oplus m_3)\tran{u}(N,m_1\oplus m_2' \oplus m_3).
\end{equation}
They give rise by SP to these cofinal firings which make up a commuting diamond:
\begin{equation}
(N,m_1\oplus m_2' \oplus m_3)\tran{t}(N,m_1' \oplus m_2' \oplus m_3)
\quad 
(N,m_1' \oplus m_2 \oplus m_3)\tran{u}(N,m_1' \oplus m_2' \oplus m_3).
\end{equation}
Generally, firings that are equivalent to those in (1), namely those that can be connected by a ladder of commuting diamonds, have the following form, for some markings $m_1^\dagger, m_2^\dagger,  m_3^\dagger$ and $m_3^{\dagger\dagger}$: 
\begin{equation}
(N,m_1\oplus m_2^\dagger \oplus m_3^\dagger) \tran{t}(N,m_1'\oplus m_2^\dagger \oplus m_3^\dagger)
\quad 
(N,m_1^\dagger \oplus m_2 \oplus m_3^{\dagger\dagger})\tran{u}(N,m_1^\dagger\oplus m_2' \oplus m_3^{\dagger\dagger})
\end{equation}

To show CIRE we assume that the events of the firings in (3) are coinitially concurrent and that the firings are coinitial. Then we show that the firings are concurrent.

 Since the firings are coinitial  we  deduce that $m_1^\dagger = m_1$, $m_2^\dagger = m_2$, and $m_3^\dagger = 
m_3^{\dagger\dagger}$. So we have
\begin{equation}
(N,m_1\oplus m_2 \oplus m_3^\dagger) \tran{t}(N,m_1'\oplus m_2 \oplus m_3^\dagger) \quad 
(N,m_1 \oplus m_2 \oplus m_3^{\dagger})\tran{u}(N,m_1\oplus m_2' \oplus m_3^{\dagger}).
\end{equation}
%Note that various markings may overlap, for example $m_1$ and $m_2$. However, 
Since  the events of the firings are coinitially concurrent we deduce that there are equivalent coinitial concurrent firings for $t$ and $u$. Assume wlog that they are the firings in (1). In order to show that the firings in (4) are concurrent, we consider three cases: $t$ and $u$ are forward, $t$ is forward and $u$ is backward, and  $t$ and $u$ are backward.

In the first case,  $t$ and $u$ in (1) being concurrent means  that they are not in an immediate conflict, so their presets do not overlap: $ m_1 \cap m_2= \emptyset$. Hence, by~\cite[Lemma 3.3]{MMU20}
the firings in (4) are also concurrent.

The second case is $t$ is forward and $u$ is backward in (1). By PCI  applied to the commuting diamond with the firings (1) we obtain that
\begin{equation}
(N,m_1\oplus m_2' \oplus m_3)\tran{t}(N,m_1' \oplus m_2' \oplus m_3)
\quad 
(N,m_1 \oplus m_2' \oplus m_3)\tran{\rev u}(N,m_1 \oplus m_2 \oplus m_3)
\end{equation}
are forward and concurrent, hence $ m_1 \cap m_2'= \emptyset$. So we can deduce that 
\begin{equation}
(N,m_1\oplus m_2' \oplus m_3^\dagger) \tran{t}(N,m_1'\oplus m_2' \oplus m_3^\dagger) \quad 
(N,m_1 \oplus m_2' \oplus m_3^{\dagger})\tran{\rev u}(N,m_1\oplus m_2 \oplus m_3^{\dagger})
\end{equation}
are concurrent,  and that there is a commuting diamond with the two firings in (4),
$ (N,m_1\oplus m_2' \oplus m_3^\dagger) \tran{t}(N,m_1'\oplus m_2' \oplus m_3^\dagger)$ and
$(N,m_1'\oplus m_2 \oplus m_3^{\dagger})\tran{u}(N,m_1'\oplus m_2' \oplus m_3^{\dagger})$.
Since firings in (6) are concurrent the firings in (4) are also concurrent by PCI. 

In the last case the firings in (1) are backwards and since they are concurrent they do not cause each 
other~(\cite[Lemma 3.3]{MMU20}). This means the postset of $t$ does not overlap with the preset of $u$, namely
$m_1'\cap m_2=\emptyset$, and the other way round: $m_2'\cap m_1=\emptyset$. Hence, the firings in (4) are also concurrent.    


%}
%
%
%
\todo{We could try to show IRE but it would require a lot of work.  Or we could use the mapping $g$ from Section~\ref{sec:coinitial}.
Since CIRE holds we get IRE and IEC by Proposition~\ref{prop:gen coinit}.
However, it remains to be checked if $co$ coincides with $g$ applied to the coinitial portion of $co$. Showing this amounts to proving IRE.

Another way to obtain IRE is to enrich labels of firings. Since there is no backward conflict transition names have unique causal histories. Let $H(t)$ stand for (an encoding of) the causal history of $t$. We can have several identical transition names in an occurrence net resulting from unfolding of a Place/Transition net but they have their different causal histories. If we use firings with labels that combine a transition name with its causal history, then we can obtain LG: firings with labels $t, H(t)$ and $u, H(u)$ are defined to be independent iff $t$ and $u$ (with histories $H(t)$ and $H(u)$ respectively) are concurrent.  }

\Comment{
%
% Commented in May
%
The equivalence relation $\sim$ on firings in commuting diamonds of firings can be defined as in Definition~\ref{def:sqeqt simp}, thus giving the notion of events as firings in the same equivalence class (Definitions~\ref{def:sqeqt}). We can then have $\coind$ as in Definition~\ref{def:CIRE}.
Firings with transition $t$ are equivalent if there is a ladder of commuting diamonds connecting them. Consider forward coinitial and concurrent firings 
\begin{equation}
(N,m_1\oplus m_2 \oplus m_3)\tran{t}(N,m_1'\oplus m_2 \oplus m_3)
\quad 
(N,m_1\oplus m_2 \oplus m_3)\tran{u}(N,m_1\oplus m_2' \oplus m_3).
\end{equation}
By SP, we obtain these cofinal firings which make up a commuting diamond:
\begin{equation}
(N,m_1\oplus m_2' \oplus m_3)\tran{t}(N,m_1' \oplus m_2' \oplus m_3)
\quad 
(N,m_1' \oplus m_2 \oplus m_3)\tran{u}(N,m_1' \oplus m_2' \oplus m_3).
\end{equation}
For presets of $t$ and $u$, namely
$\preS{t}$ and $\preS{u}$,
 we have $\preS{t}\subseteq m_1$ and $\preS{u}\subseteq m_2$. Moreover, $ m_1 \cap m_2= \emptyset$ since the firings are concurrent.
Generally, firings that are equivalent to those in (1), namely those that can be connected by a ladder of commuting diamonds, have the following form, for some $m_1^\dagger, m_2^\dagger,  m_3^\dagger$ and $m_3^{\dagger\dagger}$: 
\begin{equation}
(N,m_1\oplus m_2^\dagger \oplus m_3^\dagger) \tran{t}(N,m_1'\oplus m_2^\dagger \oplus m_3^\dagger)
\quad 
(N,m_1^\dagger \oplus m_2 \oplus m_3^{\dagger\dagger})\tran{u}(N,m_1^\dagger\oplus m_2' \oplus m_3^{\dagger\dagger})
\end{equation}


%where none of $(m_1, m_2^\dagger)$ and $(m_1^\dagger, m_2)$ overlap. 
%$\coind$
To show CIRE assume that the events of the firings in (3) are related by $\coind$, written as $[t] \coind [u]$, and that the firings are coinitial.

 By SP, we obtain these cofinal firings which make up a commuting diamond:
\begin{equation}
(N,m_1\oplus m_2^\dagger \oplus m_3^\dagger) \tran{t}(N,m_1'\oplus m_2^\dagger \oplus m_3^\dagger)
\quad 
(N,m_1^\dagger \oplus m_2 \oplus m_3^{\dagger\dagger})\tran{u}(N,m_1^\dagger\oplus m_2' \oplus m_3^{\dagger\dagger})
\end{equation}


 and both are forward. The last implies that $m_1^\dagger = m_1$, $m_2^\dagger = m_2$, and $m_3^\dagger = 
m_3^{\dagger\dagger}$. So we have
$(N,m_1\oplus m_2 \oplus m_3^\dagger) \tran{t}(N,m_1'\oplus m_2 \oplus m_3^\dagger)$ and 
$(N,m_1 \oplus m_2 \oplus m_3^{\dagger})\tran{u}(N,m_1\oplus m_2' \oplus m_3^{\dagger})$.
%Note that various markings may overlap, for example $m_1$ and $m_2$. However, 
Since  $[t] \coind [u]$ we deduce that there are equivalent coinitial firings for $t$ and $u$, for some $m_3'$, 
$$(N,m_1\oplus m_2 \oplus m_3')\tran{t}(N,m_1'\oplus m_2 \oplus m_3')
\quad 
(N,m_1\oplus m_2 \oplus m_3')\tran{u}(N,m_1\oplus m_2' \oplus m_3'),
$$
which are concurrent, implying  $ m_1 \cap m_2= \emptyset$. Hence, the firings in (3) are concurrent.

The second case is $t$ is forward and $u$ is backward (and coinitial). By PCI  we have that
\begin{equation}
(N,m_1\oplus m_2' \oplus m_3)\tran{t}(N,m_1' \oplus m_2' \oplus m_3)
\quad 
(N,m_1 \oplus m_2' \oplus m_3)\tran{\rev u}(N,m_1 \oplus m_2 \oplus m_3)
\end{equation}
are concurrent. In a commuting diamond for the firings in (3), and have that the firings corresponding to those in (4) are concurrent. ince they are forward we can prove that  
}
%}
\Comment{Events can be defined on firings in such commuting diagrams as in our Definitions~\ref{def:sqeqt} and~\ref{def:sqeqt simp}, and then IRE holds as 
the concurrency relation preserves such events.\il{*** maybe rewrite the previous sentence?***}
}
}
\subsection{Reversible sequential systems}
%\il{
In \emph{sequential systems} there is no concurrency.
Hence, in this section, we represent them as LTSIs where the independence
relation, modelling concurrency, is empty.
This is for instance the case for Janus programs~\cite{YokoyamaG07} or CCSK processes without parallel composition.
%}
%% In this section a \emph{sequential system} is defined to be an LTSI where the independence
%% relation is the empty relation, which
%% is for instance the case for Janus programs~\cite{YokoyamaG07} or CCSK processes without parallel composition.
%% \todo{Omit?:This holds in particular when
%% selecting the concurrency relation as independence relation.}
In this setting,
% because the independence relation is empty,
SP, PCI, IRE and IEC hold trivially. 
Moreover, BTI is equivalent to backward determinism, which is
the main condition required for reversibility in a sequential setting (see, e.g., Janus~\cite{YokoyamaG07}).

\begin{definition}[Backward determinism]
An LTSI is backward deterministic iff $P \ftran{a} Q$ and $P'
\ftran{a'} Q$ imply $P = P'$ and $a=a'$.
\end{definition}

\begin{proposition}\label{prop:sequential}
A sequential system satisfies BTI iff it is backward deterministic.
\end{proposition}
\begin{proof}
For the left to right implication, assume towards a contradiction that
the system satisfies BTI but it is not backward deterministic. Then
there are $P \ftran{a} Q$ and $P' \ftran{a'} Q$ with $P \neq P'$ or $a
\neq a'$. By the Loop Lemma we have the reverse transitions, which are
coinitial and backwards, hence by BTI they need to be independent,
what is a contradiction since the independence relation is empty.

For the right to left implication, take two backward coinitial
transitions $t,t'$. By applying the Loop Lemma there exist $\rev t,
\rev t'$. One can notice that $\rev t,\rev t'$ satisfy the hypothesis
of backward determinism. Hence, $\rev t=\rev t'$ and $t=t'$. Hence BTI
trivially holds.
\end{proof}

WF does not hold in general and needs to be assumed.

If we assume WF then all our results hold, but they all become trivial
or almost trivial.  E.g., all events are singletons. Also, all the notions of causal liveness coincide, and they state that the last
transition can always be undone, but this is just one direction of the
Loop Lemma. Similarly, all the notions of causal safety do coincide, and they require that only the last transition can
be undone.


Most academic vocabulary lists have been developed in the context of English for Academic Purposes (EAP). On the whole, two categories of lists exist. One list type aims to identify academic words commonly used in EAP across disciplines, which students could be made aware of. The studies aiming to provide cross-disciplinary academic word lists usually use large corpora containing expert academic writing from various disciplines. The widely used lists of this type are the Academic Word List (AWL) \cite{coxhead2000new} and the Academic Vocabulary List (AVL) \cite{gardner2014new}. The second type of list seeks to identify discipline or field-specific words worth teaching. Various specialised lists have been developed for fields such as veterinary medicine \cite{ohashi2020esp} or nursing \cite{yang2015nursing}.

While there is a growing interest in building cross-disciplinary academic word lists for languages other than English, these academic word lists remain few. See, for example studies conducted for French \cite{cobb2004there}, Persian \cite{rezvanifirst}, Portuguese \cite{baptista2010p}, Swedish \cite{carlund2012academic}, and Norwegian \cite{johannessen2016constructing}. An explanation for this scarcity might be that academic language data sets are rare and often not freely available due to copyright. This can be especially true for low-resource languages, such as Romanian. Access to a representative corpus is crucial, as the validity and reliability of an academic word list highly depend on the quality of the data set. 

Apart from the limited availability of academic writing corpora, an additional challenge may be that there is no standard procedure for extracting academic word lists. Scholars are still exploring and testing various methodologies. For example, some studies build on the methods used for the AWL or the AVL \cite{johannessen2016constructing,rezvanifirst}. One study uses the translated version of the AVL in Portuguese as a starting point for its investigation \cite{baptista2010p}. Another study proposes a new word list extraction method different from previous ones \cite{carlund2012academic}.  

In the case of Romanian, no previous studies have compiled specialised or general academic word lists. Although in the last 10-15 years, several research institutions and projects have been involved in developing corpus resources in Romanian, relatively few have focused exclusively on general academic writing. Some of the most significant corpora recently compiled, such as ROMBAC (Romanian Balanced Annotated Corpus, see \citet{ion2012rombac}), with more than 30 million words, CoRoLa (Corpus of Contemporary Romanian Language, see \citet{mititelu2014corola}), or The Balanced Romanian Corpus (BRC, see \citet{midrigan2020resources}) cover only few disciplines or subsets: 5 sections for ROMBAC (journalism, literature, medical texts, legal texts, biographies), uneven and unfiltered distribution of resources in CoRoLa (the collection of academic writing texts is based on agreements with publishing houses and journals, without filtering of the content on quality criteria) and BRC (literary text samples, research articles, news, spoken data). The ROMBAC corpus (excluding the medical subcorpus) was already used to develop the Romanian Word List (RWL, see \citet{szabo2015introducing}), targeted at Romanian L2 learners (e.g. from the Hungarian minority in Romania). The list is a general list of words, not focused on academic language. As far as discipline-specific corpora are concerned, smaller corpora such as SiMoNERo (medical corpus, \citet{mitrofan2019monero}), BioRo \cite{mitrofan2018bioro}, PARSEME-Ro (news articles), LegalNERo (legal, \citet{paiș2021named}), MARCELL (legal, multilingual, see \citet{varadi2020marcell}), CURLICAT (multilingual, containing several domains: Economics, Education, Health, Sciences, etc., see \citet{varadi2022introducing}) have been compiled. However, apart from compiling the datasets and conducting a series of descriptive studies, no special attention is given to the lexical level. 

In this context, the EXPRES corpus (Corpus of Expert Writing in Romanian and English) is the first corpus of discipline-specific academic writing in the Romanian context (academic writing in Romanian L1 and academic writing in English L2 produced by Romanians) \cite{bucur2022expres,chitez2022write}. Covering four disciplines – Linguistics, Economics, Political Sciences, Information Technology –, the Romanian subset contains 200 open-access research articles from each domain, published in the past 5-10 years in peer-reviewed journals (see \citet{chitez2022expres}). The rigorous selection criteria \cite{rogobete2021challenges} contribute to the representativeness of the corpus, making it a suitable candidate for testing a possible Romanian Word List and narrowing it down to an Academic Word List. Furthermore, the EXPRES corpus is the first Romanian expert academic corpus available on an open-access query platform. Unlike other Romanian corpora, which offer limited access to third parties and poor resources for downloading search results or statistics, the EXPRES corpus support platform has been implemented as a cross-platform distributed web application  \cite{chitez2022expres}.


\section{Conclusion and Future Work}
In this work, I design corruption-robust algorithms for the Lipschitz contextual search problem. I present the \emph{agnostic checking} technique and demonstrate its effectiveness in designing corruption-robust algorithms. There are several open problems for future research. First, in the algorithm I propose for pricing loss, the schedule for agnostic checks is fixed upfront. Can the learner design an adaptive checking schedule for the pricing loss? Second, this work assumes the learner has knowledge of the Lipschitz constant $L$. Can the learner design efficient no-regret algorithms without knowledge of $L$? 

% \begin{acks}
\section*{Acknowledgements}
This work has been partially supported by COST Action IC1405 on Reversible Computation - Extending Horizons of Computing. The first author has also been partially supported by French ANR project DCore ANR-18-CE25-0007 and by INdAM as a member of GNCS (Gruppo Nazionale per il Calcolo Scientifico). The third author has been partially supported by the JSPS Invitation Fellowship S21050.
% \end{acks}

%\newpage
\bibliographystyle{plain}
% \bibliographystyle{ACM-Reference-Format}
\bibliography{axrev}

\providecommand{\url}[1]{\texttt{#1}}
\providecommand{\urlprefix}{URL }
\providecommand{\doi}[1]{https://doi.org/#1}

%\newpage
%\appendix
%\begin{comment}
\section{System Architecture}
\label{appendix:architecture}
\system has a novel modularized system architecture with three key components: 
\emph{StreamManager}, 
\emph{TxnManager} and \emph{TxnScheduler}. 
These components are instantiated in each thread locally.
The execution outline of \system is presented in Algorithm~\ref{alg:algo}.
Transactional stream processing is continuous and potentially never ends (Line 1$\sim$8).
The dependency resolution and execution of state transactions are separated into two non-overlapping phases by punctuations~\cite{Tucker:2003:EPS:776752.776780} (Line 2 and 5), which guarantees that no subsequent input event will have a smaller timestamp. 
Effectively, a batch of state transactions is collected during the first phase, and processed during the second phase.

In the first phase (i.e., stream processing phase), 
the \emph{StreamManager} conducts preprocessing for every input event ($e$). Similar to some prior works~\cite{tstream}, state transactions may be issued but not immediately processed during preprocessing (Line 3).
The \emph{pre\_processing} and \emph{post\_processing} functions are exposed as APIs to users.
The \emph{TxnManager} handles dependency resolution (Line 4) among state transactions and insert decomposed operations to construct a \tpg. We discuss the detailed two-phase \tpg construction process in Section~\ref{subsec:construction}.

In the second phase  (i.e., transaction processing phase), 
the \emph{TxnManager} is first involved again to refine (Line 6) the constructed \tpg with further dependency resolution.
The \emph{TxnScheduler} 
schedules operations for concurrent execution based on the constructed \tpg according to the three dimensions of scheduling decisions (Line 7). 
In particular, a scheduling decision model $M$ is instantiated based on the constructed \tpg (Line 14).
\textbf{\circled{1}} Guided by $M$, execution threads adopt an exploration strategy (Section~\ref{subsec:explore}) to explore the constructed \tpg for operations available to be scheduled constrained by dependencies. 
\textbf{\circled{2}} 
During exploration, one or multiple operations may be treated as the 
% basic 
unit of scheduling (Section~\ref{subsec:granularity}). 
Subsequently, \textbf{\circled{3}} every thread executes operation(s) in the unit of scheduling with various abort handling mechanisms (Section~\ref{subsec:abort_handling}).
Only when state transactions are processed (i.e., committed or aborted) can the associated input events be postprocessed (Line 8) by the \emph{StreamManager} based on transaction processing results.
\end{comment}

\begin{comment}
\begin{algorithm}
\footnotesize
    \KwData{$e$ \tcp{Input event}}
    \KwData{$txn_{ts}$ \tcp{State transaction}}
    \KwData{$G$ \tcp{The currently constructed TPG}}
    \While{!finish processing of input streams}{
        \eIf(\tcp*[h]{Phase 1}){\text{$e$ is not a $punctuation$}}{
                $txn_{ts}$ $\gets$ PRE\_Processing($e$)\;
                \textbf{TPG\_Construction}($G$, $txn_{ts}$)\; 
          }(\tcp*[h]{Phase 2}){
                \textbf{TPG\_Refinement}($G$)\; 
                \textbf{TXN\_Scheduling}($G$)\; 
                POST\_Processing()\;
          }
    }
    
    \SetKwFunction{FMain}{TPG\_Construction}
    \SetKwProg{Fn}{Function}{:}{}
    \Fn{\FMain{$G$, $txn_{ts}$}}{
        $O_{1..k}$ $\gets$ \textbf{Partition} $txn_{ts}$\;
        \ForEach{\text{operation $O_{i}$ $\in$ $O_{1..k}$}}{
            \textbf{Identify} its \ld\;
            $G$ $\gets$ $G$ + $O_{i}$ \;
        }
    }
    \SetKwFunction{FMain}{TPG\_Refinement}
    \SetKwProg{Fn}{Function}{:}{}
    \Fn{\FMain{$G$}}{
        \ForEach{\text{vertex $e_{i}$ $\in$ $G$}}{
            \textbf{Identify} its \td, \pd\;
        }
    }
    
    \SetKwFunction{FMain}{TXN\_Scheduling}
    \SetKwProg{Fn}{Function}{:}{}
    \Fn{\FMain{$G$}}{
        $M$ $\gets$ Instantiated with $G$;\tcp{A decision model}
        \While{!finish scheduling of $G$
        }{
          \textbf{\circled{2}} $Scheduling Unit$ $\gets$ \textbf{\circled{1}} \emph{Explore}($G$, $M$)\; 
            \textbf{\circled{3}} \emph{Execute with Abort Handling} ($Scheduling Unit$)\; 
        }
    }
  \caption{Execution Outline of \system}
  \label{alg:algo}
\end{algorithm}
\end{comment}

\end{document}
