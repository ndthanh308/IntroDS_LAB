\section{Conclusion and Future Work}\label{sec:conclusion}

%\todo{To be completed.  Discuss~\cite{SNW96} and~\cite{PU07a}}

The literature on causal-consistent reversibility (see, for example the
early survey~\cite{LMT14}) has a number of proofs of results
such as PL and CC, all of which are instantiated to a specific calculus, language or formalism.  
We have taken here a complementary and more general approach, analysing the properties of
interest in an abstract and language-independent setting. In particular,
we have shown how to prove the most relevant of these properties from a
small number of axioms. Among the properties, we discussed in detail the formalisation of Causal Safety and Causal Liveness, which were mostly informally discussed in the literature.


%Our approach builds upon~\cite{PU07a}, where a set of axioms for
%reverse LTSs  was given and several 
%interesting properties were shown. 
% including NRE, \emph{Reverse Diamond} and \emph{Forward Diamond}~\cite{PU07}. 
%The LTSs satisfying these axioms are called prime ltrs and are shown
%to correspond to prime event structures.
%While the idea is similar, the development is rather different since
%we consider more basic axioms (we only share WF, while many of the
%axioms in~\cite{PU07a}, such as UT, follow from ours), and since the two 
%papers focus on different properties. We focus on CC and various forms of CS
%and CL, while~\cite{PU07a} considers correspondence with prime event structures 
%and reversible bisimulations. Moreover, LTSs in~\cite{PU07a} 
%do not have a notion of independence.

%The novelty of our current approach is in the use of
%an independence relation~\cite{SNW96} with reverse LTSs.  

%We remark that \emph{Event Determinism}
%(two coinitial transitions that represent the same event are cofinal) holds
%for prime ltrs, whereas only RED, its reverse version, holds
%for LTSIs satisfying our axioms. The novelty of our current approach is in the use of an
%independence relation~\cite{SNW96} with reverse LTSs.
% Independence allows to express
% when transitions are concurrent (can happen in any order), allows to define events
% and it defines all but one (WF) of our axioms. 


%In other related work,
%we may particularly mention~\cite{DanosKS07},
%which like ours takes an abstract view, though based on
%category theory. However, its results concern irreversible actions,
%and do not provide insights in our setting, where all actions are
%reversible. The only other work which takes a general perspective
%is~\cite{BernadetL16}, which concentrates on how to derive a
%reversible extension of a given formalism. However, proofs concern a
%limited number of properties (essentially our CC), and hold only 
%for extensions built using the technique proposed there. 
%Also~\cite{PU06,PU07} are general, since they propose how to
%reverse a calculus that can be defined in a general format of SOS rules. 
%However, the format has its
%syntactic constraints while our approach abstracts from them.
%A similar approach is taken in~\cite{LaneseM20}, which focuses on
%systems modelled using reduction semantics. In order to prove
%properties of the reversible systems they build they use our theory
%(taken from the conference version of the present
%paper~\cite{LanesePU20}), hence this can be taken as an additional
%case study for our results.
%Finally,~\cite{EKM19} presents a number of properties such as, for example,
%backward confluence, which arise in the context of reversing
%of steps of executed transitions in Place/Transition nets.


The approach proposed in this paper opens a number of new
possibilities. Firstly, when devising a new reversible formalism, our
results provide a rich toolbox to prove (or disprove) relevant
properties in a simple way. Indeed, proving the axioms is usually much simpler than proving the properties directly. This is particularly relevant since
causal-consistent reversibility is getting applied to more and more
complex languages, such as Erlang~\cite{LaneseNPV18}, where direct
proofs become cumbersome and error-prone.
%
Secondly, our abstract proofs are relatively easy to formalise in a
proof-assistant, which is even more relevant given that this will
certify the correctness of the results for many possible instances.
%
Another possible extension of our work concerns integrating into our framework mechanisms to control reversibility~\cite{LaneseMS12}, such as
 a rollback operator~\cite{LMSS11} or irreversible actions~\cite{DK05}. For the latter we could take
inspiration from the above-mentioned~\cite{DanosKS07}. 
