%
\section{Labelled Transition Systems with Independence}\label{sec:LTSIs}
We want to study reversibility in a setting as general as possible.
Thus, we base on the core of the notion of \emph{labelled
  transition system with independence} (LTSI)~\cite[Definition~3.7]{SNW96}. However,
while~\cite{SNW96} requires a number of axioms on LTSI, we take the
basic definition and explore what can be done by adding or not adding
various axioms. Also, we extend LTSI with reverse transitions, since we
study reversible systems. We define first labelled transition
systems (LTSs).

We consider the LTS of the entire set of processes in a calculus,
rather than the transition graph of a particular process and its derivatives, hence we do not fix an initial state.

\begin{definition}\label{def:lts}
A {\em labelled transition system (LTS)\/} 
is a structure \mbox{$(\Proc,\Lab,\tran{})$}, where $\Proc$ is the set of states (or processes),
$\Lab$ is the set of action labels and ${\tran{}}\subseteq \Proc \times \Lab \times \Proc$
is a {\em transition relation\/}. 
\end{definition}
We let $P,Q,\ldots$ range over processes,
$a,b,c,\ldots$ range over labels,
and $t,u,v,\ldots$ range over transitions.
We can write $t:P \tran a Q$ to denote that $t = (P,a,Q)$.
We call $a$-transition a transition with label $a$.
    
\begin{definition}[LTS with independence]\label{def:ltsi}
We say that $(\Proc,\Lab,\tran{},\ind)$ is an \emph{LTS with
  independence} (LTSI) if $(\Proc,\Lab,\tran{})$ is an LTS and $\ind$
is an irreflexive symmetric binary relation on transitions.
\end{definition}
In many cases (see Section~\ref{sec:casestudies}), the notion of
independence coincides with the notion of concurrency. However, this
is not always the case. Indeed, concurrency implies that transitions
are independent since they happen in different processses, but
transitions taken by the same process can be independent as
well. Think, for instance, of a reactive process that may react in any
order to two events arriving at the same time, and the final result
does not depend on the order of reactions.

%We remark that independence may or may not coincide with concurrency.
%We will not speak about concurrency in this paper, however one can
%notice that in all the instances any reasonable notion of concurrency
%implies the notion of independence, while the opposite is not
%necessarily true.

We shall assume that all transitions are reversible,
so that the Loop Lemma~\cite[Lemma 6]{DK04} holds.
% A more general treatment would allow for some transitions to be irreversible.
This does not hold in models of reversibility with control mechanisms~\cite{LaneseMS12}
such as irreversible actions~\cite{DK05} or a rollback operator~\cite{LMSS11}.
Nevertheless,
when showing properties of models with controlled reversibility it has
proved sensible to first consider the underlying models where all transitions
are reversible, and then study how control mechanisms change the
picture~\cite{GiachinoLMT17,LaneseNPV18}.
The present work helps with the first step.
\begin{definition}[Reverse and combined LTS]\label{def:reverse transition}
Given an LTS 
$(\Proc,\Lab,\ftran{})$,
let the \emph{reverse LTS} be $(\Proc,\Lab,\rtran{})$,
where $P \rtran a Q$ iff $Q \ftran a P$.
It is convenient to combine the two LTSs (forward and reverse):
let the reverse labels be
$\rev\Lab = \{\rev a : a \in \Lab\}$,
and define the combined LTS to be
${\tran{}}\subseteq \Proc \times (\Lab \cup \rev\Lab) \times \Proc$
by $P \tran a Q$ iff $P \ftran a Q$ and $P \tran{\rev a} Q$ iff $P \rtran a Q$.
\end{definition}
We stipulate that the union $\Lab \cup \rev\Lab$ is disjoint.
We let $\alpha,\ldots$ range over $\Lab \cup \rev\Lab$.
For $\alpha \in \Lab \cup \rev\Lab$, the \emph{underlying} action label
$\und\alpha$ is defined as
$\und{a} = a$ and $\und{\rev a} = a$.
%if $\alpha \in \rev\Lab$.
%
%where $\alpha = a$ if $\alpha \in \Lab$
%and $\alpha = \rev a$ if $\alpha \in \rev\Lab$.
Let $\rev{\rev a} = a$ for $a \in \Lab$.  Given $t:P \tran \alpha Q$, let $\rev t:Q \tran {\rev\alpha} P$ be the transition which reverses $t$.
We define a labelling function $\lab$ from transitions to $\Lab\cup \rev\Lab$ by setting
$\lab((P,\alpha,Q)) = \alpha$.


We let $\rho,\sigma,\ldots$ range over finite sequences $\alpha_1 \ldots \alpha_n$,
with $\es$ representing the empty sequence. %starting and ending at $P$.
%with $\es_P$ representing the empty sequence starting and ending at $P$.
%We shall write $\es$ when $P$ is understood.
Given an LTS, a \emph{path} is a sequence of forward or reverse transitions
of the form
$P_0 \tran{\alpha_1} P_1 \cdots \tran{\alpha_n} P_n$.
We let $r,s,\ldots$ range over paths.
We may write $r:P \ptran \rho Q$ where the intermediate states are understood.
On occasion we may refer to a path simply by its sequence of labels $\rho$.
The concatenation of paths $r$ and $s$ is written $rs$.
Given a path $r:P \ptran \rho Q$, the inverse path is $\rev r:Q \ptran {\rev \rho} P$
where $\rev\es = \es$ and $\rev{\alpha\rho} = \rev \rho \; \rev \alpha$.
The length of a path $r$ (notated $\len r$) is the number of transitions in the path.
Paths $r:P \ptran \rho Q$ and $R \ptran \sigma S$ are
\emph{coinitial} if $P = R$ and \emph{cofinal} if $Q = S$.
We say that a path is \emph{forward-only} if it contains no reverse transitions;
similarly a path is \emph{backward-only} if it contains no forward transitions.
Sometimes we let $f,\ldots$ and $b,\ldots$ range over forward-only and backward-only paths, respectively;
it will be clear from the context whether $b$ represents an action label or a path.

Let $(\Proc,\Lab,\tran{})$ be an LTS. The irreversible processes in $(\Proc,\Lab,\tran{})$ are
$\Irr = \{P \in \Proc: P \not \rtran{}\}$.
A \emph{rooted path} is a
path $r:P \ptran \rho Q$ such that $P \in \Irr$.

In the following we consider LTSIs obtained by adding a notion of
independence to combined LTSs as above. We call the result a
\emph{combined LTSI}.

\begin{remark}
From now on, unless stated otherwise, we consider a combined LTSI $\mc L = (\Proc,\Lab,\tran{},\ind)$. We will refer to it simply as an LTSI.     
\end{remark}
