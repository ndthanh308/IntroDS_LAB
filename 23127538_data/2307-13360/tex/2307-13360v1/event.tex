\section{Events}\label{sec:events}
In order to define and study causal safety and liveness (Section~\ref{sec:CSCL}),
we first need the concept of event.

%Definition of $\sqeqt$ with rotational symmetry and coinitial independence:

\begin{definition}[Event, general definition]\label{def:sqeqt}
  %Let $(\Proc,\Lab,\tran{},\ind)$ be an LTSI.
Consider an LTSI.  
Let $\sqeqt$ be the smallest equivalence relation satisfying:
if $t:P \tran \alpha Q$, $u:P \tran \beta R$,
$u':Q \tran \beta S$, $t':R \tran \alpha S$,
and $t \ind u$, $\rev u \ind t'$, $\rev{t'} \ind \rev{u'}$, $u' \ind \rev t$,
and
\begin{itemize}
\item
$Q \neq R$ if $\alpha$ and $\beta$ are both forwards or both backwards;
\item
$P \neq S$ otherwise;
\end{itemize}
then $t \sqeqt t'$.
The equivalence classes of transitions, written $[t]$ or $[P,\alpha,Q]$, are the \emph{events}.
We say that an event is \emph{forward} if it is the equivalence class of a forward transition;
similarly for \emph{reverse} events.
Given an event $e = [t]$ we let $\rev e = [\rev t]$.
Also, we let $\und{e}=e$ if $e$ is forward, $\und{e}=\rev e$ if $e$ is backward. 
\end{definition}
Intuitively, events are the equivalence classes generated by equating transitions on the opposite sides of commuting squares.
Events are introduced as a derived notion in an LTS with independence in~\cite{SNW96},
in the context of forward-only computation.
We have changed their definition by using coinitial independence at all corners
of the diamond,
yielding rotational symmetry.
This reflects our view that forward and backward transitions have equal status. 

The labelling function $\lab$ can be extended to $\,{\tran{}}/\sqeqt$ since the label does not depend on the choice of the representative inside the equivalence class.

%  to $\Lab$ by setting
%$\lab([P,\alpha,Q]) = \alpha$.

\subsection{Pre-reversible LTSIs}

Our definition can be simplified if the LTSI, and independence in
particular, are well-behaved. Thus, we now add a further axiom related
to independence. This leads us to pre-reversible LTSIs.

\begin{definition}\label{def:PCI}
  {\bf Propagation of coinitial independence (PCI)}\footnote{PCI was called CPI (coinitial propagation of independence) in~\cite{LanesePU20}; we changed the terminology following a suggestion from Marco Bernardo to better match the intuition.}:
  if $t:P
    \tran \alpha Q$, $u:P \tran \beta R$, $u': Q \tran \beta S$ and
    $t':R \tran \alpha S$ with $t \ind u$, then $u' \ind \rev t$.
\end{definition}


% \begin{definition}[Independence axioms]\label{def:indep}
  % We say that a combined LTSI $\mc L = (\Proc,\Lab,\tran{},\ind)$ satisfies:
  % \begin{description}
  % \item[Reversing preserves independence (RPI)] if $t \ind t'$ then
    % $\rev t \ind t'$;
% \todo{RPI can be defined later and derived rather than being an axiom}
  % \item[Coinitial propagation of independence (PCI)] if whenever $t:P
    % \tran \alpha Q$, $u:P \tran \beta R$, $u': Q \tran \beta S$ and
    % $t':R \tran \alpha S$ with $t \ind u$, we have $u' \ind \rev t$.
  % \end{description}
% \end{definition}
% RPI fixes the interplay between independence and reversing, while
PCI
states that independence is a property of commuting diamonds more than
of their specific pairs of edges. Indeed,
% together with RPI,
it allows
independence to propagate around a commuting diamond.

\begin{definition}[Pre-reversible LTSI]\label{def:prerev}
If an LTSI satisfies axioms SP, BTI, WF and PCI,
we say that it is \emph{pre-reversible}.
\end{definition}
The name `pre-reversible' indicates that we expect to require further axioms,
but the present four are enough to ensure that LTSIs are well-behaved,
with events compatible with causal equivalence (cfr.~Lemma~\ref{lemma:cccount}). Pre-reversible axioms are separated from further
axioms by a dashed line in Table~\ref{t:list}.

A first consequence of PCI is that coinitial transitions with mutually inverse labels are not independent.
\begin{lemma}\label{lemma:revnotind}
Suppose that an LTSI satisfies PCI. 
If $t: P \tran\alpha Q$ and $u: P \tran{\rev\alpha} R$ are coinitial transitions with mutually inverse labels,
then $t \centernot\ind u$.
\end{lemma}
\begin{proof}
Suppose that $t: P \tran\alpha Q$ and $u: P \tran{\rev\alpha} R$
are independent.
Consider the degenerate diamond with two copies of $P$ and transitions $t,u,\rev t, \rev u$.
By applying PCI we deduce $\rev t \ind \rev t$, which contradicts irreflexivity of $\ind$.
\end{proof}
Additionally, we cannot have two different coinitial backward transitions with the same label.
%\todo{Two defs of backward det, one should change name, maybe Backward Label Determinism?}
\begin{definition}\label{def:BD}
{\bf Backward label determinism (BLD):
if}
$t: P \rtran a Q$ and $u: P \rtran a R$ are coinitial backward transitions
with the same label then $t = u$.
\end{definition}
\begin{proposition}\label{prop:BD}
Suppose that an LTSI satisfies SP, BTI and PCI. Then it satisfies BLD.
\end{proposition}
\begin{proof}
Suppose $t: P \rtran a Q$ and $u: P \rtran a R$.
Then if $t \neq u$ we have $t \ind u$ by BTI.
We can complete a diamond with $Q \rtran a S$, $R \rtran a S$ by SP.
But then $Q \tran a P$ and $Q \rtran a S$ are independent by PCI.
This is a contradiction of Lemma~\ref{lemma:revnotind}.
\end{proof}
A consequence of Lemma~\ref{lemma:revnotind} is that an LTSI
satisfying BTI and PCI cannot include a diamond
$P \tran a Q \tran a S$, $P \tran a R \tran a S$
where all four transitions have the same label.
This can be seen as ruling out \emph{autoconcurrency}~\cite{Bed91}.
%Note that the transitions coming from different threads in a CCSK~\cite{PU07} process such as $a \Par a$ will be distinguished \todo{actually the key could be the same, I believe in this case we have a conflict on the key?} using keys:
%$a \Par a \tran {a \key m } a \key m \Par a$.

The following non-degeneracy property was shown for occurrence transition systems with independence in~\cite[page~312]{SNW96}, which considers forward transitions only.
We have to cope with backward as well as forward transitions.
%
%\begin{restatable}{lemma}{nondegenerate}\label{lem:non-degenerate}
\begin{lemma}\label{lem:non-degenerate}
Suppose that an LTSI is pre-reversible.
If we have a diamond
$t:P \tran \alpha Q$, $u:P \tran \beta R$ with $t \ind u$
together with cofinal transitions $u': Q \tran \beta S$ and $t': R \tran\alpha S$,
then the diamond is \emph{non-degenerate},
meaning that $P,Q,R,S$ are distinct states.
\end{lemma}
%\end{restatable}
%
\begin{proof}
We note that CC holds; hence UT holds thanks to Corollary~\ref{cor:ut}.
By WF we see that $P \neq Q \neq S \neq R \neq P$.
It remains to show $Q \neq R$ and $P \neq S$.

Suppose $Q = R$.
By $t \ind u$ we know $t \neq u$.  So $\alpha \neq \beta$.
But if $\alpha$ and $\beta$ are both forward or both backward this is impossible by UT.
If one is forward and the other is backward then this is impossible by WF.
Hence $Q \neq R$.

Suppose $P = S$.
If $\alpha$ and $\beta$ are both forward or both backward this is impossible by WF.
If one is forward and the other is backward then by UT this implies that
$\alpha = \rev\beta$.
Then $t \centernot\ind u$ by Lemma~\ref{lemma:revnotind},
which is a contradiction.
% Hence $u' = \rev t$\il{, but this is not possible because of Lemma~\ref{lemma:revnotind}.}
%By PCI we have $\rev t \ind u'$,
%which contradicts irreflexivity of $\ind$.
Hence $P \neq S$.
\end{proof}

If an LTSI is pre-reversible then by
Lemma~\ref{lem:non-degenerate} and the use of PCI
we can simplify the statement of Definition~\ref{def:sqeqt}
to:

\begin{definition}[Event, simplified definition]\label{def:sqeqt simp}
  %Let $(\Proc,\Lab,\tran{},\ind)$ be
  Consider a pre-reversible LTSI.
Let $\sqeqt$ be the smallest equivalence relation satisfying:
if $t:P \tran \alpha Q$, $u:P \tran \beta R$,
$u':Q \tran \beta S$, $t':R \tran \alpha S$,
and $t \ind u$,
then $t \sqeqt t'$.
\end{definition}

We are now able to show independence of diamonds (ID), which can be seen as
dual of SP.

\begin{definition}\label{def:ID}
%  [Independence of Diamonds (ID)]
% Let $\mc L = (\Proc,\Lab,\tran{},\ind)$ be an LTSI.
% Then $\mc L$ satisfies the
%
% Could we do without stating $(\Proc,\Lab,\tran{},\ind)$ here? We have defined PCI without stating it.
%
%An LTSI $(\Proc,\Lab,\tran{},\ind)$ satisfies the
%An LTSI satisfies
{\bf Independence of Diamonds (ID)}: if we have a diamond
$t:P \tran \alpha Q$, $u:P \tran \beta R$,
$u': Q \tran \beta S$ and $t':R \tran \alpha S$,
with %$Q \neq R$
\begin{itemize}
\item
$Q \neq R$ if $\alpha$ and $\beta$ are both forwards or both backwards;
\item
$P \neq S$ otherwise;
\end{itemize}
then $t \ind u$.
\end{definition}
%\begin{restatable}{proposition}{ID}\label{prop:ID}
\begin{proposition}\label{prop:ID}
    If an LTSI satisfies BTI and PCI then it satisfies ID.
\end{proposition}
%\end{restatable}
\begin{proof}
Suppose we have a diamond
$t:P \tran \alpha Q$, $u:P \tran \beta R$,
$u': Q \tran \beta S$ and $t':R \tran \alpha S$,
with %$Q \neq R$
\begin{itemize}
\item
$Q \neq R$ if $\alpha$ and $\beta$ are both forwards or both backwards;
\item
$P \neq S$ otherwise.
\end{itemize}
We must show $t \ind u$.
There are various cases, depending on whether $\alpha$ and $\beta$ are forwards or backwards.
If they are both forwards, then $Q \neq R$.  Hence $\rev{t'} \neq \rev{u'}$
and by BTI we have $\rev{t'} \ind \rev{u'}$.
By PCI, $u' \ind \rev t$ and again by PCI $t \ind u$ as required. 
Other cases are similar.
\end{proof}
In the proof of the above proposition it must be the case that
$\und\alpha \neq \und\beta$, or else we get a contradiction using
Lemma~\ref{lemma:revnotind}.
  
%\todo{If prereversible then by BD (uses BTI, SP, PCI) we must have
%$\und\alpha \neq \und\beta$ in the above.}


% \todo{This does not make much sense if Proposition~\ref{prop:count ceqt} is in the appendix - it was needed for Example~\ref{ex:PL not CC}, but now we are using UT instead there.}
% We can now refine Proposition~\ref{prop:count ceqt} from actions to events.
\subsection{Counting occurrences of events}
We now consider the interaction between events and causal equivalence.
We need some notation first.

\begin{definition}\label{def:count events}
Let $r$ be a path in an LTSI $\mc L$ and let $e$ be an event of $\mc L$.
Let $\cte(r,e)$ be the number of occurrences of transitions $t$ in $r$
such that $t \in e$, minus the number of occurrences of transitions $t$ in $r$ such that $t \in \rev e$.
We define $\cte(r,e)$ by induction on the length of $r$ as follows:
\begin{equation*}
\begin{split}
\cte(\es,e) & = 0 \\
\cte(tr,e) & =
\begin{cases}
\cte(r,e)+1 & \text{if } [t] = e \\
\cte(r,e)-1 & \text{if } [t] = \rev e \\
\cte(r,e) & \text{otherwise}
\end{cases}
\end{split}
\end{equation*}
\end{definition}
%\todo{In Definition~\ref{def:sqeqt} event meant forward event - now changed.
%So $\cte(r,e)$ was only defined for forward events;
%now it is defined for reverse events also.
%Note that Lemma~\ref{lemma:cccount} holds for reverse events also.}

We now show that $\cte(r,e)$ is invariant under causal equivalence.

%\begin{restatable}{lemma}{cccount}\label{lemma:cccount}
\begin{lemma}\label{lemma:cccount}
Let $\mc L$ be a pre-reversible LTSI.
  Let $r \ceqt s$. Then for each event $e$ we have that $\cte(r,e) = \cte(s,e)$.
\end{lemma}
%\end{restatable}
\begin{proof}
  We prove the thesis for $r$ and $s$ being derived by a single
  application of the axioms; the thesis will follow since equality is
  an equivalence relation.

  If $r=r_1tu'r_2$ and $s=r_1ut'r_2$ then we have by definition that
  $t \ind u$.
Hence,
  $[t]=[t']$ and $[u]=[u']$ using Definition~\ref{def:sqeqt simp}. The thesis follows.

  If $r=r_1t\rev tr_2$ and $s=r_1r_2$ (the other case is analogous)
  then the contribution of $t$ and $\rev t$ to $\cte(r,[t])$ (as well
  as to $\cte(r,e)$ for $t \notin e$) is $0$; hence the thesis
  follows.
\end{proof}
Lemma~\ref{lemma:cccount} generalises what was shown for the forward-only setting
in~\cite[Corollary~4.3]{SNW96}.
%
%\begin{restatable}{proposition}{regeqzero}\label{prop:re>0}
\begin{proposition}\label{prop:regeqzero}
If an LTSI is pre-reversible,
then for any rooted path $r$ and any forward event $e$ we have $\cte(r,e) \geq 0$.
\end{proposition}
%\end{restatable}
%\Comment{
\begin{proof}
Let $r$ be a rooted path.
Using PL (Proposition~\ref{prop:PL}), we obtain a coinitial and cofinal forward-only path $s$ such that $s \ceqt r$.  Let $e$ be any forward event.  Clearly $\cte(s,e) \geq 0$.
Hence $\cte(r,e) \geq 0$ by Lemma~\ref{lemma:cccount}.
\end{proof}
%}
%
We can lift independence from transitions to events.

\begin{definition}[Coinitially independent events]\label{def:coind events}
Let events $e,e'$ be \emph{coinitially independent},
written $e \coind e'$, iff there are coinitial transitions $t,t'$ such that
$[t] = e$, $[t'] = e'$ and $t \ind t'$.
\end{definition}
\begin{lemma}\label{lem:coind rev}
Assume an LTSI is pre-reversible. If $e \coind e'$ then we have also
$\rev e \coind e'$.
\end{lemma}
\begin{proof}
Suppose that $e \coind e'$.
Then there are coinitial $t, u$ such that $[t] = e$, $[u] = e'$ and $t \ind u$.
Use SP to complete a diamond with transitions $t' \sqeqt t$, $u' \sqeqt u$.
By PCI we have $\rev t \ind u'$.
Hence $\rev e \coind e'$ as required.
\end{proof}
Thus in pre-reversible LTSIs, $\coind$ is fully determined just considering forward events.
By Lemma~\ref{lem:coind rev},
if we know $e \coind e'$ then we know $\und e \coind \und{e'}$.
% \todo{IVAN: now und defined on events as well (Def. 5.1)}
\begin{proposition}\label{prop:coind irref}
Assume an LTSI is pre-reversible.  Then $\coind$ is irreflexive.
\end{proposition}
\begin{proof}
Suppose for a contradiction that $e \coind e$ for some event $e$.
By Lemma~\ref{lem:coind rev},
we can assume that $e$ is forward.
Then there are coinitial transitions $t,u \in e$ such that $t \ind u$.
We can use SP to complete a square with $t' \sqeqt t$ and $u' \sqeqt u$.
This square is non-degenerate by Lemma~\ref{lem:non-degenerate}.
% All transitions in the square belong to the same event.
% Hence there are two distinct reverse coinitial transitions from the same event,
% contradicting RED.
But now $\rev{t'}$ and $\rev{u'}$ are two distinct coinitial backward transitions with the same label, contradicting BLD (Proposition~\ref{prop:BD}).
\end{proof}
We can slightly strengthen the previous result as follows:
\begin{proposition}\label{prop:coind und}
Assume an LTSI is pre-reversible.
If $t:P \tran\alpha Q$ and $u:R \tran\beta S$ with $[t] \coind [u]$
then $\und\alpha \neq \und\beta$.
\end{proposition}
\begin{proof}
Similar to the proof of Proposition~\ref{prop:coind irref}.
\end{proof}

In pre-reversible LTSIs each event can occur at most once in a rooted path.
\begin{definition}\label{def:NRE}
{\bf No repeated events (NRE)}: for
any rooted path $r$ and any forward event $e$ we have $\cte(r,e) \leq 1$.
\end{definition}
In order to prove NRE we need the following lemmas.
\begin{lemma}[Ladder Lemma]\label{lem:ladder}
Assume an LTSI is pre-reversible.
Suppose that $t:P \tran \alpha Q$ and $t':P' \tran \alpha Q'$ with $t \sqeqt t'$.
Then there is a path $s$ from $Q$ to $Q'$ such that for all $u$ in $s$
% we have $t \sqeqt t'' \ind u' \sqeqt u$ (for some $t'',u'$),
% which we may write $t \sqeqt \ind \sqeqt u$.
we have $[t] \coind [u]$.
\end{lemma}
\begin{proof}
By the definition of $\sqeqt$ there is a ladder of diamonds connecting $t$ to $t'$.
This gives a path $s$ from $Q$ to $Q'$.
Take any $u$ in $s$, and consider the diamond containing $u$.
Let $u'$ be on the opposite side from $u$, so that $u' \sqeqt u$,
and let $t''$ be the rung nearest to $t$, so that $t \sqeqt t''$.
We have $t'' \ind u'$.  Hence result.
\end{proof}
\begin{lemma}\label{lem:cte zero}
Let $\mc L$ be a pre-reversible LTSI.
Suppose $t: P \tran\alpha Q$ and $t': P' \tran\alpha Q'$
with $t \sqeqt t'$, and suppose $r$ is a path from $Q$ to $Q'$.
Then $\cte(r,[t]) = 0$.
\end{lemma}
\begin{proof}
By Lemma~\ref{lem:ladder} there is a path $s$ from $Q$ to $Q'$
such that for all $u$ in $s$ we have $[t] \coind [u]$.
Let $\lab(t)=\alpha$ and $\lab(u)=\beta$.
By Proposition~\ref{prop:coind und} we have  $\und\alpha \neq \und\beta$.
%\todo{$\lab$ defined on events rather than transitions - might be good to define $\lab$ on transitions}
%Since $\coind$ is irreflexive (Proposition~\ref{prop:coind irref}),
%for all $u$ in $s$ we have $[t] \neq [u]$.
%Before we could have had $[\rev t] in r$
Hence $\cte(s,[t]) = 0$,
and by Lemma~\ref{lemma:cccount} $\cte(r,[t]) = 0$ as required.
\end{proof}

\begin{proposition}\label{prop:NRE}
If an LTSI is pre-reversible then it satisfies NRE.
\end{proposition}
\begin{proof}
Let $e$ be a forward event and $r$ be a rooted path from $I$ to $R$, and suppose for a contradiction that
$\cte(r,e) > 1$.
Using PL we can obtain a forward-only path $r'$ from $I$ to $R$ with $r \ceqt r'$.
By Lemma~\ref{lemma:cccount}, $\cte(r',e) > 1$.
Suppose $r'$ contains
$t:P \tran a Q$ followed later by $t':P' \tran a Q'$ where $t, t' \in e$.
Let $r''$ be the portion of $r'$ from $Q$ to $P'$.
% By Lemma~\ref{lem:ladder} there is a path $s$ from $Q$ to $Q'$ such that for all $u$ in $s$
% % we have $t \sqeqt t'' \ind u' \sqeqt u$ (some $t'',u'$).
% we have $[t] \coind [u]$.
% By CC, $s \ceqt r''t'$.
% By Lemma~\ref{lemma:cccount}, $\cte(s,[t']) > 0$, since $r''$ is forward-only.
% Hence there is $u$ in $s$ such that $u \sqeqt t' \sqeqt t$.
% % But then $t \sqeqt t'' \ind u' \sqeqt t$,
% % where $t'',u'$ are as above.
% But then $[t] \coind [u] = [t]$,
% contradicting
% % our assumption that $\coind$ is irreflexive.
% Proposition~\ref{prop:coind irref}.
By Lemma~\ref{lem:cte zero} applied to $t,t'$ and path $r''t'$
we have $\cte(r''t',[t]) = 0$.
This is a contradiction since $r''$ is forward-only.
\end{proof}
NRE was shown in the forward-only setting of occurrence transition systems with independence in~\cite[Corollary~4.6]{SNW96}.
It was also shown in the reversible setting without independence
in~\cite[Proposition~2.10]{PU07a}.
% \il{**** We cannot use Figure~\ref{fig:repeated1} any more, look for another example or to prove the thesis ****}
% \il{In the proposition above, one would expect} $0\leq \cte(r,e) \leq 1$, but this is not guaranteed  in pre-reversible
% LTSIs as Figure~\ref{fig:repeated1} at page \pageref{fig:repeated1} shows paths with repeated events $a$ and $b$. We shall later on 
% add one more axiom (CIRE, Definition~\ref{def:CIRE}) that would ensure that rooted paths have no repeated events
% (see Definition~\ref{def:NRE} and Proposition~\ref{prop:CIRE NRE}).

\begin{example}\label{ex:repeated}
Consider the LTSI in Figure~\ref{fig:repeated}.
% Figure environment removed
Independence holds only between coinitial transitions and is given by closing under BTI and propagating
independence around the corners of diamonds as in PCI whenever possible.
Note however that PCI does not hold, since we have coinitial independent $a$ and $\rev a$-transitions,
contradicting Lemma~\ref{lemma:revnotind}.
As well as BTI, axioms SP and WF hold, so that CC holds.
All $a$-transitions belong to the same event,
and all $b$-transitions belong to the same event.
We have rooted paths where the same event is repeated,
contradicting NRE.
Note also that BLD fails and that $\coind$ is reflexive.\finex
\end{example}

\subsection{Polychotomy}
We now show what we call \emph{polychotomy},
which states that if forward events do not cause each other and are not in conflict,
then they must be independent.
This will help us to relate the different notions of causal safety and liveness (Section~\ref{sec:CSCL}).
We first define causality and conflict relations on forward events.
\begin{definition}[Causality relation on forward events]\label{def:ordering}
  Let $\mc L$
  %= (\Proc,\Lab,\tran{},\ind)$
  be an LTSI.
Let $e,e'$ be forward events of $\mc L$.
Let $e \leq e'$ iff
for all rooted paths $r$, if $\cte(r,e') > 0$ then $\cte(r,e) > 0$.
As usual $e < e'$ means $e \leq e'$ and $e \neq e'$.
If $e < e'$ we say that
% $e'$ is an \emph{effect} of $e$ and
$e$ is a \emph{cause} of $e'$.
%The notions of $<$, cause and effect
%apply to transitions as well:
%Let $\pi=sts't's''$ be a path with transitions $t,t'$. We say that $t<t'$ if $[t]<[t']$. Also
%$t'$ is an effect of $t$ (or
%$t$ causes $t'$) if {\bf every} rooted path that contains $t'_1\in[t']$ also contains
%$t_1\in [t]$.
\end{definition}
As expected, the causality relation is a partial ordering (i.e., a reflexive, transitive and antisymmetric relation).
\begin{lemma}\label{lem:po}
If an LTSI is pre-reversible then
$\leq$ is a partial ordering on events.
\end{lemma}
\begin{proof}
Reflexivity and transitivity are immediate.
For antisymmetry, suppose that $e_1 \leq e_2$ and $e_2 \leq e_1$,
where $e_1,e_2$ are forward events.
Then for all rooted $r$, $\cte(r,e_1) >0$ iff $\cte(r,e_2) >0$.
Since the LTSI is pre-reversible, by Proposition~\ref{prop:regeqzero},
for all rooted $r$, $\cte(r,e_1) \geq 0$ and $\cte(r,e_2) \geq 0$.
Let $r$ be a shortest rooted path such that $\cte(r,e_1) >0$.
We can use WF to show that $r$ must exist.
Then $\cte(r,e_2) > 0$.
Also $r = r't$, where $\cte(r',e_1) = 0$ (otherwise $r$ would not be a shortest path) and so $\cte(r',e_2) = 0$.
We see that both $[t] = e_1$ and $[t] = e_2$,
showing that $e_1 = e_2$ as required.
\end{proof}
In~\cite{vGV97,PU07a}, orderings on forward events have been defined using forward-only rooted paths;
in fact, the definitions coincide for pre-reversible LTSIs.
% satisfying SP, BTI, WF and PCI.
\begin{definition}[\cite{vGV97,PU07a}]\label{def:ordering fwd}
  Let $\mc L$ % = (\Proc,\Lab,\tran{},\ind)$
  be an LTSI.
Let $e,e'$ be forward events of $\mc L$.
Let $e \leqf e'$ iff
for all rooted forward-only paths $r$,
if  $\cte(r,e') >0$
%$r$ contains a representative of $e'$
then $\cte(r,e) >0$.
%$r$ also contains a representative of $e$.
\end{definition}
\begin{lemma}\label{lem:ordering}
For any LTSI, and any forward events $e,e'$,
$e \leq e'$ implies $e \leqf e'$.
If an LTSI is pre-reversible then
$e \leqf e'$ implies $e \leq e'$.
\end{lemma}
\begin{proof}
Straightforward using PL and Lemma~\ref{lemma:cccount}.
\end{proof}
% We start by defining a conflict relation on events.
\begin{definition}\label{def:conflict}
Two forward events $e,e'$ are in \emph{conflict}, written $e \cf e'$,
if there is no rooted path $r$ such that $\cte(r,e) >0$ and $\cte(r,e') > 0$.
\end{definition}
Much as for orderings, conflict on events has been defined previously
using forward-only rooted paths~\cite{vGV97,PU07a};
in fact, the definitions coincide for pre-reversible LTSIs.
We omit the details.

We can now introduce the main result of this section.
\begin{definition}[Polychotomy]\label{def:poly}
% Suppose that WF, SP, PL, CC, PCI, NRE hold.
Let $\mc L$ be a pre-reversible LTSI.
We say that $\mc L$ satisfies \emph{polychotomy} if whenever
$e,e'$ are \emph{forward} events, then exactly one of the following holds: %\\
\begin{enumerate}
\item $e = e'$;\quad%\quad
\item $e < e'$;\quad%\quad %(in every rooted forward-only path containing $t' \in e'$ there is an earlier $t \in e$)
\item $e' < e$;\quad%\quad
\item $e \cf e'$; or\quad %\quad %(no rooted forward-only path contains both $t \in e$ and $t' \in e'$)
\item $e \coind e'$. %(there are \emph{coinitial} $t \in e$ and $t' \in e'$ such that $t \ind t'$)
\end{enumerate}
\end{definition}
%
%\begin{restatable}[Polychotomy]{lemma}{poly}\label{prop:poly}
\begin{proposition}[Polychotomy]\label{prop:poly}
Assume an LTSI is pre-reversible.
Then polychotomy holds.
% Let $e,e'$ be \emph{forward} events.  Then exactly one of the following holds:
% \begin{enumerate}
% \item
% $e = e'$
% \item
% $e < e'$ %(in every rooted forward-only path containing $t' \in e'$ there is an earlier $t \in e$)
% \item
% $e' < e$
% \item
% $e$ and $e'$ are in conflict %(no rooted forward-only path contains both $t \in e$ and $t' \in e'$)
% \item
% $e \coind e'$ %(there are \emph{coinitial} $t \in e$ and $t' \in e'$ such that $t \ind t'$)
% \end{enumerate}
% \todo{If true, this conjecture implies Conjecture~\ref{conj:poly coinitial}.}
\end{proposition}
%\end{restatable}
\begin{proof}
%Let event $e,e'$ have labels $a,b$ respectively ($a$ and $b$ may or may not be equal).
Consider two forward events $e$ and $e'$ which may or may not be equal.

We first check mutual exclusivity.
Suppose $e = e'$.
Then $e < e$ is impossible by definition of $<$.
%WF \todo{$e < e$ implies $e \neq e$ by definition - no need for WF}.
Also $e$ cannot be in conflict with itself (we can use WF to show that there is at least one rooted path).
Finally, $e \coind e$ is impossible by Proposition~\ref{prop:coind irref}.
% implies that there are two distinct coinitial transitions $t_1,t_2 \in e$
% which are independent.
% By SP we can complete the square to get a forward-only path with two occurrences of $e$,
% which is impossible by NRE \todo{uses WF?}\todo{Ivan: don't think so}.
From now on we assume $e \neq e'$.

Next suppose $e < e'$.
We can rule out $e' < e$ using %WF.
Lemma~\ref{lem:po}.

Using Lemma~\ref{lem:ordering}, we know that $e \ltf e'$, hence there must be some rooted forward-only path with $e$ followed by $e'$ (WF ensures at least one rooted path exists),
and so $e$ and $e'$ are not in conflict.
Finally $e \coind e'$ implies that there are two coinitial transitions $t \in e$, $t' \in e'$
which are independent.
Using SP to complete the square we see that $e < e'$ is impossible by NRE,
which holds by Proposition~\ref{prop:NRE}.

Similarly we see that $e' < e$ implies that $e$ and $e'$ are not in conflict and not independent.

Next suppose that $e \cf e'$.
If $e \coind e'$
then there are two coinitial transitions $t \in e$, $t' \in e'$
which are independent.
Using SP to complete the square and WF we see that we have a rooted forward-only path
containing occurrences of both $e$ and $e'$ contradicting them being in conflict.

Suppose that none of (1)-(4) hold.
We must show (5).
Since $e,e'$ do not conflict, there is a
%forward-only path
rooted path $r$ starting at some irreversible $I$ such that $\cte(r,e) > 0$ and $\cte(r,e') > 0$. If more than one such path exists, choose one of minimal length.
 W.l.o.g.~suppose that $r$ finishes with $t' \in e'$ at $P$.
Since not $e < e'$, using Lemma~\ref{lem:ordering} also $e \ltf e'$ does not hold; hence there is another forward-only path~$r'$ from some irreversible $I'$
finishing with $t'' \in e'$ at $Q$
such that $\cte(r',e) = 0$.
% and not containing $a$.
% There is a ladder of diamonds connecting the two $b$s creating a path $s$ from $Q$ to $P$.
By Lemma~\ref{lem:ladder} there is a path $s$ from $Q$ to $P$
such that $e' \coind [u]$ for every $u$ in $s$.
Using Proposition~\ref{prop:unique irrev} we deduce that $I' = I$.
By CC $r \ceqt r's$ and so by Lemma~\ref{lemma:cccount} $\cte(s,e) > 0$ and $s$ must contain $u \in e$,
% We then deduce that there is a diamond of $a$ and $b$ in the ladder,
yielding $e \coind e'$ as required.
\end{proof}
