\section{Structured notions of independence}\label{sec:coinitial}
In this section we consider two structured notions of independence, namely independence defined on coinitial transitions only and independence determined by labels only.
To this end, we introduce `structural axioms' in Definitions~\ref{def:coinitial LTSI},~\ref{def:CLG} and~\ref{def:LG}.  These have a different status from the axioms already introduced: rather than expressing fundamental properties that are desirable in LTSIs,
they are properties that hold in various reversible formalisms %calculi
(as we shall see in Section~\ref{sec:casestudies}), are easy to verify,
and can be used to derive other
%particular
axioms in a generic fashion.

\subsection{Coinitial independence}
In this section we discuss coinitial LTSIs, defined as follows,
and their relationship with LTSIs in general.

\begin{definition}\label{def:coinitial LTSI}
  {\bf Independence is coinitial (IC)}:
  for all transitions $t,u$,
if $t \ind u$ then $t$ and $u$ are coinitial.
\end{definition}

We say that an LTSI $\mc L$ is coinitial if it satisfies IC.
We also say that its independence relation $\ind$ is coinitial.

Coinitial independence is of interest since in many cases it is easier
to define independence only on coinitial transitions. Indeed,
coinitial independence arises, e.g., from the notions of concurrency
in~\cite[Definition 7]{DK04} for RCCS and in~\cite[Definition 5]{LaneseNPV18} for Core Erlang.

The next example satisfies IC and all and only the properties in Figure~\ref{fig:diagprerev1} implied by it. In particular, it shows that IC does not imply CL$\indt$, CL$\ci$, or
CS$\indt$ (this last follows from Proposition~\ref{prop:CSi CLci}).
\begin{example}\label{ex:IC}
Consider the LTSI in Figure~\ref{fig:IC}.
% Figure environment removed
All independence is coinitial as generated by BTI and PCI,
and the LTSI is pre-reversible.
%[Would be good to use the tool to check that this is pre-reversible.]
%\il{[IL: Done, it is pre-reversible, see exICwithtool.png]}
There are three events, which we denote by $e_a,e_b,e_c$,
with labels $a,b,c$, respectively. 
%Moreover, there are reverse events,
%for example the event $\rev{e_b}$ for the transitions with label $b$.
% Note that $e_a \coind e_b$, $e_b \coind e_c$ and $e_c \coind e_a$.
CL$\indt$ fails:
let $t: P \tran a Q$ and let $r$ from $Q$ to $R$ be $\rev b b$ (dashed transitions).
We have $\cte(r,e_b) = 0$;
however $a$ cannot be reversed at $R$, as CL$\indt$ would yield.
Also CS$\indt$ fails:
let $t: P \tran a Q$ and let $r'$ be $\rev c\, \rev b$ from $Q$ to $S$ (bold transitions).
After $r'$, $\rev a$ is possible.
However $\rev t$ is not independent with the $\rev b$ transition,
as CS$\indt$ would yield.
%EIT fails: start at the state reached by performing the leftmost $b$ transition.
%We can reach $R$ by $\rev b cab$ and also by $ac$.
%However we cannot reach $R$ by performing $ca$, as EIT would yield.
Also CL$\ci$ fails:
let $t_0$ be the $a$ following the leftmost $b$, and let $r''$ be the $c$
transition with target $R$.  We have $e_a \coind e_c$.
However $a$ cannot be reversed at $R$, as CL$\ci$ would yield.
\finex
\end{example}

Coinitial independence is inconsistent with the axiom IRE,
showing that IRE is only appropriate for the setting of general,
rather than coinitial independence:
\begin{proposition}\label{prop:IC IRE}
Let a pre-reversible LTSI have a non-empty independence relation,
and satisfy IC.  Then IRE does not hold.
\end{proposition}
\begin{proof}
Suppose for a contradiction that IRE holds.
Since the independence relation is non-empty and IC holds,
we have $t \ind u$ with $t,u$ coinitial.
By SP and PCI we can complete a diamond with $t'\sqeqt t$, $u' \sqeqt u$.
Since $t' \sqeqt t \ind u$ we deduce by IRE that $t' \ind u$.
However $t'$ and $u$ are not coinitial,
contradicting IC.
\end{proof}

We define
a mapping~$\mathrel{c}$ restricting general independence to coinitial transitions and
a mapping~$\mathrel{g}$ extending independence along events.
% so to satisfy IRE and IEC.
\begin{definition}\label{def:gen coinit}
  Given an LTSI %$(\Proc,\Lab,\tran{},\ind)$
  $\mc L$, define
$t \mathrel{g(\ind)} u$ iff \mbox{$t \sqeqt t' \ind u' \sqeqt u$}
for some $t',u'$.
Furthermore, define $t \mathrel{c(\ind)} u$
iff $t \ind u$ and $t,u$ are coinitial.
\end{definition}

  We extend $\mathrel{c}$ and $\mathrel{g}$ to LTSIs $(\Proc,\Lab,\tran{},\ind)$:
  they behave as the identity of the first three components, and as expected on the fourth. Similarly, we write $\mathrel{c(\sqeqt)}$ and $\mathrel{g(\sqeqt)}$ for the equivalence relations in $\mathrel{c(\mc L)}$ and $\mathrel{g(\mc L)}$, respectively.
  
We now show that $c$ and $g$ play well with events.
\begin{lemma}
 Given an LTSI $\mc L$, ${\sqeqt} = {c(\sqeqt)}$.
\end{lemma}
\begin{proof}
Follows by noticing that the definition of event only exploits independence on coinitial transitions.
\end{proof}

\begin{lemma}
 Given an LTSI $\mc L$, $t \sqeqt u$ implies $t \mathrel{g(\sqeqt)} u$.
\end{lemma}
\begin{proof}
By definition of $\sqeqt$, noticing that ${\ind} \subseteq {g(\ind)}$.
\end{proof}

\begin{lemma}
 Given a pre-reversible LTSI $\mc L$, $t \mathrel{g(\sqeqt)} u$ implies $t \sqeqt u$.
\end{lemma}
\begin{proof}
By definition of $\sqeqt$, we have $t \, g(\sqeqt) \, u$ if there is a chain of commuting squares connecting $t$ and $u$. Thanks to ID (which holds in pre-reversible LTSIs) all such squares are commuting squares in $\mc L$, hence $t \sqeqt u$ as desired.
\end{proof}

We can now study the impact of $c$ and $g$ on the axioms satisfied by the LTSI to which they are applied.
%
%\begin{restatable}{proposition}{gencinit}\label{prop:gen coinit}
\begin{proposition}\label{prop:gen coinit}
Let $\mc L = (\Proc,\Lab,\tran{},\ind)$ be a pre-reversible LTSI. %Then
\begin{enumerate}
\item
if $\mc L$ is coinitial and satisfies CIRE then $c(g(\ind)) = {\ind}$;
\item
if $\mc L$ satisfies IRE and IEC then $g(c(\ind)) = {\ind}$;
\item\label{item:gprop}
If $\mc L$ is coinitial and satisfies CIRE then
$g(\mc L)$
is a pre-reversible LTSI and satisfies IRE and IEC.
\item
if $\mc L$ satisfies IRE then
$c(\mc L)$
is a pre-reversible coinitial LTSI and satisfies CIRE.
\end{enumerate}
\end{proposition}
%\end{restatable}
\begin{proof}
\begin{enumerate}
\item
Clearly ${\ind} \subseteq c(g(\ind))$.
For the converse,
suppose $t \sqeqt t' \ind u' \sqeqt u$ and $t',u'$ are coinitial and $t,u$ are coinitial.
Then $t \ind u$ by CIRE.
\item
Suppose $t \ind u$.
By IEC we have $t \sqeqt t' \ind u' \sqeqt u$ with $t',u'$ coinitial.
Hence $t \mathrel{g(c(\ind))} u$.
Conversely, suppose $t \mathrel{g(c(\ind))} u$.
Then $t \ind u$ by IRE.
\item
%Suppose that $t \ind u$.  Then by SP there is a diamond with $t',u'$ and by PCI we have $\rev t \ind u'$, $\rev {u'} \ind \rev {t'}$ and $t' \ind \rev{u'}$.
%Furthermore $t \sqeqt t'$ and $u \sqeqt u'$.
Suppose $t \mathrel{g(\ind)} u$ and $t,u$ are coinitial.
Then by CIRE $t \ind u$.  So we can use SP for $\ind$ to complete the diamond.
Hence SP holds for $\mc L'$.

Clearly PCI holds for
$g(\mc L)$
since $g(\ind)$ and $\ind$ agree on coinitial transitions by CIRE.

For IRE, suppose $t' \sqeqt t \mathrel{g(\ind)} u \sqeqt u'$.  Then clearly $t' \mathrel{g(\ind)} u'$.

Finally, for IEC suppose $t \mathrel{g(\ind)} u$.  Then
$t \sqeqt t' \ind u' \sqeqt u$ with $t',u'$ coinitial, which is exactly what is needed
for IEC.
\item
Immediate.
\qedhere
\end{enumerate}
\end{proof}
Thanks to Proposition~\ref{prop:gen coinit}, we can extend
a coinitial pre-reversible LTSI satisfying CIRE in a canonical way to
a pre-reversible LTSI satisfying IRE and IEC.

Note that $g(\mc L)$ satisfies IRE (and hence ECh) by construction, since $t \mathrel{g(\ind)} u \sqeqt t'$ implies $t \mathrel{g(\ind)} t'$. Conditions in Proposition~\ref{prop:gen coinit}, item~(\ref{item:gprop}) are only needed for the other properties.

%will hold for the LTSI \il{$g(\mc L)$} generated from coinitial $\mc
%L$ using $g$ in Proposition~\ref{prop:gen coinit}, since


\subsection{Label-generated independence}
In some reversible calculi (such as RCCS) independence of coinitial transitions is defined purely
by reference to the labels.

\begin{definition}\label{def:CLG}
%Let $\mc L = (\Proc,\Lab,\tran{},\ind)$ be an LTSI.
  %Then $\mc L$ is
  {\bf Coinitial label-generated (CLG)}: if there is an irreflexive binary relation $I$ on $\Lab$ such that for any
transitions $t:P \tran\alpha Q$ and $u:P \tran\beta R$
we have $t \ind u$ iff $t$ and $u$ are coinitial and $I(a,b)$, where $a$ and $b$ are the underlying labels
$a = \und\alpha$, $b = \und\beta$.
\end{definition}

If this is the case then %it is a simple matter to verify
the axioms IC, PCI and CIRE hold by construction.
\begin{proposition}\label{prop:CLG}
%[replaces Proposition~\ref{prop:underlying}]
If an LTSI is CLG then it satisfies IC, PCI and CIRE.
\end{proposition}
\begin{proof}
Straightforward, noting for PCI and CIRE that labels on opposite sides of a diamond of transitions must be equal.
\end{proof}


%% \begin{proposition}\label{prop:underlying}
%%   %  Let $\mc L = (\Proc,\Lab,\tran{},\ind)$ be
%% \todo{Replace with the one below?}
%% \il{Consider} a coinitial LTSI.
%% Suppose that $I$ is an irreflexive binary relation on $\Lab$ such that for any
%% coinitial transitions $t:P \tran\alpha Q$ and $u:P \tran\beta R$
%% we have $t \ind u$ iff $I(a,b)$, where $a$ and $b$ are the underlying labels
%% $a = \und\alpha$, $b = \und\beta$.
%% Then $\mc L$ satisfies PCI and CIRE.
%% \end{proposition}
%% \begin{proof}
%% Straightforward, noting that labels on opposite sides of a diamond of transitions must be equal.
%% \end{proof}
Note that $I$ must be irreflexive, since $\ind$ is irreflexive by definition. Even more, we already have seen that for a pre-reversible LTSI there cannot be independent coinitial transitions $t$, $u$ with the same underlying label (as a consequence of Lemma~\ref{lemma:revnotind} and BLD).


\begin{definition}\label{def:LG}
  {\bf Label-generated (LG)}: if there is an irreflexive binary relation $I$ on $\Lab$ such that for any
transitions $t:P \tran\alpha Q$ and $u:R \tran\beta S$
we have $t \ind u$ iff $I(a,b)$, where $a$ and $b$ are the underlying labels
$a = \und\alpha$, $b = \und\beta$.
\end{definition}
\begin{proposition}\label{prop:LG}
If an LTSI is LG then it satisfies PCI, IRE and RPI.
\end{proposition}
\begin{proof}
Straightforward.
\end{proof}
Note that LG does not imply IEC, in view of the following example.
\begin{example}\label{ex:LG}
Consider the LTSI with two transitions $t:P \tran a Q$ and $u:R \tran b S$, where all states are distinct (as in Example~\ref{ex:IRE1}) and $a \neq b$.
Let independence be generated by the relation $I = \{(a,b)\}$.
Then LG holds, but not IEC, since $t \ind u$ but not $[t] \coind [u]$.\finex
\end{example}

However, LG is compatible with IEC, in view of the following example.
\begin{example}\label{ex:LG+IEC}
Let $t:P \tran a Q$, $u:P \tran b R$,
$u':Q \tran b S$, $t':R \tran a S$,
where all states are distinct and $a \neq b$.
Let independence be generated by the relation $I = \{(a,b)\}$.
Then both LG and IEC hold.
% Here we have two forward events, labelled with $a$ and $b$ respectively.
% Axioms SP, BTI, WF, PCI, IRE and IEC hold.
However IC fails.
\finex
\end{example}

All the axioms and properties we have considered previously are closed under
%taking
disjoint unions of LTSIs, defined as follows.
\begin{definition}[Disjoint union of LTSIs]
Take two LTSIs $(\Proc_1,\Lab_1,\tran{}_1,\ind_1)$ and $(\Proc_2,\Lab_2,\tran{}_2,\ind_2)$. Their disjoint union is $(\Proc_1 \cup \Proc_2,\Lab_1 \cup \Lab_2,\tran{}_1\cup\tran{}_2,\ind_1\cup\ind_2)$ provided that $\Proc_1 \cap \Proc_2 = \emptyset$, undefined otherwise.
\end{definition}
However LG and CLG are not necessarily closed under disjoint unions of LTSIs,
in view of the following examples.
% \todo{examples are useful for the diagram of implications} 
\begin{example}\label{ex:IRE+IEC}
Take the disjoint union of the LTSI of Example~\ref{ex:LG+IEC}
together with a further transition
$T \tran a U$ with an empty generator relation
(this component satisfies LG).
Then LG fails; however IEC and IRE still hold.
\finex
\end{example}
\begin{example}\label{ex:IC+CIRE}
Take the disjoint union of the LTSI of
%Example~\ref{ex:CLG CSi}
Example~\ref{ex:prerev not CSi}
(which satisfies CLG)
together with further transitions $T \tran a U$ and $T \tran b V$
with an empty generator relation
(this component satisfies CLG).
Then CLG fails; however IC and CIRE still hold.
% along with CS$\indt$.
\finex
\end{example}

The mapping $g$ converts an LTSI satisfying CLG into one satisfying LG+IEC.
The mapping $c$ converts an LTSI satisfying LG into one satisfying CLG.
Note that there is an alternative way to convert an LTSI satisfying CLG into one satisfying LG:
simply use the relation $I$ applied to any pair of transitions.
This will in general create more independent transitions than using $g$,
and so the result may not satisfy IEC.



% \begin{remark}\label{rem:coinit to gen}
% As a particular case of this,
% suppose that $\mc L = (\Proc,\Lab,\tran{},\ind)$ is a coinitial LTSI satisfying SP, BTI and WF,
% and suppose that there is a relation $I$ such that $t \ind u$ iff $I(a,b)$,
% where $a,b$ are the underlying labels of $t,u$.
% Then by Proposition~\ref{prop:underlying}, $\mc L$ satisfies PCI and CIRE.
% Let $\mc L' = (\Proc,\Lab,\tran{},g(\ind))$
% where $g(\ind)$ is as in Definition~\ref{def:gen coinit}.
% By Proposition~\ref{prop:gen coinit}, $\mc L'$ is pre-reversible and satisfies IRE and IEC.
% Note that if $t \mathrel{g(\ind)} u$ then $I(a,b)$, where $a,b$ are the respective underlying labels of $t,u$.
% However the converse does not necessarily hold;
% we also require that $t \sqeqt t'$, $u \sqeqt u'$ where $t',u'$ are coinitial.
% \begin{remark}\label{rem:CS CL coinit gen}
%\todo{add references to results used?}
%
%As a consequence of Proposition~\ref{prop:gen coinit} and results from Section~\ref{sec:further},
\subsection{Relating different forms of CS/CL}\label{subsec:comparison}
We now  discuss the relationships between different forms of CS/CL 
and consider which ones to work with in particular reversible settings. 
The starting point is how independence is or can be defined in such settings,
and whether it is general or coinitial. We explain how structural axioms and results of this section, 
together with our axioms, can be used to arrive at the most appropriate causal safety and liveness 
properties for such reversible settings. 
%beyond a basic comparison in Section~\ref{sub:comparing}. 

We can sometimes move between
LTSIs satisfying CS$\ci$ and CL$\ci$ (or equivalently  CS$_<$ and CL$_<$), all defined in terms of coinitial independence, and LTSIs satisfying CS$\indt$ and
CL$\indt$, which are based on general independence, using mappings $c$ and $g$.
Thus, if we have a coinitial pre-reversible LTSI $\mc L$ satisfying CIRE then CS$\ci$ and CL$\ci$
hold (using Theorems~\ref{thm:CS coind} and~\ref{thm:CL coind}, respectively).  
The LTSI $g(\mc L)$ is pre-reversible and satisfies IRE and IEC 
by Proposition~\ref{prop:gen coinit}. This will satisfy CS$\indt$ and CL$\indt$ as a result of 
applying Theorems~\ref{thm:CS} and~\ref{thm:CL}, respectively.
It will also satisfy CS$\ci$ and CL$\ci$. 
%
Conversely, if we have a general pre-reversible LTSI $\mc L'$ satisfying IRE then CS$\indt$ and CL$\indt$
hold by Theorems~\ref{thm:CS} and~\ref{thm:CL}, respectively. The LTSI $c(\mc L')$ 
is a coinitial pre-reversible LTSI satisfying CIRE.
This will satisfy CS$\ci$  and CL$\ci$.

Intuitively, one can think of coinitial independence as a compact way
of representing general independence (provided that this is
well-behaved, in that it satisfies IRE and IEC), and $c$ and $g$ as
ways of moving between the two representations (Proposition~\ref{prop:gen coinit}). 
CS$\indt$ and CL$\indt$ work on the general
representation only, since they check independence between transitions that
may be far apart. The other two forms of CS/CL can instead work with
both the representations, and they are equivalent (Figure~\ref{fig:simpleCSCL}). 
Moreover, once we have LTSI with general independence we can work immediately with 
CS$\indt$ and CL$\indt$. On the other hand, when independence is coinitial, we
need to instantiate the notion of event, and understand whether events are causally dependent or coinitial independent,
%develop the notion of events
% \todo{Iain: not sure about this distinction}
before we can use the other two notions of CS/CL.
The choice between CS$_<$/CL$_<$ and CS$\ci$/CL$\ci$ depends on whether independence 
or ordering is more easily or naturally defined on events. 

In some process calculi and programming languages, as can be seen in the next section,  
independence can be defined in terms of transition labels, which gives us structural axioms 
CLG and LG. So, to show CS/CL we tend to show CLG (RCCS, CCSK, HO$\pi$, Erlang) 
or we prove CIRE (R$\pi$, reversible occurrence nets) and then use $g$. 
Alternatively, we show LG ($\pi$IH).

Note that whether or not CLG/LG can be applied to a reversible formalism may depend on the level of abstraction adopted in the transition labels.


\Comment{
We provide here some insights on the relations between different forms
of CS/CL, beyond the comparison in
Section~\ref{sub:comparing}. Indeed, we can sometimes move between
LTSIs satisfying \todo{CS$\ci$ and CL$\ci$ (or equivalently  CS$_<$ and CL$_<$) } and LTSIs satisfying CS$\indt$ and
CL$\indt$, using mappings $c$ and $g$.

\todo{Thus, if we have a coinitial pre-reversible LTSI $\mc L$ satisfying CIRE then CS$\ci$ and CL$\ci$
hold (using Theorems~\ref{thm:CS coind} and~\ref{thm:CL coind}, respectively).  
The LTSI $g(\mc L)$ is pre-reversible and satisfies IRE and IEC 
by Proposition~\ref{prop:gen coinit}. This will satisfy CS$\indt$ and CL$\indt$ as a result of 
applying Theorems~\ref{thm:CS} and~\ref{thm:CL}, respectively.
It will also satisfy CS$\ci$ and CL$\ci$. 
%
Conversely, if we have a general pre-reversible LTSI $\mc L'$ satisfying IRE then CS$\indt$ and CL$\indt$
hold by Theorems~\ref{thm:CS} and~\ref{thm:CL}, respectively. The LTSI $c(\mc L')$ 
is a coinitial pre-reversible LTSI satisfying CIRE.
This will satisfy CS$\ci$  and CL$\ci$.}

Intuitively, one can think as coinitial independence as a compact way
to represent general independence (provided that this is
well-behaved, in that it satisfies IRE and IEC), and $c$ and $g$ as
ways of moving between the two representations (as shown in Proposition~\ref{prop:gen coinit}). CS$\indt$ and
CL$\indt$ can be thought of as notions that work on the general
representation only, since they check independence between transitions that
may be far away. The other two forms of CS/CL can instead work with
both the representations, and they are equivalent as shown in
Figure~\ref{fig:simpleCSCL}.
\il{On the other side CS$\indt$ and CL$\indt$
directly rely on independence, hence they are \todo{easier} to work with.  Thus,
one can decide to work either with CS$\indt$ and CL$\indt$ using a
general notion of independence, possibly generated from a coinitial
one using $g$, or on a more compact coinitial independence,
but using the two more \todo{demanding} notions of CS/CL. The choice between
CS$_<$/CL$_<$ and CS$\ci$/CL$\ci$ depends on whether independence or
ordering is more easily defined on events.  }

\todo{In the case studies, to show CS/CL we tend to show CLG (RCCS, CCSK, Ho$\pi$, Erlang) or CIRE (R$\pi$, reversible occurrence nets) and then use $g$. Or show LG ($\pi$IH).

Note that whether CLG/LG can be applied to a reversible formalism may depend on the level of abstraction adopted in the transition labels.}
}

% \end{remark}
%
% Of course CS and CL will not hold, as all independence is coinitial.
% \end{remark}

% The further axioms RPI, PI and IRE rely on independence being defined for
% non-coinitial transitions, unlike the basic axioms BTI, SP, WF.
% However some of the consequences of the further axioms can be formulated
% with independence only defined for coinitial transitions, e.g. FD, RED, NRE.
% So we might want to present axioms for the two scenarios (coinitial or general),
% with the possibility that the more general scenario has easier proofs
% or simpler axioms.
% The coinitial scenario corresponds to the development for most reversible calculi
% in the literature.

% We would like examples to show that CC does not imply CS or CL.
% \begin{example}\label{ex:coinitial CC not CL}
% Consider the ltr in Figure~\ref{fig:coinitial repeated1}.
% % Figure environment removed
% CC holds (with SP, BTI, WF, PCI).
% Also CS$\ci$ seems to hold using Definition~\ref{def:coind safe live}.
% However CL$\ci$ fails using Definition~\ref{def:coind safe live}.
% Indeed, consider a path $\ptran {abb}$ from the start (either possibility).
% Since $[a] \ind [b]$ according to Definition~\ref{def:coind events},
% we should be able to reverse $a$, but this is not possible.
% Also CIRE and NRE fail.
% \end{example}

