\documentclass[onefignum, onetabnum,onealgnum,onethmnum, oneeqnum]{siamonline190516}
\input{nd_preamble.tex}  
\usepackage[symbol]{footmisc}
\renewcommand{\thefootnote}{\arabic{footnote}}
\usepackage{ericmath}
\usepackage[createShortEnv]{proofs}


\definecolor{fig1green}{rgb}{0.01, .7, 0.1}
\definecolor{fig1orange}{rgb}{0.7, 0.34, 0.005}

\newcommand{\dowhy}{\sct{doWhy}}
\newcommand{\anova}{\sct{anova}}
\newcommand{\vm}{\sct{VM}}
\newcommand{\hsic}{\sct{Hsic}}
\newcommand{\cdcorr}{\sct{cDcorr}}
\newcommand{\pdcorr}{\sct{pDcorr}}
\newcommand{\Noccco}{\sct{Noc3o}}
\newcommand{\Manova}{\sct{manova}}
\newcommand{\manova}{\sct{manova}}
\newcommand{\cmanova}{\sct{cManova}}
\newcommand{\whell}{\sct{wHell}}
\newcommand{\kcd}{\sct{kernelCDTest}}
\newcommand{\causal}{\sct{Causal}}
\newcommand{\gcm}{\sct{GCM}}
\newcommand{\rcit}{\sct{RCIT}}
\newcommand{\rcot}{\sct{RCoT}}
\newcommand{\ccdcorr}{\causal~\cdcorr}

\title{
Learning sources of variability from high-dimensional observational studies
}

\author{
    Eric W.~Bridgeford$^{1,\dagger}$, 
    Jaewon Chung$^1$,
    Brian Gilbert$^1$,
    Sambit Panda$^1$,
    Adam Li$^3$,
    Cencheng Shen$^2$,
    Alexandra Badea$^4$,
    Brian Caffo$^1$,
    Joshua T.~Vogelstein$^1$.
    \thanks{
     $^1$ Johns Hopkins University, $^2$ University of Delaware, $^3$ Columbia University, $^4$ Duke University
    $^\dagger$ Corresponding author:
      Eric W.~Bridgeford (\email{ebridge2@jhu.edu}).}
}


\makeatletter
\def\thanks#1{\protected@xdef\@thanks{\@thanks
        \protect\footnotetext{#1}}}
\makeatother
\makeatletter
\renewcommand\@biblabel[1]{#1.}
\makeatother
\begin{document}

\maketitle

\begin{abstract}
Causal inference studies whether the presence of a variable influences an observed outcome. As measured by quantities such as the ``average treatment effect,'' this paradigm is employed across numerous biological fields, from vaccine and drug development to policy interventions. Unfortunately, the majority of these methods are often limited to univariate outcomes. Our work generalizes causal estimands to outcomes with any number of dimensions or any measurable space, and formulates traditional causal estimands for nominal variables as causal discrepancy tests. We propose a simple framework under which universally consistent conditional independence tests are universally consistent causal discrepancy tests. Numerical experiments illustrate that our method, \ccdcorr, leads to improvements in both finite sample validity and power when compared to existing strategies when the assumptions of this framework are violated. %We provide real data examples from neuroimaging and genomics datasets to demonstrate the utility and flexibility of our proposed methodologies. 
Our methods are all open source and available at \href{github.com/ebridge2/cdcorr}{github.com/ebridge2/cdcorr}.
% We believe that the results in this paper suggest the utility of the framework described for causal structure learning and conditional discrepancy testing in observational data.% We provide all of our code and results open source in a leading package for causal inference, called \dowhy.
\end{abstract}


% Figure environment removed

\section{Introduction}
Automatic 3D reconstruction of clothed humans using image inputs has gained increasing significance due to its potential applications in a wide array of AR/VR scenarios. High-fidelity reconstructions typically depend on sophisticated capture systems, which are developed with dense camera arrays~\cite{collet2015high,joo2015panoptic,joo2018total}, programmable light-stages~\cite{Vlasic2009, guo2019relightables}, and depth sensors~\cite{newcombe2011kinectfusion,DoubleFusion,BodyFusion,dou2016fusion4d,newcombe2015dynamicfusion}. However, stringent capture environments equipped with complex hardware pose significant challenges for consumer-level applications.


In this context, considerable research effort has been dedicated to developing methods that allow for more flexible capture configurations, such as utilizing a few RGB inputs. Among these works, learning implicit functions \cite{iccv2020PIFu, saito2020pifuhd, hong2021stereopifu} has proven effective in achieving highly detailed reconstructions by integrating the advancements of deep neural networks. These methods employ large multi-layer perceptrons (MLPs) to predict the occupancy probability or truncated signed distance function (TSDF) value of every queried 3D point based on its associated local feature, which is extracted from images. They can recover a continuous surface at arbitrary resolutions without topology restrictions.


However, in typical MLP-based implicit networks, the occupancy or TSDF value at each location is solved independently with planar image features, rendering them less capable of addressing challenging cases such as occlusions. Consequently, these methods suffer from generalization and robustness issues, particularly when tackling strong occlusions caused by large motion or multiple interacting humans. 
Some follow-up studies  \cite{zheng2021deepmulticap,zheng2021pamir,huang2020arch} utilize an extra geometric model, SMPL~\cite{Loper2015}, to improve robustness by introducing strong shape priors. 
Their success typically relies on the assumption of geometrical similarity \cite{huang2020arch} between the shape prior and target reconstruction, making them intractable for handling complex cases with loose clothes and sensitive to errors in SMPL model fitting.



%\ping{this paragraph sounds like `TSDF is better than MLP/SMPL, and we use TSDF to solve the problem'. But in Sec 3, we are telling a different story, saying `MLP needs a 3D convolutional encoder'. We need to make these two sections consistent.}\sicong{I think in this paragraph we claim that the TSDF}


%We opt for Trucated Signed Distance Funtion (TSDF) volumetric representations as they are naturally suitable for convolution operations, which have shown remarkable performance for learning hierarchical features on 2D visual perception tasks \cite{SunXLW19}. 
%Meanwhile, TSDF also describes the gradual geometry change around shape surface, which is not reflected by occupancy volume. 

We instead revisit the 3D volumetric representation and resort to 3D convolutional neural networks (CNNs) for feature learning, due to their impressive performance in feature learning and the ability to incorporate spatial context. However, volumetric methods and 3D convolution involve discretization, which might raise concerns regarding whether a discretized volume can preserve subtle geometric details as continuous representations learned in implicit functions. We investigate the relationship between volume resolution and quantization error on synthetic data by converting target mesh objects to TSDF volumes, as shown in Figure~\ref{fig:quantization_error}. We observe that the quantization errors are significantly reduced by increasing volume resolution and become nearly negligible when reaching a relatively high resolution (e.g., 512 or higher). In other words, achieving fine-detailed reconstruction is not supposed to be restricted by the use of volume representations as long as a proper volume resolution is utilized. Therefore, we present a method with high-resolution feature volumes, e.g., 256 and 512, while traditional volumetric methods \cite{varol18_bodynet,gilbert2018volumetric} are often limited to much lower resolutions, such as 32 or 128.



On the other hand, an increase in volume resolution may lead to a cubic growth of memory overhead \cite{8100085}. Reducing memory costs while guaranteeing the granularity of volumetric representations is necessary for pursuing high-quality reconstruction. Thus, we adopt a coarse-to-fine approach and cull away irrelevant voxels to build a sparse high-resolution feature volume. At the coarse level, the network computes an initial TSDF by applying a U-Net with sparse 3D CNN \cite{3DSemanticSegmentationWithSubmanifoldSparseConvNet} on the sparse feature volume, which is carved by a visual hull. Through our experiments, it turns out that more than 95\% of the volume grids are discarded by the visual hull culling, making the sparse 3D CNN efficient. At the fine level, the network focuses on a narrow band near the zero-level set of the initial TSDF and discretizes the narrow band with smaller voxels. By employing this narrow-band culling, we further shrink the sampling space, resulting in a relatively small range of grid numbers (usually 300K--500K in our experiments) even with a high volume resolution of 512. The remaining voxels in the narrow band are associated with features that fuse high-frequency information from the computed normal maps upon the low-frequency shape from the coarse level to compute the TSDF at high resolution. The final mesh is then extracted from the TSDF using the Marching-Cube algorithm ~\cite{Lorensen87marchingcubes}.
% Different from the u-net sturcture to preserve global topology context, we then apply a shallow 3dcnn to compute the final TSDF $D_{final}$ which contain more local geometry detail.




% \ping{this paragraph can be expanded. It is an important contribution and often ignored by other works. stress on the novel idea of regressing blending weights instead of colors}

In addition to geometry, high-quality mesh texture is also a crucial factor contributing to visual appearance. Directly computing a color field in 3D space, as in \cite{iccv2020PIFu}, struggles to capture high-frequency texture details, while the neural radiance field (NeRF) \cite{yu2020pixelnerf} or the DoubleField~\cite{shao2022doublefield} require expensive per-instance optimization and are often unstable for sparse input images. In contrast, we adopt an image-based rendering approach to compute a texture atlas map, which is efficient and widely supported in existing computer graphics tools. 
Specifically, we compute a blending weight at each 3D point on the mesh surface to determine its color as a weighted average of the colors at its image projections. The blending weights can be computed at a relatively coarse resolution, e.g., 512 volume resolution in our case, and leave texture details to the high-resolution images, such as 1K or 2K. Unlike previous methods that generate blurry texturing results under sparse input, our method generalizes well on both synthetic and real data with just a few input views. 
Figure~\ref{fig:teaser} shows two examples reconstructed by our method. Despite the challenging garment, pose, and occlusion, our method recovers faithful shape, normal, and texture on the right.

%with a wide variety of poses and clothing styles, and it is also adaptive to handle input image with arbitrary resolutions.
%\sicong{For this concern we claim that when the resolution of dicretized volume meets certain threshold (which is 256 in our experiment), the quantization error can be neglected.} 



In summary, the main contributions of this paper are as follows:
\begin{itemize}
\vspace{-0.1in}
  \item 
  We revisit the 3D volumetric representation and demonstrate that it can support clothed human reconstruction with equal or even better performance compared to implicit representation. 
  \item 
  We develop a memory and computation-efficient method for high-resolution volumetric reconstruction using sophisticated sparse 3D CNN, coarse-to-fine estimation, and voxel culling by visual hull and narrow bands. 
  \item 
  We introduce a novel method to compute a texture atlas map, which captures rich appearance details from high-resolution input images.
  \item 
  We achieve impressive results on standard benchmark datasets Twindom and MultiHuman, significantly reducing the point-2-surface (P2S) precision to approximately 0.2cm from just six input views, with more than $50\%$ error reduction compared to the state-of-the-art methods, including DoubleField~\cite{shao2022doublefield} and PIFuHD~\cite{saito2020pifuhd}.
\end{itemize}
\section{Preliminaries}

\subsection{\texorpdfstring{$(n, m,\ell, \beta)$}{(n, m, l, beta)} Set-System}
The constructions of the integrality gap instance in \Cref{sec:int-gap} and the reduction from label cover to GMP presented in \Cref{sec: reduction} both use the following set system as a building block. 
\begin{definition}[$(n, \m,  \ell, \bt)$ Set-system] \label{def: mlb-set-system}Let $n, m,\ell$ be positive integers, $\beta \in (0,1)$, $U$ be a set with $|U| = n$, and $A_1, \ldots,  A_{\m}$ be subsets of $U$. The sets $(U; A_1, \ldots, A_m)$ form an $(n, \m,  \ell, \bt)$ set-system if for every set $I$ of at most $ \ell$ indices from $[\m]$, $\left\lvert\cup_{i \in I} B_i\right\rvert \leq (1 - \bt) |U|$, where $B_i$ is either $A_i$ or $\overline{A}_i$. 
\end{definition}

Intuitively, an $(n, m, \ell, \beta)$ set-system has the property that any set cover which uses at most $\ell$ subsets necessarily uses a complementary pair of subsets $A_i$ and $\overline{A}_i$. Moreover, any collection of at most $\ell$ subsets that do not contain any complementary pair can cover at most a $(1-\beta)$ fraction of the elements in $U$. The following lemma shows that for a particular choice of parameters $n, m,\ell, $ and $\beta$, there is a simple and efficient randomized construction of an $(n, m, \ell, \beta)$ set-system.

\begin{lemma}\label{lem: mlb-set-system}
    For a sufficiently large positive integer $n$ and a positive integer $m \in [\sqrt{\log n}, 2\sqrt{\log n}]$ there exists an $(n, \m, \ell,\bt)$ set-system $(U; A_1, \ldots, A_m)$
 with $\,\,\ell = m/10$ and $\bt = \exp(-m) = \exp(-O(\sqrt{\log n}))$. There is a polynomial-time algorithm that constructs such a set system with high probability.
\end{lemma}
\begin{proof}
    Let $U$ be a set of $n$ elements, and initialize $m$ empty sets $A_1, \ldots, A_m$. For each element $e \in U$, sample a random index set $J \subset [\m]$ of size exactly $m/2$ and add $e$ to sets $A_j$ for $j \in J$.

We show that this construction gives an $(n, m,  \ell, \bt)$ set-system with high probability.  
    Consider an index set $I \subset[m]$ with $|I| = \ell$ and a collection of sets $B_i$ for $i \in I$ such that each $B_i$ is either $A_i$ or $\overline{A}_i$. For a fixed $e \in U$, let $p$ denote the probability that $e$ is not contained in $\cup_{i \in I} B_i$, i.e., $p = \Pr[e \notin \cup_{i \in I} B_i]$. We have,
    \[
         p \geq \binom{m-\ell}{m/2}/\binom{m}{m/2}
        \geq \left(\frac{m-\ell}{m/2}\right)^{m/2}/\left(\frac{em}{m/2}\right)^{m/2} 
        \geq \left(\frac{m-\ell}{em} \right)^{m/2}
        \geq \exp(-0.6 m),
    \]
where the second inequality uses that  $\left(\frac{n}{k}\right)^k \leq \binom{n}{k} \leq \left(\frac{en}{k}\right)^k$.

The probability that  $\cup_{i \in I}B_i$  contains a fixed subset of cardinality greater than or equal to $(1-\beta)n$ is at most $(1 - p)^{(1-\beta)n}$. By a union bound over at most $ \binom{n}{\beta n}n$ possible subsets with cardinality at least $(1-\beta)n$,
    \begin{align*}\Pr\left[\lvert \cup_{i \in I}B_i \rvert \geq (1 - \bt) n\right] &\leq \binom{n}{\bt n} n \cdot (1 - p)^{(1-\bt)n}
    \leq \left(e/\bt\right)^{\bt n}n \cdot e^{-(1-\bt)np}\\
    &\leq \exp\left(3n\bt  \log(1/\bt) - np/2\right) \leq \exp(-np/4) \leq \exp(-n^{0.9}),
    \end{align*}
where in the second to last inequality we use that $3\bt \log(1/\bt) < p/4$.

A union bound over the at most $2^\ell \cdot \binom{m}{\ell} \leq \exp( \sqrt{\log n})$ possible choices to pick the $\ell$ sets  $B_i$, gives that with high probability, the union of any $\ell$ sets $B_i$ has cardinality less than $(1-\bt)n$. 
\end{proof}

\subsection{Label Cover}
In \Cref{sec: reduction}, we prove the hardness of approximation of GMP via a reduction from the standard label cover problem as defined below. 
\begin{definition} \label{def: label-cover}
A \emph{label cover} instance $\mathcal{L}$ is defined by a tuple $((U,V, E), \labels, \Pi)$. Here $(U,V,E)$ is a bipartite graph with vertices $U \cup V$ and edges $E \subseteq U \times V$; $\labels$ is a positive integer and $\Pi$ is a set of functions one for each edge $e \in E$ i.e., $\Pi = \{\pi_e: [\labels] \rightarrow [\labels] \,\, |\,\, e \in E \}$. A labeling of the vertices $\sigma: U \cup V \rightarrow [\labels]$ is said to satisfy an edge $e = (u,v)$ if $\pi_e(\sigma(u)) = \sigma(v)$. Given $\mathcal{L}$, the goal of the label cover problem is to find a labeling $\sigma^*$ that satisfies the maximum number of edges in $E$. We use $OPT(\mathcal{L})$ to denote the fraction of the edges in $E$ satisfied by $\sigma^*$.
\end{definition}

As we will need the explicit dependence between the number of labels and the soundness, for completeness we 
sketch below the precise gap version of the label cover problem that we will use. 
\begin{lemma}[Hardness of  Gap  Label Cover] \label{lem: hardness-of-label-cover}
Given a label cover instance $\mathcal{L} = ((U,V, E), \labels, \Pi)$ satisfying:
\begin{enumerate}
  \item[(i)] $|U| = |V| = \lsize$
  \item[(ii)]  The degree of every vertex in $U \cup V$ is $d = O((\log \lsize)^{c_1})$ for some constant $c_1$.
  \item[(iii)]
  $\labels = \sqrt{\log \lsize}$
\end{enumerate}
There is some constant $\razconstant > 0$, for which there is no polynomial-time algorithm to decide if $OPT(\mathcal{L}) = 1$ or $OPT(\mathcal{L}) \leq (\log \lsize)^{-\razconstant}$ provided that $\emph{\textsf{NP}} \not\subseteq \emph{\textsf{DTIME}}(\lsize^{O(\log \log \lsize)})$.
\end{lemma}
\begin{proof}
Using a standard argument (see for ex., \cite{feige1996threshold, arora1996hardness}) one can obtain a reduction from a 3SAT-5 instance $\phi$ with $\satvars$ variables to a Label Cover instance $\mathcal{L}_1 = ((U_1, V_1, E_1), 8, \Pi)$, where $|U_1| = |V_1|=O(\satvars)$ and the graph $(U_1, V_1, E)$ is $15$-regular. The  instance $\mathcal{L}_1$ has the following property: if $\phi$ has a satisfying assignment then $OPT(\mathcal L_1) = 1$; else if any assignment satisfies at most $(1-\epsilon)$ fraction of clauses in $\phi$, then $OPT(\mathcal{L}_1) \leq (1 - \Theta(\epsilon))$. By the PCP-theorem \cite{10.1145/278298.278306} it follows that, for some constant $\epsilon_0 > 0$, deciding if $OPT(\mathcal L_1) = 1$ or $OPT(\mathcal L_1) \leq 1 - \epsilon_0$ is \textsf{NP}-hard.


The following well-known construction \cite{arora1996hardness}
gives stronger inapproximability results for label cover. We define the $k$th power of the label cover instance $\mathcal{L}_k = ((U_k,V_k,E_k), 8^k, \Pi^k)$, where $U_k$, $V_k$ are $k$-tuples of vertices in $U_1$, $V_1$ respectively, $E_k$ is the set of all $k$-tuples of edges in $E_1$. The resulting graph has $N = \satvars^{O(k)}$ vertices and is $(15)^k$-regular. The new set of labels\footnote{The labels are essentially numbers from $1$ to $8^k$.} consist of $k$-tuples of $\{1,\ldots,8\}$.  For an edge $e = (e_1, \ldots, e_k) \in E_k$, we define the function $\pi_e^k(a_1, \ldots, a_k) = (\pi_{e_1}(a_1), \ldots, \pi_{e_k}(a_k))$. Raz's Parallel Repetition Theorem \cite{raz1995parallel}, shows that for the label cover instance constructed above, there exists a constant $\alpha$ such that $OPT(\mathcal L_k) \leq (OPT(\mathcal L_1))^{\alpha k}$. 

We now pick $k$ so that $L = \sqrt{\log N}$. Since $L = 8^k$ and $N = \satvars^{O(k)}$, this gives $k = \Theta(\log \log \satvars)$. This choice of $k$ ensures that $d = (15)^k =(\log N)^{c_1}$ for some constant $c_1$. Moreover, if $OPT(\mathcal{L}_1) \leq (1-\epsilon_0)$, then $OPT(\mathcal L_k) \leq (1-\epsilon_0)^{\alpha k} \leq (\log t)^{-c'} \leq (\log N)^{-\razconstant}$ for some positive constants $\razconstant, c'$.
\end{proof}

\section{METHODS}
\label{sec:methods}
\subsection{Problem Definition and Proposed Framework}
The objective is to reconstruct a dense point cloud that precisely represents the shape of unknown transparent objects from sparse point clouds extracted with active tactile interactive perception. To this end, we propose a novel framework termed ACTOR shown in Fig.~\ref{fig:framework}. In Fig.~\ref{fig:framework}(a) we propose a self-surpervised learning approach with an autoencoder network that is trained on subsampled pointclouds from synthetic objects belonging to the same category but not identical as the real objects. In Fig.~\ref{fig:framework}(b), we propose a novel active tactile-based unknown transparent object exploration strategy which is used for inference with our trained model to reconstruct a dense point cloud. We demonstrate downstream tasks such as tactile-based pose estimation.
% and tactile-based object recognition. 

\subsection{Deep Self-Supervised Learning for 3D Object Reconstruction}
\label{ssec:deep_reconstruction}
We generate a dataset $\mathcal{D}$\footnote{\url{https://www.robotact.de/tactile-reconstruction}} of synthetic object models from the ShapeNet repository~\cite{chang2015shapenet} in order to leverage the open-source datasets and avoid expensive real tactile-data collection. The synthetic object models belong to the same category but are different from the real unknown transparent objects. 
We uniformly sample $N_{in} = 2048$ points from the synthetic object meshes. These pointclouds are normalized and scaled to fit into a $[0,1]^3$ cube and added to the dataset, $\mathcal{P}_{in} \in \mathcal{D}$. 
% The generated dataset is provided in the project page\footnote{\url{https://robotac-bmw.github.io/tactile_reconstruction/}}.
In order to generate the input point clouds $\mathcal{P}^{\bullet}_{in}$ to the network, we randomly subsample the $\mathcal{P}_{in}$ by voxel-grid subsampling by the factor $k$ i.e., $\mathcal{P}^{\bullet}_{in} \in \mathbb{R}^{\lceil \frac{1}{k}N_{in} \rceil \times 3}$.  This creates a challenging task for reconstruction with higher values for $k$ as simpler techniques based on interpolation with neighborhood points cannot be used. 

\subsubsection*{Feature-Extraction Encoder}
The network architecture shown in Figure~\ref{fig:framework}(a) is proposed as an autoencoder (AE) that uses a self-supervised approach to reconstruct the original point cloud from a subsampled point cloud. 
The encoder takes subsampled point clouds as inputs and generates a high dimensional feature vector. The feature vector captures the global geometric shape information of the input point cloud. 
In general, any deep network that works on raw input point clouds to provide a high dimensional feature vector can be used as an encoder. In particular,
we use a modified PointNet architecture~\cite{qi2017pointnet} for the encoder. PointNet takes unordered point clouds and generates a global feature descriptor vector of size 1024. The network learns a set of optimization functions that select interesting or informative points of the point cloud. The encoder consists of $[1\times1]$ convolutions with output channels size $(64, 64, 128, 1024)$ with the first convolutional layer with kernel size $[1\times3]$ to encode the input pointcloud of $N\times3$ dimension. The convolution layers are aggregated by a max-pooling layer. We introduce a self-attention layer~\cite{zhang2019self} whose outputs are aggregated with the max-pooled features to provide the global feature vector.  
We have summarized the encoder in Figure~\ref{fig:framework}(a).
% As the encoder provides a high-dimensional global feature vector, we term it as feature-extraction encoder.

\textbf{Self-Attention (SA) Layer:} The SA layer is introduced as it can encode meaningful spatial relationships between features and focus on important local features. From the input layer ($\mathtt{conv2d-1024}$), two separate multi-layer perceptrons (MLPs) are used to get features $\mathbf{G}$ and $\mathbf{H}$ which are subsequently used to get the weights as $\mathbf{W} = softmax(\mathbf{G}^T\mathbf{H})$. The input features are transformed using another MLP to obtain $\mathbf{K}$ and multiplied with the weights as $\mathbf{W}^T\mathbf{K}$.
These vectors are summed with the input vector to produce the output features.
% The SA layer description is shown in Fig.~\ref{fig:self_atten}.  
% \setlength{\columnsep}{0pt}
% \begin{wrapfigure}[12]{r}{0.8\linewidth}
%   \centering
%     % \vspace{-0.5cm}
%     % Figure removed
%   \caption{The self-attention unit.}
%     % \vspace{-0.5cm}
%   \label{fig:self_atten}
% \end{wrapfigure}
% % Figure environment removed

\subsubsection*{Upsampling Decoder}
We design an upsampling decoder that upsamples the input global feature vector to provide the reconstructed dense output point cloud $\mathcal{P}_{out}$. The upsampling decoder is composed by a fully connected layer with output dimension of 1024 and five deconvolutional layers with kernel sizes and output channels shown in Fig.~\ref{fig:framework}(a).  
The decoder produces the output point cloud with point size set to 2048 while training as this is sufficiently dense for reconstruction purposes. 

\subsubsection*{Loss Function}
In order to encourage the upsampled point cloud to be in proximity to the original input point cloud and follow the underlying geometrical surface of the object, we use the Chamfer distance metric~\cite{borgefors1986distance} as the loss. Given the input point cloud prior to subsampling, $\mathcal{P}_{in}$ and the reconstructed output point cloud $\mathcal{P}_{out}$, the loss is defined as:
\begin{align}
    \mathcal{L}_{CD}(\mathcal{P}_{in}, \mathcal{P}_{out}) &= \frac{1}{|\mathcal{P}_{in}|}\sum_{p_1 \in \mathcal{P}_{in}} \min_{p_2 \in \mathcal{P}_{out}} ||p_1 - p_2||_{2} + \\ & \frac{1}{|\mathcal{P}_{out}|}\sum_{p_2 \in \mathcal{P}_{out}} \min_{p_1 \in \mathcal{P}_{in}} ||p_2 - p_1||_{2} \nonumber,
    \label{eq:chamfer_dist}
\end{align}
where $|\bullet|$ refers to the number of points in the point cloud and $||\bullet||_2$ refers to the L2 norm. The loss $\mathcal{L}_{CD}$ represents the average distance between the \textit{closest} points in the two point clouds. We use the weighted loss for learning stability as the reconstruction loss $\mathcal{L}_{rec} = \alpha\mathcal{L}_{CD}$ with $\alpha = 100$ set empirically.
For surface reconstruction from the dense reconstructed point cloud, we use the ball-pivoting algorithm~\cite{bernardini1999ball}.

% \subsubsection*{Recognition Network}
% \label{ssec:recog_net}
% The pretrained encoder layers for reconstruction task are frozen for category-level classification. We employ three fully-connected layers with parameters 512, 256, and $n$ respectively where $n$ represents the number of categories of the objects.
% The softmax cross-entropy loss is used for training the recognition network. The recognition head is shown in Fig.~\ref{fig:framework}(a.I). The subsampled sparse point clouds from our synthetic dataset with different subsampling ratios and data augmentation with random rotations are used. Network implementation details are provided in Sec.~\ref{ssec:setup}.



%%%%%%%%%%%%%%%%%%%%%%%%%%%%%%%%%%%%%%%%%%%%%%%%%%%%%%%%%%%%%%%%%%%%%%%
%%%%%%%%%%%%%%%%%%%%%%%%%%%%%%%%%%%%%%%%%%%%%%%%%%%%%%%%%%%%%%%%%%%%%%%
\subsection{Active Deep Tactile-based Unknown Transparent Object Reconstruction and Pose Estimation}
\subsubsection{Active Tactile-based Transparent Object Reconstruction}
The model trained with only \textit{synthetic data} as described in Sec.~\ref{ssec:deep_reconstruction} is used during the inference with \textit{real-world} transparent objects. The sparse tactile point cloud data is collected autonomously by the robot using an information gain-based active strategy. We define two types of tactile actions for data acquisition: touch and pinch actions as shown in Figure~\ref{fig:occupancy_grid}.
% The action nomenclature is derived from human grasp taxonomy studies~\cite{feix2015grasp}.
The touch action is executed as a guarded horizontal straight-line motion wherein the object is not moved upon contact. The touch action is defined by a tuple $\mathbf{a}^{t} = \{\mathbf{s}^t, \overrightarrow{\mathbf{d}^t} \}$ where $\mathbf{s}^t \in \mathbb{R}^3$ is the start point of the tactile-sensorised gripper and $\overrightarrow{\mathbf{d}^t} \in \mathbb{R}^3$ is the direction of the gripper-motion defined in the world-coordinate frame $\mathcal{W}$. During the pinch action the robot approaches the object in a vertical straight-line motion with a completely open gripper and performs an antipodal enclosure grasp on the object. The fingers of the gripper are closed until the force on the tactile sensors exceeds a predefined threshold.
The pinch action is characterized by $\mathbf{a}^{p} = \{\mathbf{s}^p \}$ where $\mathbf{s}^p \in \mathbb{R}^3 $ is the start position of the gripper motion vertically above the object at a predefined height as shown in Figure~\ref{fig:occupancy_grid}. Given the 2D bounding box of the object (a priori known or through a RGB camera), a probabilistic occupancy grid $\mathcal{OG}_i$ of preset height and resolution $og_{res}$ is defined. Each cell of the occupancy grid $c_i$ is represented by an occupancy probability $p(c_i)$ which is initially set to 0.5. During exploration, if a cell is discovered to belong to the object, the probability is set to 1 and similarly, if the cell belongs to free space, the probability is set to 0. The probabilities are updated through ray intersections based on the virtual sensor model. We define a virtual sensor model of the tactile sensor which casts a set of rays $\mathcal{R} = \{r_1, r_2, \dots, r_{n_{taxel}} \}$ where ${n_{taxel}} $ refers to the number of taxels in the sensor array. The independence assumption of the probability of each grid cell with one another allows us to calculate the overall entropy of the $\mathcal{OG}$ as the summation of the entropy of each cell. The Shannon entropy of the overall occupancy grid is calculated as:
\begin{equation}
    \mathbb{H}(\mathcal{OG}) = \sum_{c_i \in \mathcal{OG}} p(c_i)log(p(c_i)) + (1 - p(c_i))(1 - log(p(c_i))).
    \label{eq:entropy}
\end{equation}
Monte-Carlo sampling of possible tactile actions $N_{nbt}$ are performed for computing the next best tactile (NBT) action. The actions space $\mathcal{A}_{nbt}$ is comprised of an equal number of touch and pinch respectively as $\mathcal{A}_{nbt} = \{a^p, a^t\}_{N_{nbt}}$. The expected measurements $\hat{\mathbf{z}}_t$ for each action $a_t \in \mathcal{A}$ is computed using ray-traversal algorithms~\cite{hornung2013octomap}. 
Given the observed grid cell $c$ and the measurement from sensor observation $z$, the log-odds is updated as $L(c|z) = L(c) + l(z)$ wherein $L(c) = log\frac{p(c)}{1-p(c)}$ and  
\begin{equation}
    l(z) = \left\{
                \begin{array}{ll}
                  log\frac{p_h}{1-p_h}  \quad \mathrm{if} \ z \widehat{=} \textit{ hit} \\
                  log\frac{p_m}{1-p_m} \quad \mathrm{if} \ z \widehat{=} \textit{ miss} 
                \end{array}
              \right.
    \label{eq:log-odds}
\end{equation}
where $p_h$ and $p_m$ are the probabilities of hit and miss which are user-defined values set to 0.7 and 0.4 respectively as in~\cite{hornung2013octomap}. The posterior probability $p(c|z)$ can be computed by inverting $L(c|z)$. The expected information gain by taking an action $a_t \in \mathcal{A}_{nbt}$ with expected measurement $\hat{\mathbf{z}}_t$ is provided by the Kullback-Liebler divergence of the posterior entropy and the prior entropy as: 
\begin{equation}
    E[\mathbb{I}(p(c_i | \mathbf{a}_t,  \hat{z}_t))] = \mathbb{H}(p(c_i)) - \mathbb{H}(p(c_i | \mathbf{a}_t,  \hat{z}_t))
    \label{eq:kl_view}
\end{equation}
Therefore, the action that maximizes the expected information gain is considered as the NBT action:
\begin{equation}
    \mathbf{a}^{nbt*}_t = \argmax_{\mathbf{a} \in \mathcal{A}}(E[\mathbb{I}(p(c_i | \mathbf{a}_t,  \hat{z}_t))])
    \label{eq:kl_view_max}
\end{equation}
Each tactile action extracts contact positions in 3D space and contact forces. The direction of the normal force is used to extract the normal direction $\hat{n}$ of the object surface. The contact points are aggregated into the tactile point cloud $\mathcal{P}^t$. In order to initialize the NBT action calculation, an initial point cloud (with $N_{\mathcal{P}^t} = 20$) is required, which is extracted by randomised touch actions. Further points are collected in an active manner using the NBT criteria. A minimum number of points in the tactile point cloud is required to perform model inference $N_{\mathcal{P}^t} > N_{min}$ which is tuned empirically. The tactile point cloud is provided as input to the trained network and the reconstructed point cloud $\mathcal{P}_{out}$ is obtained . 
% This is used for downstream task Section~\ref{ssec:pose_estimation}. 
% For acceptable reconstruction accuracy around 100 tactile points is required.  

% [TODO:] check for action taxonomy if its correct

%%%%%%%%%%%%%%%%%%%%%%%%%%%%%%%%%%%%%%%%%%%%%%%%%%%%%%%%%%%%%%%%%%%%%%%
% Figure environment removed
%%%%%%%%%%%%%%%%%%%%%%%%%%%%%%%%%%%%%%%%%%%%%%%%%%%%%%%%%%%%%%%%%%%%%%%


%%%%%%%%%%%%%%%%%%%%%%%%%%%%%%%%%%%%%%%%%%%%%%%%%%%%%%%%%%%%%%%%%%%%%%%
%%%%%%%%%%%%%%%%%%%%%%%%%%%%%%%%%%%%%%%%%%%%%%%%%%%%%%%%%%%%%%%%%%%%%%%
\subsubsection{Tactile-Based Object Pose Estimation}
\label{ssec:pose_estimation}

We perform the 6D pose estimation through dense to sparse point cloud registration. The sparse scene point cloud $\mathbf{s}_i \in \mathcal{S}$ is represented by the tactile points and the dense object point cloud $\mathbf{o}_i \in \mathcal{O}$ is represented by the reconstructed point cloud in~\ref{ssec:deep_reconstruction} without the need for the object model. Point cloud registration problem with $M$ known correspondences can be formulated as:
\begin{equation}
     \mathbf{s}_i = \mathbf{S}\cdot(\mathbf{R}\mathbf{o}_i) + \mathbf{t} \quad i = 1, \dots M,
     \label{eq:generativemodel}
 \end{equation}
where $\mathbf{S} \in \mathbb{R}^3$ represents scale, $\mathbf{R} \in SO(3)$ represents rotation and $\mathbf{t} \in \mathbb{R}^3$ represents translation which are unknown and to be estimated and $\cdot$ is the element-wise product. 
%% [TODO] : check derivation

We perform the point cloud registration using our novel translation-invariant Quaternion filter (TIQF) presented in~\cite{murali2022active} to determine $\mathbf{R}$, $\mathbf{S}$ and $\mathbf{t}$. 
The scale, rotation and translation are decoupled by finding the relative vectors between corresponding points, i.e., $\forall o_i, o_j \in \mathcal{O}, s_i, s_j \in \mathcal{S}$ the relative vectors are $\mathbf{s}_{ji} = \mathbf{s}_j - \mathbf{s}_i$ and $\mathbf{o}_{ji} = \mathbf{o}_j - \mathbf{o}_i$. Equation~\eqref{eq:generativemodel} is reformulated as:
\begin{align}
    \mathbf{s}_j - \mathbf{s}_i &= (\mathbf{S}\cdot\mathbf{R}\mathbf{o}_j + \mathbf{t}) - (\mathbf{S}\cdot\mathbf{R}\mathbf{o}_i + \mathbf{t}) ,\\
    \mathbf{s}_{ji} &= \mathbf{S}\cdot\mathbf{R}\mathbf{o}_{ji} \quad .
    \label{eq:trans_invariance}
\end{align}

We note that equation~\eqref{eq:trans_invariance} is independent of translation. Taking the L2-norm on both sides for Eq.~\eqref{eq:trans_invariance} and recalling that norm is rotation invariant we get:
\begin{equation}
    \mathbf{||s||}_{ji} = \mathbf{||S||}\cdot\mathbf{||o||}_{ji} \quad .
    \label{eq:rot_invariance}
\end{equation}
The scale $\mathbf{S}$ is estimated by taking the ratio of the axis aligned bounding box (AABB) of the scene and object point clouds, i.e., if $\mathcal{X}_{AABB} = \{ (x_{min}, x_{max}), (y_{min}, y_{max}), (z_{min}, z_{max}) \}$ represents the AABB for a point cloud $\mathcal{X}$, then:
\begin{align}
     \mathbf{S} &= \{ \frac{|x_{max} - x_{min}|_{\mathcal{S}}}{|x_{max} - x_{min}|_{\mathcal{O}}}, \frac{|y_{max} - y_{min}|_{\mathcal{S}}}{|y_{max} - y_{min}|_{\mathcal{O}}} , \frac{|z_{max} - z_{min}|_{\mathcal{S}}}{|z_{max} - z_{min}|_{\mathcal{O}}}    \}
     \label{eq:scale}
 \end{align}
Using the estimated scale and using $\tilde{\mathbf{o}}_{ji} = \mathbf{S}\mathbf{o}_{ji}$ for convenience we are left with a pure rotation to estimate:  
\begin{align}
    \tilde{\mathbf{s}}_{ji} &= \mathbf{R}\tilde{\mathbf{o}}_{ji} \quad .
    \label{eq:trans_scale_invariance}
\end{align}
 We cast the rotation estimation problem into a recursive Bayesian estimation framework and derive a linear state and measurement model. Reformulating Eq.\eqref{eq:trans_scale_invariance} using quaternions we get: 
 \begin{equation}
    \overline{\mathbf{s}}_{ji} = \mathbf{x} \odot \overline{\mathbf{o}}_{ji} \odot \mathbf{x}^{*}, 
    \label{eq:quat_objective}
\end{equation}
where $\mathbf{x}$ is the quaternion form of $\mathbf{R}$, $\odot$ is the quaternion product, ${\mathbf{x}}^{*}$ is the conjugate of $\mathbf{x}$, and $\overline{\mathbf{s}}_{ji}=\{0,\tilde{\mathbf{s}}_{ji}\}$ and $\overline{\mathbf{o}}_{ji}=\{0,\tilde{\mathbf{o}}_{ji}\}$.
Using the matrix form of quaternion product, we can rewrite Eq.\eqref{eq:quat_objective} as:
\begin{align}
    \begin{bmatrix}
        0 & -\tilde{\mathbf{s}}_{ji}^T \\
        \tilde{\mathbf{s}}_{ji} & \tilde{\mathbf{s}}_{ji}^{\times}
    \end{bmatrix}\mathbf{x} -  \begin{bmatrix}
        0 & -\tilde{\mathbf{o}}_{ji}^T \\
        \tilde{\mathbf{o}}_{ji} & -\tilde{\mathbf{o}}_{ji}^{\times}
    \end{bmatrix} \mathbf{x} = \mathbf{0} \\
    \underbrace{\begin{bmatrix}
        0 & -(\tilde{\mathbf{s}}_{ji} - \tilde{\mathbf{o}}_{ij})^T \\
        (\tilde{\mathbf{s}}_{ji} - \tilde{\mathbf{o}}_{ji}) & (\tilde{\mathbf{s}}_j + \tilde{\mathbf{s}}_i + \tilde{\mathbf{o}}_j + \tilde{\mathbf{o}}_i)^{\times}
        \end{bmatrix}_{4 \times 4}}_{\mathbf{H}_t} \mathbf{x} &= \mathbf{0} \quad ,
        \label{eq:expected_measurement}
\end{align}
where $(\ )^\times$ denotes the skew-symmetric matrix formulation. Equation~\eqref{eq:expected_measurement} is of the form $\mathbf{H}_t\mathbf{x} = 0$ where $\mathbf{H}_t$ is the pseudo-measurement matrix~\cite{choukroun2006novel}. We note that Eq.~\eqref{eq:expected_measurement} represents a noise-free state estimation where $\mathbf{H}_t$ depends only on sparse and dense point correspondences which are $\tilde{\mathbf{s}}_{ji}$ and $\tilde{\mathbf{o}}_{ji}$. We design a pseudo-measurement model as $ \mathbf{H}_t \mathbf{x} = \mathbf{z}^h$
% \begin{align}
%     \mathbf{H}_t \mathbf{x} &= \mathbf{z}^h,
%     \label{eq:measurement_model}
% \end{align}
and set $\mathbf{z}^h = 0$. Since we have a static process model, the object does not move and $\mathbf{x}$ and $\mathbf{z}_t$ are Gaussian distributed, 
the state $\mathbf{x}_t$ and covariance matrix $\Sigma^{\mathbf{x}}_{t}$ at each timestep $t$ are computed through a linear Kalman filter. The Kalman filter equations are skipped for brevity and a in-depth derivation is provided in our prior work~\cite{murali2022active}.
As the Kalman filter does not implicitly ensure the constraints on the quaternion as $||\mathbf{x}|| = 1$, we normalise the state and uncertainty after each update step as $\bar{\mathbf{x}}_{t} = \frac{\mathbf{x}_{t}}{||\mathbf{x}_{t}||_2} \quad, \bar{\Sigma}^{\mathbf{x}}_{t} = \frac{\Sigma^{\mathbf{x}}_{t}}{||\mathbf{x}_{t}||_2^2}$. We convert the estimated rotation $\Bar{\mathbf{x}}_t$ to its equivalent rotation matrix $\mathbf{R}$. It used to estimate the translation using the following relation: $\mathbf{t} = \frac{1}{N} \sum_{i=0}^{N} (\Bar{\mathbf{s}}_i - \mathbf{R} \Bar{\mathbf{o}}_i).$
% \begin{equation}
%     \mathbf{t} = \frac{1}{N} \sum_{i=0}^{N} (\Bar{\mathbf{s}}_i - \mathbf{R} \Bar{\mathbf{o}}_i).
%     \label{eq:translation_solution}
% \end{equation}
% \setlength{\columnsep}{1pt}
% \begin{wrapfigure}[18]{r}{0.6\linewidth}
%   \centering
%     \vspace{-0.5cm}
%     % Figure removed
%   \caption{Translation-invariant measurements}
%     % \vspace{-0.5cm}
%   \label{fig:TIMS}
% \end{wrapfigure}
At each iteration, a rotation and translation estimate is found which is used to transform the object point cloud and the process is repeated by re-estimating the correspondence points. The convergence criteria are set by (a) maximum number of iterations or (b) the relative change in estimated pose parameters is less than a predefined threshold ($0.1mm$ and $0.1^o$). 

% the linear Kalman filter equations are given as:
% \begin{align}
%     \mathbf{x}_{t} &= \bar{\mathbf{x}}_{t-1} - \mathbf{K}_t \left( \mathbf{H}_t \bar{\mathbf{x}}_{t-1} \right) \\
%     \Sigma^{\mathbf{x}}_{t} &= \left( \mathbf{I} - \mathbf{K}_t \mathbf{H}_t \right) \bar{\Sigma}^{\mathbf{x}}_{t-1} \\
%     \mathbf{K}_t &= \bar{\Sigma}^\mathbf{x}_{t-1} \mathbf{H}_t^T \left( \mathbf{H}_t\bar{\Sigma}^\mathbf{x}_{t-1} \mathbf{H}_t^T + \Sigma_t^{\mathbf{h}}\right)^{-1}, 
%     \label{eq:kalman_equations}
% \end{align}
% where $\bar{\mathbf{x}}_{t-1}$ refers to the normalized mean of the state at $t-1$, Kalman gain $\mathbf{K}_t$ and $\bar{\Sigma}^{\mathbf{x}}_{t-1}$ is the covariance matrix of the state at $t-1$. 
% The parameter $\Sigma_t^{\mathbf{h}}$ is referred as the measurement uncertainty during time $t$. It is dependent on the state and is provided by~\cite{choukroun2006novel}:
% \begin{align}
%     \Sigma_t^{\mathbf{h}} = \frac{1}{4}\rho\left[ tr(\bar{\mathbf{x}}_{t-1}\bar{\mathbf{x}}_{t-1}^T + \bar{\Sigma}^{x}_{t-1})\mathbb{I}_4 - (\bar{\mathbf{x}}_{t-1}\bar{\mathbf{x}}_{t-1}^T + \bar{\Sigma}^{x}_{t-1} )\right], 
%     \label{eq:choukron}
% \end{align}
% wherein the constant $\rho$ corresponds to the uncertainty of the correspondence measurements and $tr$ refers to trace.


%%%%%%%%%%%%%%%%%%%%%%%%%%%%%%%%%%%%%%%%%%%%%%%%%%%%%%%%%%%%%%%%%%%%%%%
%%%%%%%%%%%%%%%%%%%%%%%%%%%%%%%%%%%%%%%%%%%%%%%%%%%%%%%%%%%%%%%%%%%%%%%
% \subsubsection{Transparent Object Manipulation}
% \label{ssec:tactile_manipulation}
% With the computed 6D pose and estimated CAD model, we design a simple grasping technique in order to grasp and lift the transparent objects. For each \textit{category} of objects, we generated several grasp plans using GraspIt~\cite{miller2004graspit}. Each grasp plan includes the grasp position, orientation and approach vector relative to the model of the object and a grasp quality score. With the pose of the object, the grasp plans are filtered based on kinematic constraints of the robot, workspace limitations and possible collisions with other objects in the scene. Among the remaining grasp plans, the plan with the highest score is chosen and executed. The robot lifts the transparent object and places it in a pre-defined position.
% An online grasp planning and collision avoidance framework is out of the scope of this current work but can be readily integrated into the current framework.

%%%%%%%%%%%%%%%%%%%%%%%%%%%%%%%%%%%%%%%%%%%%%%%%%%%%%%%%%%%%%%%%%%%%%%%
%%%%%%%%%%%%%%%%%%%%%%%%%%%%%%%%%%%%%%%%%%%%%%%%%%%%%%%%%%%%%%%%%%%%%%%

% \subsubsection{Tactile-based Transparent Object Recognition}
% \label{ssec:classification}
% % Figure environment removed
% We use the pretrained encoder model with fixed weights for category-level classification. We employ three fully-connected layers with parameters 512, 256 and $n$ respectively where $n$ represents the number of categories of the objects. Transfer learning is employed to fine-tune the classification network shown in Figure~\ref{fig:framework}(a) on the sparse pointclouds from ShapeNet database.
% During inference, the real sparse tactile pointclouds are used as input to the network for recognition network described in Sec.~\ref{ssec:recog_net}. While the task is challenging, the real-world tactile data are not used during fine-tuning intentionally as collection of large-scale datasets is prohibitively time consuming. The input pointcloud is pre-processed prior to inference by normalising and scaling to fit in $[0,1]^3$ cube to be uniform with the training dataset.
\section{Experimental Results}\label{sec:results}
    \subsection{General Results}
        The basic ResSAN model is used to determine reference results which our expanded model can be compared to as it is structurally similar to ResLAN but does not possess the Lidar adaptive components of it. Further, we compare with the full-size PackNet-SAN and the unmodified NLSPN architecture. 
        As it can be seen from Tab.\,\ref{tab:sota-results}, our LiDAR-adaptive ResLAN achieves competitive performance compared to state-of-the-art standard depth completion methods, which are specialized to the unfiltered 64-beam-LiDAR. The performance differences are in the range of a few centimetres in terms of MAE, which is acceptable given the practical advantage that ResLAN can generalize to different beam patterns as will be shown below.

        Furthermore, we compared the architectures for a set of three different input types that contained 64, 32 or 16 LiDAR channels using both filter types on the metrics from the KITTI benchmark. The NLSPN model was trained for the standard depth completion task and then evaluated with different input data. As for the ResSAN models, we trained one model for each input type and tested it for the corresponding one which serve serve as the \emph{Baseline} in Tab.\,\ref{tab:overall-results}. Our ResLAN model was jointly trained for all three settings. As listed in Tab.\,\ref{tab:overall-results}, the ResLAN models outperform the challenging baseline in all metrics for FOV filtering and all but one for sparse filtering. This implies that our LiDAR adaptive model is able to outperform dedicated models in case of very sparse input depth. Fig.\,\ref{fig:comp-plot} shows this is indeed the case for 32 and even more for 16 channels. For FOV-filtered inputs with 16 channels, the ResLAN exhibits approx. $10\%$ smaller MAE than the baseline. As for the NLSPN, it becomes apparent that it is not capable of generalizing to other input types since it shows clearly worse results. The difference is especially pronounced for the FOV filtering where on average more than every fourth predicted pixel is more than $25 \%$ deviating from the ground truth\,($\delta_{1.25}$). Therefore, using a weight-adapting network in combination with differently filtered input depths allows us to train models that outperform their non-adaptive counterparts.

        \begin{table}[]
            \centering
    	    \small
            \vspace{0.4cm}
            \caption{\textbf{Depth estimation result for standard depth completion} when the ResSAN model was only trained for 64 channels and the ResLAN model for multiple tasks. The PackNet-SAN and NLSPN models were trained with the setup that was also used for our model architecture.}
            \footnotesize
            \setlength{\tabcolsep}{5pt}
            \begin{tabular}{@{}lrrrrl@{}}
            \toprule
            \multicolumn{6}{c}{\textbf{Standard LiDAR Depth Completion}}                                                                                                                         \\ \midrule
            \multicolumn{1}{l|}{Method}          & RMSE $\downarrow$            & MAE  $\downarrow$            & iRMSE $\downarrow$             & iMAE $\downarrow$ & $\delta_{1.25}$ $\uparrow$ \\
            \multicolumn{1}{l|}{}                & \multicolumn{1}{l}{{[}mm{]}} & \multicolumn{1}{l}{{[}mm{]}} & \multicolumn{1}{l}{{[}1/km{]}} & {[}1/km{]}        &                            \\ \midrule
            \multicolumn{1}{l|}{PackNet-SAN}     &  914                            &  298                            &  2.78                              &  1.4                 &  99.65 \%                          \\
            \multicolumn{1}{l|}{NLSPN}           &  \textbf{889}                            &   \textbf{263}                           &  \textbf{2.62}                              &   \textbf{1.3}                &   \textbf{99.61} \%                         \\ \midrule
            \multicolumn{1}{l|}{ResSAN (Ours)}   & 948                             &  275                            &  2.75                              &    1.4               &   99.58 \%                         \\
            \multicolumn{1}{l|}{ResLAN (Ours)} &   969                           &  283                            &   2.83                             &   1.4                &  99.56 \%                          \\ \bottomrule
            \end{tabular}
            \vspace{0.2cm}
            \label{tab:sota-results}
        \end{table}

        \begin{table}[]
    	    \centering
    	    \small
    	    \caption{\textbf{Depth estimation results of the two baseline setups and the explicit and implicit ResSAN} when evaluated on a combination of 16, 32 and 64 channel depth inputs. Please note that Specialist Methods need to train three specialized networks, one for each of the three types of inputs while our method only uses one network.}
            \footnotesize
            \setlength{\tabcolsep}{4.8pt}
            \begin{tabular}{@{}lrrrrl@{}}
                \toprule
                \multicolumn{6}{c}{\textbf{Sparse Channel Filter}}                                                                                                                                  \\ \midrule
                \multicolumn{1}{l|}{Method}        & RMSE $\downarrow$            & MAE  $\downarrow$            & iRMSE $\downarrow$             & iMAE $\downarrow$ & $\delta_{1.25}$ $\uparrow$  \\
                \multicolumn{1}{l|}{}              & \multicolumn{1}{l}{{[}mm{]}} & \multicolumn{1}{l}{{[}mm{]}} & \multicolumn{1}{l}{{[}1/km{]}} & {[}1/km{]}        &                             \\ \midrule
                \multicolumn{1}{l|}{NLSPN}         &  1396                            &  437                            & 5.54                               &  2.2                 &  98.82 \%                           \\
                \multicolumn{1}{l|}{Baseline}      & \textbf{1207}                             &  381                            & 4.41                               &  1.8                 &  \textbf{99.37} \%                           \\
                \multicolumn{1}{l|}{ResLAN (Ours)} &  1215                            &  \textbf{378}                            &  \textbf{4.27}                              &  \textbf{1.7}                 &  99.31 \%                           \\ \toprule
                \multicolumn{6}{c}{\textbf{Field-of-View Filter}}                                                                                                                                   \\ \midrule
                \multicolumn{1}{l|}{Method}        & RMSE $\downarrow$            & MAE  $\downarrow$            & iRMSE $\downarrow$             & iMAE $\downarrow$ & $\delta_{1.25}$ $\uparrow$ \\
                \multicolumn{1}{l|}{}              & \multicolumn{1}{l}{{[}mm{]}} & \multicolumn{1}{l}{{[}mm{]}} & \multicolumn{1}{l}{{[}1/km{]}} & {[}1/km{]}        &                             \\ \midrule
                \multicolumn{1}{l|}{NLSPN}         &  2738                            &  1702                            & 12.3                              &  4.3                 &  74.69 \%                           \\
                \multicolumn{1}{l|}{Baseline}      &  1556                            &  525                            &  6.8                              &  3.0                 & 98.14 \%                            \\
                \multicolumn{1}{l|}{ResLAN (Ours)} &  \textbf{1548}                            &  \textbf{519}                            &  \textbf{6.44}                              &  \textbf{2.8}                 & \textbf{98.52 \%}                            \\ \bottomrule
            \end{tabular}
            \label{tab:overall-results}
        \end{table}

        
        
        % Figure environment removed
        
        % Figure environment removed

    \subsection{Filter Effects}
        Comparing the effect of the two different types of depth input filters on the model performance, it becomes apparent that FOV filtering is the more challenging task. In that setting, reducing LiDAR channels is more detrimental to the performance than sparse filtering as it creates regions where no depth information is available. Effectively, the model is forced to perform depth prediction in these regions. These effects are highlighted in the depth images in Fig.\,\ref{fig:dense-maps} where the effect of a 16-channel sparse depth filter and a 16-channel FOV can be compared.

    \subsection{Generalization Capabilities}
        We trained three models for both filter types eaach, so the combinations and number of filtered depth inputs they receive are different. This serves the purpose of testing the generalization capabilities of the ResLAN architecture as well as the robustness to different filter settings. After training, the models were evaluated for the depth input settings they were trained for, as well as for ones they weren't exposed to. Overall, ResLAN shows good generalization capabilities. As one can gather from Fig.\,\ref{fig:explicit-comp} and Fig.\,\ref{fig:implicit-comp}, the consequences of slightly varying sets of input depth settings are limited. The most considerable deviations can be seen when the model is tasked to extrapolate. For instance, the model $\{64, 32, 16\}$ shows a noticeably higher MAE for eight-channel depth inputs than the model that was trained for it. Similar behaviour can be seen for the FOV filtering case as well for the model $\{64, 48, 32\}$ when tasked to generalize for a 16-channel input. There is no such pronounced effect for generalization tasks that lie between two filter settings the model was trained for. At most, it can be observed that models that were trained for a smaller range of filter values perform slightly better than ones that have to cover a wider range. The number of filter settings used in a fixed range does not relevantly influence the model performance, as can be seen, when comparing the two models in Fig.\,\ref{fig:implicit-comp}, which are both trained for a range of 64 to 32 channels but one with three filter settings and the other one with five.
    
    % Figure environment removed
    
    
    % Figure environment removed
\section{Discussion}
\label{sec: discussion}
\kmsdelete{In this work} We study \kmsreplace{Fairness-Aware PAC learning}{Fair-ERM} in the malicious noise model, and  in some cases allow 
the learner to maintain optimal overall accuracy despite the signal in Group $B$ being almost entirely washed out.
%when we allow learners to use the
%$\PQ$ randomized expansion of the hypothesis class $\mathcal{H}$
In particular we show that different fairness constraints have fundamentally different behavior in the presence of Malicious Noise, in terms of the amount of accuracy loss that a given level of Malicious Noise could cause a fairness-constrained learner to incur. 
The key to achieving our results, which are more optimistic than those in \cite{lampert}, is allowing for improper learners using the (P,Q)-randomized expansions of the given class $\mathcal{H}$.
%We \kmsreplace{present a picture of the}{prove upper and lower bounds on}
%accuracy loss for a range of fairness notions, given \kmsreplace{this simple randomization step.}{learning over $\PQ$.
%In general our results indicate Fair-ERM (given learning over $\PQ$) is more robust than claimed in \cite{lampert}.
The type of smoothness we create by using $\PQ$ seems to be a natural property that is likely shared by many natural hypothesis classes.

Fairness notions are motivated as a response to learned disparities when there is \kmsdelete{data corruption or} systemic error affecting \kmsdelete{the data for}
one group. 
Fairness notions are supposed to mitigate this by ruling out classifiers that have worse performance on a sub-group. 
This can peg both classifiers at a lower level of performance \kmsdelete{(e.g that the lower subgroup)} in order to \emph{motivate} \cite{hardt16} improving the data collection or labelling process to obtain more reliable performance. 
%So in \kmsreplace{some}{a} sense, sensitivity of the fairness notion to poor sub-group performance caused by malicious noise is the \textit{point} of fairness constraints! 
However, it also desirable that fairness constraints perform gracefully when subject to Malicious Noise because fairness constraints will be used in contexts where the data is unreliable and noisy and this might not be known to the learner.
This tension, exposed by our work, motivates 
%a revisiting of fairness notions from first principles approach and trying to axiomatize the 
%desired properties of a fairness intervention a la cryptography and privacy. \footnote{Work in multi-calibration \cite{multicalib} is a viable direction for this problem but it is unclear how 
%that and related notions behave with unreliable data. }
on going work studying the sensitivity level of fairness constraints. 
%If we we are to take a view, if a classifier is deployed 

% look to https://rdrr.io/cran/extracat/man/dcor.html for guidance for weighted distance correlation analogues that could be used for ``doubly-robuxst'' k-sample testing?

\begin{comment}
Notes - Adam Li (02/02/23)

- Add 1-2 paragraphs on implications wrt causal discovery

\section{Notes and Comments - 01/04/23}
I had meant to read this and provide some thoughts as Jovo suggested, but just got around to it. I held off on suggesting direct edits to not be rude :p.

Current Claims: A new method for estimating average treatment effect, that overcomes the major limitation of univariate outcomes.
\begin{enumerate}
    \item I think the claim can be stronger
    \item A new method for estimating average treatment effects that not only overcomes major limitations of univariate outcomes, but also serves as a conditional discrepancy test with applications to constraint-based causal discovery.
\end{enumerate}


Preliminaries: Lemma 4 basically enables us to test $P_t(y | x) = P_{t'}(y | x)$ even when T is categorical and Y and X are continuous(?). Note that you call this "Theorem 4" in the text.

Lemma 4 can also be stated using a graphical model with a graphical proof. I'm happy to try to provide an analagous proof, which I think could be an insightful appendix item. I would probably need to loop in a co-author with me to help write it though if that is fine with you. If it's too crowded, lmk.

2198q   109
Methods: As we discussed, I think it would be important to highlight the practical pros of using the causal d-corr vs conditional d-corr: I.e. better performance on finite samples? Also would be nice to explicitly state the difference between the two. And also state the difference compared to the CODiTE scholkopf paper.

% L2 penalty MANOVA?
% something like a permutation test, preserving uninteresting structure
% just unconditionally switching labels will destroy structure
% fit the model, take the residuals, allocate nk residuals to K groups, refit, and redo, and then compute the statistic for each permutation. just permuting the error distribution (if the model is right!)
% permutation version of residual bootstrap
% residual bootstrap, using permutation rather than a bootstrap (Efron book?)
\end{comment}

\bibliographystyle{IEEEtranSN}
\bibliography{batch}

\newpage

\appendix
\begin{comment}
\section{System Architecture}
\label{appendix:architecture}
\system has a novel modularized system architecture with three key components: 
\emph{StreamManager}, 
\emph{TxnManager} and \emph{TxnScheduler}. 
These components are instantiated in each thread locally.
The execution outline of \system is presented in Algorithm~\ref{alg:algo}.
Transactional stream processing is continuous and potentially never ends (Line 1$\sim$8).
The dependency resolution and execution of state transactions are separated into two non-overlapping phases by punctuations~\cite{Tucker:2003:EPS:776752.776780} (Line 2 and 5), which guarantees that no subsequent input event will have a smaller timestamp. 
Effectively, a batch of state transactions is collected during the first phase, and processed during the second phase.

In the first phase (i.e., stream processing phase), 
the \emph{StreamManager} conducts preprocessing for every input event ($e$). Similar to some prior works~\cite{tstream}, state transactions may be issued but not immediately processed during preprocessing (Line 3).
The \emph{pre\_processing} and \emph{post\_processing} functions are exposed as APIs to users.
The \emph{TxnManager} handles dependency resolution (Line 4) among state transactions and insert decomposed operations to construct a \tpg. We discuss the detailed two-phase \tpg construction process in Section~\ref{subsec:construction}.

In the second phase  (i.e., transaction processing phase), 
the \emph{TxnManager} is first involved again to refine (Line 6) the constructed \tpg with further dependency resolution.
The \emph{TxnScheduler} 
schedules operations for concurrent execution based on the constructed \tpg according to the three dimensions of scheduling decisions (Line 7). 
In particular, a scheduling decision model $M$ is instantiated based on the constructed \tpg (Line 14).
\textbf{\circled{1}} Guided by $M$, execution threads adopt an exploration strategy (Section~\ref{subsec:explore}) to explore the constructed \tpg for operations available to be scheduled constrained by dependencies. 
\textbf{\circled{2}} 
During exploration, one or multiple operations may be treated as the 
% basic 
unit of scheduling (Section~\ref{subsec:granularity}). 
Subsequently, \textbf{\circled{3}} every thread executes operation(s) in the unit of scheduling with various abort handling mechanisms (Section~\ref{subsec:abort_handling}).
Only when state transactions are processed (i.e., committed or aborted) can the associated input events be postprocessed (Line 8) by the \emph{StreamManager} based on transaction processing results.
\end{comment}

\begin{comment}
\begin{algorithm}
\footnotesize
    \KwData{$e$ \tcp{Input event}}
    \KwData{$txn_{ts}$ \tcp{State transaction}}
    \KwData{$G$ \tcp{The currently constructed TPG}}
    \While{!finish processing of input streams}{
        \eIf(\tcp*[h]{Phase 1}){\text{$e$ is not a $punctuation$}}{
                $txn_{ts}$ $\gets$ PRE\_Processing($e$)\;
                \textbf{TPG\_Construction}($G$, $txn_{ts}$)\; 
          }(\tcp*[h]{Phase 2}){
                \textbf{TPG\_Refinement}($G$)\; 
                \textbf{TXN\_Scheduling}($G$)\; 
                POST\_Processing()\;
          }
    }
    
    \SetKwFunction{FMain}{TPG\_Construction}
    \SetKwProg{Fn}{Function}{:}{}
    \Fn{\FMain{$G$, $txn_{ts}$}}{
        $O_{1..k}$ $\gets$ \textbf{Partition} $txn_{ts}$\;
        \ForEach{\text{operation $O_{i}$ $\in$ $O_{1..k}$}}{
            \textbf{Identify} its \ld\;
            $G$ $\gets$ $G$ + $O_{i}$ \;
        }
    }
    \SetKwFunction{FMain}{TPG\_Refinement}
    \SetKwProg{Fn}{Function}{:}{}
    \Fn{\FMain{$G$}}{
        \ForEach{\text{vertex $e_{i}$ $\in$ $G$}}{
            \textbf{Identify} its \td, \pd\;
        }
    }
    
    \SetKwFunction{FMain}{TXN\_Scheduling}
    \SetKwProg{Fn}{Function}{:}{}
    \Fn{\FMain{$G$}}{
        $M$ $\gets$ Instantiated with $G$;\tcp{A decision model}
        \While{!finish scheduling of $G$
        }{
          \textbf{\circled{2}} $Scheduling Unit$ $\gets$ \textbf{\circled{1}} \emph{Explore}($G$, $M$)\; 
            \textbf{\circled{3}} \emph{Execute with Abort Handling} ($Scheduling Unit$)\; 
        }
    }
  \caption{Execution Outline of \system}
  \label{alg:algo}
\end{algorithm}
\end{comment}

\end{document}



