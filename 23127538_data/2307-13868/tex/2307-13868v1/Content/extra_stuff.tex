
\begin{comment}
\begin{lemma}[Equivalent characterization for density of a potential outcome]
Assume the setup described in \ref{setup}. Further, suppose that:
\begin{enumerate}
    \item The treatment assignment is ignorable: $\parens*{\mathbf y_i(1), ..., \mathbf y_i(K)} \indep \mathbf t_i \cond \mathbf x_i$,
    \item The treatment assignments are positive for all levels of the covariates: $\prob{\mathbf t_i = k \cond \mathbf x_i = x} > 0$ for any $x \in \mathcal X$, and
    \item No interference: if $i \neq j$, then $\mathbf y_i(k) \indep \mathbf t_j$ for all $k \in [K]$.
\end{enumerate}
Then:
\begin{align*}
    f_{\mathbf y_i(k)}(y) &= \int_{\mathcal X}f_{\mathbf y_i | \mathbf t_i = k, x}(y) f_{\mathbf x_i}(x) \,\text d x.
\end{align*}
\label{lem:equiv_char}
\end{lemma}
\begin{proof}
By definition, $f_{\mathbf y_i(k)}(y)$ can be expressed as a marginalization over the joint density $f_{\mathbf y_i(k), \mathbf x_i, \mathbf t_i}(y,t,x)$ with respect to $\mathbf x_i$ and $\mathbf t_i$:
\begin{align*}
    f_{\mathbf y_i(k)}(y) &= \int_{\mathcal X \times \mathcal T}
    f_{\mathbf y_i(k) | t, x}(y)f_{\mathbf x_i, \mathbf t_i}(x, t)\,\text d (x,t),
\end{align*}

Using the definition of conditional probability gives:
\begin{align*}
    f_{\mathbf y_i(k)}(y) &= \int_{\mathcal X \times \mathcal T}
    f_{\mathbf y_i(k) | t, x}(y)\prob{\mathbf t_i = t | \mathbf x_i = x}f_{\mathbf x_i}(x)\,\text d (x,t).
\end{align*}
That the treatment levels are positive for all levels of the covariates gives that this quantity is well-defined. By Fubini's theorem, and using that $\mathcal T = [K]$ is discrete:
\begin{align*}
    f_{\mathbf y_i(k)}(y) &= \int_{\mathcal X}\bracks*{\sum_{t \in [K]}\parens*{
    f_{\mathbf y_i(k) | t, x}(y) - f_{\mathbf y_i(l) | t, x}(y)}\prob{\mathbf t_i = t | \mathbf x_i = x}} f_{\mathbf x_i}(x)\,\text d x.
\end{align*}

By ignorability, $f_{\mathbf y_i(k) | t, x}(y) = f_{\mathbf y_i(k) | k, x}(y)$, so:
\begin{align*}
    f_{\mathbf y_i(k)}(y) - f_{\mathbf y_i(l)}(y) &= \int_{\mathcal X}\bracks*{\sum_{t \in [K]}
    f_{\mathbf y_i(k) | k, x}(y)\prob{\mathbf t_i = t | \mathbf x_i = x}} f_{\mathbf x_i}(x)\,\text d x.
\end{align*}
By consistency:
\begin{align*}
    f_{\mathbf y_i(k)}(y) &= \int_{\mathcal X}\bracks*{\sum_{t \in [K]}
    f_{\mathbf y_i | k, x}(y)\prob{\mathbf t_i = t | \mathbf x_i = x}} f_{\mathbf x_i}(x)\,\text d x.
\end{align*}
Finally, since $f_{\mathbf y_i | k, x}(y)$ is constant with respect to $t$:
\begin{align*}
    f_{\mathbf y_i(k)}(y) &= \int_{\mathcal X}
    f_{\mathbf y_i | k, x}(y)\bracks*{\sum_{t \in [K]}\prob{\mathbf t_i = t | \mathbf x_i = x}} f_{\mathbf x_i}(x)\,\text d x \\
    &= \int_{\mathcal X}
    f_{\mathbf y_i | k, x}(y) f_{\mathbf x_i}(x)\,\text d x, \,\,\,\, \sum_{t \in [K]}\prob{\mathbf t_i = t | \mathbf x_i = x} = 1.
\end{align*}
\end{proof}
\end{comment}


\begin{comment}
\paragraph{Causal treatment effects on the treated}
A related causal estimand which is commonly estimated in practice is the \textit{conditional average treatment effect on the treated} (CATT). This estimand is practically related to the CATE in \ref{eqn:hypo_cate}, but with a few subtle differences:
\begin{align*}
    \tau_x &\triangleq \expect{\mathbf y_i(2) \cond \mathbf x_i = x, \mathbf t_i = 2} - \expect{\mathbf y_i(1) \cond  \mathbf x_i = x, \mathbf t_i = 2}.
\end{align*}
The idea here is that, whereas $\gamma_x$ represents the realized difference in expectation of an item $i$'s counterfactuals conditional on having baseline covariates $x$, $\tau_x$ makes the more generic step of conditioning to take the difference in expectation of an item $i$'s counterfactuals \textit{only for the treated group}. The subtle difference is that with $\tau_x$, we make no statements about the entire population under consideration: we make only statements about the treated population. 

Since this concept is less restrictive than $\gamma_x$, intuitively, fewer requirements have to be met in order to estimate $\tau_x$ identifiably from the data. In particular, to estimate $\tau_x$, the only criteria we need are \textit{ignorability}, \textit{consistency} and \textit{exact matching}. While we already learned about consistency and ignorability, \textbf{exact matching} means that $F_{\mathbf x_i | \mathbf t_i = 2} = F_{\mathbf x_i | \mathbf t_i = 1}$: the distribution of the covariates are the same in the treatment and control groups. This has a key implication for condition 4 under Setup \ref{setup}: the covariates conditional on the treatments have positive support. For any $x \in \mathcal X$:
\begin{align*}
    f_{\mathbf x_i}(x) &= \sum_{t \in [2]}f_{\mathbf x_i | \mathbf t_i = t}(x)\prob{\mathbf t_i = t} \\
    &= f_{\mathbf x_i | \mathbf t_i = t'}(x)\sum_{t \in [2]}\prob{\mathbf t_i = t},\,\,\,\,F_{\mathbf x_i | \mathbf t_i = 1} = F_{\mathbf x_i | \mathbf t_i = 2} \\
    &= f_{\mathbf x_i | \mathbf t_i = t'}(x).\,\,\,\,\sum_{t \in [2]}\prob{\mathbf t_i = t} = 1\numberthis \label{eqn:cond4_CDTT}
\end{align*}
which holds for any $t' \in [2]$. This fact will be useful for hypothesis testing, in Section \ref{sec:methods}, for delineating that we are testing well-defined quantities.

To generalize the concept of an average treatment effect on the treated, we define the estimand:

\begin{definition}[Distributional treatment effect on the treated]
Suppose the setup described in \ref{setup} for the two-group setting, where group $2$ is defined as the treatment group, and group $1$ is defined as the control group. A causal categorical treatment effect on the treated (\textbf{DTT}) exists if for some $x \in \mathcal X$:
\begin{align*}
    F_{\mathbf y_i(2) | \mathbf t_i = 2} &\neq F_{\mathbf y_i(1) | \mathbf t_i = 2}
\end{align*}
\label{defn:DTT}
\end{definition}

The nuance here that differentiates this concept from the CoDiCE in Definition \ref{def:CDTE} for the case of $K=2$ is that the distributions of interest here are \textit{conditional} on the treatment group, and \textit{unconditional} on the covariates. This has the intuitive implication that, unlike the CoDiCE, inference here is limited to a population which is distributed like the treatment group, rather than like the entire population under investigation. However, we have strengthened the detected effect to be \textit{global} across levels of the covariates. A suitable hypothesis for a DTT is:

\begin{align}
    H_0 : F_{\mathbf y_i(2) | 2, x} = F_{\mathbf y_i(1) | 2, x}\text{    against     }H_A : F_{\mathbf y_i(2) | \mathbf t_i = 2} \neq F_{\mathbf y_i(1) | \mathbf t_i = 2}
    \label{eqn:hypo_causal_tt}
\end{align}

As before, with some general assumptions, we can express the densities of these potential outcomes in terms of the conditional densities:
\begin{lemma}
Suppose the setup describe in \ref{setup}, where group $2$ is defined as the treatment group, and group $1$ is defined as the control group. Further, suppose that:
\begin{enumerate}
    \item The treatment assignment is ignorable: $\parens*{\mathbf y_i(1), \mathbf y_i(2)} \indep \mathbf t_i \cond \mathbf x_i$,
    \item The covariates are exactly matched: $F_{\mathbf x_i | \mathbf t_i = 2} = F_{\mathbf x_i | \mathbf t_i = 1}$, and
    \item No interference: if $i \neq j$, then $\mathbf y_i(k) \indep \mathbf t_j$ for $k \in [2]$.
\end{enumerate}
Then:
\begin{enumerate}
    \item $$f_{\mathbf y_i(2) | \mathbf t_i = 2}(y) = \int_{\mathcal X}f_{\mathbf y_i | 2, x}(y)f_{\mathbf x_i}(x)\,\text d x,\text{ and}$$
    \item $$f_{\mathbf y_i(1) | \mathbf t_i = 2}(y) = \int_{\mathcal X}f_{\mathbf y_i | 1, x}f_{\mathbf x_i}(x)\,\text d x.$$
\end{enumerate}
\label{lem:CDTT_helper}
\end{lemma}
\begin{proof}
\, \\
1. Note that $f_{\mathbf y_i(2) | \mathbf t_i = 2}(y)$ can be expressed as a marginalization of $f_{\mathbf y_i(2), \mathbf x_i | \mathbf t_i = 2}(y)$ over $\mathbf x_i$. So:
\begin{align*}
f_{\mathbf y_i(2) | \mathbf t_i = 2}(y) &= \int_{\mathcal X}f_{\mathbf y_i(2) | 2, x}(y)f_{\mathbf x_i | \mathbf t_i = 2}(x) \,\text d x.
\end{align*}
Using that consistency gives that $f_{\mathbf y_i(2)|2, x}(y) = f_{\mathbf y_i | 2, x}(y)$ and Equation \ref{eqn:cond4_CDTT} gives:
\begin{align*}
f_{\mathbf y_i(2) | \mathbf t_i = 2}(y) &= \int_{\mathcal X}f_{\mathbf y_i | 2, x}(y)f_{\mathbf x_i | \mathbf t_i = 2}(x) \,\text d x.
\end{align*}
2. Note that $f_{\mathbf y_i(1) | \mathbf t_i = 2}(y)$ can be expressed as a marginalization of $f_{\mathbf y_i(2), \mathbf x_i | \mathbf t_i = 2}(y)$ over $\mathbf x_i$:
\begin{align*}
f_{\mathbf y_i(1) | 2, x}(y) &= \int_{\mathcal X}f_{\mathbf y_i(1) | 2, x}(y)f_{\mathbf x_i | \mathbf t_i = 2}(x)\,\text d x.
\end{align*}
By ignorability, $f_{\mathbf y_i(1) | 2, x}(y) = f_{\mathbf y_i(1) | 1, x}(y)$, so:
\begin{align*}
f_{\mathbf y_i(1) | 2, x}(y) &= \int_{\mathcal X}f_{\mathbf y_i(1) | 1, x}(y)f_{\mathbf x_i | \mathbf t_i = 2}(x)\,\text d x.
\end{align*}
Using consistency, $f_{\mathbf y_i(1) | 1, x}(y) = f_{\mathbf y_i | 1, x}(y)$:
\begin{align*}
f_{\mathbf y_i(1) | 2, x}(y) &= \int_{\mathcal X}f_{\mathbf y_i | 1, x}(y)f_{\mathbf x_i | \mathbf t_i = 2}(x)\,\text d x \\
&= \int_{\mathcal X}f_{\mathbf y_i | 1, x}(y)f_{\mathbf x_i}(x)\,\text d x.\,\,\,\,\text{Equation \ref{eqn:cond4_CDTT}}
\end{align*}
\end{proof}

Under assumptions given in Lemma \ref{lem:CDTT_helper}, a test of Equation \eqref{eqn:hypo_cond_kst} is a test of a CDTT:
\begin{theorem}[$k$-sample testing and CDTT]
Suppose the setup describe in \ref{setup}, where group $2$ is defined as the treatment group, and group $1$ is defined as the control group. Further, suppose that:
\begin{enumerate}
    \item The treatment assignment is ignorable: $\parens*{\mathbf y_i(1), \mathbf y_i(2)} \indep \mathbf t_i \cond \mathbf x_i$,
    \item The covariates are exactly matched: $F_{\mathbf x_i | \mathbf t_i = 2} = F_{\mathbf x_i | \mathbf t_i = 1}$, and
    \item No interference: if $i \neq j$, then $\mathbf y_i(k) \indep \mathbf t_j$ for $k \in [2]$.
\end{enumerate}
Then a consistent test of Equation \eqref{eqn:hypo_cond_kst} is equivalent to a consistent test of \eqref{eqn:hypo_causal_tt} for a CDTT.
\label{thm:CDTT}
\end{theorem}

\begin{proof}
Recall that for Equations \ref{eqn:hypo_cond_kst} and \ref{eqn:hypo_causal_tt} that since the density fully determines a distribution, that a difference in distribution exists $\iff$ a difference in the densities exists.

By Lemma \ref{lem:CDTT_helper}, Equation \ref{eqn:hypo_causal_tt} can be expressed in terms of the relevant densities from Equation \ref{eqn:hypo_cond_kst}:
\begin{align*}
    f_{\mathbf y_i(k) | \mathbf t_i = 2}(y) - f_{\mathbf y_i(l) \mathbf t_i = 2}(y) &= \int_{\mathcal X}\parens*{f_{\mathbf y_i | k, x}(y) - f_{\mathbf y_i | l, x}(y)} f_{\mathbf x_i}(x) \,\text d x.
\end{align*}
By supposition 4 in Setup \ref{setup}, $f_{\mathbf x_i}(x) > 0$ for all $x \in \mathcal X$. Therefore, $f_{\mathbf y_i(k) | \mathbf t_i = 2}(y) \neq f_{\mathbf y_i(l) \mathbf t_i = 2}(y)$ if and only if there exists some $x \in \mathcal X$ s.t. $f_{\mathbf y_i | k, x}(y) \neq f_{\mathbf y_i | l, x}(y)$, as desired.
\end{proof}
The subtle difference between Theorems \ref{thm:CDTT} and \ref{thm:CDTE} is that we have replaced the \textit{positivity} condition in \ref{thm:CDTE} with the \textit{exact matching} condition in \ref{thm:CDTT}. In theory, the positivity condition might sound easier to achieve: all that we need to know is that there is a non-zero probability of an element being in a particular group given a baseline covariate. On the other hand, with exact matching, we need to \textit{know} that the covariate distributions are equal in each group. In practice, however, the implications of this are substantial. Without a fully randomized design, it is difficult to determine whether the positivity condition holds. However, we can take steps to ensure that the exact matching criterion \textit{approximately} holds. To do this, we leverage re-weighting approaches. Through re-weighting, we upweight (or downweight) individuals in the control group until the covariate distributions are conditionally approximately equivalent between the control and the treatment groups. In this sense, this estimand prioritizes \textit{internal validity} of inference, or the degree to which inference learned from the study sample (which, distributional, is \textit{aligned} to the treatment group) applies to the population of interest (which, in this case, is a population which has a distribution similar to the treatment group.
\end{comment}

% Kernel conditional discrepency test
% CoDiTE