%%%% ijcai23.tex

\typeout{IJCAI--23 Instructions for Authors}

% These are the instructions for authors for IJCAI-23.

\documentclass{article}
\pdfpagewidth=8.5in
\pdfpageheight=11in

% The file ijcai23.sty is a copy from ijcai22.sty
% The file ijcai22.sty is NOT the same as previous years'
\usepackage{ijcai23}

% Use the postscript times font!
\usepackage{times}
\usepackage{soul}
\usepackage{url}
\usepackage[hidelinks]{hyperref}
\usepackage[utf8]{inputenc}
\usepackage[small]{caption}
\usepackage{graphicx}
\usepackage{amsmath}
\usepackage{amsthm}
\usepackage{booktabs}
\usepackage{algorithm}
\usepackage{algorithmic}
\usepackage[switch]{lineno}
%added by xenia
\usepackage{graphics}
\usepackage{multicol}
\usepackage{multirow}
\usepackage{xcolor}
\usepackage{amsmath}
\usepackage{amssymb}
\usepackage{amsthm}
\usepackage{enumitem}
\usepackage{bm}
\usepackage{booktabs}
\usepackage{units}
\definecolor{orange}{RGB}{255, 127, 0}
\newcommand{\xenia}[1]{\textcolor{orange}{[#1--lxt]}}
\newcommand{\ccc}[1]{\textcolor{blue}{[#1--ccc]}}
\newcommand{\lwm}[1]{\textcolor{red}{[#1--lwm]}}
\newcommand{\zpy}[1]{\textcolor{green}{[#1--zpy]}}
\newtheorem{theorem}{Theorem}
\theoremstyle{definition}
\newtheorem{definition}{Definition}%[section]
\newtheorem{proposition}{Proposition}%[section]
\newtheorem{assumption}{Assumption}%[section]
\newcommand{\modelname}{\texttt{HyperFed}}
\newcommand{\moduleAname}{\textsl{SPPG}} %, 
\newcommand{\moduleBname}{\textsl{RCDR}}
\usepackage{subfigure}
% \newcommand{\nosection}[1]{\vspace{2pt}\noindent\textbf{#1.}}
% end added by xenia
% Comment out this line in the camera-ready submission
% \linenumbers

\urlstyle{same}

% the following package is optional:
%\usepackage{latexsym}

% See https://www.overleaf.com/learn/latex/theorems_and_proofs
% for a nice explanation of how to define new theorems, but keep
% in mind that the amsthm package is already included in this
% template and that you must *not* alter the styling.
\newtheorem{example}{Example}

% Following comment is from ijcai97-submit.tex:
% The preparation of these files was supported by Schlumberger Palo Alto
% Research, AT\&T Bell Laboratories, and Morgan Kaufmann Publishers.
% Shirley Jowell, of Morgan Kaufmann Publishers, and Peter F.
% Patel-Schneider, of AT\&T Bell Laboratories collaborated on their
% preparation.

% These instructions can be modified and used in other conferences as long
% as credit to the authors and supporting agencies is retained, this notice
% is not changed, and further modification or reuse is not restricted.
% Neither Shirley Jowell nor Peter F. Patel-Schneider can be listed as
% contacts for providing assistance without their prior permission.

% To use for other conferences, change references to files and the
% conference appropriate and use other authors, contacts, publishers, and
% organizations.
% Also change the deadline and address for returning papers and the length and
% page charge instructions.
% Put where the files are available in the appropriate places.


% PDF Info Is REQUIRED.
% Please **do not** include Title and Author information
\pdfinfo{
/TemplateVersion (IJCAI.2023.0)
}

\title{HyperFed: Hyperbolic Prototypes Exploration with Consistent Aggregation for Non-IID Data in Federated Learning}

% % Single author syntax
% \author{
%     Author Name
%     \affiliations
%     Affiliation
%     \emails
%     email@example.com
% }

% Multiple author syntax (remove the single-author syntax above and the \iffalse ... \fi here)
% \iffalse
\author{
Xinting Liao$^1$
\and
Weiming Liu$^1$\and
Chaochao Chen$^{1}$\thanks{Chaochao Chen is the corresponding author.}\and
Pengyang Zhou$^1$\and
Huabin Zhu$^1$\and\\
Yanchao Tan$^2$\and
Jun Wang$^3$\And
Yue Qi$^3$
\affiliations
$^1$College of Computer Science and Technology, Zhejiang University, Hangzhou, China\\
$^2$College of Computer and Data Science, Fuzhou University, Fuzhou, China\\
$^3$OPPO Research Institute, Shenzhen, China\\
\emails
\{xintingliao, 21831010, zjuccc, zhoupy, zhb2000\}@zju.edu.cn,
yctan@fzu.edu.cn,
junwang.lu@gmail.com,
qiyue@oppo.com
}
% \fi

\begin{document}

\maketitle

\begin{abstract}
    Federated learning (FL) collaboratively models user data in a decentralized way.
    % 
    However, in the real world, 
    non-identical and independent data distributions (non-IID) among clients hinder the performance of FL due to three issues, i.e., (1) the class statistics shifting, (2) the insufficient hierarchical information utilization, and (3) the inconsistency in aggregating clients.
    % 
    To address the above issues, we propose \modelname~which contains three main modules, i.e., hyperbolic prototype Tammes initialization (HPTI), hyperbolic prototype learning (HPL), and consistent aggregation (CA).
    % 
     Firstly, HPTI in the server constructs uniformly distributed and fixed class prototypes, and shares them with clients to match class statistics, further guiding consistent feature representation for local clients.
    % 
    Secondly, HPL in each client captures the hierarchical information in local data with the supervision of shared class prototypes in the hyperbolic model space.
    % 
    % For handling issue 2, the hyperbolic prototype learning module in each client model local data in the hyperbolic model space with the supervision of fixed and shared class prototypes.
    % 
    Additionally, CA in the server mitigates the impact of the inconsistent deviations from clients to server. 
    % 
    Extensive studies of four datasets prove that \modelname~is effective in enhancing the  performance of FL under the non-IID setting.

\end{abstract}

% Figure environment removed

\section{Introduction}
Automatic 3D reconstruction of clothed humans using image inputs has gained increasing significance due to its potential applications in a wide array of AR/VR scenarios. High-fidelity reconstructions typically depend on sophisticated capture systems, which are developed with dense camera arrays~\cite{collet2015high,joo2015panoptic,joo2018total}, programmable light-stages~\cite{Vlasic2009, guo2019relightables}, and depth sensors~\cite{newcombe2011kinectfusion,DoubleFusion,BodyFusion,dou2016fusion4d,newcombe2015dynamicfusion}. However, stringent capture environments equipped with complex hardware pose significant challenges for consumer-level applications.


In this context, considerable research effort has been dedicated to developing methods that allow for more flexible capture configurations, such as utilizing a few RGB inputs. Among these works, learning implicit functions \cite{iccv2020PIFu, saito2020pifuhd, hong2021stereopifu} has proven effective in achieving highly detailed reconstructions by integrating the advancements of deep neural networks. These methods employ large multi-layer perceptrons (MLPs) to predict the occupancy probability or truncated signed distance function (TSDF) value of every queried 3D point based on its associated local feature, which is extracted from images. They can recover a continuous surface at arbitrary resolutions without topology restrictions.


However, in typical MLP-based implicit networks, the occupancy or TSDF value at each location is solved independently with planar image features, rendering them less capable of addressing challenging cases such as occlusions. Consequently, these methods suffer from generalization and robustness issues, particularly when tackling strong occlusions caused by large motion or multiple interacting humans. 
Some follow-up studies  \cite{zheng2021deepmulticap,zheng2021pamir,huang2020arch} utilize an extra geometric model, SMPL~\cite{Loper2015}, to improve robustness by introducing strong shape priors. 
Their success typically relies on the assumption of geometrical similarity \cite{huang2020arch} between the shape prior and target reconstruction, making them intractable for handling complex cases with loose clothes and sensitive to errors in SMPL model fitting.



%\ping{this paragraph sounds like `TSDF is better than MLP/SMPL, and we use TSDF to solve the problem'. But in Sec 3, we are telling a different story, saying `MLP needs a 3D convolutional encoder'. We need to make these two sections consistent.}\sicong{I think in this paragraph we claim that the TSDF}


%We opt for Trucated Signed Distance Funtion (TSDF) volumetric representations as they are naturally suitable for convolution operations, which have shown remarkable performance for learning hierarchical features on 2D visual perception tasks \cite{SunXLW19}. 
%Meanwhile, TSDF also describes the gradual geometry change around shape surface, which is not reflected by occupancy volume. 

We instead revisit the 3D volumetric representation and resort to 3D convolutional neural networks (CNNs) for feature learning, due to their impressive performance in feature learning and the ability to incorporate spatial context. However, volumetric methods and 3D convolution involve discretization, which might raise concerns regarding whether a discretized volume can preserve subtle geometric details as continuous representations learned in implicit functions. We investigate the relationship between volume resolution and quantization error on synthetic data by converting target mesh objects to TSDF volumes, as shown in Figure~\ref{fig:quantization_error}. We observe that the quantization errors are significantly reduced by increasing volume resolution and become nearly negligible when reaching a relatively high resolution (e.g., 512 or higher). In other words, achieving fine-detailed reconstruction is not supposed to be restricted by the use of volume representations as long as a proper volume resolution is utilized. Therefore, we present a method with high-resolution feature volumes, e.g., 256 and 512, while traditional volumetric methods \cite{varol18_bodynet,gilbert2018volumetric} are often limited to much lower resolutions, such as 32 or 128.



On the other hand, an increase in volume resolution may lead to a cubic growth of memory overhead \cite{8100085}. Reducing memory costs while guaranteeing the granularity of volumetric representations is necessary for pursuing high-quality reconstruction. Thus, we adopt a coarse-to-fine approach and cull away irrelevant voxels to build a sparse high-resolution feature volume. At the coarse level, the network computes an initial TSDF by applying a U-Net with sparse 3D CNN \cite{3DSemanticSegmentationWithSubmanifoldSparseConvNet} on the sparse feature volume, which is carved by a visual hull. Through our experiments, it turns out that more than 95\% of the volume grids are discarded by the visual hull culling, making the sparse 3D CNN efficient. At the fine level, the network focuses on a narrow band near the zero-level set of the initial TSDF and discretizes the narrow band with smaller voxels. By employing this narrow-band culling, we further shrink the sampling space, resulting in a relatively small range of grid numbers (usually 300K--500K in our experiments) even with a high volume resolution of 512. The remaining voxels in the narrow band are associated with features that fuse high-frequency information from the computed normal maps upon the low-frequency shape from the coarse level to compute the TSDF at high resolution. The final mesh is then extracted from the TSDF using the Marching-Cube algorithm ~\cite{Lorensen87marchingcubes}.
% Different from the u-net sturcture to preserve global topology context, we then apply a shallow 3dcnn to compute the final TSDF $D_{final}$ which contain more local geometry detail.




% \ping{this paragraph can be expanded. It is an important contribution and often ignored by other works. stress on the novel idea of regressing blending weights instead of colors}

In addition to geometry, high-quality mesh texture is also a crucial factor contributing to visual appearance. Directly computing a color field in 3D space, as in \cite{iccv2020PIFu}, struggles to capture high-frequency texture details, while the neural radiance field (NeRF) \cite{yu2020pixelnerf} or the DoubleField~\cite{shao2022doublefield} require expensive per-instance optimization and are often unstable for sparse input images. In contrast, we adopt an image-based rendering approach to compute a texture atlas map, which is efficient and widely supported in existing computer graphics tools. 
Specifically, we compute a blending weight at each 3D point on the mesh surface to determine its color as a weighted average of the colors at its image projections. The blending weights can be computed at a relatively coarse resolution, e.g., 512 volume resolution in our case, and leave texture details to the high-resolution images, such as 1K or 2K. Unlike previous methods that generate blurry texturing results under sparse input, our method generalizes well on both synthetic and real data with just a few input views. 
Figure~\ref{fig:teaser} shows two examples reconstructed by our method. Despite the challenging garment, pose, and occlusion, our method recovers faithful shape, normal, and texture on the right.

%with a wide variety of poses and clothing styles, and it is also adaptive to handle input image with arbitrary resolutions.
%\sicong{For this concern we claim that when the resolution of dicretized volume meets certain threshold (which is 256 in our experiment), the quantization error can be neglected.} 



In summary, the main contributions of this paper are as follows:
\begin{itemize}
\vspace{-0.1in}
  \item 
  We revisit the 3D volumetric representation and demonstrate that it can support clothed human reconstruction with equal or even better performance compared to implicit representation. 
  \item 
  We develop a memory and computation-efficient method for high-resolution volumetric reconstruction using sophisticated sparse 3D CNN, coarse-to-fine estimation, and voxel culling by visual hull and narrow bands. 
  \item 
  We introduce a novel method to compute a texture atlas map, which captures rich appearance details from high-resolution input images.
  \item 
  We achieve impressive results on standard benchmark datasets Twindom and MultiHuman, significantly reducing the point-2-surface (P2S) precision to approximately 0.2cm from just six input views, with more than $50\%$ error reduction compared to the state-of-the-art methods, including DoubleField~\cite{shao2022doublefield} and PIFuHD~\cite{saito2020pifuhd}.
\end{itemize}
%
\section{Related Work}
\label{appsec: related work}
Bayesian causal discovery literature has primarily focused on inference in linear models with closed-form posteriors or marginalized parameters. Early works considered sampling directed acyclic graphs (DAGs) for discrete~\cite{cooper1992bayesian, madigan1995bayesian, heckerman2006bayesian} and Gaussian random variables~\cite{friedman2003being, tong2001active} using Markov chain Monte Carlo (MCMC) in the DAG space. However, these approaches exhibit slow mixing and convergence~\cite{eaton2012bayesian,grzegorczyk2008improving}, often requiring restrictions on number of parents~\cite{kuipers2017partition}. %Alternative exact dynamic programming methods are limited to small settings~\cite{koivisto2012advances}. 

Recent advances in variational inference~\cite{zhang2018advances} have facilitated graph inference in DAG space, with gradient-based methods employing the NOTEARS DAG penalty \cite{zheng2018dags}.\cite{annadani2021variational} samples DAGs from autoregressive adjacency matrix distributions, while \cite{lorch2021dibs} utilizes Stein variational approach \cite{liu2016stein} for DAGs and causal model parameters. \cite{cundy2021bcd} proposed a variational inference framework on node orderings using the gumbel-sinkhorn gradient estimator \cite{mena2018learning}. \cite{deleu2022bayesian,nishikawa2022bayesian} employ the GFlowNet framework \cite{bengio2021gflownet} for inferring the DAG posterior. Most methods, except\cite{lorch2021dibs} are restricted to linear models, while \cite{lorch2021dibs} has high computational costs and lacks DAG generation guarantees compared to our method.
% at least quadratic scaling complexity, both with respect to the number of nodes (due to the DAG penalty) as well as number of posterior samples. Our proposed approach instead has linear complexity with respect to number of posterior samples and does not require any additional DAG penalty.     

In contrast, \emph{quasi-Bayesian} methods, such as DAG bootstrap \cite{friedman2013data}, demonstrate competitive performance. DAG bootstrap resamples data and estimates a single DAG using PC \cite{spirtes2000causation}, GES \cite{chickering2002optimal}, or similar algorithms, weighting the obtained DAGs by their unnormalized posterior probabilities. Recent neural network-based works employ variational inference to learn DAG distributions and point estimates for nonlinear model parameters \cite{charpentier2022differentiable,geffner2022deep}.
%
\section{Preliminaries}
In this section, we describe the necessary background for automated planning and the significance of the International Planning Competition. 

% \subsection{Ontology}
% A formal ontology is typically represented as a set of concepts, relations, and axioms. A concept represents a set of objects or entities that share common properties, while a relation represents a connection or association between two or more concepts. Axioms are statements that define the relationships between concepts and relations. It is a formal representation of knowledge that is designed to facilitate automated reasoning and information processing. It acts as a structured vocabulary that describes a domain and promotes interoperability, data integration, and communication between humans and machines. Formally, an ontology $O$ can be represented as a tuple $(C, R, A)$, where $C$ is the set of concepts, $R$ is the set of relations, and $A$ is the set of axioms. Each concept \textit{c} $\in$ $C$ can be represented as a set of attributes, denoted as $Att(c)$. Similarly, each relation \textit{r} $\in$ $R$ can be represented as a set of attributes, denoted as $Att(r)$.

% Ontology is a branch of philosophy that deals with the nature of existence and being. In the field of computer science, however, ontology refers to a formal representation of knowledge that is designed to facilitate automated reasoning and information processing. It is a structured vocabulary that describes a domain and promotes interoperability, data integration, and communication between humans and machines. Various tools and methodologies, including Protege and ontology editors, are available for ontology creation. Ontologies are increasingly important in artificial intelligence, knowledge engineering, and the semantic web, and researchers are exploring their potential in diverse domains and applications.

% Figure environment removed

\subsection{Automated Planning}

Automated planning, also known as AI planning, is the process of finding a sequence of actions that will transform an initial state of the world into a desired goal state \cite{ghallab2004automated}. It involves constructing a plan or a sequence of actions that will achieve a specified objective while respecting any constraints or limitations that may be present. Formally, automated planning can be defined as a tuple $(S, A, T, I, G)$, where:
\begin{itemize}
    \item $S$ is the set of possible states of the world
    \item $A$ is the set of possible actions that can be taken
    \item $T$ is the transition function that describes the effects of taking an action on the current state of the world
    \item $I$ is the initial state of the world
    \item $G$ is the desired goal state
\end{itemize}
Using this notation, the problem of automated planning can be framed as finding a sequence of actions $\prec a_1, a_2, ..., a_k\succ$ that will transform the initial state $I$ into the goal state $G$, while respecting any constraints or limitations on the actions. 
 % In automated planning, 
 A problem is defined in terms of a domain and a problem instance. The domain defines the possible actions that can be taken and the effects of each action, while the problem instance specifies the initial state of the world and the desired goal state. 
Various techniques can be used to solve the planning problem, such as search algorithms, constraint-based reasoning, and optimization methods. These techniques involve exploring the space of possible plans and selecting the one that satisfies the objective and any constraints. Figure \ref{fig:planning_bw} illustrates an automated planning scenario for the blocksworld domain, where an initial state can be transformed into a goal state by executing a sequence of actions.

% \noindent \textbf{Attributes modeled about a domain.}
%   %\noindent \textbf{Attributes modeled in a domain file}
%  \begin{enumerate}
%      \item \textbf{Requirements:} A list of requirements that the planner must satisfy in order to solve the domain. Requirements include durative actions, conditional effects, or negative preconditions. For example, in blocksworld domain with types involved, one of the requirements is \emph{typing}.
%     \item \textbf{Predicates:} Predicates are fundamental elements in the planning domain that define the properties of the world. They are used to describe the initial and goal states, as well as the preconditions and effects of actions. Predicates are usually defined as logical expressions over a set of variables, where each variable can take on a finite number of values. In the context of planning, predicates are typically used to represent facts about the world that can be true or false, such as the location of an object or the status of a machine. For example, in blocksworld domain, the predicate \verb|(on b1 b2)| could indicate that block 'b2' is on top of block 'b1'.
%      \item \textbf{Actions:} Actions are the basic units of change in the planning domain. They represent atomic operations that can be performed to transform the world from one state to another. Each action has a name, a set of parameters, preconditions that must be satisfied before the action can be executed, and effects that describe the changes that the action makes to the world. Actions can be used to model a wide variety of operations, ranging from simple movements or transformations to complex processes such as planning or decision-making. For example, in blocksworld domain, the action \verb|unstack b2 b1| can be used to unstack block 'b2' from block 'b1'. 
     
%      \item \textbf{Preconditions:} Preconditions are the conditions that must be true before an action can be executed. They are usually defined using predicates and can involve multiple variables. Preconditions can also be negative, which means that a certain condition must not be true for an action to be executed. In planning, preconditions ensure that actions are only executed when the necessary conditions have been met, such as ensuring that a machine is turned off before it is serviced. For example, in blocksworld domain, the action \verb|unstack b2 b1| has a precondition of \verb|(on b1 b2)|, meaning that for the action to be valid, the block 'b2' should be on top of block 'b1'.
     
%      \item \textbf{Effects:} Effects describe the changes that an action makes to the world. They are usually defined using predicates and can involve multiple variables. Effects can be positive, which means that a certain condition becomes true after the action is executed, or negative, which means that a certain condition becomes false after the action is executed. In the context of planning, effects are used to model the changes that result from executing an action, such as moving an object from one location to another or turning a machine on. For example, in blocksworld domain, when the action \verb|unstack b2 b1| is executed, one of its effect is \verb|(not (on b1 b2))|, indicating that block 'b2' is no longer on top of block 'b1'.
     
%      \item \textbf{Constants:} Constants are values that are fixed and do not change during the execution of the planning problem. They are used to represent objects or entities in the world that have a fixed value, such as the speed limit on a road. Constants can be used to simplify the planning problem by reducing the number of variables that need to be considered and by providing a fixed set of values that can be used in predicates and actions. For example, in blocksworld domain, the constant \emph{table} could represent the surface on which the blocks are initially placed.
     
%      \item \textbf{Types:} Types are used to classify objects or entities in the world based on their attributes or properties. They are used to define the domain of values that a variable can take on and can be used to constrain the values that are assigned to variables. In the context of planning, types are typically used to group related objects or entities together, such as cars or bicycles, and to specify the properties that are common to all members of a type, such as their color or size. For example, in blocksworld domain with types involved, one can represent the predicate as \verb|(on ?x - block ?y - block)| stating that the parameters in the predicate are of type \emph{block}.

%  \end{enumerate}


% ######### Shorter version for AI Planning preliminaries
% \subsection{Automated Planning}

% Automated planning, also known as AI planning, finds actions transforming an initial world state into a goal state \cite{ghallab2004automated}. It involves creating a plan, respecting constraints, defined as $(S, A, T, I, G)$ where $S$ is the world states set, $A$ is the actions set, $T$ is the state transition function, $I$ is the initial state, and $G$ is the goal state. The challenge is to find actions $\prec a_1, a_2, ..., a_k\succ$ converting $I$ to $G$ under constraints. 

% A problem has a domain (defining actions and effects) and an instance (specifying initial and goal states). Various techniques can be used to solve the planning problem, such as search algorithms, constraint-based reasoning, and optimization methods. These techniques involve exploring the space of possible plans and selecting the one that satisfies the objective and any constraints. Figure \ref{fig:planning_bw} illustrates an automated planning scenario for the blocksworld domain, where an initial state can be transformed into a goal state by executing a sequence of actions.

\noindent \textbf{Attributes modeled about a domain.}
 \begin{enumerate}
     \item \textbf{Requirements:} A list of requirements that the planner must satisfy to solve the given domain, e.g., \emph{typing} in blocksworld with types.
     \item \textbf{Predicates:} Define world properties, e.g., \verb|(on b1 b2)| in blocksworld.
     \item \textbf{Actions:} Units of change with preconditions and effects, e.g., \verb|unstack b2 b1| in blocksworld.
     \item \textbf{Preconditions:} Conditions for action execution, e.g., \verb|(on b1 b2)| for \\ \verb|unstack b2 b1|.
     \item \textbf{Effects:} Post-action world changes, e.g., \verb|(not (on b1 b2))| after \\ \verb|unstack b2 b1|.
     \item \textbf{Constants:} Fixed values, e.g., \emph{table} in blocksworld.
     \item \textbf{Types:} Classifications based on attributes, e.g., \\ \verb|(on ?x - block ?y - block)| in typed blocksworld.
 \end{enumerate}

\noindent \textbf{Attributes modeled about a problem instance from a domain.}
\begin{enumerate}
    \item \textbf{Name:} The name of the planning problem.
    \item \textbf{Domain:} The name of the planning domain that the problem belongs to.
    \item \textbf{Objects:} A list of objects that are present in the planning problem. Objects are typically defined in terms of their type and name. In the example shown in Figure \ref{fig:planning_bw}, objects are b1, b2, and b3.
    \item \textbf{Initial State:} A description of the initial state of the world, including the values of all relevant predicates. Figure \ref{fig:planning_bw} represents an example initial state.
    \item \textbf{Goal State:} A description of the desired goal state of the world, including the values of all relevant predicates. Figure \ref{fig:planning_bw} represents an example goal state.
\end{enumerate}

% \vspace{2cm}
\subsection{International Planning Competition (IPC)}

% IPC serves as a significant means of assessing and comparing various planning systems. By presenting new planners and benchmark problems each year, the competitions aim to stimulate the advancement of new planning methodologies and reflect current trends and challenges in the field. The competition comprises multiple tracks, each covering various planning problems such as classical, temporal, and probabilistic planning. These tracks include benchmark problems that evaluate the performance of planners concerning parameters such as plan quality, plan length, and run time. The results of these competitions provide insights into the current state-of-the-art in planning and help identify the strengths and weaknesses of different planning systems. IPC can serve as an excellent starting point for building a planning-related ontology as the benchmark problems used in these competitions can provide a comprehensive overview of the domain and the types of problems that planners need to solve. 

IPC is pivotal for evaluating and contrasting planning systems. Introducing new planners and benchmarks, it promotes innovative planning methodologies and reflects the field's evolving challenges. The competition has multiple tracks, such as classical and probabilistic planning, with benchmarks assessing plan quality, length, and run time. IPC results offer a glimpse into the latest in planning, highlighting system pros and cons. The benchmarks from IPC are ideal for crafting a planning-related ontology, encapsulating the domain's breadth and planners' challenges.

%
\section{Method} \label{method_hybridaugment}
In this section, we formally define the problem, motivate our work and then present our proposed techniques.


\subsection{Preliminaries}
Let $\mathcal{F}(x;W)$ be an image classification CNN trained on the training set $\mathcal{T}_\text{train} = (x_{i}, y_{i})^{N}_{i=1}$  with $N$ samples, where $x$ and $y$ correspond to images and labels. The clean accuracy (CA) of $\mathcal{F}(x;W)$ is formally defined as its accuracy over a clean test set $\mathcal{T}_\text{test} = (x_{j}, y_{j})^{M}_{j=1}$. Assume two operators ${A}(\cdot)$ and ${C}(c, s)$ that adversarially attacks or corrupts a given set of images with the corruption category $c$ and severity $s$, respectively.  Let $A\mathcal{T}_\text{test}$ and $C\mathcal{T}_\text{test}$ be the adversarially attacked and corrupted versions of $\mathcal{T}_\text{test}$, and let $\mathcal{F}(x;W)$ have a robust accuracy (RA) on $A\mathcal{T}_\text{test}$ and a corruption accuracy (CRA) on $C\mathcal{T}_\text{test}$. 
The aim is to fit $\mathcal{F}(x;W)$ such that the model gains robustness (\ie. increased RA and CRA compared its the baseline version), while retaining (or improving) the clean accuracy of its baseline version trained without robustness concerns.


\noindent \textbf{What we know.} Our work builds on the following crucial observations: i) CNNs favour high-frequency content \cite{wang2020high}, ii) adversaries and corruptions often reside in high-frequency \cite{wang2020towards}, iii) images are dominated by low-frequency \cite{Saikia_2021_ICCV} and iv) models relying on low-frequency components are more robust \cite{li2022robust,wang2020towards}. The robustness-accuracy trade-off is visible; low-frequency reliant models are more robust, but tend to miss out on clean accuracy brought by the high-frequency components. 

\subsection{HybridAugment}
We hypothesize that a \textit{sweet spot} in the robustness-accuracy trade-off can be found. Unlike the \textit{hard} approaches that completely rule out the reliance on high-frequency components (i.e. low-pass filters), we propose to \textit{reduce} the reliance on them. To this end, we adopt a data augmentation approach that aims to diversify $\mathcal{T}_\text{train}$ by an operation $\mathcal{HA(\cdot)}$. Keeping the strong relation intact between labels and low-frequency content (i.e. labels come from low-frequency-component image), we propose to swap high and low-frequency components of images in a batch on-the-fly. Unlike \cite{mukai2022improving}, we \textit{do not} restrict the images to belong to the same class; this diversifies the training distribution even further while preserving the image semantics. We call this basic version of our approach \textit{HybridAugment}, which corresponds to: 
%
\begin{equation} \label{hybrid_augment_paired}
    \mathcal{HA_{P}}(x_{i}, x_{j}) = \mathcal{LF}(x_{i}) + \mathcal{HF}(x_{j})
\end{equation}
%
where $x_{i}$ is the input image and $x_{j}$ is a randomly sampled image from the whole training set, which we simply sample from the mini batch at each training iteration in practice. $\mathcal{HF}$ and $\mathcal{LF}$ operators select the high and low-frequency components of an input image, for which we use:
%
\begin{equation} \label{eq:cutoff}
\begin{split}
    \mathcal{LF}(x) = GaussBlur(x) \\
    \mathcal{HF}(x) = x - \mathcal{LF}(x)
    \end{split}
\end{equation}
%
where $GaussBlur$ is used as a low-pass filter. Note that a similar outcome is possible by using Discrete Fourier Transforms (DFT), swapping the frequency bands and then applying Inverse DFT (IDFT). We find the gaussian blur operation to be faster and better in practice. 


Inspired from \cite{chen2021amplitude}, in addition to the image-pair scheme in Eq.~\ref{hybrid_augment_paired}, we propose a single image variant of \textit{HybridAugment}. In the single image variant, instead of combining two images, $x_i$ and $x_{j}$ are obtained by applying randomly sampled augmentations to a single image. The single image variant $\mathcal{HA_{S}}$ can therefore be defined as 
%
\begin{equation} \label{hybrid_augment_single}
    \mathcal{HA_{S}}(x_{i}) = \mathcal{LF}(Aug(x_{i})) + \mathcal{HF}(\hat{Aug}(x_{i}))
\end{equation}
%
where $Aug$ and $\hat{Aug}$ correspond to two sets of randomly sampled augmentation operations. Note that paired and single versions can work in tandem ($\mathcal{HA_{PS}}$), and actually outperform single or paired image versions. 


\subsection{HybridAugment++}


The frequency analysis is a vast literature, however, two core aspects often stand out; frequency-band analysis (i.e. low, high) and the decomposition of signals into amplitude and phase. \textit{HybridAugment} covers the former and shows competitive results in various benchmarks (see Section \ref{sec:exp_hybridaugment}). The latter is investigated in $\mathcal{APR}$ \cite{chen2021amplitude}, where phase is shown to be the more relevant component for correct classification, and training models based on their phase labels and swapping amplitude components of images randomly lead to more robust models. Note that frequency-band and phase/amplitude discussions are arguably orthogonal, since frequency, phase and amplitude provide distinct characterizations of a signal: intuitively speaking, frequency, phase and amplitude can be seen as the separation of visual patterns in terms of scale, location and significance. 


We hypothesize these two approaches can be complementary; a model reliant on low-frequency and spatial information (i.e. phase) can further improve robustness. Inspired by the successes of cascaded augmentation methods \cite{hendrycks2019augmix,wang2021augmax,calian2022defending}, we unify these two core aspects into a single, hierarchical augmentation method. We refer to this method as \textit{HybridAugment++} and define its paired version as:
%
\begin{equation}
  \mathcal{HA_{P}}^{++}(x_{i}, x_{j}, x_{z}) = \mathcal{APR_{P}}(\mathcal{LF}(x_{i}), x_{z}) + \mathcal{HF}(x_{j})
\end{equation}
%
where $x_{i}$, $x_{j}$ and $x_{z}$ are images sampled from the same batch. Here, $\mathcal{APR_{P}}$~\cite{chen2021amplitude} is defined as
\begin{equation}
    \mathcal{APR_{P}}(x_{i}, x_{z}) = \mathcal{IDFT}(A_{x_{z}} \otimes e^{i. P_{x_{i}}}) \\
\end{equation}
%
where $\otimes$ is element-wise multiplication, $A$ is the amplitude and $P$ is the phase component. Similar to $\mathcal{HA}$ and $\mathcal{APR}$, we also define a single-image version of \textit{HybridAugment++} as
%
\begin{equation}
 \mathcal{HA_{S}}^{++}(x_{i}) = \mathcal{APR_{S}}(\mathcal{LF}(Aug(x_{i}))) + \mathcal{HF}(\hat{Aug}(x_{i}))
\end{equation}
%
where $\mathcal{APR_{S}}$~\cite{chen2021amplitude} is defined as
%
\begin{equation}
\mathcal{APR_{S}}(x_{i}) = \mathcal{IDFT}\left(A_{\bar{Aug}(x_{i})} \otimes e^{i. P_{\overline{Aug}\left(x_{i}\right)}}\right)    
\end{equation}
%
where $Aug$, $\hat{Aug}$, $\bar{Aug}$ and $\overline{Aug}$ are different sets of randomly sampled augmentation operations. Note that we essentially propose a framework; one can use different single and paired image augmentations, either individually or together, and can still achieve competitive results (see ablations in Section \ref{sec:exp_hybridaugment}). There are also other alternatives, such as swapping phase/amplitude first and then performing $\mathcal{HA}$, but we observe poor performance in practice; dividing the phase component into frequency-bands is not interpretable as frequencies of the phase component are not well defined. The pseudo-code of our methods can be found in the supplementary material.




%
% \subsection{Performance Indicators}\label{subsection:performance_indicators_definition}
We use three categories of indicators to analyze the performance of a scenario definition:

\textbf{Inefficiency Rate}: it is the difference in fuel spent between a traffic scenario in which the aircraft comply to DAA resolution advisories (RA), and the analogous scenario in which each vehicle follows its optimal path, without observation of traffic separation rules. In case of vertical deviations, a higher fuel rate is required for climb maneuvers, a lower fuel rate in descent maneuver, which increase the net total for the mission. The resulting value of this indicator is the average value over all traffic configurations in a scenario set. 

\textbf{Loss of Separation (LoS) Rate}: despite there being several ways of defining traffic separation, we examine just the simplest one, which is checking whether or not the vehicles are separated by at least a fixed \emph{minimum distance}. In order to include both DAIDALUS and ACAS sXu in the same tables, we use one separation distance from each one, respectively: 4,000 ft, which is the Horizontal Miss Distance, or HMD, used in DO-365B \cite{DO_365} to define the so-called \emph{Hazard Alert Zone} (HAZ), associated to the DAA Well-Clear (DWC) concept of separation; and 2,000 ft, used in DO-396 \cite{DO_396}, that defines the Loss of Well-Clear (LoWC) event in relation to large UAVs or manned aircraft. These indicators will denote the rate of scenarios, in a scenario set, where the distance between any aircraft pair fell below the afore mentioned threshold values. 

\textbf{Timeout Rate}: this indicates, in a scenario set, the rate of scenarios where any aircraft exceeded a maximum time without reaching its destination point. As pointed out in section \ref{section:scenario_definitions}, this phenomenon occurs because DAA Resolution Advisories (RAs) cause long chains of maneuvers that extend beyond the energy/fuel allowance of the vehicle, due to shortcomings in coordination. We use a time threshold of 1,000 seconds but, in practice, the timeout would be determined by the energy/fuel capacity of the vehicle.

\textbf{Scenario Computing Time}: the time needed to simulate a single scenario instance, in a single core of an Intel Xeon CPU, discounted the fact that multiple scenario instances can be run in parallel in a multi-core CPU. In our simulated scenarios, the DAA algorithm is called at least each 2 seconds, for each aircraft, but when the aircraft is in avoidance mode, that can happen more often. In the case of ACAS sXu, the requirement of receiving various messages to update a single track contribute to result in multiple calls per simulated second.


\subsection{Scenario Specifications and Labels}
In this study, a scenario specification is defined by features such as: the DAA algorithm used, the dimensionality (2-D or 3-D), if it uses extrinsic priorities or not, the target separation parameter, and possibly other features. The scenario labels used in this section encode these attributes:
\begin{itemize}
\item \texttt{dai\_ip\_2d\_4k}: DAIDALUS without extrinsic priorities, 2-D maneuvering and regular 4 kft Horizontal Miss Distance (HMD);
\item \texttt{dai\_ep\_2d\_4k}: similar to the above, with extrinsic priorities;
\item \texttt{sxu\_ip\_2d\_2k}: ACAS sXu with intrinsic priorities only, 2-D maneuvering and regular 2 kft LoWC threshold;
\item \texttt{sxu\_ep\_2d\_2k}: similar to the above, with extrinsic priorities;
\item \texttt{dai\_ep\_3d\_4k}: similar to \texttt{dai\_ep\_2d\_4k}, with 3-D maneuvering;
\item \texttt{dai\_ep\_2d\_2k}: similar to \texttt{dai\_ep\_2d\_4k}, with HMD reduced to 2 kft.
\end{itemize}

And there are other features and labels that will be mentioned below as needed.

\subsection{Performance Analysis}
The analysis of selected scenario specifications is summarized in table~\ref{table:performance_analysis}. The first notorious observation in this table is the effect of extrinsic priorities to decrease inefficiency. With regards to safety indicators, their effect is mixed, and we have to observe each case separately. In the case of DAIDALUS, priorities decreased the 2 kft LoS rate and, most drastically, the timeout rate, while increased the 4 kft LoS rate. In the case of ACAS sXu, priorities increased both LoS rate indicators, but decreased the timeout rate drastically. Based on our rule of thumb assessment, it can be said that DAIDALUS works better with extrinsic priorities, while ACAS sXu works better without them. We conjecture that the following reasons explain this fact: i) that DAIDALUS has more built-in symmetries than ACAS sXu; ii) that ACAS sXu has already built-in priority rules for multi-aircraft encounters, and extrinsic priorities may contradict with them.

\begin{table}[h]
    \caption{Summary of closed-loop performance indicators per scenario.}
    \label{table:performance_analysis}
    \begin{center}
    \begin{tabularx}{\columnwidth}{|p{0.20\columnwidth}|p{0.11\columnwidth}|p{0.09\columnwidth}|p{0.09\columnwidth}|p{0.1\columnwidth}|p{0.105\columnwidth}|}
        \hline
    Scenario spec. & Ineffici-ency rate & LoS rate 4~kft & LoS rate 2~kft & Timeout rate & Scenario comp. time (s)\\
    \hline
    \texttt{dai\_ip\_2d\_4k} & 9.71\% & 1.4E-2 & 6.5E-5 & 1.6E-2 & 6.5E-2\\
    \texttt{dai\_ep\_2d\_4k} & 4.83\% & 2.4E-2 & 4.1E-5 & 0 & 5.1E-2\\
    \texttt{sxu\_ip\_2d\_2k} & 20.7\% & 8.7E-1 & 4.1E-2 & 1.6E-3 & 7.8E+1\\
    \texttt{sxu\_ep\_2d\_2k} & 10.9\% & 9.0E-1 & 2.0E-1 & 1.8E-5 & 7.8E+1\\
    \texttt{dai\_ep\_3d\_4k} & 4.38\% & 8.9E-6 & 0 & 0 & 7.4E-2\\
    \texttt{dai\_ep\_2d\_2k} & 1.3\% & 9.0E-1 & 1.1E-1 & 0 & 3.9E-2\\
    \hline
    \end{tabularx}
    \end{center}
\end{table}

According to a line of reasoning, it would be expected, that, in the more efficient scenarios, the aircraft fly closer to each other and, therefore, there should be a higher probability of losing separation. But this is not the only principle at play, because, if the aircraft perform deviations with the least extra distance, while keeping separation, they stay less in the air and decrease the total number of conflicts. This becomes more understandable when we compare \texttt{dai\_ep\_2d\_4k} with \texttt{dai\_ep\_3d\_4k}, where the latter achieved a small advantage in efficiency, but a huge one in safety. \texttt{dai\_ep\_3d\_4k} is capable of shortening the total distances, but has a residual cost associated to vertical maneuvers, where the climb maneuvers spend fuel at higher rates. 

It cannot escape from observation that DAIDALUS performed much better than ACAS sXu in almost all indicators. In our opinion, it would be reasonable to expect that ACAS sXu would not excel in the LoS rates, especially that of 4 kft, because its first protection criterion is 2 kft, as a built-in feature. However, with smaller protection volumes, the deviations should be smaller and, by this reasoning, its expected inefficiency would be lower than that of DAIDALUS. But our results show otherwise when we compare the cases of DAIDALUS with those of ACAS sXu. The only case in which ACAS sXu obtained an advantage was for the LoS rates comparison between \texttt{sxu\_ip\_2d\_2k} and \texttt{dai\_ep\_2d\_2k}, which have the same separation target. In any case, the Los rate obtained for ACAS sXu meets the performance requirement of ASTM F3442 \cite{ASTM}, which uses the definition of LoWC Ratio (LR), which is the ratio between the LoS rate of 2 kft shown in table~\ref{table:performance_analysis}, with DAA active, and corresponding LoS rate with DAA inactive, the latter being 0.785 according  to our simulations. Thus, the resulting LR scores for ACAS sXu here are 0.052 and 0.253, respectively for the two ACAS sXu specs, which are well below the value of 0.4 from \cite{ASTM}, and consistent with the performance analysis of \cite{DO_396}. We conjecture that these scores would be lower in a future 3-D scenario spec of a ACAS sXu, by following the same improvement obtained with DAIDALUS. 

\subsection{Possible approximations to the closed-loop behavior}
Trying to alleviate the heavy computational load to simulate large numbers of different traffic configurations, in this multi-aircraft, closed-loop setup, we considered some approximated solutions, such as the use of Deep Neural Networks to emulate the ACAS Xu/sXu behavior, in the lines followed by \cite{Julian2018,Bak2022}. However, the existing solutions that we found available were developed for just one intruder aircraft, so they were not suitable for our study. Another possibility would be the exploitation of symmetry transformations \cite{Sibai2020}, however the history-dependent nature of the ACAS Xu/sXu algorithms, associated to the present closed-loop setup, make this possibility unpractical. Thus, we started exploring simpler ways of deducing closed-loop behavior without having to perform the full simulation of a scenario. So far, we tried to analyze correlations between measures of open-loop maneuvers and the closed loop performance. The features that we explored are:
\begin{itemize}
    \item \textbf{Distance flown until the end of the first deviation maneuver} ($\overline{M/D}$): we consider the total distance flown until a ``Clear-of-Conflict'' (CoC) event happens, that is, after one or more divergent maneuvers start in a scenario instance, we stop the scenario when the first divergent maneuver of any aircraft finishes and that individual aircraft is clear of conflict, the moment from which some decision must be made on how to continue the mission. We count the total number of maneuvers started, and divide it by the sum of the flown distances, across all scenario instances in an execution set associated to a scenario spec.
    \item \textbf{Average angle deviation maneuver} ($\overline{\alpha}$): using the same stopping rule of above, we account the angle difference between the heading angles of the aircraft at the beginning of the divergent maneuver and at the stopping moment. 
\end{itemize}
For each of these measures, we ran the 122,416 traffic configuration instances with the open-loop stopping rule. Here, all the scenario specifications are 2-dimensional, and we use abbreviated lables to achieve a better display in the graph legends. Namely, the scenario specifications in this subsection are defined as:
\begin{itemize}
    \item \texttt{D1}: DAIDALUS without extrinsic priorities and with deterministic sensor data;
    \item \texttt{D2}: DAIDALUS with extrinsic priorities and deterministic sensor data;
    \item \texttt{D3}: DAIDALUS with extrinsic priorities and Sensor Uncertainty Mitigation (SUM). This is a design feature \cite{Narkawicz2018} to mitigate uncertainty in sensor data, as for example, to determine the position of an intruder aircraft;
    \item \texttt{D4}: DAIDALUS with its Horizontal Miss Distance (HMD) set to 200 ft (the standard is 4,000 ft) and uncertain sensor data;
    \item \texttt{X1}: ACAS sXu without extrinsic priorities;
    \item \texttt{X2}: ACAS sXu with extrinsic priorities;
    \item \texttt{X3}: ACAS sXu with scenario downscaled to speed of 43 knots and cell radius of 1 km.
\end{itemize}
We used the values of $\overline{M/D}$ and $\overline{\alpha}$ obtained for each of the specs above as inputs to a linear regressor of inefficiency, as defined in section~\ref{subsection:performance_indicators_definition}, and generated a plot with the pairs of (true, predicted) values from this regression, as shown in fig.~\ref{fig:inefficiency_regression}. The trend line in the figure, which depicts the regressor, seems to represent a strong correlation, which is confirmed by the value of $R^2$. When considering each of the regression inputs separately, we obtain $R^2=0.73$ for $\overline{M/D}$ and $R^2=0.77$ for $\overline{\alpha}$, which show that they contribute with approximately equal predicting power. 
% Figure environment removed

We performed a similar analysis for the LoS indicators, but we found very little correlation, as $R^2=0.11$ for the 2~kft LoS indicator. Nevertheless, it can be concluded that these open-loop measurements are a good proxy for the closed-loop inefficiency, with the advantage that the predictor discounts the bias that may have been introduced by the closed-loop mission management system, which is not part of the DAA specification. A rough estimate for the computing time saved is of 68\% in the inefficiency case.      

% 

\section{Experiment}

\subsection{Datasets and metrics}

% \noindent\textbf{Dataset.}
\subsubsection{Dataset}
% We adopt three datasets in our experiments, i.e., ClearGrasp \cite{sajjan2020clear}, TransCG \cite{fang2022transcg} and ClearPose \cite{chen2022clearpose}. The ClearGrasp dataset is the pioneering large-scale synthetic dataset that specifically focused on transparent objects. It provids a large-scale synthetic dataset as well as a real-world benchmark. The TransCG dataset comprises 57K RGB-D images from 130 different real-world scenes. 
% ClearPose dataset contains 350K RGB-D images of 63 household objects in real-world settings. Depth completion experiments and generalization verification (reported respectively in Section \ref{sec:depth} and \ref{sec:generalization}) are conducted on ClearGrasp, TransCG and ClearPose. Ablation study (reported in Section \ref{sec:ablation}) is performed on TransCG.
We use three datasets including ClearGrasp \cite{sajjan2020clear}, TransCG \cite{fang2022transcg}, and ClearPose \cite{chen2022clearpose}. The ClearGrasp dataset is a pioneering large-scale synthetic dataset that specifically focuses on transparent objects. It provides a large-scale synthetic dataset as well as a real-world benchmark. The TransCG dataset comprises 57K RGB-D images from 130 different real-world scenes. The ClearPose dataset contains 350K RGB-D images of 63 household objects in real-world settings. 
% We conducted depth completion experiments and generalization verification on ClearGrasp, TransCG, and ClearPose, reported respectively in Section \ref{sec:depth} and \ref{sec:generalization}. We performed an ablation study on TransCG, which is reported in Section \ref{sec:ablation}.

% ClearGrasp\cite{sajjan2020clear} is the first large-scale synthetic dataset as well as a real-world test benchmark focusing on transparent objects. TransCG\cite{fang2022transcg} is a large-scale real-world dataset, which contains 57K RGB-D images from 130 different scenes. ClearPose\cite{chen2022clearpose} is a recentily proposed real-world dataset, containing 350K RGB-D images covering 63 household objects.

% \newgeometry{letterpaper,top=60pt,bottom=43pt,left=48pt,right=48pt}
% \begin{table*}[!t]
% \caption{Ablation study. We show the impact of progressively substituting the components of the DFNet with ours. \label{tab:table1}
% }
% \centering
% \resizebox{\linewidth}{!}{%
% \begin{tabular}{cccccccccc}
% \toprule
% Model/Metric    & RMSE  & REL   & MAE   & $\delta$1.05 & $\delta$1.10 & $\delta$1.25          & Inference time (s)& Parameters & Size (MB)   \\ \midrule
% DFNet\cite{fang2022transcg}          & 0.018 & 0.027 & 0.012 & 83.76 & 95.67 & 99.71          & 0.0244s        & 1.25M & 4.819MB \\ \midrule
% New Loss        & 0.017 & 0.026 & 0.012 & 84.42 & 96.30 & \textbf{99.81} & 0.0244s        & 1.25M & 4.819MB \\ \midrule
% Shortcut Fusion & 0.017 & 0.024 & 0.011 & 86.18 & 96.67 & 99.79          & 0.0218s        & 1.02M & 3.919MB \\ \midrule
% Ours(slim) & 0.016          & 0.024          & 0.011          & 86.22          & 96.64          & \textbf{99.81} & \textbf{0.0143s} & \textbf{0.39M} & \textbf{1.518MB} \\ \midrule
% Ours       & \textbf{0.015} & \textbf{0.022} & \textbf{0.010} & \textbf{88.18} & \textbf{97.15} & \textbf{99.81} & 0.0153s          & 1.25M          & 4.803MB          \\
% \bottomrule
% \end{tabular}%
% }
% \end{table*}
\begin{table}[!t]
\renewcommand{\arraystretch}{1.05}
\setlength{\tabcolsep}{5pt}
\caption{Ablation study. We show the impact of progressively substituting the components of the DFNet with ours. \label{tab:table1}
}
\centering
\resizebox{\linewidth}{!}{%
\begin{threeparttable}
\begin{tabular}{cccccccccc}
\toprule
Model   & RMSE  & REL   & MAE   & $\delta$1.05 & $\delta$1.10 & $\delta$1.25          & Time(s)& Para(M) & Size (MB)   \\ \midrule
DFNet\cite{fang2022transcg}          & 0.018 & 0.027 & 0.012 & 83.76 & 95.67 & 99.71          & 0.0244        & 1.25 & 4.819 \\ \midrule
Huber Loss &0.017   &0.027  &0.012  &84.10  &95.82  &99.74 &0.0244  &1.25   &4.819  \\ \midrule
New Loss        & 0.017 & 0.026 & 0.012 & 84.42 & 96.30 & \textbf{99.81} & 0.0244        & 1.25 & 4.819 \\ \midrule
SF* & 0.017 & 0.024 & 0.011 & 86.18 & 96.67 & 99.79          & 0.0218        & 1.02 & 3.919 \\ \midrule
Ours(s)* & 0.016          & 0.024          & 0.011          & 86.22          & 96.64          & \textbf{99.81} & \textbf{0.0143} & \textbf{0.39} & \textbf{1.518} \\ \midrule
Ours       & \textbf{0.015} & \textbf{0.022} & \textbf{0.010} & \textbf{88.18} & \textbf{97.15} & \textbf{99.81} & 0.0153          & 1.25          & 4.803          \\
\bottomrule
\end{tabular}%
% \multicolumn{10}{l}{Note: NL* represents New Loss, SF* represents Shortcut Fusion and Ours(s)* represents Ours(slim).}
\begin{tablenotes}
\footnotesize
\item Note: SF* represents Shortcut Fusion and Ours(s)* represents Ours(slim).
\end{tablenotes}

\end{threeparttable}
}


\end{table}
% \vspace{-0.5cm}
\subsubsection{Metrics}
For evaluating the performance of our depth completion model, we employ four common metrics: RMSE, REL, MAE and Threshold $\delta$ (where $\delta$ is set to 1.05, 1.10, and 1.25). These metrics are calculated only on the transparent areas, as determined by transparent masks.
% Me use common metrics RMSE, REL, MAE and Threshold $\delta$ ($\delta$ is set to 1.05, 1.10 and 1.25) to evaluate our model. All metrics are calculated on the transparent areas according to transparent masks.


% We use three metrics to evaluate performance on pose estimation task. The average closest point distance (ADD-S)\cite{xiang2017posecnn} calculates the mean distance from each 3D model point to its closest neighbor on the target model. Followed DenseFusion\cite{wang2019densefusion} we report the area under the ADD-S curve (AUC) and the percentage of ADD-S smaller than 2cm ($<$2cm).

\subsection{Implementation Details}
% \noindent
% \textbf{Network configuration.}
\subsubsection{\bf Network Configuration}
% \textcolor{blue}{
In the network architecture, the number of hidden channels, \textbf{$C$}, is set to 64. Each FFEB/DFCB contains a single OSA module. Each OSA module is composed of 5 layers with stage channels of 20. The SFM module maintains \textbf{$C$} channels throughout the pipeline, while cross-layer shortcuts have only 1 channel. Residual connections between the encoder and decoder retain only \textbf{$C$} channels. The input head module and output head module use $3\times3$ convolution to adjust the number of channels and resolution (with resolution changes only occurring in the input head module). For the slim version, \textbf{$C$} is set to 32, and the OSA block contains 4 layers with stage channels of 16.
% }
% The hidden channels \textbf{$C$} in the network is set to 64. Each FFEB/DFCB contains one OSA module, in which, we use 5 layers per block and set stage channels \textbf{$C'$} to 20. SFM keeps \textbf{$C$} channels throughout the pipeline while cross-layer shortcuts take 1 channel only. Residual connections between encoder and decoder just keep channel \textbf{$C$}. $3\times3$ convolution is used in the input head module and the output head module to modify channels and resolution (resolution modified in the input head module only). For slim version, \textbf{$C$} is set to 32, \textbf{$C'$} is set to 16 and uses 4 layers per OSA block.

\subsubsection{\bf Training Details}
% \noindent
% \textbf{Training details.}
All experiments are carried out using the AdamW optimizer with an initial learning rate of $10^{-3}$. The learning rate is reduced by half after 5, 15, 25, and 35 epochs, and training continues for a total of 40 epochs with a batch size of 32. The threshold $\delta$ is kept constant at 0.1 during the training process. The weights $\alpha$ and $\beta$ for the loss function are set to 0.1 and 0.001, respectively. The images are resized to $320\times240$ for both training and testing. The experiments were conducted using an NVIDIA GeForce RTX 3090 GPU.
% We use AdamW optimizer with initial learning rate of $10^{-3}$ and multi-step learning rate scheduler which decays the learning rate by half after 5, 15, 25, 35 epochs. We train the model for 40 epochs with the batch size of 32. Threshold $\delta$ keeps 0.1 during training. Considering loss, we set $\alpha=0.1$, $\beta=0.001$. For all methods, we scale the images to $320\times240$ during training and testing. We use NVIDIA GeForce RTX 3090 for training and testing. 

 % Depth completion task and generalization ability are tested on ClearGrasp, TransCG and ClearPose. Pose estimation task is carried out on the set1 of ClearPose, since Clearpose has an accurate pose annotation without sticker. We use typical network DenseFusion\cite{wang2019densefusion} as pose estimation network. Following the learning strategy of DenseFusion, we train the network on 12G NVIDIA TITAN Xp GPU for 5 epochs with batch size of 128. The margin of refinement is set to 0.03. For fair comparison, we evaluate others works using their released source codes and optimal hyper-parameters or statistics reported in their paper.

\subsection{Ablation study} \label{sec:ablation}
We conduct an ablation study to investigate the effectiveness of our proposed components, including  new loss function, fusion branch, cross-layer shortcut and backbone structure. We take DFNet as baseline method since it is constructed following UNet structure. We  gradually replace its original components by our proposed ones and show the influence of using our proposed components. All the experiments of the ablation study are conducted on TransCG dataset.

% In view that DFNet is also constructed based on UNet, We here gradually replace its original components by our proposed. This study is conducted on TransCG dataset.
% To study the impact of each component in our proposed method, we perform experiments with different configurations of loss functions, network architecture, and backbones. Our method is compared against the recent transparent object depth completion work DFNet, which serves as our baseline. The ablation study experiments are all performed on the TransCG dataset.
% To verify the effectiveness of each component in our method, we evaluate the performance w.r.t. different configurations of loss functions, network architecture, and backbones. We use recently proposed transparent objects depth completion work DFNet as baseline. Ablation study is carried out on TransCG.




\subsubsection{\bf Loss Function}
The training of DFNet employs the mean squared error (MSE) and smooth loss as its loss function. However, these simple loss functions can lead to overfitting to local features, which makes the model more sensitive to the noise from low-level features such as edges and positions, negatively impacting its accuracy. To validate our proposed loss function, we first replaced the MSE loss with Huber loss in DFNet and termed it as Huber Loss. And then we replaced the loss function of DFNet with ours, leaving all other aspects unchanged and termed it as New Loss in Table \ref{tab:table1}. It can be observed by comparing New Loss with DFNet that all metrics showed improvement without requiring any additional parameters. 

% Qualitatively, the use of our proposed loss function can let the network to concentrate on the global structure rather than local details. By comparing the rows 3 and 4 of Figure \ref{fig:figure5}, the boundaries become smoother and even less distinct.
% The training of DFNET uses MSE and cosine distance. The simple loss function may lead to overfit to local features during training. This makes the model more sensitive to the noise of low-level features such as edge and position, which in turn affects its accuracy. So we propose a loss function consisting of Huber loss, SSIM loss and Smooth loss to suppress it. To verify its validity, we replaced the loss function of DFNet with ours and remain its other parts unchanged, then compared the results output by the mixed model (New Loss in Table \ref{tab:table1}) with the original one.
% All metrics are improved without extra parameters. Furthermore, we manually designed a feature to describe those pixels by computing the gradient of depth image and doing Gaussian blur to form an 'edge mask'. As their wights drop, the performance of the model is improved (Edge weight modified in Table \ref{tab:table2}), suggesting that it is necessary to treat pixels differently.
%and lower their weight during training. Specifically, we compute the gradient of depth image and do gaussian blur to form an 'edge mask'. Result (Edge weight modified in Table \ref{tab:table2}) supports our idea and shows it is necessary to treat pixels differently. 

\subsubsection{\bf Fusion Branch and Cross-layer Shortcuts}
In order to evaluate the impact of our proposed fusion branch and cross-layer shortcuts, we make changes to DFNet's architecture. First, we remove the redundant CDC blocks in DFNet from its skip connections, in line with our insight of preserving low-level features and the purpose of light weighting. Then, we added cross-layer shortcuts and a fusion branch to the modified network. It can be seen in Table \ref{tab:table1} that adopting this new architecture (referred to as Shortcut Fusion), almost all metrics show improvement with fewer parameters. 

\subsubsection{\bf Backbone}
We finally replace the denseblock in DFNet with our OSA module and utilized max pooling as the downsampling method. This final modification has transformed DFNet into our network. As shown in Table \ref{tab:table1}, our network outperforms the previous state-of-the-art (SOTA) by at least 16\% on difference-based metrics and improves ratio-based metrics by up to 4.42\%, resulting in a new SOTA performance. To make it practical for low-power robots, we created a slim version to balance speed and accuracy. 


% Qualitatively, figure \ref{fig:figure5} shows our method predicts clearer edges and is better handling crowded area.

% The fusion branch in our proposed network introduces a rich collection of low-level features, while the OSA module promotes feature reuse. Additionally, raw depth information is provided throughout the network, which enhances the representation of low-level features but may also hinder the learning of high-level semantic information. Our hypothesis is that the use of max pooling as a less aggressive downsampling method can mitigate these side effects while also reducing the number of parameters. The results in Table \ref{tab:table2} support our viewpoint.
% We fianlly relace the denseblock in DFNet by our used OSA module, and use max pooling as downsampling method. After this final modification, DFNet is tranformed to our proposed network. We thus show the performance by :Our"  in Table \ref{tab:table1}. It can be observed that ours outperforms previous SOTA by at least 16\% on difference-based metrics and improves ratio-based metrics by 0.1\% to 4.42\%, achieving the new state-of-the-art performance. In order to be capable in real applications, we also construct a slim version for speed/accuracy trade-off. 
% As we mentioned above, fusion branch introduces abundant low-level features and OSA encourages feature reuse. Furthermore, Raw depth is provided throughout the network. They enrich the representation of low-level features but may also harm to the learning of high-level semantic information. We suppose that using maxpooling to loosely downsampling may reduce their side effects as well as parameters saving. Result in Table \ref{tab:table2} proved our point of view.

% For summary, with our loss function, network tend to learn high-level features, with fusion branch, raw depth image and shortcuts, network can take advantage of low-level features. These components working together gives the network ability to take into account both local details and global structures. OSA module and max-pooling downsampling accelerate inference speed and reduce side effects.




% To intuitively show the impact of the proposed components, we visualize the predicted depth on TransCG and CleargGrasp dataset in Figure \ref{fig:figure5}. All networks are trained on TransCG dataset. Qualitatively, with our loss function, network is likely to focus on global structure rather than local detail. Red rectangle in row 3 and 4 show that with our loss function, boundaries become smoothy and even ambiguous, and outliers in the bottom right corner of the second column are suppressed. 



% FDCT performs domain adaption to the concatenation of raw depth and deep features and adopts maxpooling to lossly downsampling. It is supposed to reduce the disadvantage of the inaccuracy of raw depth. Our method predicts more accuracy and smooth edge as shown by the red circle on the left and the black square on the right. And even correct the ground truth as depicted in black circle on the right. The light spot reflected on the apple significantly affects the performance in row 2,3,5, but has little impact on row 4,6. Our methods successfully overcome the side effect of the raw depth information.

\subsection{Depth Completion Experiments} \label{sec:depth}

We compare our method with others on synthetic dataset ClearGrasp and real-world dataset TransCG. The quantitative results are respectively reported in Table \ref{tab:table2} and Table \ref{tab:table3}. Our proposed network surpasses others in almost every metric on these datasets which contain  synthetic and real-world scenes. Our method achieves a new state-of-the-art performance with a smaller model size and faster inference time, making it a highly competitive solution in this field.
%except on ClearGrasp synthetic validation set. It may be result of that the local implicit depth function which is environment-dependent, as well as the extra training data. 

% {\color{blue}
Specifically, our method outperforms the other methods by a larger margin in terms of REL and $\delta1.05$ metrics. This indicates its robustness to noise in the raw depth information, as these metrics are computed based on relative values and are sensitive to noise. Additionally, the gap between our method and others is larger in tests involving novel objects in ClearGrasp (CG Syn-novel in Table \ref{tab:table4} and the ClearGrasp column in Figure \ref{fig:figure5}), indicating that our method has a better ability to generalize to unseen objects. The qualitative results is reported in Figure \ref{fig:figure5}. The prediction of our method exhibits a clearer boundary and finer details than DFNet.
% }
% Specifically, our method has a bigger gap in REL and $\delta1.05$ to others most of the time. It demonstrates that our method is more stable to the noise in raw depth information of pixels, because these metrics are computed by relative value and significantly affected by noise. Noteworthy, the gap between our method and others getting bigger in the test of novel objects in most cases, indicates our method is able to generalize better to unseen objects.

\begin{table}[!t]
\caption{Depth Completion Result on TransCG dataset.}
\label{tab:table2}

\centering
\resizebox{\linewidth}{!}{%
\begin{tabular}{ccccccccc}
\toprule
Model & RMSE  & REL   & MAE   & $\delta1.05$ & $\delta1.10$ & $\delta1.25$ & Time ($\second$)   & Size ($\mega$B)    \\ \midrule
ClearGrasp\cite{sajjan2020clear}   & 0.054 & 0.083 & 0.037 & 50.48 & 68.68 & 95.28 & 2.281          & 934          \\
LIDF-Refine\cite{zhou2021pr}  & 0.019 & 0.034 & 0.015 & 78.22 & 94.26 & 99.80 & 0.018          & 251          \\
DFNet\cite{fang2022transcg}        & 0.018 & 0.027 & 0.012 & 83.76 & 95.67 & 99.71 & 0.024          & 4.8          \\
Ours (slim)   & 0.017 & 0.025 & 0.011 & 85.53 & 96.46 & 99.79 & \textbf{0.014} & \textbf{1.6} \\
Ours & \textbf{0.015} & \textbf{0.022} & \textbf{0.010} & \textbf{88.18} & \textbf{97.15} & \textbf{99.81} & 0.015 & 4.8 \\ \bottomrule
\end{tabular}}
% \vspace{-0.5cm}
\end{table}


\begin{table}[!t]
\renewcommand{\arraystretch}{0.9}
\caption{Depth Completion Results on ClearGrasp dataset\label{tab:table3}}
\centering
\resizebox{\linewidth}{!}{%
\begin{tabular}{ccccccc}
\toprule
\multicolumn{1}{c}{Model/Metric} &
  \multicolumn{1}{c}{RMSE} &
  \multicolumn{1}{c}{REL} &
  \multicolumn{1}{c}{MAE} &
  \multicolumn{1}{c}{$\delta$1.05} &
  \multicolumn{1}{c}{$\delta$1.10} &
  $\delta$1.25 \\ \midrule
\multicolumn{7}{c}{Train CG Test CG Syn-novel} \\ \midrule
\multicolumn{1}{c}{ClearGrasp} &
  \multicolumn{1}{c}{0.040} &
  \multicolumn{1}{c}{0.071} &
  \multicolumn{1}{c}{0.035} &
  \multicolumn{1}{c}{42.95} &
  \multicolumn{1}{c}{80.04} &
  98.10 \\ 
\multicolumn{1}{c}{Local Implicit} &
  \multicolumn{1}{c}{\underline{0.028}} &
  \multicolumn{1}{c}{\underline{0.045}} &
  \multicolumn{1}{c}{\underline{0.023}} &
  \multicolumn{1}{c}{\underline{68.62}} &
  \multicolumn{1}{c}{\underline{89.10}} &
  \underline{99.20} \\ 
\multicolumn{1}{c}{DFNet} &
  \multicolumn{1}{c}{0.032} &
  \multicolumn{1}{c}{0.051} &
  \multicolumn{1}{c}{0.027} &
  \multicolumn{1}{c}{62.59} &
  \multicolumn{1}{c}{84.37} &
  98.39 \\ 
\multicolumn{1}{c}{FDCT (Ours)} &
  \multicolumn{1}{c}{\textbf{0.025}} &
  \multicolumn{1}{c}{\textbf{0.040}} &
  \multicolumn{1}{c}{\textbf{0.021}} &
  \multicolumn{1}{c}{\textbf{71.66}} &
  \multicolumn{1}{c}{\textbf{92.95}} &
  \textbf{99.64} \\ \midrule
\multicolumn{7}{c}{Train CG Test CG Syn-known} \\ \midrule
\multicolumn{1}{c}{Local Implicit} &
  \multicolumn{1}{c}{\textbf{0.012}} &
  \multicolumn{1}{c}{\textbf{0.017}} &
  \multicolumn{1}{c}{\textbf{0.009}} &
  \multicolumn{1}{c}{\textbf{94.79}} &
  \multicolumn{1}{c}{\textbf{98.52}} &
  99.67 \\ 
\multicolumn{1}{c}{ClearGrasp} &
  \multicolumn{1}{c}{0.044} &
  \multicolumn{1}{c}{0.047} &
  \multicolumn{1}{c}{0.033} &
  \multicolumn{1}{c}{71.23} &
  \multicolumn{1}{c}{92.60} &
  98.24 \\ 
\multicolumn{1}{c}{DFNet} &
  \multicolumn{1}{c}{0.018} &
  \multicolumn{1}{c}{0.023} &
  \multicolumn{1}{c}{0.013} &
  \multicolumn{1}{c}{88.85} &
  \multicolumn{1}{c}{97.57} &
  \underline{99.92} \\ 
\multicolumn{1}{c}{FDCT (Ours)} &
  \multicolumn{1}{c}{\underline{0.015}} &
  \multicolumn{1}{c}{\underline{0.020}} &
  \multicolumn{1}{c}{\underline{0.012}} &
  \multicolumn{1}{c}{\underline{90.53}} &
  \multicolumn{1}{c}{\underline{98.21}} &
  \textbf{99.99} \\ \bottomrule

\end{tabular}%
% \tablen}
}
\end{table}



\subsection{Generalization Experiment} \label{sec:generalization}
% The generalization capability of a network is essential for practical applications. We evaluated the generalization ability of our proposed method from two perspectives: from synthetic images to real-world images and from one real-world dataset to another. The results of our experiments, shown in Table \ref{tab:table6}, indicate that our method (FDCT) has a comparable generalization capability to the state-of-the-art methods in cross-dataset evaluations, and it outperforms similar works in the synthetic-to-real test. However, it lags behind methods that focus solely on sim-to-real (noted as "local implicit*").
% The generalization ability of a network is critical for real-world application. The proposed method has a generalization ability that can be trained on synthetic data and aply to real world scene (syn-to-real) or trained on one real world dataset TransCG and adap to ClearGrasp (real-to-real). Comparison result is reported in Table \ref{tab:table4}. It shows that although there is still a certain gap compared with the method Local Implicit designed for syn-to-real; compared with the similar method DFNet, our method achieves a better result in the syn-to-real setting, and a competitive result in the syn-to-syn setting.
The generalization ability of a network is critical for real-world application. Our proposed method exhibits a high degree of generalization, being able to be trained on synthetic data and applied to real-world scenes (syn-to-real), or trained on one real-world dataset TransCG and adapted to the other real-world dataset (real-to-real), such as ClearGrasp. Comparison results are reported in Table \ref{tab:table4}, which show that while there is still a certain gap compared to the syn-to-real method (Local Implicit \cite{zhu2021rgb}), our method achieves better results in the syn-to-real setting when compared to the similar method DFNet, and competitive results in the real-to-real setting.

% We inspect the generalization ability of our proposed method from two aspects, from synthetic image to real-world image and from one real-world dataset to another. Experiment results in Table \ref{tab:table5} show that FDCT has a similar generalization ability to previous SOTA in cross-dataset and get better result in synthetic-to-real test compared to similar work, but is far below to methods focusing on sim-to-real.

% Since both datasets comprise real-world image, we train models on TransCG and test it on ClearGrasp real-world set for cross-dataset test. DFNet outperformed other method with a huge gap in generalization test and is chosen to be compared with ours. Comparison result is reported in Table \ref{tab:table5}. Our method outperforms the closest work in all metrics both for known and novel objects in synthetic-to-real test. There is a bigger gap between DFNet and ours in terms of novel objects. It might owe to a better utilization of RGB cues. Our method gets similar results to DFNet in cross dataset test, showing that our method has the ability to generalize from real-world dataset to another. With a series of real-world transparent objects datasets being proposed, we believe that the generalization ability in real-world is more important than sim-to-real.



% {\color{blue}
% Figure environment removed

\begin{table}[!t]
\caption{
% Result of Synthetic to Real and Cross Dataset Generalization Experiment
Generalization test on syn-to-real and real-to-real.}
\label{tab:table4}
\renewcommand{\arraystretch}{0.95}
\centering
\resizebox{\linewidth}{!}{%
% \begin{threeparttable}
\begin{tabular}{ccclclclclcl}
\toprule
\multicolumn{1}{c}{Model/Metric} &
  \multicolumn{1}{c}{RMSE} &
  \multicolumn{2}{c}{REL} &
  \multicolumn{2}{c}{MAE} &
  \multicolumn{2}{c}{$\delta$1.05} &
  \multicolumn{2}{c}{$\delta$1.10} &
  \multicolumn{2}{c}{$\delta$1.25} \\ \midrule
\multicolumn{12}{c}{Train CG Test CG Real-known (syn-to-real)} \\ \midrule
\multicolumn{1}{c}{Local Implicit\cite{zhu2021rgb}} &
  \multicolumn{1}{c}{\textbf{0.028}} &
  \multicolumn{2}{c}{\textbf{0.033}} &
  \multicolumn{2}{c}{\textbf{0.020}} &
  \multicolumn{2}{c}{\textbf{82.37}} &
  \multicolumn{2}{c}{\textbf{92.98}} &
  \multicolumn{2}{c}{\textbf{98.63}} \\ 
\multicolumn{1}{c}{DFNet} &
  \multicolumn{1}{c}{0.068} &
  \multicolumn{2}{c}{0.107} &
  \multicolumn{2}{c}{0.059} &
  \multicolumn{2}{c}{32.42} &
  \multicolumn{2}{c}{56.88} &
  \multicolumn{2}{c}{91.47} \\ 
\multicolumn{1}{c}{FDCT (Ours)} &
  \multicolumn{1}{c}{\underline{0.065}} &
  \multicolumn{2}{c}{\underline{0.103}} &
  \multicolumn{2}{c}{\underline{0.057}} &
  \multicolumn{2}{c}{\underline{33.08}} &
  \multicolumn{2}{c}{\underline{59.81}} &
  \multicolumn{2}{c}{\underline{91.70}} \\ \midrule
\multicolumn{12}{c}{Train CG Test CG Real-novel (syn-to-real)} \\ \midrule
\multicolumn{1}{c}{Local Implicit\cite{zhu2021rgb}} &
  \multicolumn{1}{c}{\textbf{0.025}} &
  \multicolumn{2}{c}{\textbf{0.036}} &
  \multicolumn{2}{c}{\textbf{0.020}} &
  \multicolumn{2}{c}{\textbf{76.21}} &
  \multicolumn{2}{c}{\textbf{94.01}} &
  \multicolumn{2}{c}{\textbf{99.35}} \\ 
\multicolumn{1}{c}{DFNet} &
  \multicolumn{1}{c}{0.051} &
  \multicolumn{2}{c}{0.088} &
  \multicolumn{2}{c}{0.046} &
  \multicolumn{2}{c}{31.23} &
  \multicolumn{2}{c}{64.66} &
  \multicolumn{2}{c}{97.77} \\ 
\multicolumn{1}{c}{FDCT (Ours)} &
  \multicolumn{1}{c}{\underline{0.043}} &
  \multicolumn{2}{c}{\underline{0.073}} &
  \multicolumn{2}{c}{\underline{0.038}} &
  \multicolumn{2}{c}{\underline{39.42}} &
  \multicolumn{2}{c}{\underline{75.54}} &
  \multicolumn{2}{c}{\underline{99.09}} \\ \midrule
\multicolumn{12}{c}{Train TCG Test CG Real-novel (real-to-real)} \\ \midrule
\multicolumn{1}{c}{Local Implicit\cite{zhu2021rgb}} &
  \multicolumn{1}{c}{0.152} &
  \multicolumn{2}{c}{0.225} &
  \multicolumn{2}{c}{0.139} &
  \multicolumn{2}{c}{9.86} &
  \multicolumn{2}{c}{20.63} &
  \multicolumn{2}{c}{46.02} \\ 
\multicolumn{1}{c}{DFNet} &
  \multicolumn{1}{c}{\textbf{0.041}} &
  \multicolumn{2}{c}{\textbf{0.054}} &
  \multicolumn{2}{c}{\textbf{0.031}} &
  \multicolumn{2}{c}{\textbf{62.74}} &
  \multicolumn{2}{c}{\textbf{83.31}} &
  \multicolumn{2}{c}{\textbf{97.33}} \\ 
\multicolumn{1}{c}{FDCT (Ours)} &
  \multicolumn{1}{c}{\textbf{0.041}} &
  \multicolumn{2}{c}{\underline{0.055}} &
  \multicolumn{2}{c}{\underline{0.032}} &
  \multicolumn{2}{c}{\underline{61.23}} &
  \multicolumn{2}{c}{\underline{82.84}} &
  \multicolumn{2}{c}{\underline{97.28}} \\ \bottomrule
\end{tabular}
%     \begin{tablenote}
%         \footnotesize
%         \item [*]Local Implicit is method aiming at sim-to-real.
%     \end{tablenote}
% \end{threeparttable}
}
%\vspace{-0.5cm}
\end{table}
% Figure environment removed
\subsection{Analysis} \label{sec:analysis}
In our proposed method, the loss function plays a crucial role in enabling the network to focus on structural information and alleviate the effects of unstable pixels. However, this focus on structural information may come at the expense of some details. On the other hand, the fusion branch and shortcuts draw attention to the details, which can introduce extra redundancy. Nonetheless, the use of maxpooling facilitates lossy and aggressive downsampling, which can reduce redundancy and improve robustness. The convolution based fusion method make better use of the raw depth image. All components work together and complement each other to achieve the best possible balance between structural information and details. In this section, we analyze the four critical components of our method and demonstrate their effectiveness.

\subsubsection{Influence of loss term}
% As we mentioned above, some unstable pixels can unwantedly make big penalty to the loss. By computing the gradient of the depth image and applying Gaussian blur, we manually created a feature to represent these pixels. As the weights of these pixels were reduced, the model's performance improved (as seen in Experiment of weight in Table \ref{tab:table5}), indicating the importance of treating pixels differently and pointing out the necessity of the so designed loss function. However, the side effect of such loss function is that the network pays too much attention to the structure and ignores some details. The highlighted area of the feature map changes from dotted to regional in the Loss column in Figure \ref{fig:figure6}.
As mentioned in \ref{section:Loss}, unstable pixels can have a significant negative influence on the calculation of the training loss. To illustrate this issue, we manually created a feature to represent these pixels by computing the gradient of the depth image and applying a Gaussian blur. By reducing the weights of these pixels, we observed an improvement in the model's performance (as seen in the Experiment of weight in Table \ref{tab:table5}), highlighting the importance of treating pixels differently and emphasizing the necessity of the used loss functions (especially the Huber Loss). Qualitatively, as shown in Figure \ref{fig:figure6}, the New Loss model places greater emphasis on the overall structure of transparent objects, as compared to DFNet, which primarily focuses on local information. The downside of such a loss function is that the network may ignore some details.
% Figure environment removed

\subsubsection{Low-level feature preservation}
% Fusion branch and cross-layer shortcuts alleviate the indistinct boundaries and perceptual details by taking more low-level cues into consideration. The highlighted area of the feature map changes from regional to scattered in the Fusion column in Figure \ref{fig:figure6}. Loss function and low-level feature awareness components together make a good trade-off between detail and structure information.
The fusion branch and cross-layer shortcuts help alleviate the issue of blurry boundaries and low perceptual details by incorporating more low-level cues. As a result, more low-level features such as object edges and holes are preserved in the feature map of Fusion model in Figure \ref{fig:figure6}. The combination of the loss function and low-level feature awareness components strikes a good balance between detail and structural information.

\subsubsection{Influence of downsampling}
Our hypothesis is that the use of max pooling as a lossy downsampling method can mitigate the side effects of the low-level awareness components while reducing the number of parameters. The results in Table \ref{tab:table5} that are noted as ``Experiment of downsampling'' support our viewpoint. It can be observed that the performance of using convolutional downsampling and average pooling is slightly worse than that of using max pooling.

% The loss function makes the network focus on structural information and alleviating the affects of unstable pixels, but may harming to the details. The fusion branch and shortcuts draws the attention to details, but may introduce extra redundancy. Maxpooling is used to lossy and aggressively downsampling. It can reduce redundancy and improve robustness. These components work together and complement each other.
% }

\subsubsection{Fusion method of depth image}
To demonstrate that fusing the raw depth image with feature map via convolution is better than directly concatenation. We removed the convolution layers used for fusion in the model Ours and named it Ours(concat). The result labeled Table ``Experiment on fusion method'' in Table \ref{tab:table5} support our viewpoint.

\begin{table}[!ht]
\centering
\caption{Experiment Result on Weight Modification, Downsampling Implementation and Fusion Method\label{tab:table5}}

\resizebox{\linewidth}{!}{%
\begin{tabular}{ccccccc}
\toprule
\multicolumn{1}{c}{Model/Metric} &
  \multicolumn{1}{c}{RMSE} &
  \multicolumn{1}{c}{REL} &
  \multicolumn{1}{c}{MAE} &
  \multicolumn{1}{c}{$\delta$1.05} &
  \multicolumn{1}{c}{$\delta$1.10} &
  $\delta$1.25 \\ \midrule
\multicolumn{7}{c}{Experiment on weight} \\ \midrule
\multicolumn{1}{c}{Baseline} &
  \multicolumn{1}{c}{0.018} &
  \multicolumn{1}{c}{0.027} &
  \multicolumn{1}{c}{0.012} &
  \multicolumn{1}{c}{83.76} &
  \multicolumn{1}{c}{95.67} &
  99.71 \\ 
\multicolumn{1}{c}{Edge Weight Modified} &
  \multicolumn{1}{c}{\textbf{0.017}} &
  \multicolumn{1}{c}{\textbf{0.025}} &
  \multicolumn{1}{c}{\textbf{0.011}} &
  \multicolumn{1}{c}{\textbf{85.34}} &
  \multicolumn{1}{c}{\textbf{96.26}} &
  \textbf{99.75} \\ \midrule
\multicolumn{7}{c}{Experiment on downsampling} \\ \midrule
\multicolumn{1}{c}{Conv Down} &
  \multicolumn{1}{c}{0.016} &
  \multicolumn{1}{c}{0.023} &
  \multicolumn{1}{c}{0.011} &
  \multicolumn{1}{c}{87.16} &
  \multicolumn{1}{c}{96.83} &
  99.80 \\ 
\multicolumn{1}{c}{AvgPooling Down} &
  \multicolumn{1}{c}{0.016} &
  \multicolumn{1}{c}{0.024} &
  \multicolumn{1}{c}{0.011} &
  \multicolumn{1}{c}{87.16} &
  \multicolumn{1}{c}{96.93} &
  99.80 \\ 
\multicolumn{1}{c}{MaxPooling Down} &
  \multicolumn{1}{c}{\textbf{0.015}} &
  \multicolumn{1}{c}{\textbf{0.022}} &
  \multicolumn{1}{c}{\textbf{0.010}} &
  \multicolumn{1}{c}{\textbf{88.18}} &
  \multicolumn{1}{c}{\textbf{97.15}} &
  \textbf{99.81} \\ \midrule
  \multicolumn{7}{c}{Experiment on fusion method} \\ \midrule
  \multicolumn{1}{c}{Ours(concat)} &
  \multicolumn{1}{c}{\textbf{0.015}} &
  \multicolumn{1}{c}{0.023} &
  \multicolumn{1}{c}{0.011} &
  \multicolumn{1}{c}{87.90} &
  \multicolumn{1}{c}{96.68} &
  99.80 \\ 
\multicolumn{1}{c}{Ours} &
  \multicolumn{1}{c}{\textbf{0.015}} &
  \multicolumn{1}{c}{\textbf{0.022}} &
  \multicolumn{1}{c}{\textbf{0.010}} &
  \multicolumn{1}{c}{\textbf{88.18}} &
  \multicolumn{1}{c}{\textbf{97.15}} &
  \textbf{99.81} \\ 
\bottomrule
\end{tabular}%
}
%\vspace{-0.5cm}
\end{table}
\vspace{-0.2cm}


\subsection{Pose Estimation Experiment}
In this experiment, we aim to demonstrate the applicability of our network for downstream tasks and to show that it can improve the accuracy of pose estimate.
To evaluate the performance of pose estimation, we use three evaluation metrics, i.e, the average closest point distance (ADD-S), the area under the ADD-S curve (AUC), and the percentage of ADD-S values that are smaller than 2 \centi\meter.
%\cite{xiang2017posecnn}
% The higher the metrics the stronger the performance.

% This experiment is carried out on the set1 of ClearPose, since Clearpose has an accurate pose annotation without sticker. We use typical network DenseFusion \cite{wang2019densefusion} as pose estimation network. Following the learning strategy of DenseFusion, we train the network on 12G NVIDIA TITAN Xp GPU for 5 epochs with batch size of 128. The margin of refinement is set to 0.03. For fair comparison, we evaluate others works using their released source codes and optimal hyper-parameters or statistics reported in their paper.
Both our method and DFNet are trained on the ClearPose Set 1 and are used to predict the depth of Set 1-Scene 5 for pose estimation purposes. The depth completion result is reported in Table \ref{tab:table6} and a screenshot of the live demonstration is reported in Figure \ref{fig:figure7}. In our experiments, we use DenseFusion \cite{wang2019densefusion}  as the pose estimation method. We trained DenseFusion with the restored depth and tested it on 3,000 randomly selected images. Ideally, a more accurate depth prediction can lead to improved performance in pose estimation. The results of our evaluations, presented in Table \ref{tab:table7}, indicate that the depth restored by our method outperforms DFNet in almost every object in the pose estimation task. This results validate that the depth map given by our method is more appropriate for addressing the downstream task, i.e., pose estimation.
% Depth completion models are trained on ClearPose set 1 and predict the depth of set 1-scene 5 for pose estimation. We train DenseFusion with the restored depth and test on 3k randomly chosen images. Metrics for each object are reported in Table \ref{tab:table7}. Result shows that the depth restored by FDCT outperforms DFNet's in almost every object in pose estimation task.
% \todo{format of tablehead!!}
\begin{table}[!t]
\caption{Depth Completion Results on ClearPose dataset.}
\label{tab:table6}
\centering
\begin{tabular}{ccccccc}
\toprule
Model & RMSE           & REL            & MAE            & $\delta$1.05          & $\delta$1.10          & $\delta$1.25          \\ \midrule
DFNet        & 0.048          & 0.038          & 0.033          & 76.36          & 94.22          & \textbf{99.40} \\
Ours         & \textbf{0.045} & \textbf{0.033} & \textbf{0.028} & \textbf{82.15} & \textbf{94.43} & 99.25          \\
\bottomrule
\end{tabular}%
\end{table}



\begin{table}[!t]
\caption{Pose Estimation Results on ClearPose dataset\label{tab:table7}}
\centering
\resizebox{\linewidth}{!}{%
\begin{tabular}{ccccccc}
\toprule
Models &
  \multicolumn{3}{c}{DFNet} &
  \multicolumn{3}{c}{Ours} \\ \midrule
Object/Metirc &
  \multicolumn{1}{c}{AUC} &
  \multicolumn{1}{c}{\textless{}2cm} &
  ADD-S(10\%) &
  \multicolumn{1}{c}{AUC} &
  \multicolumn{1}{c}{\textless{}2cm} &
  ADD-S(10\%) \\ 
beaker\_1 &
  \multicolumn{1}{c}{79.07} &
  \multicolumn{1}{c}{\textbf{0.00}} &
  0.68 &
  \multicolumn{1}{c}{\textbf{80.44}} &
  \multicolumn{1}{c}{\textbf{0.00}} &
  \textbf{7.53} \\ 
dropper\_1 &
  \multicolumn{1}{c}{\textbf{67.76}} &
  \multicolumn{1}{c}{61.00} &
  \textbf{48.00} &
  \multicolumn{1}{c}{31.70} &
  \multicolumn{1}{c}{\textbf{65.33}} &
  0.00 \\ 
dropper\_2 &
  \multicolumn{1}{c}{81.09} &
  \multicolumn{1}{c}{\textbf{33.10}} &
  1.78 &
  \multicolumn{1}{c}{\textbf{84.24}} &
  \multicolumn{1}{c}{0.00} &
  \textbf{9.61} \\ 
flask\_1 &
  \multicolumn{1}{c}{84.96} &
  \multicolumn{1}{c}{60.33} &
  42.33 &
  \multicolumn{1}{c}{\textbf{86.71}} &
  \multicolumn{1}{c}{\textbf{68.33}} &
  \textbf{68.00} \\ 
funnel\_1 &
  \multicolumn{1}{c}{78.85} &
  \multicolumn{1}{c}{91.33} &
  0.00 &
  \multicolumn{1}{c}{\textbf{82.91}} &
  \multicolumn{1}{c}{\textbf{98.33}} &
  \textbf{12.33} \\ 
cylinder\_1 &
  \multicolumn{1}{c}{78.77} &
  \multicolumn{1}{c}{48.33} &
  28.67 &
  \multicolumn{1}{c}{\textbf{79.83}} &
  \multicolumn{1}{c}{\textbf{77.00}} &
  \textbf{33.33} \\ 
cylinder\_2 &
  \multicolumn{1}{c}{62.75} &
  \multicolumn{1}{c}{54.67} &
  3.33 &
  \multicolumn{1}{c}{\textbf{75.68}} &
  \multicolumn{1}{c}{\textbf{58.67}} &
  \textbf{29.33} \\ 
pan\_1 &
  \multicolumn{1}{c}{86.76} &
  \multicolumn{1}{c}{13.67} &
  33.33 &
  \multicolumn{1}{c}{\textbf{89.37}} &
  \multicolumn{1}{c}{\textbf{53.67}} &
  \textbf{50.00} \\ 
pan\_2 &
  \multicolumn{1}{c}{88.71} &
  \multicolumn{1}{c}{84.67} &
  44.00 &
  \multicolumn{1}{c}{\textbf{89.73}} &
  \multicolumn{1}{c}{\textbf{90.33}} &
  \textbf{56.00} \\ 
pan\_3 &
  \multicolumn{1}{c}{\textbf{88.90}} &
  \multicolumn{1}{c}{87.67} &
  \textbf{53.33} &
  \multicolumn{1}{c}{88.10} &
  \multicolumn{1}{c}{\textbf{91.00}} &
  48.00 \\ 
bottle\_1 &
  \multicolumn{1}{c}{86.05} &
  \multicolumn{1}{c}{91.53} &
  24.41 &
  \multicolumn{1}{c}{\textbf{88.71}} &
  \multicolumn{1}{c}{\textbf{93.22}} &
  \textbf{31.53} \\ 
bottle\_2 &
  \multicolumn{1}{c}{71.81} &
  \multicolumn{1}{c}{83.16} &
  4.04 &
  \multicolumn{1}{c}{\textbf{77.01}} &
  \multicolumn{1}{c}{\textbf{88.22}} &
  \textbf{13.47} \\ 
stick\_1 &
  \multicolumn{1}{c}{69.53} &
  \multicolumn{1}{c}{32.32} &
  32.66 &
  \multicolumn{1}{c}{\textbf{79.60}} &
  \multicolumn{1}{c}{\textbf{57.58}} &
  \textbf{58.92} \\ 
syringe\_1 &
  \multicolumn{1}{c}{73.03} &
  \multicolumn{1}{c}{31.67} &
  25.67 &
  \multicolumn{1}{c}{\textbf{80.15}} &
  \multicolumn{1}{c}{\textbf{57.00}} &
  \textbf{47.00} \\ 
MEAN &
  \multicolumn{1}{c}{78.43} &
  \multicolumn{1}{c}{55.25} &
  24.45 &
  \multicolumn{1}{c}{\textbf{79.58}} &
  \multicolumn{1}{c}{\textbf{64.19}} &
  \textbf{33.22} \\


  \bottomrule
  \end{tabular}%
}
\vspace{-0.5cm}
\end{table}
%

\section{Conclusion}
    % 
    To enhance federated learning (FL) with non-IID data, we propose \modelname~which contains hyperbolic prototype Tammes initialization (HPTI) module, hyperbolic prototype learning (HPL) module, and consistent aggregation (CA) module.
    % 
    Firstly, HPTI constructs uniformly distributed and fixed class prototypes on server, and
    shares them with clients to guide consistent feature representation for local clients.
% 
    secondly, HPL models client data in the hyperbolic model space with the supervision of shared class prototypes.
    % 
    Additionally, CA mitigates the impact of inconsistent deviations from clients to server. 
    % 
    Extensive studies on four datasets prove the effectiveness of \modelname.
    % that \modelname~is effective in enhancing the performance of FL with non-IID data.
    % under the non-IID setting.
    % 

\section*{Acknowledgements}
This work was supported in part by the ``Pioneer” and ``Leading Goose” R\&D Program of Zhejiang (No.2022C01126), and Leading Expert of “Ten Thousands Talent Program” of Zhejiang Province (No.2021R52001).
% This work was supported in part by the ``Pioneer" and ``Leading Goose" R\&D Program of Zhejiang (No. 2022C01126), the National Natural Science Foundation of China (No. 62172362).
% \xenia{ant}


\bibliographystyle{named}
\bibliography{ijcai23}

\appendix
\section{Datasets}
We visualize the data distributions under different degrees of non-IID, i.e., $\alpha =\{0.1, 0.5, 5\}$ for EMNIST, FMNIST, Cifar10, and Cifar100 in Fig.~\ref{fig:alpha01}-Fig.~\ref{fig:alpha5}.

% Figure environment removed


% Figure environment removed

% Figure environment removed


\section{Details of Comparison Methods}
We compare \modelname~with three categories of state-of-the-art approaches according to their optimization goals, i.e., (1) optimizing global model: \textbf{FedAvg}~\cite{mcmahan2017communication}, \textbf{FedProx}~\cite{li2020federated}, \textbf{SCAFFOLD}~\cite{karimireddy2019scaffold}, \textbf{FedDYN}~\cite{acar2020federated}, \textbf{MOON}~\cite{li2021model}, (2) optimizing local personalized models: \textbf{FedMTL}~\cite{smith2017federated}, \textbf{FedPer}~\cite{arivazhagan2019federated}, \textbf{pFedMe}~\cite{t2020personalized}, \textbf{Ditto}~\cite{li2021ditto},
\textbf{APPLE}~\cite{luo2021adapt}, and (3) optimizing both global and local models: \textbf{Fed-RoD}~\cite{chen2021bridging}, \textbf{FedBABU}~\cite{oh2021fedbabu}, \textbf{SphereFed}~\cite{dong2022spherefed}. 
% 
% 
\textbf{FedAvg} is the first vanilla federated learning framework to collaborate among server and clients.
% 
\textbf{FedProx} takes a proximal term to regularize the change from global model to the local model.
% 
\textbf{SCAFFOLD} considers the variance of global model and local model when updating local gradients.
% 
\textbf{FedDYN} applies an dynamic regularizer to pull local model close to the global model, while to push local model away from  historical local model.
% 
\textbf{MOON} introduces contrastive learning to federated learning.
% 
\textbf{FedMTL} is an algorithm that takes personalized learning as a multi-task learning objective.
% 
\textbf{FedPer} decouples the global representation and local personalized classification.
% 
\textbf{pFedMe} uses Moreau envelopes as clients’ regularized loss functions to decouple personalized model optimization from the global model learning. 
% 
\textbf{Ditto} is a multi-task learning objective for federated learning that provides personalization while retaining similar efficiency. 
% 
\textbf{APPLE} adaptively learns to personalize the client models.
% 
\textbf{Fed-RoD}  explicitly decouples a model's dual duties with two prediction tasks. 
% 
\textbf{FedBABU} only updates the representation body of the model during federated training, and the head is fine-tuned for personalization during the evaluation process. 
% 
\textbf{SphereFed} is a hyperspherical federated learning framework to 
 address PFL problem.

We conduct all of the experiments on one NVIDIA RTX 3090 GPU with 24Gb Memory.
% 
% 
In the first category of methods, a hyper-parameter $\mu$ is mainly set to balance the local objective and regularization term (between the global model and local model).
% 
According to the optimal results, we set $\mu =0.01,1,0.1, $ for FedProx,  MOON, and FedDYN, respectively.
% 
Besides, we set $\tau=0.5$, the default temperature of MOON as specified in the original paper. 
% 
For all of the second category of methods as well as the third category of methods, we follow the personalized way as their original setting.
% 
We similarly finetune \modelname~and FedBABU with additional 5 local steps in P-FL.
% 
For pFedMe, we set regularization parameter $\mu=1$ for controlling the strength of personalized model, computation complexity $k=5$, the interpolation parameter $\beta=1$ for global weights.
% 
We choose hyper-parameter controlling the interpolation between global and local model , i.e., $\lambda=1 $ and local personalized iteration as 1 for Ditto.
% 
In APPLE, we take the hyper-parameter $\mu=0.001$, and the round number  $L=0.2$ for loss scheduler.

\section{Theoretical Analysis}
\subsection{Convex Optimization}
% Figure environment removed
% % Figure environment removed

For the simplified optimization of the convex combination between the toughest deviation and the virtual deviation:
\begin{equation}
\operatornamewithlimits{min}\limits_{p_{\tau}\in [0,1]} \frac{1}{2} \|p_{\tau}\Delta_{{\tau}}+(1-p_{\tau})\Delta_{{\text{vir}}}\|^2_2,
\end{equation}
we rewrite it as:
\begin{equation}
\min _{p_{\tau} \in[0,1]} J = \frac{1}{2} \left\|p_{\tau}\left(\Delta_{{\tau}}- \Delta_{{\text{vir}}}\right)+\Delta_{{\text{vir}}}\right\|_2^2.
\end{equation}

Next, we can analyze the solution according to its geometry, i.e., analyzing the directions among the toughest deviation, the virtual deviation, and their difference.  
% 
% 
As the geometry suggests in Fig.~\ref{fig:CU}, the solution of finding the minimized norm is either a perpendicular vector or an edge case.
% 
Thus we obtain the three cases as bellow:

\paragraph{Case 1} When it has the condition as:
\begin{equation}
\left\{ 
\begin{array}{l}  
\left(\Delta_{{\tau}}-\Delta_{{\text{vir}}}\right)^{\top} \cdot \Delta_{{\tau}}\ >0  \\
\left(\Delta_{{\tau}}-\Delta_{{\text{vir}}}\right)^{\top} \cdot \Delta_{{\text{vir}}} <0 ,
\end{array}
\right.
\end{equation}
%
Then we should take the differentiation $\frac{\partial J}{\partial p_{\tau}}=0$ and get:
\begin{equation}
\left(\Delta_{{\tau}}-\Delta_{{\text{vir}}}\right)^{\top}\left(p_{\tau} \Delta_{{\tau}} + (1-p_{\tau}) \Delta_{{\text{vir}}}\right)=0.
\end{equation}
%
Hence the solution is given:
\begin{equation}
\begin{gathered}
    p_{\tau}^*=\frac{\left(\Delta_{{\text{vir}}} - \Delta_{{\tau}} \right)^{\top} \cdot \Delta_{{\text{vir}}}}{\left\|\Delta_{{\tau}}- \Delta_{{\text{vir}}}\right\|_2^2} \quad \Leftrightarrow \quad \\
\Delta_{\boldsymbol{\theta}}^{t+1} =(1-p_{\tau}^*)\Delta_{{\text{vir}}}+ {p_{\tau}^*}\Delta_{\tau}.
\end{gathered}
\end{equation}
In this scenario, we find the perpendicular vector of the difference between the virtual deviation and the toughest deviation to obtain the minimum-norm.

\paragraph{Case 2} When it has the condition as:
\begin{equation}
\left(\Delta_{{\tau}}-\Delta_{{\text{vir}}}\right)^{\top} \cdot \Delta_{{\text{vir}}} \geq 0,
\end{equation}
the solution
\begin{equation}
p_{\tau}^*=0 \quad \Leftrightarrow \quad \Delta_{\boldsymbol{\theta}}^{t+1} =\Delta_{{\text{vir}}}.
\end{equation}
In this scenario, the $\left\|\Delta_{{\text{vir}}}\right\|_2^2$ is smaller than $\left\|\Delta_{{\tau}}\right\|_2^2$. The smaller deviation will hurdle the optimization on the virtual client. Therefore, we add weight on the virtual client.

\paragraph{Case 3} When it has the condition as:
\begin{equation}
\left(\Delta_{{\tau}}-\Delta_{{\text{vir}}}\right)^{\top} \cdot \Delta_{{\tau}} \leq 0,
\end{equation}
%
the solution
\begin{equation}
p_{\tau}^*=1, \quad \Leftrightarrow \quad \Delta_{\boldsymbol{\theta}}^{t+1} =\Delta_{{\tau}}.
\end{equation}
In this scenario, the $\left\|\Delta_{{\tau}}\right\|_2^2$ is smaller than $\left\|\Delta_{{\text{vir}}}\right\|_2^2$. The smaller deviation will hurdle the optimization on the toughest client. Therefore, we add weight on the toughest client.





\end{document}

