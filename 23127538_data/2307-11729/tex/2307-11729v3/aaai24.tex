%File: anonymous-submission-latex-2024.tex
\documentclass[letterpaper]{article} % DO NOT CHANGE THIS
\usepackage[]{aaai24}  % DO NOT CHANGE THIS
\usepackage{times}  % DO NOT CHANGE THIS
\usepackage{helvet}  % DO NOT CHANGE THIS
\usepackage{courier}  % DO NOT CHANGE THIS
\usepackage[hyphens]{url}  % DO NOT CHANGE THIS
\usepackage{graphicx} % DO NOT CHANGE THIS
\urlstyle{rm} % DO NOT CHANGE THIS
\def\UrlFont{\rm}  % DO NOT CHANGE THIS
\usepackage{natbib}  % DO NOT CHANGE THIS AND DO NOT ADD ANY OPTIONS TO IT
\usepackage{caption} % DO NOT CHANGE THIS AND DO NOT ADD ANY OPTIONS TO IT
\frenchspacing  % DO NOT CHANGE THIS
\setlength{\pdfpagewidth}{8.5in} % DO NOT CHANGE THIS
\setlength{\pdfpageheight}{11in} % DO NOT CHANGE THIS
%
% These are recommended to typeset algorithms but not required. See the subsubsection on algorithms. Remove them if you don't have algorithms in your paper.
\usepackage{algorithm}
\usepackage{algorithmic}

\newcommand{\ctext}[1]{\raise0.2ex\hbox{\textcircled{\scriptsize{#1}}}}
\usepackage{multirow}
\usepackage{color, colortbl}
\definecolor{Gray}{gray}{0.8}
\usepackage{amsmath}
\interfootnotelinepenalty=10000
% \renewcommand\UrlFont{\color{blue}\tt}
\usepackage{footmisc}
\usepackage{longtable}
\usepackage{multicol}
%
% These are are recommended to typeset listings but not required. See the subsubsection on listing. Remove this block if you don't have listings in your paper.
\usepackage{newfloat}
\usepackage{listings}
\usepackage{here}
\DeclareCaptionStyle{ruled}{labelfont=normalfont,labelsep=colon,strut=off} % DO NOT CHANGE THIS
\lstset{%
	basicstyle={\footnotesize\ttfamily},% footnotesize acceptable for monospace
	numbers=left,numberstyle=\footnotesize,xleftmargin=2em,% show line numbers, remove this entire line if you don't want the numbers.
	aboveskip=0pt,belowskip=0pt,%
	showstringspaces=false,tabsize=2,breaklines=true}
\floatstyle{ruled}
\newfloat{listing}{tb}{lst}{}
\floatname{listing}{Listing}
%
% Keep the \pdfinfo as shown here. There's no need
% for you to add the /Title and /Author tags.
\pdfinfo{
/TemplateVersion (2024.1)
}

% DISALLOWED PACKAGES
% \usepackage{authblk} -- This package is specifically forbidden
% \usepackage{balance} -- This package is specifically forbidden
% \usepackage{color (if used in text)
% \usepackage{CJK} -- This package is specifically forbidden
% \usepackage{float} -- This package is specifically forbidden
% \usepackage{flushend} -- This package is specifically forbidden
% \usepackage{fontenc} -- This package is specifically forbidden
% \usepackage{fullpage} -- This package is specifically forbidden
% \usepackage{geometry} -- This package is specifically forbidden
% \usepackage{grffile} -- This package is specifically forbidden
% \usepackage{hyperref} -- This package is specifically forbidden
% \usepackage{navigator} -- This package is specifically forbidden
% (or any other package that embeds links such as navigator or hyperref)
% \indentfirst} -- This package is specifically forbidden
% \layout} -- This package is specifically forbidden
% \multicol} -- This package is specifically forbidden
% \nameref} -- This package is specifically forbidden
% \usepackage{savetrees} -- This package is specifically forbidden
% \usepackage{setspace} -- This package is specifically forbidden
% \usepackage{stfloats} -- This package is specifically forbidden
% \usepackage{tabu} -- This package is specifically forbidden
% \usepackage{titlesec} -- This package is specifically forbidden
% \usepackage{tocbibind} -- This package is specifically forbidden
% \usepackage{ulem} -- This package is specifically forbidden
% \usepackage{wrapfig} -- This package is specifically forbidden
% DISALLOWED COMMANDS
% \nocopyright -- Your paper will not be published if you use this command
% \addtolength -- This command may not be used
% \balance -- This command may not be used
% \baselinestretch -- Your paper will not be published if you use this command
% \clearpage -- No page breaks of any kind may be used for the final version of your paper
% \columnsep -- This command may not be used
% \newpage -- No page breaks of any kind may be used for the final version of your paper
% \pagebreak -- No page breaks of any kind may be used for the final version of your paperr
% \pagestyle -- This command may not be used
% \tiny -- This is not an acceptable font size.
% \vspace{- -- No negative value may be used in proximity of a caption, figure, table, section, subsection, subsubsection, or reference
% \vskip{- -- No negative value may be used to alter spacing above or below a caption, figure, table, section, subsection, subsubsection, or reference

\setcounter{secnumdepth}{0} %May be changed to 1 or 2 if section numbers are desired.

% The file aaai24.sty is the style file for AAAI Press
% proceedings, working notes, and technical reports.
%

% Title

% Your title must be in mixed case, not sentence case.
% That means all verbs (including short verbs like be, is, using,and go),
% nouns, adverbs, adjectives should be capitalized, including both words in hyphenated terms, while
% articles, conjunctions, and prepositions are lower case unless they
% directly follow a colon or long dash
\title{OUTFOX: LLM-Generated Essay Detection Through In-Context Learning \\with Adversarially Generated Examples}
\author {
    % Authors
    Ryuto Koike\textsuperscript{\rm 1},
    Masahiro Kaneko\textsuperscript{\rm 2,1},
    Naoaki Okazaki\textsuperscript{\rm 1}
}
\affiliations {
    % Affiliations
    \textsuperscript{\rm 1}Tokyo Institute of Technology\\
    \textsuperscript{\rm 2}MBZUAI\\
    ryuto.koike@nlp.c.titech.ac.jp, masahiro.kaneko@mbzuai.ac.ae, okazaki@c.titech.ac.jp
}
% \author{
%     %Authors
%     % All authors must be in the same font size and format.
%     Written by AAAI Press Staff\textsuperscript{\rm 1}\thanks{With help from the AAAI Publications Committee.}\\
%     AAAI Style Contributions by Pater Patel Schneider,
%     Sunil Issar,\\
%     J. Scott Penberthy,
%     George Ferguson,
%     Hans Guesgen,
%     Francisco Cruz\equalcontrib,
%     Marc Pujol-Gonzalez\equalcontrib
% }
% \affiliations{
%     %Afiliations
%     \textsuperscript{\rm 1}Association for the Advancement of Artificial Intelligence\\
%     % If you have multiple authors and multiple affiliations
%     % use superscripts in text and roman font to identify them.
%     % For example,

%     % Sunil Issar\textsuperscript{\rm 2},
%     % J. Scott Penberthy\textsuperscript{\rm 3},
%     % George Ferguson\textsuperscript{\rm 4},
%     % Hans Guesgen\textsuperscript{\rm 5}
%     % Note that the comma should be placed after the superscript

%     1900 Embarcadero Road, Suite 101\\
%     Palo Alto, California 94303-3310 USA\\
%     % email address must be in roman text type, not monospace or sans serif
%     proceedings-questions@aaai.org
% %
% % See more examples next
% }

% %Example, Single Author, ->> remove \iffalse,\fi and place them surrounding AAAI title to use it
% \iffalse
% \title{My Publication Title --- Single Author}
% \author {
%     Author Name
% }
% \affiliations{
%     Affiliation\\
%     Affiliation Line 2\\
%     name@example.com
% }
% \fi

% \iffalse
% %Example, Multiple Authors, ->> remove \iffalse,\fi and place them surrounding AAAI title to use it
% \title{My Publication Title --- Multiple Authors}
% \author {
%     % Authors
%     First Author Name\textsuperscript{\rm 1},
%     Second Author Name\textsuperscript{\rm 2},
%     Third Author Name\textsuperscript{\rm 1}
% }
% \affiliations {
%     % Affiliations
%     \textsuperscript{\rm 1}Affiliation 1\\
%     \textsuperscript{\rm 2}Affiliation 2\\
%     firstAuthor@affiliation1.com, secondAuthor@affilation2.com, thirdAuthor@affiliation1.com
% }
% \fi


% REMOVE THIS: bibentry
% This is only needed to show inline citations in the guidelines document. You should not need it and can safely delete it.
% \usepackage{bibentry}
% END REMOVE bibentry

\begin{document}

\maketitle

\begin{abstract}
Large Language Models (LLMs) have achieved human-level fluency in text generation, making it difficult to distinguish between human-written and LLM-generated texts. 
This poses a growing risk of misuse of LLMs and demands the development of detectors to identify LLM-generated texts.
However, existing detectors lack robustness against attacks: they degrade detection accuracy by simply paraphrasing LLM-generated texts.
Furthermore, a malicious user might attempt to deliberately evade the detectors based on detection results, but this has not been assumed in previous studies.
In this paper, we propose \textbf{OUTFOX}, a framework that improves the robustness of LLM-generated-text detectors by allowing both the detector and the attacker to consider each other's output.
In this framework, the attacker uses the detector's prediction labels as examples for in-context learning and adversarially generates essays that are harder to detect, while the detector uses the adversarially generated essays as examples for in-context learning to learn to detect essays from a strong attacker.
Experiments in the domain of student essays show that the proposed detector improves the detection performance on the attacker-generated texts by up to +41.3 points F1-score. 
Furthermore, the proposed detector shows a state-of-the-art detection performance: up to 96.9 points F1-score, beating existing detectors on non-attacked texts.
Finally, the proposed attacker drastically degrades the performance of detectors by up to -57.0 points F1-score, massively outperforming the baseline paraphrasing method for evading detection.
\end{abstract}

% Figure environment removed

\section{Introduction}
LLMs, characterized by their enormous model size and vast training data, have demonstrated emergent abilities with impressive performance across various tasks \citep{wei2022emergent}. 
These abilities include a high degree of language comprehension, fluent generation, and the capacity to handle tasks unseen during training through in-context learning \citep{brown2020language,ouyang2022training,sanh2022multitask}.

% LM検出の必要性と現状
Despite these successes, there are growing concerns about the potential misuse of LLMs.
A notable example is in education, where students might copy and paste text generated by LLMs, such as ChatGPT \citep{chatgpt}, for their assignments. 
This concern has led to the development of detectors designed to identify LLM-generated text with promising detection performance \citep{kirchenbauer2023watermark,mitchell2023detectgpt,aitextclassifier,tang2023science}.

% LM検出における課題
%% 既存の検出器はパラフレーズアタックに弱く、そのディフェンスも未だ不十分
Unfortunately, existing detectors often perform poorly against simple attacks (e.g., paraphrasing), as highlighted by recent studies \citep{sadasivan2023aigenerated,krishna2023paraphrasing}.
A recent survey called for developing robust detection methods against other potential attacks designed to deceive the detectors \citep{tang2023science}.
Given the human-like generative abilities of LLMs, there's the unexplored risk that malicious users might exploit LLMs to create texts specifically designed to evade detection.

% 課題に対する我々の提案
%% アタックに対する検出器のロバスト性向上のためのUTFOXフレームワークの提案
Motivated by this need, we propose \textbf{OUTFOX}, a novel framework designed to enhance the robustness and applicability of LLM-generated text detectors. 
As illustrated in Figure \ref{fig:outfox}, OUTFOX introduces an approach where both the detector and the attacker learn from each other's output. The attacker uses the detector's predictions as examples for in-context learning to generate more sophisticated attacks. In contrast, the detector uses these adversarially generated texts as examples for in-context learning to improve its detection abilities against a strong attacker.
%% 実際に検出すべきエッセイドメインで検証
% To validate our approach in a realistic domain, we constructed a dataset consisting of 15,400 triplets of essay problem statements, native student-written essays, and LLM-generated essays.
To validate our approach in a realistic domain, we create a dataset of native-speaker student essay writing to detect LLM-generated essays. 
Our dataset contains 15,400 triplets of essay problem statements, student-written essays, and LLM-generated essays.

% 実験結果
%% 1. アタックを考慮した検出器がアタックに対してロバスト性(検出性能)向上が見られた
Experiments show that our OUTFOX detector substantially improves the detection performance on the attacker-generated texts by up to +41.3 points F1-score compared to without considering attacks.
%%% 1. in-context learningを用いることで、数個の事例からLLM(特にChatGPT)はアタックされた文とアタックされていない文の特徴の違いを学習できることが示唆された
This result empirically suggests that LLMs, especially ChatGPT, might learn implicit differences between non-attacked and attacked texts via in-context examples.
%% 2. アタックを考慮した検出器のオリジナルデータに対する副作用(性能低下)は最小限にとどまり、それどころかエッセイ生成モデルによっては性能向上が見られた
Interestingly, our OUTFOX detector performs consistently as well or even better on non-attacked texts, resulting in the state-of-the-art detection performance of up to 96.9 points F1-score on non-attacked texts.
% with an average decrease of only a -0.1 point F1-score.
%%% 2. 提案する検出器において、アタックを考慮することはアタック前のデータの検出に対する副作用が小さい
This result demonstrates that considering attacks in our detector has little negative effect on detection performance on non-attacked texts.
%% 3. 提案するアタックはパラフレーズよりも強力であった
Furthermore, our OUTFOX attacker can drastically degrade the performance of detectors, with a decrease of up to -57.0 points F1-score, which massively outperforms the baseline paraphrasing method for evading detection.
In our analysis, we explore the semantic differences between non-attacked and attacked essays. The analysis reveals that our attacker-generated essays are semantically closer to human-written essays than non-attacked essays, leading to success in such effective attacking. For reproducibility, we release our code and dataset.\footnote{\url{https://github.com/ryuryukke/OUTFOX}}
%%% (検出を間違えたラベルが事例に入ってなくてもアタックを生成できたかは調べたいところ)
%The details of our approach and experimental results will be discussed in the following sections.

% Figure environment removed

% Figure environment removed

\section{Related Work}
% In this section, we provide brief overviews in terms of LLM-generated Text Detection, Detection for assessing academic dishonesty, Attacking on LLM-generated Text Detection, and Defence against attacking LLM-generated Text Detection.

\paragraph{LLM-Generated Text Detection}
\label{baselines}
Tackling the malicious uses of LLMs, recent studies have proposed detectors to identify LLM-generated texts.
These detectors can be mainly categorized into watermarking algorithms, statistical outlier detection methods, and supervised classifiers.
Watermarking algorithms use token-level secret markers in texts that humans cannot recognize for detection \citep{kirchenbauer2023watermark}.
In order to embed the markers into texts, the probabilities of selected tokens by a hash function are modified to be higher in text generation at each time step.
Our work focuses not on the watermark-enhanced LMs, but on LLMs that are openly used in our daily life---hence our chosen domain of student essays.
Statistical outlier detection methods exploit statistical differences in linguistic features between human-written and LLM-generated texts. These includes n-gram frequencies \citep{badaskar-etal-2008-identifying}, entropy \citep{ent08}, perplexity \citep{bere16}, token log probabilities \citep{solaiman2019release}, token ranks \citep{ippolito-etal-2020-automatic}, and negative curvature regions of the model’s log probability \citep{mitchell2023detectgpt}. 
% These are zero-shot detectors and don't need to train a classifier.
% While these methods can reliably detect the texts regardless of text domains, they require white-box access to the model logits.
Supervised classifiers are models specifically trained to discriminate human-written and LLM-generated texts with labels. The classifiers range from classical methods \citep{ippolito-etal-2020-automatic,crothers2023machine} to neural-based methods \citep{solaiman2019release,bakhtin2019real,uchendu-etal-2020-authorship,rodriguez-etal-2022-cross,guo2023close}. 
% These methods can be reliable for detecting texts from various LMs without white-box access to the model logits.
% However, their detection performance can degrade quickly when shown texts outside their training distribution.

\paragraph{Detection for Assessing Academic Dishonesty}
In the educational context, LLM-based services (e.g., ChatGPT and Bard \cite{bard}) have the potential to help students with their writing. However, many schools consider these services as academic dishonesty because students can misuse these tools to cheat on assignments \citep{educatorconsiderations4chatgpt}.
Based on this, recent studies have investigated LLM-generated text detection for student assignments, such as argumentative essays \citep{liu2023argugpt} and university-level course problems \citep{ibrahim2023perception,vasilatos2023howkgpt}.
Specifically, \citet{liu2023argugpt} targeted the essay domain written by non-native English learners. However, considering the human-level generative capabilities of LLMs, it can be more difficult to classify LLM-generated and native-written essays than non-native-written ones. Thus, as a more challenging setting, we create a dataset to distinguish native-student-written from LLM-generated essays.

\noindent
\paragraph{Attacking LLM-Generated Text Detection}
\label{attacking}
Most recent studies have reported the effectiveness of paraphrasing attacks, where existing detectors experience a large loss in accuracy when given a paraphrased version of an LLM-generated text \citep{sadasivan2023aigenerated,krishna2023paraphrasing}.
For instance, \citet{krishna2023paraphrasing} proposed DIPPER, the 11B document-level paraphraser, controlling output diversity on vocabulary and content re-ordering.
Considering the human-level generative abilities of LLMs, malicious users might instruct LLMs to adversarially generate texts based on detection results, but this has not been explored in prior work.

\noindent
\paragraph{Defense against Attacking LLM-Generated Text Detection}
% アタックに対するディフェンスとその課題
The existence of these attacking methods, in turn, poses a need for a defense against such attacks. There are few studies specifically about defending against the attacks. \citet{krishna2023paraphrasing} proposed a retrieval-based defense that detects a text that is semantically similar to one of the LLM-generated responses in an API database.
However, it needs active actions by API providers, and based on the nature of the method, a false positive rate can be higher with a larger database.
In addition, \citet{sadasivan2023aigenerated} recently showed that the retrieval-based defense method is vulnerable against recursive paraphrasing. The detection accuracy significantly drops to 25\% after 5 rounds of paraphrasing by \citet{krishna2023paraphrasing}'s paraphraser.

In recent concurrent work by \citet{hu2023radar}, they proposed RADAR, a framework that jointly trains a robust LLM-generated-text classifier against the paraphrasing attack via adversarial learning.
In our OUTFOX framework, the detector and the attacker can learn from each other's adversarially generated output via in-context learning, without any parameter updates.
Thus, the detector in our framework can handle new attacks by simply adding these to in-context examples without additional fine-tuning.
Additionally, the attacker in our framework is not limited to the paraphrasing attack but is directly designed to deceive the detector.

\section{OUTFOX Framework}
\label{methodlogy}

\subsection{Task Formulation}
Our work focuses on distinguishing LLM-generated essays from human-written essays. We assume that a training instance for detecting LLM-generated essays consists of a triplet of an essay-problem statement, a human-written essay, and an LLM-generated essay.
In our framework, the attacker adversarially generates \textit{attacked} text, designed to fool the detector, while the detector tries to learn from examples to better distinguish LLM-generated and human-written text. Both the detector and attacker are able to \textit{consider} each other's outputs by including examples in their input prompts for in-context learning. The detector can consider essay-label pairs that can consist of human-written, \textit{attacked} (adversarially-generated by an attacker), and \textit{non-attacked} (generated by an LLM, but not adversarially) essays by including them in its input prompt. On the other hand, the attacker can consider examples of non-attacked text and the detector's label predictions in its input prompt.
% The details of our approach are as follows.

\subsection{The OUTFOX Detector}
\label{detector}
Figure \ref{fig:detector} illustrates the outline of our OUTFOX detector.
Given a target essay $e$ paired with a problem statement $p$, we retrieve the top-$k$ problem statements that are semantically close\footnote{We follow the setting of \citet{prabhumoye2022fewshot}, which found that leveraging semantically close examples is effective for in-context learning.
We use the vector space of all problem statements computed by Term Frequency-Inverse Document Frequency (TF-IDF) implemented in scikit-learn:\linebreak\url{https://tinyurl.com/scikitlearn-TF-IDF-vectorizer}. The closeness is computed by $(1 - s)$, where $s$ presents the cosine similarity of the vectors of two problem statements.\label{similarity}} to the target problem statement $p$ from the training set, together with human-written and LLM-generated essays associated with the retrieved problem statements.
To generate attacks to be considered in detection, we randomly select $j$ out of the $k$ problem statements ($0 \leq j \leq k$).
Consequently, our OUTFOX attacker adversarially generates an essay for each selected problem statement.
In this way, we obtain $k$ human-written, $(k-j)$ LLM-generated, and $j$ attacked essays and construct a mixture of these essays with labels $R_{\rm det}=\{(e_i, l_i)\}_{i=1}^{2k}$.
Here, $l_i$ is \texttt{Human} when the essay $e_i$ is written by a human, or \texttt{LM} when the essay $e_i$ is written by either the LLM or attacker.
Tagging attacker-generated texts with \texttt{LM} label encourages our detector to learn the implicit characteristics of attacker-generated texts for detection.
Finally, the detector predicts a label $\hat{l}$ for the target essay $e$ by using the LLM with the task instruction $I$ and $R_{\rm det}$ as a prompt,
\begin{equation}
\hat{l} = \underset{l \in \{ \text{\texttt{Human}}, \text{\texttt{LM}}\}}{\operatorname{argmax}} P\left(l \mid I, R_{\rm det}, e\right) ,
\end{equation}
\begin{equation}
R_{\rm det}=\{(e_i, l_i)\}_{i=1}^{2k} .
\end{equation}
Here, the task instruction $I$ is ``Please classify whether the text is generated by a human or a Language Model (LM).''
% An in-context examples $\{(h'_{i}, \texttt{Human}), \cdots, (a_{j}, \texttt{LM})\}$ are \texttt{Text:} $h'_{i}$ \texttt{Answer: Human}\textbackslash n\texttt{Text:} $m'_{i}$ \texttt{Answer: LM}...\textbackslash n\texttt{Text:} $a_{i}$ \texttt{Answer: LM}.
% An example of an essay with a correct label $(Essay, Label)$ is \texttt{Text:} $Essay$ \texttt{Answer:} $Label$.
Each example in $R_{\rm det}$ is converted into the text format, ``Text: $e_i$ Answer: $l_i$.''
The target essay $e$ is converted into ``Text: $e$. Answer: '' so that the LLM predicts a label as a continuation of the prompt.

% An example of attacked essay with a correct label $(a_{j}, l_{j}^{\rm m})$ is \texttt{Text:} $a_{j}$ \texttt{Answer: LM}.

\subsection{The OUTFOX Attacker}
\label{attacker}
Figure \ref{fig:attacker} illustrates the workflow of our OUTFOX attacker. Unlike the paraphrasing attack, our attacker generates an essay from a problem statement so that it fools our detector. Given a problem statement $p$, we first retrieve the top-$k$ problem statements $p_{i}\;(i \in \{1, \dots, k\})$ that are semantically close\footref{similarity} to the problem statement $p$ from the training set, along with human-written and LLM-generated essays $e_i$ associated with the retrieved problem statements.
Then, our detector predicts a label $\hat{l}_i \in \{\texttt{Human}, \texttt{LM}\}$ on each retrieved LLM-generated essay $e_i$ using in-context examples from the training set.
At this time, our detector does not consider an attack.\footnote{Although our framework theoretically allows the detector and attacker to iteratively strengthen each other many times, we focus on only once.}
Consequently, we create a mixture of the retrieved problem statements, the predicted labels, and the retrieved LLM-generated essays $R_{\rm atk}=\{(p_i, \hat{l}_i, e_i)\}_{i=1}^{k}$.
Finally, our attacker adversarially generates an essay $a$ as a continuation for the prompt rendered from $I_{\rm d}$, $R_{\rm atk}$, the task instruction of essay-generation $I$, and the given problem statement $p$. We use an LLM to predict the next token $\hat{a}_t$ for the previously predicted tokens $\hat{a}_{1:t-1}$,
\begin{equation}
\hat{a}_t \approx \underset{a_t \in V}{\operatorname{argmax}} P\left(a_t \mid I_{\rm d}, R_{\rm atk}, I, p, \hat{a}_{1:t-1}\right) ,
\end{equation}
\begin{equation}
R_{\rm atk}=\{(p_i, \hat{l}_i, e_i)\}_{i=1}^{k} .
\end{equation}
Here, $V$ is the vocabulary of the LLM.
The description $I_{\rm d}$ is ``Here are the results of detecting whether each essay from each problem statement is generated by a Human or a Language Model (LM)''. Each example in $R_{\rm atk}$ is converted to the text format, ``Problem Statement: $p_{i}$. Answer: $\hat{l}_{i}$. Essay: $m_{i}$.''
The task instruction $I$ is ``Given the following problem statement, please write an essay detected as Human in $N$ words with a clear opinion''.
$N$ is the number of words in the human-written essay paired with the given problem statement $p$.
The problem statement $p$ is converted into ``Problem Statement: $p$. Answer: Human. Essay: '' so that the LLM adversarially generates an essay to fool the detector as a continuation of the prompt, including explicit \texttt{Human} label.

\begin{table*}[t]
\centering
\small
\setlength{\tabcolsep}{6pt} % Default value: 6pt
\renewcommand{\arraystretch}{1.05} % Default value: 1
\begin{tabular}{cccccc}\hline
% \multicolumn{6}{c}{\textbf{The detection by our detector considering attacks on attacked essays}}\\
\multirow{2}{*}{\textbf{Attacker}}& \multirow{2}{*}{\textbf{Detector}} & \multicolumn{4}{c}{\textbf{Metrics (\%) ↑}}\\\cline{3-6}
& & \textbf{HumanRec} & \textbf{MachineRec} & \textbf{AvgRec} & \textbf{F1}\\\hline
\multirow{3}{*}{DIPPER}&\multicolumn{1}{l}{w/o Attacks} & 98.6 & 66.2 & 82.4 & 79.0\\
&\multicolumn{1}{l}{w/\hspace{2.5mm}DIPPER} & 98.2 & 79.6 & \textbf{88.9} & \textbf{87.8} \\
&\multicolumn{1}{l}{w/\hspace{2.5mm}OUTFOX} & 97.8 & 72.4 & 85.1 & 82.9\\
\hline
\multirow{3}{*}{OUTFOX}&\multicolumn{1}{l}{w/o Attacks} & 98.8 & 24.8 & 61.8 & 39.4\\
&\multicolumn{1}{l}{w/\hspace{2.5mm}DIPPER} & 98.6 & 20.8 & 59.7 & 34.0 \\
&\multicolumn{1}{l}{w/\hspace{2.5mm}OUTFOX} & 97.2 & 69.6 & \textbf{83.4} & \textbf{80.7}\\
\hline
\end{tabular}
\caption{Comparison of the detection performances of our OUTFOX detector on attacked essays, with and without considering attacks: our OUTFOX attack and the DIPPER attack. The DIPPER paraphrases ChatGPT-generated essays for attacking. In the rows of ``w/o Attacks'', we show the detection performances of our detector, without considering attacks, on attacked essays by each attacker.}
\label{robustness_of_our_detector}
\end{table*}

\section{Constructing a Dataset to Detect LLM-Generated Essays}
\label{dataset}
We build a dataset for student essay writing, specifically to detect LLM-generated essays. There are already some datasets for the purpose of automatically scoring student-written essays \citep{asap-aes,feedback-prize-effectiveness}, but few of them have abundant essay-problem statement pairs. To get LLM-generated essays, we need essay problem statements.
We focus on the essay dataset of \citet{feedback-prize-effectiveness}, consisting of argumentative essays written by native students from 6th to 12th grade in the U.S.
Firstly, we instruct ChatGPT to generate a pseudo-problem-statement to mimic a setting where a student would produce the supplied essay. Afterward, we instruct an instruction-tuned LLM to generate an essay based on each generated problem statement.
For each LLM, our dataset contains 15,400 triplets of essay problem statements, student-written essays, and LLM-generated essays. In our evaluation, we split the dataset into three parts: train/validation/test with 14400/500/500 examples, respectively.
Besides the non-attacked LLM-generated essays, to evaluate each attacker in our experiments, we also build 500 attacked essays associated with problem statements in our test set for each attacker.
% We provide details of instructions and the data statistics of our dataset in Appendix \ref{sec:appendix_a}.

\section{Experiments and Results}
In our experiments, we investigate the following aspects: 
\begin{itemize}
\item How robust is our detector, considering attacks, against attacked texts? 
\item Does our detector, considering attacks, consistently perform well even on non-attacked texts? 
\item Is our attacker stronger than the previous paraphrasing attack approach?
\end{itemize}

\subsection{Overall Setup}
\label{experimental_setup}
\paragraph{Essay Generation Models}
To generate non-attacked essays, we instruct the instruction-tuned LMs: ChatGPT (gpt-3.5-turbo-0301), GPT-3.5 (text-davinci-003), and FLAN-T5-XXL\footnote{\url{https://huggingface.co/google/flan-t5-xxl}}.
In each, we set the $\mathsf{temperature}$ parameter to 1.3.
% we don't focus on open-ended generation: feeding the first $n$ tokens of a human essay and completing the continuous tokens.

\paragraph{Evaluation Metrics and Dataset}
Area Under Receiver Operating Characteristic curve (AUROC) can be applied only to the detectors which output real number prediction scores.
Since our proposed detector outputs a binary label, we employ the F1-score on LLM-generated texts as our first metric.
Our second metric for detection performance is AvgRec, following \citet{li2023deepfake}.
AvgRec is the average of HumanRec and MachineRec.
In our evaluation, HumanRec is the recall for detecting Human-written texts, and MachineRec is the recall for detecting LLM-generated texts. 
We compute a classification threshold for each baseline detector on our validation set where the Youden Index\footnote{The Youden Index is the difference between True Positive Rate (TPR) and False Positive Rate (FPR). The cut-off point in the ROC curve, where the Youden Index is maximum, is the best trade-off between TPR and FPR.} \cite{1950youden} is maximum in the ROC curve.
Finally, we evaluate detectors with these metrics and thresholds on our test set: a mixture of 500 human-written and 500 non-attacked essays.
To evaluate detectors on attacked essays, we swap only LLM-generated essays from non-attacked to attacked ones. Additionally, the threshold for each detector is fixed on both non-attacked and attacked essays.

\begin{table*}[t]
\centering
\small
\setlength{\tabcolsep}{6pt} % Default value: 6pt
\renewcommand{\arraystretch}{1.05} % Default value: 1
\begin{tabular}{cccccc}\hline
% \multicolumn{6}{c}{\textbf{The detection by our detector considering attacks on non-attacked essays}}\\
\multirow{2}{*}{\textbf{Essay Generator}}& \multirow{2}{*}{\textbf{Detector}} & \multicolumn{4}{c}{\textbf{Metrics (\%) ↑}}\\\cline{3-6}
& & \textbf{HumanRec} & \textbf{MachineRec} & \textbf{AvgRec} & \textbf{F1}\\\hline
\multirow{3}{*}{ChatGPT}&\multicolumn{1}{l}{w/o Attacks} & 99.0 & 94.0 & \textbf{96.5} & \textbf{96.4}\\
&\multicolumn{1}{l}{w/\hspace{2.5mm}DIPPER} & 99.2 & 87.8 & 93.5 & 93.1 \\
&\multicolumn{1}{l}{w/\hspace{2.5mm}OUTFOX} & 97.8 & 92.4 & 95.1 & 95.0\\
\hline
\multirow{3}{*}{GPT-3.5}&\multicolumn{1}{l}{w/o Attacks} & 98.6 & 95.2 & \textbf{96.9} & 96.8\\
&\multicolumn{1}{l}{w/\hspace{2.5mm}DIPPER} & 98.8 & 92.4 & 95.6 & 95.5 \\
&\multicolumn{1}{l}{w/\hspace{2.5mm}OUTFOX} & 97.6 & 96.2 & \textbf{96.9} & \textbf{96.9}\\
\hline
\multirow{3}{*}{FLAN-T5-XXL}&\multicolumn{1}{l}{w/o Attacks} & 98.8 & 68.2 & 83.5 & 80.5\\
&\multicolumn{1}{l}{w/\hspace{2.5mm}DIPPER} & 99.2 & 72.0 & \textbf{85.6} & \textbf{83.3} \\
&\multicolumn{1}{l}{w/\hspace{2.5mm}OUTFOX} & 97.0 & 73.4 & 85.2 & 83.2\\
\hline
\end{tabular}
\caption{
Comparison of the detection performances of our OUTFOX detector on non-attacked essays, with and without considering the attacks: our OUTFOX attack and the DIPPER attack.
In the rows of ``w/o Attacks'', we show the detection performances of our detector, without considering attacks, on non-attacked essays.
}
\label{side_effect}
\end{table*}

\begin{table*}[t]
\centering
\small
\setlength{\tabcolsep}{6pt} % Default value: 6pt
\renewcommand{\arraystretch}{1.05} % Default value: 1
\begin{tabular}{cccccc}\hline
% \multicolumn{6}{c}{\textbf{The attacking by various attackers}}\\
\multirow{2}{*}{\textbf{Detector}}& \multirow{2}{*}{\textbf{Attacker}} & \multicolumn{4}{c}{\textbf{Metrics (\%) ↓}}\\\cline{3-6}
& & \textbf{HumanRec} & \textbf{MachineRec} & \textbf{AvgRec} & \textbf{F1}\\\hline
\multirow{3}{*}{RoBERTa-base}&\multicolumn{1}{l}{Non-attacked} & 93.8 & 92.2 & 93.0 & 92.9\\
&\multicolumn{1}{l}{DIPPER} & 93.8 & 89.2 & 91.5 & 91.3 \\
&\multicolumn{1}{l}{OUTFOX} & 93.8 & 69.2 & \textbf{81.5} & \textbf{78.9}\\
\hline
\multirow{3}{*}{RoBERTa-large}&\multicolumn{1}{l}{Non-attacked} & 91.6 & 90.0 & 90.8 & 90.7\\
&\multicolumn{1}{l}{DIPPER} & 91.6 & 97.0 & 94.3 & 94.4 \\
&\multicolumn{1}{l}{OUTFOX} & 91.6 & 56.2 & \textbf{73.9} & \textbf{68.3}\\
\hline
\multirow{3}{*}{HC3 detector}&\multicolumn{1}{l}{Non-attacked} & 79.2 & 70.6 & 74.9 & 73.8\\
&\multicolumn{1}{l}{DIPPER} & 79.2 & 3.4 & 41.3 & 5.5 \\
&\multicolumn{1}{l}{OUTFOX} & 79.2 & 0.4 & \textbf{39.8} & \textbf{0.7}\\
\hline
\multirow{3}{*}{OUTFOX}&\multicolumn{1}{l}{Non-attacked} & 99.0 & 94.0 & 96.5 & 96.4\\
&\multicolumn{1}{l}{DIPPER} & 98.6 & 66.2 & 82.4 & 79.0 \\
&\multicolumn{1}{l}{OUTFOX} & 98.8 & 24.8 & \textbf{61.8} & \textbf{39.4}\\
\hline
\end{tabular}
\caption{
Comparison of the detection performances of the detectors on ChatGPT-generated essays, before and after being attacked: our OUTFOX attack and the DIPPER attack. In the rows of ``Non-attacked'', we show the detection performances of each detector on non-attacked essays.
}
\label{our_stronger_attacker}
\end{table*}

\paragraph{Detection Methods}
Our OUTFOX detector is based on ChatGPT (gpt-3.5-turbo-0301). We set the $\mathsf{temperature}$ and $\mathsf{top\_p}$ parameters to 0 in order to eliminate the randomness of our detection.
Our detector takes retrieved in-context examples for an essay.
In detection without considering attacks, the in-context examples are 5 human-written and 5 LLM-generated essays.
In detection with considering attacks, the in-context examples are 5 human-written, 3 attacked, and 2 LLM-generated essays.
Here, regardless of the essay generation models to be detected, our detector takes ChatGPT-generated essays as part of in-context examples.
As a comparison with prior work, we compare our detector to the following detectors, divided into two groups: statistical outlier approaches and supervised classifiers.
The first group covers Rank, LogRank, Log Probability, and DetectGPT (as explained in \S\ref{baselines}).
The second group contains OpenAI's RoBERTa-based GPT-2 classifiers\footnote{\url{https://github.com/openai/gpt-2-output-dataset/tree/master/detector}} (Base, Large) and HC3 ChatGPT detector\footnote{\url{https://huggingface.co/Hello-SimpleAI/chatgpt-detector-roberta}}. HC3 is the latest corpus targeted for detecting ChatGPT-generated texts \cite{guo2023close}.
We employ default parameters for each detection method.\footnote{To adopt DetectGPT, we change only $\mathsf{buffer\_size}$ from default to 2 in order to escape being stuck in the perturbation step.}

\paragraph{Attacking Methods}
Our OUTFOX attacker is based on ChatGPT (gpt-3.5-turbo-0301).
In generating attacks, we configure the $\mathsf{temperature}$ parameter to 1.3.
Our attacker takes retrieved in-context examples for a problem statement. The number of in-context examples is 10.
We compare our attacker to the paraphrasing attack by the DIPPER (as explained \S\ref{attacking}). The DIPPER paraphrases the non-attacked essays for attacking.
We configure both parameters for vocabulary $L$ and content re-ordering $O$ to 60, which are the parameters found to produce the strongest attack in \citet{krishna2023paraphrasing}. Other hyperparameters are set to the defaults.

\label{results}
\subsection{Results}
\paragraph{How Robust Is Our Detector, considering Attacks, against Attacked Texts?}
Table \ref{robustness_of_our_detector} presents the difference in detection performance of our OUTFOX detector with and without considering attacks on attacked essays. The attacking models include our OUTFOX attacker and the DIPPER.
Throughout all attackers, our detector improves the detection performance when considering attacks.
For instance, on our attacker-generated essays, our detector shows +41.3 points F1-score and +21.6 points AvgRec improvements when considering our attack. 
From this result, we empirically observe that our detector learns to detect essays from attackers via in-context examples. 
Notably, our detector, when considering our attack, shows the performance improvements on any attacked essay, while our detector considering the DIPPER improves only on the attacked essays by the DIPPER, but not our attacker.
This observation suggests that our attacker is not merely paraphrasing but may generate semantically diverse essays from given problem statements. Consequently, our detector is enabled to identify texts that employ a broader spectrum of attacks.

\paragraph{Does Our Detector, considering Attacks, Consistently Perform Well Even on Non-attacked Texts?}
Table \ref{side_effect} shows the difference in the detection performance of our OUTFOX detector on non-attacked essays, with and without considering attacks.
% The generation models include ChatGPT, GPT-3.5, and FLAN-T5-XXL.
Our detector consistently performs well, even on non-attacked essays. The difference is minimal: an average decrease of only a -0.1 point F1-score and a -0.32 point AvgRec across all comparisons.
Furthermore, in non-attacked essays by FLAN-T5-XXL, our detector performs better than without considering attacks.
% : up to +2.8 points F1-score and +2.1 points AvgRec.
These results empirically show that considering attacks has little negative effects on the detection performance of our detector on the non-attacked texts.

\begin{table*}[t]
\centering
\small
\setlength{\tabcolsep}{6pt} % Default value: 6pt
\renewcommand{\arraystretch}{1.05} % Default value: 1
\begin{tabular}{ccccccc}\hline
% \multicolumn{7}{c}{\textbf{Comparison our detector and prior detectors on non-attacked essays}}\\
\multirow{2}{*}{\textbf{Baseline type}}&\multirow{2}{*}{\textbf{Essay Generator}}& \multirow{2}{*}{\textbf{Detector}} & \multicolumn{4}{c}{\textbf{Metrics (\%) ↑}}\\\cline{4-7}
& & & \textbf{HumanRec} & \textbf{MachineRec} & \textbf{AvgRec} & \textbf{F1}\\\hline
\multirow{6}{*}{Statistical outlier methods}&\multirow{6}{*}{FLAN-T5-XXL}&\multicolumn{1}{l}{log $p(x)$} & 2.0 & 97.6 & 49.8 & 66.0\\
& &\multicolumn{1}{l}{Rank} & 28.8 & 86.2 & 57.5 & 67.0 \\
& &\multicolumn{1}{l}{LogRank} & 12.0 & 90.6 & 51.3 & 65.0\\
& &\multicolumn{1}{l}{Entropy} & 39.4 & 80.4 & 59.9 & 66.7\\
& &\multicolumn{1}{l}{DetectGPT} & 29.8 & 76.2 & 53.0 & 61.9\\
& &\multicolumn{1}{l}{OUTFOX} & 97.0 & 73.4 & \textbf{85.2} & \textbf{83.2}\\\hline
\multirow{8}{*}{Supervised classifiers}&\multirow{4}{*}{ChatGPT}&\multicolumn{1}{l}{RoBERTa-base} & 93.8 & 92.2 & 93.0 & 92.9\\
& &\multicolumn{1}{l}{RoBERTa-large} & 91.6 & 90.0 & 90.8 & 90.7 \\
& &\multicolumn{1}{l}{HC3 detector} & 79.2 & 70.6 & 74.9 & 73.8 \\
& &\multicolumn{1}{l}{OUTFOX} & 97.8 & 92.4 &\textbf{95.1} & \textbf{95.0}\\
\cline{2-7}
& \multirow{4}{*}{GPT-3.5}&\multicolumn{1}{l}{RoBERTa-base} & 93.8 & 92.0 & 92.9 & 92.8\\
& &\multicolumn{1}{l}{RoBERTa-large} & 92.6 & 92.0 & 92.3 & 92.3 \\
& &\multicolumn{1}{l}{HC3 detector} & 79.2 & 85.0 & 82.1 & 82.6 \\
& &\multicolumn{1}{l}{OUTFOX} & 97.6 & 96.2 & \textbf{96.9} & \textbf{96.9}\\
\hline
\end{tabular}
\caption{
Comparison of the detection performances of our OUTFOX detector and prior approaches on non-attacked essays. Prior approaches include statistical outlier detectors and supervised classifiers. We compare our detector with statistical outlier detectors on the essays by FLAN-T5-XXL and supervised classifiers on the essays by ChatGPT and GPT-3.5.
}
\label{comparison_baseline}
\end{table*}

\paragraph{Is Our Attacker Stronger than the Previous Paraphrasing Attack Approach?}
Table \ref{our_stronger_attacker} provides the detection performance on attacked essays by different attacking approaches: our attack and the DIPPER attack. Here, our detector does not consider any attacks.
Our attacker drastically degrades the detection performance of our detector and supervised classifiers the most by up to -57.0 points F1-score and -69.2 points MachineRec. 
In addition, our attacker has a more detrimental impact on the detection performance of all detectors than the DIPPER by up to -39.6 points F1-score.
We also find that the DIPPER doesn't degrade the detection performance much and conversely improves the detection performance, especially on RoBERTa-large. This is partially because the DIPPER attack is based on paraphrasing and not designed specifically to attack detectors resulting in the opposite effect.

\section{Comparison with Prior Work}
\label{comparison}
In this section, we compare the detection performance of our OUTFOX detector, when considering our attack, and prior detectors on non-attacked essays.
The prior detectors are divided into two groups: statistical outlier approaches and supervised classifiers.
% Throughout all essay generation models, 

\paragraph{Statistical Outlier Approaches}
Statistical outlier approaches need access to model logits for their detection.
Thus, we contrast our detector with these statistical outlier approaches in detecting essays by FLAN-T5-XXL.
As shown in Table \ref{comparison_baseline}, our detector has a far superior detection performance of 80.5 points F1-score and 83.5 points AvgRec to previous statistical approaches. 
We find that statistical outlier approaches tend to aggressively label Human-written essays as LLM-generated from the contrast between the low HumanRec and the high MachineRec.
While the HumanRec of our detector is high: 98.8 points, partially because our detector is based on ChatGPT, which is trained with human feedback, thus avoiding aggressive detection.

\paragraph{Supervised Classifiers}
We compare our detector and supervised classifiers in detecting essays generated by each ChatGPT and GPT-3.5.
Table \ref{comparison_baseline} shows that our detector has better detection performance on both essays by ChatGPT and GPT-3.5 than supervised classifiers.
We empirically find that ChatGPT has the few-shot ability to detect LLM-generated texts via labeled in-context examples without any parameter updates.

In summary, our OUTFOX detector shows the state-of-the-art detection performance on non-attacked essays.

% Figure environment removed

\section{The Semantic Similarity of Our OUTFOX Attacks with Human-Written Essays}
\label{analysis}
To explore why our attacker substantially degrades the detectors' performance, we investigate the difference between the semantic similarity of non-attacked essays and our attacker-generated essays with human-written essays.
We focus on our test set, consisting of 500 human-written, 500 ChatGPT-generated, and 500 attacked essays generated by our attacker.
For determining semantic similarity between two essays, we employ a pre-trained BERT model\footnote{\url{https://huggingface.co/bert-large-cased}} to each essay and compute a cosine similarity of the resulting embeddings.
Figure \ref{fig:analysis} presents cosine similarity distributions of non-attacked and our attacker-generated essays with human-written essays.
A rightward shift implies that our attacker-generated essays are more semantically similar to human-written essays than non-attacked essays.
From the shift, we empirically find that our OUTFOX attacker can generate essays that are more semantically similar to human-written essays than the non-attacking normal LLM, leading to difficulty in detecting attacker-generated essays.
% We present examples of essays with cosine similarity scores in Appendix \ref{sec:appendix_b}.

\section{Conclusion}
We proposed OUTFOX, a framework that improves the robustness of the detector against attacks by allowing both the detector and the attacker to consider each other's outputs as examples for in-context learning.
The experiments in the domain of student essays demonstrate that 1) Our detector can learn to detect essays from attackers via in-context examples and 2) Notably, considering attacks of our detector has little negative effect on the detection of non-attacked texts. and 3) Our attacker, which is designed specifically to deceive the detector, can evade current LLM-generated text detectors more effectively than the previous paraphrasing attack. Furthermore, our analysis reveals that our attacker can generate an essay that is semantically closer to a human-written essay than a non-attacked essay, leading to success in effective attacking.
% We believe that our OUTFOX framework can be applied to other domains, such as fake news generation and academic paper writing.
In future work, we will apply our framework to other domains, such as fake news generation and academic paper writing.

% \newpage
% \section*{Limitations}
% aa
%% Limitation %%
% 1. 我々のOUTFOXフレームワークは、クエリとそのレスポンスのペアが人間と機械の文で一定数あることが前提で、事例に関係のないものを持ってきた(しか持ってこれない)場合、性能は一定低下する。
% 2. アタックを考慮した検出は、アタック→検出を全てchatgptに行わせるため、API使用コストが高い

% \section*{Ethics Statement}
% it's safe.




\section*{Acknowledgements}
These research results were obtained from the commissioned research (No.22501) by National Institute of Information and Communications Technology (NICT), Japan.

\bibliography{aaai24}
\clearpage
% \appendix

% \onecolumn
% \begin{center}
% \huge \textbf{Technical Appendix}
% \end{center}
% % Figure environment removed

% \begin{table*}[h]
% \centering
% \small
% \renewcommand{\arraystretch}{1.3}
% \begin{tabular}{ccccc}\hline
% \textbf{Data Source} &  & \textbf{Average Document Length} & \textbf{Average \#Sentence} & \textbf{Average Sentence Length}\\\hline
% Human-written&  & 440.3 & 18.6 & 23.7 \\
% \hline
% \multirow{3}{*}{LLM-generated}&\multicolumn{1}{l}{ChatGPT} & 463.7 & 15.8 & 29.4\\
% &\multicolumn{1}{l}{GPT-3.5} & 434.3 & 12.9 & 33.6\\
% &\multicolumn{1}{l}{FLAN-T5-XXL} & 354.8 & 14.8 & 24.0\\
% \hline
% \end{tabular}
% \caption{
% Data statistics on length for our essay dataset. \#Sentence is the number of sentences.}
% \label{data_statistics}
% \end{table*}

% \begin{multicols}{2}
% \section{The Details of Our Essay Dataset}
% \label{sec:appendix_a}
% Figure \ref{fig:dataset} illustrates the workflow of constructing our essay dataset, including instructions to generate pseudo-essay problem statements and essays by LLMs.
% The instruction for generating a pseudo-problem statement from a student-written essay $e$ is ``Please write the most appropriate homework problem statement that would have you write the following essay statement. Essay Statement: $e$ Homework Problem Statement:''.
% The instruction to generate an essay by LLMs from a problem statement $p$ is ``Given the following statement, please write an essay in $N$ words with a clear opinion. Problem statement: $p$ Essay:''. $N$ is the number of words in the human-written essay paired with the given problem statement $p$.

% For each LLM, our dataset contains 15,400 triplets of essay problem statements, student-written essays, and LLM-generated essays.
% We provide the data statistics on the length of our essay dataset in table \ref{data_statistics}.
% For text length, we count the number of words of a split text by whitespace. To calculate the number of sentences in a document, we split a document into sentences using the \texttt{nltk} library\footnote{\url{https://www.nltk.org/api/nltk.tokenize.sent_tokenize.html}}.

% \section{The Examples of Essays When Our Attacker Succeeds and Fails}
% \label{sec:appendix_b}
% In our analysis, we empirically found that our OUTFOX attacker can generate essays that are more semantically similar to human-written essays than non-attacking LLM, leading to success in effective attacking.
% In this appendix, we qualitatively show the findings by presenting actual generated essays.
% Table \ref{unsuccessful-attack} provides examples of a human-written, non-attacked, and \textit{unsuccessfully attacked} essay when our OUTFOX attacker fails to deceive our detector. 
% Table \ref{successful-attack} provides examples of a human-written, non-attacked, and \textit{successfully attacked} essay when our attacker successfully deceives our detector. 
% Here, our detector does not consider any attacks.
% The non-attacked essays are generated by ChatGPT, and the attacked essays are adversarially generated by our OUTFOX attacker.

% In the example of the unsuccessfully attacked essay, the human-written essay focuses on how people should act before using an emotion-reading technology, while the non-attacked and attacked essays describe the consequences after using the technology.
% On the other hand, in the example of the successfully attacked essay, the human-written and the attacked essays mainly focus on the harsh environmental condition on Venus that would prevent NASA exploration, while the non-attacked essay primarily focuses on the high cost of the exploration.
% Additionally, the cosine similarity of the human-written essay is higher with the successfully attacked essay than with the unsuccessfully attacked essay.
% In this way, these examples suggest that a successfully attacked essay should consist of similar contents that a human-written essay has, and our attacker learns its topic selection from in-context examples.
% \end{multicols}

% \begin{table*}[t]
% \centering
% \small
% \begin{tabular}{lp{.78\textwidth}}\hline
% % \textbf{Data Source}& \textbf{Cosine Similarity Score} & \textbf{Content}\\\hline
% Problem Statement & Discuss the potential negative effects of using technology to alter or read emotional expressions in students, and argue for the importance of accepting and supporting individuals for who they are rather than trying to change them.\\\hline
% Human-written essay & In my ophion, technology reading students emotional expressions is the dumbest thing ever First, the apps they are using just changes the student indivaully like you can not make someone be happy on a computer you can make them look happy they should not change someone for who they want them to be vs what they are. Second, you should not change a photo or use any edits because thats changing you and you should just be you if you dont smile thats okay dont let people judge you on how you look you should impress your self not somebody eles. Third, People today need to realize that changing someone in any type of way could hurt them even worse. Yes, messing with peoples emotion or sometimes trying to figure it out whats wrong with them can set off alot of things. Instead of making someone look happy people should talk to the person that is upset, trying to be there for them maybe thats all they need or tell them soemthing postive to brighten there day. In Conclusion, I think changing someones pictures to make them look like they are happy is the wose idea ever because deep down they know they are, but do you just because someone makes and upset face dosnt always mean they are.\\\hline
% Non-attacked essay & In today's society, technology plays a major role in our lives. Arriving at school, students are inundated with gadgets and apps that promise to help in studies and socializing. While the use of these technologies is necessary, there have been recent concerns about using technology to alter or read emotional expressions in students, particularly by schools trying to remedy emotional problems objectively. Such a possibility could have detrimental effects on students' emotional state, as it would compromise their genuine emotions.
% Using technology to control or read emotional displays poses a significant risk to students' well-being. If students feel closely scrutinized for their feelings, they become self-conscious and perhaps anxious, potentially tampering with any chance to feel safe, secure and respected. Therefore the need to help students feel attuned, understood, in an environment that treats everyone with empathy is vital. Nonetheless, we may be suppressing one's vision for suitably changing oneself to appreciate where one is situated but not completely uproot oneself.
% Accepting and supporting individuals for who they are is therefore critical in achieving fulfilling lives and building cohesive communities. We will operate confidently and motivatedly toward achieving an assimilated and encouraging ecology – better place, shall all appreciate who we are and anticipate the responsibility we have towards ourselves and our fellow human beings rather than prioritizing changing them somehow. Only once recognizing one's sense of integrity can a self-defined education ecosystem be described, backed by a truly situated educational outcome. Indisputably, it remains my opinion that harmon\\
% A cosine similarity & 0.782\\\hline
% Attacked essay & In today's digital age, technology has become a significant influence in many aspects of our lives, including education. The use of technology to alter or read emotional expressions in students may seem enticing, but it can have negative effects that we need to carefully consider.
% First and foremost, altering or manipulating students' emotional expressions through technology can undermine their individuality and authenticity. Each individual has a unique set of emotions and expressions that make them who they are. By trying to change or alter these expressions, we run the risk of suppressing their true selves and promoting conformity.
% Moreover, constant monitoring and manipulation of emotional expressions can create a sense of unease and a lack of privacy for students. It is important to recognize that emotions are personal and should be respected as such. By intruding on students' emotional lives, we may unintentionally create a hostile classroom environment and hinder their emotional well-being.
% Instead of trying to change or manipulate students' emotional expressions, it is crucial to accept and support them for who they are. By providing a safe and accepting space, we can allow students to express their emotions freely without judgment or manipulation. This promotes a sense of belonging and self-acceptance, which are essential for emotional well-being and personal growth.
% In conclusion, while technology offers many possibilities in the field of education, we must be cautious when it comes to altering or reading students' emotional expressions. It is important to prioritize acceptance and support for individuals, allowing them to express their emotions authentically and\\
% A cosine similarity & 0.831\\\hline
% \end{tabular}
% \caption{
% An example of a human-written, non-attacked, and unsuccessfully attacked essay and their cosine similarity with the human-written essay. The non-attacked essay is generated by ChatGPT, and the attacked essay is adversarially generated by our OUTFOX attacker. 
% }
% \label{unsuccessful-attack}
% \end{table*}

% \begin{table*}[t]
% \centering
% \small
% \begin{tabular}{lp{.78\textwidth}}
% \hline
% % \textbf{Data Source}& \textbf{Cosine Similarity Score} & \textbf{Content}\\\hline
% Problem Statement & Discuss the challenges and potential risks associated with NASA's plan to explore Venus using a blimp-like vehicle, and argue whether or not it is worth the investment of time, money, and human lives.\\\hline
% Human-written essay & Scientists have been studying Venus for generations, wondering what could possibly be on a planet so close to ours. Now that we don't have to wonder anymore, NASA will try to take it one step higher. They want to be as close to Venus' ground as possible, without dying.
% Their plans are to use a blimp-like vehicle that will withstand the great amount of pressure and heat. It will fly thirty miles above the Venusian landscape, avoiding any ground conditions that would do them harm. It would still be 170 degrees F., but they would have an abundance of solar energy, and the radiation woud not exceed Earth levels. However, in order to learn all there is about Venus, you would have to go onto Venus. This is almost an impossible task though. There are many dangers on Venus, including sulfuric acid rain and an atmosphere of nintey seven percent carbon dioxide. Hovering over it would be useless. In paragraph six, it states, "... peering at Venus from a ship orbiting or hovering safely far above the planet can provide only limited insight on ground conditions because most forms of light cannot penetrate the dense atmosphere, rendering standard forms of photography and videography ineffective." This states that by hovering on top of the planet, they can't learn anyting they don't already know. My honest oppinion on this topic is that we should not be exploring Venus right now. We just don't have the right technology for it right now. We need to wait however many years before we can officially explore our sister planet. Trying to figure out its secrets now would just be a waste of time, money, and possibly lives. We shouldn't risk a human being just because this planet used to be like ours millions of years ago. \\\hline
% Non-attacked essay & Exploring Venus using a blimp-like vehicle poses several challenges and risks for NASA. One significant risk that NASA must consider before pushing through with the expedition is the harsh and almost uninhabitable environmental conditions of Venus. Venus' atmosphere contains corrosive sulfuric acid, and its surface temperature reaches up to 800 degrees Fahrenheit, which could potentially cause damage not just to the blimp but also to humans on the expedition. There are also uncertainties regarding the blimp's durability and the possibility of equipment failure fueled by a situation as harsh as Venus that would render life support critical. The longevity of the expedition activity also factors, following in-depth analysis with the burdensome costs involved to develop the kinds of balloons being considered.
% Although exploring Venus' atmosphere could undoubtedly expand our scientific knowledge of the universe, the dangers of putting lives in danger complicate matters significantly. The level of uncertainty that pertains solely to Venus demands technology specially developed toysurf the perilous area. Providing worthwhile resources to NASA's effort currently poses economic trilemmas at a global level while diverting scientific attention from other unfathomable terrestrial facts requiring exploration.
% Ultimately, taking back the mission proves plausible for scientific endeavors, what being implemented should be systemically chosen and approached with insightful technical accuracy and faultless scrutiny. The worth of investing valuable time and economic resources for the primary purpose aside, there are a battle and security concerns that should not be overtly displaced in difficult space exploration. In particular stages seemingly behind the wave to place humans on Mars, it begs companies that searching on such challenging spatial spheres open critical portals rather than high reward regions that lack technical accessibility. Must I realistically ponder if a less high-risk alternative wouldn't be more expedient?
% In conclusion, while it may sound like a worthwhile mission and obviously offer advancements on our scientific access to other celestial bodies in our system, reasonable pragmatic assessment on human resource intensiveness seems limited at the apex. The extreme challenges exceed the excellence one would high eventually meet—a high volume of attendees primed to consume futurist terra\\
% A cosine similarity & 0.841\\\hline
% Attacked essay & NASA's plan to explore Venus using a blimp-like vehicle raises several challenges and potential risks that need to be carefully considered. It is important to evaluate whether the investment of time, money, and human lives is worth these potential drawbacks.
% Firstly, the extreme conditions on Venus pose significant challenges to any exploration mission. The planet's surface temperature is hot enough to melt lead, and its thick atmosphere is composed mainly of carbon dioxide, creating a hostile environment for human and mechanical exploration. Navigating and finding protective materials for these extreme temperatures will require extensive research and development. Furthermore, the deep clouds covering Venus can impede visibility and affect navigation systems, making it difficult to maneuver the blimp-like vehicle.
% The risks associated with sending humans to explore Venus cannot be overlooked. The health and safety of astronauts should be a top priority. Exposure to the high temperatures and pressure on Venus can have severe consequences for human life. The health risks involved in such an endeavor need to be carefully evaluated and mitigated to ensure the well-being of those involved.
% However, despite these challenges and risks, exploring Venus has the potential to greatly expand our scientific knowledge and understanding of the universe. Venus is often referred to as Earth's twin planet due to its similar size and location in the solar system. By exploring Venus, scientists can gain valuable insights into the processes that shape rocky planets like ours. Studying Venus can also provide crucial information about the potential habitability of exoplanets orbiting distant stars.
% In conclusion, the exploration of Venus using a blimp-like vehicle comes with substantial challenges and potential risks. The investment of time, money, and human lives must be carefully evaluated. While the risks are significant, the scientific discoveries and advancements that can be made from studying Venus may outweigh these drawbacks. With proper research, planning, and mitigation strategies, it is possible to conduct a successful exploration mission to Venus and further our understanding of the universe.\\
% A cosine similarity & 0.902\\\hline
% \end{tabular}
% \caption{
% An Example of a human-written, non-attacked, and successfully attacked essay and their cosine similarity with the human-written essay. The non-attacked essay is generated by ChatGPT, and the attacked essay is adversarially generated by our OUTFOX attacker.
% }
% \label{successful-attack}
% \end{table*}

\end{document}