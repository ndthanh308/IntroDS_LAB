\documentclass[10pt,twocolumn,letterpaper]{article}

\usepackage{iccv}
\usepackage{times}
\usepackage{epsfig}
\usepackage{graphicx}
\usepackage{amsmath}
\usepackage{amssymb}
\usepackage{algorithm}
\usepackage{algorithmic}
\usepackage{multirow}
\usepackage{array}
\usepackage[nocompress]{cite}
\usepackage{subcaption}
\usepackage{arydshln} 
\usepackage{xr}
\usepackage{float}
% \usepackage{pdfpages}
% Include other packages here, before hyperref.

% If you comment hyperref and then uncomment it, you should delete
% egpaper.aux before re-running latex.  (Or just hit 'q' on the first latex
% run, let it finish, and you should be clear).
\usepackage[pagebackref=true,breaklinks=true,letterpaper=true,colorlinks,bookmarks=false]{hyperref}
\usepackage[breaklinks=true,bookmarks=false]{hyperref}

\newcommand{\todo}[1]{{\color{red}{\bf todo:} #1}}
\newcommand{\JW}[1]{{\color{magenta}{\bf JW:} #1}}


\iccvfinalcopy % *** Uncomment this line for the final submission

\def\iccvPaperID{3201} % *** Enter the ICCV Paper ID here
\def\httilde{\mbox{\tt\raisebox{-.5ex}{\symbol{126}}}}

\usepackage[capitalize]{cleveref}
\creflabelformat{equation}{#2#1#3}

% Pages are numbered in submission mode, and unnumbered in camera-ready
\ificcvfinal\pagestyle{empty}\fi

\begin{document}

%%%%%%%%% TITLE
\title{Augmented Box Replay: Overcoming Foreground Shift \\ for Incremental Object Detection}
\author{Yuyang Liu\textsuperscript{1,2,3} \enspace Yang Cong\textsuperscript{4} \enspace Dipam Goswami\textsuperscript{5} \enspace Xialei Liu\textsuperscript{6} \enspace Joost van de Weijer\textsuperscript{5,7} \and\vspace{-13pt}\\ 
\textsuperscript{1}State Key Laboratory of Robotics, Shenyang Institute of Automation, Chinese Academy of Sciences \\
\textsuperscript{2} Institutes for Robotics and Intelligent Manufacturing, Chinese Academy of Sciences \\ 
\textsuperscript{3}University of Chinese Academy of Sciences 
\space\space \space\space 
\textsuperscript{4}South China University of Technology\\ 
\textsuperscript{5}Computer Vision Center, Barcelona \space\space\space\space
\textsuperscript{6}VCIP, CS, Nankai University\\
\textsuperscript{7}Department of Computer Science, Universitat Autònoma de Barcelona  \\ 
{\tt\small liuyuyang@sia.cn, congyang81@gmail.com \{dgoswami, joost\}@cvc.uab.es, xialei@nankai.edu.cn}
}

\maketitle
% Remove page # from the first page of camera-ready.
\ificcvfinal\thispagestyle{empty}\fi

%%%%%%%%% ABSTRACT
\begin{abstract}
In incremental learning, replaying stored samples from previous tasks together with current task samples is one of the most efficient approaches to address catastrophic forgetting. However, unlike incremental classification, image replay has not been successfully applied to incremental object detection (IOD). In this paper, we identify the overlooked problem of foreground shift as the main reason for this. Foreground shift only occurs when replaying images of previous tasks and refers to the fact that their background might contain foreground objects of the current task. To overcome this problem, a novel and efficient Augmented Box Replay (ABR) method is developed that only stores and replays foreground objects and thereby circumvents the foreground shift problem. In addition, we propose an innovative Attentive RoI Distillation loss that uses spatial attention from region-of-interest (RoI) features to constrain current model to focus on the most important information from old model. ABR significantly reduces forgetting of previous classes while maintaining high plasticity in current classes. Moreover, it considerably reduces the storage requirements when compared to standard image replay. Comprehensive experiments on Pascal-VOC and COCO datasets support the state-of-the-art performance of our model~\footnote{Code is available at \href{https://github.com/YuyangSunshine/ABR_IOD.git}{https://github.com/YuyangSunshine/ABR\_IOD.git}}.
\end{abstract}


%%%%%%%%% INTRO
The problem of the presence or absence of phase transition is central in statistical mechanics. To prove the existence of phase transition, the standard idea is to define a notion of contour and use \textit{Peierls' argument} \cite{Peierls.1936}. In the usual Ising model \cite{Ising_25}, particles of the system interact only with their nearest-neighbors. On ferromagnetic long-range Ising models \cite{Anderson_Yuval_69}, there is interaction between each pair of spins in the lattice. The Hamiltonian of the model is given formally by
\begin{equation*}
    H(\sigma) = - \sum_{x,y\in \Z^d}J_{xy}\sigma_x\sigma_y,
\end{equation*}
where $J_{xy}=J|x-y|^{-\alpha}$, $J>0$, $\alpha > d$. It is well-known that the Peierls' argument in dimension 2 implies phase transition for Ising models with nearest-neighbors or long-range interactions when $d\geq 2$, using correlation inequalities. For the unidimensional lattice, it was known that short-range models do not present phase transition. In the long-range case, a different behavior was expected depending on the exponent $\alpha$ (see \cite{Kac_Thompson_69}), but the problem was challenging since contours were first created as multidimensional objects.

In dimension $d=1$, phase transition was proved first in 1969 by Dyson \cite{Dyson.69}, for $\alpha \in (1,2)$, by proving phase transition in an auxiliary model and then using correlation inequalities. In 1982, Fr{\"o}hlich and Spencer \cite{Frohlich.Spencer.82} introduced a notion of one-dimensional contours and then applied the Peierls' argument to show phase transition for the critical value $\alpha = 2$. These contours were inspired by the multiscale techniques previously introduced to study the Berezinskii-Kosterlitz-Thouless transition in two-dimensional continuous spin systems \cite{FS81}. Later, Cassandro, Ferrari, Merola and Presutti  \cite{Cassandro.05} extended the contour argument previously available for $\alpha=2$ to exponents $\alpha\in (3-\frac{\ln 3}{\ln 2}, 2)$, with the additional restriction that the nearest-neighbor interaction is strong, i.e.,  ${J(1)\gg 1}$; this restriction was removed for a subclass of interactions in \cite{Bissacot.Endo.18}. Further results were obtained using contour arguments, such as the decay of correlations, cluster expansions, phase transition with random interactions, etc; some references with these results are \cite{ Cassandro.Merola.Picco.17, Cassandro.Merola.Picco.Rozikov.14, Imbrie.82, Imbrie.Newman.88, Johansson.91}. 

In the multidimensional setting ($d\geq 2$), Ginibre, Grossmann, and Ruelle, in \cite{Ginibre.Grossmann.Ruelle.66}, proved the phase transition for $\alpha > d+1$, using an enhanced version of Peierls' argument and the usual contours. Park proposed a different notion of contour for long-range systems in \cite{Park.88.I, Park.88.II}, extending the Pirogov-Sinai theory available for short-range interactions assuming $\alpha > 3d+1$, although he can also consider Potts models with his methods. Some results in the literature suggest that truly long-range effects appear only when $d < \alpha \leq d+1$, see for instance, \cite{Biskup_Chayes_Kivelson_07}. Recently, Affonso, Bissacot, Endo and Handa \cite{Affonso.2021}, inspired by the ideas from Fr{\"o}hlich and Spencer in \cite{FS81, Frohlich.Spencer.82}, introduced a version of multiscale multidimensional contour and proved phase transition by a contour argument in the whole region $\alpha > d$. They can consider long-range Ising models with deterministic decaying fields, first introduced in the context of nearest-neighbor interactions in \cite{Bissacot_Cioletti_10}. For these models, the lack of analyticity of the free energy does not imply phase transition since these models have the same free energy as the models with zero field. It is expected that fields decaying slowly imply uniqueness. In this setting, a contour argument is useful for proofs of phase transitions as well for uniqueness, some papers with models with deterministic decaying fields are \cite{Aoun_Ott_Velenik_23, Bissacot_Cass_Cio_Pres_15, Bissacot.Endo.18, Cioletti_Vila_2016}.

The Random Field Ising model (RFIM) \cite{Imry.Ma.75} is the nearest-neighbor Ising model with an additional external field acting on each site $(h_x)_{x\in\Z^d}$ that is a family of i.i.d. Gaussian random variable with mean 0 and variance 1. Formally, the Hamiltonian of the model is given by
\begin{equation*}
    H(\sigma) = - \sum_{\substack{x,y\in \Z^d \\|x-y|=1}}J\sigma_x\sigma_y  - \varepsilon\sum_{x\in\Z^d}h_x\sigma_x,
\end{equation*}
where $J>0$, $\varepsilon>0$, $\alpha > d$ and $d \geq 1$. A detailed account of the history of the phase transition problem for this model, as well as detailed proofs, was given in \cite{Bovier.06}. Here we present a brief overview.

During the 1980s, the question of the specific dimension where phase transition for the RFIM should happen attracted much attention and was a topic of heated debate. Two convincing arguments were dividing the physics community. One of them, due to Imry and Ma \cite{Imry.Ma.75}, was a non-rigorous application of the Peierls' argument together with the use of the isoperimetric inequality. The key idea of Peierls' argument is to define a notion of contour and calculate the energy cost of "erasing" each contour, i.e., the energy cost of flipping all spins inside the contour. When there is no external field, that energy necessary to flip the spins in a region $A\subset \Z^d$ is of the order of the boundary $|\partial A|$. When we add an external field, we get an extra cost depending on this field. Imry and Ma argued that this cost should be approximately $\sqrt{|A|}$, which is smaller than $|\partial A|$ for all regions only when $d\geq 3$, so this should be the region where phase transition occurs. The other argument, due to Parisi and Sourlas \cite{Parisi.Sourlas.79}, based on dimensional reduction, predicted that the $d$-dimensional RFIM would behave like the $d-2$-dimensional nearest-neighbor Ising model, therefore presenting phase transition only when $d\geq 4$. 

The question was settled by two celebrated papers showing that Imry and Ma's prediction was correct. First, in 1988, Bricmont and Kupiainen \cite{Bricmont.Kupiainen.88} showed that there is phase transition almost surely in $d\geq3$, for low temperatures and variance $\varepsilon$ small enough. Their proof uses a rigorous renormalization group analysis for the short-range case and it is considered involved. Still, they claimed that the result works for any model with a suitable contour representation and centered sub-gaussian external field. Later on, Aizenman and Wehr \cite{Aizenman.Wehr.90} proved uniqueness for $d\leq 2$. For detailed proofs of these results, we refer the reader to \cite{Bovier.06} (see also \cite{Berretti.85, Camia.18, Frohlich.Imbre.84,  Klein.Masooman.97} for more uniqueness results). 

Recently, Ding and Zhuang, see \cite{Ding2021}, provided a simpler proof of the phase transition, not using RGM. And in  \cite{Ding.Liu.Xia.22}, Ding, Liu and Xia proved that if $\beta_c(d)$ is the critical inverse of the temperature of the Ising model with no field, for all $\beta>\beta_c(d)$ there exists a critical value $\varepsilon_0(d, \beta)$ such that the RFIM with $\varepsilon \leq \varepsilon_0$ presents phase transition. 

In the present paper, we are considering a long-range Ising model with a random field, whose Hamiltonian is given formally by
\begin{equation*}
    H(\sigma) = - \sum_{x,y\in \Z^d}J_{xy}\sigma_x\sigma_y - \varepsilon\sum_{x\in\Z^d}h_x\sigma_x,
\end{equation*}
where $J_{xy}=J|x-y|^{-\alpha}$, $J, \varepsilon>0$, $\alpha > d$ and $h_x\in\mathbb{R}$, $d\geq 3$.
Until now, the only known result in the long-range setting is for the one-dimensional long-range Ising model with a random field, by Cassandro, Orlandi, and Picco \cite{Cassandro.Picco.09}. They used the contours of \cite{Cassandro.05} to show the phase transition for the model when $\alpha\in (3-\frac{\ln 3}{\ln 2}, \frac{3}{2})$, under the assumption $J(1) \gg 1$. We stress that, as remarked by Aizenman, Greenblatt, and Lebowitz \cite{Aizenman_Greenblatt_Lebowitz_2012}, although their argument does not work for the whole region for the exponent $\alpha$, the phase transition holds for values close to the critical value $\alpha=3/2$, since by the Aizenman-Wehr theorem we know that there is uniqueness for $\alpha>3/2$.

The argument from Ding and Zhuang in \cite{Ding2021}, for $d\geq3$, involves controlling the probability of a bad event, which is closely related to controlling the quantity $$\sup_{\substack{0\in A\subset\Z^d \\ A \text{ connected }}}\frac{\sum_{x\in A}h_x}{|\partial A|},$$ known as the greedy animal lattice normalized by the boundary. The greedy animal lattice normalized by the size, instead of the boundary, was extensively studied for general distributions of $(h_x)_{x\in\Z^d}$, see \cite{Cox_Gandolfi_Griffin_Kesten_93, Gandolfi_Kesten_94, Hammond_06, Martin_02}. When we normalize by the boundary, an argument by Fisher, Fr\"{o}hlich and Spencer \cite{FFS84} shows that the expected value of the greedy animal lattice is constant. In dimension $d=2$, the expected value is not finite, see \cite{Ding.Wirth.20}. The supremum is taken over connected regions containing the origin since the interiors of the usual Peierls contours are of this form.


For the long-range model, the interior of contours is not necessarily connected. In fact, long-range contours may have considerably large diameters with respect to their size, so their interiors can be very sparse. To avoid this, we define contours, strongly inspired by the $(M,a,r)$-partition in \cite{Affonso.2021}, using a multiscaled procedure that assures that the contours have no cluster with small density.  With them, we generalize the arguments by Fisher-Fr\"{o}hlich-Spencer \cite{FFS84}, and prove that the expected value of the greedy animal lattice is constant, even considering regions not necessarily connected in the supremum. Then, we prove the phase transition for $d\geq 3$. The main result of this paper is the following.
\begin{theorem*}Given $d\geq 3$, $\alpha>d$, there exists $\beta_c\coloneqq\beta(d, \alpha)$ and $\varepsilon_c\coloneqq\varepsilon(d, \alpha)$ such that, for $\beta >\beta_c$ and $\varepsilon\leq \varepsilon_c$, the extremal Gibbs measures $\mu_{\beta, \varepsilon}^+$ and $\mu_{\beta, \varepsilon}^-$ are distinct, that is, $\mu_{\beta, \varepsilon}^+ \neq \mu_{\beta, \varepsilon}^-$ $\mathbb{P}$-almost surely. Therefore the long-range random field Ising model presents phase transition.
\end{theorem*}

This paper is divided as follows. In Section 2, we define the model and the contours, and suitable generalizations to the constructions in \cite{Affonso.2021} are introduced.  In Section 3, we define two bad events of the external field and prove that they occur with a small probability.  In Section 4, we present the proof of the phase transition.

%%%%%%%%% RELATED WORK
\section{Related work}
% The prediction of epitopes and paratopes, the binding sites on the antigen and antibody, respectively, is a fundamental problem in protein-protein interaction. The prediction of one strongly influences the prediction of the other. Therefore, many methods have been proposed for predicting both epitopes and paratopes simultaneously \cite{del2021neural, PiNet}.

The structure of proteins provides crucial information about the location and orientation of the binding sites. Various approaches have been taken in the literature to address the task of epitope and paratope prediction, including sequential \cite{liberis2018parapred,deac2019fastparapred} and structural \cite{krawczyk2014improving,del2021neural} methods. 
Furthermore, Geometric deep learning has emerged as a powerful tool for predicting protein-protein interactions \cite{isert2023structure}, with graph-based representations being one of the most common approaches \cite{tubiana2022scannet,stark2022equibind}. These methods leverage the geometric information of the molecules to learn complex relationships between epitopes and paratopes. For instance, some approaches \cite{del2021neural,da2022epitope3d} use the graph structure to compute features based on neighbouring residues, which are then aggregated to highlight the most probable region of interaction.

An alternative approach is to represent proteins as surfaces. % which is an effective way to capture the geometric properties of the epitope and paratope.
MaSIF \cite{gainza2020MaSif} focuses on the more general problem of protein interaction region prediction and uses a surface representation learned through convolutions defined on the surface.
PiNet \cite{PiNet} represents the protein surface as a point cloud and employs PointNet \cite{qi2017pointnet} to classify points as interacting or not. On the contrary, \citet{zhang2023equipocket} model the surface of a molecule as a graph and apply an equivariant graph neural network (EGNN, \cite{satorras2021n}) for binding site prediction. 

Integrating structural and geometric information has proven to be a promising approach for improving protein interaction prediction. Still, few studies have focused on the specific case of epitope and paratope prediction \cite{cia2023critical}. Our work supports this view by showing that considering the problem as a geometric one can effectively improve performance.


%%%%%%%%% METHOD
While it is not tractable to directly compute the integral of a function represented by a neural network, it is straightforward to take the analytical derivative. In this paper, we leverage the fundamental theorem of calculus in order to implicitly learn the integral of a function.

Suppose we wish to learn some function $f: \mathbb{R}^n \mapsto \mathbb{R}$. Instead of directly parametrising $f$, we represent it implicitly by parametrising its indefinite integral $F_\theta$ with a neural network:

\begin{equation}
    F_\theta(\vec{x}) = \int \int \cdots \int f(\vec{x}) \; dx_1 dx_2 \ldots dx_n
\end{equation}

Note that as $f$ is defined implicitly as a function of $F_\theta$, this is not an approximation---it is the \textit{exact} analytical integral. In order to solve for $f$, we must differentiate $F_\theta$:

\begin{equation}
    f(\vec{x}) = \frac{\partial}{\partial x_1} \frac{\partial}{\partial x_2} \cdots \frac{\partial}{\partial x_n} F_\theta(\vec{x})
    \label{eq:f_F}
\end{equation}

Although we parametrise its integral $F_\theta$, the function we wish to learn is $f$. Consequently, during the learning process, the loss is applied directly to $f$:

\begin{equation}
    \mathcal{L} = \mathbb{E}\left[ (y - f(\vec{x}))^2 \right]
\end{equation}

\input{schema/eps-fig}

\subsection{Integral Constraints}

Since $f$ is defined as a function of $F_\theta$, we can apply constraints directly to its integral. For example, by using an equality constraint, we can define the class of functions $f$ that integrate to a given value $\epsilon$ over a given domain $\mathcal{D}$. The same principle can be utilised to apply inequality constraints or transformations.

We start by considering rectangular domains defined by intervals $[a_i,b_i]$ for $i \in [1..n]$. The definite integral of $f$ over rectangular domain $\mathcal{D}$ is defined:

\begin{equation}
    F_\theta \Big\vert_\mathcal{D} = \sum_{p_1 \in [a_1, b_1]} \sum_{p_2 \in [a_2,b_2]} \cdots \sum_{p_n \in [a_n,b_n]} (-1)^{^{\sum \mathbbm{1}(p_i=a_i)}} \cdot F_\theta(\langle p_1, p_2, \ldots, p_n \rangle)
    \label{eq:int_eval}
\end{equation}

That is, in order to calculate the definite integral over an $n$-dimensional box, we must evaluate all of its vertices. This is because evaluating $F_\theta$ at a point will yield the ``area'' from negative infinity up to that point, so multiple points must be evaluated to determine the area of a finite region. The $(-1)$ exponent in the equation determines which regions must be subtracted and which must be added in order to determine that area (Figure \ref{fig:int_eval}).

% Figure environment removed

In order to parametrise the class of functions which integrate to $\epsilon$, we start by defining $F'_\theta$, the integral of some unconstrained function $f'$. Then, we define the integral $F_\theta$ of our constrained function $f$ by rescaling $F'_\theta$:

\begin{equation}
    F_\theta(\vec{x}) = \frac{\epsilon}{F'_\theta \big\vert_\mathcal{D}} F'_\theta(\vec{x})
\end{equation}

Since the term $\frac{\epsilon}{F'_\theta \big\vert_\mathcal{D}}$ is a scalar, we can move it inside the integral:

\begin{align}
    F_\theta(\vec{x}) &= \frac{\epsilon}{F'_\theta \big\vert_\mathcal{D}} F'_\theta(\vec{x}) \\
    F_\theta(\vec{x}) &= \frac{\epsilon}{F'_\theta \big\vert_\mathcal{D}} \int \int \cdots \int f'(\vec{x}) \; dx_1 dx_2 \ldots dx_n \\
    F_\theta(\vec{x}) &= \int \int \cdots \int \frac{\epsilon}{F'_\theta \big\vert_\mathcal{D}} f'(\vec{x}) \; dx_1 dx_2 \ldots dx_n \\
\end{align}

Therefore, we can write the constrained $f$ as a function of the unconstrained integral $F'_\theta$, which carries the learnable parameters:

\begin{align}
    f(\vec{x}) &= \frac{\epsilon}{F'_\theta \big\vert_\mathcal{D}} f'(\vec{x}) \\
    f(\vec{x}) &= \frac{\epsilon}{F'_\theta \big\vert_\mathcal{D}} \cdot \frac{\partial}{\partial x_1} \frac{\partial}{\partial x_2} \cdots \frac{\partial}{\partial x_n} F'_\theta(\vec{x})
\end{align}


% \subsection{Integration Over Arbitrary Domains}

% In the previous section we focused on the special case of rectangular domains (\textit{i.e.} where the limits of integration are constants). However, it is also possible to integrate over arbitrary domains by reparametrising with $u$-substitution.

% Consider a parametric function $\vec{x} = \vec{r}(\vec{u})$ which defines a transformation from euclidean space to some domain where the limits of integration are constant. Furthermore, as the indefinite integral depends on our choice of $\vec{r}$, we parametrise $F_\theta(\vec{u})$ instead of $F_\theta(\vec{x})$. Simplifying the notation of our iterated integral and applying this reparametrisation, our equation becomes:

% \begin{align}
%     F_\theta(\vec{u}) &= \int f(\vec{r}(\vec{u})) \; \lvert \nabla \vec{r}(\vec{u}) \rvert \, d\vec{u} \\
%     f(\vec{r}(\vec{u})) &= \frac{1}{\lvert \nabla \vec{r}(\vec{u}) \rvert} \cdot \frac{\partial}{\partial u_1} \frac{\partial}{\partial u_2} \cdots \frac{\partial}{\partial u_n} F_\theta(\vec{u}) \\
%     f(\vec{x}) &= \frac{1}{\lvert \nabla \vec{r}(\vec{u}) \rvert} \cdot \frac{\partial}{\partial u_1} \frac{\partial}{\partial u_2} \cdots \frac{\partial}{\partial u_n} F_\theta(\vec{r}^{-1}(\vec{x}))
% \end{align}

% In this formulation, we must select $\vec{r}$ according to our desired domain. For example, if we wish to integrate over the unit circle in $\mathbb{R}^2$, we can select $\langle x_1, x_2 \rangle = \vec{r}(\vec{u}) = \langle u_1 \cos(u_2), u_1 \sin(u_2) \rangle$. The differential after reparametrising with $\vec{r}$ is given by the determinant of the Jacobian $\lvert \nabla \vec{r}(\vec{u}) \rvert$. In this case, the differential is given by $|\nabla \langle u_1 \cos(u_2), u_1 \sin(u_2) \rangle| = \cos(u_2) \cdot u_1 \cos(u_2) - \sin(u_2) \cdot (-u_1 \sin(u_2)) = u_1$. 

\subsection{Positivity Constraint}

In many applications of FINN, it is necessary to constrain $f$ to be non-negative. This is useful in cases where $f$ is only defined in the positive domain (see \autoref{sec:Applications}).

Following from \autoref{eq:f_F}, in order to apply this constraint, we must ensure that the mixed partial of $F_\theta$ is non-negative. To do this, we define a new neural network layer to construct our multi-layer perceptron (MLP):

\begin{equation}
    \sigma_n \left( \lvert W \! \rvert \, \vec{x} + b \right)
\end{equation}

In this layer, we apply an absolute value to the weights (but not the bias), and we use a custom activation function $\sigma_n$, which is conditioned on the dimension of the input. Note that although we wish the mixed partial of $F_\theta$ to be non-negative, it is too constraining to restrict further derivatives to be non-negative as well. The derivative of our function $\dot{f}$ (with respect to any input dimension) should be able to represent positive \textit{or} negative values. To satisfy these criteria, we define the following activation function using the error function $\mathrm{erf}(x) = \frac{2}{\sqrt{\pi}} \int_0^x e^{-t^2} dt$:

\begin{equation}
    \sigma_n = \underbrace{\int \int \cdots \int}_{n-1} \frac{\mathrm{erf}(x)+1}{2} \; \underbrace{dx \cdots \, dx \, dx}_{n-1}
\end{equation}

For $n=1$, this simplifies to $\frac{\mathrm{erf}(x)+1}{2}$, which closely resembles sigmoid. For $n=2$, it resembles softplus, which is the integral of sigmoid. While they are similar, it is crucial that we use this custom activation instead of sigmoid, because the higher-order integrals of sigmoid evaluate to the polylogarithm, which has no closed-form solution. Conversely, all of the integrals of the error function have analytical solutions in terms of linear compositions of constants, power functions $x^k$, exponentials $e^{-x^2}$, and the error function itself $\mathrm{erf}(x)$ (all of which have efficient implementations). In practice, we use symbolic math to compute the integral once at initialisation time, and then each forward pass evaluates the resulting expression.


%%%%%%%%% EXPERIMENTS
\section{Experimental Evaluations}\label{sec:experiment}

\textbf{Implementation.}
We implement \puma\ on top of SecretFlow~\citep{spu} in \textrm{C++} and Python. SecretFlow compiles a high-level Flax code to secure computation protocols, which are then executed by our designed cryptographic backends, and we encode the floating-ponit values as $64$-bit integers in ring $\mathbb{Z}_{2^{64}}$ with $18$-bit fractional part. 
Our experiments are run on 3 Alibaba Cloud ecs.g7.8xlarge servers with 32 vCPU and 128GB RAM each. The CPU model is Intel Xeon(Ice Lake) Platinum 8369B CPU @ 2.70GHz. We evaluate \puma\ on Ubuntu 20.04.6 LTS with Linux kernel 5.4.0-144-generic. Our bandwidth is about 5Gbps and round trip time is about 1ms. %\cheng{Describe fixed point parameters: scale, share bits.}

\textbf{Models \& Datasets.}
We evaluate \puma\ on seven NLP models: Bert-Base, Roberta-Base, and Bert-Large~\citep{bert}; GPT2-Base, GPT2-Medium, and GPT2-Large~\citep{gpt}; and LLaMA-7B~\citep{touvron2023llama}. We measure the Bert performance for three NLP tasks over the datasets of Corpus of Linguistic Acceptability (CoLA), Recognizing Textual Entailment (RTE), Stanford Question Answering Dataset (QNLI) from GLUE benchmarks~\citep{wang2018glue}, and GPT2 performance on Wikitext-103 V1~\citep{merity2016pointer}.

\textbf{Baseline.}
We compare \puma\ to the most similar prior work \mpcformer~\citep{li2023mpcformer}. But for fair comparison, we have the following considerations:
\romannumeral1) As \mpcformer\ neither supports loading pretrained transformer models nor implements LayerNorm faithfully\footnote{ As \mpcformer~does not support loading pre-trained Transformer models, we did an experiment in plaintext Bert-Base that replaced LayerNorm with BatchNorm  as \mpcformer~did. This  resulted in a significant drop in the MCC score for CoLA task from $0.616$ to $-0.020$. On the contrary, \puma~achieves an MCC score of $0.613$. }, we cannot achieve meaningful secure inference results using their framework.
Therefore, we compare our secure Transformer models inference performance to that of plaintext (floating-point) to show our precision guarantee.
\romannumeral2) \mpcformer\ with \textit{Quad} approximations (for both $\gelu$ and $\softmax$) requires retraining the  modified models. As \puma\ does not require retraining, we compare our cost to that of \mpcformer\ without \textit{Quad} approximations. Also, we re-run \mpcformer~in our environment.



\subsection{Precision}\label{sec:accuracy}

% Figure environment removed

%\begin{table}
\centering
\caption{Performance on GLUE benchmark of Bert-Base, Roberta-Base, and Bert-Large on CoLA, RTE, and QNLI, Matthews correlation is reported for CoLA. Accuracy is reported for other datasets.}\label{table:bertacc}
\begin{tabular}{c|ccc|ccc|ccc}
\hline \hline
 Model & \multicolumn{3}{c|}{Bert-Base} & \multicolumn{3}{c|}{Roberta-Base} & \multicolumn{3}{c}{Bert-Large} \\ \hline
 TASK & CoLA & RTE & QNLI & CoLA & RTE & QNLI & CoLA & RTE & QNLI \\ \hline
CPU & $0.616$     & $0.700$      & $0.916$     & $0.629$ & $0.805$ & $0.920$  & $0.686$   & $0.755$ & $0.922$ \\
\puma   & $0.613$     & $0.700$     & $0.916$     & $0.618$ & $0.805$ & $0.918$ & $0.690$ & $0.747$ & $0.918$ \\ \hline \hline
\end{tabular}
\end{table}

\begin{table}[]
    \centering
    \caption{Perplexity of GPT2-Base, GPT2-Medium, and GPT2-Large on Wikitext-103 V1.}
    \label{tab:gpot2ppl}
    \begin{tabular}{c|c|c|c}
    \hline \hline
      Model & GPT2-Base & GPT2-Medium & GPT2-Large \\ \hline
      CPU & $16.284$ & $12.536$ & $10.142$ \\
      \puma & $16.284$ & $12.540$ & $10.161$ \\
      \hline \hline
    \end{tabular}
    
\end{table}

We compare our secure model 
inference performance to that of plaintext (floating-point) in Figure~\ref{fig:performance} to show our precision guarantee.

In Figure~\ref{fig:bert-base}-\ref{fig:bert-large}, we show the Matthews correlation/accuracy of plaintext and \puma\ on the Bert-Base, Roberta-base, and Bert-Large. We observe that the accuracy achieved by \puma~ matches the accuracy of the plaintext Flax code. Specifically, the accuracy difference does
not exceed $0.011$ over all datasets. 

Moreover, in Figure~\ref{fig:gpt2}, we also compare our perplexity on dataset Wikitext-103 V1 with the plaintext baseline on models GPT2-Base, GPT2-Medium, and GPT2-Large. The results are similar and the perplexity differences do not exceed $0.02$ over all models.

The above accuracy and perplexity advantages experimentally validate that our protocols are numerically precise. 

\subsection{Inference cost}\label{sec:efficiency}
\begin{table}[h]
    \centering
    \caption{Costs of Bert-Base, Roberta-Base, and Bert-Large for one sentence of length $128$. Time is in seconds and Communication (Comm. for short) is in GB, which is the same for the following tables.}\label{tab:costbert}
    \begin{tabular}{c|cc|cc|cc}
    \hline \hline
       Model & \multicolumn{2}{c|}{Bert-Base} & \multicolumn{2}{c|}{Roberta-Base} & \multicolumn{2}{c}{Bert-Large} \\ \hline
       Costs & Time & Comm. & Time & Comm. & Time & Comm. \\ \hline
       \mpcformer & $55.320$ & $12.089$ & $57.256$ & $12.373$ & $141.222$ & $32.577$ \\
       \puma & $33.913$ & $10.773$ & $41.641$ & $11.463$ & $73.720$ & $27.246$ \\
       \cellcolor{mygray} Improv. & \cellcolor{mygray} $1.631\times$ & \cellcolor{mygray} $1.122\times$ & \cellcolor{mygray} $1.375\times$ & \cellcolor{mygray} $1.079\times$ & \cellcolor{mygray} $1.916\times$ & \cellcolor{mygray} $1.195\times$ \\
       \hline \hline
    \end{tabular}
    \vspace{-0.2cm}
\end{table}

\begin{table}[]
    \centering
    \caption{Costs of GPT2-Base, GPT2-Medium, and GPT2-Large. The input sentence is of length $32$, all of the costs are for generating $1$ token.}\label{tab:costgpt2}
    \begin{tabular}{c|cc|cc|cc}
    \hline \hline
       Model & \multicolumn{2}{c|}{GPT2-Base} & \multicolumn{2}{c|}{GPT2-Medium} & \multicolumn{2}{c}{GPT2-Large} \\ \hline
       Costs & Time & Comm. & Time & Comm. & Time & Comm. \\ \hline
       \mpcformer & $34.889$ & $4.999$ & $73.078$ & $11.766$ & $129.095$ & $22.522$  \\
       \puma & $15.506$ & $3.774$ & $30.272$ & $7.059$ & $54.154$ & $11.952$ \\
       \cellcolor{mygray} Improv. & \cellcolor{mygray} $2.250\times$ & \cellcolor{mygray} $1.325\times$ & \cellcolor{mygray} $2.414\times$ & \cellcolor{mygray} $1.667\times$ & \cellcolor{mygray} $2.383\times$ & \cellcolor{mygray} $1.884\times$ \\
       \hline \hline
    \end{tabular}
    \vspace{-0.2cm}
\end{table}

In this subsection, we compare \puma's inference cost to that of \mpcformer. 
We evaluate  three Bert models (Bert-Base, Roberta-Base, and Bert-Large) and three GPT2 models (GPT2-Base, GPT2-Medium, and GPT2-Large).
The costs are for processing one input sentence: \romannumeral1) For Bert models the input sentence is of length $128$. \romannumeral2) GPT2 models input one length-32 sentence and generate $1$ new word. 

On the 3 Bert models in Table~\ref{tab:costbert}, \puma\ is  $1.375\sim 1.916\times$ faster than  \mpcformer, and is $1.079\sim 1.195\times$ more communication-efficient. For the GPT2 models in Table~\ref{tab:costgpt2}, \puma\ is $2.250\sim 2.414\times$ faster than \mpcformer, and is $1.325\sim 1.884\times$ more communication-efficient. 
    
We observe that \puma's improvements increase as the model size grows, particularly for the GPT2 models. This trend is because our specialized optimizations are more effective when processing large-scale evaluations.



\subsection{Scalability}\label{sec:scala}

In this subsection, we measure the costs of evaluating \puma\ on Bert-Base and GPT2-Base models for varying-length inputs, and varying-length outputs (only for GPT2-Base). We also compare our costs to those of \mpcformer~to demonstrate our improvements.





\begin{table}[]
    \centering
    \caption{Costs of Bert-Base and GPT2-Base for different input length (denoted as \#Input). The input lengths for Bert-Base and GPT2-Base are respective $\{64, 128, 256, 512\}$ and $\{16, 32, 64, 128\}$. GPT2-Base generates $1$ token.}\label{tab:costbertinput}
    \begin{tabular}{cc|cc|cc|cc|cc}
    \hline \hline
       \multicolumn{2}{c|}{\#Input} & \multicolumn{2}{c|}{$64 / 16$} & \multicolumn{2}{c|}{$128 / 32$} & \multicolumn{2}{c|}{$256 / 64$} & \multicolumn{2}{c}{$512 / 128$}  \\ \hline
       \multicolumn{2}{c|}{Costs} & Time & Comm. & Time & Comm. & Time & Comm. & Time & Comm. \\ \hline
       \multirow{3}{*}{Bert}& \mpcformer & $46.428$ & $4.750$ & $85.887$ & $9.673$ & $196.372$ & $23.443$ & $582.787$ & $68.069$ \\
       & \puma & $24.345$ & $1.627$ & $42.525$ & $3.591$ & $87.561$ & $8.668$ & $212.600$ & $23.439$\\
       & \cellcolor{mygray} Improv. & \cellcolor{mygray} $1.907\times$ & \cellcolor{mygray} $2.919\times$ & \cellcolor{mygray} $2.020\times$ & \cellcolor{mygray} $2.694\times$ & \cellcolor{mygray} $2.243\times$ & \cellcolor{mygray} $2.705\times$ & \cellcolor{mygray} $2.741\times$ & \cellcolor{mygray} $2.904$ \\
       \hline
       \multirow{3}{*}{GPT2}& \mpcformer & $34.522$ & $3.767$ & $42.615$ & $4.516$ & $60.451$ & $6.281$ & $105.028$ & $11.225$  \\
       & \puma & $20.692$ & $0.625$ & $29.248$ & $1.258$ & $40.968$ & $2.607$ & $74.529$ & $5.611$\\
       &\cellcolor{mygray} Improv. & \cellcolor{mygray} $1.668\times$ & \cellcolor{mygray} $6.027\times$ & \cellcolor{mygray} $1.457\times$ & \cellcolor{mygray} $3.590\times$ & \cellcolor{mygray} $1.476\times$ & \cellcolor{mygray} $2.409\times$ & \cellcolor{mygray} $1.409\times$ & \cellcolor{mygray} $2.001\times$\\
       \hline \hline
    \end{tabular}
\end{table}
\textbf{Input Length Evaluation.}
Table~\ref{tab:costbertinput} shows our costs on varying-length inputs, we evaluate Bert-Base on the inputs of length $\{64, 128, 256, 512\}$, and GPT2-Base on the inputs of length $\{16, 32, 64, 128\}$.
For Bert-Base, \puma\ is $1.720\sim 2.282\times$ faster, and for GPT2-Base, \puma\ is $1.550\sim 2.686\times$ faster. Unlike the observations in Section~\ref{sec:efficiency}, our efficiency gains decrease with increasing input sizes in GPT2, and \puma\ requires more communication when the input length is greater than 64. This phenomenon is attributed to the interesting fact: To directly support pre-trained plaintext models, \puma\ strictly follows the plaintext model format that only accept token ids as input, so \puma\ has to compute the one-hot vectors from token ids in an MPC way. On the other hand, \mpcformer\ uses modified models that accept one-hot vectors as input, so the one-hot function could be computed at the client side in plaintext. Nevertheless, \puma\ remains faster than \mpcformer.

%\begin{table}[]
    \centering
    \caption{Costs of GPT2-small for generating different output tokens (denoted as \#Output), the input length is set as $32$.}\label{tab:costgpt2tokens}
    \begin{tabular}{c|cc|cc|cc|cc}
    \hline \hline
       \#Output & \multicolumn{2}{c|}{2} & \multicolumn{2}{c|}{4} & \multicolumn{2}{c|}{8} & \multicolumn{2}{c}{16}  \\ \hline
       Costs & Time & Comm. & Time & Comm. & Time & Comm. & Time & Comm. \\ \hline
       \mpcformer & $72.833$ & $7.676$ & $132.644$ & $13.998$ & $252.796$ & $26.648$ & $494.509$ & $51.972$ \\
       \puma & $53.191$ & $2.549$ & $111.457$ & $5.167$ & $215.352$ & $11.115$ & $457.994$ & $24.917$ \\
       Improv. & $1.369\times$ & $3.011\times$ & $1.190\times$ & $2.709\times$ & $1.174\times$ & $2.397\times$ & $1.080\times$ & $2.086\times$ \\
       \hline \hline
    \end{tabular}
\end{table}

\begin{wrapfigure}{r}{0.4\textwidth}
    % Figure removed
    \caption{Runtime of GPT2-Base for generating different number of output tokens, the input length is of length $32$.} 
    \label{fig:gptwoutcosts}
\end{wrapfigure}

\textbf{Output Length Evaluation.}
Fig~\ref{fig:gptwoutcosts} presents our costs on varying-length outputs for GPT2-Base, and compares our costs to those of \mpcformer. Our improvements in runtime range from $1.279\sim 2.700\times$ respectively.
As more output tokens are generated, both costs increase in a linear way, this is because each output token must be input back into the model to generate the next token, increasing the required one-hot embedding costs. We should emphasize
again that although the time costs might be close for long outputs, \puma\ could achieve a similar accuracy as plaintext models while \mpcformer\  could not. 


\begin{table}[]
    \centering
    \caption{Costs of the secure inference of LLaMA-7B, \#Input denotes the length of input sentence and \#Output denotes the number of generated tokens.}\label{tab:llama7b}
    \begin{tabular}{c|cc|cc|cc}
    \hline \hline
       (\#Input, \#Output) & \multicolumn{2}{c|}{$(4,1)$} & \multicolumn{2}{c|}{$(8,1)$} & \multicolumn{2}{c}{$(8,2)$} \\ \hline
       Costs & Time & Comm. & Time & Comm. & Time & Comm. \\ \hline
       \puma & $122.004$ & $0.907$ & $200.473$ & $1.794$ & $364.527$ & $3.857$ \\
       \hline \hline
    \end{tabular}
    \vspace{-0.2cm}
\end{table}

\textbf{Scale to LLaMA-7B in Five Minutes.}
We evaluated the large language model LLaMA-7B using \puma\ under 3 Alibaba Cloud
ecs.r7.32xlarge servers, each has 128 threads and 1TB RAM, with 20GB bandwidth, 0.06ms round-trip-time. 
As shown in Table~\ref{tab:llama7b}, \puma\ can support the secure inference of large language model LLaMA-7B with reasonable costs. For example, given an input sentence of 8 tokens, \puma\ can output one token in around $346.126$ seconds with communication costs of $1.865$ GB. To our knowledge, this is the first time that LLaMA-7B has been evaluated using MPC.


%Llama-7B, LAN=(20GB, 0.06ms), 128 threads, input length=8, output=1 token, costs: 346.126s, 2002213760 bytes

%%%%%%%%% CONCLUSION
\section{Conclusion}\label{sec:conclusion}

This paper presents our empirical domain knowledge distillation framework using ChatGPT and discusses our observations from the framework application experiments in the autonomous driving domain. The key finding is that: 1) with proper design of prompt engineering and execution flow, fully automated domain knowledge (in the ontology format) distillation is possible. However, due to the randomness in the response and the butterfly effect, the quality of fully automated distillation results is not guaranteed. To address this, we develop a web-based assistant to enable manual supervision and early intervention at runtime. We hope our findings and tools inspire future research toward revolutionizing the engineering processes of knowledge-based systems across domains.

\subsection*{Acknowledgements}

\noindent
USA {\textendash} U.S. National Science Foundation-Office of Polar Programs,
U.S. National Science Foundation-Physics Division,
U.S. National Science Foundation-EPSCoR,
Wisconsin Alumni Research Foundation,
Center for High Throughput Computing (CHTC) at the University of Wisconsin{\textendash}Madison,
Open Science Grid (OSG),
Extreme Science and Engineering Discovery Environment (XSEDE),
Frontera computing project at the Texas Advanced Computing Center,
U.S. Department of Energy-National Energy Research Scientific Computing Center,
Particle astrophysics research computing center at the University of Maryland,
Institute for Cyber-Enabled Research at Michigan State University,
and Astroparticle physics computational facility at Marquette University;
Belgium {\textendash} Funds for Scientific Research (FRS-FNRS and FWO),
FWO Odysseus and Big Science programmes,
and Belgian Federal Science Policy Office (Belspo);
Germany {\textendash} Bundesministerium f{\"u}r Bildung und Forschung (BMBF),
Deutsche Forschungsgemeinschaft (DFG),
Helmholtz Alliance for Astroparticle Physics (HAP),
Initiative and Networking Fund of the Helmholtz Association,
Deutsches Elektronen Synchrotron (DESY),
and High Performance Computing cluster of the RWTH Aachen;
Sweden {\textendash} Swedish Research Council,
Swedish Polar Research Secretariat,
Swedish National Infrastructure for Computing (SNIC),
and Knut and Alice Wallenberg Foundation;
Australia {\textendash} Australian Research Council;
Canada {\textendash} Natural Sciences and Engineering Research Council of Canada,
Calcul Qu{\'e}bec, Compute Ontario, Canada Foundation for Innovation, WestGrid, and Compute Canada;
Denmark {\textendash} Villum Fonden and Carlsberg Foundation;
New Zealand {\textendash} Marsden Fund;
Japan {\textendash} Japan Society for Promotion of Science (JSPS)
and Institute for Global Prominent Research (IGPR) of Chiba University;
Korea {\textendash} National Research Foundation of Korea (NRF);
Switzerland {\textendash} Swiss National Science Foundation (SNSF);
United Kingdom {\textendash} Department of Physics, University of Oxford.


{\small
\bibliographystyle{ieee_fullname}
\bibliography{egbib}
}


\clearpage
\twocolumn[
\begin{@twocolumnfalse}
    \centering{\Large\textbf{Supplementary Materials: Augmented Box Replay: \\ Overcoming Foreground Shift for Incremental Object Detection}\vspace{20pt}\\
    \large{Yuyang Liu\textsuperscript{1,2,3} \enspace Yang Cong\textsuperscript{4} \enspace Dipam Goswami\textsuperscript{5} \enspace Xialei Liu\textsuperscript{6} \enspace Joost van de Weijer\textsuperscript{5,7}\\
    \textsuperscript{1}State Key Laboratory of Robotics, Shenyang Institute of Automation, Chinese Academy of Sciences \\
    \textsuperscript{2} Institutes for Robotics and Intelligent Manufacturing, Chinese Academy of Sciences \\ 
    \textsuperscript{3}University of Chinese Academy of Sciences 
    \space\space \space\space 
    \textsuperscript{4}South China University of Technology\\ 
    \textsuperscript{5}Computer Vision Center, Barcelona\space\space\space\space 
    \textsuperscript{6}VCIP, CS, Nankai University\\
    \textsuperscript{7}Department of Computer Science, Universitat Autònoma de Barcelona\\}
    {\tt\small liuyuyang@sia.cn, congyang81@gmail.com, \{dgoswami, joost\}@cvc.uab.es, xialei@nankai.edu.cn}\vspace{25pt}}
\end{@twocolumnfalse}
]

\ificcvfinal\thispagestyle{empty}\fi
\appendix \label{sec:appendix}
\setcounter{table}{6}
\setcounter{figure}{5}
\setcounter{equation}{7}

\section{Additional Methods}
\subsection{Prototype Box Selection}\label{sec:pbs}

This method involves selecting the most representative boxes, as prototypes, from the current training data, which are then replayed along with the future training data. The memory buffer is commonly denoted as $B^t$, where $t$ represents the current task and the size $M$ of $B^t$ is limited. Therefore, the selection is an important factor that affects the performance. We employ a frozen trained model to generate the Region of Interest (RoI)-Aligned feature maps $\{F^{t}_g \in \mathbb{R}^{C\times S \times S}\}_{g=1}^{G^t_n}$ for $G^t_n$ groundtruth boxes in the current task $t$, where $C$ is the number of feature planes and $S$ is the spatial dimension. Then, a prototype feature map $\hat{F}^{t}_{c}$ for each class $c \in \mathcal{C}^{t}$ can be computed by:
\begin{equation}
\hat{F}^{t}_{c}=\frac{1}{|{F}^{t}_{c}|}\sum_{g=1}^{G^t_n} F^{t}_{g},\quad \forall c_g = c,
\label{eq:profm}
\end{equation}
The distance between each feature map $F^{t}_{g}$ and the prototype feature map $\hat{F}^{t}_{c}$ for class $c$ is computed using the Euclidean distance:
\begin{equation}
d(F^{t}_{g}, \hat{F}^{t}_{c}) = \sqrt{\sum (F^{t}_{g}-\hat{F}^{t}_{c})^2},\quad \forall c_g = c,
\label{eq:dis}
\end{equation}
Then we sort $\{d(F^{t}_{g}, \hat{F}^{t}_{c}),\forall c_g = c\}_{g=1}^{G^t_n}$ in ascending order, and select the top $M_c=\frac{M}{|\mathcal{C}^{1:t}|}$ boxes for that class to form the box buffer $B^{t}_{c}$. The final $B^{t}$ can focus on the most relevant information for each task and avoid redundant or irrelevant information, as shown in~\cref{alg:1}.

Additionally, since boxes are typically smaller than whole images, the computational cost of training and rehearsal can be reduced, making the approach more scalable to large datasets and complex models. The entire flow of our proposed method is shown in~\cref{algorithm2}.

    \renewcommand{\algorithmicrequire}{{\textbf{Input:}}}
    \renewcommand{\algorithmicensure}{{\textbf{Output:}}}
    \begin{algorithm}[ht]
    	\caption{Prototype Box Selection (PBR)}
	\begin{algorithmic}[1]
		\REQUIRE  The frozen trained model in $f_{\theta_t}(\cdot)$, the stream data $D^t$ at current task $t$, each image $I^t_n$ has $G_n^t$ groundtruth labels $\{y_g\}_{g=1}^{G_n^t}$, the box rehearsal memory $B^{t-1}$ after task $t-1$, the box rehearsal memory size $M$, the seen classes $\mathcal{C}^{1:t}$ until task $t$.
		\ENSURE The updated $B^{t}$ after task $t$. \\
		\STATE {\textbf{Initialize}}: $B^{t} = \{\}$, $m^t$ = ceil($M/|\mathcal{C}^{1:t}|$);
            \STATE $F^t_g$ = $f_{\theta_t}(I^t_n, y_g)$, $\forall{n} \in N^t$, $\forall{g} \in G^t_n$;
            \STATE $b_g=crop(I^t_n, y_g)$, 
            $\forall{n} \in N^t$, $\forall{g} \in G^t_n$;      
        \FOR {$c$ in $\mathcal{C}^{1:t}$}
        \IF{$c \in \mathcal{C}^{t}$}
            % \STATE $\hat{F}_c^t=\left\{F^t_g \mid c_g=c\right\}$;
            \STATE Compute $\hat{F}^{t}_{c}$ for each class $c$ based on~\cref{eq:profm};
            % \STATE $\hat{F}_c^t= mean(F_c^t)$;
            \STATE $D_c = \left\{(b_g, y_g)\mid c_g=c\right\}$;
            \STATE Sort ${D}_c$ following~\cref{eq:dis};
            \STATE $B^{t}+=D_c[0:m^t]$;
        \ELSE
            \FOR {$j=1, 2, ..., m^t$}
    		\STATE $i=j *\left|{B}_c^{t-1}\right| / ceil(M/|\mathcal{C}^{1:t-1}|)$;
    		\STATE ${B}^t+={B}_c^{t-1}[i]$;
    		\ENDFOR
       \ENDIF\ENDFOR
	\end{algorithmic}
	\label{alg:1}
    \end{algorithm}

    \renewcommand{\algorithmicrequire}{{\textbf{Input:}}}
    \renewcommand{\algorithmicensure}{{\textbf{Output:}}}
    \begin{algorithm}[ht]
    	\caption{Augmented Box Replay Method}
            % \scriptsize
    	\begin{algorithmic}[1]
    		\REQUIRE  $f_{\theta_{t-1}}(\cdot)$, $D^t$=$\{I^t_n, G_n^t\}_{n=1}^{N_t}$, $B^{t-1}$ and Rat=1:1:2.
    		\ENSURE The updated $B^{t}$ and $f_{\theta_t}(\cdot)$ after task $t$. \\
    		\STATE {\textbf{Initialize}}: $\theta_{t} = \theta_{t-1}$;
            \FOR {$n$ in $N_t$}
                \STATE MIX,MOS,NEW=GenerateReplayType(Rat);
                \IF{MIX}
                    \STATE Compute $\hat{I}^t_n, \hat{G}^t_n$ by MixupBoxReply($I^t_n, G_n^t$);
                \ELSIF{MOS}
                    \STATE Compute $\hat{I}^t_n, \hat{G}^t_n$ by MosaicBoxReply($I^t_n, G_n^t$);
                \ELSIF{NEW}
                    \STATE $\{\hat{I}^t_n, \hat{G}^t_n\} = \{I^t_n, G_n^t$\};
                \ENDIF
                \STATE $\mathcal{L}_{Dis}=$ DistiallationLosses($f_{\theta_{t-1}}(\cdot), f_{\theta_{t}}(\cdot), \hat{I}^t_n$);
                \STATE $\mathcal{L}_{Det}=$ DetectionLosses($f_{\theta_{t}}(\cdot), \{\hat{I}^t_n, \hat{G}^t_n\}$);
                \STATE Update $\theta_{t}$ by $\mathcal{L}_{Dis}+\mathcal{L}_{Det}$;
           \ENDFOR
           \STATE Update $B_t$ by PBS($f_{\theta_{t}}(\cdot), D^t, B^{t-1}$);
    	\end{algorithmic}
    	\label{algorithm2}
    \end{algorithm}



\section{Additional Analysis}
\subsection{Analysis foreground shift problem}

In Table \textcolor{red}{1} and Table \textcolor{red}{2}, our algorithm demonstrates a remarkable improvement in mean Average Precision (mAP) ranging from 0.2$\sim$20\% across all categories. Additionally, it exhibits a substantial mAP boost of 4.5\% to 25.2\% in new categories (foreground categories), indicating the enhanced stability and plasticity achieved by our method.

Moreover, we conducted a comprehensive analysis of False Positives (FP)\cite{hoiem2012diagnosing} under the VOC 10-10 setting. \cref{fig:sm-fp_bg} visually represents the number of background errors, specifically detections confused with the background or unlabeled objects. Notably, our approach (ABR) demonstrates a clear advantage, exhibiting a substantial reduction of 275 errors in new (foreground) classes compared to the ImageReplay method. This compelling result strongly suggests the successful mitigation of the foreground shift problem by our proposed approach.

% Figure environment removed

\subsection{Analysis Attentive RoI Distillation (ARD)}

While existing methods have utilized attention distillation primarily on feature maps, we advance this approach by integrating location information of Region of Interest (RoI) proposals. By doing so, our model gains the capability to distill both feature and localization information from the replayed and new objects, leading to an overall performance enhancement.

\cref{fig:sm-attmap} showcases some additional attention maps, highlighting how our Attention-based RoI Distillation (ARD) loss effectively retains attention on the old class (e.g., bicycle). This observation confirms ARD's competence in alleviating catastrophic forgetting, a phenomenon that impacts model performance when learning new tasks.

Through the inclusion of location-awareness in attention distillation, our proposed ARD method exemplifies its potential to mitigate catastrophic forgetting and reinforce the preservation of crucial knowledge from previous tasks, resulting in improved overall model performance.


% Figure environment removed

\subsection{Effect of Hyperparameters}
% Figure environment removed
We conducted additional experiments under the VOC 10-10 setting to analyze the impact of all hyperparameters in our study, as depicted in~\cref{fig:sm-hyp}. 
For $\gamma$ in Eq. \textcolor{red}{5} of the overall ARD loss function, we vary it in range [0.5, 1.0, 5.0]. From the results shown in the first figure of~\cref{fig:sm-hyp}, we find that the default $\gamma=1$ provides good results. 

In consequence, we optimize the total objective function to realize incremental object detecion learning:
\begin{equation}
    \mathcal{L}_{total} = \mathcal{L}_{faster\_rcnn} + \alpha\mathcal{L}_{ID}+\beta\mathcal{L}_{ARD}
\end{equation}
where $\alpha$ and $\beta$ weight for the Inclusive Distillation Loss and Attentive RoI Distillation, respectively. We vary it in range [0.1, 0.2, 0.5, 1].
The performance varies as a function of $\alpha, \beta$ outperforming the state-of-the-art (66.8) for most combinations.

\section{Additional Results}\label{sec:ar}
\subsection{Detailed Results for the Long Sequences}
\begin{table*}[ht]
\centering
\caption{Per-Class AP@50 and Overall mAP@50 values in different task on PASCAL-VOC 2007 5-5 setting. }
\label{tab:longseq}
\renewcommand\arraystretch{1.2}
\scriptsize

\setlength{\tabcolsep}{0.5pt}
\begin{tabular}{l|l|cccccccccc:c:ccccc:c:ccccc:c|c}
\multicolumn{1}{l|}{\textbf{Class Split}} & \textbf{Method} & \textbf{aero} & \textbf{cycle} & \textbf{bird} & \textbf{boat} & \textbf{bottle} & \textbf{bus} & \textbf{car} & \textbf{cat} & \textbf{chair} & \textbf{cow} & \multicolumn{1}{c:}{\underline{\textbf{mAP-task1}}} & \textbf{table} & \textbf{dog} & \textbf{horse} & \textbf{bike} & \textbf{person} & \multicolumn{1}{c:}{\underline{\textbf{mAP-task2}}} & \multicolumn{1}{c}{\textbf{plant}} & \multicolumn{1}{c}{\textbf{sheep}} & \multicolumn{1}{c}{\textbf{sofa}} & \multicolumn{1}{c}{\textbf{train}} & \multicolumn{1}{c:}{\textbf{tv}} & \multicolumn{1}{c|}{\underline{\textbf{mAP-task3}}} & \multicolumn{1}{c}{\underline{\textbf{mAP-total}}} \\ \hline\hline
\textbf{1-20} & \textbf{JT} & 72.7  & 81.0  & 76.0  & 58.9  & 62.0  & 76.4  & 87.4  & 85.7  & 72.6  & 82.4  & 75.5  & 57.7  & 83.2  & 85.7  & 80.5  & 84.2  & 78.3  & \multicolumn{1}{c}{45.8 } & \multicolumn{1}{c}{77.1 } & \multicolumn{1}{c}{65.9 } & \multicolumn{1}{c}{75.7 } & \multicolumn{1}{c:}{74.5 } & 67.8  & 74.3  \\\hline
\multirow{2}{*}{\textbf{(1-5)+6-10}} & \textbf{MMA} & 73.8  & 80.8  & 71.2  & 52.5  & 63.3  & 55.2  & 74.9  & 65.2  & 39.1  & 73.3  & 64.9  & \multicolumn{1}{c}{} & \multicolumn{1}{c}{} & \multicolumn{1}{c}{} & \multicolumn{1}{c}{} & \multicolumn{1}{c:}{} &  & \multicolumn{1}{c}{} & \multicolumn{1}{c}{} & \multicolumn{1}{c}{} & \multicolumn{1}{c}{} & \multicolumn{1}{c:}{}  &  & 64.9  \\
 & \textbf{ABR} & 71.7  & 82.6  & 69.5  & 53.6  & 63.8  & 63.0  & 79.0  & 68.5  & 47.0  & 78.4  & \textbf{67.7}  &  &  &  &  &  &  &  &  &  &  &  &  & \textbf{67.7}  \\\hline
\multirow{2}{*}{\textbf{(1-10)+11-15}} & \textbf{MMA} & 67.4  & 78.1  & 64.5  & 49.7  & 63.5  & 23.1  & 34.5  & 26.3  & 8.7  & 35.0  & 45.1  & 47.5  & 52.8  & 67.5  & 65.9  & 76.0  & 61.9  &  &  &  &  &  &  & 50.7  \\
 & \textbf{ABR} & 68.5  & 79.6  & 67.3  & 51.9  & 56.7  & 60.2  & 75.2  & 62.8  & 38.6  & 62.0  & \textbf{62.3}  & 54.0  & 66.3  & 76.9  & 74.5  & 77.3  & \textbf{69.8}  & \multicolumn{1}{c}{} & \multicolumn{1}{c}{} & \multicolumn{1}{c}{} & \multicolumn{1}{c}{} & \multicolumn{1}{c:}{} &  & \textbf{64.8}  \\\hline
\multirow{2}{*}{\textbf{(1-15)+16-20}} & \textbf{MMA} & 72.3  & 75.5  & 57.0  & 46.9  & 59.9  & 4.8  & 32.4  & 38.5  & 3.3  & 1.4  & 39.2  & 0.7  & 28.8  & 42.2  & 44.1  & 18.2  & 26.8  & \multicolumn{1}{c}{36.0 } & \multicolumn{1}{c}{46.5 } & \multicolumn{1}{c}{52.0 } & \multicolumn{1}{c}{52.0 } & \multicolumn{1}{c:}{66.6 } & 50.6  & 38.9  \\
 & \textbf{ABR} & 69.3  & 80.0  & 65.6  & 53.9  & 54.6  & 52.2  & 75.5  & 69.4  & 34.3  & 69.6  & \textbf{62.4}  & 22.9  & 41.8  & 48.7  & 53.7  & 60.8  & \textbf{45.6}  & \multicolumn{1}{c}{39.6 } & \multicolumn{1}{c}{71.3 } & \multicolumn{1}{c}{59.2 } & \multicolumn{1}{c}{76.1 } & \multicolumn{1}{c:}{70.4 } & \textbf{63.3}  & \textbf{58.4}
\end{tabular}
\end{table*}
In \cref{tab:longseq}, we present the results of our experiments with long sequences on the PASCAL-VOC 2007 dataset. To simulate this scenario, we trained our detector on images from the first 5 classes and gradually added classes 6 to 20 in groups of five.

The table shows the class-wise average precision (AP)@0.5 and the corresponding mean average precision (mAP). The first row (JT) represents the upper-bound where the detector is trained on data from all 20 classes. The subsequent three pairs of rows demonstrate the results obtained when adding five new classes at a time. The notation (1-5)+6..10 is used to represent this setting. Our proposed ABR method outperforms the previous state-of-the-art method MMA~\cite{cermelli2022modeling} on all sequential tasks, as can be seen from the results in \cref{tab:longseq}. Therefore, the ABR method can be more useful in real-world scenarios where new object classes are frequently introduced. Additionally, the ABR method is a novel approach that may have implications for future research in object detection.

% \subsection{Additional qualitative results for False Positive}

% The experimental analysis focuses on the false positive errors in object detection, which are detections that do not correspond to the target category. The study focuses on three types of false positives: localization error, similar objects error, and confusion with the background in VOC 10-10, as described in~\cite{hoiem2012diagnosing}. To assess the performance of ABR, we compare it with other algorithms and show the pairwise statistics of false positives between ABR and other algorithms in~\cref{fig:fp}. 
% % The statistics represent the false positive difference between the other algorithm and ABR.
% % The paper mainly concentrates on these three types of errors and compares the proposed ABR algorithm with other algorithms to see its effectiveness in reducing false positives. 
% % The pairwise statistics of false positives between the ABR algorithm and other algorithms are shown in Figure~\cref{fig:fp}. 
% The statistics are the false positive difference between the other algorithm and the ABR algorithm. The results indicate that the ABR algorithm can significantly reduce background errors compared to other algorithms, including joint training. This is due to the mixup replay strategy used in ABR, which provides rich background information for both new and previous classes and improves the generalization ability of the overall model. The~\cref{fig:imageours} shows the performance of ABR in terms of localization and background errors on new categories and compares it with image playback methods. The results show that ABR outperforms image playback methods in terms of localization and background errors for new categories. This is because the augmented box replay strategy overcomes the problem of foreground drift caused by image replay.


% One major type of error is false positives, detection that do not correspond to the target category. There are different types of false positives which likely require different kinds of solutions. In this work, we mainly focus on three errors: localization error (loc), similar objects error (sim), confusion with background (bg), referencing to~\cite{hoiem2012diagnosing}. In~\cref{fig:fp}, we show the pairwise statistics of the false positives pairwise between our proposed ABR and different algorithms, which is the error-positive statistic of the other algorithm minus ours. Through observation, we can find that our algorithm can greatly reduce the generation of background errors, and even improves compared to joint training. This is due to the mixup playback strategy we adopted, which introduces rich background information beyond the original data for both new and previous classes, and improves the generalization ability of the overall model. ABR can achieve the better performance on new categories of localization and background errors than image replay methods. This is due to the replay technology of augmented box, which overcomes the problem of foreground drift caused by image replay.


\subsection{Visualization}
The inference results are presented in \cref{fig:addpred}, which demonstrate the effectiveness of our proposed ABR method in avoiding the forgetting of previous classes and improving adaptation to new classes. 
In the first two rows, our method is capable of accurately distinguishing new classes from similar classes in the previous classes, as seen in the detection of a \textit{bus} in the first row of images and a \textit{cow} in the second row of images. However, the popular MMA method misclassifies the \textit{bus} as a \textit{train} or \textit{bus} and the \textit{cow} as a \textit{dog} or \textit{cow}. In the third row, our algorithm successfully detects the new class, a \textit{dining table}, while also accurately locating a previous class, a \textit{chair}. In comparison to the MMA method, our method achieves more precise position detection, as demonstrated in the last two rows where \textit{person} and \textit{boat} are detected.

Overall, these results suggest that the proposed ABR method can more effectively handle the problem of incremental learning in object detection tasks, particularly in scenarios where new classes are similar to previous ones. The ability to avoid forgetting and adapt to new classes is crucial for practical applications, and the improved performance of our method is promising for future research in this area.



% \centering
% Figure environment removed





\nocite{hoiem2012diagnosing,cermelli2022modeling}
\end{document}
