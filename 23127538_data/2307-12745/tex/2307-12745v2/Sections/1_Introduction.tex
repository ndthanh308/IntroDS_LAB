\section{Introduction} \label{sec:intro}
We investigate representations of electroencephalogram (EEG) data obtained by self-supervised learning methods. Self-supervision is motivated by the lack of labeling in large-scale EEG datasets as labeling is both time-consuming and requires highly specialised EEG expertise.
Self-supervised models, such as BERT-inspired Neural Data Representations (BENDR) \cite{BENDR}, have the potential to overcome this challenge by learning informative representations from raw, unlabeled data. Such models can subsequently be fine-tuned for downstream classification tasks. We apply the Testing Concept Activation Vectors (TCAV) approach of Kim et al. \cite{kim2018interpretability}, an interpretability method introduced in 2018, to BENDR-based models, to provide insights into their structure and decision-making processes. See Figure \ref{fig:figure_1} for a conceptual overview.  A better understanding of EEG transformer models using TCAV could support the use of these models as diagnostic support tools for identifying EEG abnormalities, such as seizures. However, the question that arises is, what constitutes human-friendly concepts in this context? To address this, we present the following scientific contributions:
% \begin{itemize}
% \itemsep 0pt
% \parsep 0pt
%     \item The first TCAV workflows for EEG data with concepts based on human-annotated data and
%     based on spatiotemporal concepts defined by anatomy and frequency ranges.
%     \item Sanity checks for TCAV to ensure proper explanations in simple EEG settings.
%     \item Two proof-of-concept practical applications: one in seizure prediction and the other in brain-computer interfacing.
% \end{itemize}
%
\begin{itemize}
\itemsep 0pt
\parsep 0pt
    \item The first TCAV workflows for EEG data, proposing concepts based on human-annotated data as well concepts defined by human anatomy and EEG frequency ranges.
    \item Sanity checks for TCAV to ensure valid explanations in simple EEG settings.
    \item Two practical applications: seizure prediction and brain-computer interfacing.
\end{itemize}

All code used in this research, along with references to the datasets, have been made publicly accessible for validation and replication\footnote{\url{https://github.com/AndersGMadsen/TCAV-BENDR}}.

%
%\quad We will also investigate various transformer-based methods used in self-supervised learning within EEG modeling, highlighting BENDR a transformer-based learning architecture. The primary objective of this thesis is to uncover the decision structure of models based on BENDR and evaluate their interpretability using the TCAV technique. This analysis will help determine if the model's decisions align with the medical professionals' decision-making process when identifying abnormalities in EEG data relevant to medical diagnosis.