% Figure environment removed
%
\section{Results} \label{sec:results}
%
\subsection{Sanity Checks} \label{ssec:sanity_checks}
We first provide evidence that the TCAV method can be applied to explain EEG data and the LHB model. 
In Figure \ref{fig:results_sanity_check}, the high significance of class data as concepts (\textit{Left Fist Movement} with positive evidence and \textit{Right Fist Movement} with negative evidence) confirms this. Furthermore, concepts based on maximal activity in either the left or right hemisphere for the \emph{alpha} frequency band strongly indicate that lateralized cortical activity is detected by several layers in the model, as expected.

\quad Moreover, the negative alignment of a concept based on labeled artifacts with the model representation of motor task data implies that artifacts in EEG data significantly influence classification tasks. We find that \textit{eyem} has a negative impact on the classification of \textit{Left Fist Movement}. Note that this does \textit{not} mean that \textit{eyem} positively affects the opposite class, that is \textit{Right Fist Movement}, as the TCAV Score is specific to the "\textit{Left Fist Movement} dataset". Conversely, \textit{eyem} could negatively affect the classification of both \textit{Left Fist Movement} and \textit{Right Fist Movement}, due to the lower signal-to-noise ratio for classification when artifacts are present.

\subsection{Event-based concepts} \label{ssec:annotated_data_as_concepts}
We next investigate whether fine-tuning the LHB model for seizure classification on the TUSZ dataset and using explanatory concepts defined with labeled data from TUEV aligns with the model's internal representation for data labeled as containing seizures. The target of the investigation is the \textit{seizure} label and we test all bottlenecks in the LHB model. The results of this experiment are shown in Figure \ref{fig:results_seizure}.

\quad When compared to EEG data labeled as containing seizures, the epilepsy-related concepts \textit{pled}, which is present in certain brain areas, and \textit{gped}, which is present in most of the brain, exhibit high and positive evidence in nearly all bottlenecks. This observation aligns with existing literature that associates epileptiform discharges with seizures \cite{gajic2015detection}, and it is expected that the LHB model will use these properties for classification. The \textit{spsw} concept also demonstrates significant positive evidence in the \textit{encoder} bottleneck but not in the further downstream bottlenecks. Similarly, the \textit{bckg} concept shows negative evidence in the \textit{encoder} bottleneck but not in the further downstream bottlenecks. It is interesting that these concepts only come to be significant in the initial bottleneck.
 A possible explanation is that the technical artifacts \textit{artf} and \textit{bckg} are not significant for the classification, but BENDR effectively identifies seizure-related concepts and filters out noise. The results also suggest that the model's \textit{classifier} and \textit{extended classifier} can be further optimized, as \textit{artf} is near-significant level in these bottlenecks and, as a result, the noise has not been completely removed. In conclusion, these examples indicate that concept-based explainability can provide valuable model design information.

\subsection{Anatomy/Frequency-Based Concepts} \label{ssec:artefacts_as_concepts}
%
% Figure environment removed
%
We have demonstrated that labeled EEG data can generate human-aligned concepts, which are integrated into the LHB model for seizure classification. This comes quite naturally as labeled 
data is labeled by humans and tend to align with human-relatable concepts.
We then present evidence that defining explanatory concepts based on cortical activity in frequency bands may uncover patterns corresponding to the model's internal representations.

\quad In particular, for a motor classification task using the MMIDB EEG dataset and targeting the \textit{Left Fist Movement} class, we show that cortical activity in the \emph{alpha} band aligns with the model's internal representation. In Figure \ref{fig:results_lateralization}, we find that the CAV for \textit{Somatosensory and Motor Cortex} in the right hemisphere positively aligns with the activations of \textit{Left Fist Movement} class data across all bottlenecks in the model. The mean TCAV scores are also consistently positively significant. At the same time, the TCAV scores for the same cortical area in the \textit{Left Hemisphere} are either negatively significant or insignificant. These results strongly suggest that the model's internal representation incorporates lateralization, reflecting the fact that one hemisphere exhibits more electrical activity than the other. It is noteworthy that lateralization is most significant in the \textit{Encoding Augment} and \textit{Summarizer} bottlenecks, indicating that it is captured early in the network.

\quad Additionally, we observe that the \textit{Primary Visual Cortex (V1)} areas do not exhibit lateralization, and their TCAV scores are insignificant across all bottlenecks and for both hemispheres. This further supports the conclusion that the LHB model utilizes specific cortical areas in its classification rather than all areas indiscriminately.

\quad While no apparent lateralization is present in the \textit{Premotor Cortex}, this part of the cortex is negatively significant in the \textit{Encoder} and \textit{Summarizer} bottlenecks for both the left and right hemispheres. A possible explanation is that the instances we examine involve participants \textit{performing} movements; therefore, there may not necessarily be relevant activity in the \textit{Premotor Cortex}, which is primarily involved in movement planning \cite{gallego2022going}.

\quad Lastly, we observe significance in the \textit{Classifier} bottleneck for \textit{Early Visual Cortex} and \textit{Dorsal Stream Visual Cortex}. We note that the movement is activated by a visual cue; however, further experiments would be required to fully clarify the effect.