\section{Conclusion} \label{sec:conclusion}
Concept-based explainability has proven to be valuable in various domains, such as image classification and natural language understanding, where concepts are naturally defined using labeled data. In this study, we have explored the definition of concepts for EEG models for the first time. We presented two new workflows for concept-based explainability within the TCAV framework for EEG data. First, we adopted an approach akin to the original work of Kim et al.\ \cite{kim2018interpretability}, in which concepts are derived from labeled data. In this case, we utilized various annotated EEG databases, e.g., data from the Temple University Hospital EEG database. The second workflow is based on the source location of resting-state EEG data also from the Temple University Hospital database. This enables us to generate datasets for TCAV derived from anatomical  brain areas and for specific frequency bands, e.g., the \emph{alpha} band.
We demonstrated a proof of concept through several "sanity check" experiments to verify expected responses in elementary EEG settings, such as EEG lateralization during left- or right-hand movement. Lastly, we examined two practical applications: A case study involving seizure prediction, where TCAV reveals the role of fundamental spike patterns, and a brain-computer interface case, hinting at how the TCAV method can assist in debugging and offer valuable insights into classifier design for EEG data.
