% LaTeX template for MLSP papers. To be used with:
%   * mlspconf.sty - ICASSP/ICIP LaTeX style file adapted for MLSP, and
%   * IEEEbib.bst - IEEE bibliography style file.
\documentclass{article}
\usepackage{amsmath, bm, graphicx, mlspconf, glossaries}
\usepackage{blindtext}
\setlength\parindent{0pt}
\usepackage{hyperref}
\usepackage{afterpage}
\usepackage{multicol}
\usepackage{float}
\usepackage{epstopdf}
\usepackage{xcolor}


\makeatletter
\renewcommand{\section}{\@startsection{section}{1}{\z@}%
  {-2.0ex \@plus -0.5ex \@minus -.2ex}%
  {1.50ex \@plus.2ex \@minus-.2ex}%
  {\normalfont\large\bfseries\centering}}
  
\renewcommand{\subsection}{\@startsection{subsection}{2}{\z@}%
  {-1.50ex\@plus -1ex \@minus -.2ex}%
  {0.50ex \@plus .2ex}%
  {\normalfont\normalsize\bfseries}}
\makeatother

% Copyright notices.
% ------------------

% * For papers in which all authors are employed by the European Union:
% \copyrightnotice{979-8-3503-2411-2/23/\$31.00 {\copyright}2023 European Union}

% * For all other papers:
\copyrightnotice{979-8-3503-2411-2/23/\$31.00 {\copyright}2023 IEEE}

% Header
\toappear{2023 IEEE International Workshop on Machine Learning for Signal Processing, Sept.\ 17--20, 2023, Rome, Italy}

% Title.
% ------
\title{Concept-based explainability for an EEG transformer model}

% Double-blind peer review.
% -------------------------
% Anonymize your paper for the double-blind peer-review process using the following author and affiliation.


% Single address.
% ---------------
%\name{Author(s) Name(s)\thanks{Thanks to XYZ agency for funding.}}
%\address{Author Affiliation(s)}

% For example:
% ------------
%\address{%
%    School \\
%    Department \\
%    Address
%}
%
% Two addresses.
% --------------
%\twoauthors{%
%    A. Author-one, B. Author-two\sthanks{Thanks to XYZ agency for funding.}
%}{%
%    School A-B \\
%    Department A-B \\
%    Address A-B \\
%    Email A-B
%}{%
%   C. Author-three, D. Author-four\sthanks{The fourth author performed the work while at ...}
%}{%
%    School C-D \\
%    Department C-D \\
%    Address C-D \\
%    Email C-D
%}
% 
% Two or more addresses (alternative form).
% -----------------------------------------
% If you need to list more than 2 authors or the option for two options above 
% produces a poor author block, please use the following structure:

%\name{Anonymous\thanks{Anonymous.}}
%\address{Anonymous}

\name{%
    \fontsize{10.5pt}{11pt}\selectfont
    \begin{tabular}{c}
    Anders Gjølbye Madsen$^{\star\dagger}$ \qquad William Theodor Lehn-Schiøler$^{\star\dagger}$ \\
    Áshildur Jónsdóttir$^{\star}$ \qquad Bergdís Arnardóttir$^{\star}$ \qquad Lars Kai Hansen$^{\star}$
    \end{tabular}
    \thanks{This work is supported by The Pioneer Centre for AI, DNRF grant number P1, The Novo Nordisk Foundation grant NNF22OC0076907 "Cognitive spaces - Next generation explainability", and travel grants from The Danish Data Science Academy awarded to AGM and WLS.}
    \vspace{-3pt}
}
\address{%
    \small
    \begin{tabular}{c}
    \begin{tabular}{@{}c@{}}
    $^{\star}$Technical University of Denmark \\
    Department of Applied Mathematics and Computer Science \\
    2800 Kgs. Lyngby, Denmark
    \end{tabular}
    \qquad \qquad
    \begin{tabular}{@{}c@{}}
    $^{\dagger}$BrainCapture \\
    2800 Kgs. Lyngby, Denmark
    \end{tabular}
    \end{tabular}
    \vspace{-11pt}
}

\begin{document}
\ninept

\maketitle

\begin{abstract}
Deep learning models are complex due to their size, structure, and inherent randomness in training procedures. Additional complexity arises from the selection of datasets and inductive biases. Addressing these challenges for explainability, Kim et al. (2018) introduced Concept Activation Vectors (CAVs), which aim to understand deep models' internal states in terms of human-aligned concepts. These concepts correspond to directions in latent space, identified using linear discriminants. Although this method was first applied to image classification, it was later adapted to other domains, including natural language processing.
In this work, we attempt to apply the method to electroencephalogram (EEG) data for explainability in Kostas et al.'s BENDR (2021), a large-scale transformer model. A crucial part of this endeavour involves defining the explanatory concepts and selecting relevant datasets to ground concepts in the latent space. Our focus is on two mechanisms for EEG concept formation: the use of externally labelled EEG datasets, and the application of anatomically defined concepts. The former approach is a straightforward generalization of methods used in image classification, while the latter is novel and specific to EEG.
We present evidence that both approaches to concept formation yield valuable insights into the representations learned by deep EEG models.
\end{abstract}

\begin{keywords}
Explainable AI, EEG Concepts, TCAV, BENDR
\end{keywords}

%%%%%%%%%%%%%%%%%%%%%%%%%%%%%%%%%%%%%%%%%%%%%%%%%%%%%%%%%%%%%%%%%%%%%%%%%%%%%%%%
\section{Introduction}

Autonomous driving (AD) %with deep learning networks 
has shown promising achievements and is considered an important technological breakthrough that could revolutionize the future of transportation. Currently, ensuring the safety of autonomous driving systems has become a topic of extensive development.
% There has been much discussion on how to verify the safety of autonomous driving systems.
One traditional solution for safety tests is to exhaustively enumerate real scenarios for validation. Nevertheless, this process is not only labor-intensive and costly but also dangerous. Simulation has emerged as a robust, safe, and efficient alternative for training and evaluating AD software and algorithms~\cite{li2019aads, amini2020learning, amini2022vista}.

% Figure environment removed

Recently, neural radiance field (NeRF)~\cite{mildenhall2020nerf} has gained significant attention in AD simulation~\cite{drivesim}. This approach leverages multi-view images to construct a 3D scene and enable novel view synthesis for both indoor and outdoor applications. When it comes to constructing NeRF models in AD simulation, there are two options available: 1) collecting a large amount of data to cover as many viewpoints as possible, and constructing a fine-grained scene offline; 2) directly using log data from road tests to quickly create an environment and dynamically simulate driving scenarios. The first choice can deliver high-quality simulation~\cite{tancik2022block} by transforming the problem of view extrapolation into view interpolation through the use of large amounts of data. However, it is time- and cost-intensive, which makes it challenging to generalize. As for the second choice, the collected images from log data are usually similar to each other along the running trajectory, which may result in unsatisfactory outcomes, particularly when the camera pose is placed out-of-trajectory (see \figref{figSupportComp} as an example), semantic consistency cannot be guaranteed when synthesizing images from deviated views. We observe this problem under this data condition in all neural radiance approaches, and to the best of our knowledge, none of the existing work has solved this issue.
In our opinion, semantic consistency is crucial for AD simulation, and synthesizing on deviated views is unavoidable for scalability.

AD simulation usually involves map data for planning and control, which can be obtained from a prebuilt High-Definition Map (HD Map) or an online mapping module. While the map data may not be pixel-perfect, it can provide semantic-level information that is useful for enhancing the semantic consistency of the trained neural radiance field.
In this paper, we propose incorporating map priors into neural radiance fields to enhance the semantic consistency and rendering quality of deviated driving view synthesis. Firstly, we employ ground information from maps to supervise the density field of NeRF, providing a more reliable road base for semantic entities. Next, we propose sampling rays to simulate unseen views. Unlike most NeRF augmentation methods~\cite{zhang2022ray, chen2022geoaug}, we utilize ground and lane information in sampling computations to guide the radiance field. More importantly, we model the above two supervision methods as weak supervision by using an uncertainty parameter and propose an uncertainty tempering scheme to increase the uncertainty. This ensures that map priors only guide the training process rather than enforce it towards their absolute values. As a result, our proposed method not only improves the rendering quality of interpolated novel view synthesis quantitatively but also enhances the semantic consistency of deviated novel view synthesis. 
Our contributions can be summarized as follows:
% We summarize the contributions of this paper as follows.



% To overcome the limitations of the collected data, this paper proposes a novel approach that leverages map information to enhance the semantic consistency of the synthesized driving views. 

% Autonomous driving (AD) vehicles are being trained with the help of deep learning networks and have shown promising achievements. This technology is considered to be a breakthrough that could change the way of transportation in the near future. However, there are many discussions on how to verify or judge the safety of autonomous driving systems.
% A straightforward solution towards the safety tests is to exhaustively enumerate real scenarios for validation as many as possible. However, the process of implementing different real scenarios is not only labor-intensive and costly, but also dangerous. Simulation has been proved to be an alternative, which is robust, safe, efficient in training, and evaluating AD software and algorithms.
% Now, the emerging technology of neural radiance field (NeRF)~\cite{} leverages multi-view images to construct a 3D scene and enable novel view synthesis for many indoor and outdoor applications. For AD simulation, there are two choices for constructing NeRF models: 1) collect a large amount of data, such as LiDAR and camera data, similar to mapping, to construct a fine-grained scene offline; or 2) directly use the log file (typically in the format of ROS bag) to rapidly create an environment and then dynamically simulate the driving scenarios.
% The first choice can achieve high-quality simulation, but it is time-consuming and expensive, making it difficult to generalize to very large scales. On the other hand, the second option is fast but can lead to low-quality simulation due to the data being sparse and similar to each other in log data. This paper tackles the problem raised by choosing the latter option and attempts to improve the quality of out-of-trajectory driving view synthesis by incorporating map information. This approach is practical for many autonomous driving tests.
% In conclusion, the use of NeRF technology for AD simulation is a promising avenue for training and evaluating AD software and algorithms. While both options for constructing NeRF models have their pros and cons, this paper addresses the challenges of the second option and proposes a potential solution to improve the quality of simulation.

%There exist a few attempts to facilitate training a NeRF model for synthesizing out-of-trajectory (or called as extrapo trajectory) views.


\begin{itemize}
    \item We propose a novel method to incorporate commonly used map priors in AD scenes into neural radiance fields to improve the out-of-trajectory driving view synthesis.
    \item We explicitly model the uncertainty in map priors as a parameter and propose an uncertainty tempering scheme to guide the training process of the neural radiance field.
    \item Experiments demonstrated that the proposed method can improve the semantic consistency of out-of-trajectory views and the rendering quality of novel view trajectory interpolation.
\end{itemize}

Our proposed method is easy to implement, can be easily plugged into existing NeRF algorithms, and has the capability of extending to other formats of priors.
\section{Theory} \label{sec:theory}
%
% Figure environment removed
%
%
\subsection{BERT-inspired Neural Data Representations}
\label{ssec:bendr}
BENDR \cite{BENDR} is inspired by language modeling techniques that have found success also outside text analysis, in self-supervised end-to-end speech recognition and image recognition. It aims to develop EEG models for better brain-computer interface (BCI) classification, diagnosis support, and other EEG-based analyses. Importantly, the approach being based on self-supervision can learn from any EEG data using only unlabeled data.
The main goal of BENDR is to create self-supervised representations with minimal robust to context boundaries  like datasets and  human subjects. The approach is expected to be transferable to future unseen EEG datasets recorded from unseen subjects, different hardware, and different tasks. It can be used as-is or fine-tuned for various downstream EEG classification tasks.

\quad The architecture is based on \verb|wav2vec 2.0| \cite{wave2vec} developed for speech processing and  consists of two stages. The first stage takes raw data, and down-samples it using a stack of short-receptive field 1D convolutions, resulting in a sequence of vectors called BENDR. The second stage uses a transformer encoder \cite{vaswani2017attention} to map BENDR to a new sequence related to the target task. Down-sampling is achieved through strides, and the transformer follows the standard implementation with some modifications. The entire sequence is then classified, with a fixed token implemented as the first input for downstream tasks \cite{BERT}.
BENDR differs from the speech-specific architecture  in two ways: (1) BENDR is not quantized for pre-training targets, and (2) it has many incoming channels, unlike \verb|wav2vec 2.0| which uses quantization and is based on a single channel of raw audio. The 1D convolutions are preserved in BENDR, to reduce complexity. We note that BENDR down-samples at a lower factor than \verb|wav2vec 2.0|, here resulting in an effective sampling rate of $\approx 2.67$ Hz equivalent to a feature window of $\approx 375$ ms.

%\subsubsection{Downstream architecture}
\subsection{Linear Head BENDR}
For downstream fine-tuning, we use a version where the pre-trained transformer modules are ignored, such that the pre-trained convolutional BENDR stage is used as representation, see \cite{BENDR}. A consistent-length representation is created by dividing the BENDRs into four contiguous sub-sequences, averaging each sub-sequence, and concatenating them. A new linear layer with softmax activation is added to classify the downstream targets based on this concatenated vector of averaged BENDR. We call this the Linear Head BENDR (LHB) model and the structure is illustrated in Figure \ref{fig:linear_head_bendr}. 

The final LHB architecture consists of the following components:
\begin{enumerate}
    \item \textbf{Feature encoder:} Fine-tunes the pre-trained parameters and uses six convolution blocks, each containing a temporal convolution, group normalization, and a GELU activation function to produce a BENDR of length 512.
    \item \textbf{Encoding augment:} Involves masking and contextualizing the BENDR, with 10\% of the BENDR masked and 10\% of the channels dropped, while relative positional embeddings from the pre-trained task are added to the BENDR and further preprocessed.
    \item \textbf{Summarizer:} Applies adaptive average pooling to create four contiguous sub-sequences, averaging each sub-sequence to ensure the model's independence from the input length of EEG recordings.
    \item \textbf{Extended classifier:} Flattens the four sub-sequences, passes them through a fully connected layer to reduce their dimension, applies a dropout layer, uses a ReLU activation function, and normalizes the output using batch normalization.
    \item \textbf{Classifier:} Consists of a linear layer with a softmax activation function, which performs the classification task.
\end{enumerate}

% \begin{enumerate}
%     \item \textbf{Feature encoder:} Fine-tunes the pre-trained parameters and uses six convolution blocks, each containing a 1D temporal convolution with 512 filters, group normalization, and GELU activation function to produce BENDR of length 512.
%     \item \textbf{Encoding augment:} Involves masking and contextualizing the BENDR, with 10\% of the BENDR masked and 10\% of the channels dropped, while relative positional embeddings from the pre-trained task are added to the BENDR and further preprocessed.
%     \item \textbf{Summarizer:} Applies 1D adaptive average pooling to create four contiguous sub-sequences, averaging each sub-sequence to ensure the model's independence from the input length of EEG recordings.
%     \item \textbf{Extended classifier:} Flattens the four sub-sequences, passes them through a fully connected layer to reduce their dimension, applies a dropout layer, uses a ReLU activation function, and normalizes the output using batch normalization.
%     \item \textbf{Classifier:} Consists of a linear layer with a softmax activation function, which performs the classification task.
% \end{enumerate}
%
% Figure environment removed
%
\subsection{Testing with Concept Activation Vectors (TCAV)}
\label{ssec:tcav}
%
Testing with Concept Activation Vectors (TCAV) is a technique used to quantify the degree to which layers of neural networks align with human-defined concepts \cite{kim2018interpretability}. The method is general in the sense that it is not confined to the particular structure of the network nor to the data type. In its essence, TCAV can be broken down into five steps

\quad First, the process involves defining human-aligned concepts and representing them in the data. Alongside these, data from the target class must also be present for evaluation purposes. Furthermore, to establish the directions of the concept activation vector in the latent space, it is necessary to have a collection of concept-negative or random examples.

\quad Second, the layer activations of the concept input and the random input, respectively, are collected and separated by training 
a binary linear classifier. Then, the concept activation vector, $v_c^l$ is defined as the normal vector to the hyperplane that separates the two classes (concept vs. random).

\quad Third, for a layer $l$ in the network, the directional derivatives for the target class $k$ along the
learned activation vector for concept $C$ is used to calculate how sensitive the prediction of the network is to changes in the input data in the direction of $C$. We can quantify the sensitivity by
\begin{equation}
    S_{C,k,l}(\bm{x}) = \nabla h_{l,k} (f_l(\bm{x}))\cdot \bm{v}_C^l,
\end{equation}
where $h_{l,k}$ is defined as the function that maps activations in layer $l$ through the remaining network and predicts class $k$.

\quad Fourth, computing the sensitivity for several target examples, $\bm{x} \in X_k$, the TCAV score is defined as the ratio of examples that have positive sensitivity, i.e.,
\begin{equation}
    \vspace{-3pt}
    \text{TCAV}_{C,k,l} = \frac{|\{\bm{x} \in X_k : S_{C,k,l}(\bm{x}) > 0 \}|}{|X_k|}.
\end{equation}
In this way, concept activation vectors that are positively aligned with target activations have a TCAV score close to 1 and concept activation vectors that are negatively aligned with target activations have a TCAV score close to 0.

\quad Fifth and final, collecting samples of TCAV scores over several training runs, a suitable statistical test is used to assess the statistical significance of concept activation vectors aligning with the activation of target examples. The null hypothesis of the test is that half of the examples have positive sensitivity and the other half have negative or zero sensitivity, i.e.,
%
\begin{equation}
    H_0: \text{TCAV}_{C,k,l} = 0.5.
\end{equation}
%
Concepts $C$ for which the null hypothesis is rejected thus relate to the target class prediction, and may bring positive or negative evidence for the given target $k$.
%
\subsection{Source localization}
\label{ssec:sourcelocalisation}
Source localization for EEG data involves mapping electrical signals recorded on the scalp surface to corresponding regions on the cortical surface of the brain. This process uses a head model and the EEG data collected from electrodes placed on the scalp. The reconstruction is a grid of dipolar sources. The solution to this ill-posed problem is called the lead field and there exist many different ways to obtain this solution. In this work, we use the exact low-resolution electromagnetic tomography (eLORETA) method implemented in the MNE library \cite{doi:10.1098/rsta.2011.0081}.

\quad The eLORETA approach presupposes that the EEG measurements of the electric field present on the scalp reflect dipolar sources located in the cerebral cortex. These are conceptually modeled as a three-dimensional distribution of dipoles. The spatial resolution of eLORETA is relatively coarse, which can make pinpointing exact cortical sources challenging. However, for our purpose of estimating aggregated source activity over broadly defined brain regions, such reduced resolution is not an issue.

%\quad eLORETA assumes that the EEG measurements of the quasi-static electric field on the scalp reflect dipolar sources in the cortex, modeled as a three-dimensional distribution of dipoles. It is worth noting that the spatial resolution of eLORETA is relatively low, making it difficult to precisely locate cortical sources. However, as we are simply interested in reconstructing aggregate sources  across extended cortical areas, the comparatively low resolution is not a problem.

\section{Methods} \label{sec:methods}
\subsection{Data} \label{ssec:data}

EEG is a non-invasive technique to record the brain's electrical activity. EEG data in this paper refers to these measurements, used often in research and healthcare to identify neurological conditions. In this work, we use five publicly accessible datasets, namely TUH EEG Corpus \cite{tuheeg}, TUH EEG Artifact (TUAR) Corpus, TUH EEG Events (TUEV) Corpus, TUH EEG Seizure (TUSZ) Corpus \cite{tuhz} and the EEG Motor Movement/Imagery (MMIDB) Dataset \cite{mmidb}.

\quad The TUH EEG Corpus contains 69,652 clinical and unlabeled EEG recordings obtained from Temple University Hospital (TUH). The TUH EEG Artifact Corpus, a labeled subset of the TUH EEG Corpus, includes annotations for five distinct artifacts including eye movement artifact (\emph{eyem}). The TUEV is a subset of the TUH EEG Corpus and includes annotations of event-based EEG segments. There are numerous categories, but we primarily focus on five key classes: (1) technical artifacts (\textit{artf}), (2) background (\textit{bckg}), (3) generalized periodic epileptiform discharge (\textit{gped}), (4) periodic lateralized epileptiform discharge (\textit{pled}), and (5) spike and slow wave (\textit{spsw}). The TUSZ contains EEG signals with manually annotated data for seizure events.

\quad The MMIDB EEG dataset consists of data from 109 participants who are performing or imagining specific motor tasks; our main interest is the moments when subjects either close or imagine closing their left or right fist following a visual cue. We are excluding participants S088, S090, S092, and S100 due to missing data, resulting in 105 participants.

%\subsubsection{Resting-State Dataset} \label{sssec:resting}
\quad In the construction of brain anatomy concepts, it is imperative to obtain an extensive collection of resting-state EEG data. Due to the limited availability of public datasets with the requisite size and reliability, we utilized The TUH EEG Corpus and source localization to develop a dedicated anatomically labeled resting-state dataset. A set of predefined criteria were employed, including the number of EEG channels, minimum duration, minimum sampling frequency, scaling, and the exclusion of extreme values, which led to the elimination of approximately 90\% of the initial EEG recordings. Following this, a manual examination of a part of the remaining data was performed, ultimately yielding 200 human-verified resting-state EEG recordings, corresponding to an aggregate of about 70 hours of EEG data.

%\subsection{Preprocessing of data}
\quad In the process of downstream fine-tuning and concept formation, we employ 19 EEG channels, namely $\textit{Fp1}$, $\textit{Fp2}$, $\textit{F7}$, $\textit{F3}$, $\textit{Fz}$, $\textit{F4}$, $\textit{F8}$, $\textit{T7}$, $\textit{C3}$, $\textit{Cz}$, $\textit{C4}$, $\textit{T8}$, $\textit{T5}$, $\textit{P3}$, $\textit{Pz}$, $\textit{P4}$, $\textit{T6}$, $\textit{O1}$, and $\textit{O2}$ (see the MNE documentation \cite{doi:10.1098/rsta.2011.0081} for more information). These channels originate from the initial pre-training of BENDR using The TUH EEG Corpus. In instances where the datasets lack these channels, we establish the following mapping: $T3 \mapsto T7$, $T4 \mapsto T8$, $P7 \mapsto T5$, and $P8 \mapsto T6$. We also resample the corresponding EEG data to a 256 Hz sampling frequency and apply a high-pass FIRWIN filter with a 0.1 Hz cutoff, a low-pass FIRWIN filter with a 100.0 Hz cutoff, and a 60 Hz FIRWIN notch filter to eliminate powerline noise. In situations where preprocessing cannot be performed, the EEG is excluded. Finally, we scale each trial to the range $[-1, 1]$ and append a relative amplitude channel, see \cite{BENDR}, resulting in a total of 20 channels.

%\subsection{Training of BENDR}

\subsection{Training}
Pre-training of BENDR is based on the large set of unlabelled EEG data from The TUH EEG Corpus. The pre-training procedure is largely based on \verb|wav2vec 2.0| and involves two main stages: The convolutional stage and the transformer stage. The convolutional stage generates a sequence of representations (BENDRs) that summarize the original input. This sequence is then fed into the transformer stage, which adjusts its output to be most similar to the encoded representation at each position. The layers affected during pre-training are the feature encoder and the transformer. Kostas et al.\ \cite{BENDR} kindly made the pre-trained weights of the encoder and contextualizer publicly available, and this is the model that we have employed here.
%\footnote{Pre-trained model weights:\\ \url{https://github.com/SPOClab-ca/BENDR/releases/tag/v0.1-alpha}}.

%\subsubsection{Downstream fine-tuning}
\quad The LHB model architecture described in Figure \ref{fig:linear_head_bendr} is used for downstream fine-tuning. We aim to optimize the model for two distinct binary classification objectives. First, the model is fine-tuned for the differentiation between \emph{seizure} and \emph{non-seizure} events, using the TUSZ Corpus with 60-second window segments. The hyperparameters are determined using Bayesian optimization to maximize the validation $F_1$-score. The fine-tuning employs a batch size of 80, a learning rate of $1 \times 10^{-4}$, and $30$ epochs. This results in a model with a balanced accuracy of $0.73 \pm 0.07$. 

\quad In our second fine-tuning example, the model is adapted for the differentiation between \emph{Left Fist Movement} versus \emph{Right Fist Movement}, using the MMIDB EEG Dataset with 4-second window segments. We are using both the imaginary and performed task data from the 105 participants. We train the model for $7$ epochs with a batch size of $4$ and a learning rate of $1 \times 10^{-5}$. The hyperparameters were chosen based on the best validation balanced accuracy from leave-one-subject-out cross-validation where the model was trained for 50 epochs and the best model was retained. The specific hyperparameter configuration aligns with the optimal hyperparameters found by the original authors \cite{BENDR} and we find a similar balanced accuracy of $0.83 \pm 0.02$.


\subsection{Constructing Concepts}
\label{subsec:explanatory_concepts}

To construct human-aligned explanatory EEG concepts, a number of initial investigations were conducted. The data processing involved follows the methodology previously mentioned. In this section, we provide a general pipeline overview and discuss several choices made throughout the process.

\vspace{0.5em}
{\bf Concepts from Labeled EEG Data}: Using the labeled EEG data from the TUAR and TUEV Corpus and the MMIDB EEG Dataset, we create concepts representing activities within specific time windows. Each annotated segment of the EEG data is divided into windows of predetermined length and assigned the corresponding label.

\quad In the TUEV Corpus, we define concepts for the spike/short wave (\textit{spsw}), periodic lateralized epileptic discharge (\textit{pled}), general period epileptic discharge (\textit{gped}), technical artifact (\textit{artf}), and background (bckg) with 60-second windows. This approach aligns with the length of the \textit{seizure} classifier.

\quad Lastly, we examine the eye movement (\textit{eyem}) from the TUAR Corpus and \textit{Left Fist Movement} and \textit{Right Fist Movement} from the MMIDB EEG Dataset, both using 4-second windows. These different-sized windows then constitute examples of concepts defined based on their labels.


\vspace{0.5em}
{\bf Anatomical Concepts from Unlabeled EEG Data}:
The objective is to identify concepts representing specific frequency bands within distinct areas of the cortex, e.g. \emph{alpha activity in pre-motor cortex} or \emph{gamma activity in early visual cortex}. To obtain a non-task-specific representation of each cortical area, we utilize resting-state EEG data, as it spontaneously generates activity throughout the cortex. For this purpose, we use a subset of The TUH EEG Corpus, as described above.

\quad To define anatomical concepts, EEG data is segmented into 4-second windows, with the first and last 5 seconds of each sequence  excluded to minimize artifact contamination. The data is then divided into five frequency bands with a FIRWIN bandpass filter: \emph{delta} (0.5-4Hz), \emph{theta} (4-8Hz), \emph{alpha} (8-12Hz), \emph{beta} (12-30Hz), and \emph{gamma} (30-70Hz). The inverse operator for the forward model is computed using eLORETA \cite{doi:10.1098/rsta.2011.0081} via the MNE Python library. Since the spatial resolution is not critical, minimal regularization of $1 \times 10^{-4}$ is applied.

\quad Using the combined version of the multi-modal parcellation of the human cerebral cortex, HCPMMP1 \cite{f8095709e11547daa07262682e1545f2} and the inverse operator, the average power of electrical activity in 23 cortical areas for each hemisphere is determined.

\quad Our interest lies in cortical areas exhibiting the greatest deviation from typical activity within a specific frequency band. However, cortical areas are not equidistant from the scalp or consistent in baseline activity across bands. To normalize for these differences in the distribution of cortical activity, we compute the mean and standard deviation of the power in each cortical area for each frequency band on an EEG session level, which will be employed in various ways. We call these the baseline mean and the baseline standard deviation.

\quad We explore possible approaches to how the baseline means and standard deviation for each EEG session could be used to normalize the power of 4-second windows within that session. The options include dividing by the baseline standard deviation to account for scalp source variation, subtracting or dividing by the baseline mean to identify the cortical area with the greatest deviation, taking the absolute difference or not, and selecting a single cortical area across all frequency bands or only within a specific band.

\quad Identifying a single frequency and cortical area for each 4-second window of EEG data is a challenging task without prior work to guide the process, and each method presents its own limitations. We specifically look for \emph{alpha} desynchronization in the cerebral cortex during imagined or actual movement and closed or open eyes in the MMIDB EEG dataset, i.e., that \emph{alpha} activity in cortical areas decreases when activated.
Using a paired t-test to examine the presence of lateralization in cortical activities for different methods, we found that the preferred approach is to choose the area which maximizes the absolute difference between the given time window's power and the baseline mean, divided by the baseline standard deviation, only within specific frequency bands.

\vspace{0.5em}
{\bf Random Concepts:} Construction of CAVs calls for data examples that are considered random with respect to the concept of interest. In all experiments, random concepts consisting of 4-second or 60-second windows were randomly sampled from resting-state data obtained from the subset of the TUH EEG Corpus and unannotated sections of the TUAR dataset. 

 \subsection{Experiments} \label{subsec:experiments}
We investigate two approaches for defining explanatory concepts in EEG data. The TCAV method is then employed to evaluate whether the LHB model uses specifically defined human-aligned concepts of EEG data. For all concepts, the resulting activation vectors for all five bottlenecks in the LHB model architecture are examined to determine if they significantly align with the latent representations of class data in the model. We conduct the following experiments:
% Figure environment removed
%
\begin{enumerate}
\item \textbf{Sanity Checks:} We verify the TCAV method and construction of concepts function as intended through a series of sanity checks when classifying \textit{Left Fist Movement}.
\item \textbf{Event-based Concepts:} We assess whether the LHB model leverages specific EEG events in the classification of \textit{seizure}.
\item \textbf{Anatomy/Frequency-based Concepts:} We investigate if the LHB model employs lateralization in cortical activity in the \emph{alpha} band for classifying \textit{Left Fist Movement}. The chosen cortical areas are based on their relevance to the classification task.
\end{enumerate}
%
In the experiments, we use the TCAV method with a regularized linear model and stochastic gradient descent (SGD) learning, setting the regularization parameter $\alpha = 0.1$ to learn the decision boundary between explanatory and random concepts. We employ 50 random concepts and a maximum of 40 examples per concept. These parameters were chosen to increase statistical power. The mean TCAV scores for the target concept examples and the random examples are compared using the non-parametric Mann-Whitney U Rank test, as opposed to the t-test used in the original TCAV method, as we observed a clear violation of the normality assumption for the TCAV scores. To mitigate Type I errors, the p-values are corrected for each experiment employing the conservative Bonferroni method, after which we claim significance if the corrected p-value is below $0.05$.


% Figure environment removed
%
\section{Results} \label{sec:results}
%
\subsection{Sanity Checks} \label{ssec:sanity_checks}
We first provide evidence that the TCAV method can be applied to explain EEG data and the LHB model. 
In Figure \ref{fig:results_sanity_check}, the high significance of class data as concepts (\textit{Left Fist Movement} with positive evidence and \textit{Right Fist Movement} with negative evidence) confirms this. Furthermore, concepts based on maximal activity in either the left or right hemisphere for the \emph{alpha} frequency band strongly indicate that lateralized cortical activity is detected by several layers in the model, as expected.

\quad Moreover, the negative alignment of a concept based on labeled artifacts with the model representation of motor task data implies that artifacts in EEG data significantly influence classification tasks. We find that \textit{eyem} has a negative impact on the classification of \textit{Left Fist Movement}. Note that this does \textit{not} mean that \textit{eyem} positively affects the opposite class, that is \textit{Right Fist Movement}, as the TCAV Score is specific to the "\textit{Left Fist Movement} dataset". Conversely, \textit{eyem} could negatively affect the classification of both \textit{Left Fist Movement} and \textit{Right Fist Movement}, due to the lower signal-to-noise ratio for classification when artifacts are present.

\subsection{Event-based concepts} \label{ssec:annotated_data_as_concepts}
We next investigate whether fine-tuning the LHB model for seizure classification on the TUSZ dataset and using explanatory concepts defined with labeled data from TUEV aligns with the model's internal representation for data labeled as containing seizures. The target of the investigation is the \textit{seizure} label and we test all bottlenecks in the LHB model. The results of this experiment are shown in Figure \ref{fig:results_seizure}.

\quad When compared to EEG data labeled as containing seizures, the epilepsy-related concepts \textit{pled}, which is present in certain brain areas, and \textit{gped}, which is present in most of the brain, exhibit high and positive evidence in nearly all bottlenecks. This observation aligns with existing literature that associates epileptiform discharges with seizures \cite{gajic2015detection}, and it is expected that the LHB model will use these properties for classification. The \textit{spsw} concept also demonstrates significant positive evidence in the \textit{encoder} bottleneck but not in the further downstream bottlenecks. Similarly, the \textit{bckg} concept shows negative evidence in the \textit{encoder} bottleneck but not in the further downstream bottlenecks. It is interesting that these concepts only come to be significant in the initial bottleneck.
 A possible explanation is that the technical artifacts \textit{artf} and \textit{bckg} are not significant for the classification, but BENDR effectively identifies seizure-related concepts and filters out noise. The results also suggest that the model's \textit{classifier} and \textit{extended classifier} can be further optimized, as \textit{artf} is near-significant level in these bottlenecks and, as a result, the noise has not been completely removed. In conclusion, these examples indicate that concept-based explainability can provide valuable model design information.

\subsection{Anatomy/Frequency-Based Concepts} \label{ssec:artefacts_as_concepts}
%
% Figure environment removed
%
We have demonstrated that labeled EEG data can generate human-aligned concepts, which are integrated into the LHB model for seizure classification. This comes quite naturally as labeled 
data is labeled by humans and tend to align with human-relatable concepts.
We then present evidence that defining explanatory concepts based on cortical activity in frequency bands may uncover patterns corresponding to the model's internal representations.

\quad In particular, for a motor classification task using the MMIDB EEG dataset and targeting the \textit{Left Fist Movement} class, we show that cortical activity in the \emph{alpha} band aligns with the model's internal representation. In Figure \ref{fig:results_lateralization}, we find that the CAV for \textit{Somatosensory and Motor Cortex} in the right hemisphere positively aligns with the activations of \textit{Left Fist Movement} class data across all bottlenecks in the model. The mean TCAV scores are also consistently positively significant. At the same time, the TCAV scores for the same cortical area in the \textit{Left Hemisphere} are either negatively significant or insignificant. These results strongly suggest that the model's internal representation incorporates lateralization, reflecting the fact that one hemisphere exhibits more electrical activity than the other. It is noteworthy that lateralization is most significant in the \textit{Encoding Augment} and \textit{Summarizer} bottlenecks, indicating that it is captured early in the network.

\quad Additionally, we observe that the \textit{Primary Visual Cortex (V1)} areas do not exhibit lateralization, and their TCAV scores are insignificant across all bottlenecks and for both hemispheres. This further supports the conclusion that the LHB model utilizes specific cortical areas in its classification rather than all areas indiscriminately.

\quad While no apparent lateralization is present in the \textit{Premotor Cortex}, this part of the cortex is negatively significant in the \textit{Encoder} and \textit{Summarizer} bottlenecks for both the left and right hemispheres. A possible explanation is that the instances we examine involve participants \textit{performing} movements; therefore, there may not necessarily be relevant activity in the \textit{Premotor Cortex}, which is primarily involved in movement planning \cite{gallego2022going}.

\quad Lastly, we observe significance in the \textit{Classifier} bottleneck for \textit{Early Visual Cortex} and \textit{Dorsal Stream Visual Cortex}. We note that the movement is activated by a visual cue; however, further experiments would be required to fully clarify the effect.

%The proposed method is trained on a set of public datasets available in the ultrasound toolbox [10]. The proposed approach has been accepted for presentation during the Challenge on Ultrasound Beamforming with Deep Learning (CUBDL) at the 2020 IEEE International Ultrasonics Symposium (IUS) [11], [12]. synthetic / PICMUS difference

%ALSO DISCUSS ABOUT TYPE OF PRETRAINED VS TRAINED
Regarding the computing time, \yz{our approaches need 3-4 minutes to form one image, which is slower than DAS1, PCF~\cite{PCF} and MNV2~\cite{MNV2}, but faster \ji{than EMV ~\cite{asl_eigenspace-based_2010} and RED~\cite{RED_USIPB}, which need 8 and 20 minutes, respectively.} RED is slow because each iteration contains an inner iteration while \ji{EMV spends time on covariance matrix evaluation and decomposition.} Our iteration restoration approaches require multiple multiplication operations with the singular vector matrix, which currently hinders real-time imaging. }Accelerating this process is one of our key focuses for future work.

\begin{comment}
the computationally expensive SVD for DDRM actually does not affect imaging time since the SVD results can be precomputed, but the multiplication operation with the singular vector matrix during the image reconstruction process currently hinders real-time imaging. On our machine equipped with the GPU NVIDIA Quadro RTX 3000, each iteration takes approximately 4.5 seconds.
\end{comment}

\YZ{In conclusion, for the first time, we achieve the reconstruction of ultrasound images with} 
\DM{ two adapted diffusion models, DRUS and WDRUS. }
\yz{Different from previous model-based deep learning methods which are task-specific and require a large amount of data pairs for supervised training, our approach requires none or just a small fine-tuning dataset composed of high-quality (e.g., DAS101) images only (there is no need for paired data). Furthermore, the fine-tuned diffusion model can be used}
%applied to} 
\dm{for other US related inverse problems.}
%diverse inverse problems, e.g., DRUS and WDRUS, as long as the same prior knowledge.}
\YZ{Finally, our method demonstrated competitive performance compared to DAS75, and other state-of-the-art approaches on the PICMUS dataset.}



\begin{comment}
\YZ{Our approach has demonstrated superior performance compared to DAS, both in terms of visual quality and evaluation metrics, on both synthetic and PICMUS datasets}, \DM{even though we fine-tuned the \texttt{f-number} to fit DDRM while kept the default values from the open-source code for DAS, as 
%While we used slightly different parameters, e.g. \texttt{f-number} for DAS and our methods when testing on the PICMUS dataset, such as using default values from the PICMUS open-source code for DAS while fine-tuning these parameters for our methods to fit DDRM, 
fundamentally, the number of plane waves affects the image quality of DAS.} 
\YZ{Our method is able to compete with 75 plane waves, which is sufficient evidence of its effectiveness. As for the distortion observed in the WDRUS results, it may be due to the amplification of errors in the ultrasound model by the whitening operator, but the specific reason requires further investigation.
}

\YZ{
Our method provides an important insight for the medical imaging field by addressing the challenge of training} \DM{model-based deep learning methods
%neural networks 
}\YZ{when access to datasets is restricted due to privacy concerns. The model we used was trained on ImageNet only, without any ultrasound data.
}
\DM{However, it} 
\YZ{
%It 
should be noted that  %images in the 
ImageNet data %dataset 
are significantly different from ultrasound images. For example, pixel values in natural images are always positive, while} 
%in ultrasound image reconstruction, 
 \DM{the reconstructed ultrasound 
$\xv$ contains both positive and negative values, and 
%ultrasound images 
are typically displayed after log compression.}
\YZ{Therefore, fine-tuning existing models with a small amount of ultrasound data may lead to better results.
}

\YZ{
Although DDRM relies on the computationally expensive SVD, it does not affect imaging time since} 
%its results can be saved and repeatedly used. 
\DM{ the SVD results can be precomputed.}
\YZ{However, the multiplication operation} \DM{ with the singular vector matrix 
%between the singular matrix and other vectors 
}
\YZ{during the image reconstruction process currently hinders real-time imaging. On our machine equipped with the GPU NVIDIA Quadro RTX 3000, each iteration takes approximately 4.5 seconds. Accelerating this process is one of our key focuses for future work.}

\YZ{
In conclusion, for the first time, we achieve the reconstruction of ultrasound images with} 
%a diffusion model and test two ultrasound models, DRUS and WDRUS. 
\DM{ two adapted diffusion models, DRUS and WDRUS. }
\YZ{Our method with single plane wave is even comparable to DAS with 75 plane waves,}
%which are often used to produce target images, in the case where the generative model has never been trained on ultrasound data.
\DM{which is often used as reference to train generative models, whereas our diffusion model was never trained on ultrasound data}
\end{comment}


\bibliographystyle{IEEEbib}
\bibliography{main}

%\def\bC{{\beta_{_{C}}}}
\def\nL{{\mathcal{L}}}
\def\nW{{\mathcal{W}}}

%  LaTeX support: latex@mdpi.com 
%  For support, please attach all files needed for compiling as well as the log file, and specify your operating system, LaTeX version, and LaTeX editor.

%=================================================================
\documentclass[journal,article,submit,pdftex,moreauthors]{Definitions/mdpi} 
\renewcommand{\linenumbers}{}
%--------------------
% Class Options:
%--------------------
%----------
% journal
%----------
% Choose between the following MDPI journals:
% acoustics, actuators, addictions, admsci, adolescents, aerobiology, aerospace, agriculture, agriengineering, agrochemicals, agronomy, ai, air, algorithms, allergies, alloys, analytica, analytics, anatomia, animals, antibiotics, antibodies, antioxidants, applbiosci, appliedchem, appliedmath, applmech, applmicrobiol, applnano, applsci, aquacj, architecture, arm, arthropoda, arts, asc, asi, astronomy, atmosphere, atoms, audiolres, automation, axioms, bacteria, batteries, bdcc, behavsci, beverages, biochem, bioengineering, biologics, biology, biomass, biomechanics, biomed, biomedicines, biomedinformatics, biomimetics, biomolecules, biophysica, biosensors, biotech, birds, bloods, blsf, brainsci, breath, buildings, businesses, cancers, carbon, cardiogenetics, catalysts, cells, ceramics, challenges, chemengineering, chemistry, chemosensors, chemproc, children, chips, cimb, civileng, cleantechnol, climate, clinpract, clockssleep, cmd, coasts, coatings, colloids, colorants, commodities, compounds, computation, computers, condensedmatter, conservation, constrmater, cosmetics, covid, crops, cryptography, crystals, csmf, ctn, curroncol, cyber, dairy, data, ddc, dentistry, dermato, dermatopathology, designs, devices, diabetology, diagnostics, dietetics, digital, disabilities, diseases, diversity, dna, drones, dynamics, earth, ebj, ecologies, econometrics, economies, education, ejihpe, electricity, electrochem, electronicmat, electronics, encyclopedia, endocrines, energies, eng, engproc, entomology, entropy, environments, environsciproc, epidemiologia, epigenomes, est, fermentation, fibers, fintech, fire, fishes, fluids, foods, forecasting, forensicsci, forests, foundations, fractalfract, fuels, future, futureinternet, futurepharmacol, futurephys, futuretransp, galaxies, games, gases, gastroent, gastrointestdisord, gels, genealogy, genes, geographies, geohazards, geomatics, geosciences, geotechnics, geriatrics, grasses, gucdd, hazardousmatters, healthcare, hearts, hemato, hematolrep, heritage, higheredu, highthroughput, histories, horticulturae, hospitals, humanities, humans, hydrobiology, hydrogen, hydrology, hygiene, idr, ijerph, ijfs, ijgi, ijms, ijns, ijpb, ijtm, ijtpp, ime, immuno, informatics, information, infrastructures, inorganics, insects, instruments, inventions, iot, j, jal, jcdd, jcm, jcp, jcs, jcto, jdb, jeta, jfb, jfmk, jimaging, jintelligence, jlpea, jmmp, jmp, jmse, jne, jnt, jof, joitmc, jor, journalmedia, jox, jpm, jrfm, jsan, jtaer, jvd, jzbg, kidneydial, kinasesphosphatases, knowledge, land, languages, laws, life, liquids, literature, livers, logics, logistics, lubricants, lymphatics, machines, macromol, magnetism, magnetochemistry, make, marinedrugs, materials, materproc, mathematics, mca, measurements, medicina, medicines, medsci, membranes, merits, metabolites, metals, meteorology, methane, metrology, micro, microarrays, microbiolres, micromachines, microorganisms, microplastics, minerals, mining, modelling, molbank, molecules, mps, msf, mti, muscles, nanoenergyadv, nanomanufacturing,\gdef\@continuouspages{yes}} nanomaterials, ncrna, ndt, network, neuroglia, neurolint, neurosci, nitrogen, notspecified, %%nri, nursrep, nutraceuticals, nutrients, obesities, oceans, ohbm, onco, %oncopathology, optics, oral, organics, organoids, osteology, oxygen, parasites, parasitologia, particles, pathogens, pathophysiology, pediatrrep, pharmaceuticals, pharmaceutics, pharmacoepidemiology,\gdef\@ISSN{2813-0618}\gdef\@continuous pharmacy, philosophies, photochem, photonics, phycology, physchem, physics, physiologia, plants, plasma, platforms, pollutants, polymers, polysaccharides, poultry, powders, preprints, proceedings, processes, prosthesis, proteomes, psf, psych, psychiatryint, psychoactives, publications, quantumrep, quaternary, qubs, radiation, reactions, receptors, recycling, regeneration, religions, remotesensing, reports, reprodmed, resources, rheumato, risks, robotics, ruminants, safety, sci, scipharm, sclerosis, seeds, sensors, separations, sexes, signals, sinusitis, skins, smartcities, sna, societies, socsci, software, soilsystems, solar, solids, spectroscj, sports, standards, stats, std, stresses, surfaces, surgeries, suschem, sustainability, symmetry, synbio, systems, targets, taxonomy, technologies, telecom, test, textiles, thalassrep, thermo, tomography, tourismhosp, toxics, toxins, transplantology, transportation, traumacare, traumas, tropicalmed, universe, urbansci, uro, vaccines, vehicles, venereology, vetsci, vibration, virtualworlds, viruses, vision, waste, water, wem, wevj, wind, women, world, youth, zoonoticdis 
% For posting an early version of this manuscript as a preprint, you may use "preprints" as the journal. Changing "submit" to "accept" before posting will remove line numbers.

%---------
% article
%---------
% The default type of manuscript is "article", but can be replaced by: 
% abstract, addendum, article, book, bookreview, briefreport, casereport, comment, commentary, communication, conferenceproceedings, correction, conferencereport, entry, expressionofconcern, extendedabstract, datadescriptor, editorial, essay, erratum, hypothesis, interestingimage, obituary, opinion, projectreport, reply, retraction, review, perspective, protocol, shortnote, studyprotocol, systematicreview, supfile, technicalnote, viewpoint, guidelines, registeredreport, tutorial
% supfile = supplementary materials

%----------
% submit
%----------
% The class option "submit" will be changed to "accept" by the Editorial Office when the paper is accepted. This will only make changes to the frontpage (e.g., the logo of the journal will get visible), the headings, and the copyright information. Also, line numbering will be removed. Journal info and pagination for accepted papers will also be assigned by the Editorial Office.

%------------------
% moreauthors
%------------------
% If there is only one author the class option oneauthor should be used. Otherwise use the class option moreauthors.

%---------
% pdftex
%---------
% The option pdftex is for use with pdfLaTeX. Remove "pdftex" for (1) compiling with LaTeX & dvi2pdf (if eps figures are used) or for (2) compiling with XeLaTeX.

%=================================================================
% MDPI internal commands - do not modify
\firstpage{1} 
\makeatletter 
\setcounter{page}{\@firstpage} 
\makeatother
\pubvolume{1}
\issuenum{1}
\articlenumber{0}
\pubyear{2023}
\copyrightyear{2023}
%\externaleditor{Academic Editor: Firstname Lastname}
\datereceived{ } 
\daterevised{ } % Comment out if no revised date
\dateaccepted{ } 
\datepublished{ } 
%\datecorrected{} % For corrected papers: "Corrected: XXX" date in the original paper.
%\dateretracted{} % For corrected papers: "Retracted: XXX" date in the original paper.
\hreflink{https://doi.org/} % If needed use \linebreak
%\doinum{}
%\pdfoutput=1 % Uncommented for upload to arXiv.org

%=================================================================
% Add packages and commands here. The following packages are loaded in our class file: fontenc, inputenc, calc, indentfirst, fancyhdr, graphicx, epstopdf, lastpage, ifthen, float, amsmath, amssymb, lineno, setspace, enumitem, mathpazo, booktabs, titlesec, etoolbox, tabto, xcolor, colortbl, soul, multirow, microtype, tikz, totcount, changepage, attrib, upgreek, array, tabularx, pbox, ragged2e, tocloft, marginnote, marginfix, enotez, amsthm, natbib, hyperref, cleveref, scrextend, url, geometry, newfloat, caption, draftwatermark, seqsplit
% cleveref: load \crefname definitions after \begin{document}

%=================================================================
% Please use the following mathematics environments: Theorem, Lemma, Corollary, Proposition, Characterization, Property, Problem, Example, ExamplesandDefinitions, Hypothesis, Remark, Definition, Notation, Assumption
%% For proofs, please use the proof environment (the amsthm package is loaded by the MDPI class).

%=================================================================
% Full title of the paper (Capitalized)
\Title{Axion field influence on Josephson junction quasipotential}

% MDPI internal command: Title for citation in the left column
\TitleCitation{Axion field influence on Josephson junction quasipotential}

% Author Orchid ID: enter ID or remove command
\newcommand{\orcidauthorA}{0000-0003-2363-7699} 
\newcommand{\orcidauthorB}{0000-0001-5496-1518} 
\newcommand{\orcidauthorC}{0000-0002-6625-3989} 
\newcommand{\orcidauthorD}{0000-0002-4348-9956}
\newcommand{\orcidauthorE}{0000-0003-3546-8618} 
\newcommand{\orcidauthorF}{0000-0002-3683-2509} % Add \orcidA{} behind the author's name
%\newcommand{\orcidauthorB}{0000-0000-0000-000X} % Add \orcidB{} behind the author's name

% Authors, for the paper (add full first names)
\Author{Roberto Grimaudo $^{1}$\orcidA{}, Davide Valenti $^{1}$\orcidB{}, Bernardo Spagnolo $^{1,2}$\orcidC{}, Antonio Troisi $^{3}$\orcidD{}, Giovanni Filatrella $^{3,4}$\orcidE{} and Claudio Guarcello $^{5,4}$\orcidF{}}

%\longauthorlist{yes}

% MDPI internal command: Authors, for metadata in PDF
\AuthorNames{Roberto Grimaudo, Davide Valenti, Bernardo Spagnolo, Antonio Troisi, Giovanni Filatrella and Claudio Guarcello}

% MDPI internal command: Authors, for citation in the left column
\AuthorCitation{Grimaudo R.; Valenti D.; Spagnolo B.; Troisi A.; Filatrella G.; Guarcello C.}
% If this is a Chicago style journal: Lastname, Firstname, Firstname Lastname, and Firstname Lastname.

% Affiliations / Addresses (Add [1] after \address if there is only one affiliation.)
\address{%
$^{1}$ \quad Dipartimento di Fisica e Chimica ``E. Segr\`e'', Group of Theoretical Interdisciplinary Physics, Universit\`a degli Studi di Palermo, Viale delle Scienze, Ed. 18, I-90128 Palermo, Italy;\\
$^{2}$ \quad Lobachevskii University of Nizhnii Novgorod, 23 Gagarin Ave. Nizhnii Novgorod 603950 Russia;\\
$^{3}$ \quad Dep. of Sciences and Technologies, University of Sannio, Via De Sanctis, Benevento I-82100, Italy;\\
$^{4}$ \quad INFN, Sezione di Napoli Gruppo Collegato di Salerno, Complesso Universitario di Monte S. Angelo, I-80126 Napoli, Italy;\\
$^{5}$ \quad Dipartimento di Fisica ``E.R. Caianiello'', Universit\`a di Salerno, Via Giovanni Paolo II, 132, I-84084 Fisciano (SA), Italy;}

% Contact information of the corresponding author
\corres{Correspondence: cguarcello@unisa.it (C.G) giovanni.filatrella@unisannio.it (G.F.)}

% Current address and/or shared authorship
%\firstnote{Current address: Affiliation 3.} 
%\secondnote{These authors contributed equally to this work.}
% The commands \thirdnote{} till \eighthnote{} are available for further notes

%\simplesumm{} % Simple summary

%\conference{} % An extended version of a conference paper

% Abstract (Do not insert blank lines, i.e. \\) 
\abstract{The direct effect of an axion field on Josephson junctions is analyzed through the consequences on the effective potential barrier that prevents the junction from switching from the superconducting to the finite-voltage state. 
We describe a method to reliably compute the quasipotential with stochastic simulations, which allows to span the coupling parameter from weakly interacting axion to tight interactions.
As a result, we obtain that the axion field induces a change in the potential barrier, therefore determining a significant detectable effect for such a kind of elusive particle. }

% Keywords
\keyword{Josephson junction; axion; quasipotential; switching dynamics; noise; detection} 

% The fields PACS, MSC, and JEL may be left empty or commented out if not applicable
%\PACS{J0101}
%\MSC{}
%\JEL{}

%%%%%%%%%%%%%%%%%%%%%%%%%%%%%%%%%%%%%%%%%%
% Only for the journal Diversity
%\LSID{\url{http://}}

%%%%%%%%%%%%%%%%%%%%%%%%%%%%%%%%%%%%%%%%%%
% Only for the journal Applied Sciences
%\featuredapplication{Authors are encouraged to provide a concise description of the specific application or a potential application of the work. This section is not mandatory.}
%%%%%%%%%%%%%%%%%%%%%%%%%%%%%%%%%%%%%%%%%%

%%%%%%%%%%%%%%%%%%%%%%%%%%%%%%%%%%%%%%%%%%
% Only for the journal Data
%\dataset{DOI number or link to the deposited data set if the data set is published separately. If the data set shall be published as a supplement to this paper, this field will be filled by the journal editors. In this case, please submit the data set as a supplement.}
%\datasetlicense{License under which the data set is made available (CC0, CC-BY, CC-BY-SA, CC-BY-NC, etc.)}

%%%%%%%%%%%%%%%%%%%%%%%%%%%%%%%%%%%%%%%%%%
% Only for the journal Toxins
%\keycontribution{The breakthroughs or highlights of the manuscript. Authors can write one or two sentences to describe the most important part of the paper.}

%%%%%%%%%%%%%%%%%%%%%%%%%%%%%%%%%%%%%%%%%%
% Only for the journal Encyclopedia
%\encyclopediadef{For entry manuscripts only: please provide a brief overview of the entry title instead of an abstract.}

%%%%%%%%%%%%%%%%%%%%%%%%%%%%%%%%%%%%%%%%%%
% Only for the journal Advances in Respiratory Medicine
%\addhighlights{yes}
%\renewcommand{\addhighlights}{%

%\noindent This is an obligatory section in “Advances in Respiratory Medicine”, whose goal is to increase the discoverability and readability of the article via search engines and other scholars. Highlights should not be a copy of the abstract, but a simple text allowing the reader to quickly and simplified find out what the article is about and what can be cited from it. Each of these parts should be devoted up to 2~bullet points.\vspace{3pt}\\
%\textbf{What are the main findings?}
% \begin{itemize}[labelsep=2.5mm,topsep=-3pt]
% \item First bullet.
% \item Second bullet.
% \end{itemize}\vspace{3pt}
%\textbf{What is the implication of the main finding?}
% \begin{itemize}[labelsep=2.5mm,topsep=-3pt]
% \item First bullet.
% \item Second bullet.
% \end{itemize}
%}

%%%%%%%%%%%%%%%%%%%%%%%%%%%%%%%%%%%%%%%%%%
\begin{document}

%%%%%%%%%%%%%%%%%%%%%%%%%%%%%%%%%%%%%%%%%%
\section{Introduction}

Nowadays, in the search for cold dark matter candidates, among others, axion particles were theoretically predicted, but their detection remains elusive, for the very weak interaction that they are supposed to have with ordinary matter~\cite{Preskili83}.
Since Josephson junctions (JJs) proved to be very sensitive devices, close to detecting a single photon~\cite{Alesini20}, a natural idea was to exploit them to detect the electromagnetic field produced by axion decay~\cite{Rettaroli21}.
A different possibility, which is the framework for the present paper, is to exploit the direct interaction between JJ and axions~\cite{Bec13} in a detector~\cite{Grimaudo22,Gri23}. This scheme was also proposed as a key to understand unclear ``events'' in Josephson's response~\cite{Bec17,Hof04,Bae08,He11,Gol12,Bre13,Wang22} and would simplify the usual detection schemes through a reduced setting.
To date, the main idea behind this direct interaction is that the axion decays into the junction with a rather high probability, which would be achieved in a resonant cavity through the Primakoff effect only through a huge magnetic field (orders of magnitude above any realistic field): thus, a "Josephson cavity" is much more effective to detect the axion than a resonant cavity~\cite{Bec13} .
This scheme, however, misses a full theory of the Josephson-axion interaction and a detailed scheme of the changes induced in the JJ dynamics, which could possibly lead to detectable consequences. 
We here concentrate on the latter problem, assuming that the axion-JJ interaction exists, although the interaction parameter is unknown.
In the original proposal it was suggested to look at deviations from the locked dynamics of the JJ to an external radio frequency -- the so-called Shapiro steps -- that could be altered by the extra Cooper's pairs created by the axion decay. 
More recently, it has been proposed by some of the authors of the present paper to assume a different standpoint: to bias the JJ in the superconducting metastable state through a dc external drive and to observe the passages to the finite voltage state in the presence of the axion-JJ coupling, under the influence of thermal fluctuations~\cite{Grimaudo22}. 
The idea is that these switches are altered by the interaction of the JJ with the axions, and it is therefore possible to infer the existence of the axions if the switching is, in some statistical sense, changed; a more detailed analysis can also offer an estimate of the JJ-axion coupling from switching time measurements~\cite{Grimaudo22}. A successive approach assumed the JJ operating as a qubit, and in such a way the qubit-axion interaction being detected as axion-induced oscillations of the qubit state~\cite{Gri23}.
The general idea of exploiting a JJ to detect a weak signal, even embedded in a noisy background, is fairly well-established, a JJ being essentially a threshold device operating via a switching mechanism. The presence of a noise background is a condition typical of open systems, such as, for instance, biological and ecological systems~\cite{Val12,Lis15,Valenti16} and financial markets~\cite{Valenti18}, which has to be taken into account in view of better modeling their dynamics. Josephson junctions have been also proposed as noise detectors~\cite{Tob04,Pek04,Ank07,Suk07,Tim07,Hua07,Gra08,Fil10,Add12,Gua13,Gua19,Gua20,Gua21-2} and play a leading role in the search for possible protocols and schemes for the detection of single photons~\cite{Wal17,Kuz18,GuaBra19,GuaBraSol19,Rev20,Yab21,Pied21,Gua21,Pan22-1,Pan22-2}. 

In this paper we further investigate the consequences of a direct interaction between axions and JJ, to show that a certain quantity, namely, the \emph{quasipotential}~\cite{Graham85}, can be introduced for this non-equilibrium system and that it is possible to determine the quasipotential in the presence of the axions.
To investigate the quasipotential is an advantage, as it can be determined with numerical simulations at a relatively high noise intensity, i.e., high temperature. 
Indeed, the quasipotential (as the ordinary potential) is not affected by noise; if the quasipotential is known, it is possible to predict the average escape time at very low temperature with the Arrhenius law, with a considerable saving of simulation time~\cite{Kau88}. 
We do so with a twofold objective: in the first place, as already mentioned, to better understand the consequences that a direct interaction between JJ and axion would have, and therefore to pave the way towards a practical implementation of the device to detect axions; on the second hand, not less important, to compute the quasipotential of the axion-JJ system, a very convenient quantity in the analysis of low temperature devices (as already demonstrated for Shapiro steps~\cite{Kau96} and cavity-induced synchronization~\cite{Pou19}) to infer the properties at a very low noise and, consequently, very long escape times. In fact, the characteristic time scale of the system is of the order of $[1-10]\;\text{ps}$, being the inverse of Josephson characteristic frequency which, as we shall see later, generally falls in the range $[0.1-1]\;\text{THz}$; this means that numerical realizations, even close to real experimental times (which could take up to milliseconds, e.g., for experiments involving switching current distributions), require extremely long simulations. Moreover, these have to be repeated several times in order to obtain complete statistics. Indeed, stochastic analyses, such as the one we propose, require the repetition of the same experiment, i.e., of the same numerical simulation, for a reasonably large number of times under the same conditions, in order to allow for reliable statistical analyses. In conclusion, quasipotential analysis makes it possible to extract useful information at reasonably high temperatures, that means within reasonable simulation times, and then allows to extrapolate relevant information even at low temperatures, where numerical simulations time would become prohibitive. 

The paper is organized as follows: Sect.~\ref{Model} presents the model to describe: the JJ (Sect.~\ref{ModelJJ}), the axion field (Sect.~\ref{Modelaxions}), and the interacting axion-JJ system (Sect.~\ref{ModelJJaxions}). Sect.~\ref{Results} defines the quasipotential for this system and computes its behavior as a function of the interaction. 
Finally, in Sect.~\ref{Conclusions} conclusions are drawn.




\section{Model} 
\label{Model}

In this section we outline the models for the JJ, see Sect.~\ref{ModelJJ}, the axion, see Sect.~\ref{Modelaxions}, and their interaction, see Sect.~\ref{ModelJJaxions}. 
We show the potential of the JJ alone -- a cosinusoidal washboard potential, see Eq.~\eqref{Washboard App} below -- that exhibits an activation energy barrier, whose changes due to the interaction with the axions are the focus of the present work.
Also, some details for the numerical simulations of the stochastic equations are given in Sect.~\ref{ModelJJ}.


\subsection{RCSJ Model}
\label{ModelJJ}

Let us consider the usual model for a superconducting junction, schematically represented in Fig.~\ref{fig: Device}(a), given by the following equations~\cite{Bar82,Lik86}:
\begin{eqnarray}
\label{JJcurrent}
I_\varphi = I_c \sin{\varphi},\\
\label{JJvoltage}
V = \frac{\Phi_0}{2\pi}\frac{d\varphi}{dt},
\end{eqnarray}
where $\Phi_0=h/(2e)$ is the flux quantum, with $e$ and $h$ being the electron charge and the Planck constant, respectively, $I_c$ is the maximum Josephson current that can flow through the device, and $\varphi$ is the Josephson phase difference.

For a real device, one assumes for instance that the two superconductors have lateral dimensions $\nL$ and $\nW$ smaller than the Josephson penetration depth, $\lambda_{_{J}} = \sqrt{\Phi_0/(2\pi \mu_0 t_d J_c)}$ (here, $t_d=\lambda_{L,1}+\lambda_{L,2}+d$ is the effective magnetic thickness, with $\lambda_{L}$ and $d$ being the London penetration depths and the insulating layer thickness, respectively, $\mu_0$ is the vacuum permeability, and $J_c$ is the critical current area density).
The dynamics of the Josephson phase $\varphi$ for a dissipative, current-biased small JJ can thus be studied within the resistively and capacitively shunted junction (RCSJ) framework~\cite{Bar82,GuaVal15,Spa17,McC68,Gua19,Gua20}

%
%
\begin{equation}
\left ( \frac{\Phi_0}{2\pi} \right )^{\!\!2}\!\! C \frac{d^2 \varphi}{d t^2}+\left ( \frac{\Phi_0}{2\pi} \right )^{\!\!2}\!\!\frac{1}{R} \frac{d \varphi}{d t}+\frac{d }{d \varphi}U 
= \left ( \frac{\Phi_0}{2\pi} \right )( I_N+I_b),
\label{RCSJ App}
\end{equation}
%
with $R$ and $C$ the normal-state resistance and capacitance of the JJ, respectively, and $I_N$ and $I_b$ the thermal noise and the bias current, respectively.
The corresponding normalized dynamics can be reformulated (for sinusoidal potential of standard tunnel JJ, albeit other shapes are possible~\cite{Bee92}) through the equation:
%
%
\begin{equation}
\beta_c\frac{d^2 \varphi (\tau_c)}{d\tau_c^2}+ \frac{d \varphi (\tau_c)}{d\tau_c} +\frac{d }{d \varphi}\mathcal{U}(\varphi,i_b) = i_{n}(\tau_c) + i_b,
\label{RCSJnormOc}
\end{equation}
%
%
where time is normalized to the inverse of the characteristic frequency, that is $\tau_c = \omega_c~t$ with $\omega_c=\left ( 2\pi/\Phi_0 \right )I_cR$, $i_b= I_b / I_c$ and $i_n= I_n / I_c$ are the normalized external bias current and thermal noise current, and $\beta_c=\omega_c RC$ is the Stewart-McCumber parameter. 
We stress that the JJ response is usually quite fast, since the characteristic frequency of JJ falls within the range $ [0.1,1]\;\text{THz}$. 
Another way to obtain a dimensionless form of Eq.~\eqref{RCSJ App} consists in normalizing with respect to the plasma frequency $\omega_p=\sqrt{2eI_c/\hbar C}$.
In this case, time is normalized respect to the inverse plasma frequency, i.e., $\tau_p = \omega_p~t$, and the equation in normalized units contains a damping parameter $\alpha=\beta_{_{C}}^{-1/2}$, which multiplies the first time-derivative of the phase. 

The normalized potential, $\mathcal{U}$, is the so-called \emph{washboard potential}, which depends upon the normalized bias current, $i_b$, and the Josephson phase according to
%
% 
\begin{equation}
\mathcal{U}(\varphi,i_b)=\frac{U(\varphi,i_b)}{{E_{J_0}}}=\left [1- \cos(\varphi) -i_b\varphi\right ].
\label{Washboard App}
\end{equation}
%
%
The potential can be expressed in physical units defining the Josephson energy $E_{J_0}=\left ( \Phi_0/2\pi \right )I_c$. 
The resulting activation energy barrier, $\Delta U(i_b)$, confines the phase $\varphi$ in a metastable potential minimum and can be calculated as the difference between the maximum and minimum value of the normalized potential $U(\varphi,i_b)$, see Fig.~\ref{fig: Device}(b).
In units of $E_{J_0}$, it can be expressed as
%
% 
\begin{equation}
{\Delta \mathcal{U}(i_b)}=\frac{\Delta {U}(i_b)}{{E_{J_0}}}=2 \left [ \sqrt{1-i_b^2} -i_b\arccos(i_b)\right ].
\label{activationenergybarrier App}
\end{equation}
%
%
In the phase particle picture, the term $i_b$ represents the tilting of the potential profile; increasing $i_b$ the slope of the washboard increases and the height 
$\Delta \mathcal{U}(i_b)$ of the rightward potential barrier reduces, until this activation energy vanishes altogether for $i_b=1$, that is when the bias current reaches its critical value $I_c$. 
During the motion, different regimes are governed by the Stewart-McCumber parameter $\bC$.
 A highly damped (or overdamped) junction corresponds to $\bC\ll 1$, that is a small capacitance and/or a small resistance. 
 Instead, a junction with $\bC\gg 1$ has a large capacitance and/or a large resistance, and is weakly damped (or underdamped)~\footnote{With the alternative normalized mentioned before, the under- and overdamped regimes correspond to $\alpha \ll 1$ and $\alpha \gg 1$, respectively.}.
 For the purposes of this work, it is important to notice that in the underdamped regime once the phase has passed the barrier, a finite velocity, and hence, as per Eq.~\eqref{JJvoltage}, a finite voltage, appears.
It is therefore possible to detect the passage of the Josephson phase over the barrier through the appearance of a finite voltage, a key point to employ a JJ as a detector. In fact, the phase $\varphi$ itself is not directly accessible, while the passage over the barrier is signaled by a measurable voltage drop across the junction.
The procedure can be briefly schematized as follows. 
The JJ is prepared in some static configuration (at which corresponds a zero net voltage), exposed to some supposedly existing perturbation, and the junction is left to evolve. 
If the signal was not present, the passage only occurs under the effect of thermal noise, and it is given by the usual Kramers law~\cite{Kra40}. The presence of the signal is ascertained through deviations of the thermal escapes~\cite{Fil10,Pie21} -- as will be discussed in more details below.


In this work, the random current is modeled as a delta-correlated Gaussian white noise associated to the normal-state resistance of the junction, $R$, with the usual statistical properties:
\begin{eqnarray}
\label{averageD}
\langle i_n (\tau)\rangle\ &=& 0,\\
\label{sigmaD}
\langle i_n (\tau) i_n (\tau+\tilde{\tau})\rangle &=& 2D\,\delta (\tilde{\tau}). 
\end{eqnarray}
The amplitude of the normalized correlation is connected with the physical temperature $T$ through the relation~\cite{Bar82}
%
\begin{eqnarray}
\label{WNAmp}
D= \frac{k_BT}{R}\frac{\omega_c}{I^2_c},%=\frac{k_BT}{E_{J_0}},
\end{eqnarray}
%
here $k_B$ is the Boltzmann constant.
We note that, by normalizing time with respect to the characteristic frequency $\omega_c$ (as we do in our numerical simulations), the normalized noise intensity in Eq.~\eqref{WNAmp} can be recast as $D={k_BT}/{E_{J_0}}$, i.e, the ratio between the thermal energy and the Josephson coupling energy, $E_{J_0}$, without reference to the damping; instead, normalizing with respect to the plasma frequency, $\omega_p$, the normalized noise intensity becomes $D=\alpha{k_BT}/{E_{J_0}}$.
Thus, for Gaussian fluctuations of amplitude $D$, the stochastic independent increment employed in the numerical simulations reads
$\Delta i_N \simeq \sqrt{ 2 D \Delta t\; }\; N\left(0,1 \right)$.
Here, $N\left(0, 1 \right)$ indicates a Gaussianly distributed random function with zero mean and unit standard deviation. 




%Another way to obtain a dimensionless form of Eq.~\eqref{RCSJ App} consists in normalizing with respect to the plasma frequency $\omega_p=\sqrt{2eI_c/\hbar C}$.
%In the latter case, the normalized RCSJ equation \eqref{RCSJ App} reads
%
%
%\begin{equation}
%\frac{d^2 \varphi (\tau_p)}{d\tau_p^2}+ \alpha \frac{d \varphi (\tau_p)}{d\tau_p} + \sin \left [ \varphi\left ( \tau_p \right ) \right ] = i_{n}(\tau_p) + i_b,
%\label{RCSJnormOp App}
%\end{equation}
%
%
%where time is normalized respect to the inverse plasma frequency (that reads $\omega_p=\sqrt{{\Phi_0}/{2\pi C}}$ ), $\tau_p = \omega_p~t$, $\alpha=1/\sqrt(\omega_p~R~C)=1/\sqrt{\bC}$ is the damping parameter. 
%With this time normalization the under- and over-damped regimes correspond to $\alpha \ll 1$ and $\alpha \gg 1$, respectively.


\subsection{Axion}
\label{Modelaxions}

If one describes the axion field $a$ in the Friedman-Robertson-Walker metric, the equation of motion of the axion misalignment angle $\theta$ and the axion coupling constant $f_a$, namely $a=f_a\,\theta$~\cite{Sik83,Vis13}, reads
%
\begin{equation}
\frac{d^2 \theta (t)}{dt^2}+ H \frac{d \theta (t)}{dt} + \frac{m_a^2c^4}{\hbar^2} \sin \left [ \theta \left ( t \right ) \right ] = 0,
\label{AxionEq}
\end{equation}
%
complemented with spatial gradients that are here omitted. 
The above model includes the forcing term $\sin(\theta)$ due to quantum chromodynamics instanton effects. As one can observe, there is a formal similarity between the equation of motion governing the axion and the RCSJ systems, being the axion dynamics analogous to an unbiased RCSJ. Moreover, in normalized units, the parameters are of the same order of magnitude.
In Eq.~\eqref{AxionEq}, $H \approx 2 \times 10^{-18} ~ s^{-1}$ is the Hubble parameter and $m_a$ is the axion mass. 
The typical ranges of parameters that are allowed for dark matter axions are~\cite{sik09,Duf09}: 
$ 3 \times 10^9 ~ \text{GeV} \leq f_a \leq 10^{12} ~ \text{GeV}$ and $ 6 \times 10^{-6} ~ \text{eV} \leq m_a c^2 \leq 2 \times 10^{-3} ~ \text{eV}$. 
The prediction of the axion's mass, based on the average of the results from five independent condensed matter experiments, is $ m_a c^2 = (106 \pm 6) \mu eV$~\cite{,Hof04,Bae08,He11,Gol12,Bre13,Wang22}.
%

%
% Figure environment removed
%


\subsection{Axion-JJ System}
\label{ModelJJaxions}

According to the approach of Refs.~\cite{Grimaudo22,Yan20}, the interaction between axion and JJ can be formally written as 
%
%
\begin{subequations}\label{Orig Diff Eqs Syst}
\begin{align}
\ddot{\varphi} + a_1 \dot{\varphi} + b_1 \sin(\varphi) &= \gamma (\ddot{\theta} - \ddot{\varphi}) \label{Orig Diff Eqs Syst a},\\
\ddot{\theta} + a_2 \dot{\theta} + b_2 \sin(\theta) &= \gamma (\ddot{\varphi} - \ddot{\theta}),
\end{align}
\label{Orig Diff Eqs Syst}
\end{subequations}
%
%
%
\noindent where $(a_1, a_2)$ and $(b_1, b_2)$ are the dissipation and frequency parameters, respectively; $\gamma$ is the coupling constant between the two systems, whose values one wants to infer from the experiments. 
This model, which succeeds in explaining some experimental anomalies~\cite{Hof04,Bae08,He11,Gol12,Bre13,Wang22}, is based on the possibility to formally treat the axion as an effective JJ, and therefore to consider the system in Eqs.~\eqref{Orig Diff Eqs Syst} as equivalent to two capacitively coupled JJs~\cite{Blac09}.

To model the Josephson phase dynamics with a bias current and thermal fluctuations, Eqs.~\eqref{Orig Diff Eqs Syst} can be conveniently rewritten as 
%
\begin{subequations}
\label{Diff Eqs Syst omegac}
\begin{align}
%\label{DiffEqsJJ}
{\beta_c \over k_2}~\ddot{\varphi}+\dot{\varphi}+\sin(\varphi)+{k_1 \over k_2}~\varepsilon~\sin(\theta) &= i_b+i_n, \label{Diff Eqs Syst omegac phi} \\
%\label{DiffEqsaxion}
{\beta_c \over k_1}~\ddot{\theta}+\dot{\varphi}+\sin(\varphi)+{k_2 \over k_1}~\varepsilon~\sin(\theta) &= i_b+i_n, \label{Diff Eqs Syst omegac theta}
\end{align}
\end{subequations}
%
%
with 
%
%
\begin{equation}
\begin{aligned}
&k_1  =  {\gamma \over 1+2\gamma}, \qquad k_2  =  {1+\gamma \over 1+2\gamma}, \qquad \varepsilon  =  \left(\frac{m_ac^2}{\hbar\omega_p}\right)^2.
\label{epsilon}
\end{aligned}
\end{equation}
%
%&k_1 = {\gamma \over 1+2\gamma}, \qquad k_2 = {1+\gamma \over 1+2\gamma}, \\
%& \beta_c = \left( \frac{\omega_c}{\omega_p} \right)^2,\quad \varepsilon = \left({m_ac^2 \over \hbar\omega_p}\right)^2,
%
%
%
 The $\varepsilon$ parameter indicates the ratio between the axion energy and the 
Josephson plasma energy, $\hbar\omega_p$, and can be chosen -- within the JJ fabrication constraints -- to select the most convenient working point for the detection of an axion field interacting with the JJ. 
Indeed, the Josephson plasma frequency, and therefore the energy ratio $\varepsilon$, can be ``adjusted'' as needed, for $I_c$ can be lowered either by raising the temperature~\cite{Dub01} or by applying a magnetic field~\cite{Ber08} or a gate voltage~\cite{Du08}. 
The purpose is to determine the working point at which the system is most responsive to the axion perturbation. 
As the detection is performed through the analysis of the escape times, the response is measured in the precise sense that the distribution of the escape times for the axion-JJ coupled system deviates the most from the Josephson response in the absence of axions. In Ref.~\cite{Grimaudo22} it was in fact showed that at $\epsilon\lesssim1$ the average switching time approaches a minimum due to an axion-induced resonant activation phenomenon, for the occurrence of an effective frequency matching between axion and JJ, whereas the optimal experimental conditions for a JJ-based axion detection scheme should involve a Josephson plasma energy lower than the axion energy, i.e., $\epsilon>1$.


In this work, we trace the change in escape time (that makes the axion-JJ interaction detectable) back to the change in the effective potential barrier that confines the system to the static zero-voltage configuration. In the following Sect.~\ref{Results} we discuss how to compute the effective energy.
%, while we now discuss the numerical values of the axion parameters that enter Eq.(\ref{Diff Eqs Syst omegac}(b)). 

The integration of the stochastic Eqs.~\eqref{Diff Eqs Syst omegac} is performed with a finite-difference explicit method, using a time integration step $\Delta t=10^{-2}$, a maximum integration time $t_{max}=10^6$, initial conditions $\varphi(0)=\arcsin{(i_b)}$ 
 and $\theta(0)=\dot{\varphi}(0)=\dot{\theta}(0)=0$, and repeating each simulation $N=10^4$ times for each set of parameter values. Other parameters useful for the calculations are set as $\beta_c=100$ (i.e., an underdamped regime) and $\varepsilon=1$.


\section{Calculation of the Quasipotential}
\label{Results}

The non-equilibrium system in Eq.~\eqref{Orig Diff Eqs Syst} does not admit an ordinary potential. However, it is possible to define an effective, or quasi, potential that keeps the system in the static configuration. The axion-JJ coupled system eventually switches from the superconducting state ($V\propto d\varphi/dt=0$) to the resistive state ($V\propto d\varphi/dt\neq 0$), when the combined effect of noise and axion interaction allows the JJ to overcome the effective energy barrier. 
In this picture, one can think of the axion effect on the JJ as some perturbation that changes (more precisely, lowers, as we shall demonstrate below) the effective energy barrier. The advantage is that the change of this effective energy is independent of noise and therefore holds at any (sufficiently low) temperature.
The main difficulty is therefore to determine how the coupling between Eqs.~\eqref{Diff Eqs Syst omegac} amounts to a change in the quasipotential. 
To begin with, we show how to compute the quasipotential, that is the effective energy that must be overcome to induce a switch. 
The basic logic is as follows: suppose that the Arrhenius behavior~\cite{Graham85}
\begin{equation}
\label{Kramers}
\tau = \lim_{D\to0}\tau_0 \exp{\frac{\kappa}{D} }
\end{equation}
is valid. 
The hypothesis obviously holds for the "pure" JJ system, i.e, Eq.~\eqref{RCSJnormOc}, for which $\kappa\equiv\Delta \mathcal{U}$. 
One can make the further conjecture that the average escape time is exponential in the inverse of the noise intensity, with some coefficient $\kappa \neq \Delta \mathcal{U}$, such that, in the limit of small noise,
\begin{equation}
\label{exprelation}
 \log{\frac{\tau}{\tau_0}} = \kappa \frac{1}{D}.
\end{equation}
Under general assumptions~\footnote{For out-of-equilibrium systems that do not admit an ordinary potential, it is possible to define a non-equilibrium potential with properties analogous to those of an ordinary potential, as long as there is a single time-independent probability distribution that can be reached from any initial distribution as the weakly stochastic dynamical system approaches its steady state~\cite{Graham86}.}, it is fair to suppose that the exponential relation holds in the limit of vanishing noise, and one can thus interpret the coefficient $\kappa$ as an effective energy barrier:
\begin{equation}\label{quasipotential}
\Delta \mathcal{U}_{eff} \equiv \kappa = \lim_{D \rightarrow \infty}\frac{ \log{\tau/\tau_0}}{1/D}.
\end{equation}
In other words, the slope $\kappa$ of the relation \eqref{exprelation} can be interpreted as a {\it bona fide} potential barrier, in the limit of small noise. 
The advantage of this interpretation is twofold. 
On the one end, it gives a physically intuitive interpretation to the effect of the axion field; as we shall prove in the following subsection, the axion lowers the confining barrier, and the lowering is enhanced by the coupling $\gamma$. 
On the other hand the quasipotential offers a practical advantage, because it allows to extrapolate the results to very low values of noise, that is in the region where escape times are prohibitively long and extremely difficult to reach with simulations.
It is in fact enough to enter the regime of exponential decay to determine $\kappa$, and then to exploit such value for any lower value of the noise intensity $D$. 
The effective potential \eqref{quasipotential} can be numerically retrieved with several estimates of the escape time $\tau$ as a function of the noise amplitude $D$; in the plot $\log{\tau}$ vs $1/D$ the prefactor $\tau_0$ is the $y$-axis intercept and $\kappa$ the slope of the relationship. 
More precisely, the two quantities should be computed in the exponential regime, that is discarding the data for high $D$ to ensure that the asymptotic regime \eqref{Kramers} has been entered, as shown in Fig.~\ref{Kramers_fit}. 

% Figure environment removed


%\subsection{Behavior of the Quasi-Potential}

The axions are revealed through the difference between the electrical responses of an ideal JJ without external perturbation, and the same junction under the influence of an axion field. This difference is determined on average by the quasipotential. However, for any finite sampling the actual measured average is subject to fluctuations. Therefore, the better detection is obtained with the method to which pertains the best signal-to-noise ratio (SNR), where the signal is the measured average difference and the noise is due to the fluctuations around the average, i.e, the standard deviation of the sampled mean. The SNR thus computed is often measured through the Kumar-Carroll index~\cite{Kumar84}. However, for simplicity in this work we merely observe the effect of axion on the effective potential felt by the axion-JJ system.


For the uncoupled regime, $\gamma =0$, the numerical quasipotential should be equal to the ordinary potential; in fact, in Fig.~\ref{fig:QP} one only observes a modest discrepancy, $\eta<10\%$, due to the finite temperature (the quasipotential should be computed in the limit $D^{-1}\rightarrow \infty$, or more accurately $\Delta \mathcal{U}_{eff}/D \rightarrow \infty$) and to the finite number of realizations over which the average is calculated. 
This discrepancy is obtained as the percentage difference between $\Delta \mathcal{U}_{eff}$ estimated from the data of Fig.~\ref{fig:QP} at the lowest noise, i.e., $\gamma=0.001$, and the analytical washboard activation energy $\Delta \mathcal{U}$, see Eq.~\eqref{activationenergybarrier App}, that is $ \eta=\left ( \Delta \mathcal{U}-\Delta \mathcal{U}_{eff} \right )/\Delta \mathcal{U}=\{7.4\%, 8.0\%,\text{ and }6.1\%\}$ for $i_b=\{0.1, 0.5,\text{ and }0.8\}$, respectively.
It is indeed remarkable that, despite the relatively high noise ($\Delta \mathcal{U}_{eff}/ D \in [3-6]$ from data in Figs.~\ref{Kramers_fit} and~\ref{fig:QP}), the agreement is good.
This observation highlights the advantage of the quasipotential method, because one can use short escape times to effectively evaluate the effective quasipotential through Eq.~\eqref{quasipotential}. Shortly, we have demonstrated that the quasipotential for the JJ-axion system can be numerically evaluated with relative ease, while the result can be extrapolated to much lower values of noise, and hence to much longer escape times.

Finally, we want to exploit the estimation of the effect of the quasipotential for the detection of the axion. 
This is summarized in Fig.~\ref{fig:QP}, where the slope of the escape times versus $\gamma$ is identified with the quasipotential.
For each of the three different values of the bias current considered, $i_b=\{0.1, 0.5,\text{ and }0.8\}$, it is proven that the coupling $\gamma$ lowers the effective energy barrier. 
The effect seems more uniform for $i_b = 0.1$, while for shallow barriers [e.g., see $i_b=0.8$ in Fig.~\ref{fig:QP}c)] the change is more evident only for $\gamma\gtrsim0.1$. In fact, the greater $i_b$, the higher the $\gamma$ value above which the coupling with the axion produces an effect on the quasipotential. Interestingly, the bias current, which actually represents an easily controllable parameter, was also proven to have a significant impact on the emerging of resonances in the switching times discussed in Ref.~\cite{Grimaudo22}.
It is therefore evident from our simulations that a lower bias current is more convenient. Moreover, a different $\gamma$ gives quite different quasipotential values, such that the overall effect of the coupling between JJ and axion is to reduce the effective height of the potential barrier. 
%\newpage



% Figure environment removed

\section{Conclusions}
\label{Conclusions}

It has been shown that if the direct interaction between a solid state superconducting device, i.e., a Josephson junction, and the dark matter candidate named axion is assumed, a modification of the response to noise of the former arises \cite{Grimaudo22,Gri23}. This aspect offers an opportunity for the detection of this elusive particle.
Using JJs to detect axions has been shown to be beneficial for several reasons. First, JJs are superconducting devices that can operate at very low temperatures, and, hence, at very low noise. Second, they are very fast elements, with typical characteristic frequencies from GHz to THz, and therefore large amount of data can be collected in a brief time. Third, some parameters of the Josephson device can be adjusted to tune the coupling with the axion. The bias current is a further degree of freedom that can be exploited to tune the effective barrier, as shown in Fig.~\ref{fig:QP}.

In a nutshell, axion signature can be sketched as follows: the JJ-axion interaction facilitates, in the presence of noise, the escapes of the Josephson phase from the superconducting to the finite-voltage state. 
This change can be described through the quasipotential, which is an effective energy barrier that summarizes the response to noise in the limit of small fluctuations. 
The introduction of the quasipotential allows to extrapolate the behavior at very low noise values, at which numerical simulations become prohibitively long. 
It is thus possible to reconstruct the response at low temperature through simulations performed at relatively high values of fluctuations, an advantage that has been already exploited in several applications, as for instance the Josephson voltage standard for which even very rare escapes are relevant to maintain the high accuracy required by metrological standards~\cite{Kau96}. 
Analogously, for weak signal detection it is important to mimic occurrence of rare events in a quite noisy environment~\cite{Pie21}.
In this work we have extended the method to the interaction between an underdamped JJ and an axion field. 
Within this framework, it has been possible to demonstrate that the interaction is summarized by a quasipotential, and to determine the behavior of this effect through the quasipotential as a function of the JJ-axion interaction. 
In particular, it has been established that the quasipotential depends upon the strength of the interaction, in the precise meaning that the stronger the interaction, the lower the quasipotential effective barrier.
An ideal experiment comparing the escape times of a JJ subject only to noise with those of a junction subject to the same noise and an axion field could  reveal the presence of axions. This would be evidenced by a decrease in the mean escape time, and the magnitude of this decrease would provide a quantitative estimate of the JJ-axion interaction.
Furthermore, the decrease should persist at any noise value, even if very small, i.e. as small as necessary to achieve the desired SNR. Finally, numerical simulations demonstrated that the observed behavior holds for different bias points, thus providing an additional tunable parameter for experimental setup.



A word of caution: the change in potential energy is only one of the ingredients for an accurate calculation of SNR, which requires the estimation of fluctuations for finite sampling, as previously done for Josephson-based single photon detection schemes~\cite{Pied21,Gua21,Piedjou21}, and provides additional information on the sample size needed to determine exclusion graphs or the residual operator characteristic~\cite{Add12,Filatrella23}.

A further refinement of this approach could be achieved through the \emph{principle of minimum available noise energy}~\cite{Kautz88}. Shortly, the idea could be to follow the path of the unperturbed JJ to determine the critical point, i.e., the separatrix between the two basins of attraction belonging to different stable points. By doing this for the deterministic noise-free system, it is possible to calculate the energy in the minimum and the minimum work required to bring the system from the initial stable point to this critical point, i.e., exactly the quasipotential. Any other work involving a different trajectory between these two stable points is larger than the actual quasipotential.
If the method provides a close estimate of the quasipotential through stochastic lengthy numerical calculations, it may be exploited to search the vast parameter space with straightforward deterministic analysis. 

To conclude, our study aims to provide further insights into the interplay between noise, switching dynamic of the JJ and available signal statistical properties, to enhance the understanding of axion JJ-based detector, and at the same time its robustness and reliability. 
Through the characterization of the noise-induced effects and the understanding of their implications, we wish to contribute to the development of better detectors and of quantum technology devices with improved performances.








%\bibliography{bibliofile}

%%%%%%%%%%%%%%%%%%%%%%%%%%%%%%%%%%%%%%%%%%
\vspace{6pt} 

%%%%%%%%%%%%%%%%%%%%%%%%%%%%%%%%%%%%%%%%%%
%% optional
%\supplementary{The following supporting information can be downloaded at:  \linksupplementary{s1}, Figure S1: title; Table S1: title; Video S1: title.}

% Only for the journal Methods and Protocols:
% If you wish to submit a video article, please do so with any other supplementary material.
% \supplementary{The following supporting information can be downloaded at: \linksupplementary{s1}, Figure S1: title; Table S1: title; Video S1: title. A supporting video article is available at doi: link.}

%%%%%%%%%%%%%%%%%%%%%%%%%%%%%%%%%%%%%%%%%%
\authorcontributions{Conceptualization, R.G. and C.G.; methodology, R.G., D.V., and C.G.; software, R.G.; validation, R.G., D.V. and C.G.; formal analysis, R.G.; investigation, R.G. and C.G.; resources, D.V.; data curation, R.G. and C.G.; writing---original draft preparation, G.F. and C.G.; writing---review and editing, R.G., D.V., B.S., A.T., G.F. and C.G.; visualization, R.G., D.V., B.S., A.T., G.F. and C.G.; supervision, D.V., B.S., G.F. and C.G; funding acquisition, D.V. All authors have read and agreed to the published version of the manuscript.}

%\funding{Please add: ``This research received no external funding'' or ``This research was funded by NAME OF FUNDER grant number XXX.'' and  and ``The APC was funded by XXX''. Check carefully that the details given are accurate and use the standard spelling of funding agency names at \url{https://search.crossref.org/funding}, any errors may affect your future funding.}

%\institutionalreview{In this section, you should add the Institutional Review Board Statement and approval number, if relevant to your study. You might choose to exclude this statement if the study did not require ethical approval. Please note that the Editorial Office might ask you for further information. Please add “The study was conducted in accordance with the Declaration of Helsinki, and approved by the Institutional Review Board (or Ethics Committee) of NAME OF INSTITUTE (protocol code XXX and date of approval).” for studies involving humans. OR “The animal study protocol was approved by the Institutional Review Board (or Ethics Committee) of NAME OF INSTITUTE (protocol code XXX and date of approval).” for studies involving animals. OR “Ethical review and approval were waived for this study due to REASON (please provide a detailed justification).” OR “Not applicable” for studies not involving humans or animals.}

%\informedconsent{Any research article describing a study involving humans should contain this statement. Please add ``Informed consent was obtained from all subjects involved in the study.'' OR ``Patient consent was waived due to REASON (please provide a detailed justification).'' OR ``Not applicable'' for studies not involving humans. You might also choose to exclude this statement if the study did not involve humans.

%Written informed consent for publication must be obtained from participating patients who can be identified (including by the patients themselves). Please state ``Written informed consent has been obtained from the patient(s) to publish this paper'' if applicable.}

\dataavailability{The data presented in this study are available on reasonable request from the corresponding authors.} 

\acknowledgments{R.G. acknowledges financial support from the PRIN Project PRJ-0232 - Impact of Climate Change on the biogeochemistry of Contaminants in the Mediterranean sea (ICCC). All the authors acknowledge the support of the Ministry of University and Research of Italian Government.}

\conflictsofinterest{The authors declare no conflict of interest.} 

%%%%%%%%%%%%%%%%%%%%%%%%%%%%%%%%%%%%%%%%%%
%% Optional
%\sampleavailability{Samples of the compounds ... are available from the authors.}

%% Only for journal Encyclopedia
%\entrylink{The Link to this entry published on the encyclopedia platform.}

\abbreviations{Abbreviations}{
The following abbreviations are used in this manuscript:\\

\noindent 
\begin{tabular}{@{}ll}
JJ & Josephson junction\\
RCSJ & resistively and capacitively shunted junction
\end{tabular}
}

%%%%%%%%%%%%%%%%%%%%%%%%%%%%%%%%%%%%%%%%%%


%%%%%%%%%%%%%%%%%%%%%%%%%%%%%%%%%%%%%%%%%%
\begin{adjustwidth}{-\extralength}{0cm}
%\printendnotes[custom] % Un-comment to print a list of endnotes

%\bibliography{bibliofile}

\begin{thebibliography}{999}

\bibitem[Preskill \em{et~al.}(1983)Preskill, Wise, and Wilczek]{Preskili83}
Preskill, J.; Wise, M.B.; Wilczek, F.
\newblock Cosmology of the invisible axion.
\newblock {\em Physics Letters B} {\bf 1983}, {\em 120},~127--132.
\newblock {\url{https://doi.org/https://doi.org/10.1016/0370-2693(83)90637-8}}.

\bibitem[Alesini \em{et~al.}(2020)Alesini, Babusci, Barone, B., Beretta,
  Bianchini, Castellano, Chiarello, Di~Gioacchino, Falferi, Felici, Filatrella,
  Foggetta, Gallo, Gatti, Giazotto, Lamanna, Ligabue, Ligato, Ligi, Maccarrone,
  Margesin, Mattioli, Monticone, Oberto, Pagano, Paolucci, Rajteri, Rettaroli,
  Rolandi, Spagnolo, Toncelli, and Torrioli]{Alesini20}
Alesini, D.; Babusci, D.; Barone, C.; B., B.; Beretta, M.M.; Bianchini, L.;
  Castellano, G.; Chiarello, F.; Di~Gioacchino, D.; Falferi, P.;  et~al.
\newblock Status of the SIMP Project: Toward the Single Microwave Photon
  Detection.
\newblock {\em Journal of Low Temperature Physics} {\bf 2020}, {\em 199},~348--
  354.
\newblock {\url{https://doi.org/10.1007/s10909-020-02381-x}}.

\bibitem[Rettaroli \em{et~al.}(2021)Rettaroli, Alesini, Babusci, Barone,
  Buonomo, Beretta, Castellano, Chiarello, Di~Gioacchino, Felici, Filatrella,
  Foggetta, Gallo, Gatti, Ligi, Maccarrone, Mattioli, Pagano, Tocci, and
  Torrioli]{Rettaroli21}
Rettaroli, A.; Alesini, D.; Babusci, D.; Barone, C.; Buonomo, B.; Beretta,
  M.M.; Castellano, G.; Chiarello, F.; Di~Gioacchino, D.; Felici, G.;  et~al.
\newblock Josephson Junctions as Single Microwave Photon Counters: Simulation
  and Characterization.
\newblock {\em Instruments} {\bf 2021}, {\em 5}.
\newblock {\url{https://doi.org/10.3390/instruments5030025}}.

\bibitem[Beck(2013)]{Bec13}
Beck, C.
\newblock Possible Resonance Effect of Axionic Dark Matter in Josephson
  Junctions.
\newblock {\em Phys. Rev. Lett.} {\bf 2013}, {\em 111},~231801.

\bibitem[Grimaudo \em{et~al.}(2022)Grimaudo, Valenti, Spagnolo, Filatrella, and
  Guarcello]{Grimaudo22}
Grimaudo, R.; Valenti, D.; Spagnolo, B.; Filatrella, G.; Guarcello, C.
\newblock Josephson-junction-based axion detection through resonant activation.
\newblock {\em Phys. Rev. D} {\bf 2022}, {\em 105},~033007.
\newblock {\url{https://doi.org/10.1103/PhysRevD.105.033007}}.

\bibitem[Grimaudo \em{et~al.}(2023)Grimaudo, Valenti, Filatrella, Spagnolo, and
  Guarcello]{Gri23}
Grimaudo, R.; Valenti, D.; Filatrella, G.; Spagnolo, B.; Guarcello, C.
\newblock Coupled quantum pendula as a possible model for
  Josephson-junction-based axion detection.
\newblock {\em Chaos, Solitons \& Fractals} {\bf 2023}, {\em 173},~113745.
\newblock {\url{https://doi.org/https://doi.org/10.1016/j.chaos.2023.113745}}.

\bibitem[Beck(2017)]{Bec17}
Beck, C.
\newblock Possible resonance effect of dark matter axions in SNS Josephson
  junctions.
\newblock {\em PoS} {\bf 2017}, {\em EPS-HEP2017},~058.

\bibitem[Hoffmann \em{et~al.}(2004)Hoffmann, Lefloch, Sanquer, and
  Pannetier]{Hof04}
Hoffmann, C.; Lefloch, F.; Sanquer, M.; Pannetier, B.
\newblock Mesoscopic transition in the shot noise of diffusive
  superconductor--normal-metal--superconductor junctions.
\newblock {\em Phys. Rev. B} {\bf 2004}, {\em 70},~180503.

\bibitem[Bae \em{et~al.}(2008)Bae, Dinsmore~III, Sahu, Lee, and
  Bezryadin]{Bae08}
Bae, M.H.; Dinsmore~III, R.C.; Sahu, M.; Lee, H.J.; Bezryadin, A.
\newblock Zero-crossing Shapiro steps in high-${T}_{c}$ superconducting
  microstructures tailored by a focused ion beam.
\newblock {\em Phys. Rev. B} {\bf 2008}, {\em 77},~144501.

\bibitem[He \em{et~al.}(2011)He, Wang, and Chan]{He11}
He, L.; Wang, J.; Chan, M.H.
\newblock Shapiro Steps in the Absence of Microwave Radiation.
\newblock {\em arXiv preprint arXiv:1107.0061} {\bf 2011}.

\bibitem[Golikova \em{et~al.}(2012)Golikova, H\"ubler, Beckmann, Batov,
  Karminskaya, Kupriyanov, Golubov, and Ryazanov]{Gol12}
Golikova, T.E.; H\"ubler, F.; Beckmann, D.; Batov, I.E.; Karminskaya, T.Y.;
  Kupriyanov, M.Y.; Golubov, A.A.; Ryazanov, V.V.
\newblock Double proximity effect in hybrid planar superconductor-(normal
  metal/ferromagnet)-superconductor structures.
\newblock {\em Phys. Rev. B} {\bf 2012}, {\em 86},~064416.

\bibitem[Bretheau \em{et~al.}(2013)Bretheau, Girit, Pothier, Esteve, and
  Urbina]{Bre13}
Bretheau, L.; Girit, {\c{C}}.{\"O}.; Pothier, H.; Esteve, D.; Urbina, C.
\newblock Exciting Andreev pairs in a superconducting atomic contact.
\newblock {\em Nature} {\bf 2013}, {\em 499},~312--315.

\bibitem[Wang \em{et~al.}(2022)Wang, Wang, and Wang]{Wang22}
Wang, J.; Wang, Z.; Wang, P.
\newblock Magnetic field enhanced critical current and subharmonic structures
  in dissipative superconducting gold nanowires.
\newblock {\em Quantum Frontiers} {\bf 2022}, {\em 1},~21.
\newblock {\url{https://doi.org/10.1007/s44214-022-00021-x}}.

\bibitem[Valenti \em{et~al.}(2012)Valenti, Denaro, La~Cognata, Spagnolo,
  Bonanno, Basilone, Mazzola, Zgozi, and Aronica]{Val12}
Valenti, D.; Denaro, G.; La~Cognata, A.; Spagnolo, B.; Bonanno, A.; Basilone,
  G.; Mazzola, S.; Zgozi, S.; Aronica, S.
\newblock Picophytoplankton Dynamics in Noisy Marine Environment.
\newblock {\em Acta Phys. Pol. B} {\bf 2012}, {\em 43},~1227.
\newblock {\url{https://doi.org/https://doi.org/10.5506/APhysPolB.43.1227}}.

\bibitem[Lisowski \em{et~al.}(2015)Lisowski, Valenti, Spagnolo, Bier, and
  Gudowska-Nowak]{Lis15}
Lisowski, B.; Valenti, D.; Spagnolo, B.; Bier, M.; Gudowska-Nowak, E.
\newblock Stepping molecular motor amid L\'evy white noise.
\newblock {\em Phys. Rev. E} {\bf 2015}, {\em 91},~042713.
\newblock {\url{https://doi.org/10.1103/PhysRevE.91.042713}}.

\bibitem[Valenti \em{et~al.}(2016)Valenti, Denaro, Spagnolo, Mazzola, Basilone,
  Conversano, Brunet, and Bonanno]{Valenti16}
Valenti, D.; Denaro, G.; Spagnolo, B.; Mazzola, S.; Basilone, G.; Conversano,
  F.; Brunet, C.; Bonanno, A.
\newblock Stochastic models for phytoplankton dynamics in Mediterranean Sea.
\newblock {\em Ecological Complexity} {\bf 2016}, {\em 27},~84--103.
\newblock Mathematical Ecology and Epidemiology,
  {\url{https://doi.org/https://doi.org/10.1016/j.ecocom.2015.06.001}}.

\bibitem[Valenti \em{et~al.}(2018)Valenti, Fazio, and Spagnolo]{Valenti18}
Valenti, D.; Fazio, G.; Spagnolo, B.
\newblock Stabilizing effect of volatility in financial markets.
\newblock {\em Phys. Rev. E} {\bf 2018}, {\em 97},~062307.
\newblock {\url{https://doi.org/10.1103/PhysRevE.97.062307}}.

\bibitem[Tobiska and Nazarov(2004)]{Tob04}
Tobiska, J.; Nazarov, Y.V.
\newblock Josephson Junctions as Threshold Detectors for Full Counting
  Statistics.
\newblock {\em Phys. Rev. Lett.} {\bf 2004}, {\em 93},~106801.
\newblock {\url{https://doi.org/10.1103/PhysRevLett.93.106801}}.

\bibitem[Pekola(2004)]{Pek04}
Pekola, J.P.
\newblock Josephson Junction as a Detector of Poissonian Charge Injection.
\newblock {\em Phys. Rev. Lett.} {\bf 2004}, {\em 93},~206601.
\newblock {\url{https://doi.org/10.1103/PhysRevLett.93.206601}}.

\bibitem[Ankerhold(2007)]{Ank07}
Ankerhold, J.
\newblock Detecting Charge Noise with a Josephson Junction: A Problem of
  Thermal Escape in Presence of Non-Gaussian Fluctuations.
\newblock {\em Phys. Rev. Lett.} {\bf 2007}, {\em 98},~036601.
\newblock {\url{https://doi.org/10.1103/PhysRevLett.98.036601}}.

\bibitem[Sukhorukov and Jordan(2007)]{Suk07}
Sukhorukov, E.V.; Jordan, A.N.
\newblock Stochastic Dynamics of a Josephson Junction Threshold Detector.
\newblock {\em Phys. Rev. Lett.} {\bf 2007}, {\em 98},~136803.
\newblock {\url{https://doi.org/10.1103/PhysRevLett.98.136803}}.

\bibitem[Timofeev \em{et~al.}(2007)Timofeev, Meschke, Peltonen, Heikkil\"a, and
  Pekola]{Tim07}
Timofeev, A.V.; Meschke, M.; Peltonen, J.T.; Heikkil\"a, T.T.; Pekola, J.P.
\newblock Wideband Detection of the Third Moment of Shot Noise by a Hysteretic
  Josephson Junction.
\newblock {\em Phys. Rev. Lett.} {\bf 2007}, {\em 98},~207001.
\newblock {\url{https://doi.org/10.1103/PhysRevLett.98.207001}}.

\bibitem[Huard \em{et~al.}(2007)Huard, Pothier, Birge, Esteve, Waintal, and
  Ankerhold]{Hua07}
Huard, B.; Pothier, H.; Birge, N.O.; Esteve, D.; Waintal, X.; Ankerhold, J.
\newblock Josephson junctions as detectors for non-Gaussian noise.
\newblock {\em Annalen der Physik} {\bf 2007}, {\em 16},~736--750.

\bibitem[Grabert(2008)]{Gra08}
Grabert, H.
\newblock Theory of a Josephson junction detector of non-Gaussian noise.
\newblock {\em Phys. Rev. B} {\bf 2008}, {\em 77},~205315.
\newblock {\url{https://doi.org/10.1103/PhysRevB.77.205315}}.

\bibitem[Filatrella and Pierro(2010)]{Fil10}
Filatrella, G.; Pierro, V.
\newblock Detection of noise-corrupted sinusoidal signals with Josephson
  junctions.
\newblock {\em Phys. Rev. E} {\bf 2010}, {\em 82},~046712.
\newblock {\url{https://doi.org/10.1103/PhysRevE.82.046712}}.

\bibitem[Addesso \em{et~al.}(2012)Addesso, Filatrella, and Pierro]{Add12}
Addesso, P.; Filatrella, G.; Pierro, V.
\newblock Characterization of escape times of Josephson junctions for signal
  detection.
\newblock {\em Phys. Rev. E} {\bf 2012}, {\em 85},~016708.
\newblock {\url{https://doi.org/10.1103/PhysRevE.85.016708}}.

\bibitem[Guarcello \em{et~al.}(2013)Guarcello, Valenti, Augello, and
  Spagnolo]{Gua13}
Guarcello, C.; Valenti, D.; Augello, G.; Spagnolo, B.
\newblock The Role of Non-Gaussian Sources in the Transient Dynamics of Long
  Josephson Junctions.
\newblock {\em Acta Phys. Pol. B} {\bf 2013}, {\em 44},~997--1005.
\newblock {\url{https://doi.org/10.5506/APhysPolB.44.997}}.

\bibitem[Guarcello \em{et~al.}(2019)Guarcello, Valenti, Spagnolo, Pierro, and
  Filatrella]{Gua19}
Guarcello, C.; Valenti, D.; Spagnolo, B.; Pierro, V.; Filatrella, G.
\newblock Josephson-based Threshold Detector for L\'evy-Distributed Current
  Fluctuations.
\newblock {\em Physical Review Applied} {\bf 2019}, {\em 11},~044078.

\bibitem[Guarcello \em{et~al.}(2020)Guarcello, Filatrella, Spagnolo, Pierro,
  and Valenti]{Gua20}
Guarcello, C.; Filatrella, G.; Spagnolo, B.; Pierro, V.; Valenti, D.
\newblock Voltage drop across Josephson junctions for L\'evy noise detection.
\newblock {\em Physical Review Research} {\bf 2020}, {\em 2},~043332.

\bibitem[Guarcello(2021)]{Gua21-2}
Guarcello, C.
\newblock L\'evy noise effects on Josephson junctions.
\newblock {\em Chaos Solitons Fract} {\bf 2021}, {\em 153},~111531.
\newblock {\url{https://doi.org/https://doi.org/10.1016/j.chaos.2021.111531}}.

\bibitem[Walsh \em{et~al.}(2017)Walsh, Efetov, Lee, Heuck, Crossno, Ohki, Kim,
  Englund, and Fong]{Wal17}
Walsh, E.D.; Efetov, D.K.; Lee, G.H.; Heuck, M.; Crossno, J.; Ohki, T.A.; Kim,
  P.; Englund, D.; Fong, K.C.
\newblock Graphene-Based Josephson-Junction Single-Photon Detector.
\newblock {\em Phys. Rev. Applied} {\bf 2017}, {\em 8},~024022.
\newblock {\url{https://doi.org/10.1103/PhysRevApplied.8.024022}}.

\bibitem[{Kuzmin} \em{et~al.}(2018){Kuzmin}, {Sobolev}, {Gatti}, {Di
  Gioacchino}, {Crescini}, {Gordeeva}, and {Il'ichev}]{Kuz18}
{Kuzmin}, L.S.; {Sobolev}, A.S.; {Gatti}, C.; {Di Gioacchino}, D.; {Crescini},
  N.; {Gordeeva}, A.; {Il'ichev}, E.
\newblock Single Photon Counter Based on a Josephson Junction at 14 GHz for
  Searching Galactic Axions.
\newblock {\em IEEE Transactions on Applied Superconductivity} {\bf 2018}, {\em
  28},~1--5.

\bibitem[Guarcello \em{et~al.}(2019{\natexlab{a}})Guarcello, Braggio, Solinas,
  and Giazotto]{GuaBra19}
Guarcello, C.; Braggio, A.; Solinas, P.; Giazotto, F.
\newblock Nonlinear Critical-Current Thermal Response of an Asymmetric
  Josephson Tunnel Junction.
\newblock {\em Phys. Rev. Applied} {\bf 2019}, {\em 11},~024002.

\bibitem[Guarcello \em{et~al.}(2019{\natexlab{b}})Guarcello, Braggio, Solinas,
  Pepe, and Giazotto]{GuaBraSol19}
Guarcello, C.; Braggio, A.; Solinas, P.; Pepe, G.P.; Giazotto, F.
\newblock Josephson-Threshold Calorimeter.
\newblock {\em Phys. Rev. Applied} {\bf 2019}, {\em 11},~054074.

\bibitem[Revin \em{et~al.}(2020)Revin, Pankratov, Gordeeva, Yablokov, Rakut,
  Zbrozhek, and Kuzmin]{Rev20}
Revin, L.S.; Pankratov, A.L.; Gordeeva, A.V.; Yablokov, A.A.; Rakut, I.V.;
  Zbrozhek, V.O.; Kuzmin, L.S.
\newblock Microwave photon detection by an Al Josephson junction.
\newblock {\em Beilstein Journal of Nanotechnology} {\bf 2020}, {\em
  11},~960--965.

\bibitem[Yablokov \em{et~al.}(2021)Yablokov, Glushkov, Pankratov, Gordeeva,
  Kuzmin, and Il’ichev]{Yab21}
Yablokov, A.; Glushkov, E.; Pankratov, A.; Gordeeva, A.; Kuzmin, L.;
  Il’ichev, E.
\newblock Resonant response drives sensitivity of Josephson escape detector.
\newblock {\em Chaos Solitons Fract} {\bf 2021}, {\em 148},~111058.
\newblock {\url{https://doi.org/https://doi.org/10.1016/j.chaos.2021.111058}}.

\bibitem[{Piedjou Komnang} \em{et~al.}(2021){Piedjou Komnang}, Guarcello,
  Barone, Gatti, Pagano, Pierro, Rettaroli, and Filatrella]{Pied21}
{Piedjou Komnang}, A.; Guarcello, C.; Barone, C.; Gatti, C.; Pagano, S.;
  Pierro, V.; Rettaroli, A.; Filatrella, G.
\newblock Analysis of Josephson junctions switching time distributions for the
  detection of single microwave photons.
\newblock {\em Chaos, Solitons \& Fractals} {\bf 2021}, {\em 142},~110496.
\newblock {\url{https://doi.org/https://doi.org/10.1016/j.chaos.2020.110496}}.

\bibitem[Guarcello \em{et~al.}(2021)Guarcello, Piedjou~Komnang, Barone,
  Rettaroli, Gatti, Pagano, and Filatrella]{Gua21}
Guarcello, C.; Piedjou~Komnang, A.S.; Barone, C.; Rettaroli, A.; Gatti, C.;
  Pagano, S.; Filatrella, G.
\newblock Josephson-Based Scheme for the Detection of Microwave Photons.
\newblock {\em Phys. Rev. Applied} {\bf 2021}, {\em 16},~054015.

\bibitem[Pankratov \em{et~al.}(2022{\natexlab{a}})Pankratov, Revin, Gordeeva,
  Yablokov, Kuzmin, and Il'ichev]{Pan22-1}
Pankratov, A.L.; Revin, L.S.; Gordeeva, A.V.; Yablokov, A.A.; Kuzmin, L.S.;
  Il'ichev, E.
\newblock Towards a microwave single-photon counter for searching axions.
\newblock {\em npj Quantum Information} {\bf 2022}, {\em 8},~61.
\newblock {\url{https://doi.org/10.1038/s41534-022-00569-5}}.

\bibitem[Pankratov \em{et~al.}(2022{\natexlab{b}})Pankratov, Gordeeva, Revin,
  Ladeynov, Yablokov, and Kuzmin]{Pan22-2}
Pankratov, A.L.; Gordeeva, A.V.; Revin, L.S.; Ladeynov, D.A.; Yablokov, A.A.;
  Kuzmin, L.S.
\newblock Approaching microwave photon sensitivity with Al Josephson junctions.
\newblock {\em Beilstein Journal of Nanotechnology} {\bf 2022}, {\em
  13},~582--589.

\bibitem[Graham and T\'el(1985)]{Graham85}
Graham, R.; T\'el, T.
\newblock Weak-noise limit of Fokker-Planck models and non differentiable
  potentials for dissipative dynamical systems.
\newblock {\em Phys. Rev. A} {\bf 1985}, {\em 31},~1109.

\bibitem[Kautz(1988)]{Kau88}
Kautz, R.L.
\newblock Thermally induced escape: The principle of minimum available noise
  energy.
\newblock {\em Physical Review A} {\bf 1988}, {\em 38},~2066--2080.

\bibitem[Kautz(1996)]{Kau96}
Kautz, R.L.
\newblock Noise, chaos, and the Josephson voltage standard.
\newblock {\em Reports on Progress in Physics} {\bf 1996}, {\em 59},~935--992.

\bibitem[Pountougnigni \em{et~al.}(2019)Pountougnigni, Yamapi, Filatrella, and
  Tchawoua]{Pou19}
Pountougnigni, O.V.; Yamapi, R.; Filatrella, G.; Tchawoua, C.
\newblock Noise and disorder effects in a series of birhythmic Josephson
  junctions coupled to a resonator.
\newblock {\em Phys. Rev. E} {\bf 2019}, {\em 99},~032220.
\newblock {\url{https://doi.org/10.1103/PhysRevE.99.032220}}.

\bibitem[Barone and Paterno(1982)]{Bar82}
Barone, A.; Paterno, G.
\newblock {\em Physics and applications of the Josephson effect}; Wiley, New
  York,  1982.

\bibitem[Likharev(1986)]{Lik86}
Likharev, K.
\newblock {\em Dynamics of Josephson Junctions and Circuits} {\bf 1986}.

\bibitem[Guarcello \em{et~al.}(2015)Guarcello, Valenti, and Spagnolo]{GuaVal15}
Guarcello, C.; Valenti, D.; Spagnolo, B.
\newblock Phase dynamics in graphene-based Josephson junctions in the presence
  of thermal and correlated fluctuations.
\newblock {\em Phys. Rev. B} {\bf 2015}, {\em 92},~174519.

\bibitem[Spagnolo \em{et~al.}(2017)Spagnolo, Guarcello, Magazz\'u, Carollo,
  Persano~Adorno, and Valenti]{Spa17}
Spagnolo, B.; Guarcello, C.; Magazz\'u, L.; Carollo, A.; Persano~Adorno, D.;
  Valenti, D.
\newblock Nonlinear Relaxation Phenomena in Metastable Condensed Matter
  Systems.
\newblock {\em Entropy} {\bf 2017}, {\em 19}.

\bibitem[McCumber(1968)]{McC68}
McCumber, D.E.
\newblock Effect of ac Impedance on dc Voltage‐Current Characteristics of
  Superconductor Weak‐Link Junctions.
\newblock {\em Journal of Applied Physics} {\bf 1968}, {\em 39},~3113--3118.

\bibitem[Beenakker(1992)]{Bee92}
Beenakker, C.W.J.
\newblock Three ``Universal'' Mesoscopic Josephson Effects.
\newblock In Proceedings of the Low-Dimensional Electronic Systems; Bauer, G.;
  Kuchar, F.; Heinrich, H., Eds.; Springer Berlin Heidelberg: Berlin,
  Heidelberg,  1992; pp. 78--82.

\bibitem[Kramers(1940)]{Kra40}
Kramers, H.
\newblock Brownian motion in a field of force and the diffusion model of
  chemical reactions.
\newblock {\em Physica} {\bf 1940}, {\em 7},~284 -- 304.
\newblock
  {\url{https://doi.org/http://dx.doi.org/10.1016/S0031-8914(40)90098-2}}.

\bibitem[{Piedjou Komnang} \em{et~al.}(2021){Piedjou Komnang}, Guarcello,
  Barone, Gatti, Pagano, Pierro, Rettaroli, and Filatrella]{Pie21}
{Piedjou Komnang}, A.; Guarcello, C.; Barone, C.; Gatti, C.; Pagano, S.;
  Pierro, V.; Rettaroli, A.; Filatrella, G.
\newblock Analysis of Josephson junctions switching time distributions for the
  detection of single microwave photons.
\newblock {\em Chaos Solitons Fract} {\bf 2021}, {\em 142},~110496.
\newblock {\url{https://doi.org/https://doi.org/10.1016/j.chaos.2020.110496}}.

\bibitem[Sikivie(1983)]{Sik83}
Sikivie, P.
\newblock Experimental Tests of the "Invisible" Axion.
\newblock {\em Phys. Rev. Lett.} {\bf 1983}, {\em 51},~1415--1417.

\bibitem[Visinelli(2013)]{Vis13}
Visinelli, L.
\newblock Axion-electromagnetic waves.
\newblock {\em Modern Physics Letters A} {\bf 2013}, {\em 28},~1350162.

\bibitem[Sikivie and Yang(2009)]{sik09}
Sikivie, P.; Yang, Q.
\newblock Bose-Einstein condensation of dark matter axions.
\newblock {\em Physical Review Letters} {\bf 2009}, {\em 103},~111301.

\bibitem[Duffy and van Bibber(2009)]{Duf09}
Duffy, L.D.; van Bibber, K.
\newblock Axions as dark matter particles.
\newblock {\em New Journal of Physics} {\bf 2009}, {\em 11},~105008.

\bibitem[Yan and Beck(2020)]{Yan20}
Yan, J.; Beck, C.
\newblock Nonlinear dynamics of coupled axion-Josephson junction systems.
\newblock {\em Physica D: Nonlinear Phenomena} {\bf 2020}, {\em 403},~132294.

\bibitem[Blackburn \em{et~al.}(2009)Blackburn, Marchese, Cirillo, and
  Gr{\o}nbech-Jensen]{Blac09}
Blackburn, J.A.; Marchese, J.E.; Cirillo, M.; Gr{\o}nbech-Jensen, N.
\newblock Classical analysis of capacitively coupled superconducting qubits.
\newblock {\em Physical Review B} {\bf 2009}, {\em 79},~054516.

\bibitem[Dubos \em{et~al.}(2001)Dubos, Courtois, Pannetier, Wilhelm, Zaikin,
  and Sch\"on]{Dub01}
Dubos, P.; Courtois, H.; Pannetier, B.; Wilhelm, F.K.; Zaikin, A.D.; Sch\"on,
  G.
\newblock Josephson critical current in a long mesoscopic S-N-S junction.
\newblock {\em Phys. Rev. B} {\bf 2001}, {\em 63},~064502.

\bibitem[Bergeret and Cuevas(2008)]{Ber08}
Bergeret, F.S.; Cuevas, J.C.
\newblock The Vortex State and Josephson Critical Current of a Diffusive SNS
  Junction.
\newblock {\em Journal of Low Temperature Physics} {\bf 2008}, {\em
  153},~304--324.

\bibitem[Du \em{et~al.}(2008)Du, Skachko, and Andrei]{Du08}
Du, X.; Skachko, I.; Andrei, E.Y.
\newblock Josephson current and multiple Andreev reflections in graphene SNS
  junctions.
\newblock {\em Phys. Rev. B} {\bf 2008}, {\em 77},~184507.

\bibitem[Graham and T\'el(1986)]{Graham86}
Graham, R.; T\'el, T.
\newblock Nonequilibrium potential for coexisting attractors.
\newblock {\em Phys. Rev. A} {\bf 1986}, {\em 33},~1322--1337.
\newblock {\url{https://doi.org/10.1103/PhysRevA.33.1322}}.

\bibitem[Risken(1989)]{Risken89}
Risken, H.
\newblock {\em The Fokker-Planck Equation: Methods of solution and
  Applications}; Springer, Berlin,  1989.

\bibitem[Kumar and Carroll(1984)]{Kumar84}
Kumar, B.V.K.V.; Carroll, C.W.
\newblock {Performance Of Wigner Distribution Function Based Detection
  Methods}.
\newblock {\em Optical Engineering} {\bf 1984}, {\em 23},~732 -- 737.
\newblock {\url{https://doi.org/10.1117/12.7973372}}.

\bibitem[Piedjou~Komnang \em{et~al.}(2021)Piedjou~Komnang, Guarcello, Barone,
  Pagano, and Filatrella]{Piedjou21}
Piedjou~Komnang, A.S.; Guarcello, C.; Barone, C.; Pagano, S.; Filatrella, G.
\newblock Analysis of Josephson Junction Lifetimes for the Detection of Single
  Photons in a Thermal Noise Background.
\newblock In Proceedings of the 2021 IEEE 14th Workshop on Low Temperature
  Electronics (WOLTE),  2021, pp. 1--4.
\newblock {\url{https://doi.org/10.1109/WOLTE49037.2021.9555447}}.

\bibitem[Filatrella \em{et~al.}(2023)Filatrella, Barone, Carapella, Gatti,
  Granata, Guarcello, Mauro, Komnang, Pierro, Rettaroli, and
  Pagano]{Filatrella23}
Filatrella, G.; Barone, C.; Carapella, G.; Gatti, C.; Granata, V.; Guarcello,
  C.; Mauro, C.; Komnang, A.P.; Pierro, V.; Rettaroli, A.;  et~al.
\newblock Theoretical and Numerical Estimate of Signal-to-Noise Ratio in the
  Analysis of Josephson Junctions Lifetime for Photon Detection.
\newblock {\em IEEE Transactions on Applied Superconductivity} {\bf 2023}, {\em
  33},~1--5.
\newblock {\url{https://doi.org/10.1109/TASC.2022.3214500}}.

\bibitem[Kautz(1988)]{Kautz88}
Kautz, R.L.
\newblock Thermally induced escape: The principle of minimum available noise
  energy.
\newblock {\em Phys. Rev. A} {\bf 1988}, {\em 38},~2066--2080.
\newblock {\url{https://doi.org/10.1103/PhysRevA.38.2066}}.

\end{thebibliography}



\PublishersNote{}
\end{adjustwidth}
\end{document}



\end{document}
