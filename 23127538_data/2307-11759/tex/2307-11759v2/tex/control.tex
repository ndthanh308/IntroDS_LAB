\chapter{Aerobat Modeling}
\label{chap:control}

This chapter describes the progress made towards developing a control model of Aerobat capable of executing trajectories. In order to do this, a model must be developed mapping between robot motion and control inputs to the actuators. \cite{sihite_unsteady_2022} makes progress towards this with a description of the aerodynamic model.

% Figure environment removed

The dynamic modeling is derived using an unsteady aerodynamic model from the Wagner model and lifting-line theory \cite{boutet_unsteady_2018}. Aerobat has 20 degrees of freedom, but due to the nature of the kinetic sculpture of Aerobat's mechanism, this can be reduced to just 7 degrees of freedom (6 for the body and 1 for the motor that controls the flapping) with the rest expressed as kinematic constraints.

The dynamical equation of motion used in the simulation can be derived using Euler-Lagrangian dynamical formulations. Figure \ref{fig:fbd} shows the free-body diagram of the robot, which can be presented using 5 bodies: main body, proximal and distal wings of both sides. The synchronized wing trajectory allows us to just use one side of the wing in the states.

Let $\bm q = [\bm p^\top, \bm \theta^\top, q_s, q_e]^\top$ be the generalized coordinates, where $\bm p$ is the body center of mass inertial position, $\bm \theta$ is the Euler angles of the body, $q_s$ and $q_e$ are the left wing's shoulder and elbow angles, respectively. The dynamical equation of motion of the simplified system can be defined as follows:
%
\begin{equation}
\begin{aligned}
    \bm M(\bm q) \, \ddot{\bm q} &= \bm h(\bm q, \dot{\bm q}) + \bm u_a + \bm u_t + \bm J_c^\top \bm \lambda  \\
    \bm J_c \, \ddot{\bm q} &= [\ddot q_s, \ddot q_e]^\top = \bm y_{ks}, 
\end{aligned}
\label{eq:dynamic_eom}
\end{equation}

\noindent where $M$ is the inertial matrix, $\bm h$ is the gravitational and Coriolis forces, $\bm u_a$ and $\bm u_t$ are the generalized aerodynamic and thruster forces, respectively. $\bm \lambda$ is the Lagrangian multiplier which enforces the constraint forces acting on $q_s$ and $q_e$ to track the KS flapping acceleration $\bm y_{ks}$. $\bm \lambda$ can be solved algebraically from \ref{eq:dynamic_eom} given the states $\bm x = [\bm q^\top, \dot{\bm q}^\top]^\top$ and both generalized forces $\bm u_a$ and $\bm u_t$. These generalized forces can be derived using virtual displacement, as follows:

\begin{equation}
\begin{aligned}
    \bm u_a &= \sum_{i=1}^{N_b} B_{a,i}(\bm q)\, \bm f_{a,i} \quad &
    \bm u_t &= \sum_{i=1}^{N_t} B_{t,i}(\bm q)\, \bm f_{t,i}
\end{aligned}
\label{eq:generalized_forces}
\end{equation}
%
where $B$ matrices map the forces $\bm f \in \mathbb{R}^3$ to the generalized coordinates $\bm q$, $N_b$ is  the number of blade elements, and $N_b$ is the number of thrusters. Let the position $\bm p_k(\bm q)$ be the inertial position where the force $\bm f_k$ defined in the inertial frame is applied. The matrix $B_k$ for this force can be derived as follows: $B_k = \left( \partial \dot{\bm p}_{k} / \partial \dot{\bm q} \right)^\top$. The aerodynamic forces generated on each blade elements and thrust forces are combined to form $\bm u_a$ and $\bm u_t$, respectively.

The aerodynamics can be derived using discrete blade elements following the derivations in \cite{boutet_unsteady_2018}. This model uses the lifting line theory and Wagner's function to develop a model for calculating the lift coefficient. Let $S$ be the total wingspan and $y \in [-S/2, S/2]$ represents a position along the wingspan. The vortex shedding distribution can be defined as a function of truncated Fourier series of size $m$ across the wingspan, as follows:
%
\begin{equation}
\begin{gathered}
    \Gamma(t,y) = \frac{1}{2} a_0 \, c_0 \, U \, \sum^{m}_{n=1} a_n(t) \, \sin(n\,\theta(y))
\end{gathered}
\end{equation}
%
\noindent where $a_n$ is the Fourier coefficients, $a_0$ is the slope of the angle of attack, $c_0$ is the chord length at wing's axis of symmetry, and $U$ is the free stream airspeed. Let $\theta$ be the change of variable defined by $y = (S/2)\cos(\theta)$ for describing a position along the wingspan $y \in (-S/2, S/2)$. From $\Gamma(t,y)$, we can derive the additional downwash induced by the vortices, defined as follows:
%
\begin{equation}
\begin{aligned}
    w_{y}(t,y) &
    = - \frac{a_0 c_0 U}{4S} \sum^{m}_{n=1} n a_n(t)  \frac{\sin(n \theta)}{\sin(\theta)}.
\end{aligned}
\label{eq:induced_downwash}
\end{equation}

Following the unsteady Kutta-Joukowski theorem, the sectional lift coefficient can be expressed as follows:
%
\begin{equation}
\begin{aligned}
    C_L(t,y) &= a_0 \sum^{m}_{n=1} \left( \frac{c_0}{c(y)} a_n(t) + \frac{c_0}{U} \dot{a}_n(t) \right) \sin(n\theta),
\end{aligned}
\label{eq:lift_coeff_fourier}
\end{equation}
%
where $c(y)$ is the chord length at the wingspan position $y$. The computation of the sectional lift coefficient response of an airfoil undergoing a step change in downwash $\Delta w(y) << U$ can be expressed using Wagner function $\Phi(t)$:
%
\begin{equation}
\begin{aligned}
    c_L(t,y) &= \frac{a_0}{U} \Delta w(t,y) \Phi(\tilde t) \\
    \Phi(\tilde t)    &= 1 - \psi_1 e^{-\epsilon_1 \tilde t} - \psi_2 e^{-\epsilon_2 \tilde t}
\end{aligned}
\label{eq:lift_coeff_wagner}
\end{equation}

where $\tilde t(t) = \int_0^t (v_e^i/b) dt$ is the normalized time which is defined as the distance traveled divided by half chord length ($b = c/2$). Here, $v_e^i$ is defined as the velocity of the quarter chord distance from the leading edge in the direction perpendicular to the wing sweep. For the condition where the freestream airflow dominates $v_e$, then we can approximate the normalized time as $\tilde t = Ut/b$. The Wagner model in \eqref{eq:lift_coeff_wagner} uses Jones' approximation \cite{boutet_unsteady_2018}, with the following coefficients: $\psi_1 = 0.165$, $\psi_2 = 0.335$, $\epsilon_1 = 0.0455$, and $\epsilon_2 = 0.3$.

Duhamel's principles can be used to superimpose the transient response due to a step change in downwash as defined in \eqref{eq:lift_coeff_wagner}. Additionally, integration by parts can be used to simplify the equation further, resulting in the following equation:
%
\begin{equation}
\begin{aligned}
    C_L(t,y) &= \frac{a_0}{U} \left( w(t,y) \Phi(0) - \int_{0}^{t} \frac{\partial \Phi(t - \tau)}{\partial \tau} w(\tau, y) d\tau \right).
\end{aligned}
\label{eq:aero_CL_base}
\end{equation}
%
\begin{equation}
\begin{aligned}
    \frac{\partial \Phi(t - \tau)}{\partial \tau} &=
    -\frac{\psi_1 \epsilon_1 U}{b} e^{-\frac{\epsilon_1 U}{b}(t-\tau)}
    -\frac{\psi_2 \epsilon_2 U}{b} e^{-\frac{\epsilon_2 U}{b}(t-\tau)}
\end{aligned}
\label{eq:partial_phi}
\end{equation}
%
Here, $w(t,y)$ is the total downwash defined as:
%
\begin{equation}
    w(t,y) = v_n(t,y) + w_y(t,y),
\label{eq:total_downwash}
\end{equation}
%
where $v_n$ is the airfoil velocity normal to the wing surface which depends on the freestream velocity and the inertial dynamics. Finally, we can represent the integrals as the following states:
%
\begin{equation}
\begin{aligned}
    z_{1} (t,y) &= \int_{0}^{t} \frac{\psi_1 \epsilon_1 U}{b} e^{-\frac{\epsilon_1 U}{b}(t-\tau)} w(\tau,y) d\tau
    \\
    z_{2} (t,y) &= \int_{0}^{t} \frac{\psi_2 \epsilon_2 U}{b} e^{-\frac{\epsilon_2 
    U}{b}(t-\tau)} w(\tau,y) d\tau.
\end{aligned}
\label{eq:aero_states_z}
\end{equation}
%
Both of these states can be expressed as an ODE by deriving the time derivatives of \eqref{eq:aero_states_z}. They can be derived using Leibniz integral rule, yielding the following equations:
%
\begin{equation}
\begin{aligned}
    \dot z_{1} (t,y) &= \frac{\psi_1 \epsilon_1 U}{b} \left( w(t,y) - \frac{\epsilon_1 U}{b} z_1(t,y) \right) \\
    \dot z_{2} (t,y) &= \frac{\psi_2 \epsilon_2 U}{b} \left( w(t,y) - \frac{\epsilon_2 U}{b} z_2(t,y) \right).
\end{aligned}
\label{eq:aero_states_dz}
\end{equation}
%
The sectional lift coefficient can then be defined as:
%
\begin{equation}
\begin{aligned}
    c_L(t,y)  = \frac{a_0}{U} \left( w(t,y) \phi(0) + z_1(t,y) + z_2(t,y) \right),
\end{aligned}
\label{eq:aero_CL_final}
\end{equation}
%
and we can march the aerodynamic states $z_1$ and $z_2$ forward in time using \eqref{eq:aero_states_dz}. Finally, we can relate the both sectional lift coefficient equations in \eqref{eq:lift_coeff_fourier} and \eqref{eq:aero_CL_final} to solve for the Fourier coefficient rate of change, $\dot{a}_n$. 

The aerodynamic states are defined along the span of the wing and can be discretized into $m$ blade elements. Therefore, we can derive the $m$ equations relating \eqref{eq:lift_coeff_fourier} and \eqref{eq:aero_CL_final} on each blade element to solve for the $\dot{a}_n$. Then, including $z_1$ and $z_2$ on each blade elements, we will have $3m$ ODE equations to solve. We can represent $a_n$, $z_1$, and $z_2$ of all blade elements as the vector $\bm a_n \in \mathbb{R}^{m}$, $\bm {z}_1 \in \mathbb{R}^{m}$, and $\bm z_2 \in \mathbb{R}^{m}$, respectively. 

This model was simulated in \cite{sihite_bang-bang_2022} (Fig. \ref{fig:simulation_results}) and partially validated by the IMU data from untethered flight tests . However, to fully validate the model and close the loop, a more controlled testing setup is required.

% Figure environment removed

\section{Validation of Aerodynamic Model}

% Figure environment removed

Using RISE Arena, the aerodynamic model presented in \cite{sihite_unsteady_2022} was validated. Aerobat was set to flap at a fixed known frequency of about 2 Hz and load cell measurements were taken for headwind speeds of 0.5, 1.0, and 1.5 m/s. The results closely match the simulation, validating this model (Fig. \ref{fig:validation}) .


\section{Concluding Remarks}


This chapter presented the aerodynamic model of Aerobat and the steps taken towards validating it using the newly setup RISE Arena (Section \ref{sec:rise}), taking Aerobat one step closer to closed-loop control. Future development will be focused towards system identification and addition of more degrees of actuation into the wings, allowing Aerobat to control roll and pitch dynamics. 
% --- EOF ---
