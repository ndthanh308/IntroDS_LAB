\chapter{Towards untethered flight}
\label{chap:untethered}

Northeastern University's Aerobat is a project in development since 2016. \cite{sihite_mechanism_2020, sihite_enforcing_2020, hoff_optimizing_2018, hoff_reducing_2017, ramezani_biomimetic_2017} describe the development of the mechanical structure and actuation mechanism. \cite{sihite_unsteady_2022, ghanem_efficient_2021} describe the development of simulation models and trajectory planning. \cite{sihite_unsteady_2022} achieved tethered hovering flight indoors using these models on our indoor tethered test platform Aerobat Gamma.

The next stage of development was focused towards developing a second version of Aerobat, called Aerobat Beta for testing untethered flight outdoors.

% Figure environment removed

Aerobat Beta was designed and built as part of an earlier Master's thesis presented in \cite{hu_bang-bang_2022}. As a test platform, Aerobat Beta has lightweight laser-cut foam wings that are easily replaceable. The original goal of Aerobat Beta was to test lift-generation capabilities in isolation, and to that end, stabilizers were added to stabilize the roll and pitch axes in flight, allowing the wings to generate lift based on an open loop PWM signal sent to the motor. Stabilization was carried out using a simple PD controller that read acceleration and gyroscope values from an onboard IMU to calculate roll and pitch. PWM signal data and IMU data were relayed to a ground-station computer over Bluetooth for debugging. All this was controlled onboard by an Arduino Pico micro-controller weighing 1g, with 24kB of memory.

At the start of the work presented in this thesis, Aerobat Beta was flying with intermittent success over short 3-5m distances. Testing was carried out indoors and the main focus was on increasing consistency of flight. The primary source of flight inconsistencies was found stem from the gear mechanism that keeps the two wings in sync. Additional inconsistencies came from poorly calibrated ESCs for the stabilizers and imbalanced weight. After strengthening 3D printed parts, cleaning up the gear mechanism and calibrating the ESCs, more consistent flight was observed, until finally 5-7m untethered flight was consistently achievable indoors. Figure \ref{fig:indoor-untethered} shows one such flight. The snapshots show untethered flight before Aerobat hits the safety net, showing orientation correction in the process. 

% Figure environment removed

The modifications that allowed this to happen served only as temporary fixes and necessitated constant maintenance of the hardware to keep Aerobat in fly-worthy condition. As an early test platform, however, this was acceptable at the time and testing was continued. With consistent flight demonstrated indoors, Aerobat was taken outdoors for longer distance flights than could be executed in the indoor space available. 

Outdoor tests pose an additional challenge in the form of wind. Without closed loop control and only orientation based stabilization, testing can be difficult. However, with intermittent consistency, 10m outdoor flight was demonstrated. Figure \ref{fig:outdoor-untethered} shows one such flight, again showing Aerobat correcting undesired roll to continue flying.


% Figure environment removed

This result sufficiently demonstrated lift generation capabilities of Aerobat Beta, and focus was shifted towards a long term fix for the gear mechanism and development of closed loop control. Chapter \ref{chap:control} describes the progress made towards development of closed-loop control. The rest of this chapter, however, will be dedicated to describing the development carried out to enable the work in Chapter \ref{chap:control} and beyond.

\section{Towards Safe Testing of Untethered Flight}
\label{sec:guard}

One of the issues faced while testing Aerobat outdoors was crashes. With foam wings and no protection, each crash would lead to large reset times, allowing for only a few tests to be conducted in a given time period. As more aspects of control are developed, the ability to perform multiple repeatable tests quickly will become very important. To this end, a guard design was proposed that would protect Aerobat in the event of crashes, reduce reset times and allow a large number of tests to be carried out.

% Figure environment removed

Figure \ref{fig:guard} shows the proposed guard design with Aerobat mounted at the center. It has been named Kongming Lamp after the traditional Chinese lantern for it's distinctive shape and the safety it represents for Aerobat. Consisting of three concentric ellipses covering each of the three axes, this is designed to be a lightweight compliant addition that protects the robot in the event of a crash. Made of 11 lightweight carbon fiber rods, the structure provides strength and elasticity that would absorb impact in a crash. The rods are bound together by small snap-fit 3D printed parts that are optimized to reduce the weight to the minimum required. To test the strength of the guard, it was drop tested to see how a load at the center equivalent to the robot would survive. Figure \ref{fig:drop-test} shows the compliance of the structure absorbing the impact and protecting the representative weight.

% Figure environment removed

An additional modification made in the interest of testing more advanced control is shifting the stabilizers from Aerobat to the guard and providing the guard with its own IMU. Having the guard independently stabilized isolates the robot from the guard dynamics and allows it to be used as much or as little as needed. Eventually, these stabilizers and the guard itself will be phased out and Aerobat will be robust enough to fly on its own. Figure \ref{fig:guard} shows the full guard design with stabilizers and IMU.

The guard is stabilized with the help of four BLDC motors arranged in a quad-copter-like configuration. The control algorithm for the guard runs on Aerobat's processor and uses feedback from its own IMU for independent control. Within RISE Arena, it is fitted with markers and tracked using Optitrack Motion Capture to provide pose information to the controller. Figure \ref{fig:guard_control} shows the controller logic used to stabilize the guard. For simplicity, only the roll and pitch orientations of the guard are stabilized, and velocity in only the x and y directions is considered. Altitude control will be part of future development.

% Figure environment removed

Stabilizing the guard is challenging due to the compliant nature of the structure. The motors and IMU are mounted on snap-fit 3D printed parts that may slide along the carbon fiber rod. The rods themselves also stretch over time and the relative positioning between the motors is not rigid. This leads to challenges in tuning the controls for the guard as it needs to be robust enough to compensate for all these inconsistencies.

\section{Robotics-Inspired Study and Experimentation (RISE) Arena}
\label{sec:rise}

In order to develop controls for Aerobat, a fully controlled and repeatable environment is required where each aspect of Aerobat's dynamics may be isolated and individually studied. It needs a safe environment to test and tune controls in a rigorous and repeatable manner before it is ready to be taken outdoors for fully untethered flight.

% Figure environment removed

The Robotics-Inspired Study and Experimentation (RISE) Arena was created to provide this controlled test environment. Figure \ref{fig:rise} shows the setup of RISE Arena. At the center of it is the indoor tethered test platform Aerobat Gamma (Fig. \ref{fig:gamma}). Aerobat Gamma is a tethered version of Aerobat with flexible electronics in its wings. It is mounted on a highly sensitive ATI 6-axis load cell (shown in Fig. \ref{fig:gamma_closeup}). The robot and the load cell together are mounted at the end of a programmable 6 DOF manipulator. One side of RISE Arena is entirely covered by a large array of fans that can generate wind speeds of up to 2 m/s and the whole area is covered by 6 Optitrack Motion Capture Cameras.



% Figure environment removed

% Figure environment removed

The robotic arm offers the ability to create trajectories with precise ground truth information available and do highly repeatable experiments. The arm is interfaced through Ethernet using a Python API. A wrapper was developed for the API that added new functionality, making it easier to interface with the arm, generate trajectories and execute predefined movements. Using the wrapper, keyboard teleoperation of the arm was developed, allowing a user to move the arm to any location and save the coordinates as waypoints in a trajectory. The waypoints may be saved and fed to different programs that execute different trajectories, controlling the duration and smoothness of the trajectory, and the number of loops of the trajectory to execute. It also enables setting protection zones (Fig. \ref{fig:protection_zones}) to protect the arm and the robot from collisions within RISE arena, allowing safe testing of controls.

% Figure environment removed

From the motion capture cameras and the load cell, RISE Arena provides ground truth for flapping frequency, robot pose, lift generated, and aerodynamic forces on the robot, allowing controlled motion and pose within known stable wind conditions, making this a powerful tool for testing.

RISE Arena has been used throughout this work, from validating the the aerodynamic model to testing the guard controller to calibrating sensors and testing perception.

\section{Concluding remarks}

In this chapter, the preliminary results for untethered flight was presented with successful indoor and outdoor flight tests demonstrating a proof-of-concept for untethered flight. These flights were open loop. Future development will be focused towards developing closed loop control, with initial steps for this described in Chapter \ref{chap:control}. To better enable testing controls in closed loop flight, this chapter also describes the development of Kongming Lamp, a lightweight protective guard around Aerobat to save it from crashes and stabilize it while controls are being tuned. Finally, this chapter describes the development of indoor test setup RISE Arena, providing elaborate ground truth and a controlled repeatable environment for system identification and testing of controls. RISE Arena is far from a finished product, with many developments planned, including "free flight" of the robot while still attached to the manipulator using admittance control, incorporating more precise aerodynamic sensing and wind pattern detection and adding offboard processing to test more experimental and advanced algorithms. 