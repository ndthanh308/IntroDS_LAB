\chapter{Conclusion}
\label{chap:conclude}

This thesis presents the progress made towards Autonomous Untethered Flight on Northeastern University's Aerobat. This was broken down into three primary goals with the progress towards each described in their own chapter.

\section{Chapter \ref{chap:untethered}}
Chapter \ref{chap:untethered} described progress made towards untethered flight. A proof-of-concept 10m outdoor untethered flight was demonstrated and two additional development was presented, towards enabling future testing for untethered flight. The first of these was the protective guard, Kongming Lamp (Sec. \ref{sec:guard}), which was drop tested with a representative weight at the center to demonstrate protection for Aerobat from crashes. The second development was RISE Arena (Sec. \ref{sec:rise}), providing elaborate ground truth information for controlled and repeatable testing and system identification.

\subsection{Thesis Contributions}
For the outdoor untethered flight demonstration, stability of flight was improved by tuning the complementary filter applied to calculate orientation from IMU for stabilization. For the design of Kongming Lamp, in addition to conceptual inputs, control code for stabilization using IMU and pose information was developed and tuned. In addition, RISE Arena was fully developed as a part of this thesis, including interfacing and control code for the manipulator, calibration of Optitrack system and integration of processing and sensing onto Aerobat-Gamma.

\section{Chapter \ref{chap:control}}

Chapter \ref{chap:control} described the aerodynamic model of Aerobat and the steps taken towards validating the model. Preliminary results indicate the model is accurate, but further system identification is required to fully map out the control system and experimentally test the model in-flight under different wind conditions. Predicated on the success of this, outdoor closed-loop tests may be performed with the help of Kongming Lamp (Sec. \ref{sec:guard}) until Aerobat is ready to fly completely unsupported. 

\subsection{Thesis Contributions}
Created the manipulator trajectories and performed the experiment using RISE Arena to collect data for validation of the aerodynamic model.

\section{Chapter \ref{chap:perception}}

Chapter \ref{chap:perception} described the progress made towards onboard perception and state estimation. Processors and sensors were selected and integrated onto the robot (Sec. \ref{sec:electronics}). Sensor drivers were written and iteratively optimized for timing issues and speed of processing (Sec. \ref{sec:sensor_integration}). ROS was installed and tested on the limited processing power available on Aerobat and preliminary data for VIO was collected with the help of RISE Arena. As an immediate next goal, this data will be run through different VIO algorithms to verify the quality of the data and benchmark the algorithms.

\subsection{Thesis Contributions}
This chapter describes research and development fully carried out as part of this thesis.

\section{Future Work}
This work will be continued as part of my doctoral study, and as further progress is made on each of these goals, Aerobat will be at a mature stage where technology demonstrations may be made such as:
\begin{itemize}
    \item Controlled near ground flight akin to birds and bats demonstrating higher efficiency of near ground flight
    \item Long distance flights demonstrating the high efficiency of flapping wing systems
    \item Autonomous flight within a straight tunnel demonstrating precise closed loop control in confined areas
    \item Autonomous flight within a tunnel maze demonstrating the high agility of flapping wing systems and their ability to open up previously inaccessible spaces
\end{itemize}