\chapter{Introduction}
\label{chap:introduction}

Flapping Wing aerial locomotion is an interesting field of study that is gaining a lot of research interest \cite{di_luca_bioinspired_2017, eguiluz_towards_2019, phan_kubeetle-s_2019, chukewad_robofly_2021}. Flapping robots offer a number of advantages over conventional aerial robots such as quad-copters, which rely on propeller based lift generation. The biggest of these is their ability to fly in confined spaces. Quad-copters and other multi-rotor vehicles are heavily affected by turbulent air flow when flying in confined spaces or close to the ground \cite{matus-vargas_ground_2021}. On the other hand, flapping wing robots have the opposite effect, not only being able to fly in tight spaces aided by their high agility, but also showing higher efficiency when flying close to the ground, a phenomenon well studied in birds \cite{rayner_aerodynamics_1991}. This makes flapping wing robots a huge potential asset for applications in disaster management, for example flying through the narrow spaces inside a collapsed building, for applications in inspection such as flying through sewers or air vents that are inaccessible to humans and other types of robots, or even for data collection for scientific research in previously inaccessible areas. A further advantage of flapping wing robots is their relative safety to operate. With soft deformable wings and significantly smaller weight density, they are not only safer than propeller based aerial robots to operate around people, they are less affected by crashes into walls or ceilings and can continue flying. And finally, flapping wing robots are extremely agile, able to perform zero momentum turns, and are more efficient in their agility when compared with multi-rotor systems that rely on thrust vectoring for their agility, which is very power hungry. \cite{de_croon_flapping_2020, tu_acting_2019}

For all these advantages, however, flapping wing robots still pose a number of challenges that must be solved before they may fully reach the impact that multi-rotors have had. Flapping wing systems generate much less thrust when compared to multi-rotors of similar size. This severely impacts the available payload for sensors and other electronics that would enable the robot to be fully autonomous. Further, these are highly dynamic platforms, with flapping motions causing vibrations that an onboard perception system must deal with \cite{eguiluz_towards_2019}. Also, unlike multi-rotors, flapping systems have a constantly shifting center of mass, affected not only by the wing position, but also by the variable deformations in the wings and any inherent compliance in their structure due to their lightweight designs. These factors make localization and autonomous control of the robot a challenge.

In order to develop autonomous flight, two things are required:
\begin{enumerate}
    \item \textbf{Low Level Control}: The ability to track any desired trajectory and accurately execute any desired motion
    \item \textbf{High level control}: The ability to decide what trajectory or motion to execute based on knowledge about the robot state and it's surroundings. High level control may be further divided into two sub-goals:
    \begin{enumerate}
        \item \textbf{Perception and State Estimation}: Understand the surrounding environment and localize the robot within this space
        \item \textbf{Trajectory Planning}: Decide a trajectory to follow based on the perception and state estimation
    \end{enumerate}
\end{enumerate}

All of these are eventual goals for Aerobat. However, this work focuses on making progress towards Perception, State Estimation and Low-level control.

The thesis is organized according to these goals as follows. Chapter \ref{chap:related} goes through contemporary works on aerial and flapping wing systems, focusing specifically on works that have had success with autonomous flight. Chapter \ref{chap:untethered} describes initial results with open loop untethered flight and the development made towards safe and controlled testing of untethered flight. Chapter \ref{chap:control} describes the progress made towards low level control of Aerobat, describing the aerodynamic model of Aerobat and validation of the aerodynamic model. Chapter \ref{chap:perception} describes the progress made towards developing onboard perception and state estimation, with a special focus on the limited payload capacity available and the challenges in implementation on limited computation hardware. Finally, Chapter \ref{chap:conclude} presents an overview of the milestones reached, challenges faced and future development to take place towards untethered autonomous flight.

\section{About Aerobat}

Northeastern University's Aerobat is a tail-less flapping wing robot that, unlike existing examples, is capable of significantly morphing it's wing structure during each gait cycle. The robot, with a weight of 40g (when carrying a battery and a basic microcontroller) and a wingspan of 30 cm, was initially developed to study the flapping-wing flight of bats.

Aerobat utilizes a computational structure, called the \textit{Kinetic Sculpture} (KS) \cite{sihite_computational_2020}, that introduces computational resources for wing morphing. The KS is designed to actuate the robot's wings as it is split into two wing segments: the proximal and distal wings, which are actuated by what is the equivalent of shoulder and elbow joints, respectively. The gait captures the wing folding during the upstroke motion, which is one of the key modes in bat flight. The wing folding reduces the wing surface area and minimizes the negative lift during the upstroke and results in a more efficient flight. Aerobat is capable of flapping at a frequency of up to 8 Hz. Without a tail, Aerobat is unstable in its longitudinal (pitch dynamics) and frontal (roll dynamics) planes of flight. 
% --- EOF ---
