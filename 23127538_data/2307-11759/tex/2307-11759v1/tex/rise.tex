\chapter{RISE Arena: Towards Untethered Autonomous Flight}
\label{chap:rise_arena}

In order to develop controls for Aerobat, a fully controlled and repeatable environment is required where each aspect of Aerobat's dynamics may be isolated and individually studied. It needs a safe environment to test and tune controls in a rigorous and repeatable manner before it is ready to be taken outdoors for fully untethered flight.

% Figure environment removed

RISE Arena was created to provide this controlled test environment. Figure \ref{fig:rise} shows the setup of RISE Arena. At the center of it is the indoor tethered test platform Aerobat Gamma (Fig. \ref{fig:gamma}). Aerobat Gamma is a tethered version of Aerobat with flexible electronics in its wings. It is mounted on a highly sensitive ATI 6-axis load cell (shown in Fig. \ref{fig:gamma_closeup}). The robot and the load cell together are mounted at the end of a programmable 6 DOF manipulator. One side of RISE Arena is entirely covered by a large array of fans that can generate wind speeds of up to 2 m/s and the whole area is covered by 6 Optitrack Motion Capture Cameras.



% Figure environment removed

% Figure environment removed

The robotic arm offers the ability to create trajectories with precise ground truth information available and do highly repeatable experiments. The arm is interfaced through Ethernet using a Python API. A wrapper was developed for the API that added new functionality, making it easier to interface with the arm, generate trajectories and execute predefined movements. Using the wrapper, keyboard teleoperation of the arm was developed, allowing a user to move the arm to any location and save the coordinates as waypoints in a trajectory. The waypoints may be saved and fed to different programs that execute different trajectories, controlling the duration and smoothness of the trajectory, and the number of loops of the trajectory to execute. It also enables setting protection zones (Fig. \ref{fig:protection_zones}) to protect the arm and the robot from collisions within RISE arena, allowing safe testing of controls.

% Figure environment removed

From the motion capture cameras and the load cell, RISE Arena provides ground truth information for flapping frequency, robot pose, lift generated, and aerodynamic forces on the robot, allowing controlled motion and pose within known stable wind conditions, making this a powerful tool for testing.

RISE Arena has been used throughout this work, from validating the the aerodynamic model to testing the guard controller to calibrating sensors and testing perception.

\section{Next Steps}

The next step towards understanding the dynamics of Aerobat is to be able to fly Aerobat safely within RISE Arena. Before it can do this unassisted, an intermediate step is to use the robotic arm for support. The robotic arm is capable of force based control, and using feedback from the load cell, the arm may be moved around to minimize this force