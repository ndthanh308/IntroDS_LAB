\section{Related work}
% The prediction of epitopes and paratopes, the binding sites on the antigen and antibody, respectively, is a fundamental problem in protein-protein interaction. The prediction of one strongly influences the prediction of the other. Therefore, many methods have been proposed for predicting both epitopes and paratopes simultaneously \cite{del2021neural, PiNet}.

The structure of proteins provides crucial information about the location and orientation of the binding sites. Various approaches have been taken in the literature to address the task of epitope and paratope prediction, including sequential \cite{liberis2018parapred,deac2019fastparapred} and structural \cite{krawczyk2014improving,del2021neural} methods. 
Furthermore, Geometric deep learning has emerged as a powerful tool for predicting protein-protein interactions \cite{isert2023structure}, with graph-based representations being one of the most common approaches \cite{tubiana2022scannet,stark2022equibind}. These methods leverage the geometric information of the molecules to learn complex relationships between epitopes and paratopes. For instance, some approaches \cite{del2021neural,da2022epitope3d} use the graph structure to compute features based on neighbouring residues, which are then aggregated to highlight the most probable region of interaction.

An alternative approach is to represent proteins as surfaces. % which is an effective way to capture the geometric properties of the epitope and paratope.
MaSIF \cite{gainza2020MaSif} focuses on the more general problem of protein interaction region prediction and uses a surface representation learned through convolutions defined on the surface.
PiNet \cite{PiNet} represents the protein surface as a point cloud and employs PointNet \cite{qi2017pointnet} to classify points as interacting or not. On the contrary, \citet{zhang2023equipocket} model the surface of a molecule as a graph and apply an equivariant graph neural network (EGNN, \cite{satorras2021n}) for binding site prediction. 

Integrating structural and geometric information has proven to be a promising approach for improving protein interaction prediction. Still, few studies have focused on the specific case of epitope and paratope prediction \cite{cia2023critical}. Our work supports this view by showing that considering the problem as a geometric one can effectively improve performance.
