\section{Conclusions}

%Our O-GEP model with diffusion layers and HKS achieves state-of-the-art performance in predicting epitope and paratope, demonstrating the importance of incorporating geometric information in the model.

We investigated the effectiveness of geometric deep learning techniques in predicting antibody-antigen interactions. Our results indicate that incorporating geometric information is crucial for accurately predicting epitope and paratope regions. Specifically, the use of invariant representation in I-GEP models outperformed previous models, and O-GEP models with diffusion layers and additional geometric features achieved state-of-the-art performance. Our study highlights the potential of geometric deep learning in computational biology. Future research could explore using spectral shape analysis to address the more complex problem of conformational rearrangement in antigen-antibody binding \cite{stanfield1994major}.
% Future work could explore the use of more complex geometrical features or incorporate other types of data to further improve the accuracy of interaction prediction models.% Our findings suggest that considering the inner and outer structures of proteins and leveraging their intrinsic geometry can lead to significant improvements in prediction accuracy. 
% Additionally, the qualitative examples shown in Figures \ref{fig:mol_res}, provide evidence of the effectiveness of our proposed methods.