
\section{Data}

Comparing methods across different molecular representations is crucial for advancing research in molecular modelling. We developed a reusable pipeline that generates a dataset to evaluate methods using inner and outer structure representations. 
% Our pipeline computes the surface of each molecule and maps residues to corresponding points on the surface. This allows for easy evaluation and comparison of methods relying on inner and outer structures.

We collected a dataset of 133 protein complexes from Epipred \cite{krawczyk2014improving}, with 103 for training and 30 for testing. The training and test sets have been selected to share no more
than 90\% pairwise sequence identity. The PDB files were obtained from the Sabdab database \cite{dunbar2014sabdab}. In the test set, 7.8\% of antigen residues were labelled as positive. Additionally, we used a separate set of 27 protein complexes from PECAN derived from a subset of the Docking Benchmark v5 \cite{vreven2015updates} to validate our results.

% The datasets used for training and testing were taken from Epipred \cite{krawczyk2014improving} for a total of 133 complexes (103 in the training set and 30 in the test set. The PDB files for these sets were acquired from the Sabdab database \cite{dunbar2014sabdab}. In the test set, 7.8\% of the antigen residues were classified as part of the positive class.
% A separate set of 27 complexes from PECAN was used to validate the results. This set was created from a subset of the Docking Benchmark v5 \cite{vreven2015updates}.
% The dataset is split into a training set of 103 complexes and a test set of 30 complexes.
% The training and validation set contains 132 complexes, while the test set contains 30 complexes.
%The CDR regions of the antibodies are utilized to provide precise contextual information. 


We construct a residue graph (Figure \ref{fig:graph_res}) for each protein, where a 28-dimensional physicochemical feature vector represents each residue. This vector comprises a one-hot encoding of the amino acid (including 20 possible types and one for an unknown type), in addition to seven other features representing the physical, chemical, and structural properties of the amino acid type. These additional features can be considered a fixed embedding, as described in \cite{meiler2001generation}.

For each protein, we generated a surface mesh (Figure \ref{fig:surf_res}) using the PyMOL API with a 1.4~\AA~water probe radius. We associated each point on the protein's surface with a residue by finding the closest atom to that point. 
This association was then used to transfer the feature of each residue to the points on the surface.
% This mapping allows for easy evaluation and comparison of methods relying on inner and outer structures.
% This procedure ensures the features are properly aligned and localized on the protein's surface, allowing for accurate predictions and analysis.

 
 

