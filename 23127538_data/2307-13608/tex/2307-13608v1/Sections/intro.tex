\section{Introduction}


% Antibodies are proteins in the immune system that bind to pathogens (antigens) to detect and neutralize them. Identifying their binding region is key to developing vaccines and synthetic antibodies. However, current state of the art technic to determine the residues which are spatially close in the 3D bound structure, rely on time consuming and expensive experimental methods such as X-ray crystallography \cite{smyth2000x}. Therefore, computational techniques are needed to tackle these problems and facilitate the speedy development of therapeutics. This need was further highlighted during the recent epidemic of  COVID- 19, where it was shown that  mutations in the antigen caused changes to the binding mechanism \cite{thomson2021circulating}, potentially impacting the efficacy of existing treatments. The use of a computational tool would enable rapid assessment of the effects of these mutations, thereby keeping pace with the swiftly emerging mutants and expediting the development of therapeutics.

% Predicting the binding site of an antibody-antigen interaction can be formulated as a binary classification problem where the goal is determining whether each amino acid residue in the antibody and antigen participates in the binding. For the prediction of epitopes,  the region on the antigen that interacts with the 
% antibody, the entire antigen needs to be considered. In contrast, for the prediction of the paratope, the area on the antibody that interacts with the antigen only requires a strongly localized region at the tips of the Y-shaped antibody. Therefore sequential method using the amino acid residue sequence to describe the proteins has seen much more success for the paratope prediction while remaining difficult for the epitope.

Identifying the binding sites of antibodies is essential for developing vaccines and synthetic antibodies. These binding sites, called paratopes, can bind to antigens, wherein the corresponding binding site is known as the epitope, thus neutralizing harmful foreign molecules in the body.  %The antigen region that binds to the localized region of the antibody, called the paratope, is known as the epitope. 
Experimental methods for determining the residues that belong to the paratope and epitope are time-consuming and expensive, highlighting the need for computational tools to facilitate the rapid development of therapeutics. The recent COVID-19 epidemic highlighted this need further, as mutations in the antigen were shown to impact the binding mechanism, potentially reducing the efficacy of existing treatments \cite{thomson2021circulating}. Predicting the binding sites of an antibody-antigen interaction requires considering the entire antigen for epitope prediction and a localized region of the antibody, known as the Complementarity-Determining Region (CDR), for paratope prediction. 
% Labelling the relevant residues can be formulated as a binary classification problem.


The integration of geometric and structural information in protein-to-protein interaction studies has led to significant progress \cite{stark2022equibind, PiNet}.  While several methods have concentrated on the 3D graph representation, few methods \cite{PiNet, zhang2023equipocket} have investigated the 3D surface representation. We aim to assess the impact of utilizing the geometric representation of the antigen and antibody in the task of epitope-paratope prediction.
Our approach, GEP (Geometric Epitope-Paratope) Prediction, proposes different geometric representations of the molecules to create accurate predictors for predicting antibody-antigen binding sites.
% We show that considering the surface instead of the inner structure can effectively enhance the accuracy of epitope-paratope site prediction. Our analysis  demonstrates the potential of surface-based epitope prediction, highlighting its value as a complementary approach to 3D graph-based methods.
The use of geometrical information is further justified by the emergence of technology predicting the single-protein structure, such as AlphaFold 2 \cite{jumper2021highly}, which has comparable accuracy to experimental methods. 
 We present the following contributions in our paper:
\begin{itemize}
\item We analyze the significance of geometric information within the context of graph learning, using equivariant layers that enable more robust and accurate predictions. 
\item Additionally, we fully exploit the geometric information in molecules by representing them as surfaces and applying techniques based on spectral geometry, leading to state-of-the-art performance.
\item We will release a pipeline for generating a dataset from PDB molecules that produces molecular representations in graph and surface formats, enabling cross-method comparisons.
\end{itemize}