\section{Results}
In this section, we report the results of our experiments and demonstrate the contribution of geometric information on the task of epitope-paratope prediction.

% \begin{table}[h]
% \caption{\label{tab:graph_res} Results on graph methods}
% \vskip 0.15in
% \centering
% \begin{sc}
% \scriptsize
% \setlength{\tabcolsep}{3pt}
% \begin{tabular}{l c c | c c}
%  & \multicolumn{2}{c}{Antigen} & \multicolumn{2}{c}{Antibody} \\ \cmidrule(lr){2-3} \cmidrule(lr){4-5}
%  & MCC& AUC ROC& MCC& AUC ROC\\ \toprule

% % egat&$ 0.02 \pm 0.04$&$ 0.51 \pm 0.06$&$ 0.05 \pm 0.04$&$ 0.60 \pm 0.05$\\
% % gat&$ 0.07 \pm 0.06$&$ 0.60 \pm 0.05$&$ 0.22 \pm 0.09$&$ 0.67 \pm 0.05$\\ \hline
% EPMP&$ 0.09 \pm 0.01$&$ 0.61 \pm 0.01$&$ 0.39 \pm 0.02$&$ 0.79 \pm 0.01$\\
% EPMP$_{nc}$&$ 0.10 \pm 0.02$&$ 0.62 \pm 0.01$&$ 0.35 \pm 0.02$&$ 0.78 \pm 0.00$\\
% EPMP$_c$&$ 0.10 \pm 0.01$&$ 0.63 \pm 0.01$&$ 0.38 \pm 0.02$&$ 0.79 \pm 0.01$\\ \hline
% EEPMP &$ 0.12 \pm 0.05$&$ 0.65 \pm 0.05$&$ 0.39 \pm 0.16$&$ 0.79 \pm 0.10$\\ 
% EEPMP\_l2&$\mathbf{ 0.15} \pm 0.02$&$ \mathbf{0.69} \pm 0.03$&$ \mathbf{0.49} \pm 0.06$&$ \mathbf{0.86} \pm 0.02$\\
% % egnn\_shared&$ 0.06 \pm 0.07$&$ 0.56 \pm 0.07$&$ 0.33 \pm 0.11$&$ 0.83 \pm 0.02$\\
% \end{tabular}
% \end{sc}
% \end{table}

\begin{table}[t]
\caption{\label{tab:graph_res} Results from I-GEP models}
\vskip 0.01in
\centering
\begin{sc}
\scriptsize
\setlength{\tabcolsep}{3pt}
\begin{tabular}{l c c c c}
 & MCC& AUC ROC& AUC PR\\ \cmidrule(lr){2-4}
 & \multicolumn{3}{c}{Antigen}\\ \toprule
 EPMP&$ 0.09 \pm 0.01$&$ 0.61 \pm 0.01$&$ 0.12 \pm 0.00$\\ \midrule[0.2pt]
EPMP$_{xyz}$&$ 0.10 \pm 0.01$&$ 0.63 \pm 0.01$&$ 0.15 \pm 0.01$ & \\ 
$E(n)$-EPMP&$ \mathbf{0.14 \pm 0.01}$&$ \mathbf{0.68 \pm 0.02}$&$ \mathbf{0.16 \pm 0.01}$\\ \midrule
% EEPMP$_{l2}$&$\mathbf{ 0.15} \pm 0.02$&$ \mathbf{0.69} \pm 0.03$&\\ 
 & \multicolumn{3}{c}{Antibody} \\  \toprule
 EPMP&$ 0.39 \pm 0.02$&$ 0.79 \pm 0.01$&$ 0.53 \pm 0.01$\\ \midrule[0.2pt]
EPMP$_{xyz}$&$ 0.38 \pm 0.02$&$ 0.79 \pm 0.01$&$ 0.53 \pm 0.01$\\ 
$E(n)$-EPMP&$ \mathbf{0.44 \pm 0.11}$&$ \mathbf{0.82 \pm 0.07}$&$ \mathbf{0.60 \pm 0.10}$\\\midrule
% EEPMP$_{l2}$&$ \mathbf{0.49} \pm 0.06$&$ \mathbf{0.86} \pm 0.02$ & \\\midrule
\end{tabular}
\end{sc}
\vskip -0.02in
\end{table}
\textbf{I-GEP results}
We conducted experiments to evaluate the effectiveness of incorporating geometric information by comparing our proposed models from Section \ref{sec:IGEP} with the EPMP model proposed in \cite{del2021neural}. Our results, presented in Table \ref{tab:graph_res}, clearly demonstrate that the inclusion of geometric information leads to a meaningful increase in performance. Specifically, the use of the $E(n)$ invariant layer ({\sc $E(n)$-EPMP}) resulted in an improvement in all metrics for both antibody and antigen.

% \begin{table*}[h]
% \caption{\label{tab:surface_res}Results on surface methods}
% \begin{tiny}
% \begin{sc}
% \centering
% \setlength{\tabcolsep}{2pt}
% \begin{tabular}{l c c  c c | c c  c c}
%  & \multicolumn{4}{c}{Surface} & \multicolumn{4}{c}{Residuals} \\ \cmidrule(lr){2-5} \cmidrule(lr){6-9}
%  & \multicolumn{2}{c}{Antigen} & \multicolumn{2}{c}{Antibody} & \multicolumn{2}{c}{Antigen} & \multicolumn{2}{c}{Antibody} \\ \cmidrule(lr){2-3} \cmidrule(lr){4-5} \cmidrule(lr){6-7} \cmidrule(lr){8-9}
%  & MCC& AUC ROC& MCC& AUC ROC & MCC& AUC ROC& MCC& AUC ROC \\ \toprule
%  PiNet&$ 0.41 \pm 0.04$&$ 0.88 \pm 0.02$&$ 0.19 \pm 0.09$&$ 0.67 \pm 0.04$&$ 0.39 \pm 0.05$&$ 0.89 \pm 0.01$&$ 0.26 \pm 0.12$&$ 0.77 \pm 0.03$\\
% PiNet hks&$ 0.32 \pm 0.05$&$ 0.86 \pm 0.01$&$ 0.15 \pm 0.05$&$ 0.63 \pm 0.00$&$ 0.30 \pm 0.04$&$ 0.87 \pm 0.02$&$ 0.22 \pm 0.05$&$ 0.74 \pm 0.00$\\ \midrule

% DiffNet&$ 0.45 \pm 0.01$&$ 0.89 \pm 0.00$&$ 0.13 \pm 0.01$&$ 0.65 \pm 0.07$&$ 0.44 \pm 0.01$&$ 0.89 \pm 0.00$&$ 0.19 \pm 0.03$&$ 0.75 \pm 0.05$\\
% DiffNet hks&$ 0.43 \pm 0.02$&$ 0.89 \pm 0.01$&$ 0.23 \pm 0.07$&$ 0.69 \pm 0.02$&$ 0.43 \pm 0.03$&$ 0.90 \pm 0.01$&$ 0.29 \pm 0.08$&$ 0.79 \pm 0.02$\\ \midrule
% DiffNet mesh&$ 0.44 \pm 0.02$&$ 0.89 \pm 0.01$&$ 0.20 \pm 0.06$&$ 0.68 \pm 0.03$&$ 0.43 \pm 0.03$&$ 0.89 \pm 0.01$&$ 0.25 \pm 0.09$&$ 0.78 \pm 0.02$\\
% DiffNet mesh hks&$ 0.44 \pm 0.04$&$ 0.89 \pm 0.01$&$ 0.21 \pm 0.06$&$ 0.67 \pm 0.04$&$ 0.44 \pm 0.06$&$ 0.90 \pm 0.01$&$ 0.26 \pm 0.07$&$ 0.77 \pm 0.03$\\ \midrule
% \end{tabular}
% \end{sc}
% \end{tiny}
% \end{table*}

% \begin{table}[h]
% \caption{\label{tab:surface_res}Results on surface methods}
% \vskip 0.15in
% \centering
% \begin{sc}
% \scriptsize
% \setlength{\tabcolsep}{3pt}
% \begin{tabular}{l c c c| c c c}
%  & \multicolumn{3}{c}{Antigen} & \multicolumn{3}{c}{Antibody} \\ \cmidrule(lr){2-4} \cmidrule(lr){5-7}
%  & MCC& AUC ROC& AUC PR& MCC& AUC ROC& AUC PR\\ \toprule
% DiffNet&$ 0.44 \pm 0.01$&$ 0.89 \pm 0.00$&$ 0.49 \pm 0.02$&$ 0.19 \pm 0.03$&$ 0.75 \pm 0.05$&$ 0.47 \pm 0.06$\\
% DiffNet hks&$ 0.43 \pm 0.03$&$ 0.90 \pm 0.01$&$ 0.49 \pm 0.02$&$ 0.29 \pm 0.08$&$ 0.79 \pm 0.02$&$ 0.56 \pm 0.04$\\ \midrule
% DiffNet mesh&$ 0.43 \pm 0.03$&$ 0.89 \pm 0.01$&$ 0.48 \pm 0.05$&$ 0.25 \pm 0.09$&$ 0.78 \pm 0.02$&$ 0.52 \pm 0.04$\\
% DiffNet mesh hks&$ 0.44 \pm 0.06$&$ 0.90 \pm 0.01$&$ 0.47 \pm 0.08$&$ 0.26 \pm 0.07$&$ 0.77 \pm 0.03$&$ 0.52 \pm 0.05$\\ \midrule
% PiNet&$ 0.39 \pm 0.05$&$ 0.89 \pm 0.01$&$ 0.44 \pm 0.02$&$ 0.26 \pm 0.12$&$ 0.77 \pm 0.03$&$ 0.52 \pm 0.08$\\
% PiNet hks&$ 0.30 \pm 0.04$&$ 0.87 \pm 0.02$&$ 0.37 \pm 0.06$&$ 0.22 \pm 0.05$&$ 0.74 \pm 0.00$&$ 0.47 \pm 0.02$\\ \midrule

% \end{tabular}
% \end{sc}
% \end{table}


\begin{table}[t]
\caption{\label{tab:surface_res}Results from O-GEP models. In addition to the physicochemical features, we test different combination of geometric information: 3d coordinates {\sc (xyz)} and Heat Kernel Signature {\sc (HKS)}. For the {\sc DiffNet} models, we consider both the point cloud ($_{pc}$) and the mesh ($_{m}$) of the surface.}
\vskip 0.00in
\centering
\begin{sc}
\scriptsize
\setlength{\tabcolsep}{3pt}
\begin{tabular}{l c c c}
 & MCC& AUC ROC& AUC PR\\ \cmidrule(lr){2-4}
 & \multicolumn{3}{c}{Antigen}\\ \toprule
 PiNet {\tiny(xyz)} &$ 0.39 \pm 0.05$&$ 0.89 \pm 0.01$&$ 0.44 \pm 0.02$\\
PiNet {\tiny(xyz+hks)}&$ 0.30 \pm 0.04$&$ 0.87 \pm 0.02$&$ 0.37 \pm 0.06$\\ \midrule
DiffNet$_{pc}$ {\tiny(xyz)}&$ 0.41 \pm 0.06$&$ \mathbf{0.90 \pm 0.01}$&$ 0.49 \pm 0.02$\\
DiffNet$_{pc}$ {\tiny(hks)}&$ 0.07 \pm 0.05$&$ 0.66 \pm 0.02$&$ 0.14 \pm 0.01$\\ 
DiffNet$_{pc}$ {\tiny(xyz+hks)} &$ \mathbf{0.44 \pm 0.03}$&$ \mathbf{0.90 \pm 0.01}$&$ \mathbf{0.50 \pm 0.02}$\\ \midrule
DiffNet$_{m}$ {\tiny(xyz)}&$ 0.42 \pm 0.03$&$ \mathbf{0.90 \pm 0.01}$&$ 0.48 \pm 0.05$\\
DiffNet$_{m}$ {\tiny(hks)}&$ 0.09 \pm 0.02$&$ 0.64 \pm 0.02$&$ 0.14 \pm 0.01$\\ 
DiffNet$_{m}$ {\tiny(xyz+hks)}&$ 0.42 \pm 0.06$&$ \mathbf{0.90 \pm 0.01}$&$ 0.46 \pm 0.07$\\\midrule
 & \multicolumn{3}{c}{Antibody} \\  \midrule
 PiNet {\tiny(xyz)}&$ 0.26 \pm 0.12$&$ 0.77 \pm 0.03$&$ 0.52 \pm 0.08$\\
PiNet {\tiny(xyz+hks)}&$ 0.22 \pm 0.05$&$ 0.74 \pm 0.00$&$ 0.47 \pm 0.02$\\ \midrule
DiffNet$_{pc}$ {\tiny(xyz)}&$ 0.30 \pm 0.06$&$ 0.79 \pm 0.01$&$ 0.56 \pm 0.03$\\ 
DiffNet$_{pc}$ {\tiny(hks)}&$ 0.44 \pm 0.03$&$ \mathbf{0.85 \pm 0.00}$&$ 0.68 \pm 0.01$\\
DiffNet$_{pc}$ {\tiny(xyz+hks)}&$ 0.23 \pm 0.06$&$ 0.77 \pm 0.04$&$ 0.51 \pm 0.05$\\\midrule
DiffNet$_{m}$ {\tiny(xyz)}&$ 0.24 \pm 0.08$&$ 0.78 \pm 0.02$&$ 0.52 \pm 0.03$\\
DiffNet$_{m}$ {\tiny(hks)}&$ \mathbf{0.49 \pm 0.01}$&$ \mathbf{0.85 \pm 0.00}$&$ \mathbf{0.69 \pm 0.01}$\\ 
DiffNet$_{m}$ {\tiny(xyz+hks)}&$ 0.28 \pm 0.06$&$ 0.77 \pm 0.02$&$ 0.52 \pm 0.04$\\ \midrule
\end{tabular}
\end{sc}
\end{table}
\textbf{O-GEP results}
To test the performance of O-GEP models, we consider the methods proposed in Section \ref{sec:OGEP} with different combinations of input features.
The results are summarized in Table \ref{tab:surface_res}. Incorporating diffusion layers ({\sc DiffNet}) along with 3D coordinates and Heat Kernel Signature as additional features consistently outperformed the baseline method {\sc PiNet}. % I-GEP models in predicting epitopes. 
The use of these techniques led to an MCC score twice as high as that obtained by the I-GEP models. However, unlike epitope prediction, the paratope prediction did not show the same level of improvement with O-GEP models. In this case, the best results were achieved by considering only the HKS features and diffusion layers.


% \newcommand\wc{0.2\textwidth}

% % Figure environment removed

\newcommand\wc{0.23\textwidth}
% Figure environment removed


\textbf{Qualitative results} 
The qualitative examples shown in Figure \ref{fig:mol_res} clearly demonstrate the improved performance of O-GEP models over I-GEP.
Figure \ref{fig:graph_res} shows the results of the \textit{$E(n)$-EPMP} on the residual graph.  
% The epitope prediction is concentrated on the spiky edge of the molecule, while the paratope prediction is focused on the extremity closest to the antigen. 
The epitope prediction focuses on sparse regions of the antigene, such as the spiky edges. In contrast, paratope prediction concentrates on the residues closest to the antigen.
In Figure \ref{fig:surf_res}, the predictions of {\sc DiffNet$_{pc}$ {\tiny(xyz+hks)}} are shown on both the surface and residues of the molecules. The predictions are highly localized on the region nearest to the binding molecule.
It's worth noticing that the 3d coordinates given as input to the models are centred and randomly rotated, providing no prior knowledge of the binding region.



