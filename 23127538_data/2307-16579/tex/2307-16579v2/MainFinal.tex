\documentclass[lettersize,journal]{IEEEtran}
\usepackage{amsmath,amsfonts}
%\usepackage{algorithmic}
\usepackage{algorithmicx}
\usepackage{algorithm}
\usepackage{array}
\usepackage[caption=false,font=normalsize,labelfont=sf,textfont=sf]{subfig}
\usepackage{textcomp}
\usepackage{stfloats}
\usepackage{url}
\usepackage{verbatim}
\usepackage{enumitem}
\usepackage{graphicx}
\usepackage{cite}
\hyphenation{op-tical net-works semi-conduc-tor IEEE-Xplore}
% updated with editorial comments 8/9/2021

% extra packages
\usepackage{booktabs}
\usepackage{amssymb}
\usepackage{multirow}
\usepackage{csquotes}
\usepackage{paralist}
\usepackage{threeparttable}
\usepackage{colortbl}

\newcommand{\figref}[1]{Fig.~\ref{#1}}
\newcommand{\equref}[1]{Eq.~\eqref{#1}}
\newcommand{\secref}[1]{Sec.~\ref{#1}}
\newcommand{\tabref}[1]{Table~\ref{#1}}


\newcommand\eg{\emph{e.g.}} \newcommand\Eg{\emph{E.g.}}
\newcommand\ie{\emph{i.e.}} \newcommand\Ie{\emph{I.e.}}
\newcommand\ien{\emph{i.e. }} \newcommand\Ien{\emph{I.e. }}
\newcommand\cf{\emph{c.f.}} \newcommand\Cf{\emph{C.f.}}
\newcommand\etc{\emph{etc. }}
\newcommand\wrt{w.r.t.} \newcommand\dof{d.o.f.}
\newcommand\etal{\emph{et al.}}

% \newcommand{\secref}[1]{Sec. \ref{#1}}
\usepackage{algpseudocode}
\usepackage[pagebackref,breaklinks,colorlinks]{hyperref}
\newcommand{\red}[1]{{\textcolor{red}{#1}}} % ranking the first
\newcommand{\blue}[1]{{\textcolor{blue}{#1}}} % ranking the second  

\def\Jing#1{{\color{magenta}{\bf [Jing:} {\it{#1}}{\bf ]}}}
\def\MYX#1{{\color{red}{\bf [Yuxin:} {\it{#1}}{\bf ]}}}
\def\YC#1{{\color{blue}{\bf [Yuchao:} {\it{#1}}{\bf ]}}}
\newcommand{\rev}[1]{{\textcolor{blue}{#1}}}

\begin{document}

\title{Contrastive Conditional Latent Diffusion for Audio-visual Segmentation}


\author{Yuxin Mao,~
Jing Zhang*,~
Mochu Xiang,~
Yunqiu Lv,~
Dong Li,~
Yiran Zhong,~
Yuchao Dai*\\
\IEEEcompsocitemizethanks{\IEEEcompsocthanksitem Yuxin Mao, Jing Zhang, Mochu Xiang, Yunqiu Lv, and Yuchao Dai are with School of Electronics and Information, Northwestern Polytechnical University, and Shaanxi Key Laboratory of Information Acquisition and Processing, Xi'an, China.
% \IEEEcompsocthanksitem Jing Zhang is with School of Computing, the Australian National University, Canberra, Australia.
\IEEEcompsocthanksitem Dong Li and Yiran Zhong are with Shanghai AI Laboratory, China.
\IEEEcompsocthanksitem
% Y. Lv and J. Zhang contributed equally. J. Zhang (zjnwpu@gmail.com) and 
Jing Zhang (zjnwpu@gmail.com) and Yuchao Dai (daiyuchao@nwpu.edu.cn) are the corresponding authors. 
\IEEEcompsocthanksitem This research was supported in part by National Natural Science Foundation of China (62271410, 12150007) and by the Fundamental Research Funds for the Central Universities.
Yuxin Mao is sponsored by the Innovation Foundation for Doctoral Dissertation of Northwestern Polytechnical University (CX2024014).

% \IEEEcompsocthanksitem The source code and experimental results are publicly available via our project page: \url{xx}.
}
}

\newcommand{\toreviewer}[1]{\vspace{0.1em}\noindent \textcolor{blue}{\textbf{#1 \hspace{0.1em}}}}
\def\Fix#1{{\color{black}{{#1}}}}
\def\Fixtwo#1{{\color{black}{{#1}}}}


% The paper headers
\markboth{Journal of \LaTeX\ Class Files,~Vol.~14, No.~8, August~2021}%
{Shell \MakeLowercase{\textit{et al.}}: A Sample Article Using IEEEtran.cls for IEEE Journals}

%\IEEEpubid{0000--0000/00\$00.00~\copyright~2021 IEEE}
% Remember, if you use this you must call \IEEEpubidadjcol in the second
% column for its text to clear the IEEEpubid mark.

\maketitle

\begin{abstract}
% The task of AVS 
Audio-visual Segmentation (AVS) is conceptualized as a conditional generation task, where audio is considered as the conditional variable for segmenting the sound producer(s). In this case, audio should be extensively explored to maximize its contribution for the final segmentation task. We propose a contrastive conditional latent diffusion model for audio-visual segmentation (AVS) to thoroughly investigate the impact of audio, where the correlation between audio and the final segmentation map is modeled to guarantee the strong correlation between them.
To achieve semantic-correlated representation learning, our framework incorporates a latent diffusion model. 
The diffusion model learns the conditional generation process of the ground-truth segmentation map, resulting in ground-truth aware inference during the denoising process at the test stage. As our model is conditional, it is vital to ensure that the conditional variable contributes to the model output.
We thus extensively model the contribution of the audio signal by minimizing the density ratio between the conditional probability of the multimodal data, \eg~conditioned on the audio-visual data, and that of the unimodal data, \eg~conditioned on the audio data only.
In this way, our latent diffusion model via density ratio optimization explicitly maximizes the contribution of audio for AVS, which can then be achieved with contrastive learning as a constraint, where the diffusion part serves as the main objective to achieve maximum likelihood estimation, and the density ratio optimization part imposes the constraint. 
By adopting this latent diffusion model via contrastive learning, we effectively enhance the contribution of audio for AVS. The effectiveness of our solution is validated through experimental results on the benchmark dataset.
Code and results are online via our project page: \url{https://github.com/OpenNLPLab/DiffusionAVS}.



% We propose a latent diffusion model with contrastive learning for audio-visual segmentation (AVS) to thoroughly investigate the impact of audio. 
% The task of AVS is conceptualized as conditional generation, where audio is considered the conditional variable for segmenting sound producer(s). The correlation between audio and the final segmentation map is modeled to ensure its contribution to this new approach.
% To achieve semantic-correlated representation learning, our framework incorporates a latent diffusion model. This diffusion model learns the conditional generation process of the ground-truth segmentation map, resulting in ground-truth aware inference during the denoising process at the test stage. As our model is conditional, it is vital to ensure that the conditional variable contributes to the model output.
% Furthermore, contrastive learning is incorporated into our framework to capture audio-visual correspondence. This approach consistently maximizes the mutual information between model predictions and audio data.
% By adopting this latent diffusion model via contrastive learning, we effectively enhance the contribution of audio for AVS. The effectiveness of our solution is validated through experimental results on the benchmark dataset.
% Code and results are online via our project page: \url{https://github.com/OpenNLPLab/DiffusionAVS}.

\end{abstract}

\begin{IEEEkeywords}
Audio-visual segmentation, Conditional latent diffusion model, Contrastive learning.
\end{IEEEkeywords}


% \section{Reviews}
% \toreviewer{To Reviewer 1:}\\
% \noindent\textbf{{R1.1}: This framework is based on the baseline [1] but adds the diffusion module. The diffusion process is operated on the latent embedding of the ground truth mask, can the diffusion model be directly used on the ground truth image?}

% \noindent\textbf{{R1.2}: Can the contrastive loss also work with hat(z\_0)? This may constrain the denoised hat(z\_0) to be consistent with the audio-visual condition. The denoised hat(z\_0) is expected to restore the ground truth information, does the learned hat(z\_0) belonging to the same class in the S4 subset have a close/similar distribution in the feature space?}

% \noindent\textbf{{R1.3}: There are still some typos to be fixed, such as 'per-trained' at Line 623, and repeated encoder at Line 679.}

% \toreviewer{To Reviewer 2:}\\
% \noindent\textbf{{R2.1}: What is the application of AVS in real world? In addition to the definition of the task, it is necessary to explain the usefulness of the research. Also, in the case of applications that must operate in real-time, it is necessary to verify and analysis the amount of computational cost.}


% \noindent\textbf{{R2.2}: Diffusion models are generally slow than pixel-wise classification models including segmentation networks due to step-wise repetition in inference. Is it reasonable computation?}

% \noindent\textbf{{R2.3}: The diffusion-based segmentation methods should be clearly compared with the proposed method, in regard to the novelty, architecture design and so on.}

% \noindent\textbf{{R2.4}: What is the performance gap according to the denoise step?}

% \noindent\textbf{{R2.5}: It is interesting to learn the distance function of features through continuous learning, but the performance improvement is poor than expected.}


% \toreviewer{To Reviewer 3:}\\
% \noindent\textbf{{R3.1}: The idea of using diffusion models for segmentation is not quite new. Overall speaking, this paper just use the diffusion model to replace the previous segmentation backbones, which seems not that novel.}


% \noindent\textbf{{R3.2}: After I check the related work, I found [36] already studied diffusion + contrastive learning. Thus the contrastive part (Positive/Negative Pair Construction) could not be viewed as brand new. Besides, the performance improvement brought by the contrastive learning is very marginal.}



% \noindent\textbf{{R3.3}: The experiments are not sufficient to evaluate the proposed method. The authors only tested it on a small dataset (about 5,000 short videos). And the authors only compared to simple baselines (AVSBench proposed by the dataset authors).}

% \noindent\textbf{{R3.4}:  What is the GFLOPS of the proposed method? I believe this is not a fair comparison since the proposed model requires much more computation cost than the compared simple baselines.}

\section{Introduction}

Argumentation is the field of elaboration and presentation of arguments to debate, persuade, and agree, where an argument is made of a conclusion (i.e., a claim) supported by reasons (i.e., premises)~\cite{walton2008argumentation}. By analogy with computational linguistics, \textit{computational argumentation} refers to the use of computer-based methods to analyze and create arguments and debates~\cite{gurevych-etal-2016}. It is a subfield of artificial intelligence that deals with the automated representation, evaluation, and generation of arguments. This field includes important applications such as mining arguments~\cite{al-khatib-etal-2016-cross}, assessing an argument's quality~\cite{el-baff-etal-2018-challenge}, reconstructing implicit assumptions in arguments~\cite{habernal-etal-2018-argument} or even providing constructive feedback for improving arguments~\cite{naito-etal-2022-typic}, to name a few.

In the context of education, learning argumentation (e.g., writing argumentative essays, debates, etc.) has been shown to improve students' critical thinking skills~\cite{pitchers-sodden-2000, behar-horenstein-etal-2011-teaching}. To further improve critical thinking skills, several researchers have been working on computational argumentation to support and provide tools to assist learners in improving the quality of their arguments.

Although computational models for argumentation are proven to assist students' learning and reduce teachers' workload~\cite{twardy-2004, wambsganss-etal-2021-arguetutor}, such models still lack the ability to \emph{explain} how an argument can be improved efficiently; e.g., why a particular argument was labeled bad or given a low score by their automatic evaluation rubrics. In other words, the model should be not only able to provide its results but also be able to \textit{explain and visualize the results in a comprehensive way} for the users so that users can understand, and ultimately improve their argumentation skills.

% Figure environment removed

We argue that the output for current computational models for argumentation act as a type of explanation and must be the focus of future work. For our survey, we categorize works into four different dimensions (cf., Figure~\ref{fig:overview}):
\begin{itemize}
    \item \textit{Richness}: Level of feedback details given by a model, i.e., \textit{what} is the error identified by the model and \textit{why} it is an error;
    \item \textit{Visualization}: Way of presenting feedback, i.e., \textit{how} the explanation is shown;
    \item \textit{Interactivity}: Ability to communicate with the model, other users, or a third-person, i.e., with \textit{whom} the user is talking;
    \item \textit{Personalization}: Ability to adapt the feedback to the users' background, i.e., \textit{to whom} the feedback is given.
\end{itemize}

% Figure environment removed

In Figure~\ref{fig:ex-types}, for a given argument consisting of two claims and one premise, four different feedback are shown, each highlighting a different dimension of feedback (\textit{Richness}, \textit{Visualization}, \textit{Interactivity}, and \textit{Personalization}).

Towards explainable computational argumentation, this survey aims to give an overview of computational argumentation on automated quality assessment.
We explore work providing explanations answering the following: \textit{What} (\S\ref{sec:richness-what}), \textit{Why} (\S\ref{sec:richness-why}), \textit{How} (\S\ref{sec:visualization}), \textit{Who} (\S\ref{sec:interactivity}), and \textit{To Whom} (\S\ref{sec:personalization}).
Finally, we discuss remaining challenges and potential ways to overcome them (\S\ref{sec:open_issues}) in order to develop systems that provide explanations in a way in which learners can improve their critical thinking skills.
We believe this survey can aid researchers in understanding current explanations in argumentation and broaden their horizon on argumentative feedback.\footnote{\label{foot:website}For more details, papers mentioned in this survey are categorized at \url{https://anonymized}.}

%and focus more on explanation in argumentation and apply it to newer models, thus making the system more explainable.




\section{Related Work}
\noindent\textbf{Audio-Visual Segmentation.}
Audio-visual segmentation (AVS) is a challenging, newly proposed problem that predicts pixel-wise masks for the sound producer(s) in a video sequence given audio information. 
To tackle this issue, Zhou~\etal~\cite{zhou_AVSBench_ECCV_2022} propose an audio-visual
segmentation benchmark and provide pixel-level annotations. The dataset contains five-second videos and audio, and the binary mask is used to indicate the pixels of sounding objects for the corresponding audio.
Subsequently, they present a simple baseline, an encoder-decoder network based on temporal pixel-wise audio-visual interaction.
Building upon this work, CATR~\cite{li_catr_acmmm_2023} introduces a comprehensive approach that incorporates both spatial and temporal dependencies in an audio-visual combination.
CMMS~\cite{liu_avs_acmmm_2023} extends the AVS tasks to the instance level.
Hao~\etal~\cite{hao_aaai_2024_avsbg} present an audio-visual correlation module with a bidirectional generation consistency module to ensure audio-visual signal consistency.
However, this fusion strategy only considers correlations at the feature level and does not capture the intrinsic characteristic of AVS, namely, the guiding role of audio.
Considering the role of audio as guidance for guided
% uniqueness of this task is that audio serves as guidance, leading to guided 
multimodal binary segmentation, Mao~\etal~\cite{mao_iccv_2023_ecmvae} employ a multimodal VAE with latent space factorization to model the distribution of audio and visual, aiming to maximize the contribution of audio for AVS.



% Due to its binary segmentation nature, models for salient object detection (SOD) and video foreground segmentation (VOS)~\cite{mahadevan_3DC_VOS_2020,duke_sstvos_cvpr_2021,zhang_ebm_sod_nips_2021,mao_transformerSOD_2021} (segmenting the foreground attracts human attention) are usually treated as baselines. However, the uniqueness of this task is that audio serves as guidance, leading to guided multimodal binary segmentation.


\noindent\textbf{Diffusion Models for Segmentation.}
Diffusion model~\cite{diffusion_model_raw,ho_ddpm_NIPS_2020, denoising_diffuion,mao2024tavgbench,li2024tri} is the most popular image-generation approach aiming to learn data distribution through the iterative forward noise-adding process and the reverse denoising process. In recent days, researchers have found that it is also an effective representative learning method to capture essential features or structures~\cite{diffusion_representation_learning,Preechakul_2022_CVPR,traub2022representation,kingma2021on,label_efficient_diffusion,zhu_CDCD_ICLR_2023}. For image segmentation,
% many works have studied how diffusion model extract strong features and help improve the performance. In  
\cite{baranchuk_label_efficient_DDPMSeg_2021}
% it first 
demonstrates that the feature representation learned by a pre-trained diffusion model can significantly benefit zero-shot image segmentation. Pix2Seq-D~\cite{chen2022generalist} extend the bit-diffusion~\cite{chen2022analog} for panoptic segmentation.
\cite{asiedu_DecoderPretrain_arxiv_2022} propose a decoder pre-training strategy to pre-train the decoder of the diffusion UNet for image segmentation.
\cite{lai2023ddps} use the diffusion model for the mask prior modeling.
\cite{segdiff_image_segmentation_with_diffusion, Rahman_2023_CVPR} utilize diffusion probabilistic model for medical image segmentation. 
Most of the mentioned works only study unimodal image segmentation. However, our work investigates representative features across multiple modalities and semantic connections between them.

% where we explore the diffusion model for a unique multimodal task, where one modality serves as guidance to achieve guided segmentation.

% The essential idea of diffusion model~\cite{diffusion_model_raw,ho_ddpm_NIPS_2020} is to systematically and slowly destroy the structure in a data distribution through an iterative forward diffusion process. The reverse diffusion process then restores structures in data.
% Denoising Diffusion Probabilistic Model (DDPM)~\cite{denoising_diffuion} extends diffusion models to generate high quality samples. Especially, DDPM establishes a connection between diffusion models and denoising score matching, leading to a simplified, weighted variational bound objective for diffusion models.
% % The original goal of diffusion models is to generate high quality images, making super-resolution as a straightforward application~\cite{saharia_SRDiffusion_PAMI_2022}. 
% % Later, diffusion models have been explored for segmentation tasks~\cite{baranchuk_label_efficient_DDPMSeg_2021,wu2023promptunet}.
% Given its feature encoding nature, diffusion models have also been used for representation learning~\cite{diffusion_representation_learning,Preechakul_2022_CVPR,traub2022representation,kingma2021on,label_efficient_diffusion,zhu_CDCD_ICLR_2023}, classification and regression~\cite{han2022card}.
% For multimodal generation~\cite{weinbach_M_VADER_arxiv_2022,gopalakrishnan2022image}, each unimodal is treated equally, which is different from our setting that one modality serves as guidance for the conditional generation process. 

% Diffusion models have been studied in the image segmentation field.
% \cite{baranchuk_label_efficient_DDPMSeg_2021} use a well-trained diffusion model as a feature extractor, to achieve zero-shot image segmentation.
% Pix2Seq-D~\cite{chen2022generalist} extend the bit-diffusion~\cite{chen2022analog} for panoptic segmentation.
% \cite{asiedu_DecoderPretrain_arxiv_2022} propose a decoder pre-training strategy to per-train the decoder of the diffusion UNet for image segmentation.
% \cite{lai2023ddps} use the diffusion model for the mask prior modeling.
% \cite{segdiff_image_segmentation_with_diffusion, Rahman_2023_CVPR} utilize diffusion probabilistic model for medical image segmentation.
% Most of these works study unimodal image segmentation, where we explore the diffusion model for a unique multimodal task, where one modality serves as guidance to achieve guided segmentation.


% The essential idea of diffusion model~\cite{diffusion_model_raw,ho_ddpm_NIPS_2020,song2021scorebased} is to systematically and slowly destroy the structure in a data distribution through an iterative forward diffusion process. The reverse diffusion process then restores structures in data. 
% \cite{diffusion_model_raw} claims that generative models suffer from a trade-off between tractability and flexibility. A tractable model can be easily analyzed and evaluated to fit the data. A flexible model can fit structure in arbitrary data. The diffusion model allows 1) extreme flexibility in the model structure; 2) exact sampling; 3) easy multiplication with other distributions; 4) model log-likelihood and individual states can be cheaply evaluated, achieving both tractable and flexible models.
% DDPM~\cite{denoising_diffuion} extends diffusion models~\cite{diffusion_model_raw} to generate high quality samples. Especially, DDPM~\cite{denoising_diffuion} establishes a connection between diffusion models and denoising score matching, leading to a simplified, weighted variational bound objective for diffusion models.
% The original goal of diffusion models is to generate high quality images, making super-resolution as a straightforward application~\cite{saharia_SRDiffusion_PAMI_2022}. Later, diffusion models have been explored for segmentation tasks~\cite{asiedu_DecoderPretrain_arxiv_2022,chen2022generalist,conditional_diffusion_iterative_seg,segdiff_image_segmentation_with_diffusion,baranchuk_label_efficient_DDPMSeg_2021,diffusion_implicit_image_segmentation_ensemble,Xu_2023_CVPR,Rahman_2023_CVPR,kim2022diffusion,ji2023ddp,zbinden2023stochastic,bogensperger2023score,chen2023berdiff,wu2023promptunet,brempongdecoder}.
% Given its feature encoding nature, diffusion models have also been used for representation learning~\cite{diffusion_representation_learning,Preechakul_2022_CVPR,traub2022representation,kingma2021on,label_efficient_diffusion,zhu_CDCD_ICLR_2023}, classification and regression~\cite{han2022card}.
% Diffusion models have also been extended in 3D vision~\cite{cheng_3DDiffusion_arxiv_2022}. For multimodal generation~\cite{weinbach_M_VADER_arxiv_2022,gopalakrishnan2022image}, each unimodal is treated equally, which is different from our setting that one modality serves as guidance for the conditional generation process. A comprehensive survey on diffusion models can be found at~\cite{yang2022diffusion}.



% \cite{song2021scorebased} present a stochastic differential equation (SDE) that smoothly transforms a data distribution to a known prior distribution by slowly injecting noise, and a corresponding reverse-time SDE that restores the data distribution from the prior distribution by slowly removing the noise. The reverse-time SDE depends only on the time-dependent gradient (score) of the noisy data, where the score can be accurately estimated, with which samples can be generated by using numerical SDE.
% M-VADER: A Model for Diffusion with Multimodal Context~\cite{weinbach_M_VADER_arxiv_2022}. \\
% % Image super-resolution via iterative refinement~\cite{saharia_SRDiffusion_PAMI_2022}. \\
% SDFusion: Multimodal 3D Shape Completion, Reconstruction, and Generation~\cite{cheng_3DDiffusion_arxiv_2022}. \\
% \cite{zhu_CDCD_ICLR_2023}
% \noindent{Latent Diffusion Models:}

\noindent\textbf{Contrastive Learning for Representation Learning.}
Contrastive loss~\cite{chopra2005learning,dimension_reduction_lecun,chen2020simple} is introduced for distance metric learning to decide whether the pair of samples is similar or dissimilar.
% which takes pair of examples ($\mathbf{x}$ and $\mathbf{x}'$) as input and train the network ($E$) to predict whether they are similar (from the same class: $\mathbf{y}_\mathbf{x}=\mathbf{y}_{\mathbf{x}'}$) or dissimilar (from different classes: $\mathbf{y}_\mathbf{x}\neq \mathbf{y}_{\mathbf{x}'}$). 
% Taking a step further, triplet loss~\cite{Distance_Metric_Learning,large_scale_online_learning,facenet} achieves distance metric learning by using triplets, including a query sample ($\mathbf{x}$), it is a positive sample ($\mathbf{x}^{+}$) and a negative sample ($\mathbf{x}^{-}$). The goal of triplet loss is to push the difference of similarity between positive and negative samples to the query sample to be greater than a predefined margin parameter. 
Taking a step further, triplet loss~\cite{Distance_Metric_Learning,large_scale_online_learning,facenet} uses triplets to push the difference of similarity between positive ($\mathbf{x}^{+}$) and negative samples ($\mathbf{x}^{-}$) to the query sample ($\mathbf{x}$) to be greater than a predefined threshold and achieves better feature representation learning.
% However, the unbalanced problem will limit the performance due to the fact that it only learns from one negative samples while ignoring other negative samples.  
% achieves distance metric learning by using triplets, including a query sample ($\mathbf{x}$), it is a positive sample ($\mathbf{x}^{+}$) and a negative sample ($\mathbf{x}^{-}$). The goal of triplet loss is to push the difference of similarity between positive and negative samples to the query sample to be greater than a predefined margin parameter.
% By pulling similar concepts to be closer in the embedding space and pushing the dissimilar ones to be far apart, triplet loss achieves better feature representation learning.
% However, one of the main issues is that it only learns from one negative sample, ignoring the dissimilarity with all the other candidate negative samples, leading to unbalanced metric learning. 
% To solve this problem, 
Later,~\cite{npair_loss} introduces N-pair loss to learn from multiple negative samples.  
% for balanced metric learning.
% Contrastive loss, triplet loss, and N-pair loss produce one positive sample for the query sample $\mathbf{x}$, 
% SimCLR~\cite{chen2020simple} produces two noise versions of $\mathbf{x}$ via different data augmentation strategies, and then maximizes agreement between differently augmented views of the same sample via a contrastive loss in latent space. 
% Consider a sample $\mathbf{x}$ and its two correlated views $\tilde{\mathbf{x}}_i$ and $\tilde{\mathbf{x}}_j$, 
% % where $\tilde{\mathbf{x}}_i$ and $\tilde{\mathbf{x}}_j$ is defined 
% termed as the positive pair, \cite{chen2020simple} design a self-supervised feature representation framework with a base encoder $f(\cdot)$ (\eg~ResNet backbone) and a projection head $g(\cdot)$ (\eg~a MLP), where the former is for image backbone feature extraction, and the latter produces latent space representation, where contrastive loss is applied. 
% In this case, the contrastive prediction task aims to identify $\tilde{\mathbf{x}}_j$ from $\{\tilde{\mathbf{x}}_k\}_{k\neq i}$ for a given $\tilde{\mathbf{x}}_i$. 
The main strategy to achieve self-supervised contrastive learning is constructing positive/negative pairs via data augmentation~\cite{xie2021propagate,li2021dense,wang2021dense,van2021unsupervised,o2020unsupervised,chaitanya2020contrastive,xie2021detco}.
% Prototypical contrastive learning (PCL)~\cite{li2020prototypical,li2020prototypical,lin2022prototypical,Prototypical_Graph_Contrastive_Learning,Prototypical_Momentum_Contrastive_Learning,du2022weakly,mo2022siamese,Yue_2021_CVPR,toering2022self} aims to bridge contrastive learning with clustering, where prototypes are introduced as latent variables to find the maximum-likelihood estimation of model parameters. Specifically, prototypical contrastive learning optimize both instance discrimination and semantic structure similarity, where semantic structure of the data is also encoded into the embedding space for more fine-grained representation learning. 
Instead of model instance/image discrimination, dense contrastive learning~\cite{Wang_2022_CVPR_SimSet,wang2021exploring}, widely used in segmentation tasks, aims to explore pixel-level similarity. 
% \cite{wang2021exploring} introduces pixel-wise contrastive learning
% and applies it 
% to semantic segmentation. 
Specifically, 
% the positive/negative pairs, namely 
memory bank~\cite{wang2020xbm,chen2020improved,he2020momentum,chen2020simple,wang2022contrastive} stores historical samples from the same image or different images to make up the positive/negative pool, thereby improving the discriminative capabilities of the model. In this paper, we explore contrastive learning to extensively maximize the alignment of model output with the audio signal from a density ratio perspective, leading to both effective guided segmentation and distinction of our solution with existing techniques~\cite{sun2023learning}.

\noindent\textbf{Uniqueness of Our Solutions.}
Although diffusion models have been explored in segmentation tasks, our method aims to use a conditional latent diffusion model to learn an effective multimodal latent space.
Within the representation learning method, we learn an effective representation of the ground-truth segmentation maps, which is subsequently used to support in the segmentation results.
Instead of employing the diffusion model directly in a separate pipeline, we propose a strategy that utilizes contrastive learning to minimize the audio density ratio. This strategy imposes an explicit constraint on the latent space of the diffusion model and allows us to maximize the contribution of audio for localizing the sound source to achieve high quality segmentation.
% extract the audio information's contribution.

















% Contrastive loss~\cite{chopra2005learning,dimension_reduction_lecun} was introduced for distance metric learning to decide whether the pair of data is similar or dissimilar.
% Taking a step further, triplet loss~\cite{Distance_Metric_Learning,large_scale_online_learning,facenet} achieves distance metric learning by using triplets, including a query sample ($\mathbf{x}$), it is a positive sample ($\mathbf{x}^{+}$) and a negative sample ($\mathbf{x}^{-}$). The goal of triplet loss is to push the difference of similarity between positive and negative samples to the query sample to be greater than a predefined margin parameter. By pulling similar concepts to be closer in the embedding space and pushing the dissimilar ones to be far apart, triplet loss achieves better feature representation learning. However, one of the main issues is that it only learns from one negative sample, ignoring the dissimilarity with all the other candidate negative samples, leading to unbalanced metric learning. To solve this problem, \cite{npair_loss} introduces N-pair loss to learn from multiple negative samples for balanced metric learning.
% % Contrastive loss, triplet loss, and N-pair loss produce one positive sample for the query sample $\mathbf{x}$, 
% SimCLR~\cite{chen2020simple} produces two noise versions of $\mathbf{x}$ via different data augmentation strategies, and it then maximizes agreement between differently augmented views of the same sample via a contrastive loss in latent space. Consider a sample $\mathbf{x}$ and its two correlated views $\tilde{\mathbf{x}}_i$ and $\tilde{\mathbf{x}}_j$, where $\tilde{\mathbf{x}}_i$ and $\tilde{\mathbf{x}}_j$ is defined as the positive pair, \cite{chen2020simple} design a self-supervised feature representation framework with a base encoder $f(\cdot)$ (\eg~ResNet backbone) and a projection head $g(\cdot)$ (\eg~a MLP), where the former is for image backbone feature extraction, and the latter produces latent space representation, where contrastive loss is applied. In this case, the contrastive prediction task aims to identify $\tilde{\mathbf{x}}_j$ in $\{\tilde{\mathbf{x}}_k\}_{k\neq i}$ for a given $\tilde{\mathbf{x}}_i$. 
% The main strategy to achieve self-supervised contrastive learning is constructing positive/negative pairs via data augmentation techniques~\cite{xie2021propagate,li2021dense,wang2021dense,van2021unsupervised,o2020unsupervised,chaitanya2020contrastive,xie2021detco}.
% Instead of model instance/image discrimination, dense contrastive learning~\cite{Wang_2022_CVPR_SimSet,wang2021exploring} aims to explore pixel-level similarity.
% \cite{wang2021exploring} introduces pixel-wise contrastive learning, and applies it to semantic segmentation. Specifically, the positive/negative pairs, namely memory bank~\cite{wang2020xbm,chen2020improved,he2020momentum,chen2020simple,wang2022contrastive}, can be from the same image or different images, leading to intra/inter-image level pixel-wise contrastive learning.



\section{Our NEON System}\label{sec::method}

To address the challenges mentioned in the introduction, we develop the NEON system made up of three phases: feature mining, feature fusion layer, and multitask prediction. 
First, in the feature mining phase, we address the first challenge by carefully designing spatiotemporal features for individual-level users and address the second challenge by extracting behavioral-pattern features for group-level users.
The feature fusion phase then employs a feature-fusion neural network to seamlessly integrate internal preferences, spatiotemporal context impact, and group behavior patterns to generate complete user representations, overcoming both challenges.
Last, in the multitask prediction stage, to enhance the model's understanding of spatiotemporal context, we introduce an auxiliary task of needs-meeting way prediction to the main goal of predicting living needs, providing additional support in addressing the first challenge. The deep feature fusion layer and the multi-task prediction parts of our system are illustrated in Figure~\ref{fig::model}.

% Figure environment removed

\subsection{Feature Mining}\label{sec::FeatureMining}
First of all, we use features that directly reflect user traits, such as users' profile, their recent behavior sequence, and their historical behaviors, as inputs for the model.

 As mentioned above, for a specific user, his living needs are greatly affected by the spatial and temporal scenarios in which he is located. For example, \textit{on a rainy midday, a person at work probably has the need to order food delivery; but on a sunny noon, he/she may have another need of going out to eat in the restaurant}. This complexity and variability of human living needs, driven by the flux of time and space, pose a considerable challenge in accurately modeling the impact of the spatiotemporal context. In order to tackle this challenge, we incorporate spatiotemporal context features as an integral part of our system's input.

What's more, on the platform, users may have potential needs with sparse or even non-existent history records. For example, \textit{a person who never buys medicine on the platform may have a cold and need to buy medicine online one day.} Such potential needs are difficult for the model to grasp. To address the challenge of modeling potential needs, we introduce group behavior pattern features to help the model learn the potential living needs of users. 

Below we give a detailed description of the three categories of features.
\subsubsection{User Features}
This group of features includes user profiles and user history behavior sequences.
\begin{itemize}[leftmargin=*]
\item\textbf{User profiles $\bm{f}^U_p$.} The user's profile, including their age, gender, etc.
\item\textbf{User recent online behavior sequence $\bm{f}^U_{rb}$.} The sequence of items recently clicked by the user in the platform; the sequence of items recently ordered by the user in the platform.
\item\textbf{User aggregated historical online behavior $\bm{f}^U_{hb}$.} The percentage of times users buy each type of life service.
\item\textbf{User offline visitation record $\bm{f}^U_{ov}$.} The 50 most visited POIs (point of interest) by the user in the last six months; the 50 most visited AOIs (area of interest) by the user in the last six months. 
\end{itemize}
We concatenate all the mentioned features above to get a sparse user feature vector $f^U$, formulated as follows:
\begin{equation}
f^U=\left[f^U_p, f^U_{rb}, f^U_{hb}, f^U_{ov}\right].
\end{equation}
\subsubsection{Spatiotemporal Context Features}
Users' living needs are greatly affected by time, location, and other environmental factors. Thus, we introduce spatiotemporal context features as part of the input of our system to help our system model the complex impact of spatiotemporal context, which can be listed as follows.
\begin{itemize}[leftmargin=*]
\item\textbf{Time $\bm{f}^{ST}_t$.} Current time period. More than one time period feature of different granularity is applied, including hour, day, whether it is a holiday, etc.
\item\textbf{Location $\bm{f}^{ST}_l$.} The POI (point of interest) embedding of the user's real-time location; the AOI (area of interest) embedding of the user's real-time location; the city embedding of the user's real-time location. The location features are hourly real-time features.
\item\textbf{Weather $\bm{f}^{ST}_w$.} Weather information for the user's city or region, including wind, humidity, temperature, and weather type (sunny, rainy, snowy, etc.). Weather features are refined to hourly granularity.
\item\textbf{Travel state $\bm{f}^{ST}_{ts}$.} Information about whether the user is located in his/her resident city. Possible states include \textit{based in resident city}, \textit{about to travel}, and \textit{on travel}.
\end{itemize}
The dense spatiotemporal context feature vector $f^{ST}$ is created by concatenating all previously mentioned context features, formulated as follows:
\begin{equation}
f^{ST}=\left[f^{ST}_t, f^{ST}_l, f^{ST}_w, f^{ST}_{ts}\right].
\end{equation}
\subsubsection{Group Behavior Pattern Features}
We introduce group behavior pattern features to supplement the sparse individual behavior of users, in order to assist in identifying the potential living needs of individual users. 
\begin{itemize}[leftmargin=*]
\item\textbf{Group aggregated behavior $\bm{f}^G_{a}$.} We first segment users into groups based on their profiles. In each group, we get the group aggregated behavior by calculating the percentage of views, clicks, and purchases of each type of life service among all views, clicks, and purchases initiated by the group. For each user, the group aggregated behaviors of the groups the user belongs to are used as features. For example, a middle-aged person has group aggregated behaviors feature of middle-aged users and other groups he/she is in. 
\item\textbf{Popularity in the current time period $\bm{f}^G_{ct}$.} We cut all time into time periods according to different criteria, such as whether it is 
a holiday, if it is morning, noon, or night, etc. Then in each time period, we calculate the popularity of each type of life service by calculating the percentage of times the life service is viewed/clicked/purchased among all views/clicks/purchases happening in this time period. We determine the time periods in which the current time is located, and use the popularity of each type of life service in these time periods as a feature. For example, if the user opens the app on Christmas night, popularity on holiday and popularity at night of each kind of life service are set as features.
\item\textbf{Group behavior pattern in spatiotemporal context $\bm{f}^G_{st}$.} By discovering group preferences in different spatiotemporal contexts, we further capture more fine-grained group behavior patterns. We calculate the percentage of views/clicks/purchases of each type of life service initiated by each group in each kind of spatiotemporal context. These fine-grained patterns are used as features of the model. For example, the group preference of middle-aged people at work at noon on working days are used to enrich the representations of every individual within this demographic in such spatiotemporal scenario.
\item\textbf{User behaviors augmented by inter-need correlation $\bm{f}^G_{ic}$.} There is an inherent association across different types of users' living needs. This association can be leveraged to improve prediction performance. For example, \textit{a user who frequently purchases hairdressing services may also be inclined to purchase beauty services.} We use the association rule mining algorithm to analyze the co-occurrence of different life service categories, filter out high-correlation relationships, and employ them to augment user behavior as input features.
\end{itemize}
We combine all previously mentioned group behavior pattern features to generate the dense group behavior pattern feature vector $f^{G}$, which is formulated as follows:
\begin{equation}
f^{G}=\left[f^{G}_a, f^{G}_{ct}, f^{G}_{st}, f^{G}_{ic}\right].
\end{equation}

\subsection{Feature Fusion Layer}\label{sec::FeatureIntegration}
As mentioned in Section~\ref{sec::profdef}, we refer to a user in a specific spatiotemporal context as a user scene. For a user scene $i$, after feature mining, we have dense spatiotemporal features $f^{ST}_i$, dense group pattern features $f^G_i$, and sparse user features $f^U_i$. For brevity of presentation, we omit the subscript $i$ in some of the expressions below. We designed a feature fusion layer to integrate these features into the input of the subsequent prediction module.

We first set up an embedding layer, which processes the high-dimensional sparse user feature vector $f^U$ into a low-dimensional dense vector $v^U$.  To address the challenges of complex impact of spatiotemporal context and users' potential needs, we mine spatiotemporal features and group behavior pattern features in the feature mining phase, respectively. With these features as input, we use a feature merging network to model the interaction between spatiotemporal contexts, group behavior patterns, and  users as follows,
\begin{equation}
x^M=h^M\left(\left[f^{ST}, f^G, v^U\right]\right),
\end{equation}
where $\left[ \cdot \right]$ denotes concatenation operation. Here $h^M$ is the feature merging network, which merges three information sources of spatiotemporal contexts, group behavior patterns, and user preference into a fusion representation $x^M$.

Moreover, users have their own internal characteristics that are independent of the spatiotemporal scene they are in and the group they belong to. To model the internal characteristics of users, we generate a representation as follows,
\begin{equation}
x^U=h^U\left(f^U\right),
\end{equation}
where $h^U$ denotes the user preference network that turns raw user features into dense user preference representation. We then concatenate the two parts of representations into the full representation of the user scene:
\begin{equation}
x=\left[x^M, x^U\right].
\end{equation}
In brief, we design a feature fusion layer to tackle both challenges by considering the influence of spatiotemporal context, incorporating group behavior patterns, as well as extracting individual preferences.

\subsection{Multitask Prediction}\label{sec::Model}
We further design a prediction module which takes user scene representation as input to predict users' living needs. The module is tasked with two objectives: \textit{fine-grained need prediction} and \textit{needs-meeting way prediction}. Fine-grained need prediction is to predict the specific living need of the user. Neets-meeting way prediction is to predict the preferred way of the user to meet their needs. 

Specifically, among the ten kinds of needs which we mentioned in the problem formulation, there are two ways to satisfy the needs: in-store and via-delivery. In other words, consumers can choose to satisfy their needs by visiting a physical store or by ordering online and then receiving goods via delivery. Each type of the 10 needs can be classified into one of two categories, in-store needs or via-delivery needs. We show the classification in Table~\ref{tab::nc}. Actually, needs-meeting way prediction is to predict whether the preferred way of the user to meet their needs is in-store or via-delivery.
\begin{table}[]
\vspace{-0.3cm}
\caption{The classification of the 10 living needs that can be satisfied on Meituan}
\label{tab::nc}
\begin{tabular}{|l|l|}
\hline Living needs that& \fontfamily{ppl}\selectfont Specialty shopping online,\\
 can be satisfied&  \fontfamily{ppl}\selectfont Grocecy shopping online, \\
via delivery& \fontfamily{ppl}\selectfont Ordering food delivery, Buying medicine \\
\hline Living needs that& \fontfamily{ppl}\selectfont Eating in a restaurant, Hotel, \\
can be satisfied& \fontfamily{ppl}\selectfont Hair-dressing, Beauty,  \\
in store& \fontfamily{ppl}\selectfont Tourism, Entertainment \\
\hline
\end{tabular}
\vspace{-0.6cm}
\end{table}

Users' preferences for needs-meeting ways are strongly affected by the spatiotemporal context. For example, \textit{a person at work during lunchtime on a weekday is more likely to have the need to order food delivery (via delivery), while the same person in a shopping district on a weekend evening is more likely to visit a store for a meal (in-store)}. With this in mind, we include the needs-meeting way prediction task which is jointed trained with the main task of need prediction to enhance the model's ability to learn spatiotemporal context information.
Next, we describe how we get the prediction results of the two tasks. 
We use $y^W$ and $y^N$ to denote the prediction result of needs-meeting way and specific need. $y^k, k \in \{W, N\}$ can be generated as follows,
\begin{equation}
 \begin{aligned}
y^k &= t^k(z^k),\\
\text{where }z^k &= g^k(x)_0E^k(x)+g^k(x)_1E^S(x).
\end{aligned}
\end{equation}
Here $t^k$ is the prediction neural network for task $k$. There are a variety of choices in the specific structure of the neural network. In Section~\ref{sec::implementation}, we will state our specific choice. To avoid verbosity, we will use \textit{network} to replace ~\textit{neural network} in the following text. The output of $t^N$, $y^N$, is the scores of ten types of living needs, and the output of $t^W$, $y^W$, is the scores of in-store and via-delivery needs-meeting ways. We use $s^N_{im}$ to denote the score of need $m$ for user scene $i$, and use $s^W_{in}$ to denote the score of needs-meeting way $n$ for user scene $i$. $E^k$ is the expert network~\cite{ma2018modeling,tang2020progressive} for task $k$. $E^S$ is the shared network between the two tasks. $E^S$ is responsible for generating general representations that are common to both tasks, while $E^k$ is responsible for learning task-specific representations that are more fine-tuned to the specific task $k$. The gating network $g_k$ determines what proportion of information input each task's prediction network receives from the shared network and the expert network. We formulate the gating network as follows,
\begin{equation}
g^k(x)=\operatorname{Softmax}(W_k x),
\end{equation}
where $W_k \in \mathbb{R}^{2 \times d}$ is trainable weights for task $k$. The gating network takes $x$ as input, and outputs the relative importance of the shared and task-specific representations for a given tasking, allowing the model to selectively attend to the most relevant information and improve its performance.
In summary, to address the complexity of spatiotemporal context impact, we introduce an auxiliary task of needs-meeting way prediction which is jointly trained with the main task of fine-grained living needs prediction to enhance our system's learning of spatiotemporal context. The multitask prediction module in our system produces a score for each living need and needs-meeting way.
\subsection{Model Training}\label{sec::train}
In this section we describe how our system is trained. Corresponding to the two tasks, we design two parts of loss. We design need prediction loss taking into account the fact that the frequency of different needs arising in users' lives is different. For example, \textit{a user may need to order food delivery for lunch every workday, but rarely need to buy medicine}. In order to address the class imbalance issue for different living needs, we propose using a multi-class focal loss which can decrease the effect of needs with a high volume of training data on the final prediction loss. The need prediction loss can be formulated as follows,
\begin{align}
\text {Loss}_{\text{need}}&=-\sum_{i\in O}\left(\sum_{n=1}^{10}\left(1-q^N_{i n}\right)^\gamma \chi^N_{i n} \log \left(q^N_{i n}\right)\right),\\
\text{where } q^N_{in}&=\operatorname{Softmax}\left(s^N_{i n}\right)=\frac{e^{s^N_{i n}}}{\sum_n e^{s^N_{i n}}}.
\end{align}
Here $O$ is the training set, $s^N_{in}$ is the score of living need $n$ for user scene $i$, $\gamma$ is the hyperparameter which decides the importance of difficult samples, $\chi^N_{in}$ is 1 if $n$ is the ground truth need for user scene $i$, else it is 0.
For the needs-meeting way prediction task, we use BCE loss as prediction loss. We formulate it as follows,
\begin{align}
\text{Loss}_{\text{way}}&=-\sum_{i\in O}\left(\sum_{m=1}^{2}\chi^W_{im}
\log \left(q^W_{im})\right)\right),\\
\text{where } q^W_{im}&=\operatorname{Softmax}\left(s^W_{i m}\right)=\frac{e^{s^W_{i m}}}{\sum_m e^{s^W_{i m}}}.
\end{align}
Here $O$ is the training set, $s^W_{im}$ is the score of needs-meeting way $m$ (in store or via delivery )for user scene $i$.
$\chi^W_{im}$ is 1 if $m$ is the ground truth needs-meeting way for user scene $i$, else it is 0.
In our system, the feature integration module and multitask prediction module are trained end to end. The entire loss function is:
\begin{equation}
\text{Loss} = \lambda_1 \text{Loss}_{\text{need}}+\lambda_2 \text{Loss}_{\text{way}}.
\end{equation}
$\lambda_1$ and $\lambda_2$ are hyperparameters that control the importance of the two parts of loss.


61\section{Experimental Results}
\subsection{Setup}
\noindent\textbf{Datasets.}
We utilize the AVSBench dataset~\cite{zhou_AVSBench_ECCV_2022}, which consists of 5,356 audio-video pairs with pixel-wise annotations.
% for conducting our experiments. 
Each audio-video pair in the dataset spans 5 seconds, and we trim the video to include five consecutive frames by extracting the video frame at the end of each second. 
The AVSBench dataset is further divided into two subsets: semi-supervised Single Sound Source Segmentation (S4), where only the first frame is labeled, and fully supervised Multiple Sound Source Segmentation (MS3), where all frames are labeled. 
The S4 subset contains 4,922 videos, while the MS3 subset contains 424 videos. 
For training and testing, we follow the conventional splitting from the AVSBench dataset~\cite{zhou_AVSBench_ECCV_2022} and perform training and testing with S4 and MS3, respectively.
% and We train our model using the training set and evaluate its performance on the testing set.



\noindent\textbf{Evaluation Metrics.}
We assess the audio-visual segmentation performance using the same evaluation metrics as AVSBench~\cite{zhou_AVSBench_ECCV_2022}, namely Mean Intersection over Union (mIoU) and F-score. 
The F-score is formulated as follows: $F_{\beta}=\frac{(1+\beta^2 \times \text{precision} \times \text{recall})}{\beta^2\times \text{precision} + \text{recall}}, \beta^2=0.3$. 
Here, both precision and recall are computed based on a binary segmentation map, which is obtained by applying 256 uniformly distributed binarization thresholds in the range $[0, 255]$.



\noindent\textbf{Compared Methods.} We compare our method with published AVS methods, including AVSBench~\cite{zhou_AVSBench_ECCV_2022}, AVS-BiGen~\cite{hao_aaai_2024_avsbg}, ECMVAE~\cite{mao_iccv_2023_ecmvae}, CATR~\cite{li_catr_acmmm_2023}, CMMS~\cite{liu_avs_acmmm_2023} \Fix{and AVSegFormer~\cite{gao2024avsegformer}}.
To strictly keep accordance with the settings in previous work~\cite{zhou_AVSBench_ECCV_2022}, we also compare the performance with related segmentation tasks, such as video foreground segmentation models (VOS)~\cite{mahadevan_3DC_VOS_2020, duke_sstvos_cvpr_2021}, RGB image based salient object detection models~\cite{mao_transformerSOD_2021, zhang_ebm_sod_nips_2021}.
We set up the comparison due to the binary video segmentation nature of AVS. 
Being consistent with AVSBench, we also use two backbones, ResNet50~\cite{he_resnet_cvpr_2016} and PVT~\cite{wang_Pvtv2_CVM_2022} initialized with ImageNet~\cite{deng_imagenet_cvpr_2009} per-trained weights, to demonstrate that our proposed model achieves consistent performance improvement under different backbones.
\Fixtwo{For a fair comparison, we establish consistent experimental protocols across methods. Regarding CATR, their paper presents two experimental settings: (1) a baseline setting that maintains consistency with the training protocol of AVSBench, and (2) an enhanced setting utilizing additional AOT-enhanced annotations~\cite{yang2021associating_aot}. To ensure fair comparison, we specifically reference their results from the baseline setting. Similarly, AVSegFormer presents two training configurations: (1) a standard setting with $224\times 224$ input resolution that aligns with the configuration of AVSBench, and (2) an enhanced setting with $512\times 512$ resolution that achieves better performance through increased input size. To maintain consistent experimental protocols, we specifically compare with their $224\times 224$ configuration results, ensuring architectural comparisons are conducted under equivalent conditions.}

\noindent\textbf{Implementation Details.}
Our proposed method is trained end-to-end using the Adam optimizer~\cite{Kingma_Adam_ICLR_2015} with default hyper-parameters for 15 and 30 epochs on the S4 and MS3 subsets. The learning rate is set to $10^{-4}$ and the batch size is 4. All the video frames are resized to the shape of $224\times 224$. 
For the latent diffusion model, we use the cosine noise schedule and the noise prediction objective in \equref{ddpm_loss} for all experiments. The diffusion steps $K$ is set as 20. To accelerate sampling, we use the DDIM~\cite{song_DDIM_ICLR_2020} with 10 sampling steps.


\begin{table}[t]
    \caption{\textbf{Quantitative results on the AVSBench dataset} in terms of mIOU and F-score under S4 and MS3 settings. 
    We both report the performance with R50 and PVT as a backbone for the results of comparison methods and Ours.
    \Fix{* denotes that the training datasets are supplemented annotation with AOT~\cite{yang2021associating_aot}.
    For AVSegFormer~\cite{gao2024avsegformer}, we only report the performance when trained at the common $224\times 224$ resolution.}}
    \label{tab:main_results_on_avsbench}
    \centering
    \small
    \setlength{\tabcolsep}{1.0mm}
      \renewcommand{\arraystretch}{1.3}
    {
        \begin{threeparttable}
        \begin{tabular}{cccccc}
        \toprule[1.1pt]
        & \multirow{2}{*}{Methods} & \multicolumn{2}{c}{S4}                      & \multicolumn{2}{c}{MS3}                     \\
        \cmidrule(r){3-4}  \cmidrule(r){5-6}
                     &         & mIoU                 & F-score              & mIoU                 & F-score   \\
        \midrule
        \multirow{2}{*}{VOS}& 3DC~\cite{mahadevan_3DC_VOS_2020}    & 57.10   & 0.759   & 36.92   & 0.503    \\
        & SST~\cite{duke_sstvos_cvpr_2021}                        & 66.29   & 0.801   & 42.57   & 0.572    \\
        \midrule
        \multirow{2}{*}{SOD}& iGAN~\cite{mao_transformerSOD_2021}   & 61.59   & 0.778   & 42.89   & 0.544    \\
        & LGVT~\cite{zhang_ebm_sod_nips_2021}                       & 74.94   & 0.873   & 40.71   & 0.593    \\
        \midrule
        & AVSBench (R50)~\cite{zhou_AVSBench_ECCV_2022}   & 72.79   & 0.848   & 47.88   & 0.578    \\
        & AVSBench (PVT)~\cite{zhou_AVSBench_ECCV_2022}   & 78.74   & 0.879   & 54.00   & 0.645    \\
        & AVS-BiGen (R50)~\cite{hao_aaai_2024_avsbg}   & 74.13   & 0.854   & 44.95   & 0.568 \\
        & AVS-BiGen (PVT)~\cite{hao_aaai_2024_avsbg}   & 81.71   & 0.904   & 55.10   & 0.668 \\
        & ECMVAE (R50)~\cite{mao_iccv_2023_ecmvae}     & 76.33   & 0.865   & 48.69   & 0.607  \\
        \multirow{2}{*}{AVS}& ECMVAE (PVT)~\cite{mao_iccv_2023_ecmvae}     & 81.74   & 0.901   & 57.84   & 0.708  \\
        & CATR (R50)~\cite{li_catr_acmmm_2023}         & 74.8    & 0.866   & 52.8    & 0.653           \\
        & CATR (PVT)~\cite{li_catr_acmmm_2023}         & 81.4    & 0.896   & 59.0    & 0.700  \\
        & \Fix{CATR (R50)*~\cite{li_catr_acmmm_2023}}  & \Fix{74.9}    & \Fix{0.871}   & \Fix{53.1}    & \Fix{0.656}           \\
        & \Fix{CATR (PVT)*~\cite{li_catr_acmmm_2023}}  & \Fix{84.4}    & \Fix{0.913}   & \Fix{62.7}    & \Fix{0.745}  \\
        & CMMS~\cite{liu_avs_acmmm_2023}               & 81.29   & 0.886   & 59.5    & 0.657           \\
        & \Fix{AVSegFormer (R50)~\cite{gao2024avsegformer}}  & \Fix{76.45} & \Fix{0.859} & \Fix{49.53} & \Fix{0.628}           \\
        & \Fix{AVSegFormer (PVT)~\cite{gao2024avsegformer}}  & \Fix{\textbf{82.06}} & \Fix{0.899} & \Fix{58.36} & \Fix{0.693}   \\ 
        & Ours (R50)               & 75.80          & 0.869          & 49.77          & 0.621           \\
        & Ours (PVT)               & 81.51 & \textbf{0.903} & \textbf{59.62} & \textbf{0.712}  \\
        \bottomrule[1.1pt]
        \end{tabular}
        \end{threeparttable}
    }
    % \vspace{-5.0mm}
\end{table}

% Figure environment removed




\subsection{Performance Comparison}
% \Jing{I'm here}
\noindent\textbf{Quantitative Comparison.} Generally, we define our task as a multimodal binary segmentation task, where the input includes both visual and audio, and the output is a binary map showing the sound producer(s). We find a related and similar setting is salient object detection, where the output is also a binary map, localizing the foreground object(s) that attract human attention. In this way, to prepare the comparison methods, we also adapt the existing state-of-the-art (SOTA) salient object detection models to our multimodal binary segmentation task and show the performance of those models in \tabref{tab:main_results_on_avsbench}, where \enquote{VOS} contains video salient object detection models, and \enquote{SOD} lists the SOTA salient object detection models.
Based on the quantitative results obtained from \tabref{tab:main_results_on_avsbench}, we observe that direct adaptation of salient object detection models to AVS fails to achieve reasonable performance. The main reason is that although both salient object detection and AVS are categorized as binary segmentation, the former relies mainly on the visual input, while the latter depends greatly on the audio modality to localize the sound producer(s).

In the \enquote{AVS} section of \tabref{tab:main_results_on_avsbench}, we show performance comparison of various methods and ours on the AVSBench dataset under different settings (S4 and MS3).
\Fix{Our method consistently outperforms state-of-the-art AVS methods on both MS3 and S4 subsets, achieving notable improvements in mIoU (59.62) and F-score (0.712).}
There is a consistent performance improvement of our proposed method compared to CATR~\cite{li_catr_acmmm_2023}, regardless of whether \enquote{R50} or \enquote{PVT} is used as the backbone. In particular, 0.11 and 0.62 higher mIOU than CATR is obtained on the two subsets with the \enquote{PVT} backbone.
Moreover, the performance of our method significantly surpasses that of ECMVAE~\cite{mao_iccv_2023_ecmvae}, an AVS method based on generative models (VAE). This comparison highlights that, despite the fact that ECMVAE employs intricate strategies involving complex multimodal latent space factorization and constraints, its capacity to model the latent space falls short in comparison to our approach utilizing a conditional latent diffusion model.
It is worth noting that our \enquote{R50} based model slightly outperforms the LGVT~\cite{zhang_ebm_sod_nips_2021} under the S4 subset, despite LGVT using a swin transformer~\cite{liu_swin_iccv_2021} backbone, while AVSBench (R50) performs worse than LGVT. 
This suggests that exploring matching relationships between visual objects and sounds is more important than using a better visual backbone for AVS tasks.
\Fixtwo{Notably, our method demonstrates superior performance over AVSegFormer~\cite{gao2024avsegformer} in three out of four metrics across both datasets. This performance advantage stems from our latent diffusion architecture and contrastive loss design, which effectively model the correlation between video and audio modalities, leading to better audio-guided segmentation results. Specifically, on the S4 dataset, while achieving higher F-score due to our strength in sounding object localization, we observe slightly lower mIoU performance. This can be attributed to the single-source characteristic of S4 dataset, where mIoU primarily reflects the refinement of segmentation boundaries rather than the accuracy of sounding object localization, which is relatively straightforward in single-source scenarios.} Despite these achievements, our model maintains a lightweight architecture where $E_{\varphi}$, $E_{\varphi}$, $D_{\tau}$, and $\epsilon_\theta$ collectively contribute only 4M parameters, resulting in a total of 94.48M parameters when incorporating the PVT backbone. \Fixtwo{This parameter count is substantially more efficient compared to AVSegFormer's 186.05M parameters and CATR's 118.38M parameters while achieving better performance.}


\Fixtwo{We further compare the performance of our model with AVSSBench~\cite{zhou2024avss}, CATR~\cite{li_catr_acmmm_2023} and AVSegFormer~\cite{gao2024avsegformer}} on the AVSBench-semantic datasets (AVSS)~\cite{zhou2024avss} dataset. 
\Fixtwo{Compared to AVSegFormer, our model demonstrates consistent improvements with absolute margins of 1.4 and 1.3 in mIoU and F-score metrics, respectively. These performance gains are particularly pronounced on complex datasets containing multiple sounding targets and rich semantic information, as shown in~\tabref{tab:avss-comparison}. This superior performance can be attributed to our model's enhanced capability in modeling audio-visual correlations using the proposed diffusion framework, which becomes more evident when handling sophisticated scenarios with diverse audio sources and semantic contexts.}
The consistent performance across multiple datasets (AVSBench-S4, MS3, and now AVSS) provides substantial evidence for the robustness and adaptability of our approach. This additional experiment reinforces our claim that recasting AVS as a conditional generation task with audio guidance offers a generalizable framework for audio-visual segmentation challenges.

\begin{table}[t]
\caption{Quantitative comparisons on AVSBench-semantic datasets (AVSS)~\cite{zhou2024avss} in terms of mIoU and F-score.}
\label{tab:avss-comparison}
\centering
\small
\setlength{\tabcolsep}{1.5mm}
\renewcommand{\arraystretch}{1.3}
\begin{threeparttable}
\begin{tabular}{ccccc}
\toprule[1.1pt]
Task & Method & Backbone & mIoU & F-score \\
\midrule
\multirow{2}{*}{VOS} & 3DC~\cite{mahadevan_3DC_VOS_2020} & R18 & 17.3 & 0.210 \\
                     & AOT~\cite{yang2021associating_aot} & R50 & 25.4 & 0.310 \\
\midrule
\multirow{4}{*}{AVSS} & AVSSBench~\cite{zhou2024avss} & PVT & 29.8 & 0.352 \\
                      & CATR~\cite{li_catr_acmmm_2023}          & PVT & 32.8 & 0.385 \\
                      & AVSegFormer~\cite{gao2024avsegformer}   & PVT & 36.7 & 0.420 \\
                      & \textbf{Ours}                           & \textbf{PVT} & \textbf{38.1} & \textbf{0.430} \\
\bottomrule[1.1pt]
\end{tabular}
\end{threeparttable}
\end{table}


\noindent\textbf{Qualitative Comparison.}
In Fig.~3, we show the qualitative comparison of our method with
\Fixtwo{AVSBench~\cite{zhou_AVSBench_ECCV_2022}, ECMVAE~\cite{mao_iccv_2023_ecmvae} and AVSegFormer~\cite{gao2024avsegformer}.
Among them, AVSBench is the baseline model, ECMVAE is also a generative AVS model similar to ours. Furthermore, AVSegFormer is the most advanced model.}
The visualization samples in Fig.~\ref{fig:main_compare} are selected from the more challenging MS3 subset.
It can be observed that our method tends to output segmentation results with finer details, \ie~an accurate segmentation of the \emph{bow of the violin} and the \emph{piano-key} in the left sample in Fig.~\ref{fig:main_compare}. 
In addition, our method also has the ability to identify the true sound producer, such as the \emph{boy} in the right sample in Fig.~\ref{fig:main_compare}, indicating a better sound localization capability.
\Fixtwo{Compared to AVSegFormer, which adopts a transformer architecture, our model incorporates audio cues explicitly via a conditional latent diffusion process. This enables more accurate localization of sounding objects, especially in complex scenes. 
As a result, AVSegFormer tends to highlight visually salient regions, whereas our model focuses more accurately on sounding objects.}


% In Fig.~\ref{fig:main_compare}, we show the qualitative comparison of our method with \Fixtwo{AVSBench~\cite{zhou_AVSBench_ECCV_2022}, ECMVAE~\cite{mao_iccv_2023_ecmvae} and AVSegFormer~\cite{gao2024avsegformer}.
% Among them, AVSBench is the baseline model, ECMVAE is also a generative AVS model similar to ours. Furthermore, AVSegFormer is the most advanced model.}
% The visualization samples in Fig.~\ref{fig:main_compare} are selected from the more challenging MS3 subset.
% It can be observed that our method tends to output segmentation results with finer details, \ie~an accurate segmentation of the \emph{bow of the violin} and the \emph{piano-key} in the left sample in Fig.~\ref{fig:main_compare}. 
% In addition, our method also has the ability to identify the true sound producer, such as the \emph{boy} in the right sample in Fig.~\ref{fig:main_compare}, indicating a better sound localization capability. 
% While the compared methods segment the two salient foreground objects, ignoring the audio information as guidance.

% % We illustrate the qualitative results under the challenge MS3 subset in Fig.~\ref{fig:main_compare}. 



\subsection{Ablation Studies}
We conduct ablation studies to analyze the effectiveness of our proposed method. All variations of the experiments are trained with the PVT backbone.

\begin{table}[!htp]
    \caption{\textbf{Ablation on the latent diffusion model.} \enquote{E-D} indicates the deterministic \enquote{encoder-decoder} structure. \enquote{CVAE} denotes using CVAE to generate the latent code. \enquote{LDM} is our proposed latent diffusion model}
    % \vspace{-2.0mm}
    \label{tab:ablation_on_ldm}
    \centering
    \small
    \setlength{\tabcolsep}{2.5mm}
    \renewcommand{\arraystretch}{1.3}
    {
        \begin{threeparttable}
        \begin{tabular}{ccccc}
        \toprule[1.1pt]
        \multirow{2}{*}{Methods} & \multicolumn{2}{c}{S4}    & \multicolumn{2}{c}{MS3}         \\
        \cmidrule(r){2-3}  \cmidrule(r){4-5}
                & mIoU    & F-score & mIoU    & F-score        \\
        \midrule
        E-D      & 78.89      & 0.881      & 54.28      & 0.648       \\
        CVAE     & 79.97      & 0.888      & 55.21      & 0.661       \\
        % \midrule
        LDM (Ours)& \textbf{81.02} & \textbf{0.894} & \textbf{57.67} & \textbf{0.698}  \\
        \bottomrule[1.1pt]
        \end{tabular}
        \end{threeparttable}
    }
    % \vspace{-2.0mm}
\end{table}

\noindent\textbf{Ablation on Latent Diffusion Model.}
As discussed in the introduction section (Sec.~\ref{sec:intro}), a likelihood conditional generative model exactly fits our current conditional generation setting, thus a conditional variational auto-encoder~\cite{structure_output,kingma2013auto} can be a straightforward solution. 
To verify the effectiveness of our latent diffusion model, we design two baselines and show the comparison results
% As the critical component of our proposed method, to verify the impact of the latent diffusion model, we provide two baseline models and show the results
in \tabref{tab:ablation_on_ldm}. Firstly, we design
% : \textbf{1}) 
a deterministic model with a simple encoder-decoder structure (\enquote{E-D}), where the input data encoding $\{\mathbf{G}\}_{l=1}^4$ is feed directly to the prediction decoder (see Fig.~\ref{fig:model_overview}). Note that \enquote{E-D} is the same as AVSBench~\cite{zhou_AVSBench_ECCV_2022}, and we retrain it in our framework and get similar performance as the original numbers reported in their paper. 
Secondly, to explain the superiority of the diffusion model compared with other likelihood based generative models, namely conditional variational auto-encoder~\cite{structure_output} in our scenario, we follow~\cite{ucnet_sal,mao_iccv_2023_ecmvae} and design an AVS model based on CVAE (\enquote{CVAE}).
The full pipeline of the \enquote{CVAE} for the audio-visual segmentation task can be shown in Fig.~\ref{fig:model_overview_vae}.
Note that this structure can be regarded as a simplified version of ECMVAE~\cite{mao_iccv_2023_ecmvae}, which removes the complex multimodal factorization and other latent space constraints.
% \textbf{2}) a conditional variational auto-encoder (\enquote{CVAE}) following~\cite{ucnet_sal}. The structure of \enquote{E-D} is the same as AVSBench~\cite{zhou_AVSBench_ECCV_2022}, we retrain it in our framework and get similar performance as the original report in their paper. 
% CVAE~\cite{structure_output} is introduced to RGB-Depth salient object detection in~\cite{ucnet_sal}, where early fusion is used for latent feature encoding. 
We follow a similar pipeline and perform latent feature encoding based on the fused feature $\{\mathbf{G}_l\}_{l=0}^4$ instead of the early fusion feature due to our audio-visual setting, which is different from the visual-visual setting in~\cite{ucnet_sal}.
Specifically, the CVAE~\cite{structure_output} pipeline for our AVS task consists of an inference process and a generative process, where the inference process infers the latent variable $\mathbf{z}$ by $p_\theta(\mathbf{z}|\mathbf{X})$, and the generative process
% builds a latent variable $\mathbf{z}$ by $p_\theta(\mathbf{z}|\mathbf{X})$ and obtains 
produces the output via $p_\theta(\mathbf{y}|\mathbf{X},\mathbf{z})$. 


% Figure environment removed

% \footnote{We will explain the detailed structure of CVAE for the AVS task in the supplementary material.}.
% based generative model. 
Results in~\tabref{tab:ablation_on_ldm} show that generative models can improve the performance of AVS by yielding more meaningful latent space compared with the deterministic models. 
Additionally, the latent diffusion model (LDM) exhibits a more powerful latent space modeling capability than our implemented CVAE counterpart. Note that, as no latent code is involved in \enquote{E-D}, we do not perform contrastive learning. For a fair comparison, the contrastive learning objective $\mathcal{L}_\text{contrastive}$ is not involved in \enquote{CVAE} or \enquote{LDM (Ours)} either.



\begin{table}[t!]
    \caption{\textbf{Ablation on the conditional variable,} where we remove the conditional variable (\enquote{None}), or replace the conditional variable with only audio or visual representation.
    % with  We remove the audio-visual condition, the audio condition, and the visual condition as three comparison variants.
    }
    % \vspace{-2.0mm}
    \label{tab:ablation_on_audio_visual_condition}
    \centering
    \small
    \setlength{\tabcolsep}{2.4mm}
    \renewcommand{\arraystretch}{1.3}
    {
        \begin{threeparttable}
        \begin{tabular}{ccccc}
        \toprule[1.1pt]
        \multirow{2}{*}{Methods} & \multicolumn{2}{c}{S4}    & \multicolumn{2}{c}{MS3}         \\
        \cmidrule(r){2-3}  \cmidrule(r){4-5}
                & mIoU    & F-score & mIoU    & F-score        \\
        \midrule
        None       & 80.04      & 0.889      & 56.12      & 0.671       \\
        Audio      & 80.29      & 0.892      & 56.59      & 0.680       \\
        Visual     & 80.68      & 0.892      & 57.21      & 0.688       \\
        % \midrule
        Audio-Visual (Ours) & \textbf{81.02} & \textbf{0.894} & \textbf{57.67} & \textbf{0.698}  \\
        \bottomrule[1.1pt]
        \end{tabular}
        \end{threeparttable}
    }
    % \vspace{-2.0mm}
\end{table}


\begin{table}[t!]
    \caption{\textbf{Ablation of contrastive learning.} We perform experiments without the $\mathcal{L}_\text{contrastive}$ to show its effectiveness.}
    % \vspace{-2.0mm}
    \label{tab:ablation_on_contrastive_learning}
    \centering
    \small
    \setlength{\tabcolsep}{3.4mm}
    \renewcommand{\arraystretch}{1.3}
    {
        \begin{threeparttable}
        \begin{tabular}{ccccc}
        \toprule[1.1pt]
        \multirow{2}{*}{Methods} & \multicolumn{2}{c}{S4}    & \multicolumn{2}{c}{MS3}         \\
        \cmidrule(r){2-3}  \cmidrule(r){4-5}
                & mIoU    & F-score & mIoU    & F-score        \\
        \midrule
        w/o $\mathcal{L}_\text{contrastive}$    & 81.02 & 0.894 & 57.67 & 0.698  \\
        w   $\mathcal{L}_\text{contrastive}$    & \textbf{81.51} & \textbf{0.903} & \textbf{59.62} & \textbf{0.712} \\
        \bottomrule[1.1pt]
        \end{tabular}
        \end{threeparttable}
    }
    % \vspace{-2.0mm}
\end{table}
\noindent\textbf{Ablation on Audio-Visual Condition.}
To further investigate the effectiveness of the audio-visual conditioning in the training process of the latent diffusion model, we train three models by incorporating different conditional variables $\mathbf{c}$, and present their performance in Table~\ref{tab:ablation_on_audio_visual_condition}. 
Initially, we remove the conditional variable, leading to unconditional generation with $p_\theta(\mathbf{z}_{k-1}|\mathbf{z}_{k})$, which is represented as \enquote{None} in the table. 
Subsequently, we consider unimodal audio or visual as only one conditional variable. 
For this purpose, we simply use the feature of each individual modality before multimodal feature concatenation (refer to $E_\psi$ in Sec.\ref{subsec_conditional_latent_diffusion}), leading to audio/visual as conditional variable based models referred to as \enquote{Audio} and \enquote{Visual} in Table~\ref{tab:ablation_on_audio_visual_condition}.
% by removing the audio condition, the visual condition, and the audio-visual condition in the latent diffusion model. 
% As shown in Table~\ref{tab:ablation_on_audio_visual_condition}, we explore four variants of the different condition types. 
Compared to unconditional generation, conditional generation can provide performance improvements, with the best results achieved when using the audio-visual condition. Furthermore, we can also observe that the performance of using visual data as the conditional variable yields superior performance compared to using audio.
We attribute this observation to two main factors. Firstly, our dataset is small and less diverse, leading to less effective audio information exploration as we pre-trained our model on a large visual image dataset. 
Secondly, the audio encoder is smaller compared with the visual encoder. More investigation will be conducted to address and balance the distribution of data.
In order to ensure a fair comparison, we opted not to perform contrastive learning in the related experiments outlined in Table~\ref{tab:ablation_on_audio_visual_condition}, similar to the ablation on the latent diffusion model.





\noindent\textbf{Ablation on Contrastive Learning.}
We introduce contrastive learning to our framework to learn the discriminative conditional variable $\mathbf{c}$. We then train our model directly without contrastive learning and show its performance as \enquote{w/o $\mathcal{L}_\text{contrastive}$} in Table~\ref{tab:ablation_on_contrastive_learning}, where \enquote{w $\mathcal{L}_\text{contrastive}$} is our final performance in Table~\ref{tab:main_results_on_avsbench}. The improved performance of \enquote{w $\mathcal{L}_\text{contrastive}$} indicates the effectiveness of contrastive learning in our framework.
\begin{table}[t]
    \caption{\textbf{Ablation on the size of the latent space,} where we conduct experiments with different latent sizes.}
    % \vspace{-2.0mm}
    \label{tab:ablation_on_ldm_dimension}
    \centering
    \small
    \setlength{\tabcolsep}{3.8mm}
    \renewcommand{\arraystretch}{1.3}
    {
        \begin{threeparttable}
        \begin{tabular}{ccccc}
        \toprule[1.1pt]
        \multirow{2}{*}{Latent Size} & \multicolumn{2}{c}{S4}    & \multicolumn{2}{c}{MS3}         \\
        \cmidrule(r){2-3}  \cmidrule(r){4-5}
                & mIoU    & F-score & mIoU    & F-score        \\
        \midrule
        $ {D}= 8 $  & 81.04          & 0.892          & 57.28          & 0.689           \\
        $ {D}=16 $  & 81.18          & 0.895          & 57.98          & 0.704           \\
        $ {D}=24 $  & \textbf{81.51} & \textbf{0.903} & \textbf{59.62} & \textbf{0.712}  \\
        $ {D}=32 $  & 80.78          & 0.891          & 57.01          & 0.687           \\
        \bottomrule[1.1pt]
        \end{tabular}
        \end{threeparttable}
    }
    % \vspace{-2.0mm}
\end{table}
Additionally, we observe that contrastive learning performs poorly with the naive encoder-decoder framework, especially with our limited computation configuration, where we cannot construct large enough positive/negative pools. 
However, we find the improvement is insignificant compared to using contrastive learning in other tasks~\cite{han2022expanding}. 
We argue the main reason for this lies in our dataset being less diverse to learn distinctive enough features. 
We will investigate self-supervised learning to further explore the effectiveness of contrastive learning in our framework.
% our model relies on the representation $z$ in this case. Large positive/negative pools can relax the necessity for semantic-correlated representation, as the large sample pools can guarantee the sample-wise distinction.


\begin{table}[t]
\caption{\Fix{\textbf{Ablation on the prediction decoder,} where we conduct experiments under the AVSegFormer architecture.}}
\label{tab:AVSegFormer_diffusion}
\small
\centering
\setlength{\tabcolsep}{1.0mm}
\renewcommand{\arraystretch}{1.3}
\begin{threeparttable}
\begin{tabular}{lcccc}
\toprule[1.1pt]
\multirow{2}{*}{Method} & \multicolumn{2}{c}{S4} & \multicolumn{2}{c}{MS3} \\
\cmidrule(r){2-3}  \cmidrule(r){4-5}
 & mIoU & F-score & mIoU & F-score \\
\midrule
AVSegFormer~\cite{gao2024avsegformer} & 82.06 & 0.899 & 58.36 & 0.693 \\
AVSegFormer~w.~Diffusion (Ours)       & \textbf{82.79} & \textbf{0.910} & \textbf{59.94} & \textbf{0.715} \\
\bottomrule[1.1pt]
\end{tabular}
\end{threeparttable}
\end{table}



\noindent\textbf{Ablation on Size of the Latent Space.} 
We conduct additional ablation experiments to investigate the impact of the latent space size. In the main experiment, we perform parameter tuning and determine that $D = 24$ yields the best results. Here, we proceed to conduct experiments with varied latent sizes and present the performance outcomes in Table~\ref{tab:ablation_on_ldm_dimension}. An obvious observation is that the size of the latent space should not exceed a certain threshold ($D=32$) for the diffusion model, as doing so can lead to significant performance degradation. Conversely, we find that relatively stable predictions are achieved within the latent code dimension range of $D\in [16, 24]$.


\Fix{\noindent\textbf{Ablation on Prediction Decoder.}
We replace the decoder of the model with the transformer decoder in AVSegFormer~\cite{gao2024avsegformer} to demonstrate the applicability of our proposed conditional generation framework under different model frameworks. 
The experimental results are shown in~\tabref{tab:AVSegFormer_diffusion}. 
This demonstrates that our method's contribution extends beyond a specific architecture and represents a general enhancement that can benefit various AVS base models. Note that although alternative decoders such as transformer-based structures (\eg, AVSegFormer) demonstrate strong performance, their higher computational overhead and larger parameter counts motivated us to adopt the more lightweight Panoptic-FPN decoder.}




\subsection{Analysis}
\noindent\textbf{Pre-training Strategy Analysis.} 
As discussed in~\cite{zhou_AVSBench_ECCV_2022}, we also train our model with the full parameters initialized by the weight per-trained on the S4 subset. The performance comparison is shown in \tabref{tab:results_for_pertrain}. 
% The pre-training strategy can facilitate the modeling of audio-visual correspondence by 
It is verified that an effective pre-training strategy is beneficial in all the settings with our proposed method, using \enquote{R50} or \enquote{PVT} as a backbone. We argue the main reason lies in the less diverse and small amount of dataset. In this case, effective transfer learning with suitable model tuning strategies can be a promising research direction to improve the effectiveness of our solution further, \eg~prompt tuning~\cite{lester-etal-2021-power,han2021ptr,li-liang-2021-prefix}.

\begin{table}[t]
    \caption{\textbf{Performance comparison with different initialization strategies} (train from scratch or pre-train on S4) under MS3 setting in terms of mIoU.
    We use the arrows with specific values to indicate the performance gain.
    % the performance mIOU after using the weights pre-trained on the S4 subset.
    }
    % \vspace{-2.0mm}
    \label{tab:results_for_pertrain}
    \centering
    \small
    \setlength{\tabcolsep}{0.8mm}{
        \begin{threeparttable}
        \begin{tabular}{cccc}
        \toprule[1.1pt]
        {Methods} & {From scratch}  &  & {Pre-trained on S4}         \\
        \midrule
        AVSBench (R50)~\cite{zhou_AVSBench_ECCV_2022}   & 47.88 & $\stackrel{+ 6.45}{\longrightarrow}$ & 54.33  \\
        AVSBench (PVT)~\cite{zhou_AVSBench_ECCV_2022}   & 54.00 & $\stackrel{+ 3.34}{\longrightarrow}$ & 57.34  \\
        ECMVAE (R50)~\cite{mao_iccv_2023_ecmvae}   & 48.69 & $\stackrel{+ 8.87}{\longrightarrow}$ & 57.56  \\
        ECMVAE (PVT)~\cite{mao_iccv_2023_ecmvae}   & 57.84 & $\stackrel{+ 2.97}{\longrightarrow}$ & 60.81  \\
        Ours (R50)                             & \textbf{49.77} & $\stackrel{+ 7.82}{\longrightarrow}$ & \textbf{57.59}  \\
        Ours (PVT)                             & \textbf{59.62} & $\stackrel{+ 2.32}{\longrightarrow}$ & \textbf{61.94}  \\
        \bottomrule[1.1pt]
        \end{tabular}
        \end{threeparttable}
    }
    % \vspace{-2.0mm}
\end{table}

% Figure environment removed



\noindent\textbf{Performance with Different Denoising Steps.}
The denoising step in diffusion models is usually pre-defined empirically. We set the denoising step in this paper following the conventional practice.
We thus evaluate the effect of the re-spaced inference denoising steps driven by the DDIM scheduler~\cite{song_DDIM_ICLR_2020}.
The change in testing performance for our model across the MS3 and S4 datasets with varying denoising steps is presented in Fig.~\ref{fig:ddim_step}.
Although the model is trained with 50 DDPM steps, employing 10 steps during inference is sufficient to achieve accurate results.
As expected, increasing the number of denoising steps leads to improved performance.
We observe that the elbow point of marginal returns given more denoising steps depends on the dataset but is always under 10 steps.
Hence, we determine that a denoising step value of 10 strikes an optimal trade-off between sampling efficiency and sample quality.


% It can be seen that improved performance is achieved with increased sampling steps.
% And when denoising step is set to 10, a saturated performance can be obtained. 
% Therefore, for the trade-off of sampling efficiency and sample quality, the value 10 is optimal for denoising step.
% larger
% along with the change in the 
% denoising steps is indeed beneficial for our task. 
% 
% To achieve trade-off between sampling efficiency and sample quality, we set denosing step as 10, and  in this paper.
% , the performance of our model shows a gradual increase.


% Figure environment removed


\noindent\textbf{Failure Case Analysis.}
\Fixtwo{We conduct a failure case analysis on our proposed method, AVSBench~\cite{zhou_AVSBench_ECCV_2022} and AVSegFormer~\cite{gao2024avsegformer}.}
In Fig.~\ref{fig:Failure}, it can be observed that our method, AVSbench, and \Fixtwo{AVSegFormer can not handle the absence of segmented objects resulting from sound interruptions.}
This limitation arises from the fact that neither our method nor AVSBench considered the \enquote{timing discontinuity} of the sound during the modeling process. 
Nevertheless, our proposed method is still able to achieve accurate sound source localization and then deliver high-quality segmentation results.
We believe that modeling from a temporal perspective, \ie~an audio-visual temporal correlation latent space, is one way to think about this problem.




\section{Conclusion}\label{sec:conclusion}
In this work, we design a new type of tuning method, termed regularized mask tuning, that masks the network parameters under a learnable selection. 
Specifically, we first identify a set of parameters that are key to a given downstream task, then attach a binary mask to this parameter, and finally optimize these masks on the downstream data with the parameters frozen.
When updating the mask, we introduce a novel gradient dropout strategy to regularize the parameter selection, to prevent the model from forgetting and overfitting.
% Meanwhile, our method is synergistic with most existing prompt tuning methods and provides the capacity to customize parameter settings based on downstream needs.
Extensive experiments demonstrate that our method consistently outperforms existing methods and is synergistic with them.
Future work will explore applying mask tuning to other visual tasks such as segmentation.
% \todo{why not add the mask in the text encoder?}











%%%%%%%%% REFERENCES
{
\bibliographystyle{ieeetr}
\bibliography{egbib}
}

%%%%%%%%% Biography
%\bf{If you will not include a photo:}\vspace{-33pt}
% \begin{IEEEbiographynophoto}{John Doe}
% Use $\backslash${\tt{begin\{IEEEbiographynophoto\}}} and the author name as the argument followed by the biography text.
% \end{IEEEbiographynophoto}

\end{document}


