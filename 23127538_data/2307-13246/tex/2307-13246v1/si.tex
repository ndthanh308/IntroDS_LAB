\documentclass[aip,jcp,reprint,onecolumn,amsmath,amssymb,floatfix,citeautoscript]{revtex4-2}
%\documentclass{article}
\usepackage{cancel}
\usepackage{amsmath}
\usepackage{mathtools}
\usepackage{physics}
\usepackage{graphicx}
\usepackage{amssymb}
\usepackage{amsthm}
\usepackage{bm}
\usepackage{dcolumn}
\usepackage{braket}
\usepackage{ragged2e}
\usepackage{txfonts}
\usepackage[version=3]{mhchem}
\usepackage[T1]{fontenc}
\usepackage[colorlinks=true,allcolors=blue]{hyperref}
\usepackage[capitalise]{cleveref}

\usepackage{enumitem}
\setlistdepth{5}
\newlist{myEnumerate}{enumerate}{9}
\setlist[myEnumerate,1]{label=\arabic*.}
\setlist[myEnumerate,2]{label=\Roman*.}
\setlist[myEnumerate,3]{label=\Alph*.}
\setlist[myEnumerate,4]{label=\roman*.}
\setlist[myEnumerate,5]{label=\alph*.}
\usepackage{subcaption}

\newcommand*\red[1]{\textcolor{red}{#1}}
\newcommand*\tcb[1]{\textcolor{red}{[TCB: #1]}}
\newcommand*\hkt[1]{\textcolor{blue}{[HKT: #1]}}

\newcommand{\vq}{{\bm{q}}}
\newcommand{\vn}{{\bm{n}}}
\newcommand{\vm}{{\bm{m}}}
\newcommand{\vp}{{\bm{p}}}
\newcommand{\vA}{{\bm{a}}}
\newcommand{\statem}{\ket{\vm}}
\newcommand{\staten}{\ket{\vn}}

\newcommand{\beginsupplement}{%
	\setcounter{table}{0}
	\renewcommand{\thetable}{S\arabic{table}}%
	\setcounter{figure}{0}
	\renewcommand{\thefigure}{S\arabic{figure}}%
	\renewcommand{\thesection}{S\arabic{section}}
	\renewcommand{\theequation}{S\arabic{equation}}
	\renewcommand{\thetable}{S\arabic{table}}
}

\allowdisplaybreaks

\begin{document}

\title{Supplementary Material: Vibrational heat-bath configuration interaction with semistochastic perturbation theory 
	using harmonic oscillator or VSCF modals}

\author{Henry K. Tran and Timothy C. Berkelbach}

\maketitle
\beginsupplement

\section{Potential Energy Surfaces}
The potential energy surfaces used in the main text are presented in the supplementary information. The format is as follows. \\
\indent\texttt{Modes: n} \\
\indent\texttt{0 w\textsubscript{1}} \\
\indent\texttt{1 w\textsubscript{2}} \\
\indent$\vdots$\\
\indent\texttt{n w\textsubscript{n+1}} \\
\indent\texttt{Force\_constants: N}\\
\indent\texttt{p\textsubscript{1} q\textsubscript{1} $\cdots$ q\textsubscript{p\textsubscript{1}} F\textsubscript{1}}\\
\indent$\vdots$\\
\indent\texttt{p\textsubscript{N} q\textsubscript{1} $\cdots$ q\textsubscript{p\textsubscript{N}} F\textsubscript{N}}\\
where \texttt{w\textsubscript{i}} is the normal mode frequency in wavenumbers of the $n$ different modes. \texttt{p\textsubscript{i}} is the power of the $i$\textsuperscript{th} force constant, and every term \texttt{q\textsubscript{j}} after it are the modes for which the derivatives are taken with respect to. The value of the force constant, scaled by the square root of the normal mode frequencies, is \texttt{F\textsubscript{i}}. For example, the line \\
\indent \texttt{3 0 0 1 100} \\
indicates
\begin{equation}
	\frac{1}{\sqrt{\text{\texttt{w\textsubscript{1} w\textsubscript{1} w\textsubscript{2}}}}} \frac{\partial^3 V}{\partial^2\texttt{q\textsubscript{1}} \partial \texttt{q\textsubscript{2}}} = 100
\end{equation}
The PES of systems studied in this paper are given with the filename suffix \texttt{.inp}. 

\section{Individual State Results}
The state energies from each calculation is provided as well with format \texttt{mol\_e1\_e2.csv} where \text{mol} is the molecule, \texttt{e1} is $\epsilon_1$, and \texttt{e2} is $\epsilon_2$.

\section{Results with VSCF-VHCI} \label{sec:results-main}
In this section, VSCF-VHCI results, which were excluded from the main text, are shown.

The first 70 vibrational states of acetonitrile are compared to A-VCI results that were used as benchmarks in the first VHCI paper. The parameters used are $\epsilon_1 = 2.0$, $\epsilon_2 = 0.1$, and $\epsilon_2^d = 1.0$. All calculations involving acetonitrile are displayed in Figure \ref{fig:ac}. The root means squared (RMS) errors are displayed in Figure \ref{fig:ac_error}. The HO-VHCI and VSCF-VHCI calculations use a variational space of 23845 and 18925, respectively. The average number of configurations used for the PT2 corrections of each vibrational state is plotted in Figure \ref{fig:ac_pt2states}. The CPU timing for each part of the calculation is shown in Figure \ref{fig:ac_timing} and the CPU timing for the VHCI calculation in particular is shown in Figure \ref{fig:ac_timing_vhci}. 
% Figure environment removed
% Figure environment removed
% Figure environment removed
% Figure environment removed

The first 200 vibrational states of ethylene oxide are compared to A-VCI results that were used as benchmarks in the first VHCI paper. The parameters used are $\epsilon_1 = 5.0$, $\epsilon_2 = 0.1$, and $\epsilon_2^d=1.0$. All calculations involving ethylene oxide are displayed in Figure \ref{fig:eo}. The root means squared (RMS) errors are displayed in Figure \ref{fig:eo_error}. The HO-VHCI and VSCF-VHCI calculations use a variational space of 113272 and 104873, respectively. The average number of configurations used for the PT2 corrections of each vibrational state is plotted in Figure \ref{fig:eo_pt2states}. The CPU timing for each part of the calculation is shown in Figure \ref{fig:eo_timing} and the CPU timing for the VHCI calculation in particular is shown in Figure \ref{fig:eo_timing_vhci}. 
% Figure environment removed
% Figure environment removed
% Figure environment removed
% Figure environment removed

The first 100 vibrational states of ethylene are compared to VSCF-VHCI+SPT2 results using a tighter set of parameters ($\epsilon_1 = 0.5$ and $\epsilon_ 2 = 0.01$). All calculations involving ethylene are displayed in Figure \ref{fig:c2h4}. The parameters used are $\epsilon_1 = 2.0$, $\epsilon_2 = 0.1$ and $\epsilon_2^d=1.0$. The root means squared (RMS) errors are displayed in Figure \ref{fig:c2h4_error}. The HO-VHCI and VSCF-VHCI calculations use a variational space of 117649 and 113175, respectively. The average number of configurations used for the PT2 corrections of each vibrational state is plotted in Figure \ref{fig:c2h4_pt2states}. The CPU timing for each part of the calculation is shown in Figure \ref{fig:c2h4_timing} and the CPU timing for the VHCI calculation in particular is shown in Figure \ref{fig:c2h4_timing_vhci}. 
% Figure environment removed
% Figure environment removed
% Figure environment removed
% Figure environment removed

While VSCF-VHCI seems to improve over HO-VHCI generally, this is not always the case. It is interesting to note that the difference between the perturbative spaces of HO-VHCI+XPT2 and VSCF-VHCI+XPT2 is not much, and in fact, VSCF usually leads to a larger perturbative space. Despite this, the variational space is smaller for VSCF and the total space is consistently smaller for VSCF. However, the difference is still relatively small. For ethylene, the size of the total space shrinks by only $\sim$3\% between harmonic oscillators and VSCF.

The timing data is seen in Figures \ref{fig:ac_timing}, \ref{fig:eo_timing}, and \ref{fig:c2h4_timing}. First, since a relatively looser set of parameters are used for these comparisons, the immediate timing advantage is not apparent. While the SPT2 and SSPT2 calculations use a smaller perturbative space, the advantage is outweighed through the need to run multiple samples. Figure \ref{fig:tight_timing} shows the timing results using a smaller set of parameters where the advantages are more prominent. In these results, the smaller perturbative space clearly outweighs the need for multiple samples. Note that this is the total CPU time and each sample is embarrassingly parallel.
% Figure environment removed
% Figure environment removed
% Figure environment removed

The benefit of SPT2 is especially seen for VSCF-VHCI+S(S)PT2. Figure \ref{fig:ac_timing} has data for all methods, and VSCF-VHCI+PT2 is by far the most expensive method. It is generally too expensive to be applied to larger systems. The main difficulty is in generating the matrix elements from a VSCF reference. Harmonic oscillator reference matrix elements are much sparser, and thus the matrix is easier to generate. This can be seen in Figures \ref{fig:ac_timing_vhci}, \ref{fig:eo_timing_vhci}, and \ref{fig:c2h4_timing_vhci}. The generation of matrix elements is much more expensive for VSCF references, but the rest of the VHCI calculation is similar across harmonic oscillator and VSCF references. Hence, with SPT2 and SSPT2, the smaller perturbative space is a huge advantage for VSCF-VHCI+SPT2 and VSCF-VHCI+SSPT2 because fewer matrix elements have to be generated. This brings the computational cost of HO-VHCI+SPT2 on par with VSCF-VHCI+SPT2. VSCF-VHCI+SSPT2 tends to be quite expensive because of the deterministic space.

\section{Additional Results}
In this section, we provide calculations not presented in the main text. Most of these are calculations using a tighter set of parameters. Figure \ref{fig:ac2} shows the results for acetonitrile. The biggest difference is that it appears that HO-VHCI+SSPT2 might outperform VSCF-VHCI+SSPT2, but the difference is on the order of 10$^{-3}$ cm$^{-1}$, and it is not clear that the reference A-VCI values are correct to that order either. Figure \ref{fig:c2h4_timing_vhci2} and Figure \ref{fig:eo2} display results for ethylene and ethylene oxide. The results don't change any conclusions in the main text.
% Figure environment removed
% Figure environment removed
% Figure environment removed

% Figure environment removed

% Figure environment removed
% Figure environment removed
% Figure environment removed

\section{SPT2 Statistics}

The main text analyzed the statistics of SSPT2 for ethylene oxide. The same analyses for SPT2 are shown in Figure \ref{fig:spt2_stats}.
% Figure environment removed
% Figure environment removed

\end{document}
