\documentclass[10pt,twoside,english,reqno,a4paper]{amsart}
\usepackage{listings,graphicx,amsmath,varioref,amscd,amssymb,color,bm,stmaryrd,amsthm,amsfonts,graphics,geometry,latexsym,pgf,pst-all}
\setlength{\parindent}{0pt}
\theoremstyle{plain}
\usepackage{esint}
\usepackage{amsthm}



\theoremstyle{plain}
\newtheorem{theorem}{Theorem}[section]
\newtheorem{proposition}[theorem]{Proposition}
\newtheorem{Proposition}[theorem]{Proposition}
\newtheorem{lemma}[theorem]{Lemma}
\newtheorem*{result}{Result}
\newtheorem{corollary}[theorem]{Corollary}
\newtheorem{Corollary}[theorem]{Corollary}

\usepackage{geometry}
\geometry{
	a4paper,
	left=27mm,
	right=27mm,
	top=30mm,
	bottom=30mm,
}


\usepackage[colorlinks=false]{hyperref}

\theoremstyle{definition}
\newtheorem{defin}[theorem]{Definition}
\newtheorem{definitions}[theorem]{Definitions}
%\newtheorem*{proof}{Proof}
\newtheorem*{dimos}{dim}
\newtheorem{remark}[theorem]{Remark}
\newtheorem{example}{Example}
\newtheorem{op}[theorem]{Open Problem}
\theoremstyle{remark}
\newtheorem{conjecture}[theorem]{Conjecture}
\renewcommand{\theequation}{\thesection.\arabic{equation}}
%\@addtoreset{equation}{section}
\renewcommand{\theequation}{\thesection.\arabic{equation}}

%\usepackage[light,math, condensed]{iwona}
%\usepackage[light,math]{iwona}
%\usepackage[T1]{fontenc}


\def\red{\color{red}}
\def\bl{\color{blue}}
\def\bk{\color{black}}

\numberwithin{equation}{section}


\def\into{\int_{\Omega}}
\def\dis{\displaystyle}
\def\supp{\text{\text{supp}}}
\DeclareMathOperator{\diver}{div}
\DeclareMathOperator{\sgn}{sgn}
\DeclareMathOperator{\R}{\mathbb{R}}
\DeclareMathOperator{\N}{\mathbb{N}}
\DeclareMathOperator{\io}{\int_\Omega}
\DeclareMathOperator{\perogni}{\: \forall}
\newcommand{\car}[1]{\raise1pt\hbox{$\chi$}_{#1}}
\DeclareMathOperator*{\esssup}{ess\,sup}
\newcommand{\blu}{\textcolor{blue}}
\newcommand{\DM }{\mathcal{DM}^\infty }
\def\rn{\mathbb{R}^N}
\def\re{\mathbb{R}}

%\usepackage{refcheck}


\newcommand{\res}{\!\!\mathop{\hbox{
			\vrule height 7pt width .5pt depth 0pt
			\vrule height .5pt width 6pt depth 0pt}}
	\nolimits}

\newcommand{\se}[1]{{\color{red!50!black}{#1}}}
\newcommand{\secomm}[1]{{\color{red!50!black}{\bf[#1]}}}

\newcommand{\fo}[1]{{\color{green!50!black}{#1}}}
\newcommand{\focomm}[1]{{\color{green!50!black}{\bf[#1]}}}


\def\fp{\color{teal}}
%\newcommand{\fp}[1]{{\color{teal!50!black}{#1}}}
%\newcommand{\fpcomm}[1]{{\color{teal!50!black}{\bf[#1]}}}

\makeatletter
\@namedef{subjclassname@2020}{%
  \textup{2020} Mathematics Subject Classification}
\makeatother
\begin{document}
\title[Absorption terms in  Dirichlet problems for the prescribed mean curvature equation]{The role of absorption terms in  Dirichlet problems for the prescribed mean curvature equation}


%	

\author[F. Oliva]{Francescantonio Oliva}
\author[F. Petitta]{Francesco Petitta}
\author[S. Segura de Le\'on]{Sergio Segura de Le\'on}

\address{Francescantonio Oliva
	\hfill \break\indent
Dipartimento di Scienze di Base e Applicate per l' Ingegneria, Sapienza Universit\`a di Roma
	\hfill \break\indent
	Via Scarpa 16, 00161 Roma, Italy}
\email{\tt francescantonio.oliva@uniroma1.it}
\address{Francesco Petitta 	\hfill \break\indent Dipartimento di Scienze di Base e Applicate per l' Ingegneria, Sapienza Universit\`a di Roma
		\hfill \break\indent
		Via Scarpa 16, 00161 Roma, Italy}
\email{\tt francesco.petitta@uniroma1.it}
\address{Sergio Segura de Le\'on
	\hfill \break\indent Departament d'An\`alisi Matem\`atica,
	Universitat de Val\`encia, \hfill\break\indent Dr. Moliner 50,
	46100 Burjassot, Val\`encia, Spain.} \email{{\tt
		sergio.segura@uv.es}}


\keywords{prescribed mean curvature, functions of bounded variation, non-parametric minimal surfaces, nonlinear elliptic equations, $L^1$-data} \subjclass[2020]{35J25, 35J60,  35J75, 35J93, 35A01}



\begin{abstract}
In this paper we study existence and uniqueness of solutions to Dirichlet problems as 
$$
	\begin{cases}
		u \dis -\operatorname{div}\left(\frac{D u}{\sqrt{1+|D u|^2}}\right) = f & \text{in}\;\Omega,\\
		u=0 & \text{on}\;\partial\Omega,
	\end{cases}
$$
where $\Omega$  is an open bounded subset of $\R^N$ ($N\ge 2$) with Lipschitz boundary. In particular we explore the regularizing effect given by the absorption term in order to get a unique solutions for data $f$ merely belonging to $L^1(\Omega)$ and with no smallness assumptions. We also prove a sharp boundedness result for solutions for data in $L^{N}(\Omega)$.
\end{abstract}

\maketitle
\tableofcontents



\section{Introduction}



We aim to study existence and uniqueness of solutions to problem
\begin{equation}
	\label{pbintro}
	\begin{cases}
		u \dis -\operatorname{div}\left(\frac{D u}{\sqrt{1+|D u|^2}}\right) = f & \text{in}\;\Omega,\\
		u=0 & \text{on}\;\partial\Omega,
	\end{cases}
\end{equation}
where $\Omega$  is an open bounded subset of $\R^N$ ($N\ge 2$) with Lipschitz boundary and the datum $f$ belongs  to $L^1(\Omega)$.

The main purpose of this paper is to  describe the regularizing effect of zero order absorption terms on the existence of solutions for  boundary value problems as in \eqref{pbintro}. 


Let us recall that equation in \eqref{pbintro}, if $f=0$ and without any absorption,  falls in the well known case of minimal surface equation
$$
\dis \operatorname{div}\left(\frac{D u}{\sqrt{1+|D u|^2}}\right) =0
$$
   the name deriving from the fact that, for a smooth  function $u$, the involved operator calculates the mean curvature of the graph of $u$ at each point $(x,u(x))$; due to this fact such an operator is also called {\it non-parametric mean curvature operator}. 

Several cases  of  (non-parametric) prescribed mean curvature equation of the type
\begin{equation}
\label{pb0}
\begin{cases}
\dis -\operatorname{div}\left(\frac{D u}{\sqrt{1+|D u|^2}}\right) = f & \text{in}\;\Omega,\\
u=0 & \text{on}\;\partial\Omega,
\end{cases}
\end{equation}
have been considered as well  in  literature    starting by \cite{se}, \cite{g76,g78}, \cite{g}, and   \cite{fg84} to present a non-complete list. 

 Another motivation of our work  comes from the fact that the  equation in \eqref{pbintro} corresponds to the
resolvent equation of the following evolution equation

\begin{equation}\label{acm} u_t  =\operatorname{div}\left(\frac{D u}{\sqrt{1+|D u|^2}}\right) \,;\end{equation}
 roughly speaking, proving existence and uniqueness for \eqref{pbintro},  can be considered as a first step in order to apply Crandall-Liggett theory (\cite{cl}) to look for   mild solutions to the corresponding evolution problem. In a more general context these type of arguments  have been successfully applied in order to get existence and uniqueness of Cauchy initial-boundary value problems involving equations as in  \eqref{acm}  in the framework of entropy type solutions and with $L^1$-initial data (see \cite{ACM_MA, ACM_RMI, ACM, gm} for a quite exaustive account on this issue).\bk 
 \medskip 
 
 
Concerning less theoretical issues, problems  as in \eqref{pb0} arise     in the study of  combustible gas dynamics (see \cite{yy} and references therein) as well as in  surfaces capillary problem as pendant liquid drops (\cite{finn74,cf1,cf2, fg84}) and also in design of water-walking devices (\cite{ww}, see also \cite{lc}). 

\medskip

Prescribed mean curvature equations as in \eqref{pb0}  formally represent the Euler-Lagrange equation of the functional 
$$
\mathcal{J}(v)=\int_\Omega \sqrt{1+|\nabla v|^2}  \, dx - \into fv  dx\,, 
$$
involving the area functional. 

As regards the solvability of problems as in \eqref{pb0}, a smallness assumption on the data naturally appears: indeed, if we  formally integrate the equation in \eqref{pb0} in a smooth sub-domain of $A\subset\Omega$, an application of 
 the divergence theorem gives  the following {\it necessary condition}
$$
\left|\int_A f(x) dx\right| =\left| \int_{\partial A} \frac{D u}{\sqrt{1+|D u|^2}}\cdot \nu_A ds \right|< {\rm Per}(A)
$$
where ${\rm Per}(A)$ indicates the perimeter of $A$ and $\nu_A$ is the outer normal unit vector. 
That is,  some sort of  smallness assumption on the datum $f$ is needed in order to get existence in problems as \eqref{pb0}. This is a typical feature of problems arising from functionals with linear growth as, for instance,  the one driven by the  $1$-laplacian  (see for instance \cite{CT,KS, DGOP}). See also Remark \ref{rem:1} below for more details on this structural obstruction. In \cite{g} M. Giaquinta  shows the unique solvability in the space of functions with bounded variation, in a variational sense,  if $f$ is measurable and there exists $\varepsilon_0>0$ such that for every smooth $A\subseteq \Omega$
\begin{equation}\label{giac}
\left|\int_A f(x) \,dx\right|  \leq (1-\varepsilon_0){\rm Per}(A)\,. 
\end{equation}

In \cite{g76} it is shown that 
$$
||f||_{L^N (\Omega)}< N\omega_{N}^{\frac1N} \,,
$$
is a general condition under which \eqref{giac} holds, where $\omega_N$ is the measure of the unit ball of $\rn$ and it is a sharp request in order to get  bounded solutions for problem \eqref{pb0} (see \cite{gop2}). 

\medskip

Less regularity for data $f\in L^q(\Omega)$  below the threshold $q=N$ is known to be mainly forbidden in the classical setting of $BV$-solutions for equations arising from functionals with linear growth and one need a different approach as for instance an Entropy (or Renormalized) one, see for instance \cite{mst2} and \cite{lops}. As far as we know, this is the first paper handling the prescribed mean curvature equation with $L^1$-data. We also point out that  these generalized solutions are,   in general,  bounded only for data $f \in L^{N,\infty} (\Omega)$ with small norm.  

\medskip 

As already mentioned, our  main focus \bk  consists in  analyze the regularizing effect of zero order absorption terms for problems as in \eqref{pbintro}, also in the case of a more general continuous nonlinearity $g(u)$ that mimics $u$ at infinity (namely $g(s) \to \pm \infty$ as $s\to\pm\infty$).  In particular, for problems as   
 \begin{equation}
	\label{pbintrog}
	\begin{cases}
		g(u) \dis -\operatorname{div}\left(\frac{D u}{\sqrt{1+|D u|^2}}\right) = f & \text{in}\;\Omega,\\
		u=0 & \text{on}\;\partial\Omega,
	\end{cases}
\end{equation}
 we show that solutions do  exist for general data $f\in L^1(\Omega) $ no matter of the size of $f$. Moreover,  if $f\in L^N(\Omega)$, then solutions to  \eqref{pbintrog} lie in $L^{\infty}(\Omega)$, again, without any restriction on the norm of $f$. As a remarkable fact this result is sharp at Lorentz scale since, as  we will  show by means of an explicit  counter-example,  unbounded solutions may exist for data in $f\in L^{N,\infty}(\Omega)$. 
 
  Finally, if $g:\re\to\re$ is increasing, then the solution is unique. 

\medskip
In the first part we work by approximation proving existence of a $BV$-solution of problem \eqref{pbintro} when $f\in L^2(\Omega)$; in this case the regular approximation scheme is suitably chosen involving   $p$-Laplacian type operators. In the second part we look for  infinite energy solutions of problem \eqref{pbintro} when $f$ is a merely integrable function. In this case the approximation scheme is given by solutions to problems as \eqref{pbintro} whose existence has been proven in  the first part and we only approximate the datum $f$.

\medskip \medskip 

The plan of the paper is the following: in Section \ref{due} we set the basic machinery on $BV$ spaces (the natural space in which these problems are well settled), and the 
Anzellotti-Chen-Frid type theory of pairings between bounded  vector fields whose divergence lies in some Lebesgue spaces  and gradients of $BV$ functions. Section \ref{secL2} is devoted to present the existence and uniqueness theory of finite energy solutions to problem \eqref{pbintro}  in case of data $f\in L^2(\Omega)$.  
The core of the paper is the content of Section \ref{secL1} in which we prove existence and uniqueness of infinite energy  solutions for \eqref{pbintro} in full generality. In Section \ref{boun} we discuss the boundedness of solutions of problem \eqref{pbintro} in case $f\in L^{N}(\Omega)$,  while if  $f\in L^{N,\infty}(\Omega)$ a smallness assumption is needed as shown by an explicit  example. Finally, in  Section \ref{more}, we address to the  extension of the previous results to the case of a more general nonlinear lower order term as in \eqref{pbintrog}. 

\color{black}

\section{Notation and preparatory tools}  \label{due}



From here on  $\Omega$  will always represent  an open bounded subset of $\R^N$ ($N\ge 2$) with Lipschitz boundary.
We denote by $\mathcal H^{N-1}(E)$ the $(N - 1)$-dimensional Hausdorff measure of a set $E$,  while $|A|$ stands for the $N$-dimensional  Lebesgue measure $\mathcal{L}^N$ of a set $A\subset \rn$. We denote by $\chi_{A}$ the characteristic function of a set $A\subset \rn$. 


By $\mathcal{M}(\Omega)$ we indicate the space of Radon measures with finite total variation over $\Omega$ and we will call mutually singular (or mutually orthogonal) two Radon measures $\mu$ and $\nu$ in $\mathcal{M}(\Omega)$ such that there exists a measurable set $A\subset \Omega$ satisfying 
$$
\mu\res A=\nu\res (\Omega\backslash A)=0. 
$$ 

For a fixed $k>0$, we use the truncation functions $T_{k}:\R\to\R$ and $G_{k}:\R\to\R$ defined, respectively,  by
\begin{align*}
T_k(s):=&\max (-k,\min (s,k))\ \ \text{\rm and} \ \ G_k(s):=s- T_k(s).
\end{align*}


If no otherwise specified, we denote by $C$ several positive constants whose value may change from line to line and, sometimes, on the same line. These values will only depend on the data but they will never depend on the indexes of the sequences we will gradually introduce. Let us explicitly mention that we will not relabel an extracted compact subsequence.


For simplicity's sake, and if there is no ambiguity,  we will often use the following notation:
$$
 \int_\Omega f:=\int_\Omega f(x)\ dx.
$$

Finally, we will denote by ${\rm sgn} (s)$ the multi-valued sign function defined by 
$$
\sgn (s):=\begin{cases}
1 & \text{if }\ s>0\\
[-1,1] & \text{if }\ s=0\\
-1 & \text{if }\ s<0.
\end{cases}
$$



\subsection{$BV$ spaces and the area functional}
We refer to \cite{afp} for a complete account on $BV$-spaces.

Let
$$BV(\Omega):=\{ u\in L^1(\Omega) : Du \in \mathcal{M}(\Omega)^N \}.$$
By $Du \in \mathcal{M}(\Omega)^N$ we mean that  each distributional partial derivative of $u$ is a Radon measure with finite total variation. Then the total variation of $D u$ is given by
$$\displaystyle |D u| = \sup\left\{\int_\Omega u\sum_{i=1}^{N}  \operatorname{\frac{\partial \phi_i}{\partial  x_i}}, \ \phi_i \in C^1_0(\Omega, \mathbb{R}), \ |\phi_i|\le 1, \forall i=1,...,N\right\}.$$

We underline that the $BV(\Omega)$ space endowed with the norm
$$\displaystyle ||u||_{BV(\Omega)}=\int_{\partial\Omega}
|u|\, d\mathcal H^{N-1}+ \int_\Omega|Du|,$$
is a Banach space.


A Radon measure $\mu$ can be uniquely decomposed as
$\mu=\mu^a+\mu^s$
where $\mu^a$ is absolutely continuous with respect to the Lebesgue measure $\mathcal{L}^N$ while $\mu^s$ is concentrated on a set of zero Lebesgue measure, i.e. $\mu^a$ and $\mu^s$ are mutually singular.



\medskip

If $u\in BV(\Omega)$ the measure $\sqrt{1+|Du|^2}$ is defined as
$$\displaystyle  \sqrt{1+ | D u|^2} (E)= \sup\left\{\int_E \phi_{N+1} - \int_E u\sum_{i=1}^{N}  \operatorname{\frac{\partial \phi_i}{\partial x_i}}, \ \phi_i \in C^1_0(\Omega, \mathbb{R}), \ |\phi_i|\le 1, \forall i =1,...,N+1\right\}\,,$$
for any Borel set $E\subseteq \Omega$.
The notation
$$\int_\Omega \sqrt{1+ | D u|^2}$$
stands for  the total variation of the  $\mathbb{R}^{N+1}$-valued measure which formally represents $(\mathcal{L}^N, Du)$. Indeed, if $u$ is smooth,  then
$$|(\mathcal{L}^N, \nabla u)| (\Omega)=\int_\Omega \sqrt{1+ |\nabla u|^2}$$
gives the area of the graph of $u$.
Let also observe that it follows from the decomposition in absolutely continuous and singular part with respect to the Lebesgue measure that one has
\begin{equation*}
	\sqrt{1+ | D u|^2}= \sqrt{1+ | D^a u|^2}\mathcal{L}^N+|D^s u| \,,
\end{equation*}
 where we use the following notations $D^a u:=(D u)^a$ and $D^s u:=(D u)^s$. \bk

In what follows we will use the following semicontinuity classical results; firstly, the  functional $$J_1(v)=\int_\Omega \sqrt{1+ | D v|^2}\varphi + \int_{\partial\Omega} |v|\varphi \, d\mathcal H^{N-1} , \ \ \text{for all} \ 0\le \varphi \in C^1(\overline{\Omega})$$
is lower semicontinuous in $BV(\Omega)$ with respect to the $L^1(\Omega)$ convergence. On the other hand  the functional \begin{equation}\label{j2} J_2(v) = \int_\Omega \sqrt{1-|v|^2}\varphi \ \ \text{for all} \ 0\le \varphi \in C^1(\overline{\Omega})\end{equation} is weakly upper semicontinuous in $L^1(\Omega)$  (see Corollary $3.9$ of \cite{brezis}).


\subsection{The Anzellotti-Chen-Frid theory} Let briefly present the  $L^\infty$-vector fields  theory due to \cite{An} and \cite{CF} in the case   of bounded fields $z$ whose divergence is in $L^q(\Omega)$. 


Let $q\ge 1$ and 
$$X(\Omega)_q:=\{ z\in L^\infty(\Omega)^N : \operatorname{div}z \in L^q(\Omega) \}.$$


In \cite{An}, under suitable compatibility conditions that we shall outline later,  given a function $v\in BV(\Omega)$ and a bounded vector field $z\in X(\Omega)_q$, the following distribution $(z,Dv): C^1_c(\Omega)\to \mathbb{R}$ is defined:
\begin{equation}\label{dist1}
\langle(z,Dv),\varphi\rangle:=-\int_\Omega v\varphi\operatorname{div}z-\int_\Omega
vz\cdot\nabla\varphi,\quad \varphi\in C_c^1(\Omega)\,.
\end{equation}


Let us stress that  \eqref{dist1} is well defined provided one of the following {\it compatibility conditions} hold:  
\begin{equation}\label{cc1} v \in BV(\Omega)\  \text{and}\ \  z\in X(\Omega)_N\,,\end{equation} 

\begin{equation}\label{cc2} v \in BV(\Omega)\cap L^2(\Omega)\  \text{and}\  z\in X(\Omega)_2\,, \end{equation}
or  
\begin{equation}\label{cc3} v \in BV(\Omega)\cap L^\infty(\Omega)\  \text{and}\  z\in X(\Omega)_1\,.\end{equation} 

We point out that an admissible compatibility condition is also  $v \in BV(\Omega)$ and 
$\operatorname{div}z\in L^{N,\infty}(\Omega)$, where $L^{N,\infty}(\Omega)$ is the usual Lorentz space (see \cite{PKF} for an introduction on such function spaces) also known as Marcinkiewicz space of exponent $N$. \bk


\medskip 
Moreover, it holds
\begin{equation*}\label{finitetotal}
|\langle   (z, Dv), \varphi\rangle| \le ||\varphi||_{L^{\infty}(U) } ||z||_{L^\infty(U)^N} \int_{U} |Dv|\,,
\end{equation*}
for all open set $U \subset\subset \Omega$ and for all $\varphi\in C_c^1(U)$, and 
\begin{equation*}\label{finitetotal1}
\left| \int_B (z, Dv) \right|  \le  \int_B \left|(z, Dv)\right| \le  ||z||_{L^\infty(U)^N} \int_{B} |Dv|\,,
\end{equation*}
for all Borel sets $B$ and for all open sets $U$ such that $B\subset U  \subset \subset \Omega$.
Every $z \in X(\Omega)_q$ has a weak trace on $\partial \Omega$ of its normal component which is denoted by
$[z, \nu]$, where $\nu(x)$ is the outward normal unit vector defined for $\mathcal H^{N-1}$-almost every $x\in\partial\Omega$ (see  \cite{An}), such that
\begin{equation*}\label{des1}
||[z,\nu]||_{L^\infty(\partial\Omega)}\le ||z||_{L^\infty(\Omega)^N}\,.
\end{equation*}



The following Green formula holds (see for instance \cite[Proposition 2.5]{dgs}):
\begin{lemma}\label{21}
Let $z \in L^{\infty}(\Omega)^N$ and $v\in BV(\Omega)$, then it holds
	\begin{equation}\label{green}
		\int_{\Omega} v \operatorname{div}z + \int_{\Omega} (z, Dv) = \int_{\partial \Omega} v[z, \nu] \ d\mathcal H^{N-1}\,,
	\end{equation}
	provided one of the compatibility conditions \eqref{cc1}, \eqref{cc2}, \eqref{cc3} is in force. 
\end{lemma}	
Let us recall the following technical result due again to \cite[Theorem 2.4]{An}. \begin{lemma}\label{lemanzas}
Let $z \in L^{\infty}(\Omega)^N$ and $v\in BV(\Omega)$, then it holds
$$(z, D u)^a = z \cdot D^a u.$$
provided one of the compatibility conditions \eqref{cc1}, \eqref{cc2}, \eqref{cc3} is in force. 

\end{lemma}

{\subsection{An algebraic inequality} In what follows we will have to apply an algebraic inequality, which is next set for the sake of completeness.
If $a\ge 0$ and $0\le b\le 1$, then
\begin{equation}\label{ine:1}
  ab\le \sqrt{1+a^2}-\sqrt{1-b^2}.
\end{equation}
To check it, just realize that writing as
\[ab+\sqrt{1-b^2}\le \sqrt{1+a^2}\]
squaring and simplifying, we get \eqref{ine:1} is equivalent to
\[a^2(1-b^2)-2ab\sqrt{1-b^2}+b^2\ge0.\]
This inequality holds since the left-hand side is a square.
As a consequence of \eqref{ine:1} and the Cauchy-Schwarz inequality, we deduce that if $A, B\in\R^N$ with $|B|\le 1$, then
\begin{equation}\label{ine:2}
  A\cdot B\le \sqrt{1+|A|^2}-\sqrt{1-|B|^2}.
\end{equation}
}


\section{$BV$-solutions in presence of $L^2$-data}  \label{secL2}


In this section we deal with the following problem:
\begin{equation}
\label{pb}
\begin{cases}
u\dis -\operatorname{div}\left(\frac{D u}{\sqrt{1+|D u|^2}}\right) = f & \text{in}\;\Omega,\\
u=0 & \text{on}\;\partial\Omega,
\end{cases}
\end{equation}
where $f$ belongs  to $L^2(\Omega)$.

\medskip 
Let us start by  specifying  what we mean by a solution to \eqref{pb}.
\begin{defin}
	\label{weakdef}
	Let $f\in L^2(\Omega)$. A function {$u\in BV(\Omega)\cap L^2(\Omega)$} is a solution to problem \eqref{pb} if there exists $z\in X(\Omega)_2$ with $||z||_{L^\infty(\Omega)^N}\le 1$ such that
	\begin{align}
		&u-\operatorname{div}z = f \ \ \text{in}\ \ \mathcal{D}'(\Omega), \label{def_distrp=1}
		\\
		&(z,Du)=\sqrt{1+|Du|^2} - \sqrt{1-|z|^2} \label{def_zp=1} \ \ \ \ \text{as measures in } \Omega,
		\\
		&u(\sgn{u} + [z,\nu])(x)=0 \label{def_bordop=1}\ \ \ \text{for  $\mathcal{H}^{N-1}$-a.e. } x \in \partial\Omega.
	\end{align}
\end{defin}

\begin{remark}\label{remdef}
	Let underline that \eqref{def_zp=1} aims to give an interpretation to $D u/\sqrt{1+|D u|^2}$. Indeed, if $u$ is smooth and $z=\frac{\nabla u}{\sqrt{1+|\nabla u|^2}}$, then one has
	$$
	(z, \nabla u)= z\cdot \nabla u=\frac{|\nabla u|^2}{\sqrt{1+|\nabla u|^2}} \,,
	$$
	which, after a simple calculation, gives the right-hand of \eqref{def_zp=1}.
	
	It is also worth mentioning that, under the assumptions of Lemma \ref{lemanzas}, \eqref{def_zp=1} turns out to be equivalent to  require that both
	\begin{equation}\label{remdef1}
		z\cdot D^a u = \sqrt{1+ |D^a u|^2} - \sqrt{1-|z|^2}
	\end{equation}
	and
	\begin{equation*} \label{remdef2}
		(z,Du)^s = |D^s u|,
	\end{equation*}
		holds. 
	Let us also stress that, once \eqref{remdef1} is in force, $z$ is uniquely defined by	
	\begin{equation}\label{zesplicito}
		z= \frac{D^a u}{\sqrt{1+|D^a u |^2}}.
	\end{equation}
	 This is a striking  difference with some others  flux-limited diffusion operators as the $1$-laplacian or the transparent media one \cite{ACM, ABCM, GMP}).



	With regard to \eqref{def_bordop=1}, let only say that it is nowadays the classical way the Dirichlet datum is meant for these type of equations as,  the  trace of the solutions needs not to be attained, in general, pointwise.   Roughly speaking, it means that  at any  point of $\partial\Omega$ either $u$ is zero  or the modulus of the weak trace of the normal component of $z$ is highest possible at the boundary. 
	
	We conclude the discussion on Definition \ref{weakdef} by underlining that it is simple to convince that \eqref{def_distrp=1} also holds tested with functions in $BV(\Omega)\cap L^2(\Omega)$. \bk
\end{remark}


Let us state the existence result of this section:
\begin{theorem}\label{teomain}
	Let $f\in L^2(\Omega)$ then there exists a solution to problem \eqref{pb} in the sense of Definition \ref{weakdef}.
\end{theorem}

\begin{remark}\label{rem:1}
	Let us stress again  that, in absence of the absorption zero order term, existence of $BV$-solutions are expected only for small $f$'s belonging to $L^N(\Omega)$.
	To check that a smallness condition is needed in this case, assume that there exists a solution of problem \eqref{pb} without the absorption term and let $z$ be the associated vector field. Then, for every $v\in W_0^{1,1}(\Omega)$, Green's formula implies
	\[\left|\int_\Omega fv\right|=\left|\int_\Omega z\cdot \nabla v\right|\le \int_\Omega |\nabla v|\]
	Thus $f\in W^{-1,\infty}(\Omega)$, the dual space of $W_0^{1,1}(\Omega)$, and $\| f\|_{W^{-1,\infty}(\Omega)}\le 1$.
	Hence, Theorem \ref{teomain} shows that when dealing with the regularizing absorption term one gains that a solution always exists, and it belongs to $BV(\Omega)$, avoiding any small condition on the size of $f$.
\end{remark}
{
 By appealing  to the presence of the regularizing zero order term, we show that the $BV$-solution of \eqref{pb} is unique. 
\begin{theorem}\label{teomainuniqueL2}
	There is at most one solution to problem \eqref{pb} in the sense of Definition \ref{weakdef}.
\end{theorem}
}

\subsection{Existence of finite energy solutions}

In order to prove Theorem \ref{teomain} we consider the following scheme of approximation for $1<p<2$:
\begin{equation}
	\label{pbp}
	\begin{cases}
		u_p\dis \dis -\operatorname{div}\left(\frac{\nabla u_p}{\sqrt{1+|\nabla  u_p|^2}}\right) - (p-1)\operatorname{div}\left(|\nabla u_p|^{p-2}\nabla u_p\right)  = f_p & \text{in}\;\Omega,\\
		u_p=0 & \text{on}\;\partial\Omega,
	\end{cases}
\end{equation}
where $f_p= T_{\frac{1}{p-1}}(f)$. The existence of $u_p \in W^{1,p}_0(\Omega)\cap L^\infty(\Omega)$ such that
\begin{equation}\label{pbpw}
\int_\Omega u_pv - \into \frac{\nabla u_p}{\sqrt{1+|\nabla  u_p|^2}}\cdot \nabla v + (p-1)\into |\nabla u_p|^{p-2}\nabla u_p\cdot \nabla v = \into f_pv\,, \ \ \forall v\in W^{1,p}_0(\Omega),
\end{equation}
follows by  standard monotonicity arguments (\cite{ll}).


We start proving some estimates for $u_p$ which are independent of $p$.

\begin{lemma}\label{lemma_stimeL2}
	Let $f\in L^2(\Omega)$ and let $u_p$ be a solution to \eqref{pbp}. Then $u_p$ is unifomly  bounded in $BV(\Omega)\cap L^2(\Omega)$ (with respect to $p$), and  it also holds
	\begin{equation}\label{stimatermineenergia}
		(p-1)\int_\Omega |\nabla u_p|^p\le C,
	\end{equation}
	for some constant $C$ independent of $p$.
\end{lemma}
\begin{proof}
	It is sufficient to pick $v=u_p$ as a test function in \eqref{pbpw} obtaining
	\begin{equation}\begin{aligned}\label{stimeaprioriN1}
			\int_\Omega u_p^2+ \int_\Omega \frac{|\nabla u_p|^2}{\sqrt{1+|\nabla u_p|^2}} +(p-1)\int_\Omega |\nabla u_p|^p \le  \int_\Omega f_p u_p  \le \frac{1}{2} \int_\Omega u_p^2 + \frac{1}{2}\int_\Omega f^2
	\end{aligned}\end{equation}
after an application of the Young inequality.
Then \eqref{stimeaprioriN1} gives that	
	\begin{equation}\begin{aligned}\label{stimeaprioriN2}
		\frac{1}{2}\int_\Omega u_p^2+ \int_\Omega \frac{|\nabla u_p|^2}{\sqrt{1+|\nabla u_p|^2}} +(p-1)\int_\Omega |\nabla u_p|^p \le  C,
\end{aligned}\end{equation}
where $C>0$ does not depend on $p$.
Now observe that
	\begin{equation}\label{stimeaprioriN3}
		\int_\Omega \frac{|\nabla u_p|^2}{\sqrt{1+|\nabla u_p|^2}}  = \int_{\Omega} {\sqrt{1+|\nabla u_p|^2}}  - \int_{\Omega} \frac{1}{\sqrt{1+|\nabla u_p|^2}}  \ge \int_\Omega |\nabla u_p| - |\Omega|.
	\end{equation}
Therefore it follows from gathering \eqref{stimeaprioriN2} and \eqref{stimeaprioriN3} into \eqref{stimeaprioriN1} that 
\begin{equation}
	\begin{aligned}\label{3.12bis}
		\int_\Omega u_p^2 + \int_\Omega |\nabla u_p| +(p-1)\int_\Omega |\nabla u_p|^p  \le C,
	\end{aligned}
\end{equation}
which concludes the proof.
\end{proof}
{From the previous lemma we immediately deduce the following corollary.
\begin{corollary}\label{cor_u}
		Let $f\in L^2(\Omega)$ and let $u_p$ be a solution to \eqref{pbp}. Then there exists $u\in BV(\Omega)\cap L^2 (\Omega)$ such that, up to subsequences, $u_p$ strongly converges to $u$ in $L^q(\Omega)$ for every $q<2$, weakly in $L^2(\Omega)$, and  $\nabla u_p$ converges to $D u$ weak$^*$ as measures,  as $p\to 1^+$.
\end{corollary}

Hence, from now on, when referring to $u$ we mean the function found in Corollary \ref{cor_u}.
}

\begin{lemma}\label{lemma_esistenzazL2}
		Let $f\in L^2(\Omega)$. Then there exists $z\in X(\Omega)_2$ such that
	\begin{equation}\label{esistenzaz1}
		u-\operatorname{div}z= f  \ \ \ \text{in  } \mathcal{D'}(\Omega),
	\end{equation}
	and
	\begin{equation}\label{esistenzaz2}
		(z,Du)=\sqrt{1+|Du|^2} - \sqrt{1-|z|^2} \ \ \text{as measures in } \Omega,
	\end{equation}
	where $u$ is the function found in Corollary \ref{cor_u}. In particular it holds
	\begin{equation}\label{equint}
		\int_{\{|u|\ge k\}} |u| \le \int_{\{|u| \ge k\}} |f|,
	\end{equation}
	for any $k>0$.
\end{lemma}
\begin{proof}
Let $u_p$ be the solution of \eqref{pbp}. 	Firstly observe that, since ${|\nabla u_p|}{({1+|\nabla  u_p|^2})^{-\frac12}}\le 1$, there exists $z\in L^\infty(\Omega)^N$ such that ${\nabla u_p}{({1+|\nabla  u_p|^2})^{-\frac12}}$ converges to $z$ weak$^*$ in $L^\infty(\Omega)^N$ as $p\to 1^+$. It also follows from the weak lower semicontinuity of the norm that $||z||_{L^\infty(\Omega)^N} \le 1$.
	
	Moreover, the above argument, Lemma \ref{lemma_stimeL2} and Corollary \ref{cor_u}  give that \eqref{esistenzaz1} holds true. Indeed, we only need to show that the third term in \eqref{pbp} goes to zero in the sense of distributions as $p\to 1^+$; to do that, consider $\varphi \in C^1_c(\Omega)$ and observe that from the H\"older inequality and from \eqref{3.12bis}, one has 
	\begin{equation}\label{termineenergiazero}
		\begin{aligned}
			(p-1)\left|\int_\Omega |\nabla u_p|^{p-2}\nabla u_p\cdot\nabla \varphi \right| &\le (p-1) \left(\int_\Omega |\nabla u_p|^p\right)^{\frac{p-1}{p}} \left(\int_\Omega |\nabla \varphi|^p\right)^{\frac{1}{p}}
			\\
			&\le (p-1)^{\frac{1}{p}} ||\nabla \varphi||_{L^\infty(\Omega)^N}|\Omega|^{\frac{1}{p}} \left((p-1)\int_\Omega |\nabla u_p|^p\right)^{\frac{p-1}{p}}
			\\
			&\stackrel{\eqref{stimatermineenergia}}{\le}(p-1)^{\frac{1}{p}} ||\nabla \varphi||_{L^\infty(\Omega)^N}|\Omega|^{\frac{1}{p}} C^{\frac{p-1}{p}},
		\end{aligned}
	\end{equation}
	which gives that
	$$\lim_{p\to 1^+}(p-1)\int_\Omega |\nabla u_p|^{p-2}\nabla u_p\cdot\nabla \varphi =0.$$
	
	This implies  \eqref{esistenzaz1} and, in particular,  that $z\in X(\Omega)_2$.
	
	\medskip
	
	Let us also underline, for later purposes,  that,  since $u$ and $f$ are in $L^2(\Omega)$, it holds
	\begin{equation}\label{moltperu}
	-u\operatorname{div}z =  (f-u) u  \ \ \text{  in  } \mathcal{D}'(\Omega).
	\end{equation}
	




	\medskip
	
	Now we have to show \eqref{esistenzaz2} which  consists (recall Remark \ref{remdef}) in  proving both 
		\begin{equation}\label{zducontinua}
		z\cdot D^a u = \sqrt{1+ |D^a u|^2} + \sqrt{1-|z|^2}
	\end{equation}
	and
	\begin{equation} \label{zdusingolare}
		(z,Du)^s = |D^s u|.
	\end{equation}
	
	\medskip
	
	Hence,  let $0\le \varphi \in C^1_c(\Omega)$ and consider $v=u_p\varphi$ in  \eqref{pbpw}; this takes to
	
	\begin{equation}\begin{aligned}\label{lemmaz1}	
			& \int_\Omega u_p^2\varphi +\int_{\Omega} \frac{|\nabla u_p|^{2}\varphi}{\sqrt{1 +|\nabla u_p|^2}} + \int_{\Omega} \frac{\nabla u_p \cdot \nabla \varphi u_p}{\sqrt{1 +|\nabla u_p|^2}} + (p-1)\int_\Omega |\nabla u_p|^p\varphi
			\\
			& + (p-1)\int_\Omega |\nabla u_p|^{p-2}\nabla u_p\cdot \nabla \varphi u_p = \int_{\Omega}  f_p u_p \varphi.	
		\end{aligned}
	\end{equation}
	Dropping the nonnegative fourth term in \eqref{lemmaz1}, one gets
	\begin{equation}\begin{aligned}\label{lemmaz2}	
			&\int_\Omega u_p^2\varphi + \int_{\Omega} \sqrt{1 +|\nabla u_p|^2}\varphi - \int_{\Omega} \sqrt{1-\frac{|\nabla u_p|^{2}}{1 +|\nabla u_p|^2}}\varphi + \int_{\Omega} \frac{\nabla u_p \cdot \nabla \varphi u_p}{\sqrt{1 +|\nabla u_p|^2}}
			\\
			&+  (p-1)\int_\Omega |\nabla u_p|^{p-2}\nabla u_p\cdot \nabla \varphi u_p \le \int_{\Omega}  f_p u_p \varphi,	
	\end{aligned}\end{equation}
	where we used that	
	\begin{equation}
		\label{firstterm}
		\int_{\Omega} \frac{|\nabla u_p|^{2}\varphi}{\sqrt{1 +|\nabla u_p|^2}} = \int_{\Omega} \sqrt{1 +|\nabla u_p|^2}\varphi - \int_{\Omega} \sqrt{1-\frac{|\nabla u_p|^{2}}{1 +|\nabla u_p|^2}}\varphi.
	\end{equation}
	Now we aim to take the liminf as $p\to 1^+$ in \eqref{lemmaz2}. For the first term on the left-hand side of \eqref{lemmaz2} we can apply the Fatou Lemma. The second term on the left-hand side of \eqref{lemmaz2} is lower semicontinuous with respect to the $L^1$ convergence. The nonpositive third term on the left-hand side of \eqref{lemmaz2} is weakly lower semicontinuous with respect to the $L^1$ convergence (recall \eqref{j2}\bk).	Concerning  the fourth  term on the left-hand side of \eqref{lemmaz2} we use the weak$^*$ convergence of $\nabla u_p(1 +|\nabla u_p|^2)^{-\frac{1}{2}}$ to $z$ in $L^\infty(\Omega)^N$ as well as the strong convergence of $u_p$ in $L^q(\Omega)$ for any $q<2$ as $p\to 1^+$.
	Since $u_p$ weakly converges to $u$ in $L^2(\Omega)$ and $f_p$ strongly converges to $f$ in $L^2(\Omega)$ as $p\to 1^+$, then one can pass to the limit on the right-hand of \eqref{lemmaz2} as well.


	Let us finally focus on the  last term on the left-hand side of \eqref{lemmaz2} for which   we reason as for \eqref{termineenergiazero}. Indeed one can apply the H\"older inequality with indexes $\left(\frac{p}{p-1}, 2, \frac{2p}{2-p}\right)$ (recall that $1<p<2$) obtaining that
	
		
			\begin{equation*}\label{termineenergiazero2}
			\begin{aligned}
				(p-1)\left|\int_\Omega |\nabla u_p|^{p-2}\nabla u_p\cdot\nabla \varphi u_p \right| &\le (p-1) \left(\int_\Omega |\nabla u_p|^p\right)^{\frac{p-1}{p}} \left(\int_\Omega |\nabla \varphi|^{\frac{2p}{2-p}}\right)^{\frac{2-p}{2p}} \left(\int_\Omega |u_p|^2\right)^{\frac{1}{2}}
				\\
				&\le (p-1)^{\frac{1}{p}} ||\nabla \varphi||_{L^\infty(\Omega)^N}|\Omega|^{\frac{2-p}{2p}} \left((p-1)\int_\Omega |\nabla u_p|^p\right)^{\frac{p-1}{p}}\left(\int_\Omega |u_p|^2\right)^{\frac{1}{2}}
			\end{aligned}
		\end{equation*}
	whose right-hand goes to zero as $p\to1^+$ thanks to \eqref{stimatermineenergia} and to the boundedness in $L^2(\Omega)$ of $u_p$ with respect to $p$.
	
	
	Then we have proved that
	\begin{equation*}	
	\begin{aligned}	
		\int_\Omega u^2\varphi + \int_{\Omega} \sqrt{1 +|D u|^2}\varphi - \int_\Omega \sqrt{1-|z|^2}\varphi &\le - \int_{\Omega} uz\cdot \nabla \varphi + \int_{\Omega}  f u \varphi
		\\
		&\overset{\eqref{moltperu}}{=} -\int_{\Omega} uz\cdot \nabla \varphi-\int_\Omega u\operatorname{div}z \varphi + \int_\Omega u^2\varphi.	
	\end{aligned}
	\end{equation*}
	Hence, using  \eqref{dist1}, it holds
	\begin{equation}\label{zdu0}	
		\int_{\Omega} \sqrt{1 +|D u|^2}\varphi - \int_\Omega \sqrt{1-|z|^2}\varphi \le \int_\Omega (z, Du)\varphi,	\ \ \forall  \varphi \in C^1_c(\Omega), \ \varphi \ge 0.
	\end{equation}
	Now observe that since $\operatorname{div}z, u \in L^2(\Omega)$ one can apply Lemma \ref{lemanzas} which allows to deduce from inequality \eqref{zdu0} that
	\begin{equation*}\label{zdu1}
		z\cdot D^a u \ge \sqrt{1 +|D^a u|^2} - \sqrt{1-|z|^2},
	\end{equation*}
	almost everywhere in $\Omega$.
	This gives \eqref{zducontinua} since the reverse inequality is given by \eqref{ine:2}.
	
	Proving \eqref{zdusingolare} is immediate by observing that $||z||_{L^\infty(\Omega)^N}\le 1$ implies
	$$(z,Du)^s \le |D u|^s= |D^s u|,$$
	as measures in $\Omega$. The reverse inequality follows by \eqref{zdu0}.
	
	Finally let us show \eqref{equint};	we define the function ($k\ge\delta>0$):
	
	$$
	S_{\delta,k}(s)=
	\begin{cases}
		\sgn(s) &\text{if} \ \ |s|> k,
		\\
		0 &\text{if} \ \ |s|\le k-\delta,
		\\
		\displaystyle \frac{\sgn(s) (|s|-k+\delta)}{\delta} \ \ &\text{if} \ \ k-\delta< |s|\le k,
	\end{cases}
	$$
	and we take $v=S_{\delta,k}(u_p)$ in \eqref{pbpw} yielding to
	\begin{equation*}
		\int_\Omega u_p S_{\delta,k}(u_p) \le \int_\Omega f_pS_{\delta,k}(u_p)\le \int_{\{|u_p|> k-\delta\}} |f_p|,
	\end{equation*}
	getting rid of the nonnegative second and third term. Then taking the limsup first as  $p\to 1^+$ and then as $\delta \to 0^+$,  one obtains \eqref{equint}.
\end{proof}


\begin{lemma}\label{lemma_datoalbordoL2}
		Let $f\in L^2(\Omega)$. Then it holds
	$$u(\sgn{u} + [z,\nu])(x)=0 \ \text{for  $\mathcal{H}^{N-1}$-a.e. } x \in \partial\Omega,$$
	where $u$ and $z$ are the function and the vector field found in Corollary \ref{cor_u} and in Lemma \ref{lemma_esistenzazL2}.	
\end{lemma}
\begin{proof}
	 Let $u_p$ be a solution to \eqref{pbp} and let us take $v=u_p$ in \eqref{pbpw} yielding to
	\begin{equation*}
		\int_\Omega u_p^2 + \int_{\Omega} \frac{|\nabla u_p|^2}{\sqrt{1+|\nabla u_p|^2}} + \int_{\partial \Omega} |u_p| d\mathcal{H}^{N-1}  \le  \int_{\Omega}  f_pu_p,	
	\end{equation*}		
	since  $u_p$ has zero Sobolev trace.
	
	
	Moreover reasoning as for \eqref {firstterm} (there $\varphi =1$) one obtains
	\begin{equation*}\label{boundary0}
		\int_\Omega u_p^2 +\int_{\Omega} \sqrt{1+|\nabla u_p|^2} - \int_{\Omega} \sqrt{1-\frac{|\nabla u_p|^2}{1+|\nabla u_p|^2}} + \int_{\partial \Omega}|u_p| d\mathcal{H}^{N-1} \le \int_{\Omega}  f u_p.	
	\end{equation*}
	
	Now we can take the liminf as $p\to 1^+$ acting similarly to what done in Lemma \ref{lemma_esistenzazL2}. This allows to deduce that
	\begin{equation*}\label{boundary1}
		\int_\Omega u^2 + \int_{\Omega} \sqrt{1+|D u|^2} - \int_{\Omega} \sqrt{1-|z|^2} + \int_{\partial \Omega}|u| d\mathcal{H}^{N-1} \le  \int_{\Omega}  f u.	
	\end{equation*}
	Now, recalling \eqref{moltperu} and \eqref{green}, one can write 
	\begin{equation*}\label{bordo1}
		\begin{aligned}
			\int_\Omega u^2 + \int_{\Omega} \sqrt{1+|Du|^2} - \int_{\Omega} \sqrt{1-|z|^2} + \int_{\partial \Omega}|u| d\mathcal{H}^{N-1}  & \stackrel{\eqref{moltperu}}{\le}  -\int_\Omega u\operatorname{div}z +\int_\Omega u^2
			\\
			&\stackrel{\eqref{green}}{=} \int_{\Omega}(z,Du) - \int_{\partial \Omega} u[ z,\nu]d\mathcal{H}^{N-1} + \int_\Omega u^2.
		\end{aligned}	
	\end{equation*}
	Then the proof of Lemma \ref{lemma_datoalbordoL2} easily follows thanks  to \eqref{esistenzaz2} and to the fact that $|[z,\nu]|\leq 1$.
\end{proof}

\begin{proof}[Proof of Theorem \ref{teomain}]
	Let $u_p$ be a solution to \eqref{pbp}. Then it follows from Lemma \ref{lemma_stimeL2} that $u_p$ is bounded in $BV(\Omega)\cap L^2(\Omega)$ with respect to $p$. Corollary \ref{cor_u} guarantees that $u_p$ converges, up to subsequences, to $u$ in $L^q(\Omega)$ for every $q<2$, weak in $L^2(\Omega)$, and  $\nabla u_p$ converges to $D u$ weak$^*$ as measures as $p\to 1^+$.
	
	Then \eqref{def_distrp=1} and \eqref {def_zp=1} are proved in Lemma \ref{lemma_esistenzazL2}. Finally \eqref{def_bordop=1} follows from Lemma \ref{lemma_datoalbordoL2}. The proof is concluded.
\end{proof}

\subsection{Uniqueness of finite energy solutions}

In this section we prove Theorem \ref{teomainuniqueL2}. Let explicitly highlight that our proof of the uniqueness result is strongly related to the presence of the absorption term. \bk

\begin{proof}[Proof of Theorem \ref{teomainuniqueL2}]
	Let $u_1$ and $u_2$ be solutions to \eqref{pb} and let $z_1$ and $z_2$ be the corresponding vector fields. Using  \eqref{def_distrp=1} (recall Remark \ref{remdef}), we readily have 
	\begin{equation}\label{testL2}
		\int_\Omega u_i v -\int_\Omega v\,\operatorname{div} z_i = \int_\Omega  fv, \ \ \forall v\in BV(\Omega)\cap L^2(\Omega), \ \ i=1,2.
	\end{equation}	
	We take $v=u_1 - u_2$ in the difference between  two weak formulations \eqref{testL2} related to $u_1$ and $u_2$, obtaining
	
	\begin{equation}\label{unique1L2}
		\begin{aligned}
			&\int_\Omega (u_1-u_2)^2 + \int_\Omega (z_1, Du_1)-\int_\Omega(z_2, Du_1)
			\\
			&+ \int_\Omega(z_2, Du_2) - \int_\Omega(z_1, Du_2) - \int_{\partial\Omega}(u_1-u_2)[z_1,\nu]\, d\mathcal H^{N-1}				    	
			\\
			&+\int_{\partial\Omega}(u_1-u_2)[z_2,\nu]\, d\mathcal H^{N-1}						    	
			= 0,
		\end{aligned}
	\end{equation}
	
	after an application of \eqref{green}.
	
	Observe first that from \eqref{def_bordop=1} it holds
	$$
	u_{i}(\sgn{u_i}+[z_{i},\nu])=0\ \ \mathcal H^{N-1}-\text{a.e. on}\ \partial\Omega\  \ \text{for}\ \  i=1,2.
	$$
	
	Hence one can rewrite the boundary terms as
	
	\begin{equation}\label{boundaryuniqueL2}
		\begin{aligned}
			&-\int_{\partial\Omega}(u_1-u_2)[z_1,\nu]\, d\mathcal H^{N-1}		+ \int_{\partial\Omega}(u_1-u_2)[z_2,\nu]\, d\mathcal H^{N-1}
			\\
			&= 	\int_{\partial\Omega}(|u_1|  + u_2[z_1,\nu])\, d\mathcal H^{N-1}	+ \int_{\partial\Omega}(|u_2| + u_1[z_2,\nu])\, d\mathcal H^{N-1}	
			\\
			&= 	\int_{\partial\Omega}(|u_1|  + u_1[z_2,\nu])\, d\mathcal H^{N-1}	+ \int_{\partial\Omega}(|u_2| + u_2[z_1,\nu])\, d\mathcal H^{N-1},	
		\end{aligned}
	\end{equation}
	which is nonnegative since $|[z_i,\nu]|\le 1$ for $i=1,2$.
	
	Gathering \eqref{boundaryuniqueL2} into \eqref{unique1L2} gives that
	
	\begin{equation*}
		\begin{aligned}
			\int_{\Omega} (u_1-u_2)^2 &+ \int_\Omega (z_1, Du_1)-\int_\Omega(z_2, Du_1)
			\\
			&+ \int_\Omega(z_2, Du_2) - \int_\Omega(z_1, Du_2)  \le 0.
		\end{aligned}
	\end{equation*}
	
	Moreover, using \eqref{def_zp=1}, one gets
	
	\begin{equation*}
		\begin{aligned}
			&\int_{\Omega} (u_1-u_2)^2 + \int_\Omega\sqrt{1+|Du_1|^2} - \int_\Omega \sqrt{1-|z_1|^2} - \int_\Omega(z_2, Du_1)
			\\
			&+ \int_\Omega \sqrt{1+|Du_2|^2} - \int_\Omega \sqrt{1-|z_2|^2} - \int_\Omega(z_1, Du_2)  \le 0.
		\end{aligned}
	\end{equation*}
	
	
	Now we aim to prove that
	$$\sqrt{1+|Du_1|^2} - \sqrt{1-|z_2|^2}  \ge (z_2,Du_1)$$
	and that
	$$\sqrt{1+|Du_2|^2} - \sqrt{1-|z_1|^2}  \ge (z_1,Du_2)\,,$$
	as measures in $\Omega$.
	This easily follows by splitting the measures in the absolutely continuous and singular parts.
	Let observe that for the absolutely continuous part of the measures one needs that
	$$\sqrt{1+|D^a u_1|^2} - \sqrt{1-|z_2|^2}  \ge (z_2,Du_1)^a=z_2\cdot D^a u_1$$
	and that
	$$\sqrt{1+|D^a u_2|^2} - \sqrt{1-|z_1|^2}  \ge (z_1,Du_2)^a=z_1\cdot D^a u_2,$$
	which are given by  \eqref{ine:2} once one recalls that
	$$|z_i| = \frac{|D^a u_i|}{\sqrt{1+|D^a u_i|^2}}\leq 1\ \ \ i=1,2.$$
	For the singular part it is sufficient to recall that $||z_i||_{L^\infty(\Omega)^N}\le 1$.
	
	Hence we have shown that
	
	\begin{equation*}
		\begin{aligned}
			\int_{\Omega} (u_1-u_2)^2=0,
		\end{aligned}
	\end{equation*}
	which concludes the proof.
\end{proof}

\section{The  case of $L^1$ data}
\label{secL1}

Here we deal with \eqref{pb} in presence of a merely integrable datum $f$.

In this case one can not expect finite energy solutions. We specify how a weak solution of problem \eqref{pb} is meant in this case.
\begin{defin}
	\label{weakdefL1}
	Let $f\in L^1(\Omega)$. A function $u\in L^1(\Omega)$ with $T_k(u)\in BV(\Omega)$ for any $k>0$ is a solution to problem \eqref{pb} if there exists $z\in X(\Omega)_1$ with $||z||_{L^\infty(\Omega)^N}\le 1$ such that
	\begin{align}
		&u-\operatorname{div}z = f \ \ \text{in}\ \ \mathcal{D}'(\Omega), \label{def_distrp=1L1}
		\\
		&(z,DT_k(u))=\sqrt{1+|DT_k(u)|^2} - \sqrt{1-|z_k|^2} \label{def_zp=1L1} \ \text{as measures in } \Omega \ \text{with} \ z_k:=z\chi_{\{|u|\le k\}},
		\\
		&T_k(u)(\sgn{T_k(u)} + [z,\nu])(x)=0 \label{def_bordop=1L1} \ \text{and for  $\mathcal{H}^{N-1}$-a.e. } x \in \partial\Omega,
	\end{align}
	for any $k>0$.
\end{defin}

\begin{remark}\label{remarkdefinizioni}
	Let explicitly observe that a function $u$, solution to \eqref{pb} in the sense of Definition \ref{weakdef}, is also a solution to the same problem in the sense of Definition \ref{weakdefL1}. Indeed, if $z_k=z\chi_{\{|u|\le k\}}$, it follows from \eqref{zesplicito} that
	\begin{equation}\label{1remzL1}
		z \cdot D^aT_k(u)=\sqrt{1+|D^aT_k(u)|^2} - \sqrt{1-|z_k|^2}.
	\end{equation}
	Moreover, since $(z,DT_k(u))\le |DT_k(u)|$ one has that $(z,DT_k(u))^s\le |D^sT_k(u)|$. For the reverse inequality it is sufficient to observe that
	\begin{equation*}
		(z,Du)^s = (z,DT_k(u))^s + (z,DG_k(u))^s=|D^s u| \ge |D^s T_k(u)|.
	\end{equation*}
	As $(z,DT_k(u))$ and $(z,DG_k(u))$ are mutually singular measures, one yields to
	\begin{equation}\label{2remzL1}
		(z,DT_k(u))^s= |D^s T_k(u)|.
	\end{equation}
	Equations \eqref{1remzL1} and \eqref{2remzL1} show that \eqref{def_zp=1L1} holds. This is sufficient to conclude that $u$ is a solution to \eqref{pb} in the sense of Definition \ref{weakdefL1}.
\end{remark}

\begin{theorem}\label{teoL1}
	Let $f\in L^1(\Omega)$ then there exists a solution to problem \eqref{pb} in the sense of Definition \ref{weakdefL1}.
\end{theorem}

We also state the following uniqueness result:
\begin{theorem}\label{teomainunique}
There is at most one solution to problem \eqref{pb} in the sense of Definition \ref{weakdefL1}.
\end{theorem}


\subsection{Existence of infinite energy solutions}

By appealing to the results of Section \ref{secL2},  we work by approximation via the following problems
\begin{equation}
	\label{pbL1approx}
	\begin{cases}
		u_n\dis -\operatorname{div}\left(\frac{D u_n}{\sqrt{1+|D u_n|^2}}\right) = f_n & \text{in}\;\Omega,\\
		u_n=0 & \text{on}\;\partial\Omega,
	\end{cases}
\end{equation}
where $f_n:= T_n(f)$. The existence of a unique solution  $u_n\in BV(\Omega)\cap L^{2}(\Omega)$ is proved in Theorem \ref{teomain} and Theorem \ref{teomainuniqueL2}. This means that there exists $z_n\in X(\Omega)_2$ with $||z_n||_{L^\infty(\Omega)^N}\le 1$ such that it holds
\begin{align}
	&	\int_\Omega u_n v -\int_\Omega v\,\operatorname{div} z_n = \int_\Omega  f_nv, \ \ \forall v\in BV(\Omega)\cap L^{2}(\Omega),\ \  \label{def_distrp=1approx}
	\\
	&(z_n,Du_n)=\sqrt{1+|Du_n|^2} - \sqrt{1-|z_n|^2} \label{def_zp=1approx} \ \ \ \ \text{as measures in } \Omega,
	\\
	&u_n(\sgn{u_n} + [z_n,\nu])(x)=0 \label{def_bordop=1approx}\ \ \ \text{for  $\mathcal{H}^{N-1}$-a.e. } x \in \partial\Omega.
\end{align}

We begin by proving estimates in $BV(\Omega)$ with respect to $n$ for any truncation of the approximating solutions.

\begin{lemma}\label{lemma_stimeL1}
	Let $f\in L^1(\Omega)$ and let $u_n$ be the solution to \eqref{pbL1approx}. Then $T_k(u_n)$ is bounded in $BV(\Omega)$ with respect to $n$ and for any $k>0$. Moreover there exists an almost everywhere finite function $u$ such that $T_k(u)\in BV(\Omega)$ for any $k>0$ and such that $u_n$, up to subsequences, converges to $u$ in $L^1(\Omega)$ as $n\to\infty$.
\end{lemma}
\begin{proof}
Let $k>0$ and let us fix $v=T_k(u_n)$ in \eqref{def_distrp=1approx} obtaining
\begin{equation*}
	\int_\Omega u_n T_k(u_n) - \int_\Omega T_k(u_n)\,\operatorname{div} z_n \le k \|f\|_{L^1(\Omega)}.
\end{equation*}
Then recalling \eqref{green}, \eqref{def_bordop=1approx} and the fact that $u_n\in BV(\Omega)$, one gets
\begin{equation*}
	\int_\Omega u_n T_k(u_n) + \int_\Omega (z_n, D T_k(u_n)) + \int_{\partial\Omega} |T_k(u_n)| \ d \mathcal{H}^{N-1} \le k \|f\|_{L^1(\Omega)}.
\end{equation*}
Now, recalling \eqref{def_zp=1approx} and the discussion in Remark \ref{remarkdefinizioni}, one has
\begin{equation}\label{stimaL1}
	\int_\Omega u_n T_k(u_n) + \int_\Omega \sqrt{1 + |D T_k(u_n)|^2} - \int_\Omega \sqrt{1-|z_n|^2\chi_{\{|u_n|\le k\}}} + \int_{\partial\Omega} |T_k(u_n)| \ d \mathcal{H}^{N-1} \le k \|f\|_{L^1(\Omega)},
\end{equation}
which means that $T_k(u_n)$ is bounded in $BV(\Omega)$ with respect to $n$ for any $k>0$. This is sufficient to deduce the existence of a limit function $u$ to whom $u_n$ converges, up to subsequences, almost everywhere in $\Omega$ as $n\to\infty$. Moreover $T_k(u) \in BV(\Omega)$. Finally,  \eqref{equint} implies that $u_n$ is bounded in $L^1(\Omega)$ with respect to $n$, one has that $u$ is almost everywhere finite in $\Omega$; moreover,  
	\begin{equation}\label{equi}
	\int_{\{|u_n|\ge k\}} |u_n| \le \int_{\{|u_n| \ge k\}} |f_n|,
\end{equation}
which means that $u_n$ is equi-integrable since $f_n$ converges in $L^1(\Omega)$ and $u_n$ is bounded in $L^1(\Omega)$ with respect to $n$. Hence this is sufficient to deduce that $u_n$ converges to $u$ in $L^1(\Omega)$ as $n\to\infty$.

\end{proof}


\begin{lemma}\label{lemma_esistenzaL1}
	Let $f\in L^1(\Omega)$. Then there exists $z\in X(\Omega)_1$ such that
	\begin{equation}\label{esistenzaz1L1}
		u-\operatorname{div}z= f  \ \ \ \text{in  } \mathcal{D'}(\Omega),
	\end{equation}
	and
	\begin{equation}\label{esistenzaz2L1}
		(z,DT_k(u))=\sqrt{1+|DT_k(u)|^2} - \sqrt{1-|z_k|^2} \ \ \text{as measures in } \Omega \ \text{and for any }k>0,
	\end{equation}
	where $u$ is the function found in Lemma \ref{lemma_stimeL1} and $z_k=z\chi_{\{|u|\le k\}}$. 
\end{lemma}
\begin{proof}
Let $u_n$ be the solution to \eqref{pbL1approx} with vector field $z_n$ such that $|z_n|\le 1$. Then there exists $z\in L^\infty(\Omega)^N$ such that $z_n$ converges to $z$ weak$^*$ in $L^\infty(\Omega)^N$ as $n\to\infty$ and such that $||z||_{L^\infty(\Omega)^N} \le 1$. Then, recalling that from Lemma \ref{lemma_stimeL1} $u_n$ converges, up to subsequences, to $u$ in $L^1(\Omega)$ as $n\to\infty$, it is easy to prove that \eqref{esistenzaz1L1} holds since $f_n$ converges to $f$	in $L^1(\Omega)$.

Now in order to prove \eqref{esistenzaz2L1} one can take $v=T_k(u_n)\varphi$ ($k>0$ and $0\le\varphi\in C^1_c(\Omega)$) in \eqref{def_distrp=1approx} getting to

\begin{equation*}
\int_\Omega u_n T_k(u_n)\varphi -\int_\Omega T_k(u_n)\varphi\operatorname{div} z_n = \int_\Omega  f_nT_k(u_n)\varphi,
\end{equation*}
that, using \eqref{dist1}, gives

\begin{equation}\label{campozTK}
	\int_\Omega (z_n,D T_k(u_n))\varphi =  \int_\Omega  (f_n-u_n)T_k(u_n)\varphi - \int_\Omega z_n\cdot\nabla \varphi T_k(u_n).
\end{equation}
Then, recalling  Remark \ref{remarkdefinizioni}, one has  that,  for every $k>0$
	\begin{equation}\label{esistenzaz2TK}
 (z_n,DT_k(u_n))=\sqrt{1+|DT_k(u_n)|^2} - \sqrt{1-|z_n|^2\chi_{\{|u_n|\le k\}}},\ \text{as measures in $\Omega$},
 \end{equation}
which gathered in \eqref{campozTK} yields to 
\begin{equation*}
	\int_\Omega \sqrt{1+|D T_k(u_n)|^2}\varphi -\int_\Omega \sqrt{1-|z_n|^2\chi_{\{|u_n|\le k\}}}\varphi =  \int_\Omega  (f_n-u_n)T_k(u_n)\varphi - \int_\Omega z_n\cdot\nabla \varphi T_k(u_n).
\end{equation*}
Now one can let $n$ go to $\infty$ in the previous identity recalling that the left-hand is lower semicontinuous as already shown in the proof of Lemma \ref{lemma_esistenzazL2}. In particular, for the second term on the left-hand one uses that $z_n\chi_{\{|u_n|\le k\}}$ converges to $z\chi_{\{|u|\le k\}}$ weakly in $L^1(\Omega)^N$ as $n\to\infty$, for almost every $k>0$. 
Moreover the first term on the right-hand simply passes to the limit since $f_n$,$u_n$ converge in $L^1(\Omega)$ and $T_k(u_n)$ converges weak$^*$ in $L^\infty(\Omega)$.  This argument takes to (recall that $z_k:=z\chi_{\{|u|\le k\}}$)

\begin{equation*}
	\begin{aligned}
	\int_\Omega \sqrt{1+|D T_k(u)|^2}\varphi -\int_\Omega \sqrt{1-|z_k|^2}\varphi &\le  \int_\Omega  (f-u)T_k(u)\varphi - \int_\Omega z\cdot\nabla \varphi T_k(u)
	\\
	&= -\int_\Omega \operatorname{div}z T_k(u)\varphi - \int_\Omega z\cdot\nabla \varphi T_k(u)
	\\
	&=\int_\Omega (z,DT_k(u))\varphi,
	\end{aligned}
\end{equation*}
where the last passages follow from \eqref{esistenzaz1L1} and \eqref{dist1} respectively. From now on the reasoning to deduce \eqref{esistenzaz2L1} is similar to the one given in the proof of Lemma \ref{lemma_esistenzazL2}. Indeed it is sufficient to observe that $z\cdot D^aT_k(u) = z_k\cdot D^aT_k(u)$. This shows \eqref{esistenzaz2L1} for almost every $k>0$. Now observe that, reasoning as in Remark \ref{remdef}, from \eqref{esistenzaz2L1} one readily gets
\begin{equation}\label{zesplicitok}
	z_k= \frac{D^a T_k(u)}{\sqrt{1+|D^a T_k(u)|^2}},
\end{equation}
for almost every $k>0$. We claim that, for any fixed $k>0$, $z=0$ almost everywhere in $\{|u|=k\}$. If this is the case, then $z_n\chi_{\{|u_n|\le k\}}$ converges to $z\chi_{\{|u|\le k\}}$ weakly in $L^1(\Omega)^N$ as $n\to\infty$, for every $k>0$ and this concludes the proof. 
Let finally check the claim; let us fix $h>k$ such that \eqref{zesplicitok} holds for $z_h=z\chi_{\{|u|\le h\}}$. Then, since $z_h=0$ almost everywhere in $\{|u|=k\}$, also $z=0$ almost everywhere on the same set.  



\end{proof}
\begin{remark}\label{benilan}
Looking at \eqref{zesplicitok} one would like to identify $z$
 as
\begin{equation}\label{ben}
 z= \frac{D^a u}{\sqrt{1+|D^a u|^2}}\,.
 \end{equation}
 This is not accurate as we only ask for $u$ to have truncations in $BV(\Omega)$ so that $D^a u$ is not well defined in general. 
 
 Nevertheless,  reasoning as in \cite{b6} it is possible to define (see fon instance \cite[Lemma 1]{ACM_MA}\bk) a generalized gradient for functions whose truncation is in  $BV(\Omega)$ which,  in turn, \eqref{ben} holds a.e. in $\Omega$. 
Indeed, let $u$ be a measurable function finite a.e. on $\Omega$ such that $T_k(u)\in BV(\Omega)$ for any $k>0$. Then $D^a T_k(u)$ is well defined for any $k>0$. A standard argument allows us to select a unique measurable vector function $v:\Omega\to\rn$ that satisfies 
$$
v_{\chi_{\{|u|\leq k\}}} = D^a T_k(u). 
$$

It is possible to show that, if $u\in BV(\Omega)$, then $v=D^a u$. 

Using this generalized gradient, the vector field $z$ given in Definition \ref{weakdefL1} can be uniquely identified by \eqref{ben}.  
\end{remark}


\begin{remark}\label{conv_zn}
	For subsequent use, we underline that in the previous proof we have shown that  $z_n\chi_{\{|u_n|\le k\}}$ converges to $z\chi_{\{|u|\le k\}}$ weakly in $L^1(\Omega)^N$ as $n\to\infty$ and for every $k>0$.
\end{remark}




\begin{lemma}\label{lemma_datoalbordoL1}
	Let $f\in L^1(\Omega)$. Then it holds for any $k>0$
	$$T_k(u)(\sgn{T_k(u)} + [z,\nu])(x)=0 \ \text{for  $\mathcal{H}^{N-1}$-a.e. } x \in \partial\Omega,$$
	where $u$ and $z$ are the function and the vector field found in Lemma \ref{lemma_stimeL1} and in Lemma \ref{lemma_esistenzaL1}.	
\end{lemma}
\begin{proof}
	 Let $u_n$ be the solution of \eqref{pbL1approx} and let us pick $v=T_k(u_n)$ in \eqref{pbL1approx} yielding to
	\begin{equation*}
		\int_\Omega u_nT_k(u_n) - \int_{\Omega} \operatorname{div}z_n T_k(u_n) =  \int_{\Omega}  f_nT_k(u_n),	
	\end{equation*}		
	and, after an application of the \eqref{green}, to 
	\begin{equation}\label{bordoL1_1}
		\int_{\Omega} (z_n,DT_k(u_n)) - \int_{\partial\Omega} T_k(u_n)[z_n,\nu] =  \int_{\Omega}  (f_n-u_n)T_k(u_n).	
	\end{equation}	
	Now using both \eqref{esistenzaz2TK} and \eqref{def_bordop=1approx}, it follows from \eqref{bordoL1_1} that
	\begin{equation}\label{bordoL1_2}
	\int_{\Omega} \sqrt{1+|DT_k(u_n)|^2} - \int_\Omega \sqrt{1-|z_n|^2\chi_{\{|u_n|\le k\}}} + \int_{\partial\Omega} |T_k(u_n)| =  \int_{\Omega}  (f_n-u_n)T_k(u_n).	
	\end{equation}		
	Recalling also Remark \ref{conv_zn}, we can take $n\to\infty$ by lower semicontinuity of the left-hand of \eqref{bordoL1_2}. For the right-hand it is  sufficient to use  the strong convergence of both $f_n$ and $u_n$ in $L^1(\Omega)$ and the $*$-weak convergence in $L^\infty(\Omega)$ of $T_k(u_n)$ as $n\to\infty$. Then one deduces
	
	\begin{equation}\label{bordoL1_3}
		\int_{\Omega} \sqrt{1+|DT_k(u)|^2} - \int_\Omega \sqrt{1-|z_k|^2} + \int_{\partial\Omega} |T_k(u)| \le  \int_{\Omega}  (f-u)T_k(u).	
	\end{equation}
	Now observe that from \eqref{esistenzaz1L1} one has that $(f-u)T_k(u)=-\operatorname{div}zT_k(u)$. Then an application of \eqref{green} in \eqref{bordoL1_3} gives
	\begin{equation*}
	\int_{\Omega} \sqrt{1+|DT_k(u)|^2} - \int_\Omega \sqrt{1-|z_k|^2} + \int_{\partial\Omega} |T_k(u)| \le   \int_{\Omega}  (z,DT_k(u)) - \int_{\partial\Omega} T_k(u)[z,\nu],	
	\end{equation*}	
	which, from \eqref{esistenzaz2L1}, implies
	\begin{equation*}
		\int_{\partial\Omega} |T_k(u)| + \int_{\partial\Omega} T_k(u)[z,\nu] \le 0,	
	\end{equation*}
	and this concludes the proof since $|[z,\nu]|\le 1$.
\end{proof}


\begin{proof}[Proof of Theorem \ref{teoL1}]
	Let $u_n$ be the solution to \eqref{pbL1approx}. It follows from Lemma \ref{lemma_stimeL1} that $T_k(u_n)$ is bounded in $BV(\Omega)$ with respect to $n$ and for any $k>0$. Moreover, up to subsequences, $u_n$  converges to $u$ in $L^1(\Omega)$ as $n\to\infty$.
	Requests \eqref{def_distrp=1L1} and \eqref {def_zp=1L1} are proved in Lemma \ref{lemma_esistenzaL1}. The boundary condition \eqref{def_bordop=1L1} is shown in Lemma \ref{lemma_datoalbordoL1}. The proof is concluded.
\end{proof}


\subsection{Uniqueness of infinite energy solutions}


In this section we prove the uniqueness Theorem \ref{teomainunique} by strictly follow the lines of the proof of Theorem \ref{teomainuniqueL2}  and appealing  from the presence of the absorption term. \bk 

\begin{proof}[Proof of Theorem \ref{teomainunique}]
	Let $u_1$ and $u_2$ be solutions to \eqref{pb} and let $z_1$ and $z_2$ be the corresponding vector fields.
	
	Then one has that
	\begin{equation}\label{test}
		\int_\Omega u_i v -\int_\Omega v\,\operatorname{div} z_i = \int_\Omega  fv, \ \ \forall v\in BV(\Omega)\cap L^\infty(\Omega), \ \ i=1,2.
	\end{equation}	
	Let observe that the main difference with respect to the proof of Theorem \ref{teomainuniqueL2} relies on the fact that $u_1,u_2$ are not suitable test functions in \eqref{test} anymore.
	
	Hence we have to take $v=T_k(u_1) - T_k(u_2)$ in the difference between two weak formulations \eqref{test} related to $u_1$ and $u_2$, yielding to
	
	\begin{equation}\label{unique1}
		\begin{aligned}
			&\int_\Omega (u_1-u_2)(T_k(u_1) - T_k(u_2)) + \int_\Omega (z_1, DT_k(u_1))-\int_\Omega(z_2, DT_k(u_1))
			\\
			&+ \int_\Omega(z_2, DT_k(u_2)) - \int_\Omega(z_1, DT_k(u_2)) - \int_{\partial\Omega}(T_k(u_1)-T_k(u_2))[z_1,\nu]\, d\mathcal H^{N-1}				    	
			\\
			&+\int_{\partial\Omega}(T_k(u_1)-T_k(u_2))[z_2,\nu]\, d\mathcal H^{N-1}						    	
			= 0,
		\end{aligned}
	\end{equation}
	
	where we also used \eqref{green}.
	
	From \eqref{def_bordop=1L1} one has
	$$
	T_k(u_{i})(\sgn{T_k(u_i)}+[z_{i},\nu])=0\ \ \mathcal H^{N-1}-\text{a.e. on}\ \partial\Omega\  \ \text{for}\ \  i=1,2.
	$$
	
This means that
	
	\begin{equation}\label{boundaryunique}
		\begin{aligned}
			&-\int_{\partial\Omega}(T_k(u_1)-T_k(u_2))[z_1,\nu]\, d\mathcal H^{N-1}		+ \int_{\partial\Omega}(T_k(u_1)-T_k(u_2))[z_2,\nu]\, d\mathcal H^{N-1}
			\\
			&= 	\int_{\partial\Omega}(|T_k(u_1)|  + T_k(u_2)[z_1,\nu])\, d\mathcal H^{N-1}	+ \int_{\partial\Omega}(|T_k(u_2)| + T_k(u_1)[z_2,\nu])\, d\mathcal H^{N-1}	
			\\
			&= 	\int_{\partial\Omega}(|T_k(u_1)|  + T_k(u_1)[z_2,\nu])\, d\mathcal H^{N-1}	+ \int_{\partial\Omega}(|T_k(u_2)| + T_k(u_2)[z_1,\nu])\, d\mathcal H^{N-1},	
		\end{aligned}
	\end{equation}
	which is nonnegative since $|[z_i,\nu]|\le 1$ for $i=1,2$.
	
	Gathering \eqref{boundaryunique} into \eqref{unique1}, it  holds that
	
	\begin{equation*}
		\begin{aligned}
			\int_{\{|u_1|\le k,|u_2|\le k\}} (u_1-u_2)^2 &+ \int_\Omega (z_1, DT_k(u_1))-\int_\Omega(z_2, DT_k(u_1))
			\\
			&+ \int_\Omega(z_2, DT_k(u_2)) - \int_\Omega(z_1, DT_k(u_2))  \le 0.
		\end{aligned}
	\end{equation*}
	
	Moreover, using \eqref{def_zp=1L1}, one gets ($z_{i,k}:= z_i\chi_{\{|u_i|\le k\}}$ for $i=1,2$)
		
		\begin{equation*}
			\begin{aligned}
				&\int_{\{|u_1|\le k,|u_2|\le k\}} (u_1-u_2)^2 + \int_\Omega\sqrt{1+|DT_k(u_1)|^2} - \int_\Omega \sqrt{1-|z_{1,k}|^2} - \int_\Omega(z_2, DT_k(u_1))
				\\
				&+ \int_\Omega \sqrt{1+|DT_k(u_2)|^2} - \int_\Omega \sqrt{1-|z_{2,k}|^2} - \int_\Omega(z_1, DT_k(u_2))  \le 0.
			\end{aligned}
		\end{equation*}
		
		
		Now we  claim  that both
		$$\sqrt{1+|DT_k(u_1)|^2} - \sqrt{1-|z_{2,k}|^2}  \ge (z_2,DT_k(u_1))$$
		and 
		$$\sqrt{1+|DT_k(u_2)|^2} - \sqrt{1-|z_{1,k}|^2}  \ge (z_1,DT_k(u_2))\,$$
		hold as measures in $\Omega$.
		Once again, this follows by splitting it in the absolutely continuous and singular parts.
		For the absolutely continuous part of the measures one needs that
		$$\sqrt{1+|D^a T_k(u_1)|^2} - \sqrt{1-|z_{2,k}|^2}  \ge (z_2,DT_k(u_1))^a=z_2\cdot D^a T_k(u_1)$$
		and that
		$$\sqrt{1+|D^a T_k(u_2)|^2} - \sqrt{1-|z_{1,k}|^2}  \ge (z_1,DT_k(u_2))^a=z_1\cdot D^a T_k(u_2),$$
		which is inequality \eqref{ine:2} once one notices that $z_2\cdot D^a T_k(u_1)=z_{2,k}\cdot D^aT_k(u_1)$ and $z_1\cdot D^a T_k(u_2)=z_{1,k}\cdot D^aT_k(u_2)$.
	
	For the singular part it is sufficient to recall that $||z_i||_{L^\infty(\Omega)^N}\le 1$.
	
	This proves that
	
	\begin{equation*}
		\begin{aligned}
			\int_{\{|u_1|\le k,|u_2|\le k\}} (u_1-u_2)^2=0,
		\end{aligned}
	\end{equation*}
for any $k>0$. The proof is concluded.
\end{proof}


\section{Bounded and unbounded solutions}
\label{boun}
In this section we aim to give a sharp condition under which solutions to \eqref{pb} are bounded.

If one considers $f\in L^N(\Omega)$ then, as $N\ge 2$, it follows from Theorem \ref{teomain} and Theorem \ref{teomainuniqueL2} that there exists a unique solution $u$ to \eqref{pb} in the sense of Definition \ref{weakdef}. In the next theorem we prove that actually $u$ is bounded no matter of any smallness condition on $f$. The proof is based on that of the $L^\infty$--estimate in \cite[Theorem 1]{DS}.

\begin{theorem}\label{teobounded}
	Let $f\in L^N(\Omega)$ then the solution to \eqref{pb} in the sense of Definition \ref{weakdef} is bounded.
\end{theorem}
\begin{proof}
	Let us test \eqref{def_distrp=1} with $G_k(u)\in BV(\Omega)\cap L^2(\Omega)$ where $k>0$, yielding to
	\begin{equation}\label{bounded1}
		\int_\Omega uG_k(u) - \int_\Omega G_k(u)\,\operatorname{div} z = \int_\Omega f G_k(u).
	\end{equation}	
	For the right-hand of \eqref{bounded1} we write
	\begin{equation}\label{bounded2}
		\begin{aligned}
		\int_\Omega f G_k(u) &= \int_{\{|f|\le k\}} f G_k(u) + \int_{\{|f|> k\}}f G_k(u)
		\\
		&\le k\int_\Omega G_k(u) + \|f\chi_{\{|f|> k\}}\|_{L^N(\Omega)} \left(\int_{\Omega} G_k^{\frac{N}{N-1}}(u)\right)^{\frac{N-1}{N}}
		\\
		&\le k\int_\Omega G_k(u) + \|f\chi_{\{|f|> k\}}\|_{L^N(\Omega)} \mathcal{S}_1 \left(\int_{\Omega} |DG_k(u)| +\int_{\partial \Omega} |G_k(u)|\, d\mathcal H^{N-1}\right),
	\end{aligned}
	\end{equation}	
	after applications of the H\"older and Sobolev inequalities where $\mathcal{S}_1$ is the best constant in the Sobolev inequality for functions in $BV$. Now we gather \eqref{bounded2} into \eqref{bounded1} noticing that the first term on the right-hand of \eqref{bounded2} is less or equal than the first term on the left-hand of \eqref{bounded1}; this allows to deduce
		\begin{equation}\label{bounded3}
		\int_\Omega (z, DG_k(u)) - \int_{\partial\Omega} G_k(u)[z,\nu]\, d\mathcal H^{N-1} \le \|f\chi_{\{|f|> k\}}\|_{L^N(\Omega)} \mathcal{S}_1 \left(\int_{\Omega} |DG_k(u)| +\int_{\partial \Omega} |G_k(u)|\, d\mathcal H^{N-1} \right),
	\end{equation}	
	where we also used \eqref{green}. In particular, since from \eqref{def_bordop=1} $[z,\nu] = -\sgn u$ if $u \not=0 $ on $\partial\Omega$, one has
	$$- \int_{\partial\Omega} G_k(u)[z,\nu]\, d\mathcal H^{N-1} = \int_{\partial\Omega} |G_k(u)|\, d\mathcal H^{N-1},$$
	which gathered in \eqref{bounded3} gives
			\begin{equation}\label{bounded4}
		\int_\Omega (z, DG_k(u)) + \int_{\partial\Omega} |G_k(u)|\, d\mathcal H^{N-1} \le \|f\chi_{\{|f|> k\}}\|_{L^N(\Omega)} \mathcal{S}_1 \left(\int_{\Omega} |DG_k(u)| +\int_{\partial \Omega} |G_k(u)|\, d\mathcal H^{N-1} \right).
	\end{equation}
	Now observe that from \eqref{def_zp=1} one can write
		\begin{equation*}\label{zTkGk}
		(z,DT_k(u)) + (z,DG_k(u)) \ge  |DG_k(u)| - \sqrt{1-|z|^2}\chi_{\{|u|\le k\}} - \sqrt{1-|z|^2}\chi_{\{|u|> k\}},
	\end{equation*}
 	as measures in $\Omega$. Moreover, as $(z,DT_k(u))$ and $(z,DG_k(u))$ are orthogonal measures then one gets
 			\begin{equation}\label{zTkGk2}
 		(z,DG_k(u)) \ge  |DG_k(u)| - \sqrt{1-|z|^2}\chi_{\{|u|> k\}},
 	\end{equation}
 as measures in $\Omega$. Collecting \eqref{zTkGk2} into \eqref{bounded4} one has
 			\begin{equation*}\label{bounded5}
 				\begin{aligned}
 	\int_\Omega |DG_k(u)| + \int_{\partial\Omega} |G_k(u)|\, d\mathcal H^{N-1} &\le \|f\chi_{\{|f|> k\}}\|_{L^N(\Omega)} \mathcal{S}_1 \left(\int_{\Omega} |DG_k(u)| +\int_{\partial \Omega} |G_k(u)|\, d\mathcal H^{N-1} \right)
 	\\
 	&+ \int_\Omega \sqrt{1-|z|^2}\chi_{\{|u|> k\}}.
 \end{aligned}	
 \end{equation*}
Now one can choose $k$ great enough such that $\|f\chi_{\{|f|> k\}}\|_{L^N(\Omega)} \mathcal{S}_1<1$ in order to deduce that
  			\begin{equation*}\label{bounded6}
 		\int_\Omega |DG_k(u)| + \int_{\partial\Omega} |G_k(u)|\, d\mathcal H^{N-1} \le \frac{|\{|u|> k\}|}{1-\|f\chi_{\{|f|> k\}}\|_{L^N(\Omega)} \mathcal{S}_1}.
 \end{equation*}
An application of the Sobolev inequality gives that ($h>k$)
  	\begin{equation*}\label{bounded7}
	|h-k| |\{|u|> h\}|^{\frac{N-1}{N}}\le \left(\int_\Omega |G_k^{\frac{N}{N-1}}(u)|\right)^{\frac{N-1}{N}}  \le \frac{\mathcal{S}_1|\{|u|> k\}|}{1-\|f\chi_{\{|f|> k\}}\|_{L^N(\Omega)} \mathcal{S}_1},
\end{equation*}
which is
  	\begin{equation}\label{bounded8}
	 |\{|u|> h\}|\le \frac{\mathcal{S}^{\frac{N}{N-1}}_1|\{|u|> k\}|^{\frac{N}{N-1}}}{\left(\left(1-\|f\chi_{\{|f|> k\}}\|_{L^N(\Omega)} \mathcal{S}_1\right) |h-k|\right)^{\frac{N}{N-1}}}.
\end{equation}
Estimate \eqref{bounded8} is sufficient to apply standard Stampacchia machinery (see \cite{st}) in order to deduce that $u \in L^\infty(\Omega)$. The proof is concluded.
\end{proof}

Theorem \ref{teobounded} is sharp as the next example shows. We consider data lying in the Marcinkiewicz space $L^{N,\infty}(\Omega)$; again,  refer to the monograph \cite{PKF} for an introduction and basic properties.
In particular for functions in $W^{1,1}_0(\Omega)$ the natural embedding is in $L^{N,\infty}(\Omega)$ where the best Sobolev constant is given by $\tilde{\mathcal S}_{1}= [(N-1)\omega_{N}^{\frac{1}{N}}]^{-1}$.

It is worth mentioning that, reasoning similarly to the proof of Theorem \ref{teobounded}, one can show that the solution to \eqref{pb} is bounded if $\|f\|_{L^{N,\infty}(\Omega)}<\tilde{\mathcal S}_{1}^{-1}$.

Anyway the following example shows that above this critical threshold the unique solution is not bounded anymore and that the absorption term does not regularize the solution that much.

\begin{example}\label{example}
	Let us fix $\Omega = B_1(0)$, let $N\geq 3$ and let $0<\alpha < N-1$; it is not difficult to be  convinced that a radial solution to problem
	$$
	\label{pbie}
	\begin{cases}
		\dis -\operatorname{div}\left(\frac{D u_\alpha }{\sqrt{1+|D u_\alpha|^2}}\right) + u_\alpha = \frac{N-1}{|x|} \left(v_\alpha (|x|)+\frac{r^{-\alpha+1}-r}{N-1}\right)=:f_\alpha(x)& \text{in}\; B_1(0),\\
		u_\alpha=0 & \text{on}\;\partial B_1(0)\,
	\end{cases}
	$$
	is given, if $|x|=r$,  by $u_\alpha (x)=r^{-\alpha} - 1$ provided $v_\alpha:(0,1)\mapsto \mathbb{R}^+$ is given by
	
	$$
	v_{\alpha}(r)=\frac{\alpha r^{-\alpha-1}(\alpha^2 r^{-2\alpha -2}- \frac{\alpha+2 - N}{N-1})}{(1+\alpha^2 r^{-2\alpha-2})^{\frac{3}{2}}}\,.
	$$
	An explicit calculation of the norm gives that
	$$
	\left\|f_\alpha(x)\right\|_{L^{N,\infty}(B_1(0))}  = (N-1)\omega_{N}^{\frac{1}{N}} = \tilde{\mathcal S}_{1}^{-1};
	$$
 a similar computation can be found in Example $1$ of \cite{gop2}.
\end{example}



\section{More general lower order terms}
\label{more}
With the approach of previous sections in mind, it is possible to deduce an existence and uniqueness  result when dealing with  a more general lower order term involving a nonlinear function of  $u$. Let us  consider
\begin{equation}
	\label{pbL1g}
	\begin{cases}
		g(u)\dis -\operatorname{div}\left(\frac{D u}{\sqrt{1+|D u|^2}}\right) = f & \text{in}\;\Omega,\\
		u=0 & \text{on}\;\partial\Omega,
	\end{cases}
\end{equation}
where $f \in L^1(\Omega)$ and $g$ is a continuous function such that
\begin{equation}\label{condg} \lim_{s\to \pm \infty} g(s) = \pm\infty, \ \ \text{and}\  \ g(s)s\geq 0\  \ s\in \re.\end{equation}
The notion of solution for problem \eqref{pbL1g} is readily re-adapted  from  Definition \ref{weakdefL1} by requiring $g(u)\in L^1(\Omega)$ instead  of $u$.  In particular this gives that $u$ is almost everywhere finite in $\Omega$.

Summarizing, we have the following

\begin{theorem}
	Let $f\in L^1(\Omega)$ then there exists a solution $u$ to problem \eqref{pbL1g}. In particular,   if $f\in L^N(\Omega)$ then $u\in BV(\Omega)\cap L^\infty(\Omega)$. The solution to \eqref{pbL1g} is unique if g is  increasing. 
\end{theorem}
\begin{proof}[Sketch of the proof]
The existence result strictly follows the lines of  Theorems \ref{teomain} and \ref{teoL1}. A key estimate in this case is represented by the analogous to  \eqref{equi} which is necessary to pass to the limit in the term involving $g$.

Indeed, if $u_n$ is a suitable approximating solution,  by the Fatou Lemma one gets that $g(u)\in L^1(\Omega)$.  Hence, one has that  $\forall \varepsilon>0$  there exists $h$ such that $|\{|g(u_n)|\geq h\}|< \varepsilon$. Using the first assumption in \eqref{condg}  there exists a increasing sequence $k_h>0$ such that $\{|u_n|\geq k_h\}\subseteq \{|g(u_n)|\geq h\}$. Therefore,  the equi-integrability of $g(u_n)$ will be a consequence of
$$
\int_{\{|u_n|\ge k_h\}} |g(u_n)| \le \int_{\{|u_n| \ge k_h\}} |f|,
$$
which, as already remarked, can be proven as for \eqref{equi}.

 This would be sufficient to pass to the limit the approximation scheme. 

 \medskip
 
 The $L^\infty$-estimate in case $f\in L^N(\Omega)$ once again, with many simplications,  follows the idea presented in \cite[Theorem 1]{DS}. Indeed, as a sketch of that, one choose $G_k(u)$ to test 
\eqref{pbL1g} obtaining, for $h>0$ to be chosen later 
$$
\begin{array}{l}
	\displaystyle 	\int_\Omega g(u) G_k(u) - \int_\Omega G_k(u)\,\operatorname{div} z = \int_{\{|f|\leq h\}} f G_k(u) +  \int_{\{|f|> h\}} f G_k(u)\\\\
	\displaystyle  \leq  \int_{\{|f|\leq h\}} h G_k(u) +  \int_{\{|f|> h\}} f G_k(u)\,,
\end{array}
$$
and one can proceed as in the proof of Theorem \ref{teobounded}; in fact it sufficies to fix  $\overline{h}$ large enough in order to get 
$$
 \|f\chi_{\{|f|> \overline{h}\}}\|_{L^N(\Omega)} \mathcal{S}_1<1\,.
$$
Thus  we take $\overline{k}$ such that 
$$\inf_{s\in[\overline{k},\infty)}{g(s)}>\overline{h}$$  and we have 
$$
\begin{array}{l}
	\displaystyle 	g(\overline{k})\int_\Omega G_k(u) - \int_\Omega G_k(u)\,\operatorname{div} z  \leq  \int_\Omega \overline{h} G_k(u) +  \int_{\{|f|> \overline{h}\}} f G_k(u)\,,
\end{array}
$$for any $k\geq \overline{k}$, and so 
$$
 - \int_\Omega G_k(u)\,\operatorname{div} z  \leq   \int_{\{|f|> \overline{h}\}} f G_k(u)
 \le \|f\chi_{\{|f|> \overline{h}\}}\|_{L^N(\Omega)}\| G_k(u)\|_{L^\frac{N}{N-1}(\Omega)}, 
$$
which allow to conclude as in the proof of Theorem \ref{teobounded} as from \eqref{bounded3} on.

 
 \medskip
 
 Finally one can reshape Theorem's \ref{teomainunique} proof in order to show that there is at most one solution  if $g$ is a increasing function.
 
\end{proof}



\section*{Acknowledgements}
F. Oliva and F. Petitta are partially supported by the Gruppo Nazionale per l’Analisi Matematica, la Probabilità e le loro Applicazioni (GNAMPA) of the Istituto Nazionale di Alta Matematica (INdAM). 
S. Segura de Le\'on is partially supported by  Grant PID2022-136589NB-I00 founded  by MCIN/AEI/10.13039/5
01100011033 as well as by Grant RED2022-134784-T funded by MCIN/AEI/1
0.13039/501100011033.


\begin{thebibliography}{10}	
\bibitem{afp} L. Ambrosio, N. Fusco and D. Pallara, Functions of Bounded Variation and Free Discontinuity Problems, Oxford Mathematical Monographs, 2000


\bibitem{ABCM} F. Andreu, C. Ballester, V. Caselles and J. M. Maz\'on,
The Dirichlet problem for the total variation flow, J. Funct. Anal.  180 (2), 347-403 (2001)

 \bibitem{ACM_RMI} F. Andreu, V. Caselles, and J.M. Maz\'on, A parabolic quasilinear problem for linear growth functionals, Rev. Mat. Iberoamericana 18 (2002), 135-185

 \bibitem{ACM_MA} F. Andreu, V. Caselles, and J.M. Maz\'on, Existence and uniqueness of solution for a parabolic quasi-
linear problem for linear growth functionals with L1 data, Math. Ann. 322 (2002), 139-206\bk

\bibitem{ACM} F. Andreu, V. Caselles and J. M. Maz\'on, Parabolic quasilinear equations minimizing linear growth functionals, Progress in
Mathematics, 223, Birkh\"auser Verlag, Basel, 2004

\bibitem{An} G. Anzellotti, Pairings between measures and bounded functions and compensated compactness, Ann. Mat. Pura Appl. 135 (4),  293-318 (1983)


\bibitem{b6} P. Benilan, L. Boccardo, T. Gallou\"{e}t, R.Gariepy, M. Pierre and J.L. Vazquez, An $L^1$ theory of existence and uniqueness of nonlinear elliptic equations,  Ann. Scuola Norm. Sup. Pisa 22 (1995) 240-273.
\bibitem{brezis} H. Brezis, Functional Analysis, Sobolev Spaces and Partial	Differential Equations,  Universitext. Springer, New York, 2011


\bibitem{CF} G. Q. Chen and H. Frid, Divergence-measure fields and hyperbolic conservation laws, Arch. Ration. Mech. Anal. 147 (2),  89-118 (1999)

\bibitem{CT} M. Cicalese and C. Trombetti, Asymptotic behaviour of solutions to $p$-Laplacian equation, Asymptotic Analysis
35, 27-40 (2003)

\bibitem{cf1} P. Concus and R. Finn, A singular solution of the capillary equation. I. Existence,	Invent. Math. 29 (2), 143-148 (1975)

\bibitem{cf2} P. Concus and R. Finn, A singular solution of the capillary equation. II. Uniqueness, Invent. Math. 29 (2), 149-159 (1975)


\bibitem{cl} M. G. Crandall and  T. M. Liggett,  Generation of semi-groups of nonlinear transformations on general Banach spaces. Amer. J. Math. 93 (1971), 265-298

\bibitem{DS} A. Dall'Aglio and S. Segura de Le\'on, Bounded solutions to the 1--Laplacian equation with a total variation term, Ric. Mat. 68 (2), 597-614 (2019)

\bibitem{DGOP} V. De Cicco, D. Giachetti, F. Oliva and  F. Petitta, The Dirichlet problem for singular elliptic equations with general nonlinearities, Calc. Var. Partial Differential Equations 58 (4), Paper No. 129 (2019)

\bibitem{dgs}  V. De Cicco, D. Giachetti and S. Segura de Le\'on, Elliptic problems involving the 1-Laplacian and a singular lower order term,  J. Lond. Math. Soc. 99 (2), 349-376 (2019)

\bibitem{finn74} R. Finn, A note on the capillary problem, Acta Mathematica 132, 199-205 (1974)

\bibitem{fg84} R. Finn and E. Giusti, On nonparametric surfaces of constant mean curvature, Ann Sc. Norm Sup. Pisa 17, 389-403 (1984)


\bibitem{gop2} D. Giachetti, F. Oliva and F. Petitta, Bounded solutions for non-parametric mean curvature problems with nonlinear terms, arXiv: 2304.13611

\bibitem{GMP} L. Giacomelli, S. Moll and F.  Petitta,  Nonlinear diffusion in transparent media: the resolvent equation, Adv. Calc. Var. { 11}  (4), 405-432 (2018)

\bibitem{g} M. Giaquinta, On the Dirichlet problem for surfaces of prescribed mean curvature, Manuscripta Math. 12, 73-86 (1974)

\bibitem{g76} E. Giusti, Boundary value problems for non-parametric  surfaces of prescribed mean curvature, Ann. Scuola Norm. Sup. Pisa Cl. Sci. (4) 3 (3), 501-548 (1976)

\bibitem{g78} E. Giusti,  On the Equation of Surfaces of Prescribed Mean Curvature. Existence and uniqueness without
boundary conditions, Invent. math. 46, 111-137 (1978)

 \bibitem{gm} W. G\'orny and J. M. Maz\'on,  A duality-based approach to gradient flows of linear growth functionals,  arXiv preprint arXiv:2212.08725\bk



\bibitem{ww} D.L. Hu, M. Prakash, B. Chan et al., Water-walking devices, Exp Fluids 43, 769-778 (2007)

\bibitem{KS}  B.  Kawohl and F.  Schuricht, Dirichlet problems for the $1$-Laplace operator, including the eigenvalue problem, Commun. Contemp. Math. 9 (4), 515-543 (2007)


\bibitem{lops} M.  Latorre, F. Oliva and F. Petitta, S. Segura de Le\'on,  
The Dirichlet problem for the $1$-Laplacian with a general singular term and $L^1$-data, Nonlinearity 34 1791-1816 (2021)


\bibitem{lc}  G.P. Leonardi and G.E. Comi, The prescribed mean curvature measure equation in non-parametric form, arXiv:2302.10592 (2023)

\bibitem{ll} J. Leray and J. L. Lions, Quelques r\'esulatats de Vi\text{$\check{s}$}ik sur les probl\'emes elliptiques nonlin\'eaires par les m\'ethodes de Minty-Browder, Bull. Soc. Math. France 93, 97-107 (1965)


\bibitem{yy} YY. Liu, J. Xin and Y. Yu, Asymptotics for Turbulent Flame Speeds of the Viscous G-Equation Enhanced by Cellular and Shear Flows, Arch Rational Mech Anal 202, 461-492 (2011)

\bibitem{mst2} A. Mercaldo, S. Segura de Le\'on and C. Trombetti, On the solutions to $1$-Laplacian equation with $L^1$ data,   J. Funct. Anal. 256 (8), 2387-2416 (2009)


\bibitem{PKF}  L. Pick, A. Kufner, J. Oldrich and S. Fuc\'ik,  Function Spaces, 1. Berlin, Boston: De Gruyter, 2012


\bibitem{se} J. Serrin, The problem of Dirichlet for quasilinear elliptic equations with many independent variables, Philos. Trans. Roy. Soc. London Ser. A 264, 413-496 (1969)

\bibitem{st} G. Stampacchia, Le probl\`eme de Dirichlet pour les \'equations elliptiques du seconde ordre  \`a coefficientes discontinus,  Ann. Inst. Fourier (Grenoble) 15, 189-258  (1965)

	

\end{thebibliography}

\end{document}
