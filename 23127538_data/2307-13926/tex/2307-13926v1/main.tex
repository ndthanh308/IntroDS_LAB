\documentclass[dvipsnames,11pt]{article}
%\pdfoutput=1

\usepackage{amsmath,amssymb,amsthm,bm,color,mathrsfs,extarrows,mathtools,enumitem,fancybox,makecell}
\usepackage[square,sort,comma,numbers]{natbib}
\usepackage{subcaption}
\usepackage[margin=1.in]{geometry}
\usepackage{xcolor,bbm}
\usepackage[pagebackref]{hyperref}
\usepackage{bm}
\usepackage{bbm}
\usepackage{comment}
\allowdisplaybreaks
\newif\ifsubmit   

\usepackage[normalem]{ulem}

\setlist[itemize]{itemsep=0pt}
\setlist[enumerate]{itemsep=0pt}
\hypersetup{colorlinks=true,urlcolor=blue,linkcolor=blue,citecolor=[rgb]{.42,.56,.14},}
\def\UrlBreaks{\do\A\do\B\do\C\do\D\do\E\do\F\do\G\do\H\do\I\do\J\do\K\do\L\do\M\do\N\do\O\do\P\do\Q\do\R\do\S\do\T\do\U\do\V\do\W\do\X\do\Y\do\Z\do\[\do\\\do\]\do\^\do\_\do\`\do\a\do\b\do\c\do\d\do\e\do\f\do\g\do\h\do\i\do\j\do\k\do\l\do\m\do\n\do\o\do\p\do\q\do\r\do\s\do\t\do\u\do\v\do\w\do\x\do\y\do\z\do\0\do\1\do\2\do\3\do\4\do\5\do\6\do\7\do\8\do\9\do\.\do\@\do\\\do\/\do\!\do\_\do\|\do\;\do\>\do\]\do\)\do\,\do\?\do\'\do+\do\=\do\#}
\usepackage[mathcal]{eucal}

\usepackage[capitalise,nameinlink]{cleveref}
\Crefname{lemma}{Lemma}{Lemmas}
\Crefname{fact}{Fact}{Facts}
%\newtheorem{remark}[Remark]{Remark}
\Crefname{theorem}{Theorem}{Theorems}
\Crefname{corollary}{Corollary}{Corollaries}
\Crefname{claim}{Claim}{Claims}
\Crefname{example}{Example}{Examples}
\Crefname{problem}{Problem}{Problems}
\Crefname{definition}{Definition}{Definitions}
\Crefname{notation}{Notation}{Notations}
\Crefname{assumption}{Assumption}{Assumptions}
\Crefname{subsection}{Subsection}{Subsections}
\Crefname{section}{Section}{Sections}
\Crefformat{equation}{(#2#1#3)}

\newtheorem{theorem}{Theorem}[section]
\newtheorem*{theorem*}{Theorem}
\newtheorem{itheorem}{Theorem}

\newtheorem{subclaim}{Claim}[theorem]
\newtheorem{proposition}[theorem]{Proposition}
\newtheorem*{proposition*}{Proposition}
\newtheorem{lemma}[theorem]{Lemma}
\newtheorem*{lemma*}{Lemma}
\newtheorem{corollary}[theorem]{Corollary}
\newtheorem*{corollary*}{Corollary}
\newtheorem*{conjecture*}{Conjecture}
\newtheorem{fact}[theorem]{Fact}
\newtheorem*{fact*}{Fact}
\newtheorem{exercise}[theorem]{Exercise}
\newtheorem*{exercise*}{Exercise}
\newtheorem{hypothesis}[theorem]{Hypothesis}
\newtheorem*{hypothesis*}{Hypothesis}
\newtheorem{conjecture}[theorem]{Conjecture}

\theoremstyle{definition}
\newtheorem{definition}[theorem]{Definition}
\newtheorem{notation}[theorem]{Notation}
\newtheorem{construction}[theorem]{Construction}
\newtheorem{example}[theorem]{Example}
\newtheorem{question}[theorem]{Question}
\newtheorem{openquestion}[theorem]{Open Question}
% \newtheorem{algorithm}[theorem]{Algorithm}
\newtheorem{problem}[theorem]{Problem}
\newtheorem{protocol}[theorem]{Protocol}
\newtheorem{assumption}[theorem]{Assumption}
\newtheorem{claim}[theorem]{Claim}
\newtheorem*{claim*}{Claim}

\newtheorem{remark}[theorem]{Remark}
\newtheorem*{remark*}{Remark}
\newtheorem{observation}[theorem]{Observation}
\newtheorem*{observation*}{Observation}
\numberwithin{equation}{section}

\newcommand{\sd}[1]{\mathrm{d}#1}
\DeclareMathOperator*{\E}{\mathbb E}
\DeclareMathOperator*{\Var}{\mathrm{Var}}
\DeclareMathOperator*{\Cov}{\mathrm{Cov}}
\DeclareMathOperator*{\argmax}{\arg\,\max}
\DeclareMathOperator*{\argmin}{\arg\,\min}
\renewcommand{\Pr}{\operatorname*{\mathbf{Pr}}}
\newcommand{\questioneq}{\stackrel{?}{=}}

\newcommand{\eps}{\varepsilon}
\newcommand{\abs}[1]{\left| #1 \right|}
\newcommand{\vabs}[1]{\left\| #1 \right\|}
\newcommand{\abra}[1]{\left\langle #1 \right\rangle}
\newcommand{\sabra}[1]{\langle #1 \rangle}
\newcommand{\pbra}[1]{\left( #1 \right)}
\newcommand{\sbra}[1]{\left[ #1 \right]}
\newcommand{\cbra}[1]{\left\{ #1 \right\}}
\newcommand{\ceilbra}[1]{\left\lceil #1 \right\rceil}
\newcommand{\floorbra}[1]{\left\lfloor #1 \right\rfloor}
\newcommand{\braket}[2]{\left\langle #1 \mid #2\right\rangle}
\renewcommand{\mid}{\,\middle\vert\,}
\newcommand{\bin}{\{0,1\}}
\newcommand{\binpm}{\{\pm1\}}
\newcommand{\binTF}{\{\mathrm{True},\mathrm{False}\}}
\newcommand{\binBT}{\{\bot,\top\}}
\newcommand{\True}{\mathsf{True}}
\newcommand{\False}{\mathsf{False}}
\newcommand{\TVdist}{d_{\mathrm{TV}}}
\newcommand{\dist}{\mathrm{dist}}
\newcommand{\poly}{\mathrm{poly}}
\newcommand{\polylog}{\mathrm{polylog}}
\newcommand{\sgn}{\mathrm{sgn}}
\newcommand{\indicator}{\mathbf{1}}
\newcommand{\KL}{\mathrm{KL}}
\newcommand{\id}{\mathrm{id}}
\newcommand{\unit}{\mathrm{unit}}
\newcommand{\frob}[1]{\vabs{#1}}
\newcommand{\SendReal}{\mathrm{trunc}}
\newcommand{\opnorm}[1]{\vabs{#1}_\mathrm{op}}
\newcommand{\Forr}{\mathrm{Forr}}
\newcommand{\com}{\mu}
\newcommand{\comtwo}{\sigma}

\makeatletter
\newcommand*\bigcdot{\mathpalette\bigcdot@{.5}}
\newcommand*\bigcdot@[2]{\mathbin{{\hbox{\scalebox{#2}{$\m@th#1\bullet$}}}}}
\makeatother

\newcommand{\tensor}{\overset{\bigcdot}{\otimes}}

\newcommand{\Cbb}{\mathbb{C}}
\newcommand{\Fbb}{\mathbb{F}}
\newcommand{\Nbb}{\mathbb{N}}
\newcommand{\Rbb}{\mathbb{R}}
\newcommand{\Sbb}{\mathbb{S}}
\newcommand{\Zbb}{\mathbb{Z}}

\newcommand{\Acal}{\mathcal{A}}
\newcommand{\Bcal}{\mathcal{B}}
\newcommand{\Ccal}{\mathcal{C}}
\newcommand{\Dcal}{\mathcal{D}}
\newcommand{\Ecal}{\mathcal{E}}
\newcommand{\Fcal}{\mathcal{F}}
\newcommand{\Gcal}{\mathcal{G}}
\newcommand{\Hcal}{\mathcal{H}}
\newcommand{\Ical}{\mathcal{I}}
\newcommand{\Lcal}{\mathcal{L}}
\newcommand{\Mcal}{\mathcal{M}}
\newcommand{\Ncal}{\mathcal{N}}
\newcommand{\Ocal}{\mathcal{O}}
\newcommand{\Pcal}{\mathcal{P}}
\newcommand{\Qcal}{\mathcal{Q}}
\newcommand{\Rcal}{\mathcal{R}}
\newcommand{\Scal}{\mathcal{S}}
\newcommand{\Tcal}{\mathcal{T}}
\newcommand{\Ucal}{\unif}
\newcommand{\Vcal}{\mathcal{V}}
\newcommand{\Xcal}{\mathcal{X}}
\newcommand{\Zcal}{\mathcal{Z}}

\newcommand{\Fsf}{\mathsf{F}}
\newcommand{\Psf}{\mathsf{P}}
\newcommand{\Tsf}{\mathsf{T}}
\newcommand{\Wsf}{\mathsf{W}}

\newcommand{\rbm}{\bm{r}}
\newcommand{\xbm}{\bm{x}}
\newcommand{\ybm}{\bm{y}}
\newcommand{\zbm}{\bm{z}}

\newcommand{\Sbm}{\bm{S}}
\newcommand{\Tbm}{\bm{T}}
\newcommand{\Vbm}{\bm{V}}
\newcommand{\Ubm}{\bm{U}}

\usepackage{braket}

%\newcommand{\kewen}[1]{\authnote{Kewen}{#1}{red}}
%\newcommand{\mnote}[1]{\authnote{Makrand}{#1}{teal}}
%\newcommand{\uma}[1]{\authnote{Uma}{#1}{purple}}
%\newcommand{\avishay}[1]{\authnote{Avishay}{#1}{orange}}

\renewcommand{\tilde}{\widetilde}
\renewcommand{\bar}{\overline}
\renewcommand{\hat}{\widehat}

\newcommand{\supp}{\mathrm{supp}}
\newcommand{\rel}{\mathrm{rel}}
\newcommand{\dkl}{d_{\mathrm{KL}}}

\newcommand{\N}{\mathbb{N}}
\newcommand{\X}{\bm{X}}
\newcommand{\Y}{\bm{Y}}
\newcommand{\Z}{\bm{Z}}
\newcommand{\A}{\bm{A}}
\newcommand{\B}{\bm{B}}
\newcommand{\U}{\bm{u}}
\newcommand{\V}{\bm{v}}
%\newcommand{\u}{\bm{u}}
%\newcommand{\v}{\bm{v}}
\newcommand{\bmR}{\bm{R}}
\newcommand{\D}{\bm{d}}
\newcommand{\balpha}{\bm{\beta}}
\newcommand{\bmL}{\bm{L}}
\newcommand{\Q}{\bm{Q}}
\newcommand{\bmP}{\bm{P}}
\newcommand{\lX}{\bm{x}}
\newcommand{\lY}{\bm{y}}
\newcommand{\lZ}{\bm{z}}
\newcommand{\lA}{\bm{a}}
\newcommand{\lB}{\bm{b}}
\newcommand{\bell}{\bm{\ell}}
\newcommand{\la}{\lA}
\newcommand{\lb}{\lB}
\newcommand{\lc}{\bm{c}}
\newcommand{\lx}{\lX}
\newcommand{\ly}{\lY}
\newcommand{\lz}{\lZ}
\newcommand{\lQ}{\bm{q}}
%\newcommand{\lq}{\bm{q}}
\newcommand{\lP}{\bm{p}}
\newcommand{\lp}{\bm{p}}
\newcommand{\btau}{\bm{\tau}}
%\newcommand{\d}{\D}

\newcommand{\midd}{~\vert~}

\newcommand{\pmones}{\{\pm1\}^n}
\newcommand{\pmone}{\{\pm1\}}
\newcommand{\unif}{\nu}
\newcommand{\BE}{\E}
\newcommand{\supA}[1]{\A^{(#1)}}
\newcommand{\supB}[1]{\B^{(#1)}}
\newcommand{\supu}[1]{\bm{u}^{(#1)}}
\newcommand{\supU}[1]{\bm{U}^{(#1)}}
\newcommand{\supv}[1]{\bm{v}^{(#1)}}
\newcommand{\supV}[1]{\bm{V}^{(#1)}}
\newcommand{\supa}[1]{\lA^{(#1)}}
\newcommand{\supb}[1]{\lB^{(#1)}}
\newcommand{\supcbar}[1]{\bar\lc^{(#1)}}
\newcommand{\supX}[1]{\X^{(#1)}}
\newcommand{\supY}[1]{\Y^{(#1)}}
\newcommand{\supZ}[1]{\lZ^{(#1)}}
\newcommand{\supR}[1]{\bmR^{(#1)}}
\newcommand{\supF}[1]{\Fcal^{(#1)}}
\newcommand{\F}{\mathcal{F}}
\newcommand{\ip}[2]{\abra{#1, #2}}
\newcommand{\R}{\mathbb{R}}
\renewcommand{\Lambda}{\eta}
\newcommand{\ind}{\mathbf{1}}

\newcommand{\BH}{\bm{H}}
\newcommand{\bPi}{\bm{\Pi}}
\newcommand{\bP}{\bm{p}}
\renewcommand{\Q}{\bm{q}}
\newcommand{\bQ}{\bm{q}}
\newcommand{\K}{\bm{k}}
\newcommand{\lw}{\bm{w}}
\newcommand{\bc}{\bm{c}}
\newcommand{\br}{\bm{r}}
\newcommand{\bs}{\bm{s}}
\newcommand{\Xell}{\X_{\bell}}
\newcommand{\Yell}{\Y_{\bell}}

\newcommand{\biased}[1]{\pi^{\otimes n}_{#1}}

\usepackage{xspace}
\newcommand{\complexityclass}[1]{{\bf{#1}}\xspace}
\renewcommand{\P}{\complexityclass{P}}
\newcommand{\NP}{\complexityclass{NP}}
\newcommand{\Ppoly}{\complexityclass{P/\mathsf{poly}}}
\newcommand{\CONP}{\complexityclass{coNP}}
\newcommand{\BPP}{\complexityclass{BPP}}
\newcommand{\BQP}{\complexityclass{BQP}}
\newcommand{\PSPACE}{\complexityclass{PSPACE}}
\newcommand{\flift}{f\circ{g}}

\newcommand{\Goos}{G\"{o}\"{o}s\xspace}

\title{Fourier Growth of Communication Protocols for XOR Functions}
%Old title: XOR-Lifting and Fourier Growth of Communication Protocols
\author{
Uma Girish\thanks{Princeton University. Email: \texttt{ugirish@cs.princeton.edu}}
\and
Makrand Sinha\thanks{Simons Institute and University of California at Berkeley. Email: \texttt{makrand@berkeley.edu}}
\and
Avishay Tal\thanks{University of California at Berkeley. Email: \texttt{atal@berkeley.edu}}
\and
Kewen Wu\thanks{University of California at Berkeley. Email: \texttt{shlw\_kevin@hotmail.com}}
}
\date{}

\begin{document}
\maketitle

\begin{abstract}
The level-$k$ $\ell_1$-Fourier weight of a Boolean function refers to the sum of absolute values of its level-$k$ Fourier coefficients. Fourier growth refers to the growth of these weights as $k$ grows.
It has been extensively studied for various computational models, and bounds on the Fourier growth, even for the first few levels, have proven useful in learning theory, circuit lower bounds, pseudorandomness, and quantum-classical separations.

In this work, we investigate the Fourier growth of certain functions that naturally arise from communication protocols for XOR functions (partial functions evaluated on the bitwise XOR of the inputs $x$ and $y$ to Alice and Bob). If a protocol $\mathcal{C}$ computes an XOR function, then $\mathcal{C}(x, y)$ is a function of the parity $x \oplus y$. This motivates us to analyze the \textit{XOR-fiber} of the communication protocol $\mathcal{C}$, defined as $h(z) := \mathbb{E}_{\bm{x},\bm{y}}[\mathcal{C}(\bm{x}, \bm{y}) | \bm{x}\oplus \bm{y} = z]$.

We present improved Fourier growth bounds for the XOR-fibers of randomized protocols that communicate $d$ bits. For the first level, we show a tight $O(\sqrt{d})$ bound and obtain a new coin theorem, as well as an alternative proof for the tight randomized communication lower bound for the Gap-Hamming problem. For the second level, we show an $d^{3/2} \cdot \polylog(n)$ bound, which improves the previous $O(d^2)$ bound by Girish, Raz, and Tal (ITCS 2021) and implies a polynomial improvement on the randomized communication lower bound for the XOR-lift of the Forrelation problem, which extends the quantum-classical gap for this problem.

Our analysis is based on a new way of adaptively partitioning a relatively large set in Gaussian space to control its moments in all directions. We achieve this via martingale arguments and allowing protocols to transmit real values. We also show a connection between Fourier growth and lifting theorems with constant-sized gadgets as a potential approach to prove optimal bounds for the second level and beyond.
\end{abstract}


% STOC submission Abstract:
%Motivated by new separations between quantum and randomized communication complexity, we study the Fourier growth of Boolean functions associated with communication protocols. Suppose Alice and Bob want to compute some (partial) function $f$ on the bit-wise XOR of their inputs by exchanging as few bits as possible. Let $\mathcal{C}(x,y)$ be the value that their communication protocol yields on inputs $x$ and $y$, respectively. If $\mathcal{C}$ computes $f$, then $\mathcal{C}(x, y) = f(x\oplus y)$ on every $x, y$ for which $f(x \oplus y)$ is defined. This gives a natural motivation to analyze the \textit{XOR-fiber} of the communication protocol $\mathcal{C}$, defined as $h(z) := \mathbb{E}_{\bm{x},\bm{y}}[\mathcal{C}(\bm{x}, \bm{y}) | \bm{x}\oplus \bm{y} = z]$.

%Recent work by Girish, Raz, and Tal (ITCS 2021) showed that when the function $f$ is the Forrelation problem:
%\begin{itemize}
%\item A simultaneous-message quantum protocol exists with communication cost $\mathrm{polylog}(n)$ where each player implements an efficient quantum circuit of size $\mathrm{polylog}(n)$.
%\item Any randomized protocol requires communication complexity $\tilde{\Omega}(n^{1/4})$.
%\end{itemize}
%The randomized communication complexity lower bound was achieved via Fourier analysis of the XOR-fibers of randomized protocols with communication cost at most $d$. However, the analysis was not known to be tight, with a potential quadratic improvement to the lower bound  (that would match a simple $\tilde{O}(\sqrt{n})$ upper bound).

%In this work, we obtain improved $\ell_1$-Fourier growth bounds for the XOR-fibers of randomized protocols of cost $d$:
%\begin{enumerate}
%\item For the first Fourier level, we show a tight $O(\sqrt{d})$ bound. This implies a new coin theorem, and an alternative proof for the celebrated $\Omega(n)$ randomized communication lower bound on the Gap-Hamming problem by Chakrabarti and Regev (SICOMP 2012).
%\item For the second Fourier level, we show a $d^{3/2} \cdot \mathrm{polylog}(n)$ bound, improving the previous $O(d^2)$ bound. This implies that the randomized communication complexity for the Forrelation problem is $\tilde{\Omega}(n^{1/3})$.
%\end{enumerate}
%We believe our second-level bound can be further improved (to $d \cdot \mathrm{polylog}(n)$) and highlight some directions towards such an improvement --- one of which is via a connection to lifting with constant-sized gadgets.

%%% Version one %%%%
% Fourier growth of a function captures how the sum of absolute values of its level-$k$ Fourier coefficients scales with $k$ and has been well-studied for various computational models. $\ell_1$-Fourier weight bounds even for low Fourier levels have seen a variety of applications in learning theory, circuit lower bounds, pseudorandomness and quantum-classical separations.\\
% %Let $\mathcal{C}(x,y)$ be the value that their communication protocol yields on inputs $x$ and $y$, respectively. 

% In this work, we study the Fourier growth of certain functions that naturally arise from communication protocols for XOR functions --- a (partial) function $f$ evaluated on the bit-wise XOR of the inputs to Alice and Bob. If a protocol $\mathcal{C}$ computes an XOR function $f$, then $\mathcal{C}(x, y) = f(x\oplus y)$ on every $x, y$ for which $f(x \oplus y)$ is defined. This gives a natural motivation to analyze the \textit{XOR-fiber} of the communication protocol $\mathcal{C}$, defined as
%  $h(z) := \mathbb{E}_{\bm{x},\bm{y}}[\mathcal{C}(\bm{x}, \bm{y}) | \bm{x}\oplus \bm{y} = z]$.\\

% We obtain improved $\ell_1$-Fourier growth bounds for the XOR-fibers of randomized
% protocols that communicate $d$ bits:
% \begin{itemize}
%     \item For the first Fourier level, we show a tight $O(\sqrt{d})$ bound. As level-one Fourier bounds related to correlation with the majority function, we obtain new coin theorem, and a new alternative proof for the tight randomized lower bound for the well-studied Gap-Hamming problem as a byproduct.
% \item For the second Fourier level, we show a $d^{3/2} \cdot \mathrm{polylog}(n)$ bound, improving the previous $O(d^2)$ bound (Girish,Raz,Tal ITCS 2021). As a byproduct, we obtain that the randomized communication complexity for the Forrelation problem is $\Omega(n^1/3)$, polynomially improving the previous bound.
% \end{itemize}



% Our analysis is based on a new adaptive process to partition a relatively-large set in Gaussian space so that its moments are well-behaved in all directions and we achieve this via martingale arguments and communication protocols that communicate real values. We believe our level-$2$ bound can be further improved and highlight some directions towards such an improvement --- one of which is via a connection to lifting with constant-sized gadgets.\\


%%%%% version one end 

%%%% version zero start

% The \textit{level-$k$ $\ell_1$-Fourier weight} of a (partial) boolean function is the sum of absolute values of its level-$k$ Fourier coefficients. 
% Small Fourier growth implies efficient learning algorithms, simple pseudorandom generators, and separations between classical and quantum computational models.
% \kewen{make a point that low levels are important already}

% \mnote{Make a point that XOR functions are well-studied and connect it to Avishay's motivation for XOR fiber.}
% In this work, we prove Fourier growth bounds for XOR functions in communication complexity.

% Let $\mathcal{C}$ be a classical two-party communication protocol.
% We say $\mathcal{C}$ is the \textit{XOR function} of $f$ if $\mathcal{C}(x,y)=f(x\oplus y)$ on every $x,y$ for which $f(x\oplus y)$ is defined.
% For $k\in\{1,2\}$, we bound the level-$k$ Fourier growth of $f$ when $\mathcal C$ has low cost.
% In fact, our results hold more generally for an arbitrary protocol $\mathcal{C}$ and its \textit{XOR-fiber} $f(z):=\mathbb{E}_{\bm{x},\bm{y}}[\mathcal{C}(\bm x,\bm y)|\bm x\oplus\bm y=z]$.

% As applications, our level-$1$ bound implies a new coin theorem and an alternative proof for the randomized communication lower bound on the Gap-Hamming problem.
% Our level-$2$ bound implies that the randomized communication complexity for the Forrelation problem is $\tilde{\Omega}(n^{1/3})$, improving the previous $\tilde{\Omega}(n^{1/4})$ bound (Girish-Raz-Tal ITCS'21).

% Our analysis relies on  martingale arguments on the communication protocol tree and geometric analysis in Gaussian spaces.

% We believe our level-$2$ bound can be further improved and highlight some directions towards such an improvement --- one of which is via a connection to lifting with constant-sized gadgets.

%%% Old version end


% \mnote{Some ideas for the introduction:
% \begin{itemize}
%     \item Start with Fourier growth and its applications such as pseudorandomness and separations.
%     \item Discuss how to bound Fourier growth and the importance of exploiting cancellations, using decision trees and parity decision trees as examples. Move PDT paragraph on pg 5 here.
%     \item Introduce XOR-fibers of communication protocols, highlighting their unique challenges compared to the PDT case.
%     \item Emphasize the main contribution of the Fourier growth results. Move applications to Gap Hamming, Lifted Forrelation, and k-fold Forrelation, and Lifting with constant-sized gadgets a separate section.
%     \item Change pairwise independence with isotropy and geometric arguments in the proof overview and ensure consistency later in the proofs.
% \end{itemize}
% }

\newpage 
\tableofcontents
\thispagestyle{empty}
\newpage
\setcounter{page}{1}

%% Figure environment removed

\section{Introduction}
Automatic 3D reconstruction of clothed humans using image inputs has gained increasing significance due to its potential applications in a wide array of AR/VR scenarios. High-fidelity reconstructions typically depend on sophisticated capture systems, which are developed with dense camera arrays~\cite{collet2015high,joo2015panoptic,joo2018total}, programmable light-stages~\cite{Vlasic2009, guo2019relightables}, and depth sensors~\cite{newcombe2011kinectfusion,DoubleFusion,BodyFusion,dou2016fusion4d,newcombe2015dynamicfusion}. However, stringent capture environments equipped with complex hardware pose significant challenges for consumer-level applications.


In this context, considerable research effort has been dedicated to developing methods that allow for more flexible capture configurations, such as utilizing a few RGB inputs. Among these works, learning implicit functions \cite{iccv2020PIFu, saito2020pifuhd, hong2021stereopifu} has proven effective in achieving highly detailed reconstructions by integrating the advancements of deep neural networks. These methods employ large multi-layer perceptrons (MLPs) to predict the occupancy probability or truncated signed distance function (TSDF) value of every queried 3D point based on its associated local feature, which is extracted from images. They can recover a continuous surface at arbitrary resolutions without topology restrictions.


However, in typical MLP-based implicit networks, the occupancy or TSDF value at each location is solved independently with planar image features, rendering them less capable of addressing challenging cases such as occlusions. Consequently, these methods suffer from generalization and robustness issues, particularly when tackling strong occlusions caused by large motion or multiple interacting humans. 
Some follow-up studies  \cite{zheng2021deepmulticap,zheng2021pamir,huang2020arch} utilize an extra geometric model, SMPL~\cite{Loper2015}, to improve robustness by introducing strong shape priors. 
Their success typically relies on the assumption of geometrical similarity \cite{huang2020arch} between the shape prior and target reconstruction, making them intractable for handling complex cases with loose clothes and sensitive to errors in SMPL model fitting.



%\ping{this paragraph sounds like `TSDF is better than MLP/SMPL, and we use TSDF to solve the problem'. But in Sec 3, we are telling a different story, saying `MLP needs a 3D convolutional encoder'. We need to make these two sections consistent.}\sicong{I think in this paragraph we claim that the TSDF}


%We opt for Trucated Signed Distance Funtion (TSDF) volumetric representations as they are naturally suitable for convolution operations, which have shown remarkable performance for learning hierarchical features on 2D visual perception tasks \cite{SunXLW19}. 
%Meanwhile, TSDF also describes the gradual geometry change around shape surface, which is not reflected by occupancy volume. 

We instead revisit the 3D volumetric representation and resort to 3D convolutional neural networks (CNNs) for feature learning, due to their impressive performance in feature learning and the ability to incorporate spatial context. However, volumetric methods and 3D convolution involve discretization, which might raise concerns regarding whether a discretized volume can preserve subtle geometric details as continuous representations learned in implicit functions. We investigate the relationship between volume resolution and quantization error on synthetic data by converting target mesh objects to TSDF volumes, as shown in Figure~\ref{fig:quantization_error}. We observe that the quantization errors are significantly reduced by increasing volume resolution and become nearly negligible when reaching a relatively high resolution (e.g., 512 or higher). In other words, achieving fine-detailed reconstruction is not supposed to be restricted by the use of volume representations as long as a proper volume resolution is utilized. Therefore, we present a method with high-resolution feature volumes, e.g., 256 and 512, while traditional volumetric methods \cite{varol18_bodynet,gilbert2018volumetric} are often limited to much lower resolutions, such as 32 or 128.



On the other hand, an increase in volume resolution may lead to a cubic growth of memory overhead \cite{8100085}. Reducing memory costs while guaranteeing the granularity of volumetric representations is necessary for pursuing high-quality reconstruction. Thus, we adopt a coarse-to-fine approach and cull away irrelevant voxels to build a sparse high-resolution feature volume. At the coarse level, the network computes an initial TSDF by applying a U-Net with sparse 3D CNN \cite{3DSemanticSegmentationWithSubmanifoldSparseConvNet} on the sparse feature volume, which is carved by a visual hull. Through our experiments, it turns out that more than 95\% of the volume grids are discarded by the visual hull culling, making the sparse 3D CNN efficient. At the fine level, the network focuses on a narrow band near the zero-level set of the initial TSDF and discretizes the narrow band with smaller voxels. By employing this narrow-band culling, we further shrink the sampling space, resulting in a relatively small range of grid numbers (usually 300K--500K in our experiments) even with a high volume resolution of 512. The remaining voxels in the narrow band are associated with features that fuse high-frequency information from the computed normal maps upon the low-frequency shape from the coarse level to compute the TSDF at high resolution. The final mesh is then extracted from the TSDF using the Marching-Cube algorithm ~\cite{Lorensen87marchingcubes}.
% Different from the u-net sturcture to preserve global topology context, we then apply a shallow 3dcnn to compute the final TSDF $D_{final}$ which contain more local geometry detail.




% \ping{this paragraph can be expanded. It is an important contribution and often ignored by other works. stress on the novel idea of regressing blending weights instead of colors}

In addition to geometry, high-quality mesh texture is also a crucial factor contributing to visual appearance. Directly computing a color field in 3D space, as in \cite{iccv2020PIFu}, struggles to capture high-frequency texture details, while the neural radiance field (NeRF) \cite{yu2020pixelnerf} or the DoubleField~\cite{shao2022doublefield} require expensive per-instance optimization and are often unstable for sparse input images. In contrast, we adopt an image-based rendering approach to compute a texture atlas map, which is efficient and widely supported in existing computer graphics tools. 
Specifically, we compute a blending weight at each 3D point on the mesh surface to determine its color as a weighted average of the colors at its image projections. The blending weights can be computed at a relatively coarse resolution, e.g., 512 volume resolution in our case, and leave texture details to the high-resolution images, such as 1K or 2K. Unlike previous methods that generate blurry texturing results under sparse input, our method generalizes well on both synthetic and real data with just a few input views. 
Figure~\ref{fig:teaser} shows two examples reconstructed by our method. Despite the challenging garment, pose, and occlusion, our method recovers faithful shape, normal, and texture on the right.

%with a wide variety of poses and clothing styles, and it is also adaptive to handle input image with arbitrary resolutions.
%\sicong{For this concern we claim that when the resolution of dicretized volume meets certain threshold (which is 256 in our experiment), the quantization error can be neglected.} 



In summary, the main contributions of this paper are as follows:
\begin{itemize}
\vspace{-0.1in}
  \item 
  We revisit the 3D volumetric representation and demonstrate that it can support clothed human reconstruction with equal or even better performance compared to implicit representation. 
  \item 
  We develop a memory and computation-efficient method for high-resolution volumetric reconstruction using sophisticated sparse 3D CNN, coarse-to-fine estimation, and voxel culling by visual hull and narrow bands. 
  \item 
  We introduce a novel method to compute a texture atlas map, which captures rich appearance details from high-resolution input images.
  \item 
  We achieve impressive results on standard benchmark datasets Twindom and MultiHuman, significantly reducing the point-2-surface (P2S) precision to approximately 0.2cm from just six input views, with more than $50\%$ error reduction compared to the state-of-the-art methods, including DoubleField~\cite{shao2022doublefield} and PIFuHD~\cite{saito2020pifuhd}.
\end{itemize}
% Figure environment removed

\section{Introduction}
Automatic 3D reconstruction of clothed humans using image inputs has gained increasing significance due to its potential applications in a wide array of AR/VR scenarios. High-fidelity reconstructions typically depend on sophisticated capture systems, which are developed with dense camera arrays~\cite{collet2015high,joo2015panoptic,joo2018total}, programmable light-stages~\cite{Vlasic2009, guo2019relightables}, and depth sensors~\cite{newcombe2011kinectfusion,DoubleFusion,BodyFusion,dou2016fusion4d,newcombe2015dynamicfusion}. However, stringent capture environments equipped with complex hardware pose significant challenges for consumer-level applications.


In this context, considerable research effort has been dedicated to developing methods that allow for more flexible capture configurations, such as utilizing a few RGB inputs. Among these works, learning implicit functions \cite{iccv2020PIFu, saito2020pifuhd, hong2021stereopifu} has proven effective in achieving highly detailed reconstructions by integrating the advancements of deep neural networks. These methods employ large multi-layer perceptrons (MLPs) to predict the occupancy probability or truncated signed distance function (TSDF) value of every queried 3D point based on its associated local feature, which is extracted from images. They can recover a continuous surface at arbitrary resolutions without topology restrictions.


However, in typical MLP-based implicit networks, the occupancy or TSDF value at each location is solved independently with planar image features, rendering them less capable of addressing challenging cases such as occlusions. Consequently, these methods suffer from generalization and robustness issues, particularly when tackling strong occlusions caused by large motion or multiple interacting humans. 
Some follow-up studies  \cite{zheng2021deepmulticap,zheng2021pamir,huang2020arch} utilize an extra geometric model, SMPL~\cite{Loper2015}, to improve robustness by introducing strong shape priors. 
Their success typically relies on the assumption of geometrical similarity \cite{huang2020arch} between the shape prior and target reconstruction, making them intractable for handling complex cases with loose clothes and sensitive to errors in SMPL model fitting.



%\ping{this paragraph sounds like `TSDF is better than MLP/SMPL, and we use TSDF to solve the problem'. But in Sec 3, we are telling a different story, saying `MLP needs a 3D convolutional encoder'. We need to make these two sections consistent.}\sicong{I think in this paragraph we claim that the TSDF}


%We opt for Trucated Signed Distance Funtion (TSDF) volumetric representations as they are naturally suitable for convolution operations, which have shown remarkable performance for learning hierarchical features on 2D visual perception tasks \cite{SunXLW19}. 
%Meanwhile, TSDF also describes the gradual geometry change around shape surface, which is not reflected by occupancy volume. 

We instead revisit the 3D volumetric representation and resort to 3D convolutional neural networks (CNNs) for feature learning, due to their impressive performance in feature learning and the ability to incorporate spatial context. However, volumetric methods and 3D convolution involve discretization, which might raise concerns regarding whether a discretized volume can preserve subtle geometric details as continuous representations learned in implicit functions. We investigate the relationship between volume resolution and quantization error on synthetic data by converting target mesh objects to TSDF volumes, as shown in Figure~\ref{fig:quantization_error}. We observe that the quantization errors are significantly reduced by increasing volume resolution and become nearly negligible when reaching a relatively high resolution (e.g., 512 or higher). In other words, achieving fine-detailed reconstruction is not supposed to be restricted by the use of volume representations as long as a proper volume resolution is utilized. Therefore, we present a method with high-resolution feature volumes, e.g., 256 and 512, while traditional volumetric methods \cite{varol18_bodynet,gilbert2018volumetric} are often limited to much lower resolutions, such as 32 or 128.



On the other hand, an increase in volume resolution may lead to a cubic growth of memory overhead \cite{8100085}. Reducing memory costs while guaranteeing the granularity of volumetric representations is necessary for pursuing high-quality reconstruction. Thus, we adopt a coarse-to-fine approach and cull away irrelevant voxels to build a sparse high-resolution feature volume. At the coarse level, the network computes an initial TSDF by applying a U-Net with sparse 3D CNN \cite{3DSemanticSegmentationWithSubmanifoldSparseConvNet} on the sparse feature volume, which is carved by a visual hull. Through our experiments, it turns out that more than 95\% of the volume grids are discarded by the visual hull culling, making the sparse 3D CNN efficient. At the fine level, the network focuses on a narrow band near the zero-level set of the initial TSDF and discretizes the narrow band with smaller voxels. By employing this narrow-band culling, we further shrink the sampling space, resulting in a relatively small range of grid numbers (usually 300K--500K in our experiments) even with a high volume resolution of 512. The remaining voxels in the narrow band are associated with features that fuse high-frequency information from the computed normal maps upon the low-frequency shape from the coarse level to compute the TSDF at high resolution. The final mesh is then extracted from the TSDF using the Marching-Cube algorithm ~\cite{Lorensen87marchingcubes}.
% Different from the u-net sturcture to preserve global topology context, we then apply a shallow 3dcnn to compute the final TSDF $D_{final}$ which contain more local geometry detail.




% \ping{this paragraph can be expanded. It is an important contribution and often ignored by other works. stress on the novel idea of regressing blending weights instead of colors}

In addition to geometry, high-quality mesh texture is also a crucial factor contributing to visual appearance. Directly computing a color field in 3D space, as in \cite{iccv2020PIFu}, struggles to capture high-frequency texture details, while the neural radiance field (NeRF) \cite{yu2020pixelnerf} or the DoubleField~\cite{shao2022doublefield} require expensive per-instance optimization and are often unstable for sparse input images. In contrast, we adopt an image-based rendering approach to compute a texture atlas map, which is efficient and widely supported in existing computer graphics tools. 
Specifically, we compute a blending weight at each 3D point on the mesh surface to determine its color as a weighted average of the colors at its image projections. The blending weights can be computed at a relatively coarse resolution, e.g., 512 volume resolution in our case, and leave texture details to the high-resolution images, such as 1K or 2K. Unlike previous methods that generate blurry texturing results under sparse input, our method generalizes well on both synthetic and real data with just a few input views. 
Figure~\ref{fig:teaser} shows two examples reconstructed by our method. Despite the challenging garment, pose, and occlusion, our method recovers faithful shape, normal, and texture on the right.

%with a wide variety of poses and clothing styles, and it is also adaptive to handle input image with arbitrary resolutions.
%\sicong{For this concern we claim that when the resolution of dicretized volume meets certain threshold (which is 256 in our experiment), the quantization error can be neglected.} 



In summary, the main contributions of this paper are as follows:
\begin{itemize}
\vspace{-0.1in}
  \item 
  We revisit the 3D volumetric representation and demonstrate that it can support clothed human reconstruction with equal or even better performance compared to implicit representation. 
  \item 
  We develop a memory and computation-efficient method for high-resolution volumetric reconstruction using sophisticated sparse 3D CNN, coarse-to-fine estimation, and voxel culling by visual hull and narrow bands. 
  \item 
  We introduce a novel method to compute a texture atlas map, which captures rich appearance details from high-resolution input images.
  \item 
  We achieve impressive results on standard benchmark datasets Twindom and MultiHuman, significantly reducing the point-2-surface (P2S) precision to approximately 0.2cm from just six input views, with more than $50\%$ error reduction compared to the state-of-the-art methods, including DoubleField~\cite{shao2022doublefield} and PIFuHD~\cite{saito2020pifuhd}.
\end{itemize}
\section{Secure Design of \puma}\label{sec:design}
In this section, we first present an overview of \puma, and present the protocols for secure $\gelu$ , $\softmax$, embedding, and $\layernorm$ used by \puma. Note that the linear layers such as matrix multiplication are straightforward in replicated secret sharing, so we mainly describe our protocols for non-linear layers in this manuscript.

\subsection{Overview of \puma}\label{sec:overview}
To achieve secure inference of Transformer models, \puma\ defines three kinds of roles: one model owner, one client, and three computing parties. The model owner and the client  provide their models or inputs to the computing parties (i.e., $P_0$, $P_1$, and $P_2$) in a secret-shared form, then the computing parties execute the MPC protocols and send the results back to the client. Note that the model owner and client can also act as one of the computing party, we describe them separately for generality. \eg, when the model owner acts as $P_0$, the client acts as  $P_1$, a third-party dealer acts as $P_2$, the system model becomes the same with \mpcformer~\citep{li2023mpcformer}.

During the secure inference process, a key invariant is maintained: For any layer, the computing parties always start with 2-out-of-3 replicated secret shares of the previous layer's output and the model weights, and end with 2-out-of-3 replicated secret shares of this layer's output. As the shares do not leak any information to each party, this ensures that the layers can be sequentially combined for arbitrary depths to obtain a secure computation scheme for any Transformer-based model.
%The main focus of \puma\ is to reduce the computation and communication costs between the computing parties while maintaining the desired level of security. 



\iffalse
\textbf{Threat Model.}
Following previous works~\citep{aby3,li2023mpcformer},
\puma\ resists a semi-honest (a.k.a., honest-but-curious) adversary in honest-majority~\citep{lindell2009proof}, where the adversary passively corrupts no more than one computing party. Such an adversary follows the protocol specification exactly, but may try to learn more information than permitted. Please note that \puma\ cannot protect against the extraction of information from the inference results, and the examination of mitigating solutions (\eg, differential privacy~\citep{abadi2016deep}) falls outside the scope of this study.
\fi 

\subsection{Protocol for Secure GeLU}\label{sec:gelu}
Most of the current approaches view the $\gelu$ function as a composition of smaller functions and try to optimize each piece of them, making them to miss the
chance of optimizing the private $\gelu$ as a whole. Given the $\gelu$ function:
\begin{equation}\label{eq:gelu}
\begin{split}
    \gelu(x) &= \frac{x}{2} \cdot \left(1 + \tanh \left( \sqrt{\frac{2}{\pi}} \cdot \left(x + 0.044715 \cdot x^3 \right) \right) \right)\\
    &\approx x\cdot \mathsf{sigmoid}(0.071355\cdot x^3 + 1.595769\cdot x) 
\end{split},
\end{equation}
these approaches~\citep{hao2022iron,characmpctranformer} focus either on designing efficient protocols for function $\tanh$
or using the existing MPC protocols of exponentiation and reciprocal for $\mathsf{sigmoid}$. 

However, none of current approaches have utilized the fact that $\gelu$ function is almost linear on the two sides (\ie, $\gelu(x)\approx 0$ for $x<-4$ and $\gelu(x)\approx x$ for $x>3$). 
Within the short interval $[-4,3]$ of $\gelu$,
we suggest a piece-wise approximation of low-degree polynomials is a more efficient and easy-to-implement choice for its secure protocol. Concretely, our piece-wise low-degree polynomials are shown as equation~(\ref{eq:geluapprox}):
\begin{equation}\label{eq:geluapprox}
\gelu(x)=
\begin{cases}
0, & x<-4 \\
F_0(x), & -4 \le x < -1.95 \\
F_1(x), & -1.95 \le x \le 3 \\
x, & x >3
\end{cases},
\end{equation}
where polynomials $F_0()$ and $F_1()$ are computed by library $\mathsf{numpy.ployfit}$\footnote{\url{https://numpy.org/doc/stable/reference/generated/numpy.polyfit.html}} as equation~(\ref{eq:f0f1}). Surprsingly, the above simple poly fit works very well and our $\mathsf{max\ error}< 0.01403$, $\mathsf{median\ error}< 4.41e-05$, and $\mathsf{mean\ error}< 0.00168$.
\begin{equation}\label{eq:f0f1}
\begin{cases}
F_0(x) &= -0.011034134030615728 x^3 -0.11807612951181953 x^2 \\
&- 0.42226581151983866 x -0.5054031199708174\\
F_1(x) &= 0.0018067462606141187x^6 -0.037688200365904236 x^4 \\
&+ 0.3603292692789629x^2 + 0.5x + 0.008526321541038084
\end{cases}
\end{equation}

Formally, given secret input $\share{x}$, our secure $\gelu$ protocol $\Pi_{\gelu}$ is constructed as algorithm~\ref{protocol:gelu}. 
\iffalse
\begin{itemize}
    \item The parties jointly compute
$\share{b_0}^2 = \Pi_{\mathsf{LT}}(\share{x}, 4)$,
$\share{b_1}^2 = \Pi_{\mathsf{LT}}(\share{x}, -1.95)$, and
$\share{b_2}^2 = \Pi_{\mathsf{LT}}(3, \share{x})$.

\item  Then, each $P_i$ locally compute
$\share{b_3}^2 = \share{b_1}^2 \oplus \share{b_2}^ \oplus 1$ and
$\share{b_4}^2 = \share{b_0}^2 \oplus \share{b_1}^2$

\item Finally, the parties compute and return 
$\share{b_2}^2 \cdot \share{x} + \share{b_4}^2 \cdot F_0(\share{x}) + \share{b_3}^2 \cdot F_1(\share{x})$, where polynomials $(F_0, F_1)$ can be computed easily using secure addition and multiplication (and its variants, \eg, secure square)~\citep{spu}. 
\end{itemize}
\fi 

\begin{algorithm}[tp]
\caption{Secure $\gelu$ Protocol $\Pi_{\mathsf{GeLU}}$}\label{protocol:gelu}
\begin{algorithmic}[1]
\REQUIRE
$P_i$ holds the 2-out-of-3 replicate secret share $\share{x}_i$ for $i\in \{0,1,2\}$ 
\ENSURE
$P_i$ gets the 2-out-of-3 replicate secret share $\share{y}_i$ for $i\in \{0,1,2\}$, where $y=\gelu(x)$.

\STATE $P_0$, $P_1$, and $P_2$ jointly compute
\begin{equation*}
\begin{split}
&\shareb{b_0} = \Pi_{\mathsf{LT}}(\share{x}, -4),~~~\vartriangleright b_0 = 1\{x<-4\}\\
&\shareb{b_1} = \Pi_{\mathsf{LT}}(\share{x}, -1.95),~~~\vartriangleright b_1 = 1\{x<-1.95\} \\
&\shareb{b_2} = \Pi_{\mathsf{LT}}(3, \share{x}),~~~~~~\vartriangleright b_2 = 1\{3<x\}
\end{split}
\end{equation*}
and compute 
$\shareb{z_0} = \shareb{b_0} \oplus \shareb{b_1}$,
$\shareb{z_1} = \shareb{b_1} \oplus \shareb{b_2} \oplus 1$, and $\shareb{z_2}=\shareb{b_2}$. Note that $z_0 = 1\{-4\le x < -1.95\}$, $z_1 = 1\{-1.95\le x\le 3\}$, and $z_2 = 1\{x>3\}$.

\STATE Jointly compute $\share{x^2} = \Pi_{\mathsf{Square}}(\share{x})$, $\share{x^3} = \Pi_{\mathsf{Mul}}(\share{x}, \share{x^2})$, $\share{x^4} = \Pi_{\mathsf{Square}}(\share{x^2})$, and $\share{x^6} = \Pi_{\mathsf{Square}}(\share{x^3})$.

\STATE Computing polynomials $\share{F_0(x)}$ and $\share{F_1(x)}$ based on $\{\share{x}, \share{x^2}, \share{x^3}, \share{x^4}, \share{x^6}\}$ as equation~(\ref{eq:geluapprox}) securely.


\RETURN$\share{y} = \Pi_{\mathsf{Mul_{BA}}}(\shareb{z_0}, \share{F_0(x)}) + \Pi_{\mathsf{Mul_{BA}}}(\shareb{z_1}, \share{F_1(x)})+\Pi_{\mathsf{Mul_{BA}}}(\shareb{z_2}, \share{x})$.

\end{algorithmic}
\end{algorithm}



\subsection{Protocol for Secure Softmax}\label{sec:secureatten}

In the function $\attention(\Q,\K,\V)=
\softmax(\Q \cdot \K^\mathsf{T} + \M) \cdot \V$, where $\M$ can be viewed as a bias matrix, the key challenge is computing function $\softmax$. For the sake of numerical stability, the $\softmax$ function is computed as
\begin{equation}\label{eq:softmax}
    \softmax(\x)[i]=\frac{\exp(\x[i] - \bar{x} - \epsilon)}{\sum_i \exp(\x[i] - \bar{x} - \epsilon)},
\end{equation}
where $\bar{x}$ is the maximum element of the input vector $\x$. 
For the normal plaintext softmax, $\epsilon=0$. For a two-dimension matrix, we apply equation~(\ref{eq:softmax}) to each of its row vector.

Formally, our detailed secure protocol  $\Pi_{\softmax}$ is illustrated in algorithm~\ref{protocol:softmax}, where we propose two optimizations:
\begin{itemize}
\item 
For the first optimization, we set $\epsilon$ in equation~\ref{eq:softmax} to a tiny and positive
value, e.g., $\epsilon =
10^{-6}$, so that the inputs to exponentiation
in equation~\ref{eq:softmax} are all negative. We exploit the negative operands
for acceleration. Particularly, we compute the exponentiation using the Taylor series~\citep{tan2021cryptgpu} with a simple clipping
\begin{equation}\label{eq:negexp}
\mathsf{negExp}(x) = \begin{cases}
    0, &x < T_{\exp} \\
    (1+\frac{x}{2^t})^{2^t}, &x\in [T_{\exp},0].
\end{cases}
\end{equation}
Indeed, we apply the less-than for the branch $x < T_{\exp}$
The division by $2^t$ can be achieved using
$\Pi_{\mathsf{Trunc}}^t$ since the input is already negative. Also, we can
compute the power-of-$2^t$ using $t$-step sequences of square function $\Pi_{\mathsf{square}}$ and $\Pi_{\mathsf{Trunc}}^f$. Suppose our MPC program uses
$18$-bit fixed-point precision. Then we set $T_{\exp}=-14$ given $\exp(-14) < 2^{-18}$, and empirically set $t = 5$.


\item 
Our second optimization is to reduce the number of divisions, which ultimately saves computation and communication costs.
To achieve this, for a vector $\x$ of size $n$, we have replaced the operation $\mathsf{Div}(\x, \mathsf{Broadcast}(y))$ with $\x \cdot  \mathsf{Broadcast}(\frac{1}{y})$, where $y=\sum_{i=1}^n\x[i]$. By making this replacement, we effectively reduce $n$ divisions to just one reciprocal operation and $n$ multiplications.
This optimization is particularly beneficial in the case of the $\softmax$ operation. The $\frac{1}{y}$ in the $\softmax$ operation is still large enough to maintain sufficient accuracy under fixed-point values. As a result, this optimization can significantly reduce the computational and communication costs while still providing accurate results.
\end{itemize}

\begin{algorithm}[tp]
\caption{Secure $\softmax$ Protocol $\Pi_{\softmax}$}\label{protocol:softmax}
\begin{algorithmic}[1]
\REQUIRE
$P_i$ holds the 2-out-of-3 replicate secret share $\share{\x}_i$ for $i\in \{0,1,2\}$, and $\x$ is a vector of size $n$. 
\ENSURE
$P_i$ gets the 2-out-of-3 replicate secret share $\share{\y}_i$ for $i\in \{0,1,2\}$, where $\y=\softmax(\x)$.

\STATE $P_0$, $P_1$, and $P_2$ jointly compute
$\shareb{\mathbf{b}} = \Pi_{\mathsf{LT}}(T_{\exp}, \share{\x})$ and the maximum $\share{\bar{x}} = \Pi_{\mathsf{Max}}(\share{\x})$.

\STATE Parties locally computes $\share{\hat{\x}} = \share{\x} - \share{\bar{x}} - \epsilon$, and jointly compute $\share{\z_0} = 1+  \Pi_{\mathsf{Trunc}}^t(\share{\hat{\x}})$.

\FOR{$j=1,2,\dots, t$}
\STATE $\share{\z_j} = \Pi_{\mathsf{Square}}(\share{\z_{j-1}})$.
\ENDFOR

\STATE Parties locally compute $\share{z} = \sum_{i=1}^n \share{\z[i]}$ and jointly compute $\share{1/z} = \Pi_{\mathsf{Recip}}(\share{z})$.

\STATE Parties jointly compute $\share{\z / z} = \Pi_{\mathsf{Mul}}(\share{\z}, \share{1/z})$

\RETURN $\share{\y} = \Pi_{\mathsf{Mul}_{\mathsf{BA}}}( \shareb{\mathbf{b}}, \share{\z / z})$.

\end{algorithmic}
\end{algorithm}

\subsection{Protocol for Secure Embedding}\label{sec:embed}


The current secure embedding procedure described in~\citep{li2023mpcformer} necessitates the client to  generate a one-hot vector using the token $\tokenid$ locally. This deviates from a plaintext Transformer workflow where the one-hot vector is generated inside the model. As a result, they have to carefully strip off the one-hot step from the pre-trained models, and add the step to the client side, which could be an obstacle for deployment. 



To address this issue, we propose a secure embedding design as follows. Assuming that the token $\tokenid\in [n]$ and all embedding vectors are denoted by $\E= (\e_1^T, \e_2^T, \dots, \e_n^T)$, the embedding can be formulated as $\e_{\tokenid} = \mathbf{E}[\tokenid]$. Given $(\tokenid, \E)$ are in secret-shared fashion, our secure embedding protocol $\Pi_{\mathsf{Embed}}$ works as follows:
\begin{itemize}
    \item The computing parties securely compute the one-hot vector $\shareb{\mathbf{o}}$ after receiving $\share{\tokenid}$ from the client. Specifically, $\shareb{\mathbf{o}[i]}=\Pi_{\mathsf{Eq}}(i,\share{\tokenid})$ for $i\in [n]$.
    \item The parties can compute the embedded vector via $\share{\e_{\tokenid}} = \Pi_{\mathsf{Mul_{BA}}}(\share{\E}, \shareb{\mathbf{o}})$, where  does not require secure truncation.
\end{itemize}
In this way, our $\Pi_{\mathsf{Embed}}$ does not require explicit modification of the workflow of plaintext Transformer models, at the cost of more $\Pi_{\mathsf{Eq}}$ and $\Pi_{\mathsf{Mul_{BA}}}$ operations. 



\subsection{Protocol for Secure LayerNorm}\label{sec:seclayernorm}
Recall that given a vector $\x$ of size $n$, $\layernorm(\x)[i] =  \gamma \cdot \frac{\x[i]-\mu}{\sqrt{\sigma}} + \beta$, where $(\gamma, \beta)$ are trained parameters, $\mu = \frac{\sum_{i=1}^n \x[i]}{n}$, and $\sigma = \sum_{i=1}^n (\x[i] - \mu)^2$. In MPC, the key challenge is the evaluation of the divide-square-root $\frac{\x[i]-\mu}{\sqrt{\sigma}}$ formula. To securely evaluate this formula, CrypTen sequentially executes the MPC protocols of square-root, reciprocal, and multiplication. However, we observe that $\frac{\x[i]-\mu}{\sqrt{\sigma}}$ is equal to $(\x[i]-\mu)\cdot \sigma^{-1/2}$. And in the MPC side, the costs of computing the inverse-square-root $\sigma^{-1/2}$ is similar to that of the square-root operation~\citep{rSqrt}. Besides, inspired by the second optimization of \S~\ref{sec:secureatten}, we can first compute $\sigma^{-1/2}$ and then $\mathsf{Broadcast}(\sigma^{-1/2})$ to support fast and secure $\layernorm(\x)$. And our formal protocol $\Pi_{\layernorm}$ is shown in algorithm~\ref{protocol:layernorm}.

\begin{algorithm}[tp]
\caption{Secure $\mathsf{LayerNorm}$ Protocol $\Pi_{\mathsf{LayerNorm}}$}\label{protocol:layernorm}
\begin{algorithmic}[1]
\REQUIRE
$P_i$ holds the 2-out-of-3 replicate secret share $\share{\x}_i$ for $i\in \{0,1,2\}$, and $\x$ is a vector of size $n$. 
\ENSURE
$P_i$ gets the 2-out-of-3 replicate secret share $\share{\y}_i$ for $i\in \{0,1,2\}$, where $\y=\mathsf{LayerNorm}(\x)$.

\STATE $P_0$, $P_1$, and $P_2$ compute $\share{\mu} = \frac{1}{n}\cdot \sum_{i=1}^n\share{\x[i]}$ and $\share{\sigma} = \sum_{i=1}^n \Pi_{\mathsf{Square}}(\share{\x} - \share{\mu})[i]$.

\STATE Parties jointly compute $\share{\sigma^{-1/2}} = \Pi_{\mathsf{rSqrt}}(\share{\sigma})$.

\STATE Parties jointly compute $\share{\mathbf{c}} = \Pi_{\mathsf{Mul}}((\share{\x} - \share{\mu}), \share{\sigma^{-1/2}})$

\RETURN $\share{\y} = \Pi_{\mathsf{Mul}}(\share{\gamma}, \share{\mathbf{c}}) + \share{\beta}$.

\end{algorithmic}
\end{algorithm}
We first review some basic concepts from probability theory (see standard textbooks such as \cite{pollard2002user,williams1991probability} for a detailed treatment), 
%the background of Bayesian inference, and finally 
%We first review some basic concepts from probability theory, 
and then present the Bayesian probabilistic programming language and the normalised posterior distribution (NPD) problem.
%we consider in this work. 
Throughout the paper,
we denote by $\Nset$, $\Zset$ and $\Rset$ the sets of all natural numbers (including zero), integers, and real numbers, respectively.

\vspace{-1.5ex}
\subsection{Basics of Probability Theory}
%We assume familiarity with basic probability theory (see \cref{app:prelim} for details). 

A \emph{measurable space} is a pair $(U,\Sigma_U)$, where $U$ is a nonempty set and $\Sigma_U$ is a $\sigma$-algebra on $U$, i.e., a family of subsets of $U$ such that $\Sigma_U\subseteq \mathcal{P}(U)$ contains $\emptyset$ and is closed under complementation and countable union. Elements of $\Sigma_U$ are called \emph{measurable} sets. A function $f$ from a measurable space $(U_1,\Sigma_{U_1})$ to another measurable space $(U_2,\Sigma_{U_2})$ is \emph{measurable} if $f^{-1}(A)\in\Sigma_{U_1}$ for all $A\in\Sigma_{U_2}$.

A \emph{measure} $\mu$ on a measurable space $(U,\Sigma_U)$ is a mapping from $\Sigma_U$ to $[0,\infty]$ such that (i) $\mu(\emptyset)=0$ and (ii) $\mu$ 
%satisfies the
is countably additive:
%condition: 
for every pairwise-disjoint set sequence $\{A_n\}_{n\in\Nset}$ in $\Sigma_U$, it holds that $\mu(\bigcup_{n\in\Nset}A_n)=\sum_{n\in\Nset}\mu(A_n)$. We call the triple $(U,\Sigma_U,\mu)$ a \emph{measure space}. 
%If $\mu(U)\le 1$, we call $\mu$ a \emph{subprobability measure}. 
If $\mu(U)=1$, we call $\mu$ a \emph{probability measure}, and $(U,\Sigma_U,\mu)$ a \emph{probability space}.
The Lebesgue measure $\lambda$ is the unique measure on $(\Rset,\Sigma_{\Rset})$ satisfying $\lambda([a,b))=b-a$ for all valid intervals $[a,b)$ in $\Sigma_{\Rset}$. For each $n\in\Nset$, we have a measurable space $(\Rset^n,\Sigma_{\Rset^n})$ 
%such that there exists 
and
a unique product measure $\lambda_n$ on $\Rset^n$ satisfying $\lambda_n(\prod_{i=1}^n A_i)=\prod_{i=1}^n \lambda(A_i)$ for all $A_i\in\Sigma_{\Rset}$.


The \emph{Lebesgue} integral operator $\int$ is a partial operator that maps a measure $\mu$ on $(U,\Sigma_U)$ and a real-valued function $f$ on the same space $(U,\Sigma_U)$ to a real number or infinity, which is denoted by $\int f \mathrm{d}\mu$ or $\int f(x)\mu(\mathrm{d}x)$. 
The detailed definition of Lebesgue integral is somewhat technical, see \cite{rankin1968real,rudin1976principles} for more details. 
Given a measurable set $A\in\Sigma_U$, the integral of $f$ over $A$ is defined by $\int_A f(x)\mu(\mathrm{d} x):=\int f(x) \cdot [x\in A] \mu(\mathrm{d}x)$
%\begin{align*}
%\textstyle\int_A f(x)\mu(\mathrm{d} x):=\int f(x) \cdot [x\in A] \mu(\mathrm{d}x)
%\end{align*} 
where $[-]$ is the Iverson bracket such that $[\phi]=1$ if 
%the predicate 
$\phi$ is true, and $0$ otherwise. If $\mu$ is a probability measure, then we call the integral as the \emph{expectation} of $f$, denoted by $\expectdist{x\sim\mu;A}{f}$, or $\expv[f]$ when the scope is clear from the context.

For a measure $v$ on $(U,\Sigma_U)$, a measurable function $f:U\to \Rset_{\ge 0}$ is the \emph{density} of $v$ with respect to $\mu$ if $v(A)=\int f(x)\cdot [x\in A] \mu(\mathrm{d} x)$ for all measurable $A\in\Sigma_U$, and $\mu$ is called the \emph{reference measure} (most often $\mu$ is the Lebesgue measure). Common families of probability distributions on the reals, e.g., uniform, normal distributions, are measures on $(\Rset,\Sigma_{\Rset})$. Most often these are defined in terms of probability density functions with respect to the Lebesgue measure. That is, for each $\mu_D$ there is a measurable function $\text{pdf}_D:\Rset\to\Rset_{\ge 0}$ that determines it: $\mu_D(A):=\int_A \text{pdf}_D (\mathrm{d}\lambda) $. As we will see, density functions such as $\text{pdf}_D$ play an important role in Bayesian inference.

Given a probability space $\pspace$, a \emph{random variable} is an $\mathcal{F}$-measurable function $X: \Omega \rightarrow \Rset \cup \{+\infty,-\infty\}$. The expectation of a random variable $X$, denoted by $\expv(X)$, is the Lebesgue integral of $X$ w.r.t. $\probm$, i.e., $\int X\,\mathrm{d}\probm$. A \emph{filtration} of $\pspace$ is an infinite sequence $\{ \mathcal{F}_n \}_{n=0}^{\infty}$ such that for every $n\ge 0$, the triple $(\Omega, \mathcal{F}_n, \probm)$ is a probability space and $\mathcal{F}_n \subseteq \mathcal{F}_{n+1} \subseteq \mathcal{F}$. A \emph{stopping time} w.r.t. $\{ \mathcal{F}_n \}_{n=0}^{\infty}$ is a random variable $T: \Omega \rightarrow \Nset \cup \{0, \infty\}$ such that for every $n \geq 0$, the event \{$T \leq n$\} is in $\mathcal{F}_n$. 

A \emph{discrete-time stochastic process} is a sequence $\Gamma = \{X_n\}_{n=0}^\infty$ of random variables in $\pspace$. The process $\Gamma$ is \emph{adapted} to a filtration $\{ \mathcal{F}_n \}_{n=0}^{\infty}$, if for all $n \geq 0$, $X_n$ is a random variable in $(\Omega, \mathcal{F}_n, \probm)$. A discrete-time stochastic process $\Gamma=\{X_n\}_{n=0}^\infty$ adapted to a filtration $\{\mathcal{F}_n\}_{n=0}^\infty$ is a \emph{martingale} (resp. \emph{supermartingale}, \emph{submartingale})
if for all $n \geq 0$, $\expv(|X_n|)<\infty$ and it holds almost surely (i.e.,~with probability $1$) that
$\condexpv{X_{n+1}}{\mathcal{F}_n}=X_n$ (\mbox{resp. } $\condexpv{X_{n+1}}{\mathcal{F}_n}\le X_n$, $\condexpv{X_{n+1}}{\mathcal{F}_n}\ge X_n$).
See~\cite{williams1991probability} for details.
%Intuitively, a martingale is a discrete-time stochastic process, in which at any time $n$, the expected value $\condexpv{X_{n+1}}{\mathcal{F}_n}$ in the next step, given all previous values, is equal to the current value $X_n$. In a supermartingale, this expected value is less than or equal to the current value and a submartingale is defined conversely.
Applying martingales to qualitative and quantitative analysis of probabilistic programs is a well-studied technique~\cite{SriramCAV,ChatterjeeFG16,ChatterjeeNZ2017}.


\subsection{Bayesian Probabilistic Programming Language}

%We consider an imperative arithmetic probabilistic programming language. 
The syntax of our probabilistic programming language (PPL) is given in \cref{fig:syntax}, where the metavariables $S$, $B$ and $E$ stand for statements, boolean expressions and arithmetic expressions, respectively.   
Our PPL is imperative with the usual conditional and loop structures (i.e.,~\textbf{if} and \textbf{while}), as well as the following new structures: (a)~sample constructs of the form ``$\textbf{sample}\  D$'' that sample a value from a prescribed distribution $D$ over $\mathbb{R}$ and then assign this value to a sampling variable $r$; (b)~score statements of the form ``\textbf{score}($EW$)'' that weight the current execution with a value expressed by $EW$ (note that $\textit{pdf}(D,x)$ means the value of a probability density function w.r.t. $D$ at $x$);
%\footnote{Instead of the hard conditioning that refutes the execution when the observation mismatches the value of the sampling variable, we use the more general soft conditioning and assume the existence of a global weight variable initialized  to $1$.}
%for each program
(c)~probabilistic branching statements of the form
``$\textbf{if}\ \textbf{prob}(p)\dots$'' that lead to the then part with probability
$p\in (0,1]$ and to the else part with probability $1-p$. We also have sequential compositions (i.e., ";") and support return statements (i.e., \textbf{return}) that 
return the value of the program variable of interest. %The set of all statements is denoted by $Stmt$.
Note that $c,c_1,c_2\in\Rset$ are constants, and our language supports any distributions with continuous density functions and infinite supports, 
including but not limited to uniform and normal distributions. 



% Figure environment removed





Given a probabilistic program in our language, we distinguish two disjoint sets of variables in the program: (i) the set $\pvars$ of \emph{program variables} whose values are determined by assignments in the program (i.e., the expressions at the LHS of ``:="); (ii)~the set $\rvars$ of \emph{sampling variables} whose values are independently sampled from prescribed probability distributions each time they are accessed (i.e., each ``$\textbf{sample}\ D$" can be regarded as a sampling variable $r$). 




\begin{example}\label{ex:pedestrian-program}

%Consider the pedestrian random walk example~\cite{DBLP:conf/esop/MakOPW21}, a pedestrian is lost on a road, and she only knows that she is away from her house at most $3$ km. Thus, she starts to repeatedly walk a uniformly random distance of at most $1$ km in either direction, until reaching her house. Upon she arrives, an  odometer tells that she has walked $1.1$ km totally. However, this odometer was once broken and the measured distance is normally distributed around the true distance with standard deviation $0.1$ km. 
\cref{fig:pedestrian-program} shows a Bayesian probabilistic program written in our PPL language. In this program, the set of program variables is $\pvars=\{start,pos,dis,step\}$, and the set of sampling variables is $\rvars=\{ \textbf{sample uniform}(0,1)\}$. Each time $\textbf{sample uniform}(0,1)$ is executed, it samples a value uniformly from $[0,1]$ and then assigns the value to the variable $step$. 
%Thus, $step$ is associated with the probability distribution $\textbf{uniform}(0,1)$.
\qed


	
% Figure environment removed
\end{example}

\subsection{The Semantics of Our Programming Language}

%To relate variables with their values, we introduce the notion of valuations. 
Let $V$ be a finite set of variables with an implicit linear order over its elements. A \emph{valuation} on $V$ is a function $\pv: V \rightarrow \Rset$ that assigns a real value to each variable in $V$. We denote the set of all valuations on $V$ by $\val{V}$. For each $1\le i\le |V|$, we denote the value of the $i$-th variable (in the implicit linear order) in $\pv$ by $\pv[i]$, so that we can view each valuation as a real vector on $V$. A \emph{program} (resp. \emph{sampling}) valuation is a valuation on $\pvars$ (resp. $\rvars$), respectively. 
For the sake of convenience, we fix the notations in the following way, i.e., we always use $\pv\in\val{\pvars}$ to denote a program valuation, and $\rv\in\val{\rvars}$ to denote a sampling valuation; we also write $\pv[\mathit{ret}]$ for the value of the return variable in $\pv$. 



Below we present the semantics for our programming language. Existing semantics in the literature are either measure-\cite{DBLP:conf/lics/StatonYWHK16,LeeYRY20} or sampling-based  \cite{DBLP:conf/esop/MakOPW21,Beutner2022b}. To facilitate the development of our algorithm, we consider the \emph{transition-based} semantics~\cite{DBLP:conf/cav/ChakarovS13,DBLP:conf/popl/ChatterjeeFNH16} to our language and 
%To apply template-based algorithmic approaches to NPD problems, we consider  that 
treat each probabilistic program as a \emph{weighted probabilistic transition system} (WPTS). A WPTS extends a PTS  ~\cite{DBLP:conf/cav/ChakarovS13,DBLP:conf/popl/ChatterjeeFNH16} with weights and an initial probability distribution. 





%Below we present a variant of probabilistic transition systems \cite{DBLP:conf/cav/ChakarovS13}.
\begin{definition}
%[Weighted Probabilistic Transition Systems]
[WPTS]\label{def:wpts}
	A \emph{weighted probabilistic transition system} (WPTS) $\Pi$
	is a tuple
\begin{equation}\label{eq:wpts} 
\tag{\dag}
\Pi = (\pvars, \rvars,  L,\lin,\lout,\mu_{\mathrm{init}}, \rdvarjdis,\transset)%\win)
\end{equation}
for which:
	\begin{itemize}
		\item
		$\pvars$ and $\rvars$ are finite disjoint sets of \emph{program} and resp. \emph{sampling} variables.
%  (variables}) 
%  such that $\pvars\cap \rvars=\emptyset$.
    \item $\locs$ is a finite set of \emph{locations} 
  %or \emph{program counters} 
  with special locations $\lin,\lout\in \locs$. Informally, $\lin$ is the initial location and $\lout$ represents program termination. 
		\item
		$\mu_{\mathrm{init}}$ is the \emph{initial probability distribution} over $\mathbb{R}^{\pvars}$ with a finite support (denoted by $\supp{\mu_{\mathrm{init}}}$), 
  %from which the initial program valuation %$\valin$ is sampled, 
  while $\rdvarjdis$ is a function that assigns a probability distribution $\rdvarjdis(r)$ to each 
  %sampling variable 
  $r \in \rvars$. We call each $\pv\in\supp{\mu_{\mathrm{init}}}$ an \emph{initial program valuation}, and abuse the notation so that $\rdvarjdis$ also denotes the independent joint distribution of all $\rdvarjdis(r)$'s ($r\in \rvars$).
		\item 
		$\transset$ is a finite set of \emph{transitions} where
		each transition $\tau \in \transset$ is a tuple $\langle \loc, \phi, F_1,\dots,F_k \rangle$ such that 
(i) $\loc\in L$ is the \emph{source location} of the transition, 
%\item 
(ii) $\phi$ is the \emph{guard condition} which is a predicate over variables $\pvars$, %which serves as the \emph{guard condition}, 
and (iii) each $F_j:=\langle \loc'_j, p_j, \upd_j,\wet_j \rangle$ is called a \emph{weighted fork} for which (a) $\loc'_j\in L$ is the \emph{destination location} of the fork, (b) $p_j\in (0,1]$ is the probability of this fork, (c) $\upd_j:\Rset^{|\pvars|} \times \Rset^{|\rvars|} \rightarrow \Rset^{|\pvars|}$ is an {\em update function} that takes as inputs the current program and sampling valuations  and returns an updated program valuation in the next step, and (d) $\wet_j:\Rset^{|\pvars|} \times \Rset^{|\rvars|}\to [0,\infty)$ is a \emph{score function} that gives the likelihood weight of this fork depending on the current program and sampling valuations.	
\end{itemize}
\end{definition}


In a WPTS, we use update and score functions to model the update on the program variables and resp. the likelihood weight when running a basic block of statements in a program, respectively.  
%and use score functions to model  caused by the execution of the score statements (if exists) in this block. 
If there is no score statement in the block, then the score function is constantly $1$. 
We always assume that a WPTS $\Pi$ is \emph{deterministic} and \emph{total}, i.e., (i) there is no program valuation that simultaneously satisfies the guard conditions of two distinct transitions from the same source location, and (ii) the disjunction of the guard conditions of all the transitions from any source location is a tautology. 
The transformation from a probabilistic program into its WPTS can be done in a straightforward way (see e.g.~\cite{DBLP:journals/toplas/ChatterjeeFNH18,DBLP:conf/cav/ChakarovS13}). 

\begin{example}\label{ex:pedestrian-semantics} 
\cref{fig:pedestrian-wpts} shows the WPTS of the program in \cref{fig:pedestrian-program} which has two locations $\lin,\lout$. 
 %In the WPTS, 
The circle nodes represent locations and square nodes model the forking behavior of transitions. An edge entering a square node is labeled with the condition of its respective transition, while an edge entering a circle node stands for a fork, which is associated with its probability, update functions and score functions that marked by $w$.\footnote{Here we omit the update functions if the values of program variables are unchanged.} The value of $step$ is initialised to $0$. An the initial probability distribution $\mu_{\mathrm{init}}$ is determined by the joint distribution of $(start,pos,dis,step)$ where $start\sim uniform(0,3)$ and $pos,dis,step$ observe the Dirac measures $Dirac(\{start\})$, $Dirac(\{0\})$ and $Dirac(\{0\})$, respectively, e.g., the probability of the event ``$dis\in\{0\}$'' equals $1$. As $step$ simply receives values from a sampling variable, we neglect it in the WPTS.\qed
\end{example}

%\paragraph{Score-recursive WPTS.} 

We say that a WPTS is \emph{non-score-recursive} if for all transitions $\tau=\langle \loc, \phi, F_1,  F_2,\dots,F_k \rangle$ in the WPTS with each fork $F_j=\langle \loc'_j, p_j, \upd_j,\wet_j \rangle$ ($1\le j\le k$), we have that each score function $\wet_j$ is constantly $1$ (i.e., the multiplicative weight does not change) for every $\loc'_j\ne \lout$. Otherwise, the WPTS is \emph{score-recursive}.
Informally, a non-score-recursive WPTS has non-trivial score functions only on the transitions to the termination of a program, while a score-recursive WPTS has {\tt score} statements in the execution of the program. 
For example, the WPTS of the program in~\cref{sec3:pedestrian} is non-score-recursive as the nontrivial (i.e., score values not equal to $1$) {\tt score} statement only appears to the termination, while the WPTS of the program in \cref{sec3:phylogenetic} is score recursive since it has {\tt score} statements inside the loop body.
In the case of a non-score-recursive WPTS, we say that the WPTS is \emph{score-bounded} by a positive real $M>0$ if for every $\tau=\langle \loc, \phi, F_1, F_2,\dots,F_k \rangle$ in the WPTS with $F_j=\langle \loc'_j, p_j, \upd_j,\wet_j \rangle$ ($1\le j\le k$), we have that 
$|\wet_j|\le M$ whenever $\loc'_j=\lout$.


Given a program valuation $\mathbf{v}$ and a predicate $\phi$ over variables $\pvars$, we say that $\mathbf{v}$ \emph{satisfies} $\phi$ (written as $\mathbf{v}\models\phi$) if $\phi$ holds when the variables in $\phi$ are substituted by their values in $\mathbf{v}$. 
A \emph{state} 
%of the WPTS $\Pi$ 
is a pair $\Xi=(\loc, \pv)$ where $\loc \in L$ (resp. $\pv \in \Rset^{|\pvars|}$) represents the current location (resp. program valuation), respectively, while a \emph{weighted state} is a triple 
%$\Xi^w:=(\loc, \pv,w)$ 
$\Theta=(\loc, \pv, w)$ 
where $(\loc, \pv)$ is a state and $w\in [0,\infty)$ represents the multiplicative likelihood weight accumulated so far. 


 
%\paragraph{Semantics.} 
Below we specify the semantics of a WPTS. Consider a WPTS $\Pi$ in the form of \eqref{eq:wpts}. The semantics of $\Pi$ is formalized by the infinite sequence $\Gamma=\{\widehat{\Theta}_n=(\widehat{\loc}_n,\widehat{\pv}_n,\widehat{w}_n)\}_{n\ge 0}$ 
%of \emph{random weighted states} 
where each $(\widehat{\loc}_n,\widehat{\pv}_n,\widehat{w}_n)$ is the random weighted state at the $n$th execution step of the WPTS such that $\widehat{\loc}_n$ (resp. $\widehat{\pv}_n$, $\widehat{w}_n$) is the random variable for the location (resp. the random vector 
%of random variables 
for the program valuation, the random variable for the multiplicative likelihood weight) at the $n$th step, respectively. %The initial random state $\widehat{\Theta}_0$ is constant and equals $(\lin,\valin,\win)$. 
%its corresponding stochastic process $\Gamma:=\{\hat{\Xi}_n\}_{n\ge 0}$ on states.
The sequence $\Gamma$ starts with the initial random weighted state 
$\widehat{\Theta}_0=(\widehat{\loc}_0,\widehat{\pv}_0,\widehat{w}_0)$ such that $\widehat{\loc}_0$ is constantly $\lin$, $\widehat{\pv}_0\in \supp{\mu_\mathrm{init}}$ is sampled from the initial distribution $\mu_\mathrm{init}$ and the initial weight $\widehat{w}_0$ is constantly set to $1$\footnote{This follows the traditional setting in e.g.~\cite{Beutner2022b}.}. 
Then, given the current random weighted state $\widehat{\Theta}_n=(\widehat{\loc}_n,\widehat{\pv}_n,\widehat{w}_n)$ at the $n$th step, the next random weighted state $\widehat{\Theta}_{n+1}=(\widehat{\loc}_{n+1},\widehat{\pv}_{n+1},\widehat{w}_{n+1})$ is determined by:
(a) If $\widehat{\loc}_n=\lout$, then $(\widehat{\loc}_{n+1}, \widehat{\pv}_{n+1},\widehat{w}_{n+1})$ takes the same weighted state as $(\widehat{\loc}_n,\widehat{\pv}_n,\widehat{w}_n)$ (i.e., the next weighted state stays at the termination location $\lout$);
(b) Otherwise, $\widehat{\Theta}_{n+1}$ is determined by the following procedure:
\begin{itemize}
\item First, since the WPTS $\Pi$ is deterministic and total, we take the unique transition $\tau=\langle \hat{\loc}_n,\phi,F_1,\dots, F_k \rangle$ such that $\hat{\pv}_n\models\phi$. 
\item Second, we choose a fork $F_j=\langle \loc_j, p_j,\upd_j,\wet_j\rangle$ with probability $p_j$.
\item 
Third, we obtain a sampling valuation $\rv\in \supp{\rdvarjdis}$ 
%over the sampling variables $\rvars$ 
by sampling each $r \in \rvars$ independently w.r.t the probability distribution $\rdvarjdis(r).$
\item Finally, the value of the next random weighted state $(\widehat{\loc}_{n+1}, \widehat{\pv}_{n+1},\widehat{w}_{n+1})$ is determined as that of 
$(\loc'_j, \upd_j(\hat{\pv}_n,\rv),\widehat{w}_n\cdot \wet_j(\widehat{\pv}_n,\rv))$. 
\end{itemize}


Given the semantics, a \emph{program run} of the WPTS $\Pi$ is a concrete instance of $\Gamma$, i.e., an infinite sequence $\omega=\{\Theta_n\}_{n\ge 0}$ of weighted states where each $\Theta_n=(\loc_n,\pv_n,w_n)$ is the concrete weighted state at the $n$th step in this program run with location $\loc_n$, program valuation $\pv_n$ and multiplicative likelihood weight $w_n$. A state $(\loc,\pv)$ is called \emph{reachable} if there exists a program run $\omega=\{\Theta_n\}_{n\ge 0}$ such that $\Theta_n=(\loc,\pv,w_n)$ for some $n$. 


 
\begin{example}\label{ex:pedestrian-run}
Consider the WPTS in \cref{ex:pedestrian-semantics}. Consider an initial program valuation $(1,1,0)$ which means that the initial values of $start,pos,dis$ are $1,1,0$, respectively. Then starting from the initial weighted state $(\lin,(1,1,0),1)$, a program run w.r.t the WPTS semantics above could be 
\[
(\lin,(1,1,0),1)\to (\lin,(1,0.5,0.5),1)\to (\lin,(1,-0.1,1.1),1)\to (\lout,(1,-0.1,1.1),3.9894).\qed
\]
\end{example}

Given an initial program valuation $\valin$ of a WPTS, one could construct a probability space over the program runs by their probabilistic evolution described above and standard constructions such as general state space Markov chains~\cite{meyn2012markov}. We denote the probability measure in the probability space by $\probm_{\valin}(-)$ and the expectation operator by $\expectdist{\valin}{-}$.  



\subsection{Normalised Posterior Distribution}\label{sec2:NPD}


Before presenting the central problem of Bayesian probabilistic programming, i.e., analyzing normalised posterior distribution with our WPTS models, we introduce some technical concepts.

%\paragraph{Termination.}
\begin{definition}[Termination]
The \emph{termination time} of a WPTS
%The \emph{termination time} of the WPTS 
$\Pi$ 
%is a random variable $T$ defined on programs runs given 
is the random variable $T$ given by
%a program run  $\omega=\{\Xi_n=(\loc_n,\pv_n,w_n)\}_{n\in\Nset}$,
%\begin{align*}	
$T(\omega):=\text{min}\{n\in\Nset\mid \loc_n=\lout\}$ for every program run  $\omega=\{(\loc_n,\pv_n,w_n)\}_{n\ge 0}$
%\end{align*}
where $\text{min}\,\emptyset:=\infty$. That is, $T(\omega)$ is the number of steps a program run $\omega$ takes to reach the termination location $\lout$. A WPTS $\Pi$ is \emph{almost-surely terminating} (AST) if $\probm_{\valin}(T<\infty)=1$ for all initial program valuations $\valin\in \supp{\mu_{\mathrm{init}}}$.  
%in the case that the program run never terminates. 
\end{definition}




\begin{definition}[Expected Weights]\label{def:exp-wt}
 Given a WPTS $\Pi$ in the form of \eqref{eq:wpts}, a designated initial program valuation $\valin$ and a measurable subset $\calU\in\Sigma_{\Rset^{|\pvars|}}$, the \emph{expected weight} $\measureSem{\Pi}_{\valin}(\calU)$ 
%$\measureSem{\Pi}(\valin)$ 
%of $\Pi$ w.r.t $\pv$ 
is defined as
%$\measureSem{\Pi}_\calU(\valin):=\expectdist{\valin}{\widehat{w}_T}$. 
$\measureSem{\Pi}_{\valin}(\calU):=\expectdist{\valin}{[\widehat{\pv}_T\in \calU]\cdot\widehat{w}_T}$. 
\end{definition}

By definition, we have that $\widehat{\pv}_T$ (resp. $\widehat{w}_T$) is the random vector (resp. variable) of the program valuation (resp. the multiplicative likelihood weight) at termination, respectively. Thus, $\measureSem{\Pi}_{\valin}(\calU)$ is the expectation of $\widehat{w}_T$ 
%over all program runs 
that start from the state $(\lin,\valin,1)$ and end with $\widehat{\pv}_T\in\calU$. If $\calU=\Rset^{|\pvars|}$, the restriction of $\widehat{\pv}_T\in\calU$ can be removed.

Below we define the normalised posterior distribution (NPD) problem. %under our WPTS semantics. 

 
\begin{definition}[Normalised Posterior Distribution]\label{def:npd}
Given a WPTS $\Pi$ in the form of \eqref{eq:wpts},
%We write $\measureSem{\Pi}(\valin)$ iff $\calU=\Rset^{|\pvars|}$.)
%Then given a probability distribution $\mu$ over initial program valuations, 
the \emph{normalised posterior distribution} (NPD) $\posterior_\Pi$ of $\Pi$ 
%over $U$ 
is defined by:
\begin{align*}
\posterior_{\Pi}(\calU):=\measureSem{\Pi}(\calU)/Z_\Pi\mbox{ for all measurable subsets } \calU\in \Sigma_{\Rset^{|\pvars|}},   
\end{align*}	
where 
$\measureSem{\Pi}(\calU):=\int_{\calV} \measureSem{\Pi}_{\pv}(\calU)\cdot \mu_{\mathrm{init}}(\mathrm{d} \pv)$ is the \emph{unnormalised posterior distribution} w.r.t. $\calU$, $\calV:=\supp{\mu_{\mathrm{init}}}$, %is the support of $\mu_{\mathrm{init}}$
%is the integral of all expected weights with an initial program valuation $\pv\in U$, 
and $Z_\Pi:=\measureSem{\Pi}(\Rset^{|\pvars|})$ is the \emph{normalising constant}.  
The WPTS $\Pi$ is called \emph{integrable} 
%w.r.t a probability distribution (for initial program valuations) 
if we have $0<Z_{\Pi}<\infty$. 
%\pw{Shall we mention that $\measureSem{\Pi}_{\pv}(\calU)$ is an integrable function here?}
\end{definition}

%We call a WPTS $\Pi$ \emph{integrable} 
%w.r.t a probability distribution (for initial program valuations) 
%if the normalising constant is finite, i.e., ~$0<Z_{\Pi}<\infty$. %for any $\pv\in\val{\pvars}$. 
%Given an integrable program, we are interested in deriving lower and upper bounds on the normalised posterior distribution over some measurable set $U\in \Sigma_\Rset$.
\paragraph{Interval Bounds for NPD.} In this work, we consider the automated interval-bound analysis for NPD of a WPTS. Formally, we aim to derive an interval $[l,u]\subseteq [0,\infty)$ 
for an integrable WPTS $\Pi$ and any measurable sets $\calU\in\Sigma_{\Rset^{|\pvars|}}$ as tight as possible such that $l\le \posterior_{\Pi}(\calU) \le u$. 
%$l,u$ are called \emph{interval bounds} for the NPD $\posterior_{\Pi}(\calU)$. 
%To achieve this, in the following (\cref{sec:math}) we develop approaches to obtain interval bounds for expected weights as $\measureSem{\Pi}(\calU)$ and $Z_\Pi$ are integrations of expected weights over $\calV$. 
 



To achieve interval bounds for NPD, below we introduce the construction of a new WPTS $\Pi_\calU$ based on the original WPTS $\Pi$ and a measurable set $\calU\in \Sigma_{\Rset^{|\pvars|}}$.  

\paragraph{Construction of $\Pi_\calU$.} Consider a probabilistic program $P$ and its WPTS $\Pi$, given a measurable set $\calU\in\Sigma_{\Rset^{|\pvars|}}$, we construct a new program $P_\calU$ by adding a conditional branch of the form ``\textbf{if} $\pv_T\notin\calU$ \textbf{then} \textbf{score}($0$) \textbf{fi}'' immediately after the termination of $P$ and obtain the WPTS $\Pi_\calU$ of $P_\calU$. Therefore, $\Pi$ and $\Pi_\calU$ have the same initial probability distribution $\mu_{\mathrm{init}}$ and the same finite support $\calV=\supp{\mu_{\mathrm{init}}}$. The following proposition shows that interval-bound analysis for NPD can be reduced to interval-bound analysis for expected weights in the form $\llbracket \Pi\rrbracket_{\pv}(\Rset^{|\pvars|})$. 

\begin{proposition}\label{prop:unnorm-norm}
   Given a WPTS $\Pi$ in the form of \eqref{eq:wpts}, a measurable set $\calU\in\Sigma_{\Rset^{|\pvars|}}$ and the WPTS $\Pi_\calU$ constructed as above, we have that $\llbracket \Pi \rrbracket_{\pv}(\calU)=\llbracket \Pi_\calU\rrbracket_{\pv}(\Rset^{|\pvars|})$ for any $\pv\in\calV=\supp{\mu_{\mathrm{init}}}$. Furthermore,
   if there exist intervals $[l_1,u_1],[l_2,u_2]\subseteq [0,\infty)$ such that $\llbracket \Pi_\calU\rrbracket_{\pv}(\Rset^{|\pvars|})\in [l_1,u_1]$ and $\llbracket \Pi\rrbracket_{\pv}(\Rset^{|\pvars|})\in [l_2,u_2 ]$ for any $\pv\in\calV$, then we have two intervals $[l_\calU,u_\calU],[l_Z,u_Z]\subseteq [0,\infty)$ such that the unnormalised posterior distribution $\llbracket \Pi\rrbracket (\calU)\in [l_\calU,u_\calU]$ and the normalising constant $Z_\Pi\in [l_Z,u_Z]$. Moreover, if $\Pi$ is integrable, i.e., $[l_Z,u_Z]\subseteq (0,\infty)$, then we can obtain the NPD $\posterior_{\Pi}(\calU)\in [\frac{l_\calU}{u_Z},\frac{u_\calU}{l_Z}]$.\footnote{The interval bounds derived in this manner may be loose, but they are definitely correct.}  Note that by \cref{def:npd}, $l_\calU=\int_\calV l_1 \cdot\mu_{\mathrm{init}}(\mathrm{d} \pv)$, $u_\calU=\int_\calV u_1 \cdot\mu_{\mathrm{init}}(\mathrm{d} \pv)$, $l_Z=\int_\calV l_2 \cdot\mu_{\mathrm{init}}(\mathrm{d} \pv)$ and $u_Z=\int_\calV u_1 \cdot\mu_{\mathrm{init}}(\mathrm{d} \pv)$.

\end{proposition}

The proof of \cref{prop:unnorm-norm} is relegated to \cref{app:sec2-prop}. In the following, we will develop approaches to obtain interval bounds for expected weights.
%in the form $\llbracket \Pi \rrbracket_{\pv}(\Rset^{|\pvars|})$ where $\pv$ is an initial program valuation.











\section{Background and Problem Statement}
\label{sec:setup}
We consider the problem of an agent interacting with an SCM for $T$ rounds in order to maximize the value of a reward variable. We start by introducing SCMs, the soft intervention model used in this work, and then define the adversarial sequential decision-making problem we study. In the following, we denote with $[m]$ the set of integers $\{0, \dots, m\}$. \looseness-1

\paragraph{Structural Causal Models}
Our SCM is described by a tuple $\langle \G,  Y, \bX, \fs, \snoiserv \rangle$ of the following elements: $\G$ is a \emph{known} DAG; $Y$ is the reward variable; $\bX = {\{X_i\}_{i=0}^{m-1}}$ is a set of observed scalar random variables; the set $\fs = \{\fofi\}_{i=0}^m$ defines the \emph{unknown} functional relations between these variables; and $\snoiserv = \{\snoiserv_i \}_{i=0}^{m}$ is a set of independent noise variables with zero-mean and known distribution. % \looseness-1
 We use the notation $Y$ and $X_m$ interchangeably and assume the elements of $\bX$ are topologically ordered, i.e., $X_0$ is a root and $X_m$ is a leaf.  We denote with $\pa_i \subset \{0, \dots, m\}$ the indices of the parents of the $i$th node, and use the notation $\bZi = \{ X_j\}_{j \in \pa_i}$ for the parents this node. We sometimes use $X_i$ to refer to both the $i$th node and the $i$th random variable. \looseness-1\looseness-1

Each $X_i$ is generated according to the function $\fofi: \calZ_i \rightarrow \calX_i$, taking the parent nodes $\bZi$ of $X_i$ as input: $\si =\fofi(\zi) + \noisei$, where lowercase denotes a realization of the corresponding random variable. The reward is a scalar $x_m \in [0,1]$ while observation $X_i$ is defined over a compact set $\si \in \calX_i \subset \R$, and its parents are defined over $\calZ_i = \prod_{j \in pa_i} \calX_j$ for $i\in [m-1]$.\footnote{Here we consider scalar observations for ease of presentation, but we note that the methodology and analysis can be easily extended to vector observations as in \citet{sussex2022model}}  \looseness-1

\paragraph{Interventions}

\looseness -1 In our setup, an agent and an adversary both perform \emph{interventions} on the SCM~\footnote{Our framework allows for there to be potentially multiple adversaries, but since we consider everything from a single player's perspective, it is sufficient to combine all the other agents into a single adversary.}. 
We consider a soft intervention model \citep{eberhardt2007interventions} where interventions are parameterized by controllable \emph{action variables}. A simple example of a soft intervention is a shift intervention, where actions affect their outputs additively \citep{zhang2021matching}.

First, consider the agent and its action variables $\bm a = {\{ \ai\}_{i=0}^{m}}$. Each action $a_i$ is a real number chosen from some finite set. That is, the space $\calA_i $  of action $a_i$ is   $\calA_i \subset \R_{[0, 1]}$ where $\abs{\calA_i} = K_i$  for some $K_i \in \nN$. Let $\calA$ be the space of all actions $\bm a = {\{ \ai\}_{i=0}^{m}}$. 
% Let $\calA$ be the space of all actions $\bm a = {\{ \ai\}_{i=0}^{m}}$.
We represent the actions as additional nodes in $\G$ (see \cref{fig:overview}): $\ai$ is a parent of only $X_i$, and hence an additional input to $\fofi$. Since $\fofi$ is unknown, the agent does not know apriori the functional effect of $\ai$ on $X_i$. Not intervening on a node $X_i$ can be considered equivalent to selecting $\ai = 0$. For nodes that cannot be intervened on by our agent, we set $K_i = 1$ and do not include the action in diagrams, meaning that without loss of generality we consider the number of action variables to be equal to the number of nodes $m$.
\footnote{There may be constraints on the actions our agent can take. We refer the reader to \citet{sussex2022model} for how our setup can be extended to handle constraints.}

For the adversary we consider the same intervention model but denote their actions by $\a'$ with each $\ai'$ defined over $\calA_i' \subset \R_{[0, 1]}$ where $\abs{\calA_i'} = K_i'$ and $K_i'$ is not necessarily equal to $K_i$. 

According to the causal graph, actions $\a, \a'$ induce a realization of the graph nodes: 
\begin{align}
\label{eq:groud_truth}
& \si = \fofi(\zi, \ai, \ai') + \noisei, \ \ \forall i \in [m].
\end{align}
 
If an index $i$ corresponds to a root node, the parent vector $\zi$ denotes an empty vector, and the output of $\fofi$ only depends on the actions.

\looseness-1

\paragraph{Problem statement}
Over multiple rounds, the agent and adversary intervene simultaneously on the SCM, with known DAG $\calG$ and fixed but unknown functions $\fs = \{\fofi\}_{i=1}^m$ with $\fofi: \calZ_i \times \A_i \times \A_i' \rightarrow \calX_i$. \looseness-1
At round $t$ the agent selects actions $\at = \{\ait\}_{i=0}^m$ and obtains observations $\st = \{\sit\}_{i=0}^m$, where we add an additional subscript to denote the round of interaction. When obtaining observations, the agent also observes what actions the adversary chose $\at' = \{\ait'\}_{i=0}^m$.  We assume the adversary does not have the power to know $\at$ when selecting $\at'$, but only has access to the history of interactions until round $t$. The agent obtains a reward given by \looseness-1
\begin{align}
\label{eq:groud_truth_target}
& y_t = f_m(\bm z_{m, t}, a_{m, t}, a_{m, t}') + \noise_{m, t},
\end{align}
which implicitly depends on the whole action vector $\at$ and adversary actions $\at'$. 

The agent's goal is to select a sequence of actions that maximizes their cumulative expected reward $\sum_{t=1}^T 
r(\at, \at')$ where $r(\at, \at') = \E{y_t\mid \at, \at'}$ and expectations are taken over $\snoise$ unless otherwise stated. The challenge for the agent lies in not knowing a-priori neither the causal model (i.e., the functions $\fs = \{\fofi\}_{i=1}^m$), nor the sequence of adversarial actions $\{\at'\}_{t=1}^{\cdots}$.

\paragraph{Performance metric} 

After $T$ timesteps, we can measure the performance of the agent via the notion of regret:
\begin{align}
    R(T) = \max_{\a \in \A} \sum_{t=1}^T r(\a, \at') - \sum_{t=1}^T r(\at, \at'),
    \label{eq:regret}
\end{align}
\ie, the difference between the best cumulative expected reward obtainable by playing a single fixed action if the adversary's action sequence and $\fs$ were known in hindsight, and the agent's cumulative expected reward. We seek to design algorithms for the agent that are \emph{no-regret}, meaning that $R(T)/T \rightarrow 0$ as $T\rightarrow \infty$, for any sequence $\at'$. We emphasize that while we use the term `adversary', our regret notion encompasses all strategies that the adversary could use to select actions. This might include cooperative agents or mechanism non-stationarities. \looseness -1


 For simplicity, we consider only adversary actions observed after the agent chooses actions. Our methods can be extended to also consider adversary actions observed \emph{before} the agent chooses actions, i.e., a \textit{context}. This results in learning a policy that returns actions depending on the context, rather than just learning a fixed action. This extension is straightforward and we briefly discuss it in~\Cref{app:contextual}. \looseness-1

\textbf{Regularity assumptions} We consider standard smoothness assumptions for the unknown functions $\fofi:\mathcal{S} \rightarrow \X_i$ defined over a compact domain $\mathcal{S}$ \citep{srinivas10}. In particular, for each node $i \in [m]$, we assume that $\fofi(\cdot)$ belongs to a reproducing kernel Hilbert space (RKHS) $\mathcal{H}_{k_i}$, a space of smooth functions defined on $\calS = \calZ_i \times \calA_i \times \calA_i'$.
This means that $\fofil \in \mathcal{H}_{k_i}$ is induced by a kernel function $k_i: \calS \times  \calS \rightarrow \mathbb{R}$. 
We also assume that $k_i(s,s') \leq 1$ for every $s, s' \in \calS$\footnote{This is known as the bounded variance property, and it holds for many common kernels.}. Moreover, the RKHS norm of $\fofi(\cdot)$ is assumed to be bounded $\|\fofi\|_{k_i} \leq \mathcal{B}_i$ for some fixed constant $\mathcal{B}_i>0$.  Finally, to ensure the compactness of the domains $\Z_i$, we assume that the noise $\snoise$ is bounded, i.e., $\noisei \in \left[-1,1\right]^{d}$. \looseness-1

\section{Level-One Fourier Growth}\label{sec:proof_of_level_one}

In this section, we will give a proof of
\Cref{thm:boolean_bound_level_one} that $L_{1,1}(h) = O(\sqrt{d})$. We start with a $d$-round communication protocol $\tilde{\Ccal}$ over the Gaussian space as defined in \Cref{sec:boolean_to_real}. 
Given the discussion in the previous section and \Cref{prop:fwt-to-qv}, our task ultimately reduces to bounding the expected quadratic variation of the martingale that results from the protocol $\bar{\Ccal}$. For example, one can simply take $\bar\Ccal=\tilde\Ccal$, but, as discussed in \Cref{sec:overview}, the individual step sizes of this martingale can be quite large in the worst-case and it is not so easy to leverage cancellations here to bound the quadratic variation by $O(d)$. 

So, we first define a \emph{generalized} communication protocol $\bar{\Ccal}$ that is equivalent to the original protocol $\tilde\Ccal$ but has additional ``cleanup'' rounds where Alice and Bob reveal certain linear forms of their inputs so that their sets are pairwise clean in the sense described in the overview. These cleanup steps allow us to keep track of the quadratic variation more easily.

\subsection{Pairwise Clean Protocols}\label{sec:pairwise_clean_protocols}

To define a clean protocol, we first define the notion of a pairwise clean set.
Let $X\subseteq \Rbb^n$. We say that the set $X$ is \emph{pairwise clean in a direction $a \in \mathbb{S}^{n-1}$} with parameter $\lambda$ if 
\begin{equation}\label{eqn:pairwiseclean}
 \E_{\lX\sim \gamma}\sbra{ \abra{\lX-\com(X),a}^2 \mid \lX\in X }\le \lambda,
\end{equation}
where we recall that $\com(X) = \E_{\lX\sim \gamma}\sbra{\lX\mid \lX \in X}$ is the level-one center of mass of $X$. 

The above condition implies that for a random vector $\lX$ sampled from $\gamma$ conditioned on $X$, its variance along the direction $a$ is bounded by $\lambda$. We say that the set $X$ is \emph{pairwise clean} (with parameter $\lambda$) if it is clean in \emph{every direction $a \in \mathbb{S}^{n-1}$}. Equivalently, the operator norm of the covariance matrix of the random vector $\lX$ is bounded by $\lambda$.

We call a generalized communication protocol pairwise clean with parameter $\lambda$ if at the start of a new ``phase'' of the protocol, the corresponding rectangle $X \times Y$ satisfies that both $X$ and $Y$ are pairwise clean. Starting from a communication protocol $\tilde{\Ccal}$ in the Gaussian space, we will transform it into a pairwise clean protocol $\bar \Ccal$ by proceeding from top to bottom and adding certain Gram-Schmidt orthogonalization and cleanup steps.

In particular, consider an intermediate node in the protocol tree of $\tilde{\Ccal}$. Before Alice sends her bit as in the original protocol $\tilde\Ccal$, she first performs an orthogonalization step by revealing the inner-product between her input and Bob's current level-one center of mass. After this, she sends her bit according to the original protocol and afterwards she repeatedly cleans her current set $X$ by revealing $\abra{x,a}\in \Rbb$ while $X$ is not clean along the direction $a$ orthogonal to previous directions. 
Once $X$ becomes clean, they proceed to the next round. 
We now describe this formally.

\paragraph*{Construction of pairwise clean protocol $\bar \Ccal$ from $\tilde \Ccal$.}

We set $\lambda = 100$. The construction of the new protocol is recursive and we first define some notation. Consider an intermediate node of the new protocol $\bar \Ccal$ at depth $t$. We use the random variable $\supX{t}\subseteq\Rbb^n$ (resp., $\supY{t}\subseteq \Rbb^n$) to denote the set of inputs of Alice (resp., Bob) reaching the node. 
If Alice reveals a linear form in this step, we use $\supa{t}\in \Rbb^n$ to denote the vector of the linear form; otherwise, we set $\supa{t}$ to be the all-zeroes vector. 
We define $\supb{t}$ similarly for Bob. Throughout the protocol, we will abbreviate $\supu{t} = \com(\supX{t})$ and $\supv{t} = \com(\supY{t})$ for Alice's and Bob's current center of mass respectively. 
\begin{enumerate}
	\item At the beginning, Alice receives an input $x\in\Rbb^n$ and Bob receives an input $y\in\Rbb^n$.
	\item We initialize $t\gets0$, $\supX{0},\supY{0}\gets\Rbb^n$, and $\supa{0},\supb{0}\gets0^{n}$. 
	\item For each phase $i=1,2,\ldots,d$: suppose we are starting the cleanup for a node at depth $i$ in the original protocol $\tilde\Ccal$ and suppose we are at a node of depth $t$ in the new protocol $\bar\Ccal$. If it is Alice's turn to speak in $\tilde{\Ccal}$:
	\begin{enumerate}
		\item \textbf{Orthogonalization by revealing the correlation with Bob's center of mass.}\\
		Alice begins by revealing the inner product of her input $x$ with Bob's current (signed) center of mass $\Lambda\odot \supv{t}$. Since in the previous steps, she has already revealed the inner product with Bob's previous centers of mass, for technical reasons, we will only have Alice announce the inner product with the component of $\Lambda\odot \supv{t}$ that is orthogonal to the previous directions along which Alice announced the inner product. More formally, let $\la^{(t+1)}$ be the component of $\Lambda\odot \supv{t}$ that is orthonormal to all previous directions $\supa{1},\dots, \supa{t}$, i.e.,
		$$\textstyle
		\la^{(t+1)}=\unit\pbra{ 
		\Lambda\odot \supv{t}
		- \sum_{\tau=1}^{t}	\abra{\Lambda \odot \supv{t},\supa{\tau}} \cdot \supa{\tau}}.$$
		
		Alice computes $\bar \lc^{(t+1)}\gets \abra{x,\la^{(t+1)}}$ and sends $\bar \lc^{(t+1)}$ to Bob. Set $\lb^{(t+1)}\gets 0^ n$. Increment $t$ by 1 and go to step (b). 
		\item \textbf{Original communication.} Alice sends the bit $\supcbar{t+1}$ that she was supposed to send in $\tilde\Ccal$ based on previous messages and the input $x$. Set $\la^{(t+1)},\lb^{(t+1)}\gets 0^n$. Increment $t$ by 1 and go to step (c). 
		\item \textbf{Cleanup steps.} While there exists some direction $a\in\mathbb{S}^{n-1}$ orthogonal to the previous directions (i.e., satisfying $\abra{a,\la^{(\tau)}}=0$ for all $\tau\in [t]$) such that $\X^{(t)}$ is \emph{not pairwise clean} in direction $a$, Alice computes $\bar \lc^{(t+1)}\gets\abra{x,a}$ and sends this to Bob. 
		Set $\la^{(t+1)}\gets a$ and $\lb^{(t+1)}\gets0^n$. Increment $t$ by 1. Repeat step (c) as long as $\X^{(t)}$ is not pairwise clean; otherwise increment $i$ by 1 and go back to the for-loop in step 3 which starts the new phase.
	\end{enumerate}
	If it is Bob's turn to speak, we define everything similarly with the role of $x,\la,\X,\V$ switched with $y,\lb,\Y,\U$.
	\item Finally at the end of the protocol, the value $\bar\Ccal(x,y)$ is determined based on all the previous communication and the corresponding output it defines in $\tilde\Ccal$.
\end{enumerate}

We note some basic properties that directly follow from the description. First we note that the steps 3(a), 3(b), and 3(c) always occur in sequence for each party and we refer to such a sequence of steps as a \emph{phase} for that party. Note that there are at most $d$ phases. If a new phase starts at time $t$, then the current rectangle $\supX{t} \times \supY{t}$ is pairwise clean for both parties by construction. Also, note that the non-zero vectors in the sequence $(\supa{t})_t$ (resp., $(\supb{t})_t$) form an orthonormal set. We also note that the Boolean communication in step 3(b) is solely determined by the original protocol and hence only depends on the previous Boolean messages.

Lastly, each phase has one 3(a) and 3(b) step, followed by potentially many 3(c) steps. However, the following claim shows that it is always finite.
\begin{claim}\label{clm:finite_steps_level_one}
Let $\ell$ be an arbitrary leaf of the protocol $\bar\Ccal$ and $D(\ell)$ be its depth. Then $D(\ell) \le 2n + 2d$. 
Moreover, along this path there are at most $2d$ many steps 3(a) and 3(b).
\end{claim}
\begin{proof}
	We count the number of communication steps separately:
	\begin{itemize}
		\item \textbf{Steps 3(a) and 3(b).} Steps 3(a) and 3(b)  occur once in every phase, thus at most $d$ times.
		\item \textbf{Step 3(c).} For Alice, each time she communicates at step 3(c) $a\in\Rbb^n$, the direction is orthogonal to all previous $\supa{t}$'s. Since the dimension of $\Rbb^n$ is $n$, this happens at most $n$ times. Similar argument works for Bob.
	\end{itemize}
	Thus in total we have at most $2n+2d$ steps.
\end{proof}
We will eventually show that the \emph{expected} depth of the protocol $\bar \Ccal$ is $O(d)$ when $\lX, \lY \sim \gamma_n$.

\subsection{Bounding the Expected Quadratic Variation}

Consider a random walk on the protocol tree generated by the new protocol $\bar \Ccal$ when the parties are given independent inputs $\lx, \ly \sim \gamma_n$.
Consider the corresponding level-one martingale process defined in \Cref{eqn:def-martingale}. Formally, at time $t$ the process is defined by
\[ \supZ{t}_1 = \ip{\supu{t}}{\eta \odot \supv{t}},\]
where we recall that $\supu{t} = \com(\supX{t})$ and $\supv{t} = \com(\supY{t})$ and $\eta \in \pmones$ is a fixed sign vector. 

The martingale process stops once it hits a leaf of the protocol $\bar \Ccal$. Let $\D$ denote the (stopping) time when this happens. Note that $\BE[\D]$ is exactly the expected depth of the protocol $\bar \Ccal$. Then, in light of \Cref{prop:fwt-to-qv}, to prove \Cref{thm:boolean_bound_level_one}, it suffices to prove the following.

\begin{lemma}\label{lem:qv-level-one}
$\E\sbra{\sum_{t=1}^{\D} \pbra{\Delta\supZ{t}_1}^2} = O(d)$.
\end{lemma}

We will prove this in two steps. We first show that the only change in the value of the martingale occurs during the orthogonalization step 3(a). 
This is because in each phase, Alice's change of center of mass in steps 3(b) and 3(c) is always orthogonal to $\eta \odot \supv{t}$ so they do not change the value of the martingale $\supZ{t}_1$ as discussed in \Cref{sec:overview}. 
Moreover, recalling \Cref{eqn:overview}, since Alice's node was pairwise clean just before Alice sent the message in step 3(a), the expected change $\BE\left[\left(\Delta\supZ{t+1}_1 \right)^2\right]$ can be bounded in terms of the squared norm of the change that occurred in $\supu{t}$ between the current round and the last round where Alice was in step 3(a). 
A similar argument works for Bob.

Formally, this is encapsulated by the next lemma for which we need some additional definition.
Let $(\supF{t})_t$ be the natural filtration induced by the random walk on the generalized protocol tree with respect to which $\supZ{t}_1$ is a Doob martingale and also $\supu{t}, \supv{t}$ form vector-valued martingales (recall \Cref{prop:vec-martingale}). 
Note that $\supF{t}$ fixes all the rectangles encountered during times $0,\ldots, t$ and thus for $\tau \le t$, the random variables $\supu{\tau},\supv{\tau},\supZ{\tau}_1$ are determined, in particular, they are $\supF{t}$-measurable. Recalling that $\lambda = 100$ is the cleanup parameter, we then have the following. Below we assume without any loss of generality that Alice speaks first and, in particular, we note that Alice speaks in step 3(a) for the first time at time zero. 

\begin{lemma}[Step Size]\label{lem:step_size_square_level_one}
Let $0=\btau_1 < \btau_2 < \cdots \le \D$ be a sequence of stopping times with $\btau_m$ being the index of the round where Alice speaks in step 3(a) for the $m^\text{th}$ time or $\D$ if there is no such round. 
Then, for any integer $m \ge 2$, 
$$
\BE\left[ \left(\Delta\supZ{\btau_m+1}_1\right)^2 ~\bigg|~ \supF{\btau_m}\right] \le \lambda \cdot \vabs{\supv{\btau_m} - \supv{\btau_{m-1}}}^2,
$$
and moreover, for any $t \in \N$, we have that
$$
\BE\left[ \left(\Delta\supZ{t+1}_1\right)^2 ~\bigg|~ \supF{t}, \btau_{m-1} < t <\btau_{m}, \text{Alice speaks at time }t \right] = 0.
$$
A similar statement also holds if Bob speaks where $\V$ is replaced by $\U$ and the sequence $(\btau_m)$ is replaced by $(\btau'_m)$ where $\btau'_m$ is the index of the round where Bob speaks in step 3(a) for the $m^\text{th}$ time or $\D$ if there is no such round. 
\end{lemma}
In particular, we see that the steps 3(b) and 3(c) do not contribute to the quadratic variation and only the steps 3(a) do. Also, since the first time Alice and Bob speak, they start in step 3(a), we also note that $\supu{\btau_1}$ and $\supv{\btau'_1}$ are their initial centers of mass which are both zero.  

We shall prove the above lemma in \Cref{sec:step_size_level_one} and continue with the bound on the quadratic variation here. Using \Cref{lem:step_size_square_level_one}, we have
\begin{align*}
\E\sbra{\sum_{t=1}^{\D} \pbra{\Delta\supZ{t}_1}^2} \le \lambda \cdot\E\sbra{\sum_{m\ge 2} \vabs{\V^{(\btau_m)}-\V^{(\btau_{m-1})}}^2+\vabs{\U^{(\btau'_m)}-\U^{(\btau'_{m-1})}}^2}.
\end{align*}
On the other hand, by the orthogonality of vector-valued martingale differences from \Cref{eqn:martingale-orthogonality-vec}, we have
\begin{align*}
	\E\sbra{\sum_{m \ge 2} \vabs{\V^{(\btau_m)}-\V^{(\btau_{m-1})}}^2} = \E\sbra{\vabs{\V^{(\D)}}^2}.
\end{align*}
A similar statement holds for $(\supu{t})_t$. Therefore, 
\begin{align}\label{eqn:qv-upper-bound}
	 \E\sbra{\sum_{t=1}^{\D} \pbra{\Delta\supZ{t}_1}^2} \le \lambda \cdot\pbra{\E\sbra{\frob{\U^{(\D)}}^2}+\E\sbra{\frob{\V^{(\D)}}^2}}.
\end{align}

We prove the following in \Cref{sec:expected_cleanup_depth} to upper bound the quantity on the right hand side above. Loosely speaking, by an application of level-one inequalities (see \Cref{thm:level_k_ineq}), the lemma below ultimately boils down to a bound on the expected number of cleanup steps. 

\begin{lemma}[Final Center of Mass]\label{lem:expected_norm_level_one}
$
\E\sbra{\vabs{\supu{\D}}^2+\vabs{\supv{\D}}^2} = O(d).
$
\end{lemma}

Since $\lambda = 100$, plugging in the bounds from the above into \Cref{eqn:qv-upper-bound} readily implies \Cref{lem:qv-level-one}. Together with \Cref{prop:fwt-to-qv}, this completes the proof of \Cref{thm:boolean_bound_level_one}.

\subsection[Bounds on Step Sizes]{Bounds on Step Sizes (Proof of \Cref{lem:step_size_square_level_one})}\label{sec:step_size_level_one}

Let us abbreviate $\btau = \btau_m$. Observe that
\begin{align}
\E\sbra{\pbra{\Delta\lZ^{(\btau+1)}_1}^2\mid \Fcal^{(\btau)}}&=\E\sbra{\abra{\U^{(\btau+1)}-\U^{(\btau)}, \Lambda\odot \V^{(\btau)}}^2\mid \Fcal^{(\btau)}}\notag\\
&=\E\sbra{ \abra{\U^{(\btau+1)}, \Lambda\odot \V^{(\btau)} } ^2-\abra{\U^{(\btau)},\Lambda\odot \V^{(\btau)}}^2\mid \Fcal^{(\btau)}},
\label{step_size_level_one_alpha_3}
\end{align}
where the second line is due to $(\supu{t})_t$ being a vector-valued martingale and thus $\E\sbra{\U^{(\btau+1)}\mid \Fcal^{(\btau)}}=\U^{(\btau)}$.  

We first consider the case that at time $\btau$ a new phase starts for Alice. By construction, this means that the current rectangle $\supX{\btau} \times \supY{\btau}$ determined by $\supF{\btau}$ is pairwise clean with parameter $\lambda$, and since Alice is in step 3(a) at the start of a new phase, $\supa{\btau+1}$ is chosen to be the (normalized) component of $\Lambda\odot \V^{(\btau)}$ that is orthogonal to previous directions $\supa{0}, \ldots, \supa{\btau}$. Let $\balpha^{(\btau+1)}:= \abra{\Lambda\odot \V^{(\btau)},\la^{(\btau+1)}}$ be the length of this component before normalization. Note that $\balpha^{(\btau+1)}$ is $\Fcal^{(\btau)}$-measurable (i.e., it is determined by $\Fcal^{(\btau)}$). 

We now claim that components of $\supu{\btau+1}$ and $\supu{\btau}$ are the same along any of the previous directions $\supa{0}, \ldots, \supa{\btau}$. So in \Cref{step_size_level_one_alpha_3}, they cancel out and the only relevant quantity is the component in the direction $\supa{\btau+1}$. 
This follows since, in all the previous steps $t \le \btau$, Alice has already fixed $\sabra{x,\la^{(t)}}$. 
This implies that for any $\supX{\btau}$ and $\supX{\btau+1}$ that are determined by $\supF{\btau+1}$, the inner product with all the previous $\la^{(0)}, \ldots, \la^{(\btau)}$ is fixed over the choice of $x$ from both rectangles. 
Formally, we have that for any $x\in \X^{(\btau)}$ and $x'\in \X^{(\btau+1)} $, it holds that $\sabra{x,\la^{(t)}}=\sabra{x',\la^{(t)}}$ for any $t \le \btau$. In particular, since $\U^{(\btau)}=\com(\X^{(\btau)})$ and $\U^{(\btau+1)}=\com(\X^{(\btau+1)})$ are the corresponding centers of mass, we have that
\begin{equation}\label{step_size_level_one}
\abra{\U^{(\btau+1)}, \la^{(t)}}=\abra{\U^{(\btau)},\la^{(t)}} \text{ for all } t\le \btau.
\end{equation}

This, together with \Cref{step_size_level_one_alpha_3} and recalling that $\balpha^{(\btau+1)}$ is determined by $\Fcal^{(\btau)}$, implies that 
\begin{align}\label{eqn:step_size_level_one_alpha_4}
 \E\sbra{\pbra{\Delta\lZ^{(\btau+1)}_1}^2\mid \Fcal^{(\btau)}} &=\pbra{\balpha^{(\btau+1)}}^2\cdot \E\sbra{ \abra{\U^{(\btau+1)},\la^{(\btau+1)} } ^2-\abra{\U^{(\btau)}, \la^{(\btau+1)}}^2\mid \Fcal^{(\btau)}}.
\end{align}

We now bound the term outside the expectation by the change in the center of mass $\supv{\cdot}$ and the term inside the expectation by the fact that the set is pairwise clean.

\paragraph*{Term Outside the Expectation.} 
Recall that $\supa{\btau+1}$ is chosen to be the (normalized) component of $\Lambda\odot \V^{(\btau)}$ that is orthogonal to the span of $\supa{0}, \ldots, \supa{\btau}$. Since $\Lambda\odot \V^{(\btau_{m-1})}$ is in the span of $\supa{1}, \ldots, \supa{\btau_{m-1}+1}$ and $\btau_{m-1} + 1 \le \btau=\btau_m$, it is orthogonal to $\la^{(\btau+1)}$. Hence,
	\[ 
	\balpha^{(\btau+1)} = \abra{\Lambda\odot \V^{(\btau)},\la^{(\btau+1)}}= \abra{\Lambda\odot\pbra{ \V^{(\btau)}-\V^{(\btau_{m-1})}},\la^{(\btau+1)}}.
	\]
Since $\la^{(\btau+1)}$ is a unit vector and each entry of $\Lambda$ is in $\binpm$, this implies that
\begin{equation}\label{step_size_level_one_alpha}
	\pbra{\balpha^{(\btau+1)}}^2\le \vabs{\V^{(\btau)}-\V^{(\btau_{m-1})}}^2.
	\end{equation}
	
\paragraph*{Term Inside the Expectation.}
Since $(\supu{\tau})$ is a vector-valued martingale with respect to $\supF{\tau}$, and $\supa{\tau+1}$ is $\supF{\tau}$-measurable (determined by $\supF{\tau}$), we have that
\begin{align*}
    \E\sbra{ \abra{\U^{(\btau+1)},\la^{(\btau+1)} } ^2-\abra{\U^{(\btau)}, \la^{(\btau+1)}}^2\mid \Fcal^{(\btau)}}
 = \E\sbra{ \abra{\supu{\tau+1} - \supu{\tau} ,\la^{(\btau+1)} } ^2\mid \supF{\tau}}.
\end{align*}

Since Alice is in step 3(a), her message fixes $\abra{x,\la^{(\btau+1)}}$ at time $\btau$ for every $x \in \supX{\btau+1}$. Thus,
\begin{align}\label{eqn:step_size_level_one_alpha_2}
\E\sbra{\abra{\U^{(\btau+1)} - \U^{(\btau)}, \la^{(\btau+1)}}^2 \mid \supF{\btau}}
&= \E\sbra{\abra{\E_{\lx\sim \gamma}\sbra{\lx\mid \lx\in \X^{(\btau+1)}} - \supu{\tau},\la^{(\btau+1)}}^2 \mid \supF{\btau}} 
\notag\\
&= \E\sbra{\E_{\lx\sim \gamma}\sbra{\abra{\lx-\supu{\btau},\la^{(\btau+1)}}^2\mid \lx\in \X^{(\btau+1)}} \mid \supF{\btau}} \notag \\
&= \E\sbra{\abra{\lx-\supu{\btau},\la^{(\btau+1)}}^2\mid \supF{\btau}},
\end{align}
where the last line follows from the tower property of conditional expectation.

Recall that $\supu{\btau} = \mu(\supX{\btau})$ is the center of mass. Moreover, the unit vector $\supa{\tau+1}$ is determined by $\supF{\tau}$ and also the conditional distribution of $\lx$ conditioned on $\supF{\tau}$ is that of $\lx \sim \gamma$ conditioned on $\lx \in \supX{\tau}$. Thus, using the fact that $\X^{(\btau)}$ is pairwise clean since Alice is in step 3(a), the right hand side in \Cref{eqn:step_size_level_one_alpha_2} is at most $\lambda$.

\paragraph*{Final Bound.} 
Substituting the above in \Cref{eqn:step_size_level_one_alpha_4}, we have 
\begin{align*} 
\E\sbra{\pbra{\Delta\supZ{\btau+1}_1}^2\mid \Fcal^{(\btau)}}
&\le \lambda\cdot \pbra{\balpha^{(\btau+1)}}^2 \le \lambda \cdot \vabs{\V^{(\btau)}-\V^{(\btau_{m-1})}}^2,
\end{align*}
where the second inequality follows from \Cref{step_size_level_one_alpha}. This completes the proof of the first statement.

For the moreover part, let us condition on the event $\btau_{m-1} < t <\btau_{m}$ where Alice speaks at time $t$. Note that such $t$ must all lie in the same phase of the protocol where Alice is the only one speaking. 
So, Bob's center of mass does not change from the time $\btau_{m-1}$ till $t$, i.e., $\V^{(t+1)}=\V^{(\btau_{m-1})}$. 
Thus we have $\Delta\supZ{t+1}_1=\abra{\U^{(t+1)}-\U^{(t)}, \Lambda\odot \V^{(\btau_{m-1})}}$. 
Analogous to \Cref{step_size_level_one},
the component of Alice's center of mass along the previous directions are fixed.
Thus $\abra{\U^{(t+1)}, \la^{(r)}}=\abra{\U^{(t)},\la^{(r)}}$ for all $r \le t$. Furthermore, by construction, $\Lambda \odot \V^{(\btau_{m-1})}$ lies in the linear subspace spanned by $\la^{(0)},\ldots,\la^{(\btau_{m-1} +1)}$. Therefore, since $\btau_{m-1}+1\le t$, it follows that $\Delta\supZ{t+1}_1=0$. 

\subsection[Expected Norm of Final Center of Mass]{Expected Norm of Final Center of Mass (Proof of \Cref{lem:expected_norm_level_one})}
\label{sec:expected_cleanup_depth}

Let $\BH_A = \BH_A^{(\D)}$ be the (random) linear subspace spanned by the vectors $\supa{0}, \ldots, \supa{\D}$ and similarly, let $\BH_B = \BH_B^{(\D)}$ be the linear subspace spanned by the vectors $\supb{0}, \ldots, \supb{\D}$. 
For any linear subspace $V$ of $\Rbb^n$, we denote by $\bPi_V$ and $\bPi_{V^\bot}$ the projectors on the subspace $V$ and its orthogonal complement $V^\bot$ respectively. 
Then, we have that
\[
\vabs{\supu{\D}}^2 = \vabs{\bPi_{H_A} \supu{\D}}^2 + \vabs{\bPi_{H_A^\bot} \supu{\D}}^2 
\text{ and } 
\vabs{\supv{\D}}^2 = \vabs{\bPi_{H_B} \supv{\D}}^2 + \vabs{\bPi_{H_B^\bot} \supv{\D}}^2.
\]

Note that the non-zero vectors in $(\supa{t})_t$ and $(\supb{t})_t$ form an orthonormal basis for the subspaces $\BH_A$ and $\BH_B$ respectively. Moreover, for each $t \le \D$, the inner product $\ip{x}{\supa{t}}$ is fixed for every $x \in \supX{\D}$ and the inner product $\ip{y}{\supb{t}}$ is also fixed for every $y \in \supY{\D}$ where $\supX{\D} \times \supY{\D}$ is the current rectangle determined by $\supF{\D}$. In particular, since $\supu{\D}$ is the center of mass of $\supX{\D}$, this implies that 
\begin{align*}
\vabs{\bPi_{H_A} \supu{\D}}^2 = \sum_{t=1}^{\D} \abra{\supu{\D},\la^{(t)}}^2 &= \sum_{t=1}^{\D} \pbra{\E_{\lx\sim \gamma} \sbra{\abra{\lx,\la^{(t)}}\mid \lx\in \X^{(\D)}}}^2\\
& = \sum_{t=1}^{\D} \E_{\lx\sim \gamma} \sbra{\abra{\lx,\la^{(t)}}^2\mid \lx\in \X^{(\D)}},
\end{align*}
where the second line follows from the inner product being fixed in $\X^{(\D)}$. 
Therefore, we have 
\[ 
\vabs{\supu{\D}}^2 = \underbrace{\sum_{t=1}^{\D} {\E_{\lx\sim \gamma} \sbra{\abra{\lx,\la^{(t)}}^2\mid \lx\in \X^{(\D)}}}}_{\bP_A} + \underbrace{\vabs{\bPi_{H_A^\bot} \supu{\D}}^2}_{\bQ_A}.
\]
In an analogous fashion, 
\[ 
\vabs{\supv{\D}}^2 = \underbrace{\sum_{t=1}^{\D} {\E_{\ly\sim \gamma} \sbra{\abra{\ly,\lb^{(t)}}^2\mid \ly\in \Y^{(\D)}}}}_{\bP_B} + \underbrace{\vabs{\bPi_{H_B^\bot} \supv{\D}}^2}_{\bQ_B}.
\]

We next show that both $\E[\bP_A+\bP_B]$ and $\E[\Q_A+\Q_B]$ are at most $O(d)$. 
The former follows from stopping time and concentration arguments laid out in the overview that there cannot be too many orthogonal directions where $\BE\sbra{\ip{\lx}{\supa{t}}^2}$ is large. 
The latter follows from an application of level-one inequalities.

We will bound the norm of the projection on the subspaces $\BH_A$ and $\BH_B$, which corresponds to the quantity $\BE[\bP_A+\bP_B]$, in \Cref{sec:level_one_inside_subspace} and bound the norm of the projection on the orthogonal subspaces $\BH_A^\bot$ and $\BH_B^\bot$,  which corresponds to the quantity $\BE[\Q_A+\Q_B]$, in \Cref{sec:level_one_inside_complement_subspace}.

\subsubsection{Projection on the Subspaces \texorpdfstring{$\BH_A$}{H\textunderscore A} and \texorpdfstring{$\BH_B$}{H\textunderscore B}}\label{sec:level_one_inside_subspace}

We shall show that the expected norm of the final center of mass when projected on the subspaces $\BH_A$ and $\BH_B$ is 
\[ \E[\bP_A+\bP_B] = O(d).\]

Towards this end, define the random variable $\K_t = \K_t(\lx,\ly) =\abra{\lx,\la^{(t)}}^2+ \abra{\ly,\lb^{(t)}}^2$ for each $t \in \Nbb$. 
Note that the vectors $\la^{(t)}$'s are being chosen adaptively depending on the previous inner products $\ip{\lx}{\la^{(\tau)}}$ for $\tau < t$, as well as the Boolean communication bits from step 3(b), thus they are functions of $\lx$ and $\ly$ as well here. Observe that
\[
\E\sbra{\bP_A+\bP_B}= \E\sbra{\sum_{t=1}^{\D} \E\sbra{\K_t \mid \supF{\D}}}= \E_{\lx,\ly\sim \gamma} \sbra{ \sum_{t=1}^{\D} \K_t}.
\]

We now divide the time sequence into successive intervals of different lengths $r\cdot 4d$ for $r=1,2, \ldots$.
Then we bound the expected sum of $\K_t$ within each time interval by $O(r d)$. 
We further argue that the probability that the stopping time $\D$ lies in the $r$-th interval is at most $2\cdot 2^{-r}$. In particular, for $r\in\Nbb$, letting interval 
$I_r=\cbra{\binom{r}{2}\cdot 4d+1,\ldots,\binom{r+1}{2}\cdot 4d}$,
which is of length $4dr$, we show the following.
\begin{claim}\label{lem:depth_tail_bound_level_one}
For any $r\in \Nbb$, we have 
\[ \E_{\lx,\ly\sim \gamma}
\sbra{\sum_{t\in I_r} \K_t \mid \D>\binom{r}{2}\cdot 4d}
\le 20d r
+4 \ln\pbra{\dfrac{1}{\Pr\sbra{\D>\binom{r}{2}\cdot 4d}}}.\]
\end{claim}

We shall prove the above claim later since it is the most involved part of the proof. The previous claim readily implies the following probability bounds.
\begin{claim}\label{lem:depth_tail_bound_level_one_part_two}
For any $r\in \Nbb$, we have $\Pr\sbra{\D >\binom{r}{2}\cdot 4d}\le 2\cdot 2^{-r}$.
\end{claim}

\begin{proof}[Proof of \Cref{lem:depth_tail_bound_level_one_part_two}] 
We bound $\Pr\sbra{\D > \binom{r}{2}\cdot 4d}$ by induction on $r$. The claim trivially holds for $r=1$. 

Now we proceed to analyze the event $\D\ge\tbinom{r+1}2\cdot4d$.
Observe that \Cref{clm:finite_steps_level_one} implies that there are at most $2d$ many step 3(a) and 3(b) throughout the protocol.
Thus if the event above occurs, there are at least $4dr-2d\ge 2dr$ many time steps $t \in I_r$ where the process is in step 3(c).

By the definition of the cleanup step, if $X \times Y$ is a rectangle determined\footnote{It suffices to consider such events since we have a product measure on $\supX{t} \times \supY{t}$ conditioned on $\supF{t}$ and $\D$ is a stopping time and is $\supF{t}$-measurable (i.e., determined by the randomness in $\supF{t}$).} by $\supF{t-1} \cap \{\D > \binom{r}{2}\cdot 4d\}$ where the process is in step 3(c) and Alice speaks, then
\[ 
\BE_{\lx \sim \gamma}\sbra{\K_t \mid (\lx,\ly) \in X \times Y} = \BE_{\lx \sim \gamma}\sbra{\ip{\lx}{\supa{t}}^2 \mid \lx\in X} \ge \BE_{\lx \sim \gamma}\sbra{\ip{\lx - \com(X)}{\supa{t}}^2 \mid \lx\in X} \ge \lambda, 
\]
where $\lambda=100$ is the cleanup parameter and $\com(X) = \BE_{\lx \sim \gamma}[\lx \midd \lx \in X]$ is the center of mass. This is because $\supa{t}$ is chosen to be a unit vector in a direction where the current set (conditioned on the history) is not pairwise clean. 
A similar statement holds if Bob speaks in step 3(c) for the random variable $\ip{\ly}{\supb{t}}^2$ where $\ly$ is sampled from $\gamma$ conditioned on $Y$. 

By the tower property of conditional expectation, the above implies that
\[
100\cdot2dr\cdot \Pr\sbra{\D > {\textstyle\binom{r+1}{2}}\cdot 4d \mid \D > {\textstyle\binom{r}{2}}\cdot 4d} \le \E\sbra{\sum_{t\in I_r} \K_t \mid \D > {\textstyle\binom{r}{2}}\cdot 4d}.
\]
Recall that \Cref{lem:depth_tail_bound_level_one} implies that the right hand side is at most $\le 20d r + 4 \ln\pbra{\frac1{\Pr[\D>\tbinom{r}{2} \cdot 4d]}}$.
We consider two cases:
\begin{enumerate}
\item[(i)] if $\Pr[\D>\binom{r}{2} \cdot 4d] \le 2^{-r}$, then clearly $\Pr[\D>\binom{r+1}{2} \cdot 4d] \le 2^{-r}$ as well as required;
\item[(ii)] otherwise $\Pr[\D>\binom{r}{2} \cdot 4d] \ge 2^{-r}$ and $20d r + 4\pbra{\frac1{\Pr[\D>\tbinom{r}{2} \cdot 4d]}} \le 20dr + 4r$, then it follows that 
\[ 
\Pr\sbra{\D > \textstyle\binom{r+1}{2}\cdot 4d \mid \D > \textstyle\binom{r}{2}\cdot 4d}\le 1/2,
\]
and by induction this implies that $\Pr\sbra{\D > \textstyle\binom{r+1}{2}\cdot 4d} \le 1/2\cdot\Pr\sbra{\D > \textstyle\binom{r}{2}\cdot 4d}\le2^{-r}$.\qedhere
\end{enumerate} 
\end{proof}

These claims imply that 
\begin{align*} 
\E[\bP_A+\bP_B] & \le \E\sbra{\sum_{r=0}^\infty 1\sbra{\D>{\textstyle\binom{r}{2}}\cdot 4d}\cdot \sum_{t\in I_r} \K_t}
\\&= \sum_{r=0}^\infty \Pr[ \D> {\textstyle\binom{r}{2}}\cdot 4d] \cdot\E\sbra{\sum_{t\in I_r} \K_t\mid \D > {\textstyle\binom{r}{2}}\cdot 4d} \\
&\le \sum_{r=0}^\infty\pbra{2^{1-r}\cdot O(r d)  +   4 \cdot\Pr[ \D> {\textstyle\binom{r}{2}}\cdot 4d] \cdot \ln\pbra{\tfrac{1}{ \Pr\sbra{\D> {\textstyle\binom{r}{2}}\cdot 4d}}}}\\
&\le \sum_{r=0}^\infty\pbra{2^{1-r}\cdot O(r d)  +  O\pbra{(r+1)2^{-r}}}\le O(d),
\end{align*}
where the last line uses the fact that $x\ln(1/x)\le O((r+1)2^{-r})$ for $0\le x\le2\cdot2^{-r}$ and $r\in\N$.
This proves the desired bound on $\E[\bP_A+\bP_B]$ assuming \Cref{lem:depth_tail_bound_level_one}  which we prove next.

\begin{proof}[Proof of \Cref{lem:depth_tail_bound_level_one}] 
To prove the claim, we need to analyze the expectation of $\sum_{t \in I_r} \K_t$ under $\lx, \ly$ sampled from $\gamma$ conditioned on the event $\D \ge \binom{r}{2} \cdot 4d$. 

We first describe an equivalent way of sampling from this distribution which will be easier for analysis. 
First, we recall that the definition of the cleanup protocol implies that the Boolean communication in $\bar \Ccal$ is solely determined by the previous Boolean communication, since it is specified by the original protocol $\tilde{\Ccal}$ (and thus $\Ccal$) before the cleanup. 

Let us fix any Boolean string $c\in\{0,1\}^*$ that is a valid Boolean transcript in the original communication protocol $\tilde \Ccal$.
This defines a rectangle $X_c\times Y_c\subseteq\Rbb^n\times\Rbb^n$ consisting of all pairs of inputs to Alice and Bob that result in the Boolean transcript $c$ in $\tilde\Ccal$.
If we sample $\lx,\ly\sim \gamma$ conditioned on $\D>\binom{r}{2}\cdot 4d$ and output the unique $(\X_c,\Y_c)$ such that $(\lx,\ly)\in \X_c\times \Y_c$, 
we obtain a distribution on rectangles. We use $\gamma(X_c\times Y_c\,|\,\D >\binom{r}{2}\cdot 4d)$ 
to denote the probability of obtaining $X_c\times Y_c$ by this sampling process so that $\sum_c \gamma(X_c\times Y_c\,|\,\D >\binom{r}{2}\cdot 4d)=1$. 

Now consider the following two-stage sampling process. First, we sample a rectangle $X_c \times Y_c$ according to the above distribution, and then we sample  the inputs $\lx, \ly$ sampled from $\gamma_n$ conditioned on the event that $\{(\lx,\ly)\in X_c \times Y_c\} \wedge \{\D> \binom{r}{2}\cdot 4d\}$. We shall show the following claim for any rectangle $X_c \times Y_c$ that could be sampled in the first step. 

\begin{claim}\label{eq:depth_tail_bound_level_one_eq1}
$\E_{\lx,\ly\sim \gamma}\sbra{\sum_{t\in I_r} \K_t \mid \D > 4d\tbinom{r}{2},(\lx,\ly)\in X_c\times Y_c } \le 12dr + 4\ln
\pbra{\tfrac{1}{\Pr[\D > 4d\tbinom{r}{2},(\lx,\ly)\in X_c\times Y_c]}}$.
\end{claim}

Assuming the above, and taking an expectation over $X_c\times Y_c$ drawn with probability $\gamma(X_c\times Y_c\,|\,\D >\binom{r}{2}\cdot 4d)$, we immediately obtain \Cref{lem:depth_tail_bound_level_one}:
\begin{align*} 
&\E_{\lx,\ly\sim \gamma}\sbra{\sum_{t\in I_r} \K_t \mid \D> {\textstyle\binom{r}{2}}\cdot 4d}\\
&\le 12dr + 4\cdot\sum_{\substack{c\in\{0,1\}^*,|c|\le d}} \gamma(X_c\times Y_c|\D >{\textstyle\binom{r}{2}}\cdot 4d)\cdot 
\pbra{\ln\pbra{\tfrac{1}{\gamma(X_c\times Y_c|\D >\binom{r}{2}\cdot 4d)}}+
\ln\pbra{\tfrac{1}{\Pr[\D >\binom{r}{2}\cdot 4d]}}}\\
&\le 12dr +  4\cdot \ln(3^d) + 4\cdot \ln\pbra{\tfrac{1}{\Pr[\D >\binom{r}{2}\cdot 4d]}} 
\tag{by concavity of $\ln(\cdot)$}
\\
&\le 20dr + 4\cdot \ln\pbra{\tfrac{1}{\Pr[\D >\binom{r}{2}\cdot  4d]}}.
\tag*{\qedhere}
\end{align*}
\end{proof}

To complete the proof, we now prove \Cref{eq:depth_tail_bound_level_one_eq1}.

\begin{proof}[Proof of \cref{eq:depth_tail_bound_level_one_eq1}]
Fix any $c$ such that $\gamma(X_c\times Y_c\,|\,\D >\binom{r}{2}\cdot 4d)>0$. 
We will bound the expectation of the quantity $\sum_{t\in I_r} \K_t = \sum_{t\in I_r} \abra{\lx,\la^{(t)}}^2 +\abra{\ly,\lb^{(t)}}^2$ where $\lx, \ly$ are sampled from $\gamma_n$ conditioned on the event that $\{(\lx,\ly)\in X_c \times Y_c\} \wedge \{\D > \binom{r}{2}\cdot 4d\}$.
Note that $\supa{t}, \supb{t},\D$ are functions of the previous messages of the protocol and hence also the inputs $\lx, \ly$. Once we condition on the above event, the Boolean communication is also fixed to be $c$.

To analyze the above conditioning, we first do a thought experiment and consider a different protocol that takes standard Gaussian inputs (without any conditioning) and show a tail bound for the random variable $\sum_{t \in I_r} \K_t$ for this new protocol. In the last step, we will use it to compute the expectation we ultimately want.

\paragraph*{Protocol $\Ccal_c$.} 
The protocol $\Ccal_c$ always communicates according to the fixed transcript $c$ in a Boolean communication step and otherwise according to the cleanup protocol $\bar\Ccal$ on any input $x,y$. Consider the random walk on this new protocol tree where the inputs $\lx, \ly \sim \gamma$ (without any conditioning). Let $(\Gcal^{(t)})_t$ be the associated filtration of the new protocol $\Ccal_c$ which can be identified with the collection of all partial transcripts till time $t$. Note that the vectors $\supa{t}$ and $\supb{t}$ in this new protocol are determined only by the previous real communication since the Boolean communication is fixed to $c$. This also implies that the vectors $\supa{t}$ and $\supb{t}$ form a predictable sequence with respect to the filtration $(\Gcal^{(t)})_t$. Moreover, by the definition of the protocol the next non-zero vector $\supa{\cdot}$ is chosen to be a unit vector orthogonal to the previously chosen $\supa{\cdot}$'s and the same holds for the vectors $\supb{\cdot}$.

We denote by $\K_t^{(c)}$ the random variable that captures $\K_t$ for the protocol $\Ccal_c$, i.e., $\K_t^{(c)} = \abra{\lx,\la^{(t)}}^2 +\abra{\ly,\lb^{(t)}}^2$ for $\lx, \ly \sim \gamma$ and $\la^{(t)}, \lb^{(t)}$ defined by the protocol $\Ccal_c$.
Observe that if $(\lx, \ly) \in X_c \times Y_c$ then $\K_t^{(c)} = \K_t$.

Consider the behavior of the protocol $\Ccal_c$ at some fixed time $t$. The nice thing about the protocol $\Ccal_c$ is that conditioned on all previous real messages for $\tau < t$, both $\lx$ and $\ly$ are standard Gaussian distributions on an affine subspace of $\R^n$ (defined by the previous messages).
Then, at time $t$, since $\supa{t}$ is orthogonal to the directions used in all previous real messages, it follows that the distribution of $\abra{\lx,\la^{(t)}}$ conditioned on any event in $\Gcal^{(t-1)}$ is an independent standard Gaussian for every $t$ if $\supa{t}$ is non-zero. The same holds for $\abra{\ly,\lb^{(t)}}$ as well. This last fact uses that the projection of a multi-variate standard Gaussian $\gamma_n$ in orthonormal directions yields independent real-valued standard Gaussians.

This implies that each new $\abra{\lx,\la^{(t)}}^2$ and $\abra{\ly,\lb^{(t)}}^2$ is an independent chi-squared random variable conditioned on the history (up to depth $\binom r2\cdot4d$) of the random walk. Therefore, \Cref{thm:chi_squared_concentration} implies that
\[ 
\Pr_{\lx,\ly\sim \gamma}\sbra{ \sum_{t\in I_r} \K_t^{(c)}(\lx,\ly) \ge 2|I_r|+ s \mid \Gcal^{(\binom{r}{2}\cdot 4d)}} \le e^{-s/4}. 
\]
Since $|I_r|\le 4dr$, we have 
$\Pr_{\lx,\ly\sim \gamma}\sbra{ \sum_{t\in I_r} \K^{(c)}_t(\lx,\ly) \ge 8dr+ s } \le e^{-s/4}.$

\paragraph*{Computing the Original Expectation.} 
Let us compare the probability of the above tail event in the original protocol $\bar\Ccal$ 
where inputs $\lx, \ly$ are sampled from  $\gamma$ conditioned on the event that $\{(\lx,\ly)\in X_c \times Y_c\} \wedge \{\D > \binom{r}{2}\cdot 4d\}$. 
We can write
\begin{align}\label{eq:tail_of_kt}
&\phantom{\le}
\Pr_{(\lx,\ly)\sim \gamma}\sbra{ \sum_{t\in I_r} \K_t(\lx,\ly)\ge 8dr + s \mid \D> {\textstyle\binom{r}{2}}\cdot 4d, (\lx,\ly) \in X_c\times Y_c}\\
&= \frac{\Pr_{\lx,\ly\sim \gamma}\sbra{ \sum_{t\in I_r} \K_t(\lx,\ly)\ge 8dr+ s,(\lx,\ly) \in X_c\times Y_c, \D > \binom{r}{2}\cdot 4d}}{ \Pr_{\lx,\ly\sim \gamma}\sbra{ (\lx,\ly)\in X_c\times Y_c, \D >\binom{r}{2}\cdot 4d}}.
\notag
\end{align}
We then bound the numerator by
\begin{align*}
&\phantom{\le}
\Pr_{\lx,\ly\sim \gamma}\sbra{ \sum_{t\in I_r} \K_t(\lx,\ly)\ge 8dr+ s, (\lx,\ly) \in X_c\times Y_c, \D > {\textstyle\binom{r}{2}}\cdot 4d}\\
&= \Pr_{\lx,\ly\sim \gamma}\sbra{ \sum_{t\in I_r} \K_t^{(c)}(\lx,\ly)\ge 8dr+ s, (\lx,\ly) \in X_c\times Y_c, \D > {\textstyle\binom{r}{2}}\cdot 4d}
\tag{if $(\lx,\ly)\in X_c \times Y_c$ then $\K_t^{(c)} = \K_t$}
\\
&\le  \Pr_{\lx,\ly\sim \gamma}\sbra{ \sum_{t\in I_r} \K_t^{(c)}(\lx,\ly)\ge 8dr+ s}  \le e^{-s/4}.
\end{align*}

Note that the inequality gives us an exponential tail on \Cref{eq:tail_of_kt}:
$$
\Cref{eq:tail_of_kt}
\le e^{-s/4}\cdot\pbra{\Pr_{\lx,\ly\sim \gamma}\sbra{ (\lx,\ly)\in X_c\times Y_c, \D >\binom{r}{2}\cdot 4d}}^{-1}.
$$
We can now integrate the above inequality to get an upper bound on the expected value of $\sum_{t \in I_r} \K_t$ under the distribution of interest. In particular, since for any non-negative random variable $\lw$, the following holds for any parameter $\alpha\ge0$:
\[ 
\BE[\lw] = \int_0^{+\infty} \Pr[\lw \ge z]\sd{z} 
\le \alpha + \int_\alpha^{+\infty} \Pr[\lw \ge z]\sd{z}
=\alpha + \int_0^{+\infty} \Pr[\lw \ge\alpha+z]\sd{z},
\]
we derive the following by taking $\alpha = 8dr+4\ln\pbra{\frac{1}{\Pr_{\lx,\ly\sim \gamma}\sbra{(\lx,\ly)\in X_c\times Y_c, \D >\binom{r}{2}\cdot 4d}}}$:
\begin{align*}
&\E_{(\lx,\ly)\sim \gamma}\sbra{ \sum_{i\in I_r} \K_t(\lx,\ly) \mid \D> {\textstyle\binom{r}{2}}\cdot 4d, (\lx,\ly)\in X_c\times Y_c } \\
&\qquad\le\alpha+\int_0^{+\infty}e^{-z/4}\sd z
=\alpha + 4\\
&\qquad \le 12dr + 4\ln\pbra{\dfrac{1}{\Pr_{\lx,\ly\sim \gamma}\sbra{ (\lx,\ly)\in X_c\times Y_c, \D >\binom{r}{2}\cdot 4d}}}.
\end{align*}
This completes the proof of \Cref{eq:depth_tail_bound_level_one_eq1}.
\end{proof}

\subsubsection{Projection on the Orthogonal Subspaces \texorpdfstring{$\BH_A^\bot$}{of H\textunderscore A} and \texorpdfstring{$\BH_B^\bot$}{of H\textunderscore B}}\label{sec:level_one_inside_complement_subspace}

We shall show that the expected norm of the final center of mass when projected on the subspaces $\BH_A^\bot$ and $\BH_B^\bot$ is 
\[ \E[\Q_A+\Q_B] = O(d).\]

Recall that $\bQ_{A} = \vabs{\bPi_{\BH_A^\bot} \supu{\D}}^2$ where $\BH_A$ is the (random) linear subspace spanned by the orthonormal set of vectors $\supa{0}, \ldots, \supa{\D}$ and $\BH_A^\bot$ its orthogonal complement. 
Moreover, the vectors $\supa{t}$ are determined by the previous Boolean and real communication. A similar statement holds for $\bQ_B$ and the vectors $\supb{t}$ as well.

The proof will follow in two steps. We will first show that one can bound the norm of the projection $\bPi_{\BH_A^\bot} \supu{d}$, which turns out to be the Gaussian center of mass of a set that lives in the subspace $\BH_A^\bot$, in terms of the logarithm of the inverse relative measure with respect to the subspace. Note that the Gaussian measure here is the Gaussian measure $\gamma_{\BH_A^\bot}$ on the subspace $\BH_A^\bot$. The case for $\bPi_{\BH_B^\bot} \supu{d}$ will be similar. The second step will use information theory-esque convexity argument to show that on average the logarithm of the inverse relative measure is small.

For the first part, we observe that if we sample $\lx,\ly \sim \gamma$ and take a random walk on this protocol tree, we obtain a probability measure over transcripts which includes both real and Boolean values. Recall that the Boolean transcript is determined by the original protocol and only depends on the previous Boolean communication and the real transcript is sandwiched between the Boolean communication. 
Let $\bell = (\bc, \br)$ denote the random variable representing the full transcript of the generalized protocol where $\bc$ is the Boolean communication and $\br$ is the real communication. For any given transcript $\bell$, let $\X_{\bell} \times \Y_{\bell}$ denote the corresponding rectangle consists of inputs reaching the leaf, and let $\X_{\bc}\times \Y_{\bc}$ (for $\X_{\bc},\Y_{\bc} \subseteq\Rbb^n$) denote the rectangle consisting of all pairs of inputs to Alice and Bob that result in the Boolean transcript $\bc$. Note that the real communication $\br$ together with $\bc$ fixes the subspaces $\BH_A$ and $\BH_B$ and particular affine shifts $\bs_A$ and $\bs_B$ of those subspaces depending on the value of the inner products determined by the full transcript. In particular, the rectangle $\X_{\bell} \times \Y_{\bell}$ consistent with the full transcript $\bell = (\bc,\br)$ is given by $\X_{\bell} = \X_{\bc} \cap (\BH_A + \bs_A)$ and $\Y_{\bell} = \Y_{\bc} \cap (\BH_B + \bs_B)$, i.e., taking (random) affine slices of the original sets.

Note that $\supu{\D}$ and $\supv{\D}$ are distributed as the center of masses of the final rectangle $\X_{\bell} \times \Y_{\bell}$, and thus is suffices to look at the rectangles for the rest of the argument. Since $\X_{\bell}$ (resp., $\Y_{\bell}$) lies in some affine shift of $\BH_A^\bot$ (resp., $\BH_B^\bot$), defining the relative center of mass for a set $A$ that lives in the ambient linear subspace $V$, as $\com_V(A) = \BE_{\lx \sim \gamma_V}[\lx \midd \lx \in A]$ where the Gaussian measure $\gamma_V$ is on the subspace $V$, it follows that
\begin{align*}
	 \E\sbra{\bQ_A +\bQ_B} &= \BE\left[\vabs{\bPi_{\BH_A^\bot} \supu{\D}}^2 + \vabs{\bPi_{\BH_A^\bot} \supu{\D}}^2\right] =  \BE_{\bell}\left[\|\com_{\BH_A^\perp}(\bPi_{\BH_A^\bot}\X_{\bell})\|^2 + \|\com_{\BH_B^\perp}(\bPi_{\BH_B^\bot}\Y_{\bell})\|^2\right].
\end{align*}

Recalling that $\gamma_\rel$ is the Gaussian measure of a set relative to its ambient space, we will show: 

\begin{claim}\label{eqn:relative-measure}
   $\|\com_{\BH_A^\perp}(\bPi_{\BH_A^\bot}\X_{\bell})\|^2 \le 2e^2 \ln\pbra{\dfrac{e}{\gamma_\rel\pbra{\X_{\bell}}}}$ and $\|\com_{\BH_B^\perp}(\bPi_{\BH_B^\bot}\Y_{\bell})\|^2 \le 2e^2 \ln\pbra{\dfrac{e}{\gamma_\rel\pbra{\Y_{\bell}}}}$.
\end{claim}
Note that we can ignore the case when $\gamma_\rel(\X_{\bell})$ is zero above, since we will eventually take an expectation over $\bell$ and almost surely this measure is non-zero.  

Using the previous claim, 
\begin{align*}
	 \E\sbra{\bQ_A +\bQ_B} &= \BE\left[\vabs{\bPi_{\BH_A^\bot} \supu{\D}}^2 + \vabs{\bPi_{\BH_A^\bot} \supu{\D}}^2\right] \le2e^2 \cdot \E_{\bell}\sbra{\ln\pbra{\frac{e}{\gamma_\rel\pbra{{\X_{\bell} \times \Y_{\bell}}}}}}.
\end{align*}

For the second step of the proof, we show the next claim which relies on convexity arguments to bound  the right hand side above by $O(d)$. This is similar in spirit to chain-style arguments from information theory.

\begin{claim}\label{claim:convexity}
$\E_{\bell}\sbra{\ln\pbra{\dfrac{e}{\gamma_\rel\pbra{{\X_{\bell} \times \Y_{\bell}}}}}} = O(d)$.
\end{claim}

This gives us the final bound $\E\sbra{\bQ_A +\bQ_B} = O(d)$ assuming the claims which we now prove.

\begin{proof}[Proof of \Cref{eqn:relative-measure}]
    
We can bound the norm of the above projection by an application of the Gaussian level-one inequality (\Cref{thm:level_k_ineq}), which, by rotational symmetry, implies that if $A$ is a subset of a linear subspace $V$ with non-zero measure, then 
\begin{align}\label{eqn:level-one-inequality}
    \|\com_V(A)\|^2 \le 2e^2 \ln\left(\frac{e}{\gamma_V(A)}\right),
\end{align}
where recall that $\com_V(A) = \BE_{\lx \sim \gamma_V}[\lx \midd \lx \in A]$ is the center of mass with respect to the Gaussian measure $\gamma_V$ on the subspace $V$.

If we run the generalized protocol on $\lx, \ly \sim \gamma$ and condition on getting the full transcript $\bell$, the conditional probability measure on $\bPi_{\BH_A^\bot}\lx$ is that of the Gaussian measure $\gamma_{\BH_A^\bot}$ conditioned on $\lx \in \X_{\bell} - \bs_A$ and $\bPi_{\BH_A^\bot}\ly$ is that of the Gaussian measure $\gamma_{\BH_B^\bot}$ conditioned on $\ly \in \Y_{\bell} - \bs_B$ and they are independent. 
This follows from the fact that so far the parties have fixed inner products along a basis for the orthogonal subspaces $\BH_A$ and $\BH_B$ and the fact the projection of a standard Gaussian on orthogonal subspaces are independent.

Thus, applying \Cref{eqn:level-one-inequality}, we have 
\begin{align*}
    \|\com_{\BH_A^\bot}(\bPi_{\BH_A^\bot}\X_{\bell})\|^2 \le 2e^2 \ln\left(\frac{e}{\gamma_{\BH_A^\bot}(\X_{\bell}-\bs_A)}\right) = 2e^2 \ln\left(\frac{e}{\gamma_\rel(\X_{\bell})}\right),
\end{align*}
where the last line follows since $\BH_A+\bs_A$ is the ambient space for $\X_{\bell}$ (this holds almost surely) and $\gamma_\rel(S) = \gamma_V(S-t)$ if $V+t$ is the ambient space of $S$. A similar argument proves the bound on $\|\com_{\BH_B^\bot}(\bPi_{\BH_B^\bot}\Y_{\bell})\|^2$.
\end{proof}

\begin{proof}[Proof of \Cref{claim:convexity}]
    For this claim, it will be convenient to consider a different generalized protocol $\Ccal'$ that generates the same distribution on the leaves $\bell$. In particular, since the Boolean messages in the generalized protocol $\bar\Ccal$ only depend on the previous Boolean messages, one can first send all the Boolean messages $\bc$, and then send all the real messages $\br$ choosing them according to the protocol $\bar\Ccal$ depending on the previous real messages and the (partial) Boolean transcript. Note that the protocol $\Ccal'$ generates the same distribution on the leaves $\bell$ when the inputs $\lx, \ly \sim \gamma_n$. In particular, the real communication only partitions \footnote{We remark that this protocol $\Ccal'$ suffices for proving this claim since we are looking only at the leaves. However, unlike \Cref{lem:step_size_square_level_one}, directly bounding the expected quadratic variation of the martingale corresponding to the protocol $\Ccal'$ is difficult.} each rectangles $X_c \times Y_c$ that corresponds to the Boolean transcript $c$ into affine slices.

    For rest of the claim, we now work with the protocol $\Ccal'$ where the Boolean communication happens first. To prove the claim, we condition on a Boolean transcript $\bc=c$ and by induction show that 
    \begin{align}\label{eqn:rectangle-measure}
          \BE_{\br}\sbra{\ln\pbra{\dfrac{e}{\gamma_\rel(\X_{(c,\br)} \times \Y_{(c,\br)})}} \mid \bc=c} \le \ln\pbra{\dfrac{e}{\gamma_\rel(X_c \times Y_c)}},     
    \end{align}
     where $(c, r)$ is the full transcript and  $X_c \times Y_c$ is the rectangle containing all the inputs such that Boolean transcript is $c$. Note that $\gamma_\rel(X_c \times Y_c)$ is the probability of obtaining the Boolean transcript $c$ since the ambient space of $X_c$ and $Y_c$ is $\Rbb^n$.

    Then, taking expectation over the Boolean transcript $c$,
    \begin{align*}
         \BE_{\bell}\sbra{\ln\pbra{\dfrac{e}{\gamma_\rel(\X_{\bell} \times \Y_{\bell})}}} &\le \BE_{\bc}\sbra{\ln\pbra{\dfrac{e}{\gamma_\rel(\X_{\bc} \times \Y_{\bc})}}}  \\
         &= \sum_{\substack{c\in\{0,1\}^*,|c|\le d}} \Pr[\bc=c] \ln\pbra{\dfrac{e}{\Pr[\bc=c]}} \\
          & \le \ln(2e\cdot 2^d) = O(d),  
    \end{align*}
    where the last line follows from concavity.

    \paragraph*{Induction.} 
    To complete the proof, we now show \Cref{eqn:rectangle-measure} by induction. For this, let us look at an intermediate step $t$ in $\Ccal'$ where the Boolean communication is fixed to $c$ and Alice and Bob have exchanged some real messages $r_{<t} := r_1,\ldots, r_{t-1}$. Let the current rectangle be $X_{(c,r_{<t})} \times Y_{(c,r_{<t})}$ and it is Alice's turn to speak. Note that $X_{(c,r_{<t})}$ and $Y_{(c,r_{<t})}$ live in some affine subspaces at this point and in the current round, Alice sends the inner product of her input $x$ with a vector $a^{(t)}$ that is determined by the previous messages and orthogonal to the ambient space of $X_{(c,r_{<t})}$. At this step, Bob's set $Y_{(c,r_{<t})}$ does not change at all. 
    We shall show that in each step, the log of the inverse of the relative measure of the current rectangle does not increase on average over the next message:
    \begin{align}\label{eqn:rectangle-measure-one-step}
          \BE_{\br_{\le t}}\sbra{\ln\pbra{\dfrac{e}{\gamma_\rel(\X_{(c,\br_{\le t})})}} \mid \bc=c, \br_{<t}=r_{<t}} \le \ln\pbra{\dfrac{e}{\gamma_\rel(X_{(c,r_{<t})})}},   
    \end{align}
   and an analogous statement holds when Bob speaks. 
   Taking an expectation over $\br_{<t}$, the above directly applies \eqref{eqn:rectangle-measure} by a straightforward backward induction:
    \begin{align*}
          \BE_{\br_{\le t}}\sbra{\ln\pbra{\dfrac{e}{\gamma_\rel(\X_{(c,\br_{\le t})} \times \Y_{(c,{\br_{\le t})}})}} \mid \bc=c} &\le \BE_{\br_{<t }}\sbra{\ln\pbra{\dfrac{e}{\gamma_\rel(\X_{(c,\br_{< t})} \times \Y_{(c,{\br_{< t})}})}} \mid \bc=c}\\
          & \le \cdots \le  \ln\pbra{\dfrac{e}{\gamma_\rel(X_c \times Y_c)}}.   
    \end{align*}
    
   To see \Cref{eqn:rectangle-measure-one-step}, let us write $X := X_{(c,r_{<t})}$ for Alice's current set.  Recall that since we have fixed the history, Alice has fixed inner product with some orthogonal directions $a^{(1)}, \ldots, a^{(t-1)}$ and she has decided on the next direction $a := a^{(t)}$ along which she will send the next inner product.
   Thus, $X$ lives in some fixed affine subspace $V^\bot+s$ where $V$  is the span of $a^{(1)},\ldots,a^{(t-1)}$ and the next message $r := r_t = \ip{x}{a}$.
   Moreover, conditioned on the history till this point, the conditional probability distribution on Alice's input $\lx \in \Rbb^n$ can be described as follows: the projections corresponding to the non-zero vectors in the sequence $a^{(1)}, \ldots, a^{(t-1)}$, i.e., the inner products $\abra{\lx,a^{(\tau)}}$ where $a^{(\tau)}\neq0$ for $\tau < t$, are fixed according to the shift $s$, while the distribution on the orthogonal complement $V^\bot$ is that of the Gaussian measure $\gamma_{V^\bot}$ on the subspace $V^\bot$ after conditioning on the event that $\lx \in X - s$ (which lives in $V^\bot$). This uses that projections of a standard $n$-dimensional Gaussian in orthogonal directions are independent.

Let $k$ be the dimension of $V$ where $k<n$. Then, by doing a linear transformation, we may assume that $V^\bot= \Rbb^{n-k}$ (and thus $X \subseteq \Rbb^{n-k}$ and the shift $s$ fixes the coordinates $n-k+1$ through $n$) and $a = e_1$, i.e., in the current message Alice reveals the first coordinate of $\lx \in \Rbb^{n-k}$ where $\lx$ is sampled from $\gamma_{n-k}$ conditioned on $\lx \in X$. In this case, $\gamma_\rel$ in the left hand side of \Cref{eqn:rectangle-measure-one-step} is exactly $\gamma_\rel(X \cap \{x_1=r\})$ if Alice sends $r$ as the message, while for the right hand side of  \Cref{eqn:rectangle-measure-one-step}, we have $\gamma_\rel(X) = \gamma_{n-k}(X)$. Denoting by  $\sd\mu_{x_1}$ the probability density function of $\lx_1$, our statement boils down to showing
   \begin{align*}
       \int_{\Rbb} \ln\pbra{\dfrac{e}{\gamma_\rel(X \cap \{x_1=r\})}} \sd\mu_{x_1}(r) &\le  \ln\pbra{\dfrac{e}{\gamma_{n-k}(X)}}.  
   \end{align*}
   
    We show the above by explicitly writing the probability density function $\sd\mu_{x_1}$. Denote by $\sd\gamma_{n-k}(x_1,\ldots,x_{n-k})$ the standard Gaussian density function\footnote{Explicitly $\sd\gamma_m(x_1,\ldots,x_m)=\prod_{i=1}^m\sd\gamma_1(x_i)$ where $\sd\gamma_1(r) = \frac{1}{\sqrt{2\pi}} e^{-r^2/2}$ is the density function for one-dimensional standard Gaussian.} in $\Rbb^{n-k}$.
    The density function of the random vector $\lx$ sampled from $\gamma_{n-k}$ conditioned on $x \in X$, is given ${\gamma_{n-k}(X)}^{-1} \cdot {\sd\gamma_{n-k}(x_1,\ldots,x_{n-k})}$ for $x\in X$ and zero outside. Thus, we have 
   \begin{align*}
        \sd\mu_{x_1}(r) &= \frac{\int_{X \cap \{x_1=r\}} \sd\gamma_{n-k}(x_1,\ldots,x_{n-k})}{\gamma_{n-k}(X)}\\
        & = \sd\gamma_1(r) \cdot \frac{\int_{X \cap \{x_1=r\}} \sd\gamma_{n-k-1}(x_2,\ldots,x_{n-k})}{\gamma_{n-k}(X)}  = \sd\gamma_1(r) \cdot \frac{\gamma_\rel(X \cap \{x_1=r\})}{\gamma_{n-k}(X)}.   
   \end{align*}

   Then, by concavity, the left hand side of \Cref{eqn:rectangle-measure-one-step} is exactly given by
   \begin{align*}
       \int_{\Rbb} \ln\pbra{\dfrac{e}{\gamma_\rel(X \cap \{x_1=r\})}} \sd\mu_{x_1}(r) &\le  \ln\pbra{\int_{\Rbb} \dfrac{e}{\gamma_\rel(X \cap \{x_1=r\})} \sd\mu_{x_1}(r)}  \\
       &=  \ln\pbra{\dfrac{e}{\gamma_{n-k}(X)} \int_{\Rbb}  \sd\gamma_1(r)} = \ln\pbra{\dfrac{e}{\gamma_{n-k}(X)}}.  
       \qedhere
   \end{align*}

\end{proof}


\section{Level-Two Fourier Growth}\label{sec:proof_of_level_two}

In this section, we prove \Cref{thm:boolean_bound_level_two} that $L_{1,2}(h)=O\pbra{d^{3/2}\log^3(n)}$. Similar to the proof of level-one bound \Cref{thm:boolean_bound_level_one}, we start with a $d$-round communication protocol $\tilde\Ccal$ over the Gaussian space as defined in \Cref{sec:fourier_via_martingale}.
Note that $\tilde\Ccal$ in turn comes from the original Boolean communication protocol $\Ccal$. Thus in the following we assume without loss of generality $d\le n$.

Given the discussion in \Cref{sec:fourier_weights_via_martingales}, to bound the second-level Fourier growth, one can attempt to bound the expected quadratic variation of the martingale that results from the protocol $\bar{\Ccal}$ directly, but similar to the case of level-one it is hard to leverage cancellations here to prove the bound we aim for. So, starting from $\tilde{\Ccal}$, we will define a communication protocol $\bar{\Ccal}$ that computes the same function as $\tilde\Ccal$, but satisfies some additional ``clean" property where it is easier to control the quadratic variation. This new protocol will differ from $\tilde\Ccal$ in two ways. Firstly, the protocol $\bar{\Ccal}$ will consist of additional ``cleanup steps'' where Alice and Bob reveal certain {\em quadratic forms} of their input. Secondly, the protocol $\bar\Ccal$ will send the real value of the quadratic form {\em with certain precision}. Note that this protocol will not involve sending real messages at all, instead, any potential real messages will be truncated to a few bits of precision and be sent as Boolean messages. 

We emphasize that the main difference in the protocol $\bar\Ccal$ from the corresponding level-one variant comes from the precision control, which is not needed there due to the fact that Gaussian distribution remains a (lower-dimensional) Gaussian under linear projections. For technical reasons we shall also need to analyze the martingale under a truncated Gaussian distribution, where all coordinates are  bounded in some large interval $[-T, T]$. This intuitively doesn't incur a noticeable difference on the distribution since it is highly unlikely that coordinates drawn from Gaussian distribution will be outside such intervals and recalling \Cref{rem:symmetric} and \Cref{prop:fwt-to-qv}, it still suffices to analyze the corresponding martingale under the truncated Gaussian distribution. 

We next define the notion of a $4$-wise clean protocol. 

\subsection[4-Wise Clean Protocols]{$4$-Wise Clean Protocols}

Consider an intermediate node in the protocol and let $X\subseteq \Rbb^n$ refer to the set of Alice's inputs reaching this node.
We denote by  $\Sbb^{n\times n-1}$ the set of all matrices in $\Rbb^{n\times n}$ with zero diagonal and unit norm (when viewed as $n^2$-dimensional vectors).
For a parameter $\lambda>0$, we say that the set $X$ is \emph{$4$-wise clean in a direction $a\in \Sbb^{n\times n-1}$} if 
\[
\E_{\lX\sim \gamma}\sbra{ \abra{\lX\tensor\lX - \sigma(X), a}^2 \mid\lX\in X} < \lambda,
\]
where we recall that $\comtwo(X)=\E_{\lX\sim \gamma}\sbra{\lX\tensor\lX\mid\lX\in X}$ is the level-two center of mass of $X$ under the Gaussian measure.
We say that the set $X$ is \emph{$4$-wise clean} if it is $4$-wise clean in \emph{every direction $a$}. 
Our new protocol will consist of the original protocol, interspersed by cleaning steps. Once Alice sends her bit as in the original protocol, she cleans $X$ by revealing $\abra{x\tensor x, a}$ with a few bits of precision while there exists direction $a \in \Sbb^{n\times n-1}$  such that $X$ not clean in direction $a$. Once $X$ becomes clean, Alice proceeds to the next round and Bob does an analogous cleanup. We now describe this formally.
 
\paragraph*{Communication with Finite Precision.}
Let positive integer $L$ be a precision parameter that we will use for truncation.
In our new communication protocol, we will send real numbers with precision $2^{-L}$.
This is formalized as the $\SendReal_L(z)$ function defined at $z\in \R$ as
$$
\SendReal_L(z)=\floorbra{z\cdot 2^{L}}/2^L.
$$

\paragraph*{Construct $\bar \Ccal$ from $\tilde \Ccal$.} 
As described before, $\bar\Ccal$ will consist of the original protocol along with extra steps where Alice or Bob reveal the (approximate) value of a quadratic form on their input. Consider an intermediate node of this new protocol at depth $t$. We always use the random variable $\supX{t}$ (resp., $\supY{t}$) to denote the set of inputs of Alice (resp., Bob) reaching the node. If Alice is revealing a quadratic form in this step, we use $\supa{t}$ to denote the matrix of the quadratic form revealed at this node, otherwise set $\supa{t}$ to be the all-zeroes matrix. We define $\supb{t}$ similarly for Bob. Throughout the protocol, we will always set $\supu{t}$ and $\supv{t}$ to denote $\comtwo(\supX{t})$ and $\comtwo(\supY{t})$ respectively.

Recall that $\lambda>0$ is the parameter for cleanup to be optimized later. Since we will now send real numbers (with certain precision) as bit-strings, their magnitudes should also be well controlled to guarantee bounded message length.
This is managed by a parameter $T>0$ and we will restrict the inputs to the parties in $\bar\Ccal$ to come from the box $[-T,T]^n$. Note that, by Gaussian concentration, $T=\Theta\pbra{\sqrt{\log(n)}}$ suffices. 
\begin{enumerate}
\item At the beginning, Alice receives an input $x\in[-T,T]^n$ and Bob receives an input $y\in[-T,T]^n$.
\item We initialize $t\gets0$, $\supX{0},\supY{0}\gets[-T,T]^n$, and $\supa{0},\supb{0}\gets0^{n\times n}$.
\item For each phase $i=1,2,\ldots,d$: suppose we are starting the cleanup for a node at depth $i$ in the original protocol $\tilde\Ccal$ and suppose we are at a node of depth $t$ in the new protocol $\bar\Ccal$. If it is Alice's turn to speak in $\tilde\Ccal$:
\begin{enumerate}
\item \textbf{Orthogonalization by revealing the correlation with Bob's center of mass.}\\
Alice begins by revealing the inner product of her input $x$ with Bob's current (signed) level-two center of mass $\Lambda\odot \supv{t}$. Since in the previous steps, she has already revealed the inner product with Bob's previous centers of mass, for technical reasons, we will only have Alice announce the inner product with the component of $\Lambda\odot \supv{t}$ that is orthogonal to the previous directions along which Alice announced the inner product. More formally, let $\supa{t+1}$ be the component of $ \Lambda\odot \supv{t}$ that is orthonormal to the span of the previous directions $\supa{\tau}$ for $\tau\le t$, i.e.,
$$\textstyle
\supa{t+1}=\unit\pbra{ \Lambda\odot \supv{t}-\sum_{\tau=1}^t\abra{ \Lambda \odot \supv{t},\supa{\tau}}\cdot\supa{\tau}}.
$$
Alice computes $\supcbar{t+1}\gets\SendReal_L\pbra{\abra{x\tensor x,\supa{t+1}}}$ and sends $\supcbar{t+1}$ to Bob. 
Set $\supb{t+1}\gets 0^{n\times n}$.
Increment $t$ by $1$ and go to step (b). 
\item \textbf{Original communication.}
Alice sends the bit $\supcbar{t+1}$ that she was supposed to send in $\tilde\Ccal$ based on previous messages and $x$. Set $\supa{t+1},\supb{t+1}\gets 0^{n\times n}$. 
Increment $t$ by 1 and go to step (c). 
\item \textbf{Cleanup steps.}
While there exists some direction $a\in\Sbb^{n\times n-1}$ orthogonal to previous directions, i.e., $\abra{a,\supa{\tau}}=0$ for all $\tau\le t$, and $\supX{t}$ is \emph{not $4$-wise clean} in direction $a$, Alice computes $\supcbar{t+1}\gets\SendReal_L\pbra{\abra{x\tensor x,a}}$ and sends $\supcbar{t+1}$ to Bob. 
Set $\supa{t+1}\gets a$ and $\supb{t+1}\gets0^{n\times n}$. Increment $t$ by 1. 
Repeat step (c) while $\supX{t}$ is not $4$-wise clean; otherwise, increment $i$ by 1 and go back to the for-loop in step 3 which starts a new phase.
\end{enumerate}
If it is Bob's turn to speak, we define everything similarly with the role of $x,\lA,\X,\U$ switched with $y,\lB,\Y,\V$.
\item Finally at the end of the protocol, the value $\bar\Ccal(x,y)$ is determined based on all the previous communication and the corresponding output it defines in $\tilde\Ccal$.
\end{enumerate}

\begin{remark}
Note that by construction, the non-zero matrices among $\supa{1},\supa{2}, \ldots $ form an orthonormal set when viewed as $n^2$-dimensional vectors (similarly for $\supb{1},\supb{2}, \ldots $) and moreover, their diagonals are zero. Lastly, $\supa{t}$ and $\supb{t}$ are known to both Alice and Bob as they are canonically determined by previous messages.
\end{remark}

We remark that the steps 3(a), 3(b), and 3(c) always occur in sequence for each party and we refer to such a sequence of steps as a \emph{phase} for that party. Note that there are at most $d$ phases. 
If a new phase starts at time $t$, then the current rectangle $\supX{t} \times \supY{t}$ is $4$-wise clean for both parties by construction. 

Now we formalize a few useful properties regarding the communication protocol $\bar\Ccal$. The first fact below follows since each $\supu{t}$ is an expectation of $\lx\tensor \lx$ over some distribution and $\lx\tensor \lx$ has zero diagonal.

\begin{fact}\label{fct:starting_point}
$\supu{0}=\supv{0}=0^{n\times n}$ and each $\supu{t},\supv{t}$ has zero diagonal.
\end{fact}

The following follows from tail bounds for the univariate standard normal distribution.

\begin{fact}\label{fct:gammastar}
Let $\gamma^*=\gamma(\supX{0})\cdot\gamma(\supY{0})$. Then $\gamma^*\ge1-O\pbra{n\cdot e^{-T^2/2}}$.
\end{fact}

The next fact says that when a node fixes a quadratic form with $2^{-L}$ precision, for any two inputs that reach this node, the quadratic forms differ by at most $2^{-L}$. 

\begin{fact}\label{fct:sendreal_error}
In step 3(a) and 3(c), any $x,x'\in \supX{t+1}$ satisfies $\abs{\abra{x\tensor x,\supa{t+1}}-\abra{x'\tensor x',\supa{t+1}}}<2^{-L}$.
Similarly any $y,y'\in \supY{t+1}$ satisfies $\abs{\abra{y\tensor y,\supb{t+1}}-\abra{y'\tensor y',\supb{t+1}}}<2^{-L}$.
\end{fact}

The next claim bounds the maximum attainable norms for Alice and Bob's level-two center of masses at any point in the protocol. This uses the fact that the inputs come from the truncated Gaussian distribution.

\begin{claim}\label{clm:frob_norm_ub}
$\frob{\supu{t}}=\frob{ \Lambda\odot \supu{t}}<nT$ and $\frob{\supv{t}}=\frob{ \Lambda\odot \supv{t}}<nT$ for all possible $t$ and $\supu{t},\supv{t}$ throughout the communication.
\end{claim}
\begin{proof}
Since $\Lambda$ is a matrix with zero diagonal and $\binpm$ entries off diagonal and $\supu{t}$ also has zero diagonal, $\frob{\supu{t}}=\frob{ \Lambda\odot \supu{t}}$.
In addition, since $\supX{t}\subseteq \supX{0}=[-T,T]^n$, we have
$$
\frob{\supu{t}}
\le\E_{\lx\sim\gamma}\sbra{\frob{\pbra{\lx\tensor\lx}}\mid\lx\in \supX{t}}
\le\sqrt{(n^2-n)\cdot T^2}<nT.
$$
A similar analysis works for $\supv{t}$.
\end{proof}

The next claim gives a bound on the length of any message in the protocol $\bar \Ccal$.

\begin{claim}\label{clm:short_messages}
For any $x\in \supX{0}$ and $y\in \supY{0}$, any message in $\bar\Ccal(x,y)$ consists of at most $L + \log(Tn)$ many bits.
\end{claim}
\begin{proof}
Assume without loss of generality it is Alice's turn to speak. On step 3(b) she sends one bits. On steps 3(a) and 3(c), she computes $\SendReal_L(\sabra{x\tensor x,a})$ for some  $a\in \Sbb^{n\times n-1}$ and send the result. Since
$$
\abs{\abra{x\tensor x,a}}\le\frob{x\tensor x}\cdot \frob{a}\le\sqrt{(n^2-n)\cdot T^2}<nT,
$$
and the message is a multiple of $2^{-L}$
that means $\SendReal_L$ yields a message with $L+ \log(nT)$ many bits.
\end{proof}

The last claim bounds the maximum depth of the new protocol $\bar \Ccal$.

\begin{claim}\label{clm:finite_steps}
Let $\ell$ be an arbitrary leaf of the protocol $\bar\Ccal$ and $D(\ell)$ be its depth.
Then $D(\ell)\le2n^2$.
Moreover, along this path there are at most $n^2-n$ many non-zero $\supa{t}$ and at most $n^2-n$ many non-zero $\supb{t}$ for $t\in\{1,\ldots,D(\ell)\}$.
\end{claim}
\begin{proof}
We count the number of communication steps separately:
\begin{itemize}
\item \textbf{Steps 3(a) and 3(b).} Steps 3(a) and 3(b) occur once in every phase, thus at most $d$ times.
\item \textbf{Step 3(c).} For Alice, each time she communicates at step 3(c), the direction $a\in\Sbb^{n\times n-1}$ is non-zero and orthogonal to all previous $\supa{t}$'s. Since the dimension of $\Sbb^{n\times n-1}$ is $n^2-n$, this happens at most $n^2-n$ times. Similar argument works for Bob.
\end{itemize}
Thus in total we have at most $2(n^2-n)+2d \le 2n^2$ steps.
\end{proof}


We will eventually show that, with suitable choice of $\lambda,T,L$, typically $D(\ell)$ is at most $d\cdot\polylog(n)$.

\subsection{Bounding the Expected Quadratic Variation}\label{sec:expected_quadratic_variation}

Consider the martingale process defined in \Cref{eqn:def-martingale} from a random walk on the protocol tree generated by $\bar\Ccal$ where the inputs $\lx, \ly$ are sampled from $\gamma_n$ conditioned on being in the bounded cube $[-T,T]^n$. Recall that \Cref{prop:vec-martingale} still holds (see \Cref{rem:martingale}).

Formally, at time $t$ the process is defined by
$$
\supZ{t}_2=\abra{\supu{t},\eta\odot\supv{t}},
$$
where we recall that $\supu{t}=\comtwo(\supX{t})$ and $\supv{t}=\comtwo(\supY{t)})$ and $\eta$ is a fixed sign matrix with a zero diagonal.
The martingale process stops once it hits a leaf of $\bar\Ccal$.
Let $\D$ denote the (stopping) time when this happens.
Note that $\E[\D]$ is exactly the expected depth of the protocol $\bar\Ccal$.

In light of \Cref{rem:symmetric} and \Cref{prop:fwt-to-qv}, to prove \Cref{thm:boolean_bound_level_two}, it suffices to prove the following.
\begin{lemma}\label{lem:qv-level-two}
$\E\sbra{\sum_{t=1}^{\D} \pbra{\Delta\supZ{t}_2}^2} = O\pbra{d^3\log^6(n)}.$
\end{lemma}

\Cref{lem:qv-level-two} is proved in three steps.
We first show that essentially the only change in the value of the martingale is the orthogonalization step 3(a).
The reason is the same as the level-one bound: Alice's messages sent in step 3(b) and 3(c) are always near-orthogonal to Bob's current level-two center of mass, thus they do not change the value of the martingale $\supZ{t}_2$ much.
Moreover, by level-two analog of \Cref{eqn:overview}, since Alice's current node was clean just before Alice sent the message in step 3(a), the expected change $\E\sbra{\pbra{\Delta\supZ{t+1}_2}^2}$ can be bounded in terms of the squared norm of the change that occurred in $\supu{t}$ (or $\supv{t}$) between the current round and the last round where Alice was in step 3(a). Similar argument works for Bob.

Formally, this is encapsulated by the next lemma for which we need some additional definitions. Let $(\supF{t})_t$ denote the natural filtration induced by the random walk on the generalized protocol tree with respect to which $\supZ{t}_2$ is a Doob martingale and also $\supu{t}, \supv{t}$ form vector-valued martingales (recall \Cref{prop:vec-martingale}). Note that $\supF{t}$ fixes all the rectangles encountered during times $0,\ldots, t$ and thus for $\tau \le t$, the random variables $\supu{\tau},\supv{\tau},\supZ{\tau}_2$ are determined, in particular, they are $\supF{t}$-measurable. Recalling that $\lambda$ is the cleanup parameter to be optimized later, we then have the following. Below we assume without any loss of generality that Alice speaks first and, in particular, we note that Alice speaks in step 3(a) for the first time at time zero when both Alice and Bob's center of masses are at zero: $\supu{0}=\supv{0}=0$. 

\begin{lemma}[Step Size]\label{lem:step_size_square_level_two}
    Let $0= \btau_1 < \btau_2 < \cdots \le \D$ be a sequence of stopping times with $\btau_m$ being the index of the round where Alice speaks in step 3(a) for the $m^\text{th}$ time or $\D$ if there is no such round. 
    Then, for any integer $m \ge 2$, 
	$$
	\E\sbra{\pbra{\Delta\supZ{\btau_m+1}_2}^2 \mid \supF{\btau_m}}  \le \lambda \cdot \vabs{\supv{\btau_m} - \supv{\btau_{m-1}}}^2+ 16n^7T^3 \cdot  2^{-L}.
	$$
	and moreover, for any $t \in \N$, we have that 
	$$
	\E\sbra{\pbra{\Delta\supZ{t+1}_2}^2\mid\supF{t}, \btau_{m-1} < t <\btau_{m}, \text{Alice speaks at time }t}\le 4 n^6T^2 \cdot 2^{-2L}
	$$
 A similar statement also holds if Bob speaks where $\V$ is replaced by $\U$ and the sequence $(\btau_m)$ is replaced by $(\btau'_m)$ where $\btau'_m$ is the index of the round where Bob speaks in step 3(a) for the $m^\text{th}$ time or $\D$ if there is no such round. 
\end{lemma}

We indeed see that, if $L=\Omega(\log(n))$ and $T=O(\sqrt{\log(n)})$, then $\poly(T,n)\cdot 2^{-L} = o(1)$, and steps~3(b) and~3(c) do not contribute much to the quadratic variation and only the steps 3(a) do.  Also, since the first time Alice and Bob speak, they start in step 3(a), we also note that $\supu{\btau_1}$ and $\supv{\btau'_1}$ are their initial centers of mass which are both zero.  

We shall prove the above lemma in \Cref{sec:step_size_leve_two} and continue with the bound on the quadratic variation here.
Using the bounds on the step sizes from \Cref{lem:step_size_square_level_two},
\begin{align*}
\E\sbra{\sum_{t=1}^{\D} \pbra{\Delta\supZ{t}_2}^2} 
&\le\lambda \cdot\E\sbra{\sum_{m\ge 2} \vabs{\V^{(\btau_m)}-\V^{(\btau_{m-1})}}^2+\vabs{\U^{(\btau'_m)}-\U^{(\btau'_{m-1})}}^2}+
16n^7T^3 \cdot  2^{-L}
\cdot\E[\D]\\
&\le\lambda \cdot\E\sbra{\sum_{m \ge 2} \vabs{\V^{(\btau_m)}-\V^{(\btau_{m-1})}}^2+\vabs{\U^{(\btau'_m)}-\U^{(\btau'_{m-1})}}^2}+16n^7T^3 \cdot  2^{-L}
\cdot2n^2.
\tag{by \Cref{clm:finite_steps}}
\end{align*}
On the other hand, using the orthogonality of vector-valued martingale differences from  \Cref{eqn:martingale-orthogonality-vec},
\begin{align*}
	\E\sbra{\sum_{m \ge 2} \vabs{\V^{(\btau_m)}-\V^{(\btau_{m-1})}}^2} = \E\sbra{\vabs{\V^{(\D)}}^2}.
\end{align*}
A similar statement holds for $(\supu{t})$ as well. Therefore, 
\begin{align}\label{eqn:qv-upper-bound-level-two}
\E\sbra{\sum_{t=1}^{\D} \pbra{\Delta\supZ{t}_2}^2} \le\lambda \cdot\pbra{\E\sbra{\frob{\U^{(\D)}}^2}+\E\sbra{\frob{\V^{(\D)}}^2}}+64n^9T^3 \cdot  2^{-L}.
\end{align}

Then in \Cref{sec:to_depth} we will apply level-two inequalities (see \Cref{thm:level_k_ineq}) to convert the bounding $\E\sbra{\frob{\U^{(\D)}}^2+\frob{\V^{(\D)}}^2}$ into bounding the second moment $\E[\D^2]$. This reduction is formalized as \Cref{lem:to_depth} below and its proof is similar to \cite[Claim 1]{GRT21}.

For each leaf $\ell$, let $\gamma(\ell)=\gamma(\supX{D(\ell)})\cdot\gamma(\supY{D(\ell)})$ be the Gaussian measure of the rectangle at $\ell$. 
Recall $\gamma^*=\gamma(\supX{0})\times\gamma(\supY{0})$.
\begin{lemma}\label{lem:to_depth}
$\E\sbra{\frob{\supu{\D}}^2+\frob{\supv{\D}}^2}\le O\pbra{\frac1{\gamma^*}+L^2\E[\D^2]}$.
\end{lemma}

Finally, in \Cref{sec:depth_tail_bounds}, we bound the second moment $\E[\D^2]$ for a suitable choice of parameters.
\begin{lemma}\label{lem:second_moment} It holds that
$\E[\D^2]=O(d^2)$ and $\gamma^*\ge\frac{3}{4}$ for $L=\Theta(\log(n))$, $T=\Theta(\sqrt{\log(n)})$, and $\lambda=\Theta(d\log^4(n))$. 
\end{lemma}

Given \Cref{lem:to_depth,lem:second_moment},the proof of \Cref{lem:qv-level-two} naturally follows.
\begin{proof}[Proof of \Cref{lem:qv-level-two}]
With the parameters chosen in \Cref{lem:second_moment}, we have
\begin{align*}
\E\sbra{\sum_{t=1}^{\D} \pbra{\Delta\supZ{t}_2}^2} 
&\le O(d\log^4(n))\cdot\pbra{\E\sbra{\frob{\U^{(\D)}}^2}+\E\sbra{\frob{\V^{(\D)}}^2}}+1
\tag{by \Cref{eqn:qv-upper-bound-level-two}}\\
&\le O(d\log^4(n))\cdot\pbra{1+\log^2(n)\cdot\E[\D^2]}+1
\tag{by \Cref{lem:to_depth}}\\
&\le O(d\log^4(n))\cdot\pbra{1+\log^2(n)\cdot d^2}+1
\tag{by \Cref{lem:second_moment}}\\
&=O(d^3\log^6(n)).
\tag*{\qedhere}
\end{align*}
\end{proof}

\begin{remark}
Note that our proof for level-two Fourier growth actually holds for a slightly more general setting, where Alice and Bob are allowed to send $O(L)=O(\log(n))$ bits during each original communication round.
This can be viewed as balancing the length of the messages in step 3(b) with step 3(a) and step 3(c).

Since one can always convert a $d$-round $1$-bit communication protocol into a $\frac{2d}{\log\log(n)}$-round $\log(n)$-bit communication protocol, we obtain a slightly better level-two Fourier growth bound of 
$$
O\pbra{\frac{d^{3/2}\log^3(n)}{\pbra{\log\log(n)}^{3/2}}}.
$$ 
The conversion is done by Alice (resp., Bob) enumerating the next $\log\log(n)/2$ bits from Bob (resp., Alice), and providing the corresponding $\log\log(n)/2$ bits responses for each possibility.

It is also possible to improve the $\log^3(n)$ factor to $\log^2(n)$ by varying the cleanup parameter $\lambda$ with depth.
For example, for depth in the interval $[4rd, 4(r+1)d]$, one could pick $\lambda_r = \Theta( d \cdot \log^2(n) \cdot r^2)$.
Since our focus is mostly on improving the polynomial dependence in $d$ where there is still room for improvement, we do not make an effort here to improve the polylog terms.
\end{remark}

\subsection[Bounds on Step Sizes]{Bounds on Step Sizes (Proof of \Cref{lem:step_size_square_level_two})}\label{sec:step_size_leve_two}

Let us abbreviate $\btau = \btau_m$ and note that at time $\btau$ a new phase starts for Alice. 
By construction, this means that the current rectangle $\supX{\btau} \times \supY{\btau}$ determined by $\supF{\btau}$ is $4$-wise clean with parameter $\lambda$, and since Alice is in step 3(a) at the start of a new phase, $\supa{\btau+1}$ is chosen to be the (normalized) component of $\Lambda\odot \V^{(\btau)}$ that is orthogonal to previous directions $\supa{1}, \ldots, \supa{\btau}$. 

For each $r=1,\ldots,\btau+1$, let $\balpha^{(r)}:= \abra{\Lambda\odot \V^{(\btau)},\la^{(r)}}$ be the length of $\Lambda\odot \V^{(\btau)}$ along direction $\la^{(r)}$.
Each $\balpha^{(r)}$ is $\Fcal^{(\btau)}$-measurable (i.e., it is determined by $\Fcal^{(\btau)}$) and $\eta\odot\supv{\btau}=\sum_{r\le\btau+1}\balpha^{(r)}\cdot\supa{r}$. 
In this case, we have
\begin{align}\label{eq:step_size_level_two}
\E\sbra{\pbra{\Delta\lZ^{(\btau+1)}_2}^2\mid \Fcal^{(\btau)}}&=\E\sbra{\abra{\U^{(\btau+1)}-\U^{(\btau)}, \Lambda\odot \V^{(\btau)}}^2\mid \Fcal^{(\btau)}}\notag\\
&=\E\sbra{\pbra{\sum_{r=1}^{\btau+1}\balpha^{(r)}\cdot\abra{\supu{\btau+1}-\supu{\btau},\supa{r}}}^2\mid \Fcal^{(\btau)}}.
\end{align}

Similar to the level-one proof, the components of $\supu{\btau+1}$ and $\supu{\btau}$ are roughly the same along any of the previous directions $\supa{1},\ldots,\supa{\btau}$ and so they almost cancel out and the major quantity is in the direction $\supa{\btau+1}$.
This follows since, in all the previous steps $r\le\btau$, Alice has already fixed $\abra{x\tensor x,\supa{r}}$ with precision $2^{-L}$.
This implies that for any $\supX{\btau}$ and $\supX{\btau+1}$ that are determined by $\supF{\btau+1}$, the inner product with all the previous $\supa{1},\ldots,\supa{\btau}$ is fixed with precision $2^{-L}$ over the choice of $x$.
Formally, by \Cref{fct:sendreal_error}, we have that for any $x\in\supX{\btau}$ and $x'\in\supX{\btau+1}$, it holds that $\abs{\abra{x\tensor x,\supa{r}}-\abra{x'\tensor x',\supa{r}}}\le2^{-L}$ for all $r\le\btau$.
In particular, since $\supu{\btau}=\comtwo(\supX{\btau})$ and $\supu{\btau+1}=\comtwo(\supX{\btau+1})$ are the corresponding centers of mass, we have that
\begin{equation}\label{eq:step_size_level_two_error_term}
\abs{\abra{\supu{\btau+1}-\supu{\btau},\supa{r}}}\le2^{-L}
\quad\text{for all $r\le\btau$.}
\end{equation}
On the other hand, 
since $\supX{\btau+1}\subseteq\supX{\btau}\subseteq\supX{0}=[-T,T]^n$ and $\supa{\btau+1}$ is a unit direction, we have
\begin{equation}\label{eq:step_size_level_two_frob_bound}
\abs{\abra{\supu{\btau+1}-\supu{\btau},\supa{\btau+1}}}
\le\vabs{\supu{\btau+1}-\supu{\btau}}
\le2nT.
\end{equation}
Similarly, 
noting that $\eta$ is a sign matrix, we can bound
\begin{equation}\label{eq:step_size_level_two_beta}
\abs{\balpha^{(r)}}
=\abs{\abra{\eta\odot\supv{\btau},\supa{r}}}
\le\vabs{\eta\odot\supv{\btau}}
\le\vabs{\supv{\btau}}
\le nT
\quad\text{for all $r\le\btau+1$.}
\end{equation}
Expanding the square in \Cref{eq:step_size_level_two} and plugging these estimates to each one of the $(\btau+1)^2$ terms gives
\begin{align}
\E\sbra{\pbra{\Delta\lZ^{(\btau+1)}_2}^2\mid \Fcal^{(\btau)}}
&\le\E\sbra{\pbra{\balpha^{(\btau+1)}}^2\abra{\supu{\btau+1}-\supu{\btau},\supa{\btau+1}}^2 
+ ((\btau+1)^2-1)\cdot  \tfrac{2(nT)^3}{2^{L}}
\mid \Fcal^{(\btau)}}\notag
\\
&\le\pbra{\balpha^{(\btau+1)}}^2\E\sbra{\abra{\supu{\btau+1}-\supu{\btau},\supa{\btau+1}}^2\mid\supF{\btau}}+12n^7T^3 \cdot  2^{-L},
\label{eq:step_size_level_two_no_error_term}
\end{align}
where the second line follows from \Cref{clm:finite_steps}.

We now bound the term outside the expectation by the change in the center of mass $\supv{\cdot}$ and the term inside the expectation by the fact that the set is $4$-wise clean.

\paragraph*{Term Outside the Expectation.}
Recall that $\supa{\btau+1}$ is chosen to be the (normalized) component of $\eta\odot\supv{\btau}$ that is orthogonal to the span of $\supa{1},\ldots,\supa{\btau}$.
Since $\eta\odot\supv{\btau_m-1}$ is in the span of $\supa{1},\ldots,\supa{\btau_{m-1}+1}$ and $\btau_{m-1}+1\le\btau=\btau_m$, it is orthogonal to $\supa{\btau+1}$. Hence
$$
\balpha^{(\btau+1)}=\abra{\eta\odot\supv{\btau},\supa{\btau+1}}=\abra{\eta\odot\pbra{\supv{\btau}-\supv{\btau_{m-1}}},\supa{\btau+1}}.
$$
Since $\supa{\btau+1}$ is a unit direction and $\eta$ is a sign matrix, this implies that
\begin{equation}\label{eq:level_two_term_outside}
\pbra{\balpha^{(\btau+1)}}^2\le\vabs{\supv{\btau}-\supv{\btau_{m-1}}}^2.
\end{equation}

\paragraph*{Term Inside the Expectation.}
Recall that Alice is in step 3(a), she sends $\abra{x\tensor x,\supa{\btau+1}}$ with precision $2^{-L}$ at time $\btau$, and thus the same inner product with $\supa{\btau+1}$ is fixed with precision $2^{-L}$ for every point in $\supX{\btau+1}$ determined by $\supF{\btau+1}$.
Thus
\begin{align}
\abra{\supu{\btau+1},\supa{\btau+1}}^2
&=\pbra{\E_{\lx\sim \gamma}\sbra{\abra{\lx\tensor\lx,\la^{(\btau+1)}}\mid \lx\in \X^{(\btau+1)}}}^2\notag\\
&=\pbra{\abra{x\tensor x,\la^{(\btau+1)}}+\E_{\lx\sim \gamma}\sbra{\eps_{\lx}\mid \lx\in \X^{(\btau+1)}}}^2
\tag{$|\eps_{\lx}|\le2^{-L}$ is the truncation error by \Cref{fct:sendreal_error}}\\
&\le\abra{x\tensor x,\la^{(\btau+1)}}^2+2^{-2L}+2^{1-L}\cdot\abs{\abra{x\tensor x,\la^{(\btau+1)}}}\notag\\
&\le\abra{x\tensor x,\la^{(\btau+1)}}^2+nT\cdot2^{2-L},
\label{eq:level_two_term_inside}
\end{align}
where the last line follows from $\abs{\abra{x\tensor x,\la^{(\btau+1)}}}\le\vabs{x\tensor x}$ and $x\in\supX{0}=[-T,T]^n$.

\paragraph*{Final Bound.}

Since $(\supu{r})_r$ is a matrix-valued martingale and thus $\E\sbra{\supu{\btau+1}\mid\supF{\btau}}=\supu{\btau}$, we have
$$
\E\sbra{\abra{\supu{\btau+1}-\supu{\btau},\supa{\btau+1}}^2\mid\supF{\btau}}
=\E\sbra{\abra{\supu{\btau+1},\supa{\btau+1}}^2-\abra{\supu{\btau},\supa{\btau+1}}^2\mid\supF{\btau}}
$$
Then by \Cref{eq:level_two_term_inside}, we upper bound the right hand side by
\begin{align*}
nT\cdot2^{2-L}+\E_{\lx\sim\gamma}\sbra{\abra{\lx\tensor\lx,\supa{\btau+1}}^2-\abra{\supu{\btau},\supa{\btau+1}}^2\mid\supF{\btau}}.
\end{align*}
Since $\supX{\btau}$ is $4$-wise clean with parameter $\lambda$, it can be bounded by $nT\cdot2^{2-L}+\lambda$:
\begin{equation}\label{eq:level_two_exp_inside}
\E\sbra{\abra{\supu{\btau+1}-\supu{\btau},\supa{\btau+1}}^2\mid\supF{\btau}}
\le nT\cdot2^{2-L}+\lambda
\end{equation}
Putting everything together, we have
\begin{align*}
\E\sbra{\pbra{\Delta\lZ^{(\btau+1)}_2}^2\mid \Fcal^{(\btau)}}
&\le\pbra{\balpha^{(\btau+1)}}^2\E\sbra{\abra{\supu{\btau+1}-\supu{\btau},\supa{\btau+1}}^2\mid\supF{\btau}}+
12n^7T^3 \cdot  2^{-L}
\tag{by \Cref{eq:step_size_level_two_no_error_term}}\\
&\le\pbra{\balpha^{(\btau+1)}}^2\cdot\pbra{nT\cdot2^{2-L}+\lambda}+12n^7T^3 \cdot  2^{-L}
\tag{by \Cref{eq:level_two_exp_inside}}\\
&\le\lambda\cdot\pbra{\balpha^{(\btau+1)}}^2+n^3T^3 \cdot 2^{2-L}+12n^7T^3 \cdot  2^{-L}
\tag{by \Cref{eq:step_size_level_two_beta}}\\
&\le\lambda\cdot\vabs{\supv{\btau}-\supv{\btau_{m-1}}}^2+n^3T^3 \cdot 2^{2-L}+12n^7T^3 \cdot  2^{-L}
\tag{by \Cref{eq:level_two_term_outside}}\\
&\le\lambda\cdot\vabs{\supv{\btau}-\supv{\btau_{m-1}}}^2+16n^7T^3 \cdot  2^{-L}.
\end{align*}
This completes the proof of the first statement in the lemma.

For the moreover part, let us condition on the event $\btau_{m-1}<t<\btau_m$ where Alice speaks at time $t$. 
Note that such $t$ must all lie in the same phase of the protocol where Alice is the only one speaking.
So, Bob's center of mass does not change from the time $\btau_{m-1}$ till $t$, i.e., $\supv{t+1}=\supv{\btau_{m-1}}$.
Thus we have 
\begin{equation}\label{eq:level_two_moreover}
\Delta\supZ{t+1}_2=\abra{\supu{t+1}-\supu{t},\eta\odot\supv{\btau_{m-1}}}.
\end{equation}
Analogous to \Cref{eq:step_size_level_two_error_term}, the component of Alice's center of mass along the previous directions are fixed with precision $2^{-L}$.
Thus by \Cref{fct:sendreal_error}, 
\begin{equation}\label{eq:level_two_moreover_error}
\abs{\abra{\supu{t+1}-\supu{t},\supa{r}}}\le2^{-L}
\quad\text{for all $r\le t$.}
\end{equation}
Furthermore, by construction, $\eta\odot\supv{\btau_{m-1}}$ lies in the space spanned by $\supa{1},\ldots,\supa{\btau_{m-1}+1}$.
Note that $\btau_{m-1}+1\le t$.
Similar to the previous analysis, for each $r=1,\ldots,t$, let $\balpha^{(r)}:=\abra{\eta\odot\supv{t},\supa{r}}$ be the length of $\eta\odot\supv{t}$ along direction $\supa{r}$.
Then \Cref{eq:step_size_level_two_beta} also holds here.
Therefore
\begin{align*}
\abs{\Delta\supZ{t+1}_2}
&=\abs{\sum_{r=1}^t\balpha^{(r)}\cdot\abra{\supu{t+1}-\supu{t},\supa{r}}}
\tag{by \Cref{eq:level_two_moreover}}\\
&\le\sum_{r=1}^t\abs{\balpha^{(r)}}\cdot\abs{\abra{\supu{t+1}-\supu{t},\supa{r}}}
\le\sum_{r=1}^tnT\cdot2^{-L}
\tag{by \Cref{eq:step_size_level_two_beta} and \Cref{eq:level_two_moreover_error}}\\
&\le 2n^3T\cdot2^{-L}.
\tag{by \Cref{clm:finite_steps}}
\end{align*}

\subsection[Conversion to Second Moment Bounds of the Depth]{Conversion to Second Moment Bounds of the Depth (Proof of \Cref{lem:to_depth})}\label{sec:to_depth}

Recall $\gamma^*=\gamma(\supX{0})\times\gamma(\supY{0})$ and $\gamma(\ell)=\gamma(\supX{D(\ell}))\cdot\gamma(\supY{D(\ell)})$ for each leaf $\ell$.
The goal of this subsection is to prove \Cref{lem:to_depth}.

We first note the following basic fact.

\begin{fact}\label{fct:path_probability}
$\sum_\ell\gamma(\ell)=\gamma^*$ and
$$
\Pr_{\lx\sim \supX{0},\ly\sim \supY{0}}\sbra{\bar\Ccal(\lx,\ly)\text{ reaches leaf }\ell}=\gamma(\ell)/\gamma^*.
$$
\end{fact}

Now we apply \Cref{thm:level_k_ineq} with $k=2$ to relate the LHS of \Cref{lem:to_depth} with an entropy-type bound.

\begin{lemma}\label{lem:frob_to_path_prob}
$\E\sbra{\frob{\supu{\D}}^2+\frob{\supv{\D}}^2}\le \frac{4e^2}{\gamma^*}\sum_\ell\gamma(\ell)\cdot\ln^2\pbra{\frac{e}{\gamma(\ell)}}$.
\end{lemma}
\begin{proof}
Let $\ell$ be a fixed leaf and $D=D(\ell)$ be its depth.
Note that this also fixes the rectangle $X^{(D)}\times Y^{(D)}$ and thus the centers of mass $u^{(D)},v^{(D)}$.
Define the indicator function $\indicator_\ell\colon\Rbb^{2n}\to\bin$ by
$$
\indicator_\ell(x,y)=\begin{cases}
1 & (x,y)\in X^{(D)}\times Y^{(D)},\\
0 & \text{otherwise.}
\end{cases}
$$
Then we have
\begin{align*}
&\phantom{\le}\frob{u^{(D)}}^2+\frob{v^{(D)}}^2\\
&=\frob{\E_{\lx\sim\gamma}\sbra{\lx\tensor \lx\mid \lx\in X^{(D)}}}^2+\frob{\E_{\ly\sim\gamma}\sbra{\ly\tensor \ly\mid \ly\in Y^{(D)}}}^2\\
&=\sum_{\substack{i,j=1\\i\neq j}}^n\pbra{\E_{\lx\sim\gamma}\sbra{\lx_i\lx_j\mid \lx\in X^{(D)}}}^2+\sum_{\substack{i,j=1\\i\neq j}}^n\pbra{\E_{\ly\sim\gamma}\sbra{\ly_i\ly_j\mid \ly\in Y^{(D)}}}^2\\
&=\sum_{\substack{i,j=1\\i\neq j}}^n\pbra{\E_{\lx,\ly\sim\gamma}\sbra{\lx_i\lx_j\mid(\lx,\ly)\in X^{(D)}\times Y^{(D)}}}^2+\sum_{\substack{i,j=1\\i\neq j}}^n\pbra{\E_{\lx,\ly\sim\gamma}\sbra{\ly_i\ly_j\mid(\lx,\ly)\in X^{(D)}\times Y^{(D)}}}^2\\
&=\frac2{\gamma(\ell)^2}\pbra{\sum_{S\in\binom{[n]}2}\pbra{\E_{\lx\sim\gamma,\ly\sim\gamma}\sbra{\indicator_\ell(\lx,\ly)\lx_S}}^2+\sum_{S\in\binom{[n]}2}\pbra{\E_{\lx\sim\gamma,\ly\sim\gamma}\sbra{\indicator_\ell(\lx,\ly)\ly_S}}^2}\\
&\le\frac2{\gamma(\ell)^2}\sum_{S\in\binom{[2n]}2}\pbra{\E_{\lw\sim\gamma_n\times\gamma_n}\sbra{\indicator_\ell(\lw)\lw_S}}^2\\
&\le\frac2{\gamma(\ell)^2}\cdot 2e^2\gamma(\ell)^2\cdot\ln^2\pbra{\frac{e}{\gamma(\ell)}}
\tag{by \Cref{thm:level_k_ineq}}\\
&=4e^2\cdot\ln^2\pbra{\frac{e}{\gamma(\ell)}}.
\end{align*}
Therefore taking expectation over a random $\ell$, by \Cref{fct:path_probability}, we have
\begin{equation*}
\E\sbra{\frob{\supu{\D}}^2+\frob{\supv{\D}}^2}
\le 4e^2\cdot\E_{\bell}\sbra{\ln^2\pbra{\frac{e}{\gamma(\bell)}}}
=\frac{4e^2}{\gamma^*}\sum_\ell\gamma(\ell)\cdot\ln^2\pbra{\frac{e}{\gamma(\ell)}}.
\tag*{\qedhere}
\end{equation*}
\end{proof}

Now in the next lemma, we bound the right hand side of \Cref{lem:frob_to_path_prob} in terms of the second moment of the depth, which immediately proves \Cref{lem:to_depth}.
\begin{lemma}\label{lem:path_prob_to_depth}
Assume that $Tn \le  2^L$. Then,
$\sum_\ell\gamma(\ell)\cdot\ln^2\pbra{e/{\gamma(\ell)}}\le O(1+\gamma^*\cdot L^2\E[\D^2])$.
\end{lemma}
\begin{proof}
By \Cref{clm:short_messages}, and the assumption $Tn \le 2^{L}$ each message is of length at most $L+\log(Tn)\le 2L$.
We divide $\ell$ into two cases based on $\gamma(\ell)$:
\begin{align*}
&\sum_{\ell:\gamma(\ell)<2^{-3L\cdot D(\ell)}}\gamma(\ell)\cdot\ln^2\pbra{\frac{e}{\gamma(\ell)}}\\
&\le\sum_{\ell:\gamma(\ell)<2^{-3L\cdot D(\ell)}}2^{-3L\cdot D(\ell)}\cdot\ln^2\pbra{e \cdot 2^{3L\cdot D(\ell)}}
\tag{$x\ln^2(e/x)$ is increasing when $0\le x\le 0.2$}\\
&\le\sum_{t=1}^{\infty}2^{-3L\cdot t}\cdot 2(9L^2t^2+1)\cdot\abs{\cbra{\ell:D(\ell)=t}} 
\tag{since $\ln^2(ab) \le 2\ln^2(a) + 2\ln^2(b)$}\\
&\le\sum_{t=1}^{\infty}2^{-3L\cdot t}\cdot 2(9L^2t^2+1)\cdot2^{(2L)\cdot t}
\tag{each message is of length $\le 2L$}\\
&\le\sum_{t=1}^{\infty}2(9L^2t^2+1)\cdot2^{-Lt}=O(1)
\tag{since $L\ge2$}
\end{align*}
and
\begin{align*}
\sum_{\ell:\gamma(\ell)\ge2^{-3L\cdot D(\ell)}}\gamma(\ell)\cdot\ln^2\pbra{\frac{e}{\gamma(\ell)}}
&\le\sum_{\ell:\gamma(\ell)\ge2^{-3L\cdot D(\ell)}}\gamma(\ell)\cdot\ln^2\pbra{e \cdot 2^{3L\cdot D(\ell)}}\\
&\le 2 \cdot 9L^2 \sum_\ell\gamma(\ell) D(\ell)^2 + 2\sum_\ell\gamma(\ell)\\
&=18L^2\gamma^*\cdot\E_{\bell}\sbra{D(\bell)^2} + 2\\
&=18L^2\gamma^*\cdot\E\sbra{\D^2} + 2.
\end{align*}
Adding up the two estimates above gives the desired bound.
\end{proof}

\subsection[Second Moment Bounds for the Depth]{Second Moment Bounds for the Depth (Proof of \Cref{lem:second_moment})}\label{sec:depth_tail_bounds}

The final ingredient is an estimate for the second moment $\E[\D^2]$.
This subsection is devoted to this goal and proving \Cref{lem:second_moment}.

For messages $\ell'=(\supcbar{1},\ldots,\supcbar{t})$, we define $\gamma(\ell')=\gamma(\supX{t})\cdot\gamma(\supY{t})$ where $\supX{t},\supY{t}$ is defined by the protocol using the messages $\ell'$.
Note that this definition is consistent with $\gamma(\ell)$ from \Cref{sec:to_depth} for a leaf $\ell$.

\begin{lemma}\label{lem:depth_tail_bound}
There exists a universal constant $\alpha>0$ such that the following holds.
Let $0\le d_1<d_2$ be two arbitrary integers with $d_2-d_1\ge2d+1$.
Let $\ell^*=(\supcbar{1},\ldots,\supcbar{d_1})$ be arbitrary messages of the first $d_1$ communication steps.
Assume $2^L\ge8n^4T^2$. Then
$$
\Pr\sbra{\D\ge d_2\mid\ell^*}\le\frac{\alpha\cdot d_2^2L^2}{\lambda\cdot(d_2-d_1-2d)}+\frac14\cdot\frac{2^{-3L\cdot d_1}}{\gamma(\ell^*)}.
$$
\end{lemma}
\begin{proof}
Let $\lx,\ly$ be sampled from $\gamma$ conditioned on $\lx\in\supX{0},\ly\in\supY{0}$.
Let $\bell$ be its corresponding leaf in $\bar\Ccal$ and $\D$ be the depth of $\bell$.
By \Cref{clm:finite_steps}, $\bell$ always has finite depth.
We extend $\supa{t}=\supb{t}=0^{n\times n}$ and $\supX{t}=\supX{\D},\supY{t}=\supY{\D}$ for all $t>\D$.
Then define
$$
\K(\lx,\ly)=\sum_{t=d_1+1}^{d_2}\pbra{\abra{\lx\tensor \lx,\supa{t}}^2+\abra{\ly\tensor\ly,\supb{t}}^2}
\quad\text{and}\quad
K=\E_{\lx,\ly\sim\gamma}\sbra{\K(\lx,\ly)\mid\ell^*},
$$
where $\supa{\cdot}$'s and $\supb{\cdot}$'s depend only on $\bell$.\footnote{Note that $\bell$ specifies all the communication messages, which allows us to simulate the protocol and obtain each $\supa{\cdot}$ and $\supb{\cdot}$.}
Equivalently, we can write $K$ as
$$
K=\E_{\lx,\ly\sim\gamma}\sbra{\K(\lx,\ly)\mid(\lx,\ly)\in X^{(d_1)}\times Y^{(d_1)}},
$$
where $X^{(d_1)}$ and $Y^{(d_1)}$ are fixed due to $\ell^*$.

Observe that for any fixed $t\ge d_1$, $\supX{t}\times \supY{t}$ induced by different $\bell$, conditioned on $\ell^*$, is a disjoint partition of $X^{(d_1)}\times Y^{(d_1)}$. 
Therefore sampling $\lx,\ly\sim\gamma$ conditioned on $(\lx,\ly)\in X^{(d_1)}\times Y^{(d_1)}$ is equivalent to 
\begin{itemize}
\item first sample random messages $\bell'=(\supcbar{d_1+1},\ldots,\supcbar{t})$ conditioned on $\ell^*$,
\item then sample $\lx,\ly\sim\gamma$ conditioned on $(\lx,\ly)\in \supX{t}\times \supY{t}$ given $\bell'$.
\end{itemize}
Note that we can further expand $\bell'$ to a leaf $\bell$ as a full communication path, and obtain the following equivalent sampling process:
\begin{itemize}
\item Sample a random leaf $\bell$ conditioned on $\ell^*$.
\item Sample $\lx,\ly\sim\gamma$ conditioned on $(\lx,\ly)\in \supX{t}\times \supY{t}$ defined by the first $t$ messages of $\bell$.
\end{itemize}
As a result, we have
\begin{align*}
K
&=\sum_{t=d_1+1}^{d_2}\E_{\bell}\sbra{\E_{\lx,\ly\sim\gamma}\sbra{\abra{\lx\tensor \lx,\supa{t}}^2+\abra{\ly\tensor \ly,\supb{t}}^2\mid(\lx,\ly)\in \supX{t}\times \supY{t}}\mid\ell^*}\\
&=\E_{\bell}\sbra{\sum_{t=d_1+1}^{d_2}\E_{\lx\sim\gamma}\sbra{\abra{\lx\tensor \lx,\supa{t}}^2\mid \lx\in \supX{t}}+\E_{\ly\sim\gamma}\sbra{\abra{\ly\tensor \ly,\supb{t}}^2\mid \ly\in \supY{t}}\mid\ell^*}.
\end{align*}
Observe that there are at most $2d$ many step 3(a) and 3(b) in $\bell$.
This means, if $\D\ge d_2$, then from the $(d_1+1)$-th to the $d_2$-th communication steps, there are at least $d_2-d_1-2d$ cleanup steps (i.e., step 3(c)), each of which contributes at least $\lambda$ to $K$.
Thus we can lower bound $K$ by
\begin{equation}\label{eq:lem:depth_tail_bound_1}
K\ge \lambda\cdot(d_2-d_1-2d)\cdot\Pr\sbra{\D\ge d_2\mid\ell^*}.
\end{equation}

On the other hand by \Cref{clm:finite_steps}, there are at most $n^2$ non-zero $\supa{\cdot}$'s and at most $n^2$ non-zero $\supb{\cdot}$'s in each communication path.
Thus
\begin{equation}\label{eq:lem:depth_tail_bound_2}
\K(\lx,\ly)\le n^2\cdot\pbra{\max_{x\in \supX{0}}\frob{x\tensor x}^2+\max_{y\in \supY{0}}\frob{y\tensor y}^2}<2n^4T^2.
\end{equation}

We now obtain another upper bound using \Cref{thm:quadratic_concentration}.
Let $\bar\bell=(\supcbar{1},\ldots,\supcbar{d_2})$ extend $\ell^*$ for the next $d_2-d_1$ messages.\footnote{If $\bar\bell$ becomes a leaf before $d_2$, then we can simply pad dummy messages to it.}
Then $K=\E_{\bar\bell}\sbra{\K(\bar\bell)\mid\ell^*}$ where $
\K(\bar\ell):=\E_{\lx,\ly\sim\gamma}\sbra{\K(\lx,\ly)\mid\bar\ell}$.
Note that $\bar\ell$ fixes $a^{(\cdot)}$'s and $b^{(\cdot)}$'s in $\K(\lx,\ly)$.
Therefore we use $\K_{\bar\ell}(\lx,\ly)$ to denote $\K(\lx,\ly)$ with the directions $a^{(\cdot)}$'s and $b^{(\cdot)}$'s fixed by $\bar\ell$.
We now continue the bound on $\K(\bar\ell)$:
\begin{align}
\K(\bar\ell)
&\le\sum_{t=0}^{\infty}\Pr_{\lx,\ly\sim\gamma}\sbra{\K_{\bar\ell}(\lx,\ly)\ge t\mid\bar\ell}
=\sum_{t=0}^{\infty}\frac{\Pr_{\lx,\ly\sim\gamma}\sbra{\K_{\bar\ell}(\lx,\ly)\ge t,\bar\ell}}{\Pr_{\lx,\ly\sim\gamma}\sbra{\bar\ell}}
\notag\\
&=\sum_{t=0}^{\infty}\min\cbra{1,\frac{\Pr_{\lx,\ly\sim\gamma}\sbra{\K_{\bar\ell}(\lx,\ly)\ge t,\bar\ell}}{\gamma(\bar\ell)}}
\tag{by the definition of $\gamma(\cdot)$}\\
&\le\sum_{t=0}^{\infty}\min\cbra{1,\frac{\Pr_{\lx,\ly\sim\gamma}\sbra{\K_{\bar\ell}(\lx,\ly)\ge t}}{\gamma(\bar\ell)}}.
\label{eq:lem:depth_tail_bound_3}
\end{align}

We now analyze $\Pr_{\lx,\ly\sim\gamma}\sbra{\K_{\bar\ell}(\lx,\ly)\ge t}$ using \Cref{thm:quadratic_concentration}.
Since $a^{(t)},b^{(t)}$ cannot be non-zero simultaneously, we rearrange the matrices and assume $a^{(d_1+1)},\ldots,a^{(d')},b^{(d'+1)},\ldots,b^{(d'')}$ are the only non-zero matrices where $d''\le d_2$.
Then
$$
\K_{\bar\ell}(\lx,\ly)=\sum_{t=d_1+1}^{d'}\abra{\lx\tensor \lx,a^{(t)}}^2+\sum_{t=d'+1}^{d''}\abra{\ly\tensor \ly,b^{(t)}}^2.
$$
Note that $a$'s (resp., $b$'s) satisfy the condition in \Cref{thm:quadratic_concentration}. 
Let $1/\kappa$ be the constant\footnote{In particular $\kappa=56448$ suffices from our proof in \Cref{app:thm:quadratic_concentration}.} in $\Omega$ in \Cref{thm:quadratic_concentration}.
Hence
\begin{align*}
\Pr\sbra{\K_{\bar\ell}(\lx,\ly)\ge t}
&\le\Pr\sbra{\sum_{t=d_1+1}^{d'}\abra{\lx\tensor \lx,a^{(t)}}^2\ge t/2}+\Pr\sbra{\sum_{t=d'+1}^{d''}\abra{\ly\tensor \ly,b^{(t)}}^2\ge t/2}\\
&\le2\exp\cbra{-\frac1\kappa\cdot\frac{t/2}{d'-d_1+\sqrt{t/2}}}+2\exp\cbra{-\frac1\kappa\cdot\frac{t/2}{d''-d'+\sqrt{t/2}}}
\tag{by \Cref{thm:quadratic_concentration} and assuming $t\ge196\cdot\max\cbra{d'-d_1,d''-d'}$}\\
&\le4\exp\cbra{-\frac1\kappa\cdot\frac{t/2}{d_2-d_1+\sqrt{t/2}}}.
\tag{since $d_1\le d'\le d''\le d_2$}
\end{align*}
Thus for any $t\ge196\cdot(d_2-d_1)\ge196\cdot\max\cbra{d'-d_1,d''-d'}$, we have
\begin{equation}\label{eq:lem:depth_tail_bound_5}
\Pr\sbra{\K_{\bar\ell}(\lx,\ly)\ge t}\le4\exp\cbra{-\frac1\kappa\cdot\frac{t/2}{d_2-d_1+\sqrt{t/2}}}.
\end{equation}

For $\gamma(\bar\ell)\ge2^{-3L\cdot d_2}$, we plug \Cref{eq:lem:depth_tail_bound_5} into \Cref{eq:lem:depth_tail_bound_3} and obtain
\begin{align}
\K(\bar\ell)
&\le\sum_{t=0}^{196\cdot(d_2-d_1)^2}1+\sum_{t>196\cdot(d_2-d_1)^2}\min\cbra{1,2^{3L\cdot d_2+1}\cdot\exp\cbra{-\frac1\kappa\cdot\frac{t/2}{d_2-d_1+\sqrt{t/2}}}}
\tag{by \Cref{eq:lem:depth_tail_bound_5}}\\
&\le196\cdot(d_2-d_1)^2+1+\sum_{t\ge196\cdot(d_2-d_1)^2}\min\cbra{1,2^{3L\cdot d_2+1}\cdot e^{-\frac1\kappa\cdot\frac{t/2}{2\sqrt{t/2}}}}
\notag\\
&\le197\cdot d_2^2+\sum_{t\ge1}\min\cbra{1,2^{3L\cdot d_2+1}\cdot e^{-\frac{\sqrt{t/2}}{2\kappa}}}
\notag\\
&\le\alpha\cdot d_2^2L^2,
\label{eq:lem:depth_tail_bound_6}
\end{align}
where $\alpha$ is another universal constant.
Now we have
$$
K
=\E_{\bar\bell}\sbra{\K(\bar\bell)\mid\ell^*}
=\sum_{\bar\ell}\frac{\gamma(\bar\ell)}{\gamma(\ell^*)}\cdot \K(\bar\ell)
=\sum_{\bar\ell:\gamma(\bar\ell)<2^{-3L\cdot d_2}}\frac{\gamma(\bar\ell)}{\gamma(\ell^*)}\cdot \K(\bar\ell)+\sum_{\bar\ell:\gamma(\bar\ell)\ge2^{-3L\cdot d_2}}\frac{\gamma(\bar\ell)}{\gamma(\ell^*)}\cdot \K(\bar\ell),
$$
where the first summation can be bounded by
\begin{align*}
\sum_{\bar\ell:\gamma(\bar\ell)<2^{-3L\cdot d_2}}\frac{\gamma(\bar\ell)}{\gamma(\ell^*)}\cdot \K(\bar\ell)
&\le\frac{2^{-3L\cdot d_1}}{\gamma(\ell^*)}\cdot\sum_{\bar\ell}2^{-3L\cdot(d_2-d_1)}\cdot n^4T^2
\tag{by \Cref{eq:lem:depth_tail_bound_2}}\\
&\le\frac{2^{-3L\cdot d_1}}{\gamma(\ell^*)}\cdot2^{2L\cdot(d_2-d_1)}\cdot2^{-3L\cdot(d_2-d_1)}\cdot n^4T^2
\tag{since $\ell^*$ is fixed and each message is at most $2L$ bits}\\
&=\frac{2^{-3L\cdot d_1}}{\gamma(\ell^*)}\cdot\frac{2n^4T^2}{2^L}
\tag{since $d_2-d_1\ge1$}
\end{align*}
and the second summation is bounded by
\begin{equation*}
\sum_{\bar\ell:\gamma(\bar\ell)\ge2^{-3L\cdot d_2}}\frac{\gamma(\bar\ell)}{\gamma(\ell^*)}\cdot \K(\bar\ell)
\le\sum_{\bar\ell}\frac{\gamma(\bar\ell)}{\gamma(\ell^*)}\cdot\alpha\cdot d_2^2L^2
=\alpha\cdot d_2^2L^2.
\tag{by \Cref{eq:lem:depth_tail_bound_6}}
\end{equation*}
Then combining \Cref{eq:lem:depth_tail_bound_1}, we have
$$
\lambda\cdot(d_2-d_1-2d)\cdot\Pr\sbra{\D\ge d_2\mid\ell^*}\le \alpha\cdot d_2^2L^2+\frac{2^{-3L\cdot d_1}}{\gamma(\ell^*)}\cdot\frac{2n^4T^2}{2^L}.
$$
Assume $2^L\ge8n^4T^2$ and $d_2-d_1\ge2d+1$. Then
\begin{equation*}
\Pr\sbra{\D\ge d_2\mid\ell^*}\le\frac{\alpha\cdot d_2^2L^2}{\lambda\cdot(d_2-d_1-2d)}+\frac14\cdot\frac{2^{-3L\cdot d_1}}{\gamma(\ell^*)}.
\tag*{\qedhere}
\end{equation*}
\end{proof}

\begin{corollary}\label{cor:depth_tail_bound}
Assume $\gamma^*\ge3/4$, $T\le n$, $L\ge\Theta(\log(n))$, and $\lambda\ge\Theta(dL^2\log^2(n))$. Then for each $k=0,1,\ldots,4\log(n)$, we have
$$
\Pr\sbra{\D\ge4kd}\le2^{-k}+\frac k{n^5}.
$$
\end{corollary}
\begin{proof}
We prove the bound by induction on $k$. The base case $k=0$ is trivial.
For the inductive case, let $\ell^*$ be the first $4(k-1)d$ communication messages. Then we bound
$$
P:=\sum_{\ell^*:\gamma(\ell^*)/\gamma^*<2^{-3L\cdot4(k-1)d}}\frac{\gamma(\ell^*)}{\gamma^*}\cdot\Pr\sbra{\D\ge4kd\mid\ell^*}
$$
and
$$
Q:=\sum_{\ell^*:\gamma(\ell^*)/\gamma^*\ge2^{-3L\cdot4(k-1)d}}\frac{\gamma(\ell^*)}{\gamma^*}\cdot\Pr\sbra{\D\ge4kd\mid\ell^*}
$$
separately.

For $P$, observe that if $k=1$ then $\ell^*$ is root of the protocol, thus $\gamma(\ell^*)=\gamma^*$ and $P=0$. 
On the other hand, if $k\ge2$, then
\begin{align*}
P
&\le\sum_{\ell^*:\gamma(\ell^*)/\gamma^*<2^{-3L\cdot4(k-1)d}}2^{-3L\cdot4(k-1)d}
\le\sum_{\ell^*}2^{-3L\cdot4(k-1)d}\\
&\le2^{2L\cdot4(k-1)d}\cdot2^{-3L\cdot4(k-1)d}
\tag{each communication message is at most $2L$ bits}\\
&=2^{-L\cdot4(k-1)d}\le n^{-5}.
\tag{since $k\ge2$ and $L\ge\Theta(\log(n))$}
\end{align*}
Now we turn to $Q$.
Applying \Cref{lem:depth_tail_bound} with $\ell^*$ and $d_1=4(k-1)d,d_2=4kd$, we have
\begin{align*}
Q
&\le\sum_{\ell^*:\gamma(\ell^*)/\gamma^*\ge2^{-3L\cdot4(k-1)d}}\frac{\gamma(\ell^*)}{\gamma^*}\cdot\pbra{\frac{16\alpha\cdot k^2d^2L^2}{2dR}+\frac14\cdot\frac{2^{-3L\cdot4(k-1)d}}{\gamma(\ell^*)}}\\
&\le\sum_{\ell^*}\frac{\gamma(\ell^*)}{\gamma^*}\cdot\pbra{\frac{8\alpha\cdot k^2dL^2}{\lambda}+\frac1{4\gamma^*}}\\
&=\Pr\sbra{\D\ge4(k-1)d}\cdot\pbra{\frac{8\alpha\cdot k^2dL^2}{\lambda}+\frac1{4\gamma^*}}\\
&\le\Pr\sbra{\D\ge4(k-1)d}\cdot\frac12
\tag{since $\gamma^*\ge3/4$ and $\lambda\ge\Theta(dL^2\log^2(n)),k\le4\log(n)$}\\
&\le\pbra{2^{-(k-1)}+\frac{k-1}{n^5}}\cdot\frac12
\le2^{-k}+\frac{k-1}{n^5}.
\tag{by induction hypothesis}
\end{align*}
By adding up $P$ and $Q$, we complete the induction.
\end{proof}

Given \Cref{cor:depth_tail_bound} and suitable choice of the parameters, we now prove the second moment bound.
\begin{proof}[Proof of \Cref{lem:second_moment}]
With $L=\Theta(\log(n))$, $T=\Theta(\sqrt{\log(n)})$, and $\lambda=\Theta(d\log^4(n))$, by \Cref{fct:gammastar}, we have $\gamma^*\ge3/4$.
Therefore the second moment of $\D$ is 
\begin{align*}
\E[\D^2]
&\le\sum_{k=0}^{4\log(n)}\pbra{4(k+1)d}^2\cdot\Pr\sbra{\D\ge4kd}+\Pr\sbra{\D\ge16 d\log(n)}\cdot(2n^2)^2
\tag{by \Cref{clm:finite_steps}}\\
&\le\sum_{k=0}^{4\log(n)}\pbra{4(k+1)d}^2\cdot\pbra{2^{-k}+\frac k{n^5}}+\pbra{n^{-4}+\frac{4\log(n)}{n^5}}\cdot(2n^2)^2
\tag{by \Cref{cor:depth_tail_bound}}\\
&=O(d^2).
\tag*{\qedhere}
\end{align*}
\end{proof}
\section{Fourier Growth Reductions For General Gadgets}\label{sec:gadget}

In this section, we show that Fourier growth bounds of communication protocols for general (constant-sized) gadgets can be reduced to the bounds of XOR-fiber, and vice versa.
This implies that in the study of Fourier growth, they are all equivalent.

Let $m_1,m_2$ be two positive integers.
Let $g\colon\binpm^{m_1}\times\binpm^{m_2}\to\binpm$ be a gadget.
Recall that $\unif$ is the uniform distribution over $\binpm^n$.
We now use $\unif_1,\unif_2,\bar\unif_1,\bar\unif_2$ to denote the uniform distributions over $\binpm^{m_1},\binpm^{m_2},(\binpm^{m_1})^n,(\binpm^{m_2})^n$ respectively.
We define the $g$-fiber of communication protocols similar to the XOR-fiber:

\begin{definition}\label{def:g-fiber}
For any randomized two-party protocol $\Ccal\colon(\binpm^{m_1})^n\times(\binpm^{m_2})^n\to[-1,1]$, its $g$-fiber, denoted by $\Ccal_{\downarrow g}\colon\binpm^n\to[-1,1]$, is defined by
$$
\Ccal_{\downarrow g}(z)=\E_{\xbm\sim\bar\unif_1,\ybm\sim\bar\unif_2}\sbra{\Ccal(\xbm,\ybm)\mid g(\xbm_i,\ybm_i)=z_i,~\forall i},
$$
where the expectation is also over the internal randomness of $\Ccal$.
\end{definition}

To compare the Fourier growth bounds between gadgets, we use $L_{1,k}(g,d,m_1,m_2,n)$ to denote the upper bound of the level-$k$ Fourier growth for the $g$-fiber of an arbitrary randomized communication protocol $\Ccal\colon(\binpm^{m_1})^n\times(\binpm^{m_2})^n\to[-1,1]$ with at most $d$ bits of communication, where $g\colon\binpm^{m_1}\times\binpm^{m_2}\to\binpm$ is the gadget.
Since randomized protocols are convex combinations of deterministic protocols of the same cost, using this notation, our main results \Cref{thm:boolean_bound_level_one,thm:boolean_bound_level_two} can be rephrased as
$$
L_{1,1}(\mathrm{XOR},d,1,1,n)\le O\pbra{\sqrt d}
\quad\text{and}\quad
L_{1,2}(\mathrm{XOR},d,1,1,n)\le O\pbra{d^{3/2}\log^3(n)}.
$$

For any set $S\subseteq[m_1]$, define $x_S=\prod_{i\in S}x_i$, and similarly for $y_T$ with $T\subseteq[m_2]$.
Similar to the standard Fourier representation of Boolean functions, the gadget $g$, which is a two-party function, also has Fourier representation:
$$
g(x,y)=\sum_{S\subseteq[m_1],T\subseteq[m_2]}\hat g(S,T)\cdot x_Sy_T,
\quad\text{where}\quad
\hat g(S,T)=\E_{\xbm\sim\unif_1,\ybm\sim\unif_2}\sbra{g(\xbm,\ybm)\cdot\xbm_S\ybm_T}.
$$

For convenience, we will assume $g$ satisfies the following assumption.
It's easy to see that the XOR gadget satisfies it.
\begin{assumption}\label{as:balance}
$\hat g(S,T)=0$ if $S=\emptyset$ or $T=\emptyset$.
\end{assumption}
\begin{remark}
This assumption is equivalent to the fact that, restricted on any input to Alice's side, the remaining function on Bob's side is balanced, and vice versa.

Even if $g$ does not satisfy the assumption, then we can embed it inside a similar gadget $g'\colon\binpm^{m_1+1}\times\binpm^{m_2+1}\to\binpm$, where we XOR the last bit of Alice and the last bit of Bob to the old gadget $g$ applied to Alice's first $m_1$ bits and Bob's first $m_2$ bits, i.e.,
$$
g'(x,y)=x_{m_1+1}y_{m_2+1}\cdot g(x_{\le m_1},y_{\le m_2}).
$$
Then $g'$ satisfies the assumption and inherits most properties of $g$ sufficient for studies in communication complexity tasks.
\end{remark}

Now for a protocol $\Ccal\colon(\binpm^{m_1})^n\times(\binpm^{m_2})^n\to[-1,1]$, it is also a two-party function and thus admitting similar Fourier representation.
We view an input from $(\binpm^{m_1})^n$ as indexed by a tuple in $[n]\times[m_1]$.
Therefore any subset of $(\binpm^{m_1})^n$ is uniquely identified as $\bigcup_{i\in[n]}\cbra{i}\times S_i$, where each $S_i\subseteq[m_1]$.
We use $S^{[n]}$ to denote $(S_i)_{i\in[n]}$.
Thus the Fourier coefficients of $\Ccal$ can be written as
$$
\hat\Ccal(S^{[n]},T^{[n]}):=\hat\Ccal\pbra{\bigcup_{i\in[n]}\cbra{i}\times S_i,\bigcup_{i\in[n]}\cbra{i}\times T_i},
$$
and the Fourier representation of $\Ccal$ is
$$
\Ccal(x,y)=
\sum_{S^{[n]},J^{[n]}}\hat\Ccal(S^{[n]},T^{[n]})\cdot\prod_{i\in[n]}x_{i,S_i}\cdot\prod_{j\in[n]}y_{j,T_j},
$$
where $x_{i,S}=\prod_{j\in S}x_{i,j}$ and similar for $y_{j,T}$.

Under this notation and assuming \Cref{as:balance}, we can effectively compute the Fourier coefficients of any $g$-fiber.
\begin{fact}\label{fct:g-fiber_fourier}
Assume gadget $g\colon\binpm^{m_1}\times\binpm^{m_2}\to\binpm$ satisfies \Cref{as:balance}.
Then we have
$$
\hat{\Ccal_{\downarrow g}}(I)
=\sum_{\substack{S^I,T^I\\S_i\neq\emptyset,T_i\neq\emptyset,\forall i\in I}}\hat\Ccal(S^I,T^I)\cdot\prod_{i\in I}\hat g(S_i,T_i)
\quad
\text{for any $I\subseteq[n]$,}
$$
where we use $S^I$ to denote $S^{[n]}$ with $S_j$ fixed to $\emptyset$ for all $j\notin I$.
\end{fact}
\begin{proof}
Observe that
\begin{align*}
\hat{\Ccal_{\downarrow g}}(I)
&= \E_{\zbm \sim \unif}\sbra{\Ccal_{\downarrow g}(\zbm) \cdot \prod_{i\in I}\zbm_i}\\
&= \E_{\zbm \sim \unif}\sbra{\E_{\xbm\sim\bar\unif_1,\ybm \sim \bar\unif_2}\sbra{\Ccal(\xbm,\ybm) \mid g(\xbm_i, \ybm_i)=\zbm_i,~\forall i} \cdot \prod_{i\in I} \zbm_i}\\
&= \E_{\zbm \sim \unif}\sbra{  \E_{\xbm\sim\bar\unif_1,\ybm \sim \bar\unif_2}\sbra{ \Ccal(\xbm,\ybm) \cdot \prod_{i\in I} g(\xbm_i, \ybm_i) \mid g(\xbm_i, \ybm_i)=\zbm_i,~\forall i } }.
\end{align*}
Since $\hat{g}(\emptyset, \emptyset) = 0$ by \Cref{as:balance}, every pair $(x,y)$ is sampled with the same probability under the conditional distribution. 
Thus we get 
$$
\hat{\Ccal_{\downarrow g}}(I) = \E_{\xbm\sim\bar\unif_1,\ybm\sim \bar\unif_2}\sbra{\Ccal(\xbm,\ybm) \cdot \prod_{i\in I} g(\xbm_i, \ybm_i)}.
$$
Now we expand $\Ccal$ and $g$ in the Fourier basis and obtain
\begin{align*}
\hat{\Ccal_{\downarrow g}}(I) 
&=
\E_{\xbm\sim\bar\unif_1,\ybm\sim\bar\unif_2}\sbra{
\pbra{\sum_{S^{[n]},T^{[n]}}
\hat\Ccal(S^{[n]},T^{[n]})\prod_{i\in[n]}\xbm_{i,S_i}\prod_{j\in[n]}\ybm_{j,T_j}}
\cdot
\prod_{i\in I}\pbra{
\sum_{S_i,T_i}\hat g(S_i,T_i)\xbm_{i,S_i}\ybm_{i,T_i}}}\\
&=
\E_{\xbm\sim\bar\unif_1,\ybm\sim\bar\unif_2}\sbra{
\pbra{\sum_{S^{[n]},T^{[n]}}
\hat\Ccal(S^{[n]},T^{[n]})\prod_{i\in[n]}\xbm_{i,S_i}\prod_{j\in[n]}\ybm_{j,T_j}}
\pbra{
\sum_{S^I,T^I}
\prod_{i\in I}\hat g(S_i,T_i)\xbm_{i,S_i}\ybm_{i,T_i}}}\\
&=
\sum_{S^I,T^I}\hat\Ccal(S^I,T^I)\cdot\prod_{i\in I}\hat g(S_i,T_i)\\
&=\sum_{\substack{S^I,T^I\\S_i\neq\emptyset,T_i\neq\emptyset,\forall i\in I}}\hat\Ccal(S^I,T^I)\cdot\prod_{i\in I}\hat g(S_i,T_i),
\tag{by \Cref{as:balance}}
\end{align*}
as desired.
\end{proof}

Now we present the reduction from XOR-fiber to a general $g$-fiber.
\begin{theorem}\label{thm:xor_to_g}
Assume gadget $g\colon\binpm^{m_1}\times\binpm^{m_2}\to\binpm$ satisfies \Cref{as:balance}. Then
\begin{align*}
L_{1,k}(\mathrm{XOR},d,1,1,n)
&\le\pbra{\max_{S,T}|\hat g(S,T)|}^{-k}\cdot L_{1,k}(g,d,m_1,m_2,n)\\
&\le2^{(m_1+m_2)\cdot k/2}\cdot L_{1,k}(g,d,m_1,m_2,n).
\end{align*}
\end{theorem}
\begin{proof}
Let $\Ccal\colon\binpm^n\times\binpm^n\to[-1,1]$ be an arbitrary protocol of cost at most $d$.
Then for a fixed set $I\subseteq[n]$, by \Cref{fct:g-fiber_fourier} applied to the XOR gadget, we have
\begin{equation}\label{eq:lem:xor_to_g_1}
\hat{\Ccal_{\downarrow\mathrm{XOR}}}(I)=\hat\Ccal(1^I,1^I).
\end{equation}
Let $S\subseteq[m_1]$ and $T\subseteq[m_2]$ maximize $|\hat g(S,T)|$.
Since $g$ satisfies \Cref{as:balance}, we know $S$ and $T$ are not empty sets.

Now define a different protocol $\Ccal'\colon(\binpm^{m_1})^n\times(\binpm^{m_2})^n\to[-1,1]$ as follows:
After receiving input $x$, Alice computes $x'_i=x_{i,S}$ for each block $x_i$; and Bob computes similarly $y'_i=y_{i,T}$ upon receiving input $y$.
Then they execute the protocol $\Ccal$ on $x'$ and $y'$.
That is, $\Ccal'(x,y)=\Ccal(x',y')$.
Therefore, for any $I\subseteq[n]$ and $S^I,T^I$ satisfying $S_i\neq\emptyset,T_i\neq\emptyset$ for $i\in I$, we have
$$
\hat{\Ccal'}(S^I,T^I)=
\begin{cases}
\hat\Ccal(1^I,1^I) & S_i=S,T_i=T,~\forall i\in I,\\
0 & \text{otherwise.}
\end{cases}
$$
Then by \Cref{eq:lem:xor_to_g_1} and \Cref{fct:g-fiber_fourier} applied to $\Ccal'$ with gadget $g$, we have
$$
\hat{\Ccal_{\downarrow g}'}(I)
=\hat\Ccal(1^I,1^I)\cdot\hat g(S,T)^{|I|}
=\hat{\Ccal_{\downarrow\mathrm{XOR}}}(I)\cdot\hat g(S,T)^{|I|}.
$$
Now summing over all $I\subseteq[n]$ of size $k$, we have
\begin{align*}
L_{1,k}(\Ccal_{\downarrow\mathrm{XOR}})
&=\sum_{I\subseteq[n]:|I|=k}\abs{\hat{\Ccal_{\downarrow\mathrm{XOR}}}(I)}
=|\hat g(S,T)|^{-k}\cdot\sum_{I\subseteq[n]:|I|=k}\abs{\hat{\Ccal_{\downarrow g}'}(I)}
=|\hat g(S,T)|^{-k}\cdot L_{1,k}(\Ccal'_{\downarrow g})\\
&\le
|\hat g(S,T)|^{-k}\cdot L_{1,k}(g,d,m_1,m_2,n).
\tag{since $\Ccal'$ has cost at most $d$}
\end{align*}
Since $\Ccal$ is arbitrary, this proves the first half of \Cref{thm:xor_to_g}.
To prove the second half, we use an averaging argument and Parseval's identity on $g$:
\begin{equation*}
|\hat g(S,T)|
\ge\sqrt{2^{-m_1-m_2}\sum_{S',T'}\hat g(S',T')^2}
=\sqrt{2^{-m_1-m_2}}.
\tag*{\qedhere}
\end{equation*}
\end{proof}

Using similar analysis, we also have a reduction from a general $g$-fiber to XOR-fiber.
\begin{theorem}\label{thm:g_to_xor}
Assume gadget $g\colon\binpm^{m_1}\times\binpm^{m_2}\to\binpm$ satisfies \Cref{as:balance}. Then
\begin{align*}
L_{1,k}(g,d,m_1,m_2,n)
&\le\pbra{\sum_{S,T}|\hat g(S,T)|}^k\cdot L_{1,k}(\mathrm{XOR},d,1,1,n)\\
&\le2^{(m_1+m_2)\cdot k/2}\cdot L_{1,k}(\mathrm{XOR},d,1,1,n).
\end{align*}
\end{theorem}
\begin{proof}
Let $\Ccal\colon(\binpm^{m_1})^n\times(\binpm^{m_2})^n\to[-1,1]$ be an arbitrary protocol of cost at most $d$.
Then for a fixed set $I\subseteq[n]$, by \Cref{fct:g-fiber_fourier} applied to gadget $g$ and using \Cref{as:balance}, we have
$$
\hat{\Ccal_{\downarrow g}}(I)
=\sum_{S^I,T^I}\hat\Ccal(S^I,T^I)\cdot\prod_{i\in I}\hat g(S_i,T_i).
$$
Therefore
$$
L_{1,k}(\Ccal_{\downarrow g})
\le\sum_{I\subseteq[n]:|I|=k}\sum_{S^I,T^I}\abs{\hat\Ccal(S^I,T^I)}\cdot\abs{\prod_{i\in I}\hat g(S_i,T_i)}.
$$

Now let $M=\sum_{S,T}|\hat g(S,T)|$.
Let $\rho$ be a distribution over subsets of $[m_1]\times[m_2]$ and its probability density function is defined as:
$$
\rho(S,T)=|\hat g(S,T)|/M.
$$
Then we can rewrite $L_{1,k}(\Ccal_{\downarrow g})$ as
\begin{align}
L_{1,k}(\Ccal_{\downarrow g})
&\le\sum_{I\subseteq[n]:|I|=k}\E_{(\Sbm^I,\Tbm^I)\sim\rho^I}\sbra{\abs{\hat\Ccal(\Sbm^I,\Tbm^I)}\cdot M^k}
\notag\\
&=M^k\cdot\E_{(\Sbm^{[n]},\Tbm^{[n]})\sim\rho^{[n]}}\sbra{\sum_{I\subseteq[n]:|I|=k}\abs{\hat\Ccal(\Sbm^I,\Tbm^I)}}.
\label{eq:lem:g_to_xor_1}
\end{align}

Now we fix an arbitrary $(S^{[n]},T^{[n]})$ sampled from $\rho^{[n]}$.
Note that $S_i$ and $T_i$ are not empty by the definition of $\rho$ and \Cref{as:balance}.
Then define a different protocol $\Ccal'\colon\binpm^n\times\binpm^n\to[-1,1]$ as follows:
After receiving input $x$, Alice samples $x'\in(\binpm^{m_1})^n$ uniformly conditioned on $x'_{i,S_i}=x_i$ for all $i\in[n]$; and Bob samples similarly $y'\in(\binpm^{m_2})^n$ conditioned on $y'_{i,T_i}=y_i$ for all $i\in[n]$.
Then they execute the protocol $\Ccal$ on $x'$ and $y'$.
That is, $\Ccal'(x,y)=\E_{\xbm',\ybm'}[\Ccal(\xbm',\ybm')]$.
Therefore, for any $I\subseteq[n]$, we have
$$
\hat{\Ccal'}(1^I,1^I)=\hat\Ccal(S^I,T^I).
$$
By \Cref{fct:g-fiber_fourier} applied to $\Ccal'$ and the XOR gadget, we have
$$
\hat{\Ccal'_{\downarrow\mathrm{XOR}}}(I)=\hat{\Ccal'}(1^I,1^I)=\hat\Ccal(S^I,T^I).
$$
Since $\Ccal'$ has cost at most $d$, we have
$$
\sum_{I\subseteq[n]:|I|=k}\abs{\hat\Ccal(S^I,T^I)}
=\sum_{I\subseteq[n]:|I|=k}\abs{\hat{\Ccal'_{\downarrow\mathrm{XOR}}}(I)}
=L_{1,k}(\Ccal'_{\downarrow\mathrm{XOR}})
\le L_{1,k}(\mathrm{XOR},d,1,1,n).
$$
Putting back to \Cref{eq:lem:g_to_xor_1}, we have
$$
L_{1,k}(\Ccal_{\downarrow g})
\le M^k\cdot L_{1,k}(\mathrm{XOR},d,1,1,n),
$$
which proves the first half of \Cref{thm:g_to_xor} since $\Ccal$ is arbitrary.
To prove the second half, we use Cauchy-Schwarz inequality and Parseval's identity on $g$:
\begin{equation*}
M=\sum_{S,T}|\hat g(S,T)|\le\sqrt{2^{m_1+m_2}\sum_{S,T}\hat g(S,T)^2}=\sqrt{2^{m_1+m_2}}.
\tag*{\qedhere}
\end{equation*}
\end{proof}

As a corollary, to study the Fourier growth bounds, we can switch between gadgets conveniently, as long as the gadgets have small size.

\begin{corollary}\label{cor:g_to_g'}
Assume gadgets $g\colon\binpm^{m_1}\times\binpm^{m_2}\to\binpm$ and $g'\colon\binpm^{m_1'}\times\binpm^{m_2'}\to\binpm$ satisfy \Cref{as:balance}.
Then
$$
L_{1,k}(g,d,m_1,m_2,n)\le2^{(m_1+m_2+m_1'+m_2')\cdot k/2}\cdot L_{1,k}(g',d,m_1',m_2',n).
$$
\end{corollary}
\section{Directions Towards Further Improvements}\label{sec:future}

In this section we propose potential directions for further improving our second level bounds.
In \Cref{sec:lift}, we show that better Fourier growth bounds can be obtained from strong lifting theorems in a black-box way. This relies on the Fourier growth reductions in \Cref{sec:gadget}.
In \Cref{sec:improved-hw}, we examine the bottleneck in our analysis and identify major obstacles within.


\subsection{Better Lifting Theorems Imply Better Fourier Growth}\label{sec:lift}

Let $f:\pmone^n \to \pmone$ be a Boolean function. Let $g: \pmone^{m_1} \times \pmone^{m_2} \to \pmone$ be a gadget.
A lifting theorem connects the communication complexity of $f \circ g$ with the query complexity of $f$.
Some lifting theorems show that a low-cost communication protocol can be simulated by a low-cost query algorithm.

To be more precise, let $\Ccal: (\pmone^{m_1})^n \times (\pmone^{m_2})^n \to [-1,1]$ be a randomized two-party protocol.
Recall \Cref{def:g-fiber}, the $g$-fiber of $\Ccal$, denoted $\Ccal_{\downarrow g}(z): \pmone^{n} \to [-1,1]$, is defined by
$$
\Ccal_{\downarrow g}(z) = \E_{\bm{x}\sim\bar\unif_1, \bm{y}\sim \bar\unif_2}\left[ \Ccal(\bm{x},\bm{y})\mid g(\bm{x}_i, \bm{y}_i)=z_i,~\forall i\right].
$$
We say that $g$ satisfies a strong lifting theorem if for all randomized protocols $\Ccal$ of small communication bits, there is a randomized decision tree of small depth that approximates $\Ccal_{\downarrow g}$ on each input with error $1/\poly(n)$ (see e.g., \cite{GPW20}).

\begin{theorem}\label{thm:gadget}
Assume gadget $g\colon\binpm^{m_1}\times\binpm^{m_2}\to\binpm$ satisfies \Cref{as:balance}.
Assume for any randomized protocol $\Ccal\colon(\binpm^{m_1})^n\times(\binpm^{m_2})^n\to[-1,1]$ with at most $d$ bits of communication, there exists a randomized decision tree $\Tcal$ of depth at most $D$ that approximates $\Ccal_{\downarrow g}$ with pointwise error at most $1/n^k$, i.e.,
$$
\abs{\Tcal(z)-\Ccal_{\downarrow g}(z)}\le n^{-k}
\quad\forall z\in\binpm^n.
$$

Then, for any randomized protocol $\Ccal'\colon\binpm^n\times\binpm^n\to[-1,1]$ with at most $d$ bits of communication, its XOR-fiber $\Ccal'_{\downarrow\mathrm{XOR}}$ has level-$k$ Fourier growth
\begin{align*}
L_{1,k}(\Ccal'_{\downarrow\mathrm{XOR}})
&\le\pbra{\max_{S,T}|\hat g(S,T)|}^{-k}\cdot\sqrt{D^k\cdot O\pbra{\log(n)}^{k-1}}\\
&\le2^{(m_1+m_2)\cdot k/2}\cdot\sqrt{D^k\cdot O\pbra{\log(n)}^{k-1}}.
\end{align*}
\end{theorem}

As a simple corollary, we see that if the assumption of \Cref{thm:gadget} holds with $k=2$, $D= d \cdot \polylog(n)$, and a polylogarithmic-sized gadget $g$ (i.e., $2^{m_1},2^{m_2}\le\polylog(n)$), then the second level Fourier growth of the XOR-fiber of any randomized protocol of cost $d$ is at most $d\cdot\polylog(n)$ as desired.

We also remark that state-of-the-art lifting results hold with the gadget $g$ being either:
\begin{itemize}
\item The inner product on $m_1 = m_2 =  O(\log(n))$ bits~\cite{CFKMP19}. 
However, for such $g$ the largest Fourier coefficient squared is $1/\poly(n)$, which yields a trivial bound in Theorem~\ref{thm:gadget}.
\item The index function with $m_1 = \poly(n)$, $m_2 = \log(m_1)$~\cite{GPW20}.\footnote{For deterministic lifting, a better bound $m_1=O(n\log(n))$ is known \cite{lovett2022lifting}, but it doesn't suffice for our reduction.} 
In this case the largest Fourier coefficient squared is $1/m_1^2$, which again yields a trivial bound in Theorem~\ref{thm:gadget}.
Nonetheless, even a polynomial improvement on $m_1$, say $m_1 = n^{0.01}$, would give new non-trivial bounds in Theorem~\ref{thm:gadget} and in turn improves our lower bound on the XOR-lift of Forrelation.
\end{itemize}

\begin{proof}[Proof of \Cref{thm:gadget}]
Let $\Ccal\colon(\binpm^{m_1})^n\times(\binpm^{m_2})^n\to[-1,1]$ be a randomized protocol of cost at most $d$.
Then by assumption, $\Ccal_{\downarrow g}$ can be approximated up to error $1/n^k$ by a randomized decision tree $\Tcal$ of depth at most $D$.
Thus any Fourier coefficient of $\Ccal_{\downarrow g}$ and $\Tcal$ differs by at most $1/n^k$.
Therefore by the level-$k$ Fourier growth bounds on randomized decision trees \cite{Tal20,SSW21}, we have
$$
L_{1,k}(\Ccal_{\downarrow g})
\le \sum_{S\subseteq[n]:|S|=k}\pbra{n^{-k} + \abs{\hat{\Tcal}(S)}}
\le \sqrt{D^k\cdot O(\log(n))^{k-1}}.
$$
Since $\Ccal$ is arbitrary, the claimed bound for $\Ccal'_{\downarrow\mathrm{XOR}}$ follows from \Cref{thm:xor_to_g}.
 \end{proof}

\subsection{Sums of Squares of Quadratic Forms for Pairwise Clean Sets}
\label{sec:improved-hw}

In our analysis for the level-two bound, we showed that one can transform a general protocol to a $4$-wise clean protocol with parameter $\lambda = d\cdot\polylog(n)$ by adding $O(d)$ additional cleanup steps in expectation. If one could show that with essentially the same number of steps, one could take $\lambda = \polylog(n)$, then we would obtain the optimal level-two bound of $d \cdot \polylog(n)$.

We recall that to bound the number of cleanup steps, we rely on a concentration inequality for sums of squares of orthonormal quadratic forms (\Cref{thm:quadratic_concentration}), which says that if $M_1, \ldots, M_m$ are matrices with zero diagonal and form an orthonormal set when viewed as $n^2$ dimensional vectors,
then the random variable $\lQ = \sum_{i=1}^m \ip{\lX \tensor \lX}{M_i}^2$ satisfies $\Pr_{\lx \sim \gamma_n}[\lQ \ge t] \le e^{-\Omega(\sqrt{t})}$ for any  $t\gtrsim m^2$. 
Using this tail bound for $m= \Theta(d)$ and conditioning on $\lx \in X$ where $X$ is an arbitrary subset of $\Rbb^n$ with Gaussian measure $\approx 2^{-d}$, we obtained a bound $\BE_{\lx \sim \gamma}[\lQ \midd \lx \in X] \lesssim d^2$. 
This shows that there can be at most $O(d)$ such quadratic forms $M_i$'s where the value $\BE_{\lx \sim \gamma}\sbra{\ip{\lX \tensor \lX}{M_i}^2 \mid \lx \in X}$ can be larger than $d$ and hence, the reason we can only take $\lambda \approx d$. We note that the argument just described is for the non-adaptive setting, while in our case the $M_i$'s are also being chosen adaptively, so additional work is needed. 

The next example shows that the aforementioned statement is tight even in the non-adaptive setting where the $M_i$'s are fixed: in particular, there is a set $X$ of large measure and $\approx d$ such orthonormal quadratic forms where the above expectation after conditioning on $\lx \in X$ is $\Theta(d^2)$.

\begin{example} 
For $1\le i< j\le\sqrt d$, let $M_{ij} = E_{ij}$ for $i < j$ where $E_{ij}$ denotes the $n \times n$ matrix where only the $(i,j)$ entry is one. Note that the matrices $M_{ij}$ form an orthonormal set and they all have a zero diagonal. Let $X = \cbra{x \in \Rbb^n \mid |x_i| \gtrsim d^{1/4} \text{ for all $i \le d^{1/2}$}}$. Then, the Gaussian measure $\gamma(X) = 2^{-\Theta(d)}$ but 
\[ 
\BE_{\lx \sim \gamma}\sbra{ \sum_{1\le i< j\le \sqrt d} \ip{\lX \tensor \lX}{M_{ij}}^2 \mid \lx \in X} = \Theta(d^2).
\]
\end{example}

Note that the set $X$ in the example above is not pairwise clean and for our application, one can get around it by first ensuring that the protocol is pairwise clean and then proceeding with the 4-wise cleanup process. Motivated by this, we speculate that  when the set is pairwise clean, then the expected value of the sum of squares of orthonormal quadratic forms is much smaller unlike the example above.
Assuming such a statement and combining it with our ideas for handling the adaptivity suggests a potential way of improving the level-two bounds.

\bibliographystyle{alpha} 
\bibliography{ref}

\begin{comment}
\section{System Architecture}
\label{appendix:architecture}
\system has a novel modularized system architecture with three key components: 
\emph{StreamManager}, 
\emph{TxnManager} and \emph{TxnScheduler}. 
These components are instantiated in each thread locally.
The execution outline of \system is presented in Algorithm~\ref{alg:algo}.
Transactional stream processing is continuous and potentially never ends (Line 1$\sim$8).
The dependency resolution and execution of state transactions are separated into two non-overlapping phases by punctuations~\cite{Tucker:2003:EPS:776752.776780} (Line 2 and 5), which guarantees that no subsequent input event will have a smaller timestamp. 
Effectively, a batch of state transactions is collected during the first phase, and processed during the second phase.

In the first phase (i.e., stream processing phase), 
the \emph{StreamManager} conducts preprocessing for every input event ($e$). Similar to some prior works~\cite{tstream}, state transactions may be issued but not immediately processed during preprocessing (Line 3).
The \emph{pre\_processing} and \emph{post\_processing} functions are exposed as APIs to users.
The \emph{TxnManager} handles dependency resolution (Line 4) among state transactions and insert decomposed operations to construct a \tpg. We discuss the detailed two-phase \tpg construction process in Section~\ref{subsec:construction}.

In the second phase  (i.e., transaction processing phase), 
the \emph{TxnManager} is first involved again to refine (Line 6) the constructed \tpg with further dependency resolution.
The \emph{TxnScheduler} 
schedules operations for concurrent execution based on the constructed \tpg according to the three dimensions of scheduling decisions (Line 7). 
In particular, a scheduling decision model $M$ is instantiated based on the constructed \tpg (Line 14).
\textbf{\circled{1}} Guided by $M$, execution threads adopt an exploration strategy (Section~\ref{subsec:explore}) to explore the constructed \tpg for operations available to be scheduled constrained by dependencies. 
\textbf{\circled{2}} 
During exploration, one or multiple operations may be treated as the 
% basic 
unit of scheduling (Section~\ref{subsec:granularity}). 
Subsequently, \textbf{\circled{3}} every thread executes operation(s) in the unit of scheduling with various abort handling mechanisms (Section~\ref{subsec:abort_handling}).
Only when state transactions are processed (i.e., committed or aborted) can the associated input events be postprocessed (Line 8) by the \emph{StreamManager} based on transaction processing results.
\end{comment}

\begin{comment}
\begin{algorithm}
\footnotesize
    \KwData{$e$ \tcp{Input event}}
    \KwData{$txn_{ts}$ \tcp{State transaction}}
    \KwData{$G$ \tcp{The currently constructed TPG}}
    \While{!finish processing of input streams}{
        \eIf(\tcp*[h]{Phase 1}){\text{$e$ is not a $punctuation$}}{
                $txn_{ts}$ $\gets$ PRE\_Processing($e$)\;
                \textbf{TPG\_Construction}($G$, $txn_{ts}$)\; 
          }(\tcp*[h]{Phase 2}){
                \textbf{TPG\_Refinement}($G$)\; 
                \textbf{TXN\_Scheduling}($G$)\; 
                POST\_Processing()\;
          }
    }
    
    \SetKwFunction{FMain}{TPG\_Construction}
    \SetKwProg{Fn}{Function}{:}{}
    \Fn{\FMain{$G$, $txn_{ts}$}}{
        $O_{1..k}$ $\gets$ \textbf{Partition} $txn_{ts}$\;
        \ForEach{\text{operation $O_{i}$ $\in$ $O_{1..k}$}}{
            \textbf{Identify} its \ld\;
            $G$ $\gets$ $G$ + $O_{i}$ \;
        }
    }
    \SetKwFunction{FMain}{TPG\_Refinement}
    \SetKwProg{Fn}{Function}{:}{}
    \Fn{\FMain{$G$}}{
        \ForEach{\text{vertex $e_{i}$ $\in$ $G$}}{
            \textbf{Identify} its \td, \pd\;
        }
    }
    
    \SetKwFunction{FMain}{TXN\_Scheduling}
    \SetKwProg{Fn}{Function}{:}{}
    \Fn{\FMain{$G$}}{
        $M$ $\gets$ Instantiated with $G$;\tcp{A decision model}
        \While{!finish scheduling of $G$
        }{
          \textbf{\circled{2}} $Scheduling Unit$ $\gets$ \textbf{\circled{1}} \emph{Explore}($G$, $M$)\; 
            \textbf{\circled{3}} \emph{Execute with Abort Handling} ($Scheduling Unit$)\; 
        }
    }
  \caption{Execution Outline of \system}
  \label{alg:algo}
\end{algorithm}
\end{comment}

\end{document}