\section{Fourier Growth Reductions For General Gadgets}\label{sec:gadget}

In this section, we show that Fourier growth bounds of communication protocols for general (constant-sized) gadgets can be reduced to the bounds of XOR-fiber, and vice versa.
This implies that in the study of Fourier growth, they are all equivalent.

Let $m_1,m_2$ be two positive integers.
Let $g\colon\binpm^{m_1}\times\binpm^{m_2}\to\binpm$ be a gadget.
Recall that $\unif$ is the uniform distribution over $\binpm^n$.
We now use $\unif_1,\unif_2,\bar\unif_1,\bar\unif_2$ to denote the uniform distributions over $\binpm^{m_1},\binpm^{m_2},(\binpm^{m_1})^n,(\binpm^{m_2})^n$ respectively.
We define the $g$-fiber of communication protocols similar to the XOR-fiber:

\begin{definition}\label{def:g-fiber}
For any randomized two-party protocol $\Ccal\colon(\binpm^{m_1})^n\times(\binpm^{m_2})^n\to[-1,1]$, its $g$-fiber, denoted by $\Ccal_{\downarrow g}\colon\binpm^n\to[-1,1]$, is defined by
$$
\Ccal_{\downarrow g}(z)=\E_{\xbm\sim\bar\unif_1,\ybm\sim\bar\unif_2}\sbra{\Ccal(\xbm,\ybm)\mid g(\xbm_i,\ybm_i)=z_i,~\forall i},
$$
where the expectation is also over the internal randomness of $\Ccal$.
\end{definition}

To compare the Fourier growth bounds between gadgets, we use $L_{1,k}(g,d,m_1,m_2,n)$ to denote the upper bound of the level-$k$ Fourier growth for the $g$-fiber of an arbitrary randomized communication protocol $\Ccal\colon(\binpm^{m_1})^n\times(\binpm^{m_2})^n\to[-1,1]$ with at most $d$ bits of communication, where $g\colon\binpm^{m_1}\times\binpm^{m_2}\to\binpm$ is the gadget.
Since randomized protocols are convex combinations of deterministic protocols of the same cost, using this notation, our main results \Cref{thm:boolean_bound_level_one,thm:boolean_bound_level_two} can be rephrased as
$$
L_{1,1}(\mathrm{XOR},d,1,1,n)\le O\pbra{\sqrt d}
\quad\text{and}\quad
L_{1,2}(\mathrm{XOR},d,1,1,n)\le O\pbra{d^{3/2}\log^3(n)}.
$$

For any set $S\subseteq[m_1]$, define $x_S=\prod_{i\in S}x_i$, and similarly for $y_T$ with $T\subseteq[m_2]$.
Similar to the standard Fourier representation of Boolean functions, the gadget $g$, which is a two-party function, also has Fourier representation:
$$
g(x,y)=\sum_{S\subseteq[m_1],T\subseteq[m_2]}\hat g(S,T)\cdot x_Sy_T,
\quad\text{where}\quad
\hat g(S,T)=\E_{\xbm\sim\unif_1,\ybm\sim\unif_2}\sbra{g(\xbm,\ybm)\cdot\xbm_S\ybm_T}.
$$

For convenience, we will assume $g$ satisfies the following assumption.
It's easy to see that the XOR gadget satisfies it.
\begin{assumption}\label{as:balance}
$\hat g(S,T)=0$ if $S=\emptyset$ or $T=\emptyset$.
\end{assumption}
\begin{remark}
This assumption is equivalent to the fact that, restricted on any input to Alice's side, the remaining function on Bob's side is balanced, and vice versa.

Even if $g$ does not satisfy the assumption, then we can embed it inside a similar gadget $g'\colon\binpm^{m_1+1}\times\binpm^{m_2+1}\to\binpm$, where we XOR the last bit of Alice and the last bit of Bob to the old gadget $g$ applied to Alice's first $m_1$ bits and Bob's first $m_2$ bits, i.e.,
$$
g'(x,y)=x_{m_1+1}y_{m_2+1}\cdot g(x_{\le m_1},y_{\le m_2}).
$$
Then $g'$ satisfies the assumption and inherits most properties of $g$ sufficient for studies in communication complexity tasks.
\end{remark}

Now for a protocol $\Ccal\colon(\binpm^{m_1})^n\times(\binpm^{m_2})^n\to[-1,1]$, it is also a two-party function and thus admitting similar Fourier representation.
We view an input from $(\binpm^{m_1})^n$ as indexed by a tuple in $[n]\times[m_1]$.
Therefore any subset of $(\binpm^{m_1})^n$ is uniquely identified as $\bigcup_{i\in[n]}\cbra{i}\times S_i$, where each $S_i\subseteq[m_1]$.
We use $S^{[n]}$ to denote $(S_i)_{i\in[n]}$.
Thus the Fourier coefficients of $\Ccal$ can be written as
$$
\hat\Ccal(S^{[n]},T^{[n]}):=\hat\Ccal\pbra{\bigcup_{i\in[n]}\cbra{i}\times S_i,\bigcup_{i\in[n]}\cbra{i}\times T_i},
$$
and the Fourier representation of $\Ccal$ is
$$
\Ccal(x,y)=
\sum_{S^{[n]},J^{[n]}}\hat\Ccal(S^{[n]},T^{[n]})\cdot\prod_{i\in[n]}x_{i,S_i}\cdot\prod_{j\in[n]}y_{j,T_j},
$$
where $x_{i,S}=\prod_{j\in S}x_{i,j}$ and similar for $y_{j,T}$.

Under this notation and assuming \Cref{as:balance}, we can effectively compute the Fourier coefficients of any $g$-fiber.
\begin{fact}\label{fct:g-fiber_fourier}
Assume gadget $g\colon\binpm^{m_1}\times\binpm^{m_2}\to\binpm$ satisfies \Cref{as:balance}.
Then we have
$$
\hat{\Ccal_{\downarrow g}}(I)
=\sum_{\substack{S^I,T^I\\S_i\neq\emptyset,T_i\neq\emptyset,\forall i\in I}}\hat\Ccal(S^I,T^I)\cdot\prod_{i\in I}\hat g(S_i,T_i)
\quad
\text{for any $I\subseteq[n]$,}
$$
where we use $S^I$ to denote $S^{[n]}$ with $S_j$ fixed to $\emptyset$ for all $j\notin I$.
\end{fact}
\begin{proof}
Observe that
\begin{align*}
\hat{\Ccal_{\downarrow g}}(I)
&= \E_{\zbm \sim \unif}\sbra{\Ccal_{\downarrow g}(\zbm) \cdot \prod_{i\in I}\zbm_i}\\
&= \E_{\zbm \sim \unif}\sbra{\E_{\xbm\sim\bar\unif_1,\ybm \sim \bar\unif_2}\sbra{\Ccal(\xbm,\ybm) \mid g(\xbm_i, \ybm_i)=\zbm_i,~\forall i} \cdot \prod_{i\in I} \zbm_i}\\
&= \E_{\zbm \sim \unif}\sbra{  \E_{\xbm\sim\bar\unif_1,\ybm \sim \bar\unif_2}\sbra{ \Ccal(\xbm,\ybm) \cdot \prod_{i\in I} g(\xbm_i, \ybm_i) \mid g(\xbm_i, \ybm_i)=\zbm_i,~\forall i } }.
\end{align*}
Since $\hat{g}(\emptyset, \emptyset) = 0$ by \Cref{as:balance}, every pair $(x,y)$ is sampled with the same probability under the conditional distribution. 
Thus we get 
$$
\hat{\Ccal_{\downarrow g}}(I) = \E_{\xbm\sim\bar\unif_1,\ybm\sim \bar\unif_2}\sbra{\Ccal(\xbm,\ybm) \cdot \prod_{i\in I} g(\xbm_i, \ybm_i)}.
$$
Now we expand $\Ccal$ and $g$ in the Fourier basis and obtain
\begin{align*}
\hat{\Ccal_{\downarrow g}}(I) 
&=
\E_{\xbm\sim\bar\unif_1,\ybm\sim\bar\unif_2}\sbra{
\pbra{\sum_{S^{[n]},T^{[n]}}
\hat\Ccal(S^{[n]},T^{[n]})\prod_{i\in[n]}\xbm_{i,S_i}\prod_{j\in[n]}\ybm_{j,T_j}}
\cdot
\prod_{i\in I}\pbra{
\sum_{S_i,T_i}\hat g(S_i,T_i)\xbm_{i,S_i}\ybm_{i,T_i}}}\\
&=
\E_{\xbm\sim\bar\unif_1,\ybm\sim\bar\unif_2}\sbra{
\pbra{\sum_{S^{[n]},T^{[n]}}
\hat\Ccal(S^{[n]},T^{[n]})\prod_{i\in[n]}\xbm_{i,S_i}\prod_{j\in[n]}\ybm_{j,T_j}}
\pbra{
\sum_{S^I,T^I}
\prod_{i\in I}\hat g(S_i,T_i)\xbm_{i,S_i}\ybm_{i,T_i}}}\\
&=
\sum_{S^I,T^I}\hat\Ccal(S^I,T^I)\cdot\prod_{i\in I}\hat g(S_i,T_i)\\
&=\sum_{\substack{S^I,T^I\\S_i\neq\emptyset,T_i\neq\emptyset,\forall i\in I}}\hat\Ccal(S^I,T^I)\cdot\prod_{i\in I}\hat g(S_i,T_i),
\tag{by \Cref{as:balance}}
\end{align*}
as desired.
\end{proof}

Now we present the reduction from XOR-fiber to a general $g$-fiber.
\begin{theorem}\label{thm:xor_to_g}
Assume gadget $g\colon\binpm^{m_1}\times\binpm^{m_2}\to\binpm$ satisfies \Cref{as:balance}. Then
\begin{align*}
L_{1,k}(\mathrm{XOR},d,1,1,n)
&\le\pbra{\max_{S,T}|\hat g(S,T)|}^{-k}\cdot L_{1,k}(g,d,m_1,m_2,n)\\
&\le2^{(m_1+m_2)\cdot k/2}\cdot L_{1,k}(g,d,m_1,m_2,n).
\end{align*}
\end{theorem}
\begin{proof}
Let $\Ccal\colon\binpm^n\times\binpm^n\to[-1,1]$ be an arbitrary protocol of cost at most $d$.
Then for a fixed set $I\subseteq[n]$, by \Cref{fct:g-fiber_fourier} applied to the XOR gadget, we have
\begin{equation}\label{eq:lem:xor_to_g_1}
\hat{\Ccal_{\downarrow\mathrm{XOR}}}(I)=\hat\Ccal(1^I,1^I).
\end{equation}
Let $S\subseteq[m_1]$ and $T\subseteq[m_2]$ maximize $|\hat g(S,T)|$.
Since $g$ satisfies \Cref{as:balance}, we know $S$ and $T$ are not empty sets.

Now define a different protocol $\Ccal'\colon(\binpm^{m_1})^n\times(\binpm^{m_2})^n\to[-1,1]$ as follows:
After receiving input $x$, Alice computes $x'_i=x_{i,S}$ for each block $x_i$; and Bob computes similarly $y'_i=y_{i,T}$ upon receiving input $y$.
Then they execute the protocol $\Ccal$ on $x'$ and $y'$.
That is, $\Ccal'(x,y)=\Ccal(x',y')$.
Therefore, for any $I\subseteq[n]$ and $S^I,T^I$ satisfying $S_i\neq\emptyset,T_i\neq\emptyset$ for $i\in I$, we have
$$
\hat{\Ccal'}(S^I,T^I)=
\begin{cases}
\hat\Ccal(1^I,1^I) & S_i=S,T_i=T,~\forall i\in I,\\
0 & \text{otherwise.}
\end{cases}
$$
Then by \Cref{eq:lem:xor_to_g_1} and \Cref{fct:g-fiber_fourier} applied to $\Ccal'$ with gadget $g$, we have
$$
\hat{\Ccal_{\downarrow g}'}(I)
=\hat\Ccal(1^I,1^I)\cdot\hat g(S,T)^{|I|}
=\hat{\Ccal_{\downarrow\mathrm{XOR}}}(I)\cdot\hat g(S,T)^{|I|}.
$$
Now summing over all $I\subseteq[n]$ of size $k$, we have
\begin{align*}
L_{1,k}(\Ccal_{\downarrow\mathrm{XOR}})
&=\sum_{I\subseteq[n]:|I|=k}\abs{\hat{\Ccal_{\downarrow\mathrm{XOR}}}(I)}
=|\hat g(S,T)|^{-k}\cdot\sum_{I\subseteq[n]:|I|=k}\abs{\hat{\Ccal_{\downarrow g}'}(I)}
=|\hat g(S,T)|^{-k}\cdot L_{1,k}(\Ccal'_{\downarrow g})\\
&\le
|\hat g(S,T)|^{-k}\cdot L_{1,k}(g,d,m_1,m_2,n).
\tag{since $\Ccal'$ has cost at most $d$}
\end{align*}
Since $\Ccal$ is arbitrary, this proves the first half of \Cref{thm:xor_to_g}.
To prove the second half, we use an averaging argument and Parseval's identity on $g$:
\begin{equation*}
|\hat g(S,T)|
\ge\sqrt{2^{-m_1-m_2}\sum_{S',T'}\hat g(S',T')^2}
=\sqrt{2^{-m_1-m_2}}.
\tag*{\qedhere}
\end{equation*}
\end{proof}

Using similar analysis, we also have a reduction from a general $g$-fiber to XOR-fiber.
\begin{theorem}\label{thm:g_to_xor}
Assume gadget $g\colon\binpm^{m_1}\times\binpm^{m_2}\to\binpm$ satisfies \Cref{as:balance}. Then
\begin{align*}
L_{1,k}(g,d,m_1,m_2,n)
&\le\pbra{\sum_{S,T}|\hat g(S,T)|}^k\cdot L_{1,k}(\mathrm{XOR},d,1,1,n)\\
&\le2^{(m_1+m_2)\cdot k/2}\cdot L_{1,k}(\mathrm{XOR},d,1,1,n).
\end{align*}
\end{theorem}
\begin{proof}
Let $\Ccal\colon(\binpm^{m_1})^n\times(\binpm^{m_2})^n\to[-1,1]$ be an arbitrary protocol of cost at most $d$.
Then for a fixed set $I\subseteq[n]$, by \Cref{fct:g-fiber_fourier} applied to gadget $g$ and using \Cref{as:balance}, we have
$$
\hat{\Ccal_{\downarrow g}}(I)
=\sum_{S^I,T^I}\hat\Ccal(S^I,T^I)\cdot\prod_{i\in I}\hat g(S_i,T_i).
$$
Therefore
$$
L_{1,k}(\Ccal_{\downarrow g})
\le\sum_{I\subseteq[n]:|I|=k}\sum_{S^I,T^I}\abs{\hat\Ccal(S^I,T^I)}\cdot\abs{\prod_{i\in I}\hat g(S_i,T_i)}.
$$

Now let $M=\sum_{S,T}|\hat g(S,T)|$.
Let $\rho$ be a distribution over subsets of $[m_1]\times[m_2]$ and its probability density function is defined as:
$$
\rho(S,T)=|\hat g(S,T)|/M.
$$
Then we can rewrite $L_{1,k}(\Ccal_{\downarrow g})$ as
\begin{align}
L_{1,k}(\Ccal_{\downarrow g})
&\le\sum_{I\subseteq[n]:|I|=k}\E_{(\Sbm^I,\Tbm^I)\sim\rho^I}\sbra{\abs{\hat\Ccal(\Sbm^I,\Tbm^I)}\cdot M^k}
\notag\\
&=M^k\cdot\E_{(\Sbm^{[n]},\Tbm^{[n]})\sim\rho^{[n]}}\sbra{\sum_{I\subseteq[n]:|I|=k}\abs{\hat\Ccal(\Sbm^I,\Tbm^I)}}.
\label{eq:lem:g_to_xor_1}
\end{align}

Now we fix an arbitrary $(S^{[n]},T^{[n]})$ sampled from $\rho^{[n]}$.
Note that $S_i$ and $T_i$ are not empty by the definition of $\rho$ and \Cref{as:balance}.
Then define a different protocol $\Ccal'\colon\binpm^n\times\binpm^n\to[-1,1]$ as follows:
After receiving input $x$, Alice samples $x'\in(\binpm^{m_1})^n$ uniformly conditioned on $x'_{i,S_i}=x_i$ for all $i\in[n]$; and Bob samples similarly $y'\in(\binpm^{m_2})^n$ conditioned on $y'_{i,T_i}=y_i$ for all $i\in[n]$.
Then they execute the protocol $\Ccal$ on $x'$ and $y'$.
That is, $\Ccal'(x,y)=\E_{\xbm',\ybm'}[\Ccal(\xbm',\ybm')]$.
Therefore, for any $I\subseteq[n]$, we have
$$
\hat{\Ccal'}(1^I,1^I)=\hat\Ccal(S^I,T^I).
$$
By \Cref{fct:g-fiber_fourier} applied to $\Ccal'$ and the XOR gadget, we have
$$
\hat{\Ccal'_{\downarrow\mathrm{XOR}}}(I)=\hat{\Ccal'}(1^I,1^I)=\hat\Ccal(S^I,T^I).
$$
Since $\Ccal'$ has cost at most $d$, we have
$$
\sum_{I\subseteq[n]:|I|=k}\abs{\hat\Ccal(S^I,T^I)}
=\sum_{I\subseteq[n]:|I|=k}\abs{\hat{\Ccal'_{\downarrow\mathrm{XOR}}}(I)}
=L_{1,k}(\Ccal'_{\downarrow\mathrm{XOR}})
\le L_{1,k}(\mathrm{XOR},d,1,1,n).
$$
Putting back to \Cref{eq:lem:g_to_xor_1}, we have
$$
L_{1,k}(\Ccal_{\downarrow g})
\le M^k\cdot L_{1,k}(\mathrm{XOR},d,1,1,n),
$$
which proves the first half of \Cref{thm:g_to_xor} since $\Ccal$ is arbitrary.
To prove the second half, we use Cauchy-Schwarz inequality and Parseval's identity on $g$:
\begin{equation*}
M=\sum_{S,T}|\hat g(S,T)|\le\sqrt{2^{m_1+m_2}\sum_{S,T}\hat g(S,T)^2}=\sqrt{2^{m_1+m_2}}.
\tag*{\qedhere}
\end{equation*}
\end{proof}

As a corollary, to study the Fourier growth bounds, we can switch between gadgets conveniently, as long as the gadgets have small size.

\begin{corollary}\label{cor:g_to_g'}
Assume gadgets $g\colon\binpm^{m_1}\times\binpm^{m_2}\to\binpm$ and $g'\colon\binpm^{m_1'}\times\binpm^{m_2'}\to\binpm$ satisfy \Cref{as:balance}.
Then
$$
L_{1,k}(g,d,m_1,m_2,n)\le2^{(m_1+m_2+m_1'+m_2')\cdot k/2}\cdot L_{1,k}(g',d,m_1',m_2',n).
$$
\end{corollary}