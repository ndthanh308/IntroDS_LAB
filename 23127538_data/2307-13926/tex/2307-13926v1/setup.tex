\section{Fourier Growth via Martingales in Gaussian Space}\label{sec:fourier_via_martingale}

In this section, we reduce the question of bounding the level-one and level-two Fourier growth to bounding the expected quadratic variation of certain martingales. 
To analyze these martingales and to prove the optimal bound for the level-one setting, it seems to be crucial to work in the Gaussian setting, so first we give a generic transformation from Boolean to Gaussian. We shall also additionally allow protocols that communicate real numbers to make the analysis easier.

\subsection{Communication Protocols in Gaussian Space}
\label{sec:boolean_to_real}

Let $\Ccal: \pmones \times \pmones \to \pmone$ be a communication protocol with total communication $d$ and $h$ be its XOR-fiber defined in \Cref{eqn:fiber}.

We embed the protocol in the Gaussian space by allowing Alice's and Bob's inputs, $x$ and $y$ respectively, to be real vectors in $\R^n$ --- the new protocol $\tilde{\Ccal}$ runs the original protocol $\Ccal$ with Boolean inputs $\sgn(x)$ and $\sgn(y)$ where $\sgn(v) = (\sgn(v_1), \ldots, \sgn(v_n))$ denotes the sign function applied pointwise to each coordinate for a vector $v \in \R^n$. The behavior of the communication protocol $\tilde{\Ccal}$ can be defined arbitrarily if any coordinate of $\sgn(x)$ or $\sgn(y)$ is zero since such points have zero measure under the standard $n$-dimensional Gaussian measure $\gamma_n$. 

This translation from the Boolean hypercube to the Gaussian space preserves the measure of sets: for any subset $S\subseteq \binpm^n$, we have $\Ucal_n(S)=\gamma_n(\cbra{x\in\Rbb^n\mid\sgn(x)\in S})$ where $\unif_n$ is the uniform measure over $\pmones$. 
Moreover, up to some normalizing factor, the Fourier coefficients of $h$ can also be computed by looking at Gaussian inputs. 
In particular, denoting by $x_S = \prod_{i \in S} x_i$ for a subset $S \subseteq [n]$, we have the following fact.
\begin{fact}\label{fct:boolean_to_real}
For all $S\subseteq [n]$, we have $		\E_{\lZ\sim\Ucal_n}\sbra{h(\lZ)\lZ_S}=(\pi/2)^{|S|}\E_{\lX,\lY\sim\gamma_n}\sbra{\tilde {\Ccal}(\lX,\lY) \lX_S \lY_S}$.
\end{fact}
\begin{proof}
Note that for $\lX \sim \gamma_n$, the random variable $\sgn(\lX)$ is distributed as $\unif_n$. Thus, by the definition of the XOR-fiber $h$ and the protocol $\tilde{\Ccal}$, we have
\begin{align*}
\E_{\lZ\sim\Ucal_n}\sbra{h(\lZ)\lZ_S}
&=\E_{\lX,\lY\sim\gamma_n} \left[\Ccal(\sgn(\lX), \sgn(\lY)) \cdot \prod_{i \in S} \sgn(\lX_i) \cdot \sgn(\lY_i)\right]\\
&= (\pi/2)^{|S|}\E_{\lX,\lY\sim\gamma_n} \left[\Ccal(\sgn(\lX), \sgn(\lY)) \cdot \prod_{i \in S} \lX_i \cdot \lY_i\right]\\
&=(\pi/2)^{|S|}\E_{\lX,\lY\sim\gamma_n}\sbra{\tilde {\Ccal}(\lX,\lY) \lX_S \lY_S},
\end{align*}
where the second line follows since the expected value of a standard Gaussian in $\R$ conditioned on its sign being fixed to $\eta$ is $\sqrt{\frac2\pi}\cdot \eta$ by the following calculation:
\begin{equation*}
\E_{\lX_i\sim\gamma}\sbra{\lX_i \mid\sgn(\lX_i)=\eta}= \eta \cdot\int_0^{\infty}\sqrt{\frac2\pi}\cdot r\cdot e^{-r^2/2} \sd r=\sqrt{\frac2\pi}\cdot \eta. \qedhere
\end{equation*}
\end{proof}

\begin{remark}\label{rem:symmetric}
    We remark that instead of the Gaussian distribution above, one can work with any distribution where the coordinates are i.i.d. and symmetric around zero.
    In particular, if $\xi$ is a symmetric probability measure on the real line, and $\lx,\ly$ are independently drawn vectors in $\Rbb^n$ where each coordinate is i.i.d. sampled from $\xi$, then 
    $\E_{\lZ\sim\Ucal_n}\sbra{h(\lZ)\lZ_S}=c_{\xi}^{|S|}\E_{\lX,\lY\sim\xi^{\otimes n}}\sbra{\tilde {\Ccal}(\lX,\lY) \lX_S \lY_S}$ where $c_{\xi} = (\BE_{\lx_i \sim \xi}[|\lx_i|])^{-2}$. In the case of level-two we will need to work with the truncated Gaussian distribution where each coordinate is sampled independently from the one dimensional standard Gaussian conditioned on being in some interval $[-T,T]$ for $T = \Omega(1)$ in which case $c_{\xi}$ is upper bounded by a universal constant.
\end{remark}

\subsection{Generalized Communication Protocols}\label{sec:gen-protocol}

In the protocol $\tilde{\Ccal}$ defined above, Alice and Bob's inputs $x$ and $y$ are real vectors in $\R^n$, but in each round they still exchange a single bit based on $\sgn(x)$ and $\sgn(y)$. In order to bound the Fourier growth, it will be more convenient for us to define a notion of generalized communication protocols where parties are also allowed to send real numbers with arbitrary precision in each round. To define this formally, we place certain restrictions on the real communication in the protocol. More formally, in a generalized communication protocol, in each round a player with input $z \in \R^n$ can either send:
\begin{enumerate}[label={(\roman*)}]
	\item a bit in $\bin$ which is purely a function of the Boolean input $\sgn(z)$ and the previous \emph{Boolean} messages, or
	\item a real number that is a measurable function of $z$ and the previous (real or Boolean) messages.
\end{enumerate} 

The \emph{depth} of a generalized communication protocol is defined to be the maximum number of rounds of communication. 

Note that a generalized protocol also generates a ``protocol tree'' where if in a round a real number is sent, the ``children'' of that particular ``node'' are indexed by all possible values in $\R$. A ``transcript'' of the protocol can be defined in an analogous way. The set of inputs that reach a particular node of this generalized protocol tree still form a rectangle $X \times Y$ where $X, Y \subseteq \R^n$. We say that a generalized protocol $\bar{\Ccal}$ is equivalent to the protocol $\tilde{\Ccal}$ if ${\bar{\Ccal}}(x,y) = \tilde{\Ccal}(x,y)$ for every $x, y \in \R^n$ except on a measure zero set. 

We will be interested in random walks on such generalized protocol trees when the inputs $\lx$ and $\ly$ are sampled from a product measure $\xi_x \times \xi_y$ on $\Rbb^n \times \Rbb^n$ and the parties send messages according to the protocol to reach a ``leaf''. The random variables corresponding to the messages until any time $t$ generate a filtration $(\supF{t})_t$ ---  this filtration can be thought of as specifying a particular node of the generalized protocol at depth $t$ (equivalently, a partial transcript of the protocol till time $t$) that was sampled by the process. In this case, conditioned on any event in $\supF{t}$, (e.g., any realization of the transcript till time $t$), almost surely the conditional probability measure on the inputs $\lx, \ly$ is some product measure on $\xi_x^{(t)} \times \xi_y^{(t)}$ supported on a rectangle $\supX{t} \times \supY{t}$ where $\supX{t}, \supY{t} \subseteq \Rbb^n$. We shall refer to the random variable $\supX{t} \times \supY{t}$ as the current rectangle determined by $\supF{t}$. Since we will be working with product measures on inputs $\lx, \ly$, the reader can think of conditioning on the filtration $\supF{t}$ as essentially conditioning on the inputs being in the rectangle $\supX{t} \times \supY{t}$ or equivalently a partial transcript till time $t$.

\subsection{Fourier Growth via Martingales}\label{sec:fourier_weights_via_martingales}

We will now relate Fourier growth to the quadratic variation of a martingale. Towards this end, we first note that in light of \Cref{fct:boolean_to_real}, the level-$k$ Fourier growth of the XOR-fiber $h$ of the original communication protocol is given by 
\begin{align}\label{eqn:boolean-to-real}
L_{1,k}(h) = \sum_{\substack{S \subseteq[n]\\|S|=k}} \abs{\BE_{\lZ \sim \unif_n}[h(\lZ) \lZ_S]} 
&= (\pi/2)^k \sum_{\substack{S \subseteq[n]\\|S|=k}} \abs{\BE_{\lX,\lY \sim \gamma_n}[\bar{\Ccal}(\lX,\lY)\lX_S\lY_S]} \notag \\
& = (\pi/2)^k \max_{(\eta_S)_{|S|=k}} \sum_{\substack{S \subseteq[n]\\|S|=k}} \eta_S {\BE_{\lX,\lY \sim \gamma_n}\sbra{\bar{\Ccal}(\lX,\lY)\lX_S\lY_S}},
\end{align} 
where $\bar{\Ccal}$ is any generalized protocol that is equivalent to $\tilde{\Ccal}$ and $\eta_S \in \pmone$.

We now express the right hand side above as an inner product. Let $\bell$ be a random leaf of the generalized protocol tree $\bar \Ccal$ induced by taking $\lX, \lY \sim \gamma_n$ and let $\X_\ell \times \Y_\ell$ be the corresponding rectangle in the generalized protocol tree. Then, 
\begin{align}\label{eqn:ip}
\sum_{\substack{S \subseteq[n],|S|=k}} \eta_S {\BE_{\lX,\lY \sim \gamma_n}\sbra{\bar{\Ccal}(\lX,\lY)\lX_S\lY_S}} 
&=\E_{\bell}\sbra{\E_{\lX,\lY\sim\gamma}\sbra{\sum_{\substack{S \subseteq[n],|S|=k}} \eta_S \cdot\bar\Ccal(\lX,\lY)\lX_S \lY_S\mid (\lX,\lY)\in \X_{\bell} \times \Y_{\bell}}} \notag\\
&=\E_{\bell}\sbra{\bar\Ccal(\bell)\E_{\lX,\lY\sim\gamma}\sbra{\sum_{\substack{S \subseteq[n],|S|=k}} \eta_S \cdot\lX_S \lY_S\mid (\lX,\lY)\in \X_{\bell} \times \Y_{\bell}}} \notag \\
&\le \E_{\bell}\sbra{~\left|\sum_{\substack{S \subseteq[n],|S|=k}} \eta_S \E\sbra{\lX_S \mid \lX \in \X_{\bell}} \cdot \E\sbra{\lY_S \mid \lY \in \Y_{\bell}}\right|~},
\end{align}
where the second line follows since $\bell$ is a leaf and determines the answer and the third line follows since $\lX$ and $\lY$ are independent conditioned on being in the rectangle $\X_{\bell} \times \Y_{\bell}$.

Thus, specializing \Cref{eqn:ip} to the level-one ($k=1$) and level-two cases ($k=2$), from \Cref{eqn:boolean-to-real} we get that 
\begin{align*}
L_{1,1}(h) &\le \frac{\pi}{2} \cdot \max_{\eta}~ \E_{\bell}\sbra{~\left|\sum_{i=1}^n \eta_i \cdot \E\sbra{\lX_i \mid \lX \in \X_{\bell}} \cdot \E\sbra{\lY_i \mid \lY \in \Y_{\bell}}\right|~},\\
L_{1,2}(h) &\le \frac{\pi^2}{4} \cdot \max_\eta~\E_{\bell}\sbra{~\left|\sum_{i,j=1}^n \eta_{ij} \cdot ~\E\sbra{\lX_{ij} \mid \lX \in \X_{\bell}} \cdot \E\sbra{\lY_{ij} \mid \lY \in \Y_{\bell}}\right|~},
\end{align*}
where for $L_{1,1}$ we optimize over $\eta \in \pmones$ and for $L_{1,2}$ we optimize over $\eta$ being an $n \times n$ symmetric matrix with zeros on the diagonals and $\pm 1$ entries otherwise.

To make the above more compact, we respectively define $\com(X) \in \R^n$ and $\comtwo(X) \in \R^{n\times n}$ to be the level-one and level-two centers of mass of a set $X \subseteq \R^n$:
\begin{equation}\label{def:com}
    \com(X) = \BE_{\lX \sim \gamma_n}\sbra{\lX \mid \lX \in X} 
    \quad\text{and}\quad
    \comtwo(X) = \BE_{\lX \sim \gamma_n}\sbra{\lX \tensor \lX \mid \lX \in X}.
\end{equation}
Then, upper bounding the constants in the above inequality ($\pi/2$ and $\pi^2/4$) by $4$, we get
\begin{equation}\label{eqn:fwt-to-ip}
\begin{split}
    L_{1,1}(h) &\le 4 \cdot \max_{\eta}~ \E_{\bell}\sbra{\left| \ip{\com(\X_{\bell})}{ \eta \odot \com(\Y_{\bell})} \right|},\\
    L_{1,2}(h) &\le 4 \cdot \max_\eta~\E_{\bell}\sbra{\left| \ip{\comtwo(\X_{\bell})}{ \eta \odot \comtwo(\Y_{\bell})}  \right|},
\end{split}
\end{equation}
where $\eta$ is understood to be the same as before. 

Moving forward, we fix an arbitrary $\eta$ for both cases $k \in \{1,2\}$ and define a martingale process $\left(\supZ{t}_k\right)_t$ that captures the right hand side above. For this we note that a generalized communication protocol, where Alice's and Bob's inputs are sampled from the Gaussian distribution, naturally induces a discrete-time random walk on the corresponding (generalized) protocol tree where at time $t$ we are at a node at depth $t$ with the corresponding rectangle $\supX{t}\times \supY{t}$. Then, we have the following proposition.

\begin{proposition}\label{prop:vec-martingale}
    $\com(\supX{t})$ and $\com(\supY{t})$ are vector-valued martingales taking values in $\R^n$ and $\comtwo(\supX{t})$ and  $\comtwo(\supY{t})$ are matrix-valued martingales taking values in $\R^{n \times n}$.
\end{proposition}

Note that if in the $t^{\text{th}}$ round Alice speaks, then $\com(\supY{t})$ and $\comtwo(\supY{t})$ do not change and similarly if Bob speaks, then $\com(\supX{t})$ and $\comtwo(\supX{t})$ do not change.
The above proposition implies that the real-valued processes 
\begin{equation}
    \label{eqn:def-martingale}
 \supZ{t}_1 = \ip{\com(\supX{t})}{\eta \odot \com(\supY{t})} \text{ and } \supZ{t}_2 = \ip{\comtwo(\supX{t})}{\eta \odot \comtwo(\supY{t})},
\end{equation}
each form a Doob martingale with respect to the natural filtration induced by the random walk on the protocol tree.
Note that taking a random walk on the tree until we hit a leaf generates the marginal distribution on $\bell$ given in \Cref{eqn:fwt-to-ip}. Let $\D$ be the stopping time when this martingale hits a leaf and stops (i.e., the depth of the random leaf). Thus, by the orthogonality of martingale differences $\Delta\supZ{t}_k = \supZ{t}_k - \supZ{t-1}_k$ from \Cref{eqn:martingale-orthogonality},  we get that for $k \in \{1,2\}$, one can upper bound the Fourier growth in terms of expected quadratic variation of the above martingales:
\begin{proposition}\label{prop:fwt-to-qv}
For $k\in\{1,2\}$, $\frac14\cdot  L_{1,k}(h) \le \max_{\eta}\sqrt{\BE\left[\left(\supZ{\D}_k\right)^2\right]} = \max_{\eta}\sqrt{\BE\left[\sum_{t=1}^{\D} \left(\Delta \supZ{t}_k\right)^2\right]}$.
\end{proposition}
The martingale implicitly depends on $\eta$ as used in \Cref{eqn:fwt-to-ip} and hence the maximum. Moreover, the martingale also depends on the underlying generalized communication protocol $\bar \Ccal$. In the next two sections, we will show that after transforming the original communication protocol into ``clean'' protocols, the expected quadratic variations of $(\supZ{t}_1)_t$ and $(\supZ{t}_2)_t$ are $O(d)$ and $O(d^3) \cdot \polylog(n)$ respectively. This will then imply our main theorems.

\begin{remark}\label{rem:martingale}
Note that \Cref{prop:vec-martingale} still holds even if the input distribution is not the Gaussian distribution, but some other product probability measure on the inputs $\lx, \ly$. This also implies that $\supZ{t}_k$ for $k \in \{1,2\}$ is a martingale. In particular, for the level-two case, we will need to use a truncated Gaussian distribution. In light of \Cref{rem:symmetric}, \Cref{prop:fwt-to-qv} still suffices for us with a different constant instead of $1/4$. We also remark that we shall also need to truncate the real messages being used in the protocol for the level-two case to a finite precision, so the generalized protocols for the level-two case only have Boolean communication. However, to obtain the optimal level-one bound allowing generalized protocols that communicate real values seems to be crucial.
\end{remark}