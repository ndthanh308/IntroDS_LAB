\section{Introduction}\label{sec:intro}

The Fourier spectrum of Boolean functions and their various properties have played an important role in many areas of mathematics and theoretical computer science. In this work, we study a notion called $\ell_1$-Fourier growth, which captures the scaling of the sum of absolute values of the level-$k$ Fourier coefficients of a function. In a nutshell, functions with small Fourier growth cannot aggregate many weak signals in the input to obtain a considerable effect on the output. In contrast, the Majority function, which can amplify weak biases, is an example of a Boolean function with extremely {\em high} Fourier growth.

To formally define Fourier growth, we recall that every Boolean function $f: \binpm^n \to [-1,1]$ can be uniquely represented as a multilinear polynomial 
$$
f(x) = \sum_{S \subseteq [n]} \hat{f}(S) \cdot \prod_{i\in S} x_i
$$ 
where the coefficients of the polynomial $\hat{f}(S)\in \Rbb$ are called the Fourier coefficients of $f$, and they satisfy $\hat{f}(S) = \E[f(\bm{x}) \cdot \prod_{i\in S} \bm{x}_i]$ for a uniformly random $\bm{x} \in \pmone^n$.
The level-$k$ $\ell_1$-Fourier growth of $f$ is the sum of the {\em absolute values} of its level-$k$ Fourier coefficients, 
$$
L_{1,k}(f) := \sum_{S\subseteq[n]:|S|=k}\abs{\hat{f}(S)}.
$$

The study of Fourier growth dates back to the work of Mansour \cite{Mansour95} who used it in the context of learning algorithms.
Since then, several works have shown that upper bounds on the Fourier growth, even for the first few Fourier levels, have applications to pseudorandomness, circuit complexity, and quantum-classical separations.
For example:
\begin{itemize}
\item A bound on the level-one Fourier growth is sufficient to control the advantage of distinguishing biased coins from unbiased ones \cite{agarwal20}.
\item A bound on the level-two Fourier growth already gives  pseudorandom generators \cite{CHLT19}, oracle separations between BQP and PH \cite{RT19,Wu22}, and separations between efficient quantum communication and randomized classical communication \cite{GRT21}. 
\end{itemize}
Meanwhile, Fourier growth bounds have been extensively studied and established for various computational models, including small-width DNFs/CNFs \cite{Mansour95}, $\mathsf{AC}^0$ circuits \cite{Tal17}, low-sensitivity Boolean functions \cite{GSTW16}, small-width branching programs \cite{RSV13,SteinkeVW17,CHRT18,LPV22}, small-depth decision trees \cite{OS07,Tal20,SSW21}, functions related to small-cost communication protocols \cite{GRZ21,GRT21}, low-degree $\mathsf{GF}(2)$ polynomials \cite{CHHL19,CHLT19,blasiok2021fourier}, product tests \cite{Lee19}, small-depth parity decision trees \cite{DBLP:journals/corr/BlaisTW15,GTW21}, low-degree bounded functions \cite{iyer2021tight}, and more.

For any Boolean function $f$ with outputs in $[-1,1]$, the level-$k$ Fourier growth $L_{1,k}(f)$ is at most $\sqrt{\binom nk}$. However, for many natural classes of Boolean functions, this bound is far from tight and not good enough for applications. Establishing better bounds require exploring structural properties of the specific class of functions in question. Even for low Fourier levels, this can be highly non-trivial and tight bounds remain elusive in many cases. For example, for degree-$d$ $\mathsf{GF}(2)$ polynomials
(which well-approximate $\mathsf{AC}^0[\oplus]$ when we set $d=\polylog(n)$ \cite{razborov1987lower,DBLP:conf/stoc/Smolensky87}),
while we know a level-one bound of $L_{1,1}(f)\le O(d)$ due to~\cite{CHLT19}, the current best bound for levels $k\ge2$ is roughly $2^{O(dk)}$ \cite{CHHL19}, whereas the conjectured bound is $d^{O(k)}$. Validating such a bound, even for the second level $k=2$, will imply unconditional pseudorandom generators of polylogarithmic seed length for $\mathsf{AC}^0[\oplus]$ \cite{CHLT19}, a longstanding open problem in circuit complexity and pseudorandomness.

\paragraph*{XOR Functions.}  
In this work, we study the Fourier growth of certain functions that naturally arise from communication protocols for XOR-lifted functions, also referred to as XOR functions. XOR functions are an important and well-studied class of functions in communication complexity with connections to the log-rank conjecture and quantum versus classical separations~\cite{MO09,HHL18,TWXZ13,SZ08,Zha13}. 

In this setting, Alice gets an input $x\in \binpm^n$ and Bob gets an input $y\in \binpm^n$ and they wish to compute $f(x\odot y)$ where $f$ is some partial Boolean function and $x\odot y$ is in the domain of $f$. Here, $x\odot y$ denotes the pointwise product of $x$ and $y$. Given any communication protocol $\Ccal$ that computes an XOR function exactly, the output $\Ccal(x,y)$ of the protocol depends only on the parity $x \odot y$, whenever $f$ is defined on $x \odot y$. This gives a natural motivation to analyze the XOR-fiber of a communication protocol defined below. We note that a similar notion first appeared in an earlier work of Raz \cite{DBLP:journals/cc/Raz95}.

\begin{definition}\label{eqn:fiber}
Let $\Ccal: \pmones \times \pmones \to \pmone$ be any deterministic communication protocol. The XOR-fiber of the communication protocol $\Ccal$ is the function $h\colon\binpm^n\to[-1,1]$ defined at $z\in\binpm^n$ as 
\[    h(z) = \BE_{\lx,\ly \sim \unif}[\Ccal(\lx,\ly) ~|~ \lx\odot \ly = z],\]
where $\odot$ is the entrywise product and $\unif$ is the uniform distribution over $\binpm^n$. 
\end{definition}

We remark that XOR-fiber is the ``inverse'' of XOR-lift of a function: If $\Ccal$ computes the XOR function of $f$, then the XOR-fiber $h$ of $\Ccal$ is equal to $f$ on the domain of $f$.

\vspace{10pt}

In this work, we investigate the Fourier growth of XOR-fibers of small-cost communication protocols and apply these bounds in several contexts. Before stating our results, we first discuss several related works.  

\paragraph*{Related Works.}
Showing optimal Fourier growth bounds for XOR-fibers is a complex undertaking in general and a first step towards this end is to obtain optimal Fourier growth bounds for parity decision trees. This is because a parity decision tree for a Boolean function $f$ naturally gives rise to a structured communication protocol for the XOR-function corresponding to $f$. This protocol perfectly simulates the parity decision tree by having Alice and Bob exchange one bit each to simulate a parity query. Moreover, the XOR-fiber of this protocol exactly computes the parity decision tree. As such, parity decision trees can be seen as a special case of communication protocols, and Fourier growth bounds on XOR-fibers of communication protocols immediately imply Fourier growth bounds on parity decision trees.

Fourier growth bounds for decision trees and parity decision trees are well-studied. It is not too difficult to obtain a level-$k$ bound of $O(d)^k$ for parity decision trees of depth $d$, however, obtaining improved bounds is significantly more challenging. For decision trees of depth $d$ (which form a subclass of parity decision trees of depth $d$), O'Donnell and Servedio~\cite{OS07} proved a tight bound of $O(\sqrt{d})$ on the level-one Fourier growth. 
By inductive tree decompositions, Tal~\cite{Tal20} obtained bounds for the higher levels of the form $L_{1,k}(f)\le\sqrt{d^k\cdot O(\log(n))^{k-1}}$.
This was later sharpened by Sherstov, Storozhenko, and Wu~\cite{SSW21} to the asymptotically tight bound of $L_{1,k}(f)\le\sqrt{\binom dk\cdot O(\log(n))^{k-1}}$ using a more sophisticated layered partitioning strategy on the tree.

When it comes to parity decision trees, despite all the similarities, the structural decomposition approach does not seem to carry over due to the correlations between the parity queries. For parity decision trees of depth $d$, Blais, Tan, and Wan~\cite{DBLP:journals/corr/BlaisTW15} proved a tight level-one bound of $O(\sqrt{d})$. For higher levels, Girish, Tal, and Wu~\cite{GTW21} showed that $L_{1,k}(f)\le\sqrt{d^k\cdot O(k\log(n))^{2k}}$.
%To circumvent this issue, \cite{GTW21} developed a different framework, where the Fourier growth is related to the martingale process by a random walk on the parity decision tree, and obtained $L_{1,k}(f)\le\sqrt{d^k\cdot O(k\log(n))^{2k}}$. There, to bound the martingale steps and exploit cancellations, they introduced additional queries to ensure the hypercontractive property of the spaces encountered during the random walk.
These works imply almost tight Fourier growth bounds on the XOR-fibers of structured protocols that arise from simulating decision trees or parity decision trees. 

For the case of XOR-fibers of arbitrary deterministic/randomized communication protocols (which do not necessarily simulate parity decision trees or decision trees), Girish, Raz, and Tal \cite{GRT21} showed an ${O}(d^k)$ Fourier growth\footnote{Technically, \cite{GRT21} only proved a level-two bound (as it suffices for their analysis), but a level-$k$ bound follows easily from their proof approach, as noted by~\cite{GRZ21}} for level-$k$. For level-one and level-two, these bounds are  $O(d)$ and $O(d^2)$ respectively and are sub-optimal --- as mentioned previously, such weaker bounds for parity decision trees are easy to obtain, while obtaining optimal bounds (for parity decision trees) of $O(\sqrt{d})$ for level one and $d \cdot \polylog(n)$ for level two already requires sophisticated ideas.

The bounds in~\cite{GRT21} follow by analyzing the Fourier growth of XOR-fibers of communication rectangles of measure $\approx 2^{-d}$ and then adding up the contributions from all the leaf rectangles induced by the protocol. Such a per-rectangle-based approach cannot give better bounds than the ones in~\cite{GRT21}, while  they also conjectured that the optimal Fourier growth of XOR-fibers of arbitrary protocols should match the growth for parity decision trees. 

Showing the above is a challenging task even for the first two Fourier levels. The difficulty arises primarily since in the absence of a per-rectangle-based argument, one has to crucially leverage cancellations between different  rectangles induced by the communication protocol. In the simpler case of parity decision trees (or protocols that exchange parities), such cancellations are leveraged in \cite{GTW21} by ensuring $k$-wise independence at each node of the tree --- this can be achieved by adding extra parity queries. In a general protocol, the parties can send arbitrary partial information about their inputs and correlate the coordinates in complicated ways that such methods break down.
This is one of the key difficulties we face in this paper. 

\subsection{Main Results}

We prove new and improved bounds on the Fourier growth of the XOR-fibers associated with small-cost protocols for levels $k=1$ and $k=2$.

\begin{theorem}\label{thm:boolean_bound_level_one} 
Let $\Ccal:\binpm^n\times \binpm^n\to \binpm$ be a deterministic communication protocol with at most $d$ bits of communication. Let $h$ be its XOR-fiber as in \Cref{eqn:fiber}. Then,
$L_{1,1}(h) = O\pbra{\sqrt d}$. 
\end{theorem}

\begin{theorem}\label{thm:boolean_bound_level_two}
Let $\Ccal:\binpm^n\times \binpm^n\to \binpm$ be a deterministic protocol communicating at most $d$ bits. Let $h$ be its XOR-fiber as in \Cref{eqn:fiber}. Then, $L_{1,2}(h) = O\pbra{d^{3/2} \log^3(n)}$.
\end{theorem}

Our bounds in \Cref{thm:boolean_bound_level_one,thm:boolean_bound_level_two} extend directly to randomized communication protocols. This is because $L_{1,k}$ is convex and any randomized protocol is a convex combination of deterministic protocols with the same cost. Moreover, we can use Fourier growth reductions, as described in \Cref{sec:applications_gadgets}, to demonstrate that these bounds apply to general constant-sized gadgets $g$ and the corresponding $g$-fiber.

%We remark that the bounds in \Cref{thm:boolean_bound_level_one,thm:boolean_bound_level_two} immediately extend to {\em randomized} communication protocols, since $L_{1,k}$ is convex and any randomized protocol is a convex combination of deterministic protocols of the same cost. Via Fourier growth reductions (see \Cref{sec:applications_gadgets}), these bounds also hold for general constant-sized gadget $g$ and the corresponding $g$-fiber.

Our level-one and level-two bounds improve previous bounds in \cite{GRT21} by polynomial factors. Additionally, our level-one bound is tight since a deterministic protocol with $d+1$ bits of communication can compute the majority vote of $x_1 \cdot y_1, \ldots, x_d \cdot y_d$, which corresponds to $h(z) = \mathrm{MAJ}(z_1, \ldots, z_{d})$ with $L_{1,1}(h) = \Theta(\sqrt{d})$. Furthermore, as we discuss later in \Cref{sec:apps}, level-one and level-two bounds are already sufficient for many interesting applications.

In terms of techniques, our analysis presents a key new idea that enables us to exploit cancellations between different rectangles induced by the protocol. This idea involves using a novel process to adaptively partition a relatively large set in Gaussian space, which enables us to control its $k$-wise moments in all directions --- this can be thought of as a spectral notion of almost $k$-wise independence. We achieve this by utilizing martingale arguments and allowing protocols to transmit \emph{real values} rather than just discrete bits. This notion and procedure may be of independent interest. See \Cref{sec:overview} for a detailed discussion. 

%One of the key new ideas in our  analysis --- and this is what allows to leverage cancellations between different rectangles induced by the protocol --- is a new process to adaptively partition a relatively large set in Gaussian space in order to control its moments in all directions. We achieve this via martingale arguments and allowing protocols to communicate \emph{real values} in contrast to the discrete bits. 

%\mnote{Add more details and rephrase since currently it is just copied from the abstract.}

%Our proof relies on a martingale-based approach similar to \cite{GTW21} and bounds the Fourier weights in terms of the quadratic variation of a martingale, but because communication protocols can communicate partial information, to analyze the martingale for XOR-fibers of protocols, we introduce several new ingredients, including extra cleanup steps by sending linear or quadratic forms (over the reals) in the inputs, and a spectral notion of almost $k$-wise independence, which allows for more control over the step sizes of the martingale. See \Cref{sec:overview} for a detailed discussion.

% Additionally, as mentioned before,  \cite{DBLP:journals/corr/BlaisTW15,GTW21} proved a tight bound of $O(\sqrt{d})$ on the level-one Fourier growth of depth-$d$ parity decision trees and \Cref{thm:boolean_bound_level_one} gives a new proof of this fact by applying it to structured protocols obtained from simulating parity decision. For level-two, \Cref{thm:boolean_bound_level_two} implies a bound of $d^{3/2}\cdot \polylog (n)$, compared to the optimal bound of $d\cdot \polylog (n)$ obtained in~\cite{GTW21}.

%Our proof relies on a martingale-based approach similar to \cite{GTW21} and bounds the Fourier weights in terms of the quadratic variation of a martingale, but because communication protocols can communicate partial information, to analyze the martingale for XOR-fibers of protocols, we introduce several new ingredients, including extra cleanup steps by sending linear or quadratic forms (over the reals) in the inputs, and a spectral notion of almost $k$-wise independence, which allows for more control over the step sizes of the martingale. See \Cref{sec:overview} for a detailed discussion.


\subsection{Applications and Connections}
\label{sec:apps}

Our main theorem has applications to XOR functions, and in more generality to functions lifted with constant-sized gadgets. In this setting, there is a simple gadget $g:\Sigma \times \Sigma \to \pmone$ and a Boolean function $f$ defined on inputs $z\in \pmone^n$. 
The lifted function $\flift$ is defined on $n$ pairs of symbols $(x_1, y_1), \ldots, (x_n, y_n) \in \Sigma \times \Sigma$ such that $(\flift)(x, y) = f(g(x_1, y_1), \ldots, g(x_n, y_n))$.
The function $\flift$ naturally defines a communication problem where Alice is given $x = (x_1, \ldots, x_n)$, Bob is given $y = (y_1, \ldots, y_n)$, and they are asked to compute $(\flift)(x, y)$. 

Since XOR functions are functions lifted with the XOR gadget, our main theorem implies lower bounds on the communication complexity of specific XOR functions. Additionally, we also show connections between XOR-lifting and lifting with any constant-sized gadget. Next, we describe these lower bounds and connections, with further context.

%We can view XOR functions as functions lifted with the XOR gadget and our main theorem implies lower bounds on the communication complexity of certain XOR functions. These lower bounds, which we provide below with further context, can be inferred from our theorem. 

\subsubsection{The Coin Problem and the Gap-Hamming Problem}\label{sec:applications_level_one}

The coin problem studies the advantage that a class of Boolean functions has in distinguishing biased coins from unbiased ones. 
More formally, let $\Fcal$ be a class of $n$-variate Boolean functions.
Let $\rho \in[-1,1]$ and $\biased{\rho}$ denote the product distribution over $\binpm^n$ where each coordinate has expectation $\rho$. 
The Coin Problem asks what is the maximum advantage that functions in $\Fcal$ have in distinguishing $\biased{\rho}$ from the uniform distribution $\biased{0}$. 

This quantity essentially captures how well $\Fcal$ can approximate threshold functions, and in particular, the majority function. The coin problem has been studied for various models of computation including branching programs~\cite{BV10}, $\mathsf{AC}^0$ and $\mathsf{AC}^0[\oplus]$ circuits~\cite{CGR14,LSSTV19}, product tests~\cite{Lee18}, and more. 
Recently, Agrawal \cite{agarwal20} showed that the coin problem is closely related to the level-one Fourier growth of functions in $\Fcal$.

\begin{lemma}[{\cite[Lemma 3.2]{agarwal20}}] \label{lem:coin_problem}
Assume that $\Fcal$ is closed under restrictions and satisfies $L_{1,1}(f) \le t$ for all $f\in \Fcal$. Then, for all $\rho\in(-1,1)$ and $f\in \Fcal$, 
\[ 
\abs{\E_{z\sim \biased{\rho}}[f(z)]-\E_{z\sim \biased{0}}[f(z)]}\le \ln\pbra{\tfrac{1}{1-|\rho|}}\cdot t.
\]
\end{lemma}

Note that communication protocols of small cost are closed under restrictions, so are their XOR-fibers (see \cite[Lemma 5.5]{GRT21}).
By noting that  $\ln\pbra{\frac1{1-|\rho|}} \approx |\rho|$ for small values of $\rho$, we obtain the following corollary.\footnote{Here we also use the fact that the upper bound $O(|\rho|\cdot \sqrt{d})$ is vacuous for large enough $\rho$ as it is larger than $1$.}
We also remark that, using the Fourier growth reductions (see \Cref{sec:applications_gadgets}), \Cref{thm:coin_problem} can be established for general gadgets of small size.

\begin{theorem}\label{thm:coin_problem} 
Let $h$ be the XOR-fiber of a protocol with total communication $d$. Then for all $\rho$,
\[ \abs{\E_{z\sim \biased{\rho}}[h(z)]-\E_{z\sim \biased{0}}[h(z)]} \le O\!\pbra{ |\rho|\cdot \sqrt{d}} .\]
\end{theorem}

In particular, consider the following distinguishing task: 
Alice and Bob either receive two uniformly random strings in $\binpm^n$ or they receive two uniformly random strings in $\binpm^n$ conditioned on their XOR distributed according to $\biased{\rho}$ for $\rho = 1/\sqrt{n}$ (the latter is often referred to as \emph{$\rho$-correlated strings}).
\Cref{thm:coin_problem} implies that any protocol communicating $o(n)$ bits cannot distinguish these two distributions with constant advantage. This is essentially a communication lower bound for the well-known Gap-Hamming Problem.

\paragraph*{The Gap-Hamming Problem.}
In the Gap-Hamming Problem, Alice and Bob receive strings $x,y\in\binpm^n$ respectively and they want to distinguish if $\ip{x}{y} \le -\sqrt{n}$ or $\ip{x}{y} \ge \sqrt{n}$.

This is essentially the XOR-lift of the Coin Problem with $\rho=\pm 1/\sqrt{n}$ because the distribution of $(x,y)$ conditioned on $x \odot y\sim\biased{\rho}$ with $\rho=-1/\sqrt{n}$ and $\rho=1/\sqrt{n}$ is mostly supported on the \textsc{Yes} and \textsc{No} instances of Gap-Hamming respectively.
Thus immediately from \Cref{thm:coin_problem}, we derive a new proof for the $\Omega(n)$ lower bound on the communication complexity of the Gap-Hamming Problem.
The proof is deferred to \Cref{app:thm:gap_hamming}.

\begin{theorem}\label{thm:gap_hamming}
The randomized communication complexity of the Gap-Hamming Problem is $\Omega(n)$.	
\end{theorem}

We note that there are various different proofs~\cite{DBLP:journals/siamcomp/ChakrabartiR12,DBLP:journals/toc/Sherstov12,DBLP:journals/cjtcs/Vidick12,RY22} that obtain the above lower bound but the perspective taken here is perhaps conceptually simpler: (1) Gap-Hamming is essentially the XOR-lift of the Gap-Majority function, and (2) any function that approximates the Gap-Majority function must have large level-one Fourier growth, whereas XOR-fibers of small-cost protocols have small Fourier growth. 

\subsubsection{Quantum versus Classical Communication Separation via Lifting}
\label{sec:applications_level_two}

One natural approach to proving quantum versus classical separations in communication complexity is via lifting: 
Consider a function $f$ separating quantum and classical query complexity and lift it using a gadget $g$. Naturally, an algorithm computing $f$ with few queries to $z$ can be translated into a communication protocol computing $\flift$ where we replace each query to a bit $z_i$ with a short conversation that allows the calculation of $z_i=g(x_i, y_i)$. \Goos, Pitassi, and Watson~\cite{GPW20} showed that for randomized query/communication complexity and for various gadgets, this is essentially the best possible. Such results are referred to as {\em lifting theorems}. 

Lifting theorems apply to different models of computation, such as deterministic decision trees~\cite{RM99,GPW15}, randomized decision trees~\cite{GPW20, CFKMP19}, and more. A beautiful line of work shows how to ``lift''  many lower bounds in the query model to the communication model
\cite{RM99,GPW15,GLMWZ15,Goos15,RezendeNV16,HHL18,WYY17,CKLM19,KMR17,SZ09,Sher11,RS10,RPRC16,GKPW19,LRS15}. 
For quantum query complexity, only one direction (considered the ``easier'' direction) is known:
Any quantum query algorithm for $f$ can be translated to a communication protocol for $\flift$ with a small logarithmic overhead \cite{BCW}. 
It remains widely open whether the other direction holds as well.
However, this query-to-communication direction for quantum, combined with the communication-to-query direction for classical, is already sufficient for lifting quantum versus classical separations from the query model to the communication model.
%\mnote{add references}\kewen{I think it's forklore.}

One drawback of this approach to proving communication complexity separations is that the state-of-the-art lifting results~\cite{CFKMP19,lovett2022lifting} work for gadgets with alphabet size at least $n$ (recall that $n$ denotes $f$'s input length) and it is a significant challenge to reduce the alphabet size to $O(1)$ or even $\polylog(n)$.
These large gadgets will usually result in larger overheads in terms of communication rounds, communication bits, and computations for both parties.
As demonstrated next, lifting with simpler gadgets like XOR allows for a simpler quantum protocol for the lifted problem.

\paragraph*{Lifting Forrelation with XOR.}
The Forrelation function introduced by \cite{Aaronson10} is defined as follows: on input $x=(x_1,x_2)\in\binpm^{n}$ where $n$ is a power of $2$,
$$
\Forr(x)=\frac{2}{n}\ip{H x_1}{x_2},
$$
where $H$ denotes the $(n/2)\times (n/2)$ (unitary) Hadamard matrix.

Girish, Raz, and Tal~\cite{GRT21} studied the XOR-lift of the Forrelation problem and obtained new separations between quantum and randomized communication protocols. 
In more detail, they considered the partial function\footnote{We are overloading the notation here: technically, $\Forr \circ \mathrm{XOR}$ is the XOR-lift of the partial boolean function which on input $x$ outputs $1$ if $\Forr(x)$ is large and $-1$ if $\Forr(x)$ is small.} $\Forr \circ \mathrm{XOR} \colon\binpm^{n}\times\binpm^{n}\to\binpm$ defined as
$$
\Forr \circ \mathrm{XOR}(x,y)=\begin{cases}
1 & \Forr(x\odot y)\ge\frac1{200\ln(n/2)},\\
-1 & \Forr(x\odot y)\le\frac1{400\ln(n/2)},
\end{cases}
$$
and showed that if Alice and Bob use a randomized communication protocol, then they must communicate at least $\tilde{\Omega}(n^{1/4})$ bits to compute $\Forr \circ{\mathrm{XOR}}$; while it can be solved by two entangled parties in the quantum simultaneous message passing model with a $\polylog(n)$-qubit communication protocol and additionally the parties can be implemented with efficient quantum circuits. 

The lower bound in~\cite{GRT21} was obtained from a second level Fourier growth bound (higher levels are not needed) on the XOR-fiber of classical communication protocols.
Our level-two bound strengthens their bound and immediately gives an improved communication lower bound.

\begin{theorem}\label{thm:rcc_xor-lifts_forrelation}
The randomized communication complexity of $\Forr \circ \mathrm{XOR}$ is $\tilde{\Omega}(n^{1/3})$.
\end{theorem}

\Cref{thm:rcc_xor-lifts_forrelation} above gives an $\polylog(n)$ versus $\tilde\Omega(n^{1/3})$ separation between the above quantum communication model and the randomized two-party communication model, improving upon the $\polylog(n)$ versus $\tilde\Omega(n^{1/4})$ separation from \cite{GRT21}. 
We emphasize that our separations are for players with \emph{efficient quantum} running time, where the only prior separation was shown by the aforementioned work~\cite{GRT21}.
Such efficiency features can also benefit real-world implementations to demonstrate quantum advantage in experiments; for instance, one such proposal was introduced recently by Aaronson, Buhrman, and Kretschmer~\cite{aaronson2023qubit}.
Without the efficiency assumption, a better $\polylog(n)$ versus $\tilde\Omega(\sqrt n)$ separation is known \cite{DBLP:journals/tit/Gavinsky20} (see \cite[Section 1.1]{GRT21} for a more detailed comparison). Optimal Fourier growth bounds of $d \cdot \polylog(n)$ for level two, which we state later in \Cref{conjecture:higher_level},  would also imply such a separation with XOR-lift of Forrelation.

\paragraph*{Lifting $k$-Fold Forrelation with XOR.} 

$k$-Fold Forrelation~\cite{DBLP:journals/siamcomp/AaronsonA18} is a generalization of the Forrelation problem and was originally conjectured to be a candidate that exhibits a maximal separation between quantum and classical query complexity. 
In a recent work, \cite{BS21} showed that the randomized query complexity of $k$-Fold Forrelation is $\tilde{\Omega}(n^{1-1/k})$, confirming this conjecture, and a similar separation was proven in \cite{SSW21} for variants of $k$-Fold Forrelation. 
These separations, together with lifting theorems with the \emph{inner product} gadget~\cite{CFKMP19}, imply an $O(k\log(n))$ vs $\tilde\Omega(n^{1-1/k})$ separation between two-party quantum and classical communication complexity, where additionally, the number of rounds\footnote{We remark that for $k=2$, this is exactly the XOR-lift of the Forrelation problem and can even be computed in the quantum simultaneous model, as shown in \cite{GRT21}.} in the two-party quantum protocol is $2\cdot\lceil k/2\rceil$.

Replacing the inner product gadget with the XOR gadget above would yield an improved quantum-classical communication separation where the gadget is simpler and the number of rounds required by the quantum protocol to achieve the same quantitative separation is reduced by half. 
Bansal and Sinha~\cite{BS21} showed that for any computational model, small Fourier growth for the first $O(k^2)$-levels implies hardness of $k$-Fold Forrelation in that particular model. Thus, in conjunction with their results, to prove the above XOR lifting result for the $k$-Fold Forrelation problem, it suffices to prove the following Fourier growth bounds for XOR-fibers. 
\begin{conjecture}\label{conjecture:higher_level}
Let $\Ccal:\binpm^n\times \binpm^n\to \binpm$ be a deterministic communication protocol with at most $d$ bits of communication. Let $h$ be its XOR-fiber as in \Cref{eqn:fiber}. Then for all $k\in \Nbb$, we have that
$L_{1,k}(h) \le (\sqrt{d} \cdot \poly(k,\log(n)))^k$.
\end{conjecture}

Note that these bounds are consistent with the Fourier growth of parity decision trees (or protocols that only send parities) as shown in \cite{GTW21}.

We prove the above conjecture for the case $k=1$ and make progress for the case $k=2$. While our techniques can be extended to higher levels in a straightforward manner, the bounds  obtained are farther from the conjectured ones. Thus, we decided to defer dealing with higher levels to future work as we believe one needs to first prove the \emph{optimal} bound for level $k=2$.

In the next subsection, we give another motivation to study the above conjecture by showing a connection to lifting theorems for constant-sized gadgets.



\subsubsection{General Gadgets and Fourier Growth from Lifting}\label{sec:applications_gadgets}

Our main results are Fourier growth bounds for XOR-fibers, which corresponds to XOR-lifts of functions.
To complement this, we show that similar bounds hold for general lifted functions.

Let $g\colon\Sigma\times\Sigma\to\binpm$ be a gadget and $\Ccal\colon\Sigma^n\times\Sigma^n\to\binpm$ be a communication protocol.
Define the $g$-fiber of $\Ccal$, denoted by $\Ccal_{\downarrow g}\colon\binpm^n\to[-1,1]$, as
$$
\Ccal_{\downarrow g}(z)=\E\sbra{\Ccal(\xbm,\ybm)\mid g(\xbm_i,\ybm_i)=z_i,~\forall i},
$$
where $\xbm$ and $\ybm$ are uniform over $\Sigma$.
We use $L_{1,k}(g,d)$ to denote the upper bound of the level-$k$ Fourier growth for the $g$-fibers of protocols with at most $d$ bits of communication.
Using this notation, the XOR-fiber of $\Ccal$ is simply $\Ccal_{\downarrow\mathrm{XOR}}$, and our main results \Cref{thm:boolean_bound_level_one,thm:boolean_bound_level_two} can be rephrased as
$$
L_{1,1}(\mathrm{XOR},d)\le O\pbra{\sqrt d}
\quad\text{and}\quad
L_{1,2}(\mathrm{XOR},d)\le O\pbra{d^{3/2}\log^3(n)}.
$$

In \Cref{sec:gadget}, we relate $L_{1,k}(g,d)$ to $L_{1,k}(\mathrm{XOR},d)$, and the main takeaway is, in the study of Fourier growth bounds, constant-sized gadgets are all equivalent.

\begin{theorem}[Informal, see \Cref{thm:xor_to_g} and \Cref{thm:g_to_xor}]\label{thm:informal_general_gadget}
Let $g\colon\Sigma\times\Sigma\to\binpm$ be a ``balanced'' gadget.
Then 
$$
|\Sigma|^{-k}\cdot L_{1,k}(\mathrm{XOR},d)\le L_{1,k}(g,d)\le|\Sigma|^k\cdot L_{1,k}(\mathrm{XOR},d).
$$
\end{theorem}

\Cref{thm:informal_general_gadget} also proposes a different approach towards \Cref{conjecture:higher_level}: it suffices to establish tight Fourier growth bound for $g$-fibers for some constant-sized (actually, polylogarithmic size suffices) gadget $g$, and then apply the reduction.
The benefit of switching to a different gadget is that we can perhaps first prove a lifting theorem, and then appeal to the known Fourier growth bounds of (randomized) decision trees \cite{Tal20,SSW21}.
See \Cref{sec:lift} for detail.

As mentioned earlier, lifting theorems show how to simulate communication protocols of cost $d$ for lifted functions with decision trees of depth at most $O(d)$ (see e.g., \cite{GPW20}).
A problem at the frontier of this fruitful line of work has been establishing lifting theorems for decision trees with constant-sized gadgets.
Note that the XOR gadget itself cannot have such a generic lifting result: Indeed, the parity function serves as a counterexample. 
Nevertheless, it is speculative that some larger gadget works, which suffices for our purposes.\footnote{In terms of the separations between quantum and classical communication, even restricted lifting results for the specific outer function being the Forrelation function would suffice.}
On the other hand, for lifting from \emph{parity} decision trees, we do know an XOR-lifting theorem~\cite{HHL18}. However, it only holds for deterministic communication protocols and has a sextic blowup in the cost. 

Thus, one can see \Cref{conjecture:higher_level} as either a further motivation for establishing lifting results for decision trees with constant-sized gadgets, or as a necessary milestone before proving such lifting results.

\subsubsection{Pseudorandomness for Communication Protocols}\label{sec:applications_prg}

We say $G\colon\binpm^\ell\to\binpm^n\times\binpm^n$ is a pseudorandom generator (PRG) for a (randomized) communication protocol $\Ccal\colon\binpm^n\times\binpm^n\to[-1,1]$ with error $\eps$ and seed length $\ell$ if
$$
\abs{\E_{\xbm,\ybm\sim\unif}[\Ccal(\xbm,\ybm)]-\E_{\rbm\sim\binpm^\ell}[\Ccal(G(\rbm))]}\le\eps.
$$
\cite{INW94} showed that for the class of protocols sending at most $d$ communication bits, there exists an explicit PRG of error $2^{-d}$ and seed length $n+O(d)$ from expander graphs.
Note that the overhead $n$ is inevitable even if the protocol is only sending one bit, since it can depend arbitrarily on Alice/Bob's input.

Combining \Cref{conjecture:higher_level} %our second level Fourier bound \Cref{thm:boolean_bound_level_two}
and the PRG construction from \cite[Theorem 4.5]{CHHL19}, we would obtain a completely different explicit PRG for this class with error $\eps$ and seed length $n+d\cdot\polylog(n/\eps)$.
%Though even the optimal second level bound would only improve the seed length to $n+d^{2+o(1)}\cdot\poly(\log(n),1/\eps)$, still weaker than \cite{INW94}, 
%The upshot here is that such a PRG is a black-box application of the Fourier growth bounds.

\paragraph*{Paper Organization.}
An overview of our proofs is given in \Cref{sec:overview}.
In \Cref{sec:prelim} we define necessary notation and recall useful inequalities.
\Cref{sec:fourier_via_martingale} explains a way to associate the Fourier growth to a martingale process. The proof of level-one bound (\Cref{thm:boolean_bound_level_one}) is given in \Cref{sec:proof_of_level_one}, and the level-two bound (\Cref{thm:boolean_bound_level_two}) in \Cref{sec:proof_of_level_two}.
The Fourier growth reductions between general gadgets are presented in \Cref{sec:gadget}.
The future directions are discussed in \Cref{sec:future}.
Missing proofs can be found in the appendix.