\documentclass[journal]{IEEEtran}
\usepackage[utf8]{inputenc}
\usepackage{flushend}
\usepackage{cite}
\usepackage{tabularx}
\usepackage{enumitem}
%%%%%%%%%%%%%%%%%%%%%%%%%%%%%%%%%%%%%%%%%%%%%%%%%%%%%%%%%%%%%%%%%%%%%%%%%%%%%%%%%%
\ifCLASSINFOpdf
\usepackage[pdftex]{graphicx}
\else
\fi
%%%%%%%%%%%%%%%%%%%%%%%%%%%%%%%%%%%%%%%%%%%%%%%%%%%%%%%%%%%%%%%%%%%%%%%%%%%%%%%%%%
\usepackage{amsmath,amsfonts,amsthm,amssymb}
%%%%%%%%%%%%%%%%%%%%%%%%%%%%%%%%%%%%%%%%%%%%%%%%%%%%%%%%%%%%%%%%%%%%%%%%%%%%%%%%%%
\begin{document}
\title{Physics Informed Neural Network for Head-Related Transfer 
Function Upsampling}
\author{Fei~Ma, Thushara D. Abhayapala,
Prasanga N. Samarasinghe, and~Xingyu Chen
%%%%%%%%%%%%%%%%%%%%%%%%%%%%%%%%%%%%%%%%%%%%%%%%%%%%%%%%%%%%%%%%%%%%%%%%%%%%%%%%%%%
\thanks{This work is sponsored by the Australian Research Council (ARC) 
Discovery Projects funding schemes with project numbers DP200100693.}
%%%%%%%%%%%%%%%%%%%%%%%%%%%%%%%%%%%%%%%%%%%%%%%%%%%%%%%%%%%%%%%%%%%%%%%%%%%%%%%%%%%
\thanks{Fei Ma, Thushara D. Abhayapala, and Prasanga N. Samarasinghe 
are all with the Audio and Acoustic Signal Processing Group, 
College of Engineering, Computing \& Cybernetics,
The Australian National University, Canberra, ACT 2601, Australia 
(e-mail: \{fei.ma, Thushara.Abhayapala, prasanga.samarasinghe\}@anu.edu.au).}
%%%%%%%%%%%%%%%%%%%%%%%%%%%%%%%%%%%%%%%%%%%%%%%%%%%%%%%%%%%%%%%%%%%%%%%%%%%%%%%%%%%
\thanks{Manuscript received xxx xxx, xxx; revised xxx xx, xxx.}}
\maketitle





\begin{abstract}
%%%%%%%%%%%%%%%%%%%%%%%%%%%%%%%%%%%%%%%%%%%%%%%%%%%%%%%%%%%%%%%%%%%%%%%%%%
Head-related transfer functions (HRTFs) capture the spatial and spectral 
features that  a person uses to localize sound sources in space, 
and thus are vital for creating authentic virtual acoustic experience.
%%%%%%%%%%%%%%%%%%%%%%%%%%%%%%%%%%%%%%%%%%%%%%%%%%%%%%%%%%%%%%%%%%%%%%%%%%
However, practical HRTF measurement systems can only provide an incomplete
measurement of a person's HRTFs, and this necessitates HRTF upsampling. 
%%%%%%%%%%%%%%%%%%%%%%%%%%%%%%%%%%%%%%%%%%%%%%%%%%%%%%%%%%%%%%%%%%%%%%%%%%
This paper proposes a physics-informed neural network (PINN) method for
HRTF upsampling. 
%%%%%%%%%%%%%%%%%%%%%%%%%%%%%%%%%%%%%%%%%%%%%%%%%%%%%%%%%%%%%%%%%%%%%%%%%%
Unlike other upsampling methods which are based on the measured HRTFs only, 
the PINN method exploits the Helmholtz equation as additional information for 
constraining the upsampling process. 
%%%%%%%%%%%%%%%%%%%%%%%%%%%%%%%%%%%%%%%%%%%%%%%%%%%%%%%%%%%%%%%%%%%%%%%%%%
This helps the PINN method to generate physically amiable upsamplings 
which generalize beyond the measured HRTFs. 
%%%%%%%%%%%%%%%%%%%%%%%%%%%%%%%%%%%%%%%%%%%%%%%%%%%%%%%%%%%%%%%%%%%%%%%%%%
Furthermore, the width and the depth of the PINN are set according to 
the dimensionality of HRTFs under spherical harmonic (SH) decomposition
and the Helmholtz equation.
%%%%%%%%%%%%%%%%%%%%%%%%%%%%%%%%%%%%%%%%%%%%%%%%%%%%%%%%%%%%%%%%%%%%%%%%%%
This makes the PINN to have an appropriate level of expressiveness 
and thus does not suffer from under-fitting and over-fitting problems. 
%%%%%%%%%%%%%%%%%%%%%%%%%%%%%%%%%%%%%%%%%%%%%%%%%%%%%%%%%%%%%%%%%%%%%%%%%%
Numerical experiments confirm the superior performance of the PINN method 
for HRTF upsampling in both interpolation  and  extrapolation scenarios 
over several datasets in comparison with the SH methods.
\end{abstract}


%%%%%%%%%%%%%%%%%%%%%%%%%%%%%%%%%%%%%%%%%%%%%%%%%%%%%%%%%%%%%%%%%%%%%%%%%%%%%%%%%%
\begin{IEEEkeywords}
Head-related transfer function, physics-informed neural network, spherical harmonics, 
spatial audio, virtual acoustics.  
\end{IEEEkeywords}
%%%%%%%%%%%%%%%%%%%%%%%%%%%%%%%%%%%%%%%%%%%%%%%%%%%%%%%%%%%%%%%%%%%%%%%%%%%%%%%%%%
\IEEEpeerreviewmaketitle





%%%%%%%%%%%%%%%%%%%%%%%%%%%%%%%%%%%%%%%%%%%%%%%%%%%%%%%%%%%%%%%%%%%%%%%%%%%%%%%%%%
\section{Introduction}
\label{sec:intro}
%%%%%%%%%%%%%%%%%%%%%%%%%%%%%%%%%%%%%%%%%%%%%%%%%%%%%%%%%%%%%%%%%%%%%%%%
\IEEEPARstart{H}{ead}-related transfer functions (HRTFs) denote the free 
field acoustic transfer functions between a point source and a position 
inside of a person's ear~\cite{hrtf_mea}. 
%%%%%%%%%%%%%%%%%%%%%%%%%%%%%%%%%%%%%%%%%%%%%%%%%%%%%%%%%%%%%%%%%%%%%%%%
HRTFs characterize the filtering effect of a person's torso, head, 
and ears with respect to the direction of sound~\cite{hrtf_mea}, 
and contains the spatial and spectral features that 
a person uses to localize sound sources in space.
%%%%%%%%%%%%%%%%%%%%%%%%%%%%%%%%%%%%%%%%%%%%%%%%%%%%%%%%%%%%%%%%%%%%%%%%
Spatial audio and virtual acoustic systems rely on the knowledge of 
HRTFs to reproduce artificial acoustic experience \cite{hrtf_personal}.  %%%%%%%%%%%%%%%%%%%%%%%%%%%%%%%%%%%%%%%%%%%%%%%%%%%%%%%%%%%%%%%%%%%%%%%%





%%%%%%%%%%%%%%%%%%%%%%%%%%%%%%%%%%%%%%%%%%%%%%%%%%%%%%%%%%%%%%%%%%%%%%%%%%%%
However, the dependence of HRTFs on a person's anatomy makes HRTFs highly 
individualized, and thus that accurate measurements of HRTFs over a large 
number of directions are desired for creating authentic acoustic experience~\cite{hrtf_mea}. 
%%%%%%%%%%%%%%%%%%%%%%%%%%%%%%%%%%%%%%%%%%%%%%%%%%%%%%%%%%%%%%%%%%%%%%%%%%%%
Nonetheless, the complete measurement of HRTFs is both time-consuming
and expensive, which holds most people back from taking their HRTF measurement. 
%%%%%%%%%%%%%%%%%%%%%%%%%%%%%%%%%%%%%%%%%%%%%%%%%%%%%%%%%%%%%%%%%%%%%%%%%%%%
Practical HRTF measurement systems usually have to measure HRTFs over a limited 
number of directions due to the inconvenience of arranging loudspeakers over 
a whole sphere \cite{hrtf_mea} or due to the time constraint on the measurement 
process, resulting an incomplete HRTF dataset.
%%%%%%%%%%%%%%%%%%%%%%%%%%%%%%%%%%%%%%%%%%%%%%%%%%%%%%%%%%%%%%%%%%%%%%%%%%%%
The incompleteness of HRTF datasets motivates researchers to upsample them. 
%%%%%%%%%%%%%%%%%%%%%%%%%%%%%%%%%%%%%%%%%%%%%%%%%%%%%%%%%%%%%%%%%%%%%%%%%%%%


%%%%%%%%%%%%%%%%%%%%%%%%%%%%%%%%%%%%%%%%%%%%%%%%%%%%%%%%%%%%%%%%%%%%%%%%
HRTF upsampling consists of two scenarios: interpolation and extrapolation.
%%%%%%%%%%%%%%%%%%%%%%%%%%%%%%%%%%%%%%%%%%%%%%%%%%%%%%%%%%%%%%%%%%%%%%%%
(Note that we focus on direction related HRTF upsampling.
Distance related HRTF upsampling~\cite{hrtf_dist1,hrtf_dist2,
hrtf_dist3,hrtf_dist4} is not covered by this paper.)
%%%%%%%%%%%%%%%%%%%%%%%%%%%%%%%%%%%%%%%%%%%%%%%%%%%%%%%%%%%%%%%%%%%%%%%
For the interpolation scenario, the HRTFs are measured over a limited number 
of directions due to time constraint, and the aim is to estimate 
the unknown HRTFs whose directions are between those of the measured HRTFs.
%%%%%%%%%%%%%%%%%%%%%%%%%%%%%%%%%%%%%%%%%%%%%%%%%%%%%%%%%%%%%%%%%%%%%%%
Early works on the interpolation are mainly based on the expansion of 
HRTFs on linear functions, such as the spherical harmonics 
(SHs) \cite{SH2004,SH1998,SH2012}, the principle components 
\cite{PCA2012,PCA2013,PCA2020},  the spline functions~\cite{Spline1999},
and the wavelet functions~\cite{wavelet}.
%%%%%%%%%%%%%%%%%%%%%%%%%%%%%%%%%%%%%%%%%%%%%%%%%%%%%%%%%%%%%%%%%%%%%%
Recent works of HRTF interpolation, on the other hand, are mainly 
based non-linear modeling with neural networks (NNs) such as the auto 
encoder~\cite{autoencoder}, the generative adversarial networks~\cite{GAN}, 
and the feature-wise linear modulation~\cite{film}. 
%%%%%%%%%%%%%%%%%%%%%%%%%%%%%%%%%%%%%%%%%%%%%%%%%%%%%%%%%%%%%%%%%%%%%%%%%%%%%%


%%%%%%%%%%%%%%%%%%%%%%%%%%%%%%%%%%%%%%%%%%%%%%%%%%%%%%%%%%%%%%%%%%%%%%%%%%%%%%%%%%
For the extrapolation scenario, HRTFs are measured over a limited polar angle 
range due to the inconvenience of arranging loudspeakers deep below and high 
above a person, and the aim is to estimate the unknown HRTFs beyond the range.   
%%%%%%%%%%%%%%%%%%%%%%%%%%%%%%%%%%%%%%%%%%%%%%%%%%%%%%%%%%%%%%%%%%%%%%%%%%%%%%%%%%
The missing information over a range makes the extrapolation much more 
challenging  than the interpolation. 
%%%%%%%%%%%%%%%%%%%%%%%%%%%%%%%%%%%%%%%%%%%%%%%%%%%%%%%%%%%%%%%%%%%%%%%%%%%%%%%%%%
Up to date, there are only several works about HRTF extrapolation, 
and the majority of them are based on the SH decomposition. 
%%%%%%%%%%%%%%%%%%%%%%%%%%%%%%%%%%%%%%%%%%%%%%%%%%%%%%%%%%%%%%%%%%%%%%%%%%%%%%%%%%
Zhang \textit{et al.} developed an iterative algorithm which successively fills 
and estimates the unknown HRTFs~\cite{hrtf_dir2,hrtf_dir3}, and successfully 
recovered a low order HRTF over a full sphere with one quarter of data missing. 
%%%%%%%%%%%%%%%%%%%%%%%%%%%%%%%%%%%%%%%%%%%%%%%%%%%%%%%%%%%%%%%%%%%%%%%%%%%%%%%%%%
Zotkin \textit{et al.} proposed a regularized  least-square (LS) fit method which 
estimates the unknown HRTFs at the expense of reduced accuracy for representing 
the measured HRTFs~\cite{hrtf_dir1}.  
%%%%%%%%%%%%%%%%%%%%%%%%%%%%%%%%%%%%%%%%%%%%%%%%%%%%%%%%%%%%%%%%%%%%%%%%%%%%%%%%%%
Ahrens \textit{et al.} proposed a non-regularized LS fit method which estimates 
the unknown HRTFs based on a low-order LS fit to the measured
HRTFs and estimations of the unknown HRTFs~\cite{hrtf_dir4}. 
%%%%%%%%%%%%%%%%%%%%%%%%%%%%%%%%%%%%%%%%%%%%%%%%%%%%%%%%%%%%%%%%%%%%%%%%%%%%%%%%%%




%%%%%%%%%%%%%%%%%%%%%%%%%%%%%%%%%%%%%%%%%%%%%%%%%%%%%%%%%%%%%%%%%%%%%%%
One problem shared by most of the above mentioned HRTF upsampling methods 
is that they estimate the unknown HRTFs based on the measured HRTFs only. 
%%%%%%%%%%%%%%%%%%%%%%%%%%%%%%%%%%%%%%%%%%%%%%%%%%%%%%%%%%%%%%%%%%%%%%%
Their estimations are essentially transformations of the 
information that is contained in the measured HRTFs.
They have not effectively use additional information 
to further improve the accuracy of their estimations. 
%%%%%%%%%%%%%%%%%%%%%%%%%%%%%%%%%%%%%%%%%%%%%%%%%%%%%%%%%%%%%%%%%%%%%%%
This fact prompts us to take a different approach to HRTF 
upsampling: the physics-informed neural network (PINN). 
%%%%%%%%%%%%%%%%%%%%%%%%%%%%%%%%%%%%%%%%%%%%%%%%%%%%%%%%%%%%%%%%%%%%%%%





%%%%%%%%%%%%%%%%%%%%%%%%%%%%%%%%%%%%%%%%%%%%%%%%%%%%%%%%%%%%%%%%%%%%%%%%%
The PINN is a special kind of NN which incorporates 
physical knowledge, i.e., the governing partial differential equation 
(PDE) of a physical phenomenon, into its architecture~\cite{pinn1, pinn2,pinn3,pinn4}.  
The physics knowledge is the additional information that can help a 
PINN to model the physic phenomenon beside the physical quantities. 
%%%%%%%%%%%%%%%%%%%%%%%%%%%%%%%%%%%%%%%%%%%%%%%%%%%%%%%%%%%%%%%%%%%%%%%%%
Since the seminal works of Raissi and his colleagues \cite{pinn1,pinn2}, 
PINN has been successfully applied to many areas such as earth quake modeling~\cite{pinn_earth1,pinn_earth2}, 
propeller noise prediction~\cite{pinn_drone}, room acoustics 
\cite{pinn_room}, and sound field estimation~\cite{pinn_soundfield}.    
%%%%%%%%%%%%%%%%%%%%%%%%%%%%%%%%%%%%%%%%%%%%%%%%%%%%%%%%%%%%%%%%%%%%%%%%%%


%%%%%%%%%%%%%%%%%%%%%%%%%%%%%%%%%%%%%%%%%%%%%%%%%%%%%%%%%%%%%%%%%%%%%%%%%%%%%%%%%%
The HRTFs can be regarded as the sound field around the human head, and sound fields
in space obey the Helmholtz equation, the governing equation of acoustic wave propagation~\cite{hrtf_insight}.  
%%%%%%%%%%%%%%%%%%%%%%%%%%%%%%%%%%%%%%%%%%%%%%%%%%%%%%%%%%%%%%%%%%%%%%%%%%%%%%%%%%
This inspires us to develop a PINN method for HRTF upsampling. 
%%%%%%%%%%%%%%%%%%%%%%%%%%%%%%%%%%%%%%%%%%%%%%%%%%%%%%%%%%%%%%%%%%%%%%%%%%%%%%%%%%
We inform the training and the designing of the PINN with physics knowledge 
from two aspects. 
%%%%%%%%%%%%%%%%%%%%%%%%%%%%%%%%%%%%%%%%%%%%%%%%%%%%%%%%%%%%%%%%%%%%%%%%%%%%%%%%%%
First, we use a modified form of the Helmholtz equation as part of the loss function. 
This helps the PINN to generate physically amiable upsamplings which generalize 
beyond the training data, and relieve the burden of balancing the PDE loss and 
the data loss with additional parameters. 
%%%%%%%%%%%%%%%%%%%%%%%%%%%%%%%%%%%%%%%%%%%%%%%%%%%%%%%%%%%%%%%%%%%%%%%%%%%%%%%%%%
Second, we set the size of the PINN according to SH decomposition and 
the Helmholtz equation. 
%%%%%%%%%%%%%%%%%%%%%%%%%%%%%%%%%%%%%%%%%%%%%%%%%%%%%%%%%%%%%%%%%%%%%%%%%%%%%%%%%%
Specifically, we set the width of the PINN as half of the dimensionality 
of HRTFs under SH decomposition \cite{hrtf_insight,Thushara_kr} 
and the depth of as three (the order of the Helmholtz equation plus one). 
%%%%%%%%%%%%%%%%%%%%%%%%%%%%%%%%%%%%%%%%%%%%%%%%%%%%%%%%%%%%%%%%%%%%%%%%%%%%%%%%%%
This sets the proposed PINN method apart from PINN methods in other 
works which suffers from under-fitting or over-fitting problems 
due to inappropriate design of the network~\cite{pinn_pp1,pinn_pp2,pinn_pp3}. 
%%%%%%%%%%%%%%%%%%%%%%%%%%%%%%%%%%%%%%%%%%%%%%%%%%%%%%%%%%%%%%%%%%%%%%%%%%%%%%%%%%
The superior performance of the PINN method for upsampling HRTFs is 
confirmed by numerical experiments on several datasets, and is 
compared with that of the SH method, the most widely used HRTF
upsampling framework. 
%%%%%%%%%%%%%%%%%%%%%%%%%%%%%%%%%%%%%%%%%%%%%%%%%%%%%%%%%%%%%%%%%%%%%%%%%%%%%%%%%%%




%%%%%%%%%%%%%%%%%%%%%%%%%%%%%%%%%%%%%%%%%%%%%%%%%%%%%%%%%%%%%%%%%%%%%%%%%%%%%%%%%%%
The rest of this paper is organized as follows. We introduce the problem in Sec.~II.
We review the SH method in Sec.~III and propose the PINN method in Sec.~IV. 
The performance of the PINN method and the SH method are compared by 
extrapolation and interpolation experiments in Sec. V. 
Section VI concludes this paper and points out future directions of improvement. 
%%%%%%%%%%%%%%%%%%%%%%%%%%%%%%%%%%%%%%%%%%%%%%%%%%%%%%%%%%%%%%%%%%%%%%%%%%%%%%%%%%%%%





%%%%%%%%%%%%%%%%%%%%%%%%%%%%%%%%%%%%%%%%%%%%%%%%%%%%%%%%%%%%%%%%%%%%%%%%%%
\section{Problem formulation}
\label{sec:problem}
%%%%%%%%%%%%%%%%%%%%%%%%%%%%%%%%%%%%%%%%%%%%%%%%%%%%%%%%%%%%%%%%%%%%%%%%%%%
We present the layout of a typical HRTF measurement system in Fig. 
\ref{fig:measurement}, where we  set up a Cartesian coordinate 
system and a spherical coordinate system with respect to the 
center of a person's head, point $O$.
%%%%%%%%%%%%%%%%%%%%%%%%%%%%%%%%%%%%%%%%%%%%%%%%%%%%%%%%%%%%%%%%%%%%%%%%%%%
We denote  the Cartesian coordinates and the spherical coordinates as 
$(x,y,z)$ and $(r,\theta,\phi)$, respectively. 
%are related through  $x=r\sin\theta\cos\phi$, 
%$y=r\sin\theta \sin\phi$, and $z=r\cos\theta$.
%%%%%%%%%%%%%%%%%%%%%%%%%%%%%%%%%%%%%%%%%%%%%%%%%%%%%%%%%%%%%%%%%%%%%%%%%%%
The system measures the HRTFs between the loudspeakers which are placed 
on a sphere $\mathbb{S}_2$  and the microphones which are placed inside 
of the person's ears.
%%%%%%%%%%%%%%%%%%%%%%%%%%%%%%%%%%%%%%%%%%%%%%%%%%%%%%%%%%%%%%%%%%%%%%%%%%%
We denote HRTFs as $P(\omega,r,\theta,\phi)$ in spherical coordinates or 
as $P(\omega,x,y,z)$ in Cartesian coordinates~\cite{rafaely_2015}, where   
$\omega=2\pi{f}$ is the angular frequency  and $f$ is the frequency. 
%%%%%%%%%%%%%%%%%%%%%%%%%%%%%%%%%%%%%%%%%%%%%%%%%%%%%%%%%%%%%%%%%%%%%%%%%%%
Hereafter, we evaluate HRTFs at a single frequency and on a single 
sphere, and thus we skip frequency $\omega$ and the sphere radius 
$r$ when representing acoustic quantities for notation simplicity. 
%%%%%%%%%%%%%%%%%%%%%%%%%%%%%%%%%%%%%%%%%%%%%%%%%%%%%%%%%%%%%%%%%%%%%%%%%%%

% Figure environment removed
%%%%%%%%%%%%%%%%%%%%%%%%%%%%%%%%%%%%%%%%%%%%%%%%%%%%%%%%%%%%%%%%%%%%%%%%%%





%%%%%%%%%%%%%%%%%%%%%%%%%%%%%%%%%%%%%%%%%%%%%%%%%%%%%%%%%%%%%%%%%%%%%%%%%%%
Due to the obstruction of the person's body and the inconvenience of 
arranging the loudspeakers high above the person, the measurement 
system may only be able to measure the HRTFs over a polar range
$(\theta_\mathrm{Low},\theta_\mathrm{High})$, where 
$0<\theta_\mathrm{Low}<\theta_\mathrm{High}<\pi$. 
%%%%%%%%%%%%%%%%%%%%%%%%%%%%%%%%%%%%%%%%%%%%%%%%%%%%%%%%%%%%%%%%%%%%%%%%%%%
Due to the time constraint, the measurement system  may only be able 
to measure the HRTFs over limited number of directions.
%%%%%%%%%%%%%%%%%%%%%%%%%%%%%%%%%%%%%%%%%%%%%%%%%%%%%%%%%%%%%%%%%%%%%%%%%%%
Both of these two scenarios will result in an incomplete HRTF datasets. 
%%%%%%%%%%%%%%%%%%%%%%%%%%%%%%%%%%%%%%%%%%%%%%%%%%%%%%%%%%%%%%%%%%%%%%%%%%%



%%%%%%%%%%%%%%%%%%%%%%%%%%%%%%%%%%%%%%%%%%%%%%%%%%%%%%%%%%%%%%%%%%%%%%%%%%%
In this paper, we aim to upsample an incomplete HRTF dataset 
$\{P(\theta_q,\phi_q)\}_{q=1}^{Q}$ or  equivalently $\{P(x_q,y_q,z_q)\}_{q=1}^{Q}$ into a full or dense dataset. 
%%%%%%%%%%%%%%%%%%%%%%%%%%%%%%%%%%%%%%%%%%%%%%%%%%%%%%%%%%%%%%%%%%%%%%%%%%%




%%%%%%%%%%%%%%%%%%%%%%%%%%%%%%%%%%%%%%%%%%%%%%%%%%%%%%%%%%%%%%%%%%%%%%%%%%%
\section{Spherical harmonic methods}
\label{sec:sh}
In this section, we first briefly present the SH decomposition of 
HRTFs, and then review the regularized SH method for HRTF 
upsampling~\cite{hrtf_dir1}. 
%%%%%%%%%%%%%%%%%%%%%%%%%%%%%%%%%%%%%%%%%%%%%%%%%%%%%%%%%%%%%%%%%%%%%%%%%%%
We express HRTFs in spherical coordinates for the ease of SH 
decomposition. 
%%%%%%%%%%%%%%%%%%%%%%%%%%%%%%%%%%%%%%%%%%%%%%%%%%%%%%%%%%%%%%%%%%%%%%%%%%%



%%%%%%%%%%%%%%%%%%%%%%%%%%%%%%%%%%%%%%%%%%%%%%%%%in spherical coordinates
HRTFs can be decomposed onto SHs and their coefficients as~\cite{william} 
\begin{IEEEeqnarray}{rcl}
\label{eq:SH_first}
\mathbf{P}\approx\mathbf{Y}\mathbf{A}, 
\end{IEEEeqnarray}
where 
$\mathbf{P}=[ 
P(\theta_1,\phi_1),  
P(\theta_1,\phi_2),  
...,  
P(\theta_Q,\phi_Q) ]^{\intercal}$ denote the measured HRTFs at $(\theta_q,\phi_q)_{q=1}^{Q}$ 
($(\cdot)^\intercal$ is the transpose operation),
$\mathbf{A}=[ 
A_{0,0},      
A_{1,-1}, 
A_{1,0},
A_{1,1}, 
..., 
A_{U,U}
]^{\intercal} $ 
denote the SH coefficients, 
and 
\begin{IEEEeqnarray}{rcl}
\mathbf{Y}=
\begin{bmatrix}
\label{eq:sh_second}
Y_{0}^{0}(\theta_1,\phi_1) & 
Y_{1}^{-1}(\theta_1,\phi_1) & 
... &
Y_{U}^{U}(\theta_1,\phi_1)  \\
%%%%%%%%%%%%%%%%%%%%%%%%%%%%%%%
Y_{0}^{0}(\theta_2,\phi_2) & 
Y_{1}^{-1}(\theta_2,\phi_2) & 
...&
Y_{U}^{U}(\theta_2,\phi_2)  \\
%%%%%%%%%%%%%%%%%%%%%%%%%%%%%%%
... & 
... & 
... &
... \\
%%%%%%%%%%%%%%%%%%%%%%%%%%%%%%%
Y_{0}^{0}(\theta_Q,\phi_Q) & 
Y_{1}^{-1}(\theta_Q,\phi_Q) & 
... &
Y_{U}^{U}(\theta_Q,\phi_Q)
\end{bmatrix},\qquad
\end{IEEEeqnarray}
denotes a $Q\times(U+1)^2$ matrix whose entries are order $u$ and 
degree $v$ SH $Y_{u}^{v}(\cdot,\cdot)$ evaluated at 
$(\theta_q,\phi_q)_{q=1}^{Q}$. 



%%%%%%%%%%%%%%%%%%%%%%%%%%%%%%%%%%%%%%%%%%%%%%%%%%%%%%%%%%%%%%%%%%%%%%%%%%%%%%%%
In \eqref{eq:SH_first} and \eqref{eq:sh_second}, $U$ is the dimensionality 
of HRTFs~\cite{hrtf_insight,Thushara_kr} under SH decomposition and is 
normally chosen as 
\begin{IEEEeqnarray}{rcl}
\label{eq:order}
U=\lceil{2\pi{f}r_\mathrm{h}/c}\rceil,
\end{IEEEeqnarray}
where $\lceil{\cdot}\rceil$ is the ceiling operation, $c=343$ m/s is 
the speed of sound propagation, and 
\begin{IEEEeqnarray}{rcl}
r_\mathrm{h}=
\begin{cases}
0.2 \;\mathrm{m}, & f\leq3 \mathrm{kHz},\\ 
0.09 \;\mathrm{m}, & f>3 \mathrm{kHz}, 
\end{cases}
\end{IEEEeqnarray}
is the radius of 
human head (including the head and torso scattering effect) \cite{hrtf_insight,Thushara_kr}. 
%%%%%%%%%%%%%%%%%%%%%%%%%%%%%%%%%%%%%%%%%%%%%%%%%%%%%%%%%%%%%%%%%%%%%%%%%%%%%%%%
In this paper, for simplicity, we choose 
\begin{IEEEeqnarray}{rcl}
\label{eq:size}
U=\lceil{2\pi{f}r_\mathrm{h}/c}\rceil\approx
\begin{cases}
\lceil{f/250}\rceil, & f\leq3 \mathrm{kHz},\\ 
\lceil{f/500}\rceil, & f>3 \mathrm{kHz}, 
\end{cases}
\end{IEEEeqnarray}
and show the dimensionality $U$ of HRTFs as a function of 
frequency in Fig.~\ref{fig:dimensionality} for reference. 
In Fig.~\ref{fig:dimensionality}, we choose $U=\max\{\lceil{}f/250\rceil,\lceil{}f/500\rceil\}$ 
for $3\; \mathrm{kHz} < f< 6 \;\mathrm{kHz} $~\cite{hrtf_insight}. 
%%%%%%%%%%%%%%%%%%%%%%%%%%%%%%%%%%%%%%%%%%%%%%%%%%%%%%%%%%%%%%%%%%%%%%%%%%%%%%%
{Note that the sizes of human heads vary, and thus that 
\eqref{eq:order}, \eqref{eq:size} and Fig.~\ref{fig:dimensionality} 
should be regarded as a rule of thumb and are not supposed to be followed exactly.}
%%%%%%%%%%%%%%%%%%%%%%%%%%%%%%%%%%%%%%%%%%%%%%%%%%%%%%%%%%%%%%%%%%%%%%%%%%%%%%%





% Figure environment removed




%%%%%%%%%%%%%%%%%%%%%%%%%%%%%%%%%%%%%%%%%%%%%%%%%%%%%%%%%%%%%%%%%%%%%%%%%
The regularized SH method first estimates the SH coefficients 
through~\cite{hrtf_dir1}
\begin{IEEEeqnarray}{rcl}
\label{eq:regulation}
\hat{\mathbf{A}}=(\mathbf{Y}^{\intercal}\mathbf{Y}+\gamma\mathbf{H})^{-1}\mathbf{Y}^{\intercal}
\mathbf{P}, 
\end{IEEEeqnarray}
%%%%%%%%%%%%%%%%%%%%%%%%%%%%%%%%%%%%%%%%%%%%%%%%%%%%%%%%%%%%%%%%%%%%%%%%%
where $\mathbf{H}$ is a $(U+1)^2\times(U+1)^2$ diagonal matrix whose diagonal 
entries are $h_{l,l}=1+u(u+1)$ ($u$ is the order of corresponding SH), 
and $\gamma$ is the regularization parameter. 
%%%%%%%%%%%%%%%%%%%%%%%%%%%%%%%%%%%%%%%%%%%%%%%%%%%%%%%%%%%%%%%%%%%%%%%%%
To accurately estimate the SH coefficients up to order $U$, 
we need the number of measured HRTFs to be sufficiently large, 
$Q>(U+1)^2$ or $U<\sqrt{Q}-1$~\cite{hrtf_insight,Thushara_kr}. 
%%%%%%%%%%%%%%%%%%%%%%%%%%%%%%%%%%%%%%%%%%%%%%%%%%%%%%%%%%%%%%%%%%%%%%%%%

%%%%%%%%%%%%%%%%%%%%%%%%%%%%%%%%%%%%%%%%%%%%%%%%%%%%%%%%%%%%%%%%%%%%%%%%%%%%
The regularized SH method then estimate the HRTF from an arbitrary direction 
$(\theta_e,\phi_e)$ as  
\begin{IEEEeqnarray}{rcl}
\label{eq:sh_last}
\hat{P}_{\mathrm{SH}}(\theta_e,\phi_e)
&\approx&\sum_{u=0}^{U}\sum_{v=-u}^{u}\hat{A}_{u,v}Y_{u}^{v}(\theta_e,\phi_e).
\end{IEEEeqnarray}
%%%%%%%%%%%%%%%%%%%%%%%%%%%%%%%%%%%%%%%%%%%%%%%%%%%%%%%%%%%%%%%%%%%%%%%%%%%%
%when $(\theta_e,\phi_e)\not\in(\theta_q,\phi_q)_{q=1}^{Q}$
The regularization in \eqref{eq:regulation} prevents the estimated 
HRTF \eqref{eq:sh_last} from taking exceptionally large value 
by constraining the amplitudes of the estimated SH coefficients 
%%%%%%%%%%%%%%%%%%%%%%%%%%%%%%%%%%%%%%%%%%%%%%%%%%%%%%%%%%%%%%%%%%%%%%%%%%%%
$\hat{\mathbf{A}}=[ 
\hat{A}_{0,0},        
\hat{A}_{1,-1}, 
\hat{A}_{1,0}, 
\hat{A}_{1,1}, 
..., 
\hat{A}_{U,U}
]^{\intercal}$~\cite{hrtf_dir1}, especially the high order coefficients. 
%%%%%%%%%%%%%%%%%%%%%%%%%%%%%%%%%%%%%%%%%%%%%%%%%%%%%%%%%%%%%%%%%%%%%%%%%%%%









%%%%%%%%%%%%%%%%%%%%%%%%%%%%%%%%%%%%%%%%%%%%%%%%%%%%%%%%%%%%%%%%%%%%%%%%%%%%
\section{PINN method}\label{sec:pinn}
In this section, we first briefly introduce the PINN, and then propose a PINN 
method for HRTF upsampling.
%%%%%%%%%%%%%%%%%%%%%%%%%%%%%%%%%%%%%%%%%%%%%%%%%%%%%%%%%%%%%%%%%%%%%%%%%%%%
We express HRTFs in Cartesian coordinates to simplify the calculation of the 
Laplacian by the PINN.
%%%%%%%%%%%%%%%%%%%%%%%%%%%%%%%%%%%%%%%%%%%%%%%%%%%%%%%%%%%%%%%%%%%%%%%%%%%%


%%%%%%%%%%%%%%%%%%%%%%%%%%%%%%%%%%%%%%%%%%%%%%%%%%%%%%%%%%%%%%%%%%%%%%%%%%%%
% Figure environment removed
%%%%%%%%%%%%%%%%%%%%%%%%%%%%%%%%%%%%%%%%%%%%%%%%%%%%%%%%%%%%%%%%%%%%%%%%%%%%


%%%%%%%%%%%%%%%%%%%%%%%%%%%%%%%%%%%%%%%%%%%%%%%%%%%%%%%%%%%%%%%%%%%%%%%%%%%%
We normally build a PINN as a  multiple layer fully connected feed-forward 
neural networks~\cite{pinn1,pinn2,pinn3,pinn4}. 
%%%%%%%%%%%%%%%%%%%%%%%%%%%%%%%%%%%%%%%%%%%%%%%%%%%%%%%%%%%%%%%%%%%%%%%%%%%%
The functionality of one layer is
\begin{equation}
\mathfrak{P}(\mathbf{x})=\sigma\left(\mathbf{x}^T \mathbf{w}+b\right),
\end{equation}
where $\mathbf{x}$ is the input variable vector, $\mathbf{w}$ is the weight 
vector, $b$ is the bias and $\sigma$ is the activation function. 
%%%%%%%%%%%%%%%%%%%%%%%%%%%%%%%%%%%%%%%%%%%%%%%%%%%%%%%%%%%%%%%%%%%%%%%%%%%%
The overall functionality of the PINN is the composition of $L$ layers
\begin{equation}
\Phi(\mathbf{x} ;{\zeta})
=\left(\mathfrak{P}_L \circ \circ  \circ \mathfrak{P}_2 \circ \mathfrak{P}_1\right)(\mathbf{x}),
\end{equation}
where $\zeta$ represents the set of all trainable parameters.
%%%%%%%%%%%%%%%%%%%%%%%%%%%%%%%%%%%%%%%%%%%%%%%%%%%%%%%%%%%%%%%%%%%%%%%%%%%%
We adjust the parameters $\zeta$ by minimizing a cost function 
\begin{equation}
\label{eq:conventionalloss}
\mathfrak{L}=
\frac{1}{Q}\sum_{q=1}^Q\left({p}_q-\Phi(\mathbf{x}_q;\zeta)\right)^2
+\lambda\mathfrak{L}_\mathrm{PDE}(\mathbf{x};\zeta), 
%, \mathbf{x}\in{\Omega},
\end{equation}
%%%%%%%%%%%%%%%%%%%%%%%%%%%%%%%%%%%%%%%%%%%%%%%%%%%%%%%%%%%%%%%%%%%%%%%%%%%%
where $\{\mathbf{x}_q,{p}_q\}_{q=1}^{Q}$ are input-output training data pairs 
which are obtained by testing and measuring a physical system, $\mathfrak{L}_\mathrm{PDE}
(\mathbf{x};\zeta)$ corresponds to the residual of the governing PDE, and 
$\lambda$ is a regularization parameter. 
%%%%%%%%%%%%%%%%%%%%%%%%%%%%%%%%%%%%%%%%%%%%%%%%%%%%%%%%%%%%%%%%%%%%%%%%%%%%










%%%%%%%%%%%%%%%%%%%%%%%%%%%%%%%%%%%%%%%%%%%%%%%%%%%%%%%%%%%%%%%%%%%%%%%%%%%%
For HRTF upsampling, we design a PINN whose structure is shown in
Fig.~\ref{fig:pinn}, where there are $L$ hidden layers with $W$ neurons on each hidden layer, 
with the inputs being the Cartesian coordinates, the activation function being 
$\tanh$, and the output being HRTF estimation $\hat{P}_{\mathrm{PI}}(x,y,z)$. 
%%%%%%%%%%%%%%%%%%%%%%%%%%%%%%%%%%%%%%%%%%%%%%%%%%%%%%%%%%%%%%%%%%%%%%%%%%%%
We adjust the trainable parameters by minimizing the following  cost function
\begin{IEEEeqnarray}{rcl}
\label{eq:cost}
\mathfrak{L}&=& 
\underbrace{
\frac{1}{Q}\sum_{q=1}^{Q}\|
P(x_q,y_q,z_q)-
\hat{P}_{\mathrm{PI}}(x_q,y_q,z_q)
\|_2^2
}_{\mathfrak{L}_{\mathrm{data}}}
\nonumber\\
&+&
\underbrace{
\frac{1}{D}
\sum_{d=1}^{D}
\| \frac{1}{(w/c)^2}\nabla^2{}\hat{P}_{\mathrm{PI}}(x_d,y_d,z_d) 
+\hat{P}_{\mathrm{PI}}(x_d,y_d,z_d) \|_2^2
}_{\mathfrak{L}_{\mathrm{PDE}}}, \IEEEeqnarraynumspace
\end{IEEEeqnarray}
where $\|\cdot\|_2$ is the 2-norm, 
$ \nabla^2\equiv \frac{\partial^2}{\partial{}x^2} 
+ \frac{\partial^2}{\partial{}y^2}    
+ \frac{\partial^2}{\partial{}z^2}$ 
is the Laplacian operator~\cite{william},
$\{x_q,y_q,z_q\}_{q=1}^{Q}$ are the Cartesian coordinates of the measured HRTFs, 
$\{x_d,y_d,z_d\}_{d=1}^{D}$ is a super set of $\{(x_q,y_q,z_q)\}_{q=1}^{Q}$, 
$\mathfrak{L}_{\mathrm{data}}$ and  $\mathfrak{L}_{\mathrm{PDE}}$
denote the data loss and the PDE loss, respectively. 
%%%%%%%%%%%%%%%%%%%%%%%%%%%%%%%%%%%%%%%%%%%%%%%%%%%%%%%%%%%%%%%%%%%%%%%%%%%%


Once trained, the PINN can estimate the HRTF from an arbitrary direction 
$(\theta_e,\phi_e)$ as $\hat{P}_{\mathrm{PI}}(x_e,y_e,z_e)$.
%%%%%%%%%%%%%%%%%%%%%%%%%%%%%%%%%%%%%%%%%%%%%%%%%%%%%%%%%%%%%%%%%%%%%%%%%%%%
Note that we regard the HRTFs as the sound field around the human head, 
and thus that the Cartesian coordinates in \eqref{eq:cost}
correspond to $(r_\mathrm{h},\theta,\phi)$. 
%%%%%%%%%%%%%%%%%%%%%%%%%%%%%%%%%%%%%%%%%%%%%%%%%%%%%%%%%%%%%%%%%%%%%%%%%%%%
Below we explain the design and training of the PINN in detail:
\begin{enumerate}[wide, labelwidth=!, labelindent=0pt]
\item \textbf{Loss}: 
The loss function \eqref{eq:cost} consists of two parts, 
the data loss $\mathfrak{L}_{\mathrm{data}}$ and the
PDE loss $\mathfrak{L}_{\mathrm{PDE}}$.
The data loss $\mathfrak{L}_{\mathrm{data}}$ makes the PINN output  to fit 
the measured HRTFs, or $\hat{P}_{\mathrm{PI}}(x_q,y_q,z_q)\approx{}P(x_q,y_q,z_q)$
for $q\in[1,Q]$. 
%%%%%%%%%%%%%%%%%%%%%%%%%%%%%%%%%%%%%%%%%%%%%%%%%%%%%%%%%%%%%%%%%%%%%%%%%%%%



%%%%%%%%%%%%%%%%%%%%%%%%%%%%%%%%%%%%%%%%%%%%%%%%%%%%%%%%%%%%%%%%%%%%%%%%%%%
The PDE loss $\mathfrak{L}_{\mathrm{PDE}}$ regularizes the PINN output
to conform with the Helmholtz equation, the governing equation of acoustic 
wave propagation, at $\{(x_d,y_d,z_d)\}_{d=1}^{D}$. 
This helps the PINN to generate physical amiable output at and 
beyond  the training data. The regularization in \eqref{eq:regulation},
on the other hand, does not necessarily make the SH methods to 
generate physical amiable output as shown in the experiment section. 
%%%%%%%%%%%%%%%%%%%%%%%%%%%%%%%%%%%%%%%%%%%%%%%%%%%%%%%%%%%%%%%%%%%%%%%%%%%






%%%%%%%%%%%%%%%%%%%%%%%%%%%%%%%%%%%%%%%%%%%%%%%%%%%%%%%%%%%%%%%%%%%%%%%%%%%%
\item 
\textbf{Helmholtz equation:} 
As shown in \eqref{eq:conventionalloss}, PINNs are multiple 
target  optimization problems, and one would balance different loss 
terms with additional parameters such as $\lambda$.  
%%%%%%%%%%%%%%%%%%%%%%%%%%%%%%%%%%%%%%%%%%%%%%%%%%%%%%%%%%%%%%%%%%%%%%%%%%%%
Although tuning the additional parameter may improve the performance of the 
PINN, we decide not to do so because the tuning process can be tedious \cite{multiple_target}. 
%%%%%%%%%%%%%%%%%%%%%%%%%%%%%%%%%%%%%%%%%%%%%%%%%%%%%%%%%%%%%%%%%%%%%%%%%%%%


Instead, we use a special form of the Helmholtz equation as the PDE 
loss in \eqref{eq:cost}, where the $(\omega/c)^2$ term is used as the 
denominator for the Laplacian $\nabla^2$ rather 
than as a multiplier for the PINN output $\hat{P}_{\mathrm{PI}}$. 
%%%%%%%%%%%%%%%%%%%%%%%%%%%%%%%%%%%%%%%%%%%%%%%%%%%%%%%%%%%%%%%%%%%%%%%%%%%%
The modification makes the magnitude of the PDE loss comparable with 
that of data loss, and more importantly reveals a different point of 
view of the Helmholtz equation. That is, the Helmholtz equation can be 
regraded as a fitting to the HRTFs with the Laplacian. 
%%%%%%%%%%%%%%%%%%%%%%%%%%%%%%%%%%%%%%%%%%%%%%%%%%%%%%%%%%%%%%%%%%%%%%%%%%%%

With the magnitudes of the PDE loss and the data loss comparable with 
each other and without any apparent reason to prefer the data fitting 
or the Laplacian fitting,  we simply add the two losses together without 
balancing them with additional parameters. 
%%%%%%%%%%%%%%%%%%%%%%%%%%%%%%%%%%%%%%%%%%%%%%%%%%%%%%%%%%%%%%%%%%%%%%%%%%%%
In this way, the training of the PINN is greatly simplified and does result in 
a good HRTF upsampling performance. 





%%%%%%%%%%%%%%%%%%%%%%%%%%%%%%%%%%%%%%%%%%%%%%%%%%%%%%%%%%%%%%%%%%%%%%%%%%%%
\item 
\textbf{Training 1:}
In Fig.~\ref{fig:ext_8k} (a), Fig.~\ref{fig:ext_16k} (a), and Fig.~\ref{fig:ext_20k} (a), 
we present the normalized amplitudes of HRTFs at three different frequencies.
%%%%%%%%%%%%%%%%%%%%%%%%%%%%%%%%%%%%%%%%%%%%%%%%%%%%%%%%%%%%%%%%%%%%%%%%%%%%
These figures show that the amplitudes of the left parts where $\phi<\pi$  
tend to be different from the amplitudes of the right parts where $\phi>\pi$. 
%%%%%%%%%%%%%%%%%%%%%%%%%%%%%%%%%%%%%%%%%%%%%%%%%%%%%%%%%%%%%%%%%%%%%%%%%%%%
This is typical for HRTFs due to the head shadowing effect, and makes the PINN
unable to estimate the two parts with the same accuracy 
because of their different levels of contributions to the loss (8). 
%%%%%%%%%%%%%%%%%%%%%%%%%%%%%%%%%%%%%%%%%%%%%%%%%%%%%%%%%%%%%%%%%%%%%%%%%%%%
This fact informs us to train the PINN for the left part and right part 
separately, and merge the results afterwards. 
%%%%%%%%%%%%%%%%%%%%%%%%%%%%%%%%%%%%%%%%%%%%%%%%%%%%%%%%%%%%%%%%%%%%%%%%%%%%


%%%%%%%%%%%%%%%%%%%%%%%%%%%%%%%%%%%%%%%%%%%%%%%%%%%%%%%%%%%%%%%%%%%%%%%%%%%%
\item 
\textbf{Training 2:}
%%%%%%%%%%%%%%%%%%%%%%%%%%%%%%%%%%%%%%%%%%%%%%%%%%%%%%%%%%%%%%%%%%%%%%%%%%%%
The HRTFs are complex values which are difficult to model with scalar 
activation functions. 
%%%%%%%%%%%%%%%%%%%%%%%%%%%%%%%%%%%%%%%%%%%%%%%%%%%%%%%%%%%%%%%%%%%%%%%%%%%%
To simplify the training process, we train the PINN with the real part 
and the imaginary part of HRTFs separately, and merge the results 
afterwards. 
%%%%%%%%%%%%%%%%%%%%%%%%%%%%%%%%%%%%%%%%%%%%%%%%%%%%%%%%%%%%%%%%%%%%%%%%%%%%

\item \textbf{PINN size:}
We provide guidance on the size of the PINN,
specifically its width $W$ (the number of neurons on each hidden layer) 
and depth $L$ (hidden layer number). 

%%%%%%%%%%%%%%%%%%%%%%%%%%%%%%%%%%%%%%%%%%%%%%%%%%%%%%%%%%%%%%%%%%%%%%%%%%%%%%%%
First, we set the width $W$ based on the dimensionality 
$U$ of HRTFs under SH decomposition. 
%%%%%%%%%%%%%%%%%%%%%%%%%%%%%%%%%%%%%%%%%%%%%%%%%%%%%%%%%%%%%%%%%%%%%%%%%%%%%%%%
For the PINN method, the width $W$ is the number of components
that are needed to model the HRTFs~\cite{width}.  
%%%%%%%%%%%%%%%%%%%%%%%%%%%%%%%%%%%%%%%%%%%%%%%%%%%%%%%%%%%%%%%%%%%%%%%%%%%%%%%%
For the SH method, the dimensionality $U$ is the maximum complexity of 
HRTFs under SH decomposition.
%%%%%%%%%%%%%%%%%%%%%%%%%%%%%%%%%%%%%%%%%%%%%%%%%%%%%%%%%%%%%%%%%%%%%%%%%%%%%%%%
The similar roles of the width $W$ in the PINN method and the dimensionality $U$ 
in the SH method and the fact that the HRTF is fundamentally a characteristic 
of the human subject irrespective of the basis function 
inspire us to set the width~$W$ according to $U$ \cite{width}.
%%%%%%%%%%%%%%%%%%%%%%%%%%%%%%%%%%%%%%%%%%%%%%%%%%%%%%%%%%%%%%%%%%%%%%%%%%%%%%%%




%%%%%%%%%%%%%%%%%%%%%%%%%%%%%%%%%%%%%%%%%%%%%%%%%%%%%%%%%%%%%%%%%%%%%%%%%%%%%%%%
Based on our experience about SH analysis,  
we find two possible values for the width $W$: $(U+1)^2$ and $2U+1$. 
%%%%%%%%%%%%%%%%%%%%%%%%%%%%%%%%%%%%%%%%%%%%%%%%%%%%%%%%%%%%%%%%%%%%%%%%%%%%%%%%
$(U+1)^2$ is the total number of SHs up to order $U$, and 
$2U+1$ is the number of SHs of order $U$~\cite{Thushara_kr}.  
%%%%%%%%%%%%%%%%%%%%%%%%%%%%%%%%%%%%%%%%%%%%%%%%%%%%%%%%%%%%%%%%%%%%%%%%%%%%%%%%
The SHs form an orthogonal function set where different orders of 
SHs represent different levels of complexity of HRTFs~\cite{hrtf_insight,Thushara_kr}. 
The $\tanh$ functions with different arguments are not orthogonal
with respect to each other, and thus that less number of $\tanh$ functions
together with their arguments maybe sufficient to model HRTFs. 
The difference between SHs and the $\tanh$ function prompts us to 
choose the less value of the two as the width $W=2U+1$. 
%%%%%%%%%%%%%%%%%%%%%%%%%%%%%%%%%%%%%%%%%%%%%%%%%%%%%%%%%%%%%%%%%%%%%%%%%%%%%%%%
Further considering that \textbf{Training 1} and \textbf{Training 2}
each halves the complexity of HRTFs, 
we arrive at the final choice for width $W$
\cite{hrtf_insight,Thushara_kr} 
\begin{IEEEeqnarray}{rcl}
\label{eq:width}
W&=\frac{2U+1}{2\times2}\approx U/2 \approx 
\begin{cases}
\lceil{f/500}\rceil, & f\leq3 \mathrm{kHz},\\ 
\lceil{f/1000}\rceil, & f>3 \mathrm{kHz}. 
\end{cases}
\end{IEEEeqnarray}
%%%%%%%%%%%%%%%%%%%%%%%%%%%%%%%%%%%%%%%%%%%%%%%%%%%%%%%%%%%%%%%%%%%%
We present the width $W$ as a function of frequency in Fig.~\ref{fig:dimensionality} 
for reference,  where we let $W=\max\{  \lceil{}f/500\rceil, \lceil{}f /1000\rceil\}$ 
for 3 kHz $<f<$ 6 kHz \cite{hrtf_insight}. 
%%%%%%%%%%%%%%%%%%%%%%%%%%%%%%%%%%%%%%%%%%%%%%%%%%%%%%%%%%%%%%%%%%%%
Note that similar to \eqref{eq:order} and \eqref{eq:size}, the width 
\eqref{eq:width} should be regraded as a rule of thumb and is not
supposed to be followed exactly. 
%%%%%%%%%%%%%%%%%%%%%%%%%%%%%%%%%%%%%%%%%%%%%%%%%%%%%%%%%%%%%%%%%%%%
Indeed, for HRTF upsampling, we are facing a data insufficient 
condition, and to avoid the over-fitting problem we may need to 
slightly reduce the width $W$.  
%%%%%%%%%%%%%%%%%%%%%%%%%%%%%%%%%%%%%%%%%%%%%%%%%%%%%%%%%%%%%%%%%%%%







%%%%%%%%%%%%%%%%%%%%%%%%%%%%%%%%%%%%%%%%%%%%%%%%%%%%%%%%%%%%%%%%%%%%
Second, with the width of the PINN set to be $W=U/2$, 
we find that a depth of $L=3$ is sufficient and necessary 
for HRTF modeling. 
%%%%%%%%%%%%%%%%%%%%%%%%%%%%%%%%%%%%%%%%%%%%%%%%%%%%%%%%%%%%%%%%%%%%
This maybe because the Helmholtz equation is a second order PDE which 
requires three conditions to specify a unique solution. 
%%%%%%%%%%%%%%%%%%%%%%%%%%%%%%%%%%%%%%%%%%%%%%%%%%%%%%%%%%%%%%%%%%%%

Further theoretical investigations are necessary to justify the 
choices and provide better guidance for the PINN design. 
This will be one of our future works. 
\end{enumerate}

%%%%%%%%%%%%%%%%%%%%%%%%%%%%%%%%%%%%%%%%%%%%%%%%%%%%%%%%%%%%%%%%%%%%%%%%%%%%
\section{\label{sec:exp}Numerical experiments}
In this section, we compare the PINN method performance with that of the 
SH method which has provided valuable inspirations to the design of the PINN  
and has been widely used for both extrapolation and interpolation scenarios. 
%%%%%%%%%%%%%%%%%%%%%%%%%%%%%%%%%%%%%%%%%%%%%%%%%%%%%%%%%%%%%%%%%%%%%%%%%%%%





%%%%%%%%%%%%%%%%%%%%%%%%%%%%%%%%%%%%%%%%%%%%%%%%%%%%%%%%%%%%%%%%%%%%%%%%%%%%
Let $(\theta,\phi)$ be the direction of a HRTF. We transfer the spherical
coordinates $(r_\mathrm{h},\theta,\phi)$ into Cartesian coordinates.  
%%%%%%%%%%%%%%%%%%%%%%%%%%%%%%%%%%%%%%%%%%%%%%%%%%%%%%%%%%%%%%%%%%%%%%%%%%%%
We use these Cartesian coordinates as the inputs to the PINN. 
We normalize the amplitudes of HRTFs to be within $[-1, 1]$. 
%%%%%%%%%%%%%%%%%%%%%%%%%%%%%%%%%%%%%%%%%%%%%%%%%%%%%%%%%%%%%%%%%%%%%%%%%%%%
We conduct upsampling on the amplitude of HRTFs rather on the magnitude 
to avoid potential impacts on perceptual features of HRTFs.
%%%%%%%%%%%%%%%%%%%%%%%%%%%%%%%%%%%%%%%%%%%%%%%%%%%%%%%%%%%%%%%%%%%%%%%%%%%%
We evaluate the performance of all methods by the upsampling error  
\begin{IEEEeqnarray}{rcl}
\mathcal{E}&=& 10\log_{10} 
\frac{\sum_{e=1}^{E} \| P(\theta_e,\phi_e) - \hat{P}(\theta_e,\phi_e)
\|_2^2}{\sum_{e=1}^{E}\|P(\theta_e,\phi_e)\|_2^2}, 
\end{IEEEeqnarray}
%and the fitting error 
%\begin{IEEEeqnarray}{rcl}
%\mathcal{E}_{\mathrm{fitting}} &=& 10\log_{10} 
%\frac{\sum_{q=1}^{1368} \| P(\theta_q,\phi_q) - \hat{P}(\theta_q,\phi_q)
%\|_2^2}{\sum_{q=1}^{1368}\|P(\theta_q,\phi_q)\|_2^2}, 
%\end{IEEEeqnarray}
where $P(\theta_e,\phi_e)$ and $\hat{P}(\theta_e,\phi_e)$ are the unknown 
HRTFs and are their estimations at $\{(\theta_e,\phi_e)\}_{e=1}^{E}$, 
respectively. 
%%%%%%%%%%%%%%%%%%%%%%%%%%%%%%%%%%%%%%%%%%%%%%%%%%%%%%%%%%%%%%%%%%%%%%%%%%%%


%%%%%%%%%%%%%%%%%%%%%%%%%%%%%%%%%%%%%%%%%%%%%%%%%%%%%%%%%%%
% Figure environment removed
%%%%%%%%%%%%%%%%%%%%%%%%%%%%%%%%%%%%%%%%%%%%%%%%%%%%%%%%%%%%%%%%%%%%%%%%%%%%
\subsection{\label{sec:ext}Extrapolation}
In this section, we aim to extrapolate the unknown HRTFs whose directions 
are beyond those of the measured HRTFs.
%%%%%%%%%%%%%%%%%%%%%%%%%%%%%%%%%%%%%%%%%%%%%%%%%%%%%%%%%%%%%%%%%%%%%%%%%%%%


%%%%%%%%%%%%%%%%%%%%%%%%%%%%%%%%%%%%%%%%%%%%%%%%%%%%%%%%%%%%%%%%%%%%%%%%%%%%
We conduct experiments on HRTFs from the 3D3A dataset~\cite{3d3a}. 
%%%%%%%%%%%%%%%%%%%%%%%%%%%%%%%%%%%%%%%%%%%%%%%%%%%%%%%%%%%%%%%%%%%%%%%%%%%%
We set the simulated HRTFs of subject 37 left ear at 100, 200, ..., 16000 Hz 
as the ground truth.
%%%%%%%%%%%%%%%%%%%%%%%%%%%%%%%%%%%%%%%%%%%%%%%%%%%%%%%%%%%%%%%%%%%%%%%%%%%%
We use $\{P(\theta_q,\phi_q)\}_{q=1}^{1368}$ whose 
polar angle $0.2\pi\leq\theta_q\leq0.8\pi$ as the measured HRTFs.
% for the SH regularized fit and for the PINN training. 
%%%%%%%%%%%%%%%%%%%%%%%%%%%%%%%%%%%%%%%%%%%%%%%%%%%%%%%%%%%%%%%%%%%%%%%%%%%%
We aim to extrapolate the unknown HRTFs 
$\{P(\theta_e,\phi_e)\}_{e=1}^{362}$ where $\theta_e<0.2\pi$ or $\theta_e>0.8\pi$. 
%%%%%%%%%%%%%%%%%%%%%%%%%%%%%%%%%%%%%%%%%%%%%%%%%%%%%%%%%%%%%%%%%%%%%%%%%%%%



%%%%%%%%%%%%%%%%%%%%%%%%%%%%%%%%%%%%%%%%%%%%%%%%%%%%%%%%%%%%%%%%%%%%%%%%%%%%
We implement the SH method~\cite{hrtf_dir1} following \eqref{eq:SH_first} 
- \eqref{eq:sh_last}, and set $\gamma=0.02$ in \eqref{eq:regulation} 
according to a trial-and-error process. 
%%%%%%%%%%%%%%%%%%%%%%%%%%%%%%%%%%%%%%%%%%%%%%%%%%%%%%%%%%%%%%%%%%%%%%%%%%%%

%%%%%%%%%%%%%%%%%%%%%%%%%%%%%%%%%%%%%%%%%%%%%%%%%%%%%%%%%%%%%%%%%%%%%%%%%%%%
For the PINN method, we initialize the trainable parameters with the 
Xavier initialization \cite{init}.
%%%%%%%%%%%%%%%%%%%%%%%%%%%%%%%%%%%%%%%%%%%%%%%%%%%%%%%%%%%%%%%%%%%%%%%%%%
We train the PINN for 10$^7$  epochs with a learning rate of 10$^{-5}$ 
using the ADAM optimizer. 
%%%%%%%%%%%%%%%%%%%%%%%%%%%%%%%%%%%%%%%%%%%%%%%%%%%%%%%%%%%%%%%%%%%%%%%%%%
We evaluate the data loss $\mathfrak{L}_\mathrm{data}$  
with respect to the 1368 measured HRTFs, 
%%%%%%%%%%%%%%%%%%%%%%%%%%%%%%%%%%%%%%%%%%%%%%%%%%%%%%%%%%%%%%%%%%%%%%%%%%
and the PDE loss $\mathfrak{L}_\mathrm{PDE}$ with respect to the Cartesian 
coordinates of all 1368+362=1730 HRTFs.
%%%%%%%%%%%%%%%%%%%%%%%%%%%%%%%%%%%%%%%%%%%%%%%%%%%%%%%%%%%%%%%%%%%%%%%%%%






%%%%%%%%%%%%%%%%%%%%%%%%%%%%%%%%%%%%%%%%%%%%%%%%%%%%%%%%%%%%%%%%%%%%%%%%%%
We implement an additional method which is similar to the PINN method, 
except that we do not add the PDE loss $\mathfrak{L}_\mathrm{PDE}$ 
to the loss function~\eqref{eq:cost}. Hereafter, we refer to this method 
as the NN method. 
%%%%%%%%%%%%%%%%%%%%%%%%%%%%%%%%%%%%%%%%%%%%%%%%%%%%%%%%%%%%%%%%%%%%%%%%%%



%%%%%%%%%%%%%%%%%%%%%%%%%%%%%%%%%%%%%%%%%%%%%%%%%%%%%%%%%%%%%%%%%%%%%%%%%
We first conduct the experiment at 8 kHz. We present the real 
part of the ground truth HRTF in Fig.~\ref{fig:ext_8k} (a), where
the dashed lines $\theta=0.2\pi$  and  $\theta=0.8\pi$ are the boundaries 
between the measured and the unknown HRTFs.
%%%%%%%%%%%%%%%%%%%%%%%%%%%%%%%%%%%%%%%%%%%%%%%%%%%%%%%%%%%%%%%%%%%%%%%%%
Figure \ref{fig:ext_8k} (b) denotes the estimation obtained from the 
SH method with $U=16$.
%%%%%%%%%%%%%%%%%%%%%%%%%%%%%%%%%%%%%%%%%%%%%%%%%%%%%%%%%%%%%%%%%%%%%%%%%
Figure \ref{fig:ext_8k} (c) and (d) denote the estimations obtained 
from the NN method with $L=3, W=16$ and $L=3, W=8$, respectively.
%%%%%%%%%%%%%%%%%%%%%%%%%%%%%%%%%%%%%%%%%%%%%%%%%%%%%%%%%%%%%%%%%%%%%%%%%
Figure \ref{fig:ext_8k} (e) and (f) denote the estimations obtained 
from the PINN method with $L=3, W=16$ and $L=3, W=8$, respectively.
%%%%%%%%%%%%%%%%%%%%%%%%%%%%%%%%%%%%%%%%%%%%%%%%%%%%%%%%%%%%%%%%%%%%%%%%%
Figure \ref{fig:ext_8k} (g) and (h) denote the estimations obtained 
from the PINN method with $L=2, W=8$ and $L=4, W=8$, respectively.

%%%%%%%%%%%%%%%%%%%%%%%%%%%%%%%%%%%%%%%%%%%%%%%%%%%%%%%%%%%%%%%%%%%%%%%%%
From Fig.~\ref{fig:ext_8k} (b), we can see that the upsampling accuracy 
of the SH method for the unknown HRTFs, where $\theta_e<0.2\pi$ or 
$\theta_e>0.8\pi$, is poor. 
%%%%%%%%%%%%%%%%%%%%%%%%%%%%%%%%%%%%%%%%%%%%%%%%%%%%%%%%%%%%%%%%%%%%%%%%%
From Fig.~\ref{fig:ext_8k} (c), we can see that the NN method with width 
$W=16$ estimates the unknown HRTFs with spurious values. 
%%%%%%%%%%%%%%%%%%%%%%%%%%%%%%%%%%%%%%%%%%%%%%%%%%%%%%%%%%%%%%%%%%%%%%%%%
This problem is mitigated in Fig.~\ref{fig:ext_8k} (d) by reducing the 
width to be $W=8$. However, the estimation is still not satisfying as 
the estimated unknown HRTFs show a curve around $\theta=0.9\pi$
that does not present in the ground truth. 
%%%%%%%%%%%%%%%%%%%%%%%%%%%%%%%%%%%%%%%%%%%%%%%%%%%%%%%%%%%%%%%%%%%%%%%%%
Figure~\ref{fig:ext_8k} (b), (c), and (d) reveal the problem of 
the SH method and the NN method. That is, they have no control over the 
estimations and thus can assign arbitrary values to the unknown HRTFs.
%%%%%%%%%%%%%%%%%%%%%%%%%%%%%%%%%%%%%%%%%%%%%%%%%%%%%%%%%%%%%%%%%%%%%%%%%








%%%%%%%%%%%%%%%%%%%%%%%%%%%%%%%%%%%%%%%%%%%%%%%%%%%%%%%%%%%%%%%%%%%%%%%%%
The PINN method, on the other hand, has some control over the 
estimations,  because the PDE loss $\mathfrak{L}_\mathrm{PDE}$  
contains the contribution from the unknown HRTFs. 
%%%%%%%%%%%%%%%%%%%%%%%%%%%%%%%%%%%%%%%%%%%%%%%%%%%%%%%%%%%%%%%%%%%%%%%%%
Figure~\ref{fig:ext_8k} (e) shows that the PINN method with 
$W=16$ tends to assign zero to the unknown HRTFs.  
%%%%%%%%%%%%%%%%%%%%%%%%%%%%%%%%%%%%%%%%%%%%%%%%%%%%%%%%%%%%%%%%%%%%%%%%%
Zero is a valid but trivial solution of the Helmtholtz equation~\cite{pinn_pp1}.
Large area of zero value estimation indicates that the PINN method 
with $W=16$ has more expressiveness than it needs to be, causing the over-fitting problem. 
%%%%%%%%%%%%%%%%%%%%%%%%%%%%%%%%%%%%%%%%%%%%%%%%%%%%%%%%%%%%%%%%%%%%%%%%%
Figure~\ref{fig:ext_8k} (f) shows that the PINN method with an 
appropriate level of expressiveness, or width $W=8$, 
can accurately estimate the unknown HRTFs.  
%without assigning  either large values or zero to the unknown HRTFs. 
%%%%%%%%%%%%%%%%%%%%%%%%%%%%%%%%%%%%%%%%%%%%%%%%%%%%%%%%%%%%%%%%%%%%%%%%%
Figure~\ref{fig:ext_8k} (g) shows that the PINN method with less depth 
$L=2$ and hence less expressiveness, omits some details shown in
the ground truth HRTFs. This is the under-fitting problem.   
%%%%%%%%%%%%%%%%%%%%%%%%%%%%%%%%%%%%%%%%%%%%%%%%%%%%%%%%%%%%%%%%%%%%%%%%%
Figure~\ref{fig:ext_8k} (h) shows that the PINN method with more depth 
$L=4$ and hence more expressiveness, suffers from the over-fitting 
problem as Fig.~\ref{fig:ext_8k} (e) because it also tends to assign 
zero to the unknown HRTFs.

%%%%%%%%%%%%%%%%%%%%%%%%%%%%%%%%%%%%%%%%%%%%%%%%%%%%%%%%%%%%%%%%%%%%%%%%%
The upsampling errors of the SH method Fig.~\ref{fig:ext_8k} (b), 
the NN method with $W=16$ Fig.~\ref{fig:ext_8k} (c), 
and $W=8$ Fig.~\ref{fig:ext_8k} (d) are 1.6 dB, -1.4 dB, and -10.8 dB, 
respectively.  
%%%%%%%%%%%%%%%%%%%%%%%%%%%%%%%%%%%%%%%%%%%%%%%%%%%%%%%%%%%%%%%%%%%%%%%%%
The upsampling errors of the PINN methods are -5.7 dB, -17.5 dB, -12.3 dB,
and -14.5 dB for Fig.~\ref{fig:ext_8k} (e), (f), (g), and (h), respectively.
%%%%%%%%%%%%%%%%%%%%%%%%%%%%%%%%%%%%%%%%%%%%%%%%%%%%%%%%%%%%%%%%%%%%%%%%%


%%%%%%%%%%%%%%%%%%%%%%%%%%%%%%%%%%%%%%%%%%%%%%%%%%%%%%%%%%%%%%%%%%%%%%%%%
This experiment demonstrates the superior HRTF upsampling performance of 
the PINN method.
%%%%%%%%%%%%%%%%%%%%%%%%%%%%%%%%%%%%%%%%%%%%%%%%%%%%%%%%%%%%%%%%%%%%%%%%%
However, the good performance can only be achieved with the incorporation of 
physics knowledge and proper design of the PINN.
%%%%%%%%%%%%%%%%%%%%%%%%%%%%%%%%%%%%%%%%%%%%%%%%%%%%%%%%%%%%%%%%%%%%%%%%%


%%%%%%%%%%%%%%%%%%%%%%%%%%%%%%%%%%%%%%%%%%%%%%%%%%%%%%%%%%%%%%%%%%%%%%%%%%
Hereafter, unless otherwise stated, we assume the $U$ for the SH method is 
calculated through~\eqref{eq:order}, and the depth and the width for both 
the NN method and the PINN method are $L=3$ and $W=U/2$, respectively. 
%%%%%%%%%%%%%%%%%%%%%%%%%%%%%%%%%%%%%%%%%%%%%%%%%%%%%%%%%%%%%%%%%%%%%%%%%%



%%%%%%%%%%%%%%%%%%%%%%%%%%%%%%%%%%%%%%%%%%%%%%%%%%%%%%%%%%%%%%%%%%%%%%%%%%%%
% Figure environment removed
%%%%%%%%%%%%%%%%%%%%%%%%%%%%%%%%%%%%%%%%%%%%%%%%%%%%%%%%%%%%%%%%%%%%%%%%%%
% Figure environment removed
%%%%%%%%%%%%%%%%%%%%%%%%%%%%%%%%%%%%%%%%%%%%%%%%%%%%%%%%%%%%%%%%%%%%%%%%%





%%%%%%%%%%%%%%%%%%%%%%%%%%%%%%%%%%%%%%%%%%%%%%%%%%%%%%%%%%%%%%%%%%%%%%%%%
We next repeat the experiment on the same 3D3A HRTFs but at 16 kHz.  
We present the real part  of the ground truth HRTF 
in Fig.~\ref{fig:ext_16k} (a). 
%%%%%%%%%%%%%%%%%%%%%%%%%%%%%%%%%%%%%%%%%%%%%%%%%%%%%%%%%%%%%%%%%%%%%%%%%
Figure \ref{fig:ext_16k} (b), (c), and (d) denote the estimations 
obtained from the SH method, the NN method, and the PINN method, 
respectively.
%%%%%%%%%%%%%%%%%%%%%%%%%%%%%%%%%%%%%%%%%%%%%%%%%%%%%%%%%%%%%%%%%%%%%%%%%
In this case, the SH method fails to estimate to the unknown 
HRTFs.
%%%%%%%%%%%%%%%%%%%%%%%%%%%%%%%%%%%%%%%%%%%%%%%%%%%%%%%%%%%%%%%%%%%%%%%%%
The NN method, without any control over the estimation process, 
assigns spurious values to the unknown HRTFs.
%%%%%%%%%%%%%%%%%%%%%%%%%%%%%%%%%%%%%%%%%%%%%%%%%%%%%%%%%%%%%%%%%%%%%%%%%
The PINN method, on the other hand, can estimate the unknown HRTFs 
with a better  accuracy.
%%%%%%%%%%%%%%%%%%%%%%%%%%%%%%%%%%%%%%%%%%%%%%%%%%%%%%%%%%%%%%%%%%%%%%%%%%
The upsampling errors of the SH method, the NN method, and 
the PINN method are 3.1 dB, 5.4 dB, and -4.8 dB, respectively. 
%%%%%%%%%%%%%%%%%%%%%%%%%%%%%%%%%%%%%%%%%%%%%%%%%%%%%%%%%%%%%%%%%%%%%%%%%%

\begin{table}[t]
\caption{Upsampling errors of the SH method and the PINN method for the 3D3A HRTFs}
\vspace{0.1cm}
\centering
\begin{tabularx}{0.49\textwidth} 
{ 
|>{\centering\arraybackslash}X
|>{\centering\arraybackslash}X
|>{\centering\arraybackslash}X 
|>{\centering\arraybackslash}X 
|>{\centering\arraybackslash}X|}                                   \hline
          $f$ (kHz)   & 2      &     4  &  6     & 8      \\ \hline
$\mathrm{SH}$   & -34.9 dB  &  -12.5 dB &  -1.3 dB  & 1.6 dB    \\ \hline
$\mathrm{PINN}$ & -35.8 dB  &  -30.4 dB &  -28.3 dB & -17.5 dB  \\ \hline
%%%%%%%%%%%%%%%%%%%%%%%%%%%%%%%%%%%%%%%%%%%%%%%%%%%%%%%%%%%%%%%%%%%%%%%%%%%%%%%%%%%%%%%
         $f$ (kHz)   &  10   &  12   &  14   & 16    \\ \hline
$\mathrm{SH}$   &  1.4 dB   &  2.6 dB   &  2.3 dB   &  3.1 dB   \\ \hline
$\mathrm{PINN}$ & -14.8 dB  & -9.3 dB  &  -5.5 dB  & -4.8 dB    \\ \hline
\end{tabularx}
\end{table}
%%%%%%%%%%%%%%%%%%%%%%%%%%%%%%%%%%%%%%%%%%%%%%%%%%%%%%%%%%%%%%%%%%%%%%%%%%
We present the HRTFs upsampling errors of the SH method and the PINN method
for the 3D3A HRTFs over a broad frequency range in Table 1. 
(The upsampling errors of the NN method are consistently larger than 
that of the PINN methods, and thus are not shown.)
%%%%%%%%%%%%%%%%%%%%%%%%%%%%%%%%%%%%%%%%%%%%%%%%%%%%%%%%%%%%%%%%%%%%%%%%%%
As shown in Table 1, the upsampling errors of the PINN method is smaller
than that of the SH method over the whole range. 
%%%%%%%%%%%%%%%%%%%%%%%%%%%%%%%%%%%%%%%%%%%%%%%%%%%%%%%%%%%%%%%%%%%%%%%%%%
The experiment results for upsampling the imaginary part of the 3D3A HRTFs
using the three methods are similar to Fig.~\ref{fig:ext_8k},
Fig.~\ref{fig:ext_16k}, and Table 1, and thus are not shown for brevity.  
%%%%%%%%%%%%%%%%%%%%%%%%%%%%%%%%%%%%%%%%%%%%%%%%%%%%%%%%%%%%%%%%%%%%%%%%%%



%%%%%%%%%%%%%%%%%%%%%%%%%%%%%%%%%%%%%%%%%%%%%%%%%%%%%%%%%%%%%%%%%%%%%%%%%%%%
We repeat the experiment on the high resolution spherical 
nearfield (HRSN) dataset~\cite{nf48}. 
%%%%%%%%%%%%%%%%%%%%%%%%%%%%%%%%%%%%%%%%%%%%%%%%%%%%%%%%%%%%%%%%%%%%%%%%%%%%
The dataset contains HRTFs of the Neumann KU100 dummy head which is  
measured at different distances. 
%%%%%%%%%%%%%%%%%%%%%%%%%%%%%%%%%%%%%%%%%%%%%%%%%%%%%%%%%%%%%%%%%%%%%%%%%%%%
We use the right ear HRTFs measured with a loudspeaker array on a 1.0 m radius sphere 
at frequencies 375, 750, ..., 20250 Hz as the ground truth. 
%%%%%%%%%%%%%%%%%%%%%%%%%%%%%%%%%%%%%%%%%%%%%%%%%%%%%%%%%%%%%%%%%%%%%%%%%%%%
We use $\{P(\theta_q,\phi_q)\}_{q=1}^{2157}$ whose 
polar angle $0.2\pi\leq\theta_q\leq0.8\pi$ as the measured HRTFs.
We aim to extrapolate the unknown HRTFs 
$\{P(\theta_e,\phi_e)\}_{e=1}^{545}$ where $\theta_e<0.2\pi$ or $\theta_e>0.8\pi$. 
%%%%%%%%%%%%%%%%%%%%%%%%%%%%%%%%%%%%%%%%%%%%%%%%%%%%%%%%%%%%%%%%%%%%%%%%%%%%
The implementations of the SH method, the NN method, and the PINN method
are similar to their implementations for the 3D3A HRTFs.
%%%%%%%%%%%%%%%%%%%%%%%%%%%%%%%%%%%%%%%%%%%%%%%%%%%%%%%%%%%%%%%%%%%%%%%%%%%%

%The upsampling results of three methods  at lower frequency range where $f\leq16$ 
%kHz are similar to Fig.~\ref{fig:ext_8k} and Fig.~\ref{fig:ext_16k}, and thus 
%are not shown for brevity. 
%%%%%%%%%%%%%%%%%%%%%%%%%%%%%%%%%%%%%%%%%%%%%%%%%%%%%%%%%%%%%%%%%%%%%%%%%%%%
We show the upsampling results at a high frequency
$f=20.25$ kHz in Fig.~\ref{fig:ext_20k}, where Fig.~\ref{fig:ext_20k} (a)
shows the real part of the ground truth, and Fig.~\ref{fig:ext_20k} (b), (c), 
and (d) show the estimations obtained from the SH method, the NN method, 
and the PINN method, respectively.
%%%%%%%%%%%%%%%%%%%%%%%%%%%%%%%%%%%%%%%%%%%%%%%%%%%%%%%%%%%%%%%%%%%%%%%%%
Comparing Fig.~\ref{fig:ext_20k} with Fig.~\ref{fig:ext_16k}, we can see 
that the upsampling result of three methods for the 3D3A HRTFs at 16 kHz
and for the HRSN HRTFs at 20.25 kHz are similar. The SH method fails 
the upsampling task. The NN method assigns spurious values to the unknown 
HRTFs. The upsampling accuracy of the PINN method is the best among the 
three methods.  
%%%%%%%%%%%%%%%%%%%%%%%%%%%%%%%%%%%%%%%%%%%%%%%%%%%%%%%%%%%%%%%%%%%%%%%%%%
The upsampling errors of the SH method, the NN method, and 
the PINN method are 4.6 dB, 10.7 dB, and -3.2 dB, respectively. 
%%%%%%%%%%%%%%%%%%%%%%%%%%%%%%%%%%%%%%%%%%%%%%%%%%%%%%%%%%%%%%%%%%%%%%%%%%



\begin{table}[t]
\caption{Upsampling errors of the SH method and the PINN method for the HRSN HRTFs}
\vspace{0.1cm}
\centering
\begin{tabularx}{0.49\textwidth} 
{ 
|>{\centering\arraybackslash}X
|>{\centering\arraybackslash}X
|>{\centering\arraybackslash}X 
|>{\centering\arraybackslash}X 
|>{\centering\arraybackslash}X 
|>{\centering\arraybackslash}X|}                                            \hline
          $f$ (kHz)   & 2.25     &     4.125 &  6.0    & 8.25     & 10.125  \\ \hline
$\mathrm{SH}$   & -12.8 dB  &  -3.3 dB &  0.5 dB  & 3.6 dB   & 5.5 dB       \\ \hline
$\mathrm{PINN}$ & -16.4 dB  &  -12.1 dB &  -12.3 dB & -11.7 dB & -10.3      \\ \hline
%%%%%%%%%%%%%%%%%%%%%%%%%%%%%%%%%%%%%%%%%%%%%%%%%%%%%%%%%%%%%%%%%%%%%%%%%%%%%%%%%%%%%%%
         $f$ (kHz)   &  12.0   &  14.25    &  16.125    & 18.0   & 20.25   \\ \hline
$\mathrm{SH}$   &  3.4 dB   &  6.6 dB   &  4.3 dB   &  5.1 dB  &  4.6 dB     \\ \hline
$\mathrm{PINN}$ & -8.8 dB  & -6.3 dB  &  -5.5 dB  & -4.8 dB    & -3.2 dB     \\ \hline
\end{tabularx}
\end{table}
%%%%%%%%%%%%%%%%%%%%%%%%%%%%%%%%%%%%%%%%%%%%%%%%%%%%%%%%%%%%%%%%%%%%%%%%%%
We present the HRTFs upsampling errors of the SH method and the PINN method
for the HRSN HRTFs over a broad frequency range in Table 2. 
(The upsampling errors of the NN methods are larger than that of the PINN method,
and thus are not shown.)
%%%%%%%%%%%%%%%%%%%%%%%%%%%%%%%%%%%%%%%%%%%%%%%%%%%%%%%%%%%%%%%%%%%%%%%%%%
Comparing Table 1 and Table 2, we can see that the upsampling errors of both 
methods for the measured HRSN HRTFs are larger than the corresponding errors 
for the simulated 3D3A HRTFs. 
For these two datasets, the upsampling errors the PINN methods are
consistently smaller than that of the SH method. 
%%%%%%%%%%%%%%%%%%%%%%%%%%%%%%%%%%%%%%%%%%%%%%%%%%%%%%%%%%%%%%%%%%%%%%%%%%
The experiment results for upsampling the imaginary part of the HRSN HRTFs
using the three methods are similar to Fig.~\ref{fig:ext_20k},
and Table 2, and thus are not shown for brevity.  
%%%%%%%%%%%%%%%%%%%%%%%%%%%%%%%%%%%%%%%%%%%%%%%%%%%%%%%%%%%%%%%%%%%%%%%%%%













\begin{table}[t]
\caption{Upsampling errors of PINN method for the CHEDAR dataset and HUTUBS dataset at 10 kHz
for 5 subjects}
\vspace{0.1cm}
\centering
\begin{tabularx}{0.49\textwidth} 
{ 
|>{\centering\arraybackslash}X
|>{\centering\arraybackslash}X
|>{\centering\arraybackslash}X 
|>{\centering\arraybackslash}X 
|>{\centering\arraybackslash}X 
|>{\centering\arraybackslash}X|}                                  \hline
CHEDAR Subject  & 3         &   18      &     29     &  57       &  89         \\ \hline
PINN & -11.9 dB  & -9.8 dB  &  -11.7 dB &  -8.9 dB  & -10.1 dB    \\ \hline
%%%%%%%%%%%%%%%%%%%%%%%%%%%%%%%%%%%%%%%%%%%%%%%%%%%%%%%%%%%%%%%%%%%%%%%%%%%%%%%%%%%%%%%
HUTUBS Subject &  7        &  35      &  46       &  78       &  99         \\ \hline
PINN  & -11.8 dB  & -14.0 dB &  -11.8 dB & -12.1 dB  & -12.5 dB    \\ \hline
\end{tabularx}
\end{table}



%%%%%%%%%%%%%%%%%%%%%%%%%%%%%%%%%%%%%%%%%%%%%%%%%%%%%%%%%%%%%%%%%%%%%%%%%%%%%%%%%%%%%%%
The extrapolation experiments are further conducted on the CHEDAR dataset~\cite{chedar}
and the HUTUBS dataset~\cite{hutubs}. We randomly select five subjects from these two 
datasets, and conduct HRTF upsampling at 10 kHz. Similar to the 3D3A case
and the HRSN case, we extrapolate the unknown HRTFs beyond $0.2\pi\leq\theta\leq0.8\pi$
based on the HRTFs within the range.
We present the upsampling errors of the PINN method together with subject number in 
Table III. 
%%%%%%%%%%%%%%%%%%%%%%%%%%%%%%%%%%%%%%%%%%%%%%%%%%%%%%%%%%%%%%%%%%%%%%%%%%%%%%%%%%%%%%%
As shown in Table III,  for these two datasets, the PINN method can achieve upsampling 
errors around -10 dB and -12 dB, respectively.
%%%%%%%%%%%%%%%%%%%%%%%%%%%%%%%%%%%%%%%%%%%%%%%%%%%%%%%%%%%%%%%%%%%%%%%%%%%%%%%%%%%%%%%
The upsampling errors of the PINN method on other frequencies for these two datasets
are similar to the Table I and Table II, and thus are not shown for brevity. 
%%%%%%%%%%%%%%%%%%%%%%%%%%%%%%%%%%%%%%%%%%%%%%%%%%%%%%%%%%%%%%%%%%%%%%%%%%%%%%%%%%%%%%%


\subsection{\label{sec:int}Interpolation}
In this section, we aim to interpolate the unknown HRTFs whose directions are 
between those of the measured HRTFs.


%%%%%%%%%%%%%%%%%%%%%%%%%%%%%%%%%%%%%%%%%%%%%%%%%%%%%%%%%%%%%%%%%%%%%%%%%%
We conduct the experiment on same 3D3A HRTFs. 
We use HRTFs $\{P(\theta_q,\phi_q)\}_{q=1}^{1368}$ 
whose polar angle $0.2\pi\leq\theta_q\leq0.8\pi$ as the ground truth. 
%%%%%%%%%%%%%%%%%%%%%%%%%%%%%%%%%%%%%%%%%%%%%%%%%%%%%%%%%%%%%%%%%%%%%%%%%%
We randomly select one third (456) of the ground truth HRTFs as the measured HRTFs 
and the rest (912) as the unknown HRTFs.
We show the arrangement of the measured HRTFs and the unknown HRTFs in 
Fig.~\ref{fig:scatter} for reference.   
%%%%%%%%%%%%%%%%%%%%%%%%%%%%%%%%%%%%%%%%%%%%%%%%%%%%%%%%%%%%%%%%%%%%%%%%%%
Note that as shown in Fig.~\ref{fig:scatter}, there are regions where no 
HRTFs are measured, and this make Fig.~\ref{fig:scatter} not necessary the 
best arrangement for testing the performance of the PINN method. 
%%%%%%%%%%%%%%%%%%%%%%%%%%%%%%%%%%%%%%%%%%%%%%%%%%%%%%%%%%%%%%%%%%%%%%%%%%
Further investigation of the optimal arrangement of the measured HRTFs 
is beyond the scope of this paper, and will be one of our future works. 
%%%%%%%%%%%%%%%%%%%%%%%%%%%%%%%%%%%%%%%%%%%%%%%%%%%%%%%%%%%%%%%%%%%%%%%%%%






%%%%%%%%%%%%%%%%%%%%%%%%%%%%%%%%%%%%%%%%%%%%%%%%%%%%%%%%%%%%%%%%%%%%%%%%%%%%
The implementations of the SH method and the  PINN method are the same as 
in Sec.~\ref{sec:ext}, except that for the SH method we set $\gamma=0$ 
according to a trial-and-error process. 
%%%%%%%%%%%%%%%%%%%%%%%%%%%%%%%%%%%%%%%%%%%%%%%%%%%%%%%%%%%%%%%%%%%%%%%%%%%%
We do have implemented the NN method, but the same as in Sec.~\ref{sec:ext}, 
the performance of the NN method is inferior to that of the PINN method. 
Thus, the results of the NN method are not shown for brevity. 
%%%%%%%%%%%%%%%%%%%%%%%%%%%%%%%%%%%%%%%%%%%%%%%%%%%%%%%%%%%%%%%%%%%%%%%%%%%%


%%%%%%%%%%%%%%%%%%%%%%%%%%%%%%%%%%%%%%%%%%%%%%%%%%%%%%%%%%%%%%%%%%%%%%%%%
We compare the performance of two methods at 8 kHz in Fig.~\ref{fig:int_8k},  
where Fig.~\ref{fig:int_8k} (a) shows the real part of the ground truth,
and Fig.~\ref{fig:int_8k} (b) shows the measured HRTFs.
%%%%%%%%%%%%%%%%%%%%%%%%%%%%%%%%%%%%%%%%%%%%%%%%%%%%%%%%%%%%%%%%%%%%%%%%%
Figure \ref{fig:int_8k} (c) and (d) denote the estimations 
obtained from the SH method and from the PINN method, respectively.
%%%%%%%%%%%%%%%%%%%%%%%%%%%%%%%%%%%%%%%%%%%%%%%%%%%%%%%%%%%%%%%%%%%%%%%%%
%Note that the performances of the SH method and the PINN method for 
%$\phi<\pi$ and $\phi>\pi$ are similar, and thus to to save space 
%only the $\phi<\pi$ part of the HRTFs are shown in Fig.~\ref{fig:8khz}.
%%%%%%%%%%%%%%%%%%%%%%%%%%%%%%%%%%%%%%%%%%%%%%%%%%%%%%%%%%%%%%%%%%%%%%%%%%
In this case, the dimensionality of the HRTFs under SH decomposition is 
$U=16$, which is less than $\sqrt{456}-1$~\cite{hrtf_insight,Thushara_kr}. 
The measured HRTFs provides the SH method with enough information to accurately
estimate the SH coefficients up to order $U=16$. 
The upsampling error of the SH method is -14.9 dB.
%%%%%%%%%%%%%%%%%%%%%%%%%%%%%%%%%%%%%%%%%%%%%%%%%%%%%%%%%%%%%%%%%%%%%%%%%%
Comparing Fig.~\ref{fig:int_8k} (c) and (d), we can see that the PINN method 
can estimate the unknown HRTFs with a better accuracy. 
%%%%%%%%%%%%%%%%%%%%%%%%%%%%%%%%%%%%%%%%%%%%%%%%%%%%%%%%%%%%%%%%%%%%%%%%%%
The upsampling error of the PINN method is -23.4 dB.
%%%%%%%%%%%%%%%%%%%%%%%%%%%%%%%%%%%%%%%%%%%%%%%%%%%%%%%%%%%%%%%%%%%%%%%%%%







%%%%%%%%%%%%%%%%%%%%%%%%%%%%%%%%%%%%%%%%%%%%%%%%%%%%%%%%%%%%%%%%%%%%%%%%%%%%
% Figure environment removed
%%%%%%%%%%%%%%%%%%%%%%%%%%%%%%%%%%%%%%%%%%%%%%%%%%%%%%%%%%%%%%%%%%%%%%%%%%%%



%%%%%%%%%%%%%%%%%%%%%%%%%%%%%%%%%%%%%%%%%%%%%%%%%%%%%%%%%%%%%%%%%%%%%%%%%%%%
% Figure environment removed

%%%%%%%%%%%%%%%%%%%%%%%%%%%%%%%%%%%%%%%%%%%%%%%%%%%%%%%%%%%%%%%%%%%%%%%%%%%%
% Figure environment removed
%%%%%%%%%%%%%%%%%%%%%%%%%%%%%%%%%%%%%%%%%%%%%%%%%%%%%%%%%%%%%%%%%%%%%%%%%%%%
% Figure environment removed
\textbf{check in the figure $\theta$ and $\phi$ is wrong.} 
%%%%%%%%%%%%%%%%%%%%%%%%%%%%%%%%%%%%%%%%%%%%%%%%%%%%%%%%%%%



We compare the performance of two methods at 16 kHz in Fig.~\ref{fig:int_16k},  
where Fig.~\ref{fig:int_16k} (a) shows the real part of the ground truth,
and Fig.~\ref{fig:int_16k} (b) shows the measured HRTFs.
%%%%%%%%%%%%%%%%%%%%%%%%%%%%%%%%%%%%%%%%%%%%%%%%%%%%%%%%%%%%%%%%%%%%%%%%%
Figure \ref{fig:int_16k} (c) and (d) denote the estimations 
obtained from the SH method and the PINN method, respectively. 
%%%%%%%%%%%%%%%%%%%%%%%%%%%%%%%%%%%%%%%%%%%%%%%%%%%%%%%%%%%%%%%%%%%%%%%%%
Comparing Fig.~\ref{fig:int_16k} (a) and Fig.~\ref{fig:int_8k} (a), 
we can see that the HRTFs at 16 kHz is much more complex then the 
HRTFs at 8 kHz. 
%%%%%%%%%%%%%%%%%%%%%%%%%%%%%%%%%%%%%%%%%%%%%%%%%%%%%%%%%%%%%%%%%%%%%%%%%%
Figure \ref{fig:int_16k} (b) shows more discontinues regions than 
Fig.~\ref{fig:int_8k} (b), and this makes it more challenging to 
estimate the unknown HRTFs.  
%%%%%%%%%%%%%%%%%%%%%%%%%%%%%%%%%%%%%%%%%%%%%%%%%%%%%%%%%%%%%%%%%%%%%%%%%
In this case, the dimensionality of the HRTF under SH decomposition is 
$U=32$, which is more than $\sqrt{456}-1$~\cite{hrtf_insight,Thushara_kr}. 
The measured HRTFs do not provide enough information for the SH method to 
accurately estimate the SH coefficients up to order $U=32$.  
%%%%%%%%%%%%%%%%%%%%%%%%%%%%%%%%%%%%%%%%%%%%%%%%%%%%%%%%%%%%%%%%%%%%%%%%%%
Fig.~\ref{fig:int_16k} (c) shows that the SH method estimation has 
some dark and bright spots which do not present in the ground truth, 
and the upsampling error is $-0.8$ dB.
%%%%%%%%%%%%%%%%%%%%%%%%%%%%%%%%%%%%%%%%%%%%%%%%%%%%%%%%%%%%%%%%%%%%%%%%%%
Fig.~\ref{fig:int_16k} (d) shows that the PINN method has accurately 
estimated the ground truth HRTFs, and the upsampling error is $-14.5$ dB.
%%%%%%%%%%%%%%%%%%%%%%%%%%%%%%%%%%%%%%%%%%%%%%%%%%%%%%%%%%%%%%%%%%%%%%%%%%

\begin{table}[t]
\caption{Upsampling errors of the SH method and the PINN method for the 3D3A HRTFs}
\vspace{0.1cm}
\centering
\begin{tabularx}{0.49\textwidth} 
{ 
|>{\centering\arraybackslash}X
|>{\centering\arraybackslash}X
|>{\centering\arraybackslash}X 
|>{\centering\arraybackslash}X 
|>{\centering\arraybackslash}X|}                                   \hline
          $f$ (kHz)   & 2      &     4  &  6     & 8      \\ \hline
$\mathrm{SH}$   & -21.8 dB  &  -18.4 dB &  -14.1 dB  & -14.9 dB    \\ \hline
$\mathrm{PINN}$ & -34.0 dB  &  -26.8 dB &  -21.5 dB & -23.4 dB  \\ \hline
%%%%%%%%%%%%%%%%%%%%%%%%%%%%%%%%%%%%%%%%%%%%%%%%%%%%%%%%%%%%%%%%%%%%%%%%%%%%%%%%%%%%%%%
         $f$  (kHz)  &  10   &  12   &  14    & 16    \\ \hline
$\mathrm{SH}$   &  -3.4 dB   &  0.7 dB   &  -0.3 dB   &  -0.8 dB   \\ \hline
$\mathrm{PINN}$ & -18.1 dB  & -15.6 dB  &  -14.9 dB  & -14.5 dB    \\ \hline
\end{tabularx}
\end{table}
%%%%%%%%%%%%%%%%%%%%%%%%%%%%%%%%%%%%%%%%%%%%%%%%%%%%%%%%%%%%%%%%%%%%%%%%%%
We present the HRTFs upsampling errors of the SH method and the PINN method
for the 3D3A HRTFs over a broad frequency range in Table IV. 
%%%%%%%%%%%%%%%%%%%%%%%%%%%%%%%%%%%%%%%%%%%%%%%%%%%%%%%%%%%%%%%%%%%%%%%%%%
As shown in Table IV, the upsampling errors of the PINN method is consistently 
smaller than that of the SH method. 
%%%%%%%%%%%%%%%%%%%%%%%%%%%%%%%%%%%%%%%%%%%%%%%%%%%%%%%%%%%%%%%%%%%%%%%%%%
The experiment results for upsampling the imaginary part of the 3D3A HRTFs
using the two methods are similar to Table IV, and thus are not shown for 
brevity.  
%%%%%%%%%%%%%%%%%%%%%%%%%%%%%%%%%%%%%%%%%%%%%%%%%%%%%%%%%%%%%%%%%%%%%%%%%%







We repeat the interpolation experiment on the HRSN HRTFs.  
%%%%%%%%%%%%%%%%%%%%%%%%%%%%%%%%%%%%%%%%%%%%%%%%%%%%%%%%%%%%%%%%%%%%%%%%%%%%
We use $\{P(\theta_q,\phi_q)\}_{q=1}^{2157}$ whose 
polar angle $0.2\pi\leq\theta_q\leq0.8\pi$ as the ground truth.
%%%%%%%%%%%%%%%%%%%%%%%%%%%%%%%%%%%%%%%%%%%%%%%%%%%%%%%%%%%%%%%%%%%%%%%%%%%%
Similar to the interpolation for the 3D3A HRTFs, we randomly select 
one third (719) of the ground truth HRTFs as the measured HRTFs
and the rest (1438) as the unknown HRTFs. 
%%%%%%%%%%%%%%%%%%%%%%%%%%%%%%%%%%%%%%%%%%%%%%%%%%%%%%%%%%%%%%%%%%%%%%%%%%%%
The arrangement of the measured HRTFs and the unknown HRTFs is similar to
Fig.~\ref{fig:scatter}, and thus is not shown for brevity. 
%%%%%%%%%%%%%%%%%%%%%%%%%%%%%%%%%%%%%%%%%%%%%%%%%%%%%%%%%%%%%%%%%%%%%%%%%%%%
The implementations of the SH method and the PINN method
are similar to their implementations for the 3D3A HRTFs.
%%%%%%%%%%%%%%%%%%%%%%%%%%%%%%%%%%%%%%%%%%%%%%%%%%%%%%%%%%%%%%%%%%%%%%%%%%%%
We show the upsampling results at a high frequency $f=20.25$ kHz in 
Fig.~\ref{fig:int_20k},  
where Fig.~\ref{fig:int_20k} (a) shows the real part of the ground truth,
Fig.~\ref{fig:int_20k} (b) shows the measured HRTFs.
Figure~\ref{fig:int_20k} (c) and (d) denote the estimations 
obtained from the SH method and the PINN method, respectively. 
The upsampling errors of the SH method and the PINN method are 
$-1.6$ dB and $-13.1$ dB, respectively. 
%%%%%%%%%%%%%%%%%%%%%%%%%%%%%%%%%%%%%%%%%%%%%%%%%%%%%%%%%%%%%%%%%%%%%%%%%%%%



\begin{table}[t]
\caption{Upsampling errors of the SH method and the PINN method for the HRSN HRTFs}
\vspace{0.1cm}
\centering
\begin{tabularx}{0.49\textwidth} 
{ 
|>{\centering\arraybackslash}X
|>{\centering\arraybackslash}X
|>{\centering\arraybackslash}X
|>{\centering\arraybackslash}X 
|>{\centering\arraybackslash}X 
|>{\centering\arraybackslash}X|}                                            \hline
          $f$  (kHz)  & 2.25&     4.125  &  6.0    & 8.25  & 10.125  \\ \hline
$\mathrm{SH}$   & -17.8 dB  &  -13.0 dB &  -12.1 dB  & -7.9 dB   & -5.5 dB       \\ \hline
$\mathrm{PINN}$ & -28.8 dB  &  -21.1 dB &  -18.7 dB & -15.8 dB & -15.6 dB      \\ \hline
%%%%%%%%%%%%%%%%%%%%%%%%%%%%%%%%%%%%%%%%%%%%%%%%%%%%%%%%%%%%%%%%%%%%%%%%%%%%%%%%%%%%%%%
         $f$ (kHz)   &  12.0 &  14.25    &  16.125   & 18.0    & 20.25   \\ \hline
$\mathrm{SH}$   &  -1.3 dB  &  -0.5 dB   &  -2.8 dB   &  -1.1 dB  &  -1.6 dB     \\ \hline
$\mathrm{PINN}$ & -15.9 dB  & -14.0 dB  &  -15.5 dB  & -15.3 dB    & -13.1 dB     \\ \hline
\end{tabularx}
\end{table}
%%%%%%%%%%%%%%%%%%%%%%%%%%%%%%%%%%%%%%%%%%%%%%%%%%%%%%%%%%%%%%%%%%%%%%%%%%
We present the HRTFs upsampling errors of the SH method and the PINN method
for the HRSN HRTFs over a broad frequency range in Table V. 
%%%%%%%%%%%%%%%%%%%%%%%%%%%%%%%%%%%%%%%%%%%%%%%%%%%%%%%%%%%%%%%%%%%%%%%%%%
Similar to Table IV, the upsampling errors of the PINN method is smaller
than that of the SH method over the broad frequency range. 
%%%%%%%%%%%%%%%%%%%%%%%%%%%%%%%%%%%%%%%%%%%%%%%%%%%%%%%%%%%%%%%%%%%%%%%%%%
The experiment results for upsampling the imaginary part of the HRSN HRTFs
using the two methods are similar to Table V, and thus are not shown for brevity.  
%%%%%%%%%%%%%%%%%%%%%%%%%%%%%%%%%%%%%%%%%%%%%%%%%%%%%%%%%%%%%%%%%%%%%%%%%%


\begin{table}[t]
\caption{Upsampling errors of PINN method for the HUTUBS dataset and IRCAM 
dataset at 10 kHz for 5 subjects}
\vspace{0.1cm}
\centering
\begin{tabularx}{0.49\textwidth} 
{ 
|>{\centering\arraybackslash}X
|>{\centering\arraybackslash}X
|>{\centering\arraybackslash}X 
|>{\centering\arraybackslash}X 
|>{\centering\arraybackslash}X 
|>{\centering\arraybackslash}X|}                                  \hline
%%%%%%%%%%%%%%%%%%%%%%%%%%%%%%%%%%%%%%%%%%%%%%%%%%%%%%%%%%%%%%%%%%%%%%%%%%%%%%%%%%%%%%%
HUTUBS Subject &  5        &  35      &  47       &  73       &  89         \\ \hline
PINN  & -25.1 dB  & -20.6 dB &  -23.3 dB & -22.0 dB  & -23.7 dB    \\ \hline
IRCAM Subject  & 9         &   25      &     36     &  44       &  56         \\ \hline
PINN & -11.9 dB  & -9.8 dB  &  -11.7 dB &  -8.9 dB  & -10.1 dB    \\ \hline
\end{tabularx}
\end{table}



%%%%%%%%%%%%%%%%%%%%%%%%%%%%%%%%%%%%%%%%%%%%%%%%%%%%%%%%%%%%%%%%%%%%%%%%%%%%%%%%%%%%%%%
The interpolation experiments are further conducted on the 
HUTUBS dataset~\cite{hutubs} and the IRCAM dataset~\cite{ircam}.
We randomly select five subjects from these two 
datasets, and conduct HRTF upsampling at 10 kHz. 
%%%%%%%%%%%%%%%%%%%%%%%%%%%%%%%%%%%%%%%%%%%%%%%%%%%%%%%%%%%%%%%%%%%%%%%%%%%%%%%%%%%%%%%
Similar to the 3D3A case and the HRSN case, we use the HRTFs where  
$0.2\pi\leq\theta\leq0.8\pi$ as the ground truth. We randomly select 
one third of the ground truth as the measured HRTFs and the rest 
as the unknown HRTFs.
We aim to estimate the unknown HRTFs based on the measured ones. 
%%%%%%%%%%%%%%%%%%%%%%%%%%%%%%%%%%%%%%%%%%%%%%%%%%%%%%%%%%%%%%%%%%%%%%%%%%%%%%%%%%%%%%%
We present the upsampling errors of the PINN method together with subject number in 
Table VI. 
%%%%%%%%%%%%%%%%%%%%%%%%%%%%%%%%%%%%%%%%%%%%%%%%%%%%%%%%%%%%%%%%%%%%%%%%%%%%%%%%%%%%%%%
As shown in Table VI,  for the HUTUBS dataset, the PINN method can achieve upsampling 
errors less than -20 dB.   
%%%%%%%%%%%%%%%%%%%%%%%%%%%%%%%%%%%%%%%%%%%%%%%%%%%%%%%%%%%%%%%%%%%%%%%%%%%%%%%%%%%%%%%
The upsampling errors for the IRCAM dataset are around -10 dB, which is 
not that satisfying. 
%%%%%%%%%%%%%%%%%%%%%%%%%%%%%%%%%%%%%%%%%%%%%%%%%%%%%%%%%%%%%%%%%%%%%%%%%%%%%%%%%%%%%%%
This maybe because unlike other datasets which contain either simulated or 
measured HRTFs of artificial heads, the IRCAM dataset contains measured 
HRTFs of human subjects from 1680 directions. 
It is unclear whether human subjects can keep still over the  measurement process. 
%%%%%%%%%%%%%%%%%%%%%%%%%%%%%%%%%%%%%%%%%%%%%%%%%%%%%%%%%%%%%%%%%%%%%%%%%%%%%%%%%%%%%%%
The upsampling errors of the PINN method on other frequencies for these two datasets
show similar trends as in Table IV and Table V. That is, the upsampling errors
increase with the increment of frequencies. The results are not shown for brevity. 
%%%%%%%%%%%%%%%%%%%%%%%%%%%%%%%%%%%%%%%%%%%%%%%%%%%%%%%%%%%%%%%%%%%%%%%%%%%%%%%%%%%%%%%




\section{Conclusion}\label{sec:con}
%%%%%%%%%%%%%%%%%%%%%%%%%%%%%%%%%%%%%%%%%%%%%%%%%%%%%%%%%%%%%%%%%%%%%%%%%%%%%%
This paper proposed a PINN method for upsampling HRTFs.
The performance of most existing HRTF upsampling methods 
is limited by the fact that they use the information of 
the measured HRTFs only. 
%%%%%%%%%%%%%%%%%%%%%%%%%%%%%%%%%%%%%%%%%%%%%%%%%%%%%%%%%%%%%%%%%%%%%%%%%%%%%%
The proposed PINN method exploits the Helmholtz equation,
the governing differential equation of acoustics, as additional 
information to improve the HRTF upsampling accuracy.  
%%%%%%%%%%%%%%%%%%%%%%%%%%%%%%%%%%%%%%%%%%%%%%%%%%%%%%%%%%%%%%%%%%%%%%%%%%%%%%
Furthermore, based on the SH decomposition of the HRTFs and 
the Helmholtz equation, we set the PINN with an appropriate 
width and depth. This helps the PINN to avoid under-fitting and over-fitting 
problems.   
%%%%%%%%%%%%%%%%%%%%%%%%%%%%%%%%%%%%%%%%%%%%%%%%%%%%%%%%%%%%%%%%%%%%%%%%%%%%%%
The additional information provided by the Helmholtz equation 
and a suitable size help the PINN to outperform the SH method 
in both extrapolation and interpolation scenarios.
%%%%%%%%%%%%%%%%%%%%%%%%%%%%%%%%%%%%%%%%%%%%%%%%%%%%%%%%%%%%%%%%%%%%%%%%%%%%%%


The design of the PINN is still empirical, and we need further 
theoretical investigation to determine an optimal size of the PINN.  
Another interesting extension of this work would be incorporating even 
more physics information, such as head and ear geometry, into the design 
and training of the PINN. This will be one of our future works. 



%%%%%%%%%%%%%%%%%%%%%%%%%%%%%%%%%%%%%%%%%%%%%%%%%%%%%%%%%%%%%%%%%%%%%%%%%%%%%%
%\vfill\pagebreak
\begin{thebibliography}{1,0}
\def\enquote#1,{``#1,''}
\expandafter\ifx\csname url\endcsname\relax
\def\url#1{\texttt{#1}}\fi
\expandafter\ifx\csname urlprefix\endcsname\relax\def\urlprefix{URL }\fi
\providecommand{\bibinfo}[2]{#2}
\def\plainquote#1{``#1''}
\providecommand{\noopsort}[1]{}
\providecommand{\switchargs}[2]{#2#1}
\providecommand{\dourl}[1]{\href{http://#1}{\nolinkurl{#1}}}
\providecommand{\dodoi}[1]{doi: \href{http://dx.doi.org/#1}{\nolinkurl{#1}}}
\def\eatspace #1{#1}

\bibitem{hrtf_mea}
S. Li and J. Peissig, 
``Measurement of head-related transfer functions: a review", 
{\em Appl. Sciences}, vol. 10, no. 14, pp. 5014, 2020.

\bibitem{hrtf_personal}
W. Zhang, P. N. Samarasinghe, H. Chen, and T. D. Abhayapala, 
``Surround by sound: a review of spatial audio recording and reproduction", 
{\em Appl. Sci.}, vol. 7, no. 6, May 2017.

\bibitem{hrtf_dist1}
 L. S. Zhou, C. C. Bao, M. S. Jia, and B. Bu, 
 ``Range extrapolation of head-related transfer function using improved higher order ambisonics", {\em APSIPA ASC 2014}, pp. 1-4, 2014.

\bibitem{hrtf_dist2}
 H. Gamper, 
 ``Head-related transfer function interpolation in azimuth, elevation, and distance”, 
 {\em J. Acoust. Soc. Am.}, vol. 134, no. 6, pp. 533–547, 2013.

\bibitem{hrtf_dist3}
M. Pollow, K. V. Nguyen, O. Warusfel, T. Carpentier, M. Muller-Trapet,
M. Vorlander, and M. Noisternig, 
``Calculation of head-related transfer
functions for arbitrary field points using spherical harmonics decomposition", 
{\em Acta. Acustica united with Acustica}, vol. 98, no. 1, pp. 72–82, 2012.


\bibitem{hrtf_dist4}
S. Spors and J. Ahrens, 
``Interpolation and range extrapolation of head-related transfer functions 
using virtual local sound field synthesis", {\em 130th Conv. AES}, May 2011.
%%%%%%%%%%%%%%%%%%%%%%%%%%%%%%%%%%%%%%%%%%%%%%%%%%%%%%%%%%%%%%%%%%%%%%%%%%%%%%
\bibitem{SH2004}
R. Duraiswaini, D. N Zotkin, and N. A Gumerov, 
``Interpolation and range extrapolation of head related transfer functions”, 
{\em IEEE International Conference on Acoustics,
Speech, and Signal Processing (ICASSP)}, vol. 4, pp. iv–iv, 2004. 

\bibitem{SH1998}
M. J. Evans, J. A. Angus, and A. I. Tew, 
``Analyzing head-related transfer
function measurements using surface spherical harmonics", 
 {\em J. Acoust. Soc. Am.}, vol. 104, no. 4, pp. 2400–2411, 1998.


\bibitem{SH2012}
 M. Aussal, F. Alouges, and B. Katz, 
 ``HRTF interpolation and ITD personalization for binaural synthesis using 
 spherical harmonics",  {\em Journal of Audio Engineering Society}, 2012. 	  	

\bibitem{PCA2012}
B. Xie, 
``Recovery of individual head-related transfer functions from a small set of measurements", 
 {\em J. Acoust. Soc. Am.}, vol. 132, no. 1, pp. 282–294, 2012.


\bibitem{PCA2013}
 L. Wang, F. Yin, and Z. Chen, ``Head-related transfer function interpolation through multivariate 
 polynomial fitting of principal component weights”, 
 {\em Acoust. Sci. Tech.}, vol. 30, no. 6, pp. 395–403, 2009.
 
\bibitem{PCA2020}
M. Zhang, Z. Ge, T. Liu, X. Wu, and T. Qu, 
``Modeling of individual HRTFs based on spatial principal component analysis", 
{\em IEEE/ACM Trans. on Audio, Speech, and Lang. Process.,} vol. 28, 
pp. 785-797, 2020. 

\bibitem{Spline1999}
K. Hartung, J. Braasch, and S. J. Sterbing, 
``Comparison of different methods for the interpolation of 
head-related transfer functions”, 
{\em  Proc. 16th Int. Audio Eng. Soc. Conf. Spatial Sound Reproduction},  
pp. 319–329, 1999.



\bibitem{wavelet}
J. C. B. Torres, M. R. Petraglia, 
``HRTF interpolation in the wavelet transform domain”, 
{\em IEEE Workshop on Applications of Signal
Processing to Audio and Acoustics (WASPAA)},  pp. 293-296, 2009.



\bibitem{autoencoder}
Y. Ito, T. Nakamura, S. Koyama, and H. Saruwatari, 
``Head-related transfer function interpolation from spatially sparse measurements using autoencoder with source position conditioning", 
{\em 2022 International Workshop on Acoustic Signal Enhancement (IWAENC)}, pp. 1-5, 2022.

\bibitem{GAN}
P. Siripornpitak, I. Engel, I. Squires, S. J. Cooper, and L. Picinali, ``Spatial up-sampling of HRTF sets using generative adversarial networks: A pilot study", 
{\em Frontiers in Signal Processing}, 54.


\bibitem{film}
J. W. Lee, S. Lee, and K. Lee, 
``Global HRTF interpolation vi learned affine transformation of hyper-conditioned features", arXiv preprint arXiv:2204.02637.

 
\bibitem{hrtf_dir2}
W. Zhang, R. A. Kennedy, and T. D. Abhayapala, 
``Iterative extrapolation algorithm for data reconstruction over sphere”,
{\em IEEE International Conference on Acoustics, Speech, and Signal Processing (ICASSP)}, 
pp. 3733–3736, Mar. 2008.


\bibitem{hrtf_dir3}
U. Elahi, Z. Khalid, and R. A. Kennedy, 
``An improved iterative algorithm for band-limited signal extrapolation on the sphere",
{\em IEEE International Conference on Acoustics, Speech, and Signal Processing (ICASSP)},  
pp. 4619-4623, Mar. 2018. 

\bibitem{hrtf_dir1}
D. N. Zotkin, R. Duraiswami, N. Gumerov, et al., 
``Regularized HRTF fitting using spherical harmonics", 
{\em WASPAA’09. IEEE Workshop on}, pp. 256, 2009.

\bibitem{hrtf_dir4}
 J. Ahrens, M. R. P. Thomas, and I. Tashev, 
 ``HRTF magnitude modeling using a non-regularized least-squares
fit of spherical harmonics coefficients on incomplete
data”,  {\em APSIPA 2012}, Dec. 2012.


\bibitem{rafaely_2015}	
B. Rafaely, {\em Fundamentals of Spherical Array Processing}, (New York NY USA:Springer, 2015), vol. 8.


\bibitem{pinn1}	
M. Raissi, P. Perdikaris, and G. Em Karniadakis, 
``Physics informed deep learning (part I):  data-driven solutions of 
nonlinear partial differential equations", 2017, arXiv:1711.10566. 

\bibitem{pinn2}	
M. Raissi, P. Perdikaris, and G. Em Karniadakis, 
``Physics informed deep learning (part II): data-driven discovery of 
nonlinear partial differential equations", 2017, arXiv:1711.10566.

\bibitem{pinn3}	
G. E. Karniadakis, I. G. Kevrekidis, L. Lu, P. Perdikaris, S. Wang, and
L. Yang, 
``Physics-informed machine learning,” 
{\em Nature Reviews Physics}, vol. 3, no. 6, pp. 422–440, May 2021.


\bibitem{pinn4}	
S. Cuomo, V. D. Di Cola, F.  Giampaolo, G. Rozza, M. Raissi 
and   F. Piccialli, 
``Scientific machine learning through physics-informed neural networks: Where we are and what’s next”, 2022, arXiv:2201.05624. 


\bibitem{pinn_earth1}	
C. Song, T. Alkhalifah, and U. B. Waheed, 
``Solving the frequency-domain acoustic VTI wave equation using physics-informed neural networks", 
{\em Geophys. J. Int.}, vol. 225, no. 2, pp. 846-859, 2021.


\bibitem{pinn_earth2}	
P. Ren, C. Rao, H. Sun, and Y. Liu, 
``SeismicNet: physics-informed neural networks for seismic wave modeling 
in semi-infinite domain", arXiv preprint arXiv:2210.14044.


\bibitem{pinn_drone}	
Y. Wang, K. Wang, and M. Abdel-Maksoud, 
``NoiseNet: a neural network to predict marine propellers’ underwater 
radiated noise", Ocean Engineering, vol. 236, pp. 109542, 2021.


\bibitem{pinn_room}	
K. Shigemi, S. Koyama, T. Nakamura, and H. Saruwatari, 
``Physics-informed convolutional neural network with bicubic spline interpolation for sound field estimation", 
{\em 2022 International Workshop on Acoustic Signal Enhancement (IWAENC)}, 
pp. 1-5, 2022.


\bibitem{pinn_soundfield}	
K. Shigemi, S. Koyama, T. Nakamura, and H. Saruwatari, 
``Physics-informed convolutional neural network with bicubic 
spline interpolation for sound field estimation", 
{\em arXiv preprint}, arXiv:2207.10937.



\bibitem{hrtf_insight}	
W. Zhang, T. D. Abhayapala, R. A. Kennedy, and R. Duraiswami,
``Insights into head-related transfer function: spatial 
dimensionality and continuous representation”, 
{\em J. Acoust. Soc. Amer.}, vol. 127, pp. 2347--2357, 2010.

\bibitem{Thushara_kr}
\bibinfo{author}{D.~B. Ward}
and \bibinfo{author}{T.~D. Abhayapala},
``Reproduction of a plane-wave sound field using an array of	loudspeakers",
{\em IEEE Trans. Speech Audio Process.},
vol. 9, no. 66, pp.~\bibinfo{pages}{697--707}, \bibinfo{year}{2001}.

\bibitem{pinn_pp1}	
R. Leiteritz, D. Pflüger,
``How to avoid trivial solutions in physics-informed neural networks", 
arXiv preprint arXiv:2112.05620.


\bibitem{pinn_pp2}	
S. Wang, Y. Teng and P. Perdikaris, 
``Understanding and mitigating gradient pathologies in physics-informed neural networks", arXiv:2001.04536, 2020.

\bibitem{pinn_pp3}	
F. M.  Rohrhofer, S. Posch, C. Gößnitzer, and B. C. Geiger,
``Understanding the difficulty of training physics-informed neural networks on dynamical systems", arXiv preprint arXiv:2203.13648.










\bibitem{william}
E. G. Williams, {\em Fourier Acoustics: Sound Radiation and Nearfield Acoustical Holography}, (Academic Press, 1999).

\bibitem{3d3a}
R. Sridhar, J. G. Tylka, and E. Y. Choueiri, 
``A database of head-related transfer function and morphological measurements", 
{\em 143th Conv. AES}, Oct. 2017. 


\bibitem{nf48}
J. M. Arend, A. Neidhardt, and C. Pörschmann, 
``Measurement and perceptual evaluation of a spherical near-field HRTF set", 
29 th Tonmeistertagung, 2016.

\bibitem{multiple_target}
 O. Sener and V. Koltun, 
 ``Multi-task learning as multi-objective optimization", 
 {\em Proc. Adv. Neural Inf. Process. Syst.,} pp. 525–536, 2018. 


\bibitem{width}
 Z. Lu, H. Pu, F. Wang, Z. Hu, and L. Wang, 
 ``The expressive power of neural networks: a view from the width”, 
 {\em Proc. Adv. Neural Inf. Process. Syst.}, pp. 6231–6239, 2017. 


\bibitem{init}
 X. Glorot and Y. Bengio, 
 “Understanding the difficulty of training deep feedforward neural networks”, 
 {\em Proc. AISTATS}, pp. 249–256, 2010.

\bibitem{inter}
Y. Ito, T. Nakamura, S. Koyama, and H Saruwatari,
``Head-related transfer function interpolation from spatially sparse measurements using autoencoder with source position conditioning", 
{\em arXiv preprint}, arXiv:2207.10967.
%%%%%%%%%%%%%%%%%%%%%%%%%%%%%%%%%%%%%%%%%%%%%%%%%%%%%%%%%%%%%%%%%%%%%%%%%%%%%

\bibitem{chedar}
G. Slim, B. Xavier, and S. Renaud, 
``Computed hrirs and ears database for acoustic research", 
{\em 148th Conv. AES}, May 2020.


\bibitem{hutubs}
F. Brinkmann, M. Dinakaran, R. Pelzer, P. Grosche, D. Voss, and S. Weinzierl,
``A cross-evaluated database of measured and simulated HRTFs including 3D head meshes, anthropometric features, and headphone impulse responses",
{\em AES}, 2019.

\bibitem{ircam}
W. Olivier, http://recherche.ircam.fr/equipes/salles/listen/download.html

\end{thebibliography}
\end{document}
