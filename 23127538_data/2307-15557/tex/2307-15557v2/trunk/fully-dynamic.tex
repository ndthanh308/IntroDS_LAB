\section{Fully Dynamic $k$-Center on Graphs}\label{sec:fullydynamic}
In this section we describe how to maintain a $(2+\epsilon)$-approximate solution to the $k$-center problem on fully dynamic graphs.
We start by reviewing Gonzalez's algorithm, a classical $2$-approximation algorithm to the $k$-center problem in the static setting. Afterwards, we describe how to adapt it to the fully dynamic setting by using fully dynamic approximate SSSP algorithms.

\subsection{Gonzalez's Algorithm}
Gonzalez's algorithm~\cite{Gonzalez85} is a well-known greedy algorithm for the $k$-center problem on (possibly weighted and directed) graphs\footnote{The algorithm is also used for the $k$-center problem in metric spaces.}. 
It works as follows:
\begin{enumerate}[topsep=0pt,itemsep=-1ex,partopsep=1ex,parsep=1ex]
    \item pick as first center an arbitrary vertex $c_1 \in V$ and set $C = \{c_1\}$;
    \item while $|C| < k$, pick the next center $c_i \in \argmax_{v \in V}\, d_G(C, v)$ and set $C = C \cup \{c_i\}$; 
    \item return the set of centers $C$.
\end{enumerate}
\begin{theorem}
Gonzalez's algorithm computes a 2-approximation for the $k$-center problem on graphs and a standard implementation runs in time $O(k(m + n \log n))$.
\end{theorem}
\begin{Definition}[$\alpha$-approximate Gonzalez's algorithm]
For $\alpha \ge 1$, an $\alpha$-approximate Gonzalez's algorithm is a relaxation of Gonzalez's algorithm that picks the next center $c_i$ in step 2 above such that $d_G(C, c_i) \ge \alpha^{-1}\cdot \max_{v \in V} d_G(C,v)$.
\end{Definition}
\begin{theorem}[{\cite[Lemma~4.1]{AbboudCLM23}}]\label{thm:alpha gonzalez}
For $\alpha \ge 1$, an $\alpha$-approximate Gonzalez's algorithm computes a $2\alpha$-approximation for the $k$-center problem on graphs.
\end{theorem}


\subsection{Fully Dynamic $k$-Center via Fully Dynamic $(1+\epsilon)$-SSSP}

Assuming that we have a fully dynamic $(1+\epsilon)$-SSSP data structure,
we show how to use this to get a fully dynamic $k$-center data structure in Algorithm~\ref{alg:fullydyn_kcenter}.

\begin{algorithm}[ht!]
\caption{\textsc{Fully dynamic $2(1+\epsilon)$-approximation k-center}}
\label{alg:fullydyn_kcenter}
\DontPrintSemicolon
\setcounter{AlgoLine}{0}
\SetAlgoLined

\SetKwInOut{Input}{Input}
\SetKwInOut{Output}{Output}

% \Input{~Unweighted undirected graph $G=(V,E)$; integer $k\ge 1$}
% \Output{~Set of $k$ centers $C$}

\SetKwFunction{FSimulateGonzalez}{SimulateGonzalez}
\SetKwProg{Fn}{Function}{:}{\KwRet}
\Fn{\FSimulateGonzalez{$\mathcal{D}$, s, k}} {
    $C = \emptyset$
    
    \For{$i=1,...,k$}{
        $c_i \gets x \in \argmax_{v\in V} \delta_{G'}(s, v)$
        \tcp*{$c_1 \gets \text{arbitrary } v \in V$}
        
        \tcp*{$s$ is disconnected at $i=1$; $\delta_{G'}(s, v)=\infty,\, \forall v \in V$; ties broken arbitrarily}
        
        $\mathcal{D}.insert(s, c_i)$

        $C \gets C \cup \{c_i\}$
    }
    
    \For{$i=1,...,k$}{        
        $\mathcal{D}.delete(s, c_i)$
    }
    
    \KwRet $C$
}
\vspace{1em}
\SetKwFunction{FPreprocessing}{Preprocessing}
\Fn{\FPreprocessing{G, k}} {
    $G' \gets (V \cup \{s\},\, E)$ \tcp*{augment $G$ with super source $s$}
    
    $\mathcal{D} \gets \textsc{Initialize}(G', s)$ \tcp*{$\mathcal{D}$ is fully dynamic $(1+\epsilon)$-SSSP, with approx. distance $\delta_{G'}(s, v)$}
    
    $C \gets$ \FSimulateGonzalez{$\mathcal{D}$, s, k}
}
\vspace{1em}
\SetKwFunction{FUpdate}{Update}
\Fn{\FUpdate{$u$, $v$}} {
    $\mathcal{D}.update(u,v)$ \tcp*{either insert or delete edge $(u,v)$}
    
    $C \gets$ \FSimulateGonzalez{$\mathcal{D}$, s, k}
}
\end{algorithm}

\begin{theorem}\label{thm: fully dynamic black box}
Given a graph $G=(V,E)$, a positive parameter $\epsilon\le 1/2$, and a fully dynamic data structure that maintains $(1+\epsilon)$-approximate distances from a single source $s \in V$ with worst-case update time $T(n, m, \epsilon)$, Algorithm~\ref{alg:fullydyn_kcenter} maintains a $2(1+4\epsilon)$-approximate solution to fully dynamic $k$-center in time $O(k \cdot ( T(n,m, \epsilon)+n))$. 
\end{theorem}
\begin{proof}
We prove that the procedure \textsc{SimulateGonzalez} in Algorithm~\ref{alg:fullydyn_kcenter} is a $(1+4\epsilon)$-approximate Gonzalez's algorithm, hence the claim about the approximation follows by Theorem~\ref{thm:alpha gonzalez}.

Note that the procedure runs on $G'$, which is a copy of $G$ with an additional super-source vertex $s$ which is initially disconnected. 
Let us call $\mathcal{D}$ the data structure used to maintain the $(1+\epsilon)$-approximate distances from $s$ in $G'$, e.g., the one given in Theorem~\ref{thm:fd_distances} or in Theorem~\ref{thm:weighted_fd_distances}.
Suppose to be at the $i$-th iteration of the procedure, i.e., the super-source $s$ is connected to all vertices in $C = \{c_1,...,c_i\}$ in $G'$.
Note that such additional edges imply that $d_{G'}(s, v) = 1+ d_G(C, v)$, for every $v\in V$.
Let $\delta_{G'}(s,v)$ be the approximate distance between $s$ and $v$ maintained by $\mathcal{D}$, which guarantees that $d_{G'}(s,v) \le \delta_{G'}(s,v) \le (1+\epsilon)d_{G'}(s,v)$.
Let $v_{\max} \in \argmax_{v \in V}\, d_G(C, v)$ be one among the furthest vertices from $C$.
Let $c_{i+1}$ be the next center selected by the algorithm, i.e., $c_{i+1} \in \argmax_{v \in V} \delta_{G'}(s, v)$.
Therefore, it holds that
\begin{align*}
    1+ d_G(C, v_{\max}) 
    = d_{G'}(s, v_{\max}) 
    \le \delta_{G'}(s, v_{\max})
    &\le \delta_{G'}(s, c_{i+1}) 
    \\
    &\le (1+\epsilon)d_{G'}(C, c_{i+1})
    = (1+\epsilon)(1+d_G(C, c_{i+1})).
\end{align*}
Noting that $d_G(C, v_{\max}) \ge 1$ and since by assumption $\epsilon\in(0,1/2]$, the previous equation implies
\begin{multline*}
    d_G(C, c_{i+1}) 
    \ge \frac{1+d_G(C, v_{\max})}{1+\epsilon} - 1
    = \frac{d_G(C, v_{\max})-\epsilon}{1+\epsilon} 
    = \frac{1-\epsilon/d_G(C, v_{\max})}{1+\epsilon} d_G(C, v_{\max})
    \\
    \ge \frac{1-\epsilon}{1+\epsilon} d_G(C, v_{\max}) 
    \ge \frac{1}{1+4\epsilon} \max_{v \in V} d_G(C, v),
\end{multline*}
which concludes the approximation proof.

\bigskip
The update procedure requires that $\mathcal{D}$ is updated $2k+1$ times, with a worst-case time of $T(n,m, \epsilon)$ per update, and additionally look for the approximate furthest neighbor $k$ times, each requiring time $O(n)$, i.e., querying the approximate distance $\delta_G(s,v)$, $\forall v\in V$.
\end{proof}


In particular, for the fully dynamic data structure we use the state-of-the-art algorithm for unweighted graphs by \cite{BFN22}.
\begin{theorem}[\cite{BFN22}]\label{thm:fd_distances}
Given an unweighted undirected graph $G=(V,E)$ and a single source $s$, and $0 <\epsilon <1$, there is a deterministic fully dynamic data structure for maintaining $(1+\epsilon)$-distances from $s$ with worst-case update time of $O(n^{1.529} \epsilon^{-2})$ for the current matrix multiplication exponent $\omega$.
The algorithm has preprocessing time of $O(n^{\omega}\epsilon^{-2}\log \epsilon^{-1})$, where $\omega \le 2.373$.
\end{theorem}

For weighted graphs the state-of-the-art algorithm is slower and it is given by \cite{BrandN19}.
\begin{theorem}[\cite{BrandN19}]\label{thm:weighted_fd_distances}
Given a weighted and directed graph $G=(V,E)$, a single source $s \in V$, and a positive parameter $\epsilon <1$, there is a randomized fully dynamic algorithm working against an adaptive adversary that maintains $(1+\epsilon)$-distances from $s$ with worst-case update time of $O(n^{1.823} \epsilon^{-2})$ for the current matrix multiplication exponent $\omega$.
The algorithm has preprocessing time of $O(n^{\omega}\epsilon^{-2}\log \epsilon^{-1})$, where $\omega \le 2.373$.
\end{theorem}


A combination of Theorem~\ref{thm: fully dynamic black box} with Theorems~\ref{thm:fd_distances} and \ref{thm:weighted_fd_distances} gives the following result.
\fullydynamickcenter*{}
