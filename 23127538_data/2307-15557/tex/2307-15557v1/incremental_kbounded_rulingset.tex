\section{Incremental $k$-center on graphs}\label{sec:incremental}
We start by introducing the concept of \textit{kernel-sets}, which we will exploit throughout this section.

\begin{definition}[$(\alpha, \beta)$-kernel-set]\label{def:kern_set}
    Given a graph $G = (V, E)$, an $(\alpha, \beta)$-kernel-set $S \subseteq V$ in $G$ is a subset of vertices with
    the property that any \kbabrs in $G[S]$ is a $k$-bounded $(\alpha, \beta+1)$-ruling set in $G$.
\end{definition}
\noindent
When it is clear from the context we use the term kernel-set for a $(2, 1)$-kernel-set, unless stated otherwise.

In this section, we first develop an incremental algorithm for the \kbttrs problem by finding a small kernel-set $S$ and maintaining a \kbtors in $G[S]$. The idea is to use the reduction of Lemma~\ref{lem:2b-approx_Gr} with this algorithm, to solve the incremental $k$-center problem. 
In the reduction though, notice that we need to maintain an incremental \kbttrs algorithm on $r$-threshold graphs, which is more challenging. 

Then, we relax our requirements and develop an efficient incremental \kbttrs algorithm that works on approximate versions of threshold graphs.
Thus, we apply Lemma~\ref{lem:2b_Gr_H} instead of Lemma~\ref{lem:2b-approx_Gr}, at the cost of an extra $(1+\epsilon)$ factor in the approximation ratio of $k$-center.

\subsection{Incremental \kbttrs algorithm} \label{sec:2_k_MIS}
We describe how to detect a small kernel-set $S$ and maintain a $k$-bounded $(2,1)$-ruling set on the subgraph induced by $S$.
\begin{restatable}{theorem}{reskbmis}\label{th:res_kbmis}
    Given a graph $G = (V, E)$ subject to edge insertions, and an integer $k \geq 1$, there is a randomized incremental algorithm which
    \begin{itemize}[topsep=0pt,itemsep=-1ex,partopsep=1ex,parsep=1ex]
        \item either reports that there is an independent set in $G$ of size at least $k + 1$, and this is correct w.h.p.,
        \item or finds a \emph{kernel-set} $S$ of size $\tilde{O}(k)$ in $G$ and maintains a \kbtors in $G[S]$.
    \end{itemize}
\end{restatable}

Notice that based on the definitions of kernel-set and \kbabrs problem (i.e., Definition~\ref{def:kern_set} and
Definition~\ref{def:kbabrs}), the algorithm of Theorem~\ref{th:res_kbmis} solves
the incremental \kbttrs problem in $G$.
 Before describing the algorithm we review two existing algorithms tools.
First tool is the following folklore hitting set claim (e.g., see \cite{AingworthCIM99}, also widely used in decremental settings against an oblivious adversary).
\begin{lemma} \label{lem:sampling}
    Given a graph $G = (V, E)$ and a threshold $\gamma \ge 1$, let $S$ be the set obtained by sampling each vertex independently with probability $\min(c\ln(n) / \gamma, 1)$, for a constant $c>1$. Then, with probability at least $1 - n^{-(c-1)}$, every vertex of degree more than $\gamma$ has at least one neighbor in $S$.
\end{lemma}
\noindent
As noted, e.g., in~\cite{rodittyzwick2012dynamic}, even though Lemma~\ref{lem:sampling} refers to a static graph, it is easy to see that it holds for partially dynamic graphs. Since we are assuming an oblivious adversary, the choice of the random set $S$ is independent of the graph. This and the fact that we have at most $ O(n^2)$ versions of the graph in the incremental setting, let us bound the overall probability via a straightforward union bound, and the failure probability is at most $n^{-(c-3)}$.

Second tool, is a fully dynamic \kbtors algorithm
with the following guarantees. This
algorithm is a trivial extension of any fully dynamic MIS algorithm that returns explicitly
the MIS. For that reason, we can either use the MIS algorithm of Behnezhad et al.~\cite{behnezhad2019fully},
or the algorithm of Chechik and Zhang \cite{chechik2019fully}.
 
\begin{theorem} \label{th:fully_dyn_mis}
    Given a graph $G = (V, E)$ subject to edge updates, there is a fully dynamic 
    \kbtors algorithm with $\tilde{O}(1)$ amortized update time.
\end{theorem}
\begin{proof}
    The algorithm of Behnezhad et al.~\cite{behnezhad2019fully} maintains
    an MIS $M$ under edge updates, in $\tilde{O}(1)$ amortized update time. Recall that an MIS is a \tors.
    Thus at any moment, if the size of $M$ is at least $k + 1$, we report
    that there is an independent set in $G$ of size at least $k + 1$, otherwise we return the set $M$.
\end{proof}

\paragraph{Overview of the algorithm.}
A pseudocode of the algorithm of Theorem \ref{th:res_kbmis} is provided in Algorithm~\ref{alg:2_k_MIS}.
The algorithm consists of two phases. 
Roughly speaking, the first phase either detects a kernel-set $S$ or reports that there is an independent set in $G$ of size at least $k + 1$. 
The second phase starts when such a kernel-set $S$ is detected and is only responsible for maintaining an incremental \kbtors in $G[S]$.

\begin{algorithm}[ht!]
\DontPrintSemicolon
\caption{\textsc{\kbttrs}}
\label{alg:2_k_MIS}

\setcounter{AlgoLine}{0}
\SetAlgoLined

\SetKwProg{Fn}{Procedure}{:}{\KwRet}
\SetKwFunction{FSample}{SampleSL}
\SetKwFunction{FKBRS}{$k$-Bounded-Ruling-Set}
\SetKwFunction{FUpdate}{Insert}

\tcp{In the preprocessing $i = 0$, $L_0 = V$, and $\FKBRS{}$ is called with no edge (note that Line~16 where the edge is actually used cannot be reached during the preprocessing)}
\tcp{The index $i$ and the sets $L_i,S_i$ for every $i$ are global}

\vspace{1em}

\Fn{\FKBRS{$u,v$}} {
    \If(\tcp*[h]{first phase}){$|L_i| > 4k$} {
        \If(\tcp*[h]{recursive sampling}){$i= 0$ or $|L_i|\le \frac{|L_{i-1}|}{2}$}{
            $i \gets i+1$
        
            $\gamma_i \gets \frac{|L_{i-1}|}{2k} - 1$
        
            $S_i \gets$ sample vertices of $L_{i-1}$ independently with prob.~$\min(10\ln(n) / \gamma_i, 1)$
            \tcp*{Lemma~\ref{lem:sampling}}
            
            $L_i \gets \{x \in L_{i-1} : N_{G[L_{i-1}]}(x) \cap S_i = \emptyset\}$
    
            \FKBRS{u,v}
        }
        \Else{
            \textbf{report} there is an independent set in $G$ of size at least $k+1$
            \tcp*{Lemma~\ref{lem:kvert_nomis}}
        }
    }
    \Else(\tcp*[h]{second phase}) {
        
        \If(\tcp*[h]{$\mathcal{B}$ is dynamic \kbtors algorithm (Theorem~\ref{th:fully_dyn_mis})}){$\mathcal{B}$ is not initialized}{
            $d \gets i$
            
            $S \gets \bigcup_{j=1}^d S_j \cup L_d$
            \tcp*{$S$ is a kernel-set (Lemma~\ref{lem:S_kern_G})}

            $\mathcal{B}.initialize(G[S])$
        }
        \ElseIf{$u\in S$ and $v \in S$}{
            $\mathcal{B}.update(G[S], u,v)$
        }
    }
}

\vspace{1em}

\Fn{\FUpdate{$u, v$}} {
    $G \gets (V,\, E \cup \{u,v\})$
    
    \If {$u \in L_i$ and $\exists j \leq i$ s.t. $v \in S_j\,$ (resp., $v \in L_i$ and $\exists j \leq i$ s.t. $u \in S_j$)} {
        $L_i \gets L_i \setminus \{u\}\,$ (resp., $L_i \gets L_i \setminus \{v\}$)
    }
    \FKBRS{$u,v$}
}
\end{algorithm}

\smallskip
In the first phase, the algorithm iteratively adds vertices to the kernel-set by recursively sampling a sequence of hitting sets. 
In each recursive call $i \geq 1$, we set a threshold $\gamma_i = \tfrac{|L_{i-1}|}{2k}-1$ and construct two sets $S_i$ and $L_i$.
The set $S_i$ is obtained by sampling each vertex of $L_{i-1}$ independently with probability $\min(c\ln(n) / \gamma_i, 1)$, for a sufficiently large constant $c$.
Roughly speaking the set $S_i$ is the hitting set of the vertices with degree more than $\gamma_i$.
Moreover, the set $S_i$ is w.h.p.\ small in size due to the sampling procedure.
The set $L_i$ is constructed as the subset of vertices of $L_{i-1}$ that do not belong to $S_i$ and do not have a neighbor in $S_i$. 
Given the property of the hitting set $S_i$, the set $L_i$ contains w.h.p.\ only vertices with degree at most $\gamma_i$.
The recursion starts with $L_0 = V$ and it ends when $|L_i| \le 4k$. 

In the $i_{\text{th}}$ recursive call, if the size of $L_i$ is at most $|L_{i-1}|/2$ then a new recursive call begins. This implies that the depth of the recursion over all updates 
is bounded by $O(\log n)$.
On the other hand, if the size of the set $L_i$ is greater than $|L_{i-1}|/2$, the recursion pauses and the algorithm reports that there is an independent set in $G$ of size at least $k+1$. 
In this case $i$ may not be the final recursive call of the algorithm, because on future updates the algorithm can possibly continue the recursion.

Whenever an edge $(u,v)$ is inserted to $G$ during the first phase, we update the set $L_i$ by removing from it one of the endpoints if the other one is contained in $S_i$. Observe that edge insertions will eventually shrink the size of $L_i$, forcing the recursion 
to continue.

\smallskip
The second phase begins when the size of $L_i$ is at most $4k$, and at this moment the recursion ends.
We denote by $d$ the index of the last recursive call in the first phase, and let $S := \bigcup_{j=1}^{d} S_j \cup L_d$ be the union of the hitting sets of all recursive calls and of the set $L_d$. Notice that $S$ can be constructed explicitly, during the first phase of the algorithm.
Also, in the updates following the second phase we never re-enter the first phase, and thus the set $S$ is not modified anymore.
We show in the analysis, that even though the set $S$ is random, it is always a kernel-set in $G$.

At the beginning of the second phase, the dynamic \kbtors algorithm $\mathcal{B}$ of Theorem~\ref{th:fully_dyn_mis} is initialized on $G[S]$.
Whenever an edge $(u, v)$ is inserted to $G$ during
the second phase, the algorithm simply forwards the update to $\mathcal{B}$ if $u,v \in S$, and does nothing otherwise.


\paragraph{Proof of Theorem~\ref{th:res_kbmis}.}
The analysis consists of three claims.
First, we prove that whenever the algorithm reports that there is an independent set in $G$ of size at 
least $k + 1$, this is correct with high probability.
Second, we show that there are $O(\log n)$ recursive calls and that the size of $S$ is  $\tilde O(k)$.
Third, we prove that the set $S$ detected by the algorithm is indeed a kernel-set in $G$.


\begin{lemma} \label{lem:kvert_nomis}
At any stage of the algorithm with $i \geq 1$, if $|L_i| > \frac{|L_{i-1}|}{2}$, then w.h.p.~there is an independent set in $G$ of size at least $k + 1$.
\end{lemma}
\begin{proof}
The threshold $\gamma_i$ is set to $\frac{|L_{i-1}|}{2k} - 1$,
and $S_i$ is obtained by sampling each vertex of $L_{i-1}$ independently with probability 
$\min(c\ln(n) / \gamma_i, 1)$, for a sufficiently large constant $c$.
Then by Lemma~\ref{lem:sampling}, it holds that w.h.p.~every vertex 
in $L_{i-1}$ of degree more than $\gamma_i$ in the induced subgraph $G[L_{i-1}]$ has a neighbor in $S_i$. 
Hence, w.h.p.~every vertex in $L_i$ is of degree at most $\gamma_i$ in $G[L_{i-1}]$.
As $G[L_i]$ is a subgraph of $G[L_{i-1}]$, w.h.p.~every vertex of $G[L_i]$ is of degree at most $\gamma_i$ in $G[L_i]$ as well.

Since w.h.p.~the maximum degree in $G[L_i]$ is bounded by $\gamma_i$, for any $T \subseteq L_i$ such that
$|T|= k$ (note that $|L_i|>4k$), it holds that w.h.p.~the number of vertices which are either in $T$ or have a neighbor in $T$ is
at most $k (\gamma_i+1) \le \frac{|L_{i-1}|}{2}$. 
By assumption we have that $|L_i| > \frac{|L_{i-1}|}{2}$,
and so $T$ cannot be a maximal independent set.
So it holds that w.h.p.~there is an independent set in $G[L_i]$ of size
at least $k + 1$. In turn, as $G[L_i]$ is an induced subgraph
of $G$, it holds that w.h.p.~there is an independent set in $G$ of size
at least $k + 1$ as well.
\end{proof}

\begin{lemma}\label{lem:dep_S_size}
    Over the sequence of updates, there are $d=O(\log n)$ recursive calls. Moreover, the size of $S$ is w.h.p.~$O(k \log^2 n)$.
\end{lemma}
\begin{proof}
    Regarding the first claim, at every recursive call $i \ge 1$, it holds that $|L_i| \le \frac{|L_{i-1}|}{2}$. Initially we have that $|L_0|=n$, and so, the depth of the recursion is $d=O(\log n)$.
    
    Regarding the second claim, at each recursive call $i \geq 1$, we sample each vertex of $L_{i-1}$ independently with probability $\min(c\ln(n) / \gamma_i, 1)$, for a sufficiently large constant $c$. 
    Recall that $\gamma_i = \frac{|L_{i-1}|}{2k} - 1$ and note that the sampling takes place only if $|L_{i-1}| > 4k$.
    Then
    \[
        \mathbb{E}[|S_i|] 
        \le |L_{i-1}| \cdot \frac{c \ln(n)}{\gamma_i}
        = |L_{i-1}| \cdot \frac{c \ln(n)}{\frac{|L_{i-1}|}{2k}-1}
        = |L_{i-1}| \cdot \frac{2k \cdot c\ln(n)}{|L_{i-1}| - 2k}
        = \frac{2k \cdot c\ln(n)}{1 - 2k/|L_{i-1}|}
        < 4k \cdot c \ln(n). 
    \]
    Moreover note that $|L_d|\le 4k$.
    Therefore, by linearity of expectation it holds that $\mathbb{E}[|S|] = |L_d| + \sum_{i=1}^d \mathbb{E}[|S_i|] = O(k \log^2 n)$.
    Finally, since $|S|$ is a sum of independent Poisson trials, a standard application of a Chernoff's bound implies that $|S|=O(k \log^2 n)$ with high probability.
\end{proof}
 
\begin{lemma}\label{lem:S_kern_G}
    The set $S$ is a kernel-set in $G$.
\end{lemma}
\begin{proof}
    For a fixed vertex $v \in V \setminus S$, let $i$ be the minimum index such that $v \notin L_i$. 
    Note that such an index exists since $v \in V=L_0$ and so $i\ge 1$.
    If $v \notin L_i$, then by definition of $L_i$, vertex $v$ must have a neighbor in $S_i$.
    Therefore, every vertex $v \in V \setminus S$ has a neighbor in $S$.
    
    Let $\mathcal{A}$ be a \kbabrs in $G[S]$.
    If $\mathcal{A}$ reports that there is a distance-$(\alpha-1)$ independent set in $G[S]$ of size at least 
    $k + 1$, then $\mathcal{A}$ correctly reports that there is a distance-$(\alpha-1)$
    independent set in $G$ as well. This is because $G[S]$ is an induced subgraph of $G$, and so, any
    distance-$(\alpha-1)$ independent set in $G[S]$ is also a
    distance-$(\alpha-1)$ independent set in $G$.

    Otherwise, $\mathcal{A}$ returns an \abrs $M$ of size at most $k$. 
    Then for any vertex $v \in S$, we have that $v$ is of distance
    at most $\beta$ from its closest vertex in $M$. Therefore, since every vertex $v \in V \setminus S$
    has at least one neighbor in $S$, we have that every vertex of $G$ is of distance at most $\beta + 1$ from its closest vertex in $M$. Thus, the set $M$ is an $(\alpha, \beta+1)$-ruling set in $G$ of size at most $k$, and so the claim follows.
\end{proof}