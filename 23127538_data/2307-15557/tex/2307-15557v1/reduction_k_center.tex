\section{Reduction from $2$-approximate $k$-center to $k$-bounded ruling set} \label{sec:reduction}
It is well-know that the $k$-center problem can be reduced to the problem of finding an MIS in a graph. This reduction was first given by Hochbaum and Schmoys~\cite{HochbaumS86}, 
in order to get a $2$-approximation algorithm, and also it has been used by \cite{chan2018fully, bateni2023optimal} 
for the fully dynamic $k$-center problem in metric spaces.
In particular, it is sufficient to solve a weaker version of the MIS problem,
where we only need to return an MIS of size at most $k$, or report that there is an independent set of size at least $k + 1$.
Formally, we define this problem based on an $(\alpha, \beta)$-ruling-set, as follows.
A similar definition was also given in \cite{bateni2023optimal} for the MIS. Recall that an MIS is a $(2, 1)$-ruling set. 

\begin{definition}[$k$-bounded $(\alpha, \beta)$-ruling set problem]\label{def:kbabrs}
    Given an unweighted undirected graph $G = (V, E)$, an integer $k \geq 1$, and parameters $\alpha, \beta$ such that $\beta \geq \alpha - 1 \geq 0$, the \kbabrs problem asks to either return an \abrs of size at most $k$, or to report that there is a distance-$(\alpha-1)$ independent set of size at least $k + 1$.
\end{definition}

The reduction solves the \kbabrs problem on the following type of graphs.

\begin{definition}[$r$-threshold graph]\label{def:threshold_graph}
Given a weighted graph $G = (V, E, w)$ and a parameter $r > 0$, 
the \emph{$r$-threshold graph} $G_r = (V, E_r)$ is defined as the graph with vertex set $V$ and edge set $E_r = \{(u,v) \in V \times V : d_G(u,v) \leq r \}$.    
\end{definition}
\noindent
In other words, the $r$-threshold graph $G_r$ connects all pairs of vertices that are within distance $r$ in $G$. Observe that the $r$-threshold graph $G_r$ is unweighted.

The next lemma is an adjustment of the reduction of Hochbaum and Schmoys~\cite{HochbaumS86} to Definition~\ref{def:kbabrs}.
The proof is deferred to Appendix~\ref{apx:reduction proofs}.

\begin{restatable}{lemma}{tbapproxGr}\label{lem:2b-approx_Gr}
    Consider a $k$-center instance $\left(G = (V, E, w), k\right)$, and a positive constant parameter $\epsilon$. 
    Then by running a \kbtbrs algorithm on the $r$-threshold graph $G_r$, for each $r \in \{(1 + \epsilon)^i \mid (1 + \epsilon)^i \leq nW, i \in \mathbb{N}\}$,
    we can find a $2\beta(1 + \epsilon)$-approximate solution for the $k$-center instance.
\end{restatable}

For the sake of efficiency, in the dynamic setting we do not handle $r$-threshold graphs,
but rather an approximation of them. For this reason, we generalize the previous lemma as follows.
% 
\begin{lemma}\label{lem:2b_Gr_H}
    Consider a $k$-center instance $\left(G = (V, E, w), k\right)$, constant positive parameters $\epsilon,\epsilon'$, $r \geq 0$, and $\beta\ge 1$. 
    Let $r' := (1+\epsilon') r$ and consider the threshold graphs $G_r$ and $G_{r'}$.
    Suppose that there is an algorithm $\mathcal{A}$ such that, given $G, r, \epsilon',\beta$, 
    \begin{itemize}[topsep=0pt,itemsep=-1ex,partopsep=1ex,parsep=1ex]
        \item either reports that there is an independent set in $G_r$ of size at least $k + 1$, 
        \item or runs a \kbtbrs algorithm $\mathcal{B}$ on an edge-subgraph $H$ of $G_{r'}$ with the following condition: whenever $\mathcal{B}$ reports that there is an independent set 
        in $H$ of size at least $k + 1$, then there is an independent set in $G_r$ of size at 
        least $k + 1$.
    \end{itemize}
    Then, by running $\mathcal{A}$ with input $G, r, \epsilon',\beta$, for each $r \in \{(1 + \epsilon)^i \mid (1 + \epsilon)^i \leq nW, i \in \mathbb{N}\}$, we can find a $2\beta(1 + \epsilon)(1 + \epsilon')$-approximate solution for the $k$-center instance.
\end{lemma}
\noindent
As already stated, the previous lemma is a generalization of Lemma~\ref{lem:2b-approx_Gr}. 
In fact, observe that in the definition of a \kbtbrs problem, we are allowed to report that there is an independent set of size at least $k + 1$. 
Thus by setting $H = G_r$ and $\epsilon' = 0$ in Lemma~\ref{lem:2b_Gr_H}, we get Lemma~\ref{lem:2b-approx_Gr} as a corollary.

Before proving Lemma~\ref{lem:2b_Gr_H}, we state two auxiliary results that will be useful. Their proofs are deferred to Appendix~\ref{apx:reduction proofs}.

\begin{restatable}{lemma}{kmisrtR}\label{lem:k-MIS_r>=2R^*}
    Consider a $k$-center instance $\left(G = (V, E, w), k\right)$ with optimal radius $R^*$. 
    Then for each $r \geq 2R^*$ and for every $\beta \geq 1$, it holds that every \tbrs in the $r$-threshold graph $G_r$ is of size at most $k$.
\end{restatable} 

\begin{restatable}{observation}{noIStRGr}\label{obs:noIStRGr}
Consider a $k$-center instance $\left(G = (V, E, w), k\right)$ with optimal radius $R^*$, and let
$G_r$ be the $r$-threshold graph where $r=2R^*$.
Then, there is no independent set in $G_r$ of size at least $k + 1$.
\end{restatable}

We proceed now with the proof of Lemma~\ref{lem:2b_Gr_H}.

\begin{proof}[Proof of Lemma~\ref{lem:2b_Gr_H}]
    Let $\hat{r}$ be the smallest $r \in \{(1 + \epsilon)^i \mid (1 + \epsilon)^i \leq nW, i \in \mathbb{N}\}$ such that algorithm $\mathcal{A}$ returns a \tbrs $M_r$ of size at most $k$ in an edge-subgraph $H$ of $G_{r'}$, where $r' = (1+\epsilon')r$. 
    Also, let $\hat{r}' = (1+\epsilon')\hat{r}$ and let $S := M_{\hat{r}}$ be the solution we return for the $k$-center instance.

    Since $H$ is a subgraph of $G_{\hat{r}'}$, 
    then for every edge $(u, v) \in E(H)$, the distance between $u$ and $v$ in $G$ is at most $\hat{r}'$. Hence, as $M_{\hat{r}}$ is a \tbrs in $H$, then every vertex is within distance 
    $\beta \hat{r}'$ from its closest center in $G$. Thus, the returned solution $S$ has radius
    at most $\beta  \hat{r}'$. 
    
    We show now that $\hat{r}$ 
    is at most $2(1+\epsilon)$ times larger than $R^*$.
    Based on Observation~\ref{obs:noIStRGr}, for the fixed choice of $r = 2R^*$,
    algorithm $\mathcal{A}$ always returns a \tbrs $M_r$ in $H$ of size at most $k$.
    By definition of $\hat{r}$, and since the possible values of $r$ are powers of $(1 + \epsilon)$,
    we have that $\hat{r} \leq 2(1 + \epsilon)R^*$.
    Therefore, the radius of the returned solution $S$ is at most $2\beta(1+\epsilon)
    (1+\epsilon')R^*$.
\end{proof}