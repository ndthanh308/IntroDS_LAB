\section{Introduction} \label{sec:intro}
Clustering is a key concept in data analysis that involves organizing `similar' data into groups. One of the most fundamental and well-studied objectives is the \emph{$k$-center} problem. Given a metric space with $n$ points and a positive integer $k \leq n$, the goal is to select $k$ points, referred to as \emph{centers}, such that the maximum distance of any point in the metric space to its closest center is minimized. 
It is known that any $(2-\epsilon)$-approximation for this problem is NP-hard for any $\epsilon > 0$~\cite{HsuN79}.
Due to its popularity, $k$-center has been considered under several algorithmic frameworks, including approximation algorithms~\cite{HsuN79,Gonzalez85,HochbaumS86,plesnik1980computational,FederG88}, parameterized complexity~\cite{Feldmann15,BandyapadhyayFM22}, massive parallel computation (MPC) model~\cite{CeccarelloPP19,BeraDGK22,BergBM23}, and beyond worst-case analysis~\cite{BalcanHW20}, among others. This problem also serves as a testbed for developing fundamental algorithmic definitions and paradigms, which are then often applied to solving other variants of clustering objectives.

Clustering in the \emph{dynamic} setting has received increasing attention in recent years. 
This line of work was initiated by Charikar et al.~\cite{DBLP:conf/stoc/CharikarCFM97} who considered the problem of minimizing cluster diameters under the insertions of new points in an underlying metric space.
Under both point insertions and deletions, the $k$-center objective was considered by Chan et al.~\cite{chan2018fully}, who achieved a $(2+\epsilon)$-approximation in $O(k^2 \epsilon^{-1} \log{\Delta})$ update time, where $\Delta$ is the aspect ratio of the metric space. Their running time was later improved to $O(k \epsilon^{-1} \textrm{poly} \log (n, \Delta))$ which is almost optimal considering that any $2$-approximation to metric $k$-center in the static setting must take $\Omega(nk)$ time~\cite{BateniEFHJMW23}.
These results spurred several follow-up works that studied other metric-based clustering objectives such as $k$-means~\cite{Cohen-AddadHPSS19, HenzingerK20}, $k$-median~\cite{GuoKLX21}, facility location~\cite{GoranciHL18, GuoKLX20, GoranciHLSS21, BhattacharyaLP22}, and sum-of-radii~\cite{HenzingerLM20} in the dynamic setting. 

An important case of $k$-center clustering is when the input metric is induced by a graph $G$ on $n$ vertices and $m$ edges. Naturally, any generic metric $k$-center algorithm can be applied on top of the graphical metric obtained by computing all-pairs shortest paths in $G$. However, the latter leads to slow running times, especially since it could make a sparse graph $G$ very dense. Thorup~\cite{Thorup04} gave a faster algorithm for $k$-center in the graph setting achieving $(2+\epsilon)$-approximation in $\tilde{O}(m\epsilon^{-1})$ time. This result was recently revisited by the work of Abboud et al.~\cite{AbboudCLM23} who gave a refined and simpler algorithm for $k$-center on graphs.

We note that graph clustering has also received attention in the machine learning community, albeit for the closely related objective of $k$-means~\cite{RattiganMJ07}. 
They observe the computational challenges involving graphs (see also~\cite[Section~2.3]{Aggarwal2010}) and specifically the output sensitivity due to distance changes caused by edge updates. 

Motivated by these developments, we study the fundamental problem of \emph{dynamic} $k$-center on graphs. In comparison to the dynamic metric-based counterpart, we remark that dynamic graphs are more challenging since (i) there is no guarantee of having oracle access to all-pair shortest path distances, and (ii) a single edge update may have a \emph{global} effect on the underlying graph metric, forcing a large number of vertex pairs to change their shortest path distance. This is also why we cannot use other black-box approaches such as using a distance oracle of the metric completion of the graph.

Hence we ask the natural question of to what extent one can leverage the structure of graphical $k$-center in the context of obtaining faster algorithms for dynamic $k$-center on graphs:

\begin{center}
    \emph{Are there efficient algorithms for $k$-center on graphs undergoing edge updates?}
\end{center}

\subsection{Our Contribution} 

In this paper, we answer the question in the affirmative. 
Our first contribution is a fully dynamic $k$-center algorithm that follows from prior work using a surprisingly simple trick.
\begin{restatable}{theorem}{fullydynamickcenter}\label{thm:fully-dynamic approx}
Given a $n$-vertex graph $G=(V,E)$, a parameter $1 \le k \le n$, and a positive constant parameter $\epsilon\le 1/2$, there are two fully dynamic $k$-center algorithms that maintain a $(2+\epsilon)$-approximation with the following guarantees (based on the current value of the matrix multiplication exponent):
\begin{enumerate}[topsep=0pt,itemsep=-1ex,partopsep=1ex,parsep=1ex]
    \item Deterministic algorithm with $O(kn^{1.529}\epsilon^{-2})$ worst-case update time, if $G$ is unweighted, 
    \item Randomized algorithm, against an adaptive adversary, with $O(kn^{1.823} \epsilon^{-2})$ worst-case update time, if $G$ is weighted.
\end{enumerate}
Both algorithms have preprocessing time $O(n^{2.373} \epsilon^{-2} \log \epsilon^{-1})$.
\end{restatable}


Note that our update time bounds match those of the state-of-the-art fully-dynamic single-source distance approximation algorithms with multiplicative error $(1+\epsilon)$~\cite{BrandN19,BFN22} up to an $ \tilde O (k) $ factor.
Matching this bound is a natural goal for dynamic $k$-center algorithms maintaining the $(1 + \epsilon)$-approximate distance of each vertex to its closest center, because such distance approximations are sufficient to return a $(2 + \epsilon)$-approximation for graph diameter when $ k = 1 $ and the fastest known approach for this is to use a dynamic single-source distance approximation algorithm.
Our algorithms---and to the best of our knowledge all related dynamic $k$-center algorithms on general metrics---do have this desirable property of maintaining the $(1 + \epsilon)$-approximate distance of each vertex to its closest center.

The above suggests that in order to achieve faster running times, we need to consider \emph{partially} dynamic algorithms for $k$-center clustering, where edge updates are restricted to only edge insertions or deletions. In particular, the insertions-only algorithms (also known as the \emph{incremental} setting) in the context of clustering are particularly well-motivated from a practical viewpoint, e.g., real-world graphs such as co-authorship networks are incremental since the fact that two scientists co-authoring a research paper (almost) never changes over time. Our main result is the following. 

\begin{restatable}{theorem}{incrkcentfappr}\label{th:incr_kcent_4appr}
    Given a weighted undirected graph $G = (V, E, w)$ subject to edge insertions, an integer parameter $k \geq 1$, and a positive constant parameter $\epsilon<1$, there is
    a randomized incremental $(4 + \epsilon)$-approximation algorithm
    for the $k$-center problem on graphs.
    The algorithm is correct w.h.p.\ and
    has $\tilde{O}(n^{o(1)}k)$ amortized update time over a sequence of $\Theta(m)$ updates.
\end{restatable}

To complete the picture of partially dynamic algorithms, we also study the $k$-center problem on graphs undergoing edge deletions only, known as the \emph{decremental} setting. Here, we obtain an algorithm that achieves a tight $(2+\epsilon)$ approximation ratio. 

\begin{restatable}{theorem}{deckcenttappr} \label{th:dec_kcent_2appr}
    Given a weighted undirected graph $G = (V, E, w)$ subject to edge deletions, an integer parameter $k \geq 1$,
    and a positive constant parameter $\epsilon < 1$, there is a deterministic decremental 
    $(2 + \epsilon)$-approximation  algorithm for the $k$-center problem on graphs, with
    $\tilde{O}(n^{o(1)}k)$ amortized update time over a sequence of $\Theta(m)$ updates.
\end{restatable}

We note that the $n^{o(1)}$ factors in the running time are also due to using partially dynamic approximate single-source shortest paths (SSSP) algorithms, which is inherent in our bounds based on similar reasoning as in the fully dynamic settings.





\paragraph{Outline.}
In Section~\ref{sec:overview} we give an overview of our algorithm and also discuss the main challenges we face in dynamic graphs, as opposed to metrics.
In Section~\ref{sec:reduction}, we review a well-known reduction that relates the $k$-center problem to finding a maximal independent set on a graph. This reduction is fundamental to our partially dynamic algorithms.
Section~\ref{sec:incremental} presents our primary technical contribution, showcasing the incremental algorithm of Theorem~\ref{th:incr_kcent_4appr}.
Section~\ref{sec:decremental} completes the partially dynamic picture by providing the decremental algorithm of Theorem~\ref{th:dec_kcent_2appr}.
In Section~\ref{sec:fullydynamic}, we explore the fully-dynamic setting, in which we use a different type of algorithm to prove Theorem~\ref{thm:fully-dynamic approx}. Unlike the reduction presented in Section~\ref{sec:reduction}, our approach here is based on the greedy algorithm of Gonzalez~\cite{Gonzalez85}.
In addition to the set of centers, we can also answer other natural queries, such as the corresponding center for each vertex. We briefly discuss this in Appendix~\ref{apx:queries}.

