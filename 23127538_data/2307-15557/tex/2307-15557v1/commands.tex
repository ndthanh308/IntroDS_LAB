% Multiple "thanks" that refer to the same text
% http://tex.stackexchange.com/a/4171
\newcommand*\samethanks[1][\value{footnote}]{\footnotemark[#1]}



% Allow page breaks in the middle of multi-line equations (align)
% Argument 1-4, where 1 is least permissive
\allowdisplaybreaks[1]



% Bold math in bold text
% http://tex.stackexchange.com/a/124311
\makeatletter
\g@addto@macro\bfseries{\boldmath}
\makeatother

% Using the new macro above, we also have to "undo" boldmath for mdseries. Some macros don't use the clean \mdseries to "unbold" text (e.g., amsmath's plain style uses \upshape); thus, it is necessary to modify other macros as well.
\makeatletter
\g@addto@macro\mdseries{\unboldmath}
\g@addto@macro\normalfont{\unboldmath}
\g@addto@macro\rmfamily{\unboldmath}
\g@addto@macro\upshape{\unboldmath}
\makeatother

% Fix bold math in theorem-like environments
% https://tex.stackexchange.com/a/653419
\makeatletter
\g@addto@macro\bfseries{\boldmath}
\def\thmhead@plain#1#2#3{%
  \thmname{#1}\thmnumber{\@ifnotempty{#1}{ }\@upn{#2}}%
  \thmnote{ {\the\thm@notefont\unboldmath(#3)}}}
\let\thmhead\thmhead@plain
\makeatother



% A command for "manual" citations which can be useful for providing citations in the abstract of a paper.
% \citem[Text]{key} will insert the citation [Text] and refer to the specified key in the biblography
\DeclareCiteCommand{\citem}
    {}
    {\mkbibbrackets{\bibhyperref{\usebibmacro{postnote}}}}
    {\multicitedelim}
    {}



% Change "plus" symbol for multi-author citations in alphabetic style
% http://tex.stackexchange.com/a/130031
\renewcommand*{\labelalphaothers}{\textsuperscript{+}}



% Change separator for multiple citations to a comma (instead of semicolon)
\renewcommand*{\multicitedelim}{\addcomma\space}



% Remove field "note" from full citations
\AtEveryCitekey{\clearfield{note}}



% Enable line breaks for long links using ocgcolorlinks
% http://tex.stackexchange.com/a/47309
\makeatletter
\AtBeginDocument{%
    \newlength{\temp@x}%
    \newlength{\temp@y}%
    \newlength{\temp@w}%
    \newlength{\temp@h}%
    \def\my@coords#1#2#3#4{%
      \setlength{\temp@x}{#1}%
      \setlength{\temp@y}{#2}%
      \setlength{\temp@w}{#3}%
      \setlength{\temp@h}{#4}%
      \adjustlengths{}%
      \my@pdfliteral{\strip@pt\temp@x\space\strip@pt\temp@y\space\strip@pt\temp@w\space\strip@pt\temp@h\space re}}%
    \ifpdf
      \typeout{In PDF mode}%
      \def\my@pdfliteral#1{\pdfliteral page{#1}}% I don't know why % this command...
      \def\adjustlengths{}%
    \fi
    \ifxetex
      \def\my@pdfliteral #1{\special{pdf: literal direct #1}}% isn't equivalent to this one
      \def\adjustlengths{\setlength{\temp@h}{-\temp@h}\addtolength{\temp@y}{1in}\addtolength{\temp@x}{-1in}}%
    \fi%
    \def\Hy@colorlink#1{%
      \begingroup
        \ifHy@ocgcolorlinks
          \def\Hy@ocgcolor{#1}%
          \my@pdfliteral{q}%
          \my@pdfliteral{7 Tr}% Set text mode to clipping-only
        \else
          \HyColor@UseColor#1%
        \fi
    }%
    \def\Hy@endcolorlink{%
      \ifHy@ocgcolorlinks%
        \my@pdfliteral{/OC/OCPrint BDC}%
        \my@coords{0pt}{0pt}{\pdfpagewidth}{\pdfpageheight}%
        \my@pdfliteral{F}% Fill clipping path (the url's text) with
                           % current color
        %
        \my@pdfliteral{EMC/OC/OCView BDC}%
        \begingroup%
          \expandafter\HyColor@UseColor\Hy@ocgcolor%
          \my@coords{0pt}{0pt}{\pdfpagewidth}{\pdfpageheight}%
          \my@pdfliteral{F}% Fill clipping path (the url's text)
                             % with \Hy@ocgcolor
        \endgroup%
        \my@pdfliteral{EMC}%
        \my@pdfliteral{0 Tr}% Reset text to normal mode
        \my@pdfliteral{Q}%
      \fi
      \endgroup
    }%
}
\makeatother