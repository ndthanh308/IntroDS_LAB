\documentclass[12pt,reqno]{amsart}
\def\checkmark{\tikz\fill[scale=0.4](0,.35) -- (.25,0) -- (1,.7) -- (.25,.15) -- cycle;} 
\usepackage[euler-digits]{eulervm}
\usepackage{graphicx}
\usepackage{amsfonts, amsthm, amssymb, amsmath,mathtools}
\usepackage{stmaryrd}
\usepackage{mathrsfs,array}
\usepackage{eucal,times,color,enumerate,accents}
\usepackage{url}
\usepackage{float}
\usepackage{pbox}
\usepackage[headings]{fullpage}

\usepackage{ytableau}
\usepackage{color}
\usepackage{mathrsfs}
\usepackage{amssymb}
\usepackage{bm}
\usepackage{enumerate}
\usepackage{amssymb}
%\usepackage{ulem}
\usepackage{hyperref, cleveref}
\usepackage[all,cmtip]{xy}
\usepackage{comment}



\newcommand{\bbG}{{\mathbb G}}
\newcommand{\bbC}{{\mathbb C}}
\newcommand{\bbZ}{{\mathbb Z}}
\newcommand{\bbR}{{\mathbb R}}
\newcommand{\bbP}{{\mathbb P}}
\newcommand{\bbQ}{{\mathbb Q}}
\newcommand{\Oh}{{\mathcal O}}
\newcommand{\calV}{{\mathcal V}}


\newcommand{\ncom}{\newcommand}

\ncom{\dho}{\partial}
\ncom{\rar}{\rightarrow}
\ncom{\imply}{\Rightarrow}
\ncom{\lrar}{\longrightarrow}
\ncom{\into}{\hookrightarrow}
\ncom{\onto}{\twoheadrightarrow}
\ncom{\ov}{\overline}
\ncom{\m}{\mbox}
\ncom{\sta}{\stackrel}
\ncom{\invlim}{\varprojlim}
\ncom{\xhat}{\widehat}

\ncom{\vspc}{\vspace{3mm}}
\ncom{\End}{{\cE}nd}
\ncom{\tensor}{\otimes}

\ncom{\al}{\alpha}
\ncom{\cHom}{{\mathcal Hom}}

\ncom{\A}{{\mathbb A}}
\ncom{\comx}{{\mathbb C}}
\ncom{\E}{{\mathbb E}}
\ncom{\F}{{\mathbb F}}
\ncom{\G}{{\mathbb G}}
\ncom{\K}{{\mathbb K}}
\ncom{\Le}{{\mathbb L}}
\ncom{\N}{{\mathbb N}}

\ncom{\p}{{\mathbb P}}
\ncom{\Q}{{\mathbb Q}}
\ncom{\R}{{\mathbb R}}
\ncom{\Z}{{\mathbb Z}}

\ncom{\f}{\dfrac}

\ncom{\wtil}{\widetilde}

\ncom{\ci}{{\mathpzc i}}

\ncom{\cA}{{\mathcal A}}
\ncom{\cC}{{\mathcal C}}
\ncom{\cE}{{\mathcal E}}
\ncom{\cF}{{\mathcal F}}
\ncom{\cG}{{\mathcal G}}
\ncom{\cH}{{\mathcal H}}
\ncom{\cI}{{\mathcal I}}
\ncom{\cJ}{{\mathcal J}}
\ncom{\cK}{{\mathcal K}}
\ncom{\cL}{{\mathcal L}}
\ncom{\cM}{{\mathcal M}}
\ncom{\cN}{{\mathcal N}}
\ncom{\cO}{{\mathcal O}}
\ncom{\cP}{{\mathcal P}}
\ncom{\cQ}{{\mathcal Q}}
\ncom{\cR}{{\mathcal R}}
\ncom{\cS}{{\mathcal S}}
\ncom{\cT}{{\mathcal T}}
\ncom{\cU}{{\mathcal U}}
\ncom{\cV}{{\mathcal V}}
\ncom{\cW}{{\mathcal W}}
\ncom{\cX}{{\mathcal X}}
\ncom{\cY}{{\mathcal Y}}
\ncom{\cZ}{{\mathcal Z}}



\ncom{\cSU}{{\mathcal S \mathcal U}}
\ncom{\eop}{{\hfill $\Box$}}
\ncom{\isom}{\cong}


\DeclareMathOperator{\Hilb}{Hilb}
\DeclareMathOperator{\Spec}{Spec}
\DeclareMathOperator{\HH}{H}
\DeclareMathOperator{\RR}{R}
\DeclareMathOperator{\CH}{CH}
\DeclareMathOperator{\Pic}{Pic}
\DeclareMathOperator{\MHM}{MHM}
\DeclareMathOperator{\Alb}{Alb}
\DeclareMathOperator{\im}{Im}
\DeclareMathOperator{\tr}{tr}
\DeclareMathOperator{\Sym}{Sym}
%\DeclareMathOperator{\IJ}{IJ}
\DeclareMathOperator{\Ext}{Ext}
\DeclareMathOperator{\Div}{div}
\DeclareMathOperator{\gr}{gr}
\DeclareMathOperator{\cl}{cl}
\DeclareMathOperator{\dR}{dR}
\DeclareMathOperator{\Db}{Db}
\DeclareMathOperator{\reg}{reg}
\DeclareMathOperator{\image}{im}
\DeclareMathOperator{\supp}{supp}

\newcommand{\ronto}{\twoheadleftarrow}
\newcommand{\by}[1]{\xrightarrow{#1}}
\newcommand{\ext}[1]{\operatorname{\stackrel{#1}{\wedge}}}

\newcommand\scalemath[2]{\scalebox{#1}{\mbox{\ensuremath{\displaystyle #2}}}}


%\newtheorem{theorem}{Theorem}[section]
%
%\newtheorem{corollary}[theorem]{Corollary}
%\newtheorem{proposition}[theorem]{Proposition}
%%\newtheorem{remark}[theorem]{Remark}
%\newtheorem{definition}[theorem]{Definition}

\newcommand{\abuts}{\Rightarrow}
\theoremstyle{plain}
\newtheorem{theorem}{Theorem}
\newtheorem{lemma}[theorem]{Lemma}
\newtheorem{conjecture}[theorem]{Conjecture}
\newtheorem{thm}[theorem]{Theorem}
\newtheorem{conj}{Conjecture}
\newtheorem{sublem}{Sublemma}
\newtheorem{corollary}[theorem]{Corollary}
\newtheorem{proposition}[theorem]{Proposition}
\theoremstyle{definition}
\newtheorem{assum}{Assumption}
\newtheorem{defn}{Definition}
\newtheorem{notes}{Notes}
\newtheorem{oss}{Observation}
\theoremstyle{remark}
\newtheorem{remark}{Remark}
\newtheorem{definition}{Definition}
\newtheorem{eg}{Example}
\newtheorem{problem}{Problem}
\newenvironment{diagram}[1]{\arraycolsep=\doublerulesep\begin{array}{#1}
	}{\end{array}}
\long\def\comment#1{} 


\newtheorem{exmp}{Example}[section]

%\pagenumbering{gobble}
\begin{document}
	
	\title{Some remarks on two-periodic modules over local rings}
	\author[N. Das]{Nilkantha Das}
	
	\address{Stat-Math Unit, Indian Statistical Institute, 203 B.T. Road, Kolkata 700 108, India.}
	\email{dasnilkantha17@gmail.com}
	\author[S. Dey]{Sutapa Dey}
	\address{Department of Mathematics, Indian Institute of Technology -- Hyderabad, 502285, India.}
	\email{ma20resch11002@iith.ac.in}
	
	\subjclass[2020]{13C12;  13D02; 13H10} 

\begin{abstract}
	In this note, some properties of finitely generated two-periodic modules  over commutative Noetherian local rings have been studied. We prove a weaker version of the Huneke-Wiegand conjecture for two-periodic modules \textemdash \, under certain assumptions, a two-periodic module is necessarily free. Given a two-periodic module with finite rank over a one-dimensional local ring, it is shown that the tensor product of the module with its dual has non-zero torsion. Moreover, if the base ring is one-dimensional, we show that with certain assumptions on modules, the tensor product of a two-periodic module with another finitely generated module is torsion-free if and only if the pair of modules is Tor-independent. We also discuss Auslander's  depth formula for a Tor-independent pair of modules in this setup. It is proved that such a formula holds for a Tor-independent pair of modules if one of the modules is two-periodic with finite Gorenstein dimension.  
\end{abstract}

	\keywords{Two-periodic modules; Torsion modules}
	\maketitle
	 
	\section{Introduction} 
	Throughout, $R$ denotes a commutative Noetherian local ring. All $R$-modules are assumed to be finitely generated. 
	
	The tensor product $M \otimes_RN$ of two modules $M$ and $N$ usually contains a non-zero torsion submodule. The assumption that this tensor product is ``nice"; for instance, torsion-free or reflexive, forces strong conditions on the modules $M$ and $N$.   Starting with Auslander's \cite{A1} rigidity theorem, such investigations have led to simple questions that have been hard to solve. To begin with, we consider the following open question of Huneke and Wiegand (cf. \cite[p. 473]{HW}):
	\begin{conj}[Huneke -- Wiegand] \label{conj_1}
		Let $R$ be a one-dimensional local ring. Let $M$ be a non-free and torsion-free module on $R$. Assume $M$ has rank. Then the torsion submodule of $M \otimes_R M^*$ is non-zero, i.e., $M \otimes_R M^*$ has (non-zero) torsion, where for any $R$-module $N$, $N^* := Hom_R(N,R)$.
	\end{conj} 
	
	\noindent Goto et al. state the following higher dimensional analogue of Huneke-Wiegand conjecture \cite[Conjecture 1.1]{GTTL}:
	\begin{conj}[Huneke -- Wiegand]\label{conj_2}
		Let $R$ be a Gorenstein local domain. Let $M$ be a maximal Cohen–Macaulay $R$-module. If $M \otimes_R M^*$ is torsion-free, then $M$ is free.
	\end{conj} 
	   
	In this note, we investigate some properties of two-periodic modules, including a weaker version of \Cref{conj_2}. Let $M$ be a $R$-module, and ${\bf F}: \ \ \rar F_1 \rar F_0 \rar M \rar 0$ be a minimal free resolution of $M$. Recall that $M$ is said to be \textit{two-periodic} (cf. \cite{Eisen}) if there is a map of complexes $s: {\bf F} \rar {\bf F}$ of degree $-2$ such that $s: F_{i+2} \rar F_i$ is an isomorphism for $i \geq 0$.  Equivalently, $M$ is two-periodic if and only if $M \cong \Omega_R^2 M$, where for any $i \geq 0$, $\Omega^i_R M$ denotes the $i^{\text{th}}$ syzygy of $M$. Two-periodic modules enjoy some nice consequences, for example, they have the same depth as that of the ring. %So, over Cohen-Macaulay rings, two-periodic modules are maximal Cohen-Macaulay. 
	Two-periodic modules have complexity 1 and a simpler free resolution of infinite length, thus providing a rich class of modules beyond those with finite projective dimensions. 

	
	Eisenbud in \cite[Theorem 6.1]{Eisen} showed that over a hypersurface local ring, a maximal Cohen-Macaulay module (with no free summands) is two-periodic, and all modules are eventually two-periodic. He also showed that over complete intersection local rings, all modules with bounded Betti numbers are eventually two-periodic (cf. \cite[Theorem 4.1]{Eisen}). Later, Gasharov and Peeva gave certain criteria for when modules over Cohen-Macaulay local rings and arbitrary local rings are eventually two-periodic; see Theorem 1.2 and Proposition 3.8 of \cite{GP}. 
	
	
	 Any two-periodic $R$-module $M$ is torsionless, so it is interesting to study whether for any $R$-module $N$, $M \otimes_R N$ will have torsion or not. When $N=M^*$ in particular, a weaker version of \Cref{conj_2} holds for two-periodic modules $M$ that are locally free on the punctured spectrum. 
	
	\begin{theorem} \label{thm_1} 
		Let $(R,m)$ be a local ring with $depth\, R \geq 2$.   Let $M$ be a two-periodic $R$-module which is locally free on the punctured spectrum $Spec(R) - \{m\}$. If $M \otimes_R M^*$ is reflexive, then $M$ is free.  
	\end{theorem}  
	
	
	This result is known to be true over local Cohen-Macaulay rings, see Celikbas and Dao \cite[Theorem 3.4]{CD2}. 

	However, our methods are elementary and demonstrate an approach suitable for two-periodic modules over non-Cohen-Macaulay rings as well.  
	
	Very recently, Celikbas et al. \cite[Theorem 1.3]{CUHA} solved \Cref{conj_1} for the class of two-periodic $R$-modules when $R$ is a one-dimensional domain. We point out that their proof can be modified to give: 
	
	\begin{theorem} \label{thm_cuha} 
		Let $R$ be a one-dimensional local ring and $M$ be a non-zero two-periodic $R$-module that has rank. Then $M \otimes_R M^*$ has (non-zero) torsion. 
	\end{theorem}
	
	More generally, we have the following rigidity-type result:
	\begin{theorem} \label{thm_2} 
		Let $R$ be a one-dimensional local ring. Let $M$ be a two-periodic $R$-module which is generically free, and $N$ be a torsionless $R$-module which has rank. Then $M \otimes_R N$ is torsion-free if and only if $M$ and $N$ are Tor-independent, that is, $Tor_i^R(M,N) = 0$ for $i \geq 1$.   
	\end{theorem}
	

	The last part of this note is devoted to the study of the depth of $M \otimes_R N$ when $M$ is two-periodic. 
	A pair $(M,N)$ of $R$-modules is said to satisfy the Auslander's depth formula (cf. \cite{A1}) if $$depth\, M + depth\, N = depth\, R + depth\, M \otimes_R N.$$ The depth formula is known to be true for the Tor-independent pair of modules $(M,N)$ (that is, $Tor_i^R(M,N) = 0$ for $i \geq 1$), where either $M$ or $N$ has finite projective dimension \cite[Theorem 1.2]{A1}. Huneke and Wiegand \cite[Proposition 2.5]{HW} showed that the depth formula holds for the Tor-independent pair $(M,N)$ of $R$-modules over complete intersection rings. The depth formula is shown to hold for Tor-independent pair of modules when one of the modules have finite complete intersection dimension by \cite[Theorem 2.5]{AY}, and independently by Iyenger \cite[Theorem 4.3]{Iyenger}.
	In the setup of two-periodic modules, we show the following:

	\begin{theorem}  \label{lemma_depth} Let $R$ be a local ring with positive depth and $M$ be a two-periodic $R$-module with finite Gorenstein dimension. Let $N$ be any $R$-module such that the pair $(M,N)$ is Tor-independent over $R$. Then the pair $(M,N)$ satisfies the depth formula. 
	\end{theorem}
	
	Over Gorenstein rings, it is not known if the depth formula for the Tor-independent pair $(M,N)$ holds or not. However, a derived version of the depth formula is known to be true over Gorenstein rings, which says if the pair $(M,N)$ is Tate-Tor independent, then the derived version of the depth formula holds over Gorenstein rings \cite[Theorem 2.3]{CJ}. \\
	
\noindent{\bf Acknowledgements}. The first author is supported by the INSPIRE faculty fellowship (Ref No.: IFA21-MA 161) funded by the DST, Govt. of India. The named author is partially supported by a NET Senior Research Fellowship from UGC, MHRD, Govt. of India. Both the authors are grateful to Amit Tripathi for suggesting the problem and several fruitful discussions. The second author would like to thank IIT Hyderabad for the hospitality during the Conference on Commutative Algebra and Algebraic Geometry (CoCAAG 2023), where the initial discussion about the project was started. 
	\section{Preliminary} 
	Let $R$ be a local ring and $M$, $N$ be two finitely generated $R$-modules. We say that $M$ and $N$ are projectively equivalent, written as $M \approx N$, if there exist two projective $R$-modules $P$ and $Q$ such that $M \oplus P \cong N \oplus Q$. 
	
	Let $F_1 \rar F_0 \rar M \rar 0$ be a free (equivalently, projective) presentation of $M.$  The Auslander dual of $M$, denoted by $D(M)$,  is the cokernel of the induced map $F_0^* \rar  F_1^*$ (\cite[Definition 2.5]{AB}), where $M^*$ denotes $Hom_R(M,R)$. Dualizing the above sequence, we get the following exact sequence
	\begin{equation}\label{equ_aus_dual_exact} 
		0 \rightarrow M^* \rightarrow F_0^* \rar  F_1^* \rightarrow D(M) \rightarrow 0. 
	\end{equation}
	It is worthwhile to note that $D(M)$ depends on the free presentation of $M$; however, it is determined uniquely up to projective equivalence (cf. \cite[Proposition 4]{Masek}). 
	It immediately follows from \eqref{equ_aus_dual_exact} that $M^* \approx \Omega^2_R D(M)$.  
	We refer \cite{Masek} for a few properties of the Auslander dual used in this paper, all of which are originally stated in \cite{AB}.
	 
	An $R$-module $M$ is said to be torsionless (reflexive, respectively) if the natural map $M \rar M^{\ast \ast}$ is injective (bijective, respectively). Moreover, a reflexive module is called totally reflexive if $Ext^i_R(M,R)=Ext^i_R(M^*,R)=0$ for all $i \geq 1$. It is well known that a module is torsionless if and only if it embeds inside a free $R$-module. A torsionless module is thus torsion-free. 
	 A reflexive module is a second syzygy of some $R$-module. Reflexivity and torsionlessness of $M$ can be expressed as the vanishing of certain cohomologies of $D(M)$. More precisely, the following sequence is exact
	\begin{equation}\label{eqn_1}
		0 \rar Ext^1_R\left(D(M),R\right) \rar M \rar M^{**} \rar Ext^2_R(D(M),R) \rar 0.
	\end{equation}
	Auslander and Bridger in \cite[Theorem 2.8]{AB} extended the exact sequence \eqref{eqn_1} functorially to the higher cohomologies and proved that, for any $k \geq 0$, there exist exact sequences 
	\begin{align}   
		\label{eqn_3} \scalemath{0.9}{0 \rar Ext_R^1\left(D\left(\Omega^{k}_RM\right), -\right) \rar Tor^R_{k}\left(M,-\right) \rar Hom_R\left(Ext_R^k(M,R), -\right) \rar   Ext_R^2\left(D\left(\Omega^{k}_RM\right), -\right)},\\
		\label{eqn_4} \scalemath{0.9}{Tor^R_2\left(D\left(\Omega^{k}_RM\right), -\right) \rar	Ext_R^k\left(M, R\right) \otimes (-) \rar Ext_R^k\left(M, -\right) \rar Tor^R_1\left(D\left(\Omega^{k}_RM\right), -\right) \rar 0.}
	\end{align} 
	
	Let $M$ be a torsionless $R$-module, and let $f_1,\cdots, f_n \in M^*$ be a minimal generating set of $M^*$. Define a map $f: M \rar R^n$ as $x \mapsto (f_1(x), \cdots, f_n(x))$. Then $f$ is an injective map. The universal pushforward of $M$ is the $R$-module $M_1 := coker(f)$. We note the following result
	
	\begin{lemma} \label{lemma_ext1_equals_tor1}
		Let $R$ be a local ring and $M$ be a torsionless $R$-module. Assume $M_1$ to be the universal pushforward of $M$. Then 
		\begin{enumerate}
			\item $\Omega^1_RD(M) \approx D(M_1)$. 
			\item $Ext^1_R\left(D(M), N\right) \cong Tor_1^R\left(M_1,N\right)$ for any $R$-module $N$. 
		\end{enumerate}
	\end{lemma}
	\begin{proof} 
		Consider the defining short exact sequence of the universal pushforward 
		$$0 \rar M \rar R^n \rar M_1 \rar 0.$$
		Applying \cite[Lemma 6]{Masek}, we get the following exact sequence (for some suitable choice of Auslander duals)
		$$0 \rar M_1^* \rar (R^{n})^* \rar M^* \rar D(M_1) \rar D(R^n) \rar D(M) \rar 0.$$
		Note that the map $\left(R^{n}\right)^*  \rar M^*$ is surjective by the construction of the universal pushforward. The above exact sequence gives rise to the following short exact sequence
		$$0\rar D(M_1) \rar D(R^n) \rar D(M) \rar 0,$$
		and hence $D(M_1)$ is projectively equivalent to $\Omega^1_R(D(M)$ as $D(R^n) \approx 0$.
		
		To see the second part, consider the exact sequence \eqref{eqn_4}, with $k = 1$. We obtain the following exact sequence 
		\begin{align*}
			\scalemath{0.85}{Tor_2^R\left(D\left(\Omega^1_RD(M) \right), N\right) \rar Ext^1_R\left(D(M), R\right) \otimes_R N \rar Ext^1_R\left(D(M), N\right) \rar Tor_1^R\left(D\left(\Omega^1_RD(M)\right), N\right) \rar 0.}
		\end{align*} 
		The module $M$ being torsionless, the second term in the above exact sequence vanishes. Thus, we have isomorphisms,
		$$Ext^1_R\left(D(M), N\right) \cong Tor_1^R\left(D\left(\Omega^1_RD(M)\right), N\right) \cong Tor_1^R\left(D\left(D(M_1)\right), N\right) \cong Tor_1^R\left(M_1,N\right),$$
		where the second isomorphism uses the projective equivalence stated in the first part of this lemma, and the last isomorphism follows from the fact that $D(D(M_1)) \approx M_1$ (cf. \cite[Remark 3, p. 5789]{Masek}). 
	\end{proof}
	

 A minimal free resolution ${\bf F}: \ \ \rar F_1 \rar F_0$ of a two-periodic $R$-module $M$ has simpler description. The resolution ${\bf F}$ is determined by $F_0$ and $F_1$ only; other free modules are given by the alteration of $F_0$ and $F_1$. Thus, all the information of ${\bf F}$ is encoded in the following two short exact sequences
	\begin{align}
		\label{seq_A} 0 \rar \Omega^1_RM \rar F_0 \rar M \rar 0, \\
		\label{seq_B} 0 \rar M \rar F_1 \rar \Omega^1_RM \rar 0, 
	\end{align}
	where the $M$ in the exact sequence \eqref{seq_B} is identified with $\Omega^2_RM$.
	As $M$ is contained inside the free module $F_1$, it follows that it is torsionless, and hence torsion-free, in particular.
	

	Before proceeding further, let us make a definition. Given a $R$ module $M$, we say $M$ is \textit{projectively two-periodic} if $M \approx \Omega^2_RM$. It is worthwhile to note that two-periodic modules are projectively two-periodic. The next result says for certain projectively two-periodic modules, the regular dual and the Auslander dual are projectively equivalent.
	
	\begin{lemma} \label{aus-dual} 
		Let $R$ be a local ring and $M$ be a projectively two-periodic and totally reflexive $R$-module. Then $M^* \approx D(M)$.
	\end{lemma}
	\begin{proof}
		Let $F_1 \rar F_0 \rar M \rar 0$ be a minimal presentation of $M$. 
		Since $M$ is totally reflexive, $Ext^1_R\left(M,R\right) = Ext^1_R\left(\Omega^1_RM,R\right)  = 0$. Then, dualizing the short exact sequence $0 \rar \Omega_R^1 M \rar F_0 \rar M \rar 0$, we get $Coker \left(M^* \hookrightarrow F_0\right)$ is $ \left(\Omega_R^1 M\right)^*$. 
		Therefore, the exact sequence \eqref{equ_aus_dual_exact} splits into the following two short exact sequences 
		\begin{equation*}
			0 \rar M^* \rar F_0 \rar  \left(\Omega_R^1 M\right)^* \rar 0,
		\end{equation*} 
		\begin{equation*}
			0 \rar  \left(\Omega_R^1 M\right)^* \rar F_1 \rar D(M) \rar 0.
		\end{equation*} 	
		Dualizing both the short exact sequences again and using the fact that $\Omega_R^1 M$ is totally reflexive, we derive the following two short exact sequences 
		$$0 \rar \left(\Omega_R^1 M\right)^{**} \rar F_0 \rar M^{**} \rar 0, $$
		$$\text{ and \, \, } 0 \rar D(M)^* \rar F_1 \rar  \left(\Omega_R^1 M\right)^{**} \rar 0.$$
		Combining both sequences, we get
		\begin{equation}\label{eqn_temp1}
			0 \rar D(M)^* \rar F_1 \rar F_0 \rar M \rar 0.
		\end{equation}
		On the other hand, we have the natural exact sequence 
		 
		\begin{equation}\label{eqn_temp2}
			0 \rar \Omega_R^2 M \rar  F_1 \rar F_0 \rar M \rar 0.
		\end{equation} 
		Schanuel's lemma is now applied to the pair of exact sequences \eqref{eqn_temp1} and \eqref{eqn_temp2} to conclude  $D(M)^* \approx \Omega_R^2 M$.  Since $M$ is projectively two-periodic, it follows that  $D(M)^* \approx M$. Therefore, $D(M)^{**} \approx M^*$ as projective equivalence is preserved under the process of dualizing. Since $M$ is totally reflexive, so is $D(M)$. Hence, $D(M)$ is reflexive, in particular. The result follows. 
	\end{proof} 
	
	We are now in a position to give a better description of the first syzygy of certain two-periodic modules, which will be useful in proving the main theorems.
	\begin{proposition} \label{lemma_univ_push}
		Let $R$ be a local ring and $M$ be a two-periodic totally reflexive $R$-module. Then $\Omega^1_R M \approx M_1,$ where $M_1$ is the universal pushforward of $M$.   
	\end{proposition}
	\begin{proof} 
		We first show that $M^*$ is projectively two-periodic. Let ${\bf F} \rar M$ be a minimal free resolution of $M$. Since $M$ is totally reflexive, $Ext^1_R\left(M,R\right) = Ext^1_R\left(\Omega^1_RM, R\right)  = 0$. Both the vanishing implies $Im\left(F_0^* \rar F_1^*\right) \cong \left(\Omega^1_RM\right)^*$, and thus
		\begin{equation}\label{eqn_temp_3}
			\Omega^1_RD(M) \approx \left(\Omega^1_R M\right)^*.
		\end{equation}
		Since $M$ is two-periodic and totally reflexive; dualizing both the short exact sequences \eqref{seq_A} and \eqref{seq_B}, the following short exact sequences can be obtained $$0 \rar M^* \rar F_0 ^*\rar \left(\Omega_R^1 M\right)^* \rar 0,$$ and $$ 0 \rar \left(\Omega_R^1 M\right)^* \rar F_1^* \rar M^* \rar 0.$$
		Both the exact sequences give rise to the following exact sequence $$0 \rar M^* \rar F_0 ^*\rar  F_1^* \rar M^* \rar 0.$$ On the other hand, there always exist free $R$ modules $G_1, G_2$ making the following exact sequence $$0 \rar \Omega_R^2 M^* \rar G_1 \rar  G_0 \rar M^* \rar 0.$$
		Now applying Schanuel's lemma to the last two exact sequences, we conclude that $M^* \approx \Omega_R^2 M^*$, that is, $M^*$ is projectively two periodic.  
		
		It is worthwhile to note that $\Omega_R^1M$ is two-periodic and totally reflexive as $M$ is so. By the same argument as above, we conclude that $\left(\Omega_R^1M\right)^*$ is projectively two periodic, that is, $\left(\Omega_R^1 M\right)^* \approx \Omega_R^2 \left(\Omega_R^1 M\right)^*$. Being the dual of a totally reflexive module, $\left(\Omega_R^1 M\right)^*$ is totally reflexive as well. Now apply \Cref{aus-dual} to conclude $D\left(\left(\Omega_R^1 M\right)^* \right) \approx \left(\Omega_R^1 M\right)^{**}$.
		
		By \Cref{lemma_ext1_equals_tor1}, we already know that $D(M_1) \approx \Omega^1_R D(M)$. Therefore, we get 
		$$M_1 \approx D\left(D(M_1)\right) \approx D\left(\Omega^1_RD(M)\right) \approx D\left(\left(\Omega^1_R M\right)^*\right) \approx \left(\Omega^1_R M\right)^{**} \cong \Omega^1_R M,$$
		where the third equivalence follows from the projective equivalence in \cref{eqn_temp_3} and the last isomorphism uses the fact that $\Omega^1_R(M)$ is reflexive.
	\end{proof}
	We say an $R$-module $M$ is generically free if it is locally free on $Ass (R)$.	Using a result of Huneke and Wiegand \cite{HW}, we have a better description of the torsion submodule of $M\otimes_R N$ when $M$ is two-periodic and generically free.
	
	\begin{lemma} \label{lemma_3}  Let $R$ be a local ring and $M$ be a two-periodic $R$-module which is generically free. Furthermore, assume $N$ is a torsionless $R$-module. Then the torsion part of $M \otimes_R N$, $T\left(M \otimes_R N \right) \cong Tor_2^R \left( M,N \right)$.
	\end{lemma}
	\begin{proof} Since $M$ and $N$ are both torsionless, there exist free $R$-modules $F$ and $F'$ such that $M \subset F$ and $N \subset F'$. By a result of Huneke and Wiegand \cite[Lemma 1.4]{HW}, it follows that $T\left(M \otimes_R N \right) \cong Tor_2^R\left( F/M, F'/N \right)$. Tensoring the sequence $0 \rar M \rar F \rar F/M \rar 0$ with $F'/N$, we get $$Tor_2^R\left(F/M, F'/N \right) \cong Tor_1^R \left( M, F'/N \right).$$ By two-periodicity of $M$, $Tor_1^R\left(M, F'/N \right) \cong Tor_3^R \left(M, F'/N\right).$ Tensoring the short exact sequence $0 \rar N \rar F' \rar F'/N \rar 0$ by $M$, we see that $Tor_3^R\left(M, F'/N \right)$ is isomorphic to $Tor_2^R\left(M,N\right)$ . Thus, we have shown that $$T\left(M \otimes_R N\right) \cong Tor_2^R\left(F/M, F'/N\right) \cong Tor_2^R\left(M,N\right).$$
	\end{proof}
	Given two $R$-modules $M$ and $N$, there always exists a natural map $\alpha: M^* \otimes N \rightarrow Hom_R(M,N).$ However, $\alpha$ is neither injective nor surjective in general. Under some additional hypotheses on $M$ and $N$, the map $\alpha$ will be an isomorphism. Along this line, we will state one version of a result of Jothilingam \cite{J}, as stated in Dao \cite[Theorem 2.4]{D} which will be needed in proving our main theorems. Before stating the result, let us recall the definition of Tor-rigidity of a module.
	\begin{defn} 
		Let $M$ be a finitely generated module over a local ring $R.$ We say that $M$ is Tor-rigid if, for all $R$-modules $N$, $Tor_1^R (M,N)=0$ implies $Tor_2^R (M,N)=0$. 
	\end{defn}
	We are now all set to state the result of Jothilingam.
	\begin{theorem}[Jothilingam] \label{theorem_J} 
		Let $M$ and $N$ be finitely generated modules over a local ring $R$ such that $N$ is Tor-rigid. If $Ext^1_R(M,N) = 0,$ then the canonical map $M^* \otimes_R N \rar Hom_R(M,N)$ is an isomorphism. In particular, $Ext^1_R(N,N) = 0$ if and only if $N$ is free. 
	\end{theorem}
	\begin{remark} \label{remark_3}
		The map $\alpha: M \otimes_R M^* \rar Hom_R(M,M)$ was studied by Auslander and Goldman  \cite[Theorem A.1]{AG} where they showed that surjectivity of $\alpha$ implies $M$ is free. %A more general version of this fact follows from \cite[Lemma 3.9]{Yos}. 
	\end{remark}
	
	\section{Proof of the main results}
	\subsection{Torsionness of $M \otimes M^*$}
	
	If $M$ is generically free, then $M \otimes_R M^*$ being torsion-free has some interesting consequences. For instance, putting $k=0$ and $N=M$ in the exact sequence \eqref{eqn_4}, we get the following exact sequence
	\begin{align} \label{eqn_5} 
		\rar Tor_2^R\left(D(M), M\right) \rar M \otimes_R M^* \xrightarrow{\alpha} Hom_R(M,M) \rar Tor_1^R\left(D(M), M\right) \rar 0. 
	\end{align} 
	Since $M$ is generically free, the $Tor_2$ term is a torsion module. Therefore, its image vanishes, and hence the natural map $$\alpha: M \otimes_R M^* \rar Hom_R(M,M)$$ is injective. 
	
	 
	
	For two-periodic modules, we can further exploit these properties of the map $\alpha.$
	\begin{lemma} \label{lemma_tor_1} Let $R$ be a local ring and $M$ be a non-zero $R$-module that is two-periodic. Then $Tor_1^R(M, M^*) \neq 0$.  
	\end{lemma}
	\begin{proof} Assume that $Tor_1^R (M,M^*) = 0$. Since $M$ is two-periodic, let ${\bf F}: \ \ \rar F_1 \rar F_0 \rar M \rar 0$ be a minimal free resolution of $M$ satisfying exact sequences \eqref{seq_A} and \eqref{seq_B}. Consider the following diagram 
		\begin{equation*}
			\begin{gathered}
				\xymatrix@C-=1.2cm@R-=2.2cm{0 \ar[r] & \Omega_R^1M \otimes_R M^*\ar[r] \ar[d]^{\beta}& F_0 \otimes_R M^* \ar[d] \ar[r]& M \otimes_R M^*\ar[r] \ar[d]^{\alpha} & 0 \\ 0 \ar[r] & Hom_R(M,\Omega_R^1M) \ar[r] & Hom_R(M, F_0) \ar[r]& Hom_R(M,M) \ar[r]  & }
			\end{gathered} 
		\end{equation*} Note that the vertical arrow in the middle is an isomorphism. Thus, $\beta$ is necessarily injective, thanks to the snake lemma. Consider another diagram
		\begin{equation}
			\begin{gathered}
				\xymatrix@C-=1.2cm@R-=2.2cm{ \ar[r] & M \otimes_R M^*\ar[r] \ar[d]^{\alpha}& F_1 \otimes_R M^* \ar[d] \ar[r]& \Omega^1_RM \otimes_R M^*\ar[r] \ar[d]^{\beta} & 0 \\ 0 \ar[r] & Hom_R(M,M) \ar[r] & Hom_R(M, F_1) \ar[r]& Hom_R(M,\Omega^1_R M) \ar[r]  & }
			\end{gathered} 
		\end{equation} 
		Since $\beta$ is an injective map as shown above and the middle vertical map is an isomorphism, therefore $\alpha$ is a surjective map. It follows from \cite[Theorem A.1]{AG} that $M$ is free. 
		Thus, $M$ is necessarily zero as follows from its two-periodicity. This gives us a contradiction.
	\end{proof}
	
	
	\begin{proposition} \label{prop_ext} Let $R$ be a local ring and $M$ be a two-periodic  generically free $R$-module. Let $N$ be any $R$-module. Then $M \otimes_R N^*$ is torsion-free if and only if $Ext^1_R(N,M) = 0$. In particular, $M \otimes_R M^*$ is torsion-free if and only if $Ext^1_R(M,M) =0$. 
	\end{proposition}
	\begin{proof} 
	Let ${\bf F}: \ \ \rar F_1 \rar F_0 \rar M \rar 0$ be a minimal free resolution of $M$ satisfying exact sequences \eqref{seq_A} and \eqref{seq_B}. We have the following diagram
		\begin{equation*}
			\begin{gathered}
				\xymatrix@C-=1.2cm@R-=2.2cm{\ar[r] & \Omega_R^1M \otimes_R N^*\ar[r] \ar[d]^{\delta}& F_0 \otimes_R N^* \ar[d] \ar[r]& M \otimes_R N^*\ar[r] \ar[d]^{\gamma} & 0 \\ 0 \ar[r] & Hom_R(N,\Omega_R^1M) \ar[r] & Hom_R(N, F_0) \ar[r]& Hom_R(N,M) \ar[r]  & }
			\end{gathered} 
		\end{equation*} Since the middle vertical map is an isomorphism, by snake lemma, $Ker(\gamma) = Coker(\delta).$ Consider the leftmost column -  our assumptions imply that $Hom_R(N,M)$ is torsion-free, so $$T\left(M \otimes_R N^*\right) = T\left(Ker \, \gamma\right) = T\left(Tor_2^R\left(D(N), M\right) \rar M \otimes_R N^*\right)$$ (where for any $R$-module, $N$, $T(N)$ denotes the torsion submodule of $N$). Note that $Tor_2^R(D(N), M)$ is a torsion module. Thus, $M \otimes_R N^*$ is torsion-free if and only if the map $\gamma$ is injective (equivalently, if and only if $\delta$ is surjective).  
		
		Consider the following diagram
		\begin{equation} \label{dgm_2}
			\begin{gathered}
				\xymatrix@C-=1.2cm@R-=2.2cm{ \ar[r] & M \otimes_R N^*\ar[r] \ar[d]^{\gamma}& F_1 \otimes_R N^* \ar[d] \ar[r]& \Omega^1_RM \otimes_R N^*\ar[r] \ar[d]^{\delta} & 0 \\ 0 \ar[r] & Hom_R(N,M) \ar[r] & Hom_R(N, F_1) \ar[r]& Hom_R(N,\Omega^1_RM) \ar[r]  & }
			\end{gathered} 
		\end{equation} Here again the middle map is an isomorphism. Note that $\delta$ is surjective if and only if the map $Hom_R(N, F_1) \rar Hom_R (N,\Omega^1_RM)$ is onto. This proves that $M \otimes_R N^*$ is torsion-free if and only if $Ext^1_R(N,M) = 0$.
	\end{proof}
	
	
	\begin{eg}
		Consider the one-dimensional hypersurface ring $R= k[[X,Y]]/(XY)$, where $k$ is a field. Let $M = R/(X).$ Then $M$ is a maximal Cohen-Macaulay module. Clearly, $M$ is 2-periodic and admits a minimal free resolution of the form $$\rar R \xrightarrow{Y} R \xrightarrow{X} R \rar M \rar 0.$$ One can show that $M$ is generically free,  $Hom_R(M, R) \cong M$ and $Hom_R(M,\Omega^1_R M) = 0$. In particular, $M \otimes_R M^* \cong M$ is torsion-free and $Ext^1_R(M,M) = 0$.
	\end{eg}
	
	\begin{remark}
		When $R$ is a Gorenstein ring, the one-dimensional version of \Cref{prop_ext} follows from a more general result of Huneke and Jorgensen \cite[Theorem 5.9]{HJ}, which explores the connection between the vanishing of certain $d$ consecutive Ext modules and the Cohen-Macaulayness of a tensor product.
	\end{remark}
	
	\begin{remark} \label{remark_1} With assumptions as in \Cref{prop_ext}, and putting $N = M$, we observe that if $M \otimes_R M^*$ is torsion-free, then the top exact sequence in diagram \eqref{dgm_2} shows that $Tor_1^R\left(\Omega^1_RM, M^*\right) = 0$. In particular, $Tor_2^R\left(M, M^*\right) \cong Tor_1^R\left(\Omega^1_RM, M^*\right) = 0$. We will need this observation later. 
	\end{remark}
	
	\begin{corollary}
		Let $R$ be a local ring and $M$ be a two-periodic generically free $R$-module. Assume further that $M$ is rigid. Then $M \otimes_R M^*$ has torsion.
	\end{corollary}
	\begin{proof}
		This follows immediately from \Cref{theorem_J} and \Cref{prop_ext}, taking $N = M.$ 
	\end{proof}
	
	
	\begin{proof}[Proof of \Cref{thm_1}] Any finite reflexive module is a 2nd syzygy. In particular, if $depth\, R \geq 2$, then the depth of a reflexive module is $\geq 2.$ 
		
	 By two-periodicity of $M$ and torsion-freeness of $M \otimes_R M^*$, we have a short exact sequence 
		\begin{align}\label{seq_1} 
			0 \rar M \otimes_R M^* \rar F_1 \otimes_R M^* \rar \Omega^1_RM \otimes_R M^* \rar 0.
		\end{align}
		Vanishing of $Tor_1^R\left(\Omega^1_RM, M^*\right) $ is discussed in \Cref{remark_1}.
		By \Cref{lemma_tor_1}, the finite length $R$-module  $Tor_1^R(M, M^*) = Ker \left(\Omega^1_RM \otimes_R M^* \rar F_0 \otimes_R M^*\right)$ is nonzero. Thus, $depth\,\Omega^1_R M \otimes_R M^* = 0.$ By depth lemma, the exact sequence \eqref{seq_1} gives $depth\, M \otimes_R M^* = 1$ which contradicts the reflexivity assumption. 
	\end{proof}
	\subsection{Hochster's theta invariant}
	
	In \cite{CUHA}, Celikbas et al. have given a generalization of Hochster's theta invariant (cf. \cite{H81}) for certain two-periodic modules. In this section, we briefly recall this generalization and some properties. We use these methods to prove a variant of \cite[Theorem 1.3]{CUHA}.
	
	\begin{defn}  \label{defn_hoch}
		Let $R$ be a one-dimensional local ring and $M$ be a $R$-module. Furthermore, assume $M$ is two-periodic and generically free. It follows that for any $R$-module $N$, and for any $i \geq 1$, $Tor_{i}^R (M,N) \cong Tor_{i+2}^R (M,N)$ and $Tor_{i}^R (M,N)$ has finite length.
		Then the Hochster's theta invariant for the pair $(M,N)$, denoted by $\Theta^R(M,N)$, is defined as 
		$$\Theta^R (M,N) := \lambda \left(Tor_{2n}^R \left( M,N \right)\right) - \lambda\left(Tor_{2n-1}^R \left(M,N\right)\right),$$ for some $n \geq 1$. Here $\lambda(G)$ denotes the length of the module $G$. It is clear that $\Theta^R (M,N)$ is well-defined.
	\end{defn}
	
	\begin{remark} \label{theta_additive} Our definition is subsumed by the more general definition given in \cite{CUHA}. Thus, the general properties of theta invariant that they have discussed hold for our case as well. In particular, for any short exact sequence of $R$-modules,
		$ 0 \rar N_1 \rar N_2 \rar N_3 \rar 0 $, we have $$ \Theta^R (M,N_2) = \Theta^R (M,N_1) + \Theta^R (M,N_3).$$ See \cite[Theorem 3.2]{CUHA} for a proof of this fact. 
	\end{remark}
	\begin{defn}
		We say a pair of modules $(M,N)$ over the ring $R$ is Tor-rigid if $Tor_i^R(M,N)=0$ for some $i \geq 1$ implies $Tor_j^R(M,N)=0$ for all $j \geq i$.
	\end{defn}
	
	
	We denote by $\overline{G} (R)_\mathbb{Q}$ the reduced Grothendieck group with rational coefficients, that is, $\overline{G} (R)_\mathbb{Q}=\left(G(R)/ \mathbb{Z}\cdot[R] \right) \otimes_{\mathbb{Z}} \mathbb{Q}$, where $G(R)$ is the Grothendieck group of finitely generated $R$-modules and $[R]$ denotes the class of $R$ in $G(R)$. 
	
	\begin{lemma} \label{lemma_rigid}
		Let $R$ be a one-dimensional local ring and $M$ be a non-zero two-periodic generically free $R$-module. Let $N$ be a $R$-module which has rank. Then $\Theta^R(M,N)$ vanishes. In particular, the pair $(M,N)$ is Tor-rigid. 
	\end{lemma}
	
	\begin{proof} Note that the theta invariant induces a map
		$$ \Theta^R (M,-) : \overline{G} (R)_\mathbb{Q} \rar \mathbb{Q}.$$ This map is in fact well-defined as follows from \cite[corollary 3.3]{CUHA}.
		Since $N$ is a module with rank over a one-dimensional local ring, it follows from \cite[Proposition 2.5]{CD}, as stated in \cite[2.14]{CUHA}, that $[N]=0$ in $\overline{G} (R)_\mathbb{Q}$, where $[N]$ denotes the class of $N$ in $\overline{G} (R)_\mathbb{Q}$. Hence, $\Theta^R (M,N)$ vanishes. Since $M$ is two-periodic, this implies that the pair $(M,N)$ is Tor-rigid.
	\end{proof}
	
	
	\begin{proof}[Proof of Theorem \ref{thm_cuha}]
		Assume the contrary, i.e., $M\otimes_R M^*$ is torsion-free. By \Cref{remark_1}, the hypothesis implies that $Tor_2^R(M,M^*) = 0$. Since $M^*$ has rank, by \Cref{lemma_rigid}, $\Theta^R(M,M^*)$ vanishes. Therefore, $Tor_1^R(M,M^*) = 0$ which contradicts  \Cref{lemma_tor_1}.
	\end{proof}
	
	
	\begin{proof}[Proof of Theorem \ref{thm_2}]
		By \Cref{lemma_3}, we know that $M \otimes_R N$ is torsion-free if and only if $T(M \otimes_R N) \cong Tor_2^R(M,N) = 0$. Moreover, \Cref{lemma_rigid} immediately yields that the pair $(M,N)$ is Tor-rigid. Therefore,  $Tor_2^R(M,N) = 0$ if and only if $Tor_j^R(M,N) = 0$ for $j \geq 2$. We get
		$$Tor_1^R\left(M,N\right) \cong Tor_1^R\left(\Omega_R^2 M,N\right) \cong Tor_3^R\left(M,N\right) =  0,$$
		where the first isomorphism follows from the two-periodicity of $M$. This completes the proof.
	\end{proof}
	\subsection{Auslander's depth formula}
In this subsection, our goal is to prove \Cref{lemma_depth}. The statement of the theorem involves Gorenstein dimension; we will recall the definition for the sake of completion. A $R$-module $M$ is said to have finite Gorenstein dimension (G-dim$(M)$, for short) if for some integer $k \geq 0$, there exists an exact sequence of the form
	$$0 \rar M_k \rar M_{k-1} \rar \cdots \rar M_1 \rar M_0 \rar M \rar 0,$$
where the $M_i$'s are totally reflexive $R$-modules. If G-dim$(M)$ is finite, G-dim$(M)$ is defined to be the smallest such $k$.
	\begin{proof}[Proof of \Cref{lemma_depth}]  The module $M$ is given to be two-periodic; from the depth lemma we have depth $M$ = depth $R$. Since it has finite Gorenstein dimension, by Auslander-Bridger formula  \cite[Theorem 4.13]{AB}, it must be a totally reflexive $R$-module. It suffices to show that $depth\, M \otimes N = depth\, N.$ Putting $k=0$ in exact sequence \eqref{eqn_3}, we get the following exact sequence 
		$$0 \rightarrow Ext^1_R \left(D(M),N\right) \rightarrow M \otimes_R N \rightarrow Hom_R\left(M^*,N\right) \rightarrow Ext^2_R \left(D(M),N\right).$$
		We will now show that $Ext^1_R \left(D(M),N\right)=0$ and $Ext^2_R \left(D(M),N\right)=0$. Indeed, apply \Cref{lemma_ext1_equals_tor1} and \Cref{lemma_univ_push} to conclude the following
		$$Ext^1_R \left(D(M),N\right) \cong Tor_1^R\left(M_1, N\right) \cong Tor_1^R\left(\Omega^1_R(M), N\right) = Tor_2^R\left(M, N\right)= 0.$$
		To get $Ext^2_R \left(D(M),N\right)$, we dualize the short exact sequence \eqref{seq_B}, and get the following short exact sequence
		$$0 \rar \left( \Omega_R^1 M \right)^* \rar F_1^* \rar M^* \rar 0.$$
		The right exactness follows from the fact that $M$ is totally reflexive, and thus, $Ext^1_R\left( \Omega_R^1M,R\right)=0$. Therefore, for a suitable choice of Auslander duals, the following sequence
		$$0 \rar D\left(\Omega_R^1M\right) \rar D(F_1) \rar D(M) \rar 0$$ is exact as follows from \cite[Lemma 6]{Masek}. Note that $D(F_1) \approx 0$, and hence $Ext^i_R\left( D(F_1),N\right)=0$ for $i \geq 1$ and $R$-module $N$. Thus we get 
		$$Ext^2_R(D(M), N) \cong Ext^1_R(D(\Omega_R^1M), N) \subset Tor_1^R(M,N) = 0.$$ 
		 
		Thus, we have shown that $$ M \otimes_R N \cong Hom_R(M^*,N).$$ When depth of $N$ is 0, we have an injection $R/m \hookrightarrow N$. Applying $Hom_R(M^*,-)$, we see $Hom_R(M^*,N)$ contains a non-zero module of depth 0. Therefore, the depth of $Hom_R(M^*, N)$ is $0$ as well. So, the depth formula is satisfied in this case.
		If $depth\, N > 0$, then $depth\, M \otimes_R N = depth\, Hom_R(M^*, N) > 0$. Hence $R,M,N,M \otimes_R N$ all have positive depth, so we have a common regular element. Let $d$ be the smallest dimension of $R$ for which the theorem fails to hold. Now, we go modulo the common regular element. Two-periodicity of the new quotient module is preserved by \cite[Theorem 1.1.5]{BH}, as the minimality is intact. 
		It follows from \cite[Theorem 8.7]{Gdim} that the finiteness of Gorenstein dimension is not affected after going modulo the regular element. Thus, by \cite[Lemma 10]{ST}, we have a contradiction on the minimality of $d$. 
	\end{proof}
	\begin{corollary} Let $R$ be a Gorenstein ring and let $M$ be a two-periodic $R$ module. Let $N$ be a finite $R$-module such that $M$ and $N$ are Tor-independent over $R.$ Then the pair $(M,N)$ satisfies the depth formula. 
	\end{corollary} 
	\begin{proof}
		On Gorenstein rings, any finitely generated $R$-module has finite Gorenstein dimension. The result follows from \Cref{lemma_depth}. 
	\end{proof}
	\bibliographystyle{abbrv}
	\bibliography{ref}
\end{document}

