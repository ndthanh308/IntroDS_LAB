\section{Methods}
\label{sec:methods}
\vspace*{-0.25\baselineskip}

Given our parallel language-based and item-based preferences %
and ratings of 40 items per rater, %
we compare a variety of methods to answer our research questions.  We present the traditional baselines using either item- or language-based preferences, then novel LLM approaches, using items only, language only, or a combination of items and language.

\subsection{Baselines}

To leverage the item and language preferences elicited in Phase 1, we evaluate CF methods as well as a language-based baseline previously found particularly effective \cite{dacrema:2019:progress,Balog:2019:SIGIR}.\footnote{Notably \citet{dacrema:2019:progress} observe that the neural methods do not outperform these baselines.}
Most baseline item-based CF methods use the default configuration in MyMediaLite~\cite{Gantner:2011:MyMediaLite}, including {\bf MostPopular}: ranking items by the number of ratings in the dataset, {\bf Item-kNN}: Item-based k-Nearest Neighbours \cite{Sarwar:2001:ICF}, {\bf WR-MF}: Weighted Regularized Matrix Factorization, a regularized version of singular value decomposition \cite{Hu:2008:CFI}, and {\bf BPR-SLIM}: a Sparse Linear Method (SLIM) that learns a sparse weighting vector over items rated, via a regularized optimization approach \cite{Ning:2011:SSL, Rendle:2009:BBP}. We also compare against our own implementation of the more recent state-of-the-art item-based {\bf EASE} recommender \cite{ease}. %
As a language-based baseline, we compare against {\bf BM25-Fusion}, %
described in Section~\ref{sec:phase2}.
Finally, we also evaluate a random ordering of items in the rater's pool ({\bf Random}) to calibrate against this uninformed baseline.

\subsection{Prompting Methods}

We experiment with a variety of prompting strategies using a variant of the PaLM model (62 billion parameters in size, trained over 1.4 trillion tokens) \cite{chowdhery2022palm}, that we denote moving forward as simply LLM.
Notationally, we assume $t$ is the specific target rater for the recommendation, whereas $r$ denotes a generic rater.  All prompts are presented in two parts: a prefix followed by a suffix which is always the name of the item (movie) to be scored for the target user, denoted as $\langle \mathit{item}^{t}_{*} \rangle$.  The score is computed as the log likelihood of the suffix and is used to rank all candidate item recommendations.\footnote{The full target string scored is the movie name followed by the end-of-string token, which mitigates a potential bias of penalizing longer movie names.} As such, we can evaluate the score given by the LLM to every item in our target set of 40 items collected in Phase 2 of the data collection.

Given this notation, we devise {\bf Completion}, {\bf Zero-shot}, and {\bf Few-shot} prompt templates for the case of {\bf Items only}, {\bf Language only}, and combined {\bf Language+Items} defined as follows:

\subsubsection{Items only}

The completion approach is analogous to that used for the P5 model~\cite{p5}, except that we leverage a pretrained LLM in place of a custom-trained transformer.  The remaining approaches are devised in this work:

\begin{itemize}
    \item {\bf Completion:} $\mathit{item}^{t,1}_{+}$, $\mathit{item}^{t,2}_{+}$, $\mathit{item}^{t,3}_{+}$, $\mathit{item}^{t,4}_{+}$, $\mathit{item}^{t,5}_{+}$,  $\langle \mathit{item}^{t}_{*} \rangle$
    \item {\bf Zero-shot:} I like the following movies: $\mathit{item}^{t,1}_{+}$, $\mathit{item}^{t,2}_{+}$, $\mathit{item}^{t,3}_{+}$, $\mathit{item}^{t,4}_{+}$, $\mathit{item}^{t,5}_{+}$. Then I would also like $\langle \mathit{item}^{t}_{*} \rangle$
    \item {\bf Few-shot ($k$):}
    \begin{tabular}{ll}
    \centeredcell{Repeat $r \in \{1,\ldots,k\}$} \bigg\{ & 
    \centeredcell{User Movie Preferences: $\mathit{item}^{r,1}_{+}$, $\mathit{item}^{r,2}_{+}$, $\mathit{item}^{r,3}_{+}$, $\mathit{item}^{r,4}_{+}$\\
    Additional User Movie Preference: $\mathit{item}^{r,5}_{+}$}
    \end{tabular}\\
    User Movie Preferences: $\mathit{item}^{t,1}_{+}$, $\mathit{item}^{t,2}_{+}$, $\mathit{item}^{t,3}_{+}$, $\mathit{item}^{t,4}_{+}$, $\mathit{item}^{t,5}_{+}$\\
    Additional User Movie Preference: $\langle \mathit{item}^{t}_{*} \rangle$
\end{itemize}

\subsubsection{Language only}
\begin{itemize}
    \item {\bf Completion:} $\mathit{desc}^t_{+}$ $\langle \mathit{item}^{t}_{*} \rangle$
    \item {\bf Zero-shot:} I describe the movies I like as follows: $\mathit{desc}^t_{+}$. Then I would also like $\langle \mathit{item}^{t}_{*} \rangle$
    \item {\bf Few-shot ($k$):}
    \begin{tabular}{ll}
    \centeredcell{Repeat $r \in \{1,\ldots,k\}$} \bigg\{ & 
    \centeredcell{ 
        User Description: $\mathit{desc}^r_{+}$\\
        User Movie Preferences: $\mathit{item}^{r,1}_{+}$, $\mathit{item}^{r,2}_{+}$, $\mathit{item}^{r,3}_{+}$, $\mathit{item}^{r,4}_{+}$, $\mathit{item}^{r,5}_{+}$}
    \end{tabular}\\
    User Description: $\mathit{desc}^t_{+}$\\
    User Movie Preferences: $\langle \mathit{item}^{t}_{*} \rangle$
\end{itemize}

\subsubsection{Language + item}
\begin{itemize}
    \item {\bf Completion:} $\mathit{desc}^t_{+}$ $\mathit{item}^{t,1}_{+}$, $\mathit{item}^{t,2}_{+}$, $\mathit{item}^{t,3}_{+}$, $\mathit{item}^{t,4}_{+}$, $\mathit{item}^{t,5}_{+}$, $\langle \mathit{item}^{t}_{*} \rangle$
    \item {\bf Zero-shot:} I describe the movies I like as follows: $\mathit{desc}^t_{+}$.  I like the following movies: $\mathit{item}^{t,1}_{+}$, $\mathit{item}^{t,2}_{+}$, $\mathit{item}^{t,3}_{+}$, $\mathit{item}^{t,4}_{+}$, $\mathit{item}^{t,5}_{+}$. Then I would also like $\langle \mathit{item}^{t}_{*} \rangle$
    \item {\bf Few-shot ($k$):} 
    \begin{tabular}{ll}
        \centeredcell{Repeat $r \in \{1,\ldots,k\}$} \Bigg\{ &
        \centeredcell{ 
        User Description: $\mathit{desc}^r_{+}$\\
        User Movie Preferences: $\mathit{item}^{r,1}_{+}$, $\mathit{item}^{r,2}_{+}$, $\mathit{item}^{r,3}_{+}$, $\mathit{item}^{r,4}_{+}$\\
        Additional User Movie Preference: $\mathit{item}^{r,5}_{+}$}
    \end{tabular}\\
    User Description: $\mathit{desc}^t_{+}$\\
    User Movie Preferences: $\mathit{item}^{t,1}_{+}$, $\mathit{item}^{t,2}_{+}$, $\mathit{item}^{t,3}_{+}$, $\mathit{item}^{t,4}_{+}$, $\mathit{item}^{t,5}_{+}$\\
    Additional User Movie Preference: $\langle \mathit{item}^{t}_{*} \rangle$
\end{itemize}

\subsubsection{Negative Language Variants}

For the zero-shot cases, we also experimented with negative language variants that inserted the sentences ``I dislike the following movies: $\mathit{item}^{t,1}_{-}$, $\mathit{item}^{t,2}_{-}$, $\mathit{item}^{t,3}_{-}$, $\mathit{item}^{t,4}_{-}$, $\mathit{item}^{t,5}_{-}$'' for {\bf Item} prompts and ``I describe the movies I dislike as follows: $\mathit{desc}^t_{-}$'' for {\bf Language} prompts after their positive counterparts in the prompts labeled {\bf Pos+Neg}.  %



