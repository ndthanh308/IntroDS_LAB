\documentclass[manuscript]{acmart}

\copyrightyear{2023}
\acmYear{2023}
\setcopyright{rightsretained}
\acmConference[RecSys '23]{Seventeenth ACM Conference on Recommender Systems}{September 18--22, 2023}{Singapore, Singapore}
\acmBooktitle{Seventeenth ACM Conference on Recommender Systems (RecSys '23), September 18--22, 2023, Singapore, Singapore}\acmDOI{10.1145/3604915.3608845}
\acmISBN{979-8-4007-0241-9/23/09}

\begin{CCSXML}
<ccs2012>
   <concept>
       <concept_id>10002951.10003317.10003347.10003350</concept_id>
       <concept_desc>Information systems~Recommender systems</concept_desc>
       <concept_significance>500</concept_significance>
       </concept>
 </ccs2012>
\end{CCSXML}

\ccsdesc[500]{Information systems~Recommender systems}

\keywords{recommendation; transparency; scrutability; natural language}

\usepackage{booktabs} 
\usepackage{color}
\usepackage{arydshln} %

\newcommand{\centeredcell}[1]{\begin{tabular}{l} #1 \end{tabular}}


\definecolor{cadmiumgreen}{rgb}{0.0, 0.42, 0.24}


\acmSubmissionID{582}

\begin{document}

\title[LLMs are Competitive Near Cold-start Recommenders]{Large Language Models are Competitive Near Cold-start Recommenders for Language- and Item-based Preferences}

\author{Scott Sanner}
\email{ssanner@mie.utoronto.ca}
\affiliation{%
  \institution{University of Toronto}
  \city{Toronto}
  \country{Canada}
}
\authornote{Work done while on sabbatical at Google.}
\author{Krisztian Balog}
\email{krisztianb@google.com}
\affiliation{%
  \institution{Google}
  \city{Stavanger}
  \country{Norway}
}
\author{Filip Radlinski}
\email{filiprad@google.com}
\affiliation{%
  \institution{Google}
  \city{London}
  \country{United Kingdom}
}
\author{Ben Wedin}
\email{wedin@google.com}
\affiliation{%
  \institution{Google}
  \city{Cambridge, MA}
  \country{United States}
}
\author{Lucas Dixon}
\email{ldixon@google.com}
\affiliation{%
  \institution{Google}
  \city{Paris}
  \country{France}
}




\begin{abstract}
Traditional recommender systems leverage users' item preference history to recommend novel content that users may like.  However, modern 
dialog interfaces that allow users to express language-based preferences offer a fundamentally different modality for preference input.  Inspired by recent successes of prompting paradigms for large language models (LLMs), we study their use for making recommendations from both item-based and language-based preferences in comparison to state-of-the-art item-based collaborative filtering (CF) methods.
To support this investigation, we collect a new dataset consisting of both item-based and language-based preferences elicited from users along with their ratings on a variety of (biased) recommended items and (unbiased) random items.  
Among numerous experimental results, we find that %
LLMs provide competitive recommendation performance for \emph{pure language-based preferences} (no item preferences) in the near cold-start case 
in comparison to item-based CF methods,
despite having no supervised training for this specific task (zero-shot) or only a few labels (few-shot). %
This is particularly promising as language-based preference representations are more explainable and scrutable than item-based or vector-based representations.




\end{abstract}


\maketitle


% Figure environment removed



The understanding of 3D scene geometry is essential for many down-stream applications.  In robotics, it allows for accurate manipulation and motion planning considering the surrounding environment.  In the field of augmented reality, it allows for better mapping and rendering to bridge the virtual world to the real world.  With smartphones and robots that are equipped with high quality depth sensors, the task of 3D scene reconstruction is becoming feasible in these domains. 
%
These depth sensors allow for accurate reconstruction of the observed parts of the scene. However, to reconstruct the unseen parts, we must use prior information conditioned on the observed information. The missing information in the input image combined with the diversity in shapes, sizes, and depth distribution of the household objects presents a major challenge for scene reconstruction in-the-wild. 
%
In this paper, we study this problem in a general setting, where the goal is to reconstruct a complex scene with multiple novel objects, given only one RGB-D image of the scene.  
%



We present our method Rotate-Inpaint-Complete (\ours{}), which predicts both the 3D geometry and the texture of the unseen parts of the scene in the input image by leveraging the inpainting capabilities of large visual-language models.
%
Given an RGB-D image of a scene, first we generate novel views (RGB and depth images) by rotating and then projecting the input scene. Then we use a surface-aware masking method to select regions in the image to allow us to inpaint utilizing the powerful 2D inpainting capabilities of \dalle{}~\cite{ramesh2022hierarchical} for exposing the potential object geometry not visible in the input image. 
%
Finally, we optimize the depth images using the input depth values and occlusion boundaries and normals estimated from the inpainted images. These inpainted and completed novel RGB-D views provide the reconstructed scene geometry as a fused pointcloud with associated textures.
To mitigate the object hallucination and spatial inconsistency of predictions from \dalle{}, we integrate algorithmic features such as filtering inpainting outputs and enforcing consistency across viewpoints into our method that play a crucial role for generalizable, yet accurate and robust scene reconstruction.

%-------------------------------------------------------

We demonstrate our method on cluttered scenes with unseen household objects and categories. Through a series of rigorous quantitative experiments, we show that our approach outperforms baseline methods in settings where no training data is available.

% \subsection{Statement of Contributions}
In short, the contributions of this paper can be summarized as follows. \textit{i)} We present an integrated approach for scene completion of unseen objects under occlusion and clutter, by solving the problem through novel view inpainting and 2D to 3D scene lifting. \textit{ii)} We develop a method for selectively inpainting regions in the novel views of the input scene that enables synthesis of consistent 2D geometry. \textit{iii)} We train a 2D to 3D lifting method on the YCB-V~\cite{xiang2018posecnn} dataset and demonstrate the generalization capability on novel household objects and categories which is crucial for maintaining the generalization capability of our integrated scene reconstruction method.


\begin{table}[t]
\caption{Types and percentages of charts composed of lines, circles, pies, arcs, rectangles, and other marks in the Beagle dataset~\cite{battle2018beagle}.}
    \centering
    {\small
    \setlength\tabcolsep{1.5pt}
    \begin{tabular}{lp{0.725\linewidth}r}
    \toprule
     Mark   & Chart  & Percentage\\
     \midrule
      Rectangle & bar chart~(histogram), grouped bar chart, stacked bar chart, diverging bar chart (pyramid chart), Marimekko chart, heatmap, bullet chart, treemap, waffle chart, waterfall chart, range chart, gantt chart, matrix chart, cartogram, calendar chart & 32.85\%\\
      Line  &  line graph, parallel coordinates, Kagi chart &  30.51\%\\
      Pie &  pie chart, donut chart & 16.50\%\\
      Circle &  scatter plot, bubble plot, dot plot, circle packing & 14.96\%\\
      Others & geographic map, area chart, stream graph, chord chart, hexbin plot, Sankey diagram, Voronoi diagram, word cloud, sunburst chart, boxplot, network diagram, contour plot, radial plot & 5.18\%\\
     \bottomrule
    \end{tabular}
    }
    \label{tbl:lineCircleRect}
\end{table}

\section{Related Work}   
\subsection{\revise{Chart Reuse} Approaches}\label{sec:2.1}
\revise{To create new charts, previous studies on visualization designers' practices~\cite{bigelow_reflections_2014,walny_data_2019} suggest that it is more natural to change existing graphics than to start from scratch.}
Templates are generally recognized as a user-friendly way to create charts, especially for beginners. In traditional template-based systems, templates are created by system developers and they usually suffer from limited expressivity and quantity. 
Previous research thus has investigated how to turn existing visualizations into reusable templates without involving developers. For example, D3 Deconstructor \cite{harper_deconstructing_2014,harper_converting_2017} works on basic charts created using D3.js; iVoLVER \cite{mendez_ivolver_2016} extracts data from charts and updates them with new data; %However, they both support only basic chart types. 
Ivy \cite{mcnutt_integrated_2021} supports turning JSON-based declarative specifications into parameterized templates; 
% Mystique focuses on SVG charts, and does not require the examples to be created using a specification language. 
Chen~\etal~\cite{chen_towards_2020} use deep learning to extract timelines from infographics as templates; and Chartreuse \cite{cui_mixed-initiative_2022} supports reusing infographics bar chart templates. 

Overall, D3 Deconstructor and Chartreuse are the closest work to Mystique. D3 Deconstructor only takes charts created using D3 \cite{bostock_d3_2011}, which have the source data embedded, and Chartreuse primarily works on Microsoft PowerPoint graphics assets. In contrast, Mystique works on visualizations in the general SVG format, does not require access to underlying data, \revise{and supports more advanced layouts}. 
% We elaborate on the differences between Mystique and these tools in~\cref{sec:comparison}.

% Similar to Chartreuse, Mystique adopts a mixed initiative approach to visualization reuse, the primary difference lies in the types of visualizations. Chartreuse primarily works on Microsoft PowerPoint graphics assets and designs with highly customized glyphs. While the variation of glyph design is rich, the layout and structure of the visualizations are simple: Chartreuse only works on infographics bar charts. Mystique handles real-world SVGs with diverse and complex structures such as nested hierarchy and layouts resulting from multi-variate datasets even though it focuses on a rectangle mark.

% Mystique focuses on charts composed of rectangle marks, with an emphasis on layouts and nested structures. 

\subsection{Chart Understanding and Deconstruction}
Making a visualization example reusable requires understanding and deconstructing visualizations.  
Various automated or semi-automated methods have been proposed to detect marks \cite{ying_glyphcreator_2022,chen_towards_2020} as well as axes and legends \cite{shukla_recognition_2008, choudhury_scalable_2016}, classify chart types \cite{savva_revision_2011,shukla_recognition_2008}, and extract data \cite{jung_chartsense_2017,harper_converting_2017,harper_deconstructing_2014,masson2023chartdetective} and visual encodings \cite{poco_reverse-engineering_2017,poco_extracting_2018,harper_deconstructing_2014,harper_converting_2017,cui_mixed-initiative_2022}. Due to the vast space of visualization examples, these methods typically narrow the scope by focusing on specific glyph or chart types. 

\bpstart{Mark Detection}~Many approaches assume that input visualizations are in a raster image format, where object detection is essential. For example, GlyphCreator \cite{ying_glyphcreator_2022} focuses on circular glyphs, and uses deep learning to perform object and bounding box detection. Similarly, visual elements in timeline infographics can be identified using deep learning \cite{chen_towards_2020}. OCR is typically used to recognize text elements \cite{poco_reverse-engineering_2017}. Since our input format is SVG, mark detection is not necessary.

\bpstart{Axis and Legend Detection}~Simple heuristics \cite{shukla_recognition_2008,poco_extracting_2018} or supervised learning \cite{poco_reverse-engineering_2017} can be used to extract
axes and legends.
%Poco and Heer  detected axis and legend byclassifying text role into categories such as legend title, legend label, and axis label using SVM. 
However, these methods can still be error-prone. Since it is relatively easy to indicate where the axes and legends are, some tools expect users to provide such information \cite{poco_extracting_2018}. Mystique uses heuristics to find axes and legends, and provides a user interface for authors to correct potential mistakes.

\bpstart{Data Extraction} ~Previous work also addressed extracting data values from visualization images \cite{savva_revision_2011,jung_chartsense_2017} or vector graphics \cite{harper_converting_2017,harper_deconstructing_2014, masson2023chartdetective}. In Mystique, we demonstrate that a chart can be effectively reused without recovering the original data. Thus, data extraction is not necessary. 

\bpstart{Extraction of Visual Encoding and Spatial Arrangements} Inferring a visual encoding concerns the identification of relevant visual channel, data type, and potentially scale type. For glyphs with regular shapes (e.g., rectangles), visual encodings can be inferred using heuristics by combining information from mark type and axis \cite{poco_reverse-engineering_2017}. For custom glyphs (e.g., those used in infographics), \revise{sometimes the positions are not strictly encoded by data, but instead determined by specific spatial relationships or constraints. In these cases, current approaches usually classify charts into a predefined set of spatial arrangements~\cite{cui_mixed-initiative_2022,chen_towards_2020}. In Mystique, we break down the spatial arrangement of a chart into semantic components to handle more complex layouts.}

% \subsection{Mark and Chart Type Classification}

\bpstart{Chart Type Classification} Previous work also tackled the chart type classification problem. Most approaches are based on a simple chart taxonomy that roughly corresponds to mark types. For example, Revision \cite{savva_revision_2011} classifies chart images into 10 categories using SVM: area, bar, line, map, Pareto, pie, radar, scatter plot, table, and Venn diagram. This taxonomy is used in subsequent neural network-based methods \cite{poco_reverse-engineering_2017,jung_chartsense_2017}. In this work, we decided not to classify mark or chart types because such taxonomies are inadequate to capture the richness and variations of visualization design. Instead, we deconstruct charts into finer-grained semantic components.
\section{Experimental Setup}
\label{sec:expsetup}

To study the relationship between item-based and language-based preferences, and their utility for recommendation, we require a parallel corpus from \emph{the same raters} providing both types of information that is \emph{maximally consistent}. There is a lack of existing parallel corpora of this nature, therefore a key contribution of our work is an experiment design that allows such consistent information to be collected. %
Specifically, we designed a two-phase user study where raters were (1) asked to rate items, \emph{and} to describe their preferences in natural language, then (2) recommendations generated based on both types of preferences were uniformly rated by the raters. Hence we perform our experiments in the movie domain, being frequently used for research as movie recommendation is familiar to numerous user study participants.

A key concern in any parallel corpus of this nature is that people may \emph{say} they like items with particular characteristics, but then consume and positively react to quite different items. For instance, this has been observed where people indicate aspirations (e.g.,~subscribe to particular podcasts) yet actually consume quite different items (e.g.,~listen to others)~\citep{Nazari:2022:WWW}.
In general, it has been observed that intentions (such as intending to choose healthy food) often do not lead to actual behaviors \cite{Verplanken:1999:Good}.
Such disparity between corpora could lead to inaccurate prediction about the utility of particular information for recommendation tasks. As such, one of our key considerations was to maximize consistency.



\subsection{Phase 1: Preference Elicitation}
\label{sec:phase1}
\vspace*{-0.25\baselineskip}

Our preference elicitation design collected natural language descriptions of rater interests both at the start and at the end of a questionnaire.
Specifically, raters were first asked to write short paragraphs describing the sorts of movies they liked, as well as the sorts of movies they disliked (free-form text, minimum 150 characters).  These initial liked (+) and disliked (-) self-descriptions for rater $r$ are respectively denoted as $\mathit{desc}^r_{+}$ and $\mathit{desc}^r_{-}$.

Next, raters were asked to name five example items (here, movies) that they like. This was enabled using an online query auto-completion system (similar to a modern search engine) where the rater could start typing the name of a movie and this was completed to specific (fully illustrated) movies. The auto-completion included the top 10,000 movies ranked according to the number of ratings in the MovieLens 25M dataset~\cite{MovieLens} %
to ensure coverage of even uncommon movies. As raters made choices, these were placed into a list which could then be modified.
Each rater was then asked to repeat this process to select five examples of movies they do not like.  These liked (+) and disliked (-) item selections for rater $r$ and item selection index $j \in \{1,
\ldots,5\}$ are respectively denoted as $\mathit{item}^{r,j}_{+}$ and $\mathit{item}^{r,j}_{\mathit{-}}$.

Finally, raters were shown the five liked movies and asked again to write the short paragraph describing the sorts of movies they liked (which we refer to as the \emph{final description}). The was repeated for the five disliked movies. %





\subsection{Phase 2: Recommendation Feedback Collection}
\label{sec:phase2}
\vspace*{-0.25\baselineskip}

To enable a fair comparison of item-based and language-based recommendation algorithms, a second phase of our user study requested raters to assess the quality of recommendations made by a number of recommender algorithms based on the information collected in Phase 1. In particular, past work has observed that completeness of labels is important to ensure fundamentally different algorithms can be compared reliably \cite{Balog:2019:SIGIR, Kaminskas:2016:Diversity}.

\emph{\textbf{Desiderata for recommender selection:}} We aimed for a mix of item-based, language-based, and unbiased recommendations.  
Hence, we collected user feedback (had they seen it or would they see it, and a 1--5 rating in either case) on a shuffled set of 40 movies (displaying both a thumbnail and a short plot synopsis) drawn from four sample pools:
\begin{itemize}
    \item {\bf SP-RandPop}, an unbiased sample of popular items: 10 randomly selected top popular items (ranked 1-1000 in terms of number of MovieLens ratings);
    \item {\bf SP-RandMidPop}, an unbiased sample of less popular items: 10 randomly selected less popular items (ranked 1001-5000 in terms of number of MovieLens ratings);
    \item {\bf SP-EASE}, personalized item-based recommendations: Top-10 from the strong baseline EASE~\cite{ease} collaborative filtering recommender using hyperparameter $\lambda=5000.0$ tuned on a set of held-out pilot data from 15 users; 
    \item {\bf SP-BM25-Fusion}, personalized language-based recommendations: Top-10 from Sparse Review-based Late Fusion Retrieval that, like \cite{Balog:2021:SIGIR}, computes BM25 match between all item reviews in the Amazon Movie Review corpus (v2)~\cite{zemlyanskiy-etal-2021-docent} and rater's natural language preferences ($\mathit{desc}_+$), ranking items by maximal BM25-scoring review.
     
\end{itemize}
Note that SP-RandPop and SP-RandMidPop have 10 different movies for each rater, and that these are a completely unbiased (as they do not leverage any user information, there can be no preference towards rating items that are more obvious recommendations, or other potential sources of bias).  On the other hand, SP-EASE consists of EASE recommendations (based on the user item preferences), which we also evaluate as a recommender---so there is some bias when using this set. We thus refer to the merged set of SP-RandPop and SP-RandMidPop as an {\bf Unbiased Set} in the analysis, with performance on this set being key to our conclusions. 

\subsection{Design Consequences}
Importantly, to ensure a maximally fair comparison of language-based and item-based approaches, consistency of the two types of preferences was key in our data collection approach. As such, we directly crowd-sourced both types of preferences from raters in sequence, with textual descriptions collected twice---before and after self-selected item ratings. This required control means the amount of data per rater must be small. It is also a realistic amount of preference information that may be required of a recommendation recipient in a near-cold-start conversational setting. As a consequence of the manual effort required, the number of raters recruited also took into consideration the required power of the algorithmic comparison, with a key contributions being to the protocol developed rather than data scale. 

Our approach thus contrasts with alternatives of extracting reviews or preference descriptions in bulk from online content similarly to \cite{Bogers:2017:narrative-driven,mysore2023large} (where preferences do not necessarily capture a person's interests fully) and/or relying on item preferences expressed either explicitly or implicitly over time (during which time preferences may change). %
 %
\section{Methods}
\label{sec:methods}
\vspace*{-0.25\baselineskip}

Given our parallel language-based and item-based preferences %
and ratings of 40 items per rater, %
we compare a variety of methods to answer our research questions.  We present the traditional baselines using either item- or language-based preferences, then novel LLM approaches, using items only, language only, or a combination of items and language.

\subsection{Baselines}

To leverage the item and language preferences elicited in Phase 1, we evaluate CF methods as well as a language-based baseline previously found particularly effective \cite{dacrema:2019:progress,Balog:2019:SIGIR}.\footnote{Notably \citet{dacrema:2019:progress} observe that the neural methods do not outperform these baselines.}
Most baseline item-based CF methods use the default configuration in MyMediaLite~\cite{Gantner:2011:MyMediaLite}, including {\bf MostPopular}: ranking items by the number of ratings in the dataset, {\bf Item-kNN}: Item-based k-Nearest Neighbours \cite{Sarwar:2001:ICF}, {\bf WR-MF}: Weighted Regularized Matrix Factorization, a regularized version of singular value decomposition \cite{Hu:2008:CFI}, and {\bf BPR-SLIM}: a Sparse Linear Method (SLIM) that learns a sparse weighting vector over items rated, via a regularized optimization approach \cite{Ning:2011:SSL, Rendle:2009:BBP}. We also compare against our own implementation of the more recent state-of-the-art item-based {\bf EASE} recommender \cite{ease}. %
As a language-based baseline, we compare against {\bf BM25-Fusion}, %
described in Section~\ref{sec:phase2}.
Finally, we also evaluate a random ordering of items in the rater's pool ({\bf Random}) to calibrate against this uninformed baseline.

\subsection{Prompting Methods}

We experiment with a variety of prompting strategies using a variant of the PaLM model (62 billion parameters in size, trained over 1.4 trillion tokens) \cite{chowdhery2022palm}, that we denote moving forward as simply LLM.
Notationally, we assume $t$ is the specific target rater for the recommendation, whereas $r$ denotes a generic rater.  All prompts are presented in two parts: a prefix followed by a suffix which is always the name of the item (movie) to be scored for the target user, denoted as $\langle \mathit{item}^{t}_{*} \rangle$.  The score is computed as the log likelihood of the suffix and is used to rank all candidate item recommendations.\footnote{The full target string scored is the movie name followed by the end-of-string token, which mitigates a potential bias of penalizing longer movie names.} As such, we can evaluate the score given by the LLM to every item in our target set of 40 items collected in Phase 2 of the data collection.

Given this notation, we devise {\bf Completion}, {\bf Zero-shot}, and {\bf Few-shot} prompt templates for the case of {\bf Items only}, {\bf Language only}, and combined {\bf Language+Items} defined as follows:

\subsubsection{Items only}

The completion approach is analogous to that used for the P5 model~\cite{p5}, except that we leverage a pretrained LLM in place of a custom-trained transformer.  The remaining approaches are devised in this work:

\begin{itemize}
    \item {\bf Completion:} $\mathit{item}^{t,1}_{+}$, $\mathit{item}^{t,2}_{+}$, $\mathit{item}^{t,3}_{+}$, $\mathit{item}^{t,4}_{+}$, $\mathit{item}^{t,5}_{+}$,  $\langle \mathit{item}^{t}_{*} \rangle$
    \item {\bf Zero-shot:} I like the following movies: $\mathit{item}^{t,1}_{+}$, $\mathit{item}^{t,2}_{+}$, $\mathit{item}^{t,3}_{+}$, $\mathit{item}^{t,4}_{+}$, $\mathit{item}^{t,5}_{+}$. Then I would also like $\langle \mathit{item}^{t}_{*} \rangle$
    \item {\bf Few-shot ($k$):}
    \begin{tabular}{ll}
    \centeredcell{Repeat $r \in \{1,\ldots,k\}$} \bigg\{ & 
    \centeredcell{User Movie Preferences: $\mathit{item}^{r,1}_{+}$, $\mathit{item}^{r,2}_{+}$, $\mathit{item}^{r,3}_{+}$, $\mathit{item}^{r,4}_{+}$\\
    Additional User Movie Preference: $\mathit{item}^{r,5}_{+}$}
    \end{tabular}\\
    User Movie Preferences: $\mathit{item}^{t,1}_{+}$, $\mathit{item}^{t,2}_{+}$, $\mathit{item}^{t,3}_{+}$, $\mathit{item}^{t,4}_{+}$, $\mathit{item}^{t,5}_{+}$\\
    Additional User Movie Preference: $\langle \mathit{item}^{t}_{*} \rangle$
\end{itemize}

\subsubsection{Language only}
\begin{itemize}
    \item {\bf Completion:} $\mathit{desc}^t_{+}$ $\langle \mathit{item}^{t}_{*} \rangle$
    \item {\bf Zero-shot:} I describe the movies I like as follows: $\mathit{desc}^t_{+}$. Then I would also like $\langle \mathit{item}^{t}_{*} \rangle$
    \item {\bf Few-shot ($k$):}
    \begin{tabular}{ll}
    \centeredcell{Repeat $r \in \{1,\ldots,k\}$} \bigg\{ & 
    \centeredcell{ 
        User Description: $\mathit{desc}^r_{+}$\\
        User Movie Preferences: $\mathit{item}^{r,1}_{+}$, $\mathit{item}^{r,2}_{+}$, $\mathit{item}^{r,3}_{+}$, $\mathit{item}^{r,4}_{+}$, $\mathit{item}^{r,5}_{+}$}
    \end{tabular}\\
    User Description: $\mathit{desc}^t_{+}$\\
    User Movie Preferences: $\langle \mathit{item}^{t}_{*} \rangle$
\end{itemize}

\subsubsection{Language + item}
\begin{itemize}
    \item {\bf Completion:} $\mathit{desc}^t_{+}$ $\mathit{item}^{t,1}_{+}$, $\mathit{item}^{t,2}_{+}$, $\mathit{item}^{t,3}_{+}$, $\mathit{item}^{t,4}_{+}$, $\mathit{item}^{t,5}_{+}$, $\langle \mathit{item}^{t}_{*} \rangle$
    \item {\bf Zero-shot:} I describe the movies I like as follows: $\mathit{desc}^t_{+}$.  I like the following movies: $\mathit{item}^{t,1}_{+}$, $\mathit{item}^{t,2}_{+}$, $\mathit{item}^{t,3}_{+}$, $\mathit{item}^{t,4}_{+}$, $\mathit{item}^{t,5}_{+}$. Then I would also like $\langle \mathit{item}^{t}_{*} \rangle$
    \item {\bf Few-shot ($k$):} 
    \begin{tabular}{ll}
        \centeredcell{Repeat $r \in \{1,\ldots,k\}$} \Bigg\{ &
        \centeredcell{ 
        User Description: $\mathit{desc}^r_{+}$\\
        User Movie Preferences: $\mathit{item}^{r,1}_{+}$, $\mathit{item}^{r,2}_{+}$, $\mathit{item}^{r,3}_{+}$, $\mathit{item}^{r,4}_{+}$\\
        Additional User Movie Preference: $\mathit{item}^{r,5}_{+}$}
    \end{tabular}\\
    User Description: $\mathit{desc}^t_{+}$\\
    User Movie Preferences: $\mathit{item}^{t,1}_{+}$, $\mathit{item}^{t,2}_{+}$, $\mathit{item}^{t,3}_{+}$, $\mathit{item}^{t,4}_{+}$, $\mathit{item}^{t,5}_{+}$\\
    Additional User Movie Preference: $\langle \mathit{item}^{t}_{*} \rangle$
\end{itemize}

\subsubsection{Negative Language Variants}

For the zero-shot cases, we also experimented with negative language variants that inserted the sentences ``I dislike the following movies: $\mathit{item}^{t,1}_{-}$, $\mathit{item}^{t,2}_{-}$, $\mathit{item}^{t,3}_{-}$, $\mathit{item}^{t,4}_{-}$, $\mathit{item}^{t,5}_{-}$'' for {\bf Item} prompts and ``I describe the movies I dislike as follows: $\mathit{desc}^t_{-}$'' for {\bf Language} prompts after their positive counterparts in the prompts labeled {\bf Pos+Neg}.  %




\section{Results}
\label{results_section}


To answer our research questions and confirm or reject our hypotheses, we conduct rigorous statistical and qualitative analyses. We measure appropriate reliance based on metrics defined in previous work: Relative AI reliance (RAIR) and relative self-reliance (RSR)~\cite{schemmer2023appropriate}. By doing so, we answer \textbf{RQ} \ref{rq1} and \textbf{RQ} \ref{rq2} in \Cref{section_ar}. We also calculate the participant's task accuracy before and after receiving AI and XAI advice to answer \textbf{RQ} \ref{rq3} (see \Cref{hai_performance}). This way, we can determine whether complementary team performance exists~\cite{hemmer2021human}. Additionally, based on the new metric, Deception of Reliance (DoR), we measure to what extent explanations deceive humans, answering \textbf{RQ} \ref{rq4} in \Cref{deception_section}. Lastly, we qualitatively analyze open-ended responses through rigorous inductive content analysis in \Cref{qual-res}. For all of our research questions, we look at two different types of explanations: natural language explanations that are focused on specific features present in the image and visual, example-based explanations showing the top three most similar example images from the training set. While our analyses look at both modalities, we do not intend to compare them directly. Therefore, we do not conclude one modality is better or worse than the other.

\subsection{Participant Statistics}
On average, the study takes $24$ minutes to complete.
In order to distinguish experts from non-experts, we perform K-means clustering ($k=2$) based on a principal component analysis with two components for four features from the bird species identification test (part A of \cref{experimentdesign}). These four features represent participants' scores in correctly identifying the family and species of the easy and the difficult bird images. By clustering the $136$ participants into the expert and non-expert group, we end up with $83$ experts and $53$ non-experts. With this clustering, the average bird identification test score (summing up all four scores in the identification test) for non-experts is $38.99\% (STD = 11.42\%)$ while the average test score for experts is $83.84\% (STD = 12.30\%)$\footnote{Participants performance on the bird identification test is shown in \cref{test-scores}, Appendix Section \ref{bird-test-details}.}. Of the $83$ experts, $42$ see example-based explanations, and $41$ see natural language explanations. Of the $53$ non-experts, $25$ see example-based explanations, and $28$ see natural language explanations. In terms of the fields that the $136$ participants represent, $45$ participants have an occupation primarily related to biology, conservation, and/or the environment. $26$ have an occupation primarily related to engineering and/or technology; $30$ are either researchers, students, or affiliated with education in some other way; $24$ have occupations in miscellaneous industries; and $11$ are retired. 


\subsection{Moderating Effects in Imperfect XAI Research Model}
\label{section_ar}

In order to test whether humans' level of expertise and the explanations' assertiveness moderate the relation of the correctness of explanations on humans' appropriate reliance, we conduct several moderation analyses utilizing the process macro model of \citet{hayes2017introduction}. 
An overview of the regression analyses is presented in \Cref{mod_anal}.
% on p. \pageref{mod_anal}.

\begin{table}[htbp!]

\caption{Moderation analyses of the correctness of natural language and example-based explanations on RAIR and RSR with the level of expertise and assertiveness as moderators. The coding of assertiveness used for the moderation analyses is provided. }

\begin{tabular}{P{2cm} P{1cm} P{1cm}}
\multicolumn{3}{c}{Coding of assertiveness} \\

\hline

assertiveness &  Z1 & Z2 \\ \hline \hline

\textit{neutral} & 0 & 0 \\  \hline
\textit{non-assertive} & 1 & 0 \\  \hline
\textit{assertive} & 0 & 1 \\  \hline
\\\\ \end{tabular}


\begin{threeparttable}


\begin{tabular}{m{1.5cm}R{0.9cm} R{0.9cm} R{0.002cm} R{0.9cm} R{0.9cm} R{0.002cm} R{0.9cm} R{0.9cm} R{0.002cm} R{0.9cm} R{0.9cm}} \hline
& \multicolumn{5}{c}{RAIR} && \multicolumn{5}{c}{RSR}\\
\cmidrule{2-6} \cmidrule{8-12}
& \multicolumn{2}{c}{Natural Language} &&  \multicolumn{2}{c}{Example-Based} & & \multicolumn{2}{c}{Natural Language} & & \multicolumn{2}{c}{Example-Based}  \\
\cmidrule{2-3} \cmidrule{5-6} \cmidrule{8-9} \cmidrule{11-12}
& coeff & p&  & coeff & p & & coeff & p&  & coeff & p \\
\hline \hline
const   & 1.26   & .00 & & .43 & .17 & & -17.16   & .98  & & -3.60  & .00    \\\hline
corr   & .57   & .29 & & 1.02 & .04 & & 13.25   & .98  & & -.4.36  & .74    \\\hline
exp   & 2.12  & .00 &&  -1.25 & .00 &  & 15.89   & .98 &  & 3.25  & .00    \\\hline
Z1   & -.46 & .24 &&  .26 & .47 &&  .14   & .26  & & -.09  & .83    \\\hline
Z2   & .00  & 1.00 &&  .00 & 1.00 & & -.31   & .58  & & .09  & .83    \\\hline 
exp x corr   & -1.00  & \best{.05} & & -1.04 & \best{.03} & & -13.33  & .98  & & -.73  & .57    \\\hline
Z1 x corr   & .46  & .44 &&  -.03 & .95 & & -.28   & .71  & & -.45  & .55    \\\hline
Z2 x corr   & .19  & .74 &&  -.16 & .77 & & .90   & .23  & & -.43  & .55   \\\hline
\end{tabular}
    \begin{tablenotes}
        \item[1] \textit{corr} --- \textit{correctness}; \textit{exp} --- \textit{level of expertise}
    \end{tablenotes}
    \end{threeparttable}

\label{mod_anal}

\end{table}


\subsubsection{Participants' level of expertise moderates the effect of the correctness of explanations on RAIR for natural-language explanations}
\label{mod_analysis_nle_rair}
As theoretically developed in Section \ref{theoretical_section}, we model the correctness of explanations as an independent variable. Accordingly, we model RAIR as the dependent variable. To account for the moderation effect of the level of expertise and assertiveness, we examine each variable as a moderator and report the interaction effects with the correctness of explanations. The results of this moderation analysis are shown in \Cref{mod_anal} (a detailed view is shown in \Cref{mod_anal_nle_rair} on p. \pageref{mod_anal_nle_rair} in the \Cref{mod_appendix}).

The moderation analysis shows that the interaction of the level of expertise with the correctness of explanations is significant (coeff $= -1.00$, p-value $= .05$). We observe a negative coefficient. Accordingly, the moderation effect on the relation of correctness on RAIR is higher for non-experts than for experts. \kmedit{In other words, non-experts change their initially incorrect decision to align with the correct AI advice more often than experts do when the natural language explanation is correct.} However, there is no significant effect in the interaction of assertiveness and the correctness of explanations. Thus, we conduct a regression analysis with the moderators as independent variables to evaluate for a direct effect of assertiveness as recommended by \citet{hayes2017introduction} and \citet{warner2012applied}. The results of the regression analysis show that there is no direct effect between assertiveness and RAIR (coeff $= .04$, p-value $= .77$). Thus, we confirm hypotheses \ref{hyp1} and \ref{hyp3} and reject hypothesis \ref{hyp5} for natural language explanations.


\subsubsection{Participants' level of expertise moderates the effect of the correctness of explanations on RAIR for example-based explanations}

Next, we present the moderation analysis of example-based explanations on RAIR. We set up the analysis for example-based explanations the same as the analyses for natural language explanations (see \Cref{mod_analysis_nle_rair}). As seen in \Cref{mod_anal}, there is a significant moderation effect of the level of expertise (coeff $= -1.04$, p-value $= .03$). The negative coefficient signals that this moderation is higher for non-experts than for experts. The correctness of the example-based explanations has a positive coefficient (coeff $= 1.02$, p-value $= .04$), and thus, correct explanations have a positive impact on RAIR. \psedit{Thus, if participants are provided with a correct explanation, they more often correctly follow the AI advice.} \kmedit{Overall, correct explanations result in humans changing their initially incorrect decisions to align with the correct AI advice more often, and this is especially prevalent among non-experts.}

Furthermore, the analysis reveals that assertiveness does not moderate the effect between correctness and RAIR. According to \citet{hayes2017introduction} and \citet{warner2012applied}, we drop the interaction term and conduct a regression analysis with assertiveness set as the independent variable. The result shows that there is no significant direct effect of assertiveness on RAIR (coeff $= -.04$, p-value $= .79$).
Hence, we confirm hypotheses \ref{hyp1} and \ref{hyp3} and reject hypothesis \ref{hyp5} for example-based explanations.

\subsubsection{Participant's level of expertise has a direct effect on RSR for natural language explanations}
\label{mod_analysis_nle_rsr}
In addition to analyzing whether the level of expertise and assertiveness moderate the effect of explanations' correctness on RAIR, we conduct the same analyses for the effect of explanations' correctness on RSR. For RSR, we look at all cases in which the AI prediction is giving incorrect advice (i.e., the prediction is wrong) and the initial human decision is correct \cite{schemmer2023appropriate}. 
% This next sentence seems to be out of place so I'm commenting it out for now.
% As outlined in \Cref{methodology}, correct explanations reflect the scenario in which the explanation is describing the AI predicted class.

% In this subsection, we report the moderation analysis of natural language explanations on RSR. The results can be seen . 
The moderation analysis in \Cref{mod_anal} for the natural language explanation shows that there is no significant effect of correctness on RSR moderated by level of expertise (coeff $= -13.33$, p-value $= .98$) and assertiveness (Z1 x corr.: coeff $= -.28$, p-value $= .71$; Z2 x corr.: coeff $= .90$, p-value $= .23$). 

Thus, we perform a regression analysis with the level of expertise and assertiveness as independent variables and drop the interaction terms. We observe that there is no significant effect of assertiveness on RSR (coeff $= .10$, p-value $= .58$). However, the level of expertise (coeff $= 3.18$, p-value $= .00$) has a significant effect on RSR. 
% \psedit{Thus, expert participants correctly follow the AI advice more often than non-expert participants.} 
\kmedit{With a positive coefficient, this means that experts dismiss incorrect AI advice more than non-experts when shown natural language explanations.}
\psdelete{The trend} \psedit{This can also be seen} in \cref{rsr-rair}\psedit{, which} tells us that experts have a higher RSR than non-experts. 
% This can also be seen in \Cref{rsr-rair}. 
Therefore, we reject hypothesis \ref{hyp2}, and additionally, hypothesis \ref{hyp4} as the level of expertise does not have a moderating role but has a direct effect on RSR for natural language explanations. On top of that, we reject hypothesis \ref{hyp6}.


\subsubsection{The correctness of explanations and participants' level of expertise have a direct effect on RSR for example-based explanations}

Lastly, we report the results of the moderation analysis for example-based explanations on RSR. The analysis is set up the same as it is in \Cref{mod_analysis_nle_rsr} but for the example-based explanations. 
% \Cref{mod_anal_ex_rsr} displays the results of this regression analysis.

We can see in \Cref{mod_anal} that the level of expertise does not significantly moderate the effect of correctness on RSR (coeff = -.73, p-value = .57).
Additionally, there is no significant moderation of assertiveness (Z1 x corr.: coeff $= -.45$, p-value $= .55$; Z2 x corr.: coeff $= -.43$, p-value $= .55$). Thus, we conduct a regression analysis without the interaction terms. The results of this analysis show no direct effect of assertiveness on RSR (coeff $= -.03$, p-value $= .86$). However, there is a significant direct effect of explanations' correctness on RSR (coeff $= -1.40$, p-value $= .00$) and a direct effect of level of expertise on RSR (coeff $= 3.05$, p-value $= 0.00$). \psedit{This means that experts more often correctly override the wrong AI advice and stick to their correct initial decision compared to non-experts for example-based explanations. Additionally, when the explanations are correct, participants more often correctly override wrong AI advice and stick to their correct initial decision.} Hence, experts have a higher RSR than non-experts, which can also be seen in \Cref{rsr-rair}. Moreover, as incorrect explanations have a higher impact on RSR, experts are able to identify false AI advice for incorrect explanations better. This also shows that experts are able to identify incorrect AI advice to a greater extent than non-experts; experts, in this case, rely more heavily on their own judgment.

Thus, we confirm hypothesis \ref{hyp2} and reject hypothesis \ref{hyp4} as the level of expertise does not take in a moderating role but has a direct effect on RSR for example-based explanations. On top of that, we reject hypothesis \ref{hyp6}.
\\
\\
Overall, the moderation analyses reveal that the level of expertise moderates the effect of the correctness of explanations on RAIR for both explanation modalities. Additionally, the analyses show that the level of expertise has a direct effect on RSR for both explanation modalities, and the correctness of explanations has a direct effect on RSR for example-based explanations.


\subsection{Human-AI Team Performance}
\label{hai_performance}
Hemmer et al. argue that interpretability is a key component of human-AI complementarity~\cite{hemmer2021human}. Several previous user studies have failed to show that incorporating XAI into AI systems can lead to CTP~\cite{fok2023search}. However, with a new dimension of XAI advice in \cref{reliance-model}, we can contribute to the current literature by investigating how the correctness of explanations affects CTP. By calculating the participants' performance before and after seeing the AI and XAI advice, we can determine whether CTP exists in the presence of imperfect XAI. 
As the analyses in \Cref{section_ar} reveal, the level of expertise impacts appropriate reliance in terms of RSR and RAIR. Thus, in comparing the human-AI team performance, we distinguish by participants' level of expertise. We use accuracy as the performance metric. \Cref{reliance-counts} presents the performance of AI and humans for each treatment.

% Figure environment removed

The AI's performance is always $50\%$ because the study was designed to show participants six birds that the model correctly classified and six that the model incorrectly classified. In \Cref{reliance-counts}, we see that when experts are paired with the AI, their performance improves by $8.74\%$ for the natural language modality and $9.53\%$ for the example-based modality. When experts are paired with AI, they perform $6.91\%$ better than the AI alone for natural language explanations and $5.36\%$ for example-based explanations.

While experts reach CTP, we do not see this for non-experts. However, we do see that the non-experts greatly improve their performance and nearly match the AI's performance when paired with the AI. Specifically, non-expert participants who see the natural language explanations improve their performance by $39.58\%$ (task accuracy of $45.83\%$), while non-expert participants who see the example-based explanations improve their performance by $34.67\%$ (task accuracy of $43.00\%$) when paired with the AI.

We can separate \cref{reliance-counts} into correct and incorrect explanations. When we only consider cases with correct explanations (\cref{hai-correct} in \cref{hai-appendix}), the non-experts' task accuracy is approximately the same as the AI alone: $48.81\%$ for natural language explanations and $49.33\%$ for example-based explanations. Experts reach CTP in both modalities. When only considering incorrect explanations (\cref{hai-incorrect} in \cref{hai-appendix}), we still see complementary team performance for the experts. However, the non-experts' task accuracy suffers more when shown incorrect explanations. Non-experts' task accuracy for natural language explanations is $42.86\%$ and $36.67\%$ for example-based explanations.

Additionally, we calculate two-sample t-tests to assess whether the trends in \cref{reliance-counts} are significant. The team performance of experts and AI is significantly higher than the team performance of non-experts and AI in both explanation modalities (natural language: p-value $= 0.00$, example-based: p-value $= 0.00$). Furthermore, we also compare the performance for correct and incorrect explanations. Here, we see the same results: experts achieve a significantly higher team performance than non-experts (correct explanations --- natural language: p-value = 0.00, example-based: p-value $= 0.01$; incorrect explanations --- natural language: p-value $= 0.00$, example-based: p-value $= 0.00$). 


\subsection{Deception caused by Imperfect XAI}
\label{deception_section}


In \Cref{rsr-rair}, we compare RAIR to RSR for both levels of expertise and the correctness of explanations. We show this comparison for example-based explanations (the graph on the left side of \Cref{rsr-rair}) and natural language explanations (the graph on the right side of \Cref{rsr-rair}). By measuring RAIR and RSR for incorrect and correct explanations separately, we can calculate the deception caused by imperfect XAI (refer to \cref{DAIR_aor_eq} on p. \pageref{DAIR_aor_eq}). We do not visualize assertiveness since we do not see any significant direct or moderation effects. 

%- for experts in nle: explanation is helping experts to determine why it is actually wrong
% - for example-based: incorrect ones are differing (potentially) but for correct ones its always the same thing so that's why people are switching and not based on the information content
% - we see the same thing for non-experts: the gap for RAIR is way higher 
% - being interpreted as cofnidence score (add this thought to the discussion section for this)

% when experts see nle - they are able to distinguish when the AI prediction is wrong for correct explanations because it is clear that the characteristics identified don't match 

% for ex -- RSR is dropping -- experts are being misled by the visuals for correct explanations, which is just reinforcing the AI advice  --- but showing visual explanations that are incorrect it is obvious to experts for when the AI is wrong
% Even non-experts can tell if it is three of the same birds or not 

% Assertiveness is playing out visually more than natural language explanations (add this point to the discussion section)


% Figure environment removed

The figure shows that experts have a higher RSR than non-experts for both incorrect and correct explanations across both explanation modalities, validating that experts rely more on their own initial decisions when AI advice is given. The most striking result that emerges from the data is that for example-based explanations, we observe that experts have a significantly higher RSR for incorrect explanations (RSR $= 0.57$) than correct explanations (RSR $= 0.29$), resulting in a negative $DoR_{RSR}$ of $-0.28$ (p-value $= 0.00$).
% A negative difference in deception means that correct example-based explanations are more deceptive than incorrect explanations when the AI advice is incorrect. 
As a result, experts are falsely relying on the AI advice when provided with correct example-based explanations\footnote{Note that correct example-based explanations are consistent in showing three images of the predicted class. Incorrect example-based explanations represent three images that do not correspond to the predicted class of the AI. Moreover, the examples shown are not consistent with the bird species displayed in 90\% of the \textit{correct advice, incorrect explanation} cases and in 40\% of the \textit{incorrect advice, incorrect explanation} cases in our study.}. This means that experts are prone to being misled by correct explanations when the AI advice is incorrect.
However, we do not see this trend for natural language explanations. Here, there is a positive $DoR_{RSR}$ of 0.09, which is not significant (p-value $= 0.41$).
For example-based explanations, the $DoR_{RAIR}$ is positive, meaning that experts rightly follow correct AI advice more often when provided with correct explanations than with incorrect explanations. The data shows a weak, significant positive deception of reliance for example-based explanations ($DoR_{RAIR} = 0.16$, p-value $= 0.10$). Similarly to the RSR cases, for the RAIR cases, the experts are provided with three consistent examples for correct explanations that represent the AI's correctly predicted bird species. The incorrectly provided explanations represent three images that can be inconsistent in the bird species. Thus, experts are deceived by such incorrect explanations even though the AI advice is correct.

Non-experts have, in both modalities, a similar $DoR_{RSR}$ indicating no significant difference in their RSR between correct and incorrect explanations. However, non-experts follow the correct AI advice for correct example-based explanations more often than for incorrect example-based explanations (significant with p-value $= 0.01$). For the latter, the three examples can show inconsistent bird specie(s) that are different from the ground truth of the shown image. Thus, the $DoR_{RAIR}$ for non-experts is at $0.26$. Interestingly, for natural language explanations, the incorrect explanations are not misleading as much ($DoR_{RAIR} = 0.03$, not significant, with a p-value $= 0.68$). This means that non-experts are not misled by incorrect explanations in natural language as much as by visual, example-based explanations. In general, non-experts have a higher RAIR than experts.

Overall, participants have a higher $DoR(RAIR, RSR)$  for example-based explanations (experts: $DoR(RAIR, RSR) = 0.32$; non-experts: $DoR(RAIR, RSR) = 0.26$) than for natural language explanations(experts: $DoR(RAIR, RSR) = 0.11$; non-experts: $DoR(RAIR, RSR) = 0.06$). This means that especially the correctness of example-based explanations has an impact on humans' decision-making behavior.

\subsection{Designing for Imperfect XAI} \label{qual-res}

At the end of the bird identification task, we ask participants: ``\textit{Under what circumstances would you prefer assertive (e.g.,
``definitely'', ``clearly'') versus non-assertive (e.g., ``might be'', ``appears to be'') versus neutral explanations
and why?}''. With imperfect XAI existing in human-AI collaborations, it is necessary not only to understand quantitatively how it impacts decision-makers but also qualitatively. Even though we do not observe the level of assertiveness to have a direct effect or a moderation effect on appropriate reliance, it is still valuable to analyze participants' preferences when it comes to the tone of explanations.

% Figure environment removed

Through inductive content analysis of participants' responses to this question, we derive two dimensions that researchers and designers should consider when developing and evaluating imperfect XAI in human-AI collaborations: \textit{AI Behaviors} and the \textit{Impact on Human-AI Teams}. Each dimension is made up of four themes that are derived from concepts that emerge in the responses, shown in \cref{qual-data}. We highlight those themes in bold. $25\%$ ($34$ participants) of the responses either do not provide reasoning for their opinion or do not answer the question such that it could be grouped into one of the eight themes we identify. We provide quotes from participants for each theme to structure the aggregated dimensions and shape our insights on designing for imperfect XAI. 


\subsubsection{Aggregated Dimension: AI Behaviors}

$54$ out of the $136$ participants answer the survey question with comments relating to the first aggregated dimension: AI Behavior. Participants rationalize that the AI's behavior determines when they prefer assertive, non-assertive, and neutral explanations. Within the AI Behavior dimension, four themes emerge from the participants' comments, such as the model's overall performance, whether the model's prediction is correct or not, the model's confidence in individual predictions, and the correctness/quality of the explanation for a given prediction. 

While only $6$ out of the $136$ participants make comments about the \textbf{model's performance}, it still provides interesting insight that should be considered. Instead of looking at the individual prediction level, these participants focus on the global performance of the model. One participant states that if developers find their model ``\textit{... to be 90\% accurate in your testing, use more definite language, but if it’s not there yet, consider making the tone more neutral and put more responsibility with the end user to interpret the field marks ...}''.

Looking at the individual prediction level, $8$ out of $136$ participants comment on the \textbf{correctness of the AI's prediction}. For example, one participant says that ``\textit{... a non-assertive response would be preferred since the AI selections were incorrect ...}'' while another participant says, ``\textit{I would prefer the assertive language to be accompanied by correct identifications}''.

Considering individual predictions on a more granular level, many participants ($31$ out of $136$) make comments related to \textbf{the confidence of the AI's prediction}. These participants comment on how this factor could be used to determine the tone of the explanation. Specifically, participants, ``\textit{... would prefer the level of assertiveness to depend on the level of confidence of the answer given by the AI}''. One participant expands upon that sentiment by specifying when non-assertive versus assertive tones should be used: ``\textit{I would prefer assertive sentences when the probability of the AI model is very high, while I would prefer non-assertive when the probability is very close to other classes of the model}''. 

$9$ out of $136$ participants make comments related to the \textbf{correctness and quality of the AI explanations}. One participant who sees example-based explanations comments on how some of the examples are incorrect and do not show the correct species. They use this specific situation to rationalize when they would prefer the tone of explanations to be assertive versus non-assertive:  
``\textit{I would prefer assertive explanations if the `similar' photos were actually of the correct species and if the explanation confirmed this. Otherwise, non-assertive explanations are more helpful}''. Another participant who makes a comment related to this theme agrees that assertive language should be used, ``\textit{when all of the reference pictures match up and there are no other similar-looking species}''. 
% Another participant shared a similar sentiment on using a non-assertive tone when the explanations are incorrect, ``\textit{I would prefer non-assertive explanations nearly every time because I didn't have confidence in the AI's explanations (and a few were incorrect)}''.
% When all of the reference pictures match up and there are no other similar looking species it should be assertive.

One participant who sees the natural language explanations comments on the quality and detail of the explanation being a factor to use when determining the tone of the explanation, ``\textit{If the bird description [natural language explanation] is very generic, i.e., brown wings, gray body, or yellow beak (traits that correspond to many birds), I'd rather the AI appear more cautious in its judgment and use non-assertive explanations. However, if the bird has some standout characteristic that the AI correctly identifies [through the natural language explanation], i.e., bright yellow body or red-tipped wings, etc., then assertive explanations seem more convincing}''.


\subsubsection{Aggregated Dimension: Impact on Human-AI Team} 

$48$ out of the $136$ participants answer this survey question with comments related to the human-AI team, such as the confidence and knowledge of the decision-maker, impact on the decision-maker, unambiguous input data, and characteristics of the input data. 

$10$ of $136$ prefer the level of assertiveness to align with their \textbf{own confidence level} and knowledge of the domain. For example, one participant says, ``\textit{I would prefer more assertive explanations when I don't feel very sure about my choice}''. Another participant adds that they would prefer assertive explanations if they ``\textit{... didn't know anything about the topic in question ...}''. However, one participant says, ``\textit{I would prefer neutral explanations if I'm unsure of the species and assertive if I'm confident in my identification}''.

% ``\textit{I would prefer assertive answers if I didn't know anything about the topic in question but would be satisfied with non-assertive if the answer was to back up any doubts of my own answer}''.

% ``\textit{In situations where I am not confident in my knowledge at all, I would prefer assertive explanations so that it would help me definitely understand the subject...I would like non-assertive and neutral explanations when I am somewhat confident, and I must make a decision based on my own knowledge mixed with some additional feedback}''.

$15$ out of the $136$ make comments related to the \textbf{impacts on the decision-maker}. Some comments consist of concerns related to being misled and over- or under-relying on the AI, while other comments motivate the benefits of having assertive explanations. For example, one participant states that ``\textit{Assertive words create more security while changing your opinion or trying to gain knowledge}''. Another participant who shares the same sentiment said, ``\textit{I would prefer assertive explanations because it would make me feel more secure about the answer}''.
However, some participants do not share the same sentiment about assertive explanations and pointed out the potential consequences of them: ``\textit{Assertive AI explanations were given for incorrect identifications, which would mislead users}''. Given that potential to be misled, another participant rationalizes they ``\textit{... would prefer non-assertive explanations because I [they] do not fully trust AI with bird ID just yet}''.

$17$ out of the $136$ participants rationalize that the assertiveness of explanations should be based on how \textbf{ambiguous the input} is for a given prediction. For example, one participant brings up the quality of the input image and the difficulty of the bird ID as a way to determine whether explanations should be assertive or not: ``\textit{When it comes to less distinctive IDs like most sparrows, or harder to ID circumstances like winter or females or juveniles, or situations with weird lighting or harder angles it makes sense to use non-assertive.}''. On a similar line of thought, another participant says, ``\textit{I would prefer assertive explanations for birds that have distinctive traits over similar bird species, and non-assertive or neutral explanations for birds that are similar with characteristics that are more difficult to tell apart}''.
% ``\textit{I would prefer assertive if the photo is clear and able to easily recognize most of the markings of the bird.  I would prefer non-assertive if the markings are not clear. The same applies to neutral explanations}''.

% ``Assertive makes most sense when used for very clear examples of distinctive birds, like the Magnolia Warbler or Cerulean Warbler I had in this set. When it comes to less distinctive ID’s like most sparrows, or harder to ID circumstances like winter or females or juveniles, or situations with weird lighting or harder angles it makes sense to use non-assertive. Many beginning and casual birders take ID’s from Merlin to be absolute so we need more birding tools with nuance.'' 

$6$ out of the $136$ participants commented on how the use of assertive explanations helped them realize various \textbf{characteristics of the birds}. For example, one participant values the assertive tone because it is ``\textit{... helpful for pointing out distinctive features that would help ID the bird.}''. They also think that ``\textit{... the non-assertive language was helpful for species that share similar characteristics (aka clear, non-streaked breast) with other species that share that characteristic}''.
\section{Ethical Considerations}

We briefly consider potential ethical considerations.
First, it is important to consider biases in the items recommended. For instance, it would be valuable to study how to measure whether language-driven recommenders exhibit more or less unintended bias than classic recommenders, such as perhaps preferring certain classes of items over others. Our task was constructed as ranking a fixed corpus of items. As such, all items were considered and scored by the model. Overall performance numbers would have suffered had there been a strong bias, although given the size of our experiments, the existence of bias cannot be ruled out. Larger scale studies would be needed to bound any possible biases present.

Additionally, our conclusions are based on the preferences of a relatively small pool of 153 raters. The small scale and restriction to English-only preferences means we cannot assess whether the same results would be obtained in other languages or cultures.

Finally, we note that the preference data was provided by paid contractors. They received their standard contracted wage, which is above the living wage in their country of employment. 

\section{Conclusion}

In this paper, we collected a dataset containing both item-based and language-based preferences for raters along with their ratings of an independent set of item recommendations.  Leveraging a variety of prompting strategies in large language models (LLMs), this dataset allowed us to fairly and quantitatively compare the efficacy of recommendation from pure item- or language-based preferences as well as their combination.
In our experimental results, we find that zero-shot and few-shot strategies 
in LLMs provide remarkably competitive in recommendation performance for \emph{pure language-based preferences} (no item preferences) in the near cold-start case in comparison to item-based collaborative filtering methods.
In particular, despite being general-purpose, LLMs perform competitively with fully supervised item-based CF methods when leveraging either item-based or language-based preferences. %
Finally, we observe that this LLM-based recommendation approach provides a competitive near cold-start recommender system based on an explainable and scrutable language-based preference representation, thus providing a path forward for effective and novel LLM-based recommenders using language-based preferences.



\bibliographystyle{ACM-Reference-Format}
\bibliography{references}

\end{document}
