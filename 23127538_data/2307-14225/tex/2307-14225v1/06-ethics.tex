\section{Ethical Considerations}

We briefly consider potential ethical considerations.
First, it is important to consider biases in the items recommended. For instance, it would be valuable to study how to measure whether language-driven recommenders exhibit more or less unintended bias than classic recommenders, such as perhaps preferring certain classes of items over others. Our task was constructed as ranking a fixed corpus of items. As such, all items were considered and scored by the model. Overall performance numbers would have suffered had there been a strong bias, although given the size of our experiments, the existence of bias cannot be ruled out. Larger scale studies would be needed to bound any possible biases present.

Additionally, our conclusions are based on the preferences of a relatively small pool of 153 raters. The small scale and restriction to English-only preferences means we cannot assess whether the same results would be obtained in other languages or cultures.

Finally, we note that the preference data was provided by paid contractors. They received their standard contracted wage, which is above the living wage in their country of employment. 
