%% ****** Start of file apstemplate.tex ****** %
%%
%%
%%   This file is part of the APS files in the REVTeX 4.2 distribution.
%%   Version 4.2a of REVTeX, January, 2015
%%
%%
%%   Copyright (c) 2015 The American Physical Society.
%%
%%   See the REVTeX 4 README file for restrictions and more information.
%%
%
% This is a template for producing manuscripts for use with REVTEX 4.2
% Copy this file to another name and then work on that file.
% That way, you always have this original template file to use.
%
% Group addresses by affiliation; use superscriptaddress for long
% author lists, or if there are many overlapping affiliations.
% For Phys. Rev. appearance, change preprint to twocolumn.
% Choose pra, prb, prc, prd, pre, prl, prstab, prstper, or rmp for journal
%  Add 'draft' option to mark overfull boxes with black boxes
%  Add 'showkeys' option to make keywords appear
\documentclass[aps,prl,reprint,superscriptaddress,amsmath,amssymb,longbibliography]{revtex4-2}
%\documentclass[aps,prl,preprint,superscriptaddress]{revtex4-2}
%\documentclass[aps,prl,reprint,groupedaddress]{revtex4-2}

% You should use BibTeX and apsrev.bst for references
% Choosing a journal automatically selects the correct APS
% BibTeX style file (bst file), so only uncomment the line
% below if necessary.
%\bibliographystyle{apsrev4-2}


\newcommand{\be}{\begin{equation}}
\newcommand{\ee}{\end{equation}}
\newcommand{\ba}{\begin{align}}
\newcommand{\eda}{\end{align}}
\newcommand{\nn}{\nonumber}
%underline sottolineato
%\usepackage[utf8]{inputenc}

%%\usepackage{multicol}
\usepackage{bbm}
%\usepackage{amsmath}
\usepackage{todonotes}\topmargin -1.2cm

%\usepackage{subcaption}
\usepackage[export]{adjustbox}
\usepackage{graphicx}
\usepackage{epsfig}
\usepackage{braket}
%\usepackage{amssymb}
%\usepackage[font=small,labelfont=bf]{caption}
%\usepackage{caption}
\usepackage{capt-of}
\newcommand{\om}{\omega}
\newcommand{\omi}{\omega^{-1}}
\newcommand{\ppm}{\pi^{\pm}}
\newcommand{\pmp}{\pi^{\mp}}
\newcommand{\mpm}{\\cM^{\pm}}
\newcommand{\mmp}{\\cM^{\mp}}
\newcommand{\hp}{^{+}}
\newcommand{\hm}{^{-}}
\newcommand{\hpm}{^{\pm}}
\newcommand{\hmp}{^{\mp}}
\newcommand{\tp}{\tilde{p}}
\newcommand{\uu}{_{1}}
\newcommand{\ud}{_{2}}
\newcommand{\ua}{_{a}}
\newcommand{\ub}{_{b}}
\newcommand{\us}{_{s}}
\newcommand{\2}{^{2}}
\newcommand{\cN}{\mathcal{N}}
\newcommand{\cM}{\mathcal{M}}
\newcommand{\cK}{\mathcal{K}}
\newcommand{\cJ}{\mathcal{J}}
\newcommand{\cP}{\mathcal{P}}
\newcommand{\cA}{\mathcal{A}}
\newcommand{\cB}{\mathcal{B}}
\newcommand{\ul}{_{\lambda}}
\newcommand{\hl}{^{\lambda}}
 \newcommand{\up}{_{p}}
  \newcommand{\ut}{_{t}}
 \newcommand{\upp}{_{\pi}}
  \newcommand{\np}{\slash{p}}
    \newcommand{\nk}{\slash{k}}
%    \newcommand{\er}{\eqref{#1} }
  \def\er{\eqref}
  \newcommand{\bel}[1]{\be\label{#1}}
  \newcommand{\bal}[1]{\ba\label{#1}}
  \newcommand{\dv}{\text{d}}
  \newcommand{\tr}{\text{tr}}
  \newcommand{\qs}{\sqrt{s}}
  \newcommand{\wpp}{\widehat{p}^{\, \prime}}
  \newcommand{\vpp}{\vec{p}^{\, \prime}}
  \newcommand{\wip}{\widehat{p}}
  \newcommand{\wik}{\widehat{k}}
  \newcommand{\esk}{\,\rule[-5pt]{0.4pt}{12pt}\,{}}
  
  \renewcommand\slash[1]{\not \! #1}
  \newcommand{\tnp}{\slash{\tilde{p}}}
  
 \newcommand{\bp}{\slash{p}}
 \newcommand{\bk}{\slash{k}}
 \newcommand{\bl}{\not{l}}
 \newcommand{\nldt}[1]{\overset{\approx}{\cN}      \ul{\phantom{\big{|}}}^{\!\!\!\!\!\!(#1)}}
  
  
\usepackage{hyperref}

% For EqNumber 1.1 , 1.2 ....
%TODO \numberwithin{equation}{section}
%\numberwithin{figure}{section}
%\numberwithin{table}{section}   
\usepackage[noabbrev]{cleveref}    

%\bibliographystyle{utphys}
%\bibliographystyle{apsrev4-2}

%\allowdisplaybreaks
%\underline{off-shell}

\newcommand{\PL}{\color[rgb]{0,0,1}}


\begin{document}

% Use the \preprint command to place your local institutional report
% number in the upper righthand corner of the title page in preprint mode.
% Multiple \preprint commands are allowed.
% Use the 'preprintnumbers' class option to override journal defaults
% to display numbers if necessary
%\preprint{}

%Title of paper
\title{Soft-Photon Theorem for Pion-Proton Scattering: Next to Leading Term}
%Soft-photon theorem for pion-proton scattering: the next to leading term

% repeat the \author .. \affiliation  etc. as needed
% \email, \thanks, \homepage, \altaffiliation all apply to the current
% author. Explanatory text should go in the []'s, actual e-mail
% address or url should go in the {}'s for \email and \homepage.
% Please use the appropriate macro foreach each type of information

\author{Piotr Lebiedowicz}
%\orcid{0000-0003-1963-6263}
\email{Piotr.Lebiedowicz@ifj.edu.pl}
\affiliation{Institute of Nuclear Physics Polish Academy of Sciences, 
Radzikowskiego 152, PL-31342 Krak{\'o}w, Poland}

\author{Otto Nachtmann}
\email{O.Nachtmann@thphys.uni-heidelberg.de}
\affiliation{Institut f\"ur Theoretische Physik, Universit\"at Heidelberg,
Philosophenweg 16, D-69120 Heidelberg, Germany}

\author{Antoni Szczurek}
%\orcid{0000-0001-5247-8442}
\email{Antoni.Szczurek@ifj.edu.pl}
\affiliation{Institute of Nuclear Physics Polish Academy of Sciences, 
Radzikowskiego 152, PL-31342 Krak{\'o}w, Poland}
\affiliation{College of Natural Sciences, 
Institute of Physics, University of Rzesz{\'o}w, 
Pigonia 1, PL-35310 Rzesz{\'o}w, Poland.}

%\date{\today}

\begin{abstract}
We investigate the photon emission in pion-proton scattering
in the soft-photon limit where the photon energy $\omega \to 0$.
The expansion of the $\pi^{\pm} p \to \pi^{\pm} p \gamma$ amplitudes,
satisfying the energy-momentum relations,
to the orders $\omega^{-1}$ and $\omega^{0}$ is derived.
We show that these terms can be expressed completely
in terms of the on-shell amplitudes for $\pi^{\pm} p \to \pi^{\pm} p$
and their partial derivatives with respect to $s$ and $t$.
The term of order $\omega^{-1}$ is standard, 
while our term of order $\omega^{0}$ is new.
The structure term which is non singular for $\omega \to 0$ 
is determined to the order $\omega^{0}$ from the gauge-invariance constraint
using the generalized Ward identities for pions and the proton.
\end{abstract}

% insert suggested keywords - APS authors don't need to do this
%\keywords{}

%\maketitle must follow title, authors, abstract, and keywords
\maketitle



% body of paper here - Use proper section commands
% References should be done using the \cite, \ref, and \label commands
%%\section{}
% Put \label in argument of \section for cross-referencing
%\section{\label{}}
%%\subsection{}
%%\subsubsection{}

% If in two-column mode, this environment will change to single-column
% format so that long equations can be displayed. Use
% sparingly.
%\begin{widetext}
% put long equation here
%\end{widetext}

% figures should be put into the text as floats.
% Use the graphics or graphicx packages (distributed with LaTeX2e)
% and the \includegraphics macro defined in those packages.
% See the LaTeX Graphics Companion by Michel Goosens, Sebastian Rahtz,
% and Frank Mittelbach for instance.
%
% Here is an example of the general form of a figure:
% Fill in the caption in the braces of the \caption{} command. Put the label
% that you will use with \ref{} command in the braces of the \label{} command.
% Use the figure* environment if the figure should span across the
% entire page. There is no need to do explicit centering.

% % Figure environment removed

% Surround figure environment with turnpage environment for landscape
% figure
% \begin{turnpage}
% % Figure environment removed
% \end{turnpage}

% tables should appear as floats within the text
%
% Here is an example of the general form of a table:
% Fill in the caption in the braces of the \caption{} command. Put the label
% that you will use with \ref{} command in the braces of the \label{} command.
% Insert the column specifiers (l, r, c, d, etc.) in the empty braces of the
% \begin{tabular}{} command.
% The ruledtabular enviroment adds doubled rules to table and sets a
% reasonable default table settings.
% Use the table* environment to get a full-width table in two-column
% Add \usepackage{longtable} and the longtable (or longtable*}
% environment for nicely formatted long tables. Or use the the [H]
% placement option to break a long table (with less control than 
% in longtable).
% \begin{table}%[H] add [H] placement to break table across pages
% \caption{\label{}}
% \begin{ruledtabular}
% \begin{tabular}{}
% Lines of table here ending with \\
% \end{tabular}
% \end{ruledtabular}
% \end{table}

% Surround table environment with turnpage environment for landscape
% table
% \begin{turnpage}
% \begin{table}
% \caption{\label{}}
% \begin{ruledtabular}
% \begin{tabular}{}
% \end{tabular}
% \end{ruledtabular}
% \end{table}
% \end{turnpage}

%\section{Introduction\label{sec:Introduction}}
\textit{Introduction.}---In this Letter we discuss the production of soft photons in $\pi p $ scattering, that is, the reactions 
\begin{align}
\label{1}
\ppm \,(p\ua)+ p\,(p\ub ,\lambda\ub )&\to \ppm\,(p\uu)+p\,(p\ud ,\lambda\ud )\,, \\
\label{2}
\ppm \,(p\ua)+ p\,(p\ub ,\lambda\ub )&\to \ppm\,(p'\uu)+p\,(p'\ud ,\lambda'\ud )+\gamma\,(k,\varepsilon)\,.
\end{align}
Here $p\ua, p\ub, p\uu, p\ud, p'\uu, p'\ud$ and $k$ are the momenta of the particles,
$\lambda\ub, \lambda\ud, \lambda'\ud$ are the spin indices, 
and $\varepsilon$ is the polarization vector of the photon.
Let $\omega=k^{0}$ be the photon energy in the overall c.m. system. We are interested in soft-photon production, 
$\omega\to 0$.

In a seminal paper F.E. Low \cite{1} derived the theorem that the leading term for $\omega\to 0$ in the soft-photon-production amplitudes is proportional to $\omega^{-1}$ and comes from the emission of photons from the external particles of the reaction. In \cite{1} this was shown explicitly for the scattering of a charged scalar on an uncharged scalar particle. The result was generalized to reactions with an arbitrary number of external particles in \cite{2}. In the following soft-photon production was studied by many authors and there is in general agreement on the $\omega^{-1}$ term in the amplitudes; see e.g. \cite{3,4,5,6,7,8,9,10,11}.

In \cite{1} also an expression for the next to leading term, of order $\omega^{0}$, is given for the scattering of scalars. In our study of soft-photon production in $\pi^{-}\pi^{0}$ scattering we calculated the next to leading term and found a different result \cite{12}. We analyzed this and found that in the calculation of the $\omega^{0}$ term in \cite{1} energy-momentum conservation was not taken into account correctly. We also pointed out problems in the results for the $\omega^{0}$ terms presented in \cite{4,6,8,9,10}. Our own results for the terms of order $\omega^{-1} $ and $\omega^{0}$ for the reaction $\pi^{-}\pi^{0}\to\pi^{-}\pi^{0}\gamma$ 
are given in (3.27), (3.28), and (A1) of \cite{12}.
In our calculations we respect energy-momentum conservation at all stages. 

Coming now to the reactions \er{1} and \er{2} it turned out that the evaluation of the $\omega^{0}$ term for the amplitude of~\er{2} involved a lengthy and complex analysis. Therefore, we present in this Letter the basic ingredients of the calculation and the results. All details can be found in the companion paper \cite{13}.

%%%%%%%%%%%%%%%%%%%%%%%%%%%%%%%%%%%%%%%%%%%%%%%%%%%%%%%%%%%%%%%%%%%%%%%%

\textit{Outline of the calculation for $\pi p \to \pi p \gamma$}.---We use the framework of QCD and treat electromagnetism to lowest relevant order. In QCD we have the symmetries: parity ($P$), charge conjugation ($C$), and time reversal ($T$).
These give us restrictions for the propagators, vertices, and amplitudes. Furthermore, we use the generalized Ward identities for pions and the proton \cite{14,15} and the Landau conditions for determining the singularities in amplitudes \cite{16,17}. 
All our results are derived using only these rigorous methods. 

Consider now the reactions \er{1} where energy-momentum conservation reads 
\bel{3}
p\ua+p\ub=p\uu+p\ud\,.
\ee
Thus, there are only three independent momenta which we choose as
\bal{4}
p\us &=p\ua + p\ub = p\uu +p\ud\,,\nn\\
p\ut &=p\ua - p\uu = p\ud -p\ub\,,\nn\\
p_{u} &=p\ua - p\ud = p\uu - p\ub\,.
\end{align}
We have
\bel{5}
s=p\us\2\;,\quad t=p\ut\2 \;,\quad u=p_{u}\2\,.
\ee
The amplitude for \er{1} has the general structure
\bal{6}
&\braket{\ppm (p\uu), \,  p(p\ud ,\lambda\ud)|{\mathcal T}|\ppm(p\ua), \, p(p\ub ,\lambda\ub)} \nn\\
& \quad = \bar{u}(p\ud , \lambda\ud)\Big{[}A^{(\text{on})\pm}(s,t)\nn\\
& \qquad 
+\frac{1}{2}(\slash{p}_{a}+\slash{p}_{1})
B^{(\text{on})\pm}(s,t)\Big{]}u(p\ub,\lambda\ub )\,,
\end{align}
with invariant functions $A^{(\text{on})\pm}$ and $B^{(\text{on})\pm}$; see e.g. \cite{17}. 
In the calculation of the amplitude for \er{2} we need, however, the off-shell amplitude for $\pi p \to \pi p$ which is much more complicated than \er{6}. Writing for the on- or off-shell momenta of the general reaction \er{1} 
$\tp\ua$, $\tp\ub$, $\tp\uu$, $\tp\ud$ and defining $\tp\us$, $\tp\ut$, $\tp_{u}, \tilde{s}$, $\tilde{t}$ in analogy to \er{4} and \er{5} we find for the off-shell amplitudes 
\begin{align}
\label{7}
&\cM^{(0)\pm}(\tp\uu ,\,\tp\ud ,\, \tp\ua ,\,\tp \ub)\nn\\
& \quad =
\cM_{1}^{\pm}+\tnp_{s}\cM_{2}^{\pm}+\tnp_{t}\cM_{3}^{\pm}+\tnp_{u}\cM_{4}^{\pm}\nn\\
&\qquad +i\sigma_{\mu\nu}\tp_{s}{}^{\mu}\tp_{t}{}^{\nu}\cM_{5}^{\pm}+i\sigma_{\mu\nu}\tp_{s}{}^{\mu}\tp_{u}{}^{\nu}\cM_{6}^{\pm}\nn\\
&\qquad +i\sigma_{\mu\nu}\tp_{t}{}^{\mu}\tp_{u}{}^{\nu}\cM_{7}^{\pm}\nn\\
&\qquad +i\gamma_{\mu}\gamma_{5}\varepsilon^{\mu\nu\rho\sigma}\tp_{s\nu}\tp_{t\rho}\tp_{u\sigma}\cM_{8}^{\pm}\,.
\end{align}
Here the invariant amplitudes $\cM_{j}^{\pm}$ $(j=1, \dots , 8)$
can only depend on $\tilde{s}$, $\tilde{t}$ and the invariant squared masses,
\begin{align}\label{8}
&\cM_{j}^{\pm}=\cM_{j}^{\pm}(\tilde{s},\,\tilde{t},\,m_{1}\2,\, m_{2}\2,\, m\ua\2,\,m\ub\2)\,, \nn\\
& m_{a}^{2}=\tp_{a}^{2}\,,\quad
m_{b}^{2}=\tp_{b}^{2}\,,\quad
m_{1}^{2}=\tp_{1}^{2}\,,\quad
m_{2}^{2}=\tp_{2}^{2}\,.
\end{align}
%\newpage
%\noindent
Specializing \er{7} for the on-shell case we get back \er{6} with $\tilde{s}\to s $, $\ \tilde{t} \to t$, $\ m\ua\2 = m\uu\2 =m\upp\2$, $\ m\ub\2 =m\ud\2 = m\up\2$, and 
\begin{align}
\label{9}
A^{(\text{on})\pm}(s,t)=&\;\cM_{1}^{(\text{on})\pm}
+m_{p}\cM_{2}^{(\text{on})\pm}
-m_{p}\cM_{4}^{(\text{on})\pm}\nn\\
&+(-s+m_{p}\2 +m_{\pi}\2)\cM_{5}^{(\text{on})\pm}\nn\\
&+(s+t-m_{p}\2 -m_{\pi}\2)\cM_{7}^{(\text{on})\pm}\nn \\
&-m_{p}(2s+t-2m_{p}\2 -2m_{\pi}\2)\cM_{8}^{(\text{on})\pm}
\,,\\
\label{10}
B^{(\text{on})\pm}(s,t)=&\;\cM_{2}^{(\text{on})\pm}+\cM_{4}^{(\text{on})\pm}\nn\\
&+2m_{p}\cM_{5}^{(\text{on})\pm}-2m_{p}\cM_{7}^{(\text{on})\pm}\nn\\
&+(4m_{p}^{2}-t)\cM_{8}^{(\text{on})\pm}\,.
\end{align}
On shell the amplitudes $\cM_{3}^{\pm}$ and $\cM_{6}^{\pm}$ are zero from $C$ and $T$ invariance.

Next we consider the reaction \er{2}. 
Energy-momentum conservation reads here
\bel{11}
p\ua +p\ub = p'\uu + p'\ud + k\,.
\ee
Clearly, for $k \neq 0$ we must have $(p'\uu , p'\ud )\neq (p\uu , p\ud)$; see~\er{3}. 
Therefore, we write 
\bel{12}
p'\uu=p\uu-l\uu\,,\quad p'\ud=p\ud-l\ud\,,
\ee
and determine $l\uu$, $l\ud$ from
\begin{align}\label{13}
l\uu+l\ud&=k\,,\nn\\
(p\uu-l\uu)\2&=m_{\pi}\2\,,\nn\\
(p\ud-l\ud)\2&=m_{p}\2\,.
\end{align}
We are interested in $k\to 0$ and assume, therefore, also small $l_{1,2}\;$.
Then \er{13} is easily solved for $l_{1,2}\;$.
We set in the overall c.m. system of \er{1} and \er{2} with 
$ \widehat{p}\uu=\vec{p}\uu/|\vec{p}\uu|$
\begin{align}
\label{14}
(p_{1,2}^{\mu})&=
\left(\begin{array}{l}
p_{1,2}^{0}\\
\\
\pm |\vec{p}\uu|\,\widehat{p}\uu
\end{array}\right), \quad
(k^{\mu})&=
\left( \begin{array}{l}
k^{0}\\
\\
k_{\parallel}\,\widehat{p}\uu+\vec{k}_{\perp}
\end{array}\right), \nn\\
\nn\\
 k^{0}&=\omega\;,\quad \vec{k}_{\perp}\cdot\widehat{p}\uu=0\,.
\end{align}
We get then
\begin{align}
\label{15}
(l\uu^{\mu})&=\left( \begin{array}{l}
\dfrac{p\ud\cdot k}{\sqrt{s}} \\
 \\
\dfrac{p_{1}^{0}}{|\vec{p}\uu|\,\sqrt{s}}(p\ud\cdot k )\widehat{p}\uu +\vec{l}_{1 \perp}
\end{array}\right),\nn\\
\nn\\
(l\ud^{\mu})&=\left( \begin{array}{l}
\dfrac{p\uu\cdot k}{\sqrt{s}}\\ \\
\vec{k}-\dfrac{p_{1}^{0}}{|\vec{p}\uu|\,\sqrt{s}}(p\ud\cdot k )\widehat{p}\uu -\vec{l}_{1 \perp}
\end{array}\right).
\end{align}
Here $\vec{l}_{1 \perp}\cdot\widehat{p}_{1}=0$ but otherwise $\vec{l}_{1 \perp}$ is free, except that we require it to be of order $\omega$ for $\omega\to 0$.

We have five diagrams for $\pi^{-} p\to \pi^{-} p\gamma$ as shown in Fig.~\ref{fig:1}. 
For $\pi^{+} p\to \pi^{+} p\gamma$ the diagrams are analogous.
% Figure environment removed

Let $\cM\hpm\ul$ be the amplitude without spinors for \er{2}. 
We define a matrix amplitude $\cN\hpm\ul$ by
%\label{3.16} not same
\begin{align}
\label{16}
&\mathcal{N}\hpm\ul(p'\uu, p'\ud, k, p\ua ,p\ub)\nn\\
&\quad =(\slash{p}'\ud +m_{p})\cM\hpm\ul (p'\uu, p'\ud, k, p\ua ,p\ub)(\slash{p}\ub +m_{p})\,.
\end{align}
The ${\mathcal T}$-Matrix element for \er{2} is then
%\bal{4.2} almost same (varepsilon?)
\bal{17}
&\braket{\ppm (p'\uu), \, p(p'\ud , \lambda'\ud) ,\, \gamma (k,\varepsilon)|{\mathcal T}|\ppm (p\ua),\, p(p\ub , \lambda\ub )}\nn\\
&\quad =(\varepsilon^{\lambda})^{*} \frac{1}{(2m\up)\2}
\bar{u}(p'\ud , \lambda'\ud)
\cN\hpm\ul(p'\uu , p'\ud  , k , p\ua ,  p\ub) u(p\ub , \lambda\ub)\,.
\end{align}
The advantage of working with $\cN\hpm\ul$ instead of $\cM\hpm\ul$, sandwiched between spinors, is that we do not have to specify any particular spin basis for the protons. For real photon emission we have $k\2 =0$ in \er{17} and this is what we consider here. In \cite{13} we treat the amplitude $\cN\hpm\ul$ \er{16} also for virtual photons, that is, for $k\2 \neq 0$. Our aim is to derive the expansion of 
$\cN\hpm\ul (p\uu - l\uu, p\ud -l\ud, k, p\ua, p\ub)$ 
from~\er{16} for $\omega\to 0$ and to give the terms of order $\omega^{-1}$ and $\omega^{0}$ explicitly. Note that $l\uu$, $l\ud$, and $k$ are all of order~$\omega$; see~\er{15}. 
Thus, we have to expand $\cN\hpm\ul$ with respect to \underline{all} these momenta. Setting $l\uu = l\ud =0$ and expanding then only in $k$ makes no sense since this violates energy-momentum conservation and leads outside the physical region of the amplitude.

In Figs.~\ref{fig:1}(a)--(d), the combinations of propagator times photon vertex occur for pion and proton. Using the generalized Ward identities we find for the pion 
(see Fig.~\ref{fig:1}(a))
%\bal{B41} not same, but similar
\bal{18}
&\Delta_{F}\Big{[}(p\ua-k)\2\Big{]}\, 
\widehat{\Gamma}\ul^{(\gamma \pi\pi)}(p\ua-k,\, p\ua)\nn\\
&\quad =\frac{(2p\ua -k)\ul}{-2p\ua\cdot k + k\2 +i\varepsilon}+{\mathcal O}(\omega)\,,
\end{align}
and for the proton in Fig.~\ref{fig:1}(c)
%\bal{B81}not same, but similar
\bal{19}
&S_{F}(p\ub -k)\, \widehat{\Gamma}^{(\gamma pp)\mu}(p\ub -k, p\ub )(\bp\ub +m\up)\nn\\
&\quad 
=\frac{\bp\ub +m\up- \! \bk}{-2p\ub \cdot k+k\2+i\varepsilon}\Big{[}\gamma^{\mu}-\frac{i}{2m\up}\sigma^{\mu\nu}k_{\nu}F\ud(0)\Big{]}\nn\\
&\qquad \times ( \bp\ub +m\up)+{\mathcal O}(\omega)\,.
\end{align}
Here $F\ud (0)= \mu\up / \mu_{N} -1$ with $\mu\up$ the magnetic moment of the proton and $\mu_{N}$ the nuclear magneton.
Expressions similar to \er{18} and \er{19} apply for the vertex times propagator terms in Figs.~\ref{fig:1}(b) and \ref{fig:1}(d), respectively. 

We see from \er{18} and \er{19} that for the determination of the amplitude $\cN\hpm\ul$ to the orders $\omega^{-1}$ and $\omega^{0}$ we have to know the off-shell amplitudes for $\ppm p \to \ppm p $ 
in Figs.~\ref{fig:1}(a)--(d) to the orders $\omega^{0}$ and $\omega^{1}$. Thus, we have to use \er{7} and make this expansion for the terms 
$\tnp_{s}$, $\tnp_{t}$, etc. as well as for the amplitudes 
$\cM\uu\hpm , \dots , \cM\hpm_{8}$. 
And~this expansion is \underline{different} for the terms (a)--(d).
Finally, the structure term corresponding to Fig.~\ref{fig:1}(e) 
can be determined from the gauge-invariance constraint 
\bal{20}
k^{\lambda}\cN\ul & = k^{\lambda}\Big{(}\cN\ul^{(a)}+\cN\ul^{(b)}+\cN\ul^{(c)}+\cN\ul^{(d)}+\cN\ul^{(e)}\Big{)}=0\,.
\end{align}

%%%%%%%%%%%%%%%%%%%%%%%%%%%%%%%%%%%%%%%%%%%%%%%%%%%%%%%%%%%%%%%%%5

\textit{Results.}---After a long and rather complicated calculation we find that the result for the amplitudes $\cN\hpm\ul$ \er{16} for $\ppm p \to \ppm p \gamma $ to the orders $\omega^{-1}$ and $\omega^{0}$ can be expressed completely in terms of the on-shell amplitudes $A^ {(\text{on})\pm}(s,t)$ and $B^ {(\text{on})\pm}(s,t)$ for
$\ppm p \to \ppm p$ and their partial derivatives 
with respect to $s$ and $t$, 
%\label{3.41}
\begin{align}
\label{21}
A,_{s}^{\!\!(\text{on})\pm}(s,t)&=
\frac{\partial}{\partial s}A^{(\text{on})\pm}(s,t)\,,\nn\\
A,_{t}^{\!\!(\text{on})\pm}(s,t)&=
\frac{\partial}{\partial t}A^{(\text{on})\pm}(s,t)\,,\nn\\
B,_{s}^{\!\!(\text{on})\pm}(s,t)&=
\frac{\partial}{\partial s}B^{(\text{on})\pm}(s,t)\,,\nn\\
B,_{t}^{\!\!(\text{on})\pm}(s,t)&=
\frac{\partial}{\partial t}B^{(\text{on})\pm}(s,t)\,.
\end{align}
We find for real photon emission
\bel{22}
\cN\ul\hpm = \frac{1}{\omega}\widehat{\cN}\ul^{(0)\pm} +
\widehat{\cN}\ul^{(1)\pm} + {\mathcal O}(\omega)\,,
\ee
where with $j= 0, 1$
\bel{23}
\widehat{\cN}\ul^{(j)\pm} = \widehat{\cN}\ul^{(a+b+e1)(j)\pm} +\widehat{\cN}\ul^{(c+d+e2)(j)\pm}\,, 
\ee
\begin{widetext}
\bal{24}
\widehat{\cN}\ul^{(a+b+e1)(0)\pm}=&\pm e(\bp\ud+m\up)\Big{[}A^{(\text{on})\pm}+\frac{1}{2}(\bp\ua +\bp\uu )B^{(\text{on})\pm}\Big {]}(\bp\ub +m\up)\;
\omega\Big{[}-\frac{p_{a\lambda}}{p\ua\cdot k}+\frac{p_{1\lambda}}{p\uu\cdot k}\Big{]}\,, \\
\label{25} 
\widehat{\cN}\ul^{(a+b+e1)(1)\pm}=
&\pm e(\bp\ud+m\up)\Big{[}A^{(\text{on})\pm}+\frac{1}{2}(\bp\ua +\bp\uu )B^{(\text{on})\pm}\Big {]}(\bp\ub +m\up)\frac{1}{(p\uu\cdot k)\2}\Big{[}p_{1\lambda}(l\uu\cdot k)-l_{1\lambda}(p\uu\cdot k)\Big{]}\nn\\
&\pm e(-\!\not{l}\ud)\Big{[}A^{(\text{on})\pm}+\frac{1}{2}(\bp\ua +\bp\uu )B^{(\text{on})\pm}\Big {]}(\bp\ub +m\up)
\Big{[}-\frac{p_{a\lambda}}{p\ua \cdot k}+\frac{p_{1\lambda}}{p\uu \cdot k}\Big{]}\nn\\
&\pm e(\bp\ud+m\up)\Big{[} \frac{1}{2}(-\!\not{l}\uu)B^{(\text{on})\pm}\Big {]}(\bp\ub +m\up)
\Big{[}-\frac{p_{a\lambda}}{p\ua \cdot k}+\frac{p_{1\lambda}}{p\uu \cdot k}\Big{]}\nn\\
&\pm e (\bp\ud +m\up)\bigg{\lbrace}\Big{[}A^{(\text{on})\pm}_{,s}
+\frac{1}{2}(\bp\ua+\bp\uu)
B,_{s}^{\!\!(\text{on})\pm}
\Big{]}
\Big{[} 2(p\us \cdot k) \frac{p_{a\lambda}}{p\ua\cdot k}-2p_{s\lambda}\Big{]}\nn\\
&+\Big{[}A,_{t}^{\!\!(\text{on})\pm}+\frac{1}{2}(\bp\ua+\bp\uu)B,_{t}^{\!\!(\text{on})\pm}\Big{]}
2(p\ut \cdot l\ud)
\Big{[}
\frac{p_{a\lambda}}{p\ua \cdot k}-\frac{p_{1\lambda}}{p\uu \cdot k}\Big{]}\nn\\
&+B^ {(\text{on})\pm}\Big{[}\frac{1}{2}\bk \Big{(}\frac{p_{a\lambda}}{p\ua \cdot k}+\frac{p_{1\lambda}}{p\uu \cdot k}\Big{)}-\gamma_{\lambda}\Big{]}\bigg{\rbrace}(\bp\ub+m\up)
\,,\\
%\end{align}
\label{26}
\widehat{\cN}\ul^{(c+d+e2)(0)\pm}=&\;
e(\bp\ud+m\up)
\Big{[}A^{(\text{on})\pm}+\frac{1}{2}(\bp\ua +\bp\uu)
       B^{(\text{on})\pm}\Big {]}(\bp\ub +m\up)\;
\omega\Big{[}
-\frac{p_{b\lambda}}{p\ub \cdot k}+\frac{p_{2\lambda}}{p\ud \cdot k}\Big{]}\,, 
\\
\label{27}
\widehat{\cN}\ul^{(c+d+e2)(1)\pm}=&\;
e(\bp\ud+m\up)
\Big{[}A^{(\text{on})\pm}+\frac{1}{2}(\bp\ua +\bp\uu )B^{(\text{on})\pm}\Big {]}(\bp\ub +m\up)
\frac{1}{(p\ud\cdot k)\2}\Big{[}p_{2\lambda} (l\ud \cdot k )-l_{2\lambda}(p\ud \cdot k)\Big{]}\nn\\
&+e(-\!\not{l}\ud)
\Big{[}A^{(\text{on})\pm}+\frac{1}{2}(\bp\ua +\bp\uu )B^{(\text{on})\pm}\Big {]}(\bp\ub +m\up)
\Big{[}
-\frac{p_{b\lambda}}{p\ub \cdot k}+\frac{p_{2\lambda}}{p\ud \cdot k}\Big{]}
\nn\\
&+e(\bp\ud +m\up)
\frac{1}{2}(-\!\not{l}\uu) B^{(\text{on})\pm}(\bp\ub +m\up)\Big{[}
-\frac{p_{b\lambda}}{p\ub \cdot k}+\frac{p_{2\lambda}}{p\ud \cdot k}\Big{]}
\nn\\
&+e(\bp\ud +m\up)
\Big{[}A^ {(\text{on})\pm}_{,s}+\frac{1}{2}(\bp\ua+\bp\uu)B^ {(\text{on})\pm}_{,s}\Big{]}(\bp\ub +m\up)
\Big{[}2(p\us\cdot k)\frac{p_{b\lambda}}{p\ub \cdot k}-2p_{s \lambda}\Big{]}
\nn\\
&+e(\bp\ud +m\up)\Big{[}A^ {(\text{on})\pm}_{,t}+\frac{1}{2}(\bp\ua+\bp\uu)B^ {(\text{on})\pm}_{,t}\Big{]}(\bp\ub +m\up)
2(p\ut \cdot l\uu)\Big{[}-\frac{p_{b\lambda}}{p\ub \cdot k}+\frac{p_{2\lambda}}{p\ud \cdot k}\Big{]}\nn \allowdisplaybreaks\\
&+e(\bp\ud +m\up)
\Big{[}A^ {(\text{on})\pm}+\frac{1}{2}(\bp\ua+\bp\uu)B^ {(\text{on})\pm}\Big{]}
(k\ul -\bk \gamma\ul)(\bp\ub +m\up)\frac{1}{(-2p\ub\cdot k)}\nn \\
&-e\frac{1}{(2p\ud\cdot k)}(\bp\ud +m\up)(k\ul -\gamma\ul \bk)
\Big{[}A^ {(\text{on})\pm}+\frac{1}{2}(\bp\ua+\bp\uu)B^ {(\text{on})\pm}\Big{]}(\bp\ub +m\up)\nn\\
&+e(\bp\ud + m\up)\Big{[}A^ {(\text{on})\pm}+\frac{1}{2}(\bp\ua+\bp\uu)B^ {(\text{on})\pm}\Big{]}
\Big{[}
m\up(k\ul-\bk\gamma\ul)+\big{(}p_{b\lambda}\bk -(p\ub\cdot k )\gamma\ul\big{)}\Big{]}(\bp\ub+m\up) \nn\\
&\quad \times 
\frac{F\ud(0)}{m\up}\frac{1}{(-2p\ub\cdot k)}
%\nn\\ &
-e\frac{F\ud(0)}{m\up}\frac{1}{(2p\ud\cdot k)}(\bp\ud + m\up)\Big{[} m\up (k\ul -\gamma\ul \bk)+\big{(}p_{2\lambda}\bk -(
p\ud \cdot k)\gamma\ul \big{)}\Big{]}\nn\\
&\quad \times \Big{[}A^ {(\text{on})\pm}+\frac{1}{2}(\bp\ua+\bp\uu)B^ {(\text{on})\pm}\Big{]}(\bp\ub +m\up)\,.
\end{align}
\end{widetext}


To conclude: in this Letter we have presented a strict theorem of QCD. Given the amplitudes for $\ppm p \to \ppm p$ scattering the amplitudes for soft-photon production, 
$\ppm p \to \ppm p \gamma$, 
have been calculated exactly 
to the orders $\omega^{-1}$ and $\omega^{0}$. 
These two orders of the expansion of the 
$\ppm p \to \ppm p \gamma$ amplitudes 
are completely determined by the $\ppm p \to \ppm p $ 
on-shell amplitudes. 
For real photon emission the result is given in \er{22}--\er{27}. 
The $\omega^{-1}$ term is standard \cite{1,2}, 
the $\omega^{0}$ term is new. 
The results for soft virtual photon emission and
consequences for cross sections are given in \cite{13}.

%\begin{acknowledgments}
%We thank
%\textit{Acknowledgments.}---
We thank
%The authors are grateful to 
Carlo Ewerz for discussions and useful suggestions.
This work was partially supported by
the Polish National Science Centre under Grant
No. 2018/31/B/ST2/03537.
%\end{acknowledgments}

% Create the reference section using BibTeX:\clearpage
\bibliography{kurz_tsp}

\end{document}
%
% ****** End of file apstemplate.tex ******