\documentclass[a4paper,11pt]{article}
\pdfoutput=1 % if your are submitting a pdflatex (i.e. if you have
             % images in pdf, png or jpg format)

\usepackage{jheppub} % for details on the use of the package, please
                     % see the JHEP-author-manual

\usepackage[T1]{fontenc} % if needed
\usepackage{breqn}
\usepackage{derivative}
\usepackage{amsmath,mathrsfs}
\usepackage{float}
\usepackage{subcaption}
\usepackage{appendix}
\bibliographystyle{unsrtnat}
\usepackage[numbers,sort&compress]{natbib}
\usepackage{notoccite}
\newcommand\apj{jhep}
\title{\boldmath }


\title{Dilute axion stars converting to photons in the Milky Way's magnetic field}

\author[a]{A. Kyriazis}


\affiliation[a]{Department of Physics,\\ University of Florida , \\Gainesville, FL 32611, \\ United States }


\emailAdd{akyriazis@ufl.edu}



\abstract{In this paper we examine the possibility of dilute axion stars converting to photons in the weak, large-scale magnetic field of the Milky Way and show that they can resonate with the surrounding plasma and produce a sizable signal. We consider two possibilities for the plasma: free electrons and HII regions. In the former case, we argue that the frequency of the photons will be too small to be observed even by space-based radio telescopes. In the latter case, their frequency is larger, safely above the solar wind cut-off. We provide an estimate of the flux as a function of the decay constant and show that for $f_{a} < 5 \times 10^{11} \text{GeV}$, the signal will be above the radio emission of the solar system's planets. Finally, we  calculate the time scale of decay of the axion star and demonstrate that back-reaction can be neglected for all physically interesting values of the decay constant, while the minimum time scale of decay is in the order of a few hours. }

\begin{document} 
\maketitle
\flushbottom

\section{Introduction}
\label{sec:intro}
\par The axion, initially proposed as a solution to the strong CP problem \cite{Peccei,Weinberg}, is now one of the most well-motivated candidates of dark matter \cite{Preskill,Sikivie, Willy}. The axions are stable bosons, with large occupation numbers and can re-thermalize through their gravitational interactions forming a Bose-Einstein condensate (BEC) \cite{BEC1,BEC2}. Owing to the large occupation number of the ground state, the BEC condensate has been treated classically as a localised, coherently oscillating clump called an axion star, if the kinetic pressure is balanced by gravity and axiton or oscillon, if it is balanced by self-interactions. Generally, when an axion star is supported only by its self-interactions, such as the cosine potential, it is considered dense with radius $mR \sim 1$ \cite{axion_star1} and also decays through scalar radiation on a time-scale of $10^{3} m^{-1}$ \cite{decay,Wilczek}. 
\par
On the other hand, if both gravity and the leading term in the self-interactions are taken into account, the size is $mR \sim \frac{M_{pl}}{f_{a}}$, where $M_{pl}=\frac{1}{\sqrt{8 \pi G}}= 2.4 \times 10^{18}  \text{GeV}$ is the reduced Planck mass and $f_{a}$ is the axion decay constant \cite{Chavanis, Wilczek}. Since the axion decay constant ranges from $10^{9} \text{GeV} < f_{a} < M_{pl}$ in the case where the Peccei-Quinn symmetry is broken during inflation and $10^{9} \text{GeV}< f_{a} < 10^{11} \text{GeV}$ in the classic axion window \cite{axion_cosmology}, these axion stars can be quite large for a wide range of decay constants. In contrast to dense axion stars, they are also long-lived, which makes them cosmologically very interesting \cite{axion_star2}. However, it has been shown that dilute axion stars become increasingly unstable as $f_{a}$ approaches $M_{pl}$ \cite{planck_decay}, because their binding energy becomes large and relativistic contributions need to be taken into account. Therefore, we will limit our analysis in the range $10^{9} \text{GeV} < f_{a} < 10^{15}  \text{GeV}$.
\par The axion can also interact with electromagnetic fields and several venues have been proposed for their detection \cite{axion_detection}. Of particular interest is the Primakoff effect \cite{Primakoff}, which is the interaction of an axion with a magnetic field to produce a real photon. 
\par
The universe is abundant with magnetic fields and many astrophysical settings have been considered in the literature as possible "laboratories" where the axion to photon conversion could be detected. These range from pulsars \cite{axion_neutron, Bai, transient_radio, Radio_signals,Radio_line}, to white dwarfs \cite{upper_limit_from_axion_photon,white_dwarfs,spectral_distortions}, to AGN's \cite{AGN1,AGN2}, to the galactic magnetic field \cite{turbulent,axion_dm_radio_tele,relat_axions_in_sky,axionic_radiation}. 
\par In this paper, we will consider the possibility of an axion star converting to photons in the Milky Way's magnetic field. The strength of the magnetic field is of the order $1 \mu \text{G}$ and its coherence length is of the order of the galactic scale \cite{B_field_in_milky}. We will confirm that the flux emitted from dense axion stars in this magnetic field is negligible \cite{Bai}. However, there is the possibility of resonant conversion of axion stars to photons, if they find themselves in some region with cold plasma. It has been shown that in that case the emitted power scales as $(mR)^{6}$, enhancing it significantly for dilute axion stars. \cite{dipole, time_dependent}. 
\par
When it comes to the plasma, the average electron density in the Milky Way is of the order of $n_{e} \sim 0.03 \text{cm}^{-3}$, which implies a plasma frequency of the order $\omega_{p}=\sqrt{\frac{4 \pi \alpha n_{e}}{m_{e}}} \sim 10^{-12} \text{eV}$ \cite{radio_astronomy}. In the case of diffuse nebulae consisting of ionized hydrogen, the electron density ranges from 100-1000 $cm^{-3}$, with plasma frequencies in the 100-200 kHz range. \cite{ISM} Because of energy conservation, this will be the frequency of the monochromatic photons that are emitted from the axion star. Since the QCD axion's mass and decay constant must satisfy the equation $m_{a} f_{a} \approx (10^{8} \text{eV})^{2}$, due to the range of decay constant we are considering, our analysis does not cover the QCD axion so we will only consider Axion-Like Particles (ALP's), for which the axion mass and the decay constant are independent \cite{ALP}. We will show that even in the very weak magnetic field of the Milky Way, if the axion mass is close to the plasma frequency, dilute axion stars 
can be efficiently converted to photons and produce a signal that could be detected by a future radio telescope sensitive in kHz frequencies.
\par 
In addition, we will demonstrate that our neglect of back-reaction effects is valid for $f_{a} > 10^{7} \text{GeV}$ and therefore for the entirety of the parameter space we are investigating. Finally, we will place a lower bound in the decay time scale, which will be in the order of a few hours. 
\par To motivate our idea further, we can make an order of magnitude estimate of the axion star number in the Milky Way: let's assume that only 1\% of the Milky Way's dark matter mass, $10^{12} M_{\odot}$,is distributed in axion stars. A dilute axion star has a mass of the order $10 \frac{M_{pl} f_{a}}{m}$. For $m \sim 10^{-12} \text{eV}$ and $f_{a} \sim 10^{13} \text{GeV}$, a typical axion star mass is $10^{-4} M_{\odot}$. These should be distributed over the galactic halo, but since we are only interested in those in the galactic disk, their number is $10^{14} \frac{V_{disk}}{V_{halo}} \sim 10^{11}$, where we used a typical radius of the halo 30 kpc and a disk radius of 15 kpc and height of 300 pc \cite{ANS_binaries}. This is indeed a huge number which makes the study of their resonant conversion in the Milky Way's magnetic field an intriguing possibility. 
 
\par The paper is organised as follows: In section 2, we establish our formalism that describes a dilute axion star. In section 3,  we review the interactions between the axion and electromagnetic fields. In section 4, we outline the derivation of the conversion of an axion star to photons in a constant magnetic field, in the presence of cold plasma, and apply it to the case of a dilute star, while we also give an estimate of the spectral flux density of photons that will arrive at Earth from such an event. In section 5 we estimate the decay time-scale of the axion star and we conclude in section 6 with some comments on the prospects of detection and future venues for research. 

\section{Dilute Axions stars}

\par In this paper we will mainly focus on dilute axion stars with gravitational as well as attractive self interactions. Also, we focus on ALP's for which the axion decay constant $f_{a}$ and the axion mass $m_{a}$ are independent from each other. The action for a scalar field $\phi$ which describes the axion star, coupled to a gravitational potential $\Phi$ with attractive $\lambda \phi^{4}$ interactions is \cite{linear_newt}:
\begin{equation}
    \label{eqn: action}
    S=-\int d^{4}x \sqrt{-g} \left( \frac{1}{2}g^{\mu \nu} \partial_{\mu}\phi \partial{\nu} \phi +\frac{1}{2} m^{2} \phi^{2} - \frac{\lambda}{4!} \phi^{4} \right)
\end{equation}
with metric:

\begin{equation}
    \label{eqn: metric}
    ds^{2}=-(1+2\Phi(\vec{x},t))dt^{2}+d\vec{x} \cdot d\vec{x}
\end{equation}

Throughout this paper we will assume that $\lambda=\mathcal{O}(1) \frac{m^{2}}{f_{a}^{2}}$ and we'll ignore the order 1 constant, since we are interested in order of magnitude estimates. In the non-relativistic limit:
\begin{equation}
    \label{non-relat}
    \phi=\frac{1}{\sqrt{2m}}(\psi e^{-i m t}+\psi^{\star}e^{i m t})
\end{equation}
the equation of motion is the Gross-Pitaevskii equation:
\begin{equation}
    \label{eqn: Gross-Pit}
    i \partial_{t} \psi=-\frac{1}{2m} \nabla^{2} \psi + m \Phi \psi - \frac{\lambda}{8 m^{2}} |\psi|^{2} \psi
\end{equation}
while the potential satisfies the usual Poisson equation:
\begin{equation}
    \label{eqn: Poisson}
    \nabla^{2} \Phi = 4 \pi G m |\psi|^{2}
\end{equation}
 A stationary, spherically symmetric solution to the above equations is given by $\psi= e^{-i \mu t} \chi (r)$, where $\mu$ can be considered as the chemical potential. As a side remark that will be useful later on, we observe that if we insert this Ansatz into equation \ref{non-relat}, we get:
 \begin{equation}
     \label{eqn: Ansatz}
     \phi= \sqrt{\frac{2}{m}} \chi(r) \cos(\omega t) 
 \end{equation}
 where we have identified the frequency $\omega$ with $\omega=\mu+m$ \cite{global_view}. Since we are interested in bound states, it holds that $0<\frac{\omega}{m}<1$. This matches the Ansatz of the scalar field $\phi$ that has been used by other authors to study non-relativistic axion stars \cite{Wilczek},\cite{time_dependent} \cite{dipole}, with the difference of the $\sqrt{\frac{2}{m}}$ factor in front.

% Figure environment removed

% Figure environment removed

 \par 
 To continue, we define the small parameter $\delta= \frac{4 m^{2}}{\lambda M_{pl}^{2}}$. Notice that if we plug in the value of $\lambda$ in terms of the axion mass and decay constant, we get $\delta=4 \frac{f_{a}^{2}}{M^{2}_{pl}}$, which is indeed small for a wide range of physically relevant decay constants \cite{axion_cosmology}. We rescale the wavefunction, the potential and the lengths to find the dimensionless forms of the above equations: 
\begin{equation}
    \label{eqn:rescale}
    \chi(r)=\sqrt{\frac{m}{4 \pi G}} \delta \tilde{\chi}(r), \hspace{0.4 cm} \vec{x}=\frac{\vec{\tilde{x}}}{\sqrt{\delta} m} \hspace{0.4cm}, \Phi=\delta \tilde{\Phi} + \frac{\mu}{m}
\end{equation}
The equations of motion become:
\begin{align}
    \label{eqn: dimensionless eqn}
    \tilde{\nabla}^{2} \tilde{\chi} = 2 \left( \tilde{\Phi} \tilde{\chi} - \tilde{\chi^{3}} \right)
    \\
    \tilde{\nabla}^{2} \tilde{\Phi} = \tilde{\chi^{2}}
\end{align}
They satisfy the boundary conditions $\tilde{\chi}'(0)=0, \tilde{\chi}(\tilde{x} \rightarrow \infty)=0, \tilde{\Phi}' (0) = 0$, while the condition $\Phi(\tilde{x} \rightarrow \infty)=0$ implies that $\tilde{\Phi} (\tilde{x} \rightarrow \infty) = -\frac{\mu}{\delta m} $. These equations can be solved with the shooting method: for a given central amplitude of the scalar field $\tilde{\chi_{0}}$, we vary the central amplitude of the potential $\tilde{\Phi_{0}}$ until we find a solution that satisfies the boundary conditions. As a by-product, we also find the value of the chemical potential \cite{boson_star_sidm_eby}. Having found the solution, we can also compute the rescaled mass of the axion star $\tilde{M}=\frac{4 \pi G m M}{\sqrt{\delta}}$:
\begin{equation}
    \tilde{M} \approx \int d^{3} \tilde{x} \tilde{\chi}^{2}
\end{equation}
as well as the radius that contains $99 \%$ of its mass. The mass-radius graph generally contains two different branches, the dilute and the transition branch \cite{Wilczek,global_view,boson_star_sidm_eby}. We are only interested in the dilute branch whose mass-radius graph is depicted in figure \ref{fig:mass_radius} and we confirm that $M \sim \frac{1}{R}$. We also provide the graph of the radius versus the amplitude in figure \ref{fig:radius_ampl} which confirms a well-known behavior of axion stars: as the amplitude increases, the star becomes more dense. 

\section{Axion Electrodynamics} \label{sec: axion_electro}
We now turn to the coupling of axions to electromagnetism. The Lagrangian of the axion field coupled to electromagnetism in flat space is:
\begin{equation}
    \label{eqn:Lagrangian}
    \mathcal{L}=\frac{1}{2}\partial_{\mu}\partial^{\mu}\phi-\frac{1}{2}m^{2}_{\phi}\varphi^{2}-\frac{1}{4} F_{\mu \nu}F^{\mu \nu}-\frac{g_{a\gamma}}{4}\phi F_{\mu \nu}\tilde{F}^{\mu \nu}
\end{equation}
where $\phi$ is the scalar field of the axion and:
\begin{equation}
    \label{eqn:Tensor}
    F_{\mu \nu}=\partial_{\mu}A_{\nu}-\partial_{\nu}A_{\mu} \,,
    \qquad
    \Tilde{F^{\mu \nu}}=\frac{1}{2}\epsilon^{\mu \nu \rho \sigma} F_{\rho \sigma}
\end{equation}
\par Since the occupation number of the axion field is huge, we may treat its field as a classical field. The equations of motion are:
\begin{subequations}
\label{eqn: Maxwell}
\begin{align}
(\Box + m^{2}_{a}) \varphi = & g_{a\gamma} \bf{E} \cdot \bf{B}
\\
\nabla\cdot \bf{E} & =\rho 
\\
\nabla \times \bf{B} - \pdv{\bf{E}}{t} & = \bf{J}
\\
\nabla \cdot \bf{B} & =0
\\
\nabla \times \bf{E}+\pdv{\bf{B}}{t} & = 0
\end{align}

\end{subequations}
where the charge density $\rho$ and current density $\bf{J}$ are:

\begin{subequations}
    \begin{equation}
    \label{eqn:rho}
    \rho  = -g_{a\gamma}\nabla\varphi \cdot \textbf{B} 
    \end{equation}
    \begin{equation}
    \label{eqn:current}
\textbf{J}   = g_{a \gamma} (\pdv{\varphi}{t} \textbf{B}+\nabla \varphi \times \textbf{E})
\end{equation}
\end{subequations}

\par This is a set of coupled, partial differential equations. To make progress, we follow \cite{dipole} and expand the electromagnetic fields, charge density and current density in the small parameter $g_{a\gamma}\varphi_{0}$:

\begin{subequations}
\begin{align}
\bf{E}=\bf{E}^{(0)}+\bf{E}^{(1)}+...,
\qquad
\bf{B}=\bf{B}^{(0)}+\bf{B}^{(1)}+...
\end{align}
\end{subequations}
where $\bf{E}^{(0)}$ and $\bf{B}^{(0)}$ are the background fields that mix with the axion field. The 2 dynamical Maxwell equations for the first order electromagnetic fields are:

\begin{subequations}
\begin{align}
\nabla \cdot \textbf{E}^{(1)} & = \rho^{(1)},
\\
\nabla \times \textbf{B}^{(1)}-\pdv{\textbf{E}^{(1)}}{t} & = \textbf{J}^{(1)}
\end{align}
\end{subequations}

\par Introducing now the vector and scalar potential through the standard equations $\textbf{B}^{(1)}=\nabla \times \textbf{A}$, $\textbf{E}^{(1)}=-\nabla A^{0} - \pdv{\bf{A}}{t}$, our problem is reduced to calculating $A^{0}$ and $\textbf{A}$ from the equations:
\begin{subequations}
\begin{equation}
\label{eqn:wave_equation_potential}
(-\nabla^{2} + \frac{\partial^{2}}{\partial t^{2}} )A^{0} = \rho^{(1)} \\
\end{equation}
\begin{equation}
\label{eqn:wave_equation_vector}
(-\nabla^{2} + \frac{\partial^{2}}{ \partial t^{2}} )\textbf{A}  = \textbf{J}^{(1)}
\end{equation}
\end{subequations}
We will work in the Lorentz gauge $\partial_{\mu} A^{\mu}=0$. This allows us to express the potential $A^{0}$ in terms of the vector potential $\textbf{A}$, so we only have to solve equation \ref{eqn:wave_equation_vector}.

\section{Galactic Magnetic Field and Electron Density}

\subsection{Magnetic field}
\par The magnetic field of the galaxy that we are considering in this work has two components, a large scale one and a small scale one.
\par The large scale component is coherent on length scales of the order of the galaxy and its strength is typically around $1.5 - 2 \mu G$. It reaches $6 \mu G$ in the solar neighborhood and even $10 \mu G$ towards the galactic center. Its structure also seems to follow the spiral arms \cite{B_field_in_milky}. For the purposes of this discussion, the important point is that it is coherent on scales much larger than the size of the axion star and we will consider a value of $1 \mu G$ for its strength. We will say a few things about the small scale component towards the end, but we will ignore it for the remainder of this paper.

% Figure environment removed

\subsection{Free electron density and nebulae}
\par We will consider two different galactic environments with electron densities that could trigger a resonant conversion of axion stars to photons, the free electron density in the Milky Way and diffuse nebulae with HII regions.

\par Regarding the free electron density in the Milky Way, that is of the order $0.03 \text{cm}^{-3}$. A detailed model of the electron density in the Milky Way is given in \cite{electron_den}. From figure \ref{fig:el_den}, we see that the electron density can be as high as $n_{e}=0.2 cm^{-3}$ in the thin disk, while the Local Arm has relatively low density with $n_{e}=0.0057 cm^{-3}$. 
\par The plasma frequency is given by \cite{radio_astronomy}
\begin{equation}
    \omega_{pl}= 8.97 \text{kHz} \left( \frac{n_{e}}{cm^{-3}} \right)^{1/2} \sim 6 \times 10^{-12} \text{eV}  \left( \frac{n_{e}}{cm^{-3}} \right)^{1/2}
\end{equation}
Given the range of values for the electron density quoted above, the range of plasma frequencies, and therefore axion masses, that we can probe are 0.6 \text{KHz} - 4 \text{kHz}. 
\par 

% Figure environment removed

Unfortunately, any electromagnetic waves coming from space with frequency below 30 MHz are blocked by the ionosphere. One way to observe the low frequencies we are considering here is with lunar or space based telescopes that will not face the issue of the ionosphere \cite{space_radio_Tele}. However, they will still face the issue of the solar wind which will block any waves with frequency below 30 kHz, as can be seen from figure \ref{fig:solar_wind_freq}, taken from \cite{Radio_at_long_waves}. We will have to place radio telescopes in the orbit of Saturn if we want to observe conversions of axion stars to photons from the free electron density of figure \ref{fig:el_den}. We conclude that the free electron density in the Milky Way will not help us detect axion stars in the forseeable future. 
\par
We turn to diffuse and planetary nebulae in the interstellar medium with HII regions, that is, ionized hydrogen and electrons. These are formed by stars with temperatures $T \sim 10^{4} K$ that emit UV photons that can ionize the surrounding hydrogen gas, forming the well-known Str\"{o}mgren radius \cite{Stromgren}. The ionization fraction $x=\frac{n_{e}}{n}$, where n is the number density of protons and neutral hydrogen atoms, is equal to unity and the electron densities are $100-1000 \text{cm}^{-3}$ \cite{ISM,diffuse_neb}. These correspond to plasma frequencies $90-285 \text{kHz} \Rightarrow 6 \times 10^{-11} - 2 \times 10^{-10} \text{eV} $ 
 so the frequency of photons produced by axion stars that find themselves in this environment will be safely above the solar wind cut-off. We have ignored here the contribution of protons to the plasma frequency since their mass is much greater than the electron mass and therefore their contribution to the plasma frequency is suppressed.  
\par We can now turn to the question of whether the conversion of a dilute axion star in the magnetic field and plasma described above can produce a sizable flux of radio photons arriving to Earth. 
\section{Spectral flux density}
To begin, we briefly review the derivation of the emitted flux from an axion star in an external, constant magnetic field in the presence of plasma. Based on the analysis of section \ref{sec: axion_electro}, we need to solve equation \ref{eqn:wave_equation_vector}, with current density given by \ref{eqn:current}. We make a few simplifying assumptions: firstly, we assume zero background electric field and consider the Ansatz of the axion field $\phi(r,t)=\phi_{0} \text{cos}(\omega t) \text{sech}(r/R)$. This choice implies that we will not take back-reaction effects into account, something that we will justify in the next section. The current density is then given by $\textbf{J}=-g_{a \gamma } \text{sin}(\omega t) \text{sech}(r/R) \textbf{B}_{0}$. To solve equation \ref{eqn:wave_equation_vector}, we may employ the Green fuction of the wave equation. The details are analyzed in \cite{time_dependent,dipole, Kyriazis} and we will only give the main result here for the emitted power per solid angle: 
\begin{equation}
        \label{eqn: solid angle }
        \frac{dP}{d \Omega}  = \frac{\pi^{4} (g_{a \gamma} \varphi_{0} \omega ^{2} R^{2})^{2}}{32 k \omega} \left( \frac{\tanh (\pi k R /2)}{\cosh (\pi k R /2)} \right)^{2} |\textbf{B}_{0}|^{2}
\end{equation}
where k is the wavevector $k=\omega \sqrt{1-\frac{\omega_{p}^{2}}{\omega^{2}}}$. We have implicitly assumed that the gyrofrequency $\omega_{B}=\frac{\sqrt{4 \pi \alpha} B_{0}}{m_{e}}$ is much smaller than the frequency of radiation $\omega$, which is true for the values we are considering here: $\omega_{B}=10^{-15} \text{eV} \ll \omega=10^{-10} \text{eV} $. One additional assumption is that the propagation of the photons is perpendicular to the galactic magnetic field, because we are mainly interested in an order of magnitude estimate of the effect. 

% Figure environment removed

% Figure environment removed

\par
We see that when the axion star is far from resonance, the power peaks for sizes $ \omega R \sim 1$ while it is exponentially suppressed if $\omega R \gg 1$. We are interested in studying the power emitted when the star is in resonance, so we take $ \omega \rightarrow \omega_{p} \sim 10^{-10} \text{eV} $. The power becomes:
\begin{equation}
    \frac{dP}{d\Omega}({\omega \rightarrow \omega_{p}}) \approx  \frac{(g_{a \gamma} \phi_{0})^{2}}{128} \frac{(\pi \omega R)^{6}}{\omega^{2}} \left(1-\frac{\omega_{p}^{2}}{\omega^{2}}\right)^{1/2} |\textbf{B}_{0}|^{2}
\end{equation}

Let us assume that this conversion takes place 1kpc away from Earth. We define the spectral flux density of the incoming radiation as $S=\frac{1}{r^{2} \mathcal{B}} \frac{dP}{d \Omega}$ where $\mathcal{B}$ is the Doppler shift of the central frequency and we estimate it as $\mathcal{B} \sim \frac{0.1\omega}{2 \pi}$. For a dense axion star, the value of the spectral flux density is of the order: $S \sim 10^{-21} Jy$, which is indeed negligible. 

\par However, the $(\omega R)^{6}$ term is promising: dilute axion stars with $\omega R \gg 1$ can significantly enhance the flux that arrives at Earth. To find the power emitted from a dilute axion star in resonance, we make the substitution $\phi_{0} \rightarrow \sqrt{\frac{1}{2 \pi G}}  \delta$, which comes from combining equations \ref{eqn:rescale} and \ref{eqn: Ansatz}. Assuming that $g_{a \gamma} \sim \frac{\alpha}{f_{a}}$ and with the approximation $\omega \approx m$, our estimate is:

\begin{equation}
    S \sim 10^{-21} \text{Jy} \frac{1}{\delta^{2}}  \left( \frac{1 kpc}{r} \right)^{2} \left( \frac{10^{-10} \text{eV}}{m} \right)^{3} \left( \frac{B_{0}}{1 \mu G} \right)^{2} \
\end{equation}

where we have also approximated $mR \sim \frac{1}{\sqrt{\delta}}$. We see that since $\delta \ll 1$ for a wide range of decay constants, this flux can be quite huge.
\par
Our plan is to compare this flux density with the fluxes that come from the planets of our solar system, depicted in figure \ref{fig: planets}. From the figure it is evident that in the frequency range 100-300 kHz, the flux we should be most concerned about is the one from Saturn (green line) which reaches approximately $10^{-19} \frac{\text{W}}{\text{Hz} \hspace{0.06cm} \text{m}^{2}}=10^{7} \text{Jy}$. 
\par
The spectral flux density for different axion decay constants and electron density set to $400 \text{cm}^{-3}$ is depicted in figure \ref{fig:flux_f}. The red horizontal line corresponds to the flux coming from Saturn. We see that for axion decay constants approximately smaller than $5 \times 10^{11} \text{GeV}$  the signal is above Saturn's threshold and therefore it could be detected by a space or lunar based radio telescope.

\par
The spectral flux density as a function of the electron density of the galaxy is depicted in figure \ref{fig: flux_el}, for $f_{a}=10^{13} \text{GeV}$. We see that,if the axion star finds itself in a region with electron density 100 $cm^{-3}$, the flux can be more than an order of magnitude stronger compared to a region with density 1000 $cm^{-3}$. 

\section{Decay time scales} \label{sec:decay}
We estimate in this section the time that it will take for the axion star to convert all its mass to photons. The mass of a dilute axion star is of the order of $M \sim 10 \frac{f_{a} M_{pl}}{m}$ \cite{Wilczek}. We assume the star is in resonance with the surrounding plasma. The timescale over which the star will lose the entirety of this mass is roughly given by:
\begin{equation}
    T=\frac{M}{P} \sim 10^{4} \frac{M^{2}_{pl} m \delta^{5/2}}{B_{0}^{2}}
\end{equation}
where we have estimated the emitted power to be $P \sim 10^{-3} \frac{1}{\delta^{2}} \frac{B^{2}_{0}}{m^{2}}$. We are ignoring back-reaction effects here, so in order for this estimation to make sense, we need $T> \frac{2 \pi}{m}$, the decay time scale needs to be longer than the period of the radiation \cite{time_dependent}.Solving for the small parameter $\delta$, we find 
\begin{equation}
    \delta> \left( \frac{10^{-2} B_{0}}{m M_{pl}} \right)^{4/5} \Rightarrow f_{a} > M_{pl} \left( \frac{10^{-2} B_{0}}{m M_{pl}} \right)^{2/5}
\end{equation}
For $B_{0}= 1 \mu \text{G}$ and $m=10^{-10} \text{eV}$, this gives the lower bound  $f_{a} \gtrapprox 10^{7} \text{GeV}$. This tells us that for most of the parameter space, back-reaction can indeed be ignored. For $f_{a}=10^{9} \text{GeV}$, the lower bound on the decay timescale is $T \sim 3 \text{hr}$.  

\section{Conclusions and Outlook}
\par
We have considered the possibility of axion stars converting to photons in the magnetic field of the Milky Way. The high number of axion stars in the galactic disk that we estimated in the Introduction make this a possibility worth considering. We showed that the weak magnetic field of the galaxy is not enough to efficiently convert a dense axion star to photons in vacuum. However, if an axion star is in a plasma and its frequency is close to the plasma frequency, the dependence of the emitted flux on $(mR)^{6}$ implies that a dilute axion star will produce a sizable flux. 

\par Beginning from this observation, we considered two different possibilities of a plasma in the Milky Way, the free electrons and the diffuse nebulae with HII regions. In the former case, we argued that the photons produced will have frequencies far below the solar wind cut-off and we will never be able to observe them with lunar based radio-telescopes. In the latter case, however, the electron density is $10^{4}$ time higher, leading to photons with frequencies $\nu \sim 100 \text{kHz}$ safely above the solar wind threshold and within the target range of future space based radio telescopes \cite{space_radio_Tele,NOIRE}.

\par Our main calculation involved the estimation of the spectral flux density that will arrive at Earth if a dilute axion star resonated with its surrounding plasma and converted its mass to photons. We showed that for axion decay constants $f_{a}< 5 \times 10^{11} \text{GeV}$, the flux is larger than the radio flux emitted from Saturn, which is the dominant one from the solar system's planets in this frequency range. In addition, we demonstrated that there will be a magnitude order increase in the emitted flux if the conversion happens in a region with $n_{e}=100 cm^{-3}$ compared to a region with $n_{e} = 1000 cm^{-3}$. 

\par Finally, we estimated the time scale over which the star will radiate. We demonstrated that back-reaction effects can be ignored for the entirety of the parameter space that we consider in this work and found that an axion star will need at least a few hours to lose all its mass. However, it is an open question whether the star will transition to different configurations as it decays, so that it eventually moves out of resonance. Ref. \cite{axion_backreaction} did this analysis for dense axion stars supported by their self-interactions and found that the axion stars grow in size, their frequency increases and they go out of resonance after a certain time-scale. It is not clear whether the same thing can happen with dilute stars because, to a very good approximation, $\omega \sim m$. An analysis in the vein of \cite{Levkov} could shed some light on this question. 

\par A coulpe of more comments are in order regarding this proposal. Firstly, we have ignored the small scale component of the galactic field that is related to the turbulent Interstellar Medium . This component has a shorter correlation length than the large-scale component we used in this study and its strength is $5.5 \mu G$ \cite{B_field_in_milky}. It has been shown that it can enhance the conversion of diffuse axions to photons by many orders of magnitude and it should be taken into account in future work\cite{relat_axions_in_sky,turbulent}. 

\par Also, we have not considered the distribution of axion stars in the galactic plane, which, to our knowledge, is not known. This makes it difficult to estimate the number of conversions that we could potentially observe. A more detailed study should take the axion star distribution into account, combined with the disribution of HII in the Milky Way, as shown in \cite{Paladini}. This should provide us with an accurate estimation of the frequency of these events.  




\acknowledgments
I would like to thank Yuxin Zhao, Pierre Sikivie and Joshua Eby for useful comments and discussions. This scientific paper was supported by the Onassis Foundation - Scholarship ID: F ZS 031-1/2022-2023.

\bibliography{cit}

\end{document}
