% Please make sure you insert your
% data according to the instructions in PoSauthmanual.pdf
\documentclass[a4paper,11pt]{article}
\usepackage{pos}
\usepackage{overpic}
\usepackage{subfigure}
\usepackage{multirow}
\usepackage{booktabs}
\usepackage{tabularx}

\title{Measurements of charmonia decays from BESIII}
%% \ShortTitle{Short Title for header}

\author*{Han Miao}
\author{on behalf of the BESIII collaboration}

\affiliation{Institute of High Energy Physics,\\
  19B Yuquan Road, Shijingshan District, Beijing 100049, People's Republic of China}

%\affiliation[b]{Department, University,\\
%Street number, City, Country}

\emailAdd{miaohan@ihep.ac.cn}
%\emailAdd{s.author@univ.country}

\abstract{In this talk, recent measurements of charmonium decays of BESIII are presented. Using 448 million $\psi(3686)$ events collected with the BESIII detector, the branching fractions of the decays $\chi_{cJ} \to \phi \phi (J=0,1,2)$ have been measured most precisely, and the polarization parameters of $\chi_{cJ} \to \phi \phi$ have been determined for the first time via a helicity amplitude analysis. Using the same data sample as in the previous study, first evidence of $\eta_c(2S) \to \pi^+ \pi^- \eta$ has been found in the decay sequence $\psi(3686) \to \gamma \eta_c(2S)(\to \pi^+ \pi^- \eta)$. The product of the branching fractions of $\psi(3686) \to \gamma \eta_c(2S)$ and $\eta_c(2S) \to \pi^+ \pi^- \eta$ is reported as well as the individual branching fraction of $\eta_c(2S) \to \pi^+ \pi^- \eta$. The process $e^+ e^- \to \eta J/\psi$ at a center-of-mass energy $3.773~{\rm GeV}$ is observed for the first time. Its Born cross-section is measured, and the branching fraction of $\psi(3770) \to \eta J/\psi$ is determined by a combined fit with the cross sections at other energy points, after considering the interference effect for the first time. Utilizing 2708 million $\psi(3686)$ events collected by the BESIII detector, the decays $\chi_{cJ} \to \Omega^- \bar{\Omega}^+(J=0,1,2)$ have been observed for the first time with high significance via the radiative decays of $\psi(3686) \to \gamma \chi_{cJ}$. The relevant branching fractions have been provided.}

\FullConference{21st Conference on Flavor Physics and CP Violation (FPCP 2023)\\
 29 May - 2 June 2023\\
 Lyon, France\\}

%% \tableofcontents

\begin{document}
\maketitle


\section{Introduction}

Since the discovery of $J/\psi$ in the winter of 1974, heavy quarkonia have always been an ideal field for physicists to study the main properties of quantum chromo-dynamics (QCD)~\cite{Kwong:1987mj, QuarkoniumWorkingGroup:2004kpm, Eichten:2007qx, Brambilla:2010cs, Rosner:2011eg}, among which charmonia, especially, play an important role for understanding the physics in the energy region between perturbative and non-perturbative QCD. Clear spectrum of low-lying charmonium states below the open-charm $D\bar{D}$ threshold has been observed experimentally and predicted theoretically within the potential models, which incorporate a color Coulomb term at short distances and a linear scalar confining term at large distances. Therefore, for a long time, the charmonium system has become the prototypical ``hydrogen atom'' of meson spectroscopy, while the study of charmonium decays to light hadrons has always sufferred from the difficulties of non-perturbative calculation. In history, multiple models or techniques are raised for this issue, of which none describes all the experimental measurements perfectly. The case will be much more complex if considering the decays to light hadrons, thus the experimental results are really essential for the research on charmonium decays.

In the experimental aspect, charmonium states have been studied thoroughly by CLEO-c, BES and other charm or bottom factories in history, and are still charming after nearly 50 years with the new decay channels and new states continuously discovered. As the only facility running in the tau-charm energy region around world now, BEPCII and BESIII dedicate massive investigation, arousing widespread interest for theoretical studies. The following part will be a brief introduction and summary of the recent results on the measurement of charmonium decays at BESIII.

\section{BEPCII and BESIII}

The BESIII detector~\cite{Ablikim:2009aa}, as shown in Figure~\ref{fig:BESIII}, records symmetric $e^+e^-$ collisions provided by the BEPCII storage ring~\cite{Yu:IPAC2016-TUYA01} in the center-of-mass energy range from 2.0 to 4.95~GeV, with a peak luminosity of $1.1 \times 10^{33}\;\text{cm}^{-2}\text{s}^{-1}$ achieved at $\sqrt{s} = 3.77~\text{GeV}$. BESIII has collected large data samples in this energy region~\cite{Ablikim:2019hff, EcmsMea, EventFilter}. The cylindrical core of the BESIII detector covers 93\% of the full solid angle and consists of a helium-based multilayer drift chamber~(MDC), a plastic scintillator time-of-flight system~(TOF), and a CsI(Tl) electromagnetic calorimeter~(EMC), which are all enclosed in a superconducting solenoidal magnet providing a 1.0~T magnetic field. The magnetic field was 0.9~T in 2012, which affects 10.8\% of the total $J/\psi$ data. The solenoid is supported by an octagonal flux-return yoke with resistive plate counter muon identification modules interleaved with steel. The acceptance of charged particles and photons is 93\% over $4\pi$ solid angle. The charged-particle momentum resolution at $1~{\rm GeV}/c$ is $0.5\%$, and the ${\rm d}E/{\rm d}x$ resolution is $6\%$ for electrons from Bhabha scattering. The EMC measures photon energies with a resolution of $2.5\%$ ($5\%$) at $1$~GeV in the barrel (end cap) region. The time resolution in the TOF barrel region is 68~ps, while that in the end cap region is 110~ps. The end cap TOF system was upgraded in 2015 using multigap resistive plate chamber technology, providing a time resolution of 60~ps~\cite{etof1, etof2, etof3}.

% Figure environment removed

\section{Results from BESIII}

\subsection{Helicity amplitude analysis of $\chi_{cJ} \to \phi \phi$~\cite{BESIII:2023zcs}}

In the quark model, the $\chi_{cJ}$ states are identified as $P$-wave triple charmonium states with spin, parity and charge conjugation $J^{++}~(J=0,1,2)$. At leading order, the hadronic decays of $\chi_{cJ}$ are described by annihilations of charm and anti-charm quarks into two gluons and subsequent production of light and/or strange quarks. Early theoretical calculations for exclusive decays of $\chi_{cJ}$ into light hadrons have yielded smaller branching fractions than experimental measurements~\cite{Duncan:1980qd, Jones:1981ff, Anselmino:1992rw}.

Following perturbative QCD calculations~\cite{Zhou:2004mw}, the $\chi_{c1}$ decay rate should be strongly suppressed compared to $\chi_{c0}$ and $\chi_{c2}$, due to the helicity selection rule~\cite{Chernyak:1981zz} and the requirement of identical particle symmetry~\cite{Yang:1950rg}. However, the previous BESIII measurement reported similar branching fractions of $\chi_{cJ} \to \phi \phi$ decays for $\chi_{c0}$, $\chi_{c1}$ and $\chi_{c2}$~\cite{BESII:2011hcd}, namely $\mathcal{B}(\chi_{c0} \to \phi \phi) = (7.8 \pm 0.4 \pm 0.8) \times 10^{-4}$, $\mathcal{B}(\chi_{c1} \to \phi \phi) = (4.1 \pm 0.3 \pm 0.4) \times 10^{-4}$ and $\mathcal{B}(\chi_{c2} \to \phi \phi) = (10.7 \pm 0.4 \pm 1.1) \times 10^{-4}$. Meanwhile, the analysis of $\phi$ polarization will also be a key measurement to probe hadronic-loop effects in the $\chi_{cJ} \to \phi \phi$ decays~\cite{Huang:2021kfm}. Moreover, the ratios of the helicity amplitudes are found to be effective in the discrimination between the proposed models as these ratios are less sensitive to the parameters used in the evaluation of the model prediction~\cite{Zhou:2004mw, Chen:2013gka, Huang:2021kfm}. Table~\ref{tab:ratio_amplitudes} summarizes the helicity-amplitude ratios predicted by the considered theoretical models, where the uncertainties are due to the uncertainties on parameters involved in the calculation. The variable $x$ is defined as the ratio of transverse over the longitudinal polarized helicity amplitudes of the $\phi$ meson in $\chi_{c0}\to\phi\phi$: $x = \left|F^{0}_{1,1}/F^{0}_{0,0}\right|$ and the variables $\omega_{i}$ $(i=1,2,4)$ indicate the ratios of transverse over longitudinal polarized helicity amplitudes of the $\phi$ meson in $\chi_{c2}\to\phi\phi$: $\omega_1 = \left| F^{2}_{0,1}/F^{2}_{0,0}\right|$, $\omega_2 = \left| F^{2}_{1,-1}/F^{2}_{0,0}\right|$, $\omega_4 = \left| F^{2}_{1,1}/F^{2}_{0,0}\right|$, where $\lambda_1$ and $\lambda_2$ are the helicities of the two $\phi$, and $F^{J=0,2}_{\lambda_1,\lambda_2}$ are the helicity amplitudes. The $\chi_{c1}\to\phi\phi$ helicity amplitudes allow to test the validity of the identical particle symmetry: in this context the helicity-amplitude ratios  $u_1=|F^1_{1,0}/F^1_{0,1}|$ and $u_2=|F^1_{1,1}/F^1_{1,0}|$ are expected to be 1 and 0, respectively~\cite{Chen:2013gka}.

\begin{table}[htbp]
    \caption{Numerical predictions of the helicity-amplitude ratios from pQCD~\cite{Zhou:2004mw}, $^3P_0$~\cite{Chen:2013gka} and $D\bar D$ loop models ~\cite{Huang:2021kfm}.}
    \centering
    %\vskip -0.2cm
    \begin{tabular}{l|c|ccc}
%    \toprule
    \hline
    \hline
                Decay channel     &$\chi_{c0} \to \phi\phi$     &\multicolumn{3}{c}{$\chi_{c2} \to \phi\phi$}\\\hline
                Parameter         &$x$                        &$\omega_1$  &$\omega_2$  &$\omega_4$\\
		\hline
%                \midrule
                pQCD              &$0.293 \pm 0.030$            &$0.812 \pm 0.018$     &$1.647 \pm 0.067$      &$0.344 \pm 0.020$\\
                $^3P_0$           &$0.515 \pm 0.029$            &$1.399 \pm 0.580$     &$0.971 \pm 0.275$      &$0.406 \pm 0.017$\\
        $D\bar{D}$ loop   &$0.359\pm0.019$  &$1.285\pm0.017$ &$5.110\pm0.057$ &$0.465\pm0.002$\\
    \hline
	    \hline
%    \bottomrule
    \end{tabular}
    \label{tab:ratio_amplitudes}
\end{table}

The amplitude analysis of $\chi_{cJ} \to \phi \phi$ is performed based on $448.1 \times 10^{6}$ $\psi(3686)$ events. The joint distribution for the sequential decays $e^+ e^- \to \psi(3686) \to \gamma \chi_{cJ},~\chi_{cJ} \to \phi \phi$ and $\phi \to K^+ K^-$ is constructed in the helicity system as shown in Figure~\ref{fig:helicity_system}. The joint amplitude for the sequential process is described as Eq.~(\ref{ampform}) and (\ref{BWform}).

\begin{eqnarray}\label{ampform}
\mathcal{M}(R_i)  &=& {1\over 2}\sum_{M,\lambda_{R},\lambda_{1},\lambda_{2}}
 A_{\lambda_{R},\lambda_{\gamma}}^{1}D_{M,\lambda_{R}-\lambda_{\gamma}}^{1*}(0,\theta_{0},0) F_{\lambda_{1},\lambda_{2}}^{J}D_{\lambda_{R},\lambda_{1}-\lambda_{2}}^{J*}(\phi_{1},\theta_{0},0)\nonumber\\
 &\times&B_{0,0}^{1}D_{\lambda_{1},0}^{1*}(\phi_{2},\theta_{2},0)B_{0,0}^{1}D_{\lambda_{2},0}^{1*}(\phi_{3},\theta_{3},0)BW(m_{\phi\phi},m_{i},\Gamma_{i}),
\end{eqnarray}
with
\begin{eqnarray}\label{BWform}
BW(m_{\phi\phi},m_i, \Gamma_i) = {1 \over m_{\phi\phi}^2-m_i^2 + i m_i \Gamma_i }.
\end{eqnarray}

The partial decay rate of $\psi(3686)$ is given by
\begin{equation}
\mathrm{d}\sigma \propto \frac{1}{2}{\sum}_{M,\lambda_{\gamma}}\left|\sum_{R_i}\mathcal{M}(R_i)\right|^2\mathrm{d}\Phi,
\end{equation}
where $\mathrm{d}\Phi$ is the standard phase space for the decay $\psi(3686)\to\gamma\phi\phi$ with $\phi\to K^+ K^-$. The detailed definitions of the symbols in the above equations can be found in the published paper~\cite{BESIII:2023zcs}.

% Figure environment removed

The potential interference between the $\chi_{c0}$ and non-resonant contributions is considered while the interference of $\chi_{c1}$ and $\chi_{c2}$ with the non-resonant constribution is neglected due to the quite narrow width.

The fit result is shown in Figure~\ref{fig:fitchicj}.

% Figure environment removed

The branching fractions for $\chi_{cJ} \to \phi \phi$ are measured to be
\begin{eqnarray}
\mathcal{B}(\chi_{c0} \rightarrow \phi \phi)&=& (8.48\pm0.26\pm0.27)\times10^{-4} ,\nonumber \\
\mathcal{B}(\chi_{c1} \rightarrow \phi \phi)&=& (4.36\pm0.13\pm0.18)\times10^{-4} , \\
\mathcal{B}(\chi_{c2} \rightarrow \phi \phi)&=& (13.36\pm0.29\pm0.49)\times10^{-4}\nonumber,
\end{eqnarray}
where the first uncertainties are statistical and the second systematic. Comparing these results with BESIII previous measurement~\cite{BESII:2011hcd} and PDG values~\cite{ParticleDataGroup:2022pth}, as reflected in the Table \ref{tab BFsum}, the precision is improved by a factor of about $2$, but the values are greater.

\begin{table}[htbp]
\begin{center}
\caption{ \label{tab BFsum} Comparsion of measured branching fractions (BF).}
%\vskip 0.1cm
\begin{normalsize}
%\begin{tabular}{ m{4cm}<{\centering} | m{3.5cm}<{\centering}  | m{3.5cm}<{\centering}  | m{3.4cm}<{\centering}  }
\begin{tabular}{ c|c|c|c }
\hline
\hline
Decay Mode & BF(2011 BESIII)~\cite{BESII:2011hcd} & BF(this work)& BF(PDG value)~\cite{ParticleDataGroup:2022pth}\\
\hline
Br[$\chi_{c0}$ $\rightarrow$ $\phi \phi$]($\times$$10^{-4}$) &  7.8$\pm$0.4$\pm$0.8 & $8.48\pm0.26\pm0.27$ & 7.7$\pm$0.7\\
Br[$\chi_{c1}$ $\rightarrow$ $\phi \phi$]($\times$$10^{-4}$) &  4.1$\pm$0.3$\pm$0.5 & $4.36\pm0.13\pm0.18$ & 4.2$\pm$0.5\\
Br[$\chi_{c1}$ $\rightarrow$ $\phi \phi$]($\times$$10^{-4}$) &  10.7$\pm$0.4$\pm$1.2 & $13.36\pm0.29\pm0.49$ & 11.2$\pm$1.0\\
\hline\hline
\end{tabular}
\end{normalsize}
\end{center}
\end{table}

The ratios of the amplitude moduli are measured to be
\begin{eqnarray}
\left|F_{1,1}^0\right|/\left|F_{0,0}^0\right| &=& 0.299\pm0.003\pm0.019,
\end{eqnarray}
for $\chi_{c0}\to\phi \phi$, and
\begin{eqnarray}
\left|F_{0,1}^2\right|/\left|F_{0,0}^2\right| &=& 1.265\pm0.054\pm0.079,\\
\left|F_{1,-1}^2\right|/\left|F_{0,0}^2\right| &=& 1.450\pm0.097\pm0.104,\\
\left|F_{1,1}^2\right|/\left|F_{0,0}^2\right| &=& 0.808\pm0.051\pm0.009,
\end{eqnarray}
for $\chi_{c2}\to\phi \phi$, where the first and second uncertainties are statistical and systematic, respectively. Additionally, there is no evidence of identical particle symmetry breaking from the study of $\chi_{c1}\to\phi \phi$.

Figure~\ref{cmptheo} shows a comparison of the measured amplitude ratios to the corresponding theoretical predictions. The measured ratio of amplitude moduli for the $\chi_{c0}$ is consistent with the pQCD prediction of Ref.~\cite{Zhou:2004mw}, since two independent helicity amplitudes of the $\chi_{c0}\to\phi\phi$ decay,  $F_{1,1}^0$ and $F_{0,0}^0$,  follow the helicity selection rule. For the $\chi_{c2}$ decay, the measured ratios of amplitude moduli deviate from the pQCD~\cite{Zhou:2004mw}, $^3P_0$~\cite{Chen:2013gka} and $D\bar {D}$ loop~\cite{Huang:2021kfm} predictions with $\chi^2/\mathrm{ndf}=23.2$, $23.8$, and $155.2$, respectively. The $D\bar D$ loop model can be ruled out due to the large deviation. However, the predictions of other models also differ from the experimental results. In short, all of the above theories use some of the input from the experimental results, thus this measurement can provide more constraints for further developing the models. It could also be a basis for the measurement in the future, as 2.7 billion $\psi(3686)$ events have been accumulated in BESIII~\cite{BESIII:2020nme}.

% Figure environment removed

\subsection{Evidence for the $\eta_c(2S) \to \pi^+ \pi^- \eta$ decay~\cite{BESIII:2022ksv}}

	Until now, the knowledge about $\eta_c(2S)$ is still limited~\cite{Brambilla:2010cs}, suffering from the very soft photon from $\psi(3686) \to \gamma \eta_c(2S)$. The total measured branching fraction of $\eta_c(2S)$ decays is small (less than 5\%) according to PDG~\cite{ParticleDataGroup:2022pth}.

	The decay of charmonium states into light hadrons is believed to be dominated by the annihilation of the $c\bar{c}$ pair into two or three gluons. The so-called ``12\% rule'' states that the ratio of the inclusive branching fractions of light hadron states between $\psi(3686)$ and $J/\psi$ is about 12\%~\cite{Appelquist:1974zd}. Similarly, one would expect a similar ratio of the hadronic branching fractions between $\eta_c(2S)$ and $\eta_c$ due to their analogous decaying dynamics in comparison to $\psi(3686)$ and $J/\psi$. According to Ref.~\cite{Franklin:1983ve}, for many normal light hadronic channel $h$,
\begin{equation}
	\frac{\mathcal{B}(\eta_c(2S) \to h)}{\mathcal{B}(\eta_c \to h)} \approx \frac{\mathcal{B}(\psi(3686) \to h)}{\mathcal{B}(J/\psi \to h)} = 0.128,
\end{equation}
	while there are also theoretical works resulting in a ratio close to one~\cite{Chao:1996}. The measured ratios are mostly greater than 12\% and less than one, except the ones with $p \bar{p}$ final states, so the further measurements of the decays of $\eta_c(2S)$ and $\eta_c$ are of great significance.

	Based on $448.1 \times 10^{6}$ $\psi(3686)$ events collected at BESIII, the decay $\eta_c(2S) \to \pi^+ \pi^- \eta$ is searched for. The final fit result is shown in Figure~\ref{fig:fit_etapipi}.

% Figure environment removed

Evidence for the decay $\eta_c(2S)\to\pi^+\pi^-\eta$ is found for the first time, with a statistical significance of 3.5$\sigma$. The product of the branching fractions is measured to be $Br(\psi(3686)\to\gamma\eta_c(2S))\times Br(\eta_c(2S)\to\pi^+\pi^-\eta)=$  (2.97$\pm$0.81$\pm$0.26)$\times10^{-6}$, where the first uncertainty is statistical and the second systematic. The branching fraction of $\eta_c(2S)$ decaying into $\pi^+ \pi^- \eta$ is measured to be $Br(\eta_c(2S)\to\pi^+\pi^-\eta)=$  $(42.4\pm11.6\pm3.8\pm30.3)\times10^{-4}$, with the first uncertainty being the statistical, the second the systematic uncertainty without taking into account the uncertainty of the branching fraction of $\psi(3686)\to\gamma\eta_c(2S)$. The third one is the systematic uncertainty arising from this branching fraction.

With the branching fraction $Br(\eta_c\to\pi^+\pi^-\eta)=$(1.7$\pm$0.5)\%~\cite{ParticleDataGroup:2022pth}, the ratio of the branching fractions of $\eta_c$ and $\eta_c(2S)$ decaying into $\pi^+ \pi^- \eta$ is calculated to be $\frac{Br(\eta_c(2S)\to\pi^+\pi^-\eta)}{Br(\eta_c\to\pi^+\pi^-\eta)}=0.25\pm0.20$. Combining the ratios of other hadronic decay modes of $\eta_c(2S)$ to $\eta_c$~\cite{ParticleDataGroup:2022pth,BESIII:2022hcv}, the averaged value of all these ratios including this measurement is determined to be 0.30$\pm$0.10 (see Figure~\ref{fig::comp_br}). This ratio agrees neither with the prediction in Ref.~\cite{Franklin:1983ve} nor in Ref.~\cite{Chao:1996}. With about 2.7 billion $\psi(3686)$ events to be accumulated, BESIII will make a further substantial contribution to this field~\cite{BESIII:2020nme}.

% Figure environment removed

\subsection{Observation of $\psi(3770) \to \eta J/\psi$~\cite{BESIII:2022yoo}}

Conventionally, the $\psi(3770)$  has been regarded as the lowest-mass $D$-wave charmonium state above the $D\bar{D}$ threshold, {\it i.e.} a pure $c\bar{c}$ meson in the quark model~\cite{Eichten:1978tg}, while this model cannot explain the measured large non-$D\bar{D}$ decay width of the state~\cite{He:2008xb, ParticleDataGroup:2022pth}. To solve this puzzle, various theoretical models are developed, either by introducing tetra-quark component into the wave function~\cite{Voloshin:2005sd}, or more complicated dynamics such as $2S-1D$ mixing between $\psi(3686)$ and $\psi(3770)$~\cite{Mo:2006cy, Rosner:2004wy, Zhang:2009gy, Ding:1991vu}, and re-scattering mechanism with $D$ mesons~\cite{Guo:2012tj, Wang:2011yh, Liu:2009dr, Guo:2010ak}. Until now, the only well-established non-$D\bar{D}$ channel is $\psi(3770) \to \pi^+ \pi^- J/\psi$~\cite{ParticleDataGroup:2022pth, Eichten:2007qx} so that the further sutdy of other non-$D\bar{D}$ channels is necessary and essential in both experimental and theoretical aspects. In 2005, CLEO studied the decay $\psi(3770) \to \eta J/\psi$ and reported the branching fraction to be $(8.7 \pm 3.3 \pm 2.2) \times 10^{-4}$ at a statistical significance of $3.5\sigma$ without considering the interference~\cite{CLEO:2005zky}. The branching fraction of $\psi(3770) \to \eta J/\psi$ is utilized as an input in theoretical calculations of decay properties not only for conventional charmonium states~\cite{Zhang:2009kr, Li:2013zcr} but also for exotic charmonium-like (also called $XYZ$) states~\cite{Anwar:2016mxo} observed in this energy region.

The signal yield $N_{\rm obs}$ is obtained by fitting to the $M^{\prime}(\gamma \gamma)$ ($M^{\prime}(\gamma \gamma) \equiv M(\gamma \gamma) + M(\mu \mu) - m_{J/\psi}$) distribution as shown in Figure~\ref{fig:sig_yield_eta_jpsi}. Then the Born cross section is calculated by
\begin{equation}
{\sigma^{B}(e^+e^-\rightarrow\eta J/\psi)}= \frac{N^ {\rm obs}}{\mathcal{L}\cdot(1+\delta^{\rm ISR})\cdot(1+\delta^{\rm VP})\cdot\varepsilon\cdot{\cal{B}}r} \,,
\label{eq:born}
\end{equation}
where $\mathcal{L}$ is the integrated luminosity, $(1+\delta^ {\rm ISR})$ is the ISR correction factor~\cite{Kuraev:1985hb}, $(1+\delta^{\rm VP})$ is the vacuum polarization factor taken from QED calculation~\cite{WorkingGrouponRadiativeCorrections:2010bjp}, ${\cal{B}}r$ is the product of the branching fractions of the subsequent decays of intermediate states as given by the PDG~\cite{ParticleDataGroup:2022pth}, and $\varepsilon$ is the detection efficiency. The ISR correction factor is obtained by an iterative method~\cite{Sun:2020ehv}, in which the dressed cross section measured in this study and previously with c.m.\ energies from $\sqrt{s}=3.81$ to $4.60~{\rm GeV}$~\cite{BESIII:2020bgb} are used as input. Table~\ref{table5} shows the measured Born cross section at $\sqrt{s}=3.773~{\rm GeV}$ and the values of the other parameters in Eq.~\ref{eq:born}. Finally the Born cross section is calculated to be $\sigma^{B}(e^+ e^- \to \eta J/\psi) = (8.88 \pm 0.87_{\rm stat.} \pm 0.42_{\rm syst.})~{\rm pb}$ at $3.773~{\rm GeV}$.

% Figure environment removed

\begin{table}[htbp]
	\centering
	\footnotesize
	\caption{ The values of the integrated luminosity $\mathcal{L}$, the signal yield $N_{\rm obs}$, the detection efficiency $\varepsilon$, the product of radiative correction factor and vacuum polarization factor $R = (1+\delta^{{\rm ISR}})\cdot(1+\delta^{VP})$, and the obtained Born cross section of $e^+e^-\rightarrow\eta J/\psi$ at $\sqrt{s}=3.773~{\rm GeV}$. The uncertainties on the efficiency and cross section are statistical only.}
\begin{tabular}{ccccccc}\hline\hline
	\multicolumn{1}{c}{\multirow {1}{*}{ $\mathcal L$ (pb$^{-1}$) }}  & \multicolumn{1}{c}{ $\varepsilon(\%) $ }  & \multicolumn{1}{c}{ $R$ }
%	$(1+\delta^{{\rm ISR}})\cdot(1+\delta^{VP})$ } 
	& \multicolumn{1}{c}{ ${\cal{B}}(J/ \psi\rightarrow\mu^+\mu^-)(\%)$ } & \multicolumn{1}{c}{ ${\cal{B}}(\eta\rightarrow\gamma\gamma)(\%)$ } &  \multicolumn{1}{c}{ $N^{{\rm obs}}$ } & \multicolumn{1}{c}{ $\sigma^B$(pb) } \\
\hline
 $2931\pm15$ &  $45.4\pm0.1$  & $0.80$ & $5.96\pm0.03$  & $39.4\pm0.2$ & $222\pm22$  &  $8.89\pm0.88$ \\\hline\hline
\end{tabular}
\label{table5}
\end{table}

The branching fraction of $\psi(3770) \to \eta J/\psi$ is obtained by fitting to the dressed cross section of $e^+ e^- \to \eta J/\psi$ from $\sqrt{s} = 3.773~{\rm GeV}$ to $4.60~{\rm GeV}$, combining the cross section in this work and the previous BESIII analysis~\cite{BESIII:2020bgb} with c.m.\ energies from $\sqrt{s}=3.81$ to $4.60~{\rm GeV}$. Two treatments of the $\psi(3770)$ resonant decay amplitude are considered. Assuming $\psi(3770)$ is coherent with the other amplitudes, we get
\begin{eqnarray}
	\sigma_{\rm co.} & =|C\cdot\sqrt{\Phi(s)}+e^{i\phi_1}{\rm BW}_{\psi(3770)}+e^{i\phi_2}{\rm BW}_{\psi(4040)} \nonumber  \\
			      & +e^{i\phi_3}{\rm BW}_{Y(4230)}+e^{i\phi_4}{\rm BW}_{Y(4390)}|^2.
\end{eqnarray}
If $\psi(3770)$ is incoherent with the other amplitudes, we will have
\begin{eqnarray}
\sigma_{\rm inco.} & = |{\rm BW}_{\psi(3770)}|^2 +|C\cdot\sqrt{\Phi(s)}+e^{i\phi_2}{\rm BW}_{\psi(4040)} \nonumber \\
                                & +e^{i\phi_3}{\rm BW}_{Y(4230)}+e^{i\phi_4}{\rm BW}_{Y(4390)}|^2 \, ,
\end{eqnarray}
where $\Phi(s)=q^{3}/s$ is the $P$-wave phase space factor used to parameterize the continuum term, with $q$ being the $\eta$ momentum in the $e^+ e^-$ c.m. frame, BW is the Breit-Wigner function, $\phi$ is the relative phase between the resonant decay and the phase space term, and $C$ is a real parameter.

Using the above two formulae, the dressed cross sections are fitted as shown in Figure~\ref{fig:fit_cross_section}. In the incoherent case, the branching fraction is determined to be $(8.7 \pm 1.0_{\rm stat.} \pm 0.8_{\rm syst.}) \times 10^{-4}$, close to the CLEO result~\cite{CLEO:2005zky}. In the coherent case, four solutions are obtained with branching fractions varying between $(11.6 \pm 6.1_{\rm stat.} \pm 1.0_{\rm syst.})\times10^{-4}$ and $(12.0 \pm 6.1_{\rm stat.} \pm 1.1_{\rm syst.})\times10^{-4}$. We suppose that there exists substantial interference effect, especially between $\psi(3770)$ and highly excited vector charmonium(-like) states.

% Figure environment removed

\subsection{Observation of the decay $\chi_{cJ} \to \Omega^- \bar{\Omega}^+$~\cite{BESIII:2023bnk}}

The study of charmonium decays into baryon antibaryon ($B\bar{B}$) pairs provides a powerful tool for investigating many topics in quantum chromodynamics. In contrast to $J/\psi$ decays, the decays of the $P$-wave charmonium states, $\chi_{cJ}~(J=0,1,2)$, to $B\bar{B}$ have a non-trivial color-octet contribution~\cite{DASP:1975xwv, Feldman:1975bq}. Multiple models have been raised to describe $\chi_{cJ}$ to $B\bar{B}$ decays, including $p\bar{p}$, $\Lambda \bar{\Lambda}$, $\Sigma^+ \bar{\Sigma}^-$, $\Sigma^0 \bar{\Sigma}^0$, while none of them can describe all the final states~\cite{Wong:1999hc, Liu:2009vv}. Except the ground-state octet baryons above, it is desirable to extend these studies to decays of $\chi_{cJ}$ into pairs of decuplet ground-state baryons with spin 3/2. So far only $\chi_{c0} \to \Sigma(1385)^{\pm} \bar{\Sigma}(1385)^{\mp}$ decays~\cite{BESIII:2012wgp} have been studied by the BESIII Collaboration. The decay $\chi_{cJ} \to \Omega^- \bar{\Omega}^+$ is unique due to the presence of three pairs of strange quarks in the final state. This may give a distinct way for understanding quantum chromodynamics.

Based on $2.708 \times 10^9$ $\psi(3686)$ events, the decay $\chi_{cJ} \to \Omega^- \bar{\Omega}^+$ is studied using the radiation decay $\psi(3686) \to \gamma \chi_{cJ}$. Signal yield is obtained by fitting to the recoil mass spectrum of the radiative photon ($RM_{\gamma}$) after partially reconstructing $\Omega^-(\bar{\Omega}^+)$. The fit result is shown in Figure~\ref{fig:fit_RMgamma}. Then the branching fraction is calculate by
\begin{equation}
    \mathcal{B}(\chi_{cJ} \to \Omega^-\bar{\Omega}^+) = \frac{N_{\chi_{cJ}}^{\rm obs}} {N_{\psi(3686)} \cdot \mathcal{B}_{\psi(3686) \to \gamma \chi_{cJ}} \cdot \epsilon_{\chi_{cJ}}},
\end{equation}
where $N_{\chi_{cJ}}^{\rm obs}$ is the signal yield, $N_{\psi(3686)}$ is the total number of $\psi(3686)$ events, $\epsilon_{\chi_{cJ}}$ is the detection
efficiency including the subsequent $\Omega$ and $\Lambda$ decays, and $\mathcal{B}_{\psi(3686) \to \gamma \chi_{cJ}}$ is the BF of the $\psi(3686) \to \gamma \chi_{cJ}$
decay~\cite{ParticleDataGroup:2022pth}. The measured branching fractions for the three signal modes are listed in Table~\ref{tab:yields}.

This is the first observation of $\chi_{cJ}$ decays into a pair of decuplet ground-state baryons with spin 3/2. The $\chi_{cJ} \to \Omega^- \bar{\Omega}^+$ decays can also be used to probe the spin polarization of $\Omega^-$ baryon in the charmonium production at the future tau-charm factories~\cite{Achasov:2023gey}.

% Figure environment removed

\begin{table}[htbp]
    \begin{center}
    \caption{The $\chi_{cJ}$ signal yields ($N_{\chi_{cJ}}^{\rm obs}$), detection efficiencies ($\epsilon_{\chi_{cJ}}$), BFs of $\chi_{cJ} \to \Omega^- \bar{\Omega}^+$
    ($\mathcal{B}$) and the signal significances (${\rm Sig.}$). Here the uncertainties are statistical only.
    }
    \label{tab:yields}
    \setlength{\extrarowheight}{1.0ex}
    \renewcommand{\arraystretch}{1.0}
    \vspace{0.2cm}
 \begin{tabular}{p{1cm} | m{1.6cm}<{\centering} m{1.5cm}<{\centering} m{1.2cm}<{\centering} m{3.6cm}<{\centering}}
            \hline \hline
            Mode & $N_{\chi_{cJ}}^{\rm obs}$ & $\epsilon_{\chi_{cJ}}(\%)$ & ${\rm Sig.}(\sigma)$ & $\mathcal{B}(\times 10^{-5})$\\[1mm]
            \hline
            $\chi_{c0}$ & $~284 \pm 44 $ & 3.05 & 5.6  & $3.51 \pm 0.54 \pm 0.29$  \\
            $\chi_{c1}$ & $~277 \pm 42 $ & 7.02 & 6.4  & $1.49 \pm 0.23 \pm 0.10$  \\
            $\chi_{c2}$ & $1038 \pm 56$ & 8.91 & 18 & $4.52 \pm 0.24 \pm 0.18$  \\[1mm]
            \hline \hline
        \end{tabular}
    \vspace{-0.2cm}
    \end{center}
\end{table}

\section{Summary}

In this talk, recently published analyses of BESIII are introduced briefly. BESIII has collected the largest data sample of about 10 billion $J/\psi$ and 2.7 billion $\psi(3686)$ in 2009, 2012 and 2021, which will definitely benefit not only the studies of charmonium decays but multiple fields including the transition between low-lying charmonium states ($\psi(3686) \to \eta_c/\eta_c(2S)$, $\psi(3686) \to \chi_{cJ}$...), the precise validation of the non-perturbative QCD calculation, the study of the light hadron spectroscopy, the search for the rare decays of charmonia (leptonic, semi-leptonic, invisible...) and the search for the new physics beyond standard model related with axion, dark matter etc.

\bibliographystyle{unsrt}
\begin{thebibliography}{99}

%\cite{Kwong:1987mj}
\bibitem{Kwong:1987mj}
W.~Kwong, J.~L.~Rosner and C.~Quigg,
%``Heavy Quark Systems,''
\href{https://www.annualreviews.org/doi/10.1146/annurev.ns.37.120187.001545}{Ann. Rev. Nucl. Part. Sci. \textbf{37}, 325-382 (1987)}.
%doi:10.1146/annurev.ns.37.120187.001545
%252 citations counted in INSPIRE as of 01 May 2023

%\cite{QuarkoniumWorkingGroup:2004kpm}
\bibitem{QuarkoniumWorkingGroup:2004kpm}
N.~Brambilla \textit{et al.} [Quarkonium Working Group],
%``Heavy quarkonium physics,''
%doi:10.5170/CERN-2005-005
[arXiv:\href{https://arxiv.org/abs/hep-ph/0412158}{hep-ph/0412158} [hep-ph]].
%1000 citations counted in INSPIRE as of 01 May 2023

%\cite{Eichten:2007qx}
\bibitem{Eichten:2007qx}
E.~Eichten, S.~Godfrey, H.~Mahlke and J.~L.~Rosner,
%``Quarkonia and their transitions,''
\href{https://journals.aps.org/rmp/abstract/10.1103/RevModPhys.80.1161}{Rev. Mod. Phys. \textbf{80}, 1161-1193 (2008)}
%doi:10.1103/RevModPhys.80.1161
[arXiv:\href{https://arxiv.org/abs/hep-ph/0701208}{hep-ph/0701208} [hep-ph]].
%318 citations counted in INSPIRE as of 01 May 2023

%\cite{Brambilla:2010cs}
\bibitem{Brambilla:2010cs}
N.~Brambilla, S.~Eidelman, B.~K.~Heltsley, R.~Vogt, G.~T.~Bodwin, E.~Eichten, A.~D.~Frawley, A.~B.~Meyer, R.~E.~Mitchell and V.~Papadimitriou, \textit{et al.}
%``Heavy Quarkonium: Progress, Puzzles, and Opportunities,''
\href{https://link.springer.com/article/10.1140/epjc/s10052-010-1534-9}{Eur. Phys. J. C \textbf{71}, 1534 (2011)}
%doi:10.1140/epjc/s10052-010-1534-9
[arXiv:\href{https://arxiv.org/abs/1010.5827}{1010.5827} [hep-ph]].
%1766 citations counted in INSPIRE as of 01 May 2023

%\cite{Rosner:2011eg}
\bibitem{Rosner:2011eg}
J.~L.~Rosner,
%``Quarkonium - Theory,''
[arXiv:\href{https://arxiv.org/abs/1107.1273}{1107.1273} [hep-ph]].
%4 citations counted in INSPIRE as of 01 May 2023

%\cite{BESIII:2009fln}
\bibitem{Ablikim:2009aa}
M.~Ablikim \textit{et al.} [BESIII],
%``Design and Construction of the BESIII Detector,''
\href{https://www.sciencedirect.com/science/article/abs/pii/S0168900209023870?via\%3Dihub}{Nucl. Instrum. Meth. A \textbf{614}, 345-399 (2010)}
%doi:10.1016/j.nima.2009.12.050
[arXiv:\href{https://arxiv.org/abs/0911.4960}{0911.4960} [physics.ins-det]].
%1029 citations counted in INSPIRE as of 10 Jul 2023

%\cite{Yu:2016cof}
\bibitem{Yu:IPAC2016-TUYA01}
C.~Yu, Z.~Duan, S.~Gu, Y.~Guo, X.~Huang, D.~Ji, H.~Ji, Y.~Jiao, Z.~Liu and Y.~Peng, \textit{et al.}
%``BEPCII Performance and Beam Dynamics Studies on Luminosity,''
\href{https://accelconf.web.cern.ch/ipac2016/doi/JACoW-IPAC2016-TUYA01.html}{Proceedings of IPAC2016, Busan, Korea, 2016}.
%doi:10.18429/JACoW-IPAC2016-TUYA01
%123 citations counted in INSPIRE as of 10 Jul 2023

%\cite{BESIII:2020nme}
\bibitem{Ablikim:2019hff}
M.~Ablikim \textit{et al.} [BESIII],
%``Future Physics Programme of BESIII,''
\href{https://iopscience.iop.org/article/10.1088/1674-1137/44/4/040001}{Chin. Phys. C \textbf{44}, no.4, 040001 (2020)}
%doi:10.1088/1674-1137/44/4/040001
[arXiv:\href{https://arxiv.org/abs/1912.05983}{1912.05983} [hep-ex]].
%385 citations counted in INSPIRE as of 10 Jul 2023

\bibitem{EcmsMea}
J.~Lu, Y.~Xiao, X.~Ji,
%``Online monitoring of the center-of-mass energy from real data at BESIII,''
\href{https://link.springer.com/article/10.1007/s41605-020-00188-8}{Radiat. Detect. Technol. Methods {\bf 4}, 337–344 (2020)}.
%https://doi.org/10.1007/s41605-020-00188-8

\bibitem{EventFilter}
J.~W.~Zhang, L.~H.~Wu, S.~S.~Sun {\it et al.},
%``Suppression of top-up injection backgrounds with offline event filter in the BESIII experiment,''
\href{https://link.springer.com/article/10.1007/s41605-022-00331-7}{Radiat. Detect. Technol. Methods {\bf 6}, 289–293 (2022)}.
%https://doi.org/10.1007/s41605-022-00331-7
  
\bibitem{etof1}
X.~Li {\it et al.}, \href{https://link.springer.com/article/10.1007/s41605-017-0014-2}{Radiat. Detect. Technol. Methods {\bf 1}, 13 (2017)}.

\bibitem{etof2}
Y.~X.~Guo {\it et al.}, \href{https://link.springer.com/article/10.1007/s41605-017-0012-4}{Radiat. Detect. Technol. Methods {\bf 1}, 15 (2017)}.

%\cite{Cao:2020ibk}
\bibitem{etof3}
P.~Cao, H.~F.~Chen, M.~M.~Chen, H.~L.~Dai, Y.~K.~Heng, X.~L.~Ji, X.~S.~Jiang, C.~Li, X.~Li and S.~B.~Liu, \textit{et al.}
%``Design and construction of the new BESIII endcap Time-of-Flight system with MRPC Technology,''
\href{https://www.sciencedirect.com/science/article/abs/pii/S0168900219314068?via\%3Dihub}{Nucl. Instrum. Meth. A \textbf{953}, 163053 (2020)}.
%doi:10.1016/j.nima.2019.163053
%76 citations counted in INSPIRE as of 10 Jul 2023

%\cite{BESIII:2023zcs}
\bibitem{BESIII:2023zcs}
M.~Ablikim \textit{et al.} [BESIII],
%``Helicity amplitude analysis of \ensuremath{\chi}$_{cJ}$\textrightarrow{} \ensuremath{\phi}\ensuremath{\phi},''
\href{https://link.springer.com/article/10.1007/JHEP05(2023)069}{JHEP \textbf{05}, 069 (2023)}
%doi:10.1007/JHEP05(2023)069
[arXiv:\href{https://arxiv.org/abs/2301.12922}{2301.12922} [hep-ex]].
%0 citations counted in INSPIRE as of 11 Jul 2023

%\cite{Duncan:1980qd}
\bibitem{Duncan:1980qd}
A.~Duncan and A.~H.~Mueller,
%``Heavy Quarkonium Decays and the Renormalization Group,''
\href{https://www.sciencedirect.com/science/article/abs/pii/0370269380901082?via\%3Dihub}{Phys. Lett. B \textbf{93}, 119-124 (1980)}.
%doi:10.1016/0370-2693(80)90108-2
%78 citations counted in INSPIRE as of 11 Jul 2023

%\cite{Jones:1981ff}
\bibitem{Jones:1981ff}
H.~F.~Jones and J.~Wyndham,
%``Perturbative {QCD} and the Exclusive Decay of Heavy Quarkonium Into a Photon and Two Mesons,''
\href{https://www.sciencedirect.com/science/article/abs/pii/0550321382903972?via\%3Dihub}{Nucl. Phys. B \textbf{195}, 222-236 (1982)}.
%doi:10.1016/0550-3213(82)90397-2
%10 citations counted in INSPIRE as of 11 Jul 2023

%\cite{Anselmino:1992rw}
\bibitem{Anselmino:1992rw}
M.~Anselmino and F.~Murgia,
%``Chi (c0, c2) ---\ensuremath{>} rho rho decays and the rho polarization: Massless versus constituent quarks,''
\href{https://journals.aps.org/prd/abstract/10.1103/PhysRevD.47.3977}{Phys. Rev. D \textbf{47}, 3977-3983 (1993)}.
%doi:10.1103/PhysRevD.47.3977
%20 citations counted in INSPIRE as of 11 Jul 2023

%\cite{Zhou:2004mw}
\bibitem{Zhou:2004mw}
H.~Q.~Zhou, R.~G.~Ping and B.~S.~Zou,
%``Mechanisms for chi(cJ) ---\ensuremath{>} phi phi decays,''
\href{https://www.sciencedirect.com/science/article/abs/pii/S0370269305002170?via\%3Dihub}{Phys. Lett. B \textbf{611}, 123-128 (2005)}
%doi:10.1016/j.physletb.2005.02.017
[arXiv:\href{https://arxiv.org/abs/hep-ph/0412221}{hep-ph/0412221} [hep-ph]].
%58 citations counted in INSPIRE as of 11 Jul 2023

%\cite{Chernyak:1981zz}
\bibitem{Chernyak:1981zz}
V.~L.~Chernyak and A.~R.~Zhitnitsky,
%``Exclusive Decays of Heavy Mesons,''
\href{https://www.sciencedirect.com/science/article/abs/pii/055032138290445X?via\%3Dihub}{Nucl. Phys. B \textbf{201}, 492 (1982)}
[erratum: \href{https://www.sciencedirect.com/science/article/pii/0550321383902511?via\%3Dihub}{Nucl. Phys. B \textbf{214}, 547 (1983)}].
%doi:10.1016/0550-3213(83)90251-1
%416 citations counted in INSPIRE as of 11 Jul 2023

%\cite{Yang:1950rg}
\bibitem{Yang:1950rg}
C.~N.~Yang,
%``Selection Rules for the Dematerialization of a Particle Into Two Photons,''
\href{https://journals.aps.org/pr/abstract/10.1103/PhysRev.77.242}{Phys. Rev. \textbf{77}, 242-245 (1950)}.
%doi:10.1103/PhysRev.77.242
%1171 citations counted in INSPIRE as of 11 Jul 2023

%\cite{BESII:2011hcd}
\bibitem{BESII:2011hcd}
M.~Ablikim \textit{et al.} [BESII],
%``Observation of $\chi_{c1}$ decays into vector meson pairs $\phi\phi$, $\omega\omega$, and $\omega\phi$,''
\href{https://journals.aps.org/prl/abstract/10.1103/PhysRevLett.107.092001}{Phys. Rev. Lett. \textbf{107}, 092001 (2011)}
%doi:10.1103/PhysRevLett.107.092001
[arXiv:\href{https://arxiv.org/abs/1104.5068}{1104.5068} [hep-ex]].
%34 citations counted in INSPIRE as of 11 Jul 2023

%\cite{Huang:2021kfm}
\bibitem{Huang:2021kfm}
Q.~Huang, J.~Z.~Wang, R.~G.~Ping and X.~Liu,
%``Detecting the polarization in $\chi_{cJ} \to \phi \phi $ decays to probe hadronic loop effect,''
\href{https://journals.aps.org/prd/abstract/10.1103/PhysRevD.103.096006}{Phys. Rev. D \textbf{103}, no.9, 096006 (2021)}
%doi:10.1103/PhysRevD.103.096006
[arXiv:\href{https://arxiv.org/abs/2102.07104}{2102.07104} [hep-ph]].
%2 citations counted in INSPIRE as of 11 Jul 2023

%\cite{Chen:2013gka}
\bibitem{Chen:2013gka}
H.~Chen and R.~G.~Ping,
%``Polarization in $\chi_{cJ} \to \phi \phi$ decays,''
\href{https://journals.aps.org/prd/abstract/10.1103/PhysRevD.88.034025}{Phys. Rev. D \textbf{88}, no.3, 034025 (2013)}.
%doi:10.1103/PhysRevD.88.034025
%3 citations counted in INSPIRE as of 11 Jul 2023

%\cite{ParticleDataGroup:2022pth}
\bibitem{ParticleDataGroup:2022pth}
R.~L.~Workman \textit{et al.} [Particle Data Group],
%``Review of Particle Physics,''
\href{https://academic.oup.com/ptep/article/2022/8/083C01/6651666?login=false}{PTEP \textbf{2022}, 083C01 (2022)}.
%doi:10.1093/ptep/ptac097
%1284 citations counted in INSPIRE as of 12 Jul 2023

%\cite{BESIII:2020nme}
\bibitem{BESIII:2020nme}
M.~Ablikim \textit{et al.} [BESIII],
%``Future Physics Programme of BESIII,''
\href{https://iopscience.iop.org/article/10.1088/1674-1137/44/4/040001}{Chin. Phys. C \textbf{44}, no.4, 040001 (2020)}
%doi:10.1088/1674-1137/44/4/040001
[arXiv:\href{https://arxiv.org/abs/1912.05983}{1912.05983} [hep-ex]].
%385 citations counted in INSPIRE as of 12 Jul 2023

%\cite{BESIII:2022ksv}
\bibitem{BESIII:2022ksv}
M.~Ablikim \textit{et al.} [BESIII],
%``Evidence for the \ensuremath{\eta}c(2S)\textrightarrow{}\ensuremath{\pi}+\ensuremath{\pi}-\ensuremath{\eta} decay,''
\href{https://journals.aps.org/prd/abstract/10.1103/PhysRevD.107.052007}{Phys. Rev. D \textbf{107}, no.5, 052007 (2023)}
%doi:10.1103/PhysRevD.107.052007
[arXiv:\href{https://arxiv.org/abs/2211.11935}{2211.11935} [hep-ex]].
%0 citations counted in INSPIRE as of 12 Jul 2023

%\cite{Appelquist:1974zd}
\bibitem{Appelquist:1974zd}
T.~Appelquist and H.~D.~Politzer,
%``Orthocharmonium and e+ e- Annihilation,''
\href{https://journals.aps.org/prl/abstract/10.1103/PhysRevLett.34.43}{Phys. Rev. Lett. \textbf{34}, 43 (1975)}.
%doi:10.1103/PhysRevLett.34.43
%1253 citations counted in INSPIRE as of 12 Jul 2023

%\cite{Franklin:1983ve}
\bibitem{Franklin:1983ve}
M.~E.~B.~Franklin, G.~J.~Feldman, G.~S.~Abrams, M.~S.~Alam, C.~A.~Blocker, A.~Blondel, A.~Boyarski, M.~Breidenbach, D.~L.~Burke and W.~C.~Carithers, \textit{et al.}
%``Measurement of $\psi(3097)$ and $\psi^\prime$ (3686) Decays Into Selected Hadronic Modes,''
\href{https://journals.aps.org/prl/abstract/10.1103/PhysRevLett.51.963}{Phys. Rev. Lett. \textbf{51}, 963-966 (1983)}.
%doi:10.1103/PhysRevLett.51.963
%152 citations counted in INSPIRE as of 12 Jul 2023

\bibitem{Chao:1996}
K. T. Chao, Y. F. Gu, and S. F. Tuan,
\href{https://iopscience.iop.org/article/10.1088/0253-6102/25/4/471}{Commun. Theor. Phys. 25, 471 (1996)}.

%\cite{BESIII:2022hcv}
\bibitem{BESIII:2022hcv}
M.~Ablikim \textit{et al.} [BESIII],
%``Observation of $\eta_c(2S) \to 3(\pi^+\pi^-)$ and measurements of $\chi_{cJ} \to 3(\pi^+\pi^-)$ in $\psi(3686)$ radiative transitions,''
\href{https://journals.aps.org/prd/abstract/10.1103/PhysRevD.106.032014}{Phys. Rev. D \textbf{106}, no.3, 032014 (2022)}
%doi:10.1103/PhysRevD.106.032014
[arXiv:\href{https://arxiv.org/abs/2206.08807}{2206.08807} [hep-ex]].
%2 citations counted in INSPIRE as of 12 Jul 2023

%\cite{BESIII:2022yoo}
\bibitem{BESIII:2022yoo}
M.~Ablikim \textit{et al.} [BESIII],
%``Observation of \ensuremath{\psi}(3770)\textrightarrow{}\ensuremath{\eta}J/\ensuremath{\psi},''
\href{https://journals.aps.org/prd/abstract/10.1103/PhysRevD.107.L091101}{Phys. Rev. D \textbf{107}, no.9, L091101 (2023)}
%doi:10.1103/PhysRevD.107.L091101
[arXiv:\href{https://arxiv.org/abs/2212.12165}{2212.12165} [hep-ex]].
%1 citations counted in INSPIRE as of 12 Jul 2023

%\cite{Eichten:1978tg}
\bibitem{Eichten:1978tg}
E.~Eichten, K.~Gottfried, T.~Kinoshita, K.~D.~Lane and T.~M.~Yan,
%``Charmonium: The Model,''
\href{https://journals.aps.org/prd/abstract/10.1103/PhysRevD.17.3090}{Phys. Rev. D \textbf{17}, 3090 (1978)}
[erratum: \href{https://journals.aps.org/prd/abstract/10.1103/PhysRevD.21.313.2}{Phys. Rev. D \textbf{21}, 313 (1980)}].
%doi:10.1103/PhysRevD.17.3090
%1693 citations counted in INSPIRE as of 12 Jul 2023

%\cite{He:2008xb}
\bibitem{He:2008xb}
Z.~G.~He, Y.~Fan and K.~T.~Chao,
%``QCD prediction for the non-D anti-D annihilation decay of psi(3770),''
\href{https://journals.aps.org/prl/abstract/10.1103/PhysRevLett.101.112001}{Phys. Rev. Lett. \textbf{101}, 112001 (2008)}
%doi:10.1103/PhysRevLett.101.112001
[arXiv:\href{https://arxiv.org/abs/0802.1849}{0802.1849} [hep-ph]].
%39 citations counted in INSPIRE as of 12 Jul 2023

%\cite{Voloshin:2005sd}
\bibitem{Voloshin:2005sd}
M.~B.~Voloshin,
%``The anti-c c purity of psi(3770) and psi' challenged,''
\href{https://journals.aps.org/prd/abstract/10.1103/PhysRevD.71.114003}{Phys. Rev. D \textbf{71}, 114003 (2005)}
%doi:10.1103/PhysRevD.71.114003
[arXiv:\href{https://arxiv.org/abs/hep-ph/0504197}{hep-ph/0504197} [hep-ph]].
%27 citations counted in INSPIRE as of 12 Jul 2023

%\cite{Mo:2006cy}
\bibitem{Mo:2006cy}
X.~H.~Mo, C.~Z.~Yuan and P.~Wang,
%``Study of the Rho-pi Puzzle in Charmonium Decays,''
\href{http://cpc.ihep.ac.cn/article/id/d7f266c1-9574-470c-9a0d-c9da20a5a2e2}{Chin. Phys. C \textbf{31}, 686-701 (2007)}
[arXiv:\href{https://arxiv.org/abs/hep-ph/0611214}{hep-ph/0611214} [hep-ph]].
%34 citations counted in INSPIRE as of 12 Jul 2023

%\cite{Rosner:2004wy}
\bibitem{Rosner:2004wy}
J.~L.~Rosner,
%``Psi'' decays to charmless final states,''
\href{https://www.sciencedirect.com/science/article/abs/pii/S0003491605000345?via\%3Dihub}{Annals Phys. \textbf{319}, 1-12 (2005)}
%doi:10.1016/j.aop.2005.02.004
[arXiv:\href{https://arxiv.org/abs/hep-ph/0411003}{hep-ph/0411003} [hep-ph]].
%54 citations counted in INSPIRE as of 12 Jul 2023

%\cite{Zhang:2009gy}
\bibitem{Zhang:2009gy}
Y.~J.~Zhang and Q.~Zhao,
%``The Lineshape of psi(3770) and low-lying vector charmonium resonance parameters in e+ e- ---\ensuremath{>} D anti-D,''
\href{https://journals.aps.org/prd/abstract/10.1103/PhysRevD.81.034011}{Phys. Rev. D \textbf{81}, 034011 (2010)}
%doi:10.1103/PhysRevD.81.034011
[arXiv:\href{https://arxiv.org/abs/0911.5651}{0911.5651} [hep-ph]].
%35 citations counted in INSPIRE as of 12 Jul 2023

%\cite{Ding:1991vu}
\bibitem{Ding:1991vu}
Y.~B.~Ding, D.~H.~Qin and K.~T.~Chao,
%``Electric dipole transitions of psi (3770) and S - D mixing between psi (3686) and psi (3770),''
\href{https://journals.aps.org/prd/abstract/10.1103/PhysRevD.44.3562}{Phys. Rev. D \textbf{44}, 3562-3566 (1991)}.
%doi:10.1103/PhysRevD.44.3562
%61 citations counted in INSPIRE as of 12 Jul 2023

%\cite{Guo:2012tj}
\bibitem{Guo:2012tj}
Z.~k.~Guo, S.~Narison, J.~M.~Richard and Q.~Zhao,
%``Isospin violating decay of $\psi(3770)\rightarrow J/\psi + \pi^0$,''
\href{https://journals.aps.org/prd/abstract/10.1103/PhysRevD.85.114007}{Phys. Rev. D \textbf{85}, 114007 (2012)}
%doi:10.1103/PhysRevD.85.114007
[arXiv:\href{https://arxiv.org/abs/1204.1448}{1204.1448} [hep-ph]].
%9 citations counted in INSPIRE as of 12 Jul 2023

%\cite{Wang:2011yh}
\bibitem{Wang:2011yh}
Q.~Wang, X.~H.~Liu and Q.~Zhao,
%``Open charm effects in $e^+e^-\to J/\psi \eta$, $J/\psi \pi^0$ and $\phi\eta_c$,''
\href{https://journals.aps.org/prd/abstract/10.1103/PhysRevD.84.014007}{Phys. Rev. D \textbf{84}, 014007 (2011)}
%doi:10.1103/PhysRevD.84.014007
[arXiv:\href{https://arxiv.org/abs/1103.1095}{1103.1095} [hep-ph]].
%23 citations counted in INSPIRE as of 12 Jul 2023

%\cite{Liu:2009dr}
\bibitem{Liu:2009dr}
X.~Liu, B.~Zhang and X.~Q.~Li,
%``The Puzzle of excessive non-D anti-D component of the inclusive psi(3770) decay and the long-distant contribution,''
\href{https://www.sciencedirect.com/science/article/pii/S0370269309004729?via\%3Dihub}{Phys. Lett. B \textbf{675}, 441-445 (2009)}
%doi:10.1016/j.physletb.2009.04.047
[arXiv:\href{https://arxiv.org/abs/0902.0480}{0902.0480} [hep-ph]].
%55 citations counted in INSPIRE as of 12 Jul 2023

%\cite{Guo:2010ak}
\bibitem{Guo:2010ak}
F.~K.~Guo, C.~Hanhart, G.~Li, U.~G.~Meissner and Q.~Zhao,
%``Effect of charmed meson loops on charmonium transitions,''
\href{https://journals.aps.org/prd/abstract/10.1103/PhysRevD.83.034013}{Phys. Rev. D \textbf{83}, 034013 (2011)}
%doi:10.1103/PhysRevD.83.034013
[arXiv:\href{https://arxiv.org/abs/1008.3632}{1008.3632} [hep-ph]].
%120 citations counted in INSPIRE as of 12 Jul 2023

%\cite{CLEO:2005zky}
\bibitem{CLEO:2005zky}
N.~E.~Adam \textit{et al.} [CLEO],
%``Observation of psi(3770) ---\ensuremath{>} pi pi J/psi and measurement of Gamma(ee) [psi(2S)],''
\href{https://journals.aps.org/prl/abstract/10.1103/PhysRevLett.96.082004}{Phys. Rev. Lett. \textbf{96}, 082004 (2006)}
%doi:10.1103/PhysRevLett.96.082004
[arXiv:\href{https://arxiv.org/abs/hep-ex/0508023}{hep-ex/0508023} [hep-ex]].
%94 citations counted in INSPIRE as of 12 Jul 2023

%\cite{Zhang:2009kr}
\bibitem{Zhang:2009kr}
Y.~J.~Zhang, G.~Li and Q.~Zhao,
%``Towards a dynamical understanding of the non-D anti-D decay of psi(3770),''
\href{https://journals.aps.org/prl/abstract/10.1103/PhysRevLett.102.172001}{Phys. Rev. Lett. \textbf{102}, 172001 (2009)}
%doi:10.1103/PhysRevLett.102.172001
[arXiv:\href{https://arxiv.org/abs/0902.1300}{0902.1300} [hep-ph]].
%62 citations counted in INSPIRE as of 12 Jul 2023

%\cite{Li:2013zcr}
\bibitem{Li:2013zcr}
G.~Li, X.~h.~Liu, Q.~Wang and Q.~Zhao,
%``Further understanding of the non-DD\textasciimacron{} decays of \ensuremath{\psi}(3770),''
\href{https://journals.aps.org/prd/abstract/10.1103/PhysRevD.88.014010}{Phys. Rev. D \textbf{88}, no.1, 014010 (2013)}
%doi:10.1103/PhysRevD.88.014010
[arXiv:\href{https://arxiv.org/abs/1302.1745}{1302.1745} [hep-ph]].
%28 citations counted in INSPIRE as of 12 Jul 2023

%\cite{Anwar:2016mxo}
\bibitem{Anwar:2016mxo}
M.~N.~Anwar, Y.~Lu and B.~S.~Zou,
%``Modeling Charmonium-$\eta$ Decays of $J^{PC}=1^{--}$ Higher Charmonia,''
\href{https://journals.aps.org/prd/abstract/10.1103/PhysRevD.95.114031}{Phys. Rev. D \textbf{95}, no.11, 114031 (2017)}
%doi:10.1103/PhysRevD.95.114031
[arXiv:\href{https://arxiv.org/abs/1612.05396}{1612.05396} [hep-ph]].
%19 citations counted in INSPIRE as of 12 Jul 2023

%\cite{Kuraev:1985hb}
\bibitem{Kuraev:1985hb}
E.~A.~Kuraev and V.~S.~Fadin,
%``On Radiative Corrections to e+ e- Single Photon Annihilation at High-Energy,''
Sov. J. Nucl. Phys. \textbf{41}, 466-472 (1985).
%955 citations counted in INSPIRE as of 12 Jul 2023

%\cite{WorkingGrouponRadiativeCorrections:2010bjp}
\bibitem{WorkingGrouponRadiativeCorrections:2010bjp}
S.~Actis \textit{et al.} [Working Group on Radiative Corrections and Monte Carlo Generators for Low Energies],
%``Quest for precision in hadronic cross sections at low energy: Monte Carlo tools vs. experimental data,''
\href{https://link.springer.com/article/10.1140/epjc/s10052-010-1251-4}{Eur. Phys. J. C \textbf{66}, 585-686 (2010)}
%doi:10.1140/epjc/s10052-010-1251-4
[arXiv:\href{https://arxiv.org/abs/0912.0749}{0912.0749} [hep-ph]].
%362 citations counted in INSPIRE as of 12 Jul 2023

%\cite{Sun:2020ehv}
\bibitem{Sun:2020ehv}
W.~Sun, T.~Liu, M.~Jing, L.~Wang, B.~Zhong and W.~Song,
%``An iterative weighting method to apply ISR correction to e$^{+}$e$^{−}$ hadronic cross-section measurements,''
\href{https://link.springer.com/article/10.1007/s11467-021-1085-6}{Front. Phys. (Beijing) \textbf{16}, no.6, 64501 (2021)}
%doi:10.1007/s11467-021-1085-6
[arXiv:\href{https://arxiv.org/abs/2011.07889}{2011.07889} [hep-ex]].
%10 citations counted in INSPIRE as of 12 Jul 2023

%\cite{BESIII:2020bgb}
\bibitem{BESIII:2020bgb}
M.~Ablikim \textit{et al.} [BESIII],
%``Observation of the $Y(4220)$ and $Y(4360)$ in the process $e^{+}e^{-} \to \eta J/\psi$,''
\href{https://journals.aps.org/prd/abstract/10.1103/PhysRevD.102.031101}{Phys. Rev. D \textbf{102}, no.3, 031101 (2020)}
%doi:10.1103/PhysRevD.102.031101
[arXiv:\href{https://arxiv.org/abs/2003.03705}{2003.03705} [hep-ex]].
%32 citations counted in INSPIRE as of 12 Jul 2023

%\cite{BESIII:2023bnk}
\bibitem{BESIII:2023bnk}
M.~Ablikim \textit{et al.} [BESIII],
%``Observation of the decay $\chi_{cJ} \to \Omega^- \bar{\Omega}^+$,''
\href{https://journals.aps.org/prd/abstract/10.1103/PhysRevD.107.092004}{Phys. Rev. D \textbf{107}, no.9, 092004 (2023)}
[arXiv:\href{https://arxiv.org/abs/2302.12579}{2302.12579} [hep-ex]].
%0 citations counted in INSPIRE as of 12 Jul 2023

%\cite{DASP:1975xwv}
\bibitem{DASP:1975xwv}
W.~Braunschweig \textit{et al.} [DASP],
%``Observation of the Two Photon Cascade 3.7-GeV --\ensuremath{>} 3.1-GeV + gamma gamma via an Intermediate State p - Charm,''
\href{https://www.sciencedirect.com/science/article/abs/pii/0370269375904827?via\%3Dihub}{Phys. Lett. B \textbf{57}, 407-412 (1975)}.
%doi:10.1016/0370-2693(75)90482-7
%232 citations counted in INSPIRE as of 12 Jul 2023

%\cite{Feldman:1975bq}
\bibitem{Feldman:1975bq}
G.~J.~Feldman, B.~Jean-Marie, B.~Sadoulet, F.~Vannucci, G.~S.~Abrams, A.~Boyarski, M.~Breidenbach, F.~Bulos, W.~Chinowsky and C.~E.~Friedberg, \textit{et al.}
%``psi-prime (3684) Radiative Decays to High Mass States,''
\href{https://journals.aps.org/prl/abstract/10.1103/PhysRevLett.35.821}{Phys. Rev. Lett. \textbf{35}, 821 (1975)}
[erratum: \href{https://journals.aps.org/prl/abstract/10.1103/PhysRevLett.35.1184.3}{Phys. Rev. Lett. \textbf{35}, 1184 (1975)}].
%doi:10.1103/PhysRevLett.35.821
%180 citations counted in INSPIRE as of 12 Jul 2023

%\cite{Wong:1999hc}
\bibitem{Wong:1999hc}
S.~M.~H.~Wong,
%``Color octet contribution in exclusive p wave charmonium decay into octet and decuplet baryons,''
\href{https://link.springer.com/article/10.1007/s100520000376}{Eur. Phys. J. C \textbf{14}, 643-671 (2000)}
%doi:10.1007/s100520000376
[arXiv:\href{https://arxiv.org/abs/hep-ph/9903236}{hep-ph/9903236} [hep-ph]].
%54 citations counted in INSPIRE as of 12 Jul 2023

%\cite{Liu:2009vv}
\bibitem{Liu:2009vv}
X.~H.~Liu and Q.~Zhao,
%``The Evasion of helicity selection rule in chi(c1) ---\ensuremath{>} VV and chi(c2) ---\ensuremath{>} VP via intermediate charmed meson loops,''
\href{https://journals.aps.org/prd/abstract/10.1103/PhysRevD.81.014017}{Phys. Rev. D \textbf{81}, 014017 (2010)}
%doi:10.1103/PhysRevD.81.014017
[arXiv:\href{https://arxiv.org/abs/0912.1508}{0912.1508} [hep-ph]].
%54 citations counted in INSPIRE as of 12 Jul 2023

%\cite{BESIII:2012wgp}
\bibitem{BESIII:2012wgp}
M.~Ablikim \textit{et al.} [BESIII],
%``Observation of $\chi_{cJ}$ Decays to $\Lambda \bar{\Lambda}\pi^{+}\pi^{-}$,''
\href{https://journals.aps.org/prd/abstract/10.1103/PhysRevD.86.052004}{Phys. Rev. D \textbf{86}, 052004 (2012)}
%doi:10.1103/PhysRevD.86.052004
[arXiv:\href{https://arxiv.org/abs/1207.5646}{1207.5646} [hep-ex]].
%10 citations counted in INSPIRE as of 12 Jul 2023

%\cite{Achasov:2023gey}
\bibitem{Achasov:2023gey}
M.~Achasov, X.~C.~Ai, R.~Aliberti, Q.~An, X.~Z.~Bai, Y.~Bai, O.~Bakina, A.~Barnyakov, V.~Blinov and V.~Bobrovnikov, \textit{et al.}
%``STCF Conceptual Design Report: Volume I - Physics \& Detector,''
[arXiv:\href{https://arxiv.org/abs/2303.15790}{2303.15790} [hep-ex]].
%17 citations counted in INSPIRE as of 12 Jul 2023

\end{thebibliography}

\end{document}
