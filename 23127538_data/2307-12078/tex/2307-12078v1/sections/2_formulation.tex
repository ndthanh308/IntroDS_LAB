
\section{PROBLEM FORMULATION}
\label{sec:problem_formulation}
\subsection{Sensor Network Model}
Let $\mathcal G = (\mathcal V, \mathcal E)$ be an undirected graph, where the vertices $\mathcal V=\lbrace 1, 2, \dots, |\mathcal V|\rbrace$ represent the sensor network agents and the edges $\mathcal E \subseteq \mathcal V \times \mathcal V$ represent the availability of inter-agent (distance or bearing) measurements. 
The position of each agent is represented by a vector in $\mathbb R^{d}$, where the dimension $d$ is either $2$ or $3$ depending on the application.
The collective configuration of the sensor network can be represented by the block vector
\begin{equation}
\bold p = \big[\ \bold p[1]^\top\ \bold p[2]^\top\ \bold p[3]^\top\ \dots\ \bold p[\text{\footnotesize{$|\mathcal V|$}}]^\top\ \big]^\top \in \mathbb R^{d|\mathcal V|},
\end{equation}
where $\bold p[i]\in\mathbb R^d$ is the $i^{th}$ agent's position. We denote the components of $\bold p[i]$ as $\bold p[i] = \big[\bold p[i]_1\ \bold p[i]_2\big]^\top$ when $d=2$, and similarly for $d=3$. We assume that two agents' positions cannot coincide. 
While the results of this paper can be extended to the case where $\bold p$ evolves according to a dynamical model, we discuss the static case (where $\bold p$ is fixed) in order to keep the analysis and presentation concise.

\subsection{Position Estimates}
Each agent uses a set of onboard sensors for localization (i.e., the estimation of its position). The estimated positions of the agents are collectively represented by the block vector $\hat{\bold p} \in \mathbb R^{d|\mathcal V|}$, such that $\hat{\bold p}[i] \in \mathbb R^d$ is the $i^{th}$ agent's position estimate. Let $\mathcal D\subseteq \mathcal V$ be the set of agents that have localization errors, e.g., due to sensor faults or spoofing attacks.
To keep the analysis succinct, for now we assume that the agents in $\mathcal D^\complement$ have perfect estimates, such that $\hat {\bold p}_{\mathcal D^\complement} = \bold p _{\mathcal D ^\complement}$, and
%
\[\|\hat{\bold p}[i] - \bold p[i]\|_2 > 0 \ \Leftrightarrow\  \ i\in \mathcal D\]
%
% and $\bold {x}_{\mathcal D^\complement} = \bold 0$.
We define $ \bold x \coloneqq \bold p - \hat{\bold p}$ as the vector of localization errors. Thus, $\bold x$ has exactly $|\mathcal D|$ non-zero blocks, i.e.,
$\|\bold x\|_{2,0} = |\mathcal D|$, and
$\bold x$ is said to be \textit{block $|\mathcal D|$-sparse}. In Section \ref{sec:robustness}, we extend the above formulation to incorporate the case where the agents in $\mathcal D^\complement$ have imperfect position estimates.

\begin{remark}
One can also consider an alternative problem in which the agents in $\mathcal D$ are maliciously misreporting their positions to the other agents (via broadcast-based communication), so that $\hat {\bold p}[i] \neq \bold {p}[i]\ \forall i \in \mathcal D$. In either problem, the block vector $\bold x$ needs to be reconstructed using the inter-agent measurements, where the non-zero blocks of $\bold x$ correspond to the agents in $\mathcal D$.
Hence, either problem can be solved using the proposed approach described in the remainder of this paper, due to the symmetric relationship between the two problems.
\end{remark}

\subsection{Distance and Bearing Measurements}
\label{sec:setup_measurements}
The agents adjacent to each other in $\mathcal G$ are able to obtain inter-agent measurements which can be used to reconstruct the localization error vector, $\bold x$. Given a function $\boldsymbol \phi:\mathbb R^d \times\mathbb R^d\rightarrow \mathbb R^m$, the inter-agent measurement model is defined as
\begin{equation}
\bold y_{ij} = \boldsymbol \phi(\bold p[i], \bold p[j]) + \bold e_{ij} \quad \forall (i,j)\in \mathcal E
\label{eq:edge_measurement}
\end{equation}
where $\bold e_{ij}\in \mathbb R^m$ is a bounded noise term.
% which is explored further in Section \ref{sec:robustness}. 
% Similar assumptions on bounded measurement noise were used in \cite{Reppa_Polycarpou_Panayiotou_2012} and \cite{nstacked_2015}.
Equation (\ref{eq:edge_measurement}) can be expressed in the block vector form as
\begin{equation}
    \bold y = \boldsymbol \Phi(\bold p) + \bold e
    \label{eq:edge_measurement_block}
\end{equation}
where $\bold y=[\bold y_{ij}]$, $\bold e=[\bold e_{ij}]$, and $\boldsymbol \Phi(\bold p) = \big[\boldsymbol \phi(\bold p[i], \bold p[j])\big]$ are block vectors.

% The problem under consideration is that of using the measurements $\bold y_{ij}$ to identify which (if any) agents' estimated position $\hat{\bold p} [i]$ is different from its actual position $\bold p [i]$. 
We consider two types of measurements having the form of (\ref{eq:edge_measurement}) that commonly arise in sensor network applications, namely distance and bearing measurements. In the case where the agents can measure their distances from each other, we have 
\begin{equation}
\boldsymbol \phi_D(\bold p[i], \bold p[j])\coloneqq\frac{1}{2}\|\bold p[i]-\bold p[j]\|_2^2
\label{eq:distance_model}
\end{equation}
where the constant $\sfrac{1}{2}$ is introduced to keep the forthcoming notation concise. 
% In practice, distance measurements can be obtained using ultra wideband ranging as well as received signal strength (RSS) and time-of-arrival (ToA) of communication links between the sensor network agents, serving as redundant pieces of information about the sensor network configuration $\bold p$ which can be processed to detect and identify the localization errors.
Let $\bold \Phi_D(\bold p) = \big[\boldsymbol \phi_D(\bold p[i], \bold p[j])\big]
$ denote the block vector of length $|\mathcal E|$ comprising all the inter-agent distance functions. The \textit{distance rigidity matrix} \cite{trinh2016further}, denoted by $\bold R_D(\bold p)$, is then defined as 
\begin{equation}
\bold R_D(\bold p)\coloneqq \nabla \bold \Phi _D({\bold p})\in R^{|\mathcal E|\times d|\mathcal V|},
\end{equation}
which is the Jacobian of the measurement vector.
Thus, the $k^{th}$ row of $\bold R_D(\bold p)$ corresponds to the $k^{th}$ edge of the graph and is of the form
% [R(\bold p)]_m  &= 
% \begin{bmatrix}
% \frac{1}{2}\frac{[\partial h(\text{\scriptsize{$\bold p$}})]_m}{\partial \bold p
% \text{\scriptsize{$[1]$}}
% _x}  &
% \frac{1}{2}\frac{[\partial h(\text{\scriptsize{$\bold p$}})]_m}{\partial \bold p
% \text{\scriptsize{$[1]$}}
% _y}
% & \dots &
% \end{bmatrix}
% \nonumber \\&= 
\[
\begin{bmatrix} \ 0 & \dots & 0 & (\bold p[i] - \bold p[j])^\top & \dots & (\bold p[j] - \bold p[i])^\top & \dots\  \end{bmatrix}
\]
\noindent where $(i,j)$ is the $k^{th}$ edge in $\mathcal E$. Similarly, in the case of bearing measurements, we define 
\begin{equation}
\boldsymbol \phi_B(\bold p[i], \bold p[j])\coloneqq\frac{\bold p[i] - \bold p[j]}{\|\bold p[i] - \bold p[j]\|_2}
\end{equation}
which are the inter-agent unit vectors.
In this case, the Jacobian $\nabla \bold \Phi_B(\bold p)$ is denoted by $\bold R_B(\bold p) \in \mathbb R^{d|\mathcal E|\times d|\mathcal V|}$ and is called the \textit{bearing rigidity matrix} \cite{zhao2019bearing} whose $k^{th}$ row is of the form:
\begin{align*}
\begin{bmatrix} & 0 & \dots & 0 & \frac {\bold P_{ij}(\bold p)}{\|\bold p[i] - \bold p[j]\|_2} & \dots & \frac{\bold P_{ji}(\bold p)}{\|\bold p[i] - \bold p[j]\|_2} & \dots & \end{bmatrix},
\end{align*}
where $(i,j)$ is the $k^{th}$ edge in $\mathcal E$, and 
\begin{equation}
\bold P_{ij}(\bold p)=\bold I_d - 
\text{\footnotesize{
$\frac{(\bold p[i] - \bold p[j])(\bold p[i] - \bold p[j])^\top}{\|\bold p[i] - \bold p[j]\|_2^2}$
}}
\end{equation}
is the projection matrix corresponding to the subspace orthogonal to $\bold p[i] - \bold p[j]$.

In real-time applications, such as when using an Extended Kalman Filter (EKF) estimator, the Jacobian matrix is evaluated at the estimated state instead of the true state. For example,
as the squared Euclidean norm $\|\cdot \|_2^2$ is continuously differentiable everywhere, we can use the Taylor series approximation of $\bold \Phi_D(\bold p)$ about $\hat {\bold p}$ to establish
\begin{align} 
\bold \Phi_D(\bold{p}) \approx \bold \Phi_D(\hat{\bold p}) + \bold R_D(\hat {\bold p})(\bold p - \hat{\bold p})
\label{eq:taylor_series}
\end{align}
where the approximation error, which is on the order of $\| \hat{\bold p} - \bold p \|^2_2$, is assumed to be small. This is equivalent to the assumption that $\bold x$ is bounded, i.e., the agents have not deviated too far from their estimated positions. In Section \ref{sec:robustness}, we discuss how the linearization error in (\ref{eq:taylor_series}) can be compensated by using a bootstrapping approach.
% Since $\bold R_D(\hat{\bold p} + \bold x)=\bold R_D({\bold p})$, a bootstrapped estimate of the rigidity matrix may be used (using the current estimate of $\bold x$) to further compensate the linearization error; this approach is developed further in Section \ref{sec:robustness}.
% Onboard residual testing (such as the Chi-squared test) may be used to enforce the bound on $\bold x$ as well. Similarly, derivative of $\bold R_D(\bold p)$ with respect to $\bold p$ determines the error in using $\bold R(\hat {\bold p})$ in place of $\bold R(\bold p)$ in real-time applications. 
% 