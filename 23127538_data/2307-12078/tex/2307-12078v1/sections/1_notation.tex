
The paper is organized as follows; Section \ref{sec:problem_formulation} presents the mathematical descriptions of the sensor network, localization errors, and inter-agent measurements. In Section \ref{sec:optimization}, the main ideas from compressive sensing are introduced and motivated. Section \ref{sec:recoverability} establishes the conditions for the recoverability of localization errors using inter-agent measurements in the noise-free case. Section \ref{sec:robustness} extends our results to accommodate practical considerations such as measurement noise, imperfect estimates, and linearization error, and presents the SCP algorithm to solve the $l_2/l_1$ minimization problem in a robust manner.
Finally, the applicability of our approach is demonstrated through a numerical example in Section \ref{sec:numerical}.

\textit{Notation:} 
Given a block vector $\bold v\in\mathbb R^{dn}$ which is partitioned into $n$ blocks of length $d$ each, $\bold v[i]$ refers to the $i^{th}$ block of $\bold v$. The $l_q$ norm of $\bold v$ is denoted by $\|\bold v\|_{q}$.
Similar to \cite{efficient_block_sparse_2010, wang_wang_xu_2013, latushkin_null_2015, robust_NSP_2017}, we define the following notation:
\[
\|\bold v\|_{2,q}=
\begin{cases}
\begin{array}{lc}
     \sum_{i=1}^{n} \mathbb I\Big(\|\bold v[\small i]\|_2>0\Big) \quad & q=0 \vspace{2pt}\\
     \Big( \sum_{i=1}^{n} \|\bold v[\small i]\|_2^q\Big)^{1/q} & 0<q<\infty \vspace{2pt} \\
     \max_{1\leq i\leq n} \big(\|\bold v[\small i]\|_2\big) & q = \infty,
\end{array}
\end{cases}
\]
where $\mathbb I(\|\bold v[\small i]\|_2>0)$ is the indicator function which is equal to $1$ when $\|\bold v[\small i]\|_2>0$ and $0$ otherwise.
As defined, $\|\bold v\|_{2,0}$ counts the number of non-zero blocks of $\bold v$, referred to as its \textit{block sparsity}. 
% $\|\cdot\|_{2,q}$ is a norm for $q\geq 1$, with $\|\cdot\|_{2,2}=\|\cdot\|_2$. 
Given an index set $\mathcal S\subset \{1, 2, \dots, n\}$, $\mathcal S^\complement$ denotes its complement, $\{1, 2, \dots, n\}\backslash \mathcal S$. $\bold v_{\mathcal S}$ is the vector whose support is restricted to the blocks corresponding to $\mathcal S$, i.e., 
\[\bold v_{\mathcal S}[i] = \begin{cases}
\begin{array}{ll}
\bold v[i] \quad & i\in \mathcal S\\
\bold 0 & i\in \mathcal S^\complement,
\end{array}
\end{cases}
\]
where $\bold 0$ denotes a vector of zeros. $\bold 1_d$ denotes a vector of ones of length $d$ and $\bold I_d$ is the $d\times d$ identity matrix. For a matrix $\bold A$, we let 
% $\vrange(\bold A)$ refer to its column space (i.e., the linear span of its columns), whereas 
$\ker(\bold A)$ denote its kernel or null space. 
Lastly, `$\otimes$' is the Kronecker product operation for vectors and matrices.
% Given a set $\mathcal A$ and its subset $\mathcal B\subseteq \mathcal A$, $\mathcal B^\complement$ denotes the relative complement $\mathcal A\backsl\mathcal B$, where the corresponding superset of $\mathcal B$ is understood based on context.