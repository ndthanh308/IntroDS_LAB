\documentclass[reprint,nofootinbib,aps,twocolumn]{revtex4-2}

\usepackage{graphicx}
%\usepackage{dcolumn}
%\usepackage{bm}
\usepackage{amsmath,amssymb,nccmath}
%\usepackage{xcolor}
%\usepackage[colorlinks,linkcolor=blue]{hyperref}
%\usepackage{tikz}

\newcommand{\beq}{\begin{equation}}
\newcommand{\eeq}{\end{equation}}
\newcommand{\bea}{\begin{eqnarray}}
\newcommand{\eea}{\end{eqnarray}}
\newcommand{\bal}{\begin{aligned}}
\newcommand{\eal}{\end{aligned}}
\newcommand{\rf}[1]{(\ref{#1})}

\def\al{\alpha}
\def\be{\beta}
\def\ga{\gamma}
\def\de{\delta}
\def\ep{\epsilon}
\def\ze{\zeta}
\def\et{\eta}
\def\th{\theta}
\def\ka{\kappa}
\def\la{\lambda}
\def\rh{\rho}
\def\vr{\varrho}
\def\si{\sigma}
\def\vs{\varsigma}
\def\ta{\tau}
\def\up{\upsilon}
\def\ph{\phi}
\def\ch{\chi}
\def\ps{\psi}
\def\om{\omega}
\def\Ga{\Gamma}
\def\De{\Delta}
\def\Th{\Theta}
\def\La{\Lambda}
\def\Si{\Sigma}
\def\Up{\Upsilon}
\def\Ph{\Phi}
\def\Ps{\Psi}
\def\Om{\Omega}

\def\Gat{{\tilde \Ga}}

\def\cG{{\cal G}}
\def\cL{{\cal L}}
\def\cM{{\cal M}}
\def\cN{{\cal N}}
\def\cW{{\cal W}}

%repeating greek index combination abbreviations
\def\mn{{\mu\nu}}
\def\ab{{\al\be}}
\def\gd{{\ga\de}}
\def\bg{{\be\ga}}
\def\abgd{{\al\be\ga\de}}
\def\agbd{{\al\ga\be\de}}
\def\abg{{\al\be\ga}}
\def\bgd{{\be\ga\de}}
%%%%%

\def\prt{\partial}

\def\pt#1{\phantom{#1}}
\def\kb{\overline{k}}
\def\ktw{\tilde{k}}

\def\tb{\overline{t}}
\def\sb{\overline{s}}
\def\st{\tilde{s}}
\def\ub{\overline{u}}
\def\Ct{\tilde{C}}

\def\Gt{\tilde{G}}

\def\htra{ h^\al_{\pt{\al}\al} }
\def\htrb{ h^\be_{\pt{\be}\be} }
\def\htrg{ h^\ga_{\pt{\ga}\ga} }

\def\tg{{\tilde g}}
\def\tt{{\tilde t}}
\def\tR{{\tilde R}}
\def\tr{{\tilde r}}
\def\xb{{\overline x}}
\def\oprt{{\overline \prt}}
\def\oA{{\overline A}}
\def\oF{{\overline F}}
\def\oR{{\overline R}}
\def\ot{{\overline t}}
\def\or{{\overline r}}
\def\oh{{\overline h}}

%others
\def\fr#1#2{{{#1} \over {#2}}}
\def\half{{\textstyle{1\over 2}}}
\def\quar{{\textstyle{1\over 4}}}
\def\rf#1{(\ref{#1})}

\begin{document}

\title{Classical radiation fields for scalar, electromagnetic, and gravitational waves with spacetime-symmetry breaking}

\author{Quentin G. Bailey and Alexander S.\ Gard}
\affiliation{Embry-Riddle Aeronautical University, Prescott, AZ, 86301, USA}
\author{Nils A. Nilsson}
\affiliation{SYRTE, Observatoire de Paris, Universit\'e PSL, CNRS, Sorbonne Universit\'e,
LNE, 61 avenue de l'Observatoire, 75014 Paris, France}
\author{Rui Xu and Lijing Shao}
\affiliation{Kavli Institute for Astronomy and Astrophysics, Peking University, Beijing 100871, China}

%still adding authors

\date{\today}

\begin{abstract}
An effective field theory framework is used to investigate some Lorentz and CPT-violating effects on the generation of electromagnetic and gravitational waves, 
complementing previous work on propagation.
Specifically we find solutions to a modified, 
anisotropic wave equation, 
sourced by charged or fluid matter. 
We derive the radiation fields for scalars, 
classical electromagnetic radiation, 
and preliminary results for gravitational radiation.
\end{abstract}

\maketitle



\section{Introduction}
\label{sec:intro}

%\blfootnote{$*$ baileyq@erau.edu \\ $\dagger$ nils.nilsson@obspm.fr \\ $\ddagger$ Rui Xu's email address}

Presently, 
interest in tests of foundations of General Relativity (GR) is high.
This includes both theory and experiment.
Primary motivations include the possibility that some aspects of foundations of GR may be modified in a unified theory of physics that incorporates quantum gravity.
In particular, 
suggestions that spacetime-symmetry foundations of GR, 
like local Lorentz symmetry, 
could be broken in small but potentially detectable ways \cite{ksstring89,kp95} has motivated a plethora theoretical studies and analyses
\cite{Bailey_2023,Mariz:2022oib,safronova18,Will:2014kxa,Tasson:2014dfa,Liberati13,datatables}.
%add more cites

Much theoretical work has been accomplished
within effective field theory descriptions of spacetime-symmetry breaking, and with specific models.
This includes extensive literature
on the effects for electromagnetic waves and gravitational waves propagating in the vacuum \cite{Carroll:1989vb,km02,km09}.
Furthermore, 
much has been accomplished in Quantum Field Theory studies of spacetime-symmetry breaking in flat spacetime.%cite
Much of the latter work has developed solutions to the field equations in momentum space, 
which is what is needed for QFT applications.
Relatively few works have developed (full) classical position-space solutions, 
for instance those used in classic radiation multipole expansions.
We begin to fill this gap in this work.

The purpose of this article then is to study wave generation in the context of spacetime-symmetry breaking described by EFT.
Rather than a comprehensive study, 
we focus on minimal terms in the EFT
and use a coordinate transformation trick to find the exact Green function for a modified wave operator.
Our results are applied then to scalar fields, 
the electromagnetic sector, 
and the gravitational sector with some intriguing preliminary results on gravitational wave polarizations.

Except when we discuss some results for gravitational waves, 
most of this work is in flat spacetime with metric signature $-+++$, 
and we use Greek letters for spacetime indices.
For our notation we also follow conventions of references on spacetime-symmetry breaking
\cite{k04}.

\section{Green function with modified wave operator}
\label{Green function}

In effective field theory descriptions of spacetime-symmetry breaking, 
one encounters Lagrange densities of the schematic form
$\cL \supset \et^\mn \prt_\mu \ps \prt_\nu \ps+ t^{\mu\nu\la...} \prt_\mu \psi \prt_\nu \prt_\la ... \ps $, 
for some field $\ps$
where $t^{\mu\nu\la...}$ is a generic set of coefficients describing the degree of symmetry breaking \cite{ck98,Edwards:2018lsn}.
Upon obtaining the field equations,
one typically encounters wave-type equations modified from 
the usual D'Alembertian operator $\Box = \prt^\al \prt_\al = \nabla^2 - \prt_t^2$.
To solve them, 
one can seek a Green function solution.

For actions with just two derivatives,
the typical problem involves finding a Green function $G(x,x^\prime )$ satisfying the equation
\beq
(\tilde g)^\mn \prt_\mu \prt_\nu G (x,x^\prime) = -\de^{(4)} (x-x^\prime),
\label{GreenEqn}
\eeq
where ${\tilde g}^\mn$ are constants.
These constants ${\tilde g}^\mn$ can be chosen so that we have 
a well-posed hyperbolic partial differential equation 
for a smooth source, 
(i.e., for the underlying equation we are trying to solve $(\tilde g)^\mn \prt_\mu \prt_\nu \ps=\rh$) \cite{Wald:1984rg}.
Specifically, 
we will assume the following generic form:
\beq
\tg^\mn = \et^\mn + k^\mn,
\label{modMetric}
\eeq
where $k^\mn$ are a set of constant coefficients assumed to have values 
in the chosen coordinates sufficiently less than unity, so that $\tg^\mn$ is guaranteed an inverse.
Using Fourier transform methods, 
the momentum space solution of \rf{GreenEqn} is relatively trivial, 
while to date, to our knowledge,
%please help check this
no exact position space solution has been explicitly written and studied, 
although results can be found in certain limits \cite{bk04,Ferreira:2019ygi}.

The solution to \rf{GreenEqn} can be obtained by changing coordinates \cite{bk04} so that the equation appears with the conventional wave operator.
One then transforms the Green function back to the original coordinate system to obtain,
\beq
G (x,x^\prime) = \fr {1} {2\pi \sqrt{-\tg }} \de 
\left( -(\tg^{-1})_\mn (x-x^\prime )^\mu (x-x^\prime )^\nu \right),
\label{GreenResult}
\eeq
where $\tg$ is the determinant of ${\tilde g}^\mn$.
Details of this type of coordinate transformation are in Section \ref{photon sector application}.
This single delta function form can be changed into a more pragmatic form by finding the roots of the quantity in the argument.  
Note that one can clearly see the nature of this Green function inside an integral:
it forces an evaluation along a skewed light cone $-(\tg^{-1})_\mn (x-x^\prime )^\mu (x-x^\prime )^\nu=0$.

The result in equation \rf{GreenResult} can be cast into a more practical form for integration by expanding the delta function using the roots of its argument.
Defining $\ta = t-t^\prime$, 
and $R^i = (x-x^\prime)^i$,
it can be shown,
\beq
G (x,x^\prime) = \fr {1} {4\pi \sqrt{-\tg} } 
\fr {1} { \tR }
\de 
\left( \ta + \fr { \tR +(\tg^{-1})_{0i}R^i}{(\tg^{-1})_{00}} \right),
\label{GrnExpand}
\eeq
where
\beq
\tR =\sqrt{-(\tg^{-1})_{00} (\tg^{-1})_{ij} R^i R^j + ((\tg^{-1})_{0i} R^i)^2 }.
\label{tildeR}
\eeq

To give a sense of the propagation implied with this solution, 
we include some visualization in two and three dimensions 
in the plots to follow for the scalar field case.

\section{Scalar Example}
\label{scalar example}

We apply the results of the Green function \rf{GrnExpand} 
to the case of a real scalar field with generic source function.
We solve the equation
\beq
(\et^\mn + k^\mn ) \prt_\mu \prt_\nu \ps = -\rh,
\label{scalar}
\eeq
where $\rh$ stands for a generic source for the scalar.

Using the general Green function results above, 
we obtain,
\beq
\ps = \fr {1}{4\pi \sqrt{-\tg} } 
\int d^3 r^\prime \fr {\rh (\tt_R , \vec r^\prime ) }{\tR},
\label{psi}
\eeq
where the time $\tt_R$ is given by
\beq
\tt_R = t - \fr {\tR + (\tg^{-1})_{0i}R^i } { -(\tg^{-1})_{00} }.
\label{ttR}
\eeq
The time argument of $\rh$ is evaluated along the modified retarded time $\tt_R$.

For calculations of wave zone results, 
we will explore an expansion similar to that
in Ref.\ \cite{pw14}, 
wherein the authors construct a systematic wave zone and near zone expansion.  
We start by assuming that the field point $\vec r$ is located in the ``wave zone", far outside of the source region where $\rh \neq 0$.
Thus we assume that the source $\rh$ has ``compact support".
If this is the case, 
then we may make an expansion assuming that $r=||\vec r|| >> r^\prime$.
It will be useful to use the following quantities, 
obtained by evaluating the expressions above when $\vec r^\prime =0$.
Namely,
we define
\beq
\bal
\tr &= \sqrt{-(\tg^{-1})_{00} (\tg^{-1})_{ij} r^i r^j + ((\tg^{-1})_{0i} r^i)^2 },\\
\tt_r &= t - \fr {\tr + (\tg^{-1})_{0i}r^i } { -(\tg^{-1})_{00} }.
\eal
\label{ttr}
\eeq
Following parallel steps to Ref.\ \cite{pw14}, Section 6.3, page 315, 
we arrive at the series:
\beq
\ps = \fr {1}{4\pi \sqrt{-\tg} }
\sum_{l=0}^{\infty} 
\fr {(-1)^l} {l!}
\prt_L \left( \fr {1}{\tr} 
\int d^3 r^\prime \rh (\tt_r , \vec r^\prime ) r^{\prime L} \right).
\label{psiWZ}
\eeq
In this expression we use the index abbreviation $L=i_1 i_2 i_3...i_l$.
In will be useful in what follows to define a tangent vector $N_j$ with
\beq
N_j = - \prt_j \tt_r,
\label{N}
\eeq
which reduces to $\hat n^j = r^j/r$ 
when $k_\mn \rightarrow 0$ and represents the direction of wave propagation.

It is useful to note some results that are leading order in small coefficients $k^\mn$.
Using the definition \rf{modMetric}, 
we have for the inverse
\beq
(\tg^{-1})_\mn = \et_\mn - k_\mn+ k_{\mu\al} k^\al_{\pt{\al}\nu} + ...,
\label{invMetric}
\eeq
where indices on the right-hand side are raised and lowered with the Minkowski metric $\et_\mn$.
The modified retarded time $\tt_r$ and the tangent vector $N_j$ are written to leading order as
\beq
\bal
\tt_r &= t - r (1 - \frac 12 k_{00} - \frac 12 k_{ij} n^i n^j ) + k_{0i}r^i, \\
N_i &= n_i ( 1 - \frac 12 k_{00} + \frac 12 k_{jk} n^j n^k  )
-k_{ij} n^j
-k_{0i}.
\eal
\label{N1}
\eeq
Using these approximations we obtain the first 3 terms of the series \rf{psiWZ} in the wave zone (keeping only terms with $1/r$ falloff):
\beq
\bal
\ps &= \fr {1}{4\pi r} \Big( 
Q [ 1- \frac 12 k_{00} + \frac 12 k_{ij}n^i n^j ]
\\
&
\pt{space}
+ \dot P^i [n_i (1- k_{00} - k_{jk}n^j n^k ) -k_{ij}n^j - k_{0i} ]
\\
&
\pt{space}
+\frac 12 \ddot I^{ij}
[n_i n_j ( 1- \frac 32 k_{00} + \frac 32 k_{lm}n^l n^m)
\\
&
\pt{spacespace}
-2 n_i k_{jk} n^k - 2 k_{0i}n_j]+... \Big).
\eal
\label{psiLO}
\eeq
{\bf and explain P and I }
It is important note that the terms on the right-hand side of \rf{psiLO} are to be evaluated 
at the modified retarded time $\tt_r$ (equations \rf{ttr} and \rf{N1}), 
which has a deformed dependence on the space and time coordinates of the field point.
We include here plots of how to visualize the propagation of the wave from the source point (taken as the origin) to the field point $t,x$.

% Figure environment removed

\section{Photon sector application}
\label{photon sector application}

We apply the Green function formalism of section \ref{Green function}
to the photon sector.
The field equations for the photon sector in the ``non-birefringence" mSME limit can be written in the form,
\beq
\big( 
\et^{\mu\ka} \et^{\la\nu} +   
\et^{\mu\ka} (c_F)^{\la\nu} + 
(c_F)^{\mu\ka} \et^{\la\nu}
\big)
\prt_\mu F_{\ka\la} = -j^\nu,
\label{FEph}
\eeq
where $(c_F)^\mn$ are 9 coefficients for Lorentz violation (symmetric and assumed traceless),
$F_\mn = \prt_\mu A_\nu - \prt_\nu A_\mu$ is the usual field strength tensor, 
and $j^\nu$ is the conserved current source \cite{km02,km09,Bailey:2010af}.
For what follows it is then useful to define an alternate set of coefficients $\Ct^\mn$ by the following relation
\beq
\Ct^\mu_{\pt{\mu}\ka} ( \et^\ka_{\pt{\ka}\nu} + \frac 12  \Ct^\ka_{\pt{\ka}\nu} )= (c_F)^\mu_{\pt{\mu}\nu}.
\label{ctilde}
\eeq
[Note that to leading order in small dimensionless coefficients, $\Ct^\mu_{\pt{\mu}\ka} \approx (c_F)^\mu_{\pt{\mu}\ka}$.]
With \rf{ctilde} we can write \rf{FEph} as 
\beq
\tg^{\mu\ka} \tg^{\la\nu} 
\prt_\mu F_{\ka\la} = -j^\nu,
\label{FEph2}
\eeq
where $\tg^\mn = \et^\mn + \Ct^\mn$, 
similar to definition in \rf{modMetric}.
Note that $\tg^\mn$ resembles a skewed metric.

To solve this equation, 
we change coordinates $x^\mu = x^\mu ( \xb^\nu ) $, 
in a particular way such that under this coordinate change,
we have
\beq
{\overline \tg}^\mn = \fr {\prt \xb^\mu}{\prt x^\al}
\fr {\prt \xb^\nu}{\prt x^\be} \tg^{\al\be}= \et^\mn,
\label{gchange}
\eeq
where in the last step the numerical values of the ${\overline \tg}^\mn$ tensor 
are the same as the Minkowski metric {\it the $\xb^\mu$ coordinate system }.
Care is required because the spacetime metric in the $\xb^\mu$ system is {\bf not} Minkowski.
\footnote{This procedure was carried out at leading order in the appendix of
Ref.\ \cite{bk04} to demonstrate the physical equivalence of having certain forms of Lorentz violation
in the photon sector or the matter sector.}

In the new coordinate system the field equations take the form
\beq
\et^{\mu\ka} \et^{\la\nu} 
\oprt_\mu \oF_{\ka\la} = -{\overline j}^\nu,
\label{FEph3}
\eeq
with $\oprt_\mu= \prt/\prt \xb^\mu$ and $\oF_\mn = \oprt_\mu \oA_\nu - \oprt_\nu \oA_\mu$.
It is evident then that in these new coordinates, 
the field equations resemble that of conventional electrodynamics.
In particular,
there remains the usual gauge symmetry of electrodynamics.
We therefore, 
in the $\xb^\mu$ coordinates, 
adopt the gauge choice
$\et^\mn \oprt_\mu \oA_\nu=0$, 
leaving the field equations as
\beq
\et^\mn \oprt_\mu \oprt_\nu \oA_\la = -{\overline j}^\mu \et_{\mu\la},
\label{xbFEgf}
\eeq
%gauge fixed equations in xb system
where we have multiplied by a Minkowski inverse $\et_\mn$ on both sides to isolate $\oA_\mu$ (but again note that tensors are not raised and lowered in the $\xb^\mu$ system with the Minkowski metric).
The standard wave operator appears in equation \rf{xbFEgf} and so the usual inhomogeneous solution can be used. 
The solution to \rf{xbFEgf} is therefore
\beq
\oA_\la = \int d^4 \xb^\prime
\fr {1}{4\pi \oR } 
\de ( \ot^\prime - \ot + \oR )
{\overline j}^\mu \et_{\mu\la},
\label{Abarsoln}
\eeq
where $\oR=\sqrt{\de_{ij} (\or-\or^\prime)^i (\or-\or^\prime)^j}$, 
and the current is evaluated at $\ot^\prime$ and $\vec \or^\prime$.

To transform the solution back to the original coordinate system we make use of the alternative form of the delta function as in \rf{GreenResult}:
\beq
\oA_\la = \fr {1}{2\pi}  \int d^4 \xb^\prime
\de \big( -\et_{\al\be} (\xb -\xb^\prime)^\al (\xb -\xb^\prime)^\be \big)
{\overline j}^\mu \et_{\mu\la},
\label{Abarsoln2}
\eeq
Now we may use the coordinate transformation rule
$A_\la = \prt \xb^\mu / \prt x^\la \oA_\mu$
and change the coordinates within the integral in \rf{Abarsoln2}.
First, using equation \rf{gchange}, we can show the argument of the delta function in the original $x^\mu$ coordinates takes the form like \rf{GreenResult},
namely $-(\tg^{-1})_\mn (x-x^\prime )^\mu (x-x^\prime )^\nu$.
The remainder of the transformation follows from standard formulas.
The Jacobian of the transformation can be found from \rf{gchange} and can be written $|\prt \xb / \prt x|=1/\sqrt{-\tg}$.
The originally sought solution is then
\beq
\bal
A_\la &= 
\fr {1} {2\pi \sqrt{-\tg }} 
\int d^4x^\prime 
\de \left( -(\tg^{-1})_{\al\be} (x-x^\prime )^\al (x-x^\prime )^\be \right)
\\
&
\pt{spacespace}
\times
j^\mu (\tg^{-1})_{\mu\la}.
\label{Asoln}
\eal
\eeq
It is straightforward, 
using equation \rf{GreenEqn} and \rf{GreenResult}, 
and using the gauge condition transformed to the original coordinates,
namely $\tg^\mn \prt_\mu A_\nu=0$, 
to show that this solution satisfies \rf{FEph2} and hence
solves \rf{FEph}.

Following steps similar to those for the scalar field in we can write the solution compactly as
\beq
A_\la = \fr {1} {4\pi \sqrt{-\tg }} 
\int d^3x^\prime 
 \fr {(\tg^{-1})_{\mu\la} j^\mu (\tt_R, \vec r^\prime)}{\tR},
\label{Asoln2}
\eeq
with $\tt_R$ and $\tR$ as in equations \rf{ttR} and \rf{tildeR}.

We specialize \rf{Asoln2} to a localized conserved current density 
$j^\mu = (\rh,\vec J)$
and expand the solution assuming the field point is in the radiation zone ($r>>\la >>r^\prime$).
We follow steps similar to those leading up to
\rf{psiWZ}.
Thus we arrive at
\beq
A_\la = \fr {(\tg^{-1})_{\mu\la}}{4\pi \sqrt{-\tg} }
\sum_{l=0}^{\infty} 
\fr {(-1)^l} {l!}
\prt_L \left( \fr {1}{\tr} 
\int d^3 r^\prime j^\mu (\tt_r , \vec r^\prime ) r^{\prime L} \right).
\label{AWZ}
\eeq
We write out the first few terms up to $L=1$, 
decomposing the current into charge density and current density, 
and keeping only terms that fall as $1/\tr$, 
to obtain
\beq
\bal
A_\la &= \fr {1}{4\pi \sqrt{-\tg} \tr } \Big[
(\tg^{-1})_{\mu 0} \int d^3r^\prime \rh^\prime  
\\
&
\pt{space}
+(\tg^{-1})_{\mu i} \int d^3r^\prime J^{\prime i} 
+ N_i (\tg^{-1})_{\mu 0} \int d^3r^\prime \dot {\rh^\prime}  r^{\prime i} 
\\
&
\pt{space}
+
N_k (\tg^{-1})_{\mu i} \int d^3r^\prime {\dot J}^{\prime i} r^{\prime k}
+...
\Big].
\eal
\label{AWZ2}
\eeq
The first term is proportional to the constant total charge $Q$, while the second and third term can be re-expressed in terms of the dipole moment
$p^j = \int d^3r^\prime \rh^\prime r^{\prime j}$, 
using standard techniques 
\cite{Jackson}.
The fourth and higher terms contribute to the magnetic dipole and quadrupole terms, 
which we neglect here.
So the dominant {\it radiation} terms are 
\beq
A_\la = \fr {1}{4\pi \sqrt{-\tg} \tr } 
\Big[ (\tg^{-1})_{\mu i} + (\tg^{-1})_{\mu 0} N_i \Big] {\dot p}^i.
\label{radiation}
\eeq

The radiation zone electric field, which is gauge independent, 
is found to be
\beq
\bal
F_{i0} &= -\fr {1}{4\pi \sqrt{-\tg} \tr } 
\Big[ (\tg^{-1})_{0 i} N_j + (\tg^{-1})_{0 j} N_i 
\\
&
\pt{spacespace}
+ (\tg^{-1})_{ij} 
+ (\tg^{-1})_{00}N_i N_j
\Big] {\ddot p}^j.
\label{electric}
\eal
\eeq
Adopting a leading order expansion with $(\tg^{-1})_\mn = \et_\mn - (c_F)_\mn$ and using results above such as \rf{N1}, 
we can write the electric field as
\beq
F_{i0} = -\fr {1}{4\pi \sqrt{-\tg} \tr } 
\Big[ P_{ij} + P_{ik} P_{jl} (c_F)^{kl} \Big] {\ddot p}^j,
\label{electricLO}%leading order
\eeq
where $P_{ij} = \de_{ij} - n_i n_j$
is a projection operator.
Note that while $F_{i0}n_i=0$, indicating two independent polarizations, 
$n_i$ is not the true direction of wave propagation.
There remains a projection along the true direction of propagation $N_i$ \rf{N}, 
that is not zero, $F_{0i} N_i \neq 0$, 
indicating that at least one of the two independent modes is not transverse to the wave propagation direction.

\section{Gravity sector application}

\subsection{General solution}
\label{general solution}

{\bf add double dual, action?}
For a starting point, 
we use the mSME field equations for the metric fluctuations
$h_\mn$ around a flat background.
We can write the field equations
\beq
{\hat K}^{\mu\nu\al\be} h_{\al\be} = \ka \ta^\mn,
\label{FE1}
\eeq
where $\ta^\mn$ includes the matter stress energy tensor 
as well as the pseudotensor and other possible terms like higher order terms in $h$ with coefficients for Lorentz violation.
The operator ${\hat K}^{\mu\nu\al\be}$ can be written
as
\beq
\bal
{\hat K}^{\mu\nu\al\be} &= \frac 12 \Big( 
-\et^\mn \et^\ab \et^\gd 
+ \et^{\al (\mu} \et^{\nu )\be} \et^\gd 
+\et^\mn \et^{\al\ga} \et^{\be\de}
\\
&
+ \et^\ab \et^{\mu\ga} \et^{\nu\de}
-\et^{\al (\mu} \et^{\nu ) \ga} \et^{\be\de}
-\et^{\be (\mu} \et^{\nu ) \ga} \et^{\al\de}
\Big) \prt_\ga \prt_\de \\
& 
+\frac 12 \Big( 
-\sb^\mn \et^\ab \et^\gd 
-\et^\mn \sb^\ab \et^\gd 
-\et^\mn \et^\ab \sb^\gd 
\\
&
+ \sb^{\al (\mu} \et^{\nu )\be} \et^\gd 
+ \et^{\al (\mu} \sb^{\nu )\be} \et^\gd 
+ \et^{\al (\mu} \et^{\nu )\be} \sb^\gd 
\\
&
\pt{space}+\sb^\mn \et^{\al\ga} \et^{\be\de}
+\et^\mn \sb^{\al\ga} \et^{\be\de}
+\et^\mn \et^{\al\ga} \sb^{\be\de}
\\
&
+ \sb^\ab \et^{\mu\ga} \et^{\nu\de}
+ \et^\ab \sb^{\ga (\mu} \et^{\nu) \de}
+ \et^\ab \et^{\ga (\mu} \sb^{\nu) \de}
\\
&
\pt{space}
-\sb^{\al (\mu} \et^{\nu ) \ga} \et^{\be\de}
-\et^{\al (\mu} \sb^{\nu ) \ga} \et^{\be\de}
-\et^{\al (\mu} \et^{\nu ) \ga} \sb^{\be\de}
\\
&
-\sb^{\be (\mu} \et^{\nu ) \ga} \et^{\al\de}
-\et^{\be (\mu} \sb^{\nu ) \ga} \et^{\al\de}
-\et^{\be (\mu} \et^{\nu ) \ga} \sb^{\al\de}
\Big) \prt_\ga \prt_\de .
\eal
\label{Khat}
\eeq
The first line contains the terms present in standard linearized General Relavity, namely the terms in $G^\mn$.
The next 3 lines are the leading order corrections from the mSME $\sb^\mn$ coefficients \cite{bk06,km16}.
It is useful to note that this operator can be cast into a simpler form.
We define a two tensor $\tg^\mn = \et^\mn + \sb^\mn$. 
Then to leading order in $\sb^\mn$ it can be shown that
\beq
\bal
{\hat K}^{\mu\nu\al\be} &= \frac 12 \Big( 
-\tg^\mn \tg^\ab \tg^\gd 
+ \tg^{\al (\mu} \tg^{\nu )\be} \tg^\gd 
+\tg^\mn \tg^{\al\ga} \tg^{\be\de}
\\
&
+ \tg^\ab \tg^{\mu\ga} \tg^{\nu\de}
-\tg^{\al (\mu} \tg^{\nu ) \ga} \tg^{\be\de}
-\tg^{\be (\mu} \tg^{\nu ) \ga} \tg^{\al\de}
\Big) \prt_\ga \prt_\de ,
\label{Khat2}
\eal
\eeq
which resembles the standard linearized terms but with an apparent modified metric $\tg^\mn$, 
as pointed out in \cite{kt11}.

We perform a general coordinate transformation similar to that in section \ref{photon sector application}.
We require the coordinate transformation to satisfy \rf{gchange}, 
with $\tg^\mn = \et^\mn + \sb^\mn$.
Treating the quantities in \rf{Khat2} as tensors in a flat background, 
the field equations in the 
${\overline x}^\mu$ coordinates take the form
\beq
\bal
&\frac 12 \Big( 
-\et^\mn \et^\ab \et^\gd 
+ \et^{\al (\mu} \et^{\nu )\be} \et^\gd 
+\et^\mn \et^{\al\ga} \et^{\be\de}
\\
&
+ \et^\ab \et^{\mu\ga} \et^{\nu\de}
-\et^{\al (\mu} \et^{\nu ) \ga} \et^{\be\de}
-\et^{\be (\mu} \et^{\nu ) \ga} \et^{\al\de}
\Big) 
\\
&
\hspace{4cm}
\times
{\overline \prt}_\ga {\overline \prt}_\de \oh_{\al\be} &= \ka {\overline \ta}^\mn .
\label{xbarFE}
\eal
\eeq
Thus in this coordinate system, 
the field equations appear as conventional GR with a modified source 
${\overline \ta}^\mn $.
Note that in the ${\xb^\mu }$ coordinate system $\et^\mn$ is {\it not} the background spacetime metric.
The barred notation indicates the coordinate system and is not to be confused with the common trace-reversed bar notation.

Next we exploit the gauge freedom in equation \rf{xbarFE} and choose
\beq
\et^{\al\be} {\overline \prt}_\al \oh_{\be\ga} = \frac 12 {\overline \prt}_\ga ( \et^{\al\be} \oh_{\al\be}).
\label{GravGauge}
\eeq
Note that ${\overline \prt}_\la \et^\mn =0$ holds in the ${\xb^\mu }$,
as can be verified by transforming the partial derivative term in
$\prt_\ka \et^\mn =0$.
With the gauge choice, 
the field equations become
\beq
\fr 12 \et^\ab {\overline \prt}_\al {\overline \prt}_\be \Pi^\mn = -\ka {\overline \ta}^\mn, 
\label{gfFEbar}
\eeq
where $\Pi^\mn$ stands for
\beq
\Pi^\mn = \et^{\mu\al} \et^{\nu\be} \oh_\ab - \fr 12 \et^\mn ( \et^{\al\be} \oh_{\al\be}). 
\label{Pi}
\eeq
The standard wave operator (in ${\xb^\mu}$ coordinates) appears in \rf{gfFEbar}, 
thus we can use the standard (particular) solution,
\beq
\Pi^\mn  = \fr {\ka}{\pi} \int d^4 \xb^\prime \de \big( -\et_{\al\be} (\xb -\xb^\prime)^\al (\xb -\xb^\prime)^\be \big) {\overline \ta}^\mn.
\label{PiSoln}
\eeq
Using the Minkowski metric $\et_\mn$ we can obtain $\oh_\mn$ from \rf{PiSoln}:
\beq
\bal
\oh_\mn &= \fr {\ka}{\pi} \int d^4 \xb^\prime \de \big( -\et_{\al\be} (\xb -\xb^\prime)^\al (\xb -\xb^\prime)^\be \big)  
\\
&
\pt{spacespace}
\times
\left( \et_{\ga\mu} \et_{\de\nu} {\overline \ta}^{\ga\de}  - 
\frac 12 \et_\mn \et_\gd {\overline \ta}^\gd \right).
\label{ohSoln}
\eal
\eeq

Using the coordinate transformation rule
$h_{\ka\la} = (\prt \xb^\mu /\prt x^\ka) (\prt \xb^\nu /\prt x^\la) \oh_\mn $,
we can find the solution in the original coordinates, 
similar to the approach for the vector potential in the steps leading to \rf{Asoln}.
This yields
\beq
\bal
h_\mn &= 
\fr {\ka} {\pi \sqrt{-\tg }} 
\int d^4x^\prime 
\de \left( -(\tg^{-1})_{\al\be} (x-x^\prime )^\al (x-x^\prime )^\be \right)
\\
&
\pt{space}
\times
\left( 
(\tg^{-1})_{\ga\mu} (\tg^{-1})_{\de\nu} \ta^\gd  
-\frac 12 (\tg^{-1})_\mn (\tg^{-1})_\gd \ta^\gd 
\right).
\label{hSoln}
\eal
\eeq
We have also directly verified this solution \rf{hSoln} by inserting it into equation \rf{FE1} and checking that this equation is solved to leading order in $\sb_\mn$.  Note that the gravitational wave from the source propagates along the modified light cone $-(\tg^{-1})_\mn (x-x^\prime )^\mu (x-x^\prime )^\nu$.
This is consistent with the propagation studies of other works \cite{km16,Mewes:2019,ONeal-Ault:2021uwu}.
What is new here is that we can calculate directly the effects of a given source
on the metric fluctuations and the measured spacetime curvature in a GW detector.

In a leading order approximation, 
we have $(\tg^{-1})_\mn =\et_\mn -\sb_\mn$.
If we expand the delta function as done for the scalar case and vector case above, 
and we restrict attention to leading order in $\sb_\mn$, 
then we obtain the result
\beq
\bal
h_\mn &= 
\fr {\ka} {2\pi \sqrt{-\tg }} 
\int d^3x^\prime 
\fr {1}{\tR} \Big( 
\ta_\mn - 2 \ta^\al_{\pt{\al}(\mu} \sb_{\nu )\al} 
\\
&
-\frac 12 \et_\mn (\ta^\al_{\pt{\al}\al}  - \sb_\ab \ta^\ab ) +\frac 12 \sb_\mn \ta^\al_{\pt{\al}\al}
\Big) (\tt_R, \vec r^\prime),
\label{hLO}
\eal
\eeq
with $\tt_R$ and $\tR$ defined as in \rf{ttR} and \rf{tildeR}, 
but with $k^\mn \rightarrow \sb^\mn$.
This solution is valid in the gauge
\beq
(\et^\mn + \sb^\mn ) \prt_\mu h_{\nu\la} =
\frac 12 \prt_\la 
( \et^\mn + \sb^\mn ) h_{\mn},
\label{gauge2}
\eeq
which is not the usual harmonic gauge but reduces to the harmonic gauge when $\sb_\mn=0$.
This gauge was also used for vacuum studies of GW effects of Lorentz violation \cite{Xu:2019fyt}

\subsection{Expansion of solution}
\label{expansion}

At this stage we employ the far field expansion, 
similar to \rf{AWZ}.
First we abbreviate the terms in parenthesis inside the integral \rf{hLO} as $\Th_\mn$.
We seek the solution for $h_\mn$ in the far field or wave zone, 
that is, 
the field point lies very far away from the source.
However, 
we must integrate over the
near zone $\cN$ and wave zone $\cW$ in this case because $\ta_\mn$ does not have compact support and exists in both regions:
\beq
\bal
h_\mn &= \fr {4G} {2\pi \sqrt{-\tg }} \Big( 
\int_\cN d^3x^\prime \fr {\Th_\mn (\tt_R, \vec r^\prime)}{\tR}
\\
&
+
\pt{space}
\int_\cW d^3x^\prime \fr {\Th_\mn (\tt_R, \vec r^\prime)}{\tR}
\Big).
\label{NW}
\eal
\eeq
In this paper, 
we attempt only the first integrals, 
so we seek $(h_\cN)_\mn$.
The general solution for the $\cN$ zone integrals can be put into an expansion form
like \rf{psiWZ}:
\beq
(h_N)_\mn = \fr {4G} {\sqrt{-\tg }}
\sum_{l=0}^{\infty} 
\fr {(-1)^l} {l!}
\prt_L \left( \fr {1}{\tr} 
\int_\cN d^3 r^\prime \Th_\mn (\tt_r, \vec r^\prime) r^{\prime L} \right).
\label{GWN}
\eeq
In the lowest order approximation, 
we can use the conservation law $\prt_\mu \ta^\mn = 0$, 
to express integrals in \rf{GWN} in terms of $\int d^3 r \ta^{ij}$ which can be turned 
into a inertial tensor expression $(1/2) d^2/dt^2 \int d^3 r \ta^{00} r^i r^j = (1/2) {\ddot I}^{ij}$ up to surface terms (on the surface at radius ${\cal R}$ lying between $\cN$ and $\cW$).

The radiation fields $(h_\cN)_\mn$, expressed to PNO(4), 
are given by
\beq
\bal
(h_\cN)_{00} &= \fr {G}{\tr} \big( \de_{jk} + N_j N_k 
+\sb_{00} (\de_{jk} + 2 N_j N_k) 
\\
&
\pt{spa}
+2 \sb_{0j} N_k -\sb_{jk}\big) \ddot{I}^{jk},
\\
(h_\cN)_{0j} &= \fr {G}{\tr} \big( -2 \de_{jk} N_l (1+\sb_{00})  -2 \sb_{0k}\de_{jl} + \sb_{0j}\de_{kl}
\\
&
\pt{spa}+
\sb_{0j} N_k N_l + 2 \sb_{jk} N_l \big) \ddot{I}^{kl},
\\
(h_\cN)_{jk} &= \fr {G}{\tr} 
\big(  
2 \de_{jl} \de_{km} - \de_{jk} \de_{lm} 
+\de_{jk} ( 1+\sb_{00} ) N_l N_m 
\\
&
-4 \sb_{0(j} \de_{k)m} N_l 
+ 2 \sb_{0m}\de_{jk} N_l 
+\de_{jk} \sb_{lm} +\de_{lm} \sb_{jk} 
\\
&
\pt{space}- 4 \sb_{l(j} \de_{k)m} - \sb_{jk} N_l N_m 
\big) 
\ddot{I}^{lm},
\label{componentsWZ}
\eal
\eeq
The measured curvature in a gravitational wave detector can be taken as the components $R_{0j0k}=(1/2)(\prt_0 \prt_j h_{0k}+\prt_0 \prt_k h_{0j} - \prt_j \prt_k h_{00} - \prt^2_0 h_{jk})$.  
Normally, GR, in the usual transverse traceless gauge, 
one can obtain the curvature directly from $h_{jk}$ alone.
The gauge choice here does not generally allow that.  
However, the linearized curvature is gauge independent,
hence our focus for the observable effects.
We find the curvature components to be 
\beq
\bal
R_{0j0k} &= \fr {G}{\tr} 
\big[ 
\frac 12 \de_{jk} \de_{lm} - \de_{l(j} \de_{k)m} 
-\frac 12 \big(\de_{jk}  N_l N_m 
\\
&
+\de_{lm} N_j N_k 
-4 \de_{l (j} N_{k)} N_m \big) 
(1+ \sb_{00})
\\
&
- \frac 12 N_j N_k N_l N_m (1+2 \sb_{00})
+2\sb_{0(j} \de_{k)m} N_l
\\
&
+ \sb_{0m} \de_{jk} N_l 
 +2 \sb_{0m} \de_{l(j} N_{k)} 
-\sb_{0(j} N_{k)}\de_{lm}
\\
&
- \sb_{0m} N_j N_k N_l 
- \sb_{0(j} N_{k)} N_l N_m
\\
&
-\frac 12 \big(  \de_{jk} \sb_{lm} 
+ \de_{lm} \sb_{jk} 
- 4 \sb_{l(j} \de_{k)m} - \sb_{jk} N_l N_m 
\\
&
\pt{+\frac 12}
- \sb_{lm} N_j N_k
+4 \sb_{l(j } N_{k)} N_m 
\big) \big] (\overset{(4)}{I})^{lm}
\label{curvature}
\eal
\eeq

In general metric models of gravity beyond GR, 
there are up to six possible polarizations for gravitational waves \cite{Will:2018bme}.
In the presence of Lorentz violation 
in \rf{curvature},
five of the six polarizations show up.
We can also establish the question of their independence, 
and the number of degrees of freedom.
We will identify the polarizations by taking the trace and projections
of $R_{0i0j}$.
Since the wave travels in a direction along $N_i$ we will adopt a spatial basis
$\{ {\bf e}_1, {\bf e}_2, \bf N/\sqrt{N^i N_i} \}$, 
where the basis vectors ${\bf e}_1$ and ${\bf e}_2$ span the plane perpendicular to $N_i$.
[Note that, 
due to the coefficients in \rf{N1}, 
${\bf e}_1$ and ${\bf e}_2$ are not perpendicular to $\hat n$,
except at zeroth order in the coefficients.]

First we calculate the trace of the curvature tensor $R_{0j0k}\de^{jk}$.
It will be convenient to introduce a traceless $(\sb_{tr})_{ij}=\sb_{ij} - (1/2)\de_{ij} \sb_{00}$, 
where we use the assumption $\sb^\mu_{\pt{\mu}\mu} = \sb_{jj} - \sb_{00}=0$.
The trace can be simplified to
\beq
R_{0j0j} = \fr {G}{\tr} \left[ (\sb_{tr})_{\perp ij}+ \frac 12 (\sb_{tr})_{nn} (\de_{ij} - n_i n_j ) \right] 
(\overset{(4)}{I})^{ij},
\label{trace}
\eeq
where projections of quantities along $\hat n$ are denoted with the index $n$ 
and $\perp$ indicates a projection of a tensor perpendicular to $\hat n$ like
$(V_\perp)_i = V_i - n_i V_j n^j = V_i - n_i V_n$.

Next we find the double projection of the curvature along the wave propagation direction $N_i N_j R_{0i0j}$.  
We find
\beq
N^i N^j R_{0i0j}=0+O(\sb^2),
\label{nnR}
\eeq
thus there is no leading order polarization along this projection.
%This means that a sphere of masses would not elongate along the $\pm N_i$ directions in a cylindrically symmetric  way.
However, 
the components $R_{0i0j} N^i (e_a)^j$ do not vanish (where $a=1,2$).
They are given by
\beq
\bal
R_{0i0j} N^i (e_a)^j &= \fr {G}{\tr} \big[ \frac 12 \big( (\sb_{tr})_{an}+ \sb_{0a} \big)  (\de_{ij} - n_i n_j ) 
\\
&
\pt{space}
+(e_a)_j \big( (\sb_{tr})_{nk_{\perp}} + \sb_{0k_\perp } \big)  \big] (\overset{(4)}{I})^{ij}.
\label{neproj}
\eal
\eeq

Finally, 
we display projections along the transverse directions ${\bf e}_1$ and ${\bf e}_2$, 
the ones that normally are called ``plus" and ``cross".
They are given by 
\beq
\bal
R_{0102} &= 
\fr {G}{\tr} \big[
- (e_1)_i (e_2)_j (1-\frac 23 \sb_{00} ) 
+(\sb_{tr})_{1i} (e_2)_j
\\
&
\pt{space}
+(\sb_{tr})_{2i} (e_1)_j
-\frac 12 (\sb_{tr})_{12}
(\de_{ij} - n_i n_j)
\\
&
\pt{space}
+\sb_{01} (e_2)_i n_j + \sb_{02} (e_1)_i n_j \big] (\overset{(4)}{I})^{ij},
\\
R_{0202} - R_{0101} 
&=
\fr {G}{\tr} \big[ 
(e_{1i} e_{1j} - e_{2i} e_{2j})(1-\frac 23 \sb_{00})
\\
&
\pt{space}
+ \frac 12 ( (\sb_{tr})_{11} - (\sb_{tr})_{22} ) (\de_{ij} - n_i n_j)
\\
&
\pt{space}
-2 ( (\sb_{tr})_{1i} e_{1j} -(\sb_{tr})_{2i} e_{2j} )
\\
&
\pt{space}
-2 ( \sb_{01} e_{1i} n_j - \sb_{02} e_{2i} n_j )
\big] 
(\overset{(4)}{I})^{ij},
\label{pluscross}
\eal
\eeq
where the subscripts $1$ and $2$ imply projection with the corresponding unit vectors.

In GR, 
all projections but $R_{0102}$ and $R_{0202}-R_{0101}$ vanish (when $\hat n$ is the $3$ direction), 
as can be seen by setting all $\sb_\mn$ coefficients to zero.
In the presence of the coefficients it appears $3$ additional polarizations arise.
The results above indicate that the coefficients $\sb_\mn$, 
in addition to showing up in weak-field gravity scenarios like solar system tests,
and affecting the speed of gravitational waves,
can also affect the observed polarization content in a GW detector.
The additional polarizations are of order $\sb$.  
Given the sensitivity of the current detectors to the strength of the GW signals above noise level of a couple orders of magnitude, 
it seems that these additional effects could be observed if $\sb \sim 10^{-2}$.
% check this
Constraints on $\sb_\mn$ already exist below parts in $10$ billion (e.g., from lunar laser ranging \cite{Bourgoin:2016ynf}), 
so we do not expect observable effects via this method.
However, 
we have not studied the effects of higher order terms in the action, 
and many such coefficients theirin are not well constrained, of not constrained at all.
This points to a future general study of all coefficients in the gravity sector. 

While we do not discuss details, 
the nonzero projections found are equivalent to some of the Newman-Penrose projections of the curvature tensor \cite{np62, Will:2018bme}.
Specifically we have 
\beq
\bal
R_{0j0j} &=-2\Ph_{22},
\\
N^i N^j R_{0i0j} &=-6 \Ps_2 = 0, 
\\
R_{0i0j} N^i (e_1)^j &=-2\sqrt{2} {\it Re} {\Ps_3},
\\
R_{0i0j} N^i (e_2)^j &=2\sqrt{2} {\it Im} {\Ps_3},
\\
R_{0202}-R_{0101} &= 2 {\it Re} {\Ps_4},
\\
R_{0102} &= {\it Im} {\Ps_4}.
\label{np}
\eal
\eeq
The reader can refer to depictions of the effect of these modes on a sphere of test masses in
Refs.\ \cite{Will:2018bme,Wagle:2019mdq}.

\section{Generalizations and Summary}

We comment here on how the results in this work can be generalized and improved upon.
One needs a complete evaluation of \rf{hLO} including the contributions from the wave zone integrals
$(h_\cW)_\mn$.

Note that we have not considered in detail the effects of the Nambu-Goldstone and massive modes that may occur from a spontaneous symmetry breaking scenario \cite{ks89bb,bk05,bk08}.
A general description of the dynamical terms for the $\sb_\mn$ coefficients, 
when they arise as a vacuum expectation value of a dynamical tensor $s_\mn$
has been published, 
but not yet studied in the GW context \cite{b21}.
Studies of toy models, 
like the bumblebee model \cite{ks89bb}, 
in these SSB scenarios do exist.
In Ref.\ \cite{Liang:2022hxd} they studied the number of 
propagating degrees of freedom arising from the
gravitational fluctuations and the vector field fluctuations.
Studies of the effects of the additional field fluctuations $s_\mn$ on the 
observed signal in GW detectors is an open problem.

\begin{acknowledgments}
This work was supported by the National Science Foundation under grant number 2207734.
\end{acknowledgments}

\appendix

\section{Perturbative versus exact solutions}

It is useful to compare the methods above with other methods that involve approximate solutions.
This has been carried out successfully in slow motion, weak field scenarios where wave behavior is not dealt with \cite{bk06}.
%cite more
When the full effects of time derivatives is included, 
and we are looking for complete ``inhomogeneous" solutions 
(not vacuum propagation), subtleties arise as we point out in the section.

To illustrate, 
we focus on the scalar wave equation case in \rf{scalar}.
The philosophy behind perturbative approaches is to seek solutions in powers of the small coefficients $k_\mn$.
For instance we assume
the solution can be written
$\ps=\ps^{(0)} + \ps^{(1)}+...$, 
with $(n)$ indicating order in powers of $k_\mn$.
To zeroth and first order in $k_\mn$, 
we have the two equations to solve:
\beq
\bal
\Box \ps^{(0)} &= -\rh,
\\
\Box \ps^{(1)} &= -k^\mn \prt_\mu \prt_\nu \ps^{(0)}. 
\label{firstorder}
\eal
\eeq

The formal solutions, 
assuming no homogeneous solutions and choosing the retarded time Green function, 
can be written
\beq
\bal
\ps^{(0)} &=  
\int d^3 r^\prime \fr {\rh (t-R,\vec r^\prime ) }{R},
\\
\ps^{(1)} &= 
\int d^3 r^\prime \fr {k^\mn \prt^\prime_\mu \prt^\prime_\nu \ps^{(0)} }{R}
\label{pertsolns1}
\eal
\eeq
The first solution is the conventional scalar one.
The second of these equations involves a field on the right-hand side, 
a field that in principle exists over all regions of space, 
thus the right-hand side is not guaranteed compact support.
Note that even in GR, 
order by order integration of the formal wave solution also contains source terms that composed of fields existing far from the source \cite{mtw,pw14}, 
since GR is a self-interacting field theory.
Such terms can be dealt with in GR and form part of the complete causal and properly behaved solution.
%cite will and blanchet, etc
Its not clear in equations like \rf{firstorder}
if this program will work.

Rather than solve \rf{pertsolns1} directly, 
we present an alternative.
Applying the $\Box$ operator to the equation for $\ps^{(1)}$ in \rf{firstorder}, 
we obtain,
\beq
\Box^2 \ps^{(1)} 
= k^\mn \prt_\mu \prt_\nu \rh,
\label{alteqn}
\eeq
where now the right-hand side is a source with compact support.
The price to pay is the nonlocal Green function for the fourth order (in derivatives) operator on the left side.
Green functions solving the point source equation $\Box^2 G = -\de^4 (x-x^\prime)$ take the simple form
$G_{nl}(x,x^\prime) = -(1/16\pi) {\rm sgn} (t-t^\prime \pm R)$.
Here ${\rm sgn} (x)=\pm 1$: positive if $x>1$ and negative if $x<1$.
This result can be derived from standard sources, 
for example, 
by taking the Fourier time transform of the position space Green functions in Ref.\ \cite{Lindell}.
Note that $G_{nl} (x,x^\prime)$ 
is highly nonlocal, 
it exists in all regions of spacetime: 
it is positive in the forward light cone, negative elsewhere.\footnote{The static limit of this Green function, 
which is just proportional to the distance $R$, 
is used ubiquitously in the literature for various post-Newtonian applications \cite{bk06,bh17,km17,Will:2018bme}.
Green functions for nonlocal operators have been discussed elsewhere, 
for instance Refs. \ \cite{Pais:1950za,Eliezer:1989cr,Bailey_2023}.}%add chandrasekhar cite
When derivatives are applied to $G(x,x^\prime)$, 
the light cone delta function emerges.  
For example, 
$\Box G_{nl}(x,x^\prime) = \de (t-t^\prime \pm R)/(4\pi R)$, 
which gives the usual Green function.

The solution to \rf{alteqn} then takes the form
\beq
\ps^{(1)} = -\int d^4 x^\prime 
G_{nl} (x,x^\prime ) 
k^\mn \prt^\prime_\mu \prt^\prime_\nu \rh^\prime
+\ps^{(1)}_H,
\label{nlsoln}
\eeq
where $\ps^{(1)}_H$ is a homogeneous solution satisfying $\Box^2 \ps^{(1)}_H=0$.
Convergence of the integrals for the infinite domain in \rf{nlsoln} depends on the source function $\rh$ asymptotic properties and the bounding surface of the four-dimensional integral. 
We assume $\rh$ is localized in space, 
vanishing outside some finite radius.
The time behavior is another matter.
One can always introduce a bounding surface, 
for example, 
the volume is the spacetime between two spacelike hypersurfaces at fixed values of time $t_2$ and $t_1$ (see figure 5.3a in Ref.\ \cite{mtw}).
Alternatively one can introduce an artificial exponential time falloff for the density $\rh \rightarrow \rh e^{-\ep |t|}$ to ensure the source vanishes as time approaches $\pm \infty$.

Assuming that a modification is applied to \rf{nlsoln}, 
so that it is finite, 
we proceed with integration by parts with the $\prt^\prime_\mu \prt_\nu^\prime$ derivatives.
The surface terms can either be eliminated by a choice of the homogeneous solution $\ps^{(1)}_H$ or they can be shown to vanish on the boundary with mild assumptions.
We obtain
\beq
\ps^{(1)} = -\int d^4 x^\prime 
k^\mn \prt^\prime_\mu \prt^\prime_\nu G_{nl} (x,x^\prime ) 
\rh^\prime,
\label{nlsoln2}
\eeq
and now the derivatives of the Green function $\sim {\rm sgn} (t-t^\prime \pm R)$
will always involve a delta function along the past light cone.

The result in \rf{nlsoln} is best matched to the ``exact" solution \rf{psi} 
by breaking up the summation into space and time components.
To evaluate the derivatives on the Green function we use $sgn(u)=1-2\Th(u)$, 
where $\Th(u)$ is the usual step function and $d/du \Th (u) = \de (u)$.
To abbreviate we let $u=t-t^\prime - R$.
This yields
\beq
\bal
\ps^{(1)} &= \fr {1}{8\pi} \int d^4 x^\prime \fr {d}{du} \de (u) \rh^\prime 
\big( k_{00} + 2 k_{0j} {\hat R}^j + k_{jk} {\hat R}^j {\hat R}^k \big)
\\
&
-\fr {1}{8\pi} \int d^4 x^\prime \de (u) \rh^\prime k_{jk} 
\big( \de^{jk} - {\hat R}^j {\hat R}^k \big).
\label{nlsoln3}
\eal
\eeq
The first term can be evaluated with integration by parts 
after changing variables from $u$ to $t^\prime$, 
while the second term can simply be evaluated at the argument of the delta function.
Adding in the zeroth order solution, 
we collect the terms in a suggestive form:
\beq
\bal
\ps^{(0)}+\ps^{(1)} &=  \int d^3 r^\prime  \fr {1}{4\pi R} \big[ \rh^\prime_r 
\\
&
\pt{spacespac}+ {\dot \rh}^\prime_r \frac 12 \big( k_{00} R  + 2 k_{0j} R^j + k_{jk} R^j {\hat R}^k\big)
\\
&
\pt{spacespac}
+ \rh^\prime_r \frac 12 k_{jk} (\de^{jk} - {\hat R}^j {\hat R}^k ) \big],
\label{nlsoln4}
\eal
\eeq
where the subscript on $\rh$ indicates evaluation at the retarded time $t-R$.

The first term is the unperturbed solution for when $k_\mn=0$.
The second term has an unconventional dependence on the distance $R$;
far from the source the potential has no $1/r$ suppression.
However, 
the second term can be re-interpreted as the first order term in the Taylor expansion of 
the modified retarded time argument \rf{ttR}:
$\rh (\tt_R) = \rh (t_R)+ {\dot \rh} (\tt_R - t_R)$.
This is because it can be shown, 
to leading order in the coefficients $k_\mn$,
$\tt_R - t_R = \frac 12 ( k_{00} R  + 2 k_{0j} R^j + k_{jk} R^j {\hat R}^k ) + O(k^2)$.
The third term has the usual $1/r$ suppression outside the source region.
When comparing to the exact solution \rf{psi}, 
the third term can be understand as the leading term from a series expansion 
the modified distance ${\tilde R}$ in \rf{tildeR};
$\tilde R = R (1 + \frac 12 k_{00} - \frac 12 k_{jk} {\hat R}^j {\hat R}^k + O(k^2))$.
To summarize we have shown the match of approximate and exact solutions:
\beq
\ps^{(0)}+\ps^{(1)} = \ps +O(k^2).
\label{match}
\eeq
Note that more generally we can expect that using a first order perturbative method 
could result in terms that appear nonlocal but can be re-interpreted 
as a modification to the wave propagation via the correct modified retarded time.

\bibliography{refs}

\end{document}


