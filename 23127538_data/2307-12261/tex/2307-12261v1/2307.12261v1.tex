
\documentclass[12pt,reqno]{amsart}
\usepackage{enumerate, latexsym, amsmath, amsfonts, amssymb, amsthm, graphicx, color}
 \usepackage{hyperref}
\hypersetup{hypertex=true,
            colorlinks=true,
            linkcolor=blue,
           anchorcolor=blue,
            citecolor=blue}
\usepackage{amsmath}
 \textwidth=13.5cm
 \textheight=22cm
\hoffset=-1cm\voffset-0.5truecm
\hoffset=-1cm\voffset-0.5truecm
\def\pmod #1{\ ({\rm{mod}}\ #1)}
\def\Z{\Bbb Z}
\def\N{\Bbb N}
\def\G{\Bbb G}
\def\Q{\Bbb Q}
\def\zp{\Z^+}
\def\R{\Bbb R}
\def\C{\Bbb C}
\def\F{\Bbb F}
\def\L{\Bbb L}
\def\M{\Bbb M}
\def\l{\left}
\def\r{\right}
\def\bg{\bigg}
\def\({\bg(}
\def\){\bg)}
\def\t{\text}
\def\f{\frac}
\def\mo{{\rm{mod}\ }}
\def\lcm{\rm{lcm}}
\def\ord{{\rm ord}}
\def\sgn{{\rm sgn}}
\def\adj{{\rm adj}}
\def\cs{\ldots}
\def\ov{\overline}
\def\ls{\le}
\def\gs{\geqslant}
\def\se {\subseteq}
\def\sm{\setminus}
\def\bi{\binom}
\def\al{\alpha}
\def\be{\beta}
\def\ga{\gamma}
\def\ve{\varepsilon}
\def\lr{\Leftarrow}
\def\ra{\rightarrow}
\def\Lr{\leftrightarrow}
\def\ar{\Rightarrow}
\def\Ar{\Leftrightarrow}
\def\eq{\equiv}
\def\Da{\Delta}
\def\da{\delta}
\def\la{\lambda}
\def\La{\Lambda}
\def\ta{\theta}
\def\AP{\rm{AP}}
\def\va{\varphi}
\def\supp {\rm supp}
\def\colon{{:}\;}
\def\u{{\bf u}}
\def\v{{\bf v}}
\def\Proof{\noindent{\it Proof}}
\def\Ack{\medskip\noindent {\bf Acknowledgments}}
\theoremstyle{plain}
\newtheorem{theorem}{Theorem}
\newtheorem{fact}{}
\newtheorem{lemma}{Lemma}
\newtheorem{corollary}{Corollary}
\newtheorem{conjecture}{Conjecture}
\theoremstyle{definition}
\newtheorem*{acknowledgment}{Acknowledgment}
\theoremstyle{remark}
\newtheorem{remark}{Remark}
\newtheorem{example}[theorem]{Example}
\renewcommand{\theequation}{\arabic{section}.\arabic{equation}}
\renewcommand{\thetheorem}{\arabic{section}.\arabic{theorem}}
\renewcommand{\thelemma}{\arabic{section}.\arabic{lemma}}
\renewcommand{\thecorollary}{\arabic{section}.\arabic{corollary}}
\renewcommand{\theconjecture}{\arabic{section}.\arabic{conjecture}}
\renewcommand{\theremark}{\arabic{section}.\arabic{remark}}
\makeatletter
\@namedef{subjclassname@2010}{%
  \textup{2010} Mathematics Subject Classification}
\makeatother
 \vspace{4mm}


\newcommand\blfootnote[1]{%
	\begingroup
	\renewcommand\thefootnote{}\footnote{#1}%
	\addtocounter{footnote}{-1}%
	\endgroup
}
\begin{document}


\title{On Determinants of cyclotomic matrices involving Jacobi sums}

\author[L.-Y. Wang, H.-L. Wu and H. Pan]{Li-Yuan Wang, Hai-Liang Wu, Hao Pan*}




\address {(Li-Yuan Wang) School of Physical and Mathematical Sciences, Nanjing Tech University, Nanjing 211816, People's Republic of China}
\email{\tt wly@smail.nju.edu.cn}

\address {(Hai-Liang Wu) School of Science, Nanjing University of Posts and Telecommunications, Nanjing 210023, People's Republic of China}
\email{\tt whl.math@smail.nju.edu.cn}

\address {(Hao Pan) School of Applied Mathematics, Nanjing University of Finance and Economics, Nanjing 210046, People's Republic of China}
\email{\tt haopan79@zoho.com}
\begin{abstract}
In this paper, by using Cauchy-Binet formula and charcter sums over finite fields, we evaluate determinants of two cyclotomic matrices involving Jacobi sums $J(\chi^i,\chi^j)$. For example, we show that 
$$\det[J(\chi^i,\chi^j)]_{1\le i,j\le p-2}=(p-1)^{p-3},$$
and that 
$$\det [J(\chi^{2i},\chi^{2j})]_{1\le i,j\le (p-3)/2}
=\frac{p+(-1)^{\frac{p+1}{2}}}{4} \left(\frac{p-1}{2}\right)^{\frac{p-5}{2}}(-1)^{\frac{p+1}{2}},$$
where $p$ is an odd prime and $\chi$ is a generator of the group of all multiplicative characters of $\mathbb{F}_p$.
\end{abstract}

\thanks{2020 {\it Mathematics Subject Classification}.
Primary 11C20; Secondary 11L05, 11R18.
\newline\indent {\it Keywords}. finite fields, determinants, Jacobi sums.
\newline \indent
This research was supported by the National Natural Science Foundation of China (Grant Nos. 12071208, 12201291 and 12101321) and the Natural Science Foundation of the Higher Education Institutions of Jiangsu Province (21KJB110001, 21KJB110002). 
\newline\thanks{*Corresponding author.}}
\maketitle
\section{Introduction}	
\setcounter{lemma}{0}
\setcounter{theorem}{0}
\setcounter{corollary}{0}
\setcounter{remark}{0}
\setcounter{equation}{0}
\setcounter{conjecture}{0}
\subsection{Background}
Determinants of cyclotomic matrices have extensive applications in algebra, number theory and combinatorics. Readers may refer to the survey papers \cite{K1,K2} for recent progress on this topic. 

The earliest research on this issue came from Lehmer \cite{Lehmer} and Carlitz \cite{carlitz}. For instance, let $p$ be an odd prime and let $(\cdot/p)$ be the Legendre symbol, i.e., the quadratic multiplicative character of the finite field $\F_p$. Carlitz showed that 
$$\det\left[\left(\frac{j-i}{p}\right)\right]_{1\le i,j\le p-1}=p^{(p-3)/2}.$$
Along this line, Chapman \cite{chapman} initiated the study of several variants of Carlitz's results. Let $$\varepsilon_p^{(2-(\frac{2}{p}))h_p}=a_p+b_p\sqrt{p},\ a_p,b_p\in\mathbb{Q},$$
where $\varepsilon_p>1$ and $h_p$ denote the fundamental unit and class number
of the real quadratic field $\mathbb{Q}(\sqrt{p})$. Chapman conjectured that 
$$\det\left[\left(\frac{j-i}{p}\right)\right]_{1\le i,j\le \frac{p+1}{2}}=\begin{cases}
	-a_p & \mbox{if}\ p\equiv 1\pmod4,\\
	1    & \mbox{if}\ p\equiv 3\pmod4.
\end{cases}$$
This conjecture was later known as the ``evil determinant conjecture". This tough conjecture was later comfirmed by Vsemirnov \cite{K1,K2}. In 2019, Sun \cite{S19} further investigated some variants of the above deteminants. For example, Sun proved that $$-\det\left[\left(\frac{i^2+j^2}{p}\right)\right]_{1\le i,j\le p-1}$$
is a quadratic residue modulo $p$. Later the second author \cite{W21} revealed the relationship between this kind of determinants and elliptic curves over finite fields. For example, it is proved that for any $c\in\F_p$ with $c\neq\pm2$, 
$$\det\left[\left(\frac{i^2+cij+j^2}{p}\right)\right]_{0\le i,j\le p-1}=0$$
if the elliptic curve defined by the equation $y^2=x(x^2+cx+1)$ is a supersingular elliptic curve over $\F_p$. 

\subsection{Main Results}
 To state our results, we first introduce some notations (readers may refer to \cite{classical,LN}). Let $\F_p^{\times}$ be the group of all nonzero elements of $\F_p$, and let $\widehat{\mathbb{F}_p^{\times}}$ denote the group of all multiplicative character of $\F_p$ (we also define $\psi(0)=0$ for any $\psi\in\widehat{\mathbb{F}_p^{\times}}$). For any $\psi,\lambda\in\widehat{\mathbb{F}_p^{\times}}$, the Jacobi sum $J(\psi,\lambda)$ of $\psi$ and $\lambda$ is defined by
  $$J(\psi,\lambda)=\sum_{a+b=1}^{}\psi(a)\lambda(b)=\sum_{k=0}^{p-1}\psi(k)\lambda(1-k).$$
As $\widehat{\mathbb{F}_p^{\times}}$ is a cyclic group of order $p-1$, each Jacobi sum over $\mathbb{F}_p$ can be written as $J(\chi^i,\chi^j)$ for some $0\le i,j\le p-2$, where $\chi$ is a generator of $\widehat{\mathbb{F}_p^{\times}}$. Hence it is natural to study the matrices with $J(\chi^i,\chi^j)$ as their entry. In this paper, we first consider the following matrix involving the Jacobi sums of non-trivial characters. 

$$J_p:=[J(\chi^i,\chi^j)]_{1\le i,j\le p-2}. $$
By the Galois theory, it is clear that $\det J_p\in\mathbb{Z}$. As our first result, we give the explicit value of $\det J_p$. 

\begin{theorem}\label{thjp}
	Let $p$ be an odd prime and let $\chi$ be a generator of $\widehat{\mathbb{F}_p^{\times}}$. Then
	$$ \det J_p=(p-1)^{p-3}.$$
\end{theorem}

We next consider the matrix involving the Jacobi sums of non-trivial even characters of $\mathbb{F}_p$. Define
$$M_p:=[J(\chi^{2i},\chi^{2j})]_{1\le i,j\le (p-3)/2}.$$
We have the following result.

\begin{theorem}\label{thmp}
  Let $p$ be an odd prime and let $\chi$ be a generator of $\widehat{\mathbb{F}_p^{\times}}$. Then
 $$ \det M_p=\frac{p+(-1)^{\frac{p+1}{2}}}{4}  \(\frac{p-1}{2}\)^{\frac{p-5}{2}}(-1)^{\frac{p+1}{2}}.$$
\end{theorem}




The proofs of Theorem \ref{thjp} and Theorem \ref{thmp}  will be given in Section 2 and Section 3 respectively.
\maketitle

\section{Proof of Theorem \ref{thjp}}
\setcounter{lemma}{0}
\setcounter{theorem}{0}
\setcounter{corollary}{0}
\setcounter{remark}{0}
\setcounter{equation}{0}
\setcounter{conjecture}{0}

{\bf Proof of Theorem \ref{thjp}.}
As $J_p$ is a $(p-2)\times (p-2)$ matrix with the $(i,j)$-th entry
$$
J(\chi^i,\chi^j)=\sum_{k=0}^{p-1}\chi^i(k)\chi^j(1-k)=\sum_{k=2}^{p-1}\chi^i(k)\chi^j(1-k),
$$
one can verify that 
\begin{equation}\label{jmn}
 J_p=MN,
\end{equation}
 where
$$M=[\chi^i(j)]_{1\le i\le p-2,\ 2\le j\le p-1},$$
and
$$N=[\chi^j(1-i)]_{2\le i\le p-1,\ 1\le j\le p-2}.$$


Let $g(t)=t^{p-1}-1$ and let $g'(t)=(p-1)t^{p-2}$ be the derivative of $g(t)$. As $\chi$ is a generator of $\widehat{\mathbb{F}_p^{\times}}$, we have 
$$\{\chi(x): x\in\F_p^{\times}\}=\{e^{2\pi ai/(p-1)}:\ a\in\Z\},$$
and hence 
$g(t)=\prod\limits_{j=1}^{p-1}(t-\chi(j))$. This implies that  $\prod\limits_{j=1}^{p-1}\chi(j)=-1.$
Since $\chi(1)=1$ and $\chi(-1)=-1$, we have $\prod\limits_{j=2}^{p-1}\chi(j)=-1$ and $\prod\limits_{j=1}^{p-2}\chi(j)=1.$

For any integer $1\le k\le p-1$ we have
\begin{align}\label{1nekp-1}
   \prod\limits_{\substack{i=1\\i\ne k}}^{p-1}(\chi (k)-\chi(i))&=\lim\limits_{z\rightarrow \chi(k)}\frac{\prod\limits_{i=1}^{p-1}(z-\chi(i))}{z-\chi(k)}  \nonumber \\
   &=\lim\limits_{z\rightarrow \chi(k)}\frac{x^{p-1}-1}{z-\chi(k)}  \nonumber \\
   &=\lim\limits_{z\rightarrow \chi(k)}(p-1)\cdot z^{p-2} \nonumber \\
   &=(p-1)\cdot\chi(k)^{-1}.
\end{align}
Also, it is clear that 
\begin{align}\label{1ijp-1}
  \(\prod_{1\le i<j\le p-1}(\chi (j)-\chi(i))\)^2
   =& (-1)^{\binom{p-1}{2}}\cdot \prod_{1\le i\ne j\le p-1}(\chi (j)-\chi(i))\  \nonumber \\
   =& (-1)^{\frac{p-1}{2}}\cdot \prod_{1\le j\le p-1}\prod_{i\neq j}(\chi (j)-\chi(i))\  \nonumber \\
  =&(-1)^{\frac{p-1}{2}}\cdot\prod_{1\le j\le p-1}(p-1)\cdot\chi(k)^{-1}\nonumber\\
  =&  (-1)^{\frac{p+1}{2}}\cdot(p-1)^{p-1}.
\end{align}
Now we evaluate the determinants of $M$ and $N$. For $\det M$, we have 
\begin{align}\label{detm}
  \det M &=\left|
  \begin{array}{cccc}
    \chi (2)& \chi (3)&\cdots  & \chi (p-1) \\
    \chi^2 (2)& \chi^2 (3)&\cdots  & \chi^2 (p-1) \\
    \vdots  & \vdots  & \ddots &\vdots \\
     \chi^{p-2} (2)& \chi^{p-2} (3)&\cdots  & \chi^{p-2} (p-1)  \\
  \end{array}
  \right |   \nonumber  \\
  &=\(\prod\limits_{\substack{k=2}}^{p-1}\chi (k)\)\cdot \left|
  \begin{array}{cccc}
    1& 1&\cdots  & 1 \\
    \chi (2)& \chi (3)&\cdots  & \chi (p-1) \\
    \vdots  & \vdots  & \ddots &\vdots \\
     \chi^{p-3} (2)& \chi^{p-3} (3)&\cdots  & \chi^{p-3} (p-1) \\
  \end{array}
  \right |     \nonumber \\
  &=-\prod\limits_{2\le i<j\le p-1}(\chi (j)-\chi(i)) \nonumber \\
  &=-\frac{\prod\limits_{1\le i<j\le p-1}(\chi (j)-\chi(i))}{ \prod\limits_{\substack{j=2}}^{p-1}(\chi (j)-\chi(1))}   \nonumber\\
  &=\frac{\prod\limits_{1\le i<j\le p-1}(\chi (j)-\chi(i))}{p-1}.
\end{align}
For $\det N$, we have 
\begin{align}\label{detn}
  \det N &=\left|
  \begin{array}{cccc}
    \chi (-1)&\chi^2 (-1) &\cdots  & \chi^{p-2} (-1)\\
   \chi (-2) & \chi^2 (-2)&\cdots  & \chi^{p-2} (-2) \\
    \vdots  & \vdots  & \ddots &\vdots \\
    \chi (-(p-2))  & \chi^2 (-(p-2))&\cdots  & \chi^{p-2} (-(p-2))  \\
  \end{array}
  \right |   \nonumber  \\
  &=\(\prod\limits_{\substack{k=1}}^{p-2}\chi^k (-1)\)\cdot \left|
  \begin{array}{cccc}
    \chi (1)&\chi^2 (1) &\cdots  & \chi^{p-2} (1)\\
   \chi (2) & \chi^2 (2)&\cdots  & \chi^{p-2} (2) \\
    \vdots  & \vdots  & \ddots &\vdots \\
    \chi (p-2)  & \chi^2 (p-2)&\cdots  & \chi^{p-2} (p-2)  \\
  \end{array}
  \right |     \nonumber\\
  &=(-1)^{\frac{p-1}{2}}\cdot\(\prod\limits_{\substack{k=1}}^{p-2}\chi (k)\)\cdot\prod\limits_{1\le i<j\le p-2}(\chi (j)-\chi(i))  \nonumber\\
   &=(-1)^{\frac{p-1}{2}}\cdot \prod\limits_{1\le i<j\le p-2}(\chi (j)-\chi(i)) \nonumber\\
   &=(-1)^{\frac{p-1}{2}}\cdot\frac{\prod\limits_{1\le i<j\le p-1}(\chi (j)-\chi(i))}{ \prod\limits_{\substack{i=1}}^{p-2}(\chi (p-1)-\chi(i))} \nonumber   \\
    &=(-1)^{\frac{p+1}{2}}\cdot\frac{\prod\limits_{1\le i<j\le p-1}(\chi (j)-\chi(i))}{p-1}.
\end{align}
By (\ref{jmn}) and (\ref{1ijp-1})--(\ref{detn}) we obtain 
\begin{align}\label{jab}
 \det J_p&=\det M\cdot \det N  \nonumber\\
  &=(-1)^{\frac{p+1}{2}}\cdot\frac{\(\prod\limits_{1\le i<j\le p-1}(\chi (j)-\chi(i))\)^2}{(p-1)^2} \nonumber\\
  &=(p-1)^{p-3}
\end{align}


This completes the proof of Theorem \ref{thjp}.\qed

\section{Proof of Theorem \ref{thmp}}
\setcounter{lemma}{0}
\setcounter{theorem}{0}
\setcounter{corollary}{0}
\setcounter{remark}{0}
\setcounter{equation}{0}
\setcounter{conjecture}{0}


Let $M$ be an $m\times n$  complex matrix with $m\le n$ and let $N$ be an $n\times m$ complex matrix. Set
$$S_m=\{s=(j_1,j_2,\cdots,j_m):\ 1\le j_1<j_2<\cdots<j_m\le n\}.$$
For any $s\in S_m$, we define $M_s$ (respectively $N^s$) to be the $m\times m$ submatrix of $M$ (respectively submatrix of $N$) obtained by deleting all columns (respectively all rows) except those with indices in $s$. We begin with the well-known Cauchy-Binet formula.

\begin{lemma}[Cauchy-Binet formula]\label{cauchy-binet} Let $M,N$ be two complex matrices of sizes $m\times n$ and $n\times m$ respectively with $m\le n$. Then
$$\det(MN)=\sum_{s\in S_m}\det(M_s)\det(N^s).$$
\end{lemma}

{\bf\noindent Proof of Theorem \ref{thmp}.}
As $\chi$ is a generator of $\widehat{\mathbb{F}_p^{\times}}$, we see that $\{\chi(j^2): 1\le j\le (p-1)/2\}$ are precisely all the roots of $z^{\frac{p-1}{2}}-1=0$. %Let $\Zeta=E^{2\Pi I/(P-1)}$
Thus
\begin{equation}\label{roots}
  z^{\frac{p-1}{2}}-1=(z-\chi(1^2))(z-\chi(2^2))(z-\chi(3^2))\cdots (z-\chi((\frac{p-1}{2})^2)).
\end{equation}
By the above one deduces that
\begin{align}\label{prodsquare}
 \prod\limits_{\substack{i=1}}^{(p-1)/2}\chi (i^2)=-1\cdot (-1)^{(p-1)/2}=(-1)^{(p+1)/2}.
\end{align}
For $1\le k\le (p-1)/2$ we have


\begin{align}\label{inek}
   \prod\limits_{\substack{i=1\\i\neq k}}^{(p-1)/2}(\chi (k^2)-\chi(i^2))&=\lim\limits_{z\rightarrow \chi(k^2)}\frac{ \prod\limits_{\substack{i=1}}^{(p-1)/2}(z-\chi(i^2))}{z-\chi(k^2)}  \nonumber \\
   &=\lim\limits_{z\rightarrow \chi(k^2)}\frac{z^{\frac{p-1}{2}}-1}{z-\chi(k^2)}  \nonumber \\
   &=\frac{p-1}{2}\cdot \chi(k^2)^{\frac{p-1}{2}-1}=\frac{p-1}{2}\chi(k^2)^{-1}.
\end{align}
Let 
$$
I_p=\prod_{1\le i<j\le (p-1)/2}(\chi (j^2)-\chi(i^2)).   \nonumber
$$
Then one can verify that 
\begin{align}\label{ip}
  &\prod_{1\le i\ne j\le (p-1)/2}(\chi (j^2)-\chi(i^2))\nonumber\\
  =&\prod_{1\le i<j\le (p-1)/2}(\chi (j^2)-\chi(i^2))\cdot \prod_{1\le j<i \le (p-1)/2}(\chi (j^2)-\chi(i^2)).   \nonumber \\
  =&(-1)^{
    \binom{(p-1)/2}{2} }\cdot I_p^2.
\end{align}
On the other hand, by (\ref{prodsquare}) and (\ref{inek}).
\begin{align}\label{inej}
  &\prod_{1\le i\ne j\le (p-1)/2}(\chi (j^2)-\chi(i^2))\nonumber\\
   =&\prod_{1\le j\le (p-1)/2}\prod_{i\neq j}(\chi (j^2)-\chi(i^2)) \nonumber \\
  =&\prod_{1\le j\le (p-1)/2}\(\frac{p-1}{2}\chi(j^2)^{-1}\)  \nonumber\\
  =&\(\frac{p-1}{2}\)^{\frac{p-1}{2}}\(\prod_{1\le j\le (p-1)/2}\chi(j^2)\)^{-1}   \nonumber \\
  =&(-1)^{\frac{p+1}{2}}\(\frac{p-1}{2}\)^{\frac{p-1}{2}}
\end{align}
Combining (\ref{ip}) with (\ref{inej}) we have 
\begin{equation}\label{ipsquare}
I_p^2= (-1)^{
    \binom{(p-1)/2}{2} +\frac{p+1}{2}}\cdot \(\frac{p-1}{2}\)^{\frac{p-1}{2}}.
\end{equation}
%One may further check that
%\begin{align}\label{ipsquare}
%  I_p^2=\begin{cases}
%        (\frac{p-1}{2})^{\frac{p-1}{2}}\cdot (-1)^{\frac{p+3}{4}} & \mbox{if } p \equiv 1 \pmod 4, \\
%        (\frac{p-1}{2})^{\frac{p-1}{2}}\cdot (-1)^{\frac{p-3}{4}} & \mbox{if } p \equiv 3 \pmod 4.
%      \end{cases}
%\end{align}


Let $x_1,\ldots,x_n$ be variables and $p_k(x_1,\ldots,x_n)$ the $k$-th power sum for any $ k \ge 1$ defined by :
$$
 p_{k}(x_{1},\ldots ,x_{n})=x_{1}^{k}+\cdots +x_{n}^{k}.
$$
Also, the $k$-th elementary symmetric polynomial of $x_1,\cdots,x_n$ is denoted by $e_k(x_1, ..., x_n)$, i.e., 
$$
\displaystyle {\begin{aligned}e_{0}(x_{1},\ldots ,x_{n})&=1,\\e_{1}(x_{1},\ldots ,x_{n})&=x_{1}+x_{2}+\cdots +x_{n},\\e_{2}(x_{1},\ldots ,x_{n})&=\sum _{1\leq i<j\leq n}x_{i}x_{j},\\&\;\;\vdots \\e_{n}(x_{1},\ldots ,x_{n})&=x_{1}x_{2}\cdots x_{n}.\\\end{aligned}}
$$
Then (\ref{roots}) implies that
\begin{equation}\label{ek}
  e_{k}(\chi (1^2),\chi (2^2),\ldots ,\chi ((\frac{p-1}{2})^2))=0
\end{equation}
for any $1\le k\le (p-3)/2$.


Recall that Newton's identities allows us to recursively express the $p_k$ in terms of the $e_i $'s.
Consequently, we have $p_k=0$ for all $1\le k\le (p-3)/2$.

Since $M_p$ is a $\frac{p-3}{2}\times \frac{p-3}{2}$ matrix with the $(i,j)$-th entry
$$
J(\chi^{2i},\chi^{2j})=\sum_{k=0}^{p-1}\chi^{2i}(k)\chi^{2j}(1-k)=\sum_{k=1}^{p-1}\chi^{i}(k^2)\chi^{j}((1-k)^2), 
$$
and $-k$ runs through $\{\frac{p+1}{2},\frac{p+3}{2},\cdots,p-2,p-1\}$ in $\mathbb{F}_p$ as $k$ runs through $\{1,2,\cdots,\frac{p-1}{2}\}$, we have
\begin{align}
  J(\chi^{2i},\chi^{2j})&= \sum_{k=1}^{(p-1)/2}\chi^{i}(k^2)\chi^{j}((1-k)^2)+\sum_{k=1}^{(p-1)/2}\chi^{i}((-k)^2)\chi^{j}((1+k)^2)\nonumber  \\
  &=\sum_{k=1}^{(p-1)/2}\chi^{i}(k^2)\(\chi^{j}((k-1)^2)+\chi^{j}((k+1)^2)\).
\end{align}
This leads us to view $M_p$ as the product of two matrices like the forms in Lemma \ref{cauchy-binet}. To be precise, let
$$A=[\chi^{i}(j^2)]_{1\le i\le \frac{p-3}{2},\ 1\le j\le \frac{p-1}{2}},$$
and let
$$B=[\chi^{j}((i-1)^2)+\chi^{j}((i+1)^2)]_{1\le i\le \frac{p-1}{2},\ 1\le j\le \frac{p-3}{2}}.$$
Then clearly
 $$M_p=AB.$$
Now let $A_{(k)}$ be the submatrix of $A$ obtained by deleting the $k$-th column of $A$ and let $B^{(k)}$ be the submatrix of $B$ obtained by deleting the $k$-th row of $B$.
By Lemma \ref{cauchy-binet} we have 
\begin{align}\label{detmp}
 \det M_p=\sum_{k=1}^{(p-1)/2}\det A_{(k)} \cdot \det B^{(k)}.
\end{align}
Let $V(x_1,x_2,\ldots,x_n)$ be the Vandermonde matrix $[x_j^{i-1}]_{1\le i,j\le n}$. For $1\le k\le (p-1)/2 $, we see that $\det A_{(k)}$ actually equals to 

\begin{displaymath}
\left| \begin{array}{cccccccc}
\chi(1^2) & \chi(2^2)  & \ldots & \chi((k-1)^2)  &\chi((k+1)^2)    & \ldots & \chi((\frac{p-1}{2})^2)\\

\chi^{2}(1^2) & \chi^{2}(2^2)  & \ldots & \chi^{2}((k-1)^2)  &\chi^{2}((k+1)^2)    & \ldots & \chi^{2}((\frac{p-1}{2})^2) \\

\chi^{3}(1^2) & \chi^{3}(2^2)  & \ldots & \chi^{3}((k-1)^2)  &\chi^{3}((k+1)^2)    & \ldots & \chi^{3}((\frac{p-1}{2})^2)\\

\vdots & \vdots & \ddots  & \vdots  & \vdots  & \ddots  & \vdots  \\

\chi^{\frac{p-5}{2}}(1^2) & \chi^{\frac{p-5}{2}}(2^2)  & \ldots & \chi^{\frac{p-5}{2}}((k-1)^2)  &\chi^{\frac{p-5}{2}}((k+1)^2)    & \ldots & \chi^{\frac{p-5}{2}}((\frac{p-1}{2})^2)\\

\chi^{\frac{p-3}{2}}(1^2) & \chi^{\frac{p-3}{2}}(2^2)  & \ldots & \chi^{\frac{p-3}{2}}((k-1)^2)  &\chi^{\frac{p-3}{2}}((k+1)^2)    & \ldots & \chi^{\frac{p-3}{2}}((\frac{p-1}{2})^2)\\
\end{array} \right|.
\end{displaymath}
Taking the common factor $\chi(j^2)$ from  the $j$-th column for $1\le j\ne k\le (p-1)/2 $ we obtain
\begin{align}
  &  \det A_{(k)}\nonumber \\
  =&\(\prod\limits_{\substack{i=1\\i\neq k}}^{(p-1)/2}\chi (i^2)\)\cdot \det (V(\chi(1^2),\ldots,\chi((k-1)^2),\chi((k+1)^2),\ldots,\chi((\frac{p-1}{2})^2)))\nonumber \\
  =& \frac{\prod\limits_{\substack{i=1}}^{(p-1)/2}\chi (i^2)}{\chi(k^2)} \cdot \frac{\prod\limits_{1\le i<j\le (p-1)/2}(\chi (j^2)-\chi(i^2))}{(-1)^{\frac{p-1}{2}-k}\cdot \prod\limits_{\substack{i=1\\i\neq k}}^{(p-1)/2}(\chi (k^2)-\chi(i^2))}.
\end{align}
It follows from  (\ref{prodsquare}) and (\ref{inek}) that
\begin{equation}\label{ak}
  \det A_{(k)}=(-1)^{k+1}\cdot \frac{2}{p-1}\cdot I_p.
\end{equation}
Combining the above equalities with (\ref{detmp}) we obtain
\begin{align}\label{mp2}
 \det M_p=\frac{2}{p-1}\cdot I_p \cdot \sum_{k=1}^{(p-1)/2}(-1)^{k+1} \cdot \det B^{(k)}.
\end{align}
Let $\widetilde{B}$ be the matrix obtained by adding a column vector of order $(p-1)/2$ with all elements $1$ before the first column of $B$. By determinant expansion formula with respect to the first column we have
  \begin{align}\label{mp3}
 \det M_p=\frac{2}{p-1}\cdot I_p \cdot \det  \widetilde{B}.
\end{align}
We next evaluate $ \det \widetilde{B}$. Let $\chi(j^2)=x_j$ for any $0\le j\le (p+1)/2$. Note that $x_0=0$, $x_1=1$ and $x_{\frac{p-1}{2}}=x_{\frac{p+1}{2}}$.
Then
\begin{align}
 \det  \widetilde{B} & =
 \left | \begin{array}{ccccc}
   1 & x_0+x_2 & x_0^2+x_2^2 &\cdots &   x_0^{\frac{p-3}{2}}+x_2^{\frac{p-3}{2}} \\
   1 & x_1+x_3 & x_1^2+x_3^2 &\cdots &   x_1^{\frac{p-3}{2}}+x_3^{\frac{p-3}{2}} \\
  1 & x_2+x_4 & x_2^2+x_4^2 &\cdots &   x_2^{\frac{p-3}{2}}+x_4^{\frac{p-3}{2}} \\
  \vdots & \vdots & \vdots & \ddots & \vdots \\
    1 & x_{\frac{p-3}{2}}+x_\frac{p+1}{2} & x_\frac{p-3}{2}^2+x_\frac{p+1}{2}^2 &\cdots &   x_{\frac{p-3}{2}}^{\frac{p-3}{2}}+x_{\frac{p+1}{2}}^{\frac{p-3}{2}} \\
 \end{array} \right |  .
\end{align}

 We first add the second, third,...,until the last row to the first row.
By (\ref{ek}) we have 
\begin{align}
&(x_0+x_2)+(x_1+x_3)+(x_2+x_4)+\ldots+(x_{\frac{p-3}{2}}+x_{\frac{p+1}{2}}) \nonumber   \\
&= -x_1+2\cdot (x_1+x_2+\cdots+x_{\frac{p-3}{2}}+x_{\frac{p-1}{2}})= -x_1
\end{align}

Similarly, 
 \begin{align}\label{ksum}
&(x_0^k+x_2^k)+(x_1^k+x_3^k)+(x_2^k+x_4^k)+\ldots+(x_{\frac{p-3}{2}}^k+x_{\frac{p+1}{2}}^k) =-x_1^k
\end{align}
for any $1\le k\le \frac{p-3}{2}$.
Consequently,

\begin{align}
 \det  \widetilde{B} & =
 \left | \begin{array}{ccccc}
   \frac{p-1}{2} & -x_1 & -x_1^2 &\cdots & -x_1^{\frac{p-3}{2}} \\
   1 & x_1+x_3 & x_1^2+x_3^2 &\cdots &   x_1^{\frac{p-3}{2}}+x_3^{\frac{p-3}{2}} \\
  1 & x_2+x_4 & x_2^2+x_4^2 &\cdots &   x_2^{\frac{p-3}{2}}+x_4^{\frac{p-3}{2}} \\
  \vdots & \vdots & \vdots & \ddots & \vdots \\
    1 & x_{\frac{p-3}{2}}+x_\frac{p+1}{2} & x_\frac{p-3}{2}^2+x_\frac{p+1}{2}^2 &\cdots &   x_{\frac{p-3}{2}}^{\frac{p-3}{2}}+x_{\frac{p+1}{2}}^{\frac{p-3}{2}} \\
 \end{array} \right | .
\end{align}
We next add the first row to the second row and get

\begin{align}
 \det  \widetilde{B}
 =& \left | \begin{array}{ccccc}
   \frac{p-1}{2} & -x_1 & -x_1^2 &\cdots & -x_1^{\frac{p-3}{2}} \\
    \frac{p+1}{2}&x_3 & x_3^2 &\cdots &   x_3^{\frac{p-3}{2}} \\
  1 & x_2+x_4 & x_2^2+x_4^2 &\cdots &   x_2^{\frac{p-3}{2}}+x_4^{\frac{p-3}{2}} \\
  \vdots & \vdots & \vdots & \ddots & \vdots \\
    1 & x_{\frac{p-3}{2}}+x_\frac{p+1}{2} & x_\frac{p-3}{2}^2+x_\frac{p+1}{2}^2 &\cdots &   x_{\frac{p-3}{2}}^{\frac{p-3}{2}}+x_{\frac{p+1}{2}}^{\frac{p-3}{2}} \\
 \end{array} \right | \nonumber  \\
 =-\frac{1}{2}\cdot &\left | \begin{array}{ccccc}
   -(p-1) & 2x_1 & 2x_1^2 &\cdots & 2x_1^{\frac{p-3}{2}} \\
    \frac{p+1}{2}&x_3 & x_3^2 &\cdots &   x_3^{\frac{p-3}{2}} \\
  1 & x_2+x_4 & x_2^2+x_4^2 &\cdots &   x_2^{\frac{p-3}{2}}+x_4^{\frac{p-3}{2}} \\
  \vdots & \vdots & \vdots & \ddots & \vdots \\
    1 & x_{\frac{p-3}{2}}+x_\frac{p+1}{2} & x_\frac{p-3}{2}^2+x_\frac{p+1}{2}^2 &\cdots &   x_{\frac{p-3}{2}}^{\frac{p-3}{2}}+x_{\frac{p+1}{2}}^{\frac{p-3}{2}}\\
 \end{array} \right | .
\end{align}
For the above determinant, we add the second, third,...,until the last row to the first row. Noting that for any $1\le k\le \frac{p-3}{2}$,
\begin{align}
&2x_1^k+x_3^k+(x_2^k+x_4^k)+\cdots+(x_{\frac{p-3}{2}}^k+x_{\frac{p+1}{2}}^k) \nonumber   \\
=& -x_2^k+2\cdot (x_1^k+x_2^k+\cdots+x_{\frac{p-3}{2}}^k+x_{\frac{p-1}{2}}^k)= -x_2^k.
\end{align}
we therefore obtain 
\begin{align}
 \det  \widetilde{B}
 =-\frac{1}{2}\cdot &\left | \begin{array}{ccccc}
   -1 & -x_2 &- x_2^2 &\cdots & -x_2^{\frac{p-3}{2}} \\
    \frac{p+1}{2}&x_3 & x_3^2 &\cdots &   x_3^{\frac{p-3}{2}} \\
  1 & x_2+x_4 & x_2^2+x_4^2 &\cdots &   x_2^{\frac{p-3}{2}}+x_4^{\frac{p-3}{2}} \\
  \vdots & \vdots & \vdots & \ddots & \vdots \\
    1 & x_{\frac{p-3}{2}}+x_\frac{p+1}{2} & x_\frac{p-3}{2}^2+x_\frac{p+1}{2}^2 &\cdots &   x_{\frac{p-3}{2}}^{\frac{p-3}{2}}+x_{\frac{p+1}{2}}^{\frac{p-3}{2}} \\
 \end{array} \right |  \nonumber \\
 =\frac{1}{2}\cdot &\left | \begin{array}{ccccc}
   1 & x_2 & x_2^2 &\cdots & x_2^{\frac{p-3}{2}} \\
    \frac{p+1}{2}&x_3 & x_3^2 &\cdots &   x_3^{\frac{p-3}{2}} \\
  1 & x_2+x_4 & x_2^2+x_4^2 &\cdots &   x_2^{\frac{p-3}{2}}+x_4^{\frac{p-3}{2}} \\
  \vdots & \vdots & \vdots & \ddots & \vdots \\
    1 & x_{\frac{p-3}{2}}+x_\frac{p+1}{2} & x_\frac{p-3}{2}^2+x_\frac{p+1}{2}^2 &\cdots &   x_{\frac{p-3}{2}}^{\frac{p-3}{2}}+x_{\frac{p+1}{2}}^{\frac{p-3}{2}} \\
 \end{array} \right |.
\end{align}
Finally, we recursively  use  row operation method to deduce that

\begin{align}
 \det  \widetilde{B}
 =\frac{1}{2}\cdot &\left | \begin{array}{ccccc}
   1 & x_2 & x_2^2 &\cdots & x_2^{\frac{p-3}{2}} \\
    \frac{p+1}{2}&x_3 & x_3^2 &\cdots &   x_3^{\frac{p-3}{2}} \\
   \alpha_4& x_4 & x_4^2 &\cdots &   x_4^{\frac{p-3}{2}} \\
  \vdots & \vdots & \vdots & \ddots & \vdots \\
    \alpha_\frac{p-1}{2} &  x_\frac{p-1}{2} & x_\frac{p-1}{2}^2 &\cdots &   +x_{\frac{p-1}{2}}^{\frac{p-3}{2}} \\
    \alpha_\frac{p+1}{2} &  x_\frac{p+1}{2} & x_\frac{p+1}{2}^2 &\cdots &   +x_{\frac{p+1}{2}}^{\frac{p-3}{2}} \\
 \end{array} \right |
\end{align}
where
\begin{equation}\label{alpha}
  \alpha_k=
  \begin{cases}
    \frac{1-(-1)^\frac{k}{2}}{2} & \mbox{if}\ 2\mid k,\\
    \frac{p+1}{2}\cdot (-1)^\frac{k+1}{2}+\frac{1+(-1)^\frac{k-1}{2}}{2}& \mbox{otherwise}. 
  \end{cases}
\end{equation}
Let $\beta =\alpha_\frac{p+1}{2}-\alpha_\frac{p-1}{2}$. One can verify that 
\begin{equation}
\beta=
\begin{cases}
  (-1)^\frac{p+3}{4}\cdot \frac{p-1}{2}, & \mbox{if } p \equiv 1 \pmod 4,\\
  (-1)^\frac{p-3}{4}\cdot \frac{p+1}{2}, & \mbox{if } p \equiv 3 \pmod 4.
\end{cases}
\end{equation}
As
\begin{align}
  (-1)^{\binom{(p-1)/2}{2} }=\begin{cases}
  (-1)^\frac{p-1}{4} & \mbox{if } p \equiv 1 \pmod 4,\\
  (-1)^\frac{p-3}{4} & \mbox{if } p \equiv 3 \pmod 4,
                             \end{cases}
\end{align}
we have 
\begin{equation}\label{beta}
  \beta=(-1)^{
    \binom{(p-1)/2}{2} +\frac{p+1}{2}}\cdot \frac{p+(-1)^\frac{p+1}{2}}{2}.
\end{equation}
Now by subtracting the $\frac{p-3}{2}$-th row from the $\frac{p-1}{2}$-th row and   $x_{\frac{p-1}{2}}=x_{\frac{p+1}{2}}$ we obtain 


\begin{align}\label{1B}
 \det  \widetilde{B}
 &=\frac{1}{2}\cdot \left | \begin{array}{ccccc}
   1 & x_2 & x_2^2 &\cdots & x_2^{\frac{p-3}{2}} \\
    \frac{p+1}{2}&x_3 & x_3^2 &\cdots &   x_3^{\frac{p-3}{2}} \\
   \alpha_4& x_4 & x_4^2 &\cdots &   x_4^{\frac{p-3}{2}} \\
  \vdots & \vdots & \vdots & \ddots & \vdots \\
    \alpha_\frac{p-1}{2} &  x_\frac{p-1}{2} & x_\frac{p-1}{2}^2 &\cdots &   +x_{\frac{p-1}{2}}^{\frac{p-3}{2}} \\
    \alpha_\frac{p+1}{2}-\alpha_\frac{p-1}{2} &  0 & 0 &\cdots &   0 \\
 \end{array} \right |  \nonumber \\
 &= \frac{1}{2}\cdot (-1)^{\frac{p+1}{2}} \cdot \beta\cdot \left | \begin{array}{ccccc}
   & x_2 & x_2^2 &\cdots & x_2^{\frac{p-3}{2}} \\
   &x_3 & x_3^2 &\cdots &   x_3^{\frac{p-3}{2}} \\
  & x_4 & x_4^2 &\cdots &   x_4^{\frac{p-3}{2}} \\
  & \vdots & \vdots & \ddots & \vdots \\
     &  x_\frac{p-1}{2} & x_\frac{p-1}{2}^2 &\cdots &   +x_{\frac{p-1}{2}}^{\frac{p-3}{2}} \\
 \end{array} \right |  \nonumber \\
  &=\frac{1}{2}\cdot (-1)^{\frac{p+1}{2}}\cdot \beta\cdot \(\prod\limits_{\substack{i=2}}^{(p-1)/2}x_i\)\cdot \det (V(x_2,x_3,\cdots,x_{\frac{p-1}{2}})) \nonumber \\
  &=\frac{1}{2}\cdot   \beta \cdot \prod_{2\le i<j\le (p-1)/2}(\chi (j^2)-\chi(i^2))  \nonumber\\
   &=\frac{1}{2}\cdot   \beta \cdot \frac{\prod\limits_{1\le i<j\le (p-1)/2}(\chi (j^2)-\chi(i^2)) }{ \prod\limits_{\substack{j=2}}^{(p-1)/2}(\chi (j^2)-1)} .
\end{align}
Combining (\ref{1B}) with (\ref{inek}) we see that 
\begin{equation}
  \det  \widetilde{B}=(-1)^{\frac{p-3}{2}}\cdot \frac{1}{p-1}\cdot \beta\cdot I_p,
\end{equation}
and hence 
 \begin{align}\label{mpbetaip}
 \det M_p&=\frac{2}{p-1}\cdot I_p \cdot \det  \widetilde{B}\nonumber \\
 &=(-1)^{\frac{p+1}{2}}\cdot \frac{2}{(p-1)^2} \cdot \beta\cdot I_p^2.
\end{align}
By (\ref{ipsquare}), (\ref{beta}) and (\ref{mpbetaip}) we finally obtain 
$$ \det M_p=\frac{p+(-1)^{\frac{p+1}{2}}}{4}  \(\frac{p-1}{2}\)^{\frac{p-5}{2}}(-1)^{\frac{p+1}{2}}.$$

This completes the proof.\qed


\begin{thebibliography}{0}
	
\bibitem{carlitz}  L. Carlitz, Some cyclotomic matrices, Acta Arith. 5 (1959), 293--308.
	
\bibitem{chapman} R. Chapman, Determinants of Legendre symbol matrices, Acta Arith. 115 (2004), 231--244.

\bibitem{classical} K. Ireland, M. Rosen, A Classical Introduction to Modern Number Theory, second ed., Grad. Texts Math., vol. 84, Springer, New York, 1990.

\bibitem{Lehmer} D. H. Lehmer, On certain character matrices, Pacific J. Math. 6 (1956), 491--499.

\bibitem{LN} R. Lidl and H. Niederreiter, Finite Fields, 2nd ed., Cambridge Univ. Press, Cambridge, 1997.

\bibitem{K1} C. Krattenthaler, Advanced Determinant Calculus. In: Foata D., Han GN. (eds) The Andrews Festschrift. Springer, Berlin, Heidelberg 2001.

\bibitem{K2}  C. Krattenthaler, Advanced determinant calculus: a complement, Linear Algebra
Appl. 411 (2005) 68--166.

\bibitem{S19} Z.-W. Sun, On some determinants with Legendre symbols entries, Finite Fields Appl. 56 (2019), 285-307.

\bibitem{V1} M. Vsemirnov, On the evaluation of R. Chapman's ``evil determinant", Linear Algebra Appl. 436 (2012), 4101--4106.

\bibitem{V2} M. Vsemirnov, On R. Chapman's ``evil determinant": case $p\equiv1\pmod 4$, Acta Arith. 159 (2013), 331--344.

\bibitem{W21} H.-L. Wu, Elliptic curves over $\mathbb{F}_p$ and determinants of Legendre matrices, Finite Fields Appl. 76 (2021), 101929.

\end{thebibliography}



\end{document}



