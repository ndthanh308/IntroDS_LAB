
\documentclass[12pt,reqno]{amsart}
\usepackage{enumerate, latexsym, amsmath, amsfonts, amssymb, amsthm, graphicx, color}
 \usepackage{hyperref}
\hypersetup{hypertex=true,
            colorlinks=true,
            linkcolor=blue,
           anchorcolor=blue,
            citecolor=blue}
\usepackage{amsmath}
 \textwidth=13.5cm
 \textheight=22cm
\hoffset=-1cm\voffset-0.5truecm
\hoffset=-1cm\voffset-0.5truecm
\def\pmod #1{\ ({\rm{mod}}\ #1)}
\def\Z{\Bbb Z}
\def\N{\Bbb N}
\def\G{\Bbb G}
\def\Q{\Bbb Q}
\def\zp{\Z^+}
\def\R{\Bbb R}
\def\C{\Bbb C}
\def\F{\Bbb F}
\def\L{\Bbb L}
\def\M{\Bbb M}
\def\l{\left}
\def\r{\right}
\def\bg{\bigg}
\def\({\bg(}
\def\){\bg)}
\def\t{\text}
\def\f{\frac}
\def\mo{{\rm{mod}\ }}
\def\lcm{\rm{lcm}}
\def\ord{{\rm ord}}
\def\sgn{{\rm sgn}}
\def\adj{{\rm adj}}
\def\cs{\ldots}
\def\ov{\overline}
\def\ls{\le}
\def\gs{\geqslant}
\def\se {\subseteq}
\def\sm{\setminus}
\def\bi{\binom}
\def\al{\alpha}
\def\be{\beta}
\def\ga{\gamma}
\def\ve{\varepsilon}
\def\lr{\Leftarrow}
\def\ra{\rightarrow}
\def\Lr{\leftrightarrow}
\def\ar{\Rightarrow}
\def\Ar{\Leftrightarrow}
\def\eq{\equiv}
\def\Da{\Delta}
\def\da{\delta}
\def\la{\lambda}
\def\La{\Lambda}
\def\ta{\theta}
\def\AP{\rm{AP}}
\def\va{\varphi}
\def\supp {\rm supp}
\def\colon{{:}\;}
\def\u{{\bf u}}
\def\v{{\bf v}}
\def\Proof{\noindent{\it Proof}}
\def\Ack{\medskip\noindent {\bf Acknowledgments}}
\theoremstyle{plain}
\newtheorem{theorem}{Theorem}
\newtheorem{fact}{}
\newtheorem{lemma}{Lemma}
\newtheorem{corollary}{Corollary}
\newtheorem{conjecture}{Conjecture}
\theoremstyle{definition}
\newtheorem*{acknowledgment}{Acknowledgment}
\theoremstyle{remark}
\newtheorem{remark}{Remark}
\newtheorem{example}[theorem]{Example}
\renewcommand{\theequation}{\arabic{section}.\arabic{equation}}
\renewcommand{\thetheorem}{\arabic{section}.\arabic{theorem}}
\renewcommand{\thelemma}{\arabic{section}.\arabic{lemma}}
\renewcommand{\thecorollary}{\arabic{section}.\arabic{corollary}}
\renewcommand{\theconjecture}{\arabic{section}.\arabic{conjecture}}
\renewcommand{\theremark}{\arabic{section}.\arabic{remark}}
\makeatletter
\@namedef{subjclassname@2010}{%
  \textup{2010} Mathematics Subject Classification}
\makeatother
 \vspace{4mm}


\newcommand\blfootnote[1]{%
	\begingroup
	\renewcommand\thefootnote{}\footnote{#1}%
	\addtocounter{footnote}{-1}%
	\endgroup
}
\begin{document}


\title{On finite field analogues of determinants involving binomial coefficients}

\author[H.-L. Wu, L.-Y. Wang and H. Pan]{Hai-Liang Wu, Li-Yuan Wang, Hao Pan*}


\address {(Hai-Liang Wu) School of Science, Nanjing University of Posts and Telecommunications, Nanjing 210023, People's Republic of China}
\email{\tt whl.math@smail.nju.edu.cn}

\address {(Hao Pan) School of Applied Mathematics, Nanjing University of Finance and Economics, Nanjing 210046, People's Republic of China}
\email{\tt haopan79@zoho.com}

\address {(Li-Yuan Wang) School of Physical and Mathematical Sciences, Nanjing Tech University, Nanjing 211816, People's Republic of China}
\email{\tt wly@smail.nju.edu.cn}

\begin{abstract}
In this paper, we establish some finite field analogues of certain determinants involving binomial coefficients. For example, let $p$ be an odd prime and let $\chi$ be a generator of the group of all multiplicative character of $\mathbb{F}_p$. We show that  
$$\det[J(\chi^i,\chi^j)]_{1\le i,j\le p-2}=(p-1)^{p-3},$$
where $J(\chi^i,\chi^j)$ is the Jacobi sum of $\chi^i$ and $\chi^j$. This result can be viewed as a finite field analogue of $\det\left[\binom{i+j}{i}\right]_{1\le i,j\le n}=n+1$. 
Also, we prove that 
$$\det [J(\chi^{2i},\chi^{2j})]_{1\le i,j\le (p-3)/2}
=\frac{1+(-1)^{\frac{p+1}{2}}p}{4}\left(\frac{p-1}{2}\right)^{\frac{p-5}{2}}.$$
\end{abstract}

\thanks{2020 {\it Mathematics Subject Classification}.
Primary 11C20; Secondary 11L05, 11R18.
\newline\indent {\it Keywords}. finite fields, determinants, Jacobi sums.
\newline \indent
This research was supported by the National Natural Science Foundation of China (Grant Nos. 12101321, 12201291 and 12071208) and the Natural Science Foundation of the Higher Education Institutions of Jiangsu Province (21KJB110001, 21KJB110002). 
\newline\thanks{*Corresponding author.}}
\maketitle
\section{Introduction}	
\setcounter{lemma}{0}
\setcounter{theorem}{0}
\setcounter{corollary}{0}
\setcounter{remark}{0}
\setcounter{equation}{0}
\setcounter{conjecture}{0}
\subsection{Background and Motivations}
Determinants of cyclotomic matrices have extensive applications in algebra, number theory and combinatorics. Readers may refer to the survey papers \cite{K1,K2} for recent progress on this topic. 

We first introduce some notations (readers may refer to \cite{classical,LN}). Let $p$ be an odd prime and let $\mathbb{F}_p$ be the finite field of $p$ elements. Also, let $\F_p^{\times}$ be the group of all nonzero elements of $\F_p$, and let $\widehat{\mathbb{F}_p^{\times}}$ denote the group of all multiplicative character of $\F_p$ (we also define $\psi(0)=0$ for any $\psi\in\widehat{\mathbb{F}_p^{\times}}$). 

The earliest research on cyclotomic matrices came from Lehmer \cite{Lehmer} and Carlitz \cite{carlitz}. For example, 
given a nontrivial character $\psi\in\widehat{\mathbb{F}_p^{\times}}$, 
Carlitz studied the characteristic polynomial of the cyclotomic matrix 
$$C_p(\psi):=\left[\psi(j-i)\right]_{1\le i,j\le p-1}.$$ In particular, when $\psi$ is a quadratic character, Carlitz showed that the characteristic polynomial of $C_p(\psi)$ is 
$$\left(t^2-(-1)^{\frac{p-1}{2}}p\right)^{\frac{p-3}{2}}\left(t^2-(-1)^{\frac{p-1}{2}}\right).$$ 

Along this line, Chapman \cite{chapman} initiated the study of several variants of Carlitz's results. Surprisingly, these variants have close connections with the real quadratic field $\mathbb{Q}(\sqrt{p})$. For example, let $(\frac{\cdot}{p})$ be the Legendre symbol and let $$\varepsilon_p^{(2-(\frac{2}{p}))h_p}=a_p+b_p\sqrt{p}\ (a_p,b_p\in\mathbb{Q}),$$
where $\varepsilon_p>1$ and $h_p$ are the fundamental unit and class number
of $\mathbb{Q}(\sqrt{p})$. Chapman conjectured that 
$$\det\left[\left(\frac{j-i}{p}\right)\right]_{1\le i,j\le \frac{p+1}{2}}=\begin{cases}
	-a_p & \mbox{if}\ p\equiv 1\pmod4,\\
	1    & \mbox{if}\ p\equiv 3\pmod4.
\end{cases}$$
This conjecture was known as Chapman's ``evil determinant" conjecture. This challenging conjecture was later confirmed by Vsemirnov \cite{K1,K2}. 

Recently, Sun \cite{S19} and Krachun and his collaborators \cite{Krachun} studied some  variants of the above determinant. For example, Sun proved that $$-\det\left[\left(\frac{i^2+j^2}{p}\right)\right]_{1\le i,j\le (p-1)/2}$$
is a quadratic residue modulo $p$. Later the first author \cite{W21} proved that for any $c\in\F_p$ with $c\neq\pm2$, 
$$\det\left[\left(\frac{i^2+cij+j^2}{p}\right)\right]_{0\le i,j\le p-1}=0$$
whenever the curve defined by the equation $y^2=x(x^2+cx+1)$ is a supersingular elliptic curve over $\F_p$. Readers may refer to \cite{Krachun,S19,W21,WSW} for recent progress on this topic. 

On the other hand, let $n$ be a positive integer. Consider the following matrix involving binomial coefficients:
\begin{equation*}
	C_n=\left[\binom{i+j}{i}\right]_{1\le i,j\le n}=\left[\frac{\Gamma(i+j+1)}{\Gamma(i+1)\Gamma(j+1)}\right]_{1\le i,j\le n},
\end{equation*}
where $\Gamma(\cdot)$ is the Gamma function. One can easily verify that (we will give a detailed proof in the appendix of this paper)
\begin{equation}\label{Eq. det Cn}
	\det C_n=n+1.
\end{equation}
Also, for any $\psi,\lambda\in\widehat{\mathbb{F}_p^{\times}}$, the Jacobi sum $J(\psi,\lambda)$ of $\psi$ and $\lambda$ is defined by
$$J(\psi,\lambda):=\sum_{a+b=1}^{}\psi(a)\lambda(b)=\sum_{x\in\mathbb{F}_p}\psi(x)\lambda(1-x).$$
Let $\chi$ be a generator of $\widehat{\mathbb{F}_p^{\times}}$ and let $1\le i,j\le p-2$ be integers. Then it is known that (see \cite[Proposition 3.6.4]{Cohen}, and note that our expression below is slightly different from \cite[Proposition 3.6.4]{Cohen}) there exists a prime ideal $\mathfrak{p}$ of $\mathbb{Z}[e^{2\pi{\bf i}/(p(p-1))}]$ such that $p\in\mathfrak{p}$ and 
\begin{equation}\label{Eq. connection between jacobi sums and binomial coefficients}
J(\chi^i,\chi^j)\equiv -\binom{i+j}{i}\equiv -\frac{\Gamma(i+j+1)}{\Gamma(i+1)\Gamma(j+1)}\pmod{\mathfrak{p}}.
\end{equation}
Now congruence (\ref{Eq. connection between jacobi sums and binomial coefficients}) establishes a connection between binomial coefficients and Jacobi sums. 

Motivated by the above, to obtain a reasonable finite field analogue of (\ref{Eq. det Cn}), it is natural to consider the matrices with $J(\chi^i,\chi^j)$ as their entries. 

\subsection{Main Results}
Let notations be as above. Define 
$$J_p:=[J(\chi^i,\chi^j)]_{1\le i,j\le p-2}. $$
By the Galois theory, one can verify that $\det J_p\in\mathbb{Z}$ and is independent of the choice of the generator $\chi$. As our first result, we determine the explicit value of $\det J_p$, which can be viewed as a finite field analogue of (\ref{Eq. det Cn}).

\begin{theorem}\label{thjp}
	Let $p$ be an odd prime and let $\chi$ be a generator of $\widehat{\mathbb{F}_p^{\times}}$. Then
	$$ \det J_p=(p-1)^{p-3}.$$
\end{theorem}

As a direct consequence of Theorem \ref{thjp}, we have 
\begin{corollary}\label{Cor of Thm. 1}
	Let $p$ be an odd prime and let $\chi$ be a generator of $\widehat{\mathbb{F}_p^{\times}}$. Then
	$$\det J_p\equiv1\pmod p.$$
\end{corollary}

As the second result of this paper, we consider the matrix involving the Jacobi sums of nontrivial even characters of $\mathbb{F}_p$. Define
$$M_p:=[J(\chi^{2i},\chi^{2j})]_{1\le i,j\le (p-3)/2}.$$
We have the following result.

\begin{theorem}\label{thmp}
  Let $p$ be an odd prime and let $\chi$ be a generator of $\widehat{\mathbb{F}_p^{\times}}$. Then
 $$ \det M_p=\frac{1+(-1)^{\frac{p+1}{2}}p}{4}\left(\frac{p-1}{2}\right)^{\frac{p-5}{2}}.$$
\end{theorem}

Similar to Corollary \ref{Cor of Thm. 1}, we also have 
\begin{corollary}\label{Cor. of Thm. 2}
	Let $p$ be an odd prime and let $\chi$ be a generator of $\widehat{\mathbb{F}_p^{\times}}$. Then
	$$\det M_p\equiv \left(\frac{-2}{p}\right)\pmod p.$$
\end{corollary}

\subsection{Outline of this paper} The proofs of Theorems \ref{thjp}--\ref{thmp}  will be given in Section 2 and Section 3 respectively. In Section 4, we will give some remarks and pose some open problems for further research. Finally, we add an appendix to provide the proof of (\ref{Eq. det Cn}).
\maketitle

\section{Proof of Theorem \ref{thjp}}
\setcounter{lemma}{0}
\setcounter{theorem}{0}
\setcounter{corollary}{0}
\setcounter{remark}{0}
\setcounter{equation}{0}
\setcounter{conjecture}{0}

{\noindent\bf Proof of Theorem \ref{thjp}.}
As $J_p$ is a $(p-2)\times (p-2)$ matrix with the $(i,j)$-th entry
$$
J(\chi^i,\chi^j)=\sum_{k=0}^{p-1}\chi^i(k)\chi^j(1-k)=\sum_{k=2}^{p-1}\chi^i(k)\chi^j(1-k),
$$
one can verify that 
\begin{equation}\label{jmn}
 J_p=MN,
\end{equation}
 where
$$M=[\chi^i(j)]_{1\le i\le p-2,\ 2\le j\le p-1},$$
and
$$N=[\chi^j(1-i)]_{2\le i\le p-1,\ 1\le j\le p-2}.$$

Let $g(t)=t^{p-1}-1$ and let $g'(t)=(p-1)t^{p-2}$ be the derivative of $g(t)$. As $\chi$ is a generator of $\widehat{\mathbb{F}_p^{\times}}$, we have 
$\{\chi(x): x\in\F_p^{\times}\}=\{e^{2\pi a{\bf i}/(p-1)}:\ a\in\Z\}$, and hence 
$g(t)=\prod\limits_{j=1}^{p-1}(t-\chi(j))$. This implies $\prod\limits_{j=1}^{p-1}\chi(j)=-1$. 
Since $\chi(1)=1$ and $\chi(-1)=-1$, we also have $\prod\limits_{j=2}^{p-1}\chi(j)=-1$ and $\prod\limits_{j=1}^{p-2}\chi(j)=1$. We next verify the following equalities. Given an integer $1\le k\le p-1$, we have 
\begin{align}\label{1nekp-1}
   \prod_{i\in\mathbb{F}_p^{\times}\setminus\{k\}}(\chi (k)-\chi(i))&=\lim\limits_{z\rightarrow \chi(k)}\frac{\prod\limits_{i=1}^{p-1}(z-\chi(i))}{z-\chi(k)}  \nonumber \\
   &=\lim\limits_{z\rightarrow \chi(k)}\frac{z^{p-1}-1}{z-\chi(k)}  \nonumber \\
   &=\lim\limits_{z\rightarrow \chi(k)}(p-1)\cdot z^{p-2} \nonumber \\
   &=(p-1)\cdot\chi(k)^{-1}.
\end{align}
Also, one can verify that  
\begin{align}\label{1ijp-1}
  \(\prod_{1\le i<j\le p-1}(\chi (j)-\chi(i))\)^2
   =& (-1)^{\binom{p-1}{2}}\cdot \prod_{1\le i\ne j\le p-1}(\chi (j)-\chi(i))\  \nonumber \\
   =& (-1)^{\frac{p-1}{2}}\cdot \prod_{1\le j\le p-1}\prod_{i\neq j}(\chi (j)-\chi(i))\  \nonumber \\
  =&(-1)^{\frac{p-1}{2}}\cdot\prod_{1\le j\le p-1}\left((p-1)\cdot\chi(j)^{-1}\right)\nonumber\\
  =&  (-1)^{\frac{p+1}{2}}\cdot(p-1)^{p-1}.
\end{align}
Now we evaluate the determinants of $M$ and $N$. For $\det M$, we have 
\begin{align}\label{detm}
  \det M &=\left|
  \begin{array}{cccc}
    \chi (2)& \chi (3)&\cdots  & \chi (p-1) \\
    \chi^2 (2)& \chi^2 (3)&\cdots  & \chi^2 (p-1) \\
    \vdots  & \vdots  & \ddots &\vdots \\
     \chi^{p-2} (2)& \chi^{p-2} (3)&\cdots  & \chi^{p-2} (p-1)  \\
  \end{array}
  \right |   \nonumber  \\
  &=\(\prod\limits_{\substack{k=2}}^{p-1}\chi (k)\)\cdot \left|
  \begin{array}{cccc}
    1& 1&\cdots  & 1 \\
    \chi (2)& \chi (3)&\cdots  & \chi (p-1) \\
    \vdots  & \vdots  & \ddots &\vdots \\
     \chi^{p-3} (2)& \chi^{p-3} (3)&\cdots  & \chi^{p-3} (p-1) \\
  \end{array}
  \right |     \nonumber \\
  &=-\prod\limits_{2\le i<j\le p-1}(\chi (j)-\chi(i)) \nonumber \\
  &=-\frac{\prod\limits_{1\le i<j\le p-1}(\chi (j)-\chi(i))}{ \prod\limits_{\substack{j=2}}^{p-1}(\chi (j)-\chi(1))}   \nonumber\\
  &=\frac{\prod\limits_{1\le i<j\le p-1}(\chi (j)-\chi(i))}{p-1}.
\end{align}
For $\det N$, we have 
\begin{align}\label{detn}
  \det N &=\left|
  \begin{array}{cccc}
    \chi (-1)&\chi^2 (-1) &\cdots  & \chi^{p-2} (-1)\\
   \chi (-2) & \chi^2 (-2)&\cdots  & \chi^{p-2} (-2) \\
    \vdots  & \vdots  & \ddots &\vdots \\
    \chi (-(p-2))  & \chi^2 (-(p-2))&\cdots  & \chi^{p-2} (-(p-2))  \\
  \end{array}
  \right |   \nonumber  \\
  &=\(\prod\limits_{\substack{k=1}}^{p-2}\chi^k (-1)\)\cdot \left|
  \begin{array}{cccc}
    \chi (1)&\chi^2 (1) &\cdots  & \chi^{p-2} (1)\\
   \chi (2) & \chi^2 (2)&\cdots  & \chi^{p-2} (2) \\
    \vdots  & \vdots  & \ddots &\vdots \\
    \chi (p-2)  & \chi^2 (p-2)&\cdots  & \chi^{p-2} (p-2)  \\
  \end{array}
  \right |     \nonumber\\
  &=(-1)^{\frac{p-1}{2}}\cdot\(\prod\limits_{\substack{k=1}}^{p-2}\chi (k)\)\cdot\prod\limits_{1\le i<j\le p-2}(\chi (j)-\chi(i))  \nonumber\\
   &=(-1)^{\frac{p-1}{2}}\cdot \prod\limits_{1\le i<j\le p-2}(\chi (j)-\chi(i)) \nonumber\\
   &=(-1)^{\frac{p-1}{2}}\cdot\frac{\prod\limits_{1\le i<j\le p-1}(\chi (j)-\chi(i))}{ \prod\limits_{\substack{i=1}}^{p-2}(\chi (p-1)-\chi(i))} \nonumber   \\
    &=(-1)^{\frac{p+1}{2}}\cdot\frac{\prod\limits_{1\le i<j\le p-1}(\chi (j)-\chi(i))}{p-1}.
\end{align}
By (\ref{jmn}) and (\ref{1ijp-1})--(\ref{detn}) we obtain 
\begin{align}\label{jab}
 \det J_p&=\det M\cdot \det N  \nonumber\\
  &=(-1)^{\frac{p+1}{2}}\cdot\frac{\(\prod\limits_{1\le i<j\le p-1}(\chi (j)-\chi(i))\)^2}{(p-1)^2} \nonumber\\
  &=(p-1)^{p-3}.
\end{align}


This completes the proof of Theorem \ref{thjp}.\qed

\section{Proof of Theorem \ref{thmp}}
\setcounter{lemma}{0}
\setcounter{theorem}{0}
\setcounter{corollary}{0}
\setcounter{remark}{0}
\setcounter{equation}{0}
\setcounter{conjecture}{0}


Let $M$ be an $m\times n$  complex matrix with $m\le n$ and let $N$ be an $n\times m$ complex matrix. Set
$$S_m=\{s=(j_1,j_2,\cdots,j_m):\ 1\le j_1<j_2<\cdots<j_m\le n\}.$$
For any $s\in S_m$, we define $M_s$ (respectively $N^s$) to be the $m\times m$ submatrix of $M$ (respectively submatrix of $N$) obtained by deleting all columns (respectively all rows) except those with indices in $s$. We begin with the well-known Cauchy-Binet formula.

\begin{lemma}[Cauchy-Binet formula]\label{cauchy-binet} Let $M,N$ be two complex matrices of sizes $m\times n$ and $n\times m$ respectively with $m\le n$. Then
$$\det(MN)=\sum_{s\in S_m}\det(M_s)\det(N^s).$$
\end{lemma}

{\bf\noindent Proof of Theorem \ref{thmp}.}
As $\chi$ is a generator of $\widehat{\mathbb{F}_p^{\times}}$, we see that $\chi(j^2)$ $(1\le j\le (p-1)/2)$ are precisely all the roots of $z^{\frac{p-1}{2}}-1=0$. %Let $\Zeta=E^{2\Pi I/(P-1)}$
Thus
\begin{equation}\label{roots}
  z^{\frac{p-1}{2}}-1=(z-\chi(1^2))(z-\chi(2^2))(z-\chi(3^2))\cdots (z-\chi((\frac{p-1}{2})^2)).
\end{equation}
By the above we have 
\begin{align}\label{prodsquare}
 \prod\limits_{\substack{i=1}}^{(p-1)/2}\chi (i^2)=-1\cdot (-1)^{(p-1)/2}=(-1)^{(p+1)/2}.
\end{align}
For $1\le k\le (p-1)/2$, one can verify that 
\begin{align}\label{inek}
   \prod\limits_{\substack{i=1\\i\neq k}}^{(p-1)/2}(\chi (k^2)-\chi(i^2))&=\lim\limits_{z\rightarrow \chi(k^2)}\frac{ \prod\limits_{\substack{i=1}}^{(p-1)/2}(z-\chi(i^2))}{z-\chi(k^2)}  \nonumber \\
   &=\lim\limits_{z\rightarrow \chi(k^2)}\frac{z^{\frac{p-1}{2}}-1}{z-\chi(k^2)}  \nonumber \\
   &=\frac{p-1}{2}\cdot \chi(k^2)^{\frac{p-1}{2}-1}=\frac{p-1}{2}\chi(k^2)^{-1}.
\end{align}
Let 
$$
I_p=\prod_{1\le i<j\le (p-1)/2}(\chi (j^2)-\chi(i^2)).   \nonumber
$$
Then 
\begin{align}\label{ip}
  &\prod_{1\le i\ne j\le (p-1)/2}(\chi (j^2)-\chi(i^2))\nonumber\\
  =&\prod_{1\le i<j\le (p-1)/2}(\chi (j^2)-\chi(i^2))\cdot \prod_{1\le j<i \le (p-1)/2}(\chi (j^2)-\chi(i^2)).   \nonumber \\
  =&(-1)^{
    \binom{(p-1)/2}{2} }\cdot I_p^2.
\end{align}
On the other hand, by (\ref{prodsquare}) and (\ref{inek}) we have 
\begin{align}\label{inej}
  &\prod_{1\le i\ne j\le (p-1)/2}(\chi (j^2)-\chi(i^2))\nonumber\\
   =&\prod_{1\le j\le (p-1)/2}\prod_{i\neq j}(\chi (j^2)-\chi(i^2)) \nonumber \\
  =&\prod_{1\le j\le (p-1)/2}\(\frac{p-1}{2}\chi(j^2)^{-1}\)  \nonumber\\
  =&\(\frac{p-1}{2}\)^{\frac{p-1}{2}}\(\prod_{1\le j\le (p-1)/2}\chi(j^2)\)^{-1}   \nonumber \\
  =&(-1)^{\frac{p+1}{2}}\(\frac{p-1}{2}\)^{\frac{p-1}{2}}.
\end{align}
Combining (\ref{ip}) with (\ref{inej}) we obtain 
\begin{equation}\label{ipsquare}
I_p^2= (-1)^{
    \binom{(p-1)/2}{2} +\frac{p+1}{2}}\cdot \(\frac{p-1}{2}\)^{\frac{p-1}{2}}.
\end{equation}

Let $x_1,\ldots,x_n$ be variables. For any positive integer $k$, let $p_k(x_1,\ldots,x_n)$ be the $k$-th power sum defined by
$$p_{k}(x_{1},\ldots ,x_{n})=x_{1}^{k}+\cdots +x_{n}^{k}.$$
Also, the $k$th elementary symmetric polynomial of $x_1,\ldots,x_n$ is denoted by $e_k(x_1, ..., x_n)$, i.e., 
$$
\displaystyle {\begin{aligned}e_{0}(x_{1},\ldots ,x_{n})&=1,\\e_{1}(x_{1},\ldots ,x_{n})&=x_{1}+x_{2}+\cdots +x_{n},\\e_{2}(x_{1},\ldots ,x_{n})&=\sum _{1\leq i<j\leq n}x_{i}x_{j},\\&\;\;\vdots \\e_{n}(x_{1},\ldots ,x_{n})&=x_{1}x_{2}\cdots x_{n}.\\\end{aligned}}
$$
Then (\ref{roots}) implies that
\begin{equation}\label{ek}
  e_{k}(\chi (1^2),\chi (2^2),\ldots ,\chi ((\frac{p-1}{2})^2))=0
\end{equation}
for any $1\le k\le (p-3)/2$.


Recall that Newton's identities allow us to recursively express the $p_k$'s in terms of the $e_i $'s.
Consequently, we have $p_k=0$ for all $1\le k\le (p-3)/2$.

Since $M_p$ is a $\frac{p-3}{2}\times \frac{p-3}{2}$ matrix with the $(i,j)$-th entry
$$
J(\chi^{2i},\chi^{2j})=\sum_{k=0}^{p-1}\chi^{2i}(k)\chi^{2j}(1-k)=\sum_{k=1}^{p-1}\chi^{i}(k^2)\chi^{j}((1-k)^2), 
$$
and $-k$ runs through $\{\frac{p+1}{2},\frac{p+3}{2},\cdots,p-2,p-1\}$ in $\mathbb{F}_p$ as $k$ runs through $\{1,2,\cdots,\frac{p-1}{2}\}$, we have
\begin{align}
  J(\chi^{2i},\chi^{2j})&= \sum_{k=1}^{(p-1)/2}\chi^{i}(k^2)\chi^{j}((1-k)^2)+\sum_{k=1}^{(p-1)/2}\chi^{i}((-k)^2)\chi^{j}((1+k)^2)\nonumber  \\
  &=\sum_{k=1}^{(p-1)/2}\chi^{i}(k^2)\(\chi^{j}((k-1)^2)+\chi^{j}((k+1)^2)\).
\end{align}
This leads us to view $M_p$ as the product of two matrices like the forms in Lemma \ref{cauchy-binet}. To be precise, let
$$A=[\chi^{i}(j^2)]_{1\le i\le \frac{p-3}{2},\ 1\le j\le \frac{p-1}{2}},$$
and let
$$B=[\chi^{j}((i-1)^2)+\chi^{j}((i+1)^2)]_{1\le i\le \frac{p-1}{2},\ 1\le j\le \frac{p-3}{2}}.$$
Then clearly
 $$M_p=AB.$$
Now let $A_{(k)}$ be the submatrix of $A$ obtained by deleting the $k$-th column of $A$ and let $B^{(k)}$ be the submatrix of $B$ obtained by deleting the $k$-th row of $B$.
By Lemma \ref{cauchy-binet} we have 
\begin{align}\label{detmp}
 \det M_p=\sum_{k=1}^{(p-1)/2}\det A_{(k)} \cdot \det B^{(k)}.
\end{align}
Let $V(x_1,x_2,\ldots,x_n)$ be the Vandermonde matrix $[x_j^{i-1}]_{1\le i,j\le n}$. For $1\le k\le (p-1)/2 $, we see that $\det A_{(k)}$ actually equals to 

\begin{displaymath}
\left| \begin{array}{cccccccc}
\chi(1^2) & \chi(2^2)  & \ldots & \chi((k-1)^2)  &\chi((k+1)^2)    & \ldots & \chi((\frac{p-1}{2})^2)\\

\chi^{2}(1^2) & \chi^{2}(2^2)  & \ldots & \chi^{2}((k-1)^2)  &\chi^{2}((k+1)^2)    & \ldots & \chi^{2}((\frac{p-1}{2})^2) \\

\chi^{3}(1^2) & \chi^{3}(2^2)  & \ldots & \chi^{3}((k-1)^2)  &\chi^{3}((k+1)^2)    & \ldots & \chi^{3}((\frac{p-1}{2})^2)\\

\vdots & \vdots & \ddots  & \vdots  & \vdots  & \ddots  & \vdots  \\

\chi^{\frac{p-5}{2}}(1^2) & \chi^{\frac{p-5}{2}}(2^2)  & \ldots & \chi^{\frac{p-5}{2}}((k-1)^2)  &\chi^{\frac{p-5}{2}}((k+1)^2)    & \ldots & \chi^{\frac{p-5}{2}}((\frac{p-1}{2})^2)\\

\chi^{\frac{p-3}{2}}(1^2) & \chi^{\frac{p-3}{2}}(2^2)  & \ldots & \chi^{\frac{p-3}{2}}((k-1)^2)  &\chi^{\frac{p-3}{2}}((k+1)^2)    & \ldots & \chi^{\frac{p-3}{2}}((\frac{p-1}{2})^2)\\
\end{array} \right|.
\end{displaymath}
Taking the common factor $\chi(j^2)$ from  the $j$-th column for $1\le j\ne k\le (p-1)/2$, we obtain
\begin{align}
  &  \det A_{(k)}\nonumber \\
  =&\(\prod\limits_{\substack{i=1\\i\neq k}}^{(p-1)/2}\chi (i^2)\)\cdot \det (V(\chi(1^2),\ldots,\chi((k-1)^2),\chi((k+1)^2),\ldots,\chi((\frac{p-1}{2})^2)))\nonumber \\
  =& \frac{\prod\limits_{\substack{i=1}}^{(p-1)/2}\chi (i^2)}{\chi(k^2)} \cdot \frac{\prod\limits_{1\le i<j\le (p-1)/2}(\chi (j^2)-\chi(i^2))}{(-1)^{\frac{p-1}{2}-k}\cdot \prod\limits_{\substack{i=1\\i\neq k}}^{(p-1)/2}(\chi (k^2)-\chi(i^2))}.
\end{align}
It follows from  (\ref{prodsquare}) and (\ref{inek}) that
\begin{equation}\label{ak}
  \det A_{(k)}=(-1)^{k+1}\cdot \frac{2}{p-1}\cdot I_p.
\end{equation}
Combining the above equalities with (\ref{detmp}) we obtain
\begin{align}\label{mp2}
 \det M_p=\frac{2}{p-1}\cdot I_p \cdot \sum_{k=1}^{(p-1)/2}(-1)^{k+1} \cdot \det B^{(k)}.
\end{align}

Now let $\widetilde{B}$ be the matrix obtained by adding a column vector of order $(p-1)/2$ with all elements $1$ before the first column of $B$. By determinant expansion formula with respect to the first column we have
  \begin{align}\label{mp3}
 \det M_p=\frac{2}{p-1}\cdot I_p \cdot \det  \widetilde{B}.
\end{align}
We next evaluate $ \det \widetilde{B}$. Let $\chi(j^2)=x_j$ for any $0\le j\le (p+1)/2$. Note that $x_0=0$, $x_1=1$ and $x_{\frac{p-1}{2}}=x_{\frac{p+1}{2}}$.
Then
\begin{align}
 \det  \widetilde{B} & =
 \left | \begin{array}{ccccc}
   1 & x_0+x_2 & x_0^2+x_2^2 &\cdots &   x_0^{\frac{p-3}{2}}+x_2^{\frac{p-3}{2}} \\
   1 & x_1+x_3 & x_1^2+x_3^2 &\cdots &   x_1^{\frac{p-3}{2}}+x_3^{\frac{p-3}{2}} \\
  1 & x_2+x_4 & x_2^2+x_4^2 &\cdots &   x_2^{\frac{p-3}{2}}+x_4^{\frac{p-3}{2}} \\
  \vdots & \vdots & \vdots & \ddots & \vdots \\
    1 & x_{\frac{p-3}{2}}+x_\frac{p+1}{2} & x_\frac{p-3}{2}^2+x_\frac{p+1}{2}^2 &\cdots &   x_{\frac{p-3}{2}}^{\frac{p-3}{2}}+x_{\frac{p+1}{2}}^{\frac{p-3}{2}} \\
 \end{array} \right |  .
\end{align}

 We first add the second, third,...,until the last row to the first row.
By (\ref{ek}) we have 
\begin{align}
&(x_0+x_2)+(x_1+x_3)+(x_2+x_4)+\ldots+(x_{\frac{p-3}{2}}+x_{\frac{p+1}{2}}) \nonumber   \\
&= -x_1+2\cdot (x_1+x_2+\cdots+x_{\frac{p-3}{2}}+x_{\frac{p-1}{2}})= -x_1
\end{align}

Similarly, 
 \begin{align}\label{ksum}
&(x_0^k+x_2^k)+(x_1^k+x_3^k)+(x_2^k+x_4^k)+\ldots+(x_{\frac{p-3}{2}}^k+x_{\frac{p+1}{2}}^k) =-x_1^k
\end{align}
for any $1\le k\le \frac{p-3}{2}$.
Consequently,

\begin{align}
 \det  \widetilde{B} & =
 \left | \begin{array}{ccccc}
   \frac{p-1}{2} & -x_1 & -x_1^2 &\cdots & -x_1^{\frac{p-3}{2}} \\
   1 & x_1+x_3 & x_1^2+x_3^2 &\cdots &   x_1^{\frac{p-3}{2}}+x_3^{\frac{p-3}{2}} \\
  1 & x_2+x_4 & x_2^2+x_4^2 &\cdots &   x_2^{\frac{p-3}{2}}+x_4^{\frac{p-3}{2}} \\
  \vdots & \vdots & \vdots & \ddots & \vdots \\
    1 & x_{\frac{p-3}{2}}+x_\frac{p+1}{2} & x_\frac{p-3}{2}^2+x_\frac{p+1}{2}^2 &\cdots &   x_{\frac{p-3}{2}}^{\frac{p-3}{2}}+x_{\frac{p+1}{2}}^{\frac{p-3}{2}} \\
 \end{array} \right | .
\end{align}
We next add the first row to the second row and get

\begin{align}
 \det  \widetilde{B}
 =& \left | \begin{array}{ccccc}
   \frac{p-1}{2} & -x_1 & -x_1^2 &\cdots & -x_1^{\frac{p-3}{2}} \\
    \frac{p+1}{2}&x_3 & x_3^2 &\cdots &   x_3^{\frac{p-3}{2}} \\
  1 & x_2+x_4 & x_2^2+x_4^2 &\cdots &   x_2^{\frac{p-3}{2}}+x_4^{\frac{p-3}{2}} \\
  \vdots & \vdots & \vdots & \ddots & \vdots \\
    1 & x_{\frac{p-3}{2}}+x_\frac{p+1}{2} & x_\frac{p-3}{2}^2+x_\frac{p+1}{2}^2 &\cdots &   x_{\frac{p-3}{2}}^{\frac{p-3}{2}}+x_{\frac{p+1}{2}}^{\frac{p-3}{2}} \\
 \end{array} \right | \nonumber  \\
 =-\frac{1}{2}\cdot &\left | \begin{array}{ccccc}
   -(p-1) & 2x_1 & 2x_1^2 &\cdots & 2x_1^{\frac{p-3}{2}} \\
    \frac{p+1}{2}&x_3 & x_3^2 &\cdots &   x_3^{\frac{p-3}{2}} \\
  1 & x_2+x_4 & x_2^2+x_4^2 &\cdots &   x_2^{\frac{p-3}{2}}+x_4^{\frac{p-3}{2}} \\
  \vdots & \vdots & \vdots & \ddots & \vdots \\
    1 & x_{\frac{p-3}{2}}+x_\frac{p+1}{2} & x_\frac{p-3}{2}^2+x_\frac{p+1}{2}^2 &\cdots &   x_{\frac{p-3}{2}}^{\frac{p-3}{2}}+x_{\frac{p+1}{2}}^{\frac{p-3}{2}}\\
 \end{array} \right | .
\end{align}
For the above determinant, we add the second, third,...,until the last row to the first row. Noting that for any $1\le k\le \frac{p-3}{2}$,
\begin{align}
&2x_1^k+x_3^k+(x_2^k+x_4^k)+\cdots+(x_{\frac{p-3}{2}}^k+x_{\frac{p+1}{2}}^k) \nonumber   \\
=& -x_2^k+2\cdot (x_1^k+x_2^k+\cdots+x_{\frac{p-3}{2}}^k+x_{\frac{p-1}{2}}^k)= -x_2^k.
\end{align}
we therefore obtain 
\begin{align}
 \det  \widetilde{B}
 =-\frac{1}{2}\cdot &\left | \begin{array}{ccccc}
   -1 & -x_2 &- x_2^2 &\cdots & -x_2^{\frac{p-3}{2}} \\
    \frac{p+1}{2}&x_3 & x_3^2 &\cdots &   x_3^{\frac{p-3}{2}} \\
  1 & x_2+x_4 & x_2^2+x_4^2 &\cdots &   x_2^{\frac{p-3}{2}}+x_4^{\frac{p-3}{2}} \\
  \vdots & \vdots & \vdots & \ddots & \vdots \\
    1 & x_{\frac{p-3}{2}}+x_\frac{p+1}{2} & x_\frac{p-3}{2}^2+x_\frac{p+1}{2}^2 &\cdots &   x_{\frac{p-3}{2}}^{\frac{p-3}{2}}+x_{\frac{p+1}{2}}^{\frac{p-3}{2}} \\
 \end{array} \right |  \nonumber \\
 =\frac{1}{2}\cdot &\left | \begin{array}{ccccc}
   1 & x_2 & x_2^2 &\cdots & x_2^{\frac{p-3}{2}} \\
    \frac{p+1}{2}&x_3 & x_3^2 &\cdots &   x_3^{\frac{p-3}{2}} \\
  1 & x_2+x_4 & x_2^2+x_4^2 &\cdots &   x_2^{\frac{p-3}{2}}+x_4^{\frac{p-3}{2}} \\
  \vdots & \vdots & \vdots & \ddots & \vdots \\
    1 & x_{\frac{p-3}{2}}+x_\frac{p+1}{2} & x_\frac{p-3}{2}^2+x_\frac{p+1}{2}^2 &\cdots &   x_{\frac{p-3}{2}}^{\frac{p-3}{2}}+x_{\frac{p+1}{2}}^{\frac{p-3}{2}} \\
 \end{array} \right |.
\end{align}
Finally, we recursively  use  row operation method to deduce that

\begin{align}
 \det  \widetilde{B}
 =\frac{1}{2}\cdot &\left | \begin{array}{ccccc}
   1 & x_2 & x_2^2 &\cdots & x_2^{\frac{p-3}{2}} \\
    \frac{p+1}{2}&x_3 & x_3^2 &\cdots &   x_3^{\frac{p-3}{2}} \\
   \alpha_4& x_4 & x_4^2 &\cdots &   x_4^{\frac{p-3}{2}} \\
  \vdots & \vdots & \vdots & \ddots & \vdots \\
    \alpha_\frac{p-1}{2} &  x_\frac{p-1}{2} & x_\frac{p-1}{2}^2 &\cdots &   x_{\frac{p-1}{2}}^{\frac{p-3}{2}} \\
    \alpha_\frac{p+1}{2} &  x_\frac{p+1}{2} & x_\frac{p+1}{2}^2 &\cdots &   x_{\frac{p+1}{2}}^{\frac{p-3}{2}} \\
 \end{array} \right |
\end{align}
where
\begin{equation}\label{alpha}
  \alpha_k=
  \begin{cases}
    \frac{1-(-1)^\frac{k}{2}}{2} & \mbox{if}\ 2\mid k,\\
    \frac{p+1}{2}\cdot (-1)^\frac{k+1}{2}+\frac{1+(-1)^\frac{k-1}{2}}{2}& \mbox{otherwise}. 
  \end{cases}
\end{equation}
Let $\beta =\alpha_\frac{p+1}{2}-\alpha_\frac{p-1}{2}$. One can verify that 
\begin{equation}
\beta=
\begin{cases}
  (-1)^\frac{p+3}{4}\cdot \frac{p-1}{2}, & \mbox{if } p \equiv 1 \pmod 4,\\
  (-1)^\frac{p-3}{4}\cdot \frac{p+1}{2}, & \mbox{if } p \equiv 3 \pmod 4.
\end{cases}
\end{equation}
As
\begin{align}
  (-1)^{\binom{(p-1)/2}{2} }=\begin{cases}
  (-1)^\frac{p-1}{4} & \mbox{if } p \equiv 1 \pmod 4,\\
  (-1)^\frac{p-3}{4} & \mbox{if } p \equiv 3 \pmod 4,
                             \end{cases}
\end{align}
we have 
\begin{equation}\label{beta}
  \beta=(-1)^{
    \binom{(p-1)/2}{2} +\frac{p+1}{2}}\cdot \frac{p+(-1)^\frac{p+1}{2}}{2}.
\end{equation}
Now by subtracting the $\frac{p-3}{2}$-th row from the $\frac{p-1}{2}$-th row and noting that $x_{\frac{p-1}{2}}=x_{\frac{p+1}{2}}$, we obtain 

\begin{align}\label{1B}
 \det  \widetilde{B}
 &=\frac{1}{2}\cdot \left | \begin{array}{ccccc}
   1 & x_2 & x_2^2 &\cdots & x_2^{\frac{p-3}{2}} \\
    \frac{p+1}{2}&x_3 & x_3^2 &\cdots &   x_3^{\frac{p-3}{2}} \\
   \alpha_4& x_4 & x_4^2 &\cdots &   x_4^{\frac{p-3}{2}} \\
  \vdots & \vdots & \vdots & \ddots & \vdots \\
    \alpha_\frac{p-1}{2} &  x_\frac{p-1}{2} & x_\frac{p-1}{2}^2 &\cdots &   x_{\frac{p-1}{2}}^{\frac{p-3}{2}} \\
    \alpha_\frac{p+1}{2}-\alpha_\frac{p-1}{2} &  0 & 0 &\cdots &   0 \\
 \end{array} \right |  \nonumber \\
 &= \frac{1}{2}\cdot (-1)^{\frac{p+1}{2}} \cdot \beta\cdot \left |\begin{array}{ccccc}
   x_2 & x_2^2 &\cdots & x_2^{\frac{p-3}{2}} \\
   x_3 & x_3^2 &\cdots &   x_3^{\frac{p-3}{2}} \\
  x_4 & x_4^2 &\cdots &   x_4^{\frac{p-3}{2}} \\
  \vdots & \vdots & \ddots & \vdots \\
   x_\frac{p-1}{2} & x_\frac{p-1}{2}^2 &\cdots &   x_{\frac{p-1}{2}}^{\frac{p-3}{2}} \\
 \end{array}\right|  \nonumber \\
  &=\frac{1}{2}\cdot (-1)^{\frac{p+1}{2}}\cdot \beta\cdot \(\prod\limits_{\substack{i=2}}^{(p-1)/2}x_i\)\cdot \det (V(x_2,x_3,\cdots,x_{\frac{p-1}{2}})) \nonumber \\
  &=\frac{1}{2}\cdot   \beta \cdot \prod_{2\le i<j\le (p-1)/2}(\chi (j^2)-\chi(i^2))  \nonumber\\
   &=\frac{1}{2}\cdot   \beta \cdot \frac{\prod\limits_{1\le i<j\le (p-1)/2}(\chi (j^2)-\chi(i^2)) }{ \prod\limits_{\substack{j=2}}^{(p-1)/2}(\chi (j^2)-1)} .
\end{align}
Combining (\ref{1B}) with (\ref{inek}) we see that 
\begin{equation}
  \det  \widetilde{B}=(-1)^{\frac{p-3}{2}}\cdot \frac{1}{p-1}\cdot \beta\cdot I_p,
\end{equation}
and hence 
 \begin{align}\label{mpbetaip}
 \det M_p&=\frac{2}{p-1}\cdot I_p \cdot \det  \widetilde{B}\nonumber \\
 &=(-1)^{\frac{p+1}{2}}\cdot \frac{2}{(p-1)^2} \cdot \beta\cdot I_p^2.
\end{align}
By (\ref{ipsquare}), (\ref{beta}) and (\ref{mpbetaip}) we finally obtain 
$$ \det M_p=\frac{p+(-1)^{\frac{p+1}{2}}}{4}  \(\frac{p-1}{2}\)^{\frac{p-5}{2}}(-1)^{\frac{p+1}{2}}.$$

This completes the proof.\qed

\section{Concluding Remarks}

Fix a positive integer $k$. Let $p\equiv 1\pmod k$ be a prime and let $\chi$ be a generator of $\widehat{\mathbb{F}_p^{\times}}$. Define 
$$J_p(k):=\left[J(\chi^{ki},\chi^{kj})\right]_{1\le i,j\le \frac{p-1}{k}-1},$$
which concerns the matrices involving Jacobi sums of $k$-th nontrivial multiplicative characters of $\mathbb{F}_p$. 

Theorems \ref{thjp}-\ref{thmp} have determined the explicit values of $\det J_p(1)$ and $\det J_p(2)$ resepectively. However, for an arbitrary integer $k\ge3$, it seems that determining $\det J_p(k)$ is very difficult, and that our method used in this paper cannot be directly applied to the calculations of general $\det J_p(k)$. 

On the other hand, Greene \cite[Definition 2.4]{Greene} posed an analogue of binomial coefficients. For any $A,B\in\widehat{\mathbb{F}_p^{\times}}$, Greene defined 
$$\binom{A}{B}:=\frac{B(-1)}{p}J(A,\overline{B}),$$
where $\overline{B}\in\widehat{\mathbb{F}_p^{\times}}$ such that $\overline{B}(x)=B(x)^{-1}$ for any $x\in\mathbb{F}_p^{\times}$. Inspired by Greene's analogue and Theorems \ref{thjp}--\ref{thmp}, investigating
$$\det\left[\binom{\chi^{i+j}}{\chi^i}\right]_{1\le i,j\le p-2}\ \text{and}\ 
\det \left[\binom{\chi^{2i+2j}}{\chi^{2i}}\right]_{1\le i,j\le (p-3)/2}$$
might be also meaningful. However, we cannot solve this problem currently. 

\section{Appendix}

In this appendix we prove (\ref{Eq. det Cn}). It is known that for any positive integers $r,l$, we have 
\begin{equation}\label{Eq. Appendix A}
	\binom{r}{l}=\binom{r-1}{l-1}+\binom{r-1}{l}.
\end{equation}

Now we calculate $\det C_n=\det \left[\binom{i+j}{i}\right]_{1\le i,j\le n}$. The case $n=1$ is trivial. Assume now $n\ge2$. Subtracting the $(n-1-i)$-th row from the $(n-i)$-th row for $i=0,1,\cdots,n-2$ sequentially and by (\ref{Eq. Appendix A}), one can verify that 
\begin{equation}\label{Eq. Appendix B}
	\det C_n=\det \left[\binom{i+j}{i}\right]_{1\le i,j\le n}=
	\left|\begin{array}{ccccc}
		           2 &            3 & \cdots & n+1\\
		\binom{2}{2} & \binom{3}{2} & \cdots & \binom{n+1}{2}\\
		\vdots       & \vdots       & \ddots & \vdots \\
		\binom{n}{n} & \binom{n+1}{n} & \cdots & \binom{2n-1}{n}
	\end{array}
    \right|.
\end{equation}
We next consider the right hand side of (\ref{Eq. Appendix B}). Subtracting the $(n-1-i)$-th column from the $(n-i)$-th column for $i=0,1,\cdots,n-2$ sequentially and by (\ref{Eq. Appendix A}) again, we obtain 
\begin{equation}\label{Eq. Appendix C}
	\det C_n=
	\left|\begin{array}{ccccc}
	   1+1 &            1+0   & \cdots & 1+0               & 1+0 \\
	     1 &   \binom{2}{1}   & \cdots & \binom{n-1}{1}    & \binom{n}{1} \\
	\vdots &   \vdots         & \ddots & \vdots            & \vdots \\
	     1 & \binom{n-1}{n-2} & \cdots & \binom{2n-4}{n-2} & \binom{2n-3}{n-2} \\
	     1 & \binom{n}{n-1}   & \cdots & \binom{2n-3}{n-1} & \binom{2n-2}{n-1}
	\end{array}
	\right|=\det D_n+\det C_{n-1}, 
\end{equation}
where 
$$\det D_n=\left|\begin{array}{ccccc}
	1      &            1     & \cdots & 1                 & 1 \\
	1      &  \binom{2}{1}    & \cdots & \binom{n-1}{1}    & \binom{n}{1} \\
	\vdots &  \vdots          & \ddots & \vdots            & \vdots \\
	1      & \binom{n-1}{n-2} & \cdots & \binom{2n-4}{n-2} & \binom{2n-3}{n-2} \\
	1      & \binom{n}{n-1}   & \cdots & \binom{2n-3}{n-1} & \binom{2n-2}{n-1}
\end{array}
\right|.$$
For $\det D_n$, subtracting the $(n-1-i)$-th row from the $(n-i)$-th row for $i=0,1,\cdots,n-2$ sequentially and using (\ref{Eq. Appendix A}), we see that 
$$\det D_n=\left|\begin{array}{ccccc}
	1      &            1     & \cdots & 1                 & 1 \\
	0      &  \binom{1}{1}    & \cdots & \binom{n-2}{1}    & \binom{n-1}{1} \\
	\vdots &  \vdots          & \ddots & \vdots            & \vdots \\
	0      & \binom{n-2}{n-2} & \cdots & \binom{2n-5}{n-2} & \binom{2n-4}{n-2} \\
	0      & \binom{n-1}{n-1} & \cdots & \binom{2n-4}{n-1} & \binom{2n-3}{n-1}
\end{array}
\right|=\left|\begin{array}{ccccc}
	1      & \cdots & \binom{n-2}{1}    & \binom{n-1}{1} \\
	\vdots & \ddots & \vdots            & \vdots \\
	1      & \cdots & \binom{2n-5}{n-2} & \binom{2n-4}{n-2} \\
	1      & \cdots & \binom{2n-4}{n-1} & \binom{2n-3}{n-1}
\end{array}
\right|.$$
Repeating the above procedure, one can easily verify that 
\begin{equation}\label{Eq. Appendix D}
	\det D_n=1.
\end{equation}
Combining (\ref{Eq. Appendix D}) with (\ref{Eq. Appendix C}) and noting that $\det C_1=2$, we finally obtain $\det C_n=n+1$. This completes the proof of (\ref{Eq. det Cn}).


\Ack\ This research was supported by the National Natural Science Foundation of China (Grant Nos. 12101321, 12201291 and 12071208).





\begin{thebibliography}{0}
	
\bibitem{carlitz}  L. Carlitz, Some cyclotomic matrices, Acta Arith. 5 (1959), 293--308.
	
\bibitem{chapman} R. Chapman, Determinants of Legendre symbol matrices, Acta Arith. 115 (2004), 231--244.

\bibitem{Cohen} H. Cohen, Number Theory, Vol. I: Tools and Diophantine Equations,  Springer, New York, 2007.

\bibitem{classical} K. Ireland, M. Rosen, A Classical Introduction to Modern Number Theory, 2nd ed., Springer, New York, 1990.

\bibitem{Greene} J. Greene, Hypergeometric functions over finite fields, Trans. Am. Math. Soc. 301 (1987), 77--101.

\bibitem{Lehmer} D. H. Lehmer, On certain character matrices, Pacific J. Math. 6 (1956), 491--499.

\bibitem{LN} R. Lidl and H. Niederreiter, Finite Fields, 2nd ed., Cambridge Univ. Press, Cambridge, 1997.

\bibitem{Krachun}  D. Krachun, F. Petrov, Z.-W. Sun, M. Vsemirnov, On some determinants involving Jacobi symbols, Finite Fields Appl. 64 (2020) 101672.

\bibitem{K1} C. Krattenthaler, Advanced Determinant Calculus. In: Foata D., Han GN. (eds) The Andrews Festschrift. Springer, Berlin, Heidelberg 2001.

\bibitem{K2}  C. Krattenthaler, Advanced determinant calculus: a complement, Linear Algebra Appl. 411 (2005) 68--166.

\bibitem{S19} Z.-W. Sun, On some determinants with Legendre symbols entries, Finite Fields Appl. 56 (2019), 285-307.

\bibitem{V1} M. Vsemirnov, On the evaluation of R. Chapman's ``evil determinant", Linear Algebra Appl. 436 (2012), 4101--4106.

\bibitem{V2} M. Vsemirnov, On R. Chapman's ``evil determinant": case $p\equiv1\pmod 4$, Acta Arith. 159 (2013), 331--344.

\bibitem{W21} H.-L. Wu, Elliptic curves over $\mathbb{F}_p$ and determinants of Legendre matrices, Finite Fields Appl. 76 (2021), 101929.

\bibitem{WSW} H.-L. Wu, Y.-F. She and L.-Y. Wang, Cyclotomic matrices and hypergeometric functions over finite fields, Finite Fields and Their Applications, 82 (2022), 102054.

\end{thebibliography}



\end{document}



