\section{Visualization - How to Show the Error?}
\label{sec:visualization}

The effectiveness of any argument does not solely rely on its content but also on its presentation.
This is where the art of visualization of argumentative feedback emerges as a crucial factor.
Visualizing feedback empowers individuals to perceive the intricacies of an argument in a more comprehensive and accessible manner.
By using visual aids like graphs and charts, feedback becomes more accessible and engaging, fostering constructive discussions.
In this section, we will see how visualization impacts argumentative feedback.

\paragraph{Highlights}

A first simple approach of visualization is highlighting, i.e., application of visual emphasis on specific pattern with the intention of drawing the viewer's attention on this specific pattern. For example, ~\citet{lauscher-etal-2018-arguminsci} identify the argument component (Claim, background, data) and visualizes them by highlighting the text in different colors. Similarly, ~\citet{chernodub-etal-2019-targer} allow the user to choose the model to use and the components to highlight. ~\citet{wambsganss-etal-2022-highlight} take a step further by presenting scores giving a quick overview of the user's skills. 

Highlighting serves as an essential key step in the cognitive input process, enabling viewers to quickly identify crucial argumentative structure. However, its use should be complemented with other visualization techniques to ensure a more profound exploration and comprehension of complex explanations. Studies conducted by ~\citet{lauscher-etal-2018-arguminsci, chernodub-etal-2019-targer, wambsganss-etal-2022-highlight} shed light on the potentials and limitations of highlighting, paving the way for future advancements in data visualization methodologies.

\paragraph{Multiple views}

To overcome the shallowness of highlighting, several researchers add not only a text editor to their system but also other views such as diagrams showing the argumentative structure. 
For example to compare two drafts of an essay, ~\citet{zhang-etal-2016-argrewrite, afrin-etal-2021-visualization} use a revision map made of color-coded tiles, whereas ~\citet{putra-etal-2021-tiara2} rely on a tree to reorder arguments.

Based on the work of ~\citet{wambsganss-etal-2020-al}, ~\citet{xia-etal-2022-persua, wambsganss-etal-2022-alen} use a text editor which highlights components, a graph view which shows an overview of the argumentative structure, and a score view showing the user's performance. Based on the classroom-setting evaluation, students using such systems wrote texts with a better formal quality of argumentation compared to the ones using the traditional approach. Nevertheless, the current accuracy of such systems' feedback still leave a large improvement space in order for users to be motivated to use them.

More recent work such as ~\citet{zhang-etal-2023-visar} incorporate feedback generated by state-of-the-art LLMs in their graphical systems. Nonetheless, factual inaccuracies, inconsistent or contradictory statements are still generated, exposing the user to confusion and leaving room for improvement.

\paragraph{Dialogue Systems}

In the realm of visualization, a novel approach gaining traction is the integration of dialogue systems to enhance the interaction between users and visual representations. Dialogue systems, commonly known as chatbots like ChatGPT, have been increasingly explored for their potential to improve critical thinking skills and facilitate information comprehension~\cite{rach-etal-2020-evaluation, wambsganss-etal-2021-arguetutor}. 

This kind of representation is challenging in term of the application's user-friendliness. Indeed, in an pedagogical context, it can be hard to track and visualize the user's progress. The user may also have difficulties in finding his previous assignments.

Despite the growing popularity of both graphs and chatbots in data visualization, limited work has directly compared their effectiveness in improving critical thinking skills. Further research is needed to provide more nuanced insights on the comparison on one hand between both approaches, on the other between works among the same approach. 

The importance of visualization in argumentative feedback lies in its ability to enhance the presentation and understanding of complex ideas.
This introductory section delved into the significance of visualization in argumentative feedback, highlighting its potential to improve students' learning process.