\section{Personalization - To Whom is it For?}
\label{sec:personalization}

Even if the explanations mentioned in the sections(\S\ref{sec:richness-what}) and (\S\ref{sec:richness-why}) are a step towards good guidance, they are static, which can be problematic depending on the end-user. Indeed beginners or professionals in argumentation do not need the same amount of feedback. A child and an adult have different levels of understanding and knowledge. Therefore it is essential that a model knows \textit{to whom} it should explain the errors and hence adapts its output by providing \textit{personalized} explanations.

\paragraph{Levels of explanations}

A first approach to personalization is to discretize the different users' backgrounds into a small number of categories based on their level of proficiency in argumentation. For example, with the systems described in ~\citet{wambsganss-etal-2020-al, wambsganss-etal-2022-alen}, users can choose their level (Novice, Advanced, Competent, Proficient, Expert).

Although ~\citet{wambsganss-etal-2020-al, wambsganss-etal-2022-alen} propose different granularity levels of explanations, their study is restrained to students from their university.
Having end-users from different backgrounds may imply the need for new levels of explanations. Indeed, ~\citet{wachsmuth-alshomary-2022-mama} show that the age of the explainee changes the way an explainer explains the topic at hand. Information such as the learner's age should be considered in future interactive argumentative feedback systems, where terminology such as \textit{fallacy} and their existence would require different approaches of explanation for younger students (i.e., elementary) compared to older students.

\paragraph{Self-personalization}
For more personalized feedback, some systems such as ~\citet{hunter-metal-2019-personalization, putra-etal-2020-tiara} rely on the user's input.
For example, they allow users to make their custom tags or to choose their preferences among a set of rubrics. Nevertheless, letting the user manually personalize the system can be overwhelming and time-consuming for users. 

\paragraph{Next directions}

~\citet{hunter-metal-2019-personalization} argue that the next direction for personalized argumentative feedback would be to develop argumentation chatbots for persuasion and infer the user's stance based on the discussion. Chatbots' personalization capabilities enabling them to tailor their responses to individual learners' needs and learning styles, potentially enhancing the effectiveness of the tutoring process\cite{lin-etal-2023-reviewChatbot}. 
Bridging the gap between personalized chatbots \cite{qian-etal-2021-pchatbot, ma-etal-2021-chatperperson}, personalized educational methods \cite{gonzalez-etal-2023-persoEdu, ismail-etal-2023-persoSurvey} and argumentation has remained unexplored.

Therefore we think researchers should focus furthermore in the future on providing more \textit{personalized} explanations (i.e., precisely adjusted by considering the background of the learner) to efficiently improve the critical thinking skills of an end-user.
 