\section{Interactivity - Who Talks to the User?}
\label{sec:interactivity}

Teaching argumentation is a multifaceted task that demands more than the dissemination of theoretical knowledge; it requires fostering interactive learning environments that facilitate active engagement and practice. The traditional approach to teaching argumentation often centers on lecturing and one-way communication, where instructors impart information to students. While didactic methods have their place in education, a more interactive pedagogical approach, one that encourages learners to actively participate, can be used. In this section we will see in which ways current argumentative computational models enable a form of interaction. 

\paragraph{Interaction between different users}

NLP systems mostly allow communication between a user and a conversational agent. Nonetheless, some works chose to apply the CABLE pedagogical method (cf section (\S\ref{sec:open_issues})) and allow a user to dialog with other users. Following the footsteps of \citet{petasis-2014-nomad}, \citet{lugini-etal-2020-discussion} track real-time class discussions and help teachers annotate and analyze the discussions.

Even if recent research works such as ~\citet{zhang-etal-2023-visar} plan to add a collaborative setting, we realize through our survey that only a few works focus on collaboration between multiple users. 
The concept of collaboration between multiple users within NLP systems is promising. However, it is essential to acknowledge that some challenges and barriers have hindered its widespread adoption in research works, possibly due to the difficulty of designing and evaluating such tools, as they require important human resources. 

\paragraph{Interaction with a conversational agent}

As seen in the section (\S\ref{sec:visualization}), several research papers have showcased the feasibility of employing current conversational agents for educational purposes ~\cite{lee-etal-2022-bot, macina-etal-2023-bot, wang-etal-2023-bot}. Often based on state-of-the-art language models, these agents have shown great capabilities in understanding and generating human-like responses. They can engage in dynamic and contextually relevant conversations, making them potentially valuable tools for educational purposes.

The use of conversational agents as dialog tutors has been explored outside of argumentation ~\cite{wambsganss-etal-2021-arguetutor, mirzababaei-pammer-2022-bot, aicher-etal-2022-towards}. For instance, in ~\citet{mirzababaei-pammer-2022-bot}, an agent examines arguments to determine a claim, a warrant, and evidence, identifies any missing elements, and then assists in completing the argument accordingly. ~\cite{wambsganss-etal-2021-arguetutor} create an interactive educational system that uses interactive dialogues to teach students about the argumentative structure of a text. The system provides not only feedback on the user's texts but also learning session with different exercises.

Research on chatbots in education is at a preliminary stage due to the limited number of studies exploring the application of effective learning strategies using chatbots. This indicates a significant opportunity for further research to facilitate innovative teaching methods using conversational agents~\cite{hwang-chang-2021-chatbotChallenges}. Indeed, pre-work, extraction, and classification of useful data remain challenging as the data collected are noisy, and much effort still has to be made to make it trainable \cite{lin-etal-2023-reviewChatbot}.

Future researchers must also account for ethical considerations associated with chatbots, including value-sensitive design, biased representations, and data privacy safeguards, to ensure that these interactive tools positively impact users while upholding ethical standards \cite{kooli-2023-chatbot}. 

Overall, integrating interaction in teaching argumentation is not merely a pedagogical choice but an essential requirement to cultivate adept arguers who can navigate the intricacies of argumentation. Therefore, we encourage researchers to consider this dimension in their future pedagogical systems.