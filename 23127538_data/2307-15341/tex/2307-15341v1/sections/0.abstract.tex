\begin{abstract}

The use of argumentation in education has been shown to improve critical thinking skills for end-users such as students, and computational models for argumentation have been developed to assist in this process. Although these models are useful for evaluating the quality of an argument, they oftentimes cannot explain why a particular argument is considered poor or not, which makes it difficult to provide constructive feedback to users to strengthen their critical thinking skills.
In this survey, we aim to explore the different dimensions of feedback (Richness, Visualization, Interactivity, and Personalization) provided by the current computational models for argumentation, and the possibility of enhancing the power of explanations of such models, ultimately helping learners improve their critical thinking skills.

\end{abstract}
