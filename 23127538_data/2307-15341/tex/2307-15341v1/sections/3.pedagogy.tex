\section{Pedagogy}
\label{sec:pedagogy}

Based on ~\citet{rapanta-etal-2013-competence}, researchers do not agree on a uniform definition of argumentation competence, nor is there a universally accepted method or guideline to analyze and evaluate the various components of argumentation competence, making it challenging for NLP researchers to choose a pedagogical method to use to generate feedback. This section presents some standard pedagogical methods used in teaching argumentation.

\paragraph{Toulmin model}
The Toulmin model, often seen as the founding of teaching argumentation, is a popular framework for analyzing, constructing, and evaluating arguments, can contribute to the improvement of students’ argumentative writings ~\cite{rex-etal-2010-toulmin, yeh-1998-toulmin} as well as critical thinking skills~\cite{giri-2020-toulmin}. This approach deconstructs an argument into six elements: claim, data, warrant, backing, qualifier, and rebuttal, and students are taught to identify each element within an argument. 

\paragraph{Collaborative argumentation}
In collaborative argumentation-based learning, also described as CABLE by \citet{baker-etal-2019-collabo}, individuals or groups work together to construct, refine, and evaluate arguments on a particular topic or issue. The main goal of collaborative argumentation is to foster constructive dialogue, critical thinking, and the exploration of different perspectives.

~\citet{weinberger-fischen-2006-collabo} differentiate four dimensions of CABLE:
\begin{itemize}
    \item \textit{Participation}: Do learners participate at all? Do they participate on an equal basis? 
    \item \textit{Epistemic}: Are learners engaging in activities to solve the task (on-task discourse) or rather concerned with off-task aspect?
    \item \textit{Argumentative}: Are learners following the structural composition of arguments and their sequences?
    \item \textit{Social}: To what extent do learners refer to the contributions of their learning partners? Are they gaining knowledge by asking questions?
\end{itemize}

~\citet{veerman-etal-2002-collabo, baker-etal-2019-collabo} show the positive effects of CABLE on students' development of argumentation. 

\paragraph{Tree-based feedback}
A customary approach to giving feedback to students is through the provision of free text feedback. Text-based feedback allows for more in-depth analysis and can be particularly useful when providing specific examples or referencing external resources.
Nevertheless, it is worth noting that generating free-text feedback can be time-consuming. Conversely, the advent of tools like ~\cite{putra-etal-2020-tiara} has given rise to tree-based feedback. 
This feedback form employs graphical representations, such as concept maps or mind maps, to illustrate the relationships between ideas or concepts. This visual approach can help students visualize the connections between different concepts and enhance their understanding of complex topics ~\cite{matsumura-sakamoto-2021-tree}.

\paragraph{Socratic questioning}

The Socratic questioning is a teaching strategy commonly used in education, where the student is guided, through reflexive questions, towards solving a problem on their own, instead of being given the solution directly described in ~\cite{schauer-2012-socratic, abrams-2015-socratic}.
Recently, this method has been integrated into Large Language Models (LLMs) to more effectively adhere to user-provided queries ~\cite{ang-etal-2023-socratic, pagnoni-etal-2023-socratic}, to enhance the ability of such models in generating sequential questions~\cite{shridhar-etal-2022-automatic}, but also enhancing the explainability of these models ~\cite{al-hossami-etal-2023-socratic}.

Nevertheless, the Socratic questioning is now raising debates among researchers focusing on pedagogy in argumentation. Indeed ~\citet{kerr-1999-socratic, christie-2010-socratic} pointed out its inefficiency and abusiveness as students are forced to give imperfect answers in a hurry and endure criticism.