\documentclass[aps,epsf,preprint]{revtex4}
\usepackage{amssymb}
\usepackage{amsmath}
\usepackage{color}
%\input colornam.sty
%%%%%%%%%%%%%%%%%%%%
\def\cblack{\color{black}}
\def\tcblack{\textcolor{black}}
\def\tcmagenta{\textcolor{magenta}}
\def\cblue{\color{blue}}
\def\tcblue{\textcolor{blue}}
\def\cbluel{\color{MidnightBlue}}
\def\tcbluel{\textcolor{MidnightBlue}}
\def\ccyan{\color{cyan}}
\def\tccyan{\textcolor{cyan}}
\def\cred{\color{red}}
\def\tcred{\textcolor{red}}

\def\[{\left[}
\def\]{\right]}
\def\be{\begin{eqnarray}}
\def\ee{\end{eqnarray}}
\def\bm{\begin{pmatrix}}
\def\em{\end{pmatrix}}
\def\ba{\begin{array}}
\def\ea{\end{array}}
\def\bi{\begin{itemize}}
\def\ei{\end{itemize}}
\def\nn{\nonumber}
\def\({\left(}
\def\){\right)}
\def\bk#1{\langle#1\rangle}
\def\eq#1{Eq.(\ref{#1})}
\def\a{\alpha}
\def\s{\sigma}
\def\e{\epsilon}
\def\D{\Delta}
\def\bD{\bar\Delta}
\def\f{\phi}
\def\q{\psi}
\def\l{\lambda}
\def\m{\mu}
\def\w{\omega}
\def\x{\times}
\def\ket#1{|#1\rangle}
\def\bra#1{\langle #1|}
\def\n{\nu}
\def\p{\partial}
\def\d{\delta}
%\def\labels#1{\quad [#1]\label{#1}}
\def\labels#1{\label{#1}}
\def\bn{\begin{enumerate}}
\def\i{\item}
\def\en{\end{enumerate}}
\def\b{\beta}
\def\g{\gamma}
\def\ba{\begin{array}}
\def\ea{\end{array}}
\def\bc{\begin{center}}
\def\ec{\end{center}}
\def\hs{\hskip.5cm}
\def\ni{\noindent}
\def\.{\!\cdot\!}
\def\igw#1{\includegraphics[width=#1cm]}
\def\igwg#1#2{\igw{#1}{#2.png}} %%%%
\def\+{\!+\!}
\def\-{\!-\!}
\def\vs{\vskip.5cm}
\def\bsl{\backslash}
\def\M{M\"obius\ }
\def\Q{\Psi}
\def\pf{{\rm Pf}}
\def\r{\rho}
\def\h{{1\over 2}}
\def\Rw{\Rightarrow}
\def\={\stackrel{.}{=}}
\def\c{{\mathfrak c}}
\def\CC{{\mathbb C}}
\def\ol{\overline}
\def\bn{\bar n}
\def\1{\bar 1}
\def\8{\bar 8}
\def\9{\bar 9}
\def\7{\bar 7}
\def\ul{\underline}

\usepackage{amssymb}
\usepackage{graphicx}


\begin{document}
\title{A Model of the Black Hole Interior}
\author{C.S. Lam}
\affiliation{Department of Physics, McGill University\\
 Montreal, Q.C., Canada H3A 2T8\\
Department of Physics and Astronomy, University of British Columbia,  Vancouver, BC, Canada V6T 1Z1 \\
Email: Lam@physics.mcgill.ca}



\begin{abstract}
A model is proposed for the interior of a neutral non-rotating black hole. It consists of an ideal fluid with density 
$\r$ and a negative pressure $p$, obeying an equation of state $p=-\xi\r$. In order  to have a solution, $\xi$
must lie in the narrow range between  0.1429  and 0.1716.
Pressure, density, mass, temperature, and entropy distributions in the interior are computed. Due to
the presence of pressure, the surface temperature
$T_H$ and the total entropy $S_H$ do not obey the Bekenstein relation. In this model, $S_H$ is proportional to
the surface area as usual, but entropy is concentrated near the center of the blackhole rather than on its surface, making the model non-holographic.
\end{abstract}

\maketitle

\section{Introduction}
The interior of a black hole is a mystery. When a star collapses into a black hole, its matter loses 
baryonic number, fermionic number, isotopic spin, hypercharge etc.,
to become a different kind of matter that shall be referred to as black-hole matter. The nature of this new kind of matter 
is largely unknown, and its distribution
inside the black hole is also unknown, except that there is a singularity at the center  \cite{aK56, aR57, rP65, SG14, kL22} which can likely be washed out when quantum effect is taken into account \cite{mB07, aP17, ABP19, AQS23}. 


In this article we propose a model of the black-hole matter, described by an ideal fluid with a positive mass density $\r$,
and an equation of state $p=-\xi\r$ with a negative pressure. To prevent matter from sinking to the center,
a repulsive force produced by a negative pressure is required. In order to have a solution,  it turns out that only a very narrow range of $\xi$ between 0.1429  and 0.1716 is allowed. 
In this way the present model differs from those \cite{MM15, BMN19, MM23} where black-hole matter is made up of dark energy with $\xi=1$.
The density
and the metric of this model have a singularity at the origin as per the singularity theorem, but the internal mass in a finite volume
is finite and  vanishes as the volume goes to zero. 

We will compute the density, mass, temperature and the entropy distributions in the interior of the black hole, 
from which the surface temperature $T_H$ and the total entropy
$S_H$ can be extracted. Due to the presence of pressure, Bekenstein's relation \cite{jB73} is not quite satisfied in this model.
Moreover, the black hole entropy turns out to be concentrated near the center and not on the surface, making this model  non-holographic, although $S_H$ is still proportional to the surface area of the black hole as usual.

\section{The Model}
Consider a spherical black hole given by the line element \cite{sC04}
\be ds^2&=&-e^{2\a(r)}dt^2+e^{2\b(r)}dr^2+r^2 d\Omega^2.\labels{ds2}\ee
Vacuum is assumed to be   outside the black hole, so the line element outside is given by the Schwarzschild metric, 
\be e^{2\a(r)}&=&\(1-{2GM\over r}\)\ {\rm and\quad } e^{2\b(r)}=\(1-{2GM\over r}\)^{-1}\quad {\rm for\ }r>R,\labels{sr}\ee
where $M$ is the mass of the black hole, $G$ is the gravitational constant, and $R=2GM$  is the radius of its horizon. 

Matter is assumed to have a volume distribution inside the blackhole, with its energy-momentum
tensor  given by that of an ideal fluid,
\be T_{\m\n}=(p+\r)U_\m U_\n+p g_{\m\n}. \labels{semt}\ee
The four-velocity is as usual normalized to $U^\m U_\m=-1$. The energy density $\r$ is assumed to be positive, but 
the pressure determined by the equation of state  $p=-\xi\r$ can be either positive ($\xi<0$) or negative ($\xi>0$)
at this point. 


With this energy-momentum tensor,  Einstein equation leads to the Tolman-Oppenheimer-Volkoff equation
for the pressure gradient  \cite{sC04, OV39} 
\be {dp\over dr}&=&-{(\r+p)[Gm(r)+4\pi Gr^3p]\over r[r-2Gm(r)]},\qquad {\rm where}\labels{dpdr}\\
m(r)&=&4\pi\int_0^r\r(r'){r'}^2dr'\labels{massr}\ee
is the mass of a black-hole matter ball of radius $r\le R$.
The  metric functions $\a$ and $\b$ inside the black hole are given by the equations
\be {d\a\over dr}&=&-{1\over (\r+p)}{dp\over dr},\labels{dadr}\\
e^{2\b}&=&\[1-{2Gm(r)\over r}\]^{-1}.\labels{2b}\ee
In particular, $e^{2\b}$ has the same form across the horizon.

Since there is no pressure outside the blackhole, we require $p(R)=0$ for continuity. Hence $\r(R)=0$ and
$(dm/dr)(R)=0$. For $R-r$ small and positive, \eq{dpdr}
can be approximated by
\be {dp\over dr}=-{(\r+p)GM\over R[r-2GM]}={\xi-1\over 2\xi}{p\over R-r},\quad (R-r\ll R),\labels{dpdrR}\ee
whose solution for small and positive $R-r$ is
\be p(r)\simeq -\tilde c(R-r)^{(1-\xi)/2\xi}\labels{pR}\ee
for some $\tilde c$. In order for $p(R)=0$, it is necessary to have $0<\xi<1$, so the black-hole matter necessarily carries
a negative pressure. 


For small $R-r$, \eq{dadr} and \eq{dpdr} imply
\be {d\a\over dr}\simeq {GM\over R}{1\over r-2GM}={1\over 2(r-R)},\labels{2a}\ee
so it admits the solution $e^{2\a}\simeq (r-R)/R\simeq(1-2GM/r)$, matching the value in \eq{sr}  across the horizon.



It would be simpler to write the Tolman-Oppenheimer-Volkoff equation in a dimensionless form. To that end,
let $x=r/R,\ \bar m(x)=m(r)/M$, and $\bar \r(x)=\r(r)(R^3/M)$. Then \eq{dpdr} can be written as
\be
{d\bar \r(x)\over dx}&=&-{\xi-1\over 2\xi}{\bar \r(x)\over x}{\bar m(x)-4\pi\xi x^3\bar \r(x)\over x-\bar m(x)},\labels{dless1}\\
{d\bar m(x)\over dx}&=&4\pi x^2 \bar \r(x), \labels{dless2}\ee
with the boundary condition $\bar\r(1)=0$ and $\bar m(1)=1$. The interior solution equivalent to \eq{pR} near $x=1$ is then
\be
\bar \r(x)&\simeq& c(1-x)^{(1-\xi)/2\xi},\\
\bar m(x)&\simeq&1-{8\pi \xi c\over \xi+1}(1-x)^{(\xi+1)/2\xi},\qquad (1-x\ll 1)\label{solrm}\ee
for some $c>0$. 

Let us turn to the behavior near $x=0$. Assuming the mass function $\bar m(x)$ to increase
steadily from  0 to 1 in the interval $x\in(0,1)$, then for small $x$, $\bar m(x)=\m_0x^\b$ for some $\b>0$ and $\m_0>0$. It follows from \eq{dless1} that $\bar \r(x)=(\m_0\b/4\pi) x^{\b-3}$, so $x^3\bar \r(x)$ is proportional to $\bar m(x)$,
and  $\bar m(x)-4\pi\xi x^3\bar \r(x)=\bar m(x)(1-\b\xi)$.

If $\b>1$, then $x-\bar m(x)\simeq x$ for sufficiently small $x$, in which case the $x$-behavior
on both sides of \eq{dless1} can never match. 

This shows that $\b>1$ is not allowed. As a result, $\bar\r(x)\sim x^{\b-3}$ must diverge near $x=0$. \eq{dadr} and \eq{2b} show that the metric functions also misbehave. Specifically, from \eq{dadr},
one gets $e^{2\a}\simeq (\bar\r/\bar\r_0)^{2\xi/(1-\xi)}$, which also diverges near $x=0$, but it remains positive throughout
the interior of the black hole. Hence the signature of the metric changes from $(-,+,+,+)$ outside the black hole to
$(-,-,+,+)$ inside, whereas the signature $(+,-,+,+)$ of the Schwarzschild metric   inside is that of a Kantowski-Sachs metric.

The presence of a singularity is hardly surprising \cite{aK56, aR57, rP65, kL22, SG14}. Using the Raydhuchaudri equation, it can be shown that a singularity is present at $r=0$ if the convergence condition
 $R_{\m\n}U^\m U^\n\ge 0$ is satisfied. In the present case, the Einstein equation
$R_{\m\n}-\h R g_{\m\n}=8\pi G T_{\m\n}$ implies 
$R_{\m\n}=8\pi G(T_{\m\n}-\h T^a_\a g_{\m\n})=8\pi G\[(p+\r)U_\m U_\n+\h(\r-p)g_{\m\n}\]$, hence
$R_{\m\n}U^\m U^\n=4\pi G\r(1-3\xi)$. With $\xi<1/3$, as shall be presently shown, the 
convergence condition is met so a singularity is present.

If $\b<1$, then $x-\bar m(x)\simeq -\bar m(x)$ for sufficiently small $x$, so \eq{dless1} is satisfied when
\be \b=(\-1 \+ 7 \xi)/\xi (1 \+ \xi).\labels{beta}\ee
In order for $\b$ to be between 0 and 1, $\xi$ is allowed only a narrow range of values between 
$\xi=1/7\simeq 0.1429$ (when $\b=0$) and  $\xi=3-2\sqrt{2}\simeq 0.1716$ (when $\b=1$).
For larger values of $\xi$, this formula yields $\b>1$, so it is not allowed. 

The value $\xi=3-2\sqrt{2}$ when $\b=1$ cannot be reached. With $\b=1$, both terms in $x-\bar m(x)$
in \eq{dless1} are of the same order so both must be retained. In that case, \eq{dless1} can be satisfied only if
\be
\m_0={4\xi\over 6\xi-1-\xi^2},\qquad (\b=1),\ee
but at the value $\xi=3-2\sqrt{2}$, $\m_0=\infty$, so that cannot be reached.

A numerical solution for $\xi=0.16$, somewhere in the middle of the allowed range, is shown in Fig.~1 as an illustration. 
This solution is obtained numerically by integrating \eq{dpdr} and \eq{massr}, starting at an initial $x$ so small that the approximation  $\bar m(x)=\m_0x^\b$ and $\bar \r(x)=(\m_0\b/4\pi) x^{\b-3}$ is accurate. The
value of $\b$ given by \eq{beta}, and the constant $\m_0$ is adjusted to yield $\bar m(1)=1$ and $\bar\r(1)=0$ at the
other boundary.

\bc\igwg{12}{Fig1}\\ Fig.~1.\quad Scaled mass and density distribution in the black-hole interior for $\xi=0.16$\ec

\section{Temperature and Entropy}
It is well known that a neutral and non-rotating black hole with mass $M$ satisfies the first law of black hole dynamics: $dM=(\kappa/8\pi G)dA$, where $\kappa$ is its surface
gravity and $A$ is the area of its event horizon. Moreover, it is known that $\kappa$ is constant throughout
the surface of the black hole and $A$ cannot decrease \cite{jB73, rW04}.
These properties can be easily verified for a Schwarzschild black hole where $\kappa=GM/R^2$, $R=2GM$,  and $dA=8\pi RdR$.  Their similarity  with the first and second laws of thermodynamics led Bekenstein \cite{jB73} to propose that
the black-hole entropy $S_H$ is proportional to $A$, and the black-hole temperature $T_H$ is proportional to $\kappa$,
and that the first law of black-hole dynamics $dM=(\kappa/8\pi G)dA$ can be interpreted as the first law of thermodynamics
$dM=T_HdS_H$. A more involved analysis can be found in \cite{bD11}. 
Since temperature does not enter into the usual treatment of
classical black holes, this identification with the first law of thermodynamics remains a conjecture within classical relativity.

However, if  the interior structure of the black hole is known and is a thermodynamical system, 
then  temperature,
entropy, as well as pressure and density can be computed, and the  Bekenstein's conjecture can be verified.
We shall do so below assuming the interior of the black hole to consist of the black-hole matter discussed in the last section.

 Let $S(r)$ be the entropy of a ball of black matter with radius $r\le R$, and $T(r)$ be its surface temperature. 
 Since $T$ has the dimension of $1/r$ (in $\hbar=c=1$ units) and $S$ is dimensionless, namely of the dimension
 $r^2/G$, it follows that $T^2(r)S(r)G:=\eta$ is dimensionless. We shall assume $\eta$ to be a universal constant, independent of 
 the black-hole size $R$, or $r$. 
 
 The amount of heat flowing into the black-energy ball can be computed from
 the first law of thermodynamics to be
\be T(r)dS(r)=dm(r)+p(r)(4\pi r^2dr)=4\pi r^2\[\r(r)+p(r)\]dr=(1-\xi)4\pi r^2\r(r)dr.\labels{firstlaw}\ee
Using $T(r)=\sqrt{\eta/GS(r)}$ so that $T(r)dS(r)=2\sqrt{\eta/G}\ d\(\sqrt{S(r)}\)$, we can integrate
the first law from $r=0$ to $r=R$. Assuming $S(0)=0$, one gets $\sqrt{4\eta/G}\ \sqrt{S(R)}=(1-\xi)M$,
which implies 
\be S(R)&=&{(1-\xi)^2\over 4\eta}GM^2={(1-\xi)^2\over 16\pi \eta}{A\over 4G},\\
 T(R)&=&\sqrt{{\eta\over GS(R)}}={2\eta\over 1-\xi}{1\over GM}={16\pi\eta\over 1-\xi}{1\over 4\pi R}.\ee
If we make the identification $T_H=T(R)$ and $S_H=S(R)$, then
\be T_H dS_H=(1-\xi)dM,\labels{TdS}\ee
which differs from Bekenstein conjecture by an extra term proportional to $\xi$. This comes about
 because part of the heat
 flowing into the system is used to do work, in addition to increasing the mass value of $M$.
 
 With $\sqrt{S(r)}=(1-\xi)(G/4\eta)^\h\  \bar m(r)$, and $\b<1$, $\bar m(r)$ and $S(r)$ rise fairly quickly
 with $r$, as illustrated in Fig.~1. Although $S_H$ is proportional to the surface area $A$, this model
 is not holographic because entropy as well as dynamical degrees of freedom are spread throughout the volume, rather than being concentrated on the surface of the black hole.

I am grateful to Bei-Lok Hu for discussions and suggestions.

\begin{thebibliography}{9}
\bibitem{aK56} Komar, A., Necessity of singularities in the solution of the field equations of general relativity. Phys.
Rev. 104, 544–546 (1956)
\bibitem{aR57} Raychaudhuri, A.K., Singular state in relativistic cosmology. Phys.~Rev. 106,172–173 (1957)
\bibitem{rP65} Penrose, R., Gravitational collapse and space-time singularities. Phys.~Rev.~Lett. 14, 57–59 (1965)
\bibitem{SG14} Senovilla, J.M.M and Garfinkle, D.,  The 1965 Penrose singularity
theorem. Class. Quantum Grav. 32, 124008 (2015)
\bibitem{kL22} K. Landsman, Penrose’s 1965 singularity theorem: from geodesic incompleteness to cosmic censorship.
Class. Quantum Grav. 54, 115 (2022)
\bibitem{mB07} Martin Bojowald, Singularities and Quantum Gravity.  AIP Conference Proceedings 910, 294–333 (2007)
\bibitem{aP17} Alejandro Perez, Black Holes in Loop Quantum Gravity.     Rept. Prog. Phys. 80, 126901 (2017)
\bibitem{ABP19} Emanuele Alesci, Sina Bahrami, Daniele Pranzetti, Quantum gravity predictions for black hole interior geometry. Physics Letters B797, 134908 (2019) 
\bibitem{AQS23} Abhay Ashtekar, Javier Olmedo, Parampreet Singh, Regular black holes from Loop Quantum Gravity,
 	arXiv:2301.01309 
\bibitem{MM15} P.O. Mazur and E. Mottola, Surface Tension and Negative Pressure Interior of a Non-Singular ‘Black Hole’.
Class. Quantum Grav. 32, 215024 (2015)
\bibitem{BMN19} R. Brustein and A.J.M. Medved, Resisting collapse: How matter inside a black hole can withstand gravity,
Phys. Rev. D99, 064019 (2019)
\bibitem{MM23}	  P.O. Mazur and E. Mottola, Gravitational condensate stars: an alternative to black holes. 
Universe 9, 88 (2023)
\bibitem{jB73} Jacob D. Bekenstein, Black holes and entropy. Phys. Rev. D7, 2333 (1973)
\bibitem{sC04} We follow the notations of Sec.~5.8 in the book by Sean M. Carroll, Spacetime, and Geometry: An Introduction to General
Relativity, Addison Wesley (2004).
\bibitem{OV39} Oppenheimer, J.R. and Volkoff, G.M. : On massive neutron cores. Phys. Rev. 55, 374–381 (1939)
\bibitem{rW04} Robert M. Wald, The thermodynamics of black holes. Living Reviews in Gravity 4, 6 (2001)
\bibitem{bD11} B.P. Dolan, Pressure and volume in the first law of thermodynamics.  Class. Quantum Grav. 28, 235017 (2011)
\end{thebibliography}
\end{document}
