\section{Conclusion}
\label{sec:conc}
%Heterogeneous systems with integrated accelerators are proliferating. In these systems, several microarchitectural resources such as MC and its buffers are shared across CPU cores and integrated accelerators with a high degree of parallelism and memory traffic.
%similar microarchitecutal design techniques are deployed even with the presence of integrated accelerators with high degree of parallelism and memory traffic, and which share these microarchitectural components such as MC buffers. 
In this paper, we characterized the slowdown observed in the CPU process due to iGPU kernel memory writes. We showed that it is possible to cause a very large slowdown to CPU process memory reads by just utilizing 1/3 of iGPU available resources to issue memory write requests. We confirmed that this slowdown happens due to the management policy of write buffer and not any other shared resources (i.e. LLC, ring-interconnect, or DRAM banks). We proposed two covert channel attack variants which exploit this phenomenon to leak secret information. The first variant is oblivious of the used MC channel, while the second variant is targeting a specific MC channel to improve the bit rate %since only one pair of read and write buffers is targeted. %In our future work, we aim to investigate approaches to improve the bit rate by utilizing iGPU kernel memory writes.

We believe our attacks demonstrate a critical leakage vector across the components in modern heterogeneous systems that guide future research into designs that are not only high-performance but also secure. 

\begin{comment}
This paper characterizes the contention on the memory controller resources (specifically request queues) in a heterogeneous SoC and develops high-quality covert channels between CPU and iGPU by exploiting parallel memory writes from iGPU. We also exploit the contention through parallel memory requests on the memory controller to present a proof-of-concept side channel attack in which a CPU-based spy extracts information about the application that is running on iGPU.

We believe our attacks demonstrate a critical leakage vector across the components in modern heterogeneous systems that guide future research into designs that are not only high-performance but also secure. In the future, given the growing number and range of domain-specific accelerators (such as AI accelerators, video and audio accelerators, and many others) being used in heterogeneous SoCs, the secure shared use of resources in multi-tenant environments and the secure integration of these accelerators to the rest of system merit further investigation.
\end{comment}