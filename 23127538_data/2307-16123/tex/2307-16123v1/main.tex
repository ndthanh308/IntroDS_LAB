\documentclass[pageno]{jpaper}
\pdfoutput=1
\usepackage{cite}
\usepackage{amsmath,amssymb,amsfonts}
\usepackage{algorithmic}
\usepackage{graphicx}
\usepackage{array}
\usepackage{textcomp}
\usepackage{xcolor}
%\usepackage[hyphens]{url}
\usepackage{flushend}

\usepackage{algorithmic}
\usepackage[ruled,vlined,linesnumbered]{algorithm2e}
%\usepackage{subfigure}
 

\usepackage{hyperref}
\usepackage{comment}


\usepackage{tikz}
\usepackage{xcolor}
\usepackage{pifont}
\usepackage{tcolorbox}
%\usepackage{subcaption}
\usepackage{wrapfig}

   %correct mark
\newcommand{\cmark}{\ding{52}}
%x mark
\newcommand{\xmark}{\ding{56}}%


\newcommand*\circled[1]{\tikz[baseline=(char.base)]{
            \node[shape=circle,fill,inner sep=1pt] (char) {\textcolor{white}{#1}};}}


%replace XXX with the submission number you are given from the ASPLOS submission site.
%\newcommand{\asplossubmissionnumber}{258}

\newcommand{\hoda}[1]{\textcolor{red}{Hoda: \em #1 }}
\newcommand{\ghadeer}[1]{\textcolor{blue}{Ghadeer: \em #1 }}

\usepackage[normalem]{ulem}

\begin{document}


\title{
  Exploiting Parallel Memory Write Requests for Covert Channel Attacks in Integrated CPU-GPU Systems}

%for single author (just remove % characters)
\author{%
  {\rm Ghadeer\ Almusaddar} \\
  galmusa1@binghamton.edu  \\
Binghamton University
\and
{\rm Hoda Naghibijouybari}\\
hnaghibi@binghamton.edu \\
Binghamton University
} % end author



\date{}
\maketitle

\thispagestyle{empty}

\begin{abstract}
In heterogeneous SoCs, accelerators like integrated GPUs (iGPUs) are integrated on the same chip as CPUs, sharing the memory subsystem. In such systems, the massive memory requests from throughput-oriented accelerators significantly interfere with CPU memory requests. In addition to the large performance impact, this interference provides an attacker with a strong leakage vector for covert attacks across the processors, which is hard to achieve across the cores in a multi-core CPU. In this paper, we demonstrate that parallel memory write requests of the iGPU and more specifically, the management policy of the write buffer in the memory controller (MC) can lead to significantly stalling CPU memory read requests in heterogeneous SoCs.

We characterize the slowdown on the shared read and write buffers in the memory controller and exploit it to build a cross-processor covert channel in Intel-based integrated CPU-GPU systems. We develop two attack variants that achieve a bandwidth of 1.65 kbps and 4.41 kbps and error rates of 0.49\% and 4.32\% respectively. 

%We also present a proof-of-concept CPU-GPU side channel attack in which a spy application on the CPU monitors the contention on the memory controller to extract information about a victim application running on the GPU. 
\end{abstract}

%draft #1
% \section{Introduction}

% Figure environment removed

Reinforcement Learning from Human Feedback (RLHF) has recently been used to great effect to align pretrained large language models (LLMs) to human preferences, optimizing for desirable qualities like harmlessness and helpfulness~\citep{bai2022training} and achieving state-of-the-art results across a variety of natural language tasks~\citep{openai2023gpt4}. %RLHF approaches fundamentally rely on collecting pairs of LLM outputs $(o_1, o_2)$ from a shared prompt $p$, with a human indicating which output in each pair is better on a specified attribute.
% A fundamental component of RLHF is a preference model derived from human labels, typically formatted as pairs of LLM outputs $(o_1, o_2)$ generated from a shared prompt $p$.

A standard RLHF procedure fine-tunes an initial unaligned LLM using an RL algorithm such as PPO~\citep{schulman2017proximal}, optimizing the LLM to align with human preferences. %\violet{not sure whether we need to provide this detail in the intro, especially this has nothing to do with our contribution.} % i feel like this context is useful later when e.g. explaining that context distillation is SFT
RLHF is thus critically dependent on a reward model derived from human-labeled preferences, typically \textit{pairwise preferences} on LLM outputs $(o_1, o_2)$ generated from a shared prompt $p$. % and labeled by humans. 

However, collecting human pairwise preference data, especially high-quality data, may be expensive and time consuming at scale. To address this problem, approaches have been proposed to obtain labels without human annotation, such as Reinforcement Learning from AI Feedback (RLAIF) and context distillation. 

\iffalse
raising the question of whether we can generate high-quality data for RLHF without using human labeling. %accurately-labeled preference pairs $(o_1, o_2)$
%, motivating model alignment approaches that aim to generate accurately-labeled preference pairs $(o_1, o_2)$ without human involvement. 
Two major categories of such approaches are . 
\fi

RLAIF approaches (e.g.,~\citet{bai2022constitutional}) simulate human pairwise preferences by scoring $o_1$ and $o_2$ with an LLM (Figure \ref{fig:rlcd_differences} center); the scoring LLM is often the same as the one used to generate the original pairs $(o_1, o_2)$. Of course, the resulting LLM pairwise preferences will be somewhat noisier compared to human labels. However, this problem is exacerbated by using the same prompt $p$ to generate both $o_1$ and $o_2$, causing $o_1$ and $o_2$ to often be of very similar quality and thus hard to differentiate (e.g., Table~\ref{tab:rlaif_bad_example}). Consequently, training signal can be overwhelmed by label noise, yielding lower-quality preference data. 

% While it avoids human labeling efforts, it has weakness. First, LLM preference labels will naturally be somewhat noisier compared to human labels. Furthermore, since the same prompt $p$ is used to generate both $o_1$ and $o_2$, their quality is often very similar and hard to differentiate (See Table~\ref{tab:rlaif_bad_example}). As a result, training signals can be overwhelmed by label noise, yielding lower-quality preference data. 

Meanwhile, context distillation methods (e.g., \citet{sun2023principle}) create more training signal by modifying the initial prompt $p$. 
%to create more significant training signal. 
The modified prompt $p_+$ typically contains additional context encouraging a \textit{directional attribute change} in the output $o_+$ (Figure \ref{fig:rlcd_differences} right). However, context distillation methods only generate a single output $o_+$ per prompt $p_+$, which is then used for supervised fine-tuning, losing the pairwise preferences which help RLHF-style approaches to 
%rather than using a RLHF-style preference model to 
derive signal from the contrast between outputs. 
Multiple works have observed that RL approaches using preference models for pairwise preferences can substantially improve over supervised fine-tuning by itself when aligning LLMs~\citep{ouyang2022training,dubois2023alpacafarm}. 

% conduct alignment by running supervised fine-tuning on model outputs $o_+$ generated from a modified prompt $p_+$. $p_+$ typically contains additional context encouraging desirable attributes (Figure \ref{fig:rlcd_differences} right), such as in \citet{sun2023principle}. However, multiple works have observed that RLHF-style approaches can substantially improve over supervised fine-tuning by itself when aligning LLMs~\citep{ouyang2022training,dubois2023alpacafarm}. 

Therefore, while both RLAIF and context distillation approaches have already been successfully applied in practice to align language models, we posit that it may be even more effective to combine the key advantages of both. That is, we will use RL with \textit{pairwise preferences}, while also using modified prompts to encourage \textit{directional attribute change} in outputs. %In particular, we will adapt the RLAIF data generation process with two different prompts rather than a single $p$, modifying both prompts similarly to context distillation. %\violet{this motivation is a little unexciting. I think we can more specifically discuss the potential benefits of our approach, like the benefits from RL: exploration/data generation; benefits from contrast. I don't think we get too much benefits from context distillation since we switched to the RL framework.} 

Concretely, we propose \oursfull{} (\ours{}). 
\ours{} generates preference data as follows. Rather than producing two i.i.d.\ model outputs $(o_1, o_2)$ from the same prompt $p$ as in RLAIF, \ours{} creates two variations of $p$: a \textit{positive prompt} $p_+$ similar to context distillation which encourages directional change toward a desired attribute, and a \textit{negative prompt} $p_-$ which encourages directional change \textit{against} it (Figure \ref{fig:rlcd_differences} left). We then generate model outputs $(o_+, o_-)$ respectively, and automatically label $o_+$ as preferred---that is, \ours{} automatically ``generates'' pairwise preference labels by construction. %, without further post hoc labeling.\violet{should make it clearer that our approach `generates' labels by construction} 
We then follow the standard RL pipeline of training a preference model followed by PPO. 

Compared to RLAIF-generated preference pairs $(o_1, o_2)$ from the same input prompt $p$, there is typically a clearer difference in the quality of $o_+$ and $o_-$ generated using \ours{}'s directional prompts $p_+$ and $p_-$, which may result in less label noise. %which may result in better training signal for the preference model. 
That is, intuitively, \ours{} exchanges having examples be \textit{closer to the classification boundary} for much more \textit{accurate labels} on average. Compared to standard context distillation methods, on top of leveraging pairwise preferences for RL training, \ours{} can derive signal not only from the positive prompt $p_+$ which improves output quality, but also from the negative prompt $p_-$ which degrades it. %\ours{} is not learning to imitate $o_+$, but to distill the \textit{contrast} between $o_+$ and $o_-$. 
Positive outputs $o_+$ don't need to be perfect; they only need to contrast with $o_-$ on the desired attribute while otherwise following a similar style.

% \todo{discuss our method and why intuitively it may be better.}

We evaluate the practical effectiveness of \ours{} through both human and automatic evaluations on three tasks, aiming to improve the ability of LLaMA-7B~\citep{touvron2023llama} to generate harmless outputs, helpful outputs, and high-quality story outlines. %\ours{} outperforms both RLAIF and context distillation baselines in pairwise comparisons on 
As shown in Sec. \ref{sec:experiments}, \ours{} substantially outperforms both RLAIF and context distillation baselines in pairwise comparisons when simulating preference data with LLaMA-7B, while still performing equal or better when simulating with LLaMA-30B. 
%On all three tasks, \ours{} substantially outperforms both RLAIF and context distillation baselines in pairwise comparisons---by a margin of at least 9\% and often more than 30\%---validating our method's efficacy. 
We will release all code at a later date, although in any case \ours{} is fairly easy to implement by modifying any reference RLAIF codebase. %We release all code at \todo{github link}.
% \section{Background}
\subsection{Parallel Strategy}
\label{subsec:background:parallel-strtegy}
\paragraph{Pipeline parallelism~(PP)} In PP, each worker~(machine or GPU) holds a subset of model layers. Adjacent layers on different workers need to transfer activations in the forward propagation~(FP) step and gradients in the backward propagation~(BP) step. 
\paragraph{Data parallelism~(DP)} In DP, each worker holds a replica of the whole model and partitions training samples. In each iteration, each worker computes gradients and synchronizes them with the other workers using all-reduce collective communication~(CC). All workers will have the same model parameters after the synchronization step.
\paragraph{Tensor parallelism~(TP)} In TP, each worker holds a replica of training samples and partitions within model layers. In each iteration, each worker computes its local outputs in FP and its local gradients in BP. To synchronize outputs and gradients, all workers will perform all-reduce CC in FP and BP steps according to the partition scheme.
\paragraph{Fully sharded data parallelism~(FSDP)} FSDP partitions optimizer states, parameters and gradients of the model into separate workers. During the FP and BP step of each iteration, FSDP performs an all-gather CC to obtain the complete parameters for the relevant layer, respectively. After computing the gradients, FSDP conducts a reduce-scatter CC to distribute the global gradients among the workers.

\subsection{Manual Parallelism}
MP refers to the parallel methods in which human experts design and optimize the parallel strategies. Representative MP methods include Megatron-LM~\citep{narayanan_efficient_2021}, Mesh-TensorFlow~\citep{shazeer_mesh-tensorflow_2018}, and GSPMD~\citep{xu_gspmd_2021}. Megatron-LM manually designs TP and PP strategies for training Transformer-based models and exhibits superior efficiency. Mesh-TensorFlow and GSPMD require human effort to designate and tune the intra-layer parallel strategy. These methods rely on expert design and have little flexibility, challenging their automatic application to other models.

\subsection{Automatic Parallelism}
\paragraph{Inter-layer-only AP or intra-layer-only AP} For inter-layer-only AP, GPipe~\citep{huang_gpipe_2019} and vPipe~\citep{zhao_vpipe_2022} employ a balanced partition algorithm and a dynamic layer partitioning middleware to partition pipelines, respectively. For intra-layer-only AP, OptCNN~\citep{jia_exploring_2018}, TensorOpt~\citep{cai_tensoropt_2022}, and Tofu~\citep{wang_supporting_2019} employ dynamic programming methods to optimize DP and TP strategies together. FlexFlow~\citep{jia_beyond_2019} and Automap~\citep{schaarschmidt_automap_2021} use the Monte Carlo method to find the optimal DP and TP strategy. Colossal-Auto~\citep{liu_colossal-auto_2023} utilizes integer programming techniques to generate intra-layer parallelism and activation checkpointing strategies without optimizing inter-layer parallelism. All these methods optimize only one category of parallel strategies.


\paragraph{Inter- and intra-layer AP} PipeDream~\citep{narayanan_pipedream_2019}, DAPPLE~\citep{fan_dapple_2021}, and PipeTransformer~\citep{he_pipetransformer_2021} use dynamic programming to determine optimal strategies for both DP and PP. DNN-partitioning~\citep{tarnawski_efficient_2020} adopts integer and dynamic programming to explore DP and PP strategies. Piper~\citep{tarnawski_piper_2021} and Alpa~\citep{zheng_alpa_2022} adopt a parallel method considering DP, TP, and PP.
Galvatron~\citep{miao_galvatron_2022} uses dynamic programming to determine DP, TP, and FSDP strategies in a single pipeline stage. As for PP, it partitions stages and determines micro-batch size using naive greedy algorithms. All these methods are hierarchical, which will result in sub-optimal solutions.


% \input{sections/draft_1/3_motivation&challenges.tex}
% \input{sections/draft_1/4_attack_details.tex}
% \checkednote{\subsection{Definition of a measurement}}
In compressed sensing and coding is typically defined as a measurement of an analog flux through some type of coding projection or in our example a mask. As has been shown by multiple works, this analog model of light does not account for the poisson noise inherent in any real measurements and leads to counter intuitive behavior of coding approaches.

If we instead think of our system as measuring photons through different codes, the code behavior makes intuitive sense. A photon measured through a mask with many open pixels carries less information about the scene than one captured through a raster mask because our measurement is ambiguous regarding which pixel in the mask was the origin of the photon. In a raster mask every photon can be uniquely assigned to one pixel. In essence, more photons do not equal more information.

The implication of this well documented problem become ever more important in the age of low noise and photon counting cameras where Poisson noise dominates all measurements. It is wide reaching since the projection process we study here in a specific coding experiment is part of the design of any camera. In other words: Any camera or vision system has to project data from a high dimensional scene space down into a lower dimensional sensor space where it encounters Poisson noise and then uses those noisy measurements to make inferences about the scene. 

\Xpolish{This paper has shown that the challenges for computational imaging under Poisson noise. Algorithms based on the AGN noise assumption are problematic on the modern sensors. However, if the task requires no reconstruction but direct feature extraction, the Selective Sensing using the Optical Neural Networks model can find the optimal coding methods. We have shown the feasibility of the Selective Sensing via simulations and experiments, and it demonstrated a promising classification performance on the MNIST handwritten number dataset. Also, it is robust in the application scenarios where their noise level is hard to estimate. Furthermore, the Selective Sensing provides an motivation for proposed optical ANNs or ANNs with optical layers which allow us to optimize the coding schemes wherever the Poisson noise happens. }
{Our paper highlights the challenges of computational imaging under Poisson noise and its impact on algorithms based on the AGN noise assumption. We find that for compressible measurements, and especially tasks that involve direct feature extraction instead of signal reconstruction, a Selective Sensing approach using task optimized codes provides a viable coding solution. Through simulations and experiments, we demonstrate the feasibility of Selective Sensing and its promising classification performance on the MNIST handwritten number dataset. It is also robust in application scenarios with difficult-to-estimate noise levels. Our ONN method represents a method that can generate these selective measurements. Furthermore, Selective Sensing motivates the development of optical ANNs or ANNs with optical layers to globally optimize imaging systems.}

\Xpolish{On the other hand, there are some limitations in our project. First, we employed the AGN model and reparameterization trick for the model training, which is only an approximation to the noise at the sensor. Second, our test set in the experiments only contains 10 numbers, which may not be representative enough. There is also an inconsistency as the model was trained by simulated data but tested with experimental data. Last but not the least, the Photon Distribution Factor rescales the masks $\B{M}$, but its value changes during the training and it is not evolved in the back-propagation. In general, the optimization of the ONN model still has some defects and we still need to improve the results by using better optimization methods and more experimental data.}
{Despite the promising results of our project, there are some limitations that must be acknowledged. First, we used a Gaussian noise model with reparameterization to train our model, which is only an approximation of the actual quantization noise at the sensor. Second, our test set consisted of only 10 numbers, which may not provide a comprehensive evaluation of the model's performance. Additionally, we noted an inconsistency in that the model was trained using simulated data but tested with experimental data. Lastly, the Photon Distribution Factor rescales the masks $\B{M}$, but its value changes during training and is not evolved during back-propagation. These limitations highlight the need for further improvements in the optimization of the ONN model, such as using more advanced optimization methods and larger sets of experimental data. Our work highlights the importance of the integration of imaging hardware and signal processing. In single photon accurate imaging systems, comprehensibility and sparsity of the data can be exploited to far greater effect during the measurement, as opposed to post processing.}


% %\input{sections/4_reverse_engineerin.tex}
% \input{sections/draft_1/5_eval.tex}
% \input{sections/draft_1/7_side_channel_attack.tex}
% %\input{sections/8_attack_variations.tex}
% \section{Possible Mitigations}
\label{sec:mitigations}
% discuss mitigations which proposed to solve the write buffer issue
% comment on their security, and how it can be improved with consideration to securtity
In this section, we discuss different possible mitigations to thwart the attacks or to make it challenging for attackers to exploit memory write requests.\newline
\noindent{\textbf{Prioritizing Memory Read Requests.}} One of the possible solutions for the latency observed by the CPU process due to iGPU kernel memory write requests is deploying a policy other than \textit{drain\_when\_full}. Deploying a management policy that gives priority to memory read requests whenever they are inserted in the read buffer will thwart our attack. Such an approach may prevent iGPU kernel or CPU processes from issuing write requests if there are read requests in the read buffer and the write and writeback buffers are full. This will deteriorate the performance of CPU processes or iGPU kernel if they are issuing a lot of memory write requests. Also, latency overhead will appear due to \textit{read\_after\_write} and \textit{write\_after\_read} especially if there is frequent alteration between serving memory read and write requests.

\noindent{\textbf{Directing CPU Process's Memory Reads to Ideal DRAM Banks.}} Chatterjee et al.~\cite{staged_reads} proposed Staged Reads to serve memory read requests while memory write requests are being served. This happens only if memory reads are accessing banks not used by memory writes (i.e. ideal banks). Such an approach will not completely thwart our attack, but it will make it challenging for attackers to utilize the effect of \textit{drain\_when\_full} management policy. If Staged Reads \cite{staged_reads} technique is deployed in the system, spy and trojan should pre-agree on accessing same banks for spy read requests to be stalled.

\noindent{\textbf{Channel Partitioning.}} It is possible to thwart our covert channel attack using channel partitioning in multi-channel MCs. This is because each MC channel has its own pair of read and write buffers. One of the MC channels can be dedicated to iGPU during its kernel execution. If the CPU process needs to read from memory associated with this channel, it has to stall until the iGPU kernel finishes execution. Obviously, this approach has a performance impact on CPU processes, especially in the case of kernels with long execution times.





% We believe classes of defenses that have been developed against other microarchitectural covert and side channels on a single component (either CPU or GPU) can potentially apply to cross-component attacks on heterogeneous systems. Rather than securing each component in isolation, these defense designs require some adaptations to consider the system-wide security, functionality, performance, and power goals. These solutions include: 

% \noindent \textbf{Dynamic partitioning of resources:} memory controller request queues (specifically transaction queue) or DRAM banks using dedicated hardware. These partitioning schemes need to support the isolation of memory requests from different processors in heterogeneous systems and also support the bandwidth requirement of each processor dynamically to minimize the performance overhead. 

% \noindent \textbf{Memory scheduling and traffic control in memory controllers:} eliminating the contention among processes running on different processors, such that memory requests from each processor are grouped into the same queue (or sub-queues) and possibly scheduled and routed to the same subset of memory banks or channels. 






%In this section, we suggest different possible mitigations to address contention on memory controller resources due to iGPU memory traffic.

%\textbf{Spatial Partitioning.} One of the possible approaches to avoid contention on MC resources is to partition them spatially. The high level of contention we observe in this attack is due to sharing of resources such as queues within the memory controller. If we partition these resources between CPU cores and iGPU, spy process will never observe same level of contention. Partitioning should also be considered for other resources such as memory channels. Current iGPU is usually deployed in processor chips with one memory controller with two channels. It is possible to assign one of the channels for iGPU traffic and the other channel for CPU cores traffic. The drawback of this approach is that iGPU is not always executing kernels and resources assigned to iGPU will become under utilized.

%\hoda{temporal partitioning has huge performance impact, not a good solution in any case}
%\textbf{Temporal Partitioning.} Another possible approach to avoid contention caused by iGPU memory traffic is to use temporal partitioning. In this approach, a turn is designated to each source whether its iGPU or CPU cores for its memory requests to be served. With temporal partitioning iGPU and CPU cores memory requests will not use the same resources at the same time. This approach still has to consider in flight memory requests even if resource turn finishes issuing its requests \cite{cornell_defense_hpca}.

%time and space multiplexting
%resource paritioning
%
% \input{sections/draft_1/9_related_work.tex}
% \section{Conclusions}
\label{sec:conclusions}
Organ segmentation is a fundamental task in the medical field. The volumetric data that characterize CT and MRI acquisitions make, however, the segmentation task computationally expensive. On the one hand, 2D CNNs provide a low latency solution unable to capture inter-slice information, on the other hand, 3D CNNs extract three-dimensional features at the price of high computation costs and risk of overfitting. Moreover, popular 2.5D multi-view fusion methods train three separate networks where the features of the orthogonal planes are learned independently, despite being part of the same volume. In SSH-UNet this is addressed by imposing weight sharing between convolutions so that only one network needs to be trained and multi-view features are collaboratively learned.
In this work, we introduced a novel approach for the segmentation of volumetric medical data. Inspired by works in the field of Video Action Recognition we interpret the slices of a volume as the frame of a video. Given a 2D backbone, to re-integrate the information between features belonging to adjacent slices we leverage the power of a shifting mechanism inspired by the TSM module. Spatio-temporal modeling, declined on pseudo-3D operators, despite being well-known in the Video Understanding field was never used before in the medical image analysis to extract and mingle multi-slice features. Our network, by using a 2D convolution with weight sharing mechanism and slice shift, can extract 3D features keeping low computational complexity.
In comparison to other popular state-of-the-art methods, SSH-UNet achieves an accuracy of \textbf{87.28\%} on the AMOS validation providing the smallest model in terms of parameters (6.48M) compared to the best network which has $+1.6\%$ improve in accuracy but $\times5$ increase in parameters.





%draft #2
% Figure environment removed

\section{Introduction}
Automatic 3D reconstruction of clothed humans using image inputs has gained increasing significance due to its potential applications in a wide array of AR/VR scenarios. High-fidelity reconstructions typically depend on sophisticated capture systems, which are developed with dense camera arrays~\cite{collet2015high,joo2015panoptic,joo2018total}, programmable light-stages~\cite{Vlasic2009, guo2019relightables}, and depth sensors~\cite{newcombe2011kinectfusion,DoubleFusion,BodyFusion,dou2016fusion4d,newcombe2015dynamicfusion}. However, stringent capture environments equipped with complex hardware pose significant challenges for consumer-level applications.


In this context, considerable research effort has been dedicated to developing methods that allow for more flexible capture configurations, such as utilizing a few RGB inputs. Among these works, learning implicit functions \cite{iccv2020PIFu, saito2020pifuhd, hong2021stereopifu} has proven effective in achieving highly detailed reconstructions by integrating the advancements of deep neural networks. These methods employ large multi-layer perceptrons (MLPs) to predict the occupancy probability or truncated signed distance function (TSDF) value of every queried 3D point based on its associated local feature, which is extracted from images. They can recover a continuous surface at arbitrary resolutions without topology restrictions.


However, in typical MLP-based implicit networks, the occupancy or TSDF value at each location is solved independently with planar image features, rendering them less capable of addressing challenging cases such as occlusions. Consequently, these methods suffer from generalization and robustness issues, particularly when tackling strong occlusions caused by large motion or multiple interacting humans. 
Some follow-up studies  \cite{zheng2021deepmulticap,zheng2021pamir,huang2020arch} utilize an extra geometric model, SMPL~\cite{Loper2015}, to improve robustness by introducing strong shape priors. 
Their success typically relies on the assumption of geometrical similarity \cite{huang2020arch} between the shape prior and target reconstruction, making them intractable for handling complex cases with loose clothes and sensitive to errors in SMPL model fitting.



%\ping{this paragraph sounds like `TSDF is better than MLP/SMPL, and we use TSDF to solve the problem'. But in Sec 3, we are telling a different story, saying `MLP needs a 3D convolutional encoder'. We need to make these two sections consistent.}\sicong{I think in this paragraph we claim that the TSDF}


%We opt for Trucated Signed Distance Funtion (TSDF) volumetric representations as they are naturally suitable for convolution operations, which have shown remarkable performance for learning hierarchical features on 2D visual perception tasks \cite{SunXLW19}. 
%Meanwhile, TSDF also describes the gradual geometry change around shape surface, which is not reflected by occupancy volume. 

We instead revisit the 3D volumetric representation and resort to 3D convolutional neural networks (CNNs) for feature learning, due to their impressive performance in feature learning and the ability to incorporate spatial context. However, volumetric methods and 3D convolution involve discretization, which might raise concerns regarding whether a discretized volume can preserve subtle geometric details as continuous representations learned in implicit functions. We investigate the relationship between volume resolution and quantization error on synthetic data by converting target mesh objects to TSDF volumes, as shown in Figure~\ref{fig:quantization_error}. We observe that the quantization errors are significantly reduced by increasing volume resolution and become nearly negligible when reaching a relatively high resolution (e.g., 512 or higher). In other words, achieving fine-detailed reconstruction is not supposed to be restricted by the use of volume representations as long as a proper volume resolution is utilized. Therefore, we present a method with high-resolution feature volumes, e.g., 256 and 512, while traditional volumetric methods \cite{varol18_bodynet,gilbert2018volumetric} are often limited to much lower resolutions, such as 32 or 128.



On the other hand, an increase in volume resolution may lead to a cubic growth of memory overhead \cite{8100085}. Reducing memory costs while guaranteeing the granularity of volumetric representations is necessary for pursuing high-quality reconstruction. Thus, we adopt a coarse-to-fine approach and cull away irrelevant voxels to build a sparse high-resolution feature volume. At the coarse level, the network computes an initial TSDF by applying a U-Net with sparse 3D CNN \cite{3DSemanticSegmentationWithSubmanifoldSparseConvNet} on the sparse feature volume, which is carved by a visual hull. Through our experiments, it turns out that more than 95\% of the volume grids are discarded by the visual hull culling, making the sparse 3D CNN efficient. At the fine level, the network focuses on a narrow band near the zero-level set of the initial TSDF and discretizes the narrow band with smaller voxels. By employing this narrow-band culling, we further shrink the sampling space, resulting in a relatively small range of grid numbers (usually 300K--500K in our experiments) even with a high volume resolution of 512. The remaining voxels in the narrow band are associated with features that fuse high-frequency information from the computed normal maps upon the low-frequency shape from the coarse level to compute the TSDF at high resolution. The final mesh is then extracted from the TSDF using the Marching-Cube algorithm ~\cite{Lorensen87marchingcubes}.
% Different from the u-net sturcture to preserve global topology context, we then apply a shallow 3dcnn to compute the final TSDF $D_{final}$ which contain more local geometry detail.




% \ping{this paragraph can be expanded. It is an important contribution and often ignored by other works. stress on the novel idea of regressing blending weights instead of colors}

In addition to geometry, high-quality mesh texture is also a crucial factor contributing to visual appearance. Directly computing a color field in 3D space, as in \cite{iccv2020PIFu}, struggles to capture high-frequency texture details, while the neural radiance field (NeRF) \cite{yu2020pixelnerf} or the DoubleField~\cite{shao2022doublefield} require expensive per-instance optimization and are often unstable for sparse input images. In contrast, we adopt an image-based rendering approach to compute a texture atlas map, which is efficient and widely supported in existing computer graphics tools. 
Specifically, we compute a blending weight at each 3D point on the mesh surface to determine its color as a weighted average of the colors at its image projections. The blending weights can be computed at a relatively coarse resolution, e.g., 512 volume resolution in our case, and leave texture details to the high-resolution images, such as 1K or 2K. Unlike previous methods that generate blurry texturing results under sparse input, our method generalizes well on both synthetic and real data with just a few input views. 
Figure~\ref{fig:teaser} shows two examples reconstructed by our method. Despite the challenging garment, pose, and occlusion, our method recovers faithful shape, normal, and texture on the right.

%with a wide variety of poses and clothing styles, and it is also adaptive to handle input image with arbitrary resolutions.
%\sicong{For this concern we claim that when the resolution of dicretized volume meets certain threshold (which is 256 in our experiment), the quantization error can be neglected.} 



In summary, the main contributions of this paper are as follows:
\begin{itemize}
\vspace{-0.1in}
  \item 
  We revisit the 3D volumetric representation and demonstrate that it can support clothed human reconstruction with equal or even better performance compared to implicit representation. 
  \item 
  We develop a memory and computation-efficient method for high-resolution volumetric reconstruction using sophisticated sparse 3D CNN, coarse-to-fine estimation, and voxel culling by visual hull and narrow bands. 
  \item 
  We introduce a novel method to compute a texture atlas map, which captures rich appearance details from high-resolution input images.
  \item 
  We achieve impressive results on standard benchmark datasets Twindom and MultiHuman, significantly reducing the point-2-surface (P2S) precision to approximately 0.2cm from just six input views, with more than $50\%$ error reduction compared to the state-of-the-art methods, including DoubleField~\cite{shao2022doublefield} and PIFuHD~\cite{saito2020pifuhd}.
\end{itemize}
\vspacebeforesection
\section{Background}
\label{sec:background}

In this section, we provide the necessary background information to ensure a comprehensive understanding of the attack described in this paper. We start with a description of the Distributed Hash Table (DHT) used by IPFS, followed by its content resolution mechanisms. We also detail techniques for network size estimation, necessary for our attack detection and mitigation mechanisms.

\vspacebeforesection
\subsection{IPFS DHT}
\label{sec:kad_dht}

We review the features of the Kademlia DHT~\cite{maymounkov2002kademlia} and its \texttt{libp2p} implementation~\cite{libp2p_github} that are the most relevant to our attack.
To participate in the DHT, each peer generates a public/private key pair and derives an identity $\peerid \in \{0,1\}^{256}$ as the hash of its public key.
Ideally, each peer generates a random key pair and, therefore, peer IDs are distributed uniformly and independently over the space $\{0,1\}^{256}$.
While honest nodes follow this rule, malicious nodes may generate and choose from an arbitrary number of key pairs.
Each peer maintains a routing table consisting of $m=256$ buckets.
The $i$-th bucket contains the addresses of up to $k=20$ peers whose peer IDs share a common prefix of exactly $i$ bits with the peer's own peer ID. 

%
A new participant node joins the IPFS network by contacting one of the hardcoded bootstrap nodes. This bootstrap node provides the new node with some initial peers allowing it to join the DHT. The new node uses this information to perform a walk through the DHT towards its own peer ID.
The walk allows to: \textit{(i)}~make sure that there is no other node in the network with the same ID; \textit{(ii)}~discover new peers and fill the newcomer's DHT routing table. At the same time, the newcomer establishes \bitswap~\cite{de2021accelerating} connections to a subset of encountered peers (usually around 300 of them). The core role of the \bitswap protocol is to enable bilateral content transfer and to play the role of a cache for recently-accessed content.

The main DHT operation $\Call{GetClosestPeers}{\key}$ returns the $k=20$ closest peers to $\key$. 
%
In Kademlia, the distance between two keys $x$ and $y$ in the key space is given by $x \oplus y \in \{0,...,2^{256}-1\}$, where $\oplus$ denotes the bitwise XOR operation on the keys; the resulting binary string is interpreted as an integer.
%
When a client wants to find the peers with IDs closest to $\key$, it sends a request to the $\alpha=3$ peers in its routing table whose peer IDs are closest to $\key$. Each of these peers returns the $k$ closest peers to $\key$ in its own routing table and the addresses of these peers. 
%
The client again sends a request to the $\alpha$ peers closest to $\key$, among peers in its routing table and those whose addresses it just received. This process repeats until the client does not find any more peers closer to $\key$.
Due to network churn and imperfect routing tables, we observed in our experiments that successive calls to $\Call{GetClosestPeers}{\key}$ do not always return the same set of $k=20$ peers (we provide more details in \Cref{sec:evaluation}, \Cref{fig:20closest}). This is an important limitation affecting our attack.

\vspacebeforesection
\subsection{Content Resolution in IPFS}
\label{sec:ipfs}

IPFS is a content-centric network.
It allows its participant to request files without specifying their location. 
%
Content is indexed by content IDs $\cid \in \{0,1\}^{256}$ that are derived from a hash of that content.
Both peer IDs and CIDs are used as keys in the DHT.
Each node can play the role of a \provider, \downloader, or \resolver. 
The process of content advertisement and resolution is illustrated in \Cref{fig:add_get_provider}.

%
When a \provider wishes to publish content with a given $\cid$ on IPFS, it creates a \emph{provider record} that contains $cid$ and the \provider's address.
During a $\Call{Provide}{\cid}$ operation, the \provider first uses $\Call{GetClosestPeers}{\cid}$ to locate the $k=20$ peers with their peer IDs closest to $\cid$, 
%
and then sends them a $\mathsf{PutProvider}$ message including the provider record (\Cref{fig:add_get_provider}(a)).
We call the peers that hold provider records for $\cid$ the \emph{resolvers} for $\cid$.

Each CID can have several \providers. In fact, by default, each IPFS client becomes a provider for each piece of content it downloads for a fixed amount of time (12h, 24h, or 48h depending on the client version or custom configuration). As a result, the system provides an auto-scaling feature with supply automatically rising with demand.

%
When a \downloader wishes to fetch a piece of content, it first sends a request to all its \bitswap peers. If none of them has the content, the \downloader uses the DHT-based resolution system. We stress that the \bitswap protocol plays the supporting role of a cache in the dissemination of popular files. However, the mechanism does not provide reliable content resolution, in particular for new or less popular content. %

When \bitswap unstructured search fails, the \downloader resolves $\cid$ using $\Call{FindProviders}{\cid}$. This operation uses a DHT walk identical to that of $\Call{GetClosestPeers}{\cid}$ to find $k$ \resolvers but also queries encountered nodes for a provider record for $\cid$ (\Cref{fig:add_get_provider}(b)). The process terminates when either 20 \providers have been found, or all \resolvers have been asked. Querying all encountered nodes (\ie, not only the designated \resolvers) is useful because some of the encountered nodes may have a provider record in their cache.
%

Upon receiving a provider record, the client connects to the address specified in the provider record to retrieve the actual content (\Cref{fig:add_get_provider}(c)).
Provider records are not authenticated, and therefore malicious \providers may respond with incorrect provider records (or may not respond at all). However, the integrity of the content is preserved because the hash of the retrieved content can be verified against its $\cid$.
%


%

\input{img/add_get_provider.tex}

\vspacebeforesection
\subsection{Network Size Estimator}
\label{sec:netsize}

The number of nodes in a decentralized system is generally unknown due to the avoidance of centralized membership management.
This number is nonetheless useful for optimizations, deciding on individual node configurations, or security mechanisms.
Various methods were proposed for the decentralized estimation of unstructured and structured networks~\cite{eli-sohl-dht-size-estimation,kostoulas2005decentralized, manku2003symphony}.
We use in this work a mechanism developed initially by Protocol Labs as part of a mechanism for decreasing the latency of publishing content in IPFS~\cite{network-size-estimation-notion,network-size-estimation-github-pr}.

%
%
%
%
%
%
%
%
%
%

Each node in the DHT refreshes its routing table periodically (every $10$ minutes in \texttt{libp2p}). 
For this, the node samples $m$ random keys (one for each bucket of its routing table)
%
and queries the DHT to obtain the $k=20$ closest peer IDs to each key.
Using these, the node then computes the average distance between each one of these keys $\key_j$ for $j=1,\dots,m$ and their $i$-th closest peer ID for $i=1,...,k$ (with $m=256$ and $k=20$).
\begin{equation}
    \label{equ:avg-dist}
    \overline{D}_i = \frac{1}{m} \sum_{j=1}^m \operatorname{dist}(\key_j, \peerid_{j}^{(i)})
\end{equation}
where $\peerid_{j}^{(i)}$ is the $i$-th closest peer ID to $\key_j$.
With $N$ peers in the DHT and peer IDs uniformly distributed in the hash space, the expected distance between a $\key$ and its $i$-th closest peer ID is $\frac{2^{256}i}{N+1}$. The node then runs a least square regression to compute the value of $N$ for which the expected distances best fit the empirical average distances, \ie,
\begin{equation}
    \label{equ:netsize-least-squares}
    \hat{N} = \arg\min_{N} \sum_{i=1}^k \left(\overline{D}_i - \frac{2^{256}i}{N+1}\right)^2.
\end{equation}
The resulting estimate $\hat{N}$ can be computed in closed form.
%

When a node starts running, it must perform DHT queries for a few random keys to initialize its network size estimate. 
Since a larger number of queries will result in higher accuracy, making more queries than what is needed to initialize one's routing table is recommended.
Thereafter, keeping the estimate up-to-date does not require any excess DHT queries beyond what is already used for refreshing the routing table as this is done frequently (every 10 minutes).

While the network size estimate has a stochastic variance resulting from the probability distribution of the honest peer IDs, it is hard for an attacker to bias the estimate significantly. Since the estimator uses the density of peer IDs around keys chosen uniformly at random, the adversary would require numerous Sybil nodes (on the order of the whole network size) to significantly affect the peer ID density around those keys.

\section{Threat Model}
\label{sec:threat_model}

%TODO: include security status of the targeted system
% what is enabled? what is disabled? what defense mechanisims it assumes?

Our threat model is for a cross-processor covert channel attack in Intel-based SoC (from iGPU to CPU). In a covert channel, two processes (a trojan as a sender and a spy as a receiver) communicate covertly using a shared resource. The spy runs on a CPU core and the trojan process (from another CPU core) launches the kernel through user-level OpenCL API calls to run on iGPU. Both spy and trojan need to execute on the same machine and share the MC. We assume these processes are separate user space processes and do not require any privileged support. Also, there is no shared data between spy and trojan processes.

MC serves read and write memory requests from read and write buffers. Sharing of other architectural and micro-architectural resources other than the MC is not required in our attack. Also, it is not required for CPU cores and iGPU to share the same cache as long as the same MC is serving memory requests.
%Although sharing such resources (such as LLC in Intel-based SoCs) would introduce another level of contention, we show that the contention we observe in our attack is mainly introduced on the shared memory controller resources.
Our attack targets SoCs with a single MC with dual channels which is the common case with most SoCs in desktop, workstation, and mobile processor chips \cite{gen7.5, gen9, gen11, intel_uhd, arm_npu1, snapdragon_660, qualcomm_660}. All memory requests from the CPU and iGPU are routed through this shared MC from the cache write-back buffer. We exploit the write buffer within the shared MC to develop our covert channel attack variants.

%\hoda{Ghadeer, how common is this to have just one memory controller in Intel SoCs? Is there one memory channel in your system? We should mention details of DRAM (number of channels,  size, ... here. Please write all details here.}
\begin{comment}
We also present a proof-of-concept side channel attack, in which a spy process running on a CPU core, monitors the contention on the write buffer to leak sensitive information about a victim process that is running on iGPU.
\end{comment}
%All of our experiments are implemented on a Comet Lake i7-10700K desktop processor, which features an integrated Intel’s Gen9 UHD 630 Graphics \cite{proc_spec}. We use OpenCL version 2.1 running Ubuntu 20.04.1 LTS with linux kernel 5.13.
%\hoda{complete the blank spaces.}

Our attacks are developed and tested on an unmodified system. Current Intel iGPUs are not capable of running multiple computation kernels from separate contexts concurrently and therefore no noise is expected on the GPU side. Figure \ref{fig:threat-model} demonstrates our covert channel threat model.


%The attacker exploited a vulnerability in a targeted process and now can execute code as part of this process execution. The attacker aim to leak information from attacked process (trojan process) to spy process which is also running on the same processor chip. 
% Figure environment removed
\section{Motivation}
\label{sec:motivation}

IGNORE THIS FILE, WILL DO IN INTRO

\section{Reverse Engineering: Source of CPU Process's Read Latency Overhead}
\label{sec:cont_src}

CPU cores and Intel iGPU share microarchitectural components other than MC, such as Ring interconnect, LLC, and DRAM resources. Consequently, it is possible for such resources to be the reason for the slowdown observed by the process running on the CPU core during iGPU kernel memory writes. In this section, we investigate the source of slowdown due to iGPU kernel memory writes and confirm that it is due to the management policy of the write buffer in the shared MC. 

In all of our experiments in this section, the CPU process is reading from a buffer of size 128KB except for the experiment for LLC hits (in Figure~\ref{fig:llc_cont_hits}) where the buffer size is 512KB (double the size of L2 cache). We choose a 128KB buffer size to simplify the process of reverse engineering by reducing the set of read addresses from the CPU process (2048 accesses). While a 512KB buffer is used to ensure that most of the accesses from the CPU process are LLC hits.
\subsection{Ring Interconnect and Last Level Cache (LLC)}%
In our targeted system, LLC and ring interconnect are shared between CPU and iGPU. These two components could be the cause of the slowdown the CPU process is observing. We show that neither LLC nor ring interconnect is the source of contention observed in Figure~\ref{fig:llc_cont}.

We explored the slowdown due to iGPU kernel memory write traffic when CPU process accesses are LLC hits vs. LLC misses. Figure~\ref{fig:llc_cont_hits} shows the normalized latency of CPU process during iGPU kernel writes when buffer accesses are LLC hits. Note that in Figure~\ref{fig:llc_cont_hits} the baseline latency is CPU process average latency when iGPU kernel is not issuing any memory write requests. %The baseline average latency in this case is equal to LLC hit latency. 
Figure~\ref{fig:llc_cont_misses} shows the normalized latency of the CPU process during iGPU kernel writes when buffer accesses are LLC misses. The baseline for Figure~\ref{fig:llc_cont_misses} is CPU process average latency when the iGPU kernel is not issuing memory write requests. %The baseline average latency in this case is equal to LLC miss latency.

It can be noted from Figure~\ref{fig:llc_cont}, that the CPU process is suffering a higher level of slowdown in case of LLC misses ($\times5$ baseline LLC miss latency) compared to LLC hits ($\times2$ baseline LLC hit latency). This proves that neither LLC nor ring-interconnect is the source of slowdown. The higher latency starts to appear when the number of iGPU kernel memory requests is $2^{19}$ write requests, and iGPU kernel buffer size is 32MB.

\begin{tcolorbox}[arc=5mm, outer arc=1mm, width=\linewidth,halign=flush center, left=1mm, right=1mm]

\textit{\textbf{LLC and Ring Interconnect are not the cause of CPU process's slowdown.} During iGPU kernel execution, the normalized latency of the co-running CPU process when its accesses are LLC hits is much lower than the normalized latency when its accesses are LLC misses. 
}
\end{tcolorbox}

%\hoda{The second sentence is somehow confusing. They may think that it is the access time of the cache (hit vs. miss), especially with the numbers. The above paragraph clearly shows that access times are normalized to miss latency or hit latency so the slowdown is compared to these baselines. But here it is not clear, and if the reviewers just read these frames it is misleading. I would say something like this. If it makes sense to you, let's edit it: "When the accesses from the co-running CPU are LLC hits, the slowdown (access time/hit time) is much lower than the slowdown in case of LLC misses (access time/miss time)"} 
%\ghadeer{the case is NOT (access time / hit time), the case is (hit time (during iGPU memory write requests) / hit time (when iGPU is not issuing memory write requests. Also it is NOT (access time/miss time) it is (miss time during iGPU kernel memory requests / miss time when there is no iGPU kernel memory requests. the purpose of the red line in the figure is to indicate that during iGPU kernel execution almost 100\% of CPU buffer access are misses or hits. This is to answer the question, how did you know that the CPU process during iGPU kernel execution is hit or miss in LLC and by what percentage? when i say normalized to LLC hit latency or normalized to LLC miss latency. I'm indicating the baseline which is the CPU process latency when there is no iGPU memory write requests which is equal to LLC hit latency in the first case and LLC miss latency in the second case.  I updated the y axis and caption in addition to information in text box let me know if it still not clear}
\subsubsection{Writeback Buffer of LLC}
Writeback buffer of LLC stores dirty cache lines to be up- dated in main memory. Due to the high rate of memory write requests by the iGPU kernel, writeback buffer of LLC will rapidly get filled. We showed in Figure~\ref{fig:llc_cont_hits} that the latency of the co-running CPU process when buffer accesses are LLC hits is $\times2$ baseline LLC hit latency. This is about $\times12$-$\times15$ smaller than the latency when CPU process buffer accesses are LLC misses. This indicates that LLC does not get blocked from serving CPU process read requests even though the number of write memory requests is the same in Figure~\ref{fig:llc_cont_hits} and Figure~\ref{fig:llc_cont_misses}. Also, the LLC writeback buffer is not on the critical path of process's read requests whether they are hit or miss in LLC as long as they are not dependent on write requests.


\begin{tcolorbox}[arc=5mm, outer arc=1mm, width=\linewidth, halign=flush center, left=1mm, right=1mm]
\textit{\textbf{Writeback buffer of LLC is not the source of CPU process's slowdown}. This is because the Writeback buffer is not on the critical path of memory read requests which are not dependent on writes.}
\end{tcolorbox}



\begin{table}[t]
\centering
    \begin{tabular}{||m {0.33\columnwidth}|m{0.56\columnwidth}||}
        \hline
        Memory Controller & Dual channel \\
        \hline
        DRAM & DDR4 MR[ABC]4U320GJJM16G @ 2600MT/s\\        
        \hline
        Memory Capacity & 32GB (2-16GB DIMMs) \\
        \hline
        Ranks & Single Rank \\
        \hline
        Number of Bank groups and Banks & 4 bank groups, 4 banks/bank group \\
        \hline
        Channel Addressing & $b_8\oplus b_9\oplus b_{12}\oplus b_{13}\oplus b_{15} \oplus b_{16}$\\
        \hline
        Bank Group Addressing & BG0: $b_7\oplus b_{14}$ \newline BG1: $b_{15}\oplus b_{18}$\\
        \hline
        Bank Addressing & BA0: $b_{16}\oplus b_{19}$ \newline BA1: $b_{17}\oplus b_{20}$ \\
        \hline
    \end{tabular}
\caption{Targeted DRAM details and reversed engineered channel, bank group and bank addressing.}
\label{tab:dram_re}
\vspace{-2mm}
\end{table}


\subsection{Memory Controller and DRAM Resources}


% Figure environment removed%
%\vspace{-6mm}
%% Figure environment removed



CPU cores and iGPU also share MC and main memory (DRAM) resources. We investigate whether shared resources such as channels, bank groups, or banks contribute to the high latency level observed by the co-running CPU process during iGPU kernel memory write requests.

We reversed engineered bits in the physical address which indicate the rank, channel, bank group, and bank to access based on DRAMA paper ~\cite{drama_usenix}. Table~\ref{tab:dram_re} shows reverse engineering results of the channel, bank group, and bank addressing used to determine which DRAM resource is used based on the physical address. The table also shows our targeted DRAM details.

To study the impact of the channel, bank group, or bank contention on the observed high latency, we launched two experiments (A) and (B). In both experiments, the CPU process and iGPU kernel are doing the same number of memory reads and writes; memory reads in the case of the CPU process and memory writes in the case of the iGPU kernel. The difference is that we allocated a larger buffer in experiment (B) which is accessed at a stride of one cache line.% We will show later that dirty cache line eviction plays a role in increasing CPU process latency.

Figure~\ref{fig:dram_cont_a} and Figure~\ref{fig:dram_cont_b} depict CPU process read latency distribution and channel, bank group, and bank contention cases distribution. A contention case happens when CPU process read address uses the same channel, same bank group or same bank as iGPU kernel memory write addresses based on physical addresses. We infer if a channel, bank group, or bank contention case had happened based on reverse-engineered addressing using physical address bits in Table \ref{tab:dram_re}.

In these experiments, the total number of read requests by the CPU process is 2048 requests and the total number of write requests for the iGPU kernel is $2^{18}$ memory requests. In Figures~\ref{fig:channel_a}, ~\ref{fig:channel_b}, ~\ref{fig:bankgroup_a}, ~\ref{fig:bankgroup_b}, ~\ref{fig:bank_a} and ~\ref{fig:bank_b}, the x-axis indicates the number of contention cases observed by each CPU process read access due to iGPU kernel memory writes. The y-axis indicates the count of these contention cases. Note that the total of contention cases count (y-axis) is 2048 which is equal to CPU process memory accesses.


%\hoda{Before going to explain figures, first can you please elaborate on channel/bankgroup/bank contention cases (X-axis) more in the text? Explain what do you mean by .. contention cases? I expect this part to be very hard to follow for reviewers.}
%\ghadeer{addressed}.



Figure~\ref{fig:channel_a} and Figure~\ref{fig:channel_b} show the distribution of channel contention cases observed by the CPU process due to iGPU kernel memory writes. All of the CPU process memory reads observed the same number of channel contention cases equal to half of iGPU kernel write requests in both experiments ($2^{17}$). This is because based on our reverse engineering results, we found that about half of the iGPU kernel buffer was allocated on the first channel and the other half to the second channel. The same scenario is for the CPU process buffer. We conclude that channel contention is not the reason behind the high latency observed by the CPU process since channel contention level is the same in experiments (A) and (B).

Figure~\ref{fig:bankgroup_a} and Figure~\ref{fig:bankgroup_b} depict the distribution of bank group contention cases observed by the CPU process due to iGPU kernel memory write requests. In Figure~\ref{fig:bankgroup_a}, about 1000 CPU process memory accesses observed 32200 contention cases, while the rest observed 33300 contention cases. The case for Figure~\ref{fig:bankgroup_b} is close; about 950 memory accesses suffered 32550 contention cases the rest suffered 33000 contention cases. Total bank group contention cases in both experiments are close and do not explain the large difference in latency distribution between experiments (A) and (B).

Furthermore, we investigate the difference in bank contention between experiments (A) and (B) as we show in Figure~\ref{fig:bank_a} and Figure~\ref{fig:bank_b}. In Figure~\ref{fig:bank_a}, about 50\% of CPU process accesses resulted in 8200 bank contention cases or lower. Most CPU process accesses suffered contention cases between 7800 and 8910 cases. While for the second experiment in Figure~\ref{fig:bank_b}, 50\% of CPU process accesses resulted in about 8000 bank contention cases or lower. Also, most CPU process accesses in this experiment suffered contention cases between 8000 and 8370 cases. The range of contention cases in experiment (B) is smaller than in experiment (A). From these results, we can conclude that bank contention is not the reason behind the huge latency difference between experiments (A) and (B) shown in Figure~\ref{fig:latency_a} and Figure~\ref{fig:latency_b}. 

\begin{tcolorbox}[arc=5mm, outer arc=1mm, width=\linewidth, halign=flush center, left=1mm, right=1mm]
\textit{Channel, Bank group, and Bank contention are not the cause of the high latency of co-running CPU process during iGPU kernel memory writes.}
\end{tcolorbox}


\subsection{Write and Read Buffers in Memory Controller}
Other resources which could increase the latency of memory reads performed by the CPU process are read and write buffers in the MC. The high latency observed by the co-running CPU process happens only in the case of iGPU kernel memory writes not memory reads. Also, we noticed that the slowdown experienced by the CPU process happens once during iGPU kernel kernel buffer accesses when sending single bit '1' and not periodically as we show in Figure~\ref{fig:latency_per_access}. The CPU process continue to suffer high memory access latency during iGPU kernel execution.%

Considering these circumstances, it is possible for the management policy of the write buffer in the MC to be the cause of the CPU process's higher read latency. A common management policy used with write buffer in MC is \textit{drain\_when\_full}. In this management policy, when the write buffer gets full, memory read requests are stalled until the write buffer is drained. The purpose of such a management policy is to avoid DRAM latency due to \textit{write after read} and \textit{read after write}. In fact, there is a lot of research addressing this performance issue which occurs when stalling memory read requests to serve write requests in multi-core CPU environment~\cite{staged_reads, lee_dram}.%

Due to continuous parallel write requests from iGPU, the probability of getting the write buffer full is high. Also, when the write buffer is being drained, there will be a number of standing write requests waiting to be added to the write buffer to be served. This explains the high latency observed by the CPU process when reading from the main memory.
%\vspace{-2.5mm}
\begin{tcolorbox}[arc=5mm, outer arc=1mm, width=\linewidth, halign=flush center, left=1mm, right=1mm]
\textit{Due to the write buffer management policy (\textit{drain\_when\_full}), memory read requests in the read buffer need to be stalled to serve memory write requests.}
\end{tcolorbox}


\section{Covert Channel Attack Design}
\label{sec:attack_design}

We concluded that the management policy of the write buffer is the cause of the high latency observed by the co-running CPU process. In this section, we explore different ways to exploit this phenomenon to leak secret information using covert channel attacks. 

\begin{comment}
One of the challenges in exploiting write memory requests at the write buffer level is that it is not known when dirty cache lines will be inserted in the write buffer. We explore two ways to force evictions of these cache lines by exploiting iGPU parallelism. 
\end{comment}


% Figure environment removed

The attack first starts with handshaking between trojan (CPU-side) and spy. This can be done using traditional techniques (i.e. flush+reload, prime+probe) or by using the slowdown which can be caused using iGPU kernel memory write requests. \circled{1} Trojan CPU process indicates the start of the communication to leak secret information and launches a kernel on the iGPU to issue parallel memory write requests.%\hoda{please say that it (cpu process or iGPU kernel??) sends a synchronization bit sequence for handshaking, then the spy receives it in step 2.}
\circled{2} Once the spy CPU process successfully detects the bit sequence as part of the handshaking process from trojan (CPU-side), it starts a continuous stream of memory read requests. We found that allocating a buffer with a size larger than the LLC size and accessing it at a stride of 64 cache lines is sufficient to ensure memory accesses during iGPU kernel execution without the need to use clflush instruction. This ensures continuous monitoring of the slowdown caused by the iGPU kernel. \circled{3} Cache lines in LLC will be updated based on iGPU kernel writes. %We noticed that \textit{write-no-allocate} is used for LLC since cache lines written with the same data are still added to the write buffer.

\circled{4} When dirty cache lines are evicted from LLC, they are pushed into the write back buffer of LLC. \circled {5} Eventually, dirty cache lines in the write back buffer will be inserted to write buffer in the MC to be updated in main memory. We can control filling the write back buffer and write buffer in the MC by exploiting iGPU parallelism and writing to multiple cache lines in a short time. \circled{6} When the write buffer gets full, it has to be drained due to \textit{drain\_when\_full} management policy. During draining the write buffer, any spy read requests in the read buffer will be stalled until the write buffer is drained.

To send bit "1", trojan has to do a lot of writes to different cache lines to fill and drain write buffer frequently resulting in stalling spy read requests. Writes have to be done to different cache lines to avoid the coalescing effect in iGPU. To send bit "0", trojan exploits coalescing effect, by doing the same number of writes, but to the same cache line to avoid filling the write buffer. This way, the trojan can leak secret information to the spy.

%We mentioned that one of the challenges in exploiting memory write requests is that it is not known when dirty cache lines will be inserted in the write buffer. Consequently,
We propose two different approaches to establish a covert channel which exploit the effect of write buffer management policy. The first is MC channel oblivious attack while the second attack targets one MC channel.


\subsection{Attack Variant 1: MC Channel Oblivious Attack}
\label{subsec:var1_design}

% % Figure environment removed

% Figure environment removed

% Figure environment removed

% Figure environment removed


The first attack variant we are proposing does not require spy and trojan to agree on a channel to access beforehand. Trojan (iGPU kernel) will write to different cache lines to avoid memory request coalescing. From Figure~\ref{fig:slowdown_b}, we can notice the slowdown abruptly appears when the number of write requests is $2^{19}$ memory requests when all cache lines are accessed. Consequently, the allocated buffer size should be 32MB since writes have to be done to different cache lines to avoid the coalescing effect. A buffer size of 32MB is double the size of LLC in our targeted system.

Accessing different cache lines from the allocated 32MB buffer will increase the probability of dirty cache line evictions. Considering the parallelism of iGPU, this will speed up filling the write buffer in the memory controller to stall spy (CPU process) read requests. The drawback of this approach is that it requires iGPU kernel to access a large number of cache lines to evict dirty cache lines. This will lower the bit rate as we will show later. 

To construct an efficient covert channel attack in terms of bit and error rates, we need to consider multiple attack parameters. Such parameters are the number of write requests, the number of local and global threads, and the iteration factor to send bit '0'. We investigated the role of these parameters as we show in Figure \ref{fig:zero_iter_factor} and Figure \ref{fig:glob_loc_threads}.

%TODO: talk about zero iteration factor
We mentioned that for the iGPU kernel to send bit '0', it does the same number of writes as in sending bit '1'. We noticed that based on the latency level observed by the spy process when sending bit '1', an iteration factor of one is not enough to send bit '0' after sending bit '1'. We investigated both secret detection time and error rate for different bit '0' iteration factors and different number of write requests as we depict in Figure \ref{fig:zero_iter_factor}.

In Figure \ref{fig:zero_iter_factor}, local and global threads number is 256 and the size of secret is 1024 bits. Increasing the kernel buffer access stride, will decrease the number of write requests and thus decrease spy execution time. However, using larger strides to access iGPU kernel buffer leads to higher error rates since the number of evicted cache lines will be lower. Increasing bit '0' iteration factor shows to decrease the error rate because it makes bit '0' detectable. For this attack and from Figure \ref{fig:zero_iter_factor} we can observe that accessing iGPU kernel at a stride of eight cache lines and with bit '0' iteration factor of eight results in low error rate (~0.1\%) and small execution time.

Furthermore, we investigated the role of number of local and global threads in attack's performance. Local work group size is equal to local threads and the total number of local work-groups is equal to the number of global threads divided by local work-group size. In Figure \ref{fig:local_threads}, the number of global threads is equal to 256 threads. We notice that increasing local threads decrease execution time for initial cases and starts increasing when the number of local threads is 64. In our targeted iGPU, the size of a wavefront (sub-work group size) can be 8,16, or 32. As we increase the number of local threads, the error rate decreases and this is because, with more local threads, more memory write requests are generated. Based on this experiment, we use a local work-group size of 128 threads because it achieves a low error rate and acceptable execution time compared to the case of 64 local threads which achieves ~10\% error rate.

We also explored the role of increasing the number of global threads. The execution time initially decreases until the number of global threads reaches 2048. Error rate becomes ~100\% for global threads equal to or more than 2048 threads. When global threads are 128 or 256 threads, error rate is smaller than 0.2\%. Although 512 threads achieve execution time lower than 128 and 256 threads, it has a higher error rate (~22\%). Having a larger number of global threads, means lower number of writes per thread and this will lower latency overhead caused by iGPU kernel since all of these threads can't be scheduled at the same time. Also, there is the overhead of creating and scheduling these threads.

Based on these experiments, we conclude the parameters required to achieve an efficient covert channel attack which is oblivious of accessed MC channel. iGPU kernel has to write to 32MB buffer at a stride of eight cache lines. Additionally, To make bit zero detectable by the spy process, its iteration factor must be eight considering iGPU kernel buffer size and stride. Also, the local work-group size should be 128 threads and the number of global threads should be 256. As a result, two local work-groups will be created by the kernel for this attack.

\subsection{Attack Variant 2: Targeting Single MC Channel}
\label{subsec:attack_var2}
In our targeted system, CPU core and iGPU share dual channel MC. Each channel has its own read and write buffers. From reverse engineering of MC channel bit addressing, we noticed that the allocated buffer is distributed between these two channels. Considering this, it is possible for spy (CPU process) and trojan(iGPU kernel) to pre-agree on a channel such that single pair of read and write buffers are targeted. This will reduce the total number of writes required to cause spy slowdown and as a result, improve the bit rate. 

% Figure environment removed

Similar to the experiments we have performed in Section~\ref{subsec:var1_design}, we inferred the parameters required for this attack approach. The trojan buffer should be accessed at a stride of four rather than eight. This is because half of iGPU kernel writes will access targeted channel, and we need to ensure eviction of dirty cache lines. Note that the total number of iGPU kernel writes in this attack is similar to the previous attack. The difference is that iGPU kernel traffic is directed to one pre-agreed channel.
In this attack, the iGPU kernel write requests are preceded with a buffer index read. This is the buffer index that iGPU kernel will be writing to next and which is targeting the same channel as spy reads. Because of this, the frequency of writes is lower than the previous attack and bit '0' iteration factor of two has better attack performance. Local threads of 128 and global threads of 256 achieved a good performance in terms of error and bit rates similar to previous attack.

The bit rate in this attack is higher because bit '0' iteration factor is smaller. Additionally, frequency of write requests from iGPU kernel is lower and as result some of spy read requests will suffer lower latency than the latency we show in Figure~\ref{fig:latency_per_access}.

\begin{comment}
\subsection{Attack Variant 3: Deterministic Evictions of Dirty Cache lines}

It is important to note that the size of LLC writeback buffer and write buffer in MC is smaller than $2^{19}$ entries. Meaning that smaller number of evicted dirty cache lines should be required to get the write back buffer filled. Accessing smaller number of cache lines to fill the write buffer will improve bit rate.

To cause spy process slowdown by writing to smaller number of cache lines by the iGPU kernel, we need to find a collection of addresses which map to the same LLC set and slice (eviction set). Note that we can allocate a buffer which is shared between CPU core and iGPU. Such buffer can be allocated using Intel unified shared memory API which is part of OpenCL API \cite{unified}. Thus, we can find the eviction set on the CPU side of the trojan.

The size of eviction set is dependent on the size of writeback buffer and write buffer. Note that in addition to evicting dirty cache lines from LLC, dirty cache lines has to be evicted from writeback buffer as well to eventually reach the write buffer in memory controller. 

\subsubsection{Writeback and Write Buffers}
In order to approximate the size of writeback and write buffers, we tested slowdown level for different eviction set sizes. We expect the sizes of these buffers to be between 64 and 256 entries.
\ghadeer{continue this part based on reverse engineering results, eviction set}
%TODO:

\end{comment}
\section{Evaluation of The Attack\label{sec:attack_eval}}
In this section, we present the evaluation results of our proposed attack. First, we introduce our experimental setup. Then, we provide a detailed analysis of the performance of our attack on various datasets, discussing its effectiveness and limitations.
\subsection{Experimental setup}
\subsubsection{Datasets}
In order to evaluate the effectiveness of our attack, we conducted experiments on various real-world datasets previously utilized in related research. We include the Flickr \cite{ZengZSKP20_Flickr} dataset, where nodes represent images uploaded to the Flickr platform. Edges connect nodes if the images share common properties like geographic location, gallery, or user comments. Node features contain word representations. Additionally, we utilize two Twitch datasets (TWITCH-FR and TWITCH-RU)\cite{rozemberczki2021twitch} to evaluate NILS. We use Twitch-ES to train the GNNs as done previously in \cite{linkteller} for the inductive setting. Twitch datasets \cite{rozemberczki2021twitch} illustrate follow relationships between users on the Twitch streaming platform. The objective of these datasets is to perform binary classification to determine if a streamer uses explicit language, using features such as users' preferred games, location, and streaming habits.

Furthermore, for the transductive setting, where the training and testing of the GNNs occur on the same graph, we incorporate three citation network datasets \cite{citation_datasets}, Cora, Citeseer, and Pubmed. These datasets capture citation relationships among scientific publications across various fields. The classification task of these datasets involves predicting the topic of publications based on their textual features. While Cora and Citeseer encompass general scientific publications, Pubmed is dedicated to biomedical publications. By employing these datasets in our evaluation, we aim to demonstrate the effectiveness of our proposed attack in both inductive and transductive settings, as well as across a range application domains.

\subsubsection{Models}

In our study, we follow LinkTeller's approach to training the models and selecting hyperparameters \cite{linkteller}. In LinkTeller \cite{linkteller}, the authors trained Graph Convolutional Networks (GCNs) using various configurations and hyperparameters, which encompassed normalization techniques applied to the adjacency matrix, the number of hidden layers, input and output units, and dropout rates. In order to identify the optimal set of hyperparameters, the authors employed a grid search strategy, systematically exploring combinations of hyperparameters and evaluating their performance on a validation set.
The search space for hyperparameters and the formulae for different normalization techniques were provided in \cite[Appendix F]{linkteller}. After obtaining the best set of hyperparameters, the authors trained the GCN models to minimize the cross-entropy loss for the intended tasks.

In our experiments, we adhere to the same methodology as in LinkTeller \cite{linkteller}, ensuring consistency across the studies. By utilizing the same training procedures and hyperparameter tuning strategies, we aim to provide a comprehensive understanding of the attack performance across different layer configurations (two, three, and four layers) while maintaining consistency.

\subsubsection{Evaluation of attack performance}
In accordance with the evaluation methodology presented in the LinkTeller paper \cite{linkteller}, we employ precision, recall, and the $F_1$ score as our primary evaluation metrics. These metrics are particularly suitable for addressing the imbalanced binary classification problem at hand, in which the minority class (i.e., connected nodes) is of central interest. We primarily select the set target nodes $V_{\mathcal{A}}$, such that  $|V_{\mathcal{A}}|=500$, using a uniform random sampling approach. Furthermore, following the baseline \cite{linkteller} study's example, we explore scenarios where target nodes exhibit either low or high degrees. A comprehensive discussion of the sampling strategy can be found in \cite[Section V.D.]{linkteller}. We report the results averaged over three runs with different random seeds along with the standard deviation.
% training
% hyperparameters
% GNN architecture
% Evaluation methodology of the attack
\subsection{Analysis of strategies for malicious node’s features}
In this section, we analyze the impact of different strategies, as defined in Section \ref{subsection:malicious features strategies}, for generating the features $x_m$ of the malicious node $v_m$ on the success of our attack.

The success rates of these strategies, as shown in Table \ref{tab:adv_strategies}, reveal that the All-ones, Max attributes, and Class representative strategies are the most effective in causing significant changes in the predictions of the target node's neighbors. These results suggest that injecting nodes with high-valued or class-specific features can effectively disrupt the model's output predictions.

Conversely, the All-zeros, and Identity strategies exhibit relatively lower success rates, as shown in Table \ref{tab:adv_strategies}. While these strategies offer certain benefits in terms of stealthiness, their impact on the graph structure and predictions is less pronounced, highlighting a trade-off between attack effectiveness and stealthiness.

Concerning the Influence strategy, our NILS method exhibits a modest improvement over the LinkTeller baseline for the Twitch-FR dataset, as illustrated in Table \ref{tab:adv_strategies}. This suggests that the node injection property of our NILS attack is effective in this context. However, for the Twitch-RU dataset, NILS underperforms in comparison to the LinkTeller baseline. The most significant improvement is observed in the Flickr dataset, where the node injection property of NILS considerably increases the $F_1$ score from $0.32 \pm 0.13$ of LinkTeller to $0.89 \pm 0.10$. This outcome highlights the advantage of NILS attack's node injection method within the Influence strategy, particularly when compared to the LinkTeller attack, which employs the Influence strategy without node injection. 

These findings underscore the importance of considering both the effectiveness and stealthiness of malicious feature generation strategies when devising link inference attacks on GNNs.

\begin{table}[h]

\begin{adjustbox}{width=\columnwidth,center}
\centering


\begin{tabular}{lccc}
\toprule
Method & Twitch-FR & Twitch-RU & Flickr \\
\midrule
Class Rep. & $0.94 \pm 0.01$ & $0.83 \pm 0.06$ & $0.96 \pm 0.06$ \\
Max Attr.  & $0.99 \pm 0.00$  & $0.98 \pm 0.02$ & $\boldsymbol{1.00 \pm 0.00}$ \\
All-ones   & $\boldsymbol{0.99 \pm 0.00}$ & $\boldsymbol{0.97 \pm 0.01}$ & $0.99 \pm 0.02$ \\
All-zeros  & $0.58 \pm 0.02$  & $0.48 \pm 0.01$  & $0.71 \pm 0.07$ \\
Identity   & $0.81 \pm 0.02$  & $0.69 \pm 0.01$  & $0.95 \pm 0.07$ \\
Influence NILS  & $0.81 \pm 0.02$  & $0.70 \pm 0.01$  & $0.89 \pm 0.10$ \\
Influence LinkTeller \cite{linkteller}   & $0.80 \pm 0.02$  & $0.74 \pm 0.01$  & $0.32 \pm 0.13$ \\
\bottomrule
\end{tabular}
\end{adjustbox}
\caption{$F_1$ scores and standard deviations for different attack methods and datasets.}
\label{tab:adv_strategies}
\end{table}
\subsection{Comparison with the baselines}

In this study, we conducted experiments to evaluate the performance of our proposed NILS attack in comparison to the LinkTeller attack using the same experimental setup. Our focus is on analyzing the optimal attacks for both approaches, which involved accurately estimating the number of neighbors of the target set nodes. The results, summarized in Table \ref{tab:comp_LT}, demonstrate that our attack outperforms LinkTeller on both Twitch datasets (TWITCH-FR and TWITCH-RU). Furthermore, our method exhibits a substantial improvement over LinkTeller on the Flickr dataset, achieving nearly double the precision and recall values. Notably, our attack demonstrates stable performance across varying node degrees, with only a marginal decrease in effectiveness for high-degree target nodes. This can be attributed to the smaller influence that each neighboring node has on the aggregation of the GCN layer when the target node degree is high. Overall, our proposed NILS attack demonstrates consistently a superior performance compared to the LinkTeller attack.

We further compare our attack with link-stealing attacks introduced in \cite{he2021stealing}, where the authors' various attack strategies rely on different types of background knowledge available to the adversary, such as node attributes and shadow datasets. Specifically, in their Attack-2, the adversary has access to both the features and prediction scores of the nodes. Utilizing this information, the adversary creates two types of attacks: LSA2-attr and LSA2-post. LSA2-attr calculates distances between node attributes, while LSA2-post computes distances between node prediction scores (posteriors). It is important to highlight that these two attacks align closely with our threat model, as both assume that the adversary has access to the features and prediction scores of the target node. This similarity in assumptions renders these attacks particularly relevant for comparison with our proposed NILS attack. The attacks are executed under the transductive setting, where training and inference occur on the same graph. As shown in Table \ref{tab:LST_comp}, our proposed NILS attack outperforms the LSA2-post and LSA2-attr attacks constructed in \cite{he2021stealing}. However, our attack performance is nearly equivalent to that of LinkTeller. These results demonstrate that NILS attack maintains effectiveness under the transductive setting, just as in the inductive setting.
% \usepackage{multirow}
\begin{table*}[]
%\begin{adjustbox}{width=\columnwidth,center}
\begin{tabular}{cccccccc}
\toprule
\multirow{2}{*}{Dataset} &
  \multirow{2}{*}{Method} &
  \multicolumn{2}{c}{low} &
  \multicolumn{2}{c}{uncontrained} &
  \multicolumn{2}{c}{high} \\ \cline{3-8} 
 &
   &
  precision &
  recall &
  precision &
  recall &
  precision &
  recall \\ \hline
\multirow{2}{*}{TWITCH-FR} &
  NILS (Ours) &
  $100.0 \pm \scriptstyle 0.0$ &
  $100.0 \pm \scriptstyle 0.0$ &
  $99.13 \pm \scriptstyle 0.8$ &
  $99.57 \pm \scriptstyle 0.35$ &
  $99.91 \pm \scriptstyle 2.6$ &
  $100.0\pm \scriptstyle 0.0$ \\
 &
  LinkTeller &
  $92.5 \pm \scriptstyle 5.4$ &
  $92.5 \pm \scriptstyle 5.4$ &
  $84.1 \pm \scriptstyle 3.7$ &
  $78.2 \pm \scriptstyle 1.9$ &
  $83.2 \pm \scriptstyle 1.4$ &
  $80.6 \pm \scriptstyle 6.7$ \\ \hline
\multirow{2}{*}{TWITCH-RU} &
  NILS (Ours) &
  $100.0 \pm \scriptstyle 0.0$ &
  $100.0 \pm \scriptstyle 0.0$ &
  $96.45 \pm \scriptstyle 0.4 $ &
  $ 98.34\pm \scriptstyle 0.7$ &
  $99.77 \pm \scriptstyle 0.1$ &
  $ 99.37\pm \scriptstyle 0.1$ \\
 &
  LinkTeller &
  $78.8 \pm \scriptstyle 1.9$ &
  $ 92.6 \pm \scriptstyle 5.5 $ &
  $ 71.8\pm \scriptstyle 2.2$ &
  $78.5 \pm \scriptstyle 2.4$ &
  $ 89.7\pm \scriptstyle 1.7 $ &
  $65.7 \pm \scriptstyle 3.9 $ \\ \hline
\multirow{2}{*}{Flickr} &
  NILS (Ours) &
  $100.0\pm \scriptstyle 0.0$ &
  $100.0\pm \scriptstyle 0.0$ &
  $99.11\pm \scriptstyle 1.7$ &
  $95.83\pm \scriptstyle 5.0$ &
  $93.72\pm \scriptstyle 3.1$ &
  $78.9\pm \scriptstyle 1.9 $ \\
 &
  LinkTeller &
  $51.0 \pm \scriptstyle 7.0$ &
  $53.3\pm \scriptstyle 4.7$ &
  $33.8\pm \scriptstyle 13.3$ &
  $32.1\pm \scriptstyle 13.3$ &
  $18.2\pm \scriptstyle 4.5$ &
  $18.5\pm \scriptstyle 6.1$ \\ \hline
\end{tabular}
%\end{adjustbox}
\caption{Comparative performance of our proposed attack NILS and LinkTeller across three datasets (TWITCH-FR, TWITCH-RU, and Flickr) under low, unconstrained, and high constraint settings. The results are presented in terms of precision and recall with corresponding standard deviations}
\label{tab:comp_LT}
\end{table*}

\begin{table}[]
\centering
\begin{adjustbox}{width=\columnwidth,center}
\begin{tabular}{ccccccc}
\toprule
\multirow{2}{*}{Method} &
  \multicolumn{2}{c}{Cora} &
  \multicolumn{2}{c}{Citeseer} &
  \multicolumn{2}{c}{Pubmed} \\ \cline{2-7} 
 &
  precision &
  recall &
  precision &
  recall &
  precision &
  recall \\ \midrule
NILS (Ours) &
  $ 99.7\pm \scriptstyle 0.2$ &
  $ 99.6\pm \scriptstyle 0.3 $ &
  $97.4 \pm \scriptstyle 0.2 $ &
  $98.2 \pm \scriptstyle 0.1$ &
  $ 99.7\pm \scriptstyle 0.0 $ &
  $100.0 \pm \scriptstyle 0.0 $ \\
LinkTeller &
  $99.5 \pm \scriptstyle 0.1 $ &
  $ 99.5\pm \scriptstyle 0.1$ &
  $99.7 \pm \scriptstyle 0.0$ &
  $99.7 \pm \scriptstyle 0.0$ &
  $99.7 \pm \scriptstyle 0.0$ &
  $99.7 \pm \scriptstyle 0.0$ \\
LSA2-post &
  $ 86.7 \pm \scriptstyle 0.2 $ &
  $ 86.7\pm \scriptstyle 0.2$ &
  $ 90.1 \pm \scriptstyle 0.2$ &
  $ 90.1 \pm \scriptstyle 0.2$ &
  $ 78.8\pm \scriptstyle 0.1$ &
  $ 78.8\pm \scriptstyle 0.1$ \\
LSA2-attr &
  $73.6 \pm \scriptstyle 0.1$ &
  $73.6 \pm \scriptstyle 0.1$ &
  $80.9 \pm \scriptstyle 0.1$ &
  $80.9 \pm \scriptstyle 0.1$ &
  $ 82.4\pm \scriptstyle0.1 $ &
  $ 82.4\pm \scriptstyle0.1 $ \\
\bottomrule
  
\end{tabular}
\end{adjustbox}
\caption{Comparative performance of NILS attack with LinkTeller \cite{linkteller} and link-stealing attacks in \cite{he2021stealing} across three datasets (Cora, Citeseer, and Pubmed).}
\label{tab:LST_comp}
\end{table}

\subsection{Depth of the GNN}
In this section, we examine the impact of increasing the depth of GNN on the success rate of the attack for the Twitch-Fr dataset. Our findings illustrated in Figure \ref{fig:depth_imact_LT} indicate that as the depth of the GNN increases, the attack's success rate decreases, which can be attributed to the dilution of the injected poisoning node's influence within the target node's neighborhood. As the GNN depth increases, the model aggregates information from a larger neighborhood, encompassing nodes that are $k-$hops away from the target node. Consequently, the injected malicious node's features become one among many contributing factors in the aggregated information, leading to a dilution of its influence. This reduction in the injected node's impact on the aggregated information diminishes the overall effectiveness of the attack, making it less successful in altering the predictions of the target node's neighbors.

In comparison with LinkTeller \cite{linkteller}, as shown in Table \ref{tab:depth_imact_LT}, NILS outperforms LinkTeller \cite{linkteller} across various GCN depths. Specifically, for Twitch-FR dataset, NILS demonstrates higher precision and recall values when the GCN depth is 3 (precision: $85.06 \pm \scriptstyle 1.2$, recall: $81.56 \pm \scriptstyle 1.2$) compared to the LinkTeller method (precision: $50.01 \pm \scriptstyle 5.1$, recall: $46.6 \pm \scriptstyle 5.0$). Notably, NILS consistently outperforms LinkTeller even when comparing the attack performance of LinkTeller with a GCN depth of 2 and NILS with a GCN depth of 3. Specifically, for Twitch-FR dataset, NILS demonstrates higher precision and recall values at a GCN depth of 3 (precision: $85.06 \pm \scriptstyle 1.2$, recall: $81.56 \pm \scriptstyle 1.2$) compared to the LinkTeller method with a GCN depth of 2 (precision: $84.1 \pm \scriptstyle 3.7$, recall: $78.2 \pm \scriptstyle 1.9$). These results highlight the effectiveness of our node injection strategy, as it consistently outperforms the LinkTeller method across different depths of the GCN.

% Figure environment removed

%\ayse{isn't it better to use NILS instead of Ours in the table?}
\begin{table}[]
\begin{adjustbox}{width=\columnwidth,center}
\begin{tabular}{cccccc}
\toprule
\multirow{2}{*}{Dataset} &
  \multirow{2}{*}{Method} &
  \multicolumn{2}{c}{Depth-2} &
  \multicolumn{2}{c}{Depth-3} \\ \cline{3-6} 
 &
   &
  precision &
  recall &
  precision &
  recall \\ \midrule
\multirow{2}{*}{TWITCH-FR} & NILS (Ours) & $99.13 \pm \scriptstyle 0.8$ & $99.57 \pm \scriptstyle 0.35$ & $85.06\pm \scriptstyle 1.2$   & $81.56 \pm \scriptstyle 1.2$ \\
 &
  LinkTeller &
  $84.1 \pm \scriptstyle 3.7$ &
  $78.2 \pm \scriptstyle 1.9$ &
  $50.1 \pm \scriptstyle 5.1$ &
  $46.6 \pm \scriptstyle 5.0$ \\ \midrule
\multirow{2}{*}{TWITCH-RU} & NILS (Ours) & $96.45 \pm \scriptstyle 0.4$ & $98.34 \pm \scriptstyle 0.7$  & $78.78 \pm \scriptstyle 3.8 $ & $ 76.35\pm \scriptstyle 9.3$ \\
 &
  LinkTeller &
  $71.8\pm \scriptstyle 2.2$ &
  $78.5 \pm \scriptstyle 2.4 $ &
  $45.7\pm \scriptstyle 2.2$ &
  $50.0 \pm \scriptstyle 2.8$ \\
\bottomrule
\end{tabular}
\end{adjustbox}
  \caption{Success rates of the attack for different depths in comparison with LinkTeller \cite{linkteller}. We use the all-ones strategy and Twitch-FR dataset.}
  \label{tab:depth_imact_LT}
\end{table}


%\section{Side Channel Attack}
\label{sec:side-channel}


In this section, we present a proof-of-concept side channel attack by exploiting the contention on the memory controller to leak secret information. In our side channel scenario, a victim application is running on iGPU and a CPU-based spy extracts some sensitive information from the victim by monitoring the contention on the shared memory controller.

For our victim application, we target Kmeans application from Rodinia Benchmark\cite{rodinia_paper, rodinia} on iGPU. Kmeans application in Rodinia benchmark includes two GPU kernels: {\fontfamily{pcr}\selectfont kmeans\_swap} and {\fontfamily{pcr}\selectfont kmeans\_kernel\_c}. {\fontfamily{pcr}\selectfont kmeans\_kernel\_c} is responsible for clustering and only includes buffer reads. While {\fontfamily{pcr}\selectfont kmeans\_swap} does data swapping and as a result includes both memory reads and writes.

The number of threads in each work-group is set to 256 in Kmeans benchmark and the number of work-groups is dependent on the size of input data (data points). For example, if the number of data points is 2560, then a total of 2560 threads will be used (i.e. 10 work-groups of each 256 threads). We experiment with the maximum number of features of 255 and different data point sizes. Figure \ref{fig:side_channel_mem_accesses} demonstrates the CPU-based spy average latency, when it accesses a buffer in memory. Similar to single kernel spy setting, for side channel attack, the attacker accesses a buffer of 1MB at a stride of eight cache lines. The attacker keeps flushing and accessing its buffer to monitor the contention level caused by iGPU kernel.
%\hoda{add details of spy buffer size and pattern, and flushing} \ghadeer{I will add data about attacker's buffer and its algorithm}. 

The execution behavior of two Kmeans kernel functions is clearly observed by the spy process. {\fontfamily{pcr}\selectfont kmeans\_swap} kernel is first to execute and only once. After some time based on the size of input data,  
%\hoda{What does number of iteration mean? Does it relate to the input size?}
%\ghadeer{it is the number of iteration required until error threshold is reached. yes, it is different with different input data sizes and it is correlated with number of clusters being created. it is indicated by the number of green circles. I'm showing only portion of the collected data. but attacker monitored kernel execution till the end, the number of green circles is the number of iterations.}
{\fontfamily{pcr}\selectfont kmeans\_kernel\_c} executes for a number of iterations. 



%\hoda{not clear what you mean, is the period of red circle or the time between red and green circles?} 
%\ghadeer{each iteration indicate execution of the kernel function shown in the green circle. the kernel function in red circle is executed only one time.}
The execution time of each kernel, the elapsed time between two kernels (red circle and green circle), and also the elapsed time between every two iterations of {\fontfamily{pcr}\selectfont kmeans\_kernel\_c} kernel
are dependent on data points size. Also, the attacker is able to figure out the number of iterations the algorithm would take till convergence (as shown in Figure~\ref{fig:side_channel_mem_accesses}). This can leak information about Kmean algorithm parameters such as the number of clusters, the input data size, and the error threshold. %\hoda{how the error threshold is related} \ghadeer{Kmean algorithm executes the kernel function in green circle till convergence. this convergence is dependent on the error threshold. iterations till convergence is determined by the number of green circle. it will not give attacker direct value to error threshold, but an approximate range}.

%why contention level is small
From Figure \ref{fig:side_channel_mem_accesses}, it can be noticed that the level of contention is not as high as the contention level we observed in the covert channel attack. This is mainly due to memory coalescing on the iGPU. With memory coalescing, memory accesses to same cache line are merged as a single memory request \cite{gen9}. In our covert channel, we designed our access patterns to avoid memory coalescing, however, the attacker in a side channel does not have any control over the victim process. Memory coalescing will decrease the memory traffic and as a result, the contention level compared to having memory accesses to different cache lines. However, the iGPU kernel execution behavior is still clearly visible and attacker can infer information about Kmean algorithm by monitoring the contention on the memory controller.


%role of llc contention
%To investigate the role of (potential) LLC or ring-interconnect contention, we forced the CPU-based spy buffer accesses to be LLC hits. Figure \ref{fig:side_channel_llc_accesses} illustrates the attacker's execution behavior during Kmean kernels' execution. The attacker can observe some variations in its execution behavior due to iGPU kernel memory accesses, but it does not observe as high contention level as in Figure \ref{fig:side_channel_mem_accesses}. This indicates that the main contribution behind the contention is the shared queues in the memory controller. \hoda{in this version, although the contention level is lower, the elapsed time, two kernels, ... are also visible. We can remove this figure and its text! No need to include it}
%\ghadeer{i see your point. i will try to add back propagation algorithm after finishing the draft and fixing the figures}

% Figure environment removed

\begin{comment}
% Figure environment removed
\end{comment}


\begin{comment}
% Figure environment removed

\end{comment}
\section{Discussion}
\label{sec: discussion}
\kmsdelete{In this work} We study \kmsreplace{Fairness-Aware PAC learning}{Fair-ERM} in the malicious noise model, and  in some cases allow 
the learner to maintain optimal overall accuracy despite the signal in Group $B$ being almost entirely washed out.
%when we allow learners to use the
%$\PQ$ randomized expansion of the hypothesis class $\mathcal{H}$
In particular we show that different fairness constraints have fundamentally different behavior in the presence of Malicious Noise, in terms of the amount of accuracy loss that a given level of Malicious Noise could cause a fairness-constrained learner to incur. 
The key to achieving our results, which are more optimistic than those in \cite{lampert}, is allowing for improper learners using the (P,Q)-randomized expansions of the given class $\mathcal{H}$.
%We \kmsreplace{present a picture of the}{prove upper and lower bounds on}
%accuracy loss for a range of fairness notions, given \kmsreplace{this simple randomization step.}{learning over $\PQ$.
%In general our results indicate Fair-ERM (given learning over $\PQ$) is more robust than claimed in \cite{lampert}.
The type of smoothness we create by using $\PQ$ seems to be a natural property that is likely shared by many natural hypothesis classes.

Fairness notions are motivated as a response to learned disparities when there is \kmsdelete{data corruption or} systemic error affecting \kmsdelete{the data for}
one group. 
Fairness notions are supposed to mitigate this by ruling out classifiers that have worse performance on a sub-group. 
This can peg both classifiers at a lower level of performance \kmsdelete{(e.g that the lower subgroup)} in order to \emph{motivate} \cite{hardt16} improving the data collection or labelling process to obtain more reliable performance. 
%So in \kmsreplace{some}{a} sense, sensitivity of the fairness notion to poor sub-group performance caused by malicious noise is the \textit{point} of fairness constraints! 
However, it also desirable that fairness constraints perform gracefully when subject to Malicious Noise because fairness constraints will be used in contexts where the data is unreliable and noisy and this might not be known to the learner.
This tension, exposed by our work, motivates 
%a revisiting of fairness notions from first principles approach and trying to axiomatize the 
%desired properties of a fairness intervention a la cryptography and privacy. \footnote{Work in multi-calibration \cite{multicalib} is a viable direction for this problem but it is unclear how 
%that and related notions behave with unreliable data. }
on going work studying the sensitivity level of fairness constraints. 
%If we we are to take a view, if a classifier is deployed 

\section{Possible Mitigations}
\label{sec:mitigations}
% discuss mitigations which proposed to solve the write buffer issue
% comment on their security, and how it can be improved with consideration to securtity
In this section, we discuss different possible mitigations to thwart the attacks or to make it challenging for attackers to exploit memory write requests.\newline
\noindent{\textbf{Prioritizing Memory Read Requests.}} One of the possible solutions for the latency observed by the CPU process due to iGPU kernel memory write requests is deploying a policy other than \textit{drain\_when\_full}. Deploying a management policy that gives priority to memory read requests whenever they are inserted in the read buffer will thwart our attack. Such an approach may prevent iGPU kernel or CPU processes from issuing write requests if there are read requests in the read buffer and the write and writeback buffers are full. This will deteriorate the performance of CPU processes or iGPU kernel if they are issuing a lot of memory write requests. Also, latency overhead will appear due to \textit{read\_after\_write} and \textit{write\_after\_read} especially if there is frequent alteration between serving memory read and write requests.

\noindent{\textbf{Directing CPU Process's Memory Reads to Ideal DRAM Banks.}} Chatterjee et al.~\cite{staged_reads} proposed Staged Reads to serve memory read requests while memory write requests are being served. This happens only if memory reads are accessing banks not used by memory writes (i.e. ideal banks). Such an approach will not completely thwart our attack, but it will make it challenging for attackers to utilize the effect of \textit{drain\_when\_full} management policy. If Staged Reads \cite{staged_reads} technique is deployed in the system, spy and trojan should pre-agree on accessing same banks for spy read requests to be stalled.

\noindent{\textbf{Channel Partitioning.}} It is possible to thwart our covert channel attack using channel partitioning in multi-channel MCs. This is because each MC channel has its own pair of read and write buffers. One of the MC channels can be dedicated to iGPU during its kernel execution. If the CPU process needs to read from memory associated with this channel, it has to stall until the iGPU kernel finishes execution. Obviously, this approach has a performance impact on CPU processes, especially in the case of kernels with long execution times.





% We believe classes of defenses that have been developed against other microarchitectural covert and side channels on a single component (either CPU or GPU) can potentially apply to cross-component attacks on heterogeneous systems. Rather than securing each component in isolation, these defense designs require some adaptations to consider the system-wide security, functionality, performance, and power goals. These solutions include: 

% \noindent \textbf{Dynamic partitioning of resources:} memory controller request queues (specifically transaction queue) or DRAM banks using dedicated hardware. These partitioning schemes need to support the isolation of memory requests from different processors in heterogeneous systems and also support the bandwidth requirement of each processor dynamically to minimize the performance overhead. 

% \noindent \textbf{Memory scheduling and traffic control in memory controllers:} eliminating the contention among processes running on different processors, such that memory requests from each processor are grouped into the same queue (or sub-queues) and possibly scheduled and routed to the same subset of memory banks or channels. 






%In this section, we suggest different possible mitigations to address contention on memory controller resources due to iGPU memory traffic.

%\textbf{Spatial Partitioning.} One of the possible approaches to avoid contention on MC resources is to partition them spatially. The high level of contention we observe in this attack is due to sharing of resources such as queues within the memory controller. If we partition these resources between CPU cores and iGPU, spy process will never observe same level of contention. Partitioning should also be considered for other resources such as memory channels. Current iGPU is usually deployed in processor chips with one memory controller with two channels. It is possible to assign one of the channels for iGPU traffic and the other channel for CPU cores traffic. The drawback of this approach is that iGPU is not always executing kernels and resources assigned to iGPU will become under utilized.

%\hoda{temporal partitioning has huge performance impact, not a good solution in any case}
%\textbf{Temporal Partitioning.} Another possible approach to avoid contention caused by iGPU memory traffic is to use temporal partitioning. In this approach, a turn is designated to each source whether its iGPU or CPU cores for its memory requests to be served. With temporal partitioning iGPU and CPU cores memory requests will not use the same resources at the same time. This approach still has to consider in flight memory requests even if resource turn finishes issuing its requests \cite{cornell_defense_hpca}.

%time and space multiplexting
%resource paritioning
%
\section{Related Work}
%\subsection{Cost Volume based Deep Stereo Matching}
%Stereo matching is a typical problem that has been studied for decades and a well-known four-step pipeline \cite{scharstein2002taxonomy} has been established, where cost volume construction is an indispensable step. Current state-of-the-art stereo matching methods are all cost volume based methods and they can be categorized into two types. Typically, a cost volume is a 4D tensor of height, width, disparity, and features. The first category just uses a full correlation to generate a single-feature cost volume. Such methods are usually efficient but lose much information because of the decimation of feature channels. Many previous work, including Dispnet \cite{dispnet}, MADNet \cite{madnet}, IResNet \cite{iresnet} and AANet \cite{aanet}, belong to this category. The second category usually uses concatenation \cite{gcnet} or group-wise correlation \cite{gwcnet} to generate a multi-feature 4D cost volume. Such a method can achieve better performance while requiring higher computational complexity and memory consumption. Actually, a majority of the top-performing networks in public leaderboards belong to this category, such as GANet \cite{ganet}, CSPN \cite{cspn} and ACFNet \cite{acfnet}. These methods generally employ multiple 3D convolution layers to constantly regularize the 4D cost volume and then apply softmax over the disparity dimension to produce a discrete disparity probability distribution. The final predicted disparity is obtained by softly weighting indices according to their probability, which is also called soft argmin in GCNet \cite{gcnet}. However, soft argmin leaves the output susceptible to multi-modal disparity probability distributions. ACFNet \cite{acfnet} observes this problem and proposes to directly supervise the cost volume with unimodal ground truth distributions. In contrast, we define an uncertainty estimation to quantify the degree to which the cost volume tends to be multi-modal distribution, higher implies the higher possibility of estimation error.

\subsection{Multi-scale Cost Volume based Stereo Matching}
Cost volume construction is an indispensable step in the well-known four-step pipeline for stereo matching \cite{scharstein2002taxonomy, pamisurvey1, pamisurvey2}. Typically, current state-of-the-art stereo matching methods can be categorized into two types of cost volume-based methods, where the cost volume is a 4D tensor of height, width, disparity, and features. The first category usually uses the single-feature 3D cost volume generated by full correlation, which is efficient while losing much information due to the decimation of feature channels. Many real-time methods, such as Dispnet \cite{dispnet}, MADNet \cite{madnet, madnet_pami} and AANet \cite{aanet}, belongs to the category. Moreover, two-stage refinement \cite{mcvmfc} and pyramidal towers \cite{madnet} are commonly applied in the single-feature cost volume based network to construct multi-scale cost volume. The second category usually uses the multi-feature 4D cost volume generated by concatenation \cite{gcnet} or group-wise correlation \cite{gwcnet}, which can achieve better performance with higher computational complexity and memory consumption. Most top-performing networks, including GANet \cite{ganet}, CSPN \cite{cspn} and ACFNet \cite{acfnet} belong to this category. 
% In these methods, the 4D cost volume is constantly regularized by multiple 3D convolution layers and then a discrete disparity probability distribution can be produced by softmax. Next, the final predicted disparity can be obtained by softly weighting indices according to their probability \cite{gcnet}. However, such output is susceptible to multimodal disparity probability distributions and ACFNet \cite{acfnet} gives a solution by directly supervising the cost volume with unimodal ground truth distributions to alleviate this problem. 
Recently, to alleviate the high computational complexity and memory consumption when employing multi-feature 4D cost volumes, \cite{cvpmvsnet, cascade, uscnet} propose to use cascade cost volume representation in multi-view stereo. These methods usually first predict an initial disparity at the coarsest resolution of the image and then gradually refine the disparity by narrowing down the disparity search space. More closely related to our approach is Casstereo \cite{cascade}, which first extended such representation to stereo matching. It selected to uniform sample a pre-defined range to generate the next stage’s disparity search range. Instead, we employ pixel-level uncertainty estimation to adaptively adjust the next stage disparity searching range and generate pseudo-labels for subsequent domain adaptation. Our method also shares similarities with UCSNet \cite{uscnet}, which constructs uncertainty-aware cost volume in multi-view stereo while it doesn’t employ uncertainty estimation to generate pseudo-labels.

%\subsection{Multi-scale Cost Volume based Deep Stereo Matching} 
% \subsection{Multi-scale Cost Volume based Stereo Matching} 
%Multi-scale cost volume firstly was applied in the single-feature cost volume based network with the form of two-stage refinement \cite{mcvmfc} and pyramidal towers \cite{madnet}. Recently, cascade cost volume representation \cite{cvpmvsnet, cascade, uscnet} was proposed in multi-view stereo to alleviate the high computational complexity and memory consumption when employing multi-feature 4D cost volumes. These methods generally predict an initial disparity at the coarsest resolution of the image. Then, they will narrow down the disparity search space and gradually refine the disparity. More closely related to our approach is Casstereo \cite{cascade}, which first extended such representation to stereo matching. It selected to uniform sample a pre-defined range to generate the next stage’s disparity search range. Instead, we employ uncertainty estimation to adaptively adjust the next stage pixel-level disparity searching range and push the next stage's cost volume to be predominantly unimodal.

% The single-feature cost volume based network with the form of two-stage refinement \cite{mcvmfc} and pyramidal towers \cite{madnet} first employ multi-scale cost volume for stereo matching. Recently, to alleviate the high computational complexity and memory consumption when employing multi-feature 4D cost volumes, \cite{cvpmvsnet, cascade, uscnet} propose to use cascade cost volume representation in multi-view stereo, which generally predict an initial disparity at the coarsest resolution of the image. Then, the disparity search space is narrowed down and the disparity is gradually refined. More closely related to our approach is Casstereo \cite{cascade}, which first extended such representation to stereo matching. It selected to uniform sample a pre-defined range to generate the next stage’s disparity search range. Instead, we employ uncertainty estimation to adaptively adjust the next stage pixel-level disparity searching range and push the next stage's cost volume to be predominantly unimodal.

% Figure environment removed

\subsection{Robust Stereo Matching} 
There exist three categories of generalization definitions for robust stereo matching. 1) Cross-domain Generalization: the network’s ability to perform well on unseen scenes (cannot see the image pairs of the target domain in advance). Towards this end, Jia et al \cite{sungeneralizaiton} propose to incorporate scene geometry priors into an end-to-end network. Zhang et al \cite{dsmnet} introduce a domain normalization and a trainable non-local graph-based filter to construct a domain-invariant stereo matching network. 2) Adapt Generalization: the network’s ability to adapt pre-trained models to the new domain with unlabeled target data. Previous work usually pre-trains the models on synthetic data and then adapts it to new target domains with Graph Laplacian regularization \cite{zoom}, non-adversarial progressive color transfer \cite{adastereo}, and Knowledge Reverse Distillation \cite{aohnet}. More closely related to our approach are \cite{aohnet, unsuperviseddomainadaptation} in stereo matching and Monoresmatch \cite{monoresmatch} in monocular depth estimation, which also proposes to generate a pseudo-label for domain adaptation. However, these methods all select to employ classical stereo matching methods \cite{sgm} alongside with confidence estimators, e.g., left-right consistency check to generate pseudo-labels. That is all these methods need an independent method to generate corresponding pseudo-labels. Instead, the proposed method is an end-to-end network that can generate the predicted disparity map, corresponding uncertainty map and pseudo-labels jointly, which is a more simple, yet efficient way. 
% Instead, our proposed method can employ pixel-level and area-level uncertainty estimation to self-distill the predicted disparity maps of our pre-training model and generate sparse while reliable pseudo-labels to align the domain gap, which is a more simple, yet efficient way. 
3) Joint Generalization: the network’s ability to perform well on a variety of datasets with the same model parameters. MCV-MFC \cite{mcvmfc} introduces a two-stage finetuning scheme to achieve a good trade-off between generalization and fitting capability on multiple datasets. However, it doesn’t touch the inner difference between diverse datasets, e.g, the unbalanced disparity distribution. To further address this problem, we propose a cascade cost volume to adaptively the next stage disparity searching space, where the pixel-level uncertainty estimation is at the core.

% \subsection{Monocular Depth Estimation}
% Monocular depth estimation aims to estimate depth values from a single image, instead of stereo images or multiple frames in a video. This problem is ill-posed because of the ambiguity of object sizes. However, humans could estimate the depth from a single image with prior knowledge of the scenes. Recently, learning based methods were explored to learn depth values by supervised or unsupervised learning. Eigen et al. first employed Convolutional Neural Networks (CNN) to predict depth in a coarse-to-fine manner and further improved its performance by multi-task learning. Liu et al. presented deep convolutional neural fields model by combining deep model with continuous CRF. Li et al. [22] refined deep CNN outputs with a hierarchical CRF. Multi-scale continuous CRF was formulated into a deep sequential network by Xu et al. [45] to refine depth estimation. Unsupervised methods tried to train monocular depth estimation with stereo
% image pairs or image sequences and test on single images. Garg et al. [9] used novel image view synthesis loss to train a depth estimation network in an unsupervised way. Godard et al. [11] introduced left-right consistency regularization to improve the performance of view synthesis loss. Recently, some work also propose to use the stereo matching network as a proxy to learn depth from synthetic data or directly employ traditional stereo matching methods to distill proxies labels from the target domain, which proves the feasibility of distilling stereo matching networks to learn monocular depth estimation.



\section{Conclusion and Future Work}
In this work, I design corruption-robust algorithms for the Lipschitz contextual search problem. I present the \emph{agnostic checking} technique and demonstrate its effectiveness in designing corruption-robust algorithms. There are several open problems for future research. First, in the algorithm I propose for pricing loss, the schedule for agnostic checks is fixed upfront. Can the learner design an adaptive checking schedule for the pricing loss? Second, this work assumes the learner has knowledge of the Lipschitz constant $L$. Can the learner design efficient no-regret algorithms without knowledge of $L$? 
%\begin{comment}
\section{System Architecture}
\label{appendix:architecture}
\system has a novel modularized system architecture with three key components: 
\emph{StreamManager}, 
\emph{TxnManager} and \emph{TxnScheduler}. 
These components are instantiated in each thread locally.
The execution outline of \system is presented in Algorithm~\ref{alg:algo}.
Transactional stream processing is continuous and potentially never ends (Line 1$\sim$8).
The dependency resolution and execution of state transactions are separated into two non-overlapping phases by punctuations~\cite{Tucker:2003:EPS:776752.776780} (Line 2 and 5), which guarantees that no subsequent input event will have a smaller timestamp. 
Effectively, a batch of state transactions is collected during the first phase, and processed during the second phase.

In the first phase (i.e., stream processing phase), 
the \emph{StreamManager} conducts preprocessing for every input event ($e$). Similar to some prior works~\cite{tstream}, state transactions may be issued but not immediately processed during preprocessing (Line 3).
The \emph{pre\_processing} and \emph{post\_processing} functions are exposed as APIs to users.
The \emph{TxnManager} handles dependency resolution (Line 4) among state transactions and insert decomposed operations to construct a \tpg. We discuss the detailed two-phase \tpg construction process in Section~\ref{subsec:construction}.

In the second phase  (i.e., transaction processing phase), 
the \emph{TxnManager} is first involved again to refine (Line 6) the constructed \tpg with further dependency resolution.
The \emph{TxnScheduler} 
schedules operations for concurrent execution based on the constructed \tpg according to the three dimensions of scheduling decisions (Line 7). 
In particular, a scheduling decision model $M$ is instantiated based on the constructed \tpg (Line 14).
\textbf{\circled{1}} Guided by $M$, execution threads adopt an exploration strategy (Section~\ref{subsec:explore}) to explore the constructed \tpg for operations available to be scheduled constrained by dependencies. 
\textbf{\circled{2}} 
During exploration, one or multiple operations may be treated as the 
% basic 
unit of scheduling (Section~\ref{subsec:granularity}). 
Subsequently, \textbf{\circled{3}} every thread executes operation(s) in the unit of scheduling with various abort handling mechanisms (Section~\ref{subsec:abort_handling}).
Only when state transactions are processed (i.e., committed or aborted) can the associated input events be postprocessed (Line 8) by the \emph{StreamManager} based on transaction processing results.
\end{comment}

\begin{comment}
\begin{algorithm}
\footnotesize
    \KwData{$e$ \tcp{Input event}}
    \KwData{$txn_{ts}$ \tcp{State transaction}}
    \KwData{$G$ \tcp{The currently constructed TPG}}
    \While{!finish processing of input streams}{
        \eIf(\tcp*[h]{Phase 1}){\text{$e$ is not a $punctuation$}}{
                $txn_{ts}$ $\gets$ PRE\_Processing($e$)\;
                \textbf{TPG\_Construction}($G$, $txn_{ts}$)\; 
          }(\tcp*[h]{Phase 2}){
                \textbf{TPG\_Refinement}($G$)\; 
                \textbf{TXN\_Scheduling}($G$)\; 
                POST\_Processing()\;
          }
    }
    
    \SetKwFunction{FMain}{TPG\_Construction}
    \SetKwProg{Fn}{Function}{:}{}
    \Fn{\FMain{$G$, $txn_{ts}$}}{
        $O_{1..k}$ $\gets$ \textbf{Partition} $txn_{ts}$\;
        \ForEach{\text{operation $O_{i}$ $\in$ $O_{1..k}$}}{
            \textbf{Identify} its \ld\;
            $G$ $\gets$ $G$ + $O_{i}$ \;
        }
    }
    \SetKwFunction{FMain}{TPG\_Refinement}
    \SetKwProg{Fn}{Function}{:}{}
    \Fn{\FMain{$G$}}{
        \ForEach{\text{vertex $e_{i}$ $\in$ $G$}}{
            \textbf{Identify} its \td, \pd\;
        }
    }
    
    \SetKwFunction{FMain}{TXN\_Scheduling}
    \SetKwProg{Fn}{Function}{:}{}
    \Fn{\FMain{$G$}}{
        $M$ $\gets$ Instantiated with $G$;\tcp{A decision model}
        \While{!finish scheduling of $G$
        }{
          \textbf{\circled{2}} $Scheduling Unit$ $\gets$ \textbf{\circled{1}} \emph{Explore}($G$, $M$)\; 
            \textbf{\circled{3}} \emph{Execute with Abort Handling} ($Scheduling Unit$)\; 
        }
    }
  \caption{Execution Outline of \system}
  \label{alg:algo}
\end{algorithm}
\end{comment}


%\section{Acknowledgements}


\bibliographystyle{plain}
\bibliography{main}


\end{document}

