\section{Introduction}

Task-Oriented Dialogue (TOD) systems have shown promise in achieving user goals through conversational interactions~\citep{SMDataflow2020, su2022multi, mo2022towards}. However, existing TOD systems focus on users providing information while the system performs tasks. In contrast, our task bot assists users in executing tasks themselves by providing accurate information and guidance.

However, we face several challenges, including the following: (1) Existing TOD systems prioritize functional goals at the expense of user experience. (2) Inadequate in-domain training data, as modern neural models require large amounts of data, and acquiring annotations through crowdsourcing is costly. In this paper, we present \taco{}, a task-oriented dialogue system designed to assist users in completing multi-step cooking and how-to tasks. Built upon our previous bot~\citep{chen2022bootstrapping} deployed in the Alexa Prize TaskBot Challenge~\citep{gottardi2022alexa}, \taco{} aims to deliver a collaborative and engaging user experience. Figure~\ref{dialogue_example} showcases a partial example dialogue.

Our contributions include: (1) Developing a modularized TOD framework with accurate language understanding, flexible dialogue management, and engaging response generation. (2) Exploring data augmentation strategies, such as leveraging GPT-3 to synthesize large-scale training data. (3) Introducing clarifying questions about nutrition for cooking tasks to personalize search and better cater to user needs. (4) Incorporating chit-chat functionality, allowing users to discuss open topics of interest beyond the task at hand.



% Figure environment removed








