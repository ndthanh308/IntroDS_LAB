% \documentclass[lettersize,journal]{IEEEtran}
\documentclass[letterpaper, 10 pt, conference]{ieeeconf}
\overrideIEEEmargins   
\usepackage{amsmath,amsfonts}
\usepackage{algorithmic}
\usepackage{algorithm}
\usepackage{array}
\usepackage[caption=false,font=normalsize,labelfont=sf,textfont=sf]{subfig}
\usepackage{textcomp}
\usepackage{stfloats}
\usepackage{url}
\usepackage{verbatim}
\usepackage{graphicx}
\graphicspath{ {figures/} }
\usepackage{cite}
\hyphenation{op-tical net-works semi-conduc-tor IEEE-Xplore}
\usepackage{indentfirst}
\usepackage{multirow}
\IEEEoverridecommandlockouts 
\usepackage[margincaption,outercaption,ragged,wide]{sidecap}
\sidecaptionvpos{figure}{t} 
\sidecaptionvpos{table}{t}
\usepackage[amssymb]{SIunits}
\usepackage{booktabs}
%\usepackage[misc,geometry]{ifsym}
\setlength {\marginparwidth }{1.35cm}
\usepackage[colorinlistoftodos,prependcaption,textsize=tiny]{todonotes}
\usepackage{marvosym}
\usepackage{ifsym}
% \usepackage{subfigure}
% \usepackage{subcaption}
\usepackage{caption}
% \usepackage[letterpaper,top=57pt,bottom=43pt,left=48pt,right=48pt]{geometry}
\usepackage{threeparttable}
% \usepackage{tablefootnote}
%  for type3 font
% \usepackage[T1]{fontenc}
% \usepackage{aecompl}
\newcommand{\fref}[1]{Fig. \ref{#1}}
\newcommand{\sref}[1]{Section \ref{#1}}
\newcommand{\tref}[1]{Table \ref{#1}}
% \usepackage{lineno}
% % \linenumbers
% \switchlinenumbers
% \pagewiselinenumbers

\title{\LARGE \bf FDCT: Fast Depth Completion for Transparent Objects}
\author{Tianan Li$^{1}$, Zhehan Chen$^{1,\textrm{\Letter}}$,~\IEEEmembership{Member,~IEEE}, Huan Liu$^{2}$,~\IEEEmembership{Member,~IEEE}, and Chen Wang$^{3}$
\thanks{$^{\textrm{\Letter}}$Corresponding author: {\tt chenzh\_ustb@163.com}}% <-this % stops a space
\thanks{$^{1}$The authors are with the School of Mechanical Engineering, University of Science and Technology Beijing, Beijing 100083, China.}% <-this % stops a space
\thanks{$^{2}$Huan Liu is with the Department of Electrical and Computer Engineering, McMaster University, Canada. Email: {\tt liuh127@mcmaster.ca}}
\thanks{$^{3}$Chen Wang is with the Spatial AI \& Robotics Lab at The Department of Computer Science and Engineering, State University of New York at Buffalo, NY 14260, USA. {\tt\small chenwang@dr.com}}% <-this % stops a space
%\thanks{Manuscript received April 19, 2021; revised August 16, 2021.}
}

% The paper headers
\markboth{IEEE ROBOTICS AND AUTOMATION LETTERS, VOL. X, NO. X, XXX 2022 }%
{Li \MakeLowercase{\textit{et al.}}: FDCT: Fast Depth Completion for Transparent Objects}

\usepackage{fancyhdr}
\renewcommand{\headrulewidth}{0pt}
\renewcommand{\footrulewidth}{0pt}
\usepackage{setspace}

\begin{document}

\maketitle
\thispagestyle{fancy}

\setlength{\footskip}{5mm}
\fancyhead{}
\lhead{}
\lfoot{
\small
% 两倍行距的文字
\copyright~2023 IEEE.  Personal use of this material is permitted.  Permission from IEEE must be obtained for all other uses, in any current or future media, including reprinting/republishing this material for advertising or promotional purposes, creating new collective works, for resale or redistribution to servers or lists, or reuse of any copyrighted component of this work in other works.}
\cfoot{}
\rfoot{}
\begin{abstract}
Depth completion is crucial for many robotic tasks such as autonomous driving, 3-D reconstruction, and manipulation.
Despite the significant progress, existing methods remain computationally intensive and often fail to meet the real-time requirements of low-power robotic platforms.
Additionally, most methods are designed for opaque objects and struggle with transparent objects due to the special properties of reflection and refraction.
To address these challenges, we propose a Fast Depth Completion framework for Transparent objects (FDCT), which also benefits downstream tasks like object pose estimation.
To leverage local information and avoid overfitting issues when integrating it with global information, we design a new fusion branch and shortcuts to exploit low-level features and a loss function to suppress overfitting.
This results in an accurate and user-friendly depth rectification framework which can recover dense depth estimation from RGB-D image alone.
Extensive experiments demonstrate that FDCT can run about 70 FPS with a higher accuracy than the state-of-the-art methods.
We also demonstrate that FDCT can improve pose estimation in object grasping tasks.
To benefit to the robotics community, we release the source code at \url{https://github.com/Nonmy/FDCT}.
\end{abstract}

% \begin{IEEEkeywords}
% Deep Learning for Visual Perception, deep learning in grasping and manipulation.
% \end{IEEEkeywords}

\section{Introduction}
Current quantum hardware is unable to carry out universal quantum computations due to the buildup of errors that occur during the computation. 
The magnitude of the individual error is currently above the value that the Threshold Theorem requires in order to kick-start quantum error correction and fault-tolerant quantum computation~\cite[Section 10.6]{nielsen_chuang_2010}. 
Although the experimentally achieved fidelity rates are promising and the error bounds are inching closer to the required threshold, we will have to work for the foreseeable future with quantum hardware with errors that build-up during the computation.  This implies that we can only do a limited number of steps before the output of the computation has become completely uncorrelated with the intended one.

For fault-tolerant quantum computing, we repeat four steps: 
1) We apply a number of single and two-qubit quantum gates, in parallel whenever possible; 
2) We perform a syndrome measurement on a subset of the qubits; 
3) We perform fast classical computations to determine which errors have occurred and how to correct them; 
and, 4) We apply correction terms based on the classical computations.
We then repeat these four steps with a next sequence of gates. 
These four steps are essential to fault-tolerant quantum computing. 


The starting point of this work is to use the four steps outlined above, not to carry out error correction and fault-tolerant computation, but to enhance short, constant-depth, {\em uncorrected} quantum circuits that perform single qubit gates and {\em nearest-neighbor} two qubit gates. 
Since in the long run we will have to implement error-correction and fault-tolerant computation anyhow, and this is done by such a four-step process, why not make other use of this architecture? Moreover, on some of the quantum hardware platforms, these operations are already in place.
Embracing this idea we naturally arrive at the question: what is the computational power of \textit{low-depth} quantum-classical circuits organized as in the four steps outlined above? 
We thus investigate circuits that execute a small, ideally constant, number of stages, where at each stage we may apply, in parallel, single qubit gates and {\em nearest-neighbor} two qubit gates, followed by measurements, followed by low-depth classical computations of which the outcome can control quantum gates in later stages. 
It is not clear, at first, whether such circuits, especially with constant depth, can do anything remotely useful. 
But we will see that this is indeed the case: many quantum computations can be done by such circuits in constant depth. 
By parallelizing quantum computations in this way, we improve the overall computational capabilities of these circuits, as we do not incur errors on qubits that are idle, simply because qubits are not idle for a very long time. 
Furthermore, reducing the depth of quantum circuits, at the cost of increasing width, allows the circuit to be run faster even if errors occur.

The first usage of such a four-step layout, not to do error correction, but to perform computations, can be found in the paradigm of measurement-based quantum computing~\cite{gottesman1999demonstrating,raussendorf2001one,jozsa2006introduction,clark2007generalised}: 
A universal form of quantum computing where a quantum state is prepared and operations are performed by measuring qubits in different bases, depending on previous measurements and intermediate measurements.

\citeauthor{PhamSvore2013} were the first to formalize the four-step protocol for performing computations~\cite{PhamSvore2013}. They included specific hardware topologies by considering two-dimensional graphs for imposing constraints on qubit interactions. In their model, they develop circuits for particularly useful multi-qubit gates, including specifying costs in the width, number of qubits, depth, number of concurrent time steps, size, and total number of non-Identity operations.
As a result, they find an algorithm that factors integers in polylogarithmic depth.
\citeauthor{Browne:2011} showed that the main tool in the work by \citeauthor{PhamSvore2013}, the fan-out gate, can also be replaced by additional log-depth classical computations in the measurement-based quantum computing setting~\cite{Browne:2011}.

More recently, \citeauthor{Cirac:2021} introduced a scheme to implement unitary operations involving quantum circuits combined with Local Operations and Classical Communication ($\mathsf{LOCC}$) channels: $\mathsf{LOCC}$-assisted quantum circuits~\cite{Cirac:2021}. Similarly to the four-step scheme we just described, they allow for a short depth circuit to be run on the qubits, followed by one round of $\mathsf{LOCC}$, in which ancilla qubits are measured and local unitaries are applied based on the measurement outcomes. They show that in this model any 1D transitionally invariant matrix-product state (MPS) with fixed bond dimension is in the same phase of matter as the trivial state. Similar ideas can be found in~\cite{TVV_NonAbelianTopologicalOrder_2022, tantivasadakarn2021long}.

In this work, we introduce a new model, called \textit{Local Alternating Quantum-Classical Computations} ($\LAQCC$). In this model we alternate between running quantum circuits (constrained by locality), ending in the measurement of a subset of qubits, and fast classical computations based on the measurement results. The outcome of the classical computations are then used to control future quantum circuits. We allow for flexibility in this model, by giving different constraints to the power of both the quantum circuits and the classical circuits as well as the number of alternations between them. 
Most attention will be given to $\LAQCC$ containing quantum circuits of constant depth, classical circuits of logarithmic depth and at most a constant number of alternations between them. 
Any circuit constructed in this model is considered to be of constant depth. 
We restrict ourselves to logarithmic depth classical computations, as this is the first natural and non-trivial extension beyond constant-depth classical computations. 
Constant-depth classical computations do however also have an equivalent constant-depth quantum implementation.

The definition of $\LAQCC$ sharpens the original definition of \citeauthor{PhamSvore2013} by adding constraints to the intermediate classical computations. This allows us to bound the power of $\LAQCC$ from above. 

The main result of \citeauthor{Cirac:2021}, that 1D translational invariant MPS with fixed bond dimension can be prepared by $\mathsf{LOCC}$-assisted circuits, relies on local symmetries of the MPS. These symmetries allow them to prepare local states (on a constant number of qubits) and glue them together by doing one round of the appropriate entangling measurement and corrections, after which they run a round of local unitaries to get the desired result. This general scheme for preparing states that exhibit an MPS description with the appropriate local symmetries requires only geometrically local unitaries and one round of measurement and corrections an therefore is accessible in $\LAQCC$. Studying different local symmetries, known as Symmetry Protected Topological (SPT) phases of matter, to find measurement-based constant depth circuits for states is a broad ongoing field of research~\cite{TVV_NonAbelianTopologicalOrder_2022, tantivasadakarn2021long, smith2023deterministic}. 
All these schemes have a $\LAQCC$ implementation.

%$\LAQCC$-circuits also exist for general schemes of preparing local states, based on the local tensors, and gluing them together using one round of entangled measurement and corrections, based on the local symmetry. 
%The main result of \citeauthor{Cirac:2021}, that 1D translational invariant MPS with fixed bond dimension can be prepared by $\mathsf{LOCC}$-assisted circuits, relies heavily on local symmetries of the MPS and as a result also has an equivalent $\LAQCC$ implementation. 
%The corrections applied after the measurement round are local unitaries depending on the local symmetries of the MPS. 

 

%This general scheme of preparing local states, based on the local tensors, and gluing it together by doing one round of entangled measurement and corrections, based on the local symmetry, is accessible in $\LAQCC$.
Note however that \citeauthor{Cirac:2021} also suggest a circuit for the $W$-state.
This circuit uses sequentially and dependent measurement-based corrections of the ancilla qubits. 
These dependent measurements translate to sequential alternations between the quantum and classical circuits and therefore increase the total depth to linear depth, exceeding the constant-depth constraints imposed by $\LAQCC$-circuits. 

We study the power of the $\LAQCC$ model with respect to state preparation, showing that even with only constant quantum-depth and logarithmic classical depth it remains possible to prepare states with long-range entanglement.
Another surprising result is that it is unlikely that $\LAQCC$ circuits are classically simulatable. We show that any instantaneous quantum polynomial-time (IQP) circuit~\cite{Bremner2010,Shepherd2009} has an $\LAQCC$ implementation.
Classical simulation of IQP circuits implies the collapse of the polynomial hierarchy to the third level, which is not believed to be true~\cite{Bremner2017}. Therefore, we expect that $\LAQCC$ circuits are unlikely to be classically simulatable. We bound the power of $\LAQCC$ by showing that it is contained in $\QNC^1$, the class of polynomial-size, log-depth circuits.

Next, we also study the power that intermediate classical calculations can add to quantum computations, by considering a new model that alternates between polynomially many polynomial-depth quantum circuits and unbounded classical computations
We study this model by doing a complexity theoretical analysis, where we draw inspiration from the notions of complexity given by \citeauthor{RosenthalYuen:2022}, \citeauthor{MetgerYuen:2023}, and \citeauthor{Aaronson:2004}.
All three complexity notions are based on the notion of state preparation, instead of more traditional definition of complexity such as the decidability of a computational problem. 
The first two consider classes based on sequences of quantum states preparable by a polynomial-sized quantum circuit, where the circuits are uniformly generated by a computational class, for instance, the class $\mathsf{PSPACE}$, which results in the complexity class $\mathsf{StatePSPACE}$~\cite{RosenthalYuen:2022,MetgerYuen:2023}.
The third notion considers a relative complexity, where the complexity is measured between two given states, and is measured by the number of gates, from a given gate-set, required to transform one state in another state~\cite{Aaronson:2004}. 
For our definition of state preparation complexity, we drop the uniformity constraint from~\cite{RosenthalYuen:2022,MetgerYuen:2023} and define a class as $\mathsf{StateX}$, which refers to states preparable by circuits of type $\mathsf{X}$. 
As an example, if $\mathsf{X} = \QNC^0$, this results in the class $\mathsf{StateQNC^0}$, which is the set of states preparable from the $\ket{0}^n$ state by poly-size constant-depth circuits. 
This notion is similar to the relative complexity from~\cite{Aaronson:2004}, where one state is the  $\ket{0}^n$ state and instead of counting the number of gates we consider the set of states preparable by a fixed number of gates. Using this notion of complexity we show that any state preparable by an $\LAQCC^*$ circuit is also preparable by a $\mathsf{PostQPoly}$ circuit, the class of circuits of polynomial depth with an additional post-selection gate. 

All Clifford circuits have a constant-depth $\LAQCC$ implementation, implying that any stabilizer state can be implemented by a constant-depth $\LAQCC$ circuit, see Section~\ref{sec:clifford_circuits} for a proof of this statement. 
Efficient circuits for stabilizer states have been known already through measurement-based quantum computing. Therefore this paper focuses on the preparation of non-stabilizer states, and as a surprising result we find novel constant-depth protocols for four very natural classes of non-stabilizer states.
Despite the extensive research into these four classes of non-stabilizer states and the many applications of them, no efficient constant- or low-depth state preparation protocols are known yet. We specifically consider these four classes as they are all often used as initial states in other algorithms.

The first state is a uniform superposition over an arbitrary number of states. 
This state finds applications in many quantum algorithms, as they often start with a uniform superposition over multiple states. 
This superposition is often achieved by applying Hadamard gates to every qubit due to its simplicity to prepare. 
Yet, the analysis of many algorithms, such as Shor's algorithm~\cite{Shor:1997}, would benefit from a different initial superposition. 
The circuit to prepare the uniform superposition over an arbitrary number of states uses an exact version of Grover search as a subroutine, that turns a probabilistic circuit, with a known constant probability of success, into a deterministic circuit. 
We use the circuit for preparing a uniform superposition over an arbitrary number of states as a subroutine in the next two quantum state preparation protocols. 

The second state is the $W$-state, the uniform superposition over all computational basis states of Hamming-weight~$1$, a natural long-ranged entangled state that displays a fundamentally nonequivalent type of entanglement from the Greenberger–Horne–Zeilinger state~\cite{WState:2000}, for which $\LAQCC$-type constant-depth circuits were previously known~\cite{PhamSvore2013, Cirac:2021}. 
The $W$-state is often used as benchmark for new quantum hardware~\cite{Haffner2005,Neeley2010,GarciaPerez:2021}. 
A novel way to prepare the $W$-state therefore gives a new way to benchmark different quantum devices with each other. 
A circuit for preparing the $W$-state was given in~\cite{Cirac:2021}, but this implementation requires sequentially alternating measurements followed by local unitaries, which in the $\LAQCC$ model is not considered to be of constant depth. 
We improve this protocol by giving an $\LAQCC$ implementation of the $W$-state, based on a compress-uncompress method that links the one-hot and binary encoding of integers.

The third state considered is the Dicke state, a generalization of the $W$-state, a superposition over all computational basis states with Hamming-weight $k$~\cite{Dicke:1954}. 
Dicke states have relevance in various practical settings.
For instance, for quantum game theory~\cite{zdemir2007}, quantum storage~\cite{Bacon_Compress:2006,Plesch:2010}, quantum error correction~\cite{ouyang2014permutation}, quantum metrology~\cite{toth2012multipartite}, and quantum networking~\cite{prevedel2009experimental}. 
Dicke states have been used as a starting state for variational optimization algorithms, most notably Quantum Alternating Operator Ansatz (QAOA)~\cite{Hadfield2019}, to find solutions to problems such as Maximum k-vertex Cover~\cite{Brandhofer2022,cook2020quantum}.
The ground states of physical Hamiltonians describing one-dimensional chains tend to show a resemblance to Dicke states such as states resulting from the Bethe ansatz, making them an ideal starting state when investigating the ground state behavior of these Hamiltonians~\cite{TDL_BetheAnsatzDerivation:2010,B_ExcitedStateQuantumPhaseTransitions:2013,DickeTransitions:2021}. 
For instance, the algorithm by \citeauthor{van2021preparing}, who give an algorithm to prepare the Bethe ansatz eigenstates of the spin-1/2 XXZ spin chain, starts by first preparing a Dicke state~\cite{van2021preparing}. 
A Dicke-state preparation protocol based on the compress-uncompress methodology used in the $W$-state furthermore finds applications in entanglement distillation, where the entanglement of a large state is concentrated on only a few qubits. 
Efficient deterministic circuits for preparing Dicke states have been proposed by \citeauthor{bartschi2019deterministic}~\cite{bartschi2019deterministic, bartschi2022deterministic_short_depth}. 
They provide a quantum circuit of depth $\mathO(k \log(\frac{n}{k}))$, allowing arbitrary connectivity, to prepare a Dicke state, which they conjecture to be optimal when $k$ is constant. 
In this work, we provide a constant-depth $\LAQCC$ circuit below their conjectured bound already for constant $k$. 
However, this does not directly disprove their conjecture, as we allow for intermediate measurements and classical computations. 
More significantly, we even construct constant-depth $\LAQCC$ circuits for $k = \mathO(\sqrt{n})$ greatly improving their bound.
This construction extends the compress-uncompress method for the $W$-state combined with additional subroutines. 

We continue with a log-depth state preparation protocol for the Dicke-state for arbitrary $k$. 
This protocol implements an efficient transformation between the factoradic number representation and the combinatorial number representation of a positive integer. 
The combinatorial number representation relates directly to the Dicke state. 
The provided efficient transformation between number representation systems might be of independent interest. 

We conclude by modifying our protocol for preparing a Dicke-state to a protocol that prepares quantum many-body scar states in constant-depth. 
These states have low entanglement and longer coherence times than states with similar energy density.
These characteristics make many-body scar states interesting to analyze and relevant within physics.
Many-body scar states appear for instance in the AKLT model~\cite{AKLT:1987,MRBAR:2018,MRB:2018} and different spin models~\cite{SI:2019,MOBFR:2020}.
Known methods for preparing these states have polynomial-depth~\cite{Gustafson:2023}, whereas our circuit has constant depth. 

% We conclude by studying the power that intermediate classical calculations can add to quantum computations. 
% In this study, we define a new model that relaxes constant-depth quantum circuits to polynomial depth quantum circuits, log-depth classical calculations to unbounded classical computations and a constant number of alternations to a polynomial number of alternations. 
% We call this model $\LAQCC^*$. 
% We study this model by doing a complexity theoretical analysis, where we draw inspiration from the notions of complexity given by \citeauthor{RosenthalYuen:2022}, \citeauthor{MetgerYuen:2023}, and \citeauthor{Aaronson:2004}.
% All three complexity notions are based on the notion of state preparation, instead of more traditional definition of complexity such as the decidability of a computational problem. 
% The first two consider classes based on sequences of quantum states preparable by a polynomial-sized quantum circuit, where the circuits are uniformly generated by a computational class, for instance, the class $\mathsf{PSPACE}$, which results in the complexity class $\mathsf{StatePSPACE}$~\cite{RosenthalYuen:2022,MetgerYuen:2023}.
% The third notion considers a relative complexity, where the complexity is measured between two given states, and is measured by the number of gates, from a given gate-set, required to transform one state in another state~\cite{Aaronson:2004}. 
% For our definition of state preparation complexity, we drop the uniformity constraint from~\cite{RosenthalYuen:2022,MetgerYuen:2023} and define a class as $\mathsf{StateX}$, which refers to states preparable by circuits of type $\mathsf{X}$. 
% As an example, if $\mathsf{X} = \QNC^0$, this results in the class $\mathsf{StateQNC^0}$, which is the set of states preparable from the $\ket{0}^n$ state by poly-size constant-depth circuits. 
% This notion is similar to the relative complexity from~\cite{Aaronson:2004}, where one state is the  $\ket{0}^n$ state and instead of counting the number of gates we consider the set of states preparable by a fixed number of gates. Using this notion of complexity we show that any state preparable by an $\LAQCC^*$ circuit is also preparable by a $\mathsf{PostQPoly}$ circuit, the class of circuits of polynomial depth with an additional post-selection gate. 

\paragraph{Summary of results}
\begin{itemize}
    \item We give a new definition of a computational model that captures the power of the four step process: applying a constant number of layers of one- and two-qubit gates; performing a syndrome measurement; perform a fast classical computation determining corrections; apply corrections. We call this model \emph{Local Alternating Quantum Classical Computations}, or $\LAQCC$ for short. In this model we bound the allowed quantum operations, intermediate classical calculations, and number of rounds separately. In Section~\ref{sec:LAQCC_model} we define this model and give a list of operations based on results from literature contained in this computational model. In some of these operations we explicitly use that we allow for multiple, but at most constant, rounds  of corrections.
    \item  We show show that there exist $\LAQCC$ circuits that can not be weakly simulated in Section~\ref{sec:IQP_in_LAQCC}. We further show that for every $\LAQCC$ circuit there exists a $\QNC^1$ circuit simulating it perfectly, in Section~\ref{sec:LAQCC_in_QNC1}.
    \item We introduce a new type computational complexity for preparing states and show that the extension of $\LAQCC$ where we allow a polynomial number of rounds and unbounded classical computation, is contained in $\mathsf{PostQPoly}$, the class of polynomial circuits with post-selection, in Section~\ref{sec:Complexity results}.
    \item We show a protocol to prepare the uniform superposition state of size $q$ in $\LAQCC$ using $\mathO(\ceil{\log_2(q)}^2)$ qubits in Section~\ref{sec:superposition_modulo_q}. 
    \item We show a protocol to prepare the $W_n$ state in $\LAQCC$ using $\mathO(n\log(n))$ qubits in Section~\ref{sec:W_state_in_LAQCC}.
    \item We show two ways of preparing the Dicke-$(n,k)$ state. The first method is in $\LAQCC$, works up to $k = \mathO(\sqrt{n})$, uses $\mathO(n^2\log(n))$ qubits, and is found in Section~\ref{sec:dicke:small_k}. The second method is in $\LAQCC\text{-}\mathsf{LOG}$ (an extension of $\LAQCC$ allowing for logarithmic number of alterations instead of constant), works for any $k$, uses $\mathO(\text{poly}(n))$ qubits, and is found in Section~\ref{sec:Dicke_in_LAQCC_LOG}. 
    \item We extend on our $\LAQCC$ method of generating Dicke-$(n,k)$ states for $k = \mathO(\sqrt{n})$ and show a protocol to generate many-body scar states for a particular Hamiltonian in $\LAQCC$ (Section~\ref{sec:many_body_scar}). 
\end{itemize}
Summarized in a table, we provide the following state generation protocols:
\begin{table}[htb]
\centering
\begin{tabular}{l|l|l|l}
\textbf{State description} & \textbf{Width} & \textbf{Depth} & \textbf{Implementation}\\
\hline 
Uniform superposition mod $q$: $\frac{1}{\sqrt{q}} \sum_{i = 0}^{q-1}\ket{i}$ & $\mathO(\ceil{\log^2 q})$ & $\mathO(1)$ & Section~\ref{sec:superposition_modulo_q}\\

$W$-state: $\frac{1}{\sqrt{n}}\sum_{i = 0}^{n-1}\ket{e_i}$ & $\mathO(n \log n)$ & $\mathO(1)$ & Section~\ref{sec:W_state_in_LAQCC}\\

Dicke-$(n,k)$, $k = \mathO(\sqrt{n})$: $\binom{n}{k}^{-1/2}\sum_{x \in \{0,1\}^n: |x| = k} \ket{x}$ &  $\mathO(n^2\log n)$ & $\mathO(1)$ 
&Section~\ref{sec:dicke:small_k}\\

Dicke-$(n,k)$: $\binom{n}{k}^{-1/2}\sum_{x \in \{0,1\}^n: |x| = k} \ket{x}$ & $\mathO(\text{poly}(n))$ & $\mathO(\log n)$ &Section~\ref{sec:Dicke_in_LAQCC_LOG}\\

QMBS: $\ket{S_k} = \frac{1}{k! \sqrt{\mathcal N(n,k)}}(Q^\dagger)^k \ket{\Omega}$ &  $\mathO(n^2\log n)$ & $\mathO(1)$  &  Section~\ref{sec:many_body_scar}
\end{tabular}
\caption{Summary of state preparation protocols given in this paper.}
\label{tab:sate_prep}
\end{table}
In the entry for the quantum many-body scar state $Q$ denotes the raising operator and $\mathcal N(n,k)=\binom{n-k-1}{k}$. 
Section~\ref{sec:many_body_scar} will provide more details on the variables and the implementation. 

\paragraph{Organization of the paper}
\noindent We first introduce relevant preliminaries in Section~\ref{sec:preliminaries}. 
In Section~\ref{sec:LAQCC_model} we formally define the class of Local Alternating Quantum-Classical Computations ($\LAQCC$). We also show that any Clifford circuit can be implemented in constant depth $\LAQCC$ (a result based on a result from measurement-based quantum computing~\cite{jozsa2006introduction}). 
This result allows us to give many useful multi-qubit gates and routines in Section~\ref{sec:gates_created_in_LAQCC}. 
Beyond that we show that constant depth $\LAQCC$ circuits are contained in $\QNC^1$ and that any $\mathsf{IQP}$ circuit has an $\LAQCC$ implementation.
We conclude this section with an analysis of a more powerful instantiation of $\LAQCC$ and show an inclusion with respect to the class $\mathsf{PostQPoly}$, which is the class of circuits of polynomial depth with one additional post-selection gate. 
In Section~\ref{sec:state_prep_in_LAQCC} we give $\LAQCC$ circuit implementations for preparing the uniform superposition over an arbitrary number of states, the $W$-state and the Dicke state up to $k = \mathO(\sqrt{n})$. We furthermore give a log-depth circuit implementation for preparing the Dicke state for any $k$. We conclude by showing a $\LAQCC$ circuit for generating many body scar states of a particular type of Hamiltonian.


\section{Related Work}
%\subsection{Cost Volume based Deep Stereo Matching}
%Stereo matching is a typical problem that has been studied for decades and a well-known four-step pipeline \cite{scharstein2002taxonomy} has been established, where cost volume construction is an indispensable step. Current state-of-the-art stereo matching methods are all cost volume based methods and they can be categorized into two types. Typically, a cost volume is a 4D tensor of height, width, disparity, and features. The first category just uses a full correlation to generate a single-feature cost volume. Such methods are usually efficient but lose much information because of the decimation of feature channels. Many previous work, including Dispnet \cite{dispnet}, MADNet \cite{madnet}, IResNet \cite{iresnet} and AANet \cite{aanet}, belong to this category. The second category usually uses concatenation \cite{gcnet} or group-wise correlation \cite{gwcnet} to generate a multi-feature 4D cost volume. Such a method can achieve better performance while requiring higher computational complexity and memory consumption. Actually, a majority of the top-performing networks in public leaderboards belong to this category, such as GANet \cite{ganet}, CSPN \cite{cspn} and ACFNet \cite{acfnet}. These methods generally employ multiple 3D convolution layers to constantly regularize the 4D cost volume and then apply softmax over the disparity dimension to produce a discrete disparity probability distribution. The final predicted disparity is obtained by softly weighting indices according to their probability, which is also called soft argmin in GCNet \cite{gcnet}. However, soft argmin leaves the output susceptible to multi-modal disparity probability distributions. ACFNet \cite{acfnet} observes this problem and proposes to directly supervise the cost volume with unimodal ground truth distributions. In contrast, we define an uncertainty estimation to quantify the degree to which the cost volume tends to be multi-modal distribution, higher implies the higher possibility of estimation error.

\subsection{Multi-scale Cost Volume based Stereo Matching}
Cost volume construction is an indispensable step in the well-known four-step pipeline for stereo matching \cite{scharstein2002taxonomy, pamisurvey1, pamisurvey2}. Typically, current state-of-the-art stereo matching methods can be categorized into two types of cost volume-based methods, where the cost volume is a 4D tensor of height, width, disparity, and features. The first category usually uses the single-feature 3D cost volume generated by full correlation, which is efficient while losing much information due to the decimation of feature channels. Many real-time methods, such as Dispnet \cite{dispnet}, MADNet \cite{madnet, madnet_pami} and AANet \cite{aanet}, belongs to the category. Moreover, two-stage refinement \cite{mcvmfc} and pyramidal towers \cite{madnet} are commonly applied in the single-feature cost volume based network to construct multi-scale cost volume. The second category usually uses the multi-feature 4D cost volume generated by concatenation \cite{gcnet} or group-wise correlation \cite{gwcnet}, which can achieve better performance with higher computational complexity and memory consumption. Most top-performing networks, including GANet \cite{ganet}, CSPN \cite{cspn} and ACFNet \cite{acfnet} belong to this category. 
% In these methods, the 4D cost volume is constantly regularized by multiple 3D convolution layers and then a discrete disparity probability distribution can be produced by softmax. Next, the final predicted disparity can be obtained by softly weighting indices according to their probability \cite{gcnet}. However, such output is susceptible to multimodal disparity probability distributions and ACFNet \cite{acfnet} gives a solution by directly supervising the cost volume with unimodal ground truth distributions to alleviate this problem. 
Recently, to alleviate the high computational complexity and memory consumption when employing multi-feature 4D cost volumes, \cite{cvpmvsnet, cascade, uscnet} propose to use cascade cost volume representation in multi-view stereo. These methods usually first predict an initial disparity at the coarsest resolution of the image and then gradually refine the disparity by narrowing down the disparity search space. More closely related to our approach is Casstereo \cite{cascade}, which first extended such representation to stereo matching. It selected to uniform sample a pre-defined range to generate the next stage’s disparity search range. Instead, we employ pixel-level uncertainty estimation to adaptively adjust the next stage disparity searching range and generate pseudo-labels for subsequent domain adaptation. Our method also shares similarities with UCSNet \cite{uscnet}, which constructs uncertainty-aware cost volume in multi-view stereo while it doesn’t employ uncertainty estimation to generate pseudo-labels.

%\subsection{Multi-scale Cost Volume based Deep Stereo Matching} 
% \subsection{Multi-scale Cost Volume based Stereo Matching} 
%Multi-scale cost volume firstly was applied in the single-feature cost volume based network with the form of two-stage refinement \cite{mcvmfc} and pyramidal towers \cite{madnet}. Recently, cascade cost volume representation \cite{cvpmvsnet, cascade, uscnet} was proposed in multi-view stereo to alleviate the high computational complexity and memory consumption when employing multi-feature 4D cost volumes. These methods generally predict an initial disparity at the coarsest resolution of the image. Then, they will narrow down the disparity search space and gradually refine the disparity. More closely related to our approach is Casstereo \cite{cascade}, which first extended such representation to stereo matching. It selected to uniform sample a pre-defined range to generate the next stage’s disparity search range. Instead, we employ uncertainty estimation to adaptively adjust the next stage pixel-level disparity searching range and push the next stage's cost volume to be predominantly unimodal.

% The single-feature cost volume based network with the form of two-stage refinement \cite{mcvmfc} and pyramidal towers \cite{madnet} first employ multi-scale cost volume for stereo matching. Recently, to alleviate the high computational complexity and memory consumption when employing multi-feature 4D cost volumes, \cite{cvpmvsnet, cascade, uscnet} propose to use cascade cost volume representation in multi-view stereo, which generally predict an initial disparity at the coarsest resolution of the image. Then, the disparity search space is narrowed down and the disparity is gradually refined. More closely related to our approach is Casstereo \cite{cascade}, which first extended such representation to stereo matching. It selected to uniform sample a pre-defined range to generate the next stage’s disparity search range. Instead, we employ uncertainty estimation to adaptively adjust the next stage pixel-level disparity searching range and push the next stage's cost volume to be predominantly unimodal.

% Figure environment removed

\subsection{Robust Stereo Matching} 
There exist three categories of generalization definitions for robust stereo matching. 1) Cross-domain Generalization: the network’s ability to perform well on unseen scenes (cannot see the image pairs of the target domain in advance). Towards this end, Jia et al \cite{sungeneralizaiton} propose to incorporate scene geometry priors into an end-to-end network. Zhang et al \cite{dsmnet} introduce a domain normalization and a trainable non-local graph-based filter to construct a domain-invariant stereo matching network. 2) Adapt Generalization: the network’s ability to adapt pre-trained models to the new domain with unlabeled target data. Previous work usually pre-trains the models on synthetic data and then adapts it to new target domains with Graph Laplacian regularization \cite{zoom}, non-adversarial progressive color transfer \cite{adastereo}, and Knowledge Reverse Distillation \cite{aohnet}. More closely related to our approach are \cite{aohnet, unsuperviseddomainadaptation} in stereo matching and Monoresmatch \cite{monoresmatch} in monocular depth estimation, which also proposes to generate a pseudo-label for domain adaptation. However, these methods all select to employ classical stereo matching methods \cite{sgm} alongside with confidence estimators, e.g., left-right consistency check to generate pseudo-labels. That is all these methods need an independent method to generate corresponding pseudo-labels. Instead, the proposed method is an end-to-end network that can generate the predicted disparity map, corresponding uncertainty map and pseudo-labels jointly, which is a more simple, yet efficient way. 
% Instead, our proposed method can employ pixel-level and area-level uncertainty estimation to self-distill the predicted disparity maps of our pre-training model and generate sparse while reliable pseudo-labels to align the domain gap, which is a more simple, yet efficient way. 
3) Joint Generalization: the network’s ability to perform well on a variety of datasets with the same model parameters. MCV-MFC \cite{mcvmfc} introduces a two-stage finetuning scheme to achieve a good trade-off between generalization and fitting capability on multiple datasets. However, it doesn’t touch the inner difference between diverse datasets, e.g, the unbalanced disparity distribution. To further address this problem, we propose a cascade cost volume to adaptively the next stage disparity searching space, where the pixel-level uncertainty estimation is at the core.

% \subsection{Monocular Depth Estimation}
% Monocular depth estimation aims to estimate depth values from a single image, instead of stereo images or multiple frames in a video. This problem is ill-posed because of the ambiguity of object sizes. However, humans could estimate the depth from a single image with prior knowledge of the scenes. Recently, learning based methods were explored to learn depth values by supervised or unsupervised learning. Eigen et al. first employed Convolutional Neural Networks (CNN) to predict depth in a coarse-to-fine manner and further improved its performance by multi-task learning. Liu et al. presented deep convolutional neural fields model by combining deep model with continuous CRF. Li et al. [22] refined deep CNN outputs with a hierarchical CRF. Multi-scale continuous CRF was formulated into a deep sequential network by Xu et al. [45] to refine depth estimation. Unsupervised methods tried to train monocular depth estimation with stereo
% image pairs or image sequences and test on single images. Garg et al. [9] used novel image view synthesis loss to train a depth estimation network in an unsupervised way. Godard et al. [11] introduced left-right consistency regularization to improve the performance of view synthesis loss. Recently, some work also propose to use the stereo matching network as a proxy to learn depth from synthetic data or directly employ traditional stereo matching methods to distill proxies labels from the target domain, which proves the feasibility of distilling stereo matching networks to learn monocular depth estimation.



\section{Method} \label{method_hybridaugment}
In this section, we formally define the problem, motivate our work and then present our proposed techniques.


\subsection{Preliminaries}
Let $\mathcal{F}(x;W)$ be an image classification CNN trained on the training set $\mathcal{T}_\text{train} = (x_{i}, y_{i})^{N}_{i=1}$  with $N$ samples, where $x$ and $y$ correspond to images and labels. The clean accuracy (CA) of $\mathcal{F}(x;W)$ is formally defined as its accuracy over a clean test set $\mathcal{T}_\text{test} = (x_{j}, y_{j})^{M}_{j=1}$. Assume two operators ${A}(\cdot)$ and ${C}(c, s)$ that adversarially attacks or corrupts a given set of images with the corruption category $c$ and severity $s$, respectively.  Let $A\mathcal{T}_\text{test}$ and $C\mathcal{T}_\text{test}$ be the adversarially attacked and corrupted versions of $\mathcal{T}_\text{test}$, and let $\mathcal{F}(x;W)$ have a robust accuracy (RA) on $A\mathcal{T}_\text{test}$ and a corruption accuracy (CRA) on $C\mathcal{T}_\text{test}$. 
The aim is to fit $\mathcal{F}(x;W)$ such that the model gains robustness (\ie. increased RA and CRA compared its the baseline version), while retaining (or improving) the clean accuracy of its baseline version trained without robustness concerns.


\noindent \textbf{What we know.} Our work builds on the following crucial observations: i) CNNs favour high-frequency content \cite{wang2020high}, ii) adversaries and corruptions often reside in high-frequency \cite{wang2020towards}, iii) images are dominated by low-frequency \cite{Saikia_2021_ICCV} and iv) models relying on low-frequency components are more robust \cite{li2022robust,wang2020towards}. The robustness-accuracy trade-off is visible; low-frequency reliant models are more robust, but tend to miss out on clean accuracy brought by the high-frequency components. 

\subsection{HybridAugment}
We hypothesize that a \textit{sweet spot} in the robustness-accuracy trade-off can be found. Unlike the \textit{hard} approaches that completely rule out the reliance on high-frequency components (i.e. low-pass filters), we propose to \textit{reduce} the reliance on them. To this end, we adopt a data augmentation approach that aims to diversify $\mathcal{T}_\text{train}$ by an operation $\mathcal{HA(\cdot)}$. Keeping the strong relation intact between labels and low-frequency content (i.e. labels come from low-frequency-component image), we propose to swap high and low-frequency components of images in a batch on-the-fly. Unlike \cite{mukai2022improving}, we \textit{do not} restrict the images to belong to the same class; this diversifies the training distribution even further while preserving the image semantics. We call this basic version of our approach \textit{HybridAugment}, which corresponds to: 
%
\begin{equation} \label{hybrid_augment_paired}
    \mathcal{HA_{P}}(x_{i}, x_{j}) = \mathcal{LF}(x_{i}) + \mathcal{HF}(x_{j})
\end{equation}
%
where $x_{i}$ is the input image and $x_{j}$ is a randomly sampled image from the whole training set, which we simply sample from the mini batch at each training iteration in practice. $\mathcal{HF}$ and $\mathcal{LF}$ operators select the high and low-frequency components of an input image, for which we use:
%
\begin{equation} \label{eq:cutoff}
\begin{split}
    \mathcal{LF}(x) = GaussBlur(x) \\
    \mathcal{HF}(x) = x - \mathcal{LF}(x)
    \end{split}
\end{equation}
%
where $GaussBlur$ is used as a low-pass filter. Note that a similar outcome is possible by using Discrete Fourier Transforms (DFT), swapping the frequency bands and then applying Inverse DFT (IDFT). We find the gaussian blur operation to be faster and better in practice. 


Inspired from \cite{chen2021amplitude}, in addition to the image-pair scheme in Eq.~\ref{hybrid_augment_paired}, we propose a single image variant of \textit{HybridAugment}. In the single image variant, instead of combining two images, $x_i$ and $x_{j}$ are obtained by applying randomly sampled augmentations to a single image. The single image variant $\mathcal{HA_{S}}$ can therefore be defined as 
%
\begin{equation} \label{hybrid_augment_single}
    \mathcal{HA_{S}}(x_{i}) = \mathcal{LF}(Aug(x_{i})) + \mathcal{HF}(\hat{Aug}(x_{i}))
\end{equation}
%
where $Aug$ and $\hat{Aug}$ correspond to two sets of randomly sampled augmentation operations. Note that paired and single versions can work in tandem ($\mathcal{HA_{PS}}$), and actually outperform single or paired image versions. 


\subsection{HybridAugment++}


The frequency analysis is a vast literature, however, two core aspects often stand out; frequency-band analysis (i.e. low, high) and the decomposition of signals into amplitude and phase. \textit{HybridAugment} covers the former and shows competitive results in various benchmarks (see Section \ref{sec:exp_hybridaugment}). The latter is investigated in $\mathcal{APR}$ \cite{chen2021amplitude}, where phase is shown to be the more relevant component for correct classification, and training models based on their phase labels and swapping amplitude components of images randomly lead to more robust models. Note that frequency-band and phase/amplitude discussions are arguably orthogonal, since frequency, phase and amplitude provide distinct characterizations of a signal: intuitively speaking, frequency, phase and amplitude can be seen as the separation of visual patterns in terms of scale, location and significance. 


We hypothesize these two approaches can be complementary; a model reliant on low-frequency and spatial information (i.e. phase) can further improve robustness. Inspired by the successes of cascaded augmentation methods \cite{hendrycks2019augmix,wang2021augmax,calian2022defending}, we unify these two core aspects into a single, hierarchical augmentation method. We refer to this method as \textit{HybridAugment++} and define its paired version as:
%
\begin{equation}
  \mathcal{HA_{P}}^{++}(x_{i}, x_{j}, x_{z}) = \mathcal{APR_{P}}(\mathcal{LF}(x_{i}), x_{z}) + \mathcal{HF}(x_{j})
\end{equation}
%
where $x_{i}$, $x_{j}$ and $x_{z}$ are images sampled from the same batch. Here, $\mathcal{APR_{P}}$~\cite{chen2021amplitude} is defined as
\begin{equation}
    \mathcal{APR_{P}}(x_{i}, x_{z}) = \mathcal{IDFT}(A_{x_{z}} \otimes e^{i. P_{x_{i}}}) \\
\end{equation}
%
where $\otimes$ is element-wise multiplication, $A$ is the amplitude and $P$ is the phase component. Similar to $\mathcal{HA}$ and $\mathcal{APR}$, we also define a single-image version of \textit{HybridAugment++} as
%
\begin{equation}
 \mathcal{HA_{S}}^{++}(x_{i}) = \mathcal{APR_{S}}(\mathcal{LF}(Aug(x_{i}))) + \mathcal{HF}(\hat{Aug}(x_{i}))
\end{equation}
%
where $\mathcal{APR_{S}}$~\cite{chen2021amplitude} is defined as
%
\begin{equation}
\mathcal{APR_{S}}(x_{i}) = \mathcal{IDFT}\left(A_{\bar{Aug}(x_{i})} \otimes e^{i. P_{\overline{Aug}\left(x_{i}\right)}}\right)    
\end{equation}
%
where $Aug$, $\hat{Aug}$, $\bar{Aug}$ and $\overline{Aug}$ are different sets of randomly sampled augmentation operations. Note that we essentially propose a framework; one can use different single and paired image augmentations, either individually or together, and can still achieve competitive results (see ablations in Section \ref{sec:exp_hybridaugment}). There are also other alternatives, such as swapping phase/amplitude first and then performing $\mathcal{HA}$, but we observe poor performance in practice; dividing the phase component into frequency-bands is not interpretable as frequencies of the phase component are not well defined. The pseudo-code of our methods can be found in the supplementary material.





\section{Experiment}

\subsection{Datasets and metrics}

% \noindent\textbf{Dataset.}
\subsubsection{Dataset}
% We adopt three datasets in our experiments, i.e., ClearGrasp \cite{sajjan2020clear}, TransCG \cite{fang2022transcg} and ClearPose \cite{chen2022clearpose}. The ClearGrasp dataset is the pioneering large-scale synthetic dataset that specifically focused on transparent objects. It provids a large-scale synthetic dataset as well as a real-world benchmark. The TransCG dataset comprises 57K RGB-D images from 130 different real-world scenes. 
% ClearPose dataset contains 350K RGB-D images of 63 household objects in real-world settings. Depth completion experiments and generalization verification (reported respectively in Section \ref{sec:depth} and \ref{sec:generalization}) are conducted on ClearGrasp, TransCG and ClearPose. Ablation study (reported in Section \ref{sec:ablation}) is performed on TransCG.
We use three datasets including ClearGrasp \cite{sajjan2020clear}, TransCG \cite{fang2022transcg}, and ClearPose \cite{chen2022clearpose}. The ClearGrasp dataset is a pioneering large-scale synthetic dataset that specifically focuses on transparent objects. It provides a large-scale synthetic dataset as well as a real-world benchmark. The TransCG dataset comprises 57K RGB-D images from 130 different real-world scenes. The ClearPose dataset contains 350K RGB-D images of 63 household objects in real-world settings. 
% We conducted depth completion experiments and generalization verification on ClearGrasp, TransCG, and ClearPose, reported respectively in Section \ref{sec:depth} and \ref{sec:generalization}. We performed an ablation study on TransCG, which is reported in Section \ref{sec:ablation}.

% ClearGrasp\cite{sajjan2020clear} is the first large-scale synthetic dataset as well as a real-world test benchmark focusing on transparent objects. TransCG\cite{fang2022transcg} is a large-scale real-world dataset, which contains 57K RGB-D images from 130 different scenes. ClearPose\cite{chen2022clearpose} is a recentily proposed real-world dataset, containing 350K RGB-D images covering 63 household objects.

% \newgeometry{letterpaper,top=60pt,bottom=43pt,left=48pt,right=48pt}
% \begin{table*}[!t]
% \caption{Ablation study. We show the impact of progressively substituting the components of the DFNet with ours. \label{tab:table1}
% }
% \centering
% \resizebox{\linewidth}{!}{%
% \begin{tabular}{cccccccccc}
% \toprule
% Model/Metric    & RMSE  & REL   & MAE   & $\delta$1.05 & $\delta$1.10 & $\delta$1.25          & Inference time (s)& Parameters & Size (MB)   \\ \midrule
% DFNet\cite{fang2022transcg}          & 0.018 & 0.027 & 0.012 & 83.76 & 95.67 & 99.71          & 0.0244s        & 1.25M & 4.819MB \\ \midrule
% New Loss        & 0.017 & 0.026 & 0.012 & 84.42 & 96.30 & \textbf{99.81} & 0.0244s        & 1.25M & 4.819MB \\ \midrule
% Shortcut Fusion & 0.017 & 0.024 & 0.011 & 86.18 & 96.67 & 99.79          & 0.0218s        & 1.02M & 3.919MB \\ \midrule
% Ours(slim) & 0.016          & 0.024          & 0.011          & 86.22          & 96.64          & \textbf{99.81} & \textbf{0.0143s} & \textbf{0.39M} & \textbf{1.518MB} \\ \midrule
% Ours       & \textbf{0.015} & \textbf{0.022} & \textbf{0.010} & \textbf{88.18} & \textbf{97.15} & \textbf{99.81} & 0.0153s          & 1.25M          & 4.803MB          \\
% \bottomrule
% \end{tabular}%
% }
% \end{table*}
\begin{table}[!t]
\renewcommand{\arraystretch}{1.05}
\setlength{\tabcolsep}{5pt}
\caption{Ablation study. We show the impact of progressively substituting the components of the DFNet with ours. \label{tab:table1}
}
\centering
\resizebox{\linewidth}{!}{%
\begin{threeparttable}
\begin{tabular}{cccccccccc}
\toprule
Model   & RMSE  & REL   & MAE   & $\delta$1.05 & $\delta$1.10 & $\delta$1.25          & Time(s)& Para(M) & Size (MB)   \\ \midrule
DFNet\cite{fang2022transcg}          & 0.018 & 0.027 & 0.012 & 83.76 & 95.67 & 99.71          & 0.0244        & 1.25 & 4.819 \\ \midrule
Huber Loss &0.017   &0.027  &0.012  &84.10  &95.82  &99.74 &0.0244  &1.25   &4.819  \\ \midrule
New Loss        & 0.017 & 0.026 & 0.012 & 84.42 & 96.30 & \textbf{99.81} & 0.0244        & 1.25 & 4.819 \\ \midrule
SF* & 0.017 & 0.024 & 0.011 & 86.18 & 96.67 & 99.79          & 0.0218        & 1.02 & 3.919 \\ \midrule
Ours(s)* & 0.016          & 0.024          & 0.011          & 86.22          & 96.64          & \textbf{99.81} & \textbf{0.0143} & \textbf{0.39} & \textbf{1.518} \\ \midrule
Ours       & \textbf{0.015} & \textbf{0.022} & \textbf{0.010} & \textbf{88.18} & \textbf{97.15} & \textbf{99.81} & 0.0153          & 1.25          & 4.803          \\
\bottomrule
\end{tabular}%
% \multicolumn{10}{l}{Note: NL* represents New Loss, SF* represents Shortcut Fusion and Ours(s)* represents Ours(slim).}
\begin{tablenotes}
\footnotesize
\item Note: SF* represents Shortcut Fusion and Ours(s)* represents Ours(slim).
\end{tablenotes}

\end{threeparttable}
}


\end{table}
% \vspace{-0.5cm}
\subsubsection{Metrics}
For evaluating the performance of our depth completion model, we employ four common metrics: RMSE, REL, MAE and Threshold $\delta$ (where $\delta$ is set to 1.05, 1.10, and 1.25). These metrics are calculated only on the transparent areas, as determined by transparent masks.
% Me use common metrics RMSE, REL, MAE and Threshold $\delta$ ($\delta$ is set to 1.05, 1.10 and 1.25) to evaluate our model. All metrics are calculated on the transparent areas according to transparent masks.


% We use three metrics to evaluate performance on pose estimation task. The average closest point distance (ADD-S)\cite{xiang2017posecnn} calculates the mean distance from each 3D model point to its closest neighbor on the target model. Followed DenseFusion\cite{wang2019densefusion} we report the area under the ADD-S curve (AUC) and the percentage of ADD-S smaller than 2cm ($<$2cm).

\subsection{Implementation Details}
% \noindent
% \textbf{Network configuration.}
\subsubsection{\bf Network Configuration}
% \textcolor{blue}{
In the network architecture, the number of hidden channels, \textbf{$C$}, is set to 64. Each FFEB/DFCB contains a single OSA module. Each OSA module is composed of 5 layers with stage channels of 20. The SFM module maintains \textbf{$C$} channels throughout the pipeline, while cross-layer shortcuts have only 1 channel. Residual connections between the encoder and decoder retain only \textbf{$C$} channels. The input head module and output head module use $3\times3$ convolution to adjust the number of channels and resolution (with resolution changes only occurring in the input head module). For the slim version, \textbf{$C$} is set to 32, and the OSA block contains 4 layers with stage channels of 16.
% }
% The hidden channels \textbf{$C$} in the network is set to 64. Each FFEB/DFCB contains one OSA module, in which, we use 5 layers per block and set stage channels \textbf{$C'$} to 20. SFM keeps \textbf{$C$} channels throughout the pipeline while cross-layer shortcuts take 1 channel only. Residual connections between encoder and decoder just keep channel \textbf{$C$}. $3\times3$ convolution is used in the input head module and the output head module to modify channels and resolution (resolution modified in the input head module only). For slim version, \textbf{$C$} is set to 32, \textbf{$C'$} is set to 16 and uses 4 layers per OSA block.

\subsubsection{\bf Training Details}
% \noindent
% \textbf{Training details.}
All experiments are carried out using the AdamW optimizer with an initial learning rate of $10^{-3}$. The learning rate is reduced by half after 5, 15, 25, and 35 epochs, and training continues for a total of 40 epochs with a batch size of 32. The threshold $\delta$ is kept constant at 0.1 during the training process. The weights $\alpha$ and $\beta$ for the loss function are set to 0.1 and 0.001, respectively. The images are resized to $320\times240$ for both training and testing. The experiments were conducted using an NVIDIA GeForce RTX 3090 GPU.
% We use AdamW optimizer with initial learning rate of $10^{-3}$ and multi-step learning rate scheduler which decays the learning rate by half after 5, 15, 25, 35 epochs. We train the model for 40 epochs with the batch size of 32. Threshold $\delta$ keeps 0.1 during training. Considering loss, we set $\alpha=0.1$, $\beta=0.001$. For all methods, we scale the images to $320\times240$ during training and testing. We use NVIDIA GeForce RTX 3090 for training and testing. 

 % Depth completion task and generalization ability are tested on ClearGrasp, TransCG and ClearPose. Pose estimation task is carried out on the set1 of ClearPose, since Clearpose has an accurate pose annotation without sticker. We use typical network DenseFusion\cite{wang2019densefusion} as pose estimation network. Following the learning strategy of DenseFusion, we train the network on 12G NVIDIA TITAN Xp GPU for 5 epochs with batch size of 128. The margin of refinement is set to 0.03. For fair comparison, we evaluate others works using their released source codes and optimal hyper-parameters or statistics reported in their paper.

\subsection{Ablation study} \label{sec:ablation}
We conduct an ablation study to investigate the effectiveness of our proposed components, including  new loss function, fusion branch, cross-layer shortcut and backbone structure. We take DFNet as baseline method since it is constructed following UNet structure. We  gradually replace its original components by our proposed ones and show the influence of using our proposed components. All the experiments of the ablation study are conducted on TransCG dataset.

% In view that DFNet is also constructed based on UNet, We here gradually replace its original components by our proposed. This study is conducted on TransCG dataset.
% To study the impact of each component in our proposed method, we perform experiments with different configurations of loss functions, network architecture, and backbones. Our method is compared against the recent transparent object depth completion work DFNet, which serves as our baseline. The ablation study experiments are all performed on the TransCG dataset.
% To verify the effectiveness of each component in our method, we evaluate the performance w.r.t. different configurations of loss functions, network architecture, and backbones. We use recently proposed transparent objects depth completion work DFNet as baseline. Ablation study is carried out on TransCG.




\subsubsection{\bf Loss Function}
The training of DFNet employs the mean squared error (MSE) and smooth loss as its loss function. However, these simple loss functions can lead to overfitting to local features, which makes the model more sensitive to the noise from low-level features such as edges and positions, negatively impacting its accuracy. To validate our proposed loss function, we first replaced the MSE loss with Huber loss in DFNet and termed it as Huber Loss. And then we replaced the loss function of DFNet with ours, leaving all other aspects unchanged and termed it as New Loss in Table \ref{tab:table1}. It can be observed by comparing New Loss with DFNet that all metrics showed improvement without requiring any additional parameters. 

% Qualitatively, the use of our proposed loss function can let the network to concentrate on the global structure rather than local details. By comparing the rows 3 and 4 of Figure \ref{fig:figure5}, the boundaries become smoother and even less distinct.
% The training of DFNET uses MSE and cosine distance. The simple loss function may lead to overfit to local features during training. This makes the model more sensitive to the noise of low-level features such as edge and position, which in turn affects its accuracy. So we propose a loss function consisting of Huber loss, SSIM loss and Smooth loss to suppress it. To verify its validity, we replaced the loss function of DFNet with ours and remain its other parts unchanged, then compared the results output by the mixed model (New Loss in Table \ref{tab:table1}) with the original one.
% All metrics are improved without extra parameters. Furthermore, we manually designed a feature to describe those pixels by computing the gradient of depth image and doing Gaussian blur to form an 'edge mask'. As their wights drop, the performance of the model is improved (Edge weight modified in Table \ref{tab:table2}), suggesting that it is necessary to treat pixels differently.
%and lower their weight during training. Specifically, we compute the gradient of depth image and do gaussian blur to form an 'edge mask'. Result (Edge weight modified in Table \ref{tab:table2}) supports our idea and shows it is necessary to treat pixels differently. 

\subsubsection{\bf Fusion Branch and Cross-layer Shortcuts}
In order to evaluate the impact of our proposed fusion branch and cross-layer shortcuts, we make changes to DFNet's architecture. First, we remove the redundant CDC blocks in DFNet from its skip connections, in line with our insight of preserving low-level features and the purpose of light weighting. Then, we added cross-layer shortcuts and a fusion branch to the modified network. It can be seen in Table \ref{tab:table1} that adopting this new architecture (referred to as Shortcut Fusion), almost all metrics show improvement with fewer parameters. 

\subsubsection{\bf Backbone}
We finally replace the denseblock in DFNet with our OSA module and utilized max pooling as the downsampling method. This final modification has transformed DFNet into our network. As shown in Table \ref{tab:table1}, our network outperforms the previous state-of-the-art (SOTA) by at least 16\% on difference-based metrics and improves ratio-based metrics by up to 4.42\%, resulting in a new SOTA performance. To make it practical for low-power robots, we created a slim version to balance speed and accuracy. 


% Qualitatively, figure \ref{fig:figure5} shows our method predicts clearer edges and is better handling crowded area.

% The fusion branch in our proposed network introduces a rich collection of low-level features, while the OSA module promotes feature reuse. Additionally, raw depth information is provided throughout the network, which enhances the representation of low-level features but may also hinder the learning of high-level semantic information. Our hypothesis is that the use of max pooling as a less aggressive downsampling method can mitigate these side effects while also reducing the number of parameters. The results in Table \ref{tab:table2} support our viewpoint.
% We fianlly relace the denseblock in DFNet by our used OSA module, and use max pooling as downsampling method. After this final modification, DFNet is tranformed to our proposed network. We thus show the performance by :Our"  in Table \ref{tab:table1}. It can be observed that ours outperforms previous SOTA by at least 16\% on difference-based metrics and improves ratio-based metrics by 0.1\% to 4.42\%, achieving the new state-of-the-art performance. In order to be capable in real applications, we also construct a slim version for speed/accuracy trade-off. 
% As we mentioned above, fusion branch introduces abundant low-level features and OSA encourages feature reuse. Furthermore, Raw depth is provided throughout the network. They enrich the representation of low-level features but may also harm to the learning of high-level semantic information. We suppose that using maxpooling to loosely downsampling may reduce their side effects as well as parameters saving. Result in Table \ref{tab:table2} proved our point of view.

% For summary, with our loss function, network tend to learn high-level features, with fusion branch, raw depth image and shortcuts, network can take advantage of low-level features. These components working together gives the network ability to take into account both local details and global structures. OSA module and max-pooling downsampling accelerate inference speed and reduce side effects.




% To intuitively show the impact of the proposed components, we visualize the predicted depth on TransCG and CleargGrasp dataset in Figure \ref{fig:figure5}. All networks are trained on TransCG dataset. Qualitatively, with our loss function, network is likely to focus on global structure rather than local detail. Red rectangle in row 3 and 4 show that with our loss function, boundaries become smoothy and even ambiguous, and outliers in the bottom right corner of the second column are suppressed. 



% FDCT performs domain adaption to the concatenation of raw depth and deep features and adopts maxpooling to lossly downsampling. It is supposed to reduce the disadvantage of the inaccuracy of raw depth. Our method predicts more accuracy and smooth edge as shown by the red circle on the left and the black square on the right. And even correct the ground truth as depicted in black circle on the right. The light spot reflected on the apple significantly affects the performance in row 2,3,5, but has little impact on row 4,6. Our methods successfully overcome the side effect of the raw depth information.

\subsection{Depth Completion Experiments} \label{sec:depth}

We compare our method with others on synthetic dataset ClearGrasp and real-world dataset TransCG. The quantitative results are respectively reported in Table \ref{tab:table2} and Table \ref{tab:table3}. Our proposed network surpasses others in almost every metric on these datasets which contain  synthetic and real-world scenes. Our method achieves a new state-of-the-art performance with a smaller model size and faster inference time, making it a highly competitive solution in this field.
%except on ClearGrasp synthetic validation set. It may be result of that the local implicit depth function which is environment-dependent, as well as the extra training data. 

% {\color{blue}
Specifically, our method outperforms the other methods by a larger margin in terms of REL and $\delta1.05$ metrics. This indicates its robustness to noise in the raw depth information, as these metrics are computed based on relative values and are sensitive to noise. Additionally, the gap between our method and others is larger in tests involving novel objects in ClearGrasp (CG Syn-novel in Table \ref{tab:table4} and the ClearGrasp column in Figure \ref{fig:figure5}), indicating that our method has a better ability to generalize to unseen objects. The qualitative results is reported in Figure \ref{fig:figure5}. The prediction of our method exhibits a clearer boundary and finer details than DFNet.
% }
% Specifically, our method has a bigger gap in REL and $\delta1.05$ to others most of the time. It demonstrates that our method is more stable to the noise in raw depth information of pixels, because these metrics are computed by relative value and significantly affected by noise. Noteworthy, the gap between our method and others getting bigger in the test of novel objects in most cases, indicates our method is able to generalize better to unseen objects.

\begin{table}[!t]
\caption{Depth Completion Result on TransCG dataset.}
\label{tab:table2}

\centering
\resizebox{\linewidth}{!}{%
\begin{tabular}{ccccccccc}
\toprule
Model & RMSE  & REL   & MAE   & $\delta1.05$ & $\delta1.10$ & $\delta1.25$ & Time ($\second$)   & Size ($\mega$B)    \\ \midrule
ClearGrasp\cite{sajjan2020clear}   & 0.054 & 0.083 & 0.037 & 50.48 & 68.68 & 95.28 & 2.281          & 934          \\
LIDF-Refine\cite{zhou2021pr}  & 0.019 & 0.034 & 0.015 & 78.22 & 94.26 & 99.80 & 0.018          & 251          \\
DFNet\cite{fang2022transcg}        & 0.018 & 0.027 & 0.012 & 83.76 & 95.67 & 99.71 & 0.024          & 4.8          \\
Ours (slim)   & 0.017 & 0.025 & 0.011 & 85.53 & 96.46 & 99.79 & \textbf{0.014} & \textbf{1.6} \\
Ours & \textbf{0.015} & \textbf{0.022} & \textbf{0.010} & \textbf{88.18} & \textbf{97.15} & \textbf{99.81} & 0.015 & 4.8 \\ \bottomrule
\end{tabular}}
% \vspace{-0.5cm}
\end{table}


\begin{table}[!t]
\renewcommand{\arraystretch}{0.9}
\caption{Depth Completion Results on ClearGrasp dataset\label{tab:table3}}
\centering
\resizebox{\linewidth}{!}{%
\begin{tabular}{ccccccc}
\toprule
\multicolumn{1}{c}{Model/Metric} &
  \multicolumn{1}{c}{RMSE} &
  \multicolumn{1}{c}{REL} &
  \multicolumn{1}{c}{MAE} &
  \multicolumn{1}{c}{$\delta$1.05} &
  \multicolumn{1}{c}{$\delta$1.10} &
  $\delta$1.25 \\ \midrule
\multicolumn{7}{c}{Train CG Test CG Syn-novel} \\ \midrule
\multicolumn{1}{c}{ClearGrasp} &
  \multicolumn{1}{c}{0.040} &
  \multicolumn{1}{c}{0.071} &
  \multicolumn{1}{c}{0.035} &
  \multicolumn{1}{c}{42.95} &
  \multicolumn{1}{c}{80.04} &
  98.10 \\ 
\multicolumn{1}{c}{Local Implicit} &
  \multicolumn{1}{c}{\underline{0.028}} &
  \multicolumn{1}{c}{\underline{0.045}} &
  \multicolumn{1}{c}{\underline{0.023}} &
  \multicolumn{1}{c}{\underline{68.62}} &
  \multicolumn{1}{c}{\underline{89.10}} &
  \underline{99.20} \\ 
\multicolumn{1}{c}{DFNet} &
  \multicolumn{1}{c}{0.032} &
  \multicolumn{1}{c}{0.051} &
  \multicolumn{1}{c}{0.027} &
  \multicolumn{1}{c}{62.59} &
  \multicolumn{1}{c}{84.37} &
  98.39 \\ 
\multicolumn{1}{c}{FDCT (Ours)} &
  \multicolumn{1}{c}{\textbf{0.025}} &
  \multicolumn{1}{c}{\textbf{0.040}} &
  \multicolumn{1}{c}{\textbf{0.021}} &
  \multicolumn{1}{c}{\textbf{71.66}} &
  \multicolumn{1}{c}{\textbf{92.95}} &
  \textbf{99.64} \\ \midrule
\multicolumn{7}{c}{Train CG Test CG Syn-known} \\ \midrule
\multicolumn{1}{c}{Local Implicit} &
  \multicolumn{1}{c}{\textbf{0.012}} &
  \multicolumn{1}{c}{\textbf{0.017}} &
  \multicolumn{1}{c}{\textbf{0.009}} &
  \multicolumn{1}{c}{\textbf{94.79}} &
  \multicolumn{1}{c}{\textbf{98.52}} &
  99.67 \\ 
\multicolumn{1}{c}{ClearGrasp} &
  \multicolumn{1}{c}{0.044} &
  \multicolumn{1}{c}{0.047} &
  \multicolumn{1}{c}{0.033} &
  \multicolumn{1}{c}{71.23} &
  \multicolumn{1}{c}{92.60} &
  98.24 \\ 
\multicolumn{1}{c}{DFNet} &
  \multicolumn{1}{c}{0.018} &
  \multicolumn{1}{c}{0.023} &
  \multicolumn{1}{c}{0.013} &
  \multicolumn{1}{c}{88.85} &
  \multicolumn{1}{c}{97.57} &
  \underline{99.92} \\ 
\multicolumn{1}{c}{FDCT (Ours)} &
  \multicolumn{1}{c}{\underline{0.015}} &
  \multicolumn{1}{c}{\underline{0.020}} &
  \multicolumn{1}{c}{\underline{0.012}} &
  \multicolumn{1}{c}{\underline{90.53}} &
  \multicolumn{1}{c}{\underline{98.21}} &
  \textbf{99.99} \\ \bottomrule

\end{tabular}%
% \tablen}
}
\end{table}



\subsection{Generalization Experiment} \label{sec:generalization}
% The generalization capability of a network is essential for practical applications. We evaluated the generalization ability of our proposed method from two perspectives: from synthetic images to real-world images and from one real-world dataset to another. The results of our experiments, shown in Table \ref{tab:table6}, indicate that our method (FDCT) has a comparable generalization capability to the state-of-the-art methods in cross-dataset evaluations, and it outperforms similar works in the synthetic-to-real test. However, it lags behind methods that focus solely on sim-to-real (noted as "local implicit*").
% The generalization ability of a network is critical for real-world application. The proposed method has a generalization ability that can be trained on synthetic data and aply to real world scene (syn-to-real) or trained on one real world dataset TransCG and adap to ClearGrasp (real-to-real). Comparison result is reported in Table \ref{tab:table4}. It shows that although there is still a certain gap compared with the method Local Implicit designed for syn-to-real; compared with the similar method DFNet, our method achieves a better result in the syn-to-real setting, and a competitive result in the syn-to-syn setting.
The generalization ability of a network is critical for real-world application. Our proposed method exhibits a high degree of generalization, being able to be trained on synthetic data and applied to real-world scenes (syn-to-real), or trained on one real-world dataset TransCG and adapted to the other real-world dataset (real-to-real), such as ClearGrasp. Comparison results are reported in Table \ref{tab:table4}, which show that while there is still a certain gap compared to the syn-to-real method (Local Implicit \cite{zhu2021rgb}), our method achieves better results in the syn-to-real setting when compared to the similar method DFNet, and competitive results in the real-to-real setting.

% We inspect the generalization ability of our proposed method from two aspects, from synthetic image to real-world image and from one real-world dataset to another. Experiment results in Table \ref{tab:table5} show that FDCT has a similar generalization ability to previous SOTA in cross-dataset and get better result in synthetic-to-real test compared to similar work, but is far below to methods focusing on sim-to-real.

% Since both datasets comprise real-world image, we train models on TransCG and test it on ClearGrasp real-world set for cross-dataset test. DFNet outperformed other method with a huge gap in generalization test and is chosen to be compared with ours. Comparison result is reported in Table \ref{tab:table5}. Our method outperforms the closest work in all metrics both for known and novel objects in synthetic-to-real test. There is a bigger gap between DFNet and ours in terms of novel objects. It might owe to a better utilization of RGB cues. Our method gets similar results to DFNet in cross dataset test, showing that our method has the ability to generalize from real-world dataset to another. With a series of real-world transparent objects datasets being proposed, we believe that the generalization ability in real-world is more important than sim-to-real.



% {\color{blue}
% Figure environment removed

\begin{table}[!t]
\caption{
% Result of Synthetic to Real and Cross Dataset Generalization Experiment
Generalization test on syn-to-real and real-to-real.}
\label{tab:table4}
\renewcommand{\arraystretch}{0.95}
\centering
\resizebox{\linewidth}{!}{%
% \begin{threeparttable}
\begin{tabular}{ccclclclclcl}
\toprule
\multicolumn{1}{c}{Model/Metric} &
  \multicolumn{1}{c}{RMSE} &
  \multicolumn{2}{c}{REL} &
  \multicolumn{2}{c}{MAE} &
  \multicolumn{2}{c}{$\delta$1.05} &
  \multicolumn{2}{c}{$\delta$1.10} &
  \multicolumn{2}{c}{$\delta$1.25} \\ \midrule
\multicolumn{12}{c}{Train CG Test CG Real-known (syn-to-real)} \\ \midrule
\multicolumn{1}{c}{Local Implicit\cite{zhu2021rgb}} &
  \multicolumn{1}{c}{\textbf{0.028}} &
  \multicolumn{2}{c}{\textbf{0.033}} &
  \multicolumn{2}{c}{\textbf{0.020}} &
  \multicolumn{2}{c}{\textbf{82.37}} &
  \multicolumn{2}{c}{\textbf{92.98}} &
  \multicolumn{2}{c}{\textbf{98.63}} \\ 
\multicolumn{1}{c}{DFNet} &
  \multicolumn{1}{c}{0.068} &
  \multicolumn{2}{c}{0.107} &
  \multicolumn{2}{c}{0.059} &
  \multicolumn{2}{c}{32.42} &
  \multicolumn{2}{c}{56.88} &
  \multicolumn{2}{c}{91.47} \\ 
\multicolumn{1}{c}{FDCT (Ours)} &
  \multicolumn{1}{c}{\underline{0.065}} &
  \multicolumn{2}{c}{\underline{0.103}} &
  \multicolumn{2}{c}{\underline{0.057}} &
  \multicolumn{2}{c}{\underline{33.08}} &
  \multicolumn{2}{c}{\underline{59.81}} &
  \multicolumn{2}{c}{\underline{91.70}} \\ \midrule
\multicolumn{12}{c}{Train CG Test CG Real-novel (syn-to-real)} \\ \midrule
\multicolumn{1}{c}{Local Implicit\cite{zhu2021rgb}} &
  \multicolumn{1}{c}{\textbf{0.025}} &
  \multicolumn{2}{c}{\textbf{0.036}} &
  \multicolumn{2}{c}{\textbf{0.020}} &
  \multicolumn{2}{c}{\textbf{76.21}} &
  \multicolumn{2}{c}{\textbf{94.01}} &
  \multicolumn{2}{c}{\textbf{99.35}} \\ 
\multicolumn{1}{c}{DFNet} &
  \multicolumn{1}{c}{0.051} &
  \multicolumn{2}{c}{0.088} &
  \multicolumn{2}{c}{0.046} &
  \multicolumn{2}{c}{31.23} &
  \multicolumn{2}{c}{64.66} &
  \multicolumn{2}{c}{97.77} \\ 
\multicolumn{1}{c}{FDCT (Ours)} &
  \multicolumn{1}{c}{\underline{0.043}} &
  \multicolumn{2}{c}{\underline{0.073}} &
  \multicolumn{2}{c}{\underline{0.038}} &
  \multicolumn{2}{c}{\underline{39.42}} &
  \multicolumn{2}{c}{\underline{75.54}} &
  \multicolumn{2}{c}{\underline{99.09}} \\ \midrule
\multicolumn{12}{c}{Train TCG Test CG Real-novel (real-to-real)} \\ \midrule
\multicolumn{1}{c}{Local Implicit\cite{zhu2021rgb}} &
  \multicolumn{1}{c}{0.152} &
  \multicolumn{2}{c}{0.225} &
  \multicolumn{2}{c}{0.139} &
  \multicolumn{2}{c}{9.86} &
  \multicolumn{2}{c}{20.63} &
  \multicolumn{2}{c}{46.02} \\ 
\multicolumn{1}{c}{DFNet} &
  \multicolumn{1}{c}{\textbf{0.041}} &
  \multicolumn{2}{c}{\textbf{0.054}} &
  \multicolumn{2}{c}{\textbf{0.031}} &
  \multicolumn{2}{c}{\textbf{62.74}} &
  \multicolumn{2}{c}{\textbf{83.31}} &
  \multicolumn{2}{c}{\textbf{97.33}} \\ 
\multicolumn{1}{c}{FDCT (Ours)} &
  \multicolumn{1}{c}{\textbf{0.041}} &
  \multicolumn{2}{c}{\underline{0.055}} &
  \multicolumn{2}{c}{\underline{0.032}} &
  \multicolumn{2}{c}{\underline{61.23}} &
  \multicolumn{2}{c}{\underline{82.84}} &
  \multicolumn{2}{c}{\underline{97.28}} \\ \bottomrule
\end{tabular}
%     \begin{tablenote}
%         \footnotesize
%         \item [*]Local Implicit is method aiming at sim-to-real.
%     \end{tablenote}
% \end{threeparttable}
}
%\vspace{-0.5cm}
\end{table}
% Figure environment removed
\subsection{Analysis} \label{sec:analysis}
In our proposed method, the loss function plays a crucial role in enabling the network to focus on structural information and alleviate the effects of unstable pixels. However, this focus on structural information may come at the expense of some details. On the other hand, the fusion branch and shortcuts draw attention to the details, which can introduce extra redundancy. Nonetheless, the use of maxpooling facilitates lossy and aggressive downsampling, which can reduce redundancy and improve robustness. The convolution based fusion method make better use of the raw depth image. All components work together and complement each other to achieve the best possible balance between structural information and details. In this section, we analyze the four critical components of our method and demonstrate their effectiveness.

\subsubsection{Influence of loss term}
% As we mentioned above, some unstable pixels can unwantedly make big penalty to the loss. By computing the gradient of the depth image and applying Gaussian blur, we manually created a feature to represent these pixels. As the weights of these pixels were reduced, the model's performance improved (as seen in Experiment of weight in Table \ref{tab:table5}), indicating the importance of treating pixels differently and pointing out the necessity of the so designed loss function. However, the side effect of such loss function is that the network pays too much attention to the structure and ignores some details. The highlighted area of the feature map changes from dotted to regional in the Loss column in Figure \ref{fig:figure6}.
As mentioned in \ref{section:Loss}, unstable pixels can have a significant negative influence on the calculation of the training loss. To illustrate this issue, we manually created a feature to represent these pixels by computing the gradient of the depth image and applying a Gaussian blur. By reducing the weights of these pixels, we observed an improvement in the model's performance (as seen in the Experiment of weight in Table \ref{tab:table5}), highlighting the importance of treating pixels differently and emphasizing the necessity of the used loss functions (especially the Huber Loss). Qualitatively, as shown in Figure \ref{fig:figure6}, the New Loss model places greater emphasis on the overall structure of transparent objects, as compared to DFNet, which primarily focuses on local information. The downside of such a loss function is that the network may ignore some details.
% Figure environment removed

\subsubsection{Low-level feature preservation}
% Fusion branch and cross-layer shortcuts alleviate the indistinct boundaries and perceptual details by taking more low-level cues into consideration. The highlighted area of the feature map changes from regional to scattered in the Fusion column in Figure \ref{fig:figure6}. Loss function and low-level feature awareness components together make a good trade-off between detail and structure information.
The fusion branch and cross-layer shortcuts help alleviate the issue of blurry boundaries and low perceptual details by incorporating more low-level cues. As a result, more low-level features such as object edges and holes are preserved in the feature map of Fusion model in Figure \ref{fig:figure6}. The combination of the loss function and low-level feature awareness components strikes a good balance between detail and structural information.

\subsubsection{Influence of downsampling}
Our hypothesis is that the use of max pooling as a lossy downsampling method can mitigate the side effects of the low-level awareness components while reducing the number of parameters. The results in Table \ref{tab:table5} that are noted as ``Experiment of downsampling'' support our viewpoint. It can be observed that the performance of using convolutional downsampling and average pooling is slightly worse than that of using max pooling.

% The loss function makes the network focus on structural information and alleviating the affects of unstable pixels, but may harming to the details. The fusion branch and shortcuts draws the attention to details, but may introduce extra redundancy. Maxpooling is used to lossy and aggressively downsampling. It can reduce redundancy and improve robustness. These components work together and complement each other.
% }

\subsubsection{Fusion method of depth image}
To demonstrate that fusing the raw depth image with feature map via convolution is better than directly concatenation. We removed the convolution layers used for fusion in the model Ours and named it Ours(concat). The result labeled Table ``Experiment on fusion method'' in Table \ref{tab:table5} support our viewpoint.

\begin{table}[!ht]
\centering
\caption{Experiment Result on Weight Modification, Downsampling Implementation and Fusion Method\label{tab:table5}}

\resizebox{\linewidth}{!}{%
\begin{tabular}{ccccccc}
\toprule
\multicolumn{1}{c}{Model/Metric} &
  \multicolumn{1}{c}{RMSE} &
  \multicolumn{1}{c}{REL} &
  \multicolumn{1}{c}{MAE} &
  \multicolumn{1}{c}{$\delta$1.05} &
  \multicolumn{1}{c}{$\delta$1.10} &
  $\delta$1.25 \\ \midrule
\multicolumn{7}{c}{Experiment on weight} \\ \midrule
\multicolumn{1}{c}{Baseline} &
  \multicolumn{1}{c}{0.018} &
  \multicolumn{1}{c}{0.027} &
  \multicolumn{1}{c}{0.012} &
  \multicolumn{1}{c}{83.76} &
  \multicolumn{1}{c}{95.67} &
  99.71 \\ 
\multicolumn{1}{c}{Edge Weight Modified} &
  \multicolumn{1}{c}{\textbf{0.017}} &
  \multicolumn{1}{c}{\textbf{0.025}} &
  \multicolumn{1}{c}{\textbf{0.011}} &
  \multicolumn{1}{c}{\textbf{85.34}} &
  \multicolumn{1}{c}{\textbf{96.26}} &
  \textbf{99.75} \\ \midrule
\multicolumn{7}{c}{Experiment on downsampling} \\ \midrule
\multicolumn{1}{c}{Conv Down} &
  \multicolumn{1}{c}{0.016} &
  \multicolumn{1}{c}{0.023} &
  \multicolumn{1}{c}{0.011} &
  \multicolumn{1}{c}{87.16} &
  \multicolumn{1}{c}{96.83} &
  99.80 \\ 
\multicolumn{1}{c}{AvgPooling Down} &
  \multicolumn{1}{c}{0.016} &
  \multicolumn{1}{c}{0.024} &
  \multicolumn{1}{c}{0.011} &
  \multicolumn{1}{c}{87.16} &
  \multicolumn{1}{c}{96.93} &
  99.80 \\ 
\multicolumn{1}{c}{MaxPooling Down} &
  \multicolumn{1}{c}{\textbf{0.015}} &
  \multicolumn{1}{c}{\textbf{0.022}} &
  \multicolumn{1}{c}{\textbf{0.010}} &
  \multicolumn{1}{c}{\textbf{88.18}} &
  \multicolumn{1}{c}{\textbf{97.15}} &
  \textbf{99.81} \\ \midrule
  \multicolumn{7}{c}{Experiment on fusion method} \\ \midrule
  \multicolumn{1}{c}{Ours(concat)} &
  \multicolumn{1}{c}{\textbf{0.015}} &
  \multicolumn{1}{c}{0.023} &
  \multicolumn{1}{c}{0.011} &
  \multicolumn{1}{c}{87.90} &
  \multicolumn{1}{c}{96.68} &
  99.80 \\ 
\multicolumn{1}{c}{Ours} &
  \multicolumn{1}{c}{\textbf{0.015}} &
  \multicolumn{1}{c}{\textbf{0.022}} &
  \multicolumn{1}{c}{\textbf{0.010}} &
  \multicolumn{1}{c}{\textbf{88.18}} &
  \multicolumn{1}{c}{\textbf{97.15}} &
  \textbf{99.81} \\ 
\bottomrule
\end{tabular}%
}
%\vspace{-0.5cm}
\end{table}
\vspace{-0.2cm}


\subsection{Pose Estimation Experiment}
In this experiment, we aim to demonstrate the applicability of our network for downstream tasks and to show that it can improve the accuracy of pose estimate.
To evaluate the performance of pose estimation, we use three evaluation metrics, i.e, the average closest point distance (ADD-S), the area under the ADD-S curve (AUC), and the percentage of ADD-S values that are smaller than 2 \centi\meter.
%\cite{xiang2017posecnn}
% The higher the metrics the stronger the performance.

% This experiment is carried out on the set1 of ClearPose, since Clearpose has an accurate pose annotation without sticker. We use typical network DenseFusion \cite{wang2019densefusion} as pose estimation network. Following the learning strategy of DenseFusion, we train the network on 12G NVIDIA TITAN Xp GPU for 5 epochs with batch size of 128. The margin of refinement is set to 0.03. For fair comparison, we evaluate others works using their released source codes and optimal hyper-parameters or statistics reported in their paper.
Both our method and DFNet are trained on the ClearPose Set 1 and are used to predict the depth of Set 1-Scene 5 for pose estimation purposes. The depth completion result is reported in Table \ref{tab:table6} and a screenshot of the live demonstration is reported in Figure \ref{fig:figure7}. In our experiments, we use DenseFusion \cite{wang2019densefusion}  as the pose estimation method. We trained DenseFusion with the restored depth and tested it on 3,000 randomly selected images. Ideally, a more accurate depth prediction can lead to improved performance in pose estimation. The results of our evaluations, presented in Table \ref{tab:table7}, indicate that the depth restored by our method outperforms DFNet in almost every object in the pose estimation task. This results validate that the depth map given by our method is more appropriate for addressing the downstream task, i.e., pose estimation.
% Depth completion models are trained on ClearPose set 1 and predict the depth of set 1-scene 5 for pose estimation. We train DenseFusion with the restored depth and test on 3k randomly chosen images. Metrics for each object are reported in Table \ref{tab:table7}. Result shows that the depth restored by FDCT outperforms DFNet's in almost every object in pose estimation task.
% \todo{format of tablehead!!}
\begin{table}[!t]
\caption{Depth Completion Results on ClearPose dataset.}
\label{tab:table6}
\centering
\begin{tabular}{ccccccc}
\toprule
Model & RMSE           & REL            & MAE            & $\delta$1.05          & $\delta$1.10          & $\delta$1.25          \\ \midrule
DFNet        & 0.048          & 0.038          & 0.033          & 76.36          & 94.22          & \textbf{99.40} \\
Ours         & \textbf{0.045} & \textbf{0.033} & \textbf{0.028} & \textbf{82.15} & \textbf{94.43} & 99.25          \\
\bottomrule
\end{tabular}%
\end{table}



\begin{table}[!t]
\caption{Pose Estimation Results on ClearPose dataset\label{tab:table7}}
\centering
\resizebox{\linewidth}{!}{%
\begin{tabular}{ccccccc}
\toprule
Models &
  \multicolumn{3}{c}{DFNet} &
  \multicolumn{3}{c}{Ours} \\ \midrule
Object/Metirc &
  \multicolumn{1}{c}{AUC} &
  \multicolumn{1}{c}{\textless{}2cm} &
  ADD-S(10\%) &
  \multicolumn{1}{c}{AUC} &
  \multicolumn{1}{c}{\textless{}2cm} &
  ADD-S(10\%) \\ 
beaker\_1 &
  \multicolumn{1}{c}{79.07} &
  \multicolumn{1}{c}{\textbf{0.00}} &
  0.68 &
  \multicolumn{1}{c}{\textbf{80.44}} &
  \multicolumn{1}{c}{\textbf{0.00}} &
  \textbf{7.53} \\ 
dropper\_1 &
  \multicolumn{1}{c}{\textbf{67.76}} &
  \multicolumn{1}{c}{61.00} &
  \textbf{48.00} &
  \multicolumn{1}{c}{31.70} &
  \multicolumn{1}{c}{\textbf{65.33}} &
  0.00 \\ 
dropper\_2 &
  \multicolumn{1}{c}{81.09} &
  \multicolumn{1}{c}{\textbf{33.10}} &
  1.78 &
  \multicolumn{1}{c}{\textbf{84.24}} &
  \multicolumn{1}{c}{0.00} &
  \textbf{9.61} \\ 
flask\_1 &
  \multicolumn{1}{c}{84.96} &
  \multicolumn{1}{c}{60.33} &
  42.33 &
  \multicolumn{1}{c}{\textbf{86.71}} &
  \multicolumn{1}{c}{\textbf{68.33}} &
  \textbf{68.00} \\ 
funnel\_1 &
  \multicolumn{1}{c}{78.85} &
  \multicolumn{1}{c}{91.33} &
  0.00 &
  \multicolumn{1}{c}{\textbf{82.91}} &
  \multicolumn{1}{c}{\textbf{98.33}} &
  \textbf{12.33} \\ 
cylinder\_1 &
  \multicolumn{1}{c}{78.77} &
  \multicolumn{1}{c}{48.33} &
  28.67 &
  \multicolumn{1}{c}{\textbf{79.83}} &
  \multicolumn{1}{c}{\textbf{77.00}} &
  \textbf{33.33} \\ 
cylinder\_2 &
  \multicolumn{1}{c}{62.75} &
  \multicolumn{1}{c}{54.67} &
  3.33 &
  \multicolumn{1}{c}{\textbf{75.68}} &
  \multicolumn{1}{c}{\textbf{58.67}} &
  \textbf{29.33} \\ 
pan\_1 &
  \multicolumn{1}{c}{86.76} &
  \multicolumn{1}{c}{13.67} &
  33.33 &
  \multicolumn{1}{c}{\textbf{89.37}} &
  \multicolumn{1}{c}{\textbf{53.67}} &
  \textbf{50.00} \\ 
pan\_2 &
  \multicolumn{1}{c}{88.71} &
  \multicolumn{1}{c}{84.67} &
  44.00 &
  \multicolumn{1}{c}{\textbf{89.73}} &
  \multicolumn{1}{c}{\textbf{90.33}} &
  \textbf{56.00} \\ 
pan\_3 &
  \multicolumn{1}{c}{\textbf{88.90}} &
  \multicolumn{1}{c}{87.67} &
  \textbf{53.33} &
  \multicolumn{1}{c}{88.10} &
  \multicolumn{1}{c}{\textbf{91.00}} &
  48.00 \\ 
bottle\_1 &
  \multicolumn{1}{c}{86.05} &
  \multicolumn{1}{c}{91.53} &
  24.41 &
  \multicolumn{1}{c}{\textbf{88.71}} &
  \multicolumn{1}{c}{\textbf{93.22}} &
  \textbf{31.53} \\ 
bottle\_2 &
  \multicolumn{1}{c}{71.81} &
  \multicolumn{1}{c}{83.16} &
  4.04 &
  \multicolumn{1}{c}{\textbf{77.01}} &
  \multicolumn{1}{c}{\textbf{88.22}} &
  \textbf{13.47} \\ 
stick\_1 &
  \multicolumn{1}{c}{69.53} &
  \multicolumn{1}{c}{32.32} &
  32.66 &
  \multicolumn{1}{c}{\textbf{79.60}} &
  \multicolumn{1}{c}{\textbf{57.58}} &
  \textbf{58.92} \\ 
syringe\_1 &
  \multicolumn{1}{c}{73.03} &
  \multicolumn{1}{c}{31.67} &
  25.67 &
  \multicolumn{1}{c}{\textbf{80.15}} &
  \multicolumn{1}{c}{\textbf{57.00}} &
  \textbf{47.00} \\ 
MEAN &
  \multicolumn{1}{c}{78.43} &
  \multicolumn{1}{c}{55.25} &
  24.45 &
  \multicolumn{1}{c}{\textbf{79.58}} &
  \multicolumn{1}{c}{\textbf{64.19}} &
  \textbf{33.22} \\


  \bottomrule
  \end{tabular}%
}
\vspace{-0.5cm}
\end{table}

\section{Conclusions}
We present FDCT, a novel and efficient end-to-end network for transparent object depth completion. FDCT predicts a rectified depth map using only RGB and depth images as inputs. It outperforms the state-of-the-art methods in terms of accuracy and efficiency, with the smallest number of parameters while running at a fast speed of 70 FPS. Our experiments show that FDCT has strong generalization ability, with excellent results on unseen objects and cross-dataset scenarios. In addition, the improved depth prediction by FDCT can lead to improved performance in pose estimation, demonstrating its potential as a useful auxiliary tool for ordinary pose estimation methods when dealing with transparent objects.
There is still some works could be done in the future. The hyperparameter $\delta$ in Huber loss is set manually and unchanged. 
% In this paper, we propose FDCT, an end-to-end efficient network for transparent objects depth completion. FDCT predicts rectified depth from RGB and depth image only. Our method achieves state-of-the-art performance with least parameters and runs at 70 FPS. Experiment results show that FDCT has a generalization ability to unseen object and cross-dataset. Pose estimation method designed for opaque objects achieves a better performance with depth restored by FDCT, showing its potential to be an auxiliary tool for ordinary pose estimation methods dealing with transparent objects.


\section*{Acknowledgments}
The authors thank the supports by the National Key R\&D Program of
China: [grant number 2020YFF0304004] and the Fundamental
Research Funds for the Central Universities: [grant
number FRF-TP-20-009A3].


\bibliographystyle{IEEEtran}
\bibliography{ref}

\end{document}
