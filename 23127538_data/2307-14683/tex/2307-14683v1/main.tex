\documentclass[aps,pre,twocolumn,showpacs,superscriptaddress,groupedaddress]{revtex4-2}
\usepackage{amsmath}
\usepackage{amsfonts}
\usepackage{amssymb}
\usepackage{graphicx}
\usepackage{graphics}
\usepackage{dcolumn}
\usepackage{sidecap}
\usepackage{parskip}
\usepackage{xcolor}
\usepackage{hyperref}
\usepackage{hhline}
\usepackage{mathtools}
\usepackage{multirow}
\usepackage{verbatim}
\usepackage{rotating}
\usepackage{setspace}
\usepackage{epsfig}
\usepackage{epstopdf}


\usepackage{subfigure}
\usepackage{booktabs}
\usepackage[normalem]{ulem}
\usepackage{bm}
\usepackage{verbatim}
\usepackage{comment}
\usepackage{cleveref}
\usepackage[normalem]{ulem}
\newcommand{\n}{\noindent}
\newcommand{\A}{\textbf{A}}
\newcommand{\I}{\textbf{I}}
\newcommand{\M}{\cal{M}}
\newcommand{\mk}{\langle k \rangle}
\newcommand{\iu}{\mathrm{i}\mkern1mu}

\def \hfillx {\hspace*{-\textwidth} \hfill}

\makeatletter
\def\@eqnnum{{\normalsize \normalcolor (\theequation)}}
 \makeatother

\hyphenation{ALPGEN}
\hyphenation{EVTGEN}
\hyphenation{PYTHIA}

\newcommand\filledcirc{{\color{black}\bullet}\mathllap{\circ}}

\graphicspath{{./}{ER/main/}}
\bibliographystyle{apsrev4-2}



\begin{document}
\title{Rotating synchronized clusters in phase-lagged Kuramoto oscillators with simplicial interactions}
%higher-order interactions}
\author{Bhuwan Moyal}
\author{Priyanka Rajwani}
\author{Subhasanket Dutta}
\author{Sarika Jalan}\email{sarika@iiti.ac.in}
\affiliation{1. Complex Systems Lab, Department of Physics, Indian Institute of Technology Indore, Khandwa Road, Simrol, Indore-453552, India}


\date{\today}

\begin{abstract}
The effect of phase-lag parameter in pairwise interactions has been a topic of great interest for long. 
%An inclusion of phase-lag in coupld Kuramoto oscillators has been shown to delay the occurance of synchronization tranistion. 
However, real-world systems often have interactions that are beyond pairwise and can be modeled using simplicial complexes. 
%so to fully understand real-world systems, 
We investigate the effect of the inclusion of phase-lag in coupled Kuramoto oscillators with simplicial interactions and find that it shifts the critical points at which first-order transition from cluster synchronized state to incoherent state occurs.
%a generalized version of 2-simplex model with a phase lag. 
In the thermodynamic limit, using the Ott-Antonsen approach we derive a reduced equation for order parameter measuring cluster synchronization.
%to achieve a closed expression for the order-parameter describing the synchronization profile, we 
%this problem 
%for the symmetric part using Ott-Antonsen ansatz, further to study the steady-state solutions 
 Further, we progress through the self-consistency method to achieve a closed form of the order parameter measuring global synchronization
%of the global order parameter for triadic interactions, 
which was lacking in Ott-Antonsen approach.
%This reduced form describe the effect of triadic coupling strength (having phase lag) on synchronization. 
%Apart from setting different initial conditions, 
%The interaction strength for backward transition moves towards the right due to an increase in phase lag, 
Moreover, considering polar coordinates framework we
obtain rotation frequency of the clusters which comes out to be a function of the phase-lag parameter further indicating that phase-lag can be used as a control parameter to achieve a desired cluster frequency.
 %The work may have potential applications in understanding the origin of cluster synchronization prevalent in a range of real-world complex systems such as Brain, power grids,  and social contacts having higher-order interactions and phase-lag.

%In the following model, we have two synchronized cluster states where the inclusion of the phase lag parameter causes the transition from synchronization and asymmetry in two clusters to a completely incoherent state to occur at a high value of coupling strength causing frustration in the system.
\end{abstract}

\maketitle
\paragraph{\bf{Introduction:}}
Synchronization\cite{kuramoto1984chemical,pikovsky2002synchronization,strogatz2004sync} of interacting units  occurs in many real-world complex systems ranging from circadian clock construction in brain 
 \cite{LIU1997855}, neural networks \cite{inproceedings}, power grids \cite{RePEc:spr:eurphb:v:61:y:2008:i:4:p:485-491,PhysRevLett.109.064101,RevModPhys.94.015005}, cardiac rhythms 
 \cite{karma2013physics}, to chemical oscillators \cite{article5,article6}. It was the insights of Winfree that later Kuramoto utilized to model the collective phenomenon of synchronization into a more manageable form which tells us that system progresses from an incoherent to a fully coherent state through a second-order phase transition \cite{10.1007/BFb0013365}. Following this, a considerable amount of studies was done on the Kuramoto model uncovering various different phenomena, such as global, cluster synchronization, chimera and many more\cite{PhysRevLett.100.144102,PhysRevLett.106.254101,PhysRevLett.109.164101,PhysRevLett.93.224101,rodrigues2016kuramoto,PhysRevE.57.1563}. However, coupled Kuramoto oscillators might not provide an apt model for many real-world complex systems experiencing phase frustrated coupling which is similar to time-delayed coupling \cite{Crook1997TheRO} found in various physical systems. For example, in power grids, the phase-lag parameter corresponds to the energy loss along the transmission lines \cite{article11}. 
Sakaguchi and Kuramoto investigated the effect caused by the inclusion of a phase frustration parameter in an ensemble of oscillators, also they uncovered that if a pair of oscillators are strongly coupled, they can
 %coupled strongly enough that 
 come together in a cluster that rotates with a common non-zero frequency deviating from the algebraic sum of their intrinsic frequencies which is in contrast to what was seen in the Kuramoto model (having zero phase-lag).
 Later, the phase-lag was shown to be responsible for phase turbulence in self-oscillatory diffusive systems \cite{sakaguchi1986soluble,kuramoto1984cooperative} as well. Apart from the usual behavior of the 
  transition from an incoherent to a synchronized state through a partially synchronized state, the Sakaguchi-Kuramoto model displays non-trivial synchronization transitions where an increase in the coupling strength leads to a decrease in synchronization, in which incoherence starts regaining its stability, and partially synchronized, incoherent states coexist together \cite{PhysRevLett.109.064101,article9}. This model has also become the first prototype to investigate chimera where the non-locally coupled oscillators voluntarily split into synchronized and incoherent populations \cite{PhysRevLett.101.084103,article10}. 
 Other examples of systems that have been modeled through Kuramoto oscillators with phase-lag parameter are seismology \cite{article9}, Josephson junctions \cite{PhysRevLett.76.404,PhysRevLett.82.1963,PhysRevE.61.2513}, etc. Also, a neural network with distributed time delays can be modeled as coupled Kuramoto model with a phase-lag parameter  \cite{SONG20081538}. 

 However, all these results were obtained for purely pairwise interactions. Recent advances have indicated that this simplistic view might not be sufficient to fully decipher the underlying mechanisms behind many real-world complex phenomenons where higher order or $n-$simplicial interactions are crucial. For example, real-world complex systems such as Brain \cite{article,article1,article2},  scientific collaborations \cite{article3}, social systems  \cite{article4} have underlying higher-order interactions. $n$ simplex means there exist $n+1$ interacting units. Here we consider $2-$simplex, i.e., the coupling describing three-way interactions. Skardal and Arenas have shown that 
 the Kuramoto oscillators coupled through $2-$simplex interactions manifest an abrupt first-order transition to de-synchronization with no complementary abrupt synchronization transition \cite{Skardal_prl2019}. Also, in the thermodynamic limits there exist continuum de-synchronization transition points arising due to changes in the initial conditions.
 
%Here we consider a $2$-simplex Kuramoto model with a phase lag parameter to analyze how the dynamics of synchronization get affected. 
Recent work on coupled Kuramoto oscillators model incorporating phase-lag parameter $\alpha$ in the triadic interactions  along with the pairwise interactions has considered the following form of the triadic coupling $\sin(2\theta_j+\theta_k-\theta_i-\alpha)$ 
 \cite{dutta2023impact}. Here, in this article, we consider another phase reduction form of
the complex Ginzberg Landau equation for 2-simplex interactions \cite{leon2019phase} yielding the triadic coupling as  $\sin(\theta_j+\theta_k-2\theta_i-\alpha)$. Such a form of the 2-simplex coupling in the absence of any phase-lag is known to manifest two clusters state 
%, that gives rise to a continuum of 
which gets destroyed through an abrupt de-synchronization transition as coupling strength is adiabatically decreased 
%points as well as infinite number of multistable states based upon different initial conditions 
\cite{Skardal_prl2019}. Here, we show that the inclusion of a phase frustration term shifts the critical de-synchronization point toward the right. That is, starting with a cluster-synchronized state, as coupling strength decreases adiabatically, de-synchronization to an incoherent state occurs for larger coupling strengths than that achieved for the zero phase-lag cases. 
The crucial difference between the form of the triadic interactions considered here from the other form of the triadic interactions considered in ~\cite{dutta2023impact} and other existing models having phase-lagged in the pair-wise interactions is the existence of stable 2-clusters state in contrast to a stable global synchronized state.
In the thermodynamic limit, using the Ott-Antonsen approach \cite{Ott_Antonsen2008} we first derive the reduced dimensional equation for cluster synchronization order parameter, and then by using the self-consistency method obtain the closed forms of order parameters corresponding to global and cluster synchronization. The challenge lies in deriving an analytical expression of the cluster frequency ($\Omega$)  which comes out to be different from the mean of the intrinsic frequency of the oscillators and rather manifests an explicit dependence on $\alpha$.
%There exists an unique value of the mean frequency ($\Omega$) which 
%satisfies
%the self-consistency relation. , 
Ergo,  $\alpha$ can be used as a control parameter to regulate the rotation frequency of clusters to a desired value \cite{lohe2015synchronization}.
%similarly describe for phase frustrated Kuramoto model ($\sin(\theta_j-\theta_i-\alpha)$) \cite{lohe2015synchronization}. 
Further, we present the numerical simulations for finite-size networks which show a good match with the analytical predictions performed in the thermodynamics limit. 

%along with analytical derivations for the depiction of global and cluster synchronization using the self-consistency method and the Ott-Antonsen approach. 


 
%and particularly we calculate the phase transition points.
%Along with  we present analytical derivations for the self-consistency as well as the Ott-Antonsen methods to obtain reduced order parameter equations applicable only for even part of density fuction.
%, and analyze the dynamics of the global order parameter along with two cluster states.
. %Here we analyze the effect of phase-lag on another natural interaction term in 2-simplex. 
 
%During the transition from synchronized to an incoherent state, this information on arrangements of oscillators in the two clusters is not lost and as a result, this can be viewed as a basic model for information retention.


\paragraph{\bf{Model:}} 
% Figure environment removed

We consider a higher-order extension of the Kuramoto-Sakaguchi model, in particular, 2-simplex interactions,
\begin{equation}\label{mode_eq}
\dot\theta_i=\omega_i+\frac{K_2}{N^{2}}\sum_{j=1}^{N}\sum_{k=1}^{N}\sin(\theta_j+\theta_k-2\theta_i-\alpha),
\end{equation}
where $\alpha$ depicts the phase-lag and $K_2$ is the 2-simplex coupling strength for $N$ oscillators.
To analyze the collective behavior of the oscillators we introduce a definition of the generalized order parameter $z_q={r_q}e^{\iota\psi_q}=\frac{1}{N}\sum_{j=1}^{N}e^{q\iota\theta_j}$, for ($q=1,2$), with $z_1$ and $z_2$ measuring the extent of global and 2-cluster synchronization, respectively. 

\paragraph{\bf {Mean-field and analytical approaches:}} Order-parameter notions help us to write Eq.~\ref{mode_eq} into the mean-field form such as 
\begin{equation}\label{mean_field}
    \dot\theta_i=\omega_i+{K_2}{r_1^{2}}\sin(2\psi_1-2\theta_i-\alpha). 
\end{equation}
In the continuum limit $N \rightarrow \infty$ the state of the system can be given by density function $\rho(\theta,\omega,t)$ which describes the density of oscillators with phase between $\theta$ and $\theta+\delta\theta$ and intrinsic frequencies between $\omega$ and $\omega+\delta\omega$ at time $t$. Since the number of oscillators are conserved, $\rho$ must satisfy the continuity equation 
\begin{equation}\label{cont_eq}
 \frac{\partial \rho}{\partial t}=-\frac{\partial(\rho\dot\theta)}{\partial\theta}.
\end{equation}
Considering the frequency of each oscillator drawn from a distribution $g(\omega)$, the density function can be expanded into Fourier series
\begin{equation}\nonumber
\rho(\theta,\omega,t)=\frac{g(\omega)}{2\pi}\sum_{n=-\infty}^{\infty}\rho_n(\omega,t)e^{{\iota}n\theta}, 
\end{equation}
where $\rho_n(\omega,t)$ being the $n^{th}$ Fourier coefficient and $\rho_{-n}=\rho_{n}^* e^{-\iota n \theta}$. We can write the density function into the sum of the symmetric and anti-symmetric parts; $\rho_s(\theta+\pi,\omega,t)=\rho_s(\theta,\omega,t)$ and $\rho_a(\theta+\pi,\omega,t)=-\rho_a(\theta,\omega,t)$. The linearity property of the continuity equation suggests that individually $\rho_s$ and $\rho_a$ are solutions, therefore the combination of both is also a solution. However, only the symmetric part allows for dimensionality reduction using Ott-Antonsen ansatz as all the Fourier modes decay geometrically \cite{Ott_Antonsen2008}, i.e., $\rho_{2n}(\omega,t)=\upsilon^n(\omega,t)$ where $|\upsilon(\omega,t)|\le 1$, 
\begin{equation} \label{fou_exp}
\rho_s(\theta,\omega,t)=\frac{g(\omega)}{2\pi}[1+\sum_{n=1}^{\infty}\rho_{2n}(\omega,t)e^{in\theta}+c.c].
\end{equation}
Plugging this and Eq.~\ref{mean_field} into the continuity Eq.~\ref{cont_eq}, we find that each subspace spanned by even terms $e^{2\iota n \theta}$ collapses into a one-dimensional manifold given by,
\begin{equation}\label{mid_ott}
    \frac{\partial\upsilon}{\partial t}= -2\iota\upsilon\omega+K_2({z^{*}_1}^{2}{e^{\iota\alpha}}-z_1^{2}\upsilon^{2}e^{-\iota\alpha}).
\end{equation}
In the continuum limit $N\rightarrow\infty$, we have $z_2=\int_{-\infty}^{\infty}\int_{0}^{2\pi}{\rho_s(\theta,\omega,t)e^{\iota 2\theta}g(\omega)d\theta{d\omega}}$, which after inserting the Fourier series expansion of $\rho_s(\theta,\omega,t)$ reduces to $z_2=\int_{-\infty}^{\infty}g(\omega){\upsilon^*}d\omega$. If we consider the frequency distribution $g(\omega)$ to be Lorentzian   {\large$g(\omega)=\frac{\Delta}{\pi[(\omega-\omega_0)^2+\Delta^2]}$} with mean $\omega_0=0$ and spread $\Delta=1$, the integral ${z_2}$ can be calculated using Cauchy's Residue Theorem by contour integration in the negative half-plane, yielding $z_2={\upsilon^*}(\omega_0-\iota\Delta,t)$.  After making this substitution and separating the real and imaginary parts, Eq.~\ref{mid_ott} reduces to
\begin{equation}\label{ana1}
    \dot{r_2} = -2r_2 +K_2{{r_1}^{2}}(1-r_2^{2})\cos(2\psi_1-\psi_2-\alpha).
\end{equation}
\begin{equation}\label{ana2}
    \dot\psi_2 = K_2{r_1^{2}}\frac{1+r_2^{2}}{r_2}\sin(2\psi_1-\psi_2-\alpha).
\end{equation}
Note that these equations are achieved by considering the contribution of the symmetric part  ($\rho_s$) only, and there exists no explicit relation of $r_1$ and $K_2$, and hence we proceed using the self-consistency method.

%Proceeding further to analyze the $z_1$ using the self-consistency method. 
We change the frame of reference $\theta \rightarrow \theta + \psi$ and enter into the rotating frame of $\dot\psi=\Omega$, which yields $\psi_1$ and $\psi_2$  zero. Hence, Eq.~\ref{mean_field} can be written as 
\begin{equation}\label{rot_mean_field}
    \dot\theta_i=\omega_i-\Omega-K_2{r_1}^2\sin(2\theta_i+\alpha).
\end{equation}
Note that in the case of $\alpha=0$ initially, oscillators are distributed in a complex circle around the mean $\psi$ similarly to frequency distribution $g(\omega)$. Also, on changing the value of $K_2$ the frequency range of the locked oscillators participating in clusters remains symmetric about mean zero. However, for non-zero $\alpha$ values effective clusters frequency for $K_2>K_{2c}$, i.e., the critical coupling strength where the transition occurs,  will be different from the mean of intrinsic frequencies.  Consequently, the synchronized clusters rotate with a common non-zero frequency $\Omega$ with magnitude of the maximum frequency being different from that of the $\alpha=0$ case. 
%In contrast to the case of $\alpha=0$  where the cluster is stationary and  addition of mroe oscillators in the cluster  synchronization in a symmetric manner.
% Figure environment removed
Next, based on the dynamical behavior of the oscillators the whole population can be divided into two groups of the locked and drifting oscillators such as $|{\frac{\omega_i - \Omega}{K_2r_1^2}}| \le 1$ and $|{\frac{\omega_i - \Omega}{K_2r_1^2}}|>1$, respectively. Moreover, for the locked oscillators, the coupling form of the higher-order interactions considered here, Eq.~\ref{rot_mean_field} renders two stable fixed points;  $\theta^*=\frac{1}{2}\arcsin(\frac{\omega_i - \Omega}{K_2r_1^2})-\frac{\alpha}{2}$ and $\theta^*+\pi$. Further, to study the contribution of the locked oscillator population, we can define the density function such as 
\begin{equation}
    \rho^{lock}(\theta,\omega)=\eta \delta(\theta-\theta^*)+(1-\eta)\delta(\theta-(\theta^*+\pi)),
\end{equation}
where $\eta$ and $1-\eta$ depict the probability of oscillators having value $\theta^*$ and $\theta^*+\pi$, respectively. Furthermore, the definition of $z_1=\int_{-\infty}^{\infty}\int_{0}^{2\pi}{e^{\iota \theta}\rho_{loc}(\theta,\omega)g(\omega)d\theta{d\omega}}$ provides the contribution from the locked oscillators given by,
\begin{equation}\label{locked}
    {r_1}^{lock}=(2\eta-1)\int_{-K_{2}r_1^{2}+\Omega}^{K_{2}r_1^{2}+\Omega} e^{\iota\theta^*}g(\omega)d(\omega).
\end{equation}
Moreover, for the locked state $\dot \theta=0$ from Eq.~\ref{rot_mean_field}, 
using the trigonometric identities, the above equation can be expressed as 
\begin{equation}\nonumber
\begin{split}
    \cos(\theta^*+\frac{\alpha}{2})&=\sqrt{\frac{1+\sqrt{1-(\frac{\omega_i - \Omega}{K_2r_1^2})^2}}{2}},\\
     \sin(\theta^*+\frac{\alpha}{2})&=\pm \sqrt{\frac{1-\sqrt{1-(\frac{\omega_i - \Omega}{K_2r_1^2})^2}}{2}},
\end{split}
\end{equation}
where $\theta^*=\Theta-\frac{\alpha}{2}$. The contribution of the sinusoidal term will be either positive or negative determined based on the limits of the integration over $\omega$. Hence, Eq.~\ref{locked} can be expressed as 
\begin{equation}\nonumber
    {r_1}^{lock} e^{\iota \frac{\alpha}{2}}=(2\eta-1)\int_{-{K_2{r_1}^2}+\Omega}^{{K_2{r_1}^2}+\Omega} e^{\iota \Theta} g(\omega)d\omega.
\end{equation}
By plugging the value of $\Theta$ and comparing the real and imaginary parts, the  contribution from the locked oscillators gets determined as 
\begin{align}\label{fin_locked}
{r_1}^{lock}=&(2\eta-1)[\cos{\frac{\alpha}{2}}\int_{-K_2{r_1}^2+\Omega}^{K_2{r_1}^2+\Omega}\sqrt{\frac{1+\sqrt{1-{(\frac{\omega_i-\Omega}{K_2{r_1}^2})}^2}}{2}}g(\omega)d\omega \nonumber \\
    &- \sin{\frac{\alpha}{2}}\int_{-K_2{r_1}^2+\Omega}^{\Omega}\sqrt{\frac{1-\sqrt{1-{(\frac{\omega_i-\Omega}{K_2{r_1}^2})}^2}}{2}}g(\omega)d\omega \nonumber \\
   & + \sin{\frac{\alpha}{2}}\int_{\Omega}^{K_2{r_1}^2+\Omega}\sqrt{\frac{1-\sqrt{1-{(\frac{\omega_i-\Omega}{K_2{r_1}^2})}^2}}{2}}g(\omega)d\omega ].
\end{align}
In addition, to analyze the contribution of the drifting oscillators 
\begin{equation}\label{drift}
    {r_1}^{drift}=\int_{|{\frac{\omega_i - \Omega}{K_2r_1^2}}|>1}^{} \int_{0}^{2\pi}e^{\iota\theta}\rho_{\small d}(\theta,\omega)g(\omega)d(\omega),
\end{equation}
from Eq.~\ref{cont_eq} in the steady state, $\rho\dot\theta$ should be a constant yielding $\rho_{d}=\frac{C}{\dot\theta}$, which upon normalization reduces into 
\begin{equation}\nonumber
   \rho_{d}=\frac{\sqrt{(\omega-\Omega)^2-({K_2{r_1}^2})^2}}{2\pi|\omega-\Omega-{K_2{r_1}^2}\sin(2\theta+\alpha)|},
\end{equation}
this implies that $\rho(\theta,\omega)=\rho(\theta+\pi,\omega)$. Furthermore, owing to the above integral (Eq.~\ref{drift}) value being zero,  the contribution from the drifting oscillators vanishes. Therefore, $r_1=r_1^{loc}+r_1^{drift}\approx r_1^{loc}$ given by Eq.~\ref{fin_locked}. Additionally, the mathematical expression for $r_2$ at the steady state from Eq.~\ref{ana1} can be given as,
\begin{equation}\label{r2}
    r_2=\frac{-1+\sqrt{1+{{K_2}^2{r_1}^4{\cos^2{\alpha}}}}}{{K_2{r_1}^2\cos{\alpha}}}.
\end{equation}
Further, to simplify the expression of the self-consistency equation (~\ref{fin_locked}) we have to determine $\Omega$ since the cluster of the oscillators is moving with respect to the ground frame. The general definition for $\psi_q$ can be given as,
\begin{equation}
\psi_q=arctan\frac{(\sum_{j=1}^{N}sin(q\theta_j))}{(\sum_{j=1}^{N}cos(q\theta_j))}.
\end{equation}
Since there exists no contribution from the drifting oscillators,  upon breaking the sum into the locked and drifting parts achieved from the simplification of the mean phases, we arrive to the relation $\psi_2=2\psi_1$. Next, using Eqs.~\ref{ana2} and ~\ref{r2}, we get the expression of $\dot{\psi_1}=\Omega$ such as,
\begin{equation}\label{omega}
    \Omega=-\sqrt{(1+{{K_2}^2{r_1}^4{\cos^2{\alpha}}})}\tan{\alpha}.
\end{equation}
Using the above expression, Eq.~\ref{rot_mean_field} reduces to
\begin{equation} \label{fin_model_Eq}
    \dot\theta_i=\omega_i+\tan(\alpha)\sqrt{(1+{{K_2}^2{r_1}^4{\cos^2{\alpha}}})} \,\, - \, K_2{r_1}^2\sin(2\theta_i+\alpha).
\end{equation}
This equation helps us to analyze the model (Eq.~\ref{mode_eq}) with a change of the reference frame (free of $\psi_1$) enabling us to get rid of the rotation of the cluster.
% Figure environment removed

\paragraph{\bf {Numerical results:}}
 To access the effect of the phase frustration caused by $\alpha$ on higher-order coupling, numerically we simulate the Eq.~\ref{fin_model_Eq} in a rotating cluster frame for $N=10^{4}$. RK-4 method is used with a time step $dt=0.05$, and $r_1$ and $r_2$ are obtained by averaging over $2*10^{4}$ iterations after removing the initial transient period. Numerical results are plotted against the analytical predictions of Eqs.~\ref{fin_locked} and ~\ref{r2} obtained from the self-consistency and Ott-Antonsen analysis. 
 
  Fig.~\ref{figl_alpha} plots  $r_1$ as well as $r_2$ as a function of $K_2$ for different values of asymmetry parameter $\eta$ for a fixed phase lag parameter. To analyze the phase transition, we adiabatically increase and decrease $K_2$ that represents the forward and backward direction, respectively. In the forward direction, initially, all the oscillators are distributed uniformly between [-$\pi$,$\pi$] and frequencies are drawn from the Lorentzian distribution. 
  Whereas in the backward direction, initially the oscillators are distributed into two clusters situated diametrically opposite ends described by $\eta$.  We can see that there exists no forward synchronization for any $K_2$ value, whereas the backward direction yields a first-order transition from the cluster synchronized state to the incoherent state. With an increase in $\eta$ the critical transition from the synchronized to incoherent state shifts towards the right. 
  % I hvae corrected untill here
  Additionally, we have the initial phase distribution of oscillators in such a way that some percentage of the locked state oscillators tries to cancel the effect of each other by distributing them in diametrically opposite ends which effectively renders less number of oscillators in a locked state to contribute in $r_1$, due to which decreasing $\eta$ increases the transition points for both $r_1$ and $r_2$. However, $r_2$ will be more than $r_1$ as it measures the two cluster synchronization which increases when the distribution of oscillators in two clusters becomes more and more symmetric. In the thermodynamic limit, multistable branches are seen as an infinite number of stable partially synchronized states that are obtained through different arrangements of initial conditions in two different clusters, yielding a continuum of abrupt desynchronization transitions.  


  Fig.~\ref{fig2_eta} presents $r_1$ and $r_2$ vs $K_2$ for fixed $\eta=0.9$ and different values of $\alpha$. Here purple, red, yellow, and blue curves correspond to $\alpha=0$, $\pi/6$, $\pi/4$, and $\pi/3.5$ respectively. To demonstrate cluster synchronization, $r_2$ is plotted as a function of $K_2$ for the same parameters. The solid and dashed lines are analytical predictions obtained from Eq.~\ref{fin_locked} and Eq.~\ref{r2} as a consequence of saddle-node bifurcation at critical coupling $K_{2}$. The same behavior is witnessed for $r_2$ but the value of $r_2$ will be higher than $r_1$ due to the existence of cluster synchronization  Fig.~\ref{fig2_eta}.
  
This analysis demonstrates that there exists no forward synchronization as $r_1$, $r_2$ $\rightarrow 0^{+}$ and on inverting Eq.~\ref{r2} we get the forward critical coupling at $\infty$ indicating the absence of synchronization. The backward critical coupling can be calculated by finding the minima of the analytical curve $dK_2/dr_1$ which is not possible to be determined explicitly in a closed form due to the complexity of integral in Eq.~\ref{fin_locked}. Therefore, going back to the simulation results we see that on increasing $\alpha$ the backward transition point $K_{2b}$ shifts towards the right for $r_1$ as well as for $r_2$, i.e., transition to the incoherent state occurs at higher critical coupling value. 
As it happens that a non-zero $\alpha$ value also makes the mean frequency (Eq.~\ref{omega}) non-zero. Consequently, the intrinsic frequency range of the locked oscillators satisfying the relation ($|{\frac{\omega_i - \Omega}{K_2r_1^2}}| \le 1$) no more remains symmetric around 0. For the Lorentzian distribution considered here, this symmetry-breaking around the mean will lead to less number of oscillators satisfying this relation fig.~\ref{fig2_eta}(d).
Consequently, a large $K_2$ value is required to achieve the same $r_1$ value in contrast to the  $\alpha=0$ case.

 Fig.~\ref{Omega vs K2} depicts how the mean rotation frequency of the clusters varies as a function of $K_2$. For $\alpha=0$ the cluster remains stationary for any value of $K_2$ yielding $\Omega=0$, however, for non-zero alpha value $\Omega$ has a linear dependence on $K_2$ with an increasing slope (Eq.~\ref{omega}, Fig.~\ref{Omega vs K2}) even for the mean of the intrinsic frequencies taken at zero. This clearly demonstrates that the phase-lag parameter regulates the rotation frequency of the synchronized clusters which can be adjusted to a desired value by changing $\alpha$, a similar phenomenon is demonstrated for pairwise interactions term with phase-lag parameter but for the global synchronization \cite{lohe2015synchronization}. The crucial difference of the model considered here having triadic interactions from the pairwise interactions is that the former case yields $2-$clusters in contrast to global synchronization in the latter case.

\paragraph{\bf {Conclusion and outlook:}}
To conclude, we have analyzed the effects of phase frustration parameters on the coupled Kuramoto oscillators on simplicial complexes.  We evaluated $r_1$ and $r_2$ order parameters which measure the extent of global and 2-cluster synchronization, respectively. In the absence of any pairwise interactions, $r_1=r_2=0$ remains one stable state for all $K_2$ values. Starting with a set of initial conditions corresponding to a synchronized state, as $K_2$ decreases adiabatically, there exists an abrupt transition to a completely incoherent state. With an increase in the $\alpha$ value, this transition point for both $r_1$ and $r_2$ shifts towards the right. To better comprehend the dynamics of the system, using the Ott-Antonsen dimension reduction approach we derived the time-dependent equations for the order parameters for the even part of the density function, and to close the dynamics for the asymmetric part. We proceed by self-consistency equations which provide a relation between the order-parameters measuring synchronization and asymmetry caused by $K_2$. The analytical results are shown to be in good agreement with the numerical results. Also, we found a dependence of the mean cluster frequency on the coupling strength, which
sheds light on the origin of non-zero mean cluster frequency even for intrinsic frequency distribution having zero means. Ergo, $\alpha$ can be used as a parameter to get the desired mean frequency of the synchronized clusters. For a fixed non-zero $\alpha$ value, multistable states are shown to exist as depicted in Fig.~\ref{figl_alpha}
%where information on the arrangement of clusters is preserved 
similar to the $\alpha=0$ case \cite{Skardal_prl2019}. This model can be further generalized by including phase-lagged pairwise terms as well along with triadic terms for which the self-consistency analysis becomes difficult. Further, there have been recent attempts to analyze coupled Kuramoto oscillators with inertia on simplicial complexes  \cite{sabhahit2023prolonged}. An extension of the current work is to study coupled Kuramoto model with inertia and phase-lag which makes the model more generalized and suitable for wider applications.


\section{\bf Acknowledgement}
SJ gratefully acknowledges SERB Power grant SPF/2021/000136, and useful discussions with Stefano Boccaletti under the VAJRA project VJR/2019/000034. 

\medskip
%\bibliographystyle{plain}
%\bibliographystyle{aip}
\bibliography{ref}

\end{document}