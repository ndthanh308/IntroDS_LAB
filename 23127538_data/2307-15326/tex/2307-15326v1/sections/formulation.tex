\section{Problem formulation} \label{sec:formulation}
We study three tasks in this paper as outlined below.
\paragraph*{Task 1: vanilla staging.}
In task 1, the input is a product image (without staging), and the desired output is a product image with a model generated \textit{relevant} background (stage). The model should generate the entire background as shown in Figure~\ref{fig:tasks1_2}.
% Figure environment removed
\paragraph*{Task 2: retrieval assisted copy-paste staging.}
Task 2 is a simpler version of task 1. Here, we are given a pool of existing product images $\mathcal{P}$, and we need to retrieve a similar product image with staging such that we can copy-paste the staged background from the retrieved image onto the input image as shown in Figure~\ref{fig:tasks1_2}. Image generation (in-painting) is needed to fill in the gaps after copy-pasting (since the input product and the product in the retrieved image are not identical, gaps will be created when we swap products).

\paragraph*{Task 3: image-to-parallax animation.}
In this task, the goal is to take an input image (as shown in Figure~\ref{fig:tasks1_2} for task 2), and create an animation (\emph{i.e.}, sequence of images), where the object in the input image (as in Figure \ref{fig:tasks1_2}) appears to be moving against a stationary but staged background. Such animations are expected to lead to higher user engagement \cite{verizon_media_interactive_ads_report}.
