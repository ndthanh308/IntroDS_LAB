%%
%% This is file `sample-sigconf.tex',
%% generated with the docstrip utility.
%%
%% The original source files were:
%%
%% samples.dtx  (with options: `sigconf')
%% 
%% IMPORTANT NOTICE:
%% 
%% For the copyright see the source file.
%% 
%% Any modified versions of this file must be renamed
%% with new filenames distinct from sample-sigconf.tex.
%% 
%% For distribution of the original source see the terms
%% for copying and modification in the file samples.dtx.
%% 
%% This generated file may be distributed as long as the
%% original source files, as listed above, are part of the
%% same distribution. (The sources need not necessarily be
%% in the same archive or directory.)
%%
%% Commands for TeXCount
%TC:macro \cite [option:text,text]
%TC:macro \citep [option:text,text]
%TC:macro \citet [option:text,text]
%TC:envir table 0 1
%TC:envir table* 0 1
%TC:envir tabular [ignore] word
%TC:envir displaymath 0 word
%TC:envir math 0 word
%TC:envir comment 0 0
%%
%%
%% The first command in your LaTeX source must be the \documentclass command.
\documentclass[sigconf]{acmart}
%% NOTE that a single column version is required for 
%% submission and peer review. This can be done by changing
%% the \doucmentclass[...]{acmart} in this template to 
%% \documentclass[manuscript,screen]{acmart}
%% 
%% To ensure 100% compatibility, please check the white list of
%% approved LaTeX packages to be used with the Master Article Template at
%% https://www.acm.org/publications/taps/whitelist-of-latex-packages 
%% before creating your document. The white list page provides 
%% information on how to submit additional LaTeX packages for 
%% review and adoption.
%% Fonts used in the template cannot be substituted; margin 
%% adjustments are not allowed.

%%
%% \BibTeX command to typeset BibTeX logo in the docs
\AtBeginDocument{%
  \providecommand\BibTeX{{%
    \normalfont B\kern-0.5em{\scshape i\kern-0.25em b}\kern-0.8em\TeX}}}

%% Rights management information.  This information is sent to you
%% when you complete the rights form.  These commands have SAMPLE
%% values in them; it is your responsibility as an author to replace
%% the commands and values with those provided to you when you
%% complete the rights form.
\setcopyright{acmcopyright}
\copyrightyear{2023}
\acmYear{2023}
\acmDOI{XXXXX}

%% These commands are for a PROCEEDINGS abstract or paper.
\acmConference[AdKDD 2023]{}{Long Beach}{CA}
%
%  Uncomment \acmBooktitle if th title of the proceedings is different
%  from ``Proceedings of ...''!
%
%\acmBooktitle{Woodstock '18: ACM Symposium on Neural Gaze Detection,
%  June 03--05, 2018, Woodstock, NY} 
\acmPrice{}
\acmISBN{}


%%
%% Submission ID.
%% Use this when submitting an article to a sponsored event. You'll
%% receive a unique submission ID from the organizers
%% of the event, and this ID should be used as the parameter to this command.
%%\acmSubmissionID{123-A56-BU3}

%%
%% For managing citations, it is recommended to use bibliography
%% files in BibTeX format.
%%
%% You can then either use BibTeX with the ACM-Reference-Format style,
%% or BibLaTeX with the acmnumeric or acmauthoryear sytles, that include
%% support for advanced citation of software artefact from the
%% biblatex-software package, also separately available on CTAN.
%%
%% Look at the sample-*-biblatex.tex files for templates showcasing
%% the biblatex styles.
%%

%%
%% The majority of ACM publications use numbered citations and
%% references.  The command \citestyle{authoryear} switches to the
%% "author year" style.
%%
%% If you are preparing content for an event
%% sponsored by ACM SIGGRAPH, you must use the "author year" style of
%% citations and references.
%% Uncommenting
%% the next command will enable that style.
%%\citestyle{acmauthoryear}

%%
%% end of the preamble, start of the body of the document source.


\usepackage{times}
\usepackage{epsfig}
\usepackage{graphicx}
\usepackage{amsmath}
%\usepackage{amssymb}
\usepackage{subcaption}
\usepackage{comment}

% Include other packages here, before hyperref.



%%%%%%%%%%%%%%%%%%%%%%%%%%%%%%%%%%%%%%%%%%%%%%%%%%%%%%%%%%%%%%%%%%%%%%%%%%%%%%%%
%
%%% IMPORTANT - These next three lines are crucial.
%               (1) PLEASE enter your paper ID (given by CMT) replacing the
%                   '****' right below here with the ID from CMT.
%               (2) Leave the \wacvfinacopy commented out for the submission
%                   version, but UNCOMMENT it for your CAMERA-READY upload.
%               (3) For the camera-ready version, you may be asked to set a
%                   starting page number.  If so, replace the '9876' below with
%                   the starting page number assigned by the publication chair.
 
\def\httilde{\mbox{\tt\raisebox{-.5ex}{\symbol{126}}}}


\begin{document}


%%
%% The "title" command has an optional parameter,
%% allowing the author to define a "short title" to be used in page headers.




\title{Staging E-Commerce Products for Online Advertising \\ using Retrieval Assisted Image Generation}


\begin{comment}
\author{
Yueh-Ning Ku \\
Yahoo Research\\
Sunnyvale\\
{\tt\small yuehning.ku@verizonmedia.com}
\and
Mikhail Kuznetsov \\
Yahoo Research\\
NYC \\
{\tt\small kuznetsov@verizonmedia.com}
\and
Shaunak Mishra \\
Yahoo Research\\
NYC \\
{\tt\small shaunakm@verizonmedia.com}
\and
 Paloma de Juan\\
Yahoo Research\\
NYC \\
{\tt\small pdjuan@verizonmedia.com}
}

\end{comment}

%%
%% The "author" command and its associated commands are used to define
%% the authors and their affiliations.
%% Of note is the shared affiliation of the first two authors, and the
%% "authornote" and "authornotemark" commands
%% used to denote shared contribution to the research.



    

\author{Yueh-Ning Ku}
\authornote{Work done while at Yahoo Research.}
\authornotemark[1]
\affiliation{%
%  \institution{}
  \country{Snap Inc., USA}
 }

\author{Mikhail Kuznetsov}
\authornote{Work done while at Yahoo Research.}
\affiliation{
  \country{Amazon, USA}
%  \institution{Amazon}
}
\author{Shaunak Mishra}
\authornote{Work done while at Yahoo Research.}
\affiliation{
  \country{Amazon, USA}
%  \institution{Amazon}
}


\author{Paloma de Juan}
\affiliation{
  \country{Yahoo Research, USA}
%  \institution{Yahoo Research}
}


\begin{comment}
\author{Joao Soares}
\affiliation{
\country{Yahoo Research, USA}
%  \institution{Yahoo Research}
}

\author{Erfan Eshratifar}
\affiliation{
\country{Yahoo Research, USA}
%  \institution{Yahoo Research}
}

\author{Kapil Thadani}
\affiliation{
\country{Yahoo Research, USA}
%  \institution{Yahoo Research}
}
\end{comment}

\begin{abstract}
Online ads showing e-commerce products typically rely on the product images in a catalog sent to the advertising platform by an e-commerce platform.
In the broader ads industry such ads are called dynamic product ads (DPA). It is common for DPA catalogs to be in the scale of millions (corresponding to the scale of products which can be bought from the e-commerce platform). However, not all product images in the catalog may be appealing when directly re-purposed as an ad image, and this may lead to lower click-through rates (CTRs). In particular, products just placed against a solid background may not be as enticing and realistic as a product staged in a natural environment. To address such shortcomings of DPA images at scale, we propose a generative adversarial network (GAN) based approach to generate staged backgrounds for un-staged product images. Generating the entire staged background is a challenging task susceptible to hallucinations. To get around this, we introduce a simpler approach called copy-paste staging using retrieval assisted GANs. In copy paste staging, we first retrieve (from the catalog) staged products similar to the un-staged input product, and then copy-paste the background of the retrieved product in the input image. A GAN based in-painting model is used to fill the holes left after this copy-paste operation. We show the efficacy of our copy-paste staging method via offline metrics, and human evaluation. In addition, we show how our staging approach can enable animations of moving products leading to a video ad from a product image.
\end{abstract}


\maketitle




\section{Introduction}
Current quantum hardware is unable to carry out universal quantum computations due to the buildup of errors that occur during the computation. 
The magnitude of the individual error is currently above the value that the Threshold Theorem requires in order to kick-start quantum error correction and fault-tolerant quantum computation~\cite[Section 10.6]{nielsen_chuang_2010}. 
Although the experimentally achieved fidelity rates are promising and the error bounds are inching closer to the required threshold, we will have to work for the foreseeable future with quantum hardware with errors that build-up during the computation.  This implies that we can only do a limited number of steps before the output of the computation has become completely uncorrelated with the intended one.

For fault-tolerant quantum computing, we repeat four steps: 
1) We apply a number of single and two-qubit quantum gates, in parallel whenever possible; 
2) We perform a syndrome measurement on a subset of the qubits; 
3) We perform fast classical computations to determine which errors have occurred and how to correct them; 
and, 4) We apply correction terms based on the classical computations.
We then repeat these four steps with a next sequence of gates. 
These four steps are essential to fault-tolerant quantum computing. 


The starting point of this work is to use the four steps outlined above, not to carry out error correction and fault-tolerant computation, but to enhance short, constant-depth, {\em uncorrected} quantum circuits that perform single qubit gates and {\em nearest-neighbor} two qubit gates. 
Since in the long run we will have to implement error-correction and fault-tolerant computation anyhow, and this is done by such a four-step process, why not make other use of this architecture? Moreover, on some of the quantum hardware platforms, these operations are already in place.
Embracing this idea we naturally arrive at the question: what is the computational power of \textit{low-depth} quantum-classical circuits organized as in the four steps outlined above? 
We thus investigate circuits that execute a small, ideally constant, number of stages, where at each stage we may apply, in parallel, single qubit gates and {\em nearest-neighbor} two qubit gates, followed by measurements, followed by low-depth classical computations of which the outcome can control quantum gates in later stages. 
It is not clear, at first, whether such circuits, especially with constant depth, can do anything remotely useful. 
But we will see that this is indeed the case: many quantum computations can be done by such circuits in constant depth. 
By parallelizing quantum computations in this way, we improve the overall computational capabilities of these circuits, as we do not incur errors on qubits that are idle, simply because qubits are not idle for a very long time. 
Furthermore, reducing the depth of quantum circuits, at the cost of increasing width, allows the circuit to be run faster even if errors occur.

The first usage of such a four-step layout, not to do error correction, but to perform computations, can be found in the paradigm of measurement-based quantum computing~\cite{gottesman1999demonstrating,raussendorf2001one,jozsa2006introduction,clark2007generalised}: 
A universal form of quantum computing where a quantum state is prepared and operations are performed by measuring qubits in different bases, depending on previous measurements and intermediate measurements.

\citeauthor{PhamSvore2013} were the first to formalize the four-step protocol for performing computations~\cite{PhamSvore2013}. They included specific hardware topologies by considering two-dimensional graphs for imposing constraints on qubit interactions. In their model, they develop circuits for particularly useful multi-qubit gates, including specifying costs in the width, number of qubits, depth, number of concurrent time steps, size, and total number of non-Identity operations.
As a result, they find an algorithm that factors integers in polylogarithmic depth.
\citeauthor{Browne:2011} showed that the main tool in the work by \citeauthor{PhamSvore2013}, the fan-out gate, can also be replaced by additional log-depth classical computations in the measurement-based quantum computing setting~\cite{Browne:2011}.

More recently, \citeauthor{Cirac:2021} introduced a scheme to implement unitary operations involving quantum circuits combined with Local Operations and Classical Communication ($\mathsf{LOCC}$) channels: $\mathsf{LOCC}$-assisted quantum circuits~\cite{Cirac:2021}. Similarly to the four-step scheme we just described, they allow for a short depth circuit to be run on the qubits, followed by one round of $\mathsf{LOCC}$, in which ancilla qubits are measured and local unitaries are applied based on the measurement outcomes. They show that in this model any 1D transitionally invariant matrix-product state (MPS) with fixed bond dimension is in the same phase of matter as the trivial state. Similar ideas can be found in~\cite{TVV_NonAbelianTopologicalOrder_2022, tantivasadakarn2021long}.

In this work, we introduce a new model, called \textit{Local Alternating Quantum-Classical Computations} ($\LAQCC$). In this model we alternate between running quantum circuits (constrained by locality), ending in the measurement of a subset of qubits, and fast classical computations based on the measurement results. The outcome of the classical computations are then used to control future quantum circuits. We allow for flexibility in this model, by giving different constraints to the power of both the quantum circuits and the classical circuits as well as the number of alternations between them. 
Most attention will be given to $\LAQCC$ containing quantum circuits of constant depth, classical circuits of logarithmic depth and at most a constant number of alternations between them. 
Any circuit constructed in this model is considered to be of constant depth. 
We restrict ourselves to logarithmic depth classical computations, as this is the first natural and non-trivial extension beyond constant-depth classical computations. 
Constant-depth classical computations do however also have an equivalent constant-depth quantum implementation.

The definition of $\LAQCC$ sharpens the original definition of \citeauthor{PhamSvore2013} by adding constraints to the intermediate classical computations. This allows us to bound the power of $\LAQCC$ from above. 

The main result of \citeauthor{Cirac:2021}, that 1D translational invariant MPS with fixed bond dimension can be prepared by $\mathsf{LOCC}$-assisted circuits, relies on local symmetries of the MPS. These symmetries allow them to prepare local states (on a constant number of qubits) and glue them together by doing one round of the appropriate entangling measurement and corrections, after which they run a round of local unitaries to get the desired result. This general scheme for preparing states that exhibit an MPS description with the appropriate local symmetries requires only geometrically local unitaries and one round of measurement and corrections an therefore is accessible in $\LAQCC$. Studying different local symmetries, known as Symmetry Protected Topological (SPT) phases of matter, to find measurement-based constant depth circuits for states is a broad ongoing field of research~\cite{TVV_NonAbelianTopologicalOrder_2022, tantivasadakarn2021long, smith2023deterministic}. 
All these schemes have a $\LAQCC$ implementation.

%$\LAQCC$-circuits also exist for general schemes of preparing local states, based on the local tensors, and gluing them together using one round of entangled measurement and corrections, based on the local symmetry. 
%The main result of \citeauthor{Cirac:2021}, that 1D translational invariant MPS with fixed bond dimension can be prepared by $\mathsf{LOCC}$-assisted circuits, relies heavily on local symmetries of the MPS and as a result also has an equivalent $\LAQCC$ implementation. 
%The corrections applied after the measurement round are local unitaries depending on the local symmetries of the MPS. 

 

%This general scheme of preparing local states, based on the local tensors, and gluing it together by doing one round of entangled measurement and corrections, based on the local symmetry, is accessible in $\LAQCC$.
Note however that \citeauthor{Cirac:2021} also suggest a circuit for the $W$-state.
This circuit uses sequentially and dependent measurement-based corrections of the ancilla qubits. 
These dependent measurements translate to sequential alternations between the quantum and classical circuits and therefore increase the total depth to linear depth, exceeding the constant-depth constraints imposed by $\LAQCC$-circuits. 

We study the power of the $\LAQCC$ model with respect to state preparation, showing that even with only constant quantum-depth and logarithmic classical depth it remains possible to prepare states with long-range entanglement.
Another surprising result is that it is unlikely that $\LAQCC$ circuits are classically simulatable. We show that any instantaneous quantum polynomial-time (IQP) circuit~\cite{Bremner2010,Shepherd2009} has an $\LAQCC$ implementation.
Classical simulation of IQP circuits implies the collapse of the polynomial hierarchy to the third level, which is not believed to be true~\cite{Bremner2017}. Therefore, we expect that $\LAQCC$ circuits are unlikely to be classically simulatable. We bound the power of $\LAQCC$ by showing that it is contained in $\QNC^1$, the class of polynomial-size, log-depth circuits.

Next, we also study the power that intermediate classical calculations can add to quantum computations, by considering a new model that alternates between polynomially many polynomial-depth quantum circuits and unbounded classical computations
We study this model by doing a complexity theoretical analysis, where we draw inspiration from the notions of complexity given by \citeauthor{RosenthalYuen:2022}, \citeauthor{MetgerYuen:2023}, and \citeauthor{Aaronson:2004}.
All three complexity notions are based on the notion of state preparation, instead of more traditional definition of complexity such as the decidability of a computational problem. 
The first two consider classes based on sequences of quantum states preparable by a polynomial-sized quantum circuit, where the circuits are uniformly generated by a computational class, for instance, the class $\mathsf{PSPACE}$, which results in the complexity class $\mathsf{StatePSPACE}$~\cite{RosenthalYuen:2022,MetgerYuen:2023}.
The third notion considers a relative complexity, where the complexity is measured between two given states, and is measured by the number of gates, from a given gate-set, required to transform one state in another state~\cite{Aaronson:2004}. 
For our definition of state preparation complexity, we drop the uniformity constraint from~\cite{RosenthalYuen:2022,MetgerYuen:2023} and define a class as $\mathsf{StateX}$, which refers to states preparable by circuits of type $\mathsf{X}$. 
As an example, if $\mathsf{X} = \QNC^0$, this results in the class $\mathsf{StateQNC^0}$, which is the set of states preparable from the $\ket{0}^n$ state by poly-size constant-depth circuits. 
This notion is similar to the relative complexity from~\cite{Aaronson:2004}, where one state is the  $\ket{0}^n$ state and instead of counting the number of gates we consider the set of states preparable by a fixed number of gates. Using this notion of complexity we show that any state preparable by an $\LAQCC^*$ circuit is also preparable by a $\mathsf{PostQPoly}$ circuit, the class of circuits of polynomial depth with an additional post-selection gate. 

All Clifford circuits have a constant-depth $\LAQCC$ implementation, implying that any stabilizer state can be implemented by a constant-depth $\LAQCC$ circuit, see Section~\ref{sec:clifford_circuits} for a proof of this statement. 
Efficient circuits for stabilizer states have been known already through measurement-based quantum computing. Therefore this paper focuses on the preparation of non-stabilizer states, and as a surprising result we find novel constant-depth protocols for four very natural classes of non-stabilizer states.
Despite the extensive research into these four classes of non-stabilizer states and the many applications of them, no efficient constant- or low-depth state preparation protocols are known yet. We specifically consider these four classes as they are all often used as initial states in other algorithms.

The first state is a uniform superposition over an arbitrary number of states. 
This state finds applications in many quantum algorithms, as they often start with a uniform superposition over multiple states. 
This superposition is often achieved by applying Hadamard gates to every qubit due to its simplicity to prepare. 
Yet, the analysis of many algorithms, such as Shor's algorithm~\cite{Shor:1997}, would benefit from a different initial superposition. 
The circuit to prepare the uniform superposition over an arbitrary number of states uses an exact version of Grover search as a subroutine, that turns a probabilistic circuit, with a known constant probability of success, into a deterministic circuit. 
We use the circuit for preparing a uniform superposition over an arbitrary number of states as a subroutine in the next two quantum state preparation protocols. 

The second state is the $W$-state, the uniform superposition over all computational basis states of Hamming-weight~$1$, a natural long-ranged entangled state that displays a fundamentally nonequivalent type of entanglement from the Greenberger–Horne–Zeilinger state~\cite{WState:2000}, for which $\LAQCC$-type constant-depth circuits were previously known~\cite{PhamSvore2013, Cirac:2021}. 
The $W$-state is often used as benchmark for new quantum hardware~\cite{Haffner2005,Neeley2010,GarciaPerez:2021}. 
A novel way to prepare the $W$-state therefore gives a new way to benchmark different quantum devices with each other. 
A circuit for preparing the $W$-state was given in~\cite{Cirac:2021}, but this implementation requires sequentially alternating measurements followed by local unitaries, which in the $\LAQCC$ model is not considered to be of constant depth. 
We improve this protocol by giving an $\LAQCC$ implementation of the $W$-state, based on a compress-uncompress method that links the one-hot and binary encoding of integers.

The third state considered is the Dicke state, a generalization of the $W$-state, a superposition over all computational basis states with Hamming-weight $k$~\cite{Dicke:1954}. 
Dicke states have relevance in various practical settings.
For instance, for quantum game theory~\cite{zdemir2007}, quantum storage~\cite{Bacon_Compress:2006,Plesch:2010}, quantum error correction~\cite{ouyang2014permutation}, quantum metrology~\cite{toth2012multipartite}, and quantum networking~\cite{prevedel2009experimental}. 
Dicke states have been used as a starting state for variational optimization algorithms, most notably Quantum Alternating Operator Ansatz (QAOA)~\cite{Hadfield2019}, to find solutions to problems such as Maximum k-vertex Cover~\cite{Brandhofer2022,cook2020quantum}.
The ground states of physical Hamiltonians describing one-dimensional chains tend to show a resemblance to Dicke states such as states resulting from the Bethe ansatz, making them an ideal starting state when investigating the ground state behavior of these Hamiltonians~\cite{TDL_BetheAnsatzDerivation:2010,B_ExcitedStateQuantumPhaseTransitions:2013,DickeTransitions:2021}. 
For instance, the algorithm by \citeauthor{van2021preparing}, who give an algorithm to prepare the Bethe ansatz eigenstates of the spin-1/2 XXZ spin chain, starts by first preparing a Dicke state~\cite{van2021preparing}. 
A Dicke-state preparation protocol based on the compress-uncompress methodology used in the $W$-state furthermore finds applications in entanglement distillation, where the entanglement of a large state is concentrated on only a few qubits. 
Efficient deterministic circuits for preparing Dicke states have been proposed by \citeauthor{bartschi2019deterministic}~\cite{bartschi2019deterministic, bartschi2022deterministic_short_depth}. 
They provide a quantum circuit of depth $\mathO(k \log(\frac{n}{k}))$, allowing arbitrary connectivity, to prepare a Dicke state, which they conjecture to be optimal when $k$ is constant. 
In this work, we provide a constant-depth $\LAQCC$ circuit below their conjectured bound already for constant $k$. 
However, this does not directly disprove their conjecture, as we allow for intermediate measurements and classical computations. 
More significantly, we even construct constant-depth $\LAQCC$ circuits for $k = \mathO(\sqrt{n})$ greatly improving their bound.
This construction extends the compress-uncompress method for the $W$-state combined with additional subroutines. 

We continue with a log-depth state preparation protocol for the Dicke-state for arbitrary $k$. 
This protocol implements an efficient transformation between the factoradic number representation and the combinatorial number representation of a positive integer. 
The combinatorial number representation relates directly to the Dicke state. 
The provided efficient transformation between number representation systems might be of independent interest. 

We conclude by modifying our protocol for preparing a Dicke-state to a protocol that prepares quantum many-body scar states in constant-depth. 
These states have low entanglement and longer coherence times than states with similar energy density.
These characteristics make many-body scar states interesting to analyze and relevant within physics.
Many-body scar states appear for instance in the AKLT model~\cite{AKLT:1987,MRBAR:2018,MRB:2018} and different spin models~\cite{SI:2019,MOBFR:2020}.
Known methods for preparing these states have polynomial-depth~\cite{Gustafson:2023}, whereas our circuit has constant depth. 

% We conclude by studying the power that intermediate classical calculations can add to quantum computations. 
% In this study, we define a new model that relaxes constant-depth quantum circuits to polynomial depth quantum circuits, log-depth classical calculations to unbounded classical computations and a constant number of alternations to a polynomial number of alternations. 
% We call this model $\LAQCC^*$. 
% We study this model by doing a complexity theoretical analysis, where we draw inspiration from the notions of complexity given by \citeauthor{RosenthalYuen:2022}, \citeauthor{MetgerYuen:2023}, and \citeauthor{Aaronson:2004}.
% All three complexity notions are based on the notion of state preparation, instead of more traditional definition of complexity such as the decidability of a computational problem. 
% The first two consider classes based on sequences of quantum states preparable by a polynomial-sized quantum circuit, where the circuits are uniformly generated by a computational class, for instance, the class $\mathsf{PSPACE}$, which results in the complexity class $\mathsf{StatePSPACE}$~\cite{RosenthalYuen:2022,MetgerYuen:2023}.
% The third notion considers a relative complexity, where the complexity is measured between two given states, and is measured by the number of gates, from a given gate-set, required to transform one state in another state~\cite{Aaronson:2004}. 
% For our definition of state preparation complexity, we drop the uniformity constraint from~\cite{RosenthalYuen:2022,MetgerYuen:2023} and define a class as $\mathsf{StateX}$, which refers to states preparable by circuits of type $\mathsf{X}$. 
% As an example, if $\mathsf{X} = \QNC^0$, this results in the class $\mathsf{StateQNC^0}$, which is the set of states preparable from the $\ket{0}^n$ state by poly-size constant-depth circuits. 
% This notion is similar to the relative complexity from~\cite{Aaronson:2004}, where one state is the  $\ket{0}^n$ state and instead of counting the number of gates we consider the set of states preparable by a fixed number of gates. Using this notion of complexity we show that any state preparable by an $\LAQCC^*$ circuit is also preparable by a $\mathsf{PostQPoly}$ circuit, the class of circuits of polynomial depth with an additional post-selection gate. 

\paragraph{Summary of results}
\begin{itemize}
    \item We give a new definition of a computational model that captures the power of the four step process: applying a constant number of layers of one- and two-qubit gates; performing a syndrome measurement; perform a fast classical computation determining corrections; apply corrections. We call this model \emph{Local Alternating Quantum Classical Computations}, or $\LAQCC$ for short. In this model we bound the allowed quantum operations, intermediate classical calculations, and number of rounds separately. In Section~\ref{sec:LAQCC_model} we define this model and give a list of operations based on results from literature contained in this computational model. In some of these operations we explicitly use that we allow for multiple, but at most constant, rounds  of corrections.
    \item  We show show that there exist $\LAQCC$ circuits that can not be weakly simulated in Section~\ref{sec:IQP_in_LAQCC}. We further show that for every $\LAQCC$ circuit there exists a $\QNC^1$ circuit simulating it perfectly, in Section~\ref{sec:LAQCC_in_QNC1}.
    \item We introduce a new type computational complexity for preparing states and show that the extension of $\LAQCC$ where we allow a polynomial number of rounds and unbounded classical computation, is contained in $\mathsf{PostQPoly}$, the class of polynomial circuits with post-selection, in Section~\ref{sec:Complexity results}.
    \item We show a protocol to prepare the uniform superposition state of size $q$ in $\LAQCC$ using $\mathO(\ceil{\log_2(q)}^2)$ qubits in Section~\ref{sec:superposition_modulo_q}. 
    \item We show a protocol to prepare the $W_n$ state in $\LAQCC$ using $\mathO(n\log(n))$ qubits in Section~\ref{sec:W_state_in_LAQCC}.
    \item We show two ways of preparing the Dicke-$(n,k)$ state. The first method is in $\LAQCC$, works up to $k = \mathO(\sqrt{n})$, uses $\mathO(n^2\log(n))$ qubits, and is found in Section~\ref{sec:dicke:small_k}. The second method is in $\LAQCC\text{-}\mathsf{LOG}$ (an extension of $\LAQCC$ allowing for logarithmic number of alterations instead of constant), works for any $k$, uses $\mathO(\text{poly}(n))$ qubits, and is found in Section~\ref{sec:Dicke_in_LAQCC_LOG}. 
    \item We extend on our $\LAQCC$ method of generating Dicke-$(n,k)$ states for $k = \mathO(\sqrt{n})$ and show a protocol to generate many-body scar states for a particular Hamiltonian in $\LAQCC$ (Section~\ref{sec:many_body_scar}). 
\end{itemize}
Summarized in a table, we provide the following state generation protocols:
\begin{table}[htb]
\centering
\begin{tabular}{l|l|l|l}
\textbf{State description} & \textbf{Width} & \textbf{Depth} & \textbf{Implementation}\\
\hline 
Uniform superposition mod $q$: $\frac{1}{\sqrt{q}} \sum_{i = 0}^{q-1}\ket{i}$ & $\mathO(\ceil{\log^2 q})$ & $\mathO(1)$ & Section~\ref{sec:superposition_modulo_q}\\

$W$-state: $\frac{1}{\sqrt{n}}\sum_{i = 0}^{n-1}\ket{e_i}$ & $\mathO(n \log n)$ & $\mathO(1)$ & Section~\ref{sec:W_state_in_LAQCC}\\

Dicke-$(n,k)$, $k = \mathO(\sqrt{n})$: $\binom{n}{k}^{-1/2}\sum_{x \in \{0,1\}^n: |x| = k} \ket{x}$ &  $\mathO(n^2\log n)$ & $\mathO(1)$ 
&Section~\ref{sec:dicke:small_k}\\

Dicke-$(n,k)$: $\binom{n}{k}^{-1/2}\sum_{x \in \{0,1\}^n: |x| = k} \ket{x}$ & $\mathO(\text{poly}(n))$ & $\mathO(\log n)$ &Section~\ref{sec:Dicke_in_LAQCC_LOG}\\

QMBS: $\ket{S_k} = \frac{1}{k! \sqrt{\mathcal N(n,k)}}(Q^\dagger)^k \ket{\Omega}$ &  $\mathO(n^2\log n)$ & $\mathO(1)$  &  Section~\ref{sec:many_body_scar}
\end{tabular}
\caption{Summary of state preparation protocols given in this paper.}
\label{tab:sate_prep}
\end{table}
In the entry for the quantum many-body scar state $Q$ denotes the raising operator and $\mathcal N(n,k)=\binom{n-k-1}{k}$. 
Section~\ref{sec:many_body_scar} will provide more details on the variables and the implementation. 

\paragraph{Organization of the paper}
\noindent We first introduce relevant preliminaries in Section~\ref{sec:preliminaries}. 
In Section~\ref{sec:LAQCC_model} we formally define the class of Local Alternating Quantum-Classical Computations ($\LAQCC$). We also show that any Clifford circuit can be implemented in constant depth $\LAQCC$ (a result based on a result from measurement-based quantum computing~\cite{jozsa2006introduction}). 
This result allows us to give many useful multi-qubit gates and routines in Section~\ref{sec:gates_created_in_LAQCC}. 
Beyond that we show that constant depth $\LAQCC$ circuits are contained in $\QNC^1$ and that any $\mathsf{IQP}$ circuit has an $\LAQCC$ implementation.
We conclude this section with an analysis of a more powerful instantiation of $\LAQCC$ and show an inclusion with respect to the class $\mathsf{PostQPoly}$, which is the class of circuits of polynomial depth with one additional post-selection gate. 
In Section~\ref{sec:state_prep_in_LAQCC} we give $\LAQCC$ circuit implementations for preparing the uniform superposition over an arbitrary number of states, the $W$-state and the Dicke state up to $k = \mathO(\sqrt{n})$. We furthermore give a log-depth circuit implementation for preparing the Dicke state for any $k$. We conclude by showing a $\LAQCC$ circuit for generating many body scar states of a particular type of Hamiltonian.


\section{Related Work}
\label{appsec: related work}
Bayesian causal discovery literature has primarily focused on inference in linear models with closed-form posteriors or marginalized parameters. Early works considered sampling directed acyclic graphs (DAGs) for discrete~\cite{cooper1992bayesian, madigan1995bayesian, heckerman2006bayesian} and Gaussian random variables~\cite{friedman2003being, tong2001active} using Markov chain Monte Carlo (MCMC) in the DAG space. However, these approaches exhibit slow mixing and convergence~\cite{eaton2012bayesian,grzegorczyk2008improving}, often requiring restrictions on number of parents~\cite{kuipers2017partition}. %Alternative exact dynamic programming methods are limited to small settings~\cite{koivisto2012advances}. 

Recent advances in variational inference~\cite{zhang2018advances} have facilitated graph inference in DAG space, with gradient-based methods employing the NOTEARS DAG penalty \cite{zheng2018dags}.\cite{annadani2021variational} samples DAGs from autoregressive adjacency matrix distributions, while \cite{lorch2021dibs} utilizes Stein variational approach \cite{liu2016stein} for DAGs and causal model parameters. \cite{cundy2021bcd} proposed a variational inference framework on node orderings using the gumbel-sinkhorn gradient estimator \cite{mena2018learning}. \cite{deleu2022bayesian,nishikawa2022bayesian} employ the GFlowNet framework \cite{bengio2021gflownet} for inferring the DAG posterior. Most methods, except\cite{lorch2021dibs} are restricted to linear models, while \cite{lorch2021dibs} has high computational costs and lacks DAG generation guarantees compared to our method.
% at least quadratic scaling complexity, both with respect to the number of nodes (due to the DAG penalty) as well as number of posterior samples. Our proposed approach instead has linear complexity with respect to number of posterior samples and does not require any additional DAG penalty.     

In contrast, \emph{quasi-Bayesian} methods, such as DAG bootstrap \cite{friedman2013data}, demonstrate competitive performance. DAG bootstrap resamples data and estimates a single DAG using PC \cite{spirtes2000causation}, GES \cite{chickering2002optimal}, or similar algorithms, weighting the obtained DAGs by their unnormalized posterior probabilities. Recent neural network-based works employ variational inference to learn DAG distributions and point estimates for nonlinear model parameters \cite{charpentier2022differentiable,geffner2022deep}.
\section{Problem formulation} \label{sec:formulation}
We study three tasks in this paper as outlined below.
\paragraph*{Task 1: vanilla staging.}
In task 1, the input is a product image (without staging), and the desired output is a product image with a model generated \textit{relevant} background (stage). The model should generate the entire background as shown in Figure~\ref{fig:tasks1_2}.
% Figure environment removed
\paragraph*{Task 2: retrieval assisted copy-paste staging.}
Task 2 is a simpler version of task 1. Here, we are given a pool of existing product images $\mathcal{P}$, and we need to retrieve a similar product image with staging such that we can copy-paste the staged background from the retrieved image onto the input image as shown in Figure~\ref{fig:tasks1_2}. Image generation (in-painting) is needed to fill in the gaps after copy-pasting (since the input product and the product in the retrieved image are not identical, gaps will be created when we swap products).

\paragraph*{Task 3: image-to-parallax animation.}
In this task, the goal is to take an input image (as shown in Figure~\ref{fig:tasks1_2} for task 2), and create an animation (\emph{i.e.}, sequence of images), where the object in the input image (as in Figure \ref{fig:tasks1_2}) appears to be moving against a stationary but staged background. Such animations are expected to lead to higher user engagement \cite{verizon_media_interactive_ads_report}.

\lstMakeShortInline[columns=fixed]@
% Figure environment removed
\lstDeleteShortInline@

In this section, we describe how we collect examples for learning repair strategies without any version-controlled data. Specifically, we first detect \safeprogs and corresponding witnesses using \sawitnessfull (witnesses are sanitizers and guards that protect from vulnerabilities)  in Section~\ref{subsec:sa-witness}. Using these witness annotations, we generate unsafe programs and \textit{edits} from the \safeprog using a \textbf{witness-removal} step (Section ~\ref{subsec:witness-removal}). In the following, we define terminology for the \astree  data-structure we operate on. 


\astree refers to the abstract syntax tree representation of programs, augmented with data flow edges and annotations for sources, sinks, sanitizers, guards, witnesses etc. 
An \astree is a five-tuple 
$\langle \mathcal{N},\mathcal{V},\mathcal{T},\mathcal{E}, \mathcal{A} \rangle$, where:
\begin{enumerate}
\item
$\mathcal{N}=\{\mathit{id}_0,\ldots\mathit{id}_n\}$  is a set of nodes, where  $\mathit{id_i}\in\mathbb{N}$ for 
$ 0 \leq i \leq n$.
\item
$\mathcal{V}$ is a map from nodes to program snippets
represented as strings. For a node $n$, we have that $\mathcal{V}(n)$ is a string representing the code snippet associated with $n$
\item
$\mathcal{T}$ is a map from nodes to their types defined by 
 \sa~\cite{codeqlast}. For example, \callexpr is the type of a node representing a function call, \indexexpr is the type of a node representing an array index, and \blockstmt is the type of a node representing a basic block of statements.
\item
$\mathcal{E}$ is a set of directed edges.
Each edge is of the form $(n_1,n_2,\edgetype,z)$, where
$n_1$ is a source node, $n_2$ is a target node, 
$\edgetype \in \{\T{SynParent}, \T{SynChild}, \T{SemParent},
\T{SemChild} \}$ denotes the relationship from 
$n_1$ to $n_2$, as one of syntactic parent, syntactic child, semantic parent or semantic child,
and $z\in\mathbb{Z}$ is the index of $n_2$ among $n_1's$ children if this edge is a child edge, and $-1$ if the edge is a parent edge. 
\item
$\mathcal{A}$ is a set of annotations associated with each node. The annotations are from the set $\{\T{source},
\T{sink},\T{sanitizer},\T{guard}$,\T{witness}\}. We also refer to annotations using predicates or relations. For instance, for a node $n$, if an annotation  $\T{source}$ is present, we say that
the predicate $\T{source}(n)$ is true.
\end{enumerate}

%\setlength{\grammarindent}{5em} % increase separation between LHS/RHS

% Figure environment removed



A {\em traversal} or a {\em path} in an \astree is a sequence of edges $e_0,\ldots,e_{i-1},e_i,\ldots ,e_k$ such that the target node of $e_{i-1}$ is also the source node of $e_i$, for all $i\in\{1,\ldots,k\}$. That is, $e_{i-1}$ is of the form $(\_,n,\_,\_)$ and $e_i$ is of the form $(n,\_,\_,\_,\_)$. The source node of $e_0$ is the source of this path and the target node of $e_k$ is the target of the path.


\lstMakeShortInline[columns=fixed]@
%Note that these additional edges can capture long-range dependencies in programs. E.g. edge 4 in Figure ~\ref{fig:unsafememberex} links two nodes across the function boundaries. 
Figure~\ref{fig:example1-pdg} depicts a partial \pdg corresponding to the unsafe program in Figure~\ref{fig:unsafememberex}. Each oval corresponds to an \astree-node containing a type $\tau$ and an associated value. The dark edges denote the syntactic child edges. For example, the oval with value @foo(data)@ is an \astree-node with type \callexpr and has two children -- @foo@ and @data@, both with the type \varexpr. 
%Similarly, the \blockstmt node on the top refers to the function body between Line~\ref{lst:line:handlers-run} and Line~\ref{lst:line:handlers-run-end} in Figure ~\ref{fig:unsafememberex}. As the body of a function block can contain a variable number of children, we link to @handlers[callerId](data);@ as the k-th child of the \blockstmt. 
The semantic child edges are at the bottom in cyan. These edges correspond to the ones depicted in cyan in Figure ~\ref{fig:unsafememberex}. 
\lstDeleteShortInline@

%TODO:FIX THIS

%With this simplification, 
If $\prog$ is an \pdg then
we use  $\prog.\mathtt{source}$ to denote the source node, $\prog.\mathtt{sink}$ to denote the sink node, and $\prog.\mathtt{witness}$ to denote the witness node.
If the program has several sources, sinks and sanitizers then we generate a separate \pdg for each $(\mathtt{source},\mathtt{witness},\mathtt{sink})$ triple.
For a node $n$, its syntactic parent is $n.\mathtt{parent}$, syntactic children are $n.\mathtt{children}$, semantic parent is $n.\mathtt{semparent}$, and semantic children are $n.\mathtt{semchildren}$.

%\input{ql.tex}

\subsection{Static Analysis Witnessing}
\label{subsec:sa-witness}

\newcommand{\DMethodjudge}[1]{\texttt{#1(}\checknextarga}

% Figure environment removed

%\naman{TODO - sell this more as technique to work with any \sa tool ; our master query is a general framework implemented in \codeql that can work for any vulnerability -- easily extendable to other languages }
In this section, we show how to repurpose \sa tools to generate witnesses.
\sa tools perform dataflow analysis to check for rule-violations in programs. They use pattern matching to identify known sources, sinks, sanitizers, and guards. For commercial tools, these patterns are implemented (and continuously updated) manually by developers and encode this domain knowledge. Next, 
%these patterns are used to detect sources, sinks, sanitizers, and guards in programs and
\sa checks if there exists a flow between a source and a sink that does not cross a sanitizer or guard. We capture this formally in Figure~\ref{fig:judgements} (top two rules), and explain the notation used in it below.

\sa tools encode domain knowledge about the vulnerability by annotating nodes as \T{Source}, \T{Sink}, \T{Sanitizer}, and \T{Guard}. %These relations operate on the set of dataflow nodes in the programs.
So \DMethod{Source}{\I{n}}\ is true iff the node \I{n} is a \textit{source} node for a vulnerability. Next, \sa tools perform dataflow analysis by defining the relation \DMethod{SemChild}{$n_1$}{$n_2$}\ which is true iff there is a \taintpropedge between $n_1$ and $n_2$. Then the \DMethod{Vulnerability}{$n_1$}{$n_2$}\ relation can be defined as:
\begin{enumerate}
    \item $n_1$ and $n_2$ are source and sink nodes (\DMethod{Source}{$n_1$}\ and \DMethod{Sink}{$n_2$}\ are true)
    \item There exists a \textit{path} between $n_1$ and $n_2$ which is free of sanitizers or guards (\DMethod{SanGuardFree*}{$n_1$}{$n_2$}\ is true). A path is free of sanitizers and guards iff every \textit{edge} in the \textit{path} is free of sanitizers and guards. An edge between $n_1$ and $n_2$ is considered free of sanitizers and guards (\DMethod{SanGuardFree}{$n_1$}{$n_2$}\ is true) iff $(n_1, n_2, \_, \T{SemChild}) \in \mathcal{E}$ and neither of $n_1$ or $n_2$ is a sanitizer or a guard
\end{enumerate}

Here, we make the following observation - \emph{this domain knowledge present in these annotations and relations is helpful beyond just detecting vulnerabilities}. For instance, simply using the sanitizer relation allows us to query the different kinds of sanitizers domain experts have specified. We use this observation to discover \emph{\safeprogs} i.e., programs having a source, sink, and a sanitizer or guard that \textit{blocks} the \taintprop or, in simpler terms, make the program safe. In addition, we also detect the corresponding sanitizers or guards in the programs and refer to them as \textit{witnesses} because they serve as the evidence of making the program safe. We call this procedure \sawitnessfull (abbreviated as \sawitness). 
We define this as the \T{Witness} relation in Figure~\ref{fig:judgements} (bottom two rules). Specifically, \DMethod{Witness}{$n_1$}{$n_3$}{$n_2$}\ is defined as:
\begin{enumerate}
    \item $n_1$ and $n_2$ are source and sink nodes (\DMethod{Source}{$n_1$}\ and \DMethod{Sink}{$n_2$}\ are true)
    \item There exists a node $n_3$ such that it satisfies \DMethod{SanGuardInMid}{$n_1$}{$n_3$}{$n_2$}. \DMethod{SanGuardInMid}{$n_1$}{$n_3$}{$n_2$}\ is true iff there exists a \T{SemChild}
    %\naga{notation for flow inconsistent with (2) above} 
    path between $n_1$, $n_3$, between $n_3$ and $n_2$, with the additional constraint of $n_3$ being a sanitizer or guard. 
\end{enumerate}

The difference between the \T{Vulnerability} relation (which \sa populates) and \T{Witness} relations (which we want to find) is highlighted in {\color{red} red} and {\color{ForestGreen} green}. Notice that while defining the \T{Witness} relation, we simply use the existing relations that define the \T{Vulnerability} relation. Thus, we argue that \sawitness can be implemented on top of \sa by using the intermediate relations that \sa is computing.
%for every pair of source and sink, they track taint through a taint-flow analysis. If there is a flow from a source to a sink that does not go through a sanitizer or guard, then the source-sink pair is reported as vulnerable.

%We make the following observation - \emph{the patterns defined by experts encodes domain knowledge which can be used for use cases beyond just detecting vulnerabilities}. For instance, we can use the sanitizer patterns to search for all sanitizers in source-code. In this work, we use this idea to detect \safeprogs, which we define as programs having a source, sink, and a sanitizer or guard that blocks the \unsure{flow} or in other words, makes the program safe.  \naman{highlighted part of Figure somethings shows the difference between semantics of witnessing vs traditional semantics}

%We realize the following -- the set of patterns of sources, sinks, and sanitizers are useful beyond detecting vulnerabilities. We override the existing static analysis query that detects unsafe programs and use these encoded sanitizers for detecting sanitizers and guards in programs. Specifically, in the existing query that detects unsafe programs, we modify the taint-propagation steps to propagate taints through sanitizers and guards and then use static analysis to then find these dataflows containing sanitizers and guards. Thus, we can directly find the safe programs containing these \textit{witnesses} of safety. 
%Once such a dataset is collected, we use these witnesses to convert safe  to unsafe  and thus obtain paired examples for learning repair strategies (Section~\ref{subsec:witness-removal}). 

\lstMakeShortInline[columns=fixed]@
%We instantiate our \sawitness technique using \codeql~\cite{a}. It is an open-source \sa tool that allows implementing custom static analysis as queries in a high-level object-oriented extension of datalog. These queries usually contain a \Verb|select from where| statement that allows querying the program database. \codeql maintains these patterns of sources, sinks, sanitizers, and guards using \Verb"Configuration" classes. Consider an example of a simplified \Verb"Configuration" for \xss vulnerability in Figure~\ref{fig:configuration}. It defines a set of predicates @isSource@, @isSink@, @isSanitizer@, and @isGuard@. These predicates are written manually by \codeql authors and improved through rich community support\footnote{\url{https://github.com/github/codeql}}. With this configuration, vulnerabilities are reported by selecting source-sink pairs such that the @cfg.hasFlow@ predicate is true for the source, and the sink. This predicate is internally defined by \codeql and uses the patterns defined in the configuration to check for the presence of vulnerability-causing dataflows. %\spsays{Showing corresponding programs will be useful}

%Now, we demonstrate the static-analysis-witnessing approach for collecting examples of \safeprog and witnesses in Figure~\ref{fig:safe-configuration}. Specifically, we inherit from the existing configuration, using the same @isSource@ and @isSink@ predicates while overriding the @isSanitizer@ and @isGuard@ predicates to @none()@. This ensures that all the source and sink pairs are detected independent of the presence of sanitizers/guards between them. Finally, to detect our witnesses, we define the @isWitness@ predicate which uses the @isSanitizer@ and @isGuard@ predicates from the original configuration. Specifically, witnesses are defined as sanitizers/guards that lie between a source-sink pair. Finally, to report \safeprog and witnesses, the @cfg.hasFlow@ predicate is used to select all valid source-sink pairs and the corresponding witnesses are detected via the @isWitness@ predicate. Note that Figure~\ref{fig:configuration-vs-safe-configuration} depicts the key idea behind our approach in a simplified view. In practice, additional measures need to block the taint propagation internally and we share the actual \codeql queries used as part of the Appendix~\ref{app:codeql-queries}.


\subsection{Witness Removal}
\label{subsec:witness-removal}

We obtain \safeprogs and witnesses by applying \sawitness to a snapshot of a codebase. Recall that the witnesses block the flow between a source and a sink and thus help make programs  \textit{safe}. Hence, removing these witnesses will make the programs unsafe. Recall also that the witnesses are either sanitizing functions of the form @sanitize(taintedVar)@ or guards of the form @if checkSafe(taintedVar) {executeSink(taintedVar)}@. %Usually, they are used only for ensuring the safety of programs and are not critical to the functionality of programs. Therefore, 
We implement witness-removal perturbations  that precisely remove the guard-checks and sanitizer-functions. Note that our goal here is to generate unsafe programs and corresponding edits that enable learning repair strategies that insert such witnesses. So, while we generate the unsafe programs by perturbation, they should look structurally similar to natural unsafe programs written by the developers, otherwise the repair strategies learned on this artificially generated data through perturbations would not generalize to code in the wild. 
%At the same time, minor syntactic-semantic issues in parts of unsafe programs not directly relevant to the vulnerability or repair do not impact learning.
\lstDeleteShortInline@

% Figure environment removed

\lstMakeShortInline[columns=fixed]@

\input{witnessremoval.tex}

We use \rmSan and \rmGuard functions to programmatically remove the witnesses. A high-level sketch of these functions is illustrated in Figure~\ref{fig:remove-functions}. The functions use the structure of the corresponding \astree (node types $\tau$) to decide how to remove witnesses. Consider the \rmGuard function. It first computes the parent (\witnesspar) and grand-parent (\witnessparpar) of the witness guard condition. Then if the type of \witnesspar is \ifstmt (i.e., program is of the form @if (witness) body@ then we modify the \astree edge from \witnessparpar and \witnesspar to instead point to the body of the \ifstmt (index 1 child is body of \ifstmt). Similarly, if the type of \witnesspar is \binaryexpr with operator @&&@ (i.e. of the form @if (otherCond && guard)@ or @if (guard && otherCond)@) then we again modify the edge from \witnessparpar and \witnesspar to instead point to the non-guard child of \binaryexpr (@otherCond@ in the example). Note that since \binaryexpr has 3 children, the index of non-guard child is index of guard-child subtracted from 2. 
Figure~\ref{fig:witness-removal} depicts this removal on the \astree level, where the syntactic edges in red are removed and the syntactic edges in green are inserted.
In the end, the functions returns a tuple of the \pdg of the unsafe program ($\prog_{unsafe}$), \pdg of the safe program ($\prog_{safe}$)
and an edit object (\edit) which stores


\begin{enumerate}
    \item \astree for the removed witness (referred to as \editprog)
    \item location in the \pdg where the witness is removed (referred to as editloc
    %\naga{shouldn't it be editloc to be consistent with (1)?} 
    or \editloc)
    %\item an enum (\insertsc or \replace) depending on whether \concedit is inserted or replaced 
\end{enumerate}

Since $\prog_{unsafe}$ and edit-object can generate the safe program, we only propagate the unsafe programs and edits as the output of this step. Applying \rmGuard function to the safe program in Figure~\ref{fig:safememberex} removes the \ifstmt on Line~\ref{lst:line:fix-start} while preserving the @handlers[callerId](data);@ statement and in fact produces the unsafe program in Figure~\ref{fig:unsafememberex}. Additionally, it  returns the removed witness guard  @if handlers.hasOwnProperty(data.id){ ... }@ as the \editprog and \blockstmt (blue oval in Figure~\ref{fig:example1-pdg}) as the edit location \edit.editloc. Figure~\ref{fig:example1-editprog} shows the \astree for the \editprog containing the \ifstmt. 
The dashed line and dark circle correspond to the \textit{removed} \astree edge between the \blockstmt and the \expr @handlers[callerId](data)@. 

Note that Figure~\ref{fig:remove-functions} provides a high-level sketch of witness-removal and elides over implementation details that are required to make it work for real \js programs. We discuss these issues in the implementation section (Section~\ref{subsec:impl:witness-removal}).% and include the full implementation as part of supplementing source code\naga{we should make sure we are doing these, else remove this sentence}. 
%. In practice, we need implement such decisions more carefully to cover other traditional cases in which guards occur and we document them in the supplementing source code.
\lstDeleteShortInline@

%\naman{add examples $\dots$ } \spsays{do we re-run codeql on this generated bad program? -- NO (naman)}


\section{Method} \label{method_hybridaugment}
In this section, we formally define the problem, motivate our work and then present our proposed techniques.


\subsection{Preliminaries}
Let $\mathcal{F}(x;W)$ be an image classification CNN trained on the training set $\mathcal{T}_\text{train} = (x_{i}, y_{i})^{N}_{i=1}$  with $N$ samples, where $x$ and $y$ correspond to images and labels. The clean accuracy (CA) of $\mathcal{F}(x;W)$ is formally defined as its accuracy over a clean test set $\mathcal{T}_\text{test} = (x_{j}, y_{j})^{M}_{j=1}$. Assume two operators ${A}(\cdot)$ and ${C}(c, s)$ that adversarially attacks or corrupts a given set of images with the corruption category $c$ and severity $s$, respectively.  Let $A\mathcal{T}_\text{test}$ and $C\mathcal{T}_\text{test}$ be the adversarially attacked and corrupted versions of $\mathcal{T}_\text{test}$, and let $\mathcal{F}(x;W)$ have a robust accuracy (RA) on $A\mathcal{T}_\text{test}$ and a corruption accuracy (CRA) on $C\mathcal{T}_\text{test}$. 
The aim is to fit $\mathcal{F}(x;W)$ such that the model gains robustness (\ie. increased RA and CRA compared its the baseline version), while retaining (or improving) the clean accuracy of its baseline version trained without robustness concerns.


\noindent \textbf{What we know.} Our work builds on the following crucial observations: i) CNNs favour high-frequency content \cite{wang2020high}, ii) adversaries and corruptions often reside in high-frequency \cite{wang2020towards}, iii) images are dominated by low-frequency \cite{Saikia_2021_ICCV} and iv) models relying on low-frequency components are more robust \cite{li2022robust,wang2020towards}. The robustness-accuracy trade-off is visible; low-frequency reliant models are more robust, but tend to miss out on clean accuracy brought by the high-frequency components. 

\subsection{HybridAugment}
We hypothesize that a \textit{sweet spot} in the robustness-accuracy trade-off can be found. Unlike the \textit{hard} approaches that completely rule out the reliance on high-frequency components (i.e. low-pass filters), we propose to \textit{reduce} the reliance on them. To this end, we adopt a data augmentation approach that aims to diversify $\mathcal{T}_\text{train}$ by an operation $\mathcal{HA(\cdot)}$. Keeping the strong relation intact between labels and low-frequency content (i.e. labels come from low-frequency-component image), we propose to swap high and low-frequency components of images in a batch on-the-fly. Unlike \cite{mukai2022improving}, we \textit{do not} restrict the images to belong to the same class; this diversifies the training distribution even further while preserving the image semantics. We call this basic version of our approach \textit{HybridAugment}, which corresponds to: 
%
\begin{equation} \label{hybrid_augment_paired}
    \mathcal{HA_{P}}(x_{i}, x_{j}) = \mathcal{LF}(x_{i}) + \mathcal{HF}(x_{j})
\end{equation}
%
where $x_{i}$ is the input image and $x_{j}$ is a randomly sampled image from the whole training set, which we simply sample from the mini batch at each training iteration in practice. $\mathcal{HF}$ and $\mathcal{LF}$ operators select the high and low-frequency components of an input image, for which we use:
%
\begin{equation} \label{eq:cutoff}
\begin{split}
    \mathcal{LF}(x) = GaussBlur(x) \\
    \mathcal{HF}(x) = x - \mathcal{LF}(x)
    \end{split}
\end{equation}
%
where $GaussBlur$ is used as a low-pass filter. Note that a similar outcome is possible by using Discrete Fourier Transforms (DFT), swapping the frequency bands and then applying Inverse DFT (IDFT). We find the gaussian blur operation to be faster and better in practice. 


Inspired from \cite{chen2021amplitude}, in addition to the image-pair scheme in Eq.~\ref{hybrid_augment_paired}, we propose a single image variant of \textit{HybridAugment}. In the single image variant, instead of combining two images, $x_i$ and $x_{j}$ are obtained by applying randomly sampled augmentations to a single image. The single image variant $\mathcal{HA_{S}}$ can therefore be defined as 
%
\begin{equation} \label{hybrid_augment_single}
    \mathcal{HA_{S}}(x_{i}) = \mathcal{LF}(Aug(x_{i})) + \mathcal{HF}(\hat{Aug}(x_{i}))
\end{equation}
%
where $Aug$ and $\hat{Aug}$ correspond to two sets of randomly sampled augmentation operations. Note that paired and single versions can work in tandem ($\mathcal{HA_{PS}}$), and actually outperform single or paired image versions. 


\subsection{HybridAugment++}


The frequency analysis is a vast literature, however, two core aspects often stand out; frequency-band analysis (i.e. low, high) and the decomposition of signals into amplitude and phase. \textit{HybridAugment} covers the former and shows competitive results in various benchmarks (see Section \ref{sec:exp_hybridaugment}). The latter is investigated in $\mathcal{APR}$ \cite{chen2021amplitude}, where phase is shown to be the more relevant component for correct classification, and training models based on their phase labels and swapping amplitude components of images randomly lead to more robust models. Note that frequency-band and phase/amplitude discussions are arguably orthogonal, since frequency, phase and amplitude provide distinct characterizations of a signal: intuitively speaking, frequency, phase and amplitude can be seen as the separation of visual patterns in terms of scale, location and significance. 


We hypothesize these two approaches can be complementary; a model reliant on low-frequency and spatial information (i.e. phase) can further improve robustness. Inspired by the successes of cascaded augmentation methods \cite{hendrycks2019augmix,wang2021augmax,calian2022defending}, we unify these two core aspects into a single, hierarchical augmentation method. We refer to this method as \textit{HybridAugment++} and define its paired version as:
%
\begin{equation}
  \mathcal{HA_{P}}^{++}(x_{i}, x_{j}, x_{z}) = \mathcal{APR_{P}}(\mathcal{LF}(x_{i}), x_{z}) + \mathcal{HF}(x_{j})
\end{equation}
%
where $x_{i}$, $x_{j}$ and $x_{z}$ are images sampled from the same batch. Here, $\mathcal{APR_{P}}$~\cite{chen2021amplitude} is defined as
\begin{equation}
    \mathcal{APR_{P}}(x_{i}, x_{z}) = \mathcal{IDFT}(A_{x_{z}} \otimes e^{i. P_{x_{i}}}) \\
\end{equation}
%
where $\otimes$ is element-wise multiplication, $A$ is the amplitude and $P$ is the phase component. Similar to $\mathcal{HA}$ and $\mathcal{APR}$, we also define a single-image version of \textit{HybridAugment++} as
%
\begin{equation}
 \mathcal{HA_{S}}^{++}(x_{i}) = \mathcal{APR_{S}}(\mathcal{LF}(Aug(x_{i}))) + \mathcal{HF}(\hat{Aug}(x_{i}))
\end{equation}
%
where $\mathcal{APR_{S}}$~\cite{chen2021amplitude} is defined as
%
\begin{equation}
\mathcal{APR_{S}}(x_{i}) = \mathcal{IDFT}\left(A_{\bar{Aug}(x_{i})} \otimes e^{i. P_{\overline{Aug}\left(x_{i}\right)}}\right)    
\end{equation}
%
where $Aug$, $\hat{Aug}$, $\bar{Aug}$ and $\overline{Aug}$ are different sets of randomly sampled augmentation operations. Note that we essentially propose a framework; one can use different single and paired image augmentations, either individually or together, and can still achieve competitive results (see ablations in Section \ref{sec:exp_hybridaugment}). There are also other alternatives, such as swapping phase/amplitude first and then performing $\mathcal{HA}$, but we observe poor performance in practice; dividing the phase component into frequency-bands is not interpretable as frequencies of the phase component are not well defined. The pseudo-code of our methods can be found in the supplementary material.




\section{Experimental Results}\label{sec:results}
    \subsection{General Results}
        The basic ResSAN model is used to determine reference results which our expanded model can be compared to as it is structurally similar to ResLAN but does not possess the Lidar adaptive components of it. Further, we compare with the full-size PackNet-SAN and the unmodified NLSPN architecture. 
        As it can be seen from Tab.\,\ref{tab:sota-results}, our LiDAR-adaptive ResLAN achieves competitive performance compared to state-of-the-art standard depth completion methods, which are specialized to the unfiltered 64-beam-LiDAR. The performance differences are in the range of a few centimetres in terms of MAE, which is acceptable given the practical advantage that ResLAN can generalize to different beam patterns as will be shown below.

        Furthermore, we compared the architectures for a set of three different input types that contained 64, 32 or 16 LiDAR channels using both filter types on the metrics from the KITTI benchmark. The NLSPN model was trained for the standard depth completion task and then evaluated with different input data. As for the ResSAN models, we trained one model for each input type and tested it for the corresponding one which serve serve as the \emph{Baseline} in Tab.\,\ref{tab:overall-results}. Our ResLAN model was jointly trained for all three settings. As listed in Tab.\,\ref{tab:overall-results}, the ResLAN models outperform the challenging baseline in all metrics for FOV filtering and all but one for sparse filtering. This implies that our LiDAR adaptive model is able to outperform dedicated models in case of very sparse input depth. Fig.\,\ref{fig:comp-plot} shows this is indeed the case for 32 and even more for 16 channels. For FOV-filtered inputs with 16 channels, the ResLAN exhibits approx. $10\%$ smaller MAE than the baseline. As for the NLSPN, it becomes apparent that it is not capable of generalizing to other input types since it shows clearly worse results. The difference is especially pronounced for the FOV filtering where on average more than every fourth predicted pixel is more than $25 \%$ deviating from the ground truth\,($\delta_{1.25}$). Therefore, using a weight-adapting network in combination with differently filtered input depths allows us to train models that outperform their non-adaptive counterparts.

        \begin{table}[]
            \centering
    	    \small
            \vspace{0.4cm}
            \caption{\textbf{Depth estimation result for standard depth completion} when the ResSAN model was only trained for 64 channels and the ResLAN model for multiple tasks. The PackNet-SAN and NLSPN models were trained with the setup that was also used for our model architecture.}
            \footnotesize
            \setlength{\tabcolsep}{5pt}
            \begin{tabular}{@{}lrrrrl@{}}
            \toprule
            \multicolumn{6}{c}{\textbf{Standard LiDAR Depth Completion}}                                                                                                                         \\ \midrule
            \multicolumn{1}{l|}{Method}          & RMSE $\downarrow$            & MAE  $\downarrow$            & iRMSE $\downarrow$             & iMAE $\downarrow$ & $\delta_{1.25}$ $\uparrow$ \\
            \multicolumn{1}{l|}{}                & \multicolumn{1}{l}{{[}mm{]}} & \multicolumn{1}{l}{{[}mm{]}} & \multicolumn{1}{l}{{[}1/km{]}} & {[}1/km{]}        &                            \\ \midrule
            \multicolumn{1}{l|}{PackNet-SAN}     &  914                            &  298                            &  2.78                              &  1.4                 &  99.65 \%                          \\
            \multicolumn{1}{l|}{NLSPN}           &  \textbf{889}                            &   \textbf{263}                           &  \textbf{2.62}                              &   \textbf{1.3}                &   \textbf{99.61} \%                         \\ \midrule
            \multicolumn{1}{l|}{ResSAN (Ours)}   & 948                             &  275                            &  2.75                              &    1.4               &   99.58 \%                         \\
            \multicolumn{1}{l|}{ResLAN (Ours)} &   969                           &  283                            &   2.83                             &   1.4                &  99.56 \%                          \\ \bottomrule
            \end{tabular}
            \vspace{0.2cm}
            \label{tab:sota-results}
        \end{table}

        \begin{table}[]
    	    \centering
    	    \small
    	    \caption{\textbf{Depth estimation results of the two baseline setups and the explicit and implicit ResSAN} when evaluated on a combination of 16, 32 and 64 channel depth inputs. Please note that Specialist Methods need to train three specialized networks, one for each of the three types of inputs while our method only uses one network.}
            \footnotesize
            \setlength{\tabcolsep}{4.8pt}
            \begin{tabular}{@{}lrrrrl@{}}
                \toprule
                \multicolumn{6}{c}{\textbf{Sparse Channel Filter}}                                                                                                                                  \\ \midrule
                \multicolumn{1}{l|}{Method}        & RMSE $\downarrow$            & MAE  $\downarrow$            & iRMSE $\downarrow$             & iMAE $\downarrow$ & $\delta_{1.25}$ $\uparrow$  \\
                \multicolumn{1}{l|}{}              & \multicolumn{1}{l}{{[}mm{]}} & \multicolumn{1}{l}{{[}mm{]}} & \multicolumn{1}{l}{{[}1/km{]}} & {[}1/km{]}        &                             \\ \midrule
                \multicolumn{1}{l|}{NLSPN}         &  1396                            &  437                            & 5.54                               &  2.2                 &  98.82 \%                           \\
                \multicolumn{1}{l|}{Baseline}      & \textbf{1207}                             &  381                            & 4.41                               &  1.8                 &  \textbf{99.37} \%                           \\
                \multicolumn{1}{l|}{ResLAN (Ours)} &  1215                            &  \textbf{378}                            &  \textbf{4.27}                              &  \textbf{1.7}                 &  99.31 \%                           \\ \toprule
                \multicolumn{6}{c}{\textbf{Field-of-View Filter}}                                                                                                                                   \\ \midrule
                \multicolumn{1}{l|}{Method}        & RMSE $\downarrow$            & MAE  $\downarrow$            & iRMSE $\downarrow$             & iMAE $\downarrow$ & $\delta_{1.25}$ $\uparrow$ \\
                \multicolumn{1}{l|}{}              & \multicolumn{1}{l}{{[}mm{]}} & \multicolumn{1}{l}{{[}mm{]}} & \multicolumn{1}{l}{{[}1/km{]}} & {[}1/km{]}        &                             \\ \midrule
                \multicolumn{1}{l|}{NLSPN}         &  2738                            &  1702                            & 12.3                              &  4.3                 &  74.69 \%                           \\
                \multicolumn{1}{l|}{Baseline}      &  1556                            &  525                            &  6.8                              &  3.0                 & 98.14 \%                            \\
                \multicolumn{1}{l|}{ResLAN (Ours)} &  \textbf{1548}                            &  \textbf{519}                            &  \textbf{6.44}                              &  \textbf{2.8}                 & \textbf{98.52 \%}                            \\ \bottomrule
            \end{tabular}
            \label{tab:overall-results}
        \end{table}

        
        
        % Figure environment removed
        
        % Figure environment removed

    \subsection{Filter Effects}
        Comparing the effect of the two different types of depth input filters on the model performance, it becomes apparent that FOV filtering is the more challenging task. In that setting, reducing LiDAR channels is more detrimental to the performance than sparse filtering as it creates regions where no depth information is available. Effectively, the model is forced to perform depth prediction in these regions. These effects are highlighted in the depth images in Fig.\,\ref{fig:dense-maps} where the effect of a 16-channel sparse depth filter and a 16-channel FOV can be compared.

    \subsection{Generalization Capabilities}
        We trained three models for both filter types eaach, so the combinations and number of filtered depth inputs they receive are different. This serves the purpose of testing the generalization capabilities of the ResLAN architecture as well as the robustness to different filter settings. After training, the models were evaluated for the depth input settings they were trained for, as well as for ones they weren't exposed to. Overall, ResLAN shows good generalization capabilities. As one can gather from Fig.\,\ref{fig:explicit-comp} and Fig.\,\ref{fig:implicit-comp}, the consequences of slightly varying sets of input depth settings are limited. The most considerable deviations can be seen when the model is tasked to extrapolate. For instance, the model $\{64, 32, 16\}$ shows a noticeably higher MAE for eight-channel depth inputs than the model that was trained for it. Similar behaviour can be seen for the FOV filtering case as well for the model $\{64, 48, 32\}$ when tasked to generalize for a 16-channel input. There is no such pronounced effect for generalization tasks that lie between two filter settings the model was trained for. At most, it can be observed that models that were trained for a smaller range of filter values perform slightly better than ones that have to cover a wider range. The number of filter settings used in a fixed range does not relevantly influence the model performance, as can be seen, when comparing the two models in Fig.\,\ref{fig:implicit-comp}, which are both trained for a range of 64 to 32 channels but one with three filter settings and the other one with five.
    
    % Figure environment removed
    
    
    % Figure environment removed
\section{Discussion}
\label{sec: discussion}
\kmsdelete{In this work} We study \kmsreplace{Fairness-Aware PAC learning}{Fair-ERM} in the malicious noise model, and  in some cases allow 
the learner to maintain optimal overall accuracy despite the signal in Group $B$ being almost entirely washed out.
%when we allow learners to use the
%$\PQ$ randomized expansion of the hypothesis class $\mathcal{H}$
In particular we show that different fairness constraints have fundamentally different behavior in the presence of Malicious Noise, in terms of the amount of accuracy loss that a given level of Malicious Noise could cause a fairness-constrained learner to incur. 
The key to achieving our results, which are more optimistic than those in \cite{lampert}, is allowing for improper learners using the (P,Q)-randomized expansions of the given class $\mathcal{H}$.
%We \kmsreplace{present a picture of the}{prove upper and lower bounds on}
%accuracy loss for a range of fairness notions, given \kmsreplace{this simple randomization step.}{learning over $\PQ$.
%In general our results indicate Fair-ERM (given learning over $\PQ$) is more robust than claimed in \cite{lampert}.
The type of smoothness we create by using $\PQ$ seems to be a natural property that is likely shared by many natural hypothesis classes.

Fairness notions are motivated as a response to learned disparities when there is \kmsdelete{data corruption or} systemic error affecting \kmsdelete{the data for}
one group. 
Fairness notions are supposed to mitigate this by ruling out classifiers that have worse performance on a sub-group. 
This can peg both classifiers at a lower level of performance \kmsdelete{(e.g that the lower subgroup)} in order to \emph{motivate} \cite{hardt16} improving the data collection or labelling process to obtain more reliable performance. 
%So in \kmsreplace{some}{a} sense, sensitivity of the fairness notion to poor sub-group performance caused by malicious noise is the \textit{point} of fairness constraints! 
However, it also desirable that fairness constraints perform gracefully when subject to Malicious Noise because fairness constraints will be used in contexts where the data is unreliable and noisy and this might not be known to the learner.
This tension, exposed by our work, motivates 
%a revisiting of fairness notions from first principles approach and trying to axiomatize the 
%desired properties of a fairness intervention a la cryptography and privacy. \footnote{Work in multi-calibration \cite{multicalib} is a viable direction for this problem but it is unclear how 
%that and related notions behave with unreliable data. }
on going work studying the sensitivity level of fairness constraints. 
%If we we are to take a view, if a classifier is deployed 

{\small
\bibliographystyle{ACM-Reference-Format}
%\bibliographystyle{ieee_fullname}
\bibliography{refs}
}

%%
%% The code below is generated by the tool at http://dl.acm.org/ccs.cfm.
%% Please copy and paste the code instead of the example below.
%%
%%
%% Keywords. The author(s) should pick words that accurately describe
%% the work being presented. Separate the keywords with commas.


%% A "teaser" image appears between the author and affiliation
%% information and the body of the document, and typically spans the
%% page.

%%
%% This command processes the author and affiliation and title
%% information and builds the first part of the formatted document.
%\maketitle

\end{document}
\endinput
%%
%% End of file `sample-sigconf.tex'.
