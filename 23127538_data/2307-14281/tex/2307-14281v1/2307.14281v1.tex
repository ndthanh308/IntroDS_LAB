\documentclass[11pt]{amsart}
\usepackage{fullpage}
\usepackage{amsmath,amsfonts,amssymb,amsthm,amscd,mathscinet,enumitem,stix}
\usepackage{hyperref}
\usepackage[noabbrev,capitalize]{cleveref}

\theoremstyle{plain}
\newtheorem{theorem}{Theorem}[section]
\newtheorem{proposition}[theorem]{Proposition}
\newtheorem{lemma}[theorem]{Lemma}
\newtheorem{corollary}[theorem]{Corollary}
\theoremstyle{definition}
\newtheorem{definition}[theorem]{Definition}
\newtheorem{notation}[theorem]{Notation}
\theoremstyle{remark}
\newtheorem{remark}[theorem]{Remark}

\renewcommand{\epsilon}{\varepsilon}
\renewcommand{\phi}{\varphi}

\DeclareMathOperator{\ADF}{ADF}
\DeclareMathOperator{\cs}{cs}
\DeclareMathOperator{\id}{id}
\DeclareMathOperator{\Con}{Con}
\DeclareMathOperator{\Seq}{Seq}
\DeclareMathOperator{\gelo}{GELO}
\DeclareMathOperator{\Part}{Part}
\DeclareMathOperator{\ssac}{SSAC}
\DeclareMathOperator{\As}{As}
\DeclareMathOperator{\Sat}{Sat}
\DeclareMathOperator{\Wr}{Wr}
\DeclareMathOperator{\Isom}{Isom}
\DeclareMathOperator{\Sols}{Sols}
\DeclareMathOperator{\Res}{Res}
\DeclareMathOperator{\Stab}{Stab}

\newcommand{\mso}{[\kern-1.75pt [}
\newcommand{\msc}{]\kern-1.75pt ]}
\newcommand{\ms}[1]{\mso #1 \msc}
\newcommand{\bmso}{\big[\kern-3pt \big[}
\newcommand{\bmsc}{\big]\kern-3pt \big]}
\newcommand{\bms}[1]{\bmso #1 \bmsc}
\newcommand{\E}{{\mathbb E}}
\newcommand{\N}{{\mathbb N}}
\newcommand{\Q}{{\mathbb Q}}
\newcommand{\Z}{{\mathbb Z}}
\newcommand{\cC}{{\mathcal C}}
\newcommand{\cP}{{\mathcal P}}
\newcommand{\cQ}{{\mathcal Q}}
\newcommand{\cW}{{\mathcal W}}
\newcommand{\cWp}{\cW^{(p)}}
\newcommand{\cWtwo}{\cW^{(2)}}
\newcommand{\cWthree}{\cW^{(3)}}
\newcommand{\fC}{{\mathfrak C}}
\newcommand{\fP}{{\mathfrak P}}
\newcommand{\card}[1]{\left|{#1}\right|}
\newcommand{\conj}[1]{\overline{#1}}
\newcommand{\expv}[1]{{\mathbf E}_{#1}}
\newcommand{\ev}{\E_f^\ell}
\newcommand{\mom}[1]{\mu_{#1,f}^\ell}
\newcommand{\smom}[1]{\tilde{\mu}_{#1,f}^\ell}
\newcommand{\plusmodthree}{\concavediamond}
\newcommand{\lindexset}{{[2]\times[2]}}
\newcommand{\indexset}{{[p]\times\lindexset}}
\newcommand{\Eindexset}{{E\times\lindexset}}
\newcommand{\Findexset}{{F\times\lindexset}}
\newcommand{\eindexset}{{\{e\}\times\lindexset}}
\newcommand{\sums}[1]{\sum_{\substack{#1}}}
\newcommand{\floor}[1]{\left\lfloor{#1}\right\rfloor}
\newcommand{\ceil}[1]{\lceil{#1}\rceil}
\newcommand{\res}[1]{\vert_{#1}}

\title[Moments of Autocorrelation Demerit Factors]{Moments of Autocorrelation Demerit Factors of Binary Sequences}

\author{Daniel J.\ Katz}
\author{Miriam E.\ Ramirez}
\thanks{Daniel J.\ Katz is with the Department of Mathematics, California State University, Northridge.  Miriam E.\ Ramirez was with the Department of Mathematics, California State University, Northridge, USA.  This paper is based upon work of both authors supported in part by the National Science Foundation under Grants DMS-1500856 and CCF-1815487, and by work of Daniel J.\ Katz supported in part by the National Science Foundation under Grant CCF-2206454.}
\date{26 July 2023}

\begin{document}
\maketitle
\begin{abstract}
Sequences with low aperiodic autocorrelation are used in communications and remote sensing for synchronization and ranging.
The autocorrelation demerit factor of a sequence is the sum of the squared magnitudes of its autocorrelation values at every nonzero shift when we normalize the sequence to have unit Euclidean length.
The merit factor, introduced by Golay, is the reciprocal of the demerit factor.
We consider the uniform probability measure on the $2^\ell$ binary sequences of length $\ell$ and investigate the distribution of the demerit factors of these sequences.
Previous researchers have calculated the mean and variance of this distribution.
We develop new combinatorial techniques to calculate the $p$th central moment of the demerit factor for binary sequences of length $\ell$.
These techniques prove that for $p\geq 2$ and $\ell \geq 4$, all the central moments are strictly positive.
For any given $p$, one may use the technique to obtain an exact formula for the $p$th central moment of the demerit factor as a function of the length $\ell$.
The previously obtained formula for variance is confirmed by our technique with a short calculation, and we demonstrate that our techniques go beyond this by also deriving an exact formula for the skewness.
\end{abstract}

\section{Introduction}

A {\it sequence} is a doubly infinite list $f=(\ldots,f_{-1},f_0,f_1,f_2,\ldots)$ of complex numbers in which only finitely many of the terms are nonzero.
We adopt this definition because we are thinking of our sequences aperiodically.
If $\ell$ is a nonnegative integer, then a {\it binary sequence of length $\ell$} is an $f=(\ldots,f_{-1},f_0,f_1,f_2,\ldots)$ in which $f_j\in\{-1,1\}$ for $j \in \{0,1,\ldots,\ell-1\}$ and $f_j=0$ otherwise.
Binary sequences are used to modulate signals in telecommunications and remote sensing \cite{Golomb,Golomb-Gong,Schroeder}.
Some applications, such as ranging, require very accurate timing.
For these applications, it is important that the sequence not resemble any time-delayed version of itself.

Our measure of resemblance is aperiodic autocorrelation.
If $f$ is a sequence and $s \in \Z$, then the {\it aperiodic autocorrelation of $f$ at shift $s$} is
\[
C_{f,f}(s) = \sum_{j \in \Z} f_{j+s} \conj{f_j},
\]
and since $f_k=0$ for all but finitely many $k$, this sum is always defined and is nonzero for only finitely many $s$.
Note that $C_{f,f}(0)$ is the squared Euclidean norm of $f$.
For applications, we want $|C_{f,f}(s)|$ to be small compared to $C_{f,f}(0)$ for every nonzero $s\in\Z$; this distinction is what ensures proper timing.

There are two main measures for evaluating how low the autocorrelation of a sequence $f$ is at nonzero shifts.
One measure is the {\it peak sidelobe level}, which is the maximum of $|C_{f,f}(s)|$ over all nonzero $s \in \Z$; this can be regarded as an $l^\infty$ measure.
Another important measure is the {\it demerit factor}, which is an $l^2$ measure of smallness of autocorrelation.
The {\it (autocorrelation) demerit factor} of a nonzero sequence $f$ is
\[
\ADF(f)=\frac{\sums{s \in \Z \\ s\not=0} |C_{f,f}(s)|^2}{C_{f,f}(0)^2} = -1 +\frac{\sum_{s \in \Z} |C_{f,f}(s)|^2}{C_{f,f}(0)^2},
\]
which is the sum of the squared magnitudes of all autocorrelation values at nonzero shifts for the sequence that one obtains from $f$ by normalizing it to have unit Euclidean norm.
The {\it (autocorrelation) merit factor} is the reciprocal of the autocorrelation demerit factor; it was introduced by Golay in \cite[p.\ 450]{Golay-72} as the ``factor'' for a sequence and then as the ``merit factor'' in \cite[p.\ 460]{Golay-75}, while ``demerit factor'' appears later in the work of Sarwate \cite[p.\ 102]{Sarwate}.

Sequences with low demerit factor (equivalently, high merit factor) are highly desirable for communications and ranging applications.
For each given length $\ell$, we would like to understand the distribution of the demerit factors of binary sequences of length $\ell$.
For this entire paper, $\Seq(\ell)$ denotes the set of $2^\ell$ binary sequences of length $\ell$ with the uniform probability distribution, and the expected value of a random variable $v$ with respect to this distribution is denoted by $\ev v(f)=\expv{f \in \Seq(\ell)} (v(f))$.
The $p$th central moment of the random variable $v(f)$ as $f$ ranges over the binary sequences of length $\ell$ is denoted
\[
\mom{p} v(f)=\ev \left(v(f)-\ev v(f)\right)^p,
\]
and the $p$th standardized moment is denoted by
\[
\smom{p} v(f)=\frac{\mom{p} v(f)}{\left(\mom{2} v(f)\right)^{p/2}}.
\]
Sarwate \cite[eq.\ (13)]{Sarwate} found the mean of the demerit factor for binary sequences of a given length.
\begin{theorem}[Sarwate, 1984]
If $\ell$ is a positive integer, then $\ev \ADF(f)=1-1/\ell$.
\end{theorem}
Borwein and Lockhart \cite[pp.\ 1469--1470]{Borwein-Lockhart} showed that the variance of the demerit factor for binary sequences of length $\ell$ tends to $0$ as $\ell$ tends to infinity.
Jedwab \cite[Theorem 1]{Jedwab} gives an exact formula for the variance of the demerit factor for binary sequences of length $\ell$.
We present a formula equivalent to Jedwab's.  Ours is a quasi-polynomial function of length with rational coefficients divided by the fourth power of the length.
\begin{theorem}[Jedwab, 2019]\label{Jessica}
If $\ell$ is a positive integer, then 
\[
\mom{2} \ADF(f)=\begin{cases}
\frac{16\ell^3-60\ell^2+56\ell}{3\ell^4} & \text{if $\ell$ is even,} \\[4pt]
\frac{16\ell^3-60\ell^2+56\ell -12}{3\ell^4} & \text{if $\ell$ is odd.} 
\end{cases} 
\]
\end{theorem}
The calculation of the variance of the demerit factor is considerably more difficult than the calculation of the mean.
Jedwab follows the method of Aupetit et al.\ \cite{Aupetit}, which involves many multiple summations and is therefore somewhat difficult to execute precisely: Jedwab had to correct the calculation of Aupetit et al.\ to get the right formula.

In this paper, we devise a combinatorial method for calculating the moments of the distribution of the demerit factor of binary sequences of length $\ell$.
For any given $p$, one may use the technique to obtain an exact formula for the $p$th central moment of the demerit factor as a function of the length $\ell$.
For $p=2$, this entails a short calculation that yields Jedwab's formula for variance.
To demonstrate that one can go further, we also use our formula for $p=3$ to derive an exact formula for the third central moment as a quasi-polynomial function of the length with rational coefficients divided by the sixth power of the length, from which we determine (in Corollaries \ref{Sarah} and \ref{Shirley}) the third standardized moment (skewness).
\begin{theorem}\label{Theodore}
If $\ell$ is a positive integer, then
\[
\mom{3} \ADF(f)=
\begin{cases}
\frac{160\ell^4-1296\ell^3+3296\ell^2-2496\ell}{\ell^6} &  \text{if $\ell \equiv 0 \bmod 4$,} \\[4pt]
\frac{160\ell^4-1296\ell^3+3296\ell^2-2736\ell+576}{\ell^6} &  \text{if $\ell \equiv \pm 1 \bmod 4$,} \\[4pt] 
\frac{160\ell^4-1296\ell^3+3296\ell^2-2496\ell-384}{\ell^6} &  \text{if $\ell \equiv 2 \bmod 4$,}
\end{cases}
\]
and
\[
\smom{3} \ADF(f)=
\begin{cases}
\frac{6\sqrt{3}(10\ell^4-81 \ell^3+206 \ell^2-156 \ell)}{(4 \ell^3-15 \ell^2+14 \ell)^{3/2}} &  \text{if $\ell \equiv 0 \bmod 4$,} \\[4pt]
\frac{6\sqrt{3}(10\ell^4-81 \ell^3+206 \ell^2-171 \ell+36)}{(4 \ell^3-15 \ell^2+14 \ell -3)^{3/2}} &  \text{if $\ell \equiv \pm 1 \bmod 4$,} \\[4pt]
\frac{6\sqrt{3}(10\ell^4-81 \ell^3+206 \ell^2-156 \ell-24)}{(4 \ell^3-15 \ell^2+14 \ell)^{3/2}} &  \text{if $\ell \equiv 2 \bmod 4$.}
\end{cases}
\]
\end{theorem}
Determination of the third moment is considerably more difficult than the calculation of the variance.

Out methods also shed light on interesting aspects of the distribution of demerit factors.
For instance, in \cref{Raphael} we show that our general theory implies the odd central moments are always nonnegative, and we can also determine precisely when central moments are zero.
\begin{theorem}\label{Leonard}
Let $\ell$ and $p$ be positive integers.
Then $\mom{p} \ADF(f)$ is nonnegative.
Moreover, if (i) $p=1$, (ii) $p$ is odd with $p>1$ and $\ell\leq 3$, or (iii) $p$ is even and $\ell\leq 2$, then $\mom{p} \ADF(f)$ is zero; otherwise it is strictly positive.
\end{theorem}

Our method can be developed further to prove that the $p$th central moment of the autocorrelation demerit factor of binary sequences of length $\ell$ is always a quasi-polynomial function of $\ell$ with rational coefficients divided by $\ell^{2 p}$.
It can also be used to show that in the limit as $\ell\to\infty$, all the standardized moments tend to those of the standard normal distribution.
The additional theoretical tools used to obtain these results shall be introduced and explored in a future paper.

The rest of this paper is organized as follows.
\cref{Jake} sets out the general theory that enables us to calculate central moments of the distribution of demerit factors of binary sequences.
This section employs the theory of partitions and group actions to give an exact formula for the each central moment.
Then we prove \cref{Leonard} in \cref{Prunella}.
\cref{Veronica} is a short section that applies this theory to compute the variance, thus confirming Jedwab's result in \cref{Jessica}.
\cref{Simon} follows with an exact calculation of third central moment, thus proving \cref{Theodore}.
We close with \cref{Arthur}, which contains some technical combinatorial results used in the main text.

\section{Exact combinatorial determination of moments}\label{Jake}

In this section, we prove the general results that we need to calculate central moments of the distribution of demerit factors for binary sequences.
We begin in \cref{Nellie} with basic definitions, mostly concerning partitions and functions known as {\it assignments}, which we then use in \cref{Monte} to obtain an exact formula for central moments (cf.\ \cref{Sanri}).
Typically, this formula involves many similar partitions, so in \cref{Idelphonse} we devise an equivalence relation (via a group action) to organize these partitions into classes.
This makes the moment calculations much easier, and produces a formula in \cref{Sanria}, which is used for the exact determination of variance and skewness in Sections \ref{Veronica} and \ref{Simon}.
In \cref{Scott}, we introduce some further concepts and results that elucidate the structure of the partitions relevant to our calculations; these are general ideas used in Sections \ref{Veronica} and \ref{Simon}.

\subsection{Notation and definitions}\label{Nellie}
In this paper, we use the convention that $\N=\{0,1,2,\ldots\}$.
If $\ell \in \N$, we write $[\ell]$ to mean $\{0, 1, \dots, \ell-1\}$.

We write multisets using blackboard bold square brackets: if $I$ is a set and $\{a_i\}_{i \in I}$ is a family of (not necessarily distinct) elements indexed by $I$, then $\ms{a_i: i \in I}$ is the multiset in which the multiplicity of $a$ in our multiset equals the number of $i \in I$ such that $a_i=a$.
In particular, if $a_1,\ldots,a_k$ is list, then the multiplicity of $a$ in the multiset $\ms{a_1,\ldots,a_k}$ equals the number of times $a$ appears on our list.

We use the symbol $\sqcup$ to indicate a union of disjoint sets.
If $S$ and $T$ are sets, then $T^S$ denotes the set of all functions from $S$ into $T$.
When we have a $k$-tuple of natural numbers, we sometimes abbreviate it by omitting commas and enclosing parentheses (e.g., the triple $(e,s,v)$ could be abbreviated as $(e s v)$ or even $e s v$), provided that the abbreviation does not introduce any ambiguity, for example, when the context demands a tuple with precisely $k$ terms and each of the $k$ terms is written with a single letter or digit.

A partition of a set $A$ is a collection of nonempty, disjoint subsets of $A$ whose union is $A$.
If $\cP$ is a partition of $A$, then $\cP$ induces an equivalence relation on $A$ that is written $a_1 \equiv a_2 \pmod{\cP}$, which means that there is some class $P \in \cP$ such that $a_1,a_2 \in P$.

\begin{definition}[Partition induced by a function]
Let $f\colon A \to B$ be a function.
The {\it partition induced by $f$} is the partition of $A$ that equals $\{f^{-1}(b): b \in B\}\smallsetminus\{\emptyset\}$, i.e., the classes of the partitions are the nonempty fibers of $f$.
Equivalently, the partition induced by $f$ is the partition of $A$ where $a_1, a_2 \in A$ lie in the same class if and only if $f(a_1)=f(a_2)$.
In other words, the partition induced by $f$ is the partition that induces the equivalence relation on $A$ where $a_1$ is equivalent to $a_2$ if and only if $f(a_1)=f(a_2)$.
\end{definition}
\begin{definition}[Partition type]
Let $\cP$ be a partition of a set $S$. Then the {\it type of $\cP$} is the multiset $\ms{|P|: P \in \cP}$.
\end{definition}
 \begin{definition}[Even partition] 
A partition is {\it even} if and only if the cardinality of each equivalence class is finite and even, i.e., if and only if every element of the partition's type is an even number.
\end{definition}
In order to continue with our definitions, we introduce some notation. If $E \subseteq \N$, then we define:
\begin{itemize}
\item $\Part(E)=\{\cP:\cP \text{ is a partition of } \Eindexset \}$,
\item $\Part(p)=\Part([p])$ for $p\in\N$.
\end{itemize}
Now we introduce the concept of restriction of partitions, which is used throughout the rest of this paper.
\begin{definition}[Restriction of sets and partitions]\label{Judy}
Let $F \subseteq E \subseteq \N$.
For $P\subseteq \Eindexset$, the {\it restriction of $P$ to $F$}, written $P_F$, is $P\cap(\Findexset)$.
For $\cP\in \Part(E)$, then {\it restriction of $\cP$ to $F$}, written $\cP_F$, is the partition $\{P_F : P \in \cP\} \smallsetminus \{\emptyset\}$ of $\Findexset$.
\end{definition}
In our combinatorial methods, the parity of sizes of classes in partitions is very important.
The following definition characterizes partitions of great importance in our calculations.
\begin{definition}[Globally even, locally odd (GELO) partition]\label{Hammurabi}
Let $E \subseteq \N$, then $\cP \in \Part(E)$ is said to be {\it globally even, locally odd} (abbreviated {\it GELO}) if $\cP$ is even but $\cP_{\{e\}}$ is not even for each $e \in E$. We denote the set of globally even, locally odd partitions of $\Eindexset $ as $\gelo(E)$.  For $p \in \N$, we write $\gelo(p)$ to mean $\gelo([p])$.
\end{definition}

A certain kind of structure, which we shall call an {\it assignment}, plays a critical role in our probability calculations.
\begin{definition}[Assignment]\label{Gennady}
Let $E\subseteq \N$.
An {\it assignment for $E$} is a function from $\Eindexset$ to $\N$, i.e., an element of $\N^\Eindexset$.
If $\tau$ is an assignment for $E$ and $(e,s,v) \in \Eindexset$, we typically use subscript notation $\tau_{e,s,v}$ (or just $\tau_{esv}$) rather than the parenthesized notation $\tau(e,s,v)$ to denote the evaluation of $\tau$ at $(e,s,v)$.
When we do not know or do not wish to disclose a particular coordinate or coordinates of an ordered triple in $\indexset$, we write $*$ to indicate the unstated value; for example, $(0,*,*)$ stands for an element of the form $(0,s,v)$ for some $s,v \in [2]$.
The following are notations for the set of all assignments for $E$ and some of its important subsets:
\begin{itemize}
\item $\As(E) = \N^\Eindexset$, the set of all assignments for $E$,
\item $\As(E,\ell) = [\ell]^{\Eindexset} $,
\item $\As(E,=) = \{\tau \in \As(E): \tau_{e 00} + \tau_{e 01} = \tau_{e 10} + \tau_{e 11} \text{ for every } e \in E\}$, and
\item $\As(E,=,\ell) = \As(E,=) \cap \As(E,\ell)$.
\end{itemize}
Furthermore, if $\cP \in \Part(E)$, then
\begin{itemize}
\item $\As(\cP)=\{\tau \in \As(E): \tau_\beta = \tau_\gamma \text{ iff }  \beta \equiv \gamma \bmod{\cP} \}$, that is, the set of assignments that induce the partition $\cP$,
\item $\As(\cP,\ell)=\As(\cP) \cap \As(E,\ell)$,
\item $\As(\cP,=) = \As(\cP) \cap \As(E,=)$, and
\item $\As(\cP,=,\ell) = \As(\cP) \cap \As(E,=) \cap \As(E,\ell)$.
\end{itemize}
\end{definition}
It is often necessary to restrict the domain of assignments in a way that is compatible with the restriction of sets and partitions in \cref{Judy}.
\begin{definition}[Restriction of assignments]
If $F \subseteq E\subseteq \N$, and $\tau \in \As(E)$, then we use $\tau\res{F}$ as a shorthand for $\tau\vert_{F\times\lindexset}$.
We call $\tau\res{F}$ the {\it restriction of $\tau$ to $F$}.
\end{definition}
This restriction process respects the various attributes for assignments that we specified in \cref{Gennady}.
\begin{lemma}\label{flower}
Let $p,\ell \in \N$, $F\subseteq E \subseteq [p]$, and $\cP \in \Part(E)$.
\begin{enumerate}[label=(\roman*)]
\item\label{flower-a} If $\tau\in\As(E)$, then $\tau\res{F} \in \As(F)$.
\item\label{flower-b} If $\tau\in\As(E,\ell)$, then $\tau\res{F} \in \As(F,\ell)$.
\item\label{flower-c} If $\tau\in\As(E,=)$, then $\tau\res{F} \in \As(F,=)$.
\item\label{flower-d} If $\tau\in\As(E,=,\ell)$, then $\tau\res{F} \in \As(F,=,\ell)$.
\item\label{flower-e} If $\tau\in\As(\cP)$, then $\tau\res{F} \in \As(\cP_F)$.
\item\label{flower-f} If $\tau\in\As(\cP,\ell)$, then $\tau\res{F} \in \As(\cP_F,\ell)$.
\item\label{flower-g} If $\tau\in\As(\cP,=)$, then $\tau\res{F} \in \As(\cP_F,=)$.
\item\label{flower-h} If $\tau\in\As(\cP,=,\ell)$, then $\tau\res{F} \in \As(\cP_F,=,\ell)$.
\end{enumerate}
\end{lemma}
\begin{proof}
First of all, \ref{flower-a} (resp., \ref{flower-b}) is immediate since $\tau$ is a function from $\Eindexset$ into $\N$ (resp., into $[\ell]$), so $\tau\res{F}=\tau\vert_{F\times\lindexset}$ is a function from $F\times\lindexset$ into $\N$ (resp., into $[\ell]$).
If $\tau \in \As(E,=)$, then for each $f \in F$, we have $\tau_{f00}+\tau_{f01}=\tau_{f10}+\tau_{f11}$, so $(\tau\res{F})_{f00}+(\tau\res{F})_{f01}=(\tau\res{F})_{f10}+(\tau\res{F})_{f11}$, and so $\tau\res{F} \in \As(F,=)$, thus proving \ref{flower-c}.
Then \ref{flower-d} follows from \ref{flower-b} and \ref{flower-c} because $\As(A,=,\ell)=\As(A,\ell)\cap \As(A,=)$ for $A \in \{E,F\}$.

Let $\tau\in\As(\cP)$, so that for every $\beta,\gamma \in F\times\lindexset$, we have $\tau_\beta=\tau_\gamma$ if and only if $\beta \equiv \gamma \pmod{\cP}$, and so
$(\tau\res{F})_\beta=(\tau\res{F})_\gamma$ if and only if $\beta \equiv \gamma \pmod{\cP_F}$, so then $\tau\res{F} \in \As(\cP_F)$, thus proving \ref{flower-e}.
Then \ref{flower-f}, \ref{flower-g}, and \ref{flower-h} respectively follow from \ref{flower-e} in conjunction with \ref{flower-b}, \ref{flower-c}, \ref{flower-d}, respectively.
\end{proof}
Partitions that support assignments are significant in our calculations of moments.
\begin{definition}[Satisfiable partition]\label{Sargon}
Let $E$ a finite subset of $\N$.
A partition $\cP \in \Part(E)$ is said to be {\it satisfiable} if $\As(\cP,=)$ is nonempty.
(Equivalently, there is some $\ell \in \N$ such that $\As(\cP,=,\ell)$ is nonempty.)
We denote the set of satisfiable partitions of $\Eindexset$ as $\Sat(E)$, and for $p\in\N$ we use $\Sat(p)$ as a shorthand for $\Sat([p])$.
\end{definition}

\begin{lemma}
Let $E$ be a finite subset of $\N$.
If $\cP \in \Sat(E)$ then for any $F\subseteq E$ we have $\cP_F \in \Sat(F)$.
\end{lemma}
\begin{proof}
Suppose that $\cP\in\Sat(E)$, so that there is some $\tau \in \As(\cP,=)$.
Then \cref{flower}\ref{flower-g} tells us that $\tau\res{F}\in \As(\cP_F,=)$, so $\As(\cP_F,=)$ is nonempty.
\end{proof}
When we calculate the moments of the distribution of demerit factors, it turns out that every nonzero term in our calculation corresponds to some partition combining the attributes of both \cref{Hammurabi} and \cref{Sargon}, so we name such partitions accordingly.
\begin{definition}[Contributory partition]
Let $p \in \N$.  Then $\cP \in \Part(p)$ is said to be {\it contributory} if it is globally even, locally odd and satisfiable.
We denote the set of contributory partitions of $\indexset$ as $\Con(p)$.
That is, $\Con(p)=\gelo(p)\cap\Sat(p)$.
\end{definition}

\subsection{Moments from partitions}\label{Monte}
The set of binary sequences of length $\ell$ is $\Seq(\ell)$.
Recall that we equip this set with uniform probability measure, and use $\ev v(f)=\expv{f \in \Seq(\ell)} (v(f))$ to denote the expected value of a random variable $v$ depending on $f$ with this distribution.
Also recall that the $p$th central moment of the random variable $v(f)$ as $f$ ranges over the binary sequences of length $\ell$ is denoted
\[
\mom{p} v(f)=\ev \left(v(f)-\ev v(f)\right)^p,
\]
and the $p$th standardized moment is denoted by
\[
\smom{p} v(f)=\frac{\mom{p} v(f)}{\left(\mom{2} v(f)\right)^{p/2}}.
\]

We use $\ssac(f)$ to denote the sum of the squares of the autocorrelation values for a given sequence $f$.
In this section, we will show the theory behind how we calculate the $p$th central moment of $\ssac$.
Since the demerit factor of a binary sequence $f$ of length $\ell$ is $\ADF(f)=-1+\ssac(f)/\ell^2$, it is easy to determine the $p$th central moment of the demerit factor from that of $\ssac$.
\cref{Sanri}, the main result of this section, provides a way of calculating central moments of the sum of squares of the autocorrelation in terms of contributory partitions and assignments.
The proof of this result requires some technical lemmata and their corollaries, which we defer to the end of the section.
In this section we use the notation that if $p,\ell \in \N$, $E\subseteq [p]$, $\tau \in \As(E,\ell)$ and $f=(\ldots,f_0,f_1,f_2,\ldots) \in \Seq(\ell)$, then $f^\tau= \prod_{\gamma \in \Eindexset } f_{\tau_\gamma}$.
\begin{proposition}\label{Sanri}
For $p,\ell \in \N$, we have
\[ \mom{p} \ssac(f) = \sum_{\cP \in \Con(p)} |\As(\cP,=,\ell)|. \]
\end{proposition}
\begin{proof}
We use the convention that $f_j=0$ if $j\not\in \{0,1,\ldots,\ell-1\}$, in which case \cite[eq.\ (14)]{Katz16} shows that
\[
\ssac(f)=\sums{s,t,u,v \in \Z\\ s+t=u+v} f_s f_t f_u f_v = \sums{0 \leq s,t,u,v < \ell \\ s+t=u+v} f_s f_t f_u f_v.
\]
So in fact, for any $e \in [p]$, we have $\ssac(f)=\sum_{\tau \in \As(\{e\},=,\ell)} f^\tau$.
For the rest of this proof, if $E$ is any subset of $[p]$, we write $S(E)$ for $\As(E,=,\ell)$, and if $e \in [p]$, we write $S_e$ to mean $S(\{e\})=\As(\{e\},=,\ell)$.
\begin{align*}
\mom{p} \ssac(f) & = \ev\left(\prod_{e \in [p]} \left[\sum_{\tau^{(e)} \in S_e} f^{\tau^{(e)}} - \ev\left(\sum_{\tau^{(e)} \in S_e} f^{\tau^{(e)}} \right) \right] \right)  \\
&= \ev\left(\sum_{E \subseteq [p]} \prod_{d \in [p]\smallsetminus E} \, \sum_{\tau^{(d)} \in S_d} f^{\tau^{(d)}} (-1)^{\card{E}}  \prod_{e \in E}  \ev\left(\sum_{\tau^{(e)} \in S_e} f^{\tau^{(e)}} \!\! \right)\!\! \right) \\
&= \ev\left(\sum_{E \subseteq [p]} \sum_{\upsilon \in S([p]\smallsetminus E)} f^\upsilon (-1)^{\card{E}}  \prod_{e \in E}  \ev\left(\sum_{\tau^{(e)} \in S_e} f^{\tau^{(e)}} \right) \right) \\
&= \sum_{E \subseteq [p]} (-1)^{\card{E}} \left(\sum_{\upsilon \in S([p]\smallsetminus E)} \ev(f^\upsilon)\right) \left(\prod_{e \in E}  \sum_{\tau^{(e)} \in S_e} \ev\left(f^{\tau^{(e)}} \right)\right) \\
&= \sum_{E \subseteq [p]} (-1)^{\card{E}} \sum_{\tau \in S([p])} \ev(f^{\tau\vert_{[p]\smallsetminus E}})  \prod_{e \in E}  \ev\left(f^{\tau_{\{e\}}} \right) \\
&= \sum_{E \subseteq [p]} (-1)^{\card{E}} \sum_{\cP \in \Part(p)} \sum_{\tau \in \As(\cP,=,\ell)} \ev(f^{\tau_{[p]\smallsetminus E}}) \prod_{e \in E}  \ev\left(f^{\tau_{\{e\}}} \right) \\
&= \sum_{\cP \in \Part(p)} \sum_{E \subseteq [p]} (-1)^{\card{E}}  \sum_{\tau \in \As(\cP,=,\ell)} \ev(f^{\tau_{[p]\smallsetminus E}})  \prod_{e \in E}  \ev\left(f^{\tau_{\{e\}}} \right),
\end{align*}
where the second equality is the binomial expansion, the third equality uses \cref{Valerian} below, the fifth uses \cref{Vladimir} below, and the sixth sorts the tuples in $S([p])=\As([p],=,\ell)$ according to which partitions they induce.


Let $\cP$ be a fixed element of $\Part(p)$ and examine the value of the innermost summation for this fixed $\cP$.
If $\tau \in \As(\cP,=,\ell)$, then Lemma \ref{Kirkland} below tells us that
\[
\ev(f^{\tau_{[p]\smallsetminus E}})  \prod_{e \in E}  \ev(f^{\tau_{\{e\}}})
\]
equals $1$ if and only if both $\cP_{[p]\smallsetminus E}$ is even and $\cP_{\{e\}}$ is even for every $e \in E$; otherwise the product of expectations is $0$.
This means that we get a nonzero contribution if and only if both $\cP$ is even and $\cP_{\{e\}}$ is even for every $e \in E$.

So we fix an even $\cP \in \Part(p)$, and we let $C$ be the subset of $ [p] $ such that $ \cP_{ \{c\} } $ is even for all $c \in C$ and $ \cP_{ \{d\} } $ is non-even for all $ d \in [p] \smallsetminus C$.
By the previous paragraph,
\[
\sum_{E \subseteq [p]} (-1) ^{\card{E}} \sum_{\tau \in \As(\cP,=,\ell)} \left[\ev(f^{\tau_{[p]\smallsetminus E}}) \prod_{e \in E} \ev(f^{\tau_{\{e\}}}) \right] 
= \sum_{E \subseteq C} (-1) ^{\card{E}} \card{\As(\cP,=,\ell)}.
\]
If $C$ is nonempty then the sum on the right hand side is always zero; otherwise we get $|\As(\cP,=,\ell)|$.
Thus, we only get nonzero contributions from partitions that are globally even, locally odd, so our expectation calculation becomes
\begin{align*}
\mom{p} \ssac(f) & = \sum_{\cP \in \gelo(p)} |\As(\cP,=,\ell)| \\
& = \sum_{\cP \in \Con(p)} |\As(\cP,=\ell)|,
\end{align*}
where the final step simply eliminates those $\cP$ such that $|\As(\cP,=,\ell)|=0$.
\end{proof}

\begin{remark}
\cref{Sanri} shows that each central moment is a sum of cardinalities, so it must be nonnegative.  Eventually in \cref{Lynn} we shall give the necessary and sufficient conditions for a central moment to be strictly positive.
\end{remark}

\begin{remark}
Since there is no GELO partition of $[1]\times\lindexset$, we know that $\Con(1)=\emptyset$, so then \cref{Sanri} correctly indicates that the first central moment is $0$.
\end{remark}


We conclude with the technical results that we used in the proof of \cref{Sanri}.

\begin{lemma}
Let $R$ be a commutative ring, let $p,k \in \N$, let $E_1,\ldots,E_k$ be pairwise disjoint subsets of $[p]$, let $E=\bigsqcup_{j=1}^k E_k$, and let $g_j\colon \As(E_j,=,\ell) \to R$ for each $j \in \{1,\ldots,k\}$.
Then
\[
\sum_{\tau \in \As(E,=,\ell)} \,\,\, \prod_{j=1}^k g_j(\tau\vert_{E_j}) = \prod_{j=1}^k \,\,\, \sum_{\tau^{(j)} \in \As(E_j,=,\ell)}  g_j(\tau^{(j)}).
\]
\end{lemma}
\begin{proof}
First, we let $\phi \colon \As(E,=,\ell) \to \prod_{j=1}^k \As(E_j,=,\ell)$ be the map with $\phi(\tau)=(\tau\vert_{E_1},\ldots,t\vert_{E_k})$ and let $\psi\colon \prod_{j=1}^k \As(E_j,=,\ell) \to \As(E,=,\ell)$ with $\psi((\tau^{(1)},\ldots,\tau^{(k)}))$ the element $\tau \in \As(E,=,\ell)$ such that for any $(e,s,v) \in E\times\lindexset$, we let $\tau_{e,s,v}=\tau^{(j)}_{e,s,v}$ where $j$ is the unique element of $\{1,\ldots,k\}$ with $e \in E_j$.
It is easy to see that $\phi$ and $\psi$ are inverse maps, and so they are bijective.
Now let $g\colon \prod_{j=1}^k \As(E_j,=,\ell) \to R$ with $g((\tau^{(1)},\ldots,\tau^{(k)})) = \prod_{j=1}^k g_j(\tau^{(j)})$.
Then we have
\begin{align*}
\prod_{j=1}^k \,\,\, \sum_{\tau^{(j)} \in \As(E_j,=,\ell)}  g_j(\tau^{(j)})
& = \sum_{(\tau^{(1)},\ldots,\tau^{(k)}) \in \prod_{j=1}^k \As(E_j,=,\ell)} \,\,\, \prod_{j=1}^k g_j(\tau^{(j)}) \\
& = \sum_{(\tau^{(1)},\ldots,\tau^{(k)}) \in \prod_{j=1}^k \As(E_j,=,\ell)} g((\tau^{(1)}, \cdots, \tau^{(k)})) \\
& = \sum_{\tau \in \As(E,=,\ell)} g(\phi(\tau)) \\
& = \sum_{\tau \in \As(E,=,\ell)} g((\tau\vert_{E_1},\ldots,\tau\vert_{E_k})) \\
& = \sum_{\tau \in \As(E,=,\ell)} \,\,\, \prod_{j=1}^k g_j(\tau\vert_{E_j}),
\end{align*}
where the first equality is due to the distributive law, the second and fifth are by the definition of $g$, the third is by bijectivity of $\phi$, and the fourth by the definition of $\phi$.
\end{proof}

\begin{corollary}\label{Valerian}
Let $R$ be a commutative ring, let $p,k \in \N$, let $E_1,\ldots,E_k$ be pairwise disjoint subsets of $[p]$, let $E=\bigsqcup_{j=1}^k E_k$, and let $f=(\ldots,f_0,f_1,f_2,\ldots)$ be a sequence of elements of $R$.
Then
\[
\sum_{\tau \in \As(E,=,\ell)} f^\tau = \prod_{j=1}^k \sum_{\tau^{(j)} \in \As(E_j,=,\ell)}  f^{\tau^{(j)}}.
\]
\end{corollary}
\begin{proof}
For each $j \in \{1,\ldots,k\}$, let $g_j \colon \As(E_j,=,\ell) \to R$ with $g_j(\upsilon)=f^\upsilon$, so by the lemma we have
\[
\prod_{j=1}^k \sum_{\tau^{(j)} \in \As(E_j,=,\ell)}  g_j(\tau^{(j)})= \sum_{\tau \in \As(E,=,\ell)} \prod_{j=1}^k g_j(\tau\vert_{E_j}),
\]
that is,
\begin{align*}
\prod_{j=1}^k \sum_{\tau^{(j)} \in \As(E_j,=,\ell)}  f^{\tau^{(j)}}
& = \sum_{\tau \in \As(E,=,\ell)} \prod_{j=1}^k f^{\tau\vert_{E_j}} \\
& = \sum_{\tau \in \As(E,=,\ell)} \prod_{j=1}^k \prod_{(e,s,v) \in E_j\times\lindexset} f_{\tau_{e,s,v}} \\
& = \sum_{\tau \in \As(E,=,\ell)} \prod_{(e,s,v) \in E\times\lindexset} f_{\tau_{e,s,v}} \\
& = \sum_{\tau \in \As(E,=,\ell)} f^{\tau}.\qedhere
\end{align*}
\end{proof}


\begin{corollary}\label{Vladimir}
Let $p,k,\ell \in \N$, let $E_1,\ldots,E_k$ be pairwise disjoint subsets of $[p]$, and let $E=\bigsqcup_{j=1}^k E_k$.
Then
\[
\sum_{\tau \in \As(E,=,\ell)} \prod_{j=1}^k \ev(f^{\tau\vert_{E_j}}) = \prod_{j=1}^k \sum_{\tau^{(j)} \in \As(E_j,=,\ell)}  \ev(f^{\tau^{(j)}}).
\]
\end{corollary}
\begin{proof}
For each $j \in \{1,\ldots,k\}$, let $g_j \colon \As(E_j,=,\ell) \to \Q$ with $g_j(\upsilon)=\ev(f^\upsilon)$, so by the lemma we have
\[
\prod_{j=1}^k \sum_{\tau^{(j)} \in \As(E_j,=,\ell)}  g_j(\tau^{(j)})= \sum_{\tau \in \As(E,=,\ell)} \prod_{j=1}^k g_j(\tau\vert_{E_j}),
\]
that is,
\[
\prod_{j=1}^k \sum_{\tau^{(j)} \in \As(E_j,=,\ell)}  \ev(f^{\tau^{(j)}})= \sum_{\tau \in \As(E,=,\ell)} \prod_{j=1}^k \ev(f^{\tau\vert_{E_j}}). \qedhere
\]
\end{proof}

\begin{lemma}\label{Kirkland} 
Let $p, \ell \in \N$, $\cP\in \Part(p)$, $\tau \in \As(\cP,=,\ell)$,and $E \subseteq [p]$, then
\[
\ev(f^{\tau\res{E}})=\begin{cases}
1 & \text{if $\cP_E$ is even,} \\
0 & \text{otherwise.}
\end{cases} 
\]
\end{lemma}
\begin{proof}
Note that 
$f^{\tau\res{E}}= \prod_{\gamma \in \Eindexset} f_{(\tau\res{E})_\gamma} = \prod_{j \in [\ell]} f_j^{|(\tau\res{E})^{-1}(\{j\})|}$.
Since $f_0,\ldots,f_{\ell-1}$ are independent and uniformly distributed on $\{\pm 1\}$, we see that $\ev(f^{\tau\res{E}})$ is $1$ if $|(\tau\res{E})^{-1}(\{j\})|$ is even for all $j \in [\ell]$, and $\ev(f^{\tau\res{E}})=0$ otherwise.
\cref{flower}\ref{flower-h} shows that $\tau\res{E} \in \As(\cP_E,=,\ell)$, and so $\cP_E$ is the partition of $\Eindexset$ induced by $\tau\res{E}$, so the classes of $\cP_E$ are the nonempty fibers of $\tau\res{E}$.
Thus the cardinalities of the nonempty fibers of $\tau\res{E}$ are the cardinalities of the classes of $\cP_E$, and so $|(\tau\res{E})^{-1}(\{j\})|$ is even for every $j \in [\ell]$ if and only if $\cP_E$ is even.
\end{proof}


\subsection{Moments from isomorphism classes of partitions}\label{Idelphonse}
Let $p, \ell \in \N $ be fixed and take $ \cP , \cQ \in \Part(p) $ where $ \arrowvert \cP \arrowvert = \arrowvert \cQ \arrowvert $ with equivalence classes of matching size.
We wish to find what guarantees $|\As(\cP,=,\ell)|=|\As(\cQ,=,\ell)|$.  

For example, if $\cP, \cQ \in \Part(2)$ are given by
\begin{align*}
\cP & = \{ \{(000),(001), (100), (101)  \}, \{ (010), (110) \}, \{ (011), (111)  \} \}  \\
\cQ & = \{ \{(010),(011), (100), (101)  \}, \{ (000), (111) \}, \{ (001), (110)  \} \}
\end{align*}
Let $\pi$ be the permutation of $[2]\times\lindexset$ with
\begin{equation}\label{Petra}
\pi(e,s,v)=\begin{cases}
(1-e,1-s,v) & \text{if $(e,s)=(0,0)$,} \\
(1-e,1-s,1-v) & \text{if $(e,s)=(0,1)$,} \\
(1-e,s,v) & \text{if $(e,s)=(1,0)$,} \\
(1-e,s,1-v) & \text{if $(e,s)=(1,1)$.}
\end{cases}
\end{equation}
We shall now show that the map $\tau \mapsto \tau\circ\pi$ is a bijection from $\As(\cP,=,\ell)$ to $\As(\cQ,=,\ell)$, which shows that $\card{\As(\cP,=,\ell)}=\card{\As(\cQ,=,\ell)}$.
If we start with $\tau \in \As(\cP,=,\ell)$, then there are distinct $A,B,C \in [\ell]$ such that
\begin{align*}
A & = \tau_{000} = \tau_{001} = \tau_{100} = \tau_{101}\\
B & = \tau_{010} = \tau_{110} \\
C & = \tau_{011} = \tau_{111},
\end{align*}
and 
\begin{align*}
A + A = \tau_{000} + \tau_{001} & = \tau_{010} + \tau_{011} = B+C\\
A + A = \tau_{100} + \tau_{101} & = \tau_{110} + \tau_{111} = B+C.
\end{align*}
Now set $\upsilon=\tau\circ\pi$, then 
\begin{align*}
(\upsilon_{000}, \upsilon_{001},\upsilon_{010},\upsilon_{011},\upsilon_{100},\upsilon_{101},\upsilon_{110},\upsilon_{111}) & =(\tau_{110},\tau_{111},\tau_{101},\tau_{100},\tau_{000},\tau_{001},\tau_{011},\tau_{010}) \\
& = (B,C,A,A,A,A,C,B),
\end{align*}
which has $\upsilon_{esv}=\upsilon_{e's'v'}$ if and only if $(e,s,v) \equiv (e',s',v') \pmod{\cQ}$ and also satisfies
\begin{align*}
B + C = \upsilon_{000} + \upsilon_{001} & = \upsilon_{010} + \upsilon_{011} = A+A\\
A + A = \upsilon_{100} + \upsilon_{101} & = \upsilon_{110} + \upsilon_{111} = C+B,
\end{align*}
so that $\tau\circ\pi \in \As(\cQ,=,\ell)$.
The effect of composing on the right with $\pi$ is to permute the indices of the assignment in a way that the system of equations for $\tau$ is transformed into a system of equations for $\upsilon$ by (i) first transposing the two equations, (ii) then transposing the sides of one of the equations, and (iii) then, for each of the two equations, transposing the two summands on the right-hand side.
Moving items in our system of equations in this manner will always yield a system that is still satisfied.
One can similarly show that if $\phi$ is an arbitrary element of $\As(\cQ,=,\ell)$, then $\phi \circ \pi^{-1} \in \As(\cP,=,\ell)$, and then one sees that the map $\tau \mapsto \tau \circ \pi$ from $\As(\cP,=,\ell)$ to $\As(\cQ,=,\ell)$ and the map $\phi \mapsto \phi \circ \pi^{-1}$ in the opposite direction are inverses of each other, and so $|\As(\cP,=,\ell)|=|\As(\cQ,=,\ell)|$.
Now notice that $\cP=\pi(\cQ)$.
This is no coincidence; we shall eventually show that $\{\tau\circ\pi : \tau \in \As(\pi(\cQ),=,\ell)\} = \As(\cQ,=,\ell)$ if $\cQ$ is any partition of $\indexset$ and $\pi$ is a permutation of $\indexset$ that respects the underlying structure of our system of equations.
This $\pi$ cannot be an arbitrary permutation of $\indexset$: for example, it is not hard to show that if $\pi$ is the transposition of $(0,0,0)$ and $(0,1,0)$, then $\As(\cQ,=,3)\not=\emptyset$ (one element is the $\upsilon$ displayed above with $A=1$, $B=0$, and $C=2$), but $\As(\pi(\cQ),=,\ell)=\emptyset$.

This leads us to the notion of isomorphic partitions.
To define this notion, we must first speak of groups of permutations of $\indexset$ and their action on $\Part(p)$ and $\As([p])$.
 
If $A$ is any set, we use $S_A$ to denote the group of all permutations of $A$, and $S_p$ is a shorthand for $S_{[p]}$.
If $\pi \in S_{\indexset}$ and $P \subseteq \indexset$ and $\cQ$ is a set of subsets of $\indexset$, then we let $\pi$ act on $P$ by setting $\pi(P)=\{\pi(e,s,v): (e,s,v) \in P\}$ and we let $\pi$ act on $\cQ$ by setting $\pi(\cQ) = \{\pi(Q): Q \in \cQ\}$.  This gives an action of $\pi$ on $\Part(p)$.

If $\pi \in S_\indexset$, then the map $\pi^*$ is the permutation of $\As([p])$ given by $\tau\mapsto\tau\circ\pi$; the inverse of $\pi^*$ is $(\pi^{-1})^*$.
Thus $(\pi^*(\tau))_{e,s,v}=(\tau\circ\pi)_{e,s,v}=\tau_{\pi(e,s,v)}$ for each $(e,s,v)\in\indexset$.
For example, if $\pi \in S_{[2]\times\lindexset}$ is the permutation defined in \eqref{Petra}, then we can restrict the domain and codomain of $\pi^*$ to $\As(\cP,=,\ell)$ and $\As(\cQ,=,\ell)$, respectively, thus demonstrating $\card{\As(\cP,=,\ell)}=\card{\As(\cQ,=,\ell)}$.
However, the full group $S_\indexset$ of permutations of our indices does not respect the equations $\tau_{e00}+\tau_{e01}=\tau_{e10}+\tau_{e11}$ that are defining conditions of our summations, so $\As([p],=)$ is not stable under this action.

We represent the elements of the wreath product $S_2 \Wr_{[2]} S_2$ as pairs of the form $\delta=\left((\digamma_0,\digamma_1),\sigma\right)$, where $\digamma_0,\digamma_1,\sigma \in S_2$ and $\delta$ acts on $(s,v) \in \lindexset$ by the rule
\[
\delta(s,v)=\left((\digamma_0,\digamma_1),\sigma\right)(s,v)=\left(\sigma(s),\digamma_{\sigma(s)}(v)\right) .
\]
It turns out that $S_2 \Wr_{[2]} S_2$ is isomorphic to the dihedral group of order $8$.
\begin{notation}[$\cWp$]\label{Walter}
Let $ p \in \N$.
Then we use $\cWp$ to denote the wreath product $(S_2 \Wr_{[2]} S_2) \Wr_{[p]} S_p$, whose elements are pairs of the form $\pi=\left((\delta_0,\ldots,\delta_{p-1}),\epsilon\right)$, with $\delta_0,\ldots,\delta_{p-1} \in S_2 \Wr_{[2]} S_2$ and $\epsilon \in S_p$, where $\pi$ acts on the $(e,s,v) \in \indexset$ by the rule
\[
\pi(e,s,v)=\left((\delta_0,\ldots,\delta_{p-1}),\epsilon\right)(e,s,v)=\left(\epsilon(e),\delta_{\epsilon(e)}(s,v)\right).
\]
We can make this even more explicit if for each $j \in [p]$, we write
\[
\delta_j=\left((\digamma_{j,0},\digamma_{j,1}),\sigma_j\right),
\]
and then
\[
\pi=\left(\left(\left((\digamma_{0,0},\digamma_{0,1}),\sigma_0\right),\ldots,\left((\digamma_{p-1,0},\digamma_{p-1,1}),\sigma_{p-1}\right)\right),\epsilon\right),
\]
with
\[
\pi(e,s,v)=\left(\epsilon(e),\sigma_{\epsilon(e)}(s),\digamma_{\epsilon(e),\sigma_{\epsilon(e)}}(v)\right).
\]
We say that {\it $\pi$ uses the permutation $\epsilon$ to permute the equations, then the permutation $\sigma_e$ to permute the sides of equation $e$ for each $e \in [p]$, and then the permutation $\digamma_{e,s}$ to permute the places on side $s$ of equation $e$ for each $(e,s) \in [p]\times [2]$}.
\end{notation}
Since each element $\pi \in \cWp$ permutes $\indexset$ by this action, we identify each $\pi \in \cWp$ with the permutation in $S_\indexset$ that permutes $\indexset$ the same way.
Thus, $\cWp$ is regarded as a subgroup of $S_\indexset$ of order
\[
|\cWp|=|S_p| \cdot |S_2 \Wr_{[2]} S_2|^p = p! 8^p = p! 2^{3 p}.
\] 
Now we show that our subgroup $\cWp$ preserves important properties within $\As([p])$.
\begin{lemma} \label{Cactus}
Let $p, \ell \in \N$ and suppose that $\pi \in \cWp$ and $\cP \in \Part(p)$.  Then
\begin{enumerate}[label=(\roman*)]
\item\label{BalloonCactus} $\pi^*(\As([p]))=\As([p])$,
\item\label{Saguaro} $\pi^*(\As([p],\ell))=\As([p],\ell)$,
\item\label{PricklyPear} $\pi^*(\As([p],=))=\As([p],=,\ell)$,
\item\label{BarrelCactus} $\pi^*(\As([p],=,\ell))=\As([p],=,\ell)$,
\item\label{CrownCactus} $\pi^*(\As(\pi(\cP))=\As(\cP)$,
\item\label{FeatherCactus} $\pi^*(\As(\pi(\cP),\ell))=\As(\cP,\ell)$,
\item\label{MoonCactus} $\pi^*(\As(\pi(\cP),=))=\As(\cP,=)$, and
\item\label{StarCactus} $\pi^*(\As(\pi(\cP),=,\ell))=\As(\cP,=,\ell)$.
\end{enumerate}
Therefore the action of $\cWp$ on $\As([p])$ can be restricted to any of the following subsets: $\As([p],\ell)$, $\As([p],=)$, $\As([p],=,\ell)$.
\end{lemma} 
\begin{proof}
Each part claims that there is an equality of sets, but we need only prove that the set on the left is contained in the set on the right, and then we can apply what we have proved with $\pi^{-1}$ in place of $\pi$ and $\pi(\cP)$ in place of $\cP$ to prove the other containment.  The first statement, \ref{BalloonCactus}, has already been made clear: $\pi^*$ is a permutation of $\As([p])$.

To prove \ref{Saguaro}, note that if $\tau \in \As([p],\ell)$, then $\tau(e,s,v) \in [\ell]$ for every $(e,s,v) \in \indexset$, and so of course $(\pi^*(\tau))(e,s,v)=(\tau\circ\pi)(e,s,v) \in [\ell]$ for every $(e,s,v) \in \indexset$.

To prove \ref{PricklyPear}, we define $\psi: \As([p]) \to \Z^p $ by $\psi(\tau) = (\tau_{e00}+\tau_{e01}-\tau_{e10}-\tau_{e11})_{e \in [p]}$.
Notice that if $\tau\in\As([p])$, then $\psi(\tau)=0$ if and only if $\tau\in \As([p],=)$.
Let $ \phi $ be the nontrivial homomorphism from $S_2$ onto $\{\pm 1, \cdot \}$.
Now we define an action of $\cWp$ on $\Z^p$: if $\pi\in\cWp$ with
\[ \pi=\left(\left(\left((\digamma_{0,0},\digamma_{0,1}),\sigma_0\right),\ldots,\left((\digamma_{p-1,0},\digamma_{p-1,1}),\sigma_{p-1}\right)\right),\epsilon\right),\]
with $\epsilon \in S_p$ and all $\sigma_e$ and $\digamma_{e,s}$ in $S_2$, and if $u = (u_e)_{e \in [p] } \in \Z^p$, then $\pi \cdot u = (\phi(\sigma_{\epsilon(e)}) u_{\epsilon(e)})_{e \in [p]}$.
Notice that if $\pi \in \cWp$ and $\tau \in \As([p])$, then $\pi \cdot \psi(\tau)=\psi(\pi^*(\tau)))$.
That is, the following diagram commutes:
\[
\begin{CD}
\As([p]) @> \pi^* >> \As([p])\\
@VV \psi V @VV \psi V\\
\Z^p
@> \pi \cdot  >> \Z^p.
	\end{CD}
\]
Recall that for $\tau\in\As([p])$, we have $\tau\in \As([p],=)$ if and only if $\psi(\tau)=0$.
Moreover, $\psi(\tau)=0 $ if and only if $\pi \cdot \psi(\tau) = 0$ for all $\pi \in \cWp$.
By the commutativity of our diagram $\pi \cdot \psi(\tau) = 0 $ for all $\pi \in \cWp$ if and only if $\psi(\pi^*(\tau))=0$ for all $\pi \in \cWp$, i.e., $\pi^*(\tau) \in \As([p],=)$ for all $\pi \in \cWp$.

Now \ref{BarrelCactus} follows from \ref{Saguaro} and \ref{PricklyPear} because $\As([p],=,\ell)=\As([p],\ell)\cap\As([p],=)$.

To prove \ref{CrownCactus}, suppose that $\tau \in \As(\pi(\cP))$.
Suppose $\gamma, \gamma' \in \indexset$.
Then $(\pi^*(\tau))_\gamma = (\pi^*(\tau))_{\gamma'}$ if and only if $\tau_{\pi(\gamma)} = \tau_{\pi(\gamma')}$.
Since $\tau\in \As(\pi(\cP))$, we have $\tau_{\pi(\gamma)} = \tau_{\pi(\gamma')}$ if and only if $\pi(\gamma)\equiv \pi(\gamma')  \pmod{\pi(\cP)}$.
Now, $\pi(\gamma) \equiv \pi(\gamma')  \pmod{\pi(\cP)}$ if and only if $\gamma \equiv \gamma' \pmod{\cP}$.
This shows that $\cP$ is the partition induced by $\pi^*(\tau)$, that is $\pi^*(\tau) \in \As(\cP)$.

Then \ref{FeatherCactus}, \ref{MoonCactus}, and \ref{StarCactus} respectively follow from \ref{CrownCactus} in conjunction with \ref{Saguaro}, \ref{PricklyPear}, \ref{BarrelCactus}, respectively.
\end{proof}

Recall that $\cWp$ is a subgroup of $S_\indexset$, and therefore it acts on elements of $\indexset$, and by extension on subsets of $\indexset$, that is, $\pi(P)=\{\pi(e,s,v): (e,s,v) \in P\}$ for $\pi\in\cWp$ and $P \subseteq \indexset$.
By further extension, $\cWp$ acts on sets of subsets of $\indexset$, that is, $\pi(\cQ)=\{\pi(Q): Q \in \cQ\}$ for $\pi\in\cWp$ and $\cQ$ is a sets of subsets of $\indexset$.
These give actions of $\cWp$ on $\indexset$ and on $\Part(p)$.
\begin{definition}[Isomorphic partitions]
Let $ p \in \N $ and $ \cP, \cQ \in \Part(p)$.  Then we say that $ \cP $ and $\cQ$ are {\it isomorphic} and write $\cP \cong \cQ$ to mean that there exists $ \pi \in \cWp $ such that 
$ \cQ = \pi(\cP) $.
\end{definition}
Since $\cWp$ is a group, the isomorphism relation is clearly an equivalence relation.   
We introduce the stabilizer of a partition under the action of $\cWp$; this will help us to determine the size of isomorphism classes.

\begin{notation}[$\Stab_{\cWp}(\cP)$]
Let $p \in \N$ and $ \cP \in \Part(p)$.
Then we use $ \Stab_{\cWp}(\cP)$ to denote the stabilizer in $\cWp$  of $\cP$ under the action of $\cWp$ on $\Part(p)$.
\end{notation}

Suppose that $ p \in \N$, and $ \cP \in \fP $. Then the orbit of $ \cP $ under the action of $ \cWp $ is $ \fP $. Hence, by the orbit-stabilizer formula we arrive at the following lemma.
\begin{lemma}
Let $ p \in \N $ and $ \cP \in \Part(p)$ and let $\fP$ be the equivalence class of $\cP$ under the action of $\cWp$.  Then 
\[|\fP| = \frac{|\cWp|}{|\Stab_{\cWp}(\cP)|}=\frac{p! 2^{3 p}}{|\Stab_{\cWp}(\cP)|}.\]
\end{lemma}
Recall that a permutation $ \pi \in \cWp $ acts on $\As([p],\ell)$ by composition on the right.

\begin{lemma}\label{Nicholas}
Let $ p \in \N $, $ \pi=\left((\delta_0,\ldots,\delta_{p-1}),\epsilon\right) \in \cWp$, $ \cP \in \Part(p) $ and $e\in [p] $.
Then $\pi(\cP)_{\{\epsilon(e)\}}=\pi(\cP_{\{e\}})$, which has the same type as $\cP_{\{e\}}$.
\end{lemma}
\begin{proof}
For each $k \in [p]$, we define $ \Res_k : \Part(p) \to \Part(\{k\})$ by $\Res_k(\cP) = \cP_{\{k\}}$.
Notice that $ \pi(\eindexset) =  \{\epsilon(e) \}\times \lindexset$ because $\pi \in \cWp$.
This leads to the following commutative diagram:
\[
\begin{CD}
\Part(p) @>\pi>> \Part(p) \\
@VV\Res_e V @VV\Res_{\epsilon(e)}V\\
\Part(\{e\})
@>\pi>>  \Part(\{\epsilon(e)\}).
\end{CD}
\]
That is, $\pi(\cP)_{\{\epsilon(e)\}}=\pi(\cP_{\{e\}})$. 
Since $ \pi $ permutes the elements in $ \indexset $, we have $|P|=|\pi(P)| $ for $ P \in \cP_{\{e\}} $.
It follows that $ \pi(\cP_{\{e\}})$ is of the same type as $ \cP_{\{e\}} $.
\end{proof}
\begin{corollary}
Let $ p \in \N $, $ \cP \in \Part(p) $, $ \pi=\left((\delta_0,\ldots,\delta_{p-1}),\epsilon\right) \in \Stab_{\cWp}(\cP)$ and $e \in [p]$.
Then $\cP_{\{\epsilon(e)\}}=\pi(\cP_{\{e\}})$, which has the same type as $\cP_{\{e\}}$.
\end{corollary}
The globally even, locally odd property is stable under our action.
\begin{lemma}\label{Pineapple}
Let $p\in \N$ and $ \cP, \cQ \in \Part(p)$ with $ \cP \cong \cQ $.  Then $ \cP \in \gelo(p) $ if and only if $\cQ \in \gelo(p)$.
\end{lemma}
\begin{proof}
Since $ \cP \cong \cQ $ then $ \pi(\cP) = \cQ $ for some $ \pi \in \cWp $ where $\pi=(\delta,\epsilon)$ with $\delta \in (S_2\Wr_{[2]} S_2)^p$ and $\epsilon \in S_p$.
Notice that since $\pi$ only permutes elements of the underlying set $\indexset$ of which $\cP$ and $\cQ$ are partitions, then $|P| = |\pi(P)| $ for all $ P \in \cP $.
So,  $ \cP $ is even if and only if $ \pi(\cP) $ is even.
		
Since $ \pi $ is a permutation, $\cP_{\{e\}}$ is non-even for all $ e \in [p] $ if and only if $ \pi(\cP_{\{e\}})  $ is non-even for all $ e \in [p] $.
By \cref{Nicholas} this means that $\cP_{\{e\}}  $ is non-even for all $ e \in [p] $ if and only if $ \pi(\cP)_{\{\epsilon(e)\}}  $ is non-even for all $ e \in [p] $.
Since $ \epsilon $ is a permutation of $ [p] $ this means $\cP_{\{e\}}  $ is non-even for all $ e \in [p] $ if and only if $ \pi(\cP)_{\{f\}}  $ is non-even for all $ f \in [p] $.
Thus $ \cP \in \gelo(p) $ if and only if $\pi(\cP) \in \gelo(p)$. 
\end{proof}
The property of satisfiability is also stable under our action.
\begin{lemma} \label{Celery}
Let $p,\ell \in \N$. If $\cP, \cQ \in \Part(p)$ with $\cP \cong \cQ$, then $|\As(\cP,=,\ell)|=|\As(\cQ,=,\ell)|$.
In particular, $\cP \in \Sat(p)$ if and only if $\cQ \in \Sat(p)$.
\end{lemma} 
\begin{proof}
Since $\cP \cong \cQ$, then $\cQ = \pi(\cP)$ for some $\pi \in \cWp$.
Then by \cref{Cactus}\ref{StarCactus}, we know that $\pi^*(\As(\cQ,=,\ell))=\As(\cP,=,\ell)$, and since $\pi^*$ is a permutation of $\As([p])$, we have $|(\As(\cQ,=,\ell))|=|\As(\cP,=,\ell)|$.
Since the partition $\cP$ (resp., $\cQ$) lies in $\Sat(p)$ if and only if $\As(\cP,=,\ell)$ (resp., $\As(\cQ,=,\ell)$) is nonempty for some $\ell \in \N$, we now see that $\cP$ is in $\Sat(p)$ if and only if $\cQ$ is in $\Sat(p)$.
\end{proof}
Recalling that $\Con(p)=\gelo(p)\cap\Sat(p)$, \cref{Pineapple} and \cref{Celery} show that the contributory property is also stable under our action.
\begin{corollary}\label{sticks}
Let $p, \ell \in \N $. If $ \cP, \cQ \in \Part(p)$ with $ \cP \cong \cQ $, then $ \cP \in \Con(p) $ if and only if $\cQ \in \Con(p)$, and furthermore $|\As(\cP,=,\ell)|=|\As(\cQ,=,\ell)|$.
\end{corollary}
This last result shows each equivalence class in $\Part(p)$ under the action of $\cWp$ either contains only contributory partitions or no contributory partitions at all.
Since we are primarily interested in the contributory partitions and their equivalence classes, we make a name for the set of all such classes.
\begin{definition}[$\Isom(p)$]
Let $ p \in \N $. We denote the set of isomorphism classes of partitions in $ \Con(p) $ as $ \Isom(p)$.
\end{definition}
In view of \cref{sticks}, it is helpful to have a notation for the common value of $|\As(\cP,=,\ell)|$ for all partitions $\cP$ in an isomorphism class of contributory partitions.
\begin{definition}[$\Sols(\fP,\ell)$]\label{Sonia}
Let $ p, \ell \in \N $. 
If $\fP$ is any subset of $\Part(p)$ such that all partitions in $\fP$ are isomorphic to each other, we let $\Sols(\fP,\ell)$ be the common value (by \cref{sticks}) of $|\As(\cP,=,\ell)|$ for $\cP \in \fP$.
\end{definition}
We most commonly use this definition when $\fP \in \Isom(p)$.
Now our formula \cref{Sanri} for central moments of the sum of squares of autocorrelation can be made much less unwieldy by grouping terms according to isomorphisms classes.
\begin{proposition}\label{Sanria}
If $ p,\ell \in \N$, then
\[ 
\mom{p} \ssac(f) = \sum_{ \fP \in \Isom(p)} |\fP| \Sols(\fP,\ell).
\]
\end{proposition}
\begin{proof}
By \cref{Sanri} we know
\begin{align*}
\mom{p} \ssac(f)
& = \sum_{\cP \in \Con(p)} |\As(\cP,=,\ell)| \\
& = \sum_{\fP \in \Isom(p)} \sum_{\cP \in \fP} |\As(\cP,=,\ell)|.
\end{align*}
Thus by \cref{sticks}
\[
\mom{p} \ssac(f) = \sum_{\fP \in \Isom(p)} |\fP| \Sols(\fP,\ell).  \qedhere
\]
\end{proof}

\subsection{Structure of contributory partitions}\label{Scott}
In this section, we will go over some general results that help us understand some of the basic requirements needed for a partition to be contributory. By finding which partitions are contributory, we can focus on counting only those specific partitions. This will make the calculations in Sections \ref{Veronica}--\ref{Simon} more manageable.

We first define an attribute of partitions and their classes that makes it easier to classify contributory partitions.
\begin{definition}[Cross-section]
Let $p \in \N$.
If $P \subseteq \indexset$, then the {\it $p$-$p$-cross-section} of $P$, written $\cs_p(P)$, is defined to be the multiset $\bms{|P_{\{e\}}|: e\in [p]}$.
If $\cP \in \Part(p)$, then the {\it $p$-cross-section} of $\cP$, written $\cs_p(\cP)$,  is defined to be the multiset $\ms{\cs_p(P): P \in \cP}$.
\end{definition}
We show that $p$-cross-section is stable under the action of our group $\cWp$.
\begin{lemma}\label{Brandee}
Let $p \in \N$ and let $\pi \in \cWp$.
If $P \subseteq \indexset$, then the $p$-cross-section of $\pi(P)$ is the same as that of $P$.
If $\cP \in \Part(p)$, then the $p$-cross-section of $\pi(\cP)$ is the same as that of $\cP$.
\end{lemma}
\begin{proof}
Let $P \subseteq \lindexset$.
Let $\epsilon \in S_p$ and $\delta_0,\ldots,\delta_{p-1} \in S_2 \Wr_{[2]} S_2$ such that $\pi=((\delta_0,\ldots,\delta_{p-1}),\epsilon)$.
For each $e \in [p]$, \cref{Nicholas} shows that $|\pi(\cP_{\{e\}})|=|\pi(\cP)_{\{\epsilon(e)\}}|$, and since $\pi$ is a permutation, this means that $|\cP_{\{e\}}|=|\pi(\cP)_{\{\epsilon(e)\}}|$.
Since $\epsilon$ is a permutation of $[p]$, we obtain the same multiset of values $|\cP_{\{e\}}|$ as $e$ runs through $[p]$ as the multiset of values $|\pi(\cP)_{\{e\}}|$ as $e$ runs through $[p]$.

Let $\cP \in \Part(p)$, let $m=|\cP|$, and write $\cP=\{P_0,\ldots,P_{m-1}\}$.
Then the $p$-cross-section of $\pi(\cP)$ is the $p$-cross-section of $\{\pi(P_0),\ldots,\pi(P_{m-1})\}$.
From the previous paragraph this is the same as the $p$-cross-section of $\{P_0,\ldots,P_{m-1}\}$.
\end{proof}
The following notion is also helpful in describing and classifying contributory partitions.
\begin{definition}[Twin class]
Let $ p \in \N$.  A {\it twin class} is any subset $P$ of $\indexset$ such that there exists some $(e,s) \in [p] \times [2]$ with $\{ (e,s,0), (e,s,1) \} \subseteq P$.
\end{definition} 	
The following result indicates some of the general features of contributory partitions.
\begin{lemma}\label{Gideon}
Let $p\in \Z_+$ and $\cP \in \Con(p)$.
\begin{enumerate}[label=(\roman*)]
\item\label{Anise} For each $P\in\cP$, each element of $\cs_p(P)$ is in $\{0,1,2\}$, and $\sum_{m \in \cs_p(P)} m=|P|$, so that $\cs_p(P)$ contains an even number of instances of $1$.
\item\label{Beetroot} For each $j \in \{0,1,2\}$, let $N_j$ be the total number of instances of $j$ among all the multisets in $\cs_p(\cP)$.  Then $N_0+N_1+N_2=p |\cP|$ and \[
  N_1+2 N_2=\sum_{M \in \cs_p(\cP)} \sum_{m \in M} m= \big|\indexset\big|=4 p,\] and $N_1 \geq 2 p$ while $N_2 \leq p$.
\item\label{Corriander} For each $P \in \cP$, we have $|P| \leq 2 p$, and if there is some $P \in \cP$ with $|P|=2 p$, then $\cs_p(P)=\bms{2,2,\ldots,2}$ (has $p$ instances of $2$ and no other elements).
\item\label{Epazote} For each $e\in[p]$, the restriction $\cP_{\{e\}}$ is a partition of type $\ms{2,1,1}$ or $\ms{1,1,1,1}$.  In the former case, the equivalence class with cardinality $2$ is a twin class, i.e., there is some $s \in [2]$ such that $\cP_{\{e\}} = \{\{(e,s,0),(e,s,1)\}, \{(e,1-s,0)\},\{(e,1-s,1)\}\}$.  In the latter case, $\cP_{\{e\}}= \{\{(e,0,0)\},\{(e,0,1)\},\{(e,1,0)\},\{(e,1,1)\}\}$.
\item\label{Dill} We have $3 \leq |\cP| \leq 2 p$, and if $p$ is odd then $|\cP| \geq 4$.
\end{enumerate}
\end{lemma}
\begin{proof}
We first prove \ref{Anise}.
Since $\cP$ is GELO, we know that for each $e \in [p]$, the restriction $\cP_{\{e\}}$ is a non-even partition of the set $\{e\}\times\lindexset$, so it cannot have a only one set with all four elements.  Thus, for every $P \in \cP$, we know that $|P_{\{e\}}|\not=4$, so $4$ cannot occur in $\cs_p(P)$.
Suppose that there were some $e \in [p]$ and $P \in \cP$ with $|P_{\{e\}}|=3$ to show a contradiction.  Then there are some $s,v \in [2]$ such that $(e,s,0),(e,s,1),(e,1-s,v) \in P$, and there must be some $Q \in \cP$ with $Q\not=P$ and $(e,1-s,1-v) \in Q$.
Since $\cP \in \Sat(p)$, there is some $\tau \in \As(\cP,=)$ such that $\tau_{e,s,0}=\tau_{e,s,1}=\tau_{e,1-s,v}\not=\tau_{e,1-s,1-v}$ and $\tau_{e,s,0}+\tau_{e,s,1}=\tau_{e,1-s,v}+\tau_{e,1-s,1-v}$, which is a contradiction.
So for every $P \in \cP$, the $p$-cross-section $\cs_p(P)$ cannot contain any number larger than $2$.
For each $P \in \cP$, the summation $\sum_{m \in \cs_p(P)} m=\sum_{e \in [p])} |P_{\{e\}}|=|P|$, which is an even number since $P$ is a class from a GELO partition, so the number of $1$'s that occur in $\cs_p(P)$ must be even.

Now we prove \ref{Beetroot}.
By definition, for every $P \in \cP$, the $p$-cross-section $\cs_p(P)$ is a multiset containing $p$ elements, so $N_0+N_1+N_2 = p |\cP|$.
Since $\cP$ is a partition of $\indexset$, we have $\sum_{M \in \cs_p(\cP)} \sum_{m \in M} m = \sum_{P \in \cP} \sum_{e \in [p]} |P_{\{e\}}| = \sum_{P \in \cP} |P| = \big|\indexset\big|=4 p$, but we have defined our $N_j$'s so that this sum is also $0 N_0 + 1 N_1 + 2 N_2$.
If we had $N_2 > p$, then by the pigeonhole principle, there would need to be some $e \in [p]$ and distinct $P, Q \in \cP$ such that $|P_{\{e\}}|=|Q_{\{e\}}|=2$, but this would mean that $\cP_{\{e\}}$ is a partition of type $\ms{2,2}$, which would violate the fact that $\cP$ is GELO.
Since $N_1 + 2 N_2 = 4 p$, and $N_2 \leq p$, we know that $N_1 \geq 2 p$.

Now we prove \ref{Corriander}.
Let $P \in \cP$.
Since $\cs_p(P)$ consists of $p$ numbers from $\{0,1,2\}$, we have $|P|=\sum_{m \in  \cs_p(P)} m \leq \sum_{m \in \cs_p(P)} 2 = 2 |\cs_p(P)|= 2 p$.
Furthermore, we have equality if and only $m=2$ for every $m \in \cs_p(P)$, i.e., if and only $\cs_p(P)$ consists of $p$ instances of $2$ (and nothing else).

Now we prove \ref{Epazote}.
Let $e \in [p]$.
Then $\cP_{\{e\}}$ is a non-even partition (because $\cP$ is GELO) of the set $\{(e,s,v): s,v \in [2]\}$ of $4$ elements into classes of size at most $2$ (by part \ref{Anise} of this lemma), it must be of type $\ms{2,1,1}$ or $\ms{1,1,1,1}$.
In the former case, suppose for a contradiction that the class of size $2$ in $\cP_{\{e\}}$ contained $(e,t,v)$ and $(e,1-t,w)$ for some $t,v,w \in [2]$, so that the classes of size $1$ are $\{(e,t,1-v)\}$ and $\{(e,1-t,1-w)\}$.
Since $\cP \in Sat(p)$, there is some $\tau \in \As(\cP,=)$ such that $\tau_{e,t,1-v}$ and $\tau_{e,1-t,1-w}$ are distinct from each other and from $\tau_{e,t,v}=\tau_{e,1-t,w}$.
But then this means that $\tau_{e,t,v}+\tau_{e,t,1-v}\not=\tau_{e,1-t,w}+\tau_{e,1-t,1-w}$, contradicting the fact that $\tau\in\As(\cP,=)$.
Thus the class of size $2$ in $\cP_{\{e\}}$ must contain two elements of the form $(e,s,0)$ and $(e,s,1)$ for some $s\in[2]$, while the other two classes are the singleton classes $\{(e,1-s,0)\}$ and $\{(e,1-s,1)\}$.
On the other hand, if $\cP_{\{e\}}$ is of type $\ms{1,1,1,1}$, then $\cP_{\{e\}}$ is a partition of $\{(e,0,0),(e,0,1),(e,1,0),(e,1,1)\}$ into four singleton classes.

Now we prove \ref{Dill}.
By \ref{Epazote}, $\cP_{\{0\}}$ has at least $3$ classes, so $\cP$ must have at least three classes.
Since $\cP\in\Con(p)\subseteq\gelo(p)$, all classes in the partition $\cP$ are of even cardinality, hence they have at least $2$ elements, and the sum of all these cardinalities is $\big|\indexset\big|=4 p$, so the number $|\cP|$ of classes cannot exceed $4 p/2=2 p$.

We prove the final claim in \ref{Dill} by contraposition: we assume that $\cP$ has precisely three equivalence classes and prove that $p$ is even.
By \ref{Epazote} each $\cP_{\{j\}}$ must be type $\ms{2,1,1}$, so every class $S$ of $\cP$ has $S_{\{j\}}\not=\emptyset$ and there must be at least one twin class $P$ in $\cP$.
So there are some $e \in [p]$ and $s \in [2]$ such that $(e,s,0), (e,s,1) \in P$.

We claim that the other two classes in $\cP$ are not twin classes.
Suppose that $Q$ is another twin class to show contradiction.
Then there would be some $f \in [p]$ and $t \in [2]$ such that $(f,t,0), (f,t,1) \in Q$ and some $u,v \in [2]$ such that $(e,1-s,u) \in Q$ and $(f,1-t,v) \in P$.
So if $R$ is the third class in $\cP$, then $(e,1-s,1-u),(f,1-t,1-v) \in R$.
We claim that there cannot be any $\tau \in \As(\cP,=)$, for if we let $A=\tau_{e,s,0}$, $B=\tau_{f,t,0}$, and $C=\tau_{e,1-s,1-u}$, then $A$, $B$, and $C$ must be distinct (because $\tau\in\As(\cP)$) and (because $\tau\in\As(=)$), they must satisfy
\begin{align*}
2A &=B+C \\
2B &=A+C,
\end{align*}
which implies that $A=B$, contradicting the distinctness that we just noted.

So if $\cP$ has precisely three classes, then $\cP$ has one twin class and two non-twin classes.
If $S$ is a non-twin class of $\cP$, then since we have seen that $S_{\{j\}}$ is nonempty for every $j \in [p]$, we must have $|S_{\{j\}}|=1$ for every $j \in [p]$.
This implies $|S|=p$, and since a contributory partition is even, this means that $p$ is even.
\end{proof}

\section{Positivity of moments}\label{Prunella}

We now our general theory to prove that the central moments are strictly positive, except when the sequences involved are very short (in which case the central moments are zero).
\begin{theorem}\label{Lynn}
Let $\ell$ and $p$ be positive integers.
Then $\mom{p} \ADF(f)$ is nonnegative.
Moreover, if (i) $p=1$, (ii) $p$ is odd with $p>1$ and $\ell\leq 3$, or (iii) $p$ is even and $\ell\leq 2$, then $\mom{p} \ADF(f)$ is zero; otherwise it is strictly positive.  
\end{theorem}
\begin{proof}
\cref{Sanri} gives the $p$th central moment as
\begin{equation}\label{Eustace}
\mom{p} \ssac(f) = \sum_{\cP \in \Con(p)} |\As(\cP,=,\ell)|,
\end{equation}
which is clearly nonnegative.
The first central moment is trivially zero.
By \cref{Gideon}\ref{Dill}, a contributory partition $\cP$ must have at least three classes if $p$ is even and at least four classes if $p$ is odd.
If $\ell < |\cP|$, then $|\As(\cP,=,\ell)|=0$ since the tuples in distinct classes must be assigned distinct elements of $[\ell]$.
Therefore the $p$th central moment must be zero if $p$ is even and $\ell \leq 2$ or if $p$ is odd and $\ell \leq 3$.

Now we show that the $p$th central moment is strictly positive when $p$ is even and $\ell \geq 3$.
Let $\cP=\{P_0,P_1,P_2\}$ be the following partition:
\begin{itemize}
\item $P_k=\{(e,0,k): e \in [p]\}$ for each $k \in [2]$ and
\item $P_2=\{(e,1,v): e \in [p], v \in [2] \}$.
\end{itemize}
Since $|P_0|=|P_1|=p$ and $|P_2|=2 p$, we see that $\cP$ is even.
Furthermore, $\cP_{\{e\}}$ is of type $\ms{2,1,1}$ for every $e \in [p]$, so $\cP$ is GELO.
Let $\tau \in \As(p)$ with $\tau_\gamma=0$, $2$, or $1$, respectively, when $\gamma \in P_0$, $P_1$, or $P_2$, respectively.
Then $\tau \in \As(\cP,=,\ell)$ for every $\ell \geq 3$, so then $\cP \in \Con(p)$ and $|\As(\cP,=,\ell)| > 0$ if $\ell \geq 3$.
From \eqref{Eustace}, one sees that this makes the $p$th central moment strictly positive.

Finally, we show that the $p$th central moment is strictly positive when $p>1$ is odd and $\ell \geq 4$.
Let $\cP=\{P_0,P_1,P_2,P_3\}$ be the following partition:
\begin{itemize}
\item $P_k = \{ (k,0,0), (k,0,1), (k+1 \bmod 2,1,0), (e,0,k) : 2 \leq e < p \} $ for $ k\in [2] $ and 
\item $P_k = \{ (k \bmod 2, 1,1), (e,1,k \bmod 2) : 2 \leq e < p  \} $ for $k \in \{2,3\} $.
\end{itemize}
Since $|P_0|=|P_1|=p+1$ and $|P_2|=|P_3|=p-1$, the partition $\cP$ is even.
Furthermore, $\cP_{\{0\}}$ and $\cP_{\{1\}}$ are of type $\ms{2,1,1}$ and $\cP_{\{e\}}$ is of type $\ms{1,1,1,1}$ for $2 \leq e < p$, we see that $\cP$ is GELO.
Let $\tau \in \As(p)$ with $\tau_\gamma=1$, $2$, $0$, or $3$, respectively, when $\gamma \in P_0$, $P_1$, $P_2$, or $P_3$, respectively.
Then $\tau \in \As(\cP,=,\ell)$ for every $\ell \geq 4$, so then $\cP \in \Con(p)$ and $|\As(\cP,=,\ell)| > 0$ if $\ell \geq 4$.
From \eqref{Eustace}, one sees that this makes the $p$th central moment strictly positive.
\end{proof}
\begin{remark}\label{Raphael}
\cref{Leonard} follows from \cref{Lynn} because when $\ell > 0$, we have $\ADF(f)=-1+\ssac(f)/\ell^2$.
\end{remark}

\section{Explicit calculation of variance}\label{Veronica}
In this chapter, we verify Jedwab's formula (\cref{Jessica}) for the variance of the demerit factor by finding a formula for the variance of $\ssac$, the sum of the squares of the autocorrelations.
This comes from the $p=2$ case of \cref{Sanria}, where we explicitly determine the isomorphism classes of partitions in $\Con(2)$ and the number of solutions attached to each in the formula of \cref{Sanria}.
It helps first to notate the partitions that we shall eventually show to make up $\Con(2)$, so for every $s,t,v,w \in [2]$, we define
\begin{align*} 
\cC_{1,s,t,v} = \Big\{& \{(0,s,0),(0,s,1),(1,t,0),(1,t,1)\}, \\ & \{(0,1-s,0),(1,1-t,v)\}, \{(0,1-s,1),(1,1-t,1-v)\}\Big\}\text{ and } \\
\cC_{2,s,v,w} = \Big\{& \{(0,0,0),(1,s,v)\}, \{(0,0,1),(1,s,1-v)\}, \\ & \{(0,1,0),(1,1-s,w)\}, \{(0,1,1),(1,1-s,1-w)\}\Big\}.
\end{align*}
We organize these partitions into classes as follows
\begin{align*}
\fC_1 & = \{\cC_{1,s,t,v}: s,t,v \in [2]\} \text{ and} \\
\fC_2 & = \{\cC_{2,s,v,w}: s,v,w \in [2]\}.
\end{align*}
We shall eventually show that $\Isom(2)=\{\fC_1,\fC_2\}$.

For each of the two classes $\fC_1$ and $\fC_2$, it is not difficult to show that different tuples of parameters produce different partitions.
For example, in $\cC_{1,s,t,v}$, note that $s$ is the only element of $[2]$ such two elements of the form $(0,s,*)$ appear in the same class of the partition, $t$ is the only element of $[2]$ such two elements of the form $(1,t,*)$ appear in the same class of the partition, and $v$ is the only element of $[2]$ such that an element of the form $(1,1-t,v)$ lies in the same class of the partition as $(0,1-s,0)$.
Or, in $\cC_{2,s,v,w}$, $s$ and $v$ are the unique elements of $[2]$ such that $(1,s,v)$ lies in the same class of the partition as $(0,0,0)$ and $w$ is the unique element of $[2]$ such that $(1,1-s,w)$ lies in the class of the partition containing $(0,1,0)$.
Since partitions with different parameters are distinct, this shows that $|\fC_1|=8$ and $|\fC_2|=8$.

Now we claim that partitions within the same class $\fC_m$ (for $m \in \{1,2\}$) are isomorphic to each other but are not isomorphic to those in the other class.
The non-isomorphicity is clear: partitions in $\fC_1$ have three classes and those in $\fC_2$ have four.
To see isomorphicity, recall the terminology at the end of \cref{Walter} and then notice that given an arbitrary $\cC_{1,s,t,v}$, one can arrive at $\cC_{1,0,0,v}$ by applying elements of $\cWtwo$ that permute the sides of equations $0$ or $1$, and ultimately get $\cC_{1,0,0,0}$ by applying the element of $\cWtwo$ that transposes the places on side $1$ of equation $1$ if $v\not=0$.
Or given an arbitrary $\cC_{2,s,v,w}$, one can arrive at $\cC_{2,0,v,w}$ by applying the element of $\cWtwo$ that transposes the sides of equation $1$ if $s\not=0$, and ultimately get $\cC_{2,0,0,0}$ by applying elements of $\cWtwo$ that permute places on sides of equation $1$.

Now we claim that $\Con(2) \subseteq \fC_1 \cup \fC_2$.
The evenness of contributory partitions and \cref{Gideon}\ref{Dill} forces a contributory partition $\cP$ to be of type $\ms{4,2,2}$ or $\ms{2,2,2,2}$.
In the former case, we let $P$ denote the class of size $4$, and then \cref{Gideon}\ref{Epazote} shows that there must be some $s,t \in [2]$ such that $P=\{(0,s,0),(0,s,1),(1,t,0),(1,t,1)\}$, and the two remaining classes must then be $\{(0,1-s,0),(1,1-t,v)\}$ and $\{(0,1-s,1),(1,1-t,1-v)\}$ for some $v \in [2]$, so that $\cP=\cC_{1,s,t,v} \in \fC_1$.
In the latter case, we first claim that it is impossible for any class of $\cP$ to be a twin class, for if this were the case then \cref{Gideon}\ref{Epazote} would force our classes to be $\{(0,s,0),(0,s,1)\}$, $\{(1,t,0),(1,t,1)\}$, $\{(0,1-s,0),(1,1-t,v)\}$, and $\{(0,1-s,1),(1,1-t,1-v)\}$ for some $s, t, v \in [2]$, and then any $\tau\in\As(\cP,=)$ will necessarily have $2 \tau_{0,s,0}= \tau_{0,s,0}+\tau_{0,s,1}=\tau_{0,1-s,0} + \tau_{0,1-s,1} = \tau_{1,1-t,v} + \tau_{1,1-t,1-v} = \tau_{1,t,0}+\tau_{1,t,1}=2 \tau_{1,t,0}$, which forces $\tau_{0,s,0}=\tau_{1,t,0}$ but $(0,s,0)$ and $(1,t,0)$ are not in the same class of $\cP$, so $\As(\cP,=)=\emptyset$ and $\cP$ would not be satisfiable.
So the classes of $\cP$ must be of the form $\{(0,0,0),(1,s,v)\}$, $\{(0,0,1),(1,s',v')\}$, $\{(0,1,0),(1,t,w)\}$, and $\{(0,1,1),(1,t',w')\}$ for some $s,s',t,t',v,v',w,w' \in [2]$, so that for any $\tau\in\As(\cP,=0)$ we must have $\tau_{1,s,v}+\tau_{1,s',v'}=\tau_{0,0,0}+\tau_{0,0,1}=\tau_{0,1,0}+\tau_{0,1,1}=\tau_{1,t,w}+\tau_{1,t',w'}$.
If $s$ were equal to $t$ (resp., $t'$), then we would also have the equation $\tau_{1,s,v}+\tau_{1,t,w}=\tau_{1,s',v'}+\tau_{1,t',w'}$ (resp., $\tau_{1,s,v}+\tau_{1,t',w'}=\tau_{1,s',v'}+\tau_{1,t,w}$), which along with the previous equation implies $\tau_{1,s,v}=\tau_{1,t',w'}$ (resp., $\tau_{1,s,v}=\tau_{1,t,w}$), but $(1,s,v)$ and $(1,t',w')$ (resp., $(1,t,w)$) are not in the same class of $\cP$, so $\As(\cP,=)=\emptyset$ and $\cP$ would not be satisfiable.
This forces $s'=s$, $t'=t=1-s$, $v'=1-v$, and $w'=1-w$, which makes $\cP=\cC_{2,s,v,w}\in\fC_2$.

Next we claim that that $\Isom(2)=\{\fC_1,\fC_2\}$.
It suffices to show that $\fC_1 \cup \fC_2 \subseteq \Con(2)$, since we already know that partitions in $\fC_1$ (resp. $\fC_2$) are isomorphic to each other but not isomorphic to those in $\fC_2$ (resp., $\fC_1$).
Since the partitions in $\fC_1$ and $\fC_2$ are clearly GELO, it suffices to that they are satisfiable by showing that $\Sols(\fC_1,=\ell)$ and $\Sols(\fC_2,=,\ell)$ are nonzero functions of $\ell$.
Now $\Sols(\fC_1,\ell)=|\As(\cC_{1,0,0,0},=,\ell)|$.  If $\tau \in \As(\cC_{1,0,0,0},\ell)$ and we let $A$ (resp., $B$, $C$) denote the element of $[\ell]$ such that $\tau_\gamma=A$ (resp., $B$, $C$) for $\gamma$ in $\{(0,0,0),(0,0,1),(1,0,0),(1,0,1)\}$ (resp., in $\{(0,1,0),(1,1,0)\}$, $\{(0,1,1),(1,1,1)\}$), then $\tau \in \As(\cC_{1,0,0,0},=,\ell)$ if and only if we satisfy the following equations:
\begin{align*}
A+A&=B+C\\
A+A&=B+C
\end{align*}
with distinct $A, B, C \in [\ell]$.  So we are counting the number of solutions of $2 A=B+C$ with distinct elements $A,B,C \in [\ell]$, which is given by \cref{Persephone}, so that
\[
\Sols(\fC_1,\ell) =\floor{\frac{(\ell-1)^2}{2}}.
\]
Likewise, $\Sols(\fC_2,\ell)=|\As(\cC_{2,0,0,0},=,\ell)|$.
If $\tau \in \As(\cC_{2,0,0,0},\ell)$ and we let $A$ (resp., $B$, $C$, $D$) denote the element of $[\ell]$ such that $\tau_\gamma=A$ (resp., $B$, $C$, $D$) for $\gamma$ in $\{(0,0,0),(1,0,0)\}$ (resp., in $\{(0,0,1),(1,0,1)\}$, $\{(0,1,0),(1,1,0)\}$, $\{(0,1,1),(1,1,1)\}$), then $\tau \in \As(\cC_{2,0,0,0},=,\ell)$ if and only if we satisfy the following equations:
\begin{align*}
A+B&=C+D\\
A+B&=C+D
\end{align*}
with distinct $A, B, C, D \in [\ell]$.
So we are counting the number of solutions of $A+B=C+D$ with distinct elements $A,B,C,D \in [\ell]$, which is given by \cref{wowzers}, so that
\[
\Sols(\fC_2,\ell) = \begin{cases}
\frac{2 \ell^3-9 \ell^2+10\ell}{3} & \text{if $\ell$ is even,} \\[4pt]
\frac{2 \ell^3-9 \ell^2+10\ell-3}{3} & \text{if $\ell$ is odd}.
\end{cases}
\]

Since we know that $\Isom(2)=\{\fC_1,\fC_2\}$ with $|\fC_1|=|\fC_2|=8$, and we have determined $\Sols(\fC_1,\ell)$ and $\Sols(\fC_2,\ell)$ above, we can compute the variance of $\ssac$ using \cref{Sanria} with $p=2$.
\begin{theorem}
For any positive integer $\ell$,
\[
\mom{2} \ssac(f) =
\begin{cases}
\frac{16\ell^3-60\ell^2+56\ell}{3} & \text{if $\ell$ is even,} \\[4pt]
\frac{16\ell^3-60\ell^2+56\ell -12}{3} & \text{if $\ell$ is odd.} 
\end{cases}
\]
\end{theorem}
Since $\ADF(f)=-1+\ssac(f)/\ell^2$, we can divide this result by $\ell^4$ to verify \cref{Jessica} in the Introduction.

\section{Explicit calculation of skewness}\label{Simon}

In this chapter, we will find an explicit formula for the skewness of the demerit factor of binary sequences of length $\ell $, with independent uniformly distributed entries, as a function of $\ell $.
Since the demerit factor is $\ADF(f)=-1+\ssac(f)/\ell^2$, we first find a formula for the variance of $ \ssac $, the sum of the squares of the autocorrelations.
Our calculation is based on the formula
\[ \mom{3} \ssac(f) = \sum_{ \fP \in \Isom(3)} |\fP| \Sols(\fP,\ell) , \]
in \cref{Sanria}.
We begin by identifying the isomorphism classes of contributory partitions of $[3]\times\lindexset$ in Sections \ref{Abby}--\ref{Dave}.  
On the way, we find the size of each such isomorphism class $\fP$ (in \cref{Bernard}), and then find $\Sols(\fP,\ell)$ (in \cref{Dave}).
In \cref{Egbert}, we put all this information together and arrive at our formula for the skewness of the demerit factor (\cref{Shirley}).

\subsection{Notations for partitions of interest}\label{Abby}

We now begin our study of the partitions in $\Con(3)$, whose complete classification into eight isomorphism classes we shall achieve later in \cref{Julianna}.
It helps first to notate all these partitions that we shall eventually show to be contributory.
For this purpose, it useful to define $\plusmodthree \colon [3] \times [3] \to [3]$ where $e\plusmodthree f$ is the unique element $g \in [3]$ such that $e+f \equiv g \pmod{3}$.
Then, for each $e \in [3]$ and $s$, $t$, $u$, $v$, $w$, $x$, $y$, $z$, $a \in [2]$, we define
\begin{align*}
\cC_{1,e,s,t,u,v,w,x,y}   = \Big\{ & \{(e\plusmodthree 1,s,0),(e\plusmodthree 1,s,1),(e\plusmodthree 2,1-t,v),(e,u,x)\}, \\
                             & \{(e\plusmodthree 2,t,0),(e\plusmodthree 2,t,1),(e\plusmodthree 1,1-s,w),(e,u,1-x)\}, \\
                             & \{(e\plusmodthree 1,1-s,1-w),(e,1-u,y)\}, \\
                             & \{(e\plusmodthree 2,1-t,1-v),(e,1-u,1-y)\} \Big\},
\end{align*}
\begin{align*}
\cC_{2,e,s,t,u,v,w}      = \Big\{ & \{(e\plusmodthree 1,s,0),(e\plusmodthree 1,s,1),(e\plusmodthree 2,t,0),(e\plusmodthree 2,t,1)\}, \\
                            & \{(e,u,v),     (e\plusmodthree 1,1-s,0)\}, \\
                            & \{(e,u,1-v),   (e\plusmodthree 1,1-s,1)\}, \\
                            & \{(e,1-u,w),   (e\plusmodthree 2,1-t,0)\}, \\
                            & \{(e,1-u,1-w), (e\plusmodthree 2,1-t,1)\} \Big\},
\end{align*}
\begin{align*}
\cC_{3,e,s,t,u,v,w,x}    = \Big\{ & \{(e,s,0), (e,s,1), (e\plusmodthree 1,t,v), (e\plusmodthree 2,u,w)\}, \\
                            & \{(e,1-s,x),   (e\plusmodthree 1,t,1-v)\}, \\
                            & \{(e,1-s,1-x), (e\plusmodthree 2,u,1-w)\}, \\
                            & \{(e\plusmodthree 1,1-t,0), (e\plusmodthree 1,1-t,1)\}, \\
                            & \{(e\plusmodthree 2,1-u,0), (e\plusmodthree 2,1-u,1)\} \Big\},
\end{align*}
\begin{align*}
\cC_{4,e,s,t,u,v,w,x,y,z} = \Big\{ & \{(e,s,0), (e,s,1), (e\plusmodthree 1,t,v), (e\plusmodthree 2,u,w)\}, \\
                            & \{(e,1-s,x),     (e\plusmodthree 1,1-t,y)\}, \\
                            & \{(e,1-s,1-x),   (e\plusmodthree 2,1-u,z)\}, \\
                            & \{(e\plusmodthree 1,t,1-v),   (e\plusmodthree 2,1-u,1-z)\}, \\
                            & \{(e\plusmodthree 1,1-t,1-y), (e\plusmodthree 2,u,1-w)\} \Big\},
\end{align*}
\begin{align*}
\cC_{5,s,t,u,v,w,x}       = \Big\{ & \{(0,s,0),   (0,s,1)\},\\
                             & \{(1,t,0),   (1,t,1)\}, \\
                             & \{(2,u,0),   (2,u,1)\}, \\
                             & \{(0,1-s,v), (1,1-t,1-w)\} \\
                             & \{(1,1-t,w), (2,1-u,1-x)\} \\
                             & \{(2,1-u,x), (0,1-s,1-v)\} \Big\},
\end{align*}
\begin{align*}
\cC_{6,e,s,t,v,w,x,y}     = \Big\{ & \{(e\plusmodthree 1,s,0),  (e\plusmodthree 1,s,1)\},\\
                             & \{(e\plusmodthree 2,t,0),  (e\plusmodthree 2,t,1)\}, \\
                             & \{(e,0,x),    (e\plusmodthree 1,1-s,v)\}, \\
                             & \{(e,0,1-x),  (e\plusmodthree 2,1-t,w)\} \\
                             & \{(e,1,y)  ,  (e\plusmodthree 1,1-s,1-v)\} \\
                             & \{(e,1,1-y),  (e\plusmodthree 2,1-t,1-w)\} \Big\},
\end{align*}
\begin{align*}
\cC_{7,s,t,u,v,w,x}      = \Big\{ & \{(0,s,0),    (1,1-t,v)\},\\
                             & \{(0,s,1),    (1,1-t,1-v)\}, \\
                             & \{(1,t,0),    (2,1-u,w)\}, \\
                             & \{(1,t,1),    (2,1-u,1-w)\} \\
                             & \{(2,u,0)  ,  (0,1-s,x)\} \\
                             & \{(2,u,1),    (0,1-s,1-x)\} \Big\}\text{, and}
\end{align*}
\begin{align*}
\cC_{8,s,t,v,w,x,y,z,a}   = \Big\{ & \{(0,0,v),    (1,1-s,1-z)\},\\
                             & \{(0,1,y),    (1,s,1-w)\}, \\
                             & \{(1,s,w),    (2,1-t,1-a)\}, \\
                             & \{(1,1-s,z),  (2,t,1-x)\} \\
                             & \{(2,t,x),    (0,1,1-y)\} \\
                             & \{(2,1-t,a),  (0,0,1-v)\} \Big\}.
\end{align*}
We organize these partitions into classes as follows:
\begin{align*}
\fC_1 & = \{ \cC_{1,e,s,t,u,v,w,x,y}   : e \in [3]; s,t,u,v,w,x,y \in [2] \}, \\
\fC_2 & = \{ \cC_{2,e,s,t,u,v,w}      : e \in [3]; s,t,u,v,w \in [2] \}, \\
\fC_3 & = \{ \cC_{3,e,s,t,u,v,w,x}     : e \in [3]; s,t,u,v,w,x \in [2] \}, \\
\fC_4 & = \{ \cC_{4,e,s,t,u,v,w,x,y,z}  : e \in [3]; s,t,u,v,w,x,y,z \in [2] \}, \\
\fC_5 & = \{ \cC_{5,s,t,u,v,w}        : s,t,u,v,w \in [2] \}, \\
\fC_6 & = \{ \cC_{6,e,s,t,v,w,x,y}     : e \in [3]; s,t,v,w,x,y \in [2] \}, \\
\fC_7 & = \{ \cC_{7,s,t,u,v,w,x}      : s,t,u,v,w,x \in [2] \}\text{, and} \\
\fC_8 & = \{ \cC_{8,s,t,v,w,x,y,z,a}   : s,t,v,w,x,y,z,a \in [2] \}.
\end{align*}

\subsection{Basic properties of our classes}\label{Bernard}
Now we determine the sizes of our putative isomorphism classes.
\begin{lemma}\label{Genevieve}
We have $\card{\fC_1}=3 \cdot 2^7$, $\card{\fC_2}=3 \cdot 2^5$, $\card{\fC_3}=3 \cdot 2^6$, $\card{\fC_4}=3 \cdot 2^8$, $\card{\fC_5}=2^6$, $\card{\fC_6}=3\cdot 2^6$, $\card{\fC_7}=2^6$, and $\card{\fC_8}=2^8$.
\end{lemma}
\begin{proof}
For each of the eight classes, it is not difficult to show that different tuples of parameters produce different partitions.

For example, in $\cC_{1,e,s,t,u,v,w,x,y}$, note that $e$ is the only element of $[3]$ such that all four classes contain some element of the form $(e,*,*)$.  Then $s$ (resp., $t$) is the unique element such that both $(e\plusmodthree 1,s,0)$ and $(e\plusmodthree 1,s,1)$ (resp., both $(e\plusmodthree 2,t,0)$ and $(e\plusmodthree 2,t,1)$) appear in the same class, while $u$ is the unique element such that $(e,u,0)$ and $(e,u,1)$ appear in classes of cardinality four.  Then $v$ (resp., $w$) is the unique element such that $(e\plusmodthree 2,1-t,v)$ (resp., $(e\plusmodthree 1,1-s,w)$) is in a class of cardinality four, while $x$ is the unique element such that $(e,u,x)$ lies in the class containing $(e\plusmodthree 1,s,0)$ and $(e\plusmodthree 1,s,1)$, and $y$ is the unique element such that $(e,1-u,y)$ resides in the same class as $(e\plusmodthree 1,1-s,1-w)$.

In $\cC_{2,e,s,t,u,v,w}$, note that $e$ is the only element of $[3]$ such that the class of cardinality four has no element of the form $(e,*,*)$.  Then $s$ (resp., $t$) is the unique element such that both $(e\plusmodthree 1,s,0)$ and $(e\plusmodthree 1,s,1)$ (resp., both $(e\plusmodthree 2,t,0)$ and $(e\plusmodthree 2,t,1)$) appear in the class of cardinality four.  Then $u$ and $v$ are the unique elements such that $(e,u,v)$ lies in the same class as $(e\plusmodthree 1,1-s,0)$, while $w$ is the unique element such that $(e,1-u,w)$ lies in the class containing $(e\plusmodthree 2,1-t,0)$.

In $\cC_{3,e,s,t,u,v,w,x}$, note that $e$ is the only element of $[3]$ and $s$ is the only element of $[2]$ such that the class of cardinality four has two elements of the form $(e,s,*)$.
Then $t$ and $v$ (resp., $u$ and $w$) are the unique elements such that the class of cardinality four contains an element of the form $(e\plusmodthree 1,t,v)$ (resp., $(e\plusmodthree 2,u,w)$).  Then $x$ is the unique element such that $(e,1-s,x)$ lies in the class that contains $(e\plusmodthree 1,t,1-v)$.

In $\cC_{4,e,s,t,u,v,w,x,y,z}$, note that $e$ is the only element of $[3]$ and $s$ is the only element of $[2]$ such that the class of cardinality four has two elements of the form $(e,s,*)$.
Then $t$ and $v$ (resp., $u$ and $w$) are the unique elements such that the class of cardinality four contains an element of the form $(e\plusmodthree 1,t,v)$ (resp., $(e\plusmodthree 2,u,w)$).
Then $y$ (resp., $z$) is the unique element such that $(e\plusmodthree 1,1-t,1-y)$ (resp., $(e\plusmodthree 2,1-u,1-z)$ lies in the class that contains $(e\plusmodthree 2,u,1-w)$ (resp., $(e\plusmodthree 1,t,1-v)$).
Then $x$ is the unique element such that $(e,1-s,x)$ lies in the class that contains $(e\plusmodthree 1,1-t,y)$.

In $\cC_{5,s,t,y,v,w,x}$, there is one and only one class that contains one element of the form $(0,*,*)$ and one element of the form $(1,*,*)$, and $s$, $v$, $t$, and $w$ are the unique elements such that this class contains $(0,1-s,v)$ and $(1,1-t,1-w)$.  Then $u$ and $x$ are the unique elements such that $(2,1-u,1-x)$ lies in the class containing $(1,1-t,w)$.

In $\cC_{6,e,s,t,v,w,x,y}$, note that $e$ is the only element of $[3]$ such that no class contains two elements of the form $(e,*,*)$.  Then $s$ (resp., $t$) is the unique element such that there is a class containing two elements of the form $(e\plusmodthree 1,s,*)$ (resp., $(e\plusmodthree 2,t,*)$).
There is one and only one class containing both an element of the form $(e,0,*)$ and an element of the form $(e\plusmodthree 1,*,*)$; then $x$ and $v$ are the unique elements such that this class contains $(e,0,x)$ and $(e\plusmodthree 1,1-s,v)$.
There is one and only one class containing both an element of the form $(e,1,*)$ and an element of the form $(e\plusmodthree 2,*,*)$; then $y$ and $w$ are the unique elements such that this class contains $(e,1,1-y)$ and $(e\plusmodthree 2,1-t,1-w)$.

In $\cC_{7,s,t,u,v,w,x}$, note that there is one and only one class that contains both an element of the form $(0,*,0)$ and an element of the form $(1,*,*)$; then $s$, $t$, and $v$ are the unique elements such that this class contains $(0,s,0)$ and $(1,1-t,v)$.
Then $u$ and $w$ are the unique elements such $(2,1-u,w)$ lies in the class containing $(1,t,0)$.
Then $x$ is the unique element such that $(0,1-s,x)$ lies in the class containing $(2,u,0)$.

In $\cC_{8,s,t,v,w,x,y,z,a}$, note that there is one and only one class that contains both an element of the form $(0,0,*)$ and an element of the form $(1,*,*)$; then $v$, $s$, and $z$ are the unique elements such that this class contains $(0,0,v)$ and $(1,1-s,1-z)$.
Then $t$ and $a$ are the unique elements such that $(2,1-t,a)$ lies in the class containing $(0,0,1-v)$.
Then $w$ is the unique element such that $(1,s,w)$ lies in the class containing $(2,1-t,1-a)$.
Then $y$ is the unique element such that $(0,1,y)$ lies in the class containing $(1,s,1-w)$.
Then $x$ is the unique element such that $(2,t,x)$ lies in the class containing $(0,1,1-y)$.
\end{proof}
In the next two lemmata, we show that for each of our putative isomorphism classes, the partitions within the class are indeed isomorphic to each other and not isomorphic to partitions in the other class.
\begin{lemma}\label{Elizabeth}
For each $m \in \{1,2,\ldots,8\}$, all the partitions in $\fC_m$ are isomorphic to each other.
\end{lemma}
\begin{proof}
For each of the eight classes, it is not difficult to write down an element $\pi$ in the wreath product $\cWthree$ that sends an element of the class with arbitrary parameter values to the element of the class where all the parameters have been set equal to $0$.

For example, for $\cC_{1,e,s,t,u,v,w,x,y} $ suppose that $e \in [3]$ and $s,t,u,v,w,x,y \in [2]$.
Let $\epsilon$ be the permutation of $[3]$ with $\epsilon=(0 2 1)^e$, let $\sigma_0=(0 1)^u$, $\sigma_1=(0 1)^s$, $\sigma_2=(0 1)^t$, and let $\digamma_{0,0}=(0 1)^x$, $\digamma_{0,1}=(0 1)^y$, $\digamma_{1,0}=\id$, $\digamma_{1,1}=(0 1)^w$, $\digamma_{2,0}=\id$, and $\digamma_{2,1}=(0 1)^v$.  Then set $\delta_e=((\digamma_{e,0},\digamma_{e,1}),\sigma_e)$ for each $e \in [3]$.  Then it is not hard to show that the element $\pi=((\delta_0,\delta_1,\delta_2),\epsilon) \in \cWthree$ has $\pi(\cC_{1,e,s,t,u,v,w,x,y})=\cC_{1,0,0,0,0,0,0,0,0}$.

For  $\cC_{2,e,s,t,u,v,w}$ suppose that $e \in [3]$ and $s,t,u,v,w \in [2]$.
Let $\epsilon$ be the permutation of $[3]$ with $\epsilon=(021)^e$, let $\sigma_0=(0 1)^u$, $\sigma_1=(0 1)^s$, $\sigma_2=(0 1)^t$, and let $\digamma_{0,0}=(0 1)^v$, $\digamma_{0,1}=(0 1)^w$, $\digamma_{1,0}=\id$, $\digamma_{1,1}=\id$, $\digamma_{2,0}=\id$, and $\digamma_{2,1}=\id$. Then set $\delta_e=((\digamma_{e,0},\digamma_{e,1}),\sigma_e)$ for each $e \in [3]$.  Then it is not hard to show that the element $\pi=((\delta_0,\delta_1,\delta_2),\epsilon) \in \cWthree$ has $\pi(\cC_{2,e,s,t,u,v,w})=\cC_{2,0,0,0,0,0,0}$.

For  $\cC_{3,e,s,t,u,v,w,x}$ suppose that $e \in [3]$ and $s,t,u,v,w,x\in [2]$.
Let $\epsilon$ be the permutation of $[3]$ with $\epsilon=(021)^e$, let $\sigma_0=(0 1)^s$, $\sigma_1=(0 1)^t$, $\sigma_2=(0 1)^u$, and let $\digamma_{0,0}=\id$, $\digamma_{0,1}=(0 1)^x$, $\digamma_{1,0}=(01)^v$, $\digamma_{1,1}=\id$, $\digamma_{2,0}=(01)^w$, and $\digamma_{2,1}=\id$. Then set $\delta_e=((\digamma_{e,0},\digamma_{e,1}),\sigma_e)$ for each $e \in [3]$.  Then it is not hard to show that the element $\pi=((\delta_0,\delta_1,\delta_2),\epsilon) \in \cWthree$ has $\pi(\cC_{3,e,s,t,u,v,w,x})=\cC_{3,0,0,0,0,0,0,0}$.

For  $\cC_{4,e,s,t,u,v,w,x,y,z}$ suppose that $e \in [3]$ and $s,t,u,v,w,x,y,z\in [2]$.
Let $\epsilon$ be the permutation of $[3]$ with $\epsilon=(021)^e$, let $\sigma_0=(0 1)^s$, $\sigma_1=(0 1)^t$, $\sigma_2=(0 1)^u$, and let $\digamma_{0,0}=\id$, $\digamma_{0,1}=(0 1)^x$, $\digamma_{1,0}=(01)^v$, $\digamma_{1,1}=(01)^y$, $\digamma_{2,0}=(01)^w$, and $\digamma_{2,1}=(01)^z$. Then set $\delta_e=((\digamma_{e,0},\digamma_{e,1}),\sigma_e)$ for each $e \in [3]$.  Then it is not hard to show that the element $\pi=((\delta_0,\delta_1,\delta_2),\epsilon) \in \cWthree$ has $\pi(\cC_{4,e,s,t,u,v,w,x,y,z})=\cC_{4,0,0,0,0,0,0,0,0,0}$.

For  $\cC_{5,s,t,y,v,w,x}$ suppose that $s,t,y,v,w,x\in [2]$.
Let $\epsilon$ be the identity permutation of $[3]$, let $\sigma_0=(0 1)^s$, $\sigma_1=(0 1)^t$, $\sigma_2=(0 1)^u$, and let $\digamma_{0,0}=\id$, $\digamma_{0,1}=(0 1)^v$, $\digamma_{1,0}=\id$, $\digamma_{1,1}=(01)^w$, $\digamma_{2,0}=\id$, and $\digamma_{2,1}=(01)^x$. Then set $\delta_e=((\digamma_{e,0},\digamma_{e,1}),\sigma_e)$ for each $e \in [3]$.  Then it is not hard to show that the element $\pi=((\delta_0,\delta_1,\delta_2),\epsilon) \in \cWthree$ has $\pi(\cC_{5,s,t,y,v,w,x})=\cC_{5,0,0,0,0,0,0}$.

For  $\cC_{6,e,s,t,v,w,x,y}$ suppose that $e \in [3]$ and $s,t,v,w,x,y\in [2]$.
Let $\epsilon$ be the permutation of $[3]$ with $\epsilon=(021)^e$, let $\sigma_0=\id$, $\sigma_1=(0 1)^s$, $\sigma_2=(0 1)^t$, and let $\digamma_{0,0}=(01)^x$, $\digamma_{0,1}=(0 1)^y$, $\digamma_{1,0}=\id$, $\digamma_{1,1}=(01)^v$, $\digamma_{2,0}=\id$, and $\digamma_{2,1}=(01)^w$. Then set $\delta_e=((\digamma_{e,0},\digamma_{e,1}),\sigma_e)$ for each $e \in [3]$.  Then it is not hard to show that the element $\pi=((\delta_0,\delta_1,\delta_2),\epsilon) \in \cWthree$ has $\pi(\cC_{6,e,s,t,v,w,x,y})=\cC_{6,0,0,0,0,0,0,0}$.

For  $\cC_{7,s,t,u,v,w,x}$ suppose that $s,t,v,w,x,y\in [2]$.
Let $\epsilon$ be the identity permutation of $[3]$, let $\sigma_0=(01)^s$, $\sigma_1=(0 1)^t$, $\sigma_2=(0 1)^u$, and let $\digamma_{0,0}=id$, $\digamma_{0,1}=(0 1)^x$, $\digamma_{1,0}=\id$, $\digamma_{1,1}=(01)^v$, $\digamma_{2,0}=\id$, and $\digamma_{2,1}=(01)^w$. Then set $\delta_e=((\digamma_{e,0},\digamma_{e,1}),\sigma_e)$ for each $e \in [3]$.  Then it is not hard to show that the element $\pi=((\delta_0,\delta_1,\delta_2),\epsilon) \in \cWthree$ has $\pi(\cC_{7,s,t,u,v,w,x})=\cC_{7,0,0,0,0,0,0}$.

For  $\cC_{8,s,t,v,w,x,y,z,a}$ suppose that $s,t,v,w,x,y,z,a\in [2]$.
Let $\epsilon$ be the identity permutation of $[3]$, let $\sigma_0= \id$, $\sigma_1=(0 1)^s$, $\sigma_2=(0 1)^t$, and let $\digamma_{0,0}=(01)^v$, $\digamma_{0,1}=(0 1)^y$, $\digamma_{1,0}= (01)^w$, $\digamma_{1,1}=(01)^z$, $\digamma_{2,0}= (01)^x$, and $\digamma_{2,1}=(01)^a$. Then set $\delta_e=((\digamma_{e,0},\digamma_{e,1}),\sigma_e)$ for each $e \in [3]$.  Then it is not hard to show that the element $\pi=((\delta_0,\delta_1,\delta_2),\epsilon) \in \cWthree$ has $\pi(\cC_{8,s,t,v,w,x,y,z,a})=\cC_{8,0,0,0,0,0,0,0,0}$.
\end{proof}

\begin{lemma}\label{Jacob}
If $m$ and $n$ are distinct elements of $\{1,2,\ldots,8\}$, then no partition in $\fC_m$ is isomorphic to any partition in $\fC_n$.
\end{lemma}
\begin{proof}
If $i, j$ are two distinct elements of $\{1,2,\ldots,8\}$ with $\{i,j\}\not=\{7,8\}$, then one notes that the partitions in $\fC_i$ have a different $3$-cross-section than the partitions in $\fC_j$, so by \cref{Brandee} no partition in $\fC_i$ can be isomorphic to one in $\fC_j$.
For each partition $\cP$ in $\fC_7$ and each $e \in [3]$ and $s \in [2]$, there are two distinct classes, say $P$ and $Q$, that respectively contain $(e,s,0)$ and $(e,s,1)$.  Then the unique element of $P\smallsetminus\{(e,s,0)\}$ will have the same first coordinate as the unique element of $Q\smallsetminus\{(e,s,1)\}$.  If one applies a wreath product element to a partition with this property, the output partition still has this property, but no partition in $\fC_8$ has this property.
\end{proof}

\subsection{$\Con(3)$ is covered by our classes}

In this section, we show that our putative equivalence classes contain all contributory partitions.
By classifying partitions according to various properties, Lemmata \ref{Eugene}--\ref{Beverly} show that all partitions in $\Con(3)$ are forced into one of our eight putative equivalence classes.  This is summed up in \cref{Alexander}.
\begin{lemma}\label{Eugene}
Let $\cP \in \Con(3)$ and suppose that there is some $P \in \cP$ whose $3$-cross-section contains at least two instances of $2$.
Then $\cP \in \fC_2$.
\end{lemma}
\begin{proof}
Take $e \in [3]$ such that $|P_{\{e\plusmodthree 1\}}|=|P_{\{e\plusmodthree 2\}}|=2$.
Then \cref{Gideon}\ref{Epazote} shows that there are some $s,t \in [2]$ such that $P_{\{e\plusmodthree 1\}}=\{(e\plusmodthree 1,s,0),(e\plusmodthree 1,s,1)\}$ and $P_{\{e\plusmodthree 2\}}=\{(e\plusmodthree 2,t,0),(e\plusmodthree 2,t,1)\}$.
Furthermore, \cref{Gideon}\ref{Epazote} shows that there are two classes, $Q$ and $R$, distinct from $P$ and from each other, such that $|Q_{\{e\plusmodthree 1\}}|=|R_{\{e\plusmodthree 1\}}|=1$  and that there are two classes $Q'$ and $R'$, distinct from $P$ and from each other, such that $|Q'_{\{e\plusmodthree 2\}}|=|R'_{\{e\plusmodthree 2\}}|=1$.

We suppose that $\{Q,R\}$ and $\{Q',R'\}$ are non-disjoint to show contradiction; without loss of generality, we may take $Q=Q'$.
Since $\cP \in \Con(3)$, there is some $\tau \in \As(\cP,=)$.
Let $A$ (resp., $B$, $C$, $C'$) denote the element of $\N$ such that $\tau_\gamma=A$ (resp., $B$, $C$, $C'$) for $\gamma\in P$ (resp., $Q$, $R$, $R'$).
Then looking at $\gamma$'s of the form $(e\plusmodthree 1,*,*)$ (resp., $(e\plusmodthree 2,*,*)$), we must have $2 A=B+C$ (resp., $2 A=B+C'$).
Therefore $C=C'$, which forces $R=R'$.
By \cref{Gideon}\ref{Epazote}, there is some class $S \in \cP$ such that $|S_{\{e\}}|=1$.
But since $|S_{\{e\plusmodthree 1,e\plusmodthree 2\}}|$ is even (regardless of whether $S \in \{P,Q,R\}$ or not), this makes $|S|=|S_{\{e\}} \bigsqcup S_{\{e\plusmodthree 1,e\plusmodthree 2\}}|$ odd, which is absurd because contributory partitions have classes of even size.

So $\{Q,R\}$ and $\{Q',R'\}$ must be disjoint.
Without loss of generality, we may assume $Q_{\{e\plusmodthree 1\}}=\{(e\plusmodthree 1,1-s,0)\}$, $R_{\{e\plusmodthree 1\}}=\{(e\plusmodthree 1,1-s,1)\}$, $Q'_{\{e\plusmodthree 2\}}=\{(e\plusmodthree 2,1-t,0)\}$, $R'_{\{e\plusmodthree 2\}}=\{(e\plusmodthree 2,1-t,1)\}$.
Since $\cP$ is an even partition of the $12$ elements in $[3]\times\lindexset$, and $\cP$ has the five distinct classes $P$, $Q$, $R$, $Q'$, and $R'$, whose respective sizes are at least $4$, $2$, $2$, $2$, and $2$, we know that these are the only classes in $\cP$ and their sizes are equal to the lower bounds just stated.
We still need to account for which class each element of the form $(e,*,*)$ lies in: we must place one such element in each of $Q$, $R$, $Q'$, and $R'$.
Since $\cP \in \Con(3)$, there is some $\tau \in \As(\cP,=)$.
Let $A$, $B$, $C$, $B'$, $C'$ denote the distinct elements of $\N$ such that $\tau_\gamma=A$ (resp., $B$, $C$, $B'$, $C'$) for $\gamma\in P$ (resp., $Q$, $R$, $Q'$, $R'$).
Then looking at $\gamma$'s of the form $(e\plusmodthree 1,*,*)$ (resp., $(e\plusmodthree 2,*,*)$), we must have $2 A=B+C$ (resp., $2 A=B'+C'$), so that $B+C=B'+C'$
Let $u,v \in [2]$ be such that $Q=\{(e\plusmodthree 1,1-s,0),(e,u,v)\}$.

Suppose that $(e,u,1-v) \in Q'$ (resp., $R'$) to show a contradiction.
Then looking at $\gamma$'s of the form $(e,*,*)$, we must have $B+B'=C+C'$ (resp., $B+C'=B'+C$), and if we combine these equations with the already known $B+C=B'+C'$, we obtain $B=C'$ (resp., $B=B'$), which contradicts the fact that these values must be distinct.

Thus we must have $R=\{(e\plusmodthree 1,1-s,1),(e,u,1-v)\}$.
Let $w \in [2]$ be such that $Q'=\{(e\plusmodthree 2,1-t,0),(e,1-u,w)\}$, and then $R'=\{(e\plusmodthree 2,1-t,1),(e,1-u,1-w)\}$.
Then note that our partition $\cP=\{P,Q,R,Q',R'\}=\cC_{2,e,s,t,u,v,w}$, so $\cP \in \fC_2$.
\end{proof}
\begin{lemma}\label{Francine}
Let $\cP \in \Con(3)$ and suppose that there is no $P\in \cP$ whose $3$-cross-section contains at least two instances of $2$.
Then for each $P \in \cP$, the $3$-cross-section of $P$ must be $\ms{0,1,1}$, $\ms{0,0,2}$, or $\ms{1,1,2}$.
\end{lemma}
\begin{proof}
This is an easy consequence of \cref{Gideon}\ref{Anise}.
\end{proof}
\begin{lemma}\label{Eusebius}
If $\cP \in \Con(3)$ with $|\cP|=4$, then $\cP \in \fC_1$.
\end{lemma}
\begin{proof}
Suppose that we have a partition $\cP \in \Con(3)$ with $|\cP|=4$.
We know that no class in $\cP$ can have a $3$-cross-section with multiple instances of $2$, because \cref{Eugene} would force $\cP \in \fC_2$, which would make $|\cP|=5$, contrary to our assumption.
Therefore by \cref{Francine}, we know that classes must be of size $2$ or $4$, this means that $\cP$ must have type $\ms{4,4,2,2}$, with the classes of size $4$ having $3$-cross-section $\ms{1,1,2}$.

By \cref{Gideon}\ref{Epazote}, we can choose some $e \in [3]$ and label our two classes of size $4$ as $P$ and $Q$ so that $|P_{\{e\plusmodthree 1\}}|=2$ and $|Q_{\{e\plusmodthree 2\}}|=2$ and there are $s,t \in [2]$ be such that $(e\plusmodthree 1,s,0),(e\plusmodthree 1,s,1) \in P$ and $(e\plusmodthree 2,t,0),(e\plusmodthree 2,t,1) \in Q$.
Since $|P_{\{f\}}|\not=2$ for $f\not=e\plusmodthree 1$ by the previous paragraph, the remaining two elements of $P$ must be of the form $(e\plusmodthree 2,1-t,v)$ and $(e,u,x)$ for some $v,u,x \in [2]$.
Since $|Q_{\{f\}}|\not=2$ for $f\not=e\plusmodthree 2$ by the previous paragraph, the remaining two elements of $Q$ must be of the form $(e\plusmodthree 1,1-s,w)$ for some $w \in [2]$ and $(e,*,*)$.

Assume that there is some class $T$ in $\cP$ such that $|T_{\{e\}}|=2$ to show a contradiction; this $T$ is distinct from $P$ and $Q$.
Then by \cref{Gideon}\ref{Epazote}, we must have $T=\{(e,1-u,0),(e,1-u,1)\}$ since $(e,u,x)$ is already in class $P$, and so $(e,u,1-x) \in Q$.
Let $U$ be the class of size $2$ that is not $T$; by process of elimination $U=\{(e\plusmodthree 1,1-s,1-w),(e\plusmodthree 2,1-t,1-v)\}$.
Since $\cP \in \Con(3)$, there is some $\tau \in \As(\cP,=)$.
Let $A$ (resp., $B$, $E$, $F$) denote the element of $\N$ such that $\tau_\gamma=A$ (resp., $B$, $E$, $F$) for $\gamma\in P$ (resp., $Q$, $T$, $U$).
The values $A$, $B$, $E$, $F$ must be distinct and satisfy the equations $E+E=A+B$, $A+A=B+F$, $B+B=A+F$; if we add these three equations, we find that $E=F$, a contradiction.

Thus $|X_{\{e\}}|\leq 1$ for every $X \in \cP$, and since we have four classes, this bound is always met.
Let $R$ (resp., $S$) be the class of size $2$ in $\cP$ with $(e\plusmodthree 1,1-s,1-w) \in R$ (resp., $(e\plusmodthree 2,1-t,1-v) \in S$); note that $R\not=S$ since each class of size $2$ has some element of the form $(e,*,*)$.
Since $\cP \in \Con(3)$, there is some $\tau \in \As(\cP,=)$.
Let $A$, $B$, $C$, $D$ denote the distinct elements of $\N$ such that $\tau_\gamma=A$ (resp., $B$, $C$, $D$) for $\gamma\in P$ (resp., $Q$, $R$, $S$).
Considering values of $\gamma$ of the form $(e\plusmodthree 1,*,*)$ and $(e\plusmodthree 2,*,*)$, we must have $A+A=B+C$ and $B+B=A+D$, and adding these equations yields $A+B=C+D$.
If $(e,u,1-x)$ were in $R$ (resp., $S$), then considering the values of $\gamma$ of the form $(e,*,*)$, we would have $A+C=B+D$ (resp., $A+D=B+C$), which would combine with the earlier $A+B=C+D$ to yield $B=C$ (resp., $B=D$), a contradiction.
Thus we must have $(e,u,1-x) \in Q$.
Then let $y \in [2]$ be such that $(e,1-u,y) \in R$; this means that $(e,1-u,1-y)\in S$, and we see that $\cP=\{P,Q,R,S\}=\cC_{1,e,s,t,u,v,w,x,y}$, and so $\cP \in \fC_1$.
\end{proof}
\begin{lemma}\label{Salvador}
Let $\cP \in \Con(3)$ with $|\cP|=5$.
Then the number of classes of $\cP$ that have $3$-cross-section $\ms{0,0,2}$ is either $0$ or $2$.
\end{lemma}
\begin{proof}
If any class of $\cP$ has a $3$-cross-section that contains more than one instance of $2$, then \cref{Eugene} forces $\cP \in \fC_2$, and the partitions in $\fC_2$ have no classes with $3$-cross-section $\ms{0,0,2}$.
So henceforth, we assume that no class of $\cP$ has a $3$-cross-section that contains more than one instance of $2$.

Since $\cP$ is a partition of the $12$ elements of $\indexset$ into $5$ classes of even size, its type must be $\ms{4,2,2,2,2}$.
Let $P$ be the class of size $4$ in $\cP$; it must have $3$-cross-section $\ms{1,1,2}$ by \cref{Francine}.
So by \cref{Gideon}\ref{Epazote} there are some $e \in [3]$ and $s,t,u,v,w \in [2]$ such that $(e,s,0),(e,s,1) \in P$.
By \cref{Gideon}\ref{Epazote}, $\cP_{\{e\}}$ is of type $\ms{2,1,1}$: so there are two distinct classes $Q$ and $R$ with $|Q_{\{e\}}|=|R_{\{e\}}|=1$; these classes have $3$-cross-sections $\ms{0,1,1}$ and one point of $\{(e,1-s,0),(e,1-s,1)\}$ lies in each of $Q$ and $R$.
So at most two classes of $\cP$ have $3$-cross-section $\ms{0,0,2}$.

Suppose that there is precisely one class of $\cP$, say $T$, with $3$-cross-section $\ms{0,0,2}$ to show contradiction.
Let $S$ be the last class of size $2$ that is not $Q$, $R$, or $T$; this $S$ must have $3$-cross-section $\ms{0,1,1}$.
Since $\{e\}\times\lindexset$ is covered by $P$, $Q$, and $R$, we have $|T_{\{e\}}|=0$.
Then there are some $f \in [3]$ with $f\not=e$ and $r,y \in [2]$ such that $T=\{(f,r,0),(f,r,1)\}$ and $(f,1-r,y) \in P$ and $(f,1-r,1-y) \in S$.
Let $g$ be the unique element of $[3] \smallsetminus\{e,f\}$.
Then there are some $q,z \in [2]$ such that $(g,q,z) \in P$.
Then $(g,q,1-z)$ lies either in $Q$, $R$, or $S$: in the next two paragraphs we show that any of these options leads to a contradiction.

If $(g,q,1-z) \in Q \cup R$, then let $U$ denote whichever of $Q$ or $R$ contains $(g,q,1-z)$ and let $V$ be whichever does not contain $(g,q,1-z)$.
Since $\cP \in \Con(3)$, there is some $\tau \in \As(\cP,=)$.
Let $A$ (resp., $B$, $C$, $D$, $E$) denote the element of $\N$ such that $\tau_\gamma=A$ (resp., $B$, $C$, $D$, $E$) for $\gamma\in P$ (resp., $U$, $V$, $S$, $T$).
If we consider values of $\gamma$ of the form $(e,*,*)$ (resp., $(f,*,*)$, $(g,*,*)$), we must have $2 A=B+C$ (resp., $2 E=A+D$, $C+D=A+B$).
Adding these three equations shows that $E=B$, but that contradicts the fact that $\tau \in \As(\cP,=)$.

If $(g,q,1-z) \in S$, then one element of $\{(g,q+1,0),(g,q+1,1)\}$ must lie in each of $Q$ and $R$.
Since $\cP \in \Con(3)$, there is some $\tau \in \As(\cP,=)$.
Let $A$ (resp., $B$, $C$, $D$, $E$) denote the element of $\N$ such that $\tau_\gamma=A$ (resp., $B$, $C$, $D$, $E$) for $\gamma\in P$ (resp., $Q$, $R$, $S$, $T$).
If we consider values of $\gamma$ of the form $(e,*,*)$ (resp., $(f,*,*)$, $(g,*,*)$), we must have $2 A=B+C$ (resp., $2 E=A+D$, $A+D=B+C$).
These equations imply that $A=E$, but that contradicts the fact that $\tau \in \As(\cP,=)$.
\end{proof}
\begin{lemma}\label{Zoe}
Let $\cP \in \Con(3)$ with $|\cP|=5$ and at least one class of $\cP$ has $3$-cross-section $\ms{0,0,2}$.  Then $\cP \in \fC_3$.
\end{lemma}
\begin{proof}
We can see that no class of $\cP$ has a $3$-cross-section that contains more than one instance of $2$, for if it did then \cref{Eugene} would force $\cP \in \fC_2$, but no partition in $\fC_2$ has a class with $3$-cross-section $\ms{0,0,2}$.
Since $\cP$ is a partition of the $12$ elements of $\indexset$ into $5$ classes of even size, its type must be $\ms{4,2,2,2,2}$.
Let $P$ be the class of size $4$ in $\cP$; it must have $3$-cross-section $\ms{1,1,2}$ by \cref{Francine}.
By \cref{Salvador}, $\cP$ must have precisely two classes with $3$-cross-section $\ms{0,0,2}$.
Therefore the remaining two classes of $\cP$ must have $3$-cross-section $\ms{0,1,1}$.

So by \cref{Gideon}\ref{Epazote} there are some $e \in [3]$ and $s,t,u,v,w \in [2]$ such that $P=\{(e,s,0),(e,s,1),(e\plusmodthree 1,t,v),(e\plusmodthree 2,u,w)\}$.
By \cref{Gideon}, $\cP_{\{e\}}$ is of type $\ms{2,1,1}$: so there are two distinct classes $Q$ and $R$ with $|Q_{\{e\}}|=|R_{\{e\}}|=1$; these classes have $3$-cross-sections $\ms{0,1,1}$ and one point of $\{(e,1-s,0),(e,1-s,1)\}$ lies in each of $Q$ and $R$.
Let $S$ and $T$ be the classes with $3$-cross-section $\ms{0,0,2}$.

By \cref{Gideon}\ref{Epazote}, we cannot have two distinct classes $X, Y \in \cP$ and $f \in [3]$ with $|X_{\{f\}}|=|Y_{\{f\}}|=2$.
Since $|P_{\{e\}}|=2$, without loss of generality, we may assume that $|S_{\{e\plusmodthree 1\}}|=2$ and $|T_{\{e\plusmodthree 2\}}|=2$.
Then by \cref{Gideon}\ref{Epazote}, we have $S=\{(e\plusmodthree 1,1-t,0),(e\plusmodthree 1,1-t,1)\}$ and $T=\{(e\plusmodthree 2,1-u,0),(e\plusmodthree 2,1-u,1)\}$.
Then $(e\plusmodthree 1,t,1-v)$ and $(e\plusmodthree 2,u,1-w)$ must lie in $Q \cup R$.
Without loss of generality, we may take $(e\plusmodthree 1,t,1-v) \in Q$ and $(e\plusmodthree 2,u,1-w) \in R$.
Since each of $Q$ and $R$ also contains one element from $\{(e,1-s,0),(e,1-s,1)\}$, there is some $x \in [2]$ be such that $Q=\{(e,1-s,x),(e\plusmodthree 1,t,1-v)\}$ and $R=\{(e,1-s,1-x),(e\plusmodthree 2,u,1-w)\}$.
But now we see that $\cP=\cC_{3,e,s,t,u,v,w,x}$, so $\cP \in \fC_3$.
\end{proof}
\begin{lemma}\label{Nestor}
Let. $\cP \in \Con(3)$ with $|\cP|=5$ and no class $\cP$ has $3$-cross-section $\ms{0,0,2}$.  Then $\cP \in \fC_2 \cup \fC_4$.
\end{lemma}
\begin{proof}
If there is some $P \in \cP$ whose $3$-cross-section contains at least two instances of $2$, then \cref{Eugene} shows that $\cP \in \fC_2$.
So henceforth, we assume that no class of $\cP$ has a $3$-cross-section that contains more than one instance of $2$.
Since $\cP$ is a partition of the $12$ elements of $\indexset$ into $5$ classes of even size, its type must be $\ms{4,2,2,2,2}$.
Let $P$ be the class of size $4$ in $\cP$; it must have $3$-cross-section $\ms{1,1,2}$ by \cref{Francine}, which (with the initial assumption) also shows that the four classes of size $2$ must all be have $3$-cross-section $\ms{0,1,1}$.

So by \cref{Gideon}\ref{Epazote} there are some $e \in [3]$ and $s,t,u,v,w \in [2]$ such that $P=\{(e,s,0),(e,s,1),(e\plusmodthree 1,t,v),(e\plusmodthree 2,u,w)\}$.
By \cref{Gideon}\ref{Epazote}, $\cP_{\{e\}}$ is of type $\ms{2,1,1}$: so there are two distinct classes $Q$ and $R$ with $|Q_{\{e\}}|=|R_{\{e\}}|=1$ where one point of $\{(e,1-s,0),(e,1-s,1)\}$ lies in each of $Q$ and $R$.
Let $S$ and $T$ be the remaining two classes in $\cP$ that are not $P$, $Q$, or $R$.
Since $S_{\{e\}}=T_{\{e\}}=\emptyset$ and $S$ and $T$ have $3$-cross-sections $\ms{0,1,1}$, we know that $|S_{\{f\}}|=|T_{\{f\}}|=1$ for $f \in \{e\plusmodthree 1,e\plusmodthree 2\}$.
Note that $|P_{\{f\}}|=1$ for $f \in \{e\plusmodthree 1,e\plusmodthree 2\}$, so that $|(Q\cup R)_{\{f\}}|=1$ for $f \in \{e\plusmodthree 1,e\plusmodthree 2\}$.
Without loss of generality, we suppose that $|Q_{\{e\plusmodthree 1\}}|=1$.
Thus
Since $Q$ has $3$-cross-section $\ms{0,1,1}$ and $|Q_{\{e\}}|=1$, this means that $Q_{\{e\plusmodthree 2\}}=\emptyset$.
Then $|R_{\{e\plusmodthree 2\}}|=1$ and $R_{\{e\plusmodthree 1\}}=\emptyset$.
Since one point of $\{(e,1-s,0),(e,1-s,1)\}$ lies in each of $Q$ and $R$, there is some $x \in [2]$ such that $(e,1-s,x) \in Q$ and $(e,1-s,1-x) \in R$.

Suppose that $(e\plusmodthree 1,t,1-v) \in Q$ to show a contradiction.
Then $(S\cup T)_{\{e\plusmodthree 1\}}=\{(e\plusmodthree 1,1-t,0),(e\plusmodthree 1,1-t,1)\}$.
Since $\cP \in \Con(3)$, there is some $\tau \in \As(\cP,=)$.
Let $A$, $B$, $C$, $D$, $E$ denote the distinct elements of $\N$ such that $\tau_\gamma=A$ (resp., $B$, $C$, $D$, $E$) for $\gamma\in P$ (resp., $Q$, $R$, $S$, $T$).
If we consider values of $\gamma$ of the form $(e,*,*)$ (resp., $(e\plusmodthree 1,*,*)$), we must have $2 A=B+C$ (resp., $A+B=D+E$).
We can add these two equations to get $3 A=C+D+E$.
Since $|P_{\{e\plusmodthree 2\}}|=|R_{\{e\plusmodthree 2\}}|=|S_{\{e\plusmodthree 2\}}|=|T_{\{e\plusmodthree 2\}}|=1$, if we consider values of $\gamma$ of the form $(e\plusmodthree 2,*,*)$, we must have one of (i) $A+C=D+E$, (ii) $A+D=C+E$, or (iii) $A+E=C+D$.
But in each of these three cases, when we combine with $3 A=C+D+E$, we get (i) $A=C$, (ii) $A=D$, or (iii) $A=E$, a contradiction.

So $(e\plusmodthree 1,t,1-v)\not\in Q$, and since $|Q_{\{e\plusmodthree 1\}}|=1$.
So there must be some $y \in [2]$ such that $Q=\{(e,1-s,x),(e\plusmodthree 1,1-t,y)\}$.
Then each of $S$ and $T$ has one element from $\{(e\plusmodthree 1,t,1-v),(e\plusmodthree 1,1-t,1-y)\}$.
Without loss of generality we may take $(e\plusmodthree 1,t,1-v) \in S$ and $(e\plusmodthree 1,1-t,1-y) \in T$.
Since $\cP \in \Con(3)$, there is some $\tau \in \As(\cP,=)$.
Let $A$ (resp., $B$, $C$, $D$, $E$) denote the element of $\N$ such that $\tau_\gamma=A$ (resp., $B$, $C$, $D$, $E$) for $\gamma\in P$ (resp., $Q$, $R$, $S$, $T$).
If we consider values of $\gamma$ of the form $(e,*,*)$ (resp., $(e\plusmodthree 1,*,*)$), we must have $2 A=B+C$ (resp., $A+D=B+E$).
If we subtract these equations, we obtain $A+E=C+D$.
Since $|P_{\{e\plusmodthree 2\}}|=|R_{\{e\plusmodthree 2\}}|=|S_{\{e\plusmodthree 2\}}|=|T_{\{e\plusmodthree 2\}}|=1$, if we consider values of $\gamma$ of the form $(e\plusmodthree 2,*,*)$, we must have one of (i) $A+C=D+E$, (ii) $A+D=C+E$, or (iii) $A+E=C+D$.
If we combine case (i) with the known equation $A+E=C+D$, we obtain $C=E$, a contradiction.
If we combine case (ii) with the known equation $A+E=C+D$, we obtain $D=E$, a contradiction.
So we must have case (iii), which forces $T$ to have an element of the form $(e\plusmodthree 2,u,*)$ and $R$ and $S$ to have elements of the form $(e\plusmodthree 2,1-u,*)$.
So $T=\{(e\plusmodthree 1,1-t,1-y),(e\plusmodthree 2,u,1-w)\}$ and there is some $z \in [2]$ such that $R=\{(e,1-s,1-x),(e\plusmodthree 2,1-u,z)\}$ and $S=\{(e\plusmodthree 1,t,1-v),(e\plusmodthree 2,1-u,1-z)\}$.
Thus $\cP = \cC_{4,e,s,t,u,v,w,x,y,z}$, so $\cP \in \fC_4$.
\end{proof}

\begin{lemma}\label{Abigail}
Let $\cP \in \Con(3)$ with $|\cP|=6$.
Then the number of classes of $\cP$ that have $3$-cross-section $\ms{0,0,2}$ is either $0$, $2$, or $3$.
\end{lemma}
\begin{proof}
Since $\cP$ is an even partition of the $12$ elements of $[3]\times\lindexset$ into $6$ classes, every class is of cardinality $2$.
So each class must have $3$-cross-section $\ms{0,0,2}$ or $\ms{0,1,1}$.
By \cref{Gideon}\ref{Beetroot}, no more than $3$ classes may have $3$-cross-section $\ms{0,0,2}$.

Suppose that precisely one class, $P$, has $3$-cross-section $\ms{0,0,2}$ to show contradiction.
Then we know from \cref{Gideon}\ref{Epazote} that there are some $e \in [3]$ and $s \in [2]$ such that $P=\{(e,s,0),(e,s,1)\}$.
The remaining five classes have $3$-cross-section $\ms{0,1,1}$.
Let $Q$ and $R$ be the two classes with $3$-cross-section $\ms{0,1,1}$ that contain one point each from $\{(e,1-s,0),(e,1-s,1)\}$.
Let $S$, $T$, and $U$ be the remaining three classes other than $P$, $Q$, and $R$.
We know $S_{\{e\}}=T_{\{e\}}=U_{\{e\}}=\emptyset$, and since they have $3$-cross-sections $\ms{0,1,1}$, we must have $|S_{\{f\}}|=|T_{\{f\}}|=|U_{\{f\}}|=1$ for $f \in \{e\plusmodthree 1,e\plusmodthree 2\}$.
So then $|(Q\cup R)_{\{f\}}|=1$ for $f \in \{e\plusmodthree 1,e\plusmodthree 2\}$.
Without loss of generality, we may assume that $|Q_{\{e\plusmodthree 1\}}|=1$ and $Q_{\{e\plusmodthree 2\}}=\emptyset$ and $R_{\{e\plusmodthree 1\}}=\emptyset$ and $|R_{\{e\plusmodthree 2\}}|=1$.
So there are some $t,u,v,w \in [2]$ such that $(e\plusmodthree 1,t,v) \in Q$ and $(e\plusmodthree 2,u,w) \in R$.
Then $S$, $T$, and $U$ each have one element from $\{(e\plusmodthree 1,t,1-v),(e\plusmodthree 1,1-t,0),(e\plusmodthree 1,1-t,1)\}$.
Without loss of generality, we may assume $(e\plusmodthree 1,t,1-v) \in S$, and $(T \cup U)_{\{e\plusmodthree 1\}}=\{(e\plusmodthree 1,1-t,0),(e\plusmodthree 1,1-t,1)\}$.

Suppose that $(e\plusmodthree 2,u,1-w) \in S$ to show contradiction.
Then $(T \cup U)_{\{e\plusmodthree 2\}}=\{(e\plusmodthree 2,1-u,0),(e\plusmodthree 2,1-u,1)\}$.
Since $\cP \in \Con(3)$, there is some $\tau \in \As(\cP,=)$.
Let $A$, $B$, $C$, $D$, $E$, $F$ denote the distinct elements of $\N$ such that $\tau_\gamma=A$ (resp., $B$, $C$, $D$, $E$, $F$) for $\gamma\in P$ (resp., $Q$, $R$, $S$, $T$, $U$).
If we consider values of $\gamma$ of the form $(e\plusmodthree 1,*,*)$ (resp., $(e\plusmodthree 2,*,*)$), we must have $B+D=E+F$ and $C+D=E+F$, which forces $B=C$, a contradiction.

So $(e\plusmodthree 2,u,1-w)\not\in S$, so it must be in either $T$ or $U$.
Without loss of generality, we may assume $(e\plusmodthree 2,u,1-w) \in T$, so then $(S \cup U)_{\{e\plusmodthree 2\}}=\{(e\plusmodthree 2,1-u,0),(e\plusmodthree 2,1-u,1)\}$.
Since $\cP \in \Con(3)$, there is some $\tau \in \As(\cP,=)$.
Let $A$, $B$, $C$, $D$, $E$, $F$ denote the distinct elements of $\N$ such that $\tau_\gamma=A$ (resp., $B$, $C$, $D$, $E$, $F$) for $\gamma\in P$ (resp., $Q$, $R$, $S$, $T$, $U$).
If we consider values of $\gamma$ of the form $(e,*,*)$ (resp., $(e\plusmodthree 1,*,*)$, $(e\plusmodthree 2,*,*)$), we must have $2 A=B+C$, $B+D=E+F$, and $C+E=D+F$.
If we add these three equations, we get $A=F$, a contradiction.
So it is impossible for precisely one class to have $3$-cross-section $\ms{0,0,2}$.
\end{proof}

\begin{lemma}\label{Bailey}
If $\cP \in \Con(3)$ with $|\cP|=6$ and $\cP$ contains three classes with $3$-cross-section $\ms{0,0,2}$, then $\cP \in \fC_5$.
\end{lemma}
\begin{proof}
Since $\cP$ is a partition of the twelve elements of $[3]\times\lindexset$ into six classes of even size, every class is of cardinality $2$.
Three of these have $3$-cross-section $\ms{0,0,2}$, so by \cref{Gideon}\ref{Epazote} we know that for each $e \in [3]$ we must have $\cP_{\{e\}}$ be type $\ms{2,1,1}$.
So there must be $s,t,u \in [2]$ such that the three classes of $\cP$ with $3$-cross-section $\ms{0,0,2}$ are $\{(0,s,0),(0,s,1)\}$, $\{1,t,0),(1,t,1)\}$, and $\{(2,u,0),(2,u,1)\}$.
Let $P$, $Q$, and $R$ be the remaining three classes, each of which must have $3$-cross-section equal to $\ms{0,1,1}$.
Since the classes with $3$-cross-section $\ms{0,0,2}$ have already accounted for two elements of the form $(e,*,*)$ for each $e \in [3]$, we must have $|(P\cup Q\cup R)_{\{e\}}|=2$.
For each $X \in \{P,Q,R\}$, there must be a unique $e_X \in [3]$ such that $X_{\{e_X\}}=\emptyset$ because $X$ has $3$-cross-section $\ms{0,1,1}$.
Furthermore $e_P$, $e_Q$, and $e_R$ must be distinct because if not, there would be some $e \in [3] \smallsetminus \{e_P,e_Q,e_R\}$, and then $|(P\cup Q\cup R)_{\{e\}}|=3$, a contradiction.
Without loss of generality, we may take $e_P=2$, $e_Q=0$, and $e_R=1$; thus $|P_{\{0\}}|=|P_{\{1\}}|=1$, $|Q_{\{1\}}|=|Q_{\{2\}}|=1$, and $|R_{\{2\}}|=|R_{\{0\}}|=1$.
Thus there are some $v,w,x \in [2]$ such that $(0,1-s,v) \in P$, $(1,1-t,w) \in Q$, and $(2,1-u,x) \in R$.
And by process of elimination we must have $(1,1-t,1-w) \in P$, $(2,1-u,1-x) \in Q$, and $(0,1-s,1-v) \in R$.
Thus $\cP=\cC_{5,s,t,u,v,w,x}$ and so $\cP \in \fC_5$.
\end{proof}

\begin{lemma}\label{Ophelia}
If $\cP \in \Con(3)$ with $|\cP|=6$ and $\cP$ contains precisely two classes with $3$-cross-section $\ms{0,0,2}$, then $\cP \in \fC_6$.
\end{lemma}
\begin{proof}
Since $\cP$ is a partition of the twelve elements of $[3]\times\lindexset$ into six classes of even size, every class is of cardinality $2$.
Two of these have $3$-cross-section $\ms{0,0,2}$, so by \cref{Gideon}\ref{Epazote} we know that there must be an $e \in [3]$ such that $\cP_{\{e\}}$ is of type $\ms{1,1,1,1}$ while $\cP_{\{e\plusmodthree 1\}}$ and $\cP_{\{e\plusmodthree 2\}}$ are of type $\ms{2,1,1}$.
So there must be $s,t \in [2]$ such that the two classes of $\cP$ with $3$-cross-section $\ms{0,0,2}$ are $P=\{(e\plusmodthree 1,s,0),(e\plusmodthree 1,s,1)\}$ and $Q=\{e\plusmodthree 2,t,0),(e\plusmodthree 2,t,1)\}$.
Let $R$, $S$, $T$, and $U$ be the remaining four classes, each of which must have $3$-cross-section equal to $\ms{0,1,1}$.
Since $\cP_{\{e\}}$ is of type $\ms{1,1,1,1}$, we must have $|X_{\{e\}}|=1$ for every $X \in \{R,S,T,U\}$.
Furthermore for each $f \in \{e\plusmodthree 1,e\plusmodthree 2\}$, there must be precisely two classes $X,Y \in \{R,S,T,U\}$ with $|X_{\{f\}}|=|Y_{\{f\}}|=1$.
Without loss of generality, we may assume $|R_{\{e\plusmodthree 1\}}|=|T_{\{e\plusmodthree 1\}}|=1$ and $|S_{\{e\plusmodthree 2\}}|=|U_{\{e\plusmodthree 2\}}|=1$.

Suppose for a contradiction that there is some $u \in [2]$ such that $(R \cup T)_{\{e\}}=\{(e,u,0),(e,u,1)\}$.
This forces $(Q\cup S)_{\{e\}}=\{(e,1-u,0),(e,1-u,1)\}$.
Since $\cP \in \Con(3)$, there is some $\tau \in \As(\cP,=)$.
Let $A$, $B$, $C$, $D$, $E$, $F$ denote the distinct elements of $\N$ such that $\tau_\gamma=A$ (resp., $B$, $C$, $D$, $E$, $F$) for $\gamma\in P$ (resp., $Q$, $R$, $S$, $T$, $U$).
If we consider values of $\gamma$ of the form $(e,*,*)$ (resp., $(e\plusmodthree 1,*,*)$, $(e\plusmodthree 2,*,*)$), we must have $C+E=D+F$ (resp., $2 A=C+E$, $2 B=D+F$), which forces $A=B$, a contradiction.

So one of $R$ and $T$ contains an element of the form $(e,0,*)$ (without loss of generality, we may take it to be $R$) and the other contains an element of the form $(e,1,*)$.
So there are some $x,y \in [2]$ such that $(e,0,x) \in R$ and $(e,1,y) \in T$.
Then one element of $\{(e,0,1-x), (e,1,1-y)\}$ lies in each of $S$ and $U$, and without loss of generality, we may take $(e,0,1-x) \in S$ and $(e,1,1-y) \in U$.
Recall that $|R_{\{e\plusmodthree 1\}}|=|T_{\{e\plusmodthree 1\}}|=1$ and $|S_{\{e\plusmodthree 2\}}|=|U_{\{e\plusmodthree 2\}}|=1$; thus, there are some $v, w \in [2]$ such that $(e\plusmodthree 1,1-s,v) \in R$, $(e\plusmodthree 1,1-s,1-v) \in T$, $(e\plusmodthree 2,1-t,w) \in S$, and $(e\plusmodthree 2,1-t,1-w) \in U$.
Thus $\cP=\cC_{6,s,t,v,w,x,y}$ and so $\cP \in \fC_6$.
\end{proof}

\begin{lemma}\label{Beverly}
Let $\cP \in \Con(3)$ with $|\cP|=6$ such that $\cP$ contains no class with $3$-cross-section $\ms{0,0,2}$.
Then $\cP \in \fC_7 \cup \fC_8$.
\end{lemma}
\begin{proof}
Since $\cP$ is an even partition of the $12$ elements of $[3]\times\lindexset$ into $6$ classes, every class is of cardinality $2$.
So each class must have $3$-cross-section $\ms{0,1,1}$.
We say that $\cP$ has the {\it coherence property} to mean that for every $e \in [3]$ and $s \in [2]$ and $A,B \in \cP$, if $(e,s,0) \in A$ and $(e,s,1) \in B$, then the leftmost coordinate of the triple in $A\smallsetminus\{(e,s,0)\}$ is the same as the leftmost coordinate of the triple in $B\smallsetminus\{(e,s,1)\}$.

Let us first assume that $\cP$ has the coherence property.
We shall now show that this implies that $\cP \in \fC_7$.
Then for each $j \in [3]$, we claim that at least one class in $\cP$ is of the form $\{(j,*,*),(j\plusmodthree 1,*,*)\}$.  If not, then the fact that every class has $3$-cross-section $\ms{0,1,1}$, we would have four classes of the form $\{(j,*,*),(j\plusmodthree 2,*,*)\}$, which would force the remaining two classes to be of the form $\{(j\plusmodthree 1,*,*),(j\plusmodthree 1,*,*)\}$, contradicting the fact that every class has $3$-cross-section $\ms{0,1,1}$.
Then by the coherence property, for each $j \in [3]$ there must be at least two classes of the form $\{(j,*,*),(j\plusmodthree 1,*,*)\}$, and since there are $6$ classes in $\cP$, we have exactly two classes of the form $\{(j,*,*),(j\plusmodthree 1,*,*)\}$ for each $j \in [3]$.
Let $s,t \in [2]$ be such that there is a class $M$ in $\cP$ of the form $M=\{(0,s,*),(1,1-t,*)\}$; then there must be one other class $M'$ containing an element of the form $(0,s,*)$, but by the coherence property $M'$ must also contain an element of the form $(1,*,*)$ and therefore $M$ and $M'$ must be the two classes in $\cP$ that of the form $\{(0,*,*),(1,*,*)\}$.
On the other hand, there must be a class $M''$ distinct from $M$ containing an element of the form $(1,1-t,*)$, but by the coherence property $M''$ must also contain an element of the form $(0,*,*)$ and therefore $M$ and $M''$ must be the two classes in $\cP$ that are of the form $\{(0,*,*),(1,*,*)\}$; thus $M'=M''$.
So there is some $v\in [2]$ so that $M$ and $M'$ are $\{(0,s,0),(1,1-t,v)\}$ and $\{(0,s,1),(1,1-t,1-v)\}$.
The remaining four classes must each contain an element of the form $(2,*,*)$; of these two must be of the form $\{(1,t,*),(2,*,*)\}$ and two must be of the form $\{(2,*,*),(0,1-s,*)\}$.
By the coherence property there must be some $u$ such that the former two classes are of the form $\{(1,t,*),(2,1-u,*)\}$ and the latter two classes are of the form $\{((0,1-s,*),(2,u,*)\}$.
Then there is some $w\in [2]$ so that one of the former classes is $\{(1,t,0),(2,1-u,w)\}$ and the other is $\{(1,t,1),(2,1-u,1-w)\}$.
And there is some $x \in [2]$ such one of the latter classes is $\{(2,u,0),(0,1-s,x)\}$ and the other is $\{(2,u,1),(0,1-s,1-x)\}$.
Then we see that $\cP \in \fC_7$.

From now on, we assume that $\cP$ does not have the coherence property.
We shall eventually show that this implies that $\cP \in \fC_8$.
Assume that there are $b,c, d\in [2]$ and distinct $e,f \in [3]$ such that $G=\{(e,b,0),(f,c,d)\}$ and $H=\{(e,b,1),(f,c,1-d)\}$ are classes in $\cP$ in order to show contradiction.
Let $k$ be the element of $[3]$ that is neither $e$ nor $f$.
Then each of the four remaining classes must contain an element of the form $(k,*,*)$ and since $\cP$ has the weak decoherence property, these classes must be of the form $I=\{(k,0,*),(e,1-b,*)\}$, $J=\{(k,0,*),(f,1-c,*)\}$, $K=\{(k,1,*),(e,1-b,*)\}$, $L=\{(k,1,*),(f,1-c,*)\}$.
Since $\cP \in \Con(3)$, there is some $\tau \in \As(\cP,=)$.
Let $A$, $B$, $C$, $D$, $E$, $F$ denote the distinct elements of $\N$ such that $\tau_\gamma=A$ (resp., $B$, $C$, $D$, $E$, $F$) for $\gamma\in G$ (resp., $H$, $I$, $J$, $K$, $L$).
If we consider values of $\gamma$ of the form $(e,*,*)$ (resp., $(f,*,*)$, $(k,*,*)$), we must have $A+B=C+E$ (resp., $A+B=D+F$, $C+D=E+F$), which forces $D=E$, a contradiction.

So in fact $\cP$ has the property that for every $e,f \in [3]$ and $b,c \in [2]$, if one class in $\cP$ is of the form $\{(e,b,*),(f,c,*)\}$, then there must be another class of the form $\{(e,b,*),(g,d,*)\}$ for some $g \in [3]$ and $d \in [2]$ where $(g,d)\not=(f,c)$; we call this fact the {\it weak decoherence property}.
Let us assume that there is at least one case where $g=f$ to show contradiction.
Then $d\not=c$, i.e., $d=1-c$, so that our two classes are of the form $G=\{(e,b,*),(f,c,*)\}$ and $H=\{(e,b,*),(f,1-c,*)\}$.
Let $h$ be the unique element of $[3] \smallsetminus\{e,f\}$.
Then each of the four remaining classes must contain an element of the form $(h,*,*)$ and since $\cP$ does not have the coherence property, then there must be some $i \in [2]$ such that these classes have the form $I=\{(h,i,*),(e,1-b,*)\}$, $J=\{(h,i,*),(f,c,*)\}$, $K=\{(h,1-i,*),(e,1-b,*)\}$, $L=\{(h,1-i,*),(f,1-c,*)\}$.
Since $\cP \in \Con(3)$, there is some $\tau \in \As(\cP,=)$.
Let $A$, $B$, $C$, $D$, $E$, $F$ denote the distinct elements of $\N$ such that $\tau_\gamma=A$ (resp., $B$, $C$, $D$, $E$, $F$) for $\gamma\in G$ (resp., $H$, $I$, $J$, $K$, $L$).
If we consider values of $\gamma$ of the form $(e,*,*)$ (resp., $(f,*,*)$, $(h,*,*)$), then we must have $A+B=C+E$ (resp., $A+D=B+F$, $C+D=E+F$), which forces $A=C$, a contradiction.

So in fact $\cP$ has the property that for every $e,f \in [3]$ and $b \in [2]$, if one class in $\cP$ is of the form $\{(e,b,*),(f,*,*)\}$, then there must be another class of the form $\{(e,b,*),(g,*,*)\}$ where $g$ is the unique element of $[3] \smallsetminus \{e,f\}$; we call this the {\it strong decoherence property}.
Thus, considering the two classes that contain elements of the form $(0,0,*)$, there must be some $s,t,v,z,a \in [2]$ such that these two classes must be $P=\{(0,0,v),(1,1-s,1-z)\}$ and $U=\{(0,0,1-v),(2,1-t,a)\}$.
Similarly, the two classes that contain elements of the form $(0,1,*)$ must be of the forms $\{(0,1,*),(1,*,*)\}$ and $\{(0,1,*),(2,*,*)\}$.
In fact, strong decoherence forces these to be of the form $\{(0,1,*),(1,s,*)\}$ and $\{(0,1,*),(2,t,*)\}$.
So there must be some $w,x,y\in [2]$ such that these two classes are $Q=\{(0,1,y),(1,s,1-w)\}$ and $T=\{(0,1,1-y),(2,t,x)\}$.
This leaves four elements, $(1,s,w),(1,1-s,z),(2,t,1-x),(2,1-t,1-a)$, that do not lie in the four aforementioned classes $P$, $Q$, $T$, $U$; these four elements are organized into two classes of two elements each.
Since there are no twin classes, there is one class containing $(1,s,w)$ and another class contains $(1,1-s,z)$.

Suppose that one class is $G=\{(1,s,w),(2,t,1-x)\}$ to show contradiction.  Then the classes are $P$, $Q$, $T$, $U$, $G$, and $H=\{(1,1-s,z),(2,1-t,1-a)\}$.
Since $\cP \in \Con(3)$, there is some $\tau \in \As(\cP,=)$.
Let $A$ (resp., $B$, $C$, $D$, $E$, $F$) denote the element of $\N$ such that $\tau_\gamma=A$ (resp., $B$, $C$, $D$, $E$, $F$) for $\gamma\in P$ (resp., $Q$, $G$, $H$, $T$, $U$).
If we consider values of $\gamma$ of the form $(0,*,*)$ (resp., $(1,*,*)$, $(2,*,*)$), we must have $A+F=B+E$ (resp., $C+B=D+A$, $E+C=F+D$), which forces $C=D$, a contradiction.

So the class containing $(1,s,w)$ must be $\{(1,s,w),(2,1-t,1-a)\}$; we class this class $R$.
The final class is $S=\{(1,1-s,z),(2,t,1-x)\}$.
Then $\cP=\{P,Q,R,S,T,U\} \in \fC_8$.
\end{proof}
We now observe that Lemmata \ref{Eugene}--\ref{Beverly} cumulatively show that all partitions in $\Con(3)$ reside in one of our putative equivalence classes.
\begin{lemma}\label{Alexander}
We have $\Con(3) \subseteq \bigsqcup_{m=1}^8 \fC_m$.
\end{lemma}
\begin{proof}
Let $\cP \in \Con(3)$.
\cref{Gideon}\ref{Dill} shows that $4 \leq |\cP| \leq 6$.
If $|\cP|=4$, then \cref{Eusebius} shows that $\cP \in \fC_1$.
If $|\cP|=5$, then Lemmas \ref{Zoe} and \ref{Nestor} show that $\cP \in \fC_2 \cup \fC_3 \cup \fC_4$.
If $|\cP|=6$, the \cref{Abigail} shows that the number of classes in $\cP$ with $3$-cross-section $\ms{0,0,2}$ is equal to $0$, $2$, or $3$, and then Lemmas \ref{Bailey}, \ref{Ophelia}, and \ref{Beverly} show that $\cP \in \fC_5 \cup \fC_6 \cup \fC_7 \cup \fC_8$.
Thus, in every case $\cP \in \cup_{m=1}^8 \fC_m$, and the union is disjoint by \cref{Jacob}.
\end{proof}

\subsection{Our classes constitute $\Isom(3)$}\label{Dave}

In this section, we show that our putative isomorphism classes are indeed the full set of isomorphism classes that make up $\Isom(3)$.
To do this, we need to prove that the partitions in our putative isomorphism classes are satisfiable, and since we need to know the exact value of $\Sols(\fP,\ell)$ for each $\fP \in \Isom(3)$ to calculate the third central moment of $\ssac$ using \cref{Sanria}, we prove satisfiability by obtaining $\Sols(\fC_j,\ell)$ for $1 \leq j \leq 8$ and $\ell\in\N$.
\begin{lemma} \label{Apple}
We have $\Sols(\fC_1,\ell)=\floor{(\ell-1)(\ell-2)/3}$.
\end{lemma}
\begin{proof}
By \cref{Sonia}, $\Sols(\fC_1,\ell)=|\As(\cP,=,\ell)|$, where $\cP=C_{1,0,0,0,0,0,0,0,0}=\{P,Q,R,S\}$ with
\begin{align*}
P & = \{(1,0,0),(1,0,1),(2,1,0),(0,0,0)\} \\ 
Q & = \{(2,0,0),(2,0,1),(1,1,0),(0,0,1)\} \\
R & = \{(1,1,1),(0,1,0)\} \\
S & = \{(2,1,1),(0,1,1)\}.
\end{align*}
If $\tau \in \As(\cP,\ell)$, let $A$ (resp., $B$, $C$, $D$) denote the element of $[\ell]$ such that $\tau_\gamma=A$ (resp., $B$, $C$, $D$) for $\gamma\in P$ (resp., $Q$, $R$, $S$), then $\tau \in \As(\cP,=,\ell)$ if and only if we satisfy the following equations:
\begin{align*}
A+B&=C+D \\
2A &=B+C\\
2B &=A+D
\end{align*}
with distinct $ A, B, C, D  \in [\ell]$.
The first equation is redundant because it follows from the second and third.
Since $ A $ is the average of $ B $ and $ C $, while $ B $ is the average of $ A $ and $ D $, we see that $C, A, B, D$ is an arithmetic progression.
Since we can have either $C<A<B<D$ or $D<B<A<C$, we should count twice the number of $4$-term arithmetic progressions residing in $[\ell]$, which by \cref{Fuzz} is $(\ell-r)(\ell+r-3)/6$ where $r \in \{0,1,2\}$ with $r \equiv \ell \pmod{3}$.
We conclude that $|\As(\cP,=,\ell)|$ is $(\ell^2-3\ell)/3 $ whenever $\ell $ is a multiple of $ 3 $ and $(\ell^2-3\ell+2)/3 $ otherwise.
In other words, $|\As(\cP,=,\ell)|=\floor{(\ell-1)(\ell-2)/3}$.
\end{proof}

\begin{lemma}
We have \[\Sols(\fC_2,\ell)=
\begin{cases}
\frac{\ell^3-6\ell^2+8\ell}{3} &  \text{if $\ell$ is even,} \\[4pt]
\frac{\ell^3-6\ell^2+11\ell -6}{3} & \text{if $\ell$ is odd.}
\end{cases}
\]
\end{lemma}
\begin{proof}
By \cref{Sonia}, $\Sols(\fC_2,\ell)=|\As(\cP,=,\ell)|$, where $\cP=C_{2,0,0,0,0,0,0}=\{P,Q,R,S,T\}$ with
\begin{align*}
P & = \{(1,0,0),(1,0,1),(2,0,0),(2,0,1)\} \\
Q & = \{(0,0,0),(1,1,0)\} \\
R & = \{(0,0,1),(1,1,1)\} \\
S & = \{(0,1,0),(2,1,0)\} \\
T & = \{(0,1,1),(2,1,1)\}.
\end{align*}
If $\tau \in \As(\cP,\ell)$, let $A$ (resp., $B$, $C$, $D$, $E$) denote the element of $[\ell]$ such that $\tau_\gamma=A$ (resp., $B$, $C$, $D$, $E$) for $\gamma\in P$ (resp., $Q$, $R$, $S$, $T$), then $\tau \in \As(\cP,=,\ell)$ if and only if we satisfy the following equations:
\begin{align*}
B+C &= D+E \\
2 A & = B+C \\
2 A & = D+E
\end{align*}
with distinct $ A, B, C, D, E  \in [\ell]$.
The first equation is redundant because it follows from the second and third.
Since $ A $ is the average of $ B $ and $ C $ and of $ D $ and $ E $, then if $ B,C,D $ and $ E $ are distinct then so are $ A, B, C, D, E $.
Hence, the number of solutions with distinct $ A, B, C, D, E  \in [\ell] $ to the above system equals the number of solutions with distinct $ B, C, D, E  \in [\ell] $ of 
\begin{equation*}
B+C = D+E
\end{equation*} 
such that $ B $ and $ C $ have the same parity. \cref{wowzers} gives the number of solutions.
\end{proof}

\begin{lemma}
We have \[\Sols(\fC_3,\ell)=
\begin{cases}
\frac{\ell^2-4\ell}{4} & \text{if $\ell \equiv 0 \pmod{4}$,} \\[4pt]
\frac{\ell^2-4\ell+4}{4} & \text{if $\ell \equiv 2 \pmod{4}$,} \\[4pt]
\frac{\ell^2-4\ell+3}{4} & \text{otherwise.}
\end{cases}\]
\end{lemma}
\begin{proof}
By \cref{Sonia}, $\Sols(\fC_3,\ell)=|\As(\cP,=,\ell)|$, where $\cP=C_{3,0,0,0,0,0,0,0}=\{P,Q,R,S,T\}$ with
\begin{align*}
P & = \{(0,0,0),(0,0,1),(1,0,0),(2,0,0)\} \\
Q & = \{(0,1,0),(1,0,1)\} \\
R & = \{(0,1,1),(2,0,1)\} \\
S & = \{(1,1,0),(1,1,1)\} \\
T & = \{(2,1,0),(2,1,1)\}.
\end{align*}
If $\tau \in \As(\cP,\ell)$, let $A$ (resp., $B$, $C$, $D$, $E$) denote the element of $[\ell]$ such that $\tau_\gamma=A$ (resp., $B$, $C$, $D$, $E$) for $\gamma\in P$ (resp., $Q$, $R$, $S$, $T$), then $\tau \in \As(\cP,=,\ell)$ if and only if we satisfy the following equations:
\begin{align*}
2 A & = B + C \\
A+B & = 2 D \\
A+C & = 2 E
\end{align*}
with distinct $ A, B, C, D, E  \in [\ell]$.
This system is equivalent to saying that $C,E,A,D,B$ is an arithmetic progression.
Since we can have either either $C<E<A<D<B$ or $B<D<A<E<C$, we should double the count of $5$-term arithmetic progressions residing in $[\ell]$, which by \cref{Fuzz} is equal to $(\ell-r)(\ell+r-4)/8$, where $r \in \{0,1,2,3\}$ with $r \equiv \ell \pmod{4}$.
We conclude that
\[
|\As(\cP,=,\ell)| = 
\begin{cases}
\frac{\ell^2-4\ell}{4} & \text{if $\ell \equiv 0 \pmod{4}$,} \\[4pt]
\frac{\ell^2-4\ell+4}{4} & \text{if $\ell \equiv 2 \pmod{4}$,} \\[4pt]
\frac{\ell^2-4\ell+3}{4} & \text{otherwise.} 
\end{cases} 
\]
\end{proof}
\begin{lemma}
We have \[\Sols(\fC_4,\ell)=
\begin{cases}
\frac{5\ell^3-32\ell^2+52\ell}{12} & \text{if $\ell \equiv 0 \pmod{6}$,} \\[4pt]
\frac{5\ell^3-32\ell^2+55\ell-28}{12} & \text{if $\ell \equiv \pm 1 \pmod{6}$,} \\[4pt]
\frac{5\ell^3-32\ell^2+55\ell-12}{12} & \text{if $\ell \equiv 3 \pmod{6}$,} \\[4pt]
\frac{5\ell^3-32\ell^2+52\ell-16}{12} & \text{otherwise.}
\end{cases} \]
\end{lemma}
\begin{proof}
By \cref{Sonia}, $\Sols(\fC_4,\ell)=|\As(\cP,=,\ell)|$, where $\cP=C_{4,0,0,0,0,0,0,0,0,0}=\{P,Q,R,S,T\}$ with
\begin{align*}
P & = \{(0,0,0),(0,0,1),(1,0,0),(2,0,0)\} \\
Q & = \{(0,1,0),(1,1,0)\} \\
R & = \{(0,1,1),(2,1,0)\} \\
S & = \{(1,0,1),(2,1,1)\} \\
T & = \{(1,1,1),(2,0,1)\}.
\end{align*}
If $\tau \in \As(\cP,\ell)$, let $A$ (resp., $B$, $C$, $D$, $E$) denote the element of $[\ell]$ such that $\tau_\gamma=A$ (resp., $B$, $C$, $D$, $E$) for $\gamma\in P$ (resp., $Q$, $R$, $S$, $T$), then $\tau \in \As(\cP,=,\ell)$ if and only if we satisfy the following equations:
\begin{align*}
2 A & = B + C \\
A+D & = B + E \\
A+E & = C + D.
\end{align*}
with distinct $ A, B, C, D, E  \in [\ell]$.
This system is equivalent to $E-D=A-B=C-A$, so that $A$ is the average of $B$ and $C$.
Hence, either (i) $C<A<B $ and $E<D $ or (ii) $B<A<C $ and $D<E$.
Any solution to (i) yields a solution to (ii) by interchanging the values of $C$ and $E$ with the values of $B$ and $D$, respectively.
So we restrict to (i) and then just double our counts.
Since case (i) corresponds to \cref{Light}, the number of ways for (i) to occur with distinct $A, B, C, D, E  \in [\ell] $ is
\begin{align*}
\begin{cases}
\frac{5\ell^3-32\ell^2+52\ell}{24} & \text{if $\ell \equiv 0 \pmod{6}$,} \\[4pt]
\frac{5\ell^3-32\ell^2+55\ell-28}{24} & \text{if $\ell \equiv \pm 1 \pmod{6}$,} \\[4pt]
\frac{5\ell^3-32\ell^2+52\ell-16}{24} & \text{if $\ell \equiv \pm 2 \pmod{6}$,} \\[4pt]
\frac{5\ell^3-32\ell^2+55\ell-12}{24} & \text{if $\ell \equiv 3 \pmod{6}$.}
\end{cases}
\end{align*}
By multiplying this quantity by two we find $|\As(\cP,=,\ell)|$.
\end{proof}

\begin{lemma}
We have \[\Sols(\fC_5,\ell)=
\begin{cases}
\frac{\ell^3-9\ell^2+20\ell}{4} & \text{if $\ell \equiv 0 \pmod{4}$,} \\[4pt]
\frac{\ell^3-9\ell^2+20\ell-12}{4} & \text{if $\ell \equiv 2 \pmod{4}$,} \\[4pt]
\frac{\ell^3-9\ell^2+23\ell-15}{4} & \text{if $\ell \equiv 1,3 \pmod{4}$.} \\
\end{cases} \]
\end{lemma}
\begin{proof}
By \cref{Sonia}, $\Sols(\fC_5,\ell)=|\As(\cP,=,\ell)|$, where $\cP=C_{5,0,0,0,0,0,0}=\{P,Q,R,S,T,U\}$ with
\begin{align*}
P & = \{(0,0,0),(0,0,1)\} \\
Q & = \{(1,0,0),(1,0,1)\} \\
R & = \{(2,0,0),(2,0,1)\} \\
S & = \{(0,1,0),(1,1,1)\} \\
T & = \{(1,1,0),(2,1,1)\} \\
U & = \{(2,1,0),(0,1,1)\}.
\end{align*}
If $\tau \in \As(\cP,\ell)$, let $A$ (resp., $B$, $C$, $D$, $E$, $F$) denote the element of $[\ell]$ such that $\tau_\gamma=A$ (resp., $B$, $C$, $D$, $E$, $F$) for $\gamma\in P$ (resp., $Q$, $R$, $S$, $T$, $U$), then $\tau \in \As(\cP,=,\ell)$ if and only if we satisfy the following equations:
\begin{align*}
2 A & = D+F \\
2 B & = E+D \\
2 C & = F+E    
\end{align*}
with distinct $ A, B, C, D, E, F  \in [\ell]$.
Since $ A, B $, and $ C $ are averages then $D$, $E$, and $F$ must have the same parity.
Moreover, if $D$, $E$, and $F$ were distinct then $A$, $B$, and $C $ would be distinct from each other.
However, if $D$, $E$, and $F$ form an arithmetic progression then $|\{D,E,F\} \cap \{A,B,C\}| =1$, but if $D$, $E$, and $F$ are distinct and do not form an arithmetic progression, then all six values are distinct.
Hence, we must find how many ways we can pick $D$, $E$, and $F$ such that they have the same parity but do not form an arithmetic progression.
	
Now, there are $ 6 $ ways to order $D$, $E$, and $F$, and any solution to one ordering corresponds to a solution to a different ordering by interchanging values between variables.
So, for the rest of this paragraph, we restrict to the case $ D<E<F$, and afterward we shall multiply our count by $6$.
The number of ways to have $D$, $E$, and $F$ all be even is $ \binom{\ceil{\ell/2}}{3} $ and the number of ways to have $D$, $E$, and $F$ all be odd is $ \binom{\floor{\ell/2}}{3}$.
Hence the number of ways to have $D$, $E$, and $F$ such that they have the same parity is $\binom{\ceil{\ell/2}}{3} +  \binom{\floor{\ell/2}}{3}$.
Next, let $ M \in \{ \ceil{\ell/2},\floor{\ell/2} \} $ be the number of integers of a particular parity in $ [\ell] $.
By \cref{Fuzz} if we write $ M = 2q+r $ with $ 0 \leq r <2 $ then the number of ways to have $D$, $E$, and $F$ with that particular parity in an arithmetic progression is $(M-r)(M+r-2)/4$. So our solution depends on what $\ell $ is modulo $ 4 $.
If $\ell \equiv 0 \pmod 4 $, then our solution equals 
\[2\cdot \binom{\ell/2}{3} - 2 \cdot \frac{(\ell/2)(\ell/2-2)}{4} = \frac{\ell^3-9\ell^2+20\ell}{24}. \]
If $\ell \equiv 1  \pmod 4 $, then our count equals 
\[ \binom{(\ell+1)/2}{3}  + \binom{(\ell-1)/2}{3}  -  \frac{((\ell+1)/2-1)^2}{4} - \frac{((\ell-1)/2)((\ell-1)/2-2)}{4} = \frac{\ell^3-9\ell^2+23\ell-15}{24}. \]	
If $\ell \equiv 2 \pmod 4 $, then our count equals 
\[  2\cdot \binom{\ell/2}{3} - 2 \cdot \frac{(\ell/2-1)^2}{4} = \frac{\ell^3-9\ell^2+20\ell-12}{24}. \]
If $\ell \equiv 3 \pmod 4 $, then our count equals
\[ \binom{(\ell+1)/2}{3}  +\binom{(\ell-1)/2}{3}  -  \frac{((\ell+1)/2)((\ell+1)/2-2)}{4} - \frac{((\ell-1)/2-1)^2}{4}  = \frac{\ell^3-9\ell^2+23\ell-15}{24}.  \]               

We now multiply these counts by $6$ to get the number of solutions to the system, which finishes the proof.
\end{proof}


\begin{lemma}
We have \[\Sols(\fC_6,\ell)=
\begin{cases}
\frac{\ell^3-8\ell^2+17\ell}{3} & \text{if $\ell \equiv 0 \pmod{12}$,} \\[4pt]
\frac{\ell^3-8\ell^2+17\ell-10}{3} & \text{if $\ell \equiv 1,2,5,7,10,11 \pmod{12}$,} \\[4pt]
\frac{\ell^3-8\ell^2+17\ell-6}{3}  & \text{if $\ell \equiv 3,6,9 \pmod{12}$,} \\[4pt]
\frac{\ell^3-8\ell^2+17\ell-4}{3} & \text{if $\ell \equiv 4,8 \pmod{12}$.}
\end{cases} 
 \]
\end{lemma}

\begin{proof}
By \cref{Sonia}, $\Sols(\fC_6,\ell)=|\As(\cP,=,\ell)|$, where $\cP=C_{6,0,0,0,0,0,0,0}=\{P,Q,R,S,T,U\}$ with
\begin{align*}
P & = \{(1,0,0),(1,0,1)\} \\
Q & = \{(2,0,0),(2,0,1)\} \\
R & = \{(0,0,0),(1,1,0)\} \\
S & = \{(0,0,1),(2,1,0)\} \\
T & = \{(0,1,0),(1,1,1)\} \\
U & = \{(0,1,1),(2,1,1)\}.
\end{align*}
If $\tau \in \As(\cP,\ell)$, let $A$ (resp., $B$, $C$, $D$, $E$, $F$) denote the element of $[\ell]$ such that $\tau_\gamma=A$ (resp., $B$, $C$, $D$, $E$, $F$) for $\gamma\in P$ (resp., $Q$, $R$, $S$, $T$, $U$); then $\tau \in \As(\cP,=,\ell)$ if and only if we satisfy the following equations:
\begin{align*}
C+D & = E+F \\
2 A & = C+E \\
2 B & = D+F
\end{align*}
with distinct $ A,B,C,D,E,F \in [\ell] $.  This system is equivalent to saying that $C$, $A$, $E$ and $F$, $B$, $D$ are arithmetic progressions with the same spacings. Hence we either have (i) $ E<A<C $ and $ D<B<F $ or (ii) $ C<A<E $ and $ F<B<D $. Notice that any solution to (i) yields a solution to (ii) by interchanging the values of $ E $ and $ D $ with $ C $ and $ F $, respectively. So, we restrict to (i) and double our counts. 
		
Since $ E,A,C $ is an arithmetic progression in $ [\ell] $, then if we fix a spacing $ s $, there are $\ell-2s $ ways to have $ E<A<C $. Similarly, for a fixed spacing $ s $, there are $\ell-2s $ ways to have $ D<B<F $. Thus, if we allow $A,B,C,D,E,F\in [\ell] $ to not necessarily be distinct, the number of ways for (i) to occur is 
\[ \sum_{s=1}^{\floor{(\ell-1)/2}} (\ell-2s)^2 = \frac{\ell(\ell-1)(\ell-2)}{6}.\]
Since we are only counting positive spacings, then the only ways we can have $A,B,C,D,E,F$ not all distinct are when one of the three occurs: (a) $ \{C,A,E\} = \{D,B,F\}  $, (b) $ |\{C,A,E\} \cap \{D,B,F\}| = 2 $, or (c) $ |\{C,A,E\} \cap \{D,B,F\}| = 1 $. 

Now, the only way we can have (a) is if $ C=F $, $ A=B $, and $ E=D $. If $\ell \equiv r_2 \bmod 2$ with $ 0 \leq r_2 <2 $, then by \cref{Fuzz} there are $(\ell-r_2)(\ell+r_2-2)/4$ ways for (a) to occur. Next, for (b) to occur we must have either $ A=F $ and $ E=B $ or $ C=B $ and $ A=D $. By \cref{Fuzz}, if  $\ell \equiv r_3 \bmod 3$ with $ 0 \leq r_3 <3 $ there are $2(\ell-r_3)(\ell+r_3-3)/6$ ways for (b) to occur. Finally, for (c) to occur either $ E=F $ or $ C=D $. By \cref{Fuzz}, if  $\ell \equiv r_4 \bmod 4$ with $ 0 \leq r_4 <4 $ there are $2(\ell-r_4)(\ell+r_4-4)/8$ ways for (c) to occur. Thus, the number ways (i) can take place is
\[
\frac{\ell(\ell-1)(\ell-2)}{6} - \frac{(\ell-r_2)(\ell+r_2-2)}{4} - \frac{(\ell-r_3)(\ell+r_3-3)}{3}  - \frac{(\ell-r_4)(\ell+r_4-4)}{4},
\]
and we double this to obtain the full count
\[
\Sols(\fC_6,\ell)=\frac{\ell(\ell-1)(\ell-2)}{3} - \frac{(\ell-r_2)(\ell+r_2-2)}{2} - \frac{2(\ell-r_3)(\ell+r_3-3)}{3} - \frac{(\ell-r_4)(\ell+r_4-4)}{2}.
\]
To finish the proof, one works out the values of this expression for the values $r_2$, $r_3$, and $r_4$ take when $\ell$ runs through the congruence classes modulo $12$.
\end{proof}

\begin{lemma}
We have \[
\Sols(\fC_7,\ell)=
\begin{cases}
\frac{\ell^4-10\ell^3+32\ell^2-32\ell}{2} & \text{if $\ell \equiv 0 \pmod{2}$,} \\[4pt]
\frac{\ell^4-10\ell^3+32\ell^2-38\ell+15}{2} & \text{if $\ell \equiv 1 \pmod{2}$.}
\end{cases} 
\]
\end{lemma}
\begin{proof}
By \cref{Sonia}, $\Sols(\fC_7,\ell)=|\As(\cP,=,\ell)|$, where $\cP=C_{7,0,0,0,0,0,0}=\{P,Q,R,S,T,U\}$ with
\begin{align*}
P & = \{(0,0,0),(1,1,0)\} \\
Q & = \{(0,0,1),(1,1,1)\} \\
R & = \{(1,0,0),(2,1,0)\} \\
S & = \{(1,0,1),(2,1,1)\} \\
T & = \{(2,0,0),(0,1,0)\} \\
U & = \{(2,0,1),(0,1,1)\}.
\end{align*}
If $\tau \in \As(\cP,\ell)$, let $A$ (resp., $B$, $C$, $D$, $E$, $F$) denote the element of $[\ell]$ such that $\tau_\gamma=A$ (resp., $B$, $C$, $D$, $E$, $F$) for $\gamma\in P$ (resp., $Q$, $R$, $S$, $T$, $U$), then $\tau \in \As(\cP,=,\ell)$ if and only if we satisfy the following equations:
\begin{align*}
A+B & = E+F \\
C+D & = A+B \\
E+F & = C+D.
\end{align*}
with distinct $ A,B,C,D,E,F \in [\ell] $.  This system is equivalent to $A+B=C+D=E+F$. Let $ h = A+B $. In order for $A,B,C,D,E,F \in [\ell] $ to be distinct we need to be able to write $ h $ as the sum of two different numbers with at least three different unordered pairs of numbers. If $ h \leq \ell-1 $, then  by \cref{wow} there are $ h+1 $ ordered pairs $(x,y)$ with $x,y \in [\ell]$ such that $x+y=h$ and so there are $ \ceil{h/2} $ unordered pairs of numbers $\{x,y\} \subseteq [\ell]$ with $ x\not = y $ such that $ x+y=h $. If $ h>\ell-1 $ then, by \cref{wow} there are $ 2(\ell-1)-h+1 $ ordered pairs $(x,y)$ with $x,y \in [\ell]$ such that $x+y=h$ and so there are $\ceil{(2(\ell-1)-h)/2} $ unordered pairs of numbers $ \{x,y\} \subseteq [\ell]$ with $ x\not = y $ such that $ x+y=h $. Any three pairs that work can be assigned in $ 3! $ ways to the following sets $ \{ A,B\}, \{F,E\}, \{C,D\} $. Once all pairs are assigned there are two different ways to assign numbers to variables for each $ 2 $-set. Hence, for a fixed $ h $ there are 
\begin{align*}
\begin{cases}
3!\cdot 2^3\cdot\binom{\ceil{h/2}}{3} & \text{if $h\leq \ell-1$,} \\[4pt]
3!\cdot 2^3\cdot\binom{\ceil{(2(\ell-1)-h)/2}}{3} & \text{if $h> \ell-1$}
\end{cases}
\end{align*}
ways to have $A=B=C+D=E+F=h$ with distinct $A,B,C,D,E,F \in [\ell]$.
Thus, the number of the solutions to the system must be 
\[
3! \cdot 2^3 \cdot \left[ \sum_{h=5}^{\ell-1} \binom{\ceil{h/2}}{3} + \sum_{h=\ell}^{2\ell-7}\binom{\ceil{(2(\ell-1)-h)/2}}{3}\right],
\]
which, when $\ell$ is even, is equal to
\[
3! \cdot 2^3 \cdot \left[ \binom{\ell/2}{3} + 4 \sum_{i=0}^{\ell/2-1} \binom{i}{3} \right] = 3! \cdot 2^3 \cdot \left[ \binom{\ell/2}{3} + 4 \binom{\ell/2}{4} \right],
\]
but when $\ell$ is odd, is equal to
\[
3! \cdot 2^3 \cdot \left[ -\binom{(\ell-1)/2}{3} + 4 \sum_{i=0}^{(\ell-1)/2} \binom{i}{3} \right] = 3! \cdot 2^3 \cdot \left[ -\binom{(\ell-1)/2}{3} + 4 \binom{(\ell+1)/2}{4} \right].
\]
These expressions for $|\As(\cP,=,\ell)|=\Sols(\fC_7,\ell)$ simplify to those in the statement of this lemma.
\end{proof}

\begin{lemma}\label{Orlando}
We have \[
\Sols(\fC_8,\ell)=
\begin{cases}
\frac{\ell^4-11\ell^3+39\ell^2-46\ell}{2} & \text{if $\ell \equiv 0 \pmod{6}$,} \\[4pt]
\frac{\ell^4-11\ell^3+39\ell^2-49\ell+20}{2} & \text{if $\ell \equiv \pm 1 \pmod{6}$,} \\[4pt]
\frac{\ell^4-11\ell^3+39\ell^2-46\ell+8}{2} & \text{if $\ell \equiv \pm 2 \pmod{6}$,} \\[4pt]
\frac{\ell^4-11\ell^3+39\ell^2-49\ell+12}{2} & \text{if $\ell \equiv 3 \pmod{6}$.}
\end{cases} 
\]
\end{lemma}
\begin{proof}
By \cref{Sonia}, $\Sols(\fC_8,\ell)=|\As(\cP,=,\ell)|$, where $\cP=C_{8,0,0,0,0,0,0,0,0}=\{P,Q,R,S,T,U\}$ with
\begin{align*}
P & = \{(0,0,0),(1,1,1)\} \\
Q & = \{(0,1,0),(1,0,1)\} \\
R & = \{(1,0,0),(2,1,1)\} \\
S & = \{(1,1,0),(2,0,1)\} \\
T & = \{(2,0,0),(0,1,1)\} \\
U & = \{(2,1,0),(0,0,1)\}.
\end{align*}
If $\tau \in \As(\cP,\ell)$, let $A$ (resp., $B$, $C$, $D$, $E$, $F$) denote the element of $[\ell]$ such that $\tau_\gamma=A$ (resp., $B$, $C$, $D$, $E$, $F$) for $\gamma\in P$ (resp., $Q$, $R$, $S$, $T$, $U$), then $\tau \in \As(\cP,=,\ell)$ if and only if we satisfy the following equations:
\begin{align*}
A+F & = B+E \\
C+B & = D+A \\
E+D & = F+C
\end{align*}
with distinct $ A,B,C,D,E,F \in [\ell] $.
This system is equivalent to $A-B=C-D=E-F$.
Hence, we either have (i) $ A>B $, $ C>D $, $ E>F $ or (ii) $ B>A $, $ D>C $, $ F>E $. Since (i) yields a solution to (ii) by interchanging the values of $ A $, $ C $, and $ E $ with $ B $, $ D $, and $ F $, respectively.
So, we restrict to (i) and double our counts. 
	
Notice that if $ A,C,E $ are distinct from each other, then $ B,D,F $ are also distinct from each other.
Thus, we will first count how many ways we can have $ A,C,E $ distinct from each other while $ A>B $. Now, if $ s = A-B$, then there are $\ell-s $ ways to have $ A>B $.
This leaves $\ell-s-1 $ options for $ C $ and $\ell-s-2 $ options for $ E $.
Thus, if we allow $ A,B,C,D,E,F \in [\ell] $ to not necessarily be distinct the number of ways (i) can occur is
\begin{equation}\label{pine}
\sum_{s=1}^{\ell-3} (\ell-s)(\ell-s-1)(\ell-s-2) = 3! \sum_{t=3}^{\ell-1} \binom{t}{3} = 3! \binom{\ell}{4} = \frac{\ell(\ell-1)(\ell-2)(\ell-3)}{4}.
\end{equation}
Since we are only counting with positive spacing $s$ between $A$ and $B$ while $ A,C,E $ are distinct from each other, then the only ways we can have $ A,B,C,D,E,F $ not all distinct are when either 
\begin{enumerate}[label=(\alph*).]
	\item $| \{A,B,D,C,E,F\} |=5 $ \, or
	\item $| \{A,B,D,C,E,F\} |=4 $.
\end{enumerate}
Now, the only way we can have (a) is if exactly one of $ |\{A\} \cap \{D,F\} | =1$, $ |\{C\} \cap \{B,F\} | =1$, or $ |\{E\} \cap \{B,D\} | =1$ occurs.
So, there are $ 6 $ different ways for (a) to occur.
Any case of (a) is an instance of  \cref{Light}. Hence, we must subtract $ 6 $ times 
\begin{align}
\begin{split}\label{spruce}
\begin{cases}
\frac{5\ell^3-32\ell^2+52\ell}{24} & \text{if $\ell \equiv 0 \pmod{6}$,} \\[4pt]
\frac{5\ell^3-32\ell^2+55\ell-28}{24} & \text{if $\ell \equiv \pm 1 \pmod{6}$,} \\[4pt]
\frac{5\ell^3-32\ell^2+52\ell-16}{24} & \text{if $\ell \equiv \pm 2 \pmod{6}$,} \\[4pt]
\frac{5\ell^3-32\ell^2+55\ell-12}{24} & \text{if $\ell \equiv 3 \pmod{6}$}
\end{cases}
\end{split}
\end{align}
from \eqref{pine}.
Next, the only way we can have (b) is if exactly one of $ (A,B) \in \{(D,E), (F,C)\}$, $(C,D) \in \{(B,E),(F,A) \}$, or $(E,F) \in \{(D,A),(B,C)\}$ occurs.
Hence, there are $ 6 $ different ways for (b) to occur.
Any case of (b) makes $\{A,B,C,D,E,F\}$ a $4$-term arithmetic progression, so it falls under \cref{Fuzz} with $ k=3 $. Thus, we must also subtract $ 6 $ times 
\begin{align}
\begin{split}\label{sequoia}
\begin{cases}
\frac{\ell^2-3\ell}{6} & \text{if $\ell \equiv 0 \pmod{3}$,} \\[4pt]
\frac{\ell^2 -3\ell + 2}{6} & \text{if $\ell \equiv \pm 1 \pmod{3}$}
\end{cases}
\end{split}
\end{align}
from \eqref{pine}.

When we have subtracted $6$ times \eqref{spruce} and $6$ times \eqref{sequoia} from \eqref{pine}, we get the count for case (i).  Doubling this to get the full count gives
\begin{align*}
|\As(\cP,=,\ell)|=
\begin{cases}
\frac{\ell^4-11\ell^3+39\ell^2-46\ell}{2} & \text{if $\ell \equiv 0 \pmod{6}$,} \\[4pt]
\frac{\ell^4-11\ell^3+39\ell^2-49\ell+20}{2} & \text{if $\ell \equiv \pm 1 \pmod{6}$,} \\[4pt]
\frac{\ell^4-11\ell^3+39\ell^2-46\ell+8}{2} & \text{if $\ell \equiv \pm 2 \pmod{6}$,} \\[4pt]
\frac{\ell^4-11\ell^3+39\ell^2-49\ell+12}{2} & \text{if $\ell \equiv 3 \pmod{6}$.} \hfill \quad 
\end{cases}
\end{align*}
\par \vspace{-1.7\baselineskip}
\qedhere
\end{proof}
Now that we know that the partitions in our putative isomorphism classes are satisfiable, it is not hard to show that these classes do indeed constitute $\Isom(3)$.
\begin{lemma}\label{Julianna}
The set $\Con(3)=\bigsqcup_{m=1}^8 \fC_m$, and for each $m \in [8]$, the set $\fC_i$ is a $\cWthree$-orbit.  That is, $\Isom(3)=\{\fC_j: 1 \leq j \leq 8\}$.
\end{lemma}
\begin{proof}
\cref{Alexander} shows that $\Con(3) \subseteq \bigsqcup_{m=1}^8 \fC_m$.
We claim that every partition $\cP \in \bigsqcup_{m=1}^8 \fC_m$ is GELO.
First, check the definitions and note that all the classes of $\cP$ are of size $2$ or $4$, hence $\cP$ is even.
Furthermore, for every $f \in [3]$, one can always find a class of size $2$ in $\cP$ that has one element of the form $(f,*,*)$ and another element of the form $(g,*,*)$ with $g\not=f$; this makes $\cP$ locally odd.
Furthermore, Lemmas \ref{Apple}--\ref{Orlando} show that the partitions in $\bigsqcup_{m=1}^8 \fC_m$ are satisfiable, so they are in fact contributory, so $\Con(3)=\bigsqcup_{m=1}^8 \fC_m$.

Let $m \in [8]$.
Then $\fC_m$ is contained in a $\cWthree$-orbit by \cref{Elizabeth}.
If there were some partition $\cP$ isomorphic to a partition in $\fC_m$, then $\cP \in \Con(3)$ by \cref{sticks}, so $\cP \in \bigsqcup_{m=1}^8 \fC_m$.
But $\cP$ cannot be in $\fC_{m'}$ for $m'\not=m$ by \cref{Jacob}, so this forces $\cP \in \fC_m$.
Thus $\fC_m$ is a $\cWthree$-orbit.
\end{proof}

\subsection{Calculation of skewness}\label{Egbert}

Now that we know $|\fP|$ and $\Sols(\fP,\ell)$ for each class $\fP$ in $\Isom(3)$, we use \cref{Sanria} to determine the third central moment of $\ssac$.
\begin{theorem}
If $\ell \in \Z_+$  then 
\[
\mom{3}\ssac(f) = \begin{cases}
160\ell^4-1296\ell^3+3296\ell^2-2496\ell &  \text{if $\ell \equiv 0 \pmod{4}$,} \\
160\ell^4-1296\ell^3+3296\ell^2-2736\ell+576 & \text{if $\ell \equiv \pm 1 \pmod{4}$,} \\ 
160\ell^4-1296\ell^3+3296\ell^2-2496\ell-384 & \text{if $\ell \equiv 2 \pmod{4}$.}
\end{cases} 
\]
\end{theorem}
\begin{proof}
We have 
\begin{align*}
\mom{3} \ssac(f)
& = \sum_{ \fP \in \Isom(3)} |\fP| \Sols(\fP,\ell) \\
& = \sum_{i=1}^8 |\fC_i| \Sols(\fC_i,\ell) \\
& = \begin{cases}
160\ell^4-1296\ell^3+3296\ell^2-2496\ell & \text{if $\ell \equiv 0 \pmod{4}$,} \\
160\ell^4-1296\ell^3+3296\ell^2-2736\ell+576 & \text{if $\ell \equiv \pm 1 \pmod{4}$,} \\ 
160\ell^4-1296\ell^3+3296\ell^2-2496\ell-384 & \text{if $\ell \equiv 2 \pmod{4}$.}
\end{cases}
\end{align*}
where the first equality is by \cref{Sanria}, the second by \cref{Julianna}, and the third by Lemmas \ref{Genevieve} and \ref{Apple}--\ref{Orlando}.
\end{proof}
Since $\ADF(f)=-1+\ssac(f)/\ell^2$, we can divide this result by $\ell^6$ to obtain the third central moment of the demerit factor.
\begin{corollary}\label{Sarah}
If $\ell \in \Z_+$, then 
\[
\mom{3} \ADF(f) =
\begin{cases}
\frac{160\ell^4-1296\ell^3+3296\ell^2-2496\ell}{\ell^6} & \text{if $\ell \equiv 0 \pmod{4}$,} \\[4pt]
\frac{160\ell^4-1296\ell^3+3296\ell^2-2736\ell+576}{\ell^6} & \text{if $\ell \equiv \pm 1 \pmod{4}$,} \\[4pt] 
\frac{160\ell^4-1296\ell^3+3296\ell^2-2496\ell-384}{\ell^6} & \text{if $\ell \equiv 2 \pmod{4}$.}
\end{cases}
\]
\end{corollary}
We can normalize the third central moment using our variance calculation from \cref{Jessica} to obtain the skewness.
\begin{corollary}\label{Shirley}
If $\ell \in \Z_+$, then
\begin{equation*}
\smom{3} \ADF(f)=
\begin{cases}
\frac{6\sqrt{3}(10\ell^4-81 \ell^3+206 \ell^2-156 \ell)}{(4 \ell^3-15 \ell^2+14 \ell)^{3/2}} & \text{if $\ell \equiv 0 \pmod{4}$,} \\[4pt]
\frac{6\sqrt{3}(10\ell^4-81 \ell^3+206 \ell^2-171 \ell+36)}{(4 \ell^3-15 \ell^2+14 \ell -3)^{3/2}} & \text{if $\ell \equiv \pm 1 \pmod{4}$,} \\[4pt]
\frac{6\sqrt{3}(10\ell^4-81 \ell^3+206 \ell^2-156 \ell-24)}{(4 \ell^3-15 \ell^2+14 \ell)^{3/2}} & \text{if $\ell \equiv 2 \pmod{4}$.}
\end{cases}
\end{equation*}
\end{corollary}

\appendix
\section{Technical Lemmas}\label{Arthur}

This appendix contains some results in enumerative combinatorics that are useful in calculating the cardinality of $\As(\cP,=,\ell)$ for various partitions $\cP \in \Con(p)$ for $p$ a positive integer and $\ell\in\N$.
These results are used in Sections \ref{Veronica} and \ref{Simon}.

The first result, from \cite[Lemma 12]{Abrego-Fernandez-Katz-Kolesnikov}, counts the number of arithmetic progressions of a certain size within a larger arithmetic progression.
For our purposes, an $n$-term arithmetic progression is a set of the form $\{a,a+b,a+2 b,\ldots,a+(n-1)b\}$ for some integers $a,b,n$ with $n\not=0$.
\begin{lemma} \label{Fuzz}
Let $\ell \in \N$ and $k \in \Z_+$ with $\ell=q k+r$, where $q,r\in\Z$ and $0\leq r<k$.
Then exactly $(\ell-r)(\ell+r-k)/(2 k)$ subsets of $[\ell]$ are $(k+1)$-term arithmetic progressions.
\end{lemma}
The next two counting results follow from \cref{Fuzz}.
\begin{corollary}\label{Persephone}
Let $\ell \in \N$ and let $N$ be the number of $3$-term arithmetic progressions in $[\ell]$.
Then the following four numbers are all equal
\begin{enumerate}[label=(\roman*)]
\item $2 N$,
\item the number of ordered pairs $(A,B)$ where $A,B$ are distinct elements in $[\ell]$ such that $A \equiv B \pmod{2}$,
\item the number of solutions to the equation $A+B=2 C$ with $A,B,C$ distinct elements of $[\ell]$,
\item $\floor{(\ell-1)^2/2}$.
\end{enumerate}
The number of solutions to the equation $A+B=2 C$ with $A, B, C$  (not necessarily distinct) elements of $[\ell]$ is $\floor{(\ell^2+1)/2}$.
\end{corollary}

\begin{proof}
If $(A,B)$ is a pair of distinct elements of $[\ell]$, then there is some $C \in [\ell]$ such that $A+B=2 C$ if and only if $A$ and $B$ have the same parity, in which case $\{A,C,B\}$ is a $3$-term arithmetic progression.
This progression also arises if we begin with the pair $(B,A)$ (but in no other case), so the number of progressions is half the number of pairs, and one can check that our formula for the count is indeed twice what is given by the $k=2$ case of \cref{Fuzz}.

If we have a solution $(A, B, C)$ to $A+B=2 C$ where $A, B, C$ are not all distinct, then they must all be equal to each other, so there are $\ell$ such solutions with $A, B, C \in [\ell]$.  When we add this to the number of solutions with $A, B, C$ distinct we get $\floor{(\ell^2+1)/2}$.
\end{proof}

\begin{lemma}\label{Light}
For $\ell \in \N $, the number of choices of distinct $ A,B,C,D,E \in [\ell] = \{ 0,1, \dots, \ell-1\} $ that will make (i) $B-A=C-B=E-D$, (ii) $A<B<C$, and (iii) $ D<E $ is 
\begin{align*}
\begin{cases}
\frac{5\ell^3-32\ell^2+52\ell}{24} & \text{if $\ell \equiv 0 \pmod{6}$,} \\[4pt]
\frac{5\ell^3-32\ell^2+55\ell-28}{24} & \text{if $\ell \equiv \pm 1 \pmod{6}$,} \\[4pt]
\frac{5\ell^3-32\ell^2+52\ell-16}{24} & \text{if $\ell \equiv \pm 2 \pmod{6}$,} \\[4pt]
\frac{5\ell^3-32\ell^2+55\ell-12}{24} & \text{if $\ell \equiv 3 \pmod{6}$.}
\end{cases}
\end{align*}
\end{lemma}

\begin{proof}
If $A$, $B$, $C$, $D$, $E$ satisfy our conditions, then we let $s$ be the common value of $B-A$, $C-B$, and $E-D$, so $s$ is some integer with $1\leq s \leq\floor{\ell/2}$ so that $A,B,C$ are distinct and lie within $[\ell]$ (and also $D,E$ are distinct).
For a given value of $s$, there are $\ell-2 s$ choices of $A,B,C$ and $\ell-s$ choices of $D,E$, where we are not yet insisting that $\{A,B,C\} \cap \{D,E\}=\emptyset$; the number of total ways to achieve this is 
\begin{align}
\begin{split}\label{Tybalt}
\sum_{s=1}^{\floor{\ell/2}} (\ell-2s)(\ell-s) = 			
\begin{cases}
\frac{5\ell^3-12\ell^2+4\ell}{24} & \text{if $\ell$ is even,} \\[4pt]
\frac{5\ell^3-12\ell^2+7\ell}{24} & \text{if $\ell$ is odd.}
\end{cases}
\end{split}
\end{align}
Now we need to deduct the number of cases where $\{A,B,C\} \cap \{D,E\}\not=\emptyset$, which happens in four ways: (1) $A=D$ and $B=E$, (2) $B=D$ and $C=E$, (3) $A=E$, or (4) $C=D$.
In each of cases (1) and (2) $\{A,B,C,D,E\}$ is a $3$-term arithmetic progression in $[\ell]$, so by \cref{Persephone} there are $\floor{(\ell-1)^2/2} /2$ ways for this to occur.
In each of cases (3) and (4) $\{A,B,C,D,E\}$ is a $4$-term arithmetic progression in $[\ell]$, so one can use \cref{Fuzz} to see that there are $\floor{(\ell-1)(\ell-2)/3}/2$ such progressions in $[\ell]$.
Deducting these four counts from \eqref{Tybalt} yields the desired result.
\end{proof}
The following two results are easy exercises in counting.
\begin{lemma}\label{Sam}
For $\ell,r \in \N$, we have
\begin{enumerate}[label=(\roman*)]
\item\label{Wilfred} $\displaystyle\sum_{0 \leq j \leq \ell} j^2 = \frac{\ell(\ell+1)(2\ell+1)}{6}$ and
\item\label{Xavier} $\displaystyle\sums{0 \leq j \leq \ell \\ j\equiv r \!\!\pmod{2} } j^2 = \begin{cases}
\frac{\ell(\ell+1)(\ell+2)}{6} & \text{if $\ell \equiv r \pmod{2}$,} \\[4pt]
\frac{(\ell-1)\ell(\ell+1)}{6} & \text{if $\ell \not\equiv r \pmod{2}$.}
\end{cases}$
\end{enumerate}
\end{lemma}

\begin{lemma}\label{wow} 
Let $k,\ell \in \N$.  Then the number of ordered pairs $(A,B)$ of (not necessarily distinct) elements in $[\ell]$ with $A+B=k$ is
\begin{align*}
\sums{A, B \in [\ell] \\ A+B = k } 1 =
\begin{cases}
k+1 & \text{if $0 \leq k \leq \ell-1$,} \\
2(\ell-1)-k+1 & \text{if $\ell-1 \leq k \leq 2(\ell-1)$,} \\
0 & \text{otherwise.}
\end{cases}
\end{align*}
\end{lemma}

This leads to our final result.
\begin{lemma}\label{wowzers}
Let $\ell \in \N $.
\begin{enumerate}[label=(\roman*)]
\item Then the number $(A,B,C,D) \in [\ell]^4$ with $A+B=C+D$ is $(2 \ell^3+\ell)/3$.
\item The number of $(A,B,C,D) \in [\ell]^4$ with $A$, $B$, $C$, and $D$ distinct and $A+B=C+D$ is
\[
\begin{cases}
\frac{2 \ell^3-9 \ell^2+10\ell}{3} & \text{if $\ell$ is even,} \\[4pt]
\frac{2 \ell^3-9 \ell^2+10\ell-3}{3} & \text{if $\ell$ is odd.}
\end{cases}
\]
\item The number of $(A,B,C,D) \in [\ell]^4$ where $A+B=C+D$ and $A \equiv B \pmod{2}$ is
\[
\begin{cases}
\frac{\ell^3-\ell}{3} & \text{if $\ell$ is even,} \\[4pt]
\frac{\ell^3+2\ell}{3} & \text{if $\ell$ is odd.}
\end{cases}
\]
\item The number of $(A,B,C,D) \in [\ell]^4$ with $A$, $B$, $C$, and $D$ distinct and $A+B=C+D$ and $A \equiv B \pmod{2}$ is
\[
\begin{cases}
\frac{\ell^3-6\ell^2+8\ell}{3} &  \text{if $\ell$ is even,} \\[4pt]
\frac{\ell^3-6\ell^2+11\ell -6}{3} & \text{if $\ell$ is odd.}
\end{cases}
\]
\end{enumerate}
\end{lemma}
\begin{proof}
The first count equals \[\sum_{0 \leq k \leq 2\ell-2} \left|\left\{(A,B) \in [\ell]^2:  A+B=k\right\}\right| \left|\left\{(C,D) \in [\ell]^2:  C+D=k\right\}\right|,\]
and \cref{wow} gives us the cardinalities of the sets, so we have a sum of squares $1^2+2^2+\cdots+(\ell-1)^2+\ell^2+(\ell-1)^2+\cdots+2^2+1^2$, which we total up using \cref{Sam}\ref{Wilfred} to get the final result.

To get the second count, we deduct solutions in which $A$, $B$, $C$, and $D$ are not distinct, which fall into five cases:
\begin{itemize}
\item The number of solutions when $A=B$ and $A$, $C$, and $D$ are distinct (or when $C=D$ and $A$, $B$, and $C$ are distinct) is given by \cref{Persephone} as $\floor{(\ell-1)^2/2}$.
\item The number of solutions when $A=C\not=B=D$ (or when $A=D\not=B=C$) is clearly $\ell(\ell-1)$.
\item The number of solutions when $A=B=C=D$ is clearly $\ell$.
\end{itemize}

The third count equals \[\sums{0 \leq k \leq 2\ell-2 \\ k \equiv 0 \!\!\!\!\pmod{2}} \left|\left\{(A,B) \in [\ell]^2:  A+B=k\right\}\right| \left|\left\{(C,D) \in [\ell]^2:  C+D=k\right\}\right|,\]
and \cref{wow} gives us the cardinalities of the sets, so we have a sum of squares $1^2+3^2+\cdots+(\ell-1)^2+(\ell-1)^2+\cdots+3^2+1^2$ (if $\ell$ is even) or $1^2+3^2+\cdots+(\ell-2)^2+\ell^2+(\ell-2)^2+\cdots+3^2+1^2$ (if $\ell$ is odd) which we total up using \cref{Sam}\ref{Xavier} to get the final result.

To get the fourth count, we deduct solutions in which $A$, $B$, $C$, and $D$ are not distinct, which fall into five cases:
\begin{itemize}
\item The number of solutions when $A=B$ and $A$, $C$, and $D$ are distinct (or when $C=D$ and $A$, $B$, and $C$ are distinct) is given by \cref{Persephone} as $\floor{(\ell-1)^2/2}$.
\item The number of solutions when $A \equiv B \pmod{2}$ and $A=C\not=B=D$ (or when $A\equiv B \pmod{2}$ and $A=D\not=B=C$) is $\floor{(\ell-1)^2/2}$ by \cref{Persephone}.
\item The number of solutions when $A=B=C=D$ is clearly $\ell$. \qedhere
\end{itemize}
\end{proof}

\section*{Acknowledgement}

The authors thank Bernardo \'Abrego and Silvia Fern\'andez-Merchant for helpful discussions and suggestions.

\begin{thebibliography}{AFMKK16}

\bibitem[AFMKK16]{Abrego-Fernandez-Katz-Kolesnikov}
Bernardo~M. \'Abrego, Silvia Fern\'andez-Merchant, Daniel~J. Katz, and Levon
  Kolesnikov.
\newblock On the number of similar instances of a pattern in a finite set.
\newblock {\em Electron. J. Combin.}, 23(4):Paper 4.39, 24, 2016.

\bibitem[ALS04]{Aupetit}
S.~Aupetit, P.~Liardet, and M.~Slimane.
\newblock Evolutionary search for binary strings with low aperiodic
  auto-correlations.
\newblock In P.~Liardet, P.~Collet, C.~Fonlupt, E.~Lutton, and M.~Schoenauer,
  editors, {\em Artificial Evolution}, volume 2936 of {\em Lecture Notes in
  Computer Science}, pages 39--50, 2004.

\bibitem[BL01]{Borwein-Lockhart}
Peter Borwein and Richard Lockhart.
\newblock The expected {$L_p$} norm of random polynomials.
\newblock {\em Proc. Amer. Math. Soc.}, 129(5):1463--1472, 2001.

\bibitem[GG05]{Golomb-Gong}
Solomon~W. Golomb and Guang Gong.
\newblock {\em Signal design for good correlation}.
\newblock Cambridge University Press, Cambridge, 2005.

\bibitem[Gol67]{Golomb}
Solomon~W. Golomb.
\newblock {\em Shift register sequences}.
\newblock With portions co-authored by Lloyd R. Welch, Richard M. Goldstein,
  and Alfred W. Hales. Holden-Day, Inc., San Francisco,
  Calif.-Cambridge-Amsterdam, 1967.

\bibitem[Gol72]{Golay-72}
Marcel J.~E. Golay.
\newblock {A class of finite binary sequences with alternate autocorrelation
  values equal to zero.}
\newblock {\em {IEEE Trans. Inform. Theory}}, 18:449--450, 1972.

\bibitem[Gol75]{Golay-75}
Marcel J.~E. Golay.
\newblock {Hybrid low autocorrelation sequences.}
\newblock {\em {IEEE Trans. Inform. Theory}}, 21:460--462, 1975.

\bibitem[Jed19]{Jedwab}
Jonathan Jedwab.
\newblock The distribution of the {$L_4$} norm of {L}ittlewood polynomials.
\newblock arXiv:1911.11246, 2019.

\bibitem[Kat16]{Katz16}
Daniel~J. Katz.
\newblock Aperiodic crosscorrelation of sequences derived from characters.
\newblock {\em IEEE Trans. Inform. Theory}, 62(9):5237--5259, 2016.

\bibitem[Sar84]{Sarwate}
D.~V. Sarwate.
\newblock Mean-square correlation of shift-register sequences.
\newblock {\em Communications, Radar and Signal Processing, IEE Proceedings F},
  131(2):101--106, 1984.

\bibitem[Sch06]{Schroeder}
M.~R. Schroeder.
\newblock {\em Number theory in science and communication}, volume~7 of {\em
  Springer Series in Information Sciences}.
\newblock Springer-Verlag, Berlin, fourth edition, 2006.

\end{thebibliography}

\end{document}
