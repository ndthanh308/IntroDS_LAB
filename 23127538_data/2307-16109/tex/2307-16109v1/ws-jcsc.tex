%%%%%%%%%%%%%%%%%%%%%%%%%%%%%%%%%%%%%%%%%%%%%%%%%%%%%%%%%%%%%%%%%%%%%%%%%%%%
%% Trim Size: 9.75in x 6.5in
%% Text Area: 8in (include Runningheads) x 5in
%% ws-jcsc.tex   :	1-8-2014
%% Tex file to use with ws-jcsc.cls written in Latex2E.
%% The content, structure, format and layout of this style file is the
%% property of World Scientific Publishing Co. Pte. Ltd.
%% Copyright 2014 by World Scientific Publishing Co.
%% All rights are reserved.
%%%%%%%%%%%%%%%%%%%%%%%%%%%%%%%%%%%%%%%%%%%%%%%%%%%%%%%%%%%%%%%%%%%%%%%%%%%%
%%

%\documentclass[wsdraft]{ws-jcsc}
\documentclass{ws-jcsc}

\usepackage{amsmath}
\usepackage{amssymb}
\usepackage{amsfonts}
\usepackage{color}
\usepackage{graphicx}
\usepackage{algorithm}
\usepackage{algorithmic}
\usepackage{enumerate}
%\usepackage{epstopdf}
\usepackage{multirow}
\usepackage{lipsum,mwe,cuted}
\usepackage{diagbox}
\usepackage{graphicx}
\usepackage{float}
\usepackage{subfig}

\begin{document}


\markboth{Authors' Names}{Instructions for Typesetting Manuscripts (Condensed Title for the Paper)}

%%%%%%%%%%%%%%%%%%%%% Publisher's Area please ignore %%%%%%%%%%%%%%%
%
\catchline{}{}{}{}{}
%
%%%%%%%%%%%%%%%%%%%%%%%%%%%%%%%%%%%%%%%%%%%%%%%%%%%%%%%%%%%%%%%%%%%%

\title{A Message Passing Detection based Affine Frequency Division Multiplexing Communication System
\footnote{This work was supported by the National Natural Science Foundation of China (NSFC) (grant number 61871097) and the Sichuan Science and Technology Program (grant number 2023YFG0298).}}

\author{Lifan Wu}  

\address{School of Aeronautics and Astronautics,\\
University of Electronic Science and Technology of China, Chengdu, China\\
lifan\_w@163.com}

\author{Shan Luo\footnote{Corresponding author.}}

\address{School of Aeronautics and Astronautics,\\
University of Electronic Science and Technology of China, Chengdu, China\\
luoshan@uestc.edu.cn}

\author{Dongxiao Song}

\address{School of Aeronautics and Astronautics,\\
University of Electronic Science and Technology of China, Chengdu, China\\
songdongxiao0813@foxmail.com}

\author{Fan Yang}

\address{School of Aeronautics and Astronautics,\\
University of Electronic Science and Technology of China, Chengdu, China\\
yangfan000209@163.com}

\author{Rongping Lin}

\address{School of Information and Communication Engineering,\\
University of Electronic Science and Technology of China, Chengdu, China\\
linrp@uestc.edu.cn}

\maketitle

\begin{history}
\received{(Day Month Year)}
\revised{(Day Month Year)}
\accepted{(Day Month Year)}
%\comby{(xxxxxxxxxx)}
\end{history}

\begin{abstract}
The new generation of wireless communication technology is expected to solve the reliability problem of communication in high-speed mobile communication scenarios. An orthogonal time frequency space (OTFS) system has been proposed and can effectively solve this problem. However, the pilot overhead and multiuser multiplexing overhead of the OTFS are relatively high. Therefore, a new modulation technology based on the discrete affine Fourier transform was proposed recently to address the above issues in OTFS, referred to the affine frequency division multiplexing (AFDM). The AFDM attains full diversity due to parameter adjustment according to the delay-Doppler profile of the channel and can achieve performance similar to the OTFS. Due to the limited research on the detection of AFDM currently, we propose a low-complexity yet efficient message passing (MP) algorithm for joint interference cancellation and detection, which takes advantage of the inherent channel sparsity. According to simulation results, the MP detection performs better than the minimum mean square error and maximal ratio combining detection.
\end{abstract}

\keywords{Orthogonal time frequency space; message passing; affine frequency division multiplexing; affine Fourier transform; wireless communication detection.}

%\ccode{1991 Mathematics Subject Classification: 22E46, 53C35, 57S20}

\section{Introduction}	
The 6G mobile communication system is expected to tackle reliable communication challenges in very high-speed mobile scenarios, such as high-speed rail, unmanned aerial vehicles (UAVs), and vehicle-to-vehicle communications\cite{white1,white2}. However, the Doppler shift can be substantial in high-speed mobile scenarios, resulting in multipath fading. The prevalent Doppler effect, especially on the orthogonal frequency division multiplexing (OFDM), induced by mobile vehicles or time-varying medium, can cause considerable distortion and degrade demodulation performance\cite{OFDM}.
\par To address the impact of Doppler shift on communication, several multicarrier modulation techniques have been proposed, one of which is the orthogonal chirp division multiplexing (OCDM)\cite{OCDM}, which is based on the discrete Fresnel transform - a special case of discrete affine Fourier transform (DAFT), is shown to outperform OFDM in time-dispersive channels thanks to a higher diversity order. However, OCDM cannot achieve full diversity in general linear time-varying (LTV) channels, since its diversity order depends on the delay-Doppler (DD) profile of the channel. Orthogonal time frequency space (OTFS)\cite{OTFS} is a two-dimensional (2D) modulation technique that transforms information from the DD coordinate system to the familiar time-frequency domain, commonly used in traditional modulation schemes like the OFDM, CDMA, and TDMA\cite{anmt}. By employing full diversity over time and frequency, the OTFS and equalization mitigate the effects of fading and time-varying wireless channels experienced by modulated signals such as OFDM. This transformation results in a time-independent channel with a complex channel gain that remains approximately constant for all symbols\cite{OTFS}.
\par Researchers have shown keen interest in the OTFS and explored various aspects of its performance. Some of the topics studied include coded OTFS\cite{code1,code2,code3,code4}, peak-to-average power ratio (PAPR)\cite{PAPR1,PAPR2,PAPR3}, schemes for detection\cite{detection1,detection2,detection3} and channel estimation\cite{channel1,channel2,channel3}, etc. The results indicate that the OTFS is a suitable modulation scheme for high-mobility use cases where significant Doppler shifts are encountered\cite{anms}.
\par However, the 2D transform in OTFS has certain drawbacks, such as increased pilot overhead and multiuser multiplexing overhead\cite{epce}. To overcome these limitations, Ali Bemani et al. proposed affine frequency division multiplexing (AFDM) in 2021 as an alternative. The AFDM is a novel modulation scheme based on the traditional OFDM but with significant improvements to enhance the performance of wireless communication systems. The AFDM achieves full diversity by adapting parameters based on the channel's DD profile, leading to performance comparable to the OTFS\cite{AFDM1,AFDM2}. In the AFDM, symbols are multiplexed on a set of orthogonal chirps that adapt to the doubly dispersive channel characteristics, resulting in a full and sparse DD representation of the channel in the DAFT domain\cite{amab,cdf}. Research results presented in\cite{AFDM1,AFDM2}, \cite{AFDM2} demonstrate that the AFDM exhibits similar outstanding performance to the OTFS but with lower complexity and advantages in terms of reduced channel estimation overhead.
\par Currently, research on AFDM is rapidly expanding, and the published work primarily focuses on the following areas. Pilot-aided channel estimation has been proposed in\cite{AFDMce}, aiming to accurately estimate the channel characteristics in AFDM systems. Low complexity equalization techniques have been proposed in\cite{AFDMeq}, addressing the challenges of efficient equalization in AFDM communications. And an AFDM-based integrated sensing and communications approach has been proposed in\cite{AFDMsc}, exploring the potential of AFDM in combined sensing and communication applications. However, it is noted that research on the detection method for AFDM is relatively limited. Further exploration and development in this area could contribute to advancing the understanding and practical implementation of AFDM in various communication scenarios.
\par Thus, we focus on the detection method for AFDM in this paper. We propose a low-complexity yet efficient message passing (MP) algorithm that combines interference cancellation (IC) and detection, capitalizing on the inherent channel sparsity. The MP algorithm utilizes a sparse factor graph and employs Gaussian approximation of interference terms to reduce complexity further. This approach is inspired by a similar method used in\cite{lcd}, which was applied to massive MIMO without considering the advantage of channel sparsity. In the MP algorithm, inter-carrier interference (ICI) and inter-symbol interference (ISI) can be eliminated by suitable phase shifting. In contrast, inter-Doppler interference (IDI) can be mitigated by adapting the MP algorithm to focus only on the most significant interference terms. As a result, the proposed MP algorithm effectively compensates for a wide range of channel Doppler spreads\cite{mpa}. By simulations, we observed that the performance of AFDM using the proposed MP algorithm is comparable to that of OTFS. This indicates the potential of AFDM as an efficient modulation scheme for high-mobility scenarios with significant Doppler shifts.
\par The remaining sections of this paper are organized as follows. In Section II, we provide an introduction to AFT, DAFT, and AFDM. The proposed MP algorithm is presented in Section III. In Section IV, simulation results demonstrate the performance of AFDM. Finally, Section V concludes this paper.
%
\section{RELATED WORK}
This section reviews two fundamental concepts of AFT and DAFT, which serve as the basis for AFDM. We also present related concepts and key aspects of the AFDM.
\subsection{Affine Frequency Transform}
%
\subsubsection{Continuous Affine Frequency Transform}
%
The AFT, also known as the linear canonical transform\cite{lct}, is a linear integral transform with four parameters and serves as the basis of AFDM. The AFT is mathematically defined as follows.
%
\begin{equation}
  \setcounter{equation}{1}
  S_{a,b,c,d}\left( u \right) =\begin{cases}
    \int_{-\infty}^{+\infty}{s\left( t \right)}K_{a,b,c,d}\left( t,u \right) \mathrm{d}t&		,b\ne 0\\
    s\left( du \right) \frac{e^{-j\frac{cd}{2}u^2}}{\sqrt{a}}&		,b=0\\
  \end{cases}
\end{equation}
where$(a,b,c,d)$ constitutes a matrix
$\mathbf{M}=\left[ \begin{matrix}
	a&		b\\
	c&		d\\
\end{matrix} \right] $
with a unit determinant, i.e $ad-bc=1$, and the kernel of this transformation is
\begin{equation}
  \setcounter{equation}{2}
  K_{a,b,c,d}\left( t,u \right) =\frac{1}{\sqrt{2\pi \left| b \right|}}e^{-j\left( \frac{a}{2b}u^2+\frac{1}{b}ut+\frac{d}{2b}t^2 \right)}
\end{equation}
Inverse AFT can be represented by the AFT with parameter
$\mathbf{M}^{-1}=\left[ \begin{matrix}
	d&		-b\\
	-c&		a\\
\end{matrix} \right] $
as follows
\begin{equation}
  \setcounter{equation}{3}
  s\left( t \right) =\int_{-\infty}^{+\infty}{S_{a,b,c,d}\left( u \right) K_{a,b,c,d}^{*}\left( t,u \right) \mathrm{d}u}
\end{equation}
\par The AFT generalizes several well-known mathematical transforms, including the Fourier transform $(0,1/2\pi,-2\pi,0)$, Laplace transform $(0,j(1/2\pi),j2\pi,0)$ and $\theta$ Fractional Fourier transform $(\cos \theta,(1/2\pi) \sin \theta,-2\pi \sin \theta,\cos \theta)$. The additional degrees of freedom in AFT provide flexibility and have found applications in various fields, such as filter design, time-frequency analysis, phase retrieval, and multiplexing in communications.
\par The versatility of AFT allows researchers and engineers to adapt it to specific use cases and tailor it to address various challenges in signal processing and communication systems. This flexibility has contributed to the development of novel modulation techniques like the AFDM, which exploits AFT's properties to overcome the challenges posed by high-mobility scenarios with significant Doppler shifts.
\vspace{1mm}
\subsubsection{Discrete Affine Frequency Transform}
%
The DAFT is a discrete counterpart of AFT, enabling the computation of continuous transformations in spectral analysis or processing of discrete data signals. In the spectral analysis scenario, the continuous function is sampled to provide the input for the discrete transformation. On the other hand, in the discrete data processing case, the input is a purely discrete sequence.
\par The input signal $s(t)$ and its corresponding AFT $S_{a,b,c,d}(u)$ are sampled at intervals $\Delta t$ and $\Delta u$, respectively, to facilitate the discrete transformation. This discrete form of AFT, i.e., DAFT, is useful for various practical applications where the input data is either continuous and sampled or purely discrete in nature. Sample input functions $s(t)$ and $S_{a,b,c,d}(u)$ at intervals $\Delta t$ and $\Delta u$ to obtain
\begin{equation}
  \setcounter{equation}{4}
  s_n=s\left( n\Delta t \right) ,S_m=S_{a,b,c,d}\left( m\Delta u \right)
\end{equation}
where $n=0,\dots,N-1$ and $m=0,\dots,M-1$. According to $(4)$, we can convert $(1)$ into
\begin{equation}
  \setcounter{equation}{5}
  S_m=\frac{1}{\sqrt{2\pi \left| b \right|}}\cdot \Delta t\cdot e^{-j\left( \frac{a}{2b}m^2\Delta u^2 \right)}\sum_{n=0}^{N-1}{e^{-j\left( \frac{1}{b}mn\Delta u\Delta t+\frac{d}{2b}n^2\Delta t^2 \right)}s_n}
\end{equation}
This equation can be written in the form of a transformation matrix as follows
\begin{equation}
  \setcounter{equation}{6}
  S_m=\sum_{n=0}^{N-1}{F_{a,b,c,d}\left( m,n \right) s_n}
\end{equation}
where
$$
F_{a,b,c,d}\left( m,n \right) =\frac{1}{\sqrt{2\pi \left| b \right|}}\cdot \Delta t\cdot e^{-j\left( \frac{a}{2b}m^2\Delta u^2+\frac{1}{b}mn\Delta u\Delta t+\frac{d}{2b}n^2\Delta t^2 \right)}
$$
\par In order to make $(6)$ reversible, the following conditions should hold
\begin{equation}
  \setcounter{equation}{7}
  \Delta t\Delta u=\frac{2\pi \left| b \right|}{M}
\end{equation}
Therefore, the first type of DAFT can be written as follows
\begin{equation}
  \setcounter{equation}{8}
  S_m=\frac{1}{\sqrt{M}}e^{-j\frac{a}{2b}m^2\Delta u^2}\sum_{n=0}^{N-1}{e^{-j\left( \text{sgn}(b) \frac{2\pi}{M}mn+\frac{d}{2b}n^2\Delta t^2 \right)}s_n}
\end{equation}
\par The second type of DAFT is obtained by defining $c_1=\frac{d}{4\pi b}\Delta t^2$ and $c_2=\frac{a}{4\pi b}\Delta u^2$, therefore, $S_m$ in (6) is written as $S_m=\sum_{n=0}^{N-1}{F_{c_1,c_2}\left( n,m \right) s_n}$, where
\begin{equation}
  \setcounter{equation}{9}
  F_{c_1,c_2}\left( n,m \right) \triangleq \frac{1}{\sqrt{M}}e^{-j2\pi \left( c_2m^2+\frac{\text{sgn} \left( b \right)}{M}mn+c_1n^2 \right)}
\end{equation}
\par The condition in $(7)$ becomes $c_1c_2=\frac{ad}{4M^2}$. Since $a$ and $d$ can take any real number, as long as $b$ and $c$ are adjusted to satisfy $ad-bc=1$, $c_1$ and $c_2$ have no constraints and can take any real number. Further simplification follows from fixing $\text{sgn} (b)=1$, i.e., the DAFT is defined as
\begin{equation}
  \setcounter{equation}{10}
  S_m=\frac{1}{\sqrt{M}}e^{-j2\pi c_2m^2}\sum_{n=0}^{N-1}{e^{-j2\pi \left( \frac{1}{M}mn+c_1n^2 \right)}s_n}
\end{equation}
where $M\ge N$ and its inverse transformation is as follows
\begin{equation}
  \setcounter{equation}{11}
  s_n=\frac{1}{\sqrt{M}}e^{j2\pi c_1n^2}\sum_{n=0}^{M-1}{e^{j2\pi \left( \frac{1}{M}mn+c_2m^2 \right)}S_m}
\end{equation}
From $(10)$ and $(11)$, it can be seen that the periodicity is as follows
\begin{equation}
  \setcounter{equation}{12}
  S_{m+kM}=e^{-j2\pi c_2\left( k^2M^2+2kMm \right)}S_m
\end{equation}
\begin{equation}
  \setcounter{equation}{13}
  s_{n+kN}=e^{j2\pi c_1\left( k^2N^2+2kNn \right)}s_n
\end{equation}
\par For our purposes, only constraint $(13)$ is significant, as its sole practical effect is on the type of prefix that should be added to a DAFT-based multicarrier symbol. When $M=N$, the inverse transform is the same as the forward transform with parameters $-c_1$ and $-c_2$ and conjugating the Fourier transform term.
Assuming there are signals $\mathbf{s}=\left( s_0,s_1,...,s_{N-1} \right)$ and $\mathbf{S}=\left( S_0,S_1,...,S_{N-1} \right)$, DAFT is represented in matrix form as
\begin{equation}
  \setcounter{equation}{14}
  \mathbf{S}=\mathbf{As}
\end{equation}
where $\mathbf{A}=\mathbf{\Lambda}_{c_2}\mathbf{F}\mathbf{\Lambda}_{c_1}$, $\mathbf{F}$ being the DFT matrix with entries $e^{-j2\pi mn/N}/\sqrt{N}$ and
\begin{equation}
  \setcounter{equation}{15}
  \mathbf{\Lambda}_c=\mathrm{diag}\left( e^{-j2\pi cn^2},n=0,1,\dots,N-1 \right)
\end{equation}
The inverse of the matrix $\mathbf{A}$ is given by $\mathbf{A}^{-1}=\mathbf{A}^H=\mathbf{\Lambda}_{c_1}^{H}\mathbf{F}^H\mathbf{\Lambda }_{c_2}^{H}$.

\subsection{Affine Frequency Division Multiplexing}
%
The AFDM is a multicarrier modulation concept based on DAFT. At the transmitting end, the message signal undergoes modulation through Inverse DAFT and is transformed into a time-domain signal. Upon reception, the signal is demodulated using DAFT to restore the original message signal. The AFDM block diagram is illustrated in Fig. 1.
% Figure environment removed
%
\subsubsection{Modulation}
Assuming that the message signal $\mathbf{x}\in \mathbb{A} ^{N\times 1}$ is a vector in the discrete affine Fourier domain, where $\mathbb{A}$ represents the alphabet and its elements are numerical fields of the form $z_r+z_ij$, where $z_r$ and $z_i$ are integers. In this paper, we consider 4-QAM signals. The modulated signal can be written as follows
\begin{equation}
  \setcounter{equation}{16}
  s\left[ n \right] =\sum_{m=0}^{N-1}{x\left[ m \right] \phi _n\left( m \right) ,   n=0,\dots ,N-1}
\end{equation}
where $\phi _n\left( m \right) =\frac{1}{\sqrt{N}}\cdot e^{j2\pi \left( c_1n^2+c_2m^2+nm/N \right)}$. In matrix form, $(16)$ becomes $\mathbf{s}=\mathbf{A}^H\mathbf{x}=\mathbf{\Lambda} _{c_1}^{H}\mathbf{F}^H\mathbf{\Lambda} _{c_2}^{H}\mathbf{x}$.
\par To mitigate the effects of multipath propagation and create the illusion of a periodic domain in the channel, we prepend a chirp-periodic prefix (CPP) of length $L$ here to the time-domain transmission signal due to different signal periodicity, instead of an OFDM periodic prefix (CP). Here, $L$ is an integer greater than or equal to the maximum channel delay spread in the sample. With the periodicity defined in equation $(13)$, the prefix is:
\begin{equation}
  \setcounter{equation}{17}
  s\left[ n \right]=s\left[ N+n \right] e^{-j2\pi c_1\left( N^2+2Nn \right)},   n=-L,\cdots ,-1
\end{equation}
Note that a CPP is simply a cyclic prefix (CP) whenever $2Nc_1$ is an integer value and $N$ is even.
%
\subsubsection{Channel}
After parallel to serial conversion and channel transmission, the received signal is
\begin{equation}
  r[n] =\sum_{l=0}^\infty s[n-l]g_n(l) +w[n]
\end{equation}
where $w[n]\sim \mathcal{C} \mathcal{N} \left( 0,N_0 \right)$ is an additive Gaussian noise and
\begin{equation}
  g_n(l) =\sum_{i=1}^P h_ie^{-j2\pi f_in}\delta(l-l_i)
\end{equation}
%
is the impulse response of channel at time $n$ and delay $l$, where $P\ge1$ is the number of paths, $\delta(\cdot)$ is the Dirac delta function, and $h_i$, $f_i$ and $l_i$ are the complex gain, Doppler shift (in digital frequencies), and the integer delay associated with the $i$-th path, respectively. Substitute $(19)$ into $(18)$ to obtain
\begin{equation}
  r\left[ n \right] =\sum_{i=1}^P{h_ie^{-j2\pi f_in}s\left[ n-l_i \right] +w\left[ n \right]}
\end{equation}
%
\par We define $\nu _i\triangleq Nf_i=\alpha _i+a_i$, where $\nu_i\in \left[-\nu _{\max},\nu _{\max} \right]$ is the Doppler shift normalized with respect to the subcarrier spacing, $\alpha _i\in [-\alpha _{\max},\alpha _{\max}]$ is its integer part whereas $a_i$ is the fractional part satisfying $-\frac{1}{2}<a_i\le \frac{1}{2}$. We assume that the maximum delay of the channel satisfies $l_{\max}\triangleq \max \left( l_i \right) <N$, and that the length of CPP is greater than $l_{\max}-1$.
\par After removing the CPP, we can write $(20)$ in matrix form
\begin{equation}
  \mathbf{r}=\mathbf{Hs}+\mathbf{w}
\end{equation}
with $\mathbf{w}\sim \mathcal{C} \mathcal{N} \left( 0,N_0 \right) $, $\mathbf{H}=\sum_{i=1}^P{h_i\mathbf{\Gamma} _{\mathrm{CPP}_i}\mathbf{\Delta} _{f_i}\mathbf{\Pi} ^{l_i}}$ and $\mathbf{\Pi}$ is the forward cyclic-shift matrix
\begin{equation}
  \mathbf{\Pi} =\left[ \begin{matrix}
    0&		\cdots&		0&		1\\
    1&		\cdots&		0&		0\\
    \vdots&		\ddots&		\vdots&		\vdots\\
    0&		\cdots&		1&		0\\
  \end{matrix} \right] _{N\times N}
\end{equation}
$\mathbf{\Delta} _{f_i}\triangleq \mathrm{diag}\left( e^{-j2\pi f_in},n=0,1,\dots ,N-1 \right) $ and $\mathbf{\Gamma} _{\mathrm{CPP}_i}$ is a $N\times N$ diagonal matrix
\begin{equation}
  \mathbf{\Gamma} _{\mathrm{CPP}_i}=\mathrm{diag}( \begin{cases}
    e^{-j2\pi c_1\left( N^2-2N\left( l_i-n \right) \right)}&		,n<l_i\\
    1&		,n\ge l_i\\
  \end{cases},n=0,\dots ,N-1 )
\end{equation}
We can see from $(23)$ that whenever $2Nc_1$ is an integer and $N$ is even, $\mathbf{\Gamma} _{\mathrm{CPP}_i} = \mathbf{I}$.
%
\subsubsection{Demodulation}
At the receiver, the output signal obtained after demodulation is
\begin{equation}
  y\left[ m \right] =\sum_{n=0}^{N-1}{r\left[ n \right] \phi _{n}^{*}\left( m \right)}
\end{equation}
Represented in a matrix form, the output can be written
\begin{equation}
  \mathbf{y}=\mathbf{Ar}=\sum_{i=1}^P{h_i\mathbf{A}\mathbf{\Gamma }_{\mathrm{CPP}_i}\mathbf{\Delta} _{f_i}\mathbf{\Pi} ^{l_i}\mathbf{A}^H\mathbf{x}}+\mathbf{Aw}=\mathbf{H}_{\mathrm{eff}}\mathbf{x}+\tilde{\mathbf{w}}
\end{equation}
where $\mathbf{H}_{\mathrm{eff}}\triangleq \mathbf{AHA}^H$ and $\tilde{\mathbf{w}}=\mathbf{Aw}$. Since $\mathbf{A}$ is a unitary matrix, $\tilde{\mathbf{w}}$ and $\mathbf{w}$ have the same statistical properties.
%
\subsection{AFDM Parameters}
The performance of DAFT-based modulation schemes critically depends on the choice of parameters $c_1$ and $c_2$. In OCDM, both $c_1$ and $c_2$ are set to $1/2N$. However, in DAFT-OFDM, $c_2 = 0$ while $c_1$ is adapted to the delay-Doppler channel profile to minimize Inter-Carrier Interference (ICI) and this choice simplifies detection. In the proposed AFDM, we set $c_1$ and $c_2$ in a way that the DAFT domain impulse response constitutes a full DD representation of the channel.
\par In order for the DAFT domain impulse response to constitute a full DD representation of the channel, the unique nonzero entry in each row of $\mathbf{H}_i$ for each path $i \in \{1,\dots, P\}$ should not coincide with the position of the unique nonzero entry of the same row of $\mathbf{H}_j$ for any $j \in \{1,\dots, P\}$ such that $j \ne i$. Referring to $(34)$ shows that the location of each path depends on its DD information and AFDM parameters. For the integer Doppler shift case, $\mathrm{loc}_i$ is in the following range
\begin{equation}
  -\alpha_{\max}+2Nc_1l_i\le \mathrm{loc}_i \le \alpha_{\max} + 2Nc_1l_i
\end{equation}
Therefore, for the positions of the nonzero entries of $\mathbf{H}_i$ and $\mathbf{H}_j$ to not overlap, the intersection of the corresponding ranges of $\mathrm{loc}_i$ and $\mathrm{loc}_j$ should be empty, i.e,
\begin{equation}
  \begin{aligned}
    &\{-\alpha_{\max}+2Nc_1l_i,\dots,\alpha_{\max}+2Nc_1l_i\}\cap \\
    &\{-\alpha_{\max}+2Nc_1l_j,\dots,\alpha_{\max}+2Nc_1l_j\}=\emptyset
  \end{aligned}
\end{equation}
If two paths have the same delays $ ( l_i = l_j ) $ but different Doppler shifts, then they always occupy two distinct positions in the DAFT domain. For the paths with different delays $( l_i \ne l_j )$ assuming $l_j > l_i$, satisfying $(27)$ is equivalent to the constraint
\begin{equation}
  2Nc_1>\frac{2\alpha_{\max}}{l_j-l_i}
\end{equation}
If there is no sparsity in the time-domain impulse response of the channel, then the minimum value of $l_j - l_i$ is one and $c_1$ should satisfy
\begin{equation}
  c_1=\frac{2\alpha_{\max}+1}{2N}
\end{equation}
\par With the specified parameter setting, channel paths with different delay values or Doppler frequency shifts are separated in the DAFT domain. Moreover, the only remaining condition for the DAFT-domain impulse response to constitute a full DD representation of the channel is to ensure that the nonzero entries of any two matrices $\mathbf{H}_{i_{\min}}$ and $\mathbf{H}_{i_{\max}}$ corresponding to paths $i_{\min}$ and $i_{\max}$ with delays $l_{i_{\min}}\triangleq\min_{i=1,\dots,P}l_i$ and $l_{i_{\max}}\triangleq\max_{i=1,\dots,P}l_i$ respectively do not overlap due to the modular operation in $(34)$. This overlapping never occurs if $2\alpha_{\max}l_{\max} + 2\alpha_{\max}+l_{\max} < N$. Since the wireless channels are typically under spread, i.e., $l_{\max}\ll N$ and $\alpha_{\max}\ll N$, this condition can be satisfied even with moderate values of $N$. With this parameter setting, channel paths with different delay values or different Doppler frequency shifts get separated in the DAFT domain. Thus, we get a DD representation of the channel in the DAFT domain since the DD profile can be determined from the positions of the nonzero entries in any row of $\mathbf{H}_{\mathrm{eff}}$. This feature can neither be obtained by DAFT-OFDM (since its conceptual target is making the effective channel matrix as close to being diagonal as possible to reduce ICI), nor with OCDM (since setting $c_1 = 1/2N$, there might exist two paths $i\ne j$ such that the nonzero entries of $\mathbf{H}_i$ and $\mathbf{H}_j$ coincide under some DD profiles of the channel).
%
\section{MESSAGE PASSING ALGORITHM based AFDM Communication}
%
In this section, we first provide the input-output relation of AFDM and then propose the MP algorithm for AFDM using the input-output relation in $(25)$.
\subsection{Input-Output Relation}
From $(25)$, we can observe that the received symbols are a linear combination of the transmitted symbols. By considering the definition of $\mathbf{H}_{\mathrm{eff}}$, $(25)$ can be rewritten as follows
\begin{equation}
  \mathbf{y}=\sum_{i=1}^P{h_i\mathbf{H}_i\mathbf{x}}+\tilde{\mathbf{w}}
\end{equation}
where $\mathbf{H}_i\triangleq \mathbf{A}\mathbf{\Gamma} _{\mathrm{CPP}_i}\mathbf{\Delta} _{f_i}\mathbf{\Pi} ^{l_i}\mathbf{A}^H$. It can be seen that $H_i[p,q]$ can be written as
\begin{equation}
  H_i\left[ p,q \right] =\frac{1}{N}e^{j\frac{2\pi}{N}\left( Nc_1l_{i}^{2}-ql_i+Nc_2\left( q^2-p^2 \right) \right)}\mathcal{F} _i\left( p,q \right)
\end{equation}
where we define $\mathcal{F} _i\left( p,q \right) $ as
\begin{equation}
  \begin{aligned}
    \mathcal{F} _i\left( p,q \right) &=\sum_{n=0}^{N-1}{e^{-j\frac{2\pi}{N}\left(\left( p-q+\nu _i+2Nc_1l_i \right) n \right)}}\\
  &=\frac{e^{-j2\pi \left( p-q+\nu _i+2Nc_1l_i \right)}-1}{e^{-j\frac{2\pi}{N}\left( p-q+\nu _i+2Nc_1l_i \right)}-1}
  \end{aligned}
\end{equation}
In this paper, we only discuss the case of integer Doppler shifts. $\nu _i$ is an integer for all $i\in \left\{ 1,\dots ,P \right\} $, $(32)$ is equivalent to
\begin{equation}
  \mathcal{F} _i\left( p,q \right) =\begin{cases}
    N&		,q=\left( p+\mathrm{loc}_i \right) _N\\
    0&		,\mathrm{otherwise}\\
  \end{cases}
\end{equation}
where $\mathrm{loc}_i\triangleq \left( \alpha _i+2Nc_1l_i \right) _N$, $\left( \cdot \right)_N $ is the modulo $N$ operation and $(31)$ writes as
\begin{equation}
  H_i\left[ p,q \right] =\begin{cases}
    e^{j\frac{2\pi}{N}\left( Nc_1l_{i}^{2}-ql_i+Nc_2\left( q^2-p^2 \right) \right)}&		,q=\left( p+\mathrm{loc}_i \right) _N\\
    0&		,\mathrm{otherwise}\\
  \end{cases}
\end{equation}
Hence, there is only one nonzero element in each row of $\mathbf{H}_i$ as shown in Fig. 2. And the input-output relation for $(30)$ becomes
\begin{equation}
  \begin{aligned}
    y\left[ p \right] =\sum_{i=1}^P{h_ie^{j\frac{2\pi}{N}\left( Nc_1l_{i}^{2}-ql_i+Nc_2\left( q^2-p^2 \right) \right)}x\left[ q \right] +\tilde{w}\left[ p \right]}\\
    ,0\le p\le N-1
  \end{aligned}
\end{equation}
where $q=\left( p+\mathrm{loc}_i \right) _N$.
% Figure environment removed
%
\subsection{MP based Detection}
Consider the vectorized AFDM input-output relation in $(25)$, where $\mathbf{y}$ and $\tilde{\mathrm{w}}$ are complex vectors of dimension $N \times 1$ with elements denoted by $y[d]$ and $\tilde{w}\left[ d \right] $, $1\le d\le N$, respectively; $\mathbf{H}_{\mathrm{eff}}$ is a $N \times N$ complex matrix with elements $H_{\mathrm{eff}}[d,c]$, $1\le d,c\le N$; $\mathbf{x}$ is the information vector of dimension $N\times 1$ with elements $x\left[ c \right] \in \mathbb{A}, 1\le c\le N$. The elements of $\mathbf{y}$, $\mathbf{x}$,and $\mathbf{H}_{\mathrm{eff}}$ are determined from $(35)$ and $\tilde{\mathrm{w}}$ is the noise vector. Due to the modulo $N$ operation in $(35)$, we observe that among the $N$ elements in row $d$ of $\mathbf{H}_{\mathrm{eff}}$, it is non-zero in $(d+\mathrm{loc}_i)_N$, and among the $N$ elements in column $c$, it is non-zero in $(c-\mathrm{loc}_i)_N$, where $i$ is the $i$-th path. Let $I(d)$ and $J(c)$ denote the sets of indexes with nonzero elements in the $d$-th row and $c$-th column, respectively.
\par Based on $(25)$, we model the system as a sparsely-connected factor graph with $N$ variable nodes corresponding to $\mathbf{x}$ and $N$ observation nodes corresponding to $\mathbf{y}$. In this factor graph, each observation node $y[d]$ is connected to a set of $P$ variable nodes $\{x[c],c \in I(d)\}$. Similarly, each variable node $x[c]$ is connected to a set of $P$ observation nodes $\{y[d],d \in J(c)\}$, where $P$ is the number of paths.
\par From $(25)$, the joint maximum a posterior probability (MAP) detection rule for estimating the transmitted signals is given by
$$
\hat{\mathbf{x}}=\mathrm{arg}\underset{\mathbf{x}\in \mathbb{A} ^{N\times 1}}{\max}\mathrm{Pr}\left( \mathbf{x}\left| \mathbf{y},\mathbf{H}_{\mathrm{eff}} \right. \right)
$$
which has a complexity exponential in $N$. Since the joint MAP detection can be intractable for practical values of $N$, we consider the symbol-by-symbol MAP detection rule for $c=1,\dots ,N$
\begin{subequations}
  \begin{align}
    \hat{x}\left[ c \right] &=\mathrm{arg} \underset{a_j\in \mathbb{A}}{\max}\mathrm{Pr}\left( x\left[ c \right] =a_j|\mathbf{y},\mathbf{H}_{\mathrm{eff}} \right) \label{eq30} \notag\\
    &=\mathrm{arg} \underset{a_j\in \mathbb{A}}{\max}\frac{1}{Q}\mathrm{Pr}\left( \mathbf{y}|x\left[ c \right] =a_j,\mathbf{H}_{\mathrm{eff}} \right)  \\
    &\approx \mathrm{arg} \underset{a_j\in \mathbb{A}}{\max}\prod_{d\in J_c}{\mathrm{Pr}\left( y\left[ d \right] |x\left[ c \right] =a_j,\mathbf{H}_{\mathrm{eff}} \right)}
  \end{align}
\end{subequations}
In $(36\mathrm{a})$, we assume all the transmitted symbols $a_j \in \mathbb{A}$ are equally likely and in $(36\mathrm{b})$ we assume the components of $\mathbf{y}$ are approximately independent for a given $x[c]$, due to the sparsity of $\mathbf{H}_{\mathrm{eff}}$. That is, we assume the interference terms $\zeta^{(i)}_{d,c}$ defined in $(37)$ are independent for a given $c$. To solve the approximate symbol-by-symbol MAP detection in $(36\mathrm{b})$, we propose a MP detector that has a linear complexity in $N$. For each $y[d]$, a variable $x[c]$ is isolated from the other interference terms, which are then approximated as Gaussian noise with an easily computable mean and variance.
% Figure environment removed
\par In the MP algorithm, the mean and variance of the interference terms are used as messages from observation nodes to variable nodes. On the other hand, the message passed from a variable node $x[c]$ to the observation nodes $y[d]$, $d \in J(c)$, is the probability mass function (pmf) of the alphabet $\mathbf{p}_{c,d} = \{p_{c,d}(a_j)|a_j \in \mathbb{A}\}$. Fig. 3 shows the connections and the messages passed between the observation and variable nodes. The MP algorithm is described in Algorithm 1.
%
\begin{algorithm}
  %
  \caption{MP Algorithm for AFDM Symbol Detection}\label{alg:MP}
  %
  \begin{algorithmic}[1]
  \renewcommand{\algorithmicrequire}{\textbf{Inputs:}}
  \renewcommand{\algorithmicensure}{\textbf{Output:}}
  \REQUIRE Channel matrix $\mathbf{H}_{\mathrm{eff}}$, received signal $\mathbf{y}$
  %
  \STATE \textbf{Initialize:} $\mathrm{pmf}$ $\mathbf{p}^{(0)}_{c,d}=1/Q$,$c=0,\dots,N-1$, $d\in J(c)$
  %
  \FOR {$i=1:max\; iterations$}
  %
  \STATE Observation nodes $y[d]$ compute the means $\mu_{d,c}^{(i)}$ and variances
  $(\sigma^{(i)}_{d,c})^2$ of Gaussian random variables $\zeta^{(i)}_{d,c}$ using $\mathbf{p}^{(i-1)}_{c,d}$ and pass them to variables nodes $x[c]$, $c \in I(d)$.
  \STATE Variable nodes $x[c]$ update $\mathbf{p}^{(i)}_{c,d}$ using $\mu_{d,c}^{(i)}$, $(\sigma^{(i)}_{d,c})^2$ and $\mathbf{p}^{(i-1)}_{c,d}$ and pass them to observation nodes $y[d]$, $d \in J (c)$.
  \STATE Compute convergence indicator $\eta^{(i)}$.
  \STATE Update the decision on the transmitted symbols $\hat{x}[c]$, $c = 0,...,N-1$ if needed.
  \IF {(Stopping criteria satisfied)}
  \STATE \textbf{EXIT}
  \ENDIF
  \ENDFOR
  \ENSURE The decision on transmitted symbols $\hat{x}[c]$.
  \end{algorithmic}
\end{algorithm}\\
The details of the steps in iteration $i$ in the MP algorithm are detailed below.
\vspace{1mm}
\par \textbf{1. Observation nodes $y[d]$ $\to$ variable nodes $x[c]$, $c \in I(d)$:}
\par Mean $\mu_{d,c}^{(i)}$ and variance $(\sigma^{(i)}_{d,c})^2$ of the interference, approximately modeled as a Gaussian random variable $\zeta^{(i)}_{d,c}$ defined as
\begin{equation}
  y\left[ d \right] =x\left[ c \right] H_{\mathrm{eff}}\left[ d,c \right] +\underset{\xi _{d,c}^{(i)}}{\underbrace{\sum_{e\in I\left( d \right) ,e\ne c}{x\left[ e \right] H_{\mathrm{eff}}\left[ d,e \right]}+\tilde{w}\left[ d \right] }}
\end{equation}
can be computed as
\begin{equation}
  \mu _{d,c}^{(i)}=\sum_{e\in I\left( d \right) ,e\ne c}{\sum_{j=1}^Q{p_{e,d}^{(i-1)}\left( a_j \right) a_j H_{\mathrm{eff}}\left[ d,e \right]}}
\end{equation}
and
\begin{equation}
  \begin{aligned}
    \left( \sigma _{d,c}^{(i)} \right) ^2=\sum_{e\in I\left( d \right) ,e\ne c}{\left( \sum_{j=1}^Q{p_{e,d}^{(i-1)}\left( a_j \right) \left| a_j \right|^2\left| H_{\mathrm{eff}}\left[ d,e \right] \right|^2} \right.} \\
    \left. -\left| \sum_{j=1}^Q{p_{e,d}^{(i-1)}\left( a_j \right) a_j H_{\mathrm{eff}}\left[ d,e \right]} \right|^2 \right) +\sigma ^2
  \end{aligned}
\end{equation}
From $(35)$, we can calculate $H_{\mathrm{eff}}[d,e]$ as follows
\begin{equation}
  H_{\mathrm{eff}}[d,e] = h_ie^{j\frac{2\pi}{N}\left(Nc_1l^2_i-ql_i+Nc_2\left(q^2-d^2\right)\right)}
\end{equation}
where $q=(d+\mathrm{loc}_i)_N$ and $i$ is the $i$-th path corresponding to $e$.
\vspace{1mm}
\par \textbf{2. Message passings from variable nodes $x[c]$ to observation nodes $y[d]$, $d\in J(c)$:}
\par The pmf vector $\mathbf{p}^{(i)}_{c,d}$ can be updated as
\begin{equation}
  p_{c,d}^{(i)}\left( a_j \right) =\Delta \cdot \tilde{p}_{c,d}^{(i)}\left( a_j \right) +\left( 1-\Delta \right) \cdot p_{c,d}^{(i-1)}\left( a_j \right) ,a_j\in \mathbb{A}
\end{equation}
where $\Delta \in (0, 1]$ is the damping factor used to improve the performance by controlling the convergence speed, and
\begin{equation}
  \begin{aligned}
    \tilde{p}_{c,d}^{(i)}\left( a_j \right) &\propto \prod_{e\in J\left( c \right) ,e\ne d}{\mathrm{Pr}\left( y\left[ e \right] |x\left[ c \right] =a_j,\mathbf{H}_{\mathrm{eff}} \right)} \\
  &=\prod_{e\in J\left( c \right) ,e\ne d}{\frac{\xi ^{(i)}\left( e,c,j \right)}{\sum_{k=1}^Q{\xi ^{(i)}\left( e,c,k \right)}}}
  \end{aligned}
\end{equation}
where $\xi ^{(i)}\left( e,c,k \right) =\exp \left( \frac{-\left| y\left[ e \right] -\mu _{e,c}^{(i)}-H_{\mathrm{eff}}\left[ e,c \right] a_k \right|^2}{\left( \sigma _{e,c}^{(i)} \right) ^2} \right) $. From $(35)$, we can calculate $H_{\mathrm{eff}}[e,c]$ as follows
\begin{equation}
  H_{\mathrm{eff}}[e,c] = h_ie^{j\frac{2\pi}{N}\left(Nc_1l^2_i-cl_i+Nc_2\left(c^2-p^2\right)\right)}
\end{equation}
where $p=(c-\mathrm{loc}_i)_N$ and $i$ is the $i$-th path corresponding to $e$.
\vspace{1mm}
\par \textbf{3. Convergence indicator: }Compute the convergence indicator $\eta^{(i)}$ as
\begin{equation}
  \eta ^{(i)}=\frac{1}{N}\sum_{c=1}^N{\mathbb{I} \left( \underset{a_j\in \mathbb{A}}{\max}p_{c}^{(i)}\left( a_j \right) \ge 1-\gamma \right)}
\end{equation}
for some small $\gamma>0$ and where
\begin{equation}
  p_{c}^{(i)}\left( a_j \right) =\prod_{e\in J\left( c \right)}{\frac{\xi ^{(i)}\left( e,c,j \right)}{\sum_{k=1}^Q{\xi ^{(i)}\left( e,c,k \right)}}}
\end{equation}
and $\mathbb{I} \left( \cdot \right)$ is an indicator function which gives a value of 1, if the expression in the argument is true, and 0 otherwise.
\vspace{1mm}
\par \textbf{4. Update decision: }If $\eta^{(i)} > \eta ^{(i-1)}$, then we update the decision of the transmitted symbol as
\begin{equation}
  \hat{x}\left[ c \right] =\mathrm{arg} \underset{a_j\in \mathbb{A}}{\max}p_{c}^{(i)}\left( a_j \right) , c=0,\cdots ,N-1
\end{equation}
%
We update the decision on the transmitted symbols only when the current iteration can provide better estimates than the previous iteration.
%
\vspace{1mm}
\par \textbf{5. Stopping criteria: }The MP algorithm stops when at least one of the following conditions is satisfied:
\par \textbf{a.}\quad $\eta^{(i)}=1$,
\par \textbf{b.}\quad $\eta ^{(i)}<\eta ^{(i^*)}-\epsilon $, where $i^*\in \{1,\dots,i-1\}$ is the iteration index for which $\eta ^{(i^*)}$ is maximum,
\par \textbf{c.}\quad the maximum number niter of iterations is reached.
\par We select $\epsilon =0.2$ to disregard small fluctuations of $\eta$. Here, the first condition occurs in the best case, where all the symbols have converged. The second condition is useful to stop the algorithm if the current iteration provides a worse decision than the one in previous iterations.
%
\section{SIMULATION RESULTS}
In this section, we evaluate the performance of AFDM based on simulation results. In all simulations, $c_1$ and $c_2$ in DAFT are set to $\left( 2\alpha _{\max}+1 \right) /N$ and 0, respectively, where $\alpha_{\max}$ is the maximum Doppler shift of the channel, and $N$ is the number of symbols in AFDM. The complex gain $h_i$ is generated as an independent complex Gaussian random variable with zero mean and $1/P$ variance. The carrier frequency is 4 GHz. The BER value is obtained using $10^6$ different channel implementations.
% Figure environment removed
\par In Fig. 4, we present the BER performance and the average number of iterations achieved using the MP algorithm with ideal pulses. We vary the damping factor $\Delta$ and consider the 4-QAM signal with SNR equal to 16 dB, 18 dB, and 20 dB, respectively. From Fig. 4(a), it is evident that when $\Delta\le 0.6$, the BER remains relatively stable, but it starts to deteriorate beyond this value. On the other hand, Fig. 4(b) demonstrates that the MP algorithm converges with the least number of iterations when $\Delta=0.7$. Based on these observations, we conclude that the optimum damping factor in terms of both performance and complexity is $\Delta=0.6$, and we select this value for our AFDM system.
\par Fig. 5 depicts two schemes for AFDM with $N=128$ and OTFS with $N_{\rm{OTFS}}=8$ and $M_{\rm{OTFS}}=16$, both utilizing 4-QAM symbols and MP detection, to assess the BER performance in 4-path and 5-path LTV channels, both with $l_{\max}=\alpha_{\max}=3$. From the results, it is evident that as the number of paths increases, the amount of information during MP detection also increases. Consequently, the BER performance is better at $P=5$ than at $P=4$. Additionally, it is noteworthy that the BER performance of AFDM and OTFS is similar, regardless of whether $P=4$ or $P=5$ is considered. This similarity in BER performance showcases the potential of AFDM as a competitive alternative to OTFS in scenarios with varying path conditions.
% Figure environment removed
\par In\cite{mmse}, low complexity Minimum Mean Square Error (MMSE) and Maximal Ratio Combining (MRC) detectors have been proposed. Fig. 6 demonstrates AFDM with $N=64$, employing 4-QAM signals, and utilizing MP detection, MMSE detection, and MRC detection to evaluate the BER performance in 4-path LTV channels with $l_{\max}=3$ and $\alpha_{\max}=3$. From the results, we observed that the performance of MRC is relatively close to that of MMSE, indicating that MRC provides a decent BER performance. However, the MP detection outperforms both MMSE and MRC, demonstrating its superiority in achieving lower BER values compared to the other two detection methods.
% Figure environment removed
%
\par Fig. 7 shows the BER performance of AFDM with various symbol numbers using MP detection in 4-path LTV channels with $l_{\max}=3$ and $\alpha_{\max}=3$. The considered AFDM systems have different symbol numbers, namely $N=32$, $N=64$, $N=128$, and $N=256$. From the observations in Fig. 7, we can conclude that as the symbol number $N$ increases, the BER of the AFDM system decreases, and the MP detection effect improves. In other words, larger $N$ values result in better BER performance and more robust detection for the AFDM system using MP detection.
% Figure environment removed
%
\section{Conclusion}
%\vspace*{-1mm}
This paper introduces a low-complexity yet efficient MP algorithm for symbol joint detection, designed particularly for large-scale AFDM systems. The MP algorithm effectively handles ISI and ICI by employing appropriate phase shifts and minimizes IDI by focusing on a small number of significant interference terms. Furthermore, the proposed MP algorithm exhibits remarkable compensation capabilities for wide-ranging channel Doppler spread. Through simulation results, we demonstrate that the BER performance of AFDM with MP detection closely matches that of OTFS modulation, while outperforming MMSE detection. Additionally, we observe that the BER performance of AFDM improves with larger symbol numbers $N$ and superior MP detection efficiency. These findings highlight the potential of our proposed MP algorithm in enhancing the performance of AFDM systems.




\begin{thebibliography}{0}
	\bibitem{white1}
	G. Wikström, P. Persson, S. Parkvall, G. Mildh, E. Dahlman, B. Balakrishnan, P. Öhlén, E. Trojer, G. Rune, J. Arkko, Z. Turányi, D. Roeland, B. Sahlin, W. John, J. Halén, and H. Björkegren. (Nov. 2020). {\em Ever-present intelligent communication}. Ericsson, White paper. Accessed: Oct. 10, 2021. [Online]. Available: https://www.ericsson.com/en/reports-and-papers/white-papers/a-research-outlook-towards-6g
	\bibitem{white2}
	Samsung. (Jul. 2020). {\em The Next Hyper-Connected experience for All}. Accessed: Oct. 10, 2021. [Online]. Available: https://cdn.codeground.org/nsr/downloads/researchareas/2\\0201201\_6G\_Vision\_web.pdf
	\bibitem{OFDM}
	B. Li, F. Tong, J.H. Li, S.Y. Zheng, ``Cross-correlation quasi-gradient Doppler estimation for underwater acoustic OFDM mobile communications,'' {\em Applied Acoustics}, Volume 190, 2022, 108640, ISSN 0003-682X, doi: 10.1016/j.apacoust.2022.108640.
  \bibitem{OCDM}
  X. Ouyang and J. Zhao, ``Orthogonal Chirp Division Multiplexing," {\em IEEE Transactions on Communications}, vol. 64, no. 9, pp. 3946-3957, Sept. 2016, doi: 10.1109/TCOMM.2016.2594792.
	\bibitem{OTFS}
	R. Hadani, S. Rakib, M. Tsatsanis, A. Monk, A. J. Goldsmith, A. F. Molisch, and R. Calderbank, ``Orthogonal time frequency space modulation,'' in {\em 2017 IEEE Wireless Communications and Networking Conference (WCNC)}. IEEE, 2017, pp. 1-6.
	\bibitem{anmt}
  A. Monk, R. Hadani, M. Tsatsanis, and S. Rakib, ``OTFS - Orthogonal time frequency space: A novel modulation technique meeting 5G high mobility and massive MIMO challenges." {\em Technical report}. Available online: https://arxiv.org/ftp/arxiv/papers/1608/1608.02993.pdf
	\bibitem{code1}
	C. Liu, S. Li, W. Yuan, X. Liu and D. W. K. Ng, ``Predictive Precoder Design for OTFS-Enabled URLLC: A Deep Learning Approach," {\em IEEE Journal on Selected Areas in Communications}, vol. 41, no. 7, pp. 2245-2260, July 2023, doi: 10.1109/JSAC.2023.3280984.
	\bibitem{code2}
	J. Park, J. -P. Hong, H. Kim and B. J. Jeong, ``Auto-Encoder Based Orthogonal Time Frequency Space Modulation and Detection With Meta-Learning," {\em IEEE Access}, vol. 11, pp. 43008-43018, 2023, doi: 10.1109/ACCESS.2023.3271993.
	\bibitem{code3}
	K. Deka, A. Thomas and S. Sharma, ``OTFS-SCMA: A Code-Domain NOMA Approach for Orthogonal Time Frequency Space Modulation," {\em IEEE Transactions on Communications}, vol. 69, no. 8, pp. 5043-5058, Aug. 2021, doi: 10.1109/TCOMM.2021.3075237.
	\bibitem{code4}
	Z. Kang, H. Zhao and H. Wang, ``An Efficient Two-Dimension OTFS-NOMA Scheme Based on Heterogenous Mobility Users Grouping," {\em 2021 IEEE 21st International Conference on Communication Technology (ICCT)}, Tianjin, China, 2021, pp. 726-730, doi: 10.1109/ICCT52962.2021.9657979.
	\bibitem{PAPR1}
	G. D. Surabhi, R. M. Augustine and A. Chockalingam, ``Peak-to-Average Power Ratio of OTFS Modulation," {\em IEEE Communications Letters}, vol. 23, no. 6, pp. 999-1002, June 2019, doi: 10.1109/LCOMM.2019.2914042.
	\bibitem{PAPR2}
	P. Wei, Y. Xiao, W. Feng, N. Ge and M. Xiao, ``Charactering the Peak-to-Average Power Ratio of OTFS Signals: A Large System Analysis," {\em IEEE Transactions on Wireless Communications}, vol. 21, no. 6, pp. 3705-3720, June 2022, doi: 10.1109/TWC.2021.3123397.
	\bibitem{PAPR3}
	S. Gao and J. Zheng, ``Peak-to-Average Power Ratio Reduction in Pilot-Embedded OTFS Modulation Through Iterative Clipping and Filtering," {\em IEEE Communications Letters}, vol. 24, no. 9, pp. 2055-2059, Sept. 2020, doi: 10.1109/LCOMM.2020.2993036.
	\bibitem{detection1}
	W. Yuan, Z. Wei, J. Yuan and D. W. K. Ng, ``A Simple Variational Bayes Detector for Orthogonal Time Frequency Space (OTFS) Modulation," {\em IEEE Transactions on Vehicular Technology}, vol. 69, no. 7, pp. 7976-7980, July 2020, doi: 10.1109/TVT.2020.2991443.
	\bibitem{detection2}
	S. Li et al., ``Hybrid MAP and PIC Detection for OTFS Modulation," {\em IEEE Transactions on Vehicular Technology}, vol. 70, no. 7, pp. 7193-7198, July 2021, doi: 10.1109/TVT.2021.3083181.
	\bibitem{detection3}
	Y. K. Enku et al., ``Two-Dimensional Convolutional Neural Network-Based Signal Detection for OTFS Systems," {\em IEEE Wireless Communications Letters}, vol. 10, no. 11, pp. 2514-2518, Nov. 2021, doi: 10.1109/LWC.2021.3106039.
	\bibitem{channel1}
	W. Yuan, S. Li, Z. Wei, J. Yuan and D. W. K. Ng, ``Data-Aided Channel Estimation for OTFS Systems With a Superimposed Pilot and Data Transmission Scheme," {\em IEEE Wireless Communications Letters}, vol. 10, no. 9, pp. 1954-1958, Sept. 2021, doi: 10.1109/LWC.2021.3088836.
	\bibitem{channel2}
	X. Wu, S. Ma and X. Yang, ``Tensor-based low-complexity channel estimation for mmWave massive MIMO-OTFS systems," {\em Journal of Communications and Information Networks}, vol. 5, no. 3, pp. 324-334, Sept. 2020, doi: 10.23919/JCIN.2020.9200896.
	\bibitem{channel3}
	O. K. Rasheed, G. D. Surabhi and A. Chockalingam, ``Sparse Delay-Doppler Channel Estimation in Rapidly Time-Varying Channels for Multiuser OTFS on the Uplink," {\em 2020 IEEE 91st Vehicular Technology Conference (VTC2020-Spring)}, Antwerp, Belgium, 2020, pp. 1-5, doi: 10.1109/VTC2020-Spring48590.2020.9128497.
	\bibitem{anms}
	M. Ramachandran, G. Surabhi, and A. Chockalingam,``OTFS: A New Modulation Scheme for High-Mobility Use Cases," {\em J. Indian Institute of Science}, vol. 100, no. 2, 2020, pp. 315-36.
	\bibitem{epce}
	P. Raviteja, K. T. Phan, and Y. Hong, ``Embedded pilot-aided channel estimation for OTFS in delay-doppler channels,'' {\em IEEE Trans. on Vehicular Technology}, vol. 68, no. 5, pp. 4906-4917, 2019.
	\bibitem{AFDM1}
	A. Bemani, N. Ksairi and M. Kountouris, ``AFDM: A Full Diversity Next Generation Waveform for High Mobility Communications," {\em 2021 IEEE International Conference on Communications Workshops (ICC Workshops)}, Montreal, QC, Canada, 2021, pp. 1-6, doi: 10.1109/ICCWorkshops50388.2021.9473655.
	\bibitem{AFDM2}
	A. Bemani, G. Cuozzo, N. Ksairi and M. Kountouris, ``Affine Frequency Division Multiplexing for Next-Generation Wireless Networks," {\em 2021 17th International Symposium on Wireless Communication Systems (ISWCS)}, Berlin, Germany, 2021, pp. 1-6, doi: 10.1109/ISWCS49558.2021.9562168.
	\bibitem{amab}
	T. Erseghe, N. Laurenti, and V. Cellini, ``A multicarrier architecture based upon the affine fourier transform," {\em IEEE Transactions on Communications}, vol. 53, no. 5, pp. 853-862, May 2005.
	\bibitem{cdf}
	S. Chang Pei and J. Jiun Ding, ``Closed-form discrete fractional and affine Fourier transforms," {\em IEEE Transactions on Signal Processing}, vol. 48, no. 5, pp. 1338-1353, May 2000.
	\bibitem{AFDMce}
	H. Yin and Y. Tang, ``Pilot Aided Channel Estimation for AFDM in Doubly Dispersive Channels," {\em 2022 IEEE/CIC International Conference on Communications in China (ICCC)}, Sanshui, Foshan, China, 2022, pp. 308-313, doi: 10.1109/ICCC55456.2022.9880774.
	\bibitem{AFDMeq}
	A. Bemani, N. Ksairi and M. Kountouris, ``Low Complexity Equalization for Afdm In Doubly Dispersive Channels," {\em ICASSP 2022 - 2022 IEEE International Conference on Acoustics}, Speech and Signal Processing (ICASSP), Singapore, Singapore, 2022, pp. 5273-5277, doi: 10.1109/ICASSP43922.2022.9746329.
	\bibitem{AFDMsc}
	Y. Ni, Z. Wang, P. Yuan and Q. Huang, ``An AFDM-Based Integrated Sensing and Communications," {\em 2022 International Symposium on Wireless Communication System\\s (ISWCS)}, Hangzhou, China, 2022, pp. 1-6, doi: 10.1109/ISWCS56560.2022.9940346.
	\bibitem{lcd}
	P. Som, T. Datta, N. Srinidhi, A. Chockalingam, and B. S. Rajan, ``Low-complexity detection in large-dimension MIMO-ISI channels using graphical models," {\em IEEE J. Sel. Topics Signal Process}, vol. 5, no. 8, pp. 1497-1511, Dec. 2011.
	\bibitem{mpa}
	P. Raviteja, K. T. Phan, Y. Hong and E. Viterbo, ``Interference Cancellation and Iterative Detection for Orthogonal Time Frequency Space Modulation," {\em IEEE Transactions on Wireless Communications}, vol. 17, no. 10, pp. 6501-6515, Oct. 2018, doi: 10.1109/TWC.2018.2860011.
	\bibitem{lct}
	J. J. Healy, M. A. Kutay, H. M. Ozaktas, and J. T. Sheridan, {\em Linear canonical transforms: Theory and applications}. Springer, 2015, vol. 198.
	\bibitem{mmse}
	A. Bemani, N. Ksairi and M. Kountouris, ``Low complexity equalization for AFDM in doubly dispersive channels," https://doi.org/10.48550/arXiv.2203.01875, 2022.
\end{thebibliography}

\end{document}

