\section{Introduction}\label{sec::intro}

\textit{Living needs} are the various needs generated by individuals in their daily routines for daily survival and well-being.
Typical living needs include necessities such as food, housing, and personal care as well as leisure activities such as entertainment, for which there exists a diverse array of life service providers. Meituan\footnote{https://www.metiuan.com} is a large platform connecting customers to life service providers, in which users can meet almost all kinds of living needs.
Unlike traditional information systems such as e-commerce websites, where users can only purchase products (\textit{i.e.}, meeting one kind of living needs), Meituan allows access to various living services, such as booking a hotel, ordering takeout
, etc. 
Moreover, an accurate understanding of users' needs can significantly improve user experience, for example, by enhancing the downstream recommendation tasks.
Generally speaking, generating a specific kind of living need is naturally accompanied by a specific time or location, such as ordering food delivery at the office.
Thus, the problem of \textit{living needs prediction} can be defined as predicting the type of needs of the target user's spatiotemporal context, as illustrated in Figure~\ref{fig::nms}.
% Figure environment removed

For this new problem from the real-world scenario, there are two closely-related research topics, demand forecasting~\cite{lasek2016restaurant,seyedan2020predictive,ren2020demand,song2019review,hernandez2014survey} and spatiotemporal activity prediction~\cite{fan2019personalized,zheng2010collaborative,bhargava2015and,kipf2016semi,li2022disenhcn}. 
Demand forecasting is dedicated to predicting the quantity of a product or service that consumers will purchase. 
However, these works focus on the problem of aggregated demand prediction with times-series modeling, ignoring the needs of the specific user, time, and location.
As for spatiotemporal activity prediction,
the existing works focus mainly on either online activities such as App usage, or offline activities such as location visitation, while individuals' living needs can be fulfilled both in-store (offline) and via delivery (online), leading to a completely new problem. 

There exist two critical challenges for living needs prediction as follows.
\begin{itemize}[leftmargin=*]
    \vspace{-0.1cm}
    \item \textbf{The impact of spatiotemporal context is complex.} As discussed above, for the same user, living needs at different locations or times are totally different. For example, the user in Figure~\ref{fig::nms} eats food delivery at noon and watches a movie at night. Additionally, the lifestyles, \textit{i.e., the spatiotemporal pattern of living needs}, are extremely various for different users, further making it more difficult to model the spatiotemporal context.
    \item \textbf{There is a significant gap between users' actual living needs and their historical records on the platform. }
    Typically, users will face various kinds of situation in their life, and will generate multiple living needs. But they may only choose to satisfy one or a few of them on the platform, leading to a significant gap between their actual living needs and their historical records. We refer to the needs that can not be observed from historical records as \textit{potential needs}. The case here is different from that in most recommender systems, where the actual interests and the historical record generally do not differ greatly for users, leading to another critical challenge.
\end{itemize}
To address these challenges, in this work, we described our deployed NEON system (short for living \textbf{NE}eds predicti\textbf{ON}) in Meituan, which includes three phases: \textbf{feature mining}, \textbf{feature fusion}, and \textbf{multitask prediction}. First, in the feature mining phase, to address the first challenge, we carefully design the spatial and temporal features for individual-level users, and to address the second challenge, we extract the behavioral-pattern features for group-level users.
Second, in the feature fusion phase, we develop a feature-fusion neural network that combines internal preferences, impact from spatiotemporal context, and group behavior patterns to generate user representations, addressing both challenges.
Last, as the complement to the main task of living needs prediction, we introduce the auxiliary task of needs-meeting way prediction to enhance the model's learning of spatiotemporal context, further addressing the first challenge. 

The proposed NEON system plays a critical role in Meituan's recommendation engine with various downstream applications, including homepage recommendation, \textit{Guess-you-like} page recommendation, and message pop-up recommendation, which requires the different-aspect ability of living needs prediction. 
After the deployment of the proposed system, we obtain stable and significant gains in three applications, providing strong real-world evidence for NEON's effectiveness from different perspectives.

The contribution of this work can be summarized as follows.
\begin{itemize}[leftmargin=*]
    \item To the best of our knowledge, we take the first step to study the problem of living needs prediction, which is a critical problem in real-world life service platforms but has not been well explored.
    \item We proposed the NEON system, which includes three phases of feature mining, feature fusion layer, and multitask prediction, which well addresses the two challenges, the complex impact of spatiotemporal context and missing behavioral data. 
    \item We deploy NEON in Meituan's recommendation engine and conduct large-scale online A/B tests on three typical downstream applications, along with extensive offline evaluations. Offline experimental results verify that NEON can accurately predict users' living needs. The downstream evaluations strongly confirm NEON's high application value, with significant performance improvement in three downstream applications, among which a representative result is a 1.886\% increase \textit{w.r.t.} CTCVR for Meituan homepage recommendation.
\end{itemize}

