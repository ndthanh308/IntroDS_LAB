\section{Problem Statement}\label{sec::profdef}
As we discussed above, in users' daily life, they generate various living needs, such as eating, accommodation, entertainment, beauty, etc. These needs can be fulfilled by life service providers in the city, which can be accessed through platforms connecting life service providers to customers. To enhance the user experience, it's crucial for these platforms to accurately predict users' needs and recommend appropriate services. This leads to the problem of living needs prediction. 

As defined in the introduction, living needs prediction aims to predict the specific living needs of a user given the spatiotemporal context in which they are located (in the following we also refer to this as given the \textit{user scene}). To clearly define the problem, with the help of experts, we divide all the living needs that users can satisfy on the platform into ten categories, shown in Table~\ref{tab::needs}. We use $\mathcal{N}$ as the symbol for the set of all living needs. In response to these needs, all the life services on Meituan are also divided into 10 categories. The problem can be defined as follows.

\begin{table}

\caption{10 types of living needs that can be satisfied in Meituan}
\vspace{-0.3cm}
\label{tab::needs}

\begin{tabular}{|c|l|}

\hline 
& \fontfamily{ppl}\selectfont Ordering food delivery, Eating in a restaurant, \\

 Living & \fontfamily{ppl}\selectfont Booking a hotel, Buying medicine, \\

 Needs & \fontfamily{ppl}\selectfont Specialty shopping online, Hair-dressing, \\

  &  \fontfamily{ppl}\selectfont Grocery shopping online, Beauty, \\

  & \fontfamily{ppl}\selectfont Tourism and Entertainment \\

\hline

\end{tabular}

\end{table}

\noindent \textbf{Input:}  A dataset $\mathcal{O}^+$ of real-world life service consumption records that reflect users' living needs. Each instance in the dataset tells the kind of life service a specific user purchases, which indicates the specific living need $n$ ($n\in \mathcal{N}$) of the user in a specific user scene $i$.

\noindent \textbf{Output:} A model to estimate the probability that a user will generate the living need $n$ in user scene $i$, formulated as $f(i,n|\mathcal{O}^{+})$. Here $f(\cdot)$ denotes the function that the model aims to learn. 