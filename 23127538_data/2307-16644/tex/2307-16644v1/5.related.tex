\section{Related Work}\label{sec::relatedwork}
As discussed above, in this work we explore the problem of living needs prediction, defined as predicting the specific living \textbf{needs} of a user given the \textbf{spatiotemporal} context. Thus, there are two closely related research topics: demand forecasting and spatiotemporal activity prediction.

\subsection{Demand Forecasting} 
Demand forecasting aims at predicting the quantity of a product or service that consumers will purchase. It helps in making informed decisions on inventory management, production scheduling, pricing strategy, etc. The problem of demand forecasting is broad and multifaceted, affecting many different industries, including restaurant~\cite{lasek2016restaurant}, manufacturing~\cite{seyedan2020predictive}, retail~\cite{ren2020demand}, tourism~\cite{song2019review}, energy~\cite{hernandez2014survey}, transportation~\cite{tskeris2011demand}, etc. 

To address the problem of demand forecasting, researchers have proposed various methods which can be broadly classified into three categories: statistical models~\cite{lasek2016restaurant,fattah2018forecasting,huber2017cluster,priyadarshi2019demand,wang2019selection}, machine learning models~\cite{reynolds2013econometric,sellers2010predicting,aishwarya2020food,ma2016demand}, and deep learning models~\cite{raza2017prediction,duncan2015probabilistic,lakshmanan2020sales,kilimci2019improved}. Statistical models, such as exponential smoothing~\cite{silva2019demand}, are well-suited for long-term demand forecasting as they are based on historical trends and patterns. However, they are not adept at handling variations or outliers in the data, making them unsuitable for volatile or short-term demand forecasting. On the other hand, machine learning models for demand forecasting, such as Random Forest based models~\cite{ramya2020advanced,abbasi2019short}, are efficient at short-term demand forecasting, but their performance drops when it comes to long-term forecasting. Deep learning models such as LSTM based models~\cite{lakshmanan2020sales} and GAN based models~\cite{husein2019generative} have the ability to capture complex patterns and dependencies in the data, making them suitable for both short-term and long-term demand forecasting. However, they require a large amount of data to work well. 

Existing demand forecasting methods can not handle the problem of living needs prediction. These methods focus on the overall demand for a particular product or service in a market, but in this work, we aim at predicting the need of a specific consumer. What's more, demand forecasting methods predict demands in several months or years, while in this work we predict a user's need at a specific time and location. 





\subsection{Spatiotemporal Activity Prediction}
 Spatiotemporal activity prediction aims to predict the activity of a user at a given time and location. Previous works have employed various methods to perform spatiotemporal activity prediction. One popular approach is to build a tensor using historical data and then conduct tensor factorization to learn intrinsic association~\cite{fan2019personalized,zheng2010collaborative,bhargava2015and}. For example, Fan \textit{et al.}~\cite{fan2019personalized} propose to integrate tensor factorization with transfer learning for online activity prediction. Additionally, WDGTC~\cite{li2020tensor} proposes a low-rank tensor decomposition and completion framework for passenger flow prediction by introducing L1-norm and Graph Laplacian penalties. Recently, researchers have introduced Graph Convolutional Networks~\cite{kipf2016semi} (GCNs) to achieve high performance in spatiotemporal activity prediction. For example, SA-GCN~\cite{yu2020semantic} develops a Graph Convolutional Network with meta path-based objective function for App-usage prediction task. Furthermore, DisenHCN~\cite{li2022disenhcn} utilizes a heterogeneous hypergraph to model fine-grained user similarities, resulting in significant performance gains.

However, existing works on spatiotemporal activity prediction focus on predicting the specific activities of people, while our work focuses on the general living needs which are the driving force behind specific consumption behaviors. What's more, these works mainly focus on either online or offline activities, but in our work, we predict living needs can be satisfied both in store (offline) and via delivery (online) by different kinds of life services, which is beyond the capabilities of existing methods. 