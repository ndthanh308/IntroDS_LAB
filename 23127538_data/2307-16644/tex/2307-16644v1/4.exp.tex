
\section{Offline Evaluation}\label{sec::experiments}
\subsection{Experimental Settings}
\subsubsection{Dataset}
We conduct an offline experiment on a real-world dataset at a billion-scale. The dataset comprises a sampling of all 2022 purchase records on the platform, based on the percentage of purchases for each type of life service. It consists of over 7 billion actual purchase records from 65 million users. The details of datasets
are provided in Appendix~\ref{sec::dataset}.
\subsubsection{Metrics} 
We design three metrics, namely Sort Accuracy (SA), Via-delivery Sort Accuracy (VDSA), and In-store Sort Accuracy (ISSA), to measure the performance of our system and baseline systems in predicting living needs. SA measures the overall accuracy of sorting living needs based on their scores. VDSA focuses on the accuracy of predicting needs that are satisfied via delivery, while ISSA focuses on the accuracy of predicting needs for in-store scenarios. Detailed definitions of these metrics are provided in Appendix~\ref{sec::metrics}.
\subsubsection{Baselines}
To illustrate the effectiveness of our system, we compare it with two baselines widely in actual production environments, including \textbf{DIN~\cite{zhou2018deep}}, \textbf{DNN~\cite{cheng2016wide}}, \textbf{DCN~\cite{wang2017deep}}, 
\textbf{ESMM~\cite{ma2018entire}},
and \textbf{MMOE~\cite{ma2018modeling}}. We will provide a detailed description of these baselines in Appendix~\ref{sec::baselines}.
\subsection{Overall Performance}
We test the performance of our proposed system and baselines on the living needs prediction task, and show the results in Table~\ref{tab::overallperformance}. We can have the following observations.
\begin{table}[]
\caption{Offline experimental performance of NEON and baselines.}
\vspace{-0.2cm}
\label{tab::overallperformance}
\begin{tabular}{cccc}
\hline
\textbf{Method} & \textbf{VDSA}   & \textbf{ISSA}   & \textbf{SA}     \\ \hline
DIN             & 0.9044          & 0.7467          & 0.8700          \\
DNN             & 0.9060          & 0.7500          & 0.8718          \\
DCN             & 0.9051 & 0.7466 & 0.8701\\
ESMM & 0.9080 & 0.7476 & 0.8708 \\
MMOE & 0.9097 & 0.7573 & 0.8757 \\
NEON      & \textbf{0.9175} & \textbf{0.8277} & \textbf{0.9070} \\ \hline
Improvement     & 0.86\%          & 9.30\%         & 3.57\%          \\ \hline
\vspace{-0.5cm}
\end{tabular}
\end{table}

\begin{table}[]
\caption{Performance of NEON with/without multitask prediction.}
\vspace{-0.2cm}
\label{tab::ablation}
\begin{tabular}{cccc}
\hline
 & \textbf{VDSA}   & \textbf{ISSA}   & \textbf{SA}     \\ \hline
w.o.           & 0.9089          & 0.8084          & 0.8968          \\
with             & 0.9175          & 0.8277          & 0.9070          \\ \hline
Improvement     & 0.95\%          & 2.39\%         & 1.14\%          \\ \hline
\vspace{-0.5cm}
\end{tabular}
\end{table}


\begin{itemize}[leftmargin=*]
\item{\textbf{Our system steadily outperforms all baselines on all metrics.}} The improvement of our system compared to the best baseline is 0.86\%, 9.30\%, and 3.57\% \textit{w.r.t.} VDSA, ISSA, and SA, respectively. The significant performance gain confirms the effectiveness of our system on the living needs prediction task. Furthermore, such a significant improvement in the ability to predict living needs will result in a huge benefit in real-world production scenarios, which will be further confirmed through online evaluation.
\item{\textbf{Our system achieves greater improvement on ISSA.}} The task of predicting users' in-store living needs is relatively difficult, since the in-store consumption data is sparser, and the relationships between spatiotemporal context and in-store needs are more complex. On this task, all methods perform the worst, and our system outperforms baselines on a large margin, with an improvement of 9.30\% \textit{w.r.t.} ISSA. This further confirms our model's ability to tackle the complex impact of spatiotemporal context and discovering potential needs. 
\end{itemize}
\subsection{Ablation Study}
As mentioned in Section~\ref{sec::Model}, we introduce the task of needs-meeting way prediction to enhance the system's learning of spatiotemporal context. To study the effectiveness of the multitask prediction design, we remove it from our system to observe the impact of the design on the system performance. Specifically, we change the system structure by removing the needs-meeting way prediction network $t^W$ and taking the sum of $z^N$ and $z^W$ as the input of the need prediction network $t^N$, and test the performance of the changed system. The results are shown in Table~\ref{tab::ablation}.
The results show that our proposed system outperforms the system without multitask prediction design. Our system performs better \textit{w.r.t.} VDSA by 1.16\%, \textit{w.r.t.} ISSA by 2.39\%, and \textit{w.r.t.} SA by 1.14\%. The significant performance improvement confirms the validity of the multitask prediction design.
\section{Online Evaluation}\label{sec::online}
% Figure environment removed
For life service platforms such as Meituan, understanding and predicting users' living needs are important in all business scenarios. In this section, we evaluate the performance of three downstream recommendation tasks when deploying NEON into Meituan’s recommender engine, and the illustration of deployment is shown in Figure~\ref{fig::odp}. Specifically, the three typical applications are homepage recommendation, \textit{Guess-you-like} recommendation, and message pop-up recommendation. The performance of NEON on these applications reflects its effectiveness on fine-grained need prediction, needs-meeting way prediction, and potential need prediction, respectively.
We elaborate on the online testing of three tasks one by one as follows.
\subsection{Homepage Recommendation}
In this section, we evaluate the performance of NEON when deployed to homepage recommendation, which reflects its ability on fine-grained living need prediction.
\subsubsection{Deployment Scenario}
Typically, users open Meituan mobile APP to meet their certain living need. In the overall recommendation list on homepage offered by Meituan, users probably only focus on the items which belong to the type of life service that can fulfill their living needs, and choose one item from the items. Whether an item is in the category of life service that the user \textit{need} is at least as important as whether the user \textit{likes} the item. To emphasize the importance of recommending \textit{needed} items, the engine follows a two-step approach to generate the final recommendation list from the pre-ranking result. For a user scene, it first decides the quotas for all the ten categories of life services and then generates the final list according to the quotas and the prediction scores of the items. The supply quotas are generated based on the scores our system outputs along with other criteria.

In short, our NEON system is used to generate the \textit{quotas} of different kinds of life services in the homepage recommendation list. 
\subsubsection{Experiment Setting}
We will first give a detailed description of how our system is used in generating the quotas for each category of life services. At first, we recall local life service items into the recall pool using various strategies, such as popularity and collaborative filtering. Then the engine outputs a preliminary recommendation list based on the recall pool, called pre-ranking list. After that, we take the Softmax normalized scores output by our system as the proportional quotas for each category of life service. We further adjust the quotas taking the proportion of each category in the pre-ranking list, supply distribution by category, and order distribution by category into account. The generated quotas are the proportion of each category in the lists received by users. The lists are generated considering both quotas of categories and prediction scores of items.

We compare the homepage recommendation performance of our whole recommendation engine with and without our system through online A/B tests. The tests last for one week and involved around 4.5 million users.

In our tests, the users are randomly divided into two buckets of similar size and assigned different methods for calculating quotas for each category. Specifically, for the first group, we use the method described previously, while for the second group, we generate quotas based only on the proportion of each category in the pre-ranking list, supply distribution by category, and order distribution by category. We maintain consistency across all other modules to ensure a fair comparison. 

For metrics, we use \textit{Click Through Rate} (CTR), \textit{Conversion Rate} (CVR), \textit{Click to Conversion Rate} (CTCVR) to measure the quality of the final recommendation list, which are widely-used measurements~\cite{ma2018entire,wen2020entire,liu2020autofis}. 
\begin{table}[]
\caption{Results of A/B tests on homepage recommendation}
\vspace{-0.2cm}
\label{tab::online1overall}
\begin{tabular}{cccc}
\hline
Metric & w/o NEON & with NEON & Improvement \\ \hline
CTR    & 3.0601$\times 10^{-2}$  & 3.0672$\times 10^{-2}$   & +0.230\%     \\
CVR    & 1.1497$\times 10^{-1}$ & 1.1687$\times 10^{-1}$  & +1.652\%     \\ 
CTCVR  & 3.5183$\times 10^{-3}$   & 3.5846$\times 10^{-3}$    & +1.886\%     \\  \hline
\end{tabular}
\vspace{-0.5cm}
\end{table}
\subsubsection{Performance} 
The results of our A/B tests are shown in Table~\ref{tab::online1overall}. From the results, we can have the following observations: 
\begin{itemize}[leftmargin=*]
\item{There is a significant improvement with respect to all metrics.} The increase \textit{w.r.t.} CTR and CVR are 0.230\% and 1.653\%, respectively, which is a notable improvement. 
\item{The increase \textit{w.r.t.} CTCVR is 1.886\%.} CTCVR indicates how likely users are to purchase the recommended items. Such improvement can result in a substantial rise in total consumption on the platform. 
\item{The rise \textit{w.r.t.} GTV-CC is 3.627\%.} The remarkable uplift demonstrates the outstanding capability of our system in predicting users' living needs that has no historical record. 
\end{itemize}

We further calculate the \textit{Kullback-Leibler divergence}~\cite{kullback1951information} (KLD) between real user order distribution by category and the average proportional allocation given by the online engine with/without our system. Lower KLD indicates the quotas match better with real user order distribution, which can be regarded as real user needs distribution. The results of different time periods throughout the day are shown in Table~\ref{tab::KLD}. The time periods are separated based on the business characteristics of the platform during each hour. Similar hours are grouped within a single period.
\begin{table}[]
\caption{KLD of different time periods between real user order distribution and quotas given with/without NEON.}
\vspace{-0.2cm}
\label{tab::KLD}
\begin{tabular}{cccc}
\hline
\multicolumn{1}{l}{\multirow{2}{*}{Time Period}} & \multicolumn{2}{l}{Kullback-Leibler Divergence} & \multicolumn{1}{l}{\multirow{2}{*}{Improvement}} \\ \cline{2-3}
\multicolumn{1}{l}{}                             & w/o NEON               & with NEON            & \multicolumn{1}{l}{}                             \\ \hline
0-4   & 0.2578 & 0.2201 & -14.62\% \\
5-8   & 0.2237 & 0.1907 & -14.75\% \\
9-10  & 0.2179 & 0.1902 & -12.71\% \\
11-12 & 0.2408 & 0.2138 & -11.21\% \\
13-16 & 0.2323 & 0.2053 & -11.62\% \\
17-19 & 0.2353 & 0.2096 & -10.92\% \\
20-24 & 0.2333 & 0.1995 & -14.49\% \\ \hline
\end{tabular}
\vspace{-0.5cm}
\end{table}

From the results, it can be seen that in all time periods, the Kullback-Leibler Divergence between the actual user order distribution by category and the average proportional allocation generated by the online engine with NEON is significantly less than that without our NEON. The percentage of decrease (improvement) is 12.90\% on average. This indicates that the quotas produced by the online engine with our system are more aligned with the actual user consumption distribution, or the real-life user living needs, compared to the ones generated without our system. This can be regarded as evidence of our model's effectiveness in addressing the intricate impact of spatiotemporal context, and further confirms our system's strong ability for predicting fine-grained user needs.
\subsection{Guess-you-like Recommendation}
This section assesses NEON's performance in guess-you-like recommendation, highlighting its ability on needs-meeting way prediction.
\subsubsection{Deployment Scenario}
Meituan designs a \textit{Guess-you-like} page, which user may be guided to during their leisure time. Typically, users browse this page to satisfy their \textit{unnecessary} living needs, such as entertainment or beauty, etc. In this page, they purchase items they need and also \textit{like}. To provide more choices for users in the categories of life services they need, the recommendation list should have more items in those categories. In this page, we assume that users are more concerned with being recommended items they like. Therefore, we do not maintain the proportion of a category in the "Guess-you-like" recommendation list just because there is a slight possibility that it is essential, as users can use modules other than Guess-you-like such as the search button to find necessary life services. To ensure enough items are in the needed category, the online engine generates new quotas for Guess-you-like page based on the quotas for homepage recommendation list and the in store/via delivery score output by our system. 

In summary, the needs-meeting way prediction results of NEON are used to further adjust the \textit{quotas} of different life services in \textit{Guess-you-like} recommendation list.
\subsubsection{Experiment Setting}
We conduct online A/B tests involving about 7 million users over a period of two weeks. We randomly divide users into two buckets, each of which has a similar amount of users, and assign them different methods of generating the quotas of categories in the Guess-you-like page. Specifically, for the first bucket we adopt the same strategy as in the homepage recommendation. For the second bucket, we calculated the score of both needs-meeting ways, and turn higher the quotas of categories whose needs-meeting way gets a higher score.
As for metrics, we use CTR, CVR, CTCVR, \textit{Negative Feedback Rate for Unique Visitor} (NFR-UV), \textit {Negative Feedback Rate for Page View} (NFR-PV) to measure the quality of the recommendation list in the Guess-you-like page. NFR-UV and NFR-PV emphasize that the recommendation list should not contain items users dislike. 
\subsubsection{Performance}
Under the aforementioned settings, we conduct extensive online A/B experiments. The results are shown in Table~\ref{tab::online2overall}. We list our observations as follows:
\begin{table}[]
\caption{Overall results of A/B tests on \textit{Guess-you-like} page recommendation.}
\vspace{-0.2cm}
\label{tab::online2overall}
\begin{tabular}{cccc}
\hline
Metric & w/o NEON & with NEON & Improvement \\ \hline
CTR    & 9.5707$\times 10^{-2}$  & 9.5647$\times 10^{-2}$    & -0.063\% \\
CVR    & 1.3568$\times 10^{-1}$  & 1.3606$\times 10^{-1}$    & +0.280\%     \\
CTCVR  & 1.3014$\times 10^{-2}$  & 1.2985$\times 10^{-2}$    & +0.218\%    \\
NFR-UV & 1.0128$\times 10^{-4}$  & 1.0294$\times 10^{-4}$    & +1.605\%     \\
NFR-PV & 8.3283$\times 10^{-5}$  & 8.6726$\times 10^{-5}$    & +3.342\%     \\ \hline
\end{tabular}
\end{table}
\begin{itemize}[leftmargin=*]
\item{The performance gain \textit{w.r.t.} CVR, CTCVR, and OV are 0.280\%, 0.218\%, and 0.310\%, respectively.} Such improvement resulting from adjusting the quotas with the assistance of the needs-meeting way prediction module can lead to a significant increase in the consumption amount.
\item{The NFR-UV and NFR-PV decrease by 1.57\% and 3.31\% respectively, indicating that the recommendation engine is able to recommend fewer items that users dislike in the Guess-you-like page by adjusting the quotas with the aid of the needs-meeting way prediction module in our system.} 
\end{itemize}

We further calculate the performance increase on some popular categories of life services in Guess-you-like page. The results are shown in Table~\ref{tab::bus}. 
From the result, we can observe that:
\begin{itemize}[leftmargin=*]
\item{For all these categories there is average relative improvement of 3.801\% \textit{w.r.t.} CVR and 3.765\% \textit{w.r.t} CTCVR.} With such improvements, all these categories can enjoy a notable increase in order volume on Meituan platform. 
\item{Among all these categories, the rise on Kids category is the most significant, which is 8.723\% \textit{w.r.t.} CVR and 9.691\% \textit{w.r.t.} CTCVR.} 
\item{There are also slight decreases \textit{w.r.t.} CTR of 0.034\% for Food Delivery, -0.449\% for Beauty, and -0.595\% for Hotel.} These decreases are within the typical fluctuations observed in the market. 
\end{itemize}
The above results, which demonstrate the successful implementation of our system in guess-you-like recommendation, further illustrate the efficacy of our system, particularly in needs-meeting way prediction.

\begin{table}[]
\caption{The A/B tests results on several popular categories of life services in \textit{Guess-you-like} page.}
\label{tab::bus}
\begin{tabular}{ccccc}
\hline
Category                  & Metric & w/o NEON   & with NEON  & Improvement \\ \hline
\multirow{3}{*}{Food} & CTR    & 1.8298$\times 10^{-2}$  & 1.8292$\times 10^{-2}$  & -0.034\%     \\
                          & CVR    & 2.8841$\times 10^{-1}$ & 2.8987$\times 10^{-1}$ & +0.506\%     \\
                          & CTCVR  & 5.2774$\times 10^{-3}$ & 5.3023$\times 10^{-3}$ & +0.472\%     \\ \hline
\multirow{3}{*}{Beauty}   & CTR    & 1.8769$\times 10^{-2}$  & 1.8684$\times 10^{-1}$  & -0.449\%     \\
                          & CVR    & 1.2583$\times 10^{-3}$  & 1.2889$\times 10^{-3}$ & +2.427\%     \\
                          & CTCVR  & 2.3616$\times 10^{-4}$  & 2.4081$\times 10^{-4}$  & +1.967\%     \\ \hline
\multirow{3}{*}{Kids}     & CTR    & 1.7717$\times 10^{-2}$  & 1.7874$\times 10^{-2}$  & +0.890\%     \\
                          & CVR    & 7.8825$\times 10^{-3}$  & 8.5701$\times 10^{-3}$  & +8.723\%     \\
                          & CTCVR  & 1.3965$\times 10^{-1}$  & 1.5319$\times 10^{-4}$  & +9.691\%     \\ \hline
\multirow{3}{*}{Hotel}    & CTR    & 1.6567$\times 10^{-2}$  & 1.6469$\times 10^{-2}$ & -0.595\%     \\
                          & CVR    & 4.1454$\times 10^{-2}$  & 4.2923$\times 10^{-2}$  & +3.546\%     \\
                          & CTCVR  & 6.8678$\times 10^{-1}$  & 7.0690$\times 10^{-4}$  & +2.930\%     \\ \hline
\end{tabular}
\vspace{-0.3cm}
\end{table}
\subsection{Message Pop-up Recommendation}\label{sec::pnp}
In this section, we test our model's performance in message pop-up recommendation to observe its potential need prediction ability.
\subsubsection{Deployment Scenario}
In the case where a user has a living need that has never or rarely been fulfilled on Meituan platform, he/she probably won't launch Meituan mobile app, as they may not be aware that this need can be satisfied on Meituan or may not be accustomed to fulfilling this need on Meituan. So, if our engine runs in the background of the phone and detects the user's potential living need, it will send a message pop-up to the user with a recommendation for a life service solution, if the user allows it. The message pop-up recommends the user a life service that is in the category of the highest score output by our system.

In brief, with the ability of potential needs prediction, NEON is deployed for message pop-up recommendation.
\subsubsection{Experiment Setting}
We conduct online A/B tests involving one million users over a period of two weeks. The users are randomly divided into two buckets of similar volume and are assigned different strategies for selecting items to be sent in message pop-ups.  For the first bucket, the message pop-up recommends items that belong to the category with the highest score output by our system. For the second bucket, in each hour of the day, we calculate the popularity and average CTR of each category. We distribute the traffic for message pop-ups to the categories with the most popularity and highest average CTR. 

We use CTR, CVR, CTCVR, OV, \textit{Number of Cold-start Customers} (NCC) to measure the performance of our system in potential need prediction. NCC represents the number of customers who purchase a lifestyle service that they have never acquired on the platform previously.
\subsubsection{Performance}
The results of the online A/B tests are shown in Table~\ref{tab::online3overall}. We can observe that:
\begin{itemize}[leftmargin=*]
\item{By replacing the algorithm based on popularity and average CTR with our living need prediction system NEON to determine the category of recommendation in message pop-up, all metrics show significant improvement. CTR, CVR, CTCVR, and OV increase by 8.21\%, 78.64\%, 85.71\%, and 95.92\%, respectively.} 
\item{NCC increases by 74.26\%, indicating that the online deployment of our system in message pop-up recommendations results in a 74.26\% increase in the number of customers purchasing a life service that they have not previously acquired on the Meituan platform, via the message pop-up feature.} 
\end{itemize}
Our system is able to accurately detect the potential needs of users for a specific lifestyle service, even in instances where they have never previously purchased that service on the Meituan platform, and accordingly deliver targeted message pop-ups to them. This result strongly demonstrates the exceptional capability of our system in predicting users' potential needs.
\begin{table}[]
\caption{Results of A/B tests on message pop-up recommendation.}
\vspace{-0.2cm}
\label{tab::online3overall}
\begin{tabular}{cccccc}
\hline
Metric      & CTR     & CVR      & CTCVR    & OV       & NCC      \\ \hline
Improv. & +8.21\% & +78.64\% & +85.71\% & +95.92\% & +74.26\% \\ \hline
\end{tabular}
\end{table}


In summary, the success achieved by our NEON system on three downstream recommendation tasks proves its effectiveness on fine-grained need prediction, needs-meeting way prediction, and potential need prediction, respectively. The significant performance gain in real applications can lead to huge benefits.
