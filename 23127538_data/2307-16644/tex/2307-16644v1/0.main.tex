\documentclass[sigconf]{acmart}
\settopmatter{printacmref=true}
\let\Bbbk\relax\usepackage{amssymb}
\usepackage{amsfonts}
\usepackage{algorithmic}
\usepackage{graphicx}
\usepackage{xcolor}
\usepackage{makecell}
\usepackage{booktabs} 
\usepackage{subfig}
\usepackage{amsmath}
\usepackage{accents}
\usepackage{statex}
\usepackage[normalem]{ulem}
\usepackage{enumitem}
\usepackage{multirow}
\usepackage[nomargin,inline,marginclue,draft]{fixme}
\usepackage{balance}
\usepackage{changepage}
\usepackage{bm}
\usepackage{hyperref}
\usepackage{setspace}
\usepackage{mathrsfs}
\usepackage{ulem}
\usepackage{verbatim}
\usepackage{diagbox}
\usepackage{pdftexcmds}
\usepackage{catchfile}
\usepackage{ifluatex}
\usepackage{ifplatform}
\usepackage{threeparttable}
\usepackage[normalem]{ulem}
\useunder{\uline}{\ul}{}
\usepackage{comment}

\usepackage[T1]{fontenc}
\usepackage{aecompl}
\renewcommand{\thefootnote}{}
\newcommand{\rev}[1]{\textcolor{black}{#1}}
\newcommand{\para}[1]{\noindent \textbf{#1}}

\newlength\savedwidth
\newcommand\whline{\noalign{\global\savedwidth\arrayrulewidth
		\global\arrayrulewidth 1.1pt}%
	\hline
	\noalign{\global\arrayrulewidth\savedwidth}}

\newcommand{\chen}[1]{\footnote{+chen+: #1}}
\newcommand{\chenc}[1]{\textcolor{blue}{[#1]}}


\renewcommand{\authors}{}
\renewcommand{\shortauthors}{Xiaochong Lan et al.}

\author{Xiaochong Lan}
\affiliation{%
 \institution{Department of Electronic Engineering, BNRist,}
 \institution{Tsinghua University}
 \city{Beijing}
 \country{China}
}
\email{lanxc22@mails.tsinghua.edu.cn}

\author{Chen Gao$\dag$}
\affiliation{%
 \institution{Department of Electronic Engineering, BNRist,}
 \institution{Tsinghua University}
 \city{Beijing}
 \country{China}
}
\email{chgao96@gmail.com}

\author{Shiqi Wen}
\affiliation{%
 \institution{Meituan}
 \city{Beijing}
 \country{China}
}
\email{wenshiqi@meituan.com}

\author{Xiuqi Chen}
\affiliation{%
 \institution{Meituan}
 \city{Beijing}
 \country{China}
}
\email{chenxiuqi@meituan.com}

\author{Yingge Che}
\affiliation{%
 \institution{Meituan}
 \city{Beijing}
 \country{China}
}
\email{cheyingge@meituan.com}

\author{Han Zhang}
\affiliation{%
 \institution{Meituan}
 \city{Beijing}
 \country{China}
}
\email{zhanghan56@meituan.com}

\author{Huazhou Wei}
\affiliation{%
 \institution{Meituan}
 \city{Beijing}
 \country{China}
}
\email{weihuazhou@meituan.com}

\author{Hengliang Luo}
\affiliation{%
 \institution{Meituan}
 \city{Beijing}
 \country{China}
}
\email{luohengliang@meituan.com}

\author{Yong Li}
\affiliation{%
 \institution{Department of Electronic Engineering, BNRist,}
 \institution{Tsinghua University}
 \city{Beijing}
 \country{China}
}
\email{liyong07@tsinghua.edu.cn}

                                                                                   

\settopmatter{printacmref=true}



 \providecommand\BibTeX{{%
 Bib\TeX}}

\copyrightyear{2023}
\acmYear{2023}
\setcopyright{rightsretained}
\acmConference[KDD '23]{Proceedings of the 29th ACM SIGKDD Conference on Knowledge Discovery and Data Mining}{August 6--10, 2023}{Long Beach, CA, USA}
\acmBooktitle{Proceedings of the 29th ACM SIGKDD Conference on Knowledge Discovery and Data Mining (KDD '23), August 6--10, 2023, Long Beach, CA, USA}
\acmDOI{10.1145/3580305.3599874}
\acmISBN{979-8-4007-0103-0/23/08}


\makeatletter
\gdef\@copyrightpermission{
  \begin{minipage}{0.3\columnwidth}
   \href{https://creativecommons.org/licenses/by/4.0/}{% Figure removed}
  \end{minipage}\hfill
  \begin{minipage}{0.7\columnwidth}
   \href{https://creativecommons.org/licenses/by/4.0/}{This work is licensed under a Creative Commons Attribution International 4.0 License.}
  \end{minipage}
  \vspace{5pt}
}
\makeatother

\begin{document}
\renewcommand{\shortauthors}{Xiaochong Lan et al.}
	\title{NEON: Living Needs Prediction System in Meituan}

\begin{abstract}
\footnotetext{$\dag$Chen Gao is the corresponding author (chgao96@gmail.com).}
Living needs refer to the various needs in human's daily lives for survival and well-being, including food, housing, entertainment, etc. At life service platforms that connect users to service providers, such as Meituan, the problem of living needs prediction is fundamental as it helps understand users and boost various downstream applications such as personalized recommendation. 
However, the problem has not been well explored and is faced with two critical challenges.
First, the needs are naturally connected to specific locations and times, suffering from complex impacts from the spatiotemporal context.
Second, there is a significant gap between users' actual living needs and their historical records on the platform. 
To address these two challenges, we design a system of living \textbf{NE}eds predicti\textbf{ON} named NEON, consisting of three phases: feature mining, feature fusion and multi-task prediction. 
In the feature mining phase, we carefully extract individual-level user features for spatiotemporal modeling, and aggregated-level behavioral features for enriching data,
which serve as the basis for addressing two challenges, respectively. 
Further, in the feature fusion phase, we propose a neural network that effectively fuses two parts of features into the user representation. Moreover, we design a multitask prediction phase, where the auxiliary task of needs-meeting way prediction can enhance the modeling of spatiotemporal context.
Extensive offline evaluations verify that our NEON system can effectively predict users' living needs. 
Furthermore, we deploy NEON into Meituan's algorithm engine and evaluate how it enhances the three downstream prediction applications, via large-scale online A/B testing. As a representative result, deploying our system leads to a 1.886\% increase \textit{w.r.t.} CTCVR in Meituan homepage recommendation.
The results demonstrate NEON's effectiveness in predicting fine-grained user needs, needs-meeting way, and potential needs, highlighting the immense application value of NEON.
\end{abstract}


\ccsdesc[500]{Information systems~Information systems applications}

\keywords{Living Needs Prediction; Deep Neural Networks; Multi-task Learning}

\maketitle
	
\section{Introduction}

Argumentation is the field of elaboration and presentation of arguments to debate, persuade, and agree, where an argument is made of a conclusion (i.e., a claim) supported by reasons (i.e., premises)~\cite{walton2008argumentation}. By analogy with computational linguistics, \textit{computational argumentation} refers to the use of computer-based methods to analyze and create arguments and debates~\cite{gurevych-etal-2016}. It is a subfield of artificial intelligence that deals with the automated representation, evaluation, and generation of arguments. This field includes important applications such as mining arguments~\cite{al-khatib-etal-2016-cross}, assessing an argument's quality~\cite{el-baff-etal-2018-challenge}, reconstructing implicit assumptions in arguments~\cite{habernal-etal-2018-argument} or even providing constructive feedback for improving arguments~\cite{naito-etal-2022-typic}, to name a few.

In the context of education, learning argumentation (e.g., writing argumentative essays, debates, etc.) has been shown to improve students' critical thinking skills~\cite{pitchers-sodden-2000, behar-horenstein-etal-2011-teaching}. To further improve critical thinking skills, several researchers have been working on computational argumentation to support and provide tools to assist learners in improving the quality of their arguments.

Although computational models for argumentation are proven to assist students' learning and reduce teachers' workload~\cite{twardy-2004, wambsganss-etal-2021-arguetutor}, such models still lack the ability to \emph{explain} how an argument can be improved efficiently; e.g., why a particular argument was labeled bad or given a low score by their automatic evaluation rubrics. In other words, the model should be not only able to provide its results but also be able to \textit{explain and visualize the results in a comprehensive way} for the users so that users can understand, and ultimately improve their argumentation skills.

% Figure environment removed

We argue that the output for current computational models for argumentation act as a type of explanation and must be the focus of future work. For our survey, we categorize works into four different dimensions (cf., Figure~\ref{fig:overview}):
\begin{itemize}
    \item \textit{Richness}: Level of feedback details given by a model, i.e., \textit{what} is the error identified by the model and \textit{why} it is an error;
    \item \textit{Visualization}: Way of presenting feedback, i.e., \textit{how} the explanation is shown;
    \item \textit{Interactivity}: Ability to communicate with the model, other users, or a third-person, i.e., with \textit{whom} the user is talking;
    \item \textit{Personalization}: Ability to adapt the feedback to the users' background, i.e., \textit{to whom} the feedback is given.
\end{itemize}

% Figure environment removed

In Figure~\ref{fig:ex-types}, for a given argument consisting of two claims and one premise, four different feedback are shown, each highlighting a different dimension of feedback (\textit{Richness}, \textit{Visualization}, \textit{Interactivity}, and \textit{Personalization}).

Towards explainable computational argumentation, this survey aims to give an overview of computational argumentation on automated quality assessment.
We explore work providing explanations answering the following: \textit{What} (\S\ref{sec:richness-what}), \textit{Why} (\S\ref{sec:richness-why}), \textit{How} (\S\ref{sec:visualization}), \textit{Who} (\S\ref{sec:interactivity}), and \textit{To Whom} (\S\ref{sec:personalization}).
Finally, we discuss remaining challenges and potential ways to overcome them (\S\ref{sec:open_issues}) in order to develop systems that provide explanations in a way in which learners can improve their critical thinking skills.
We believe this survey can aid researchers in understanding current explanations in argumentation and broaden their horizon on argumentative feedback.\footnote{\label{foot:website}For more details, papers mentioned in this survey are categorized at \url{https://anonymized}.}

%and focus more on explanation in argumentation and apply it to newer models, thus making the system more explainable.




\section{Problem Statement}\label{sec::profdef}
As we discussed above, in users' daily life, they generate various living needs, such as eating, accommodation, entertainment, beauty, etc. These needs can be fulfilled by life service providers in the city, which can be accessed through platforms connecting life service providers to customers. To enhance the user experience, it's crucial for these platforms to accurately predict users' needs and recommend appropriate services. This leads to the problem of living needs prediction. 

As defined in the introduction, living needs prediction aims to predict the specific living needs of a user given the spatiotemporal context in which they are located (in the following we also refer to this as given the \textit{user scene}). To clearly define the problem, with the help of experts, we divide all the living needs that users can satisfy on the platform into ten categories, shown in Table~\ref{tab::needs}. We use $\mathcal{N}$ as the symbol for the set of all living needs. In response to these needs, all the life services on Meituan are also divided into 10 categories. The problem can be defined as follows.

\begin{table}

\caption{10 types of living needs that can be satisfied in Meituan}
\vspace{-0.3cm}
\label{tab::needs}

\begin{tabular}{|c|l|}

\hline 
& \fontfamily{ppl}\selectfont Ordering food delivery, Eating in a restaurant, \\

 Living & \fontfamily{ppl}\selectfont Booking a hotel, Buying medicine, \\

 Needs & \fontfamily{ppl}\selectfont Specialty shopping online, Hair-dressing, \\

  &  \fontfamily{ppl}\selectfont Grocery shopping online, Beauty, \\

  & \fontfamily{ppl}\selectfont Tourism and Entertainment \\

\hline

\end{tabular}

\end{table}

\noindent \textbf{Input:}  A dataset $\mathcal{O}^+$ of real-world life service consumption records that reflect users' living needs. Each instance in the dataset tells the kind of life service a specific user purchases, which indicates the specific living need $n$ ($n\in \mathcal{N}$) of the user in a specific user scene $i$.

\noindent \textbf{Output:} A model to estimate the probability that a user will generate the living need $n$ in user scene $i$, formulated as $f(i,n|\mathcal{O}^{+})$. Here $f(\cdot)$ denotes the function that the model aims to learn. 


\section{Our NEON System}\label{sec::method}

To address the challenges mentioned in the introduction, we develop the NEON system made up of three phases: feature mining, feature fusion layer, and multitask prediction. 
First, in the feature mining phase, we address the first challenge by carefully designing spatiotemporal features for individual-level users and address the second challenge by extracting behavioral-pattern features for group-level users.
The feature fusion phase then employs a feature-fusion neural network to seamlessly integrate internal preferences, spatiotemporal context impact, and group behavior patterns to generate complete user representations, overcoming both challenges.
Last, in the multitask prediction stage, to enhance the model's understanding of spatiotemporal context, we introduce an auxiliary task of needs-meeting way prediction to the main goal of predicting living needs, providing additional support in addressing the first challenge. The deep feature fusion layer and the multi-task prediction parts of our system are illustrated in Figure~\ref{fig::model}.

% Figure environment removed

\subsection{Feature Mining}\label{sec::FeatureMining}
First of all, we use features that directly reflect user traits, such as users' profile, their recent behavior sequence, and their historical behaviors, as inputs for the model.

 As mentioned above, for a specific user, his living needs are greatly affected by the spatial and temporal scenarios in which he is located. For example, \textit{on a rainy midday, a person at work probably has the need to order food delivery; but on a sunny noon, he/she may have another need of going out to eat in the restaurant}. This complexity and variability of human living needs, driven by the flux of time and space, pose a considerable challenge in accurately modeling the impact of the spatiotemporal context. In order to tackle this challenge, we incorporate spatiotemporal context features as an integral part of our system's input.

What's more, on the platform, users may have potential needs with sparse or even non-existent history records. For example, \textit{a person who never buys medicine on the platform may have a cold and need to buy medicine online one day.} Such potential needs are difficult for the model to grasp. To address the challenge of modeling potential needs, we introduce group behavior pattern features to help the model learn the potential living needs of users. 

Below we give a detailed description of the three categories of features.
\subsubsection{User Features}
This group of features includes user profiles and user history behavior sequences.
\begin{itemize}[leftmargin=*]
\item\textbf{User profiles $\bm{f}^U_p$.} The user's profile, including their age, gender, etc.
\item\textbf{User recent online behavior sequence $\bm{f}^U_{rb}$.} The sequence of items recently clicked by the user in the platform; the sequence of items recently ordered by the user in the platform.
\item\textbf{User aggregated historical online behavior $\bm{f}^U_{hb}$.} The percentage of times users buy each type of life service.
\item\textbf{User offline visitation record $\bm{f}^U_{ov}$.} The 50 most visited POIs (point of interest) by the user in the last six months; the 50 most visited AOIs (area of interest) by the user in the last six months. 
\end{itemize}
We concatenate all the mentioned features above to get a sparse user feature vector $f^U$, formulated as follows:
\begin{equation}
f^U=\left[f^U_p, f^U_{rb}, f^U_{hb}, f^U_{ov}\right].
\end{equation}
\subsubsection{Spatiotemporal Context Features}
Users' living needs are greatly affected by time, location, and other environmental factors. Thus, we introduce spatiotemporal context features as part of the input of our system to help our system model the complex impact of spatiotemporal context, which can be listed as follows.
\begin{itemize}[leftmargin=*]
\item\textbf{Time $\bm{f}^{ST}_t$.} Current time period. More than one time period feature of different granularity is applied, including hour, day, whether it is a holiday, etc.
\item\textbf{Location $\bm{f}^{ST}_l$.} The POI (point of interest) embedding of the user's real-time location; the AOI (area of interest) embedding of the user's real-time location; the city embedding of the user's real-time location. The location features are hourly real-time features.
\item\textbf{Weather $\bm{f}^{ST}_w$.} Weather information for the user's city or region, including wind, humidity, temperature, and weather type (sunny, rainy, snowy, etc.). Weather features are refined to hourly granularity.
\item\textbf{Travel state $\bm{f}^{ST}_{ts}$.} Information about whether the user is located in his/her resident city. Possible states include \textit{based in resident city}, \textit{about to travel}, and \textit{on travel}.
\end{itemize}
The dense spatiotemporal context feature vector $f^{ST}$ is created by concatenating all previously mentioned context features, formulated as follows:
\begin{equation}
f^{ST}=\left[f^{ST}_t, f^{ST}_l, f^{ST}_w, f^{ST}_{ts}\right].
\end{equation}
\subsubsection{Group Behavior Pattern Features}
We introduce group behavior pattern features to supplement the sparse individual behavior of users, in order to assist in identifying the potential living needs of individual users. 
\begin{itemize}[leftmargin=*]
\item\textbf{Group aggregated behavior $\bm{f}^G_{a}$.} We first segment users into groups based on their profiles. In each group, we get the group aggregated behavior by calculating the percentage of views, clicks, and purchases of each type of life service among all views, clicks, and purchases initiated by the group. For each user, the group aggregated behaviors of the groups the user belongs to are used as features. For example, a middle-aged person has group aggregated behaviors feature of middle-aged users and other groups he/she is in. 
\item\textbf{Popularity in the current time period $\bm{f}^G_{ct}$.} We cut all time into time periods according to different criteria, such as whether it is 
a holiday, if it is morning, noon, or night, etc. Then in each time period, we calculate the popularity of each type of life service by calculating the percentage of times the life service is viewed/clicked/purchased among all views/clicks/purchases happening in this time period. We determine the time periods in which the current time is located, and use the popularity of each type of life service in these time periods as a feature. For example, if the user opens the app on Christmas night, popularity on holiday and popularity at night of each kind of life service are set as features.
\item\textbf{Group behavior pattern in spatiotemporal context $\bm{f}^G_{st}$.} By discovering group preferences in different spatiotemporal contexts, we further capture more fine-grained group behavior patterns. We calculate the percentage of views/clicks/purchases of each type of life service initiated by each group in each kind of spatiotemporal context. These fine-grained patterns are used as features of the model. For example, the group preference of middle-aged people at work at noon on working days are used to enrich the representations of every individual within this demographic in such spatiotemporal scenario.
\item\textbf{User behaviors augmented by inter-need correlation $\bm{f}^G_{ic}$.} There is an inherent association across different types of users' living needs. This association can be leveraged to improve prediction performance. For example, \textit{a user who frequently purchases hairdressing services may also be inclined to purchase beauty services.} We use the association rule mining algorithm to analyze the co-occurrence of different life service categories, filter out high-correlation relationships, and employ them to augment user behavior as input features.
\end{itemize}
We combine all previously mentioned group behavior pattern features to generate the dense group behavior pattern feature vector $f^{G}$, which is formulated as follows:
\begin{equation}
f^{G}=\left[f^{G}_a, f^{G}_{ct}, f^{G}_{st}, f^{G}_{ic}\right].
\end{equation}

\subsection{Feature Fusion Layer}\label{sec::FeatureIntegration}
As mentioned in Section~\ref{sec::profdef}, we refer to a user in a specific spatiotemporal context as a user scene. For a user scene $i$, after feature mining, we have dense spatiotemporal features $f^{ST}_i$, dense group pattern features $f^G_i$, and sparse user features $f^U_i$. For brevity of presentation, we omit the subscript $i$ in some of the expressions below. We designed a feature fusion layer to integrate these features into the input of the subsequent prediction module.

We first set up an embedding layer, which processes the high-dimensional sparse user feature vector $f^U$ into a low-dimensional dense vector $v^U$.  To address the challenges of complex impact of spatiotemporal context and users' potential needs, we mine spatiotemporal features and group behavior pattern features in the feature mining phase, respectively. With these features as input, we use a feature merging network to model the interaction between spatiotemporal contexts, group behavior patterns, and  users as follows,
\begin{equation}
x^M=h^M\left(\left[f^{ST}, f^G, v^U\right]\right),
\end{equation}
where $\left[ \cdot \right]$ denotes concatenation operation. Here $h^M$ is the feature merging network, which merges three information sources of spatiotemporal contexts, group behavior patterns, and user preference into a fusion representation $x^M$.

Moreover, users have their own internal characteristics that are independent of the spatiotemporal scene they are in and the group they belong to. To model the internal characteristics of users, we generate a representation as follows,
\begin{equation}
x^U=h^U\left(f^U\right),
\end{equation}
where $h^U$ denotes the user preference network that turns raw user features into dense user preference representation. We then concatenate the two parts of representations into the full representation of the user scene:
\begin{equation}
x=\left[x^M, x^U\right].
\end{equation}
In brief, we design a feature fusion layer to tackle both challenges by considering the influence of spatiotemporal context, incorporating group behavior patterns, as well as extracting individual preferences.

\subsection{Multitask Prediction}\label{sec::Model}
We further design a prediction module which takes user scene representation as input to predict users' living needs. The module is tasked with two objectives: \textit{fine-grained need prediction} and \textit{needs-meeting way prediction}. Fine-grained need prediction is to predict the specific living need of the user. Neets-meeting way prediction is to predict the preferred way of the user to meet their needs. 

Specifically, among the ten kinds of needs which we mentioned in the problem formulation, there are two ways to satisfy the needs: in-store and via-delivery. In other words, consumers can choose to satisfy their needs by visiting a physical store or by ordering online and then receiving goods via delivery. Each type of the 10 needs can be classified into one of two categories, in-store needs or via-delivery needs. We show the classification in Table~\ref{tab::nc}. Actually, needs-meeting way prediction is to predict whether the preferred way of the user to meet their needs is in-store or via-delivery.
\begin{table}[]
\vspace{-0.3cm}
\caption{The classification of the 10 living needs that can be satisfied on Meituan}
\label{tab::nc}
\begin{tabular}{|l|l|}
\hline Living needs that& \fontfamily{ppl}\selectfont Specialty shopping online,\\
 can be satisfied&  \fontfamily{ppl}\selectfont Grocecy shopping online, \\
via delivery& \fontfamily{ppl}\selectfont Ordering food delivery, Buying medicine \\
\hline Living needs that& \fontfamily{ppl}\selectfont Eating in a restaurant, Hotel, \\
can be satisfied& \fontfamily{ppl}\selectfont Hair-dressing, Beauty,  \\
in store& \fontfamily{ppl}\selectfont Tourism, Entertainment \\
\hline
\end{tabular}
\vspace{-0.6cm}
\end{table}

Users' preferences for needs-meeting ways are strongly affected by the spatiotemporal context. For example, \textit{a person at work during lunchtime on a weekday is more likely to have the need to order food delivery (via delivery), while the same person in a shopping district on a weekend evening is more likely to visit a store for a meal (in-store)}. With this in mind, we include the needs-meeting way prediction task which is jointed trained with the main task of need prediction to enhance the model's ability to learn spatiotemporal context information.
Next, we describe how we get the prediction results of the two tasks. 
We use $y^W$ and $y^N$ to denote the prediction result of needs-meeting way and specific need. $y^k, k \in \{W, N\}$ can be generated as follows,
\begin{equation}
 \begin{aligned}
y^k &= t^k(z^k),\\
\text{where }z^k &= g^k(x)_0E^k(x)+g^k(x)_1E^S(x).
\end{aligned}
\end{equation}
Here $t^k$ is the prediction neural network for task $k$. There are a variety of choices in the specific structure of the neural network. In Section~\ref{sec::implementation}, we will state our specific choice. To avoid verbosity, we will use \textit{network} to replace ~\textit{neural network} in the following text. The output of $t^N$, $y^N$, is the scores of ten types of living needs, and the output of $t^W$, $y^W$, is the scores of in-store and via-delivery needs-meeting ways. We use $s^N_{im}$ to denote the score of need $m$ for user scene $i$, and use $s^W_{in}$ to denote the score of needs-meeting way $n$ for user scene $i$. $E^k$ is the expert network~\cite{ma2018modeling,tang2020progressive} for task $k$. $E^S$ is the shared network between the two tasks. $E^S$ is responsible for generating general representations that are common to both tasks, while $E^k$ is responsible for learning task-specific representations that are more fine-tuned to the specific task $k$. The gating network $g_k$ determines what proportion of information input each task's prediction network receives from the shared network and the expert network. We formulate the gating network as follows,
\begin{equation}
g^k(x)=\operatorname{Softmax}(W_k x),
\end{equation}
where $W_k \in \mathbb{R}^{2 \times d}$ is trainable weights for task $k$. The gating network takes $x$ as input, and outputs the relative importance of the shared and task-specific representations for a given tasking, allowing the model to selectively attend to the most relevant information and improve its performance.
In summary, to address the complexity of spatiotemporal context impact, we introduce an auxiliary task of needs-meeting way prediction which is jointly trained with the main task of fine-grained living needs prediction to enhance our system's learning of spatiotemporal context. The multitask prediction module in our system produces a score for each living need and needs-meeting way.
\subsection{Model Training}\label{sec::train}
In this section we describe how our system is trained. Corresponding to the two tasks, we design two parts of loss. We design need prediction loss taking into account the fact that the frequency of different needs arising in users' lives is different. For example, \textit{a user may need to order food delivery for lunch every workday, but rarely need to buy medicine}. In order to address the class imbalance issue for different living needs, we propose using a multi-class focal loss which can decrease the effect of needs with a high volume of training data on the final prediction loss. The need prediction loss can be formulated as follows,
\begin{align}
\text {Loss}_{\text{need}}&=-\sum_{i\in O}\left(\sum_{n=1}^{10}\left(1-q^N_{i n}\right)^\gamma \chi^N_{i n} \log \left(q^N_{i n}\right)\right),\\
\text{where } q^N_{in}&=\operatorname{Softmax}\left(s^N_{i n}\right)=\frac{e^{s^N_{i n}}}{\sum_n e^{s^N_{i n}}}.
\end{align}
Here $O$ is the training set, $s^N_{in}$ is the score of living need $n$ for user scene $i$, $\gamma$ is the hyperparameter which decides the importance of difficult samples, $\chi^N_{in}$ is 1 if $n$ is the ground truth need for user scene $i$, else it is 0.
For the needs-meeting way prediction task, we use BCE loss as prediction loss. We formulate it as follows,
\begin{align}
\text{Loss}_{\text{way}}&=-\sum_{i\in O}\left(\sum_{m=1}^{2}\chi^W_{im}
\log \left(q^W_{im})\right)\right),\\
\text{where } q^W_{im}&=\operatorname{Softmax}\left(s^W_{i m}\right)=\frac{e^{s^W_{i m}}}{\sum_m e^{s^W_{i m}}}.
\end{align}
Here $O$ is the training set, $s^W_{im}$ is the score of needs-meeting way $m$ (in store or via delivery )for user scene $i$.
$\chi^W_{im}$ is 1 if $m$ is the ground truth needs-meeting way for user scene $i$, else it is 0.
In our system, the feature integration module and multitask prediction module are trained end to end. The entire loss function is:
\begin{equation}
\text{Loss} = \lambda_1 \text{Loss}_{\text{need}}+\lambda_2 \text{Loss}_{\text{way}}.
\end{equation}
$\lambda_1$ and $\lambda_2$ are hyperparameters that control the importance of the two parts of loss.


\section{Experiments}\label{sec:exp}
\subsection{Experimental Settings}

\noindent\textbf{Datasets}.
We conduct experiments on 11 publicly available image classification datasets following CoOP~\cite{zhou2022learning}. The datasets including ImageNet~\cite{deng2009imagenet}, FGVCAircraft~\cite{maji2013fine}, StanfordCars~\cite{Krause_2013_ICCV_Workshops}, Flowers102~\cite{nilsback2008automated}, Caltech101~\cite{fei2004learning}, DTD~\cite{cimpoi2014describing}, EuroSAT~\cite{helber2019eurosat}, Food101~\cite{bossard2014food}, UCF101~\cite{soomro2012ucf101}, OxfordPets~\cite{parkhi2012cats}, and SUN397~\cite{xiao2010sun}. 
% The detailed statistics of these datasets are illustrated in \cref{table:dataset}. 
% These datasets cover a wide range of image classification tasks, \eg, classification of objects, actions, textures, satellite images, and fine-grained classification. 
% Thus, these datasets compose a comprehensive benchmark for evaluating the transfer capacity of the V\&L model.

% \begin{table}[tb]
% 	\caption{The statistics of the 11  image classification datasets}
	
% 	\centering
% 	% \vspace{-0.20cm}
% 	\resizebox{\linewidth}{!}{
% 		\begin{tabular}{c|c|c|c}
% 		\hline
% 		\multirow{1}{*}{Dataset} & \multirow{1}{*}{\# of class} &
% 		\multirow{1}{*}{\# of testing data} &
% 		\multirow{1}{*}{Task} \\ 
% 		\hline 
		
% 		ImageNet~\cite{deng2009imagenet} & 1,000 & 50,000  & Object \\
% 		Caltech101~\cite{fei2004learning} & 101 & 2,465 & Object \\
% 		FGVCAircraft~\cite{maji2013fine} & 100 & 3,333  & Fine-grained \\
% 		StanfordCars~\cite{Krause_2013_ICCV_Workshops} & 196 & 8,041 & Fine-grained \\
% 		Flowers102~\cite{nilsback2008automated} & 102 & 2,463 & Fine-grained \\
%         OxfordPets~\cite{parkhi2012cats} & 37 & 3,669 & Fine-grained \\
%         Food101~\cite{bossard2014food} & 101 & 30,300 & Fine-grained \\
% 		DTD~\cite{cimpoi2014describing} & 47 & 1,692 & Textures \\
% 		EuroSAT~\cite{helber2019eurosat} & 10 & 8,100 & Satellite Images \\
%         UCF101~\cite{soomro2012ucf101} & 101 & 3,783 & Actions \\
% 		SUN397~\cite{xiao2010sun} & 397 & 19,850 & Scenes \\
		
% 		\hline
% 		\end{tabular}
% 		}    
% 		\label{table:dataset}
% 		\begin{tablenotes}
% 		\scriptsize
% 		\item[*] We use these datasets for few shot image classification task. The number of training data and validation data is set to be a specific multiple of the number of classes in training period. Thus, we only demonstrate the number of testing data.
% 		\end{tablenotes}
		
% 	% \vspace{-.15in}
% 	\end{table}


% Figure environment removed


\iffalse
% Figure environment removed
\fi

\noindent\textbf{Implementation Details}.
We transfer CLIP to the few-shot image classification task with AMT and R-AMT. Specifically, we use 1, 2, 4, 8, and 16-shot training sets to optimize the model and evaluate it on the full test set, following~\cite{radford2021learning}. For $n$-shot image classification, we random sample $n$ images per category for training. All results reported below are the average of three runs with different random seeds. All images are resized to $224\times 224$. Random cropping, resizing, and random horizontal flipping strategy are used for data augmentation.
We utilize ViT-B/16 as the visual backbone of CLIP. 
For a fair comparison, all experiments only use single text prompt, except learnable text prompt methods, \eg, for ImageNet and SUN397, the text prompt is set to be ``a photo of a [class].''
% , while a task-relevant sentence is added for fine-grained classification datasets, \eg, for Flowers102, the text prompt is ``a photo of a [class], a type of flower.'' For other datasets, the text prompt is set to be a task-related context, \eg, for UCF101, the text prompt is ``a photo of a person doing [class].''
We adopt Adam optimizer for optimization.
% For ImageNet, the maximum epoch is set to 10, and the learning rate is set to 3e-5.
% For other datasets, the maximum epoch is set to 30, and the learning rate is set to 8e-5.
The mask weights are initialized element-wise with $10^{-2}$. The threshold $\alpha$ is set to be $5\times10^{-3}$.
The $l$ in~\cref{eq.optimize_final} is set to 0.3 for datasets except for ImageNet, SUN397, and Food101 in 16-shot experiments. And $l$ is set to 1.0 in other experiments.
% The parameter prompts are applied on all layers for 16/8/4-shot, while the parameter prompts are only applied on the last self-attention layer for 2/1-shot (except for EuroSAT where the  parameter prompts are applied on the last layer for all settings).
% The initial learning rate is set to be 5e-5, which is decayed by 0.1 at half of the maximum epoch.




\subsection{Comparison to State-of-the-Art Methods}
\noindent
\textbf{Main Results on 11 Datasets.} We compare AMT and R-AMT with Zero-shot CLIP and five state-of-the-art methods on the 11 datasets as mentioned above, demonstrated in~\cref{fig:main_results_vit}. 
% It is mentioned that although the VPT and UPT adopt the visual/unified prompt, they are only for transformer architecture to design methods and are not universal. Thus, we do not include the VPT and UPT in this setting of 11 Datasets.
% The baseline methods include Zero-shot CLIP and Linear probe CLIP: 1) the former directly transfers to the downstream task without training; 2) the latter trains a linear classifier for downstream tasks based on the pretrained CLIP. 
Zero-shot CLIP directly transfers to the downstream task without training.
The state-of-the-art methods include prompt tuning methods, \ie, CoOP~\cite{zhou2022learning}, VPT~\cite{jia2022visual}, UPT~\cite{zang2022unified}, ProGrad~\cite{zhu2022prompt}, and adapter tuning method TIP-Adapter~\cite{zhang2022tip}.
% The prompt-based methods are CoOP~\cite{zhou2022learning} and ProGrad~\cite{zhu2022prompt}. Both of them tend to learn better text prompts for adaptation to downstream tasks. 
According to \cref{fig:Average}, the AMT and R-AMT outperform these methods on average over 11 datasets, which approves the ability of AMT and R-AMT to transfer CLIP to the downstream tasks. R-AMT achieves better performance compared with AMT. It indicates the gradient dropout regularity formalism is able to enhance the transfer ability of mask tuning in few-shot scenarios.
% Moreover, we find that the AMT and R-AMT achieve superior performance on the FGVCAircarft dataset (\cref{fig:FGVCAircraft}). It is because \todo{reason}.

% Concretely, the PPL improves the second-best method ProGrad by 3.85\% on the 16-shot setting, while 33.50\% of parameters in the image encoder are left out on average.
% Moreover, the PPL achieves the best accuracy in most of the datasets on the 16/8/4/2-shot setting. From \cref{fig:FGVCAircraft}, we find the improvement in terms of accuracy is more significant as the training few-shot data increase for the FGVCAircraft dataset.
% However, given the 1-shot data for training, there is performance degradation of PPL on some datasets, \eg Food101. This shows the reliability of the parameter prompt depends on the number of training data. A small quantity of training data for fine-grained classification tasks may lead to overfitting.
% to select preferred knowledge within the pretrained model for downstream tasks.


\noindent
\textbf{Results on base-to-new generalization setting.}
Following CoCoOP~\cite{zhou2022conditional}, we conduct experiments on base-to-new generalization setting. Concretely, the classes are split equally into the base and new classes on each dataset. The base classes are used for training. The $l$ is set to 1 in all base-to-new generalization experiments. The averaged results over 11 datasets are shown in~\cref{tab:Average_b2n}. The numerical experimental results on each dataset are shown in Supplementary.
Overall, R-AMT reaches the best performance, which surpasses the second best method CLIP-Adapter~\cite{gao2021clip} 2.00\% on the harmonic mean on average.
Notably, the AMT achieves quite high performance on the base classes. But the accuracy has significantly degraded (5.11\% on average) on the new classes compared with Zero-shot CLIP. 
We deem the degradation to be the result of overfitting since the amount of training data is too small for some datasets, \eg, Eurosat. 
The R-AMT achieves competitive results with AMT on base classes. However, the performance of R-AMT improves AMT by 3.04\% on average in new classes, which demonstrates the anti-overfitting ability of the proposed gradient dropout regularity formalism.

\begin{table}[t]
    \caption{\textbf{Comparison on the base-to-new generalization setting on the average over 11 datasets with 16 shots.} ``H'' denotes the harmonic mean of the accuracy on base and new classes. Thanks to gradient dropout regularity, R-AMT can efficiently maintain the knowledge of new classes while improving the anti-overfitting ability of the model to base classes. We report the average accuracy over three runs. The  error bar and performance of each dataset are provided in the supplementary materials.}
    \vspace{2pt}
    % \resizebox{0.7\linewidth}{!}{
    \centering\small
    \setlength{\tabcolsep}{13pt}
    \begin{tabular}{ccc|c}
    \toprule
        Method & Base & New & H \\
         \hline
       Zero-shot CLIP  &  69.34 & \textbf{74.22} & 71.70 \\
       CoCoOP~\cite{zhou2022conditional}  & 80.47 & 71.69 & 75.83 \\
       ProGrad~\cite{zhu2022prompt}  & 82.79 & 68.55 & 75.00 \\
       CLIP-adapter~\cite{gao2021clip}  & 82.62 & 70.97 & 76.35 \\
       \hline
       \multirow{1}{*}{AMT}  & \textbf{86.17} & 69.11 & \underline{76.70}  \\
       % & - & - &  - \\
       \multirow{1}{*}{R-AMT} & \underline{85.71} &\underline{72.15} & \textbf{78.35} \\
       % & - & - & - \\
    \bottomrule
    \end{tabular}
    % }
    \label{tab:Average_b2n}
    \vspace{-8pt}
\end{table}



\noindent
\textbf{The robustness to distribution shift.}
We evaluate the out-of-distribution (OOD) ability of AMT and R-AMT by training them on ImageNet and evaluating on ImageNet-V2~\cite{recht2019imagenet} and Imagenet-Sketch~\cite{wang2019learning}, following~\cite{zhang2022tip}. The evaluating datasets have compatible categories with the training set. But the three datasets are different in semantics. The OOD experimental results are shown in~\cref{table:distribution_shift}. R-AMT achieves the best performance, which surpasses TIP-Adapter~\cite{zhang2022tip} 1.06\% on ImageNet-V2 and surpasses CoOP~\cite{zhou2022learning} 0.04\% on Imagenet-Sketch. This indicates the R-AMT is also capable of OOD tasks. Moreover, R-AMT boosts AMT 0.47\%, 0.94\%, and 0.91\% on ImageNet, ImageNet-V2, and Imagenet-Sketch, respectively. It further proves that R-AMT benefits from the gradient dropout regularity technique in terms of enhancing transfer and anti-overfitting ability.

\subsection{Combination with State-of-the-Art Methods}
\noindent
To prove the R-AMT is synergistic to existing parameter-efficient methods, we combine it with CoOP~\cite{zhou2022learning} and TIP-Adapter~\cite{zhang2022tip} on 11 datasets with 16 shots, as shown in~\cref{table:combine_sota_vit}. 
Concretely, we first load the binary masks trained with R-AMT  and multiply them with the original parameters of the image encoder of CLIP. Then we train the learnable contextual prompt or adapter following CoOP and TIP-Adapter.
Particularly, the few-shot training set for R-AMT, CoOP+R-AMT, and TIP-Adapter+R-AMT is the same.
For CoOP+R-AMT, the learned text prompt is randomly initialized and the length of the text prompt is set to 16, using the same training details as CoOP~\cite{zhou2022learning}. 
% The results of CoOP reported in~\cref{table:combine_sota_vit} are obtained with the same experimental setting for a fair comparison.
The CoOP+R-AMT boosts the performance of CoOP by 3.26\% on average.
This indicates the R-AMT provides a more reliable image encoder for learning better text prompts using CoOP.
In addition, this combination approach directly uses a mask that is optimized by hand-craft text and does not update this mask for the learnable text prompts from CoOP, resulting in not completely unleashing the potential of the mask for downstream tasks (\ie, not surpass the R-AMT).
For TIP-Adapter+R-AMT, the training details are also the same as the TIP-Adapter~\cite{zhang2022tip}.
% the $\alpha$ and $\beta$ are set to be 1.0 and 5.5, respectively, following TIP-Adapter~\cite{zhang2022tip}. The adapter is training for 20 epochs. The initial learning rate is set to be 0.001. The hyper-parameter setting for training is the same for TIP-Adapter. 
The TIP-Adapter+R-AMT improves TIP-Adapter 3.13\% on average with 16 shots.
This verifies the ability of R-AMT to endow existing parameter-efficient methods with the ability to better adapt to the downstream task.

\begin{table}[t]
	\caption{\textbf{Comparison on robustness to distribution shift.}}
	\vspace{2pt}
	\centering\small
        \setlength{\tabcolsep}{2.2pt}
	% \resizebox{\linewidth}{!}{
		\begin{tabular}{c|c|cc|c}
		\toprule
            \multirow{2}{*}{Method}  & Source & \multicolumn{2}{c|}{Target}  & \multirow{2}{*}{Average} \\
		% \multirow{1}{*}{Method} & \multirow{1}{*}{Average}& Source & \multicolumn{1}{*}{Target} & \multicolumn{1}{*}{Target} \\
            & ImageNet & -V2 & -Sketch &  \\
            \hline
            
		Zero-shot CLIP & 66.73 & 60.83 & 46.15 & 57.90 \\
            Linear probe & 65.85 & 56.26 & 34.77 & 52.29 \\
            CoOP & 71.73 & 64.56 & 47.89 & 61.39 \\
            CLIP-adapter & 71.77 & 63.97 & 46.27 & 60.67 \\
            TIP-adapter & \textbf{73.08} &64.85 & 46.76 & \underline{61.56} \\
            \hline
            AMT & 72.60\std{0.12} & \underline{64.97}\std{0.11} & \underline{47.02}\std{0.13} & 61.53 \\
            R-AMT &\underline{73.07}\std{0.10} & \textbf{65.91}\std{0.34} & \textbf{47.93}\std{0.26} & \textbf{62.30}\\
            % \hline
            % R-MMT &\textbf{73.52}\std{0} & {66.09}\std{0} & {47.68}\std{0} & \textbf{62.43}\\
            % R-PMT &73.48\std{0} & \textbf{66.19}\std{0} & {47.60}\std{0} & {62.42}\\
            
		\bottomrule
		\end{tabular}
            % }
		% \begin{tablenotes}
  %   		\scriptsize
  %   		\item[*] For TIP-Adapter, the results reported by Zhang \etal~\cite{zhang2022tip} on ImageNet are based on using prompt ensembling as text input. For a fair comparison, we conduct experiments with a single text prompt for TIP-Adapter and report the results here.
		% \end{tablenotes}
		\label{table:distribution_shift}
		
	\vspace{-8pt}
\end{table}

% \begin{table}[tb]
% 	\caption{Combining with state-of-the-art methods on 16-shot ImageNet.}
% 	\centering
% 	\vspace{-0.20cm}
% 	\resizebox{0.5\linewidth}{!}{
% 		\begin{tabular}{l|c}
% 		\hline
% 		\multirow{1}{*}{Methods} & \multirow{1}{*}{ImageNet} \\ 
% 		\hline 
		
% 		Zero-shot CLIP & 66.73 \\
% 		PMT & \textbf{73.07~\textcolor{red}{+ 6.67}} \\ \hline
% 		CoOP~\cite{zhou2022learning} & 72.01  \\
% 		PMT+CoOP & \textbf{73.35~\textcolor{red}{+ 1.34}} \\ \hline
% 		TIP-Adapter~\cite{zhang2022tip} & 73.08 \\
% 		PMT+TIP-Adapter &\textbf{74.28 ~\textcolor{red}{+ 1.20}} \\
		
% 		\hline
% 		\end{tabular}
% 		} 
% 		\label{table:combine_sota}
		
% 	\vspace{-.1in}
% 	\end{table}

\begin{table*}[t]
	\caption{\textbf{Combining with the state-of-the-art methods on 16 shots.} Our mask tuning is synergistic with most existing parameter-efficient tuning methods (\eg, adapter tuning~\cite{zhang2022tip} and prompt tuning~\cite{zhou2022learning}) and can boost about $3\%$ performance on top of them.}
        \vspace{2pt}
	\centering\small
        \setlength{\tabcolsep}{3pt}
	% \vspace{-0.20cm}
	% \resizebox{\linewidth}{!}{
		\begin{tabular}{c|ccccccccccccc}
		\toprule
		\makebox[0.05\textwidth][c]{\rotatebox{45}{Method}} &
		\makebox[0.05\textwidth][c]{\rotatebox{45}{ImageNet}} &
		\makebox[0.05\textwidth][c]{\rotatebox{45}{Caltech101}} &
		\makebox[0.05\textwidth][c]{\rotatebox{45}{FGVCAircraft}} &
		\makebox[0.05\textwidth][c]{\rotatebox{45}{StanfordCars}} & 
            \makebox[0.05\textwidth][c]{\rotatebox{45}{Flowers102}} & 
            \makebox[0.05\textwidth][c]{\rotatebox{45}{OxfordPets}} &
		\makebox[0.05\textwidth][c]{\rotatebox{45}{Food101}} &
		\makebox[0.05\textwidth][c]{\rotatebox{45}{DTD}} &
		\makebox[0.05\textwidth][c]{\rotatebox{45}{EuroSAT}} &
		\makebox[0.05\textwidth][c]{\rotatebox{45}{UCF101}} &
		\makebox[0.05\textwidth][c]{\rotatebox{45}{SUN397}} & 
		\makebox[0.05\textwidth][c]{\rotatebox{45}{Average}}& 
		\makebox[0.05\textwidth][c]{\rotatebox{45}{Gain}}\\
 
		\hline 
		Zero-shot CLIP & 66.73 & 92.94 & 24.72 & 65.32 & 71.34 & 89.21 & 86.06 & 44.39 & 47.60 & 66.75 & 62.50 & 65.23 &- \\
		R-AMT & 73.07 & 97.00 & 58.47 & 85.93 & 98.17 & 93.80 & 87.47 & 74.57 & 91.80 & 86.93 & 76.40 & \textbf{83.96}&{\textcolor{red}{+18.73}}  \\
            \hline
		CoOP~\cite{zhou2022learning} & 72.01 & 95.47 & 43.29 & 82.91 & 96.93 & 91.92 & 84.33 & 69.21 & 86.05 & 82.25 & 74.58 & 79.90 &- \\
		CoOP+R-AMT& {73.35} & 96.70 & 56.37 & 85.63 & 97.83 & 93.20 & 86.13 & 73.03 & 90.20 & 86.87 & 75.45 &\textbf{83.16}&{~\textcolor{red}{+3.26}} \\ 
            \hline
		TIP-Adapter~\cite{zhang2022tip} & 73.08 & 95.63 & 45.20 & 83.04 & 96.15 & 92.66 & 87.31 & 71.57 & 88.53 & 84.24 & 76.21 & 81.24 &- \\
		TIP-Adapter+R-AMT &{74.28} & 96.97 & 61.07 & 86.27 & 97.80 & 94.07 & 87.43 & 74.77 & 91.50 & 86.93 & 76.97 & \textbf{84.37}&{~\textcolor{red}{+3.13}}\\
		
		\bottomrule
		\end{tabular}
		% }
		\label{table:combine_sota_vit}
% 		\begin{tablenotes}
% 		\scriptsize
% 		\item[*] We use these datasets for few shot image classification task. The number of training data and validation data is set to be a specific multiple of the number of classes in training period. Thus, we only demonstrate the number of testing data.
% 		\end{tablenotes}
		
	\vspace{-5pt}
	\end{table*}

\begin{table*}[t]
	\caption{\textbf{Effect of performing masking on different layers.} Attaching a binary mask to the multi-head self-attention layer (\ie, R-AMT) achieves the same performance as R-PMT but with lower computational effort.}
        \vspace{2pt}
	\centering\small
        \setlength{\tabcolsep}{4.8pt}
	% \vspace{-0.20cm}
	% \resizebox{\linewidth}{!}{
		\begin{tabular}{c|ccccccccccccc}
		\toprule
		\makebox[0.05\textwidth][c]{\rotatebox{45}{Method}} &
            % Method &
		\makebox[0.05\textwidth][c]{\rotatebox{45}{ImageNet}} &
		\makebox[0.05\textwidth][c]{\rotatebox{45}{Caltech101}} &
		\makebox[0.05\textwidth][c]{\rotatebox{45}{FGVCAircraft}} &
		\makebox[0.05\textwidth][c]{\rotatebox{45}{StanfordCars}} & 
            \makebox[0.05\textwidth][c]{\rotatebox{45}{Flowers102}} & 
            \makebox[0.05\textwidth][c]{\rotatebox{45}{OxfordPets}} &
		\makebox[0.05\textwidth][c]{\rotatebox{45}{Food101}} &
		\makebox[0.05\textwidth][c]{\rotatebox{45}{DTD}} &
		\makebox[0.05\textwidth][c]{\rotatebox{45}{EuroSAT}} &
		\makebox[0.05\textwidth][c]{\rotatebox{45}{UCF101}} &
		\makebox[0.05\textwidth][c]{\rotatebox{45}{SUN397}} & 
		\makebox[0.05\textwidth][c]{\rotatebox{45}{Average}}& 
		\makebox[0.05\textwidth][c]{\rotatebox{45}{Storage Space}}\\
 
		\hline 
		% OPPL         & 72.76 & 96.93 & 57.79 & 85.51 & 98.05 & 93.29 & 86.20 & 74.09 & 90.60 & 86.50 & 74.78 & 83.32 & 7.67M \\
		% OR-AMT& 72.79 & 96.99 & 57.19 & 85.58 & 97.82 & 93.61 & 87.36 & 74.17 & 90.16 & 86.48 & 75.39 & 83.41 & 7.67M \\
		% PMT          & 72.60 & 97.10 & 59.43 & 85.70 & 98.07 & 93.43 & 85.93 & 74.53 & 92.00 & 87.00 & 72.27 & 83.46 & 28.90M \\
		R-AMT & 73.07 & \textbf{97.00} & 58.47 & 85.93 & 98.17 & 93.80 & 87.47 & 74.57 & \textbf{91.80} & 86.93 & \textbf{76.40} & \textbf{83.96} & 6.7M \\
            R-MMT & \textbf{73.52} & 96.77 & 59.57 & \textbf{86.43} & 98.07 & \textbf{93.83} & 87.40 & 75.73 & 84.07 & 87.70 & 74.23 & 83.39 & 14M \\
		% PPL          & 72.70 & 96.80 & 59.40 & 86.57 & 97.97 & 93.60 & 86.70 & 75.33 & 88.00 & 87.27 & 74.53 & 83.63 & 85.52M \\
            R-PMT & 73.48 & 96.63 & \textbf{60.30} & 86.33 & \textbf{98.27} & 93.77 & \textbf{87.50} & \textbf{75.60} & 88.20 & \textbf{87.33} & 76.12 & \textbf{83.96} & 19M \\
		
		\bottomrule
		\end{tabular}
		% }
		\label{table:different_layes_vit}
% 		\begin{tablenotes}
% 		\scriptsize
% 		\item[*] We use these datasets for few shot image classification task. The number of training data and validation data is set to be a specific multiple of the number of classes in training period. Thus, we only demonstrate the number of testing data.
% 		\end{tablenotes}
		
	% \vspace{-.15in}
        \vspace{-.1in}
	\end{table*}

\subsection{Ablation Studies}
\label{sec:ab}
\noindent
\textbf{Analysis of Masking different layers.}
We conduct ablation studies on masking different layers of the image encoder.~\cref{table:different_layes_vit} shows the results on 16 shots over 11 datasets. Concretely, we apply the binary masks on all weight matrices of convolutional layers and fully connected layers when training R-PMT. For R-AMT, the binary masks are applied on the multi-head self-attention (MHSA) layers, while for R-MMT, the binary masks are applied on the multilayer perceptron (MLP) layers.
R-AMT achieves equal performance with R-PMT on the average of 11 datasets, which surpasses R-MMT 0.57\%. 
But the R-AMT only uses 6.7M for storing the trained model, which is 12.3M less than R-PMT.
Moreover, we find the R-AMT achieves superior performance when there are limited training classes, \eg, EuroSAT.
Thus, we deem the R-AMT to be a more practical method.
 


\noindent
\textbf{Influence of gradient dropout regularity.}
We explore the influence of gradient dropout regularity with 16 shots ImageNet. The experimental results are shown in~\cref{table:graddrop}.
The proposed gradient dropout regularity requires the guidance of KL divergence. Thus, we conduct an ablation study by directly adding the KL loss $\mathcal{L}_{kl}$ with the cross-entropy loss $\mathcal{L}_{ce}$ to training the binary mask, which is termed as AMT+KL loss. 
It shows that if we directly add these two losses, the accuracy drops by 0.68\% on 16-shot ImageNet. Because the $\mathcal{L}_{ce}$ aims to transfer the model to downstream tasks, while the $\mathcal{L}_{kl}$ requires the disparity between the classification logits of AMT and CLIP is not large.
Directly adding $\mathcal{L}_{kl}$ with $\mathcal{L}_{ce}$ limits the transfer ability of AMT.
AgreeGrad~\cite{mansilla2021domain} adopts gradient surgery to solve the domain conflict, but it excessively believes in previous knowledge from KL loss, resulting in performance degradation.
Recently, ProGrad~\cite{zhu2022prompt} proposes a gradient projection method for training text prompts. This gradient projection method and our gradient dropout regularity technique both require the guidance of KL divergence.
Thus, we employ the gradient projection method to train our AMT for comparison, named AMT+ProGrad. AMT+ProGrad surpasses AMT by 0.10\%, but is 0.37\% lower than R-AMT ($l$=1.0).
It indicates the gradient projection technique can help mask tuning. But the improvement is limited since all conflict gradients are forced to be projected in the vertical direction.
The gradient dropout regularity adds some level of randomness to the gradient guided by the KL divergence, which helps the model generalize better to downstream tasks.
% the transfer ability of mask tuning since the conflicts gradients are all forced to be projected in the vertical direction. But the gradient dropout regularity technique adds some level of randomness to the gradient guided by the KL divergence, which helps the model generalize better to downstream tasks.
Moreover, we analyze the influence of $l$ in the gradient dropout regularity technique. 
A smaller $l$ implies a higher probability of CE-related gradient maintenance, which divers the binary masks more sparse.
The R-AMT achieves the best performance on 16-shot ImageNet when $l=1.0$. When $l$ is small than 1.0, The performance degradation is caused by the leak through gradients of $\mathcal{L}_{ce}$, which conflicts with the general knowledge of CLIP.


 \begin{table}[t]
	\caption{\textbf{Ablation studies on gradient dropout regularity strategy.} The proposed gradient dropout regularity can make better use of general knowledge of KL loss while exploring the knowledge of downstream data. ``$l$'' controls the level of agreement in CE Loss.
 % Analysis of the influence of gradient dropout regularity technique on 16-shot ImageNet.
 }
        \vspace{2pt}
	\centering\small
        \setlength{\tabcolsep}{5pt}
	% \vspace{-0.20cm}
	% \resizebox{0.95\linewidth}{!}{
		\begin{tabular}{c|c|c|c|c}
		\toprule
		% \makebox[0.05\textwidth][c]{\rotatebox{0}{Method}}  &
  %           \makebox[0.05\textwidth][c]{\rotatebox{45}{Method}} &
  %           \makebox[0.05\textwidth][c]{\rotatebox{45}{$l$}} &
		% \makebox[0.05\textwidth][c]{\rotatebox{45}{ImageNet}} &
		% \makebox[0.05\textwidth][c]{\rotatebox{45}{Caltech101}} &
		% \makebox[0.05\textwidth][c]{\rotatebox{45}{FGVCAircraft}} &
		% \makebox[0.05\textwidth][c]{\rotatebox{45}{StanfordCars}} & 
  %           \makebox[0.05\textwidth][c]{\rotatebox{45}{Flowers102}} & 
  %           \makebox[0.05\textwidth][c]{\rotatebox{45}{OxfordPets}} &
		% \makebox[0.05\textwidth][c]{\rotatebox{45}{Food101}} &
		% \makebox[0.05\textwidth][c]{\rotatebox{45}{DTD}} &
		% \makebox[0.05\textwidth][c]{\rotatebox{45}{EuroSAT}} &
		% \makebox[0.05\textwidth][c]{\rotatebox{45}{UCF101}} &
		% \makebox[0.05\textwidth][c]{\rotatebox{45}{SUN397}} & 
		% \makebox[0.05\textwidth][c]{\rotatebox{45}{Average}}\\
 
  %        % R-AMT & 1.0/0.3 & 73.07  & 97.00 & 58.47 & 85.93 & 98.17 & 93.80 & 87.47 & 74.57 & 91.80 & 86.93 & 76.40  & 83.96 \\
		% \hline 
		% AMT    & -  & 72.60 & 97.10 & 59.43 & 85.70 & 98.07 & 93.43 & 85.93 & 74.53 & 92.00 & 87.00 & 72.27 & 83.46 \\
  %      R-AMTKL loss& -  & 71.92 & 95.23 & 47.09 & 80.12 & 91.09 & 92.69 & 87.06 & 67.10 & 86.02 & 79.51 & 71.38 & 79.02 \\
  %      R-AMTPCGrad & -  & 72.70 & 96.53 & 53.30 & 84.43 & 96.10 & 93.60 & 87.57 & 73.23 & 91.10 & 85.37 & 75.67 & 82.69 \\
		% R-AMT & 1.0 & 73.07 & 96.89 & 55.81 & 85.90 & 97.07 & 93.97 & 87.47 & 74.13 & 89.65 & 86.07 & 76.40 & 83.31 \\
  %           R-AMT & 0.5 & 72.95 & 97.00 & 57.83 & 85.90 & 98.12 & 94.03 & 86.71 & 74.65 & 91.34 & 86.69 & 75.96 & 83.74 \\
  %           R-AMT & 0.3 & 72.87 & 97.00 & 58.47 & 85.93 & 98.17 & 93.80 & 86.50 & 74.57 & 91.80 & 86.93 & 75.87 & 83.81 \\
  %           R-AMT & 0.1 & 72.67 & 96.87 & 59.00 & 85.90 & 98.07 & 93.57 & 86.04 & 74.53 & 91.97 & 86.84 & 75.38 & 83.71 \\

            Method & $l$ & Accuracy & Gain& Sparsity\\
            \hline 
            Zero-shot CLIP & - & 66.73 & - & -  \\
            \hline 
            AMT         & - & 72.60\std{0.12} & - & 2.64  \\
            AMT+KL loss & - & 71.92\std{0.06} & \textcolor{green}{-0.68} & 2.58 \\
            AMT+AgreeGrad~\cite{mansilla2021domain} & - & 68.82\std{0.09} & \textcolor{green}{-3.78} & 1.73 \\
            AMT+ProGrad~\cite{zhu2022prompt} & - & 72.70\std{0.22} & \textcolor{red}{+0.10} & 2.67  \\
            \hline
            R-AMT & 1.0 & \textbf{73.07}\std{0.10} & \textcolor{red}{+0.47}& 2.45 \\
            R-AMT & 0.8 & 72.97\std{0.13} & \textcolor{red}{+0.37}& 2.50 \\
            R-AMT & 0.5 & 72.95\std{0.19} & \textcolor{red}{+0.35} & 2.56 \\
            R-AMT & 0.3 & 72.87\std{0.05} & \textcolor{red}{+0.27} & 2.59 \\
            R-AMT & 0.1 & 72.67\std{0.12} & \textcolor{red}{+0.07} & 2.61 \\
            
            
		
		\bottomrule
		\end{tabular}
		% }
		\label{table:graddrop}
  \vspace{-10pt}
  \end{table}


 %   \begin{table}[ht]
	% \caption{Analysis on Gradient Drop on ImageNet on 16-shot base-to-new generalization setting.}

	% \centering
	% % \vspace{-0.20cm}
	% \resizebox{0.7\linewidth}{!}{
	% 	\begin{tabular}{c|c|cc|c}
	% 	\toprule
	% 	% \makebox[0.05\textwidth][c]{\rotatebox{0}{Method}}  &
 %  %           \makebox[0.05\textwidth][c]{\rotatebox{45}{Method}} &
 %  %           \makebox[0.05\textwidth][c]{\rotatebox{45}{$l$}} &
	% 	% \makebox[0.05\textwidth][c]{\rotatebox{45}{ImageNet}} &
	% 	% \makebox[0.05\textwidth][c]{\rotatebox{45}{Caltech101}} &
	% 	% \makebox[0.05\textwidth][c]{\rotatebox{45}{FGVCAircraft}} &
	% 	% \makebox[0.05\textwidth][c]{\rotatebox{45}{StanfordCars}} & 
 %  %           \makebox[0.05\textwidth][c]{\rotatebox{45}{Flowers102}} & 
 %  %           \makebox[0.05\textwidth][c]{\rotatebox{45}{OxfordPets}} &
	% 	% \makebox[0.05\textwidth][c]{\rotatebox{45}{Food101}} &
	% 	% \makebox[0.05\textwidth][c]{\rotatebox{45}{DTD}} &
	% 	% \makebox[0.05\textwidth][c]{\rotatebox{45}{EuroSAT}} &
	% 	% \makebox[0.05\textwidth][c]{\rotatebox{45}{UCF101}} &
	% 	% \makebox[0.05\textwidth][c]{\rotatebox{45}{SUN397}} & 
	% 	% \makebox[0.05\textwidth][c]{\rotatebox{45}{Average}}\\
 
	% 	% \hline 
	% 	% AMT         & - & 72.60 & 97.10 & 59.43 & 85.70 & 98.07 & 93.43 & 85.93 & 74.53 & 92.00 & 87.00 & 72.27 & 83.46 \\
 %  %           % R-AMT & 1.0/0.3 & 73.07  & 97.00 & 58.47 & 85.93 & 98.17 & 93.80 & 87.47 & 74.57 & 91.80 & 86.93 & 76.40  & 83.96 \\
 %  %           R-AMTKL loss & - & 71.92 & 95.23 & 47.09 & 80.12 & 91.09 & 92.69 & 87.06 & 67.10 & 86.02 & 79.51 & 71.38 & 79.02 \\
 %  %           % \hline
 %  %           R-AMTPCGrad & - & 72.70 & 96.53 & 53.30 & 84.43 & 96.10 & 93.60 & 87.57 & 73.23 & 91.10 & 85.37 & 75.67 & 82.69 \\
 %  %           \hline
	% 	% R-AMT & 1.0 & 73.07 & 96.89& 55.81 & 85.90 & 97.07 & 93.97 & 87.47 & 74.13 & 89.65 & 86.07 & 76.40 & 83.31 \\
 %  %           R-AMT & 0.5 & 72.95 & 97.00 & 57.83 & 85.90 & 98.12 & 94.03 & 86.71 & 74.65 & 91.34 &86.69 & 75.96 & 83.74 \\
 %  %           R-AMT & 0.3 & 72.87 & 97.00 & 58.47 & 85.93 & 98.17 & 93.80 & 86.50 & 74.57 & 91.80 & 86.93 & 75.87 & 83.81 \\
 %  %           R-AMT & 0.1 & 72.67 & 96.87 & 59.00 & 85.90 & 98.07 & 93.57 & 86.04 & 74.53 & 91.97 & 86.84 & 75.38 & 83.71 \\

 %            Method & $l$ & Base & New &H \\
 %            \hline 
 %            AMT         & - & 72.60\\
 %            R-AMTKL loss & - & 71.92 \\
 %            R-AMTPCGrad & - & 72.70\\
 %            R-AMT & 1.0 & 73.07\\
 %            R-AMT & 0.5 & 72.95 \\
 %            R-AMT & 0.3 & 72.87\\
 %            R-AMT & 0.1 & 72.67\\
            
            
		
	% 	\bottomrule
	% 	\end{tabular}
	% 	}
	% 	\label{table:graddrop_b2n}
	% \end{table}

\noindent
\textbf{Analysis of hard threshold $\alpha$.} We conduct ablation study on the hard threshold $\alpha$ with the initial value of mask weights fixed in~\cref{table:ablation_alpha}. It shows the binary masks are sparser as $\alpha$ gets larger. The R-AMT achieves the best accuracy when $\alpha=5\times10^{-3}$. We deem that some redundant information still has not been masked when $\alpha=4\times10^{-3}$. Thus, this information still influences the performance of the model in the downstream task. When $\alpha=6\times10^{-3}$, some valuable parameters are moved out by the binary masks. It caused performance degradation.
% \textbf{More ablation studies are included in the Supplementary.}
	
 \begin{table}[t]
	\caption{\textbf{Effect of hard threshold $\alpha$ on 16-shot ImageNet}. The threshold determines the sparsity of model.
 % Analysis of the influence of hard threshold $\alpha$ on 16-shot ImageNet.
 }
        \vspace{2pt}
	\centering\small
        \setlength{\tabcolsep}{10pt}
	% \vspace{-0.20cm}
	% \resizebox{0.9\linewidth}{!}{
		\begin{tabular}{c|c|c|c}
		\toprule

            $\alpha$  & $4\times10^{-3}$ &  $5\times10^{-3}$ & $6\times10^{-3}$ \\
            \hline 
            Accuracy      & 72.87\std{0.06}  & \textbf{73.07}\std{0.10}& 72.91\std{0.14}  \\
            Sparsity  & 1.99 & 2.45 & 3.12\\
		\bottomrule
		\end{tabular}
		% }
		\label{table:ablation_alpha}
\vspace{-8pt}
  \end{table}
  
  

% \subsubsection{Analysis of Parameter Prompts}
% \label{sec.analysis}
% \noindent
% \textbf{More ablation studies are included in the Supplementary.}

\section{Related Work}\label{sec::relatedwork}
As discussed above, in this work we explore the problem of living needs prediction, defined as predicting the specific living \textbf{needs} of a user given the \textbf{spatiotemporal} context. Thus, there are two closely related research topics: demand forecasting and spatiotemporal activity prediction.

\subsection{Demand Forecasting} 
Demand forecasting aims at predicting the quantity of a product or service that consumers will purchase. It helps in making informed decisions on inventory management, production scheduling, pricing strategy, etc. The problem of demand forecasting is broad and multifaceted, affecting many different industries, including restaurant~\cite{lasek2016restaurant}, manufacturing~\cite{seyedan2020predictive}, retail~\cite{ren2020demand}, tourism~\cite{song2019review}, energy~\cite{hernandez2014survey}, transportation~\cite{tskeris2011demand}, etc. 

To address the problem of demand forecasting, researchers have proposed various methods which can be broadly classified into three categories: statistical models~\cite{lasek2016restaurant,fattah2018forecasting,huber2017cluster,priyadarshi2019demand,wang2019selection}, machine learning models~\cite{reynolds2013econometric,sellers2010predicting,aishwarya2020food,ma2016demand}, and deep learning models~\cite{raza2017prediction,duncan2015probabilistic,lakshmanan2020sales,kilimci2019improved}. Statistical models, such as exponential smoothing~\cite{silva2019demand}, are well-suited for long-term demand forecasting as they are based on historical trends and patterns. However, they are not adept at handling variations or outliers in the data, making them unsuitable for volatile or short-term demand forecasting. On the other hand, machine learning models for demand forecasting, such as Random Forest based models~\cite{ramya2020advanced,abbasi2019short}, are efficient at short-term demand forecasting, but their performance drops when it comes to long-term forecasting. Deep learning models such as LSTM based models~\cite{lakshmanan2020sales} and GAN based models~\cite{husein2019generative} have the ability to capture complex patterns and dependencies in the data, making them suitable for both short-term and long-term demand forecasting. However, they require a large amount of data to work well. 

Existing demand forecasting methods can not handle the problem of living needs prediction. These methods focus on the overall demand for a particular product or service in a market, but in this work, we aim at predicting the need of a specific consumer. What's more, demand forecasting methods predict demands in several months or years, while in this work we predict a user's need at a specific time and location. 





\subsection{Spatiotemporal Activity Prediction}
 Spatiotemporal activity prediction aims to predict the activity of a user at a given time and location. Previous works have employed various methods to perform spatiotemporal activity prediction. One popular approach is to build a tensor using historical data and then conduct tensor factorization to learn intrinsic association~\cite{fan2019personalized,zheng2010collaborative,bhargava2015and}. For example, Fan \textit{et al.}~\cite{fan2019personalized} propose to integrate tensor factorization with transfer learning for online activity prediction. Additionally, WDGTC~\cite{li2020tensor} proposes a low-rank tensor decomposition and completion framework for passenger flow prediction by introducing L1-norm and Graph Laplacian penalties. Recently, researchers have introduced Graph Convolutional Networks~\cite{kipf2016semi} (GCNs) to achieve high performance in spatiotemporal activity prediction. For example, SA-GCN~\cite{yu2020semantic} develops a Graph Convolutional Network with meta path-based objective function for App-usage prediction task. Furthermore, DisenHCN~\cite{li2022disenhcn} utilizes a heterogeneous hypergraph to model fine-grained user similarities, resulting in significant performance gains.

However, existing works on spatiotemporal activity prediction focus on predicting the specific activities of people, while our work focuses on the general living needs which are the driving force behind specific consumption behaviors. What's more, these works mainly focus on either online or offline activities, but in our work, we predict living needs can be satisfied both in store (offline) and via delivery (online) by different kinds of life services, which is beyond the capabilities of existing methods. 
\section{Conclusion and Future Work}\label{sec::conclusion}
In this work, we approach the new problem of living needs prediction, which is critical in life services platforms.
We present the NEON system in Meituan, consisting of three phases, feature mining, feature fusion, and multitask prediction. 
Large-scale online A/B testing in three downstream applications, along with extensive offline evaluation, strongly confirm the effectiveness of our system.
As for future work, we plan to test NEON's performance in more downstream applications.

\section*{Acknowledgement}
This work is supported in part by National Key Research and Development Program of China under 2022YFB3104702. This work is supported in part by National Natural Science Foundation of China under 62272262, 61971267, and 61972223. This work is supported in part by a grant from the Guoqiang Institute, Tsinghua University under 2021GQG1005. This work is supported in part by Beijing National Research Center for Information Science and Technology. This work is also supported by Meituan.
\clearpage
\balance

\bibliographystyle{ACM-Reference-Format}
\bibliography{bibliography}
\balance
\clearpage

\appendix
\section{APPENDIX FOR REPRODUCIBILITY}
\subsection{Model Implementation Details}\label{sec::implementation}
In this section we provide implementation details of our system. Due to the massive amount of training data and the high demand for low latency in our online system, we simplify each network within the framework. In our final implementation, feature merging network $h^M$ is a one-layer fully-connected network of 120 output units (120D FC). User preference network $h^U$ is a 340D FC. Shared network $E^S$ and expert networks $E^W$/$E^N$ are 256D FCs. Need prediction network $t^N$ is a 10D FC. In-store/delivery classification network $t^W$ is a two-layer perceptron, which has 10 hidden units and 2 output units. Complexifying these neural networks can help us further optimize performance, but at the same time, it would lead to an increase in system latency. The activation function of all mentioned networks is ReLU~\cite{glorot2011deep}. We adopt batch normalization~\cite{ioffe2015batch} right after $h^M$, $h^U$, $E^S$, $E^P$, and $E^N$. Following such implementation, we conduct rich online and offline experiments to prove the effectiveness of our model, which are shown in Section~\ref{sec::experiments} and Section~\ref{sec::online}.
\subsection{Dataset}\label{sec::dataset}
We conduct our offline experiment on a real-world dataset at the scale of billions. The dataset is a sampling of all purchase records in 2022 on the platform. We sample the records according to the percentage of purchases of each kind of life service. The dataset includes more than 7 billion real purchase records from 65 million users.
Each instance in the dataset includes user profile, time, location, and other real-time environmental factors, and the kind of life service the user purchase. The type of life service consumed by the user reflects their actual living need in the spatiotemporal context. Following existing works~\cite{cheng2012fused,palumbo2018knowledge}, we randomly sample 80\% of the dataset as the training set, and 20\% as the test set.
\subsection{Metrics}\label{sec::metrics}
We design a metric named Sort Accuracy (SA) to measure the performance of systems on our problem. The metric SA can be defined as follows.
For user scene $i$, our system outputs scores for all kinds of living needs. We sort the living needs by their scores and get a list. We define \textit{Relative Ranking Error} as the difference between the actual ranking position and the ideal ranking position (the first position) of the ground truth need, and further define \textit{Maximum Ranking Error} as the maximum relative ranking error any system can give for a user scene (the number of categories of user needs - 1). For example, in our system which handles 10 types of living need, for user scene $i$, if the ground truth need is ranked third, then the relative ranking error is 2 (2 = 3-1), and the maximum ranking error is 9 (9 = 10-1). Then Sort Accuracy (SA) can be defined as follows,
\begin{equation}
\text{SA}=\text{average}_{i\in T}\left(1-\frac{\text{Relative Ranking Error}}{\text{Maximum Ranking Error}}\right)
\end{equation}
where $T$ denotes the testing set on which the metric is calculated. We first calculate $1-\text{Relative Ranking Error}/\text{Maximum Ranking Error}$ of every user scene in the testing set and then compute the average. We then define Via-delivery Sort Accuracy (VDSA) and In-store Sort Accuracy (ISSA) to measure systems' performance on delivery and in-store living needs.
\begin{equation}
\text{VDSA}=\text{average}_{i\in T_{VD}}\left(1-\frac{\text{Relative Ranking Error}}{\text{Maximum Ranking Error}}\right)
\end{equation}
\begin{equation}
\text{ISSA}=\text{average}_{i\in T_{IS}}\left(1-\frac{\text{Relative Ranking Error}}{\text{Maximum Ranking Error}}\right)
\end{equation}
$T_{VD}$ and $T_{IS}$ denote sets of testing samples where the ground truth needs-meeting way is via delivery and in store, respectively. In the following, we will use these metrics to measure the living needs prediction performance of our system and baseline systems.
\subsection{Baselines}\label{sec::baselines}
To illustrate the effectiveness of our system, we compare it with two baselines widely in actual production environments, including \textbf{DIN~\cite{zhou2018deep}}, \textbf{DNN~\cite{cheng2016wide}}, \textbf{DCN~\cite{wang2017deep}}, 
\textbf{ESMM~\cite{ma2018entire}},
and \textbf{MMOE~\cite{ma2018modeling}}. We will provide a detailed description of these baselines in the appendix. DIN is a recommendation algorithm that leverages deep neural networks to analyze users' historical behavior and make predictions about their potential interests. It uses an attention-based mechanism to weigh the importance of different historical behaviors for predicting the current interest of a user. As for DNN, We follow the design of Wide \& Deep learning to build a Deep Neural Network (DNN) based system for our task. It can learn both simple and complex relationships in the data. DCN is proposed to keep the benefits of a DNN model while introducing a cross network that is more efficient in learning certain bounded-degree feature interactions. It applies feature crossing at each layer, and it doesn't require manual feature engineering, adding minimal extra complexity to the DNN model. ESMM estimates post-click conversion rates for recommendation systems by using sequential user actions and a feature representation transfer learning strategy to alleviate sample selection bias and data sparsity. MMoE is a multi-task learning approach that learns to model task relationships from data. It adapts the Mixture-of-Experts (MoE) structure to multi-task learning by sharing the expert submodels across all tasks, while also having a gating network trained to optimize each task. 
For a clear comparison, the input features for all baselines are kept the same as those for our model.
\subsection{Additional Ablation Study}
To gain a deeper understanding of the impact of each component design of our model, we also conduct ablation studies with a particular focus on the effects of the group behavior pattern features and the spatiotemporal context features. When removing the group behavior pattern features from the model input, the performance decreased by 1.06\%. When removing the spatiotemporal context features from the model input, the performance decreased by 1.46\%. These results provide compelling evidence of the pivotal role that these two features play in accurately predicting users' daily living needs.



\end{document}
