\documentclass{ecai}

\usepackage{graphicx}
\usepackage{latexsym}
\usepackage{xcolor}
\usepackage{amsfonts}
\usepackage{amssymb}
\usepackage{amsmath}
\usepackage{amsthm}
\usepackage{pifont}
\usepackage{enumerate}

\newtheorem{thm}{Theorem}[section]
\newtheorem{proposition}[thm]{Proposition}
\newtheorem{lemma}[thm]{Lemma}
\newtheorem{corollary}[thm]{Corollary}

\newcommand{\yes}{{\color{teal}\checkmark}}
\newcommand{\no}{{\color{red} \text{\sffamily X}}}
\newcommand{\util}{\textsc{Util}}
\newcommand{\egal}{\textsc{Egal}}

\newtheorem{definition}[theorem]{Definition}
\newtheorem{example}[theorem]{Example}
\newtheorem{remark}[theorem]{Remark}

\DeclareMathOperator*{\argmin}{arg\,min}
\DeclareMathOperator*{\argmax}{arg\,max}

\begin{document}

\begin{frontmatter}

\title{Settling the Score: Portioning with Cardinal Preferences}

\author[1,2]{\fnms{Edith}~\snm{Elkind}}
\author[3]{\fnms{Warut}~\snm{Suksompong}}
\author[1]{\fnms{Nicholas}~\snm{Teh}}

\address[1]{University of Oxford, UK}
\address[2]{Alan Turing Institute, UK}
\address[3]{National University of Singapore, Singapore}

\begin{abstract}
We study a portioning setting in which a public resource such as time or money is to be divided among a given set of candidates,
and each agent proposes a division of the resource. We consider two families of aggregation rules for this setting---those based on coordinate-wise aggregation and those that optimize some notion of welfare---as well as the recently proposed Independent Markets mechanism. We provide a detailed analysis of these rules from
an axiomatic perspective, both for classic axioms, such as strategyproofness and Pareto optimality, and for novel axioms, which aim to capture proportionality in this setting. 
Our results indicate that a simple rule that computes the average of all proposals satisfies many of our axioms, including some that are violated by more sophisticated rules.
\end{abstract}
\end{frontmatter}

\section{Introduction}
A town council has just received its annual funding from the government, and it needs to determine how to split the budget among constructing new facilities, keeping the streets clean, and ensuring safety in public places.
The mayor is in favor of making decisions democratically, so she asks each resident of the town to propose a division of the budget.
After having collected the proposals, how should the council aggregate them into an actual allocation?

In the problem of \emph{portioning}, the aim is to divide a resource among a given set of candidates.
As illustrated in the example above, a prominent application of portioning is \emph{participatory budgeting}, a democratic framework initiated in Porto Alegre, Brazil in 1989 and used in over 7,000 cities around the world~\cite{PB2022}.\footnote{Participatory budgeting is a subject of much recent interest in computational social choice \cite{aziz2018proportionally,baumeister2022time,benade2017preference,chen2022participatory,fairstein2022welfare,goyal2023pb,hershkowitz2021district,jain2021partition,sreedurga2022maxmin,talmon2019budget}. 
As we discuss in the section on related work, most of the participatory budgeting literature focuses on the discrete setting.
}
Besides allotting a budget, portioning can also be used to share \emph{time}, for example, when deciding upon the proportion of time to spend on various activities at a conference (e.g., research talks, panels, social gatherings) or different types of music at a graduation party (e.g., classical, rock, jazz).

Most prior works on portioning assumed that each voter's preferences can be represented by a ranking~\cite{airiau2019portioning}, or by an approval ballot \cite{aziz2020fairmixing,bogomolnaia2005collective,duddy2015fairsharing}.
However, in some portioning scenarios these preference formats cannot fully describe the agents' desires.
For instance, if a student wants both jazz and rock music to be played at the graduation party, but with more time devoted to rock, her preference is not captured by a ranking or a list of approvals.
Likewise, a conference attendee who would like 75\% of the time to be spent on research talks, 15\% on panels, and 10\% on social gatherings ranks these activities in the same way as another attendee who prefers a 40\%--35\%--25\% split, but the actual preferences of these two attendees are quite different.

Recently, Freeman et al.~\cite{freeman2021truthfulbudget} studied portioning with cardinal preferences, wherein each voter is allowed to propose a division of the resource.
They observed that, even though the rule that maximizes the utilitarian social welfare is known to be strategyproof (for a specific tie-breaking rule) \cite{goel2019knapsack,lindner2008allocating}, it tends to put too much weight on majority preferences.
In light of this observation, they introduced the \emph{independent markets (IM)} mechanism, which is strategyproof and, in some sense, more proportional.
However, while strategyproofness is an important consideration, there may be scenarios where other features of aggregation rules are just as---if not more---desirable. Thus, to  identify a suitable aggregation rule for a specific application, it would be useful to (1) build a catalogue of axioms for the portioning setting, and (2) determine which of these axioms are satisfied by popular aggregation rules. 

\paragraph{Our Contributions}
We consider a diverse set of axioms for portioning with cardinal preferences. Besides classic axioms, such as strategyproofness and Pareto optimality, we put forward two novel proportionality axioms, namely, score-unanimity and score-representation (see Section~\ref{sec:preliminaries} for definitions).
We then conduct a systematic study of aggregation rules with respect
to these axioms. 
We focus on two families of portioning rules---those that are based on coordinate-wise aggregation and those that optimize some notion of welfare---as well as the recently proposed Independent Markets mechanism \cite{freeman2021truthfulbudget}.
We also include observations regarding relationships between the axioms.
Table~1 summarizes our results. 
An overview and discussion of our results can be found in Section \ref{sec:overview}.

\paragraph{Related Work} While there is a large body of work on participatory budgeting, most of it focuses on the discrete setting, where each project is either implemented in full or not implemented at all~\cite{aziz-shah}. Our model, where there is a unit of budget to be split arbitrarily among the projects, is usually referred to as portioning \cite{airiau2019portioning,aziz2020fairmixing,bogomolnaia2005collective,duddy2015fairsharing}. Within the portioning literature, 
only a few papers consider cardinal preferences. The ground-breaking work of Freeman et al.~\cite{freeman2021truthfulbudget} focuses on designing strategyproof mechanisms for this setting, and proposes the Independent Market mechanism. The follow-up work of Caragiannis et al.~\cite{caragiannis2022truthful} studies proportionality guarantees that can be obtained by truthful mechanisms, and provides approximation bounds. In a very recent paper, Goyal et al.~\cite{goyal2023pb} study approximation guarantees offered by mechanisms with low sample complexity.

Other related lines of work include probabilistic models in social choice \cite{fishburn1975probcomment,intriligator1973probsc,rice1977comment}, in which the output is a probability distribution over candidates.
There, however, a single candidate is to be chosen according to the probability distribution, in contrast to our setting where we portion among several alternatives.
Another relevant topic is probabilistic opinion pooling \cite{clemen1989forecasts,genest1986combining}, where the goal is to aggregate probabilistic beliefs which may represent, e.g., forecasts.

\begin{table*}[ht] 
    \label{tab:summary}
\begin{center}
\begin{tabular}{ |c||c|c|c|c|c||c|c||c| } 
 \hline
 \multicolumn{1}{|c||}{} & 
 \multicolumn{5}{|c||}{ Coordinate-wise } &
 \multicolumn{2}{|c||}{Welfare-based} &
 \multicolumn{1}{|c|}{Other}
 \\
 \hline
 $F$ & Sum & Max & Min & Med & Prod & \util{} & \egal{} & IM \\
 \hline
 Score-unanimity & \yes & \no$^\dagger$ & \no$^\dagger$ &  \no$^\ddagger$ & \no & \yes & \yes & \no$^\dagger$ \\
 \hline
 Score-monotonicity & \yes & \yes & \yes & \yes & \yes & \yes & $?^\ddagger$ & \yes \\
 \hline
 Proportionality & \yes & \no$^*$ & \no$^\ddagger$ & \no$^*$ & \no$^\ddagger$  & \no$^*$ &  \no$^*$ & \yes
 \\
  \hline
 Score-representation & \yes & \no$^\ddagger$ & \no & \no$^*$ & \no & \no$^\ddagger$ & \no$^*$ &  \no 
 \\
 \hline
 Strategyproofness & \no & \no & \no & \no & \no & \yes & \no &  \yes \\
  \hline
 Participation & \yes & \yes & \yes & $?^\ddagger$ & \no & \yes & \yes &  \yes \\
 \hline
 Pareto optimality & \no$^{\ddagger}$  & \no$^\dagger$ & \no$^\dagger$ & \no$^\ddagger$ & \no & \yes & \yes &  \no$^\dagger$ \\
  \hline
 Range-respecting & \yes & \no$^\dagger$ & \no$^\dagger$ & \no$^\ddagger$  & \no & \yes & \yes & \no$^\dagger$ \\
 \hline
 Reinforcement & \yes & \yes & \yes & $?^\ddagger$ & \no & \yes & \yes & \yes \\
 \hline
\end{tabular}
\caption[Summary]{Summary of our results.
For most of the negative results, we provide a counterexample for the simplest case, but they can be easily extended by cloning agents or adding dummy candidates.
The asterisk symbol ($^*$) indicates that the axiom is satisfied for $n=2$, but may fail when $n \geq 3$ (even if $m = 2$).
The dagger symbol ($\dagger$) indicates that the axiom is satisfied for $m=2$, but may fail when $m \geq 3$ (even if $n = 2$).
The double dagger symbol ($\ddagger$) indicates that the axiom is satisfied when $\min\{n,m\} = 2$; when used with \no, it may fail in more general cases; when used with ?, it indicates that the case $\min\{n,m\} \ge 3$ remains open.   
Some of the results on \util{} and IM were obtained by Freeman et al.~\cite{freeman2021truthfulbudget}.}
\end{center}
\end{table*}

\section{Preliminaries} \label{sec:preliminaries}
We first present the model of portioning with cardinal preferences, and then introduce the axioms and rules that we will study.
\subsection{Model}
We are given a set $N = \{1, \dots, n\}$ of $n$ \emph{agents} (or \emph{voters}), and a set $C = \{p_1, \dots, p_m\}$ of $m$ \emph{candidates}. 
Each agent $i \in N$ has a normalized preference score vector $\mathbf{s}_i \in (\mathbb{R}_{\geq 0})^m$ over the candidates, where for each $i\in N$ we have $\mathbf{s}_i=(s_{i1}, \dots, s_{im})$ and $\sum_{j \in [m]} s_{ij} = 1$. An {\em instance} of our problem is the vector $\mathcal{I} = (\mathbf{s}_1,\dots,\mathbf{s}_n)$, also referred to as a \emph{preference profile}. For any score vector $\mathbf{x} = (x_1,\dots,x_m)$, agent $i$'s \emph{disutility function} is defined as $d_i(\mathbf{x}) = \sum_{j \in [m]} |s_{ij} - x_j|$, which is the $\ell_1$ distance between the agent's score vector $\mathbf{s}_i$ and $\mathbf{x}$. 
Given an instance $\mathcal I$, we aim to find a vector $\mathbf{x}$ with $\sum_{j\in [m]} x_j = 1$ that reflects the agents' collective preferences. To do so, we use \emph{aggregation rules}, which are defined as follows.

\begin{definition}[Aggregation rule]
    An \emph{aggregation rule} $F$ is a function $F : (\mathbb{R}_{\ge 0})^{m \times n} \rightarrow (\mathbb{R}_{\ge 0})^m$ that maps a preference profile $\mathcal{I} \in (\mathbb{R}_{\ge 0})^{m \times n}$ to an outcome vector $\mathbf{x} \in (\mathbb{R}_{\ge 0})^m$.
\end{definition}

\subsection{Axioms}
We begin by introducing a basic property, which states that if all agents unanimously agree on a score for a particular candidate, then, in the outcome, this candidate should get exactly that score.

\begin{definition}[Score-unanimity]
    An aggregation rule $F$ is \emph{score-unanimous} if, 
    for every instance ${\mathcal I}=(\mathbf{s}_1, \dots, {\mathbf s}_n)$ such that for some $j\in [m]$, 
    $\gamma\in [0, 1]$ it holds that 
    $s_{ij} = \gamma$ for all $i \in N$,
    the score vector $\mathbf{x} = F({\mathcal I})$ satisfies $x_j = \gamma$.
\end{definition}

Next, we consider another intuitive property, which states that if a single agent increases the score allocated to a particular candidate (while decreasing her scores for other candidates), then this candidate's score in the outcome cannot decrease. 

\begin{definition}[Score-monotonicity] \label{defn-score_monotonicity}
    An aggregation rule $F$ is \emph{score-monotone} if the following holds: for any two instances 
    $\mathcal{I}=({\mathbf s}_1, \dots, {\mathbf s}_n)$ and 
    $\mathcal{I}'=({\mathbf s}'_1, \dots, {\mathbf s}'_n)$ with $F(\mathcal{I}) = \mathbf{x}$ and $F(\mathcal{I}') = \mathbf{x}'$
    such that 
    for some $i \in N$, $j \in [m]$ we have
    (1) $s_{ij} < s'_{ij}$;
    (2) $s_{ij'} \geq s'_{ij'}$ for all $j' \in [m]\setminus \{j\}$; and 
    (3) $s_{i'j'}=s'_{i'j'}$ for all $i'\in N\setminus\{i\}$, $j'\in [m]$,
    it holds that $x_{j} \leq x_{j}'$. 
\end{definition}

Another notion that has recently been studied in the context of portioning is \emph{proportionality}~\cite{freeman2021truthfulbudget}. 
This property requires that when agents are \emph{single-minded}, i.e., each agent places all of her score on a single candidate, the outcome score for each candidate equals the proportion of agents that favor this candidate. 
However, agents are rarely single-minded, so an appropriate notion of proportionality (or, more broadly, representation) for general preferences is needed.
We formulate one such notion for the cardinal preference setting, and refer to it as \emph{score-representation}. 

\begin{definition}[Score-representation] \label{defn-score_representation}
    An aggregation rule $F$ satisfies \emph{score-representation} if
    for every instance ${\mathcal I}=(\mathbf{s}_1, \dots, \mathbf{s}_n)$ 
    such that for some $V\subseteq N$, $j\in [m]$, 
    $\gamma\in [0, 1]$ it holds that 
    $s_{ij} \geq \gamma$ for all $i \in V$,
    the score vector $\mathbf{x} = F({\mathcal I})$ satisfies 
    $x_j \geq \gamma \cdot \frac{|V|}{n}$.
\end{definition}

Another important property of aggregation rules is strategyproofness (see, e.g., \cite{freeman2021truthfulbudget}): agents should not be able to lower their disutility by
misreporting their score vector.

\begin{definition}[Strategyproofness] \label{defn-strategyproofness}
    An aggregation rule $F$ is \emph{strategyproof} if, for any two instances $\mathcal{I}$ and $\mathcal{I}'$ such that $\mathcal{I}'$ is obtained from $\mathcal{I}$ by replacing agent $i$'s score vector $\mathbf{s}_i$ with another score vector $\mathbf{s}'$, if $F(\mathcal{I}) = \mathbf{x}$ and $F(\mathcal{I}') = \mathbf{x}'$, then $d_i(\mathbf{x}) \leq d_i(\mathbf{x}')$.
\end{definition}

A related notion is participation. This property states that each agent weakly prefers voting truthfully to withdrawing from the aggregation process. In numerous contexts (particularly for elections), this property incentivizes higher voter turnout.

\begin{definition}[Participation] \label{defn-participation}
    An aggregation rule $F$ satisfies \emph{participation} if, for any two instances $\mathcal{I}$ and $\mathcal{I}'$ such that $\mathcal{I}'$ is obtained from $\mathcal{I}$ by adding an additional agent $i$, if $F(\mathcal{I}) = \mathbf{x}$ and  $F(\mathcal{I}') = \mathbf{x}'$, then $d_i(\mathbf{x}) \geq d_i(\mathbf{x}')$.
\end{definition}

We also consider Pareto optimality, which is 
a basic notion of efficiency.
\begin{definition}[Pareto optimality]
    An aggregation rule $F$ is \emph{Pareto optimal} (PO) if, for any instance $\mathcal I$ and the outcome $\mathbf{x}=F({\mathcal I})$, there does not exist another outcome $\mathbf{x}'$ such that (1) $d_i(\mathbf{x}') \leq d_i(\mathbf{x})$ for all $i \in N$ and (2) $d_i(\mathbf{x}') < d_i(\mathbf{x})$ for some $i \in N$.
\end{definition}

The last two axioms we consider were introduced by Freeman et al.~\cite{freeman2021truthfulbudget}.
\begin{definition}[Range-respect]
    An aggregation rule $F$ is \emph{range-respecting} (RR) if
    for any instance $\mathcal I$, the outcome $\mathbf{x}=F({\mathcal I})$, and
    for all $j \in [m]$ it holds that 
    $\min_{i\in N}s_{ij}\le x_j\le \max_{i\in N}s_{ij}$.
\end{definition}

\begin{definition}[Reinforcement]
    An aggregation rule $F$ satisfies \emph{reinforcement} if, for any two instances $\mathcal{I}=(\mathbf{s}_1, \dots, \mathbf{s}_n)$ and $\mathcal{I}'=(\mathbf{s}'_1, \dots, {\mathbf s}'_{n'})$ such that $F(\mathcal{I}) = F(\mathcal{I}') = \mathbf{x}$, for the instance $\mathcal{I}^* = (\mathbf{s}_1, \dots, \mathbf{s}_n, \mathbf{s}'_1, \dots, {\mathbf s}'_{n'})$ we have $F(\mathcal{I}^*) = \mathbf{x}$.
\end{definition}

\begin{remark}
The axioms of score-unanimity, score-representation, Pareto optimality, and range-respect can be defined for outcomes rather than voting rules: e.g., we say than an outcome $\mathbf x$ for an instance ${\mathcal I}=(\mathbf{s}_1, \dots, \mathbf{s}_n)$ is range-respecting if 
$\min_{i\in N}s_{ij}\le x_j\le \max_{i\in N}s_{ij}$
for all $j \in [m]$ 
(and similarly for other axioms).
\end{remark}

\subsection{Aggregation Rules}
We focus on two classes of rules, namely, (1) rules that are based on \emph{coordinate-wise aggregation} and (2) rules that are based on \emph{welfare optimization}. In addition, we will also consider the independent markets mechanism
of Freeman et al.~\cite{freeman2021truthfulbudget}.

\paragraph{Coordinate-wise Aggregation Rules}
We start by defining the class of \emph{coordinate-wise} aggregation rules. 
\begin{definition} \label{defn:coord_rules}
    We say that an aggregation rule $F$ is \emph{coordinate-wise}
    if for each $n\ge 1$ there is a {\em coordinate-aggregation function} $f_n: (\mathbb{R}_{\ge 0})^n \rightarrow \mathbb{R}_{\ge 0}$
    such that, given an instance $\mathcal{I} = (\mathbf{s}_1,\dots,\mathbf{s}_n)$, 
    the function $F$ outputs a vector $\mathbf x$ 
    that satisfies
    $x_j = \frac{f_n(s_{1j}, \dots, s_{nj})}{\sum_{j' \in [m]} f_n(s_{1j'}, \dots, s_{nj'})}$ for each $j \in [m]$.
\end{definition}
For each $j \in [m]$, we refer to $y_j = f_n(s_{1j}, \dots, s_{nj})$ as the \emph{pre-normalization value} for $x_j$. The vector $\mathbf{y} = (y_1,\dots,y_m)$ is called the \emph{pre-normalization vector}. In what follows, we will omit
the subscript~$n$ and write $f$ instead of $f_n$.

We study five natural coordinate-wise aggregation rules, where $f$ is, respectively, the 
median (if the number of agents is even, we take the average of the two middle scores), sum, maximum, minimum, or product function (for the last two rules, if $y_j = 0$ for all $j \in [m]$, we set $x_j = 1/m$ for all $j\in [m]$). 
For brevity, we refer to these rules as med, sum, max, min, and prod rules, respectively.
These rules 
are attractive, because they are intuitive and efficiently computable.

\paragraph{Welfare-based Aggregation Rules}
We also consider rules that are based on welfare optimization. 
In particular, we focus on two popular welfare criteria: (1) maximizing utilitarian welfare (\util{}), i.e., minimizing $\sum_{i\in N}d_i(\mathbf{x})$, and (2) maximizing egalitarian welfare (\egal{}), i.e., minimizing $\max_{i\in N}d_i(\mathbf{x})$. 
Note that Nash welfare is not well-defined in this setting, as we are considering disutilities.\footnote{For example, it has been observed that there is no natural equivalent of Nash welfare in the fair allocation of chores \cite{ebadian2022chores,freeman2020chores}.}

For \util{}, tie-breaking is important. 
Following Freeman et al.~\cite{freeman2021truthfulbudget}, we break ties in favor of the \emph{maximum entropy division}. Specifically, we assume that \util{} outputs the utilitarian welfare-maximizing outcome $\mathbf{x}$ that minimizes the quantity $\sum_{j \in [m]} (x_j - 1/m)^2$,
i.e., the $\ell_2$-distance to the uniform distribution $\mathbf{x}_u = (1/m, \dots, 1/m)$.
Importantly, a \util{} outcome is PO, and can be computed in polynomial time. However, 
the \util{} rule fails proportionality~\cite{freeman2021truthfulbudget}.

For \egal{}, if there are multiple outcomes that maximize the egalitarian welfare, then we break ties in a ``leximin'' manner.
That is, we minimize the largest disutility, then subject to that, minimize the second-largest disutility, and so on.
From this definition, it is clear that \egal{} satisfies PO.
We make two further observations.

\begin{thm}
    \emph{\egal{}} can be computed in polynomial time.
\end{thm}
\begin{proof}
    Similar to \cite{airiau2019portioning}, we formulate a series of linear programs (LP) for finding an \egal{} outcome. Let the objective function be
    \begin{equation*}
        \text{minimize } \xi
    \end{equation*}
    \noindent
    subject to the following constraints:
    
    (1) $\sum_{j \in [m]} x_j = 1$;
        
    (2) $x_j \geq 0 \text{ for each $j\in [m]$ and } \xi \geq 0$;
    
    (3) $z_{ij} \geq s_{ij} - x_j \text{ and } z_{ij} \geq x_j - s_{ij}$ for each $i\in N$, $j\in [m]$;
    
    (4) $\sum_{j \in [m]} z_{ij} \leq \xi$ for each $i\in N$.\\
    
    \noindent This allows us to minimize the largest disutility~$\xi$. There is an agent $i$
    that has disutility $\xi$ in every leximin outcome: 
    indeed, if for every $i\in N$ there is an outcome in which $i$ incurs disutility less than $\xi$ and every other agent incurs disutility at most $\xi$, then, by averaging these outcomes across all $i\in N$, we obtain an outcome in which every agent's disutility is less than $\xi$, contradicting the choice of $\xi$.
    To find such an agent, for each $i\in N$, we formulate an LP that computes the maximum $\delta$ for which there exists an outcome such that agent~$i$ incurs disutility at most $\xi - \delta$ while every other agent incurs disutility at most $\xi$; an agent with the desired property will return $\delta = 0$.
    We fix the disutility of this agent to $\xi$, and continue by finding the second largest disutility, and so on.
    The total number of LPs is $O(n^2)$.
\end{proof}

\begin{lemma} \label{lem:welfare-egal_is_sum}
    When $n=2$, the output of the sum and med rules is an \emph{\egal{}} outcome.
\end{lemma}
\begin{proof}
    In the case of $n=2$, the sum rule is equivalent to the median rule.
    Thus, we focus on proving the result for the sum rule.
    Let $\mathbf{x}$ be the output of the sum rule. Then, for each $j \in [m]$, we have that $x_j = \frac{s_{1j}+s_{2j}}{2}$.
    This means that $d_1(\mathbf{x}) = d_2(\mathbf{x})$, and $d_1(\mathbf{x}) + d_2(\mathbf{x}) = \sum_{j\in [m]} |s_{1j} - s_{2j}|$. 
    Since $d_1(\mathbf{x}') + d_2(\mathbf{x}') \ge \sum_{j\in [m]} |s_{1j} - s_{2j}|$ for any outcome $\mathbf{x}'$, $\mathbf{x}$ is an \egal{} outcome.
\end{proof}

Since there may be multiple \egal{} outcomes even after the leximin tie-breaking, we assume for convenience (in light of Lemma~\ref{lem:welfare-egal_is_sum}) that for $n = 2$, the output of \egal{} is the same as that of the sum rule.
Our results for $n\ge 3$ will not depend on this choice, and we allow \egal{} to break ties in any consistent manner.

\paragraph{Independent Markets (IM) mechanism} 

Freeman et al.~\cite{freeman2021truthfulbudget} put forward two aggregation mechanisms that both rely on introducing \emph{phantoms}. The first of these mechanisms is called the Independent Markets mechanism, and is described below. The second one is equivalent to \util{} with the maximum entropy division tie-breaking rule (defined earlier). 

\begin{definition} \label{defn-IM}
For each $c \in \mathbb{R}_{\ge 0}$, the 
{\em $c$-coordinate-wise median} of candidate 
$p_j$ is defined as the median of the $2n+1$
values $\{0, \frac{c}{n}, \frac{2c}{n}, \dots, \frac{(n-1)c}{n}, c, s_{1j}, \dots, s_{nj}\}$
(i.e., the agents' scores for $p_j$ and $n+1$ ``phantoms'' that are uniformly distributed on $[0, c]$). The mechanism starts with $c=0$
and continuously increases $c$, stopping when the sum of all candidates' $c$-coordinate-wise medians is $1$ (if a phantom score is higher than $1$, it is taken to be $1$). It then outputs the vector of $c$-coordinate-wise medians.
\end{definition}

\section{Overview and Discussion} \label{sec:overview}
Our results offer several insights on portioning rules.
As shown in Table \ref{tab:summary}, the most promising rules with respect to the axioms that we study are the sum rule and \textsc{Util}, with the trade-off being that the sum rule fails strategyproofness and Pareto optimality whereas \textsc{Util} fails proportionality and score-representation. 
Additionally, while the IM mechanism satisfies both strategyproofness and proportionality, it fails other intuitive properties such as score-unanimity, range-respecting, and Pareto-optimality; these failures can lead to highly counterintuitive outcomes for the voters and cast doubt on the IM mechanism as an aggregation method.

These trade-offs between various rules may also be used to inform decision-making in a wide range of settings.
For instance, consider a scenario where a conference organizer needs to divide time among different activities at a conference (paper presentations, poster sessions, social activities, and so on). 
In this case, it is likely difficult for an attendee to accurately predict what other attendees’ preferences are, or what fraction of attendees might prefer which distribution of activities, making strategyproofness arguably less important as a consideration. 
On the other hand, strategyproofness could be more important in smaller-scale settings where voters know each other well, e.g., portioning within a family or a small organization. 
In such a setting, it may not be crucial that the outcome is exactly proportional in the way that the proportionality axiom requires. 
Additionally, intuitive properties such as score-unanimity and range-respecting may be essential in settings where votes are revealed: for example, if all voters vote $0.8$ on a certain activity but the rule outputs $0.6$ on this activity, this may well lead to dissatisfaction among voters regarding the use of that voting rule.

\section{Score-Unanimity and Range-Respect} \label{sec:score_unanimity_rr}
In this section, we focus on two
basic properties defined
in Section~\ref{sec:preliminaries}: score-unanimity and range-respect. The former states that if all agents assign a particular candidate $p_j$ the same score, then in the outcome $p_j$ should get exactly that score. The latter mandates that, in the outcome, every candidate's score should lie between the highest and the lowest score that this candidate obtains from the agents.

It is easy to see that if all agents give the exact same score $\gamma$ to a candidate $p_j$, then in the outcome of any RR rule $F$ the score of $p_j$ has to be $\gamma$, and hence $F$ is also score-unanimous.
In fact, as we will show later in Section~\ref{sec:po}, any PO outcome is also RR. 
We will now give a direct proof of a more general result.

\begin{thm} \label{thm:scoreunanimity-PO_implies_scoreunanimity}
    Any aggregation rule that is RR or PO is also score-unanimous.
\end{thm}
\begin{proof}
    The case of RR rules has been considered above. Now, let $\mathbf{x}$ be an outcome returned by a PO rule $F$ on an instance $\mathcal I$. Suppose for a contradiction that $F$ is not score-unanimous on $\mathcal I$. 
    This means that for some $j \in [m]$ and $\gamma \in [0,1]$ we have $s_{ij} = \gamma$ for all $i \in N$, but $x_j \neq \gamma$. We can assume without loss of generality that $x_j>\gamma$; a similar argument works for the case $x_j < \gamma$.
    Note that there must exist some $j' \in [m]$ such that $x_{j'} < s_{i'j'}$ for some $i' \in N$.
    Now, consider the outcome vector $\mathbf{x}'$ where $\mathbf{x}'$ is identical to $\mathbf{x}$, except $x'_j = \gamma$ and $x'_{j'} = x_{j'} + (x_j-\gamma)$. Then, at least one agent's $(i')$ disutility will decrease, and all other agents' disutility does not increase, a contradiction with $\mathbf{x}$ being Pareto optimal.
\end{proof}

Next, we show that of the five coordinate-wise aggregation rules we consider, the sum rule is the only one satisfying score-unanimity. 

\begin{thm} \label{thm:scoreunanimity-sum}
    The sum rule is score-unanimous.
\end{thm}
\begin{proof}
    Suppose for some $j \in [m]$ and $\gamma \in [0,1]$ we have $s_{ij} = \gamma$ for all $i \in N$. Then, $x_j = \frac{1}{n} \sum_{i \in N} \gamma = \frac{\gamma \cdot n}{n} = \gamma$.
\end{proof}

The next three results show that the max, min, and med rules satisfy score-unanimity in some special cases, whereas the prod rule fails it even in the simplest setting. 

\begin{thm}
    The max and min rules are score-unanimous when $m=2$, but may fail to be so when $m \geq 3$ (even when $n = 2)$.
\end{thm}
\begin{proof}
    Since the max and min rules are PO when $m=2$ (by Proposition~\ref{prop:po_maxmin}), the property follows by Theorem~\ref{thm:scoreunanimity-PO_implies_scoreunanimity}.

    Next, we show a counterexample for $n=2$ and $m=3$.
    Suppose we have two agents with score vectors
    $\mathbf{s}_1 = (0,0.2,0.8)$ and 
    $\mathbf{s}_2 = (0.8,0.2,0)$.
    Then, the max rule will return $\mathbf{x}_\text{max} = (\frac{4}{9},\frac{1}{9},\frac{4}{9})$ and the min rule will return $\mathbf{x}_\text{min} = (0,1,0)$.
    It is easy to see that score-unanimity is violated for $j=2$.
\end{proof}

\begin{thm} \label{thm:scoreunanimity-med}
    The med rule is score-unanimous when $n=2$ or $m\leq 3$, but may fail to be so when $n \geq 3$ and $m \geq 4$.
\end{thm}
\begin{proof}
    Since the med rule is equivalent to the sum rule when $n=2$, by Theorem \ref{thm:scoreunanimity-sum}, the result follows.

    When $m=2$, if all agents are unanimous on one candidate's (say $p_1$ without loss of generality) score, then for $p_2$, it will also be unanimous. 
    This means all agents will have identical score vectors. 
    Both the max and min rules will then return exactly the vector of any agent.

    When $m=3$, suppose (without loss of generality) that all agents are unanimous in the score for candidate $p_1$: $s_{i1} = \gamma$ for all $i \in N$.
    Let $\mathbf{y}$ be the pre-normalization score vector. Then $y_1 = \gamma$.
    Then consider the scores for the remaining two candidates. 
    Each agent's total score for the remaining two candidates must sum to $1 - \gamma$. 
    We have that the scores chosen for the remaining two candidates is either exactly one of the agents' score vector, or the average of the two ``middle'' score vectors. 
    In either case, $y_2 + y_3 = 1 - \gamma$.
    Then we have that $\mathbf{x} = \mathbf{y}$, and in particular, $x_1 = y_1 = \gamma$.

    Next, we show a counterexample for $n=3$ and $m=4$.
    Suppose we have three agents with score vectors
    $\mathbf{s}_1 = (0.3,0.5,0.1,0.1)$, 
    $\mathbf{s}_2 = (0.3,0.1,0.6,0)$, and 
    $\mathbf{s}_3 = (0.3,0.7,0,0)$.
    Then, the med rule will return $\mathbf{x} = (\frac{1}{3}, \frac{5}{9}, \frac{1}{9},0)$.
    Score-unanimity is violated for $p_1$.
\end{proof}

\begin{proposition} \label{prop:scoreunanimity-prod}
    The prod rule may fail score-unanimity for all $n \geq 2$ and $m \geq 2$.
\end{proposition}
\begin{proof}
    Let $n\ge 2$ and $\mathbf{s}_i = (0.8,0.2,0, \dots, 0)$ for all $i \in N$. 
    The prod rule returns $\mathbf{x} = \left( \frac{0.8^n}{0.8^n + 0.2^n}, \frac{0.2^n}{0.8^n + 0.2^n},0, \dots, 0 \right)$.
    It is easy to verify that $\frac{0.8^n}{0.8^n + 0.2^n} > 0.8$ for all $n \geq 2$.
\end{proof}
As for the welfare-based rules, since both \util{} and \egal{} are PO (by definition), we get the following as a corollary of Theorem~\ref{thm:scoreunanimity-PO_implies_scoreunanimity}.

\begin{corollary}
    \emph{\util{}} and \emph{\egal{}} are score-unanimous.
\end{corollary}

Finally, we show that the IM mechanism also fails to be score-unanimous in general.

\begin{thm} \label{thm:scoreunanimous-IM}
    The IM mechanism is score-unanimous when $m=2$, but may fail to be so when $m \geq 3$ (even when $n = 2$).
\end{thm}
\begin{proof}
    Since the IM mechanism is PO when $m=2$ (by Corollary \ref{cor:PO_IM_m=2}), the result follows from Theorem \ref{thm:scoreunanimity-PO_implies_scoreunanimity}.

    Next, we show a counterexample for $n=2$ and $m=3$. Suppose we have two agents with score vectors $\mathbf{s}_1 = (0.8,0.2,0)$ and $\mathbf{s}_2 = (0.8,0,0.2)$, then the IM mechanism returns $\mathbf{x}=(0.6,0.2,0.2)$. It is easy to see that score-unanimity is violated for $p_1$.
\end{proof}

We have argued that score-unanimity is a special case of RR. Unsurprisingly, it is known that the IM mechanism is not RR in general (one can use the same counterexample as in the proof of Theorem~\ref{thm:scoreunanimous-IM}) \cite{freeman2021truthfulbudget}. We complement this result by showing that IM satisfies RR for two candidates.

\begin{thm} \label{thm:RR-IM}
    The IM mechanism is RR when $m=2$, but may fail to be so when $m \geq 3$ (even when $n = 2$).
\end{thm}
\begin{proof}
    By the comment before the theorem, it suffices to prove this when $m=2$.
    Let $\mathbf{x}$ be the output vector of the IM mechanism.
    Recall that the stopping condition from the definition of the mechanism mandates that the entries of the output vector $\mathbf{x}$ sum to $1$.

    For each $j \in \{1,2\}$, denote $z^\text{max}_j$ and $z^\text{min}_j$ as the maximum and minimum values that the $n$ voters have for candidate~$j$, respectively. 
    Suppose for a contradiction that the IM mechanism is not RR when $m=2$. There are two cases: either all the $n+1$ phantoms are strictly less than $z^\text{min}_j$ for some $j \in \{1,2\}$, or all the $n+1$ phantoms are strictly more than $z^\text{max}_j$ for some $j \in \{1,2\}$. 
    The latter case cannot happen since one of the phantoms is always at 0. 
    Thus, we focus on the former case.

    Suppose without loss of generality that all $n+1$ phantoms are strictly less than $z^\text{min}_1$, so we have that $x_1 < z^\text{min}_1$. Then, since $z^\text{max}_2 = 1 - z^\text{min}_1$, in order for $x_1 + x_2 = 1$, we must have that $x_2 > z^\text{max}_2$. This means that all $n+1$ phantoms must be strictly more than $z^\text{min}_2$, which is a contradiction.
\end{proof}
    
In fact, out of the five coordinate-wise aggregation rules we study, the sum rule is the only one that is RR for all $n$ and~$m$.

\begin{thm} \label{thm:rr_sum}
    The sum rule is RR.
\end{thm}
\begin{proof}
    Let $\mathbf{x}$ and $\mathbf{y}$ be the output and pre-normalization vector of the sum rule, respectively.
    Then $x_j = \frac{y_j}{n}$ for all $j\in [m]$.
    For each $j \in [m]$, let $z^\text{max}_j$ and $z^\text{min}_j$ be the maximum and minimum values that the $n$ voters have for candidate $j$, respectively. 
    Then, since $y_j = \sum_{i \in N} s_{ij}$, we obtain $n \cdot z^\text{min}_j \leq y_j \leq n \cdot z^\text{max}_j$. Dividing by $n$ throughout, we get $z^\text{min}_j \leq x_j = \frac{y_j}{n} \leq z^\text{max}_j$.
\end{proof}

\begin{thm} \label{thm:RR_max_min}
    The max and min rules are RR when $m=2$, but may fail to be so when $m \geq 3$ (even when $n = 2$).
\end{thm}
\begin{proof}
    We first prove the result for the max rule when $m=2$.
    Let $\mathbf{x}$ be the output vector of the max rule. 
    For each $j \in \{1,2\}$, denote $z^\text{max}_j$ and $z^\text{min}_j$ as the maximum and minimum values that the $n$ voters have for candidate $j$, respectively. 
    Then $x_j = z_j^\text{max}/(z_1^\text{max} + z_2^\text{max})$ for $j=1, 2$.
    
    We will show that the max rule is RR for $p_1$; the same analysis applies to $p_2$. 
    We have $z^\text{max}_2 = 1- z^\text{min}_1$. 
    Given that $z_1^\text{min} \leq z_1^\text{max}$, multiplying both sides by $1-z_1^\text{min}$, we get 
    \[
    z_1^\text{min} (1-z_1^\text{min}) \leq z_1^\text{max} (1-z_1^\text{min}).
    \]
    Algebraic manipulation gives us 
    \[
    z_1^\text{min} \leq \frac{z_1^\text{max}}{{z_1^\text{max} + (1-z_1^\text{min})}} = x_1,
    \]
    as desired (since $z^\text{max}_2 = 1- z^\text{min}_1)$.
     
    For the upper bound, we know that $z_1^\text{min} \leq z_1^\text{max}$. Then, adding $1 - z_1^\text{min}$ to both sides, we get 
    \[
    1 \leq z_1^\text{max} + (1-z_1^\text{min}).
    \]
     Multiplying both sides by $\frac{z_1^\text{max}}{z_1^\text{max} + (1-z_1^\text{min})}$, we get 
     \[
     x_1 = \frac{z_1^\text{max}}{z_1^\text{max} + (1-z_1^\text{min})} \leq z_1^\text{max},
     \]
     as desired (since $z^\text{max}_2 = 1- z^\text{min}_1$).

     We now show a counterexample for the max rule when $n=2$ and $m=3$.
     Consider a profile with two agents and three candidates, where the score vectors are $\mathbf{s}_1 = (\frac{2}{3},0,\frac{1}{3})$ and $\mathbf{s}_2 = (\frac{1}{3},\frac{1}{3},\frac{1}{3})$. Then, the max rule will return $\mathbf{x} = (0.5,0.25,0.25)$, which is clearly not RR.

     Next, we prove the result for the min rule when $m=2$.
    Let $\mathbf{x}$ be the output vector of the min rule. 
    For each $j \in \{1,2\}$, denote $z^\text{max}_j$ and $z^\text{min}_j$ as the maximum and minimum values that the $n$ voters have for candidate $j$, respectively. 
    This means that for all $j \in \{1,2\}$, $x_j = \frac{z_j^\text{min}}{z_1^\text{min} + z_2^\text{min}}$.
    
    We will show that the rule is RR for $p_1$ (but the same analysis extends to $p_2$). 
    Now, we have that $z^\text{min}_2 = 1- z^\text{max}_1$. 
    Given that $z_1^\text{min} \leq z_1^\text{max}$, multiplying $1-z_1^\text{max}$ on both sides, we get
    \begin{equation*}
        z_1^\text{min} (1-z_1^\text{max}) \leq z_1^\text{max} (1-z_1^\text{max}).
    \end{equation*}
    Algebraic manipulation gives us
    \begin{equation*}
        x_1=\frac{z_1^\text{min}}{z_1^\text{min} + (1-z_1^\text{max})} \leq z_1^\text{max},
    \end{equation*}
    as desired (since $z^\text{min}_2 = 1- z^\text{max}_1$).
     
    For proving the lower bound, we know that $z_1^\text{min} \leq z_1^\text{max}$. Then, adding $1 - z_1^\text{max}$ on both sides, we get
     \begin{equation*}
         z_1^\text{min} + (1-z_1^\text{max}) \leq 1
     \end{equation*}
     Multiplying $\frac{z_1^\text{min}}{z_1^\text{min} + (1-z_1^\text{max})}$ on both sides, we get
     \begin{equation*}
         z_1^\text{min} \leq \frac{z_1^\text{min}}{z_1^\text{min} + (1-z_1^\text{max})} = x_1,
     \end{equation*}
     as desired (since $z^\text{min}_2 = 1- z^\text{max}_1$).

    Finally, we give a counterexample for the min rule when $n=2$ and $m=3$.
     Consider a profile with two agents and three candidates, where the score vectors are $\mathbf{s}_1 = (1,0,0)$ and $\mathbf{s}_2 = (0,1,0)$. Then, the min rule will return $\mathbf{x} = (\frac{1}{3},\frac{1}{3},\frac{1}{3})$, which is clearly not RR.
\end{proof}
\begin{thm} \label{thm:rr_med}
    The med rule is RR when $n=2$ or $m=2$ or $n=m=3$, but may fail to be RR when $n \geq 3$ and $m \geq 4$.
\end{thm}
\begin{proof}
    When $n=2$, the med rule is equivalent to the sum rule. By Theorem~\ref{thm:rr_sum}, the sum rule is RR, and the result follows.

    Next, we consider the case $m=2$.
    Let the sorted scores for $p_1$ and $p_2$ be $z_{1,1}\le \dots\le z_{1,n}$ and $z_{2,1}\le \dots\le z_{2,n}$, respectively. 
    Since score vectors are normalized, for each $i \in N$, we have that $z_{1,i} = 1 - z_{2,n-i+1}$.
    When $n$ is odd, the med rule will output $\mathbf{x} = \left( \frac{z_{1,(n+1)/2}}{z_{1,(n+1)/2} + z_{2,(n+1)/2}}, \frac{z_{2,(n+1)/2}}{z_{1,(n+1)/2} + z_{2,(n+1)/2}} \right)$.
    But note that $z_{1,(n+1)/2} = 1 - z_{2,n-(n+1)/2 + 1} = 1-z_{2,(n+1)/2}$.
    This means that $z_{1,(n+1)/2} + z_{2,(n+1)/2} = 1$, and hence we have that $\mathbf{x} = ( z_{1,(n+1)/2}, z_{2,(n+1)/2})$, which is clearly RR.
    
    When $n$ is even, let $A_1 = z_{1,n/2} +z_{1,n/2+1}$ and $A_2 = z_{2,n/2} +z_{2,n/2+1}$. Then, the med rule will output 
    $\mathbf{x} = \left( 
        \frac{A_1/2}{(A_1+A_2)/ 2},
        \frac{A_2/2}{(A_1+A_2)/ 2} \right)$.
    But note that $z_{1,n/2} = 1 - z_{2,n-n/2 + 1} = 1-z_{2,n/2+1}$ and $z_{2,n/2} = 1 - z_{1,n-n/2 + 1} = 1-z_{1,n/2+1}$.
    This means that $A_1+A_2 = 2$ and hence we have that $\mathbf{x} = ( A_1/2, A_2/2)$, which is clearly RR.

    When $n=m=3$, let $\mathbf{x}$ and $\mathbf{y}$ be the output and pre-normalization vector of the med rule, respectively.
    Similar to the previous case, for $j \in \{1,2,3\}$, let the sorted scores for $p_j$ be $z_{j,1},z_{j,2},z_{j,3}$, where $z_{j,1} \leq z_{j,2} \leq z_{j,3}$.
    Suppose for a contradiction, and without loss of generality, that $x_1 = 
    \frac{z_{1,2}}{z_{1,2} + z_{2,2} + z_{3,2}} > z_{1,3}$.
    Note that 
    \begin{equation} \label{eqn:medRR_1}
        z_{1,1} + z_{2,1} + z_{3,1} \leq z_{1,2} + z_{2,2} + z_{3,2} < 1.
    \end{equation}
    This also means that
    $x_2 > z_{2,2}$ and $x_3 > z_{3,2}$.
    Now, we know that
    \begin{equation} \label{eqn:medRR_2}
        x_1 + x_2 + x_3 = 1.
    \end{equation}
    Consider an agent $i$ such that $s_{i1} = z_{1,3}$.
    We split our analysis into two cases.
    \begin{description}
        \item[Case 1: $s_{i2} = z_{2,3}$ or $s_{i3} = z_{3,3}$.]
        Without loss of generality, assume that $s_{i2} = z_{2,3}$.
        This means that there exists some other agent $j \neq i$ such that $s_{j1} \leq z_{1,2}$, $s_{j2} \leq z_{2,2}$, and $s_{j3} \leq z_{3,2}$.
        However, $s_{j1} + s_{j2} + s_{j3} \leq z_{1,2}+z_{2,2}+z_{3,2} < 1$,
        where the rightmost inequality follows from (\ref{eqn:medRR_1}),
        contradicting the definition of a preference score vector.
        \item[Case 2: $s_{i2} \leq z_{2,2}$ and $s_{i3} \leq z_{3,2}$.]
        However, $s_{i1} + s_{i2} + s_{i3} \leq z_{1,3}+z_{2,2}+z_{3,2} < x_1 + x_2 + x_3 = 1$, where the rightmost equality follows from (\ref{eqn:medRR_2}), contradicting the definition of a preference score vector.
    \end{description}
    In both cases, we arrive at a contradiction, and hence the med rule is RR when $n=m=3$.

    Finally, we show a counterexample for $n=3$ and $m=4$.
    Suppose we have three agents with score vectors $\mathbf{s}_1 = (0.3,0.5,0.1,0.1)$, $\mathbf{s}_2 = (0.3,0.1,0.6,0)$, and $\mathbf{s}_3 = (0.3,0.7,0,0)$.
    Then, the med rule will return $(\frac{1}{3},\frac{5}{9},\frac{1}{9},0)$. It is easy to see that RR is violated for $p_1$.
\end{proof}

Theorem~\ref{thm:scoreunanimity-PO_implies_scoreunanimity} and Proposition~\ref{prop:scoreunanimity-prod} imply the following:

\begin{corollary}
    The prod rule may fail to be RR for all $n \geq 2$ and $m \geq 2$.
\end{corollary}

As for the welfare-based rules, from Theorems \ref{thm:n=2_po_rr}--\ref{thm:PO_implies_RR}, we have that both satisfy RR.
\begin{corollary}
  \emph{\util{}} and \emph{\egal{}} are RR.
\end{corollary}

\section{Score-Monotonicity} \label{sec:score_monotonicity}
We begin this section with a lemma that provides a more general condition for coordinate-wise aggregation rules to satisfy score-monotonicity. This result will be useful for subsequent proofs.
\begin{lemma} \label{lemma:score_monotonicity_condition}
    Suppose that a coordinate-wise aggregation rule~$F$ with coordinate-wise aggregation function $f$ is such that for every $i\in N$, $f(z_1,\dots,z_i,\dots,z_n) \geq f(z_1,\dots,z'_i,\dots,z_n)$ whenever $z_i \geq z'_i$. 
    Then $F$ is score-monotone.
\end{lemma}
\begin{proof}
    Let $\mathcal{I} = (\mathbf{s}_1,\dots, \mathbf{s}_i, \dots,\mathbf{s}_n)$ be an instance. 
    Then, suppose $\mathcal{I}' = (\mathbf{s}_1,\dots, \mathbf{s}'_i, \dots,\mathbf{s}_n)$ is another instance that is equivalent to $\mathcal{I}$, except that $s_{ij} < s'_{ij}$, and $s_{ij'} \geq s'_{ij'}$ for all other $j' \in [m] \setminus \{j\}$. 
    If $\mathbf{x}$ and $\mathbf{x}'$ are the output vectors of the rule $F$ applied to instances $\mathcal{I}$ and $\mathcal{I}'$ respectively, then we want to prove that $x_j \leq x'_j$.

    For each $j' \in [m]$, let $S_{j'} = (s_{i'j'})_{i' \in N\setminus \{i\}}$.
    Note that using the property of $f$ in the lemma statement, we get
    \begin{equation} \label{eqn:score_monotone_propertyf}
        f(s'_{ij},S_j) - f(s_{ij},S_j) \geq 0.
    \end{equation}
    Moreover, for all $j' \in [m]\setminus \{j\}$, $f(s'_{ij'},S_{j'}) = f(s_{ij'},S_{j'}) = 0$.
    
    First, we consider the case where $\sum_{j' \in [m]} f(s_{ij'}, S_{j'}) = 0$. 
    By Definition \ref{defn:coord_rules}, $x_{j'} = 1/m$ for all $j' \in [m]$.
    If (\ref{eqn:score_monotone_propertyf}) is an equality, then $x'_j = x_j = 1/m$. 
    On the other hand, if the inequality in (\ref{eqn:score_monotone_propertyf}) is strict, we get that $f(s'_{ij},S_j) > f(s_{ij},S_j) = 0$.
    This means that $x'_j = 1 \ge 1/m = x_j$.

    Next, we consider the case where $\sum_{j' \in [m]} f(s_{ij'}, S_{j'}) > 0$.
    Multiplying (\ref{eqn:score_monotone_propertyf}) on both sides of the following inequality (which is true since all terms are non-negative):
    \begin{equation*}
        \sum_{j' \in [m]} f(s_{ij'}, S_{j'}) \geq f(s_{ij},S_j),
    \end{equation*}
    we get
    \begin{equation*}
        \left( f(s'_{ij},S_j) - f(s_{ij},S_j) \right) \cdot \sum_{j' \in [m]} f(s_{ij'}, S_{j'}) 
    \end{equation*}
    \begin{equation*}
        \geq 
        f(s_{ij},S_j) \cdot \left( f(s'_{ij},S_j) - f(s_{ij},S_j) \right).
    \end{equation*}
    Algebraic manipulation gives us
    \begin{equation*}
        f(s'_{ij},S_j) \cdot \sum_{j' \in [m]} f(s_{ij'}, S_{j'}) 
    \end{equation*}
    \begin{equation*}
        \geq f(s_{ij},S_j) \cdot \left( f(s'_{ij},S_j) - f(s_{ij},S_j) + \sum_{j' \in [m]} f(s_{ij'}, S_{j'})\right).
    \end{equation*}
    Then, we can derive the following (and note that the denominators on both sides of the inequality are guaranteed to be positive by the assumptions):
    \begin{align*}
        \begin{split}
            x'_j & = \frac{f(s'_{ij},S_j)}{  \sum_{j' \in [m]} f(s'_{ij'}, S_{j'})} \\
            & \geq \frac{f(s'_{ij},S_j)}{ f(s'_{ij},S_j) - f(s_{ij},S_j) + \sum_{j' \in [m]} f(s_{ij'}, S_{j'})}  \\
            & \geq \frac{f(s_{ij},S_j)}{\sum_{j' \in [m]} f(s_{ij'}, S_{j'})} = x_j,
        \end{split}
    \end{align*}
    where the first inequality holds because $f(s'_{ij},S_j) - f(s_{ij},S_j) + f(s_{ij},S_{j}) = f(s'_{ij},S_{j})$
    and
    for all other $j' \in [m] \setminus \{j\}$,
    $s'_{ij'} \leq s_{ij'}$.
\end{proof}

For all our five coordinate-wise aggregation rules, the aggregation function clearly satisfies the condition in Lemma \ref{lemma:score_monotonicity_condition}.
Hence, we immediately have the following result.

\begin{thm} \label{thm:score_monotonicity_fiverules}
    All five coordinate-wise aggregation rules satisfy score-monotonicity.
\end{thm}

As for the remaining rules, Freeman et al.~\cite[Thm.~3]{freeman2021truthfulbudget} showed that \util{} and the IM mechanism are score-monotone. 
We show that \egal{} is score-monotone in the special cases where $n=2$ or $m=2$. The setting where $n,m \geq 3$ is left as an open question.

\begin{thm} \label{thm:score_monotone-egal}
    \emph{\egal{}} is score-monotone when $n=2$ or $m=2$.
\end{thm}
\begin{proof}
    When $n=2$, \egal{} is equivalent to the sum rule (by Lemma~\ref{lem:welfare-egal_is_sum}), which is known to satisfy score-monotonicity by Theorem~\ref{thm:score_monotonicity_fiverules}.
    Thus, we only consider the case $m=2$ and assume that $n\ge 3$.

    Let instances $\mathcal{I}$ and $\mathcal{I}'$ be such that $\mathcal{I}'$ is identical to $\mathcal{I}$, except that for some $i\in N$ we have $s_{i1} < s'_{i1}$, $s_{i2}  \geq s'_{i2}$ (this is without loss of generality; the same argument can be made for the other case), while
    other agents' score vectors remain the same.
    Suppose $\mathbf{x}$ and $\mathbf{x}'$ are the outcome vectors returned by \egal{} for instances $\mathcal{I}$ and $\mathcal{I}'$, respectively.

    Without loss of generality, let the preference profile $\mathcal{I}$ be such that $s_{11} \leq \dots \leq s_{n1}$ and $s_{12} \geq \dots \geq s_{n2}$.
    Then, \egal{} will return $\mathbf{x} = \left(\frac{s_{11} + s_{n1}}{2}, \frac{s_{n2} + s_{12}}{2} \right)$.
    For the preference profile $\mathcal{I}'$, we have $\min_{i\in N}s'_{i1} \ge s_{11}$ and $\max_{i\in N}s'_{i1} \ge s_{n1}$.
    So $x_1 = \frac{s_{11}+s_{n1}}{2} \le \frac{\min_{i\in N}s'_{i1}+\max_{i\in N}s'_{i1}}{2} = x'_1$, which means that score-monotonicity is satisfied.    
\end{proof}

\section{Proportionality and Score-Representation} \label{sec:prop_sr}
In this section, we focus on the notions of proportionality \cite{freeman2021truthfulbudget} and score-representation (which is a generalization of proportionality). Freeman et al.~\cite{freeman2021truthfulbudget} proved that the IM mechanism satisfies proportionality. However, we show that their guarantees do not carry over for score-representation in general.

\begin{thm}
    The IM mechanism may fail score-representation, even when $n=2$ and $m=3$.
\end{thm}

\begin{proof}
    Consider the case of two agents and three candidates, where the score vectors are $\mathbf{s}_1 = (0.8,0.2,0)$ and $\mathbf{s}_2 = (0.8,0,0.2)$.
    Then, score-representation mandates that the outcome $\mathbf{x}$ should satisfy $x_1 \geq 0.8$. 
    However, the outcome vector returned by the IM mechanism is $\mathbf{x}' = (0.6, 0.2, 0.2)$, where $x_1 = 0.6 < 0.8$.
\end{proof}
The question of whether the IM mechanism satisfies the score-representation axiom when $m=2$ remains open.

Now, we show that the sum rule satisfies score-representation (and hence proportionality as well) in general. In fact, among the coordinate-wise aggregation rules we consider, it is the only rule with this property.

\begin{thm} \label{thm:score_representation-sum}
    The sum rule satisfies score-representation.
\end{thm}
\begin{proof}
    Let $S \subseteq N$ be the set of agents whose score for a candidate $p_j$ is at least $\gamma$, for some $\gamma \in (0,1]$. Then $x_j = \frac{1}{n} \left( \sum_{i \in S} s_{ij} + \sum_{i' \in N\setminus S} s_{i'j} \right) \geq \frac{1}{n} \left( \sum_{i \in S} \gamma + 0 \right) = \frac{ \gamma \cdot |S|}{n}$.
\end{proof}

While the general positive result only holds for the sum rule, we show that the max and med rules satisfy score-representation when $n=2$, whereas the min and prod rules may fail to do so even in the simplest setting.

\begin{thm} \label{thm:score_representation-max}
    The max rule satisfies score-representation when $n=m=2$, but may fail to be so when  $m \geq 3$ (even for $n=2$) or $n\ge 3$ (even for $m=2$).
    It also satisfies proportionality when $n=2$, but may fail to be so when $n \geq 3$ (even for $m = 2$).
\end{thm}
\begin{proof}
    We first prove that the max rule satisfies score-representation when $n=m=2$.
    Assume without loss of generality that $s_{11} \geq s_{21}$ and $s_{22} \geq s_{12}$.
    Then, score-representation demands that (i) $x_1 \geq \frac{s_{11}}{2}$, $x_2 \geq \frac{s_{22}}{2}$ and (ii) $x_1 \geq s_{21}$, $x_2 \geq s_{12}$.
    Property (ii) follows from the fact that the max rule is RR when $m=2$ (Theorem \ref{thm:RR_max_min}).
    We will show the max rule satisfies property (i).

    Now, $x_1 = \frac{s_{11}}{s_{11} + s_{22}}$ and $x_2 = \frac{s_{22}}{s_{11} + s_{22}}$.
    Since $s_{11} + s_{22} \leq 2$,
    $x_1 = \frac{s_{11}}{s_{11} + s_{22}} \geq \frac{s_{11}}{2}$ and $x_2 = \frac{s_{22}}{s_{11} + s_{22}} \geq \frac{s_{22}}{2}$.
    Thus, property (i) is satisfied.

    We now show that the max rule may fail score-representation when $n=2$ and $m=3$.
    Consider a profile with two agents and three candidates, where the score vectors are $\mathbf{s}_1 = (\frac{2}{3},0,\frac{1}{3})$ and $\mathbf{s}_2 = (\frac{1}{3},\frac{1}{3},\frac{1}{3})$. 
    Then, since $s_{13} = s_{23} = \frac{1}{3}$, score-representation mandates that $x_3 \geq \frac{1}{3}$.
    However, the max rule will return $\mathbf{x} = (0.5,0.25,0.25)$ with $x_3 = 0.25 < \frac{1}{3}$, failing this condition.

     For $m=2$ and $n\ge 3$, we will show later that the max rule fails the weaker proportionality property.

     Next, we show that the max rule satisfies proportionality for $n=2$ and $m \geq 3$ (it already satisfies the stronger score-representation property when $n=m=2$).
     Recall that agents are single-minded when reasoning about proportionality.
     If both agents give a score of $1$ to the same candidate (let it be $p_1$), then the max rule will return $\mathbf{x} = (1,0,\dots,0)$, satisfying proportionality.
     Suppose both agents give a score of $1$ to different candidates.
     Without loss of generality, let $s_{11} = 1$ and $s_{22} = 1$, with all other scores being $0$.
     Then, the max rule will return $\mathbf{x} = (0.5,0.5,0,\dots,0)$, which also satisfies proportionality.
    
    Finally, we show that the max rule may fail proportionality for $n \geq 3$, even for $m = 2$. 
    Suppose we have $n-1$ agents with score vector $(1,0)$ and one agent with score vector $(0,1)$. 
    Then, proportionality states that the score of candidate $p_1$ should be $\frac{n-1}{n}$. 
    However, the max rule will return the vector $\mathbf{x} = (0.5,0.5)$. For any $n \geq 3$, we have $x_1 = 0.5 < \frac{n-1}{n}$.
    The case of more candidates can easily be handled by adding dummy candidates for which every agent has a score of $0$. 
\end{proof}

\begin{thm} \label{thm:score_representation-med}
    The med rule satisfies score-representation when $n=2$, but may fail to be proportional when $n \geq 3$ (even for $m = 2$).
\end{thm}
\begin{proof}
    In the case of two agents, the med rule is equivalent to the sum rule, which is known to satisfy score-representation by Theorem \ref{thm:score_representation-sum}.

    Next, we prove that the rule may fail proportionality for any $n \geq 3$ and $m \geq 2$.
    Consider a candidate $p_j$, for some $j \in [m]$. 
    Suppose $n$ is odd (respectively, even), then let $\frac{n-1}{2}$ (resp. $\frac{n}{2}-1$) voters give $p_j$ a score of $1$ and $\frac{n+1}{2}$ (resp. $\frac{n}{2}+1$) voters give $p_j$ a score of~$0$. 
    Then, proportionality states that candidate $p_j$ should receive a score of $\frac{n-1}{2}$ (resp. $\frac{n}{2} -1$). 
    However, the med rule will return $x_j = 0$.
\end{proof}

\begin{thm} \label{thm:score_representation-min_prod}
    The min and prod rules satisfies proportionality when $n=m=2$, but may fail to be so when $n \geq 3$ or $m \geq 3$.
    Both rules may fail score-representation even when $n=m=2$.
\end{thm}
\begin{proof}
    We first prove that the min and prod rules satisfy proportionality when $n=m=2$.
    Recall that when checking for proportionality, agents are single-minded.
    If both agents give a score of $1$ to the same candidate (let it be $p_1$), then both rules return $\mathbf{x} = (1,0)$, which satisfies proportionality.
    If both agents give a score of $1$ to different candidates, then both rules return $\mathbf{x} = (0.5,0.5)$, which also satisfies proportionality.
    
    Next, we show that both rules may fail score-representation in the same case where $n=m=2$.
    Consider a profile with two agents and two candidates, where the score vectors are $\mathbf{s}_1 = (1,0)$ and $\mathbf{s}_2 = (0.5,0.5)$. 
    Then, score-representation mandates that $x_2 \geq 0.25$.
    However, both rules will return $\mathbf{x} = (1,0)$ with $x_2 = 0 < 0.25$, failing this condition.

    Finally, we show that the min and prod rules may fail proportionality when $m\geq 3$ (even for $n = 2$) or $n \geq 3$ (even for $m = 2$). 
    
    When $n=2$, suppose $s_{11} = 1$ and $s_{22} = 1$, with all other scores being $0$.
    Then, both rules will return $\mathbf{x} = (\frac{1}{m},\frac{1}{m},\dots, \frac{1}{m})$, which violates proportionality.
    
    When $m=2$, Suppose we have $n-1$ agents with score vector $(1,0)$ and one agent with score vector $(0,1)$. 
    Then, proportionality states that the score of candidate $p_1$ should be $\frac{n-1}{n}$. 
    However, the min and prod rules will return the vector $\mathbf{x} = (0.5,0.5)$. For any $n \geq 3$, we have $x_1 = 0.5 < \frac{n-1}{n}$.
    The case of more candidates can easily be handled by adding dummy candidates for which every agent has a score of $0$. 
\end{proof}

Freeman et al.~\cite{freeman2021truthfulbudget} showed that \util{} may fail to satisfy proportionality in general. 
We show that it satisfies the stronger score-representation property when $n=m=2$, and proportionality when $n=2$, but fails in all other cases.

\begin{thm}
    \emph{\util{}} satisfies score-representation when $n=m=2$, but may fail to be so when $m \geq 3$. It is also proportional when $n=2$, but may fail to be so when $n \geq 3$ (even for $m=2$).
\end{thm}
\begin{proof}
    Note that for $n=2$, any outcome vector $\mathbf{x}$ is range-respecting if and only if it maximizes utilitarian welfare.
    \util{} will then select the outcome vector that is closest to the uniform distribution.
    In particular, if the outcome vector $(\frac{1}{m},\dots,\frac{1}{m})$ is range-respecting, it will be selected by \util{}.

    When $n=2$, an outcome satisfying score-representation has to satisfy two conditions: for each $j \in [m]$, (i) $x_j \geq \max(s_{1j}, s_{2j})/2$ and (ii) $x_j \geq \min(s_{1j},s_{2j})$.
    
    When $n = m = 2$, since \util{} is range-respecting, (ii) is trivially satisfied.
    As for (i), if $\max(s_{11}, s_{21}) \geq 0.5 \geq \min(s_{11},s_{21})$, then $\max(s_{12}, s_{22}) \geq 0.5 \geq \min(s_{12},s_{22})$, and \util{} will pick $\mathbf{x} = (0.5,0.5)$ and score-representation is satisfied.
    Now, note that $\min(s_{11}, s_{21}) + \max(s_{12}, s_{22})=1$ and $\max(s_{11}, s_{21}) + \min(s_{12}, s_{22})=1$.
    If $\min(s_{11}, s_{21}) > 0.5$, then \util{} will choose the outcome vector $\mathbf{x} = (\min(s_{11}, s_{21}), \max(s_{12}, s_{22}))$. Clearly, (i) is satisfied for both $j=1,2$.
    Similarly, if $\max(s_{11}, s_{21}) < 0.5$, then \util{} will choose the outcome vector $\mathbf{x} = (\max(s_{11}, s_{21}), \min(s_{12}, s_{22}))$. Since $\min(s_{12}, s_{22}) >0.5$, it is easy to see that (i) is satisfied for both $j=1,2$.

    Now, we show that \util{} may fail to satisfy score-representation when $m \geq 3$, even for $n = 2$.
    Let $\mathbf{s}_1 = (0.8,0.1,0.1)$ and $\mathbf{s}_2 = (\frac{1}{3},\frac{1}{3},\frac{1}{3})$. Then, \util{} will return $\mathbf{x} = \mathbf{s}_2 = (\frac{1}{3},\frac{1}{3},\frac{1}{3})$.
    However, score-representation mandates that $x_1 \geq 0.4$, which is clearly not the case.

    Next, we show that \util{} satisfies proportionality when $n=2$ and $m \geq 3$ (it already satisfies the stronger score-representation property when $n=m=2$).
    Recall that agents are single-minded when reasoning about proportionality.
    If both agents give a score of $1$ to the same candidate (let it be $p_1$), then \util{} will return $\mathbf{x} = (1,0,\dots,0)$, satisfying proportionality.
    Suppose both agents give a score of $1$ to different candidates.
    Without loss of generality, let $s_{11} = 1$ and $s_{22} = 1$, with all other scores being $0$.
    Then, \util{} will return $\mathbf{x} = (0.5,0.5,0\dots,0)$, which also satisfies proportionality.
    
    Finally, we show that \util{} may fail proportionality when $n \geq 3$, even for $m=2$. Consider the case where $n-1$ agents have the score vector $(1,0)$ and one agent has the score vector $(0,1)$.
    Then, proportionality mandates that $p_1$ receives a score of $\frac{n-1}{n}$. However, \util{} will return the outcome $\mathbf{x} = (1,0)$, where $x_1 = 1 > \frac{n-1}{n}$.
\end{proof}

We now show that \egal{} provides slightly better guarantees.

\begin{thm} \label{thm:prop_egal}
    \emph{\egal{}} satisfies score-representation when $n=2$, but may fail to be proportional for any $n \geq 3$ (even for $m=2$).
\end{thm}
\begin{proof}
    The proof for $n=2$ follows from the fact that \egal{} is equivalent to the sum rule in this case (Lemma \ref{lem:welfare-egal_is_sum}).
    
    Next, we show a counterexample for $n=3$ when $m=2$. 
    Consider the setting with three agents and two candidates, where $\mathbf{s}_1 = (1,0)$ and $\mathbf{s}_2 = \mathbf{s}_3 = (0,1)$. 
    Proportionality mandates that outcome must be $\mathbf{x}=(\frac{1}{3},\frac{2}{3})$, with
    $d_1(\mathbf{x}) = \frac{4}{3}$ and $d_2(\mathbf{x}) = d_3(\mathbf{x})$ = $\frac{2}{3}$. 
    However, \egal{} will output outcome vector $\mathbf{x}_\text{egal} = (0.5,0.5)$, with $d_i(\mathbf{x}_\text{egal}) = 1$ for $i=1,2,3$.
\end{proof}

\section{Strategyproofness and Participation} \label{sec:sp_participation}
Both of the phantom-based mechanisms proposed by Freeman et al.~\cite{freeman2021truthfulbudget} that we study in this work, i.e., the IM mechanism and \util{}, have been proven to satisfy strategyproofness.
Unfortunately, none of the five coordinate-wise aggregation rules enjoys this property.

\begin{proposition} \label{prop:sp_all}
    Each of the sum, med, max, min, and prod rules may fail to be strategyproof for all $n \geq 2$ and $m \geq 2$.
\end{proposition}
\begin{proof}
    First, we consider the sum, med, and max rules. 
    Note that, for $n=2$, the med rule is equivalent to the sum rule, so we omit the argument for the med rule. 
    Consider the score vectors $\mathbf{s}_1 = (0.5,0.5)$ and $\mathbf{s}_2 = (0,1)$. 
    Then, the sum rule returns $\mathbf{x}_\text{sum} = (0.25,0.75)$, with $d_1(\mathbf{x}_\text{sum}) = 0.5$. 
    However, if agent $1$ were to misreport her score vector as $\mathbf{s}_1^* = (1,0)$, the sum rule would return $\mathbf{x}^*_\text{sum} = (0.5,0.5)$, with $d_1(\mathbf{x}^*) = 0 < d_1(\mathbf{x}_\text{sum})$. 
    The max rule returns $\mathbf{x}_\text{max} = (\frac{1}{3},\frac{2}{3})$ with $d_1(\mathbf{x}_\text{max}) = \frac{1}{3}$. 
    However, under the misreported score vector $\mathbf{s}_1^*$, the max rule would return $\mathbf{x}^*_\text{max} = (0.5,0.5)$, with $d_1(\mathbf{x}_\text{max}^*) = 0 < d_1(\mathbf{x}_\text{max})$.

    Next, we consider the min and prod rules. Consider the score vectors $\mathbf{s}_1 = (0.8,0.2)$ and $\mathbf{s}_2 = (0.2,0.8)$. Then, the min and prod rules return $\mathbf{x}_\text{min} = \mathbf{x}_\text{prod} = (0.5,0.5)$. However, if agent $1$ were to misreport her score vector as $\mathbf{s}_1^* = (1,0)$, the min and prod rules would return $\mathbf{x}_\text{min}^* = \mathbf{x}_\text{prod}^* = (1,0)$. Now, we have that $d_1(\mathbf{x}_\text{min}^*) = 0.4 < 0.6 = d_1(\mathbf{x}_\text{min})$ and $d_1(\mathbf{x}_\text{prod}^*) = 0.4 < 0.6 = d_1(\mathbf{x}_\text{prod})$.

    We have focused on the case $n=m=2$, but our counterexample is easily extendable to settings with more agents and candidates.
\end{proof}

Next, we show that this negative result extends to \egal{} as well.

\begin{proposition}
    \emph{\egal{}} may fail to be strategyproof for all $n \geq 2$ and $m \geq 2$.
\end{proposition}
\begin{proof}
    Suppose we have two agents and two candidates, with agents' score vector as follows: $\mathbf{s}_1 = (0.2,0.8)$ and $\mathbf{s}_2 = (0.8,0.2)$.
    Then, \egal{} will return the outcome $\mathbf{x} = (0.5, 0.5)$.
    Agent $1$'s disutility will be $d_1(\mathbf{x}) = (0.5-0.2) + (0.8-0.5) = 0.6$.
    Consider another instance where agent $1$ misreports his preferences as $\mathbf{s}'_1 = (0,1)$.
    Then, \egal{} will return the outcome $\mathbf{x}' = (0.4, 0.6)$, where agent $1$'s disutility (with respect to her truthful score vector) is $d_1(\mathbf{x}') = (0.4-0.2) + (0.8-0.6) = 0.4 < 0.6 = d_1(\mathbf{x})$.
    Hence, agent $1$ benefited from lying and \egal{} is not strategyproof.
    
    Now we consider the case of $n\ge 3$ or $m \geq 3$. 
    We build upon the counterexample for $n=m=2$.
    If $n$ is even, make $\frac{n}{2}-1$ copies of agent $1$ and $\frac{n}{2}-1$ copies of agent $2$; otherwise if
     $n$ is odd, make $\frac{n-1}{2}-1$ copies of agent $1$, $\frac{n-1}{2}-1$ copies of agent $2$, and add one more agent with score vector $(0.5, 0.5)$.
    Then, we add $m-2$ more dummy candidates that every agent has a score of $0$ for.
    It is easy to see that \egal{} still fails strategyproofness.
\end{proof}

As seen above, the rules we consider may fail strategyproofness even in the simplest case.
We will now focus on participation, which is generally viewed
as a less demanding property.
The IM mechanism is known to satisfy participation \cite[Thm.~8]{freeman2021truthfulbudget}.
We will now show that three of the five coordinate-wise aggregation rules and \egal{} also satisfy this property.
\begin{thm}\label{thm:participation_sum}
    The sum rule satisfies participation.
\end{thm}
\begin{proof}
For any $i \in N$ and $j \in [m]$, we have that
    \begin{align*}
        & \left| s_{ij} - \frac{1}{n-1} \sum_{i' \in N \setminus \{i\}}s_{i'j} \right|
        \\
        & \geq 
        \frac{n-1}{n} \cdot \left| s_{ij} - 
 \frac{1}{n-1} \sum_{i' \in N \setminus \{i\}}s_{i'j} \right| \\
        & = \left| s_{ij} - \left(\frac{n-1}{n} \cdot \frac{1}{n-1} \sum_{i' \in N \setminus \{i\}}s_{i'j} + \frac{1}{n} \cdot s_{ij}\right)\right| \\
        & = \left| s_{ij} - \frac{1}{n} \sum_{i' \in N \setminus \{i\}} s_{i'j}  - \frac{1}{n} \cdot s_{ij}\right|\\
        & = \left| s_{ij} - \frac{1}{n} \sum_{i' \in N} s_{i'j}\right|.
    \end{align*}
    Then, summing over $j \in [m]$ on both sides, we get that agent $i$'s disutility from not participating is at least her disutility from participating.
\end{proof}

\begin{thm} \label{thm:participation_max}
    The max rule satisfies participation.
\end{thm}
\begin{proof}
    Let $\mathbf{x}$ and $\mathbf{x}'$ be the outcome vector with $i$ not participating and participating, respectively. 
    Let $\mathbf{y}$ and $\mathbf{y}'$ be the corresponding pre-normalization vectors.
    If $\mathbf{y} = \mathbf{y}'$, then the property is trivially satisfied; thus we assume $\mathbf{y} \neq \mathbf{y}'$.

    We first prove two lemmas.
    \begin{lemma} \label{lem:participation_max_condition}
        For any $j \in [m]$, if $x'_j > x_j$, then $y'_j > y_j$.
    \end{lemma}
    \begin{proof}
        Under the max rule, for all $j \in [m]$, we have that $y'_j \geq y_j$, where the inequality is strict for at least one $j \in [m]$ (otherwise, $\mathbf{y} = \mathbf{y}'$, which contradicts our assumption).
        Summing over $j \in [m]$, we get that
        \begin{equation} \label{eqn:participation_max_y}
            \sum_{j \in [m]} y'_j > \sum_{j \in [m]} y_j.
        \end{equation}

        Suppose $x'_j > x_j$, and assume that $y'_j = y_j$, we get that
        \begin{equation*}
            x_j = \frac{y_j}{\sum_{k \in [m]}y_k} = \frac{y'_j}{\sum_{k \in [m]}y_k} > \frac{y'_j}{\sum_{k \in [m]} y'_k} = x'_j,
        \end{equation*}
        where the inequality follows from (\ref{eqn:participation_max_y}).
        Then,
        \begin{equation*}
            x'_j > x_j > x'_j,
        \end{equation*}
        which is a contradiction.
    \end{proof}

    \begin{lemma} \label{lem:participation_max_y_max1}
    $\sum_{j \in [m]} y'_j \geq 1$.
    \end{lemma}
    \begin{proof}
        Suppose for a contradiction that $\sum_{j \in [m]} y'_j < 1$.
        This means that 
        \begin{equation*}
            \sum_{j \in [m]} \max_{k \in N} s_{kj} = \sum_{j \in [m]} y'_j < 1.
        \end{equation*}
        This implies that for each agent $k \in N$, $\sum_{j \in [m]} {s_{kj}} < 1$, contradicting the definition of preference score vectors.
    \end{proof}
    
    Let $[m] = S \cup T$, where $S \subset [m]$ is the set of indices where for each $\alpha \in S$,
    \begin{equation*}
        x'_\alpha > x_\alpha,
    \end{equation*}
    and $T \subset [m]$ is the set of indices where for each $\beta \in T$,
    \begin{equation*}
        x'_\beta \leq x_\beta.
    \end{equation*}
    Now, we have that 
    \begin{equation} \label{eqn:participation_max_loss=gain}
        \sum_{\alpha \in S} (x'_\alpha - x_\alpha) = \sum_{\beta \in T} (x_\beta - x'_\beta).
    \end{equation}

    We claim that for all indices $\alpha \in S$,
    \begin{equation*}
        s_{i\alpha} \geq x'_\alpha.
    \end{equation*}
    Suppose for a contradiction that $s_{i\alpha} < x'_\alpha$.
    Then,
    \begin{equation*}
        s_{i\alpha} < x'_\alpha = \frac{y'_\alpha}{\sum_{k \in [m]}y'_k} \leq y'_\alpha,
    \end{equation*}
    where the rightmost inequality follows from Lemma \ref{lem:participation_max_y_max1}.
    Then, it follows that $y'_\alpha = y_\alpha$.
    However, by the definition of $S$ and Lemma \ref{lem:participation_max_condition}, we arrive at a contradiction.
    Hence, $s_{i\alpha} \ge x'_\alpha > x_\alpha$.

    This property shows that for indices in $S$, agent $i$'s participation results in a decrease in her disutility by exactly $\sum_{\alpha \in S} (x'_\alpha - x_\alpha)$. Together with (\ref{eqn:participation_max_loss=gain}), her net disutility (across $[m]$) from participating is nonnegative, and we obtain the desired result.
\end{proof}

\begin{thm} \label{thm:participation_min}
    The min rule satisfies participation.
\end{thm}
\begin{proof}
    Let $\mathbf{x}$ and $\mathbf{x}'$ be the outcome vector with $i$ not participating and participating, respectively. 
    Let $\mathbf{y}$ and $\mathbf{y}'$ be the corresponding pre-normalization vectors.
    If $\mathbf{y} = \mathbf{y}'$, then the property is trivially satisfied; thus we assume $\mathbf{y} \neq \mathbf{y}'$.

    We first prove two lemmas.
    \begin{lemma} \label{lem:participation_min_condition}
        Suppose 
        $y'_k > 0$ for some $k \in [m]$.
        Then, for any $j \in [m]$, if $x'_j < x_j$, then $y'_j < y_j$. 
    \end{lemma}
    \begin{proof}
        Under the min rule, for all $j \in [m]$, we have that $y'_j \leq y_j$, where the inequality is strict for at least one $j \in [m]$ (otherwise, $\mathbf{y} = \mathbf{y}'$, which contradicts our assumption).
        Summing over $j \in [m]$, we get that
        \begin{equation} \label{eqn:participation_min_ycondition}
            \sum_{j \in [m]} y'_j < \sum_{j \in [m]} y_j.
        \end{equation}
        By the lemma assumption, we have $\sum_{j\in[m]}y_j > \sum_{j\in[m]}y'_j > 0$.
        Suppose $x'_j < x_j$, and assume that $y'_j = y_j$, we get that
        \begin{equation*}
            x_j = \frac{y_j}{\sum_{k\in[m]} y_k} = \frac{y'_j}{\sum_{k\in[m]} y_k} < \frac{y'_j}{\sum_{k\in[m]} y'_k} = x'_j,
        \end{equation*}
        where the inequality follows from (\ref{eqn:participation_min_ycondition}).
        Then,
        \begin{equation*}
            x'_j < x_j < x'_j,
        \end{equation*}
        which is a contradiction.        
    \end{proof}
    \begin{lemma} \label{lem:participation_min_y_max1}
        $\sum_{j \in [m]} y'_j \leq 1$.
    \end{lemma}
    \begin{proof}
        Suppose for a contradiction that $\sum_{j \in [m]} y'_j > 1$.
        This means that
        \begin{equation*}
            \sum_{j \in [m]} \min_{k \in N} s_{kj} = \sum_{j \in [m]} y'_j > 1.
        \end{equation*}
        This implies that for each agent $k \in N$, $\sum_{j \in [m]} {s_{kj}} > 1$, contradicting the definition of preference score vectors.
    \end{proof}

    Let $[m] = S \cup T$, where $S \subset [m]$ is the set of indices where for each $\alpha \in S$,
    \begin{equation*}
        x'_\alpha < x_\alpha,
    \end{equation*}
    and $T \subset [m]$ is the set of indices where for each $\beta \in T$, 
    \begin{equation*}
        x'_\beta \geq x_\beta.
    \end{equation*}
    Now, we have that
    \begin{equation} \label{eqn:participation_min_loss=gain}
        \sum_{a \in S} (x_\alpha - x'_\alpha) = \sum_{\beta \in T} (x'_\beta - x_\beta).
    \end{equation}
    We claim that for all indices $\alpha \in S$, 
    \begin{equation*}
        s_{i\alpha} \leq x'_\alpha.
    \end{equation*}

    Consider the case where $y'_j = 0$ for all $j \in [m]$.
    This means $y'_\alpha = s_{i\alpha} = 0$ for all $\alpha \in S$.
    Then, $\mathbf{x}' = (\frac{1}{m},\dots,\frac{1}{m})$.  
    For $\alpha \in S$,
    \begin{equation*}
        s_{i\alpha} = 0 \leq \frac{1}{m} = x'_{\alpha},
    \end{equation*}
    and the property holds.

    Next, we consider the case where $y'_j > 0$ for some $j \in [m]$.
    Suppose for a contradiction that $s_{i\alpha} > x'_\alpha$.
    Then,
    \begin{equation*}
        s_{i\alpha} > x'_\alpha = \frac{y'_\alpha}{\sum_{k \in [m]}y'_k} \geq y'_\alpha,
    \end{equation*}
    where the rightmost inequality follows from Lemma \ref{lem:participation_min_y_max1}, and the denominator is positive since $y'_j > 0$ for some $j \in [m]$.
    Then, it follows that $y'_\alpha = y_\alpha$.
    However, by the definition of $S$ and Lemma \ref{lem:participation_min_condition}, we arrive at a contradiction.

    This property shows that for indices in $S$, agent $i$'s participation results in a decrease in her disutility by exactly $\sum_{a \in S} (x_\alpha - x'_\alpha)$. Together with (\ref{eqn:participation_min_loss=gain}), her net disutility (across $[m]$) from participating is nonnegative, and we obtain the desired result.
\end{proof}
\begin{thm} \label{thm:participation_med}
    The med rule satisfies participation when $n\le 2$ or $m=2$.
\end{thm}
\begin{proof}
    When $n\le 2$, the med rule is equivalent to the sum rule, which is known to satisfy participation by Theorem \ref{thm:participation_sum}.

    Next, we consider the case $m=2$. 
    Let $\mathbf{x}$ and $\mathbf{x}'$ be the outcome vectors returned by the med rule when $i$ does not participate and participates, respectively.
    Let the sorted scores for $p_1$ and $p_2$ be $z_{1,1}\le \dots\le z_{1,n}$ and $z_{2,1}\le\dots\le z_{2,n}$, respectively.
    Since score vectors are normalized, for each $i \in N$, we have that $z_{1,i} = 1 - z_{2,n-i+1}$.
    
    When $n$ is odd, the med rule will output $\mathbf{x}' = (z_{1,(n+1)/2}, z_{2,(n+1)/2})$ (refer to the proof of Theorem \ref{thm:rr_med}). 
    Thus, for some agent $i$, if $s_{i1} = z_{1,(n+1)/2}$ (and thus $s_{i2} = z_{2,(n+1)/2}$), $d_i(\mathbf{x}') = 0 \leq d_i(\mathbf{x})$ for any outcome $\mathbf{x}$.
    If $i$ is not an agent with the median score, suppose without loss of generality that $s_{i1} > z_{1,(n+1)/2}$ (i.e., $i$'s score for candidate $p_1$ is above the median).
    Then by not participating, the med rule will return the outcome $\mathbf{x} = (B_1/2, B_2/2)$, where $B_1 = z_{1,(n-1)/2} + z_{1,(n+1)/2}$ and $B_2 = z_{2,(n+1)/2} + z_{2,(n+3)/2}$.
    It is easy to see that $i$'s disutility increases from not participating.
    
    When $n$ is even, the med rule will return $\mathbf{x}' = (A_1/2,A_2/2)$, where $A_1 = z_{1,n/2} +z_{1,n/2+1}$ and $A_2 = z_{2,n/2} +z_{2,n/2+1}$ (refer to the proof of Theorem \ref{thm:rr_med}). 
    Suppose without loss of generality that $s_{i1} \geq z_{1,n/2+1}$.
    Then, in the absence of $i$, the med rule will return $\mathbf{x}' = (z_{1,n/2}, z_{2,n/2+1})$.
    Again, it is easy to see that $i$'s disutility increases from not participating.
\end{proof}
\begin{proposition} \label{prop:participation_prod}
    The prod rule may fail participation for all $n \geq 2$ and $m \geq 2$.
\end{proposition}
\begin{proof}
    Consider the score vectors $\mathbf{s}_1 = (0.49,0.51)$ and $\mathbf{s}_2 = (0.1,0.9)$. Then, the prod rule will return $\mathbf{x}' = (\frac{49}{508}, \frac{459}{508})$.
    However, if agent $1$ does not participate, the rule will return $\mathbf{x} = (0.1, 0.9)$.
    The disutility of agent $1$ is lower by not participating: $d_1(\mathbf{x}') \approx 0.787 > 0.78 = d_1(\mathbf{x})$.
\end{proof}
\begin{thm} \label{thm:participation_egal}
    \emph{\egal{}} satisfies participation.
\end{thm}
\begin{proof}
    Let $\mathcal{I}$ and $\mathcal{I}'$ be the instances where $i$ does not participate and participates, respectively.
    Also let $\mathbf{x}$ and $\mathbf{x}'$ be the corresponding outcome vectors returned by \egal{}.
    Consider the case where $\mathbf{x} \neq \mathbf{x}'$; otherwise participation is trivially satisfied.
    
    Suppose for a contradiction that 
    \begin{equation} \label{participation_egal_contradiction}
        d_i(\mathbf{x}) < d_i(\mathbf{x}').
    \end{equation}
    Let $U$ be the function that takes in a multiset of nonnegative real numbers and outputs a vector containing the elements in the set, sorted in non-increasing order.

    Then, define 
    $\mathbf{z}:= U(\{d_k(\mathbf{x}) : k \in N \setminus \{i\}\})$ and
    $\mathbf{z}':= U(\{d_k(\mathbf{x}') : k \in N \setminus \{i\}\})$.

    Consider two vectors $\mathbf{v},\mathbf{v}'$ of the same length $V$.
    We say that $\mathbf{v} = \mathbf{v}'$ if $v_k = v'_k$ for all $k \in [V]$.
    Furthermore, we say that $\mathbf{v} \succ \mathbf{v}'$ if  
    $v_k > v'_k$ for some $k \in [V]$ and $v_{k'} = v'_{k'}$ for all $k' < k$.

    Now, since $\mathbf{x}'$ is chosen over $\mathbf{x}$ under instance $\mathcal{I}'$, together with (\ref{participation_egal_contradiction}), we have that $z_k > z'_{k}$ for some $k \in N \setminus \{i\}$ and $z_{k'} = z'_{k'}$ for all $k' < k$, i.e.,
    \begin{equation} 
    	\mathbf{z} \succ \mathbf{z}'.
    \end{equation}
    However, since $\mathbf{x}$ is chosen over $\mathbf{x}'$ under instance $\mathcal{I}$,    
    \begin{equation} 
    	\mathbf{z}' \succeq \mathbf{z},
    \end{equation}
    giving us a contradiction.
Therefore, the claim is proven.
\end{proof}

\section{Pareto Optimality} \label{sec:po}

Next, we turn our attention to Pareto optimality.
We first establish the relationships between PO and RR.

\begin{lemma} \label{lem:po_rr}
Every PO outcome is RR.
\end{lemma}

\begin{proof}
    Let an outcome $\mathbf{x}$ be PO. Suppose for a contradiction that $\mathbf x$ is not RR. 
    Without loss of generality, this means that there exists some $\ell \in [m]$ such that $x_\ell > s_{1\ell} \geq s_{i\ell}$ for all $i \in N\setminus \{1\}$ (a similar argument can be made if the labels of the agents are swapped, or if the value of $x_\ell$ is strictly lower instead). 
    Hence, there must exist some other $\ell' \in [m] \setminus \{\ell\}$ such that $x_{\ell'} < s_{1\ell'}$ (due to the fact that score vectors are normalized).
    \begin{description}
        \item[Case 1: $x_\ell - s_{1\ell} \leq s_{1\ell'}-x_{\ell'}$.]
        Consider the score vector $\mathbf{x}'$ where 
    \begin{equation*}
        x'_j=
        \begin{cases} 
          s_{1\ell} & \text{ if } j = \ell;  \\
          x_{\ell'} + (x_\ell - s_{1\ell}) & \text{ if } j = \ell'; \\
          x_j & \text{ otherwise. }
       \end{cases}
    \end{equation*} 
    Then, together with the fact that $x_\ell - s_{1\ell} > 0$, we have 
    \begin{equation*}
    d_1(\mathbf{x}')= d_1(\mathbf{x}) - 2(x_\ell - s_{1\ell}) < d_1(\mathbf{x}).
    \end{equation*}
    Also, under $\mathbf{x}'$, for each $i\in N \setminus \{1\}$, agent $i$'s disutility decreases by $x_\ell - s_{1\ell}$ from coordinate $\ell$, and because he can maximally gain a disutility of $x_\ell - s_{1\ell}$ from coordinate $\ell'$, we get $d_i(\mathbf{x}') \leq  d_i(\mathbf{x}) - (x_\ell - s_{1\ell}) + (x_\ell - s_{1\ell}) = d_i(\mathbf{x})$.
    \item[Case 2: $x_\ell - s_{1\ell} > s_{1\ell'}-x_{\ell'}$.]
        Consider the score vector $\mathbf{x}'$ where 
    \begin{equation*}
        x'_j=
        \begin{cases} 
          x_\ell - (s_{1\ell'} - x_{\ell'}) & \text{ if } j = \ell;  \\
          s_{1\ell'} & \text{ if } j = \ell'; \\
          x_j & \text{ otherwise. }
       \end{cases}
    \end{equation*} 
    Then, together with the fact that $s_{1\ell'} - x_{\ell'} > 0$, we have 
    \begin{equation*}
        d_1(\mathbf{x}')= d_1(\mathbf{x}) - 2(s_{1\ell'} - x_{\ell'}) < d_1(\mathbf{x}).
    \end{equation*}
    Also, under $\mathbf{x}'$, for each $i\in N \setminus \{1\}$, agent $i$'s disutility decreases by $s_{1\ell'} - x_{\ell'}$ from coordinate $\ell$, and because he can maximally gain a disutility of $s_{1\ell'} - x_{\ell'}$ from coordinate $\ell'$, we get
    \begin{equation*}
        d_i(\mathbf{x}') \leq  d_i(\mathbf{x}) - (s_{1\ell'} - x_{\ell'}) + (s_{1\ell'} - x_{\ell'}) = d_i(\mathbf{x}).
    \end{equation*}
    \end{description}
    In both cases, $\mathbf{x}'$ Pareto dominates $\mathbf{x}$, contradicting the assumption that $\mathbf{x}$ is PO.    
\end{proof}

We now show that PO and RR are equivalent in the special cases of two agents or two candidates. We start by considering the case $n=2$.

\begin{thm} \label{thm:n=2_po_rr}
    For $n = 2$, an outcome is PO if and only if it is RR.
\end{thm}
\begin{proof}
    The forward direction has been established in Lemma~\ref{lem:po_rr}.
    We prove the other direction. Let the score vectors of the two agents be 
    $\mathbf{s}_1 = (s_{11},\dots,s_{1m})$ and $\mathbf{s}_2 = (s_{21},\dots,s_{2m})$. Then, for any outcome $\mathbf x$ we have that $d_1(\mathbf{x})+d_2(\mathbf{x}) \geq \sum_{j \in [m]} |s_{1j} - s_{2j}|$.
    Also, if $\mathbf{x}$ is RR, it holds that $d_1(\mathbf{x})+d_2(\mathbf{x}) \leq \sum_{j \in [m]} |s_{1j} - s_{2j}|$.
    Combining the two inequalities, for any RR outcome $\mathbf{x}$, we have that $d_1(\mathbf{x})+d_2(\mathbf{x}) = \sum_{j \in [m]} |s_{1j} - s_{2j}|$.
    Thus, any other RR outcome will have the same sum of disutilities, and if one agent were to have a strict decrease in disutility, the other agent must have a strict increase in disutility. This shows that $\mathbf{x}$ is PO.
\end{proof}
The same property is observed in the case of two candidates.

\begin{thm} \label{thm:m=2_po_rr}
    For $m = 2$, an outcome is PO if and only if it is RR.
\end{thm}
\begin{proof}
    The forward direction has been established in Lemma~\ref{lem:po_rr}.
    We prove the other direction. Let the outcome $\mathbf{x}$ be RR. Consider any other outcome $\mathbf{x}'$. Without loss of generality, suppose that $x_1 > x'_1$. Then, it must be that $x_2 < x'_2$. 
    Since $\mathbf{x}$ is RR, there exists some agent $i \in N$ such that $s_{i1} \geq x_1$. Correspondingly, $s_{i2} \leq x_2$.
    Note that 
    \begin{equation*}
        d_i(\mathbf{x}) = (s_{i1} - x_1) + (x_2 - s_{i2}) < (s_{i1} - x'_1) + (x'_2 - s_{i2}) = d_i(\mathbf{x}').
    \end{equation*}
    This means that the disutility of agent $i$ increases when going from $\mathbf{x}$ to $\mathbf{x}'$. Thus $\mathbf{x}$ is PO.
\end{proof}
However, if the numbers of agents and candidates exceed two, the relationship between PO and RR becomes one-sided. 
\begin{thm} \label{thm:PO_implies_RR}
    For $n \geq 3$ and $m \geq 3$, every PO outcome is RR. However, the converse may not hold even when $n = m = 3$.
\end{thm}
\begin{proof}
    We proved the forward direction in Theorem \ref{thm:n=2_po_rr}.
    
    As for the counterexample for the converse, consider the score vectors $\mathbf{s}_1 = (0,0,1)$, $\mathbf{s}_2 = (0,0.5,0.5)$, and $\mathbf{s}_3 = (0.5,0.5,0)$. 
    The outcome $\mathbf{x} = (\frac{1}{6}, \frac{1}{3}, \frac{1}{2})$ is RR. However, the score vector $(0, 0.5, 0.5)$ strictly benefits agent $2$ and does not hurt the other two agents, so $\mathbf{x}$ is not PO.
\end{proof}

Building on the above observations, we obtain the following results.

\begin{proposition} \label{prop:po_sumfailn3m3}
    The sum rule is PO when $n=2$ or $m=2$, but may fail to be so when both $n,m \geq 3$.
\end{proposition}
\begin{proof}
    For $n=2$ or $m=2$, the property follows as a corollary of Theorems \ref{thm:rr_sum}, \ref{thm:n=2_po_rr}, and \ref{thm:m=2_po_rr}.

    For $n = m = 3$, consider the score vectors $\mathbf{s}_1 = (0,0,1)$, $\mathbf{s}_2 = (0,0.5,0.5)$, and $\mathbf{s}_3 = (0.5,0.5,0)$. 
    Then, the sum rule returns $\mathbf{x}_\text{sum} = (\frac{1}{6}, \frac{1}{3}, \frac{1}{2})$. 
    However, the score vector $(0, 0.5, 0.5)$ strictly benefits agent $2$ and does not hurt the other two agents.
\end{proof}
\begin{proposition} \label{prop:po_maxmin}
    The max and min rules are PO when $m=2$, but may fail to be so when $m \geq 3$ (even when $n = 2$).
\end{proposition}
\begin{proof}
    For $m=2$, the property follows for both rules as a corollary of Theorems \ref{thm:RR_max_min} and \ref{thm:m=2_po_rr}.
    
    For the max rule, consider the case of two agents and three candidates where the score vectors are $\mathbf{s}_1 = (\frac{2}{3},0,\frac{1}{3})$ and $\mathbf{s}_2 = (\frac{1}{3},\frac{1}{3},\frac{1}{3})$. 
    Then, the max rule returns $\mathbf{x}_\text{max} = (0.5,0.25,0.25)$.
    However, the score vector $(\frac{1}{2},\frac{1}{6},\frac{1}{3})$ strictly benefits agent~$1$ and does not hurt agent~$2$.
    
    For the min rule, consider the case of two agents and three candidates where the score vectors are $\mathbf{s}_1 = (1,0,0)$ and $\mathbf{s}_2 = (0,1,0)$. Then, the min rule returns $\mathbf{x}_\text{min} = (\frac{1}{3},\frac{1}{3},\frac{1}{3})$.
    However, the score vector $(0.5,0.5,0)$ strictly benefits both agents.
\end{proof}

The following results are corollaries of our results on RR (Section~\ref{sec:score_unanimity_rr}) and the relationships between RR and PO established earlier in this section.

\begin{corollary} \label{prop:po_medfailn3m4}
    The med rule is PO when $n=2$ or $m=2$, but may fail to be so when $n \geq 3$ and $m \geq 4$.
\end{corollary}

\begin{corollary}
    The prod rule may fail to be PO for all $n \geq 2$ and $m \geq 2$.
\end{corollary}
\begin{corollary} \label{cor:PO_IM_m=2}
    The IM mechanism is PO when $m = 2$, but may fail to be so when $m \geq 3$ (even when $n = 2$).
\end{corollary}

As for the welfare-based rules, \util{} and \egal{} are both PO by definition.

We will now consider PO outcomes from an algorithmic perspective. Finding a PO outcome is easy: we can simply define the vector $\mathbf{x}$ to be exactly $\mathbf{s}_i$ for some agent $i \in N$. Then, $d_i(\mathbf{x}) = 0$, whereas $d_i(\mathbf{x}') > 0$ for any other outcome vector $\mathbf{x}'$. 
In some problem domains, determining whether an outcome is PO can be computationally difficult \cite{aziz2019efficient}. In contrast, our next result shows that in our setting checking the PO property is computationally tractable.
\begin{thm} \label{thm:po_check}
    The problem of determining whether an outcome~ $\mathbf{x}'$ is PO is polynomial-time solvable.
\end{thm}

\begin{proof}
    Suppose we are given an outcome $\mathbf{x}'$ for an instance $\mathcal{I} = (\mathbf{s}_1,\dots, \mathbf{s}_n)$ and we want to determine whether ${\mathbf x}'$ is PO.

    For each $i \in N$ and $j \in [m]$, let $z'_{ij} = |x'_j - s_{ij}|$. The quantities $z'_{ij}$ and $s_{ij}$ can be computed from the input, and will appear in the constraints of the linear program below.
    
    We formulate a linear program as follows:
    \begin{equation*}
        \text{minimize } \sum_{i \in N} \sum_{j \in [m]} z_{ij}, 
    \end{equation*}
    
    subject to the following constraints:
    
    (1) $\sum_{j \in [m]} x_j = 1$;
    
    (2) $x_j \geq 0$ for each $j\in [m]$;
    
    (3) $z_{ij} \geq |s_{ij} - x_j|$ (i.e., $z_{ij} \geq s_{ij} - x_j \text{ and } z_{ij} \geq x_j - s_{ij}$) for each $i \in N$, $j \in [m]$;
    
    (4) $\sum_{j \in [m]} z_{ij} \leq \sum_{j \in [m]} z'_{ij}$ for each $i\in N$.
        
    Note that $z_{ij}$ is an upper bound on $i$'s disutility from $p_j$, so the smaller it is, the better.
    Then, for the optimal solution $(z^*_{ij})_{i\in N, j\in [m]}$ returned, if, for some $i \in N$, $\sum_{j \in [m]} z^*_{ij} < \sum_{j \in [m]} z'_{ij}$,
    then we know that $\mathbf{x}'$ is not PO; otherwise it is.
\end{proof}

\section{Reinforcement} \label{sec:reinforcement}
The last property we consider is reinforcement.
We show that three of the five coordinate-wise aggregation rules satisfy this property. In contrast, the med rule satisfies it in the special case of two agents or two candidates, while the prod rule may fail it even in the simplest setting.
\begin{thm} \label{thm:reinforcement_3rules}
    The sum, max, and min rules satisfy reinforcement. 
\end{thm}
\begin{proof}
     Suppose we have two instances $\mathcal{I} = (\mathbf{s}_1,\dots,\mathbf{s}_n)$ and $\mathcal{I}' = (\mathbf{s}'_1,\dots,\mathbf{s}'_{n'})$.
     Let $N$ and $N'$ be the corresponding sets of agents.

    First we consider the sum rule. For each $j \in [m]$, if $\frac{1}{n} \sum_{i\in N} s_{ij} = x_j$ and $\frac{1}{n'} \sum_{i\in N'} s'_{ij} = x_j$, then $\sum_{i\in N} s_{ij} = n \cdot x_j$ and $\sum_{i\in N'} s'_{ij} = n' \cdot x_j$.
    Combining the two, we have that $\sum_{i\in N} s_{ij} + \sum_{i\in N'} s'_{ij} = (n+n') \cdot x_j$.
    This gives us 
    $\frac{1}{n+n'} \cdot \left( \sum_{i\in N} s_{ij} + \sum_{i\in N'} s'_{ij} \right) = x_j$, as desired.

    Next, we prove that both the max and min rules satisfy reinforcement.
    Let $\mathbf{y}$ and $\mathbf{y}'$ be the pre-normalization vectors for $\mathcal{I}$ and $\mathcal{I}'$, respectively.
    Suppose that our rules output the score vector $\mathbf{x}$ on $\mathcal{I}$ and $\mathcal{I}'$. 
    
    Now, for the min rule, consider the case where at least one of $\sum_{k \in [m]} y_k$ and $\sum_{k \in [m]} y'_k$ is $0$.
    Without loss of generality, assume that $\sum_{k \in [m]} y_k \geq \sum_{k \in [m]} y'_k = 0$.
    This means that $\min(y_k, y'_k) = 0$ for each $k \in [m]$, and hence $\sum_{k \in [m]} \min(y_k, y'_k) = 0$, giving us that the outcome of the combined instance is $(\frac{1}{m},\dots,\frac{1}{m})$, which is the same as the outcome for $\mathcal{I}'$ (and $\mathcal{I})$.
    This shows that reinforcement holds.
    Note that this case cannot happen for the max rule, since $\sum_{k \in [m]} y_k \geq 1$ and $\sum_{k \in [m]} y'_k \geq 1$.
    
    Henceforth, assume that $\sum_{k \in [m]} y_k >0$ and $\sum_{k \in [m]} y'_k > 0$.    
    Then, since for each $j \in [m]$,
    \begin{equation*}
        x_j = \frac{y_j}{\sum_{k \in [m]} y_{k}} = \frac{y'_j}{\sum_{k \in [m]} y'_{k}},
    \end{equation*}
    it follows that there exists a $\lambda \in \mathbb{R}_{\geq 0}$ such that $y_j = \lambda \cdot y'_j$ for all $j\in [m]$ (more specifically, $\lambda = \frac{\sum_{k \in [m]} y_{k}}{\sum_{k \in [m]} y'_{k}}$). 
    Now, if $\lambda \geq 1$, then for each $j \in [m]$, we have
    \begin{equation*}
        \frac{\max(y_j,y'_j)}{\sum_{k \in [m]} \max(y_k,y'_k)} = \frac{y_j}{\sum_{k \in [m]} y_{k}} = x_j
    \end{equation*}
    and 
    \begin{equation*}
        \frac{\min(y_j,y'_j)}{\sum_{k \in [m]} \min(y_k,y'_k)} = \frac{y'_j}{\sum_{k \in [m]} y'_{k}} = x_j.
    \end{equation*}
    If $\lambda < 1$, then for each $j \in [m]$, we have
    \begin{equation*}
        \frac{\max(y_j,y'_j)}{\sum_{k \in [m]} \max(y_j,y'_j)} = \frac{y'_j}{\sum_{k \in [m]} y'_{k}} = x_j
    \end{equation*}
    and 
    \begin{equation*}
        \frac{\min(y_j,y'_j)}{\sum_{k \in [m]} \min(y_j,y'_j)} = \frac{y_j}{\sum_{k \in [m]} y_{k}} = x_j.
    \end{equation*}
    Hence, the reinforcement property holds for both the max and min rules.
\end{proof}

\begin{thm} \label{thm:reinforcement_med_n=2m=2}
    The med rule satisfies reinforcement when $n=2$ or $m=2$.
\end{thm}
\begin{proof}
    When $n=2$, the med rule is equivalent to the sum rule, which is known to satisfy reinforcement by Theorem \ref{thm:reinforcement_3rules}.

    Next, we consider the case when $m=2$. 
    Suppose we have two instances $\mathcal{I}= (\mathbf{s}_1,\dots,\mathbf{s}_n)$ and $\mathcal{I}' = (\mathbf{s}'_1,\dots,\mathbf{s}'_{n'})$.
    Let $N$ and $N'$ be the corresponding sets of agents.
    Let $\mathbf{x}$ be outcome vector returned by the med rule on both instances.

    Note that for the case of $m = 2$, it suffices to reason about the scores of $p_1$. We have the following lemma.
    
    \begin{lemma} \label{lem:reinforcement_med_sets}
        Suppose we have two sets of real numbers in $[0,1]$: $L = \{\ell_1,\dots,\ell_n \}$ and $L' = \{\ell'_1,\dots,\ell'_{n'}\}$ of cardinality $n$ and $n'$ respectively. 
        If $\text{med}(L) = \text{med}(L') = \gamma$, then $\text{med}(L \cup L') = \gamma$.
    \end{lemma}
    \begin{proof}
        Suppose without loss of generality that the elements in each set are labelled such that they are sorted in non-decreasing order, i.e., $\ell_1 \le \dots \le \ell_n$ and $\ell'_1 \le \dots \le \ell'_{n'}$.
        
        If $n$ and $n'$ are both odd, then we have that $\text{med}(L) = \ell_{(n+1)/2} \text{ and } \text{med}(L') = \ell'_{(n'+1)/2}$.
        Let $\text{med}(L) = \text{med}(L') = \gamma$.
        Also denote $L_- = \{ \ell_1,\dots, \ell_{(n-1)/2} \}$ and $L_+ =  \{ \ell_{(n+3)/2},\dots,\ell_n \}$. 
        Correspondingly for $L'$, denote
        $L'_- = \{ \ell'_1,\dots, \ell'_{(n'-1)/2} \}$ and $L'_+ =  \{ \ell'_{(n'+3)/2},\dots, \ell'_{n'} \}$.
        Consider the set $L \cup L'$.
        We have that $|L_- \cup L'_-| = \frac{n+n'}{2} - 1$ and $|L_+ \cup L'_+| = (n - \frac{n+3}{2}+1) + (n' - \frac{n'+3}{2}+1) = \frac{n+n'}{2} - 1 = |L_- \cup L'_-|$.
        This means that $\text{med}(L \cup L') = \ell_{(n+1)/2} = \ell'_{(n'+1)/2} = \gamma$.

        If $n$ and $n'$ are both even, then we have that $\text{med}(L) = \frac{1}{2}(\ell_{n/2} + \ell_{n/2+1})$ and $\text{med}(L') = \frac{1}{2}(\ell'_{n'/2} + \ell'_{n'/2+1})$.
        Let $\text{med}(L) = \text{med}(L') = \gamma$.
        Also denote $L_- = \{ \ell_1,\dots, \ell_{n/2-1} \}$ and $L_+ =  \{ \ell_{n/2+2},\dots,\ell_n \}$.
        Correspondingly for $L'$, denote
        $L'_- = \{ \ell'_1,\dots, \ell'_{n'/2-1} \}$ and $L'_+ =  \{ \ell'_{n'/2+2},\dots, \ell'_{n'} \}$.
        Consider the set $L \cup L'$.
        We have that $|L_- \cup L'_-| = \frac{n+n'}{2} - 2$ and $|L_+ \cup L'_+| = (n - \frac{n}{2}-1) + (n' - \frac{n'}{2}-1) = \frac{n+n'}{2} -2 = |L_- \cup L'_-|$.

        Now, since $\frac{1}{2}(\ell_{n/2} + \ell_{n/2+1}) =\frac{1}{2}(\ell'_{n'/2} + \ell'_{n'/2+1})$, we have that
        \begin{equation*}
            \ell_{n/2} + \ell_{n/2+1} = \ell'_{n'/2} + \ell'_{n'/2+1}.
        \end{equation*}
        This means that if $\ell_{n/2} \leq \ell'_{n'/2}$, then $\ell_{n/2+1} \geq \ell'_{n'/2+1}$.
        Similarly, if $\ell_{n/2} > \ell'_{n'/2}$, then $\ell_{n/2+1} < \ell'_{n'/2+1}$.
        Consequently, we have $\text{med}(L \cup L') = \frac{1}{2}(\ell_{n/2} + \ell_{n/2+1}) = \frac{1}{2}(\ell'_{n'/2} + \ell'_{n'/2+1}) = \gamma$.

        Consider the last case where one of $n$ and $n'$ is even and the other is odd. 
        Without loss of generality, suppose $n$ is even and $n'$ is odd.
        Then, we have $\text{med}(L) = \frac{1}{2}(\ell_{n/2} + \ell_{n/2+1})$ and $\text{med}(L') = \ell'_{(n'+1)/2}$.
        Let $\text{med}(L) = \text{med}(L') = \gamma$.
        Also denote $L_- = \{ \ell_1,\dots, \ell_{n/2} \}$ and $L_+ =  \{ \ell_{n/2+1},\dots,\ell_n \}$ (note that this is defined differently from the previous case).
        For $L'$, denote
        $L'_- = \{ \ell'_1,\dots, \ell'_{(n'-1)/2} \}$ and $L'_+ =  \{ \ell'_{(n'+3)/2},\dots, \ell'_{n'} \}$.

        Consider the set $L \cup L'$.
        Then, $|L_- \cup L'_-| = \frac{n+n'-1}{2}$ and $|L_+ \cup L'_+| = (n - \frac{n}{2}) + (n' - \frac{n'}{2} - \frac{1}{2}) = \frac{n+n'-1}{2} = |L_- \cup L'_-|$.
        This means that $\text{med}(L \cup L') =  \ell'_{(n'+1)/2} = \gamma$.
    \end{proof}
    By letting $L = \{s_{i1} :  i \in N\}$, $L' = \{ s'_{i1} : i \in N'\}$, and $\gamma = x_1$ in Lemma~\ref{lem:reinforcement_med_sets}, the result follows.
\end{proof}

\begin{proposition} \label{prop:reinforcement-prod}
    The prod rule may fail reinforcement for all $n \geq 2$ and $m \geq 2$.
\end{proposition}
\begin{proof}
    Consider the case of two agents and two candidates, where the score vectors of the first profile are $\mathbf{s}_1 = (0.8,0.2)$ and $\mathbf{s}_2 = (0.3,0.7)$, and the score vectors of the second profile are the same: $\mathbf{s}'_1 = (0.8,0.2)$ and $\mathbf{s}'_2 = (0.3,0.7)$. Then, the prod rule will return $\mathbf{x} = (\frac{12}{19},\frac{7}{19})$ for both profiles; but with the combined profile, it will return $(\frac{144}{193}, \frac{49}{193})$.
\end{proof}
As for the rest of the rules, \util{} and the IM mechanism have been shown to satisfy reinforcement \cite{freeman2021truthfulbudget}. We show that \egal{} also satisfies this axiom.

\begin{thm} \label{thm:reinforcement_egal}
    \emph{\egal{}} satisfies reinforcement.
\end{thm}
\begin{proof}
    Let $\mathcal{I}$ and $\mathcal{I}'$ be two instances, and let $\mathbf{x}$ be the outcome vector returned by \egal{} in both instances.
    Let $N$ and $N'$ be the sets of agents in those instances, respectively.
    Consider a third instance $\mathcal{I}^*$ derived by combining the two instances, with the set of agents $N^* := N \cup N'$, and let the outcome returned by \egal{} in this combined instance be $\mathbf{x}^*$.

    Let $U$ be the function that takes in a multiset of nonnegative real numbers and outputs a vector containing the elements in the set, sorted in non-increasing order.

    Then, define 
    \[
    \mathbf{z} := U(\{d_k(\mathbf{x}) : k \in N\}),\, 
    \mathbf{z}^* := U(\{d_k(\mathbf{x}^*) : k \in N\})
    ,\] 
    \[
    \mathbf{y} := U(\{d_k(\mathbf{x}) : k \in N'\}),\, 
    \mathbf{y}^* := U(\{d_k(\mathbf{x}^*) : k \in N'\}), 
    \]    
    \[
    \mathbf{w} := U(\{d_k(\mathbf{x}) : k \in N^*\}),\,
    \mathbf{w}^* := U(\{d_k(\mathbf{x}^*) : k \in N^*\}).
    \]

    Consider two vectors $\mathbf{v},\mathbf{v}'$. of the same length $V$.
    We say that $\mathbf{v} = \mathbf{v}'$ if $v_k = v'_k$ for all $k \in [V]$.
    Furthermore, we say that $\mathbf{v} \succ \mathbf{v}'$ if  
    $v_k > v'_k$ for some $k \in [V]$ and $v_{k'} = v'_{k'}$ for all $k' < k$.
    
    Additionally, for any two vectors $\mathbf{u}$ and $\mathbf{v}$ of possibly different length, let $\mathbf{u} || \mathbf{v}$ be the vector containing all the elements in both $\mathbf{u}$ and $\mathbf{v}$, sorted in non-increasing order.
    Note that if we have four vectors $\mathbf{u}, \mathbf{u}', \mathbf{v},\mathbf{v}'$ such that $\mathbf{u} \succeq \mathbf{u}'$ and $\mathbf{v} \succeq \mathbf{v}'$, then $\mathbf{u}||\mathbf{v} \succeq \mathbf{u}'||\mathbf{v}'$.

    Since the outcome returned by \egal{} under the combined instance $\mathcal{I}^*$ is $\mathbf{x}^*$, it must be that $\mathbf{w} \succeq \mathbf{w}^*$. 
    If $\mathbf{w} = \mathbf{w}^*$, then $\mathbf{x} = \mathbf{x}^*$ (since we assume that a consistent tie-breaking rule is used for \egal{}).
    Suppose for a contradiction that \egal{} does not satisfy reinforcement, i.e., $\mathbf{w} \succ \mathbf{w}^*$. 

    Now, since $\mathbf{x}$ is chosen over $\mathbf{x}^*$ in instance $\mathcal{I}$,
    \begin{equation}
       \mathbf{z}^* \succeq \mathbf{z}.
    \end{equation}
    Also, since $\mathbf{x}$ is chosen over $\mathbf{x}^*$ in instance $\mathcal{I}'$,
    \begin{equation}
       \mathbf{y}^* \succeq \mathbf{y}.
    \end{equation}
    Then, we have that
    \begin{align*}
        \mathbf{w}^*
        & = \mathbf{z}^* || \mathbf{y}^* 
        \succeq \mathbf{z} || \mathbf{y} 
        = \mathbf{w},
    \end{align*}
    which contradicts our assumption.
    Hence, the result follows.
\end{proof}

\section{Conclusion}
In this work, we analyzed two natural classes of aggregation rules for portioning with cardinal preferences (namely, those based on coordinate-wise aggregation and welfare optimization) as well as the IM mechanism with respect to a number of appealing axiomatic properties. Some of these axioms were proposed in prior work, while others, such as score-representation and score-unanimity, are new. 
We show that a simple rule that takes the average of the proposals satisfies most of our properties. In contrast, while the IM mechanism possesses the desirable strategyproofness property, it violates some of the other axioms. Thus, in settings where strategyproofness is not a major concern, IM is not necessarily the optimal aggregation rule. 

Besides resolving the open questions that remain, avenues for future research include the following: (1) considering other coordinate-wise aggregation rules or characterizing certain rules within this class, (2) studying other classes of aggregation rules, (3) investigating other disutility models (e.g., $\ell_2$ or $\ell_\infty$ norms), and (4) finding a suitable analog of Nash welfare in this setting and exploring its axiomatic properties.

\ack This work was partially supported by the AI Programme of The Alan Turing Institute, the Singapore Ministry of Education under grant number MOE-T2EP20221-0001 and by an NUS Start-up Grant.
We thank the anonymous reviewers for their constructive feedback.

\bibliography{abb,ecai}
\end{document}


