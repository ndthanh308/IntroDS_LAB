% This is samplepaper.tex, a sample chapter demonstrating the
% LLNCS macro package for Springer Computer Science proceedings;
% Version 2.21 of 2022/01/12
%
% \documentclass[runningheads]{llncs}
% %
% \usepackage[T1]{fontenc}
% \usepackage{algorithm}
% \usepackage[noend]{algpseudocode}
% \usepackage{multirow}

% % T1 fonts will be used to generate the final print and online PDFs,
% % so please use T1 fonts in your manuscript whenever possible.
% % Other font encondings may result in incorrect characters.
% %
% \usepackage{graphicx}

% \usepackage[capitalize]{cleveref}
% \crefname{section}{Sec.}{Secs.}
% \Crefname{section}{Section}{Sections}
% \Crefname{table}{Table}{Tables}
% \crefname{table}{Tab.}{Tabs.}


% Used for displaying a sample figure. If possible, figure files should
% be included in EPS format.
%
% If you use the hyperref package, please uncomment the following two lines
% to display URLs in blue roman font according to Springer's eBook style:
%\usepackage{color}
%\renewcommand\UrlFont{\color{blue}\rmfamily}
%

% \usepackage{xcolor}

% \usepackage{soul}
% \usepackage{tabularx,booktabs} 
% \usepackage{algorithm}

% \usepackage{xr}
% \usepackage{zref-xr}
% \externaldocument{main}



% \newcolumntype{P}[1]{>{\centering\arraybackslash}p{#1}}
% \newcolumntype{M}[1]{>{\centering\arraybackslash}m{#1}}

\title{Appendix for “Fed-MAS”}
% \renewcommand\thefigure{\thesection.\arabic{figure}}    


\author{Marawan Elbatel\inst{1,2} \and
Hualiang Wang\inst{1} \and Robert Martí\inst{2} \and 
Huazhu Fu\inst{3}  \and Xiaomeng Li\inst{1} 
}


\authorrunning{Elbatel et al.} 
% % First names are abbreviated in the running head.
% % If there are more than two authors, 'et al.' is used.
% %
\institute{The Hong Kong University of Science and Technology \and Computer Vision and Robotics Institute, University of Girona \and Institute of High Performance Computing (IHPC), Agency for Science, Technology and Research (A*STAR), Singapore}


\maketitle

\appendix
\setcounter{figure}{3}
\setcounter{table}{3}
% Attributes visualization
%Algorithm
%Table with ablation, and two figures of CIFAR-10-
%TSNE visualization
%Table with BSM

\begin{table}
\parbox{.45\linewidth}{
\centering
\caption{HyperKvasir $\lambda_{f}$ ablation.}
    \resizebox{0.5\columnwidth}{!}{
 \begin{tabular}{l|ccc|ccc}
        \toprule
    {\multirow{2}{*}{Method}} &\multicolumn{3}{c|}{IID} & \multicolumn{3}{c}{non-IID}          
    \\
    \cline{2-7}
        & $\lambda_{f}=0$ & $\lambda_{f}=1$ & $\lambda_{f}=3$ & $\lambda_{f}=0$ & $\lambda_{f}=1$ & $\lambda_{f}=3$
        \\
        \toprule
        Fed-MAS & 
        60.28&61.43&\textbf{62.08}&
        59.15& 61.08 & \textbf{61.61}\\
        \bottomrule
    \end{tabular}
    \label{apx:tbl:free_ablation}
    }
    \hfill
    \centering
    \caption{HyperKvasir $f_{\xi}$ ablation non-IID.}
    \resizebox{0.4\columnwidth}{!}{
 \begin{tabular}{l|cc}
\hline
  \multicolumn{1}{l|}{$f_{\xi}$} & All & B-Acc  \\
   \toprule
           CLIP-ViTB/32& 
	60.34 &61.39$\pm$2.1
 \\
        MoCo-RN50 & 
	60.57 &61.61$\pm$1.0
 \\
 \bottomrule
    \end{tabular}
    \label{apx:tbl:ablation_lunch}
    }
}
\hfill
\parbox{0.5\linewidth}{
\centering
\caption{HyperKvasir FL methods with local BRM~\cite{Ren2020balms}.}
    \resizebox{0.5\columnwidth}{!}{
\begin{tabular}{l|cc|cc}
\hline
\multicolumn{1}{c|}{\multirow{2}{*}{Method}}
& \multicolumn{2}{c|}{All}&\multicolumn{2}{c}{B-Acc}\\
\cline{2-5}
&IID & non-IID & IID & non-IID2\\
\hline

FedAvg&58.92	
        &58.24 & 60.28$\pm$0.6&59.15$\pm$1.3 \\

FedProx&59.37
        &58.86&60.47$\pm$1.3&59.64$\pm$2.0\\

Moon &59.45
     &58.72&60.69$\pm$0.9&59.66$\pm$0.8\\

Ours &\textbf{61.05}
&\textbf{60.57}&\textbf{62.08$\pm$0.2}&\textbf{61.61$\pm$1.4} 
   \\
   \bottomrule
\end{tabular}
    \label{apx:tbl:FL_BRM}
    }
      \hfill
    \centering
% <--
\caption{Using a plug-in cRT~\cite{Kang2020Decoupling} on HyperKvasir on non-IID.}
\resizebox{0.3\columnwidth}{!}{
\begin{tabular}{l|cc}
\hline
Method + cRT 
& All 
& \multicolumn{1}{c}{B-Acc}\\ \hline 
Decoupling~\cite{Kang2020Decoupling}
&54.21
&55.6
\\
BSM-FL~\cite{Ren2020balms} 
&62.67
& 63.11
\\
\textbf{Ours} 
&\textbf{65.05}
&\textbf{65.11}
\\

\bottomrule
\end{tabular}
    \label{apx:tbl:FL_CRT}
}

}
\end{table}

\begin{table}[t]
  \centering
  \caption{Flamby-ISIC~\cite{NEURIPS2022_232eee8e_flamby} results on the first fold with the global model (gFL) and the local models (pFL) with ImageNet Weight Initialization. MOON~\cite{li2021model_moon} and FedProx~\cite{LiSZSTS20_fedprox} are reported with local BRM~\cite{Ren2020balms}.}
  \setlength{\tabcolsep}{12pt}
  \label{tbl:sota_results_isic_fl_2}
  \resizebox{0.8\columnwidth}{!}{
  \begin{tabular}{l|cc|l|cc|}
    \hline
    \multirow{2}{*}{Method}
     & \multicolumn{2}{c|}{\textbf{Metric}} 
     &   \multirow{2}{*}{Method}
     & \multicolumn{2}{c|}{\textbf{Metric}}
     \\ \cline{2-3} \cline{5-6}
     & gFL &pFL&&gFL &pFL
     \\
    \hline
MOON ($\mu=0.01$)&72.13&80.03&
FedProx ($\mu=0.1$)& 72.47&79.82 \\

MOON ($\mu=0.1$)&72.45&79.64&
FedProx ($\mu=0.01$)& 73.25 &79.82 \\


MOON ($\mu=1$) &73.12&79.46&
FedProx ($\mu=0.001$)& 73.52 &79.70 \\
FedLC~\cite{pmlr-v162-zhang22p_fedlc} &68.07&78.60&
BSM-FL~\cite{Ren2020balms}& 72.83 &79.79 \\

~\cite{Ren2020balms} w/ KD ($\lambda_{f}$=1)
&72.26&79.66&
~\cite{Ren2020balms} w/ KD  ($\lambda_{f}$=3)
&72.85&80.06\\

\textbf{Fed-MAS} ($\lambda_{f}$=1)
&72.94&82.73&
\textbf{Fed-MAS} ($\lambda_{f}$=3)
&\textbf{74.12}&\textbf{83.28}\\
\bottomrule
  \end{tabular}
  }
    
    % \vspace{-0.5em}
\end{table}


\begin{algorithm}
\centering
\caption{Pseudocode for Fed-MAS.}\label{algo:fedfree}
\begin{algorithmic}[1]
\State \text{\textbf{Notations}} total number of clients (C), server (S), total communication rounds (R), local epochs (E), learning rate ($\eta$), and a set of client’s data sliced into batches of size B ($\mathcal{B}$).
\State \text{\textbf{\underline{ServerExecution:}}}
\State Init $\theta_{glob}^{1}$
\For{\textit{each round} $r = 1, ..., R$}
  % \State $S_t \gets$ (Selection of c clients from C)\;
  \For{\textit{client} $c \in C$ \textit{in parallel}}
      \State $\theta_{c}, {RF}_{c}\gets$\textbf{LocalUpdate}($\theta_{glob}^r$);\;
\EndFor
  % \State $\bar{RF}_c \gets {\frac{RF_{c}}{\sum\limits_{j}RF_{j}}}$ \;
  \State ${\theta^{r+1}_{glob}\gets}$\textbf{DLMA}($RF_{c},\theta'_{c}$,c = 1 to C); //~\cref{eq:tempavg}\; 
   \EndFor
\State \textbf{Return}  $\theta_{glob}^{R}$
%   \State 
\State \underline{\textbf{LocalUpdate}} ($\theta_{glob}$): 
    \State Init $\hat{w}_{k}=0;$\;
    \State Init $ w_k  =\mathcal{L}_{k}^{\theta_{glob}};$ \;
    % following \cref{eq:global_measurement}
    \For{\textit{each local epoch} $e = 1, ..., E$}
        \For{\textit{each batch b} $\in\mathcal{B}$}
             \State $\mathcal{L}_{total} =\mathcal{L}_{sup}+ \lambda_{f} \mathcal{L}_{f};$~//~\cref{eq:loss_total}\;
             
            \State $\theta' \leftarrow \theta' - \eta \triangledown \mathcal{L}_{total};$ \;
            \State $\hat{w}_{k} \leftarrow  \hat{w}_{k} + \mathcal{L}_f(b_k);$ // running distillation loss mean for each class k\; 
            %~\cref{eq:local_estimation} 
        \EndFor
    \EndFor
\State $RF=\sum\limits_{k=1}^{K}  w_{k} \hat{w}_{k};$ \; 
// RF~$\uparrow \approx$ divergence~$\theta_{glob},f_{\xi}\uparrow$
\State \textbf{Return} $\theta',RF$
\end{algorithmic}
\end{algorithm}

% Figure environment removed










% Figure environment removed

\clearpage

% % Figure environment removed

% \section{SISSI Training Dynamics.}
% % Figure environment removed

% % Figure environment removed 
% \subsubsection{Acknowledgements} OFI for CellLab images and annotation of testset.

% \end{document}
