% -------------------------------------------------------
\subsection{Other Considerations} \label{other-design}
% -------------------------------------------------------

\parab{From single source to multiple sources.}
Multicast source switching is common in datacenter communications. For example, in the \textit{PB} phase of HPL, different nodes play as the multicast source in different epochs. \sys can support this communication requirement by switching multicast source inside the group without reestablishing QPs. In particular, when the last multicast source finishes transmission, the multicast members use the already established QPs for the subsequent multicast communication. The multicast forwarding table maintained by the switches stays almost untouched. In Appendix~\ref{apx:source-switch}, we provide a detailed description of multicast source switching.

\parab{Congestion control.} 
As described above, we reuse the entire RC transport at the end-host for RNIC compatibility. This means that we also reuse the built-in congestion control (CC) mechanism to regulate the multicast sending rate. Rationally, a multicast congestion control algorithm should match its sending rate with the most congested path (\ie, bottleneck) of its data distribution tree~\cite{widmer2001extending, rizzo2000pgmcc}.

To this end, we enhance the switch feedback aggregation with a congestion signal filtering mechanism while remaining the end-host CC unchanged. Specifically, we maintain a congestion counter for each link at the switch, recording the congestion signal from the receivers or downstream switches. We then perform signal filtering to only pass through the congestion signal from the most congested link\footnote{There are different implementations of congestion signals. Some adopt stand-alone CNP, while others reuse the ACK to carry the congestion bits. Our congestion signal filtering design works for them all.} (forming a path when cascading each switch link end-to-end). Further, a periodic aging mechanism is added to update the congestion counter to match the frequently changing network dynamics.