%--------------------------------------------------------------------------
\section{\sys}\label{design}
%--------------------------------------------------------------------------

We first describe the key ideas and design challenges of \sys ($\S$\ref{key-chalge}). Then we overview the \sys structure ($\S$\ref{design-overview}), briefing its main components and how they work together. After that, we elaborate these design components one by one ($\S$\ref{connection-handle}-$\S$\ref{other-design}). 

\subsection{Key Ideas and Challenges} \label{key-chalge}

\sys focuses on simultaneously achieving (\romannumerber{1})~the optimal multicast forwarding, (\romannumerber{2})~the efficient utilization of the advanced capabilities of RDMA, and (\romannumerber{3})~satisfying the deployment requirements. To this end, the key ideas behind \sys are to (\romannumerber{1})~perform the optimal multicast forwarding in the in-fabric distribution manner~\cite{crowcroft1988multicast, diab2022orca, shahbaz2019elmo}; and (\romannumerber{2})~re-purpose the native RDMA RC logic with careful switch coordination for an efficient multicast transport. 

However, there are critical challenges blocking the way: \textit{how to achieve integration between the optimal multicast forwarding and the existing RDMA RC logic}. Specifically, there are two main compatibility issues.

The first is that the connection-oriented logic~\cite{rocev2} of RC cannot find the associated QPs when receiving the traditional-forwarded multicast packets, illustrated in Fig.~\ref{fig:design:incompt}. During the traditional multicast forwarding, switches won't change the packet's layer-4 header. Switches either copy the entire packet~\cite{crowcroft1988multicast, Infiniband} or only modify the layer-3 (IP) header for layer-3 multicast routing~\cite{diab2022orca, shahbaz2019elmo}. Thus all packets contain an identical layer-4 header, which only matches at most one connection. The non-matched RNIC will discard these packets as it cannot find the associated QP and Queue Pair Context (QPC) based on the non-matched layer-4 header. 

Secondly, even if receivers can accept the packets and find the associated QPs, the second incompatibility impeding the leverage of RC is the existing reliability logic~\cite{zhu2015congestion, guo2016rdma}. The standard reliability logic is designed for single feedback (including ACK, NACK, CNP, \etc) stream from a single receiver. Thus, multiple feedback streams from multiple receivers can confuse RC and disturb its loss detection and retransmission routines. 

\sys is a fabric-supported multicast protocol that substantially differs from the traditional layer-3 in-fabric approaches. \sys systematically integrates its design components to address these two incompatibilities.

%\sys systematically integrates its design components to address these two incompatibilities. \sys is a fabric-supported multicast protocol that substantially differs from the traditional layer-3 in-fabric approaches. \sys extends the legacy table-based routing approach, including the control-plane table registration and the data-plane data (feedback) packets forwarding. \sys combines layer-4 states into its table design. Based on the extended table, \sys performs two operations. Firstly, \sys modifies the connection-related states in the packet header to achieve a virtual one-to-many connection. Then \sys aggregates feedback (including ACK, NACK, CNP, \etc) packets to reuse the existing reliability logic of RDMA RC. 

% Figure environment removed

% Figure environment removed

\subsection{\sys Overview} \label{design-overview}
Fig.~\ref{fig:overview} illustrates the architecture of \sys. The working steps of \sys are composed of three main phases: the control-plane multicast group registration (\circled{1} $\&$ \circled{2}), the data-plane one-to-many data forwarding (\circled{3}), and the data-plane many-to-one feedback aggregation (\circled{4}).

For the control-plane multicast group registration, \sys elaborately extends the legacy multicast forwarding table structure by integrating layer-4 states. We develop an out-of-band UDP-based protocol called \envelope to register forwarding table to switches—the master node\footnote{For ease of understanding, readers can think master node as the multicast source. Actually, the master node can be any node in the multicast group.} in the multicast group collects the layer-4 states of other nodes, fits these states into the \envelope packet format, and transmits to switches and other nodes for building forwarding table and affirming the multicast membership, respectively (\circled{1}). The involved nodes will answer ACKs to the master node to confirm its participation (\circled{2}). Due to space limitations, we present the extended forwarding table structure in the main body, leaving the remaining control-plane details in Appendix~\ref{apx:regis}.

For the data-plane one-to-many data forwarding ($\S$\ref{connection-handle}), \sys reserves the optimal multicast forwarding and achieves a virtual one-to-many connection. Let's take the multicast communication in Fig.~\ref{fig:overview} for an example. $S$ only transmits data once via the existing RC connection. Then the switches in the multicast tree copy data and forward them to multiple receivers via the optimal paths. In addition, some specific switches replace the connection-related states in the packet header to match different QPs in different receivers (\circled{3}), thus achieving a virtual one-to-many connection. For instance, $L_1$, $S_1$, $C_2$, and $S_3$ copy and forward data packets to specific ports that are identified in the forwarding table. Additionally, $L_2$, $L_3$, and $L_4$ replace some connection-related states in packet header to match QPs in $R_1$, $R_2$, and $R_3$.

After receiving data packets, receivers, $R_1$, $R_2$, and $R_3$, generate normal ACK/NACK/CNP packets following the existing RC logic. Then the fabric aggregates ACK, filters NACK/CNP, and forwards these feedbacks to the sender (\circled{4}) ($\S$\ref{ack-aggregation} $\&$ $\S$\ref{other-design}). We take ACK as an example. As shown in Fig.~\ref{fig:overview}, $L_2$, $L_3$, $L_4$, and $C_2$ only forward ACK packets to next-hop switches, as there is only one ACK stream as input. $S_3$ and $S_1$ perform ACK aggregation, as there are multiple ACK streams as input. $L_1$ changes the connection-related states in the ACK header to match the $S$'s QP before forwarding the aggregated ACK. %When there are NACK packets, the fabric carefully filters NACK packets to enable the sender can correctly detect and retransmit the lost packet. 

The aggregated ACK and NACK packets enable the sender to transmit the subsequent new data packets or retransmit the lost ones. The filtered CNPs are used to adjust the sender's sending rate. Every data packet will go through the above four steps until the multicast communication job is finished.

%% -------------------------------------------------------
\subsection{Multicast Group Registration} \label{data-forward}
% -------------------------------------------------------
The multicast forwarding of \sys is based on forwarding table that installed on switches. Table-based forwarding is a commonly adopted approach by existing multicast solutions, \eg, the IP multicast \cite{crowcroft1988multicast}. \sys's forwarding table is indexed by \textit{multicast group IP address} (\aka, GroupIP), and \sys follows the indexed introduction to copy packets and forward them to specific output ports. Through the GroupIP-based multicast forward is similar, \sys has primary difference with the traditional IP multicast. 

%\textsf{test} \texttt{test} \textsl{test} \textit{test}

% Figure environment removed

% Figure environment removed

The first difference is that, the multicast forwarding table registration of \sys is centralized, while the IP multicast performs a distributed registration algorithm. The key insight pushing registration from distributed to centralized is that the datacenter is highly autonomous and the multicast membership is highly controlled. This centralized approach is used in various datacenter application and frameworks \cite{shahbaz2019elmo, diab2022orca}. Secondly, the IP multicast only registers layer-3 states to switches; thus only support the best-efford delivery. \sys, on the other hand, registers both layer-3 and layer-4 states to switches for the following connectivity and reliability supports.

We describe the group registration of \sys here. The registration is centralized, and mainly supported by an application-assigned master node. The master node in the multicast group collects the connection states (including the layer-3 IP and layer-4 IB information) of all other multicast nodes through an out-of-band protocol (\eg, TCP). After information collection, the master node fits those connection states into the packet format of the self-developed \envelope protocol, which is illustrated in Fig. \ref{fig:envelope-packet}. In \envelope pakcet, the IP option is GroupIP followed by a specific UDP option. The palyload contains metadata and detailed connection states of each node. In addition to node IP, the states of each node also include QPN which is used for connectivity building afterward. The metadata contains connection statistics, where $seq$ and $total$ indicate the sequence and total number of \envelope packets. Limited to MTU, one \envelope packet contains at msot 96 nodes, thus the connection states of a multicast group with more than 96 members must span of multiple \envelope packets.

%forwarding format
%After the \envelope packet preparation, master node sends \envelope packet out, and letting swithes build their multicast forwarding data and multicast member confirm their paticipation. 
The multicast forwarding of \sys is based on the forwarding table that installed on switches. We enrich the native IP multicast forwarding table by adding layer-4 information, including connection and ACK/NACK states, for connectivity and reliability supports. The enriched multicast forwarding table of \sys is illustrated in Fig. \ref{fig:table}. This table is indexed with GroupIP, and every GroupIP associates with an array with $n$ ( \# of switch ports) entries. $entry_i$ reprensets the action taken in $port_i$. Each entry has three types: $connected$, $forward$, and $excluded$, where $excluded$ means that this port isn't included in multicast tree. $Connected$ means that this port is directly connected to a multicast node, and $forward$ means that the downstream device is a switch. Each entry contains layer-3 IP address, layer-4 connection states, and ACK/NACK statistics. $\S$\ref{connection-handle} and $\S$\ref{ack-aggregation} introduce that how to support optimal distribution, connectivity and reliability, based on entry states.

%forward envelope and determine forwarding port.
Upon receiving the \envelope packet, switch builds its local forwarding table and send one or more new \envelope pakcets to downstream devices. Algorithm \ref{alg:three} illustrates the behavior of \sys switch. A \envelope packet $p$ carries a GroupIP and an array of multicast member connection states ($p.array$). The switch generates a forwarding table indexed by $p.groupIP$, and initiate all entry types to $excluded$. Then the switch iterates over $p.array$ to fill forwarding table. For every node in array, the switch find the node routing information through the normal unicast forwarding table. If this node is directly connected to $port_i$, then mark the type of $entry_i$ in forwarding table to $connected$, and fill this node's connection states into $entry_i$. Otherwise, this node isn't directly connected, then the switch find the set of accessible ports ($set_p$). If one port in $set_p$ has been marked as $forward$, the switch select this port again to optimally distribute data. If not, the switch selects the least utilized port among $set_p$ to perform a group-level load balancing. 

After processing \envelope packet and filling forwarding table, the switch generate one or more new \envelope packet to ports with types as $forward$ or $connected$. The metadata of \envelope packet is unmodified while the contained nodes is changed. The \envelope packet that sent through each port only contains nodes that selects this port. We show the actions that taken by $S_1$ through Fig. \ref{fig:envelope-transmission}, and the whole topology is shown in Fig. \ref{fig:overview}. The \envelope packet that $S_1$ receives contains $R_1$, $R_2$, and $R_3$. As instructed by Algorithm \ref{alg:three}, $S_1$ should forward an copy to $port_{L_2}$ (eventually accessing $R_1$), and an copy to $port_{C_2}$ (eventually accessing $R_2$ and $R_3$). Thus the \envelope packets forwarded to $port_{L_2}$ and $port_{C_2}$ contains information of $\{R_1\}$ and $\{R_2, R_3\}$, respectively, to let the downstream switch build its local forwarding table.
   
Finally, if one node receives an \envelope packet, and its IP address is included in the packet, this node answer ACK back to the master node to confirm its participation. After the master node collects all nodes' confirmation ACK, the multicast transmission can start. 


\begin{algorithm}[t]
\caption{Multicast Forwarding Table Registration}\label{alg:three}
\begin{algorithmic}[1]

%\Function{Receive}{$p$}\Comment{$p$: \envelope packet}
\State $p\gets $ received \envelope packet
\State $T\gets $ multicast forwarding table
\State $n\gets$ the number of switch ports
\State $R[n]\gets \{ 0 \}$ \Comment{for creating \envelope packets}
\State $key\gets p.groupIP$
\State create $T[key]$, and initiate type of all entry to $excluded$
%\State $array[n]\gets MFT[p.groupIP]$
\State \textcolor{pink}{// loop over nodes and build $T$}
\For{$node$ contained in $p$}%\Comment{loop over nodes and build $T$}
	\If{$node$ is directly connected to port $i$}
		\State $out \gets i$
		\State $T[key][out].type \gets connected$
		\State $T[key][out].vaue \gets $ node connection states
	\Else
		\State $S\gets $ the set of accessible ports
		\If{port $j\in S$ is marked as $forward$}
			\State $out \gets j$
		\Else
			\State $out \gets $ least utilized port in $S$
			\State $T[key][out].type \gets forward$
		\EndIf
	\EndIf
	\State update port utilization
	\State $R[out].append(node)$
\EndFor
\State \textcolor{pink}{// create \envelope packets and send out}
\For{port $i\gets 0,$ $n-1$}%\Comment{create \envelope packets}
	\State create \envelope packet $p$ based on $R[i]$
	\State send $p$ through port $i$
\EndFor
%\EndFunction

\end{algorithmic}
\end{algorithm}

% -------------------------------------------------------
\subsection{One-to-many Data Forwarding} \label{connection-handle}
% -------------------------------------------------------

For the data-plane one-to-many data forwarding, \sys firstly reserves the optimal multicast forwarding, adopted by the previous in-fabric multicast solutions~\cite{crowcroft1988multicast, diab2022orca, shahbaz2019elmo}. Thus the sender only needs to send one copy of data, and the fabric makes multiple copies and forwards them to multiple receivers via the optimal paths. Therefore, there is no bandwidth wastage, and the communication distance and latency are minimized. 

% Figure environment removed

Secondly, departing from the previous works that only support layer-3 routing, \sys integrates layer-4 states into the fabric and further achieves a virtual one-to-many connection. Because of this, the single connected QP in the sender can simultaneously communicate with multiple QPs on multiple receivers. Therefore, \sys can fully leverage the advanced features of connected transport service, \ie, the one-sided \rdwrite operation and extended message size, as discussed in $\S$\ref{rdma}. Besides the more efficient bandwidth utilization, \sys outperforms the application-layer multicast with shorter communication distance and lower forwarding latency, as there is no additional \rdwqe and \rdcqe processing, and the intermediate nodes are not involved in forwarding data.

\parab{Extended multicast forwarding table.} \sys elaborately extends the traditional multicast forwarding table structure by integrating layer-4 states. The extended table, illustrated in Fig.~\ref{fig:table}, is the foundation of \sys. The indexed key of the table is the \textit{multicast group IP address} (abbreviated as GroupIP). Many multicast groups can exist simultaneously, each with a unique GroupIP.

% Figure environment removed

The GroupIP indexes two types of states: the group-level states and the port-level states. The group-level states contain statistics of the multicast group, which are used for the many-to-one feedback aggregation. The port-level states are formatted into an array with at most $n$ ( \# of switch ports) entries, only containing ports included in the multicast tree. The entry with $port$ as $i$ represents states related to $port_i$. Each entry is assigned one of two types: $connected$ and $forwarded$, where $connected$ means that this port is directly connected to a receiver node, and $forwarded$ means that the next hop is a switch. The $connected$ entry contains the connected receiver's layer-3 and layer-4 states, as well as the ACK/NACK states, and the $forwarded$ entry only contains ACK/NACK states. $\S$\ref{ack-aggregation} introduces the usage of these ACK/NACK states. 

The specific memory space used by one multicast table depends on the number of ports involved in the multicast group, which is at most $n$. We calculate that 1K multicast groups at most cost 0.92MB memory when each group contains the maximum entries. Moreover, the design goal of \sys is not to compress the switch-maintained states but to provide a general multicast protocol with prominent RDMA features. We can use many approaches to extend \sys to support more groups, and we'll talk about them in $\S$\ref{works}.

\parab{Establishing QPs.} Each multicast member follows the common unicast-like steps to establish the RC QP but assigns a virtual destination to it. Specifically, the destination IP is set as a unique GroupIP, and the destination QPN can be assigned as any non-conflicting value (\eg, $0x1$). Commodity RNICs provide the application with the programming interface to specify the destination IP and QPN without modifying the RNIC circuit~\cite{qpmodi}. After QPs establishment, multicast members exchange their QPs information and register the above-described forwarding table to switches, as described in Appendix~\ref{apx:regis}. Once the registration finishes, the sender can start sending multicast data packets.
%Once the master node receives all participants' affirm, the sender can start sending data packets.% and the fabric forwards them to multiple receivers based on the registered table.

%Multicast receivers follow unicast's identical connection management logic to process multicast traffic. In particular, upon receiving a data packet, the $dest\_IP$ is first checked if it matches the node's IP. Further, the RDMA transport attempts to locate the associated QP based on the $dest\_QPN$ in the packet header. Then RNIC processes packets based on the associated QP's context (QPC), such as generating ACK, notifying the upper-layer application, \etc Therefore, the connection-related states in the packet header, including $dest\_IP$, $dest\_QPN$, \etc, have to be modified from GroupIP and $0x0$ to match each receiver, as discussed in $\S$\ref{key-chalge}.

\parab{One-to-many data forwarding.} Switches involved in the multicast tree are responsible for forwarding data via the multicast tree and modifying packet headers to match different QPs. The switch follows Algorithm~\ref{alg:data-forward} to process data packets. Upon receiving a data packet ($p$), the switch uses the destination IP ($p.dest\_IP$) in the packet header to index the associated multicast forwarding table ($T$). Then switch iterates all entries (one entry corresponds to one port) in $T$ and actions as follows: (\romannumerber{1}) if the type is $forwarded$: creates a packet copy and forwards it through this port; (\romannumerber{2}) if the type is $connected$: creates a packet copy, modifies its connection-related states, and forwards it through this port.

We take one path, $S_1$ to $L_2$ to $R_1$, of the multicast tree in Fig.~\ref{fig:overview} as an example to show the header change, which is illustrated in Fig.~\ref{fig:header-change}. Firstly, the destination IP and QPN are modified to match $R_1$'s QP identification, as described in $\S$\ref{key-chalge}. Besides, \sys changes the source IP from the sender's IP to GroupIP. As a result, when $R_1$ generates feedback, the feedback's destination IP will be the data packet's source IP, \ie, GroupIP. Thus feedback packets can also index the associated forwarding table by their destination IP. Besides, \sys switches replace the destination MAC address to avoid the receiver's MAC layer discarding the packet. 

%The switch identifies multicast data packets through the specific header region, shown in Fig. \ref{fig:packet-format} and follows Algorithm \ref{alg:data-forward} to process data packets. 
%if the type is $connected$: creates a packet copy, modifies its connection-related states in the header, and forwards the resulting packet through this port. 

\begin{algorithm}[t]
\caption{Forwarding Packets and Replacing Headers.}\label{alg:data-forward}
\begin{algorithmic}[1]
%\Function{Receive}{p}
\State $p\gets $ data packet
\State $j\gets$ port that $p$ enters
\State $T\gets $ multicast forwarding table indexed by $p.dest\_IP$
%\State $n\gets$ the number of switch ports
%\State \textcolor{pink}{// loop over $T$ and copy/forward $p$}
%\Comment{\textcolor{pink}{loop over $T$ and copy/forward $p$}}
\For{$Entry \in T$}\Comment{\textcolor{gray}{loop over $T$ and copy/forward $p$}}
	\If{$Entry.type = forwarded$ \& $Entry.port \neq j$}
		\State $\overline{p}\gets$ a copy of $p$
		\State send $\overline{p}$ out from $Entry.port$
	\EndIf
	\If{$Entry.type = connected$ \& $Entry.port \neq j$}
		\State $\overline{p}\gets$ a copy of $p$
		\State $\overline{p}.dest\_IP(QPN) =Entry.dest\_IP(QPN)$
		%\State $\overline{p}.dest\_QPN=T[i].dest\_QPN$
		\State $\overline{p}.(...)=Entry.(...)$
		\State send $\overline{p}$ out from $Entry.port$
	\EndIf
\EndFor
%\EndFunction
\end{algorithmic}
\end{algorithm}

\parab{Support for one-to-many \rdwrite.}
\sys maintains connectivity between the sender and multiple receivers, which is sufficient for \rdsend/\rdreceive. However, \rdwrite requires more support. \rdwrite allows a node to write a memory slot on a remote node. The to-be-written MR info (including the remote VA and \rkey) is indicated in the first packet of the \rdwrite request. The \rdwrite responder's RNIC will check the MR info and execute the request only when they are correct. Otherwise, the packets will be discarded. To enable one-to-many \rdwrite, \sys needs to modify the MR states in the \rdwrite request header for different receivers. 

Besides maintaining MR info for different receivers, the MR info needs to be updated for every \rdwrite request because the MR changes with different \rdwrite requests. We force the host application to invoke an extra \rdwrite message which contains the MR states of different receivers, before submitting the actual \rdwrite request. Then the leaf switch recognizes this special message, updates MR info to the table, and replaces the MR states for the subsequent real \rdwrite request. This per-request updating scheme introduces minimal extra bandwidth overhead as long as the extra \rdwrite message is small compared to the total volume of transmitted data. We evaluate the performance of one-to-many \rdwrite in $\S$\ref{eval:storage}. Moreover, we discuss a possible way to avoid this extra overhead in Appendix~\ref{apx:opwrite}. This alternative approach requires the RNIC's modification.

%% Figure environment removed

%The multicast forwarding of \sys is based on the forwarding table that installed on switches. We enrich the native IP multicast forwarding table by adding layer-4 information, including connection and ACK/NACK states, for connectivity and reliability supports. The enriched multicast forwarding table of \sys is illustrated in Fig. \ref{fig:table}. This table is indexed with GroupIP, and every GroupIP associates with an array with $n$ ( \# of switch ports) entries. $entry_i$ reprensets the action taken in $port_i$. Each entry has three types: $connected$, $forward$, and $excluded$, where $excluded$ means that this port isn't included in multicast tree. $Connected$ means that this port is directly connected to a multicast node, and $forward$ means that the downstream device is a switch. Each entry contains layer-3 IP address, layer-4 connection states, and ACK/NACK statistics. $\S$\ref{connection-handle} and $\S$\ref{ack-aggregation} introduce that how to support optimal distribution, connectivity and reliability, based on entry states.

%\sys simultaneously achieves the optimal data distribution and the efficient one-to-many connection. Thus \sys combines the advantages of in-fabric and application-layer multicast solutions togather to support high application performance. The optimal data distribution enables the sender to sends only one copy of data. Then the network copy and forward multiple copies of data to multiple receivers through the registered multicast tree path, which significantly reduce bandwidth waste. Through the efficient one-to-many connection, multicast member can leverage the advanced RDMA features in connected transport, such as the comprehensive primitives and extended message size. Besides, each node only maintains one connection, and medium nodes aren't required to forward data, thus eliminating the scalability issue in RNIC and avoiding additional CPU overhead for medium data processing.  

%Sender send packet with destination IP address out. Switch use specific header option to idendity multicast data packet and utilized the registed multicast forwarding tree to forward packets. The specific multicast data header is showed in Fig. \ref{fig:packet-header}. 

%This is because, in Ethernet protocol, switch uses ARP protocol to find MAC address. ARP protocol use the IP address to find the associated MAC address in ARP table. As the GroupIP is a virtual IP address, ARP cannot find validated MAC address. So we need to manually assign it. 

% -------------------------------------------------------
\subsection{Many-to-one Feedback Aggregation} \label{ack-aggregation}
% -------------------------------------------------------

\begin{algorithm}[t]
\caption{Many-to-one ACK/NACK Aggregation}\label{alg:ack-aggre}
\begin{algorithmic}[1]
%\Procedure{ProcessIncomingACK}{ }
\State $p\gets $ ACK/NACK packet
\State $T\gets $ forwarding table indexed by $p.dest\_IP$
%\State $i\gets$ port that $p$ enters
\State $Entry\gets$ entry in $T$ with $port$ $i$ that $p$ enters
\State $min\_port\gets$ port with minimum $ack\_psn$ last time
\State $last\_ack\_psn\gets$ last aggregated ACK's PSN
\If{$p$ is $ACK$}
\If{$p.psn \geq Entry.ack\_psn$}\Comment{\textcolor{gray}{update $Entry$}}
		\State $Entry.ack\_psn = p.psn$
		\State $Entry.(...) = p.(...)$
	\EndIf
%	\State \textcolor{gray}{// check for generation triggering condition}
	\If{$i = min\_port$ \& $p.psn \geq last\_ack\_psn$}\label{line:trigger}
		\State \Call{GenerateNewAck/Nack}{ }
	\EndIf
\Else \Comment{\textcolor{gray}{$p$ is a NACK packet}}
	\If{$p.psn - 1 \geq Entry.ack\_psn$}
		\State $Entry.ack\_psn = p.psn - 1$\Comment{\textcolor{gray}{update $Entry$}}
	\EndIf
%	\State \textcolor{gray}{// check for generation triggering condition}
	\If{$p.psn \leq T.nack.ePSN$} \label{line:nackupdate}\Comment{\textcolor{gray}{update $T.nack$}}
		\State $T.nack.ePSN = p.psn$
		\State $T.nack.(...) = p.(...)$
		\State \Call{GenerateNewAck/Nack}{ }
	\EndIf
\EndIf
%\EndProcedure
\end{algorithmic}
\end{algorithm}

With the extended one-to-many data forwarding, \sys can optimally forward data replicas to multiple receivers, and different receivers' QPs can accept the packet. However, the second incompatibility impeding the leverage of RC is the existing reliability logic~\cite{zhu2015congestion, guo2016rdma}. The current reliability logic in commodity RNICs is designed to interpret a single feedback stream from a single receiver; thus, multiple feedback streams from multiple receivers can compromise reliability. 

Feedback contains various types of packets, such as ACK, NACK, notification packets for congestion control (CC) (such as CNP~\cite{zhu2015congestion}), \etc We talk about ACK and NACK here, and the processing for CC-related feedback is described in $\S$\ref{other-design}. \sys performs the in-fabric many-to-one ACK aggregation and NACK filtering to deliver a unicast-like ACK/NACK stream to the sender. As a result, the sender can correctly interpret ACK and proceeds with data transmission. Besides, when the loss occurs, the sender can precisely detect and retransmit the lost packet.

The basic principles of ACK-aggregation/NACK-filtering are that (\romannumerber{1}) the multicast source should receive an ACK only when \textit{\textbf{all}} receivers have received the corresponding packets; (\romannumerber{2}) the multicast source should receive a NACK when \textit{\textbf{any}} receiver loses a packet. Moreover, the aggregation needs to consider the processing rules of RDMA protocol, such as the go-back-N retransmission and ACK coalescing. 

\begin{algorithm}[t]
\caption{ACK/NACK Generation}\label{alg:generation}
\begin{algorithmic}[1]
\Function{GenerateNewAck/Nack}{ }
\State $T\gets $ forwarding table indexed by $p.dest\_IP$
%\State $n\gets$ the number of switch ports
\State $ack\_out\_port\gets$ port that data packets enter
\State $min\_psn\gets \infty$
\State $min\_port\gets -1$
\For{$Entry \in T$} \Comment{\textcolor{gray}{find the  minimized $ack\_psn$}}
	% \If{$Entry.type = forwarded$ or $connected$}
		\If{$min\_psn < Entry.ack\_psn$}
			\State $min\_psn = Entry.ack\_psn$
			\State $min\_port = i$
		\EndIf
	% \EndIf
\EndFor
\State \textcolor{gray}{// generate aggregated ACK}
\State create ACK packet $p$ with ($psn = min\_psn$)
\State send $p$ through $ack\_out\_port$ %\Comment{\textcolor{gray}{Generate ACK}}
\State \textcolor{gray}{// check for generating NACK}
\If{$min\_psn\ + 1 = T.nack.ePSN$} \label{line:nack}
	\State create NACK packet $p$ %\Comment{\textcolor{gray}{Generate NACK}}
	\State $p.psn$ = $T.nack.ePSN$
	\State send $p$ through $ack\_out\_port$
\EndIf
\State update global $last\_ack\_psn$ with $min\_psn$
\EndFunction
\end{algorithmic}
\end{algorithm}

\sys maintains ACK/NACK-related information in the extended multicast forwarding table, illustrated in Fig.~\ref{fig:table}. Upon receiving an ACK/NACK packet, switches find the associated forwarding table by ACK/NACK packets' destination IP, \ie, GroupIP. We first present a working logic without packet loss and then describe how to handle NACK packets. 

\parab{Handle ACK.} The ACK-related states includes (\romannumerber{1})~the group-level data, including the PSN of the last aggregated ACK ($last\_ack\_psn$), the port from which the ACK should be sent ($ack\_out\_port$), \etc; (\romannumerber{2})~the port-level data, including the largest acked PSN ($ack\_psn$) of each port. Switches process ACK packets, update the related states in the forwarding table, and generate the aggregated ACK packet, following Algorithm~\ref{alg:ack-aggre}. 

Upon receiving an ACK packet, the switch firstly updates this port's $ack\_psn$ if the PSN of incoming ACK is larger than the old $ack\_psn$. Then, if the trigger condition (Line~\ref{line:trigger}) is satisfied, the \textsc{GenerateNewAck/Nack} function in Algorithm~\ref{alg:generation} is called to generate an aggregated ACK. The aggregated ACK contains the minimum $ack\_psn$ recorded by the switch, which is found by iterating all multicast forwarding table entries. As a result, each aggregated ACK forwarded by the switch confirms that all downstream receivers have received the corresponding data packets. 

The port that owns the minimum $ack\_psn$ is recorded as $min\_port$. Each time an ACK with a larger PSN is received from $min\_port$ (Line~\ref{line:trigger}), the ACK generation is triggered. Thus not every ACK packet will trigger the generation, and the number of ACKs received by the source is reduced.

%The switch maintains a triggering condition to generate the aggregated ACK, thus not every ACK packet will trigger the generation.
%The insight behind this triggering condition is that \sys aims to guarantee that the multicast source won't receive many duplicate ACK packets. 
%The switch records the port, called $min\_port$, that contains the minimum $ack\_psn$ of all ports in the last ACK generation procedure. Afterwards, as long as receiving a new ACK packet from $min\_port$ and its $ack\_psn$ got updated, the switch will generate ACK again. In ACK generation, switch firstly iterates all entries of multicast forwarding table, find the minimum $ack\_psn$ and the associated port. If the minimum $ack\_psn$ is larger than the PSN of last aggregated ACK ($last\_ack\_psn$), switch will generate a new ACK packet contains the new minimum $ack\_psn$, and update the group-level $min\_port$ and $last\_ack\_psn$.

%Following the above algorithm, every time the multicast sender receives an aggregated ACK containing a PSN, the sender can recoginize that all receivers have received all packets before this PSN. As a result, the sender can interpret the aggregated ACK packets similar with unicast ACK stream.


%In particular, the $group\_ack\_psn$ is the lowest $ack\_psn$ of all ports. The $input\_port$ is the port, through which aggregated ACK be forward. The $group\_nack\_psn$ is lowest $nack\_psn$ of all ports. And each $ack\_psn$ records the highest acked PSN for each receiver. Switch follows Algorithm \ref{alg:ack-aggre} to process ACK packet, update ACK/NACK related states in forwarding table and generate aggregated ACK packet and NACK packet. 

\parab{Handle NACK.} When the loss occurs, the receiver generates NACK packet to notify the multicast sender. The NACK packet ($p$) contains the receiver's expected PSN ($p.psn$). Each NACK will acknowledge all data packets with PSN smaller than the expected PSN. This rule must be carefully handled in NACK generation. %towards compatibility with RNIC's retransmission logic.

We illustrate an example in Fig.~\ref{fig:nack-delay}. There are two receivers and one switch. The switch generates two copies of data and forwards them to two receivers. There are two lost packet, $p1_{R1}$ and $p2_{R2}$, and two associated NACK packets, $nack_{p1}$ and $nack_{p2}$. The switch should forward the $nack_{p1}$ because it contains the minimum expected PSN. If the $nack_{p2}$ is forwarded to the sender first, the loss of $p1_{R1}$ will be covered because the sender will assume that all the packets before the expected PSN of $nack_{p2}$ (\ie, $p2$) have been received. Thus the sender won't retransmit $p1$ anymore, and the reliability is compromised.

% Figure environment removed

Therefore, the NACK packet should be forwarded only when all receivers have acknowledged all packets with PSN smaller than its expected PSN. We implement this judgement in Line~\ref{line:nack} of Algorithm~\ref{alg:generation}. If the condition is not satisfied, the switch keeps waiting, during which new ACK/NACK packets keep coming. If the switch receives a new NACK and its PSN is not great than the recorded $T.nack.ePSN$, the $T.nack.ePSN$ is updated, and the NACK generation condition is rechecked, as shown in Line~\ref{line:nackupdate} of Algorithm~\ref{alg:ack-aggre}.



\crefname{equation}{}{}
\Crefname{equation}{}{}
\crefname{enumi}{}{}
\Crefname{enumi}{}{}

\renewcommand{\leq}{\leqslant}
\renewcommand{\geq}{\geqslant}

% \numberwithin{equation}{section}
