%-------------------------------------------------------------------------------
\section{Related Works} \label{works}
%-------------------------------------------------------------------------------
\parab{Internet and Datacenter multicast.} Multicast has been widely applied in large-scale Internet applications, such as Internet broadcast \cite{iptv}, video conferencing \cite{chen2011celerity}, and multiplayer games \cite{cho2009enabling}, \etc Prior works for the Internet \cite{chiang2018online, huang2016multicast,diab2020oktopus,ren2018optimal, diab2022yeti} mostly focus on the multicast routing, \ie, to find promising multicast paths, inside ISPs. For instance, Yeti~\cite{diab2022yeti} supports multicast routing with traffic engineering and service chaining requirements for large-scale ISPs. Yeti creates labels representing forwarding information for multicast graphs and processes these labels to forward packets to targeted paths. Although there are a bunch of prior works on the Internet, most of them merely provide best-effort delivery, which only works for applications without reliability requirement. 

There are some works~\cite{widmer2001extending, rizzo2000pgmcc} aim to provide reliability for datacenter applications upon approaches with best-effort delivery. However, existing reliable multicast solutions mainly adopt a TCP-like software stack and cannot meet the demand for high-speed communication in datacenters. In contrast, \sys  leverages the advanced RDMA stacks to process multicast traffic, providing high-speed reliable communication.

%And there are few works for in-fabric multicast in datacenters with technical details~\cite{sharp}. \todo{software datacenter multicast.} 
%MTRSA \cite{huang2016multicast} is a multi-tree routing algorithm attempting to optimize multicast routing paths inside ISP with consideration of node and link capacity. \cite{ren2018optimal} aims to find the optimal path to forward traffic while preserving the ordered access of a sequence of network services. These services are usually deployed as virtual functions on core routers in the ISP network. 

\parab{Multicast scalability.}
Datacenter applications impose a demand for high scalability. As the traditional IP multicast~\cite{crowcroft1988multicast}, along with its native group management, IGMP and tree construction protocol, PIM~\cite{estrin1998protocol}, are poor in scalability, many works~\cite{shahbaz2019elmo, diab2022orca, li2013scaling} attempt to address the scalability issue, \ie, supporting as much as possible multicast groups. For example, Elmo~\cite{shahbaz2019elmo} encodes the routing link of a multicast tree into rules formatted as packet header. Thus Elmo switch only needs to maintain rule parsing logic, reducing the total switch-maintained states. Orca~\cite{diab2022orca} utilizes the large memory space of the server, making servers assist in forwarding packets, reducing the switch's burden on maintaining states. 

These works that address the scalability issue are orthogonal with the \sys design. Our goal in this work is to provide a general multicast protocol with prominent RDMA features and reliability guarantee rather than compressing the switch-maintained states. As mentioned before, \sys can support at least 1K multicast groups using 0.92MB space, which is acceptable for a majority of multicast applications in datacenters. \sys can support even more multicast groups when getting extended further upon these works.

%\parab{Group communication.}
%The group communication in the datacenter is not limited to one-to-many and many-to-many. There are various patterns \cite{wan2020rat, rashidi2021enabling}, such as all-gather, reduce-scatter, and all-reduce. \sys can be extended to support these group communication patterns, and we leave this as our future work.
