%%%%%%%%%%%%%%%%%%%%%%%%%%%%%%%%%%%%%%%%%%%%%%%%%%%%%%%%%%%%%%
% Template for USENIX papers.
%%%%%%%%%%%%%%%%%%%%%%%%%%%%%%%%%%%%%%%%%%%%%%%%%%%%%%%%%%%%%%%

\documentclass[letterpaper,twocolumn,10pt]{article}
\usepackage{usenix2019_v3}
\usepackage{tikz}
\usepackage{amsmath}
\usepackage{amssymb}
\usepackage{xspace}
\usepackage{breakurl}
\usepackage{url}
\usepackage{multirow}
\usepackage{pifont}% http://ctan.org/pkg/pifont
\usepackage{caption}
\usepackage{subfigure}
\usepackage{algorithm}
\usepackage[noend]{algpseudocode}
\usepackage{xcolor}
\usepackage{booktabs}
\usepackage{hyperref}
\hypersetup{
    colorlinks=true,linkcolor=black,citecolor=red
}

\newcommand{\cmark}{\ding{51}}%
\newcommand{\xmark}{\ding{55}}%
\newcommand{\paraspace}{\vspace{0.04in}}
\newcommand{\parab}[1]{\paraspace\noindent{\bf #1} }
\newcommand{\todo}[1]{\textcolor{red}{(todo)#1}}
\newcommand{\add}{\textcolor{blue}}
\newcommand{\modify}{\textcolor{blue}}
\newcommand{\red}{\textcolor{red}}
\newcommand{\blue}{\textcolor{blue}}
\newcommand{\del}{\textcolor{gray}}
\newcommand{\etal}{\textit{et al}.\xspace}
\newcommand{\etc}{\textit{etc}.\xspace}
\newcommand{\ie}{\textit{i}.\textit{e}.\xspace}
\newcommand{\eg}{\textit{e}.\textit{g}.\xspace}
\newcommand{\aka}{\textit{a}.\textit{k}.\textit{a}.\xspace}

\newcommand{\sys}{Gleam\xspace}
\newcommand{\rdverb}{\textit{verb}\xspace}
\newcommand{\rdverbs}{\textit{verbs}\xspace}
\newcommand{\rdread}{READ\xspace}
\newcommand{\rdwrite}{WRITE\xspace}
\newcommand{\rdatomic}{\textsf{atomic}\xspace}
\newcommand{\rdsend}{SEND\xspace}
\newcommand{\rdreceive}{RECEIVE\xspace}
\newcommand{\rdwqe}{WQE\xspace}
\newcommand{\lkey}{\textit{L\_key}\xspace}
\newcommand{\rkey}{\textit{R\_key}\xspace}
\newcommand{\rdcqe}{CQE\xspace}
\newcommand*\circled[1]{\tikz[baseline=(char.base)]{
            \node[shape=circle,draw,inner sep=0.1pt] (char) {#1};}}
\newcommand{\romannumerber}[1] {\romannumeral #1}
\newcommand{\envelope}{\textit{envelope}\xspace}
\newcommand{\mpibcast}{$MPI\_Bcast$\xspace}

%--------------------------------------------------------------
\begin{document}
%--------------------------------------------------------------

%don't want date printed
\date{}

% make title bold and 14 pt font (Latex default is non-bold, 16 pt)
\title{Gleam: An RDMA-accelerated Multicast Protocol for Datacenter Networks}

%for single author (just remove % characters)
%\author{{\textit{Paper ID: \# 201}}} % end author

\author{
\rm Wenxue Li$^{1}$\thanks{Co-first authors.}~~~
\rm Junyi Zhang$^{2*}$~~~
\rm Gaoxiong Zeng$^{2}$~~~
\rm Yufei Liu$^{2}$~~~
\rm Zilong Wang$^{1}$~~~\\
\rm Chaoliang Zeng$^{1}$~~~
\rm Pengpeng Zhou$^{2}$~~~
\rm Qiaoling Wang$^{2}$~~~
\rm Kai Chen$^{1}$~~~\\
\\
$^1$Hong Kong University of Science and Technology~~~~
}

%\author{
%\rm Wenxue Li$^{1}$\thanks{Co-first authors.}~~~
%\rm Junyi Zhang$^{2*}$~~~
%\rm Gaoxiong Zeng$^{2}$~~~
%\rm Yufei Liu$^{2}$~~~
%\rm Zilong Wang$^{1}$~~~\\
%\rm Chaoliang Zeng$^{1}$~~~
%\rm Pengpeng Zhou$^{2}$~~~
%\rm Qiaoling Wang$^{2}$~~~
%\rm Kai Chen$^{1}$~~~\\
%\\
%$^1$Hong Kong University of Science and Technology~~~~
%$^2$Huawei~~~~
%}

\maketitle
%--------------------------------------------------------------
\begin{abstract}
%--------------------------------------------------------------
RDMA has been widely adopted for high-speed datacenter networks. However, native RDMA merely supports one-to-one reliable connection, which mismatches various applications with group communication patterns (\eg, one-to-many). While there are some multicast enhancements to address it, they all fail to simultaneously achieve optimal multicast forwarding and fully unleash the distinguished RDMA capabilities. 

In this paper, we present Gleam, an RDMA-accelerated multicast protocol that simultaneously supports optimal multicast forwarding, efficient utilization of the prominent RDMA capabilities, and compatibility with the commodity RNICs. At its core, Gleam re-purposes the existing RDMA RC logic with careful switch coordination as an efficient multicast transport. Gleam performs the one-to-many connection maintenance and many-to-one feedback aggregation, based on an extended multicast forwarding table structure, to achieve integration between standard RC logic and in-fabric multicast. We implement a fully functional Gleam prototype. With extensive testbed experiments and simulations, we demonstrate Gleam's significant improvement in accelerating multicast communication of realistic applications. For instance, Gleam achieves 2.9$\times$ lower communication time of an HPC benchmark application and 2.7$\times$ higher data replication throughput.
%with a programmable switch platform and commodity RNICs
\end{abstract}

\section{Introduction}
Current quantum hardware is unable to carry out universal quantum computations due to the buildup of errors that occur during the computation. 
The magnitude of the individual error is currently above the value that the Threshold Theorem requires in order to kick-start quantum error correction and fault-tolerant quantum computation~\cite[Section 10.6]{nielsen_chuang_2010}. 
Although the experimentally achieved fidelity rates are promising and the error bounds are inching closer to the required threshold, we will have to work for the foreseeable future with quantum hardware with errors that build-up during the computation.  This implies that we can only do a limited number of steps before the output of the computation has become completely uncorrelated with the intended one.

For fault-tolerant quantum computing, we repeat four steps: 
1) We apply a number of single and two-qubit quantum gates, in parallel whenever possible; 
2) We perform a syndrome measurement on a subset of the qubits; 
3) We perform fast classical computations to determine which errors have occurred and how to correct them; 
and, 4) We apply correction terms based on the classical computations.
We then repeat these four steps with a next sequence of gates. 
These four steps are essential to fault-tolerant quantum computing. 


The starting point of this work is to use the four steps outlined above, not to carry out error correction and fault-tolerant computation, but to enhance short, constant-depth, {\em uncorrected} quantum circuits that perform single qubit gates and {\em nearest-neighbor} two qubit gates. 
Since in the long run we will have to implement error-correction and fault-tolerant computation anyhow, and this is done by such a four-step process, why not make other use of this architecture? Moreover, on some of the quantum hardware platforms, these operations are already in place.
Embracing this idea we naturally arrive at the question: what is the computational power of \textit{low-depth} quantum-classical circuits organized as in the four steps outlined above? 
We thus investigate circuits that execute a small, ideally constant, number of stages, where at each stage we may apply, in parallel, single qubit gates and {\em nearest-neighbor} two qubit gates, followed by measurements, followed by low-depth classical computations of which the outcome can control quantum gates in later stages. 
It is not clear, at first, whether such circuits, especially with constant depth, can do anything remotely useful. 
But we will see that this is indeed the case: many quantum computations can be done by such circuits in constant depth. 
By parallelizing quantum computations in this way, we improve the overall computational capabilities of these circuits, as we do not incur errors on qubits that are idle, simply because qubits are not idle for a very long time. 
Furthermore, reducing the depth of quantum circuits, at the cost of increasing width, allows the circuit to be run faster even if errors occur.

The first usage of such a four-step layout, not to do error correction, but to perform computations, can be found in the paradigm of measurement-based quantum computing~\cite{gottesman1999demonstrating,raussendorf2001one,jozsa2006introduction,clark2007generalised}: 
A universal form of quantum computing where a quantum state is prepared and operations are performed by measuring qubits in different bases, depending on previous measurements and intermediate measurements.

\citeauthor{PhamSvore2013} were the first to formalize the four-step protocol for performing computations~\cite{PhamSvore2013}. They included specific hardware topologies by considering two-dimensional graphs for imposing constraints on qubit interactions. In their model, they develop circuits for particularly useful multi-qubit gates, including specifying costs in the width, number of qubits, depth, number of concurrent time steps, size, and total number of non-Identity operations.
As a result, they find an algorithm that factors integers in polylogarithmic depth.
\citeauthor{Browne:2011} showed that the main tool in the work by \citeauthor{PhamSvore2013}, the fan-out gate, can also be replaced by additional log-depth classical computations in the measurement-based quantum computing setting~\cite{Browne:2011}.

More recently, \citeauthor{Cirac:2021} introduced a scheme to implement unitary operations involving quantum circuits combined with Local Operations and Classical Communication ($\mathsf{LOCC}$) channels: $\mathsf{LOCC}$-assisted quantum circuits~\cite{Cirac:2021}. Similarly to the four-step scheme we just described, they allow for a short depth circuit to be run on the qubits, followed by one round of $\mathsf{LOCC}$, in which ancilla qubits are measured and local unitaries are applied based on the measurement outcomes. They show that in this model any 1D transitionally invariant matrix-product state (MPS) with fixed bond dimension is in the same phase of matter as the trivial state. Similar ideas can be found in~\cite{TVV_NonAbelianTopologicalOrder_2022, tantivasadakarn2021long}.

In this work, we introduce a new model, called \textit{Local Alternating Quantum-Classical Computations} ($\LAQCC$). In this model we alternate between running quantum circuits (constrained by locality), ending in the measurement of a subset of qubits, and fast classical computations based on the measurement results. The outcome of the classical computations are then used to control future quantum circuits. We allow for flexibility in this model, by giving different constraints to the power of both the quantum circuits and the classical circuits as well as the number of alternations between them. 
Most attention will be given to $\LAQCC$ containing quantum circuits of constant depth, classical circuits of logarithmic depth and at most a constant number of alternations between them. 
Any circuit constructed in this model is considered to be of constant depth. 
We restrict ourselves to logarithmic depth classical computations, as this is the first natural and non-trivial extension beyond constant-depth classical computations. 
Constant-depth classical computations do however also have an equivalent constant-depth quantum implementation.

The definition of $\LAQCC$ sharpens the original definition of \citeauthor{PhamSvore2013} by adding constraints to the intermediate classical computations. This allows us to bound the power of $\LAQCC$ from above. 

The main result of \citeauthor{Cirac:2021}, that 1D translational invariant MPS with fixed bond dimension can be prepared by $\mathsf{LOCC}$-assisted circuits, relies on local symmetries of the MPS. These symmetries allow them to prepare local states (on a constant number of qubits) and glue them together by doing one round of the appropriate entangling measurement and corrections, after which they run a round of local unitaries to get the desired result. This general scheme for preparing states that exhibit an MPS description with the appropriate local symmetries requires only geometrically local unitaries and one round of measurement and corrections an therefore is accessible in $\LAQCC$. Studying different local symmetries, known as Symmetry Protected Topological (SPT) phases of matter, to find measurement-based constant depth circuits for states is a broad ongoing field of research~\cite{TVV_NonAbelianTopologicalOrder_2022, tantivasadakarn2021long, smith2023deterministic}. 
All these schemes have a $\LAQCC$ implementation.

%$\LAQCC$-circuits also exist for general schemes of preparing local states, based on the local tensors, and gluing them together using one round of entangled measurement and corrections, based on the local symmetry. 
%The main result of \citeauthor{Cirac:2021}, that 1D translational invariant MPS with fixed bond dimension can be prepared by $\mathsf{LOCC}$-assisted circuits, relies heavily on local symmetries of the MPS and as a result also has an equivalent $\LAQCC$ implementation. 
%The corrections applied after the measurement round are local unitaries depending on the local symmetries of the MPS. 

 

%This general scheme of preparing local states, based on the local tensors, and gluing it together by doing one round of entangled measurement and corrections, based on the local symmetry, is accessible in $\LAQCC$.
Note however that \citeauthor{Cirac:2021} also suggest a circuit for the $W$-state.
This circuit uses sequentially and dependent measurement-based corrections of the ancilla qubits. 
These dependent measurements translate to sequential alternations between the quantum and classical circuits and therefore increase the total depth to linear depth, exceeding the constant-depth constraints imposed by $\LAQCC$-circuits. 

We study the power of the $\LAQCC$ model with respect to state preparation, showing that even with only constant quantum-depth and logarithmic classical depth it remains possible to prepare states with long-range entanglement.
Another surprising result is that it is unlikely that $\LAQCC$ circuits are classically simulatable. We show that any instantaneous quantum polynomial-time (IQP) circuit~\cite{Bremner2010,Shepherd2009} has an $\LAQCC$ implementation.
Classical simulation of IQP circuits implies the collapse of the polynomial hierarchy to the third level, which is not believed to be true~\cite{Bremner2017}. Therefore, we expect that $\LAQCC$ circuits are unlikely to be classically simulatable. We bound the power of $\LAQCC$ by showing that it is contained in $\QNC^1$, the class of polynomial-size, log-depth circuits.

Next, we also study the power that intermediate classical calculations can add to quantum computations, by considering a new model that alternates between polynomially many polynomial-depth quantum circuits and unbounded classical computations
We study this model by doing a complexity theoretical analysis, where we draw inspiration from the notions of complexity given by \citeauthor{RosenthalYuen:2022}, \citeauthor{MetgerYuen:2023}, and \citeauthor{Aaronson:2004}.
All three complexity notions are based on the notion of state preparation, instead of more traditional definition of complexity such as the decidability of a computational problem. 
The first two consider classes based on sequences of quantum states preparable by a polynomial-sized quantum circuit, where the circuits are uniformly generated by a computational class, for instance, the class $\mathsf{PSPACE}$, which results in the complexity class $\mathsf{StatePSPACE}$~\cite{RosenthalYuen:2022,MetgerYuen:2023}.
The third notion considers a relative complexity, where the complexity is measured between two given states, and is measured by the number of gates, from a given gate-set, required to transform one state in another state~\cite{Aaronson:2004}. 
For our definition of state preparation complexity, we drop the uniformity constraint from~\cite{RosenthalYuen:2022,MetgerYuen:2023} and define a class as $\mathsf{StateX}$, which refers to states preparable by circuits of type $\mathsf{X}$. 
As an example, if $\mathsf{X} = \QNC^0$, this results in the class $\mathsf{StateQNC^0}$, which is the set of states preparable from the $\ket{0}^n$ state by poly-size constant-depth circuits. 
This notion is similar to the relative complexity from~\cite{Aaronson:2004}, where one state is the  $\ket{0}^n$ state and instead of counting the number of gates we consider the set of states preparable by a fixed number of gates. Using this notion of complexity we show that any state preparable by an $\LAQCC^*$ circuit is also preparable by a $\mathsf{PostQPoly}$ circuit, the class of circuits of polynomial depth with an additional post-selection gate. 

All Clifford circuits have a constant-depth $\LAQCC$ implementation, implying that any stabilizer state can be implemented by a constant-depth $\LAQCC$ circuit, see Section~\ref{sec:clifford_circuits} for a proof of this statement. 
Efficient circuits for stabilizer states have been known already through measurement-based quantum computing. Therefore this paper focuses on the preparation of non-stabilizer states, and as a surprising result we find novel constant-depth protocols for four very natural classes of non-stabilizer states.
Despite the extensive research into these four classes of non-stabilizer states and the many applications of them, no efficient constant- or low-depth state preparation protocols are known yet. We specifically consider these four classes as they are all often used as initial states in other algorithms.

The first state is a uniform superposition over an arbitrary number of states. 
This state finds applications in many quantum algorithms, as they often start with a uniform superposition over multiple states. 
This superposition is often achieved by applying Hadamard gates to every qubit due to its simplicity to prepare. 
Yet, the analysis of many algorithms, such as Shor's algorithm~\cite{Shor:1997}, would benefit from a different initial superposition. 
The circuit to prepare the uniform superposition over an arbitrary number of states uses an exact version of Grover search as a subroutine, that turns a probabilistic circuit, with a known constant probability of success, into a deterministic circuit. 
We use the circuit for preparing a uniform superposition over an arbitrary number of states as a subroutine in the next two quantum state preparation protocols. 

The second state is the $W$-state, the uniform superposition over all computational basis states of Hamming-weight~$1$, a natural long-ranged entangled state that displays a fundamentally nonequivalent type of entanglement from the Greenberger–Horne–Zeilinger state~\cite{WState:2000}, for which $\LAQCC$-type constant-depth circuits were previously known~\cite{PhamSvore2013, Cirac:2021}. 
The $W$-state is often used as benchmark for new quantum hardware~\cite{Haffner2005,Neeley2010,GarciaPerez:2021}. 
A novel way to prepare the $W$-state therefore gives a new way to benchmark different quantum devices with each other. 
A circuit for preparing the $W$-state was given in~\cite{Cirac:2021}, but this implementation requires sequentially alternating measurements followed by local unitaries, which in the $\LAQCC$ model is not considered to be of constant depth. 
We improve this protocol by giving an $\LAQCC$ implementation of the $W$-state, based on a compress-uncompress method that links the one-hot and binary encoding of integers.

The third state considered is the Dicke state, a generalization of the $W$-state, a superposition over all computational basis states with Hamming-weight $k$~\cite{Dicke:1954}. 
Dicke states have relevance in various practical settings.
For instance, for quantum game theory~\cite{zdemir2007}, quantum storage~\cite{Bacon_Compress:2006,Plesch:2010}, quantum error correction~\cite{ouyang2014permutation}, quantum metrology~\cite{toth2012multipartite}, and quantum networking~\cite{prevedel2009experimental}. 
Dicke states have been used as a starting state for variational optimization algorithms, most notably Quantum Alternating Operator Ansatz (QAOA)~\cite{Hadfield2019}, to find solutions to problems such as Maximum k-vertex Cover~\cite{Brandhofer2022,cook2020quantum}.
The ground states of physical Hamiltonians describing one-dimensional chains tend to show a resemblance to Dicke states such as states resulting from the Bethe ansatz, making them an ideal starting state when investigating the ground state behavior of these Hamiltonians~\cite{TDL_BetheAnsatzDerivation:2010,B_ExcitedStateQuantumPhaseTransitions:2013,DickeTransitions:2021}. 
For instance, the algorithm by \citeauthor{van2021preparing}, who give an algorithm to prepare the Bethe ansatz eigenstates of the spin-1/2 XXZ spin chain, starts by first preparing a Dicke state~\cite{van2021preparing}. 
A Dicke-state preparation protocol based on the compress-uncompress methodology used in the $W$-state furthermore finds applications in entanglement distillation, where the entanglement of a large state is concentrated on only a few qubits. 
Efficient deterministic circuits for preparing Dicke states have been proposed by \citeauthor{bartschi2019deterministic}~\cite{bartschi2019deterministic, bartschi2022deterministic_short_depth}. 
They provide a quantum circuit of depth $\mathO(k \log(\frac{n}{k}))$, allowing arbitrary connectivity, to prepare a Dicke state, which they conjecture to be optimal when $k$ is constant. 
In this work, we provide a constant-depth $\LAQCC$ circuit below their conjectured bound already for constant $k$. 
However, this does not directly disprove their conjecture, as we allow for intermediate measurements and classical computations. 
More significantly, we even construct constant-depth $\LAQCC$ circuits for $k = \mathO(\sqrt{n})$ greatly improving their bound.
This construction extends the compress-uncompress method for the $W$-state combined with additional subroutines. 

We continue with a log-depth state preparation protocol for the Dicke-state for arbitrary $k$. 
This protocol implements an efficient transformation between the factoradic number representation and the combinatorial number representation of a positive integer. 
The combinatorial number representation relates directly to the Dicke state. 
The provided efficient transformation between number representation systems might be of independent interest. 

We conclude by modifying our protocol for preparing a Dicke-state to a protocol that prepares quantum many-body scar states in constant-depth. 
These states have low entanglement and longer coherence times than states with similar energy density.
These characteristics make many-body scar states interesting to analyze and relevant within physics.
Many-body scar states appear for instance in the AKLT model~\cite{AKLT:1987,MRBAR:2018,MRB:2018} and different spin models~\cite{SI:2019,MOBFR:2020}.
Known methods for preparing these states have polynomial-depth~\cite{Gustafson:2023}, whereas our circuit has constant depth. 

% We conclude by studying the power that intermediate classical calculations can add to quantum computations. 
% In this study, we define a new model that relaxes constant-depth quantum circuits to polynomial depth quantum circuits, log-depth classical calculations to unbounded classical computations and a constant number of alternations to a polynomial number of alternations. 
% We call this model $\LAQCC^*$. 
% We study this model by doing a complexity theoretical analysis, where we draw inspiration from the notions of complexity given by \citeauthor{RosenthalYuen:2022}, \citeauthor{MetgerYuen:2023}, and \citeauthor{Aaronson:2004}.
% All three complexity notions are based on the notion of state preparation, instead of more traditional definition of complexity such as the decidability of a computational problem. 
% The first two consider classes based on sequences of quantum states preparable by a polynomial-sized quantum circuit, where the circuits are uniformly generated by a computational class, for instance, the class $\mathsf{PSPACE}$, which results in the complexity class $\mathsf{StatePSPACE}$~\cite{RosenthalYuen:2022,MetgerYuen:2023}.
% The third notion considers a relative complexity, where the complexity is measured between two given states, and is measured by the number of gates, from a given gate-set, required to transform one state in another state~\cite{Aaronson:2004}. 
% For our definition of state preparation complexity, we drop the uniformity constraint from~\cite{RosenthalYuen:2022,MetgerYuen:2023} and define a class as $\mathsf{StateX}$, which refers to states preparable by circuits of type $\mathsf{X}$. 
% As an example, if $\mathsf{X} = \QNC^0$, this results in the class $\mathsf{StateQNC^0}$, which is the set of states preparable from the $\ket{0}^n$ state by poly-size constant-depth circuits. 
% This notion is similar to the relative complexity from~\cite{Aaronson:2004}, where one state is the  $\ket{0}^n$ state and instead of counting the number of gates we consider the set of states preparable by a fixed number of gates. Using this notion of complexity we show that any state preparable by an $\LAQCC^*$ circuit is also preparable by a $\mathsf{PostQPoly}$ circuit, the class of circuits of polynomial depth with an additional post-selection gate. 

\paragraph{Summary of results}
\begin{itemize}
    \item We give a new definition of a computational model that captures the power of the four step process: applying a constant number of layers of one- and two-qubit gates; performing a syndrome measurement; perform a fast classical computation determining corrections; apply corrections. We call this model \emph{Local Alternating Quantum Classical Computations}, or $\LAQCC$ for short. In this model we bound the allowed quantum operations, intermediate classical calculations, and number of rounds separately. In Section~\ref{sec:LAQCC_model} we define this model and give a list of operations based on results from literature contained in this computational model. In some of these operations we explicitly use that we allow for multiple, but at most constant, rounds  of corrections.
    \item  We show show that there exist $\LAQCC$ circuits that can not be weakly simulated in Section~\ref{sec:IQP_in_LAQCC}. We further show that for every $\LAQCC$ circuit there exists a $\QNC^1$ circuit simulating it perfectly, in Section~\ref{sec:LAQCC_in_QNC1}.
    \item We introduce a new type computational complexity for preparing states and show that the extension of $\LAQCC$ where we allow a polynomial number of rounds and unbounded classical computation, is contained in $\mathsf{PostQPoly}$, the class of polynomial circuits with post-selection, in Section~\ref{sec:Complexity results}.
    \item We show a protocol to prepare the uniform superposition state of size $q$ in $\LAQCC$ using $\mathO(\ceil{\log_2(q)}^2)$ qubits in Section~\ref{sec:superposition_modulo_q}. 
    \item We show a protocol to prepare the $W_n$ state in $\LAQCC$ using $\mathO(n\log(n))$ qubits in Section~\ref{sec:W_state_in_LAQCC}.
    \item We show two ways of preparing the Dicke-$(n,k)$ state. The first method is in $\LAQCC$, works up to $k = \mathO(\sqrt{n})$, uses $\mathO(n^2\log(n))$ qubits, and is found in Section~\ref{sec:dicke:small_k}. The second method is in $\LAQCC\text{-}\mathsf{LOG}$ (an extension of $\LAQCC$ allowing for logarithmic number of alterations instead of constant), works for any $k$, uses $\mathO(\text{poly}(n))$ qubits, and is found in Section~\ref{sec:Dicke_in_LAQCC_LOG}. 
    \item We extend on our $\LAQCC$ method of generating Dicke-$(n,k)$ states for $k = \mathO(\sqrt{n})$ and show a protocol to generate many-body scar states for a particular Hamiltonian in $\LAQCC$ (Section~\ref{sec:many_body_scar}). 
\end{itemize}
Summarized in a table, we provide the following state generation protocols:
\begin{table}[htb]
\centering
\begin{tabular}{l|l|l|l}
\textbf{State description} & \textbf{Width} & \textbf{Depth} & \textbf{Implementation}\\
\hline 
Uniform superposition mod $q$: $\frac{1}{\sqrt{q}} \sum_{i = 0}^{q-1}\ket{i}$ & $\mathO(\ceil{\log^2 q})$ & $\mathO(1)$ & Section~\ref{sec:superposition_modulo_q}\\

$W$-state: $\frac{1}{\sqrt{n}}\sum_{i = 0}^{n-1}\ket{e_i}$ & $\mathO(n \log n)$ & $\mathO(1)$ & Section~\ref{sec:W_state_in_LAQCC}\\

Dicke-$(n,k)$, $k = \mathO(\sqrt{n})$: $\binom{n}{k}^{-1/2}\sum_{x \in \{0,1\}^n: |x| = k} \ket{x}$ &  $\mathO(n^2\log n)$ & $\mathO(1)$ 
&Section~\ref{sec:dicke:small_k}\\

Dicke-$(n,k)$: $\binom{n}{k}^{-1/2}\sum_{x \in \{0,1\}^n: |x| = k} \ket{x}$ & $\mathO(\text{poly}(n))$ & $\mathO(\log n)$ &Section~\ref{sec:Dicke_in_LAQCC_LOG}\\

QMBS: $\ket{S_k} = \frac{1}{k! \sqrt{\mathcal N(n,k)}}(Q^\dagger)^k \ket{\Omega}$ &  $\mathO(n^2\log n)$ & $\mathO(1)$  &  Section~\ref{sec:many_body_scar}
\end{tabular}
\caption{Summary of state preparation protocols given in this paper.}
\label{tab:sate_prep}
\end{table}
In the entry for the quantum many-body scar state $Q$ denotes the raising operator and $\mathcal N(n,k)=\binom{n-k-1}{k}$. 
Section~\ref{sec:many_body_scar} will provide more details on the variables and the implementation. 

\paragraph{Organization of the paper}
\noindent We first introduce relevant preliminaries in Section~\ref{sec:preliminaries}. 
In Section~\ref{sec:LAQCC_model} we formally define the class of Local Alternating Quantum-Classical Computations ($\LAQCC$). We also show that any Clifford circuit can be implemented in constant depth $\LAQCC$ (a result based on a result from measurement-based quantum computing~\cite{jozsa2006introduction}). 
This result allows us to give many useful multi-qubit gates and routines in Section~\ref{sec:gates_created_in_LAQCC}. 
Beyond that we show that constant depth $\LAQCC$ circuits are contained in $\QNC^1$ and that any $\mathsf{IQP}$ circuit has an $\LAQCC$ implementation.
We conclude this section with an analysis of a more powerful instantiation of $\LAQCC$ and show an inclusion with respect to the class $\mathsf{PostQPoly}$, which is the class of circuits of polynomial depth with one additional post-selection gate. 
In Section~\ref{sec:state_prep_in_LAQCC} we give $\LAQCC$ circuit implementations for preparing the uniform superposition over an arbitrary number of states, the $W$-state and the Dicke state up to $k = \mathO(\sqrt{n})$. We furthermore give a log-depth circuit implementation for preparing the Dicke state for any $k$. We conclude by showing a $\LAQCC$ circuit for generating many body scar states of a particular type of Hamiltonian.


\vspacebeforesection
\section{Background}
\label{sec:background}

In this section, we provide the necessary background information to ensure a comprehensive understanding of the attack described in this paper. We start with a description of the Distributed Hash Table (DHT) used by IPFS, followed by its content resolution mechanisms. We also detail techniques for network size estimation, necessary for our attack detection and mitigation mechanisms.

\vspacebeforesection
\subsection{IPFS DHT}
\label{sec:kad_dht}

We review the features of the Kademlia DHT~\cite{maymounkov2002kademlia} and its \texttt{libp2p} implementation~\cite{libp2p_github} that are the most relevant to our attack.
To participate in the DHT, each peer generates a public/private key pair and derives an identity $\peerid \in \{0,1\}^{256}$ as the hash of its public key.
Ideally, each peer generates a random key pair and, therefore, peer IDs are distributed uniformly and independently over the space $\{0,1\}^{256}$.
While honest nodes follow this rule, malicious nodes may generate and choose from an arbitrary number of key pairs.
Each peer maintains a routing table consisting of $m=256$ buckets.
The $i$-th bucket contains the addresses of up to $k=20$ peers whose peer IDs share a common prefix of exactly $i$ bits with the peer's own peer ID. 

%
A new participant node joins the IPFS network by contacting one of the hardcoded bootstrap nodes. This bootstrap node provides the new node with some initial peers allowing it to join the DHT. The new node uses this information to perform a walk through the DHT towards its own peer ID.
The walk allows to: \textit{(i)}~make sure that there is no other node in the network with the same ID; \textit{(ii)}~discover new peers and fill the newcomer's DHT routing table. At the same time, the newcomer establishes \bitswap~\cite{de2021accelerating} connections to a subset of encountered peers (usually around 300 of them). The core role of the \bitswap protocol is to enable bilateral content transfer and to play the role of a cache for recently-accessed content.

The main DHT operation $\Call{GetClosestPeers}{\key}$ returns the $k=20$ closest peers to $\key$. 
%
In Kademlia, the distance between two keys $x$ and $y$ in the key space is given by $x \oplus y \in \{0,...,2^{256}-1\}$, where $\oplus$ denotes the bitwise XOR operation on the keys; the resulting binary string is interpreted as an integer.
%
When a client wants to find the peers with IDs closest to $\key$, it sends a request to the $\alpha=3$ peers in its routing table whose peer IDs are closest to $\key$. Each of these peers returns the $k$ closest peers to $\key$ in its own routing table and the addresses of these peers. 
%
The client again sends a request to the $\alpha$ peers closest to $\key$, among peers in its routing table and those whose addresses it just received. This process repeats until the client does not find any more peers closer to $\key$.
Due to network churn and imperfect routing tables, we observed in our experiments that successive calls to $\Call{GetClosestPeers}{\key}$ do not always return the same set of $k=20$ peers (we provide more details in \Cref{sec:evaluation}, \Cref{fig:20closest}). This is an important limitation affecting our attack.

\vspacebeforesection
\subsection{Content Resolution in IPFS}
\label{sec:ipfs}

IPFS is a content-centric network.
It allows its participant to request files without specifying their location. 
%
Content is indexed by content IDs $\cid \in \{0,1\}^{256}$ that are derived from a hash of that content.
Both peer IDs and CIDs are used as keys in the DHT.
Each node can play the role of a \provider, \downloader, or \resolver. 
The process of content advertisement and resolution is illustrated in \Cref{fig:add_get_provider}.

%
When a \provider wishes to publish content with a given $\cid$ on IPFS, it creates a \emph{provider record} that contains $cid$ and the \provider's address.
During a $\Call{Provide}{\cid}$ operation, the \provider first uses $\Call{GetClosestPeers}{\cid}$ to locate the $k=20$ peers with their peer IDs closest to $\cid$, 
%
and then sends them a $\mathsf{PutProvider}$ message including the provider record (\Cref{fig:add_get_provider}(a)).
We call the peers that hold provider records for $\cid$ the \emph{resolvers} for $\cid$.

Each CID can have several \providers. In fact, by default, each IPFS client becomes a provider for each piece of content it downloads for a fixed amount of time (12h, 24h, or 48h depending on the client version or custom configuration). As a result, the system provides an auto-scaling feature with supply automatically rising with demand.

%
When a \downloader wishes to fetch a piece of content, it first sends a request to all its \bitswap peers. If none of them has the content, the \downloader uses the DHT-based resolution system. We stress that the \bitswap protocol plays the supporting role of a cache in the dissemination of popular files. However, the mechanism does not provide reliable content resolution, in particular for new or less popular content. %

When \bitswap unstructured search fails, the \downloader resolves $\cid$ using $\Call{FindProviders}{\cid}$. This operation uses a DHT walk identical to that of $\Call{GetClosestPeers}{\cid}$ to find $k$ \resolvers but also queries encountered nodes for a provider record for $\cid$ (\Cref{fig:add_get_provider}(b)). The process terminates when either 20 \providers have been found, or all \resolvers have been asked. Querying all encountered nodes (\ie, not only the designated \resolvers) is useful because some of the encountered nodes may have a provider record in their cache.
%

Upon receiving a provider record, the client connects to the address specified in the provider record to retrieve the actual content (\Cref{fig:add_get_provider}(c)).
Provider records are not authenticated, and therefore malicious \providers may respond with incorrect provider records (or may not respond at all). However, the integrity of the content is preserved because the hash of the retrieved content can be verified against its $\cid$.
%


%

\input{img/add_get_provider.tex}

\vspacebeforesection
\subsection{Network Size Estimator}
\label{sec:netsize}

The number of nodes in a decentralized system is generally unknown due to the avoidance of centralized membership management.
This number is nonetheless useful for optimizations, deciding on individual node configurations, or security mechanisms.
Various methods were proposed for the decentralized estimation of unstructured and structured networks~\cite{eli-sohl-dht-size-estimation,kostoulas2005decentralized, manku2003symphony}.
We use in this work a mechanism developed initially by Protocol Labs as part of a mechanism for decreasing the latency of publishing content in IPFS~\cite{network-size-estimation-notion,network-size-estimation-github-pr}.

%
%
%
%
%
%
%
%
%
%

Each node in the DHT refreshes its routing table periodically (every $10$ minutes in \texttt{libp2p}). 
For this, the node samples $m$ random keys (one for each bucket of its routing table)
%
and queries the DHT to obtain the $k=20$ closest peer IDs to each key.
Using these, the node then computes the average distance between each one of these keys $\key_j$ for $j=1,\dots,m$ and their $i$-th closest peer ID for $i=1,...,k$ (with $m=256$ and $k=20$).
\begin{equation}
    \label{equ:avg-dist}
    \overline{D}_i = \frac{1}{m} \sum_{j=1}^m \operatorname{dist}(\key_j, \peerid_{j}^{(i)})
\end{equation}
where $\peerid_{j}^{(i)}$ is the $i$-th closest peer ID to $\key_j$.
With $N$ peers in the DHT and peer IDs uniformly distributed in the hash space, the expected distance between a $\key$ and its $i$-th closest peer ID is $\frac{2^{256}i}{N+1}$. The node then runs a least square regression to compute the value of $N$ for which the expected distances best fit the empirical average distances, \ie,
\begin{equation}
    \label{equ:netsize-least-squares}
    \hat{N} = \arg\min_{N} \sum_{i=1}^k \left(\overline{D}_i - \frac{2^{256}i}{N+1}\right)^2.
\end{equation}
The resulting estimate $\hat{N}$ can be computed in closed form.
%

When a node starts running, it must perform DHT queries for a few random keys to initialize its network size estimate. 
Since a larger number of queries will result in higher accuracy, making more queries than what is needed to initialize one's routing table is recommended.
Thereafter, keeping the estimate up-to-date does not require any excess DHT queries beyond what is already used for refreshing the routing table as this is done frequently (every 10 minutes).

While the network size estimate has a stochastic variance resulting from the probability distribution of the honest peer IDs, it is hard for an attacker to bias the estimate significantly. Since the estimator uses the density of peer IDs around keys chosen uniformly at random, the adversary would require numerous Sybil nodes (on the order of the whole network size) to significantly affect the peer ID density around those keys.

\section{Design}
The design philosophy of causal-learn is centered around building an open-source, modular, easily extensible and embeddable Python platform for learning causality from data and making use of causality for various purposes. Due to the different goals, assumptions, and techniques between causal learning and traditional learning tasks, newcomers to the field often find it hard to get a clear picture of the developments in modern causality research. Thus, we briefly introduce the algorithms and functionalities in causal-learn with a special focus on their use cases and suitable application scenarios.

\subsection{Search methods}
Causal-learn covers representative causal discovery methods across all major categories with official implementation of most algorithms. We briefly introduce the methods as follows. It is worth noting that we are actively updating the library to incorporate latest algorithms.
\begin{itemize}
    \item \textbf{Constraint-based causal discovery methods.} Current algorithms under that category are PC \citep{spirtes2000causation}, FCI \citep{spirtes1995causal}, and CD-NOD \citep{huang2020causal}. PC is a classical and widely-used algorithm with consistency guarantee under independent and identically distributed (i.i.d.) sampling assuming no latent confounders, the faithfulness assumption, and the causal Markov condition, which has been extensively applied in many fields. By continuously applying (conditional) independence tests on subsets of variables of increasing size in a smart way, its search procedure returns a Markov Equivalence Class (MEC), of which the graphical object consists of a mixture of directed and undirected edges, known as a Completed Partially Directed Acyclic Graph (CPDAG). PC is highly adaptable to various use cases, facilitated by the selection of an appropriate independence test; it can handle data with different assumptions, such as Fisher-Z test \citep{fisher1921014} for linear Gaussian data, Chi/G-squared test \citep{tsamardinos2006max} for discrete data, and Kernel-based Conditional Independence (KCI) test \citep{zhang2011kernel} for the nonparametric case. Moreover, causal-learn provides an extension, Missing-Value PC (MV-PC) \citep{tu2019causal}, to address issues of missing data. Furthermore, we have implemented FCI for causal structures that include hidden confounders (it indicates the possible existence of hidden confounders whenever the possibility cannot be excluded, but it cannot help determine possible relations among them), and causal discovery from nonstationary/heterogeneous data (CD-NOD). These constraint-based methods offer wide applicability as they can accommodate various types of data distributions and causal relations, provided that appropriate conditional independence testing methods are utilized. However, genenerally speaking, they may not be able to determine the complete causal graph uniquely and, accordingly, there usually exist some undirected edges in the returned CPDAGs.

    \item \textbf{Score-based causal discovery methods.} Different from the search style of constraint-bed methods, score-based methods find the causal structure by optimizing a properly defined score function. Greedy Equivalence Search (GES) \citep{chickering2002optimal} is a well-known two-phase procedure that directly searches over the space of equivalence classes. Similarly, exact search (e.g., A* \citep{yuan2013learning}, Dynamic Programming \citep{silander2006simple}), and permutation-based search (e.g., GRaSP \citep{lam2022greedy}) apply different search strategies to return a set of the sparsest Directed Acyclic Graphs (DAGs) that contains the true model under assumptions strictly weaker than faithfulness. These score-based methods are versatile, able to accommodate a wide array of data and causal relations by choosing suitable score functions, such as BIC \citep{schwarz1978estimating} for linear Guassian data, BDeu \citep{buntine1991theory} for discrete data, and Generalized Score \citep{huang2018generalized} for the nonparametric case. The choice of score function can be conveniently adjusted as a hyperparameter.

    \item \looseness=-1 \textbf{Causal discovery methods based on constrained functional causal models.} While constraint-based and score-based methods offer flexibility through the selection of an appropriate independence test or score function, they are limited to returning equivalence classes, yielding non-unique solutions where the causal direction between certain variable pairs remains indeterminate. In contrast, assuming specific Functional Causal Models (FCMs)--that is, functions in a particular functional class to specify how the effect is generated from its direct causes and noise--allows for the full determination of the causal structure, albeit at the cost of certain trade-offs. Causal-learn incorporates algorithms based on several FCM variants, capable of producing unique causal directions. Examples include the linear non-Gaussian acyclic model (LiNGAM) \citep{shimizu2006linear} and its variant, i.e., DirectLiNGAM \citep{shimizu2011directlingam}, which have been extensively applied for non-Gaussian noises with linear relations. VAR-LiNGAM \citep{hyvarinen2010estimation}, which combines LiNGAM with vector autoregressive models (VAR), to estimate both time-delayed and instantaneous causal relations from time series. RCD \citep{maeda2020rcd}, an extension of LiNGAM, allows for hidden confounders, while CAM-UV \citep{maeda2021causal} further extends this to the nonlinear additive noise case. In addition, the additive noise model (ANM) \citep{hoyer2008nonlinear} has been proven to be identifiable in the presence of nonlinearity and additive noises. Furthermore, we have also incorporated the post-nonlinear (PNL) causal model \citep{zhang2009identifiability}, a highly general form (with LiNGAM and ANM as special cases) that has been demonstrated to be identifiable in the generic case, barring five specific situations described in \citep{zhang2009identifiability}.

    \item \textbf{Causal representation learning: Finding causally related hidden variables.} Latent variables play an instrumental role in a multitude of real-world scenarios, often acting as hidden confounders that influence observed variables. Unfortunately, most existing methods may fail to produce convincing results in cases with latent variables (confounders). In causal-learn, we implement the Generalized Independent Noise (GIN) condition \citep{xie2020generalized} for estimating linear non-Gaussian latent variable causal model, which allows causal relationships between latent variables and multiple latent variables behind any two observed variables. This promises to improve the detection and understanding of the complex, often hidden, causal structures that govern real-world phenomena.

\end{itemize}

Besides, causal-learn also has Granger causality \citep{granger1969investigating, granger1980testing} implemented for statistical but not causal\footnote{As mentioned by Granger, Granger causality is not necessarily true causality. In fact, If one assumes 1) that there is no latent confounding process, 2) that the data are sampled at the right causal frequency, and 3) that there are no instantaneous causal influences, then Granger causality defined by Granger \citep{granger1980testing} can be seen as causal relations that can be discovered from stochastic processes with constraints-based methods such as PC. Of course, if those assumptions are violated, one may still apply Granger causal analysis, but the estimated relations may not be true causal influences.} time series analysis. Through the collective efforts of various teams and the contributions of the open-source community, causal-learn is always under active development to incorporate the most recent advancements in causal discovery and make them available to both practitioners and researchers.

\subsection{(Conditional) independence tests}

In addition to its comprehensive search methods, causal-learn also provides a variety of (conditional) independence tests as independent modules. Besides being an essential parts of several search methods, these tests can also be independently utilized and seamlessly integrated into existing statistical analysis pipelines. Currently,the library features a diverse array of such tests including Fisher-z test \citep{fisher1921014}, Missing-value Fisher-z test, Chi-Square test, Kernel-based conditional independence (KCI) test and independence test \citep{zhang2011kernel}, and G-Square test \citep{tsamardinos2006max}, each with distinct capabilities and benefits. The Fisher-z test is ideally suited for linear-Gaussian data, while the Missing-value Fisher-z test addresses the challenges of missing values by implementing a testwise-deletion approach. For categorical variables, the Chi-Square and G-Square tests are most effective. For users interested in a nonparametric test or the case with mixed categorical and continuous data types, the KCI test is an option. Overall, the range of tests offered by causal-learn underscores its versatility in handling diverse data types.

\subsection{Score functions}
\looseness=-1
Moreover, a diverse range of score functions is available in \textit{causal-learn}. These score functions quantify the goodness of fit of a model to the data, a crucial measure in score-based causal discovery methods, and can also be utilized independently for model selection in a broader range. Among these, the Bayesian Information Criterion (BIC) score \citep{schwarz1978estimating} is used extensively, offering a balance between model complexity and fit to the data. Another important score function is the Bayesian Dirichlet equivalent uniform (BDeu) score \citep{buntine1991theory}. This score function, especially beneficial for discrete data, incorporates a uniform prior over the set of Bayesian networks. Additionally, the Generalized Score \citep{huang2018generalized} is also available in causal-learn, which offers the flexibility to accommodate more complex scenarios and is beneficial for nonparametric cases where the true data-generating process does not align with the assumptions of BIC (linear Gaussian) or BDeu (discrete).


\subsection{Utilities}

Causal-learn further offers a suite of utilities designed to streamline the assembly of causal analysis pipelines. The package features a comprehensive range of graph operations encompassing transformations among various graphical objects integral to causal discovery. These include Directed Acyclic Graphs (DAGs), Completed Partially Directed Acyclic Graphs (CPDAGs), Partially Directed Acyclic Graphs (PDAGs), and Partially Ancestral Graphs (PAGs). Additionally, to enhance the convenience of experimental processes, \textit{causal-learn} features a set of commonly used evaluation metrics to appraise the quality of the causal graphs discovered. These metrics include precision and recall for arrow directions or adjacency matrices, along with the Structural Hamming Distance \citep{acid2003searching}.

\subsection{Demos, documentation, and benchmark datasets}

The \textit{causal-learn} package also contains extensive usage examples of all search methods, (conditional) independence tests, score functions, and utilities at 
\\ \centerline{ \url{https://github.com/py-why/causal-learn/tree/main/tests}.} 
\\
Furthermore, detailed documentation is available at \\
\centerline{\url{https://causal-learn.readthedocs.io/en/latest/}.} \\
It is worth noting that it also includes a collection of well-tested benchmark datasets--since ground-truth causal relations are often unknown for real data, evaluation of causal discovery methods has been notoriously known to be hard, and we hope the availability of such benchmark datasets can help alleviate this issue and inspire the collection of more real-world datasets with (at least partially) known causal relations. 



%-------------------------------------------------------------------------------
\section{Implementation} \label{imple}
%-------------------------------------------------------------------------------

% Figure environment removed

\sys's implementation consists of (\romannumerber{1}) a fully-functional \sys switch prototype which implements the overall in-fabric logic; and (\romannumerber{2}) a set of software APIs exposed to applications. Our prototype is built upon a FPGA-assisted commodity switch while a P4 programmable switch implementation is also provided.

%\sys's implementation consists of (\romannumerber{1}) a FPGA-based fully-functional prototype which implements the self-defined switch logic and (\romannumerber{2}) a set of software APIs exposed to multicast applications. \sys's switch logic can also be implemented on the Tofino P4 switch, as described in $\S$\ref{dis}.

\parab{FPGA-based prototype.} We implement the group registration, data packet duplication, header modification, and feedback aggregation logics on an FPGA board. The board is equipped with a commodity FPGA chip~\cite{ultrascale} and four 100Gbps Ethernet interfaces. The FPGA resource utilization is shown in Table~\ref{tab:overhead}. We build our testbed with the FPGA board, a commodity Ethernet switch, and four servers, as illustrated in Fig.~\ref{fig:fpgaprototype}. Each server is equipped with a commodity RNIC. The FPGA board and four RNICs are connected to the commodity switch through 100Gbps Ethernet interfaces. 

The commodity switch is configured by Access Control List (ACL) to route the servers' multicast traffic to the FPGA board. The FPGA board identifies the multicast data (ACK\footnote{In Fig.~\ref{fig:fpgaprototype}, we use ACK to represent all types of feedback.}) packets through the specific packet header by \textit{Parser} and \textit{Arbiter}. The data (ACK) packets will be duplicated (aggregated) by \textit{Duplicator} (\textit{ACK Aggregator}). The resulting packets will be pushed in \textit{Queue System}, waiting for the \textit{Multiplexer} to schedule in case for queue competition. Finally, the duplicated data (aggregated ACK) packets are sent back to the commodity switch. During processing, the \textit{Multicast Forwarding Table} is accessed when needed. %\todo{describe Fig. 8} Note those resulting packets' destination IP would be unicast IPs, so the switch routes them as normal unicast packets.
%

\parab{P4-based implementation.}
\sys in-fabric logic can be implemented on the P4 switch as well with some special handling. For the one-to-many data forwarding, P4 switch duplicates packets at the Traffic Manager (TM). The extended table states in Fig.~\ref{fig:table} are stored in the egress pipeline, indexed by <GroupID, EgressPort>. Because the lookup key in P4 switch is at most 32bits, we can use the least significant 24bits of GroupIP plus the 8bits port number as the real index. 

For the many-to-one feedback aggregation, there are two challenges due to the limited capability of P4 switch. Commodity P4 switch switch contains many stages, each with minimal computation capability and independent memory. Firstly, a single stage cannot support the \textit{wrapped-around} PSN comparison. To handle it, we simplify the standard PSN comparison to match the stage's capability, resulting in a tighter PSN space reduced from $2^{23}$ to $2^{22}$. Secondly, a single stage cannot iterate the entire table entries and find the minimum PSN. To handle it, we leverage multiple stages, each responsible for partial entries. Thus, the maximum number of entries supported in each multicast group is limited by the total stage number.

Besides, the P4 switch lacks the computation capability to recalculate the Invariant Cyclic Redundancy Checksum (ICRC) for the modified (aggregated) data (ACK) packets. As a result, those packets would violate the ICRC validation and be discarded at the receiver. This is why we choose the FPGA-based prototype to evaluate \sys in this work. However, a recent work shows that some RNICs provide the ability to bypass ICRC validation~\cite{switchML}. 
%Although this can work, it's a compromised method having security risks. 
%So we select the more integrated FPGA-based prototype to evaluate in $\S$\ref{eva}. Other operators can choose their preferred implementation based on their requirements.

%\begin{algorithm}[t]
%	\caption{Update PSN record and find PSN minimum in P4}\label{alg:psncomp}
%	\begin{algorithmic}[1]
%		%\Function{Generation}{}
%		\State $ack.psn, ack.port\gets $ the PSN and port of ACK packet
%		\State $rec.psn, rec.port\gets$ the PSN and port recorded
%		\State $isTrigger\gets$ whether the packet is a trigger packet
%		%\State $last\_ack\_psn\gets$ last aggregated ACK's psn
%		%\State $min\_port\gets$ port with minimum $ack\_psn$ last time
%		\State min.psn = ack.psn;
%		\State \textcolor{purple}{// every stage compare the PSN}
%		\If{$ack.psn > rec.psn$ or $(ack.psn <= 24'b3fffff$ and $rec.psn >= 24'b600000)$}
%			\State min.psn = rec.psn;
%			\If{rec.port == ack.port}
%				\State rec.psn = ack.psn;
%			\EndIf
%		\EndIf
%		\State \textcolor{purple}{// last stage write the min PSN back}
%		\If{isTrigger} 
%			\State ack.psn = min.psn;
%			\State Forward ACK.
%		\EndIf
%	\end{algorithmic}
%\end{algorithm}

\parab{Software APIs.} We provide various communication libraries and middleboxes for \sys multicast support. Take the commonly-used OpenMPI as an example, we modify the OpenMPI (v4.1.1)~\cite{openmpi} and UCX (v2.3)~\cite{ucx} to adapt to \sys's design, as shown in Fig.~\ref{fig:fpgaprototype}. Specifically, we add a new implementation of $MPI\_Bcast$ and modify UCX for multicast QPs creation and data transmission. When the new $MPI\_Bcast$ is called, the MPI process calls the UCX to establish QPs for multicast. Multicast members exchange their QPs information, and the handshake starts, as described in Appendix \ref{apx:regis}. Once the multicast group is successfully established, the UCX finally calls the RDMA primitives defined in the well-known \textit{libibverbs}~\cite{libibverbs} to transmit data. The software modifications at the end-host are transparent to the upper-layer applications and don't require any RNIC or RDMA driver modification.
%\parab{Coalescence of unicast and multicast}
%When we design \sys, there is a question in our mind: \textit{which is better, maintaining unicast and multicast transports separately at end-host, or utilizing the in-network support to enabling them to match the same transport?} Because of the long-standing resource limit in RNICs and the emeging trend of shifting appropriate computation task to programmability network, we believe the latter is the correct selection.

\begin{table}[t]
	\small
    \centering
%	\begin{center}
%    \begin{tabular}{l|c|c|c}
    \begin{tabular}{|p{0.2\linewidth}|p{0.18\linewidth}|p{0.18\linewidth}|p{0.18\linewidth}|}
    \hline
    \textbf{Resource} & \hfil \textbf{LUT} & \hfil \textbf{Register} & \hfil \textbf{BRAM} \\
    \hline
   	\textbf{Usage} & \hfil 53169 & \hfil 15391 & \hfil 188 \\
    \hline
    \end{tabular}
%    \end{center}
    \caption{Resource usage of the \sys in-fabric logic.}
    \label{tab:overhead}
    \vspace{-0.25cm}
\end{table}

%\parab{Resource overhead}  Note that the size of multicast forwarding table is determined by the number of ports of the switch and doesn't scale up with the multicast group size. 2.7MB memory can support upto 1K multicast groups, which is satisfied in datacenter. We provide a detailed calculation of the maxmum group support in Appendix \ref{apx:cal}. 
\section{Evaluation} \label{sec:evaluation}

\begin{table*}[tbp]
\centering
\small
\begin{tabular}{cccccccccc}
\toprule
& \multicolumn{3}{c}{\msr} & \multicolumn{3}{c}{\negc} & \multicolumn{3}{c}{\wsj} \\
& Acc. & F1 & wF1 & Acc. & F1 & wF1 & Acc. & F1 & wF1 \\ \cmidrule(lr){2-4} \cmidrule(lr){5-7} \cmidrule(lr){8-10} 
\udel & 66.86 & 56.76 & 64.3 & \textbf{80.80} & 55.45 & 77.9 & 63.74 & 64.23 & 63.2 \\
\icsi & \underline{71.19} & 64.73 & 70.4 & 80.36 & 64.53 & \underline{78.6} & 64.62 & 64.15 & 63.4 \\
\cnts & 68.59 & 61.39 & 67.2 & 78.68 & 61.62 & 76.8 & 64.31 & 64.59 & 64.4 \\
\osu & 68.02 & 60.28 & 66.6 & 79.24 & 57.04 & 76.5 & 69.20 & 69.63 & 68.9 \\
\isg & 67.05 & 58.83 & 65.3 & 77.34 & 59.52 & 75.6 & 69.15 & 69.35 & 69.2 \\ \midrule
\bert & \textbf{71.68} & \underline{66.70} & \textbf{71.4} & 77.79 & \underline{72.87} & 77.7 & \underline{80.95} & \underline{80.93} & \underline{80.9} \\
\roberta & 70.91 & \textbf{67.53} & \underline{70.7} & \textbf{80.80} & \textbf{77.29} & \textbf{80.7} & \textbf{82.61} & \textbf{82.70} & \textbf{82.6} \\ \midrule
Average & 69.19 & 62.32 & 67.99 & 79.29 & 64.05 & 77.69 & 70.65 & 70.80 & 70.37 \\
\bottomrule
\end{tabular}
\caption{\label{tab:performance} Overall accuracy (Acc.), macro-averaged F1 (F1), and weighted-macro F1 (wF1) scores of the algorithms depicted in Section~\ref{sec:algorithm}. For instance, \msr-\udel refers to a C5.0 classifier trained on the \msr~corpus, using the feature set mentioned in \citet{greenbacker-mccoy-2009-udel}.}
%Its Acc., F1 and wF1 of this model are 66.86, 56.76, and 64.3, respectively.}
\end{table*}


In this section, we introduce the evaluation protocol and report the performance of the models.

\subsection{Implementation Details} \label{sec:implementation}

For \bert and \roberta, we used \textit{bert-base-cased} and \textit{roberta-base}, both from Hugging Face. For fine-tuning, we set the batch size to 16, the learning rate to 1e-3, the dropout rate to 0.5, and the size of the output layer to 256. We ran each model for 20 epochs and used the one that achieved the highest F1 score on the development set. The implementation details of the classic ML-based models can be found in Appendix~\ref{sec:appendixML}.

\subsection{Evaluation Protocol} \label{sec:protocol}

The main evaluation metric in the GREC-MSR shared tasks was accuracy. 
In addition to accuracy, we also report macro-F1 and weighted-macro F1. We argue that different metrics evaluate algorithms from different perspectives and provide us with different meaningful insights. 
For pragmatic tasks like REG, it makes sense to ask how well an algorithm performs on naturally distributed data which is often imbalanced. For these cases, reporting accuracy and weighted F1 are logical. 
Furthermore, analogous to other classification tasks, minority categories should not be overlooked. Take as an example the class \emph{description} in the \negc corpus, which occurs only 4\%. If a model fails to produce this class, the produced document might sound unnatural. Therefore, it is important to ensure that an algorithm is not over- or under-generating certain classes. Looking into accuracy and macro-F1 together provides insights into such cases.

\subsection{Performance of the Models}\label{subsec:overallacc}

The overall accuracy of the models, their macro F1, and their weighted-macro F1 are presented in Table \ref{tab:performance}. 
We also present the ranking of the models based on these scores in Appendix~\ref{sec:app_rank}. 


\paragraph{PLM-based Models.} The best-performing models across all corpora and metrics are PLM-based models.  In six out of nine rankings, \bert and \roberta are ranked as the top two models. The sole exception is \negc, where \bert is the second worst model. The benefit of using PLMs is the largest on the \wsj corpus. For example, \roberta improves the macro F1 score from 69.63 (i.e., the performance of the best ML-based model) to 82.70.


\paragraph{ML-based Models.} In contrast to the robust performance of the PLM models, the performance of the classic ML models is more corpus-dependent. In the case of \msr and \negc, \icsi is the best-performing model, while in the case of \wsj, it is at the bottom section of the rankings. Another interesting observation is the performance of the \udel models. In terms of accuracy, \udel has the highest performance in \negc, while it has the lowest performance in both \msr and \wsj. In terms of macro-F1 rankings, the \negc \udel model dropped from first to last place, whereas \bert improved from penultimate place to second place. In general, our ML models yielded lower scores than the original models used in the GREC study \citep{belz2009generating}. This could be attributed to a variety of factors, including differences in feature engineering and model parameters.

\paragraph{Comparing Different Metrics.} 

Upon comparing average scores across the three metrics, we observe that for \msr and \negc, PLMs are clear winners only when macro-F1 is the metric in question. However, for \wsj, PLMs are winners on all three metrics. This may be because the distribution of categories in \wsj is much more balanced than in the other two corpora.
%\section{Discussion}
\label{sec: discussion}
\kmsdelete{In this work} We study \kmsreplace{Fairness-Aware PAC learning}{Fair-ERM} in the malicious noise model, and  in some cases allow 
the learner to maintain optimal overall accuracy despite the signal in Group $B$ being almost entirely washed out.
%when we allow learners to use the
%$\PQ$ randomized expansion of the hypothesis class $\mathcal{H}$
In particular we show that different fairness constraints have fundamentally different behavior in the presence of Malicious Noise, in terms of the amount of accuracy loss that a given level of Malicious Noise could cause a fairness-constrained learner to incur. 
The key to achieving our results, which are more optimistic than those in \cite{lampert}, is allowing for improper learners using the (P,Q)-randomized expansions of the given class $\mathcal{H}$.
%We \kmsreplace{present a picture of the}{prove upper and lower bounds on}
%accuracy loss for a range of fairness notions, given \kmsreplace{this simple randomization step.}{learning over $\PQ$.
%In general our results indicate Fair-ERM (given learning over $\PQ$) is more robust than claimed in \cite{lampert}.
The type of smoothness we create by using $\PQ$ seems to be a natural property that is likely shared by many natural hypothesis classes.

Fairness notions are motivated as a response to learned disparities when there is \kmsdelete{data corruption or} systemic error affecting \kmsdelete{the data for}
one group. 
Fairness notions are supposed to mitigate this by ruling out classifiers that have worse performance on a sub-group. 
This can peg both classifiers at a lower level of performance \kmsdelete{(e.g that the lower subgroup)} in order to \emph{motivate} \cite{hardt16} improving the data collection or labelling process to obtain more reliable performance. 
%So in \kmsreplace{some}{a} sense, sensitivity of the fairness notion to poor sub-group performance caused by malicious noise is the \textit{point} of fairness constraints! 
However, it also desirable that fairness constraints perform gracefully when subject to Malicious Noise because fairness constraints will be used in contexts where the data is unreliable and noisy and this might not be known to the learner.
This tension, exposed by our work, motivates 
%a revisiting of fairness notions from first principles approach and trying to axiomatize the 
%desired properties of a fairness intervention a la cryptography and privacy. \footnote{Work in multi-calibration \cite{multicalib} is a viable direction for this problem but it is unclear how 
%that and related notions behave with unreliable data. }
on going work studying the sensitivity level of fairness constraints. 
%If we we are to take a view, if a classifier is deployed 

%-------------------------------------------------------------------------------
\section{Related Works} \label{works}
%-------------------------------------------------------------------------------
\parab{Internet and Datacenter multicast.} Multicast has been widely applied in large-scale Internet applications, such as Internet broadcast \cite{iptv}, video conferencing \cite{chen2011celerity}, and multiplayer games \cite{cho2009enabling}, \etc Prior works for the Internet \cite{chiang2018online, huang2016multicast,diab2020oktopus,ren2018optimal, diab2022yeti} mostly focus on the multicast routing, \ie, to find promising multicast paths, inside ISPs. For instance, Yeti~\cite{diab2022yeti} supports multicast routing with traffic engineering and service chaining requirements for large-scale ISPs. Yeti creates labels representing forwarding information for multicast graphs and processes these labels to forward packets to targeted paths. Although there are a bunch of prior works on the Internet, most of them merely provide best-effort delivery, which only works for applications without reliability requirement. 

There are some works~\cite{widmer2001extending, rizzo2000pgmcc} aim to provide reliability for datacenter applications upon approaches with best-effort delivery. However, existing reliable multicast solutions mainly adopt a TCP-like software stack and cannot meet the demand for high-speed communication in datacenters. In contrast, \sys  leverages the advanced RDMA stacks to process multicast traffic, providing high-speed reliable communication.

%And there are few works for in-fabric multicast in datacenters with technical details~\cite{sharp}. \todo{software datacenter multicast.} 
%MTRSA \cite{huang2016multicast} is a multi-tree routing algorithm attempting to optimize multicast routing paths inside ISP with consideration of node and link capacity. \cite{ren2018optimal} aims to find the optimal path to forward traffic while preserving the ordered access of a sequence of network services. These services are usually deployed as virtual functions on core routers in the ISP network. 

\parab{Multicast scalability.}
Datacenter applications impose a demand for high scalability. As the traditional IP multicast~\cite{crowcroft1988multicast}, along with its native group management, IGMP and tree construction protocol, PIM~\cite{estrin1998protocol}, are poor in scalability, many works~\cite{shahbaz2019elmo, diab2022orca, li2013scaling} attempt to address the scalability issue, \ie, supporting as much as possible multicast groups. For example, Elmo~\cite{shahbaz2019elmo} encodes the routing link of a multicast tree into rules formatted as packet header. Thus Elmo switch only needs to maintain rule parsing logic, reducing the total switch-maintained states. Orca~\cite{diab2022orca} utilizes the large memory space of the server, making servers assist in forwarding packets, reducing the switch's burden on maintaining states. 

These works that address the scalability issue are orthogonal with the \sys design. Our goal in this work is to provide a general multicast protocol with prominent RDMA features and reliability guarantee rather than compressing the switch-maintained states. As mentioned before, \sys can support at least 1K multicast groups using 0.92MB space, which is acceptable for a majority of multicast applications in datacenters. \sys can support even more multicast groups when getting extended further upon these works.

%\parab{Group communication.}
%The group communication in the datacenter is not limited to one-to-many and many-to-many. There are various patterns \cite{wan2020rat, rashidi2021enabling}, such as all-gather, reduce-scatter, and all-reduce. \sys can be extended to support these group communication patterns, and we leave this as our future work.


%-----------------------------------------------------------------------
\section{Conclusion}
%-----------------------------------------------------------------------
We present \sys, an RDMA-accelerated multicast protocol that significantly facilitates multicast communication while maintaining compatibility with commodity RNICs. \sys reuses existing RDMA RC transport to process multicast traffic, thus preserving the optimal bandwidth utilization and benefiting from superior RDMA functionality. \sys replaces packet headers and aggregates feedback in the fabric to remain compatible between existing RC logic and multicast traffic. We provide a fully functional \sys prototype, which requires no RNIC modification. Extensive testbed experiments and simulations demonstrate \sys's superior performance in multicast acceleration.

\sys opens the door for efficiently leveraging the prominent RDMA stacks with in-fabric assistance to accelerate group communication patterns. While this work mainly focuses on multicast; for future works, we plan to extend \sys for more group communication patterns, such as many-to-one (\eg, $MPI\_Reduce$) and many-to-many (\eg, $MPI\_Alltoall$), \etc

%-----------------------------------------------------------------------
%\section*{Acknowledgments}

%-----------------------------------------------------------------------
%-----------------------------------------------------------------------
\bibliographystyle{plain}
\bibliography{main.bib}

\begin{comment}
\section{System Architecture}
\label{appendix:architecture}
\system has a novel modularized system architecture with three key components: 
\emph{StreamManager}, 
\emph{TxnManager} and \emph{TxnScheduler}. 
These components are instantiated in each thread locally.
The execution outline of \system is presented in Algorithm~\ref{alg:algo}.
Transactional stream processing is continuous and potentially never ends (Line 1$\sim$8).
The dependency resolution and execution of state transactions are separated into two non-overlapping phases by punctuations~\cite{Tucker:2003:EPS:776752.776780} (Line 2 and 5), which guarantees that no subsequent input event will have a smaller timestamp. 
Effectively, a batch of state transactions is collected during the first phase, and processed during the second phase.

In the first phase (i.e., stream processing phase), 
the \emph{StreamManager} conducts preprocessing for every input event ($e$). Similar to some prior works~\cite{tstream}, state transactions may be issued but not immediately processed during preprocessing (Line 3).
The \emph{pre\_processing} and \emph{post\_processing} functions are exposed as APIs to users.
The \emph{TxnManager} handles dependency resolution (Line 4) among state transactions and insert decomposed operations to construct a \tpg. We discuss the detailed two-phase \tpg construction process in Section~\ref{subsec:construction}.

In the second phase  (i.e., transaction processing phase), 
the \emph{TxnManager} is first involved again to refine (Line 6) the constructed \tpg with further dependency resolution.
The \emph{TxnScheduler} 
schedules operations for concurrent execution based on the constructed \tpg according to the three dimensions of scheduling decisions (Line 7). 
In particular, a scheduling decision model $M$ is instantiated based on the constructed \tpg (Line 14).
\textbf{\circled{1}} Guided by $M$, execution threads adopt an exploration strategy (Section~\ref{subsec:explore}) to explore the constructed \tpg for operations available to be scheduled constrained by dependencies. 
\textbf{\circled{2}} 
During exploration, one or multiple operations may be treated as the 
% basic 
unit of scheduling (Section~\ref{subsec:granularity}). 
Subsequently, \textbf{\circled{3}} every thread executes operation(s) in the unit of scheduling with various abort handling mechanisms (Section~\ref{subsec:abort_handling}).
Only when state transactions are processed (i.e., committed or aborted) can the associated input events be postprocessed (Line 8) by the \emph{StreamManager} based on transaction processing results.
\end{comment}

\begin{comment}
\begin{algorithm}
\footnotesize
    \KwData{$e$ \tcp{Input event}}
    \KwData{$txn_{ts}$ \tcp{State transaction}}
    \KwData{$G$ \tcp{The currently constructed TPG}}
    \While{!finish processing of input streams}{
        \eIf(\tcp*[h]{Phase 1}){\text{$e$ is not a $punctuation$}}{
                $txn_{ts}$ $\gets$ PRE\_Processing($e$)\;
                \textbf{TPG\_Construction}($G$, $txn_{ts}$)\; 
          }(\tcp*[h]{Phase 2}){
                \textbf{TPG\_Refinement}($G$)\; 
                \textbf{TXN\_Scheduling}($G$)\; 
                POST\_Processing()\;
          }
    }
    
    \SetKwFunction{FMain}{TPG\_Construction}
    \SetKwProg{Fn}{Function}{:}{}
    \Fn{\FMain{$G$, $txn_{ts}$}}{
        $O_{1..k}$ $\gets$ \textbf{Partition} $txn_{ts}$\;
        \ForEach{\text{operation $O_{i}$ $\in$ $O_{1..k}$}}{
            \textbf{Identify} its \ld\;
            $G$ $\gets$ $G$ + $O_{i}$ \;
        }
    }
    \SetKwFunction{FMain}{TPG\_Refinement}
    \SetKwProg{Fn}{Function}{:}{}
    \Fn{\FMain{$G$}}{
        \ForEach{\text{vertex $e_{i}$ $\in$ $G$}}{
            \textbf{Identify} its \td, \pd\;
        }
    }
    
    \SetKwFunction{FMain}{TXN\_Scheduling}
    \SetKwProg{Fn}{Function}{:}{}
    \Fn{\FMain{$G$}}{
        $M$ $\gets$ Instantiated with $G$;\tcp{A decision model}
        \While{!finish scheduling of $G$
        }{
          \textbf{\circled{2}} $Scheduling Unit$ $\gets$ \textbf{\circled{1}} \emph{Explore}($G$, $M$)\; 
            \textbf{\circled{3}} \emph{Execute with Abort Handling} ($Scheduling Unit$)\; 
        }
    }
  \caption{Execution Outline of \system}
  \label{alg:algo}
\end{algorithm}
\end{comment}

%%%%%%%%%%%%%%%%%%%%%%%%%%%%%%%%%%%%%%%%%%%%%%%%%%%%%%%%%%%%%%%%%%%%%%%%
\end{document}
%%%%%%%%%%%%%%%%%%%%%%%%%%%%%%%%%%%%%%%%%%%%%%%%%%%%%%%%%%%%%%%%%%%%%%%%

%%  LocalWords:  endnotes includegraphics fread ptr nobj noindent
%%  LocalWords:  pdflatex acks