%%%%-----------------Definition/Theorems------------------%%%%
\theoremstyle{definition} %%upright text, extra space above and below
    \newtheorem{definition}{Definition}

\theoremstyle{plain} %% italic text, extra space above and below
    \newtheorem{theorem}[definition]{Theorem}
    \newtheorem{proposition}[definition]{Proposition}
    \newtheorem{lemma}[definition]{Lemma}
    \newtheorem{corollary}[definition]{Corollary}
    \newtheorem{claim}[definition]{Claim}
    \newtheorem{assumption}[definition]{Assumption}

\theoremstyle{remark} %% upright text, no extra space above or below
    \newtheorem{remark}[definition]{Remark}



%%%%-----------------Bibliography------------------%%%%

\usepackage[bibstyle=alphabetic,citestyle=alphabetic,useprefix,giveninits=true, sorting=ynt, sortcites, minbibnames=99,maxbibnames=99,backend=biber]{biblatex}  %%other styles: numeric-comp, authoryear 
    %% sorting: ynt in the text, nyt in the bibliography with additional command there
\renewbibmacro{in:}{} %% removes in: before journals
\bibliography{bibliography}  %% Name of the file with the bibliography
\emergencystretch=1em %% Adjusts the overfulls in the bibliography by allowing more space between words

%%% 
\DeclareSourcemap{
  \maps[datatype=bibtex]{
    \map[overwrite]{ %%If DOI is present, doesn't print arXiv
      \step[fieldsource=doi, final]
      \step[fieldset=url, null]
      \step[fieldset=eprint, null]
      }
     \map[overwrite]{ %% If only arXiv is present, doens't print pages and eid (eliminates duplicates)
      \step[fieldsource=eprint, final]
      \step[fieldset=pages, null]
      \step[fieldset=eid, null]
      \step[fieldset=journal, null]
    }  
  }
}

%%%%-----------------Headers------------------%%%%
    % \usepackage{fancyhdr} %% Headers
    %     \pagestyle{fancy}
    %     \fancyhf{}
    %     \fancyhead[LE,RO]{\thepage}
    %     \fancyhead[RE]{ \nouppercase{\leftmark} }
    %     \fancyhead[LO]{ \nouppercase{\rightmark} }
    %     \setlength{\headheight}{15pt}
    % \usepackage{emptypage}



%%%%-----------------Typing shortcuts------------------%%%%

    \newcommand{\Tr}[0]{\text{Tr}}
    \newcommand{\indices}{}
%% tilde variables 
    \renewcommand{\Tilde}{\widetilde}   
    \newcommand{\tc}{\widetilde{c}}
    \newcommand{\tom}{\widetilde{\omega}}
    \newcommand{\tg}{\widetilde{g}}
    \newcommand{\te}{\widetilde{e}}
    \newcommand{\txi}[1]{\widetilde{\xi}^{#1}}
    \newcommand{\tedl}{\widetilde{\underline{e}}^\dag}
    
%% Bold variables
    \newcommand{\bxi}{\boldsymbol{\xi}}
    \newcommand{\bc}{\mathbf{c}}
    \newcommand{\bom}{\boldsymbol{\omega}}
    
%% double tilde variables
    \newcommand{\ttc}{\widetilde{\tc}}
    \newcommand{\tte}{\widetilde{\te}}
    \newcommand{\ttom}{\widetilde{\tom}}
    \newcommand{\ttxi}[1]{\widetilde{\widetilde{\xi^{#1}}}}
    
%% Generic Math
    \DeclareMathOperator{\Ima}{Im}
    \newcommand{\oloc}{\Omega_{\mathrm{loc}}}
    \newcommand{\paral}{\slash\!\slash}
    \newcommand{\Ker}[1]{\mathrm{Ker}{(#1)}}
    \newcommand{\comp}[1]{\langle #1 \rangle}
    \newcommand{\X}[1]{(X_{#1})}
    \newcommand{\qsp}[2]{\,\ensuremath{\raise.5ex\hbox{$#1$}\big\slash\raise-.5ex\hbox{$#2$}}} 
    \newcommand{\pard}[2]{\frac{\delta#1}{\delta#2}}

%% BV shortcuts
    \newcommand{\FF}{\mathfrak{F}}
    \newcommand{\AKSZ}{\textsf{\tiny AKSZ}}
    \DeclareMathOperator{\BFV}{\mathit{BFV}}
    \DeclareMathOperator{\BV}{\mathit{AKSZ}}
    \newcommand{\dd}{\mathrm{d}}
    \newcommand{\ndash}{\nobreakdash-\hspace{0pt}}

%% Commands for AKSZ
    \newcommand{\Fp}[2]{\mathcal{F}_{#2}^\partial(#1)}
    \newcommand{\Sp}[2]{S_{#2}^\partial(#1)}
    \newcommand{\Qp}[2]{Q_{#2}^\partial(#1)}
    \newcommand{\varp}[2]{\varpi_{#2}^\partial(#1)}
    \newcommand{\alp}[2]{\alpha_{#2}^\partial(#1)}
    \newcommand{\uF}[1]{{\mathcal{F}}_{R}(#1)}
    \newcommand{\uFF}[1]{{\mathfrak{F}}_{R}(#1)}
    \newcommand{\uS}[1]{{S}_{R}(#1)}
    \newcommand{\uV}[1]{{\varpi}_{R}(#1)}
    \newcommand{\uQ}[1]{{Q}_{R}(#1)}
    \newcommand{\UQ}{{Q}_R}

%%filtration symbols
    \newcommand{\filt}[1]{(#1)}
    \newcommand{\filtint}[1]{M^{(#1)}}
    \newcommand{\filtBF}[1]{(#1 )}
    \newcommand{\filtintBF}[1]{M^{(#1 )}}

%%% BF non degenerate
    \newcommand{\BFnd}{BF_{*}}

%%%%-----------------For long computations------------------%%%%
    \makeatletter
        \newcommand{\zzlabel}[1]{\ifmeasuring@\else\ltx@label{#1}\fi} %%new label (necessary if amsmath is present)
    \makeatother

    \newcounter{terms}[equation] %%counter for terms in a equation
    \newcommand{\unl}[2]{\underline{#1}_{\refstepcounter{terms} \zzlabel{#2} \theterms}} %%underlines, counts a term and put the corresponding number. Put the term in the first slot and a label in the second

    \newcommand{\reft}[2]{(\ref{#1}.\ref{#2})} %%refers to a term in a equation as (equation.term)

    %\showlabels[\color{blue}]{zzlabel}  %%shows the labels of the terms

