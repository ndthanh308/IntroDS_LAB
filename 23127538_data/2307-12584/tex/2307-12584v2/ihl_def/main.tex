\section{Identifying the structural components of simulated galaxies}\label{sec:decomp}

In this section, we introduce our structural decomposition technique and apply it to two
example galaxies identified in Ref-L0100N1504 (S\ref{subsec:decomp_expl}). In S\ref{subsec:vary_nc}, we test the effect of varying
the number of Gaussians used by the GMM on the resulting disc, bulge, and IHL mass fractions, and in S\ref{subsec:decomp_alt_methods}
we compare the results of our decomposition to alternative estimates of the kinematic morphologies and IHL fractions of galaxies. The results in this section are limited to central galaxies identified at $z=0$ in Ref-L0100N1504 that have $M_{200}> 10^{11.7}\,{\rm M}_\odot$,
resulting in a sample of 2415 galaxies. This halo mass limit ensures that the structural and kinematic properties of the galaxies are
robust to spurious heating by DM particles at their half stellar mass radii \citep[$r_{50}$; see Table 2 of][]{ludlow_spurious_2023}.

\subsection{Decomposing the stellar component of \eagle\ galaxies using GMMs}\label{subsec:decomp_expl}
% Figure environment removed

We use GMMs to decompose the stellar components of \eagle\ galaxies into at most three physical components -- a disc, a centrally
concentrated bulge, and an extended IHL -- but make no attempt to isolate dynamically distinct sub-components of them (i.e. we do not
distinguish between thin and thick discs, or classic- and pseudo-bulges). We follow \citet[][see also \citealt{du_identifying_2019}]{obreja_nihao_2016}
and use the kinematic parameter space of stellar particles defined by their values of \jzjc, \jpjc, and \emin. Our GMMs find particle clusters
in this parameter space and approximate their distributions with multi-dimensional Gaussians. The GMMs evaluate the probability of each stellar particle belonging to each of these Gaussian distributions. In this work, particles are wholly allocated to the Gaussian distribution to which they are assigned the largest probability (i.e. we adopt a hard classification) and these Gaussian components are later associated with
the different structural components of a galaxy.

The optimal number of Gaussian distributions, $n_c$, required to disentangle the various components of a galaxy is not known a priori,
so assumptions must be made. \citet{obreja_nihao_2019} applied GMMs to 25 galaxies simulated at high resolution using $n_c\leq 5$ and associated
each of the best-fit Gaussians to one physical galaxy component: a thin or thick disc, classical- or pseudo-bulge, or a stellar
halo. However, the structural components of galaxies may not follow simple Gaussian distributions in the input parameter space (e.g. they may
posses kinematic substructure or exhibit non-Gaussian distribution functions), and when they do not multiple Gaussian distributions per galaxy
component seem to fare better. \citet{du_identifying_2019} used a modified Bayesian information criterion to determine that
$5\lesssim n_c \lesssim 12$ works well for galaxies in Illustris-TNG100 \citep{pillepich_first_2018} with stellar masses
$M_\star \gtrsim 10^{10}\, {\rm M_\odot}$; they assigned each Gaussian to a physical galaxy component based on the values\footnote{We follow the
nomenclature of \cite{du_identifying_2019} and denote the mean of the Gaussians output by our GMM fits using angular brackets, e.g. \meanjzjc\
or \meanemin.} of \meanjzjc and \meanemin.

\subsubsection*{Morphological classification}
We follow a different approach, and initially set $n_c=3$ in order to assess whether each galaxy possesses a significant disc component or
if it can be approximated as a pure spheroid. If one or more of the best-fit Gaussians have a mean circularity \meanjzjc $\geq 0.5$
we classify the galaxy as a disc (and hereafter refer to them as ``disc'' galaxies), otherwise it is spheroid dominated (hereafter, ``spheroid''). 
After some experimentation, we found that this initial morphological classification prevents the identification of spurious discs in
dispersion dominated galaxies for GMMs run with larger $n_c$, cases that often lead to net counter-rotating
spheroidal components. This occurs because dispersion supported systems often contain a small but significant
fraction of stellar orbits with high \jzjc\ values, and assigning those orbits to a ``disc'' skews the
distribution of the remaining \jzjc\ values lower. The top panels of Fig.~\ref{fig:flowchart} show two typical
galaxies that were classified as a disc (left panel) and spheroid (right panel) using $n_c=3$. 

After categorising galaxy morphologies this way, we again run GMMs but this time using a range of $n_c$ values. For the bulk
of our analysis we adopt $n_c=12$, but discuss below how increasing or decreasing $n_c$ affects the median mass fractions of the
various structural components of galaxies inferred from GMMs, as well as the intrinsic variation in them for individual galaxies.

\subsubsection*{Disc galaxy decomposition}
For all $n_c$, disc galaxies are modelled using the (\jzjc, \jpjc, \emin) parameter space of stellar particles and we assign the corresponding best-fit Gaussian distributions to one of the three physical galaxy components based on their values of \meanjzjc and \meanemin.
Those with \meanjzjc $\geq0.5$ are assigned to the disc, and the rest are split between the bulge and IHL. To do so, we identify the
Gaussian distributions with the maximum and minimum \meanemin\ values and define $e_{\mathrm{cut}}$ as the midpoint between them, i.e.
$e_{\mathrm{cut}}=[\min(\langle e/e_{\rm min}\rangle) + \max(\langle e/e_{\rm min}\rangle)]/2$. The remaining Gaussian clusters not
assigned to the disc (i.e. those with \meanjzjc$< 0.5$) are assigned to the bulge if \meanemin$\geq e_{\mathrm{cut}}$, or to the IHL
if \meanemin$<e_{\mathrm{cut}}$. Note that there is the chance that the IHL will not be detected, as is the case for the $n_c=3$
disc model in Fig. \ref{fig:flowchart} (upper-left panel). This occurs most often when $n_c$ is small and the IHL is significantly
less massive than the disc and bulge components.

\subsubsection*{Spheroidal galaxy decomposition}
We run the same GMMs for spheroids, but this time using (\jzjc, \emin) as the input parameter space (the broad \jpjc\ distributions
for the bulges and IHL of spheroidal galaxies overlap considerably making this quantity less useful for distinguishing these components). As for discs, Gaussians with \meanemin$\geq e_{\mathrm{cut}}$ are assigned to the
bulge; those with \meanemin$<e_{\mathrm{cut}}$ are assigned to the IHL\footnote{If we instead adopt a fixed specific binding energy cut of $e/e_{{\rm min}}=0.5$ to distinguish the bulge and IHL components, bulge masses are systematically higher by about 5 per cent; IHL masses are systematically lower by roughly 20 per cent.}.

\subsubsection*{Applying the method to example galaxies}
The middle row of Fig. \ref{fig:flowchart} show the results of running a GMM using $n_c=12$ on the disc and spheroidal galaxies
mentioned above, and in the bottom row we plot $x-z$ projections of the stellar particles assigned to each component (these galaxies are hereafter referred to as DG and SG, respectively). DG (left) is roughly the mass of the MW, having 
$M_\star \approx 10^{10.7}\,\rm{M_\odot}$ and $M_{200}\approx 10^{12.3}\,M_\odot$. SG is the central galaxy of a low-mass cluster with $M_\star\approx 10^{12.1}\,\rm{M_\odot}$ and $M_{200}=10^{14.3}\,{\rm M_\odot}$. Teal diamonds, maroon circles, and yellow squares show the values of \meanemin\ and \meanjzjc\ corresponding to the best-fit Gaussians assigned to the disc, bulge, and IHL, respectively.

The vertical black lines show the value of \meanjzjc$=0.5$ used to distinguish Gaussians assumed to
represent the disc component from those that represent the bulge or IHL. The horizontal black lines show the values of
$e_{\rm{cut}}$. For DG, we find $e_{\rm{cut}} = 0.59$ whereas for SG
$e_{\rm{cut}} = 0.50$. We note that these values are lower than the value of 0.75 adopted in previous work
\citep[e.g.][]{du_kinematic_2020, du_evolutionary_2021} to distinguish between GMM Gaussians representing the bulges and IHL of simulated
disc galaxies, but the most appropriate value of $e_{\rm cut}$ for a particular galaxy is unclear.
Recent work suggests that the optimal cut in binding energy may depend on galaxy morphology \citep[e.g.][]{zana_morphological_2022}.
Nonetheless, when applied to DG, our method effectively separates the tightly-bound stellar structures (i.e. the bulge) from the
loosely-bound ones  (i.e. the IHL). For SG, the separation between the bulge and IHL components is less
clear: several Gaussians have \meanemin$\approx e_{\rm{cut}}$, suggesting that the bulge and IHL components of this galaxy
are less dynamically distinct than those of DG.

% Figure environment removed

Fig. \ref{fig:cs_props} shows a few properties of the stellar particles belonging to the different structural
components of DG (left column) and SG (right column;
in both cases, the structural components were identified using a GMM with $n_c=12$). The top and middle
rows show the distributions of \jzjc\ and \emin, respectively; the bottom row shows the distributions of stellar ages ($t_{{\rm form}}$).
Grey histograms in each panel represent all stellar particles belonging to each galaxy, and the colored lines
show the subset of stellar particles assigned to the disc (teal dotted-dashed lines), bulge (solid maroon lines),
and IHL (dashed yellow lines). 

DG is composed of three distinct components: a rotationally supported disc, a tightly bound bulge and a loosely bound
stellar halo (the latter two components are largely dispersion supported, as expected). The bottom left panel of Fig.
\ref{fig:cs_props} shows that the disc component formed over an extended time period, with star formation peaking $\approx 9$
Gyrs ago, and gradually tapering off to roughly half of the peak value by $z=0$. The disc, which is composed primarily of stellar
particles that formed in-situ ($f_{\rm{ex-situ}}\approx 0.1$),\footnote{We use the ex-situ classification of \cite{davison_eagles_2020}.
Stellar particles are traced back to the snapshot prior to star formation: if the associated gas particle does not belong to the main branch of the $z=0$ subhalo at this snapshot, the stellar particles are flagged as having formed ``ex-situ''. All other particles are flagged as having formed ``in-situ''.} has a half mass stellar age of 6.4 Gyr, and an interquartile age range of
about 5 Gyr.

The bulge component of DG contains a slightly higher contribution from ex-situ stars ($f_{\rm{ex-situ}}\approx 0.18$) and is, on average,
composed of the oldest stellar populations in the galaxy; its half mass stellar age is $t_{50} \approx 10.1\, {\rm Gyrs}$. The IHL component is the
most extended in the galaxy (its half stellar mass radius is $r_{50}=30.9\,{\rm kpc}$; for the bulge, $r_{50}=2.8\,{\rm kpc}$) and is
dominated by stars that formed ex-situ ($f_{\rm{ex-situ}}=0.73$), suggestive of a merger-driven formation scenario. Similar to the bulge
component, the IHL hosts a relatively old stellar population with a half mass age of about 10.1 Gyrs. Neither the bulge nor the IHL contain an
appreciable number of stellar particles with ages $\lesssim 6\,{\rm Gyr}$ (the disc formed 54 per cent of its stellar mass in that time).

The bulge component of SG is similar to that of DG: it is dispersion-supported and comprised primarily of old stellar populations (the
half mass age of bulge stars in SG is $\approx 11.0\, {\rm Gyrs}$). The visible peaks in the $t_{{\rm form}}$ distribution of bulge stars
does, however, indicate that the bulge contains several distinct stellar populations. Together with the high ex-situ fraction
($f_{\rm{ex-situ}}\approx 0.80$), this implies a formation history dominated by multiple merger events, consistent with the standard
model for the formation of brightest cluster galaxies (BCGs) in $\Lambda{\rm CDM}$
\citep[e.g.][]{de_lucia_hierarchical_2007, robotham_galaxy_2014}. The IHL of SG is also dominated by ex-situ stars
($f_{\rm{ex-situ}}\approx 0.94$) and is comprised of two kinematically-distinct components: one dispersion-supported structure with a
peak orbital circularity of \jzjc$\approx 0$, and another with a peak at \jzjc $\approx -0.5$. The latter component counter-rotates with
respect to the net angular momentum of the galaxy and is visible in the plot of the component projections (see bottom panel of Fig \ref{fig:flowchart}).

% Figure environment removed

\subsection{The impact of varying $n_c$ on the decomposition results}\label{subsec:vary_nc}

Fig. \ref{fig:mass_by_n} shows the logarithmic change in the stellar mass that is allocated to the disc
(teal diamonds; upper panel), bulge (maroon circles), and IHL (yellow squares) components of individual galaxies as
$n_c$ is increased by $\Delta n_c=1$, starting from $n_c=3$ and increasing to $n_c=14$. The top and bottom panels show results separately for
the subset of disc and spheroidal galaxies, respectively. Symbols correspond to the median\footnote{When calculating
this ratio, we set $\log({\rm M}(n_c) / {\rm M}(n_c+1) ) = -1$ if a particular component is undetected for a given value of
$n_c$. Doing so allows us to include such instances in our estimates of the scatter and logarithmic change in component masses
as $n_c$ is increased, which would otherwise be biased by only including systems for which masses can be estimated. We note
that such occurrences do not affect the medians values plotted in Fig.~\ref{fig:mass_by_n}.} values of $\log[M(n_c) / M(n_c+1)]$,
and error bars show in interquartile scatter (note: $M(n_c)$ is used generically here to represent the mass assigned to a
particular galaxy component after running a GMM that uses $n_c$ Gaussians).

The top panel of Fig. \ref{fig:mass_by_n} shows that, for disc galaxies, the mass assigned to each
structural component is typically robust provided $n_c\geq 8$. 
For spheroids, the situation is similar. The median masses assigned to the bulge and IHL of individual
systems is largely unchanged when $n_c$ is increased from $\approx 4$, although the scatter in
mass assigned to the IHL of individual galaxies is larger than that of bulges for all $n_c$. The larger scatter in the mass assigned the IHL component compared to the bulge of spheroidal galaxies is due its relatively low mass, 
which makes it more susceptible to small changes in $M(n_c)$. Note, however, that both the median mass fraction and
scatter in mass assigned to each
component, including the IHL, are well-behaved provided $n_c\geq 9$, regardless of galaxy morphology. 

We adopt $n_c=12$ for our analysis for the following reason. If the Gaussian distributions are equally divided between
the structural components of galaxies, it implies that the disc, bulge, and IHL of disc-dominated galaxies will
each be identified by 4 Gaussian distributions; for spheroidal galaxies, the bulge and IHL will be identified by
6 Gaussians. The exact division of the Gaussians among the structural components of galaxies will, of course,
vary from galaxy to galaxy, but allowing for multiple Gaussians per galactic component can better accommodate
the presence of distinct sub-populations.

We stress, however, that the exact value of $n_c$ is somewhat arbitrary: provided 
$8\lesssim n_c \lesssim 15$, the masses allocated to the different structural components of most galaxies
are relatively stable. This result is shown another way in Fig.~\ref{fig:fcomp}, where
we plot the fraction of mass assigned to the various structural components of discs (left) and spheroids
(right) as a function of $M_{200}$. The thick lines in each panel correspond to results obtained for $n_c=12$;
the thin lines show results for other values of $n_c$ in the range $8\leq n_c \leq 15$. Note that the variation in the component mass fractions that arise from changing $n_c$ is smaller than the differences between the mass fractions of each component.

% Figure environment removed

\subsection{Comparison to alternative definitions of galaxy components}\label{subsec:decomp_alt_methods}

\subsubsection{Relation to kinematic estimates of galactic morphology}
% Figure environment removed

In Fig. \ref{fig:fdisc_comp} we compare the mass fraction allocated to the disc component by our GMM (\fdisk) with two other
kinematic indicators\footnote{Specifically, we define $\kappa_{\rm co}=(2\,K_\star)^{-1}\sum_{j_{z,k}>0} m_k\,(j_{z,k}/R_k)^2$, where $K_\star$ is the total kinetic energy of the stellar particles, $j_{z,k}$ is the $z$-component of the angular momentum of particle $k$, and $R_k$ is its distance from the $z$-axis. The disc-to-total ratio is defined as $D/T=1-S/T=1-2/M_\star \sum_{j_{z,k}<0} m_k$, where $S/T$ is the spheroid-to-total ratio and $m_k$ is the mass of the $k^{\rm th}$ stellar particle.} of galaxy morphology: the fraction of kinetic energy
in co-rotation (\kappaco; see \citealt{correa_relation_2017}) and the disc-to-total ratio (D/T; see e.g.
\citealt{thob_relationship_2019}). Both quantities were
measured using stellar particles that lie within a spherical 30 kpc aperture. 

The top panels of Fig. \ref{fig:fdisc_comp} plot the 2D histogram of \fdisk\ versus \kappaco\ (left) and D/T (right) for
disc galaxies. The lines show the median relations separately for three bins of $M_{200}$. The Spearman correlation
coefficient for the full sample of discs (labelled $\rho$ in the upper panels) is displayed in the bottom right corner of each panel.

Regardless of halo mass, there is a strong correlation between \fdisk\ and \kappaco\,  ($\rho=0.92$ for
all galaxies, and $\rho>0.89$ for the individual mass bins) that closely follows the one-to-one line. Close inspection,
however, reveals that the relation has a slope that is slightly steeper than 1: for \fdisk$\,\lesssim 0.4$
there is a tendency for \kappaco\, to exceed \fdisk. This is because our GMMs can in principle yield $f_{\rm disc}=0$,
whereas the minimum value of \kappaco\, for isotropic, dispersion supported systems with no rotation is
$\kappa_{\rm co}\approx 0.17$. This naturally biases the relation between \fdisk\, and \kappaco, particularly
for galaxies with small disc fractions.

The top right panel of Fig. \ref{fig:fdisc_comp} shows that \fdisk\ and D/T are also
strongly correlated ($\rho=0.87$), but that D/T is slightly higher than \fdisk\, for the majority of galaxies. 
This is true for all mass bins, but a larger offset is seen for higher $M_{200}$. This offset occurs
because the D/T statistic implicitly assumes that all spheroidal components of galaxies do not rotate and are
completely dispersion supported, attributing all net rotation in the galaxy to the disc. A similar offset between
D/T and disc mass fractions obtained from GMMs was reported by \cite{obreja_nihao_2016}. The mass dependence of the
offset hints at an increasing prevalence of rotational support in the spheroidal component for galaxies in higher
mass haloes. 

The bottom panels of Fig. \ref{fig:fdisc_comp} show the distributions of \kappaco\ and D/T for  our sample of spheroids. The vertical pink, yellow, and blue lines show the median values for each mass bin and the green 
line shows the values obtained for the spheroidal galaxy used for Figs. \ref{fig:flowchart} and \ref{fig:cs_props} (i.e. SG).
The median \kappaco\ value for all spheroids is 0.19, only slightly larger than the value expected for
isotropic, dispersion supported systems with no net rotation. There are a handful of
spheroids with relatively high values of \kappaco\ and D/T (for example, 6.6 per cent of spheroids
have \kappaco$\,\ge 0.4$, whose mean D/T ratio is 0.6). A visual inspection of these galaxies
reveal they typically have lenticular morphologies, consistent with fast rotators \citep[e.g.][]{cappellari_structure_2016}, or
have experienced recent mergers, which complicates the disc-bulge decomposition.

The strong correlations between \fdisk\, and these alternative morphology metrics are perhaps unsurprising given that both \kappaco\,
and D/T are calculated directly from the $z$-component of the angular momentum.  Although
they are not completely independent measures of galactic morphology, the close correspondence between them indicates that our galaxy
decomposition technique yields sensible results and that the kinematic morphologies we recover are in agreement with previous work. 

\subsubsection{Comparison to alternative definitions of the IHL}
% Figure environment removed

We next compare the estimates of $M_{\rm{IHL}}$ obtained from our GMMs
to those obtained using three alternative IHL definitions: 1) the total stellar mass that formed ex-situ ($M_{\rm{ex-situ}}$; e.g. \citealt{cooper_galactic_2010}); 2) the total stellar mass at
$r>100\,\rm{kpc}$ ($M_{>100\,\rm{kpc\,}}$; e.g. \citealt{pillepich_halo_2014}); and 3) the total stellar mass beyond 2 times
the stellar half-mass radius of a galaxy ($M_{>2\,r_{50}}$; e.g. \citealt{elias_stellar_2018}).

The top panel of Fig. \ref{fig:fihl_comp} shows the relationship between $M_{\rm{IHL}}$ and $M_{\rm{ex-situ}}$. Median relations
are shown separately for discs and spheroids (blue and orange lines, respectively) in four bins of $M_{200}$
(see the legend in the middle panel). Both $M_{\rm{IHL}}$ and $M_{\rm{ex-situ\,}}$ increase with increasing
$M_{200}$, as seen by the separation of lines of different type (they move up and to the right as mass increases). 
As a result, the values of $M_{\rm{IHL}}$ and $M_{\rm{ex-situ\,}}$ for the whole sample of galaxies
are strongly correlated ($\rho=0.9$), although at fixed halo mass the correlations are somewhat weaker (the Spearman rank coefficients for
the various mass bins plotted in Fig.~\ref{fig:fihl_comp} range from 0.57 to 0.79).

$M_{\rm{IHL}}$ is typically larger than $M_{\rm{ex-situ}}$ by about 0.2 dex, although this offset increases with increasing $M_{200}$. The offset between $M_{\rm{IHL}}$ and $M_{\rm{ex-situ\,}}$ among massive spheroids likely reflects the fact that both the bulge and IHL components of these galaxies are dominated by stars that formed ex-situ. Associating the IHL
exclusively with accreted stellar material is therefore inappropriate for these systems, because much of the bulge mass also formed ex-situ \citep[e.g.][]{pillepich_halo_2014}.
We will return to these points
in the next section.

Given its extended nature, defining the IHL with an aperture-based approach is common; in this case, all mass beyond some radius is assigned to the IHL, whereas the mass within that radius is assumed to belong to the other galaxy components. The middle and bottom panels of Fig.~\ref{fig:fihl_comp} show the relation between $M_{\rm IHL}$ and two aperture-based IHL mass measurements: $M_{>100\rm{kpc}}$ \citep[used in e.g.][]{pillepich_first_2018}, and  $M_{>2\,r_{50}}$ \citep[used in e.g.][]{elias_stellar_2018},
respectively.

We find that $M_{>100\rm{kpc\,}}$ is lower than $M_{\rm IHL}$ for $\approx$ 99 per cent of galaxies (not surprisingly, only the most massive galaxies have $M_{>100\rm{kpc\,}} > M_{\rm IHL}$).
The two IHL mass estimates are correlated for the whole population ($\rho=0.85$) but at fixed $M_{200}$ they are less correlated ($\rho >0.6$ only for the highest halo mass bins plotted). This
suggests that the correlation between these IHL mass estimates is primarily driven by the fact that $M_{>100\rm{kpc\,}}$
and $M_{\rm IHL}$ both increase with increasing $M_{200}$. A similar conclusion applies to the relationship between $M_{>2\,r_{50}}$ and
$M_{\rm IHL}$. Note that $M_{>2\,r_{50}}$ is typically larger than $M_{\rm IHL}$, and has a much smaller halo-to-halo scatter
\citep[see][for a similar finding]{canas_stellar_2020}.

Although the IHL masses estimated using these alternative methods correlate with our measurements, it is clear that none of them
reproduce our results, and fare even poorer when comparisons are made at fixed halo mass. This is due to the fact that the
various components of galaxies do not have well-defined edges, nor are they composed purely of in-situ or ex-situ stars.