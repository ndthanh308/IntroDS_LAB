\section{Conclusions and outlook}\label{sec:conc}

We used Gaussian Mixture Models (GMMs) to decompose the structural components of central galaxies identified in the $z=0$ output of the \eagle\ simulation. Most of our analysis was based on galaxies identified in Ref-L0100N1504, i.e. the 100 cubic Mpc flagship simulation of the \eagle\ Project \citep{schaye_eagle_2015}, which we supplemented with 30 galaxy clusters from the C-\eagle\ Project \citep{bahe_hydrangea_2017, barnes_cluster-eagle_2017}, and several hundred from 50-HiResDM \citep{ludlow_spurious_2023}. The latter run used the same numerical and subgrid model parameters as Ref-L0100N1504, but was was carried out in a smaller volume (50 cubic Mpc) and with seven times higher mass resolution in the DM component. Combined, these simulations allowed us to study the discs, bulges, and intra-halo light (IHL) of galaxies across a range of environments and over four orders of magnitude in halo mass ($10^{11.2}\,{\rm M_\odot}\leq M_{200} \leq 10^{15.4}\, {\rm M_\odot}$). Our main results are summarised below.

\begin{itemize}

\item Our galaxy decomposition technique is robust to small variations in $n_c$, i.e. the number of Gaussian distributions used by the GMM to isolate kinematically distinct galaxy components (see Section \ref{sec:decomp}). The stellar mass fractions assigned to the disc, bulge, and IHL components of galaxies are, on average, independent of $n_c$ provided $8\lesssim n_c \leq 15$ (although there is some variation on the level of individual galaxies). For most of our analysis we used $n_c=12$, which resulted in disc fractions that correlate strongly with alternative kinematic morphology estimators, such as $\kappa_{\rm co}$ (\citealt{correa_relation_2017}; see Fig. \ref{fig:fdisc_comp}). We also reproduce the morphology-dependence of the stellar-to-halo mass relation reported in previous observational and theoretical work (Fig. \ref{fig:fstar_morph}). 

\item Of the 3342 galaxies in our sample, roughly 71 per cent were classified as discs (i.e. have a non-zero disc mass fraction based on the $n_c=3$ GMM; see Section \ref{subsec:decomp_expl}), the remainder were classified as spheroids. We find that 34 per cent of the disc population are disc dominated (i.e. have $f_{\rm disc}>0.5$). Disc galaxies primarily occupy haloes with virial mass $M_{200}\lesssim 10^{12.5}\,{\rm M_\odot}$, whereas spheroids dominate in higher-mass haloes. Less than 0.5 per cent of galaxies in our sample possess no IHL component; these are primarily low-mass, disc-dominated systems (i.e. they typically have disc mass fractions $f_{\rm disc}>0.5$).
  
\item The basic properties of the disc and bulge components of galaxies are consistent with well-established observed trends: discs host younger stellar populations than bulges, have higher star formation rates, and, on comparable mass scales, are systematically larger than bulges (as quantified by their stellar half-mass radii). The half-mass radii of discs and bulges are in good agreement with observational results presented in \cite{robotham_profuse_2022} and \cite{bellstedt_resolving_2023} (see Fig. \ref{fig:r50}). Discs also contain a smaller contribution from ex-situ stars than bulges, by roughly a factor of 2 (see Fig. \ref{fig:meds}).
  
\end{itemize}

The primary aim of our work, however, was to study the properties of the IHL components of $z=0$ galaxies using a consistent methodology for IHL identification, and across a broad range of galaxy and halo masses. The results listed above give us confidence that our galaxy decomposition technique is sensible and robust, and that our IHL definition is on firm footing. The main insights into the IHL components of galaxies that our work provides are summarised as follows:

\begin{itemize}

\item Compared to discs and bulges, the IHL component of galaxies contains the highest fraction of ex-situ stars, which contribute roughly 70 per cent of the IHL mass at $M_{200}\approx 10^{12}\,{\rm M_\odot}$ and about 90 per cent at $M_{200}\approx 10^{14}\,{\rm M_\odot}$. The average SFR of the IHL (averaged over a lookback time of 500 Myr) is lower than that of bulges and discs. The IHL is, on average, composed of older stellar populations than discs and bulges at $M_{200} < 10^{12.5}\,{\rm M_\odot}$; above this mass scale the ages of bulges and the IHL overlap (Fig.~\ref{fig:meds}). At all mass scales, the IHL component is more extended than the disc or bulge component (as quantified by the half stellar mass radius of each component; Fig.~\ref{fig:r50}). The SFRs, ages, and sizes of the disc, bulge, and IHL components studied in this work are converged between the Ref-L0100N1504 and 50-HiResDM simulations, indicating that our results are robust to the effects of spurious collisional heating \citep[see][for details]{ludlow_spurious_2023}.
  
\item The fraction of mass assigned to the IHL, \fihl, increases with increasing stellar and halo mass (although slightly), but at fixed mass exhibits large scatter. For MW mass galaxies (i.e. $M_\star\approx 10^{10}\,{\rm M_\odot}$ or $M_{200}\approx 10^{12}\,{\rm M_\odot}$) we find \fihl $\approx 0.08$, which increases to \fihl $\approx 0.37$ at the group and cluster scale (i.e. $M_{200}\gtrsim 10^{13}\,{\rm M_\odot}$). The IHL fractions we obtain from our GMMs are in broad agreement with observed values obtained for disc and BCGs (Fig. \ref{fig:fihl}; although several nearby disc galaxies have systematically lower IHL fractions than discs in our simulations).

\item At halo masses $M_{200} \lesssim 10^{12.5} \rm{M_\odot}$, the scatter in \fihl\ is closely connected to the kinematic morphology of a galaxy: at a fixed halo mass, disc dominated galaxies have lower IHL fractions than spheroidal galaxies. The IHL fraction also correlates (albeit weakly) with the median age of the IHL (Fig. \ref{fig:fihl}) in such a way that the IHL of disc-dominated galaxies is systematically older than the IHL of spheroids of the same mass. This supports the idea that most disc-dominated galaxies have not undergone any recent mergers that could have increased their IHL fractions, and that, more broadly, the IHL fraction of a galaxy holds valuable information about its assembly history.

\item For halo masses $M_{200} \gtrsim 10^{13}\,{\rm M_\odot}$, the various correlations between the IHL mass fraction, its age, and galaxy morphology no longer hold. This may be due to the lack of diversity in galaxy morphologies at these mass scales (Fig. \ref{fig:fstar_morph}) or due to similarities in the merger histories of massive haloes. At these mass scales, galaxies are typically spheroidal and both the bulge and IHL components are dominated by ex-situ stars. It is not clear whether a meaningful distinction exists between the bulge and IHL components of galaxies in massive haloes, since both components exhibit similar stellar populations (Fig. \ref{fig:meds}).

\item Finally, we explored how the IHL transition radius, \rtrans (defined as the radius beyond which the stellar density profile of the IHL dominates over that of the inner galaxy), depends on halo mass and galaxy morphology. For disc galaxies, we found that \rtrans$\approx 30\,{\rm kpc}$. For spheroids, \rtrans\ depends strongly on $M_{200}$, increasing from  $r_{\rm IHL}\approx 20 \,{\rm kpc}$ at $M_{200}\approx 10^{12}\,{\rm M_\odot}$ to $r_{\rm IHL}\approx 300 \,{\rm kpc}$ at $M_{200}\approx 10^{15}\,{\rm M_\odot}$ (Fig. \ref{fig:ihl_trans}). Our methodology also predicts significant spatial overlap between galaxy components, implying that \rtrans\ (or any other spherical aperture) cannot be used to distinguish the IHL from the other structural components of a galaxy. 

\end{itemize}

We believe our results provide a sensible assessment of some of the basic properties of the disc, bulge, and IHL components of simulated galaxies. We plan to leverage the galaxy decomposition technique introduced in this work to explore properties of the progenitors of the IHL (as well as the progenitors of the ex-situ components of discs and bulges) and how they depend on halo mass.

Although our simulated galaxy sample spans a large range of halo masses ($10^{11.2}\,{\rm M_\odot}\leq M_{200} \leq 10^{15.4}\, {\rm M_\odot}$), massive galaxy clusters are sparsely sampled, and the IHL of low-mass galaxies ($M_{200}\lesssim 10^{12}\,{\rm M_\odot}$) is poorly resolved (typically containing only a few hundred to a few thousand stellar particles). Future work on the subject may benefit from recent large volume cosmological simulations that provide much larger samples of massive clusters \citep[e.g.][]{pakmor_millenniumtng_2023, kugel_flamingo_2023, schaye_flamingo_2023-2}. Likewise, cosmological simulations with higher baryonic resolution than \eagle\ will be useful for exploring the structure of the IHL of low-mass galaxies, which is crucial if we wish to properly interpret the low IHL fraction of the MW and other nearby galaxies, and to properly place their formation histories in a wider cosmological context \citep[e.g.][]{evans_how_2020}. Such simulations will also enable investigations into the IHL of dwarf galaxies, which, due to their low surface brightness, are difficult to study observationally but are potentially powerful probes of dark matter models \citep{deason_dwarf_2022}.