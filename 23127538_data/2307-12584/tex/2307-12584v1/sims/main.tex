\section{Simulations and analysis}\label{sec:sims}

\subsection{The \eagle\ simulations}

The \eagle\ Project \citep{schaye_eagle_2015, crain_eagle_2015} is a suite of cosmological, smoothed particle hydrodynamical
(SPH) simulations that model the formation and evolution of galaxies in a $\Lambda$CDM universe using cosmological parameters
consistent with the \citet{planck_collaboration_planck_2014} results. The majority of our analysis is based on the
${\rm L_{box}}=100$ cubic Mpc ``Reference'' run of the \eagle\ Project (i.e. Ref-L0100N1504; see Table 2 of
\citealt{schaye_eagle_2015}), which follows structure formation using ${\rm N_{DM}}=1504^3$ equal-mass dark matter (DM) particles
and initially the same number of baryonic particles. The mass of DM particles is $m_{\rm DM}=9.70\times 10^6\,{\rm M_\odot}$, and
$m_{\rm gas}=1.81\times 10^6\,{\rm M_\odot}$ is the initial baryonic particle mass.

Initial conditions were evolved to $z=0$ using an updated version of {\sc Gadget-3} \citep{springel_cosmological_2005,schaye_eagle_2015}
that includes subgrid models for, among other processes, radiative cooling and photoheating \citep{wiersma_chemical_2009},
star formation and stellar feedback \citep{schaye_relation_2008, dalla_vecchia_simulating_2012} the growth of supermassive black
holes (BHs) through mergers and accretion, and feedback from active galactic nuclei (AGN; \citealt{rosas-guevara_impact_2015}). The
subgrid model parameters were calibrated so that \eagle\ reproduced $z \approx 0$ observations of the galaxy stellar mass function,
the stellar size-mass relation, and the black hole mass-stellar mass relation \citep[see][for details]{crain_eagle_2015}. Subsequent work
has shown that \eagle\ also reproduces observations of galaxies at $z > 0$, such as their angular momenta \citep{lagos_angular_2017},
sizes \citep{furlong_size_2017}, velocity dispersion, and rotational velocities \citep{van_de_sande_sami_2019}, highlighting that galaxy
structure and kinematics are reproduced well by \eagle.

We supplement results from Ref-L0100N1504 at high and low halo masses using data from two other simulations.
One is the Cluster-\eagle\ project (C-\eagle; \citealt{bahe_hydrangea_2017, barnes_cluster-eagle_2017}), which is a suite of 30
resimulations of cluster-mass haloes; the other is a ${\rm L_{box}}=50\,{\rm Mpc}$ \eagle\ volume simulated using higher resolution in
the DM component while maintaining the original baryonic particle mass and force resolution that was used for \eagle\,
(see \citealt{ludlow_spurious_2023}, for details). We use the latter run, which is referred to in our paper as 50-HiResDM, to test the sensitivity of our results to the spurious collisional heating of stellar particles by DM halo particles \citep[see, e.g.,][]{ludlow_energy_2019,wilkinson_impact_2023}. 50-HiResDM employed the same subgrid models, as well as the same numerical and subgrid parameters as \eagle, but  its DM particle mass is
$m_{\rm DM}=1.39\times 10^6\,{\rm M_\odot}$, i.e. a factor of 7 lower than the value used for Ref-L0100N1504. The C-\eagle\, project used the
same subgrid and numerical set-up as \eagle, but adopted different parameters for the AGN feedback subgrid model to achieve better agreement
with the observed gas content of galaxy clusters (see \citealt{bahe_hydrangea_2017} and \citealt{barnes_cluster-eagle_2017} for details).

\subsection{Identifying DM haloes and galaxies}

DM haloes, their substructure haloes and associated galaxies were identified using \subfind\
\citep{springel_populating_2001, dolag_substructures_2009}. Haloes were first identified using a Friends-of-Friends algorithm
(FoF; \citealt{davis_evolution_1985}), which links nearby DM particles into FoF groups. Baryonic particles were assigned to the same group
as their nearest DM particle, provided it belonged to one. Each FoF halo was then divided into self-bound substructures, or ``subhaloes''
for short. One subhalo typically dominates the total mass of the FoF halo -- we refer to this as the ``central'' subhalo; lower-mass
subhaloes we refer to as ``satellite'' subhaloes. The baryonic particles associated with central and satellite subhaloes are
referred to as central and satellite galaxies, respectively. The stellar mass of a galaxy, $M_\star$, is defined as the total mass of all 
stellar particles gravitationally bound to a subhalo.

We restrict our analysis to stellar particles associated with central galaxies identified at $z=0$, but exclude those
bound to satellite galaxies. We note that this choice may be questionable for systems with large numbers of
satellite galaxies or for those undergoing mergers; in these cases, distinguishing stellar particles that are bound
to a central galaxy from those bound to its satellites is challenging. However, we find that MW-mass central
galaxies in \eagle\ typically contribute 97 per cent of the total stellar mass associated with their FoF haloes, with satellites
contributing $\lesssim 3$ per cent. At higher halo masses, however, where mergers and substructure are more prevalent,
the contribution of satellite galaxies to the total stellar mass of FoF haloes can be quite large, sometimes reaching as high
as $\approx 40$ per cent for halo masses $\gtrsim 10^{13}\,{\rm M_\odot}$. 
Satellite galaxies, however, typically dominate the stellar mass budget at larger galacto-centric radii than that which encompasses the
majority of the central's stellar material. For that reason, we neglect the possible contribution of satellites
to the IHL of galaxies.

We henceforth quantify halo masses using $M_{200}$, i.e. the total mass (DM plus baryonic) within the spherical radius 
$r_{200}$ that encloses a mean density of $200\times \rho_{\rm{crit}}(z)$, where $\rho_{\rm{crit}}(z)=3\,H(z)^2/8\pi G$ is
the critical density ($H(z)$ is the Hubble parameter and $G$ is the gravitational constant).
The centres of haloes and galaxies are defined as the location of their DM particle with the lowest gravitational potential energy.

\subsection{Kinematic quantities used for structural decomposition}\label{subsec:analysis}
We begin by repositioning the stellar particles of each central galaxy relative to its halo centre. The velocity frame of the galaxy
is at rest with respect to the centre of mass motion of the innermost 80 per cent of its stellar mass. All positions and velocities
from herein refer to these recentred quantities. 

Galaxies are then oriented such that the $z$-axis aligns with their total stellar angular momentum vector, $\vec{J_\star}$, which is
calculated using all stellar particles between 2 and 30 kpc. The lower limit of 2 kpc is imposed to minimise contributions from stellar
particles with disordered motions, or from kinematically decoupled cores\footnote{Although these are rare in \eagle\ \citep{lagos_diverse_2022},
they can impact the measured stellar angular momentum significantly.}, while the upper limit minimises the contribution from particles that
do not belong to the central disc or spheroidal component of the galaxy.

We use the following quantities to decompose \eagle\ galaxies into distinct structural components, which may include a disc, a bulge, and IHL:
\begin{itemize}

\item \jzjc: The specific angular momentum in the $z$-direction relative to the specific angular momentum of a particle on a circular orbit
  with the same binding energy. This quantity was first introduced by \cite{abadi_simulations_2003} to isolate disc stars;
  it is commonly referred to as the orbital circularity parameter. Particles with \jzjc\ values of 1 (-1) are on prograde (retrograde) circular
  orbits in the plane perpendicular to the net angular momentum vector (i.e., in the disc plane of a late type galaxy). 

\item \jpjc: The specific angular momentum in the plane parallel to $\vec{J_\star}$ (i.e., perpendicular to the disc of a late type galaxy),
  normalised by $j_{\rm{circ}}$ (note that in our coordinate system, $j_{\rm p}^2=j_x^2 + j_y^2$ ). This quantity was first introduced by
  \cite{domenech-moral_formation_2012} to aid in identifying disc stars.

\item \emin: The ratio of the specific binding energy of a particle to that of the most bound stellar particle in the galaxy.

\end{itemize}

For a given value of the specific binding energy, the maximum value of the specific angular momentum corresponds to that of a particle on
a prograde circular orbit, which we refer to as $j_{\rm{circ}}$ \citep{abadi_simulations_2003}. We estimate $j_{\rm{circ}}$ numerically using
the orbital information of stellar particles \citep[see also][]{thob_relationship_2019, kumar_galaxy_2021}. Particles are first divided into
150 equally-spaced bins of binding energy. Within each bin, the maximum value of the specific angular momentum of all particles, $j_{\rm max}$,
is taken to be the value of $j_{\rm{circ}}$ for particles within that bin. Due to the finite mass resolution of our simulations
and the diffuse nature of galaxy outskirts, this approach may be inaccurate at large galacto-centric distances, where bins naturally contain
fewer stellar particles than those in the central regions of galaxies. However, we have verified that the results of our galaxy decomposition
are insensitive to reasonable variations in the number of binding energy bins used to calculate $j_{\rm{circ}}$. This is because \jzjc\ is mostly useful for identifying disc particles, which are centrally concentrated and located in regions of a halo that are well sampled by stellar
particles.