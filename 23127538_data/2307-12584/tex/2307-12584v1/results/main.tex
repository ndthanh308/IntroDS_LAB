\section{Results}\label{sec:results}

In this section, we supplement the Ref-L0100N1504 sample with 897 additional galaxies from
50-HiResDM,\footnote{All galaxies in 50-HiResDM have virial masses $M_{200}\geq 10^{11.2}\,{\rm M_\odot}$, and are expected
to be robust to spurious collisional effects at their stellar half mass radii \citep[see][for details]{ludlow_spurious_2023}.}
as well as the 30 galaxy clusters from the C-\eagle~ suite.
This provides us with a sample of 3342 galaxies spanning the mass range 
$10^{11.2}{\rm M}_\odot\lesssim M_{200} \lesssim 10^{15.4}\,{\rm M}_\odot$ (or 
$10^{8.4} {\rm M}_\odot \lesssim M_\star\lesssim 10^{12.9}\,{\rm M}_\odot$ in stellar mass).
Of the full galaxy sample, $\approx 72$ per cent were classified as discs based on our morphological classification step
(see Section~\ref{subsec:decomp_expl}); the remainder as spheroids.
Only about 1.5 per cent of galaxies have no discernible IHL component; these are primarily low-mass, disc-dominated systems. 
All results that follow were obtained from our GMM galaxy decomposition using $n_c=12$.

\subsection{The stellar-to-halo mass relation and morphology}\label{subsec:shmr}

% Figure environment removed

In Fig. \ref{fig:fstar_morph} we plot the stellar-to-halo mass relation (SHMR) for the full galaxy sample,
colour-coded by \fdisk. The median relations obtained from Ref-L0100N1504 (shown as
a solid black line) and 50-HiResDM (thick dashed line) are in good agreement \citep[see also][]{ludlow_spurious_2023}. Fig. \ref{fig:fstar_morph}
also shows that, over the mass range $11.5 \lesssim \log\,(M_{200}/ {\rm M}_\odot) \lesssim 12.5$, there is
a clear relationship between the stellar mass fraction of a galaxy and \fdisk: at fixed halo mass, disc galaxies have higher stellar masses than spheroids. For example,
galaxies with $f_{\rm disc}>0.5$ have, on average, $M_\star\approx 2.1\times 10^{10}\,{\rm M_\odot}$; those with $f_{\rm disc}<0.1$ have
$M_\star\approx 1.4\times 10^{10}\,{\rm M_\odot}$.
These results provide additional dynamical evidence for a morphology-dependent stellar-to-halo mass relation,
consistent with the observational results of \cite{posti_dynamical_2021}, and the theoretical results of
\cite{correa_dependence_2020}, who reported a similar trend for \eagle\ galaxies but using \kappaco\ as a proxy for morphology.

Fig. \ref{fig:fstar_morph} also demonstrates that disc-dominated galaxies tend to occupy haloes 
with masses $M_{200}\lesssim 10^{12.5}\,{\rm M_\odot}$ (including 95 per cent of galaxies with \fdisk$> 0.5$), but above this
mass spheroids dominate (e.g. at $M_{200}\gtrsim 10^{12.5}\,{\rm M_\odot}$, more than
half of the galaxies in our sample have \fdisk$=0$), a trend that
is also well-established observationally \citep[e.g.][]{posti_dynamical_2021}.

\subsection{The SFRs, ex-situ fractions, and ages of discs, bulges and the IHL}\label{subsec:scaling}

Fig. \ref{fig:meds} plots the median star formation rates (SFRs), ex-situ fractions ($f_{{\rm ex-situ}}$),
and stellar half-mass formation times ($t_{50}$), versus $M_{200}$. Results are shown
separately for each galaxy component and for all simulations (Ref-L0100N1504 results are shown using solid lines;
50-HiResDM results using dashed lines; and C-\eagle~ results using crosses). The SFRs were averaged over a
lookback time of 500$\, \rm{Myr}$, but we have verified that our results are qualitatively insensitive to reasonable
variations in that timescale. Note that merger trees are not available for our 50-HiResDM run,
nor for C-\eagle~, so we only present ex-situ fractions for Ref-L0100N1504.

% Figure environment removed

Overall, the results plotted in Fig.~\ref{fig:meds} are in qualitative agreement
with observational expectations and theoretical results: compared to bulges and the IHL,
discs have the highest SFRs, the lowest fractions of ex-situ stars, and are typically
the youngest component of a galaxy \citep[see, e.g.,][for observational evidence of the latter]{robotham_profuse_2022}.
Note that discs have small but non-zero ex-situ fractions, possibly due to the
presence of stellar particles tidally stripped from satellites co-planar with the disc \citep[e.g.][]{abadi_simulations_2003}.

Results from our GMMs suggest that bulges host the oldest stellar
populations and exhibit the lowest $z=0$ SFRs. They have higher ex-situ fractions than discs, but lower ex-situ fractions
than the IHL at fixed mass. The ex-situ fraction of bulges correlates strongly with halo mass,
increasing from about 10 per cent at $M_{200}\approx 10^{12}\,{\rm M_\odot}$ to about $\approx 70$ per cent at
$M_{200}\approx 6\times 10^{13}\,{\rm M_\odot}$; for galaxies hosted by haloes of mass $M_{200}\approx 10^{13}\,{\rm M_\odot}$, roughly half of all their bulge stars were formed ex-situ.

Although the IHL is dominated by ex-situ stars at most masses, our analysis suggests that it also
contains a non-negligible fraction of stars that were born in-situ (roughly 60 per cent at the galactic scale, and
$\approx 10$ per cent at the cluster scale). This goes against a common assumption that the IHL
is comprised entirely of accreted stellar material \citep[e.g.][]{cooper_galactic_2010}.
The IHL is systematically younger, more star-forming, and comprised of more ex-situ stars than bulges, although 
the differences between them become small at mass scales $M_{200} \gtrsim 10^{14}\, \rm{M_\odot}$.
This is in qualitative agreement with observational results indicating that the stellar populations of the ICL and BCGs overlap significantly \citep{jimenez-teja_unveiling_2018}.

The median age of each component increases with increasing $M_{200}$, consistent with the concept
of galaxy ``downsizing'' \citep[e.g.][]{neistein_natural_2006}. 

Finally, note that the median SFRs and half-mass ages of the various galactic components are in good agreement between
Ref-L0100N1504 and 50-HiResDM. The slight differences in the SFRs of the bulge components in these runs are primarily due to the
different galaxy samples that they provide, rather than due to numerical artefacts affecting our decomposition technique.
These results corroborate and extend the findings of \citet{ludlow_spurious_2023}, and suggest that
spurious heating does not affect the SFRs of simulated galaxies, or even the SFRs of their dynamically-distinct components.
This is because, at the mass resolution of our simulations, gaseous baryons are largely unaffected by spurious heating because their
radiative cooling timescale is shorter than their collisional heating timescale \citep{steinmetz_two-body_1997}.

\subsection{The structure of discs, bulges and the IHL}
\subsubsection{The size-mass relation for discs, bulges, and the IHL}
% Figure environment removed


Fig. \ref{fig:r50} plots the $r_{50}-M_{200}$ relations obtained from our simulations. As above, results from
Ref-L0100N1504 are shown using solid lines, results from 50-HiResDM using dashed lines, and C-\eagle~ galaxies using crosses. Different panels
show results separately for discs (left), bulges (middle), and the IHL (right). For comparison, the grey lines in each panel show the size-mass relations
for central galaxies (i.e. for the total stellar component of each system), regardless of their morphological type.

The median disc sizes depend weakly on $M_{200}$ across the mass range plotted and, for most masses,
are well approximated by a power-law of logarithmic
slope $\approx 0.18$. For $M_{200} \lesssim 10^{12.5}\,{\rm M_\odot}$, bulge sizes also exhibit a weak dependence on
$M_{200}$, which gradually steepens at higher masses where spheroids begin dominate our sample.
Like discs, the median sizes of the IHL component are also well
approximated by a single power-law, $r_{50}\propto M_{200}^{0.67}$, over roughly four orders of magnitude in halo mass. Discs are larger than bulges by roughly 0.2 dex at any given mass, in agreement with observational results \citep[e.g.][]{lange_galaxy_2016, robotham_profuse_2022}.

\citet[][and later \citealt{ludlow_spurious_2023}]{navarro_origin_2017} showed that, on the galactic scale, stellar half-mass sizes are well approximated by the simple empirical relation $r_{50}\approx 0.2\times r_s$, where $r_s$ is the scale radius of the galaxy's DM halo. The light blue line with black boundaries plotted in the middle panel of Fig.~\ref{fig:r50} shows\footnote{The different normalisation of the relation plotted in Fig.~\ref{fig:r50} and that advocated by \citet{ludlow_spurious_2023} is likely due to the different mass-concentration relations adopted in our study and theirs. Their $r_s$ values were taken from the empirical model for the mass-concentration relation advocated by \citet{ludlow_mass-concentration-redshift_2016} whereas ours were taken from \citet{schaller_baryon_2015}, which were determined from fits to halo profiles obtained from \eagle.} $r_{50}=0.18\times r_s$, which describes the median sizes of \eagle\, centrals quite well, at least those occupying haloes with $M_{200}\lesssim 10^{14}\,{\rm M_\odot}$. Note, however, that the size-mass relations obtained for any of the individual galaxy components are not well described by this simple relation, nor are they well approximated by a fixed fraction of the halo virial radius \citep[but see][]{kravtsov_size-virial_2013}. 

Finally, note the good agreement between the various size-mass relations obtained from Ref-L0100N1504 and 50-HiResDM.
Although we employed lower limits on $M_{200}$ for these runs such that the effects of spurious heating are negligible at (and above) 
the half-mass radii of {\em all} stellar particles, it is encouraging that good convergence is also obtained for the sizes of dynamically distinct
discs, bulges, and the IHL.

\subsubsection{The IHL transition radius}\label{subsubsec:regions}
% Figure environment removed

At what radius does the IHL begin to dominate the stellar mass of a galaxy? We denote this radius \rtrans\ and explore below how it depends on
halo mass and galaxy morphology.

To compute \rtrans, we first construct spherically-averaged density profiles for the different structural components of each galaxy,
using 36 equally-spaced bins in $\log(r)$ that span the range $-0.5 < \log(r/\rm{kpc}) < 3$. We then interpolate the profiles to determine
the outer radius at which the density of the IHL exceeds the density of the remaining stellar material. This procedure yields $r_{\rm IHL}$
values for $\approx 90$ per cent of our galaxy sample. Galaxies for which $r_{\rm IHL}$ could not be determined are usually low-mass, disc galaxies
with low IHL fractions. For such systems, sampling noise in the stellar halo makes it difficult to determine an accurate value of $r_{\rm IHL}$. For
that reason, in the remainder of this section, we only consider galaxies with $M_{200} \geq 10^{11.5}\, \rm{M}_\odot$, for which $r_{\rm IHL}$ was
reliably determined.

In Fig. \ref{fig:ihl_trans}, we plot \rtrans\ versus $M_{200}$ for galaxies identified in Ref-L0100N1504.
Blue points show results for our sample of disc galaxies, and orange points show spheroids.
The shape of the \rtrans$-\,M_{200}$ relation differs from the size-mass relations of
the individual structural components of galaxies plotted in Fig.~\ref{fig:r50}.
Specifically, below $M_{200} \approx 10^{12.8}\,$\msol, \rtrans\ is largely independent of
$M_{200}$ for both discs and spheroids. There is some evidence that the $r_{\rm IHL}$ of discs is slightly larger than that of
spheroids at fixed $M_{200}$ but, for both morphologies, the IHL begins to dominate the
stellar mass distribution at \rtrans $\approx 30\, \rm{kpc}$ (shown as a horizontal dashed line in Fig.~\ref{fig:ihl_trans}).

Although $r_{\rm IHL}$ marks the radius where the IHL begins to dominate the stellar mass of a galaxy, the typical fraction of the IHL
within that radius can be quite large. For example, roughly 74 (69) per cent of the IHL mass of discs (spheroids)
hosted by haloes with $M_{200}\approx 10^{12}\,{\rm M_\odot}$ lies within $r_{\rm IHL}$. And beyond $r_{\rm IHL}$,
only about 64 percent of the stellar mass is associated with the IHL (for discs; for spheroids it is 93 per cent). Although
the exact values differ as a function of halo mass, our results suggest that there is significant overlap 
in the spatial distribution of the different galaxy components, and that \rtrans\ (or any
other characteristic radius, or fixed spherical aperture) is unlikely
to accurately distinguish the IHL from the other components of a galaxy. 
Defining the IHL as such likely excludes a significant fraction
of the total IHL mass (and includes non-negligible contributions to the IHL from bulge or disc stars)
and potentially biases estimates of IHL properties.

The strong mass dependence of \rtrans\ at $M_{200}\gtrsim 10^{13}\,{\rm M_\odot}$
differs from the findings of \cite{contini_transition_2022},
who found that $r_{\rm IHL} \approx 60\,{\rm kpc}$ at the group and cluster scale.
This discrepancy is perhaps due to the different theoretical approach to the problem (they used a semi-analytic model
that implicitly assumes that the IHL follows a NFW profile, albeit one that is more concentrated than the surrounding DM halo by a factor of 3), but may also be due to 
our different definitions of $r_{\rm IHL}$. Specifically, \cite{contini_transition_2022} identify
$r_{\rm IHL}$ with the radius at which the IHL contributes 90 per cent of the total stellar mass, whereas
in our definition it contributes 50 per cent.
Regardless of this issue, the different mass dependence of \rtrans\ above and below
$M_{200} \approx 10^{13} {\rm M_\odot}$ seen in Fig.~\ref{fig:ihl_trans} indicates that the IHL in groups and clusters
likely differs from the IHL of spheroids and discs at the galaxy scale. A comprehensive analysis of the
structure of the IHL, how it varies with mass, and its relationship to galaxy assembly histories warrants further investigation,
which we defer to future work.
    
\cite{chen_sphere_2022} showed that the IHL transition radii inferred from photometric decomposition of stacked images of
galaxy clusters in the Sloan Digital Sky Survey \citep{rykoff_redmapper_2014} are comparable to
the characteristic scale radii (inferred from weak lensing) of their surrounding DM haloes.
The grey line plotted in
Fig. \ref{fig:ihl_trans} shows the best-fitting $r_s - M_{200}$ relation proposed by \cite{schaller_baryon_2015},
which was obtained by fitting NFW profiles to the average density profiles of relaxed haloes in \eagle. 
At the group and cluster scale, \rtrans\ exhibits a similar dependence on $M_{200}$ 
to that of $r_s$. Furthermore, above $M_{200}\approx 10^{14}\,$\msol, we find  \rtrans$\approx r_s$, although 
with considerable scatter. We did not, however, find a correlation between \rtrans\ and $r_s$ among individual systems, i.e. the scatter in \rtrans\ at fixed $M_{200}$ cannot be attributed to differences in halo concentrations.
The apparent similarity between \rtrans\ and $r_s$ may be coincidental.

Finally, note that the \rtrans\ values we obtain at cluster mass scales are consistent with observational results 
suggesting that the sphere of influence of BCGs can extend to $\approx200\, \rm{kpc}$ \citep{chen_sphere_2022}, which is
significantly larger than the fixed spherical apertures often used in theoretical work \citep[e.g. 30 kpc;][]{montenegro-taborda_growth_2023}.

\subsection{Variation of the IHL fraction with host galaxy properties}\label{subsubsec:fihl}
% Figure environment removed

In the left panel of Fig. \ref{fig:fihl}, we plot \fihl\ versus $M_\star$ and compare our simulation results 
to values obtained from observations of nearby disc galaxies
(observational data were taken from Table 4 of \citealt{harmsen_diverse_2017}; our
simulated galaxies are colour-coded by their disc mass fractions). The MW and M31 are shown using green symbols, and the rest
are plotted in beige (with black edges). For the observed discs, we have included the contribution of the IHL to the total stellar mass to be
consistent with the stellar masses used for our simulated galaxies.

In the right panel of Fig. \ref{fig:fihl}, we plot \fihl\ versus $M_{200}$ and compare with observed IHL fractions
for a few galaxy clusters,\footnote{Observed IHL
fractions at the group and cluster scale are typically expressed as light fractions. Here, we assume a constant mass-to-light ratio
when comparing to our simulation results.} as well as for the MW and M31 (the
virial masses for the MW and M31 were taken from \citealt{shen_mass_2022} and \citealt{fardal_inferring_2013}, respectively;
in this case, our simulated galaxies have been coloured according to the half mass ages of their stellar IHL particles).
Note that at the galaxy cluster scale, it is common for IHL fractions to be expressed
relative to the total stellar mass bound to the halo, inclusive of satellite galaxies. For our analysis,
we only include observational data for which the contribution from satellite galaxies was excluded from the reported IHL fractions.

In both panels, solid and dashed grey lines show the running medians obtained from Ref-L0100N1504 and
50-HiResDM, respectively. The median \fihl\ values are in good agreement between the two simulations at
all mass scales plotted, suggesting that our IHL mass estimates are robust to the spurious heating of stellar particles by
DM particles. For simulated galaxies with no discernible IHL, we instead plot $f_{>100\,\rm{kpc}}$ (grey triangles in both
panels of Fig. \ref{fig:fihl}), which can be interpreted as a lower limit on their IHL fractions (see Fig. \ref{fig:fihl_comp}). 

The left panel of Fig. \ref{fig:fihl} shows that there is some overlap in the
\fihl\ values obtained for our simulated galaxies and from observations (although the simulated galaxies are biased toward
slightly higher \fihl, on average). The IHL fractions of simulated galaxies measured using aperture-based methods
\citep[e.g.][]{pillepich_first_2018, elias_stellar_2018} or 6D phase-space information
\citep[e.g.][]{canas_stellar_2020} are also typically higher than observed IHL fractions.
This suggests that the IHL of simulated galaxies may genuinely outweigh that of observed ones, although
it has been noted that \fihl\ values derived from single-band photometric data (such as those in 
\citealt{merritt_dragonfly_2016}) are likely lower limits on the true IHL fractions \citep{sanderson_reconciling_2018}.
We also stress that the observed IHL fractions plotted in the left panel of Fig. \ref{fig:fihl} correspond to
disc-dominated galaxies, and more closely coincide with the IHL fractions of
disc-dominated galaxies in our simulations (i.e. those corresponding to the blue coloured points).

The left panel of Fig. \ref{fig:fihl} also shows that there is a large diversity in the IHL fractions of simulated
galaxies at fixed stellar mass. For those that span the stellar mass range of the MW and M31
(i.e. the vertical shaded region in the left-hand panel of Fig.
\ref{fig:fihl}), the rms scatter (in $\log\,f_{\rm IHL}$) about the median IHL fraction is about $0.4$ dex, corresponding to a
factor of $\approx 3.2$ in IHL mass. The scatter, however, clearly correlates with galaxy morphology: at fixed $M_\star$, galaxies with
higher IHL fractions tend to have lower disc fractions, and vice versa.

The colour coding of points in the right-hand panel of Fig. \ref{fig:fihl} shows that there is also a correlation
between the IHL mass fraction and its half mass age, $t_{\rm{50, IHL}}$, at least among low-mass galaxies (i.e.
$M_{200}\lesssim 10^{12.5}\,{\rm M_\odot}$).
Note, however, that the correlation between $t_{\rm{50, IHL}}$ and \fihl\ disappears at high masses (see inset
in the right-hand panel Fig.~\ref{fig:fihl}). At the scale of galaxy clusters, the IHL fractions obtained from our simulations approach a
constant value of \fihl$\approx0.45$, in broad agreement with \cite{kluge_photometric_2021}, who finds \fihl$=0.52$. 

The fact that the IHL of disc dominated galaxies tends to be older and less massive
than that of spheroids of similar stellar mass can be interpreted as follows. Galaxies with high disc mass fractions are unlikely to have
experienced recent disruptive mergers that could potentially contribute to the growth of their IHL. As a result, what IHL they
do possess tends to be older than that of spheroidal galaxies, which are more likely to have experienced recent mergers.
This interpretation is supported by the conclusions of \cite{deason_total_2019}, who showed that the low IHL mass of the MW can
be largely explained by an ancient ($\approx 10\, \rm{Gyr}$) merger event with a massive progenitor whose
stellar mass now dominates its IHL.

The diagonal dashed line in the left-hand panel of Fig. \ref{fig:fihl} highlights an IHL fraction corresponding to
roughly 100 stellar particles. Clearly, the stellar haloes of many of the low-mass galaxies in our simulations (i.e. those with
$M_\star\lesssim 10^{10}\,{\rm M_\odot}$) are only resolved by a few 10s to a few 100s of stellar particles. 
Assessing the properties of their IHL components, such as their shape or spatial and kinematic structure, will likely require simulations that reach higher baryonic mass resolution than our runs.
