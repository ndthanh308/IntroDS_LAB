\section{Introduction}\label{sec:intro}

In the `$\Lambda$-cold dark matter' ($\Lambda$CDM) cosmological model, structure forms in a hierarchical manner. 
Galaxies are expected to have undergone numerous accretion events over their lifetime, building up their stellar
mass at least in part through mergers with lower-mass ones. In principle, these accretion events should leave an
observable imprint on present-day galaxies in the form of an extended stellar component: the intra-halo light
\citep[IHL; see e.g.][]{purcell_shredded_2007}. The ubiquity of merger events in the $\Lambda$CDM model implies
that the IHL should be present, to some extent, in most galaxies \citep{bullock_tracing_2005}. Indeed, the IHL has
been observed across a broad range of galaxy masses; it is typically referred to as a stellar halo at the galactic
scale \citep[see][for a review of the Milky Way's, MWs, stellar halo]{helmi_stellar_2008}, and as intra-group (cluster)
light at the galaxy group (cluster) scale \citep[see][for reviews]{contini_origin_2021, montes_new_2022}.

Our understanding of the mass fraction, spatial distribution and stellar populations of the IHL, as well as how they
vary with host halo mass, is limited, partly due to difficulties defining the IHL unambiguously
\citep[see e.g.][and references therein]{sanderson_reconciling_2018}. Nonetheless, there is mounting evidence hinting
at a connection between present day IHL properties and the formation of its host galaxy and dark matter (DM) halo.
In the case of the MW, observations of halo stars \citep[e.g., by the][]{gaia_collaboration_gaia_2021}
have revolutionised our understanding of the Galaxy's formation history through the discovery of distinct
substructures in 6D phase space \citep[e.g., the Gaia Enceladus/Sausage;][]{belokurov_co-formation_2018,helmi_merger_2018}
or distinct chemodynamical sub-populations \citep[e.g.][]{kruijssen_formation_2019, horta_evidence_2020, buder_galah_2022},
which are the likely remnants of destroyed satellites that merged with the MW early in its formation. At the galaxy cluster
scale, \cite{deason_stellar_2021} showed that the stellar density profiles of simulated clusters
exhibit a well-defined edge, coincident with the `splashback' radius of the underlying DM halo
\citep[e.g.][]{adhikari_splashback_2014, diemer_splashback_2017}. This feature has since been detected observationally
\citep{gonzalez_discovery_2021}, indicating that the IHL may provide an observable probe of the underlying DM distribution.

Despite recent observational progress, accurately characterising the IHL for a representative sample of extragalactic galaxies
remains challenging. Surveys based on resolved stellar populations provide a wealth of information on halo stars, but these
measurements can only be obtained for nearby systems \citep[e.g.][]{barker_resolving_2009, ibata_large-scale_2013, harmsen_diverse_2017}.
Deep integrated light surveys provide an alternative approach to studying the IHL of extragalactic systems, where the surface
brightness profiles of galaxies can be decomposed based on assumptions about the analytic form appropriate for specific galactic components.
\cite{merritt_dragonfly_2016} measured the IHL of 8 nearby disc galaxies by decomposing their stellar mass surface density
profiles, finding an RMS scatter in the IHL mass fraction of approximately 1 dex, indicating that the IHL fraction varies
significantly, even for galaxies with similar morphologies and stellar masses. The IHL fractions of galaxy groups and clusters
at $z\approx0$ also exhibit significant scatter, with measured fractions ranging from a few per cent to $\approx 50$ per cent
\citep[e.g.][]{montes_faint_2022}. It is unclear how much of this scatter is inherently physical in origin
\citep[due to, for example, stochastic variations in formation histories, e.g.][]{fattahi_tale_2020, rey_how_2022}, and how much
is a result of the inconsistent methodologies adopted by different observational studies \citep[see, e.g.,][]{kluge_photometric_2021}.

The outlook for characterising the IHL consistently across different environments is perhaps more promising in cosmological,
hydrodynamical simulations, in which the various structural components of galaxies can in principle be identified from the
kinematics of their stellar particles. In particular, understanding the formation mechanisms of the IHL and how they vary as a
function of mass can be addressed with large volume cosmological simulations, where galaxies that form in a broad range of
environments can be studied and tracked over time \citep[e.g.][]{schaye_eagle_2015, pillepich_first_2018, dave_simba_2019}.

\cite{canas_stellar_2020} introduced an adaptive phase-space algorithm, specifically designed to distinguish kinematically hot stellar
particles (which they identified with the IHL) from centrally concentrated stellar structures and applied it to
galaxies identified in the Horizon-AGN simulation \citep{dubois_horizon-agn_2016}. This allowed them to study the mass-dependence
of the IHL fractions of a diverse population of galaxies, which revealed that the scatter in IHL masses is correlated with kinematic galaxy
morphology. While their method was effective in identifying the IHL of galaxies in diverse environments, their results depend on a
free parameter that had to be calibrated to a small number of uncertain observations. 

Unsupervised machine learning techniques provide a viable alternative for identifying kinematically distinct
structures within simulated galaxies \citep[e.g.][]{domenech-moral_formation_2012}. \cite{obreja_nihao_2016} showed that Gaussian
Mixture Models (GMMs) can be applied to the stellar components of galaxies to identify discs, bulges, and stellar haloes, whose
properties resemble those of observed systems. Automating structural decomposition for a diverse galaxy sample is, however,
non-trivial. As a result, studies are typically limited to small samples of galaxies, for which the clusters of stars identified by
the GMMs can be manually assigned to different galactic components \citep{obreja_introducing_2018, obreja_nihao_2019}, or they are
limited to galaxies within a narrow range of mass or morphology (e.g. \citealt{ortega-martinez_milky_2022}, though see
\citealt{du_identifying_2019}).

In this work, we use GMMs to characterise the disc, bulge and IHL components of simulated galaxies across a wide range of halo masses.
We apply our methodology to galaxies from three simulations based on the \eagle\ model of galaxy formation, allowing us to study how
the mass fraction, stellar populations and structure of these components vary from the galactic scale up to the scale of massive galaxy
clusters. The remainder of this paper is organised as follows. In Section \ref{sec:sims}, we describe the \eagle\ simulations and
introduce the kinematic quantities used in our structural decomposition. In Section \ref{sec:decomp}, we introduce our decomposition
technique and compare our results to alternative techniques. In Section \ref{sec:results}, we analyse the stellar populations and
structural properties of the disc, bulge, and IHL components; and in Section \ref{sec:conc} we summarise our results.

