\documentclass{letter}

\usepackage{amssymb}
\usepackage{fullpage}

\pagestyle{empty}

\signature{\vspace{-.25in}Jackson Loper (corresponding author), Noam Solomon, Jeffrey Regier}

\address{Jackson Loper \\ Department of Statistics \\  West Hall \\ 1085 S University Ave \\ Ann Arbor, MI 48109}

\begin{document}

\begin{letter}{Editorial Board \\ Journal of the American Statistical Association
}

\opening{To whom it may concern,}

Thank you for considering our manuscript, titled ``Improving Accuracy in Cell-Perturbation Experiments by Leveraging Auxiliary Information'' for publication in the Journal of the American Statistical Association (Theory and Methods).  

This work develops new statistical methodology to analyze cell-perturbation experiments.  These experiments are a fundamental tool for assessing how various treatments affect various gene in various populations of cells.  We propose two distinct contributions.  First, we propose a new estimator for Average Treatment Effects (ATEs) in this context.   The new estimator is designed using ideas from factor analysis and a Gaussian process model.  The Gaussian process allows us to leverage known side-information the similarities between treatments.  Second, we propose a method for assessing the quality of different ATE estimators.   Assessing quality is difficult to assess in this context because ground truth about ATEs is unavailable.  Critically, the new method can be used to assess estimators regardless of whether they are based in correctly specified models.  Together, we believe these two contributions offer a promising path forwards to obtaining more accurate ATEs from these costly and important experiments.  

Thank you for your time and effort in considering this paper.

\vspace{1in}

\closing{Sincerely,}

\end{letter}

\end{document}
