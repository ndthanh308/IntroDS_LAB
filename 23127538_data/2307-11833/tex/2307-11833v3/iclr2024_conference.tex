
\documentclass{article} % For LaTeX2e
\usepackage{iclr2024_conference,times}

% Optional math commands from https://github.com/goodfeli/dlbook_notation.
%%%%% NEW MATH DEFINITIONS %%%%%
\newtheorem{property}{Property}
\newtheorem{definition}{Definition}
\newtheorem{theorem}{Theorem}
\newtheorem{lemma}{Lemma}
\newtheorem{corollary}{Corollary}
\DeclarePairedDelimiter\abs{\lvert}{\rvert}
\DeclarePairedDelimiter\norm{\lVert}{\rVert}
\makeatletter
\let\oldabs\abs
\def\abs{\@ifstar{\oldabs}{\oldabs*}}
\let\oldnorm\norm
\def\norm{\@ifstar{\oldnorm}{\oldnorm*}}
\makeatother

% Mark sections of captions for referring to divisions of figures
\newcommand{\figleft}{{\em (Left) }}
\newcommand{\figcenter}{{\em (Center) }}
\newcommand{\figright}{{\em (Right)}}
\newcommand{\figtop}{{\em (Top) }}
\newcommand{\figbottom}{{\em (Bottom) }}
\newcommand{\captiona}{{\em (a) }}
\newcommand{\captionb}{{\em (b) }}
\newcommand{\captionc}{{\em (c) }}
\newcommand{\captiond}{{\em (d) }}

% Highlight a newly defined term
\newcommand{\newterm}[1]{{\bf #1}}


\def\figref#1{figure~\ref{#1}}
\def\Figref#1{Figure~\ref{#1}}
\def\twofigref#1#2{figures \ref{#1} and \ref{#2}}
\def\quadfigref#1#2#3#4{figures \ref{#1}, \ref{#2}, \ref{#3} and \ref{#4}}
\def\secref#1{section~\ref{#1}}
\def\Secref#1{Section~\ref{#1}}
\def\twosecrefs#1#2{sections \ref{#1} and \ref{#2}}
\def\secrefs#1#2#3{sections \ref{#1}, \ref{#2} and \ref{#3}}
\def\eqref#1{equation~\ref{#1}}
\def\Eqref#1{Equation~\ref{#1}}
% A raw reference to an equation---avoid using if possible
\def\plaineqref#1{\ref{#1}}
% Reference to a chapter, lower-case.
\def\chapref#1{chapter~\ref{#1}}
% Reference to an equation, upper case.
\def\Chapref#1{Chapter~\ref{#1}}
% Reference to a range of chapters
\def\rangechapref#1#2{chapters\ref{#1}--\ref{#2}}
% Reference to an algorithm, lower-case.
\def\algref#1{algorithm~\ref{#1}}
% Reference to an algorithm, upper case.
\def\Algref#1{Algorithm~\ref{#1}}
\def\twoalgref#1#2{algorithms \ref{#1} and \ref{#2}}
\def\Twoalgref#1#2{Algorithms \ref{#1} and \ref{#2}}
% Reference to a part, lower case
\def\partref#1{part~\ref{#1}}
% Reference to a part, upper case
\def\Partref#1{Part~\ref{#1}}
\def\twopartref#1#2{parts \ref{#1} and \ref{#2}}

% Random variables
\def\reta{{\textnormal{$\eta$}}}
\def\ra{{\textnormal{a}}}

% Random vectors
\def\rvepsilon{{\mathbf{\epsilon}}}
\def\rvtheta{{\mathbf{\theta}}}
\def\rva{{\mathbf{a}}}

% Elements of random vectors
\def\erva{{\textnormal{a}}}
\def\ervb{{\textnormal{b}}}

% Random matrices
\def\rmA{{\mathbf{A}}}
\def\rmB{{\mathbf{B}}}

% Elements of random matrices
\def\ermA{{\textnormal{A}}}
\def\ermB{{\textnormal{B}}}

\def\fvec{{\mathbf{f}}}
\def\bff{{\mathbf{f}}}
\def\bfg{{\mathbf{g}}}
% Vectors
\def\vzero{{\bm{0}}}
\def\vone{{\bm{1}}}
\def\vmu{{\bm{\mu}}}
\def\vtheta{{\bm{\theta}}}
\def\va{{\bm{a}}}
\def\vb{{\bm{b}}}
\def\vc{{\bm{c}}}
\def\vd{{\bm{d}}}
\def\ve{{\bm{e}}}
\def\vf{{\bm{f}}}
\def\vg{{\bm{g}}}
\def\vh{{\bm{h}}}
\def\vi{{\bm{i}}}
\def\vj{{\bm{j}}}
\def\vk{{\bm{k}}}
\def\vl{{\bm{l}}}
\def\vm{{\bm{m}}}
\def\vn{{\bm{n}}}
\def\vo{{\bm{o}}}
\def\vp{{\bm{p}}}
\def\vq{{\bm{q}}}
\def\vr{{\bm{r}}}
\def\vs{{\bm{s}}}
\def\vt{{\bm{t}}}
\def\vu{{\bm{u}}}
\def\vv{{\bm{v}}}
\def\vw{{\bm{w}}}
\def\vx{{\bm{x}}}
\def\vy{{\bm{y}}}
\def\vz{{\bm{z}}}

% Matrix
\def\mA{{\bm{A}}}

% Tensor
\DeclareMathAlphabet{\mathsfit}{\encodingdefault}{\sfdefault}{m}{sl}
\SetMathAlphabet{\mathsfit}{bold}{\encodingdefault}{\sfdefault}{bx}{n}
\newcommand{\tens}[1]{\bm{\mathsfit{#1}}}
\def\tA{{\tens{A}}}
\def\tB{{\tens{B}}}
\def\tC{{\tens{C}}}
\def\tD{{\tens{D}}}
\def\tE{{\tens{E}}}
\def\tF{{\tens{F}}}
\def\tG{{\tens{G}}}
\def\tH{{\tens{H}}}
\def\tI{{\tens{I}}}
\def\tJ{{\tens{J}}}
\def\tK{{\tens{K}}}
\def\tL{{\tens{L}}}
\def\tM{{\tens{M}}}
\def\tN{{\tens{N}}}
\def\tO{{\tens{O}}}
\def\tP{{\tens{P}}}
\def\tQ{{\tens{Q}}}
\def\tR{{\tens{R}}}
\def\tS{{\tens{S}}}
\def\tT{{\tens{T}}}
\def\tU{{\tens{U}}}
\def\tV{{\tens{V}}}
\def\tW{{\tens{W}}}
\def\tX{{\tens{X}}}
\def\tY{{\tens{Y}}}
\def\tZ{{\tens{Z}}}


% Graph
\def\gA{{\mathcal{A}}}
\def\gB{{\mathcal{B}}}
\def\gC{{\mathcal{C}}}
\def\dataset{{\mathcal{D}}}
\def\gE{{\mathcal{E}}}
\def\gF{{\mathcal{F}}}
\def\fourier{{\mathcal{F}}}
\def\gG{{\mathcal{G}}}
\def\gH{{\mathcal{H}}}
\def\gI{{\mathcal{I}}}
\def\gJ{{\mathcal{J}}}
\def\gK{{\mathcal{K}}}
\def\gL{{\mathcal{L}}}
\def\loss{{\mathcal{L}}}
\def\gM{{\mathcal{M}}}
\def\gN{{\mathcal{N}}}
\def\normal{{\mathcal{N}}}
\def\gaussian{{\mathcal{N}}}
\def\gO{{\mathcal{O}}}
\def\gP{{\mathcal{P}}}
\def\gQ{{\mathcal{Q}}}
\def\gR{{\mathcal{R}}}
\def\gS{{\mathcal{S}}}
\def\gT{{\mathcal{T}}}
\def\gU{{\mathcal{U}}}
\def\uniform{{\mathcal{U}}}
\def\gV{{\mathcal{V}}}
\def\gW{{\mathcal{W}}}
\def\gX{{\mathcal{X}}}
\def\gY{{\mathcal{Y}}}
\def\gZ{{\mathcal{Z}}}

\def\algebra{{\mathscr{A}}}
\def\borel{{\mathscr{B}}}
\def\manifold{{\mathscr{M}}}

% Sets
\def\sA{{\mathbb{A}}}
\def\sB{{\mathbb{B}}}
\def\complex{{\mathbb{C}}}
\def\sD{{\mathbb{D}}}
\def\expectation{{\mathbb{E}}}
\newcommand{\E}{\mathbb{E}}
\def\sF{{\mathbb{F}}}
\def\sG{{\mathbb{G}}}
\def\sH{{\mathbb{H}}}
\def\sI{{\mathbb{I}}}
\def\sJ{{\mathbb{J}}}
\def\sK{{\mathbb{K}}}
\def\sL{{\mathbb{L}}}
\def\sM{{\mathbb{M}}}
\def\natural{{\mathbb{N}}}
\def\sO{{\mathbb{O}}}
\def\sP{{\mathbb{P}}}
\def\rational{{\mathbb{Q}}}
\def\real{{\mathbb{R}}}
\newcommand{\R}{\mathbb{R}}
\def\sS{{\mathbb{S}}}
\def\sphere{{\mathbb{S}}}
\def\sT{{\mathbb{T}}}
\def\sU{{\mathbb{U}}}
\def\sV{{\mathbb{V}}}
\def\sW{{\mathbb{W}}}
\def\sX{{\mathbb{X}}}
\def\sY{{\mathbb{Y}}}
\def\integer{{\mathbb{Z}}}
\def\indicator{{\mathbbm{1}}}

% Entries of a matrix
\def\emLambda{{\Lambda}}
\def\emA{{A}}
\def\emB{{B}}
\def\emC{{C}}
\def\emD{{D}}
\def\emE{{E}}
\def\emF{{F}}
\def\emG{{G}}
\def\emH{{H}}
\def\emI{{I}}
\def\emJ{{J}}
\def\emK{{K}}
\def\emL{{L}}
\def\emM{{M}}
\def\emN{{N}}
\def\emO{{O}}
\def\emP{{P}}
\def\emQ{{Q}}
\def\emR{{R}}
\def\emS{{S}}
\def\emT{{T}}
\def\emU{{U}}
\def\emV{{V}}
\def\emW{{W}}
\def\emX{{X}}
\def\emY{{Y}}
\def\emZ{{Z}}
\def\emSigma{{\Sigma}}

% entries of a tensor
% Same font as tensor, without \bm wrapper
\newcommand{\etens}[1]{\mathsfit{#1}}
\def\etLambda{{\etens{\Lambda}}}
\def\etA{{\etens{A}}}
\def\etB{{\etens{B}}}
\def\etC{{\etens{C}}}
\def\etD{{\etens{D}}}
\def\etE{{\etens{E}}}
\def\etF{{\etens{F}}}
\def\etG{{\etens{G}}}
\def\etH{{\etens{H}}}
\def\etI{{\etens{I}}}
\def\etJ{{\etens{J}}}
\def\etK{{\etens{K}}}
\def\etL{{\etens{L}}}
\def\etM{{\etens{M}}}
\def\etN{{\etens{N}}}
\def\etO{{\etens{O}}}
\def\etP{{\etens{P}}}
\def\etQ{{\etens{Q}}}
\def\etR{{\etens{R}}}
\def\etS{{\etens{S}}}
\def\etT{{\etens{T}}}
\def\etU{{\etens{U}}}
\def\etV{{\etens{V}}}
\def\etW{{\etens{W}}}
\def\etX{{\etens{X}}}
\def\etY{{\etens{Y}}}
\def\etZ{{\etens{Z}}}

\def\ceil#1{\lceil #1 \rceil}
\def\floor#1{\lfloor #1 \rfloor}
\def\eps{{\epsilon}}

\newcommand{\pder}[1]{\frac{\partial}{\partial #1}}

\newcommand{\half}{\frac{1}{2}}
\newcommand{\limNinf}{\lim_{N \to \infty}}
\newcommand{\limTzero}{\lim_{\tau \to 0}}


\newcommand{\cmark}{\ding{51}}
\newcommand{\xmark}{\ding{55}}

\newcommand{\layer}{\mathcal{H}}
\newcommand{\defeq}{\triangleq}
%\newcommand{\defeq}{vcentcolon=}
\newcommand{\domain}{\Omega}
\newcommand{\grad}{\nabla}

\newcommand{\cin}{c_{\rm{in}}}
\newcommand{\cout}{c_{\rm{out}}}
\newcommand{\intdomain}{\int_{\domain}}
\newcommand{\network}{\gT}
\newcommand{\subnet}{\gK}
\newcommand{\map}{\gR} %\gR

\newcommand{\innerproduct}[2]{\langle #1, #2 \rangle}
\newcommand{\mcsum}[1][j]{\frac{1}{N}\sum_{#1=1}^N}

\newcommand{\inrspace}[1][c]{\gF_{#1}}

\DeclareMathOperator*{\argmax}{arg\,max}
\DeclareMathOperator*{\argmin}{arg\,min}

\let\ab\allowbreak


\usepackage{hyperref}
\usepackage{url}

\usepackage{graphicx}
\usepackage{mathtools}
\usepackage{amsmath}
\usepackage{amssymb}
\usepackage{wrapfig}
\usepackage{multirow}
\usepackage{subfigure}
\usepackage{subcaption} 
\usepackage{mdwlist}
\usepackage{xspace}

\newcommand{\ourmethod}{\textrm{PINNsFormer}\xspace}
\newcommand{\aditya}[1]{\textbf{Aditya says: #1 }}
\newcommand{\rev}[1]{\textcolor{red}{\textbf{#1}}}
% \newcommand{\ourmethod}{\texttt{\textit{PINNsFormer}}\xspace}

% \usepackage{amsmath}
% \usepackage{amssymb}
% \usepackage{multirow}
% \usepackage{siunitx}
% \usepackage{graphicx}
% \usepackage{mathtools}
% \usepackage{wrapfig}
% \usepackage{subcaption}

% \usepackage{hyperref}       % hyperlinks
% \usepackage{url}            % simple URL typesetting
% \usepackage{booktabs}       % professional-quality tables
% \usepackage{amsfonts}       % blackboard math symbols
% \usepackage{nicefrac}       % compact symbols for 1/2, etc.
% \usepackage{microtype}      % microtypography
% \usepackage{xcolor}         % colors

\newtheorem{theorem}{Theorem}
\newtheorem{proposition}{Proposition}


\title{PINNsFormer: A Transformer-Based Framework For Physics-Informed Neural Networks}

% Authors must not appear in the submitted version. They should be hidden
% as long as the \iclrfinalcopy macro remains commented out below.
% Non-anonymous submissions will be rejected without review.

\author{Zhiyuan Zhao\\
  Georgia Institute of Technology\\
  Atlanta, GA 30332 \\
  \texttt{leozhao1997@gatech.edu} \\
  \And
  Xueying Ding \\
  Carnegie Mellon University \\
  Pittsburgh, PA 15213\\
  \texttt{xding2@andrew.cmu.edu} \\
  \AND
  B. Aditya Prakash \\
  Georgia Institute of Technology \\
  Atlanta, GA 30332 \\
  \texttt{badityap@cc.gatech.edu} \\
}

% The \author macro works with any number of authors. There are two commands
% used to separate the names and addresses of multiple authors: \And and \AND.
%
% Using \And between authors leaves it to \LaTeX{} to determine where to break
% the lines. Using \AND forces a linebreak at that point. So, if \LaTeX{}
% puts 3 of 4 authors names on the first line, and the last on the second
% line, try using \AND instead of \And before the third author name.

\newcommand{\fix}{\marginpar{FIX}}
\newcommand{\new}{\marginpar{NEW}}

\iclrfinalcopy % Uncomment for camera-ready version, but NOT for submission.
\begin{document}


\maketitle

\begin{abstract}
Physics-Informed Neural Networks (PINNs) have emerged as a promising deep learning framework for approximating numerical solutions to partial differential equations (PDEs). However, conventional PINNs, relying on multilayer perceptrons (MLP), neglect the crucial temporal dependencies inherent in practical physics systems and thus fail to propagate the initial condition constraints globally and accurately capture the true solutions under various scenarios. In this paper, we introduce a novel Transformer-based framework, termed \ourmethod, designed to address this limitation. \ourmethod can accurately approximate PDE solutions by utilizing multi-head attention mechanisms to capture temporal dependencies. \ourmethod transforms point-wise inputs into pseudo sequences and replaces point-wise PINNs loss with a sequential loss. Additionally, it incorporates a novel activation function, \texttt{Wavelet}, which anticipates Fourier decomposition through deep neural networks. Empirical results demonstrate that \ourmethod achieves superior generalization ability and accuracy across various scenarios, including PINNs failure modes and high-dimensional PDEs. Moreover, \ourmethod offers flexibility in integrating existing learning schemes for PINNs, further enhancing its performance.
\end{abstract}

% \section{Submission of conference papers to ICLR 2024}
% !TEX root = ../AttackGraphBasedRiskAnalysis.tex
% !TEX spellcheck = en_US
% !TEX encoding = UTF-8 Unicode

\section{Introduction}\label{sec: intro}

Traditional cities are becoming smarter. 
One of the core smart city concepts is smart mobility, which has attracted considerable attention from security researchers due to the emergence of smart vehicles and V2X communication that have given rise to novel cybersecurity threats.

Over the last decade, several trends have contributed to the automotive and railway threat landscape. 
First, sophisticated features in smart vehicles come with a higher volume of lines of code, aggravating testability and auditing and increasing the likelihood and severity of vulnerabilities. 
Second, (wireless) communication interfaces in smart vehicles come with a higher volume of external peripheral devices that can connect to smart vehicles, hence increasing the attackers' access point options, and also with a higher volume of connections, hence increasing the risk of malicious interactions. 
Finally, a higher volume of connections between smart vehicles comes with a higher volume of exchanged data, which in most cases is personal and, therefore, immensely valuable. In other words, more data is generated and needs to be considered and protected.

Graphical security modeling is a widely-used and well-established approach for representing and analyzing threat landscapes that examine vulnerabilities of systems and organizations. 
One of the primary strengths of graphical security models is that they allow for the inclusion of user-friendly visual elements with formal semantics and algorithms, enabling both qualitative and quantitative analyses. 
Over the last couple of decades, security researchers have been progressively focusing on graphical security modeling, which has gradually evolved into a valuable tool for the assessment of risks in real-life systems, such as automotive and railway environments.

Threat landscapes include (1) malicious actions of an attacker, whose goal is to harm or damage one or more assets of a system or organization, and (2) countermeasures for either preventing or mitigating such malicious actions. 
The first \emph{tree-based approach} for graphical security modeling was the \emph{threat logic trees}, which was introduced by Weiss in 1991~\cite{weiss1991}, thereby motivating the development of several subsequent frameworks, such as attack trees, which are still considered one of the most important and favored tools for the assessment of risks to date.

In all tree-based approaches, the modeling process begins with identifying a feared event, which is shown as a root node, and continues with the refinement of the attack steps, resulting in a tree model.
However, tree structures are limited to only one path between a pair of nodes. 
In other words, with tree structures, each refined node can only have one parent node. 
This limitation is addressed by the \emph{directed acyclic graph (DAG) structure}, which enables refined nodes to have multiple parent nodes. 
As a result, DAG structures can provide a higher level of detail, but they can also come with a higher level of complexity, which can nevertheless be dealt with modularization, thereby allowing the model to be subdivided into loosely-coupled, independent, and interchangeable parts that can be studied individually and in parallel. 
Finally, while the one-to-many relationship between nodes in tree structures results in a linear analysis of the threat landscape, the many-to-many relationship between nodes in DAG structures can theoretically result in an exponential analysis.
However, the complexity is kept small in practice due to the acyclic structure, and the threat landscape analysis is eventually possible.

Ensuring the security of systems is not a static process that is over after going through once.
The conditions are constantly changing, on the one hand attackers and their capabilities are evolving, and on the other hand, systems themselves are being extended and evolving.
To effectively perform the necessary continuous security management, it is necessary to know not just the threat landscape but to be able to understand the consequences and impacts if attacks are performed successfully.
Hence, it is necessary to continuously perform a risk analysis to identify the potential exposure.
Nowadays, risk management is primarily done using large tables filled with a lot of information and use cases.
Large tables only offer limited visibility, as it is challenging to maintain a comprehensive overview of risks.
With numerous rows and columns, it becomes difficult to identify trends and patterns or prioritize risks effectively.
Furthermore, managing risk can be a tedious and time-consuming process.
Updating and maintaining tables with evolving risks and mitigation measures can require significant effort, especially when dealing with a complex system or multiple risk factors.
This gets even harder when dealing with large tables that often fail to provide the necessary context and connections between different risks.
Additionally, analyzing and interpreting data from large tables can be daunting. 
It may require specialized tools or skills to extract meaningful insights from the extensive amount of information presented in the table format.
Large tables may further lack the flexibility to accommodate changing risk scenarios or evolving requirements. 
Modifying or updating the table structure to incorporate new risks or factors can be cumbersome and may hinder agility in risk management.
With numerous cells and data entries, there is also an increased risk of errors, inaccuracies, or inconsistencies in the large table. 
These issues can undermine the reliability and integrity of the risk management process.

We propose a graphical solution for the risk management process to mitigate these disadvantages of tables.
A visual representation can enhance the understanding and communication of complex risk information and make it easier to identify patterns, trends, and relationships among risks, facilitating effective decision-making.
Complex risk data is further simplified by presenting it in a clear and concise manner.
Understanding  the relationships, dependencies, and interactions between various risk elements is necessary to understand the overall risk landscape.
Visual representations of the entire risk landscape provide this overview, allowing for the identification of interdependencies, hotspots, or areas of high vulnerability.
Graphical solutions can also aid in developing and evaluating risk mitigation strategies. 
By visually representing the potential consequences and effectiveness of different mitigation measures, decision-makers can make more informed choices and allocate resources more efficiently.
Furthermore, it allows for the exploration of different risk scenarios. 
By manipulating variables or parameters within the visual representation, it becomes possible to assess the potential impact of various risk factors and evaluate the effectiveness of different response strategies.
Additionally, as graphical solutions can be more adaptable to changing requirements and evolving risks, they allow for easier updates and modifications, enabling risk management processes to be more responsive and agile.

Consequently, we believe a graphical solution for the risk assessment process improves the maintenance of risk scenarios and facilitates accessibility to different stakeholders, including non-technical audiences.
However, the existing graphical solutions are momentarily used to describe the threat landscape.
Which, of course, is helpful for the risk management process but not sufficient to represent the entire risk management process.
Therefore, motivating us to define a new graphical method for risk assessment by extending existing graphical methods for depicting the threat landscape.
Besides ways to depict attack vectors, their probability, and countermeasures, our method includes a way to depict the consequences of attack vectors and the impact level, enabling us to calculate a risk value.

The remainder of the paper is structured as follows:
After the introduction,~\cref{sec: related work} discusses the related work.
Our definition of attack graphs is given in~\cref{sec: attack graphs}.
The necessary adjustments to use these attack graphs are presented in~\cref{sec: attack graph risk assessment}, including an example of how the risk assessment is performed in our project.
\cref{sec: applicability of attack graphs to risk management standards} validates our defined method by combing it with the risk assessment processes of ISO/SAE 21434~\cite{21434} and CLC/TS 50701~\cite{50701} respectively.
The scalability and practicality are evaluated in~\cref{sec: evaluation}.
Finally,~\cref{sec: conclusion} concludes this paper.
% !TEX root = ../AttackGraphBasedRiskAnalysis.tex
% !TEX spellcheck = en_US
% !TEX encoding = UTF-8 Unicode

\section{Related Work}\label{sec: related work}
%\todo{Discuss all frameworks regarding consequences.}

Kordy et al.~\cite{DAGpaper} categorize thirty-three frameworks for graphical analysis of attack and defense scenarios into (1) \emph{attack and/or defense modeling}, which focus on the formal aspects of attacks or defenses, and (2) \emph{static or sequential modeling}, which focus on the temporal aspects or dependencies between actions. 
Using the same categorization, this section provides an overview of all the frameworks, and it describes these frameworks that fulfill the majority of properties incorporated in the framework of this article.

By reviewing frameworks from current literature, we identify seven properties for graphically modeling and managing an entire risk landscape.
The first property is \emph{attack vectors}, which enables the relations (shown as edges) between attack steps (shown as nodes) and, therefore, the formation of attack paths (i.e., attack vectors). 
The second property is the \emph{directed acyclic graph (DAG) structure}, thereby enabling linear (i.e., directed) and finite (i.e., acyclic) series of attack steps towards multiple potential attack goals (i.e., graph). 
The third property is \emph{node attributes}, which enables the quantification and, therefore, the evaluation of attack steps. 
The fourth property is \emph{dynamic connectors}, thereby enabling extensive attack refinements (besides the basic AND-OR refinements). 
The fifth property is \emph{edge attributes}, which enables the quantification and, therefore, the evaluation of relations between attack steps. 
The sixth property is \emph{countermeasure nodes}, thereby enabling actions to reduce the negative consequences of attacks.
The final property is \emph{consequence nodes}, enabling the presentation of consequences of successful attacks, which is also necessary to constitute the impact.

\begin{table*}[h]
\rowcolors{2}{gray!10}{gray!40}
\renewcommand{\arraystretch}{1.2}
\caption{Static attack modeling frameworks compared to the seven defined properties.}
\label{tab: static attack modeling}
\noindent\makebox[\textwidth]{%
\begin{tabular}[t]{>{\raggedright}p{0.15\textwidth}>{\raggedright}p{0.06\textwidth}>{\raggedright}p{0.08\textwidth}>{\raggedright}p{0.1\textwidth}>{\raggedright}p{0.09\textwidth}>{\raggedright\arraybackslash}p{0.08\textwidth}>{\raggedright\arraybackslash}p{0.15\textwidth}>{\raggedright\arraybackslash}p{0.1\textwidth}}
\toprule
 & Attack Vectors & DAG Structure & Node Attributes & Dynamic Connectors & Edge Attributes & Countermeasure Nodes & Consequence Nodes
\tabularnewline
\midrule
% \textbf{Static Attack Modeling} & & & & &
% \tabularnewline
Attack Trees & \checkmark & - & (\checkmark) & - & - & - & -
\tabularnewline
Augmented Vulnerability Trees & \checkmark & - & (\checkmark) & - & - & -  & -
\tabularnewline
Augmented Attack Trees & \checkmark & - & (\checkmark) & - & - & -  & -
\tabularnewline
OWA Trees & \checkmark & - & - & (\checkmark)  & (\checkmark) & -  & -
\tabularnewline
Parallel Model for Multi-Parameter Attack Trees & \checkmark & - & (\checkmark) & - & - & -  & -
\tabularnewline
Extended Fault Trees & \checkmark & - & (\checkmark) & - & - & -  & -
\tabularnewline
\bottomrule
\end{tabular}}
\end{table*}

Each one of the thirty-three frameworks presented in this section considers only subsets of the seven identified properties. 
None of these frameworks are suitable, as all seven properties are necessary to perform a full risk assessment.
To overcome this limitation, this article incorporates all seven identified properties into a framework for a graphical solution for performing risk analysis and examines its applicability to different risk analysis standards.

\subsection{Static Attack Modeling}\label{sec: static attack modeling}




Six frameworks for \emph{static attack modeling}, namely \emph{Attack Trees}~\cite{weiss1991}, \emph{Augmented Vulnerability Trees}~\cite{AugmentedVulnerabilityTrees}, \emph{Augmented Attack Trees}~\cite{AugmentedAttackTrees}, \emph{OWA Trees}~\cite{Yager2006OWATA}, \emph{Parallel Model for Multi-Parameter Attack Trees}~\cite{ParallelModelForMultiParameterAttackTrees}, and \emph{Extended Fault Trees}~\cite{ExtendedFaultTrees}, are summarised in Table~\ref{tab: static attack modeling}.
All frameworks fulfill the attack vectors property, but none of them supports the DAG structure, countermeasure nodes, and consequence nodes properties. 
Of the six frameworks, OWA trees stand out as they at least partially fulfill the dynamic connectors and edge attributes properties, despite being the only framework that does not fulfill the node attributes property. 
This section describes attack trees, which was the first graphical security modeling framework, and OWA trees, which is the framework that at least partially fulfills most of the seven identified properties.

\subsubsection{Attack Trees}\label{sec: attack trees}

The first \emph{tree-based approach}, shown as an AND-OR tree structure for graphical security modeling, was the \emph{threat logic trees}, which was introduced by Weiss in 1991~\cite{weiss1991}.
Today, all AND-OR tree structures are referred to as \emph{attack trees}, a term first introduced by Salter et al. in 1998~\cite{Salter1998}.

In attack trees, the root node (i.e., the tree's root) indicates the attack's main goal. 
The main goal is then conjunctively (AND) or disjunctively (OR) refined into sub-goals until they represent basic actions corresponding to atomic components that can be easily understood and quantified. 
Conjunctive refinements indicate that \emph{all} sub-goals need to be fulfilled in order to achieve the main goal, whereas disjunctive refinements indicate that \emph{at least one} sub-goal needs to be fulfilled for achieving the main goal~\cite{weiss1991}.

\subsubsection{OWA Trees}\label{sec: owa trees}

\emph{Ordered weighted averaging (OWA) trees} were proposed by Yager in 2005 to include the concept of \emph{uncertainty} into attack trees~\cite{Yager2006OWATA}. 
This was made possible by replacing the AND-OR nodes with OWA nodes (i.e., quantifiers, such as \emph{most}, \emph{some}, \emph{half of}, etc.) and therefore taking into consideration situations where the number of sub-goals that need to be fulfilled in order to achieve the main goal remains unknown. 
Finally, OWA trees allow for the evaluation of success probability and cost attributes, which can be jointly used to calculate the cheapest and most probable attack.

\subsection{Sequential Attack Modeling}\label{sec: sequential attack modeling}

\begin{table*}[h]
\rowcolors{2}{gray!10}{gray!40}
\renewcommand{\arraystretch}{1.2}
\caption{Sequential attack modeling frameworks compared to the seven defined properties.}
\label{tab: sequential attack modeling}
\noindent\makebox[\textwidth]{%
\begin{tabular}[t]{>{\raggedright}p{0.15\textwidth}>{\raggedright}p{0.06\textwidth}>{\raggedright}p{0.08\textwidth}>{\raggedright}p{0.1\textwidth}>{\raggedright}p{0.09\textwidth}>{\raggedright\arraybackslash}p{0.08\textwidth}>{\raggedright\arraybackslash}p{0.15\textwidth}>{\raggedright\arraybackslash}p{0.1\textwidth}}
\toprule
 & Attack Vectors & DAG Structure & Node Attributes & Dynamic Connectors & Edge Attributes & Countermeasure Nodes & Consequence Nodes
\tabularnewline
\midrule
% \textbf{Sequential Attack Modeling} & & & & &
% \tabularnewline
Cryptographic DAGs & \checkmark & \checkmark & - & - & - & - & - 
\tabularnewline
Fault Trees for Security & \checkmark & - & \checkmark & (\checkmark) & - & -  & -
\tabularnewline
Bayesian Networks for Security & \checkmark & \checkmark & \checkmark & - & \checkmark & - & -
\tabularnewline
Bayesian Attack Graphs & \checkmark & \checkmark & \checkmark & - & \checkmark & - & -
\tabularnewline
Compromise Graphs & \checkmark & \checkmark & - & - & (\checkmark) & - & -
\tabularnewline
Enhanced Attack Trees & \checkmark & - & \checkmark & - & (\checkmark) & - & -
\tabularnewline
Vulnerability Cause Graphs & (\checkmark) & \checkmark & - & - & - & - & -
\tabularnewline
Dynamic Fault Trees for Security & \checkmark & - & (\checkmark) & - & - & - & -
\tabularnewline
Serial Model for Multi-Parameter Attack Trees & \checkmark & - & (\checkmark) & - & - & - & -
\tabularnewline
Improved Attack Trees & \checkmark & - & (\checkmark) & - & - & - & -
\tabularnewline
Time-dependent Attack Trees & \checkmark & \checkmark & (\checkmark) & - & - & - & -
\tabularnewline
\bottomrule
\end{tabular}}
\end{table*}

Eleven frameworks for \emph{sequential attack modeling}, namely \emph{Cryptographic DAGs}~\cite{Meadows1996ARO}, \emph{Fault Trees for Security}~\cite{FaultTreesForSecurity}, \emph{Bayesian Networks for Security}~\cite{BayesianNetworksForSecurity}, \emph{Bayesian Attack Graphs}~\cite{BayesianAttackGraphs}, \emph{Compromise Graphs}~\cite{CompromiseGraphs}, \emph{Enhanced Attack Trees}~\cite{EnhancedAttackTrees}, \emph{Vulnerability Cause Graphs}~\cite{VulnerabilityCauseGraphs}, \emph{Dynamic Fault Trees for Security}~\cite{DynamicFaultTreesForSecurity}, \emph{Serial Model for Multi-Parameter Attack Trees}~\cite{SerilModelForMultiParameterAttackTrees}, \emph{Improved Attack Trees}~\cite{ImprovedAttackTrees}, and \emph{Time-dependent Attack Trees}~\cite{TimeDependentAttackTrees}, are summarised in Table~\ref{tab: sequential attack modeling}. 
Again, none of the frameworks fulfills the countermeasure and consequence nodes property. In addition, only Fault Trees for Security offer a wide range of dynamic connectors, and only Bayesian-based models fulfill the edge attributes property. 
Finally, Compromise Graphs and Enhanced Attack Trees are two frameworks that at least partially fulfill the edge attributes property, and Vulnerability Cause Graphs are the only framework that only partially fulfills the attack vectors property. 
This section describes Cryptographic DAGs, which was the first graph-based approach for security modeling, and Bayesian Attack Graphs, which combine attack trees and Bayesian networks and also fulfill four of the seven identified properties.

\subsubsection{Cryptographic DAGs}\label{sec: cryptographic dags}

\emph{Cryptographic directed acyclic graphs} were proposed by Meadows~\cite{Meadows1996ARO} in 1996 to provide a \emph{novel} simple representation of sequences and dependencies of attack steps towards the main goal of the attack. 
Instead of a tree-based approach, Cryptographic DAGs introduced a \emph{graph-based approach} for security modeling. 
However, they eventually do not offer the possibility to perform risk assessment as other properties are still not fulfilled.

\subsubsection{Bayesian Networks and Bayesian Attack Graphs}\label{sec: bayesian Networks and Attack Graphs}

For the last couple of decades, researchers have been focusing on \emph{Bayesian networks} for the purposes of security modeling.
The origin of Bayesian networks, which are also known as \emph{belief} or \emph{causal networks}, lies in artificial intelligence.
In Bayesian networks, nodes represent events or objects and are associated with probabilistic variables. 
Hence, analyzing the uncertainty of events is also possible. 
Bayesian networks follow a DAG structure, where the directed edges represent the causal dependencies between the nodes~\cite{BayesianAttackGraphs}.

\emph{Bayesian attack graphs} are a fusion of (general) attack trees and (computational procedures) of Bayesian networks, and they were first introduced by Liu and Man in 2005 to analyze network vulnerability scenarios~\cite{BayesianAttackGraphs}. 
Subsequently, calculating general security metrics regarding information system networks~\cite{Frigault2008, Noel2010} and capturing dynamic behavior~\cite{Frigault2008Dyn} was also made possible.

Finally, although Bayesian attack graphs allow for assigning values to nodes and for performing computations using the graphs, they do not allow for a dynamic selection of connectors and for including countermeasures. 
As a result, Bayesian attack graphs cannot be used to perform risk assessment.

\subsection{Static Attack and Defense Modeling}\label{sec: static attack and defense modeling}

\begin{table*}[h]
\rowcolors{2}{gray!10}{gray!40}
\renewcommand{\arraystretch}{1.2}
\caption{Static attack and defense modeling frameworks compared to the seven defined properties.}
\label{tab: static attack and defense modeling}
\noindent\makebox[\textwidth]{%
\begin{tabular}[t]{>{\raggedright}p{0.15\textwidth}>{\raggedright}p{0.06\textwidth}>{\raggedright}p{0.08\textwidth}>{\raggedright}p{0.1\textwidth}>{\raggedright}p{0.09\textwidth}>{\raggedright\arraybackslash}p{0.08\textwidth}>{\raggedright\arraybackslash}p{0.15\textwidth}>{\raggedright\arraybackslash}p{0.1\textwidth}}
\toprule
 & Attack Vectors & DAG Structure & Node Attributes & Dynamic Connectors & Edge Attributes & Countermeasure Nodes & Consequence Nodes
\tabularnewline
\midrule
% \textbf{Static Attack and Defense Modeling}
% \tabularnewline
Anti-Models & \checkmark & - & - & - & - & \checkmark & -
\tabularnewline
Defense Trees & \checkmark & - & (\checkmark) & - & - & \checkmark & -
\tabularnewline
Protection Trees & - & - & \checkmark & - & - & \checkmark & -
\tabularnewline
Security Activity Graphs & \checkmark & \checkmark & (\checkmark) & - & - & \checkmark & -
\tabularnewline
Attack Countermeasure Trees & \checkmark & - & \checkmark & - & - & \checkmark & -
\tabularnewline
Attack-Defense Trees & \checkmark & - & \checkmark & - & - & \checkmark & -
\tabularnewline
Countermeasure Graphs & \checkmark & \checkmark & \checkmark & - & - & \checkmark & -
\tabularnewline
\bottomrule
\end{tabular}}
\end{table*}

Seven frameworks for \emph{static attack and defense modeling}, namely \emph{Anti-Models}~\cite{AntiModels}, \emph{Defense Trees}~\cite{DefenseTrees}, \emph{Protection Trees}~\cite{ProtectionTrees}, \emph{Security Activity Graphs}~\cite{SecurityActivityGraphs}, \emph{Attack Countermeasure Trees}~\cite{AttackCountermeasureTrees}, \emph{Attack-Defense Trees}~\cite{AttackDefenseTrees}, and \emph{Countermeasure Graphs}~\cite{CountermeasureGraphs}, are summarised in Table~\ref{tab: static attack and defense modeling}. 
All frameworks fulfill the countermeasure nodes property, and only Protection Trees do not fulfill the attack vectors property. 
In addition, Anti-Models is the only framework that does not at least partially fulfill the node attributes property.
However, none of these frameworks considers consequence nodes in their design.
This section describes Security Activity Graphs and Countermeasure Graphs, which are the two frameworks that fulfill four of the seven identified properties.

\subsubsection{Security Activity Graphs}\label{sec: security activity graphs}

\emph{Security activity graphs (SAGs)} were developed by Ardi et al.~\cite{SecurityActivityGraphs} in 2006 to improve security throughout the software development process. 
SAGs are loosely based on fault trees, and the root of a SAG is associated with a vulnerability. 
Vulnerability mitigations are modeled using activities (i.e., leaf nodes), which are assigned boolean variables to indicate whether an activity \enquote{is implemented perfectly during software development} (true) or not (false). 
Finally, besides AND-OR gates, which follow a strictly Boolean logic, SAGs also include \emph{split gates}, which allow one activity to be used in several parent activities, thus creating a DAG structure.

However, SAGs lack the ability to represent the consequences of threats and edge attributes, and both are necessary to calculate a risk value.
Furthermore, there is only a limited option for connectors and node attributes.
Therefore, rendering SAGs impractical for risk assessment.

\subsubsection{Countermeasure Graphs}\label{sec: countermeasure graphs}

\emph{Countermeasure graphs} were introduced by Baca and Petersen~\cite{CountermeasureGraphs} in 2010 to simplify countermeasure selection through cumulative voting. 
Countermeasure graphs are created by identifying actors, goals, attacks, and countermeasures. Related events are connected with edges. 
That is, actors are connected to pursued goals and likely executable attacks, and countermeasures are connected to preventable attacks. 
Finally, actors, goals, attacks, and countermeasures are assigned priorities according to the rules of hierarchical cumulative voting. 
Higher assigned priorities imply higher threat levels of the corresponding events and vice versa. 
Using hierarchical cumulative voting, the most effective countermeasures can be identified.

Countermeasure Graphs provide a useful system overview, but the computational rules focus on finding the most effective countermeasure instead of the most likely and severe attack. 
This limitation could at least partially be addressed with the threat level. 
However, the threat level value is determined by the subjective assessment of the graph creator rather than by calculations over meaningful attributes, thereby raising issues of validity.

\subsection{Sequential Attack and Defense Modeling}\label{sec: sequential attack and defense modeling}

\begin{table*}[h]
\rowcolors{2}{gray!10}{gray!40}
\renewcommand{\arraystretch}{1.2}
\caption{Sequential attack and defense modeling frameworks compared to the seven defined properties.}
\label{tab: sequential attack and defense modeling}
\noindent\makebox[\textwidth]{%
\begin{tabular}[t]{>{\raggedright}p{0.15\textwidth}>{\raggedright}p{0.06\textwidth}>{\raggedright}p{0.08\textwidth}>{\raggedright}p{0.1\textwidth}>{\raggedright}p{0.09\textwidth}>{\raggedright\arraybackslash}p{0.08\textwidth}>{\raggedright\arraybackslash}p{0.15\textwidth}>{\raggedright\arraybackslash}p{0.1\textwidth}}
\toprule
 & Attack Vectors & DAG Structure & Node Attributes & Dynamic Connectors & Edge Attributes & Countermeasure Nodes & Consequence Nodes
\tabularnewline
\midrule
% \textbf{Sequential Attack and Defense Modeling} & & & & &
% \tabularnewline
Insecurity Flows & \checkmark & \checkmark & \checkmark & - & - & \checkmark & -
\tabularnewline
Intrusion DAGs & \checkmark & \checkmark & - & - & - & \checkmark & -
\tabularnewline
Bayesian Defense Graphs & \checkmark & \checkmark & (\checkmark) & - & - & \checkmark & -
\tabularnewline
Security Goal Indicator Trees & - & - & - & - & - & \checkmark & -
\tabularnewline
Attack Response Trees & \checkmark & -  & \checkmark & - & - & \checkmark & -
\tabularnewline
Boolean Logic Driven Markov Processes & \checkmark & \checkmark & (\checkmark) & (\checkmark) & - & \checkmark & -
\tabularnewline
Cyber Security Modeling Language & \checkmark & \checkmark & (\checkmark) & - & (\checkmark) & \checkmark & -
\tabularnewline
Security Goal Models & \checkmark & \checkmark & - & - & - & \checkmark & -
\tabularnewline
Unified Parameterizable Attack Trees & \checkmark & - & \checkmark & - & (\checkmark) & \checkmark & -
\tabularnewline
\bottomrule
\end{tabular}}
\end{table*}

Finally, nine frameworks for \emph{sequential attack and defense modeling}, namely \emph{Insecurity Flows}~\cite{InsecurityFlows}, \emph{Intrusion DAGs}~\cite{IntrusionDAGs}, \emph{Bayesian Defense Graphs}~\cite{BayesianDefenseGraphs}, \emph{Security Goal Indicator Trees}~\cite{SecurityGoalIndicatorTrees}, \emph{Attack Response Trees}~\cite{AttackResponseTrees}, \emph{Boolean Logic Driven Markov Process}~\cite{BooleanLogicDrivenMarkovProcess}, \emph{Cyber Security Modeling Language}~\cite{CyberSecurityModelingLanguage2010}, \emph{Security Goal Models}~\cite{SecurityGoalModels}, and \emph{Unified Parameterizable Attack Trees}~\cite{UnifiedParameterizableAttackTrees}, are summarized in Table~\ref{tab: sequential attack and defense modeling}. 
All frameworks fulfill the countermeasure nodes property, and only Security Goal Indicator Trees do not fulfill the attack vectors property. 
In addition, Boolean Logic Driven Markov Processes (BDMPs) is the only framework that offers a wide range of connectors and, therefore, at least partially fulfills the dynamic connectors property. 
This section describes BDMPs and Cyber Security Modeling Language (CySeMoL), which are the two frameworks that at least partially fulfill five of the seven identified properties.

\subsubsection{Boolean Logic Driven Markov Processes}\label{sec: boolean logic driven markov processes}

\emph{Boolean logic driven Markov processes (BDMPs)} are a security modeling framework, which can also be used to perform risk assessment~\cite{BooleanLogicDrivenMarkovProcess}. 
It was invented by Bouissou and Bon~\cite{BooleanLogicDrivenMarkovProcess} in 2003 for the safety and reliability domains, and it was later adapted to security modeling by Piètre-Cambacédès and Bouissou in 2010. 
BDMPs combine the readability of attack trees with the modeling power of Markov chains. 
The root (top event) of a BDMP represents the main goal of the attack, and the leaves represent the attack steps or security events.
BDMPs offer a wide range of node attributes, including time-domain metrics, such as mean-time to success, attack tree-related metrics, such as costs of attacks, boolean indicators, such as specific requirements, and risk assessment tools, such as sensibility graphs.

However, the lack of edge attributes, in addition to issues of usability with respect to leaf nodes and connectors~\cite{BDMPCritic}, render BDMPs impractical for risk assessment.

\subsubsection{Cyber Security Modeling Language}\label{sec: cyber security modeling language}

\emph{Cyber security modeling language (CySeMoL)} was developed by Sommestad et al. in 2010 to assess the cyber security of \emph{supervisory control and data acquisition (SCADA)} system architectures~\cite{CyberSecurityModelingLanguage2010, CyberSecurityModelingLanguage2013}.
Simply modeling the system architecture and the characteristics of the involved assets is sufficient, as CySeMoL already includes information about how attacks and defenses are quantitatively related. 
The attacker is assumed to be a professional penetration tester with a fixed time of one week to perform an attack.
CySeMoL was extended by Holm in 2014 and renamed to \emph{predictive, probabilistic cyber security modeling language ((P$^2$)CySeMoL)}, introducing more flexible and useful computations, the possibility to model assets, attacks, and defenses that are not necessarily SCADA-related, and the option to specify the time needed to perform an attack~\cite{PredictiveProbabilisticCyberSecurityModelingLanguage}.
Computations can be conducted automatically (i.e., without personalized inputs) as (P$^2$)CySeMoL already includes qualitative information gathered from literature reviews, empirical studies, as well as surveys involving domain experts~\cite{CyberSecurityModelingLanguage2010, CyberSecurityModelingLanguage2013, PredictiveProbabilisticCyberSecurityModelingLanguage}.

The results of the computations show the likelihood of an attack. 
However, the severity of an attack is not considered, and therefore the risk of an attack cannot be properly assessed. 
Furthermore, (P$^2$)CySeMoL does not include connectors, and therefore it seems an inconvenient tool for graphical risk assessment.

\subsection{Summary of Remarks}\label{sec2: summary of remarks}

This section provides an overview of thirty-three frameworks for analysis of attack and defense scenarios, and it describes eight of these frameworks in more detail. 
Thirty frameworks fulfill the attack vectors property, sixteen frameworks fulfill the countermeasure nodes property, and only thirteen frameworks fulfill the DAG structure property.
In addition, node/edge attributes and connectors are, in most cases, fixed and limited, thereby reducing the usability and usefulness of the frameworks with respect to the purposes of risk assessment. 
The complex nature and rapid development of (information) systems, attacks, and defenses motivate the need for proper risk management.
Existing methods are mainly consisting of tables with graphical solutions mostly utilized for support, if at all.
As shown in this section, current graphical solutions support threat or vulnerability management and sometimes even calculations to determine which attack vector might be the easiest to execute or, in other terms, which is most probable to occur.
The risk value cannot be equated with probability, though, and is usually determined using the probability of an event and its impact.
However, none of the methods described in this section can represent an event's consequences and impact, rendering them incapable of performing risk assessment.


\section{Methods}
\label{sec:method}

CTVIS builds upon Mask2Former \cite{mask2former}, which is an effective image instance segmentation model (briefly reviewed in Section~\ref{sec:mask2former})\footnote{
%We would like to
Note that CTVIS can be easily combined with other query-based instance segmentation models \cite{idol, detr, deformabledetr} with minor modifications.}. Our CTVIS is motivated by the inference of typical online VIS methods introduced in Section~\ref{sec:inference}. 
Then we detail our consistent training method in Section~\ref{sec:ct}. Finally, Section~\ref{sec:pseudo} presents our goal-oriented pseudo-video generation technique for training VIS models with sparse image-level annotations.

% We closely follow the Notations in MinVIS
\subsection{Brief Overview of Mask2Former} 
\label{sec:mask2former}
Mask2Former \cite{mask2former} composed of three main components: an \emph{image encoder} $\mathcal{E}$ (consist of a backbone and a pixel decoder), a \emph{transformer decoder} $\mathcal{T}$ and a \emph{prediction head} $\mathcal{P}$. Given an input image $I\in \mathbb{R}^{H \times W \times 3}$, $\mathcal{E}$ extracts a set of feature maps $\bm{F}=\mathcal{E}(I)$, where $\bm{F} = \{ F_0 \cdots F_{-1}\}$ is a sequence of multi-scale feature maps, and $F_{-1}$ is the final output of the $\mathcal{E}$ with $1/4$ resolution of $I$. The $N$ raw query embeddings $\hat{Q} \in \mathbb{R}^{N \times C}$ are learnable parameters, where $N$ is a large enough number of outputs and $C$ is the number of channels. Then, $\mathcal{T}$ takes both $\bm{F}$ and $\hat{Q}$ to iteratively refine query embeddings, and consequently outputs $Q \in \mathbb{R}^{N \times C}$. Finally, the prediction head outputs the segmentation masks $M$ and the classification scores $O$. For classification, $O=\mathcal{C}(Q) \in \mathbb{R}^{N \times K}$, where $K$ is the number of object categories. For  segmentation, the masks $M \in \mathbb{R}^{N \times H/4 \times W/4}$ are generated with $M = \sigma(Q \ast F_{-1})$, where $\ast$ denotes the convolution operation and $\sigma(\cdot)$ is the sigmoid function.

\noindent\textbf{Our Modification.} Because CTVIS employs instance embeddings to associate instances during inference, we add a  head (a few MLP layers) to compute the instance embeddings $E \in \mathbb{R}^{N \times C}$ based $Q$. 
% The entire process can be summarized as
% \begin{equation}
% \label{eq:mask2former}
%     O, M, E = Mask2Former(I).
% \end{equation}

\subsection{Inference of CTVIS}
\label{sec:inference}
CTVIS leverages Mask2Former\cite{mask2former} to process each frame 
%(\ie Equation~\eqref{eq:mask2former}) 
and introduces an external memory bank\cite{idol, masktrackrcnn} to store the states of previously detected instances, including classification scores, segmentation masks and instance embeddings. 
% gets the corresponding classification scores, segmentation masks and instance embeddings for each frame. 
% Specially, CTVIS makes instance association frame by frame and introduces an external memory bank to store the states of previously detected instances, including classification scores, segmentation masks and instance embeddings. 
To ease presentation, we assume that CTVIS has already processed $T$ frames out of an input video of $L$ frames, and there are $N$ predicted instances with $N$ instance embeddings $\bold{d}_i \in \mathbb{R}^C$ in the current frame. The memory bank stores for the previous $T$ frames $M$ detected instances, each of which has multiple temporal instance embeddings $\{ \bold{e}^t_j \in \mathbb{R}^C  \}^T_{t=1}$ and a momentum-averaged instance embedding $\hat{\bold{e}}_j^T$, which is computed according to the similarity-guided fusion \cite{sgf}: 
\begin{gather}
    \label{eq:sgf}
    \hat{\bold{e}}^T_j=(1-\beta^T) \hat{\bold{e}}^{T-1}_j+\beta^T \bold{e}^T_j \text {, } \\
    \beta^T=\max \left\{0, \frac{1}{T-1} \sum_{k=1}^{T-1} \Psi_d\left(e^T_j, e^{T-k}_j\right)\right\} , 
\end{gather}

% Figure environment removed 

\noindent where $\Psi_d$ denotes the cosine similarity. % This momentum type brings more flexblity for instance embedding fusion. 
We refer the reader to \cite{sgf} for more details. Next, for each instance $i$ detected in the current frame, we compute its bi-softmax similarity \cite{qdtrack} with respect to the previously detected instance $j$ using


\begin{equation}
\label{equ:bio_softmax}
    f_{i,j}=
    0.5 \cdot \left[\frac{\exp \left(\hat{\mathbf{e}}_j^T \cdot \mathbf{d}_i\right)}{\sum_k \exp \left(\hat{\mathbf{e}}_k^T \cdot \mathbf{d}_i\right)}+\frac{\exp \left(\hat{\mathbf{e}}_j^T \cdot \mathbf{d}_i\right)}{\sum_{l} \exp \left(\hat{\mathbf{e}}_j^T \cdot \mathbf{d}_l\right)}\right] %\cdot 
 %   / 2. 
\end{equation}

Finally, we find the ``best''  
instance ID for $i$ with
\begin{equation}
\hat{j}=\arg \max f_{i,j}, \forall j \in\{1,2, \ldots, M\}.
\end{equation}
If $f_{i,\hat{j}} > 0.5$, we believe that newly detected instance $i$ and instance $\hat{j}$ in the memory bank correspond to the identical target. Otherwise, we initiate a new instance ID in the memory bank. When all frames are processed, the memory bank contains a certain number of instances, each of which takes a classification score list $\{c_i^t\}_{t=1}^{L}$ and a mask list $\{m_i^t\}_{t=1}^{L}$ (recall that $L$ denotes the number of frames). For each instance $i$, we calculate its video-level classification score by averaging the frame-level scores of the object. 
% $\boldsymbol{c_i}$ as follows: $\boldsymbol{c_i} = \Sigma_{t=1}^L c_i^t$. We get the $\{(\boldsymbol{c_i}, \{m_i^t\}_{t=1}^{L})\}_{i=1}^S$ as final outputs. 

% \subsection{Constructing CIs via Consistent Training}
\subsection{Consistent Learning}
\label{sec:ct}

A reliable matching of instances (\ie using Equation~\eqref{equ:bio_softmax}) across time is required to track instances successfully. Hence the extraction of highly discriminative embeddings of objects is of great importance. 
We argue that the discrimination of instance embeddings extracted with recent models \cite{idol, stc} is still inadequate, especially for videos involving object-occlusion, shape-transformation and fast-motion. One main reason is that mainstream contrastive learning methods build CIs (\ie $\{\mathbf{v},\mathbf{k}^+,\mathbf{k}^-\}$) from the reference frame only, which results in the comparison of the anchor embedding against instantaneous instance embeddings in $\mathbf{k}^+$ and $\mathbf{k}^-$. Such embeddings are typically less discriminative and contain noise, which prevents training from learning robust representations. To address this, our CTVIS leverages a memory bank to store MA embeddings, thus supporting contrastive learning from more stable representations. Here our insight is to align the embedding comparison of training with that of inference (such that the two comparisons are consistent). Figure~\ref{fig:main} sketches our CTVIS, which processes the training video frame-by-frame. For an arbitrary frame $t$, CTVIS involves three steps: a) it first takes the Mask2Former and Hungarian matching to compute the instance embeddings, and to match them with GT (highlighted by red, green and purple); b) Then, it builds CIs using MA embeddings within the memory bank, and performs contrastive learning with CIs; and c) It updates the memory bank with noise (\eg the embedding of the \emph{cat} is deliberately added to the memory of the \emph{dog}), which serves the learning from the next frame.

\noindent\textbf{Forward passing and GT assignment.} As shown in Figure~\ref{fig:main}~(a), we first feed the current frame $t$ into Mask2Former to compute the embeddings for queries. Then we employ Hungarian matching to find an optimal match between the decoded instances and the ground truth (GT), such that each GT instance is assigned one instance embedding. Note that Hungarian matching relies on the costs calculated for all (\emph{Decoded-Instance}, \emph{GT-Instance}) pairs. Essentially, each cost measures the similarity between a pair of instances based on their labels and masks.

\noindent\textbf{Construct CIs.} 
After GT assignment, we build the contrastive items for each GT instance using a memory bank. The memory bank stores all detected instances of previous $t-1$ frames, each associated with 1) a series of instance embeddings extracted at different times, and 2) its MA embedding computed by Equation~\eqref{eq:sgf}. 
% To clarify, we only  show the contrastive item of the person in Figure~\ref{fig:main}(b), we select the instance embeddings of the person at current frames as query embedding $v$. For the positive embedding, we select the momentum-averaged embedding of person from the memroy bank
In order to prepare the CIs $\{\mathbf{v}, \mathbf{k}^+, \mathbf{k}^-\}$ for instance $i$ (termed as the \emph{anchor}, \eg the person in Figure~\ref{fig:main}~(a)) at the $t$-th frame, the instance embedding extracted from this frame is used as the anchor embedding $v$.
%
For the positive embedding, we pick from the memory bank the MA embedding of instance $i$.
%
The negative embeddings $\mathbf{k}^-$ include the major negative embeddings and the supplementary negative embeddings. We use the MA embeddings of other instances in the memory bank as the major negative embeddings. We also sample the background query embeddings of previous $t - 1$ frames to form the supplement negative embeddings. Taking as inputs the created CIs, we compute the contrastive loss with
\begin{equation}
\label{eq:loss_embed}
\begin{aligned}
    \mathcal{L}_{\text {emb}} & =-\log \frac{\exp \left(\mathbf{v} \cdot \mathbf{k}^{+}\right)}{\exp \left(\mathbf{v} \cdot \mathbf{k}^{+}\right)+\sum\nolimits_{\mathbf{k}^{-}} \exp \left(\mathbf{v} \cdot \mathbf{k}^{-}\right)} \\
    & =\log \left[1+\sum\nolimits_{\mathbf{k}^{-}} \exp \left(\mathbf{v} \cdot \mathbf{k}^{-}-\mathbf{v} \cdot \mathbf{k}^{+}\right)\right].
\end{aligned}
\end{equation}
As shown in Figure~\ref{fig:main} (c), training with $\mathcal{L}_{\text {emb}}$ pulls the embeddings of positive instances close to the anchor embedding, while pushing the negative embeddings away from it.

\noindent\textbf{Update memory bank.} After computing the $\mathcal{L}_{\text{emb}}$ for each instance in frame $t$, we need to update the memory bank, such that the updated version can be taken to build CIs for frame $t+1$.
%
Unlike the inference stage, for training we can get the ground truth ID of each instance so as to update the memory bank with their embeddings extracted from frame $t$.
%
In comparison, inference can fail to track instances across time (\ie the ID switch issue), especially for complicated scenarios. To alleviate this, we introduce noise to the update of the memory bank, which compels the contrastive learning to tackle the switch of instance IDs.
%
Specifically, each disappeared instance (\eg the dog) in frame $t$ will have a little chance to receive an embedding of other instances (\eg the cat, which is randomly picked from all available instances) in the same frame, which is called the \emph{noise}. 
%
% As illustrated in Figure~\ref{fig:main}~(c), the dog disappeared in frame $t$, and a new instance of cat presents.
%
If the generated random value exceeds a threshold (\eg 0.05), as illustrated in Figure~\ref{fig:main}~(c), we use the noise as the embedding of the disappeared instance at frame $t$. Finally, the MA embeddings are updated for all instances using Equation~\eqref{eq:sgf}. Due to the low similarity between the disappeared instance and the noise, such an update has quite a limited impact on the MA embedding of the instance, which is reidentified later. Indeed, training with noise is able to reduce the chance of ID switch, as demonstrated by the fish example in Figure~\ref{fig:video}. 

\noindent\textbf{Loss.} After processing all frames, The $\mathcal{L}_{\text {emb}}$ values of all CIs are averaged to obtain $L_{\text {emb}}$.
The total training loss is
\begin{equation}
L_{\text{total}} = \lambda_{\text{emb}}L_{\text{emb}} + \lambda_{\text{cls}} L_{\text{cls}} + \lambda_{\text{ce}} L_{\text {ce}} + \lambda_{\text{dice}} L_{\text{dice}},
\end{equation}
where $\lambda$ denotes loss weight. $L_{\text {cls}}$, $L_{\text {ce}}$ and $L_{\text {dice}}$ supervise the per-frame segmentation as suggested in \cite{mask2former}.

% when the ID of an instance changes to another instance in a complicated scene, most current methods always accumulate errors; to ease this issue, we introduce noise training, which directly simulates this situation during the construction of CI. As illustrated in Figure~\ref{fig:main}, the dog disappeared in the third frame, but a new instance of the cat appeared, and we added the cat's embedding to the external memory bank of the dog. Due to the low similarity between the instance embeddings of cats and dogs, it will have little impact on the MA embedding of further dogs that appear in the following frames. As shown in the video scene on the right of Figure~\ref{fig:video}, the wrong instances are corrected to original trajectories through noise training. 

% Here we describe how to build CIs via consistent training. Following the inference pipeline, we build the CIs frame by frame with updating the memory bank. For each frame (expect the first frame),  we will build CIs for each instances. 

% The consistent training aims at constructing CIs frame by frame following the inference stage introduced in Section~\ref{sec:inference}. As shown in Figure~\ref{fig:main}, we sample $T$ temporally adjcent frames as train video $\{ I_t\}_{t=1}^T$. 
% The first step is feeding each frame into a weight-shared Mask2Former and then matching the output with the corresponding ground truth via Hungarian matching. The output of each input frame $I_t$ comprises classification scores, segmentation masks and instance embeddings, formulated as $\{ O_t, M_t, E_t\}$. Then we calculate the pair-wise matching cost, considering both class prediction and the similarity of predicted and the ground truth masks. Next, we use Hungarian matching to assign one predicted instance to each ground truth instance. Specially, for each ground truth instance $j$ at frame $I_t$, we have a matched predicted instance embedding $\mathbf{e}^t_j$. 

% Then we construct contrastive items, each of which consists of anchor/positive/negative embeddings, for each ground truth instances at each frames. 
% % Contrastive items are composed of three parts: anchor embedding, positive embedding and negative embedding. Assume we contruct the contrastive item for the instance $i$ at the $I_t$, we build 
% Here we detail the construction of the CI of the GT instance $i$ at the frame $I_t$ (e.g. the dog at the $I_5$ shown in Figure~\ref{fig:main}). As shown in Figure~\ref{fig:main} (b), the memory bank stores the status of instances of previous frames. For each instance ID in the memory bank, we can get corresponding momentum-averaged embedding from the instance embeddings via Equation~\ref{eq:sgf}. We sample the instance embedding $e^t_j$ at the current frame as the anchor embedding $v$. For the positive embedding, we choose the momentum-averaged embedding of the same instance ID from the memory bank. The negative embeddings $k^-$ of contrastive item compose of two parts: major negative embeddings and supplement negative embeddings. And we select the momentum-averaged embeddings of other instance IDs as major negative embeddings. Furthermore, we sample the background embedding of previous $t - 1$ frames as supplement negative embeddings. Finally, we compute the contrastive loss upon each contrastive item as follows:
% \begin{equation}
% \label{eq:loss_embed}
% \begin{aligned}
%     \mathcal{L}_{\text {embed}} & =-\log \frac{\exp \left(\mathbf{v} \cdot \mathbf{k}^{+}\right)}{\exp \left(\mathbf{v} \cdot \mathbf{k}^{+}\right)+\sum_{\mathbf{k}^{-}} \exp \left(\mathbf{v} \cdot \mathbf{k}^{-}\right)} \\
%     & =\log \left[1+\sum_{\mathbf{k}^{-}} \exp \left(\mathbf{v} \cdot \mathbf{k}^{-}-\mathbf{v} \cdot \mathbf{k}^{+}\right)\right].
% \end{aligned}
% \end{equation}
% As shown in Figure~\ref{fig:main} (c), the $\mathcal{L}_{\text {embed}}$ enforces the embeddings of same instances while draws embeddings of different embeddings far away. 

% In addition, when the ID of an instance changes to another instance in a complicated scene, most current methods always accumulate errors; to ease this issue, we introduce noise training, which directly simulates this situation during the construction of CI. As illustrated in Figure~\ref{fig:main}, the dog disappeared in the third frame, but a new instance of the cat appeared, and we added the cat's embedding to the external memory bank of the dog. Due to the low similarity between the instance embeddings of cats and dogs, it will have little impact on the MA embedding of further dogs that appear in the following frames. As shown in the video scene on the right of Figure~\ref{fig:video}, the wrong instances are corrected to original trajectories through noise training. 

% Finally, we get the final contrastive loss by average on all contrastive items. Besides, we get the loss final loss as follows: $$

% For instance, a contrastive item is generated for contrastive learning that matches the ground truth (GT) on each frame. This contrastive item is mainly composed of three parts:
% \begin{itemize}
% \item The anchor, which is the instance embedding of the current frame
% \item The positive and negative samples, which are the instances matched with GT by calculating the momentum-averaged instance  embedding of the current frame through the similarity-guided fusion mentioned in Section~\ref{sec:inference}
% \item Other embeddings that do not match GT, which are directly treated as negative samples
% \end{itemize}

% When the ID of an instance changes to another instance in a complicated scene, most current methods will always be wrong; thus, we directly simulate this situation in the training phase. As illustrated in Figure~\ref{fig:main}, the dog disappeared in the third frame, but a new instance of the cat appeared, and we added the cat's embedding to the external memory bank $B_{dog}$ of the dog. Due to the low similarity between the instance embeddings of cats and dogs, it will have little impact on the MA embedding of further dogs that appear in the following frames. As shown in the video scene on the right of Figure~\ref{fig:video}, the wrong instances are corrected to original trajectories through noise training.

\vspace{-3mm}
\subsection{Learning from Sparse Annotation}
\label{sec:pseudo}

% Figure environment removed 

We now elaborate on our pseudo-video and mask generation technique, which enables the training of VIS models when only sparse annotations (\eg image data) are available. We take a few widely applied image-augmentation methods, including \emph{random rotation}, \emph{random crop} and \emph{copy\&paste} on source image to create pseudo-videos and the associated instance masks. Note that the pseudo-videos are created by no means to approximate real ones. Instead, they are taken to mimic the movement of targets in reality. 

\noindent\textbf{Rotation.} 
As shown in the first row of Figure~\ref{fig:augs}, the rotation augmentation rotates the source images with several random angles (e.g., $ [-15, 15]$ ) to introduce subtle changes between frames of the pseudo-videos. 

\noindent\textbf{Crop.} 
The rotation augmentation cannot alter the shapes and magnitudes of instances. However, instances deform or/and enter/exit the visible field due to the movement introduced either by the target or the camera. To address this, we apply random crop augmentation to the image, which allows the generated videos to mimic the zooming in/out effect of the camera lens and the shifting of targets. The second and the third rows of Figure~\ref{fig:augs} present two examples of \emph{crop-zoom} and \emph{crop-shift}, respectively. The pseudo-videos generated by such augmentations cover a large proportion of targets' movements.

\noindent\textbf{Copy and Paste.} 
As mentioned earlier, the trajectories of instances in pseudo-videos created by the augmentations share the identical motion direction. To incorporate the relative motion between instances, we also employ the \emph{copy\&paste} augmentation\cite{copypaste}, which copies the instances from another image in the dataset and pastes them into  random locations within the source image. Note that the pasting positions of an instance are typically different across time, which brings the relative motion between different instances (as shown in the fourth row of Figure~\ref{fig:augs}).
% As shown in Figure~\ref{fig:augs}, this operation brings relative between several instances.

% \noindent\textbf{Merge All.} Suppose we want to generate a pseudo-video of T frames. Given an input image $I_{des}$, we randomly select another image from datasets as $I_{src}$. Then we parallelly make the copy\&paste $T$ times each of which copy\&pastes the instances of $I_{src}$ into the $I_{des}$. We get the output as N

% which selects two images, ${img_k}$ and ${img_r}$, from the dataset by random, and applies the aforementioned augmentations to them simultaneously. Subsequently, a subset of the pixel values of the instance from ${img_r}$, exceeding a certain threshold, is selected and pasted onto ${img_k}$. Finally, the necessary modifications to the ground-truth annotation are made, the fully occluded object is removed, and the mask of the partially occluded mask and bounding box is updated. As shown in the fourth row of Figure~\ref{fig:augs}, the copy\&paste augmentation\cite{copypaste}

\section{Experiment}
\label{sec:experiment}

\noindent\textbf{Datasets.} 
The proposed methods are evaluated on three VIS benchmarks: YTVIS19\cite{masktrackrcnn}, YTVIS21\cite{masktrackrcnn} and OVIS\cite{ovis}.

% We provide more dataset details in the supplementary material.
% YTVIS19 covers 40 object classes and contains 2,238/302 videos for training/validation. YTVIS21 retains the number of categories of YTVIS19 while expanding the datasets to 2,985/421 videos for training/validation and improving the instance annotation quality. OVIS has 25 object classes and contains 607/140 videos for training/validation. While the number of videos is less, each OVIS video includes more frames (69.4 frames on average) than YTVIS19 (27.6 frames) and YTVIS21 (30.2 frames). 
% Moreover, OVIS samples typically involve more instances and severe occlusion and thus are more challenging.

\begin{table*}[!htbp]
\vspace{-2mm}
  \centering
    \resizebox{0.95\textwidth}{!}{%
    \begin{tabular}{c|c|c|ccccc|ccccc|ccccc}
    \toprule
    \multicolumn{2}{c|}{\multirow{2}[1]{*}{Methods}} & \multirow{2}[1]{*}{\shortstack{Params.}} & \multicolumn{5}{c|}{YTVIS19\cite{masktrackrcnn}}                               & \multicolumn{5}{c|}{YTVIS21\cite{masktrackrcnn}}                               & \multicolumn{5}{c}{OVIS\cite{ovis}} \\
    \multicolumn{2}{c|}{}     &             & AP          & AP$_{\mathtt{50}}$        & AP$_{\mathtt{75}}$        & AR$_{\mathtt{1}}$         & AR$_{\mathtt{10}}$        & AP          & AP$_{\mathtt{50}}$        & AP$_{\mathtt{75}}$        & AR$_{\mathtt{1}}$         & AR$_{\mathtt{10}}$        & AP          & AP$_{\mathtt{50}}$        & AP$_{\mathtt{75}}$        & AR$_{\mathtt{1}}$         & AR$_{\mathtt{10}}$ \\
    \midrule
    \multirow{11}[2]{*}{\begin{sideways}ResNet-50\cite{resnet}\end{sideways}} & MaskTrack R-CNN\cite{masktrackrcnn} & -           & 30.3        & 51.1        & 32.6        & 31          & 35.5        & 28.6        & 48.9        & 29.6        & 26.5        & 33.8        & 10.8        & 25.3        & 8.5         & 7.9         & 14.9 \\
                & SipMask\cite{sipmask}     & -           & 33.7        & 54.1        & 35.8        & 35.4        & 40.1        & 31.7        & 52.5        & 34          & 30.8        & 37.8        & 10.2        & 24.7        & 7.8         & 7.9         & 15.8 \\
                & CrossVIS\cite{crossvis}    & -           & 36.3        & 56.38       & 38.9        & 35.6        & 40.7        & 34.2        & 54.4        & 37.9        & 30.4        & 38.2        & 14.9        & 32.7        & 12.1        & 10.3        & 19.8 \\
                & IFC\cite{ifc}         & -           & 41.2        & 65.1        & 44.6        & 42.3        & 49.6        & 35.2        & 55.9        & 37.7        & 32.6        & 42.9        & 13.1        & 27.8        & 11.6        & 9.4         & 23.9 \\
                & Mask2Former-VIS\cite{mask2formervis} & 44          & 46.4        & 68          & 50          & -           & -           & 40.6        & 60.9        & 41.8        & -           & -           & 17.3        & 37.3        & 15.1        & 10.5        & 23.5 \\
                & TeViT\cite{tevit}       & -           & 46.6        & 71.3        & 51.6        & 44.9        & 54.3        & 37.9        & 61.2        & 42.1        & 35.1        & 44.6        & 17.4        & 34.9        & 15          & 11.2        & 21.8 \\
                & SeqFormer\cite{seqformer}   & 48        & 47.4        & 69.8        & 51.8        & 45.5        & 54.8        & 40.5        & 62.4        & 43.7        & 36.1        & 48.1        & 15.1        & 31.9        & 13.8        & 10.4        & 27.1 \\
                & MinVIS\cite{minvis}      & 44          & 47.4        & 69          & 52.1        & 45.7        & 55.7        & 44.2        & 66          & 48.1        & 39.2        & 51.7        & 25          & 45.5        & 24          & 13.9        & 29.7 \\
                & IDOL\cite{idol}        & \textbf{43}        & 49.5        & \underline{74}          & 52.9        & 47.7        & 58.7        & 43.9        & \underline{68}          & \underline{49.6}        & 38          & 50.9        & \underline{30.2}        & \underline{51.3}        & \underline{30}          & \underline{15}          & \underline{37.5} \\
                & VITA\cite{vita}        & 57        & \underline{49.8}        & 72.6        & \underline{54.5}        & \underline{49.4}        & \underline{61}          & \underline{45.7}        & 67.4        & 49.5        & \underline{40.9}        & \underline{53.6}        & 19.6        & 41.2        & 17.4        & 11.7        & 26 \\
                & \textbf{CTVIS (Ours)} & \underline{44}          & \textbf{55.1} & \textbf{78.2} & \textbf{59.1} & \textbf{51.9} & \textbf{63.2} & \textbf{50.1} & \textbf{73.7} & \textbf{54.7} & \textbf{41.8} & \textbf{59.5} & \textbf{35.5} & \textbf{60.8} & \textbf{34.9} & \textbf{16.1} & \textbf{41.9} \\
    \midrule
    \multirow{6}[1]{*}{\begin{sideways}Swin-L \cite{swin}\end{sideways}} 
    & SeqFormer\cite{seqformer}   & 219       & 59.3        & 82.1        & 66.4        & 51.7        & 64.6        & 51.8        & 74.6        & 58.2        & 42.8        & 58.1        & -           & -           & -           & -           & - \\
                & Mask2Former-VIS\cite{mask2formervis} & 216       & 60.4        & 84.4        & 67          & -           & -           & 52.6        & 76.4        & 57.2        & -           & -           & 25.8        & 46.5        & 24.4        & 13.7        & 32.2 \\
                & MinVIS\cite{minvis}      & 216         & 61.6        & 83.3        & 68.6        & 54.8        & 66.6        & 55.3        & 76.6        & 62          & 45.9        & 60.8        & 39.4        & 61.5        & 41.3        & \underline{18.1}        & 43.3 \\
                & VITA\cite{vita}        & 229       & 63          & 86.9        & 67.9        & \underline{56.3}        & 68.1        & \underline{57.5}        & 80.6        & 61          & \underline{47.7}        & \underline{62.6}        & 27.7        & 51.9        & 24.9        & 14.9        & 33 \\
                & IDOL\cite{idol}        & \textbf{213}       & \underline{64.3}        & \underline{87.5}        & \underline{71}          & 55.5        & \underline{69.1}        & 56.1        & \underline{80.8}        & \underline{63.5}        & 45          & 60.1        & \underline{42.6}        & \underline{65.7}        & \underline{45.2}        & 17.9        & \underline{49.6} \\
                & \textbf{CTVIS (Ours)} & \underline{216}         & \textbf{65.6} & \textbf{87.7} & \textbf{72.2} & \textbf{56.5} & \textbf{70.4} & \textbf{61.2} & \textbf{84} & \textbf{68.8} & \textbf{48} & \textbf{65.8} & \textbf{46.9} & \textbf{71.5} & \textbf{47.5} & \textbf{19.1} & \textbf{52.1} \\
    \bottomrule
    \end{tabular}%
    }
    \vspace{-1mm}
    \caption{Compare CTVIS with SOTA methods. The best and second best are highlighted by \textbf{bold} and \underline{underlined} numbers, respectively.
    %CTVIS models outperform SOTA approaches by clear margins, and achieve leading results on YTVIS19, YTVIS21 and OVIS.
    }
    \label{tab:main}%
    \vspace{-3mm}
\end{table*}%

\noindent\textbf{Metrics.}
Following prior studies \cite{idol, minvis, mask2formervis, masktrackrcnn, crossvis, ifc, seqformer, vita}, we use Average Precision (AP) and Average Recall (AR) as the evaluation metrics. 
% The intersection-over-union (IoU) thresholds, which range from 50\% to 95\% with a 5\% resolution, are used to calculate AP. 
%To be concrete, the AP and AR are first computed for each object class. Then they are averaged across all classes to get the final results. 
% Unless otherwise specified, all metrics' results are calculated using the open source platform \cite{codalab}. 
%Furthermore, we report the median of three runs to relieve the influence of random seeds.

\noindent\textbf{Implementation Details.}
%Our codebase is built upon PyTorch \cite{pytorch} and Detectron2 \cite{detectron2}. 
For the hyper-parameters of Mask2Former\cite{mask2former}, we just use its officially released version. The number of layers of the instance embedding head is 3. All models are initialized with parameters pre-trained on COCO \cite{mscoco}, and then they are trained on 8 NVIDIA A100 GPUs. Following prior works \cite{seqformer, vita, genvis},  we use the COCO joint training (CJT) setting to train our models unless otherwise specified. We set the lengths of training videos as 8 and 10 for YTVIS19\&21 and OVIS, respectively. For data augmentation, we use clip-level random crop and flip. During the training phase, we resize the input frames so that the shortest side is at least 320 and at most 640p, while the longest side is at most 768p. During inference, the input frames are downsampled to 480p. We set $\lambda_{\text{emb}}$, $\lambda_{\text{cls}}$, $\lambda_{\text{ce}}$, $\lambda_{\text{dice}}$ as 2.0, 2.0, 5.0 and 5.0, respectively. The mini-batch size is 16 and the maximum training iterations is 16,000. The initial learning rate is 0.0001 and decays at 6,000 and 12,000 iterations, respectively. 

\subsection{Main Result}
As shown in Table~\ref{tab:main}, we compare CTVIS against SOTA methods \cite{masktrackrcnn, sipmask, crossvis, ifc, mask2formervis, tevit, seqformer,minvis,idol,vita}, respectively using ResNet-50 \cite{resnet} and Swin-L \cite{swin} as the backbone on three benchmarks.

\noindent\textbf{YTVIS19 \& YTVIS21.} consist of relatively simple videos with short durations. Thanks to the introduced consistent learning paradigm and the extracted discriminative embeddings, CTVIS outperforms recent best methods on AP by $\sim5\%$ with ResNet-50 on both benchmarks. With the stronger backbone Swin-L, CTVIS surpasses the second best by $3.7\%$ on YTVIS21. Compared with IDOL\cite{idol}, CTVIS considerably improves the performance in terms of all metrics with tolerable parameter overheads. 

\noindent\textbf{OVIS.} This dataset contains longer videos and more intricate contents, on which online methods \cite{idol, minvis} perform much better than offline models \cite{vita, mask2former, seqformer}. 
Thanks to the effective embedding learning with long video samples, CTVIS gains $5.3$ and $4.3$ points in terms of AP, taking as inputs ResNet-50 and Swin-L, respectively. To summarize, CTVIS is highly competitive on benchmarks with varying complexities.

\subsection{Ablation Study}
We conduct extensive ablation to verify the effectiveness of CTVIS. Unless specified otherwise, we take the ResNet-50 as the backbone and train models under the CJT setting. Here we report AP$^{\mathtt{YV19}}$ and AP$^{\mathtt{OVIS}}$ on YTVIS19 and OVIS.

\noindent\textbf{Do improvements mainly come from better image-level instance segmentation models?} 
The answer is no. We validate this in Table~\ref{tab:detector}: 
1) Compared with IDOL with Deformable DETR, IDOL with Mask2Former is 1.0 and 1.5 points higher, suggesting the influence of a better detector is not that significant; 
2) Since our CTVIS is not restricted to a specific network, we implement Deformable DETR with CTVIS, which brings 4.2  and 3.6 points of AP gains. Similarly, CTVIS on Mask2Former also boosts the results by 3.9 and 3.8 points, which indicates that the improvements mainly come from our proposed CTVIS. 

\begin{table}[t]
\vspace{-2mm}
  \centering
  \resizebox{0.9\columnwidth}{!}{%
    \begin{tabular}{c|ll|ll}
    \toprule
    \multirow{2}[2]{*}{Methods} & \multicolumn{2}{c|}{Deformable DETR$^*$ \cite{idol}} & \multicolumn{2}{c}{Mask2Former \cite{mask2former}} \\
                           & AP$^{\mathtt{YV19}}$     & AP$^{\mathtt{OVIS}}$        & AP$^{\mathtt{YV19}}$     & AP$^{\mathtt{OVIS}}$ \\
    \midrule
    IDOL\cite{idol}        & 49.5        & 30.2        & 51.2        & 31.7 \\
    CTVIS                  &  \textbf{53.7} \blue{(+4.2)}        &  \textbf{33.8} \blue{(+3.6)}        &  \textbf{55.1} \blue{(+3.9)}        &  \textbf{35.5} \blue{(+3.8)} \\
    \bottomrule
  \end{tabular}%
  }
  \vspace{-1mm}
  \caption{Comparison of different instance segmentation methods with IDOL and CTVIS, respectively. 
  Deformable DETR$^*$ is extended to instance segmentation as suggested in \cite{idol}. }  
  \label{tab:detector}%
  \vspace{-1mm}
\end{table}%

% Figure environment removed 

\noindent\textbf{Long-video training.}
To verify the effectiveness of long-video training, we ablate the number of frames of each video used for training. For a fair comparison, we extend IDOL\cite{idol} to a multiple references (MR) version, by replacing its segmentor with the stronger Mssk2Former and using multiple reference frames.  Figure~\ref{fig:ablate_num_frame} shows the results. Thanks to the CI construction method employed by CTVIS, the performance has seen a dynamic increase by using more frames (peaked at 8 and 10 frames). In comparison, MR cannot benefit from long-video training and even degrades on OVIS. Hence we conclude that the performance of CTVIS stems from the effective video-level embedding learning (for tracking), rather than training an enhanced instance segmentor with larger batch sizes (more images per batch).

\noindent\textbf{Components of CTVIS.} First, removing all components of CTVIS sets a baseline, which utilizes a single reference to learn embeddings in a frame-by-frame way. As shown in Table~\ref{tab:ctvis}, the baseline gets 51.6 and 32.6 on YTVIS19 and OVIS. Based on this baseline, we gradually add CTVIS components: 1) We take the latest embedding of each instance to build CIs (instead of MA embeddings), which improves AP$^{\mathtt{YV19}}$ and AP$^{\mathtt{OVIS}}$ to 52.1 and 33.3. This suggests that the sampling domain CIs do indeed influence the instance embedding learning; 2) When MA is incorporated, the results see salient increases (52.1 \vs 54.2 and 33.3 \vs 34.9), which indicates that our CI-building method renders the embedding learning more stable and consistent; 3) When incorporating noise in the memory bank, which is designed to alleviate the ID switch issue, the performance sees non-trivial increases (0.9 and 0.6 on two datasets). Put all components together, CTVIS obtains remarkable results on both datasets and outperforms the strong baseline by 3.5 and 2.9 points, which validates the significance of the temporal alignment between training and inference pipelines, at least for VIS. 

\noindent\textbf{Sampling of $\mathbf{k}^-$.} 
% Recall that CTVIS samples major embedding from the MA embedding from the memory bank and supplementary embeddings from background embeddings from previous frames.  
We test different ways of building the negative embeddings $\mathbf{k}^-$. Table~\ref{tab:negative} presents four configurations and the corresponding results. Recall that the supplementary negative embeddings represent the background, and training with such negative samples only corrupts the performance (the 1st row). On the other hand, using major negative samples only gives decent results. A conjunctive usage of both negative-sampling types improves the performance significantly. In this line, we further consider sampling supplementary negative instances from either the local (sampled from the preceding frame only) or global domain (sampled from all previous frames). We found that the local setting gives the best results. This is probably because the model only needs to check the background in the local domain during inference. Hereafter we simply use the local setting.



% \begin{table*}[t]
%     \vspace{-.2em}
%     \centering
%   % \resizebox{0.3\columnwidth}{!}{%
%     \subfloat[
%     \textbf{Effect of the Components of CTVIS.}
%     \label{tab:ctvis}
%     ]{
%     \begin{minipage}{0.29\linewidth}{\begin{center}
%     \centering
%     \begin{tabular}{ccc|cc}
%     \toprule
%     Memory Bank & Momentum & Noise Training & AP$^{\mathtt{YV19}}$ & AP$^{\mathtt{OVIS}}$     \\
%     \midrule
%                      &                  &                  & 51.6  & 32.6                     \\
%     $\checkmark$     &                  &                  & 52.1  & 33.3                     \\
%     $\checkmark$     & $\checkmark$     &                  & 54.2  & 34.9                     \\
%     $\checkmark$     & $\checkmark$     & $\checkmark$     & \textbf{55.1}  & \textbf{35.5}   \\
    
%     \bottomrule
%     \end{tabular}%
%     \end{center}}\end{minipage}
%     }
%     \hspace{2em}
%     \subfloat[
%     \textbf{Ablation study of negative embeddings sampling strategy. }
%     \label{tab:negative}%
%     ]{
%     \begin{minipage}{0.29\linewidth}{\begin{center}
%     \centering
%     \begin{tabular}{cc|cc}
%     \toprule
%     Major       & Supplementary          & AP$^{\mathtt{YV19}}$ & AP$^{\mathtt{OVIS}}$ \\
%     \midrule
%                 & $\checkmark$           &   16.5      &  0.5  \\
%     $\checkmark$           &             &   50.8	   &  31.6 \\
%     $\checkmark$           & global      &   54.6      &  33.4 \\
%     $\checkmark$           & local       &   55.1      &  35.5 \\
%     \bottomrule
%     \end{tabular}%
%     \end{center}}\end{minipage}
%   }
%   % }
%   % \caption{\textbf{Effect of the Components of CTVIS.}}
%   % \label{tab:ctvis}
% \end{table*}%

\begin{table}[t]
\vspace{-2mm}
  \centering
  \resizebox{0.85\columnwidth}{!}{%
    \begin{tabular}{ccc|cc}
    \toprule
    Memory bank & Momentum & Noise & AP$^{\mathtt{YV19}}$ & AP$^{\mathtt{OVIS}}$     \\
    \midrule
                     &                  &                  & 51.6  & 32.6                     \\
    $\checkmark$     &                  &                  & 52.1  & 33.3                     \\
    $\checkmark$     & $\checkmark$     &                  & 54.2  & 34.9                     \\
    $\checkmark$     & $\checkmark$     & $\checkmark$     & \textbf{55.1}  & \textbf{35.5}   \\
    
    \bottomrule
    \end{tabular}%
  }
  \vspace{-1mm}
  \caption{Effectiveness of different CTVIS components.}
  \label{tab:ctvis}
\end{table}%

\begin{table}[t]
\vspace{-2mm}
  \centering
  \resizebox{0.7\columnwidth}{!}{%
    \begin{tabular}{cc|cc}
    \toprule
    Major       & Supplementary          & AP$^{\mathtt{YV19}}$ & AP$^{\mathtt{OVIS}}$ \\
    \midrule
                & $\checkmark$           &   16.5      &  0.5  \\
    $\checkmark$           &             &   50.8	   &  31.6 \\
    $\checkmark$           & global      &   54.6      &  33.4 \\
    $\checkmark$           & local       &   \textbf{55.1}      &  \textbf{35.5} \\
    \bottomrule
    \end{tabular}%
  }
  \vspace{-1mm}
  \caption{Ablate the sampling strategy of negative embeddings.} 
  \vspace{-3mm}
  \label{tab:negative}%
\end{table}%

\subsection{Pseudo Video as Training Example}
\label{sec:exp_pseudo}
We train VIS models on pseudo-videos, which are created with COCO images and the method described in Section~\ref{sec:pseudo}. Since COCO classes do not match that of VIS datasets, we only adopt the overlapping categories for training. For evaluation, we sample 421 and 140 videos with overlapping categories from the train sets of YTVIS21 and OVIS train sets, respectively. For more dataset information, please refer to the supplementary material. Specially, we denote the sampled version of YTVIS21 and OVIS as YTVIS21$^*$ and  OVIS$^*$. We use Swin-L as the backbone, and investigate the impacts of augmentation techniques in terms of generating pseudo-video datasets for training. Here \emph{rotation} is taken as the baseline. As shown in Table~\ref{tab:augs}, both \emph{crop} and \emph{copy\&paste} bring gains on both datasets over the baseline. Because YTVIS21 is relatively simple, \emph{crop} and \emph{copy\&paste} only improve the results by $0.2$ and $0.5$, respectively. However, for the complicated OVIS, they offer much larger performance gains, \ie $1.3$ and $2.0$ on two datasets, which suggests that pseudo videos generated with stronger augmentations are especially suitable to tackle complicated VIS tasks. We also train VITA and IDOL models using the generated pseudo-samples. Again, CTVIS surpasses them by clear margins, as that demonstrated in Table~\ref{tab:pseudo_sota}.

\subsection{Training with Limited Supervision}
Following MinVIS\cite{minvis}, we train CTVIS and MinVIS models on only a proportion ($\%$) of VIS training set. Specifically, we sample 1\%, 5\%, 10\%, and 100\% frames respectively from the training set to create pseudo videos for training. As shown in Table~\ref{tab:st_img}, with a 5\% proportion, CTVIS outperforms MinVIS with 100\% samples on all datasets.  More importantly, CTVIS trained with pseudo videos, which are created from 100\% frame samples, even surpasses the fully supervised competitors, and achieves close performance compared with CTVIS learned from genuine videos. 

\begin{table}[t]
  \centering
  \resizebox{0.8\columnwidth}{!}{%
    \begin{tabular}{ccc|cc}
    \toprule
        Rotation  &Crop  & Copy\&Paste       & AP$^{\mathtt{YV21^*}}$ & AP$^{\mathtt{OVIS^*}}$ \\
    \midrule
    $\checkmark$ &  &                                      & 48.5             & 27.3          \\
    $\checkmark$ & $\checkmark$  &                         & 48.7             & 28.6           \\
    $\checkmark$ &  & $\checkmark$                         & 49               & 29.3    \\
    $\checkmark$ & $\checkmark$ & $\checkmark$             & \textbf{49.7}             &\textbf{30.5}    \\
    \bottomrule
  \end{tabular}%
  }
  \vspace{-1mm}
  \caption{Influence of augmentations on producing pseudo-videos.}
  \label{tab:augs}%

\end{table}%

% Please add the following required packages to your document preamble:
% \usepackage{graphicx}
\begin{table}[t]
\vspace{-1mm}
\centering
\resizebox{0.8\columnwidth}{!}{%
\begin{tabular}{c|c|cc}
\toprule
    Methods &   Supervision &  AP$^{\mathtt{YV21^*}}$ & AP$^{\mathtt{OVIS^*}}$ \\ 
    \midrule
    MinVIS \cite{minvis}  &  Image  & 43.9             & 24.4 \\
    VITA   \cite{vita}    &  Pseudo video & 44.4             & 19.1 \\
    IDOL   \cite{idol}    &  Pseudo image pair& 47.8             & 27.8 \\
    CTVIS                 &  Pseudo video & \textbf{49.7}             & \textbf{30.5} \\
    \bottomrule
\end{tabular}%
}
\vspace{-1mm}
\caption{Compare with SOTA models trained with pseudo-samples, which are generated based on COCO images.}
\label{tab:pseudo_sota}
\vspace{-3mm}
\end{table}

% Figure environment removed 

\begin{table}[t]
    % \vspace{-1mm}
    \centering
    \footnotesize
    \begin{tabular}{c|c|ccc}
    \toprule
    Methods                  & Training                 & AP$^{\mathtt{YV19}}$ & AP$^{\mathtt{YV21}}$ & AP$^{\mathtt{OVIS}}$ \\
    \midrule
    VITA \cite{vita}        & \multirow{3}{*}{Full} & 63           & 57.5         & 27.7      \\
    IDOL \cite{idol}        &                       & 64.3         & 56.1         & 42.6      \\
    \textbf{CTVIS (Ours)}   &                       & \textbf{65.6}         & \textbf{61.2}         & \textbf{46.9}      \\
    \midrule
    \multirow{4}{*}{MinVIS \cite{minvis}} 
                            & 1\%                   & 59           & 52.9         & 31.7 \\
                            & 5\%                   & 59.3         & 54.3         & 35.7 \\
                            & 10\%                  & 61           & 54.9         & 37.2 \\
                            & 100\%                 & \textbf{61.6}         & \textbf{55.3}         & \textbf{39.4} \\
    \midrule
    \multirow{4}{*}{\textbf{CTVIS (Ours)}}  
                            & 1\%                   & 62.4         & 57.8        & 36.2 \\
                            & 5\%                   & 63.4         & 59.4         & 41.9 \\
                            & 10\%                  & 64.2         & 60.0         & 42.1 \\
                            & 100\%                 & \textbf{64.8}         & \textbf{60.7}         & \textbf{44.1} \\
    \bottomrule
    \end{tabular}%
    \vspace{-1mm}
    \caption{Compare with SOTA models trained with either the entire or a part ($x\%$) of training examples. Full means training with annotated videos.}
    \label{tab:st_img}
    \vspace{-3mm}
\end{table}


\subsection{Qualitative Results}
We visualize some VIS results obtained by SOTA offline\cite{vita} and online\cite{idol} approaches in Figure~\ref{fig:video}. The left example includes heavy occulusion caused by moving pedestrian, the swap of instance positions, and target-disappearing-reappearing. Under such case, VITA \cite{vita} fails to segment and track the pedestrian. IDOL\cite{idol} mistakenly assigns the ID of the dog in the two rightmost images, and the squatting person is recognized as a dog. In comparison, our proposed CTVIS is able to segment, classify and track all instances successfully.
%
For the right example, both VITA and IDOL fail to track the fish, and their ID switched after the video suddenly darkened. CTVIS also undergoes and ID switch (the middle image). Thanks to the noise introduced during training, CTVIS is more robust to tackle such occasional failure, and it reidentifies the fish later (the rightmost image).

\section{Conclusion}

In this paper, we proposed \ourmethod{}, an adaptation of diffusion models for motion synthesis which entangles the motion temporal-axis with the diffusion time-axis. This mechanism enables synthesizing arbitrarily long motion sequences in an autoregressive manner using a U-Net architecture. A unique aspect of our work is the notion of a \textit{stationary} motion buffer. Our framework continues to produce clean frames (i.e., progressing along the diffusion-time axis), without \textit{actually} incrementing the diffusion time.
The ability of our pipeline to continually generate motion along the diffusion axis is what enables our framework to robustly and continuously produce novel frames. Interestingly, the ability to naturally use diffusion in such an autoregressive fashion may have implications for other types of sequential data beyond motion, such as audio and video, or modalities where a sequential order can be defined, such as a patch-by-patch order for images.

Our system enables partially-clean-frame to be immediately (or near immediately) popped-off the motion buffer stack. However, a current limitation of our system is that computing a clean from from pure noise requires going through the chain of denoising diffusion.
In the future we are interested in leveraging ideas from DDIM~\cite{song2020denoising} to skip ahead during the denoising process to achieve even lower latency. In addition, our framework may enable future research in long-term text-conditioned motion generation. We are interested in exploring how high-level control may be coupled with low-level user-guidance for the task of long-term generation.






\bibliography{iclr2024_conference}
\bibliographystyle{iclr2024_conference}

\clearpage
\appendix
% \section{Appendix}

\section{Appendix A: Model Hyperparameters}
\label{sec:appenda}


\textbf{Model Hyperparameters.} We provide a detailed set of hyperparameters used to obtain the experiment results, shown in Table \ref{tbl:hyper}.

\begin{table}[h]
\centering
\renewcommand{\arraystretch}{1.25}
\begin{tabular}{cccc}
\hline\hline
Model                         & Hyperparameters & Value      & Model Parameters      \\ \hline\hline
\multirow{2}{*}{PINNs \& FLS} & hidden layer    & 4          & \multirow{2}{*}{527k} \\
                              & hidden size     & 512        &                       \\ \hline
\multirow{2}{*}{QRes}         & hidden layer    & 4          & \multirow{2}{*}{397k} \\
                              & hidden size     & 256        &                       \\ \hline
\multirow{7}{*}{PINNsFormer}       & $k$               & 5          & \multirow{7}{*}{454k} \\
                              & $\Delta t$               & 1e-3, 1e-4 &                       \\
                              & \# of encoder   & 1          &                       \\
                              & \# of decoder   & 1          &                       \\
                              & embedding size  & 32         &                       \\
                              & head            & 2          &                       \\
                              & hidden size     & 512        &                       \\ \hline \hline
\end{tabular}
\caption{Hyperparameters for Main Results}
\label{tbl:hyper}
\end{table}

% \begin{table}[h]
% \centering
% \renewcommand{\arraystretch}{1.5}
% \begin{tabular}{c|cc}
% Model                   & Hyparameter    & Value \\ \hline
% \multirow{2}{*}{MLP}    & hidden layers & 4     \\ 
%                         & hidden size    & 512   \\ \hline
% \multirow{7}{*}{Trans.} & $k$            & 5     \\
%                         & $\Delta t$     & 1e-4  \\
%                         & encoder        & 1    \\
%                         & decoder        & 1    \\
%                         & embedding size & 32    \\
%                         & heads          & 2     \\
%                         & hidden size    & 512  
% \end{tabular}
% \caption{Hyperparameters for Main Results}
% \label{tbl:hyper}
% \end{table}


\textbf{Training Overhead.} We compare the training overhead of PINNsFormer over PINNs, as PINNs are known as an efficient framework while Transformer-based models are known for being computationally costly. The comparison relies on solving the Convection PDEs, which are detailed in Table \ref{tbl:overhead}. Here, we vary the hyperparameter of pseudo-sequence length $k$ for validation purposes. In practice, we set $k=5$ for all the empirical experiments in this paper.

\begin{table}[h]
\centering
\renewcommand{\arraystretch}{1.25}
\begin{tabular}{cccccc}
\hline\hline
\multicolumn{2}{c}{Model}      & \begin{tabular}[c]{@{}c@{}}Training Time\\ (sec/epoch)\end{tabular} & \begin{tabular}[c]{@{}c@{}}Computational\\ Overhead\end{tabular} & \begin{tabular}[c]{@{}c@{}}GPU Memory\\ (MiB)\end{tabular} & \begin{tabular}[c]{@{}c@{}}Memory\\ Overhead\end{tabular} \\ \hline \hline
\multicolumn{2}{c}{PINNs}      & 0.80                                                                & /                                                                & 1311                                                       & /                                                         \\ \hline
\multirow{3}{*}{PINsFormer} & $k$=3  & 2.10                                                                & 2.62x                                                            & 2207                                                       & 1.68x                                                     \\
                        & $k$=5  & 2.34                                                                & 2.92x                                                            & 2827                                                       & 2.15x                                                     \\
                        & $k$=10 & 3.10                                                                & 3.87x                                                            & 4803                                                       & 3.66x                                                     \\ \hline\hline
\end{tabular}
\caption{Overhead of PINNsFormer than PINNs in varying pseudo-sequence length. Both computational and memory overhead are tolerable and grow approximately linearly as $k$ increases}
\label{tbl:overhead}
\end{table}

\textbf{Evaluation Metrics.} We present the detailed formula of rMAE and rRMSE as the following:
\begin{equation}
\begin{gathered}
    \texttt{rMAE} =  \frac{\sum_{n=1}^N |\hat{u}(x_n,t_n)-u(x_n,t_n)|}{\sum_{n=1}^{N_{\textit{res}}}|u(x_n,t_n)|}\\
    \texttt{rRMSE} = \sqrt{\frac{\sum_{n=1}^N |\hat{u}(x_n,t_n)-u(x_n,t_n)|^2}{\sum_{n=1}^N|u(x_n,t_n)|^2}}
\end{gathered}    
\end{equation}
where $N$ is the number of testing points, $\hat{u}$ is the neural network approximation, and $u$ is the ground truth.
\section{Appendix B: PDEs setups}
\label{sec:appendb}

We provide detailed PDE setups for convection, reaction-diffusion, and 1D-reaction equations.

\textbf{Convection PDE.} The one-dimensional convection problem is a hyperbolic PDE that is commonly used to model transport phenomena. The system has the formulation with periodic boundary conditions as follows:
\begin{equation}
\begin{gathered}
    \frac{\partial u}{\partial t} + \beta \frac{\partial u}{\partial x} = 0, \:\: \forall x\in [0,2\pi], \: t\in [0,1] \\
    \texttt{IC:} u(x,0)=\sin(x), \:\:\: \texttt{BC:} u(0,t)=u(2\pi,t)
\end{gathered}    
\end{equation}

where $\beta$ is the convection coefficient. As $\beta$ increases, the frequency of its solution goes higher, and it becomes harder for PINNs to approximate. Here, we set $\beta=50$.

\textbf{1D-Reaction PDE.} The one-dimensional reaction problem is a hyperbolic PDE that is commonly used to model chemical reactions. The system has the formulation with periodic boundary conditions as follows:
\begin{equation}
\begin{gathered}
    \frac{\partial u}{\partial t} - \rho u(1-u) = 0, \:\: \forall x\in [0,2\pi], \: t\in [0,1] \\
    \texttt{IC:} u(x,0)=\exp(-\frac{(x-\pi)^2}{2(\pi/4)^2}), \:\:\: \texttt{BC:} u(0,t)=u(2\pi,t)
\end{gathered}    
\end{equation}

where $\rho$ is the reaction coefficient. Here, we set $\rho=5$. The equation has a simple analytical solution:
\begin{equation}
    u_{\texttt{analytical}} = \frac{h(x) \exp(\rho t)}{h(x)\exp(\rho t)+1-h(x)}
\end{equation}

where $h(x)$ is the function of the initial condition.

% \textbf{Reaction-Diffusion PDE.} The reaction-diffusion system is where a diffusion operator is added to the reaction equation above. The system has the formulation with periodic boundary conditions as follows:
% \begin{equation}
% \begin{gathered}
%     \frac{\partial u}{\partial t} - \nu \frac{\partial^2 u}{\partial x^2} - \rho u(1-u)= 0, \:\: \forall x\in [0,2\pi], \: t\in [0,1] \\
%     \texttt{IC:} u(x,0)=\exp(-\frac{(x-\pi)^2}{2(\pi/4)^2}), \:\:\: \texttt{BC:} u(0,t)=u(2\pi,t)
% \end{gathered}    
% \end{equation}

% where $\nu>0$ is the diffusion coefficient. Here, we set $\rho=5$ and $\nu=5$. The solution of the system can be solved via Strang splitting, i.e., splitting the equation into two separate models (a reaction component and a diffusion component):
% \begin{equation}
% \begin{gathered}
%     \frac{\partial u}{\partial t} = \rho u (1-u) \\
%     \frac{\partial u}{\partial t} = \frac{\partial^2 u}{\partial x^2}
% \end{gathered}    
% \end{equation}

\textbf{1D-Wave PDE.} The 1D-Wave equation is a hyperbolic PDE that is used to describe the propagation of waves in one spatial dimension. It is often used in physics and engineering to model various wave phenomena, such as sound waves, seismic waves, and electromagnetic waves. The system has the formulation with periodic boundary conditions as follows:
\begin{equation}
\begin{gathered}
    \frac{\partial^2 u}{\partial t^2} - \beta \frac{\partial^2 u}{\partial x^2} = 0 \, \:\: \forall x\in [0,1], \: t\in [0,1] \\
    \texttt{IC:} u(x,0)=\sin (\pi x) + \frac{1}{2}\sin(\beta\pi x), \:\: \:\frac{\partial u(x,0)}{\partial t}  =0 \\
    \texttt{BC:} u(0,t)=u(1,t) = 0
\end{gathered}    
\end{equation}
where $\beta$ is the wave speed. Here, we are specifying $\beta=3$.The equation has a simple analytical solution:
\begin{equation}
\begin{gathered}
    u(x,t) = \sin (\pi x) \cos(2\pi t) + \frac{1}{2}\sin(\beta \pi x)\cos(2\beta\pi t)
\end{gathered}
\end{equation}

\textbf{2D Navier-Stokes PDE.} The 2D Navier-Stokes equation is a parabolic PDE that consists of a pair of partial differential equations that describe the behavior of incompressible fluid flow in two-dimensional space. They are widely used in fluid dynamics to model the motion of fluids, such as air and water, in various engineering and scientific applications. The system has the formulation as follows: 
\begin{equation}
\begin{gathered}
    \frac{\partial u}{\partial t} + \lambda_1 (u\frac{\partial u}{\partial x} + v \frac{\partial u}{\partial y}) = - \frac{\partial p}{\partial x} + \lambda_2 (\frac{\partial^2 u}{\partial x^2} + \frac{\partial^2 u}{\partial v^2}) \\
    \frac{\partial v}{\partial t} + \lambda_1 (u\frac{\partial v}{\partial x} + v \frac{\partial v}{\partial y}) = - \frac{\partial p}{\partial y} + \lambda_2 (\frac{\partial^2 u}{\partial x^2} + \frac{\partial^2 u}{\partial v^2})
\end{gathered}
\end{equation}
where $u(t,x,y)$ and $v(t,x,y)$ are the $x$-component and $y$-component of the velocity field separately, and $p(t,x,y)$ is the pressure. Here, we set $\lambda_1 = 1$ and $\lambda_2 = 0.01$. The system does not have an explicit analytical solution, while the simulated solution is given by~\cite{raissi2019physics}.
\section{Appendix C: Additional Results}
\label{sec:appendc}

\textbf{Ablation Study on Activation Functions.} To investigate the effectiveness of the Wavelet activation function in PINNsFormer, we compare the performance differences using Wavelet than ReLU, Sigmoid, and Sin activation functions over convection and 1D-reaction problems. In particular, we study the effects of using the same activation function in both the feed-forward layer and encoder/decoder layer (marked as ReLU, etc.) and changing the activation function of the encoder/decoder layer to LayerNorm (as vanilla Transformer does, marked as ReLU+LN, etc.). The evaluation results are shown in Table~\ref{tbl:ablation}.

{\small
\begin{table}[h]
% \vspace{-0.1in}
\centering
\renewcommand{\arraystretch}{1.1}
\begin{tabular}{ccccccc}
\hline\hline
\multirow{2}{*}{Activation} & \multicolumn{3}{c}{Convection} & \multicolumn{3}{c}{1D-Reaction} \\
& Loss & rMAE & rRMSE & Loss & rMAE & rRMSE \\ \hline
ReLU & 0.5256 & 1.001 & 1.001 & 0.2083 & 0.994 & 0.996 \\
Sigmoid & 0.1618 & 1.112 & 1.223 & 0.1998 & 0.991 & 0.993 \\
Sin & 0.3159 & 1.074 & 1.141 & 4.9e-6 & 0.017 & 0.032 \\
ReLU+LN & 0.7818 & 1.001 & 1.002 & 0.2028 & 0.992 & 0.993 \\
Sigmoid+LN & 0.0549 & 0.941 & 0.967 & 0.2063 & 0.992 & 0.990 \\
Sin+LN & 0.3219 & 1.083 & 1.156 & 4.7e-6 & 0.016 & 0.033 \\
Wavelet & \textbf{3.7e-5} & \textbf{0.023} & \textbf{0.027} & \textbf{3.0e-6} & \textbf{0.015} & \textbf{0.030} \\
Wavelet+LN & NaN & NaN & NaN & 3.9e-6 & 0.018 & 0.037 \\ \hline\hline
\end{tabular}
\caption{Results for solving convection and 1D-reaction equations using Transformer architecture with different activation functions. PINNsFormer (with Wavelet activation) consistently outperforms all other activation functions in terms of training loss, rMAE, and rRMSE}
\label{tbl:ablation}
% \vspace{-0.2in}
\end{table}
}

The ablation study results show two major conclusions: First, using wavelet activation shows constantly better performance than ReLU, Sigmoid, and Sin activations. In particular, Sin activation may show effectiveness in only certain cases, while Wavelet can generalize all cases well. Second, Introducing LayerNorm activation to the encoder/decoder does not significantly contribute to performance improvement. In contrast, LayerNorm activation may cause convergence issues when coupling with the Wavelet activation function for certain situations.

\textbf{Hyperparameter Sensitivity Study.} To investigate the possible difficulties in picking hyperparameters $k$ and $\Delta$, we compared the performance differences with a mesh choice of these two hyperparameters over the 1d-reaction problem. The evaluation results (relative-$\ell_2$ error, with failure modes bolded) are shown in Table~\ref{tbl:hyper}.

{\small
\begin{table}[h]
% \vspace{-0.1in}
\centering
\renewcommand{\arraystretch}{1.1}
\begin{tabular}{ccccc}
\hline \hline
$\Delta t$   & k=3   & k=5   & k=7   & k=10  \\ \hline
1e-1 & 0.044 & \textbf{0.514} & \textbf{0.743} & \textbf{0.731} \\
1e-2 & 0.029 & 0.035 & 0.045 & 0.049 \\
1e-3 & 0.037 & 0.037 & 0.024 & 0.035 \\
1e-4 & \textbf{0.997} & 0.030 & 0.029 & 0.046 \\
1e-5 & \textbf{0.977} & 0.026 & \textbf{0.977} & 0.021 \\ \hline \hline
\end{tabular}
\caption{Results for solving 1D-reaction equation with various combinations of $\Delta t$ and $k$. PINNsFormer shows the flexibility of a wide choice of hyperparameters on certain problems.}
\label{tbl:hyper}
% \vspace{-0.2in}
\end{table}
}

The study on hyperparameter sensitivity of $\Delta t$ and $k$ exhibits three intuitions: First, given a mesh choice of $k$ and $\Delta t$, PINNsFormer is not sensitive to a wide range of the two hyperparameters. For instance, PINNsFormer successfully mitigates the failure modes for any combinations of $k\in[1e-2, 1e-3, 1e-4]$ and $\Delta t\in[3,5,7]$. Second, the choice of $\Delta t$ should not be either too large (i.e., 1e-1) or too small (i.e., 1e-5). Intuitively, either a too-large or a too-small $\Delta t$ degrades the temporal dependencies between discrete time steps. Third, increasing the pseudo-sequence length can help mitigate PINNs failure modes (i.e., $k=3\rightarrow5$ when $\Delta t=1e-4$). However, once PINNs successfully mitigate the failure mode, the benefit of further increasing $k$ is marginal.


\textbf{Result Visualizations.} We here present the plots of ground truth solutions, neural network predictions, and absolute errors for all evaluations included in the experimental section. The plots on convection, 1D-reaction, 1D-wave, and 2D Navier-Stokes equations are shown in Figure separately.

% Figure environment removed


% Figure environment removed


% Figure environment removed


% Figure environment removed

\end{document}
