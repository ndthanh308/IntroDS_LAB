\documentclass{article}


% if you need to pass options to natbib, use, e.g.:
%     \PassOptionsToPackage{numbers, compress}{natbib}
% before loading neurips_2023


% ready for submission
% \usepackage{neurips_2023}
\usepackage{amsmath}
\usepackage{amssymb}
\usepackage{multirow}
\usepackage{siunitx}
\usepackage{graphicx}
\usepackage{mathtools}
\usepackage{wrapfig}
\usepackage{subcaption}

\newtheorem{theorem}{Theorem}
\newtheorem{proposition}{Proposition}

\newcommand{\solution}[0]{PINNsFormer\xspace}

% to compile a preprint version, e.g., for submission to arXiv, add add the
% [preprint] option:
\usepackage[nonatbib, preprint]{neurips_2023}


% to compile a camera-ready version, add the [final] option, e.g.:
    % \usepackage[final]{neurips_2023}


% to avoid loading the natbib package, add option nonatbib:
% \usepackage[nonatbib]{neurips_2023}


\usepackage[utf8]{inputenc} % allow utf-8 input
\usepackage[T1]{fontenc}    % use 8-bit T1 fonts
\usepackage{hyperref}       % hyperlinks
\usepackage{url}            % simple URL typesetting
\usepackage{booktabs}       % professional-quality tables
\usepackage{amsfonts}       % blackboard math symbols
\usepackage{nicefrac}       % compact symbols for 1/2, etc.
\usepackage{microtype}      % microtypography
\usepackage{xcolor}         % colors


\title{PINNsFormer: A Transformer-Based Framework For Physics-Informed Neural Networks}


% The \author macro works with any number of authors. There are two commands
% used to separate the names and addresses of multiple authors: \And and \AND.
%
% Using \And between authors leaves it to LaTeX to determine where to break the
% lines. Using \AND forces a line break at that point. So, if LaTeX puts 3 of 4
% authors names on the first line, and the last on the second line, try using
% \AND instead of \And before the third author name.


\author{%
  Zhiyuan Zhao\\
  % \thanks{Use footnote for providing further information about author (webpage, alternative address)---\emph{not} for acknowledging funding agencies.} \\
  % Department of Computer Science\\
  Georgia Institute of Technology\\
  Atlanta, GA 30332 \\
  \texttt{leozhao1997@gatech.edu} \\
  % examples of more authors
  \And
  Xueying Ding \\
  Carnegie Mellon University \\
  Pittsburgh, PA 15213\\
  \texttt{xding2@andrew.cmu.edu} \\
  \AND
  B. Aditya Prakash \\
  Georgia Institute of Technology \\
  Atlanta, GA 30332 \\
  \texttt{badityap@cc.gatech.edu} \\
  % \And
  % Coauthor \\
  % Affiliation \\
  % Address \\
  % \texttt{email} \\
  % \And
  % Coauthor \\
  % Affiliation \\
  % Address \\
  % \texttt{email} \\
}


\begin{document}

\maketitle


\begin{abstract}
  Physics-Informed Neural Networks (PINNs) have emerged as a promising deep learning framework for approximating numerical solutions for partial differential equations (PDEs). While conventional PINNs and most related studies adopt fully-connected multilayer perceptrons (MLP) as the backbone structure, they have neglected the temporal relations in PDEs and failed to approximate the true solution.  In this paper, we propose a novel Transformer-based framework, namely PINNsFormer, that accurately approximates PDEs' solutions by capturing the temporal dependencies with multi-head attention mechanisms in Transformer-based models. Instead of approximating point predictions, PINNsFormer adapts input vectors to pseudo sequences and point-wise PINNs loss to a sequential PINNs loss. In addition, PINNsFormer is equipped with a novel activation function, namely \texttt{Wavelet}, which anticipates the Fourier decomposition through deep neural networks. We empirically demonstrate PINNsFormer's ability to capture the PDE solutions for various scenarios, in which conventional PINNs have failed to learn. We also show that PINNsFormer achieves superior approximation accuracy on such problems than conventional PINNs with non-sensitive hyperparameters, in trade of marginal computational and memory costs, with extensive experiments.
\end{abstract}


\section{Introduction}
Deep learning models have been widely used in many applications.
For example, BERT~\citep{devlin_bert_2019}, GPT-3~\citep{brown_language_2020}, and T5~\citep{raffel_exploring_2020} achieved state-of-the-art~(SOTA) results on different natural language processing~(NLP) tasks. 
For computer vision~(CV), Transformer-like models such as ViT~\citep{dosovitskiy_image_2021} and Swin Transformer~\citep{liu_swin_2021} deliver excellent accuracy performance upon multiple tasks. 


At the same time, training deep learning models has been a critical problem troubling the community due to the long training time, especially for those large models with billions of parameters~\citep{brown_language_2020}. 
In order to enhance the training efficiency, researchers propose some manually designed parallel training strategies~\citep{narayanan_efficient_2021,shazeer_mesh-tensorflow_2018,xu_gspmd_2021}. 
However, selecting, tuning, and combining these strategies require extensive domain knowledge in deep learning models and hardware environments. With the increasing diversity of modern hardware architectures~\cite{flynn_very_1966,flynn_computer_1972} and the rapid development of deep learning models, these manually designed approaches are bringing heavier burdens to developers. 
Hence, \emph{automatic parallelism} is introduced to automate the parallel strategy searching for training models.


There are two main categories of parallelism in deep learning models: inter-layer parallelism~\citep{huang_gpipe_2019,narayanan_pipedream_2019,narayanan_memory-efficient_2021,fan_dapple_2021,li_chimera_2021,lepikhin_gshard_2021,du_glam_2022,fedus_switch_2022} and intra-layer parallelism~\citep{li_pytorch_2020,narayanan_efficient_2021,rasley_deepspeed_2020,fairscale_authors_fairscale_2021}. 
Inter-layer parallelism partitions the model into disjoint sets on different devices without slicing tensors. 
Alternatively, intra-layer parallelism partitions tensors in a layer along one or more axes and distributes them across different devices.


Current automatic parallelism techniques focus on optimizing strategies within these two categories. However, they treat these two categories separately. 
Some methods~\citep{zhao_vpipe_2022,jia_exploring_2018,cai_tensoropt_2022,wang_supporting_2019,jia_beyond_2019,schaarschmidt_automap_2021,liu_colossal-auto_2023} overlook potential opportunities for inter- or intra-layer parallelism, the others optimize inter- and intra-layer parallelism hierarchically and sequentially~\citep{narayanan_pipedream_2019,fan_dapple_2021,he_pipetransformer_2021,tarnawski_efficient_2020,tarnawski_piper_2021,zheng_alpa_2022}. 
As a result, current automatic parallelism techniques often fail to achieve the global optima and instead become trapped in local optima. 
Therefore, a unified inter- and intra-layer approach is needed to enhance the effectiveness of automatic parallelism.


This paper aims to find the optimal parallelism strategy while simultaneously considering inter- and intra-layer parallelism. 
It enables us to search in a more extensive strategy space where the globally optimal solution lurk. 
However, unifying inter- and intra-layer parallelism in automatic parallelism brings us two challenges. 
Firstly, to adopt a unified perspective on the inter- and intra-layer automatic parallelism, we should not formalize them with separate formulations as prior works. Therefore, how can we express these parallelism strategies in a unified formulation? 
Secondly, previous methods take a long time to obtain the solution with a limited strategy space. Therefore, how can we ensure that the best solution can be obtained in a reasonable time while expanding the strategy space?


To solve the above challenges, we propose UniAP. For the first challenge, UniAP adopts the mixed integer quadratic programming~(MIQP)~\citep{lazimy_mixed_1982} to search for the globally optimal parallel strategy automatically. 
It unifies the inter- and intra-layer automatic parallelism in a single MIQP formulation. 
For the second challenge, our complexity analysis and experimental results show that UniAP can obtain the globally optimal solution in a significantly shorter time.


The contributions of this paper are summarized as follows: 
\begin{itemize}
    \item We propose UniAP, the first framework to unify inter- and intra-layer automatic parallelism in model training.
    \item The optimal parallel strategies discovered by UniAP exhibit scalability on training throughput and strategy searching time.
    \item The experimental results show that UniAP speeds up model training on four Transformer-like models by up to 1.70$\times$ and reduces the strategy searching time by up to 16$\times$, compared with the SOTA method.
\end{itemize}

% \section{Preliminaries}
% \paragraph{Input feature attribution methods.}
% Consider a linear model $f(x) = w_1 x_1 + w_2 x_2$. To explain which feature is more important for predicting the value of f(x), we can compare their coefficients. If $w_1 = 1000$ and $w_2 = 0.01$, we can say that $x_1$ would be weighed more than $x_2$. This type of explanation assumes that the values of $x_1$ and $x_2$ are of the same order. This is true in the case of most inputs to the neural network models, for example image pixels. 
% Gradients are the general way of discussing the coefficient with respect to a particular feature to discuss its importance.
% \textbf{Element-wise product of gradient into input} \textsc{grad $\odot$ input} \cite{Shrikumar2016NotJA}, provides global importance about the input feature in the model's output. 
% \cite{} have used it show the feature importance in attention models. 
% , as compared to just the gradient.
% details of computing the attribution with math 
% Assume $\mathbf{x}$ is a real-valued input feature vector (for any modality). For discrete inputs, real-valued vector obtained after passing the feature through a look-up embedding.

% , but there is no clear superior attribution technique over another. 

% Instead of considering attributions over pixels, \textbf{XRAI} \cite{Kapishnikov2019XRAIBA} computes the effective attributions of integrated gradients over overly segmented image. The image is segmented based on similarity such as color, which makes the segment boundaries align with the edges. The segmentation is done at multiple scales to obtain a set of overlapping image segments.
% Assume that attribution mask over an image $I$ of size ${H\times W}$ is $A$ of the same size. 
% Using graph-based segmentations over multiple scale parameters, we obtain a set of segments $\mathcal{S}$. 
% Let a pixel be indexed by $i$ in the original image. For a segment $s$, the gain can be calculated by $g_s = \sum_{i \in s\backslash M} \frac{A_i}{area(s\backslash M)}$. 
% The segment with maximum gain is selected as  attribution to update the XRAI saliency set $\mathcal{M}$.
% The process is repeated with the remaining segments until the area of the mask set is equal to that of the image. 
% While this method seems to produce slightly better visual attributions over other variants of IG, it is sensitive to the size of segmentation scales and dilation factor. We consider  $XRAI(\cdot)$ to denote this attribution method for visual attribution analysis in \S \ref{subsec:visual_attr}.   
% which create grainy regions. 
% However, this method depends on the size of segmentation scales selected for computation. Further, dilation added to the final attribution masks to include edges may depict an inflated version of model's actual feature importance. 
% In this work, $XRAI(\cdot)$ denotes that this attribution method is applied.
%  \vspace{-0.5em}
\section{Related Work}
%  \vspace{-0.3em}
\label{sec:related_work}
\paragraph{Interpretability and explainability } Recent work in multimodal explainability in autonomous vehicles \cite{gilpin-2021-multimodal} uses symbolic explanations to debug and process outputs out of sub-components.
In contrast, we address the challenge of post-hoc multimodal interpretability for any existing end-to-end trained differentiable policies. \textsc{grad $\odot$ input}~\cite{Shrikumar2016NotJA},  a simple and modality-agnostic attribution that works on par with recent methods~\cite{Ancona2017AUV}. We use this method to compute multimodal attribution at inputs to the fusion layer to weigh how each modality contributes to the decision-making. 
% as it has been shown to work at par compared to the recent gradient-based attribution techniques~\cite{Ancona2017AUV}.
While \textsc{grad $\odot$ input} is a modality-agnostic starting point for attributions, 
it is not easy to understand, especially for images. Among recent works to improve visual attribution  \cite{Smilkov2017SmoothGradRN, Simonyan2014DeepIC, ig, sturmfels2020visualizing, Xu_2020_CVPR, Kapishnikov2021GuidedIG, Kapishnikov2019XRAIBA},  we use XRAI~\cite{Kapishnikov2019XRAIBA} for vision-specific analysis as it produces visually intuitive attributions by relying on regions, not individual pixels. 
% \cite{Smilkov2017SmoothGradRN} proposed ways to visually sharpen these vanilla gradient-based attributions. ~\cite{Simonyan2014DeepIC}  applying Gaussian noise perturbations over averaged over a sufficient number of samples.
% describe IG
% IG \cite{ig} and path methods have been studied as a cost-sharing method called Aumann-Shapley. 
% Attribution based on IG preserves axiomatic properties like \textit{sensitivity} and \textit{implementation invariance}.
% While IG aggregate the gradients on sampling inputs on a straight line between the baseline and the input, there are several paths possible in higher dimensional spaces and corresponding different attribution.
% Recent works build on IG to obtain more visually intuitive attributions, like SHAP Deep Explainer~\cite{sturmfels2020visualizing}, Blur IG ~\cite{Xu_2020_CVPR}, Guided IG~\cite{Kapishnikov2021GuidedIG} and XRAI~\cite{Kapishnikov2019XRAIBA}. Qualitatively, XRAI showed visually intuitive attributions by relying on regions and not individual pixels.  
% Interpretability using gradient-based attribution techniques is quite similar to adversarial attacks \cite{Goodfellow2015ExplainingAH} and adversarial training for robustness \cite{Bai2021RecentAI}, as both fundamentally rely on gradient of the input feature with respect to the output. 
% Do we need a figure to show the difference in attributions with just gradient vs gradxinput? 
\vspace{-0.8em}
\paragraph{Language-driven task benchmarks}

There are many benchmarks to study an agent’s ability to follow natural language instructions \cite{ALFRED20, padmakumar2022teach, gu2022vision,  mahmoudieh2022zero}. 
% While most existing settings apply only to either navigation \cite{} or manipulation \cite{}, 
% we conside one of the benchmarks which handles both, that is,
% navigation (Anderson et al., 2018; Chen
% et al., 2019), object manipulation (Misra et al.,
% 2017; Zhu et al., 2017) and embodied reasoning
% (Das et al., 2018a; Gordon et al., 2018). 
ALFRED \cite{ALFRED20} serves as a suitable testbed for this analysis as these tasks require both high reasoning for navigation and manipulation. ALFRED dataset provides visual demonstrations collected through PDDL planning in 3D Unity household environments and natural language description of the high-level goal and low-level instructions annotated by MTurkers. 
The benchmarks provide evaluation metrics for the overall task goal completion success rate (SR) and those weighted by the expert's path length (PLWSR)
% over seen and unseen tasks
and have reported a huge gap in the performance of learning algorithms and humans at these tasks. 
% ALFRED  is a benchmarking environment that provides natural language instructions annotated by MTurkers on egocentric visual sequences of actions taken for everyday household tasks. As ALFRED is a simulated environment on Unity3D game engine, the visual demonstrations are collected based on PDDL planning. 

\vspace{-0.8em}
\paragraph{End-to-end Learned Policies} We investigate the end-to-end learned policies for the task, such that, the gradient can be attributed at a task level. While we do not discuss modular yet differentiable policies like \cite{min2021film} \cite{DBLP:journals/corr/ZhouC15}, tying the gradient across multiple modular learned components is a direction for future work.
% as 
% tying the gradient among modular learned components in future work. 
In our work, we consider the checkpoints of policies trained on the ALFRED dataset. Broadly, these policies are of two types: (a) sequence-to-sequence models, that are, the one proposed with ALFRED dataset (Baseline) \cite{ALFRED20} and Modular Object-Centric Approach (MOCA) \cite{Singh2021FactorizingPA}, (b) transformer-based models, that are Episodic Transformers (ET) \cite{pashevich2021episodic}, and Hierarchical Tasks via Unified Transformers (HiTUT) \cite{Zhang2021HierarchicalTL}. Refer Table~\ref{tab:policiesarch} to compare architectural details \footnote{Previous action is modeled with learned embedding look-up in all these policies.}.
% \textbf{Seq2Seq(Baseline)} \cite{ALFRED20} is a single-stream Seq-to-Seq model with progress monitoring, processing the visual frames through  a frozen ResNet-18 encoder, language through bi-LSTM and soft attention and fusion of the latent visual, language and previous action encodings through an LSTM layer.
%%%% The visual frames are encoded by a frozen ResNet-18 encoder. The language instruction tokens are processed with a bi-LSTM and soft attention. The latent encodings for visual, language and previous action are passed through an LSTM.
% \textbf{MOCA} \cite{Singh2021FactorizingPA} presents a factorized model into two, i.e. interactive perception and action policy. The inputs to the action policy model are language encoding from bi-LSTM, visual embedding from a pretrained ResNet-18, and previous action embedding; all concatenated as input to an LSTM with residual connection.
% \textbf{Episodic Transformers} \cite{pashevich2021episodic} proposes a transformer architecture that encodes the language instructions and the sequence of visual observations and actions to predict subsequent actions per visual frame. Visual observations are processed through pretrained ResNet-50, language tokens passed through a transformer encoder pre-trained with synthetic language targets,  and action are encoded by embedding look-up. 

% Please add the following required packages to your document preamble:
% \usepackage{booktabs}
% Please add the following required packages to your document preamble:
% \usepackage{booktabs}
% Please add the following required packages to your document preamble:
% \usepackage{booktabs}
\begin{table}[t]
\centering
%  \vspace{-1em}
\caption{Policies trained on ALFRED Dataset and their architectures for each modality}
\label{tab:policiesarch}
\begin{tabular}{@{}llll@{}}
\toprule
Policies & Visual                                                                       & Language                       & Fusion                                                                   \\ \midrule
Baseline \cite{ALFRED20} & Frozen ResNet-18                                                             & Learned Embedding, Bi-LSTM     & LSTM                                                                     \\
MOCA \cite{Singh2021FactorizingPA}    & \begin{tabular}[c]{@{}l@{}}Frozen ResNet-18\\ + Dynamic Filters\end{tabular} & Learned Embedding, Bi-LSTM     & \begin{tabular}[c]{@{}l@{}}LSTM with \\ residual connection\end{tabular} \\
ET \cite{pashevich2021episodic}      & Frozen ResNet-50                                                             & Learned Embedding, Transformer & Transformer Encoder                                                      \\
HiTUT \cite{Zhang2021HierarchicalTL}   & Frozen MaskRCNN                                                              & Learned Embedding, FC, LN      & Transformer Encoder                                                  \\ \bottomrule
\end{tabular}
\vspace{-0.2em}
\end{table}
% EmBERT




 
%  provide spurious 
%  explanations and 
%  may not 
%  In cases where the attribution may 
%  this method depends on the underlying attribution methods such as IG. 

\section{Methodology}
\label{sec:method}

While most PINNs models study point-to-point predictions, learning and understanding the temporal dependency in PINNs seems to be neglected. For instance, conventional PINNs methods utilize a single pair of spatial information $\boldsymbol{x}$ and temporal information $t$ to approximate the numerical solution $u(\boldsymbol{x},t)$, without considering the dependency from its previous or later time steps. However, this is only true for elliptic PDEs, whose unknown functions and derivatives are related to each other in a way that does not involve time explicitly. Hyperbolic and parabolic PDEs involve time derivatives, indicating the state at one timestep may influence states at previous or later timesteps. Hence, it is crucial to consider the time dependency relations for solving these PDEs with PINNs.

In this section, we propose a novel framework with a Transformer-based model for solving PINNs, namely PINNsFormer, going beyond point-to-point predictions to sequential predictions. PINNsFormer allows approximating the accurate solution at a specific time step by learning and regularizing the temporal dependency of its incoming states. The framework consists of four components: Pseudo Sequence Generator, Spatio-Temporal Mixer, Encoder-Decoder with multi-head attention, and an Output Layer. In addition, we propose a novel activation function, namely \texttt{Wavelet}, that anticipates PDEs' solutions in a Fourier decomposition manner. The diagram of the framework is exhibited in Figure \ref{fig:arch}. We detail all framework components and learning schemes in the following subsections.

% Figure environment removed

\subsection{Pseudo Sequence Generator}

Despite the fact that Transformer and Transformer-based models are designed to capture the long-term dependency for sequential data, conventional PINNs generally take non-sequential data to feed forward the neural networks. Hence, to adapt PINNs with Transformer-based models, it is essential to translate the pointwise spatiotemporal inputs to temporal sequences. Thus, for any given spatial input $\boldsymbol{x}\in \mathbb{R}^{d-1}$ and temporal input $t\in \mathbb{R}$, the Pseudo Sequence Generator does the following operations:
\begin{equation}
    [\boldsymbol{x},t] \xRightarrow{\texttt{Generator}} \{[\boldsymbol{x},t], [\boldsymbol{x},t+\Delta t], \ldots, [\boldsymbol{x},t+(k-1)\Delta t]\} 
\end{equation}

where $[\cdot]$ is the concatenation operation, such that $[\boldsymbol{x},t]\in \mathbb{R}^d$ is vectorized, and the generator outputs the pseudo sequence in the shape of $\mathbb{R}^{k\times d}$. In short, the Pseudo Sequence Generator anticipates the sequential time series by expanding a single spatiotemporal input to multiple isometric discrete time steps.
$k$ and $\Delta t$ are hyperparameters, which intuitively determine how many steps the pseudo sequence needs to 'look ahead' and how 'far' each step should be. In practice, both $k$ and $\Delta t$ should not be set by large scales, as larger $k$ can cause heavy computational and memory overheads, and larger $\Delta t$ may undermine the time dependency relationships of neighboring discrete time steps. 

\subsection{Model Architecture}

In addition to the Pseudo Sequence Generator, PINNsFormer consists of three components of its model: Sptio-Temporal Mixer, Encoder-Decoder with multi-head attentions, and Output Layer. While it is trivial to interpret the Output Layer, as it is simply a fully-connected MLP attached to the tail similar to most neural networks, we detail the first two components in the following. Remarkably, PINNsFormer consists of only linear layers and non-linear activations without complicated operations such as convolutional or recurrent layers. Hence, it preserves computational efficiency as fully-connected MLPs do in practice.

\textbf{Spatio-Temporal Mixer.} The spatial information of most PDEs is typically low-dimensional. Directly feeding low-dimensional data to encoders may fail to capture the complex relationships between each feature dimension. Hence, it is necessary to embed original sequential data in higher-dimensional spaces such that more information is encoded into each vector.

Instead of embedding raw data in a high-dimensional space where the distance between vectors reflects the semantic similarity, as most language tasks do~\cite{vaswani2017attention, devlin2018bert}, PINNsFormer simply constructs a linear projection that maps spatiotemporal inputs onto a higher-dimensional space using a fully-connected MLP. The embedded data enriches the capability of information by mixing all raw spatiotemporal features together, so-called the linear projection Spatio-Temporal Mixer.

\begin{wrapfigure}{r}{0.45\textwidth}
\vspace{-0.1in}
    \begin{center}
        % Figure removed
    \end{center}
    \caption{The architecture of PINNsFormer's Encoder-Decoder Layers. The decoder is not equipped with self-attentions.}
    \label{fig:en_de}
    \vspace{-0.1in}
\end{wrapfigure}

\textbf{Encoder-Decoder with multi-head attention.} PINNsFormer employs an encoder-decoder architecture similar to Transformer~\cite{vaswani2017attention}. The encoder consists of a stack of identical layers, each of which contains a self-attention layer and a feedforward layer. The decoder is slightly different from the vanilla Transformer, in which each of the identical layers contains only an encoder-decoder attention layer and a feedforward layer. Since at decoders' level, PINNsFormer utilizes the same spatiotemporal embeddings as the encoder does, it is unnecessary for the decoder to equip with an additional self-attention and  repeatedly capture the dependencies for the same input embeddings. The diagram for the encoder-decoder architecture is shown in Figure \ref{fig:en_de}

Intuitively, our model's encoder's self-attention allows learning the dependency relationships of all spatiotemporal information. The decoder's encoder-decoder attention then enables selectively focusing on specific dependencies of the input sequence during the decoding process, capturing more information than a vanilla PINN.


\subsection{Wavelet Activation}

While Transformers are typically equipped with \texttt{LayerNorm} and \texttt{ReLU} non-linear activation functions~\cite{vaswani2017attention, gehring2017convolutional, devlin2018bert}, these activation functions might not generally be feasible in solving PINNs. In particular, employing \texttt{ReLU} activation in PINNs suffers from poor performance for PINNs, whose effectiveness relies heavily on the accurate evaluation of derivatives while \texttt{ReLU} has a discontinuous derivative~\cite{haghighat2021physics, de2021assessing}. Recent studies utilize \texttt{Sin} activation for some special cases in PINNs, aiming to mimic the periodic properties of PDEs' solutions~\cite{li2020fourier, jagtap2020adaptive, song2022versatile}. However, it requires a strong prior knowledge of the solution's behavior and only works well for limited scenarios. Tackling this issue, we proposed a novel and simple activation function, namely \texttt{Wavelet}, defined as follows:
\begin{equation}
    \texttt{Wavelet}(\boldsymbol{x}) = \sin(\boldsymbol{x})+\cos(\boldsymbol{x})
\end{equation}

The intuition behind \texttt{Wavelet} activation simply follows the Fourier Transform: While periodic signals can be decomposed into a sum of sine waves of different frequencies, all signals, including both periodic and non-periodic, can be decomposed into a sum of sine and cosine waves of different frequencies. It is then obvious that \texttt{Wavelet} activation can approximate arbitrary functions giving enough approximation power, which leads to the following proposition.

\begin{proposition}
    Let $\mathcal{N}$ be a two-hidden-layer neural network with infinite width, equipped with \texttt{Wavelet} activation function, then $\mathcal{N}$ is a universal approximator for any target f.
\end{proposition}

\textit{Proof sketch:} The proof follows the following two steps. First, for any given input $x$ and its corresponding target $f(x)$, it has the Inverse Fourier Transform:
\begin{gather*}
    f(x) = \frac{1}{2\pi}\int_{-\infty}^\infty F(\omega)e^{-j\omega x}d\omega = \frac{1}{\pi} \int_0^\infty F_c(\omega)\cos \omega x + F_s(\omega) \sin \omega x d\omega
\end{gather*}

where $F_c$ and $F_s$ are the real and imaginary parts of the Fourier transform respectively. Second, by Riemann sum approximation, the integral can be rewritten to the infinite sum such that:
\begin{gather*}
    f(x) = \frac{1}{\pi}\sum^\infty_{i=0} F_c(\omega_i) \cos \omega_i x + F_s(\omega_i) \sin \omega_i x \equiv W_2(\texttt{Wavelet}(W_1 x))
\end{gather*}

where $W_1$ and $W_2$ are the weights of $\mathcal{N}$'s first and second hidden layer. Hence, $\mathcal{N}$ is a universal approximator for any given $f$.

Despite that to our purposes, the \texttt{Wavelet} activation function is only employed by PINNsFormer to improve PINNs, it can be potentially useful for other deep learning tasks. It is similar to \texttt{ReLU}, $\sigma(\cdot)$, and \texttt{Tanh} activations, which when applied to a two-hidden-layer neural network, turns the network into a universal approximator ~\cite{cybenko1989approximation, hornik1991approximation, glorot2011deep}. Though going beyond PINNs is out of the scope of this work, we hope to demonstrate the effectiveness of \texttt{Wavelet}  with other applications.

% \textcolor{blue}{Maybe a short proposition: Given a 2-layer linear layer with infinite width of hidden, and wavelet activation, it can approximate any solutions.}

\subsection{Learning Scheme}

While conventional PINNs deal with point-to-point predictions, it remains unexplored how PINNs loss can be adapted to pseudo-sequential inputs. In PINNsFormer, each generated point in the sequence, i.e., $[\boldsymbol{x}_i,t_i+j\Delta t]$, is forwarded to the corresponding approximation, i.e., $\hat{u}(\boldsymbol{x}_i,t_i+j\Delta t)$ for any $j\in\mathbb{N}, j<k$. Hence, it is possible to take the $n$th-order gradients w.r.t. $\boldsymbol{x}$ or $t$ independently for any valid $n$. For instance, for any given input pseudo sequence $\{[\boldsymbol{x}_i,t_i], [\boldsymbol{x}_i,t_i+\Delta t], \ldots, [\boldsymbol{x}_i,t_i+(k-1)\Delta t]\}$, and the corresponding approximations $\{\hat{u}(\boldsymbol{x}_i,t_i), \hat{u}(\boldsymbol{x}_i,t_i+\Delta t), \ldots, \hat{u}(\boldsymbol{x}_i,t_i+(k-1)\Delta t)\}$, we have the first-order derivatives w.r.t. $\boldsymbol{x}$ and $t$ in the following formulations separately:
\begin{equation}
\begin{gathered}
    \frac{\partial\{\hat{u}(\boldsymbol{x}_i,t_i+j\Delta t)\}_{j=0}^{k-1}}{\partial\{t_i+j\Delta t\}_{j=0}^{k-1}} = \{\frac{\partial\hat{u}(\boldsymbol{x}_i,t_i)}{\partial t_i},\frac{\partial\hat{u}(\boldsymbol{x}_i,t_i+\Delta t)}{\partial(t_i+\Delta t)},\ldots, \frac{\partial\hat{u}(\boldsymbol{x}_i,t_i+(k-1)\Delta t)}{\partial(t_i+(k-1)\Delta t)}\} \\
     \frac{\partial\{\hat{u}(\boldsymbol{x}_i,t_i+j\Delta t)\}_{j=0}^{k-1}}{\partial\boldsymbol{x}_i} = \{\frac{\partial\hat{u}(\boldsymbol{x}_i,t_i)}{\partial\boldsymbol{x}_i},\frac{\partial\hat{u}(\boldsymbol{x}_i,t_i+\Delta t)}{\partial\boldsymbol{x}_i},\ldots, \frac{\partial\hat{u}(\boldsymbol{x}_i,t_i+(k-1)\Delta t)}{\partial\boldsymbol{x}_i}\}
\end{gathered}
\end{equation}

The above scheme of calculating the gradient of sequential approximations w.r.t. sequential inputs can be easily extended to higher-order derivatives, and is applicable for residual, boundary, and initial points. 
However, while the general PINNs optimization objective in Equation \ref{eq:pinn} unifies the initial and boundary conditions together, PINNsFormer instead clearly distinguishes the initial and boundary conditions. This is because PINNsFormer regularizes the initial and boundary points differently through its learning scheme.
% The scheme of calculating the gradient of sequential approximations w.r.t. sequential inputs is applicable for residual, boundary, and initial points. 
For residual and boundary points, all sequential outputs can be regularized through the PINNs loss. This is because all generated pseudo timesteps are at the same domain as their original input. For instance, if $[\boldsymbol{x}_i,t_i]$ is sampled from the boundary, then $[\boldsymbol{x}_i, t_i+j\Delta t]$ also locates on the boundary for any given $j\in \mathbb{N}^+$. For the initial points, only the $t=0$ is regularized, which refers to the first element of the sequential outputs. This is because only the first element of the pseudo sequence refers to exactly the initial condition, where $t=0$. All other generated time steps have $t=j\Delta t$ for any $j\in\mathbb{N}^+$ are out of the initial conditions. Combining all reasonings above, we adapt the PINNs loss to the sequential version, described as follows:
\begin{equation}
\begin{gathered}
    \mathcal{L}_\textit{res} = \frac{1}{kN_\textit{res}}\sum_{i=1}^{N_\textit{res}}\sum_{j=0}^{k-1} \|\mathcal{D}[\hat{u}(\boldsymbol{x}_i,t_i+j\Delta t)] - f(\boldsymbol{x}_i, t_i+j\Delta t)\|^2\\
    \mathcal{L}_\textit{bc} = \frac{1}{kN_\textit{bc}}\sum_{i=1}^{N_\textit{bc}}\sum_{j=0}^{k-1} \|\mathcal{B}[\hat{u}(\boldsymbol{x}_i,t_i+j\Delta t)] - g(\boldsymbol{x}_i, t_i+j\Delta t)\|^2\\
    \mathcal{L}_\textit{ic} = \frac{1}{N_\textit{ic}} \sum_{i=1}^{N_\textit{bc}} \|\mathcal{I}[\hat{u}(\boldsymbol{x}_i,0)] - h(\boldsymbol{x}_i, 0)\|^2\\
    \mathcal{L}_{\texttt{PINNsFormer}} = \lambda_\textit{res} \mathcal{L}_\textit{res} + \lambda_\textit{ic} \mathcal{L}_\textit{ic} + \lambda_\textit{bc} \mathcal{L}_\textit{bc}
\end{gathered} 
\label{eqn:obj}
\end{equation}

where $N_\textit{res}=N_r$ refers to the residual points as in Equation \ref{eq:pinn}, $N_\textit{bc}, N_\textit{ic}$ are the number of boundary and initial points separately, and $N_\textit{bc}+N_\textit{ic}=N_b$. $\lambda_\textit{res}$, $\lambda_\textit{bc}$, and $\lambda_\textit{ic}$ are the regularization parameters that balance the emphasis of PINNsFormer's loss terms, similar to conventional PINNs loss.

In the training stage, PINNsFormer forwards all residual, boundary, and initial points to observe the corresponding sequential approximations. It then optimizes the modified PINNs loss $\mathcal{L}_\texttt{PINNsFormer}$ in Equation \ref{eqn:obj} through the gradient-based optimization algorithms, i.e., L-BFGS or Adam, to update the model parameters until convergence. In the testing stage, PINNsFormer forwards an arbitrary pair of $[\boldsymbol{x},t]$ to observe the sequential approximations, while the first element of the sequential approximation is exactly the desired approximation of $\hat{u}(\boldsymbol{x},t)$.

\subsection{Loss Landscape Analysis}

While showing the convergence or generalization bound of Transformer-based models theoretically can be ambitious, the alternative approach to analyzing its performance is by visualizing the loss landscape, which has been employed for analyzing either Transformers or PINNs~\cite{krishnapriyan2021characterizing, yao2020pyhessian, park2022vision}. The loss landscape computes the loss values by making small perturbations on the trained model across the first two dominant Hessian eigenvectors, which tends to be more informative than perturbing the model parameters in random directions. Generally, the smoother the loss landscape is, or the fewer local minimums the loss landscape presents, the easier that the model can converge to the global minima. Therefore, we visualize the loss landscape for both conventional PINNs (MLPs + \texttt{Tanh}) and PINNsFormer (Transformer + \texttt{Wavelet}), trained for the reaction-diffusion problem with coefficient $\rho=5, \nu=5$ (detailed in Appendix \ref{sec:appendb}). The visualizations are shown in Figure \ref{fig:loss}.

% Figure environment removed




It is then evident that vanilla PINNs exhibit a more complicated loss landscape than PINNsFormer. More precisely, we estimate the Lipschitz constant for both loss landscapes. We observe $L_\texttt{PINNs} =776.16$, which is much larger than $L_\texttt{PINNsFormer} = 32.79$, both in log-scale. 
Moreover, the loss landscape of conventional PINNs presents several sharp cones near its optimal, indicating multiple local minima around its convergence (zero perturbation). The rugged loss landscape and multiple local minima of conventional PINNs both indicate that optimizing the objective in Equation \ref{eqn:obj} of PINNsFormer is an easier alternative approach than conventional PINNs to reach the global minima, and hence possibly avoid PINNs’ failure modes. The analysis result is further validated by the empirical experiments in the following section.




\section{Experiments}
\label{sec:exp}

\subsection{Experiments Setup}
\label{sec:setup}

\textbf{Data Preparation.} We use a unified data preparation strategy for all PDEs we are evaluating, regardless of the different domain ranges for different PDEs. For the training phase, we uniformly sample $N_{\textit{ic}} = N_{\textit{bc}}=101$ initial and boundary data. We then uniformly create a $101\times 101$ mesh data inside the residual domain, such that $N_{\textit{res}}=10201$. For the test phase, we uniformly create a $201\times 201$ mesh data inside the residual domain. The sampling for testing is denser than for training, which aims to show the flexibility of zooming in/out the resolution and approximating the solutions at arbitrary points inside the residual domain, similar to \cite{chen2018neural}.

\textbf{Training and Testing.} We build a fully-connected MLP as the baseline, which is a typical architecture of PINNs, and the PINNsFormer. We control the number of parameters for both models to be close, aiming to show the benefit of PINNsFormer is indeed by its mechanism rather than the approximation power due to the over-parameterization. We train all models with L-BFGS optimizer with Strong Wolfe linear search function for 1000 iterations. For simplicity, we set $\lambda_\textit{res}=\lambda_\textit{ic} = \lambda_\textit{bc}=1$ for the optimization objective in Equation \ref{eqn:obj}. We detail all hyperparameters in Appendix \ref{sec:appenda}. 

\textbf{Reproducibility.} All models are implemented in PyTorch~\cite{paszke2019pytorch}, and are trained separately on a single NVIDIA Tesla V100 GPU. All training code and setup are included and reproducible at \url{https://github.com/AdityaLab/pinnsformer}.

\subsection{Main Results}

The main evaluation is based on solving three specific PDEs: convection equation, 1D-reaction equation, and reaction-diffusion equation, where MLP presents significant limitations in approximating the solutions for these equations~\cite{mojgani2022lagrangian, daw2022rethinking, krishnapriyan2021characterizing}, so-called the failure modes. The detailed formulations of the PDEs are presented in Appendix \ref{sec:appendb}. For the comprehensiveness of the evaluation, we present an ablation study by the combinations between two model architectures, MLP and Transformer (Trans. in short), and four activation functions, including \texttt{ReLU}, \texttt{Tanh}, \texttt{Sin}, and proposed \texttt{Wavelet}. In particular, combining Transformer with \texttt{Wavelet} activation function is exactly the PINNsFormer. We measure both relative MAE (rMAE, or $\ell_1$ error) and relative RMSE (rRMSE, or $\ell_2$ error) as most relative works do~\cite{krishnapriyan2021characterizing, raissi2019physics, mcclenny2020self}, defined as:
\begin{equation}
\begin{gathered}
    \texttt{rMAE} =  \frac{\sum_{n=1}^N |\hat{u}(x_n,t_n)-u(x_n,t_n)|}{\sum_{n=1}^{N_{\textit{res}}}|u(x_n,t_n)|}\\
    \texttt{rRMSE} = \sqrt{\frac{\sum_{n=1}^N |\hat{u}(x_n,t_n)-u(x_n,t_n)|^2}{\sum_{n=1}^N|u(x_n,t_n)|^2}}
\end{gathered}    
\end{equation}
where $N$ is the number of test points, $\hat{u}$ is the neural network approximation, and $u$ is the ground truth. In addition, we include the training PINNs loss for all cases. The evaluations of Convection and Reaction-Diffusion PDEs are shown in Table \ref{tbl:main}. We include additional evaluations on the 1D-Reaction problem and selected prediction plots by Appendix \ref{sec:appendc}.

\begin{table}[h]
\centering
\renewcommand{\arraystretch}{1.5}
\scalebox{0.8}{
\begin{tabular}{cc|ccc|ccc}
\toprule
\multirow{2}{*}{Model}  & \multirow{2}{*}{\begin{tabular}[c]{@{}c@{}}Activation\\ Function\end{tabular}} & \multicolumn{3}{c|}{Convection} & \multicolumn{3}{c}{Reaction-Diffusion} \\
                        &                                                                                & Loss     & rMAE     & rRMSE     & Loss        & rMAE       & rRMSE       \\ \hline
\multirow{4}{*}{MLP}    & ReLU&\num{5.66e-1}&\num{1.00e0}&\num{1.01e0}&\num{2.08e-1}&\num{9.92e-1}&\num{9.93e-1}\\
                        & Tanh&\num{1.18e-2}&\num{6.75e-1}&\num{7.56e-1}&\num{1.57e-4}&\num{1.85e-1}&\num{2.39e-1}\\
                        & Sin&\num{2.97e-1}&\num{1.06e0}&\num{1.12e0}&\num{5.89e-4}&\num{2.52e-1}&\num{2.90e-1}\\
                        & Wavelet&\num{3.50e-4}&\num{5.07e-2}&\num{5.75e-2}&\num{9.09e-4}&\num{7.81e-1}&\num{8.47e-1}\\ \hline
\multirow{4}{*}{Trans.} & ReLU&\num{5.40e-1}&\num{1.00e0}&\num{1.01e0}&\num{2.10e-1}&\num{9.90e-1}&\num{9.92e-1}\\
                        & Tanh&\num{1.50e-2}&\num{7.52e-1}&\num{8.20e-1}&\num{8.19e-6}&\num{1.39e-1}&\num{1.66e-1}\\
                        & Sin&\num{1.59e-2}&\num{7.56e-1}&\num{8.23e-1}&\num{5.01e-5}&\num{2.33e-1}&\num{2.94e-1}\\
                        & Wavelet&\num{2.43e-4}&\num{4.96e-2}&\num{5.64e-2}&\num{3.86e-6}&\num{8.49e-2}&\num{1.05e-1}  \\
                        \bottomrule 
\end{tabular}}
\vspace{0.05in}
\caption{Main results in approximating Convection and Reaction-Diffusion PDEs' solutions. PINNsFormer (Trans.+\texttt{Wavelet}) outperforms other methods in training loss, rMAE, and rRMSE.}
\label{tbl:main}
\end{table}

The evaluation results show significant outperformance of PINNsFormer over any other combinations between model architectures and activation functions, where PINNsFormer achieves the lowest both training loss and validation errors. Moreover, either Transformer-based model architecture or \texttt{Wavelet} activation function is beneficial for optimizing PINNs loss. For instance, the evaluations of the convection problem with \texttt{Wavelet} activation uniformly outperform it with \texttt{Tanh} activation, and the evaluations of the reaction-diffusion problem with Transformer uniformly outperform it with MLP. However, either cannot robustly overcome PINNs' failure modes independently. Only combining both the Transformer-based model and \texttt{Wavelet} activation results in better and more robust approximations for various PDEs.

\subsection{Methodology Overheads}
\label{sec:overhead}

One major motivation for proposing neural-network-based approaches in approximating PDEs' solutions, including PINNs, is to avoid the high computational costs for high-dimensional/order PDEs, while Transformer-based models are known to be more computationally expensive than MLPs. In addition, training sequential data typically requires more memory than point data, which can be costly in memory. Therefore, studying both the computational and memory overheads of PINNsFormer is practically meaningful. For consistency, we use the same model hyperparameters as employed in Main Results. We measure the computation cost by counting the training wallclock time per iteration for learning the convection problem. We measure the memory cost by counting the model parameters and training data size. We present all overheads in Table \ref{tbl:overehead}.

\begin{table}[h]
\centering
\renewcommand{\arraystretch}{1.5}
\begin{tabular}{c|cccc|cc}
\toprule 
\multirow{2}{*}{Model}  & \multicolumn{4}{c|}{Computational Cost (sec)} & \multicolumn{2}{c}{Memory Cost}                \\

                       & ReLU      & Tanh      & Sin      & Wavelet    & Model Parameters & Data Size                   \\ \hline
MLP                    & 2.05      & 2.50      & 2.59     & 3.38       & 527k             & {[}Batch, Dim{]}    \\
Trans.            & 3.03      & 3.39      & 3.36     & 4.69       & 453k             & {[}Batch, $k$, Dim{]} \\
Overhead               & 1.48\texttt{x}     & 1.36\texttt{x}     & 1.30\texttt{x}    & 1.39\texttt{x}      & 0.86\texttt{x}   & $k$\texttt{x}  \\
\bottomrule
\end{tabular}
\vspace{0.05in}
\caption{Computational and memory overhead between Transformer-based models and MLP-based models. Transformer-based models require fewer parameters but larger memory for data, while its computational cost is within 1.5\texttt{x} compared to MLP models.}
\label{tbl:overehead}
\end{table}

For computational overhead, PINNsFormer is about 1.3\texttt{x}$\sim$1.5\texttt{x} computationally costly than vanilla PINNs under the same activation functions. The computational difference is not significant, benefitted from the model's simplicity, that is, PINNsFormer consists of only linear hidden layers with non-linear activation functions, without complicated operators such as convolutional or recurrent layers. In addition, the computational cost is dependent on different activation functions. The proposed \texttt{Wavelet} activation function requires 1.6\texttt{x} and 1.35\texttt{x} computation time than \texttt{ReLU} and \texttt{Tanh} activation functions separately. This is either because \texttt{Wavelet} is more complicated in calculation and derivation, or manually implementing \texttt{Wavelet} is not fully GPU optimized by PyTorch. 

For memory overhead, PINNsFormer actually employs fewer model parameters to achieve better performance than vanilla PINNs. Indeed, PINNsFormer requires $k$ times larger memory for pseudo-sequential data. Nevertheless, on the one hand, hyperparameter $k$ does not necessarily need to be large in practice. We achieve significant outperformance by setting $k=5$. On the other hand, the training batch size of PINNs is typically smaller than other deep-learning tasks, such as language or vision. The memory cost remains controllable even by expanding to $k$-length sequences. Hence, PINNsFormer is a practically meaningful and applicable approach to approximate PDEs' solutions.

\subsection{Hyperparameter Sensitivity}

PINNsFormer creates a generated time sequence through its Pseudo Sequence Generator. The generator has two hyperparameters: steps $k$ and stepsize $\Delta t$. Tuning these two hyperparameters might impact the dependency relationships and qualities captured by self-attentions and encoder-decoder attention, and hence leads to performance differences. Therefore, we vary the evaluations of PINNsFormer on different pairs of $\{k,\Delta t\}$ for solving the convection PDE, which can be easily failed due to the solution's high-frequency component. Without comparing with conventional PINNs, we empower the approximation ability of PINNsFormer by simply doubling the hidden dimension of the feed-forward neural network. The evaluation results are shown in Table \ref{tbl:hyper}, which can be concluded by the following two observations.

\begin{table}[h]
\centering
\renewcommand{\arraystretch}{1.2}
\begin{tabular}{cc|cccc}
\toprule
\multicolumn{2}{c|}{\multirow{2}{*}{rRMSE}} & \multicolumn{4}{c}{$k$}             \\
\multicolumn{2}{c|}{}                          & 3    & 5    & 7    & 10   \\ \hline
\multirow{6}{*}{$\Delta t$}            &\num{1e-5}&\num{3.19e-2}&\num{4.15e-2}&\num{3.12e-2}&\num{2.48e-2}\\
&\num{3e-5}&\num{3.41e-2}&\num{2.58e-2}&\num{3.89e-2}&\num{3.19e-2}\\
&\num{1e-4}&\num{3.10e-2} &\num{3.46e-2}&\num{6.94e-1}&\num{5.17e-2} \\
&\num{3e-4}&\num{5.04e-2} &\num{5.89e-1}&\num{4.64e-2}&\num{6.44e-2} \\
&\num{1e-3}&\num{1.13e0}&\num{5.81e-1}&\num{1.39e-1}&\num{2.04e-1}  \\
&\num{1e-2}&\num{0.446e-1} &\num{1.21e0} &\num{1.01e0}&\num{1.01e0} \\
\bottomrule
\end{tabular}
\vspace{0.05in}
\caption{Evaluated rRMSE of convection equation on wide ranges of hyperparameters $k$ and $\Delta t$.}
\label{tbl:hyper}
\end{table}

\textbf{Hyperparameter $\Delta t$ is not sensitive once it is lower than a threshold.} The performance differs significantly while varying the values of $\Delta t$. When $\Delta t$ remains small, PINNsFormer performs uniformly well. For instance, we observe the rRMSE are uniformly low when $\Delta t=$\texttt{1e-5} and \texttt{3e-5}, regardless of $k$. However, as $\Delta t$ increases, i.e., $\Delta t\geq$\texttt{1e-3}, the performance defers heavily. This is intuitively reasonable as learning temporal dependency between large time steps may become less meaningful. Nevertheless, since the empirical evaluations are generally robust for small $\Delta t$s,  we conclude that $\Delta t$ is not sensitive once it is smaller than a threshold.

\textbf{Hyperparameter $k$ is not sensitive, increasing $k$ can marginally improve the performance.} Intuitively, larger $k$ can enable more dependencies, and therefore, better performance. However, when $\Delta t$ is set properly, the performance of increasing pseudo sequence length $k$ is marginal. For instance, when $\Delta t=\texttt{1e-5}$, increasing $k=3$ to 10 only decrease the error by approximately $25\%$. The performance improvement of increasing $k$ when $\Delta t=\texttt{3e-5}$ is even less significant.
Thus, larger $k$ can only improve the performance marginally, and hence is not sensitive when $\Delta t$ is set properly. 


% \subsection{Robustness to non-failure PDEs}

% While the above evaluations rely on the failure PDEs of PINNs, PINNsFormer is indeed a robust framework for solving both failure and non-failure PDEs. We therefore evaluate PINNsFormer on Burger's and Helmholtz equations, in which conventional PINNs have already shown success in these scenarios~\cite{raissi2019physics}. We use the same setup as described in Section \ref{sec:setup}. We include the detailed PDEs setup and coefficients in Appendix \ref{sec:appendb} and results in Appendix \ref{sec:appendc}. In summary, PINNsFormer achieves uniformly better results under all evaluating metrics for both equations. 

\section{Conclusion}

In this work, we present \texttt{vox2vec} --- a self-supervised framework for voxel-wise representation learning in medical imaging. Our method expands the contrastive learning setup to the feature pyramid architecture allowing to pre-train effective representations in full resolution. By pre-training a FPN backbone to extract informative representations from unlabeled data, our method scales to large datasets across multiple task domains. We pre-train a FPN architecture on more than 6500 CT images and test it on various segmentation tasks, including different organs and tumors segmentation in three setups: linear probing, non-linear probing, and fine-tuning. Our model outperformed existing methods in all regimes. Moreover, \texttt{vox2vec} establishes a new state-of-the-art result on the linear and non-linear probing scenarios. 

Still, this work has a few limitations to consider. We plan to investigate further how the performance of \texttt{vox2vec} scales with the increasing size of the pre-training dataset and the pre-trained architecture size. Another interesting research direction is exploring the effectiveness of \texttt{vox2vec} in the domain adaptation and few-shot learning scenarios.






% \section*{References}

\bibliographystyle{plain}
\bibliography{ref.bib}

%%%%%%%%%%%%%%%%%%%%%%%%%%%%%%%%%%%%%%%%%%%%%%%%%%%%%%%%%%%%
% \section*{Checklist}

% %%% BEGIN INSTRUCTIONS %%%
% % The checklist follows the references.  Please
% % read the checklist guidelines carefully for information on how to answer these
% % questions.  For each question, change the default \answerTODO{} to \answerYes{},
% % \answerNo{}, or \answerNA{}.  You are strongly encouraged to include a {\bf
% % justification to your answer}, either by referencing the appropriate section of
% % your paper or providing a brief inline description.  For example:
% % \begin{itemize}
% %   \item Did you include the license to the code and datasets? \answerYes{See Section~\ref{gen_inst}.}
% %   \item Did you include the license to the code and datasets? \answerNo{The code and the data are proprietary.}
% %   \item Did you include the license to the code and datasets? \answerNA{}
% % \end{itemize}
% % Please do not modify the questions and only use the provided macros for your
% % answers.  Note that the Checklist section does not count towards the page
% % limit.  In your paper, please delete this instructions block and only keep the
% % Checklist section heading above along with the questions/answers below.
% %%% END INSTRUCTIONS %%%

% \begin{enumerate}

% \item For all authors...
% \begin{enumerate}
%   \item Do the main claims made in the abstract and introduction accurately reflect the paper's contributions and scope? \answerYes{}
%     % \answerTODO{}
%   \item Did you describe the limitations of your work? \answerYes{We discuss possible limitations in perspective of overhead of our proposed methodolgy, detailed in Section \ref{sec:overhead}}
%     % \answerTODO{}
%   \item Did you discuss any potential negative societal impacts of your work? \answerNA{}
%     % \answerTODO{}
%   \item Have you read the ethics review guidelines and ensured that your paper conforms to them? \answerYes{}
%     % \answerTODO{}
% \end{enumerate}

% \item If you are including theoretical results...
% \begin{enumerate}
%   \item Did you state the full set of assumptions of all theoretical results? \answerYes{}
%     % \answerTODO{}
% 	\item Did you include complete proofs of all theoretical results? \answerYes{}
%     % \answerTODO{}
% \end{enumerate}

% \item If you ran experiments...
% \begin{enumerate}
%   \item Did you include the code, data, and instructions needed to reproduce the main experimental results (either in the supplemental material or as a URL)? \answerYes{}
%     % \answerTODO{}
%   \item Did you specify all the training details (e.g., data splits, hyperparameters, how they were chosen)? \answerYes{}
%     % \answerTODO{}
% 	\item Did you report error bars (e.g., with respect to the random seed after running experiments multiple times)? \answerNA{}
%     % \answerTODO{}
% 	\item Did you include the total amount of compute and the type of resources used (e.g., type of GPUs, internal cluster, or cloud provider)? \answerYes{}
%     % \answerTODO{}
% \end{enumerate}

% \item If you are using existing assets (e.g., code, data, models) or curating/releasing new assets...
% \begin{enumerate}
%   \item If your work uses existing assets, did you cite the creators? \answerYes{For instance, PyTorch implementation, Transformer implementation.}
%     % \answerTODO{}
%   \item Did you mention the license of the assets? \answerYes{Properly cited.}
%     % \answerTODO{}
%   \item Did you include any new assets either in the supplemental material or as a URL? \answerNA{}
%     % \answerTODO{}
%   \item Did you discuss whether and how consent was obtained from people whose data you're using/curating? \answerNA{}
%     % \answerTODO{}
%   \item Did you discuss whether the data you are using/curating contains personally identifiable information or offensive content? \answerNA{}
%     % \answerTODO{}
% \end{enumerate}

% \item If you used crowdsourcing or conducted research with human subjects...
% \begin{enumerate}
%   \item Did you include the full text of instructions given to participants and screenshots, if applicable? \answerNA{}
%     % \answerTODO{}
%   \item Did you describe any potential participant risks, with links to Institutional Review Board (IRB) approvals, if applicable? \answerNA{}
%     % \answerTODO{}
%   \item Did you include the estimated hourly wage paid to participants and the total amount spent on participant compensation? \answerNA{}
%     % \answerTODO{}
% \end{enumerate}

% \end{enumerate}

%%%%%%%%%%%%%%%%%%%%%%%%%%%%%%%%%%%%%%%%%%%%%%%%%%%%%%%%%%%%


\clearpage

\section*{Supplementary}
\appendix
\section{Appendix A: Model Hyperparameters}
\label{sec:appenda}

We provide a detailed set of hyperparameters used to obtain the experiment results, shown in Table \ref{tbl:hyper}.

\begin{table}[h]
\centering
\renewcommand{\arraystretch}{1.5}
\begin{tabular}{c|cc}
Model                   & Hyparameter    & Value \\ \hline
\multirow{2}{*}{MLP}    & hidden layers & 4     \\ 
                        & hidden size    & 512   \\ \hline
\multirow{7}{*}{Trans.} & $k$            & 5     \\
                        & $\Delta t$     & 1e-4  \\
                        & encoder        & 1    \\
                        & decoder        & 1    \\
                        & embedding size & 32    \\
                        & heads          & 2     \\
                        & hidden size    & 512  
\end{tabular}
\caption{Hyperparameters for Main Results}
\label{tbl:hyper}
\end{table}


\section{Appendix B: PDEs setups}
\label{sec:appendb}

We provide detailed PDE setups for convection, reaction-diffusion, and 1D-reaction equations.

\textbf{Convection PDE.} The one-dimensional convection problem is a hyperbolic PDE that is commonly used to model transport phenomena. The system has the formulation with periodic boundary conditions as follows:
\begin{equation}
\begin{gathered}
    \frac{\partial u}{\partial t} + \beta \frac{\partial u}{\partial x} = 0, \:\: \forall x\in [0,2\pi], \: t\in [0,1] \\
    \texttt{IC:} u(x,0)=\sin(x), \:\:\: \texttt{BC:} u(0,t)=u(2\pi,t)
\end{gathered}    
\end{equation}

where $\beta$ is the convection coefficient. As $\beta$ increases, the frequency of its solution goes higher, and it becomes harder for PINNs to approximate. Here, we set $\beta=50$.

\textbf{1D-Reaction PDE.} The one-dimensional reaction problem is a hyperbolic PDE that is commonly used to model chemical reactions. The system has the formulation with periodic boundary conditions as follows:
\begin{equation}
\begin{gathered}
    \frac{\partial u}{\partial t} - \rho u(1-u) = 0, \:\: \forall x\in [0,2\pi], \: t\in [0,1] \\
    \texttt{IC:} u(x,0)=\exp(-\frac{(x-\pi)^2}{2(\pi/4)^2}), \:\:\: \texttt{BC:} u(0,t)=u(2\pi,t)
\end{gathered}    
\end{equation}

where $\rho$ is the reaction coefficient. Here, we set $\rho=5$. The equation has a simple analytical solution:
\begin{equation}
    u_{\texttt{analytical}} = \frac{h(x) \exp(\rho t)}{h(x)\exp(\rho t)+1-h(x)}
\end{equation}

where $h(x)$ is the function of the initial condition.

\textbf{Reaction-Diffusion PDE.} The reaction-diffusion system is where a diffusion operator is added to the reaction equation above. The system has the formulation with periodic boundary conditions as follows:
\begin{equation}
\begin{gathered}
    \frac{\partial u}{\partial t} - \nu \frac{\partial^2 u}{\partial x^2} - \rho u(1-u)= 0, \:\: \forall x\in [0,2\pi], \: t\in [0,1] \\
    \texttt{IC:} u(x,0)=\exp(-\frac{(x-\pi)^2}{2(\pi/4)^2}), \:\:\: \texttt{BC:} u(0,t)=u(2\pi,t)
\end{gathered}    
\end{equation}

where $\nu>0$ is the diffusion coefficient. Here, we set $\rho=5$ and $\nu=5$. The solution of the system can be solved via Strang splitting, i.e., splitting the equation into two separate models (a reaction component and a diffusion component):
\begin{equation}
\begin{gathered}
    \frac{\partial u}{\partial t} = \rho u (1-u) \\
    \frac{\partial u}{\partial t} = \frac{\partial^2 u}{\partial x^2}
\end{gathered}    
\end{equation}

\section{Appendix C: Additional Results}
\label{sec:appendc}

We here first include the evaluation on 1D-reaction PDE, which is the supplement for the main results in Table \ref{tbl:main}. The evaluation results are shown in Table \ref{tab:add}. Again, PINNsFormer outperforms all other methods in training loss, rMAE, and rRMSE. In particular, PINNsFormer is the only approach that avoids the convergence issue (failure modes) of PINNs for 1D-reaction PDE.

\begin{table}[h]
    \centering
    \renewcommand{\arraystretch}{1.5}
\begin{tabular}{cc|ccc}
\multirow{2}{*}{Model}  & \multirow{2}{*}{\begin{tabular}[c]{@{}c@{}}Activation\\ Function\end{tabular}} & \multicolumn{3}{c}{1D-Reaction} \\
& & Loss   & rMAE & rRMSE\\ \hline
\multirow{4}{*}{MLP}    & ReLU&\num{2.08e-1}&\num{9.91e-1}&\num{9.92e-1}\\
& Tanh&\num{1.99e-2}&\num{9.81e-1}&\num{9.80e-1}\\
& Sin&\num{1.99e-1}&\num{9.68e-1}&\num{9.67e-1}\\
& Wavelet&\num{2.04e-1}&\num{9.86e-1}&\num{9.91e-1}\\ \hline
\multirow{4}{*}{Trans.} & ReLU&\num{2.10e-1}&\num{9.93e-1}&\num{9.95e-1}\\
& Tanh&\num{1.99e-1}&\num{9.80e-1}&\num{9.80e-1}\\
& Sin&\num{1.99e-1}&\num{9.82e-1}&\num{9.81e-1}\\
& Wavelet&\num{7.56e-6}&\num{1.42e-2}&\num{2.67e-2}
\end{tabular}
    \caption{Supplement results in approximating 1D-Reaction PDE's solution. PINNsFormer (Trans.+\texttt{Wavelet}) outperforms other methods in training loss, rMAE, and rRMSE.}
    \label{tab:add}
\end{table}

For simplicity, we then show the selected prediction plots, which include the comparison between conventional PINNs (MLP+\texttt{Tanh}) and PINNsFormer (Trans.+\texttt{Wavelet}) for all three PDEs. The plots are shown in Figure \ref{fig:plot}. Clearly, we observe approximations by PINNsFormer is more accurate and closer to the ground truth, which matches the main evaluation results.

% Figure environment removed

% References follow the acknowledgments in the camera-ready paper. Use unnumbered first-level heading for
% the references. Any choice of citation style is acceptable as long as you are
% consistent. It is permissible to reduce the font size to \verb+small+ (9 point)
% when listing the references.
% Note that the Reference section does not count towards the page limit.
% \medskip


% {
% \small


% [1] Alexander, J.A.\ \& Mozer, M.C.\ (1995) Template-based algorithms for
% connectionist rule extraction. In G.\ Tesauro, D.S.\ Touretzky and T.K.\ Leen
% (eds.), {\it Advances in Neural Information Processing Systems 7},
% pp.\ 609--616. Cambridge, MA: MIT Press.


% [2] Bower, J.M.\ \& Beeman, D.\ (1995) {\it The Book of GENESIS: Exploring
%   Realistic Neural Models with the GEneral NEural SImulation System.}  New York:
% TELOS/Springer--Verlag.


% [3] Hasselmo, M.E., Schnell, E.\ \& Barkai, E.\ (1995) Dynamics of learning and
% recall at excitatory recurrent synapses and cholinergic modulation in rat
% hippocampal region CA3. {\it Journal of Neuroscience} {\bf 15}(7):5249-5262.
% }

% %%%%%%%%%%%%%%%%%%%%%%%%%%%%%%%%%%%%%%%%%%%%%%%%%%%%%%%%%%%%


\end{document}