\section{Appendix B: PDEs setups}
\label{sec:appendb}

We provide detailed PDE setups for convection, reaction-diffusion, and 1D-reaction equations.

\textbf{Convection PDE.} The one-dimensional convection problem is a hyperbolic PDE that is commonly used to model transport phenomena. The system has the formulation with periodic boundary conditions as follows:
\begin{equation}
\begin{gathered}
    \frac{\partial u}{\partial t} + \beta \frac{\partial u}{\partial x} = 0, \:\: \forall x\in [0,2\pi], \: t\in [0,1] \\
    \texttt{IC:} u(x,0)=\sin(x), \:\:\: \texttt{BC:} u(0,t)=u(2\pi,t)
\end{gathered}    
\end{equation}

where $\beta$ is the convection coefficient. As $\beta$ increases, the frequency of its solution goes higher, and it becomes harder for PINNs to approximate. Here, we set $\beta=50$.

\textbf{1D-Reaction PDE.} The one-dimensional reaction problem is a hyperbolic PDE that is commonly used to model chemical reactions. The system has the formulation with periodic boundary conditions as follows:
\begin{equation}
\begin{gathered}
    \frac{\partial u}{\partial t} - \rho u(1-u) = 0, \:\: \forall x\in [0,2\pi], \: t\in [0,1] \\
    \texttt{IC:} u(x,0)=\exp(-\frac{(x-\pi)^2}{2(\pi/4)^2}), \:\:\: \texttt{BC:} u(0,t)=u(2\pi,t)
\end{gathered}    
\end{equation}

where $\rho$ is the reaction coefficient. Here, we set $\rho=5$. The equation has a simple analytical solution:
\begin{equation}
    u_{\texttt{analytical}} = \frac{h(x) \exp(\rho t)}{h(x)\exp(\rho t)+1-h(x)}
\end{equation}

where $h(x)$ is the function of the initial condition.

% \textbf{Reaction-Diffusion PDE.} The reaction-diffusion system is where a diffusion operator is added to the reaction equation above. The system has the formulation with periodic boundary conditions as follows:
% \begin{equation}
% \begin{gathered}
%     \frac{\partial u}{\partial t} - \nu \frac{\partial^2 u}{\partial x^2} - \rho u(1-u)= 0, \:\: \forall x\in [0,2\pi], \: t\in [0,1] \\
%     \texttt{IC:} u(x,0)=\exp(-\frac{(x-\pi)^2}{2(\pi/4)^2}), \:\:\: \texttt{BC:} u(0,t)=u(2\pi,t)
% \end{gathered}    
% \end{equation}

% where $\nu>0$ is the diffusion coefficient. Here, we set $\rho=5$ and $\nu=5$. The solution of the system can be solved via Strang splitting, i.e., splitting the equation into two separate models (a reaction component and a diffusion component):
% \begin{equation}
% \begin{gathered}
%     \frac{\partial u}{\partial t} = \rho u (1-u) \\
%     \frac{\partial u}{\partial t} = \frac{\partial^2 u}{\partial x^2}
% \end{gathered}    
% \end{equation}

\textbf{1D-Wave PDE.} The 1D-Wave equation is a hyperbolic PDE that is used to describe the propagation of waves in one spatial dimension. It is often used in physics and engineering to model various wave phenomena, such as sound waves, seismic waves, and electromagnetic waves. The system has the formulation with periodic boundary conditions as follows:
\begin{equation}
\begin{gathered}
    \frac{\partial^2 u}{\partial t^2} - \beta \frac{\partial^2 u}{\partial x^2} = 0 \, \:\: \forall x\in [0,1], \: t\in [0,1] \\
    \texttt{IC:} u(x,0)=\sin (\pi x) + \frac{1}{2}\sin(\beta\pi x), \:\: \:\frac{\partial u(x,0)}{\partial t}  =0 \\
    \texttt{BC:} u(0,t)=u(1,t) = 0
\end{gathered}    
\end{equation}
where $\beta$ is the wave speed. Here, we are specifying $\beta=3$.The equation has a simple analytical solution:
\begin{equation}
\begin{gathered}
    u(x,t) = \sin (\pi x) \cos(2\pi t) + \frac{1}{2}\sin(\beta \pi x)\cos(2\beta\pi t)
\end{gathered}
\end{equation}

\textbf{2D Navier-Stokes PDE.} The 2D Navier-Stokes equation is a parabolic PDE that consists of a pair of partial differential equations that describe the behavior of incompressible fluid flow in two-dimensional space. They are widely used in fluid dynamics to model the motion of fluids, such as air and water, in various engineering and scientific applications. The system has the formulation as follows: 
\begin{equation}
\begin{gathered}
    \frac{\partial u}{\partial t} + \lambda_1 (u\frac{\partial u}{\partial x} + v \frac{\partial u}{\partial y}) = - \frac{\partial p}{\partial x} + \lambda_2 (\frac{\partial^2 u}{\partial x^2} + \frac{\partial^2 u}{\partial v^2}) \\
    \frac{\partial v}{\partial t} + \lambda_1 (u\frac{\partial v}{\partial x} + v \frac{\partial v}{\partial y}) = - \frac{\partial p}{\partial y} + \lambda_2 (\frac{\partial^2 u}{\partial x^2} + \frac{\partial^2 u}{\partial v^2})
\end{gathered}
\end{equation}
where $u(t,x,y)$ and $v(t,x,y)$ are the $x$-component and $y$-component of the velocity field separately, and $p(t,x,y)$ is the pressure. Here, we set $\lambda_1 = 1$ and $\lambda_2 = 0.01$. The system does not have an explicit analytical solution, while the simulated solution is given by~\cite{raissi2019physics}.