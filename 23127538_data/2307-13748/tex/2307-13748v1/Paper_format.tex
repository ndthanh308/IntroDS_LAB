

%% This is a skeleton file demonstrating the use of IEEEtran.cls
%% (requires IEEEtran.cls version 1.7 or later) with an IEEE conference paper.
%%
%% Support sites:
%% http://www.michaelshell.org/tex/ieeetran/
%% http://www.ctan.org/tex-archive/macros/latex/contrib/IEEEtran/
%% and
%% http://www.ieee.org/

%%*************************************************************************
%% Legal Notice:
%% This code is offered as-is without any warranty either expressed or
%% implied; without even the implied warranty of MERCHANTABILITY or
%% FITNESS FOR A PARTICULAR PURPOSE! 
%% User assumes all risk.
%% In no event shall IEEE or any contributor to this code be liable for
%% any damages or losses, including, but not limited to, incidental,
%% consequential, or any other damages, resulting from the use or misuse
%% of any information contained here.
%%
%% All comments are the opinions of their respective authors and are not
%% necessarily endorsed by the IEEE.
%%
%% This work is distributed under the LaTeX Project Public License (LPPL)
%% ( http://www.latex-project.org/ ) version 1.3, and may be freely used,
%% distributed and modified. A copy of the LPPL, version 1.3, is included
%% in the base LaTeX documentation of all distributions of LaTeX released
%% 2003/12/01 or later.
%% Retain all contribution notices and credits.
%% ** Modified files should be clearly indicated as such, including  **
%% ** renaming them and changing author support contact information. **
%%
%% File list of work: IEEEtran.cls, IEEEtran_HOWTO.pdf, bare_adv.tex,
%%                    bare_conf.tex, bare_jrnl.tex, bare_jrnl_compsoc.tex
%%*************************************************************************

% *** Authors should verify (and, if needed, correct) their LaTeX system  ***
% *** with the testflow diagnostic prior to trusting their LaTeX platform ***
% *** with production work. IEEE's font choices can trigger bugs that do  ***
% *** not appear when using other class files.                            ***
% The testflow support page is at:
% http://www.michaelshell.org/tex/testflow/



% Note that the a4paper option is mainly intended so that authors in
% countries using A4 can easily print to A4 and see how their papers will
% look in print - the typesetting of the document will not typically be
% affected with changes in paper size (but the bottom and side margins will).
% Use the testflow package mentioned above to verify correct handling of
% both paper sizes by the user's LaTeX system.
%
% Also note that the "draftcls" or "draftclsnofoot", not "draft", option
% should be used if it is desired that the figures are to be displayed in
% draft mode.
%
\documentclass[conference, letterpaper]{IEEEtran}
% Add the compsoc option for Computer Society conferences.
%
% If IEEEtran.cls has not been installed into the LaTeX system files,
% manually specify the path to it like:






% Some very useful LaTeX packages include:
% (uncomment the ones you want to load)


% *** MISC UTILITY PACKAGES ***
%
%\usepackage{ifpdf}
% Heiko Oberdiek's ifpdf.sty is very useful if you need conditional
% compilation based on whether the output is pdf or dvi.
% usage:
% \ifpdf
%   % pdf code
% \else
%   % dvi code
% \fi
% The latest version of ifpdf.sty can be obtained from:
% http://www.ctan.org/tex-archive/macros/latex/contrib/oberdiek/
% Also, note that IEEEtran.cls V1.7 and later provides a builtin
% \ifCLASSINFOpdf conditional that works the same way.
% When switching from latex to pdflatex and vice-versa, the compiler may
% have to be run twice to clear warning/error messages.






% *** CITATION PACKAGES ***
%
%\usepackage{cite}
% cite.sty was written by Donald Arseneau
% V1.6 and later of IEEEtran pre-defines the format of the cite.sty package
% \cite{} output to follow that of IEEE. Loading the cite package will
% result in citation numbers being automatically sorted and properly
% "compressed/ranged". e.g., [1], [9], [2], [7], [5], [6] without using
% cite.sty will become [1], [2], [5]--[7], [9] using cite.sty. cite.sty's
% \cite will automatically add leading space, if needed. Use cite.sty's
% noadjust option (cite.sty V3.8 and later) if you want to turn this off.
% cite.sty is already installed on most LaTeX systems. Be sure and use
% version 4.0 (2003-05-27) and later if using hyperref.sty. cite.sty does
% not currently provide for hyperlinked citations.
% The latest version can be obtained at:
% http://www.ctan.org/tex-archive/macros/latex/contrib/cite/
% The documentation is contained in the cite.sty file itself.






% *** GRAPHICS RELATED PACKAGES ***
%
\ifCLASSINFOpdf
  % \usepackage[pdftex]{graphicx}
  % declare the path(s) where your graphic files are
  % \graphicspath{{../pdf/}{../jpeg/}}
  % and their extensions so you won't have to specify these with
  % every instance of \includegraphics
  % \DeclareGraphicsExtensions{.pdf,.jpeg,.png}
\else
  % or other class option (dvipsone, dvipdf, if not using dvips). graphicx
  % will default to the driver specified in the system graphics.cfg if no
  % driver is specified.
  % \usepackage[dvips]{graphicx}
  % declare the path(s) where your graphic files are
  % \graphicspath{{../eps/}}
  % and their extensions so you won't have to specify these with
  % every instance of \includegraphics
  % \DeclareGraphicsExtensions{.eps}
\fi
% graphicx was written by David Carlisle and Sebastian Rahtz. It is
% required if you want graphics, photos, etc. graphicx.sty is already
% installed on most LaTeX systems. The latest version and documentation can
% be obtained at: 
% http://www.ctan.org/tex-archive/macros/latex/required/graphics/
% Another good source of documentation is "Using Imported Graphics in
% LaTeX2e" by Keith Reckdahl which can be found as epslatex.ps or
% epslatex.pdf at: http://www.ctan.org/tex-archive/info/
%
% latex, and pdflatex in dvi mode, support graphics in encapsulated
% postscript (.eps) format. pdflatex in pdf mode supports graphics
% in .pdf, .jpeg, .png and .mps (metapost) formats. Users should ensure
% that all non-photo figures use a vector format (.eps, .pdf, .mps) and
% not a bitmapped formats (.jpeg, .png). IEEE frowns on bitmapped formats
% which can result in "jaggedy"/blurry rendering of lines and letters as
% well as large increases in file sizes.
%
% You can find documentation about the pdfTeX application at:
% http://www.tug.org/applications/pdftex





% *** MATH PACKAGES ***
%
%\usepackage[cmex10]{amsmath}
% A popular package from the American Mathematical Society that provides
% many useful and powerful commands for dealing with mathematics. If using
% it, be sure to load this package with the cmex10 option to ensure that
% only type 1 fonts will utilized at all point sizes. Without this option,
% it is possible that some math symbols, particularly those within
% footnotes, will be rendered in bitmap form which will result in a
% document that can not be IEEE Xplore compliant!
%
% Also, note that the amsmath package sets \interdisplaylinepenalty to 10000
% thus preventing page breaks from occurring within multiline equations. Use:
%\interdisplaylinepenalty=2500
% after loading amsmath to restore such page breaks as IEEEtran.cls normally
% does. amsmath.sty is already installed on most LaTeX systems. The latest
% version and documentation can be obtained at:
% http://www.ctan.org/tex-archive/macros/latex/required/amslatex/math/





% *** SPECIALIZED LIST PACKAGES ***
%
%\usepackage{algorithmic}
% algorithmic.sty was written by Peter Williams and Rogerio Brito.
% This package provides an algorithmic environment fo describing algorithms.
% You can use the algorithmic environment in-text or within a figure
% environment to provide for a floating algorithm. Do NOT use the algorithm
% floating environment provided by algorithm.sty (by the same authors) or
% algorithm2e.sty (by Christophe Fiorio) as IEEE does not use dedicated
% algorithm float types and packages that provide these will not provide
% correct IEEE style captions. The latest version and documentation of
% algorithmic.sty can be obtained at:
% http://www.ctan.org/tex-archive/macros/latex/contrib/algorithms/
% There is also a support site at:
% http://algorithms.berlios.de/index.html
% Also of interest may be the (relatively newer and more customizable)
% algorithmicx.sty package by Szasz Janos:
% http://www.ctan.org/tex-archive/macros/latex/contrib/algorithmicx/




% *** ALIGNMENT PACKAGES ***
%
%\usepackage{array}
% Frank Mittelbach's and David Carlisle's array.sty patches and improves
% the standard LaTeX2e array and tabular environments to provide better
% appearance and additional user controls. As the default LaTeX2e table
% generation code is lacking to the point of almost being broken with
% respect to the quality of the end results, all users are strongly
% advised to use an enhanced (at the very least that provided by array.sty)
% set of table tools. array.sty is already installed on most systems. The
% latest version and documentation can be obtained at:
% http://www.ctan.org/tex-archive/macros/latex/required/tools/


%\usepackage{mdwmath}
%\usepackage{mdwtab}
% Also highly recommended is Mark Wooding's extremely powerful MDW tools,
% especially mdwmath.sty and mdwtab.sty which are used to format equations
% and tables, respectively. The MDWtools set is already installed on most
% LaTeX systems. The lastest version and documentation is available at:
% http://www.ctan.org/tex-archive/macros/latex/contrib/mdwtools/


% IEEEtran contains the IEEEeqnarray family of commands that can be used to
% generate multiline equations as well as matrices, tables, etc., of high
% quality.


%\usepackage{eqparbox}
% Also of notable interest is Scott Pakin's eqparbox package for creating
% (automatically sized) equal width boxes - aka "natural width parboxes".
% Available at:
% http://www.ctan.org/tex-archive/macros/latex/contrib/eqparbox/





% *** SUBFIGURE PACKAGES ***
%\usepackage[tight,footnotesize]{subfigure}
% subfigure.sty was written by Steven Douglas Cochran. This package makes it
% easy to put subfigures in your figures. e.g., "Figure 1a and 1b". For IEEE
% work, it is a good idea to load it with the tight package option to reduce
% the amount of white space around the subfigures. subfigure.sty is already
% installed on most LaTeX systems. The latest version and documentation can
% be obtained at:
% http://www.ctan.org/tex-archive/obsolete/macros/latex/contrib/subfigure/
% subfigure.sty has been superceeded by subfig.sty.



%\usepackage[caption=false]{caption}
%\usepackage[font=footnotesize]{subfig}
% subfig.sty, also written by Steven Douglas Cochran, is the modern
% replacement for subfigure.sty. However, subfig.sty requires and
% automatically loads Axel Sommerfeldt's caption.sty which will override
% IEEEtran.cls handling of captions and this will result in nonIEEE style
% figure/table captions. To prevent this problem, be sure and preload
% caption.sty with its "caption=false" package option. This is will preserve
% IEEEtran.cls handing of captions. Version 1.3 (2005/06/28) and later 
% (recommended due to many improvements over 1.2) of subfig.sty supports
% the caption=false option directly:
%\usepackage[caption=false,font=footnotesize]{subfig}
%
% The latest version and documentation can be obtained at:
% http://www.ctan.org/tex-archive/macros/latex/contrib/subfig/
% The latest version and documentation of caption.sty can be obtained at:
% http://www.ctan.org/tex-archive/macros/latex/contrib/caption/




% *** FLOAT PACKAGES ***
%
%\usepackage{fixltx2e}
% fixltx2e, the successor to the earlier fix2col.sty, was written by
% Frank Mittelbach and David Carlisle. This package corrects a few problems
% in the LaTeX2e kernel, the most notable of which is that in current
% LaTeX2e releases, the ordering of single and double column floats is not
% guaranteed to be preserved. Thus, an unpatched LaTeX2e can allow a
% single column figure to be placed prior to an earlier double column
% figure. The latest version and documentation can be found at:
% http://www.ctan.org/tex-archive/macros/latex/base/



%\usepackage{stfloats}
% stfloats.sty was written by Sigitas Tolusis. This package gives LaTeX2e
% the ability to do double column floats at the bottom of the page as well
% as the top. (e.g., "\begin{figure*}[!b]" is not normally possible in
% LaTeX2e). It also provides a command:
%\fnbelowfloat
% to enable the placement of footnotes below bottom floats (the standard
% LaTeX2e kernel puts them above bottom floats). This is an invasive package
% which rewrites many portions of the LaTeX2e float routines. It may not work
% with other packages that modify the LaTeX2e float routines. The latest
% version and documentation can be obtained at:
% http://www.ctan.org/tex-archive/macros/latex/contrib/sttools/
% Documentation is contained in the stfloats.sty comments as well as in the
% presfull.pdf file. Do not use the stfloats baselinefloat ability as IEEE
% does not allow \baselineskip to stretch. Authors submitting work to the
% IEEE should note that IEEE rarely uses double column equations and
% that authors should try to avoid such use. Do not be tempted to use the
% cuted.sty or midfloat.sty packages (also by Sigitas Tolusis) as IEEE does
% not format its papers in such ways.





% *** PDF, URL AND HYPERLINK PACKAGES ***
%
%\usepackage{url}
% url.sty was written by Donald Arseneau. It provides better support for
% handling and breaking URLs. url.sty is already installed on most LaTeX
% systems. The latest version can be obtained at:
% http://www.ctan.org/tex-archive/macros/latex/contrib/misc/
% Read the url.sty source comments for usage information. Basically,
% \url{my_url_here}.





% *** Do not adjust lengths that control margins, column widths, etc. ***
% *** Do not use packages that alter fonts (such as pslatex).         ***
% There should be no need to do such things with IEEEtran.cls V1.6 and later.
% (Unless specifically asked to do so by the journal or conference you plan
% to submit to, of course. )


% correct bad hyphenation here
\hyphenation{op-tical net-works semi-conduc-tor}

%\usepackage{subcaption}

% *** GRAPHICS RELATED PACKAGES ***
%
\ifCLASSINFOpdf
   \usepackage[pdftex]{graphicx}
\else
\fi

% *** MATH PACKAGES ***
%
\usepackage[cmex10]{amsmath}
\usepackage{color}
%
\usepackage{fancyhdr}
\usepackage[caption=false,font=footnotesize]{subfig}

\renewcommand{\thispagestyle}[2]{} 


\fancypagestyle{plain}{
        \fancyhead{}
        \fancyhead[C]{first page center header}
        \fancyfoot{}
        \fancyfoot[C]{first page center footer}
}
\pagestyle{fancy}


\headheight 20pt
\footskip 20pt

\rhead{}

%Enter the first page number of your paper below
\setcounter{page}{1}

%Header
%\fancyhead[R]{\textit{(IJACSA) International Journal of Advanced Computer Science and Applications, \\ Vol. XXX, No. XXX, 2014}}
\renewcommand{\headrulewidth}{0pt}

%Footer
\fancyfoot[C]{    }
\renewcommand{\footrulewidth}{0.5pt}
\fancyfoot[R]{\thepage \  $|$ P a g e }


\begin{document}

%
% paper title
% can use linebreaks \\ within to get better formatting as desired
\title{TIME SERIES ANALYSIS APPLIED TO NOTIFICATIONS OF WORK ACCIDENTS}
%FORECASTING PERUVIAN VEGETABLES IMPORTS BASED ON MACHINE LEARNING MODELS
%PERUVIAN VEGETABLE IMPORTS FORECASTING USING MACHINE LEARNING MODELS|ALGORITHM
%FORECAST IMPORTS FROM PERUVIAN VEGETABLES USING MACHINE LEARNING
%MACHINE LEARNING APPROACH TO FORECAST IMPORTS OF PERUVIAN VEGETABLES
 

% author names and affiliations
% use a multiple column layout for up to three different
% affiliations
\author{\IEEEauthorblockN{Fernandez-Cayo Tony Gabriel}
\IEEEauthorblockA{Faculty of Statistic and Computer Engineering,\\
Universidad Nacional del Altiplano de Puno, P.O. Box 291\\
Puno - Peru.\\
Email: tfernandezc@est.unap.edu.pe}
\and
\IEEEauthorblockN{Torres-Cruz Fred}
\IEEEauthorblockA{Faculty of Statistic and Computer Engineering,\\
Universidad Nacional del Altiplano de Puno,P.O. Box 291\\
Puno - Peru.\\
Email: ftorres@.unap.edu.pe}
}

% conference papers do not typically use \thanks and this command
% is locked out in conference mode. If really needed, such as for
% the acknowledgment of grants, issue a \IEEEoverridecommandlockouts
% after \documentclass

% for over three affiliations, or if they all won't fit within the width
% of the page, use this alternative format:
% 
%\author{\IEEEauthorblockN{Michael Shell\IEEEauthorrefmark{1},
%Homer Simpson\IEEEauthorrefmark{2},
%James Kirk\IEEEauthorrefmark{3}, 
%Montgomery Scott\IEEEauthorrefmark{3} and
%Eldon Tyrell\IEEEauthorrefmark{4}}
%\IEEEauthorblockA{\IEEEauthorrefmark{1}School of Electrical and Computer Engineering\\
%Georgia Institute of Technology,
%Atlanta, Georgia 30332--0250\\ Email: see http://www.michaelshell.org/contact.html}
%\IEEEauthorblockA{\IEEEauthorrefmark{2}Twentieth Century Fox, Springfield, USA\\
%Email: homer@thesimpsons.com}
%\IEEEauthorblockA{\IEEEauthorrefmark{3}Starfleet Academy, San Francisco, California 96678-2391\\
%Telephone: (800) 555--1212, Fax: (888) 555--1212}
%\IEEEauthorblockA{\IEEEauthorrefmark{4}Tyrell Inc., 123 Replicant Street, Los Angeles, California 90210--4321}}




% use for special paper notices
%\IEEEspecialpapernotice{(Invited Paper)}




% make the title area
\maketitle


\begin{abstract}
Time series analysis applied to occupational accident reports is a powerful tool for understanding the evolution of occupational accidents over time. It provides valuable information to make informed decisions. In this study, data from reports of work accidents collected from the MINISTRY OF LABOR AND EMPLOYMENT PROMOTION – MTPE were analyzed by time series. Significant patterns and trends in accident reporting have been identified, leading to more effective prevention strategies and better health and safety management.
\end{abstract}


\begin{IEEEkeywords}Time series analysis;Notifications of work accidents;Prevention of occupational hazards;ARIMA models;temporary evolution;Occupational Health and Safety

\end{IEEEkeywords}

\IEEEpeerreviewmaketitle



\section{Introduction}
Nowadays, adequate notification of work accidents is an essential component for the management of occupational safety.\cite{Fernández-MuñizSafetyScience}.
provides valuable information on the occurrence and characteristics of accidents, allowing the identification of patterns, trends and risk factors that may affect safety at work.\par

Time series analysis breaks down data into various components, such as trend, seasonality, and random component, making it easy to identify long-term patterns and seasonal changes in workplace accident notifications. In addition, statistical models such as ARIMA (Autoregressive Integrated Moving Average) models can be used to model and predict the future occurrence of accidents at work... \cite{Rana}.

By understanding the temporal evolution of work accident reports, organizations and those responsible for occupational safety can take more effective preventive measures. Time series analysis provides valuable information to identify periods of increased risk, assess the effectiveness of the preventive measures taken and develop individual strategies to improve safety at work.\cite{Rodríguez}

This article presents a time series approach applied to industrial failure reporting. Data on reported occupational accidents are collected and statistical methods are used to identify important trends and trends. The analysis of time series allows us to better understand the temporal evolution of occupational accidents, contributing to the management of occupational safety and risk prevention in the workplace.\cite{Rodríguez2016}.

%\section{DATASET}

\section{METHODS}

In this section, we describe the models used for forecasting.

\subsection{Data collection}
The first step we will do is collect the data, the source from which we obtained the data is the MINISTRY OF LABOR
AND PROMOTION OF EMPLOYMENT - MTPE, SYSTEM OF WORK ACCIDENTS - SAT for notifications of work accidents.
These data include the date and quantity of each notification which are the initials NNAT, NNAM, NNIP, NNEO.\cite{trabajo}.

% Figure environment removed


\subsection{Data preparation}

Once the data is collected, it is important to perform proper cleaning and pre-processing. This involves checking the records for consistency, removing outliers or missing data, and making sure the data is in the proper format for time series analysis. Also, it may be necessary to adjust the temporal frequency of the data.\cite{Cowpertwait}.



\subsection{exploratory visualization}

Before applying analysis techniques, we will perform an exploratory visualization of the data. This involves plotting the time series data on a graph to identify patterns, trends, and seasonality. These visualizations can include line charts, histograms, scatter plots, and time series decomposition. \cite{ Chatfield}

% Figure environment removed

% Figure environment removed


\subsection{Decomposition of the time series}

Decomposition is an important step in time series analysis. It consists of separating the time series into its main components: trend, seasonality and residual error. The trend shows the general direction of the data in the long term, the seasonality shows repeating patterns in the short term, and the residual error represents random or unexplained variations by the trend and seasonality.\cite{Hyndman}

% Figure environment removed


% Figure environment removed

% Figure environment removed



\subsection{Modeling and forecast}

Once the time series is decomposed, we will use different statistical models to forecast future work accidents. Some common techniques include ARIMA (Autoregressive Integrated Moving Average) models, exponential smoothing models, and regression models. These models allow us to predict the occurrence of accidents and evaluate the impact of possible predictor variables on the time series.\cite{Brockwell} 

% Figure environment removed



\subsection{Model evaluation}

Evaluates the accuracy of the forecast model using error measures such as the mean square error (MSE) or the mean absolute error (MAE). Compare the predicted values with the actual values of workplace accident notifications to determine the effectiveness of the model.It is a stationary Gaussian process where the mean and variance are independent and the covariance of two variables will depend on the time lag of k
\cite{Cowpertwait}



\section{RESULTS}

Studies conducted from 2012 to 2020 We identified time series patterns and trends revealed significant patterns and trends in the occurrence of work accidents over time. Seasonal fluctuations were observed, with increases in accidents during certain periods of the year. In addition, cyclical patterns were identified in which accidents tend to recur at certain times of the day or days of the month.\cite{Cowpertwait} These results provide key information to direct preventive measures at times and areas of greatest riska. Time series analysis revealed significant seasonal patterns in accident reports with increases in certain months of the year. There was also an overall downward trend in accidents over the five years of the study. In addition, drastic changes were identified in the accident report in response to the security measures of the institution.\cite{Shumway}

% Figure environment removed

Through time series analysis, forecast models were developed to predict the future occurrence of work accidents.\cite{Wang} These models make it possible to anticipate periods of increased risk and take proactive preventive measures. For example, if an upward trend in the occurrence of accidents has been identified, organizations can intensify security measures during those specific periods to reduce the probability of incidents.\cite{stoffer}


% Figure environment removed
\vspace{3mm}

% Figure environment removed
\vspace{3mm}

% Figure environment removed
\vspace{3mm}
% Figure environment removed

\vspace{2mm}
\section{Discussion}

The results of the time series analysis in the notifications of work accidents provide a solid base to improve the existing preventive measures.\cite{Wei} By understanding patterns and trends, organizations can identify recurring risk areas and focus resources and efforts on implementing more effective preventative measures. These findings support informed decision making to improve job security.\cite{box}The results indicate the importance of considering seasonal factors in accident reporting and adapting preventive measures accordingly. The general downward trend in accidents suggests that the safety actions implemented have been effective. However, continuous monitoring and evaluation of prevention strategies is required to address the identified changes in accident reporting.\cite{Chowdhury}

\vspace{5mm}

The results of the time series analysis are also useful for resource planning and job security strategies. By predicting the future occurrence of accidents, organizations can allocate resources more efficiently and plan preventative measures at times and areas of greatest risk. This contributes to a more effective use of resources and a reduction in the costs associated with work accidents.\cite{Hämäläinen}

\vspace{3mm}

\section{Conclusions}
\vspace{3mm}

Time series analysis of work accident notifications provides valuable results to improve occupational safety and prevent risks in the work environment. The patterns, trends, and changes identified through this analysis make it possible to focus preventive measures, evaluate the effectiveness of implemented interventions, and plan data-based strategies to protect workers. By using time series analysis, organizations can make more informed and effective decisions in their pursuit of a safer and healthier work environment.\cite{visuri}Time series analysis of workplace accident reporting provides deeper insight into patterns and trends in workplace safety. These findings can be used to improve risk prevention strategies and safety management in the work environment. Ongoing collaboration between employers, workers and competent authorities is needed to ensure a safe and healthy working environment.\cite{Cowpertwaitt}

%\bibliography{IEEEabrv,../bib/paper}
%
% <OR> manually copy in the resultant .bbl file
% set second argument of \begin to the number of references
% (used to reserve space for the reference number labels box)
\begin{thebibliography}{1}

%\bibitem{Fernández-MuñizSafetyScience}
%Fernández-Muñiz, B., Montes-Peón, J. M., and Vázquez-Ordás, \emph{A %Guide to \LaTeX}, 3rd~ed.\hskip 1em plus
%0.5em minus 0.4em\relax Harlow, England: Addison-Wesley, 2009.
  
\bibitem{Fernández-MuñizSafetyScience}
Fernández-Muñiz, B., Montes-Peón, J. M., and Vázquez-Ordás, C. J. (2009). Relation between occupational safety management and firm performance. Safety Science, 47(7), 980-991.
Kines, P., Lappalainen, J., Mikkelsen, K. L., Olsen, E., Pousette, A., and Tharaldsen, J. E. (2011). 

\bibitem{Rana}
Rana, S., and Ahamad, M. S. (2019). Forecasting occupational accidents in manufacturing industries using ARIMA models. Safety Science, 118, 392-400.

\bibitem{Rodríguez2016}
Rodríguez, M. A., and Barreiro, M. (2016). Time series analysis of occupational accident rates in the construction industry. Safety Science, 84, 63-75.

\bibitem{trabajo}MTC.http://tablero.trabajo.gob.pe/vp/extensions/indicadores-r2105/indicadores-r2105.html/ Fuente: MINISTERIO DE TRABAJO Y PROMOCIÓN DEL EMPLEO - MTPE
SISTEMA DE ACCIDENTES DE TRABAJO - SAT


\bibitem{Chatfield}
Cowpertwait, P. S., and Metcalfe, A. V. (2019). Introductory Time Series with R. Springer.

\bibitem{Cowpertwait}
Chatfield, C. (2019). "The Analysis of Time Series: An Introduction." Chapman and Hall/CRC.

\bibitem{Hyndman}
Hyndman, R. J., and Athanasopoulos, G. (2018). Forecasting: Principles and Practice (2nd ed.). OTexts.

\bibitem{Brockwell}
Brockwell, P. J., and Davis, R. A. (2016). Introduction to Time Series and Forecasting (3rd ed.). Springer.

\bibitem{Cowpertwait}
Cowpertwait, P. S. P., and Metcalfe, A. V. (2019). "Introductory Time Series with R." Springer.

\bibitem{Shumway}
Shumway, R. H., and Stoffer, D. S. (2017). "Time Series Analysis and Its Applications: With R Examples." Springer.

\bibitem{Cowpertwait}
Cowpertwait, P. S., and Metcalfe, A. V. (2009). Introductory time series with R. Springer Science and Business Media

\bibitem{stoffer}
Shumway, R. H., and Stoffer, D. S. (2010). Time series analysis and its applications: with R examples. Springer Science and Business Media.

\bibitem{box}
Box, G. E. P., Jenkins, G. M., Reinsel, G. C., and Ljung, G. M. (2015). "Time Series Analysis: Forecasting and Control." Wiley Series in Probability and Statistics.

\bibitem{Chowdhury}
Chowdhury, R. K., and Kanji, G. K. (2009). Time Series Analysis and Forecasting of Occupational Accident Rates in the Construction Industry. Journal of Construction Engineering and Management, 135(7), 550-558.

\bibitem{Hämäläinen}
Hämäläinen, P., and Karhula, K. (2013). Time series analysis of occupational accidents and near misses in the Finnish construction industry. Safety Science, 52(1), 105-113.

\bibitem{Rodríguez}
Rodríguez, M. A., and Barreiro, M. (2016). Time series analysis of occupational accident rates in the construction industry. Safety Science, 84, 63-75.

\bibitem{Wang}
Wang, W., Wang, D., and Yuan, Z. (2017). Time series analysis of occupational accidents in the construction industry. International Journal of Environmental Research and Public Health, 14(7), 738.

\bibitem{Wei}
Wei, C. C., Wu, C. F., and Lin, W. C. (2019). Time-series analysis of occupational accidents in the Taiwanese construction industry: 2002–2013. International Journal of Environmental Research and Public Health, 16(2), 246.

\bibitem{visuri}
Visuri, K.  (2017), R Time Series Analysis Cookbook, Packt Publishing

\bibitem{Cowpertwaitt}
Cowpertwait, P. S. P., Metcalfe, A. V. (2019), introductory Time Series with R, Springer


%\bibitem{menculini2021comparing}
% L. Menculini, A. Marini, M. Proietti, A. Garinei, A. Bozza, C. Moretti,
%and M. Marconi, “Comparing prophet and deep learning to arima in
%forecasting wholesale food prices,” Forecasting, vol. 3, no. 3, pp. %644–
%662, 2021.





\end{thebibliography}




% that's all folks
\end{document}


