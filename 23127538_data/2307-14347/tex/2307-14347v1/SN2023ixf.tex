%% Beginning of file 'sample631.tex'
%%
%% Modified 2022 May  
%%
%% This is a sample manuscript marked up using the
%% AASTeX v6.31 LaTeX 2e macros.
%%
%% AASTeX is now based on Alexey Vikhlinin's emulateapj.cls 
%% (Copyright 2000-2015).  See the classfile for details.

%% AASTeX requires revtex4-1.cls and other external packages such as
%% latexsym, graphicx, amssymb, longtable, and epsf.  Note that as of 
%% Oct 2020, APS now uses revtex4.2e for its journals but remember that 
%% AASTeX v6+ still uses v4.1. All of these external packages should 
%% already be present in the modern TeX distributions but not always.
%% For example, revtex4.1 seems to be missing in the linux version of
%% TexLive 2020. One should be able to get all packages from www.ctan.org.
%% In particular, revtex v4.1 can be found at 
%% https://www.ctan.org/pkg/revtex4-1.

%% The first piece of markup in an AASTeX v6.x document is the \documentclass
%% command. LaTeX will ignore any data that comes before this command. The 
%% documentclass can take an optional argument to modify the output style.
%% The command below calls the preprint style which will produce a tightly 
%% typeset, one-column, single-spaced document.  It is the default and thus
%% does not need to be explicitly stated.
%%
%% using aastex version 6.3
\documentclass{aastex631}
%\documentclass{aastex631}

%% The default is a single spaced, 10 point font, single spaced article.
%% There are 5 other style options available via an optional argument. They
%% can be invoked like this:
%%
%% \documentclass[arguments]{aastex631}
%% 
%% where the layout options are:
%%
%%  twocolumn   : two text columns, 10 point font, single spaced article.
%%                This is the most compact and represent the final published
%%                derived PDF copy of the accepted manuscript from the publisher
%%  manuscript  : one text column, 12 point font, double spaced article.
%%  preprint    : one text column, 12 point font, single spaced article.  
%%  preprint2   : two text columns, 12 point font, single spaced article.
%%  modern      : a stylish, single text column, 12 point font, article with
%% 		  wider left and right margins. This uses the Daniel
%% 		  Foreman-Mackey and David Hogg design.
%%  RNAAS       : Supresses an abstract. Originally for RNAAS manuscripts 
%%                but now that abstracts are required this is obsolete for
%%                AAS Journals. Authors might need it for other reasons. DO NOT
%%                use \begin{abstract} and \end{abstract} with this style.
%%
%% Note that you can submit to the AAS Journals in any of these 6 styles.
%%
%% There are other optional arguments one can invoke to allow other stylistic
%% actions. The available options are:
%%
%%   astrosymb    : Loads Astrosymb font and define \astrocommands. 
%%   tighten      : Makes baselineskip slightly smaller, only works with 
%%                  the twocolumn substyle.
%%   times        : uses times font instead of the default
%%   linenumbers  : turn on lineno package.
%%   trackchanges : required to see the revision mark up and print its output
%%   longauthor   : Do not use the more compressed footnote style (default) for 
%%                  the author/collaboration/affiliations. Instead print all
%%                  affiliation information after each name. Creates a much 
%%                  longer author list but may be desirable for short 
%%                  author papers.
%% twocolappendix : make 2 column appendix.
%%   anonymous    : Do not show the authors, affiliations and acknowledgments 
%%                  for dual anonymous review.
%%
%% these can be used in any combination, e.g.
%%
%% \documentclass[twocolumn,linenumbers,trackchanges]{aastex631}
%%
%% AASTeX v6.* now includes \hyperref support. While we have built in specific
%% defaults into the classfile you can manually override them with the
%% \hypersetup command. For example,
%%
%% \hypersetup{linkcolor=red,citecolor=green,filecolor=cyan,urlcolor=magenta}
%%
%% will change the color of the internal links to red, the links to the
%% bibliography to green, the file links to cyan, and the external links to
%% magenta. Additional information on \hyperref options can be found here:
%% https://www.tug.org/applications/hyperref/manual.html#x1-40003
%%
%% Note that in v6.3 "bookmarks" has been changed to "true" in hyperref
%% to improve the accessibility of the compiled pdf file.
%%
%% If you want to create your own macros, you can do so
%% using \newcommand. Your macros should appear before
%% the \begin{document} command.
%%
\newcommand{\vdag}{(v)^\dagger}
\newcommand\aastex{AAS\TeX}
\newcommand\latex{La\TeX}

%% Reintroduced the \received and \accepted commands from AASTeX v5.2
%\received{March 1, 2021}
%\revised{April 1, 2021}
%\accepted{\today}

%% Command to document which AAS Journal the manuscript was submitted to.
%% Adds "Submitted to " the argument.
%\submitjournal{PSJ}

%% For manuscript that include authors in collaborations, AASTeX v6.31
%% builds on the \collaboration command to allow greater freedom to 
%% keep the traditional author+affiliation information but only show
%% subsets. The \collaboration command now must appear AFTER the group
%% of authors in the collaboration and it takes TWO arguments. The last
%% is still the collaboration identifier. The text given in this
%% argument is what will be shown in the manuscript. The first argument
%% is the number of author above the \collaboration command to show with
%% the collaboration text. If there are authors that are not part of any
%% collaboration the \nocollaboration command is used. This command takes
%% one argument which is also the number of authors above to show. A
%% dashed line is shown to indicate no collaboration. This example manuscript
%% shows how these commands work to display specific set of authors 
%% on the front page.
%%
%% For manuscript without any need to use \collaboration the 
%% \AuthorCollaborationLimit command from v6.2 can still be used to 
%% show a subset of authors.
%
%\AuthorCollaborationLimit=2
%
%% will only show Schwarz & Muench on the front page of the manuscript
%% (assuming the \collaboration and \nocollaboration commands are
%% commented out).
%%
%% Note that all of the author will be shown in the published article.
%% This feature is meant to be used prior to acceptance to make the
%% front end of a long author article more manageable. Please do not use
%% this functionality for manuscripts with less than 20 authors. Conversely,
%% please do use this when the number of authors exceeds 40.
%%
%% Use \allauthors at the manuscript end to show the full author list.
%% This command should only be used with \AuthorCollaborationLimit is used.

%% The following command can be used to set the latex table counters.  It
%% is needed in this document because it uses a mix of latex tabular and
%% AASTeX deluxetables.  In general it should not be needed.
%\setcounter{table}{1}

%%%%%%%%%%%%%%%%%%%%%%%%%%%%%%%%%%%%%%%%%%%%%%%%%%%%%%%%%%%%%%%%%%%%%%%%%%%%%%%%
%%
%% The following section outlines numerous optional output that
%% can be displayed in the front matter or as running meta-data.
%%
%% If you wish, you may supply running head information, although
%% this information may be modified by the editorial offices.
%\shorttitle{AASTeX v6.3.1 Sample article}
%\shortauthors{Schwarz et al.}
%%
%% You can add a light gray and diagonal water-mark to the first page 
%% with this command:
%% \watermark{text}
%% where "text", e.g. DRAFT, is the text to appear.  If the text is 
%% long you can control the water-mark size with:
%% \setwatermarkfontsize{dimension}
%% where dimension is any recognized LaTeX dimension, e.g. pt, in, etc.
%%
%%%%%%%%%%%%%%%%%%%%%%%%%%%%%%%%%%%%%%%%%%%%%%%%%%%%%%%%%%%%%%%%%%%%%%%%%%%%%%%%
%\graphicspath{{./}{figures/}}
%% This is the end of the preamble.  Indicate the beginning of the
%% manuscript itself with \begin{document}.

\begin{document}

\title{Photometry of Type II Supernova SN 2023ixf with a Worldwide Citizen Science Network}



\author[0000-0001-6629-5399]{Lauren A. Sgro}
\affiliation{SETI Institute, 339 N Bernardo Ave Suite 200, Mountain View, CA 94043, USA}
\affiliation{Unistellar, 5 all\'ee Marcel Leclerc, b\^{a}timent B, Marseille, 13008, France}

\author[0000-0002-0792-3719]{Thomas M. Esposito}
\affiliation{SETI Institute, 339 N Bernardo Ave Suite 200, Mountain View, CA 94043, USA}
\affiliation{Unistellar, 5 all\'ee Marcel Leclerc, b\^{a}timent B, Marseille, 13008, France}
\affiliation{Department of Astronomy, University of California, Berkeley, CA 94720, USA}

\author[0000-0002-0973-4276]{Guillaume Blaclard}
\affiliation{Unistellar, 5 all\'ee Marcel Leclerc, b\^{a}timent B, Marseille, 13008, France}

\author[0000-0001-6395-6702]{Sebastian Gomez}
\affiliation{Space Telescope Science Institute, 3700 San Martin Drive, Baltimore, MD 21218, USA}

\author[0000-0001-7016-7277]{Franck Marchis}
\affiliation{SETI Institute, 339 N Bernardo Ave Suite 200, Mountain View, CA 94043, USA}
\affiliation{Unistellar, 5 all\'ee Marcel Leclerc, b\^{a}timent B, Marseille, 13008, France}

\author[0000-0003-3460-0103]{Alexei V. Filippenko}
\affiliation{Department of Astronomy, University of California, Berkeley, CA 94720, USA}

\author[0000-0002-9427-0014]{Daniel O’Conner Peluso}
\affiliation{SETI Institute, 339 N Bernardo Ave Suite 200, Mountain View, CA 94043, USA}
\affiliation{Centre for Astrophysics, University of Southern Queensland, Toowoomba, QLD, Australia}

\author[0000-0002-7491-7052]{Stephen S. Lawrence}
\affiliation{Dept. of Physics \& Astronomy, 151 Hofstra University, Hempstead, NY 11549, USA}
\affiliation{Unistellar Citizen Scientist}

\author{Aad Verveen} 
\affiliation{Unistellar Citizen Scientist}
\author{Andreas Wagner}
\affiliation{Unistellar Citizen Scientist}
\author{Anouchka Nardi}
\affiliation{Unistellar Citizen Scientist}
\author{Barbara Wiart}
\affiliation{Unistellar Citizen Scientist}
\author{Benjamin Mirwald}
\affiliation{Unistellar Citizen Scientist}
\author{Bill Christensen}
\affiliation{Unistellar Citizen Scientist}
\author{Bob Eramia}
\affiliation{Unistellar Citizen Scientist}
\author[0000-0002-9578-5765]{Bruce Parker} 
\affiliation{Unistellar Citizen Scientist}
\author[0000-0003-4091-0247]{Bruno Guillet} 
\affiliation{Unistellar Citizen Scientist}
\author{Byungki Kim}
\affiliation{Unistellar Citizen Scientist}
\author[0009-0003-4751-4906]{Chelsey A. Logan}
\affiliation{Unistellar Citizen Scientist}
\author[0000-0001-7014-1843]{Christopher C. M. Kyba}
\affiliation{Unistellar Citizen Scientist}
\author{Christopher Toulmin}
\affiliation{Unistellar Citizen Scientist}
\author{Claudio G. Vantaggiato}
\affiliation{Unistellar Citizen Scientist}
\author{Dana Adhis}
\affiliation{Unistellar Citizen Scientist}
\author{Dave Gary}
\affiliation{Unistellar Citizen Scientist}
\author{Dave Goodey}
\affiliation{Unistellar Citizen Scientist}
\author{David Dickinson}
\affiliation{Unistellar Citizen Scientist}
\author{David Koster}
\affiliation{Unistellar Citizen Scientist}
\author{Davy Martin}
\affiliation{Unistellar Citizen Scientist}
\author[0009-0003-4759-1434]{Eliud Bonilla}
\affiliation{Unistellar Citizen Scientist}
\author{Enner Chung}
\affiliation{Unistellar Citizen Scientist}
\author{Eric Miny}
\affiliation{Unistellar Citizen Scientist}
\author[0000-0002-6818-6599]{Fabrice Mortecrette}
\affiliation{Unistellar Citizen Scientist}
\author[0009-0003-6483-0433]{Fadi Saibi}
\affiliation{Unistellar Citizen Scientist}
\author[0009-0008-5454-6929]{Francois O. Gagnon}
\affiliation{Unistellar Citizen Scientist}
\author{François Simard}
\affiliation{Unistellar Citizen Scientist}
\author{Gary Vacon}
\affiliation{Unistellar Citizen Scientist}
\author{Georges Simard}
\affiliation{Unistellar Citizen Scientist}
\author{Gerrit Dreise}
\affiliation{Unistellar Citizen Scientist}
\author{Hiromi Funakoshi}
\affiliation{Unistellar Citizen Scientist}
\author{Janet Vacon}
\affiliation{Unistellar Citizen Scientist}
\author{James Yaniz}
\affiliation{Unistellar Citizen Scientist}
\author[0009-0000-0969-2216]{Jean-Charles Le Tarnec}
\affiliation{Unistellar Citizen Scientist}
\author[0000-0002-1908-6057]{Jean-Marie Laugier}
\affiliation{Unistellar Citizen Scientist}
\author{Jennifer L. W. Siders}
\affiliation{Unistellar Citizen Scientist}
\author{Jim Sweitzer}
\affiliation{Unistellar Citizen Scientist}
\author{Jiri Dvoracek}
\affiliation{Unistellar Citizen Scientist}
\author[0009-0008-1227-5083]{John Archer}
\affiliation{Unistellar Citizen Scientist}
\author{John Deitz}
\affiliation{Unistellar Citizen Scientist}
\author{John K. Bradley}
\affiliation{Unistellar Citizen Scientist}
\author[0000-0002-9297-5133]{Keiichi Fukui}
\affiliation{Unistellar Citizen Scientist}
\author[0000-0003-0690-5508]{Kendra Sibbernsen}
\affiliation{Unistellar Citizen Scientist}
\author{Kevin Borrot}
\affiliation{Unistellar Citizen Scientist}
\author{Kevin Cross}
\affiliation{Unistellar Citizen Scientist}
\author[0009-0004-1079-7196]{Kevin Heider}
\affiliation{Unistellar Citizen Scientist}
\author{Koichi Yamaguchi}
\affiliation{Unistellar Citizen Scientist}
\author[0000-0001-8058-7443]{Lea A. Hirsch}
\affiliation{Unistellar Citizen Scientist}
\author[0000-0003-3046-9187]{Liouba Leroux}
\affiliation{Unistellar Citizen Scientist}
\author[0000-0002-3278-9590]{Mario Billiani}
\affiliation{Unistellar Citizen Scientist}
\author{Markus Lorber}
\affiliation{Unistellar Citizen Scientist}
\author{Martin J. Smallen}
\affiliation{Unistellar Citizen Scientist}
\author{Masao Shimizu}
\affiliation{Unistellar Citizen Scientist}
\author{Masayoshi Nishimura}
\affiliation{Unistellar Citizen Scientist}
\author[0000-0001-8337-0020]{Matthew Ryno}
\affiliation{Unistellar Citizen Scientist}
\author{Michael Cunningham}
\affiliation{Unistellar Citizen Scientist}
\author{Michael Gagnon}
\affiliation{Unistellar Citizen Scientist}
\author[0000-0003-3462-7533]{Michael Primm}
\affiliation{Unistellar Citizen Scientist}
\author{Michael Rushton}
\affiliation{Unistellar Citizen Scientist}
\author[0000-0001-7412-0193]{Michael Sibbernsen}
\affiliation{Unistellar Citizen Scientist}
\author{Mike Mitchell}
\affiliation{Unistellar Citizen Scientist}
\author{Neil Yoblonsky}
\affiliation{Unistellar Citizen Scientist}
\author{Niniane Leroux}
\affiliation{Unistellar Citizen Scientist}
\author{Olivier Clerget}
\affiliation{Unistellar Citizen Scientist}
\author[0000-0002-9418-3754]{Ozren Stojanović}
\affiliation{Unistellar Citizen Scientist}
\author{Patrice Unique}
\affiliation{Unistellar Citizen Scientist}
\author[0000-0003-1371-4232]{Patrick Huth}
\affiliation{Unistellar Citizen Scientist}
\author[0000-0002-8703-6430]{Raymund John Ang}
\affiliation{Unistellar Citizen Scientist}
\author{Regis Santoni}
\affiliation{Unistellar Citizen Scientist}
\author{Robert Foster}
\affiliation{Unistellar Citizen Scientist}
\author{Roberto Poggiali}
\affiliation{Unistellar Citizen Scientist}
\author[0009-0004-0672-2255]{Ruyi Xu}
\affiliation{Unistellar Citizen Scientist}
\author[0000-0001-7029-644X]{Ryuichi Kukita}
\affiliation{Unistellar Citizen Scientist}
\author{Sanja Šćepanović}
\affiliation{Unistellar Citizen Scientist}
\author[0009-0006-8494-5408]{Sophie Saibi}
\affiliation{Unistellar Citizen Scientist}
\author[0000-0003-0404-6279]{Stefan Will}
\affiliation{Unistellar Citizen Scientist}
\author{Stephan Latour}
\affiliation{Unistellar Citizen Scientist}
\author{Stephen Haythornthwaite}
\affiliation{Unistellar Citizen Scientist}
\author{Sylvain Cadieux}
\affiliation{Unistellar Citizen Scientist}
\author[0000-0002-9089-6853]{Thoralf Müller}
\affiliation{Unistellar Citizen Scientist}
\author[0000-0003-1368-861X]{Tze Yang Chung}
\affiliation{Unistellar Citizen Scientist}
\author{Yoshiya Watanabe}
\affiliation{Unistellar Citizen Scientist}
\author{Yvan Arnaud}
\affiliation{Unistellar Citizen Scientist}


 
\begin{abstract}

We present highly sampled photometry of the supernova (SN) 2023ixf, a Type II SN in M101, beginning 2~days before its first known detection. To gather these data, we enlisted the global Unistellar Network of citizen scientists. These 252 observations from 115 telescopes show the SN's rising brightness associated with shock emergence followed by gradual decay. We measure a peak $M_{V} = -18.18 \pm 0.09$~mag at 2023-05-25 21:37 UTC in agreement with previously published analyses.

\end{abstract}

\keywords{Supernovae (1668) -- Type II supernovae (1731) -- Core-collapse supernovae (304)}

\section{Introduction} \label{sec:intro}
Type II supernovae (SNe) are hydrogen-rich, core-collapse SNe (see \citealt{filippenko1997optical} for a review of SN classification) and are among the most commonly observed SNe (e.g., \citealt{li2011nearby}). Despite their prevalence, early-time observations of these SNe are rarely available owing to the cadence of large surveys and other factors. Nevertheless, data within days after shock breakout, in which the shock wave from the collapsing core reaches the stellar photosphere, are imperative for gaining an understanding of the progenitor and explosion physics \citep{waxman2017handbook}.

\citet{2023TNSTR1158....1I} reported discovery of a SN in M101 (redshift $z=0.0008$) in observations from 2023-05-19 17:27 (UTC dates are used throughout this paper). This SN, designated 2023ixf, offered an opportunity for amateur and professional astronomers to collect data promptly. The earliest known detection was found during spontaneous observations by \citet{{2023TNSAN.130....1M}} at 2023-05-18 20:29. See \citet{2023arXiv230606097H} and references therein for a review of early-time photometry. 

\citet{2023TNSAN.119....1P} revealed SN 2023ixf as a Type II SN. The progenitor candidate has been identified in archival {\it Spitzer Space Telescope} and {\it Hubble Space Telescope} images \citep{szalai2023spitzer,soraisam,pledger} as a luminous red supergiant with a dense shell of circumstellar material and long-period variability in near-to-mid-infrared wavelengths \citep{Jacobson-Galan2023,Jencson,Kilpatrick}. Future studies will further constrain the progenitor and SN to increase understanding of Type II SNe.

\section{Observations \& Data Reduction} \label{sec:obs}

All data used in this work were taken by the Unistellar Network, comprised of observers worldwide who use Unistellar telescopes \textemdash~ 11.4~cm aperture digital, smart telescopes \citep{marchis2020unistellar}. Each telescope employs a CMOS sensor sensitive to blue, green, and red bandpasses via a Bayer filter. Uniform optical properties simplify combination of data from multiple telescopes, making possible results such as those described by \citet{graykowski2023light}, \citet{perrocheau202216}, and \citet{peluso2023unistellar}. 

Unistellar telescopes can record data in two modes, Enhanced Vision (EV) and Science mode, which were utilized to measure the light curve of SN 2023ixf presented in Figure \ref{fig:LC}. The EV data were taken during prediscovery and postdiscovery serendipitous imaging of M101, while coordinated Science observations commenced ${\sim}29$~hr after discovery \citep{2023TNSTR1158....1I}. 

For each observation, raw images were dark-subtracted, if dark frames were taken, and plate-solved. Images that were off-target or had insufficient quality to plate solve were discarded. Calibrated images were aligned and averaged into stacked images with integration time $\approx60$~s. To separate the SN signal from the host galaxy, stacks were high-pass filtered using a median boxcar with width equal to 6 aperture radii. Differential aperture photometry was then performed on stacked images to measure fluxes of the SN and 3--5 reference stars with known {\it Gaia} magnitudes that were transformed to the Johnson-Cousins $V$ band (\citealt{2022arXiv220800211G}). The radius of the circular aperture was minimized to enclose $\geq90$\% of the reference stars' flux, which varied from $4\arcsec$ to $12\arcsec$ (3--7 pixels) to accommodate observing conditions.

The SN $m_V$ was then calculated as $m_V=m_{Vref} + 2.5\mathrm{log}(F_{ref})/F_{SN})$, where $m_{Vref}$ is the reference star's apparent $V$ magnitude and $F_{ref}$ and $F_{SN}$ are the measured reference star and SN fluxes, respectively. This was repeated for all reference stars and stacks. Measurements of the SN $m_V$ with signal-to-noise ratios $< 5$ were discarded, and we report the mean of the remaining magnitudes. The standard deviation of magnitudes within a given observation is the reported uncertainty. We consider observations where the SN $m_V$ exceeds the standard deviation of the background noise ($m_V - \sigma_{bg}$) to be non-detections.

% Figure environment removed

\section{Results} \label{sec:data}

Figure \ref{fig:LC} contains 243 detections and 9 nondetection limits from 252 observations by 123 observers, 88 whom were identified because EV data are uploaded anonymously. Our earliest prediscovery detection was at 05-19 01:18, 4.83~hr after \citet{2023TNSAN.130....1M}'s detection and 16.15~hr before the discovery epoch. Additionally, \citet{2023TNSAN.123....1F} used a Unistellar telescope and were first to report a nonsurvey prediscovery detection at 2023-05-19 06:08.

We use a Markov Chain Monte Carlo (MCMC) sampling to obtain a best-fit model light curve with an exponential rise and linear decay. We obtain a peak $m_{v} = 11.05 \pm 0.08$ at 2023-05-25 21:37 $\pm$ 62~min, corresponding to $M_{V} = -18.18 \pm 0.09$~mag using a distance of $6.71 \pm 0.14$~Mpc and $E(B-V) = 0.031$~mag \citep{riess20162}. These values strongly agree with those found via meter-class telescopes (e.g., \citealt{Jacobson-Galan2023}). Our modeling implies an explosion time of 2023-05-18 22:43 $\pm$ 15~min, but this is 2.23~hr post earliest detection and not meaningfully constrained because our model is not well-fit to the initial rise (${\sim}1$ day, similar to \citealt{2023arXiv230606097H}).


\section{Conclusion} \label{sec:conc}

Here we present a light curve of SN 2023ixf with a 3.3~hr average sampling time over 35 days. Our modeled light-curve parameters support those gathered by professional telescopes and presented in other works. As such, this study provides crucial data to the community, but also demonstrates the power of a global observing network using telescopes with homogeneous opto-electronics, like Unistellar telescopes.



\vspace{5mm}
\facilities{Unistellar}
\vspace{15mm}

%______________________________________


\bibliography{SN2023ixf}{}
\bibliographystyle{aasjournal}

%% This command is needed to show the entire author+affiliation list when
%% the collaboration and author truncation commands are used.  It has to
%% go at the end of the manuscript.
%\allauthors

%% Include this line if you are using the \added, \replaced, \deleted
%% commands to see a summary list of all changes at the end of the article.
%\listofchanges

\end{document}

