%% 
%% Copyright 2007-2020 Elsevier Ltd
%% 
%% This file is part of the 'Elsarticle Bundle'.
%% ---------------------------------------------
%% 
%% It may be distributed under the conditions of the LaTeX Project Public
%% License, either version 1.2 of this license or (at your option) any
%% later version.  The latest version of this license is in
%%    http://www.latex-project.org/lppl.txt
%% and version 1.2 or later is part of all distributions of LaTeX
%% version 1999/12/01 or later.
%% 
%% The list of all files belonging to the 'Elsarticle Bundle' is
%% given in the file `manifest.txt'.
%% 

%% Template article for Elsevier's document class `elsarticle'
%% with numbered style bibliographic references
%% SP 2008/03/01
%%
%% 
%%
%% $Id: elsarticle-template-num.tex 190 2020-11-23 11:12:32Z rishi $
%%
%%
%%\documentclass[preprint,12pt]{elsarticle}

%% Use the option review to obtain double line spacing
%% \documentclass[authoryear,preprint,review,12pt]{elsarticle}

%% Use the options 1p,twocolumn; 3p; 3p,twocolumn; 5p; or 5p,twocolumn
%% for a journal layout:
%% \documentclass[final,1p,times]{elsarticle}
%%\documentclass[final,1p,times,twocolumn]{elsarticle}
%% \documentclass[final,3p,times]{elsarticle}
%%\documentclass[final,3p,times,twocolumn]{elsarticle}
%% \documentclass[final,5p,times]{elsarticle}
\documentclass[final,5p,times,twocolumn]{elsarticle}

%% For including figures, graphicx.sty has been loaded in
%% elsarticle.cls. If you prefer to use the old commands
%% please give \usepackage{epsfig}

%% The amssymb package provides various useful mathematical symbols
\usepackage{amssymb}
%% The amsthm package provides extended theorem environments
%% \usepackage{amsthm}
\usepackage{mathtools}

%% The lineno packages adds line numbers. Start line numbering with
%% \begin{linenumbers}, end it with \end{linenumbers}. Or switch it on
%% for the whole article with \linenumbers.
%% \usepackage{lineno}

\journal{Radiation Measurements}

\begin{document}

\begin{frontmatter}

%% Title, authors and addresses

%% use the tnoteref command within \title for footnotes;
%% use the tnotetext command for theassociated footnote;
%% use the fnref command within \author or \address for footnotes;
%% use the fntext command for theassociated footnote;
%% use the corref command within \author for corresponding author footnotes;
%% use the cortext command for theassociated footnote;
%% use the ead command for the email address,
%% and the form \ead[url] for the home page:
%% \title{Title\tnoteref{label1}}
%% \tnotetext[label1]{}
%% \author{Name\corref{cor1}\fnref{label2}}
%% \ead{email address}
%% \ead[url]{home page}
%% \fntext[label2]{}
%% \cortext[cor1]{}
%% \affiliation{organization={},
%%             addressline={},
%%             city={},
%%             postcode={},
%%             state={},
%%             country={}}
%% \fntext[label3]{}

\title{An easy tool for the Monte Carlo simulation of the passage of photons and electrons through matter}

%% use optional labels to link authors explicitly to addresses:
%% \author[label1,label2]{}
%% \affiliation[label1]{organization={},
%%             addressline={},
%%             city={},
%%             postcode={},
%%             state={},
%%             country={}}
%%
%% \affiliation[label2]{organization={},
%%             addressline={},
%%             city={},
%%             postcode={},
%%             state={},
%%             country={}}

\author{Víctor Moya}
\author{Jaime Rosado}
\author{Fernando Arqueros}

\affiliation{organization={IPARCOS-UCM, Instituto de Física de Partículas y del Cosmos and EMFTEL Department, Universidad Complutense de Madrid},%Department and Organization
            %addressline={Address One}, 
            city={Madrid},
            postcode={E-28040}, 
            %state={State One},
            country={Spain}}

\begin{abstract}
%% Text of abstract
A simple Monte Carlo (MC) algorithm for the simulation of the passage of low-energy gamma rays and electrons through any material medium is presented. The algorithm includes several approximations that accelerate the simulation while maintaining reasonably accurate results. Systematic comparisons for both photons and electrons have been made against the MC code PENELOPE and experimental data to validate the algorithm, showing deviations in the deposited energy smaller than or around 10\% in the energy interval of $0.1 - 5$~MeV in light media. The simulation is also valid for heavy media, but with less accuracy at high energy. The algorithm has been implemented in an open-source Python package called LegPy, which provides an easy-to-use framework for rapid MC simulations aiming to be useful for applications that do not require the level of detail of available well-established MC programs.
\end{abstract}

%%Graphical abstract
%\begin{graphicalabstract}
%% Figure removed
%\end{graphicalabstract}

%%Research highlights
%\begin{highlights}
%\item Research highlight 1
%\item Research highlight 2
%\end{highlights}

\begin{keyword}
%% keywords here, in the form: keyword \sep keyword
Monte Carlo simulation \sep ionizing radiation
%% PACS codes here, in the form: \PACS code \sep code
%\PACS 0000 \sep 1111
%% MSC codes here, in the form: \MSC code \sep code
%% or \MSC[2008] code \sep code (2000 is the default)
%\MSC 0000 \sep 1111
\end{keyword}

\end{frontmatter}

%% \linenumbers

%% main text
\section{Introduction}
\label{sec:intro}

The study of the interaction of ionizing radiation with matter is of utmost importance in a wide range of applications. Monte Carlo (MC) simulations are extensively used to this purpose, and several excellent MC programs are available, such as EGS4 \cite{egs4}, PENELOPE \cite{penelope}, GEANT4 \cite{geant4} and MCNP \cite{mcnp}, to name a few. These programs provide frameworks for detailed simulations of any case of study, including complex geometrical forms and all the physical processes that ionizing particles may undergo in a broad energy range. Moreover, these programs typically offer multiple options for different physical models and corrections due to the specific features of the atomic composition of the media.
Unfortunately, such accuracy comes at the cost of increasing technical complexity, which may require users to acquire expertise to obtain meaningful results. Additionally, the more detailed a simulation is, the more time and computing resources it demands, which could limit the use of these programs in some applications.
As an alternative, analytical calculations or simple models can be used to obtain estimates of the desired results. However, this approach generally lacks the accuracy and level of detail provided by MC simulations.

In this paper we present a simplified MC algorithm for the simulation of the passage of low-energy gamma rays and electrons through any material medium. The algorithm is based on several approximations that enhance simulation speed and simplicity while maintaining reasonably accurate results. In particular, we developed a very simple model for electron transportation that accelerates the simulations significantly. In this work, we analyzed the range of validity and assessed the impact of these approximations by comparing them against both experimental data and results from well-established MC programs, especially PENELOPE \cite{penelope}. We implemented the algorithm in a Python package called LegPy, released under an open-source license \cite{legpy}. This tool aims to provide an easy-to-use framework for rapid simulations in applications where minor details are not necessary.

The paper is structured as follows. In section \ref{sec:algorithm}, the physical approximations employed in our MC algorithm and the main features of the package LegPy are described. The validation of the algorithm is presented in section \ref{sec:validation}. Lastly, in section \ref{sec:conclusions}, the conclusions drawn, the potential uses of our algorithm, and the improvement plans are discussed.

\section{The algorithm}
\label{sec:algorithm}

Our Monte Carlo algorithm was conceived to provide accurate enough results for a wide range of situations using a small amount of input data and computing resources. Under this approach, we neglected several effects on the transportation of photons and electrons that are only expected to be relevant at either high energy or very small scale. For the transportation of photons, pair production (energy threshold of 1.02 MeV) was ignored. Besides, several simplifications were made in photoelectric absorption, coherent scattering, and incoherent scattering. For the transportation of electrons, we developed a novel method to account for multiple scattering and collisional energy loss in a very simple and fast way. Bremsstrahlung was also ignored because it is only significant at high energy and heavy media.

We focused on situations where a beam of photons or electrons interacts with an object made of one or more homogeneous materials. The algorithm was designed to track all the individual particles inside this object, but only keep information on the energy deposit in a voxel-based scheme. Histograms of other relevant parameters of the beam particles (e.g., the angle and energy of escaping particles, the absorbed energy, and the maximum depth of electrons) can also be computed.

Next, the approximations made in the transportation of photons and electrons are described. In the last subsection, we briefly report on the LegPy package \cite{legpy}.

\subsection{Photon transportation}
\label{ssec:photon}

The algorithm transports photons in random steps. The distance a photon travels before it undergoes its next interaction is randomly obtained from its mean free path, which is calculated from the total attenuation coefficient of the medium at the photon energy. The photon is transported this distance in the direction of its momentum vector to the next interaction point as long as it is inside the same medium, otherwise, the photon escapes the object or the medium changes. In the latter case, the photon is transported to the boundary between the two media to take another step in the new medium.

When a photon interacts, the type of interaction (i.e., photoelectric absorption, coherent and incoherent scattering) is determined at random from the relative attenuation coefficients of the different processes. Both total and relative attenuation coefficients are taken from the NIST Standard Reference Database XCOM \cite{nist_ph}. If photoelectric absorption takes place, an electron is emitted with a kinetic energy and propagation direction equal to those of the absorbed photon. In the case of incoherent scattering, the momentum vector and energy of both the photon and the electron are sampled from the differential cross section, given by the Klein-Nishina formula \cite{evans}, and the energy-momentum conservation laws. If the photon undergoes a coherent scattering, the Thompson scattering law \cite{evans} is used to randomly deviate its track.

As already mentioned, pair production is not included in our simplified algorithm. Nevertheless, the attenuation coefficient corresponding to this process is added to that of photoelectric absorption so that the attenuation of a photon beam is properly simulated. Furthermore, the atomic effects on the above processes are ignored. In particular, the fluorescence subsequent to photoelectric absorption or incoherent scattering, i.e., the emission of X-rays by the excited atom, is not included. This is one of the main limitations of our approximation, because X-rays spread the energy deposit at larger distances than photoelectrons do. Neglecting this effect has a significant impact for heavy elements, as will be discussed later. However, our approximation is accurate enough for light elements.


\subsection{Electron transportation}
\label{ssec:electron}

The electron transportation is based on the continuous slowing down approximation (CSDA), that is, the rate of energy loss of an electron along its track is assumed to be determined by the total stopping power neglecting fluctuations. Therefore, the total path traveled by an electron is assumed to be equal to the CSDA range in the medium. Both the total stopping power and the CSDA range are taken from the NIST Standard Reference Database ESTAR \cite{nist_estar}. The electron path is divided into a number of steps with length chosen adequately to the desired precision, e.g., equal to the voxel size used in the simulation. Since energy-loss fluctuations are neglected, the simulation uses precomputed tables with the electron energies at the endpoints of all the steps in each constituent material of the object. When an electron is generated with a given initial energy in a medium, the energy loss and distance traveled in its first step are obtained by interpolation from the corresponding table. All the subsequent steps follow this table until the electron stops or the medium changes. In the latter case, the last step in the first medium is shortened to end at the boundary between the two media, and the energy loss is calculated accordingly. Then, the electron continues to be transported in the second media (if any) as if it had been generated at that point.

In each step, the end position of the electron is determined by its starting position, its propagation direction, and the step length, i.e., the electron is assumed to travel in a straight line ignoring the lateral displacement due to multiple scattering. On the other hand, after the step is taken, the electron propagation direction is randomly deviated. The axial angle of the deviation is a random number in the $0-2\pi$ interval and the scattering angle $\theta$ is sampled from a Gaussian distribution \cite{theta_0}
\begin{equation} \label{eq:gauss}
P(\theta) = \frac{1}{\sqrt{2\pi}\theta_0} \exp\left({-\frac{\theta^2}{2\theta_0^2}}\right)\,,
\end{equation}
where the average scattering angle $\theta_0$ is given by
\begin{equation} \label{eq:theta_0}
\theta_0 = \frac{E_0}{\beta cp} \sqrt{\frac{s}{X_0}} \left(1+0.038\ln\frac{s}{X_0 \beta^2} \right)\,.
\end{equation}
Here, $s$ is the step length, $X_0$ is the radiation length of the medium, $\beta$ is the ratio of the electron velocity and the speed of light $c$, $p$ is the electron momentum and $E_0$ is a model parameter. In the Gaussian approximation described in \cite{theta_0}, $E_0$ takes the value 13.6~MeV. However, we modified this parameter to compensate for the various simplifications made in electron transportation. First, we searched the values of $E_0$ that make our algorithm reproduce approximately both the energy deposit distribution and the backscattering factor obtained by PENELOPE \cite{penelope} for the energy range $0.1-5.0$~MeV and a number of media. Then, from these fitted values, we obtained the following parameterization of $E_0$ in MeV:
\begin{equation}
\label{eq:E0}
\begin{split}
      E_0 = &13.6\,(1.56+0.130x) \\
          &\left[ 1 - (0.0471-0.0182\ln\,x)\ln\,E \right]  
\end{split}
\end{equation}
%where $x=X_0^{-0.5}$ for $X_0$ expressed in cm and $E$ in MeV.
where $x=1/\sqrt{X_0}$ for $X_0$ expressed in cm and $E$ in MeV.

The necessary $\theta_0$ values are calculated at the beginning of the simulation and stored in the table of steps for each medium. The results obtained with this simple model of multiple scattering were checked to be almost independent of the step length as long as it is smaller than $10\%$ of the CSDA range.

As a further simplification, half of the energy lost by an electron in a step is deposited in the starting voxel and the other half in the end one. To avoid discontinuities in the energy deposit distribution, the step length should be smaller than the voxel size. Besides, when an electron reaches the boundary between two media, the energy loss in that step is assumed to be fully deposited in the starting voxel to prevent artifacts just at the boundary. In our approximation, no secondary particles are simulated. In particular, Bremsstrahlung is ignored, which limits the applicability of our logarithm to moderate electron energies.

% Figure environment removed

The use of all these approximations speeds up the simulation considerably while a reasonable accuracy is achieved, as will be shown in section \ref{sec:validation}. The main limitation is the underestimation of the maximum range of electrons because their total path length is imposed to equal the CSDA range.

\subsection{LegPy framework}
\label{ssec:legpy}

LegPy stands for ``low energy gamma-ray simulation in Python'', but it allows the simulation of both photons and electrons using the approximations described above. The code is organized as a Python package containing classes to define the basic ingredients of a simulation: the beam geometry, the energy spectrum and type of the incident particles, the object geometry, the medium (or media) the object is made of, and the simulation settings. LegPy is designed to be used in Google Colab \cite{gcolab} or Jupyter Notebook, which are interactive environments that improve user experience. Several notebook examples are included in the last release of this software package.

The package includes a library of media containing all the necessary data taken from NIST databases, so the user only has to choose the media from this library. Adding new media to the library is also quite easy. Presently, only three object geometries (i.e., sphere, cylinder, and orthohedron) are supported. Any object can be divided into voxels in Cartesian coordinates, although spherical and cylindrical voxelizations are also available for objects having these symmetries. The user only needs to select the geometry, input the dimensions of the object, and give the number of voxels along each dimension. The beam of particles is assumed to enter the bottom surface of the object, which is always oriented along the z-axis. Both parallel and divergent beams can be used and a set of predefined energy spectra are available. For photon beams, the tracking of secondary electrons can be turned on and off. All these simplifications and default options provide a simple but flexible simulation framework.

In Fig. \ref{fig:legpy}, a sample of code that executes a simulation with LegPy and a couple of output plots are shown. The object dubbed ``result'' in this example stores the results of the simulation and has several methods to plot the energy deposit distribution, the particle tracks and histograms of various relevant parameters. The package includes additional functions to analyze the simulation results. All the results shown in this paper have been obtained with LegPy.

%In case a particular medium of interest were not available, a generic procedure has been implemented in LegPy to obtain these coefficients from the atomic composition, the atomic mass of each element and the density. In this generic option, coherent scattering is neglected and the total cross section of incoherent scattering is given by the Klein-Nishina formula. The photoelectric cross section is calculated using the approximation described in \cite{amjphy}. Although this procedure might lead to non-negligible deviations in some energy intervals, it might be useful to obtain rough estimates of many parameters of interest. We will show some comparisons later.


\section{Validation of the algorithm}
\label{sec:validation}
In order to check the validity of the code, a number of tests were made to compare our results with experimental data and those from other MC codes. In particular, we carried out systematic comparisons with the well-established code PENELOPE \cite{penelope}. These tests were done for the geometries currently supported by LegPy, i.e., cylinder, orthoedron, and sphere, and for the cases of a single medium and two media. Monoenergetic beams of photons and electrons in the range $0.1 - 5.0$~MeV were used. 

\subsection{Photon beams}
\label{sec:photons}
For these tests, the LegPy simulations were carried out without the transportation of secondary electrons, that is, the electrons generated by a photoelectric absorption or a Compton effect are assumed to be absorbed depositing their kinetic energy in the production point. In the first set of tests, the dose in depth calculated with LegPy for several media at various energies was compared with the results from PENELOPE and MCNP5. Other tests were focused on the angular distribution of escaping photons and the energy spectra of absorbed energy relevant to the characterization of scintillators.

%JR: La explicación de los buildup factors me sigue pareciendo un poco larga. Yo diría lo siguiente para acortar:
% FA: OK lo he cambiado así.

%The dose in depth comparisons were done through the buildup factor for a point source in an infinite medium, which is defined as....
%Here, D_r(R) is the ``real'' dose as a function of the distance from the source, which can be calculated by simulation.  The parameter D_t(R) is the ``theoretical'' dose that would be obtained if secondary radiation (e.g., Compton scattered photons) is ignored, which is given by...
%where \mu and ...
%Buildup factors are very relevant [...] in medical physics. Note that comparing the buildup factor is equivalente to comparing... as long as the same coefficients...
%Using LegPy, we calculated B(R)....
%
The dose in depth comparisons were done through the buildup factor $B$ for a point source in an infinite medium, which is defined as,
%As is well known, the passage of a beam of gamma-rays through a medium gives rise to secondary radiation (e.g., Compton scattered photons) that contributes, along with the primary one, to the dose. This effect is customarily quantified by the so-called buildup factor $B$ defined as 
\begin{equation} \label{eq:B}
B = \frac{D_{\rm r}}{D_{\rm t}}
\end{equation}
where $D_{\rm r}(R)$ is the ``real'' dose as a function of the distance from the source $R$, which can be calculated by simulation. The ``theoretical'' dose $D_t(R)$, that is, the expected one if secondary radiation (e.g., Compton scattered photons) is ignored, can be calculated from the expression
\begin{equation} \label{eq:Dt}
D_{\rm t}(R) = \frac{1}{4\pi R^2}  \exp\left({-\mu R}\right) \left( \frac{\mu_{\rm en}}{\rho} \right) E
\end{equation}
where $\mu$ and $\mu_{\rm en}/\rho$ are respectively the linear attenuation coefficient and the mass absorption coefficient of the medium at the photon energy $E$, tabulated in \cite{nist_ph}.


%and $D_{\rm t}$ are respectively the real dose including scattered radiation and its theoretical value if that contribution is ignored. The $B$ value depends on the energy spectrum and geometrical features of the photon beam as well as the geometry and composition of the medium. For a given setup, $B$ is a function of the penetration depth of the radiation.

Buildup factors are very relevant in engineering materials for shielding calculations. A database (ANSI/ANS-6.1.1, 1991) is available in \cite{ansi-ans91} and their values are confirmed by MC simulations with EGS4 \cite{hirayama}, \cite{UEI}. These parameters are also of the most relevance in medical physics \cite{Manohara}, \cite{kadri}. 

%For the calculation of the buildup factor from Eq. (\ref{eq:B}), the dose as a function of the distance $R$ from the isotropic point source, $D_{\rm r}(R)$, was obtained by simulation while $D_{\rm t}(R)$ was calculated from the expression 
Using LegPy, we calculated $B(R)$ for water, iron, and lead at energies of 0.1, 0.3, 1.0, and 5.0 MeV. We repeated the simulations using PENELOPE and recalculated the buildup factors. In addition, we used the updated ANSI/ANS-6.1.1 buildup factor database \cite{ansi_up_low}, \cite{ansi_up_high} obtained by MCNP5 simulations \cite{mcnp5}.
The results are compared in figures \ref{fig:B_w} - \ref{fig:B_Pb}. Note that comparing the buildup factor is equivalent to comparing the dose, assuming that the same $\mu$ and $\mu_{\rm en}/\rho$ coefficients are used. 

% Figure environment removed
%
% Figure environment removed
%
% Figure environment removed

As can be seen in Fig.\ref{fig:B_w} (upper plot), the buildup factors for water obtained with LegPy are in good agreement with those obtained with both PENELOPE and MCNP5. The deviations of LegPy results with respect to those from PENELOPE (lower plot) are at the level of less than 5\% for $R < 4$~mfp, growing up to about 15\% at 10~mfp. Notably, the agreement is very good even at 5.0~MeV, confirming that the impact of ignoring pair creation and Bremsstrahlung on the calculation of the dose in depth is very weak in light elements even at energies of a few MeV.

Results for iron at 0.3, 1.0 and 5.0 MeV are shown in Fig. \ref{fig:B_Fe}. Again, the simplifications made in LegPy do not seem to have a relevant impact in the calculation of the dose at these energies. At 0.1 MeV (not shown in the figure), LegPy also reproduces the result from PENELOPE up to about 4~mfp but underestimates it by 30\% at 10~mfp. This deviation is attributed to the simplified treatment of the coherent scattering in LegPy.

The results for lead at 0.3, 1.0, and 5.0 MeV are shown in Fig. \ref{fig:B_Pb}. LegPy is in agreement with PENELOPE up to about 4~mfp with larger deviations (up to 30\%) at large depths. Again, the effect of ignoring pair creation and Bremsstrahlung is not very relevant in the depth dose calculation even at 5 MeV. However, we found very large deviations with respect to both PENELOPE and MCNP5 at 0.1 MeV, which are attributed to ignoring X-ray fluorescence. It is well known \cite{subbaiah} that, for heavy atoms, X-ray fluorescence leads to a dramatic increase in buildup factors for photon energies slightly above the K edge of the photoelectric cross section. Indeed, the value of $B(R)$ obtained by PENELOPE is larger than that obtained with LegPy by a factor of 6 (50) at a depth of 5~mfp (10~mfp).

% Figure environment removed
%
% Figure environment removed
%
% Figure environment removed

We carried out another set of tests for the angular distribution of photons escaping the medium. We simulated pencil beams of various energies traversing a cylinder of water with both height and diameter equals to one mfp at the corresponding energy. The LegPy results for 0.2, 1.0, and 5.0 MeV are compared with those from PENELOPE in Fig. \ref{fig:ang_w}. We found that the agreement is very good up to about 2.0 MeV while LegPy results deviate significantly from those from PENELOPE at higher energies. From these simulations, we also calculated the distribution of energy absorbed by the medium. The results are shown in Fig. \ref{fig:eab_w}. The agreement is very good even at 5 MeV, although LegPy does not reproduce some spectral features associated with processes ignored in our algorithm, e.g., escape of annihilating photons subsequent to pair creation.

An interesting test was done for isotropic beams of 0.667 and 1.275 MeV (emulating radioactive point sources of $^{137}$Cs and $^{22}$Na) at a distance of 2 cm from a NaI cylinder with both diameter and height of one mean free path (4.34 cm for $^{137}$Cs and 5.32 cm for $^{22}$Na). As can be seen in Fig. \ref{fig:eab_nai_667},  the spectrum of absorbed energy (i.e., the shape of the Compton profile and the height of the photopeak) obtained with LegPy is in very good agreement with the  one obtained with PENELOPE. However, PENELOPE predicts a small peak shifted by about 30 keV from the photopeak due to the escape of X-rays produced after the photoelectric absorption in the K shell of Iodine. This feature, which is generally irrelevant for the characterization of a scintillator,  cannot be reproduced by LegPy because X-ray fluorescence is ignored in our algorithm.

\subsection{Electron beams}
\label{sec:electrons}
For the validation of LegPy for the transportation of electrons, the dose in depth, the backscattering coefficient and the angular distribution of backscattered electrons were obtained for a pencil electron beam going along the axis of a semi-infinite cylinder (i.e.,  larger than the corresponding CSDA range). Different media were tested in the energy range $0.1 - 5.0$~MeV. As in the previous section, we repeated the simulations with PENELOPE to compare with LegPy. The simulation results were also compared with experimental data on energy deposition of electron beams on different media and backscattering coefficients, which are available in the literature for a wide energy range \cite{sandia_el_1}, \cite{sandia_el_2} and have been used since long ago to benchmark PENELOPE \cite{Sempau} and  GEANT4 \cite{Kadri}, \cite{P_Arce}. 

In figures \ref{fig:w_0.1_0.5} and \ref{fig:w_1_5}, depth dose in water for electrons at several energies in the range $0.1 - 5.0$ MeV obtained with LegPy are compared with the results from PENELOPE. As explained above, the fluctuations in energy deposition are ignored in our simplified algorithm in such a way that the path length of all electrons equals the CSDA range. As a consequence, the dose at depth beyond the CSDA range obtained with LegPy is zero while the energy is spread to larger distances for the PENELOPE results. Apart from this discrepancy, LegPy reproduces reasonably the results from PENELOPE with deviations in the maximum of the dose in depth function smaller or around 10\%. These discrepancies are a consequence of the approximations made in the multiple scattering and collisional energy losses in our algorithm.

% Figure environment removed
%
% Figure environment removed
%
% Figure environment removed
%
% Figure environment removed
%
% Figure environment removed

The results for aluminum are shown in Fig. \ref{fig:Al_0.3-0.5-1.0}.  LegPy reproduces the PENELOPE results with an accuracy similar to that obtained for water. Available experimental data at 0.31, 0.52, and 1.03 MeV \cite{sandia_el_1} are in very good agreement with PENELOPE confirming the reasonable accuracy of LegPy. Similar results were found up in the whole range between 0.1 and 5.0 MeV.
%% A 0.11 tambien comparé pero no puse la figura. 

The results for copper are shown in Fig. \ref{fig:Cu_0.1_0.5}. LegPy reproduces well the dose in depth at energies below 1 MeV. The discrepancy turns out significant at 5.0 MeV (16 \% at the depth dose maximum), very likely due to ignoring the Bremsstrahlung production. For very heavy elements, such as uranium, this discrepancy exists even at lower energies, as expected (Fig. \ref{fig:U_0.2_1.0}).

Electron backscattering in LegPy was also compared with the results from PENELOPE. In general, we found a good agreement in the shape of the angular distribution of backscattered electrons obtained with both simulation codes. As an example, we show the results for copper at 0.1 and 2.0 MeV in Fig. \ref{fig:Cu_bck}. On the other hand, LegPy tends to underestimate the backscattering coefficient $\eta$ (i.e., the ratio between incoming and backscattered electrons) for light elements. For instance, in water at 1.0 MeV, the PENELOPE result is $\eta = 3.0\%$ while it is half that value for LegPy. The agreement in the backscattering coefficient improves for heavier elements. For aluminum at 1.03 MeV, LegPy gives $\eta =8.0\%$ to be compared with the value of $\eta = 9.5\%$ obtained with PENELOPE, which is in good agreement with the experimental value of 9.2\% \cite{sandia_el_2}. For copper at 2.0 MeV (Fig. \ref{fig:Cu_bck}), LegPy gives  $\eta = 28\% $ while PENELOPE gives $\eta = 29.5\% $. For uranium at 1 MeV, the PENELOPE result is $\eta = 51\%$ versus $\eta = 44\%$ from LegPy.

% Figure environment removed

\subsection{Beams in a two media object}
\label{sec:2M_ph}
Several tests were made to check the performance of LegPy for transporting photons and electrons through an object composed by two different media. As an example, in Fig. \ref{fig:w-Pb_1.0}, we show the dose in depth for a photon beam of 1.0 MeV crossing along its axis a cylinder of 2.0 cm length with a 10.0 cm diameter (i.e., laterally infinite) to assure that all scattered radiation except the backscattered one is absorbed. The transportation of secondary electrons is included in the simulation. The cylinder is made of water and lead, the boundary between the two media being at a depth of 1.0 cm. As can be seen in the figure, the dose in depth obtained with LegPy is in good agreement with the result from PENELOPE. Note the relevant role of secondary electrons in the dose in depth. Electronic equilibrium is reached at about 2.5 mm and the backscattering in lead is very strong. We checked that similar agreements with PENELOPE are achieved at 0.3 and 5.0 MeV.

Tests for an electron beam traversing a two-media cylinder were also carried out. As an example, in Fig. \ref{fig:Al-Pb_1MeV} it is shown the dose in depth for a pencil beam of 1.0 MeV electrons crossing a cylinder of 0.164 cm length and 0.20 cm diameter, made of aluminum and lead. Several cases were studied varying the depth $z_{\rm b}$ of the boundary between the two media. The results of the figure corresponds to $z_{\rm b}$ values of 0.0411, 0.082, and 0.123 cm as well as to the only-Al case. The LegPy results are in very reasonable agreement with those from PENELOPE. Backscattering leads to significant features that LegPy adequately reproduces. Deviations with respect to PENELOPE are basically due to the approximations in the electron transportation, as already discussed in \ref{sec:electrons}.

% Figure environment removed

% Figure environment removed

\section{Conclusions}
\label{sec:conclusions}
In this paper, we have presented a simplified algorithm for the simulation of the passage of low-energy electrons and gamma rays through any medium. The algorithm has been realized in the Python package called LegPy available under an open source license. 

The algorithm has been validated by comparing a set of results with the code PENELOPE as well as with available experimental data. From the comparisons for photon beams, we can state that LegPy is able to calculate with reasonable accuracy (i.e., around 10\%) the dose deposited by photons of up to 5 MeV in light media at depths of up to several mean free paths. This is particularly interesting for applications in medical physics. The algorithm can also reproduce accurately the spectrum of absorbed energy in light elements as well in typical scintillator materials used in gamma-ray spectrometry. The results of the angular distribution of escaping photons obtained with LegPy are also in general good agreement with those from PENELOPE, but results for water at 5 MeV have significant deviations, indicating that the scattering of photons is more sensitive to the approximations employed in our algorithm. We found that the main limitation of our algorithm for the simulation of low-energy gamma rays is that it is not valid at energies close but over that of the K shell of heavy elements. We plan to implement soon a simplified model for the X-ray fluorescence in our algorithm to solve this problem. 

Regarding the transportation of electrons, we have presented a simple model that provides reasonable results of both the dose in depth and backscattering. Our results of the dose in depth in light elements deviate from those from PENELOPE by less than or around 10\% at the maximum of the depth in dose function. For heavy elements, deviations remain small up to a few MeV, but LegPy underestimates significantly the energy deposition at higher energies as a consequence of ignoring the Bremsstrahlung effect. In general, backscattering is properly simulated with LegPy. The most significant deviations with respect to PENELOPE were found for light elements, although they are expected to have a minor effect in most practical cases because backscattering is only relevant for heavy elements.

LegPy has also been tested for the simulation of photons and electrons around the boundary between two media. The expected effects are properly observed and results are in good agreement with those obtained with PENELOPE for both photon and electron beams.

The several simplifications of our algorithm greatly improves the simulation speed. For example, the computing times needed to get the results of dose deposited by 5~MeV photons in lead shown in Fig. \ref{fig:B_Pb} were 130 times smaller for LegPy than for PENELOPE, in spite that LegPy simulations were run using a Python interpreter. At lower photon energies and for lighter elements, the number of interactions is reduced and the speed of both simulation codes becomes more similar. The increase in speed is also significant for electrons. For example, LegPy is faster than PENELOPE by a factor of 36 for 1.03 MeV electrons in aluminum (Fig. \ref{fig:Al_0.3-0.5-1.0}) if using a step length of 10\% of the CSDA range in LegPy.
For a step length of 1\% of the CSDA, the speed of LegPy decreases approximately by a factor of 10, but it is still faster than PENELOPE.

In summary, an easy-to-use tool for fast simulations of low energy gamma-rays and electrons is available. LegPy aims to be useful for researchers that need simulations on simple geometries with reasonable accuracy but without the technical complexity of other MC codes. As the tool is very easy to use, it may also be useful for teaching purposes in undergraduate or postgraduate degree programs as well as for the training of experts in the field of medical applications of ionizing radiation.

\section*{Acknoledgements}
We thank Francesc Salvat for providing help and guidance for the use of PENELOPE program. We gratefully acknowledge financial support from the Spanish Research State Agency (AEI) through the grant PID2019-104114RB-C32. V. Moya also acknowledges the research grant CT19/23-INVM-109 funded by NextGenerationEU.



%see appendix~\ref{sec:sample:appendix}.

%% The Appendices part is started with the command \appendix;
%% appendix sections are then done as normal sections
%\appendix

%\section{Sample Appendix Section}
%\label{sec:sample:appendix}
%Lorem ipsum dolor sit amet, consectetur adipiscing elit, sed do eiusmod tempor section \ref{sec:sample1} incididunt ut labore et dolore magna aliqua. Ut enim ad minim veniam, quis nostrud exercitation ullamco laboris nisi ut aliquip ex ea commodo consequat. Duis aute irure dolor in reprehenderit in voluptate velit esse cillum dolore eu fugiat nulla pariatur. Excepteur sint occaecat cupidatat non proident, sunt in culpa qui officia deserunt mollit anim id est laborum.

%% If you have bibdatabase file and want bibtex to generate the
%% bibitems, please use
%%
% \bibliographystyle{elsarticle-num} 
% \bibliography{cas-refs}

%% else use the following coding to input the bibitems directly in the
%% TeX file.

\begin{thebibliography}{00}

\bibitem{egs4}
W.R. Nelson, H. Hirayama, D.W.O. Rogers, The EGS4 Code System, SLAC-265 (1985).

\bibitem{penelope}
J. Baró, J. Sempau, J.M. Fernández-Varea, F. Salvat., Nucl. Instrum. Meth. Phys. Res. Sect B 100, 31 (1995); doi: 10.1016/0168-583X(95)00349-5

\bibitem{geant4} 
S. Agostinelli, J. Allison, {\sl et al.},  Nucl. Instrum. Meth. Phys. Res. Sect A. 506, 250 (2003); doi: 10.1016/S0168-9002(03)01368-8

\bibitem{mcnp}
%T. Goorleya, M. Jamesb, {\sl et al.}, Nuclear Technology 180, 298 (2012); doi: 10.13182/NT11-135
https://mcnp.lanl.gov/

\bibitem{legpy}
F. Arqueros, J. Rosado, V. Moya, LegPy (Version v1.2) [Computer software] (2022);   doi: 10.5281/zenodo.7882780

\bibitem{nist_ph}
M.J. Berger, J.H. Hubbell, S.M. Seltzer, {\sl et al.}, XCOM: Photon Cross Section Database (version 1.5) (2010). [Online] http://physics.nist.gov/xcom. National Institute of Standards and Technology, Gaithersburg, MD.

\bibitem{evans}
R. D. Evans, The Atomic Nucleus, McGraw-Hill (1955).
%reprinted (1982).R. E. Krieger, Malabar, FL.

\bibitem{nist_estar}
M. Berger, J. Coursey, M. Zucker, ESTAR, PSTAR, and ASTAR: Computer Programs for Calculating Stopping-Power and Range Tables for Electrons, Protons, and Helium Ions (version 1.21) (1999). [Online] http://physics.nist.gov/Star

\bibitem{theta_0}
G.R. Lynch and O.I. Dahl, Nucl. Instrum. Meth. Phys. Res. Sect B 58, 6 (1991); doi:10.1016/0168-583X(91)95671-Y

\bibitem{gcolab}
colab.research.google.com/

%\bibitem{amjphy}
%F. Arqueros, G. D. Montesinos, Am. J. Phys. 71, 38 (2003); doi: 10.1119/1.1509416

%\bibitem{coh_ef}
%D.E. Cullen, J.H. Hubbell, L. Kissel, EPDL97 The Evaluated Photon Data Library, 097 Version, Technical Report UCRL-50400, Lawrence Livermore National Laboratory, Livermore, California  (1997).

\bibitem{ansi-ans91}
American National Standard Institute. Gamma-ray attenuation coefficients and buildup factors for engineering materials. Report ANSI/ANS-6.4.3-1991. La Grange Park, Illinois: American Nuclear Society (1991).

\bibitem{hirayama}
H. Hirayama, J. Nucl. Sci. Technol. 32, 1201 (1995); doi: 10.1080/18811248.1995. 9731842

%H. Hirayama, Journal of Nuclear Science and Technology, Calculation of Gamma-ray Exposure Buildup Factors up to 40mfp using the EGS4 Monte Carlo Code with a Particle Splitting, Journal of Nuclear Science and Technology, 32:12, 1201-1207 (1995), DOI: 10.1080/18811248.1995.9731842

\bibitem{UEI} 
Uei-Tyng Lin, Shiang-Huei Jiang, Radiat. Phys. Chem. 48, 389 (1996); doi: 10.1016/0969-806X(95)00461-6

\bibitem{Manohara}
S.R. Manohara, S.M. Hanagodimath, L. Gerward,  J. Appl. Clin. Med. Phys. 12, 296 (2011); doi: 10.1120/jacmp.v12i4.3557. 
%Journal of Applied clinical Medical Physics (JACMP)

\bibitem{kadri}
O. Kadri, A. Alfuraih  Nucl. Sci. Tech. 30, 176 (2019); doi:10.1007/s41365-019-0701-4
%Photon energy absorption and exposure buildup factors for deep penetration in human tissues.
\bibitem{ansi_up_low}
L. Durani, "Update to ANSI/ANS-6.4.3-1991 for low-Z and compound materials and review of particle transport theory" (2009). UNLV Theses, Dissertations, Professional Papers, and Capstones. 43; 
doi: 10.34917/1363554

\bibitem{ansi_up_high}
L.P. Ruggieri, "Update to ANSI/ANS-643-1991 for high-Z materials and review of particle transport theory" (2008). UNLV Retrospective Theses Dissertations; doi:10.25669/xc1i-qbff

\bibitem{mcnp5}
X-5 Monte Carlo Team, MCNP — A General Monte Carlo N-Particle Transport Code, Version 5, Volume I: Overview and Theory, LA-UR-03-1987, Los Alamos National Laboratory (2003).

\bibitem{subbaiah}
K.V. Subbaiah, A. Natarajan, Nuclear Science and Engineering 96, 330 (1987); doi: 10.13182/NSE87-A16396
%Effect of Fluorescence in Deep Penetration of Gamma Rays. 

%\bibitem{subbaiah_2}
%Effect of Fluorescence, Bremsstrahlung, and Annihilation Radiation on the Spectra and Energy Deposition of Gamma Rays in Bulk Media
%K. V. Subbaiah, A. Natarajan, D. V. Gopinath & D. K. Trubey, Nuclear Science and Engineering 81, 172 (1982); doi: 10.13182/NSE82-A20084 

\bibitem{sandia_el_1}
G.J. Lockwood, G.H. Miller, L.E. Ruggles, J.A. Halbleib, Calorimetric measurement of electron energy deposition in extended media, SANDIA Tech. Rep. SAND79-0414, UC-34a, Albuquerque, NM (1987).

\bibitem{sandia_el_2}
G.J. Lockwood, G.H. Miller, L.E. Ruggles, J.A. Halbleib, Electron Energy and Charge Albedos — Calorimetric Measurement vs Monte Carlo Theory, Tech. Rep. SAND80-1968, UC-34a, Albuquerque, NM (1987)
%SANDIA REPORT SAND79-0414.UC-34a (1987)
%SAND80– 1968 UC–34a (1987)

%G.J. Lockwood, L.E. Ruggles, G.H. Miller, J.A. Halbleib, Calorimetric Measurement of Electron Energy Deposition in Extended Media—Theory vs Experiment, Sandia Laboratories Tech. Rep. SAND79-0414, Albuquerque, NM, 1980.

%\bibitem{Tabata_bck}
%T. Tabata,  Phys. Rev. 162, 336 (1967); doi:10.1103/PhysRev.162.336
%Backscattering of Electrons from 3.2 to 14 MeV,

\bibitem{Sempau}
J. Sempau, J.M. Fernandez-Varea, E. Acosta, F. Salvat, Nucl. Instrum. Meth. Phys. Res. Sect B 
207, 107 (2003); doi:10.1016/S0168-583X(03)00453-1 
%Experimental benchmarks of the Monte Carlo code PENELOPE

\bibitem{Kadri}
O. Kadri, V.N. Ivanchenko, F. Gharbi, A. Trabelsi, Nucl. Instrum. Meth. Phys. Res. Sect B 258, 381 (2007); doi:10.1016/j.nimb.2007.02.088
%GEANT4 simulation of electron energy deposition in extended media

\bibitem{P_Arce}
P. Arce,{\sl et al.}, Med. Phys. 48, 19 (2021); doi:10.1002/mp.14226
% Report on G4‐Med, a Geant4 benchmarking system for medical physics applications developed by the Geant4 Medical Simulation Benchmarking Group.

\end{thebibliography}
\end{document}
\endinput
%%
%% End of file `elsarticle-template-num.tex'.
