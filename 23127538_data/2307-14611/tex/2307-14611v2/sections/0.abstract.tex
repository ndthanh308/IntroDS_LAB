% In this paper, w
% The 
Recent label mix-based augmentation methods have shown their effectiveness in generalization despite their simplicity, and their favorable effects are often attributed to semantic-level augmentation.
% also perturb the semantic meaning of the data and are effective in generalization, 
However, we found that they are vulnerable to highly skewed class distribution,
% the class distribution having high skewness, because perturbation is done in an inter-class manner.
because scarce data classes are rarely sampled for inter-class perturbation.
We propose $\TextMani$, a text-driven manifold augmentation method that semantically enriches visual feature spaces, regardless of data distribution.
%We propose a text-driven 
% manipulation and 
%augmentation method, $\TextMani$, to enrich the visual feature space given scarce data in a semantically meaningful way, regardless of the class distribution.
% handle the sparse sample problems often occurring in the real-world.
$\TextMani$ augments visual data with intra-class semantic perturbation
% in a semantically meaningful way
by exploiting easy-to-understand 
visually mimetic words, i.e., attributes.
% text.
% , making $\TextMani$ human interpretable, physically relevant, and controllable.
% $\TextMani$, on the other hand, augments around all the given individual visual samples by reflecting the changes stemming from adding visually mimetic words, i.e., attribute.
% , on the visual feature space by attribute embedded vectors.
To this end, we bridge between the text representation and a target visual feature space,
% through a pre-trained joint visual-language embedding space, 
and propose an efficient vector augmentation.
% is proposed.
To empirically support the validity of our design, 
we devise two visualization-based analyses and show the plausibility of the bridge between two different modality spaces.
% that the attribute vector has the meaning of attributes, 
% accomplish
% and show empirical evidence.
Our experiments demonstrate that $\TextMani$ is powerful in scarce samples with class imbalance as well as even distribution.
We also show compatibility with the label mix-based approaches in evenly distributed scarce data.
% Our experiments demonstrate that $\TextMani$ helps to improve the performance in the limited data conditions and the compatibility with other augmentation methods.
% \textcolor{postechred}{Our code will be publicly available if accepted.}