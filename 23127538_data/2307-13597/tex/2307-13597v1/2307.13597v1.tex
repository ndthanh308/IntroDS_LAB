\documentclass[10pt,notitlepage,a4paper,reqno,ps]{amsart}

\usepackage[utf8]{inputenc}
\usepackage[T1]{fontenc}
\usepackage[english]{babel}
\usepackage{amsmath}
\usepackage{amsfonts}
\usepackage{amssymb}
\usepackage{amsthm}
\usepackage{bbm}
\usepackage{mathrsfs}
\usepackage{graphicx}
\usepackage{braket}
\usepackage{color}
\usepackage{amscd,epsfig,esint,graphics}
\usepackage{wrapfig}
\usepackage{verbatim}
\usepackage{mathtools}
\usepackage{color}
%\usepackage[notcite,notref]{showkeys}
\renewcommand{\baselinestretch}{1.15}

\textwidth=165mm \textheight=220mm \hoffset=-20mm \voffset=-10mm

\theoremstyle{plain}
\newtheorem{theorem}{Theorem}[section]
\newtheorem{lemma}[theorem]{Lemma}
\newtheorem{proposition}[theorem]{Proposition}
\newtheorem{example}[theorem]{Example}
\newtheorem{corollary}[theorem]{Corollary}
\newtheorem{remark}[theorem]{Remark}
\newtheorem{definition}[theorem]{Definition}
\newtheorem{conj}[theorem]{Conjecture}
\theoremstyle{definition}
\theoremstyle{remark}
\numberwithin{equation}{section}
\newenvironment{oss}{\begin{remark} \begin{rm}}{\end{rm} \end{remark}}

\newcommand{\R}{\mathbb{R}}
\newcommand{\Q}{\mathbb{Q}}
\newcommand{\N}{\mathbb{N}}
\newcommand{\K}{\mathbb{K}_\psi(\Omega)}
\newcommand{\m}{\mu}
\newcommand{\e}{\varepsilon}
\newcommand{\C}{\mathcal{C}}
\newcommand{\capp}{\text{Cap}}
\newcommand{\dive}{\text{div}}
\newcommand{\om}{\Omega}
\newcommand{\leb}{\mathcal{L}}
\newcommand{\supp}{\text{supp}\,}
\DeclareMathOperator*{\esssup}{ess\,sup}

\renewcommand{\comment}[1]{\vskip.3cm
	\fbox{%
		\parbox{0.93\linewidth}{\footnotesize #1}}
	\vskip.3cm}

\def\tcr{\textcolor{red}}
\def\tcb{\textcolor{blue}}
\definecolor{vg}{rgb}{0.0, 0.26, 0.15}

\title{Relaxation of one-dimensional nonlocal supremal functionals in the Sobolev setting} %and dimension reduction}
\author[A.\,Torricelli]
{Andrea Torricelli}
\address[A.\,Torricelli]{Dipartimento di Scienze Fisiche, Informatiche e Matematiche, Universit\`a degli Studi di Modena e Reggio Emilia, via Campi 213/b, 41125, Modena, Italy.}
\email{andrea.torricelli@unipr.it}

\author[E.\,Zappale]
{Elvira Zappale}
\address[E.\,Zappale]{Dipartimento di Scienze di Base ed Applicate per l'Ingegneria, Sapienza-Universit\`a di Roma, via A. Scarpa, 16, 00161 Roma, Italy}
\email{elvira.zappale@uniroma1.it}


%la funzione \chi_K potrebbe avere un significato ambiguo, controlla di averla sempre usata con una sola definizione.

%devi definire la funzione dist(\cdot, K) con K compatto. Vedi se esite anche un modo di scrivere "dist" meglio.





\begin{document}
	
	\baselineskip 3.4ex
	\vspace{0.5cm}
	\maketitle
	
	\begin{abstract}
		
		We provide necessary and sufficient conditions on the density $W:\R^d\times\R^d\to\R$ in order to ensure the sequential weak* lower semicontinuity of the functional $J: W^{1,\infty}(I;\mathbb R^d)\to \mathbb R$, defined as \begin{align*}
			J(u):=\esssup_{I\times I}W(u'(x), u'(y)),
		\end{align*}
		when $I$ is an open and bounded interval of $\mathbb R$.
		We also show that, when $d=1$, the lower semicontinuous envelope of $I$ in general can be
		obtained by replacing $W$ by its separately level convex envelope.  
		
		\textsc{MSC (2020):} 49J45 (primary), 47J22, 26B25.
		
		
		\noindent\textsc{Keywords: convexity, nonlocality, supremal functionals, $L^\infty$- variational problems, nonlocal differential inclusions.} 
	\end{abstract}

%%%%%%%%%%%%%%%%%%%%%%%%%%%%%%%%%%%%%%%%%%%%%%%


\section{Introduction}

 A growing interest is developping towards nonlocal functionals in recent years, due to their many applications, both from the application view point (peridynamics, image processing, artificial intelligence, etc), and the theoretical one, e.g. \cite{BMCP, MD, DFKS, EDLL14, KS, CKS} among a pletora of scientific contributions.
 In particular a big attention is devoted to nonlocal integrals, we refer to \cite{Pm} and the bibliography contained therein for an overview and to \cite{BDM2} for the interactions with local energies, among a wide literature.
 
 On the other hand, $L^\infty$ variational problems arise when studying  optimal design problems, in the control theory, optimal transport, etc, and in these past decades a wide theory has been built, we refer to \cite{BL, ABPrin} for the pioneering theoretical papers and to \cite{EP, PZ, CK}  and the bibliography contained therein just to mention some more theoretical recent contribution among a much wider literature.    
% \color{red}
 %In \cite{KZ} Kreisbeck and Zappale investigated the necessary and sufficient condition that makes the nonlocal supremal functionals \eqref{functional} lower semicontinuous when tested along sequences of $L^\infty$ functions converging weakly*. In their paper, they prove that such condition is the separate level convexity of the supremand function. Moreover, they also prove that the relaxed functional is structure preserving, i.e. that is still a supremal functional, and that its supremand is the separately convex envelope of $W.$ This seems to be only natural since in \cite{BL} the authors proved that level convexity is a necessary and sufficient condition for the lower semicontinuity in the local case, i.e. for the functional
 %\begin{equation}
%L^\infty(\Omega)\owns u\mapsto\esssup_{x \in \Omega}f(x),
 %\end{equation}
 %while in \cite{ABPrin},\cite{CP} and \cite{Prin} the authors proved that its relaxation is, again, structure preserving.
 %Furthermore, in \cite{KRZ} Kreisbeck, Ritorno and Zappale studied the case when the supremand of the functional is defined on $\R^d\times\R^d$, with $d\in\N$. In this case the necessary and sufficient condition on the supremand is called Cartesian convexity. The notion of Cartesian convexity coincide with the separate level convexity when $d=1$, but it is strictly weaker for $d>1$. As extensively explained in both \cite{KRZ} and \cite{KZ}, nonlocal supremal functionals are strongly connected to nonlocal differential inclusions and nonlocal integral functionals and in particular to nonlocal indicator functionals.
 %On one hand, in \cite{BM-C}, the authors prove a charcterization of the weak lower semicontinuity in $L^p$ of the nonlocal integral functional
 %\begin{equation}
 %	\int\int_{\Omega\times \Omega}f(x,y,u(x),u(y))dxdy, \quad u\in W^{1,p}(\Omega,\R^d),\, \Omega\subset\R^n,\, n\in\N.
 %\end{equation}
 Our contribution inserts in the framework of detecting necessary and sufficient conditions for the lower semicontinuity and relaxation of nonlocal $L^\infty$ functionals depending on the derivatives of Sobolev fields defined on the real line. 
 
 
 
 
In this paper we  focus on $L^\infty$ variational models, for which one might be interested in detecting equilibrium configurations, which consist of determining the lower semicontinuity of suitable functional, or, when this is not the case in their relaxation. When $\Omega$ is a bounded open subset of $\mathbb R^n$, $n > 1$, it is  well known that in general there is no explicit representation for the relaxation of nonlocal supremal functionals of the type
\begin{equation}\label{tildeJ}\tilde J(u):=\esssup_{\om\times \om}W(\nabla u(x), \nabla u(y))
	\end{equation}
with respect to the $W^{1,\infty}-weak^*$ topology. As anticipated, the question has been completely answered in \cite{KRZ} in the case where $\nabla u$ is replaced by $u$, while, in the case of gradients, a sufficient condition on $W$ ensuring lower semicontinuity has been provided in \cite{GZ}.

The current analysis, focusing on fields defined on an interval strongly rely on the results contained in \cite{KRZ}. 
% very much related with the analogous one treated in \cite{KRZ}. DISCUSS MORE ANG MAKE COMMENTS AND REFERENCES.
The study of the lower semicontinuity of $\tilde J$ with respect to the  $W^{1,\infty}-weak^*$ topology is  equivalent to characterize, for every $c \in \mathbb R$, the $W^{1,\infty}-weak^*$ closure of $B_{L_c(W)}$, with  $L_c(W)$ being $c$-(sub)level set of $W$, i.e.
\begin{align*}
	L_c(W):=\left\{(\xi, \eta) \in \R^{d\times n}\times\R^{d\times n}: W(\xi,\eta)\le c\right\},
\end{align*}
and, for every $K \subset \R^{d \times n} \times \R^{d\times n}$
\begin{align}\label{AK}
	B_K:=\left\{u \in W^{1,\infty}(\om;\mathbb R^d): (\nabla u (x), \nabla u(y)) \in K \text{ for a.e. }x\in \om\times\om \right\}.
\end{align}
In this note, stemming from the results in \cite{KZ, KRZ}, we provide an answer when $n=1$ in terms of the notion of Cartesian convexity introduced in \cite{KRZ} for the set $K$ and an equivalent characterization in term of separate convexity when also $d=1$. 
%\color{red}
%Indeed a function is weakly lower semicontinuous if and only if all its level sets are weakly closed. It is easy to see that $B_{L_c(W)}$ is the $c$-level set of $J$, so our result can be seen as a necessary and sufficient condition on $W$ that ensure that $B_{L_c(W)}$ is $W^{1,\infty}-weak^*$ closed. \\

We will also provide the supremal counterpart of the results for nonlocal integral functionals depending on derivatives of Sobolev functions contained in \cite{BP}. In that paper the authors find that the necessary and sufficient condition for the $L^p$ weak lower semicontinuity of functionals such as
\begin{equation}
	\label{nonlocalint}
	\iint_{I\times I}f(u'(x),u'(y))dxdy, \quad u\in W^{1,p}(I),\, I \text{ interval of }\R,
\end{equation}
is the separate convexity of $f$.


We will characterize the $W^{1,\infty}$-lower semicontinuity of the functional $\tilde J$ in \eqref{tildeJ} in terms of properties on the supremand $W$ when the open set $\Omega$ is an open interval on the real line.  For the sake of exposition we will mainly focus on the case when $d=1$, i.e. the field $u$ is scalar valued. On the other hand the results concerning fields in $W^{1,\infty}(I;\mathbb R^d)$ can be deduced exploiting analogous arguments, see Theorem \ref{cnsvec}. Actually, Theorem \ref{cns} could be deduced as a corollary of Theorem \ref{cnsvec}, but we opted for presenting the proof in the scalar valued setting for the sake of exposition.

We emphasize also that, while in \cite{BP} it has been proven that the relaxation of a nonlocal integral functional defined in open subsets of $\mathbb R$, does not admit, in general, a representation of the same type (see also \cite{BM-C, KZdintegral, BDM3}), we will show that, under suitable assumptions on $W$, in the case $n=1$, the relaxed functional,  can be represented in supremal form with a suitable density. Furthermore if also $d=1$, without any restriction on $W$, the relaxed energy is of the same type with a density coinciding with the separately level convex envelope of the original one. In the vector valued case the relaxation will be deduced in view of \cite[Theorem 1.2]{KRZ}, see Theorem \ref{relaxvec}. 


To ease the parallel with the nonlocal integral setting, it is worth to mention that in \cite{M}, Mu$\tilde{n}$oz proved that separate convexity of the integrand is a necessary and sufficient condition for the weak lower semicontinuity of the functional \eqref{nonlocalint} when it depends on $\nabla u$, with $u\in W^{1,p}(\Omega)$ and $\Omega\subset \R^n$. The same question in the supremal setting is currently still open and it is not clear if the Cartesian level convexity of the supremand $W$ (see Section 3 for the definition introduced in \cite{KRZ}) is also necessary for the lower semicontinuity of $\tilde J$. 
We also stress that, in contrast with the integral framework, even in the local scalar setting, i.e. $n=1$ or $d=1$, the convexity notions arising in the supremal framework are not equivalent, see \cite{RZ2}, indeed this latter aspect prevents us from using the same strategies adopted in \cite{M}.

%On the other hand, in \cite{KZdintegral} the authors proved that the relaxation of such nonlocal integral functional is not structure preserving, i.e. its relaxation is not a nonlocal integral functional.







From now on, we will denote with $I$ the interval $(a,b)$, for some $a,b\in\R$, $a<b$. Referring to Section \ref{pre} for definitions and properties of the nonlocal densities below, 
the first result that we prove is the following 
\begin{theorem}
	\label{cns}
	Let  $J_{1d}:W^{1,\infty}(I)\to\R$ be defined as
	\begin{align}
		\label{functional1d}
		J_{1d}(u):=\esssup_{I\times I}V(u'(x), u'(y)),
	\end{align} with $V:\mathbb R \times \mathbb R \to \mathbb R$ coercive, lower semicontinuous, diagonal and symmetric, then $J_{1d}$ is lower semicontinuous with respect to the $W^{1,\infty}-weak^*$ topology if and only if $V$ is separately level convex.
\end{theorem}
Moreover the following result holds: 
\begin{theorem}
	\label{relax}
	Let $V$ and $J_{1d}$ be as in Theorem \ref{cns}. %defined as in \eqref{functional1d} with $V$ coercive, lower semicontinuous, diagonal and symmetric.
	Let $J_{1d}^{rlx}$ be the sequentially $W^{1,\infty}-$weak$^*$ lower semicontinuous envelope of $J_{1d}$, i.e.
	\begin{equation}\label{Jrlx1d}
	J_{1d}^{rlx}:=\inf\{\liminf_{j\to +\infty} J_{1d}(u_j): u_j \overset{*}{\rightharpoonup} u \hbox{ in }W^{1,\infty}(I)\}.
	\end{equation}
	Then
	$$J^{rlx}_{1d}(u)=\esssup_{I\times I} V^{slc}(u'(x), u'(y)),$$ for every $u \in W^{1,\infty}(I)$, where $V^{slc}$ is the separately level convex envelope of $V$.
\end{theorem}
For what concerns the vector valued case, we have
\begin{theorem}
	\label{cnsvec}
	Let  $J:W^{1,\infty}(I;\R^d)\to\R$ be defined as
	\begin{align}
		\label{functionald}
		J(u):=\esssup_{I\times I}W(u'(x), u'(y)),
	\end{align} with $W:\mathbb R^d \times \mathbb R^d \to \mathbb R$ coercive, lower semicontinuous, diagonal and symmetric, then $J$ is lower semicontinuous with respect to the $W^{1,\infty}-weak^*$ topology if and only if $W$ is Cartesian level convex.
\end{theorem}

Furthermore 
\begin{theorem}
	\label{relaxvec}
	Let $W$ and $J$ be as in Theorem \ref{cnsvec}. %defined as in \eqref{functional1d} with $V$ coercive, lower semicontinuous, diagonal and symmetric.
	Let $J^{rlx}$ be the sequentially $W^{1,\infty}-$weak$^*$ lower semicontinuous envelope of $J$, i.e.
%	\begin{equation}\label{Jrlxd}
$		J^{rlx}:=\inf\{\liminf_{j\to +\infty} J(u_j): u_j \overset{*}{\rightharpoonup} u \hbox{ in }W^{1,\infty}(I;\R^d)\}. $
%	\end{equation}
	If every sublevel set of $W$ has a basic Cartesian convexification, then
	$$J^{rlx}(u)=\esssup_{I\times I} {W}^{\times lc}(u'(x), u'(y)),$$ for every $u \in W^{1,\infty}(I;\mathbb R^d)$, where ${W}^{\times lc}$ stands for the Cartesian level convex envelope of $ W$.
\end{theorem}




	As already mentioned above, our results can be understood as results on non-local differential inclusions. Indeed, let $E$ be a compact subset of $\mathbb R\times \mathbb R$, then Theorem \ref{relax} states that for every sequence $(u_j)_j\subset W^{1,\infty}(I)$ such that $u_j\overset{*}{\rightharpoonup}u$ in $W^{1,\infty}(I)$ and
	\begin{align*}
		(u'_j(t),u'_j(s))\in E
	\end{align*}
for a.e. $(s,t) \in I\times I$, then 
\begin{align*}
	(u'(t),u'(s))\in {\widehat E}^{sc},
\end{align*}
where
\begin{align}\label{Ehat}
	%	&K^{sym}:= \left\{ (\xi, \eta)\in K: (\eta, \xi) \in K \right\},\\ %&K^{diag}:=\left\{(\xi,\eta)\in K: (\xi,\xi),(\eta,\eta)\in K\right\},\\ &
	\widehat{E}:=\left\{(\xi,\eta)\in E: (\eta,\xi),(\eta,\eta),(\xi, \xi)\in E\right\}%=K^{sym}\cap K^{diag},	
\end{align}
and the supersript {\rm sc} stands for its separately level convex hull.

Consequently Theorem \ref{cns} states that 
	$(u'(t),u'(s))\in E$ for a.e. $(s,t)\in I\times I$ if and only if $E$ is separately convex.

The differential inclusions counterpart of Theorems \ref{cnsvec} and \ref{relaxvec} will be discussed in Section \ref{proofs}. 

The paper is organized as follows: in section \ref{pre} we start fixing notation and  providing some preliminary results. The proofs of the main results are given in the last section.

\section{Preliminaries and Notation}\label{pre}
In the sequel $n$ ad $d$ will denote elements of $\mathbb N$. In this section we will give some defintion and recall some known results that will be useful for the proofs of our theorems.
 
\color{black}

%\begin{definition}
%	\label{sections}
%	Given a non-empty set $A\subset\R^{d}\times\R^d$ we denote by $\chi_A$ the characteristic function defined as 
%	\begin{align*}
%		\chi_A(\xi):=
%		\begin{cases}
%			& 0 \quad \text{if }\xi\in A\\
%			& \infty \quad \text{otherwise.}
%		\end{cases}
%	\end{align*}

%Moreover, we denote with $\mathbbm{1}_A$ the indicator function defined as
%\begin{align*}
%	\mathbbm{1}_A(\xi):=
%	\begin{cases}
%		& 1 \quad \text{if }\xi \in A\\
%		& 0 \quad \text{otherwise.}
%	\end{cases}
%\end{align*}
%Finally given $A\subset\R^n$ we denote with $dist (\xi,A):= \inf_{a\in A}|\xi-a|$ the distance of a point $\xi\in\R^n$ from a $A$, while given $B\subset\R^n$ we denote with $dist^n_H(A,B):=\sup_{a\in A}dist(a,B)+\sup_{b\in B}dist(b,A)$ the Hausdorff distance between $A,B$
%\end{definition}

%\color{vg}
%Our analysis will rely on $\Gamma-$convergence. We recall the definition.
%\begin{definition}
%	Given a family of functionals $(J_k)_{k\in\N}$ on a metric space $(X, d)$,  $J_k$ $\Gamma-$converges (with respect to $d$) to a functional $J$ on $X$ if 
%\begin{align*}
%	J(u)=(\Gamma-\liminf_{k\to \infty}J_k )(u)=(\Gamma-\limsup_{k\to \infty}J_k )(u),
%\end{align*}
%where
%\begin{align*}
%	&(\Gamma-\liminf_{k\to \infty}J_k )(u):=\inf\left\{\liminf_{k \to +\infty} J_k(u_k): u_k \stackrel{d}{\to} u \text{ in } X \right\}\\
%	&(\Gamma-\limsup_{k\to \infty}J_k )(u):=\inf\left\{\limsup_{k \to +\infty} J_k(u_k): u_k\stackrel{d}{\to} u \text{ in } X \right\},
%\end{align*} i.e. if for every $(u_k)_{k\in\N}\subset X$ such that %$u_k\stackrel{d}{\to}u$
%	\begin{align*}
%		J(u)\le\liminf_{k \to +\infty}J_k(u_k),
%	\end{align*}
%	and if for every $v\in X$ there exists a sequence $(v_k)_{k\in\N}\subset X$ such that $v_k\stackrel{d}{\to}u$ and
%	\begin{align*}
%		J(v)\ge\limsup_{k\to+\infty}J_k(v_k), 
%	\end{align*}
%\end{definition}
%For a detailed treatment of $\Gamma$-convergence we refer to \cite{DM}.

%\begin{remark}
%	\label{Glimrelax}
%	An equivalent definition of $\Gamma-$limit rely on the $\Gamma-\limsup$ and $\Gamma-\liminf$ functions, defined as
%By construction, both $\Gamma-\liminf$ and $\Gamma-\limsup$ are lower semicontinous with respect to the metric $d$ in $X$.
% and thus so must be the $\Gamma$-limit $J$. It follows that 
%Moreover if for every $k \in \mathbb N$, $J_k:=J_0$, then the $\Gamma$-limit of $J_k$ is nothing but the lower semicontinous envelope of $J_0$.

Moreover we recall the notions of \textit{level convexity, separate convexity} and \textit{separate level convexity.}
\begin{definition}
	\label{sep_conv}
	A set $A \subset \R^d\times\R^d$ is said  {\it separately convex} if for every $(\xi_1,\xi_2), (\eta_1,\eta_2) \in \R^d\times \R^d$ with $\xi_1=\eta_1$ or $\xi_2=\eta_2$ it holds
	\begin{align*}
		t(\xi_1,\xi_2)+(1-t)(\eta_1,\eta_2) \in A,
	\end{align*}
	for every $t \in (0,1).$ We denote with $A^{sc}$ the separate convex hull of $A$, namely the smallest separately convex set containing $A$.
\end{definition}


\begin{remark}
	\label{compactness}
	If $A\subset \mathbb R^d\times \mathbb R^d$ is open then so is $A^{sc}$. This is in general not true for compactness, see \cite[Remark 7.18 (ii)]{Dac}, but, as observed in \cite[Proposition 2.3]{K} the property is preserved if $A\subset \R\times\R$. 
\end{remark}
For any function $f:\R^d\times\R^d \to \R \cup \{\infty\}$ and any real number $c$, by $L_c(f)$ we denote the level set $c$ of $f$, i.e.
$$
L_c(f):=\{(\xi, \eta) \in \R^d \times \R^d: f(\xi, \eta)\leq c\}.
$$
\begin{definition}
	\label{sep_level_conv}
	A function $f:\R^d\times\R^d \to \R \cup \{\infty\}$ is said to be \begin{itemize}
		\item [1.] {\it level convex} if for every $c \in \R$ the set $L_c(f)$ is convex; 
		\item[2.]  {\it separately convex} if for every $\xi \in \R^d$ both $f(\cdot, \xi)$ and $f(\xi, \cdot)$ are convex functions. 
		\item[3.]  {\it separately level convex} if for every $c \in \R$  $L_c(f)$ is separately convex.
	\end{itemize}
	We denote the separately  (level) convex envelope of $f$, namely the greatest (level) convex function less than or equal to $f$ by ($f^{slc}$) $f^{sc}$.
\end{definition}


\begin{definition}\label{WdiagDEF}
	Given $W:\R^d\times\R^d\to\R\cup \{\infty\}$ we say that $W$ is symmetric if $W(\xi,\eta)=W(\eta, \xi)$ for every $(\xi,\eta)\in\R^d\times\R^d$. Also, we say that $W$ is diagonal if 
	\begin{align*}
		(\xi,\eta)\in L_c(W) \Rightarrow (\xi,\xi),(\eta,\eta)\in L_c(W).
	\end{align*}
	In other words, following \cite[eq. (28)]{KRZ},
	\begin{align*}
		W(\xi,\eta)=\max\{W(\xi,\xi), W(\xi,\eta), W(\eta, \eta)\}.
	\end{align*}
\end{definition}


\begin{definition}
	We say that a function $W:\R^d\times\R^d\to\R\cup\{\infty\}$ is coercive if there exists $C'>0$ such that
	\begin{align*}%\label{coerciW}
		C'|(\xi,\eta)| \leq W(\xi,\eta),
	\end{align*}
	for every $\xi,\eta\in\R^d.$
\end{definition}
\begin{remark}
	\label{identity}
	From Definitions \ref{sep_conv} and \ref{sep_level_conv},  for any $W:\R^d\times\R^d\to\R$,
	\begin{align*}
		L_c(W)^{sc}\subset L_c(W^{slc}),
	\end{align*}
	The inverse inclusion in general does not hold, while if $d=1$, in  \cite[Lemma 7.4]{KZ} it has been proven that if $W$ is symmetric and diagonal then 
	\begin{align*}
		L_c(W)^{sc}=L_c(W^{slc}).
	\end{align*}
\end{remark}

\subsection{Properties of the supremal nonlocal functionals}

In this subsection we will describe the properties of the energy densities appearing in Theorems \ref{cns}- \ref{relaxvec}.
First of all we observe that there is no loss of generality in supposing that the supremands $V$ and $W$ in \eqref{functional1d} and \eqref{functionald} are diagonal and symmetric.


In fact, let $W: \mathbb R^{d\times n} \times \mathbb R^{d\times n} \to \mathbb R$ and  define the function $\widehat W:\mathbb R^{d\times n} \times \mathbb R^{d\times n} \to \mathbb R$ by
\begin{align}
	\label{simdiag}
	\widehat{W}(\xi,\eta):=\inf\left\{ c\in\R: (\xi,\eta)\in\widehat{L_c(W)} \right\},
\end{align}
as in \cite[(7.1)]{KZ}, with $\widehat{L_c(W)}$ given by \eqref{Ehat}. $\widehat{W}$ is the symmetric and diagonal, and $\widehat{W}\ge W, $ so the coercivity of $W$ is inherited by $\widehat{W}$. 
Following \cite{KZ}, define for $\Omega \subset \mathbb R^n$, and every $K \subset \R^{d\times n} \times \R^{d\times n}$
\begin{align}\label{AKdef}
	A_K:=\left\{v \in L^{\infty}(\om;\mathbb R^{d\times n}): (v (x), v(y)) \in K \text{ for a.e. }x\in \om\times\om \right\}.
\end{align} 
In \cite[Proposition 5.1]{KZ} it has been proven that $A_K = A_{\widehat K}$, from which follows 
\cite[eq (7.3)]{KZ}, i.e. for every $u \in W^{1,\infty}(\Omega;\mathbb R^d)$,
%by \cite[eq (7.2), Proposition 5.1 and eq. (7.4)]{KZ} ,
\begin{align*}
	\esssup_{\om\times\om}\widehat{W}(\nabla u(x),\nabla u(y))=%\inf\left\{ c\in\R: \nabla u\in A_{L_c(\widehat{W})} \right\}\\
	%&=\inf\left\{ c\in\R: \nabla u\in A_{\widehat{L_c(W)}} \right\}=
	%\inf\left\{ c\in\R: \nabla u\in A_{L_c(W)} \right\}\\
	\esssup_{\om\times\om}W(\nabla u(x),\nabla u(y)).
\end{align*}
In particular the last equality holds when $W=V:\mathbb R\times \mathbb R\to\R$ and $\Omega=I$.


%Moreover if $W$ is a supremand symmetric, diagonal and coercive, and if  $W_0$ is the density defined by \eqref{W0def}, then $W_0$ enjoys the same properties.  

%If $\xi \in \mathbb R^d$, we will denote its first $d-1$ components, by $\xi_\alpha \in \mathbb R^{d-1}$, i.e. $\xi= (\xi_\alpha, \xi_d)$.

%\begin{proposition}\label{diagsym}
%	Let $W:\mathbb R^d \times \mathbb R^d\to \mathbb R$ be lower  semicontinuous (continuous), symmetric, diagonal and coercive, then the function $W_0:\R^{d-1}\times \R^{d-1}\to \R$, defined as
%	\begin{equation}\label{W02}W_0(\xi_\alpha, \eta_\alpha):=\inf_{a,b \in \mathbb R}W((\xi_\alpha, a), (\eta_\alpha, b))
%		\end{equation}
%	is also lower semicontinuous (continuous), symmetric, diagonal and coercive. 
%\end{proposition}

%\begin{proof}[Proof]
%	By the very definition of $W_0$, which goes back to \cite{ABP}, and by \cite[Proposition 1]{LeDretRaoult'95}, $W_0$ turns out to be lower semicontinuous (continuous) and coercive. Moreover due to the coerciveness of $W$, the infimum in the definition of $W$ is attained.
	
%	Now we prove its symmetry.
%	\begin{align*}
%		&W_0(c_\alpha,d_\alpha)=\inf_{a,b\in\R} W((c_\alpha,a),(d_\alpha,b))=W((c_\alpha, c') (d_\alpha,d'))=W((d_\alpha, d'),(c_\alpha, c'))\\
%		&\geq \inf_{a,b \in \R}W((d_\alpha, a), (c_\alpha, b))= %\color{red} \hbox{erase}W((d_\alpha, d''), (c_\alpha, c''))=\color{black}
%		W_0(d_\alpha, c_\alpha).
%	\end{align*}
%	Changing the role of $c_\alpha$ and $d_\alpha$ we obtain that $W_0$ is symmetric.
%\color{red} to erase	On the other hand, the same arguments below
%	\begin{align*}
%		&\inf_{a,b\in\R} W((c_\alpha,a),(d_\alpha,b))=W((c_\alpha, c'),(d_\alpha,d'))=
%		W((d_\alpha, d'),(c_\alpha, c'))\\
%		&\geq\inf_{a,b \in \R}W((d_\alpha, a), (c_\alpha, b))= W((d_\alpha, d''), (c_\alpha, c''))=W((c_\alpha, c''), (d_\alpha, d''))\\
%		&\geq \inf_{a,b\in \mathbb R} W((c_\alpha, a), (d_\alpha, b))
%	\end{align*}
%	prove that $c'=c''$ and $d'=d''$ in the definition of $W_0$.
%	\color{vg}
%	For what concerns the diagonality,  we observe that, being $W$ diagonal, for every $c=(c_\alpha, c')$ and $d=(d_\alpha, d')$ such that $W_0(c_\alpha, d_\alpha)= W(c,d)$, by \eqref{Wdiagdef}, it results
%	$W(c, d)=\max\{W(c,d), W(c,c), W(d,d), W(d,c)\}$,
%	hence 
%	\begin{align*}
%		&W_0(c_\alpha, d_\alpha)= W(c,d) =\max\{ W(c,c), W(d,d), W(d,c)\}\\
%		&\geq \max\{W_0(c_\alpha,c_\alpha),W_0(d_\alpha, d_\alpha), W(d_\alpha, c_\alpha)\},
%	\end{align*}
%	from which, in view of Definition \ref{WdiagDEF}, the diagonality of $W$ follows.
	
%The proof of the coerciveness of $W_0$ is trivial, i.e. it suffices to infimize over $(s,t) \in \mathbb R$ in both sides of the inequality
%	$$
%	C'|((\xi_\alpha, t), (\eta_\alpha, s))|\leq W(((\xi_\alpha, t), (\eta_\alpha, s))).
%	$$
%\end{proof}



%\begin{remark}
%	\label{uniqueness}
%	If $W$ is as in Proposition \ref{diagsym} and $W_0$ is as in \eqref{W02}, from the above arguments, it follows that for a.e. $t,s \in I$, and every $v \in W^{1,\infty}(I;\R^{d-1})$
%	\begin{align*}
%		W_0(v'(t),v'(s))=W_0(v'(s),v'(t)).
%	\end{align*}
%	\color{red} This means that there exists a unique measurable function $u:I^2\to \mathbb R$ such that
%	\begin{equation}
%		\label{representation}
%		W_0(v'(t), v'(s))= W((v'(t), u(s,t)), (v'(s), u(s,t))).
%	\end{equation}
%	Indeed, by the coercivity assumption of $W$ and a measurable selection lemma, as in \cite[Proposition 7]{LeDretRaoult'95}, there exists a measurable function $(w_1,w_2): I\times I \to \R^2$ such that
	% if we suppose that 
%	\begin{align*}
%		W_0(v'(s),v'(t))= W(((v'(s),w_2(s,t)),(v'(t),w_1(s,t))),
%	\end{align*}
%for a.e. $(s,t)\in I\times I$.
%Then
%	\begin{align*}
%		&W_0(v'(s),v'(t))= W(((v'(s),w_2(s,t)),(v'(t),w_1(s,t)))\\
%		&=W((v'(t),w_1(s,t)),(v'(s),w_2(s,t)))= W_0(v'(t), v'(s)),
%	\end{align*}
%	so, from the arbitrariness of $s$ and $t$, without loss of generality we can suppose that there is a unique function $w_2=w_1$, symmetric and such that...
%	\begin{align*}
%		w_2(s,t)=w_1(s,t),
%	\end{align*}
%	for a.e. $(s,t) \in I^2$.
	
%	Moreover by the very definition \eqref{J0def}, if $W$ is lower semicontinuous, symmetric, diagonal and satisfies \eqref{coerciW}, it results
%	$$
%	J^0(v)=\esssup_{(t,s)\in I\times I} W_0(v'(t), v'(s))= \inf_{u\in L^\infty(I^2)}\esssup_{(t,s)\in I\times I}W((v'(t), u(t,s)), (v'(s), u(s,t))).
%	$$
%	Indeed, fixed $u\in L^\infty(I^2)$ and $t,s \in I$ it is trivial to see that
%	\begin{align*}
%		W_0(v'(t),v'(s)) \le W((v'(t),u(s,t)),(v'(s),u(s,t))),
%	\end{align*}
%so
%\begin{align*}
%		\esssup_{(t,s)\in I\times I} W_0(v'(t),v'(s)) \le \esssup_{(t,s)\in I\times I} W((v'(t),u(s,t)),(v'(s),u(s,t))),
%\end{align*}
%that leads to
%\begin{align*}
%	\esssup_{(t,s)\in I\times I} W_0(v'(t),v'(s)) \le \inf_{u\in L^\infty(I^2)} \esssup_{(t,s)\in I\times I} W((v'(t),u(s,t)),(v'(s),u(s,t))).
%\end{align*}
%On the other hand, by \eqref{representation} we have that for every $t,s \in I$
%\begin{align*}
%	W_0(v'(t), v'(s))= W((v'(t), w(s,t)), (v'(s), w(s,t)))
%\end{align*}
%and thus
%\begin{align*}
%	\esssup_{(t,s)\in I\times I} W_0(v'(t), v'(s))= \esssup_{(t,s)\in I\times I} W((v'(t), w(s,t)), (v'(s), w(s,t))).
%\end{align*}
%Since $W$ is coercive then it follows that $w\in L^\infty(I^2)$. Passing to the infimum over $L^\infty(I^2)$ on the right hand side we get
%\begin{align*}
%	\esssup_{(t,s)\in I\times I} W_0(v'(t), v'(s))\ge \inf_{u \in L^\infty(I^2)}\esssup_{(t,s)\in I\times I} W((v'(t), u(s,t)), (v'(s), u(s,t))),
%\end{align*}
%which grant us the desired identity.
%	\color{black}
%\end{remark}


%\begin{definition}
%	Given $E\subset\R\times\R$ the projections of $E$ on its first and second component are
%	\begin{align*}
%		\pi_1(E)=\bigcup_{(a,b)\in E}\{a\}, \quad \pi_2(E)=\bigcup_{(a,b)\in E}\{b\}.
%	\end{align*}
%Fixed $\xi\in \R$ the sections of $E$ in $\xi$ are denoted as $\mathfrak{E}^\xi_1$ and $\mathfrak{E}^\xi_2$ and are defined as 
%\begin{align*}
%	\mathfrak{E}^\xi_1:=\left\{ a\in\R: (a,\xi)\in E \right\}, \quad \mathfrak{E}^\xi_2:=\left\{ b\in\R: (\xi,b)\in E \right\}
%\end{align*}
%\end{definition}
%
%\begin{remark}
%	If the set $E$ is symmetric then $\pi_1(E)=\pi_2(E)$ and $\mathfrak{E}_\xi=\mathfrak{E}_2^\xi$ for every $\xi \in \R^n$.
%\end{remark}
%
%The following results are slight modifications of results proved in \cite{KZ}.
%
%
%\begin{proposition}
%	\label{prop1}
%	Given $B\in \R\times\R$ closed and $F\in \R\times\R$ such that
%	\begin{align}
%		\label{hprop1}
%		\left[\pi_1(F)\times\pi_1(F)\right]\cap B=\emptyset \quad or \quad \left[\pi_2(F)\times\pi_2(F)\right]\cap B=\emptyset,
%	\end{align}
%then 
%\begin{align*}
%	A_B=A_{B\setminus F}.
%\end{align*}
%\end{proposition}
%
%\begin{proof}
%	Let $F$ satisfy the first condition of \eqref{hprop1} and fix $u \in A_B$ such that $u\notin A_{B\setminus F}$. By hypothesis there exist a measurable set $N \subset \Omega\times\Omega$ with positive measure such that $(u'(x), u'(y))\in F$ for all $(x,y)\in N$. Tonelli's theorem tells us that there exist $\bar{y}\in\Omega$ such that $\leb^n(\mathfrak{N}_1^{\bar{y}})>0$. By construction 
%	\begin{align*}
%		(u'(x), u'(\bar{y}))\in F \quad\text{for all } x \in \mathfrak{N}_1^{\bar{y}},
%	\end{align*}
%wich means that
%\begin{align*}
%	u'(x)\in\pi_1(F) \quad\text{for all } x\in \mathfrak{N}_1^{\bar{y}},
%\end{align*}
%and so 
%\begin{align*}
%	(u'(x),u'(y))\in \pi_1(F)\times\pi_2(F) \quad\text{for all } \mathfrak{N}_1^{\bar{y}}\times\mathfrak{N}_1^{\bar{y}}.
%\end{align*}
%By \eqref{hprop1} we deduce that $(u'(x),u'(y))\notin B$ for $(x,y) \in \mathfrak{N}_1^{\bar{y}}\times\mathfrak{N}_1^{\bar{y}}$, but this contradict the hypothesis that $u \in A_B$ and this conclude the proof.
%\end{proof}
%
%\begin{proposition}
%	\label{prop2}
%	Given $(F_k)_{k\in\N}$ a family of sets of $\R\times\R$ then
%	\begin{align*}
%		\bigcap_{k\in\N}A_{F_k}=A_{\cap_{k\in\N}F_k.}
%	\end{align*}	
%\end{proposition}
%
%\begin{proof}
%It is trivial to prove that $A_{\cap_{k\in\N}F_k} \subset \bigcap_{k\in\N}A_{F_k}$. Let's prove the inverse inclusion. Fix $u\in\bigcap_{k\in\N}A_{F_k}$, then, by definition, for every $k\in\N$ there exists a set $N_k\subset\R\times\R$ of Lebesgue measure zero such that
%\begin{align*}
%	(u'(x),u'(y)) \in F_k \quad \text{for every } (x,y)\in\R\times\R\setminus N_k.
%\end{align*}
%If we define $N:=\cap_{k\in\N}N_k$, then $N$ is a set of Lebesgue measure zero such that
%\begin{align*}
%	(u'(x),u'(y)) \in \bigcap_{k \in \N}F_k \quad \text{for every } (x,y)\in\R\times\R\setminus N
%\end{align*}
%and this proves that $\bigcap_{k\in\N}A_{F_k}\subset A_{\cap_{k\in\N}F_k.}$
%\end{proof}
%
%The next Proposition will show that given a compact set $K\subset \R\times\R$ the definition of $A_K$ is invariant under symmetrization and diagonalization of $K.$
%
%\begin{proposition}
%	\label{invariance}
%	Given $ B,C \in \R\times\R$ closed, then $A_B=A_C$ if and only if $\widehat{B}=\widehat{C}$. Moreover $A_B=A_{\widehat{B}}$.
%\end{proposition}
%
%\begin{proof}
%	Let's start proving that if $A_B=A_C$ then $\widehat{B}\subset\widehat{C}$. Suppose $\widehat{B}\neq\emptyset$ and fix $(\xi,\eta)\in \widehat{B}$. Consider the affine function $u$ on $\om$ such that
%	\begin{align}
%		u'(x)=
%		\begin{cases}
%			& \xi \quad \text{if }x \in \om_\xi\\
%			& \eta \quad \text{if }x \in \om_\eta,
%		\end{cases}
%	\end{align}
%for some subset $\om_\xi, \om_\eta$ of $\om$ such that $\om_\eta=\om\setminus\om_\xi$. Since $u$ is affine then $|\om_\xi|, |\om_\eta|>0$, moreover $u \in W^{1,\infty}(\om)$. Since $(\xi, \eta)\in\widehat{B}\subset B$ then $(\xi,\xi), (\eta, \eta), (\eta,\xi) \in B$. Since $A_B=A_C$ then $(\xi,\eta),(\xi,\xi),(\eta,\eta),(\eta,\xi) \in C$ and thus $(\xi, \eta) \in\ \widehat{C}.$ Since $B$ and $C$ are arbitrary if follows that $\widehat{B}=\widehat{C}.$
%To prove that if $\widehat{B}=\widehat{C}$ then $A_B=A_C$, is a consequence of $A_B=A_{\widehat{B}}$, so let's prove this second identity.\\
%First we observe that $A_B=A_{B^{sym}}$, indeed if $u\in A_B$ then by definition $u \in A_{B^T}$ and thus $u \in A_{B^{sym}}=A_{B\cap B^T}$. Without loss of generality we can assume $B$ symmetric. For the next step we will decompose $B\setminus\widehat{B}$ using suitable sets satisfying \eqref{hprop1}. Since $B$ is closed then $B^c$ is open, thus for any $\xi\in\Q$ such that $(\xi,\xi)\notin B$ we can construct an open cube $]a_\xi,b_\xi[\times]a_\xi,b_\xi[ \subset B^c$ for some $a_\xi,b_\xi\in \R$
%such that $\xi \in ]a_\xi,b_\xi[ .$ Every such cube satisfy hypothesis \eqref{hprop1} for both $F=]a_\xi,b_\xi[ \times \R$ and $F=\R \times ]a_\xi,b_\xi[$ and so by Proposition \eqref{prop2} it holds true that
%\begin{align*}
%	A_B=A_{B\setminus F_\cup},
%\end{align*}
%with
%\begin{align*}
%	F_\cup:=\bigcup_{\xi \in \Q,\, (\xi,\xi)\notin B} (\R \times ]a_\xi,b_\xi[) \times (]a_\xi,b_\xi[ \times \R),
%\end{align*}
%indeed if we fix an enumeration $(\xi_i)_{i \in \N}$ of $\left\{\xi \in \Q: (\xi,\xi)\notin B \right\}$ and define for $k \in \N$
%\begin{align*}
%	F_\cup^k:=\bigcup_{i=1}^k (\R \times ]a_{\xi_i},b_{\xi_i}[) \times (]a_{\xi_i},b_{\xi_i}[ \times \R),
%\end{align*}
%Since for a $E\subset\R\times\R$ it holds
%\begin{align*}
%	\widehat{E}=E^{sym}\setminus F_E, \quad F_E:=\bigcup_{(\xi,\xi)\notin E}\R\times(\left\{\xi\right\})\cup(\left\{\xi\right\}\times\R),
%ea\end{align*}
%then since $B$ is symmetric and since $F_\cup=F_B$ we get $B\setminus F_\cup=\widehat{B}$. Using Proposition \eqref{prop1} we get
%\begin{align*}
%	A_B=\bigcap_{k \in \N} A_{B\setminus F_\cup^k}=A_{\cap_{k \in \N}B\setminus F_\cup^k}=A_{B\setminus \cup_{k \in \N} F_\cup^k}=A_{B\setminus F_\cup}=A_{\widehat{B}}
%\end{align*}
%\end{proof}
%


\color{black}

 
%This shows that the functional $J$ is invariant by symmetrization and diagonalization of its supremand.


%	On another note, for a generic $W$ symmetrical and diagonal we have that
%	\begin{align*}
%		L_c(W)=\widehat{L_c(W)},
%	\end{align*}
%	where we recall that $\widehat{L_c(W)}$ is the diagonalization and symmetrization of $L_c(W)$. 
%\end{remark}

\color{black}
\section{Proofs}\label{proofs}
In this section, first we prove that the separate level convexity of the supremand $V$ appearing in \eqref{functional1d} %being separately level convex 
is a sufficient and necessary condition for the sequential $W^{1,\infty}-$weak$^*$ lower semicontinuity of fhe functional $J_{1d}$  defined in \eqref{functional1d}.
%$J_{1d}(u):=\esssup_{(x,y) \in I\times I}V(u'(x), u'(y))$.
To this end, we recall  for every $E \subset  \mathbb R\times \mathbb R$, the set \begin{align}\label{BEdef}
	B_E:=\left\{u \in W^{1,\infty}(I): (u' (x), u'(y)) \in E \text{ for a.e. }(x,y)\in I\times I \right\}.
\end{align} 
Moreover, we recall that $J_{1d}$ is sequentially $W^{1,\infty}-$ weakly$^*$ lower semicontinuous if and only if $B_{L_c(V)}$ is sequentially $W^{1,\infty}-$weakly$^*$ closed for every $c \in \mathbb R$. With the following results we will prove that $B_{L_c(V)}$ is sequentially $W^{1,\infty}-$weakly$^*$ closed if and only if $B_{L_c(V)}=B_{L_c(V)^{sc}}$ and by Remark \ref{identity} we have that this is true if and only if $V$ is separately level convex. 


The next lemma is key to prove of Theorem \ref{cns}. Let $I=(a,b)\subset\R$, we will denote with $S^\infty_E(I)$ the set of all the simple functions $s$ (see for instance \cite[eq (5.6)]{KZ}), and with $D^\infty(I)$ the set of all the integral functions
	
	\begin{align}\label{integralfunction}
		u_s(x,c):=c+\int_a^x s(t)dt,\quad x\in I,
	\end{align}
	with $s\in S^\infty(I)$ and $c\in\R$. 
\begin{remark}
	\label{Amerio}
	We recall that,  given $v\in L^\infty(I)$, the function $u_v(\cdot,c) \in L^\infty(I)$ defined, for every $c\in\R$ as \eqref{integralfunction}, i.e.
	$$
	u_v(x,c):=c+ \int_a^x v(t)dt,
	$$  
	is an element of $W^{1,\infty}(I)$. Moreover, the distributional derivative exists also in the classical sense and
	\begin{align*}
		\frac{du_v(x,c)}{dx}=v(x)\quad \text{for a.e. }x\in I.
	\end{align*}
Since for our subsequent analysis the role of $c$ can be neglected, we will omit it from the notation and denote the integral function just by $u_v$.

Thus, given a function $u_v$ with $v\in L^\infty(I)$ and a sequence of functions $(v_j)_{j\in\N}\subset L^\infty(I)$ such that $v_j\overset{*}{\rightharpoonup} v$ in $L^\infty(I)$, then $u_{v_j}\overset{*}{\rightharpoonup}u_v$ in $W^{1,\infty}(I)$ (see \cite{F}).  In particular if $v$ is a simple function $s\in S^\infty(I)$, it is immediately verified that $D^\infty(I)\subset 
W^{1,\infty}(I).$



\end{remark}

\begin{lemma}
	\label{dense}
	Given $E\subset\R\times\R$ symmetric and diagonal, let $B_E$ be as in \eqref{BEdef}, then, for every $u\in B_E$ there exists a sequence $(u_j)_{j\in\N}\subset B_E\cap D^\infty(I)$ such that $u_j\overset{*}{\rightharpoonup}u$ in $W^{1,\infty}(I)$.
\end{lemma}
\begin{proof}
	Since $B_E\subset W^{1,\infty}(I)$ we can 
define $B'_E:=\{u': u \in  B_E\}$. Clearly $B'_E\subset L^\infty(I)$ and it is a subset of 
$A_E:=\{v \in L^\infty(I): (v(x), v(y))\in E \}$ in \eqref{AKdef}.
Hence, by \cite[Lemma 5.4 and Corollary 5.5]{KZ} any element  $v \in A_E$ can be approximated in $L^\infty(I)$ strong convergence by a sequence $\{s_j\}_j \subset A_E\cap S^\infty(I)$. In particular this happens for $v=u' \in B'_E$.
Thus, by Remark \ref{Amerio}, given $u \in B_E$, there exists $\{u_{s_j}\}_j\subset D^\infty_I\cap B_E$
weakly* converging to $u$ in $W^{1,\infty}(I)$.
\end{proof}
%
%	
%	TOGLIEREI LA VECCHIA DIMOSTRAZIONE sotto SE QUESTA di spora TI CONVINCE
%	
%	assume that $m\le u'(x)\le M$ for some $m,M \in \R$ for every $x\in\Omega$. Fixed $k\in \N$, we partition the interval $[m,M]$ in $p\in \N$ half-open intervals $I_k^i=[m_i, M_i)$ of lenght $|I_k^i|<\frac{1}{k}$ for every $i=1,...,p$. If we define the sets
%	\begin{align*}
%		\Omega_k^i=u^{-1}(I_k^i), \quad i=1,...,p
%	\end{align*}
%then for every $\Omega_k^i$ is a $\leb^1$-measurable set for every $i$ and also $\Omega=\cup_{i=1}^p\Omega_k^i$. We denote with $J_k$ the subset of $\left\{1,...,p\right\}$ defined by
%\begin{align*}
%	|I_k^i|>0, \quad i\in J_k,
%\end{align*}
%and we suppose that $J_k=\left\{1,...,l\right\}$ for some $l\in\N$ with $l\le p.$
%We define the simple function
%\begin{align*}
%	s_k(x)=\sum_{1}^{l}\mathbbm{1}_{\Omega_k^i}u(x_k^i), \quad x\in\Omega,
%\end{align*}ù
%where the points $x_k^i$ are constructed as follow. We set $F=\left\{ (x,y)\in\Omega\times\Omega:(u'(x),u'(y))\in E\right\}$ and we notice that since $E$ is diagonal and symmetric so is $F$, i.e. $(y,x),(y,y),(x,x)\in M$ for every $(x,y)\in M.$ Recolling the notation for sections of a set introduced in Definition \ref{sections} then we define
%\begin{align*}
%	G=\left\{ x\in\Omega: |\mathfrak{F}^x|=|\Omega| \right\}.
%\end{align*}
%G has positive measure, indeed $\leb^1\otimes\leb^1(\Omega)=\leb^1\otimes\leb^1(F)=\int_\om\mathfrak{F^x}dx$, so $|\mathfrak{F^x}|=|\Omega|$ for a.e. $x\in\Omega$, which means that $|G|=|\Omega|.$
%Since $|G|>0$, then we can chose $x_k^1\in \Omega_j^1\cap G$ and the iteratively for every $i>1$
%\begin{align*}
%	x_k^i\in\Omega_k^i\cap G\cap \left( \bigcup_{t=1}^{i-1}\mathfrak{F}^{x_k^t} \right).
%\end{align*}
%By construction $u'(x_k^i)\in Q_k^i$ for $i=1,...,l$, and 
%\begin{align*}
%	(x_k^i,x_k^{i'})\in F, \quad i,i'=1,...,l.
%\end{align*}
%Furthermore 
%\begin{align*}
%	(s_k(x),s_k(y))\in\bigcup_{i,i'\in\left\{1,...,l\right\}} \left\{(u'(x_k^i),u'(x_k^{i'}))\right\}\subset E\quad \text{for }\leb^1\otimes\leb^1-\text{a.e. }(x,y)\in\Omega\times\Omega,
%\end{align*}
%and 
%\begin{align*}
%	|u'(x)-s_k(x)|\le \frac{1}{k} \quad \text{for a.e. }x\in\Omega,
%\end{align*}
%so $s_k\to u'$ in $L^\infty(\Omega)$. Withouth loss of generality we can suppose $\Omega=(a,b)$, so by Remark \ref{simple} we have that $(u_{s_k})\subset D_E^\infty(\Omega)$ and $u_{s_k}(\cdot,u(a))\rightharpoonup^*u$, i.e. $D_E^\infty(\Omega)$ is dense in $\Omega$ with respect to the $W^{1,\infty}(\Omega)-weakly^*$ topology.
%
%FINO A QUI
%\end{proof}

\begin{remark}
In the previous lemma we supposed that $E \subset \R \times \R$ is symmetric and diagonal, but from \cite[Proposition 5.1]{KZ} we can deduce that the previous lemma holds for every $E$ closed and contained in $\mathbb R^{d\times n} \times \mathbb R^{d\times n}$, with $n, d \in \mathbb N$.
\end{remark}
\color{black}


\begin{proof}[Proof of Theorem \ref{cns}]
	Due to the coercivity assumption on $V$ and the metrizability of the weak* topology on bounded sets we will omit the word 'sequentially'.
	The sufficiency of the separate level convexity of $V$ follows from one of the implications of \cite[Theorem 1.3(i)]{KZ}, taking into account the weak* convergence of $(u'_k)_k$ for every sequence $(u_k)_k$ weakly* converging to $u$ in $W^{1,\infty}(I)$.
	
	For what concerns the reverse implication, we claim \color{black} that the $W^{1,\infty}-$weak* lower semicontinuity of the functional $J_{1d}$ in \eqref{functional1d}  ensures the $L^\infty-$weak* lower semicontinuity of the functional in $L^\infty(I)$ defined as $\esssup_{I\times I}
V(v(x), v(y))$ 	along sequences $(v_k)_k \subset L^\infty(I)$. Indeed, if $v_k \overset{\ast}{\rightharpoonup} v$, in $L^\infty (I)$, then, as observed in Remark \ref{Amerio}, it is possible to define \begin{align*}%\label{uk}
	u_k(x):=\int_a^x v_k(t)dt \in L^\infty(I)
\end{align*} and $u_k$ is in $L^\infty(I)$. Indeed the fact that $v_k$ is bounded in $L^\infty(I)$, entails also that
$$
\esssup_{x \in I}|u_k(x)| \leq \esssup_{x \in I}\int_a^x |v_k|(x)dx \leq \|v_k\|_{L^{\infty}(I)}(b-a).$$ Moreover $\frac{d u_k}{dx} $ exists for a.e. $x \in I$ and coincides with $v_k$. 
Hence $u_k \in W^{1,\infty}(I)$. Moreover $u_k$ is unifomrly bounded in the $W^{1,\infty}$ norm, thus, up to a not relabelled subsequence, $u_k \overset{\ast}{\rightharpoonup} u$ in $W^{1,\infty}(I)$ with $\frac{d u}{dx}= v$ a.e. in $I$. This proves our claim.
%The above observations allow us to assert the weak* lower semicontinuity of  $\esssup_{I\times I}
%V(v(x), v(y))$. 
Consequently the proof is concluded by the reverse implication of \cite[(i)Theorem 1.3]{KZ}.   
 \end{proof}
\color{black}


%QUESTA PARTE DI SOTTO NON L'HO CONTROLLATA MA SE SERVISSE LE NOTAZIONI ANDREBBERO CAMBIATE
%The next result was proved in \cite{KZ}, this is a slight modification.
%\begin{proposition}
%	\label{KZ5.6}
%Given $E\subset\R\times\R$ closed, then 
%\begin{align*}
%	A_E=\bigcup_{P\in P_E} A_P,
%\end{align*}
%where $P_E$ is defined as in Definition \ref{maximal}.	
%\end{proposition}
%
%\begin{proof}
%	The inclusion $\bigcup_{P\in P_E} A_P \subset A_E$ is trivial. We will now show the inverse inclusion. If $u\in D_E^\infty(\Omega)$ then $u'$ is of the form 
%	\begin{align*}
%		u'(x)=\sum_{1}^{j}\mathbbm{1}_{\Omega_k}c_k, \quad x\in\Omega,
%	\end{align*}
%	for some non-empty sets $\Omega_1,...,\Omega_j$ and for some constant $c_1,...,c_k,$ with $(c_k,c_{k'})\in E$ for all $k,k'=1,...,j$. It from $u\in A_E$ follow that
%	\begin{align*}
%		v_{u'}(\Omega,\Omega)=u'(\Omega)\times u'(\Omega)=\bigcup_{k,k'=1}^j u'(\Omega_k)\times u'(\Omega_{k'})=\bigcup_{k,k'=1}^j\left\{ (c_k,c_{k'}) \right\}\subset E.
%	\end{align*}
%Fixed $u\in A_E$, by lemma \ref{dense} we can chose a sequence $(u_k)_{k\in\N}\subset D_E^\infty(\Omega)$ such that $u_k\rightharpoonup^*u$ in $W^{1,\infty}(\Omega).$ Since the chosen sequence is uniformly bounded, then we can assume that $E$ is also bounded, and so compact. Since $u'_k$ is simple, then there exist $A_k\subset\R$ compact such that defined $P_k=A_k\times A_k$ then $u'_k\in A_{P_k}$. By Blaschke selection theorem (see\cite{AFP}, \cite{Rogers}) there exists a non-relabeld subsequence $(A_k)_k$ and $A\subset\R$ compact such that 
%\begin{align*}
%	dist^1_H(A_k,A)\to 0, \text{ as } k\to \infty,
%\end{align*}
%Since it holds
%\begin{align*}
%	dist^2_H(C\times C, B\times B)\le 2dist^1_H(C,B),
%\end{align*}
%for every non-empty sets $C,B\subset\R$, then it is obvious that
%\begin{align*}
%	dist_H^2(P_k,A\times A)\to 0, \text{ as }k\to \infty,
%\end{align*}
%so since $P_k\subset E$, we get $A\times A\subset E$. Finally, since $u_k\rightharpoonup^*u$, then by Dominated convergence theorem we have
%
%
%\begin{align*}
%	\int_\om&\int_\om dist(v_{u'},A\times A)dxdy=\lim_{k \to +\infty}\int_\om\int_\om dist(v_{u'_k},A\times A)dxdy \\ 
%	&\le\lim_{k \to +\infty}\int_\om\int_\om dist(v_{u'_k},P_k)dxdy+\lim_{k \to +\infty}dist_H^2(P_k,A\times A)=0,
%\end{align*}
%thus $v_{u'}\in A\times A$ for a.e. $(x,y)\in\Omega$, i.e. $u\in A_{A\times A}$.
%\end{proof}
%
%\begin{lemma}
%	\label{KZ5.8}
%	Given $a,b\in\R$ and $K=\{a,b\}\times\{a,b\}$, then
%	\begin{align*}
%		A_{Q_{a,b}}\subset A_K^\infty.
%	\end{align*}
%\end{lemma}
%\begin{proof}
%	The proof follows the line of Lemma 5.8 of \cite{KZ} but uses $D_E^\infty(\Omega)$ instead of $S_E^\infty(\Omega)$ together with Remark \ref{simple}.
%\end{proof}
%
%
%
%\begin{theorem}
%	Given a compact set $K\subset\R\times\R$, then $A^\infty_K=A_{\widehat{K}^{sc}}$.
%\end{theorem}
%
%\begin{proof}
%	We begin proving that $A^\infty_K\subset A_{\widehat{K}^{sc}}.$ Fix $u\in A^\infty_K$, then by Proposition \ref{invariance} there exists a sequence $(u_i)_{i\in\N}\subset L^\infty(\Omega)$ such that $v_{u_i}(x,y)\in K$ for a.e. $(x,y)\in\Omega\times\Omega$ and such that $u_i\rightharpoonup^*u$ in $W^{1,\infty}(\Omega)$, so in particular it will be $u'_i\rightharpoonup^*u'$ in $L^\infty(\Omega)$. Let $\left\{\nu_x\otimes\nu_y\right\}_{(x,y)\in\Omega\times\Omega}$ be the Young measure generated from $(v_{u_i})_{i\in \N}$ by Lemma \ref{P}. $\widehat{K}$ is compact since $K$ is compact, so by Remark \ref{compactness} then also $\widehat{K}^{sc}$ is compact. This grants us that the map
%	\begin{align*}
%		\R\times\R \owns (\xi,\eta) \mapsto dist^2\left((\xi,\eta),K\right),
%	\end{align*}
%	is lower semicontinuous, so from the Foundametal theorem of Young measure (Theorem \ref{foundamental}) we get
%	\begin{align*}
%		0=\lim_{i\to\infty}\int_\om\int_\om dist^2(v_{u_i},K)dxdy\ge\int_\om\int_\om\int_{\R}\int_{\R} dist^2\left((\xi,\eta),K\right)d\nu_x(\xi)\otimes d\nu_y(\eta)dxdy\ge0
%	\end{align*}
%	Which tells us that $\nu_x\otimes\nu_y$ has support in $\widehat{K}\subset\widehat{K}^{sc}$ for a.e. $(x,y)\in\Omega\times\Omega$. Appplying Lemma \ref{KZ2.7} to the function $\chi_{\widehat{K}^{sc}}$ then it follows that $(u'(x),u'(y))=([\nu_x],[\nu_y])\in\widehat{K}^{sc}$ for a.e. $(x,y)\in\Omega\times\Omega$ and so $u\in A_{\widehat{K}^{sc}}$. Now we have to prove the inverse inclusion. We combine Lemma \ref{KZ4.7} with Proposition \ref{KZ5.6} and Lemma \ref{KZ5.8} and we obtain
%	\begin{align*}
%		A_{\widehat{K}^{sc}}=A_{\bigcup_{P\in P_{\widehat{K}^{sc}}}P}=\bigcup_{P\in P_{\widehat{K}^{sc}}}A_P\subset\bigcup_{(a,b)\in\widehat{K}}A_{Q_{a,b}}\subset \bigcup_{(a,b)\in\widehat{K}}A_{\{a,b\}\times\{a,b\}}^\infty\subset         A_K^\infty.
%	\end{align*}
%\end{proof}
%
%Since $W$ is coercive, diagonal and symmetric, then $L_c(W)=\widehat{L_c(W)}$ is compact. Applying the previous Theorem and Remark \ref{identity} to $L_c(W)$ we conclude the proof. 
%
%FINO A QUI NON HO RIVISTO PERCH\'E NON VORREI USARE QUESTA PARTE




\begin{remark}\label{remvec}
It is worth to observe that the level convexity of $W:\mathbb R^{d\times n}\times \R^{d\times n}\to \mathbb R$ is sufficient to ensure the weak* lower semicontinuity of the functional $\esssup_{\om\times \om}W(\nabla u(x),\nabla u(y))$ in view of the same arguments exploited in the first part of the proof of Theorem \ref{cns}.  In view of \cite[Theorem 1.1]{KRZ} the reverse implication does not hold, even if one assumes $n=1$: we refer to Theorem \ref{cnsvec} below.
\end{remark}


Now we prove that the relaxed functional $J_{1d}^{rlx}$ in \eqref{Jrlx1d} is given by
\begin{align*}
	%\label{thm17thesis}
	J_{1d}^{rlx}(u)=\esssup_{ I\times I}V^{slc}(u'(x),u'(y)).
\end{align*}

\begin{proof}[Proof of Theorem \ref{relax}]
	
The proof follows by \cite[Proposition 7.5]{KZ} and Theorem \ref{cns}. We present some details for the reader's convenience. 
Let us define 
\begin{align*}
	J^{slc}(u):=\esssup_{I\times I}V^{slc}(u'(x),u'(y)).
\end{align*}
By definition of separately level convex envelope (see Definition \ref{sep_level_conv}) we have that 
\begin{align*}
	V\ge V^{slc},
\end{align*}
so, by \cite[Theorem 1.3]{KZ} for every $(u_k)_{k\in\N}\subset W^{1,\infty}(I)$ such that $u_k\overset{*}{\rightharpoonup}u$ in $W^{1,\infty}(I)$ we have that $u'_k \overset{\ast}{\rightharpoonup}u'$ in $L^\infty(I)$, hence
\begin{align*}%\label{lb}
	J^{slc}(u)\le\liminf_{k \to \infty}J^{slc}(u_k)\le\liminf_{k \to \infty}J_{1d}(u_k).
\end{align*}
To prove the other inequality, fix $u\in W^{1,\infty}(I)$. 

%we can define
%\begin{align*}
%	c:=J^{slc}(u)<+\infty,
%\end{align*}
%and we can choose a sequence $(c_k)_{k\in\N}\subset\R$ such that $c_k \searrow c$ as $k\to\infty$ and such that
%\begin{align*}
%	u\in B_{L_c{_k(V^{slc})}}=B_{L_c{_k(V)^{sc}}},
%\end{align*}
%where in the last equality we have exploited \cite[Lemma 7.4]{KZ}.

By \cite[Theorem 1.3]{KZ} %Proposition \ref{cns} 
there exists a sequence $(v_k)_{k\in\N}\subset L^{\infty}(I)$ such that $v_k\overset{*}{\rightharpoonup}u'$ in $L^{\infty}(I)$ as $k\to+\infty$ and
\begin{align}\label{almostrecovery}
\lim_{k \to +\infty}\esssup_{I\times I} V(v_k(x),v_k(y))= \esssup_{I\times I} V^{slc}(u'(x), u'(y)).
\end{align}
Again the possibility of defining  $u_k(x):=\int_a^x v_k(x)dx$, see Remark \ref{Amerio}, allows us to say that $(u_k)_{k\in \mathbb N} \subset W^{1,\infty}(I)$, and $u_k \overset{\ast}{\rightharpoonup} w$ in $W^{1,\infty}(I)$ with $\frac{d w}{dx}= u'$, a.e. in $I$, hence $w(x)= u (x)+ C$. This fact, together with \eqref{almostrecovery} guarantee that
$$
\lim_{k\to +\infty} \esssup_{I\times I}V(u_k'(x), u'_k(y))=\esssup_{I\times I}V^{slc}(u'(x), u'(y)),
$$ 
which ensures the existence of a recovery sequence, i.e. $(u_k)_{k \in \mathbb N} \subset W^{1,\infty}(I)$, weakly* converging to $u$ and that concludes the proof.
\end{proof}

In \cite{KRZ} the following notions have been introduced.
\begin{definition}\label{defvec}
	A set $E \subset \R^d \times \R^d$ is {\it Cartesian convex} if $A \times A \subset E$ implies $A^{co} \times A^{co} \subset  E$, where $A^{co}$ denotes the convex envelope of the set $A$.
	
	If $E$ is not Cartesian convex, the smallest Cartesian convex set containing $E$ is, by definition, its Cartesian convex hull and it is denoted by $E^{\times sc}$. 
	
	$E$ admits a basic Cartesian convexification if 
	$E^{\times xc}= \bigcup_{A \times A \in \mathcal P_E} A^{co}\times A^{co}$,
	where $\mathcal P_E$ represents the set of all the maximal Cartesian squares i.e. sets $B\times B$, contained in $E$ and with the property that if there exists a set $C$ such that $C\times C \subset E$ and $B\subset C$, then $B=C.$ 
	
	A function $f:\R^d\times\R^d \to \R \cup \{\infty\}$ is said to be 
	{\it Cartesian level convex} if for every $c \in \R$ the set $L_c(f)$ is Cartesian convex.
	
	
	When $f$ is not Cartesian level convex, the corresponding envelope, namely 
	$$f^{\times lc}(\xi, \eta) := \sup\{g (\xi, \eta) |\, f : \R^d \times \R^d \to \R \hbox{ is Cartesian level convex and }g \leq f\},
	$$ 
can be described as
	$$f^{\times lc}(\xi, \eta) = \inf\{c \in \mathbb R : (\xi, \eta) \in L_c(f)^{\times c}
	\}$$ for $(\xi, \eta) \in \R^d \times \R^d$.
\end{definition}



\begin{proof}[Proof of Theorem \ref{cnsvec}]
	The proof follows along the lines of Theorem \ref{cns}, considering vector valued functions in $W^{1,\infty}(I;\mathbb R^d)$, replacing \cite[Theorem 1.3(ii)]{KZ} by \cite[Theorem 1.1]{KRZ}, i.e. exploiting the necessary and sufficient condition given by the Cartesian level convexity of $W$ for nonlocal supremal functionals depending on $L^\infty(I;\mathbb R^d)$ fields, instead of the separate level convexity.   
\end{proof}

We also observe, as already mentioned in the introduction, that Theorem \ref{cns} is a corollary of Theorem \ref{cnsvec}, since Cartesian level convexity reduces to separate level convexity when $d=1$, as proven in \cite{KRZ}.


In terms of differential inclusions, given a compact set $K\subset\R^d\times\R^d$ we denote the $W^{1,\infty}-$weak$^*$ closure of $B_K$ in \eqref{AK} as $B^\infty_K$, i.e.
\begin{align}\label{Bkinfty}
	B^\infty_K:=\left\{ u\in W^{1,\infty}(I;\mathbb R^d)| \exists (u_j)_{j\in\N}\subset B_K : u_j \overset{*}{\rightharpoonup}u \in W^{1,\infty}(I;\mathbb R^d) \right\}.
\end{align}

Theorem \ref{cnsvec} states that $B^\infty_K= B_K$ if and only if $K$ is Cartesian convex.



The same arguments of Theorem \ref{relax} can be applied to describe the relaxation of the functional \ref{functionald} in the case when $W$ admits a {\it basic Cartesian convexification}

\begin{proof}[Proof of Theorem \ref{relaxvec}]
	The proof relies on the same arguments of Theorem \ref{relax}, exploiting \cite[Theorem 1.2]{KRZ} instead of \cite[Theorem 1.3]{KZ}. 
	\end{proof}



\begin{remark}
	\label{lastrem}
	It is worth to observe that when the diagonality assumption on the supremands $V$ and $W$ above is dropped the relaxation results hold in terms of the envelopes of $\widehat V$ and $\widehat W$, respectively, defined accordingly to \eqref{simdiag}.
\end{remark}

Clearly Theorem \ref{relaxvec} in terms of differential inclusions says that $B^\infty_K$  in \eqref{Bkinfty} coincides with $B_{K^{\times c}}$ if $K$ is symmetric, diagonal and admitting a basic Cartesian convexification.
 
We conclude by observing that all the above results could be rephrased in terms of the lower semicontinuity or relaxation of nonlocal indicator functionals of the type $$\iint_{I\times I}\chi_K (u'(x), u'(y)) dx dy,$$
where for $K \subset \R^d \times \R^d$, $\chi_K$ stands for its characteristic function defined as 
$	\chi_K(\xi,\eta):=
	\begin{cases}
		&0 \hbox{ if }(\xi, \eta)\in K,\\
		&\infty \hbox{ otherwise}.
	\end{cases} $
This leads to a characterization of the lower semicontinuity for the above unbounded nonlocal integrals in terms of the Cartesian convexity of $K$ and a characterization of its relaxation in terms of its Cartesian convex envelope, if and only if $K$ satisfies the assumptions of \cite[Theorem 1.2]{KRZ}. 
 
\bigskip


{\bf Acknowledgements}
The authors are members of INdAM-GNAMPA and they gratefully acknowledge the support of Progetto 'GNAMPA Prospettive nelle scienze dei materiali: modelli variazionali, analisi asintotica e omogeneizzazione'.
EZ thanks Dipartimento di Scienze Fisiche, Informatiche e Matematiche of University of Modena and Reggio Emilia for its support and kind hospitality.
%	
%	%DA QUI, 
%	%\color{red}
%	%IN THE FOLLOWING I AM ASSUMING THAT $W_0(v'(x_1), v'(y_1)) = W(v'(x_1), u_0(x_1), v'(y_1), u_0(y_1)$ for the same $u_0$ which in principle might not be true....maybe this follows by symmetry and diagonality of $W$ DOUBLE CHECK 
%	%
%	%
%	%a parte la densit\'a che nella migliore delle ipoteis va al contrario bisogna aggiustare.... non so se serva un'approssimazione $L^p$
%	Since
%	\begin{align*}
%		\lim_{p\to+\infty}\left(\iint_{I\times I} W((v'(x_1),u(x_1)), (v'(y_1),u(y_1)))^pdx_1dy_1\right)^{\frac{1}{p}}=\esssup_{(x_1,y_1)\in I\times I}W((v'(x_1),u(x_1)), (v'(y_1),u(y_1)),
%	\end{align*}
%	then fixed $\e>0$ there exists $\bar{p}$ such that for $p>\bar{p}$ it holds
%	\begin{align*}
%		\esssup_{(x_1,y_1)\in I\times I}W((v'(x_1),u(x_1)), (v'(y_1),u(y_1))<\left(\iint_{I\times I} W((v'(x_1),u(x_1)), (v'(y_1),u(y_1)))^pdx_1dy_1\right)^{\frac{1}{p}}+\e.
%	\end{align*}
%	If we define
%	\begin{align*}
%		J_p(v,u):=\left(\iint_{I\times I} W((v'(x_1),u(x_1)), (v'(y_1),u(y_1)))^pdx_1dy_1\right)^{\frac{1}{p}},
%	\end{align*}
%	then the previous inequality reads
%	\begin{align*}
%		\esssup_{(x_1,y_1)\in I\times I}W((v'(x_1),u(x_1)), (v'(y_1),u(y_1))<J_p(v,u)+\e,
%	\end{align*}
%	for every  $v\in W^{1,\infty}(I)$ and $u \in W^{1,\infty}_0(I)$,
%	and thus it follows that
%	\begin{align}
%		\label{preEkeland}
%		\inf_{u\in W_0^{1,\infty}(I)}\esssup_{(x_1,y_1)\in I\times I}W((v'(x_1),u(x_1)), (v'(y_1),u(y_1))<\inf_{u\in W_0^{1,\infty}(I)} J_p(v,u)+\e,
%	\end{align}
%	for every $p\geq \bar p$.
%	Now we observe that
%	\begin{align}\label{estLinfty}
%		\inf_{u\in W_0^{1,\infty}(I)} J_p(v,u)=\inf_{L^p(I)} J_p(v,u)\le \inf_{ L^\infty(I)} J_p(v,u).
%	\end{align}
%	The second inequality is trivial. To prove the first identity, since $W_0^{1,\infty}(I)\subset L^P(I)$, it remains to prove
%	\begin{align*}
%		\inf_{u\in W_0^{1,\infty}(I)} J_p(v,u)\le\inf_{L^p(I)} J_p(v,u).
%	\end{align*}
%	Recall that 
%	\begin{align*}
%		%&W_0^{1,\infty}(I):=\overline{C_0^\infty(I)}^{W^{1,\infty}},\\
%		&C_0^\infty(I) \text{ is dense in }L^p(I).
%	\end{align*}
%	Fixed $\e>0$ there exists $\eta_\e\in L^P(I)$ such that
%	\begin{align}\label{est11}
%		J_p(v,\eta_\e)\le\inf_{\eta \in L^p(I)}J_p(v,\eta)+\frac{\e}{2} <+\infty,
%	\end{align}
%	where in the latter inequality we have used \eqref{growthW} and the fact that $v \in W^{1,\infty}(I)$.
%	Correspondingly, there exists a sequence $(u_k^\e)_k\subset C_0^\infty(I)$ such that $u_k^\e\to\eta_{\e}$ in $L^p(I)$ as $k\to\infty$. From $(u_k^\e)_k$ we can extract a non relabeled subsequence $(u_k^\e)_k$ converging to $\eta_\e$ a.e. in $I$, and since $W$ is continuous on $\R^2\times\R^2$ then %there must exists $k(\e)\in\N$ such that 
%	\begin{align*}
%		W^p((v'(x),u_{k(\e)}^\e(x)),(v'(y),u_{k(\e)}^\e(y)))\to W^p((v'(x),\eta_\e(x)),(v'(y),\eta_\e(y)))%|\le\frac{\e}{2|I|}, 
%	\end{align*}
%	as $k \to +\infty$, a.e. in $I\times I$. By  the above convergence, \eqref{est11}, and Vitali-Lebesgue convergence theorem
%	$$J_p^p (v,u^\e_{k(\e)})\to J_p^p(v,\eta_\e)$$ as $k \to +\infty$. 
%	Hence, taking the $1/p$-th power, it results
%	\begin{align*}
%		\inf_{u \in W^{1,\infty}_0(I)}J_p(v,u)\le \inf_{u \in C^{\infty}_0(I)}J_p(v,u)\le \inf_{u \in L^p(I)}J_p(v,u)+\e.
%	\end{align*}
%	Since $\e$ is arbitrary, we get
%	
%	\begin{align*}
%		\inf_{u\in W_0^{1,\infty}(I)} J_p(v,u)\le\inf_{L^p(I)} J_p(v,u),
%	\end{align*}
%	hence
%	\begin{align*}
%		\inf_{u\in W_0^{1,\infty}(I)} J_p(v,u)=\inf_{L^p(I)} J_p(v,u).
%	\end{align*}
%	Arguing as in \cite[Proposition 7]{LeDretRaoult'95}, based on a measurable selection lemma (see \cite{Ekeland})  we have 
%	$$
%	\inf_{u\in W_0^{1,\infty}(I^2)} J_p(v,u)+\e \leq \\\left(\iint_{I\times I} W_0(v'(x_1), v'(y_1))^pdx_1 dy_1\right)^{\frac{1}{p}} +\varepsilon
%	$$
%	for every $p \geq \bar p$.
%	Thus, putting together \eqref{preEkeland} and the fact the $L^p$ norm approaches from below the $L^\infty$ norm, we have 
%	%together with the arguments in Remark \ref{uniqueness} we have
%	\begin{align*}
%		\inf_{u\in W_0^{1,\infty}(I^2)}\esssup_{(x_1,y_1)\in I\times I}W((v'(x_1),u(x_1)), (v'(y_1),u(y_1))<\inf_{u\in W_0^{1,\infty}(I^2)} J_p(v,u)+\e \leq \\\left(\iint_{I\times I} W_0(v'(x_1), v'(y_1))^pdx_1 dy_1\right)^{\frac{1}{p}} +\varepsilon\leq \\
%		\esssup_{I\times I} W_0(v'(x_1), v'(y_1)) +\varepsilon.
%	\end{align*}
%	From \eqref{GammalimsupestII}, and the arbitrariness of $\varepsilon$, it follows that 
%	\begin{align}\label{Gammalimsupineq}
%		(\Gamma-\limsup_{h\to 0}J_h)(v)\le\esssup_{I\times I} W_0(v'(x_1), v'(y_1)).
%	\end{align}
%	Hence the proof is concluded once we show that 
%	\begin{align*}
%		(\Gamma-\limsup_{h\to 0}J_h)(v)\le\esssup_{I\times I} (W_0)^{slc}(v'(x_1), v'(y_1)).
%	\end{align*}
%	
%	First, by applying Theorem \ref{relax}, the $W^{1,\infty}(I)-$weak* sequential lower semicontinuous envelope of
%	\begin{align*}
%		\esssup_{(t,s)\in I\times I}W_0(v'(t),v'(s))
%	\end{align*}
%	is
%	\begin{align*}
%		\esssup_{I\times I}(W_0)^{slc}(v'(t),v'(s))
%	\end{align*}
%	\color{vg}
%	Then, since the $(\Gamma-\limsup_{h\to 0}J_h)$ is sequentially weakly* $W^{1,\infty}(I)$-lower semicontinuous, it coincides with its $W^{1,\infty}(I)-$weak* sequential lower semicontinuous envelope.


%If we define 
%\begin{align*}
%	J(v)=
%	\begin{cases}
%		\esssup_{I^2\times I^2} W_0(\frac{\partial v(x_1,x_2)}{\partial x_1},\frac{\partial v(y_1,y_2)}{\partial y_1}) \text{ if }v\in W^{1,\infty}(I^2), \frac{\partial v}{\partial x_2}=0\\
%		+\infty \text{ elsewhere}.
%	\end{cases}
%\end{align*}
%It is clear that
%\begin{align*}
%	(\Gamma-\lim_{h\to 0}J_h)(v)\le J(v),
%\end{align*}
%and since $(\Gamma-\lim_{h\to 0}J_h)(v)$ is $W^{1,\infty}(I^2)$ weak * lower semicontinuous it follows that 
%\begin{align*}
%	(\Gamma-\lim_{h\to 0}J_h)(v)\le J^{rlx}(v),
%\end{align*}
%where $J^{rlx}$ is the lower semicontnuous elvelope of $J$ with respect to the $W^{1,\infty}(I^2)$ weak * convergence.\\
%
%\comment{Andrea: Questo è praticamente l'equivalente del Lemma 5 di Le Dret e Rault.}
%Now we observe that $W^{1,\infty}(I)$ is continuously embedded in $W^{1,\infty}(I^2)$ and in particular that it is isomorphic to $V:=\left\{v\in W^{1,\infty}(I^2), \frac{\partial v}{\partial x_2}=0 \right\}$. Fixed $v\in V$ we denote the function representing $v$ in $W^{1,\infty}(I)$ with $\bar{v}.$ What we want to show now is that 
%\begin{align*}
%	J^{rlx}(v)=
%	\begin{cases}
%		\esssup_{I^2\times I^2} (W_0)^{slc}(\bar{v}'(x_1),\bar{v}'(y_1)) \text{ if }v\in V\\
%		+\infty \text{ elsewhere}.
%	\end{cases}
%\end{align*}
%Since $J^{rlx}$ is the $\Gamma$-limit of $J$ with respect to the $W^{1,\infty}(I^2)$ weak * convergence, then fixed $v \in V$ there must exist a recovery sequence $(v_\e)_\e\subset V$ such that
%\begin{align*}
%	&v_\e \to v \text{ in } W^{1,\infty}(I^2) \text{weak *}\\
%	J^{rlx}(v)&=\lim_{\e\to 0}J(v_\e)=\lim_{\e\to 0}\esssup_{I^2\times I^2} W_0(\frac{\partial v_\e(x_1,x_2)}{\partial x_1},\frac{\partial v_\e(y_1,y_2)}{\partial y_1})\\
%	&=\lim_{\e \to 0}\esssup_{I\times I} W_0(\bar{v}_\e'(x),\bar{v}_\e'(y)).
%\end{align*}
%Since $v_\e \to v \text{ in } W^{1,\infty}(I^2)$ weak * it follows from the definition that $\bar{v}_e$ converges to $\bar{v}$ in  $W^{1,\infty}(I)$ weak *, so passing to the lower semicontinuous envelope it follows that
%\begin{align*}
%	J^{rlx}(v)&\ge W^{1,\infty}(I) \text{ weak * lower semicontinuous envelope of }\esssup_{I\times I} W_0(\bar{v}'(x),\bar{v}'(y))\\
%	&=\esssup_{I\times I} (W_0)^{slc}(\bar{v}'(x),\bar{v}'(y))
%\end{align*}
%Now we have to prove the inverse inequality.
%\comment{Andrea: Qui non so come procedere. Le Dret e Rault usano il fatto che il loro spazio è immerso con compattezza mentre il nostro è immerso solo con continuità (credo). Tu avevi parlato di utilizzare un procedimento diagonala ma non mi è ancora chiaro come. Continuerò a pensarci.}
%Finally we have to prove that if $v\notin V$ then $J^{rlx}(v)=+\infty$.
%\comment{Andrea: Questo dovrebbe essere abbastanza facile da far vedere seguendo Le Dret e Rault. Il problema con il loro metodo è che loro lavorano in spazi riflessivi e quindi per avere la compattezza gli basta la coercività, mentre noi abbiamo bisogno di ipotesi ulteriori}
%\end{proof}





%\begin{proposition}
%	\label{liminf}
%	Let $J$ be as in \eqref{functional} and suppose $W$ lower semicontinuous, coercive, and separately level convex. Then $J$ is $W^{1,\infty}-weakly^*$ lower semicontinuous.
%\end{proposition}
%
%\begin{proof}
%	Fix $(u_k)_{k \in \N}\subset W^{1,\infty}(\om)$ such that $u_k \rightharpoonup^*u$ in $ W^{1,\infty}$, for some $u \in  W^{1,\infty}(\om)$, and let $\nu=\{\nu_x\}_{x\in \om}$ be the Young measure generated by $(u'_k)_{k \in \N}$ for some non-relabeled subsequence of $(u'_k)_{k \in \N}$. Define $v_{u_k}(x,y):= (u'_k(x), u'_k(y))$, clearly $(v_{u_k})_{k \in \N}\subset L^\infty(\om \times \om, \R \times\R)$. For every $x,y \in \om$ define also $V=\{V_{(x,y)}\}_{(x,y)\in\om\times\om}:=\nu_x\otimes\nu_y$ the Young measure generated by Lemma \ref{P}. By Lemma \ref{KZ2.5} we get
%	\begin{align}
%		\label{linfpt1}
%		\liminf_{k \to \infty}J(u_k)=\liminf_{k \to \infty}\esssup_{(x,y)\in \om\times\om}W(u'_k(x),u'_k(y))\ge\esssup_{(x,y)\in \om\times\om} \bar{W}(x,y),
%	\end{align}
%with $\bar{W}(x,y):=V_{(x,y)}-\esssup_{(\xi,\eta)\in \R\times\R}W(\xi,\eta).$ By Lemma \ref{P} we have that for a.e. $(x,y)\in\om\times\om$
%\begin{align*}
%	\bar{W}(x,y)=\nu_x\otimes\nu_y-\esssup_{(\xi,\eta)\in \R\times\R}W(\xi,\eta)=\nu_x-\esssup_{\xi\in \R}(\nu_y-\esssup_{\eta\in \R}W(\xi,\eta));
%\end{align*}
%since $W$ is separately convex, by point $3)$ of Lemma \ref{KZ3.5} and by \eqref{barycenter} it holds
%\begin{align}
%	\label{linfpt2}
%	\bar{W}(x,y)\ge W([\nu_x], [\nu_y])=W(\nabla u(x),\nabla u(y)).
%\end{align}
%Combining \eqref{linfpt1} and \eqref{linfpt2} we obtain
%\begin{align*}
%	W(u(x),u(y))\le\liminf_{k \to \infty}W(u_k(x),u_k(y)),
%\end{align*}
%and this conclude the proof.
%\end{proof}



\color{black}

\begin{thebibliography}{99}
	%\bibitem{ABP} \textsc{ E. Acerbi, G. Buttazzo and D. Percivale:}
	
	
	\bibitem{ABPrin} \textsc{E. Acerbi, G. Buttazzo, F. Prinari: }\emph{The class of functionals which can be represented by a supremum.} J. Convex Anal. 9 (1) (2002) 225–236
	
	%\bibitem{AFP} \textsc{L. Ambrosio, N. Fusco and D. Pallara:} \emph{Functions of bounded variation and free discontinuity problems.} The Claredon Press, Oxford University Press, New York, 2000
	
	%\bibitem{Amerio} \textsc{L. Amerio:} \emph{Analisi matematica Vol.3 parte 2.} Unione Tipografica Editrice Torinese, UTET Torino, 1982
	
%	\bibitem{B} \textsc{E. N. Barron:} \emph{Viscosity solutions and analysis in $L^\infty$}. Nonlinear analysis, differential equations and control (Montreal, QC, 1998), volume 528 of NATO Sci. Ser. C Math. Phys. Sci., pages 1-60. Kluwer Acad. Publ., Dordrecht, 1999
	
	\bibitem{BL} \textsc{E.N. Barron, W. Liu: }\emph{Calculus of variations in $L^\infty$. }Appl. Math. Optim. 35 (3) (1997) 237–263.
	
	%\bibitem{BJW} \textsc{E. N. Barron, R. R. Jensen and C. Y. Wang:} \emph{Lower semicontinuity of $L^\infty$ functionals.} Ann. Inst. H. Poincaré Anal. Non Linéaire, 18(4):495-517, 2001
	
	\bibitem{BP}\textsc{J. Bevan and P. Pedregal: }\emph{A necessary and sufficient condition for the weak lower semicontinuity of one-dimensional non-local variational integrals}, Proc. Roy. Soc. Edinburgh Sect. A. {\bf 136}, n. 4, 701-708, 2006.
	
	\bibitem{BM-C}\textsc{J. C. Bellido, C. Mora-Corral: }\emph{Lower semicontinuity and relaxation via Young measures for nonlocal variational problems and applications to peridynamics.} SIAM J. Math. Anal. {\bf 50}, No. 1, pp. 779-809, 2019.
	
	\bibitem{BMCP}
	\textsc{J. C. Bellido, C. Mora-Corral,  P. Pedregal:}
	\emph{Hyperelasticity as a {$\Gamma$}-limit of peridynamics when the
			horizon goes to zero},
		{Calc. Var. Partial Differential Equations},
	{\bf 54}, n. 2, 1643--1670, 2015.
	
	%\bibitem{BDM1}
	%\textsc{A. Braides, G. Dal Maso:}
%\emph{Continuity of some non-local functionals with respect to a
%convergence of the underlying measures},
%J. Math. Pures Appl. (9), {\bf 170}, 136-149,
%2023.

\bibitem{BDM2}
\textsc{A. Braides, and G. Dal Maso:}
\emph{Compactness for a class of integral functionals with
interacting local and non-local terms.}
Calc. Var. Partial Differential Equations,
{\bf 62}, n. 5,  {Paper No. 148, 28 pp.}, 2023.
	
	
	\bibitem{BDM3}
	\textsc{A. Braides, and G. Dal Maso:}
	\emph{Validity and failure of the integral representation of $\Gamma$-limits of convex non-local functionals}, https://arxiv.org/abs/2305.04679.

\bibitem{CK}
\textsc{E. Clark, N. Katzourakis:}, 
\emph{On isosupremic vectorial minimisation problems in $L^\infty$ with general nonlinear constraints.} Adv. in Calc. Var., 2023.
\bibitem{CKS} 
\textsc{J. Cueto, C. Kreisbeck, H. Schönberger:}
\emph{A variational theory for integral functionals involving finite-horizon fractional gradients.}, https://arxiv.org/abs/2302.05569.


%	\bibitem{CP}\textsc{P. Cardaliaguet, F. Prinari: }\emph{Supremal representation of $L^\infty$ functionals.} Appl. Math. Optim. 52 (2) 129–141, 2005..
	
%	\bibitem{Dac}\textsc{B. Dacorogna:} {\emph Direct methods in the calculus of variations }. Second edition. Applied Mathematical Sciences, {\bf 78}. Springer, New York, 2008.

	\bibitem{EP}
	\textsc{M. Eleuteri, F. Prinari}
	\emph:{{$\Gamma$}-convergence for power-law functionals with variable
		exponents}, {Nonlinear Anal. Real World Appl.},
{\bf 58}, 	Paper No. 103221, 21 pp., 2021.

\bibitem{EDLL14}
		\textsc{A. Elmoataz, X. Desquesnes, Z.  
			Lakhdari, O. L\'{e}zoray:}
		\emph{Nonlocal infinity {L}aplacian equation on graphs with
			applications in image processing and machine learning},
		{Math. Comput. Simulation}, {\bf 102}, 	153--163, 2014.
	
	%\bibitem{DM} \textsc{G. Dal Maso:} {\emph An introduction to $\Gamma$-convergence}.....
	
	%\bibitem{Ekeland} \textsc{I. Ekeland, R. Temam: } \emph{Convex Analysis and Variational Problems}, Classics in Applied Mathematics, SIAM Philadelphia, {\bf 28} (1999).
	
	%\bibitem{FL} \textsc{I. Fonseca and G. Leoni:} \emph{Modern Methods in the calculus of variations: $L^p$ spaces.} Springer Monographs in Mathematics. Springer, New York, 2007
	
	%\bibitem{LeDretRaoult'95} \textsc{H. Le Dret and A. Raoult:}\emph{he nonlinear membrane model as variational limit of nonlinear three-dimensional elasticity}, J. Math. Pures Appl. (9), {\bf 74}, n. 6, 549-578, 1995.
	
	\bibitem{DFKS} \textsc{E. Davoli, R. Ferreira, C. Kreisbeck and
		H. Sch\"{o}nberger:} \emph{Structural changes in nonlocal denoising models arising
		through bi-level parameter learning},
	Appl. Math. Optim., {\bf 88}, n.1, Paper No. 9, 47pp., 2023.
	
	
	
	\bibitem{F} \textsc{F. G. Friedlander: }\emph{Introduction to the Theory of Distributions}, Cambridge Univ. Press., 157 pp. 1982
	
	\bibitem{GZ}\textsc{G. Gargiulo, E. Zappale:} \emph{A sufficient condition for the lower semicontinuity of nonlocal supremal functionals}, https://arxiv.org/abs/2305.14801, to appear in European Journal of Mathematics, 2023.
	
%	\bibitem{GP}\textsc{M.S. Gelli, F. Prinari:}
%	\emph{The role of intrinsic distances in the relaxation of
	%	$L^\infty$-functionals},
%	Nonlinear Anal., {\bf 204}, {Paper No. 112202, 30},
%	2021.
	
	\bibitem{KRZ} \textsc{C. Kreisbeck, A. Ritorto, E. Zappale: }\emph{Cartesian convexity as the key notion in the variational existence theory for nonlocal supremal functionals}, Nonlinear Anal. {\bf 225}, Paper No. 113111, 33, 2022.
	
	\bibitem{KS}
	\textsc{C. Kreisbeck, H. Sch\"{o}nberger: }
\emph{Quasiconvexity in the fractional calculus of variations:
characterization of lower semicontinuity and relaxation},
Nonlinear Anal. {\bf 215}, Paper No. 112625, 26 pp,  2022.

	
	\bibitem{KZdintegral} \textsc{C. Kreisbeck, E. Zappale: }\emph{Loss of double integral character during relaxation.} SIAM J. Math. Anal. Vol. 53, No. 1, pp. 351--385, 2021.
	
	\bibitem{KZ} \textsc{C. Kreisbeck, E. Zappale: }\emph{Lower semicontinuity and relaxation of nonlocal $L^\infty$ functionals.} Calc. Var. Partial Differential Equations, {\bf 59}, n. 4, Paper 138, 36, 2020. 
	
	\bibitem{K} \textsc{J. Kol\'ar: } \emph{Non-compact lamination convex hulls.} Ann. Inst. H. Poincaré Anal. Non Linéaire, 20(3):391-403, 2003
	
	\bibitem{MD}\textsc{T. Mengesha, Q. Du:}
		\emph{On the variational limit of a class of nonlocal functionals
			related to peridynamics},
		{Nonlinearity},
	{\bf 28}, n. 11, 3999--4035, 2015
	\bibitem{M} \textsc{J. Mu$\tilde{n}$oz:} \emph{Characterisation of the weak lower semicontinuity for a type of nonlocal integral functional: The n-dimensional scalar case.} J. Math. Anal. Appl. 360 (2009) 495–502
	
	%\bibitem{P} \textsc{P. Pedregal:} \emph{Nonlocal variational principles.} Nonlinear Anal., 29(12):1379-1392, 1997
	
	\bibitem{Pm}
		\textsc{P. Pedregal:}
		\emph{On non-locality in the calculus of variations},
		SeMA J., {\bf 78}, n. 4, 435-456, 2021.
	
	\bibitem{P} \textsc{F. Prinari:}\emph{Relaxation and $\Gamma$-convergence of supremal functionals.} Boll. Unione Mat. Ital. Sez. B Artic. Ric. Mat. (8) 9 (1), 101–132, 2006.
	
	\bibitem{PZ}
	\textsc{F. Prinari, E. Zappale:}
	\emph{A relaxation result in the vectorial setting and power law
		approximation for supremal functionals},
	{J. Optim. Theory Appl.},
	{\bf 186}, n. 2, 412-452,
	2020.
	\bibitem{RZ2} \textsc{A. M. Ribeiro, E. Zappale:} \emph{Revisited convexity notions for $L^\infty$ variational problems,} in preparation. 
	
	%\bibitem{R} \textsc{F. Rindler:} \emph{Calculus of variations.} Universitext. Springer, Cham, 2018
	
	%\bibitem{Rogers} \textsc{C. A. Rogers} \emph{Hausdorff measures.} Cambridge University Press, London-New York, 1970
		
	
\end{thebibliography}

\end{document}