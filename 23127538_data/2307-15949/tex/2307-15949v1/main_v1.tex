\documentclass[aps,pra,showpacs,twoside,10pt,floatfix,nofootinbib,longbibliography,twocolumn]{revtex4-1}
\usepackage[colorlinks=true, citecolor=red, urlcolor=blue ]{hyperref}
\usepackage{epsfig,newlfont,amssymb,amsfonts,amsmath,bm,subfigure,palatino,mathtools,amsthm,braket,soul,enumitem,color,times,comment,geometry,xcolor,float}
\usepackage[normalem]{ulem}
\geometry{lmargin=2cm,rmargin=2cm,tmargin=2cm,bmargin=2cm}
\definecolor{iitcolor}{HTML}{F7A600}
\definecolor{checkcolor}{HTML}{7541C0}
\bibliographystyle{apsrev4-1}
%\onecolumngrid
\newcommand{\AKP}[1]{{\color{red}#1}}
\newcommand{\JK}[1]{{\color{blue}#1}}
\newcommand{\HK}[1]{{\color{magenta}#1}}
\newcommand{\TOCHECK}[1]{{\color{checkcolor}#1}}
\newcommand{\propose}[1]{{\noindent\textbf{$\blacksquare$ Proposition #1.}}}
\newcommand{\corollary}[1]{{\noindent\textbf{$\diamond$ Corollary #1.}}}
\newcommand{\newpar}[1]{{\noindent$\circ$ \emph{#1.}}}

\begin{document}


\title{Entanglement in XYZ model on a spin-star system: Anisotropy vs. field-induced dynamics}
\author{Jithin G. Krishnan, Harikrishnan K. J., and Amit Kumar Pal}
\affiliation{Department of Physics, Indian Institute of Technology Palakkad, Palakkad 678 623, India}
\date{\today}


\begin{abstract}
We consider a star-network of $n=n_0+n_p$ spin-$\frac{1}{2}$ particles, where interaction between $n_0$ central spins and $n_p$ peripheral spins are of the XYZ-type.  In the limit $n_0/n_p\ll 1$,  we show that for odd $n$, the ground state is doubly degenerate, while for even $n$, the energy gap becomes negligible when $n$ is large, inducing an \emph{effective} double degeneracy. In the same limit, we show that for vanishing $xy$-anisotropy $\gamma$, bipartite entanglement on the peripheral spins computed using either a partial trace-based, or a measurement-based approach exhibits a logarithmic growth with $n_p$, where the sizes of the partitions are typically $\sim n_p/2$. This feature disappears for $\gamma\neq 0$, which we refer to as the \emph{anisotropy effect}. Interestingly, when the system is taken out of equilibrium by the introduction of a magnetic field of constant strength on all spins, the time-averaged bipartite entanglement on the periphery at the long-time limit exhibits a logarithmic growth with $n_p$ irrespective of the value of $\gamma$. We further study the $n_0/n_p\gg 1$ and $n_0/n_p\rightarrow 1$ limits of the model, and show that the behaviour of bipartite peripheral entanglement is qualitatively different from that of the $n_0/n_p\ll 1$ limit.    
\end{abstract}

\maketitle

\section{Introduction}
\label{sec:intro}

Entangled quantum states~\cite{horodecki2009,*guhne2009} have been shown to be a key resource for several quantum technological applications including quantum communication~\cite{bennett1992,*bennett1993,*mattle1996,*bouwmeester1997}, quantum cryptography~\cite{ekert1991,*jennewein2000,*Yin2020,*Schimpf2021}, quantum simulation~\cite{Dalmonte2018,*Kokail2021}, quantum metrology~\cite{Riedel2010,*Joo2011,*Demkowicz2014}, measurement-based quantum computation~\cite{raussendorf2001,*briegel2009}, and quantum algorithms~\cite{Bruss2011}. Being sources of entangled states, quantum many-body systems~\cite{amico2008,*Latorre2009,*dechiara2018} have been identified as candidate systems for implementing these applications. The last two decades have also witnessed the laboratory realization of quantum many-body Hamiltonians leading to quantum states possessing bipartite and multipartite entanglement using trapped ions~\cite{leibfried2003,*Porras2004,*Deng2005}, nuclear magnetic resonance systems~\cite{Vandersypen2005,*Rao2013}, ultra-cold atoms and optical lattices~\cite{Duan2003,*mandel2003,*bloch2005,*Treutlein2006,*Simon2011,*Cramer2013}, and solid-state systems~\cite{Schechter2008}. This has made testing the theoretically predicted entanglement properties of these systems possible. From this motivation, interface of quantum information theory and quantum many-body systems has emerged as a vibrant field of research advancing along two complementary directions.  On one hand, quantum technological applications have been implemented using quantum many-body systems, eg. quantum state transfer through spin chains~\cite{Bose2003,*Bose2013_chapter} and one-way quantum computation~\cite{raussendorf2001,*briegel2009} using cluster states~\cite{hein2006}.  On the other hand, quantum information-theoretic concepts have been extensively used to probe quantum many-body systems~\cite{amico2008,*Latorre2009,*dechiara2018}, leading to the introduction of projected entangled pair states~\cite{Schollwock2005,*Verstraete2008,*Schollowck2011,*Orus2014,*Bridgeman_2017}, and multiscale entanglement renormalization ansatz~\cite{Verstraete2004b,*Vidal2007,*Vidal2008,*Rizzi2008,*Aguado2008,*Cincio2008,*Evenbly2009}. 

% Figure environment removed


Among the variety of quantum many-body systems, interacting quantum spin models~\cite{amico2008,*Latorre2009,*dechiara2018} have attracted bulk of the attention due to their natural representation of multi-qubit or multi-qudit systems pertinent to quantum protocols. Although one- and two-dimensional regular lattices have primarily been in focus~\cite{amico2008,*Latorre2009,*dechiara2018}, achievement of control over the connection between any two qubits in an array realized in different substrates~\cite{Kane1998,*Makhlin1999,*Imam1999,*Zheng2000,*Cirac2000} have also given rise to  studies on quantum spin systems having a lattice structure modelled for specific purposes. Prime examples along this line of studies are (a) a star network of spins~\cite{Hutton2004}, where a number of  \emph{central spins} are surrounded by a number of  \emph{peripheral spins} such that each of the central spin interacts with all of the peripheral spins (see Fig.~\ref{fig:system} for examples with one and two central spin(s)), and (b) a star-chain network of spins, where a number of spin-chains have a common boundary spin~\cite{Yao2011,*Ping2013,*ZHU2018,*Grimaudo2022}. While such a network was originally envisioned for achieving control over entanglement through one, or a group of \emph{preferred} (central) spins~\cite{Hutton2004}, they have also been used as switch in quantum networks~\cite{Yung_2011}, in quantum state transfer and cloning~\cite{Deng_2008}, in magnetic resonance imaging~\cite{Sushkov2014}, in studying measurement uncertainty~\cite{Haddadi2021}, and to implement quantum heat engines~\cite{Turkpence_2017}, refrigerators~\cite{Arisoy2021}, and quantum batteries~\cite{Liu2021,*Peng2021}. The static and dynamical properties of entanglement~\cite{Hutton2004,Anza_2010,*Militello2011,*Ma2013,*He2019,*Karlova2023,Haddadi2019} in a star network of spins have also been extensively explored, along with the quantum correlations not belonging to the entanglement-separability paradigm~\cite{Radhakrishnan2019,Haddadi2019} (cf.~\cite{modi2012,*bera2017}).



In this paper, we consider a star-network of $n=n_0+n_p$ spins with $n_0$ central and $n_p$ peripheral spins, where each central (peripheral) spin interacts with each peripheral (central) spin via a fully anisotropic XYZ interaction~\cite{korepin_bogoliubov_izergin_1993,*Mila_2000,*Giamarchi2004,*Franchini2017}. We consider three limits of the model --  the \emph{large periphery limit} given by $n_0/n_p\ll 1$, the \emph{large center limit} given by $n_0/n_p\gg 1$,  and the \emph{competing center limit}, represented by $n_0/n_p\rightarrow 1$. In the large periphery limit, for non-zero values of the $xy$- and the $z$-anisotropy parameters, we investigate the structure of the ground state of the system. We show that for odd $n$, the ground states are doubly degenerate for all values of the anisotropy parameters.  However, for even $n$, there is a finite energy gap between the ground and the first excited states for low and moderately high $n$. This energy gap decreases as $\exp{(-b n_p^m)}$ for $\gamma=0$, and as $bn_p^{-m}$ for $\gamma\neq 0$ with increasing $n_p$, and the ground state becomes \emph{effectively} doubly degenerate with a negligible energy gap once a \emph{critical} size $n_p^c$ of the periphery (and hence, a critical size $n_c$ of the system) is achieved.
%\HK{(We clearly observed this only in the case $n_0<<n_p$, whereas in the $n_0=n_p$ limit upto the numerical limits we can check, we saw the energy gap between ground and first excited states is decreasing but saturating to $0.25$. So to our numerical limits it is inconclusive that effect degeneracy always happens for large even n)}. 



We next investigate the interplay of the \emph{bipartite entanglement} over the peripheral spins and the size of the partitions in the zero-temperature and large periphery limits of the system. We take two separate approaches for quantifying bipartite entanglement in the periphery. One is the partial trace-based approach, where the central spins are traced out of the state of the system, leading to a mixed state on the peripheral spins which can be used to compute a bipartite entanglement measure over a pre-fixed bipartition of the peripheral spins~\cite{horodecki2009,*guhne2009}. The other avenue is to \emph{localize}~\cite{divincenzo1998,verstraete2004,*popp2005,sadhukhan2017,*Krishnan2023,banerjee2020,*Banerjee2022} non-zero average bipartite entanglement on the peripheral spins  by performing a judiciously chosen projection measurement on the central spins. For both cases, we show that the bipartite entanglement over the equal bipartition of the peripheral spins exhibits a growth as $\log_2 n_p$ with the periphery-size when the system has no $xy$-anisotropy (cf.~\cite{Latorre2005,*Unanyan2005} for a similar finding in  the Lipkin-Meshkov-Glick (LMG) model), while for non-zero $xy$-anisotropy, this feature is absent. We refer to this as the \emph{anisotropy effect}. Moreover, for a fixed system-size $n$ with a partition-size $n^\prime$, both types of bipartite entanglement varies as $a\left[1-\exp\left(-b(n^\prime)^m\right)\right]$.

Next, we consider taking the $n_0/n_p\ll 1$ limit of the system out of equilibrium by turning on a magnetic field of constant strength on all spins. As the system evolves, we calculate the bipartite entanglement from both approaches as a function of time at different points in the parameter space of the $xy$- and $z$-anisotropy parameters. The time-averaged bipartite entanglement corresponding to both approaches in the long-time limit are shown to have a logarithmic growth with $n_p$ for all values of the $xy$-anisotropy parameter, implying a negation of the anisotropy effect due to the field-induced dynamics. Also, similar to the static scenario, both types of average bipartite entanglement varies as $a\left[1-\exp\left(-b(n^\prime)^m\right)\right]$ for a fixed $n_p$. We further consider the $n_0/n_p\gg1$ and $n_0/n_p\rightarrow 1$ limits of the spin-star system, and find that the behaviours of bipartite peripheral entanglement in these limits are qualitatively different from the same for the $n_0/n_p\ll 1$ limit. In the case of $n_0/n_p\gg 1$, the bipartite peripheral entanglement decreases monotonically with increasing $n_0$ as $\sim bn_0^{-m}$, and vanishes asymptotically, while $n_p$ is kept fixed. For $n_0/n_p\rightarrow 1$, the bipartite peripheral entanglement saturates with $n_p$. 

The rest of the paper is organized as follows. In Sec.~\ref{sec:def}, we introduce the XYZ model on the spin-star system, and provide brief definitions of the partial trace-based and measurement-based approaches for quantifying bipartite entanglement on the peripheral spins. In Sec.~\ref{subsec:diagonalization}, we discuss the diagonalization of the system-Hamiltonian in the $n_0/n_p\ll 1$ limit, and explain the degeneracy and effective degeneracy of the ground state of the system for odd and even system sizes in Sec.~\ref{subsec:single_central_spin_ground_state}. The static properties of peripheral bipartite entanglement in the large periphery limit of the system is discussed in Sec.~\ref{subsec:static_entanglement}, while the features of the dynamics of bipartite entanglement in the periphery due to the introduction of a magnetic field of constant magnitude is explored in Sec.~\ref{subsec:field_induced_dynamics}. The limits $n_0/n_p\gg 1$ and $n_0/n_p\rightarrow 1$ are discussed in   Sec.~\ref{subsec:large_center} and Sec.~\ref{sec:coupled_central} respectively. Sec.~\ref{sec:conclude} contains the concluding remarks and outlook.
  

\section{Definitions}
\label{sec:def}

In this section, we set notations for describing the system, and present the necessary details for partial trace-based and measurement-based entanglement computed in the star network of spins.

\subsection{Star network of spins }
\label{subsec:star_network}

We consider a star-network~\cite{Hutton2004,Yung_2011,Deng_2008,Sushkov2014,Haddadi2021,Turkpence_2017,Arisoy2021,Liu2021,*Peng2021,Anza_2010,*Militello2011,*Ma2013,*He2019,*Karlova2023,Radhakrishnan2019,Haddadi2019} of $n=n_0+n_p$ spin-$1/2$ particles, where $n_0$ \emph{central spins} are surrounded by $n_p$ \emph{peripheral spins} such that each of the central spin interacts with all of the peripheral spins, but there is no interaction among the central spins and the peripheral spins themselves.  Without any loss in generality, we label the central spins as $0,1,2,\cdots,n_0-1$, and the $n_p$ peripheral spins as $n_0,n_0+1,n_0+2,\cdots,n_0+n_p-1$. We consider three limits of the model, 
\begin{enumerate} 
\item[(a)] the \emph{large periphery limit}, $n_0/n_p\ll 1$, where the size of the periphery is far larger than the size of the center, 
\item[(b)] the \emph{large center limit}, $n_0/n_p\gg 1$, where the central spins outnumber the peripheral spins by a large number,  and 
\item[(c)] the \emph{competing center limit}, $n_0/n_p\rightarrow 1$, where the number of central spins is comparable to the number of peripheral spins.  
\end{enumerate}
See Fig. \ref{fig:system} for pictorial representations of these limits of the system. We assume each of the central spins $j$ interacting with all of the peripheral spins via the interaction Hamiltonian~\cite{korepin_bogoliubov_izergin_1993,*Mila_2000,*Giamarchi2004,*Franchini2017} 
\begin{eqnarray}
H_{s}&=& K\sum_{i=0}^{n_0-1}\sum_{j=n_0}^{n_0+n_p-1}\left[(1+\gamma)S^x_i S^x_{j}+(1-\gamma) S^y_i S^y_{j}\right]\nonumber\\ &&+\Delta K\sum_{i=0}^{n_0-1}\sum_{j=n_0}^{n_0+n_p-1}S^z_i S^z_{j},
\label{eq:star-ham}
\end{eqnarray} 
Here,  $i$ and $j$ are the spin indices corresponding to the central and the peripheral spins respectively, $S^{\alpha}_i=\sigma^{\alpha}_i/2$  is the standard representation of spin operators corresponding to the lattice site $i$, with $\sigma^{\alpha}$ being the Pauli matrices, $\alpha=x,y,z$,  and $K$ is the strength of the exchange interaction. The dimensionless parameters $\gamma$ and $\Delta$ respectively represent the $xy$- and the $z$-anisotropy corresponding to all pairs of spins, with $-1\leq\gamma\leq1$ and $-1\leq\Delta\leq1$. The Hamiltonian $H_s$ represents a number of paradigmatic quantum spin Hamiltonians on the star network for different values of $\gamma$ and $\Delta$, including the XY model ($0<|\gamma|<1$, $\Delta=0$), the XX model ($\gamma=0,\Delta=0$) the classical Ising model ($\gamma=\pm 1$, $\Delta=0$), the isotropic Heisenberg model $(\gamma=0,\Delta=1)$, and the Heisenberg model  with a $z$-anisotropy  ($\gamma=0$, $0<|\Delta|<1$). In this paper, we consider the most general XYZ model  ($0< |\gamma|< 1$, $0 < |\Delta| < 1$) on the spin-star system in all three limits. 



%In the star-ring configuration (Fig. \ref{fig:system}\textbf{(b)}), in addition to the star structure, the peripheral spins are situated on a one-dimensional (1d) lattice with periodic boundary condition (PBC) $n+1\equiv 1 $. The spins interact via nearest-neighbor (NN)  interaction described by the Hamiltonian  
%\begin{eqnarray}
%H_{r}&=& \sum_{i=1}^{n}K_{i}\left[(1+\gamma)S^x_i S^x_{i+1}+(1-\gamma) S^y_i S^y_{i+1}\right]\nonumber\\
%&&+\Delta\sum_{i=1}^{n}K_{i} S^z_i S^z_{i+1},
%\label{eq:ring-ham}
%\end{eqnarray}\normalsize
%where $K_i$ is the strength of the exchange interaction between the NN spins $(i,i+1)$, and $\gamma$ and $\Delta$ are respectively the $x-y$ and the $z$ anisotropy parameters corresponding to all pairs of spins on the ring. Similar to the star network, the peripheral ring can be ordered or disordered depending on whether $K_{i}=K$ for all $i$ or not.   The total \emph{star-ring} system is described by the Hamiltonian $H_{sr}=H_s+H_r$. 

%In this work, we specifically focus on the TXY model and the XXZ model in an external field on the star-ring network of spins. In the ordered scenario, we identify $h$ as the energy scale of the system, and work with the scaled Hamiltonians $H_s/h\rightarrow H_s(g_s)$ and $H_r/h\rightarrow H_r(g_r)$, such that the eigenvalues of $H_s$ and $H_r$ are dimensionless, where $g_s=K_s/h$ and $g_r=K_r/h$ are dimensionless parameters. The scaled Hamiltonian for the total system is $H_{sr}=H_{s}(g_s)+H_{r}(g_r)$.  

 


\subsection{Entanglement in the star network of spins}
\label{subsec:entanglement}



To quantify entanglement in subsystems of the star network of spins in the state $\rho$, we consider two separate approaches, namely, (a) the partial trace-based approach, and (b) the measurement-based approach. The descriptions of these approaches are given below. 

\paragraph{Partial trace-based approach.} Let us consider a bipartition $A:B$ of the system, where we aim to quantify entanglement in $B$. The size of the individual partitions $A$ and $B$ are respectively determined by the number of spins $N_A$ and $N_B$ in them. In the partial trace-based approach, the state $\rho_B$ on $B$ is determined by tracing out the spin degrees of freedom corresponding to all spins $a=1,2,\cdots,N_A$ in $A$, i.e., 
\begin{eqnarray}
\rho_B=\text{Tr}_A[\rho].
\end{eqnarray}
Using $\rho_B$, a suitable bipartite or a multipartite entanglement measure $E_B=E(\rho_B)$ is computed over the spins $b=1,2,\cdots,N_B$ in $B$~\cite{horodecki2009,*guhne2009}. In this paper, we consider the situation where $A$ is constituted only of the  central spins, thereby having size $N_A=n_0$, while the peripheral spins constitute the partition $B$ of size $N_B = n_p$. We specifically focus on bipartite entanglement measures on $B$, and denote a bipartition of $B$ as $B_1:B_2$, 
with partitions $B_1$ and $B_2$ being of the  sizes $n^\prime$ and $n_p-n^\prime$ respectively. For even $n_p$, $1\leq n^\prime \leq n_p/2$, while for odd $n_p$, $1\leq n^\prime\leq (n_p-1)/2$.  Without any loss in generality, we assume that the spins $b=1,2,\cdots,n^\prime$ $\in B_1$, and $b=n^\prime+1,n^\prime+2,\cdots,n_p$ $\in B_2$. For clarity, we denote $E_B\equiv E_{B_1:B_2}$.  

\paragraph{Measurement-based approach.} On the other hand, in the measurement based approach~\cite{divincenzo1998,verstraete2004,*popp2005,sadhukhan2017,*Krishnan2023,banerjee2020,*Banerjee2022}, independent projection measurements are performed on all spins in $A$, giving rise to an ensemble of post-measured states 
\begin{eqnarray}
    \varrho_{\mathbf{k}}=\frac{1}{p_{\mathbf{k}}}\left[(P_A^{\mathbf{k}}\otimes I_B)\rho(P_A^{\mathbf{k}}\otimes I_B)\right],
\end{eqnarray}
where each state $\varrho_{\mathbf{k}}$ has a probability of occurrence $p_{\mathbf{k}}=\text{Tr}\left[(P_A^{\mathbf{k}}\otimes I_B)\rho(P_A^{\mathbf{k}}\otimes I_B)\right]$, with $P_A^{\mathbf{k}}=\ket{\mathbf{k}}\bra{\mathbf{k}}$ being the projector corresponding to the measurement-basis $\ket{\mathbf{k}}$. For independent projection measurements on all spins in $A$, 
\begin{eqnarray}
P_A^{\mathbf{k}}=\otimes_{a\in A}\ket{\mathbf{k}_a}\bra{\mathbf{k}_a},
\end{eqnarray}
with $\mathbf{k}_a=\mathbf{0},\mathbf{1}$, where 
\begin{eqnarray}
    \ket{\mathbf{0}_a}&=&\cos\frac{\theta_a}{2}\ket{0_a}+\text{e}^{\text{i}\phi_a}\sin\frac{\theta_a}{2}\ket{1_a},\nonumber\\
    \ket{\mathbf{1}_a}&=&\sin\frac{\theta_a}{2}\ket{0_a}-\text{e}^{\text{i}\phi_a}\cos\frac{\theta_a}{2}\ket{1_a}, 
\end{eqnarray}
$\{\ket{0_a},\ket{1_a}\}$ being the computational basis. For each measurement outcome $\mathbf{k}$ on $A$, the post-measured state $\varrho_{\mathbf{k}}$ has the form 
\begin{eqnarray}
    \varrho_{\mathbf{k}}=P_A^{\mathbf{k}}\otimes\varrho_B^{\mathbf{k}},
\end{eqnarray}
so that an average entanglement over the ensemble of all possible post-measured states on $B$ can be defined as $\sum_{\mathbf{k}}p_{\mathbf{k}}E_B^{\mathbf{k}}$, with $E_B^{\mathbf{k}}=E(\varrho_B^{\mathbf{k}})$ is an entanglement measure, bipartite or multipartite, computed with the state $\varrho_B^{\mathbf{k}}$ on $B$. A maximization of $\langle E_B\rangle$ with respect to the $2N_A$ real parameters $\{\theta_a,\phi_a\}$ $\forall a\in A$ leads to the definition of \emph{localizable entanglement}~\cite{verstraete2004,*popp2005,sadhukhan2017,*Krishnan2023,banerjee2020,*Banerjee2022} over the subsystem $B$, given by 
\begin{eqnarray}
\langle E_B\rangle = \max_{\{\theta_a,\phi_a\}}\left[\sum_{\mathbf{k}}p_{\mathbf{k}}E_B^{\mathbf{k}}\right],
\label{eq:LE}
\end{eqnarray}
via single-spin projection measurements over all spins in $A$. Similar to the case of the partial trace-based approach, we focus on bipartition $B_1:B_2$ in $B$, and write $E_{B}^{\mathbf{k}}=E_{B_1:B_2}^{\mathbf{k}}$, and $\langle E_{B}\rangle=\langle E_{B_1:B_2}\rangle$. Note that the basis-independence of partial trace and the convexity property~\cite{horodecki2009,*guhne2009} of the chosen entanglement measure $E$ implies  
\begin{eqnarray}
    \langle E_{B_1:B_2}\rangle \geq E_{B_1:B_2}. 
\end{eqnarray}

To quantify bipartite entanglement over a partition $B_1:B_2$ in a state $\rho_B$ (or $\varrho_B^{\mathbf{k}}$) on $B$, we use \emph{logarithmic negativity}~\cite{plenio2005} as an entanglement measure, defined as 
\begin{eqnarray}
   \mathcal{L} &=& \log_2(\mathcal{N}+1),
\end{eqnarray}
where $\mathcal{N}$ is the \emph{negativity}~\cite{peres1996,*horodecki1996,*zyczkowski1998,*vidal2002,*lee2000} of $\rho_B$, defined as 
\begin{eqnarray} 
\mathcal{N}&=&||\rho_{B}^{T_{B_1}}||-1=2\left|\sum_{\lambda_i<0}\lambda_i\right|.
\end{eqnarray}
Here,  $\rho_{B}^{T_{B_1}}$ is the partially transposed density matrix $\rho_{B}$ with respect to the subsystem  $B_1$, $||.||$ denotes the trace norm, and $\lambda$ are the negative eigenvalues of $\rho_{B}^{T_{B_1}}$. 

%\HK{PPT is necessary and sufficient for separability of Dicke diagonal states\cite{Yu2016}.}


%These choices are as follows.
%\begin{enumerate}
%    \item[(a)] 
%    \item[(b)] A bipartition $A:B$ where $A$ includes $n-1$ peripheral spins, and $B$ includes a pair of spins including the central spin, and any one of the spins from the periphery. See Fig.~\ref{fig:partition}(b)\HK{((b) is not possible even numerically from this idea of permutation symmetry among peripheries.)}. 
%    \item[(c)] The third scenario, depicted in Fig.~\ref{fig:partition}(c), corresponds to the case where the partition $A$  is made of $n-2$ peripheral spins and the central spin, while the partition $B$ is made of a pair of NN spins on the peripheral ring.  Note that in the absence of interactions between the spins in the periphery, $B$ can be any pair of peripheral spins. 
%\end{enumerate}
 

\section{XYZ model on a star network in the large periphery limit}
\label{sec:xyz_model_single_central}

 



%We start with the ordered case $(K_i=K\neq 0, h_{i}=h\sqrt{S_0J_r}\neq 0\;\forall i)$, and 

We now consider the XYZ model on a star network of spins in the large periphery limit. We choose the case of a single central spin ($n_0=1$), and increase $n_p$ in order to attain this limit. We identify $|K|$ as the natural energy scale of the system, and write the dimensionless spin-star Hamiltonian   
$H_s/|K|\rightarrow H_s$  as 
\begin{eqnarray}
\label{eq:star-xyz_ordered} 
H_s&=& \pm\frac{1}{2}\left\{S_0^+(J_p^- + \gamma J_p^+) + S_0^-(J_p^+ + \gamma J_p^-)\right\} \nonumber\\&& + \Delta S_{0}^{z} J_{p}^{z}, 
\end{eqnarray}
where  
$J_p^\alpha=\sum_{i=1}^{n_p}S^\alpha_{i}$, $\alpha=x,y,z$, are defined on the Hilbert space of the peripheral spins, and the signs of $H_s$ is determined by whether the spin-spin interaction is ferromagnetic (FM, $K>0$) or antiferromagnetic (AFM, $K<0$).  
Note that the operators $J_p^{\alpha}$, $\alpha=x,y,z$, obeys the usual commutation and anticommutation relation of angular momentum operators, implying that the peripheral spins collectively behave as a spin-$n_p/2$ particle with the spin operator $\mathbf{J}_p=(J_p^x,J_p^y,J_p^z)$, while the total spin operator for the star network is given by $\mathbf{S}_0+\mathbf{J}_p$, with $\mathbf{S}_0=(S^x_0,S^y_0,S^z_0)$. We also define $S^\pm_i=S^x_i\pm\text{i}S^y_i$ for the $i$th spin, and subsequently $J^\pm_p=\sum_{i=1}^{n_p} S^\pm_i$. 


% Figure environment removed


\subsection{Diagonalization}
\label{subsec:diagonalization}


Since $[H_s,\mathbf{J}_p^2]=[H_s,\mathbf{S}_0^2]=0$, $H_s$ is block diagonal in the basis $\ket{b}=\ket{S_0,m_0}\otimes\ket{J_p,m_p}$, where
\begin{eqnarray}
\mathbf{S}_0^2\ket{S_0,m_0}&=&S_0 (S_0+1)\ket{S_0,m_0},\nonumber\\
\mathbf{J}_p^2\ket{J_p,m_p}&=&J_p (J_p+1)\ket{J_p,m_p}.
\end{eqnarray} 
Noticing that  
\begin{enumerate}
\item[(a)] $S_0$ only has one allowed value $S_0=\frac{1}{2}$ implying $m_0=\pm 1/2$, and
\item[(b)] different blocks of $H_s$ corresponding to a fixed value of $m_0$ can be labelled by specific values of $J_p$, where $m_p$ in each block can take $2J_p+1$ values, $-J_p\leq m_p\leq J_p$, 
\end{enumerate} 
for each block, it is sufficient to represent the basis as $\ket{b}=\ket{m_0,m_p}=\ket{m_0}\otimes\ket{m_p}$. Evidently, the ground state of $H_s$ belongs to one of these blocks with a specific value of $J_p$. Our numerical analysis suggests that irrespective of the system size $n$, the ground state $\ket{\Psi_0}$  always belongs to the sector with $J_p=n_p/2$ (see, for example,~\cite{Latorre2005,*Unanyan2005} for a similar property in the LMG model), while the same is true for the first excited state $\ket{\Psi_1}$ for $n_p\geq 3$ (see Appendix~\ref{app:small_system} for a demonstration with $n_p=2,3$).  


Let us now set $\ket{m_i=\pm 1/2}$ basis as the computational basis $\{\ket{0}\equiv\ket{1/2},\ket{1}\equiv\ket{-1/2}\}$ for the spin $i$, such that 
\begin{eqnarray}
    S^z_i\ket{+1/2}&=&(+1/2)\ket{+1/2},\nonumber\\ 
    S^z_i\ket{-1/2}&=&(-1/2)\ket{-1/2}.
\end{eqnarray}
Using this, we write the permutation-invariant Dicke state~\cite{sadhukhan2017,*Krishnan2023,dicke1954,*bergmann2013,*lucke2014,*kumar2017} on the $n_p$  peripheral spins with $n_p-l$ excitations:
\begin{eqnarray}
    \ket{D^{n_p}_l} &=& \frac{1}{\sqrt{\genfrac(){0pt}{1}{n_p}{l}}}\sum_i\mathcal{P}_i\left(\ket{1}^{\otimes l}\ket{0}^{\otimes n_p-l}\right)
\end{eqnarray}
with the summation being over all possible permutations $\mathcal{P}_i$ over $n_p$-spin product states, where $l$ spins are in the ground state $\ket{1}$, and the rest $n_p-l$ spins are in the excited state $\ket{0}$, $0\leq l\leq n_p$. Noting that 
\begin{eqnarray}
    J_p^z\ket{D_l^{n_p}}&=&\lambda_l\ket{D_l^{n_p}}, \\ J_p^2\ket{D_l^{n_p}}&=&\lambda\ket{D_l^{n_p}},\\
    J_p^\pm\ket{D_l^{n_p}}&=&\Big[\lambda-\lambda_l\left(\lambda_l \pm 1 \right)\Big]^{\frac{1}{2}}\ket{D_{l\mp 1}^{n_p}},
\end{eqnarray}
where $\lambda_l=\frac{n_p}{2}-l$ and $\lambda=\frac{n_p}{2}\left(\frac{n_p}{2}+1\right)$, we further identify
\begin{eqnarray}
\ket{D_l^{n_p}}\equiv\left|J_p=\frac{n_p}{2},m_p=\frac{n_p}{2}-l\right\rangle;\quad 0\leq l\leq n_p.
\end{eqnarray}
Therefore, the  $J_p=n_p/2$  block of $H_s$, which we denote by $\mathcal{H}_{n_p/2}$, is a $2(n_p+1)$-dimensional subspace spanned by the permutation-invariant $n_p$ qubit Dicke states, with
\begin{eqnarray}
    \ket{m_0,m_p}\equiv\ket{m_0}\otimes\ket{D^{n_p}_l};\quad 0\leq l\leq n_p,
\end{eqnarray}
and $\mathcal{H}_{n_p/2}$ takes the form
\begin{eqnarray}
    \mathcal{H}_{n_p/2}=\begin{bmatrix}
    B & A \\ A^T & B^\prime
    \end{bmatrix},
    \label{eq:n_2_block}
\end{eqnarray}
with the matrix elements for the $n_p+1$ dimensional matrices $A$ and $B$ as  
\begin{eqnarray}\label{eq:A}
A_{i,j} &=& \left[\lambda- \lambda_j(\lambda_j-1)\right]^{\frac{1}{2}} \delta_{i-1,j}/2\nonumber\\&&  + \gamma \left[\lambda- \lambda_j(\lambda_j+1)\right]^{\frac{1}{2}} \delta_{i+1,j}/2, \\
\label{eq:B}
B_{i,j}&=& \Delta\lambda_i\delta_{i,j}/2 = -B_{i,j}^\prime,
\end{eqnarray}
where $i,j\in [0,n_p]$. Due to the linear increase in dimension with increasing $n_p$, $\mathcal{H}_{n_p/2}$ can be numerically diagonalized to obtain the ground state of the model even for large $n_p$.

We further re-arrange the basis corresponding to the subspace hosting $\mathcal{H}_{n_p/2}$ such that $\mathcal{H}_{n_p/2}$ takes the form 
\begin{eqnarray}
    \mathcal{H}_{n_p/2} &=&
    \begin{bmatrix}
       A_1 & 0 \\ 0 & A_2
    \end{bmatrix},
    \label{eq:block_diagonal_forms}
\end{eqnarray}
where each of $A_1,A_2$ are $(n_p+1)$-dimensional matrices, and the basis has been grouped as 
\begin{eqnarray}
    A_1 &:& \{\ket{+1/2}\ket{-n_p/2+2k},\ket{-1/2}\ket{-n_p/2+2k+1}\},\nonumber\\
    A_2 &:& \{\ket{-1/2}\ket{-n_p/2+2k+1},\ket{+1/2}\ket{-n_p/2+2k}\},\nonumber\\   
    \label{eq:grouping}
\end{eqnarray}
with $0\leq k\leq n_p/2$ for even $n_p$, and $0\leq k\leq (n_p-1)/2$ when $n_p$ is odd. The utility of this re-arrangement will be clear in subsequent discussions. 


\subsection{On ground state degeneracy}
\label{subsec:single_central_spin_ground_state}


Let us now define 
\begin{eqnarray}
    \mathcal{O}=S_0^x\bigotimes_{i=1}^{n_p}S_i^x, 
\end{eqnarray} 
where $[H_s,\mathcal{O}]=0$. We first focus on the case of even $n_p$, and notice that $\mathcal{O}$ connects the basis elements corresponding to $A_1$ and that corresponding to $A_2$. Let us denote the eigenstate corresponding to the ground state energy $\mathcal{E}_0$ coming from $A_1$ is $\ket{\psi_0}_{A_1}$, and the same for $A_2$ is $\ket{\psi_0}_{A_2}$, while $\ket{\psi_0}_{A_1}=\mathcal{O}\ket{\psi_0}_{A_2}$, and vice-versa. The doubly-degenerate (DD) ground state of the system in the common eigenspace of $\mathcal{O}$ and $H_s$ is given by (see Appendix~\ref{app:small_system} for an example) 
\begin{eqnarray}
    \ket{\Psi_0^{\pm}}=\frac{1}{\sqrt{2}}\left(\ket{\psi_0}_{A_1}\pm\ket{\psi_0}_{A_2}\right). 
    \label{eq:degenerate_ground_states}
\end{eqnarray}
The double degeneracy in the ground state of $H_s$ is found for all the allowed values of $0\leq|\gamma|\leq 1$, and $0\leq |\Delta|< 1$ when $n_p$ is even (i.e., when $n=n_p+1$ is odd). 






On the other hand, in the case of odd $n_p$ (i.e., for even $n=n_p+1$), application  of $\mathcal{O}$ on the eigenstates of $A_1$ ($A_2$) does not take the state out of the subspace of $A_1$ ($A_2$). Apart from the lines $|\Delta|=1$ in the $(\gamma,\Delta)$ parameter space where the ground state is DD (see Appendix~\ref{app:small_system} for examples), at all other allowed values of the anisotropy parameters ($0\leq|\gamma|< 1$\footnote{At $\gamma=1$, ground states in the cases of both odd and even $n_p$ are doubly degenerate.} and $0\leq|\Delta|<1$), the ground state in non-degenerate (ND). However, the energy gap $\delta \mathcal{E}=\mathcal{E}_1-\mathcal{E}_0$ of the system, $\mathcal{E}_1$ $(\mathcal{E}_0)$ being the energy corresponding to the first excited (ground) state, decreases with increasing $n_p$ as (see Fig.~\ref{fig:energy_gap})
\begin{eqnarray}
    \delta \mathcal{E}&\sim& \left\{\begin{array}{cc}
    a+bn_p^{-m} & \text{for }\gamma=0 \\
    a+\exp\left(-bn_p^{m}\right) & \text{for }\gamma\neq 0
    \end{array}\right.
    \label{eq:double_exponential}
\end{eqnarray}
where the parameters $a$, $b$ and $m$ can be obtained by fitting numerical data, as shown in Table~\ref{tab:fitting_parameters} (see Appendix~\ref{app:fitting_params}). For $\gamma=0$, $\delta\mathcal{E}$ decreases slowly, and \emph{vanishes}\footnote{We assume $\delta\mathcal{E}=0$ when $\delta\mathcal{E}<10^{-4}$. This, indeed, depends on one's precision of numerical simulation.}  when the periphery-size $n_p$ is beyond a critical value $n_p^c\geq 5\times 10^3$ (i.e., when the system size is beyond a critical size $n_c=1+n_p^c$). This critical periphery-size, $n_p^c$, is a function of the chosen system parameters $(\gamma,\Delta)$.  The collapse of $\delta \mathcal{E}\rightarrow 0$ becomes more  rapid as the value of $\gamma$ increases, and is achieved for $n_p^c\geq 13$, when $\gamma\geq 10^{-1}$, while the effect of $\Delta$ on $n_p^c$ for a fixed $\gamma$ is nominal (see Fig.~\ref{fig:nc}). For $n_p\geq n_p^c$ with odd $n_p$, the ground states of the system, denoted by $\ket{\Psi_0^{\pm}}$ for consistency with energy eigenvalues $\mathcal{E}_1$ and $\mathcal{E}_0$ respectively, can be considered to be \emph{effectively}  DD (EDD).


% Figure environment removed




Noticing the invariance of $H_s$ under permutation of spins on the periphery irrespective of the value of $n_p$, the general form of the ground state $\ket{\Psi_0}$ (degenerate, or non-degenerate), obtained from diagonalizing $\mathcal{H}_{n_p/2}$, can be written as 
\begin{eqnarray}
    \ket{\Psi_0}&=&\sum_{l=0}^{n_p} \frac{c_{l}}{\sqrt{2}}\ket{\psi_0}\otimes\ket{D^{n_p}_l}+\sum_{l^\prime=0}^{n_p} \frac{d_{l^\prime}}{\sqrt{2}}\ket{\psi_0^\perp}\otimes\ket{D^{n_p}_{l^\prime}},\nonumber\\
    \label{eq:ground_state_form}
\end{eqnarray}
where $\sum_{l=0}^{n_p}|c_l|^2=\sum_{l^\prime=0}^{n_p}|d_{l^\prime}|^2=1$, and $\{\ket{\psi_0},\ket{\psi_0^\perp}\}$ constitute a complete orthonormal basis in the Hilbert space of the central spin. The coefficients $\{c_l\}$ and $\{d_{l^\prime}\}$ can be obtained by diagonalizing $\mathcal{H}_{n_p/2}$. If the ground state is doubly degenerate,  one works with the \emph{thermal ground state} (TGS) of $H_s$~\cite{osborne2002}, given by an equal mixture of the degenerate ground states $\ket{\Psi_0^{\pm}}$ as 
\begin{eqnarray}
    \rho_0=\frac{1}{2}\left(\ket{\Psi_0^+}\bra{\Psi_0^+}+\ket{\Psi_0^-}\bra{\Psi_0^-}\right). \label{eq:thermal_ground_state}
\end{eqnarray}
In the case of EDD ground states occurring for large odd $n_p$, one can also work with $\rho_0$, and consider it as the \emph{effective} TGS (ETGS), as there is negligible difference between $\mathcal{E}_0$ and $\mathcal{E}_1$. 


\subsection{Static entanglement properties: Anisotropy effect}
\label{subsec:static_entanglement}

In this paper, we are interested in the partial trace-based and measurement-based quantification of entanglement in $\ket{\Psi_0}$ (for ND ground state), or $\rho_0$ (for DD or EDD ground state). We note that tracing out the central spin from either of these two states leads to a state of the form 
\begin{eqnarray}
    \rho_p&=&\sum_{l,l^\prime=0}^{n_p}c_{l,l^\prime}\ket{D^{n_p}_l}\bra{D^{n_p}_{l^\prime}},
    \label{eq:mixed_state_dicke_basis}
\end{eqnarray}
for which prescription for computing entanglement $E_{B_1:B_2}$, quantified by the negativity or the logarithmic negativity, over arbitrary bipartition $B_1:B_2$ exists~\cite{Stockton2003}. On the other hand, a projection measurement on the central spin in the basis $\{\ket{0},\ket{1}\}$ results in a post-measured ensemble $\{\rho_p^1,\rho_p^2\}$, with $\rho_p^{1,2}$ both being of the form (\ref{eq:mixed_state_dicke_basis}), such that  $\langle E_{B_1:B_2}\rangle$ can be computed. We point out here  that the projection measurement on the central spin in the basis $\{\ket{0},\ket{1}\}$ may not be optimal, and would only provide a lower bound corresponding to the actual localizable entanglement (see Eq.~(\ref{eq:LE})). However, our numerical analysis for small systems indicate that for $n_0\ll n_p$, $\{\ket{0},\ket{1}\}$ is the optimal basis.  




\paragraph{Peripheral entanglement of EDD states.} Note that in the case of even $n_p$, the degenerate ground states $\ket{\Psi_0^\pm}$ are connected by the local unitary operator $\mathcal{O}$, and therefore have identical entanglement properties. However, this is not the case for odd $n_p$, and an interesting question would be how the two EDD states in the case of odd $n_p$  differ from each other in terms of values of $E_{B_1:B_2}$ and $\langle E_{B_1:B_2}\rangle$ where $n^\prime$ is typically fixed at $n^\prime=n_p/2$ for even $n_p$ ($n^\prime=(n_p-1)/2$ for odd $n_p$), when $n_p$ is increased. In Figs.~\ref{fig:energy_gap}(b), (c), we plot $\delta E_{B_1:B_2}=\left|E_{B_1:B_2}^+-E_{B_1:B_2}^-\right|$ and  $\delta \langle E_{B_1:B_2}\rangle=\left|\langle E_{B_1:B_2}^+\rangle-\langle E_{B_1:B_2}^-\rangle\right|$ as functions of $n_p$ for different values of $\gamma$ and  $\Delta$, where the $\pm$ in the superscripts in the expressions for $\delta E_{B_1:B_2}$ and $\delta \langle E_{B_1:B_2}\rangle$ denote the states $\ket{\Psi_0^\pm}$   (see discussions below Eq.~(\ref{eq:double_exponential})) for which entanglement is calculated. We find that $\delta E_{B_1:B_2}$ (and also $\delta\langle E_{B_1:B_2}\rangle$) approach zero as 
\begin{eqnarray}
\label{eq:entanglement gap fitt}
    \delta E_{B_1:B_2} &\sim& a+ \exp\left(-b n_p^{m}\right),
\end{eqnarray}
similar to the $\gamma\neq 0$ case of
Eq.~(\ref{eq:double_exponential}) for the energy gap. 


% Figure environment removed
 


\paragraph{Peripheral entanglement against system-size.} We now discuss the features of $E_{B_1:B_2}$ and $\langle E_{B_1:B_2}\rangle$ in the case of the XYZ model in the star network of spins with $n_0=1$. For demonstration, we choose $n^\prime=n_p/2$ (for even $n_p$), and $(n_p-1)/2$ (for odd $n_p$). Fig.~\ref{fig:zero_field_dynamic}(a)-(b) depicts the variations of $E_{B_1:B_2}$ and $\langle E_{B_1:B_2}\rangle$ as functions of $n_p$ for different values of $\gamma$ and $\Delta$. The qualitative features of the partial trace-based and the measurement-based entanglement are similar, as indicated from Fig.~\ref{fig:zero_field_dynamic}. It is important to note that on the $(\gamma,\Delta)$ plane for odd $n_p$, the onset of $\delta\mathcal{E}=0$ (i.e., $n_p^c$) is different for different points. This is indicated by the discontinuities in the variations of entanglement for odd $n_p$, computed from the ND ground state $\ket{\Psi_0}$ (Eq.~(\ref{eq:ground_state_form})) for $n_p \leq n_p^c$,  and from the mixed ETGS $\rho_0$ (Eq.~(\ref{eq:thermal_ground_state})) for $n_p> n_p^c$. For $\gamma=0$, $E_{B_1:B_2}$ (and also $\langle E_{B_1:B_2}\rangle$) exhibit a logarithmic dependence on $n_{p}$ as 
\begin{eqnarray}
    E_{B_1:B_2}&\sim& a+b \log_2 n_p,
    \label{eq:entanglement_fit_n}
\end{eqnarray}
where the fitting parameters $a$ and $b$ can be obtained by fitting the data for $E_{B_1:B_2}$ and $\langle E_{B_1:B_2}\rangle$ against $n_p$ (see Fig.~\ref{fig:zero_field_dynamic}(a)-(b) for the example of $\Delta=0$) when $n_p>n_p^c$. Also our numerical analysis suggest that for $\gamma=0$, $a$ and $b$ depends on the choice of $\Delta$ weakly for even $n_p$, and are independent of $\Delta$ for odd $n_p$.  However, in stark contrast, the logarithmic dependence is absent for $\gamma\neq 0$. This result indicate a prominent change in the variations of $E_{B_1:B_2}$ and  $\langle E_{B_1:B_2}\rangle$ with periphery-size for vanishing and non-vanishing anisotropy parameter $\gamma$. We refer to this as the \emph{anisotropy effect}. We point out here that this feature is unchanged for all constant ratios $n^\prime/n_p$
, where for the logarithmic dependence on $n_p$ in case of  $\gamma=0$, only values of $a$ and $b$ change.


For a fixed periphery-size $n_p$ (and subsequently for a fixed system-size $n$) and a varying partition size $n^\prime$ where the maximum value of $n^\prime$ is typically $n_p/2$,  for all $\gamma$, $E_{B_1:B_2}$ (and $\langle E_{B_1:B_2}\rangle$) vary as 
\begin{eqnarray}
    E_{B_1:B_2} &\sim& a\left[1-\exp\left(-b(n^\prime)^m\right)\right],
    \label{eq:entanglement_fit_nprime}
\end{eqnarray}
which is demonstrated in Figs.~\ref{fig:zero_field_dynamic}(c)-(d) for both types of entanglement. Note that for  $\gamma=0$, the fitting parameters $a, b$ and $m$ remains unaffected up to three decimal places for $\langle E_{B_1:B_2}\rangle$ when $0\leq \Delta< 1$. For $E_{B_1:B_2}$, fitting parameter $a$ has slight dependence on $\Delta$ which changes even the first decimal place. Whereas fitting parameters $b$ and $c$ remain unchanged up to two and three decimal places respectively. On the other hand, for $\gamma \neq 0$, all the fitting parameters depends on $\Delta$.

   
% Figure environment removed



\subsection{Field-induced dynamics of entanglement}
\label{subsec:field_induced_dynamics}

We now consider a situation where a time-dependent local magnetic field of strength $f(t)$ is applied to all spins in the star network. The field strength is such that 
\begin{eqnarray}
    f(t)=\left\{
    \begin{array}{cc}
    0; & t=0\\
    h; & t>0
    \end{array}
    \right.,
\end{eqnarray}
where $h>0$ (see~\cite{Barouch1970,*Barouch1971,*Barouch1971a,Chanda2016} for use of similar magnetic field). The system at $t>0$ therefore evolves in time due to the time-independent Hamiltonian
\begin{eqnarray} 
H_{\text{tot}}=H_s+h\sum_{i=0}^{n_0+n_p-1}S_i^z, 
\label{eq:total_Hamiltonian}
\end{eqnarray} 
where we redefine $h\rightarrow h/|K|$ to keep $H_{\text{tot}}$ dimensionless. Note that the features of block-diagonalization of $H$ discussed in Sec.~\ref{subsec:diagonalization} remain unchanged for $H_{\text{tot}}$ also,  irrespective of the value of $h$. %\AKP{In this case, the grouping of the  basis elements corresponding to Eq.~(\ref{eq:grouping}) is obtained by replacing $n_p/2$  with the corresponding value of $J_p$ in Eq.~(\ref{eq:grouping}), and in the ranges for~$k$.}

%\HK{(Again this is not true. Throughout the paper $J_p=n_p/2$ is the only sector we are looking for states, whether static or dynamic. Addition of a magnetic field results in change of matrix elements $B,B^\prime$ in Eq.~\ref{eq:n_2_block}). Which is equivalent to adding the $2(n_p+1) \times 2(n_p+1)$ matrix, $h(S_0^z\otimes \mathbf{I_{n_p+1}}+\mathbf{I_{2}}\otimes J_p^z)$ to $ \mathcal{H}_{n_p/2}$. 

%I think we only have to say this, $[H_{tot},\mathbf{J}_p^2]=0$ so time evolved state under $H_{tot}$ will be again contained in the same $J_p=n_p/2$ block if we start from the ground state of $H_s$ which is already in this block. The diagonalization of $H_{tot}$ is not an important thing to mention as we are not looking at the ground state of $H_{tot}$
%}


Starting from an initial state $\rho_{0}$ of the system, the time-evolved state of the full system at time $t$ is given by 
\begin{eqnarray}
    \rho(t)= \text{e}^{-\text{i}H_{\text{tot}}t}\rho_0\text{e}^{\text{i}H_{\text{tot}}t}.
\end{eqnarray}
The time-dependence of the partial trace-based entanglement on the peripheral qubits can be computed using
$\rho_p(t)=\text{Tr}_{0}\rho(t)$, while the measurement-based entanglement can be computed by performing single-qubit projective measurements on the central qubit, when the system is in state $\rho(t)$. In Fig.~\ref{fig:time_series}, we plot the time-evolution of $E_{B_1:B_2}$ and $\langle E_{B_1:B_2}\rangle$ for a system with $n_p=30$ , and $n^\prime=15$.  Both $E_{B_1:B_2}$ and $\langle E_{B_1:B_2}\rangle$ oscillate with $t$, and the average values, $\overline{E_{B_1:B_2}}$ and $\overline{\langle E_{B_1:B_2}\rangle}$, are obtained by averaging over the time-series data in the long-time limit. We probe the long-time limit of the time-evolving spin-star system using  $\overline{E_{B_1:B_2}}$ and $\overline{\langle E_{B_1:B_2}\rangle}$. Note that for $\gamma=0$, the field term commutes with the Hamiltonian irrespective of $\Delta$, leading to no time-evolution of the initial state. Therefore, we specifically focus on the cases with $\gamma\neq 0$. In Figs.~\ref{fig:zero_field_dynamic}(e) and (f), we respectively plot $\overline{E_{B_1:B_2}}$ and $\overline{\langle E_{B_1:B_2}\rangle}$ as functions of $n_p$, where the partition-size $n^\prime=n_p/2$ $((n_p-1)/2)$ for even (odd) $n_p$. In contrast to the static case discussed in Sec.\ref{subsec:static_entanglement}, for large $n_p$, both $\overline{E_{B_1:B_2}}$ and $\overline{\langle E_{B_1:B_2}\rangle}$ exhibit logarithmic growth with $n_p$ in both $xy$-isotropic ($\gamma=0$) and $xy$-anisotropic ($\gamma\neq 0$) conditions, thereby negating the \emph{anisotropy effect}. The discrepancies in the logarithmic variation due to the finite-size of the system vanishes within $n_p\sim 20$. We further test the variations of  $\overline{E_{B_1:B_2}}$ and $\overline{\langle E_{B_1:B_2}\rangle}$ with $n^\prime$ for a fixed $n_p$, and found them to be similar to  the static case (see Eq.~(\ref{eq:entanglement_fit_nprime})). See Figs.~\ref{fig:zero_field_dynamic}(g)-(h). 




  
\subsection{Large center limit}
\label{subsec:large_center}

We now consider the XYZ model on the star network of spins when the size of the center is far larger than the size of the periphery, i.e., $n_0/n_p\gg 1$. Note that this limit is identical to the $n_0/n_p\ll 1$ limit of the model through an interchange of the central spins with peripheral spins, thereby exhibiting similar features of ground state energy as the limit $n_0/n_p\ll 1$.  We present the variation of $E_{B_1:B_2}$ with $n_0$ for a fixed $n_p$ in the limit $n_0/n_p\gg 1$ in Fig.~\ref{fig:other_limits}(a) (the minimal instance of two peripheral spins and $n_p$ central spins is represented in Fig.~\ref{fig:system}(b)). Our numerical analysis suggests that in contrast to the results obtained in the limit $n_0/n_p\ll 1$,  for increasing $n_0$ with $n_0/n_p\gg 1$, $E_{B_1:B_2}$ decreases monotonically and approaches zero asymptotically as
\begin{eqnarray}\label{eq:fit_ent_third_limit}
E_{B_1:B_2}\sim a+ bn_0^{-m},
\end{eqnarray}
where the fitting parameters are obtained from the numerical data (see Fig.~\ref{fig:other_limits}(a)).  

% Figure environment removed

\section{XYZ model on a star network in the competing center limit}
\label{sec:coupled_central}

A question that naturally arises at this point is whether the features of bipartite peripheral entanglement remain the same as one considers  the limit $n_0/n_p\rightarrow 1$. In this situation, the dimensionless spin-star Hamiltonian is given by (see Eq.~(\ref{eq:star-xyz_ordered}))  
\begin{eqnarray}
    H_s&=&\pm\frac{1}{2}\Big[J_0^+(J_p^- + \gamma J_p^+) + J_0^-(J_p^+ + \gamma J_p^-)  +\Delta   J_0^z J_p^z\Big],\nonumber\\
\label{eq:Hamiltonian_coupled_central}
\end{eqnarray}
where $J_0^{\pm}=\sum_{i=0}^{n_0-1}S_i^{\pm}$, $J_0^{z}=\sum_{i=0}^{n_0-1}S_i^{z}$, $J_p^{\pm}=\sum_{i=n_0}^{n_0+n_p-1}S_i^{\pm}$, and $J_p^{z}=\sum_{i=n_0}^{n_0+n_p-1}S_i^{z}$. Analysis for the Hamiltonian (\ref{eq:Hamiltonian_coupled_central}) is similar to that of the Hamiltonian (\ref{eq:star-xyz_ordered}) in the case of the single central spin, and the values of $J_0$ and $J_p$ corresponding to the ground and the first excited states of $H_s$ are $n_0/2$ and $n_p/2$ respectively, independent of the system parameters $\gamma$, and $\Delta$. 
%\footnote{For $n_0=2$, the ground and the first excited states of (\ref{eq:Hamiltonian_coupled_central}) belong to the block $\mathcal{H}_{n_p/2}$ irrespective of the values of the system parameters $\gamma$, and $\Delta$, as in the case described in Sec.~\ref{sec:xyz_model_single_central}. 
%For $n_0\geq 3$, the values of $J_p$ depends on the choice of $(\gamma,\Delta)$}. 
Upon diagonalization, doubly degenerate ground states are found when $n=n_0+n_p$ is odd irrespective  of the values of $(\gamma,\Delta)$, while the effective degeneracy in ground states for large system size is observed when $n$ is even only for $\gamma\neq 0$ (see Fig.~\ref{fig:other_limits}(b)). Note that in the present case, $\delta \mathcal{E}$ approaches saturation (for $\gamma=0$) or vanishing (for $\gamma\neq 0$) as 
\begin{eqnarray}
\delta\mathcal{E}\sim a + \exp\left(-bn_p^{m}\right). 
\label{eq:energy_fit_np}
\end{eqnarray}
For $\gamma=0$, according to our numerical investigation up to $n_p=n_0=50$, $\delta\mathcal{E}$ saturates to a non-zero value, which increases with an increase in the value of $\Delta$, implying a ND ground state for $\gamma=0$ up to the maximum system-size investigated in this paper. 

%For $\gamma\neq 0$ and for even $n$, however, $\delta\mathcal{E}$ approaches zero as in Eq.~(\ref{eq:double_exponential}). 





In the limit $n_0/n_p\rightarrow 1$, the structure of the ground states  with permutation symmetry in both central and peripheral qubits is given by 
\begin{eqnarray}
    \ket{\Psi_0}&=&\sum_{l^\prime=0}^{n_0}\sum_{l=0}^{n_p} c_{l,l^\prime}\ket{D_{l^\prime}^{n_0}}\otimes\ket{D^{n_p}_l},
    \label{eq:ground_state_form_coupled_central}
\end{eqnarray}
where $\sum_{l^\prime}\sum_l |c_{l,l^\prime}|^2=1$, and the definition of $\ket{D_{l^\prime}^{n_0}}$ is similar to that of $\ket{D_l^{n_p}}$. Similar to the case of single central qubit,  the TGS $\rho_0$ corresponding to the degenerate and the effectively degenerate ground states $\ket{\Psi^\pm_0}$ is of the form (\ref{eq:thermal_ground_state}), while the mixed state $\rho_p$ on the peripheral qubits obtained via tracing out the central qubits is of the form (\ref{eq:mixed_state_dicke_basis}), enabling one to compute the bipartite entanglement $E_{B_1:B_2}$ on the peripheral spins. 
%On the other hand,  a projection measurement on the central qubits results in $2^{n_0}$ possible outcomes with an  ensemble of $2^{n_0}$  post-measured states $\{\rho_p^l;l=1,2,3,4\}$ on the $n_p$ peripheral qubits, with the form of each post-measured states being that of Eq.~(\ref{eq:mixed_state_dicke_basis}). Therefore, the average entanglement $\langle E_{B_1:B_2}\rangle$ can also be computed, although performing the optimization and determining the localizable entanglement become increasingly difficult with increasing $n_0$ \JK{the whole paragraph is taking about two central spin and $n_{p}$ peripheral spin system}. 
In Fig.~\ref{fig:other_limits}(c), we plot the variations of $E_{B_1:B_2}$ as a function of $n_p$, where $n_p=n_0$. In contrast to the limit $n_0/n_p\ll 1$, $E_{B_1:B_2}$ for $\gamma=0$ saturates with increasing $n_p$ as 
\begin{eqnarray}
    E_{B_1:B_2} &\sim& a\left[1-\exp\left(-bn_p^m\right)\right].
    \label{eq:entanglement_fit_np}
\end{eqnarray}
For $\gamma\neq 0$ also, $E_{B_1:B_2}$ saturates with $n_p$, although the approach to saturation is different from that of the $\gamma=0$, and the saturation value decreases with increasing $\gamma$, implying an \emph{anisotropy effect}. However, the time-averaged bipartite peripheral entanglement, $\overline{E_{B_1:B_2}}$, in contrast to the $n_0/n_p\ll 1$ limit, does not negate the anisotropy effect. 







\section{Conclusion and Outlook}
\label{sec:conclude}

In this paper, we study the XYZ model on a spin-star system, where a number of central spins are connected to a number of peripheral spins in such a way that each central spin interacts with all peripheral spins. We specifically consider the limit of the system where the number of central spins are very small compared to the number of peripheral spins. In this limit, the ground state of the system is doubly degenerate when the system-size is even, while there is a finite energy gap between the ground and the first excited states of the system with odd number of spins. This energy gap decreases rapidly with increasing system-size, inducing an \emph{effective} double degeneracy of the ground states. When the system size is large (i.e., when the periphery-size is large), the bipartite entanglement over typically equal bipartition of the periphery, computed from either partial trace-based, or measurement-based approaches, exhibits a logarithmic growth with the periphery-size when the system is $xy$-isotropic (i.e., $\gamma=0$). This feature vanishes for $\gamma\neq 0$, which we refer to as the anisotropy effect. Interestingly, the anisotropy effect can be nullified by kick-starting a magnetic field of constant strength on all spins in the system, where the average values of the time-series of bipartite peripheral entanglement computed from either of the approaches exhibit a logarithmic growth with the periphery-size irrespective of the values of $\gamma$. We further consider the large center and the competing center limits of the model, and find the features of bipartite peripheral entanglement to be qualitatively different from that of the large periphery limit of the model. 

A number of possible research directions open up from this study. Within the range of the results reported in this paper, it would be interesting to investigate whether similar features in entanglement exists if interactions are present within the central and/or the peripheral spins themselves. Also, a similar investigation  in a star-chain configuration of spin-$\frac{1}{2}$ particles~\cite{Yao2011,*Ping2013,*ZHU2018,*Grimaudo2022} is yet to be performed. Another interesting question would be whether these entanglement properties of the spin-star system are robust against the presence of quenched disorder in the system~\cite{de_dominicis_giardina_2006}. Besides, similarity of our results with the already existing results for the LMG model~\cite{Latorre2005,*Unanyan2005} raises the question as to what would be the underlying principle to model a spin-lattice system that exhibits similar entanglement properties. Moreover, recent interests in exploring quantum technological aspects involving $d$-level systems, or qudits~\cite{Wang2020} highlights the relevance of exploring similar models where lattice sites are populated with spins having a higher spin quantum number.       

\acknowledgements 

We acknowledge the use of \href{https://github.com/titaschanda/QIClib}{QIClib} -- a modern C++ library for general purpose quantum information processing and quantum computing. We also thank Aditi Sen (De) for useful discussions. 

\appendix



\section{Small systems with single central spin}
\label{app:small_system}


In this section, we look into two specific examples for $n_p=2$ and $n_p=3$ in the case of a star network of $(n_p+1)$ spins, as follows.  


\subsection{\texorpdfstring{$n_p=2$}{np=2}}



We first consider the case of $J_{p}=1$  ($m_p=-1,0,1$) among the allowed values $J_p=0,1$ for $n_p=2$. We group the basis states according to Eq.~(\ref{eq:grouping}) as 
\[\{\ket{-1/2}\ket{-1}, \ket{1/2}\ket{0} ,\ket{-1/2}\ket{1} \}\]   
  and  
\[\{\ket{1/2}\ket{-1}, \ket{-1/2}\ket{0} ,\ket{1/2}\ket{1}\}\] respectively, leading to the $\mathcal{H}_{1}$ block having the form (\ref{eq:block_diagonal_forms}),  with $A_1$ and $A_2$ being two $3\times 3$ matrices given by 
\begin{eqnarray}
A_1 &=& \frac{1}{2} \begin{bmatrix}
    -\Delta  &  0   &   \alpha^{-}_{0}  \\
    0  &  \Delta &  \gamma \alpha^{+}_{0}   \\
    \alpha^{+}_{-1}    &    \gamma \alpha^{-}_{1}    &    0 
\end{bmatrix},\\ 
A_2&=& \frac{1}{2}
\begin{bmatrix}
   0  &  \gamma \alpha^{+}_{-1}   &  \alpha^{-}_{1}   \\
\gamma \alpha^{-}_{0}  &  \Delta &  0   \\
    \alpha^{+}_{0}    &    0    &    -\Delta 
\end{bmatrix}, 
\end{eqnarray} 
upto a normalization factor, where 
\begin{eqnarray}
\alpha^{\pm}_{m_{p}} = \sqrt{J_{p}(J_{p}+1) - m_{p}(m_{p}\pm 1)}.
\label{eq:alpha}
\end{eqnarray}
Note that $|A_1|=|A_2|=2\Delta(\gamma^2-1)$, which vanishes for $\Delta=0$ or $\gamma=\pm 1$, implying that at least one of the eigenvalues of both $A_1$ and $A_2$ vanishes at these conditions. For $\Delta=0$, the eigenvalues are $0,\pm\sqrt{2(\gamma^{2}+1)}/2$ for both  $A_{1}$ and $A_{2}$, leading to the doubly-degenerate ground state energy $-\sqrt{2(\gamma^2+1)}/2$. The corresponding eigenvectors belonging to the blocks $A_1$ and $A_2$ are given by
\begin{eqnarray}
    \ket{\psi}_{0,{A_{1}}} &=& \frac{1}{\sqrt{2(\gamma^{2}+1)}} \ket{-1/2}_0\Big(\ket{-1}_p+\gamma\ket{1}_p\Big)\nonumber\\ &&-\frac{1}{\sqrt{2}}\ket{1/2}_0\ket{0}_p,\nonumber\\
    \ket{\psi}_{0,{A_{2}}} &=& \frac{1}{\sqrt{2(\gamma^{2}+1)}} \ket{1/2}_0\Big(\gamma\ket{-1}_p+\ket{1}_p\Big)\nonumber\\ &&-\frac{1}{\sqrt{2}}\ket{-1/2}_0\ket{0}_p,
\end{eqnarray} 
where clearly $\ket{\psi}_{0,A_1}=\mathcal{O}\ket{\psi}_{0,A_2}$, and vice-versa (see Sec.~\ref{subsec:single_central_spin_ground_state}). Consequently, the doubly degenerate ground-states corresponding to the ground-state energy $-\sqrt{2(\gamma^2+1)}/2$ and in the common eigenspace of $H_s$ and $\mathcal{O}$ are  given by Eq.~(\ref{eq:degenerate_ground_states}).   Similarly for  $\gamma=\pm1$, the eigenvalues of either of $A_1$ and $A_2$ are $0,\pm\sqrt{4+\Delta^{2}}/2$, with the doubly-degenerate ground state energy $-\sqrt{4+\Delta^2}/2$, and the corresponding eigenvectors being  
\small 
\begin{eqnarray}
    \ket{\psi}_{0,{A_{1}}} &=& \Big(\frac{\sqrt{4+\Delta^{2}}-\Delta}{\sqrt{4+\Delta^{2}}+\Delta}\Big) \ket{-1/2}_{0}\ket{1}_{p} \nonumber\\ && +\ket{-1/2}_{0}\ket{-1}_{p}+ \Big(\frac{\Delta-\sqrt{4+\Delta^{2}}}{\sqrt{2}}\Big) \ket{1/2}_{0}\ket{0}_{p},\nonumber\\
   \ket{\psi}_{0,{A_{2}}} &=& \Big(\frac{\sqrt{4+\Delta^{2}}-\Delta}{\sqrt{4+\Delta^{2}}+\Delta}\Big) \ket{1/2}_{0}\ket{-1}_{p}  \nonumber\\ && +\ket{1/2}_{0}\ket{1}_{p}+ \Big(\frac{\Delta-\sqrt{4+\Delta^{2}}}{\sqrt{2}}\Big) \ket{-1/2}_{0}\ket{0}_{p},\nonumber\\ 
\end{eqnarray} \normalsize 
up to a normalization factor. It is easy to see that similar to the case of $\Delta=0$, $\ket{\psi}_{0,A_1}$ and $\ket{\psi}_{0,A_2}$ are connected via $\mathcal{O}$, and the doubly-degenerate ground states in the common eigenspace of $H_s$ and $\mathcal{O}$ are given by Eq.~(\ref{eq:degenerate_ground_states}), as in the case of $\Delta=0$. The double degeneracy in ground state is maintained in the entire range $0\leq |\gamma|\leq 1$, $0\leq |\Delta|\leq 1$.  

We point out here that all sectors of $H_s$ corresponding to values of $J_p\neq n_p/2$ can be block-diagonalized using a similar grouping of basis elements.  In the present example, the block $\mathcal{H}_0$ corresponding to $J_p=0$, is diagonal in the basis  $\{\ket{\frac{1}{2}}\otimes\ket{0}, \ket{-\frac{1}{2}}\otimes \ket{0}\}$, with both eigenvalues vanishing irrespective of the values of $(\gamma,\Delta)$. 



\subsection{\texorpdfstring{$n_p=3$}{np=3}}


In this case, $J_p=3/2,1/2,1/2$, and we start with the block $H_s(J_p=3/2)$, where grouping the basis as
\[
\{ \ket{-1/2}\ket{3/2}, \ket{1/2}\ket{1/2}, \ket{-1/2}\ket{-1/2},  \ket{1/2}\ket{-3/2}\}\]
 and 
 \[\{\ket{1/2}\ket{3/2}, \ket{-1/2}\ket{1/2}, \ket{1/2}\ket{-1/2},  \ket{-1/2}\ket{-3/2}\}\] 
 makes $\mathcal{H}_{3/2}$ of the form~(\ref{eq:block_diagonal_forms}) with the $4\times4$ matrices $A_1$ and $A_2$ given by 
\begin{eqnarray}
A_1&=& \frac{1}{2}
\begin{bmatrix}
    \frac{3\Delta}{2}  &  0   &   \gamma \alpha^{+}_{\frac{1}{2}}   &    0   \\
    0  &  -\frac{\Delta}{2}   &  \alpha^{-}_{\frac{1}{2}} &\gamma\alpha^{+}_{-\frac{3}{2}}   \\
    \gamma \alpha^{-}_{\frac{3}{2}} &  \alpha^{+}_{-\frac{1}{2}}  &   -\frac{\Delta}{2}   &  0  \\
    0    &    \gamma \alpha^{-}_{-\frac{1}{2}}    &    0    &    \frac{3\Delta}{2}
\end{bmatrix},\\ 
A_2&=& \frac{1}{2}
\begin{bmatrix}
    \frac{\Delta}{2}  &  0  & \alpha^{-}_{\frac{3}{2}}   &    \gamma \alpha^{+}_{-\frac{1}{2}}   \\
    0  &  -\frac{3\Delta}{2} &  0 &    \alpha^{-}_{-\frac{1}{2}}   \\
     \alpha^{+}_{\frac{1}{2}} &  0  &   -\frac{3\Delta}{2}   &  0  \\
     \gamma \alpha^{-}_{\frac{1}{2}}    &  \alpha^{+}_{-\frac{3}{2}}     &    0    &    \frac{\Delta}{2}
\end{bmatrix}
\end{eqnarray} 
The eigenvalues of $\mathcal{H}_{3/2}$ are
\begin{eqnarray} 
\lambda^1_{\pm}&=&\frac{1}{2}\left( \frac{\Delta}{2}+1\pm\sqrt{(\Delta-1)^{2}+3\gamma^{2}}\right),\nonumber\\  
\lambda^2_\pm&=&\frac{1}{2}\left( \frac{\Delta}{2}-1\pm\sqrt{(\Delta+1)^{2}+3\gamma^{2}}\right),\nonumber\\ 
\lambda^3_{\pm}&=&\frac{1}{2}\left(-\frac{\Delta}{2}+\gamma \pm\sqrt{(\Delta+\gamma)^{2}+3}\right),\nonumber\\ 
\lambda^4_\pm &=& \frac{1}{2}\left(-\frac{\Delta}{2}-\gamma \pm\sqrt{(\Delta-\gamma)^{2}+3}\right), 
\end{eqnarray} 
none of which are degenerate for $\Delta=0$ or $\gamma=0$. On the other hand, $\mathcal{H}_{1/2}$ also takes the form~(\ref{eq:block_diagonal_forms}) upon grouping the basis as 
\[\{\ket{1/2}\ket{1/2}, \ket{-1/2}\ket{-1/2}\}\] 
and 
\[\{\ket{1/2}\ket{-1/2}, \ket{-1/2}\ket{1/2}\}\] 
with the two-dimensional $A_1$ and $A_2$ matrices  given as
\begin{eqnarray}
A_1&=& \frac{1}{2}
\begin{bmatrix}
    \frac{\Delta}{2} & \gamma \alpha^{+}_{-\frac{1}{2}} \\
    \gamma \alpha^{-}_{\frac{1}{2}} & \frac{\Delta}{2}
\end{bmatrix},
A_2= \frac{1}{2}
\begin{bmatrix}
    -\frac{\Delta}{2} & \alpha^{-}_{\frac{1}{2}} \\
    \alpha^{+}_{-\frac{1}{2}} & -\frac{\Delta}{2}
\end{bmatrix}. 
\end{eqnarray} 
The eigenvalues of $\mathcal{H}_{1/2}$ are 
\begin{eqnarray} 
\lambda^5_\pm&=&\frac{\Delta}{4} \pm \frac{\gamma}{2}, \lambda^6_\pm = -\frac{\Delta}{4}\pm \frac{1}{2},
\end{eqnarray} 
each of which is doubly degenerate due to the presence of a second identical block $\mathcal{H}_{1/2}$. 

Note that for $\Delta=0$ and for arbitrary \emph{positive} values of $\gamma$ within its allowed range, the ground state energy is given by either $\lambda^4_-$, or $\lambda^2_-$, depending on the value of $\gamma$. The eigenvectors corresponding to these two eigenvalues are given by 
\small  
\begin{eqnarray}
    \ket{\psi(\lambda^2_-)} &=& -\ket{-1/2}\ket{3/2} -\frac{1+\sqrt{1+\gamma^2}}{\sqrt{3}\gamma}\ket{1/2}\ket{1/2} \nonumber\\ &&+\frac{1+\sqrt{1+\gamma^2}}{\sqrt{3}\gamma}\ket{-1/2}\ket{-1/2}+\ket{1/2}\ket{-3/2},\nonumber\\
    \ket{\psi(\lambda_-^4)} &=&-\ket{1/2}\ket{3/2} +\frac{\gamma-\sqrt{3+\gamma^2}}{\sqrt{3}}\ket{-1/2}\ket{1/2}  \nonumber\\ &&-\frac{\gamma-\sqrt{3+\gamma^2}}{\sqrt{3}}\ket{1/2}\ket{-1/2}+\ket{-1/2}\ket{-3/2},\nonumber\\ 
\end{eqnarray}\normalsize  
upto normalization, both of which are eigenvectors of $\mathcal{O}$ with an eigenvalue $-1$. Similar result can be obtained for negative values of $\gamma$ as well. On the other hand, for $\gamma=0$ and $0<|\Delta|<1$, the ground state is non-degenerate with the ground state eigenvalue given by $\lambda^2_-$, and the corresponding eigenvector is 
\begin{eqnarray}
    \ket{\psi(\lambda^2_-)} &=& \frac{1}{\sqrt{2}}\left(\ket{-1/2}\ket{-1/2}-\ket{1/2}\ket{1/2}\right). \nonumber\\
\end{eqnarray}
For $\gamma=0$ and $\Delta = 1$, the ground state is two-fold degenerate with eigenvalue $\lambda^3_-(=\lambda^4_-)$ and corresponding eigenvectors, up to normalization, are 
\begin{eqnarray}
    \ket{\psi(\lambda^3_-)} &=&  \frac{a}{\sqrt{2}} \big(\ket{1/2}\ket{3/2} -\ket{-1/2}\ket{-3/2}\big) \nonumber\\ &&
    +\frac{\sqrt{1-a^2}}{\sqrt{2}}\big(\ket{-1/2}\ket{1/2}-\ket{1/2}\ket{-1/2}\big),\nonumber \\
    \ket{\psi(\lambda^4_-)} &=& \frac{a}{\sqrt{2}} \big(\ket{1/2}\ket{3/2}+\ket{-1/2}\ket{-3/2}\big)\nonumber\\ &&+\frac{\sqrt{1-a^2}}{\sqrt{2}}\big(\ket{-1/2}\ket{1/2}+\ket{1/2}\ket{-1/2}\big),    \nonumber\\ 
\end{eqnarray} 
where \small 
\begin{eqnarray} 
a&=&\frac{-(\Delta+\sqrt{3+\Delta^2})}{\sqrt{3+(\Delta+\sqrt{3+\Delta^2})^2}}+\frac{\sqrt{3}}{\sqrt{3+(\Delta-\sqrt{3+\Delta^2})^2}},\nonumber\\ 
\end{eqnarray} \normalsize  
and we note that $\mathcal{O}\ket{\psi(\lambda^2_-)}=-\ket{\psi(\lambda_-^2)}$,
$\mathcal{O}\ket{\psi(\lambda^3_-)}=-\ket{\psi(\lambda_-^3)}$, and 
$\mathcal{O}\ket{\psi(\lambda^4_-)}=\ket{\psi(\lambda_-^4)}$. 

\section{Fitting parameters}
\label{app:fitting_params}
Here we present the information regarding the fitting of Eqs.~(\ref{eq:double_exponential}), (\ref{eq:entanglement gap fitt}), (\ref{eq:entanglement_fit_nprime}), (\ref{eq:fit_ent_third_limit}), (\ref{eq:energy_fit_np}), and (\ref{eq:entanglement_fit_np}) to the corresponding data presented in Figs.~\ref{fig:energy_gap}, \ref{fig:zero_field_dynamic}, and~\ref{fig:other_limits}.  
The fitting parameters $a$, $b$, and $m$ in the above equations are consolidated in Table~\ref{tab:fitting_parameters}. 

\begin{table*}
    \centering
    \begin{tabular}{|c|c|c|c|c|c||c|c|c|c|c|c|}
         \hline 
         Fig. No. & $n$ odd/even & $\gamma$ &  $a$ & $b$ & $m$ & Fig. No. & $n$ odd/even & $\gamma$ & $a$ & $b$ & $m$ \\
         \hline 
         Fig.~\ref{fig:energy_gap}(a) & even & $0.0$ & $0.0008$ & $0.391$ & $0.922$ & Fig.~\ref{fig:zero_field_dynamic}(e) & odd & $0.0$ & $-0.609$ & $0.490$ & $-$ \\
         \hline
          Fig.~\ref{fig:energy_gap}(a) & even & $0.05$ & $0.0$ & $0.605$ & $0.888$ & Fig.~\ref{fig:zero_field_dynamic}(e) & odd & $0.05$ & $-0.564$ & $0.437$ & $-$ \\
          \hline
           Fig.~\ref{fig:energy_gap}(a) & even & $0.1$ & $0.0$ & $0.644$ & $0.983$ & Fig.~\ref{fig:zero_field_dynamic}(e) & odd & $0.1$ & $-0.531$ & $0.436$ & $-$ \\
         \hline
         Fig.~\ref{fig:energy_gap}(b) & even & $0.0$ & $0.001$ & $0.929$ & $0.559$ & Fig.~\ref{fig:zero_field_dynamic}(f) & even & $0.0$ & $0.362$ & $0.495$ & $-$ \\
         \hline
         Fig.~\ref{fig:energy_gap}(b) & even & $0.05$ & $0.0$ & $0.080$ & $1.280$ & Fig.~\ref{fig:zero_field_dynamic}(f) & even & $0.05$ & $-0.276$ & $0.485$ & $-$ \\
         \hline
          Fig.~\ref{fig:energy_gap}(b) & even & $0.1$ & $0.0$ & $0.131$ & $1.285$ & Fig.~\ref{fig:zero_field_dynamic}(f) & even & $0.1$ & $-0.283$ & $0.494$ & $-$ \\
        \hline
        Fig.~\ref{fig:energy_gap}(c) & even & $0.0$ & $0.0014$ & $0.705$ & $0.644$ & Fig.~\ref{fig:zero_field_dynamic}(f) & odd & $0.0$ & $-0.253$ & $0.489$ & $-$\\
        \hline
        Fig.~\ref{fig:energy_gap}(c) & even & $0.05$ & $0.0$ & $0.086$ & $1.265$ & Fig.~\ref{fig:zero_field_dynamic}(f) & odd & $0.05$ & $-0.249$ & $0.480$ & $-$\\
        \hline
         Fig.~\ref{fig:energy_gap}(c) & even & $0.1$ & $0.0$ & $0.133$ & $1.288$ & Fig.~\ref{fig:zero_field_dynamic}(f) & odd & $0.1$ & $-0.283$ & $0.494$ & $-$ \\
        \hline
        Fig.~\ref{fig:zero_field_dynamic}(a) & even & $0.0$ & $-0.253$ & $0.489$ & $-$ & Fig.~\ref{fig:zero_field_dynamic}(g) & odd & $0.0$ & $2.218$ & $0.287$ & $0.798$ \\
        \hline
        Fig.~\ref{fig:zero_field_dynamic}(a) & odd & $0.0$ & $-0.609$ & $0.490$ & $-$ & Fig.~\ref{fig:zero_field_dynamic}(g) & odd & $0.05$ & $1.943$ & $0.382$ & $0.722$\\
         \hline  
        Fig.~\ref{fig:zero_field_dynamic}(b) & even & $0.0$ & $0.362$ & $0.495$ & $-$ & Fig.~\ref{fig:zero_field_dynamic}(g) & odd & $0.1$ & $1.966$ & $0.394$ & $0.727$ \\
        \hline
        Fig.~\ref{fig:zero_field_dynamic}(b) & odd & $0.0$ & $-0.253$ & $0.489$ & $-$ & Fig.~\ref{fig:zero_field_dynamic}(h) & odd & $0.0$ & $2.565$ & $0.327$ & $0.773$\\
         \hline
        Fig.~\ref{fig:zero_field_dynamic}(c) & odd & $0.0$ & $2.218$ & $0.287$ & $0.798$ & Fig.~\ref{fig:zero_field_dynamic}(h) & odd & $0.05$ & $2.488$ & $0.471$ & $0.689$ \\
        \hline
        Fig.~\ref{fig:zero_field_dynamic}(c) & odd & $0.05$ & $0.688$ & $0.247$ & $0.714$ & Fig.~\ref{fig:zero_field_dynamic}(h) & odd & $0.1$ & $2.517$ & $0.473$ & $0.704$ \\
        \hline
        Fig.~\ref{fig:zero_field_dynamic}(c) & odd & $0.1$ & $0.449$ & $0.245$ & $0.706$ & Fig.~\ref{fig:other_limits}(a) & odd & $0.0$ & $-0.004$ & $7.624$ & $1.591$ \\
        \hline
        Fig.~\ref{fig:zero_field_dynamic}(d) & odd & $0.0$ & $2.565$ & $0.327$ & $0.773$ & Fig.~\ref{fig:other_limits}(a) & even & $0.0$ & $-0.002$ & $1.123$ & $1.719$ \\
        \hline
        Fig.~\ref{fig:zero_field_dynamic}(d) & odd & $0.05$ & $0.725$ & $0.261$ & $0.690$ & Fig.~\ref{fig:other_limits}(b) & even & $0.0$ & $0.255$ & $1.274$ & $0.542$ \\
        \hline
        Fig.~\ref{fig:zero_field_dynamic}(d) & odd & $0.1$ & $0.471$ & $0.258$ & $0.682$ & Fig.~\ref{fig:other_limits}(b) & even & $0.05$ & $-0.001$ & $0.562$ & $0.790$ \\
        \hline
        Fig.~\ref{fig:zero_field_dynamic}(e) & even & $0.0$ & $-0.253$ & $0.489$ & $-$ & Fig.~\ref{fig:other_limits}(b) & even & $0.1$ & $-0.0006$ & $0.588$ & $0.934$ \\
        \hline
        Fig.~\ref{fig:zero_field_dynamic}(e) & even & $0.05$ & $-0.547$ & $0.434$ & $-$ & Fig.~\ref{fig:other_limits}(c) & even & $0.0$ & $0.245$ & $0.650$ & $0.469$ \\
        \hline
        Fig.~\ref{fig:zero_field_dynamic}(e) & even & $0.1$ & $-0.505$ & $0.432$ & $-$ &  &  &  &  &  & \\
         \hline
    \end{tabular}
    \caption{Values of fitting parameters $a$, $b$, and $m$ from Eqs.~(\ref{eq:double_exponential}), (\ref{eq:entanglement gap fitt}), (\ref{eq:entanglement_fit_nprime}), (\ref{eq:fit_ent_third_limit}), (\ref{eq:energy_fit_np}), and (\ref{eq:entanglement_fit_np}), fitted to the corresponding data presented in Figs.~\ref{fig:energy_gap}, \ref{fig:zero_field_dynamic}, and~\ref{fig:other_limits}.}
    \label{tab:fitting_parameters}
\end{table*}

\bibliography{ref}

\end{document} 




%We further note that \AKP{there may be a factor coming in. check!}
%\begin{eqnarray}
    %S_0^zH_s(\gamma_s,\Delta_s)S_0^z=-H_s(\gamma_s,-\Delta_s),
%\end{eqnarray}
%which implies that $\ket{\Psi^\prime}=S_0^z\ket{\Psi}$ is the ground state of $-H_s(\gamma_s,-\Delta_s)$ irrespective of the values of $\gamma_s$ and $\Delta_s$. 

%\AKP{why is this information important for us?}\HK{(This means both ferromagnetic($K_s<0$) and antiferromagnetic($K_s>0$) coupling case have ground state in this block. So in complete ($K_s,\gamma_s,\Delta_s$) space this idea of constraining the hamiltonian to $2n+2$ dimensions to get the ground state holds. This is not for any system. For example spin clusters which also have permutation invariant hamiltonian, have ground state from this block only in ferromagnetic coupling.)}














%\propose{I} \emph{The block of XYZ hamiltonian designated by the eigenvalue of the operator $\mathbf{J}_r^2$ ($J_{r}$)  can be  further made block diagonal upon grouping the constituent basis states as follows,}

%\emph{When $J_r$ is half integer multiple.}

%Block 1:
%$\ket{\frac{1}{2}}\ket{-J_r},\ket{-\frac{1}{2}}\ket{-J_r+1},\ket{\frac{1}{2}}\ket{-J_r+2}$
%... $\ket{\frac{1}{2}}\ket{J_r-1},\ket{-\frac{1}{2}}\ket{J_r}$    

%Block 2:
%$\ket{-\frac{1}{2}}\ket{-J_r},\ket{\frac{1}{2}}\ket{-J_r+1},\ket{-\frac{1}{2}}\ket{-J_r+2}$
%... $\ket{-\frac{1}{2}}\ket{J_r-1},\ket{\frac{1}{2}}\ket{J_r}$    

%\emph{When $J_r$ is integer multiple.}

%Block 1:
%$\ket{\frac{1}{2}}\ket{-J_r},\ket{-\frac{1}{2}}\ket{-J_r+1},\ket{\frac{1}{2}}\ket{-J_r+2}$
%... $\ket{-\frac{1}{2}}\ket{J_r-1},\ket{\frac{1}{2}}\ket{J_r}$    

%Block 2:
%$\ket{-\frac{1}{2}}\ket{-J_r},\ket{\frac{1}{2}}\ket{-J_r+1},\ket{-\frac{1}{2}}\ket{-J_r+2}$
%... $\ket{\frac{1}{2}}\ket{J_r-1},\ket{-\frac{1}{2}}\ket{J_r}$    

%\propose{II} \emph{The extreme energy states will come from the $J_r=\frac{n}{2}$ block of hamiltonian.}
%\begin{proof}
%Our numerical analysis shows that ground state always belongs to $J_r=\frac{n}{2}$ block. As general analytical treatment for $n$ peripheral qubit is not viable see appendix~(\ref{app:small_system}) for explicit calculations of system sizes $n=2,3$ where it can be seen that extreme energy states indeed belong to the $J_r=1,\frac{3}{2}$ blocks respectively.

%Now it can be seen that,
%\begin{equation}
%    \sigma_0^z H_s(K_s,\gamma_s,\Delta_s) \sigma_0^z = -H_s(K_s,\gamma_s,-\Delta_s)
%\end{equation}
%which shows that if $\ket{\phi_G}$ is the ground state of $H_s(K_s,\gamma_s,\Delta_s)$ then $\sigma_0\ket{\phi_G}$ is the most energetic state of $H_s(K_s,\gamma_s,-\Delta_s)$. As this must be true for all $\Delta_s$ and we already now ground state belong to $J_r=\frac{n}{2}$ block, we have the most energetic state of $H_s$ also belong to the same block. Hence the proof. 
%\end{proof}

\HK{
In order to make the phase transition point independent of system size, the Hamiltonian have to be re-written with a scaling factor for field term as follows,
\begin{eqnarray}
H_s(\gamma_s,\Delta_s,g)&=& \pm[\frac{1}{2}\{S_0^+(J_r^- + \gamma_s J_r^+) + S_0^-(J_r^+ + \gamma_s J_r^-)\} \nonumber\\
&& + \Delta_s S_0^z J_r^z]+g\sqrt{S_0J_r}\left(S_0^z+J^z_r\right) 
\end{eqnarray}
where in the case of spin star with one central qubit and $n$ peripheral qubits  $\sqrt{S_0J_r}=\frac{\sqrt{n}}{2}$. 

\textbf{Purpose of the scaling factor}
\begin{enumerate}
    \item The XYZ interaction term of $H_s(\gamma_s,\Delta_s,g)$ is proportional to $S_0J_r$ whereas the field term is only proportional to $S_0$ and $J_r$ thus a scaling factor on either of the terms is necessary to make the transition points size independent.  
    \item In the case of coupled spins with equal magnitude$(S_0=J_r)$ there is a quantum phase transition happening at $h_c=0.5$ when $\gamma_s=1,\Delta_s=0$\cite{Mondal2020}.
    \item For any other values of $S_0,J_r$ it can be shown with $\sqrt{S_0J_r}$ scaling of field term that the system is effectively a coupled spins with reciprocal spins i.e, $S_1=\sqrt{\frac{S_0}{J_r}},S_2=\sqrt{\frac{J_r}{S_0}}$.
    \item Hence the question is whether there is a phase transition in a coupled reciprocal spin system(probably yes)?
\end{enumerate}

}




%Next, consider the situation where a cluster of identical spin-$1/2$ particles, labelled by $0,1,\cdots,n_0$,  ``$0$" and ``$1$", are surrounded by $n$ peripheral spins, labelled by $2,3,\cdots,n+1$, such that each spin in the peripheral ring interacts with each of the  central spins via XYZ interaction. The Hamiltonian $H=H_{0}+H_s$ of the full system is given by 
%\begin{eqnarray}
%\label{eq:star-xyz_ordered_center_couple} 
%H_s&=& \frac{K}{2}\sum_{i=0}^1\{S_i^+(J_r^- + \gamma J_r^+) + S_i^-(J_r^+ + \gamma J_r^-)\} \nonumber\\
%&& +\Delta K\sum_{i=0}^1  S_i^z J_r^z,
%\end{eqnarray}
%and 
%\begin{eqnarray}
%    H_{0} &=& K^\prime\left\{(1+\gamma)S_0^xS_1^x+(1-\gamma)S_0^yS_1^y+\Delta S_0^zS_1^z\right\}\nonumber\\ 
%    \label{eq:central_couple_Hamiltonian}
%\end{eqnarray}
%where $K^\prime$ is the strength of the exchange interaction between the spin ``$0$" and spin ``$1$".  Choosing $|K|$ to be the energy scale, the dimensionless system Hamiltonian $H/|K|\rightarrow H$ is given by


The block $\mathcal{H}_{n_p/2}$ in this case takes the $4(n_p+1)$-dimensional form\footnote{It is worthwhile to note that for $n_0\geq 2$, dimension of the $\mathcal{H}_{n_p/2}$ block is $2^{n_0}(n_p+1)$.} 
\begin{eqnarray}
    \mathcal{H}_{n_p/2}=\begin{bmatrix}
    B & A & A & 0 \\ A^T & 0 & 0 &  A \\A^T & 0 & 0 &  A \\ 0 & A^T & A^T &  B^\prime 0 \\
    \end{bmatrix},
    \label{eq:block_coupled_central_spins}
\end{eqnarray}
with the $(n_p+1)$-dimensional matrices $A$ and $B$ given by~Eqs.~(\ref{eq:A})-(\ref{eq:B}), and $I$ being the $(n+1)$-dimensional identity matrix. 


doubly degenerate ground states are found when $n$ is odd, while a non-degenerate ground state for small $n$ and effectively doubly-degenerate ground states for large $n$ with an $n_c$ value dependent on $(\gamma,\Delta)$ are found for even $n$\footnote{}, similar to the case of odd $n$ described in Sec.~\ref{subsubsec:single_central_spin_ground_state}. 


%Thus we have two of the eight eigen energies of the system. Since the block corresponding to $J_{r}=0$ has only "0" as eigen energies, it is sufficient to show that the $J_{r}=1$ block provides non zero eigen energies. It can be seen that the determinant of both the blocks corresponding to  $J_{r}=1$ is $2\Delta(\gamma^{2}-1)$. For $\Delta=0$ or $\gamma \pm 1$ or both, which are the cases where the determinant vanishes, it can be explicitly shown that two equal and opposite eigen energies emerges from each of these blocks.  For other values of $\gamma$ and $\Delta$, a non zero value of determinant implies non zero eigen energies, thus validating Proposition II. 

% \begin{eqnarray} 
%     \ket{\psi_{1}} &=& \ket{\frac{1}{2}}\otimes\big(C_{1}\ket{1,-1}+C_{2}\ket{1,1}\big) \nonumber\\ &&+\ket{-\frac{1}{2}}\otimes C_{3}\ket{1,0}
% \end{eqnarray}
% and,
% \begin{eqnarray}
%     \ket{\psi_{2}} &=& \ket{\frac{1}{2}}\otimes C_{1}\ket{1,0}+\ket{-\frac{1}{2}}\otimes \big(C_{2}\ket{1,-1}  \nonumber\\
%     && +C_{3}\ket{1,1}\big)
% \end{eqnarray}
% \begin{eqnarray}
%     H\ket{\psi_{1}} &=& \ket{\frac{1}{2}}\otimes \big( [-C_{1}\Delta + C_{3}\alpha^{+}_{1,0}]\ket{1,-1}+\nonumber\\ 
%     && [C_{2}\Delta + C_{3}\gamma \alpha^{+}_{1,0}]\ket{1,1}+\nonumber \\
%     &&\ket{-\frac{1}{2}}\otimes [C_{1} \alpha^{+}_{1,-1}+C_{2}\gamma \alpha^{-}_{1,1}]\ket{1,0} \big) 
% \end{eqnarray}