\documentclass[aps,onecolumn,10pt]{revtex4-2}
\usepackage{amsmath}
\usepackage{graphicx}
\usepackage{xcolor}
\usepackage{hyperref}
\usepackage{amssymb}
\usepackage{ulem} % pour barrer du texte avec \sout{texte}

\def \degre {$^\mathrm{o}$}
\def \mps {m~s$^{-1}$}

\usepackage{geometry}
\geometry{ a4paper, total={210mm,297mm}, left=29mm, right=29mm, top=40mm, bottom=45mm}

\newcommand{\Marc}[1]{\textcolor{red}{#1}}
\newcommand{\Fred}[1]{\textcolor{brown}{#1}}
\newcommand{\Jishen}[1]{\textcolor{blue}{#1}}

\DeclareUnicodeCharacter{2212}{-}

\begin{document}

\title{Wind-wave growth over a viscous liquid}
\author{J. Zhang$^{1,2}$}
\author{A. Hector$^1$}
\author{M. Rabaud$^1$}
\author{F. Moisy$^1$}
\email{frederic.moisy@universite-paris-saclay.fr}
\affiliation{$^1$Universit\'e Paris-Saclay, CNRS, FAST, F-91405 Orsay, France. \\
$^2$PMMH, CNRS, ESPCI Paris, Universit\'e PSL, Sorbonne Universit\'e, Universit\'e Paris-Cit\'e, F-75005, Paris, France.}

\date{\today}

\begin{abstract}
\vspace{0.4 cm}

Experimental and theoretical studies on wind-wave generation have focused primarily on the air-water interface, where viscous effects are small. Here we characterize the influence of the liquid viscosity on the the growth of mechanically generated waves. In our experiment, wind is blowing over a layer of silicon oil, of viscosity 20 and 50 times that of water, and waves of small amplitude are excited by an immersed wave-maker. We measure the spatial evolution of the wave slope envelope using Free-Surface Synthetic Schlieren, a refraction-based optical method. Through spatiotemporal band-pass filtering of the surface slope, we selectively determine the spatial growth rate for each forcing frequency, even when the forced wave is damped and coexists with naturally amplified waves at other frequencies. Systematic measurements of the growth rate for various wind velocities and wave frequencies are obtained, enabling precise determination of the onset of wave growth and the marginal stability curve. From these measurements, we show that Miles' model, which is commonly applied to water waves, offers a reasonable description of the growth rate for more viscous liquids.  We finally discuss the scaling of the growth rate of the most amplified wave and the critical friction velocity with the liquid viscosity.

\end{abstract}

\maketitle


\section{Introduction} \label{sec:intro}

The generation of waves by the wind, whether on a puddle, a pond, or at sea, remains a fascinating phenomenon that has challenged researchers for decades~\cite{Janssen_2004,sullivan2010dynamics,Rabaud_2020,pizzo2021,ayet2022}.
Gaining a comprehensive understanding of this fundamental problem is of first importance for wave forecasting, assessing heat and mass transfers in the ocean, and also in engineering applications involving liquid and gas transport in pipes or heat exchangers~\cite{boomkamp1996classification}.

Since the pioneering works of Phillips~\cite{phillips1957generation} and Miles~\cite{Miles_1957}, the following two-stage picture has emerged~\cite{Miles1960generation}. In the first stage (or principal stage), incoherent wrinkles of very low amplitude are excited by the turbulent pressure fluctuations of the wind, resulting in a {\it linear} increase in wave energy over time~\cite{phillips1957generation,zavadsky2017water,li_shen_2022}, with a saturation to a finite amplitude governed by the liquid viscosity for moderate wind~\cite{Paquier_2016,Perrard2019,Nove_2020}. In the second stage, for sufficient wind, the air-water shear flow becomes unstable, leading to a subsequent {\it exponential} increase of the wave energy~\cite{Miles_1957,plant1977growth,shemer2021spatially}. However, there is still no consensus on the specific details of this general scenario, particularly regarding the nature of the transition between the two stages and its dependence on the various physical parameters of the problem~\cite{Perrard2019,li_shen_2022}. In this context, accurate laboratory measurements of growth rates under controlled conditions, for physical parameters distinct from the usual air-water configuration, are highly valuable.

The difficulty in modeling the wind-wave generation problem has several origins. First, the flow on the air side is turbulent, and both the detailed characteristics of the mean velocity profile and the statistical properties of the turbulent stress fluctuations play significant roles. Second, because of the transport of wave energy, the {\it temporal} growth of waves translates into {\it spatial} growth~\cite{plant1977growth}, but a quantitative relationship between the two requires steady forcing conditions over long distances and durations, which are rarely satisfied in open sea conditions. Third, the wind not only forces waves but also currents, which may reach saturation on a timescale distinct from the growth time~\cite{zavadsky2017water}. Fourth, surface contamination, which is inevitable in open sea environments and large water tanks, can significantly impact the onset of wave generation~\cite{ryan2010recurrent}. Lastly, even in controlled laboratory experiments, the presence of walls introduces additional damping effects and undesirable wave reflections~\cite{hidy1966wind, plant1977growth, Caulliez_1998}.

In this paper, we consider the problem of wind-wave generation on viscous liquids (silicon oils of viscosity 20 and 50 times greater than that of water), with the aim of circumventing some of the previous difficulties: the flow in the liquid is purely laminar, enabling accurate analytical modeling; the surface drift remains moderate, simplifying the air-liquid energy transfer problem; the use of a liquid of low surface tension reduces the impact of surface contamination; the influence of reflected waves is reduced by the strong damping.

As long as the liquid viscosity is not too high (typically less than 100~mm$^2$~s$^{-1}$), the two-stage scenario of the air-water configuration, with linearly increasing wrinkles at small wind and exponentially growing waves at large wind, still applies (larger liquid viscosities are not considered here, as they exhibit a very distinct behavior governed by the Kelvin-Helmholtz instability~\cite{Francis_1954,Miles1959generation,aulnette2019wind,Rabaud_2020,aulnette2022kelvin}). While previous studies using viscous liquids focused on waves naturally generated by the wind~\cite{Paquier_2015,Paquier_2016}, here we are interested in waves triggered by a small mechanical disturbance. Provided that their initial amplitude is sufficient, this forcing bypasses the first stage of Phillips~\cite{Miles_1957}, and the system directly enters the second stage of Miles~\cite{Miles_1957}. This shortcut in the two-stage wave generation problem has the advantage of directly assessing the growth rate of each wave number, at the cost of disregarding the realistic conditions required for the ``true'' wave generation process starting from an initially undisturbed surface, which may result in distinct critical wind velocities for natural and mechanically generated waves.

In the exponential growth regime, Miles demonstrated that the normalized temporal growth rate of a given wave number $k$ is proportional to the normalized wind shear stress $(u^*/c)^2$, where $u^*$ is the friction velocity and $c(k)$ the phase velocity, with a coefficient $\beta$ that depends on the curvature of the mean velocity profile at the ($k$-dependent) critical height where the mean flow velocity matches the phase velocity. This growth is mitigated by the dissipation in the liquid, either laminar or turbulent, which defines in principle a critical friction velocity for the growth of this wave number $k$. However, unlike standard hydrodynamic instabilities arising from the infinitesimal disturbance of a base state, the two-stage transition here also depends  on the incoherent base state from which the waves grow, which itself depends on the liquid viscosity and the wind forcing conditions (intensity, duration)~\cite{Perrard2019}. This complexity likely contributes to the persistent challenge in defining a clear critical wind velocity for the onset of wave generation under realistic conditions.

The growth rate of wind-generated waves has been the subject of a large number of experimental works, either from their temporal growth at a fixed point~\cite{larson1975wind, Kawai1979generation, geva2022excitation}, or from their spatial growth in steady conditions~\cite{hidy1966wind,Wilson_1973, mitsuyasu1982wind, grass2001measurement, tsai2005spatial, Liberzon_2011} (the equivalence between these two approaches is discussed in Ref.~\cite{plant1977growth}), but only in the air-water configuration. These measurements, initially limited to one-point probes, are now possible using optical methods resolved in space and time~\cite{Moisy09,Paquier_2015,Paquier_2016,shemer2019,yousefi2020,liu2022} and direct numerical simulations~\cite{lin2008direct,zonta2015growth,li_shen_2022,Wu_2021,wu2022revisiting,burdairon2023}, revisiting this old problem with new, high quality data. 

In our experiments, the spatial evolution of mechanically generated waves are measured using Free-Surface Synthetic Schlieren~\cite{Moisy09}, a refraction-based optical method offering a wave slope resolution of $3 \times 10^{-3}$. To isolate the spatial evolution of the wave at each forcing frequency, we use a dedicated spatiotemporal filtering. This method enables us to determine both the growth and damping rate for each forcing frequency, even for cases above wind onset where natural waves are also present. From this, we determine the marginal stability curve of the instability with unprecedented resolution. We show that Miles’ model, which is commonly applied to water waves, offers a reasonable description of the growth rate for more viscous liquids. We finally discuss the scaling of the growth rate of the most amplified wave with the liquid viscosity and friction velocity.



\section{Wave growth and attenuation mechanisms}
\label{Sec_growth_attenuation}

The spatial evolution of wind-generated waves results from a combination of energy transferred from the air flow and dissipated in the liquid. We first recall here the modeling of these two contributions. We start from a simple harmonic surface deformation, $\zeta\left ( x,t \right ) = \zeta_0e^{i\left ( kx-\omega t \right)}$, of wave number $k$ and angular frequency $\omega$, propagating over a layer of a liquid of depth $h$, density $\rho_\ell$, and surface tension $\sigma$. In the absence of dissipation and forcing, $\omega$ and $k$ are real and are related by the dispersion relation
\begin{equation}
        \omega^2=\left ( gk+\frac{\sigma k^3}{\rho_\ell} \right) \tanh \left( kh \right)=gk\left ( 1+\frac{k^2}{k_{cap}^2} \right) \tanh \left( kh \right),
\label{eq:SDR}
\end{equation}
where $k_{cap}=2\pi/\lambda_{cap}=\sqrt{{\rho_\ell g}/{\sigma}}$ is the capillary wave number. The energy density (per unit surface) of the wave writes $E = \rho_\ell g \zeta_0^2 [ 1+(k/k_{cap})^2]$. Neglecting nonlinear interactions and surface currents, the transport and growth of the wave energy is governed by~\cite{Kahma_1988,grare2013growth,PEIRSON:2008}
\begin{equation}
\left(\frac{\partial}{\partial t} + c_g \frac{\partial}{\partial x} \right) E = P - D,
\label{eq:EPD}
\end{equation}
with $c_g = \partial \omega / \partial k$ the group velocity, $P$ the local power injected by the the wind and $D$ the local power dissipated by viscosity in the liquid.

The injected power is the work per unit time of the wind stress (pressure and viscous shear stress) applied on the liquid surface, and can be modeled in two ways, depending on the wave amplitude. For very low wave amplitude, typically smaller than a fraction of the viscous sublayer thickness, the wavy shape of the liquid surface does not modify the flow in the air. In this case, only the turbulent pressure fluctuations in the air deform the interface, resulting in an injected power $P$ independent of the wave energy $E$, and hence a linear growth of $E$: this is the starting hypothesis of the Phillips model~\cite{phillips1957generation}. On the other hand, for larger wave amplitude, a feedback of the wave on the air flow takes place, and the wind stress becomes modulated by the wave profile. The resulting injected power $P$ is now proportional to $E$, yielding an exponential growth of the wave energy: this is the Miles model~\cite{Miles_1957}.

In the present paper, we are interested in mechanically generated waves of initial amplitude larger than the viscous sublayer thickness, which places us directly in the second stage of wave growth. Both $P$ and $D$ are therefore proportional to $E$, so the right-hand-side of Eq.~(\ref{eq:EPD}) can be written in the form $P-D = \gamma E$, with $\gamma$ the temporal growth rate. In this framework, the onset of wave growth is defined by $\gamma=0$, when the injected power overcomes the viscous dissipation. 

We first focus on the injected power $P$. Since the flow in the air is necessarily turbulent in our problem, the wave-induced wind stress variations (pressure and viscous shear stress) are conveniently scaled with the applied mean shear stress $\tau = \rho_a u^{*2}$, with $u^*$ the friction velocity and $\rho_a$ the air density. The friction velocity $u^*$ depends on the free-stream wind $U_a$ and the fetch $x$ (distance along which the boundary layer develops), with $u^*/U_a \simeq 0.03-0.05$ in usual laboratory experiments close to the onset of wave generation.

To first order in wave slope, since the velocity of the surface is essentially vertical, the work per unit time of the horizontal component of the stress (shear stress) can be neglected, and only the vertical component (pressure) contributes to the injected power. This contribution  writes  $P = \Re \langle p^\dagger \dot \zeta \rangle$, where $ p^\dagger$ is the complex conjugate of the oscillating component of the pressure along the surface and $\dot \zeta$ the vertical velocity of the surface~\cite{Miles_1957,Janssen_2004}. It therefore depends on the phase relation between the surface elevation and the wave-induced pressure oscillation. Following Miles~\cite{Miles_1957}, we can write the oscillating pressure in the form $p=(\alpha + i \beta) \rho_a u^{*2} k \zeta$ to first order in wave slope, with $\alpha + i \beta$ a non-dimensional parameter coding for this phase shift~\cite{wu2022revisiting}. Only the pressure contribution in phase with the wave slope ($\beta \neq 0$) provides work, yielding a growth rate $\gamma = P / E$ in the form
\begin{equation}
   \gamma = \beta \omega \frac{\rho_a}{\rho_\ell} \left( \frac{u^*}{c} \right)^2.
\label{eq:Miles} % Miles 
\end{equation}

The normalized phase velocity $c/u^*$ is often referred to as the ``wave age'', because young waves are short and have small phase velocity $c$. The key result of Miles theory~\cite{Miles_1957,Lighthill_1962} is that $\beta$ is proportional to the curvature $-U''(z_c)$, with $z_c$ the elevation of the critical layer where the wind velocity $U(z_c)$ matches the phase velocity of the considered wave. This indicates that too small or too large wavelengths, for which $z_c$ is such that $U''(z_c)$ vanishes (in the viscous sublayer or outside the boundary layer), cannot be amplified by this mechanism. The main difficulty in the determination of $\beta$ therefore relies in the detailed knowledge of the velocity profile in the air.

Since the original work of Miles, the experimental verification of the scaling (\ref{eq:Miles}), and the question of the extent to which the parameter $\beta$ can be considered a constant, have been the subject of numerous studies. For naturally growing waves, $\beta$ corresponds to the most amplified wave number, but it is possible to examine its dependence with $k$ using mechanically generated waves. Compilation of field and laboratory experiments suggests that the scaling $(u^*/c)^2$ approximately holds for the air-water interface over a significant range of wave ages $(1<c/u^*<20)$, with an average value $\beta \simeq 32\pm{16}$~\cite{plant1982relationship,Janssen_2004}. This large variability can be ascribed to a number of parameters, such as the dependence with wave number, the presence of currents, the incorrect modeling of wave dissipation for non-laminar flow in the liquid, nonlinear effects (dependence with wave slope~\cite{PEIRSON:2008}), etc.


We now describe the dissipation in the liquid, focusing on the laminar dissipation mechanisms, which are relevant for the viscous liquids considered here. Viscosity introduces a negative imaginary part in the dispersion relation, $\omega = \omega_r + i \omega_i$, yielding a temporal damping rate $\gamma = -D/E = 2\omega_i < 0$ of the wave energy. The full dispersion relation for arbitrary viscosity (without side effects) is given by Lamb~\cite{Lamb} in infinite depth, later improved by LeBlond and Mainardi~\cite{Leblond87} for arbitrary depth. For moderate viscosity (including that considered in the present work), the real part $\omega_r$ remains close to the inviscid dispersion relation (\ref{eq:SDR}), i.e. the phase velocity $c=\omega_r/k$ is only slightly affected by viscosity. The damping rate $\gamma = 2 \omega_i$ can be approximately split in two contributions: bulk dissipation due to the internal shear stress,
\begin{equation}
        \gamma_{\rm bulk} = -4\nu_\ell k^2,
\label{eq:QIT} % Quasi-inviscid Theory
\end{equation}
 and dissipation at the bottom wall due to the shear stress in the Stokes boundary layer of thickness $\delta = \sqrt{\nu_\ell /\omega_r}$ (provided that $\delta$ is much smaller than $h$)~\cite{Lighthill},
\begin{equation}
        \gamma_{\rm BW} = - \frac{ \sqrt{2 \nu_\ell \omega_r} \, k}{\sinh{2kh}}.
\label{eq:QITH} % Quasi-inviscid Theory
\end{equation}
The bulk dissipation is dominant at small wavelength, while the bottom-wall dissipation is dominant for wavelength much larger than liquid depth.

Finally, for waves in a channel of finite width $W$, a third contribution arises from the friction with  the side walls~\cite{hunt1952viscous, tsai2005spatial, peirson2013rain},
\begin{equation}
        \gamma_{\rm SW} = - \frac{ \sqrt{2 \nu_\ell \omega_r}}{W}.
\label{eq:SWD} % side wall dissipation
\end{equation}
In the following, we use the numerically resolved LeBlond-Mainardi solution, noted $\gamma_{\rm LM}$, which has the bulk and bottom-wall approximations (\ref{eq:QIT})-(\ref{eq:QITH}) as asymptotic solutions for large and small $k$ respectively, to which we add the side-wall contribution $\gamma_{\rm SW}$ (\ref{eq:SWD}). The relative magnitude of these contributions, for the specific geometry of our experimental setup, is discussed in Appendix~\ref{Appendix_contributions}.


To summarize, the net temporal growth rate $\gamma$ accessible to experiments is the sum of the positive contribution (\ref{eq:Miles}) and a combination of negative contributions (\ref{eq:QIT})-(\ref{eq:SWD}). In most practical configurations (including the present paper), waves are forced under permanent conditions and develop spatially, so the energy transport equation (\ref{eq:EPD}) reduces to 
$$
\frac{d}{d x} E = \gamma_s E,
$$
with $\gamma_s = \gamma / c_g$ the {\it spatial} growth rate~\cite{Gaster}. Assuming homogeneous forcing and dissipation (independent of $x$), this yields an exponential variation of the energy profile, $E(x) = E_0 e^{\gamma_s x}$, with $\gamma_s > 0$ for growing waves and $\gamma_s<0$ for damped waves. In the following, we perform measurements of $\gamma_s$ by fitting $E(x)$ in the range of exponential growth or decay for various $k$ and friction velocities $u^*$, and discuss the relevance of the various growth and damping rate contributions in the case of viscous liquids.


\section{Experiments and measurements}

\subsection{Experimental setup} \label{sec:exp_setup}

% Figure environment removed

The experimental setup is composed of a rectangular tank of length $L=1.5$~m and width $W=296$~mm, located at the bottom of a wind tunnel -- see~Fig.~\ref{fig:setup} and Refs.~\cite{Paquier_2015,Paquier_2016,aulnette2019wind} for details. The wind velocity $U_a$, measured in the center of the cross-section, ranges from~$0$~to~$7~\mathrm{m~s^{-1}}$. The tank, of depth $h=35~\mathrm{mm}$, is filled with silicon oil (Rhodorsil\textsuperscript{\textregistered}~oil~47~V20 and V50). The experiments are carried out at temperature of $24.0\pm1.0^\circ$C, giving an oil density $\rho_\ell=956~\mathrm{kg~m^{-3}}$ and surface tension $\sigma=
20.7~\mathrm{mN~m^{-1}}$. The corresponding capillary wavelength and capillary-gravity frequency are $\lambda_{cap}= 9.33$ mm and $f_{cap}=18.3$ Hz. Two kinematic viscosities are used: $\nu_\ell=20.5\pm0.4~\mathrm{mm}^2~s^{-1}$ and $\nu_\ell = 50.1\pm0.1~\mathrm{mm}^2~s^{-1}$. In order to minimize wave reflection, two inclined perforated planes of length $85~\mathrm{mm}$ and slope $30^\circ$ are added at both ends of the tank to absorb the wave energy. For visualisation purpose all walls are made of transparent Plexiglas.


Waves are generated using an immersed wave maker, which consists in a stainless steel cylinder of diameter $5~\mathrm{mm}$ and length equal to the tank width. The wave maker is located $125$~mm after the beginning of the tank and defines the origin $x=0$ of our measurements. It is hung by a rectangular frame and oscillated vertically using an electromagnetic shaker (Sinocera~JZK-20) via a crank rod system [Fig.~\ref{fig:setup}(b)]. The upper part of the cylinder is at $z(t) = z_0 + A \cos(\omega t)$, with $z_0 = - 14$~mm the mean depth and $A$ the stroke amplitude. The forcing frequency $f=\omega/2\pi$ ranges from $3$ to $12~\mathrm{Hz}$. This range is below the capillary-gravity crossover frequency $f_{cap}$, indicating that excited waves are in the gravity regime. The corresponding wavelengths range from 150~mm to 15~mm. The frequency range is limited at low frequency by the fact that the small diameter of the cylinder cannot excite efficiently large wavelengths, and at large frequency by the fact that the wavelength and the attenuation length become too small to be measurable.

The stroke amplitude of the wave maker is kept fixed to $A=1~\mathrm{mm}$ in all this study. This forcing induces waves of initial amplitude $\zeta_0$ between 0.1 and 0.4~mm depending on the forcing frequency.  This initial amplitude is small enough to satisfy the linear wave approximation (the maximum wave slope near the wave maker remains below $k\zeta_0 \simeq 0.02$). On the other hand, this initial amplitude is large enough to be directly in the Miles' exponential growth regime: the characteristic surface deformation corresponding to the transition from the Phillips' incoherent wrinkles to the Miles' regular waves is estimated to $\zeta \simeq (0.11-0.14) \delta_\nu$ in Refs.~\cite{Perrard2019,li_shen_2022}, with $\delta_\nu = \nu_a / u^*$ the thickness of the viscous sublayer, which lies here in the range $40-100~\mu$m.


The friction velocity $u^*$, which measures the mean wind shear stress $\tau = \rho_a u^{*2}$ at the surface, is obtained from the wind-induced current $U_s$ that develops at the liquid surface~\cite{Paquier_2015}. The flow in the liquid being laminar (the maximum Reynolds number $U_s h/\nu_\ell$ is less than 70), a stationary Couette-Poiseuille flow of zero flow rate develops, with a surface current $U_s$ given by the stress continuity at the surface,
\begin{equation}
    U_s = \frac{\rho_a u^{*2}h}{4\rho_\ell \nu_\ell}.
\label{eq:surf_vel_current}
\end{equation}
This surface current is of the order of $0.02~\mathrm{m~s^{-1}}$ close to the onset of wave growth, a value typically less than 10\% of the typical phase velocity. The resulting friction velocity deduced from $U_s$ at various $x$ locations is in good agreement with the classical Schlichtling relation for a developing turbulent boundary layer~\cite{Schlichtling}
\begin{equation}
      \frac{u^{*2}(x)}{{U_a}^2}\simeq C\left(\frac{(x+x_0)U_a }{\nu_a}\right)^{-0.2},
\label{eq:Sch} % Schlichting
\end{equation}
with $x_0= 475$~mm the distance between the beginning of the flat plate at the end of the wind-tunnel convergent and the wave maker, and $C=0.029$~\cite{aulnette2019wind}. In the present study, measurements are performed in the interval $x = 0-800$~mm. For the range of wind velocity considered here, Eq.~(\ref{eq:Sch}) gives $u^* / U_a \simeq 0.05$ close to the wave maker, with a decrease of 10\% at the end of the measurement area.



\subsection{Wave measurement}

The surface deformations are determined with micrometer precision using Free-Surface Synthetic Schlieren (FS-SS)~\cite{Moisy09}. This method is based on the analysis of the refracted image of a random-dot pattern through the interface. The dots are $1~\mathrm{mm}$ in diameter (corresponding to 5 pixels) with a density of $60$ dots per $\mathrm{cm^2}$. The pattern is printed on a transparent film located at the bottom of the liquid tank and illuminated from below by a LED panel. Images of the pattern are taken through the transparent upper-wall of the wind tunnel. To reduce parallax errors, images are acquired through a 45$^\mathrm{o}$ mirror located above the wave tank by a camera located $3~\mathrm{m}$ from the liquid surface (Fig.~\ref{fig:setup}b). Images are acquired using a high-speed camera (Photron~FASTCAM~Mini~WX50), fitted with a Canon $85~\mathrm{mm}~f/1.8$ macro lens. Images of dimension of $890 \times 320~\mathrm{mm^2}$ are recorded at a frame rate at least $10$ times larger than the wave maker frequency $f$, for a recording duration of at least 20 wave periods. 


The surface slope field is obtained by computing the apparent displacement field $\delta {\bf r}$ of the dot pattern induced by the surface deformation using an image correlation algorithm~\cite{Moisy09}. In the limit of waves of weak slope and within the paraxial approximation, the displacement field is proportional to the surface height gradient,
$$
\delta {\bf r} = - h^* \nabla\zeta,
$$
with $h^* = 14$~mm an effective distance that includes the surface-pattern distance ($h_p= 48.2~\mathrm{mm}$) and the refraction indices of the liquid and intermediate layers between the surface and the pattern~\cite{Moisy09}. Although the surface height field $\zeta({\bf r},t)$ can be reconstructed by inverting the $\nabla$ operator, here we work directly with the refraction-induced displacement field $\delta {\bf r}$: for a quasi-monochromatic wave propagating in the $x$ direction, $\zeta(x,t) = \zeta_0(x) \cos(kx-\omega t)$, where $\zeta_0(x)$ is a slowly varying amplitude, the displacement is simply proportional to the wave height,
$$
\delta x = - h^* \frac{\partial \zeta}{\partial x} \simeq h^* k \zeta_0(x) \sin(kx-\omega t),
$$
so the spatial growth rate of the wave amplitude $\zeta_0(x)$ can be obtained directly from that of the displacement $\delta x$.

The image correlation algorithm is based on interrogation windows of size 12 pixels (5.2~mm) with 50\% overlap, yielding a spatial resolution of $2.6~\mathrm{mm}$. This is approximately 6 times smaller than the smallest wavelength considered here, $\lambda \simeq 15~\mathrm{mm}$ for $f=12~\mathrm{Hz}$. The resolution in the apparent displacement $\delta x$ is 0.1~pixel (0.04~mm), yielding a resolution in surface slope $k\zeta \simeq 0.003$.


\subsection{Data processing} \label{sec:data_proc}

% Figure environment removed

Figure~\ref{fig:exampleRMSField}(a,b) shows two snapshots of the magnitude of the wave gradient field $\nabla \zeta = - \delta {\bf r} / h^*$ measured at a wind velocity $U_a = 5.0$~m~s$^{-1}$ in the case of an amplified wave ($f=5$~Hz, left) and damped wave ($f=9.5$~Hz, right), for the oil of viscosity $\nu_\ell=20.5~\mathrm{mm^2~s^{-1}}$. From the wave slope in the $x$ direction, we compute the mean square envelope, which writes for a quasi-monochromatic wave as
\begin{equation}
\langle (\partial \zeta / \partial x)^2 \rangle \simeq \frac{1}{2} k^2 \zeta_0^2(x),
\label{eq:RMS_zeta}
\end{equation}
with $\zeta_0(x)$ the slowly varying wave amplitude and $\left\langle \cdot \right\rangle$ the average over time and $y$. Examples of wave slope profiles $\partial \zeta / \partial x$ and their corresponding mean square envelopes are shown as red dashed lines in Fig.~\ref{fig:exampleRMSField}(e-h). In the amplified case, the envelope shows a well defined exponential growth at moderate fetch, followed by a saturation at large fetch [Fig.~\ref{fig:exampleRMSField}(g)]. This saturation, which results from a combination of nonlinear interactions with other waves and the small decrease of the wind shear stress along the developing boundary layer, is not considered in the following. The damped case is more complex [Fig.~\ref{fig:exampleRMSField}(h)]: after a short range of exponential decay, the envelope saturates or even slightly increases at large fetch, because of the presence of other waves of frequencies different from the forcing frequency. These unforced waves are the amplified waves naturally present for a wind velocity above the critical threshold, and may hide the underlying decay of the forced wave, which makes it difficult to accurately measure the damping rate.

To extract the forced waves from the sea of ambient waves present in the field, we perform a spatiotemporal band-pass filtering of the displacement field $\delta x = -h^* \partial \zeta / \partial x$. Figure~\ref{fig:BW}(a) shows the power spectrum $|\delta \hat x|^2(k,\omega)$ (averaged in the $y$ direction) in the case of a damped wave. The displacement field being real, its Fourier transform is symmetric under $(k,\omega) \rightarrow (-k,-\omega)$. The quadrants ($+,+$) and ($-,-$) correspond to waves propagating in the $x>0$ direction, while the quadrants ($+,-$) and ($-,+$) correspond to waves propagating in the $x<0$ direction. In addition to the injected energy around the forcing $\pm (k_0, \omega_0)$ (red ellipses), energy is also visible in the reflected waves $\pm (-k_0, \omega_0)$, in harmonics, and along the viscous dispersion relation (white dashed lines). Energy along the dispersion relation is systematically found for a wind velocity above the onset, and is due to the natural waves growing from initial disturbances and amplified by the wind.

To filter out the undesired wave components, we compute the space-time Fourier transform $\delta \hat x(k,\omega)$, convolute it with a Butterworth filter kernel
\begin{equation}
        G\left(k,\omega\right)=\frac{H(k\omega)}{1+\left[\left(\frac{k-k_0}{\Delta k}\right)^2 + \left(\frac{\omega-\omega_0}{\Delta\omega}\right)^2\right]^{N/2}},
\label{eq:Butterworth}
\end{equation}
and compute the inverse Fourier transform to reconstruct the filtered displacement field $\delta \bar x(x,t)$, and hence the filtered wave slope $\partial \bar \zeta / \partial x$. We choose a filter kernel centered around the forcing $\pm (k_0, \omega_0)$, with a spectral bandwidth ($\Delta k$, $\Delta \omega$) and a decay characterized by the exponent $N$, chosen here equal to 6. The Heaviside operator $H(k\omega)$ filters the reflected waves ($k\omega<0$). A narrow temporal bandwidth $\Delta \omega$ is chosen to select precisely the forced waves, but a wider spatial bandwidth $\Delta k$ is chosen to correctly measure the spatial decay of the wave. The temporal bandwidth is chosen as $\pm 3$ spectral points $\pi/T_{acq}$ (with $T_{acq}$ the acquisition duration) surrounding the forcing frequency $\omega_0$. The spatial bandwidth $\Delta k$ is chosen equal to $k_0$, so the smallest measurable damping length is of the order of the wavelength. The resulting filtered spectrum $| G(k,\omega) \delta \hat x|^2(k,\omega)$ in Fig.~\ref{fig:BW}(b) shows that the energy of the forced wave is correctly selected by these parameters.

% Figure environment removed

The snapshots for the filtered instantaneous wave gradient $\nabla \bar \zeta = - \delta \bar {\bf r} / h^*$, shown in Fig.~\ref{fig:exampleRMSField}(c,d), are very close to the raw data in the amplified case, but are significantly modified in the damped case, indicating that the naturally amplified waves are efficiently filtered out by our procedure. The corresponding wave slope profiles and envelopes are also shown as continuous black lines in Fig.~\ref{fig:exampleRMSField}(e-h). We finally measure the spatial growth rate $\gamma_{s}$ by fitting the mean-square filtered wave slope by an exponential,
\begin{equation}
    \langle  (\partial \bar \zeta / \partial x)^2 \rangle \propto e^{\gamma_{s}x}.
\label{eq:RMS_exp} 
\end{equation}
The fit is performed in the range $x \in [1.8 ,6] \lambda$, indicated as dots in Fig.~\ref{fig:exampleRMSField}(g,h). Taking into account the parameters of the spatiotemporal filtering and the extent of the fitting interval, the overall uncertainty on $\gamma_s$ is less than 8\%.



\section{Results}
\label{secResultsDisccution}


\subsection{Phase velocity}

To compare the spatial growth rates $\gamma_s$ determined experimentally to the temporal growth rates $\gamma$ predicted theoretically, it is necessary to convert the measured spatial variations of the wave energy to the corresponding temporal variations. In the limit of small viscosity, the ratio between the two growth rates is given by the group velocity~\cite{Gaster}, which cannot be directly measured in our system. We must therefore use the group velocity calculated with the dispersion relation of the free waves in a viscous liquid, i.e. including viscous effects but without wind, as discussed in Sec.~\ref{Sec_growth_attenuation}. To check to what extent we can rely on this viscous free-wave dispersion relation, we first measure here the phase velocity of the waves and compare it with this prediction.

% Figure environment removed

Figure~\ref{fig:c_c}(a,b) shows the phase velocity $c_{exp} = \omega /k$ for the two liquid viscosities as a function of the wave number for various wind velocities $U_a$ (here $k$ is simply measured from the distance between wave crests, ensuring high accuracy). The data are compared to the inviscid prediction (\ref{eq:SDR}) in solid line, and to the numerically computed viscous prediction of LeBlond and Mainardi~\cite{Leblond87} including depth effects in dashed line. These predictions are very close in our range of frequencies ($3-12$~Hz), within 2\%, so we will simply consider the inviscid prediction in the following. At low frequency, the inviscid phase velocity tends to the non-dispersive shallow water limit $\sqrt{gh} \simeq 0.58$~m/s, while the viscous prediction falls to 0 when the thickness of the Stokes boundary layer becomes of the order of the liquid depth (this cutoff is well below the  frequencies relevant to our study).

For our range of frequencies, the measured phase velocity without wind, $c_{0-exp}$, is almost indistinguishable from the prediction (either inviscid or viscous). In the presence of wind, it still remains close to the prediction, to within 3\% at $\nu_\ell=20.5~\mathrm{mm^2~s^{-1}}$ and 8\% at $\nu_\ell=50.1~\mathrm{mm^2~s^{-1}}$. We can therefore use with reasonable accuracy the predicted group velocity computed from the inviscid dispersion relation (\ref{eq:SDR}) to convert spatial growth rates to temporal ones in the following.

It is interesting to analyze in more detail the small discrepancies between the measured phase velocity with and without wind ($c_{exp}$ and $c_{0-exp}$, respectively). The difference $c_{exp} - c_{0-exp}$ is plotted in Fig.~\ref{fig:c_c}(c,d), with uncertainties shown as shaded areas. It can be explained by a combination of two antagonistic effects~\cite{valenzuela1976growth}, the importance of which depends on the liquid viscosity. The first one corresponds to an increase of the phase velocity by the surface current induced by the wind shear stress, which dominates at small viscosity, and the second one corresponds to a decrease of the phase velocity due to the wave-induced aerodynamic pressure, which dominates at large viscosity.

The increase in phase velocity by surface current is simply given by $U_s$ for short wavelengths [see Eq.~(\ref{eq:surf_vel_current}), with $U_s \simeq 0.02$~m~s$^{-1}$ close to the wave onset], but is less pronounced for long wavelengths because the perturbation they induce penetrates deeper into the liquid, where the current is lower. This transport effect therefore depends on the mean velocity profile in the liquid. For the Couette-Poiseuille profile considered here~\cite{Paquier_2015}, the modification in the phase velocity has been computed by Lilly, and can be found in the appendix of the paper by Hidy and Plate~\cite{hidy1966wind},
\begin{equation}
\Delta c_w(k) = U_s\left(1-\frac{1+2\cosh{2kh}}{kh\sinh{2kh}} + \frac{3}{2k^2h^2} \right).
\label{eq:celerity_current}
\end{equation}

The second effect is the slowdown of the phase velocity by the variations of the aerodynamic pressure over the wave profile, and can be explained qualitatively as follows. The phase velocity of gravity waves is governed by gravity, i.e. by the weight of the liquid deformation. In the presence of wind, a fraction of this weight is carried by the wave-induced aerodynamic pressure (Bernoulli effect), of the order of $\rho_a U_a^2 k \zeta$ -- suction over the wave crests and over-pressure on the wave troughs. The surface deformation therefore experiences an effective reduced gravity, yielding a reduced phase velocity, which can be written as~\cite{plant1980phase}
\begin{equation}
c(k) =\sqrt{c_0(k)^2-\frac{\rho_a}{\rho_\ell}U(z_k)^2},
\label{eq:celerity_correction}
\end{equation}
with $U(z_k)$ the wind velocity evaluated at a characteristic elevation that depends on the wave number $k$. We evaluate this effect by considering a classical logarithmic profile $U(z)=u^*[\kappa^{-1}\log(zu^*/\nu_a)+B]$ (with $\kappa= 0.41$ and $B=3.5$) computed at the elevation $z_k = \epsilon/k$ with $\epsilon \simeq 0.23$~\cite{aulnette2022kelvin}.

The correction to the phase velocity, combining the surface current effect (\ref{eq:celerity_current}) and the aerodynamic pressure effect (\ref{eq:celerity_correction}), is plotted in Fig.~\ref{fig:c_c}(c,d). It provides a reasonable description of the data: At low viscosity and large wave numbers, the waves are essentially transported by the surface current, yielding a phase velocity larger than the inviscid prediction. On the other hand, waves at low wave numbers are systematically slower than the free-wave celerity, as a result of the dominant aerodynamic pressure effect.

To summarize, the phase velocity of the waves in presence of wind can be correctly described by the finite-depth inviscid phase velocity, by including the corrections due to the surface drift and the aerodynamic pressure. However, in the range of liquid viscosities and wind velocities considered here, these corrections remain small: for a wind velocity close to the wave onset, the measured phase velocity matches the free-wave inviscid prediction to within 10\%. By extension, we can consider that the group velocity is also only marginally affected by the surface current and aerodynamic pressure, which allows us to use the free-wave inviscid prediction $c_g$ derived from Eq.~(\ref{eq:SDR}) to infer the temporal growth rate $\gamma$ from the measured spatial growth rate $\gamma_s$.


\subsection{Spatial growth rates} 
\label{Sec_Growth_rate}

% Figure environment removed


Figure~\ref{fig:beta_sm}(a,b) shows the measured spatial growth rate $\gamma_{s}$ as a function of the measured wave number, for different wind velocities. As expected, without wind $\gamma_s$ is negative for all $k$. The corresponding damping rate is compared to the bulk prediction $\gamma_s = -4 \nu_\ell k^2 / c_g$ (solid line) and to the numerically solved LeBlond-Mainardi prediction~\cite{Leblond87} (including bulk and bottom-wall dissipation) completed with side-wall dissipation (\ref{eq:SWD}) (dashed line). The large wave numbers ($k>70$~m$^{-1}$) are equally well described by the two models, but the smaller $k$ are much better described by the refined model, especially at large viscosity, highlighting the significant effect of the bottom dissipation at small $k$ (see Appendix~\ref{Appendix_contributions}).

\begin{table}[b!]
\caption{Critical parameters at the onset of wave generation for the two liquid viscosities.}
\begin{center}
%\begin{tabular}{|c|c|c|c|c|c|}
\begin{tabular}{p{2.5cm}p{2.5cm}p{3cm}p{2.5cm}p{2.5cm}}
    \hline
    \hline
     $\nu_\ell~(\mathrm{mm^2~s^{-1}})$ & $U_{ac}~(\mathrm{m~s^{-1}})$ & $u_c^*~(\mathrm{m~s^{-1}})$ & $k_c$ (m$^{-1}$)  & $f_c$~(Hz) \\ 
     \hline
     $20.5$ & $3.18~\pm 0.04$ & $0.173~\pm0.005$ & $80 \pm 5$ & $4.35~\pm 0.1$ \\ 
     %  \hline
     $50.1$ &  $5.92~\pm 0.04$ & $0.303~\pm0.005$ & $100 \pm 6$ & $4.6~\pm 0.1$ \\ 
    \hline
    \hline
\end {tabular}
\end{center} 
\label{Table-onset}
\end{table}


As the wind velocity is increased, the waves become gradually less damped and, above a critical wind $u^*_c$, the growth rate becomes positive in a finite range of wave numbers. From the set of growth rate curves $\gamma_s(k)$ at various velocities, we can compute the marginal stability curve $u^*(k)$, defined as the minimum friction velocity required for a given wave number $k$ to become unstable. It is plotted in Fig.~\ref{fig:beta_sm}(c,d) for the two liquid viscosities. Here the friction velocity $u^*$ is deduced from the free-stream velocity $U_a$ by using Eq.~(\ref{eq:Sch}) for $x$ close to the wave maker.  The lowest point  $(k_c, u^*_c)$ of the marginal stability curve defines the critical wave number and critical friction velocity, which are summarized in table \ref{Table-onset} for the two liquid viscosities. The critical wave numbers correspond to wavelengths $\lambda_c \simeq 60-80$~mm, much larger than the capillary wavelength $\lambda_{cap}= 9.33$~mm.

As the liquid viscosity is increased, the critical friction velocity strongly increases, but the critical wave number is marginally modified. Increasing $u^*$ shifts the most amplified wave number (shown as blue triangles) to larger values, and rapidly widens the domain of unstable wave numbers. The low-$k$ branch of the stability curve, for which the bottom-wall and the bulk dissipation are of comparable importance (see Appendix A), shows a weak dependence with $u^*$. On the other hand, the large-$k$ branch, which is essentially governed by the bulk dissipation, strongly varies with $u^*$. We note that this higher branch is defined with high accuracy, because $\gamma_s$ sharply crosses 0, but that the lower branch is more difficult to measure, because $\gamma_s$ shows little variations with $u^*$ for small $k$.


%%%%%%%%%%%%%%%%%%%%%%%%%%%%%%%%%%%%%%%%%%%%%%%%%%%%%%%%%%%%%%%%%%%%%%%%%%%%%%%%%%%%%%%%
\subsection{Comparison with Miles' scaling}
\label{Sec_Miles}

We now investigate to what extent Miles' model, usually applied to the air-water interface, also applies to the case of the more viscous liquids considered here. Following Eq.~(\ref{eq:Miles}), we plot in Fig.~\ref{fig:beta_uc}(a,b) the temporal growth rate $\gamma= \gamma_s c_g$ normalized by the wave frequency $\omega$ as a function of the inverse wave age $u^*/c$  (only amplified waves are considered here). In this traditional representation, ``young'' waves (high $k$, low $c$) are one the right and ``old'' waves (small $k$, large $c$) are on the left. However we note that, strictly speaking, there is wave aging in our experiments: each data point corresponds to a single wave excited at a given frequency.

% Figure environment removed

% Figure environment removed

Although individual curves for each $u^*$ fail to follow the $(u^*/c)^2$ scaling, the upper envelope of the set of curves is compatible with this scaling. The strong departure at small and large wave age results from a combination of dissipation effects (dominated by the bulk dissipation for large $u^*/c$ and bottom-wall dissipation for small $u^*/c$) and the selective amplification of the Miles' mechanism: wave numbers $k$ far from the most amplified wave numbers correspond to a critical height where the curvature of the mean velocity profile $U''(z)$, and consequently  the growth rate, becomes zero.

To further test the $(u^*/c)^2$ scaling, it is therefore necessary to subtract the contribution from the viscous dissipation. In order not to rely on a specific dissipation model, we use the measured damping rate $\gamma_0$ without wind as the best estimate for the global dissipation. We therefore plot the normalized net growth rate $(\gamma - \gamma_0)/\omega$ in Fig.~\ref{fig:beta_uc}(c,d), which is now consistent with the scaling $(u^*/c)^2$, at least for large wind velocity and large wave numbers (large $u^*/c$). The best fit with Eq.~(\ref{eq:Miles}) gives the average value $\bar \beta \simeq 45$ with an uncertainty range $\beta \simeq 23-75$ (shown in grey area).


The significant scatter in the dimensionless growth rates is not specific to our experiments performed with viscous oils, and is also present in experiments in the air-water case. In Fig.~\ref{fig:comparaisonMiles} we superimpose our normalized net growth rates to the classical Plant's compilation of laboratory and field measurements~\cite{plant1982relationship}, reproduced from Janssen~\cite{Janssen_2004}.  The correct agreement between our data in viscous oils and the data in water supports the robustness of the scaling $(u^*/c)^2$ at large $u^*/c$, with a comparable scatter.  The solid line shows the prediction of Miles~\cite{Miles1959_part2}, which gives the asymptotic scaling $(u^*/c)^2$ at large $u^*/c$, relevant to our ``young'' (slow) waves, but a lower growth rate for ``older'' (faster) waves when the phase velocity becomes of the order of the wind velocity.


\subsection{Maximum growth rate}

We now turn to the maximum growth rate $\gamma_{max}$ (growth rate of the most amplified wave number) and its dependence on the liquid viscosity. Figure~\ref{fig:beta_ut}(a) shows $\gamma_{max}$ as a function of the friction velocity $u^*$ for the two viscosities. By definition $\gamma_{max}$ increases with $u^*$ and crosses 0 at the critical friction velocity $u^*_c$ (given in table~\ref{Table-onset}). We note that negative values of $\gamma_{max}$ can be defined for $u^* < u^*_c$ because slightly below $u^*_c$ the growth rate shows a local (negative) maximum, corresponding to the least damped wave number.

% Figure environment removed

Given the various sources of dissipation, with distinct scaling laws with respect to $\nu_\ell$ (bulk dissipation in $\nu_\ell$ and wall dissipation in $\nu_\ell^{1/2}$), a unique scaling for $\gamma_{max}$ with $\nu_\ell$ cannot be derived in general. However, an approximate trend can be proposed, at the cost of a number of approximations: (i) we ignore the $k$ dependence of the parameter $\beta$ in Eq.~(\ref{eq:Miles}), which is acceptable close to the most amplified $k$, and consider the average value $\bar \beta=45$; (ii) we consider that waves are in the deep-water gravity regime, $\omega = \sqrt{gk}$, which again is acceptable because $k_c / k_{cap} < 0.15$ and $\tanh(k_c h) \simeq 0.99$; (iii) we consider the bulk dissipation $\gamma_{\rm bulk} = -4 \nu_\ell k^2$ as the only source of dissipation. This third approximation is the most questionable in our experiments, with a bulk-bottom transition at $k\simeq 70$~m$^{-1}$ (see Appendix A), which is close to the critical wave number $k_c$. Under these approximations, the temporal growth rate can be modeled as
\begin{equation}
\gamma \simeq \bar \beta s \frac{u^{*2} k^{3/2}} {g^{1/2}} - 4 \nu_\ell k^2,
\label{eq:gam_bulk}
\end{equation}
with $s = \rho_a / \rho_\ell$. Solving for $\partial \gamma / \partial k = 0$ yields the maximum growth rate
\begin{equation}
    \gamma_{max}=\frac{3^3}{2^{14}}  \bar \beta^4 s^4 \frac{u^{*8}}{g^2\nu_{\ell}^3}.
\label{eq:beta_max_gravity}
\end{equation}


To verify this scaling, we plot in Fig.~\ref{fig:beta_ut}(b) the combination $\gamma_{max} \nu_\ell^3$ as a function of $u^*$. The data for the two viscosities approximately gather along the power law $u^{*8}$, with variations of the same order as the uncertainty on the parameter $\beta$ found previsouly (the black line gives the average value $\bar \beta = 45$, and the shaded area shows the same range of $\beta$ as in Fig.~\ref{fig:beta_uc}). Although the data for the two viscosities do not overlap, because of the very limited range of accessible $u^*$  (at most twice the critical value $u^*_c$), the high-$\nu_\ell$ curve reasonably extends the low-$\nu_\ell$ curve, which is consistent with the scaling law (\ref{eq:beta_max_gravity}).

The variation of $\gamma_{max}$ with $u^*$ has been the subject of many studies, but only in the air-water case~\cite{Kawai1979generation,tsai2004stability,creamer1992surface,larson1975wind}. Approximate power laws $\gamma_{max} \sim u^{*n}$ are reported, but with significantly smaller exponents $n$, in the range 1.5--3.5. The asymptotic scaling $n=8$ may be difficult to access experimentally in water experiments because nonlinear effects (wave saturation, frequency downshift, wave breaking) rapidly appear as $u^*$ is increased, even close to the onset $u^*_c$.


\subsection{Critical friction velocity}

We finally consider the influence of the liquid viscosity on the critical friction velocity $u^*_c$ for the growth of mechanically-generated waves. This critical friction velocity cannot be directly inferred from Eq.~(\ref{eq:gam_bulk}), based on the sole bulk dissipation, because it predicts a range of unstable $k$ starting from $k=0$ as soon as $u^*>0$. This directly follows from the approximation $\beta(k) \simeq \bar \beta$ and the omission of other dissipation sources. In our experiments, the dissipation at low $k$ being dominated by the bottom wall friction (see Appendix A), we can modify Eq.~(\ref{eq:gam_bulk}) by still assuming a constant coefficient $\beta$, while now including  the Leblond-Mainardi damping rate (including bulk and bottom wall contributions)
\begin{equation}
\gamma \simeq \bar \beta s \frac{u^{*2} k^{3/2}} {g^{1/2}} + \gamma_{\rm LM}(k).
\label{eq:gam_lm}
\end{equation}
The critical friction velocity $u_c^*$, obtained by solving numerically $\gamma=0$, is plotted in Fig.~\ref{fig:u_viscosite} as a function of $\nu_\ell$. The numerical solution, shown again for the average value $\bar \beta=45$ (black line) and  its uncertainty range (shaded area), is in good agreement with the measured $u_c^*$ for the two values of $\nu_\ell$. Interestingly, over this range of viscosity, this numerical solution is locally well described by the approximate power law $u^*_c \sim \nu_\ell^{0.5}$. Although this power law cannot be directly derived from Eq.~(\ref{eq:gam_lm}), it is compatible with the fact that the most unstable wave number $k_c$ shows little variation with $\nu_\ell$, and that  solving $\gamma=0$ in Eq.~(\ref{eq:gam_bulk}) for fixed $k$ indeed yields $u^*_c \sim \nu_\ell^{1/2}$.

We close this discussion by comparing the critical friction velocity for natural waves and for mechanically generated waves. In Fig.~\ref{fig:u_viscosite} we add $u^*_c$ for natural waves as measured in the same setup in Refs.~\cite{Paquier_2016,aulnette2022kelvin}, using silicon oils and water-glycerol mixtures. Here $u^*_c$ is defined as the transition from the small-amplitude wrinkles (Phillips regime) to the exponentially growing waves (Miles regime). This natural-wave threshold is larger than the forced-wave threshold, but shows a shallower dependence with the liquid viscosity, $u^*_c \sim \nu_\ell^{0.2}$, with a very similar numerical prefactor (within 5\%) for the two types of fluids.  

The two-stage scenario provides a natural framework to interpret this difference: 
Mechanically-generated waves grow from a disturbance of initial amplitude sufficient to disturb the air flow and trigger the Miles instability mechanism. On the other hand, natural waves grow from a base state defined by the low-amplitude incoherent wrinkles excited by the turbulent pressure fluctuations in the air, so they require a larger wind velocity for the wrinkle amplitude to disturb the flow and trigger the instability. More specifically, for $u^* < u^*_c$, the wrinkle amplitude results from the balance between the power injected by the turbulent pressure fluctuations and the viscous dissipation in the liquid, $\zeta_w \simeq  s \delta (u^{*3}/g \nu_\ell)^{1/2}$, with $\delta$ the boundary layer thickness. Assuming a Phillips-Miles transition when the wrinkle amplitude $\zeta_w$ becomes of the order of the viscous sublayer thickness $\delta_\nu = \nu_a / u^*$~\cite{Perrard2019,li_shen_2022}, we obtain $u^*_c \sim \nu_\ell^{1/5}$, in good agreement with the data in Fig.~\ref{fig:u_viscosite}. The strong difference between these two scalings, $u^*_c \sim \nu_\ell^{1/2}$ for forced waves and $u^*_c \sim  \nu_\ell^{1/5}$ for natural waves, emphasizes the difficulty in defining a unique onset for the wave generation problem, and the interest of varying the liquid viscosity to gain more insight in this problem.

% Figure environment removed



\section{Conclusion}
\label{sec:conclusion}

In this paper we investigated the wind-induced growth of mechanically generated waves over liquids of viscosity 20 and 50 times larger than that of water, with the aim to gain insight into the scaling of the growth rate in the exponential regime. This growth rate has been extensively studied in the literature but only on the air-water configuration. Varying the liquid viscosity offers an interesting opportunity to test existing theories and shed new light on this problem.

Using Free-Surface Synthetic Schlieren with a spatio-temporal filtering provides wave slope measurements with a very good resolution, and the possibility to access the growth and damping rates of the forced waves even when masked by other naturally growing waves. From these measurements we reconstruct the marginal stability curves with unprecedented accuracy. Once the viscous dissipation is subtracted, our measurements show correct overall agreement with Miles' scaling, with a significant scatter comparable to that of experiments performed in water.  Our analysis suggests a maximum growth rate (growth rate of the most amplified wave number) that scales as $u^{*8} \nu_\ell^{-3}$, in reasonable agreement with our data.

Our work raises the question of the origin of the critical friction velocity $u^*_c$ for wave growth, and its dependence with liquid viscosity $\nu_\ell$. For mechanically generated waves, provided that the initial wave amplitude is not too small, the system is directly in the second stage of wave growth (exponential amplification of the wave energy). In this case, the threshold is governed by the balance between the power injected by the wind and the dissipation in the liquid. On the other hand, for natural waves, a larger threshold $u^*_c$ is found, because the wave instability is triggered from the low-amplitude incoherent wrinkles excited by the turbulent pressure fluctuations in the air~\cite{Perrard2019,li_shen_2022}. Previous works~\cite{Paquier_2016,aulnette2022kelvin} suggested a dependence $u^*_c \sim \nu_\ell^{1/5}$ for natural waves, while the present work, currently available for two values of the liquid viscosities only, suggests a dependence $u^*_c \sim \nu_\ell^{1/2}$ for mechanically-generated waves. Extending these results to arbitrary liquid depth and to a larger range of viscosities, including the case of water, would be valuable in order to clarify this difference.


\begin{acknowledgments}

We are grateful to F. Charru, J. Magnaudet, R. Mathis, F. Burdairon, M. Aulnette and S. Perrard for fruitful discussions, and to P. Balondrade and J. Sant-Anna for assistance with measurement and data processing. We thank A. Aubertin, L. Auffray, J. Amarni and R. Pidoux for experimental help. This work was supported by the project ``ViscousWindWaves'' (ANR-18-CE30-0003) of the French National Research Agency.
\end{acknowledgments}

\appendix

\section{Relative importance of the various contributions to the wave dissipation}
\label{Appendix_contributions}

The damping rate of waves at the surface of a viscous liquid in a container, provided that the thickness of the boundary layers is much smaller than the container dimension, can be split in three contributions:  bulk ($\gamma_{\rm bulk}$), bottom-wall ($\gamma_{\rm BW}$) and side-wall ($\gamma_{\rm SW}$), given by Eqs.~(\ref{eq:QIT}), (\ref{eq:QITH}), (\ref{eq:SWD}) respectively. The first two contributions are contained in the finite-depth damping rate $\gamma_{\rm LM}$ for arbitrary viscosity derived by LeBlond and Mainardi~\cite{Leblond87}, which has $\gamma_{\rm bulk}$ and $\gamma_{\rm BW}$ as asymptotic solutions at low viscosity for large and small $k$ respectively.

% Figure environment removed

To estimate which contributions dominate in our experiments (tank depth $h=35$~mm, width $W=296$~mm), we plot in Fig.~\ref{fig:gs} the spatial damping rates $-\gamma_s = -\gamma / c_g$ from the numerically resolved Leblond-Mainardi dispersion relation ($\gamma_{\rm LM}$) and from the three separate contributions ($\gamma_{\rm bulk}$, $\gamma_{\rm BW}$, $\gamma_{\rm SW}$) for the two liquid viscosities $\nu_\ell = 20.5$ and 50.1~$\mathrm{mm^2~s^{-1}}$. In the range of accessible wave numbers, shown as gray bars, the bulk dissipation $\gamma_{\rm bulk}$ (\ref{eq:QIT}) is the dominant contribution for $k >70$~m$^{-1}$, while the bottom-wall contribution $\gamma_{\rm BW}$ becomes significant for lower $k$. The side-wall contribution $\gamma_{\rm SW}$, shown in dashed blue line, is never dominant, but it contributes approximately 20\% to the wall dissipation at low $k$.


\bibliography{Biblio_Viscous_Waves}

\end{document}
