 	\section{Derivation of the time-fractional FENE-type system} \label{Sec:Derivation}
	


 

In this section we derive the time-fractional Navier--Stokes--Fokker--Planck model, admitting both Hookean and FENE-type bead-spring-chains. The classical Hookean bead-spring-chain model is derived from a system of stochastic differential equations; see, for example, the articles \cite{suli2018mckeanvlasov,barrett2007existence} and the thesis \cite[Section 1.3]{ye2018numerical}. The time-fractional Fokker--Planck equation is derived from a Langevin equation in \cite{magdziarz2009stochastic}. We shall emphasize the differences in the derivation of the time-fractional Navier--Stokes--Fokker--Planck model and examine the steps where the time-fractional derivative is introduced. 

In this work, the time-fractional derivative of order $\alpha \in (0,1)$ is understood in the sense of Riemann--Liouville and is given by 
\begin{equation} \label{Eq:RL} 
	\pta w(t) := \pt \int_0^t \frac{(t-s)^{-\alpha}}{\Gamma(1-\alpha)} w(s) \ds,\end{equation} 
where $\Gamma(\alpha):=\int_0^\infty t^{\alpha-1}e^{-t}\dt$ is Euler's Gamma function.



%Later in this section, we shall state suitable assumptions concerning the quantities that appear in the upcoming PDE model and will supplement the system with appropriate initial and boundary conditions.





\subsection{Derivation} We shall idealize each polymer molecule as a pair of massless beads connected by a massless elastic spring. It is assumed that the resulting, so called, \textit{dumbbell} is suspended in a Newtonian solvent, whose motion is governed by the incompressible Navier--Stokes equations for the velocity $u$ and the pressure $p$ of the fluid. Let us denote the position vectors (with respect to an arbitrary, but fixed, reference point in $\mathbb{R}^3$) of the centers of mass of the two beads at time $t$ by $x_i(t) \in \R^3$ for $i \in \{1,2\}$. At time $t$, the center of mass of the dumbbell is at $x_c(t):=\frac12\big(x_1(t)+x_2(t)\big)$ and the elongation (or conformation) vector pointing from $x_1(t)$ to $x_2(t)$ is $q_1(t):=x_2(t)-x_1(t)$. We denote the vector pointing in the opposite direction by $q_2(t):=-q_1(t)$ and assume that $q_1(t)$ and $q_2(t)$ are contained, for all $t \geq 0$, within a given convex open set $D \subset \R^3$ that satisfies $0 \in D$, and $-q \in D$ whenever $q\in D$.

Three kinds of force act on the $i$-th bead in the dumbbell suspended in the fluid: a drag force (Stokes drag) arising from the motion of the bead through the solvent, an elastic spring force, and a random force, which is assumed to be Brownian, modelling the random collisions of the bead with the molecules of the surrounding Newtonian solvent. As each of the two beads is assumed to be massless, Newton's second law implies that
\begin{equation} \label{Eq:Newton} \text{Drag Force}_i +\text{Spring Force}_i + \text{Brownian Force}_i =0,\quad i=1,2.\end{equation} 
To define the drag force 
 acting on the $i$-th bead of the dumbbell suspended in the fluid, we apply  Stokes' law and get
\begin{eqnarray*}
- \zeta\Big(\ddt x_i(t)-u(x_i,t)\Big),\quad i \in \{1,2\}.
\end{eqnarray*} Here $\zeta$ is the friction coefficient and $u(\cdot,\cdot) $ stands for the fluid velocity. We note that $\zeta$ carries the SI unit $[\text{kg}/\text{s}] $ and is linear in the dynamic viscosity $\eta$. For a sphere of radius $R$ it reads $\zeta = 6 \pi R \eta$.  


The elastic spring force $F:D \to \R^3$ of the spring connecting the two beads is $$F(q):=HU'(\tfrac12 |q|^2)q,$$ where $U$ is a nonnegative continuously differentiable potential and $H$ is the spring constant having the SI unit $[\text{kg}/\text{s}^2]$.
In the case of the Hookean dumbbell model the spring force is linear and is given by $F(q)=Hq$ with $q\in D=\R^3$ and the corresponding potential is  $U(s)=s$ for $s \in [0,\infty)$, while in the case of a classical FENE model one has, instead, $$D=B_{|q_\text{max}|}(0), \quad F(q)=\frac{Hq}{1-|q|^2/|q_\text{max}|^2}, \quad U(s)=-\frac{|q_\text{max}|^2}{2}\ln\!\bigg(1-\frac{2s}{|q_\text{max}|^2}\bigg)$$ for $q\in D$ and $s\in [0,\tfrac{1}{2}|q_\text{max}|^2)$, where $B_{|q_\text{max}|}(0)$ is an open ball in $\R^3$ with radius $|q_\text{max}|$ centered at the origin and $|q_\text{max}|>0$ is a strict upper bound on the maximal extension to which a dumbbell can be stretched.

The Brownian force acting on the $i$-th bead at time $t$ is denoted by $B_i(t)$  and is formally defined by 
$$B_i(t):= \sqrt{2k_B\mu_T\zeta}\, \frac{\dd W_i(t)}{\text{d}t },\quad i \in \{1,2\},$$
where $k_B$ is the Boltzmann constant in $[\text{kg m}^2 /(\text{s}^2 \text{K})]$,  $\mu_T$ denotes the  temperature in $[\text{K}]$, and $W(t)=\big(W_1(t),W_2(t)\big)^{\mathrm{T}}$ is a vector of two independent Wiener processes. 
As each Wiener process is distributed according to $\mathcal{N} (0,t)$, we find that the SI unit for $ \frac{\dd W_i(t)}{\text{d}t }$ is $[1/\sqrt{\text{s }}]$.


By introducing the notation
\begin{equation} \label{Def:Vectorb} X(t):=\begin{pmatrix} x_1(t) \\ x_2(t) \end{pmatrix}, \quad b(X(t),t):=\begin{pmatrix} u(x_1(t),t)+\zeta^{-1} F(q_1(t)) \\ u(x_2(t),t)+\zeta^{-1} F(q_2(t)) \end{pmatrix},\end{equation}
and dividing \cref{Eq:Newton} by $\zeta$, we get as impulse balance
 the so-called Langevin equation, an It$\hat{\rm o}$ stochastic differential equation of the form
\begin{equation*} %
\dd X(t) = b(X(t),t) \dt +   \sqrt{\frac{2k_B\mu_T}{\zeta}}       \dd W(t), \quad t \geq 0. \end{equation*}
%As initial condition, we set $X(0)=0$.\footnote{\color{red} I included the initial condition here (not yet done for Y) however setting the initial condition for X to zero results in the delta for psi) ??? and thus there would be no freedom in setting $psi_0$ anymore ???
%but I might be completely wrong - on the other hand not fixing X to zero at initial time is also no way out}
%Considering the integral form and introducing the time-average, 
The partial differential equation describing the evolution of the probability density function 
%
%first moment
%(i.e. the mathematical expectation) 
$\tilde\psi:=\tilde\psi(x_1,x_2,t)$
%:=\langle X(t) \rangle$ 
%
of the random variable $X(t)$ is the standard Fokker--Planck equation \cite{pavliotis2014stochastic}:
\begin{equation} \label{eq:stFP}
\pt \tilde\psi= \sum_{i=1}^2 \left( -\div_{x_i}\!\big( b_i((x_1^{\rm{T}},x_2^{\rm {T}})^{\rm T} 
,t)\, \tilde\psi \big)  + \frac{k_B\mu_T}{\zeta} \Delta_{x_i} \tilde\psi \right),
\end{equation}
with $x_1$ and $x_2$ considered to be column-vectors in $\mathbb{R}^3$.
%, see also \cite[Equation (11)]{sokolov2006field} and \cite[Proof of Theorem 1]{magdziarz2009stochastic} for a computation of the moments to the time-fractional Fokker--Planck equation.



At this point, we deviate from the usual derivation of the Fokker--Planck equation for the evolution of the probability density function of the stochastic process $X$ 
%(see, \cite[Section 5.3.2]{pavliotis2014stochastic}), 
and introduce a subordination of the Langevin equation. This allows us to model trapping events to the motion of the particles. For $\alpha \in (0,1)$,  let $U_\alpha (\cdot)$ be an $\alpha$-dependent subordinator %for which the Laplace transform of its probability density function is 
with expectation %Laplace transform 
$\mathbb{E}[e^{-\lambda U_\alpha(\tau)}]= \text{exp}( -\tau \Phi_\alpha(\lambda) )$, where
\begin{equation*}
\Phi_\alpha(\lambda ) := \tau_0^{\alpha -1} \lambda^\alpha
\end{equation*}
is the so-called Laplace exponent and $\tau_0$ is a characteristic time-scale (to be fixed). We note that the limiting value of $\alpha =1 $ results in the standard integer-order case. The inverse subordinator $S_\alpha ^t $, defined as the first-passage time of $U_\alpha$, is then given by 
\begin{equation*}
S_\alpha^t := \inf_{\tau > 0} \{ \tau \,:\,  U_\alpha(\tau) > t \} .
\end{equation*}
%We keep the notation and denote from now on the new  parent process by $X$.
%It is assumed to satisfy
%
Suppose that $Y_\alpha(t)$ is a solution of the stochastic differential equation
%
\[ \mathrm{d}Y_\alpha(t) =  b(Y_\alpha(t),U_\alpha(t))\mathrm{d}t +   \sqrt{\frac{2k_B\mu_T}{\zeta}}  \mathrm{d} W(t),\quad t \geq 0.
\]
Define $X(t):=Y_\alpha(S_\alpha^t)$. It then follows from Theorem 1 in \cite{Magd2014} that if $b$ is twice continuously differentiable with respect to its
variables and satisfies the usual Lipschitz condition, then the probability density function of the process $X$ is a solution of the time-fractional
Fokker--Planck equation
%
%It follows that the the subordinated 
%Langevin equation 
%\begin{equation} \label{Eq:Langevin} \dd X(t) = b\big(X(U_\alpha(t)),U_\alpha(t)\big) \dt +  \sqrt{\frac{2k_B\mu_T}{\zeta}}   \dd W(t). 
%\end{equation} 
%By integrating the subordinated Langevin equation \cref{Eq:Langevin} on the time interval $(0,t)$, the subordinated process $Y_\alpha(t):=X(S_\alpha^t)$ can be written as
%\begin{equation} \label{eq:sub}
%Y_\alpha(t)= \int_0^t b(X(\tau),\tau) \dd S_\alpha^\tau +   \sqrt{\frac{2k_B\mu_T}{\zeta}}  W(S_\alpha^t),
%\end{equation}
%where the Lebesgue--Stieltjes integral of the vector function $b(\cdot,\cdot)$ is to be understood componentwise. 
%Equation \eqref{eq:sub} is a direct consequence of \eqref{Eq:Langevin} and the fact that $U_\alpha (S_\alpha^t) =t$.
%Since the Fourier transform of $Y_\alpha(t)$ is holomorphic in a neighborhood of zero (see, for example, \cite[Proof of Theorem 1]%{magdziarz2009stochastic}), the moments determine the distribution in a unique way; see \cite[Section VII.3]{feller}.
%Compared with the standard Fokker--Planck equation, we have to consider the inverse Laplace transform of the symbol of $\tau_0^{1-\alpha} \lambda^{1-%\alpha}$, which results in the presence of the differential operator $\tau_0^{1-\alpha} \ptb$.
% We recall that the limiting case of
%$\alpha =1$ corresponds to the inverse Laplace transform of $1$ and thus to the operator $\delta(t)$. As in the standard case, we define the first moment, still denoted by $\tilde\psi$, by $ \tilde \psi =\tilde\psi(x_1,x_2,t):=\langle Y_\alpha(t) \rangle$.
%These preliminary observations 
resulting from the replacement of $\tilde \psi$ on the right-hand side  of \eqref{eq:stFP} by $ \tau_0^{1 -\alpha} \ptb \tilde \psi$, i.e., 
$$
\pt \tilde\psi= \sum_{i=1}^2 \left( -\div_{x_i}\!\big( b_i((x_1^{\rm{T}},x_2^{\rm {T}})^{\rm T} 
,t)\,\tau_0^{1 -\alpha} \ptb \tilde\psi \big)  + \frac{k_B\mu_T}{\zeta} \Delta_{x_i}  \tau_0^{1 -\alpha} \ptb \tilde\psi \right).$$

Next, we perform the linear change of variables $(x_1,x_2) \mapsto (\frac{1}{2}(x_1 + x_2), x_2 - x_1)=:(x,q)$, whereby we have identified a point $x \in \Omega \subset \mathbb{R}^3$ in the (macroscopic) flow domain $\Omega$ with the center of mass of the dumbbell, and have, as before, denoted by $q$ the vector pointing from $x_1$ to $x_2$. 
 By recalling the definition \cref{Def:Vectorb} of $b(\cdot,\cdot)$ and setting $\psi(x,q,t):=\tilde\psi(x-\tfrac12 q,x+\tfrac12 q,t)$, we find that
\begin{equation} \label{Eq:Langevin3} 
\begin{aligned}
 \pt \psi \, + \,  & \tau_0^{1 -\alpha} \div_x \bigg(\frac{u(x-\tfrac12q,t)+u(x+\tfrac12q,t)}{2} \ptb\psi \bigg)\\ &\quad+ \tau_0^{1 -\alpha} \div_q\bigg(\big(u(x+\tfrac12q,t)-u(x-\tfrac12q,t)\big) \ptb\psi - \frac{2F(q)}{\zeta}  \ptb\psi \bigg) \\ &= \frac{k_B\mu_T \tau_0^{1 -\alpha}}{2\zeta} \Delta_x \ptb\psi + \frac{2k_B\mu_T \tau_0^{1 -\alpha}}{\zeta} \Delta_q \ptb\psi. 
\end{aligned} 
\end{equation}
To proceed, we assume `local homogeneity', i.e., that the spatial variation of the velocity field over the microscopic length-scale of a single dumbbell is negligibly small. Consequently, the arithmetic mean $\big(u(x-\frac12q,t)+u(x+\frac12q,t)\big)/2$ can be approximated by $u(x,t)$ in the second term on the left-hand side of \cref{Eq:Langevin3}.  In the case of the third term, we use Taylor expansion of $u$ about the point $x$ to obtain 
\begin{equation} \begin{aligned} u(x+\tfrac12 q,t)-u(x-\tfrac12 q,t) &= \nablax u(x,t) q + \mathcal{O}(|q|^3) \\ &=\Big(\sigma\big(u(x,t)\big) + \omega\big(u(x,t)\big)  \Big)q + \mathcal{O}(|q|^3),
		\end{aligned}
	\label{Eq:ApproxTaylor}
	\end{equation}
where we have further split the gradient of $u$ into its symmetric and antisymmetic parts, which are, respectively, defined as follows:
 \begin{equation} \label{Eq:Omega} \sigma(u):=\frac{\nablax u + (\nablax u)^{\mathrm{T}}}{2}, \qquad \omega(u):=\frac{\nablax u - (\nablax u)^{\mathrm{T}}}{2}.
 	\end{equation}
In the approximation \cref{Eq:ApproxTaylor}, we omit the $\mathcal{O}(|q|^3)$ term and further, we also omit the symmetric part of the gradient of $u$, i.e., we consider the, so called, \textit{corotational model}. 
While the omission of the $\mathcal{O}(|q|^3)$ term from \cref{Eq:ApproxTaylor} can be justified on the grounds that $|q| \ll 1$, our omission of the term $\sigma(u(x,t))$ from the additive decomposition $\nabla_x u(x,t)=\sigma(u(x,t))+\omega(u(x,t))$ is for purely technical reasons.
%
%and has no physical justification: unlike the case of $\alpha =1$, with our current mathematical tools we are unable to address the question of %existence of global weak solutions to the general noncorotational Navier--Stokes--Fokker--Planck system in the time-fractional case, where the full %gradient $\nabla_x u$ features in equation \cref{Eq:DerivFP} below instead of the skew-symmetric gradient $\omega(u)$. 
In the time-fractional corotational model considered here, polymer molecules are therefore allowed to rotate, but they are forced to do so without stretching. This modelling assumption weakens the coupling between the Navier--Stokes equation and the (time-fractional) Fokker--Planck equation; for example, if the initial datum for $\psi$ happens to be spherically symmetric with respect to $q$ and independent of the spatial variable $x$, then the Fokker--Planck equation is decoupled from the Navier--Stokes equation. In this respect, the model that we study here is no
different from the corotational model considered (in the case of $\alpha=1$) in the works  \cite{lions2000global,lions2007global,schonbek2009existence,masmoudi2008well,barrett2005existence,barrett2009numerical,debiec2023corotational,lin2008global,busuioc2014fene}.
Thus, we consider the following corotational Fokker--Planck equation with a time-fractional derivative:
\begin{equation} \label{Eq:FokkerDim1}\begin{aligned} 
&\pt \psi + \tau_0^{1- \alpha}
\div_x(u \ptb\psi) + \tau_0^{1- \alpha} \div_q\big(\omega( u) q \ptb\psi \big) \\ 
&\quad = \frac{k_B \mu_T \tau_0^{1- \alpha}}{2\zeta} \Delta_x \ptb\psi + \frac{2k_B\mu_T \tau_0^{1- \alpha}}{\zeta} \Delta_q \ptb\psi+ \tau_0^{1- \alpha} \div_q\Big( \frac{2F(q)}{\zeta}  \ptb\psi\Big).
\end{aligned} 
\end{equation}
We shall rely on the fact that $q^{\mathrm{T}} \omega(u) q \equiv 0$ thanks to the skew-symmetry of $\omega(u)$. In the general noncorotational case $q^{\mathrm{T}} (\nabla_x u)q$ is of course not identically nonzero in the flow domain. For $\alpha=1$ at least, the proof of the existence of large-data global-in-time weak solutions to the general noncorotational Navier--Stokes--Fokker--Planck system relies, instead, on an entropy estimate (cf.  \cite{barrett2011existence}). It is this entropy estimate that needs to be replicated in the time-fractional case, for $\alpha \in (0,1)$, as a key
ingredient of the proof of the existence of global-in-time large-data weak solutions. The analysis of the  general noncorotational time-fractional model is deferred to future work. 



%Because the continuum mechanical “macroscopic” equations of incompressible fluid flow are coupled to a “microscopic” model, the polymer model under consideration is a microscopic–macroscopic model. Here, the microscopic equation is the time-fractional Fokker--Planck equation
%\eqref{Eq:DerivFP}, describing the statistical properties of polymer molecules in the continuum. We begin by presenting these equations and collecting the relevant assumptions on the various parameters featuring in the model.

%Let $\Omega \subset \R^3$ be a Lipschitz domain and $D \subset \R^3$ a bounded open ball (centered at the origin) of admissible elongation vectors $q$. Let $u:\Omega  \times [0,T) \to \R^3$ be the velocity field and $\psi:\Omega \times D \times [0,T) \to \R$ the probability density function that represents the probability at time $t$ of finding the center of mass of a dumbbell in the volume element $x+\d x$ with the end-point of its elongation vector within the volume element $q+\dq$. 
%\\[7cm]

  Let $\psi(x,q,t) $ denote from now on the probability density function for a collection of $N\gg 1$ dumbbells. As we are dealing with a dilute polymeric fluid, the polymer molecules suspended in the fluid are assumed not to interact with each other and they move without self-interaction. The function $\psi$ therefore satisfies the same partial differential equation, \eqref{Eq:FokkerDim1}, as in the case of a single dumbbell. The only difference is in the choice of the initial datum $\psi^0 \geq 0$ for the Fokker--Planck equation. 
A typical choice of $\psi^0$ in the present context is $$\psi^0(x,q) = \frac{1}{N}\sum_
{j=1}^N \alpha_j(q) \Psi_j(x),$$ where $\alpha_j(q) \geq 0$ for all $q \in D$ and $\int_D \alpha_j(q)\d q = 1$, $j=1,\ldots,N$, and $\Psi_j \geq 0$, for all $x \in \Omega$ and $\int_\Omega \Psi_j (x) \d x =1$, $j=1,\ldots,N$. For example, one may choose $\Psi_j$  as a mollifier (i.e. a nonnegative $C^\infty_0$ approximation to the Dirac measure) concentrated at a point $z_j \in \Omega$, $j=1,\ldots,N$, with $\{\Psi_j\}_{j=1}^N$ forming a scaled partition of  unity; here $z_j$ can be thought of as the initial location of the center of mass of the $j$-th dumbbell.

For the sake of simplicity we shall confine our attention to the case when the boundary $\partial\Omega$ of the macroscopic flow domain $\Omega \subset \mathbb{R}^3$ has no inflow or outflow parts. The polymeric fluid under consideration is therefore confined to $\Omega$, and its velocity will be supposed to satisfy the no-slip boundary condition $u(x,t)=0$ for all $(x,t) \in \partial\Omega \times (0,T)$. 
Thus the
total number $N\gg 1$ of polymer molecules contained in $\Omega$ remains constant in time. The number density 
$\rho_P := \rho_P(x,t)$, in $[m^{-3}]$, of the polymer molecules contained in $\Omega$ is called the \textit{polymer number density} and it is related to $N$  by $ N = \int_\Omega \rho_P(x,t) \d x$. The polymer number density is further related to the probability density function $\psi$ satisfying the Fokker--Planck equation  \eqref{Eq:FokkerDim1} by 
%
\begin{align}\label{eq:rho-psi} 
\rho_P(x,t) = N \int_D \psi(x,q,t) \d q,
\end{align}
%
with the understanding that $\psi$ has the SI unit of $[m^{-6}]$. We shall supplement the Fokker--Planck equation \eqref{Eq:FokkerDim1} with no-flux (homogeneous Neumann) boundary conditions on $\partial\Omega \times D \times (0,T)$ and on $\Omega \times \partial D \times (0,T)$, which will ensure that $\int_{\Omega}\int_D \psi(x,q,t) \d q \d x$ is constant in time, and is therefore equal to $$\int_{\Omega}\int_D \psi(x,q,0) \d q \d x =  \int_{\Omega}\int_D \psi^0(x,q) \d q \d x=1.$$

%So far we have considered
% the statistical properties of one polymer molecule within a given macroscopic velocity field. Under the assumption of a dilute solution different polymer chains do not directly interact with each other. Let $N$ be the total number of polymer chains within the physical domain $\Omega$, and let us assume that the net flux is zero then the total number is a constant over time. The individual polymer chain influence the macroscopic velocity and induce an extra stress tensor within the Navier--Stokes system. Let us denote by $\psi_k $ the probability density function of the kth polymer chain. 
%We note that %all $\psi_k$ satisfy the Fokker--Planck equation \eqref{Eq:FokkerDim1} with the same macroscopic velocity but  have possibly different initial conditions but all are normalized, i.e., 
%$\int_\Omega \int_D \psi(x,q,0) \d x \d q =1 $. 
%For the choice of a consistent, i.e., number of polymer chain preserving, boundary condition, we refer to Subsection \ref{se:non}. The extra stress depends only on $\psi:= \frac{1}{N} \sum_{k=1}^N \psi_k$ which has due to the pre-factor $1/N$ also a normalized initial condition, and thus the macroscopic velocity depends on $\psi$ but not on each individual $\psi_k$ separately. 
%Summing the two nonlinear terms on the left of  \eqref{Eq:FokkerDim1} over all individual polymer molecules  results in
%\begin{align*}
%\sum_{k=1}^N  \left( \div_x \left(u \left(\sum_{l=1}^N \psi_l \right)  \ptb\psi_k \right) +  \div_q\left(\omega \left( u \left(\sum_{l=1}^N \psi_l\right) \right) q \ptb\psi_k \right) \right) \\
%= \div_x \left(u \left(\psi \right)  \ptb\psi \right) + \div_q\left(\omega  \left( u \left(\psi\right) \right) q \ptb\psi \right)  .
%\end{align*}
%All other terms are linear, and thus we obtain the coupled nonlinear Fokker--Planck equation 
%\begin{equation} \label{Eq:FokkerDim}\begin{aligned} 
%&\pt \psi + \tau_0^{1- \alpha}
%\div_x \left(u(\psi) \ptb\psi \right) + \tau_0^{1- \alpha} \div_q\left(\omega \left( u (\psi  ) \right)q \ptb\psi \right) \\ 
%&\quad = \frac{k_B \mu_T \tau_0^{1- \alpha}}{2\zeta} \Delta_x \ptb\psi + \frac{2k_B\mu_T \tau_0^{1- \alpha}}{\zeta} \Delta_q \ptb\psi+ \tau_0^{1- \alpha} \div_q\Big( \frac{2F(q)}{\zeta}  \ptb\psi\Big).
%\end{aligned} \end{equation}
Having established the “microscopic” equation that describes the statistical properties of polymer molecules in the continuum, we turn our attention to the continuum mechanical “macroscopic” equations of motion of the incompressible fluid in which the polymer molecules are suspended. We note that polymeric fluids are non-Newtonian fluids and the presence of the $N \gg 1$ polymer molecules contributes an additional term, $\tau=\tau(\psi)$, to the stress tensor appearing in the balance of linear momentum equation in the Navier--Stokes system: a symmetric polymeric extra stress tensor, which we shall define below. We assume that the evolution of the  velocity field $u$ and the pressure $p$ is governed by the incompressible Navier--Stokes system
\begin{equation} \label{Def:NS} \begin{aligned}
		\rho(\pt u + (u \cdot \nablax) u )- \eta \Delta_x u + \nablax p &= \div_x \tau (\psi) &&\quad\text{in }\Omega\times (0,T), \\
		\div_x u &=0 &&\quad\text{in }\Omega\times (0,T),\end{aligned} \end{equation}
supplemented with the homogeneous Dirichlet boundary condition $u=0$ on $\partial \Omega \times (0,T)$ and the initial condition $u(0)=u^0$ in $\Omega$
at $t=0$. 
As before, $\eta$ denotes the dynamic viscosity and $\rho$ stands for the macroscopic density, which is assumed to be constant in space and time.   The polymeric extra-stress tensor $\tau=\tau(\psi)$ is defined by the so-called \textit{Kramers expression}, see, e.g., \cite{bird1987dynamics2},
\begin{equation} 
	\label{Def:taud} \tau(\psi):=  \rho_P  \big(\C(\psi)- k_B \mu_T I_3\big),
\end{equation}
where, as before, $\rho_P$ is the polymer number density, $k_B$ and $\mu_T$ are the Boltzmann constant and the absolute temperature, respectively, and $I_3$ is the $3\times 3$ identity matrix. 
Finally,  the $3\times 3$ symmetric  tensor $\C$ appearing in the expression for  $\tau$ is defined by 
\begin{equation} 
\label{Def:C} \C(\psi):= \frac{\int_D  F(q) q^{\mathrm{T}} \psi \d  q }{ \int_D   \psi \d  q }  
,
\end{equation}
see, e.g., \cite{suli2018mckeanvlasov}. Because the polymer number density $\rho_P$ is related to $\psi$ by \eqref{eq:rho-psi}, the polymeric extra stress tensor is simplified to 
%
\[ \tau(\psi)  =\tau^1(\psi)+ \tau^2(\psi),\]
%
where
%
\begin{align} \label{eq:tau1}
 \tau^1(\psi) &:=   N \int_D  F(q) q^{\mathrm{T}} \psi \d  q , \\ \label{eq:tau2}
 \tau^2(\psi) &:=       - N k_B \mu_T  \int_D \psi \d q \ I_3.
\end{align}
Recall that $\psi$ is assumed to have the SI unit $[m^{-6}]$.




We note that the definition of $\C(\psi)$  results in the SI unit $[ kg/(s^2 m^2)] $ for the polymeric extra stress tensor $\tau=\tau(\psi)$.  
 The tensor $\tau$ is responsible for coupling the velocity $u$ and the pressure $p$ to the probability density function  $\psi$. 
Dividing the Navier--Stokes momentum equation by the macroscopic density $\varrho$ and introducing the kinematic viscosity as $\nu := \eta/\rho$, we arrive at the Navier--Stokes system in its dimensional form:
\begin{equation} \label{Eq:NavierDim}
\begin{aligned}
	\pt u + (u \cdot \nablax) u - \nu \Delta_x u + \frac 1\rho\nablax p &= \frac 1\rho\div_x \tau(\psi) &&\quad\text{in }\Omega\times (0,T), \\
	\div_x u &=0 &&\quad\text{in }\Omega\times (0,T).
\end{aligned} 
\end{equation}
In the next section we shall perform a nondimensionalization of the Navier--Stokes--Fokker--Planck system. In order to distinguish a nondimensionalized
quantity from its dimensional form, we shall use the subscript $_{\mathrm{dim}}$ for dimensional quantities in cases where confusion might arise; so, for example, we shall write $\tau_{\text{dim}}$ to indicate the original dimensional form of the polymeric extra stress tensor, while $\tau$ will henceforth signify its nondimensionalized form. Similarly $\mathcal{C}_{\mathrm{dim}}$ will denote the original dimensional form of $\mathcal{C}$ (cf. \eqref{Def:C}), while $\mathcal{C}$ will signify its form following nondimensionalization.


\subsection{Nondimensionalization} \label{se:non}
Next, we transform the Navier--Stokes--Fokker--Planck system, see \eqref{Eq:NavierDim} and  \eqref{Eq:FokkerDim1}, into its nondimensionalized form. To this end, we define the quantities $\hat x$, $\hat t$ and $\hat u$ by setting
$$x=L_0 \hat x, \quad t=T_0 \hat t, \quad u(x,t)=U_0 \hat u(\hat x,\hat t), $$
where	$L_0$ and $T_0$ stand for  the characteristic macroscopic length-scale and time-scale, respectively, and $U_0$ denotes the characteristic velocity of the macroscopic flow. 
 In a similar manner, we introduce $\hat q$ by letting $q=l_0 \hat q $, where $l_0$ is a characteristic microscopic length-scale, recall that $\psi$ was assumed to have the SI unit $[m^{-6}]$, and we define the dimensionless quantity $\hat\psi$ by 
\begin{align} \label{eq:hatpsi}
 \hat \psi(\hat x, \hat q, \hat t) := (L_0 l_0)^{3}\, \psi(x,q,t) . 
\end{align}
 We point out that scaling $\psi$ by a constant does not affect the definition of the $\C_{\text{dim}}(\psi)$ given by \eqref{Def:C}; i.e., for any positive constant $a$ we have $\C_{\text{dim}}(a \psi)  =\C_{\text{dim}}(\psi)$. 
Scaling $\psi$ by a positive constant, however, affects its relationship to the polymer number density $\rho_P (x,t)$. To avoid the nondimensionalization of $\rho_P (x,t)$, we do not consider the  definition \eqref{Def:taud} of $ \tau_{\text{dim}}(\psi) $ but of its equivalent form given by \eqref{eq:tau1} and \eqref{eq:tau2}.


%For ease of readability, we omit the hats in the arguments and in the partial differential symbols but keep the ones in the velocity and the probability density function.
 Consequently, we obtain from the time-fractional Fokker--Planck equation \eqref{Eq:FokkerDim1} the following partial differential equation
\begin{align} \label{Eq:FokkerNondim1}
& \nonumber \frac{1 }{T_0}\partial_{\hat t} \hat \psi + \frac{\tau_0^{1- \alpha} U_0 }{T_0^{1-\alpha} L_0} \div_{\hat{x}}( \hat u (\psi) \partial_{\hat t}^{1-\alpha} \hat \psi) + \frac{ \tau_0^{1- \alpha} U_0}{T_0^{1-\alpha} L_0} \div_{\hat q} \big(\omega( \hat u (\psi) ) \hat{q}\, \partial_{\hat t}^{1-\alpha} \hat \psi\big)  \\ 
&\quad =    \frac{k_B \mu_T \tau_0^{1- \alpha}}{2\zeta T_0^{1-\alpha} L_0^{2}}  \Delta_{\hat x} \,\partial_{\hat t}^{1-\alpha} \hat \psi +  \frac{2k_B\mu_T \tau_0^{1- \alpha}}{\zeta l_0^2 T_0^{1-\alpha}} \Delta_{\hat q} \, \partial_{\hat t}^{1-\alpha} \hat \psi +  \frac{2 H \tau_0^{1- \alpha}}{\zeta T_0^{1-\alpha}} \div_{\hat q}( \hat F(\hat q)  \partial_{\hat t}^{1-\alpha} \hat \psi) 
\end{align}
defined on the nondimensionalized domain $\widehat \Omega \times \widehat D \times (0, \hat T)$. Here we have used  the notation $\hat u(\psi) $ to indicate that the velocity field is understood to depend on the original (dimensional) probability density function $\psi$, and will only be expressed as a function depending on the nondimensionalized probability density function $\hat \psi$ in the next step. 
We set the characteristic macroscopic and microscopic time-scales to, respectively, $T_0:=L_0/U_0$, $\tau_0 := T_0$  and the nondimensionalized force in case of the standard FENE type model to
\begin{align} \label{eq:hatF}
\hat F(\hat q) ;= \frac{\hat q}{1 - |\hat q|^2/|\hat q_{\max} |^2},  \qquad \hat q_{\max} := q_{\max}/l_0.
\end{align}
The  prefactors of the last two terms on the right-hand side of \eqref{Eq:FokkerNondim1} are equal if the microscopic length-scale $\ell_0$ is defined as follows: 
\begin{align} \label{Eq:Def:l_0}
\ell_0^2:= \frac{k_B \mu_T}{H}.\end{align}
Next we introduce the Deborah number $\lambda$, defined as the ratio of the time it takes for the material to adjust to applied stresses/deformations and the characteristic time scale, $T_0$, by
%
\[ \lambda := \frac{\zeta/(4H)}{T_0} = \frac{\zeta}{4H T_0}  = \frac{\zeta \ell_0^2}{4 k_B \mu_T T_0}.\]
%
Finally, we define the nondimensional parameter 
%
\[ \varepsilon := \frac{k_B \mu_T}{2\zeta  U_0 L_0},\] 
%
whose numerator and denominator both carry the SI unit of $\mathrm{J = [kg\,m^2/s^2]}$ corresponding to energy. Thus, we obtain from \eqref{Eq:FokkerNondim1} by multiplying it with $T_0$ the following nondimensional equation:
\begin{align} \label{Eq:DerivFP}
\nonumber 	\partial_{\hat t} \hat \psi + (\hat u (\psi) \cdot \nabla_{\hat x}) \partial_{\hat t}^{1 - \alpha} \hat \psi + \div_{\hat q}\,\big(\omega(\hat u (\psi)) {\hat q}\, \partial_{\hat t}^{1- \alpha} \hat \psi\big) \, \qquad \qquad \qquad \\  = \varepsilon \Delta_{\hat x}\, \partial_{\hat t}^{1- \alpha} \hat \psi + \frac{1}{2\lambda} \div_{\hat q}\,(\nabla_{\hat q} \,\partial_{\hat t}^{1 - \alpha} \hat \psi + \hat F(\hat q)\, \partial_{\hat t}^{1- \alpha} \hat \psi).
\end{align} 
 We note in passing that the ratio of the diffusion coefficients $\varepsilon$ and $1/(2\lambda)$ featuring in the equation \eqref{Eq:DerivFP} is equal to $(\ell_0/(2L_0))^2$, which is $\ll 1$, so the first-term on the right-hand side of \eqref{Eq:DerivFP}, called the center-of-mass diffusion term, is frequently neglected in practical considerations. Crucially, we shall retain this term in the equation and will continue to work with a strictly positive center-of-mass diffusion coefficient $\varepsilon$, as is implied by the derivation of the Fokker--Planck equation \eqref{Eq:DerivFP} performed above. 


In the same manner, we multiply the Navier--Stokes system \eqref{Eq:NavierDim} by $T_0/U_0$ and arrive at the following nondimensionalized system posed on $\hat \Omega \times (0,\hat T)$:
\begin{equation} \label{Eq:NavierNonDim}
\begin{aligned}
	\partial_{\hat t} \hat u + (\hat u \cdot \nabla_{\hat x}) \hat u - \frac{1}{\text{Re}} \Delta_{\hat x} \hat u + \nabla_{\hat x} \hat p &= \div_{\hat x} \hat \tau(\hat \psi), \\  %&&\quad\text{in }\Omega\times (0,T), \\
	\div_{\hat x} \hat u &=0,% &&\quad\text{in }\Omega\times (0,T), 
\end{aligned} 
\end{equation}
where $\text{Re}$ stands for the Reynolds number and $\hat p$ for a scaled pressure. Defining $\tau(\hat \psi )$ in a proper way allows us to replace $\hat u(\psi)$ in our nondimensionalized Fokker--Planck equation by $\hat u(\hat \psi)$ or, in compact notation, by $\hat u$.
Thanks to the reformulation of the stress, we reconsider the two terms $\hat \tau^1(\hat \psi)$ and $\hat \tau^2 (\hat \psi)$, appearing in $\hat \tau (\hat \psi) := \hat \tau^1(\hat \psi) + \hat \tau^2 (\hat \psi)$, separately. The definition \eqref{eq:tau1} of $\tau^1 (\psi)$ in combination with the nondimensionalization steps and the definition of $\hat F$ and $\hat \psi$, see \eqref{eq:hatF} and \eqref{eq:hatpsi}, respectively yields
\begin{align} \label{eq:tau1nd}
\frac{1}{\rho} \frac{T_0}{U_0} \frac{1}{L_0} H l_0^2 l_0^3  &\int_{\hat D} \hat F(\hat q) \hat q^{\mathrm T} \hat \psi \d \hat q\, (L_0 l_0)^{-3}  N  \\ & \nonumber 
= \ \frac{N}{\rho L_0^3}  \frac{k_B \mu_T}{U_0^2} \int_{\hat D} \hat F(\hat q) \hat q^{\mathrm T} \hat \psi(\hat q) \d \hat q  =  \gamma\, \hat \C(\hat \psi) =: \hat \tau^1(\hat \psi),
\end{align}
with the dimensionless quantities $\gamma$ and $\hat \C(\hat \psi)$ being defined by
\begin{equation*}
	\gamma := \frac{ k_B \mu_T N }{\rho U_0^2 L_0^3}  \quad\mbox{and}\quad  \hat \C(\hat \psi):= \int_{\hat D} \hat F(\hat q) \hat q^{\mathrm T} \hat \psi \d \hat q.
\end{equation*}
To define $\hat \tau^2 (\hat \psi) $, we start from the scaled macroscopic momentum balance equation and the definition \eqref{eq:tau2}:
\begin{align} \label{eq:tau2nd}
\frac{1}{\rho} \frac{T_0}{U_0} \frac{1}{L_0} (- k_b \mu_T) l_0^3 & \int_{\hat D} \hat \psi \d \hat q \, (L_0 l_0)^{-3}  N   \ I_3  \\ & \nonumber
= \ -\frac{N}{\rho L_0^3} \frac{k_B \mu_T}{U_0^2}  \int_{\hat D} \hat \psi \d \hat q  \ I_3
 =  \gamma    \int_{\hat D} \hat \psi \d \hat q  \ I_3  =: \hat \tau^2(\hat \psi).
\end{align}
%
For the sake of notational simplicity we drop from now on all $\, \hat \cdot \,$ symbols.

We close this section by recalling from \eqref{Eq:NavierNonDim} and \eqref{Eq:DerivFP} our nondimensionalized model problem:
\begin{equation} \label{Def:FP}  \begin{aligned}
	\pt \psi + (u \cdot \nablax) \ptb \psi + \div_q\big(\omega(u) &q \ptb \psi\big) &&  \\  - \tfrac{1}{2\lambda} \div_q(\nablaq \ptb  \psi +  \ptb  \psi)  &=\eps \ptb \Delta_x \psi  &&\quad\text{in } \Omega \times D \times (0,T),  \\
\pt u + (u \cdot \nablax) u - \frac{1}{\text{Re}} \Delta_x u + \nablax p &= \div_x \tau(\psi) &&\quad\text{in }\Omega\times (0,T), \\
\div_x u &=0 &&\quad\text{in }\Omega\times (0,T),
\end{aligned} \end{equation}
and we supplement this system of equations with the following boundary conditions:
$$
\begin{aligned}
\left(\tfrac{1}{2\lambda} (\nablaq \ptb \psi + U'q \ptb\psi)-\omega(u) q \ptb \psi \right) \cdot n_{\partial D} &=0 \qquad \text{on } \Omega \times \partial D \times (0,T), \\
\eps \nablax \ptb \psi \cdot n_{\partial \Omega} &=0 \qquad \text{on } \partial\Omega \times  D \times (0,T), \\
	u  &= 0 \qquad \text{on } \partial\Omega  \times (0,T).
\end{aligned}$$
%
The nondimensional initial data $u^0(x)$ and $\psi^0(x,q)$ for the velocity field and the probability density function, respectively, are obtained from their dimensional counterparts. We note that the scaling \eqref{eq:hatpsi} ensures
that $\int_\Omega \int_D \psi^0(x,q) \d x \d q = 1$; this then guarantees, thanks to the homogeneous Neumann boundary conditions on $\psi$, that $\int_\Omega \int_D \psi(x,q,t) \d x \d q = 1$ for all $t \geq 0$. 

