Having proved that a weak solution tuple $(u,\hphi)$ to the time-fractional system in the sense of \cref{Eq:DefWeak} exists, we return to the original variable $\psi:=\gb* ( M \hphi)$, whose evolution is governed by the time-fractional Fokker--Planck equation \cref{Eq:DerivFP}. Indeed, $\gb*\psi=g_{2-2\alpha}*\phi$, which is continuous for $\alpha> 1/2$ and $(\gb*\psi)(0)=0$ as we have originally assumed in the model transformation. In this sense, we have also shown the existence of a variational solution tuple $(u,\psi)$ to the original time-fractional model.


\section*{Conclusions and outlook}
	In this paper, we investigated the well-posedness of a coupled Navier--Stokes--Fokker--Planck system with a time-fractional derivative. Such systems arise in the kinetic theory of polymeric liquid solutions with noninteracting polymer chains. We outlined the derivation of the model from a subordinated Langevin equation and considered the case of a finitely extensible nonlinear elastic (FENE-type) dumbbell model with a corotational drag term. We proved the existence of large-data global-in-time weak solutions to the corotational time-fractional model of order $\alpha \in (\tfrac12,1)$ and derived a uniform energy inequality by considering a nonstandard and novel testing procedure.
The existence of weak solutions to the general noncorotational time-fractional FENE model is an open problem, which will be studied in a forthcoming paper by using a different testing procedure; see \cite{barrett2011existence} for the integer-order setting (corresponding to $\alpha=1$). Concerning the numerical approximation of the time-fractional system considered here we refer the reader to the recent paper
%We bear in mind that numerical simulations have been omitted to emphasize the analytical findings.
%To discretize a system in both $\Omega$ and $D$, one must explore strategies such as the heterogeneous alternating-direction method, as done in \cite{knezevic2009heterogeneous} for a FENE-type dumbbell model.
%Simultaneously, the fractional derivative has to be discretized, and we recommend a modern approach to avoid the costly memory effect, e.g., the rational approximation scheme \cite{khristenko}. This is done by the submitted paper 
\cite{beddrich2023numerical}.


