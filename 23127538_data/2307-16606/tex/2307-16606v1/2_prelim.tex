\section{Mathematical preliminaries} \label{Sec:Prelim}
In this section, we introduce some useful definitions and results regarding the fractional derivative in the sense of Riemann--Liouville and recall the Aubin--Lions lemma, which is a key result featuring in proofs of existence of weak solutions to nonlinear PDEs based on compactness arguments. %

For a Hilbert space $H$ with inner product $(\cdot,\cdot)_H$ and norm $\|\cdot\|_H$, we shall denote the duality pairing between $H$ and its dual space $H'$ by $\langle \cdot,\cdot\rangle_H$. We shall denote the inner product on the Bochner space $L^2(0,T;H)$ by $(\cdot,\cdot)_{L^2H}$, and we shall write $(\cdot,\cdot)_{L^2_tH}$ when in this inner product the temporal interval of integration is $(0,t)$ for some $t \in (0,T)$ rather than the complete interval $(0,T)$, i.e., $$(u,v)_{L^2_tH}:=\int_0^t (u(s),v(s))_H \, \text{d}s \qquad \forall\, u,v \in L^2(0,T;H).$$
The norm induced by this inner product will be denoted by $\|\cdot\|_{L^2_t H}$.



\subsection{Riemann--Liouville kernels}
The Riemann--Liouville kernel function $g_\alpha$ of order $\alpha$ is defined by $g_\alpha(t):=t^{\alpha-1}/\Gamma(\alpha)$, $t \in (0,T)$, for $\alpha > 0$ and $g_0(t):=\delta_0(t)$ (the Dirac distribution concentrated at $0$) for $\alpha=0$. We observe that  $g_\alpha \in L^p(0,T)$ for any $\alpha\in (1-1/p,1)$ and $p \in [1,\infty)$, and the kernel function satisfies the following semigroup property; see \cite[Theorem 2.4]{diethelm2010analysis}:
\begin{equation} \label{Eq:Semigroup}
	\ga * g_\beta = g_{\alpha+\beta} \qquad \forall\, \alpha,\beta \geq 0.
\end{equation} 
%This can be proved as follows: One applies Fubini's theorem to interchange the order of integration 
%$$\begin{aligned}(\ga * g_\beta *u)(t) &=\frac{1}{\Gamma(\alpha)\Gamma(\beta)} \int_0^t (t-s)^{\alpha-1} \int_0^s (s-\tau)^{\beta-1} u(\tau) \dd \tau \dd s  \\
%	&=\frac{1}{\Gamma(\alpha)\Gamma(\beta)} \int_0^t u(\tau) \int_\tau^t (t-s)^{\alpha-1} (s-\tau)^{\beta-1}   \dd s \dd \tau,
%\end{aligned}$$
%and the substitution $s=\tau+\sigma(t-\tau)$ then yields
%$$\begin{aligned}(\ga * g_\beta *u)(t) 
%	&=\frac{1}{\Gamma(\alpha)\Gamma(\beta)} \int_0^t u(\tau) (t-\tau)^{\alpha+\beta-1} \int_0^1 (1-\sigma)^{\alpha-1} \sigma^{\beta}   \dd \sigma \dd \tau.
%\end{aligned}$$
%Lastly, we observe using the fundamental property of the Gamma function that the second integral is equal to $\Gamma(\alpha)\Gamma(\beta)/\Gamma(\alpha+\beta)$, see \cite[Theorem D.6]{diethelm2010analysis}, from which we deduce the desired semigroup property \cref{Eq:Semigroup} of $g_\alpha$.

We note that when $\alpha \in (0,1)$, one can bound the $L^p(0,t)$-norm of a function $u:(0,T) \to \R$ by its convolution with $\ga$ as follows: for any $t \in (0,T]$, we have that
\begin{equation}\begin{aligned} \|u\|_{L^p(0,t)}^p := \int_0^t |u(s)|^p \ds  &\leq t^{1-\alpha} \int_0^t (t-s)^{\alpha-1} |u(s)|^p \ds \\ &\leq T^{1-\alpha} \Gamma(\alpha) \big(\ga * |u|^p\big)(t).	
\end{aligned} 
\label{Eq:KernelNorm}
\end{equation}
This implies that the space $$L^p_\alpha(0,T):=\big\{u:(0,T) \to \R:\sup_{t \in (0,T)} (\ga*|u|^p)(t) < \infty \big\},$$ is indeed a subspace of $L^p(0,T)$.
%Further, this estimate can be generalized for a nonnegative function $u:(0,T) \to \R_{\geq 0}$ and for $0<\beta<\alpha<1$ in the following way:
%$$(\ga * u)(t)=\frac{1}{\Gamma(\alpha)} \int_0^t (t-s)^{\beta-1} \frac{(t-s)^{\alpha-1}}{(t-s)^{\beta-1}} u(s) \ds \leq \frac{T^{\alpha-\beta}\Gamma(\beta)}{\Gamma(\alpha)} (g_\beta * u)(t).$$
If the order $\alpha$ of the kernel function $g_\alpha$  is larger than 1, then one can exploit the semigroup property of the kernel and apply Young's convolution inequality (cf. Lemma 3.2 in \cite{Oparnica}) as follows:
$$(g_{1+\alpha}*u)(t)=(g_1*\ga*u)(t)=\int_0^t (\ga*u)(s) \ds \leq \|\ga\|_{L^1(0,t)} \|u\|_{L^1(0,t)},$$
for any $u \in L^1(0,T)$ and any $t \in (0,T]$.


\subsection{Time-fractional derivative} 
We can rewrite the definition of the Riemann--Liouville derivative stated in \cref{Eq:RL} in a compact form by using the convolution operator $*$ as $\pta w=\pt (\gb * w)$.
We refer to the classical textbooks \cite{diethelm2010analysis,baleanu2012fractional} and the newer monographs \cite{jin2021fractional,chen2022fractional} regarding fractional calculus and fractional differential equations.

 We define the fractional Riemann--Liouville--Bochner space   for $\alpha \in (0,1)$ and $p \in [1,\infty)$ on $(0,T)$ with values in $H$ by $$\W^{\alpha,p}(0,T;H):=\big\{u \in L^p(0,T;H) : \gb * u \in W^{1,p}(0,T;H)\big\}.$$
Here, the convolution $\ast$ is of course understood to be with respect to the temporal variable $t \in (0,T)$. In the limit, 
when $\alpha=1$, we have that $g_{1-\alpha} = g_0=\delta$, and then $$\W^{1,p}(0,T;H):=W^{1,p}(0,T;H):=\big\{u \in L^p(0,T;H) : \pt u \in L^p(0,T;H)\big\}.$$
However for $0 < \alpha < 1$,
the Riemann--Liouville space $\W^{\alpha,p}(0,T;H)$ differs from the fractional-order Sobolev--Bochner space
$$W^{\alpha,p}(0,T;H):=\Big\{u \in L^p(0,T;H) : (s,t) \mapsto \tfrac{\|u(t)-u(s)\|_H}{|t-s|^{\alpha+1/p}} \in L^{p}((0,T)\times(0,T))\Big\},$$
which can be confirmed by noting that the function $g_\alpha$ is an element of $\W^{\alpha,p}(0,T):=\W^{\alpha,p}(0,T;\R)$ for $\alpha \in (1-\tfrac{1}{p},1)$ but not of $W^{\alpha,p}(0,T)$; see \cite[Proposition 3.13]{carbotti2021note}. 
%Therefore, we are not able to apply classical results such as embedding theorems for Sobolev--Bochner spaces.

\begin{remark}  
Even though the space $\W^{\alpha,p}(0,T)$ is not a subspace of the Sobolev--Slobodecki\u{\i} space $W^{\alpha,p}(0,T)$, it is nevertheless continuously embedded into $C([0,T])$, the space of uniformly continuous functions defined on $[0,T]$, for $\alpha \in (1- \frac{1}{p},1]$ and $p \in [1,\infty)$; see, \cite[Remark 6.2]{carbotti2021note}. 
\end{remark}

\begin{comment}
Therefore, small values of $\alpha$ have to be studied carefully.
Further, we observe that $\ga$ does not belong to $L^p(0,T)$ for $\alpha \in (0,1-\tfrac{1}{p}]$ (e.g., $\ga \notin L^2(0,T)$ for $\alpha \in (0,\frac12]$) and therefore, we find that $$(\gb*\phi)(0)=0 \qquad \forall\, \phi \in \W^{\alpha,p}(0,T;H),~ \alpha \in (0,1-\tfrac{1}{p}],$$
by the inverse convolution property  \cref{Eq:InverseConvolution}. However, this might contradict a given nontrivial initial condition $\phi^0$. E.g., for $\phi \in \W^{\alpha,2}(0,T;H):=\W^{\alpha,2}(0,T;H)$ it has to hold that $(\gb*\phi)(0)=0$ for $\alpha\in (0,\tfrac12]$ and therefore, PDE solutions with this regularity are only well-posed for $\phi^0=0$. Such an issue can be avoided by studying PDEs of the form $\pta(\phi-\phi^0)=f(\phi)$ and considering instead the regularity of $\phi-\phi^0$, i.e., $\phi \in \W^{\alpha,2}_{\phi^0}(0,T;H)$. However, the time-fractional model \cref{Eq:System} in this work is not of this translated form and therefore, we cannot expect that this system is well-posed for non-zero initials. We note that $\psi_0=0$ is physically unreasonable anyway for probability density functions and, moreover, we will naturally observe in the existence's proof below that the restriction $\alpha \geq \frac12$ naturally appears in the energy estimates.
\end{comment}




We also introduce the following Riemann--Liouville space incorporating a homogeneous initial condition at $t=0$, albeit in a somewhat nonstandard manner:
$$\begin{aligned}
\W^{\alpha,p}_{0}(0,T;H)&:=\big\{u \in \W^{\alpha,p}(0,T;H) : (\gb*u)(0)=0 \big\}.
%\W^{\alpha,p}_{u^0}(0,T;H)&:=\big\{u \in L^p(0,T;H) : u-u^0 \in \W_0^{\alpha,p}(0,T;H) \big\}.
\end{aligned}$$ 
We note that the function $\gb*u:[0,T] \to H$ has a well-defined trace at $t=0$ (even when the function $u$ itself might not have one) thanks to the continuous embedding
$$\gb*u \in W^{1,p}(0,T;H) \hookrightarrow AC([0,T];H).$$
For a given element $z \in H$, the convolution $\gb*z$ should be understood to mean the function $t \mapsto (\gb*g_1)(t) z \in H$; recall that $g_1(t)\equiv 1$ for all $t\geq 0$. Thus, $z \in H$ is in this context now thought of as the mapping $t \mapsto g_1(t)z \in \mathcal{W}^{\alpha,p}(0,T;H)$,  for $\alpha \in (0,1)$, $p \in [1,\infty)$ and $0<\alpha p < 1$, or if $\alpha =1$ and $p \in [1,\infty)$. 
Thanks to the semigroup property \eqref{Eq:Semigroup} we then have that $$t\mapsto (\gb*z)(t)=z\,g_{2-\alpha}(t)=\frac{z}{\Gamma(2-\alpha)}  t^{1-\alpha} \in C([0,T];H)$$ for any $\alpha \in [0,1]$. Thus, for $\alpha \in (0,1]$, $p \in [1,\infty)$ and $z \in H$ we define the following `translated' Riemann--Liouville space:
\begin{equation} \label{Eq:RLSpaceU0}
    \W^{\alpha,p}_{z}(0,T;H):=\big\{u \in L^p(0,T;H) : u-z \in \W^{\alpha,p}(0,T;H), ~ (g_{1-\alpha}*u)(0)=0 \big\}.
\end{equation}
Note that if $\alpha \in (0,1)$, $p \in [1,\infty)$ and $0<\alpha p<1$, or if $\alpha=1$ and $p \in [1,\infty)$, then $u-z \in \W^{\alpha,p}(0,T;H)$ if, and only if $u \in \W^{\alpha,p}_0(0,T;H)$, and therefore, for such $\alpha$ and $p$ we have that $\W^{\alpha,p}_{z}(0,T;H) = \W^{\alpha,p}_0(0,T;H)$ irrespective of the choice of $z \in H$.

Next, we state an inverse convolution (or deconvolution) property. Its name stems from the fact that convolution with the kernel $\ga$ acts as an inverse mapping on the operator of taking $\alpha$-th fractional derivative, up to a term that involves the initial value at $t=0$.
%
\begin{lemma}[Inverse convolution] Let $\alpha \in (0,1]$ and $p\in [1,\infty)$. Suppose further that $H$ is a Hilbert space and $z \in H$. Then, for any $t \in (0,T)$, we have the following equalities:
\begin{align} 	\label{Eq:InverseConvolution1}
(\ga * \pta u)(t) &= u(t) - (\gb*u)(0)\ga(t) \quad &&\forall\, u \in \W^{\alpha,p}(0,T;H), \\ 	\label{Eq:InverseConvolution}
		(\ga* \pta u)(t)    &=u(t)  &&\forall\, u \in \W_{z}^{\alpha,p}(0,T;H). \end{align} 
\end{lemma}
\begin{proof}
	We start with the proof of the equality \cref{Eq:InverseConvolution1}.
Recall that for any function $u \in \W^{\alpha,p}(0,T;H)$ we have $\gb*u \in  AC([0,T];H)$, and the fundamental theorem of calculus for absolutely continuous functions therefore yields, for any $t\in [0,T]$,
$$(\gb *u)(t) - (\gb*u)(0) = \int_0^t \partial_s (\gb * u)(s) \ds=(g_1 * \pta u)(t).$$
We convolve this equality with the kernel $\ga$ and make use of the semigroup property \cref{Eq:Semigroup} to obtain
$$(g_1*u)(t) - (\gb*u)(0) g_{1+\alpha}(t)=g_{1+\alpha}*\pta u,$$
where we have used that  $g_\alpha*1=g_\alpha*g_1=g_{1+\alpha}$, because $\alpha \Gamma(\alpha) = \Gamma(1+\alpha)$.
Next, we differentiate this equality in $t$ and observe that $\pt (g_1*u)=u$, $\pt g_{1+\alpha}=g_\alpha$, and $\pt (g_{1+\alpha}*v)=g_\alpha*v$, which yields \cref{Eq:InverseConvolution1}.
We finally note that  \cref{Eq:InverseConvolution} follows trivially from \cref{Eq:InverseConvolution1} and \cref{Eq:RLSpaceU0}.
%We consider an element $u \in \W_{u^0}^{\alpha,p}(0,T;H)$, i.e., there exists an element $v \in \W_{0}^{\alpha,p}(0,T;H)$ with $u-u^0=v$ and using $v$ in \cref{Eq:InverseConvolution1}, we obtain $\ga*\pta v=v$, i.e.,
%\begin{equation*} \begin{aligned}\ga* \pta (u-u^0)   &= u-u^0 &&\forall\, u \in \W_{u^0}^{\alpha,p}(0,T;H).
	%\end{aligned} \end{equation*} 
%Moreover, we can split the left-hand side thanks to the linearity of the fractional derivative and obtain 
 %\begin{equation*} \begin{aligned}
	%	\ga* \pta u    &= u-u^0  + \ga * \pta u^0 =u  &&\forall\, u \in \W_{u^0}^{\alpha,p}(0,T;H),
	%\end{aligned} \end{equation*} 
	%where we have used that   $\ga * \pta 1 = \ga*\gb =1$ thanks to the semigroup property  \cref{Eq:Semigroup}.
 \end{proof}
 
The following result is a direct consequence of the interaction between fractional derivatives and kernel functions.
\begin{corollary} The following identities hold:
\begin{equation} \label{Eq:DerivativeofKernel}  \begin{aligned}
	\pta (\ga * u ) &=\pt ( \gb * \ga * u) = \pt (1*u) = u &&\forall\, u \in L^1(0,T;H), \\
	\ptb \pta u &=  \pt (\ga * \pta u) = \pt u &&\forall\, u \in  W_{0}^{1,1}(0,T;H).
\end{aligned}
\end{equation}
\end{corollary}




%However, in our setting of the time-fractional Navier--Stokes--Fokker--Planck system, we have already seen that the Riemann--Liouville derivative appears on the left-hand side without the translation of an initial value. This already explains intuitively the restriction on the values of $\alpha$ in the theorem of the system's well-posedness, see \cref{Thm:WellPosedness} below.


%As in the integer-order setting, there are continuous and compact embedding results for Riemann--Liouville spaces.
We shall require the following special case of the classical Aubin--Lions lemma; see \cite{simon1986compact}. Suppose that the Hilbert spaces $V,H,Z$ form a Gelfand triple $V \com H \con Z$. then, the following classical compact embeddings hold: 
\begin{equation} \begin{aligned} \label{Eq:aubin} 
W^{1,1}(0,T;Z) \cap L^p(0,T;V) &\com L^{p}(0,T;H), &&p \in [1,\infty), \\
W^{1,r}(0,T;Z) \cap L^\infty(0,T;V) &\com C([0,T];H), && r \in (1,\infty);
\end{aligned}\end{equation} 
see \cite{simon1986compact}. Several fractional counterparts of the Aubin--Lions lemma have been proposed; see \cite{ouedjedi2019galerkin,wittbold2020bounded,li2018some}. We make use of  the following result; see \cite[Corollary 3.2]{ouedjedi2019galerkin}:
\begin{equation*} \begin{aligned} %
\W^{\alpha,1}(0,T;Z) \cap L^p(0,T;V) &\com L^r(0,T;H), &&p \in (1,\infty), \quad r \in [1,p), \quad \alpha \in (0,1).
\end{aligned}\end{equation*} 
The proof can be easily adapted to the limit case $r=p$ if the $\alpha$-th fractional derivative is in a better space than $L^1(0,T;Z)$. This is done for Caputo derivatives in \cite{li2018some}. In fact, we obtain
%$r \in (\frac{p}{1+\alpha p},\infty) \cap [1,\infty)$.  In the spacial case when $p=2$ and $\alpha \in (\frac12,1]$ it yields
\begin{equation} \begin{aligned} \label{Eq:aubinfractional2} %
		\W^{\alpha,r}(0,T;Z) \cap L^p(0,T;V) &\com L^p(0,T;H), &&r\in (1,\infty), \quad \alpha \in (0,1).
\end{aligned}\end{equation} 

\begin{comment}
Next, we require a Gronwall-type inequality that allows an additional nonnegative factor $b \in L^1(0,T)$ in the integrand on the right-hand side of the inequality. Particularly, this function is only assumed to be integrable, and it is allowed to degenerate.
\begin{lemma}[Gronwall, cf. {\cite[Lemma II.4.10]{boyer2012mathematical}}] \label{Lem:Gron4}
    Let $C_1,C_2$ be  nonnegative constants and let $b\in L^1(0,T)$ be nonnegative. If the  function $u \in L^\infty(0,T)$  satisfies the inequality
    $$u(t) \leq C_1+C_2 \int_0^t b(s) u(s) \, \textup{d}s \qquad \text{for a.a. } t \in (0,T], $$
    then 
    $$u(t) \leq C_1 \textup{exp}\Big(C_2\int_0^t  b(s) \, \textup{d}s\Big) \qquad \text{for a.a. } t \in (0,T]. $$
\end{lemma}
\end{comment}


%\subsection{Fractional chain inequality}
The classical chain rule does not hold for fractional derivatives, but one can use the following inequality as a remedy; see \cite[Theorem 2.1]{vergara2008lyapunov}:
\begin{equation} \label{Eq:ChainOriginal}  \frac12 \pta \|u\|^2_H +\frac12 \gb(t) \|u\|_H^2 \leq (u,\pta u)_H \quad \forall\, u \in \W_{z}^{\alpha,2}(0,T;H),
\end{equation}
for $z \in H$ and almost all $t \in (0,T)$.
%Here, it has to be assumed that  $\big(\gb*(u-u^0)\big)(0)=0$. 

\begin{comment}
We will see that the initial condition $\psi^0$ to the time-fractional system considered later on does not satisfy $\big(\gb*(\phi-\psi^0)\big)(0)=0$. Instead, we say that the solution satisfies the initial condition if $(\gb*\phi)(0)=\psi^0$. One can transform this into the form from before by noting that
$$0=(\gb*\phi)(0)-\psi^0=(\gb*\phi)(0)-(\gb*\ga \psi^0)(0)=(\gb*(\phi-\ga \psi^0))(0).$$
Next, we derive a fractional chain inequality for such functions so that we can apply such a result later on in the existence proof.

We consider a fixed $u \in \H_{0}^\alpha(0,T;H)$ and we  introduce $v=u+\ga z$. Since in this case  $u^0=0$, this gives $(\gb*v)(0)=z$. 
%We note that $\pta v= \pta u$ and therefore we obtain
%$$\begin{aligned}
	%(v,\pta v)_H  &=(u,\pta %u)_H+(\ga z, \pta u)_H 
	%\\ &\geq \frac12 \pta %\|u\|_H^2 + \frac12 \gb \|u\|_H^2 + \ga  (z,\pta v)_H,
%\end{aligned}$$
Using the fractional chain inequality 
\cref{Eq:ChainOriginal} for $u=v-\ga z$, we trivially find
\begin{equation} \label{Eq:Chain} (v-\ga z,\pta v)_H \geq \frac12 \pta \|v-\ga z\|_H^2.\end{equation}

%We note that $u=\ga * \pta u$ thanks to the  inverse convolution property \cref{Eq:InverseConvolution} and inserting  $u=v-g_\alpha z$ yields $v-g_\alpha z=g_\alpha * \pta v$. Therefore, we can also write, instead of \cref{Eq:Chain},
%\begin{equation*}  (\ga * \pta v,\pta v)_H \geq \frac12 \pta \|\ga*\pta v\|_H^2.\end{equation*}
One might ask oneself if there is a generalized inequality of the form of $(\ga*u,u)_H \geq \frac12 \pta \|\ga * u\|_H^2$ for a sufficiently smooth function $u$, i.e., whether the convolution with $\ga$ is coercive in some sense. We partly answer this question in the next lemma.
\end{comment}
%
%We shall also require the inequalities stated in the next lemma. 

%\begin{lemma} Let $z\in H$ be given. For any $u \in \W_{0}^{\alpha,2}(0,T;H)$ and any $v \in \W^{\alpha,2}(0,T;H)$ with $(\gb * v)(0)=z \in H$ we have the following inequalities:
%\begin{align} \label{Eq:Coercive}   \int_0^t (u,\pta u)_H \ds &\geq \cos(\alpha \pi/2)\|\pt^{\alpha/2} u\|_{L^2_tH}^2, \\
%\int_0^t \! (v\!-\!\ga z,\pta v)_H \ds  &\geq \cos(\alpha \pi/2)  \Big( \tfrac12 \|\pt^{\alpha/2} v\|^2_{L^2_tH}-\tfrac{\Gamma(\alpha-1)}{\Gamma(\alpha/2)^2}g_{\alpha}(t) \|z\|_H^2 \Big).\label{Eq:Coercive2} 
%\end{align}
%\end{lemma}
%\begin{proof}
%By \cite[Lemma 3.1]{mustapha2014well} we have that
%\begin{equation} \label{Eq:Mustapha} \int_0^t (\ga*w,w)_H \ds \geq \cos(\alpha \pi/2) \|g_{\alpha/2} * w\|_{L^2_tH}^2 \qquad \forall\, w \in L^2_t(0,T;H).
%\end{equation}
%Hence, with $w=\pta u$ we obtain the inequality
%$$
%\int_0^t (u,\pta u)_H \ds \geq \cos(\alpha \pi/2) \|g_{\alpha/2} * \pta u \|_{L^2_tH}^2,$$
%where we have used the inverse convolution property $g_\alpha * \pta u=u$, see \cref{Eq:InverseConvolution} with $z=0$ to simplify the left-hand side. On the right-hand side, we can make use of the fact that $\pt^{\alpha}u =\pt^{\alpha/2}\pt^{\alpha/2} u$ and again the inverse convolution property \cref{Eq:InverseConvolution} to deduce that
%$$g_{\alpha/2}*\pt^{\alpha} u = g_{\alpha/2}*\pt^{\alpha/2} \pt^{\alpha/2} u = \pt^{\alpha/2} u.$$
%Therefore, we obtain the first of the inequalities stated in the lemma. We note that we can only split the $\alpha$-th derivative into the composition of two $\frac{\alpha}{2}$-th derivatives if the function $u$ satisfies an initial condition of the form $\big(\gb*u\big)(0)=0$. 
%
%
%For $v \in \W^{\alpha,2}(0,T;H)$ with $(g_{1-\alpha} \ast v)(0)=z \in H$, we define $u:=v-g_\alpha z$, and we find that $u \in \W_0^{\alpha,2}(0,T;H)$ and thus
%\begin{equation} \label{Eq:LemmaIneq} \begin{aligned} \int_0^t (v -\ga z,\pta v)_H \ds 
%%&\geq \cos(\alpha \pi/2) \|g_{\alpha/2} * \pta u \|_{L^2_tH}^2 
%%\\ &=\cos(\alpha \pi/2) \|g_{\alpha/2} * \pt^{\alpha/2} \pt^{\alpha/2} u\|_{L^2_tH}^2 \\ 
%\geq \cos(\alpha \pi/2)\|\pt^{\alpha/2} u\|_{L^2_tH}^2. \end{aligned} \end{equation}
%As, trivially, $(a/\sqrt{2} - \sqrt{2}b)^2 \geq 0$ for all $a, b \in \mathbb{R}$, it follows that  $|a-b|^2 \geq \frac{a^2}{2} - b^2$. Therefore, noting $(\pt^{\alpha/2} g_\alpha)^2= (g_{\alpha/2})^2=\tfrac{\Gamma(\alpha-1)}{\Gamma(\alpha/2)^2} g_{\alpha-1}$ and inserting $u=v-\ga z$ into the right-hand side of  \eqref{Eq:LemmaIneq} yields
%\begin{equation*} 
%\begin{aligned}
%\|\pt^{\alpha/2} u\|_{L^2_tH}^2  &=  \|\pt^{\alpha/2} (v-g_\alpha z)\|_{L^2_tH}^2 \\ 
%&\geq \frac12 \|\pt^{\alpha/2} v\|_{L^2_tH}^2 - \|\pt^{\alpha/2} \ga z\|_{L^2_tH}^2 \\
%&\geq    \frac12 \|\pt^{\alpha/2} v\|^2_{L^2_tH}-\tfrac{\Gamma(\alpha-1)}{\Gamma(\alpha/2)^2} \|z\|_H^2 \int_0^t g_{\alpha-1}(s)\ds
%%&\geq  \big|  \|\pt^{\alpha/2} v\|_{L^2_tH}-\big( \tfrac{\Gamma(\alpha-1)}{\Gamma(\alpha/2)^2} g_{\alpha}(t) \big)^{1/2} \|z\|_H \big|^2 
%\\
%&=    \frac12 \|\pt^{\alpha/2} v\|^2_{L^2_tH}-\tfrac{\Gamma(\alpha-1)}{\Gamma(\alpha/2)^2}g_{\alpha}(t) \|z\|_H^2.
%\end{aligned}
%\end{equation*}
%That completes the proof of the lemma.
%\end{proof}





\section{Model revisited} \label{Sec:Form}

Having summarised the required results from fractional calculus, we revisit the mathematical model that we have derived in \Cref{Sec:Derivation}.
Let us assume for the moment that the solution $\psi$ to the Fokker--Planck equation belongs to $\mathcal{W}^{1-\alpha,p}(0,T;H) \cap C([0,T];H)$ for some $\alpha \in (0,1)$ and a suitable Hilbert space $H$, to be chosen. As $\psi \in C([0,T];H)$, it follows that $\|(g_\alpha \ast \psi)(t)\|_H \leq \frac{t^\alpha}{\Gamma(1+\alpha)}\|\psi\|_{C([0,T];H)}$, and therefore  $(g_\alpha \ast \psi)(0)=0$. Hence, $\psi \in \mathcal{W}^{1-\alpha,p}_0(0,T;H)$. It then follows from \eqref{Eq:InverseConvolution1}, with $\alpha$ replaced by $1-\alpha$ and $u=\psi$ there, that $(g_{1-\alpha}\ast \partial_t^{1-\alpha} \psi)(t) = \psi(t)$ for $t \in (0,T)$. 
Motivated by these properties, we introduce the auxiliary function $\phi$ by
\begin{equation} \label{Eq:Substitute} \phi := \ptb \psi = \pt (\ga * \psi),
\end{equation} 
whereby $\psi=g_{1-\alpha}*\phi$. We then have that $\pt \psi = \pt(g_{1-\alpha} \ast \phi) = \pta \phi$; and, thanks to the assumed continuity of $\psi$ (i.e. $\psi \in C([0,T];H)$) it
makes sense to require attainment of the initial condition $\psi(0) = \psi^0$, i.e. $(g_{1-\alpha} \ast \phi)(0) = \psi^0$. We shall therefore introduce the substitution $\phi:=\partial_t^{1-\alpha} \psi$ in \eqref{Def:FP}, which results in the following system of equations:
\begin{equation} \begin{aligned}
	\pt u + (u \cdot \nablax) u - \nu \Delta_x u + \nablax p - \div_x \tau( \gb * \phi) &=0, \\
	\div_x u &=0, \\
	\pta \phi  + (u \cdot \nablax) \phi + \div_q (\omega(u) q \phi)-\tfrac{1}{2\lambda} \div_q(\nablaq \phi + U'q  \phi)   -\eps  \Delta_x  \phi&=0, \end{aligned} 
\label{Eq:System}
\end{equation}
subject to the initial conditions $u(0)=u^0$ and $(\gb*\phi)(0)=\psi^0$ for a given nonnegative $\psi^0$ that fulfils $\int_D \psi^0 \dq=1$. Furthermore, we equip the system with the following boundary conditions:
\begin{align}\label{eq:neumannbc}
\begin{aligned}
u &=0 \qquad\text{on } \partial \Omega \times (0,T), \\
\left(\tfrac{1}{2\lambda} (\nablaq \phi + U'q \phi)-\omega(u) q \phi \right) \cdot n_{\partial D} &=0 \qquad\text{on } \Omega \times \partial D \times (0,T), \\
\eps \nablax \phi \cdot n_{\partial \Omega} &=0 \qquad\text{on } \partial\Omega \times  D \times (0,T).
\end{aligned}
\end{align}




\subsection{The Maxwellian and Maxwellian-weighted function spaces} \label{Sec:Maxwell}
We introduce the normalized Maxwellian by 
\begin{equation}
	\label{Def:Max} M(q)=\frac{e^{-U(\tfrac12 |q|^2)}}{\int_D e^{-U(\tfrac12 |s|^2)} \dd s}.
\end{equation}
Moreover, we define the (Maxwellian-weighted) Hilbert spaces
$$\begin{alignedat}{5}
	&\mathcal{H}=\{h \in L^2(\Omega;\R^d): \div \, h =0\}, \qquad ~\mathcal{H}_0&&=\{h \in \mathcal{H} : h \cdot n_{\partial \Omega} = 0 \text{ on } \partial \Omega\}, 
	\\ &\mathcal{V}=\{v \in H^1(\Omega;\R^d) : \div \, v = 0\}, \qquad ~ \mathcal{V}_0&&=\{v \in \mathcal{V} : v|_{\partial \Omega} = 0 \text{ on } \partial \Omega\},
	\\  &\mathcal{Y}=L^2(\Omega \times D),  \qquad\quad\,\,\,   \widehat{\mathcal{Y}}= L^2_M(\Omega \times D) &&=\{y \in \mathcal{Y}: \|y\|_{\widehat{\mathcal{Y}}} := \|M^{1/2}y \|_\mathcal{Y}<\infty \}, 
	\\ &\mathcal{X} = H^1(\Omega \times D), \qquad\quad \widehat{\mathcal{X}}= H_M^1(\Omega \times D) &&= \{\phi \in \mathcal{X}: \|\phi\|_\hX <\infty  \},  \\
&\mathcal{Z}=H^1(D;H^1(\Omega)), ~ \widehat{\mathcal{Z}} =H_M^1(D; H^1(\Omega))&&=\{\zeta \in \mathcal{Z}: \|\zeta\|_\hZ <\infty  \}, 
\end{alignedat}$$
where the norms on $\hX$ and $\hZ$ are defined by $\|\phi\|_\hX^2:= \|\phi\|_\hY^2 + \|\nablaq \phi\|_\hY^2 + \|\nablax \phi\|_\hY^2$ and $\|\zeta\|_\hZ^2 :=\|\zeta\|_\hX^2 + \|\nablax \nablaq \zeta\|_\hY^2$. Obviously, $H_M^2(\Omega \times D) \subseteq \hZ$, where $H^2_M(\Omega \times D)$ is the subspace of $H^1_M(\Omega \times D)$ consisting of all functions defined on $\Omega \times D$ whose second (weak) partial derivatives belong to $\hY=L^2_M(\Omega \times D)$.
We refer to \cite{barrett2005existence} regarding theoretical results on these weighted Hilbert spaces. In particular, we have the Gelfand triples
$$\begin{aligned} &\HSV \com \HS \hookrightarrow \HSV', \quad &&\HSV_0 \com \HS_0 \hookrightarrow \HSV_0', \\
	&\mathcal{X} \com \mathcal{Y} \hookrightarrow \mathcal{X}', \quad &&\hX \com \hY \hookrightarrow \hX',
\end{aligned}$$
where $\mathcal{V}'$, $\mathcal{V}'_0$, $\mathcal{X}'$ and $\hX'$ denote the dual space of, respectively, 
$\mathcal{V}$, $\mathcal{V}_0$, $\mathcal{X}$ and $\hX$. 

Using the definition of the normalized Maxwellian $M$, see \cref{Def:Max}, we have that $$M(q)\nablaq M(q)^{-1}=-M(q)^{-1} \nablaq M(q) =\nablaq U(\tfrac12 |q|^2) = U'(\tfrac12 |q|^2) q.$$ We introduce the scaled variable $\hphi=\phi/M$ and  with the formula 
$$M \nablaq \hphi = \nablaq \phi + M \nablaq M^{-1} \phi  = \nablaq \phi + U'q \phi$$
we can rewrite the fractional Fokker--Planck equation in \cref{Eq:System} as
$$ \pta \phi + (u \cdot \nablax) \phi + \div_q \big(\omega(u) q \phi\big) = \tfrac{1}{2\lambda} \div_q(M \nablaq \phim) + \eps \Delta_x \phi.$$
As was indicated earlier, we shall confine ourselves here to considering the corotational model, i.e.,
$\omega(v)=-\omega(v)^{\mathrm{T}}$, $q^{\mathrm{T}} \omega(v) q = 0$; if $\div\, v = 0$ it then follows that 
\begin{equation}\label{Eq:SigmaZero} 
\big(M \hphi \,\omega(v)q,\nablaq \hphi\big)_{\mathcal{Y}} = \frac12 \big(M \omega(v)q, \nablaq \hphi^2\big)_{\mathcal{Y}}=-\frac12 \big(\div_q(M \omega(v) q),\hphi^2\big)_{\mathcal{Y}} = 0; \end{equation}
see \cite{barrett2009numerical,barrett2005existence}.
%
We note in passing that partial integration yields the following equalities: 
\begin{equation} \label{Eq:IntParts} \begin{aligned}
		-2\big(M \omega(u) q \hat\varphi,\nablaq \hphi\big)_{\mathcal{Y}} &=  \big(\nablax(M\hat\varphi \nablaq \hphi)q,u\big)_{\mathcal{Y}} + \big(u\cdot q, \div_x (M\hat\varphi\nablaq \hphi)\big)_{\mathcal{Y}}
		\\ &= \big(M \nablax \hat\varphi (\nablaq \hphi)^{\mathrm{T}} q,u\big)_{\mathcal{Y}} + \big(M \hat\varphi \nablax \nablaq \hphi\, q,u\big)_{\mathcal{Y}} \\ &\quad + \big(u \cdot q,M \nablax \hat\varphi \cdot \nablaq \hphi\big)_{\mathcal{Y}} + \big(u \cdot q,M \hat\varphi \,\div_x \nablaq \hphi\big)_{\mathcal{Y}}.
\end{aligned} \end{equation} 
%

We recall that the stress tensor $\tau(\psi) = \tau_1(\psi) + \tau_2(\psi)$ is of the form  given by \eqref{eq:tau1nd} and \eqref{eq:tau2nd}; i.e., 
%%%%%%%%%%%%%
%
\begin{align}\label{Eq:tau1tau2}\tau^1(\psi)=\gamma\, \C(\psi),\quad \tau^2(\psi)= \gamma \int_{D} \psi \d q  \ I_3,\quad  \C(\psi):= \int_{D} F(q) q^{\mathrm T} \psi \d q,
\end{align}
%
where $\gamma>0$ is a dimensionless constant.


%%%%%%%%%%%%
For $\C(M \hpsi)$, we are in a setting that allows us to deduce the following bound; see also \cite[Eq. (3.7)]{barrett2009numerical}:
\begin{equation} \label{Eq:C}  \begin{aligned}
		\int_\Omega |\C(M \hpsi)|^2 \d x & =
		\int_\Omega \left| \, \int_{D} F(q)  q^{\mathrm T} M \hpsi \d q \, \right|^2 \d x
		\\ &\leq \int_D M  | F(q)  q^{\mathrm T} |^2   \dq \, \int_{\Omega \times D} M|\hpsi|^2 \d(x,q)  \\ & \leq C  \| \hpsi\|_{\hY}^2 \quad \forall\, \hpsi \in \hY. \end{aligned}\end{equation}



