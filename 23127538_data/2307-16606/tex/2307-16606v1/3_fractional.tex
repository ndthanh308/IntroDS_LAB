\section{Existence of weak solutions} \label{Sec:Analysis}
In this section, we prove the existence of a weak solution to the time-fractional Navier--Stokes--Fokker--Planck system. We use a Galerkin procedure and discretize the partial differential equations in space and derive suitable energy estimates. We will emphasize the places where the time-fractional derivative comes into play. We shall then pass to the limit in the sequence of Galerkin approximations to deduce the existence of a weak solution. We shall proceed step by step and prove this result through several lemmas. While, for physical reasons, the derivation of the system was in the previous sections discussed in the case of $d=3$ space dimensions, the analysis below applies to both two and three space dimensions. We begin by introducing the concept of a weak solution to the time-fractional corotational Navier--Stokes--Fokker--Planck system under consideration. %\medskip

\begin{definition} \label{Eq:DefWeak} Suppose that $d \in \{2,3\}$.
We call the pair $(u,\hphi)$ a weak solution to the system \cref{Eq:System}, \cref{eq:neumannbc} provided that   
\begin{align*}
u &\in L^\infty(0,T;\mathcal{H}_0) \cap L^2(0,T;\mathcal{V}_0) \cap W^{1,\tfrac{8}{4+d}}(0,T;\mathcal{V}_0'), \\ 
%L^{
\hphi &\in L^2(0,T;\hX), \quad \pta \hphi \in  L^{\tfrac{8}{4+d}}\big(0,T; \hZ' \big),
\end{align*}
satisfies the initial conditions $u(0)=u^0$, $(\gb*\hphi)(0)=\hpsi^0 := \psi^0/M$ and  the variational problems
\begin{align}
		\langle \pt u,v \rangle_{L^{8/(4-d)}\mathcal{V}}  + \big((u \cdot \nablax)u,v\big)_{L^2\mathcal{H}} + \nu (\nablax u,\nablax v)_{L^2\mathcal{H}} && \label{Eq:NS} \\[-.1cm] + k_B \mu_T \big(\C(M \gb *\hphi),\nablax v\big)_{L^2\mathcal{H}} =0, && \notag  \\[.1cm]
		\langle  \pta \hphi,\hat\zeta\rangle_{L^{8/(4-d)} \hZ}  - (u \hphi,\nablax \hat\zeta)_{L^2\hY}+ \frac{1}{2\lambda} (\nablaq \hphi,\nablaq \hat\zeta)_{L^2\hY}  + \eps (\nablax \hphi,\nablax \hat\zeta)_{L^2\hY} && \label{Eq:FP} \\[-.2cm]   + \frac12 (\nablax(\hphi\nablaq \hzeta)q,u)_{L^2\hY} - \frac12 \big(u\cdot q,\div_x (\hphi\nablaq \hzeta)\big)_{L^2\hY} = 0. &&  \notag
\end{align}
for all $v \in L^{8/(4-d)}(0,T;\mathcal{V}_0)$ and $\hat\zeta \in L^{8/(4-d)}(0,T;\hZ)$. In these variational problems and hereafter, for a Hilbert space $H$ and $p \in [1,\infty]$, subscripts of the form $L^pH$ and $L^p_tH$ appearing in the various inner products, norms, and duality pairings, signify $L^p(0,T;H)$ and $L^p(0,t;H)$, respectively. %Similarly, for a reflexive Banach space $B$, $\langle \cdot, \cdot \rangle_{C B}$ denotes the duality pairing between $C([0,T];B)$ and its dual space, 
%$\mathcal{M}(0,T;B')$ the space of all $B'$-valued finitely additive finite signed measures defined on $(0,T)$, which are absolutely continuous with respect to the Lebesgues measure, equipped with the total variation norm over $(0,T)$.
\end{definition} %\medskip


We summarize the assumptions that we require for proving the existence of a weak solution in the sense of \cref{Eq:DefWeak} in Assumption \ref{Ass:WellPosedness} below. %\medskip

\begin{assumption} \label{Ass:WellPosedness}
	Let the following assumptions hold:
	\begin{itemize}
		\item $D \subset \R^d$,  $d\in\{2,3\}$, is a bounded open ball centered at the origin, $\Omega \subseteq \R^d$ is a Lipschitz domain (i.e., bounded, open, connected set in $\R^d$, with a Lipschitz-continuous boundary $\partial\Omega$), and $T<\infty$ is a fixed final time;
  \item $u^0 \in \mathcal{H}_0$, $\hpsi^0 \in \hX$ with $\hpsi^0 \in H^1_M\big(D; H^{1+d/2+\delta}(\Omega)\big)$ for $\delta>0$ arbitrarily small;
  \item $\tau(\psi) =\tau_1(\psi) + \tau_2(\psi)$ is given by \eqref{Eq:tau1tau2}, with the identity matrix $I_3 \in \mathbb{R}^{3 \times 3}$ replaced by the identity matrix $I_d \in \mathbb{R}^{d \times d}$ in the definition of $\tau_2(\psi)$,  and $\C$ satisfies \cref{Eq:C};
    \item $\alpha \in (1/2,1)$;
  \item $k_B,\mu_T, \nu,\lambda,\eps >0$.
	\end{itemize}
\end{assumption} %\medskip

The main result of this paper is the following theorem asserting the existence of global-in-time large-data weak solutions to the time-fractional Navier--Stokes--Fokker--Planck system under consideration. 

%\medskip

\begin{theorem} \label{Thm:WellPosedness}
	Let \cref{Ass:WellPosedness} hold; 
	then, there exists a weak solution $(u,\hphi)$ to the system \cref{Eq:System}, \cref{eq:neumannbc} in the sense of \cref{Eq:DefWeak}.
\end{theorem}%\medskip

In order to prove this theorem, we state several lemmas, which will eventually imply \cref{Thm:WellPosedness}. We begin by constructing a sequence of Galerkin approximations $\{(u_k,\hphi_k)\}_{k=1}^\infty$ to the system of partial differential equations under consideration, resulting in a system of fractional-order ordinary differential equations, which admits a local-in-time solution $(u_k,\hphi_k)$
for each $k \geq 1$ thanks to  standard theory. We then derive an energy estimate for the sequence of Galerkin approximations, which is uniform with respect to $k$; this then implies that, for each $k \geq 1$, the local-in-time solution of the fractional-order system of ordinary differential equations can be extended to the entire time-interval $[0,T]$; it also implies the existence of a weakly/weakly-$*$ convergent subsequence $(u_{k_j},\hphi_{k_j})$. Finally, we pass to the limit $j \to \infty$ and apply a compactness argument to deduce that the limiting pair of functions 
$(u,\hphi)$ is in fact a weak solution to the system of partial differential equations in the sense of \cref{Eq:DefWeak}. The Galerkin method has been applied to various time-fractional PDEs; see, e.g., \cite{fritz2020time,fritz2021sub,fritz2021equivalence,vergara2015optimal}; it has also been applied to Navier--Stokes--Fokker--Planck systems in \cite{knezevic2009heterogeneous,barrett2009numerical,barrett2011finite,barrett2012finite}, with an integer-order Fokker--Planck equation. 

\subsection{Galerkin discretization}
We follow the construction of \cite[Section 2.1]{bulicek2013existence} and conclude by the Hilbert--Schmidt theorem \cite[Lemma A.4]{bulicek2013existence} the existence of a countable set $\{h_j\}_{j=1}^\infty$ of eigenfunctions in $\mathcal{V}_0 \cap H^{1+\frac{d}{2}+\delta}(\Omega)^d$, with $\delta>0$ arbitrarily small, whose linear span is dense in $\mathcal{H}_0$ such that the $h_j$, $j\in \{1,2,\dots\}$, are orthonormal in $\mathcal{H}$ and orthogonal in $H^{1+\frac{d}{2}+\delta}(\Omega)^d$ in the sense that $(h_j,h_i)_{H^{1+\frac{d}{2}+\delta}(\Omega)}=\lambda_j \delta_{i,j}$ for any $i,j \in \{1,2,\ldots\}$ and $\lambda_j>0$ for all $j=1,2,\ldots$. Similarly, we fix a countable set $\{y_j\}_{j=1}^\infty$ in $H^2_M(\Omega \times D)$ that forms an orthogonal system in $\hX$ and an orthonormal system in $\hY$. 
%Consider the high-order elliptic problem of finding a solution tuple $(u,\lambda) \in H^{1+d}(\Omega)^d \times \R$ of
%$$(u,v)_{H^{1+d}(\Omega)} + (\nabla u,\nabla v)_{L^2(\Omega)}  = \lambda (u,v)_{\mathcal{H}} \qquad \forall  v \in W^{1+d}(\Omega)^d.$$
%By a version of the Hilbert--Schmidt theorem, see \cite[Lemma A.4]{bulicek2013existence}, there is a countable set of eigenfunctions $\{h_j\}_{j=1}^\infty$, which are orthogonal in the inner product of $H^{1+d}(\Omega)^d$ and orthonormal in the inner product of $L^2(\Omega)^d$. We note that $H^{1+d}(\Omega)^d$ is continuously embedded into $W^{1,\infty}(\Omega)$
We then define the $k$-dimensional linear spaces
\begin{align*}
	\mathcal{H}_k  :=\text{span}\{ h_1,\dots,h_k\}, \quad 
	\widehat{\mathcal{Y}}_k  :=\text{span}\{ y_1,\dots,y_k\},
	%	Z_K &=\text{span}\{ z_1,\dots,z_k\},
\end{align*}
%where $h_j: \Omega \to \R$ and $y_j : \Lambda \to \R$, $j \in \{1,\dots,k\}$, are the eigenfunctions corresponding to the eigenvalues $\lambda_{j}, \mu_{j} \in \R$ of the following respective problems
%$$\begin{aligned}
	%(h_j,v)_{H^{1+d}(\Omega)} + (\nabla h_j,\nabla v)_{L^2(\Omega)}  &= \lambda_{j} (h_j,v)_{\mathcal{H}} &&\forall  v \in W^{1+d}(\Omega;\R^d),  \\
	%(\nabla y_j,\nabla y)_\hY &= \mu_{j} (y_j,y)_\hY &&\forall  y \in \hY. 
%\end{aligned}$$
%\begin{alignat*}{1}
%	 & \begin{cases} \begin{aligned}
		%			-\Delta h_j          & = \lambda_{h,j} h_j &  & \text{in } \Omega,         \\
		%			\nabla h_j \cdot n & = 0               &  & \text{on } \partial\Omega,
		%		\end{aligned} \end{cases} 
%			  \begin{cases} \begin{aligned}
		%		-\Delta h_j^0          & = \lambda_{h^0,j} h_j^0 &  & \text{in } \Omega,         \\
		%		\nabla h_j^0 \cdot n & = 0             &  & \text{on } \partial\Omega \backslash \p \Omega_D, \\
		%				h_j^0 & = 0               &  & \text{on } \partial\Omega_D, 
		%		\end{aligned} \end{cases} 
%	  \begin{cases} \begin{aligned}
		%			-\Deltala y_j                  & = \lambda_{y,j} y_j &  & \text{in } \Lambda,         \\
		%			\nablala y_j \cdot n_\Lambda & = 0               &  & \text{on } \partial\Lambda_N, \\
		%			y_j & = 0               &  & \text{on } \partial\Lambda_D.
		%		\end{aligned} \end{cases}
%\end{alignat*}
%As stated in \cite{bulicek2013existence}, there is a countable 
%Since both the inverse Stokes and the inverse Neumann--Laplace operator are compact, self-adjoint, injective, positive operators on $\mathcal{H}_0$ and $\hY$, respectively, we conclude by the spectral theorem, see e.g. \cite[12.12 and 12.13]{alt2016linear}, that
%\begin{alignat*}{3}
	%& \{h_j\}_{j \in \mathbb{N}} &  & \text{ is an orthonormal basis in } \mathcal{H}_0 &  & \text{ and orthogonal in } \mathcal{V}_0, \\
	%	& \{y_j\}_{j \in \mathbb{N}} &  & \text{ is an orthonormal basis in } \hY &  & \text{ and orthogonal in } \hX.
%\end{alignat*}
%By the orthonormality of the eigenfunctions, we conclude that ${\cup_{k\in\mathbb{N}}} \mathcal{H}_k$ and ${\cup_{k\in\mathbb{N}}} \hY_k$ are dense in $\mathcal{V}_0$ in $\hX$, respectively. 
and we consider the Galerkin approximations
\begin{equation}\begin{gathered}
		u_k (t) = \sum_{j=1}^k u_k^j(t) h_j,
		\quad \hphi_k (t) = \sum_{j=1}^k \hphi^j_k(t) y_j,
	\end{gathered}
	\label{Eq:GalerkinAnsatzFunctions}
\end{equation}
where 
$u^{j}_k$ and $\hphi^j_k$ are real-valued coefficient functions for all $j \in \{1,\dots,k\}$.
 %Let $h>0$ denote a discretization parameter tending to zero. As in \cite{barrett2009numerical}, we choose finite-dimensional spaces $\hY_k^x \subset W^{1,\infty}(\Omega)$ and $\hY_k^q \subset W^{1,\infty}(D)$ such that it holds
%$$\text{dist}_{W^{1,\infty}(\Omega)} (\eta,\hY_k^x) \to 0, \qquad \text{dist}_{W^{1,\infty}(D)} (\xi,\hY_k^q) \to 0,$$
%as $h\to 0$ for all $\eta \in C^\infty(\overline\Omega)$ and $\xi \in C^\infty(\overline D)$. Moreover, we define the tensor space $\hY_k=\hY_k^x \otimes \hY_k^q \subset W^{1,\infty}(\Omega \times D)$ and note that $\widehat{X}_h \subset X \subset \hX$. 
%Further, we define the finite-dimensional spaces $W_h$, $R_h$ and $\mathcal{H}_k$ such that
%$$\begin{aligned}W_h &\subset \mathcal{H}_0^1(\Omega;\R^d) \cap W^{1,\infty}(\Omega;\R^d), \quad R_h \subset L_0^2(\Omega), \\ \mathcal{H}_k&=\{w_h\in W_h: (\div_x w_h,r_h)_{\mathcal{H}} \,\forall r_h \in R_h \},\end{aligned}$$
%where $\cup_{h>0} W_h$ and $\cup_{h>0} R_h$ are supposed to be dense in $\mathcal{H}_0^1(\Omega;\R^d)$ and $L_0^2(\Omega;\R^d)$, respectively. Further, we assume that for all $v \in V$ there exists a sequence $v_k \in \mathcal{H}_k$ such that $v_k \to v$ in $H^1(\Omega)$ for $h \to 0$. This holds for the typical Galerkin approximation with the availability of an uniform inf-sup condition.
The canonical orthogonal projection onto the finite-dimensional space $\mathcal{H}_k$ is defined by $\Pi_{\mathcal{H}_k}: \mathcal{H} \to \mathcal{H}_k$, $h \mapsto \sum_{j=1}^k (h,h_j)_{\mathcal{H}} h_j$,  and in the same way for $\Pi_{\hY_k}:\hY \to \hY_k$. 
For $h=\sum_{j=1}^\infty (h,h_j)_{\mathcal{H}}h_j$ we have that $$\|h\|^2_{H^{1+\frac{d}{2}+\delta}(\Omega)} = \sum_{j=1}^\infty \lambda_j |(h,h_j)_{\mathcal{H}}|^2, \quad \|\Pi_{\mathcal{H}_k} h\|^2_{H^{1+\frac{d}{2}+\delta}(\Omega)} = \sum_{j=1}^k \lambda_j |(h,h_j)_{\mathcal{H}}|^2,$$
from which we conclude via the Sobolev embedding theorem that, for each $k \geq 1$, 
$$\|\Pi_{\mathcal{H}_k} h\|_{W^{1,\infty}(\Omega)} \leq C\|\Pi_{\mathcal{H}_k} h\|_{H^{1+\frac{d}{2}+\delta}(\Omega)} \leq C\|h\|_{H^{1+\frac{d}{2}+\delta}(\Omega)}.$$


%Thanks to \cite[Theorem 8.1.11]{brenner2008mathematical} and \cite{guzman2009holder}, we have that $\Pi_{\mathcal{H}_k}$ is uniformly $H^1$-stable and $\Pi_{\hY_k}$ is $W^{1,\infty}$-stable.
%Given the initial data $u^0$ and $\psi^0$ from the continuous system, we choose $u_k^0 \in \mathcal{H}_k$ and $\hphi_k^0 \in \hY_k$ such that $u_k^0 = \Pi_{\mathcal{H}_k} u^0$ and $\hpsi_k^0=\Pi_{\hY_k} \hpsi^0$. 

The Galerkin equations read as follows: we wish to find a tuple $(u_k,\hphi_k) \in \mathcal{H}_k \times \hY_k$ for each $k \geq 1$ such that $u_k(0)=u_k^0:= \Pi_{\mathcal{H}_k}u^0$, $(\gb*\hphi_k)(0)=\hpsi_k^0:=\Pi_{\hY_k} \hpsi^0$, and
\begin{align}(\pt u_k,v_k )_{\mathcal{H}}  &+ ((u_k \cdot \nablax)u_k,v_k)_{\mathcal{H}} + \nu (\nablax u_k,\nablax v_k)_{\mathcal{H}}  \label{Eq:NS_dis} \\ 
& + \,k_B \mu_T (\C(M \gb *\hphi_k),\nablax v_k)_{\mathcal{H}} =0, \notag \\
	(\pt (\gb*\hphi_k),\hat\zeta_k)_\hY  &+  
	((u_k \cdot \nablax) \hphi_k, \hzeta_k)_\hY + \frac{1}{2\lambda} (\nablaq \hphi_k,\nablaq \hat\zeta_k)_\hY  \label{Eq:FP_dis} \\ &+ \eps (\nablax \hphi_k,\nabla_x \hat\zeta_k)_\hY- (\omega(u_k) q \hphi_k, \nablaq \hat\zeta_k)_\hY=0, \notag \end{align} for all $v_k \in \mathcal{H}_k$ and $\hat\zeta_k \in \hY_k$.  %\medskip

\begin{lemma} \label{Eq:LemmaExistenceODE}
	Suppose that \cref{Ass:WellPosedness} holds; then, for each $k \geq 1$,
	there exists a local-in-time solution $(u_k,\hphi_k)$ to the Galerkin system \cref{Eq:NS_dis}, \cref{Eq:FP_dis}.
	\end{lemma} %\medskip

\begin{proof}
Let $U_k(t):=(u_k^1(t),\ldots,u_k^k(t))^{\mathrm{T}}$ and $\widehat\Phi_k(t)=(\widehat{\phi}_1^k,\ldots, \widehat{\phi}_k^k)^{\rm T}$. With this notation the Galerkin subsystem \eqref{Eq:NS_dis} becomes an initial-value problem for a system of ordinary differential equations of the form $\ddt U_k  = F(t, U_k, \widehat\Phi_k)$, while, by noting that $\ddt (\gb*\hphi_k)=\pta \hphi_k$, the Galerkin subsystem \eqref{Eq:FP_dis} takes the form of an initial-value problem for a system of fractional-order ordinary differential equations $(\ddt)^\alpha \widehat\Phi_k  = G(t, U_k, \widehat\Phi_k)$. As the functions $F$ and $G$ are continuous with respect to their arguments and locally Lipschitz continuous with respect to their second and third arguments, we can appeal to the generalization of the Cauchy--Lipschitz theorem stated in Theorem 5.1 of \cite{diethelm2010analysis} to deduce the existence of a unique continuous solution, defined on a time interval $[0, T_k]$ where $0<T_k \leq T$, where $U_k$ is, in fact, a continuously differentiable function of $t$ by the classical Cauchy--Lipschitz theorem. 
\end{proof}

\subsubsection{Energy estimates} Next, we derive a $k$-uniform energy estimate, which will allow us to extend, for each $k \geq 1$, the corresponding local-in-time Galerkin solution, whose existence is guaranteed by Lemma \ref{Eq:LemmaExistenceODE}, to the entire time interval $[0,T]$; it will also enable us to extract weakly converging subsequences of Galerkin approximations.  We begin by deriving a bound on the solution to the Galerkin approximation of the Navier--Stokes equation; we shall then derive a bound on the solution to the Galerkin approximation of the Fokker--Planck equation. At the end, we will add the two bounds and apply Gronwall's lemma to obtain a $k$-uniform energy estimate. %\medskip

\begin{lemma} \label{Lem:EstU} Let \cref{Ass:WellPosedness} hold; then the following bound on the Galerkin solution $u_k$, in terms of $u_k^0$ and $\hphi_k$, holds for all $t \in (0,T_k)$:
	\begin{equation} \label{Eq:EnergyU2}
		\frac12 \|u_k(t)\|_{\mathcal{H}}^2  + \frac{\nu}{2} \|\nablax u_k\|_{L^2_t{\mathcal{H}}}^2  \leq \frac12 \|u^0_h\|_{\mathcal{H}}^2 +\frac{ T^{2-2\alpha} k_B^2\mu_T^2}{2\nu (1-\alpha)^2}  \|\hphi_k\|_{L^2_t\hY}^2.
	\end{equation}
\end{lemma} %\medskip

\begin{proof}
We take the test function $v_k=u_k(t)$ in the equation \cref{Eq:NS}, which gives
$$\begin{aligned}
	\frac12 \frac{\dd}{\dd t} \|u_k\|_{\mathcal{H}}^2  + \nu \|\nablax u_k\|_{\mathcal{H}}^2  &= -k_B \mu_T \big(\C(M\gb*\hphi),\nablax u_k\big)_{\mathcal{H}}, 
	\end{aligned}$$
and we can further bound the right-hand side from above to deduce that
$$\frac12 \frac{\dd}{\dd t} \|u_k\|_{\mathcal{H}}^2  + \nu \|\nablax u_k\|_{\mathcal{H}}^2  \leq \frac{\nu}{2} \|\nablax u_k\|_{\mathcal{H}}^2 + \frac{k_B^2\mu_T^2}{2\nu} \|\C(M\gb*\hphi_k)\|_{\mathcal{H}}^2. $$
By bounding the term  $\C(M\gb * \hphi_k)$ as in \cref{Eq:C} we arrive at the inequality
\begin{equation} \begin{aligned}\frac12 \frac{\dd}{\dd t} \|u_k\|_{\mathcal{H}}^2  + \frac{\nu}{2} \|\nablax u_k\|_{\mathcal{H}}^2  &\leq  \frac{k_B^2\mu_T^2}{2\nu} \|\gb*\hphi_k\|_\hY^2. %
\end{aligned} \label{Eq:EnergyU}
\end{equation}
Next, we note that, for any $t \in (0,T_k)$,
%
\begin{align}\label{Eq:EnergyU1} 
\int_0^t \|(\gb*\hphi_k)(s)\|_\hY^2 \ds \leq \int_0^t  \big(\gb * \|\hphi_k\|_\hY\big)^2(s) \ds\leq \|\gb\|_{L^1_t}^2 \|\hphi_k\|_{L^2_t\hY}^2.
\end{align}
This follows by observing that $g_{1-\alpha}$ only depends on the scalar variable $s$, which permits pulling the $\hY$-norm inside of the convolution, followed by applying Young's convolution inequality (cf. Lemma 3.2 in \cite{Oparnica}) in the resulting integrand. 

We then integrate \eqref{Eq:EnergyU} over the interval $[0,t]$ where $t \in (0,T_k)$ and use \eqref{Eq:EnergyU1} to bound the right-hand side of the resulting inequality. Finally we
note that $g_{1-\alpha}$ is integrable on $(0,t)$ and its integral is bounded by $T^{1-\alpha}/(1-\alpha)$. 
This gives \cref{Eq:EnergyU2}.
\end{proof} %%\medskip

Having derived a bound on $u_k$, we move on to the derivation of a bound on $\hphi^k$ by testing the Galerkin system \eqref{Eq:FP_dis}.  %\medskip

\begin{lemma} \label{Lem:EstPhi} Let \cref{Ass:WellPosedness} hold and let $\gamma>0$ be arbitrary but fixed; then, the following bound on the sequence of Galerkin solutions $\{(u_k,\hphi_k)\}_{k=1}^\infty$ holds:
\begin{equation} \label{Est:SolFP} \begin{aligned} & \frac{\gamma}{2} \big(\gb* \|\hphi_k-\init \ga\|_\hY^2\big)(t) + \frac{\gamma \,T^{-\alpha}}{16\, \Gamma(1-\alpha)} \|\hphi_k\|_{L^2_t\hY}^2  +\frac{\gamma}{2\lambda} \|\nablaq \hphi_k\|_{L^2_t\hY}^2 + \gamma \eps \| \nablax \hphi_k\|_{L^2_t \hY}^2  \\
	&\quad \leq C(\alpha,\gamma) \|M^{1/2} \init\|_{H^1(D;W^{1,\infty}(\Omega)) }^2  \int_0^t g_{2\alpha-1}(s) \|u_k(s)\|_{\mathcal{H}}^2 \ds
	+ C(\alpha,\gamma,T)\| \init\|_\hX^2.
\end{aligned}\end{equation}
\end{lemma} %\medskip

\begin{proof}
We note again that,  thanks to the inverse convolution property, see \cref{Eq:InverseConvolution}, $\hphi_k - \init \ga = \ga * \pta \hphi_k$.
We take this function as the test function in the variational Fokker--Planck equation \cref{Eq:FP}, i.e., $\hat\zeta= \hphi_k - \init \ga = \ga * \pta \hphi_k$, which gives  
\begin{equation} \label{Eq:TestingRHS}\begin{aligned} &(\pta \hphi_k,\ga * \pta \hphi_k)_\hY +\frac{1}{2\lambda}  \|\nablaq \hphi_k\|_\hY^2 + \eps \| \nablax \hphi_k\|_\hY^2\\ %
&\quad =\ga(t) \cdot \Big( \big((u_k \cdot \nablax) \hphi_k, \init\big)_\hY + \frac{1}{2\lambda} \big(\nablaq \hphi_k,\nablaq \init\big)_\hY  \\[0cm] &\qquad + \eps \big(\nablax \hphi_k,\nabla \init\big)_\hY- \big(\omega(u_k) q \hphi_k, \nablaq \init\big)_\hY \Big) =:R.
\end{aligned}\end{equation}
We then use the fractional chain inequality \cref{Eq:ChainOriginal} to bound the left-hand side of \cref{Eq:TestingRHS} from below, which yields
$$
\frac12 \pta \|\hphi_k-\ga \init\|^2_\hY  \leq (\pta \hphi_k,\hphi_k-\ga \init)_\hY = (\pta \hphi_k,\ga * \pta \hphi_k)_\hY. 
$$
Regarding the right-hand side of \cref{Eq:TestingRHS}, we integrate the last term containing $\omega(u_k)$ by parts, see
\cref{Eq:IntParts}, and get
$$\begin{aligned} -  \big(M\omega(u_k)  q \hphi, \nablaq \init\big)_{\mathcal{Y}} &=  \big(M \nablax \hphi_k (\nablaq \init)^{\mathrm{T}} q,u_k\big)_{\mathcal{Y}} + \big(M \hphi_k \nablax \nablaq \init q,u_k\big)_{\mathcal{Y}} \\ &\quad + \big(u_k \cdot q,M \nablax \hphi_k \cdot \nablaq \init\big)_{\mathcal{Y}} + \big(u_k \cdot q,M \hphi_k \div_x \nablaq \init\big)_{\mathcal{Y}}. 
\end{aligned} $$
We apply H\"older's inequality to obtain the following bound on the right-hand side, $R$, of the equality \cref{Eq:TestingRHS}:
$$\begin{aligned} R \leq{}& \ga(t) \cdot\! \Big(  \|u_k\|_{\mathcal{H}}  \|\nablax\hphi_k\|_\hY \|M^{1/2}\init\|_{L^2(D;L^\infty(\Omega)) }\\
%
%[-.2cm]  &+\|u_k\|_{\mathcal{H}}  \|\hphi_k\|_\hY \|M^{1/2}
%\init\|_{L^2(D;W^{1,\infty}(\Omega)) }  \\[0cm]  
%
&+ \frac{1}{2\lambda} \|\nablaq \hphi_k\|_\hY  \|\nablaq \init\|_\hY  + \eps \|\nablax \hphi_k\|_\hY \|\nablax \init\|_\hY   \\
%[-.1cm]  
&+ C\|u_k\|_{\mathcal{H}} \|q\|_{L^\infty(D)}  \|M^{1/2}\nablaq \init\|_{L^2(D;W^{1,\infty}(\Omega)) } \big(\|\hphi_k\|_\hY + \|\nablax \hphi_k\|_\hY\big) \Big).
\end{aligned}$$
Hence, thanks to Young's inequality, we arrive at the following bound on $R$: 
$$\begin{aligned} R \leq&~{} \frac{1}{\eps} \ga(t)^2 \|M^{1/2} \init\|_{L^2(D;W^{1,\infty}(\Omega)) }^2  \|u_k\|_{\mathcal{H}}^2  + \frac{\eps}{4} \|\nablax \hphi_k\|_\hY^2  \\ &{} + \frac{1}{4\lambda} \|\nablaq \hphi_k\|_\hY^2   + \frac{\ga(t)^2}{4\lambda} \|\nablaq \init\|_\hY^2  +\frac{\eps}{4} \|\nablax \hphi_k\|_\hY^2 + \eps \ga(t)^2 \|\nablax \init\|_\hY^2 \\
&{}+\delta  \|\hphi_k\|_\hY + \frac{\eps}{4} \|\nablax \hphi_k\|_\hY^2 + C(\eps,\delta) \ga(t)^2 \|M^{1/2}\nablaq \init\|_{L^2(D;W^{1,\infty}(\Omega)) }^2 \|u_k\|_{\mathcal{H}}^2,
\end{aligned}$$
where $\delta>0$ is sufficiently small, to be chosen appropriately later on. After combining the lower bound on the left-hand side of \cref{Eq:TestingRHS} and the upper bound on the right-hand side we have that
$$\begin{aligned} &\frac12 \pta \|\hphi_k-\ga \init\|^2_\hY +\frac{1}{2\lambda}  \|\nablaq \hphi_k\|_\hY^2 + \eps \| \nablax \hphi_k\|_\hY^2\\ &\quad \leq \frac{1}{\eps}\ga(t)^2 \|M^{1/2} \init\|_{L^2(D;W^{1,\infty}(\Omega)) }^2  \|u_k\|_{\mathcal{H}}^2  + \frac{\eps}{4} \|\nablax \hphi_k\|_\hY^2  \\ &{} \qquad + \frac{1}{4\lambda} \|\nablaq \hphi_k\|_\hY^2   + \frac{\ga(t)^2}{4\lambda} \|\nablaq \init\|_\hY^2  +\frac{\eps}{4} \|\nablax \hphi_k\|_\hY^2 + \eps \ga(t)^2 \|\nablax \init\|_\hY^2 \\
&{}\qquad +\delta  \|\hphi_k\|_\hY^2 + \frac{\eps}{4} \|\nablax \hphi_k\|_\hY^2 + C(\eps,\delta) \ga(t)^2 \|M^{1/2}\nablaq \init\|_{L^2(D;W^{1,\infty}(\Omega)) }^2 \|u_k\|_{\mathcal{H}}^2,\end{aligned}$$
and absorbing terms on the right-hand side into the left-hand side gives
$$\begin{aligned} &\frac12 \pta \|\hphi_k-\ga \init\|^2_\hY +\frac{1}{4\lambda}  \|\nablaq \hphi_k\|_\hY^2 + \frac{\eps}{2} \| \nablax \hphi_k\|_\hY^2\\ &\quad \leq C(\eps,\delta) \ga(t)^2  \|M^{1/2} \init\|_{H^1(D;W^{1,\infty}(\Omega))}   \|u_k\|_{\mathcal{H}}^2  + \delta \|\hphi_k\|_\hY^2 + C(\eps,\delta)\ga(t)^2 \|\init\|_\hX^2.
\end{aligned}$$

We note that $\ga^2=\frac{\Gamma(2\alpha-1)}{\Gamma(\alpha)^2}g_{2\alpha-1}$ is integrable for $\alpha\in (\tfrac12,1)$ and $g_1*g_{2\alpha-1}=g_{2\alpha}$, which is continuous, bounded, and monotonically increasing on $[0,T]$ for $\alpha\in(\tfrac12,1)$.  We integrate the inequality over $(0,t)$ and exploit the representation $\pta v = \pt (g_{1-\alpha} * v)$ of the Riemann--Liouville derivative, which gives
\begin{equation} \label{Eq:EnergyFP}\begin{aligned} & \frac12 \big(\gb* \|\hphi_k-\init \ga\|_\hY^2\big)(t)  +\frac{1}{4\lambda} \|\nablaq \hphi_k\|_{L^2_t\hY}^2 + \frac{\eps}{2} \| \nablax \hphi_k\|_{L^2_t \hY}^2 - \delta \|\hphi_k\|^2_{L^2_t\hY}  \\
&\quad \leq C(\delta,\alpha) \|M^{1/2} \init\|_{H^1(D;W^{1,\infty}(\Omega)) }^2  \int_0^t g_{2\alpha-1}(s) \|u_k(s)\|_{\mathcal{H}}^2 \ds
+ C(\delta)\| \init\|_\hX^2  g_{2\alpha}(T).
\end{aligned}\end{equation}
Further, we derive a lower bound on the first term of the left-hand side by noting that $(g_1*v)(t) \leq T^{\alpha} \Gamma(1-\alpha) (\gb*v)(t)$, see \cref{Eq:KernelNorm}, and therefore we have that
$$\begin{aligned} &\frac12 \big(\gb* \|\hphi_k-\init \ga\|_\hY^2\big)(t) \\ &\quad \geq \frac{T^{-\alpha}}{2\Gamma(1-\alpha)} \int_0^t  \|\hphi_k(s)-\init \ga(s)\|_{\hY}^2 \ds \\ &\quad \geq \frac{T^{-\alpha}}{2\Gamma(1-\alpha)} \int_0^t \big|\,\|\hphi_k(s)\|_\hY-\ga(s)\|\init \|_{\hY} \big|^2\ds \\
&\quad =\frac{T^{-\alpha}}{2\Gamma(1-\alpha)} \int_0^t \|\hphi_k(s)\|_\hY^2 - 2\ga(s)\|\hphi_k(s)\|_\hY\|\init \|_{\hY} +g_{\alpha}^2(s)\|\init \|_{\hY}^2\ds,
\end{aligned} $$
where we applied the reverse triangle inequality in the second estimate.
The function $g_{\alpha}$ belongs to $L^2(0,t)$ for any $\alpha\in (\tfrac12,1)$ and the integral of $g_\alpha^2$ is positive. We apply H\"older's inequality in the second term of the integrand and note that the $L^2(0,t)$-norm of $g_\alpha$ has the upper bound $C(\alpha)T^{\alpha-1/2}$. We thus have that
$$\begin{aligned}   \frac12 \big(\gb* \|\hphi_k-\init \ga\|_\hY^2\big)(t)   &\geq 
\frac{T^{-\alpha}}{2\Gamma(1-\alpha)} \left[
\frac12 \|\hphi_k\|_{L^2_t\hY}^2 - 2 C(\alpha) T^{\alpha-1/2} \|\init \|_{\hY}  \|\hphi_k\|_{L^2_t\hY} \right]\\ &\geq \frac{T^{-\alpha}}{2\Gamma(1-\alpha)}\left[\frac14 \|\hphi_k\|_{L^2_t\hY}^2 - C(T,\alpha) \|\init \|_{\hY}^2\right],
\end{aligned} $$
where we have applied Young's inequality in the last step. We multiply the energy estimate \cref{Eq:EnergyFP} by $\gamma>0$ and obtain for $\delta=\frac{T^{-\alpha}}{16 \Gamma(1-\alpha)}$ the estimate \cref{Est:SolFP}.
\end{proof} %\medskip

Next, we combine the estimates on $u_k$ and $\hphi_k$, see \cref{Lem:EstU} and \cref{Lem:EstPhi}, to obtain a $k$-uniform bound. %\medskip

\begin{lemma} \label{Lem:EstComb}
	Let \cref{Ass:WellPosedness} hold; then, the following $k$-uniform estimate on the Galerkin solution $(u_k,\hphi_k)$ holds:
	\begin{equation} \label{Eq:EnergyIneq}
		\begin{aligned} &  \big(\gb* \|\hphi_k-\init \ga\|_\hY^2\big)(t) +  \|\hphi_k\|_{L^2_t\hX}^2    + \|u_k(t)\|_{\mathcal{H}}^2 +  \|\nabla u_k\|_{L^2_t{\mathcal{H}}}^2 \\
			&\quad \leq C\big(\alpha,T, \|u^0\|_{\mathcal{H}}^2,\|M^{1/2} \hpsi^0\|^2_{H^1(D;H^{1+d/2+\delta}(\Omega)) }\big).
	\end{aligned}\end{equation}
\end{lemma} 

\begin{proof}
We add the integrated velocity inequality \cref{Eq:EnergyU2} to the bound \cref{Est:SolFP} and obtain the following combined bound:
$$\begin{aligned} &\frac{\gamma}{2} \big(\gb* \|\hphi_k-\init \ga\|_\hY^2\big)(t) + \frac{\gamma\, T^{-\alpha}}{16\,\Gamma(1-\alpha)} \|\hphi_k\|_{L^2_t\hY}^2  +\frac{\gamma}{2\lambda} \|\nablaq \hphi_k\|_{L^2_t\hY}^2 + \gamma \eps \| \nablax \hphi_k\|_{L^2_t \hY}^2  \\ &\quad + \frac12 \|u_k(t)\|_{\mathcal{H}}^2 + \frac{\nu}{2} \|\nabla u_k\|_{L^2_t{\mathcal{H}}}^2 \\
&\leq C(\alpha,\gamma) \|M^{1/2} \init\|_{H^1(D;W^{1,\infty}(\Omega)) }^2  \int_0^t g_{2\alpha-1}(s) \|u_k(s)\|_{\mathcal{H}}^2 \ds
+ C(\alpha,\gamma,T)\| \init\|_\hX^2 \\&\quad + \frac12 \|u_k^0\|_{\mathcal{H}}^2 + %
\frac{ T^{2-2\alpha} k_B^2\mu_T^2}{2\nu (1-\alpha)^2}  \|\hphi_k\|_{L^2_t\hY}^2.
\end{aligned}$$
We now choose $\gamma$ such that 
$$\frac{\gamma\, T^{-\alpha}}{16\, \Gamma(1-\alpha)} \geq \frac{ T^{2-2\alpha} k_B^2\mu_T^2}{\nu (1-\alpha)^2}.$$
Hence, we can absorb the last term on the right-hand side into the second term on the left-hand side, and 
 the combined energy inequality thus becomes
\begin{equation} \label{Eq:EnergyBack}\begin{aligned} & \big(\gb* \|\hphi_k-\init \ga\|_\hY^2\big)(t) +  \|\hphi_k\|_{L^2_t\hX}^2    + \|u_k(t)\|_{\mathcal{H}}^2 +  \|\nabla u_k\|_{L^2_t{\mathcal{H}}}^2 \\
&\quad\leq C(\alpha,T) \|M^{1/2} \init\|_{H^1(D;W^{1,\infty}(\Omega)) }^2  \int_0^t g_{2\alpha-1}(s) \|u_k(s)\|_{\mathcal{H}}^2 \ds
\\ &\qquad + C(\alpha,T) \big( \| \init\|_\hX^2 + \|u_k^0\|_{\mathcal{H}}^2\big),
\end{aligned}\end{equation}
where we took the minimum of each prefactor of the norms on the left-hand side and divided the inequality by this value. Gronwall's lemma then implies that
\begin{equation} 
\begin{aligned} \label{Eq:EnergyBack1}
& \big(\gb* \|\hphi_k-\init \ga\|_\hY^2\big)(t) +  \|\hphi_k\|_{L^2_t\hX}^2    + \|u_k(t)\|_{\mathcal{H}}^2 +  \|\nabla u_k\|_{L^2_t{\mathcal{H}}}^2 \\
&\quad \leq C(\alpha,T) \cdot \big(  \| \init\|_\hX^2  + \|u_k^0\|_{\mathcal{H}}^2  \big) \cdot \textup{exp}\bigg(\frac{T^{2\alpha-1}}{2\alpha-1} \|M^{1/2} \init\|^2_{H^1(D;W^{1,\infty}(\Omega)) }\bigg).
\end{aligned}
\end{equation}


We note that the initial conditions of the Galerkin system are defined by $u_k^0=\Pi_{\mathcal{H}_k} u^0$ and $\init=\Pi_{\hY_k} \hpsi^0$. Therefore, we have that $\|u_k^0\|_{\mathcal{H}}^2 \leq \|u^0\|_{\mathcal{H}}^2$ and $$\|M^{1/2}\init\|_{H^1(D;W^{1,\infty}(\Omega)) }^2 \leq C \|M^{1/2}\hpsi^0\|_{H^1(D;H^{1+d/2+\delta}(\Omega)) }^2.$$
%see \cite[Theorem 8.1.11]{brenner2008mathematical} with regards to the stability of the Ritz projection in $W^{1,\infty}$.
We insert these bounds into the right-hand side of the inequality \cref{Eq:EnergyBack1} and we thus arrive at the desired $k$-uniform energy estimate \cref{Eq:EnergyIneq}.
\end{proof}





\subsection{Convergence of subsequences} 
Having derived the $k$-uniform energy estimate \cref{Eq:EnergyIneq} stated in \cref{Lem:EstComb},
we shall extract weakly/weakly-$*$ converging subsequences of Galerkin solutions $(u_k,\hphi_k)$. We shall also prove strong convergence of a subsequence $u_{k_j}$ in $L^2(0,T;\mathcal{H}_0)$ in order to pass to the limit $j\to \infty$ in the nonlinear terms in the variational Navier--Stokes--Fokker--Planck system.   %\medskip

\begin{lemma} Let \cref{Ass:WellPosedness} hold and assume that $r \in [1,\infty)$ for $d=2$ and $r \in [1,6)$ for $d=3$; then, the sequence of Galerkin solutions $(u_k,\hphi_k)$ from \cref{Eq:LemmaExistenceODE} contains a subsequence $(u_{k_j},\hphi_{k_j})$ that admits the following convergences as $j \to \infty$:
	%\footnote{\color{red}~I don't see where the 7th weak convergence, with spatial function space $\hY$, is coming from. Surely this is incorrect. -- This should be corrected now} 
	\begin{equation} \label{Eq:Weak} \begin{aligned}	
			u_{k_j} &\longweak u &&\text{weakly-$*$ in } L^\infty(0,T;\mathcal{H}_0), \\
			u_{k_j} &\longweak u &&\text{weakly\phantom{-*} in } L^2(0,T;\mathcal{V}_0) \cap L^{8/d}(0,T;L^4(\Omega)^d), \\
			\hphi_{k_j} &\longweak 		 \hphi &&\text{weakly\phantom{-*} in }
			L^2(0,T;\hX), \\
		\pt u_{k_j} &\longweak \pt u &&\text{weakly\phantom{-*} in } L^{8/(4+d)}(0,T;\mathcal{V}_0'), \\
		u_{k_j} &\longrightarrow  u &&\text{strongly\hspace{1mm} in } L^2\big(0,T;L^{r}(\Omega;\R^d)\big), \\
		u_{k_j} &\longrightarrow  u &&\text{strongly\hspace{1mm} in } C([0,T];\mathcal{V}_0'),\\
		\pt^{\alpha} \hphi_{k_j} &\longweak \pt^{\alpha} \hphi &&\text{weakly\phantom{-*} in } L^{8/(4+d)}(0,T;\hZ'),\\
		\hphi_{k_j} &\longrightarrow  \hphi &&\text{strongly\hspace{1mm} in } L^2(0,T;\hY),
\\ 
\C(M \gb * \hphi_{k_j}) &\longrightarrow \C(M\gb *\hphi) &&\text{strongly\hspace{1mm} in }
		L^2\big(0,T;L^2(\Omega;\R^{d\times d})\big), \\
		\gb*\hphi_{k_j} &\longweak \gb *\hphi &&\text{weakly-$*$ in } L^\infty(0,T;\hY) \cap L^2(0,T;\hX), \\
		\gb*\hphi_{k_j} &\longrightarrow \gb *\hphi &&\text{strongly\hspace{1mm} in }
		C([0,T];\hX') \cap L^2(0,T;\hY).
		\end{aligned}
	\end{equation} 
	\end{lemma}
\begin{proof}
In Lemma \ref{Lem:EstComb} we stated various $k$-uniform bounds on $u_k$ and $\hphi_k$. Thanks to the Banach--Alaoglu and Eberlein--\v{S}mulian theorems, see \cite[Theorem 8.10]{alt2016linear}, there are weakly/weakly-$*$ converging subsequences $u_{k_j}$ and $\hphi_{k_j}$. In particular, we obtain the convergences
\begin{equation} \label{Eq:Weak1} \begin{aligned}	
u_{k_j} &\longweak u &&\text{weakly-$*$ in } L^\infty(0,T;\mathcal{H}_0), \\
u_{k_j} &\longweak u &&\text{weakly\phantom{-*} in } L^2(0,T;\mathcal{V}_0), \\
\hphi_{k_j} &\longweak 		 \hphi &&\text{weakly\phantom{-*} in }
L^2(0,T;\hX). 
\end{aligned}\end{equation}


We shall establish the strong convergence of $u_{k_j}$ in $L^2(0,T;\mathcal{H}_0)$ by applying the Aubin--Lions compactness lemma; see \cref{Eq:aubin}. To this end, we need to bound the time derivative of $u_k$ in a suitable dual space. Let us therefore consider an arbitrary element $v \in L^{8/(4-d)}(0,T;\mathcal{V}_0)$ and bound each of the terms appearing on the right-hand side of \cref{Eq:NS_dis} below by means of H\"older's inequality:
$$\begin{aligned}\int_0^T | \langle \pt u_k,v \rangle_{\mathcal{V}_0}| \dt   &= \int_0^T  \Big| -((u_k \cdot \nablax)u_k,\Pi_{\mathcal{H}_k} v)_{\mathcal{H}}  \\ 
&\quad  - \nu (\nablax u_k,\nablax \Pi_{\mathcal{H}_k} v)_{\mathcal{H}} - k_B \mu_T \big(\C(M\gb*\hphi_k),\nablax \Pi_{\mathcal{H}_k} v \big)_{\mathcal{H}} \Big| \dt
	\\ 
& \leq C \int_0^T \Big( \|u_k\|_{L^4(\Omega)} \|u_k\|_{\mathcal{V}} \|\Pi_{\mathcal{H}_k} v\|_{L^4(\Omega)}  \\ &\quad  + \|u_k\|_{\mathcal{V}} \|\Pi_{\mathcal{H}_k} v\|_{\mathcal{V}} +  \|\C(M\gb*\hphi_k)\|_{\mathcal{H}}  \|\Pi_{\mathcal{H}_k} v\|_{\mathcal{V}} \Big) \dt. 
\end{aligned}$$
Hence, using Ladyzhenskaya's inequality, we have that
$$\begin{aligned}
&\int_0^T | \langle \pt u_k,v \rangle_{\mathcal{V}_0}| \dt\\  
 &\leq C \int_0^T \Big( \|u_k\|_{\mathcal H}^{1-d/4} \|u_k\|_{\mathcal{V}}^{1+d/4} \|\Pi_{\mathcal{H}_k} v\|_{L^4(\Omega)} + \|u_k\|_{\mathcal{V}} \|\Pi_{\mathcal{H}_k} v\|_{\mathcal{V}} +  \|\hphi_k\|_\hY  \|\Pi_{\mathcal{H}_k} v\|_{\mathcal{V}} \Big) \dt 
 \\
&\leq C \Big( \|u_k\|_{L^\infty {\mathcal{H}}}^{1-d/4} \|u_k\|_{L^2{\mathcal{V}}}^{1+d/4} \|v\|_{L^{8/(4-d)} {\mathcal{V}}} + \|u_k\|_{L^2{\mathcal{V}}} \|v\|_{L^2{\mathcal{V}}} +  \|\hphi_k\|_{L^2\hY} \|v\|_{L^2{\mathcal{V}}} \Big)
	\\
	&\leq C\|v\|_{L^{8/(4-d)} {\mathcal{V}}}.
\end{aligned}$$
This then implies that $\pt u_k$ is bounded in $L^{8/(4+d)}(0,T;\mathcal{V}_0
 ')$. 
It follows by the Aubin--Lions lemma \cref{Eq:aubin} that  %
\begin{equation} \label{Eq:Weak2} \begin{aligned}	
\pt u_{k_j} &\longweak \pt u &&\text{weakly\phantom{-*} in } L^{8/(4+d)}(0,T;\mathcal{V}_0'), \\
u_{k_j} &\longrightarrow  u &&\text{strongly\hspace{1mm} in } L^2\big(0,T;L^{r}(\Omega;\R^d)\big), \\
u_{k_j} &\longrightarrow  u &&\text{strongly\hspace{1mm} in } C([0,T];\mathcal{V}_0'),
\end{aligned}\end{equation}
where $r \in [1,\infty)$ for $d=2$ and $r \in [1,6)$ for $d=3$.


Similarly, we consider an arbitrary element $\hzeta \in L^\frac{8}{4-d}(0,T;\hZ)$ and we recall that $\hZ$ was defined in the beginning of \cref{Sec:Maxwell}. We test the Galerkin equation of $\hphi_k$ with $\Pi_{\mathcal H_k} \hzeta$ giving
\begin{equation} \label{Eq:BoundDerivative} \begin{aligned}\int_0^T \!\! |\langle \pta \hphi_k,\Pi_{\mathcal H_k} \hzeta \rangle_{\hX}| \dt  &=\!
\int_0^T\!\! \Big| -((u_k \cdot \nablax) \hphi_k, \Pi_{\mathcal H_k} \hzeta)_\hY - \frac{1}{2\lambda} (\nablaq \hphi_k,\nablaq \Pi_{\mathcal H_k} \hzeta)_\hY   \\ &\quad - \eps (\nablax \hphi_k,\nabla_x \Pi_{\mathcal H_k} \hzeta)_\hY+ (\omega(u_k) q \hphi_k, \nablaq \Pi_{\mathcal H_k} \hzeta)_\hY \Big|\d t.\end{aligned} \end{equation}
We note that $u_k$ is bounded in $L^{8/d}(0,T;L^4(\Omega)^d)$ by the following interpolation result
$$\int_0^T \|u_k\|_{L^4}^{8/d} \dt \leq \int_0^T  \|u_k\|_{\mathcal H}^{8/d-2} \|u_k\|^{2}_{\mathcal V} \dt \leq \|u_k\|_{L^\infty \mathcal{H}}^{8/d-2} \|u_k\|_{L^{2} \mathcal{V}}^{2}.$$
Regarding the last term in \eqref{Eq:BoundDerivative}, we integrate by parts and estimate by H\"older's inequality
$$\begin{aligned} -  &\big(\omega(u_k)  q \hphi, \nablaq \hzeta\big)_\hY \\ &=  \big(\nablax \hphi_k (\nablaq \hzeta)^{\mathrm{T}} q,u_k\big)_\hY + \big( \hphi_k \nablax \nablaq \hzeta q,u_k\big)_\hY + \big(u_k \cdot q, \nablax \hphi_k \cdot \nablaq \hzeta\big)_\hY \\ &\quad  + \big(u_k \cdot q, \hphi_k \div_x \nablaq \hzeta\big)_\hY \\
	&\leq C  \| \nablax \hphi_k\|_{L^2\hY}
	\big( \|\nablaq \hzeta\|_{L^{8/(4-d)} \hY}  +  \|\nablax \nablaq \hzeta\|_{L^{8/(4-d)} \hY}  \big) \|q\|_{L^\infty} \|u_k\|_{L^{8/d} L^4}  \\ &\quad+ \| \hphi_k\|_{L^2\hX} \|\nablax \nablaq \hzeta\|_{{L^{8/(4-d)} \hY}} \|q\|_{L^\infty} \|u_k\|_{L^{8/d} L^4}  \\ &\quad + C \|u_k\|_{L^{8/d} L^4} \|q\|_{L^\infty} \|\nablax \hphi_k\|_{L^2 \hY} \big(\|\nablaq \hzeta\|_{L^{8/(4-d)}\hY}  + \|\nablax \nablaq \hzeta\|_{L^{8/(4-d)}\hY} \big) \\ &\quad +\|u_k\|_{L^{8/d} L^4} \|q\|_{L^\infty}  \|\hphi_k\|_{L^2 \hX} \|\div_x \nablaq \hzeta\|_{L^{8/(4-d)}\hY}  \\
	&\leq C \|u_k\|_{L^{8/d} L^4} \|q\|_{L^\infty}  \|\hphi_k\|_{L^2 \hX} \| \hzeta\|_{L^{8/(4-d)}(0,T;\hZ) } .
\end{aligned} $$
Using this estimate, we can bound \eqref{Eq:BoundDerivative} as follows:
\begin{equation} \label{Eq:BoundDerivativePhi} \begin{aligned}& \int_0^T |\langle \pta \hphi_k,\Pi_{\mathcal H_k} \hzeta \rangle_{\hX}| \dt  \\ &\leq C \Big( \|u_k \|_{L^\infty\mathcal{H}_0 }  \|\hphi_k \|_{L^2 \hX}   \|\hzeta \|_{L^2\hX} +  \|\nablaq \hphi_k\|_{L^2\hY}   \|\nablaq \hzeta \|_{L^2\hY}    \\ &\quad +  \| \nablax \hphi_k\|_{L^2\hY}   \|\nabla_x\hzeta \|_{L^2\hY} + \|u_k\|_{L^{8/d} L^4} \|q\|_{L^\infty}  \|\hphi_k\|_{L^2 \hX} \| \hzeta\|_{L^{8/(4-d)}(0,T;\hZ) } \Big) \\
		&\leq  C \|\hzeta\|_{L^{8/(4-d)}(0,T;\hZ) } .
\end{aligned} \end{equation}
Hence, we obtain the $k$-uniform boundedness of $\pt(\gb*\hphi_{k})=\pt^{\alpha} \hphi_k$ in the space $L^{8/(4+d)}(0,T;\hZ')$, which is continuously embedded  in $L^{8/(4+d)}(0,T;(H^2_M(\Omega \times D))')$. 
Therefore, we are in the setting of the Gelfand triple
$$\hX \com \hY\hookrightarrow \big( H_M^2(\Omega \times D)\big)'.$$ 
%and $\hphi_{k}$ is bounded in the space 
%$$L^2(0,T;\hX) \cap W^{\alpha,8/(4+d)}(0,T;(H^2(\Omega \times D;M))').$$
We thus obtain from the fractional Aubin--Lions lemma, see \eqref{Eq:aubinfractional2}, that
\begin{equation} \label{Eq:Weak3} \begin{aligned}	
		\pt^{\alpha} \hphi_{k_j} &\longweak \pt^{\alpha} \hphi &&\text{weakly\phantom{-*} in } L^{8/(4+d)}(0,T;\hZ'), \\
		\hphi_{k_j} &\longrightarrow  \hphi &&\text{strongly\hspace{1mm} in } L^2(0,T;\hY).
\end{aligned} \end{equation}

%We note that we can test the weak form of $\hphi_k$ again by $\hzeta=g_\alpha * \pta \hphi_k$, i.e., we consider the tested weak form \cref{Eq:TestingRHS}. However, this time we do not apply the fractional chain inequality on the first term on the left-hand side of \cref{Eq:TestingRHS} but we exploit the coercivity of the kernel function in the integrated weak form, see
% \cref{Eq:Coercive2}, to obtain
% $$\int_0^t (\pta \hphi_k,\hphi_k - \ga \hpsi_k^0)_\hY \ds  \geq  \cos(\alpha \pi/2)  \Big( \tfrac12 \|\pt^{\alpha/2} \hphi_k\|^2_{L^2_t \hY}-\tfrac{\Gamma(\alpha-1)}{\Gamma(\alpha/2)^2}g_{\alpha}(t) \|\hpsi_k^0\|_\hY^2 \Big). $$
  %We take $t=T$ and use the derived energy estimates to bound the right-hand side of \cref{Eq:TestingRHS} similiarly to before. 
 % We can apply the fractional Aubin--Lions lemma, see \eqref{Eq:aubinfractional2}, to obtain the strong convergence of $\hphi_{k_j}$ in $L^2(0,T;\hY)$; in other words, 
The convolution $\gb*\hphi_{k}$ is bounded in $L^2(0,T;\hX)$ thanks to Young's convolution inequality 
$$\|\gb * \hphi_k\|_{L^2_t\hX} \leq \|g_{1-\alpha}\|_{L^1_t} \|\hphi_{k}\|_{L^2_t\hX} \leq C T^{1-\alpha} \|\hphi_{k}\|_{L^2_t\hX}.$$
Moreover,  $\gb*\hphi_{k}$ is bounded in $L^\infty(0,T;\hY)$ by the following chain of estimates:
$$\begin{aligned} &\|\gb*\hphi_{k}\|_{L^\infty \hY} \\&\quad \leq \sup_{t \in (0,T)} \int_0^t \gb(t-s) \|\hphi_{k}(s)\|_\hY \ds \\ 
	&\quad \leq \sup_{t \in (0,T)} \int_0^t \gb(t-s) \|\hphi_{k}(s)-\hpsi_{k}^0 \ga(s)\|_\hY \ds + (\gb*\ga)(t) \|\hpsi_{k}^0\|_\hY   \\ 
	&\quad \leq   \sup_{t \in (0,T)} \int_0^t \gb(t-s) \|\hphi_{k}(s)-\hpsi_{k}^0 \ga(s)\|^2_\hY \ds + \frac14 \int_0^t \gb(t-s) \ds + \|\hpsi_{k}^0\|_\hY  
	\\ &\quad = \sup_{t \in (0,T)} (\gb*\|\hphi_{k}-\hpsi_{k}^0 \ga\|_\hY^2)(t) + \frac14 g_{2-\alpha}(T) + \|\hpsi_{k}^0\|_\hY,
\end{aligned}$$
and the first term on the right-hand side is bounded by \cref{Lem:EstComb}. Since we have already proved a bound on $\pt(\gb*\hphi_{k})=\pta \hphi_{k}$, see \eqref{Eq:BoundDerivativePhi}, we may use the Aubin--Lions lemma, see \cref{Eq:aubin}, to obtain the following strong convergence results: 
\begin{equation} \label{Eq:Weak5} \begin{aligned}	
		\gb*\hphi_{k_j} &\longrightarrow  \gb*\hphi &&\text{strongly\hspace{1mm} in } L^2(0,T;\hY), \\
		\gb*\hphi_{k_j} &\longrightarrow  \gb*\hphi &&\text{strongly\hspace{1mm} in } C([0,T];\hX').
\end{aligned} \end{equation}
Lastly, we note that the mapping $M\gb*\varphi \mapsto \C(M\gb*\varphi)$ is linear and continuous thanks to \eqref{Eq:C},  %\leq C \|g_{1-\alpha}\|_{L^1_t} \|\varphi\|_{L^2_t\hY} \leq C T^{1-\alpha} \|\varphi\|_{L^2_t\hY}, $$
and therefore we have from \eqref{Eq:Weak5}$_1$ that
\begin{equation} \label{Eq:Weak4} \C(M \gb * \hphi_{k_j}) \longrightarrow \C(M\gb *\hphi) \quad \text{ strongly in }
L^2\big(0,T;L^2(\Omega;\R^{d\times d})\big).
\end{equation} 
\end{proof}

\subsection{Passage to the limit}
Next, we pass to the limit $j \to \infty$ in the  time-integrated $k_j$-th Galerkin system \cref{Eq:NS_dis}, \cref{Eq:FP_dis}. Specifically, we shall use the convergence results stated in the preceding lemma to show that the weak limits, $u$ and $\hphi$, satisfy the variational Navier--Stokes--Fokker--Planck system in the sense of \cref{Eq:DefWeak}. 
%
\begin{proof}[Proof of \cref{Thm:WellPosedness}]
We consider the time-integrated Galerkin system
\begin{align} 
&\int_0^T \Big(\langle\pt u_{k_j},v \rangle_V  + ((u_{k_j} \cdot \nablax)u_{k_j},v)_{\mathcal{H}} \label{Eq:NS_dis_time} \\ 
& \quad + \nu (\nablax u_{k_j},\nablax v)_{\mathcal{H}}+ k \mu (C(M \gb *\hphi_{k_j}),\nablax v)_{\mathcal{H}} \Big) \eta(t) \dt =0  \notag\\
&\int_0^T  -(\gb*\hphi_{k_j},\hzeta)_\hY \eta'(t) + \Big(    ((u_{k_j}\cdot \nablax) \hphi_{k_j}, \hzeta)_\hY+ \frac{1}{2\lambda} ( \nablaq \hphi_{k_j},\nablaq \hat\zeta)_\hY \label{Eq:FP_dis_time} \\
&\quad + \eps (\nablax \hphi_{k_j},\nabla \hat\zeta)_\hY- (\omega(u_{k_j}) q \hphi_{k_j}, \nablaq \hat\zeta)_\hY \Big) \eta(t) \dt=0, \notag \end{align} for all $v \in \H_{k_j}$, $\eta \in C_0^\infty(0,T)$ and $\hat\zeta \in \hY_{k_j}$.  
Passing to the limit $j \rightarrow \infty$ in \eqref{Eq:NS_dis_time} using \cref{Eq:Weak1}--\cref{Eq:Weak4}
is standard, and results in \eqref{Eq:NS}. It therefore remains to pass to the limit $j \rightarrow \infty$ in \eqref{Eq:FP_dis_time}. In particular, the convergence of the linear terms follow immediately by weak convergence, and we only consider the two nonlinear terms. 
 We note that $\hphi_{k_j} \to \hphi$ strongly in $L^2\big(0,T;\hY)$ and $\omega(u_{k_j}) \to \omega(u)$ weakly in $L^2(0,T;\H_0)$, from which we deduce that
$$\int_0^T (\omega(u_{k_j}) q \hphi_{k_j}, \nablaq \hat\zeta)_\hY  \eta(t) \dt \longrightarrow  \int_0^T (\omega(u) q \hphi, \nablaq \hat\zeta)_\hY  \eta(t) \dt,$$
as $j \to \infty$. With the same reasoning, we are able to show that
$$\int_0^T  ((u_{k_j}\cdot \nablax) \hphi_{k_j}, \hat\zeta)_{\hY} \eta(t) \dt \longrightarrow \int_0^T  ( (u \cdot\nablax) \hphi, \hat\zeta)_{\hY} \eta(t) \dt\quad \mbox{as $j \to \infty$}.$$

%To this end we apply integration by parts in time in \eqref{Eq:FP_dis_time}, which yields
%
%$$\begin{aligned} 0=&- \int_0^T ( M\gb * \hphi_{k_j},\hat\zeta)_{\mathcal{Y}} \eta'(t) \dt - ( M\hpsi_0,\hat\zeta)_{\mathcal{Y}} \eta(0) && \\ &+  
%\int_0^T \Big( ((u_{k_j}\cdot \nablax) \hphi_{k_j}, \hzeta)_\hY+ \frac{1}{2\lambda} ( \nablaq \hphi_{k_j},\nablaq \hat\zeta)_\hY \\
%&\quad + \eps (\nablax \hphi_{k_j},\nabla \hat\zeta)_\hY- (\omega(u_{k_j}) q \hphi_{k_j}, \nablaq \hat\zeta)_\hY \Big) \eta(t) \dt=0, \notag 
%\end{aligned}$$
%for all $\hat\zeta \in \hX_{k_j}$ and $\eta \in C_0^\infty(-T,T)$. We apply Young's convolution inequality to the first term on the right-hand side to obtain
%$$\begin{aligned} - \int_0^T (M \gb * \hphi_{k_j},\hat\zeta)_{\mathcal{Y}} \eta'(t) \dt &\leq \|\gb * \hphi_{k_j}\|_{L^1\hY} \|\hat\zeta\|_\hY \|\eta'\|_{L^\infty_T}
%\\&\leq C \|\gb\|_{L^1_T} \|\hphi_{k_j}\|_{L^1 \hY} \|\hat\zeta\|_\hY  \|\eta'\|_{L^\infty_T},
%\end{aligned} $$
%from which we conclude the convergence of the first term.
%We exploit the corotationality of $\omega$ in the form of
%$$(M\omega(u)q,\eta)_{\mathcal{Y}} = \frac12 (M v \cdot q, \div_x \eta)_{\mathcal{Y}} - \frac12 (M\nablax \eta q,v)_{\mathcal{Y}},$$
%which allows us to rewrite \cref{Eq:FP_dis_time} as
%$$\begin{aligned} 0=&- \int_0^T ( M\gb * \hphi_{k_j},\hat\zeta)_{\mathcal{Y}} \eta'(t) \dt - ( M\hpsi_0,\hat\zeta)_{\mathcal{Y}} \eta(0) && \\  &+ \frac12 \int_0^T \Big(  2 (M (u_{k_j} \cdot \nablax) \hphi_{k_j}, \hat\zeta)_{\mathcal{Y}}  + \frac{1}{\lambda} (M \nablaq \hphi_{k_j},\nablaq \hat\zeta)_{\mathcal{Y}} + 2\eps (M\nablax \hphi_{k_j},\nablax \hat\zeta)_{\mathcal{Y}} && \\   
%&\qquad \qquad - \big(M u_{k_j} \cdot q, \div_x(\hphi_{k_j} \nablaq \hat\zeta)\big)_{\mathcal{Y}} +  (M\nablax (\hphi_{k_j} \nablaq \hat\zeta) q,u_{k_j})_{\mathcal{Y}} \Big) \eta(t) \dt,
%\end{aligned}$$
%for all $\hat\zeta \in \hX_{k_j}$ and $\eta \in C_0^\infty(-T,T)$.

%{\color{red}~ I don't understand why it was necessary in the 7 lines, immediately above, to rewrite $(M\omega(u)q,\eta)$ as $\frac12 (M v \cdot q, \div_x \eta)_{\mathcal{Y}} - \frac12 (M\nablax \eta q,v)_{\mathcal{Y}}$. I also don't understand the argument below. What space does $\zeta$ belong to? }

%\medskip

%{\color{blue} Since $u_{k_j} \to u$ strongly in $L^2\big(0,T;L^4(\Omega)\big)$, $u_{k_j} \zeta \to u \zeta$ strongly in $L^2\big(0,T;L^2(\Omega \times D;\R^d)\big)$  and $\nablax \hphi_{k_j} \rightharpoonup \nablax \hphi$ weakly in $L^2(0,T;\hY)$, it follows that
%$$\int_0^T  (M (u_{k_j}\cdot \nablax) \hphi_{k_j}, \hat\zeta)_{\mathcal{Y}} \eta(t) \dt \longrightarrow \int_0^T  (M (u \cdot\nablax) \hphi, \hat\zeta)_{\mathcal{Y}} \eta(t) \dt.$$
%Passing to the limit $k \rightarrow \infty$ in the terms with prefactors $1/\lambda$ and $2\eps$ in the second integral on the right-hand side is straightforward using weak convergence 
%of $\hphi_k$ to $\hphi$ in $\widehat{\mathcal{X}}=L^2(0,T; H^1(\Omega \times D;M))$.


%It remains to deal with the last two terms. For the nonlinearities that couple the Fokker--Planck equation to the Navier--Stokes equation, we note again
%$$\int_0^T  ((u_k \cdot \nablax)\xi,\varphi)_{\mathcal{H}} + ((u_k \cdot \nablax)\varphi,\xi)_{\mathcal{H}} \dt = \int_0^T \int_\Omega \div_x((\xi \cdot \varphi)u_k) \, \text{d}x \dt=0,$$
%for any $\xi,\varphi \in L^2(0,T;\hX)$.}


We use the density of $\cup_{k=1}^\infty \H_{k}$ in $\mathcal{V}$ and of $\cup_{k=1}^\infty \hY_{k}$ in $H^2_M(\Omega \times D)$, which completes the proof by observing that the tuple $(u,\hphi)$ satisfies the variational form of the time-fractional Navier--Stokes--Fokker--Planck system as stated in \cref{Eq:DefWeak}. 

It remains to check that the initial conditions are satisfied. First, we obtain the convergence $u_{k_j}(0) \to u(0)$ in $\mathcal{V}_0'$  as $j \rightarrow \infty$; see again \cref{Eq:Weak}. However, by definition, $u_{k_j}(0)=\Pi_{\H_{k_j}} u^0$, which converges to $u^0$ in $\mathcal{H}_0$
as $j \rightarrow \infty$. By the uniqueness of the limit it follows that $u(0)=u^0$. Regarding the solution of the Fokker--Planck equation, we use again the strong convergence \cref{Eq:Weak} to conclude $(\gb*\hphi)(0) = \hpsi^0$.
\end{proof}


 









