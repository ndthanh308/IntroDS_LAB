	\section{Introduction}
	
This paper is concerned with the existence of weak solutions to a system of nonlinear partial differential equations that arises in the kinetic theory of dilute solutions of polymeric fluids. Within this class of models we focus on finitely-extensible nonlinear elastic, FENE-type, dumbbell models with a corotational drag term. In contrast to previous literature on the analysis of these models we assume power law waiting times in the derivation of the system, which results in the appearance of a time-fractional derivative in the Fokker--Planck equation describing the evolution of the probability density function. This raises new questions about the study of well-posedness, and we provide rigorous results concerning the existence of global-in-time weak solutions to the system of partial differential equations featuring in the model.

Dilute polymer models are derived and extensively described in the monograph \cite{bird1987dynamics2} and in the book by \"Ottinger \cite{ottinger2012stochastic}; see also \cite{suli2018mckeanvlasov} for a mathematically rigorous derivation of the Hookean bead-spring-chain model from Brownian dynamics. We also refer to the papers \cite{lemou2002viscoelastic,herrchen1997a} for a comparison of several FENE-type dumbbell models. Such systems are of microscopic-macroscopic type since they involve a coupling of the (macroscopic) Navier--Stokes equations for the description of incompressible fluid flow and the Fokker--Planck equation for the microscopic processes associated with 
the statistical properties of polymer molecules immersed in the fluid. 
Concerning the weak and strong well-posedness of FENE-type models, we refer to the works \cite{jourdain2004existence,kreml2010on,masmoudi2013global,zhang2006local,renardy1991an}. More general dilute polymer models are analyzed in \cite{barrett2005existence,barrett2007existence,barrett2008existence,barrett2010existence,barrett2010existence2}. Further, we mention the papers \cite{lions2000global,lions2007global,schonbek2009existence,masmoudi2008well,barrett2005existence,barrett2009numerical,debiec2023corotational,lin2008global,busuioc2014fene}, which, similarly to the discussion herein, are concerned with dumbbell models that assume a corotational drag term in the Fokker--Planck equation.  In such models it is supposed that polymer molecules are not stretched by the surrounding solvent, although they are allowed to rotate without stretching; see, for example, \cite{la2020diffusive}.


Time-fractional differential equations have been the focus of considerable attention in the mathematical and engineering literature in recent years. Such equations are nonlocal in time and have an innate history effect. They are of relevance in applications where memory effects are present and hereditary properties of materials are studied; see, for example, the textbooks on viscoelasticity \cite{mainardi2022fractional,yang2020general}, hydrology \cite{su2020fractional}, financial economics \cite{fallahgoul2016fractional}, and mechanical processes \cite{atanackovic2014fractional,pilipovic2014fractional}. The time-fractional Fokker--Planck system, in particular, allows subdiffusive behaviour and has been previously studied in \cite{metzler1999anomalous,metzler1999deriving,metzler2000random,barkai2000continuous,barkai2001fractional,henry2006anomalous,henry2010fractional,henry2010introduction,langlands2008anomalous} with regards to its derivation and applicability. The articles \cite{pinto2017numerical,le2016numerical,le2018a,le2019existence,le2021alpha} have investigated the numerical analysis and the simulation of solutions to the time-fractional Fokker--Planck equation. The time-fractional model considered herein has been explored computationally in \cite{beddrich2023numerical}, albeit in the simpler setting of a linear (Hookean) elastic spring force instead of the FENE spring model that we study here.
 
We employ a spatial Galerkin approximation in conjunction with a compactness argument to prove the existence of weak solutions to the time-fractional Navier--Stokes--Fokker--Planck system under consideration. More specifically, we discretize the system in space and derive appropriate energy bounds, which then enable us to pass to the limit in the discretized system.  Spatial discretizations of dilute polymer models were previously considered  in  \cite{barrett2009numerical,barrett2011finite,barrett2012finite}. In addition, weak solutions to time-fractional PDEs have been investigated using the Galerkin approach in the publications \cite{fritz2021sub,fritz2020time,fritz2021equivalence}. There have also been initial steps in the analysis of a decoupled time-fractional Fokker--Planck equation with time-dependent forces; see the papers \cite{fritz2023well,mclean2020regularity,le2019existence,le2021alpha,mclean2021uniform}. However, the coupling of the time-fractional Fokker--Planck equation to the Navier--Stokes system gives rise to new technical complications, which have not been addressed previously.


In \Cref{Sec:Derivation} we derive the model from the Langevin equation assuming power-law waiting time. In this way time-fractional derivatives in the sense of Riemann--Liouville appear in the associated  Fokker--Planck equation. By mimicking the technique for the derivation of the standard dumbbell model, a time-fractional Navier--Stokes--Fokker--Planck system is obtained.  In \Cref{Sec:Prelim} we introduce several function spaces of Sobolev-type and recall some important results from the theory of fractional derivatives, including chain inequalities and embedding theorems. In \Cref{Sec:Form} we transform the model in order to make it amenable to the subsequent analysis.  We then equip the model with suitable initial and boundary conditions and we make use of the associated Maxwellian to rescale the Navier--Stokes--Fokker--Planck system. In \Cref{Sec:Analysis} we finally state and prove a theorem asserting the existence of large-data global-in-time weak solutions to the model with a time-fractional derivative of order $\alpha \in (\tfrac12,1)$. 

