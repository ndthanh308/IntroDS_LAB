\documentclass[prl,reprint,superscriptaddress]{revtex4-2}
\pdfoutput=1
\usepackage{braket,dsfont,subfigure,amsfonts,amssymb,bm,graphicx,float,amsmath,hyperref,amsthm}
\usepackage{mathrsfs}
\newcommand{\Tr}{{\rm Tr}} 
\newcommand{\llvert}{\left|\hspace{-0.48mm}\left|}
\newcommand{\rrvert}{\right|\hspace{-0.48mm}\right|}
\newcommand{\rrangle}{\rangle\!\rangle}
\newcommand{\llangle}{\langle\!\langle}
\newcommand{\rrpipe}{|}
\newcommand{\llpipe}{|}
\newcommand{\sket}[1]{\ensuremath{\llpipe#1\rrangle}}
\begin{document}
\title{Scalable Quantum State Tomography with Locally Purified Density Operators and Local Measurements}
\author{Yuchen Guo}
\affiliation{State Key Laboratory of Low Dimensional Quantum Physics and Department of Physics, Tsinghua University, Beijing 100084, China}
\author{Shuo Yang}
\email{shuoyang@tsinghua.edu.cn}
\affiliation{State Key Laboratory of Low Dimensional Quantum Physics and Department of Physics, Tsinghua University, Beijing 100084, China}
\affiliation{Frontier Science Center for Quantum Information, Beijing 100084, China}
\affiliation{Hefei National Laboratory, Hefei 230088, China}
\begin{abstract}
    Understanding quantum systems holds significant importance for assessing the performance of quantum hardware and software, as well as exploring quantum control and quantum sensing.
    An efficient representation of quantum states enables realizing quantum state tomography with minimal measurements.
    In this study, we propose a new approach to state tomography that uses tensor network representations of mixed states through locally purified density operators and employs a classical optimization algorithm requiring only local measurements.
    Through numerical simulations of one-dimensional pure and mixed states and two-dimensional random tensor network states up to size $8\times 8$, we demonstrate the efficiency, accuracy, and robustness of our proposed methods.
    Experiments on the IBM Quantum platform complement these numerical simulations.
    Our study opens new avenues in quantum state tomography for two-dimensional systems using tensor network formalism.
\end{abstract}
\maketitle
\section{Introduction}
Quantum state tomography plays a fundamental role in characterizing and evaluating the quality of quantum states produced by quantum devices.  
It serves as a crucial element in the advancement of quantum hardware and software, regardless of the underlying physical implementation and potential applications~\cite{Nielsen2009, Preskill2018, Gebhart2023}.
However, reconstructing the full quantum state becomes prohibitively expensive for large-scale quantum systems that exhibit potential quantum advantages~\cite{Arute2019, Kim2023}, as the number of measurements required increases exponentially with system size.

Recent protocols try to solve this challenge through two main steps: efficient parameterization of quantum states and utilization of carefully designed classical optimization algorithms.
For one-dimensional systems with area law entanglement, matrix product state (MPS)~\cite{Verstraete2006A, Perez2007, Verstraete2008, Schollwoeck2011, Orus2014, Cirac2021} provides a compressed representation.
It requires only a polynomial number of parameters that can be determined from local or global measurement results.
Two iterative algorithms using local measurements, singular value thresholding (SVT)~\cite{Cramer2010} and maximum likelihood (ML)~\cite{Baumgratz2013A}, have been demonstrated in trapped-ion quantum simulators with up to $14$ qubits~\cite{Lanyon2017}.
However, SVT is limited to pure states and thus impractical for noisy intermediate-scale quantum (NISQ) systems. 
Meanwhile, although ML can handle mixed states represented as matrix product operators (MPOs)~\cite{Pirvu2010, Jarkovsky2020}, it suffers from inefficient classical optimization.
Another scheme reconstructs the quantum state by inverting local measurements, but the resulting MPO is not necessarily positive~\cite{Baumgratz2013B}.

On the other hand, some approaches are also based on tensor networks (TN) but update parameters from the sampling output of global measurements across the entire system~\cite{Han2018, Wang2020}, which is significantly more demanding than local measurements.
Moreover, these approaches cannot readily incorporate error mitigation techniques that focus mainly on estimators rather than samplers~\cite{Li2017, Temme2017, Endo2018, McArdle2019, Cai2021, Guo2022, Berg2023, Cai2023}, especially for readout errors~\cite{Berg2022, Yang2022}.
To scale to larger systems and integrate with error mitigation techniques, TN-based methods using only local measurements are preferred. 
However, TN state tomography remains unexplored for higher-dimensional quantum systems.

In this work, we introduce a new approach to reconstructing mixed quantum states using only local expectation values, for which readout error mitigation is feasible.
We propose representing mixed states as locally purified density operators (LPDOs)~\cite{Verstraete2004A} and optimizing their parameters using a variant of gradient descent for a local loss function.
To validate the effectiveness of our method, we perform numerical simulations using typical quantum states: 1D critical Ising ground states, 1D gapless spin-$\frac{1}{2}$ Heisenberg ground states subjected to various noise models, and two-dimensional (2D) random projected entangled pair states (PEPSs)~\cite{Verstraete2006B, Schuch2007, Perez2008}.
Furthermore, we implement experiments on real quantum devices accessible through the IBM Quantum platform.
 
\section{Quantum state tomography with LPDO and local loss function}
In this section, we introduce our new reconstruction approach for scalable tomography of mixed states. 
We begin by efficiently parameterizing mixed states using LPDOs, which have been shown efficient in simulating thermal or dissipative many-body systems in 1D~\cite{Werner2016, Cheng2021}.
LPDO variants have also been applied to quantum process tomography for small systems~\cite{Torlai2023}.

In the LPDO form, a 1D mixed quantum state is expressed as follows
\begin{align}
    \begin{aligned}
        \hat{\rho} = \sum_{\{\bm{\mu}, \bm{\nu}\}}\sum_{\{\bm{\kappa}\}}\prod_{j=1}^N[A_j]^{\tau_j, \kappa_{j}}_{\mu_{j-1}, \mu_{j}}[A_j^*]^{\omega_j, \kappa_{j}}_{\nu_{j-1}, \nu_{j}}\\
        \ket{\tau_1,\cdots,\tau_N}\hspace{-1mm}\bra{\omega_1,\cdots,\omega_N},
    \end{aligned}
\end{align}
where $\bm{\mu}$ and $\bm{\nu}$ denote virtual indices that describe quantum entanglement, and $\bm{\kappa}$ represents inner indices (also known as Kraus indices) introduced for open systems~\cite{Cheng2021}, as depicted in Fig.~\ref{Fig: Compare}(a).
Importantly, an LPDO is constructed to be Hermitian and positive semidefinite by design.
In scenarios with weak local noise, the Kraus dimension $d_{\kappa}$ is typically a small constant, independent of the system size $N$.
This enables a significantly more efficient tomography approach compared to directly reconstructing an MPO.
Note that the ML method proposed in Ref.~\cite{Baumgratz2013A} could be adapted and integrated into this framework by replacing MPO iterations with computationally less expensive LPDO operations.
However, we do not explore this further as the exponential computational cost of iteration (shown later) is not easily reduced.

Next, we propose a loss function constructed from local measurements and a variant of gradient descent to update the local LPDO tensors.
The loss function for our problem is chosen as
\begin{align}
    \Theta = \sum_{i}{||\hat{\sigma}_{\braket{i}} - \hat{\rho}_{\braket{i}}||_F^2} \equiv \sum_{i} \Theta_{\braket{i}},\label{equ: Cost}
\end{align}
where $\hat{\sigma}_{\braket{i}}$ and $\hat{\rho}_{\braket{i}}$ are the reduced density matrices for the sites $\{i, \cdots, i+L-1\}$ of the target state obtained from experiments (see Methods) and the reconstructed state respectively, as shown in Fig.~\ref{Fig: Compare}(a). 
Expanding each term in the loss function gives
\begin{align}
    \Theta_{\braket{i}} = \Tr{\left[\hat{\sigma}_{\braket{i}}^2 - 2\hat{\sigma}_{\braket{i}}\hat{\rho}_{\braket{i}}+\hat{\rho}_{\braket{i}}^2\right]}.
\end{align}
To calculate the gradients, terms such as
\begin{align}
    \frac{\partial \Theta_{\braket{i}}}{\partial A_j^*} = 2 \Tr{\left[\left(\hat{\rho}_{\braket{i}}-\hat{\sigma}_{\braket{i}}\right)\frac{\partial\hat{\rho}_{\braket{i}}}{\partial A_j^*}\right]},
\end{align}
need to be computed, requiring $O(N^2)$ computational complexity.

However, instead of optimizing the entire loss function directly, we update each local tensor $A_j$ by only considering the adjacent terms $\Theta_{\langle i \rangle}$ that involve the target site $j$ in the loss function.
Specifically, we update $A_j$ according to the following rule
\begin{align}
    A_{j} \rightarrow A_{j} - \eta \sum_{i=j-L+1}^{j}\frac{\partial \Theta_{\braket{i}}}{\partial A_{j}^{*}},\label{equ: Gradient}
\end{align}
where $\eta$ is the learning rate, automatically adjusted using the Adam optimizer~\cite{Kingma2017}.
Evaluating the gradient in each iteration step has a time complexity of $O(ND^3)$, where $D$ is the virtual bond dimension. 
This approach converges to a high-quality approximation of the target state in only $O(\log(N))$ iterative steps without encountering the issue of local minima or barren plateaus, as demonstrated below.
LPDO and the local loss function together constitute our Grad-LPDO method.
Importantly, for pure states, one can set $d_{\kappa}=1$ and optimize over the MPS manifold using the same gradient method, which is treated as a special version of Grad-LPDO.

\section{Numerical simulations for 1D systems}
% Figure environment removed
In this section, we present numerical demonstrations of our Grad-LPDO method for both pure and mixed states of special interest.
We begin by comparing the performance of the ML-MPS method and our proposed Grad-LPDO method for reconstructing the ground state of the 1D transverse field Ising model at the critical point $g=1$ under open boundary condition (OBC). 
The Hamiltonian is given by
\begin{align*}
    H = - \sum_{\braket{i, j}} Z_iZ_j+g\sum_i X_i.
\end{align*}

We first obtain the ground state approximated by an MPS with $D_0 = 16$ using the standard variational method~\cite{Crosswhite2008, Orus2014}.
This MPS serves as the target state for subsequent reconstruction.
We first assume that all measurements are ideal and directly construct the reduced density matrices $\hat{\sigma}_i$ from the target state.
Thus, any reconstruction errors are solely attributed to insufficient local measurements and inefficient classical optimization.
We perform both algorithms for different system sizes $N$, with $D = D_0 = 16$ and $L = 2$, where the initial states are chosen as the paramagnetic ground state for $g\rightarrow+\infty$.
The hyperparameters in the Adam optimizer are set as $\xi_1 = \xi_2 = 0.8$ and $\epsilon = 10^{-8}$.
Fig.~\ref{Fig: Compare}(b)-(c) shows the number of iterative steps required to achieve fidelity $f = 0.9$ for different system sizes, representing the convergence speed of the algorithms.

The inset of Fig.~\ref{Fig: Compare}(b) clearly shows that the time complexity of the ML-MPS method scales exponentially with system size for a given reconstruction accuracy.
In contrast, the results shown in Fig.~\ref{Fig: Compare}(c) are highly promising, indicating that the number of gradient steps required for convergence scales only as $O(\log{N})$.
In other words, the overall time complexity of our Grad-LPDO method to reconstruct a pure 1D state is $O(N\log(N)D^3)$.
This time complexity significantly outperforms ML-MPS (and SVT-MPS with time complexity $O(N^4D^4)$~\cite{Cramer2010}), meaning that our method saves considerable time compared to ML-MPS, especially for large systems.
Even for small systems, ML-MPS holds no efficiency advantage since truncating MPS in each iteration, as required in the ML method (see Methods), is generally more computationally expensive than simply contracting the environment in Grad-LPDO.

To provide a comprehensive comparison of these two methods, we calculate the maximal fidelity achieved during iterations for different $N$ in Fig.~\ref{Fig: Compare}(d), which demonstrates the higher accuracy of our method.
At the same time, we also consider convergence stability when comparing different methods. 
Typical iteration curves plotted in Fig.~\ref{Fig: Compare}(e)-(f) reveal that Grad-LPDO consistently converges to maximal fidelity, while the convergence of ML-MPS is less stable, regardless of the update or learning rate per step.
In conclusion, our Grad-LPDO method surpasses the previous ML-MPS method in three key aspects: efficiency, accuracy, and stability.

% Figure environment removed
We now examine the performance of our Grad-LPDO method on the ground state of the 1D gapless spin-$\frac{1}{2}$ Heisenberg model
\begin{align}
    H = \sum_{\braket{i, j}}\mathbf{S}_{i}\cdot\mathbf{S}_{j}
\end{align}
with $N=20$ and $D_0=16$.
The ZZ correlation function $C^{\rm{ZZ}}_{ij} = \braket{Z_iZ_j}-\braket{Z_i}\braket{Z_j}$ are calculated for both the target and the reconstructed states with $D=D_0=16$ and $L=2$ in Fig.~\ref{Fig: Mixed}(a).
Their high consistency indicates that Grad-LPDO can capture most of the long-range correlation and antiferromagnetic order from only local measurements.
Furthermore, we compare the performance of ML-MPS and Grad-LPDO in Fig.~\ref{Fig: Mixed}(b), where the differences in correlation between the target state and the reconstructed states are shown for both methods.
Unlike Grad-LPDO, ML-MPS cannot accurately reproduce the correlation between sites with intervals $L\geq 3$, hindering its application to systems with nontrivial orders.

Next, we consider Heisenberg ground states with $D_0=16$ that undergo different types of local noise.
Specifically, we add four types of single-qubit noise with an equal error rate $\varepsilon=0.01$ to each qubit of the ideal state, including depolarizing (DP), bit flipping (BF), amplitude damping (AD), and phase damping (PD).
Fig.~\ref{Fig: Mixed}(b) shows numerical results, where we reconstruct target states using LPDOs with $D=16$ and different $d_{\kappa}$, along with local measurements of length $L=2$.
The fidelity between two density matrices is defined as their inner product in operator space $f\left(\hat{\rho}_1, \hat{\rho}_2\right) \equiv \Tr{\left(\hat{\rho}_1\hat{\rho}_2\right)} / \sqrt{\Tr{\left(\hat{\rho}_1^2\right)}\Tr{\left(\hat{\rho}_2^2\right)}}$, which can be efficiently calculated for LPDOs (see Methods).

Introducing noise to the target state will generally reduce the reconstruction fidelity $f$, which cannot be improved by only increasing $d_{\kappa}$ as implied in Fig.~\ref{Fig: Mixed}(c).
This decrease arises because local measurements alone cannot distinguish between mixtures in the reduced density matrix from entanglement with unmeasured qubits (represented by virtual indices) and from quantum noise (represented by Kraus indices), as confirmed by our numerical simulations.
For example, reconstructing quantum states with depolarizing or bit-flipping noise poses greater challenges, since these noise models are stochastic in nature and directly introduce mixtures across different trajectories.
Furthermore, the purity of the target state $\mathcal{P} = \Tr[\hat{\rho}_0^2]$ is plotted in Fig.~\ref{Fig: Mixed}(d) for different noise types and in Fig.~\ref{Fig: Mixed}(e) for depolarizing noise across varying system sizes $N$.
They show a clear dependence of the reconstruction fidelity $f$ on $\mathcal{P}$, supporting our earlier arguments.

To overcome this challenge, we increase the measurement length $L$ in Fig.~\ref{Fig: Mixed}(d)-(e), significantly improving the accuracy for all types of noise.
With adjacent $4$-site measurements, the reconstruction fidelity $f$ exceeds 0.985 even for the most challenging depolarizing noise with up to $N=20$ qubits.
However, the experimental and numerical costs scale exponentially with $L$, requiring a trade-off between accuracy and efficiency.
Our results indicate that even for moderate-sized critical systems with typical noise levels, a small constant $L$ suffices for high-fidelity reconstruction at an acceptable cost using modern experimental and numerical techniques.

\section{Generalization to 2D systems}
% Figure environment removed
We now generalize our Grad-LPDO method to pure states of two-dimensional systems on square lattices, represented as PEPSs.
The key idea and procedure remain the same, where the local measurements in Eq.~\eqref{equ: Cost} are applied to $L_1\times L_2$ subsystems.
When updating the corresponding local tensor $A_j$, gradients in Eq.~\eqref{equ: Gradient} contain terms with local measurements covering the site $j$.

In general, the contraction of a 2D TN is computationally expensive.
Here, we adopt the standard truncation method for finite-size systems~\cite{verstraete2004B, Orus2014}.
The 2D TN with bond dimension $D$ is contracted layer by layer as the evolution of a 1D MPS, truncating the bond dimension to $\chi=D^2$ after each contraction.

We simulate random PEPS target states with fixed $D_0=3$ and varying system sizes $N\times N$, where random PEPSs with different $D$ are chosen as initial states for the iteration.
Local measurements are performed on all $1\times 2$ and $2\times 1$ subsystems.
Fig.~\ref{Fig: PEPS} shows the reconstruction fidelity of $100$ shots for each pair of $N$ and $D$, with error bars giving the standard deviation.
The average fidelity reaches $0.995$ for random $8\times8$ PEPSs with $D_0 = 3$, which are generally non-critical.

\section{Experiments on IBM Quantum platform}
% Figure environment removed
To demonstrate our method on real quantum hardware, we conduct experiments on the IBM Quantum platform.
Specifically, we use the `ibm\_nairobi' quantum computer, a superconducting processor with $N=7$ available qubits, whose qubit configuration and noise information are shown in Fig.~\ref{Fig: IBM}(a).
The input state is the trivial product state $\ket{\psi_0} = \ket{0}^{\otimes N}$, then random Haar circuits are implemented to generate the output states to be reconstructed.
$4^L$ ($L=2$ here) numbers of Pauli strings are measured to estimate each reduced density matrix $\hat{\sigma}_i$ for the target state, with each observable measured using 10,000 shots.
To mitigate errors in measurement results, twirled readout error extinction (T-REx)~\cite{Berg2022} is employed.
Other error mitigation techniques for circuit errors are not considered, as one of the main objectives of tomography is to learn about the noise information in the system.
Since the target state is unknown in practice, we use the loss function $\Theta$ defined in Eq.~\eqref{equ: Cost} to represent the residual error for tomography, which is highly related to the final reconstruction fidelity that cannot be directly evaluated in practical scenarios~\cite{Cramer2010}.

In Fig.~\ref{Fig: IBM}(b), the iteration process of one typical circuit realization (seed $57$ for the random circuit) is plotted for $d_{\kappa}=2$ and different $D$.
The experimentally measured expectation values $\Tr{[\hat{\sigma}_{\braket{i}}\hat{P}_{\braket{i}}]}$ and the corresponding estimated values from the reconstructed states $\Tr{[\hat{\rho}_{\braket{i}}\hat{P}_{\braket{i}}]}$ are calculated for each reduced density matrix with $d_{\kappa}=2$ and $D=4$, where $\hat{P}_{\braket{i}}$ refers to Pauli strings.
These results are visualized in Fig.~\ref{Fig: IBM}(d)-(i) to demonstrate their high consistency.
Fig.~\ref{Fig: IBM}(c) shows the residual error averaged over $64$ random circuit realizations along with the standard deviation.
For each circuit realization, we choose $100$ random LPDOs as initial states and record the average residual error.
These results confirm that LPDOs with small $d_{\kappa}$ and $D$ serve as good approximations for quantum states generated from noisy quantum circuits, validating the performance of our tomography scheme.

\section{Discussion}
In this study, we introduce a new state tomography scheme that uses the LPDO parameterization of mixed states and classical optimization to estimate unknown tensors from only local measurements. 
Our approach demonstrates enhanced efficiency, accuracy, and robustness compared to previous methods relying on MPS representations.
In particular, we extend TN state tomography, originally developed for 1D systems, to higher spatial dimensions.
Our optimization method demonstrates efficacy in alleviating the curse of dimensionality for high-dimensional systems.

The findings suggest that the recently proposed process tomography method~\cite{Torlai2023} and the error mitigation approach~\cite{Guo2022} may generalize to higher-dimensional circuits, enabling a deeper understanding of noise and its effects in such systems~\cite{Guo2023}.
Our protocol facilitates the realization of quantum state tomography for large systems with potential quantum advantages, promoting advancements in precise quantum control and complex algorithm implementation~\cite{Preskill2018, Endo2021}.

\section{Methods}
\subsection{Local measurements and reduced density matrices}
To obtain the reduced density matrix $\hat{\sigma}_{\braket{i}}$ for the sites $\{i, \cdots, i+L-1\}$ in experiments, one needs to implement an informationally complete set of measurements in this subsystem.
Specifically, we consider all possible products of Pauli operators $\hat{P}_{\braket{i}}^{\bm{m}}$ acting on these adjacent $L$ sites, where $\bm{m}=\{m_i, \cdots, m_{i+L-1}\}$ with $m_{i}\in\{I, X, Y, Z\}$.
Then $\hat{\sigma}_{\braket{i}}$ can be expanded as~\cite{Nielsen2009, Cramer2010}
\begin{align}
    \hat{\sigma}_{\braket{i}} = \frac{1}{2^L}\sum_{\bm{m}}{\Tr{\left[\hat{\sigma}_{\braket{i}}\hat{P}_{\braket{i}}^{\bm{m}}\right]}\hat{P}_{\braket{i}}^{\bm{m}}}
\end{align}
since $\hat{P}_{\braket{i}}^{\bm{m}}$ constitute a set of complete and orthogonal basis in the operator space.
To reconstruct each $\hat{\sigma}_{\braket{i}}$, the expectation values of all $4^L$ Pauli strings must be measured.

In our numerical simulations for 1D pure and mixed states and 2D PEPSs, we directly construct $\hat{\sigma}_{\braket{i}}$ from the target states.
In experiments on the IBM Quantum platform, we estimate $\hat{\sigma}_{\braket{i}}$ by applying measurements on the target states.

\subsection{SVT-MPS method and ML-MPS method}
We briefly review two QST methods for 1D systems based on MPS and local measurements~\cite{Cramer2010, Baumgratz2013A}.

The SVT-MPS method~\cite{Cramer2010} is inspired by the SVT algorithm in computer science for matrix completion~\cite{Cai2008}.
The target state can be approached iteratively by solving a local Hamiltonian at each step.
Specifically, in the $n$-th iterative step we construct and find the dominant eigenstate of the following local Hamiltonian 
\begin{align}
    \hat{Y}_{n+1} = \hat{Y}_n + \delta_n\left(\sum_i \hat{\sigma}_{\braket{i}}-E_n \sum_i \hat{\rho}_{n\braket{i}}\right).
\end{align}
Here, $\delta_n$ is the `update rate' for each step, $\hat{\sigma}_{\braket{i}}$ are reduced density matrices for adjacent $L$ sites of the target state obtained through direct measurements, while $\hat{\rho}_{n\braket{i}}$ are reduced density matrices of the dominant eigenstate of $\hat{Y}_n$ with eigenvalue $E_n$.
The authors have shown that the number of iterative steps to achieve fixed fidelity typically scales as $O(N^2)$, where $N$ is the number of qubits.
Additionally, in each step, a variational method is needed to solve a local Hamiltonian, involving $O(N)$ sweep back and forth and $O(N)$ calculations of the environment per sweep.
Updating local tensors for each site also requires calculating the dominant eigenvector of a $D^2d_p\times D^2d_p$ matrix, with complexity at least $O(D^4)$ assuming sparsity.
Therefore, the total computational cost of SVT-MPS is $O(N^4D^4)$, which is scalable but limited to pure states.
However, since real experiments encounter mixed states with decoherence, this reconstruction scheme has limited practicality for characterizing the noise effect in quantum devices.

The ML-MPS method directly searches for the target state that maximizes the log-likelihood function
\begin{align}
    \log{\mathcal{L}\left(\hat{\rho}\right)} = \sum_{i, j} n_{\braket{i}}^j \log{\left(\Tr\left[\hat{\Pi}_{\braket{i}}^j\hat{\rho}\right]\right)}
\end{align}
via a fixed-point iterative algorithm.
Here, $\hat{\Pi}_{\braket{i}}^j$ are local projectors labeled with $j$ applied at adjacent $L$ sites $\{i, \cdots, i+L-1\}$, and $n_{\braket{i}}^j$ are the corresponding measurement outcomes.
The solution $\hat{\rho}_{\textrm{ML}}$ that maximizes the above function satisfies
\begin{align}
    \hat{\rho}_{\textrm{ML}} = \frac{1}{M}\sum_{i,j}\frac{n_{\braket{i}}^j}{\Tr{\left[\hat{\Pi}_{\braket{i}}^j\hat{\rho}_{\text{ML}}\right]}}\hat{\Pi}_{\braket{i}}^j\hat{\rho}_{\text{ML}}\equiv \mathcal{R}(\hat{\rho}_{\text{ML}})\hat{\rho}_{\text{ML}},
\end{align}
which corresponds to the fixed-point equation
\begin{align}
    \hat{\rho} = \mathcal{R}(\hat{\rho})\hat{\rho}\mathcal{R}(\hat{\rho}).
    \label{equ: ML-Fixed}
\end{align}
In practice, one can replace $\mathcal{R}$ by $(\mathcal{I}+\epsilon\mathcal{R}) / (1+\epsilon)$ with $\epsilon \ll 1$.
Furthermore, under the assumption of pure state, we only need to iterate on the pure state manifold $\ket{\psi} = \mathcal{R}\ket{\psi}$.
To implement this, we construct the MPO representation of $\mathcal{R}$ in each iteration and truncate the resulting MPS $\mathcal{R}\ket{\psi}$.
Truncation can be done variationally by minimizing the error $e = (\bra{\psi}-\bra{\psi^{\prime}})(\ket{\psi}-\ket{\psi^{\prime}})$ or more efficiently by using SVD from site to site in canonical form, which requires $O(ND^{3})$ operations.

In the following table, we summarize the computational complexity and application scope of the previous two methods and our Grad-LPDO method.
\begin{table}[H]
    \centering
    \begin{tabular}{c|cc}\hline
        Method & Complexity & Application\\\hline
        SVT-MPS & $O(N^4D^4)$ & Pure states\\
        ML-MPS (MPO) & $O(\exp(N)ND^3)$ & Pure (Mixed) states \\
        Grad-LPDO & $O(\log(N)ND^3)$ & Mixed states\\\hline
    \end{tabular}
\end{table}

\subsection{Fidelity between two states}
The fidelity between two normalized pure states is defined as
\begin{align}
    f\left(\ket{\psi}, \ket{\phi}\right) = \left|\braket{\psi|\phi}\right|^2.
    \label{equ: fidelity}
\end{align}
This is usually generalized for two mixed states as
\begin{align}
    f\left(\hat{\rho}_1, \hat{\rho}_2\right) = \left(\Tr{\sqrt{\sqrt{\hat{\rho}_1}\hat{\rho}_2\sqrt{\hat{\rho}_1}}}\right)^2
\end{align}
with normalized $\Tr{\left[\hat{\rho}_1\right]}=\Tr{\left[\hat{\rho}_2\right]}=1$.
However, this definition cannot be directly estimated for two mixed states in their LPDO form, which prevents direct benchmarking of our tomography method for mixed states in large systems.
Therefore, we adopt an alternative definition
\begin{align}
    f\left(\hat{\rho}_1, \hat{\rho}_2\right) \equiv \Tr{\left(\hat{\rho}_1\hat{\rho}_2\right)} / \sqrt{\Tr{\left(\hat{\rho}_1^2\right)}\Tr{\left(\hat{\rho}_2^2\right)}},
\end{align}
which is the inner product in the operator space and equals the overlap between two superoperators $\sket{\rho_1}$ and $\sket{\rho_2}$.
In particular, this alternative definition of fidelity reduces to Eq.~\eqref{equ: fidelity} for pure states.

\subsection{Noise models}
In our numerical simulations for mixed states, the noise added after each state includes four types.
The depolarizing noise is defined as
\begin{align}
    \mathcal{E}\left(\hat{\rho}\right) = \left(1-\frac{4}{3}\varepsilon\right)\hat{\rho} + \frac{1}{3}\varepsilon\sum_{i=0}^{3}\sigma_i\hat{\rho}\sigma_i.
\end{align}
The bit flipping noise is defined as
\begin{align}
    \mathcal{E}\left(\hat{\rho}\right) = \left(1-\varepsilon\right)\hat{\rho} + \varepsilon\sigma_x\hat{\rho}\sigma_x.
\end{align}
The amplitude damping noise is defined by the Kraus operator $E_0 = \ket{0}\hspace{-1mm}\bra{0} + \sqrt{1-\varepsilon}\ket{1}\hspace{-1mm}\bra{1}$ and $E_1 = \sqrt{\varepsilon}\ket{0}\hspace{-1mm}\bra{1}$ with the operator-sum representation
\begin{align}
    \mathcal{E}\left(\hat{\rho}\right) = E_0\hat{\rho} E_0^{\dagger} + E_1\hat{\rho} E_1^{\dagger}.
\end{align}
Phase damping noise is defined similarly, with the Kraus operator $E_0 = \ket{0}\hspace{-1mm}\bra{0} + \sqrt{1-\varepsilon}\ket{1}\hspace{-1mm}\bra{1}$ and $E_1 = \sqrt{\varepsilon}\ket{1}\hspace{-1mm}\bra{1}$.

\section{Data availability}
The datasets generated and analyzed during the current study are available from the corresponding author upon reasonable request.

\bibliography{ref}

\section{Acknowledgements}
This work is supported by the National Natural Science Foundation of China (NSFC) (Grant No. 12174214 and No. 92065205), the National Key R\&D Program of China (Grant No. 2018YFA0306504), the Innovation Program for Quantum Science and Technology (Grant No. 2021ZD0302100), and the Tsinghua University Initiative Scientific Research Program.
The IBM Quantum device `ibm\_nairobi' is accessible at https://quantum-computing.ibm.com.

\section{Author contributions}
Y.G. conceived, designed, and performed the experiments.
Y.G. and S.Y. analyzed the data and wrote the paper.
S.Y. contributed analysis tools and supervised the project.

\section{Competing interests}
The authors declare no competing interests.

\end{document}