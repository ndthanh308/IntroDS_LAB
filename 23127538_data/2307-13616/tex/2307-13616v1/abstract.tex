The development of Machine Learning is experiencing growing interest from the general public, and in recent years there have been numerous press articles questioning its objectivity: racism, sexism, \dots  Driven by the growing attention of regulators on the ethical use of data in insurance, the actuarial community must rethink pricing and risk selection practices for fairer insurance. Equity is a philosophy concept that has many different definitions in every jurisdiction that influence each other without currently reaching consensus. In Europe, the Charter of Fundamental Rights defines guidelines on discrimination, and the use of sensitive personal data in algorithms is regulated. If the simple removal of the protected variables prevents any so-called `direct' discrimination, models are still able to `indirectly' discriminate between individuals thanks to latent interactions between variables, which bring better performance (and therefore a better quantification of risk, segmentation of prices, and so on). After introducing the key concepts related to discrimination, we illustrate the complexity of quantifying them. We then propose an innovative method, not yet met in the literature, to reduce the risks of indirect discrimination thanks to mathematical concepts of linear algebra. This technique is illustrated in a concrete case of risk selection in life insurance, demonstrating its simplicity of use and its promising performance.