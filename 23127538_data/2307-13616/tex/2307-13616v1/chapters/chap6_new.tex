\section*{\centering Conclusion}
\addcontentsline{toc}{section}{Conclusion}
\vspace{4cm}
\paragraph{} The goal of this thesis was to mitigate indirect discrimination when using Machine Learning models for insurance mortality data. We have provided the reader with an innovative method based on mathematical concepts of linear algebra to achieve this goal. Looking first at a simulated case allowed us to grasp the intricate effects of discrimination mechanisms, and gave us insights as to how to tackle the real use case. \par
We have made a few approximations, as reminded throughout the thesis. Firstly, we have chosen to focus on only one definition of fairness, statistical parity. Although it is the most popular in research articles, we are not certain that it will be the one chosen by regulators and policy makers in their control for fairness. Secondly, statistical parity is an independence condition, and we have approximated dependence with its first-order component, correlation. Our method is therefore a first step towards fairness, but might not be sufficient if variables have intricate and complex relationships with each other. Although, as we have mentioned, if regulators decide to use such a metric to control fairness, they will need to take into account acceptability threshold around the goal value of the metric, because of its statistical nature. Lastly, with our transformation of the non-sensitive variables, we have interpretability issues as the new vectors are a combination of the old ones. Explaining to an insured or even to other less technical employees of the insurance company, why premiums vary may be tricky as they are based on the mortality rates that come from the use of the transformed variables. \par
Further developments around this subject should thus focus on expanding the method to all forms of dependence instead of correlation only, and even looking at ways of adapting the method to other fairness definitions. Other improvements may include adaptation to regression problems, which are also encountered by insurers. \par

% La conclusion est aussi importante que l’introduction. Elle donne la dernière impression au lecteur du mémoire, l'image finale qui influencera "fortement" le jury dans son évaluation. En aucun cas, elle ne devra laisser le lecteur sur une impression d’inachevé !
% La conclusion doit être synthétique et elle comprend, en générale, les éléments suivants :
% - Un rappel de la problématique actuarielle ;
% - Les principaux résultats des travaux ;
% - Les limites des travaux et de la méthodologie utilisée ;
% - Les voies futures pour ouvrir le débat sur une question plus large.
% La conclusion correspond à la vraie dernière partie du mémoire et elle doit être la synthèse des apports du travail réalisé, et proposer, le cas échéant des pistes à explorer pour un prochain travail.
% Elle doit clairement indiquer ce que le travail a amené à la « science actuarielle », que les résultats soient positifs et intéressants ou qu’il n’y ait pas eu de résultats probants. La conclusion doit donner la vision de l’étudiant sur sa contribution à la problématique actuarielle.
% La conclusion devrait représenter 2 à 3 pages.

% ouverture : on a choisi une métrique mais qui dit que c'est la bonne
% on s'est placé dans un cadre de classif mais régression ?, on a approximé dépendance avec corrélation
% question encore ouverte, mouvement mondial
% interprétabilité vecteurs transformés?
