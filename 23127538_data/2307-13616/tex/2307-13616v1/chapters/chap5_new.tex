
\section{Discrimination mitigation applied to real mortality data}
\subsection{Adapting the logistic regression to survival data}

\paragraph{} If we considered a simple logistic regression to predict the event of death $\delta_{i,j}\sim \mathcal{B}(q_j)$, we would not take into account censoring. Instead, we assign weights to the sample with the individual initial exposure. This way, we estimate the mortality rate $q_j=\mathbb{P}(T\leq\tau_j+1|T>\tau_j)$ with $\hat{q}_j=\frac{\sum_i \delta_{i,j}}{\sum_ie_{i,j}}$0.
\begin{proof}
The weighted likelihood of the model is
\begin{equation*}
    L=\prod_{i=1}^{l_j} q_j^{\delta_{i,j}e_{i,j}}(1-q_j)^{(1-\delta_{i,j})e_{i,j}}
\end{equation*}
The log likelihood to maximize is
\begin{equation*}
    log(L)=\sum_{i=1}^{l_j} \Bigl( \delta_{i,j}e_{i,j}log(q_j)+(1-\delta_{i,j})e_{i,j}log(1-q_j) \Bigr)
\end{equation*}
Deriving with respect to $q_j$, we obtain
\begin{equation*}
    \frac{dlog(L)}{dq_j} = \sum_{i=1}^{l_j} \Bigl( \frac{\delta_{i,j}e_{i,j}}{q_j} - \frac{e_{i,j}-\delta_{i,j}e_{i,j}}{1-q_j} \Bigr)
\end{equation*}
Equalizing with 0, we get
\begin{equation*}
    \begin{split}
         & (1-\hat{q}_j)\sum_{i=1}^{l_j} \delta_{i,j}e_{i,j}=\hat{q}_j\sum_{i=1}^{l_j} \bigl( e_{i,j}-\delta_{i,j}e_{i,j} \bigr) \\
        \mbox{ie } & \hat{q}_j=\frac{\sum_i \delta_{i,j}e_{i,j}}{\sum_i e_{i,j}}=\frac{\sum_i \delta_{i,j}}{\sum_ie_{i,j}} \mbox{ as } \delta=1 \Rightarrow e=1
    \end{split}
\end{equation*}
\end{proof}

% Figure environment removed

\subsection{Variable selection}

\paragraph{} In order to select the variables that are relevant to the model performance, we begin with a model with all variables. Table~\ref{tab:coeff complete} gives the coefficients of the regression, with their standard errors and p-values. We will only keep the variables with a p-value below or equal to 0.005, as it is good practice in modeling. The p-values that are above 0.005 are highlighted in orange. Variables that are not statistically significant are: \begin{itemize}
    \item \texttt{AGE}
    \item \texttt{reg\_nod\_ex}
    \item \texttt{Tot\_num\_benin}
    \item \texttt{AYA\_site\_rec20}
    \item \texttt{Surg\_LN}
    \item \texttt{Surg\_oth}
\end{itemize}
This does not necessarily mean that these variables are not medically relevant. \par

\begin{longtable}[c]{|l|c|c|c|}
    \hline
    Variable & Coefficient & Standard error & P-value \\ \hline
    \endfirsthead
    %
    \endhead
    %
    const & 174.9219 & 13.260 & 0.000 \\ \hline
    DURATION & -0.0191 & 0.003 & 0.000 \\ \hline
    Yr\_dx & -0.0358 & 0.004 & 0.000 \\ \hline
    Age\_dx & 0.0207 & 0.002 & 0.000 \\ \hline
    YEAR & -0.0549 & 0.003 & 0.000 \\ \hline
    AGE & 0.0016 & 0.002 & \cellcolor[HTML]{FFCC67}0.396 \\ \hline
    Sex\_Female & -0.3927 & 0.043 & 0.000 \\ \hline
    reg\_nod\_pos & 0.0289 & 0.007 & 0.000 \\ \hline
    reg\_nod\_ex & 0.0003 & 0.002 & \cellcolor[HTML]{FFCC67}0.849 \\ \hline
    Mitotic rate & 0.0528 & 0.007 & 0.000 \\ \hline
    Tot\_num\_in\_situ & -0.0742 & 0.023 & 0.002 \\ \hline
    Tot\_num\_benin & 0.0777 & 0.170 & \cellcolor[HTML]{FFCC67}0.648 \\ \hline
    Laterality\_Bilateral & 0.6875 & 0.331 & \cellcolor[HTML]{FFCC67}0.038 \\ \hline
    Laterality\_Not a paired site & 0.4738 & 0.061 & 0.000 \\ \hline
    Laterality\_One side & 0.0624 & 0.200 & \cellcolor[HTML]{FFCC67}0.756 \\ \hline
    Laterality\_Paired site & 0.2974 & 0.093 & 0.001 \\ \hline
    Laterality\_Paired site midline tumor & 0.1334 & 0.128 & \cellcolor[HTML]{FFCC67}0.296 \\ \hline
    Laterality\_Right origin & -0.0520 & 0.044 & \cellcolor[HTML]{FFCC67}0.236 \\ \hline
    Ulceration\_Missing & 0.1514 & 0.099 & \cellcolor[HTML]{FFCC67}0.128 \\ \hline
    Ulceration\_Ulceration & 0.5980 & 0.048 & 0.000 \\ \hline
    Site\_rec\_WHO08\_Male Genital Organs & 0.0514 & 0.565 & \cellcolor[HTML]{FFCC67}0.928 \\ \hline
    Site\_rec\_WHO08\_Vulva & 0.5930 & 0.183 & 0.001 \\ \hline
    Origin\_American Indian/AK Native & 0.3195 & 0.293 & \cellcolor[HTML]{FFCC67}0.276 \\ \hline
    Origin\_Asian or Pacific Islander & 0.1691 & 0.156 & \cellcolor[HTML]{FFCC67}0.280 \\ \hline
    Origin\_Black & 0.4672 & 0.167 & 0.005 \\ \hline
    Origin\_Hispanic & 0.2042 & 0.108 & \cellcolor[HTML]{FFCC67}0.060 \\ \hline
    Origin\_Missing & -3.0099 & 1.001 & 0.003 \\ \hline
    AYA\_site\_rec20\_Nodular melanoma & 0.1249 & 0.052 & \cellcolor[HTML]{FFCC67}0.016 \\ \hline
    \begin{tabular}[c]{@{}l@{}}AYA\_site\_rec20\_Superficial spreading\\ /low cumulative sun damage melanoma\end{tabular} & -0.0319 & 0.048 & \cellcolor[HTML]{FFCC67}0.510 \\ \hline
    Mar\_stat\_Divorced & 0.3849 & 0.064 & 0.000 \\ \hline
    Mar\_stat\_Missing & -0.4411 & 0.072 & 0.000 \\ \hline
    Mar\_stat\_Separated & 0.3068 & 0.225 & \cellcolor[HTML]{FFCC67}0.173 \\ \hline
    Mar\_stat\_Single & 0.2578 & 0.055 & 0.000 \\ \hline
    Mar\_stat\_Unmarried & 0.3127 & 0.426 & \cellcolor[HTML]{FFCC67}0.463 \\ \hline
    Mar\_stat\_Widowed & 0.3041 & 0.087 & 0.000 \\ \hline
    Extent\_Distant & 0.7494 & 0.078 & 0.000 \\ \hline
    Extent\_Missing & 0.6607 & 0.171 & 0.000 \\ \hline
    Extent\_Regional & 0.4792 & 0.057 & 0.000 \\ \hline
    Surg\_LN\_Missing & 0.0226 & 0.347 & \cellcolor[HTML]{FFCC67}0.948 \\ \hline
    Surg\_LN\_Surgery & -0.1323 & 0.057 & \cellcolor[HTML]{FFCC67}0.020 \\ \hline
    Surg\_primsite\_Missing & 0.7475 & 0.335 & \cellcolor[HTML]{FFCC67}0.025 \\ \hline
    Surg\_primsite\_No surgery & 0.5662 & 0.136 & 0.000 \\ \hline
    Surg\_oth\_Missing & -0.6858 & 0.474 & \cellcolor[HTML]{FFCC67}0.148 \\ \hline
    Surg\_oth\_Surgery & 0.2390 & 0.095 & \cellcolor[HTML]{FFCC67}0.012 \\ \hline
    Tumor\_Missing & 1.0042 & 0.173 & 0.000 \\ \hline
    Tumor\_T0 & 0.4037 & 0.203 & \cellcolor[HTML]{FFCC67}0.047 \\ \hline
    Tumor\_T2 & 0.9187 & 0.076 & 0.000 \\ \hline
    Tumor\_T3 & 1.0184 & 0.094 & 0.000 \\ \hline
    Tumor\_T4 & 1.2887 & 0.097 & 0.000 \\ \hline
    Positive\_Node\_Missing & 0.9338 & 0.214 & 0.000 \\ \hline
    Positive\_Node\_N1 & -0.1761 & 0.588 & \cellcolor[HTML]{FFCC67}0.764 \\ \hline
    Positive\_Node\_N2 & 0.0468 & 0.590 & \cellcolor[HTML]{FFCC67}0.937 \\ \hline
    Positive\_Node\_N3 & 0.4810 & 0.594 & \cellcolor[HTML]{FFCC67}0.418 \\ \hline
    Stage\_II & 0.5843 & 0.083 & 0.000 \\ \hline
    Stage\_III & 1.5105 & 0.590 & \cellcolor[HTML]{FFCC67}0.011 \\ \hline
    Stage\_Missing & -0.3659 & 0.207 & \cellcolor[HTML]{FFCC67}0.077 \\ \hline
    \captionsetup{justification=centering}
    \caption{Coefficients of the complete model, \\p-values in orange when above the threshold of 0.005}
    \label{tab:coeff complete}\\
\end{longtable}

For our final model, the selected variables need to be statistically relevant, consistent with medical literature and available at the underwriting stage. We have discussed with the oncologist of the team, drawing the following conclusions: \begin{itemize}
    \item For time variables, the most important ones are the current age and duration. As we had strong correlations between age and age at diagnosis, we will only keep age. The year does not influence mortality, as we only have data from recent years.
    \item Laterality does not influence mortality, but it statistically relevant so we will keep the variable in the model.
    \item Fairer complexions are at a higher risk of developing the cancer, but once the cancer has developed, it does not influence mortality, so all things equal otherwise, we should obtain the same mortality rates for all Origins. Unfortunately, and especially in the US, Origin is very often a proxy for social economic status, which influences access to healthcare for example, which in turn has consequences on mortality risks.
    \item Stage is a combination of the TNM staging variables, so there are strong correlations between them, and there is no use in keeping them all. So we will not keep the \texttt{Stage} variable.
\end{itemize}
To conclude, we will keep the \texttt{AGE} and \texttt{Laterality} variables. We note that we are left with the three sensitive variables, \texttt{Sex}, \texttt{Origin} and \texttt{Mar\_stat}. \par

% Figure environment removed

\subsection{Regression model with no pre-processing step}

\paragraph{} As we have done in the simulated case, we begin by applying the regression model without any particular modifications. This will allow a comparison with other methods. As we are left with many variables, the coefficients of the regression are given in appendix~\ref{appendix:tab:coeff selected}. \par
Figure~\ref{fig:seer coeff sel} gives a visual understanding of the coefficients. We have an intercept and only kept the least represented classes for the categorical variables, so the coefficients for the remaining categorical variables represent the relative risk compared to the most represented class. For example, for the variable \texttt{Sex}, as we had more observations for Male, we only kept the variable \texttt{Sex\_Female} in our model. The coefficient attributed to this variable by the logistic regression indicates how much more chances of dying a Female has, compared to a Male. For numerical variables, the coefficients indicate how much the chances of dying change for an increase of one in the variable value. We have also included the standard errors, represented by the black lines.
\begin{itemize}
    \item \texttt{DURATION}: the higher the duration, the lower the probability of dying.
    \item \texttt{AGE}: the older the individual, the higher his probability of dying.
    \item \texttt{Sex}: compared to being a Male, being a Female decreases the probability of dying.
    \item \texttt{reg\_nod\_pos}: the higher the number of positive regional nodes, the higher the probability of dying.
    \item \texttt{Mitotic rate}: the higher the density of mitoses, the higher the probability of dying.
    \item \texttt{Laterality}: compared to the Left origin category, a Right origin lowers the probability of death and other categories increases it.
    \item \texttt{Ulceration}: compared to No ulceration, other categories increases the death probability, Ulceration more than Missing.
    \item \texttt{Site\_rec\_WHO08}: compared to tumors originating on the Pleura, tumors originated on the vulva and on male genital organs increase the probability of dying.
    \item \texttt{Origin}: compared to White, Missing largely decreases the probability of dying. Other categories increase it, in the increasing order we have Asian or Pacific Islander, Hispanic, American Indian/AK Native and Black. 
    \item \texttt{Marital status}: compared to Married, Missing decreases the probability of death. Other categories increase it, in the increasing order we have Unmarried, Single, Separated, Widowed and Divorced. We must take into account the standard errors, which are larger than the coefficients for Unmarried and Separated.
    \item \texttt{Extent}: compared to Localized, all categories increase the probability of dying, in the increasing order we have Missing, Regional and Distant.
    \item \texttt{Surg\_primsite}: compared to Surgery, other categories increase the probability of dying, No surgery more than Missing.
    \item \texttt{Tumor}: compared to T1, larger tumors increase the probability of dying in size order. Missing increases the same way as T2. But we have an anomaly for T0 as it seems it increases the probability of dying compared to T1. This might be due to the small sample size, as we saw when we were looking at the five-year mortality rates.
    \item \texttt{Positive\_Node}: compared to N0, larger numbers of positives nodes increase the probability of dying. Missing increases it, but not at much as the larger sizes.
\end{itemize}
The coefficients are coherent with medical findings: a higher number of positive regional nodes, the presence of ulceration, a further extent, and a larger tumor all point towards a severe cancer, which implies lower survival odds.

% Figure environment removed

\paragraph{Performance evaluation} Figure~\ref{fig:ROC selected var} gives the ROC curve. The performance of the model is almost the same as with all variables, with an AUC of 0.8769. \par

% Figure environment removed

In order to compute fairness metrics, we need to classify individuals into two categories $\hat{Y}=0$ or $1$, $\hat{Y}$ being the event of death due to skin melanoma, 1 if the event occurred ie the individual died of this cause and 0 otherwise. As the probabilities of death are all very close to 0, we cannot use the standard threshold of 0.5 to classify individuals. We will select the threshold by looking at the true percentage of dead individuals. In the entire dataset, we have about 0.7\% of dead individuals, as we saw in the section on descriptive statistics. We will therefore look for a threshold $\tau$ such that
\begin{equation*}
    \hat{Y}=1 \mbox{ if } \mathbb{P}(\hat{Y}=1)\geq \tau \mbox{ and } 0 \mbox{ otherwise, giving 0.7\% of individuals with }\hat{Y}=1
\end{equation*}
We find $\tau=0.1088$0. This value gives the confusion matrix as shown in figure~\ref{tab:seer confusion matrix sel variables}. We have, as planned, about 0.7\% individuals classified with $\hat{Y}=1$0. The class $\hat{Y}=0$ is the positive class here. \par

\begin{table}[H]
    \begin{center}
        \begin{tabular}{ccccc}
            \cline{2-3}
            \multicolumn{1}{c|}{}  & \multicolumn{1}{c|}{$\hat{Y}$=0} & \multicolumn{1}{c|}{$\hat{Y}$=1} &  &  \\ \cline{1-3}
            \multicolumn{1}{|c|}{$Y=0$} & \multicolumn{1}{c|}{106,236}  & \multicolumn{1}{c|}{699}  &  &  \\ \cline{1-3}
            \multicolumn{1}{|c|}{$Y=1$} & \multicolumn{1}{c|}{662}  & \multicolumn{1}{c|}{95}  &  &  \\ \cline{1-3}
        \end{tabular}
    \end{center}
    \caption{Confusion matrix (model with selected variables)}
    \label{tab:seer confusion matrix sel variables}
\end{table}

\paragraph{Fairness evaluation} We computed the acceptance rate and the true and false positive rates, and looked at them globally, by sex, by origin and by marital status. To conduct an extensive analysis, we should also look at subgroups, crossing information about the three sensitive attributes, like we did in the simulated case, but it becomes intricate as we have many different categories, and we risk having too few observations in one of the subcategories. \par
\begin{itemize}
    \item Globally, we have values close to 100\% for the acceptance and true positive rates, due to the numerous classifications into the positive class compared to the negative class. The false positive rate is quite high, as we have 6 times more false positives than true negatives.
    
\begin{table}[H]
    \centering
    \begin{tabular}{c|c}
        (\%) & Global \\ \hline
        AR & 99.26 \\
        TPR & 99.35 \\
        FPR & 87.45
    \end{tabular}
    \caption{Global fairness metrics (model with selected variables)}
    \label{tab:seer metrics sel variables global}
\end{table}

    \item By sex \begin{itemize}
        \item The acceptance rates are very close, only 0.76 points apart, but higher for Female.
        \item The true positive rates are very close for both genders, only 0.69 points apart, but higher for Female.
        \item The false positive rates are 3.87 points part, and higher for Female.
    \end{itemize}
    All in all, looking at the three definitions of fairness, the Male sex is always disadvantaged by this model, although the gaps between metrics are not large.

\begin{table}[H]
    \centering
    \begin{tabular}{c|cc|c}
        (\%) & Female & Male & Difference \\ \hline
        AR & 99.67 & 98.91 & 0.76 \\
        TPR & 99.71 & 99.03 & 0.69\\
        FPR & 90.18 & 86.30 & 3.87
    \end{tabular}
    \caption{Fairness metrics by sex (model with selected variables)}
    \label{tab:seer metrics sel variables sex}
\end{table}

    \item By origin \begin{itemize}
        \item The acceptance rates are between 92.53\% and 100\%. The lowest is for Black and the highest is for Missing.
        \item The true positive rates are between 93.21\% and 100\%. The lowest is for Black and the highest for Missing.
        \item The false positive rates are very distant. The lowest FPR is for Missing at 0\%, then 89.20\% for White. The highest is for American Indian/Alaska Native at 100.00\%.
    \end{itemize}
    It is interesting to note that the Missing category only has true positive classifications, which explains the values for the fairness metrics. Going back on the regression weights of figure~\ref{fig:seer coeff sel}, the \texttt{Origin\_Missing} variable has a colossal negative coefficient compared to other dummy variables, and especially compared to other variables for \texttt{Origin}. Whenever the origin of an individual was not collected, his probability of dying within the year is null. We can wonder about the data collection process: perhaps individuals with zero to no chance of dying were not as `interesting' for the database. \\
    Black is the most disadvantaged group under the statistical parity and equal opportunity definitions. For the equalized odds definition, Black has the lowest TPR but White has the lowest FPR, so we cannot conclude on the most disadvantaged group.
\begin{table}[H]
    \centering
    \begin{tabular}{c|ccccccc}
        (\%)& White & Hispanic & Black & \begin{tabular}[c]{@{}c@{}}Asian or\\ Pacific Islander\end{tabular} & \begin{tabular}[c]{@{}c@{}}American Indian\\ /AK Native\end{tabular} & Missing & \multicolumn{1}{|c}{Var} \\ \hline
        AR & 99.33 & 97.70 & 92.53 & 96.63 & 98.72 & 100.00 & \multicolumn{1}{|c}{$6.10\mathrm{e}{-4}$} \\
        TPR & 99.40 & 97.94 & 93.21& 97.53 & 100.00 & 100.00 & \multicolumn{1}{|c}{$5.52\mathrm{e}{-4}$} \\
        FPR & 89.20 & 97.94 & 93.21 & 97.53 & 100.00 & 0.00 & \multicolumn{1}{|c}{$8.35\mathrm{e}{-2}$}
    \end{tabular}
    \caption{Fairness metrics by origin (model with selected variables)}
    \label{tab:seer metrics sel variables race}
\end{table}

    \item By marital status \begin{itemize}
        \item The acceptance rate is lowest for Separated and highest for Missing, then Married.
        \item The true positive rates are between 97.54\% and 99.99\%, lowest for Widowed and highest for Missing, then Married.
        \item The false positive rate is lowest for Widowed, at 78.79\% and highest for Separated and Unmarried, at 100\%.
    \end{itemize}
    The Separated class is the most disadvantaged under the statistical parity definition, Widowed under the equal opportunity and equalized odds definitions.
\end{itemize}

\begin{table}[H]
    \centering
    \begin{tabular}{c|cccccccc}
         (\%) & Married & Missing & Single & Widowed & Divorced & Separated & Unmarried & \multicolumn{1}{|c}{Var} \\ \hline
        AR & 99.23 & 99.99 & 98.82 & 97.29 & 97.47 & 95.75 & 98.44 & \multicolumn{1}{|c}{$1.72\mathrm{e}{-4}$} \\
        TPR & 99.33 & 99.99 & 98.95 & 97.54 & 97.69 & 95.71 & 98.43 & \multicolumn{1}{|c}{$1.71\mathrm{e}{-4}$} \\
        FPR & 88.16 & 97.10 & 84.30 & 78.79 & 82.19 & 100.00 & 100.00 & \multicolumn{1}{|c}{$6.76\mathrm{e}{-3}$}
    \end{tabular}
    \caption{Fairness metrics by marital status (model with selected variables)}
    \label{tab:seer metrics sel variables mar stat}
\end{table}

To conclude, the most disadvantaged group is not the same depending on the definition of fairness that is used.

% Figure environment removed

\subsection{Removing protected variables to avoid direct discrimination}

\paragraph{} We will now compare the model with the protected variables to the model without protected variables. As we saw in the simulated case, removing the protected variables is not the perfect solution as it does not prevent discrimination. Simply ignoring the sensitive attributes gives larger weights to variables that are correlated with them. \par
Here, we will apply a regression model to all variables except the sensitive ones. The coefficients are given in appendix~\ref{appendix:tab:coeff without A}. \par
Figure~\ref{fig:seer coeff without A} shows the importance of each variable. The main difference that we can observe, compared to the model with all variables, is that the variable \texttt{Site\_rec\_WHO08\_Male Genital Organs} now has a 
negative coefficient, with a consequent standard error. Other than that, the weights stayed in the same order, with some amplitude variations.

% Figure environment removed

\paragraph{Performance evaluation} Figure~\ref{fig:ROC without A} gives the ROC curve. The model with all selected variables had an AUC of 0.8769, and without protected variables the model has an AUC of 0.8778, which is slightly higher. \par

% Figure environment removed

\paragraph{Fairness evaluation}
\begin{itemize}
    \item Globally \begin{itemize}
        \item[] The acceptance rate, true positive rate and false positive rate have slightly decreased compared to the model using all variables.
    \end{itemize}

\begin{table}[H]
    \centering
    \begin{tabular}{ccccc}
        \multicolumn{2}{c}{With all variables} &  & \multicolumn{2}{c}{Without protected variables} \\
        \multicolumn{1}{c|}{(\%)} & Global &  & \multicolumn{1}{c|}{(\%)} & Global \\ \cline{1-2} \cline{4-5} 
        \multicolumn{1}{c|}{AR} & 99.26 &  & \multicolumn{1}{c|}{AR} & 99.20 \\
        \multicolumn{1}{c|}{TPR} & 99.35&  & \multicolumn{1}{c|}{TPR} & 99.29 \\
        \multicolumn{1}{c|}{FPR} & 87.45 &  & \multicolumn{1}{c|}{FPR} & 86.26
    \end{tabular}
    \caption{Global fairness metrics}
    \label{tab:seer metrics without A global}
\end{table}

    \item By sex \begin{itemize}
        \item The acceptance rates difference has decreased, and Female kept the higher value.
        \item The true positive rates for Female and Male are closer than in the model with all variables. The Female category still has a higher TPR.
        \item The false positive rates are slightly closer than previously. The Female group still has a higher FPR.
    \end{itemize}
    Male is still disadvantaged under all three fairness definitions, although the gaps in metrics have narrowed.

\begin{table}[H]
    \centering
    \begin{tabular}{ccccccccc}
        \multicolumn{4}{c}{With all variables} &  & \multicolumn{4}{c}{Without protected variables} \\
        \multicolumn{1}{c|}{(\%)} & Female & Male & \multicolumn{1}{|c}{Difference} &  & \multicolumn{1}{c|}{Sex} & Female & Male & \multicolumn{1}{|c}{Difference} \\ \cline{1-4} \cline{6-9} 
        \multicolumn{1}{c|}{AR} & 99.67 & 98.91 & \multicolumn{1}{|c}{0.76} &  & \multicolumn{1}{c|}{AR} & 99.46 & 98.97 & \multicolumn{1}{|c}{0.49} \\
        \multicolumn{1}{c|}{TPR} & 99.71 & 99.03 & \multicolumn{1}{|c}{0.69} &  & \multicolumn{1}{c|}{TPR} & 99.52 & 99.10 & \multicolumn{1}{|c}{0.42} \\
        \multicolumn{1}{c|}{FPR} & 90.18 & 86.30 & \multicolumn{1}{|c}{3.87} &  & \multicolumn{1}{c|}{FPR} & 88.76 & 84.97 & \multicolumn{1}{|c}{3.79}
    \end{tabular}
    \caption{Fairness metrics by sex}
    \label{tab:seer metrics without A sex}
\end{table}

    \item By origin \begin{itemize}
        \item The lowest acceptance rate has increased and the highest has stayed at 100\%, for Missing. Aside from this category, the highest AR is still for White and the lowest still for Black.
        \item Black still has the lowest true positive rate, but it has increased. Missing still has the highest TPR, at 100\%. The second highest is for American Indian/AK Native, which had a TPR of 100\% in the previous model.
        \item Apart from Missing, White had the lowest false positive rate but now Black does. American Indian/AK Native had the highest FPR but now Hispanic does.
    \end{itemize}
    Black is still the most disadvantaged group under the statistical parity and equal opportunity definitions. With the previous model, we could not conclude on the most disadvantaged group under the equalized odds definition, but here without taking into account the value for Missing, Black is the most disadvantaged group.
    
\begin{table}[H]
    \centering
    \begin{tabular}{cccccccc}
        \multicolumn{8}{c}{With all variables} \\
        \multicolumn{1}{c|}{(\%)} & White & Hispanic & Black & \begin{tabular}[c]{@{}c@{}}Asian or\\ Pacific Islander\end{tabular} & \begin{tabular}[c]{@{}c@{}}American Indian\\ /AK Native\end{tabular} & Missing & \multicolumn{1}{|c}{Var} \\ \hline
        \multicolumn{1}{c|}{AR} & 99.33 & 97.70 & 92.53 & 96.63 & 98.72 & 100.00 & \multicolumn{1}{|c}{$6.10\mathrm{e}{-4}$} \\
        \multicolumn{1}{c|}{TPR} & 99.40 & 97.94 & 93.21& 97.53 & 100.00 & 100.00 & \multicolumn{1}{|c}{$5.52\mathrm{e}{-4}$} \\
        \multicolumn{1}{c|}{FPR} & 89.20 & 97.94 & 93.21 & 97.53 & 100.00 & 0.00 & \multicolumn{1}{|c}{$8.35\mathrm{e}{-2}$} \\
         &  &  &  &  &  & \\
        \multicolumn{8}{c}{Without protected variables} \\
        \multicolumn{1}{c|}{(\%)} & White & Hispanic & Black & \begin{tabular}[c]{@{}c@{}}Asian or\\ Pacific Islander\end{tabular} & \begin{tabular}[c]{@{}c@{}}American Indian\\ /AK Native\end{tabular} & Missing & \multicolumn{1}{|c}{Var} \\ \hline
        \multicolumn{1}{c|}{AR} & 99.23 & 98.76 & 93.85 & 96.68 & 99.10 & 100.00 & \multicolumn{1}{|c}{$4.37\mathrm{e}{-4}$} \\
        \multicolumn{1}{c|}{TPR} & 99.31 & 98.88 & 95.32 & 97.05 & 99.54 & 100.00 & \multicolumn{1}{|c}{$2.71\mathrm{e}{-4}$} \\
        \multicolumn{1}{c|}{FPR} & 87.07 & 89.66 & 45.45 & 70.00 & 66.67 & 0.00 & \multicolumn{1}{|c}{$9.28\mathrm{e}{-2}$}
    \end{tabular}
    \caption{Fairness metrics by origin}
    \label{tab:seer metrics without A origin}
\end{table}

    \item By marital status \begin{itemize}
        \item Compared to the previous model, the acceptance rates are closer to each other, with the lowest value going from 95.75\% for Separated to 97.48\% for Unmarried and the highest value going from 99.99\% to 99.89\%, both for Missing.
        \item The true positive rates are also closer to each other. Widowed had the lowest TPR, but now Unmarried does. Missing still has the highest.
        \item The false positive rates are also closer to each other. Widowed still has the lowest FPR and Unmarried still has the highest. Separated had the highest with Unmarried, but now has a lower FPR.
    \end{itemize}
    For statistical parity, we are closer to fairness but the most disadvantaged group is not the same. For equal opportunity and equalized odds, we are a little closer to fairness and Widowed is still the most disadvantaged group.
    
\begin{table}[H]
    \centering
    \begin{tabular}{ccccccccc}
        \multicolumn{9}{c}{With all variables} \\
        \multicolumn{1}{c|}{(\%)} & Married & Missing & Single & Widowed & Divorced & Separated & Unmarried & \multicolumn{1}{|c}{Var} \\ \hline
        \multicolumn{1}{c|}{AR} & 99.23 & 99.99 & 98.82 & 97.29 & 97.47 & 95.75 & 98.44 & \multicolumn{1}{|c}{$1.72\mathrm{e}{-4}$} \\
        \multicolumn{1}{c|}{TPR} & 99.33 & 99.99 & 98.95 & 97.54 & 97.69 & 95.71 & 98.43 & \multicolumn{1}{|c}{$1.71\mathrm{e}{-4}$} \\
        \multicolumn{1}{c|}{FPR} & 88.16 & 97.10 & 84.30 & 78.79 & 82.19 & 100.00 & 100.00 & \multicolumn{1}{|c}{$6.76\mathrm{e}{-2}$} \\
        \multicolumn{9}{c}{ } \\
        \multicolumn{9}{c}{Without protected variables} \\
        \multicolumn{1}{c|}{(\%)} & Married & Missing & Single & Widowed & Divorced & Separated & Unmarried & \multicolumn{1}{|c}{Var} \\ \hline
        \multicolumn{1}{c|}{AR} & 99.01 & 99.89 & 99.06 & 97.74 & 98.14 & 97.75 & 97.48 & \multicolumn{1}{|c}{$6.85\mathrm{e}{-5}$} \\
        \multicolumn{1}{c|}{TPR} & 99.12 & 99.91 & 99.15 & 98.13 & 98.29 & 97.96 & 97.44 & \multicolumn{1}{|c}{$6.23\mathrm{e}{-5}$} \\
        \multicolumn{1}{c|}{FPR} & 85.43 & 92.75 & 87.96 & 78.72 & 87.32 & 85.71 & 100.00 & \multicolumn{1}{|c}{$3.78\mathrm{e}{-3}$}
    \end{tabular}
    \caption{Fairness metrics by marital status}
    \label{tab:seer metrics without A mar stat}
\end{table}
\end{itemize}

To conclude, the performance metrics have not deteriorated too much compared to the model using all variables. For Sex, the fairness is better under all three definitions and the same group remains in a disadvantage position. For Origin, we also have the same most disadvantaged group as in the model using all variables, for some metrics the fairness has improved and for others it has deteriorated. For marital status, fairness has improved but for one of the definitions the most diasadvantaged group is not the same as before.

% Figure environment removed

\subsection{Transforming the non-protected variables to mitigate indirect discrimination}

\paragraph{} As we have done for the simulated data, we will change the basis formed by the centered variables to obtain transformed non-sensitive explanatory variables uncorrelated with the sensitive ones. We are dealing with 12 sensitive variables, because the sensitive variables \texttt{Sex}, \texttt{Origin} and \texttt{Mar\_stat} have been converted to dummy variables. The procedure is the exact same as in the simulated case, but with more variables.

\paragraph{Application to the data} We apply the procedure to all explanatory variables (ie not to the exposure nor the variable of interest) and obtain the projected non-sensitive variables that are uncorrelated to the sensitive ones. Figure~\ref{fig:seer heatmap proj} shows the correlation matrix before and after orthogonal projection. The correlations between the sensitive and other variables are framed in blacked. On the left, we can see a few correlations (blue and red boxes) and on the right, they are all null (only grey boxes). As expected, these results are the same as in the simulated case. 

% Figure environment removed

% Figure environment removed

\paragraph{Applying the model} We will now apply the logistic regression model to the projected non-sensitive variables, keeping the exposure as weights, to predict the probabilities of death due to melanoma of the skin. We will compare the performance and fairness metrics of this model to the ones of the model without protected variables. The coefficients are presented in appendix~\ref{appendix:tab:coeff proj}. \par
Figure~\ref{fig:seer coeff proj} helps us visualize the coefficients and their standard errors. The most noticeable difference with the coefficients of the model without protected variables concerns \texttt{Site\_rec\_WHO08\_Male Genital Organs} which still has a large standard error but a positive coefficient instead of a negative one. We have the oppposite effect for \texttt{Laterality\_Paired site midline tumor}. \par

% Figure environment removed

\paragraph{Performance evaluation} Looking at the ROC curve in figure~\ref{fig:ROC proj}, the model performances have once again downgraded: we went from an AUC of 0.8778 (model without the protected variables) to an AUC of 0.8534, which still indicates good predictive performance. \par

% Figure environment removed

% Figure environment removed

\paragraph{Fairness evaluation}
\begin{itemize}
    \item Globally, the AR, TPR and FPR are higher than for the model without protected variables. The AR and FPR are 0.1 point higher and the FPR 0.7 point higher.
    
\begin{table}[H]
    \centering
    \begin{tabular}{ccccc}
        \multicolumn{2}{c}{Without protected variables}  &  & \multicolumn{2}{c}{Transformed variables} \\
        \multicolumn{1}{c|}{(\%)} & Global  &  & \multicolumn{1}{c|}{(\%)} & Global \\ \cline{1-2} \cline{4-5} 
        \multicolumn{1}{c|}{AR} & 99.20 &  & \multicolumn{1}{c|}{AR} & 99.30 \\
        \multicolumn{1}{c|}{TPR} & 99.29 &  & \multicolumn{1}{c|}{TPR} & 99.39 \\
        \multicolumn{1}{c|}{FPR} & 86.26 &  & \multicolumn{1}{c|}{FPR} & 86.92
              
    \end{tabular}
    \caption{Global fairness metrics}
    \label{tab:seer metrics proj global}
\end{table}
    
    \item By sex \begin{itemize}
        \item The acceptance rates are now only 0.03 point apart.
        \item The true positive rates are only 0.04 point apart.
        \item The false positive rates are further apart and have changed signs.
    \end{itemize}
    We have approximately reached statistical parity when looking at the Sex variable: both groups are treated fairly by the model. We have also approximately reached equal opportunity. As the FPR are now 9.02 points apart, we are further away from the definition of equalized odds, with now the Female category being disadvantaged by the model.
\begin{table}[H]
    \centering
    \begin{tabular}{ccccccccc}
        \multicolumn{4}{c}{Without protected variables} &  & \multicolumn{4}{c}{Transformed variables} \\
        \multicolumn{1}{c|}{(\%)} & Female & Male & \multicolumn{1}{|c}{Difference} &  & \multicolumn{1}{c|}{Sex} & Female & Male & \multicolumn{1}{|c}{Difference} \\ \cline{1-4} \cline{6-9} 
        \multicolumn{1}{c|}{AR} & 99.46 & 98.97 & \multicolumn{1}{|c}{0.49} &  & \multicolumn{1}{c|}{AR} & 99.32 & 99.29 & \multicolumn{1}{|c}{0.03} \\
        \multicolumn{1}{c|}{TPR} & 99.52 & 99.10 & \multicolumn{1}{|c}{0.42} &  & \multicolumn{1}{c|}{TPR} & 99.41 & 99.37 & \multicolumn{1}{|c}{0.04} \\
        \multicolumn{1}{c|}{FPR} & 88.76 & 84.97 & \multicolumn{1}{|c}{3.79} &  & \multicolumn{1}{c|}{FPR} & 80.75 & 89.77 & \multicolumn{1}{|c}{-9.02}
    \end{tabular}
    \caption{Fairness metrics by sex}
    \label{tab:seer metrics proj sex}
\end{table}

    \item By origin \begin{itemize}
        \item The acceptances rates are now very close to each other. The order is the same except for Black and Asian that switched places (the two lowest acceptance rates). The variance in acceptances rates is of $1.82\mathrm{e}{-5}$0.
        \item The true positive rates are closer to each other than in the case without protected variables. The order is the same except for the American Indian/AK Native that went from second highest TPR to second lowest. The variance in true positive rates is of $1.23\mathrm{e}{-5}$0.
        \item The false positive rates are further apart than before, with a variance of 0.12. The order has been changed: Hispanic had the highest FPR but now has the fourth highest. Black and American Indian/AK Native respectively had the fifth and fourth highest but now have the same highest. Missing still has the lowest.
    \end{itemize}
    We have approximately reached statistical parity and equal opportunity when looking at the Origin variable, as we have very close acceptance and true positive rates. For equalized odds, we are further away from fairness as the false positive rates are further apart.

\begin{table}[H]
    \centering
    \begin{tabular}{cccccccc}
        \multicolumn{8}{c}{Without protected variables} \\
        \multicolumn{1}{c|}{(\%)} & White & Hispanic & Black & \begin{tabular}[c]{@{}c@{}}Asian or\\ Pacific Islander\end{tabular} & \begin{tabular}[c]{@{}c@{}}American Indian\\ /AK Native\end{tabular} & Missing & \multicolumn{1}{|c}{Var} \\ \hline
        \multicolumn{1}{c|}{AR} & 99.23 & 98.76 & 93.85 & 96.68 & 99.10 & 100.00 & \multicolumn{1}{|c}{$4.37\mathrm{e}{-4}$} \\
        \multicolumn{1}{c|}{TPR} & 99.31 & 98.88 & 95.32 & 97.05 & 99.54 & 100.00 & \multicolumn{1}{|c}{$2.71\mathrm{e}{-4}$} \\
        \multicolumn{1}{c|}{FPR} & 87.07 & 89.66 & 45.45 & 70.00 & 66.67 & 0.00 & \multicolumn{1}{|c}{$9.28\mathrm{e}{-2}$} \\
         &  &  &  &  &  &  &  \\
        \multicolumn{8}{c}{Transformed variables} \\
        \multicolumn{1}{c|}{(\%)} & White & Hispanic & Black & \begin{tabular}[c]{@{}c@{}}Asian or\\ Pacific Islander\end{tabular} & \begin{tabular}[c]{@{}c@{}}American Indian\\ /AK Native\end{tabular} & Missing & \multicolumn{1}{|c}{Var} \\ \hline
        \multicolumn{1}{c|}{AR} & 99.30 & 98.93 & 98.88 & 98.62 & 99.04 & 99.96 & \multicolumn{1}{|c}{$1.82\mathrm{e}{-5}$} \\
        \multicolumn{1}{c|}{TPR} & 99.39 & 99.18 & 98.85 & 99.16 & 99.02 & 99.96 & \multicolumn{1}{|c}{$1.23\mathrm{e}{-5}$} \\
        \multicolumn{1}{c|}{FPR} & 87.45 & 76.92 & 100.00 & 63.64 & 100.00 & 0.00 & \multicolumn{1}{|c}{$1.18\mathrm{e}{-1}$}
    \end{tabular}
    \caption{Fairness metrics by origin}
    \label{tab:seer metrics proj race}
\end{table}

    \item By marital status \begin{itemize}
        \item The acceptance rates are now almost equal, with a variance of $1.83\mathrm{e}{-5}$0.
        \item The same goes for the true positive rates, with a variance of $1.70\mathrm{e}{-5}$0.
        \item The false positive rates are a lot further apart.
    \end{itemize}
    We have approximately reached statistical parity and equal opportunity. For equalized odds, we are further away.
\end{itemize}

\begin{table}[H]
    \centering
    \begin{tabular}{ccccccccc}
        \multicolumn{9}{c}{Without protected variables} \\
        \multicolumn{1}{c|}{(\%)} & Married & Missing & Single & Widowed & Divorced & Separated & Unmarried & \multicolumn{1}{|c}{Var} \\ \hline
        \multicolumn{1}{c|}{AR} & 99.01 & 99.89 & 99.06 & 97.74 & 98.14 & 97.75 & 97.48 & \multicolumn{1}{|c}{$6.85\mathrm{e}{-5}$} \\
        \multicolumn{1}{c|}{TPR} & 99.12 & 99.91 & 99.15 & 98.13 & 98.29 & 97.96 & 97.44 & \multicolumn{1}{|c}{$6.23\mathrm{e}{-5}$} \\
        \multicolumn{1}{c|}{FPR} & 85.43 & 92.75 & 87.96 & 78.72 & 87.32 & 85.71 & 100.00 & \multicolumn{1}{|c}{$3.78\mathrm{e}{-3}$} \\
         &  &  &  &  &  &  &  &  \\
        \multicolumn{9}{c}{With transformed variables} \\
        \multicolumn{1}{c|}{(\%)} & Married & Missing & Single & Widowed & Divorced & Separated & Unmarried & \multicolumn{1}{|c}{Var} \\ \hline
        \multicolumn{1}{c|}{AR} & 99.12 & 99.77 & 99.06 & 99.19 & 99.07 & 98.40 & 98.47 & \multicolumn{1}{|c}{$1.83\mathrm{e}{-5}$} \\
        \multicolumn{1}{c|}{TPR} & 99.23 & 99.79 & 99.18 & 99.42 & 99.21 & 98.66 & 98.47 & \multicolumn{1}{|c}{$1.70\mathrm{e}{-5}$} \\
        \multicolumn{1}{c|}{FPR} & 86.13 & 92.86 & 84.40 & 86.96 & 90.12 & 75.00 & 0.00 & \multicolumn{1}{|c}{$9.31\mathrm{e}{-2}$}
    \end{tabular}
    \caption{Fairness metrics by marital status}
    \label{tab:seer metrics proj mar stat}
\end{table}

% Figure environment removed

\subsection{Conclusion on the methods}

\paragraph{} Just like in the simulated case, we compared the performance and fairness of the logistic regression model using three different types of explanatory variables: all variables, only non-sensitive variables, and transformed non-sensitive variables. Visual results by sensitive variables can be found in appendices~\ref{appendix:fig:metrics by sex seer}, \ref{appendix:fig:metrics by origin seer} and \ref{appendix:fig:metrics by marital status seer}. \par
The model using all variables has the best results in terms of AUC, with only non-sensitive variables we have a decrease in AUC and finally with transformed non-sensitive variables the AUC is the lowest. This all shows the trade-off between performance and fairness. \par
In terms of fairness, the model using all variables treats unfairly the different protected groups. By Sex, Male is disadvantaged under all three definitions of fairness. By Origin, Black is the most disadvantaged group under the statistical parity and equal opportunity definitions. For equalized odds, we cannot conclude. By Marital status, Separated is the most disadvantaged category under the statistical parity definition and Widowed under the two others. \par
When we remove the protected variables, thus avoiding direct discrimination, the results depend on the relationships the protected variables have with the other variables. By sex, fairness improves under all definitions, but Male is still disadvantaged. By Origin, fairness improves under the statistical parity and equal opportunity definitions, with Black still being the most disadvantaged group. Under the equalized odds definition, one involved metric improves but the other does not. By Marital status, fairness improves under all definitions. For statistical parity, Unmarried is now the most disadvantaged group instead of Separated. For the other definitions, Widowed is still the most disadvantaged group. \par
Finally, applying the change of basis method transformed the non-sensitive variables and we now have a null correlation with the sensitive variables. Applying the model, we have the expected results: we are now very close to respecting statistical parity. Looking at the two other definitions, we are closer to equal opportunity. Looking at the metrics for equalized odds, the fairness is worse, so we can conclude that all fairness definitions are not compatible with each other. \par


\newpage