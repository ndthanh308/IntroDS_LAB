
\section{Use case: mortality of individuals with melanoma of the skin}
\paragraph{} In this section, we will study a real-life use case: the mortality of individuals with melanoma skin cancer. It is the 17th most common cancer worldwide, with over 300,000 new cases in 2020 \cite{WHO20}. In the US, the 5-year survival rate is of 93.7\% over the period of 2012 to 2018 \cite{SEER}. \par

\subsection{Some information about skin cancer}
\paragraph{} The skin, the body's largest organ, consists of several lays. The two main ones are the epidermis and the dermis, as shown in figure~\ref{fig:skin}. Skin cancer begins in the epidermis, which consists of three types of cells: squamous cells, basal cells and melanocytes. Melanocytes are cells that can make melanin, which is the pigment giving the skin its color. \par
Two different cancers can start in the skin: non-melanoma and melanoma, each representing about half of skin cancers. Non-melanoma skin cancer forms in the lower part of the epidermis or in squamous cells, but not in melanocytes. Melanoma forms in melanocytes and is more likely to spread out to other parts of the body. It can start in the skin, but also in mucous membranes such as parts of the eye. In this thesis, we focus on melanoma. \par

% Figure environment removed


Medical literature \cite{SEER} identifies several risk factors for melanoma, including:
\begin{itemize}
    \item Fair complexion
    \item Exposure to natural or artificial sunlight
    \item Exposure to certain environmental factors (radiation, solvents, \dots)
    \item History of blistering sunburns
    \item Presence of several large or many small moles
    \item Family history of unusual moles or melanoma
    \item Weakened immune system
    \item Changes in genes linked to melanoma
\end{itemize} \par

Melanomas are mainly characterized by the location, thickness and ulceration \label{ulceration} (ie whether it has broken through the skin) of the tumor, the speed at which cancer cells divide, its spread to lymph nodes and other parts of the body. All of these factors can impact the severity of the condition. The TNM staging system is an internationally recognized standard for classifying cancers:
\begin{itemize}
    \item T describes the size of the primary tumor, as shown in figure~\ref{fig:skin T}:
    \begin{itemize}
        \item Tis: in situ
        \item T1: less than 1 mm thick
        \item T2: between 1 and 2 mm thick
        \item T3: between 2 and 4 mm thick
        \item T4: more than 4 mm thick
    \end{itemize}
% Figure environment removed

    \item N describes the spread to regional lymph nodes:
    \begin{itemize}
        \item N0: no melanoma cells in the nearby lymph nodes
        \item N1: melanoma cells in one lymph node or in-transit, satellite or microsatellite metastases
        \item N2: melanoma cells in 2 or 3 lymph nodes or in one lymph node and in-transit, satellite or microsatellite metastases
        \item N3: melanoma cells in 4 or more lymph nodes or in 2 or 3 lymph nodes and in-transit, satellite or microsatellite metastases or in any number of lymph nodes stuck to each other (matted)
    \end{itemize}
    \item M describes the presence of metastasis (M0 for non-metastatic and M1 for metastatic).
\end{itemize}
% The higher the values of T, N and M, the higher the mortality risk. 
The more well-known overall stage grouping describes the progression of a cancer thanks to five categories:
\begin{itemize}
    \item Stage 0: also called in situ, when abnormal melanocytes are found. They can become cancer and spread.
    \item Stage I: cancer has formed and is localized.
    \item Stage II: the cancer is locally advanced, in early stages.
    \item Stage III: the cancer is locally advanced, in late stages.
    \item Stage IV: the cancer has spread to other parts of the body which may be distant to the origin site, it is metastatic.
\end{itemize}
The link between the two staging systems is straightforward, as seen in table~\ref{tab:stage TNM}.
\begin{table}[H]
    \centering
    \begin{tabular}{|c|c|c|c|}
        \hline
        Stage & T & N & M \\ \hline
        0 & 0 & 0 & 0 \\ \hline
        I & 1-2 & 0 & 0 \\ \hline
        II & 3-4 & 0 & 0 \\ \hline
        III & 1-4 & 1-3 & 0 \\ \hline
        IV & 1-4 & 1-3 & 1 \\ \hline
    \end{tabular}
    \caption{Staging and its relation to the TNM system}
    \label{tab:stage TNM}
\end{table}


\subsection{Database presentation and mapping}

\paragraph{Presentation} In order to model the mortality of melanoma of the skin cancer patients, we used the public research SEER database. SEER is the Surveillance, Epidemiology, and End Results program of the National Cancer Institute of the United States that collects and publishes cancer information about around 48\% of the American population. This data collection process dates back to 1973, with around 400,000 new cases collected yearly in the most recent years. It is the largest and one of the most reliable cancer database, which makes it a dependable source of information for all types of studies. It is also representative of the US population in terms of measures of poverty and education. \par
The variables describe:
\begin{itemize}
    \item the patient: ID number, sex, age at diagnosis, year of diagnosis, origin, marital status at diagnosis, \dots
    \item the cancer (characteristics at diagnosis): type, site of origin, tumor size, spread to lymph nodes, metastatic state, stage, \dots
\end{itemize}

% Figure environment removed

\paragraph{Mapping} This database is a huge source of information: there are 5,075,266 observations and 164 variables. Melanoma of the skin cancers represent about 4\% of all observations. Along the years, the way cancers are described has drastically changed, mostly because of the changes in classification of diseases standards. We therefore had to preprocess the data by studying the significance of variables in order to have the same standards throughout the years. For example, we had to regroup 8 different variables to determine the size of the tumor. \par
 Figure~\ref{fig:missings values raw and clean} shows how much preprocessing had to be done: the white spaces represent the missing values. In the raw data, we had more than 5 million rows and 164 columns. After keeping only patients with melanoma of the skin, mapping and deleting variables that had nothing to do with skin cancer (about breast cancer for example), we have 177,960 rows and 47 columns. \par

% Figure environment removed

\paragraph{Missing values and initial variable selection} After the mapping step, we still have many missing values. As we saw in section~\ref{constraints}, there are many constraints to be taken into account when choosing which variables to keep for our model.
\begin{itemize}
    \item Medical constraints:\\
    The probability of dying from cancer greatly depends on the metastatic state, so we cannot do without it. We remove individuals with missing metastatic state from the database. \\
    We also need to have the information on whether the individual is alive or not at the end of the observation period, so we remove individuals with missing information. \\
    Some variables are not medically relevant, for example the type of reporting source: if the information about an individual was collected by a hospital, a lab or another medical facility. So we delete this type of variable. \\

% Figure environment removed

    As mortality rates evolve and change over the years, we need to set our time window so that the rates are still valid. Figure~\ref{fig:Yr_dx_count} shows the year of diagnosis distribution. We already have only recent years, mainly because we deleted individuals with unspecified metastatic state, and this variable was not collected in the past.

% Figure environment removed

    \item Underwriting constraints: \\
    We only keep individuals who are between 18 and 80 years old, as they are the ones that are covered by most markets. \\
    Normally, underwriting constraints impose, depending on the local regulation, the deletion of sensitive variables, like the origin of the individual. But this thesis focuses on the subject of discrimination, so we have to keep these variables to be able to measure bias.
    \item Modeling constraints: we do not keep variables to are strongly correlated with each other. This is the case for variables representing redundant information: we have 3 different variables for age at diagnosis, 4 different variables for origin, \dots We only keep one of each, the most complete one.
\end{itemize} \par
For the remaining missing values, we decided to delete the columns with almost only missing values, as they will not be useful for the model. Then, for numerical variables, the median of the series is assigned to any missing value. For categorical variables, we replace the missing values by a category `Missing'. In the end, we are left with 101,797 rows and 28 columns. \par

\subsection{Specificity of survival analysis}

\paragraph{} Most of the time, survival duration is observed partially. This can be due to the occurrence of the event of interest - in our case, death - outside the observation period, or to other events that result in the individual leaving the study (eg lapses, hospital transfers, recording systems failure). This censoring and truncation characterize survival data. If these effects were ignored, the probability of the event of interest would be underestimated. Another mistake would be to remove individuals for which the observation is incomplete from the study, because it would once again lead to biased estimations. \par
To deal with these issues, we either need to use specific models or modify the structure of the data to use standard models. We will use this second approach, which transforms the data by separating it into small time intervals. This allows to model the number of deaths by standard Machine Learning techniques using the exposure to risk as weights, taking into account the censoring. \par

\subsubsection{Exposure}

A first step is therefore to compute the exposure to risk of each individual. It can be computed differently depending on the hypothesis made on mortality. We compute the initial exposure which represents the time each individual was exposed to risk in the interval. It is called initial because it is based on the information at the beginning of the interval. \par
The individual initial exposure $e_{i,j}$ for individual i in time interval $[\tau_j,\tau_{j+1}]$ takes value:
\begin{itemize}
    \item 1 if the individual is alive during the entire time interval
    \item 1 if the individual dies during the time interval
    \item the fraction of the time interval he was observed if the individual is not observed during the entire time interval
\end{itemize}
Formally, denoting
\begin{itemize}
    \item[] $c_{i,j}$ the censoring time of individual i in time interval $[\tau_j,\tau_j+1]$
    \item[] $t_{i,j}$ the death time of individual i in time interval $[\tau_j,\tau_j+1]$
    \item[] $w_j$ the number of individuals withdrawn from the study in time interval $[\tau_j,\tau_j+1]$
    \item[] $l_j$ the number of individuals that are alive during the entire time interval $[\tau_j,\tau_j+1]$
    \item[] $d_j$ the number of individuals that died during the time interval $[\tau_j,\tau_j+1]$
\end{itemize}
we can write the individual initial exposure as:
\begin{equation*}
    \begin{split}
        e_{i,j} & =\left\{
        \begin{array}{ll}
            1 & \mbox{if } t_{i,j}>1 \mbox{ and } c_{i,j}>1  \\
            1 & \mbox{if } t_{i,j} <1 \\
            c_{i,j} & \mbox{if } c_{i,j}<1
        \end{array}
        \right. \\
         & =\underbrace{\mathbb{1}_{t_{i,j}>1}\times\mathbb{1}_{c_{i,j}>1}+\mathbb{1}_{t_{i,j}<1}}_{1-\mathbb{1}_{c_{i,j}<1}}+c_{i,j}\mathbb{1}_{c_{i,j}<1}
    \end{split}
\end{equation*}
The global initial exposure for all individuals in time interval $[\tau_j,\tau_j+1]$ is
\begin{equation*}
    \begin{split}
        E_j & =\sum_{i=1}^{l_j} e_{i,j} \\
         & =\sum_{i=1}^{l_j} (1-\mathbb{1}_{c_{i,j}<1}+c_{i,j}\mathbb{1}_{c_{i,j}<1}) \\
         & =l_j-w_j+\sum_{i=1}^{w_j}c_{i,j}
    \end{split}
\end{equation*}
The number of deaths is the sum of the number of observed deaths and expected deaths (from censored individuals). Writing down
\begin{itemize}
    \item[] $q_j=\mathbb{P}(T<\tau_j+1|T>\tau_j)$ the mortality rate in time interval $[\tau_j,\tau_j+1]$
    \item[] $_{c_{i,j}}q_j=\mathbb{P}(T<\tau_j+c_{i,j}|T>\tau_j)$ the mortality rate in time interval $[\tau_j,c_{i,j}]$
\end{itemize}
we can compute
\begin{equation*}
    \begin{split}
        d_j & =(l_j-w_j)q_j+\sum_{i=1}^{w_j} {}_{c_{i,j}}q_j \\
         & =l_j q_j-\sum_{i=1}^{w_j} {}_{1-c_{i,j}}q_{j+c_{i,j}}
    \end{split}
\end{equation*}
The Balducci hypothesis supposes that mortality rates decrease over the interval and are defined as:
\begin{equation*}
    \begin{split}
        {}_{1-c_{i,j}}q_{j+c_{i,j}} & =\mathbb{P}(T_i\leq\tau_j+1|T_i>\tau_j+c_{i,j}) \\
         & =(1-c_{i,j})\mathbb{P}(T_i\leq\tau_j+1|T_i>\tau_j) \\
         & =(1-c_{i,j})q_j
    \end{split}
\end{equation*}
So we can write
\begin{equation*}
    d_j=l_jq_j-q_j\sum_{i=1}^{w_j} (1-c_{i,j})
\end{equation*}
Solving the formula for $q_j$ gives us:
\begin{equation*}
        \hat{q}_j=\frac{d_j}{l_j-\sum_{i=1}^{w_j} (1-c_{i,j})}=\frac{d_j}{l_j-w_j+\sum_{i=1}^{w_j} c_{i,j}}
\end{equation*}
\begin{equation*}
    \boxed{\hat{q}_j=\frac{d_j}{E_j}}
\end{equation*}
We obtained the estimation for mortality rates on time interval $[\tau_j,\tau_j+1]$, corrected for censoring, using the Balducci hypothesis. Although it is generally not verified because mortality rates increase with time, the errors can be ignored as withdrawals are small compared to the population.

% Figure environment removed

\subsubsection{Five-year mortality rates: a first look into the influence of each variable}

\paragraph{} In order to better understand the influence of certain factors on the mortality of melanoma of the skin patients and to compare our data with medical literature, we will compute the five-year mortality rates depending on these factors. In relation to the previous section, we need to compute the number of deaths and the global initial exposure in time interval $[0,5]$ (in years). For the number of deaths, we need to separate the cases of non-metastatic and metastatic patients, because metastasis implies the spread of the cancer to other sites, meaning that the cause of death can be a different cancer or disease that was caused by the initial cancer. For non-metastatic patients, the number of deaths will be computed with the number of deaths caused by skin melanoma and for metastatic patients, all causes of death will be taken into account. For individual i, we compute the death variable:
\begin{equation*}
    \begin{split}
        d_{i,5}^{M0}= & \left\{
        \begin{array}{ll}
            1 & \mbox{if Dead (due to skin melanoma) and survival < 5 years} \\
            0 & \mbox{else}
        \end{array}
        \right. \\
        d_{i,5}^{M1}= & \left\{
        \begin{array}{ll}
            1 & \mbox{if Dead (all causes) and survival < 5 years} \\
            0 & \mbox{else}
        \end{array}
        \right.
    \end{split}
\end{equation*}

Then, we compute the individual initial exposure as presented in the last section:
\begin{equation*}
    \begin{split}
        e_{i,5} & =\left\{
        \begin{array}{ll}
            1 & \mbox{if } t_{i,5}>1 \mbox{ and } c_{i,5}>1  \\
            1 & \mbox{if } t_{i,5} <1 \\
            c_{i,5} & \mbox{ if } c_{i,5}<1
        \end{array}
        \right. \\
         & =\left\{
        \begin{array}{ll}
            1 & \mbox{ if Vital\_status=0 or Survival\_years} \geq 5 \mbox{ Years} \\
            \frac{\mbox{Survival\_years}}{5} & \mbox{ else}
        \end{array}
        \right.
    \end{split}
\end{equation*}
Then, the estimation of the 5-year mortality rate is:
\begin{equation*}
    \hat{q}_5=\frac{\sum_{i=1}^{l_5} d_{i,5}}{\sum_{i=1}^{l_5} e_{i,5}}
\end{equation*}

We will compare the estimated 5-year mortality rates depending on the following factors: metastatic state (M), tumor size (T), spread to regional lymph nodes (N), stage, gender, origin and age. As we stated earlier, we need to separate the cases of non-metastatic and metastatic cancers, but without taking into account the metastatic state and looking at deaths caused by melanoma of the skin only, the 5-year mortality rate is of 7.59\%, ie the 5-year survival rate is of 92.41\% which is very close to the survival rate of 93.7\% given by the National Cancer Institute. \par

\noindent \begin{center}
    \fbox{
\begin{minipage}{0.9\textwidth}
   For the following, it is important to note that looking at mortality rates in different categories is tricky as groups may have different age distributions. Mortality rates by sex might capture correlation with age, if for example there are more observations of one gender than the other in some age groups.
\end{minipage}
}
\end{center}

The five-year mortality rate by metastatic state is given in table~\ref{tab:qx M}. As expected, the mortality rate for metastatic patients is considerably higher than for non-metastatic patients. \par

\begin{table}[H]
    \centering
    \begin{tabular}{c|c|c|}
        \cline{2-3}
         & Melanoma of the skin & All causes \\ \hline
        \multicolumn{1}{|c|}{M0} & 5.28\% & \cellcolor[HTML]{9B9B9B} \\ \hline
        \multicolumn{1}{|c|}{M1} & \cellcolor[HTML]{9B9B9B} & 83.00\% \\ \hline
    \end{tabular}
    \caption{Five-year mortality rate by presence of metastasis}
    \label{tab:qx M}
\end{table}


We expect mortality rates to go up with larger tumors. In the non-metastatic case, except for missing and T0 tumor sizes, the thicker the tumor, the higher the mortality rate. For a tumor size T1, the 5-year mortality rate is of 1.23\%, going up to 28.13\% for tumor size T4. The high mortality rate for T0 can be explained by the small number of observations: there are only 496 records of it, whereas for T1 for example there are 67,243 observations. A small number of observations results in a high variance in the estimation. Individuals with missing tumor sizes have a five-year mortality rate which is close to individuals with T2 tumors. In the metastatic case, we have the same problem for individuals with missing and T0 tumor sizes. Another unexpected result is that $\hat{q}_5^{M1,T1}>\hat{q}_5^{M1,T2}$0. This last result can come from the small sample sizes that result in a great estimation variance: there are respectively 202 and 154 individuals with tumors sizes T1 and T2. All mortality rates for metastatic cancers are in a smaller interval than for non-metastatic cancers. This might be interpreted as a lower predictive importance of the size of the primary tumor for metastatic cancers, and this might be due to the fact that these cancers have already spread to other sites. \par

\begin{table}[H]
    \centering
    \begin{tabular}{cc|c|c|}
            \cline{3-4}
         &  & Melanoma of the skin & All causes \\ \hline
        \multicolumn{1}{|c|}{} & Missing & 8.41\% & \cellcolor[HTML]{9B9B9B}{\color[HTML]{9B9B9B} } \\ \cline{2-4} 
        \multicolumn{1}{|c|}{} & T0 & 32.56\% & \cellcolor[HTML]{9B9B9B}{\color[HTML]{9B9B9B} } \\ \cline{2-4} 
        \multicolumn{1}{|c|}{} & T1 & 1.23\% & \cellcolor[HTML]{9B9B9B}{\color[HTML]{9B9B9B} } \\ \cline{2-4} 
        \multicolumn{1}{|c|}{} & T2 & 7.00\% & \cellcolor[HTML]{9B9B9B}{\color[HTML]{9B9B9B} } \\ \cline{2-4} 
        \multicolumn{1}{|c|}{} & T3 & 16.53\% & \cellcolor[HTML]{9B9B9B}{\color[HTML]{9B9B9B} } \\ \cline{2-4} 
        \multicolumn{1}{|c|}{\multirow{-6}{*}{M0}} & T4 & 28.13\% & \cellcolor[HTML]{9B9B9B} \\ \hline
        \multicolumn{1}{|c|}{} & Missing & \cellcolor[HTML]{9B9B9B} & 85.93\% \\ \cline{2-4} 
        \multicolumn{1}{|c|}{} & T0 & \cellcolor[HTML]{9B9B9B} & 82.09\% \\ \cline{2-4} 
        \multicolumn{1}{|c|}{} & T1 & \cellcolor[HTML]{9B9B9B} & 79.87\% \\ \cline{2-4} 
        \multicolumn{1}{|c|}{} & T2 & \cellcolor[HTML]{9B9B9B} & 75.54\% \\ \cline{2-4} 
        \multicolumn{1}{|c|}{} & T3 & \cellcolor[HTML]{9B9B9B} & 82.90\% \\ \cline{2-4} 
        \multicolumn{1}{|c|}{\multirow{-6}{*}{M1}} & T4 & \cellcolor[HTML]{9B9B9B} & 83.52\% \\ \hline
    \end{tabular}
    \caption{Five-year mortality rate by tumor size}
    \label{tab:qx T}
\end{table}

The spread to regional lymph nodes indicates how far a cancer has spread. The mortality rates are expected to go up with a more extensive spread. Only stages III and IV exhibit regional lymph node spread. For non-metastatic cases, we observe as expected an increase in mortality rates with greater spread. For the metastatic case, we have the same issues as in the previous paragraphs: mortality rates do not behave as expected. This can be due either to the little number of observations or the fact that with metastasis, the cancer has necessarily spread to regional lymph nodes, so the information collection process was faulty.

\begin{table}[H]
    \centering
    \begin{tabular}{cc|c|c|}
        \cline{3-4}
         &  & Melanoma of the skin & All causes \\ \hline
        \multicolumn{1}{|c|}{} & Missing & 6.02\% & \cellcolor[HTML]{9B9B9B}{\color[HTML]{9B9B9B} } \\ \cline{2-4} 
        \multicolumn{1}{|c|}{} & N0 & 3.30\% & \cellcolor[HTML]{9B9B9B}{\color[HTML]{9B9B9B} } \\ \cline{2-4} 
        \multicolumn{1}{|c|}{} & N1 & 23.59\% & \cellcolor[HTML]{9B9B9B}{\color[HTML]{9B9B9B} } \\ \cline{2-4} 
        \multicolumn{1}{|c|}{} & N2 & 33.11\% & \cellcolor[HTML]{9B9B9B}{\color[HTML]{9B9B9B} } \\ \cline{2-4} 
        \multicolumn{1}{|c|}{\multirow{-5}{*}{M0}} & N3 & 47.55\% & \cellcolor[HTML]{9B9B9B}{\color[HTML]{9B9B9B} } \\ \hline
        \multicolumn{1}{|c|}{} & Missing & \cellcolor[HTML]{9B9B9B} & 86.49\% \\ \cline{2-4} 
        \multicolumn{1}{|c|}{} & N0 & \cellcolor[HTML]{9B9B9B} & 78.99\% \\ \cline{2-4} 
        \multicolumn{1}{|c|}{} & N1 & \cellcolor[HTML]{9B9B9B} & 85.06\% \\ \cline{2-4} 
        \multicolumn{1}{|c|}{} & N2 & \cellcolor[HTML]{9B9B9B} & 74.21\% \\ \cline{2-4} 
        \multicolumn{1}{|c|}{\multirow{-5}{*}{M1}} & N3 & \cellcolor[HTML]{9B9B9B} & 84.38\% \\ \hline
    \end{tabular}
    \caption{Five-year mortality rate by spread to regional lymph nodes}
    \label{tab:qx N}
\end{table}

Looking at the stage of the cancer, it is common knowledge that more advanced stages imply higher mortality rates. This is what we observe here. Non-metastatic cancers can have grades ranging from I to III, and stage IV is defined by the presence of metastases. There are no missing stage information for individuals with stage IV cancers, because we imputed the missing values: remembering table~\ref{tab:stage TNM}, metastatic cancers are stage IV, which is how the information was completed for the variable.

\begin{table}[H]
    \centering
    \begin{tabular}{cc|c|c|}
        \cline{3-4}
         &  & Melanoma of the skin & All causes \\ \hline
        \multicolumn{1}{|c|}{} & Missing & 5.36\% & \cellcolor[HTML]{9B9B9B}{\color[HTML]{9B9B9B} } \\ \cline{2-4} 
        \multicolumn{1}{|c|}{} & I & 1.44\% & \cellcolor[HTML]{9B9B9B}{\color[HTML]{9B9B9B} } \\ \cline{2-4} 
        \multicolumn{1}{|c|}{} & II & 14.79\% & \cellcolor[HTML]{9B9B9B}{\color[HTML]{9B9B9B} } \\ \cline{2-4} 
        \multicolumn{1}{|c|}{\multirow{-4}{*}{M0}} & III & 29.87\% & \cellcolor[HTML]{9B9B9B}{\color[HTML]{9B9B9B} } \\ \hline
        \multicolumn{1}{|c|}{M1} & IV & \cellcolor[HTML]{9B9B9B} & 83.00\% \\ \hline
    \end{tabular}
    \caption{Five-year mortality rate by stage}
    \label{tab:qx stage}
\end{table}

Medically speaking, skin cancer behaves the exact same way for both genders. Mortality rates for the general population in the US are higher for men than women, which can be explained by numerous factors such as behavior differences. In insurance, gender is a sensitive variable and risk selection or pricing cannot discriminate by gender, so the same mortality rates have to be used for both men and women. In the case of melanoma of the skin, mortality rates are also higher for men. In the case of non-metastatic cancers, the five-year mortality rate for men is 2.56 points higher than for women and for metastatic cancers, it is 2.83 points higher. \par

\begin{table}[H]
    \centering
    \begin{tabular}{cc|c|c|}
        \cline{3-4}
         &  & Melanoma of the skin & All causes \\ \hline
        \multicolumn{1}{|c|}{} & Women & 3.82\% & \cellcolor[HTML]{9B9B9B} \\ \cline{2-4} 
        \multicolumn{1}{|c|}{\multirow{-2}{*}{M0}} & Men & 6.38\% & \cellcolor[HTML]{9B9B9B} \\ \hline
        \multicolumn{1}{|c|}{} & Women & \cellcolor[HTML]{9B9B9B} & 81.09\% \\ \cline{2-4} 
        \multicolumn{1}{|c|}{\multirow{-2}{*}{M1}} & Men & \cellcolor[HTML]{9B9B9B} & 83.92\% \\ \hline
    \end{tabular}
    \caption{Five-year mortality rate by gender}
    \label{tab:qx gender}
\end{table}

When looking at the origin variable provided by the SEER database, which classifies individuals into five categories (Hispanic, American Indian/Alaska Native, Asian or Pacific Islander, Black, White), we observe great disparity. The five-year mortality rates vary greatly between the categories, both for non-metastatic and metastatic cancers. As we saw previously, skin cancer is more common in individuals with fair complexions because of the lower melanin production. As a result, individuals with darker complexions are less likely to get this type of cancer and it often goes undetected for a longer time, which explains higher mortality rates. It could also be due to other factors such as socioeconomic status, which is still very related to origin in the US, where this data comes from. Another reason for these gaps is the variance of the estimation, as not all categories contain enough observations to have a robust estimation. It is interesting to notice that individuals with missing information about race/origin are predicted very low five-year mortality rates compared to all classes. This is caused by the low number of such observations, leading to a less robust estimator. \par

\begin{table}[H]
    \centering
    \begin{tabular}{cc|c|c|}
        \cline{3-4}
         &  & Melanoma of the skin & All causes \\ \hline
        \multicolumn{1}{|c|}{} & Hispanic & 7.34\% & \cellcolor[HTML]{9B9B9B} \\ \cline{2-4} 
        \multicolumn{1}{|c|}{} & Missing & 0.07\% & \cellcolor[HTML]{9B9B9B} \\ \cline{2-4} 
        \multicolumn{1}{|c|}{} & American Indian/AK Native & 9.86\% & \cellcolor[HTML]{9B9B9B} \\ \cline{2-4} 
        \multicolumn{1}{|c|}{} & Asian or Pacific Islander & 11.73\% & \cellcolor[HTML]{9B9B9B} \\ \cline{2-4} 
        \multicolumn{1}{|c|}{} & Black & 18.82\% & \cellcolor[HTML]{9B9B9B} \\ \cline{2-4} 
        \multicolumn{1}{|c|}{\multirow{-6}{*}{M0}} & White & 5.21\% & \cellcolor[HTML]{9B9B9B} \\ \hline
        \multicolumn{1}{|c|}{} & Hispanic & \cellcolor[HTML]{9B9B9B} & 86.53\% \\ \cline{2-4} 
        \multicolumn{1}{|c|}{} & Missing & \cellcolor[HTML]{9B9B9B} & 35.71\% \\ \cline{2-4} 
        \multicolumn{1}{|c|}{} & American Indian/AK Native & \cellcolor[HTML]{9B9B9B} & 88.45\% \\ \cline{2-4} 
        \multicolumn{1}{|c|}{} & Asian or Pacific Islander & \cellcolor[HTML]{9B9B9B} & 93.85\% \\ \cline{2-4} 
        \multicolumn{1}{|c|}{} & Black & \cellcolor[HTML]{9B9B9B} & 81.85\% \\ \cline{2-4} 
        \multicolumn{1}{|c|}{\multirow{-6}{*}{M1}} & White & \cellcolor[HTML]{9B9B9B} & 82.71\% \\ \hline
    \end{tabular}
    \caption{Five-year mortality rates by origin}
    \label{tab:qx origin}
\end{table}

Marital status can impact health behavior, leading to differences in mortality rates. In our data, we indeed have different estimated five-year mortality rates for all marital status classes. The individuals with missing information about their marital status have the lowest mortality rates, for both non-metastatic and metastatic cancers. For both metastatic states, divorced and widowed individuals have the highest five-year mortality rates, but these categories also correspond to an older population. \par

\begin{table}[H]
    \centering
    \begin{tabular}{cc|c|c|}
        \cline{3-4}
         &  & Melanoma of the skin & All causes \\ \hline
        \multicolumn{1}{|c|}{} & Divorced & 9.71\% & \cellcolor[HTML]{9B9B9B} \\ \cline{2-4} 
        \multicolumn{1}{|c|}{} & Married & 5.86\% & \cellcolor[HTML]{9B9B9B} \\ \cline{2-4} 
        \multicolumn{1}{|c|}{} & Missing & 1.75\% & \cellcolor[HTML]{9B9B9B} \\ \cline{2-4} 
        \multicolumn{1}{|c|}{} & Separated & 8.56\% & \cellcolor[HTML]{9B9B9B} \\ \cline{2-4} 
        \multicolumn{1}{|c|}{} & Single & 6.27\% & \cellcolor[HTML]{9B9B9B} \\ \cline{2-4} 
        \multicolumn{1}{|c|}{} & Unmarried & 7.71\% & \cellcolor[HTML]{9B9B9B} \\ \cline{2-4} 
        \multicolumn{1}{|c|}{\multirow{-7}{*}{M0}} & Widowed & 11.17\% & \cellcolor[HTML]{9B9B9B} \\ \hline
        \multicolumn{1}{|c|}{} & Divorced & \cellcolor[HTML]{9B9B9B} & 85.15\% \\ \cline{2-4} 
        \multicolumn{1}{|c|}{} & Married & \cellcolor[HTML]{9B9B9B} & 81.61\% \\ \cline{2-4} 
        \multicolumn{1}{|c|}{} & Missing & \cellcolor[HTML]{9B9B9B} & 77.78\% \\ \cline{2-4} 
        \multicolumn{1}{|c|}{} & Separated & \cellcolor[HTML]{9B9B9B} & 80.51\% \\ \cline{2-4} 
        \multicolumn{1}{|c|}{} & Single & \cellcolor[HTML]{9B9B9B} & 82.94\% \\ \cline{2-4} 
        \multicolumn{1}{|c|}{} & Unmarried & \cellcolor[HTML]{9B9B9B} & 74.38\% \\ \cline{2-4} 
        \multicolumn{1}{|c|}{\multirow{-7}{*}{M1}} & Widowed & \cellcolor[HTML]{9B9B9B} & 90.75\% \\ \hline
    \end{tabular}
    \caption{Five-year mortality rates by marital status}
    \label{tab:qx mar stat}
\end{table}

% Figure environment removed

\subsubsection{A different data structure needed to use standard models: pseudo table}

\paragraph{} As mentioned previously, time discretization and data structure modification allow the use of standard Machine Learning models to predict the number of death in the desired time interval. This data structure modification is done by creating pseudo tables: for each individual, we create as many rows as the maximal duration. We consider times in years rather than in months, as it seems granular enough for our modeling purposes. In each row, we will have a death variable indicating if the individual died during that time interval and the initial exposure, which will then be used as a weight or an offset when applying the standard models to predict the number of deaths. Another advantage of the approach is that we can have time-varying information about an individual, such as the evolution of his marital status or of the size of his tumor. Unfortunately, in the SEER database, we do not have access to this kind of information, so we can question the relevance of using such variables. \par
The step-by-step process is as follows. We have for each individual i his age at diagnosis, year of diagnosis, survival time in months and a variable indicating if he died because of melanoma of the skin. The first step is to compute the survival time in years and the maximal duration, which is the ceiling of the survival time in years. Then, we create as many rows for each individual i as his/her maximal duration. We can then compute for each of the lines j the duration since diagnosis in years, and the age, year and death due to melanoma in that time interval. Finally, we can compute the individual exposure are we saw in the previous section. For each time interval j: $[\tau_j,\tau_j+1]$ which corresponds to a year, we compute the variables of interest. The duration since diagnosis (in years) is
\begin{equation*}
    \mbox{Duration}_{j}=j-1
\end{equation*}
the death variable (due to melanoma, in the time interval k) is
\begin{equation*}
    d_{i,j}=\left\{
    \begin{array}{ll}
        1 & \mbox{if Death\_melanoma}=1 \mbox{ and } j=\mbox{Max\_duration} \\
        0 & \mbox{ else}
    \end{array}
    \right.
\end{equation*}
the individual initial exposure is
\begin{equation*}
    e_{i,j}=\left\{
    \begin{array}{ll}
        1 & \mbox{if } Y_{i,j}=1 \mbox{ or } j<\mbox{Max\_duration} \\
        S_i-\mbox{Duration}_{j} & \mbox{ else}
    \end{array}
    \right.
\end{equation*}

We will create the pseudo table step-by-step, starting from the raw data from table~\ref{tab:raw survival}:
\begin{itemize}
    \item Individual i=1: \begin{itemize}
        \item For the survival time in years, we just convert the number of months into a number of years: $S_1=25/12=2.08$0.
        \item For the maximal duration in years, we take the ceiling of the survival time in years: $\mbox{Max\_duration}=\lceil S_1 \rceil = \lceil 2.08 \rceil = 3$0.
        \item So we create 3 lines $j=1,2,3$ for this individual: \begin{itemize}
            \item j=1: $\mbox{Duration}_{1}=j-1=0$ \\
            Age=Age\_dx=25 \\
            Year=Year\_dx=2002 \\
            Death\_melanoma $d_{1,1}=0$ because $\mbox{Death\_melanoma}\neq1$ \\
            Exposure $e_{1,1}=1$ because $j=1<3=\mbox{Max\_duration}$
            \item j=2: $\mbox{Duration}_{2}=j-1=1$ \\
            Age=Age\_dx+1=26 \\
            Year=Year\_dx+1=2003 \\
            Death\_melanoma $d_{1,2}=0$ because $\mbox{Death\_melanoma}\neq1$ \\
            Exposure $e_{1,2}=1$ because $j=2<3=\mbox{Max\_duration}$
            \item j=3: $\mbox{Duration}_{3}=j-1=2$ \\
            Age=Age\_dx+2=27 \\
            Year=Year\_dx+2=2004 \\
            Death\_melanoma $d_{1,3}=0$ because $\mbox{Death\_melanoma}\neq1$ \\
            Exposure $e_{1,3}=S_1-D_{1,3}=2.08-2=0.08$ because $Y_{1,3}\neq 1$ and $j=3=\mbox{Max\_duration}$
        \end{itemize}
    \end{itemize}
    \item Individual i=2: \begin{itemize}
        \item $S_2=4/12=0.33$0.
        \item $\mbox{Max\_duration}=\lceil S_2 \rceil = \lceil 0.33 \rceil = 1$0.
        \item So we create 1 line $j=1$ for this individual: \begin{itemize}
            \item j=1: $\mbox{Duration}_{1}=j-1=0$ \\
            Age=Age\_dx=37 \\
            Year=Year\_dx=2004 \\
            Death\_melanoma $d_{2,1}=1$ because $\mbox{Death\_melanoma}=1$ and $j=1=\mbox{Max\_duration}$\\
            Exposure $e_{2,1}=1$ because $Y_{2,1}=1$
        \end{itemize}
    \end{itemize}
    \item Individual i=3: \begin{itemize}
        \item $S_3=58/12=4.83$0.
        \item $\mbox{Max\_duration}=\lceil S_3 \rceil = \lceil 4.83 \rceil = 5$0.
        \item So we create 5 lines $j=1,2,3,4,5$ for this individual: \begin{itemize}
            \item j=1: $\mbox{Duration}_{1}=j-1=0$ \\
            Age=Age\_dx=56 \\
            Year=Year\_dx=2010 \\
            Death\_melanoma $d_{3,1}=0$ because $j=1\neq5=\mbox{Max\_duration}$ \\
            Exposure $e_{3,1}=1$ because $j=1<5=\mbox{Max\_duration}$
            \item j=2: $\mbox{Duration}_{2}=j-1=1$ \\
            Age=Age\_dx+1=57 \\
            Year=Year\_dx+1=2011 \\
            Death\_melanoma $d_{3,2}=0$ because $j=2\neq5=\mbox{Max\_duration}$ \\
            Exposure $e_{3,2}=1$ because $j=2<5=\mbox{Max\_duration}$
            \item j=3: $\mbox{Duration}_{3}=j-1=2$ \\
            Age=Age\_dx+2=58 \\
            Year=Year\_dx+2=2012 \\
            Death\_melanoma $d_{3,3}=0$ because $j=3\neq5=\mbox{Max\_duration}$ \\
            Exposure $e_{3,3}=1$ because $j=3<5=\mbox{Max\_duration}$
            \item j=4: $\mbox{Duration}_{4}=j-1=3$ \\
            Age=Age\_dx+3=59 \\
            Year=Year\_dx+3=2013 \\
            Death\_melanoma $d_{3,4}=0$ because $j=4\neq5=\mbox{Max\_duration}$ \\
            Exposure $e_{3,4}=1$ because $j=4<5=\mbox{Max\_duration}$
            \item j=5: $\mbox{Duration}_{5}=j-1=4$ \\
            Age=Age\_dx+4=60 \\
            Year=Year\_dx+4=2014 \\
            Death\_melanoma $d_{3,5}=1$ because $\mbox{Death\_melanoma}=1$ and $j=5=\mbox{Max\_duration}$ \\
            Exposure $e_{3,5}=1$ because $Y_{3,5}=1$
        \end{itemize}
    \end{itemize}
\end{itemize}
In the end, we get table~\ref{tab:pseudo survival}.

\begin{table}[H]
    \centering
    \begin{tabular}{|c|c|c|c|c|}
        \hline
        i & Age\_dx & Year\_dx & Survival (months) & Death\_melanoma \\ \hline
        1 & 25 & 2002 & 25 & 0 \\ \hline
        2 & 37 & 2004 & 4 & 1 \\ \hline
        3 & 56 & 2010 & 58 & 1 \\ \hline
    \end{tabular}
    \caption{Raw survival data}
    \label{tab:raw survival}
\end{table}

\begin{table}[H]
    \centering
    \makebox[\linewidth]{
    \begin{tabularx}{17,77cm}{|c|c|c|c|c|c|c|c|c|}
    % \begin{tabular}{|c|c|c|c|c|c|c|c|c|}
        \hline
        i & \begin{tabular}[c]{@{}c@{}}Survival \\ (years)\\$S_j$\end{tabular} & \begin{tabular}[c]{@{}c@{}}Max\_duration\\ (years)\end{tabular} & j & \begin{tabular}[c]{@{}c@{}}$\mbox{Duration}_{j}$\\ (years)\end{tabular} & Age & Year & \begin{tabular}[c]{@{}c@{}}Death\_melanoma\_j\\$d_{i,j}$\end{tabular} & \begin{tabular}[c]{@{}c@{}}Exposure\\ $e_{i,j}$\end{tabular} \\ \hline
        \multirow{3}{*}{1} & \multirow{3}{*}{25/12=2.08} & \multirow{3}{*}{3} & 1 & 0 & 25 & 2002 & 0 & 1 \\ \cline{4-9} 
         &  &  & 2 & 1 & 26 & 2003 & 0 & 1 \\ \cline{4-9} 
         &  &  & 3 & 2 & 27 & 2004 & 0 & 0.08 \\ \hline
        2 & 4/12=0.33 & 1 & 1 & 0 & 37 & 2004 & 1 & 1 \\ \hline
        \multirow{5}{*}{3} & \multirow{5}{*}{58/12=4.83} & \multirow{5}{*}{5} & 1 & 0 & 56 & 2010 & 0 & 1 \\ \cline{4-9} 
         &  &  & 2 & 1 & 57 & 2011 & 0 & 1 \\ \cline{4-9} 
         &  &  & 3 & 2 & 58 & 2012 & 0 & 1 \\ \cline{4-9} 
         &  &  & 4 & 3 & 59 & 2013 & 0 & 1 \\ \cline{4-9} 
         &  &  & 5 & 4 & 60 & 2014 & 1 & 1 \\ \hline
    % \end{tabular}
    \end{tabularx}
    }
    \caption{Pseudo table created from the raw survival data}
    \label{tab:pseudo survival}
\end{table}

% Figure environment removed

\subsection{Descriptive statistics}

\paragraph{} We will perform the descriptive statistics analysis on the pseudo table. As mentioned before, our study is restricted to non-metastatic cancers.

\subsubsection{Variable identification}

\paragraph{} The 28 variables available after mapping and applying underwriting and medical constraints are described in appendix~\ref{appendix:tab:variable description discretized}. They can be categorized into 3 groups: the explanatory variables, the variable of interest and the weight variable. Within the explanatory variables category, we have non-sensitive and sensitive variables. \par

\paragraph{Non-sensitive explanatory variables} They are the ones that are pure descriptors of the medical situation, for example the age, the type of cancer (\texttt{AYA\_site\_recode20}), the time since diagnosis (\texttt{DURATION}) or how far the cancer has spread (\texttt{Extent}). Some of them are continuous and others categorical. \par

\paragraph{Sensitive explanatory variables} They are the ones that can introduce discrimination. They directly concern the individual: sex, origin and marital status. \par
Table~\ref{tab:obs sex} gives the number of observations by sex. The group imbalance ratio is
\begin{equation*}
    \mbox{IR}_{Sex}=\frac{287,820}{249,118}=1.16
\end{equation*}
which is close to 1.

\begin{table}[H]
    \centering
    \begin{tabular}{|c|c|}
        \hline
        Sex & Observations \\ \hline
        Female & 249,118 \\ \hline
        Male & 287,820 \\ \hline
    \end{tabular}
    \caption{Number of observations by sex}
    \label{tab:obs sex}
\end{table}

Table~\ref{tab:obs origin} gives the number of observations by origin. The majority population is the `White' category, with almost 45 times more observations than the second biggest category, `Missing'. \par

\begin{table}[H]
    \centering
    \begin{tabular}{|c|c|}
        \hline
        Origin & Observations \\ \hline
        Hispanic & 11,034 \\ \hline
        Missing & 11,510 \\ \hline
        American Indian/AK Native & 1,157 \\ \hline
        Asian or Pacific Islander & 3,597 \\ \hline
        Black & 1,767 \\ \hline
        White & 507,873 \\ \hline
    \end{tabular}
    \caption{Number of observations by origin}
    \label{tab:obs origin}
\end{table}

Table~\ref{tab:obs mar stat} gives the number of observations by marital status. The majority class is `Married' and the second largest is `Missing'. \par

\begin{table}[H]
    \centering
    \begin{tabular}{|c|c|}
        \hline
        Marital status & Observations \\ \hline
        Divorced & 26,160 \\ \hline
        Married & 272,743 \\ \hline
        Missing & 159,649 \\ \hline
        Separated & 1,946 \\ \hline
        Single & 63,729 \\ \hline
        Unmarried & 674 \\ \hline
        Widowed & 12,037 \\ \hline
    \end{tabular}
    \caption{Number of observations by marital status}
    \label{tab:obs mar stat}
\end{table}

\paragraph{Variable of interest} It is the variable that we will predict with our model: the event of interest is the death due to skin melanoma of the individual in the year (\texttt{Death\_skin}). Table~\ref{tab:obs death skin} gives the number of observations for each value of \texttt{Death\_skin}. There is a great imbalance between classes: the group imbalance ratio is
\begin{equation*}
    \mbox{IR}_{Death\_skin}=\frac{533,167}{3,771}=141.39
\end{equation*}
meaning there are 141.39 more observations for \texttt{Death\_skin}=0 than for \texttt{Death\_skin}=1. \par

\begin{table}[H]
    \centering
    \begin{tabular}{|c|c|}
        \hline
        Death\_skin & Observations \\ \hline
        0 & 533,167 \\ \hline
        1 & 3,771 \\ \hline
    \end{tabular}
    \caption{Number of observations by death\_skin}
    \label{tab:obs death skin}
\end{table}

\paragraph{The weight variable} As mentioned previously, because of the characteristics of survival data, we need to take into account the exposure of individuals to predict mortality. The exposure variable will therefore be used as a weight in the model. \par

\subsubsection{Multivariate analysis}

\paragraph{} We will now study the relationships between variables. For that, we will focus on correlations. As we have already mentioned multiple times before, absence of correlation is not equivalent to independence. \par
As we have many variables and had to turn categorical features into binary data, we end up with a large correlation matrix. The keys points are:
\begin{itemize}
    \item The bright blue colors are negative correlations between different categories of the same variables, for example between \texttt{Sex\_Female} and \texttt{Sex\_Male}.
    \item Generally speaking, there are correlations between \texttt{Tumor}, \texttt{Positive\_Node}, \texttt{Stage}, \texttt{Extent}, \texttt{Ulceration}, \texttt{Mitotic rate}, and \texttt{Surg\_LN/oth/primsite}, which are linked with each other as they are the medical variables that allow the diagnosis.
    \item Time variables are naturally correlated with each other: age with age at diagnosis, year with year of diagnosis and duration.
    \item There are a few light correlations with the variable of interest, all less than 0.13 (in absolute value). 
    \item Concerning the missing values, the dummy variables for Missing values of \texttt{Tumor}, \texttt{Stage} and \texttt{Positive\_Nodes} are strongly positively correlated with each other. It seems coherent, as the stage is defined thanks to the TNM system. Missing categories of other variables are correlated with these variables and with each other: \texttt{Surg\_primsite}, \texttt{Surg\_LN}, \texttt{Surg\_oth}, \texttt{Extent}, \texttt{Ulceration}, and these variables also describe the gravity of the cancer. A Missing origin or marital status is lightly positively correlated with the previously mentioned variables. We can conclude that when there are Missing values for an individual, they are for all or most of the mentioned categories.
    % As we saw with the 5-year mortality rates, individuals with Missing values for Origin and Marital Status have lower mortality rates than other classes. The reason for these values to be missing might be that the collection of the information was not done properly because they were not seen as important to the study because of their low risk.
\end{itemize}
With this study of correlations, we can conclude that under the modeling constraints, not all variables can be kept.
% \begin{itemize}
%     \item For all categorical variables, we will use dummies like in the correlation matrix, drop the category with the most observations and add a constant to find the intercept of the regression. This will remove some of the strong correlations between variables and improve model performance. It will also allow the comparison of coefficients: the coefficient attributed to a category will indicate how it influences the prediction compared to the majority class.
%     \item As the year corresponding to the time interval and the year of diagnosis are strongly correlated with each other, we will only keep one them. The same goes for the age. Discussions with an oncologist led to the decision to only keep the current age and year.
%     \item As the stage is a summary of the TNM system variables, and as we saw they were strongly correlated with each other, we will not keep the Stage variable.
% \end{itemize}
% With the oncologist, we also learned that the Laterality variable was not relevant for the mortality of patients diagnosed with melanoma of the skin, so we will not keep this variable for the model.


% Figure environment removed

% Figure environment removed

\newpage
