\documentclass[prl,superscriptaddress,onecolumn,notitlepage]{revtex4-2}
\usepackage[latin9]{inputenc}
\usepackage{color}
\usepackage{amsmath}
\usepackage{amssymb}
\usepackage{graphicx}
\usepackage[unicode=true,
 bookmarks=true,bookmarksnumbered=false,bookmarksopen=false,
 breaklinks=false,pdfborder={0 0 1},backref=false,colorlinks=true]
 {hyperref}
\hypersetup{
 linkcolor=magenta,urlcolor=blue,citecolor=blue,pdfstartview={FitH},urlcolor=blue}
\setcounter{secnumdepth}{3}
\makeatletter

\usepackage{amsmath}
\usepackage{amssymb}
\usepackage{graphicx}
\usepackage{graphbox}

\usepackage{amsfonts}
\usepackage{tabularx}
\usepackage{dcolumn}
\usepackage{bm}
\usepackage{epstopdf}
\usepackage{times}

%\usepackage{mathbbold}ss
\usepackage{mathrsfs}
%\documentclass[aps,preprint,showpacs,superscriptaddress,endfloats]{revtex4}
\usepackage{epsfig}
\usepackage{graphicx}
\usepackage{amsfonts}
\usepackage[figuresright]{rotating}
\usepackage{amssymb}
\usepackage{amsmath}
\usepackage{dcolumn}
\usepackage{bm}
\usepackage{color}
\usepackage{MnSymbol}
\usepackage{verbatim}
\usepackage{bm}
%\usepackage{psfrag}
%\renewcommand{\baselinestretch}{2.0}
\usepackage{color}
\usepackage{amssymb}
\usepackage{bm}
\usepackage{braket}
\usepackage{mathtools}
\def\sign{\text{sign}}
\def\Tr{\text{Tr}}
\def\gtd{\tilde{g}}
\def\ve{\varepsilon}
\newcommand{\spinup}{\uparrow}
\newcommand{\spindown}{\downarrow}
\newcommand{\up}{\uparrow}
\newcommand{\down}{\downarrow}
\newcommand{\mr}{moir\'{e} }
\newcommand{\BZ}{Brillouin zone }
\newcommand{\tr}{\text{Tr}}
\newcommand{\Blue}[1]{{\color{blue}{#1}}}
\newcommand{\red}[1]{{\color{red}{#1}}}

\newcommand{\innerproduct}[2]{\langle #1 | #2 \rangle}
\newcommand{\avr}[1]{\langle #1 \rangle}
\newcommand{\RNum}[1]{\uppercase\expandafter{\romannumeral #1\relax}}

\DeclareMathOperator{\sgn}{sgn}

\DeclareMathOperator{\Ree}{Re}
\DeclareMathOperator{\Imm}{Im}

\setcounter{MaxMatrixCols}{10}



\setcounter{figure}{0}
\makeatletter
\renewcommand{\thefigure}{S\@arabic\c@figure}
\setcounter{equation}{0} \makeatletter
\renewcommand \theequation{S\@arabic\c@equation}
\renewcommand \thetable{S\@arabic\c@table}

\begin{comment}
\usepackage{amsmath}
\usepackage{graphicx}
\usepackage{lmodern}
\usepackage{amsmath}
\usepackage[dvipsnames]{xcolor}
\newcommand{\addYW}[1]{\textcolor{Purple}{#1}}
\usepackage{color}
\usepackage{amssymb}
\usepackage{bm}
\usepackage{braket}
\usepackage{mathtools}
\def\sign{\text{sign}}
\def\Tr{\text{Tr}}
\def\gtd{\tilde{g}}
\def\ve{\varepsilon}
\newcommand{\spinup}{\uparrow}
\newcommand{\spindown}{\downarrow}
\newcommand{\up}{\uparrow}
\newcommand{\down}{\downarrow}
\newcommand{\mr}{moir\'{e} }
\newcommand{\BZ}{Brillouin zone }
\newcommand{\tr}{\text{Tr}}
\newcommand{\Blue}[1]{{\color{blue}{#1}}}
\newcommand{\red}[1]{{\color{red}{#1}}}
\DeclareMathOperator{\sgn}{sgn}
\DeclareMathOperator{\Tr}{Tr}
\DeclareMathOperator{\Ree}{Re}
\DeclareMathOperator{\Imm}{Im}

\newcommand{\darkred}[1]{{\color[rgb]{0.6,0,0}{#1}}}
\newcommand{\green}[1]{{\color[rgb]{0,0.6,0}{#1}}}
\newcommand{\blue}[1]{{\color{blue}{#1}}}
\newcommand{\magenta}[1]{{\color[cmyk]{0,.9,0,0.2}{#1}}}

\usepackage[dvipsnames]{xcolor}
\usepackage[colorlinks=true]{hyperref}
\hypersetup{
    colorlinks=true,
    linkcolor=blue,
    citecolor=blue,
    urlcolor=blue
} 
\setcounter{equation}{0}
\setcounter{figure}{0}
\setcounter{table}{0}
%\setcounter{page}{1}
\makeatletter
\renewcommand{\theequation}{S\arabic{equation}}
\renewcommand{\thefigure}{S\arabic{figure}}
\renewcommand{\bibnumfmt}[1]{[S#1]}
\renewcommand{\citenumfont}[1]{S#1}
\makeatletter
\def\maketitle{
\@author@finish
\title@column\titleblock@produce
\suppressfloats[t]}
\makeatother
%\newpage
%\begin{widetext}
%\onecolumngrid
%\vspace{1cm}
%\begin{center}
%{\bf\large Supplemental Materials}
%\end{center}
\end{comment}

\begin{document}

\title {Supplementary Material for ``Mixed-state Topological Order Without Quantum Memory"}
\author{Zijian Wang}
\thanks{These authors contributed equally to this work.}
\affiliation{Institute for Advanced Study, Tsinghua University, Beijing 100084,
People's Republic of China}
\author{Zhengzhi Wu}
\thanks{These authors contributed equally to this work.}
\affiliation{Institute for Advanced Study, Tsinghua University, Beijing 100084,
People's Republic of China}

\author{Zhong Wang}
\email{wangzhongemail@tsinghua.edu.cn}
\affiliation{Institute for Advanced Study, Tsinghua University, Beijing 100084,
People's Republic of China}
\date{\today}

\maketitle

%\newpage
%\begin{widetext}
\onecolumngrid

%\title {Supplementary Material for `Pair density wave and topological superconductivity from twisted homo-bilayer transition metal dichalcogenides'.}
%\maketitle
\setcounter{equation}{0}
\setcounter{figure}{0}
\setcounter{table}{0}
%\setcounter{page}{1}
\makeatletter
\renewcommand{\theequation}{S\arabic{equation}}
\renewcommand{\thefigure}{S\arabic{figure}}
% \newcommand{\addYW}[1]{\textcolor{Purple}{#1}}
\renewcommand{\bibnumfmt}[1]{[S#1]}
\renewcommand{\citenumfont}[1]{S#1}
\section{Toric code model under decoherence}

\subsection{Calculation of the coherent information and mapping to the RBIM}
In this section, we investigate the critical error rate $p_c$, or the accuracy threshold, of the quantum memory. We are interested in the intrinsic threshold, which means if the error rate $p>p_c$ \footnote{For simplicity, we drop the subscript of $p_\varepsilon$ in sections \RNum{1}. A--\RNum{1}. D}, there is no decoding algorithm to recover the encoded quantum information. For this purpose, we calculate of the coherent information $I_c$ \cite{schumacher1996quantum}, which quantifies the amount of information surviving under quantum channel, and has recently been investigated in noisy toric code models with single-qubit errors \cite{fan2023diagnostics}. To define the coherent information of a decohered quantum system $\rho_Q=\mathcal{N}[\rho_{Q,0}]$, we first introduce two reference qubits $R$ to purify $\rho_{Q,0}$ as $
\rho_{Q,0}=\operatorname{Tr}_{R}\left|\Psi^{R Q}\right\rangle\left\langle\Psi^{R Q}\right|
$.  Then the coherent information is defined as:
\begin{equation}
    I_c=S(\rho_{Q})-S(\rho_{RQ}),
\end{equation}
where $S$ is the von Neumann entropy, and $S(\rho_{RQ})=\mathcal{N}\left[|\Psi^{RQ}\rangle\langle\Psi^{RQ}|\right]$. $I_c$ is bounded by $-S(\rho_R)\leq I_c\leq S(\rho_R)$ due to the subadditivity of the von Neumann entropy, and it measures the ability to restore the information encoded in the initial state via error correction. The sufficient and necessary condition for the existence of perfect quantum error correction is that $I_c=S(\rho_{R})$ \cite{schumacher1996quantum}. In the next two subsections, we use the replica trick to calculate the two von Neumann entropy $S(\rho_{Q})$ and $S(\rho_{RQ})$ respectively. The von Neumann entropy is obtained by taking the replica limit: $S=-\Tr(\rho\log{\rho})=-\lim_{n\rightarrow1}\frac{\partial }{\partial n}\Tr(\rho^n)$.


\subsubsection{von Neumann entropy $S(\rho_{RQ})$ }
We begin with the calculation of $S(\rho_{RQ})$. First, the purification of the initial state $\rho_0$ is similar to the construction in \cite{fan2023diagnostics}. We introduce 2 reference qubits, and maximally entangle them with the two logical qubits in the ground state subspace of the system:
% Following Fan et al\cite{fan2023diagnostics}, we use coherent information\cite{schumacher1996quantum} to give a quantitative analysis. We first introduce 2 reference qubits(denoted by $R$), and maximally entangle them with the two logical qubits in the ground state subspace of the system(denoted by Q). That is, the initial state 
\begin{equation}
\rho_{RQ,0}=\frac{1+W^x_{\gamma_y}\sigma_{\gamma_x}^x}{2}\frac{1+W^z_{\gamma_x}\sigma_{\gamma_x}^z}{2}\frac{1+W^x_{\gamma_x}\sigma_{\gamma_y}^x}{2}\frac{1+W^z_{\gamma_y}\sigma_{\gamma_y}^z}{2}\prod_v\frac{1+A_v}{2}\prod_p\frac{1+B_p}{2},
\end{equation}
where $W^{x,z}_{\gamma_{x,y}}$ are non-contractible $X,Z$ loops along cycle $\gamma_{x,y}$, and $\sigma_{\gamma_x},\sigma_{\gamma_y}$ are Pauli operators acting on the two Reference qubits. And the final decohered state is $\rho_{RQ}=\mathcal{N}^\varepsilon[\rho_{RQ,0}]$.

To see why $\rho_{RQ,0}$ is a purification, or in other words: $\rho_{RQ,0}=\ket{\psi_{RQ}}\bra{\psi_{RQ}}$, we begin with the simpler maximally mixed density matrix $\rho_{Q,0}$:
\begin{equation}
    \rho_{Q,0}=\prod_v\frac{1+A_v}{2}\prod_p\frac{1+B_p}{2}=\frac{1}{4}\sum_{a,b=\pm 1}\ket{a,b}\bra{a,b},
\end{equation}
where $a,b$ label the eigenvalues of the two non-contractible Wilson loops $W_{\gamma_x}^z,W_{\gamma_y}^z$ respectively. The four orthogonal ground states $\ket{\pm1,\pm1}$ actually have a $U(1)$ phase ambiguity, and we fix the relative phase such that the three states $\ket{1,-1},\ket{-1,1},\ket{-1,-1}$ can be obtained from $\ket{1,1}$ through the application of non-contractible 't Hooft loops:
\begin{equation}
    \ket{-1,1}=W_{\gamma_x}^x\ket{1,1}, \quad  \ket{1,-1}=W_{\gamma_y}^x\ket{1,1}, \quad \ket{-1,-1}=W_{\gamma_y}^xW_{\gamma_x}^x\ket{1,1}.
\end{equation}

What's more, the pure state $\ket{\psi_{RQ}}$ involves two references qubits $R$, so we introduce two additional $Z_2$ indices after $a$ and $b$ as: $\ket{a,b;c=\pm1,d=\pm1}$, where $c,d=\pm1$ represent the spin up and down in the z-direction of the reference qubits $\sigma_{\gamma_x}$ and $\sigma_{\gamma_y}$ respectively. With all these notations, we can now write down the pure state $\ket{\psi_{RQ}}$ explicitly as : $ \ket{\psi_{RQ}}= \frac{1}{2}\sum_{a,b=\pm 1}\ket{a,b;a,b}$.



To facilitate the calculation of the coherent information, we write the decohered $\rho_{RQ}$ in the error chain representation: %and for simplicity, we take $p_x=p_y=p$ in the quantum channel: 
\begin{equation}
\rho_{RQ}=\sum_{C}P(C)w^\varepsilon_C\rho_{RQ,0}
w^\varepsilon_C
\label{eq:errorstring}
\end{equation}
where $C$ stands for error chain configurations with total length $|C|$.  $P(C)=p^{|C|}(1-p)^{N-|C|}$ is  the occurrence probability of the error chain $C$. $w^\varepsilon_C$ is the (product of) open string operators which  create $\varepsilon$ anyons at the ends of the $C$. See Fig. 1 (d) ) in the main text for an illustration. 

Now the trace $\Tr(\rho_{RQ}^n)$ is:
\begin{equation}
\begin{aligned} 
\Tr(\rho_{RQ}^n)&= \sum_{\left\{C^{(s)}\right\}} \prod_{s=1}^{n} P\left(C^{(s)}\right) \operatorname{tr}\left[\prod_{s=1}^{n} \left(w^\varepsilon_{C^{(s)}} \rho_{RQ,0}w^\varepsilon_{C^{(s)}}\right) \right],\\
&=\sum_{\left\{C^{(s)}\right\}}\prod_{s=1}^{n} P\left(C^{(s)}\right)\left\langle\Psi_{0}\left|w^\varepsilon_{C^{(s)}} w^\varepsilon_{C^{(s+1)}}\right| \Psi_{0}\right\rangle,
\end{aligned}
\end{equation}
where $w^\varepsilon_{C^{(n+1)}}\equiv w^\varepsilon_{C^{(1)}}$ and the loops $C^{(s)}$ satisfy:
\begin{equation}
C^{(s+1)}=C^{(1)}+\partial v^{(s)}, \quad s=1,2, \ldots, n-1
\end{equation}
to give nonzero contribution. $\partial v^{(s)}$ are boundaries of a set of plaquettes $v^{(s)}$, so they are homologically trivial loops. Then $\Tr(\rho_{RQ}^n)$ can be further simplified as:
\begin{equation}
\Tr(\rho_{RQ}^n)=\frac{1}{2^{n-1}}\sum_{C^{(1)}} P\left(C^{(1)}\right) \sum_{\left\{v^{(s)}\right\}} \prod_{s=1}^{n-1} P\left(C^{(1)}+\partial v^{(s)}\right)
\end{equation}
The prefactor $\frac{1}{2^{n-1}}$ is due to the fact that for each replica $s=1,2,\cdots n-1$, there are two plaquette sets $v^{(s)}$ giving the same boundary $\partial v^{(s)}$. $\Tr(\rho_{RQ}^n)$ can be mapped to the partition function of a classical Ising model with $n-1$ flavors of Ising spin and a defect line at $C^1$. Concretely, we introduce $Z_2$ variables $n_{v^{(s)}}(l)=1,0$ to denote whether link $l$ is occupied in the error loop $\partial v^{(s)}$ or not. Then we can express the probability $P\left(C^{(1)}+\partial v^{(s)}\right)$  by the $Z_2$ variables $n_{v^{(s)}}(l)$. For example, if a link $l\in C^{(1)}$ and $n_{v^{(s)}}(l)=1$, then link $l$ does not occur in the error chain $C^{(1)}+\partial v^{(s)}$ and contributes a factor $(1-p)^{n_{v^{(s)}}(l)}p^{1-n_{v^{(s)}}(l)}$ in $P\left(C^{(1)}+\partial v^{(s)}\right)$.  As a result, the probability $P\left(C^{(1)}+\partial v^{(s)}\right)$  can be written as:
\begin{equation}
\begin{aligned}
    &P\left(C^{(1)}+\partial v^{(s)}\right)=\left[\Pi_{l\in C^{(1)}}\left( (1-p)^{n_{v^{(s)}}(l)} p^{1-n_{v^{(s)}}(l)}\right)\right]\left[\Pi_{l\notin C^{(1)}}\left( p^{n_{v^{(s)}}(l)} (1-p)^{1-n_{v^{(s)}}(l)}\right)\right].
\end{aligned}
\end{equation}
The first part with those links belonging to the error chain $C^{(1)}$ can be made symmetric as:
\begin{equation}
\begin{aligned}
    \Pi_{l\in C^{(1)}}\left( (1-p)^{n_{v^{(s)}}(l)} p^{1-n_{v^{(s)}}(l)}\right)&=\Pi_{l\in C^{(1)}}\left(\sqrt{p(1-p)}(\frac{1-p}{p})^{n_{v^{(s)}}(l)-\frac{1}{2}}\right)\\
    &=\sqrt{p(1-p)}^{|C^{(1)}|}\Pi_{l\in C^{(1)}}(\frac{1-p}{p})^{n_{v^{(s)}}(l)-\frac{1}{2}}
    \end{aligned}
\end{equation}
Similarly, we also make the second part symmetric as:
\begin{equation}
\begin{aligned}
    \Pi_{l\notin C^{(1)}}\left( p^{n_{v^{(s)}}(l)} (1-p)^{1-n_{v^{(s)}}(l)}\right)&=\Pi_{l\notin C^{(1)}}\left(\sqrt{(1-p)p}(\frac{p}{1-p})^{n_{v^{(s)}}(l)-\frac{1}{2}}\right)\\
    &=\sqrt{p(1-p)}^{(N-|C^{(1)}|)}\Pi_{l\notin C^{(1)}}(\frac{p}{1-p})^{n_{v^{(s)}}(l)-\frac{1}{2}}
\end{aligned}
\end{equation}
Then we can express the link probability part $(\frac{p}{1-p})^{n_{v^{(s)}}(l)-\frac{1}{2}}$ or $(\frac{1-p}{p})^{n_{v^{(s)}}(l)-\frac{1}{2}}$ as an Ising coupling between two nearest-neighbour plaquettes which share the link $l$. Concretely, we introduce $n-1$ flavours of Ising spins $\tau^{(s)}=\pm1,s=1,2,...n-1$ on each plaquette, and introduce the Ising coupling constant $J$ as: $e^{-2 J}=p /(1-p)$. Then the link probability part $(\frac{p}{1-p})^{n_{v^{(s)}}(l)-\frac{1}{2}}$ or $(\frac{1-p}{p})^{n_{v^{(s)}}(l)-\frac{1}{2}}$ can be written as $\exp[J\eta_{ij}\tau_i^{(s)}\tau_j^{(s)}]$, where $i,j$ are the dual lattice site coordinates of the two plaquettes, and $\eta_{ij}=-1\ (1)$ for $l$ belonging to (not belonging to) the error chain $C^{(1)}$. Then $p$ is the probability of antiferromagnetic coupling  for each bond.

As a result, $\Tr(\rho_{RQ}^n)$ can be expressed as the partition function of a random bond Ising model (RBIM) with $n-1$ flavours of Ising spins and periodic boundary condition (PBC). 
\begin{equation}
\begin{aligned}
    \Tr(\rho_{RQ}^n)&=\frac{1}{2^{n-1}}\sum_{C^{(1)}} P\left(C^{(1)}\right) \sum_{\left\{v^{(s)}\right\}} \prod_{s=1}^{n-1} P\left(C^{(1)}+\partial v^{(s)}\right)\\
    &=\frac{1}{2^{n-1}}\left(\sqrt{(1-p)p}\right)^{(n-1)N}\sum_{C^{(1)}} P\left(\{\eta\}\right) \sum_{\left\{\tau^{(s)}\right\}} \prod_{s=1}^{n-1} \exp[J\eta_{ij}\tau_i^{(s)}\tau_j^{(s)}]\\
    &=\frac{1}{2^{n-1}}\left(\sqrt{(1-p)p}\right)^{(n-1)N}\sum_{C^{(1)}} P\left(\{\eta\}\right)\prod_{s=1}^{n-1}\sum_{\left\{\tau^{s}\right\}}\exp[J\eta_{ij}\tau_i^{(s)}\tau_j^{(s)}]\\
    &=\frac{1}{2^{n-1}}\left(\sqrt{(1-p)p}\right)^{(n-1)N}\sum_{C^{(1)}} P\left(\{\eta\}\right)\left(Z[J,\{\eta\}]\right)^{n-1}.
\end{aligned}
\end{equation}
Finally we take the replica limit $n\rightarrow 1$ to derive the von Neumann entropy $S(\rho_{RQ})$:
\begin{equation}
\begin{aligned}
   S(\rho_{RQ})&=- \lim_{n\rightarrow1}\frac{\partial }{\partial n}\Tr(\rho_{RQ}^n)\\
   &=-\frac{N}{2}\log [p(1-p)]+\log 2-\sum_{\{\eta_l\}}P(\{\eta\})\log Z[J,\{\eta\}]\\
   &\equiv -\overline{\log Z^{\text{RBIM}}_{\text{PBC}}}+\log2-\frac{N}{2}\log [p(1-p)] ,
\end{aligned}
\end{equation}
where the first term is the average free energy of the RBIM along the Nishimori line: $e^{-2J}=\frac{p}{1-p}$.
\subsubsection{von Neumann entropy $S(\rho_{Q})$ }
The von Neumann entropy $S(\rho_{Q})$ can be derived similar to $S(\rho_{RQ})$. The inital density matrix is :
\begin{equation}
    \rho_{Q,0}=\frac{1}{4}\sum_{a,b=\pm 1}\ket{a,b}\bra{a,b}.
\end{equation}
Then $ \Tr(\rho_{Q}^n)$ is:
\begin{equation}
\begin{aligned} 
\Tr(\rho_{Q}^n)&=\sum_{\left\{C^{(s)}\right\}}\sum_{a^{(s)},b^{(s)}}\prod_{s=1}^{n} P\left(C^{(s)}\right)\left(\frac{1}{4}\bra{a^{(s)},b^{(s)}}w^\varepsilon_{C^{(s)}} w^\varepsilon_{C^{(s+1)}} \ket{a^{(s+1)},b^{(s+1)}}\right)\\
&=\frac{1}{2^{n-1}}\cdot\frac{1}{4^{n-1}}\sum_{C^{(1)}} P\left(C^{(1)}\right) \prod_{s=1}^{n-1} \sum_{\left\{v^{(s)}\right\}}\sum_{d_x^{(s)},d_y^{(s)}=0,1} P\left(C^{(1)}+\partial v^{(s)}+d_x^{(s)}\gamma_x+d_y^{(s)}\gamma_y\right),
\end{aligned}
\label{zhenji2}
\end{equation}
where $n+1\equiv 1$, and $d^{(s)}_{x,y}=0,1$ denotes whether or not $C^{(s)}$ lies in the same homological class as $C^{(1)}$. 
\begin{comment}
Recalling that our convention of the four ground states $\ket{\pm1,\pm1}$ is:
\begin{equation}
    \ket{-1,1}=W_{\gamma_x}^{\varepsilon}\ket{1,1}, \quad  \ket{1,-1}=W_{\gamma_y}^{\varepsilon}\ket{1,1}, \quad \ket{-1,-1}=W_{\gamma_y}^{\varepsilon}W_{\gamma_x}^{\varepsilon}\ket{1,1}.
\end{equation}
Then $ \Tr(\rho_{Q}^n)$ can be rewritten into the error chain probability as:
\begin{equation}
\begin{aligned}
\Tr(\rho_{Q}^n)&=(\frac{1}{4})^{(n-1)}\sum_{C^{(1)}} P\left(C^{(1)}\right) \prod_{s=1}^{n-1} \sum_{\left\{v^{(s)}\right\}}\sum_{d_x^{(s)},d_y^{(s)}=0,1} P\left(C^{(1)}+\partial v^{(s)}+\gamma_x^{d_x^{(s)}}+\gamma_y^{d_y^{(s)}}\right),
\end{aligned}
\label{zhenji2}
\end{equation}

where $\gamma_x(y)^{d_{x(y),s}}$ denotes whether or not we insert an non-contractible loop in the $x$ $(y)$ direction in the error chain when $d_{x(y),s}=1,0$. Compared with \eqref{zhenji1}, the extra overall factor of 4 in \eqref{zhenji2} comes from the trace over the four ground states $\ket{\pm1,\pm1}$ in \eqref{zhenji1}; equivalently speaking, there are $n-1$ free summation over $\{d_x^{(s)},d_y^{(s)}\}$ in \eqref{zhenji2}, but $n$ free summation over $\{a^{(s)},b^{(s)}\}$ in \eqref{zhenji1}.
\end{comment}
Similar to the mapping of $\Tr(\rho_{Q}^n)$ to the partition function of RBIM, we can also map $\Tr(\rho_{Q}^n)$ to the  partition function of RBIM, except that here we must sum over the four contributions of inserting or not the two non-contractible defect lines on the torus:
\begin{equation}
\begin{aligned}
\Tr(\rho_{Q}^n)&=\frac{1}{2^{n-1}}\cdot\frac{1}{4^{n-1}}\left(\sqrt{(1-p)p}\right)^{(n-1)N}\sum_{C^{(1)}} P\left(\{\eta\}\right)\left(\sum_{d_{x},d_{y}=0,1} Z_{d_x,d_y}[J,\{\eta\}]\right)^{n-1}\\
&\equiv \frac{1}{2^{n-1}}\cdot\frac{1}{4^{n-1}}\left(\sqrt{(1-p)p}\right)^{(n-1)N}\overline{\left(\sum_{d_x,d_y=0,1}Z^{\text{RBIM}}_{d_x,d_y}\right)^{n-1}}
\end{aligned}
\label{eq:RBIM2}
\end{equation}
where $Z^{\text{RBIM}}_{d_x,d_y}$ is the partition function with $d_a$ non-contractible defect lines inserted along the cycle $\gamma_a$. Along the defect line the coupling changes from $\eta J$ to $-\eta J$. This is equivalent to taking the anti-periodic boundary condition (APBC). 

$S_{\rho_Q}$ can in turn be obtained by taking the replica limit:
\begin{equation}
S(\rho_Q)=- \lim_{n\rightarrow1}\frac{\partial }{\partial n}\Tr(\rho_{Q}^n)=3\log 2-\overline{ \log \left[\sum_{d_x,d_y=0,1 }Z^{\text{RBIM}}_{d_x,d_y}\right]     }-\frac{N}{2}\log[p(1-p)].
\end{equation}
We note that the second term can also be understood as the free energy of RBIM with all four types of boundary condition (PBC/APBC along $x,y$ direction) into account. %This is related to the fact that we take $\rho_{Q,0}$ to be the maximally mixed state in the ground state subspace.  Instead, if we take the initial state to be eigen state of $W^z_{\gamma_x},W^z_{\gamma_y}$, we would obtain a RBIM under PBC.

\subsubsection{Critical error rate and classical memory from coherent information }
As we have demonstrated in the previous subsections, $S_{\rho_{Q}}$ and $S_{\rho_{RQ}}$ can be mapped to the free energy of the RBIM with or without the insertion of non-contractible defect lines (plus some constants),  so the coherent information $I_c$ is related the excess free energy of the defect line:

\begin{equation}
I_c=2\log 2-\overline{\log \frac{\sum_{d_x,d_y=0,1 }Z^{\text{RBIM}}_{d_x,d_y}}{Z^{\text{RBIM}}_{00}}}=2\log 2-\overline{\log\left[\sum_{d_x,d_y=0,1 }e^{\Delta F_{d_x,d_y}}\right]},
\end{equation}
where $\Delta F_{d_x,d_y}$ is the excess free energy with the insertion of a non-contractible defect line. For small $p$, the RBIM is in the ferromagnetic (FM) phase, and the excess free energy of a defect line is extensive, $\Delta F_{\{d_x,d_y\}\neq\{0,0\}}\sim O(L)$, which leads to $I_c=2\log 2$. On the other hand, when $p$ is sufficiently large, the RBIM undergoes a phase transition to a paramagnetic (PM) phase and $I_c<2\log2$. This is exactly the same as the situation with single-qubit errors. So the threshold is also the same: $p_c\approx 0.109$. The classical memory of $\rho_\varepsilon$ reflected in the fact that $I_c\geq 0$ for $p\leq \frac{1}{2}$. 
\subsection{Relative entropy}
In this section we calculate the relative entropy $
D(\rho_\varepsilon||\rho_\varepsilon^\alpha)\equiv\Tr(\rho_\varepsilon\log\rho_\varepsilon)-\Tr(\rho_\varepsilon\log\rho_\varepsilon^\alpha)$, with $\rho_\varepsilon^\alpha=\mathcal{N}^\varepsilon[w^\alpha \rho_0w^\alpha]$, and $w^\alpha$ are open string operators creating a pair of $\alpha$ anyons on the ends of the string. Again, we first calculate the Renyi relative entropy,
\begin{equation}
D^{(n)}(\rho_\varepsilon||\rho^\alpha_\varepsilon)\equiv\frac{1}{1-n}\log\frac{\Tr\rho_\varepsilon(\rho_\varepsilon^\alpha)^{n-1}}{\Tr\rho_\varepsilon^n},
\end{equation}
and revover $D(\rho_\varepsilon||\rho^\alpha_\varepsilon)$ by taking the limit $n\rightarrow 1$. 

Using error chain expansion, we get
\begin{equation}
\begin{aligned} 
\Tr\rho_{\varepsilon}(\rho_\varepsilon^\alpha)^{n-1}&=\sum_{\{C^{(s)}\}}\prod_{s=1}^{n}P(C^{(s)})\Tr\left(w^\varepsilon_{C^{(1)}}\rho_0 w^\varepsilon_{C^{(1)}}\prod_{s=2}^n w^\varepsilon_{C^{(s)}}w^\alpha\rho_0 w^\alpha w^\varepsilon_{C^{(s)}}\right)\\
&=\sum_{\left\{C^{(s)}\right\}}\sum_{a^{(s)},b^{(s)}}\left[\prod_{s=1}^{n} \frac{1}{4}P\left(C^{(s)}\right)\right]\bra{a^{(1)},b^{(1)}}w^\varepsilon_{C^{(1)}}w^\varepsilon_{C^{(2)}}w^\alpha\ket{a^{(2)},b^{(2)}}\bra{a^{(n)},b^{(n)}}w^\alpha w^\varepsilon_{C^{(n)}} w^\varepsilon_{C^{(1)}} \ket{a^{(1)},b^{(1)}}\\
&\qquad\qquad\qquad\ \prod_{s=2}^{n-1}\bra{a^{(s)},b^{(s)}}w^\alpha w^\varepsilon_{C^{(s)}}  w^\varepsilon_{C^{(s+1)}}w^\alpha \ket{a^{(s+1)},b^{(s+1)}}.
\end{aligned}
\end{equation}
Clearly, for $\alpha=e,m,\varepsilon'$ , $\Tr\rho_{\varepsilon}(\rho_\varepsilon^\alpha)^{n-1}=0$ , so the relative entropy diveges. Thus we only focus on $\alpha=\varepsilon$ next. Terms in the summation is nonvanishing only if the error chain configurations satisfy the following condition:
\begin{equation}
C^{(s)}=C^{(1)}+\partial v^{(s)}+d^{(s)}_x\gamma_x+d^{(s)}_y\gamma_y+A,\ s=1,2,\cdots ,n-1,
\end{equation}
where $d_{x/y}^{(s)}=0,1$ and $A$ denotes the string where $w^{\alpha=\varepsilon}$ acts nontrivially. Compared to \eqref{zhenji2}, \eqref{eq:RBIM2}, we can see that the insertion of $w^\varepsilon_A$ corresponds to inserting an additional defect line along $A$ in the RBIM, which means the Ising coupling flips sign along $A$. Denote the partition function of RBIM with defect line along $A$ as $Z^{\text{RBIM}}[A]$, where we implicitly sum over the four types of boundary conditions $\{d_x,d_y\}$. Then
\begin{equation}
D^{(n)}(\rho_\varepsilon||\rho^\varepsilon_\varepsilon)=\frac{1}{1-n}\log\frac{\overline{(Z^{\text{RBIM}}[A])^{n-1}}}{\overline{(Z^{\text{RBIM}})^{n-1}}},
\end{equation}
Taking the replica limit $n\rightarrow 1$, we obtain the relative entropy:
\begin{equation}
D(\rho_\varepsilon||\rho_\varepsilon^\varepsilon)=\overline{\log Z^{\text{RBIM}}}-\overline{\log Z^{\text{RBIM}}[A]},
\end{equation}
which is mapped to the excess free energy of defect line $A$. In the FM phase $(p<p_c)$, it diverges as the distance between the two ends of $A$ goes to infinity. However, in the PM phase $(p>p_c)$, it is finite, which means $\varepsilon$ is no longer a well-defined excitation. Since $\varepsilon\times e=m$, $e$ and $m$ can be viewed as the same type of excitations. 
\subsection{Calculation of the entanglement negativity}

In this section, we derive the entanglement negativity $\varepsilon_A(\rho)\equiv\log ||\rho^{T_A}_\varepsilon||_1$. It turns out that to calculate $\varepsilon_A(\rho)$, it's more convenient to use the loop expansion (Eq. (5) in the main text) instead of the error chain expansion used in the last two sections. 
We start from Eq. (6) in the main text.

\begin{equation}
\rho_{\ve}^{T_A}=\frac{1}{2^N}\sum_{g\in G}(1-2p)^{l_g}y_A(g)g,
\end{equation}
where $l_g$ is the length of the segment where $g_x$ and $g_z$ does not coincidence, and
$$
y_A(g)\equiv \sign_A(g_x,g_z)\equiv
\left\{
\begin{array}{ll}
1,\quad &\text{if } g_{xA},g_{zA} \text{ commute.} \\
-1,\quad &\text{if } g_{xA}, g_{zA} \text{ anticommute.}
\end{array}
\right.
$$  
 To calculate $\varepsilon_A(\rho_\ve)$, we utilize the replica trick, that is, we first calculate the $2n^{th}$ Renyi negativity $\varepsilon^{(2n)}_A(\rho_\ve):=\frac{1}{2-2n}\log \frac{\Tr(\rho^{T_A}_\ve)^{2n}}{\Tr\rho_\ve^{2n}}$ and finally take the replica limit $2n\rightarrow 1$.
To start with, we calculate $(\rho^{T_A})^2$,
\begin{equation}
\begin{aligned}
(\rho^{T_A}_\ve)^2 &=\frac{1}{2^N}\sum_{g,h\in G}(1-2p)^{l_g+l_h}y_A(g)y_A(h)gh\\
&=\frac{1}{2^N}\sum_{g,h\in G}(1-2p)^{l_g+l_h}y_A(gh)\sign_A(g,h)gh\\
&=\frac{1}{2^N}\sum_{g,\tilde{g}\in G}(1-2p)^{l_g+l_{g\tilde{g}}}y_A( \tilde{g})\sign_A(g,\tilde{g})\tilde{g}.
\end{aligned}
\label{eq:rhoT2}
\end{equation}
In the last step we use the substitution $h=g\gtd$. 

To simplify the expression, we deal with the summation over $g$ first. The crucial part in this expression is $\sign_A(g,\gtd)$ which leads to complete destructive inteference when $\gtd$ cross the boundary $\partial A$. To be more precise, we define the subgroup $H$ of $G$:
\begin{equation}
H\equiv\{g\in G|g_A\in G\}.
\end{equation}
$H$ contains all loops that do not cross the boundary $\partial A$. Then $\sign_A(g,\gtd)=1,\forall g\in G,\gtd \in H$. Then we can simplify  \eqref{eq:rhoT2} :
\begin{equation}
(\rho^{T_A}_\ve)^2=\frac{1}{2^N}\sum_{g\in G,\tilde{g}\in H}(1-2p)^{l_g+l_{g\tilde{g}}}y_A( \tilde{g})\tilde{g}.
\end{equation}
Thus

\begin{equation}
\begin{aligned}
\Tr(\rho^{T_A}_\ve)^{2n}&=2^{-2nN}\prod_{s=1}^{n}\sum_{\gtd^{(s)}\in H}\prod_{s=1}^{n}\sum_{g^({s)}\in G}(1-2p)^{\sum_{s=1}^{n}l_{g^{(s)}}+l_{g^{(s)}\gtd^{(s)}} } y_A(\prod_{s=1}^n \gtd^{(s)})\Tr\left(\prod_{s=1}^{n}\gtd^{(s)}\right)\\
&=2^{(1-2n)N}\sum_{\gtd^{(1)},\cdots,\gtd^{(n-1)}\in H} \sum_{g^{(1)},\cdots,g^{(n)}\in G}(1-2p)^{ \sum_{s=1}^{n-1}(l_{g^{(s)}}+l_{g^{(s)}\gtd^{(s)}}) +l_{\prod_{s=1}^{n-1}g^{(s)} }+l_{\prod_{s=1}^{n-1}g^{(s)}\gtd^{(s)} } }
\end{aligned}
\end{equation}


In a similar manner we can obtain $\Tr\rho_\ve^{2n}$, resulting in a similar expression with the summation over $H$ replaced by a summation over $G$:
\begin{equation}
 \Tr\rho_\ve^{2n}=2^{(1-2n)N}\sum_{\gtd^1,\cdots,\gtd^{n-1}\in G} \sum_{g^1,\cdots,g^n\in G}(1-2p)^{ \sum_{s=1}^{n-1}(l_{g^{(s)}}+l_{g^{(s)}\gtd^{(s)}}) +l_{\prod_{s=1}^{n-1}g^{(s)} }+l_{\prod_{s=1}^{n-1}g^{(s)}\gtd^{(s)} } }   
\end{equation}
Since we are only concerned about their ratio, we can extract the common part in the expression and rename it as $O_{ \{\gtd\} }$.
\begin{equation}
O_{\{ \gtd \}  }=2^{(1-2n)N} \sum_{g^1,\cdots,g^n\in G}(1-2p)^{ \sum_{s=1}^{n-1}(l_{g^{(s)}}+l_{g^{(s)}\gtd^{(s)}}) +l_{\prod_{s=1}^{n-1}g^{(s)} }+l_{\prod_{s=1}^{n-1}g^{(s)}\gtd^{(s)} } }
\end{equation}
It is straightforward to show that $O_{\{ \gtd \}  }$ is actually only a function of $l_{\gtd^{(s)}}$. 
Based on this observation, we can divide the summation over $\{\gtd\}$ into different classes. First, we define the invariant subgroup $G_\varepsilon$ of $G$, generated by $A_vB_{p=v^*=v+\bm{\delta}}$. In other words, elements in $G_\varepsilon$ are tensionless loops with $l_g=0$. Similarly, we define the invariant subgroup $H_\varepsilon$ of $H$ to be $H_\varepsilon\equiv\{g\in G_\varepsilon|g_A\in G\}$. Then
\begin{equation}
\begin{aligned}
\Tr(\rho_\ve^{T_A})^{2n}&=\sum_{\gtd^1_\varepsilon,\cdots \gtd^{n-1}_\varepsilon\in H_\varepsilon}\sum_{\tilde{u}^1,\cdots,\tilde{u}^{n-1}\in H/H_\varepsilon } O_{\{\tilde{u}\}},\\
\Tr\rho_\ve^{2n}&=\sum_{\gtd^1_\varepsilon,\cdots \gtd^{n-1}_\varepsilon\in G_\varepsilon}\sum_{\tilde{u}^1,\cdots,\tilde{u}^{n-1}\in G/G_\varepsilon } O_{\{\tilde{u}\}}
\end{aligned}
\end{equation}
Since $G/G_{\varepsilon}=H/H_{\varepsilon}$, we finally get 
\begin{equation}
\frac{  \Tr(\rho_\ve^{T_A})^{2n}\ } { \Tr\rho_\ve^{2n}  }=\left(\frac{|H_\varepsilon| } { |G_\varepsilon|   } \right)^{n-1}=2^{(2-2n)(|\partial A|-1)}
\end{equation}
So the Renyi negativity is:
\begin{equation}
\varepsilon^{(2n)}_A(\rho)=(|\partial A|-1)\log 2,\forall n
\end{equation}
Of course this result also holds in the replica limit $2n\rightarrow 1$, so $\varepsilon_A(\rho)=(|\partial A|-1)\log 2$. The subleading term, $\varepsilon^{\text{topo}}=\log 2$ is the topological entanglement negativity (TEN). We can also adopt the Kitaev-Preskill scheme to extract the TEN (See Fig.\ref{fig:multipartition}) \cite{kitaev2006topological}.
% Figure environment removed
\begin{equation}
\varepsilon^{\text{topo}}=-\varepsilon_A-\varepsilon_B-\varepsilon_C+\varepsilon_{AB}+\varepsilon_{AC}+\varepsilon_{BC}-\varepsilon_{ABC}=\log 2.
\label{eq:TEN}
\end{equation}
\subsection{Anyon braiding}
In this section, we discuss the possibility of performing anyon braiding on the topologically ordered mixed-state. Remarkably, for $p>p_c$, there still exists nontrivial anyon statistics, even if there are only two superselection sectors. First, although $\varepsilon$ particles are no longer well-defined excitations for $\rho_\varepsilon$, we can choose an alternative way of defining fermionic anyons, with $m$ sitting on the upper left side of $e$, and denote such composite particle as $\varepsilon'$, as shown in Fig.\ref{fig:braiding}. , and can be created at the ends of string operators $w_{\varepsilon'}=\prod_{i\in \gamma}X_iZ_{i+\bm{\delta}}$. Although $\varepsilon'$ lies in the same superselection sector as $\varepsilon$, it remains well-defined particles, because $D(\rho_\varepsilon||\rho^{\varepsilon'}_\varepsilon)$ is divergent. We demonstrate below that it has nontrivial braiding with $e/m$ anyons. 

We can perform anyon braiding in the following way (See Fig.\ref{fig:braiding}): First, create two $\varepsilon'$ anyons and two $e$ anyons, $\rho_{\varepsilon}^{e,\varepsilon'}=\mathcal{N}^\varepsilon[w_ew_{\varepsilon'}\rho_0w_{\varepsilon'}w_e]$ and drag one $\varepsilon'$ around one $e$ anyon along a cycle, which can be done by applying the loop operator $w^l_{\varepsilon'}=\prod_{i\in \text{loop }l}X_iZ_{i+\bm{\delta}}$. $\varepsilon'$ acquires a definite phase, $\theta=\pi$ , which is not affected by the $\mathcal{N}^\varepsilon$ channel during the process because $w_{\varepsilon'}$ commute with the Kraus operators, $[w_{\varepsilon'},Z_iX_{i+\bm{\delta}}]=0$. Hence, the mutual semion statistics between $\varepsilon'$ and $e$ is well preserved.   

%One may concern if this phase factor is really detectable for a mixed state, since we always have $w^l_\varepsilon\rho^{e,\varepsilon'}_\varepsilon w^l_\varepsilon=\rho^{e,\varepsilon'}_\varepsilon$, regardless of whether $l$ encloses a $e/m$ anyon. Actually, the same issue is encountered for pure states. The only result of such braiding process is only an undetectable overall phase, after all. 
To turn this braiding phase into observable effect, we can use the same protocol as in Google's experiment \cite{satzinger2021realizing}. That is, we introduce an ancilla qubit and prepare the initial state $|+\rangle\langle +|\otimes \rho_\varepsilon^{e,\varepsilon'}$, where $|+\rangle\equiv\frac{|1\rangle+|0\rangle}{\sqrt{2}}$. Then we can use the ancilla qubit to control the braiding process, by constructing the unitary gate: $U=|0\rangle\langle0|\otimes I+|1\rangle\langle 1|\otimes w^l_{\varepsilon'} $ Then the phase obtained from braiding statistics can be transformed into rotation of the ancilla qubit:
\begin{equation}
U\left( |+\rangle\langle +|\otimes\rho_\varepsilon^{e,\varepsilon'}\right)U^\dagger=|-\rangle\langle-|\otimes\rho_\varepsilon^{e,\varepsilon'},
\end{equation}
where $|-\rangle=\frac{|0\rangle-|1\rangle}{\sqrt{2}}$.
% Figure environment removed
 The existence of nontrivial braiding statistics can be viewed as a nontrivial consequence of long-range entanglement of $\rho_\varepsilon$. However, this braiding process is only immune to noise channels with Kraus operators commuting with the $w^l_{\varepsilon'}$, and is not robust against other types of noise/errors. This is because $\varepsilon'$ is in the same superselection sector as $\varepsilon$, and thus, the vaccum, for $p>p_c$. As a result, $\varepsilon'$ is no longer a well-defined excitation in the presence of generic types of errors, and thus the braiding between $\varepsilon'$ and $e/m$ is no longer well-defined under generic errors.
\\

\subsection{Adding phase errors}
In this section, we aim to discuss the robustness of the intrinsic mixed-state quantum topological order under other noises. For simplicity, we consider the effect of additional single-qubit phase errors here. That is, we consider the following mixed state:
\begin{equation}
{\rho}_{\varepsilon,e}=\mathcal{N}^z\circ\mathcal{N}^\varepsilon[\rho_0]
\end{equation}
Similar to section A, B, we analyze the property of $\tilde{\rho}_\varepsilon$ by calculating the von Neumann entropy $S(\tilde{\rho}_\varepsilon)$ and map it to statistical models. 

We denote the error rate and error chain configuration of $\mathcal{N}^z$ $(\mathcal{N}^\varepsilon)$ as $p_z$ $(p_\varepsilon)$ and $C_z$ $(C_\varepsilon)$, respectively. Then ${\rho}_{\varepsilon,e}$ can be represented by error chain expansion as in \eqref{eq:errorstring} :
\begin{equation}
\rho_{\varepsilon,e}=\sum_{C_z,C_\varepsilon}P_\varepsilon(C_\varepsilon)P_z(C_z)w^\varepsilon_{C_\varepsilon}w^e_{C_z}\rho_0 w^e_{C_z}w^\varepsilon_{C_\varepsilon}.
\end{equation}
Then we can write the $n$-th moment as:
\begin{equation}
\begin{aligned} 
\Tr(\rho_{\varepsilon,e}^n)&=\sum_{\left\{C^{(s)}\right\}}\sum_{a^{(s)},b^{(s)}}\prod_{s=1}^{n} \frac{1}{4}P_\varepsilon\left(C_\varepsilon^{(s)}\right)P_z\left(C_z^{(s)}\right)\bra{a^{(s)},b^{(s)}}w^e_{C_z^{(s)}}w^\varepsilon_{C_\varepsilon^{(s)}} w^\varepsilon_{C_\varepsilon^{(s+1)}}w^e_{C_z^{(s+1)}} \ket{a^{(s+1)},b^{(s+1)}}\\
\end{aligned}
\label{eq:moment2}
\end{equation}
Nonzero contributions only come from error chain configurations satisfying: 
\begin{equation}
C_z^{(s)}=C_{z}^{(1)}+\partial v^{(s)}+{d_x^{z,(s)}}\gamma_x+{d_x^{z,(s)}}\gamma_y,\ C_{\varepsilon}^{(s)}=C_{\varepsilon}^{(1)}+\partial v^{(s)}+{d_x^{\varepsilon,(s)}}\gamma_x+{d_x^{\varepsilon,(s)}}\gamma_y,
\end{equation}
so \eqref{eq:moment2} can be simplified as:
\begin{equation}
\begin{aligned}
\Tr(\rho_{\varepsilon,e}^n)=\frac{1}{4^{n-1}}\cdot\frac{1}{4^{n-1}}\prod_{\alpha=z,\varepsilon}\left[\sum_{C_\alpha^{(s)}} P\left(C_\alpha^{(1)}\right) \prod_{s=1}^{n-1} \sum_{\left\{v_\alpha^{(s)}\right\}}\sum_{d_x^{\alpha,(s)},d_y^{\alpha,(s)}=0,1} P\left(C_\alpha^{(1)}+\partial v_\alpha^{(s)}+d_x^{\alpha,(s)}\gamma_x+{d_y^{\alpha,(s)}}\gamma_y\right)\right].
\end{aligned}
\end{equation}
As in \eqref{eq:RBIM2}, the collection of terms in the bracket for each $\alpha$ can be mapped to the partition function of a $(n-1)$-flavor RBIM: 

\begin{equation}
\begin{aligned}
\Tr(\rho_{\varepsilon,e}^n)=\frac{1}{4^{2n-2}}\left(\sqrt{(1-p_\varepsilon)p_\varepsilon}\right)^{(n-1)N}\left(\sqrt{(1-p_z)p_z}\right)^{(n-1)N}\overline{\left(Z^{\text{RBIM}}_{p_\varepsilon}(J_\varepsilon)\right)^{n-1}}\cdot\overline{\left(Z^{\text{RBIM}}_{p_z}(J_z)\right)^{n-1}},
\end{aligned}
\label{eq:RBIM3}
\end{equation}
where $J_\alpha$ $(\alpha=z,\varepsilon)$ is the strength of Ising coupling for each of the two RBIMs. $p_\alpha$ denotes, the probability of antiferromagnetic coupling on each bond. Both RBIMs are situated along the Nishimori line: $e^{-2J_\alpha}=\frac{p_\alpha}{1-p_\alpha}$. Again, the partition functions implicitly contain summations over the four boundary conditions. The von Neumann entropy can be obtained by taking the $n\rightarrow 1$ limit:
\begin{equation}
\begin{aligned}
S({\rho}_{\varepsilon,e})&=- \lim_{n\rightarrow1}\frac{\partial }{\partial n}\Tr({\rho}_{\varepsilon,e})\\
&=-\overline{ \log Z^{\text{RBIM}}_{p_z}(J_z)}-\overline{ \log Z^{\text{RBIM}}_{p_\varepsilon}(J_\varepsilon)} +\left(-4\log 2-\frac{N}{2}\log[p_z(1-p_z)]-\frac{N}{2}\log[p_\varepsilon(1-p_\varepsilon)]\right),
\end{aligned}
\end{equation}
The terms in the parentheses is always regular for finite $p_{z},p_\varepsilon$, so we can focus on the first two terms, which is the free energy of two decoupled RBIMs. We denote the Ising spin variables of the two RBIMs as $\sigma$ and $\tau$, respectively. Then we can straightforwardly get the phase diagram of the statistical model, shown in Fig. \ref{fig:phasediagram}. 

% Figure environment removed
Applying the same strategy as in section A, B, we can map the relative entropy and coherent information to observables in the RBIM. We directly list the results here. 
\begin{enumerate}
\item Relative entropy. $D(\rho_{\varepsilon,e}||\rho^e_{\varepsilon,e})$ is mapped to the excess free energy of the defect line (connecting the inserted pair of $e$ anyons) of the RBIM of spin $\sigma$; $D(\rho_{\varepsilon,e}||\rho^\varepsilon_{\varepsilon,e})$ is mapped to the excess free energy of the defect line (connecting the inserted pair of $\varepsilon$ anyons) of the RBIM of spin $\tau$;  $D(\rho_{\varepsilon,e}||\rho^m_{\varepsilon,e})$  is mapped to the sum of the excess free energy of the defect line (connecting the inserted pair of $m$ anyons) of the two RBIMs, because $w^m=w^ew^{\varepsilon}$.
\item Coherent information. 
\begin{equation}
    I_c=2\log 2-\overline{\log\left[\sum_{d_x,d_y=0,1 }e^{-\Delta F^\sigma_{d_x,d_y}}\right]}-\overline{\log\left[\sum_{d_x,d_y=0,1 }e^{-\Delta F^\tau_{d_x,d_y}}\right]}
\end{equation}

\end{enumerate}

where $\Delta F_{d_x,d_y}^{\sigma(\tau)}$ is the excess free energy of non-contractible defect lines in the RBIM of spin $\sigma\ (\tau)$. Across transitions at $p_z\approx 0.109$ and $p_\varepsilon\approx 0.109$, $I_c$ change discontinuously. Quantum memory can only be realized when both $\sigma$ and $\tau$ are in the FM phase, while classical memory corresponds to one of the spin species is in the FM phase while the other in the PM phase, and the topological memory is completely lost when both RBIMs are in the PM phase.

From the above mapping, we can relate the four phases of the RBIM in Fig. \ref{fig:phasediagram} to the four types of topological order (including the trivial one). See Fig. 3 in the main text for a summary of the properties of these phases. 

Unfortunately, since $\varepsilon'$ anyons belong to the same sector as the vacuum in the intrinsic mixed-state quantum TO, they are no longer well-defined at nonzero $p_z$. Actually, the relative entropy $D(\rho_{\varepsilon,e}||\rho^{\varepsilon'}_{\varepsilon,e})$ can be mapped to the excess free energy of some short-distance defect line, so it is finite in general. Thus, for this phase, it seems anyon braiding is only well-defined at $p_z=0$. Also, it is extremely challenging to calculate the TEN when both types of channels are present, both analytically or numerically. However, based on the mapping to the RBIM and results of the relative entropy, we expect that TEN would remain nonzero for $p_z<p_c\approx 0.109$.   
\section{Kitaev honeycomb model under decoherence}
\subsection{The model}
In this section, we investigate the effect of the incoherent proliferation of the fermionic anyons (the $f$ anyons) in the Kitaev honeycomb model:
\begin{equation}
H=-J_{x} \sum_{x \text {-bonds }} \sigma_{j}^{x} \sigma_{k}^{x}-J_{y} \sum_{y \text {-bonds }} \sigma_{j}^{y} \sigma_{k}^{y}-J_{z} \sum_{z \text {-bonds }} \sigma_{j}^{z} \sigma_{k}^{z},
\label{kitaev_ham}
\end{equation}
It can be exactly solved by introducing the Majorana fermion operators: $\Vec{\sigma}=i\Vec{b}b^0$. After fixing the $Z_2$ gauge fields as $\hat{u}_{j k}=i b_{j}^{\alpha} b_{k}^{\alpha}=1$, the Kitaev Hamiltonian \eqref{kitaev_ham} becomes a free fermion Hamiltonian, and the physical eigenstates $|\Psi\rangle$ can be obtained by the application of the projection operator $\hat{P}=\Pi_{i}\frac{1+\hat{D}_i}{2}$, i.e., $|\Psi\rangle=\Pi_{i}\frac{1+\hat{D}_i}{2}|\{u_{ij}=1\}\rangle\otimes|\psi_F\rangle$, where $\hat{D}_i=b^{x}_i b^{y}_i b^{z}_i b^0_i$.

We start with the Abelian phase A, and take the limit $|J_z|\gg |J_x|,|J_y|$ for simplicity. However, the essential physics we discuss below remains unchanged away fom this limit, as long as the initial state stays in the A phase. In this limit, the ground state $\rho_0=|\Psi_0\rangle\langle \Psi_0|$ is the vacuum of the complex fermions $f_i$, composed of the two Majorana fermions $b^0$ on the $z$-bonds. $f_i=\frac{1}{\sqrt{2}}(b^0_{i,A}+ib^0_{i,B})$. $f$ is a fermionic anyon that belongs to the superselection sector $\varepsilon$. The ground state is $4$-fold degenerate on a torus. For simplicity, we take $|\Psi_0\rangle$ to be the one without flux threading through cycles $\gamma_x,\gamma_y$.
\begin{equation}
|\Psi_0\rangle=\hat P|\{u_{ij}=1\}\rangle\otimes|0_F\rangle. 
\end{equation}
%We denote the density matrix of the ground state as: $\rho_0=|\Psi_0\rangle\langle\Psi_0|$.
%where $|\psi_0\rangle$ is the ground state:  $|\psi_0\rangle=\Pi_{i}\frac{1+\hat{D}_i}{2}|\{u_{ij}=1\}\rangle\otimes|\psi_f\rangle$. 

To proliferate the $f$ anyons,  we impose the following quantum channels on the ground state:
\begin{equation}
\begin{aligned} 
\mathcal{N}^{X}_{\langle ij\rangle\in x \text {-bonds }}[\rho_0] & =\left(1-p_{x}\right) \rho_0+p_{x} \sigma^x_{i}\sigma^x_{j}\rho_0 \sigma^x_{i}\sigma^x_{j},\\ 
\mathcal{N}^Y_{\langle ij\rangle\in y \text {-bonds }}[\rho_0] & =\left(1-p_{y}\right) \rho_0+p_{y} \sigma^y_{i}\sigma^y_{j}  \rho_0\sigma^y_{i}\sigma^y_{j}.
\end{aligned}
\end{equation}
And the decohered density matrix is: $\rho_f=\mathcal{N}^{X}\circ \mathcal{N}^{Y}\left[\rho_{0}\right]$, where $\mathcal{N}^{X(Y)}=\prod_{\langle ij \rangle} \mathcal{N}^{X(Y)}_{\langle ij \rangle }$. Compared to Eq. (10) in the main text, the $\mathcal{N}^{Z}$ channel is neglected since it has no effect on the particular choice of $\rho_0$ here. Physically, the quantum channel induce hopping, pair creation and annhilation of $f$. For simplicity, we take the error rate $p_x=p_y=p$. In other words, the only effect of the channel is to change the fermion part of the density matrix and leave the gauge field configuration invariant. This can be shown by writing the Kraus operators in terms of Majorana fermion operators: $\sigma^{x(y)}_{i}\sigma^{x(y)}_{j}=-i\hat{u}_{ij}b_i^0b_j^0$. Thus the density matrix $\rho_f$ can be written as: 
\begin{equation}
    \rho_f=\hat{P}|\{u_{ij}=1\}\rangle\langle\{u_{ij}=1\}|\otimes\rho_F\hat{P}, 
\end{equation}
where $\rho_F$ is the Majorana fermion density matrix, which is $\rho_F=\mathcal{N}^{X,F} \circ \mathcal{N}^{Y,F}\left[\rho_{F,0}\right]$, where $\rho_{F,0}=|0_F\rangle\langle 0_F|$, and  $\mathcal{N}^{X(Y),F}=\prod_{\langle ij \rangle} \mathcal{N}^{X(Y), \langle ij \rangle,F}$:
\begin{equation}  
\mathcal{N}^{X(Y),F}_{\langle ij\rangle\in x(y) \text {-links }}[\rho_{F,0}] =\left(1-p_{x}(p_{y})\right) \rho_{F,0}+p_{x}(p_{y}) b_i^0b_j^0\rho_{F,0} b_j^0b_i^0.
\end{equation}
Similar to the case in the toric code model, the error-corrupted state $\rho_f$ undergoes a phase transition in the mixed-state topological order, at a critical error rate $p_c\approx 0.109$, which can be determined by mapping to the RBIM, analogous to the case in the toric code. The topological quantum memory also breaks down to classical topological memory after the transition, with two remaining commuting logical operators:
\begin{equation}
W_{\gamma_x}=\prod_{\langle ij\rangle\in\gamma_x}\sigma_i^{\alpha_{\langle ij\rangle}}\sigma_j^{\alpha_{\langle ij\rangle}},\ W_{\gamma_y}=\prod_{\langle ij\rangle\in\gamma_y}\sigma_i^{\alpha_{\langle ij\rangle}}\sigma_j^{\alpha_{\langle ij\rangle}},
\end{equation}
where $\alpha_{\langle ij\rangle}=x,y,z$  for $\langle ij\rangle \in$ $x$-links, $y$-links, $z$-links. The residual classical memory is due to the fact that $[W_{\gamma_x},\sigma_i^{\alpha_{\langle ij\rangle}}\sigma_j^{\alpha_{\langle ij\rangle}}]=[W_{\gamma_y},\sigma_i^{\alpha_{\langle ij\rangle}}\sigma_j^{\alpha_{\langle ij\rangle}}]=0, \ \forall \langle ij \rangle.$
In terms of anyon excitations, there are four superselection sectors before error corruption, $1,e,m,f\ (\text{or }\varepsilon)$. Gauge flux excitations on even and odd rows corresponds to $e,m$ anyons, respectively, which is termed as weak translation symmetry breaking. Above $p_c$, $f$ anyons are no-longer well-defined excitations, so $e$ and $m$ now belongs to the same superselectors, and translation symmetry is restored. 
% Figure environment removed
\subsection{Entanglement negativity}
In this section, we compute the entanglement negativity of the $\rho_f$ in the decohered Kitaev honeycomb model. We will show that its topological entanglement negativity is the same as the ground state $\rho_0$ and is independent of the error rate $p$. 

Again, we use the replica trick to compute the entanglement negativity:
\begin{equation}
\mathcal{E}_{A}(\rho):=\log \left\|\rho^{T_{A}}\right\|_{1}=\lim_{2n\rightarrow 1}\frac{1}{2-2 n} \log \frac{\Tr\left(\rho^{T_{A}}\right)^{2 n}}{\Tr \rho^{2 n}}.
\end{equation}
First it is easy to show: $\lim_{2n\rightarrow 1} \Tr \rho_{f}^{2 n}=\Tr \rho_{f}=1$, due to the trace-preserving property of quantum channels. So we only need to deal with the numerator: $\Tr\left(\rho^{T_{A}}\right)^{2 n}$.

A general curve $\gamma$, which bipartite the honeycomb lattice into A and B subregions, intersects the bonds of the honeycomb lattice. In order to partial transpose the degrees of freedom in the A subregion, we should define new $Z_2$ gauge fields to replace the gauge fields on the intersected bonds. Following the notation in \cite{PhysRevLett.105.080501}, we assume $\gamma$ intersects $2L$ bonds, and we denote the bonds intersected by the curve $\gamma$ as: $\overline{a_n b_n},n=1,2,...,2L$. If the $Z_2$ gauge field on the bonds $\overline{a_{2n-1}b_{2n-1}}$ and $\overline{a_{2n}b_{2n}}$  are $\hat{u}_{a_{2n-1} b_{2n-1}}=ib_{a_{2n-1}}^\alpha b_{b_{2n-1}}^\alpha,\hat{u}_{a_{2n} b_{2n}}=ib_{a_{2n}}^\beta b_{b_{2n}}^\beta $; then we introduce two $Z_2$ gauge fields in the A and B subregions respectively: $w_{A,n}=ib^{\alpha}_{a_{2n-1}}b^{\beta}_{a_{2n}},w_{B,n}=ib^{\alpha}_{b_{2n-1}}b^{\beta}_{b_{2n}}$. Then the ground state configuration of the gauge fields on the intersected links can be written as:
\begin{equation}
\left|\{u_{p}\}\right\rangle=\frac{1}{\sqrt{2^{L}}} \sum_{w_{A}=w_{B}=\{ \pm 1\}}\left|w_{A}, w_{B}\right\rangle,
\end{equation}
where $\ket{\{u_p\}}$ is the direct product of  $\ket{u_{a_n b_n}=1}$. As a result, the ground state density matrix can be written as:
\begin{equation}
\begin{aligned}
    \rho_0&=\frac{1}{2^{N+L+1}} \sum_{g,g^{\prime}, w,w^{\prime}} D_{g}\ket{u_{A}, w} \ket{u_{B},w}\bra{u_A,w^{\prime}}\bra{u_B,w^{\prime}}\otimes \rho_{F,0}D_{g^{\prime}},
\end{aligned}
\end{equation}
where the summation over $g,g^{\prime}$ is over all the possible sets of the lattice sites, and $D_g=\prod_{i\in g} D_i$.  What's more, we can simply replace the $\rho_{F,0}$ with $\rho_{F}$ to get the decohered density matrix $\rho_f$. The partial trace of the density matrix is:
\begin{equation}
    \rho_f^{T_{A}}=\frac{1}{2^{N+L+1}} \sum_{g,g^{\prime}, w,w^{\prime}} D_{g_A^{\prime}}D_{g_B}\ket{u_{A}, w^{\prime}} \ket{u_{B},w}\bra{u_A,w}\bra{u_B,w^{\prime}}\otimes \rho^{T_A}_F D_{g_B^{\prime}}D_{g_A}.
\end{equation}
Thus
\begin{equation}
    \begin{aligned}
        (\rho_f^{T_{A}})^2&=(\frac{1}{2^{N+L+1}})^2 \sum_{g,g^{\prime}, w,w^{\prime}}\sum_{g_2,g_2^{\prime}, w_2,w_2^{\prime}} D_{g_A^{\prime}}D_{g_B}\left[\ket{u_{A}, w^{\prime}}\ket{u_{B},w} \otimes \rho_F\right]\\
     &\bra{u_A,w}\bra{u_B,w^{\prime}} D_{g^{\prime}_B}D_{g_A}D_{g_{2,B}}D_{g^{\prime}_{2,A}}\ket{u_{A}, w_2^{\prime}}\ket{u_{B},w_2}\\
     &\rho_F^{T_A}\otimes\bra{u_{A}, w_2}\bra{u_{B},w_2^{\prime}},
    \end{aligned}
\end{equation}
where $g_A$ ,$g_B$, are the sets of lattice sites $g\bigcap A, g\bigcap B$, respectively. Now we split $D_g$ into the gauge field part and fermion part :$D_g=X_gY_g$, where $X_{g}=i^{|g|(|g|-1) / 2} \prod_{j \in g} b_{j}^{x} b_{j}^{y} b_{j}^{z}$ and $Y_{g}=i^{|g|(|g|-1) / 2} \prod_{j \in g} b^0_{j}$, where $|g|$ is the number of lattice sites in region $g$. The inner product can be simplified as:
\begin{equation}
\begin{aligned}
       & \bra{u_A,w}\bra{u_B,w^{\prime}} D_{g_B^{\prime}}D_{g_A}D_{g_{2,B}}D_{g_{2,A^{\prime}}}\ket{u_{A}, w_2^{\prime}}\ket{u_{B},w_2}\\
       &=\delta_{w, w_2^{\prime}}\left(\delta_{g_{A}, g_{2,A}^{\prime}}+x_{A}(w) \delta_{g_{A}, A-g^{\prime}_{2,A}}Y_{A}\right)\delta_{w^{\prime}, w_2}\left(\delta_{g^{\prime}_{B}, g_{2,B}}+x_{A}(w) \delta_{g^{\prime}_{B}, B-g_{2,B}}Y_{B}\right)\\
       &=\left(2\delta_{w, w_2^{\prime}}P_{F,A}^{x_A(w)}\right)\left(2\delta_{w^{\prime}, w_2}P_{F,B}^{x_B(w^{\prime})}\right),
\end{aligned}
\label{eq:innerproduct}
\end{equation}
where $P_{F,A(B)}^{x}=\frac{1+xY_{A(B)}}{2}$ is the projection to the subspace with fixed Fermi parity $x$ in subregion $A$ $(B)$, and $x_{A(B)}(w)=\left\langle u_{A(B)}, w\left|X_{A(B)}\right| u_{A(B)}, w\right\rangle=p_{A(B)} \prod_{n=1}^{L} w_{n}$, where we define $p_{A(B)}\equiv\prod_{i,j \in A(B)} u_{i j}$.  Puting this inner product back into the $(\rho^{T_{A}})^2$, we obtain:
\begin{equation}
    \begin{aligned}
        (\rho^{T_{A}})^2&=(\frac{1}{2^{N+L+1}})^2 \sum_{g,g^{\prime}, w,w^{\prime}} D_{g}\ket{u_{A}, w^{\prime}}\ket{u_{B},w}\bra{u_{A}, w^{\prime}}\bra{u_{B},w} \otimes \rho_F^{T_A}2^N\left(2P_{F,A}^{x_A(w)}\right)\left(2P_{F,B}^{x_B(w^{\prime})}\right)\rho_F^{T_A}D_{g^{\prime}}\\
        &=\frac{1}{2^{N+2L}} \sum_{g,g^{\prime}, w,w^{\prime}} D_{g}\ket{u_{A}, w^{\prime}}\ket{u_{B},w}\bra{u_{A}, w^{\prime}}\bra{u_{B},w} \otimes \rho_F^{T_A}P_{F,A}^{x_A(w)}P_{F,B}^{x_B(w^{\prime})}\rho_F^{T_A}D_{g^{\prime}}
    \end{aligned}
\end{equation}
Now we move one step further to calculate $(\rho^{T_{A}})^4$:
\begin{equation}
\begin{aligned}
        (\rho^{T_{A}})^4&= \left((\rho^{T_{A}})^2\right)^2=(\frac{1}{2^{N+2L}})^2 \sum_{g,g^{\prime}, w,w^{\prime}} D_{g}\ket{u_{A}, w^{\prime}}\ket{u_{B},w}\bra{u_{A}, w^{\prime}}\bra{u_{B},w} \otimes \rho_F^{T_A}P_{F,A}^{x_A(w)}P_{F,B}^{x_B(w^{\prime})}\rho_F^{T_A}D_{g^{\prime}}\\
        &\sum_{g_2,g_2^{\prime}, w_2,w_2^{\prime}} D_{g_2}\ket{u_{A}, w_2^{\prime}}\ket{u_{B},w_2}\bra{u_{A}, w_2^{\prime}}\bra{u_{B},w_2} \otimes \rho_F^{T_A}P_{F,A}^{x_A(w_2)}P_{F,B}^{x_B(w_2^{\prime})}\rho_F^{T_A}D_{g_2^{\prime}}\\
        &=(\frac{1}{2^{N+2L}})^2 \sum_{g,g^{\prime}, w,w^{\prime}} D_{g}\ket{u_{A}, w^{\prime}}\ket{u_{B},w}\bra{u_{A}, w^{\prime}}\bra{u_{B},w}\otimes \rho_F^{T_A}P_{F,A}^{x_A(w)}P_{F,B}^{x_B(w^{\prime})}\rho_F^{T_A}\\
        &2^N\left(2P_{F,A}^{x_A(w^{\prime})}\right)\left(2P_{F,B}^{x_B(w)}\right)\rho_F^{T_A}P_{F,A}^{x_A(w)}P_{F,B}^{x_B(w^{\prime})}\rho_F^{T_A}D_{g_2^{\prime}}\\
        &=\frac{1}{2^{N+4L-2}}\sum_{g,g^{\prime}, w,w^{\prime}} D_{g}\ket{u_{A}, w^{\prime}}\ket{u_{B},w}\bra{u_{A}, w^{\prime}}\bra{u_{B},w}\\
        &\otimes \left(\rho_F^{T_A}P_{F,A}^{x_A(w)}P_{F,B}^{x_B(w^{\prime})}\right)\left(\rho_F^{T_A}P_{F,A}^{x_A(w^{\prime})}P_{F,B}^{x_B(w)}\right)\left(\rho_F^{T_A}P_{F,A}^{x_A(w)}P_{F,B}^{x_B(w^{\prime})}\right)\rho_F^{T_A} D_{g^{\prime}}.
    \end{aligned}
\end{equation}
With these results, we can now arrive at $(\rho^{T_{A}})^{2n}$ by the iteration and induction:
\begin{equation}
\begin{aligned}
    (\rho^{T_{A}})^{2n}&=\frac{1}{2^{N+2nL-2(n-1)}}\sum_{g,g^{\prime}, w,w^{\prime}} D_{g}\ket{u_{A}, w^{\prime}}\ket{u_{B},w}\bra{u_{A}, w^{\prime}}\bra{u_{B},w}\\
    &\left[\left(\rho_F^{T_A}P_{F,A}^{x_A(w)}P_{F,B}^{x_B(w^{\prime})}\right)\left(\rho_F^{T_A}P_{F,A}^{x_A(w^{\prime})}P_{F,B}^{x_B(w)}\right)\right]^{n-1}\left(\rho_F^{T_A}P_{F,A}^{x_A(w)}P_{F,B}^{x_B(w^{\prime})}\right)D_{g^{\prime}}.
    \end{aligned}
\end{equation}
Using \eqref{eq:innerproduct} again, we can obtain the trace:
\begin{equation}
\begin{aligned}
    \Tr(\rho^{T_{A}})^{2n}&=\frac{1}{2^{2nL-2n}}\sum_{w,w^{\prime}} \Tr_F\left(\rho_F^{T_A}P_{F,A}^{x_A(w^{\prime})}P_{F,B}^{x_B(w)}\rho_F^{T_A}P_{F,A}^{x_A(w)}P_{F,B}^{x_B(w^\prime)}\right)^n\\
    &=\frac{2^{L-1}\cdot2^{L-1}}{2^{2n(L-1)}}\Tr_F\bigg(\rho_F^{T_A}P_{F,A}^{p_A}P_{F,B}^{p_B}\rho_F^{T_A}P_{F,A}^{p_A}P_{F,B}^{p_B}+\rho_F^{T_A}P_{F,A}^{p_A}P_{F,B}^{-p_B}\rho_F^{T_A}P_{F,A}^{-p_A}P_{F,B}^{p_B}\\
    &\qquad\qquad\qquad\qquad+\rho_F^{T_A}P_{F,A}^{-p_A}P_{F,B}^{p_B}\rho_F^{T_A}P_{F,A}^{p_A}P_{F,B}^{-p_B}+\rho_F^{T_A}P_{F,A}^{-p_A}P_{F,B}^{-p_B}\rho_F^{T_A}P_{F,A}^{-p_A}P_{F,B}^{-p_B}\bigg)^n
    \end{aligned}
\end{equation}
Since the projector $D_{\text{tot}}=\prod_i D_i=X_{\text{tot}}Y_{\text{tot}}=1$, the total fermion parity of the whole system is fixed by $Y_{\text{tot}}=X_{\text{tot}}=p_Ap_B$. Therefore, terms in the bracket can be simplified as:
\begin{equation}
\begin{aligned}
&\rho_F^{T_A}P_{F,A}^{p_A}P_{F,B}^{p_B}\rho_F^{T_A}P_{F,A}^{p_A}P_{F,B}^{p_B}+\rho_F^{T_A}P_{F,A}^{p_A}P_{F,B}^{-p_B}\rho_F^{T_A}P_{F,A}^{-p_A}P_{F,B}^{p_B}+\rho_F^{T_A}P_{F,A}^{-p_A}P_{F,B}^{p_B}\rho_F^{T_A}P_{F,A}^{p_A}P_{F,B}^{-p_B}+\rho_F^{T_A}P_{F,A}^{-p_A}P_{F,B}^{-p_B}\rho_F^{T_A}P_{F,A}^{-p_A}P_{F,B}^{-p_B}\\
=&\rho_F^{T_A}(P^+_{F,A}+P^-_{F,A})(P^+_{F,B}+P^-_{F,B})\rho_F^{T_A}(P^+_{F,A}+P^-_{F,A})(P^+_{F,B}+P^-_{F,B})\\
=&(\rho^{T_A}_{F})^2
\end{aligned}
\end{equation}
Then we can finally get the entanglement negativity:
\begin{equation}
    \begin{aligned}
        \mathcal{E}_{A}(\rho)&=\log(\tr(\rho^{T_A}))\\
        &=\lim_{2n\rightarrow 1}\frac{1}{2-2 n} \log \tr\left(\rho^{T_{A}}\right)^{2 n}\\
        &=L\log 2-\log 2 +\log||\rho_F^{T_A}||_1,
    \end{aligned}
\end{equation}
where the last term is the entanglement negativity $\varepsilon_A(\rho_F)$ of the fermion density matrix. In the gapped phase, $\rho_{F,0}$ is a Gaussian state with finite correlation length, so $\varepsilon_A(\rho_{F,0})$ satisfies an area law with a vanishing subleading term for $L\rightarrow \infty$ (For example, in the case $|J_z|\rightarrow \infty$, $\varepsilon_A(\rho_{F,0})=\log 2\cdot|\# \text{ of } z\text{-bonds across the boundary }\partial A|$). We expect $\varepsilon_A(
\rho_{F})$ also have the same property after applying the local channels. Using \eqref{eq:TEN} again, the entanglement contributed by fermions cancel out completely (along with the first term), and we can get the TEN $\varepsilon^{\text{topo}}=\log2$.    
\section{Exact solution to the gapless spin liquid phase of the toric code model through fermionization }
% Figure environment removed
In this section we analyze the properties the toric code model with additional $ZX$ terms in the Hamiltonian,
\begin{equation}
H=-\sum_v A_v-\sum_p B_p -\sum_i h_{xz} Z_i X_{i+\bm{\delta}}.
\label{eq:ZX}
\end{equation}
We show that this model can be solved exactly using fermionization analogous to the method in \cite{CHEN2018234, PhysRevResearch.4.043003}. Firstly, we note that the model has an extensive number of local conserved quantities. $[H,W_p]=0$ with $W_p=A_{v=p-\bm{\delta}} B_{p}$, which follows from the fact that the $e$ anyons and adjacent $m$ anyons are always created or annihilated in pairs, so we can solve the model in each simultaneous eigenspace of $W_p$. Secondly, the role of $ZX$ term is to induce pair creation, annihilation and hopping of $\varepsilon$ anyons, which are fermions. Then on a infinite lattice or a topologically trivial lattice, in each sector $\{W_p=w_p\}$, the only degrees of freedom are the $\varepsilon$ anyons, so we expect in each sector the model can be described by a fermion tight-binding model. We assume the fermions are defined on the vertices of the lattice, with the mapping 
\begin{equation}
n^f_v\longleftrightarrow \frac{1-A_v}{2}, 
\label{eq:map1}
\end{equation}
where $n^f_v=f^\dagger_v f_v$ is the fermion number operator. This mapping follows naturally from the observation that in the ground state sector $w_p=1$ (at least for small $h_{xz}$, and we numerically verify that this is always the lowest energy sector for any value of $h_{xz}$ ), $\frac{1-A_v}{2}$ corresponds to the occupation number of the $\varepsilon$ anyon on $v$.  Finally, because $\varepsilon$ and $e/m$ anyons are mutual semions, an $\varepsilon$ anyon can acquire a nontrivial phase depending on $w_p$ when moving around the plaquette $p$. Thus $W_p$ should correspond to static $Z_2$ flux on each plaquette in the fermion model, so we have the following mapping,
\begin{equation}
T_{v_iv'_i}\equiv iu_{i}\gamma_{v_i}\gamma'_{v'_i} \longleftrightarrow Z_i X_{i+\bm{\delta}} ,\quad \text{link }i\equiv \langle v_iv'_i\rangle
\label{eq:map2}
\end{equation}
where $\gamma_v=f_v+f^\dagger_v,\gamma'_{v'}=-i(f_{v'}-f^\dagger_{v'})$ are Majorana fermion operators and $u_{i}=\pm 1$ are static $Z_2$ gauge fields defined on links. It's straightforward to check that the commutation and anti-commutation relation between $Z_i X_{i+\bm{\delta}}$ is preserved under the above mapping:
\begin{equation}
\left\{
\begin{aligned}
\{T_{v_iv'_i},T_{v_jv'_j}\}=0,\quad  & \text{if } i=j\pm \bm{\delta}\ \\ 
{[T_{v_iv'_i},T_{v_jv'_j}]=0,}\quad & \text { otherwise }
\end{aligned}
\right. \longleftrightarrow 
\left\{\begin{array}{cl}
\{Z_iX_{i+\bm{\delta}},Z_jX_{j+\bm{\delta}}\}=0,\quad & \text{if } i=j\pm \bm{\delta} \\
{[Z_iX_{i+\bm{\delta}},Z_jX_{j+\bm{\delta}}]=0,}\quad & \text{ otherwise } 
\end{array}
\right.
% \left\{
% \begin{aligned}
% \{T_{v_iv'_i},T_{v_jv'_j}\}=0,\quad &\text{if } i=j\pm \bm{\delta}\\%v_i=v_j\text{ or } v'_i=v'_j\\
% [T_{v_iv'_i},T_{v_jv'_j}]=0,\quad &\text{ otherwise } 
% \end{aligned}
% \right.
\end{equation}


 The commutation and anti-commutation relation between $Z_iX_{i+\bm{\delta}}$ and $A_v$ (and similarly for $B_p=A_{v=p-\bm{\delta}}W_p$) is also preserved:
 \begin{equation}
\left\{
\begin{aligned}
\{T_{v_iv'_i},n^f_v\}=0,\quad  & \text{if } v\in\partial i \\ 
{[T_{v_iv'_i},n^f_v]=0,}\quad & \text { otherwise }
\end{aligned}
\right. \longleftrightarrow 
\left\{\begin{array}{cl}
\{Z_iX_{i+\bm{\delta}},A_v\}=0,\quad &\text{if } v\in\partial i \\
{[Z_iX_{i+\bm{\delta}},A_v]=0, }\quad &\text{otherwise } 
\end{array}
\right.
\end{equation}
 Besides, $Z_iX_{i+\bm{\delta}}$ and $A_v,B_p$ satisfy an additional relation, 
 \begin{equation}
 \prod_{i\in\partial p}Z_iX_{i+\bm{\delta}}=B_p A_{v=p+\bm{\delta}}
 \label{eq:additional}
 \end{equation}
 
 Under the mapping in \eqref{eq:map2}, the left hand side of \eqref{eq:additional} is mapped to $\prod_{i\in\partial p}T_{v_iv'_i}=(1-2n^f_{p-\bm{\delta}})(1-2n^f_{p+\bm{\delta}})\prod_{i\in\partial p}\hat{u}_{i}$. The right hand side of  \eqref{eq:additional} can be rewritten as $W_pA_{p-\bm{\delta}}A_{p+\bm{\delta}}$. Then \eqref{eq:additional} together with the \eqref{eq:map1}) determines the $Z_2$ flux configuration in the fermion model: 
 \begin{equation}
 \prod_{i\in\partial p}\hat{u}_{i}\longleftrightarrow A_{p-\bm{\delta}}B_{p}=W_p
 \label{eq:map3}
 \end{equation}
as expected.
\eqref{eq:map1,eq:map2,eq:map3} form the complete the dictionary of the fermionization procedure on an infinite lattice or a topologically trivial lattice. However, on the fermionic side, under periodic boundary condition, i.e., on a torus, there are additional $Z_2$ fluxes threading the two non-contractible cycles $\gamma_x,\gamma_y$ along the $x,y$ direction: $\hat{w}_{x,y}=\prod_{i\in\gamma_{x,y}}\hat{u}_i$. We need to figure out what is the counterpart of $\hat{w}_{x,y}$ on the toric code side. This can be done by using again the mapping in \eqref{eq:map2}, which leads to:
\begin{equation}
-\left(\prod_{i\in\gamma_{x,y}}\hat{u}_i\right)\prod_{i\in\gamma_{x,y}}(1-2n^f_{v_i}) \longleftrightarrow \prod_{i\in\gamma_{x,y}}Z_iX_{i+\bm{\delta}}
\end{equation}
By using \eqref{eq:map1}, we obtain
\begin{equation}
-\hat{w}_{x(y)}\longleftrightarrow \prod_{i\in\gamma_{x(y)}}A_{v_i}Z_iX_{i+\bm{\delta}}=\prod_{i\in\gamma_{x(y)}}Z_iX_{i-\bm{\delta}}\equiv\hat{W}^{\varepsilon'}_{\gamma_{x(y)}}.
\end{equation}
Indeed, $\hat{W}^{\varepsilon'}_{\gamma_x},\hat{W}^{\varepsilon'}_{\gamma_y}$ are also conserved quantities in the original model, $[\hat{W}^{\varepsilon'}_{\gamma_{x(y)}},H]=0$. 

In the end, we map the model in \eqref{eq:ZX} to a quadratic fermion model with static $Z_2$ gauge field:
\begin{equation}
H\leftrightarrow \tilde{H}=\sum_v(2n^f_v-1)\cdot(1+\hat{w}_p)-h_{xz}\sum_{\langle vv'\rangle}i\hat{u}_{vv'}\gamma_v\gamma'_{v'} 
\end{equation}
where $\hat{w}_p=\prod_{\langle vv'\rangle\in \partial p}\hat{u}_{vv'}$. We note that the above results can be viewed as a generalization of the 2d bosonization proposed in Ref.\cite{CHEN2018234}, where they construct the above mapping for fermion models without gauge field, so they add the restriction $\hat{W}_p=1$ to $H$. Also, they only consider the case in an infinite/ topologically trivial lattice. 

Thanks to the extensive number of conserved quantities, $[\hat{u}_{vv'},\tilde{H}]=[\hat{w}_{p/x/y},\tilde{H}]=0$, $\tilde{H}$ can be reduced to a free fermion model in each $Z_2$ flux sector $\{\hat{w}_p=w_p,\hat{w}_x=w_x,\hat{w}_y=w_y\}$, and thus can be easily solved. In the case $h_{xz}=0$, it is obvious that the ground state (the vaccum of $f$) lies in the zero flux sector $w_p=1$, and the lowest energy state in the four sectors with distinct $\{w_x=\pm 1,w_y=\pm 1\}$ has degenerate eigenenergy. This is just another viewpoint of the well-known topological degeneracy. 

Via numerical investigation we find that the ground state always stay in the sector with $w_p=1$, irrespective of the value of $h_{xz}$, so we will mainly restrict our discussion to this case. 
\begin{equation}
    \tilde{H}=4\sum_v n^f_v-h_{xz}\sum_{\langle vv'\rangle} f^\dagger_vf_{v'}+f_vf_{v'}+h.c.
\end{equation}
and ${w_{a}}=1,-1(a=x,y)$ corresponds to PBC and APBC along direction $a$, respectively. Then $\tilde{H}$ can be solved via Fourier transformation, $f_v=\frac{1}{\sqrt{L_xL_y}}\sum_{k_a=\frac{2n_a\pi}{L_a}} f_k e^{ik_x{v_x} +ik_yv_y}$, where $n\in \mathbb{Z}$ for PBC and $n\in \mathbb{Z}+\frac{1}{2}$ for APBC.

\begin{equation}
\tilde{H}=\sum_k (f^\dagger_k, f_{-k})\left(\begin{array}{cc}
	2-h_{xz}(\cos k_x+\cos k_y) & -2i(\sin k_x+\sin k_y) \\
	2i(\sin k_x+\sin k_y) & -2+h_{xz}(\cos k_x+\cos k_y)
\end{array} \right)\left(\begin{array}{c}
	f_k \\
	f^\dagger_{-k}
\end{array} \right)
\end{equation}

The dispersion of Bogoliubov quasiparticle excitation can be easily obtained:
\begin{equation}
\xi_k=\sqrt{[2-h_{xz}(\cos k_x+\cos k_y)]^2+4(\sin k_x+\sin k_y)^2},
\end{equation}
and the ground state energy is $E_g=-\sum_k\xi_k$. For $h_{xz}<1$, the spectrum is gapped and the ground energy is nearly degenerate (with an exponentially small splitting) for the 4 types of boundary conditions. This corresponds to the gapped topologically ordered phase of $H$. For $h_{xz}=1$, the gap closes at $k_x=k_y=0$, and remain closed for $h_{xz}>1$, with linear dispersion at two Dirac points $k_x=-k_y=\pm \arccos \frac{1}{h_{xz}}$. So for $h_{xz}>1$ the original model lies in a gapless spin liquid phase, reminiscent of the gapless phase of the Kitaev honeycomb model. In this phase the topological degeneracy is lifted by an algebraically small gap, but the ground state remains long-range entangled.  

\bibliography{supp_wzj}
\end{document}