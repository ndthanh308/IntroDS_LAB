\documentclass[aps,prl,twocolumn,superscriptaddress]{revtex4-1}
%\usepackage{mathbbold}
\usepackage{mathrsfs}
%\documentclass[aps,preprint,showpacs,superscriptaddress,endfloats]{revtex4}
\usepackage{epsfig}
\usepackage{graphicx}
\usepackage{amsfonts}
\usepackage[figuresright]{rotating}
\usepackage{amssymb}
\usepackage{amsmath}
\usepackage{dcolumn}
\usepackage{bm}
\usepackage{color}
\usepackage[colorlinks, citecolor=blue]{hyperref}
\hypersetup{linkcolor=magenta,urlcolor=blue,citecolor=blue,pdfstartview={FitH},urlcolor=blue}

%\usepackage{psfrag}
%\renewcommand{\baselinestretch}{2.0}


\def\avg#1{\langle#1\rangle}
\def\Re{\rm{Re}}
\def\Im{\rm{Im}}
\def\be{\begin{equation}} \def\ee{\end{equation}}
\def\bea{\begin{eqnarray}} \def\eea{\end{eqnarray}}
\def\PRB{Phys. Rev. B}
\def\PRA{Phys. Rev. A}
\def\PRL{Phys. Rev. Lett.}
\def\nn{\nonumber}
\def\pp{\parallel}
\def\sign{\text{sign}}
\def\Tr{\text{Tr}}
\def\tr{\text{tr}}
\def\gtd{\tilde{g}}
\def\ve{\varepsilon}
\newcommand{\ket}[1]{| #1 \rangle}
\newcommand{\bra}[1]{\langle #1 |}

%%% define  shorthand colors
\def\black{\color{black}}
\def\clr{\color{red}}
\def\clb{\color{blue}}
\def\clp{\color{purple}}
\def\clv{\color{violet}}
\def\clg{\color{green}}

\newcommand{\CIT} {Department of Physics and Jiangsu Laboratory of Advanced Functional Material, Changshu Institute of Technology, Changshu 215500, China}

\newcommand{\sjtu} {Key Laboratory of Artificial Structures and Quantum
Control, Department of Physics and Astronomy, Shanghai Jiao Tong University, Shanghai 200240, People's Republic of China}

\newcommand{\CSRC} {Beijing Computational Science Research Center, Beijing 100193, China}

\newcommand{\WQCASQC} {Wilczek Quantum Center, School of Physics and Astronomy, Shanghai Jiao Tong University, Shanghai 200240, China}

\newcommand{\wqctdli} {Wilczek Quantum Center, School of Physics and Astronomy and T. D. Lee Institute,
Shanghai Jiao Tong University, Shanghai 200240, China}
\newcommand{\tdli}{T. D. Lee Institute, Shanghai Jiao Tong University, Shanghai 200240, China}

\newcommand{\Pittsburgh} {Department of Physics and Astronomy, University of Pittsburgh, Pittsburgh, Pennsylvania 15260, USA}

\newcommand{\SUSTech}{
		Shenzhen Institute for Quantum Science and Engineering and Department of Physics, Southern University of Science and Technology, Shenzhen 518055, China
		}
		
\newcommand{\SRCQC}{Shanghai Research Center for Quantum Sciences, Shanghai 201315, China}

\begin{document}

\title{Intrinsic Mixed-state Topological Order Without Quantum Memory}

\author{Zijian Wang}
\thanks{These authors contributed equally to this work.}
\affiliation{Institute for Advanced Study, Tsinghua University, Beijing 100084,
People's Republic of China}
\author{Zhengzhi Wu}
\thanks{These authors contributed equally to this work.}
\affiliation{Institute for Advanced Study, Tsinghua University, Beijing 100084,
People's Republic of China}

\author{Zhong Wang}
\email{wangzhongemail@tsinghua.edu.cn}
\affiliation{Institute for Advanced Study, Tsinghua University, Beijing 100084,
People's Republic of China}




\begin{abstract}  
Decoherence is a major obstacle to the preparation of topological order in noisy intermediate-scale quantum devices. Here, we show that decoherence can also give rise to new types of topological order. Specifically, we construct two such examples by proliferating fermionic anyons in the two-dimensional toric code model and the Kitaev honeycomb model through certain local quantum channels. The resulting mixed states retain long-range entanglement, which manifests in the nonzero topological entanglement negativity, though the topological quantum memory is destroyed by decoherence. We argue that these properties are stable against perturbations. Therefore, the identified states represent a novel intrinsic mixed-state quantum topological order, which has no counterpart in pure states. 

\end{abstract}


\maketitle

{\it Introduction}.---
As long-range entangled quantum matter, topologically ordered phases have attracted extensive  attention in the past few decades \cite{wen1990topological,chen2010local,wen2017colloquium,savary2017spinliquid,zhou2017spinliquid,sachdev2018topological}. 
Recently, there is a growing number of theoretical proposals \cite{aguado2008creation,verresen2021prediction,piroli2021quantum,tantivasadakarn2021long,tantivasadakarn2022shortest,tantivasadakarn2023hierarchy,lee2022decoding,bravyi2022adaptive,lu2023mixed} as well as experimental evidence \cite{semeghini2021probing,satzinger2021realizing,google2023non,xu2023digital,iqbal2023topological,iqbal2023creation} showing that topological order (TO) can be prepared in current many-body quantum simulation platforms, such as superconducting qubit arrays, Rydberg atom arrays, and trapped ion systems, etc. A key feature of these noisy intermediate-scale quantum (NISQ) devices is the inevitable presence of decoherence, which renders the quantum state a mixed state \cite{preskill2018}. Substantial progress has been made in diagnosing nontrivial topological phases subject to decoherence \cite{dennis2002topological,mcginley2020fragility,deng2021stability,wang2021symmetry,de2022symmetry,ma2022average,lee2022symmetry,zhang2022strange,fan2023diagnostics,bao2023mixed,lee2023quantum,ma2023topological,su2023conformal,wang2023topologically}. Particularly, the decoherence-induced breakdown of topological quantum memory 
in the toric code model is investigated, and is related to the transition in the mixed-state topological order \cite{fan2023diagnostics,lee2023quantum}. 
The topological order therein is inherited from the pure-state counterpart, which is resistant to modest decoherence. Above certain critical error rate, the long-range entanglement is destroyed.  

We are interested in the following question: Other than destroying the pure-state topological order, can decoherence give rise to novel types of quantum topological order that are intrinsically mixed? The possibility of such intriguing scenario arises from new mechanisms of anyon proliferation provided by decoherence, distinct from anyon condensation in pure states. As a starting point, we explore such possibility in the context of $Z_2$ topological order \cite{anderson1973,read1989statistics,read1991largeN,wen1991mean,anderson1987RVB,senthil2000Z2,moessner2001dimer,kitaev2003fault}, which comprises three types of anyon excitations, $e,m,$ and $\varepsilon=e\times m$. In pure states, either $e$ or $m$ (both being self-bosons) can condense, leading to a topologically trivial Higgs/confined phase. Correspondingly, proliferation of $e,m$ anyons induced by decoherence also destroys long-range entanglement. In contrast, $\varepsilon$ anyons are self-fermions, and therefore they cannot condense in pure state; instead, strong fluctuation of $\varepsilon$ anyons typically leads to a gapless spin liquid, which remains long-range entangled. This motivates us to study the fate of $Z_2$ topological order when $\varepsilon$ anyons proliferate under decoherence. 

Specifically, we study the behavior of the toric code model \cite{kitaev2003fault} under local quantum channels that solely create $\varepsilon$ anyons. The topological quantum memory degrades to classical memory above certain deoherence threshold, aligning with previous studies. However, it turns out that the mixed state still possesses nontrivial quantum topological order even in the absence of quantum memory. To diagnose the mixed-state topological order, we employ the topological entanglement negativity (TEN) \cite{vidal2013TEN,castelnovo2013TEN,Ryu2016TEN,Ryu2016edge}, which is a natural generalization of the topological entanglement entropy (TEE) \cite{kitaev2006topological,levin2006detecting}. TEN has been utilized to probe topological order in thermal equilibrium \cite{hart2018entanglement, lu2020detecting,lu2022entanglement,lu2022characterizing} or under shallow-depth noise channels \cite{fan2023diagnostics}, effectively acting as a faithful indicator of topological quantum memory in those studied cases. However, we find that the TEN fails to reflect topological quantum memory in this scenario, and remains unchanged across the transition. Nevertheless, the nonzero TEN still points to the persistence of long-range entanglement, a hallmark of quantum topological order. Moreover, the absence of topological quantum memory indicates that the identified topological order has no pure-state counterpart and, therefore, is termed ``intrinsic mixed-state topological order" here. 

Expanding on this idea, we extend our analysis to the Kitaev honeycomb model \cite{kitaev2006anyons}, where we apply channels that incoherently proliferate the fermionic excitations while preserving the $Z_2$ gauge flux. Interestingly, we find that it also results in mixed states with classical memory above the decoherence threshold, yet still exhibiting nonzero TEN. Hence, we can identify these states as the same type of intrinsic mixed-state topological order as the previous example. 

   

 
 


{\it The model}.---In this letter we primarily study the 2D $Z_2$ toric code model on a square lattice:
$$H_{\text{TC}}=-\sum_v A_v-\sum_p B_p,\ A_v\equiv\prod_{i\in v}X_i,\ B_p\equiv\prod_{i\in p}Z_i,$$ where $X_i,Z_i$ are Pauli matrices. The ground states are 4-fold degenerate and can be used to encode two logical qubits, amenable to fault-tolerant quantum information processing. The tolerance of the topological quantum memory against local phase errors and bit flip errors has been investigated in \cite{dennis2002topological,fan2023diagnostics,lee2023quantum}, where errors are modeled as quantum channels,  $\mathcal{N}^z$ and $\mathcal{N}^x$ \footnote{$\mathcal{N}^{x(z)}$ are the composition of single-qubit quantum channels: $\mathcal{N}^{x(z)}=\prod_i\mathcal{N}_i^{x(z)},\mathcal{N}_i^{x(z)}[\cdot]\equiv (1-p_{x(z)})\cdot+p_{x(z)}X_i(Z_i)\cdot X_i(Z_i) $, where $p_{x},p_z$ are the error rates of bit flip and phase errors respectively.}. Since these channels incoherently create $e$, $m$ anyons, respectively, we denote the corresponding error-corrupted states as $\rho_e$, $\rho_m$. It has been shown that above the error threshold, the proliferation of either bosonic anyon ($e$ or $m$) would degrade the quantum memory to classical memory, accompanied by a sudden drop of TEN from $\log 2$ to $0$. This motivates us to investigate the incoherent proliferation of the fermionic $\varepsilon$ anyons of the $Z_2$ topological order, which can be realized by the following 2-qubit quantum channel: 


\begin{equation}
\mathcal{N}^{\varepsilon}=\prod_i \mathcal{N}^{\varepsilon}_i,\ \mathcal{N}^\varepsilon_i[\cdot]:=(1-p_\varepsilon)\cdot+p_\varepsilon Z_iX_{i+\bm{\delta}}\cdot X_{i+\bm{\delta}}Z_i,
\end{equation}
where $\bm{\delta}=(\frac{1}{2},-\frac{1}{2})$ (the lattice constant is taken to be 1) and $0<p_\varepsilon<\frac{1}{2}$. In this way $\rho_0$ is turned into a mixed state, $\rho_\varepsilon=\mathcal{N}^\varepsilon[\rho_0]$. As depicted in Fig.\ref{fig:fig1}(b), $\mathcal{N}^\varepsilon$ exclusively creates $\varepsilon$ anyons. In contrast, certain other types of errors, like the Pauli-Y errors, locally create pairs of $\varepsilon$ anyons as well, but globally they also produce $e$ and $m$ anyons, resulting in completely different outcomes. In the following sections, we demonstrate that this simple model surprisingly realizes an exotic intrinsic mixed-state topological order through analytical exact investigations of its topological memory, topological entanglement negativity, and anyon superselection sectors.  
% Figure environment removed

{\it Breakdown of quantum memory}.---Under the $\mathcal{N}^\varepsilon$ channel, the mixed state undergoes an error-induced transition, similar to the case with bit flip and phase errors. Such transitions can only be probed by information quantities nonlinear in the density matrix, such as the von Neumann entropy $S=-\Tr(\rho_{\varepsilon}\log \rho_{\varepsilon})$ and the coherent information $I_c$ \cite{schumacher1996quantum,schumacher2001approximate,fan2023diagnostics}. 
We find that $S$ can be mapped to the free energy of the random bond Ising model (RBIM) along the Nishimori line. At a critical error rate  $p_c\approx 0.109$, the RBIM undergoes a ferromagnet-to-paramagnet phase transition, with an abrupt drop of coherent information, which determines the threshold where the topological quantum memory is damaged beyond recovery. The details of the mappings and calculations of $S$ and $I_c$ can be found in \cite{SupMat}. Nevertheless, we note that $\rho_\varepsilon$ still retains classical memory for $p_\varepsilon>p_c$. Suppose $\rho_0=\ket{\psi_0}\bra{\psi_0}$ is an eigenstate of the logical operator $W_{\gamma_{x,y}}=\prod_{i\in \gamma_{x,y}}Z_{i}X_{i-\bm{\delta}}$, where $\gamma_{x},\gamma_{y}$ are two non-contractible loops along the torus, then $\rho_\varepsilon$ always stays in the same eigenspace of $W_{\gamma_{x,y}}$ under the quantum channel, as $[W_\gamma,Z_iX_{i+\bm{\delta}}]=0$.


{\it Topological entanglement negativity}.---Based on the above analysis, it may seem that $\rho_\varepsilon$ closely resembles $\rho_e$ and $\rho_m$. However, surprisingly, we demonstrate below that even when the quantum memory breaks down for $p_\varepsilon > p_c$, $\rho_\varepsilon$ retains long-range entanglement with a nonzero TEN, indicating the emergence of a distinct quantum topological order.
 
To evaluate the logarithmic entanglement negativity of $\rho_\varepsilon$, we consider the bipartition $A\cup B, B=\Bar{A}$ of the system:
\begin{equation}
\varepsilon_A(\rho_\varepsilon) \equiv \log ||\rho_\varepsilon^{T_A}||1 = \varepsilon_B(\rho\varepsilon),
\end{equation}
where $T_A$ denotes the partial transpose of $\rho_\varepsilon$ in subregion A, and $||\cdot||_1$ represents the trace norm. As an entanglement monotone, logarithmic negativity is commonly used to quantify quantum entanglement in mixed states, excluding the contribution from classical correlation \cite{separability,horodecki19961,zyczkowski1998volume,vidal2002computable}. As such, it is considered a natural generalization of entanglement entropy in pure states.

For convenience, we take the initial state $\rho_0$ to be the maximally mixed state of the toric-code ground-state subspace:

\begin{equation}
\rho_0=\frac{1}{4}\prod_v\frac{1+A_v}{2}\prod_p\frac{1+B_p}{2}.
\end{equation}
We denote the groups generated by $A_v(B_p)$ as $G_{x(z)}$. Each group element $g_{x(z)}$ corresponds to a loop configuration on the dual lattice (original lattice), as shown in Fig.\ref{fig:fig1}(e), (f). Thus, $\rho_0$ can be represented by an equal weight expansion of loop configurations:
\begin{equation}
\begin{aligned}
    \rho_0=\frac{1}{2^N}\sum_{g_x\in G_x}\sum_{g_z\in G_z}g_xg_z=\frac{1}{2^N}\sum_{g\in G\equiv G_x\times G_z}g.
\end{aligned}
\end{equation}

The effect of $\mathcal{N}^\varepsilon$ is to introduce loop tension. Specifically, for a given loop $g=g_xg_z$, $\mathcal{N}^\varepsilon$ assigns a weight $1-2p_\varepsilon$ to each segment where $g_x$ and $g_z$ does not coincide (up to a shift by $\bm{\delta}$). Consequently, we have,
\begin{equation}
\rho_\varepsilon=\mathcal{N}^\varepsilon[\rho_0]=\frac{1}{2^N}\sum_{g\in G} (1-2p_\varepsilon)^{l_g}g,
\end{equation}
where $l_g$ is the length of segments where $g_x$ and $g_z$ do not coincide. In Fig.\ref{fig:fig1}(e) we illustrate how to count such segments, and in Fig.\ref{fig:fig1}(f) we give an exmple of a tensionless loop.

Now we take the partial transpose for subregion $A$. For convenience we assume $A$ is simply connected. We denote $g=g_A g_B$, where $g_{A(B)}$ is the restriction of operator $g$ to subregion $A(B)$.
\begin{equation}
\begin{aligned}
\rho_\ve^{T_A}=\frac{1}{2^N}\sum_{g\in G}(1-2p_\varepsilon)^{l_g}y_A(g)g,
\end{aligned}
\end{equation}
where $y_A(g)=1(-1)$ when $g_{xA}$ and $g_{zA}$ commute (anti-commute). 
To calculate $\varepsilon_A(\rho_\ve)$, we employ the replica trick, that is, we first compute the $2n^{th}$ Renyi negativity $\varepsilon^{(2n)}_A(\rho_\ve)\equiv\frac{1}{2-2n}\log \frac{\Tr(\rho^{T_A}_\ve)^{2n}}{\Tr\rho_\ve^{2n}}$ and then take the replica limit $2n\rightarrow 1$.

Remarkably, it turns out that the final result of Renyi negativity is rather simple, and is independent of $p_c$:
\begin{equation}
\varepsilon^{(2n)}_A(\rho)=(|\partial A|-1)\log 2,\forall n.
\end{equation}
This result also holds in the replica limit $2n\rightarrow 1$, yielding $\varepsilon_A(\rho)=(|\partial A|-1)\log 2$. 

The first term is the usual area law part of the entanglement negativity, where $|\partial A|$ is the length of the boundary of subregion $A$. The second term, known as topological entanglement negativity, is a generalization of TEE. A nonzero value of TEN signals nontrivial topological order, as it arises solely from long-rang entanglement. Hence, it has been used to diagnose topological order in both finite-temperature systems and states subject to local errors. Notably, the logarithmic negativity remains exactly the same as that of the ground state, irrespective of the value of $p_\varepsilon$. This stands in sharp contrast to the case with single-qubit errors. We emphasize that the persistence of long-range entanglement signifies a genuine quantum topological order, which distinguishes $\rho_{\varepsilon}$ from the so-called classical topological order, a concept raised in the study of finite-temperature topological order \cite{castelnovo2007classical,castelnovo2007finiteT,castelnovo2008topological,lu2020detecting}. States with classical TO also have topological classical memory, but zero TEN. One typical example of classical TO is the low-temperature phase of the 3D toric code model, where the point-like $e$ anyons proliferate while the loop excitations do not. In this sense, $\rho_e$ and $\rho_m$ (above the error threshold) also have classical TO. $\rho_\varepsilon$ is qualitatively different from these known examples. 

{\it Anyon superselection sectors}.---As mentioned above, the phase transition at $p_\varepsilon=p_c$ is driven by the proliferation of $\varepsilon$ anyons, so we expect that $\varepsilon$ anyons are no longer well-defined excitations after the transition. In this section we give a more quantitative analysis from this aspect. We denote the string operators creating $\alpha$ $(\alpha=e,m,\varepsilon)$ anyons at the ends of the string as $w_\alpha$. To determine whether these anyon excitations remain well-defined under the channel $\mathcal{N}^\varepsilon$, we investigate whether $\rho^\alpha_{\varepsilon}\equiv\mathcal{N}^\varepsilon[w_\alpha\rho_0 w_\alpha]$ is really a distinct state from $\rho_\varepsilon$. Quantitatively, we calculate the relative entropy:
\begin{equation}
D(\rho_\varepsilon||\rho_\varepsilon^\alpha)\equiv\Tr(\rho_\varepsilon\log\rho_\varepsilon)-\Tr(\rho_\varepsilon\log\rho_\varepsilon^\alpha),
\end{equation}
and examine whether it diverges as the length of $w_\alpha$ approaches infinity \cite{fan2023diagnostics}, which is proposed as a generalization of Fredenhagen-Marcu order parameter for ground states \cite{fredenhagen1983charged,gregor2011diagnosing}.

It turns out that, for $p_\varepsilon<p_c$, $D(\rho_\varepsilon||\rho_\varepsilon^\alpha)$ diverges for all three types of anyons, while for $p_\varepsilon>p_c$, $D(\rho_\varepsilon||\rho_\varepsilon^\ve)$ becomes finite , indicating that $\varepsilon$ anyons are no longer well-defined, in agreement with our expectation. Additionally, although $D(\rho_\ve||\rho_\ve^{e(m)})$ is divergent, $e,m$ cease to be distinct excitations since $e\times \varepsilon=m$. Consequently, there are only two superselection sectors after the transition: $1=\varepsilon,e=m$. Surprisingly, we find that nontrivial anyon braiding can still be performed \cite{SupMat},  which is enabled by the long-range entanglement of $\rho_\varepsilon$. 

{\it Comparison to anyon condensation and gapless spin liquid}.---From the previous analysis, we see that although $\rho_\varepsilon$ only exhibits classical memory (for $p_\varepsilon>p_c$), it is fundamentally distinct from $\rho_e$ and $\rho_m$. To gain deeper insight into this counter-intuitive result, we draw comparisons between the error-induced anyon proliferation and anyon condensation in pure states. Instead of applying local quantum channels, we analyze the case when the $X_i,Z_i,Z_iX_{i+\bm{\delta}}$ terms are directly introduced into the toric code Hamiltonian: 
\begin{equation}
H=H_{\text{TC}}-\sum_ih_xX_i-h_zZ_i-h_{xz}Z_iX_{i+\bm{\delta}},
\end{equation} 

The ground state phase diagram for $h_{xz}=0$ has been extensively studied. For sufficiently large $h_z$ or $h_x$, it leads to the condensation of $e$ or $m$ anyons, respectively, resulting in the destruction of long-range entanglement. Analogously, the local $Z,X$ errors induce $e,m$ anyon proliferation in an incoherent manner, which also destroys the long-rang entanglement. Despite the similarities between decoherence-induced anyon proliferation and anyon condensation, there are still noteworthy distinctions. In mixed states, incoherent proliferation of either $e$ or $m$ does not completely trivialize the phase but rather leads to classical TO. Contrarily, in pure states, condensation of either $e$ or $m$ already leads to the trivial Higgs/confined phase. 

The distinction becomes much more significant for $\varepsilon$ anyons. As fermions, they cannot condense in pure states. Actually, when $h_{xz}$ is sufficiently large ($h_{x,z}=0$), corresponding to large fluctuations of $\varepsilon$ particles, the system enters a gapless spin liquid phase, where $\varepsilon$ particles form a $p-$wave superconductor with conic dispersions \cite{SupMat}. The gapless spin liquid phase remains long-range entangled, but it loses well-defined topological degeneracy. In this perspective, $\rho_\varepsilon$ can be understood as the mixed-state analogue of the gapless spin liquid phase. Then it is reasonable to find $\rho_{\varepsilon}$ to be long-range entangled while $\rho_{e},\rho_{m}$ are not. However, there are notable differences between $\rho_{\varepsilon}$ and the gapless spin liquid phase. Gapless phases typically exhibit some critical behavior, including algebraically decaying correlation functions and subleading logarithmic corner contributions to entanglement entropy/negativity \cite{corner2006,corner2009,corner2015}. However, since $\rho_{\varepsilon}$ is obtained by applying local quantum channels on a gapped phase, no power-law corrleation of local operators can be generated. Moreover, as we have seen, the subleading term in the entanglement negativity $\varepsilon_A(\rho_\varepsilon)$ does not depend on the geometry of $A$, so there is no obstruction to define TEN for $\rho_{\varepsilon}$. In this sense, $\rho_{\varepsilon}$ also retains some properties of gapped topological order. Therefore, $\rho_{\varepsilon}$ indeed represents a new type of topological order that is only possible in mixed states. 


{\it Generalization to the Kitaev honeycomb model}.---The above construction can be generalized to other lattice models or topological order with fractionalized fermionic excitations. Here we take the Kitaev honeycomb model as another example: $H=-J_{x} \sum_{x \text {-bonds }} \sigma_{j}^{x} \sigma_{k}^{x}-J_{y} \sum_{y \text {-bonds }} \sigma_{j}^{y} \sigma_{k}^{y}-J_{z} \sum_{z \text {-bonds }} \sigma_{j}^{z} \sigma_{k}^{z}$. This model can be exactly solved by mapping it to Majorana fermions coupled to static $Z_2$ gauge fields. In the parameter regime $|J_z|>|J_x|+|J_y|$,  the ground state is an Abelian $Z_2$ spin liquid (the same phase as the toric code model). By increasing $J_x,J_y$, it is known that the fluctuation of Majorana fermions would drive the system to a gapless Kitaev spin liquid \cite{kitaev2006anyons}. Thus it is natural to ask whether a similar intrinsic mixed-state quantum TO can be obtained via quantum channels. 

For this purpose, we apply the following channel, which creates fermion excitations but no $Z_2$ gauge flux:
\begin{equation}
\begin{aligned}
&\rho_f=\mathcal{N}^X\circ\mathcal{N}^Y\circ\mathcal{N}^Z[\rho_0],\quad \mathcal{N}^\alpha=\prod_{\langle ij\rangle\in \alpha\text{-bonds}}\mathcal{N}^\alpha_{\langle ij\rangle},\\
 &\mathcal{N}^\alpha_{\langle ij\rangle}[\rho_0]= p\sigma^\alpha_i\sigma^\alpha_j\rho_0\sigma^\alpha_j\sigma^\alpha_i +(1-p)\rho_0.
\end{aligned}
\end{equation}
We find that $\rho_f$ has similar properties as $\rho_\varepsilon$. Above some error threshold $p_c$, the quantum memory is reduced to classical memory, while $\rho_f$ still possess the same TEN as $\rho_0$; the number of superselection sectors is reduced from 4 to 2, restoring the translation symmetry from weak symmetry breaking in the Abelian phase. We can also take $\rho_0$ in the gapless phase or the non-Abelian phase and apply the same channel. At $p=\frac{1}{2}$, they all result in the same mixed state. More detailed analysis of this model is given in \cite{SupMat}. We note that a similar model in the context of Lindblad equations was discussed in \cite{hwang2023mixed}. 

% Figure environment removed

{\it Discussion}.---Although in our contruction we need to use specific two-qubit channels that looks a bit unconventional, $\rho_{\ve}$ represents a new type of topologically ordered phase, instead of a fine-tuned exception. We can consider the case when single-qubit phase errors are also present: ${\rho}_{\varepsilon,e}=\mathcal{N}^z[\rho_\varepsilon]$, with error rate $p_z$. By mapping to two decoupled RBIMs \cite{SupMat}, we demonstrate that for small $p_z$, ${\rho}_{\varepsilon,e}$ stays in the same phase as $\rho_\varepsilon$, while for $p_z>p_c\approx 0.109$, the state undergoes another transition to the trivial phase, with no memory and zero TEN \cite{SupMat}. 

Furthermore, it is worth noting that the construction of such mixed states is experimentally feasible in current NISQ devices \cite{satzinger2021realizing}. For example, one can realize a noisy toric code model by implementing incomplete error correction, where only the error syndrome with $A_vB_{p+\bm{\delta}}=-1$ is corrected after the syndrome measurement using string operators $w_e$ or $w_m$. This partial error correction would lead to a mixed state similar to $\rho_\varepsilon$.

Our work introduces a promising mechanism for creating novel topologically ordered phases in mixed states. One of our observations is that while the ways of anyon condensation are limited for pure states, anyon proliferation in mixed states can occur in more general ways, offering new possibilities for topological order. In the two models studied in this letter, we propose a new topological order arising from incoherent proliferation of fermionic anyons in $Z_2$ topological order. We conjecture here that proliferation of anyons with nontrivial topological spin ($\theta\neq 1$), e.g., semions in the doubled semion model, may generally give rise to exotic types of mixed-state topological order. Moreover, in our construction, we start from a topologically ordered state and get its descendants via noise channels. It is also tempting to find systematic ways to prepare mixed-state topological order from short-range entangled mixed states, for example, by measurng mixed-state symmetry protected topological order \cite{ma2022average,lee2022symmetry,zhang2022strange,ma2023topological,lu2023mixed}.  

\begin{acknowledgements}
{\it Acknowledgments}.---We thank Yingfei Gu, Zhen Bi, and He-Ran Wang for helpful discussions. We are especially grateful to Ruihua Fan for many pieces of valuable advice and feedback on the manuscript. This work is supported by NSFC under Grant No.
12125405.
\end{acknowledgements}
\bibliography{wzj}

\end{document}
