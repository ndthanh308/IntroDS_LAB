\documentclass[fleqn,usenatbib]{mnras}
\usepackage{newtxtext,newtxmath}
\usepackage[T1]{fontenc}
\usepackage{ae,aecompl}
\usepackage{graphicx}
\usepackage{subfig}
\usepackage{ulem}
\usepackage{booktabs}
\usepackage{xcolor}

% Comment:
\newcommand{\com}[1]{{\sf\color[rgb]{0,0,1}{[#1]}}}
% Marking old and new text:
\newcommand{\new}[1]{{\color[rgb]{0.7,0,0}#1}}
\newcommand{\neu}[1]{{\color[rgb]{0.0,0.7,0}#1}}
\newcommand{\old}[1]{{\color[rgb]{0.5,0.5,0.5}\sout{#1}}}
%%%%%%%%%%%%%%%%%%%%%%%%%%%%%%%%
\newcommand{\bk}[1]{\textcolor{magenta}{BK: #1}}
\newcommand{\JAP}[1]{\textcolor{teal}{JAP: #1}}
%%%%%%%%%
\title[DMANS in the light of observational constraints]{Dark Matter Admixed Neutron Star in the light of HESS J1731-347 and PSR J0952-0607}
\author[Routaray et al.]{
Pinku Routaray$^{1}$,
H C Das $^{2}$,
\thanks{E-mail: XXXX@gmail.com},
Jeet Amrit Pattnaik$^{2}$,
Bharat Kumar$^{1}$,
\thanks{E-mail: kumarbh@nitrkl.ac.in},
\\
$^{1}$ Department of Physics and Astronomy, National Institute of Technology, Rourkela 769008, India\\
$^{2}$ Institute of Physics, Sachivalaya Marg, Bhubaneswar 751005, India\\}
%%%%%%%%%%%%%%%%
\begin{document}
\maketitle
\date{\today}
%%%%%%%%%%%%%%%%    
\begin{abstract}
This study explores the implications of Dark Matter in Neutron Stars (DMANS) by focusing on two specific astronomical objects: HESS J1731-347 and PSR J0952-0607. Varying the Fermi momentum k$f^{\rm DM}$ of DM, the study analyzes the EOS for the INRS model with and without DM. Results show the robustness of the model, with most EOS curves within chiral Effective Field Theory bounds.  Our model predicts a maximum mass of $2.343 \ M_\odot$ for PSR J0952-0607, satisfying NICER bounds. The analysis suggests HESS J1731-347 could be a DMANS. Constraints on DM within NSs are established, and tidal deformability lies within GW event limits. Nonradial $f$-mode oscillations increase with DM, concluding low mass stars pulsate at higher frequencies. 
\end{abstract}
%%%%%%%%%%%%%%%%
%%%%%%%%%%%%%%%%%%%%%%
\section{Introduction}

In the last decades, significant progress has been achieved in understanding the properties of compact stars, such as mass, radius, and tidal deformability ~\citep{Hinderer_2008}. This advancement enhances the exploration of the highly dense neutron star (NS) and, at the same time, it restricts different theories. One pivotal breakthrough came from the observation of gravitational waves (GWs) from the merger of two NSs in the GW170817 event ~\citep{Abbott_2017}. This observation provided valuable insights into the masses of binary systems and their tidal deformabilities. Another binary coalescence event, GW190814 ~\citep{RAbbott_2020}, involving a black hole and a compact object with a mass range of 2.50-2.67 $M_\odot$, has raised a compelling argument about the nature of the secondary compact object-whether it is the lightest black hole or a massive NS.

In addition to GWs observations, there are concurrent measurements of mass and radius of PSR J0030+0451 have been made through pulse profile modeling of X-ray emissions from hot spots on the surface of the isolated NS observed by Neutron Star Interior Composition Explorer (NICER) telescope ~\citep{Riley-nicer_2019, Miller-nicer_2019}. The discovery of the Galactic NS (PSR J0952-0607), named the ``Black widow pulsar", has been unveiled in the Milky Way disc. It has been distinguished as the fastest and heaviest of its kind, having a mass of $M=2.35\pm0.17 \ M_\odot$ ~\citep{Romani_2022}. Recently, a central compact object within the supernova remnant HESS J1731-347 was investigated through the analysis of its X-ray spectrum, revealing it to be a low-mass object ~\citep{HESS_2022}. Remarkably, its mass has been determined to be $M=0.77^{+0.20}_{-0.17} \ M_\odot$ having radius $10.4^{+0.86}_{-0.78}$ km, and it opens a debate whether this object is either the lightest NS or a strange star with a more exotic equation of state (EOS). 

Several studies have already hypothesized the nature of the HESS object. For example, Horvath {\it et al.} ~\citep{Horvath-hess_2023} employed the quark model and lead to the conclusion that the object might be a light strange star. The notion of a strange nature for HESS J1731-347 was also put forth in Refs.  ~\citep{clemente_2023hess, hcdas_2023hess, rather_2023quark}. On the other hand, some works proposed it might be the lightest NS. For example, Kubis {\it et al.} incorporated the $\omega-\delta$ cross-coupling into their established a new relativistic mean-field (RMF) model by varying the slope parameter, which satisfied the HESS constraints \citep{kurbis_2019_prcmodel,kubis_2023hess}. Additionally, Huang {\it et al.} ~\citep{huang_2023hess} explored the possibility of the lightest NS by considering the RMF model with tensor coupling. A comprehensive study by Sagun et al. ~\citep{sagun_2023hess} explored the nature of the CCO in HESS as baryonic, strange, or even admixed with dark matter (DM), particularly a two-fluid approach. Hence, there is still a window to explore the nature of the HESS object. 

In contrast to the aforementioned efforts, this study introduces a novel approach to studying the CCO in HESS J1731-347, specifically focusing on the inclusion of a DM admixed NS (DMANS) within a single fluid framework. This study begins by formulating a new RMF model called ``INRS" ({\bf I}OP-{\bf N}IT{\bf R}-{\bf S}OA), which is formulated to reproduce the mass ($M = 2.35 \ M_\odot$) of PSR J0952-0607 ~\citep{Romani_2022}. In addition, we also incorporate the DM inside the NS, which satisfies several observational constraints that depends on the amount of DM present inside it. These constraints include the $2 \ M_\odot$ upper limit, as well as constraints imposed by NICER  ~\citep{Riley-nicer_2019, Miller-nicer_2019}, the multimessenger observation for canonical radius as discussed in ~\cite{Capano_2020}, the stringent constraint imposed by HESS J1731-347 ~\citep{HESS_2022}, and the tidal deformability constraints given by GW170817 ~\citep{Abbott_2017} and GW190814 ~\citep{RAbbott_2020} events. By considering these constraints, we want to fix the amount of DM that could exist inside the NS. The oscillation frequency of the NS is also a crucial quantity that can be used to explore the star's internal structure ~\citep{Andersson_1996, athul_2022, souhardya_2023, harishprd_2021, pinku_prd_2023, Bikram_2021-fmode}. Therefore, we also calculate the non-radial oscillation frequency within the relativistic Cowling framework.

%%%%%%%%%%%%%%%%%%%%%%
\section{The Model}
\subsection{RMF model}
Within the framework of relativistic mean field (RMF) formalism, we develop a new model named "INRS" (\textbf{I}OP-\textbf{N}IT\textbf{R}-\textbf{S}OA). The Lagrangian density of the RMF model is given by ~\citep{FURNSTAHL_1996, Frun_1997, singh_2014, Kumar_2017, Kumar_2018},
%%%%%%%%%%%%%%%%
\begin{eqnarray}
{\cal L}_{\rm nucl.} & = &  \sum_{\alpha=p,n} \bar\psi_{\alpha}
\Bigg\{\gamma_{\mu}\bigg(i\partial^{\mu}-g_{\omega}\omega^{\mu}-\frac{1}{2}g_{\rho}\vec{\tau}_{\alpha}\!\cdot\!\vec{\rho}^{\,\mu}\bigg)
\nonumber \\
&&
-\bigg(M_{\rm nucl.} 
-g_{\sigma}\sigma\bigg)\Bigg\} \psi_{\alpha} +\frac{1}{2}\partial^{\mu}\sigma\,\partial_{\mu}\sigma-\frac{1}{2}m_{\sigma}^{2}\sigma^2
\nonumber \\
&&
+\frac{\zeta_0}{4!}g_\omega^2(\omega^{\mu}\omega_{\mu})^2-\frac{\kappa_3}{3!}\frac{g_{\sigma}m_{\sigma}^2\sigma^3}{M_{nucl.}}-\frac{\kappa_4}{4!}\frac{g_{\sigma}^2m_{\sigma}^2\sigma^4}{M_{nucl.}^2}
\nonumber\\
&&
+\frac{1}{2}m_{\omega}^{2}\omega^{\mu}\omega_{\mu}-\frac{1}{4}W^{\mu\nu}W_{\mu\nu}
+\frac{1}{2}m_{\rho}^{2}\bigg(\vec\rho^{\mu}\!\cdot\!\vec\rho_{\mu}\bigg)
\nonumber\\
&&
-\frac{1}{4}\vec R^{\mu\nu}\!\cdot\!\vec R_{\mu\nu} - \Lambda_{\omega}g_{\omega}^2g_{\rho}^2\big(\omega^{\mu}\omega_{\mu}\big)\big(\vec\rho^{\,\mu}\!\cdot\!\vec\rho_{\mu}\big)\, .
\label{RMF}
\end{eqnarray}
%%%%%%%%%%%%%%

The INRS model employed in this study was successfully calibrated using the simulation annealing method, allowing for precise determination of its various parameters of the RMF model, as mentioned above. The detailed parameterization procedure has been extensively described in the references ~\citep{BKAgrawal_2005, BKAgrawal_2006, Kumar_2017, Kumar_2018}. To achieve this calibration, coupling constants and nuclear matter properties were carefully fitted by employing experimental data on binding energy and charge radii for $^{16}$O, $^{40}$Ca, $^{48}$Ca, $^{68}$Ni, $^{90}$Zr, $^{100}$Sn, $^{132}$Sn, and $^{208}$Pb nuclei. The resulting values of these calibrated parameters are summarized in Table \ref{tab:parameter}. This rigorous calibration process ensures that the INRS model accurately represents the properties of neutron stars, both with and without the presence of dark matter, providing a reliable and consistent framework for our investigation.

The above Lagrangian density can be solved in mean field approach using beta equilibrium condition and from stress energy-momentum tensor we can calculate the energy density (${\cal{E}}_{\rm NS}$) and pressure ($P_{\rm NS}$) for the NS ~\citep{Kumar_2018}.
\begin{align}
{\cal{E}}_{\rm NS} & = &  \frac{2}{(2\pi)^{3}}\int d^{3}k E_{i}^\ast (k)+\rho  W+
\frac{ m_{s}^2\Phi^{2}}{g_{s}^2}\Bigg(\frac{1}{2}+\frac{\kappa_{3}}{3!}
\frac{\Phi }{M} + \frac{\kappa_4}{4!}\frac{\Phi^2}{M^2}\Bigg)
\nonumber\\
&&
 -\frac{1}{2}m_{\omega}^2\frac{W^{2}}{g_{\omega}^2}-\frac{1}{4!}\frac{\zeta_{0}W^{4}} {g_{\omega}^2}+\frac{1}{2}\rho_{3} R 
 -\frac{1}{2}\frac{m_{\rho}^2}{g_{\rho}^2}R^{2}-\Lambda_{\omega}  (R^{2}\times W^{2})
\label{enrns}
\end{align}
\begin{eqnarray}
P_{\rm NS} & = &  \frac{2}{3 (2\pi)^{3}}\int d^{3}k \frac{k^2}{E_{i}^\ast (k)}- \frac{m_{s}^2\Phi^{2}}{g_{s}^2}\Bigg(\frac{1}{2}+\frac{\kappa_{3}}{3!}
\frac{\Phi }{M}+ \frac{\kappa_4}{4!}\frac{\Phi^2}{M^2}  \Bigg)\nonumber\\ 
& & +\frac{1}{2}m_{\omega}^2\frac{W^{2}}{g_{\omega}^2} +\frac{1}{4!}\frac{\zeta_{0}W^{4}}{g_{\omega}^2} +\frac{1}{2} \frac{m_{\rho}^2}{g_{\rho}^2}R^{2}+\Lambda_{\omega} (R^{2}\times W^{2})
\label{presns}
\end{eqnarray}
Where $\Phi$, $W$ and $R$ are the fields associated with $\sigma$, $\omega$ and $\rho$ mesons respectively.
%%%%%%%%%%%%%%
%%%%%%%%%%%%%%
\begin{table*}
\centering
\caption{Different parameters of "INRS" model is shown along with its nuclear matter properties. Mass of the $\sigma$,$\omega$ and $\rho$ mesons are presented in MeV.}
\renewcommand{\tabcolsep}{0.04cm}
\renewcommand{\arraystretch}{1.0}
\begin{tabular}{ccccccccccc}
\hline \hline
Parameter & $m_\sigma$ &  $m_\omega$   & $m_\rho$   & $g_\sigma$   & $g_\omega$   &$g_\rho$   & $\kappa_3$   &$\kappa_4$  & $\zeta_0$   & $\Lambda_{\omega}$\\
\hline
 INRS & 492.988 & 782.5 & 763.0  & 9.869  & 12.682  & 10.296  & 1.724  & -5.764  & 1.189  & 0.0156 \\
\hline
% &&&&& NM Parameters &&&&& \\ \hline 
$\rho_0$ (fm$^{-3}$) &  & $\mathcal{E}_0$ (MeV) &  &$K$ (MeV)&  &$J$ (MeV) & &$L$ (MeV) & & \\ \hline
0.155 &  &-16.34 &  &225.12 &  &36.094 & &78.474 & &  \\
\hline \hline
 % Expt. & $0.148-0.185^{[a]}$ & $-15 - - 17^{[a]}$  & $220-260^{[b]}$ & $30.20-33.70^{[c]}$ & $35.00-70.00^{[c]}$ \\
\end{tabular}
\label{tab:parameter}
\end{table*}
%%%%%%%%%%%%%
\subsection{Model for the NS with DM admixed}

The exploration of the true nature of DM remains an ongoing challenge in the field of cosmology. Over the course of many years since the Big Bang, numerous DM candidates have been proposed, with weakly interacting massive particles (WIMPs) gaining significant popularity due to their thermal relic nature. In our study, we focus on a specific class of WIMPs, known as non-annihilating WIMPs or neutralinos, which we consider a potential candidate for dark matter. These neutralinos are hypothesized to have already accreted within the NS due to their high baryon density and enormous gravitational potential ~\citep{Kouvaris_2011, Goldman_1989, arpan_2019}.


To elucidate the nature of DMANS, we construct a Lagrangian density within the framework of the standard model (SM). This Lagrangian density accounts for the coupling between baryons and dark matter, involving Higgs exchange. The formulation is influenced by previous works ~\citep{Grigorious_2017, arpan_2019, harishmnras_2020, harishjcap_2021, harishprd_2021, pinku_prd_2023, pinku_nitr_dm_2023} and is as follows:
%%%%%%%%%%%%%%%%
\begin{eqnarray}
{\cal{L}}_{\rm DM} & = & \bar \chi \left[ i \gamma^\mu \partial_\mu - M_\chi + y h \right] \chi +  \frac{1}{2}\partial_\mu h \partial^\mu h  \nonumber\\
& &
- \frac{1}{2} M_h^2 h^2 + f \frac{M_{\rm nucl.}}{v} \bar \varphi h \varphi \, , 
\label{eq:LDM}
\end{eqnarray}
%%%%%%%%%%%%%%
The symbols $\varphi$ and $\chi$ represent the nucleonic and dark matter wave functions, respectively. The Higgs field is denoted by the symbol $h$. The masses $M_\chi$ and $M_h$ represent the neutralino mass and Higgs mass, respectively. The coupling constants between the DM and SM Higgs Bosons are denoted as $y$. Consequently, $f M_{\rm nucl.}/v$ denotes the nucleon-Higgs field effective Yukawa coupling with proton-Higgs form factor $f$ and vacuum expectation of Higgs $v$. 

The above Lagrangian can be solved to obtain the EOS for the DM and is given by ~\citep{Grigorious_2017,harishmnras_2020,pinku_prd_2023},
%%%%%%%%%%%%%%%%
\begin{eqnarray}
{\cal{E}}_{\rm DM} = \frac{2}{(2\pi)^{3}}\int_0^{k_f^{\rm DM}} d^{3}k \sqrt{k^2 + (M_\chi^\star)^2 } + \frac{1}{2}M_h^2 h_0^2 \, ,
\label{eq:edm}
\end{eqnarray}
%%%%%%%%%%%%%%%%
\begin{eqnarray}
P_{\rm DM} = \frac{2}{3(2\pi)^{3}}\int_0^{k_f^{\rm DM}} \frac{d^{3}k \hspace{1mm}k^2} {\sqrt{k^2 + (M_\chi^\star)^2}} - \frac{1}{2}M_h^2 h_0^2 \, ,
\label{eq:pres}
\end{eqnarray} 
%%%%%%%%%%%%%%
Where $M_\chi^\star$ represents the effective mass of the DM.

Hence, in the context of DMANS, the expressions for the total energy density and pressure can be formulated as follows. 
%%%%%%%%%%%%%%%%
\begin{eqnarray}
{\cal{E}}={\cal{E}}_{\rm NS}+ {\cal{E}}_{\rm DM} \, ,
\nonumber
\\
{\rm and}
\hspace{1cm}
P=P_{\rm NS} + P_{\rm DM} \, ,
\label{eq:EOS_total}
\end{eqnarray}
%%%%%%%%%%%%%%

Once the EOS is determined as an input, the Tolman–Oppenheimer–Volkoff (TOV) equation can be solved to compute the mass and radius of the dark matter admixed neutron star  ~\citep{Tolman_1939, Oppenheimer_1939, pinku_prd_2023}. And the calculation for the dimensionless tidal deformability ($\Lambda$) ~\citep{Hinderer_2008, Kumartidal_2017} and non-radial $f$-mode oscillation ~\citep{athul_2022,harishprd_2021, Bikram_2021-fmode} can be done by solving their respective differential equations along with the TOV equation.
%%%%%%%%%%%%%%%%%%%%%%
\section{Results and Discussion}
%%%%%%%%%%%%%%    
% Figure environment removed
%%%%%%%%%%%%%%
In Figure \ref{fig:eos}, we illustrate the EOS for the INRS model both with and without the presence of dark matter. The shaded region represents the chiral Effective Field Theory (EFT) bounds ~\citep{Drischler_2021}, providing a benchmark for our analysis. To assess the influence of DM, we explore various scenarios by varying the Fermi momentum of the dark matter (k$_f^{\rm DM}$) from $0.00$ GeV to $0.05$ GeV. Remarkably, nearly all of the EOS curves align within the chiral EFT bounds, affirming the robustness and consistency of our model. It's important to note that the unified EOS is constructed using the SLY4 crust ~\citep{Douchin_2001}. In addition to the main plot, we provide an inset plot illustrating the sound speed as a function of density, ensuring that the causality limit is satisfied. This ensures that the speed of sound within the neutron star does not exceed the speed of light, thus preserving causality.

%%%%%%%%%%%%%%
% Figure environment removed
%%%%%%%%%%%%%%

\begin{table*}
\centering
\caption{Neutron star properties such as mass(M), radius(R), dimensionless tidal deformibility($\Lambda$) and nonradial $f$-mode frequency shown with varying k$_f^{\rm DM}$. }
\renewcommand{\tabcolsep}{0.2cm}
\renewcommand{\arraystretch}{1.5}
\begin{tabular}{lllllllllll}
\hline \hline
$k_f^{\rm DM}$ (GeV) & \multicolumn{3}{l}{$M=0.77 M_\odot$} & \multicolumn{3}{l}{$M=1.4 M_\odot$} & \multicolumn{4}{l}{$M=M_{\rm max} (M_\odot)$} \\
\cmidrule(lr){2-4}\cmidrule(lr){5-7}\cmidrule(lr){8-11}
& $R$ & $\Lambda$ & $f$ & $R$ & $\Lambda$ & $f$ & $M$ & $R$ & $\Lambda$ & $f$ \\
\hline
0.000 & 12.585 & 17328.172 & 1.741 & 13.040 & 732.105 & 2.013 & 2.343 & 12.164 & 9.757 & 2.392\\
0.005 & 12.573 & 17280.384 & 1.742 & 13.032 & 730.754 & 2.014 & 2.343 & 12.160 & 9.760 & 2.392\\
0.010 & 12.485 & 16944.979 & 1.753 & 12.975 & 720.574 & 2.021 & 2.336 & 12.134 & 9.802 & 2.396\\
0.015 & 12.268 & 16101.535 & 1.782 & 12.833 & 694.021 & 2.040 & 2.329 & 11.922 & 8.624 & 2.405\\
0.020 & 11.905 & 14688.758 & 1.834 & 12.587 & 646.907 & 2.076 & 2.311 & 11.790 & 8.672 & 2.424\\
0.025 & 11.426 & 12823.245 & 1.911 & 12.244 & 580.448 & 2.130 & 2.282 & 11.610 & 8.936 & 2.454\\
0.030 & 10.885 & 10758.035 & 2.012 & 11.827 & 499.789 & 2.206 & 2.240 & 11.316 & 8.783 & 2.523\\
0.035 & 10.323 & 8734.671 & 2.134 & 11.364 & 413.899 & 2.301 & 2.185 & 10.974 & 8.794 & 2.582\\
0.040 & 9.772 & 6906.368 & 2.276 & 10.881 & 330.210 & 2.415 & 2.119 & 10.580 & 8.785 & 2.682\\
0.045 & 9.247 & 5360.532 & 2.434 & 10.388 & 255.626 & 2.547 & 2.043 & 10.128 & 8.609 & 2.773\\
0.050 & 8.752 & 4101.288 & 2.608 & 9.900 & 190.975 & 2.694 & 1.959 & 9.690 & 8.895 & 2.907\\
\hline \hline
\end{tabular}
\label{tab:total}
\end{table*}

The resulting mass and radius values are depicted in Figure \ref{fig:mr}, with the color bar indicating the variation of k$_f^{\rm DM}$ with a step of $0.005$ GeV. In our analysis, we consider several observational constraints for mass and radius. The mass constraints from the black widow pulsar PSR J0952-0607 (purple) ~\citep{Romani_2022}, PSR J0348+0432 (blue) ~\citep{Antoniadis_2013}, and PSR J0740+6620 (orange) are represented by shaded regions. Additionally, the radius constraints from NICER ~\citep{Riley-nicer_2019, Miller-nicer_2019} and multimessenger observation ~\citep{Capano_2020} are also taken into account. Importantly, the mass-radius constraint from HESS J1731-347 is represented by the pink shaded region ~\citep{HESS_2022}. As we increase the value of k$_f^{\rm DM}$, the maximum mass of the neutron star decreases, and the resulting $M-R$ curve shifts towards the left. When k$_f^{\rm DM}$ is approximately $0.03$ GeV, the $M-R$ curve aligns well with the observational data. At this specific value of k$_f^{\rm DM}$, the radius of the canonical star, as presented by both NICER and Capano et al., is also satisfied. Furthermore, this k$_f^{\rm DM}$ value also fulfills the mass-radius constraint of HESS J1731-347. For larger values of k$_f^{\rm DM}$, up to $0.045$ GeV, the constraints of HESS J1731-347 and the canonical radius for Capano et al. continue to be satisfied.
%%%%%%%%%%%%%%
% Figure environment removed
%%%%%%%%%%%%%%

In Figure \ref{fig:tidal}, we present the calculated $\Lambda$ values for both the baryonic and DM cases, providing essential insights into the impact of DM inclusion on the deformability of the neutron star. The error bars in the plot represent the estimated values for the tidal deformability of the canonical star, which correspond to the values obtained from gravitational wave measurements of GW170817 ~\citep{Abbott_2017} and GW190814 ~\citep{Abbott_2020}. The blue shaded region corresponds to the constraints derived from the GW170817 measurements, and the color bar represents the variation of k$_f^{\rm DM}$ in the GeV scale. From the figure, we observe that the tidal deformability decreases with the inclusion of dark matter, indicating that an increased amount of DM within the neutron star makes it more resistant to deformation in response to tidal forces. This result aligns with our previous findings, reinforcing the notion that dark matter has a significant impact on the structural properties of neutron stars. Moreover, our EOS considered in this study demonstrates remarkable consistency with the canonical bound of the GW190814 event, as well as the observational constraint of GW170817, as indicated by the blue shaded region. 


%%%%%%%%%%%%%%
% Figure environment removed
%%%%%%%%%%%%%

In Figure \ref{fig:nrad}, we present the variation of the nonradial $f$-mode oscillation frequency with the mass of the neutron star. Notably, we also consider the theoretical limits of the $f$-mode frequency in light of the gravitational wave events GW170817 ~\citep{Abbott_2017} and GW190814 ~\citep{Abbott_2020}. As we increase the parameter k$_f^{\rm DM}$, we observe a corresponding increase in the $f$-mode frequency, implying that the neutron star pulsates with a higher frequency. This effect is consistent with the trend of the maximum mass of the neutron star decreasing with the increase in k$_f^{\rm DM}$.
\section{CONCLUSIONS}
In this paper, we have investigated the implications of DMANS in the context of two specific astronomical objects, namely HESS J1731-347 and PSR J0952-0607. To explore the effects of DM, we varied the Fermi momentum k$_f^{\rm DM}$ of DM from 0.00 GeV to 0.05 GeV. Our analysis focused on the EOS for the INRS model, both with and without the presence of dark matter. We found that nearly all EOS curves remained within the chiral Effective Field Theory  bounds, demonstrating the robustness and consistency of our model. In the presence of DM, the EOS became softer, affecting the mass and radius of the NSs. However, without the inclusion of DM, our model successfully produced a maximum mass of approximately $2.343 \ M_\odot$ for PSR J0952-0607, satisfying the NICER bounds for canonical radius. Furthermore, our analysis also helped in determining the nature of HESS J1731-347, suggesting that it could be a DMANS. We were able to constrain the amount of DM within the NSs based on the bounds from HESS J1731-347. For a Fermi momentum k$_f^{\rm DM}$ of approximately 0.03 GeV, our model satisfied the constraints from HESS J1731-347 as well as the radius constraint by Capano $et.al.$ Increasing the value of k$_f^{\rm DM}$ up to approximately 0.045 GeV still allowed our model to satisfy these constraints. Moreover, we investigated the tidal deformability of the NSs in our novel model, both with and without DM, and found that the values lie within the limits of GW events. Finally, we calculated the nonradial $f$-mode oscillation and observed that with the inclusion of DM, the amplitude of the $f$-mode increased as the DM's Fermi momentum increased. The softening effect of the EOS due to the addition of DM resulted in a decrease in the maximum mass of the NSs, leading to the conclusion that a low mass star pulsates with a higher frequency than a high mass star in our model.

%\clearpage
%%%%%%%%%%%%%%%%%%%%%%%%
\section{Acknowledgments}
%%%%%%%%%%%%%%%%%%%%%%%
B.K. acknowledges partial support from the Department of Science and Technology, Government of India with grant no. CRG/2021/000101.
%%%%%%%%%%%%%%%%%%%%%%%%%%%%
\section*{DATA AVAILABILITY}
%%%%%%%%%%%%%%%%%%%%%%%%%%%%
This manuscript has no associated data, or the data will not be deposited. Data sharing is not applicable to this article as no data sets were generated or analyzed during the current study.
%%%%%%%%%%%%%%%%%%%%%%%%%%%%%%
\bibliography{hess}
\bibliographystyle{mnras}
%%%%%%%%%%%%%%%%%%%%%%%%%%%%%%
\end{document}
%%%%%%%%%%%%%%%%%%%%%%%%%%%%%%%%%%%%%%%%%%%%%% END %%%%%%%%%%%%%%%%%%%%%%%%%%%%%%%%%%%%%%%%%%%%%%%%

