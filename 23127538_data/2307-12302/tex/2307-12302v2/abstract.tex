% !TEX root = main.tex


Saturation is a fundamental game-semantic property satisfied by strategies that interpret higher-order concurrent programs.
It states that the strategy must be closed under certain rearrangements of moves, and corresponds to the intuition
that program moves (P-moves) may depend only on moves made by the environment (O-moves).

We propose an automata model over an infinite alphabet, called saturating automata, for which all accepted languages 
are guaranteed to satisfy a closure property mimicking saturation.

We show how to translate the finitary fragment of Idealized Concurrent Algol ($\fica$) into saturating automata,
confirming their suitability for modelling higher-order concurrency.
Moreover, we find that, for terms in normal form, 
the resultant automaton has linearly many 
transitions and states with respect to term size, 
and can be constructed in polynomial time.
This is in contrast to earlier
attempts at finding automata-theoretic models of $\fica$, 
which did not guarantee saturation 
and involved an exponential blow-up during translation,
even for normal forms.