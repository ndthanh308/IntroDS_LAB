% !TEX root = main.tex


\section{Game semantics\label{sec:gs}}

In this section, we briefly present 
the fully abstract game model for $\fica$ from~\cite{GM08}, which we rely on in the paper.
Game semantics for $\fica$ involves
two players, called Opponent (O) and Proponent (P),
and the sequences of moves made by them can be viewed as interactions between 
a program (P) and a surrounding context (O).
The games are defined using an auxiliary concept of an arena.
\begin{definition}
An \emph{arena} $A$ is a tuple $\langle{M_A,\lambda_A,\vdash_A, I_A}\rangle$, where:
\begin{itemize}
\item $M_A$ is a set of \emph{moves};
\item $\lambda_A:M_A\rarr\makeset{O,P}\times\makeset{Q,A}$
is a function determining for each $m\in M_A$ whether
it is an \emph{Opponent} or a \emph{Proponent move}, 
and a \emph{question} or an \emph{answer};
we write $\lambda_A^{OP},\lambda_A^{QA}$ for the composite
of $\lambda_A$ with respectively the first and second projections;
\item $\vdash_A$ is a binary relation on $M_A$, called \emph{enabling},
satisfying: if $m\vdash_A n$ then $\lambda_A^{OP}(m)\neq\lambda_A^{OP}(n)$ and $\lambda_A^{QA}(m)=Q$;
\item $I_A\subseteq M_A$ is a set of \emph{initial moves} such that $\lambda_A(I_A)\subseteq\{(O,Q)\}$ and $\vdash_A \cap (M_A\times I_A) =\emptyset$ (no enablers).
\end{itemize}
\end{definition}
%If $m\vdash_A n$ we say that $m$ \emph{enables} $n$.
%We shall write $I_A$ for the set of all moves of $A$ which have no enabler; such moves are called \emph{initial}.
Note that an initial move must be an O-question (OQ).
In arenas used to interpret base types all questions are initial -
the possible P-answers (PA) are listed below ($0\le\mi\le \imax$).
\[\centering\renewcommand\arraystretch{0.9}\begin{array}{c|c|c}
~\word{Arena}~ & ~\word{OQ}~  & ~\word{PA}~ \\
\hline
\sem{\comt} & \mrun & \mdone \\\hline
\sem{\vart} & \mread & i \\
          & \mwrite{i} & \mok 
          \end{array}\qquad\qquad
          \begin{array}{c|c|c}
~\word{Arena}~ & ~\word{OQ}~  &~\word{PA}~\\
\hline
\sem{\expt} & \mq &   i\\
\hline
\sem{\semt} & \mgrb & \mok \\
          & \mrls & \mok
\end{array}
\]


% Figure environment removed
More complicated types are interpreted inductively using
the \emph{product} ($A\times B$) 
and \emph{arrow} ($A\Rightarrow B$) constructions, given in Figure~\ref{fig:gs1}. 

% Figure environment removed
We write $\sem{\theta}$ for the arena corresponding to type $\theta$. In Figure~\ref{fig:gs2a}, we give (the enabling relation of)
the arena
$A=(\sem{\comt\rarr\comt}\times\sem{\comt})\Rightarrow \sem{\expt}$,
which needs to be constructed to interpret the term from Example~\ref{ex:term}.
We use superscripts to distinguish copies of the same move
(the use of superscripts is consistent with our future convention,
which will be introduced in Definition~\ref{def:tags}).

Given an arena $A$, we specify next what it means to be a legal play in $A$.
For a start, the moves that players exchange will have to form
a \emph{justified sequence}, which is a finite sequence of moves
of $A$ equipped with pointers. Its first move is always initial and has no
pointer, but each subsequent move $n$ must have a unique pointer to an
earlier occurrence of a move $m$ such that $m\vdash_A n$.  We say that
$n$ is (explicitly) \emph{justified by} $m$ or, when $n$ is an answer, that
$n$ \emph{answers} $m$.
If a question does not have an answer in a justified sequence, we say
that it is \emph{pending} in that sequence.  
%and $m_A$ a move from $M_A$.
In Figures~\ref{fig:gs2b},~\ref{fig:gs2c} we give two justified sequences $s_1$ and $s_2$
over $A$.

Not all justified sequences are valid.  In order to constitute a legal
play, a justified sequence must satisfy a well-formedness condition
that reflects the ``static'' style of concurrency of our programming
language: any started sub-processes must end before the parent process terminates.
\cutout{any process starting sub-processes must wait for the
children to terminate in order to continue. In game terms: if a
question is answered then that question and all questions justified by
must have been answered (exactly once).} This is formalised as follows,
where the letters $q$ and $a$ to refer to question- and answer-moves
respectively, while $m$ denotes arbitrary moves.
\begin{definition}
The set $P_A$ of  \emph{plays over $A$} 
consists of the justified sequences $s$ over $A$ that satisfy
the two conditions below.
\begin{description}
\item[FORK]: In any prefix $s'= \cdots\rnode{A}{q} \cdots\rnode{B}{m}\justf{B}{A}$ of $s$, the question $q$ must be pending when $m$ is played.
\item[WAIT]: In any prefix $s'= \cdots\rnode{A}{q} \cdots\rnode{B}{a}\justf{B}{A}$ of $s$, all questions justified by $q$ must be answered.
\end{description}
\end{definition}
\cutout{For two shuffled sequences $s_1$ and $s_2$, $s_1\amalg s_2$ denotes
the set of all interleavings of $s_1$ and $s_2$.
For two sets of shuffled sequences $S_1$ and $S_2$,
$S_1\amalg S_2=\bigcup_{s_1\in S_1,s_2\in S_2}s_1\amalg s_2$.
Given a set $X$ of shuffled sequences, we define $X^0=X$, $X^{i+1}=X^i \amalg X$. 
Then $X^\circledast$, called \emph{iterated shuffle} of $X$, is defined to
be $\bigcup_{i\in\N}X^i$. }

It is easy to check that the justified sequences $s_1, s_2$ from~\cref{fig:gs2b,fig:gs2c} are plays.
\begin{remark}\label{rem:swaps}
It is worth noting that the notion of play is stable with respect to swaps of adjacent
moves except when the swaps involve occurrences of moves $m_1 m_2$ 
related by the pointer structure:
$\rnode{A}{m_1}\,\rnode{B}{m_2}\justf{B}{A}$
or $m_1, m_2$ are answers to questions $q_1, q_2$ such that
$q_2$ justifies $q_1$.
\end{remark}
A subset $\sigma$ of $P_A$ is \emph{O-complete} if $s\in \sigma$
and $s o\in P_A$ imply $so\in\sigma$, when $o$ is an O-move.
\begin{definition}
  A \emph{strategy} on $A$, written $\sigma:A$, is a
  prefix-closed O-complete subset of $P_A$.
\end{definition}
\cutout{
Recall that O represents the role of the environment/context in game semantics.
Thus, strategies record all potential environment actions.

The game model of $\fica$ consists of \emph{saturated} strategies only: the saturation
condition stipulates that all possible (sequential) observations of
(parallel) interactions must be present in a strategy: actions of the
environment (O) can always be observed earlier if possible, actions of the
program (P) can be observed later. To formalize this, for any arena
$A$, we define a preorder $\preceq$ on $P_A$, as the least transitive
relation $\preceq$ satisfying 
$s\, o\, m\, s'\preceq s\, m\, o\, s'$ and $s\, m\, p\, s'\preceq s\, p\, m\, s'$
for all $s,s'$,
where $o$ and $p$ are an O- and  a P-move respectively (in the above pairs of plays 
moves on the left-hand-side of $\preceq$ are assumed to have the same justifiers as on the right-hand-side). 
\begin{definition}\label{def:sat}
A strategy $\sigma:A$ is \emph{saturated} iff, for all $s,s'\in P_A$,
if $s\in \sigma$ and $s'\preceq s$ then $s'\in\sigma$.
\end{definition}
\begin{remark}\label{rem:causal}
Definition~\ref{def:sat} states that saturated strategies are stable 
under certain rearrangements of moves.
Note that $s_0\,  p\, o\, s_1\not \preceq s_0\, o\, p\, s_1$, while other move-permutations are allowed.
Thus, saturated strategies express causal dependencies of P-moves on O-moves. This partial-order aspect 
is captured explicitly in concurrent games based on event structures~\cite{CCRW17}.
\end{remark}
}

Suppose
$\Gamma=\{x_1:\theta_1,\cdots, x_l:\theta_l\}$ 
and $\seq{\Gamma}{M:\theta}$ is a $\fica$-term.
Let us write $\sem{\seq{\Gamma}{\theta}}$ for the arena $\sem{\theta_1}\times\cdots\times\sem{\theta_l}\Rightarrow\sem{\theta}$.
In~\cite{GM08} it is shown how to assign 
a strategy on $\sem{\seq{\Gamma}{\theta}}$ to any $\fica$-term
$\seq{\Gamma}{M:\theta}$. 
We write $\sem{\seq{\Gamma}{M}}$ to refer to that strategy.
For example, $\sem{\seq{\Gamma}{\divcom}}=\{\epsilon, \mrun\}$
and $\sem{\seq{\Gamma}{\skipcom}} = \{\epsilon,\mrun,\rnode{A}{\mrun}\, \rnode{B}{\mdone}\justf{B}{A}\}$.
The plays $s_1,s_2$ turn out to belong to the strategy that interprets the term from Example~\ref{ex:term}. 
\cutout{
$\seq{\Gamma}{M:\theta}$, where $\Gamma=\{x_1:\theta_1,\cdots, x_l:\theta_l\}$, using a strategy, written 
through strategies, written $\seq{\Gamma}{M:\theta}$, where $\Gamma=\{x_1:\theta_1,\cdots, x_l:\theta_l\}$,
are interpreted as saturated strategies (written $\sem{\seq{\Gamma}{M}}$) in the arena 
$\sem{\seq{\Gamma}{\theta}}=\sem{\theta_1}\times\cdots\times\sem{\theta_l}\Rightarrow\sem{\theta}$.
To model free identifiers $\seq{\Gamma,x:\theta}{x:\theta}$, one uses (the least saturated strategy generated by) 
alternating plays in which P simply copies moves between the two instances of $\sem{\theta}$.
Other elements of the syntax are interpreted using strategy composition with special strategies.
Below we give a selection of constructs along with the plays that generate the corresponding special strategies.
%\rlnote{Explaining at least one of these in words would be nice.}

\noindent
\renewcommand\arraystretch{1}
\[\begin{array}{lclclcl}
;& & \rnode{A}{q}\,\,\rnode{B}{\mrun}^2\justh{B}{A}\,\,\rnode{C}{\mdone^2}\justh{C}{B}\, \,\rnode{D}{q^1}\justn{D}{A}{140}\, \,\rnode{E}{a^1}\justh{E}{D}\, \,\rnode{F}{a}\justh{F}{A} & & 
||& & \rnode{A}{\mrun}\,\,\rnode{B}{\mrun}^1\justh{B}{A}\,\,\rnode{C}{\mrun}^2\justh{C}{A}\, \,\rnode{D}{\mdone^1}\justn{D}{B}{140}\,\, \rnode{E}{\mdone^2}\justn{E}{C}{140}\, \,\rnode{F}{\mdone}\justn{F}{A}{155}\\[2ex]
\raisebox{0.065ex}{:}{=}& & \rnode{A}{\mrun}\,\,\rnode{B}{\mq^1}\justh{B}{A}\,\,\rnode{C}{i^1}\justh{C}{B}\, \,\rnode{D}{\mwrite{i}^2}\justn{D}{A}{140}\,\, \rnode{E}{\mok^2}\justh{E}{D}\, \,\rnode{F}{\mdone}\justn{F}{A}{150} & &
{!} & & \rnode{A}{\mq}\,\,\rnode{B}{\mread}^1\justn{B}{A}{120}\,\,\rnode{C}{i^1}\justf{C}{B}\,\, \rnode{D}{i}\justn{D}{A}{120}\\[2ex]
{\bf grab} && \rnode{A}{\mrun}\,\,\rnode{B}{\mgrb}^1\justn{B}{A}{110}\,\,\rnode{C}{\mok^1}\justf{C}{B}\, \,\rnode{D}{\mdone}\justn{D}{A}{135} & \qquad\qquad &
{\bf release} && \rnode{A}{\mrun}\,\,\rnode{B}{\mrls}^1\justn{B}{A}{110}\,\,\rnode{C}{\mok^1}\justn{C}{B}{120}\, \rnode{D}{\mdone}\justn{D}{A}{135}
  \end{array}\]
\medskip

\begin{tabular}{ll}
${\bf newvar}\,x\aasg i$ & $\quad \rnode{A}{q} \,\, \rnode{B}{q^1}\justj{B}{A}\,\, (\rnode{C}{\mread^{11}}\justn{C}{B}{160}\,\, \rnode{D}{i^{11}}\justf{D}{C})^\ast\, \,
\big(\sum_{j=0}^\imax(\rnode{E}{\mwrite{j}^{11}}\justj{E}{B}\,\, \rnode{F}{\mok^{11}}\justh{F}{E}\,\, (\rnode{G}{\mread^{11}}\justn{G}{B}{160} \,\, \rnode{H}{j^{11}}
\justh{H}{G})^\ast)\big)^\ast
\,\, a^1\, \,a$\\[2ex]
${\bf newsem}\, x\aasg 0$&
 $\quad \rnode{A}{q} \,\, \rnode{B}{q^1}\justf{B}{A}\,\, (
 \rnode{C}{\mgrb^{11}}\justn{C}{B}{160}\,\, \rnode{D}{\mok^{11}}\justf{D}{C}\,\, \rnode{E}{\mrls^{11}}\justn{E}{B}{155}\,\, 
 \rnode{F}{\mok^{11}}\justh{F}{E})^\ast\, \, (\rnode{G}{\mgrb^{11}}\justn{G}{B}{160}\,\,\rnode{H}{\mok^{11}}\justf{H}{G}+\epsilon)\,\, a^1\, \,a$ \\[1ex]
%${\bf newsem}\, x\aasg 1$& 
%$\quad q\, q\,(\mrls\,\mok\,\mgrb\,\mok)^\ast\,(\mrls\,\mok+\epsilon)\, a\, a$.
\end{tabular}\\[1.5ex]
}
Given a strategy $\sigma$,
we denote by $\comp\sigma$ the set of non-empty \emph{complete} plays of $\sigma$, i.e. those in which all questions have been
answered. For example, $s_1$ (\cref{fig:gs2b}) is not complete, but $s_2$ (\cref{fig:gs2c}) is.

The game-semantic interpretation $\sem{\cdots}$ can be viewed as a faithful record of all possible interactions between the
term and its contexts.
It provides a fully abstract model in the sense that contextual equivalence is characterized by the sets of non-empty complete plays.
\begin{theorem}[\cite{GM08}]\label{thm:full}
\cutout{$\Gamma\vdash M_1\sqsubsim  M_2$ iff
$\comp{\sem{\Gamma\vdash M_1}}\subseteq \comp{\sem{\Gamma\vdash M_2}}$.}
We have $\Gamma\vdash M_1\cong  M_2$ if and only if  $\comp{\sem{\Gamma\vdash M_1}}=\comp{\sem{\Gamma\vdash M_2}}$.
\end{theorem}
The strategies corresponding to $\fica$ terms turn out to be
closed under swaps of adjacent moves as long as
the earlier move is a P-move or the later one is an O-move, and the swap produces a play.
Formally, for any arena $A$, let us define $\succeq\subseteq P_A\times P_A$ 
to be the least preorder satisfying
$s\, m\, o\, s' \succeq s\, o\, m\, s'$ and $s\, p\, m\, s' \succeq s\, m\, p\, s'$,
where $m,o,p$ range over moves, O-moves and P-moves respectively.
In the pairs of plays above, we assume that, during a swap, the justification pointers from the two moves also move with them.

\begin{example}
Consider the following play.
\begin{center}
\begin{tikzpicture}[->,>=stealth',auto, node distance=1.5cm,
  thick,main node/.style={},player/.style={}, bend right=25]
 \node[main node] (s) {$s_3=$};
  \node[main node] (q) [right of=s] {$\mq^{\vphantom{f}}$};
  \node[main node] (rf) [right of=q] {$\mrun^f$};
  \node[main node] (rf1) [right of=rf] {$\mrun^{f1}$};
      \node[main node] (rc) [right of=rf1] {$\mrun^{c}$};
  \node[main node] (dc) [right of=rc] {$\mdone^{c}$};
    \node[main node] (df1) [right of=dc] {$\mdone^{f1}$};
  \node[main node] (df) [right of=df1] {$\mdone^{f}$};
  \node[main node] (1) [right of=df] {$1$};

  \node[player] (pr) [below = -1mm of q] {$O$};
  \node[player] (prf) [below = -1mm of rf] {$P$};
  \node[player] (prf1) [below = -1mm of rf1] {$O$};
    \node[player] (prc) [below = -1mm of rc] {$P$};
  \node[player] (pdf1) [below = -1mm of df1] {$P$};
  \node[player] (pdf) [below = -1mm of df] {$O$};
    \node[player] (pdc) [below = -1mm of dc] {$O$};
  \node[player] (p1) [below = -1mm of 1] {$P$};

  \path (rf) edge[bend right] node [left] {} (q);
  \path (rf1) edge[bend right] node [left] {} (rf);
  \path (df1) edge[bend right] node [left] {} (rf1);
  \path (rc) edge[bend right] node [left] {} (q);
  \path (dc) edge[bend right] node [left] {} (rc);
  \path (df) edge[bend right] node [left] {} (rf);
  \path (1) edge[bend right] node [left] {} (q);
\end{tikzpicture}
\end{center}
\noindent
Observe that $s_2 \succeq s_3$, where $s_2$ is the play from~\cref{fig:gs2c},
because the P-move $\mdone^{f1}$ moved to the right
past a P-move ($\mrun^c$) and an O-move ($\mdone^c$).
In contrast, we do not have $s_3\succeq s_2$,
as this would involve moving a P-move ($\mdone^{f1}$) left
past an O-move ($\mdone^c$).
%Note that, although swapping $\mdone^{f}$ (O-move) 
%with $\mrun^c$ (P-move) yields a play $s$, we 
%do \emph{not} have $s_3\succeq s$ (instead we have $s\succeq s_3$).
\end{example}
\begin{example}\label{ex:para}
Consider the plays $s_4, s_5$ given below (in the arena 
$\sem{\comt\rarr\comt\rarr\comt}$), which correspond to
parallel and sequential composition respectively.
Observe that $s_4 \succeq s_5$. Note that the witnessing swap
involves swapping $\mrun^2$ (P-move) with $\mdone^1$ (O-move),
which is permitted by the definition of $\succeq$.
\cutout{
\begin{tikzpicture}[->,>=stealth',auto, node distance=1.1cm,
  thick,main node/.style={},player/.style={}, bend right=20]
 \node[main node] (s) {$s_4=$};
  \node[main node] (m1) [right=-2mm of s] {$\mrun^{\vphantom{f}}$};
  \node[main node] (m2) [right of=m1] {$\mrun^1$};
  \node[main node] (m3) [right of=m2] {$\mrun^2$};
    \node[main node] (m4) [right of=m3] {$\mdone^1$};
  \node[main node] (m5) [right of=m4] {$\mdone^2$};
  \node[main node] (m6) [right of=m5] {$\mdone$};

  \node[player] (p1) [below = -1mm of m1] {$O$};
  \node[player] (p2) [below = -1mm of m2] {$P$};
  \node[player] (p3) [below = -1mm of m3] {$P$};
    \node[player] (p4) [below = -1mm of m4] {$O$};
  \node[player] (p5) [below = -1mm of m5] {$O$};
  \node[player] (p6) [below = -1mm of m6] {$P$};

  \path (m2) edge[bend right] node [left] {} (m1);
  \path (m3) edge[bend right] node [left] {} (m1);
  \path (m4) edge[bend right] node [left] {} (m2);
  \path (m5) edge[bend right] node [left] {} (m3);
  \path (m6) edge[bend right] node [left] {} (m1);

\end{tikzpicture}

\begin{tikzpicture}[->,>=stealth',auto, node distance=1.5cm,
  thick,main node/.style={},player/.style={}, bend right=15]
 \node[main node] (s) {$s_5=$};
  \node[main node] (m1) [right of=s] {$\mrun^{\vphantom{f}}$};
  \node[main node] (m2) [right of=m1] {$\mrun^1$};
      \node[main node] (m4) [right of=m2] {$\mdone^1$};
  \node[main node] (m3) [right of=m4] {$\mrun^2$};
  \node[main node] (m5) [right of=m3] {$\mdone^2$};
  \node[main node] (m6) [right of=m5] {$\mdone$};

  \node[player] (p1) [below = -1mm of m1] {$O$};
  \node[player] (p2) [below = -1mm of m2] {$P$};
      \node[player] (p4) [below = -1mm of m4] {$O$};
  \node[player] (p3) [below = -1mm of m3] {$P$};
  \node[player] (p5) [below = -1mm of m5] {$O$};
  \node[player] (p6) [below = -1mm of m6] {$P$};

  \path (m2) edge[bend right] node [left] {} (m1);
  \path (m3) edge[bend right] node [left] {} (m1);
  \path (m4) edge[bend right] node [left] {} (m2);
  \path (m5) edge[bend right] node [left] {} (m3);
  \path (m6) edge[bend right] node [left] {} (m1);



\end{tikzpicture}
}
\bigskip
\[\xymatrix@C=0.5mm@R=-.5mm{
s_4= &\rnode{A}{\mrun}&\rnode{B}{\mrun^1}&\rnode{C}{\mrun^2}&
\rnode{D}{\mdone^1} & \rnode{E}{\mdone^2} &\rnode{F}{\mdone}\\
&O&P&P&O&O&P
\justf{B}{A}\justf{C}{A}\justf{D}{B}\justf{E}{C}\justn{F}{A}{130}
}\qquad\xymatrix@C=0.5mm@R=-.5mm{
s_5= &
\rnode{A}{\mrun}&\rnode{B}{\mrun^1}&\rnode{C}{\mdone^1}&
\rnode{D}{\mrun^2} & \rnode{E}{\mdone^2} &\rnode{F}{\mdone}\\
&O&P&O&P&O&P
\justf{B}{A}\justf{C}{B}\justf{D}{A}\justf{E}{D}\justn{F}{A}{130}}
\]
\end{example}
\begin{definition}\label{def:sat}
A strategy $\sigma:A$ is \emph{saturated} if, for all $s,s'\in P_A$,
if $s\in \sigma$ and $s\succeq s'$ then $s'\in\sigma$.
\end{definition}
\begin{remark}\label{rem:causal}
Definition~\ref{def:sat} states that saturated strategies are stable under $\succeq$.
Note that $s\,  o\, p\, s'\not \succeq s\, p\, o\, s'$, while other $o/p$ combinations are allowed in $\succeq$. Thus, saturated strategies allow one to 
express causal dependencies of P-moves on O-moves. 
This aspect of strategies is captured explicitly 
in concurrent games based on event structures~\cite{CCRW17}.
\end{remark}
\begin{theorem}[\cite{GM08}]
For any $\fica$-term $\seq{\Gamma}{M}$, the strategy
$\sem{\seq{\Gamma}{M}}$ is saturated.
\end{theorem}

In the next section we will introduce an automata-theoretic model
for representing plays. 
In contrast to earlier attempts, languages accepted by the automata
will satisfy a language-theoretic equivalent of the saturation condition.
