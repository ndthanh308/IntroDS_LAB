% !TEX root = main.tex


\section{Finitary Idealised Concurrent Algol ($\fica$)}
\label{sec:fica}

% Figure environment removed


Idealised Concurrent Algol~\cite{GM08} is a paradigmatic 
call-by-name language 
combining higher-order computation with imperative constructs 
in the style of Reynolds~\cite{Rey78}, 
extended to concurrency with parallel composition ($\parc$) and binary semaphores.
We consider its finitary variant, $\fica$,
defined over a finite datatype $\makeset{0,\ldots,\imax}$ ($\imax\ge 0$), with no recursion, but with iteration.
Its types $\theta$ are generated by the grammar 
\[
\theta::=\beta\mid \theta\rarr\theta\qquad\qquad
  \beta::=\comt\mid\expt\mid\vart\mid\semt
\]
where 
$\comt$ is the type of commands;
$\expt$ that of $\makeset{0,\ldots,\imax}$-valued expressions;
$\vart$ that of assignable variables;
and $\semt$ that of semaphores.
The typing judgments are displayed in Figure~\ref{fig:icatypes}.
Here, $\skipcom$ and $\divcom_\theta$ are constants representing termination and divergence respectively,
$i$ ranges over $\{0,\ldots,\imax\}$,
and $\mathbf{op}$ represents unary arithmetic operations, such as successor or predecessor (since we work over a finite datatype, operations of bigger arity can be defined using conditionals).
Variables and semaphores can be declared locally via $\mathbf{newvar}$ and $\mathbf{newsem}$.
Variables are dereferenced using $!M$, and semaphores are manipulated using two (blocking) primitives,
$\grb{s}$ and $\rls{s}$, which  grab and release the semaphore respectively. 
% !TEX root = main.tex
%\section{Additional material for \cref{sec:fica}}
%\subsection{Operational semantics of $\fica$}
We assume that variables are initialised to $0$
and semaphores are initially released.

In reduction rules, it will be convenient to use
the syntax
$\newvar{x\aasg i}{M}$ and $\newsem{x\aasg i}{M}$,
which allows us to specify initial values more flexibly,
i.e. $\newvar{x}{M}$ and $\newsem{x}{M}$ should be viewed as
$\newvar{x\aasg 0}{M}$ and $\newsem{x\aasg 0}{M}$ respectively.


{
The operational semantics is defined using a (small-step) transition
relation $\step{s}{M}{s'}{M'}$, where $\mem$ is a set of variable names
denoting active \emph{memory cells} and \emph{semaphore locks}.
$s,s'$ are states, i.e.\ functions $s,s':\mem\rightarrow\makeset{0,\cdots,\imax}$, and $M,M'$ are
terms.  We write $s\otimes (v\mapsto i)$ for the state obtained by augmenting $s$ with $(v\mapsto i)$, assuming $v\not\in \dom{s}$.
The basic reduction rules are given in Figure~\ref{fig:os},
where $c$ stands for any language constant ($i$ or $\skipcom$)
and $\widehat{\mathbf{op}}:\{0,\cdots,\imax\}\rarr\{0,\cdots,\imax\}$
is the function corresponding to $\mathbf{op}$.
In-context reduction is given by the schemata:

\begin{center}
\AxiomC{$\mem,v\vdash  M[v/x],s\otimes(v\mapsto i)\longrightarrow M',s'\otimes(v\mapsto i') $ \quad $M\neq c$}
\UnaryInfC{$\mem\vdash\newin{x\aasg i}{M},s\longrightarrow \newin{x\aasg i'}{M'[x/v]}, s' $}
\DisplayProof\\[2ex]
\AxiomC{$\mem,v\vdash  M[v/x],s\otimes(v\mapsto i)\longrightarrow M',s'\otimes(v\mapsto i') $\quad $M\neq c$}
\UnaryInfC{$\mem\vdash\newsem{x\aasg i}{M},s\longrightarrow \newsem{x\aasg i'}{M'[x/v]}, s' $}
\DisplayProof\\[2ex]
  \AxiomC{$\step{s}{M}{s'}{M'}$}
  \UnaryInfC{$\step{s}{\mathcal E[M]}{s'}{\mathcal E[M']}$}
  \DisplayProof
\end{center}
where reduction contexts $\mathcal E[-]$ are produced by the
grammar:
\[\begin{array}{rcl}
  \mathcal E[-] &::=& [-] \mid \mathcal E;N
  \mid (\mathcal E\,\parc\, N)
  \mid (M\,\parc\, \mathcal E)
  \mid {\mathcal E} N 
  \mid \arop{\mathcal E} 
  \mid \cond{\mathcal E}{N_1}{N_2}\\
  &&\mid {!}\mathcal E
  \mid \mathcal E\aasg m
  \mid M\aasg\mathcal E
  \mid \grb{\mathcal E}
  \mid \rls{\mathcal E}.
\end{array}\]
% Figure environment removed
We say that a term $\seq{}{M:\comt}$ \emph{may terminate},  written $M\Downarrow$, if 
$\emptyset \vdash \emptyset,\,M\longrightarrow^\ast \emptyset,\skipcom$.

\cutout{
Idealized Concurrent Algol~\cite{GM08} also features variable and semaphore constructors, 
called $\textbf{mkvar}$ and $\textbf{mksem}$ respectively,
which play a technical role in the full abstraction argument, similarly to~\cite{AM97a}.
We omit them in the main body of the paper, because
they do not present technical challenges, but they are covered here for the sake of completeness.


\paragraph*{Typing rules}
\[
\AxiomC{$\Gamma\vdash M:\expt\rarr\comt$}
  \AxiomC{$\Gamma\vdash N:\expt$}
  \BinaryInfC{$\Gamma\vdash \mkvar{M}{N}:\vart$}
  \DisplayProof
\quad
 \AxiomC{$\Gamma\vdash M:\comt$}
  \AxiomC{$\Gamma\vdash N:\comt$}
  \BinaryInfC{$\Gamma\vdash  \mksem{M}{N}:\semt$}
  \DisplayProof
\]


\paragraph*{Reduction rules}

\begin{align*}
  \step {s&}{(\mkvar{M}{N})\aasg M'}{s}{M M'}\\
  \step {s&}{{!}(\mkvar{M}{N}}{s}{N}\\
  \step {s&}{\grb {\mathbf{mksem}\,M N}}{s}{M}\\
  \step {s&}{\rls {\mathbf{mksem}\,M N}}{s}{N}
\end{align*}

\paragraph*{$\eta$ rules for $\vart,\semt$}
\[\begin{array}{rcl}
M &\longrightarrow & \mkvar{(\lambda x^\expt. M\aasg x)}{!M}\\
M &\longrightarrow & \mksem{\grb{M}}{\rls{M}}
\end{array}\]
\cutout{
Using $\mathbf{mkvar}$ and $\mathbf{mksem}$,
one can define $\divcom_\theta$ as syntactic sugar using $\divcom=\divcom_\comt$ only.
\[
\divcom_\theta=\left\{
\begin{array}{lcl}
\divcom && \theta=\comt\\
\divcom;0 && \theta=\expt\\
\mkvar{\lambda x^\expt.\divcom}{\divcom_\expt} && \theta=\vart\\
\mksem{\divcom}{\divcom} & &\theta=\semt\\
\lambda x^{\theta_1}.\divcom_{\theta_2} & & \theta=\theta_1\rarr\theta_2\\
\end{array}\right.
\]}
}

$\fica$ terms can be compared using a notion of
\emph{contextual (may-)equivalence}, denoted $\Gamma\vdash M_1\cong M_2$.
Two terms of the same type and with the same free variables 
are equivalent if 
they cannot be distinguished with respect to termination by any context:
for all contexts $\ctx$ such that $\seq{}{\ctx[M_1]:\comt}$, we have
$\ctx[M_1]\!\Downarrow$ if and only if $\ctx[M_2]\!\Downarrow$.
Using game semantics, one can reduce $\cong$ to equality of
the associated sets of complete plays (Theorem~\ref{thm:full}).
%However, due to quantification over all contexts, even very simple instances of equivalence are undecidable~\cite{GMO06}.
\begin{example}\label{ex:term}
Consider the term
\[
\seq{f:\comt\rarr\comt, c:\comt}{
\newvar{x}{(f\,( x\,\aasg{1} )\,\,\parc\,\, \cond{\,\deref{x}}{\,c\,}{\,\divcom}_\comt);\, \deref{x}} : \expt}
\]
The free variable $f$ can be viewed as representing an unknown
function, to be bound to concrete code by a context. Since
we work in a call-by-name setting, that function may evaluate
its argument arbitrarily many times, including none.
If  the function does not use its argument, the value of $x$
will always be $0$ (we assume that local variables are initialised to $0$)
and the term will never terminate, because the right term inside $\parc$
will always diverge, preventing the whole term from terminating.
On the other hand, as long as $f$ evaluates its argument at least once and terminates,
and the right-hand side of $\parc$ is scheduled after the assignment $x\aasg 1$ (and code bound to $c$ terminates) then the whole term will terminate too, returning $1$.
\end{example}
In the next section we sketch the game semantics of $\fica$.
