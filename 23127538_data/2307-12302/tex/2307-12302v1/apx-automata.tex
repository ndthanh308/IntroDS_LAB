% !TEX root = main.tex


% \section{Additional material for~\cref{sec:sata}}
% \label{apx:sata}

% \subsection{A transition sequence for~\cref{ex:aut}}

{
\newcommand{\ex}[1]{\ensuremath{#1_{\mathsf{ex}}}}
\newcommand{\sset}[1]{\{#1\}}

We give a possible transition sequence for $\Aut$.
For the sake of simplicity, data values from $\mathcal{D}$ will be subscripted with a number corresponding to their level, and superscripted with zero or more primes to distinguish within each level. 
%(Since there are no children of distinct nodes at level 2 in this example, this schema is sufficient.) 
Configurations are denoted as a tree of nodes, reflecting the subtree of $\mathcal{D}$ currently maintained in the automaton. 

Nodes at even levels $2i$ are written $d(X)$ or $d(X, v)$, where $d$ is a level-$2i$ data value, $X \in \mul{C^{(2i)}}$ and $v$ represents the memory value maintained at that node
(in this case always a single number).
Nodes at odd levels $2i-1$ have the form $d(X)$, where $d$ is a level-$(2i-1)$ data value and $X \in C^{(2i-1)}$.
%the control state (or multiset of control states, at even levels) at that node, 
The complete transition sequence is given in Figure~\ref{fig:trans-seq-s2}.
It witnesses the acceptance of a data word corresponding to the play $s_2$ from  Figure~\ref{fig:gs2c}.


{
\tikzset{
 auto,
 node distance=1.25cm,
 anchor=base,baseline= (current bounding box.center)
}
% Figure environment removed
}

}