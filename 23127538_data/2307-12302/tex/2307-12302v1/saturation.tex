% !TEX root = main.tex



\section{Saturation}\label{sec:sat}

In this section we define a language variant of saturation and show
that languages traced and accepted by $\sata$ satisfy it.
$d_1, d_2\in \D$ will be called \emph{independent} if neither $d_1=\predd{k}{d_2}$ nor $d_2=\predd{k}{d_1}$ for $k\ge 0$, 
i.e. the data lie on different branches. 
Let $\Sigma_O=\Sigma_{OQ}+\Sigma_{OA}$ and $\Sigma_P=\Sigma_{PQ}+\Sigma_{PA}$.
\begin{definition}
We shall say that $L\subseteq (\Sigma\times\D)^\ast$ is \emph{saturated} iff,
for any $w\in L$ and independent $d_1,d_2$,
$w=w_1 (t_1,d_1)(t_2,d_2) w_2\in L$ implies $w_1 (t_2,d_2)(t_1,d_1) w_2\in L$
whenever $t_1\in\Sigma_P$ or $t_2\in \Sigma_O$.
\end{definition}

\begin{remark}
The condition ``$t_1\in\Sigma_P$ or $t_2\in \Sigma_O$'' is the negation
of ``$t_1\in\Sigma_O$ and $t_2\in\Sigma_P$'', i.e. 
the swap is allowed unless the first letter is from $\Sigma_O$ and the second one from $\Sigma_P$.
Note that this is analogous to the game-semantic saturation condition (Definition~\ref{def:sat}).
The definition above uses independent $d_1,d_2$. It would not make sense
to extend it to any dependent cases: one can show that in such cases
the swap will never result in a trace%(\cref{apx:sat}, cf. Remark~\ref{rem:swaps})
.
\end{remark}
To show that saturating automata are bound to produce saturated sets of traces/accepted words, we establish a series of lemmas
about commutativity 
between various kinds of transitions.
\begin{lemma}[$\epsilon O\mapsto O\epsilon$]\label{lem:eo}
If $\kappa_1 \trans{\epsilon} \kappa_2 \trans{(t,d)} \kappa_3$ 
and $t\in\Sigma_O$ then 
$\kappa_1\trans{(t,d)}\kappa_2' \trans{\epsilon} \kappa_3$
for some $\kappa_2'$.
\end{lemma}
\begin{proof}
We need to consider all combinations of the transitions listed below.


\begin{tabular}{c|c}
$\epsilon$ & $O$ \\
\hline
$\DS{2i}\trans{\epsilon}\ES{2i}$ \quad or \quad $(\mop{2j}{h}{v},\cs{2i})\trans{\epsilon}(v',\ds{2i})$ 
& $\cs{2i'-1}\trans{q_O} \DS{2i'}$\quad or \quad $\cs{2i'+1}\trans{a_O}\ds{2i'}$
\end{tabular}

\smallskip

\noindent
Observe that the EPS transitions do not modify states at odd levels
or add nodes. Thus, the $\Sigma_O$ transitions could be fired from $\kappa_1$.
Now note that the $\Sigma_O$ transitions cannot prevent the EPS transitions 
from being executed next, because they do not change states at even levels (though
they add new ones).
\end{proof}
\begin{remark}
The converse to Lemma~\ref{lem:eo} is false.
If a $\Sigma_O$ transition is followed by an $\eps$ transition, it may be impossible to swap them, 
because the latter could rely on states introduced by the former.
\end{remark}
\begin{lemma}[$ P \epsilon \mapsto \epsilon P$]\label{lem:pe}
If $\kappa_1 \trans{(t,d)} \kappa_2 \trans{\epsilon} \kappa_3$ 
and $t\in\Sigma_P$ then 
$\kappa_1\trans{\epsilon}\kappa_2' \trans{(t,d)} \kappa_3$
for some $\kappa_2'$.
\end{lemma}
\begin{proof}
We inspect the shape of the relevant rules, which are listed below.

\begin{tabular}{c|c}
$P$ & $\epsilon$ \\
\hline
$\cs{2i}\trans{q_P} \ds{2i+1}$\quad or\quad $\DS{2i}\trans{a_P} \dagger$ & $\DS{2i'}\trans{\epsilon}\ES{2i'}$\quad or\quad $(\mop{2j}{h}{v},\cs{2i'})\trans{\epsilon}(v',\ds{2i'})$
\end{tabular}

\smallskip
\noindent
Observe that the $\epsilon$ transitions do not depend on any information
introduced by transitions on $\Sigma_P$. Hence, they are executable 
from $\kappa_1$. 
Note also that they will not
destroy any information needed to execute the $\Sigma_P$ transitions when fired, 
as there must already have been enough copies of any information to fire the transitions in the original order.
% \am{as they were executable in the opposite order}.
\end{proof}
\begin{remark}
The converse to Lemma~\ref{lem:pe} is false:
an $\epsilon$ transition may well be followed by a transition
on $\Sigma_P$ that relies on the states introduced by the $\epsilon$ transition.
\end{remark}
\begin{remark}\label{rem:permut}
One can use Lemmata~\ref{lem:eo} and~\ref{lem:pe} to replace sequences of
transitions of the form $\kappa\trans{(t_1,d_1)}(\trans{\epsilon})^\ast\trans{(t_2,d_2)}\kappa'$
with sequences of transitions between the same configurations
such that the transitions on $(t_1,d_1)$ and $(t_2,d_2)$
will be adjacent. 
\begin{itemize}
\item If $t_1\in\Sigma_P$ then, using Lemma~\ref{lem:pe} repeatedly,
one can obtain $\kappa_1(\trans{\epsilon})^\ast \trans{(t_1,d_1)}\trans{(t_2,d_2)} \kappa'$.

\item If $t_2\in\Sigma_O$ then, using Lemma~\ref{lem:eo} this time,
one can obtain
$\kappa_1\trans{(t_1,d_1)}\trans{(t_2,d_2)}(\trans{\epsilon})^\ast \kappa'$.
\end{itemize}
Note that these transformations require either
$t_1\in\Sigma_P$ or $t_2\in\Sigma_O$,
so they cannot be carried out if $t_1\in\Sigma_O$ and $t_2\in\Sigma_P$.
\end{remark}
Next we examine permutability of  consecutive transitions
involving independent data values. 
%We delegate the argument to Appendix~\ref{apx:sat}.
\begin{lemma}\label{lem:swap}
Suppose $d_1,d_2$ are independent and
$\kappa_1 \trans{(t_1,d_1)} \kappa_2\trans{(t_2,d_2)}\kappa_3$, where $t_1\in\Sigma_P$ or $t_2\in \Sigma_O$. 
Then there exists $\kappa_2'$ such that
$\kappa_1 \trans{(t_2,d_2)} \kappa_2'\trans{(t_1,d_1)}\kappa_3$.
\end{lemma}
\begin{proof}
Recall that non-$\epsilon$ transitions rely only on two consecutive levels of the configuration tree.
Consequently, if $d_1, d_2$ are independent and $\pred{d_1}\neq \pred{d_2}$ then the transitions operate on disjoint regions of the configuration and can be swapped.

Now suppose $\pred{d_1}=\pred{d_2}$ and note that, because of independence, we have $d_1\neq d_2$. Consequently, the transitions must operate at the same level
and concern different children of the same node.
\begin{itemize}
\item If the level is even, we need to consider 
the following combinations of transitions:
$\add(2i)\,\add(2i)$, $\del(2i)\,\add(2i)$,
$\del(2i)\,\del(2i)$ (other cases can be ignored
due to the $t_1\in\Sigma_P$ or $t_2\in\Sigma_O$ constraint).
Recalling that $\add(2i)$ and $\del(2i)$ transitions have the form
$\cs{2i-1}\trans{q_O} \DS{2i}$ and $\DS{2i}\trans{a_P} \dagger$ respectively,
we can confirm that the Lemma holds, because 
the state $\cs{2i-1}$ associated with $\pred{d_1}=\pred{d_2}$ is not modified
and there is no scope for interference between the transitions.
%\adnote{The sequences of transitions are not very aesthetically pleasing to read...}

\item If the level is odd, we need to consider 
the following combinations of transitions:
$\add(2i+1)\add(2i+1)$, $\add(2i+1)\del(2i+1)$,
$\del(2i+1)\del(2i+1)$ (other cases can be ignored
due to the $t_1\in\Sigma_P$ or $t_2\in\Sigma_O$ constraint).
Recalling that $\add(2i+1)$ and $\del(2i+1)$ transitions have the form
$\cs{2i}\trans{q_P} \ds{2i+1}$ and $\cs{2i+1}\trans{a_O} \ds{2i}$ respectively,
we can confirm that the Lemma holds, because the transitions will not
interfere. In particular, due to $d_1\neq d_2$,
the $\del(2i+1)$ transition in $\add(2i+1)\del(2i+1)$ cannot use
the state introduced by the preceding $\add(2i+1)$ transition.
\end{itemize}
\end{proof}


\begin{remark}
Note that the ``$t_1\in \Sigma_P$ or $t_2\in\Sigma_O$'' condition is necessary:
in the $\del(2i+1)\,\add(2i+1)$ case (i.e. $a_O q_P$), 
it is possible
for the latter transition to use the target state of the former.
\end{remark}

\begin{thm}
For any $\sata$ $\Aut$, the sets $\trace{\Aut},\lang{\Aut}$ 
are saturated.
\end{thm}
\begin{proof}
Consider $t_1,t_2,d_1,d_2$ such that 
$t_1\in \Sigma_P$ or $t_2\in\Sigma_O$,
$d_1, d_2$ are independent and 
$w_1 (t_1,d_1)(t_2,d_2) w_2\in\trace{\Aut}$.
Thus, there exist $\kappa_1,\kappa_2$ such that
$\kappa_1 \trans{(t_1,d_1)}(\trans{\epsilon})^\ast\trans{(t_2,d_2)}\kappa_2$.
By Remark~\ref{rem:permut}, we can rearrange the transitions to get
$\kappa_1 (\trans{\epsilon})^\ast \trans{(t_1,d_1)}\trans{(t_2,d_2)}
(\trans{\epsilon})^\ast\kappa_2$.
By Lemma~\ref{lem:swap}, we then obtain
$\kappa_1 (\trans{\epsilon})^\ast \trans{(t_2,d_2)}\trans{(t_1,d_1)}
(\trans{\epsilon})^\ast\kappa_2$, i.e.
$w_1 (t_2,d_2)(t_1,d_1) w_2 \in\trace{\Aut}$.
Hence, $\trace{\Aut}$ is saturated.
As $\lang{\Aut}$ is a subset of $\trace{\Aut}$ in which
all questions have answers, $\lang{\Aut}$ is also saturated,
because the  swaps do not affect membership in $\lang{\Aut}$.
\end{proof}

\begin{remark}\label{rem:discussion}
Earlier proposals for automata models of $\fica$~\cite{DLMW21,DLMW21b} 
failed to satisfy saturation.
In retrospect, this was because they allowed for too much communication
between control states at various levels.

Leafy automata~\cite{DLMW21} could  access the whole branch 
of the configuration tree at each transition and modify it during transition.
In particular, each move could access and update the state at the root.
This feature could easily be used to define leafy automata that
are very rigid and not closed under any kind of transition swaps. 
Local leafy automata, also introduced in \cite{DLMW21}, restrict access only to the local part of the branch but still allow communication (thus preventing swaps) between nodes sharing a parent or great-grandparent.

Split automata~\cite{DLMW21b} in turn featured restricted access
to control states at various levels, but their transitions still
allowed for state-based communication between siblings,
through transitions $\cs{2i}\trans{q_P} (\ds{2i},\ds{2i+1})$
and $(\cs{2i},\cs{2i+1})\trans{a_O} \ds{2i}$.
The first rule could be used to create two child nodes in
a specific order only, violating Lemma~\ref{lem:swap} for $t_1, t_2\in \Sigma_P$.
The second rule could be used to delete child nodes in a specific
order only, violating the same lemma for $t_1, t_2\in \Sigma_O$.
Finally, the fact that the two rules can communicate through level $2i$
means that we can make the second one conditional on the first one,
meaning that Lemma~\ref{lem:swap} would be violated for $t_1\in \Sigma_P$
and $t_2\in \Sigma_O$.  Consequently, split automata did not offer
native support for saturation, regardless of the polarity of letters.
\end{remark}