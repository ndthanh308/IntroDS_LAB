%\documentclass[10pt,a4paper]{article}
\documentclass[]{jfm}
\usepackage{newtxtext}
\usepackage{newtxmath}

%\usepackage[utf8]{inputenc}
\usepackage{amsmath}
\usepackage{amsfonts}
\usepackage{amssymb}

\usepackage{pgfplots}
\usetikzlibrary{shapes.geometric}
%%

%\usepackage[bmargin=2.cm,tmargin=2.cm,lmargin=2cm,rmargin=2cm]{geometry}
\usepackage{natbib}

\usepackage{psfrag}
\usepackage{pstool}
\usepackage{graphicx}
\usepackage{grffile}%allow .1.jpeg

\usepackage{hyperref}
\usepackage[final]{pdfpages}
\usepackage{url} %citer un lien

\usepackage[]{subcaption}
\captionsetup[subfigure]{skip=+10pt}

\usepackage{tikz}
\usetikzlibrary{positioning}

\usepackage{authblk}

\definecolor{blue_py}{RGB}{31,119,180}
\definecolor{orange_py}{RGB}{255, 127, 14} 
\definecolor{green_py}{RGB}{44, 160, 44}
\definecolor{red_py}{RGB}{214, 39, 40} 




\begin{document}


%\title{The lubrication force between two translating and growing bubbles at arbitrary Reynolds number} 
%\title{The lubrication force between translating and growing bubbles at arbitrary Reynolds number}
%\title{The lubrication forces between translating and growing bubbles: from viscous to inertial regimes}
\title{Viscous and inertial forces between translating and growing bubbles in close proximity}
\author[1]{\small Jean-Lou Pierson}
\affil[1]{\small IFP Energies Nouvelles, Solaize, 69360, France}
\affil[*]{ jean-lou.pierson@ifpen.fr}
%\homepage[]{Your web page}
%\thanks{}
%\altaffiliation{}





\date{\today}




\maketitle 

\begin{abstract}
Motivated by the dynamics of microbubbles in dissolved gas flotation processes, we consider theoretically the approach between two shear-free translating and growing bubbles. We make use of the lubrication assumption to obtain the thin film flow between the bubbles. We demonstrate that the lubrication force between the bubbles involves two distinct components: one viscous and one inertial. Both components exhibit weak singular behavior, scaling logarithmically with the ratio of bubble radius to film thickness. To assess the accuracy of our findings, we compare the obtained solution to results from Stokes flow and potential flow theory. The comparison demonstrates that our current results are reliable, provided that we combine the lubrication forces with a constant term that cannot be derived from the lubrication assumption alone. We illustrate the relevance of the solution to determine the drainage time of a small bubble rising to a free surface, and the drainage rate of expanding bubbles under force-free conditions. Remarkably, our findings exhibit good agreement with available experimental data.
% finding very good agreement with available experimental results.

%We emphasize the significance of our solution in determining the coalescence time of small bubbles rising towards a free surface and the drainage rate of expanding bubbles under force-free conditions. Remarkably, our findings exhibit excellent agreement with available experimental data.

%By comparing this solution to Stokes flow results and Porential flow results we demonstrate that the current results are accurate as long as we combine the lubrication forces with a constant term which cannot be obtained by lubrication force alone. We illustrate the relevance of the solution to determine the coalescence time of a small bubble rising to a free surface, and the drainage rate of expanding bubbles under force-free conditions finding very good agreement with available experimental results. %IN   


%Motivated by the dynamics observed in dissolved gas flotation processes involving microbubbles, we conduct a theoretical investigation on the approach between two shear-free bubbles that are both translating and growing. To analyze the thin film flow between these bubbles, we employ the lubrication assumption. Our study reveals that the lubrication force consists of two distinct components: one associated with viscosity and the other with inertia. Both components exhibit weak singular behavior, scaling logarithmically with the ratio of bubble radius to initial film thickness.

%To assess the accuracy of our findings, we compare the obtained solution to results from Stokes flow and potential flow analyses. The comparison demonstrates that our current results are reliable, provided that we combine the lubrication forces with a constant term that cannot be derived from the lubrication force alone. We emphasize the significance of our solution in determining the coalescence time of small bubbles rising towards a free surface and the drainage rate of expanding bubbles under force-free conditions. Remarkably, our findings exhibit excellent agreement with available experimental data.

 %in both viscous dominated  where a i

%In this note, we consider the thin film flow between two shear free bubbles. We show that the lubrication force are weakly singular and behaves as $\log(a/h_0)$ in both viscous dominated  where a is the bubble radius
% La seul différence entre le cas de de deux bulles et le cas d'une bulle a sunr surface libre est qu'il faut mulitplier par 2 ou diviser par 2 h. Ensuite en fonction de la puisscane de h de la solution on trouve different resultat. POur le cas immobile on a une facteur 2, pour le cas particllement mobile je pense qu'on a un facteur racine 2 et ici c'est un terme en O (1)
%Recent images of the early stages of this process by ...show that the radius of the hole...
%bien dire que le terme d'ordre 1 est necessaire pour etre suffisament precis
\end{abstract}


%%%%%%%%%%%%%%%%%%%%%%%%%%%%%%%%%%%%%%%%%%%%%%%%%%%%%%%%%%%%%%%%%%%%%%%%%%%%%%%%%%%%%%%%%%%%%%%%%%%%%%%%%%%%%%%%%%%%%%%%%%%%%%%%%%%%%%%%%%%%%%%

\section{Introduction}
\label{sec:intro}

The coalescence of bubbles is ubiquitous in nature and industry. For instance, the coalescence of bubbles at a free surface significantly affects the production of sea aerosol \citep{deike2022}. In the industry, the size of bubbles has a profound impact on the efficiency of flotation processes \citep{nguyen2003}. These flotation processes are particularly interesting for the recovery of microparticles like microplastics or fine particles \citep{swart2022}. However, one major limitation of this technique is the requirement for generating extremely small bubbles to capture the smallest particles, as emphasized by \citet{yoon1989}. 
Dissolved gas flotation is one existing technology used to generate micron-sized bubbles. In this process, the pressure of a liquid containing dissolved air is reduced, thereby releasing micron-sized bubbles that grow in size as they translate. The dynamics of these bubbles, especially when they are in close proximity and about to coalesce, strongly influences the process efficiency which motivates the present study.

%Dissolved gas flotation is one existing technology used to generate micron-sized bubbles. In this process, the pressure of a liquid containing dissolved air is reduced, thereby releasing micron-sized bubbles that grow in size as they move. The dynamics of these bubbles strongly influence the process efficiency, especially when they are in close proximity to coalescence, which motivates the present study.

%For instance, the coalescence of bubbles with a free surface influences the production of sea aerosol \citep{deike2022}, while in the industry the size of the bubble dramatically influences the efficiency of flotation processes \citep{nguyen2003}. Flotation processes are particularly interesting to recover microparticles such as microplastics or fine. One of the main drawbacks of this technique is the necessity of generating very small bubbles to catch the smallest particles \citep{yoon1989}. One of the existing technology to generate micron-sized bubbles is dissolved gas flotation. In this process, the pressure in a liquid within which air is dissolved is reduced thereby releasing micron-sized bubbles that grow in radius as they translate. The dynamics of those bubbles have a strong impact on the efficiency of the process especially when there are sufficiently close to coalesce which motivates the present study.

%The coalescence of bubbles plays a pervasive role in both natural and industrial settings. For instance, the coalescence of bubbles at a free surface significantly affects the production of sea aerosol, as noted by Deike et al. (2022). In the industrial context, the size of bubbles has a profound impact on the efficiency of flotation processes, as highlighted by Nguyen et al. (2003). These flotation processes are particularly crucial for the recovery of microparticles like microplastics and fine particles. However, one major limitation of this technique is the requirement for generating extremely small bubbles to capture the smallest particles, as emphasized by Yoon et al. (1989). Dissolved gas flotation is one existing technology used to generate micron-sized bubbles. In this process, the pressure of a liquid containing dissolved air is reduced, thereby releasing micron-sized bubbles that grow in size as they move. The dynamics of these bubbles strongly influence the process efficiency, especially when they are in close proximity to coalescence, which motivates the present study.


% what are the forces between growing and translating bubbles in closed proximity.

%In practice geenrating very small bubble. Dissolved gas flotation is usually prefered. In this process predictiing the size of the bubble wich translate due to the ambient flow and or gravity as well as grow in radius which motivate the present study.
%bubble columns are routinely used in chemical engineering industry. 

%The coalescence of bubble is usually decomposeed into three stages (trouver une ref) : the approach, the film drainage and the coalscence. Here we focus on the film drianage part and we consider the lubrication force acting on the two bubble.

%In the Stokes flow regime the approach and drianage of two bubble is shown to ewhibit a weak singularity (Kim et Karrila +other reference).

%Les équations sont elles singulieres ? que donnent elles en régime inertiel ? dans le cas de deux bulles peut on utiliser Davis ?


%The coalescence of b



The hydrodynamic resistance between two spherical drops or bubbles in the Stokes regime has been studied by \citep{haber1973} using a bispherical coordinate solution of the Stokes equation. \citep{davis1989} used lubrication theory to obtain the leading order forces on normally translating drops when the film thickness is significantly smaller than the drop radius. They observed good agreement between their results and the bispherical solution for sufficiently thin films. They also highlight the effect of the viscosity ratio on the force and mobility of the interface. In particular, the interface can be considered fully mobile or shear-free only for very small viscosity ratios.
%They found very good agreement with the bi-spherical solution for sufficiently small thicknesses. They also, highlight the effect of the viscosity ratio on the force and mobility of the interface. The interface can be considered fully mobile or shear-free only for a very small very viscosity ratio.
%The investigation of hydrodynamic resistance between two spherical drops or bubbles in the Stokes regime was conducted by Haber (1973), who employed a bispherical coordinate solution of the Stokes equation. Davis (1989), on the other hand, utilized lubrication theory to determine the dominant forces acting on drops that translate normally, specifically focusing on cases where the film thickness is significantly smaller than the drop radius. Remarkably, they observed excellent agreement between their results and the bispherical solution for sufficiently thin films. Davis also emphasized the influence of viscosity ratio on both the force and mobility of the interface. It was noted that the interface can be considered fully mobile or shear-free only for viscosity ratios that are extremely small.
%For very large viscosity ratio the interface can be considered as immobile and the interface ... in the case of a fully mobile interface however the singularity of the film is still a matter of question.
%In parallel many experimental works have been performed on fully mobile interface for finiter or important deformation of the bubble or drop. 
%Assuming the interfaces to be shear-free \citep{chesters1982} studied the film drainage between two small bubbles approaching at a constant velocity. They find two different regimes of film drainage depending on the viscous or inertia dominated. In the former case, coalescence occurs in a finite time, even aside from van der Waals effects and a dimple is observed. In viscous dominated regimes no dimple is observed.  \citep{pigeonneau2011} performed simulations of the rising of a bubble toward a free surface using  boundary integral methods. They found a significant influence of the deformation on the drainage dynamic with respect to the bi-spherical solution results, the deformation tending to delay the drainage of the film. 
\citep{chesters1982} investigated the film drainage between two approaching bubbles, assuming shear-free interfaces. Their study revealed two distinct regimes of film drainage based on whether viscosity or inertia dominates. In the latter regime, coalescence occurs within a finite time, even without considering van der Waals effects, and a dimple is observed. Instead, in viscous-dominated regimes, no dimple is observed. \citep{pigeonneau2011} performed simulations using boundary integral methods to study the rise of a bubble towards a free surface in viscous dominated flows. Their findings indicated a significant influence of deformation on the dynamics of film drainage, compared to the results obtained from the bi-spherical solution. The deformation tended to delay the drainage of the film.

%parler des papiers de degennes ensuite.

%il faut aussi parler des bulles gui grossissent.

%There are much less works on translating and growing bubbles. \citet{michelin2018} derived the bi-spherical coordinate solution for two growing bubbles. They showed that hydrodynamic effects are mostly negligible except in the case of bubbles in close proximity. In the nearly inviscid limit \citet{vandergeld2002} derived the force on growing and translating bubbles near a plane wall. Thanks to an extended Rayleigh–Plesset equation they investigated the trajectory of a bubble with an oscillating radius.

Translating and growing bubbles has received relatively less attention in the literature. \citet{michelin2018} derived a solution using bi-spherical coordinates for two expanding bubbles. Their analysis revealed that hydrodynamic effects are mostly negligible, except when the bubbles are in close proximity. In the nearly inviscid limit \citet{van2002} derived the force on growing and translating bubbles near a plane wall. Through an extended Rayleigh-Plesset equation, they investigated the trajectory of a bubble with an oscillating radius.


%force to the distance from the wall and the bubble growth rate are derived

%among many other. where the pressure gradient in the longituinal direction equilibrate a Poiseuille flow or a plug flow dominated by the velocity in the bubble. There is much less work on coaslescing bubble with shear free interface.


Focusing on translating and growing bubbles with shear-free interfaces, the present study aims at disentangling the various contributions to the forces. For this purpose, we make use of lubrication theory to derive closed-form solutions to the forces. In addition, we validate our theory by comparing it with both Stokes flow and potential flow solutions. We then apply our results to two canonical configurations from existing literature, specifically examining the scenarios of a bubble approaching a free interface and the interaction of two growing bubbles under force-free conditions. The problem and current assumptions are introduced in \ref{sec:problem}. The viscous stress and pressure distribution within the liquid film are derived in \ref{sec:stress}. Then, the forces on the bubbles are derived and compared to Stokes and potential flow solutions in \ref{sec:force}. Predictions from the theory in two different physical configurations are discussed in \ref{sec:vaka} and \ref{sec:ohashi}. The main conclusions and some open issues are summarized in section \ref{sec:conc}. 

%we provide validation to our theory stokes flow solution and potential flow solution. We then applied our results to two canonical configurations of the literature in viscous dominated regimes: the approach of a bubble to a free interface, and two growing bubbles in force-free conditions. The problem and current assumptions are introduced in 2, while 3 summarizes the numerical method.  

%We validate our theory by comparing it with both Stokes flow and potential flow solutions. We then apply our results to two well-known configurations from existing literature, specifically examining the scenarios of a bubble approaching a free interface and the interaction of two growing bubbles under force-free conditions. Section 2 introduces the problem and the assumptions made, while Section 3 provides a summary of the numerical method employed in this study.
%Before embarking on the discussion of numerical results, the fundamental
%mechanisms involved in the interaction process for spherical and deform


%\section{Problem definition}
\section{Problem definition}
\label{sec:problem}

% Figure environment removed
%In this study, we investigate the dynamics of thin film flow between two shear-free interfaces of bubbles with radius $a$. The external fluid between the bubbles has a density of $\rho$ and dynamic viscosity of $\mu$. The film, located between the two bubbles, is modeled using axisymmetric cylindrical polar coordinates $(r,z)$. The two bubbles approach each other with a velocity of $\mathbf{V}=V\mathbf{e}_z$, where $\mathbf{e}_z$ represents the unit vector in the $z$ direction (as shown in Fig. \ref{fig:scheme}). The film thickness is denoted by $h(r,t)$, while the shortest distance between the two bubbles is given by $h_0(t)$. The thin film limit is considered, in which $h_0/a=\epsilon \ll 1$. Close to the axis of symmetry, i.e., for $r \ll a$, $h(r,t) \sim h_0(t)+r^2/a$. Therefore, in this region, $r$ scales as $\sqrt{ah_0}$ \citep{davis1989}. The outward normal to the top bubble can be expressed as follows: 

%We consider the thin film flow between two shear-free interfaces bubbles of radius $a$. The external fluid between the bubbles has density $\rho$ and dynamic viscosity $\mu$. The film located between the two bubbles is described in axisymmetric cylindrical polar coordinates $(r,z)$. The two bubbles approach with a velocity $\mathbf{V} = V \mathbf{e}_z$ where $\mathbf{e}_z$ is the unit vector in $z$ direction (Fig. \ref{fig:scheme}).   The thickness of the film is written $h(r,t)$ while the shortest distance from the two bubbles $h_0(t)$. We consider the thin film limit for which $h_0/a =\epsilon \ll 1$. Near the axis of symmetry, i.e in the limit $r \ll a$, $h(r,t) \sim h_0(t) + r^2/a$. Hence in this region $r$ scales as $\sqrt{ah_0}$ \citep{davis1989}. The outward normal to the top bubble reads 

We consider the thin film flow between two translating and expanding bubbles moving along their line of centers (Figure \ref{fig:scheme}). The radius of the bubbles is noted $a(t)$ and the rate of change of the bubble radius is noted $\dot{a}(t)$. Both bubbles have shear-free interfaces. The external fluid between the bubbles has a density of $\rho$ and dynamic viscosity of $\mu$. The film, located between the two bubbles, is modeled using axisymmetric cylindrical polar coordinates $(r,z)$. The film thickness is denoted by $h(r,t)$, while the shortest distance between the two bubbles is given by $h_0(t)$. Hence, the axial centroid velocity of the top bubble is $V(t) = \dot{h}_0/2+\dot{a}$. The thin film limit is considered, in which $h_0/a=\epsilon \ll 1$. Close to the axis of symmetry, i.e., for $r \ll a$, $h(r,t) \sim h_0(t)+r^2/a$. Therefore, in this region, $r$ scales as $\sqrt{ah_0}$ \citep{davis1989}. The outward normal to the top bubble can be expressed as follows 

\begin{equation}
\mathbf{n}=-\left(1-\frac{r^2}{a^2}\right)^{1/2}\mathbf{e}_z + \frac{r}{a} \mathbf{e}_r.
\end{equation}
Hence to leading order $\mathbf{n} = -\mathbf{e}_z +\mathcal{O}(\epsilon^{1/2})$. Using this approximation the lubrication force on the top bubble reads 
\begin{equation}
F = 2\pi\int _0 ^{R_\infty} \left(p-2\mu \frac{\partial w}{\partial z}\right)r dr, 
\label{eq:force}
\end{equation}
where $w$ is the axial velocity and $R_\infty$ is the extent of the gap for which lubrication effects dominate. Hence, $R_\infty$ should be in the range $\sqrt{ah_0} \ll R_\infty \ll a$ \citep{davis1989}. To determine the leading order force, it is necessary to compute both the velocity profile and the pressure distribution. It should be noted that the present study differs from classical lubrication analysis in a significant aspect. Specifically, in contrast to \citet{davis1989} and \citep{leal2007}, the pressure and viscous stress in our analysis have comparable magnitudes in Equation \ref{eq:force} due to the assumption of shear-free interfaces \citep{savva2009}.

%Hence to obtain the leading order force one need to compute the velocity profile as well has the pressure distribution. We would like to mention an important difference between the present work and the work of \citet{davis1989}. In constrast to \citet{davis1989} both pressure and viscous stress have the same order of magnitude.

 %  have the same order of magnitude than the pressure force.

%between this work and the work of . 


%We consider that both bubbles have shear-free interfaces.
%$\sigma _zz$ is the  

%In the present analysis one want to compute the hydrodynamic force which reads 

%\begin{equation}
%F = (\int _0 ^R \boldsymbol{\sigma}\cdot\mathbf{n} dS)\cdot e_z. 
%\end{equation}
%where $R$ is assumed to be sufficiently large to represent the point for which the thin film region breakdown.
%Since 




%+ \mathcal{O}\left(\frac{r^4}{a^3}\right)$
%ajouter le O comme Michelin

%\section{Equations of motion}
\section{Viscous stress and pressure distribution}
\label{sec:stress}
We begin this section with the calcultation of the viscous stress. Conservation of mass in axisymetric cylindrical coordinate reads
\begin{equation}
\frac{\partial w}{\partial z} + \frac{1}{r}\frac{\partial }{\partial r} (ur) = 0,
\label{eq:cdm}
\end{equation}
where $u$ is the velocity in the radial direction. Hence the viscous stress can be easily evaluated through the calculation of the radial velocity profile. Since the thin film is long and thin ($\epsilon \ll 1$), and due to the shear-free conditions the radial velocity $u$ is the sum of a leading order plug flow profile and a second-order parabolic flow profile \citep{savva2009}. On the other hand the axial velocity is a linear function of $z$. Hence the kinematic condition on the interface may be written

\begin{equation}
\frac{\partial h}{\partial t} + \frac{1}{r} \frac{\partial }{\partial r} (urh) =0. \label{eq:cdmh}
\end{equation}
%Inserting the kinematic boundary condition in Equation \ref{eq:cdm}.
Since $\partial h /\partial t (r,t) = \dot{h}_0 + \mathcal{O}(\epsilon)$ and $u(r=0,t)=0$, one obtain the radial velocity profile
\begin{equation}
u= -\frac{\dot{h}_0}{2}\frac{r}{h} %= (V + \dot{a} )\frac{r}{h_0 + r^2/a}, 
\label{eq:u}
\end{equation}
and the viscous stress
\begin{equation}
\mu \frac{\partial w}{\partial z} = - \mu \frac{\dot{h}_0}{h}\left(-1+\frac{r^2}{ah}\right).
\label{eq:viscous_stress}
\end{equation}
Since the flow is dominantly in the radial direction the momentum equation reads \citep{chesters1982,savva2009}

%The momentum equation for a 
%Now let-s integrate the equation of motion on $r$. Le mieux est juste de commencer avec la conservation de la masse puis ensuite de parler de la pression.


%Using ... we obtain
 

 
%is dominantly in the radial direction $r$ and obeys the following equations \citep{chesters1984,savva2009}:
%\begin{align}
%\frac{\partial h}{\partial t} + \frac{1}{r} \frac{\partial }{\partial r} (urh) &=0, \\
%\frac{\partial w}{\partial z} +  \frac{1}{r}\frac{\partial }{\partial r} (ur) &= 0,\\
%\rho\left(\frac{\partial u}{\partial t} + u\frac{\partial u}{\partial r}\right) &= - \frac{\partial p}{\partial r} + 2 \mu \left(\frac{\partial u^2}{\partial r^2}+\frac{1}{r}\frac{\partial u}{\partial r}-\frac{u}{r^2}+\frac{1}{h}\frac{\partial h}{\partial r}\left(2\frac{\partial u}{\partial r} + \frac{u}{r}\right)\right)\label{eq:qdm},
%\end{align}

\begin{equation}
\rho\left(\frac{\partial u}{\partial t} + u\frac{\partial u}{\partial r}\right) = - \frac{\partial p}{\partial r} + 2 \mu \left(\frac{\partial u^2}{\partial r^2}+\frac{1}{r}\frac{\partial u}{\partial r}-\frac{u}{r^2}+\frac{1}{h}\frac{\partial h}{\partial r}\left(2\frac{\partial u}{\partial r} + \frac{u}{r}\right)\right).\label{eq:qdm}
\end{equation}
Equations \ref{eq:cdmh} and \ref{eq:qdm} constitute a system of lubrication equations that describe the thinning of a film located between two shear-free bubbles. This system of equations differs significantly from the standard lubrication equations \citep{davis1989,leal2007,michelin2019}. First, in the present configuration, the inertial term is comparable in magnitude to the viscous term. This contrasts with the classical lubrication technique for which the inertial term is negligible with respect to the viscous term for Reynolds number of order one \citep{leal2007}. Second, one may observe a numerical prefactor multiplying the viscous term in \ref{eq:qdm}, which can be interpreted as a Trouton viscosity \citep{howell1996}. This raises the question of why such a notable difference exists compared to the standard lubrication technique. The explanation lies in the boundary conditions at the interfaces. In the standard lubrication technique, the flow is resisted either by the no-slip boundary condition at one of the interfaces \citep{leal2007,michelin2019} or by the flow inside one of the drops \citep{davis1989}. Here both interfaces are free and the flow is extensional \citep{howell1996}.



Integrating Equation \ref{eq:qdm} in $r$, and making use of Equation \ref{eq:u} the pressure profile reads %Using mass conservation equation (Eq. 1), the shear stress can be easily evaluated. 




%Also, by using the kinematic boundary condition Equation ... can be rewritten as
%\begin{equation}
%\frac{\partial h}{\partial t} + \frac{1}{r} \frac{\partial }{\partial r} (urh).
%\end{equation}


 
%\partial w}{\partial z}  = -\left(\frac{\partial u}{\partial r}+\frac{u}{ r}\right)
%The presence of the viscous stress is one of the major difference between the classical lubrication equation and the present analysis.
%The normal stress in the film reads (cf Munro et Lister):
%\begin{equation}
%\sigma_{zz} = -p +2\mu \frac{\partial w}{\partial z}
%\end{equation}
%Since conservation of mass can be written as :
%\begin{equation}
%\frac{\partial w}{\partial z} +  \frac{1}{r}\frac{\partial }{\partial r} (ur) =0
%\end{equation}

%one get :
%\begin{equation}
%\frac{\partial w}{\partial z}  = -\left(\frac{\partial u}{\partial r}+\frac{u}{ r}\right)
%\end{equation}
%hence
%\begin{equation}
%\sigma_{zz} = -p  - 2\mu \left(\frac{\partial u}{\partial r}+\frac{u}{ r}\right)
%\end{equation}
%The pressure distribution inside the film can be obtained by integrating Equation \ref{eq:qdm} with respect to $r$. Once the pressure and the shear stress the force can be calculated.

%Conservation of mass + kinematic bounday condition :

%Pour ce qui est de l'équation pour la pression, il est possible de la trouver de plusieurs manière. En utilisant 3.10 de Savva. Mais en fait on peut carément utiliser son résultat final et remplacé le terme capillaire par la pression +/- les contraintes visqueuses. Dans les faits on retombe sur ce que donne Chesters - une fois la dérive des contraintes visqueuses :

%assume that the sheet between the two bubbles is slender.
%bubble are inviscid 


%Since $\partial h /\partial t = d h_0 /d t =V $ does not depend on $r$, one obtain for the velocity (using the mass conservation) :
%\begin{equation}
%u=-\frac{V}{2} \frac{r}{h} = -\frac{V}{2} \frac{r}{h_0 + r^2/a} 
%\end{equation}
%Using ... we obtain

%\begin{equation}
%\frac{\partial w}{\partial z} = 
%\end{equation}








%Ainsi meme si la pression ne depend pas de la viscosité dans le cas d'un film // on a quand meme les contraintes visqueuses à prendre en compte dans la force car $u/r$ non nul.





%\begin{equation}
%p  = p_\infty -\rho\left(...\right)  + 2 \mu \left(\frac{\partial u}{\partial r }+\frac{u}{r} + \frac{4u^2}{Va}- \frac{u}{r} \right)
%\end{equation}

\begin{equation}
p  = p_\infty -\rho \frac{\dot{h}_0^2}{4}\left(-\frac{a}{h}+\frac{1}{2}\frac{r^2}{h^2}\right)+\rho \frac{\ddot{h}_0}{4}a\log (ha)  - \frac{\mu \dot{h}_0}{h}
\label{eq:pressure}
\end{equation}
where $p_\infty$ is a $\mathcal{O}(1)$ term related to the pressure outside the film. In viscous dominated flows both the viscous component of the pressure (represented by the last term in equation \ref{eq:pressure}) and the normal viscous stress  \ref{eq:viscous_stress} combine to form the overall stress within the thin film. These quantities are singular in the limit of small thickness as they scale proportionally to $\mu \dot{h}_0 / h_0$. Figure \ref{fig:pressure} (a) illustrates their behavior. Interestingly, in contrast to the pressure, the viscous stress cancels near the symmetry axis. Nevertheless both quantities approach zero as the radial distance $r$ approaches infinity. In the opposite limit of inertia-dominated flows, the pressure scales as $\rho\dot{h}_0^2a/h_0$. In this regime also the pressure $p$ decreases as the radial distance $r$ increases (Figure \ref{fig:pressure} b).


%both the viscous part of the pressure (last term in \ref{eq:pressure}) and the viscous stress superimpose to obtain the whole stress in the film. Both are singular in the limit of small thickness since they scale as $\mu \dot{h}_0 / h_0$. They are displayed in figure \ref{fig:pressure} (a). In contrast to the pressure the viscous stress cancels close to the symmetry axis but both tend to zero as $r$ tends to infinity. In the opposite limit of inertia-dominated flows the pressure scales as  $\rho\dot{h}_0^2a/h_0$. We also observe in this regime that $p$ is a decreasing function of $r$.



%In fluid flows where viscosity plays a dominant role, both the viscous component of the pressure (represented by the last term in equation \ref{eq:pressure}) and the viscous stress combine to form the overall stress within the thin film. These quantities exhibit singularity as the film thickness approaches zero, as they scale proportionally to $\mu \dot{h}_0 / h_0$. Figure \ref{fig:pressure} (a) illustrates their behavior.

%Interestingly, in contrast to the pressure, the viscous stress tends to cancel out near the symmetry axis but approaches zero as the radial distance $r$ approaches infinity. On the other hand, in fluid flows where inertia is dominant, the pressure scales as $\rho\dot{h}_0^2a/h_0$. Furthermore, it is noteworthy that in this regime, the pressure $p$ decreases as the radial distance $r$ increases.


% Hence the viscous stress is singular and increases as $\mu (V+ \dot{a}) / h$ as $h$ decreases.


% Figure environment removed



%Bien préciser egalement que l'on a supposé que la vitesse etait constante.


%la conclusion de tout cela est que l'on ne peut pas obtenir quelque chose qui ne depend pas de Rinfty, mais que ce dernier peut etre obtenu par une solution numerique par exemple.

%On va remplacer V/2 par V ca sera plus claire;



%the viscous term introduces a numerical factor in front of the viscosity, which can be interpreted as Trouton viscosity. 


%First, in the present configuration, the inertial term has the same order of magnitude as the viscous term. This contrasts with the classical lubrication technique for which the inertial term is negligible with respect to the viscous term for Reynolds number of order one. Second the viscous term makes appear a numerical factor in front of the viscosity which may be interpreted as Trouton viscosity. Then the question is why is the difference so important with the standard lubircation technique? The explanation relies on the boundary conditions on the interfaces. In the standard lubrication technique, the flow is resisted by either the no-slip boundary condition on one of the interface, or by the flow inside one of the drops. In our case both interfaces are free and the flow is extensional \citep{howell1999}. %A mon avis c'est 'l'extensional stress' qui s'oppose au louvement dans ce cas ou l'inertie.

%Equations \ref{eq:cdmf} and \ref{eq:qdmf} form a set of lubrication equations that govern the thinning process of a slender film situated between two shear-free bubbles. These equations differ significantly from the conventional lubrication equations \citep{davis1989,leal2007,michelin2019} in a couple of ways. Firstly, in the present configuration, the inertial term is comparable in magnitude to the viscous term. This stands in contrast to the classical lubrication approach where the inertial term is considered negligible compared to the viscous term when the Reynolds number is on the order of one. Secondly, the viscous term introduces a numerical factor in front of the viscosity, which can be interpreted as Trouton viscosity. This raises the question of why such a notable difference exists compared to the standard lubrication technique. The explanation lies in the boundary conditions at the interfaces. In the standard lubrication technique, the flow is impeded either by the no-slip boundary condition at one of the interfaces or by the flow inside one of the drops. However, in our case, both interfaces are free, and the flow is characterized as extensional \citep{howell1999}.



\section{Forces on the bubble}
\label{sec:force}

\subsection{Lubrication forces}
The leading-order contribution to the force on the bubble can be obtained by integrating Equations \ref{eq:viscous_stress} and \ref{eq:pressure}. In particular since $\int _0 ^{R_\infty} r/h dr \sim a/2\log\left(R_\infty ^2/(ah_0)\right)$ and $\int _0 ^{R_\infty} r^3/h^2 dr \sim a^2/2\log\left(R_\infty ^2/(ah_0)\right)$ we obtain

\begin{equation}
F = - \pi \mu \dot{h}_0 a \log\left(\frac{R_\infty ^2}{ah_0}\right) + \frac{\pi}{8} \rho \dot{h}_0^2 a^2 \log\left(\frac{R_\infty ^2}{ah_0}\right) + \mathcal{O}(1).
\label{eq:forcef}
\end{equation}
%Note that the viscous stress does not contribute to the force since the integral of the viscous stress vanishes to leading-order. Equation \ref{eq:force} shows that the force is singular when the two drops approach each other but this takes the form of a weak logarithm singularity. However, formula \ref{eq:force} is not closed since $R_\infty$ is a priori unknown. The exact value of $R_\infty$ may be obtained by matching the thin film solution with the outer solution flow solution at least in the Stokes regime \citep{oneill1967}. Here, we make use of the asymptotic formula proposed by \citep{kim1991} based on the bi-spherical coordinate solution of \citet{haber1973} to obtain the value of $R_\infty$. Their solution reads 
%Moreover, the unsteady term proportionnal to $(\dot{V} + \ddot{a})$ gives contribution to the force which are of order one. This is in line with the fact that the added mass do not diverge as the distance between the bubble tends to zero \citep{miloh1977}.
It should be noted that the viscous stress doesn't contribute to the force, as the integral of the viscous stress vanishes to leading-order. Additionally, the force contribution from the unsteady term, which is proportional to $\ddot{h}_0$, is $\mathcal{O}(1)$ and is negligible to leading order. This observation is consistent with the bounded increase of the added mass force as the distance between the bubble approaches zero \citep{miloh1977}. Equation \ref{eq:forcef} indicates that the force has two contributions:  one viscous and one inertial. Both become singular when the two drops get close to each other, but they take the form of weak logarithmic singularities. However, formula \ref{eq:force} is not closed since $R_\infty$ is a priori unknown. The exact value of $R_\infty$ may be obtained by matching the thin film solution with the outer flow solution at least in the Stokes regime \citep{oneill1967}. Alternatively, we can use the trick proposed by \citep{kim1991} based on the matching principle proposed in \citep{oneill1967} and \citep{cooley1969} for the shearing and squeezing motions of rigid particles and replace $\log(R_\infty^2/ah_0)$ with $\log(a/h_0)$. While it is not immediately clear whether this principle can be extended to apply to inertial flow and shear-free surfaces, subsequent analysis reveals that it is indeed valid to the leading order. As a result, it can be inferred that $R_\infty$ varies proportionally with $a$. This may appear to contradict the assumption $R_\infty \ll a$, which is required to satisfy the thin-film assumption. However, it is important to note that because the solution has a logarithmic form, we can choose $R_\infty$ to be equal to $a$ multiplied by a small parameter to satisty $R_\infty \ll a$ without affecting the leading-order solution. Since our derivation for the force is not limited to the Stokes regime, we impose $R_\infty \propto a$ in both term of equation \ref{eq:forcef} such that

%While it is not immediately clear whether this principle can be extended to apply to inertial flow and shear-free surfaces, subsequent analysis reveals that it is indeed valid to the leading order. As a result, it can be inferred that $R_\infty$ varies proportionally with $a$.
%There is \textit{a priori} no reason for this principle to be applicable for shear-free surface and inertial flow but we will see below that it is correct to leading order. Hence $R_\infty \propto a$.

%However, it's important to note that formula \ref{eq:force} is not self-contained because the value of $R_\infty$ is not known a priori. To obtain the exact value of $R_\infty$, one can match the thin film solution with the outer flow solution, at least in the Stokes regime \citep{oneill1967}. Alternatively, we can use the asymptotic formula proposed by \citep{kim1991} based on the bi-spherical coordinate solution of \citet{haber1973} to determine the value of $R_\infty$. Their solution is expressed as follows:


%Based on the solution proposed by ... in bi-spherical coordinate and making use of the decomposition of the infinite series in an inner sum and outer sum, \citet{kim1991} have shown that in sthe Stokes flow regime :

%\begin{equation}

%\end{equation}
%Hence, to leading order $R_\infty \propto a$. We may think that this results contradicts the assumption $R_\infty \ll a$ made to fullfill the thin-film assumption. However, one may recall, that due to the logarithmic form of the solution one may choose $R_\infty = a \epsilon$, where $\epsilon$ is an arbitrary small parameter, whithout changing the leading order solution.
%However in this specific configuration the outer solution can only be obtained in the Stokes flow and potential flow regimes.

%\color{blue}
%a verifier
%\color{black}

\begin{equation}
F = 2\pi \mu (\dot{a} - V) a \log\left(\frac{a}{h_0}\right) + \frac{\pi}{2} \rho ( \dot{a} -V)^2 a^2 \log\left(\frac{a}{h_0}\right).
\label{eq:forcea}
\end{equation}
Formula \ref{eq:forcea} has the advantage of being applicable for arbitrary Reynolds number ($Re = \rho \dot{h}_0a/\mu$), making it a useful tool for analyzing bubble collisions as long as the bubbles experience minimal deformation and the interface can be assumed to be shear free. In the following, we demonstrate that our choice for $R_\infty$ is appropriate.


\subsection{Comparison with exact solutions in the Stokes flow regime}
\label{sec:viscf}

In the Stokes flow regime the first term in equation \ref{eq:forcea} matches the asymptotic formula proposed by \citep{kim1991} in the case of two translating bubbles. Their solution is expressed as $F = -2\pi\mu a V \left[\log\left(a/h_0\right) + A\right]$, where $A= 2(\gamma + \log 2)$ and $\gamma$ is the Euler's constant. \citep{kim1991} have shown that this formula matches very well the exact bi-spherical coordinate solution proposed by \citet{haber1973} up to $h_0/a \approx 0.1$. 
%For two growing bubbles in the Stokes regime, one may use the bi-spherical coordinate solution derived in \citet{michelin2018}. Also  \citet{michelin2018} do not provide the force on the bubble in closed form but require solving a linear system of 4 equations. In the case of two identical bubbles the system can be simplified and explicit form for the force can be derived as shown in Appendix \ref{app:visc}. %\citet{michelin2018} do not provide the force on the bubble explicitly

%to compute the force on the bubbles (see Appendix \ref{app:visc}).
% Figure environment removed

To the best of our knowledge the growing viscous contribution in formula \ref{eq:forcea} has not been derived so far. To verify the accuracy of this result, the bi-spherical coordinate solution proposed in \citet{michelin2018} can be used. However, it is important to note that \citet{michelin2018} did not provide an explicit expression for the force acting on the bubble, instead requiring the solution of a linear system comprising four equations. In the specific scenario of two identical bubbles, the complexity of the system can be reduced, allowing for the derivation of an explicit form for the force, as demonstrated in Appendix \ref{app:visc}. A very good match between the bi-spherical coordinate solution (Equation \ref{eq:bispherical}) and Equation \ref{eq:forcea} prediction is observed by adding a $\mathcal{O}(1)$ term to the analytical expression (Figure \ref{fig:force_visc}). This term cannot be obtained from lubrication theory alone but is easily fitted from the bi-spherical coordinate solution.

%To analyze the dynamics of expanding bubbles in the Stokes regime,





 %asymptotic formula proposed by \citep{kim1991} based on the bi-spherical coordinate solution of \citet{haber1973} to determine the value of $R_\infty$ in the case of two translating bubble. Their solution is expressed as $F = \pi\mu a V \left[ 2\log\left(a/h_0\right) + 4\gamma + 4\log 2\right]$
%\begin{equation}
%F = 2\pi\mu a V \left[ \log\left(\frac{a}{h_0}\right) + 2\gamma + 2\log 2\right]
%\end{equation}
%where $\gamma$ is the Euler constant.


\subsection{Comparison with exact solutions in the potential flow regime}

%\color{blue}
%It is not immediate to know to which type of results we could compare the inertial force. 
It is tempting to compare the inertial force contribution to potential flow results.%to the force derived under the potential flow assumption. 
However one may recall that our solution was not derived under the potential flow assumption. In practice, it appears that to leading order the solution presented here derived from a potential. Indeed to leading order, the flow in the film is a plug flow $\mathbf{u} \approx u(r,t)\mathbf{e}_r$ \citep{savva2009}. Hence $\nabla \times \mathbf{u} \approx \mathbf{0}$. Additionally, it's worth mentioning that the shear-free boundary condition only appears in the viscous term. Therefore to leading order and under the conditions of sufficiently high Reynolds numbers, the present set of equations simplifies to the potential flow limit.

%Hence to leading order, and for sufficiently high Reynolds numbers, the present set of equations reduces to the potential flow limit.
%\color{black}


% Figure environment removed

To verify the accuracy of the second term in \ref{eq:forcea}, we compared it to the potential flow prediction. We make use of the force expression provided by \citet{miloh1977} in the form of infinite linear systems (Appendix \ref{app:pot}) . The forces resisting bubble translation and growth are shown in Figure \ref{fig:forces}. By adding $\mathcal{O}(1)$ terms to the analytical expressions, the agreement between the inertial contribution of the formula \ref{eq:forcea} and the numerical results of \citet{miloh1977} are excellent. Once again, these terms cannot be obtained from lubrication theory but are easily fitted from the exact results. Furthermore, the accurate agreement between the slope of the theoretical curves and the lubrication predictions in the limit of $\epsilon \ll 1$ serves as evidence supporting the validity of our choice for the parameter $R_\infty$. Additionnaly the inertial contribution in equation \ref{eq:forcea} matches the analytical solution obtained by \citep{voinov1969} in the case of purely translating spheres and for $\epsilon \ll 1$. Finally, formula \ref{eq:forcea} yields a significant enhancement for small values of $h_0/a$ compared to the formulae obtained by assuming successive images illustrated as dashed-dotted lines \citep{miloh1977}.

%In summary, we have demonstrated that formula \ref{eq:forcea} provides an accurate way of computing the lubrication force between translating and growing bubble as long as we take into account the order one correction. All the expressions derived here are summarized in table \ref{tab:force}.

%\begin{table}
%  \begin{center}
%  \def~{\hphantom{0}}
%    \begin{tabular}{c c c} 
    %\hline
%                    & Viscous flow & Potential flow \\ 
% Translating force  & $-\mu a V \left(2\pi\log\left(a/h_0\right) + 4\pi(\gamma + \log 2)\right)$        & $\rho V^2 a^2 (\pi / 2 \log (a/ h_0)- 1.96)$ \\ 
% Growing force      & $\mu \dot{a} a (2 \pi \log (a/ h_0) - 4.95)$        & $\rho \dot{a}^2 a^2 (\pi / 2\log (a/ h_0) + 1.28)$ \\ 
    %\hline
%    \end{tabular}
%  \caption{Forces on two growing and translating bubbles.}
%  \label{tab:force}
%  \end{center}
%\end{table}




%, as noted in previous works such as \citet{bjerknes1906,milne1996}. 
 %we have compared it to result derived under the potential flow assumption (Appendix \ref{app:pot}). We make used of the force expression provided by \citet{miloh1977} in the form of infinite linear systems. Figure \ref{fig:forces} displays the force resiting bubble translation and the force resisting bubble growth. The agreement between the inertial contribution of formula \ref{eq:forcea} and the numerical results of \citet{miloh1977} is very good as long as we add a $\mathcal{O}(1)$ term to the analytical expression. This term cannot be obtained from lubirciation theory but is easily fitted from the numerical results.


%To ensure the accuracy of the result, we compared it to the one obtained through the assumption of potential flow (refer to Appendix \ref{app:pot}) using the force expression provided by Miloh (1977) in the form of infinite linear systems. The force resisting bubble translation and growth are shown in Figure \ref{fig:forces}. By adding a $\mathcal{O}(1)$ term to the analytical expression, the agreement between the inertial contribution of formula \ref{eq:forcea} and the numerical results of \citet{miloh1977} is excellent. This term cannot be obtained from lubrication theory but can be easily fitted from the numerical results.


% We see that formula \ref{eq:forcea} provide a dramatic improvement for small $h_0/a$ with respect to the formula derived under the assumption of successive images depicted as dashed dotted lines (see \citet{bjerknes1906,milne1996}). Although, we add a constant to this formula to best match the numerical results.



%All this results prove that our choice for $R_\infty$ is perfectly adequate.


\subsection{Added mass force on purely translating bubbles in close proximity}

%The added mass loads are not singular and cannot be obtained from the lubrication equation alone. However, one may used potential flow results to obtain a closed form expression for the added mass force on a translating spherical particle. In particular in the Lagrangian formulation we have in the case of purely translating bubble :
The added mass force is not singular and cannot be obtained from the lubrication equation alone. However, one may use potential flow properties to obtain a closed-form expression for the added mass force on a translating spherical particle. In particular, Lagrange's equation in the case of purely translating bubbles translating toward each other reads \citep{lamb1993}


\begin{equation}
\frac{d}{dt}\frac{\partial T}{\partial V} - \frac{\partial T}{\partial z_b} = - F
\label{eq:lagrange}
\end{equation}
where $z_b=h_0/2+a$ is the center of mass position, $T = 1/2m_aV^2$ is the kinetic energy of the bubble, $m_a = 4/3\pi a^3\rho c_a(h_0)$ is the added mass and $F$ is the inertial force for translating bubble. Inserting those estimate in equation \ref{eq:lagrange} one obtain a differential equation governing the behaviour of the added mass coefficient $c_a$
\begin{equation}
\frac{d c_a}{dh_0} = -\frac{3}{8}\frac{1}{a}\left(\log \left( \frac{a}{h_0}\right)+C\right).
\label{eq:cad}
\end{equation}
When $h_0/a=0$, \text{i.e.} when the bubbles are touching, \citet{miloh1977} have shown that the added mass coefficient reads $c_a(0) = 3/2\zeta(3) - 1$ where $\zeta$ is the Riemann zeta function . Hence, integrating Equation \ref{eq:cad} one get to leading order
\begin{equation}
c_a = \frac{3}{2}\zeta(3) - 1 - \frac{3}{8}\frac{h_0}{a}\log \left( \frac{a}{h_0}\right) + \mathcal{O}\left(\frac{h_0}{a}\right)
\label{eq:ca}
\end{equation}

% Figure environment removed
We observe a very good match between the potential flow solution and formula \ref{eq:ca} for $h_0/a\leq 0.2$ (Figure \ref{fig:ca}). Unfortunately \citep{miloh1977} did not provide the added mass coefficient related to growing motion. The solution provided by \citep{van2002} might be used, but it is not immediately obvious if one may directly relate the added mass coefficient to the inertial force for purely growing bubble using Lagrange formalism. 


%\color{blue}

%A la fin noter que le calcul de la masse ajoutee pour le cas rayon variable nest pas donne par Miloh mais peut etre trouve chez Van der Geld. Par contre la relation liant ce dernier a la force inertiel est plus complexe
%\color{black}


\section{Film drainage time scale of small translating bubbles in viscous dominated flows}
\label{sec:vaka}
In this section, we explore the extent to which our findings can be used to determinate the film drainage time scale for two translating bubbles approaching each other. A significant limitation of the current analysis, which may restrict its practical application, is the assumption of negligible deformations in the bubble shape relative to a spherical geometry.
In practice, deformations can occur when the two bubbles are in close proximity \citep{chesters1982}. By employing the normal stress balance and equating the hydrodynamic stress to the capillary pressure, which scales as $\gamma / a$ where $\gamma$ represents the surface tension, a criterion for negligible deformations may be established. In viscous dominated flows, the deformation is negligible if the capillary number $Ca = \mu V / \gamma$ is much smaller than $\epsilon$. Conversely, in regimes dominated by inertia, the deformation is negligible if the Weber number $We = \rho V^2 a / \gamma$ is much smaller than $\epsilon$. In practice, these two assumptions necessitate extremely small bubbles and high surface tension. For a bubble rising in water, we can anticipate the current analysis to be applicable for bubbles with a radius of around $10\mu m$. If this condition is fullfilled the present analysis provides a way to calculate the film drainage characteristic time which is the time it takes for the bubbles to drain the film located between them. In the subsequent section we will focus on viscous dominated flow. %and inertia dominated flow. 

%In the upcoming sections, our attention will be directed towards two distinct regimes: flow dominated by viscosity and flow dominated by inertia.

%\subsection{Viscous dominated drainage}

We are not aware of any experimental results measuring the film drainage time scale for small bubbles in viscous flow except the recent experiments of \citet{vakarelski2018}. The absence of prior research can be attributed to two reasons. First, it is very difficult experimentally to avoid the contamination of the interface with impurities \citep{vakarelski2018}. Second, tracking of bubble smaller than 100 $\mu m$ requires high-speed video cameras equipped with microscope \citep{vakarelski2018}. \citet{vakarelski2018} studied the free rise and coalescence of small air-bubbles ($a \geq 50 \mu m$) at a liquid-air interface. They make use of a fluorocarbon liquid with a density of $\rho = 2030$ kg.m$^{-3}$ and a viscosity of $\mu = 0.0192$ Pa.s, which is highly resistant to surface-active contamination. Their findings reveal that the coalescence time $t_c$ for bubbles with radii $250 \mu$m is approximately $3.6$ ms. Although the bubble size in their experiment is larger than that required for the negligible deformation assumption, we will demonstrate that our model agrees with their experimental results. Indeed our model can also be applied to the rising of a bubble at a free surface since the plane of symmetry separating the two bubbles is a shear free surface. Replacing $h_0$ by $h_0/2$ in the viscous translating force given by \citet{kim1991} and balancing it with the buoyancy force $4/3\pi a^3\rho g$ we obtain

%($\rho = 2030$ kg.m$^{-3}$, $\mu = 0.0192$ Pa.s) which is highly resistant to surface active contamination. They measured the coalescence time $t_c$ for bubbles having radii larger than $100 \mu$m and found $t_c \approx 2 - 3$ ms. Although the size of the bubble is higher than the one prescribed to fulfil the negligible deformation assumption before we will see that our model gives pretty good agreement with the experimental results. Balancing  the first term of Equation \ref{eq:forcea} with the buoyancy force $4/3\pi a^3\rho g $ and integrating we obtain



%Regrettably, there are no existing experimental findings on the coalescence time of small bubbles, apart from the recent work by \citet{vakarelski2018}. The absence of prior research can be attributed to two reasons. Firstly, experimental contamination of the interface with impurities is highly challenging to avoid \citep{duineveld1994, vakarelski2018}. Secondly, monitoring bubbles smaller than 100 $\mu m$ requires high-speed video cameras equipped with microscopes \citep{vakarelski2018}. In their study, \citet{vakarelski2018} explored the free rise and coalescence of small air bubbles ($a \geq 50 \mu m$) at a liquid-air interface. They utilized a fluorocarbon liquid with a density of $\rho = 2030$ kg.m$^{-3}$ and a viscosity of $\mu = 0.0192$ Pa.s, which is highly resistant to surface-active contamination. Their findings revealed that the coalescence time $t_c$ for bubbles with radii exceeding $100 \mu$m was approximately $2-3$ ms. Although the bubble size in their experiment was larger than that required for the negligible deformation assumption, we will demonstrate that our model agrees significantly with their experimental results. By balancing the first term of Equation \ref{eq:forcea} with the buoyancy force $4/3\pi a^3\rho g$ and performing integration, we obtain...


%To compute the coalescence time we make use of the first term of Equation \ref{...} Balancing ... with the buoynacy force $4/3\pi a^3\rho g $ on the bubble and intergrating one get. This problem is similar to the rising of a bubble at a free surface  

%To the best of the author knowledge the experiments dealing with the smallest bubble are the one of \citet{varaleski2018}.
%In the following We compare ou results to the experimental results of Varaleski 2018. 


%Consequently, this analysis provides a way to calculate the coalescence time of very small bubbles, which is the time it takes for the bubble to drain the film located between them.

%Regrettably, there are no experimental findings for such small bubbles that we are aware of.



\begin{equation}
(2\gamma + \ln 2 + 1)h_0^* - h_0^*\log(h_0^*) = (2\gamma + \ln 2 + 1)h_0^*(0) - h_0^*(0)\log(h_0^*(0))-\frac{2}{3}t^*
\end{equation}
where the equation of motion has been integrated once with respect to time and the stared variables are dimensionless quantities. Specifically $h_0 = a h_0^*$ and $t = \mu /(\rho a g) t^*$. In our analysis, we have neglected the effects of bubble inertia and inertial lubrication force on the force balance, which is acceptable since the Reynolds number associated with bubble motion is significantly less than unity. Since the coalescence time $t_c$ is defined as the duration required for the film to reach a zero thickness we get

% To obtain this results we have neglected the bubble inertia as well a the inertial lubrication force in the force balance. Those assumptions are valid since the Reynolds number related to the bubble motion is much smaller than one. Since the coalescence time $t_c$ is defined as the time needed for the film to reach a zero thickness we get

% to coaslece, we inject $h_0=0$ in ... and find At the end, one get the dimensionless induction time
\begin{equation}
t_c^* = \frac{3}{2}h_0^*(0)\left(2\gamma + \ln 2 + 1 - \log(h_0^*(0))\right).
\end{equation}
%\citet{vakarelski2018} do not specifiy the thickness of the film at which the coaslecence time starts to be measured denoted $h(0)$, but specify that this time is calulated once the bubble reaches the interface. Since their spatial resolution is larger than $2\mu$m, one may expect $h(0)$ to be ranging between $2\mu$m and a few pixels let say $10\mu$m. Taking $h_0^*(0)=0.1$ ($h(0)=10\mu$m) we get $t_c^*\approx 0.5$ and $t_c \approx 4.8$ ms while for $h_0^*(0)=0.02$ ($h(0)=2\mu$m)we get $t_c^*\approx 0.15$ and $t_c \approx 1.4$ ms in very good agreement with the experimental values.
One may note that in contrast to solid particles, the film drainage process occurs in a finite time. \citet{vakarelski2018} did not explicitly state the initial thickness $h_0(0)$ at which coalescence time measurements were conducted. However, they indicated that the time was determined once the bubble reached the interface. Given their spatial resolution exceeding $2\mu$m, it is reasonable to assume that $h_0(0)$ falls within the range of $2\mu$m to a few pixels, around $10\mu$m. Using $h_0^*(0)=0.1$ ($h_0(0)=10\mu$m) yields $t_c^*\approx 0.77$ and $t_c \approx 6$ ms, while $h_0^*(0)=0.02$ ($h_0(0)=2\mu$m) leads to $t_c^*\approx 0.2$ and $t_c \approx 1.6$ ms. These values are in good agreement with the experimental findings. However the agreement between the model and the experiments is questionable. Specifically, in the experiments conducted by \citet{vakarelski2018}, the capillary number based on the terminal velocity is approximately $2.9 \times 10^{-3}$. Consequently, for $h_0/a < 10^{-2}$, significant interface deformation is expected, indicating the possibility of an alternative regime of film drainage. Measurements of the film thickness of gas bubbles ascending towards a free surface under gravity were performed by \citet{kovcarkova2013}. They observed an exponential reduction in film thickness over time, with a characteristic time scale proportional to $\mu a / \gamma$ for small bubbles. For the present system, $\mu a / \gamma$ is approximately $0.09$ ms, indicating that the time scale for drainage after deformation is considerably shorter than the previously calculated value, providing strong confidence in the estimated coalescence time. It should be noted that our coalescence time prediction exhibits a behavior of $a^{-1}$ for a fixed initial film thickness $h_0(0)$, while \citet{vakarelski2018} observed an increase in coalescence time proportional to $a^2$ for larger bubbles. This probably indicates that the results reported by \citet{vakarelski2018} for the smallest bubble lie at the boundary between the Taylor regime of film drainage, where deformation is negligible, and the Reynolds regime, where deformation is non-negligible \citep{ivanov1999}.



\section{Dynamics of two growing bubbles in viscous dominated flows}
\label{sec:ohashi}

%It is interesting to discuss the ability of formula \ref{eq:forcea} to predict the dynamics of two force-free growing bubble in the Stokes regime. A similar analysis have been performed in \citet{michelin2019} for a bubble closed to the wall. They found that depending on the boundary condition on the bubble surface the bubble can either monotonically drain the fluid separating it from the wall or it will bounce from the surface once before draining the film. Specifically in their configuration the translational lubrication force which scales as $1/\epsilon$ is significantly more singular than the inflating lubrication force (with $h_0$ fixed) which leads to a weak logartihm singularity. In the present configuration, the the translational lubrication force is itself a weak logartihm singularity while the inflating contribution to the lubrication force (with $h_0$ fixed) leads to term smaller than the order $1$, since the part of the velocity involved in this case is $-r^2 \dot{a}/a^2\sim \epsilon \dot{a}$. As a result if we consider two force free bubble in the stokes regime, we obtain from equation ... 



Formula \ref{eq:forcea} may be used in principle to predict the dynamics of two growing bubbles in the Stokes regime without external forces. A related analysis was conducted by \citet{michelin2019} for a bubble in close proximity to a wall. Their findings indicated that the boundary conditions on the bubble surface determined whether the bubble would either continuously drain the fluid separating it from the wall or rebound once before draining the film. In their configuration, the translational lubrication force has a singularity that scales as $1/\epsilon$, which is significantly more singular than the inflating lubrication force (with $h_0$ fixed) that produced a weak logarithmic singularity. In contrast, in the present configuration, the translational lubrication force itself has a weak logarithmic singularity, while the inflating contribution to the lubrication force (with $h_0$ fixed) generates terms smaller than the order one, as the velocity $u$ involved in this case is $-r^2 \dot{a}/a^2\sim \epsilon \dot{a}$ is negligible. Consequently, when considering two force-free bubbles in the Stokes regime, Equation \ref{eq:forcea} leads to $\dot{h}_0=0$, which implies that the film thickness does not vary over time. This result contrasts with those of a recent experimental investigation by \citet{ohashi2022} who observed the decrease in time of the film thickness of two adjacent, expanding bubbles with low capillary number. To accurately describe the dynamics of two growing bubbles, it is necessary to include the $\mathcal{O}(1)$ contributions to both the inflating and translating force. In the case of force-free bubbles, and due to the linearity of the Stokes equation this yields $-(\log(a/h_0) + A)V + (\log(a/h_0) + B)\dot{a} = 0$, where $A$ and $B$ are defined in section \ref{sec:viscf}. The force-free condition finally reads

%$A = 2(\gamma + \log 2)\approx 2.54 $ represents the $\mathcal{O}(1)$ term associated with the translational force (divided by $2\pi$), as determined by \citet{kim1991}, and $B \approx -4.95/(2\pi)\approx -0.79$ represents the $\mathcal{O}(1)$ term associated with the growing force (see Figure \ref{fig:force_visc}). 


%The findings contrast with those of a recent investigation by \citet{ohashi2022} who observed the coalescence of two adjacent, expanding bubbles with low capillary number within a finite time.
%. In principle a numerical estimate of $B$ might be obtained using the bi-spherical coordinate solution proposed by \citet{michelin2018} but since it does not support our point. 


%This contradict the recent results of \citet{ohashi2022} in which two adjacent growing bubbles with low capillary number coalesce in a finite time. As a result to properly capture the dynamics of the two growing bubble one need to compute the $\mathcal{O}(1)$ contribution to both the tranlating and inflating force problem. In particular, in the case of force free bubble one will obtain $(\ln (a/h_0)+A)V+(\ln (a/h_0)+B)\dot{a}=0$ where $A$ is the $\mathcal{O}(1)$ term related to translational motion which have been computed by \citet{kim2013} and $B$ a $\mathcal{O}(1)$ term related to growing motion.

%Indeed if we do not do this we get $\dot{h_0}=0$. Which mean that the thickness do not evolves as function of time ! 

%Discuter le fait qu'à l'ordre dominant les forces a vitesse ou rayon variable sont les memes. En fait c'est en accord avec Michelin 2019. Les différences n'apparaissant que si l'on considere le terme en epsilon dans la vitesse. Pour des bulles je pense que d'autre terme sont important avant celui la. Par contre ssans la solution d'ordre 1ce n'est pas tres precis. Deriver le cas 2D ? et comparer aux cas en cellcule de Hele-Shaw de lequipe japonaise ?

\begin{equation}
%\frac{\dot{h}_0}{\dot{a}} = -\frac{4.95 + 4\pi(\gamma + \log 2)}{\log(a/h_0) + 4\pi(\gamma + \log 2)}
\frac{\dot{h}_0}{\dot{a}} = \frac{2(B - A)}{\log(a/h_0) + A}
\label{eq:h_0dot}
\end{equation}

Since $(B-A) \approx -3.28$ is negative Equation \ref{eq:h_0dot} indicates that the film thickness is a decreasing function of time for two growing bubbles in agreement with the experimental results of \citet{ohashi2022}. They consider the coalescence of two growing bubbles in highly viscous liquids. Two approximatively equal-sized bubbles were injected thanks to syringe inside a closed box. Then the pressure inside the box was reduced.
We can go further in the comparison with \citet{ohashi2022} experiments by directly comparing $\dot{h}_0/\dot{a}$ to their experimental values. We consider the experiment with the smallest capillary number based on the grow rate $Ca = \mu \dot{a}/\gamma \approx 0.005$ originally denoted case (c) in the article of \citet{ohashi2022}. The agreement with the experimental results is very good in the range $0.02\leq \epsilon \leq 0.2$ (Figure \ref{fig:ohashi}). For larger thickness, our solution is not supposed to work, while for smaller thickness, deformation can be significant. %Also the good agreement is particularly 

% ce wui est normal car pour des epaisseurs trop forte ce n'est pas cense fonctionner mais pour des epaisseurs trop importante ni pour des epaisseurs trop faible.

%smallest bubble of the experiments ($a(0) \approx $) such that to maintain the capillary number as small as possible. 


% Figure environment removed

%discuter des limites de Ohashi : on est pas exactement force free, nombre capillaire etc.

%PLus montrer si ou non on a coalescence en un temps fini. Ne semble pas faisable en pratique

%on va ajouter le cas 2D de Ohashi ce serait trop bete de pas essayer. sauf que il faut se coltiner la derivation de tout car ce que donne Savva c'est juste la formule avec la capillarite (cela n'apporte pas grand chose). En pratique dire que l'on a une decroissance plus forte en racine de epsilon et mettre cela dans la discussion.


\section{Conclusion}
\label{sec:conc}

In this article, we have computed the lubrication force between two translating and expanding bubbles. Our findings demonstrate that the lubrication theory successfully captures the dominant term in the force expression for small inter-bubble gaps, regardless of the Reynolds number. However, for more accurate results, it is necessary to include $\mathcal{O}(1)$ terms in the force expression. Although, we obtained those terms in the Stokes and potential flow limits, one may anticipate obtaining it for arbitrary Reynolds numbers. One potential approach is to employ direct numerical simulations, as demonstrated in the works of \citet{teng2022,terrington2023}. They explicitly compute the $\mathcal{O}(1)$ term in the force expression in the case of a rotating and possibly translating cylinder nearby to a plane wall.

%We have shown that the lubrication theory provided the leading order term solution in the limit of small thickness for arbitrary Reynolds number. However, O(1) terms in the force expression are needed to accurately match the numerical results. Although those numerical coefficients were obtained from previous solutions in the Stokes and potential flow regimes, one may anticipate obtaining those constants for arbitrary Reynolds numbers. In practice, one may use direct numerical simulation as in the work of \citet{teng2022,terrington2023} to compute explicitly calculate the O(1) term in the force expression.

%In this article, we have performed calculations to determine the lubrication force between two translating and expanding bubbles, considering a wide range of Reynolds numbers. Our findings demonstrate that the lubrication theory successfully captures the dominant term in the force expression for small inter-bubble gaps, regardless of the Reynolds number. However, for more accurate results, it is necessary to include terms of order O(1) in the force expression, which can be obtained by incorporating numerical coefficients derived from previous solutions in the Stokes and potential flow regimes. It is worth noting that obtaining these coefficients for arbitrary Reynolds numbers is feasible, and one potential approach is to employ direct numerical simulation techniques, as demonstrated in the works of \citet{teng2022,terrington2023}, which explicitly calculate the O(1) term in the force expression.


%O(1) term in the force expression

%We then applied our results to two well-known problems of the literature in viscous dominated flows: the coalescence of a small bubble at a free surface, and the dynamics of two force-free growing bubbles. In both problems, a good agreement with the experimental results was obtained giving strong support to the theory at least in the Stokes regime. A natural extension of the present work might be to compare the present theory to experimental results in inertia dominated regime. However, to the best of the author's knowledge, there are very few experiments to validate the present theory since it appears difficult in practice to verify $Re \gg 1$ and $We\ll 1$. Moreover, the force balance in inertia-dominated flows involves more forces than that given by the present theory. In particular, it involves the drag force on the bubbles at high Reynolds numberq, which may be calculated thanks to the dissipation in the liquid outside the bubbles. It is not immediately clear whether this force may be obtained using the current methodology.


Subsequently, we applied our findings to two well-known problems in viscous-dominated flows: the coalescence of a small bubble at a free surface and the dynamics of two force-free expanding bubbles. Remarkably, both scenarios exhibited excellent agreement with experimental observations, providing strong support for our theory in the Stokes regime. A natural extension of the present work would involve comparing our theory with experimental results in the regime of significant inertia. However, to the best of our knowledge, there exist only a limited number of experiments capable of validating our theory under the conditions of high Reynolds numbers ($Re \gg 1$) and low Weber numbers ($We \ll 1$). Furthermore, in inertia-dominated flows, the force balance involves additional forces beyond those accounted presently. In particular, it involves the drag force exerted on the bubbles at high Reynolds numbers, which can be calculated using dissipation within the surrounding liquid \citep{van2002}. The feasibility of obtaining this force using the present methodology is a matter of future research.%It is not sure whether our current methodology can be used to derive this force.

%In the case of translating bubbles toward each other the force balance reads (neglecting the viscous dissipation and Levich contribution)
%Dire que les résultats peuvent aussi sappliquer en pratique dans des cas tres inertiels, meme si il est diffiicle en pratique davoir $Re>>1$ et $We \ll 1$. Discuter des deux cas (translation \& grossissement dans ce cadre). L'inertie va un peu jouer mais a priori pas un effet majeur. La force dissipation de Levich aussi mais également difficile den dire plus. je vais regarder ca.


%on ne parle pas du cas 2D cela nous emmene trop loin.
%\begin{equation}
%mettre masse ajoutee plus force
%\end{equation}
%plus disucter du temps d'induction 


%Ensuite on passe au cas force free ac inertie. En fait dans ce cas on peut considerer que l'on commence par exemple avec une vitesse initale nulle et on fait grossir les bulles. on peut juste discuter de l'effet d'inertie plus le fait que 




%plus discuter que les constantes peuvent etre trouves via des resultats numeriques comme dans le papier rcent de Terrington et hourigan. oui car en pratique les constantes que l'on donne ne sont valables que pour les regimes stokes vs potentiel. entre les deux les constantes sont sans doute différentes

%Throughout this paper, the bubble interfaces were assumed to be shear-free. In practice, this assumption imposes a rather severe constraint on the viscosity ratio. Indeed the shear stress within the liquid film scales as $\mu u / a$ since the flow is dominantly radial and the shear stress is only the result of a small Poiseuille contribution. Within the bubble, the appropriate length scale is $\sqrt{ah_0}$ and the shear stress scales as $\mu _b u/\sqrt{ah_0}$ where $\mu_b$ is the gas velocity. For the interfaces to be shear free one should have $\mu _b/\mu \ll \sqrt{h_0/a}$ \citep{davis1989}. In practice, one may expect a significant effect of the gas viscosity in the case of an air-water system ($\mu _b/\mu \approx 0.018$) for $h_0/a \approx 1e-3$. Given its practical relevance, there is a definite need to examine these effects in future studies.


%3 / 3

Throughout this paper, the bubble interfaces were assumed to be shear-free. However, this assumption imposes a rather stringent constraint on the viscosity ratio. The shear stress at the interface arises primarily from the small Poiseuille contribution within the liquid film. Hence the shear stress in the film scales as $\mu u / a$. Within the bubble, the appropriate length scale is $\sqrt{ah_0}$, and the shear stress scales as $\mu_b u/\sqrt{ah_0}$, where $\mu_b$ is the gas viscosity. In order for the interfaces to be shear-free, it is necessary to satisfy the condition $\mu_b/\mu \ll \sqrt{\epsilon}$ \citep{davis1989}. In practical scenarios, one can expect significant effects of gas viscosity, especially in an air-water system ($\mu_b/\mu \approx 0.018$) when $\epsilon \approx 0.001$. Considering the practical relevance of that effect, there is a definite need to examine it in future studies.


%(je ne sais pas si on doit aller dans ce detail mais peut ere juste donne le scaling de fully mobile). Also one may expect a singificant effect of the viscotiy ratio for smalle thickness

%Throughout this paper, the bubble interfaces were assumed to be shear-free. In practice this assumption impose a rather constraint on the viscosity ratio. The shear stress within the liquid film scales as $\mu u / a$ since the flow is dominantly radial and the shear stress is only the results of the small poiseuille contribution. Whithin the drop the appropriate time scale is (je ne sais pas si on doit aller dans ce detail mais peut ere juste donne le scaling de fully mobile). Also one may expect a singificant effect of the viscotiy ratio for smalle thickness

%Comment font les auteurs qui ont des solutions logarithtmiques (Voinov etc) pour ne pas se coltiner un R infini ?


%The main results of this paper are the derivation of the lubrication force between two shear free bubbles. 



%compliqué d'inclire la cavitation car cette derniere induit une forte deformation des inclusons.

%However, one should point out that our prediction for the coaslescence time behaves as $a{^-1}$, for a given $h_0(0)$, while \citet{vakarelski2018} reported an increase of the coalescence time with $a^2$ for larger bubbles. Hence the results of \citet{vakarelski2018} for the smallest bubble are tipically located at the junction between the so called Taylor regime  of film drainage (no deformation) and Reynolds regime with non-negigible deformation \citep{ivanov1999}.  


%Indeed $Ca\approx 2.9\times 10^{-3}$, based on the terminal velocity, in the experiments of \citet{vakarelski2018}. Hence the deformation will be non-negligible for $h_0/a < 10^{-2}$and one may expecect another regime of film drainage with significant deformation of the interface. \citet{kovcarkova2013} measured expreimentally the film thickness of gas bubbles rising under gravity toward a free surface. They reported an exponential decrease with time of the film thiskness  whose time scales as $\mu R \gamma$ for small bubbles. In the present configuration $\mu a / \gamma \approx 0.09$ms. Hence the charcateristic time of drainage once deofmration occurs is much smaller than the one found before giving strong confidence in our results. 



%The validity of the model's predictions has been called into question due to discrepancies with experimental data. Specifically, in the experiments conducted by \citet{vakarelski2018}, the capillary number ($Ca$) based on the terminal velocity is approximately $2.9 \times 10^{-3}$. Consequently, for $h_0/a < 10^{-2}$, significant interface deformation is expected, indicating the possibility of an alternative regime of film drainage. Empirical measurements of the film thickness of gas bubbles ascending towards a free surface under gravity were performed by \citet{kovcarkova2013}. They observed an exponential reduction in film thickness over time, with a characteristic time scale proportional to $\mu R \gamma$ for small bubbles. For the present system, $\mu a / \gamma$ is approximately $0.09$ ms, indicating that the time scale for drainage after deformation is considerably shorter than the previously calculated value, providing strong confidence in our results.


%ONe may ask why the drainage is so fast once deformation occurs. This is related to the

%It is in some sense interesting to note that the drinage is so fast. This is due to the small spherical cap area to be drained. %This also mean that the present results for the coalescence time are valid for $\mu a / \gamma \ll \mu /(\rho a g)h_0/a$ n'apporte rien


%Alhough the velocity  and hence the Capillary number decrease with time as the film thickness decreases, this value is correct ate leat in order of magnitude.

%a value very closed to the experimental one. Hence one may safely says that the induction time can be correctely predicted using the present model.




%In Varaleski experiments for the smallest bubble of diameter $a\approx 100 \mu m$, one have : $t_i^* \approx 0.4$. Also, Varaleski et al. do not specifiy the point for whichthe induction is measured. Taking $h_0^*(0)=0$ we get $t_i^*=0.5$ a value very closed to the experimental one. Hence one may safely says that the induction time can be correctely predicted using the present model.


%en fait quand on compare Varaleski 2018, on a une tres legere deviation par rapport a la solution de Bart pr des distances faible. on peut au moins le discuter.

\backsection[Acknowledgements]{The support of V\'eronique Lachet is gratefully acknowledged. The author is indebted to Professor Masatoshi Ohashi for providing the experimental datas of \citet{ohashi2022}.}%I would like to thank Professor Ohashi for providing his experimental results}

\backsection[Funding]{The financial support of IFP Energies Nouvelles is acknowledged.}

\backsection[Declaration of interests]{The authors report no conflict of interest.}

\appendix

\section{Force on growing bubbles in viscous flow}
\label{app:visc}
%In this appendix we compute the force on two identical growing bubble based on the bi-spherical solution proposed by \citep{michelin2018}. Since the two bubble are identical the plane of symmetry separating the bubbles is a shear-free boundary. As a result the coefficients $A_n$ and $C_n$ defined in \citep{michelin2018} are zero and the force reads 

In this appendix, we calculate the force exerted on two identical expanding bubbles using the bi-spherical solution proposed by \citep{michelin2018}. Due to the identical nature of the bubbles, the plane of symmetry that separates them is a shear-free boundary. Consequently, the coefficients $A_n$ and $C_n$ defined in \citep{michelin2018} are both zero, leading to the following expression for the force

\begin{equation}
F = \frac{2\pi \sqrt{2}\mu \dot{a}a^2}{k}\sum_{n=1}^{\infty}\left(\frac{2n+1}{4n+2}\right)(B_n+D_n)
\label{eq:bispherical} 
\end{equation}
with 
\begin{equation}
B_n = \frac{-U_n''+(n+3/2)^2U_n}{\sinh[(n-1/2)\eta]} \quad \text{and} \quad D_n = \frac{U_n''-(n-1/2)^2U_n}{\sinh[(n+3/2)\eta]}.
\end{equation}
The functions $U_n$ and $U_n''$ are defined as
\begin{equation}
U_n(\eta) = -\frac{3\sqrt{2}\sinh^2(\eta/2)}{2(2n+1)}  \left(\frac{e^{-(n+3/2)\eta}}{2n+3} - \frac{e^{-(n-1/2)\eta}}{2n-1}\right) - \frac{\delta_{n1}\sqrt{2}}{3}(e^{\eta/2}-e^{-\eta/2})+\frac{\sqrt{2}\sinh^2\eta}{2(2n+1)}e^{-(n+1/2)\eta},
\end{equation}
and
\begin{align}
U_n''(\eta) =& -\frac{3\sqrt{2}\sinh \eta}{8}  \left(-\frac{2\sinh^2 \eta e^{-(n+1/2)\eta}}{1+\cosh\eta}+(2n+3)e^{-(n-1/2)\eta}-(2n-1)e^{-(n+3/2)\eta}\right) \nonumber\\
 &- \frac{\delta_{n1}}{\sqrt{2}}(e^{\eta/2}-e^{-\eta/2}) - \left(n-\frac{1}{2}\right)\left(n+\frac{3}{2}\right)U_n(\eta),
\end{align}
where $k = \sqrt{(h_0+2a)^2/4-a^2}$ and $\sinh \eta = k/a$. In practice we truncate the infinite sum to a finite number $N \geq 1000$.
%\begin{equation}
%B_n = \frac{-U_n''+(n+3/2)^2U_n}{\sinh(n-1/2)\eta(4n+2)}
%\end{equation}

\section{Forces on translating and growing bubbles in potential flow}
\label{app:pot}
In this appendix, we compute the inertial and added mass forces based on the potential flow solution provided by \citet{miloh1977}. He makes use of the Lagally theorem to compute the inertial force on spherical particles in various configurations. Specifically, he considers a bubble approaching a wall and a bubble growing next to a wall. Due to symmetry reason, these two scenarios can be equivalently represented as two bubbles approaching each other and two bubbles undergoing growth. In the former case, the inertial force reads

%Due to symmetry reason those two problems are equivalent to two bubbles approaching each other and two growing bubbles. 

%The author utilizes the Lagally theorem to calculate the inertial force acting on spherical particles in different arrangements. Specifically, the investigation focuses on a bubble approaching a wall and a bubble expanding adjacent to a wall. By considering the symmetry of these scenarios, it is established that these two problems can be equivalently represented as two bubbles approaching each other and two expanding bubbles.

\begin{equation}
F = 4\pi \rho a^2 V^2\sum_{n=0}^\infty (n+2)A_nA_{n+1},
\label{eq:F_zMiloh}
\end{equation}
where the coefficients $A_n$ obeys the following equation
%formula 29 de Miloh
\begin{equation}
A_n = -\frac{1}{2}\delta(n-1) + \frac{n}{(n+1)!}\sum _{m=1}^\infty A_m \frac{(m+n)!}{m!}\left(\frac{1}{2+h_0/a}\right)^{m+n+1}. 
\end{equation}
The function $\delta(n)$ equals 1 for $n=0$ and equals 0 otherwise. In this case the added mass coefficient reads $c_a = |1+3A_1|$. In the case of two growing bubbles, the force reads 
\begin{equation}
F = 4\pi \rho a^2 \dot{a}^2\sum_{n=0}^\infty (n+2)D_nD_{n+1},
\label{eq:F_zMiloha}
\end{equation}
where the coefficients $D_n$ follow
\begin{equation}
D_n = -\delta(n) + \frac{n}{(n+1)!}\sum _{m=0}^\infty D_m \frac{(m+n)!}{m!}\left(\frac{1}{2+h_0/a}\right)^{m+n+1}. 
\end{equation}
Solving the infinite sets of linear equations above involves truncating the infinite sum to a finite number $N$ and solving a sequence of finite linear systems. To avoid overflow errors arising from the presence of factorial terms during computation, we have selected a practical value of $N=86$ in the main body of the paper. This value is significantly larger than the initial value chosen by \citet{miloh1977} $N=30$. 

%\color{blue}
%ajouter la masse ajoutee.
%\color{black}


%In practice increasing $N$ has significant effect for small value of $h_0/a$.

%\color{blue}
%\section{Forces on translating and growing bubbles in potential flow}
%\label{app:pot_vandergeld}
%ajouter la masse ajoutee. et citer le papier de Yang (totalement empirique la facon de faire donc on ajoute clairement quelque chose). Non il donne la constante necessaire pour la masse ajoutee de maniere analytique !)
%\color{black}



% demonstrated by Figure \ref{fig:force_N} increasing $N$ has significant effect for small value of $h_0/a$. This is due to the fact that the convergence rate for small distances is slow.

%Those infinite sets of linear equations can be solved by truncating the infinite sum beyond a finite number $N$ and solve the sequence of finite linear systems. In practice we have choosen $N=86$ since due to the presence of factorial terms the computation lead to overflow error. This value is significantly larger than the one initially choosen by Miloh, since in the inertial force slowly converge for small distance as shown by figure ... Significant imporvement is obseverd in figure ...
%can be solved using the method of reduction. In brief, we


%% Figure environment removed



%consider a finite number 

%\section{formula}
%\begin{equation}
%\int _0 ^{R_\infty} \frac{r}{h} dr \sim \frac{1}{2}a\log\left(\frac{R_\infty ^2}{ah_0}\right)  
%\end{equation}
%\begin{equation}
%\int _0 ^{R_\infty} \frac{r^3}{h^2} dr \sim \frac{1}{2}a^2\log\left(\frac{R_\infty ^2}{ah_0}\right)  
%\end{equation}
%Donc les contraintes visqueuses ne contribuent pas à la force (au moins a l'ordre dominant)


%\begin{equation}
%\frac{\partial p}{\partial r} = -\rho\left(\frac{\partial u}{\partial t} + u\frac{\partial u}{\partial r}\right)  + 2 \mu \left(\frac{\partial u^2}{\partial r^2}+\frac{1}{r}\frac{\partial u}{\partial r}-\frac{u}{r^2}+\frac{1}{h}\frac{\partial h}{\partial r}\left(2\frac{\partial u}{\partial r} + \frac{u}{r}\right)\right)
%\end{equation}


%\begin{align}
%\int \frac{\partial u ^2}{\partial r ^2} dr= \frac{\partial u}{\partial r } +   cste = \frac{V}{h} \left(1- \frac{2r^2}{ah}\right) \\
%\int \frac{1}{r}\frac{\partial u}{\partial r}-\frac{u}{r^2} dr = \frac{u}{r} +   cste,\\
%\int \frac{1}{h}\frac{\partial h}{\partial r}2\frac{\partial u}{\partial r} dr = \frac{2V}{a}\frac{r^2}{h^2} \\
%\int \frac{1}{h}\frac{\partial h}{\partial r}\frac{u}{r} dr = -\frac{1}{2}\frac{\partial h_0}{\partial t} \int \frac{1}{h^2}\frac{\partial h}{\partial r} = - \frac{u}{r} + cste
%\end{align}

%\section{Lubrication force on 2D bubbles}


%\section{Acknowledgment}
%This research was funded by grants from IFP Energies Nouvelles.
\bibliography{biblio.bib}
\bibliographystyle{apalike-fr}


\end{document}
%
% ****** End of file aiptemplate.tex ******
