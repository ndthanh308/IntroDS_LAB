%\documentclass[10pt,a4paper]{article}
\documentclass[]{jfm}
%\usepackage{newtxtext}
%\usepackage{newtxmath}

%\usepackage[utf8]{inputenc}
%\usepackage{amsmath}
\usepackage{amsfonts}
\usepackage{amssymb}
\usepackage{amsmath,bm}
%\usepackage{pgfplots}
%\usetikzlibrary{shapes.geometric}
%%

%\usepackage[bmargin=2.cm,tmargin=2.cm,lmargin=2cm,rmargin=2cm]{geometry}
\usepackage{natbib}

\usepackage{psfrag}
%\usepackage{pstool}
\usepackage{graphicx}
\usepackage{grffile}%allow .1.jpeg

\usepackage{hyperref}
\usepackage[final]{pdfpages}
\usepackage{url} %citer un lien

\usepackage[]{subcaption}
\captionsetup[subfigure]{skip=+10pt}

\usepackage{tikz}
\usetikzlibrary{positioning}

\usepackage{authblk}

\definecolor{blue_py}{RGB}{31,119,180}
\definecolor{orange_py}{RGB}{255, 127, 14} 
\definecolor{green_py}{RGB}{44, 160, 44}
\definecolor{red_py}{RGB}{214, 39, 40} 




\begin{document}

\title{Viscous lubrication force between spherical bubbles with time-dependent radii}
%\title{Viscous lubrication force between spherical bubbles or low viscosity droplets with time-dependent radii}
\author[1]{\small Jean-Lou Pierson}
\affil[1]{\small IFP Energies Nouvelles, Solaize, 69360, France}
\affil[*]{ jean-lou.pierson@ifpen.fr}
%\homepage[]{Your web page}
%\thanks{}
%\altaffiliation{}





\date{\today}




\maketitle 

\begin{abstract}
Motivated by the dynamics of microbubbles in dissolved gas flotation processes, we consider theoretically the approach between two shear-free spherical bubbles with time-dependent radii. We make use of the lubrication assumption to obtain the thin film flow between the bubbles. Our analysis underscores that for the shear-free condition and spherical shape assumption to hold, both the viscosity ratio and the capillary number must be significantly smaller than the thickness of the film. We demonstrate that the lubrication force exhibits weak singular behavior, scaling logarithmically with the ratio of bubble radius to film thickness. To assess the accuracy of our findings, we compare the obtained solution to results from Stokes flow theory. The comparison demonstrates that our current results are reliable, provided that we combine the lubrication forces with subdominant corrections, which require proper matching and computation to the solution far from the film. In practice, we compute these subdominant corrections in the case of two equal bubbles or a bubble close to a plane-free surface either by a curve fit of numerical results from bi-spherical coordinate solutions or by using results from the literature.
We illustrate the relevance of the solution to determine the drainage time of a small bubble rising to a free surface and the drainage rate of expanding bubbles under force-free conditions.  
Finally, in the discussion, we relax the assumption of negligible shear and show that even a small but non-negligible shear induced by fluid motion within the bubble introduces a singular term in the lubrication force.\end{abstract}
%In both cases, we discuss the assumption of non-deformable interfaces, low Reynolds number, and shear-free conditions.

%%%%%%%%%%%%%%%%%%%%%%%%%%%%%%%%%%%%%%%%%%%%%%%%%%%%%%%%%%%%%%%%%%%%%%%%%%%%%%%%%%%%%%%%%%%%%%%%%%%%%%%%%%%%%%%%%%%%%%%%%%%%%%%%%%%%%%%%%%%%%%%

\section{Introduction}
\label{sec:intro}

The coalescence of bubbles is ubiquitous in nature and industry. For instance, the coalescence of bubbles at a free surface significantly affects the production of sea aerosol \citep{deike2022}. In the industry, the size of bubbles profoundly impacts the efficiency of flotation processes \citep{nguyen2003}. These flotation processes are particularly interesting for recovering microparticles like microplastics or fine particles \citep{swart2022}. However, one major limitation of this technique is the requirement for generating small bubbles to capture the smallest particles, as emphasized by \citet{yoon1989}. 
Dissolved gas flotation is one existing technology used to generate micron-sized bubbles. In this process, the pressure of a liquid containing dissolved air is reduced, thereby releasing micron-sized bubbles that grow in size as they translate. The dynamics of these bubbles with time-dependent radii, especially when they are nearby and about to coalesce, strongly influence the process efficiency, which motivates the present study. The current investigation also holds relevance in the investigation of coalescing bubbles within magma or the coalescence phenomenon observed in water electrolysis processes. %for the examination of coalescing bubbles within magma, or the process of bubble coalescence in water electrolysis.%The present study may also be of some interest in the studying of coalescisincg bubbles in magma, or the coalescence of bubbles in water electrolysis.





The hydrodynamic resistance between two spherical drops or bubbles in the Stokes regime has been studied by \citet{haber1973} using a bispherical coordinate solution of the Stokes equation. \citet{davis1989} used lubrication theory to obtain the leading order forces on normally translating drops when the film thickness is significantly smaller than the drop radius. They observed good agreement between their results and the bispherical solution for sufficiently thin films. They also highlight the effect of the viscosity ratio on the force and mobility of the interface. In particular, the interface can be considered fully mobile or shear-free only for very small viscosity ratios.
 
\citet{chesters1982} investigated the film drainage between two approaching bubbles, assuming the interfaces to be deformable and shear-free. Their study revealed two distinct regimes of film drainage based on whether viscosity or inertia dominates. Coalescence occurs within a finite time, even without considering van der Waals effects, and a dimple is observed in the latter regime. Instead, in viscous-dominated regimes, no dimple is observed. \citet{pigeonneau2011} performed simulations using boundary integral methods to study the rise of a deformable bubble toward a free surface in viscous-dominated flows. Their findings indicated a significant influence of deformation on the dynamics of film drainage compared to the results obtained from the bi-spherical solution. The deformation tended to delay the drainage of the film.


Translating bubbles with time-dependent radii has received relatively less attention in the literature. \citet{michelin2018} derived a solution using bi-spherical coordinates for two bubbles with time-dependent radii. Their analysis revealed that hydrodynamic effects are mostly negligible, except when the bubbles are in close proximity. In the nearly inviscid limit \citet{van2002} derived the force on translating bubbles with time-dependent radii near a plane wall. Through an extended Rayleigh-Plesset equation, they investigated the trajectory of a bubble with an oscillating radius.






Focusing on translating spherical bubbles with time-dependent radii and shear-free interfaces in the Stokes regime, the present study aims to disentangle the various contributions to the forces. For this purpose, we use lubrication theory to derive closed-form solutions to the forces. In addition, we validate our theory by comparing it with Stokes flow theory. We then apply our results to two canonical configurations from existing literature, specifically examining the scenarios of a bubble approaching a free interface and the interaction of two growing bubbles under force-free conditions. The problem and current assumptions are introduced in section \ref{sec:problem}. %The viscous stress and pressure distribution within the liquid film are derived in section \ref{sec:stress}. 
In section \ref{sec:lubrication} we consider the asymptotic limit of a thin film.   
Then, the forces on the bubbles are derived and compared to Stokes flow solutions in section \ref{sec:force}. Predictions from the theory in two different physical configurations are discussed in section \ref{sec:examples}. The outcomes and assumptions of our investigation are discussed in section \ref{sec:conc}. Specifically, we relax the assumption of shear-free interfaces by considering small but non-negligible shear at the bubbles interfaces.


\section{Formulation of the problem}
\label{sec:problem}

\subsection{Description of the problem}

% Figure environment removed

We consider the thin film flow between two translating spherical bubbles moving along their line of centers (Figure \ref{fig:scheme}). The radius of the bubble $i$ ($i=1,2$) is denoted $a_i(t)$, and the rate of change of the bubble radius is noted $\dot{a}_i(t)$. 
The bubbles center move towards each other with velocity $\bm{V}_i(t)$ in a quiescent fluid of density $\rho$ and dynamic viscosity $\mu$. The fluid density and viscosity within the bubbles are denoted respectively $\zeta \rho$ and $\lambda \mu$ where $\zeta$ and $\lambda$ are the density and viscosity ratios, respectively.
The film, located between the two bubbles, is modeled using axisymmetric cylindrical polar coordinates $(r,z)$. The origin of the coordinate system $(O)$ is located at the center of the film. The film thickness is denoted by $h(r,t)$, while the shortest distance between the two bubbles is given by $d(t)$. From those definitions we have $V_2-V_1 = \dot{d} + \dot{a}_1 + \dot{a}_2$ where $V_1 = \bm{V}_1\cdot \bm{e}_z$ and $V_2 = \bm{V}_2\cdot \bm{e}_z$. As the velocities of the bubbles may vary with time, the frame of reference is not inertial in the most general configuration. The position of the two bubble surfaces is given by $z = h_1(r,t)$ and $z = h_2(r,t)$ (Figure \ref{fig:scheme}). 


\subsection{Governing equations}


We assume that the flow within the film satisfies the Stokes equations. The Stokes flow solution provides a good approximation to the flow field near the bubbles when the Reynolds number, denoted as $Re=\rho \dot{d} \bar{a}/\mu$ is much less than unity \citep{kim1991}. Specifically, when $Re \ll 1$, the advection term in the Navier-Stokes equation, scaling as $\rho \dot{d}^2/\bar{a}$, is significantly smaller compared to the viscous term, which scales as $\mu \dot{d}/\bar{a}^2$. Here, $\bar{a}$ represents the reduced radius defined as $\bar{a}=a_1a_2/(a_1+a_2)$. The fluid motion is unsteady as the separating distance between the bubbles continuously changes. Nevertheless, if we consider the time scale to be on the order of $\bar{a}/\dot{d}$, the acceleration term becomes negligible compared to the viscous terms under the condition that $Re \ll 1$. Under this condition, we may disregard the acceleration of the film center relative to a stationary frame of reference.


The axisymmetric Stokes equations written in cylindrical coordinates read
\begin{align}
\frac{1}{r}\frac{\partial}{\partial r}(r u)+\frac{\partial w}{\partial z} &= 0 \label{eq:cdm}\\
0 &= -\frac{\partial p}{\partial r} + \mu \left(\frac{\partial ^2 u}{\partial r^2} + \frac{\partial ^2 u}{\partial z^2} + \frac{1}{r}\frac{\partial u}{\partial r}-\frac{u}{r^2}\right) \label{eq:qdmr}\\
0 &= -\frac{\partial p}{\partial z} + \mu \left(\frac{\partial  ^2 w}{\partial r^2} + \frac{\partial ^2 w}{\partial z^2} + \frac{1}{r}\frac{\partial w}{\partial r}\right)  \label{eq:qdmz}
\end{align}
where $u$ is the radial velocity, $w$ the axial velocity and $p$ the fluid pressure.
Equations \eqref{eq:cdm}-\eqref{eq:qdmz} are also valid within the bubbles (replacing $\rho$ by $\zeta\rho$ and $\mu$ by $\lambda \mu$ and adding a subscript $i$ to all the fields) provided that the condition $Re \ll \lambda$ is satisfied.


Before specifying the boundary conditions on the interface we define the tangential and normal unit vectors $\bm{n}_i$ and $\bm{t}_i$ as 

\begin{align}
%\bm{n} = \frac{1}{(1+(\partial h /\partial r )^2)^{1/2}}\left(\frac{1}{2}\frac{\partial h}{\partial r} \bm{e}_r-\bm{e}_z\right), \quad
\bm{n}_i = \left[1+\left(\frac{\partial h_i}{\partial r}\right)^2\right]^{-1/2}\left(\frac{\partial h_i}{\partial r} \bm{e}_r-\bm{e}_z\right), \quad
%\bm{t} = \frac{\bm{e}_r + \partial h /\partial r \bm{e}_r}{(1+(\partial h /\partial r )^2)^{1/2}}
\bm{t}_i = \left[1+\left(\frac{\partial h_i}{\partial r}\right)^2\right]^{-1/2}\left(\bm{e}_r+\frac{\partial h_i}{\partial r}\bm{e}_z\right),
\label{eq:nt}
\end{align}
Note that $\bm{n}_2$ is pointing outward to the bubble surface while $\bm{n}_1$ is pointing inward. At the surface of bubble $i$ the impermeability condition reads

\begin{align}
\bm{u}\cdot\bm{n}_i = (-1)^i \dot{a}_i+\bm{V}_i\cdot\bm{n}_i \quad \text{on $z= h_i$}.
\end{align}



The tangential stress boundary condition may be expressed as $\bm{e}:(\bm{t}_i\otimes\bm{n}_i) = \bm{e}_i:(\bm{t}_i\otimes\bm{n}_i)$ on $z= h_i$ where $\bm{e}$ is the strain-rate tensor within the liquid and $\bm{e}_i$ the strain-rate tensor within bubble $i$. Except in the final section, the specific details regarding the internal flow within the bubble will not be taken into consideration.
Hence, for simplicity, we denote $\bm{e}_i:(\bm{t}_i\otimes\bm{n}_i) = f_i$, where $f_i(r,t)$ is the tangential stress exerted by the gas in the bubble $i$ on the interface.
Using \eqref{eq:nt} the tangential stress boundary condition reads
\begin{align}
&\mu\left[1+\left(\frac{\partial h_i}{\partial r}\right)^2\right]^{-1}\left[2\frac{\partial h_i }{\partial r } \left(\frac{\partial u}{\partial r}-\frac{\partial w}{\partial z} \right) + \left(\left(\frac{\partial h _i }{\partial r }\right) ^2 -1 \right)\left(\frac{\partial u}{\partial z}+\frac{\partial w}{\partial r} \right) \right] \nonumber \\
&= f_i \quad \text{on $z= h_i$}.%\text{on $z= H \pm h/2$} 
\label{eq:shear}
\end{align}

The assumption of spherical shapes is fulfilled as long as surface tension effects are much larger than dynamic normal stresses. This may be rationalized using the normal stress boundary conditions: $-p+2\mu\bm{e}:(\bm{n}_i\otimes\bm{n}_i) = -p_i+2\mu\bm{e}_i:(\bm{n}_i\otimes\bm{n}_i) + \gamma \kappa _i$ on $z= h_i$ where $\gamma$ is the surface tension and $\kappa _i$ the mean curvature of interface $i$. This boundary condition will be used to demonstrate \textit{a posteriori} the conditions under which the spherical assumption remains valid. As before, we define a function $g_i$ equals to the dynamic normal stress exerted by the bubble on the interface: $g_i (r,t) = -p_i+2\mu\bm{e}_i:(\bm{n}_i\otimes\bm{n}_i)$. Therefore,
\begin{align}
&-p + 2\mu \left[1+\left(\frac{\partial h_i}{\partial r}\right)^2\right]^{-1}\left(\left(\frac{\partial h _i }{\partial r }\right) ^2\frac{\partial u}{\partial r} - \frac{\partial h _i }{\partial r }\left(\frac{\partial u}{\partial z}+\frac{\partial w}{\partial r} \right)+\frac{\partial w}{\partial z}\right) \nonumber \\
&= g_i + \gamma \kappa _i \quad \text{on $z= h_i$} \label{eq:normal}.
\end{align}
The force on the $i$-th bubble reads

\begin{equation}
\bm{F}_i = (-1)^i\int _{S_i} (-p \bm{I} +2\mu \bm{e})\cdot \bm{n}_i dS
\end{equation}
where $S_i$ is the surface of bubble $i$ and $\bm{I}$ the identity matrix.
 Since the motion is axisymmetric, the only nonzero components of the force lies in the $z$ direction. From equation \eqref{eq:nt} we obtain for the force on the $i$-th bubble %second bubble %on the $2$th bubble
%(in dimensional form)
\begin{equation}
F_i =  2\pi (-1)^i \left[\int_0^{a_i}\left(p-2\mu \frac{\partial w}{\partial z}\right)rdr + \mu \int_0^{a_i}\left(\frac{\partial u}{\partial z}+ \frac{\partial w}{\partial r}\right) \frac{\partial h_i}{\partial r}rdr \right] \label{eq:force}
\end{equation}


\section{Lubrication theory}
\label{sec:lubrication}

In the subsequent section, we shall derive the lubrication equations. Although similarities exist between our derivation and prior investigations in the literature \citep{chesters1982,howell1996,savva2009}, a comprehensive derivation for the specific problem under consideration has not been previously done. Moreover, the previous derivations in the literature presuppose shear-free boundaries at the bubble surface, while in the present approach, the shear-free assumption will be justified \textit{a posteriori}. 
The assumption of shear-free interface is of primary importance since the set of lubrication equations between shear-free interfaces differs significantly from the standard lubrication equations \citep{davis1989,leal2007,michelin2019}. In the latter case, the flow is resisted either by the no-slip boundary condition at one of the interfaces \citep{leal2007,michelin2019} or by the flow inside one of the drops \citep{davis1989}. On the other hand, when both interfaces are free, the flow is resisted by the extensional effective viscosity \citep{howell1996}.

We consider the thin film limit, in which $d/\bar{a} = \epsilon ^2\ll 1$. Close to the axis of symmetry, i.e., for $r \ll \bar{a}$, we have $h_1 \sim - 1/2(d+r^2/a_1)$ and $h_2 \sim 1/2(d+r^2/a_2)$. Since $h=h_2-h_1$ we obtain
\begin{equation}
h(r,t) = d(t)+\frac{r^2}{2\bar{a}(t)} + \mathcal{O} \left(\frac{r^4}{a^4}\right)
\label{eq:h}
\end{equation}
%where $a=a_1a_2/(a_1+a_2)$ is the reduced radius \citep{davis1989}. 
Therefore in the film region, $r$ scales as $\sqrt{\bar{a}d}$ \citep{davis1989}. 
The appropriate normalizations for the lengths within the gap are $r=\sqrt{\bar{a}d} r^*$, $z=dz^*$ and $h = d h^*$, and the starred quantities are dimensionless. Equation \eqref{eq:h} written in non-dimensional form reads

\begin{equation}
h^*(r^*) = 1+\frac{r^{*2}}{2} + \mathcal{O} \left(\epsilon ^2\right)
\label{eq:h_adim}
\end{equation}

%Denoting $\dot{d}_{0} = \dot{d}(0)$ the initial axial velocity 
We non-dimensionalize the axial velocity as follows $w=\dot{d}w^*$. We get $u=u^*\dot{d}/\epsilon $ from the incompressibility condition. From the momentum equation in the $z$-direction, where viscous and pressure forces are in
balance, we obtain $p = p^*\mu \dot{d}/d$. Within the bubble the radial and axial length scales are $\sqrt{\bar{a}d}$ \citep{davis1989}. Hence, $f_i = f_i^*\lambda \mu \dot{d}/d $ and $g _i= g_i^*\lambda \mu \dot{d}/d $. Since the bubbles are spherical, we normalize the mean curvature by the inverse of $\bar{a}$ such that $\kappa = 1/\bar{a} \kappa^*$. We normalize the velocities as $\dot{a}_i = \dot{a}_i^*\dot{d}$ and $V_i = V_i^*\dot{d}$. %\color{blue}le scaling pr la pression nest a priori pas si simple. Si l'on suppose que levoulement dans la gouttes est visqueux, on a priori $p = \mu U/l$ OK donne la meme chose \color{black}.
 After this non-dimensionalization, the Navier-Stokes equations \eqref{eq:cdm} - \eqref{eq:qdmz} take the form

%The typical time scale may be written $t=t^* d/\dot{d}$

\begin{align}
\frac{1}{r^*}\frac{\partial}{\partial r^*}(r^* u^*)+\frac{\partial w^*}{\partial z^*} &= 0, \label{eq:cdm_adim}\\
0 &= -\epsilon^2\frac{\partial p^*}{\partial r^*} +\epsilon^2\frac{\partial ^2 u^*}{\partial r^{*2}} + \frac{\partial ^2 u^*}{\partial z^{*2}} + \epsilon^2\frac{1}{r^*}\frac{\partial u^*}{\partial r^*}-\epsilon^2\frac{u^*}{r^{*2}}, \label{eq:qdmr_adim}\\
0 &= -\frac{\partial p^*}{\partial z^*} +\epsilon^2\frac{\partial ^2 w^*}{\partial r^{*2}} + \frac{\partial ^2 w^*}{\partial z^{*2}} +\epsilon^2 \frac{1}{r^*}\frac{\partial w^*}{\partial r^*}, \label{eq:qdmz_adim}
\end{align}
The impermeability condition reads

\begin{align}
\epsilon^2\frac{\partial h_i^*}{\partial r^*}u^* - w^* = (-1)^i \dot{a}_i^* \left(1+\epsilon^2\left(\frac{\partial h_i^*}{\partial r^*}\right)^2\right)-\bm{V}_i^* \quad \text{on $z^*= h_i^*$}.
\end{align}
%\color{black}
while the dynamic boundary conditions give
\begin{align}
&\left[1+\epsilon^2\left(\frac{\partial h_i^*}{\partial r^*}\right)^2\right]^{-1}\left[2\epsilon^2\frac{\partial h_i^* }{\partial r ^*}   \left(\frac{\partial u^*}{\partial r^*}-\frac{\partial w^*}{\partial z^*} \right) + \left(\epsilon^2\left(\frac{\partial h _i ^*}{\partial r ^*}\right) ^2 -1 \right)\left(\frac{\partial u^*}{\partial z^*}+\epsilon^2\frac{\partial w^*}{\partial r^*} \right)\right] \nonumber\\
&= \epsilon \lambda f_i^* \quad \text{on $z^*= h_i^*$} \label{eq:shear_adim},\\
&-p^* + 2\left[1+\epsilon^2\left(\frac{\partial h_i^*}{\partial r^*}\right)^2\right]^{-1}\left(\epsilon^2\left(\frac{\partial h _i^* }{\partial r ^*}\right) ^2\frac{\partial u^*}{\partial r^*} - \frac{\partial h _i ^*}{\partial r^* }\left(\frac{\partial u^*}{\partial z^*}+\epsilon^2\frac{\partial w^*}{\partial r^*} \right)+\frac{\partial w^*}{\partial z^*}\right)  \nonumber \\
&= \lambda g_i^* + \frac{\epsilon ^2}{Ca} \kappa _i^* \quad \text{on $z= h_i^*$},
\label{eq:normal_adim}
\end{align}
where $Ca = \mu \dot{d}/\gamma$ is the capillary number. As it can be seen from equation \eqref{eq:normal_adim}, the deformation relative to the initial spherical shapes remains small provided that $Ca \ll \epsilon ^2$. 

%Discuter de al validité de cela dans les cas applicatifs
%\color{black} 

We seek solutions of this set of equations in the form of expansions in powers of the small parameter $\epsilon ^2$ \citep{howell1996}: $u^* = u_0^* + \epsilon^2 u_2^* + ...$, $w^* = w_0^* + \epsilon^2 w_2^* + ...$,  $p^* = p_0^* + \epsilon^2 p_2^* + ...$. %\footnote{We have omitted to write $h$ and $f$ as expansions of the small parameters for the sake of brevity. This does not affect the leading-order solution, as only the first term of these expansions is significant.}. 
Inserting this expansion in Equations \eqref{eq:cdm_adim} - \eqref{eq:qdmz_adim}, we obtain to the lowest order

\begin{align}
\frac{1}{r^*}\frac{\partial}{\partial r^*}(r^* u_0^*)+\frac{\partial w_0^*}{\partial z^*} &= 0, \label{eq:cdm_adim0}\\
0 &= \frac{\partial ^2 u_0^*}{\partial z^{*2}}, \label{eq:qdmr_adim0}\\
0 &= -\frac{\partial p_0^*}{\partial z^*} + \frac{\partial ^2 w_0^*}{\partial z^{*2}}. \label{eq:qdmz_adim0}
\end{align}
%\color{blue}
The impermeability condition at the bubble surface is given by
\begin{align}
- w_0^* = (-1)^i \dot{a}_i^* -\bm{V}_i^* \quad \text{on $z^*= h_i^*$}. \label{eq:kinematic0}
\end{align}
%\color{black}
%and
%\begin{equation}
%\frac{\partial h_i}{\partial t} + u_0\frac{\partial h_i}{\partial r} = w_0 \quad \text{on $z= h_i$} \label{eq:kin_adim0}.
%\end{equation}
Assuming that the viscosity ratio is small, \textit{i.e.} $\lambda \sim \epsilon$ the tangential boundary condition \eqref{eq:shear_adim} gives 

\begin{equation}
\frac{\partial u _0^*}{\partial z^*} = 0 \quad \text{on $z^*= h_i^*$} \label{eq:shear_adim0}.
\end{equation}
Making use of Equations \eqref{eq:qdmr_adim0} and \eqref{eq:shear_adim0}, we obtain $u _0^*\equiv u_0^*(r^*)$. Hence, the radial velocity is uniform along the film. As a consequence, equation \eqref{eq:cdm_adim0} and $w_0^*(r^*,z^*=0) =0$ yields
\begin{equation}
 w_0^*(r^*,z^*) = - \frac{1}{r^*}\frac{\partial}{\partial r^*}(r^* u_0^*)z^*.
\label{eq:w0}
\end{equation}
Substitution into \eqref{eq:qdmz_adim0} yields $p _0^* \equiv  p_0^*(r^*)$. We now insert the $\mathcal{O}(\epsilon ^2)$ terms in the expansions in equation \eqref{eq:qdmr_adim} to obtain an equation for the radial velocity %from equation \eqref{eq:qdmr_adim} as
\begin{equation}
0 = -\frac{\partial p_0^*}{\partial r^*} + \frac{\partial^2 u_0^*}{\partial r^{*2}} + \frac{\partial^2 u_2^*}{\partial z^{*2}} + \frac{1}{r^*}\frac{\partial u_0^*}{\partial r^*}-\frac{u_0^*}{r^{*2}}. \label{eq:qdmr2}
\end{equation}
% equation for the leading-order radial velocity may be obtained from equation \eqref{eq:qdmr_adim} as
%A l'ordre 2
%\color{red}
%ajouter des details sur la derivation nottament sur w0
%\color{black}
Similarly, by inserting the $\mathcal{O}(\epsilon ^2)$ terms in the tangential stress boundary conditions \eqref{eq:shear_adim} we obtain
\begin{equation}
%2\frac{\partial h_i }{\partial r }  \left(\frac{\partial u_0}{\partial r}-\frac{\partial w_0}{\partial z} \right) + \left(\epsilon^2\left(\frac{\partial h _i }{\partial r }\right) ^2 -1 \right)\left(\frac{\partial u_2}{\partial z}+\frac{\partial w_0}{\partial r} \right) = \frac{\lambda}{\epsilon}f_i(r,t) \quad \text{on $z=\pm h_i$} .
2\frac{\partial h_i^* }{\partial r^* }  \left(\frac{\partial u_0^*}{\partial r^*}-\frac{\partial w_0^*}{\partial z^*} \right) -\left(\frac{\partial u_2^*}{\partial z^*}+\frac{\partial w_0^*}{\partial r^*} \right) = \frac{\lambda}{\epsilon}f_i^* \quad \text{on $z^*= h_i^*$} \label{eq:shear_adim2}.
\end{equation}
Upon integrating equation \eqref{eq:qdmr2} from $h_1^*$ to $h_2^*$ and making use of equations \eqref{eq:w0} and \eqref{eq:shear_adim2} we find 


\begin{equation}
0 = - \frac{\partial p_0^*}{\partial r^*} + 2\left[\frac{\partial}{\partial r^*}\left(\frac{1}{r^*}\frac{\partial}{\partial r^*}(r^* u_0^*)\right) + \frac{1}{h^*}\frac{\partial h^*}{\partial r ^*}  \left(2\frac{\partial u_0^*}{\partial r^*}+\frac{u_0^*}{r^*} \right)\right] - \frac{\lambda}{\epsilon h^*}(f_2^*(r,t)  - f_1^*(r,t)).
\label{eq:qdm_shear}
\end{equation}

Then, under the assumption of negligible shear stress at the interfaces ($\lambda \ll \epsilon$), the last equation may be written %in dimensional form as 
\begin{equation}
0  = - \frac{\partial p _0^*}{\partial r^*} + 2 \left[\frac{\partial}{\partial r^*}\left(\frac{1}{r^*}\frac{\partial}{\partial r^*}(r^* u_0^*)\right) + \frac{1}{h^*}\frac{\partial h^*}{\partial r^* }  \left(2\frac{\partial u_0^*}{\partial r^*}+\frac{u_0^*}{r^*} \right)\right] \label{eq:qdm_final},
%\rho \left(\frac{\partial u}{\partial t} + u\frac{\partial u}{\partial r} \right)  = - \frac{\partial p}{\partial r} + 2 \mu \left[\frac{\partial}{\partial r}\left(\frac{1}{r}\frac{\partial}{\partial r}(r u)\right) + \frac{1}{h}\frac{\partial h}{\partial r }  \left(2\frac{\partial u}{\partial r}+\frac{u}{r} \right)\right] \label{eq:qdm_final}.
\end{equation}
%where $h$ is given by \eqref{eq:h_adim}.
Equations \eqref{eq:h_adim}, \eqref{eq:cdm_adim0}, \eqref{eq:kinematic0} and \eqref{eq:qdm_final} constitute a system of lubrication equations that describe
the flow within the thin gap located between the two spherical bubbles. 
The main limitations of those equations are their limited range of validity: $Re \ll 1$, $Ca \ll \epsilon ^2$, and $\lambda \ll \epsilon$. We will drop the subscript 0 for convenience in the next sections.



\section{Forces on the bubble}
\label{sec:force}


Each of the two integrals on the right-hand side of \eqref{eq:force} may be decomposed over an inner region where lubrication assumption is valid and an outer region on the scale of the bubble \citep{cox1967}. More precisely, we may define $R_\infty$ such that $r = R_\infty$ may be considered lying between the inner and outer regions. Hence, $R_\infty$ should be in the range $\sqrt{\bar{a}d} \ll R_\infty \ll \bar{a}$ \citep{davis1989}. Because of the somewhat arbitrary manner in which $R_\infty$ is defined, the final expression of the force should not depend on this parameter. It will naturally disappear from the total force when adding the outer solution to the inner lubrication region \citep{cox1967,oneill1967}. 
In the present configuration, the outer solution is known analytically for $Re=0$ \citep{haber1973}. Nevertheless, a detailed knowledge of the outer solution is not required by restricting our attention to the singular terms of the expansion, which may be obtained from the inner solution alone. 

%Only the behaviour of the outer solution in the inner limit is required. Using the "matching principle" \citep{van1964}, this information may be obtained solely from the inner solution. This technique was pioneered by \citet{cox1967} when investigating the inertial correction to the force for a sphere approaching a wall. S
%This methodology is commonly employed in literature dedicated to lubrication equations (see for instance Kim and Karrila (1991) and Michelin et al. (2019)) 
%\color{red} il y a un papier de Goldman et Cox ou il trouve une solution logartihtmiqueent singuliere mais que avec de la lurbif. Non il ajoute bien la solution exterieur\color{black}

 

Using the normalizations defined in section \ref{sec:lubrication} and expressing the force on the $i$-th bubble as $F_i = \mu \dot{d} \bar{a}F_i^*$ and $R_\infty= \bar{a} R_\infty^*$ the inner force contribution reads  %inner force contribution as $F^i = \mu \epsilon U aF^{i*}$ and $R_\infty$ as $R_\infty= a R_\infty^*$ the inner force contribution reads (dropping the stars)

%Utilizing the normalizations outlined in Section \ref{sec:moderate} and expressing the inner force contribution as $F^i = \mu \epsilon U aF^{i*}$ and $R_\infty = a R_\infty^*$, we arrive at the simplified form (sans the asterisks).

\begin{equation}
F_i^{i*} = 2\pi (-1)^i \left[\int_0^{R_\infty^*/\epsilon}\left(p^*-2\frac{\partial w^*}{\partial z^*}\right)r^*dr^* + \int_0^{R_\infty^*/\epsilon}\left(\frac{\partial u^*}{\partial z^*}+ \epsilon ^2\frac{\partial w^*}{\partial r^*}\right) \frac{\partial h_i^*}{\partial r^*}r^*dr^*. \right] \label{eq:forcen}
\end{equation} 
Since $F_1^{i*} = -F_2^{i*}$, we denote $F^*$ the inner force on the first bubble for convenience. %dropping the index $i$ for convenience
%Inserting the expansions for $u$, $w$ and $p$ in powers of the small parameter and noting that to the lowest order $u$ is independent of $z$ we may write 
By substituting the expansions for $u^*$, $w^*$, and $p^*$ in terms of the small parameter in equation \eqref{eq:forcen} and recognizing that, at the lowest order, $u^*$ is independent of $z^*$, we may write %(dropping the subscript 0)
%Since $u$ is indepent of z and to the order of approximation considered here

\begin{equation}
F^* = -2\pi \int_0^{R_\infty^*/\epsilon}\left(p^*-2\frac{\partial w^*}{\partial z^*}\right)r^*dr^* \label{eq:forcen2}
\end{equation} 
%the "inner" contribution to
Hence, to determine the force, it is necessary to compute both the velocity profile and the pressure distribution within the film.
One may also note that in contrast to \citet{davis1989} and \citet{leal2007}, the pressure and viscous stress in our analysis have comparable magnitudes in Equation \eqref{eq:forcen2} due to the assumption of shear-free interfaces \citep{savva2009}.


\subsection{Viscous stress and pressure distribution}
\label{sec:stress}

To obtain the radial velocity profile we integrate \eqref{eq:cdm_adim0} with
respect to $z^*$

\begin{equation}
\int_{h_1^*}^{h_2^*}\frac{1}{r^*}\frac{\partial}{\partial r^*}(r^* u^*)dz^*+w^*(h_2^*)-w^*(h_1^*) = 0.
\end{equation}
If we apply the boundary conditions for $w^*$ from \eqref{eq:kinematic0} we obtain the radial velocity

\begin{equation}
u^*(r^*)= -\frac{1}{2}\frac{r^*}{h^*(r^*)},
\label{eq:u}
\end{equation}
where $h^*$ is given by \eqref{eq:h_adim}.
Substitution into \eqref{eq:cdm_adim0} thus yields 
\begin{equation}
\frac{\partial w^*}{\partial z^*}(r^*) =  \frac{1}{h^{*2}(r^*)}.
%\frac{\partial w}{\partial z} = \frac{4\dot{d}a^2d}{(2da+r^2)^2}
\label{eq:viscous_stress}
\end{equation}





Integration of Equation \eqref{eq:qdm_final} in $r^*$ (see appendix \ref{app:pressure} for the details), and making use of Equation \eqref{eq:u} the pressure profile reads %Using mass conservation equation (Eq. 1), the shear stress can be easily evaluated. 


\begin{equation}
p^*(r^*)  = - \frac{1}{h^*(r^*)} + p_\infty ^*
\label{eq:pressure}
\end{equation}
where $p_\infty^*$ is a $\mathcal{O}(1)$ term related to the pressure outside the film. Without loss of generality, we assume that $p_\infty ^* = 0$. Both the pressure \eqref{eq:pressure} and the normal viscous stress  \eqref{eq:viscous_stress} combine to form the overall stress within the thin film. These quantities are singular in the limit of small thickness as they scale proportionally to $1/h^*$ or $1/h^{*2}$, respectively. Figure \ref{fig:pressure} illustrates their behavior. Both quantities approach zero as the radial distance $r$ approaches infinity. 

% Figure environment removed




\subsection{Lubrication forces}
The innner contribution to the force on the bubble can be obtained by inserting Equations \eqref{eq:viscous_stress} and \eqref{eq:pressure} in \eqref{eq:forcen2}. The calculation details are summarized in appendix \ref{app:force}. This yields,

%In particular since $\int _0 ^{R_\infty} r/h dr \sim a/2\log\left(R_\infty ^2/(ah_0)\right)$ and $\int _0 ^{R_\infty} r^3/h^2 dr \sim a^2/2\log\left(R_\infty ^2/(ah_0)\right)$ we obtain

%\begin{equation}
%F = - \pi \mu \dot{h}_0 a \log\left(\frac{R_\infty ^2}{ah_0}\right) + \frac{\pi}{8} \rho \dot{h}_0^2 a^2 \log\left(\frac{R_\infty ^2}{ah_0}\right) + \mathcal{O}(1).
%\label{eq:forcef}
%\end{equation}
%Note that the viscous stress does not contribute to the force since the integral of the viscous stress vanishes to leading-order. Equation \ref{eq:force} shows that the force is singular when the two drops approach each other but this takes the form of a weak logarithm singularity. However, formula \ref{eq:force} is not closed since $R_\infty$ is a priori unknown. The exact value of $R_\infty$ may be obtained by matching the thin film solution with the outer solution flow solution at least in the Stokes regime \citep{oneill1967}. Here, we make use of the asymptotic formula proposed by \citep{kim1991} based on the bi-spherical coordinate solution of \citet{haber1973} to obtain the value of $R_\infty$. Their solution reads 
%Moreover, the unsteady term proportionnal to $(\dot{V} + \ddot{a})$ gives contribution to the force which are of order one. This is in line with the fact that the added mass do not diverge as the distance between the bubble tends to zero \citep{miloh1977}.
\begin{equation}
F^* = -4\pi\log\epsilon + \mathcal{O}(1).
\label{eq:forcea}
\end{equation}
Note that the viscous stress contributes only to the $\mathcal{O}(1)$ term. %force since the integral of the viscous stress vanishes to leading-order.
The preceding formula may be rewritten in dimensional form as
\begin{equation}
%F_i = (-1)^i2\pi\mu\dot{d}a\log\left(\frac{d}{a}\right)
%F = -2\pi\mu a \dot{d} \log\left(\frac{d}{a}\right)
F = -4\pi\mu \bar{a} \dot{d} \log \epsilon
\label{eq:forced}
\end{equation}
%The force on the second bubble is the opposite of \eqref{eq:forced}. 
where we recall that $\dot{d} = V_2-V_1 - (\dot{a}_1 +\dot{a}_2)$.
The formula \eqref{eq:forced} has the advantage of being applicable for drops with different radii as long as $Re \ll 1$, $Ca \ll \epsilon ^2$ and $\lambda \ll \epsilon$. However, since it is a weak logarithm singularity, its accuracy is low except for very thin film, potentially violating the conditions $Ca \ll \epsilon ^2$ and $\lambda \ll \epsilon$ in practical conditions. Hence, there is a need to derive the $\mathcal{O}(1)$ terms to enhance solution accuracy. These terms may be obtained by properly matching the inner solution with the outer solution \citep{oneill1967}. Instead, we use previous findings from the literature to compute them. Additionally, given that the $\mathcal{O}(1)$ term depends on the specific configuration under consideration, we focus on cases involving bubbles with identical radii either moving at identical velocities or experiencing a constant rate of change in radius. We also consider the case of a bubble translating or expanding/draining near a plane-free surface.  





\subsection{Determination of the $\mathcal{O}(1)$ contribution to the force}
\label{sec:viscf}

%In the case of two identical translating bubbles $a = a_i/2$. 
We first consider the case of two identical bubbles translating with opposite and equal velocity $V$ such that $\dot{d} = -2V$. %Denoting $\bar{a}$ the bubble radius such that $\bar{a}=a_1=a_2$ %their radius, the relative velocity between the two bubbles $V = V_2-V_1$ 
Formula \eqref{eq:forced} may be expressed as $F = 8\pi\mu \bar{a} V \log \epsilon$. \citet{kim1991} derived an asymptotic formula based on the bi-spherical coordinate solution proposed by \citet{haber1973}. Their solution may be expressed as 

\begin{equation}
F = 8\pi\mu \bar{a} V \left(\log \epsilon + A\right)
\label{eq:kim_A}
\end{equation}
where $A = -\gamma -3/2\log 2$ and $\gamma$ is the Euler's constant. We recall that $\bar{a}$ denotes the reduced radius, which is half the radius of the bubbles in this case, and $\epsilon$ is defined as $\epsilon = (d/\bar{a})^{1/2}$. Figure \ref{fig:kim} displays the variation of force with respect to film thickness. %Figure \ref{fig:kim}
%displays the evolution of the force as a function of the film thickness. 
The formula \eqref{eq:kim_A} (depicted with dashed line) is in good agreement with the bi-spherical coordinate solution (depicted with continuous line) up to $\epsilon \approx 1$. In contrast, the solution derived solely from the inner lubrication region (dotted line) proves inaccurate even for small $\epsilon$.
%Figure \ref{fig:kim} illustrates the variation of force with respect to film thickness. Equation \eqref{eq:kim_A} aligns well with the bi-spherical coordinate solution for film thicknesses up to $\epsilon \approx 1$. Conversely, the solution obtained solely from the lubrication area proves inaccurate even for small $\epsilon$.

% Figure environment removed


%such that $a=a_1$, $a_2 \gg a_1$ and $\dot{d} = -V$. 

%We recall that $a$ is the reduced radius (half the radius of the bubbles in this case) and $\epsilon = (d/a)^{1/2}$. 
We now consider the case of a bubble translating toward a free surface with a velocity $V$. 
In this case, Equation \eqref{eq:forced} reads $F = 4\pi\mu \bar{a} V \log \epsilon$. Additionally, solution \eqref{eq:kim_A} may be readily extended to the case of a free surface by noting that the plane of symmetry separating the two bubbles is a shear-free boundary. By substituting $\bar{a}$ with $\bar{a}/2$ and $d$ with $2d$ in \eqref{eq:kim_A}, we obtain

\begin{equation}
F = 4\pi\mu \bar{a} V \left(\log \epsilon + A_\infty\right)
\label{eq:kim_planar}
\end{equation}
where $A_\infty = -\gamma -\log (2)/2$.
%\color{red} a finalsier tranquillement\color{black}









To the best of our knowledge, the contribution arising from the variation of the bubble radius in formula \eqref{eq:forcea} has not been derived so far. To verify the accuracy of this result, the bi-spherical coordinate solution proposed in \citet{michelin2018} can be used. However, it is important to note that \citet{michelin2018} did not explicitly express the force acting on the bubble, instead requiring the solution of a linear system comprising four equations. In the specific scenario of two identical bubbles, the system's complexity can be reduced, allowing for the derivation of an explicit form for the force, as demonstrated in Appendix \ref{app:visc}. %Figure \ref{fig:kim} displays the force variation with respect to film thickness. 
Denoting $\dot{a}$ the time derivative of the bubble radius, we have $\dot{d} = -2\dot{a}$. Formula \eqref{eq:forced} can be expressed as $F = 8\pi\mu \bar{a} \dot{a} \log \epsilon$. Consequently, similar to the case of purely translating bubbles, one can seek the force expression as

%Thus, as in the case of purely translating bubble one may seek the force expression as 
%As in the case of purely translating bubble one may seek the $\mathcal{O}(1)$ term
\begin{equation}
F = 8\pi\mu \bar{a} \dot{a} \left(\log \epsilon + B\right)
\label{eq:michelin}
\end{equation}
where $B$ represents a term of order $\mathcal{O}(1)$, which may be fitted from the bi-spherical coordinate solution. In our specific scenario, the best-fit yields $B \approx 0$. Indeed, as depicted in Figure \ref{fig:force_visc}, there is an excellent agreement between the bi-spherical coordinate solution (depicted with a continuous line) and the predictions of lubrication theory (depicted with a dotted line).
%where $B$ is $\mathcal{O}(1)$ term which may be fitted from the bi-spherical coordinate solution. It appears that in this specific scenario the best fit we obtain is $B\approx 0$. Indeed as shown in figure Figure \ref{fig:force_visc} a very good match between the bi-spherical coordinate solution and the prediction of the lubrication theory.
% Figure environment removed
Formula \eqref{eq:michelin} may be readily extended to the case of a bubble with a time-dependent radii near a free surface. In this case, the force may be expressed as

\begin{equation}
F = 4\pi\mu \bar{a} \dot{a} \left(\log \epsilon + B_\infty\right)
\label{eq:michelin2}
\end{equation}
where $B_\infty \approx \log 2 $.








\section{Some applications to bubble dynamics}
\label{sec:examples}
In this section, we explore how our findings can be used to determine the dynamics of translating bubbles with time-dependent radii. We first consider the case of a bubble translating toward a free surface under the action of a constant body force. Subsequently, we analyze the dynamics of two expanding bubbles in the absence of external forces.

%Then, we consider the dynamics of two growing bubbles without external forces.

%under the action of a constant body force.



%\subsection{Small translating bubbles under the action of a constant body force}
\subsection{Bubble translating toward a free surface under the action of gravity}
%\color{blue}
%Reformuler pour contextualiser mieux le probleme dans le cadre constant body force.
%\color{black}
%\subsection{Film drainage time scale of small translating bubbles in viscous dominated flows}
\label{sec:vaka}
%In this section, we explore the extent to which our findings can be used to determine the dynamics of translating bubbles under the action of a constant body force. In practice, we consider the case of a bubble translating toward a free surface under the action of gravity.


% Figure environment removed

We consider a bubble of radius $a$ translating toward a free surface under the action of gravity denoted by $g$ (Figure \ref{fig:scheme_free}). Our primary focus is determining the film drainage time, which is the duration required for the bubble to drain the film between its upper surface and the free surface.
A significant limitation of the current analysis, which restricts its practical application, is the assumption of negligible deformations of the free surface and of the bubble shape relative to a spherical geometry, which is expressed as $Ca \ll \epsilon ^2$. 
This assumption necessitates extremely small bubbles and high surface tension, as demonstrated later in this section.


We are unaware of any experimental results measuring the film drainage time scale for small bubbles in viscous flow except the recent experiments of \citet{vakarelski2018}. The absence of prior research can be attributed to two reasons. First, it is very difficult experimentally to avoid the contamination of the interface with impurities \citep{vakarelski2018}. Second, tracking of bubbles smaller than approximately 200 $\mu m$ requires high-speed video cameras equipped with microscope \citep{vakarelski2018}. \citet{vakarelski2018} studied the free rise and coalescence of small air-bubbles ($a \geq 100 \mu m$) at a liquid-air interface. They use a fluorocarbon liquid with a density of $\rho = 2030$ kg.m$^{-3}$ and a viscosity of $\mu = 0.0192$ Pa.s, which is highly resistant to surface-active contamination. Their findings reveal that the coalescence time $t_c$ for bubbles with radii $125 \mu$m is approximately $3.6$ ms and increases for larger bubbles. %Although the bubble size in their experiment is larger than that required for the negligible deformation assumption, 
In the following, we will demonstrate that our model agrees with their experimental results. Balancing the buoyancy force (neglecting the gas density and noting that $\bar{a}=a$) with equation \eqref{eq:kim_planar}, one obtain 
%Indeed our model can also be applied to the rising of a bubble at a free surface since the plane of symmetry separating the two bubbles is a shear free surface. 

\begin{equation}
- \frac{d\epsilon^2}{dt^*}(\log \epsilon ^2 + 2 A_\infty) + \frac{2}{3} =0
\label{eq:force_balance}
\end{equation}
where $t = \mu /(\rho a g) t^*$. Integration of this equation with respect to time yields %Replacing $h_0$ by $h_0/2$ in the viscous translating force given by \citet{kim1991} and balancing it with the buoyancy force $4/3\pi a^3\rho g$ we obtain

%($\rho = 2030$ kg.m$^{-3}$, $\mu = 0.0192$ Pa.s) which is highly resistant to surface active contamination. They measured the coalescence time $t_c$ for bubbles having radii larger than $100 \mu$m and found $t_c \approx 2 - 3$ ms. Although the size of the bubble is higher than the one prescribed to fulfil the negligible deformation assumption before we will see that our model gives pretty good agreement with the experimental results. Balancing  the first term of Equation \ref{eq:forcea} with the buoyancy force $4/3\pi a^3\rho g $ and integrating we obtain



%Regrettably, there are no existing experimental findings on the coalescence time of small bubbles, apart from the recent work by \citet{vakarelski2018}. The absence of prior research can be attributed to two reasons. Firstly, experimental contamination of the interface with impurities is highly challenging to avoid \citep{duineveld1994, vakarelski2018}. Secondly, monitoring bubbles smaller than 100 $\mu m$ requires high-speed video cameras equipped with microscopes \citep{vakarelski2018}. In their study, \citet{vakarelski2018} explored the free rise and coalescence of small air bubbles ($a \geq 50 \mu m$) at a liquid-air interface. They utilized a fluorocarbon liquid with a density of $\rho = 2030$ kg.m$^{-3}$ and a viscosity of $\mu = 0.0192$ Pa.s, which is highly resistant to surface-active contamination. Their findings revealed that the coalescence time $t_c$ for bubbles with radii exceeding $100 \mu$m was approximately $2-3$ ms. Although the bubble size in their experiment was larger than that required for the negligible deformation assumption, we will demonstrate that our model agrees significantly with their experimental results. By balancing the first term of Equation \ref{eq:forcea} with the buoyancy force $4/3\pi a^3\rho g$ and performing integration, we obtain...


%To compute the coalescence time we make use of the first term of Equation \ref{...} Balancing ... with the buoynacy force $4/3\pi a^3\rho g $ on the bubble and intergrating one get. This problem is similar to the rising of a bubble at a free surface  

%To the best of the author knowledge the experiments dealing with the smallest bubble are the one of \citet{varaleski2018}.
%In the following We compare ou results to the experimental results of Varaleski 2018. 


%Consequently, this analysis provides a way to calculate the coalescence time of very small bubbles, which is the time it takes for the bubble to drain the film located between them.

%Regrettably, there are no experimental findings for such small bubbles that we are aware of.

\begin{equation}
%(2\gamma + \ln 2 + 1)h_0^* - h_0^*\log(h_0^*) = (2\gamma + \ln 2 + 1)h_0^*(0) - h_0^*(0)\log(h_0^*(0))-\frac{2}{3}t^*
(1-2A_\infty )\epsilon^2 - \epsilon^2\log(\epsilon^2) = (1-2A_\infty )\epsilon^2(0) - \epsilon^2(0)\log(\epsilon^2(0))-\frac{2}{3}t^*
\end{equation}
%where the equation of motion has been integrated once with respect to time and the stared variables are dimensionless quantities. Specifically $h_0 = a h_0^*$ and $t = \mu /(\rho a g) t^*$. 
Since the coalescence time $t_c$ is defined as the duration required for the film to reach a zero thickness we get

\begin{equation}
%t_c^* = \frac{3}{2}h_0^*(0)\left(2\gamma + \ln 2 + 1 - \log(h_0^*(0))\right).
t_c^* = \frac{3}{2}\epsilon^2(0)\left( 1 -2A_\infty - \log(\epsilon^2(0))\right).
\label{eq:tc}
\end{equation}

One may note that in contrast to solid particles, the film drainage process occurs in a finite time. \citet{vakarelski2018} did not explicitly state the initial thickness $d(0)$ at which coalescence time measurements were conducted. However, they indicated that the time was determined once the bubble reached the interface. Given their spatial resolution exceeding $1.7\mu$m, it is reasonable to assume that $d(0)$ falls within the range of 2 pixels (3.4$\mu$m) to six pixels ($10.2\mu$m). 
%Using $h_0^*(0)=0.1$ ($h_0(0)=10\mu$m) yields $t_c^*\approx 0.77$ and $t_c \approx 6$ ms, while $h_0^*(0)=0.02$ ($h_0(0)=2\mu$m) leads to $t_c^*\approx 0.2$ and $t_c \approx 1.6$ ms.
% Figure environment removed
Equation \eqref{eq:tc} is in good agreement with the experimental findings for $a \leq 150 \mu$m (Figure \ref{fig:vakarelski}). However, it should be noted that our coalescence time prediction exhibits a behavior of $a^{-1}$ for a fixed initial film thickness $d(0)$, while \citet{vakarelski2018} observed an increase in coalescence time proportional to $a^2$ for larger bubbles (Figure \ref{fig:vakarelski}). This probably indicates that the results reported by \citet{vakarelski2018} for the smallest bubbles lie at the boundary between the Taylor regime of film drainage, where deformation is negligible, and the Reynolds regime, where deformation is non-negligible \citep{ivanov1999}. 


Our analysis assumes that the conditions $Ca \ll \epsilon ^2$, $Re \ll 1$, and $\lambda \ll \epsilon$ are met. In the subsequent discussion, we address the validity of each of these assumptions. %In the following we discuss the validity of each of these assumptions. 
First, we have neglected (convective) inertia effects, which is acceptable since the Reynolds number associated with bubble motion is significantly less than unity. Indeed, to leading order, the velocity of the bubble reads $1/(3\log\epsilon)$. Hence, the assumption $Re \ll 1$ requires

\begin{equation}
Ar \ll \frac{\log \epsilon}{\epsilon}
\label{eq:Ar}
\end{equation}
where $Ar = \rho ^2 a^3 g/\mu ^2 $ is the Archimedes number. Typically, for $\epsilon = 0.1$, we find $|\log(\epsilon)|/\epsilon \approx 23$. Consequently, condition \eqref{eq:Ar} is met in the experiments conducted by \citet{vakarelski2018} for $a = 150 \mu m$, since $Ar \approx 0.4$. %Tipically for $\epsilon =0.1$ we get$|\log(0.1)|/0.1 \approx 23$. Hence the requirement is tipicaylly satisfied in the experiments of \citet{vakarelski2018} for $a=150 \mu m$ as $Ar \approx 0.4$.
%\color{red}
%($Re \approx 0.1$ for the smallest bubbles in \citet{vakarelski2018} experiments). 
%Since the velocity of the bubble reads $2/(6\log\epsilon)$ to leading order, the assumption of small Reynolds number require ...
%\begin{equation}
%toto
%\end{equation}
However, the effect of fluid unsteadiness on the force balance requires further consideration. From Equation \eqref{eq:force_balance}, we note that the timescale $\tau$ governing the balance between drag and buoyancy forces is $\mu/(\rho a g)\epsilon^2 \log(\epsilon)$. This result can also be derived from Equation \eqref{eq:tc}. The unsteady term in the Navier-Stokes equation becomes negligible compared to the viscous term only if $\tau \gg \rho a^2/\mu$ \citep{kim1991}, which requires



\begin{equation}
%Ar \ll \frac{3}{2}\epsilon^2 \log(\epsilon^2)
Ar \ll \epsilon^2 \log \epsilon
\label{eq:Ar_unsteady}
\end{equation}

This criterion is much more restrictive than \eqref{eq:Ar}. To relax this condition, one may consider the influence of unsteady forces in the viscous regime, such as added mass and history force, in the force balance (equation \eqref{eq:force_balance}). However, while the added mass as a function of the free surface distance may be obtained \citep{miloh1977}, to our knowledge, no expression exists for the history force of a bubble approaching a free surface. Furthermore, as illustrated in Figure \ref{fig:vakarelski} and as discussed in the original study by \citet{vakarelski2018}, the force balance \eqref{eq:force_balance} proves to be sufficiently accurate to predict the dynamics of small bubbles with radii up to approximately $100\mu$m. However, it must be said that in this case, the criterion \eqref{eq:Ar_unsteady} is not satisfied. This requires further investigation of the impact of unsteady forces on the dynamics of bubbles in proximity to a boundary.

%on the influence of unsteady forces on the dynamic of bubbles close to a boundary.
% although in this case the criterion \eqref{eq:Ar_unsteady} is not met.

%Additionally, as evidenced in Figure \ref{fig:vakarelski} and noted in the original work by \citet{vakarelski2018}, the force balance \eqref{force_balance} is sufficient to predict the dynamic of small bubbles with radius up to $a \approx 100\mu$m. % one may consider the unsteady forces in the viscous regime such as the added mass and history force in the force balance. However, if the added mass as function of the distance of the wall is known, to the best of our knowdleged no expression exist for the history force of a bubble approahcing a wall. Aditionnaly as observed on figure ... and in the original paper of Vakarelski, the force balance witout 

%For $\epsilon(0) = 0.1$ we find $|\log(\epsilon)|\epsilon\approx 0.23$.   

%\color{black}

We now consider the criterion $Ca \ll \epsilon ^2$. To leading order, the velocity of the bubble reads $1/(3\log\epsilon)$. %Assuming that the bubble has reached its steady velocity (given by a balance between the Hadamard-Rybczynski and the buoyancy forces) before reaching the plane surface given, we obtain (neglecting the density within the bubble) $V = \frac{1}{3}\frac{\rho g a^2}{\mu}$. 
Substituting this estimate into the Capillary number yields

%Assuming that the bubble has achieved a steady velocity before reaching the plane surface due to a balance between its drag force and weight, we obtain (neglecting the density within the bubble) the expression $V = \frac{1}{3}\frac{\rho g a^2}{\mu}$. Substituting this estimate into the Capillary number yields:
\begin{equation}
Bo \ll \epsilon \log \epsilon
\end{equation}
where $Bo = \rho g a^2/\gamma$ is the Bond number. %Thus $Ca \ll \epsilon^2$ implies $Bo \ll  \epsilon ^2$.
%For various liquid-air combinations, surface tension typically ranges between $20.10^{-3}$ N.m$^{-1}$ and $70.10^{-3}$ N.m$^{-1}$. In our analysis, we assume a surface tension value of approximately $\gamma \approx 40 \times 10^{-3}$ N.m$^{-1}$ and a liquid density of approximately $1000$ kg.m$^{-3}$. 
In the experiments of \citet{vakarelski2018}, when considering a bubble with a radius of $100\mu$m immersed in a liquid, the resulting Bond number is approximately $Bo \approx 9 \times 10^{-3}$, whereas for a bubble with a radius of $200\mu$m, $Bo \approx 3.6 \times 10^{-2}$. Subsequently, assuming significant deformation occurs when the Bond number approaches $\epsilon \log \epsilon$, for the $100\mu$m bubble, one might expect deformation when $d \approx a Bo ^2 \approx 8 \times 10^{-9}$m, and for the $200\mu$m bubble, $d \approx a Bo^2 \approx 2.6 \times 10^{-7}$m. Assuming that the critical thickness at which the film ruptures is typically around $10^{-8}$m \citep{chatzigiannakis2020}, we anticipate that the current analysis is applicable to bubbles with a radius of approximately $100\mu$m. Indeed, in such cases, deformation tends to occur at smaller film thicknesses. For larger bubbles the agreement between the model and the experiments is questionable since significant interface deformations are expected, indicating the possibility of an alternative regime of film drainage. Measurements of the film thickness of gas bubbles ascending towards a free surface under gravity were performed by \citet{kovcarkova2013}. They observed an exponential reduction in film thickness over time, with a characteristic time scale proportional to $\mu a / \gamma$ for small bubbles. For the present system with $a\approx 150 \mu$m, $\mu a / \gamma$ is approximately $0.1$ ms, indicating that the time scale for drainage after deformation is considerably shorter than the previously calculated value, providing confidence in the estimated coalescence time for $a \leq 150 \mu$m.

Finally, the condition $\lambda \ll \epsilon$ is fulfilled since the viscosity ratio is very small $\lambda\approx 10^{-3}$ in the experiments of \citet{vakarelski2018}. Indeed, the influence of shear stress within the bubble will manifest at film thicknesses smaller than the critical thickness at which film rupture occurs. However, one may expect a significant effect of the gas viscosity in the case of an air-water system ($\lambda \approx 0.018$). This topic will be discussed in the last section. %As a results effect of the shear-stress within the bubble will occur at smaller film thicknesses than the critical thickness at which the film ruptures.


\subsection{Dynamics of two growing bubbles}
\label{sec:ohashi}



Equation \eqref{eq:forcea} may be used in principle to predict the dynamics of two bubbles with time-dependent radii in the Stokes regime without external forces. A related analysis was conducted by \citet{michelin2019} for a bubble near a wall. Their findings indicated that the boundary conditions on the bubble surface determined whether the bubble would either continuously drain the fluid separating it from the wall or rebound once before draining the film. In their configuration, the translational lubrication force has a singularity that scales as $1/\epsilon^2$, which is significantly more singular than the inflating lubrication force (with $d$ fixed) that produced a weak logarithmic singularity. In contrast, in the present configuration, the translational lubrication force itself has a weak logarithmic singularity, while the inflating contribution to the lubrication force (with $d$ fixed) generates terms smaller than the order one, as the velocity $w$ involved in this case is proportional to $\epsilon ^2 \dot{a}$ is negligible. Consequently, when considering two force-free bubbles in the Stokes regime, Equation \eqref{eq:forcea} equals zero, leading to $\dot{d}=0$. This implies that the film thickness does not vary over time, %\color{blue}
%Consequently, the leading order lubrication force becomes zero when the relative displacement rate $\dot{d}$ equals zero, 
which requires that $V_2 - V_1 = \dot{a}_1 + \dot{a}_2$. When bubbles approach each other ($V_2 - V_1 < 0$), the force vanishes if the bubbles shrink at half their relative velocity, for instance. Conversely, when bubbles move apart, the force vanishes if they expand at half their relative velocity. 

This result contrasts with those of a recent experimental investigation by \citet{ohashi2022}, who observed the decrease in time of the film thickness $d$ of two adjacent, expanding bubbles with low capillary numbers. To accurately describe the dynamics of two bubbles with time-dependent radii, it is necessary to include the $\mathcal{O}(1)$ contributions to both the inflating (or draining) and translating force. In the case of force-free bubbles with equal velocity $V$ and expanding rate $\dot{a}$ and to the linearity of the Stokes equation, we may sum equations and \eqref{eq:kim_A} and \eqref{eq:michelin}. This yields 
\begin{equation}
-(\log \epsilon + A)V + (\log \epsilon + B)\dot{a} = 0
\end{equation}
where $A$ and $B$ are defined in section \ref{sec:viscf}. Noting that $\dot{d}=2V - 2\dot{a}$ the force-free condition finally reads

%$A = 2(\gamma + \log 2)\approx 2.54 $ represents the $\mathcal{O}(1)$ term associated with the translational force (divided by $2\pi$), as determined by \citet{kim1991}, and $B \approx -4.95/(2\pi)\approx -0.79$ represents the $\mathcal{O}(1)$ term associated with the growing force (see Figure \ref{fig:force_visc}). 


%The findings contrast with those of a recent investigation by \citet{ohashi2022} who observed the coalescence of two adjacent, expanding bubbles with low capillary number within a finite time.
%. In principle a numerical estimate of $B$ might be obtained using the bi-spherical coordinate solution proposed by \citet{michelin2018} but since it does not support our point. 


%This contradict the recent results of \citet{ohashi2022} in which two adjacent growing bubbles with low capillary number coalesce in a finite time. As a result to properly capture the dynamics of the two growing bubble one need to compute the $\mathcal{O}(1)$ contribution to both the tranlating and inflating force problem. In particular, in the case of force free bubble one will obtain $(\ln (a/h_0)+A)V+(\ln (a/h_0)+B)\dot{a}=0$ where $A$ is the $\mathcal{O}(1)$ term related to translational motion which have been computed by \citet{kim2013} and $B$ a $\mathcal{O}(1)$ term related to growing motion.

%Indeed if we do not do this we get $\dot{h_0}=0$. Which mean that the thickness do not evolves as function of time ! 

%Discuter le fait qu'à l'ordre dominant les forces a vitesse ou rayon variable sont les memes. En fait c'est en accord avec Michelin 2019. Les différences n'apparaissant que si l'on considere le terme en epsilon dans la vitesse. Pour des bulles je pense que d'autre terme sont important avant celui la. Par contre ssans la solution d'ordre 1ce n'est pas tres precis. Deriver le cas 2D ? et comparer aux cas en cellcule de Hele-Shaw de lequipe japonaise ?

\begin{equation}
%\frac{\dot{h}_0}{\dot{a}} = -\frac{4.95 + 4\pi(\gamma + \log 2)}{\log(a/h_0) + 4\pi(\gamma + \log 2)}
%\frac{\dot{h}_0}{\dot{a}} = \frac{2(B - A)}{\log(a/h_0) + A}
\frac{\dot{d}}{\dot{a}} = \frac{2(B - A)}{\log(\epsilon) + A}
\label{eq:h_0dot}
\end{equation}
Equation \eqref{eq:h_0dot} indicates that the film thickness is a decreasing function of time for two growing bubbles in agreement with the experimental results of \citet{ohashi2022}. They consider the coalescence of two growing bubbles in highly viscous liquids ($5.10 ^{-6}\lesssim Re \lesssim 10 ^{-4}$, $2.10^{-5}\lesssim \lambda \lesssim 2.10^{-4}$). Two approximatively equal-sized bubbles were injected thanks to a syringe inside a closed box. Then, the pressure inside the box was reduced. 
We can go further in the comparison with \citet{ohashi2022} experiments by directly comparing $\dot{d}/\dot{a}$ to their experimental values. Specifically, we focus on experiments characterized by the smallest capillary numbers based on the growth rate, denoted as $Ca_{\dot{a}} = \mu \dot{a}/\gamma \approx 0.1$ and $Ca_{\dot{a}} \approx 0.005$, originally labeled as cases (a) and (c) in \citet{ohashi2022}. As depicted in Figure \ref{fig:ohashi}, there is excellent agreement between the force-free bi-spherical solution (obtained by equating equations \eqref{eq:bispherical_trans} and \eqref{eq:bispherical}) and the experimental findings of \citet{ohashi2022} up to $\epsilon \approx 1$ for $Ca_{\dot{a}} \approx 0.1$ and up to $\epsilon \approx 0.2$ for $Ca_{\dot{a}} \approx 0.005$. However, for thinner films, noticeable bubble deformation occurs, as illustrated in the original article of \citet{ohashi2022}. We recall that the non-deformation assumption requires $Ca \ll \epsilon ^2$ which may rewritten using \eqref{eq:h_0dot}

\begin{equation}
Ca_{\dot{a}} \ll \epsilon^2\log(\epsilon).
\end{equation}
In the case where $\epsilon = 0.2$, it follows that $\epsilon^2|\log(\epsilon)|$ yields an approximate value of 0.064 which is indeed much larger than the minimum capillary number examined in the study conducted by \citet{ohashi2022}. %Typically for $\epsilon =0.2$, $\epsilon^2|\log(\epsilon)| \approx 0.064$ which is indeed much larger than the smallest capillary number considered by \citet{ohashi2022}. %. Hence one may expect agreement for the smallest capillary number up to this value of $\epsilon$. 
As a consequence of deformation and the departure of our solution from the bi-spherical solution due %and from the fact that our solution deviates from the bi-spherical solution 
to the increased film thickness the agreement between formula \eqref{eq:h_0dot} and the experimental data is only noticeable within a relatively limited range of film thickness ($0.25 \lesssim \epsilon \lesssim 0.5$) (see Figure \ref{fig:ohashi}).


%Notably, beyond this range, our solution deviates from the bi-spherical solution due to the increased film thickness. 




%We consider the experiments with the smallest capillary number based on the growth rate $Ca_{\dot{a}} = \mu \dot{a}/\gamma \approx 0.1$ and $Ca_{\dot{a}} \approx 0.005 $ originally denoted case (a) and (c) in the article of \citet{ohashi2022}. As shown in Figure \ref{fig:ohashi} the agreement between the full bi-spherical solution (based on equations \eqref{eq:bispherical_trans} and \eqref{eq:bispherical}) and the experimental results of \citet{ohashi2022} is very good up to $\epsilon \approx 1$ for $Ca_{\dot{a}} \approx 0.1 $ and up to $\epsilon \approx 0.25$ for $Ca_{\dot{a}} \approx 0.005 $. For smaller film thickness significant bubbles deformation are observed as demonstrated in the paper of \citet{ohashi2022}. As a result, the agreement between formula \ref{eq:h_0dot} and the experimental results is only observable for a realtively narrow range of film thickness ($0.25\leq \epsilon \leq 0.5)$ (Figure \ref{fig:ohashi}). Indeed for larger thickness, our solution is not supposed to work the solution deviates from the bi-spherical solution.

 %while for smaller thickness, deformation can be significant. %Also the good agreement is particularly 

% ce wui est normal car pour des epaisseurs trop forte ce n'est pas cense fonctionner mais pour des epaisseurs trop importante ni pour des epaisseurs trop faible.

%smallest bubble of the experiments ($a(0) \approx $) such that to maintain the capillary number as small as possible. 

% Figure environment removed


%We now discuss the range of validity of the agreement. 
%We may note that in the experiments of \citet{ohashi2022} the bubble are not exactly force-free. Indeed due to the buoyancy force the bubble are rising under gravity. In the limit of small but finite Reynolds number (based on the rising velocity) limit two bubble rising side by side are known to experience a lift force which scales proportionnaly with the Reynolds number \citep{vasseur1976}. \citet{legendre2003} reported direct numerical simulation of bubble rising side by side and show that this coefficient is not singular in the limit $\epsilon \leq 1$. Under the assumption of small Reynolds number (based on the rising velocity), and small film tchickness under which the repulsive force between the two bubble is singular one may confidently disregard this term. 

%The assumption $Re \ll 1$ may also be written

%\begin{equation}
%\frac{\rho \dot{a} a}{\log \epsilon \mu} \ll 1
%\end{equation}
%Hence is the growth rate is sufficiently small thhis assumption should be satisfied. 

%Thus this force may be neglected for sufficiently small film tichckness and Reynolds number. %and for the smallest separation distance in their paper they found the force to be equal to $F$


% The $Re$ number in this donner lequation avec lexpression de d point, may be expressed as


%which is small is the rate of change of the radisu is small
%\color{black}


%Despite this limitation \eqref{eq:h_0dot} provides a convenient appproximate tool to investigate in the evolution of the film thickness as function of the growth rate in the case of small bubbles. We did not succeeded in finding a closed form  solution to \eqref{eq:h_0dot} in the case where $a$ is an arbitrary function of time. However one can make analytical progress by 

%However one may investiagte the case of a constant growth rate ? meme pas sur. ce qui est sur c'est que d point sanulle quand epsilon devient tres faible ? cela veut il dire que l'on a atteint d=0 en un temps fini ?

%discuter des limites de Ohashi : on est pas exactement force free, nombre capillaire etc.

%PLus montrer si ou non on a coalescence en un temps fini. Ne semble pas faisable en pratique

%on va ajouter le cas 2D de Ohashi ce serait trop bete de pas essayer. sauf que il faut se coltiner la derivation de tout car ce que donne Savva c'est juste la formule avec la capillarite (cela n'apporte pas grand chose). En pratique dire que l'on a une decroissance plus forte en racine de epsilon et mettre cela dans la discussion.


\section{Discussion}
\label{sec:conc}

In this article, we have computed the lubrication force between two translating bubbles with time-dependent radii in viscous-dominated flows. Our findings demonstrate that the lubrication theory successfully captures the dominant term in the force expression for small inter-bubble gaps. However, for more accurate results, including $\mathcal{O}(1)$ terms in the force expression is necessary. Subsequently, we applied our findings to two problems in viscous-dominated flows: the coalescence of a small bubble at a free surface and the dynamics of two force-free expanding bubbles. The theory exhibited reasonable agreement with experimental observations within the prescribed range of applicability. 


In Section \ref{sec:vaka}, it was established that the negligible deformations condition necessitates the Bond number to be very small. In practical applications, it is observed that for a given inclusion size, the Bond number is approximately two orders of magnitude smaller for droplets ascending in liquids than bubbles \citep{balla2020}. Consequently, our theoretical framework may encompass a broader range of droplet sizes than bubble sizes. However, given that droplets typically exhibit higher viscosity ratios than bubbles, there exists a practical need to relax the shear-free boundary condition ($\lambda \ll \epsilon$) and consider scenarios where $\lambda \sim \epsilon$. Although the derivation of the governing equations remains the same (see section \ref{sec:lubrication}), it becomes necessary to incorporate the internal flow within the droplet by accounting for the tangential stress at the interface (as expressed in Equation \eqref{eq:qdm_shear}). Within the Stokes flow regime, the velocity distribution inside the bubble or droplet can be determined by employing the boundary integral formulation of the Stokes equations. \citet{davis1989} have previously obtained an equation characterizing the shear stress distribution along an axisymmetric plane interface as a function of the velocity. Their expression is as follows

\begin{equation}
f^*(r^*)=4\int_0^\infty \phi\left(\frac{R^*}{r^*}\right)\left(\frac{u^*}{R^{*2}}-\frac{1}{R^*}\frac{\partial u^*}{\partial R^*}-\frac{\partial^2 u^*}{\partial R^{*2}}\right)dR^*
\end{equation}
where $\phi$ is a Green function defined in Appendix \ref{app:f}. Although this formulation has been used by various authors \citep{davis1989,rother1997,nemer2013}, the computation of $f^*$ via numerical integration can pose challenges, as evidenced by the recent investigation by \citet{ozan2019}. The numerical integration process is detailed in Appendix \ref{app:f} and a Python script performing the integration can be found here \citep{piersongit}. Given that $f^*$ remains unaffected by the drop radius within the scope of the approximation considered here (nearly flat fluid interfaces), we may substitute $2f^*$ for $f_2^* - f_1^*$ in Equation \eqref{eq:qdm_shear}. Upon integrating Equation \eqref{eq:qdm_shear} with respect to $r^*$ and using Equation \eqref{eq:u}, the resulting pressure profile can be expressed as%Although this formulation has been used by many authors \citep{davis1989,rother1997,nemer2013}, the numerical integration to compute $f^*$ may be delicate to perform as demonstrated in the recent work of \citet{ozan2019}. In appendix \ref{app:f} we detailed how the numerical integration is performed. Since $f^*$ is independent of the radius of the drop, to the order of appoximation considered here (almost flat fluid interface) we may replace $f_2 - f_1$ in formula \eqref{eq:qdm_shear} by $2f$. Integration of Equation \eqref{eq:qdm_shear} in $r^*$ and making use of Equation \eqref{eq:u} the pressure profile reads

\begin{equation}
p^* = -\frac{1}{h^*} + 2 \frac{\lambda}{\epsilon} \int _{r^*} ^\infty\frac{f^*}{h^*}dr^*
\label{eq:p_lambda}
\end{equation}
Numerically integrating the last term in the previous expression produces the function $p^*$ as depicted in Figure \ref{fig:pressure_lambda}. The contribution arising from the non-negligible shear at the interface significantly affects the overall pressure distribution within the film.
%Performing the integration numerically, yields the function $p^*$ shown in Figure \ref{fig:pressure_lambda}. The term arising form the non-geligible shear at the interface has a non-neglible effect on the whole pressure within the film.

% Figure environment removed
The force acting on the bubble (or droplet) can be calculated using the methodology detailed in Appendix \ref{app:f}, resulting in

\begin{equation}
F = -4\pi\mu \bar{a} \dot{d} \left( \log \epsilon - C \frac{\lambda}{\epsilon} \right) .
%F = -2\pi \mu V \log \left( \frac{a}{h_0}\right) +(a/h_0)^{1/2}*...
\label{eq:force_lambda_num}
\end{equation}
where $C \approx 1.3085$. Equation \eqref{eq:force_lambda_num} reveals that the force comprises two distinct contributions. One arises from the pressure within the film when the bubble interfaces experience negligible shear, while the other arises due to the assumption of non-negligible shear at the interface. Both contributions exhibit similar orders of magnitude when $\lambda \sim \epsilon \log \epsilon$. Furthermore, it is evident from \eqref{eq:force_lambda_num} that for small film thickness, i.e., when $\epsilon \log \epsilon \ll \lambda  $, the term associated with non-negligible shear becomes dominant. Additionally, this term is similar to the analysis conducted by \citet{davis1989}. 

Although equation \eqref{eq:force_lambda_num} may be used in practical scenarios involving bubbles or droplets with varying and time-dependent radii, section \ref{sec:viscf} has demonstrated that incorporating the subdominant $\mathcal{O}(1)$ terms in the expansion enhances the accuracy of the results. Given that in the limit where $\lambda \sim \epsilon$, the correction to the $\mathcal{O}(1)$ term derived in part \ref{sec:viscf} is negligible the force on two equal bubbles approaching with a velocity $V$ may be expressed as 

%compared to the $\mathcal{O}(1)$ term derived under the assumption $\lambda = 0$, one can readily employ the findings from part ...

%Although \ref{eq:force_lambda_num} may be applied in practical configurations involving bubbles or droplets of different  and time-dependent radii we have seen previsously that adding the subdominant  \mathcal{O}(1) terms in the expansion improve the accuracy of the results. Since in the limit $\lambda \sim \epsilon$, the correction to the O(1) term is negliglble in comparison to the O(1) term derived under the assumption $\lambda =0$, one may simply use the results of part ... %is the same is negliglbe IN the case of two equal bubbles one may write 

%more desriable to find the 

\begin{equation}
F = 8\pi \mu \bar{a} V \left( \log \epsilon - C \frac{\lambda}{\epsilon} + A \right).
%F = -2\pi \mu V \log \left( \frac{a}{h_0}\right) +(a/h_0)^{1/2}*...
\label{eq:force_lambda_num_equals}
\end{equation}
% Figure environment removed
In figure \ref{fig:two_drops}, we compare this asymptotic solution with the exact solution obtained using bi-spherical coordinates. A very good match is observed between both solutions. Also, one may see that for the highest viscosity ratio considered in the figure, the influence of the term due to non-negligible shear becomes predominant for small $\epsilon$.


Significant effects of gas viscosity are anticipated, particularly in an air-water system where the viscosity ratio ($\lambda \approx 0.018$) is non-negligible. Given the practical importance of the air-water system, it warrants discussion. We anticipate no significant influence from the viscosity ratio prior to film rupture for very small bubbles, typically smaller than $100 \mu m$. However, as illustrated in Fig. \ref{fig:two_drops}, for larger bubbles, the effect of the viscosity ratio becomes non-negligible when the bubbles experience minimal deformation and most probably subsequently when substantial deformation occurs, leading to film drainage \citep{liu2019}. Moreover, since the pressure within the film depends on the viscosity ratio, one may expect a significant influence of the viscosity ratio on the thickness for which bubble deformations occur. The intricate interplay between bubble deformations and viscosity ratio remains a subject for future investigation. This problem could be addressed by employing an asymptotic expansion for small $Ca$ as demonstrated in \citet{yiantsios1990}, for instance. Nonetheless, given that this method is constrained to small deformations, employing numerical simulations may be better suited for investigating scenarios involving significant deformations.%This problem may be undertake using an asymptotic expansion as a function of the small parameter $Ca$ as in the work of \citet{yiantsios1990} for instance. However, since this approach is limited to small but finite deformation the use of a numerical code might be more approriate to investigate the regime of large deformation. 



%\color{red}

Although we derived the lubrication force in the Stokes limit, one may anticipate obtaining it for moderate Reynolds numbers. Indeed one may consider inertia in the derivation of lubrication equations \citep{chesters1982,howell1996,savva2009} and consider its effect on the lubrication force. We may also compute the subdominant $\mathcal{O}(1)$ terms to improve the accuracy of the solution. One potential approach is to employ direct numerical simulations, as demonstrated in the works of \citet{teng2022,terrington2023}. They explicitly compute the $\mathcal{O}(1)$ term in the force expression in the case of a rotating and possibly translating cylinder nearby to a plane wall. Another related methodology was proposed by \citet{kropinski1995} in their investigation of low Reynolds number flows past a cylindrical body. %Another related approach was introduced by \citet{kropinski1995} when studying the low Reynolds number flows past a cylindrical body. 
Finally, a natural extension of the present work would involve deriving the theory in potential flow regime \citep{van2002}. 


%\color{black}


\backsection[Acknowledgements]{The support of V\'eronique Lachet and Benoit Cr\'eton from IFP Energies Nouvelles is gratefully acknowledged. The author is indebted to Professor Masatoshi Ohashi for providing the experimental datas of \citet{ohashi2022}. The author wishes to thank the referees for their valuable comments which helped him to improve the presentation.}%Finally, he thanks the three anonymous referees whose constructive criticism served as motivation to substantially improve the first draft of the manuscript.} %the author would like to thank to the three anonymous referees who by their contructif cirticism, mottivate the author to significantly improve the manuscript.}%I would like to thank Professor Ohashi for providing his experimental results}
%In conclusion, the author expresses gratitude to the three anonymous referees whose constructive criticism served as motivation to substantially enhance the manuscript.

\backsection[Funding]{The financial support of IFP Energies Nouvelles is acknowledged.}

\backsection[Declaration of interests]{The authors report no conflict of interest.}

\appendix

\section{Details on the calculation of the lubrication force}
\subsection{Pressure distribution}
\label{app:pressure}

The viscous term may be separated into two parts. Using equation \eqref{eq:u} the integral of the first part can be expressed as follows
\begin{equation}
2 \int \frac{\partial}{\partial r^*}\left(\frac{1}{r^*}\frac{\partial}{\partial r^*}(r^* u^*)\right) dr^* = \frac{2}{r^*}\frac{\partial}{\partial r^*}(r^* u^*) + C = -\frac{2}{h^{*2}} + C,
\end{equation}
while the integral of the second part reads
\begin{equation}
2 \int \frac{1}{h^*}\frac{\partial h^*}{\partial r^* }  \left(2\frac{\partial u^*}{\partial r^*}+\frac{u^*}{r^*} \right)dr^* = \frac{2}{h^{*2}} - \frac{1}{h^*}+ C
\end{equation}
where $C$ is a constant.
Consequently the antiderivative of the viscous terms is $-1/h^* +C$.
%\begin{equation}
%2 \int \frac{\partial}{\partial r}\left(\frac{1}{r}\frac{\partial}{\partial r}(r u)\right) + \frac{1}{h}\frac{\partial h}{\partial r }  \left(2\frac{\partial u}{\partial r}+\frac{u}{r} \right)dr = -\frac{1}{h} + C.
%\end{equation}


\subsection{Lubrication force}
\label{app:force}
The expression of the lubrication force is given by \eqref{eq:forcen2}. In this appendix, we provide the integral of each term appearing in the lubrification force.
%First one may remark tha
Since,
\begin{equation}
\int _0^{R_\infty^*/\epsilon} \frac{r^*}{h^*}dr^* = [\log(2+r^{*2})]_{0}^{R_\infty^*/\epsilon}
\end{equation}
Then,
\begin{align}
\int _0^{R_\infty^*/\epsilon} \frac{r^*}{h^*}dr^* &= \log\left(2+\frac{R_\infty^{*2}}{\epsilon^2}\right) -  \log(2) \\
                                          &\sim  -2\log \epsilon \quad \text{as } \epsilon \rightarrow 0.
\end{align}
%\begin{equation}
%\int _0^{R_\infty/\epsilon} \frac{r^3}{h^2}dr = \left[\frac{4}{r^2+2}+2\log(2+r^2)\right]_{0}^{R_\infty/\epsilon}
%\end{equation}
Moreover,
\begin{equation}
\int _0^{R_\infty^*/\epsilon} \frac{r^*}{h^{*2}}dr^* = \left[-\frac{2}{r^{*2}+2}\right]_{0}^{R_\infty^*/\epsilon},
\end{equation}
Therefore
\begin{align}
\int _0^{R_\infty^*/\epsilon} \frac{r^*}{h^{*2}}dr^* &= -\frac{2}{R_\infty^{*2}/\epsilon^2+2} +2 \\
                                            &\sim 2  \quad \text{as } \epsilon \rightarrow 0.
\end{align}

%Then, 

%\begin{equation}
%\int _0^{R_\infty/\epsilon} \frac{r^3}{h^2}dr \sim  -4\log \epsilon \quad \text{as } \epsilon \rightarrow 0
%\end{equation}

\section{Bi-spherical coordinate solution}
%\subsection{Bi-spherical coordinate solution for two identical droplets approahcing each other with equal speed} %(necessaire ?)}
\subsection{Two identical droplets with equal speed} 
\label{app:lub_stokes}
The lubrication force between two non-deformable drops moving with equal speed toward each other has been obtained in closed form by \citet{haber1973} (see also \citep{kim1991}). %As a result, the outer region needs to be taken into account to obtain a more accurate prediction.  One has two relate to the solution provided by \citet{bart1968} obtained using bi-spherical coordinate. 
The force may be expressed as 
\begin{equation}
F = 12 \pi \Lambda \mu a V
\label{eq:bispherical_trans}
\end{equation}
where $V$ is the velocity of each drop and $a$ the reduced radius. The coefficient $\Lambda$ reads

%In this appendix, we explain how to derive the force based on the results of \citet{kim1991} (using the methodology proposed originally by \citet{cox1967}) who derived the force between two identical bubbles. 
%In this appendix, we explain how to derive this force based on the methodology of \citet{cox1967} (see also \citet{kim1991})
\begin{equation}
\Lambda = \frac{2}{3} \sinh \alpha \sum_{n=1}^{\infty} \frac{n(n+1)}{(2n-1)(2n+3)}\left(\frac{A_n(\alpha)+\lambda B_n(\alpha)}{C_n(\alpha)+\lambda D_n(\alpha)}\right)
% \frac{n(n+1)}{(2n-1)(2n+3)}\left(\frac{2\sinh(2n+1)\alpha +(2n+1)\sinh 2 \alpha}{2(\cosh(2n+1)\alpha - \cosh 2\alpha)}-1\right).
\end{equation}
where 
\begin{equation}
A_n(\alpha) = 2((n+1)\sinh 2\alpha+2\cosh 2\alpha-2e^{-(2n+1)\alpha)},
\end{equation}
\begin{equation}
B_n(\alpha)=(2n+1)^2\cosh 2 \alpha-2(2n+1)\sinh 2\alpha -(2n+3)(2n-1)+4e^{-(2n+1)\alpha}, 
\end{equation}
\begin{equation}
C_n(\alpha) = 4 \sinh [(n-1/2)\alpha] \sinh [(n+3/2)\alpha],
\end{equation}
\begin{equation}
D_n(\alpha) = 2 \sinh[(2n+1)\alpha] - (2n+1)\sinh 2\alpha
\end{equation}
The parameters $\alpha$ is related to the film thickness as  $1 + \epsilon ^2/4 = \cosh \alpha $. %Remarkably the solution derived by \citet{bart1968} is identical to the one for which two identical bubbles approach each other \citep{haber1973}. The only difference appears in the definition of the coefficient $\alpha$, which reads $h_0/D+1=\cosh \alpha$ in the cases of two identical bubbles. Hence the lubrication force between two bubbles derived by \citet{kim1991}  can be used directly here. In the limit of $h_0/D \ll 1$, $\alpha \sim 2 \sqrt{h_0/D}$. Replacing this expression in equation (9.42) of \citet{kim1991} one recover equation \ref{eq:force_fmob} and the expression of \citet{pigeonneau2011}.


%\subsection{Bi-spherical coordinate solution for two identical expanding bubbles}%Force on growing bubbles}
\subsection{Two identical bubbles with time-dependent radii}%Force on growing bubbles}
\label{app:visc}
%In this appendix we compute the force on two identical growing bubble based on the bi-spherical solution proposed by \citep{michelin2018}. Since the two bubble are identical the plane of symmetry separating the bubbles is a shear-free boundary. As a result the coefficients $A_n$ and $C_n$ defined in \citep{michelin2018} are zero and the force reads 

In this appendix, we calculate the force exerted on two identical bubbles with time-dependent radii using the bi-spherical solution proposed by \citep{michelin2018}. Due to the identical nature of the bubbles, the plane of symmetry that separates them is a shear-free boundary. Consequently, the coefficients $A_n$ and $C_n$ defined in \citep{michelin2018} are both zero, leading to the following expression for the force

\begin{equation}
F = \frac{8\pi \sqrt{2}\mu \dot{a} a^2}{k}\sum_{n=1}^{\infty}\left(\frac{2n+1}{4n+2}\right)(B_n+D_n)
\label{eq:bispherical} 
\end{equation}
where $\dot{a}$ is the time rate of change of the bubble radius and
\begin{equation}
B_n = \frac{-U_n''+(n+3/2)^2U_n}{\sinh[(n-1/2)\eta]} \quad \text{and} \quad D_n = \frac{U_n''-(n-1/2)^2U_n}{\sinh[(n+3/2)\eta]}.
\end{equation}
The functions $U_n$ and $U_n''$ are defined as
\begin{align}
U_n(\eta) = &-\frac{3\sqrt{2}\sinh^2(\eta/2)}{2(2n+1)}  \left(\frac{e^{-(n+3/2)\eta}}{2n+3} - \frac{e^{-(n-1/2)\eta}}{2n-1}\right) \\
            &- \frac{\delta_{n1}\sqrt{2}}{3}(e^{\eta/2}-e^{-\eta/2})+\frac{\sqrt{2}\sinh^2\eta}{2(2n+1)}e^{-(n+1/2)\eta},
\end{align}
and
\begin{align}
U_n''(\eta) =& -\frac{3\sqrt{2}\sinh \eta}{8}  \left(-\frac{2\sinh^2 \eta e^{-(n+1/2)\eta}}{1+\cosh\eta}+(2n+3)e^{-(n-1/2)\eta}-(2n-1)e^{-(n+3/2)\eta}\right) \nonumber\\
 &- \frac{\delta_{n1}}{\sqrt{2}}(e^{\eta/2}-e^{-\eta/2}) - \left(n-\frac{1}{2}\right)\left(n+\frac{3}{2}\right)U_n(\eta),
\end{align}
where $k = \sqrt{(h_0+2a)^2/4-a^2}$ and $\sinh \eta = k/a$. In practice we truncate the infinite sum to a finite number $N \geq 1000$.
%\begin{equation}
%B_n = \frac{-U_n''+(n+3/2)^2U_n}{\sinh(n-1/2)\eta(4n+2)}
%\end{equation}

\section{Computation of the tangential shear stress $f^*(r^*)$ and the force $F^*$}
\label{app:f}
In this appendix, we give details on the way $f$ and $F$ are computed. The function $\phi$ can be expressed as follows \citep{davis1989,rother1997,nemer2013}


%(on reprends la formulation de Nemer qui est la plus claire, le moins sen va quand on explicite omega) (il semble que radoa et milex est deja fait ce travail)








%$\phi$ ext explicite chez Nemer en fonction d'integrale elliptique. 
\begin{equation}
\phi(x) = \frac{1}{2\pi}\left[\frac{1+x^2}{1+x}K\left(\frac{4x}{1+x^2}\right)-(1+x)E\left(\frac{4x}{1+x^2}\right)\right]
\end{equation}
where $x = R/r$ and $K$ and $E$ are the first and second-kind elliptic integrals. The primary difficulty encountered in the computation of function $f$ arises from the divergence of $\phi$ as the radius $R^*$ approaches $r^*$. Indeed,%The main difficulty when computing $f$ is that $\phi$ diverges as $R^*$ tends to $r^*$.


%Le probleme est que $\phi$ diverge (faiblement) quand R tends vers r. En particulier (cf Barnocky et Davis):

\begin{equation}
\phi\left(\frac{R^*}{r^*}\right) \sim -\frac{1}{2\pi}\ln\left(\frac{| R^*-r^* |}{r^*}\right) \quad \text{as} \quad  R^* \rightarrow r^*. 
\end{equation}
To perform the integration we define two integrals by adding or substrating the previous "weak" logarithm singularity
%Nous definissons donc deux integrales (avec ou non la partie singuliere) 
%\begin{equation}
%f_i(r,t)=4\int_0^\infty \left(\phi\left(\frac{R}{r}\right)+\frac{1}{2\pi}\ln\left(\frac{| R-r |}{r}\right)\right)\left(\frac{u}{R^2}-\frac{1}{R}\frac{\partial u}{\partial R}-\frac{\partial^2 u}{\partial R^2}\right)dR
%\end{equation}

%\begin{equation}
%f_s(r,t)=4\int_0^\infty -\frac{1}{2\pi}\ln\left(\frac{| R-r |}{r}\right)\left(\frac{u}{R^2}-\frac{1}{R}\frac{\partial u}{\partial R}-\frac{\partial^2 u}{\partial R^2}\right)dR
%\end{equation}

%Normalizing the quantities as follow $h(r,t)=h^*(r^*,t^*)h_0(t)$, $r=r^*\sqrt{ah_0}$, $f=f^*V/h_0$, $u=u^*V(a/h_0)^{1/2}$.
\begin{equation}
f_i^*(r^*)=4\int_0^\infty \left[\phi\left(\frac{R^*}{r^*}\right)+\frac{1}{2\pi}\ln\left(\frac{| R^*-r^* |}{r^*}\right)\right]\left(\frac{u^*}{R^{*2}}-\frac{1}{R^*}\frac{\partial u^*}{\partial R^*}-\frac{\partial^2 u^*}{\partial R^{*2}}\right)dR^*,
\end{equation}

\begin{equation}
f_s^*(r^*)=4\int_0^\infty -\frac{1}{2\pi}\ln\left(\frac{| R^*-r^* |}{r^*}\right)\left(\frac{u}{R^{*2}}-\frac{1}{R^*}\frac{\partial u^*}{\partial R^*}-\frac{\partial^2 u^*}{\partial R^{*2}}\right)dR^*,
\end{equation}
The first integral is calculated numerically whereas the second integral can be obtained analytically, resulting in %while the second can be obtained analitically and yields
\begin{equation}
f_s^*(r^*)= -\frac{2+4\ln r^*+3\pi r^*+2r^{*2}+\pi r^{*3}}{\pi (1+r^{*2})^2}.
\end{equation}
%La premiere integrale est calculee numeriquement et nous obtenons un tres bon accord avec la solution de Davis
%Attention il y a deux petites choses qui viennent modifier notre convention. La premiere est que $a$ chez Davis et $a/2$ ches nous. Cela induit un facetur racine de 2. Par ailleurs on a un facteur 2 car chez Davis W=2V.
%% Figure environment removed
In order to compute the pressure within the thin film, as expressed by formula \eqref{eq:p_lambda}, we decompose it into two distinct contributions. The first contribution, denoted as $p^*_{\lambda \ll \epsilon}$, arises solely from the viscous stress within the film when the shear at the interface is negligible, and can be expressed as $p^*_{\lambda \ll \epsilon} = -1 /h^*$. The second contribution, denoted as $p^*_{\lambda \sim \epsilon}$, emerges due to the non-negligible shear at the interfaces and is given by the expression $p^*_{\lambda \sim \epsilon} = 2 \frac{\lambda}{\epsilon} \int _{r^*} ^\infty\frac{f^*}{h^*}dr^*$. By noting that  $f_i(r^*) -4/\pi\ln r^*$ and $f_s+4/\pi \ln r^*$ can be integrated numerically we can compute $p^*_{\lambda \sim \epsilon}$ as depicted in Figure \ref{fig:pressure_lambda}. 
%To compute the pressure within the thin film which is given by formula \eqref{eq:p_lambda} We separate the pressure it in two contributions. The first $p^*_{\lambda \ll \epsilon} =-1 /h^*$ is only due to the viscous stress within the film when the shear at the interface is negligible. The second $p^*_{\lambda \sim \epsilon} =  2 \frac{\lambda}{\epsilon} \int _{r^*} ^\infty\frac{f^*}{h^*}dr^* $ is due to the non-negligible shear at the interface. Then by noting that the following $f_i(r^*) -\frac{4}{\pi}\ln r^*$ and $f_s+4/\pi \ln r^*$ can be integrated numerically we may 

%By assuming that the pressure far from the film tends to zero To compute the force we note that. We recall that the force  may be obstained uusing formula and the pressure from  \eqref{eq:p_lambda}. By integration by part we obtain 

We now compute the contribution fo the force due to the non-negligible shear at the interfaces denoted as $F^*_{\lambda\sim \epsilon }$. This contribution may be expressed as follows

\begin{equation}
F^*_{\lambda\sim \epsilon } = 2 \pi \int _0 ^{R_\infty ^* /\epsilon}  p^*_{\lambda \sim \epsilon} dr^*. 
\end{equation}
This last expression can be simplified in several ways. First, the upper limit of the preceding expression should be in the range $1 \ll R_\infty ^* /\epsilon \ll 1/\epsilon$. However, given that the parameter $p^*$ tends towards zero as the radial coordinate $r^*$ approaches infinity, we may substitute the upper limit with $r^*=\infty$ \citep{davis1989}. Second, this integral may be integrated by parts. Assuming that $p^*r^{*2}$ tends towards zero as $r^*$ approaches infinity, we get %Second, the integrand in the preceding expression may be integrate by part which gives (assuming that $p^*r^{*2}$ $\rightarrow 0$ as $r^*$ $\rightarrow \infty$)

\begin{equation}
F^*_{\lambda\sim \epsilon } = 2 \pi \frac{\lambda}{\epsilon} \int _0 ^{\infty} \frac{f^*}{h^*} r^{*2} dr^*.
\end{equation}
Performing the numerical integration yields an approximate result of 

\begin{equation}
F^*_{\lambda\sim \epsilon } \approx -16.4435605 \frac{\lambda}{\epsilon}. 
\end{equation}
This approximated result can be compared to the exact coefficient as reported by \citet{kim1991} $-\frac{3 \sqrt{2}}{8}\pi^3 \approx -16.4435614$. For implementation details, a Python script facilitating the integration process can be found here \citep{piersongit}.

%First, the upperlimit of the previous expression should be in the range $1 \ll R_\infty ^* /\epsilon \ll 1/\epsilon$, but since $p^*$ $\rightarrow 0$ for $r$ $\rightarrow  \infty$, we may replace the upper limit. $r*=\infty$.

 %Initially, the upper limit of the preceding expression is constrained to the interval $1 \ll R_\infty ^* /\epsilon \ll 1/\epsilon$. However, given that the parameter $p^$ tends towards zero as the radial coordinate $r$ approaches infinity, it is permissible to substitute the upper limit with $r^=\infty$.


%\begin{equation}
%f_i(r,t)-\frac{4}{\pi}\ln r^*
%\end{equation}
%est reguliere. Et inversement $f_s+4/\pi \ln r^*$ est reguliere. Cela facilite tres largement l'integration numerique. On integre ces deux quantites pour obtenir la pression et la force. Par integration par partie nous obtenons

%\begin{equation}
%F^* = 2\pi\int _0^{R_\infty}\frac{f^*}{h^*}r^{*2}dr^*
%\end{equation}


%Perfoming the numerical integration one obtain $F^* \approx -16.4435605$ while the exact value is $F^*=-\frac{3 \sqrt{2}}{8}\pi^3 \approx -16.4435614$ \citep{kim1991}. A python script perofming the integration is available \citep{piersongit}.

%\begin{equation}
%\int _0^{\infty}\frac{-4r^{*2}}{(1+r^{*2})^3}dr^* = -\frac{\pi}{4}
%\end{equation}

%C'est le cas le plus complique. en fait il s'annule car la primitive se comporte de la meme maniere en 0 et l'infini.
%\begin{equation}
%\int _0^{\infty}\frac{-8\ln r r^{*2}}{(1+r^{*2})^3}dr^* = 0
%\end{equation}


%\begin{equation}
%\int _0^{\infty}\frac{-6\pi r^{*3}}{(1+r^{*2})^3}dr^* = -\frac{3\pi}{2}
%\end{equation}

%\begin{equation}
%\int _0^{\infty}\frac{-4 r^{*4}}{(1+r^{*2})^3}dr^* = -\frac{3\pi}{4}
%\end{equation}

%\begin{equation}
%\int _0^{\infty}\frac{-2\pi r^{*5}}{(1+r^{*2})^3}dr^* = \frac{3\pi}{2} -\ln(R_\infty/(ah_0)^{1/2})
%\end{equation}

%Pour des raisons qui m'echappent on ne capture pas le terme en ln epsilon avec la solution BIM. pourtant on a bien un terme en epsilon ln epsilon (qui sannule en fait a priori avec le terme totalement numerique).




%\section{Acknowledgment}
%This research was funded by grants from IFP Energies Nouvelles.
\bibliography{biblio.bib}
%\bibliographystyle{apalike-fr}
\bibliographystyle{apalike}


\end{document}
%
% ****** End of file aiptemplate.tex ******
