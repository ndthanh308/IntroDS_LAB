\documentclass[prb,preprint,superscriptaddress,onecolumn,12pt]{revtex4-1} 
% The line above defines the type of LaTeX document.
% The % character begins a comment, which continues to the end of the line.

\usepackage{amsmath}  % needed for \tfrac, \bmatrix, etc.
\usepackage{amsfonts} % needed for bold Greek, Fraktur, and blackboard bold
\usepackage{graphicx} % needed for figures
\usepackage{esvect}
\usepackage{mathtools}
\usepackage{subcaption}
\usepackage{xcolor}
\usepackage{appendix}
\usepackage{siunitx} 
\sisetup{
  round-mode          = places, % Rounds numbers
  round-precision     = 2, % to 2 places
}

\usepackage{placeins}

\usepackage{float} 

\usepackage{pbox}

\usepackage{mathtools}
\makeatletter
\newcommand{\vast}{\bBigg@{3}}
\newcommand{\Vast}{\bBigg@{3.5}}
\makeatother

\usepackage[mathscr]{euscript}
\DeclareSymbolFont{rsfs}{U}{rsfs}{m}{n}
\DeclareSymbolFontAlphabet{\mathscrsfs}{rsfs}

\newcommand{\dd}{\mathrm{d}}
\newcommand{\bd}{\mathscrsfs{D}}


\DeclareMathOperator{\expint}{E} 
\DeclareMathOperator{\hyperu}{U} 
\DeclareMathOperator{\laguerrel}{L} 

\begin{document}

% Be sure to use the \title, \author, \affiliation, and \abstract macros
% to format your title page.  Don't use lower-level macros to  manually
% adjust the fonts and centering.

\title{A Schr\"{o}dinger-Bloch Equation for Evolutionary Dynamics}
% In a long title you can use \\ to force a line break at a certain location.

\author{Duy V. Tran}
\thanks{These authors contributed equally to this work.}
\affiliation{Department of Mechanical Engineering, VNUHCM University of Technology, 226 Ly Thuong Kiet, Ho Chi Minh, 11000, Vietnam.}

\author{Vi D. Ao}
\thanks{These authors contributed equally to this work.}
\affiliation{Department of Physics, VNUHCM University of Science, 227 Nguyen Van Cu, Ho Chi Minh, 700000, Vietnam.}

\author{Kien T. Pham}
\affiliation{Department of Aerospace Engineering, School of Transportation Engineering, Hanoi University of Science and Technology, 01 Dai Co Viet, Hanoi, 100000, Vietnam.}

\author{Duc M. Nguyen}
\affiliation{University of Chicago, 5801 S Ellis Ave, Chicago, IL 60637, USA.}

\author{Huy D. Tran}
\affiliation{Department of Physics, The Hong Kong University of Science and Technology, Clear Water Bay, Kowloon, Hong Kong, 999077, PR China.}

\author{Tuan K. Do}
\affiliation{Department of Mathematics, UCLA, 520 Portola Plaza, Los Angeles, CA 90095-1555, USA.}

\author{Van H. Do}
\affiliation{Homer L. Dodge Department of Physics and Astronomy, Unversity of Oklahoma, 440 W. Brooks St. Norman, OK 73019, USA.}

\author{Trung V. Phan}
\email{trung.phan@yale.edu}
\affiliation{Department of Molecular, Cellular and Developmental Biology, Yale University, 260 Whitney Ave, New Haven, CT 06511, USA.}

\date{\today}

\begin{abstract}
We establish an analogy between the Fokker-Planck equation describing evolutionary landscape dynamics and the Schr\"{o}dinger-Bloch equation which characterizes quantum mechanical particles, showing how a population with multiple genetic traits evolves analogously to a wavefunction under a multi-dimensional energy potential in imaginary time. Furthermore, we discover within this analogy that the stationary population distribution on the landscape corresponds exactly to the ground-state wavefunction. This mathematical equivalence grants entry to a wide range of analytical tools developed by the quantum mechanics community, such as the Rayleigh-Ritz variational method and the Rayleigh-Schr\"{o}dinger perturbation theory, allowing us to not only make reasonable quantitative assessments but also explore fundamental biological inquiries. We demonstrate the effectiveness of these tools by estimating the population success on landscapes where precise answers are elusive, and unveiling the ecological consequences of stress-induced mutagenesis -- a prevalent evolutionary mechanism in pathogenic and neoplastic systems. We show that, even in a unchanging environment, a sharp mutational burst resulting from stress can always be advantageous, while a gradual increase only enhances population size when the number of relevant evolving traits is limited. Our interdisciplinary approach offers novel insights, opening up new avenues for deeper understanding and predictive capability regarding the complex dynamics of evolving populations.
\end{abstract}

\maketitle % title page is now complete

\section{Introduction}

Evolution is the primary driving force behind the diversity and complexity of life on Earth for billions of years \cite{darwin2004origin,dawkins2016selfish}, allowing organisms to change and adapt over time \cite{waddington1959evolutionary,rose1996adaptation}. It emerges through the synergy between natural selection and genetic mutations, in which natural selection favors combinations of traits that enhance fitness \cite{endler1986natural} while genetic mutations introduce genetic variations that facilitate the emergence of new advantageous traits \cite{nevo1978genetic}. Within populations, ecological factors such as niche constraint \cite{tsoularis2002analysis, getz1991unified,getz1994metaphysiological} and environmental stress \cite{bjedov2003stress,fitzgerald2017stress} can exert their influence on these processes \cite{conrad1990geometry}, even molding the trajectory and tempo of evolution \cite{zhang2011acceleration}. 

The complex dynamics of evolving population can be captured by a Fokker-Planck equation on the evolutionary landscape \cite{risken1991fokker}, an abstract space of all possible genetic variations and their corresponding biological properties within a given ecological context \cite{wright1932roles}. The number of relevant evolving genetic traits corresponds to the dimensionality of this space \cite{vincent2005evolutionary}, where every combination corresponds to an unique position. On the landscape, together with the ecological influence, we represent the mutation process with an effective diffusion \cite{kimura1964diffusion} and the natural selection pressure with a fitness potential \cite{sergey2004fitness}. We show that there is an analogy between this formulation of evolutionary dynamics and the Schordinger-Bloch description for quantum mechanical particles \cite{schrodinger1926quantisierung,bloch1929quantenmechanik}, in which how a population evolves on the multi-dimensional landscape is almost similar to how a wavefunction behave under a multi-dimensional energy potential in imaginary time \cite{popov2005imaginary,tran2022numerical}. Following this observation, we further discover that the stationary population distribution on the landscape corresponds exactly to the quantum ground-state wavefunction \cite{schrodinger1926quantisierung}. Such curious connection enables us to utilize various quantitative tools borrowed directly from quantum mechanics literature to quickly extract information about the steady population state of complex landscapes, even ones that lack exact analytical comprehension. In this study, we apply the Rayleigh-Ritz variational method \cite{rayleigh1907dynamical,ritz1909neue} to estimate the stationary population size for a family of single-peak landscapes in all dimensions.

Understanding the consequences of evolution has always been a cornerstone of biological research \cite{williams1970deducing}, serving as a fundamental pursuit aimed at unravelling the possible outcomes arising from this transformative force, and shedding light on the foundational principles that shape and govern all life on Earth. There exists many distinct evolutionary regimes \cite{mayr2001evolution}. For pathogenic and neoplastic populations such as microbial organisms and cancer cells, stress-induced mutagenesis \cite{bjedov2003stress,fitzgerald2017stress}, in which mutational increase can be triggered due to high biological stress, assumes a prominent role. Consider {\it E.coli} bacteria after experiencing exposure to antibiotics \cite{imasheva1999environmental,hoffmann2000environmental}, they undergo elongation and stop dividing \cite{phan2018emergence}. At the same time inside the bacterium, the SOS response switches on, leading to the induction of low-fidelity error-prone replication polymerases and, consequently, there is a sharp increase in the mutation rate during DNA replication from the typically low value of $D_l \propto 10^{-9}$ to a high rate of $D_h \propto 10^{-5}$ mutations per base pair per generation \cite{cirz2005inhibition, bos2015emergence}. There are also several other mechanisms by which genetic change can occur when organisms are under stress \cite{foster2007stress}. Laboratory studies have shown that at least $80\%$ of natural isolates of {\it E.coli} from diverse environments worldwide can exhibit stress-induced mutagenesis \cite{pribis2022stress}, highlighting its significance as an essential evolutionary dynamic in the realm of microbiology. Knowing the extensive effects of stress-induced mutagenesis in pathogenic and neoplastic systems is vital for developing strategies to combat their adaptive capabilities and improve therapeutic interventions \cite{govindaraj2018global}.

Here, thanks to the Schr\"{o}dinger-Bloch analogy, we can conveniently employ the Rayleigh-Schordinger perturbation theory \cite{cohen1982rayleigh} to assess the tendency of population change resulting from stress-induced mutagenesis. We consider two extremes: gradual increases in mutation rate with stress and a sharp mutational burst when stress levels surpass a certain fitness threshold. We demonstrate in an unchanging environment that, unlike the former case, the latter consistently leads to a net gain in the total population size. This finding offers an explanation for the frequent appearance of mutational switches observed in nature \cite{cirz2005inhibition, bos2015emergence}.

\section{Evolutionary Landscape and Ecological Influence}

The landscape is typically represented as a multi-dimensional Euclidean space $\mathbb{R}^{\mathscrsfs{D}}$, where each point $\vec{x}$ represents a unique combination of $\mathscrsfs{D}$ scalar-strategy genetic traits \cite{vincent2005evolutionary}. When the maximum fitness $R(\vec{x})$ (which represents the selection pressure) remains constant over time \cite{wright1932roles,sergey2004fitness}, the population distribution density $b(\vec{x},t)$ within this landscape evolves via the Fokker-Planck equation \cite{risken1991fokker}:
\begin{equation}
\partial_t b = \nabla^2 \left( D b \right) + R b \ ,
\label{fokker_planck}
\end{equation}
where the effective diffusivity $D$ represents the local speed of mutations. In other words, higher value of $D$ results in a faster population diversification. 

Our mathematical model is still incomplete as it assumes unlimited population growth. In any natural ecological system, the population growth of an organism should be limited by the resources available in its environment. The logistic model of population growth provides a better description of the population dynamics in a finite environment by taking into account the carrying capacity of the environment \cite{tsoularis2002analysis}. The carrying capacity $K$ in the logistic model of population growth \cite{getz1991unified,getz1994metaphysiological} represents the maximum number of individuals that can be sustained in a given environment. When the population size approaches the carrying capacity, the growth rate decreases until the population stabilizes at the carrying capacity. This carrying capacity can be incorporated into the mathematical framework by modifying Eq. \eqref{fokker_planck} into an integro-differential equation:
\begin{equation}
\partial_t b = \nabla^2 \left( D b \right) + \left[ 1- \frac{\displaystyle\int d^{\mathscrsfs{D}} \vec{x} b(\vec{x},t)}{K} \right] R b \ ,
\label{fokker_planck_logistic}
\end{equation}
in which the integration of population density distribution is the total population size $B(t)=\int d^{\mathscrsfs{D}} \vec{x} b(\vec{x},t)$. We can define the metric for population success as \cite{phan2021it}:
\begin{equation}
S(t) =\frac{B(t)}{K} = \frac{\displaystyle\int d^{\mathscrsfs{D}} \vec{x} b(\vec{x},t)}{K} \ ,
\label{success}
\end{equation}
then the expression for the growth rate is just $G = \left( 1-S \right) R$. This is a valid description at $S\leq 1$, and also exhibits a decrease of not only birth but also death rate at large population. For an example, bacteria such as {\it E.coli} can signal each others via quorum sensing, which can lead to a collective slowdown in metabolic rate at a dense bacterial population \cite{an2014bacterial}. In general, at high cell densities, the rate of cell death may decrease due to various reasons. One factor contributing to decreased cell death is the activation of stress responses and mechanisms that enhance cell survival. Bacteria can sense and respond to stressful conditions, such as nutrient limitation or high cell density, by activating protective mechanisms that increase cell viability and reduce cell death. This adaptive response can help bacteria survive and maintain population stability in crowded environments, which has been observed with bacteria living in biofilms \cite{mooney2018periprosthetic}. Rearrange the terms in Eq. \eqref{fokker_planck_logistic} and define a rescaled time $\tilde{t} = 2Dt$, we can arrive at:
\begin{equation}
-\partial_{\tilde{t}} b = \left( -\frac12 \nabla^2  - \frac{1-S}{2D} R \right) b \ ,
\label{fokker_planck_logistic_rearrange}
\end{equation}
which has the form of a hyperbolic differential equation if the success is treated as a constant.

 In order to describe stress-induced mutagenesis, the effective diffusivity $D$ should not be a constant. This evolutionary regime is a major concern for medical research due to its ability to accelerate the development of drug resistance in pathogenic and neoplastic systems \cite{zhang2011acceleration,wu2014game,li2021acceleration}, while also creating other complications in the treatment of infectious diseases \cite{ram2014stress}. According to the World Health Organization, antibiotic-resistant infections caused an estimated $1.27$ million deaths worldwide in 2019 \cite{murray2022global}. Stress-induced mutagenesis has also been found to have a notable impact on the evolution of the SARS-CoV-2 virus and the emergence of novel variants \cite{kemp2021sars}, highlighting the need for better understanding. One of the most crucial biological inquiries one could ask about stress-induced mutagenesis is why it behaves in the way it does. There are many possible functional-dependence of mutation rate on stress, yet nature somehow seems favor a mutational switch \cite{cirz2005inhibition,bos2015emergence}. To explore this further, we cast this question into the mathematical framework presented by Eq. \eqref{fokker_planck_logistic_rearrange}. To capture the intricacies of stress-induced mutagenesis, we incorporate a heterogeneous effective diffusivity into the landscape. We seek to investigate the theoretical distinctions between the outcomes of these two different possibilities for the diffusivity $D$ as a function of fitness, which is defined as the ability to reproduce -- the growth rate $G$. In the gradual case, a linearity governs:
\begin{equation}
D_\text{gradual}[G] = D^{(0)}_\text{gradual} - D^{(1)}_\text{gradual} G \ ,
\label{gradual_diff}
\end{equation}
whereas the sharp case follows a Heaviside step-function in which the transition happens right at the boundary between the fit and the unfit regions:
\begin{equation}
D_\text{sharp}[G] = D^{(0)}_\text{sharp} + D^{(1)}_\text{sharp} \Theta(-G) \ .
\label{sharp_diff}
\end{equation}
$\Theta(\zeta)$ is the Heaviside function, in which $\Theta(\zeta<0)=0$ and $\Theta(\zeta>0)=1$. Fig. \ref{fig1} serves as a visual representation of the basic postulations underlying our theoretical analysis.

% Figure environment removed

\section{An Analogy to Schr\"{o}dinger-Bloch Equation} 

The analogy between the Fokker-Planck equation as in Eq. \eqref{fokker_planck_logistic_rearrange} and the Schr\"{o}dinger-Bloch equation \cite{schrodinger1926quantisierung,bloch1929quantenmechanik} can be elucidated by considering the following identification identifications. We introduce an imaginary time variable $i\tau$ related to the diffusion coefficient $D$ and the physical time $t$ and an energy potential $V(S,\vec{x})$ related to the population success $S$ maximum growth rate $R(\vec{x})$: 
\begin{equation}
i\tau \leftrightarrow \tilde{t} \ , \ V(S,\vec{x}) \leftrightarrow -\frac{1-S}{2D} R(\vec{x}) \ .
\label{identification}
\end{equation}
If we treat $S$ as a constant parameter, then for Planck constant $\hbar=1$ and mass $m=1$, we can recast Eq. \eqref{fokker_planck_logistic_rearrange} into a form that closely resembles the Schr\"{o}dinger-Bloch equation in imaginary time $\tau$:
\begin{equation}
i\hbar \partial_\tau \Psi = \hat{H} \Psi \ , \ \hat{H} = \frac{\hat{p}^2}{2m} + V(S,\vec{x}) \ , 
\label{Schrodinger_Bloch_imaginary_time}
\end{equation}
where $\hat{p} = - i \hbar \nabla$ is the momentum operator and $\Psi(\vec{x},t) \propto b(\vec{x},t)$ represents the wavefunction of a single quantum mechanical particle of mass $m$ moving in our multi-dimensional landscape. The Hamiltonian operator $\hat{H}$ governs the behavior of this particle under the influence of the energy potential $V(S,\vec{x})$. From here on, we drop $\hbar$ and $m$ out of the analysis.

While this analogy is not exact, as it assumes that success $S$ is unchanging, independent of distribution density $b(\vec{x},t)$, and therefore neglects the influence of the total population number on the potential energy function, it can still provide a powerful framework for understanding the dynamics of evolution. We demonstrate that by looking at the stationary state, where $b=b_\text{st}(\vec{x})$ is an unchanging spatial-function and thus $S=S_\text{st}$ is fixed. We now have an exact correspondence between Eq. \eqref{Schrodinger_Bloch_imaginary_time} and a time-independent Schr\"{o}dinger-Bloch equation associated with $E=0$ eigenstate: 
\begin{equation} \label{Schrodinger_stationary}
0 = \left( -\frac12 \nabla^2  - \frac{1-S_\text{st}}{2D} R \right) b_\text{st} \ \longleftrightarrow \ E \Psi_\text{st} = \left( \frac{\hat{p}^2}{2m} + V(S_\text{st},\vec{x}) \right) \Psi_\text{st} \ .
\end{equation}
Here we obtain a powerful constraint -- the stationary population success $S_\text{st}$ must correspond to an energy potential $V(S_\text{st},\vec{x})$ that has a zero-eigenenery. Moreover, since the population density $b_\text{st}(\vec{x})$ is a non-negative physical field, its associated wavefunction $\Psi_\text{st}(\vec{x})$ should not change sign and cross zero anywhere on the entire landscape (one exception is at impenetrable boundaries, where the wave function is forced to vanish) \cite{van2007fundamental}. This further restriction implies that the wavefunction should also be the ground state of the energy potential $V(S_\text{st},\vec{x})$. Together, we require $V(S_\text{st},\vec{x})$ to be a potential energy function that possesses a ground-state with zero-energy, and $\Psi(\vec{x})$ must be the ground-state wavefunction $\Psi_{\Omega}(\vec{x})$.

This kind of quantum mechanical analogy in biological phenomena has been discovered in other contexts as well. For instance, the very same Schr\"{o}dinger-Bloch equation we consider in our paper also emerges in bacterial chemotaxis \cite{rosen1983theoretical} as well, although only at a very special subset -- but experimentally has been observed in actual bacteria populations -- of the parameter space.

Let us show how to utilize this convenient constraint in practice. Consider an inverse-quadratic maximum growth rate $R(x)$ peaked and centered around the optimal combinations of genetic traits which is chosen to be at $\vec{x}_{op} = 0$:
\begin{equation}
R(\vec{x}) = R_0 \left[ 1 - \left( \frac{\vec{x}}{\lambda} \right)^2 \right]
\label{toy_growth} \ ,
\end{equation}
It means the further away from the origin, the less fit an organism becomes. We call $\vert \vec{x} \vert <\lambda$ the {\it fit region} where $R>0$, and $\vert \vec{x} \vert >\lambda$ the {\it unfit region} where $R<0$ \cite{phan2021it}, as already shown in Fig. \ref{fig1}. Following Eq. \eqref{identification}, this fitness landscape corresponds to a simple harmonic oscillator potential energy $U_{\text{SHO}}(\vec{x})$ up to a shift $U_0$:
\begin{equation}
V(S_\text{st},\vec{x}) = U_{\text{SHO}} (\vec{x}) + U_0 
 \ , \ U_{\text{SHO}} (\vec{x}) = \frac12 \omega^2 \vec{x}^2 \ ,
 \label{SHO_and_shift}
\end{equation}
in which the angular oscillation frequency and the downward shift are:
\begin{equation}
\omega^2 = \frac{1-S_\text{st}}{D\lambda^2} R_0 \ , \ U_0 = -\frac12 \omega^2 \lambda^2 \ .
\label{freq_and_shift}
\end{equation}
The ground-state energy of a $\mathscrsfs{D}$-dimensional oscillator, which corresponding to the purely-quadratic potential $U_{\text{SHO}}(\vec{x})$, is a fundamental result that can be found in pretty much every quantum mechanics textbooks \cite{dirac2001lectures,landau2013quantum,griffiths2018introduction,sakurai1995modern}): 
\begin{equation}
E_{\Omega} = \mathscrsfs{D} \frac{\omega}2 \ .
\label{SHO_groundstate_energy}
\end{equation}
So for it to be $0$ after the energy shift, we need:
\begin{equation}
E = E_{\Omega} + U_0 = 0 \ \Longrightarrow \ \omega = \frac{\mathscrsfs{D}}{\lambda^2} \ , 
\label{set_to_0}
\end{equation}
which directly gives us the stationary population success from Eq. \eqref{freq_and_shift}:
\begin{equation}
S_\text{st} = 1 - \frac{\mathscrsfs{D}^2 D \lambda^2}{R_0} \ .
\end{equation}
If $S_\text{st}<0$, it means there is no sustainable success, and the population eventually goes extinct on such ecological system. We get the stationary population distribution density $b_\text{st}(\vec{x})$, starting from the Gaussian ground-state wavefunction of a simple harmonic oscillator \cite{dirac2001lectures,landau2013quantum,griffiths2018introduction,sakurai1995modern}:
\begin{equation}
b_\text{st}(\vec{x}) \propto \Psi_{\Omega}(\vec{x}) \propto \exp\left( -\frac12 \omega \vec{x}^2 \right) = \exp\left[ -\frac{\mathscrsfs{D}}2 \left( \frac{\vec{x}}{\lambda} \right)^2 \right] \ .
\label{bst_from_PsiOmega}
\end{equation}
Using Eq. \eqref{success}, we can determine the pre-factor and obtain:
\begin{equation}
b_\text{st}(\vec{x}) = \frac{K }{\sqrt{2\pi \lambda^2 / \mathscrsfs{D}}^{\mathscrsfs{D}}} \left( 1 - \frac{\mathscrsfs{D}^2 D \lambda^2}{R_0}\right)\exp\left[ -\frac{\mathscrsfs{D}}2 \left( \frac{\vec{x}}{\lambda} \right)^2 \right] \ .
\label{bst_SHO}
\end{equation}
Similar analytical investigations can be done to extract $S_\text{st}$ and $b_\text{st}(\vec{x})$ from the quantum mechanical ground-state, for any function $R(\vec{x})$ defined on the landscape. 

We can utilize the Rayleigh-Ritz variational method \cite{rayleigh1907dynamical,ritz1909neue} to estimate the upper-bound (and also the Weinstein method \cite{weinstein1934modified,lee1987upper} for the lower-bound) of the population success $S_\text{st}$, in any dimensions. As a demonstration, we consider a class of landscapes with power-law dependency fitness  $R(\vec{x})=R_0 \left[ 1- (|\vec{x}|/\lambda)^\gamma \right]$. The fitness we considered before, given by Eq. \eqref{toy_growth}, belongs to this class and corresponds to the exponent value $\gamma = 2$. For a general value of $\mathcal{D}$ and $\gamma$, the exact solution for the ground-state is not known. But using a Gaussian ansatz-wavefunction we can quickly estimate that:
\begin{equation}
S_{st} \leq 1 - 2^{-\frac{\gamma}{2}} \frac{D \bd \lambda^2}{R_0} \left(\frac{\gamma + 2}{\gamma} \right)^{\frac{\gamma + 2}{\gamma}} \left[\frac{\Gamma(\frac{\bd}{2})}{\Gamma (\frac{\bd + \gamma}{2})} \right]^{-\frac{2}{\gamma}} \ .
\end{equation}
We will carry out in details this estimation with a our choice of ansatz-wavefunction in Appendix \ref{rayleigh_ritz}.

\section{Applying Rayleigh-Schordinger Perturbation Theory to Stress-Induced Mutagenesis}

We investigate the impact of stress-induced mutagenesis on the stationary population size $B_\text{st}$ in both cases, as listed in Eq. \eqref{gradual_diff} and Eq. \eqref{sharp_diff}, for all natural dimensionality $\mathscrsfs{D} \in \mathbb{N}$ of the landscape. Rather than attempting to solve the exact, non-tractable evolution dynamics on the landscape, we adopt a perturbative approach. This enables us to reveal distinctions between the two cases in a tractable manner. We split the Hamiltonian in Eq. \eqref{Schrodinger_Bloch_imaginary_time} into the unperturbed $\hat{H}_0$ and the perturbed $\epsilon \hat{H}_p$. At the stationary state:
\begin{equation}
\hat{H} = \hat{H}_0 + \epsilon \hat{H}_p \ , \ \hat{H}_0 = \frac12 \hat{p}^2 + V(S_\text{st},\vec{x}) \ ,
\label{unperturbed_perturbed}
\end{equation}
where the energy potential $V(S_\text{st},\vec{x})$ is as given in Eq. \eqref{SHO_and_shift} and Eq. \eqref{freq_and_shift}. At the lowest-order of perturbation $\mathscrsfs{O}(\epsilon)$, the correction $\epsilon \delta E_{\Omega}$ to the ground-state energy in Eq. \eqref{SHO_groundstate_energy} can be estimated via Rayleigh-Schr\"{o}dinger perturbation theory even for a non-Hermittian $\hat{H}_p$ \cite{cohen1982rayleigh}: 
\begin{equation}
E_\Omega = \mathscrsfs{D} \frac{\omega}2 + \epsilon \delta E^{(1)}_{\Omega} \ , \ \delta E^{(1)}_{\Omega} = \frac{\displaystyle\int d^{\mathscrsfs{D}}\vec{x} \  \Psi_\Omega^{\dagger}(\vec{x}).\hat{H}_p. \Psi_\Omega(\vec{x})}{\displaystyle\int d^{\mathscrsfs{D}}\vec{x} \  \Psi_\Omega^{\dagger}(\vec{x}) \Psi_\Omega(\vec{x})} \ ,
\label{perturbative_correction}
\end{equation}
where $\Psi_\Omega(\vec{x})$ is the ground-state wavefunction of the unperturbed Hamiltonian as given by Eq. \eqref{bst_from_PsiOmega}. Our analysis can unveil the tendencies with which different manifestations of stress-induced mutagenesis affect the population, either boosting or suppressing success $S_\text{st}$.

\subsection{A Gradual Change}

The diffusivity on the landscape corresponding to a gradual change in mutation rates can be rewritten as:
\begin{equation}
D_\text{gradual} = D(\omega) \left[1 + \epsilon(\omega) \left( \frac{\vert \vec{x} \vert}{\lambda} \right)^2 \right] \ , \ 
\end{equation}
in which we consider the quadratic contribution as perturbation $\epsilon(\omega) \ll 1$ for simplications. Here the functions $D(\omega)$ and $\epsilon(\omega)$, as coming from Eq. \eqref{toy_growth} and Eq. \eqref{gradual_diff}, satisfy:
\begin{equation}
D(\omega) = \tilde{D} \left( 1 - \tilde{\epsilon} \frac{\omega^2}{D(\omega) \lambda^2} \right) \ , \ \epsilon(\omega) = \left( \left[ \tilde{\epsilon} \frac{\omega^2}{D(\omega) \lambda^2} \right]^{-1}-1 \right)^{-1} \ ,
\end{equation}
where we define the constants $\tilde{D}$ and $\tilde{\epsilon}$ from Eq. \eqref{freq_and_shift} to be:
\begin{equation}
\tilde{D} = D^{(0)}_{gradual} \ , \ \tilde{\epsilon} = \frac{D^{(1)}_\text{gradual}}{D_\text{gradual}^{(0)}}  \ .
\end{equation}
In the limit $\tilde{\epsilon} \rightarrow 0$, perturbative expansions in $\epsilon(\omega)$ are applicable.

The perturbed Hamiltonian in Eq. \eqref{unperturbed_perturbed} for this regime of stress-induced mutagenesis is given by a non-Hermitian operator:
\begin{equation}
\hat{H}_p = \frac12 \hat{p}^2 \left( \frac{\vert \vec{x} \vert}{\lambda} \right) ^2 \ .
\label{gradual_perturb_Hamiltonian}
\end{equation}
Applying Eq. \eqref{perturbative_correction}, we obtain:
\begin{equation}
\delta E^{(1)}_{\Omega} = \frac{1}{8\lambda^2} \mathscrsfs{D}\left( \mathscrsfs{D}-2\right) \ .
\end{equation}
The detail of this calculation can be found in Appendix \ref{perturbative_calc_1}.

Applying Eq. \eqref{set_to_0} including the ground-state energy correction:
\begin{equation}
E = E_\Omega + U_0 = \mathscrsfs{D} \frac{\omega}2 + \epsilon(\omega) \frac{1}{8\lambda^2} \mathscrsfs{D}\left( \mathscrsfs{D}-2\right) - \frac12 \omega^2 \lambda^2 = 0\ ,
\end{equation}
we can approximate $\omega$ at the first-order of $\tilde{\epsilon}$-expansion:
\begin{equation}
\omega \approx \frac{\mathscrsfs{D}}{\lambda^2} \left[ 1 + \tilde{\epsilon} \eta(\mathscrsfs{D}) \right] \ , \ \eta(\mathscrsfs{D}) = \frac{\mathscrsfs{D} - 2}{4\mathscrsfs{D}} \ .
\label{eta_gradual}
\end{equation}
For a high dimensionality $\mathscrsfs{D} > 2$, $\eta(\mathscrsfs{D})$ is a positive value. Perturbative stress-induced mutagenesis in this regime increases $\omega \uparrow$. Since Eq. 
\eqref{freq_and_shift} indicates that $\omega$ and $S_{st}$ have  an inverse monotonic relationship, this means we get a reduction in success $S_{st} \downarrow$. In other words, stress-induced mutagenesis tends to suppress the population success when the number of relevant genetic traits on the landscape is high.

The above statement does not change if we consider another power-law dependency for the perturbative Hamiltonian in Eq. \eqref{gradual_perturb_Hamiltonian}, since:
\begin{equation}
\hat{H}_p \propto \hat{p}^2 \vert \vec{x} \vert^\kappa \ \ \Longrightarrow \ \ \eta (\mathscrsfs{D}) \propto \mathscrsfs{D} - \kappa \ . 
\end{equation}
We get $\eta (\mathscrsfs{D}) > 0$ when $\mathscrsfs{D} > \kappa$. The derivation can be found in Appendix \ref{perturbative_calc_2}.

\subsection{A Sharp Change}

We can rewrite Eq. \eqref{sharp_diff}, which described the diffusivity on the landscape associated with a sharp change in mutation rates, as follows:
\begin{equation}
D_\text{gradual} = D\left[1 + \epsilon \Theta \left( \frac{\vert \vec{x} \vert}{\lambda} - 1 \right) \right] \ . \ 
\end{equation}
From Eq. \eqref{toy_growth} and Eq. \eqref{sharp_diff}, we define the constants $D$ and $\epsilon$ to be:
\begin{equation}
D=D^{(0)}_\text{sharp} \ , \ \epsilon = \frac{D^{(1)}_\text{sharp}}{D^{(0)}_\text{sharp}} \ . \ 
\end{equation}
Here we treat the up-step contribution as perturbation $\epsilon \ll 1$, which requires $D^{(0)}_\text{sharp} \gg D^{(1)}_\text{sharp}$, for subsequent calculations to be analytically tractable.

For this posibility of stress-induced mutagenesis, the perturbed Hamiltonian in Eq. \eqref{unperturbed_perturbed} is given by the following non-Hermitian operator:
\begin{equation}
\hat{H}_p = \frac12 \hat{p}^2 \Theta \left( \frac{\vert \vec{x} \vert}{\lambda} - 1 \right) \ ,
\label{Sharp_perturb_Hamiltonian}
\end{equation}
Following Eq. \eqref{perturbative_correction}, we can make the estimation:
\begin{equation}
\delta E^{(1)}_{\Omega} = \frac1{\lambda^2} \frac{(\omega\lambda^2)^{\frac{\mathscrsfs{D}}2 +1} \left[ -2e^{-\omega\lambda^2} + \omega \lambda^2\expint_{-\frac{\mathscrsfs{D}}2} \left( \omega\lambda^2 \right) \right]}{2\Gamma\left( \frac{\mathscrsfs{D}}2 \right)} \ ,
\end{equation}
where we remind that $\omega$ is as given in Eq. \eqref{freq_and_shift}. We carry out the details of this calculation in Appendix \ref{perturbative_calc_3}. We can then determine $\omega$ using Eq. \eqref{set_to_0}:
\begin{equation}
\begin{split}
E & = E_{\Omega} + U_0  
\\
& = \mathscrsfs{D} \frac{\omega}2 + \epsilon \frac1{\lambda^2} \frac{(\omega\lambda^2)^{\frac{\mathscrsfs{D}}2 +1} \left[ -2e^{-\omega\lambda^2} + \omega \lambda^2\expint_{-\frac{\mathscrsfs{D}}2} \left( \omega\lambda^2 \right) \right]}{2\Gamma\left( \frac{\mathscrsfs{D}}2 \right)}   - \frac12 \omega^2 \lambda^2 = 0 \ , 
\end{split}
\end{equation}
in which we obtain the approximate solution:
\begin{equation}
\omega \approx \frac{\mathscrsfs{D}}{\lambda^2} \left[ 1 + \epsilon \eta(\mathscrsfs{D}) \right] \ , \ \eta(\mathscrsfs{D}) =\frac{\mathscrsfs{D}^{\frac{\mathscrsfs{D}}2-1} \left[ -2e^{-\mathscrsfs{D}} + \mathscrsfs{D} \expint_{-\frac{\mathscrsfs{D}}2} \left( \mathscrsfs{D} \right) \right]}{\Gamma\left( \frac{\mathscrsfs{D}}2 \right)}  \ ,
\label{eta_sharp}
\end{equation}
where $\expint_{\ldots}(\ldots)$ is the generalized exponential integral \cite{olver1994generalized}. Since $\eta(\mathscrsfs{D})<0$ for every natural dimensionality $\mathscrsfs{D} \in \mathbb{N}$, this perturbative stress-induced mutagenesis effect always reduces $\omega \downarrow$, thus increases success $S_{st} \uparrow$ as follows from Eq. \eqref{freq_and_shift}. 

The nonperturbative stationary solution of this case has been studied in our previous work \cite{2303.09084}. For a sanity check, one can show that what we have found here using Rayleigh-Schr\"{o}dinger perturbation theory agrees with the exact result at the leading-order of the perturbative parameters $\epsilon$.

\subsection{A Comparison between two Possibilities}

The expression for the total diffusive flux of population on the landscape is given by $J = \nabla(Db)$, which can be further broken down into two contributions: the diffusion-gradient contribution $J_\text{diff} = (\nabla D) b$ which is driven by the local slope of the diffusivity, and the density-gradient contribution $J_\text{dens} = D(\nabla b)$ which is generated by the heterogeneity of population distribution. In the case of a gradually changing $D_\text{gradual}[G]$, the contribution $J_\text{diff}$  exists generally everywhere, pointing towards the optimal combination of genetic traits. This means there is a clear guidance toward peak fitness on the $\mathscrsfs{D}$-dimensional space of genetic variations, focusing the population into a specific hypervolume. In contrast, for a sharp transition $D_\text{sharp}[G]$, this flux vanishes everywhere except at the $(\mathscrsfs{D}-1)$-dimensional boundary hypersurface between fit and unfit regions. Thus, intuitively, we expect that abruptly increasing mutation rates via stress-induced mutagenesis may be less effective in maintaining large stationary population size.

Our application of Rayleigh-Schr\"{o}dinger perturbation theory has quickly revealed a paradoxical outcome that influence evolutionary dynamics in the presence of multiple relevant evolving genetic traits. We interpret this mathematical finding as follows: in the case of gradual stress-induced mutagenesis, the diffusive-gradient flux becomes less effective at population concentration towards the fit region, as it spreads out excessively as the number of landscape dimensions increases. Conversely, sharp stress-induced mutagenesis enables the diffusive-gradient flux to remain spatially focused, even singular, and thus boosts the population success consistently, regardless of the number of genetic traits involved. It has been obsered that natural selection favors species that can optimize multiple biological capabilities simultaneously \cite{endler1986natural,arnold1983morphology}. As a result, the sharp regime of stress-induced mutagenesis may be preferred. Empirical evidence supports this notion \cite{cirz2005inhibition, bos2015emergence}. Here we have shown a quantitative argument for why this might be the case.

\section{Discussion} 

In contrast to physics, where phenomena may arise spontaneously, the complex emergent behavior observed in biology is the product of billions of years of natural selection, which has relentlessly honed and optimized the living systems we see today \cite{darwin2004origin,dawkins2016selfish}. It is therefore essential to focus on understanding why biological phenomena occur, rather than simply how they occur \cite{elsasser2016physical,dawkins1996blind}. 

In this study, we reveal a curious analogy between a fundamental equation in quantum mechanics and the equation that governs the evolution of multiple genetic traits in a population. By utilizing Rayleigh-Schr\"{o}dinger perturbation theory \cite{cohen1982rayleigh}, a standard tool developed by the quantum physics community, we have demonstrated how this analogy can be utilized to answer a biological ``why'' question. Specifically, we have shown quantitative supporting evidence for why stress-induced mutagenesis exhibits a sharp transition rather than a gradual change in mutation rates, which is commonly observed in microbial and cancer cells \cite{cirz2005inhibition,bos2015emergence}. Our findings highlight the significance of our approach as a valuable framework for modeling biological evolution, rather than just a mere mathematical exercise.

 The methodology presented in this paper open up many avenues for future research. Although our study focused on a static ecological system, it is essential to acknowledge that most ecological systems in the real world are highly complex \cite{bhattacharjee2019bacterial,phan2020bacterial} and dynamic in nature \cite{fu2018spatial}. Therefore, one possible direction for future research is to extend our analogy to incorporate the effects of dynamical ecological systems, such as seasonal change or periodic cycle of drug-administration \cite{phan2021it}. Another exciting adventure is to explore the impact of landscape topology. In particular, it would be interesting to investigate whether certain topological features of the fitness landscape can amplify or suppress the effects of stress-induced mutagenesis on population dynamics. Finally, there have been recent attempts to investigate exotic collective behaviors and new sectors of evolutionary dynamics using robots with engineered ecological interactions \cite{wang2021emergent,phan2021bootstrapped} and stress-induced mutable genomes \cite{wang2022robots}, which might allow us to see the realization of our analogy in a physical evolvable system beyond biology. So much to do; the future seems bright and exciting.

\section{Acknowledgement}

We thank Robert H. Austin, Ramzi Khuri, Kenneth J. Pienta, Joel Brown, Emma U. Hammarlund and Sarah R. Amend for the chance to give a talk on this simple but curious finding at Moffit Cancer Center (2021) and many useful discussions followed, which motivated us to share it with a wider audience. We also thank Truong H. Cai, Ramzi Khuri, Thierry Emonet, Henry Mattingly and Lam Vo for insightful comments and supports.

\section{Declarations}

\begin{itemize}
\item Funding: This research received no external funding.
\item Conflict of interest: The authors declare no conflict of interest. 
\item Ethics approval: Not applicable. 
\item Consent to participate: Not applicable.
\item Consent for publication: Not applicable.
\item Availability of data and materials: Not applicable.
\item Code availability: Not applicable. 
\item Authors' contributions: Conceptualization, T.V.P and D.K.T; analytical investigation, D.V.T., V.D.A. and K.T.P.; writing---original draft preparation, D.V.T., V.D.A., K.T.P., D.K.T. and T.V.P.; writing---review and editing, D.V.T., V.D.A., K.T.P., D.M.N., D.K.T. and T.V.P.; visualization, T.V.P.; supervision, H.D.T., V.H.D. and T.V.P.; project administration, T.V.P.. All authors provided critical feedback and helped shape the research, analysis and manuscript. All authors have read and agreed to the published version of the manuscript.
\end{itemize}

\appendix

\section{Application of the Rayleigh-Ritz Variational Method for the Stationary Population Success \label{rayleigh_ritz}}

Refer to Eq. \eqref{identification}, we obtain the potential of:
\begin{equation}
V(S, \vec{x}) = - \frac{(1-S)R_0}{2D} + \frac{(1-S)R_0}{2D \lambda^{\gamma}} \vert \vec{x} \vert ^{\gamma}.
\label{potentail_1}
\end{equation}
It is known earlier that the stationary population success $S_\text{st}$ must correspond to an potential $V(S_st, \vec{x})$ that has a zero-eigenenergy.
It is known that the ground state energy $E_{\Omega}$ satisfied:
\begin{equation}
E_{\Omega} \le \frac{\langle f \vert \hat{H} \vert f \rangle}{\langle f \vert f \rangle}.
\label{Energy_definition}
\end{equation}
with the Hamiltonian operator $\hat{H}$ already defined in Eq. \eqref{Schrodinger_Bloch_imaginary_time} and $f$ is dummy function representing the wavefunction.
Let us try with with Gaussian ansantz $f = e^{-ax^2}$ with one-parameter $a$. Defined $\mathcal{E}(a)$ is the function in which
\begin{equation}
\mathcal{E}(a) \equiv \frac{\displaystyle\int \dd^{\bd} \vec{x} f \hat{H} f}{\displaystyle\int \dd^{\bd} \vec{x} f^2} = \frac{\displaystyle\int \dd^{\bd} \vec{x} e^{-ax^2} \left[\left(- \frac{1}{2} \nabla ^2 + V(x) \right) e^{-ax^2} \right]}{\displaystyle\int \dd^{\bd} \vec{x} e^{-2ax^2}}.
\label{g_definition}
\end{equation}
for a generalized potential
\begin{equation}
V(x) = \Phi x^{\gamma}
\label{pot_definition}
\end{equation}

Expand the space differential $\dd^{\bd} \vert \vec{x} \vert$ and Laplacian in $\bd$-dimension gives:
\begin{equation}
\mathcal{E}(a) = \frac{\displaystyle \int e^{-ax^2} \left[-\frac{1}{2} \frac{1}{x^{\bd - 1}} \frac{\partial}{\partial x} \left(x^{\bd - 1} \frac{\partial e^{-ax^2}}{\partial x}\right) + \Phi x^{\gamma} e^{-ax^2}\right] x^{\bd - 1} \dd x}{\displaystyle\int e^{-2ax^2} x^{\bd - 1} \dd x}.
\end{equation}
The nominator reduced to
\begin{equation}
\begin{split}
\int_0^{\infty} \left[a \bd x^{\bd - 1} - 2a^2 x^{\bd + 1} + \Phi x^{\bd + \gamma - 1}  \right] e^{-2ax^2} \dd x \\
= a \bd \frac{\Gamma (\frac{\bd}{2})}{2 (2a)^{\frac{\bd}{2}}} - 2a^2 \frac{\Gamma(\frac{\bd + 2}{2})}{2 (2a)^{\frac{\bd + 2}{2}}} + \Phi \frac{\Gamma(\frac{\bd + \gamma}{2})}{2 (2a)^{\frac{\bd + \gamma}{2}}}
\end{split}
\end{equation}
The denominator is reduced to
\begin{equation}
\frac{\Gamma(\frac{\bd}{2})}{2 (2a)^{\frac{bd}{2}}}
\end{equation}
Therefore, $\mathcal{E}(a)$ is simplified to
\begin{equation}
\mathcal{E}(a) = \frac{a \bd}{2} + \Phi (2a)^{\frac{\gamma}{2}} \frac{\Gamma (\frac{\bd + \gamma}{2})}{\Gamma(\frac{\bd}{2})}
\end{equation}
A simple analysis would reveal that $g(a)$ reached a minimum of:
\begin{equation}
g_{\mathrm{min}} = \left(\frac{\bd}{\Phi \gamma}\frac{\Gamma(\frac{\bd}{2})}{\Gamma(\frac{\bd + \gamma}{2})} \right) ^{-\frac{2}{\gamma + 2}} \left(\frac{\bd}{2} + \frac{\bd}{\gamma} \right)
\end{equation}

We can conclude that this is the ground state energy for the power part of the potential. For the eigenenergy of $V$ to equals $0$, the following condition must be satisfied:
\begin{equation}
-\frac{(1-S)R_0}{2D} + \left(\frac{2\bd D \lambda^{\gamma}}{\gamma (1-S) R_0} \frac{\Gamma(\frac{\bd}{2})}{\Gamma (\frac{\bd + \gamma}{2})} \right)^{-\frac{2}{\gamma + 2}} \left(\frac{\bd}{2} + \frac{\bd}{\gamma} \right) = 0
\end{equation}

Solving the above equation for $S_{st}$ gives:
\begin{equation}
S_{st} = 1 - 2^{-\frac{\gamma}{2}} \frac{D \bd \lambda^2}{R_0} \left(\frac{\gamma + 2}{\gamma} \right)^{\frac{\gamma + 2}{\gamma}} \left(\frac{\Gamma(\frac{\bd}{2})}{\Gamma (\frac{\bd + \gamma}{2})} \right)^{-\frac{2}{\gamma}}
\end{equation}

\section{Perturbative Corrections}
 
\subsection{With perturbed Hamiltonian contains $x^2$ \label{perturbative_calc_1}}
Following Eq. \eqref{perturbative_correction}, with perturbed Hamiltonian Eq. \eqref{gradual_perturb_Hamiltonian}, we expand the space differential $\dd^{\bd} \Vec{x}$:
\begin{equation}
      \delta E_{\Omega}^{(1)} =\frac{\displaystyle \int_0^{\infty}x^{\bd-1} \dd x  e^{-\frac{1}{2}\omega x^2} \cdot \frac{1}{2}\hat{p}^2\left(\frac{\vert x \vert}{\lambda} \right)^2 \cdot  e^{-\frac{1}{2}\omega x^2}}{\displaystyle \int_0^{\infty} x^{\bd-1} \dd x e^{- \frac{1}{2}\omega x^2} \cdot e^{- \frac{1}{2}\omega x^2}} 
\end{equation}
 With mathematical transformation and do calculation on the numerator, we evaluate the numerator: 
\begin{equation}
    \begin{split}
    -\frac{1}{2 \lambda^2} \int_0^{\infty} x^{\bd-1} e^{-\frac{1}{2}\omega x^2} \frac{1}{x^{\bd-1}} \partial_x[ x^{\bd-1} \partial_x (x^2 e^{-\frac{1}{2}\omega x^2})] \dd x \\
    = \frac{(-2+\bd)\omega^{-\frac{\bd}{2}}\Gamma(1+\frac{\bd
     }{2})}{8\lambda^2}
\end{split}
\end{equation}
And also with the denominator, we get:
\begin{equation}
        \int_0^{\infty}x^{\bd-1} e^{-\omega x^2} \dd x  = \frac{1}{2}\omega^{-\frac{\bd}{2}}{\Gamma\Big(\frac{\bd}{2}\Big)}
\end{equation}
Which  $\delta E_{\Omega}^{(1)}$ is simplified to:
\begin{equation}
        \frac{(-2+\bd)\omega^{-\frac{\bd}{2}}\Gamma(1+\frac{\bd
     }{2})}{8\lambda^2} \frac{2}{\omega^{-\frac{\bd}{2}}{\Gamma(\frac{\bd}{2})}} = \frac{\bd(\bd-2)}{8 \lambda^2}. \label{app1}
\end{equation}


%%%%%%%%%%%%%%%%%%%%%%%
%%%%%%%%%%%%%%%%%%%%%%% 


\subsection{With perturbed Hamiltonian contains $x^\kappa$ \label{perturbative_calc_2}}

In order to evaluate $\delta E_\Omega^{(1)}$ for the $\vert \Vec{x}\vert^\kappa$ non-Hamiltonian operator, we would like to expand Eq. \eqref{perturbative_correction}:

\begin{equation}
      \delta E_{\Omega}^{(1)} =\dfrac{\displaystyle \int_0^\infty x^{\bd - 1} e^{-\frac{1}{2} \omega x^2} \cdot\dfrac{1}{2} \hat{p}^2 \left( \frac{x}{\lambda}\right)^\kappa \cdot e^{-\frac{1}{2} \omega x^2} \dd x}{\displaystyle\int_0^\infty x^{\bd - 1} e^{-\omega x^2} \dd x}.
\end{equation}
After integrating both numerator and denominator part of the above expansion, we have 
\begin{equation}
    \delta E_{\Omega}^{(1)} =\frac{1}{\lambda^2} \frac{\displaystyle(\bd - \kappa ) \omega^{1 - \frac{\kappa}{2}} \Gamma \left(\frac{\bd + \kappa}{2}\right)}{\displaystyle4\Gamma\left( \frac{\bd}{2}\right)}.
\end{equation}
For the sanity check, when we substitute $\kappa = 2$, the result $ \delta E_{\Omega}^{(1)}$ would degenerate to Eq. \eqref{app1}.


%%%%%%%%%%%%%%%%%%%%%%%%
%%%%%%%%%%%%%%%%%%%%%%%%

\subsection{With perturbed Hamiltonian contains Heaviside function \label{perturbative_calc_3}}

In order to evaluate $\delta E_\Omega^{(1)}$ for the non-Hermitian operator containing Heaviside function, which is given in Eq. \eqref{Sharp_perturb_Hamiltonian}, we would like to expand Eq. \eqref{perturbative_correction}:
\begin{equation}
      \delta E_{\Omega}^{(1)} =\dfrac{\displaystyle \int_0^\infty x^{\bd - 1} e^{-\frac{1}{2} \omega x^2} \cdot\dfrac{1}{2} \hat{p}^2 \Theta\left(\frac{\vert\vec{x}\vert}{\lambda} - 1 \right) \cdot e^{-\frac{1}{2} \omega x^2} \dd x}{\displaystyle\int_0^\infty x^{\bd - 1} e^{-\omega x^2} \dd x}.
\end{equation}
After completing numerous mathematical transformation and calculation on each integral part in the numerator and denominator, we have the following result:
\begin{equation}
    \begin{split}
        \delta E_{\Omega}^{(1)} &=\dfrac{-\dfrac{1}{2}\displaystyle \int_0^\infty x^{\bd - 1} e^{-\frac{1}{2} \omega x^2} \dfrac{1}{x^{\bd-1}} \partial_x\Big[x^{\bd-1}\partial_x\Big(\Theta\left(\frac{\vec{x}}{\lambda} - 1 \right) e^{-\frac{1}{2} \omega x^2} \Big)\Big]\dd x}{\displaystyle\int_0^\infty x^{\bd - 1} e^{-\omega x^2} \dd x} \\
        &=\frac{1}{\lambda^2}\dfrac{\displaystyle(\omega \lambda^2)^{\frac{\bd}{2}+1}\left[-2 e^{-\omega \lambda^2}+\omega \lambda^2 \expint_{-\frac{\bd}{2}}\left(\omega \lambda^2\right)\right]}{\displaystyle2\Gamma\left(\frac{\bd}{2}\right)}.
\end{split}
\end{equation}

\phantomsection
\bibliography{main}
\bibliographystyle{unsrt}

\end{document}




