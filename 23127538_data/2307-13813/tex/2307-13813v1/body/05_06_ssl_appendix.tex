\FloatBarrier

\section{Additional details and results for self-supervised image representation learning}
\label{app:ssl}

\paragraph{Organization} 
This appendix is structured into three sections.
We first give an overview of our chosen 
\gls{ssl} method \gls{byol} (\Cref{app:sec-components-ssl}),
our recipe for training \gls{byol} using ResNet 18s (\Cref{subsec:byol-r18}), 
our recipe for training \gls{byol} using \glspl{vit} (\Cref{app:byol-vit}), 
ablations of normalization approaches that lead to the development of this recipe (\Cref{app:byol-vit-ln-vs-bn}), 
and additional results corresponding to longer training duration (\Cref{app:byol-waterfall}) and further understanding the impact of Progressive Scaling (\Cref{app:byol-progressive-scaling-regimes}).

Second, we demonstrate that the \gls{ema} Scaling Rule combined with Progressive Scaling can scale a ResNet-50 BYOL model trained with LARS to batch size 32,768 without performance drop, demonstrating the empirical utility of the tools we provide outside of their theoretical validity (\Cref{subsec:byol-additional}).

Finally, we show that it is possible to systematically scale DINO \citep{DBLP:journals/corr/abs-2104-14294} using a combination of Progressive Scaling and the \gls{ema} Scaling Rule,
providing a solution for researchers and practitioners wanting to train DINO at scale.

\subsection{Components of self-supervised learning}
\label{app:sec-components-ssl}

First, a key component of many \gls{ssl} methods is the \emph{stop-gradient} 
or $\text{StopGrad}$ (\Cref{def:stop-grad}).
\begin{definition}[Stop Gradient/$\text{StopGrad}(\,\cdot\,)$] 
     The \emph{stop-gradient} operator $\text{StopGrad}(\,\cdot\,)$ prevents the flow of gradient information
    \begin{equation}
    \frac{df(\text{StopGrad}(h(x ; \rvomega)); \rvtheta)}{d\rvomega}\equiv 0
    \end{equation}
    \label{def:stop-grad}
    for all parameteric functions $h$ and $f$ and for all parameters $\rvtheta$ and $\rvomega$.
\end{definition}
Applying a \emph{stop-gradient} is sometimes called \emph{detaching} in the literature. 
Now, we 
introduce the update rule of our representative \gls{ssl} method \gls{byol} in \Cref{def:byol}.
\begin{definition}[BYOL Update] 
     \gls{byol} learns unsupervised features by minimizing the cosine distance between the predictions of a student backbone
     $f(\,\cdot\,;\rvtheta)$ (typically a ResNet or Vision Transformer),
     projected through 
     $h(\,\cdot\,;\rvomega$) (typically a \gls{mlp}), and the predictions of an \gls{ema} teacher $f(\,\cdot\,;\rvzeta)$ \citep{DBLP:conf/nips/GrillSATRBDPGAP20}. 
     The update for the parameters of \gls{byol} is then
    \begin{align}
    (\rvtheta_{t+1},\rvomega_{t+1})
    &=
    (\rvtheta_t,\rvomega_t) - \eta \times \frac1B
    \sum_{x\in\sB} \nabla_{(\rvtheta,\rvomega)} \Ls(x;\rvtheta_{t},\rvomega_{t},\rvzeta_{t})\\
    \rvzeta_{t+1}
    &=
    \rho \,\rvzeta_t + (1-\rho)\,\rvtheta_{t+1} \label{eq:byol-ema-update}\\
    \text{with} \;\; 
    \Ls(x;\rvtheta_{t},\rvomega_{t},\rvzeta_{t})
    &=
    \frac12
    \cos \big[ h( f(x_1;\rvtheta_t);\rvomega_t), \text{StopGrad}(f(x_2;\rvzeta_t))  \big]  + (x_1\leftrightarrow x_2),
    \end{align}
    where $\cos (\va,\vb)\equiv 1-\va\cdot \vb/(||\va||\,||\vb||)$ is the cosine distance, 
    and
    $x_1$ and $x_2$ are two views of a single variate $x$, often produced by augmentations,
    and
    $x_1\leftrightarrow x_2$ denotes symmetrization over $x_1$ and $x_2$.
    \label{def:byol}
\end{definition}
As noted in \Cref{subsec:self-supervised},the  \gls{byol} \gls{ema} update (\Cref{eq:byol-ema-update}) uses $\rvtheta_{t+1}$ instead of our analyzed $\rvtheta_{t}$ (\Cref{eq:scalingRuleSummaryEquation}).
The effect upon the overall \gls{ema} update is $\mathcal{O}(\eta\times\beta_\rho)$ and so is captured by the \gls{ema} Scaling Rule (\Cref{def:ema-sr}).

One more piece of technology typically employed in \gls{ssl} is a \emph{tracking probe} (\Cref{def:linear-probe}) which we will use to evaluate the performance of \gls{byol} on downstream tasks of interest, for example, image classification.
\begin{definition}[Tracking Probe/Linear Probe] 
    When optimizing model parameters $\rvomega_t$ of an \gls{ssl} method, 
    simultaneously optimize the parameters $\rvxi$ of a probe model $r(\,\cdot\,;\rvxi)$
    under a downstream objective $\Ls^{(d)}$.
    For example, in classification, with data $x$ and samples $y$
    \begin{align}
    \Ls^{(d)}(x,y,\rvtheta_{t},\rvxi_t)&= - \log P(y|r(\text{StopGrad}(h(x;\rvomega_t));\rvxi))\\
    \Ls^{(\text{total})}(x,y;\rvtheta_{t},\rvomega_{t},\rvzeta_{t},\rvxi_t) & = \Ls(x;\rvtheta_{t},\rvomega_{t},\rvzeta_{t})+\Ls^{(d)}(x,y,\rvomega_{t},\rvxi_t),
    \end{align}
    The is a probe for the teacher, which is typically the better choice due to Polyak-Ruppert averaging effects (see \Cref{subsec:supervised-polyakking}).
    \label{def:linear-probe}
    When the $r$ is a linear model, the tracking probe is called a linear probe.
\end{definition}
It is also typical to use a Batch Normalization layer \emph{without} trainable affine terms before this linear layer as in \citet{DBLP:conf/cvpr/HeCXLDG22} to stabilize probe training.
In this case, the running statistics can be absorbed into a definition of the linear layer weights and biases, and so this is still a \emph{linear probe}, although we will call this a \emph{pre-bn linear probe} to remove ambiguity.

\subsection{A ResNet18 recipe for BYOL}
\label{subsec:byol-r18}

\paragraph{Hyperparameters} We present the base hyperparameters for training \gls{byol} with a ResNet 18 backbone using \gls{sgd} in \Cref{tab:byol-r18-recipe}.
This recipe was developed by starting from a well-known BYOL ResNet 50 recipe \citep{DBLP:conf/nips/GrillSATRBDPGAP20}, adapting the input augmentations for CIFAR10, and performing a search over learning rate choices for an SGD optimizer.

\compute

\begin{table}[t]
  \caption{BYOL ResNet 18 hyperparameters for CIFAR10}
  \label{tab:byol-r18-recipe}
  \centering
  \small
  \begin{tabular}{lc}
    \toprule
    & ResNet 18 \\
    \midrule
    Weight initialization & \texttt{kaiming\_uniform} \citep{DBLP:conf/iccv/HeZRS15} \\
    Backbone normalization    & BatchNorm  \\
    Head normalization    & BatchNorm  \\
    Synchronized BatchNorm over replicas & Yes \\     
    Learning rate schedule & Single Cycle Cosine \\    
    Learning rate warmup (epochs) & 20 \\    
    Learning rate minimum value & $0$  \\    
    Training duration (epochs) & 100 \\
    Optimizer & SGD \\    
    Optimizer scaling rule & SGD \\
    Optimizer momentum & $0.9$ \\
    Gradient clipping & $0.1$ \\
    Base learning rate & $0.02$  \\
    Base batch size & $1024$  \\
    Base teacher momentum & $0.992$ \\    
    Weight decay & $1\times 10^{-6}$ \\
    Weight decay scaling rule & None \\
    Weight decay skip bias & Yes \\
    Numerical precision & \texttt{tf32} \\
    Augmentation stack & \texttt{BYOL CIFAR10} \\   
    \bottomrule
  \end{tabular}
\end{table}

\FloatBarrier

\subsection{A Vision Transformer recipe for BYOL}
\label{app:byol-vit}

\paragraph{Hyperparameters} We present the base hyperparameters for training \gls{byol} with a ViT-B/16 backbone in \Cref{tab:byol-recipe}.
This recipe was developed by starting from a well-known supervised ViT-B/16 recipe \citep{DBLP:conf/cvpr/HeCXLDG22} and performing a search over weight decay and learning rate hyperparameter choices. 
We find that \gls{byol} performs well with heavy weight decay ($\lambda=0.3$)
and a low learning rate
($\eta=10^{-3}$)
at a base batch size $B=4096$.
The AdamW optimizer
is used, and so for scaling to other batch sizes $\hat B=\kappa B$ we use the Adam Scaling Rule (\Cref{def:adam-sr})\footnote{We note that Adam \citep{DBLP:journals/corr/KingmaB14} and AdamW  \citep{DBLP:conf/iclr/LoshchilovH19} are equivalent in the limit of zero weight decay, and that
the Adam Scaling Rule (\Cref{def:adam-sr}) was derived with zero weight decay \citep{DBLP:conf/nips/MalladiLPA22}.
}
We use a pre-bn linear probe as discussed in \Cref{app:sec-components-ssl}.
Finally, the performance of \gls{byol} can be further improved by employing multicrop \citep{DBLP:conf/nips/CaronMMGBJ20} by $\approx$ +2\% in absolute test top-1 performance on ImageNet1k compared to without multicrop, however, as this is not our focus, we omit this from the presented recipe.

\compute

\begin{table}[t]
  \caption{BYOL ViT-B/16 hyperparameters.}
  \label{tab:byol-recipe}
  \centering
  \small
  \begin{tabular}{lc}
    \toprule
    & BYOL ViT-B/16 \\
    \midrule
    ImageNet1k Linear Probe Test Top-1 & 74.47\% (\Cref{fig:vitb-byol-ln-vs-bn}) \\
    \midrule
    Weight initialization & \texttt{trunc\_normal(.02)}  \\
    Backbone normalization    & LayerNorm  \\
    Head normalization    & BatchNorm  \\
    Synchronized BatchNorm over replicas & No \\     
    Learning rate schedule & Single Cycle Cosine \\    
    Learning rate warmup (epochs) & 40 \\    
    Learning rate minimum value & $1\times 10^{-6}$  \\    
    Training duration (epochs) & 480 \\
    Optimizer & AdamW \\    
    Optimizer scaling rule & Adam \\
    Base ($\beta_1, \beta_2$) & (0.9, 0.95) \\
    Base learning rate & $1\times 10^{-3}$  \\
    Base batch size & 4096  \\
    Base teacher momentum & 0.99 \\    
    Weight decay & 0.3 \\
    Weight decay scaling rule & None \\
    Weight decay skip bias & Yes \\
    Numerical precision & \texttt{bf16} \\
    Augmentation stack & 
    \texttt{BYOL} \citep{DBLP:conf/nips/GrillSATRBDPGAP20} \\   
    Stochastic depth & 0.1 \\
    \bottomrule
  \end{tabular}
\end{table}

\paragraph{Additional background} 
Achieving large scale \gls{ssl} training with \glspl{vit} to large scale \gls{ssl} training has been a long standing goal in the community. 
MoCo-v3 \citep{DBLP:conf/iccv/ChenXH21} enables the use of \gls{vit}s with contrastive learning, but achieves this through modificatinos of the \gls{vit} training procedures, including gradient freezing on the image patching layer, and re-introducing Batch Normalization to post-attention MLP layers.
Despite these modifications, MoCo-v3 was only trained up to a batch size of 6144, where model performance begins to suffer \citep{DBLP:conf/iccv/ChenXH21}. 
In \Cref{fig:vitb-byol} we demonstrate that combining dynamic batch scaling (\Cref{subsec:dynamic-batch-scaling}) with the \gls{ema} Scaling Rule (\Cref{def:ema-sr}) enables \gls{byol} to be trained using \glspl{vit} to batch sizes of 24,576 without any drop in performance compared to the reference batch size of 4096. 
We emphasize that the piecewise transitions in the schedules are important for preserving training dynamics.

\subsection{The role of Batch Normalization and Layer Normalization in BYOL with ViTs}
\label{app:byol-vit-ln-vs-bn}
% Figure environment removed

Here we compare the roles of Batch Normalization (BatchNorm, \cite{DBLP:conf/icml/IoffeS15}) and  Layer Normalization (LayerNorm, \cite{DBLP:journals/corr/BaKH16})
in the projection and prediction heads of \gls{byol} \citep{DBLP:conf/nips/GrillSATRBDPGAP20} using \glspl{vit}.

It has been observed that BatchNorm 
plays a critical role in \gls{byol} predictor and projector dynamics \citep{Fetterman_Albrecht_2020}, 
and using either LayerNorm or \emph{no normalization} significantly decrease in model performance.
Subsequently, it was demonstrated \citep{DBLP:journals/corr/abs-2010-10241} that competitive \gls{byol} performance could be achieved through a combination of Group Normalization (GroupNorm, \cite{DBLP:conf/eccv/WuH18}) and Weight Standardization \citep{DBLP:journals/corr/abs-1903-10520}.
Additionally, \citet{DBLP:journals/corr/abs-2010-10241} showed that if BatchNorm is used in the backbone, one can use LayerNorm or \emph{no normalization} in the predictor and projector without any performance drop.

In this work, we we show it is possible to train \gls{byol} \gls{vit} using \emph{only LayerNorm} across the backbone, projector and predictor (see \Cref{fig:vitb-byol-ln-vs-bn}),
decoupling \gls{byol}'s reliance on batch statistics, a desirable trait for a representation learning algorithm \citep{DBLP:conf/icml/BrockDSS21}. 
At batch size 3072, using LayerNorm in the predictor and projector achieves competitive performance (74.10\%), performing slightly worse than using BatchNorm (74.47\%). 
At the larger batch size of 24,576, runs perform significantly worse as the \gls{ema} Scaling Rule was not applied.

\subsection{Longer training duration with incremental Progressive Scaling}
\label{app:byol-waterfall}
% Figure environment removed

Here we use the same base hyperparameters as \Cref{tab:byol-recipe}, except that we train for 480 instead of 300 epochs.
To mitigate the student impulse phenomena discussed in \Cref{subsec:self-supervised}, in \Cref{fig:vitb-byol-waterfall} we investigate increasing the batch size every 60 epochs using Progressive Scaling  (\Cref{def:progressive-scaling}).
We observe that this more gradual procedure enables closer tracking of the baseline train loss trajectory.
Additionally, this procedure results in a scaled linear probe performance that outperforms the baseline (75.64\% compared to the baseline performance of 74.47\%).
The same procedure can be applied to the LayerNorm variant discussed in \Cref{app:byol-vit-ln-vs-bn}, which produces a similar result (75.09\% compared to the baseline performance of 74.10\%).

\subsection{Building intuition around Progressive Scaling and momentum sensitivity}
\label{app:byol-progressive-scaling-regimes}
% Figure environment removed
Our final \gls{byol} \gls{vit} results are to help build intuition around Progressive Scaling (\Cref{def:progressive-scaling}),
as well as when the \gls{ema} Scaling Rule is most important.
In \Cref{fig:vitb-byol-rho-ablations} we explore transitioning from the baseline batch size 4096 model to batch size 24,576 in a \emph{single transition} after 60 epochs.
After this transition, we continue training for 240 epochs for a range of momenta: $\rho \in \{0.8, 0.9, 0.95, 0.97, 0.9867, 0.994, 0.999\}$ \emph{without} the EMA Scaling Rule. 

We observe that after the transition, any $0.9\leq \rho\leq0.994$ produces a linear probe performance that matches or outperforms the baseline at the end of training. 
This indicates that after the initial training period, BYOL becomes less sensitive to the choice of teacher momentum. 
Note that without the initial 60 epochs of training with batch size 4096, \emph{all models}, including those employing the \gls{ema} Scaling Rule diverge (see $B=24,576$ in \Cref{fig:vitb-byol}).

We present an illustration for why this might happen in \Cref{fig:robustness-cartoon}.
First, we see that using the \gls{ema} Scaling Rule \emph{always} keeps the model within the acceptable momentum region.
We also wee that \emph{not} using the \gls{ema} Scaling Rule can keep the model  within the acceptable momentum region for a range of batch sizes, depending on how large wide in momenta the acceptable region is at the base batch size.
Finally, we see that the momentum value matters much more at low values of momenta (the acceptable momentum region shrinks), whereas at large momenta, this region of acceptability widens.

% Figure environment removed

\ifthenelse{\equal{\anonymous}{0}}{
\subsection{Compute usage for ViT BYOL investigation}

We now summarize the compute usage for the BYOL 
\gls{vit} experiments.
First we detail cost of reproducing 
\Cref{fig:vitb-byol} in \Cref{tab:byol-vit-compute},
\Cref{fig:vitb-byol-ln-vs-bn} in \Cref{tab:byol-vit-compute-2},
and \Cref{fig:vitb-byol-waterfall} in \Cref{tab:byol-vit-compute-3}.

\begin{table}[ht]
  \caption{
  Compute usage for baseline ViT BYOL investigation in \Cref{fig:vitb-byol}. 
  Values \emph{include} node allocation times (typically a small \% of corresponding total runtime), giving a practical   estimate of reproduction cost. 
  All experiments conducted are using 80Gb A100s, and experiments indicated by $(\dagger)$ have a faster interconnect.}
  \label{tab:byol-vit-compute}
  \centering
  \small
  \begin{tabular}{ccccccc}
\toprule
 Batch Size &  GPUs &  Time (h) &  Compute/Run (GPUh) &  Runs &  Compute (GPUh) \\
\midrule
       4,096 &      32 &          16.6 &               531.4 &       2 &          1,062.7 \\
       8,192 &      48 &          14.1 &               678.0 &       2 &          1,356.1 \\
      16,384 &      96 &           8.3 &               800.3 &       2 &          1,600.6 \\
      24,576 &     128 &           6.6 &               850.1 &       4 &          3,400.4 \\
      32,768${}^{\dagger}$ &     176 &           4.1 &               721.7 &       2 &          1,443.4 \\ \midrule 
      \textbf{Total} &&&&& \textbf{8,863.1} \\
\bottomrule
  \end{tabular}
\vspace{-0.2cm}
\end{table}

\begin{table}[ht]
  \caption{
  Compute usage for ViT BYOL investigation into BatchNorm and LayerNorm variants in \Cref{fig:vitb-byol-ln-vs-bn}.
  Values \emph{include} node allocation times (typically a small \% of corresponding total runtime), giving a practical   estimate of reproduction cost. 
  All experiments conducted are using 80Gb A100s, and were run for 480 epochs, except those indicated by $(*)$ were truncated early (see \Cref{fig:vitb-byol-ln-vs-bn} for more details).}
  \label{tab:byol-vit-compute-2}
  \centering
  \small
\begin{tabular}{cccccc}
\toprule
 Batch Size & Normalization &  GPUs &  Time (h) &  Compute (GPUh) \\
\midrule
       3,072 &     BatchNorm &    16 &      47.9 &           766.0 \\
       3,072 &     LayerNorm &    16 &      48.0 &           768.7 \\
      24,576 &     BatchNorm &   128 &      14.8 &          1900.4 \\
      24,576${}^{*}$ &     LayerNorm &   128 &       3.5 &           451.1 \\ \midrule
      \textbf{Total} &&&& \textbf{3,886.2} \\
\bottomrule
\end{tabular}
\vspace{-0.2cm}
\end{table}

\begin{table}[ht]
  \caption{
  Compute usage for ViT BYOL investigation into incremental scaling in \Cref{fig:vitb-byol-waterfall}.
  Values \emph{include} node allocation times (typically a small \% of corresponding total runtime), giving a practical   estimate of reproduction cost. 
  All experiments conducted are using 80Gb A100s for 60 epochs. Stage 0 corresponding to the baseline in \Cref{fig:vitb-byol-waterfall} is the run detailed in the first row of \Cref{tab:byol-vit-compute-2}, using a batch size of 3,072, Batch Normalization, and 16 GPUs. Computing only the first 60 epochs of stage 0 corresponds to approximately 127.7 GPUh, which would bring the total cost of \Cref{fig:vitb-byol-waterfall} to 1,432.9 GPUh.}
  \label{tab:byol-vit-compute-3}
  \centering
  \small
\begin{tabular}{cccccc}
\toprule
 Stage &  Batch Size &    GPUs &  Time (h) &  Compute (GPUh) \\
\midrule
     1 &        6,144 &        32 &       3.5 &           113.0 \\
     2 &        9,216 &          48 &       3.1 &           149.8 \\
     3 &       12,288 &          64 &       2.8 &           176.0 \\
     4 &       15,360 &          80 &       2.3 &           186.5 \\
     5 &       18,432 &          96 &       2.1 &           202.9 \\
     6 &       21,504 &         112 &       2.1 &           235.8 \\
     7 &       24,576 &         128 &       1.9 &           241.3 \\ \midrule 
     \textbf{Total} &&&& \textbf{1,305.2} \\
\bottomrule
\end{tabular}
\vspace{-0.2cm}
\end{table}

\FloatBarrier

Next, the cost of a single momentum ablation presented in \Cref{fig:vitb-byol-rho-ablations}
is 240 epochs at batch size 24,576, which is $\approx 240/480 \times 1900.4\text{ GPUh}=950.2\text{ GPUh}$, giving a total cost over seven runs of $\mathbf{6651.4}\text{ GPUh}$.

Finally, providing a full view of the investigations carried out for the \gls{vit} \gls{byol} is given in \Cref{tab:byol-vit-compute-summary}.

\begin{table}[ht]
  \caption{
  Total compute usage for ViT BYOL investigations.}
  \label{tab:byol-vit-compute-summary}
  \centering
  \small
\begin{tabular}{lc}
\toprule
 &  Compute (GPUh) \\
\midrule
Baselines (\Cref{fig:vitb-byol} and \Cref{tab:byol-vit-compute}) & 8,863.1 \\
BatchNorm and LayerNorm (\Cref{fig:vitb-byol-ln-vs-bn} and \Cref{tab:byol-vit-compute-2}) & 3,886.2 \\
Incremental scaling (\Cref{fig:vitb-byol-waterfall} and \Cref{tab:byol-vit-compute-3}) & 1,305.2 \\
Momentum ablations (\Cref{fig:vitb-byol-rho-ablations}) & 6,651.4 \\
All other compute, e.g. code development, runs with errors, and debugging & 84,984.1 \\
\midrule 
     \textbf{Total} & \textbf{105,690.0} \\
\bottomrule
\end{tabular}
\vspace{-0.2cm}
\end{table}

}{\compute}


\FloatBarrier
\subsection{Scaling a ResNet-50 BYOL using LARS and Progressive Scaling}
\label{subsec:byol-additional}

Here we investigate whether Progressive Scaling and the \gls{ema} Scaling Rule can be used in practice where there is no known optimizer \gls{sde} approximation.
We use the default 300 epoch configuration for \gls{byol} \citep{DBLP:conf/nips/GrillSATRBDPGAP20} in \Cref{fig:r50-byol}.
We see that although trajectories during training do not match, we are able to match or surpass the linear probe performance of the \gls{byol} baseline at the larger batch size if 32,768.
\emph{This indicates that the contributions of our work have practical utility beyond the theoretical constraints.}

% Figure environment removed

\ifthenelse{\equal{\anonymous}{0}}{
\FloatBarrier

The compute usage for the BYOL ResNet using LARS is detailed in  
\Cref{tab:byol-lars-compute}.

\begin{table}[ht]
  \caption{
  Compute usage for ResNet 50 LARS investigation in \Cref{fig:r50-byol}. 
  Values \emph{include} node allocation times (typically a small \% of corresponding total runtime), giving a practical estimate of reproduction cost. 
  All experiments conducted are using 80Gb A100s.}
  \label{tab:byol-lars-compute}
  \centering
  \small
\begin{tabular}{cccc}
\toprule
                              Batch Size &  GPUs &  Time (h) &  Compute (GPUh) \\
\midrule
$4,096\rightarrow 32,768$ (120 Epochs) &   128 &      14.1 &          1809.8 \\
 $4,096\rightarrow 32,768$ (60 Epochs) &   128 &      12.9 &          1655.9 \\
                               $4,096$ &    16 &      32.8 &           524.9 \\ \midrule 
\multicolumn{3}{l}{All other compute, e.g. code development, runs with errors, and debugging} & 17,654.6 \\ \midrule 
      \textbf{Total} &&& \textbf{21645.2} \\
\bottomrule
  \end{tabular}
\end{table}

}{\compute}



\FloatBarrier
\subsection{Preventing collapse phenomena in DINO at scale}
\label{subsec:dino}

Until now, our representatives \gls{ssl} method has been \gls{byol} for reasons discussed in \Cref{subsec:self-supervised}.
Here, we will turn our attention to \gls{dino} \citep{DBLP:journals/corr/abs-2104-14294},
which has the update rule presented in \Cref{def:dino}.
\begin{definition}[DINO Update] 
     \gls{dino} learns unsupervised features by 
     matching predictions over emergent pseudo-labels of a student backbone and head $f(\,\cdot\,;\rvtheta)$ 
     to those of an \gls{ema} teacher $f(\,\cdot\,;\rvzeta)$ through a 
     cross-entropy guided distillation procedure.
    \gls{dino} has a additional centering procedure, which is a form of batch normalization with momentum $\rho_c=0.9$ which we do not scale using the \gls{ema} Scaling Rule. 
    The update for the parameters of \gls{dino} is
    \begin{align}
    \rvtheta_{t+1}
    &=
    \rvtheta_t - \eta \times \frac1B
    \sum_{x\in\sB} \nabla_{\rvtheta} \Ls(x;\rvtheta_{t},\rvzeta_{t}, \rvc_t)
    \\
    \rvzeta_{t+1}
    &=
    \rho \,\rvzeta_t + (1-\rho)\,\rvtheta_{t+1}
    \\
    \rvc_{t+1}
    &= 
    \rho_{c} \, \rvc_t + (1 - \rho_{c}) \,\E_{x^\prime} \rvzeta(x^\prime) \\
    \text{with} \;\; \Ls(x;\rvtheta_{t},\rvzeta_{t}, \rvc_t)
    &=
    H\big( f(x_1, \rvtheta_{t}), f(x_2, \rvzeta_{t}) - \rvc_{t} \big)  + (x_1\leftrightarrow x_2),
    \end{align}
    where $H(\va,\vb)\equiv - \sum_{m=1}^M p_m(\va)\,\log p_m(\vb)$
    is the cross-entropy between categorical distributions over $M$ (emergent pseudo-)classes given logits $\va,\vb\in\R^{M}$, 
    $x_1$ and $x_2$ are two views of a single variate $x$, often produced by augmentations,
    and $x_1\leftrightarrow x_2$ denotes symmetrization over $x_1$ and $x_2$.
    \label{def:dino}
\end{definition}
In practice, \gls{dino} employs multi-crop \citep{DBLP:journals/corr/abs-2104-14294}.
We omit this detail for clarity of presentation, although we \emph{do} use multi-crop in the experiments that follow.

Our interest \gls{dino} is due to the  difficulty in its optimization\footnote{For an example, see
\url{https://github.com/facebookresearch/dino/issues/43\#issuecomment-881453515}.}, and in particular, preventing collapse phenomena in \gls{dino} at batch sizes above 1024, which is an open research problem.
In this section, we will show that a combination of the \gls{ema} Scaling Rule (\Cref{def:ema-sr}) and 
Progressive Scaling (\Cref{def:progressive-scaling}) enable training of \gls{dino} beyond batch size 1024 without sacrificing performance.

\paragraph{Hyperparameters} Base hyperparameters are presented in \Cref{tab:dino-hp}.

\compute

\begin{table}[t]
  \caption{\gls{dino} ViT-B/16 Training hyperparameters.}
  \label{tab:dino-hp}
  \centering
  \small
  \begin{tabular}{lcccc}
    \toprule
    & \gls{dino} ViT-B/16 \\
    \midrule
    CIFAR10 Linear Probe Top-1 ($\rho_B=0.996$) & 85.38\%  \\
    CIFAR10 Linear Probe Top-1 ($\rho_B=0.992$) & 86.96\%  \\
    \midrule
    Weight initialization & \texttt{trunc\_normal(.02)}  \\
    Normalization    & Layer Norm  \\
    Learning rate schedule & Single Cycle Cosine \\    
    Learning rate warmup (epochs) & 50 \\    
    Learning rate minimum value & $1\times10^{-6}$ \\        
    Training duration (epochs) & 280 \\    
    Optimizer & AdamW \\    
    Optimizer scaling rule & Adam \\
    Base ($\beta_1$, $\beta_2$) & (0.9, 0.95) \\
    Base learning rate & $3\times 10^{-4}$  \\
    Base batch size ($B$) & 1024  \\
    Base teacher momentum ($\rho_B$) & 0.992 or 0.996  \\
    Base weight decay & 0.04 \\
    Weight decay scaling rule & Linear \\
    Weight decay skip bias & Yes \\
    Center Momentum & 0.9 \\
    Center Momentum Scaling Rule & None \\
    Precision & \texttt{bf16} \\
    Augmentation stack & \texttt{DINO multi-crop} \citep{DBLP:conf/nips/CaronMMGBJ20} \\   
    \bottomrule
  \end{tabular}
\end{table}

\paragraph{Results}
In \Cref{fig:dino-cifar10-0.996,fig:dino-cifar10-0.992} we show the results obtained training \gls{dino} on CIFAR-10 with $\rho_B=0.996$ and $\rho_B=0.992$ respectively at the reference batch size of 1024. 
We employ smooth Progressive Scaling (\Cref{def:progressive-scaling}) between epochs 120 and 180.

At batch size 2048, the training loss matches the reference \emph{only} when the \gls{ema} Scaling Rule is applied, whereas the run \emph{without} the scaling rule diverges from the reference. 
The impact of this divergence is emphasized as we consider the larger batch size of 4096.
Here. there is also a gap \emph{with} the \gls{ema} Scaling Rule, however is approximately three times smaller than the gap \emph{without} the \gls{ema} Scaling Rule.

Additionally, we observe that using $\rho_B=0.992$ yields higher Top-1 accuracy over $\rho_B=0.996$, and in our experiments, using the \gls{ema} Scaling Rule \emph{always} performs better in terms of linear probe performance than not using the scaling rule.

% Figure environment removed

% Figure environment removed

\FloatBarrier

In \Cref{fig:dino-cifar10-0.996-hp} we show how the hyperparameters $\rho$, $B$ and learning rate change with the progressive scaling in \Cref{def:progressive-scaling}.

% Figure environment removed

\FloatBarrier
We also attempted to use a sharp batch size transition (\Cref{fig:dino-cifar10-0.992-step,fig:dino-cifar10-0.992-hp-step}), which leads to the collapse pheonomena observed in prior work. 
This collapse happens with and without the \gls{ema} Scaling Rule.
We suspect this is due to dynamics specific to \gls{dino}'s early phase that are even more challenging to replicate under discretization than those of \gls{byol}.

% Figure environment removed

% Figure environment removed

\ifthenelse{\equal{\anonymous}{0}}{
\FloatBarrier

\paragraph{Compute} The compute usage for the DINO investigations is detailed in 
\Cref{tab:dino-compute}.

\begin{table}[ht]
  \caption{
  Compute usage for DINO investigations. 
  Values \emph{include} node allocation times (typically a small \% of corresponding total runtime), giving a practical estimate of reproduction cost. 
  All experiments conducted are using 80Gb A100s.}
  \label{tab:dino-compute}
  \centering
  \small
\begin{tabular}{cccccc}
\toprule
                             Batch Size &  GPUs &  Time (h) &  Compute/Run (GPUh) &  Runs &  Compute (GPUh) \\
\midrule
                              $1,024$ &    24 &       6.8 &               163.5 &     2 &           327.0 \\
             $1,024\rightarrow 2,048$ &    40 &       4.6 &               182.4 &     1 &           182.4 \\
             $1,024\rightarrow 3,072$ &    48 &       4.0 &               189.9 &     1 &           189.9 \\
             $1,024\rightarrow 4,096$ &    64 &       3.3 &               212.3 &     1 &           212.3 \\
$1,024\rightarrow 2,048$ (100 Epochs) &    40 &       4.8 &               190.6 &     4 &           762.3 \\
$1,024\rightarrow 3,072$ (100 Epochs) &    48 &       4.0 &               192.5 &     4 &           769.9 \\
$1,024\rightarrow 4,096$ (100 Epochs) &    64 &       3.6 &               232.1 &     4 &           928.2 \\ \midrule 
\multicolumn{5}{l}{All other compute, e.g. code development, runs with errors, and debugging} & 38239.2 \\ \midrule 
      \textbf{Total} &&&&& \textbf{41,611.1} \\
\bottomrule
  \end{tabular}
\end{table}

}{\compute}


Our results in this section show it is possible to scale \gls{dino} to large batch sizes \emph{without} sacrificing performance by using \emph{both} the \gls{ema} Scaling Rule and Progressive Scaling, providing the batch size schedule of Progressive Scaling is not sudden. 

