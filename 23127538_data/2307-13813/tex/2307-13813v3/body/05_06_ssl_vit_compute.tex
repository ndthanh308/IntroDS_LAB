\subsection{Compute usage for ViT BYOL investigation}

We now summarize the compute usage for the BYOL 
\gls{vit} experiments.
First we detail cost of reproducing 
\Cref{fig:vitb-byol} in \Cref{tab:byol-vit-compute},
\Cref{fig:vitb-byol-ln-vs-bn} in \Cref{tab:byol-vit-compute-2},
and \Cref{fig:vitb-byol-waterfall} in \Cref{tab:byol-vit-compute-3}.

\begin{table}[ht]
  \caption{
  Compute usage for baseline ViT BYOL investigation in \Cref{fig:vitb-byol}. 
  Values \emph{include} node allocation times (typically a small \% of corresponding total runtime), giving a practical   estimate of reproduction cost. 
  All experiments conducted are using 80Gb A100s, and experiments indicated by $(\dagger)$ have a faster interconnect.}
  \label{tab:byol-vit-compute}
  \centering
  \small
  \begin{tabular}{ccccccc}
\toprule
 Batch Size &  GPUs &  Time (h) &  Compute/Run (GPUh) &  Runs &  Compute (GPUh) \\
\midrule
       4,096 &      32 &          16.6 &               531.4 &       2 &          1,062.7 \\
       8,192 &      48 &          14.1 &               678.0 &       2 &          1,356.1 \\
      16,384 &      96 &           8.3 &               800.3 &       2 &          1,600.6 \\
      24,576 &     128 &           6.6 &               850.1 &       4 &          3,400.4 \\
      32,768${}^{\dagger}$ &     176 &           4.1 &               721.7 &       2 &          1,443.4 \\ \midrule 
      \textbf{Total} &&&&& \textbf{8,863.1} \\
\bottomrule
  \end{tabular}
\vspace{-0.2cm}
\end{table}

\begin{table}[ht]
  \caption{
  Compute usage for ViT BYOL investigation into BatchNorm and LayerNorm variants in \Cref{fig:vitb-byol-ln-vs-bn}.
  Values \emph{include} node allocation times (typically a small \% of corresponding total runtime), giving a practical   estimate of reproduction cost. 
  All experiments conducted are using 80Gb A100s, and were run for 480 epochs, except those indicated by $(*)$ were truncated early (see \Cref{fig:vitb-byol-ln-vs-bn} for more details).}
  \label{tab:byol-vit-compute-2}
  \centering
  \small
\begin{tabular}{cccccc}
\toprule
 Batch Size & Normalization &  GPUs &  Time (h) &  Compute (GPUh) \\
\midrule
       3,072 &     BatchNorm &    16 &      47.9 &           766.0 \\
       3,072 &     LayerNorm &    16 &      48.0 &           768.7 \\
      24,576 &     BatchNorm &   128 &      14.8 &          1900.4 \\
      24,576${}^{*}$ &     LayerNorm &   128 &       3.5 &           451.1 \\ \midrule
      \textbf{Total} &&&& \textbf{3,886.2} \\
\bottomrule
\end{tabular}
\vspace{-0.2cm}
\end{table}

\begin{table}[ht]
  \caption{
  Compute usage for ViT BYOL investigation into incremental scaling in \Cref{fig:vitb-byol-waterfall}.
  Values \emph{include} node allocation times (typically a small \% of corresponding total runtime), giving a practical   estimate of reproduction cost. 
  All experiments conducted are using 80Gb A100s for 60 epochs. Stage 0 corresponding to the baseline in \Cref{fig:vitb-byol-waterfall} is the run detailed in the first row of \Cref{tab:byol-vit-compute-2}, using a batch size of 3,072, Batch Normalization, and 16 GPUs. Computing only the first 60 epochs of stage 0 corresponds to approximately 127.7 GPUh, which would bring the total cost of \Cref{fig:vitb-byol-waterfall} to 1,432.9 GPUh.}
  \label{tab:byol-vit-compute-3}
  \centering
  \small
\begin{tabular}{cccccc}
\toprule
 Stage &  Batch Size &    GPUs &  Time (h) &  Compute (GPUh) \\
\midrule
     1 &        6,144 &        32 &       3.5 &           113.0 \\
     2 &        9,216 &          48 &       3.1 &           149.8 \\
     3 &       12,288 &          64 &       2.8 &           176.0 \\
     4 &       15,360 &          80 &       2.3 &           186.5 \\
     5 &       18,432 &          96 &       2.1 &           202.9 \\
     6 &       21,504 &         112 &       2.1 &           235.8 \\
     7 &       24,576 &         128 &       1.9 &           241.3 \\ \midrule 
     \textbf{Total} &&&& \textbf{1,305.2} \\
\bottomrule
\end{tabular}
\vspace{-0.2cm}
\end{table}

\FloatBarrier

Next, the cost of a single momentum ablation presented in \Cref{fig:vitb-byol-rho-ablations}
is 240 epochs at batch size 24,576, which is $\approx 240/480 \times 1900.4\text{ GPUh}=950.2\text{ GPUh}$, giving a total cost over seven runs of $\mathbf{6651.4}\text{ GPUh}$.

Finally, providing a full view of the investigations carried out for the \gls{vit} \gls{byol} is given in \Cref{tab:byol-vit-compute-summary}.

\begin{table}[ht]
  \caption{
  Total compute usage for ViT BYOL investigations.}
  \label{tab:byol-vit-compute-summary}
  \centering
  \small
\begin{tabular}{lc}
\toprule
 &  Compute (GPUh) \\
\midrule
Baselines (\Cref{fig:vitb-byol} and \Cref{tab:byol-vit-compute}) & 8,863.1 \\
BatchNorm and LayerNorm (\Cref{fig:vitb-byol-ln-vs-bn} and \Cref{tab:byol-vit-compute-2}) & 3,886.2 \\
Incremental scaling (\Cref{fig:vitb-byol-waterfall} and \Cref{tab:byol-vit-compute-3}) & 1,305.2 \\
Momentum ablations (\Cref{fig:vitb-byol-rho-ablations}) & 6,651.4 \\
All other compute, e.g. code development, runs with errors, and debugging & 84,984.1 \\
\midrule 
     \textbf{Total} & \textbf{105,690.0} \\
\bottomrule
\end{tabular}
\vspace{-0.2cm}
\end{table}
