\FloatBarrier

\subsection{Compute usage for image classification Polyak-Ruppert averaging}

The compute usage image classification Polyak-Ruppert averaging is summarized in   \Cref{tab:polyak-image-compute}

\begin{table}[ht]
  \caption{
  Compute usage for image classification Polyak-Ruppert averaging in \Cref{fig:r50-polyak,fig:r50-polyak-full-bn}. 
  The three runs for the batch size 1,024 baseline correspond to three seeds, and the nine runs for all other batch sizes correspond to using and not using the \gls{ema} Scaling Rule shown in \Cref{fig:r50-polyak}, and its application to Batch Normalization shown in \Cref{fig:r50-polyak-full-bn}. All experiments conducted are using 80Gb A100s.}
  \label{tab:polyak-image-compute}
  \centering
  \small
\begin{tabular}{cccccc}
\toprule
 Batch Size &  GPUs &  Time (h) &  Compute/Run (GPUh) &  Runs &  Compute (GPUh) \\
\midrule
        512 &     8 &      35.3 &               282.4 &     9 &          2,541.6 \\
       1,024 &     8 &      17.1 &               137.0 &     3 &           410.9 \\
       2,048 &     8 &      13.3 &               106.7 &     9 &           960.6 \\
       4,096 &     8 &       4.2 &                33.5 &     9 &           301.9 \\
       8,192 &    16 &       2.8 &                44.8 &     9 &           403.6  \\ \midrule 
\multicolumn{5}{l}{All other compute, e.g. code development, runs with errors, and debugging} & 25,768.3 \\ \midrule        
      \textbf{Total} &&&&& \textbf{30386.8} \\
\bottomrule
\end{tabular}
\end{table}
