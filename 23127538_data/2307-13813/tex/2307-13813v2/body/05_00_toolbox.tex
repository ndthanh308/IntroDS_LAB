\section{The scaling toolbox: practical methods for enabling systematic scaling}
\label{app:scaling-toolbox}

There are many different components involved in preserving optimization dynamics at different batch sizes. 
In this appendix we collect into a single place the different concepts and values that we found useful in practice, in an attempt to make the practice of scaling as accessible as possible.

\subsection{The continuous time/SDE perspective}
\label{sec:app-sde-perspective}

Here we discuss the mindset difference required when trying to preserve training dynamics.
In \gls{ml} we typically use stochastic optimization, leading us to think of the optimization in terms of \emph{performing updates}, or \emph{stepping the optimizer}.
This notion has become more common in the era of large datasets, where it may be the case that we only see a fraction of the dataset during optimization.

For dynamics preservation under scaling, we suggest that it is simpler to consider the \emph{amount of data} seen by the training process, or alternatively, the amount of \emph{continuous time} in the discretization of \glspl{sde} view.
The reason is the following.
The \gls{sde} scaling rule results 
(\Cref{def:ema-sr}, \cite{li2019stochastic,DBLP:conf/nips/LiMA21,DBLP:conf/nips/MalladiLPA22}) follow from showing that different discretizations of the \gls{sde} are close to that \gls{sde}, providing we appropriately scale hyperparameters (see \Cref{subsec:ema-sdes}).
Each of these discretizations shares the \emph{total continuous time} $T=\hat\eta\times \widehat {N}_{\text{iter}}$\footnote{This is in the case of \gls{sgd}, for RMSProp and Adam one should use $T=\hat{\eta}^2\times \widehat {N}_{\text{iter}}$ \citep{DBLP:conf/nips/MalladiLPA22}.} of the underlying \gls{sde}, but each discretization has a \emph{different} number of iterations $\widehat {N}_{\text{iter}}={N}_{\text{iter}}/\kappa $.

This perspective is already adopted, perhaps by accident in some domains.
For example, in \gls{cv}, 
it is typical to compare model performance after optimization on ImageNet1k after a \emph{number of epochs},
whilst also specifing a learning rate warmup after a \emph{number of epochs}.
This transforms the schedule into the form \emph{wait until the process meets [condition]}, where here \emph{[condition]} is \emph{when the process has seen sufficiently many samples.}

More generally, we can specify any \emph{condition} that is not a property of the discretization procedure itself.
Instead, the discretization procedure should be viewed as a numerical approximation method for the \gls{sde} we are evolving, and the properties of that discretization process (like \emph{number of steps}) are not \emph{of specific interest} in the world view where we do decouple optimization from the batch size.
A specific example of this more general case is present in \Cref{subsec:semi-supervised}, where for scaling $\kappa>2$ we wait until the pre-training \gls{wer} is sufficiently low.

There may be cases where one is working with a setup that is explicitly defined in terms of quantities related to the discretization process.
Indeed, the optimizer hyperparameters are examples of these, and need to be scaled accordingly with $\kappa$.
The other typical example of this is conditions based on the \emph{number of optimizer steps}, rather than the number of epochs.
In this case, these quantities should be scaled to achieve the desired condition in the same amount of time, i.e. as above $\widehat {N}_{\text{iter}}={N}_{\text{iter}}/\kappa$, where ${N}_{\text{iter}}$ is the number of iterations specified at the base batch size $B$. 
Concretely, if training is specified in a number of steps, then doubling the batch size implies you should train for half the number of steps.

\subsection{Scaling rules for optimization}

For ease of reference, we collect all the scaling rules related to batch size modification we are aware of.
We begin with the most well-known, the \gls{sgd} Scaling Rule (\Cref{def:lsr,def:app-sgd-scaling-rule}).
\begin{definition}[\gls{sgd} Scaling Rule]
    When running \gls{sgd} (\Cref{def:sgd}) with batch size $\hat B=\kappa B$,
    use a learning rate $\hat\eta=\kappa\eta$ \citep{DBLP:journals/corr/Krizhevsky14,DBLP:journals/corr/GoyalDGNWKTJH17}. 
    \label{def:app-sgd-scaling-rule}
\end{definition}
The \gls{sgd} Scaling Rule is also known as the Linear Scaling Rule (LSR), although for clarity, this work adopts the naming convention \emph{[Algorithm Name] Scaling Rule},
which means all parameters of those algorithms are appropriately scaled from batch size $B$ to $\kappa B$.

Next, we give the two scaling rules known for the adapative optimizers RMSProp \citep{rmsprop} and Adam \citep{DBLP:journals/corr/KingmaB14} in \Cref{def:rmpsprop-sr} and \Cref{def:adam-sr} respectively.

\begin{definition}[RMSProp Scaling Rule]
    When running RMSProp  \citep{rmsprop} with batch size $\hat B=\kappa B$,
    use a learning rate $\hat\eta=\sqrt\kappa\eta$,
    beta coefficient $\hat\beta=1-\kappa\times(1-\beta)$, and adaptivity parameter
    $\hat\epsilon=\frac\epsilon{\sqrt\kappa}$
    \citep{DBLP:conf/nips/MalladiLPA22}.
    \label{def:rmpsprop-sr}
\end{definition}

\begin{definition}[Adam Scaling Rule]
    When running Adam \citep{DBLP:journals/corr/KingmaB14} with batch size $\hat B=\kappa B$,
    use a learning rate $\hat\eta=\sqrt\kappa\eta$,
    beta coefficients 
    $\hat\beta_1=1-\kappa\times(1-\beta_1)$,
    $\hat\beta_2=1-\kappa\times(1-\beta_2)$, and adaptivity parameter
    $\hat\epsilon=\frac\epsilon{\sqrt\kappa}$
    \citep{DBLP:conf/nips/MalladiLPA22}.
    \label{def:adam-sr}
\end{definition}

Next, we present a contribution of this work, the \gls{ema} Scaling Rule
(\Cref{def:ema-sr,def:app-ema-scaling-rule}), which extends the above scaling rules to allow the presence of a model \gls{ema} which is able to contribute to the overall optimization (see \Cref{app:ema-approximation-theorem,app:matrix-calculations} for derivations).

\begin{definition}[\gls{ema} Scaling Rule]
    When computing the \gls{ema} update (\Cref{def:ema}) of a model undergoing stochastic optimization with batch size $\hat B=\kappa B$,
    use a momentum $\hat\rho=\rho^\kappa$ and scale other optimizers according to their own scaling rules.
\label{def:app-ema-scaling-rule}
\end{definition}

Concretely, if we are using \gls{sgd} in the presence of a model \gls{ema},
\Cref{def:app-sgd-scaling-rule,def:app-ema-scaling-rule} state that we should take 
$\hat\eta=\kappa\eta$ \emph{and} $\hat\rho=\rho^\kappa$ when scaling by $\kappa=\hat B/B$.

The final scaling rule is for weight decay, and follows from the scaling logic discussed in \Cref{sec:app-sde-perspective} and \citet{DBLP:journals/corr/Krizhevsky14}.
If we take the weight decay regularization penalty $\lambda$ defined at batch size $B$, what should the weight decay $\hat\lambda$ be for batch size $\hat B=\kappa B$?
For simplicity, consider $\kappa$ updates of optimization of parameters $\rvtheta_t$ in the presence of weight decay only
\begin{equation}
    \rvtheta_{t+\kappa}
    =\rvtheta_{t+\kappa-1}-\eta\,\lambda\,\rvtheta_{t+\kappa-1}
    =(1 - \eta\,\lambda)\,\rvtheta_{t+\kappa-1}
    =(1 - \eta\,\lambda)^\kappa\,\rvtheta_{t}.
\end{equation}
Therefore, to match the effect of weight decay with a single iteration step, we need to match
\begin{equation}
    1-\hat\eta \,\hat \lambda = (1 - \eta\,\lambda)^\kappa.
\end{equation}
Solving for $\hat\lambda$ and expanding around $\eta\approx 0$ gives
\begin{equation}
    \hat\lambda
    =\frac{1-(1 - \eta\,\lambda)^\kappa}{\hat\eta}
    \approx 
    \frac{\eta}{\hat \eta}\times 
    \kappa \,\lambda
    +\mathcal{O}(\eta).
\end{equation}
This leads to the Weight Decay Scaling Rule (\Cref{def:wd-sr}).

\begin{definition}[Weight Decay Scaling Rule]
    When using weight decay with batch size $\hat B=\kappa B$,
    use a penalty term $\hat\lambda=(\kappa \hat\eta / \eta)\,\lambda$,
    where $\hat\eta$ and $\eta$ represent the scaled and unscaled learning rates of the corresponding optimizer
    \citep{DBLP:journals/corr/Krizhevsky14,DBLP:conf/nips/Li0TSG18,DBLP:conf/iclr/LoshchilovH19}.
    \label{def:wd-sr}
\end{definition}

The Weight Decay Scaling Rule implies that using \emph{linear} scaling for the learning rate $\eta$ then the weight decay penalty is automatically scaled,
and when using \emph{square-root} scaling for the learning rate $\eta$ (e.g. in the case of the Adam Scaling Rule (\Cref{def:adam-sr})) then the weight decay penalty should also be scaled with a \emph{square-root} as is proposed in \citet{DBLP:conf/iclr/LoshchilovH19}.

Finally, we see that if the implementation of weight decay does not have an update scaled by the learning rate, i.e. the update is $\rvtheta_{t+1}=(1-\lambda)\,\rvtheta_t$, then the scaling rule is optimizer-independent, and becomes linear for small weight decay, i.e. $\hat\lambda=\kappa\lambda$,
and for arbitrary $\lambda$ takes the form $\hat\lambda=1-(1-\lambda)^\kappa$.

\subsection{Commonly used values of hyperparameters at different batch sizes}

In the literature it is common to give a base learning rate $\eta$ defined at batch size 256, implicitly using the \gls{sgd} Scaling Rule, even when using the Adam optimizer.
Because the scaling of other optimization hyperparameters was not understood until recently, it is also common to just present these \emph{for the experiment}, e.g. the Adam betas and epsilon, and the \gls{ema} momentum, implicitly defined at the scale of the experiment, for example at batch size 4096.
One way to deal with this in practice is to define a single reference batch size $B$ at which \emph{all} hyperparameters are defined, and then scale from there.
In this case, it is easiest to compute \emph{using linear scaling} the learning rate at the redefined base batch size 
$\eta = \tilde\kappa\,\eta_{\text{orig}}$,
where $\tilde\kappa=B/B_{\text{orig}}$,
and then scale this new reference $\eta$ as $\hat\eta=\kappa\eta$, $\kappa=\hat B/B$, along with e.g. the momentum defined at $B$.

As this process can be slightly frustrating, we have provided tables of typical learning rates in \Cref{tab:common-hparams-lr} and momenta in \Cref{tab:common-hparams-momenta}.

\newcommand{\firstmidrules}{
\cmidrule(r){2-4}
\cmidrule(r){5-7}
}
\newcommand{\secondmidrules}{
\cmidrule(r){2-3}
\cmidrule(r){4-4}
\cmidrule(r){5-5}
\cmidrule(r){6-7}
}

\begin{table}[t]
\centering
\caption{Scaled learning rates $\hat\eta$ at different batch sizes $\hat B=\kappa B$ given reference learning rates $\eta$ defined at batch size $B$. The reference values of each column are boldened. 
Note that this is only valid when there is a notion of \emph{single sample}. In the sequence learning setup (for example, in \Cref{subsec:semi-supervised}), the notion of batch size should be appropriately replaced with the \emph{dynamic batch size}, i.e. total sequence length.}
\small
\label{tab:common-hparams-lr}
\begin{tabular}{cllllll}
\toprule
{} & \multicolumn{3}{c}{$\hat\eta=\kappa\eta$ [SGD]} & \multicolumn{3}{c}{$\hat\eta=\sqrt\kappa\eta$ [RMSProp, Adam]} \\ \firstmidrules
{} & \multicolumn{2}{c}{$B=256$} &         $B=512$ &                                    $B=256$ & \multicolumn{2}{c}{$B=4096$} \\ \secondmidrules
Batch size $\hat B$&                  $\eta=0.1$ &      $\eta=0.3$ &      $\eta=0.1$ &                             $\eta=10^{-3}$ &      $\eta=4.8$ &    $\eta=10^{-3}$ \\
\midrule
$32$                &                    $0.0125$ &        $0.0375$ &       $0.00625$ &                                  $0.00035$ &       $0.42426$ &         $0.00009$ \\
$64$                &                     $0.025$ &         $0.075$ &        $0.0125$ &                                   $0.0005$ &           $0.6$ &         $0.00013$ \\
$128$               &                      $0.05$ &          $0.15$ &         $0.025$ &                                  $0.00071$ &       $0.84853$ &         $0.00018$ \\
$256$               &              $\mathbf{0.1}$ &  $\mathbf{0.3}$ &          $0.05$ &                           $\mathbf{0.001}$ &           $1.2$ &         $0.00025$ \\
$512$               &                       $0.2$ &           $0.6$ &  $\mathbf{0.1}$ &                                  $0.00141$ &       $1.69706$ &         $0.00035$ \\
$1024$              &                       $0.4$ &           $1.2$ &           $0.2$ &                                    $0.002$ &           $2.4$ &          $0.0005$ \\
$2048$              &                       $0.8$ &           $2.4$ &           $0.4$ &                                  $0.00283$ &       $3.39411$ &         $0.00071$ \\
$4096$              &                       $1.6$ &           $4.8$ &           $0.8$ &                                    $0.004$ &  $\mathbf{4.8}$ &  $\mathbf{0.001}$ \\
$8192$              &                       $3.2$ &           $9.6$ &           $1.6$ &                                  $0.00566$ &       $6.78823$ &         $0.00141$ \\
$16384$             &                       $6.4$ &          $19.2$ &           $3.2$ &                                    $0.008$ &           $9.6$ &           $0.002$ \\
$32768$             &                      $12.8$ &          $38.4$ &           $6.4$ &                                  $0.01131$ &      $13.57645$ &         $0.00283$ \\
$65536$             &                      $25.6$ &          $76.8$ &          $12.8$ &                                    $0.016$ &          $19.2$ &           $0.004$ \\
\bottomrule
\end{tabular}
\end{table}

\newcommand{\thirdmidrules}{
\cmidrule(r){2-4}
\cmidrule(r){5-8}
}

\begin{table}[t]
\centering
\caption{Scaled EMA momenta $\hat\rho=\rho^\kappa$ at different batch sizes $\hat B=\kappa B$ given reference momenta $\rho$ defined at batch size $B$. The reference values of each column are boldened. Again in the sequence learning setup, batch size should be appropriately replaced with a notion of sequence length.}
\small
\label{tab:common-hparams-momenta}
\begin{tabular}{clllllll}
\toprule
{} & \multicolumn{3}{c}{$B=256$} & \multicolumn{4}{c}{$B=4096$} \\ \thirdmidrules
Batch size $\hat B$&      $\rho=0.9999$ &      $\rho=0.999$ &      $\rho=0.99$ &      $\rho=0.996$ &      $\rho=0.992$ &      $\rho=0.99$ &      $\rho=0.97$ \\
\midrule
$32$                &          $0.99999$ &         $0.99987$ &        $0.99874$ &         $0.99997$ &         $0.99994$ &        $0.99992$ &        $0.99976$ \\
$64$                &          $0.99997$ &         $0.99975$ &        $0.99749$ &         $0.99994$ &         $0.99987$ &        $0.99984$ &        $0.99952$ \\
$128$               &          $0.99995$ &          $0.9995$ &        $0.99499$ &         $0.99987$ &         $0.99975$ &        $0.99969$ &        $0.99905$ \\
$256$               &  $\mathbf{0.9999}$ &  $\mathbf{0.999}$ &  $\mathbf{0.99}$ &         $0.99975$ &          $0.9995$ &        $0.99937$ &         $0.9981$ \\
$512$               &           $0.9998$ &           $0.998$ &         $0.9801$ &          $0.9995$ &           $0.999$ &        $0.99874$ &         $0.9962$ \\
$1024$              &           $0.9996$ &         $0.99601$ &         $0.9606$ &           $0.999$ &         $0.99799$ &        $0.99749$ &        $0.99241$ \\
$2048$              &           $0.9992$ &         $0.99203$ &        $0.92274$ &           $0.998$ &         $0.99599$ &        $0.99499$ &        $0.98489$ \\
$4096$              &           $0.9984$ &         $0.98412$ &        $0.85146$ &  $\mathbf{0.996}$ &  $\mathbf{0.992}$ &  $\mathbf{0.99}$ &  $\mathbf{0.97}$ \\
$8192$              &           $0.9968$ &         $0.96849$ &        $0.72498$ &         $0.99202$ &         $0.98406$ &         $0.9801$ &         $0.9409$ \\
$16384$             &          $0.99362$ &         $0.93798$ &         $0.5256$ &          $0.9841$ &         $0.96838$ &         $0.9606$ &        $0.88529$ \\
$32768$             &          $0.98728$ &          $0.8798$ &        $0.27625$ &         $0.96844$ &         $0.93776$ &        $0.92274$ &        $0.78374$ \\
$65536$             &          $0.97472$ &         $0.77405$ &        $0.07632$ &         $0.93788$ &          $0.8794$ &        $0.85146$ &        $0.61425$ \\
\bottomrule
\end{tabular}
\end{table}





\subsection{Progressive scaling}
\label{subsec:dynamic-batch-scaling}

In \Cref{subsec:self-supervised} we introduced Progressive Scaling (\Cref{def:progressive-scaling}) to test our hypothesis that early in the \gls{byol} training procedure, there are dynamics that are challenging to replicate at larger batch sizes.
To remove ambiguity, in \Cref{alg:progressive-scaling} we provide pseudo-code for how to use Progressive Scaling.

\begin{algorithm}[t!]
\caption{Stochastic Gradient Descent with Progressive Scaling}\label{alg:progressive-scaling}
\begin{algorithmic}
\Require Base learning rate $\eta$, base momentum $\rho$ for base batch size $B$
\Require Initial target model parameters $\rvtheta$ and model \gls{ema} parameters $\rvzeta$
\Require Epochs $E$ and schedule of batch sizes $\mathcal B=B_1,B_2,\ldots,B_{E}$
\Require Loss function $\Ls$
\For{$e$ in $1,2\ldots,E$}
    \State $\hat B \gets \mathcal B[e]$ \Comment{Get current batch size}
    \State $\kappa \gets \hat B/B$      \Comment{Compute scaling factor}
    \State $\hat\eta \gets \kappa \eta$ \Comment{Get scaled learning rate}
    \State $\hat\rho \gets \rho^\kappa$ \Comment{Get scaled momentum}
    \For{$b$ in $1,2\ldots,\text{floor}(E/\hat B)$}
        \State Sample a minibatch of $\hat B$ samples $\mathcal X=\{\vx^{(1)},\ldots,\vx^{(\hat B)}\}$
        \State $\rvtheta \gets \rvtheta - 
        (\hat \eta / \hat B)
        \sum_{x\in\mathcal X} \nabla_{\rvtheta} \Ls(x;\rvtheta,\rvzeta)$ \Comment{SGD Update}
        \State $\rvzeta \gets \hat\rho \,\rvzeta+(1-\hat\rho)\,\rvtheta$ \Comment{EMA Update}
    \EndFor
\EndFor
\end{algorithmic}
\end{algorithm}

In \Cref{alg:progressive-scaling}, the prefactor of the \gls{sgd} update could also have been written $\eta/B$, although an equivalent use of the base momentum is not possible.

Finally, we outline how to extend \Cref{alg:progressive-scaling} to more complex setups, like those presented in 
\Cref{subsec:self-supervised}:
\begin{enumerate}[leftmargin=0.75cm]
    \item Optimizer scaling rules are used appropriately, for example the Adam scaling rule in case of using the Adam optimizer to update parameters $\rvtheta$.
    \item Schedules for hyperparameters are computed using the base hyperparameters, and are then modified by application of the scaling law in epoch (outer) loop.
    \item Schedules for hyperparameters at the \emph{step} rather than epoch level can be achieved in practice through recomputing the schedule and updating the notion of minibatch index appropriately throughout training.
\end{enumerate}
All of the above techniques are used in \Cref{subsec:self-supervised}.
In addition, scheduling batch sizes within epoch is possible, providing one maintains a notion of computation within some fixed continuous time $T_{\text{fixed}}$. 
We did not investigate this scenario.

