\documentclass[aps,prb,twocolumn,showpacs,amsmath,amssymb]{revtex4-2}
\pdfoutput=1
\usepackage[colorlinks=true,citecolor=blue]{hyperref}

\begin{document}
 
\title{Supplemental Material for ``Adiabatic Cooper Pair Splitter"}
\author{Fredrik Brange}
\affiliation{Department of Applied Physics, Aalto University, 00076 Aalto, Finland}
\author{Riya Baruah}
\affiliation{Department of Applied Physics, Aalto University, 00076 Aalto, Finland}
\author{Christian Flindt}
\affiliation{Department of Applied Physics, Aalto University, 00076 Aalto, Finland}

\maketitle

\onecolumngrid

\renewcommand{\theequation}{S\arabic{equation}}%
\renewcommand{\thefigure}{S\arabic{figure}}%

\section{Lindblad equation \& vectorization}\label{sec:mastereq}
\noindent
As explained in the main text, the Cooper pair splitter can be described by the Lindblad equation~\cite{Sauret:Quantum,Walldorf2020}
\begin{equation}\label{eq:vonNeumannEq}
	\frac{d}{dt}\hat \rho(t)=\mathcal{L}(t)\hat \rho(t)=\frac{1}{i\hbar}[\hat H(t),\hat \rho(t)]+\Gamma\sum_{\ell\sigma}\big( \hat d_{\ell\sigma}^{\phantom\dagger}\hat \rho(t) \hat d_{\ell\sigma}^\dagger-\frac{1}{2}\{\hat \rho(t),\hat d_{\ell\sigma}^\dagger \hat d_{\ell\sigma}^{\phantom\dagger}\}\big),
\end{equation}
where 
\begin{equation}\label{eq:Heff}
	\hat{H}(t)=\sum_{\ell\sigma}\varepsilon_\ell(t)\hat{d}_{\ell\sigma}^\dagger \hat{d}_{\ell\sigma}^{\phantom\dagger}-\gamma(\hat d_S^\dagger+\hat d_S) -\kappa\sum_\sigma (\hat{d}_{L\sigma}^\dagger \hat{d}_{R\sigma}^{\phantom\dagger}+
	\mathrm{h.c.}),
\end{equation}
is the effective Hamiltonian of the two quantum dots with time-dependent level positions, $\varepsilon_\ell(t)$. To carry out our calculations, we vectorize the density matrix of the quantum dots and implement a matrix representation of the Liouvillian. Here, we do not explicitly consider the spin-degrees of freedom, and it therefore suffices to express the density matrix and the Liouvillian in the charge basis only. In this representation, the density matrix takes the form
\begin{equation}
	\hat\rho= \left( 
	\begin{array}{cccc}
		\rho_{00} & 0 & 0 &  \rho_{S0} \\
		0 & \rho_{LL}  & \rho_{RL} & 0  \\
		0 & \rho_{LR} & \rho_{RR} & 0\\
		\rho_{0S} & 0 & 0 & \rho_{SS}
	\end{array}
	\right),
\end{equation}
where $\rho_{\ell \ell'}=\sum_{\sigma}\rho_{\ell\sigma,\ell'\sigma}$ are given by traces over the spins. The density matrix can be written on the vectorized form 
\begin{equation}
	\hat\rho= (\rho_{00},\rho_{LL},\rho_{RR},\rho_{SS},\rho_{0S}, \rho_{S0},\rho_{LR},\rho_{RL})^T, 
\end{equation}
where the first four elements are the populations, and the others are the coherences. The Liouvillian then becomes
\begin{equation}\label{eq:kernelmatrix}
\mathcal{L}(\chi,t)=
\left( 
	\begin{array}{cccccccc}
	0 & \Gamma e^{i\chi} & \Gamma & 0 & -i\gamma & i\gamma & 0 & 0 \\
	0 & -\Gamma  & 0 & \Gamma  & 0 & 0 &  -i\kappa &  i\kappa \\
	0 & 0 & -\Gamma  & \Gamma  e^{i\chi} & 0 & 0 & i\kappa & -i\kappa \\
	0 & 0 & 0 & -2\Gamma & i\gamma & -i\gamma & 0 & 0 \\
	-i\gamma & 0 & 0 & i\gamma & i\epsilon(t)-\Gamma  & 0 & 0 & 0 \\
	i\gamma  & 0 & 0 & -i\gamma  & 0 & -i\epsilon(t)-\Gamma  & 0 & 0 \\
	0 & -i\kappa & i\kappa & 0 & 0 & 0 & -i\delta(t)-\Gamma  & 0 \\
	0 & i\kappa & -i\kappa & 0 & 0 & 0 & 0 & i\delta(t)-\Gamma 
	\end{array}
\right),
\end{equation}
where we have included a counting field, $\chi$, that couples to transitions into the left lead, and we have defined the detuning and the sum of the energy levels, $\delta=\varepsilon_L-\varepsilon_R$ and $\epsilon=\varepsilon_L+\varepsilon_R$, with $\hbar,e=1$ from now on. We note that a matrix representation of the spin-resolved Liouvillian can be found in the appendix of Ref.~\cite{Walldorf2020}.

\section{Time-dependent \& period-averaged current}\label{sec:current}
\noindent
In the main text, we show results for the time-dependent current running into the left lead, given by the expression
\begin{equation}
	I_L(t)=\mathrm{tr}\{\mathcal{J}_{L}\hat \rho_C(t)\}
	\label{eq:time_curr}
\end{equation}
in terms of the jump operator $\mathcal{J}_{L}\hat \rho \equiv \Gamma \sum_\sigma \hat d_{L\sigma}^{\phantom\dagger}\hat \rho \hat d_{L\sigma}^\dagger$ and the periodic state of the system with the property $\hat \rho_C(t)=\hat \rho_C(t+\mathcal T)$.   To find the periodic state, we need the time-evolution operator
\begin{equation}
	\mathcal{U}(t,t_0) = \hat{T}\left\{ e^{\int_{t_0}^t \mathcal{L}(t') dt'}\right\}\simeq \prod_i e^{\mathcal{L}(t_i) \Delta t},
\end{equation} 
where $\hat{T}$ is the time-ordering operator, and $\mathcal{L}(t)$ is the Liouvillian without the counting field. We also show how we evaluate the time-evolution operator by discretizing the interval $[t_0,t]$ in small steps of size $\Delta t$, during which the Liouvillian is roughly constant. The periodic state can be found from the eigenproblem, $\mathcal{U}(t+\mathcal{T},t)\hat \rho_C(t)=\hat \rho_C(t)$, and the trace operation in Eq.~(\ref{eq:time_curr}) is implemented by summing over the first four elements of $\mathcal{J}_{L}\hat \rho_C(t)$ in its vectorized form. Moreover, the period-averaged current can be obtained as $I_L = \int_0^\mathcal{T}dt I_L(t)/\mathcal{T}$. Without the driving, one can analytically find the stationary state, defined by  $\mathcal{L}\hat\rho_S=0$, and Eq.~(2) of the main text then follows as $I_L=\mathrm{tr}\{\mathcal{J}_{L}\hat \rho_S\}$.

\section{Low-frequency noise}\label{sec:noise}
\noindent We find the low-frequency noise using techniques from full counting statistics by including a counting field as in Eq.~(\ref{eq:kernelmatrix})~\cite{Bagrets:2003,Pistolesi:2004,Flindt2005,Benito:2016,Potanina:2019}. The moment generating function for the number of emitted electrons after $N$ periods then reads 
\begin{equation}
	M(\chi,N) = \text{tr}\left\{ [\mathcal{U}(\chi,\mathcal T,0)]^N \hat\rho_C(0) \right\},
	\label{MGF}
\end{equation}
where we have defined $\mathcal{U}(\chi,t,t_0) = \hat{T} \{e^{\int_{t_0}^t \mathcal{L}(\chi,t') dt'}\}$. The cumulant generating function of the current is given as
\begin{equation}
F(\chi)=\lim_{N\rightarrow \infty}\ln[M(\chi,N)]/N\mathcal T = \ln[\max_i\{\lambda_i(\chi)\}]/\mathcal T
\end{equation}
in terms of the eigenvalue of $\mathcal{U}(\chi,\mathcal{T},0)$ with the largest absolute value. Moreover, the zero-frequency cumulants of the current are given by derivatives with respect to the counting field as $\langle \!\langle I_L^n \rangle\!\rangle = \partial_{i \chi}^n F(\chi)|_{\chi=0}$. Specifically, the average current and the noise are the first and second cumulants of the current, $I_L = \partial_{i \chi} F(\chi)|_{\chi=0}$
and $S_L = \partial^2_{i \chi} F(\chi)|_{\chi=0}$.

\section{Waiting time distribution}\label{sec:wtd}
\noindent
To evaluate the distribution of waiting times between electrons tunneling into the left lead, we use the expression
\begin{equation}
	\mathcal{W}_{L}(\tau) =\overline{\mathrm{tr}\{ \mathcal{J}_{L} \mathcal{U}_L(t+\tau,t)\mathcal{J}_{L} \hat{\rho}_C(t)\}}/ I_{L},
\end{equation} 
where the overline denotes an average over a period of the drive, while $\mathcal{U}_L(t,t_0) = \hat{T}\{e^{\int_{t_0}^t \left(\mathcal{L}(t')-\mathcal{J}_L\right) dt'} \}$ is the time-evolution operator, which excludes electron tunneling into the left drain~\cite{Brandes:Waiting,Walldorf2018,Brange:2021}. By evaluating this expression, we obtain the waiting time distributions presented in the main text.

\begin{thebibliography}{10}%
	\makeatletter
	\providecommand \@ifxundefined [1]{%
		\@ifx{#1\undefined}
	}%
	\providecommand \@ifnum [1]{%
		\ifnum #1\expandafter \@firstoftwo
		\else \expandafter \@secondoftwo
		\fi
	}%
	\providecommand \@ifx [1]{%
		\ifx #1\expandafter \@firstoftwo
		\else \expandafter \@secondoftwo
		\fi
	}%
	\providecommand \natexlab [1]{#1}%
	\providecommand \enquote  [1]{``#1''}%
	\providecommand \bibnamefont  [1]{#1}%
	\providecommand \bibfnamefont [1]{#1}%
	\providecommand \citenamefont [1]{#1}%
	\providecommand \href@noop [0]{\@secondoftwo}%
	\providecommand \href [0]{\begingroup \@sanitize@url \@href}%
	\providecommand \@href[1]{\@@startlink{#1}\@@href}%
	\providecommand \@@href[1]{\endgroup#1\@@endlink}%
	\providecommand \@sanitize@url [0]{\catcode `\\12\catcode `\$12\catcode
		`\&12\catcode `\#12\catcode `\^12\catcode `\_12\catcode `\%12\relax}%
	\providecommand \@@startlink[1]{}%
	\providecommand \@@endlink[0]{}%
	\providecommand \url  [0]{\begingroup\@sanitize@url \@url }%
	\providecommand \@url [1]{\endgroup\@href {#1}{\urlprefix }}%
	\providecommand \urlprefix  [0]{URL }%
	\providecommand \Eprint [0]{\href }%
	\providecommand \doibase [0]{https://doi.org/}%
	\providecommand \selectlanguage [0]{\@gobble}%
	\providecommand \bibinfo  [0]{\@secondoftwo}%
	\providecommand \bibfield  [0]{\@secondoftwo}%
	\providecommand \translation [1]{[#1]}%
	\providecommand \BibitemOpen [0]{}%
	\providecommand \bibitemStop [0]{}%
	\providecommand \bibitemNoStop [0]{.\EOS\space}%
	\providecommand \EOS [0]{\spacefactor3000\relax}%
	\providecommand \BibitemShut  [1]{\csname bibitem#1\endcsname}%
	\let\auto@bib@innerbib\@empty
	%</preamble>
	\bibitem [{\citenamefont {Sauret}\ \emph {et~al.}(2004)\citenamefont {Sauret},
		\citenamefont {Feinberg},\ and\ \citenamefont {Martin}}]{Sauret:Quantum}%
	\BibitemOpen
	\bibfield  {author} {\bibinfo {author} {\bibfnamefont {O.}~\bibnamefont
			{Sauret}}, \bibinfo {author} {\bibfnamefont {D.}~\bibnamefont {Feinberg}},\
		and\ \bibinfo {author} {\bibfnamefont {T.}~\bibnamefont {Martin}},\
	}\bibfield  {title} {\bibinfo {title} {Quantum master equations for the
			superconductor-quantum dot entangler},\ }\href
	{https://doi.org/10.1103/PhysRevB.70.245313} {\bibfield  {journal} {\bibinfo
			{journal} {Phys. Rev. B}\ }\textbf {\bibinfo {volume} {70}},\ \bibinfo
		{pages} {245313} (\bibinfo {year} {2004})}\BibitemShut {NoStop}%
	\bibitem [{\citenamefont {Walldorf}\ \emph {et~al.}(2020)\citenamefont
		{Walldorf}, \citenamefont {Brange}, \citenamefont {Padurariu},\ and\
		\citenamefont {Flindt}}]{Walldorf2020}%
	\BibitemOpen
	\bibfield  {author} {\bibinfo {author} {\bibfnamefont {N.}~\bibnamefont
			{Walldorf}}, \bibinfo {author} {\bibfnamefont {F.}~\bibnamefont {Brange}},
		\bibinfo {author} {\bibfnamefont {C.}~\bibnamefont {Padurariu}},\ and\
		\bibinfo {author} {\bibfnamefont {C.}~\bibnamefont {Flindt}},\ }\bibfield
	{title} {\bibinfo {title} {Noise and full counting statistics of a {C}ooper
			pair splitter},\ }\href {https://doi.org/10.1103/PhysRevB.101.205422}
	{\bibfield  {journal} {\bibinfo  {journal} {Phys. Rev. B}\ }\textbf {\bibinfo
			{volume} {101}},\ \bibinfo {pages} {205422} (\bibinfo {year}
		{2020})}\BibitemShut {NoStop}%
	\bibitem [{\citenamefont {Bagrets}\ and\ \citenamefont
		{Nazarov}(2003)}]{Bagrets:2003}%
	\BibitemOpen
	\bibfield  {author} {\bibinfo {author} {\bibfnamefont {D.~A.}\ \bibnamefont
			{Bagrets}}\ and\ \bibinfo {author} {\bibfnamefont {Yu.~V.}\ \bibnamefont
			{Nazarov}},\ }\bibfield  {title} {\bibinfo {title} {{Full counting statistics
				of charge transfer in Coulomb blockade systems}},\ }\href
	{https://doi.org/10.1103/PhysRevB.67.085316} {\bibfield  {journal} {\bibinfo
			{journal} {Phys. Rev. B}\ }\textbf {\bibinfo {volume} {67}},\ \bibinfo
		{pages} {085316} (\bibinfo {year} {2003})}\BibitemShut {NoStop}%
	\bibitem [{\citenamefont {Pistolesi}(2004)}]{Pistolesi:2004}%
	\BibitemOpen
	\bibfield  {author} {\bibinfo {author} {\bibfnamefont {F.}~\bibnamefont
			{Pistolesi}},\ }\bibfield  {title} {\bibinfo {title} {Full counting
			statistics of a charge shuttle},\ }\href
	{https://doi.org/10.1103/PhysRevB.69.245409} {\bibfield  {journal} {\bibinfo
			{journal} {Phys. Rev. B}\ }\textbf {\bibinfo {volume} {69}},\ \bibinfo
		{pages} {245409} (\bibinfo {year} {2004})}\BibitemShut {NoStop}%
	\bibitem [{\citenamefont {Flindt}\ \emph {et~al.}(2005)\citenamefont {Flindt},
		\citenamefont {Novotn\'y},\ and\ \citenamefont {Jauho}}]{Flindt2005}%
	\BibitemOpen
	\bibfield  {author} {\bibinfo {author} {\bibfnamefont {C.}~\bibnamefont
			{Flindt}}, \bibinfo {author} {\bibfnamefont {T.}~\bibnamefont {Novotn\'y}},\
		and\ \bibinfo {author} {\bibfnamefont {A.-P.}\ \bibnamefont {Jauho}},\
	}\bibfield  {title} {\bibinfo {title} {Full counting statistics of
			nano-electromechanical systems},\ }\href
	{https://doi.org/10.1209/epl/i2004-10351-x} {\bibfield  {journal} {\bibinfo
			{journal} {EPL}\ }\textbf {\bibinfo {volume} {69}},\ \bibinfo {pages} {475}
		(\bibinfo {year} {2005})}\BibitemShut {NoStop}%
	\bibitem [{\citenamefont {Benito}\ \emph {et~al.}(2016)\citenamefont {Benito},
		\citenamefont {Niklas},\ and\ \citenamefont {Kohler}}]{Benito:2016}%
	\BibitemOpen
	\bibfield  {author} {\bibinfo {author} {\bibfnamefont {M.}~\bibnamefont
			{Benito}}, \bibinfo {author} {\bibfnamefont {M.}~\bibnamefont {Niklas}},\
		and\ \bibinfo {author} {\bibfnamefont {S.}~\bibnamefont {Kohler}},\
	}\bibfield  {title} {\bibinfo {title} {Full-counting statistics of
			time-dependent conductors},\ }\href
	{https://doi.org/10.1103/PhysRevB.94.195433} {\bibfield  {journal} {\bibinfo
			{journal} {Phys. Rev. B}\ }\textbf {\bibinfo {volume} {94}},\ \bibinfo
		{pages} {195433} (\bibinfo {year} {2016})}\BibitemShut {NoStop}%
	\bibitem [{\citenamefont {Potanina}\ \emph {et~al.}(2019)\citenamefont
		{Potanina}, \citenamefont {Brandner},\ and\ \citenamefont
		{Flindt}}]{Potanina:2019}%
	\BibitemOpen
	\bibfield  {author} {\bibinfo {author} {\bibfnamefont {E.}~\bibnamefont
			{Potanina}}, \bibinfo {author} {\bibfnamefont {K.}~\bibnamefont {Brandner}},\
		and\ \bibinfo {author} {\bibfnamefont {C.}~\bibnamefont {Flindt}},\
	}\bibfield  {title} {\bibinfo {title} {Optimization of quantized charge
			pumping using full counting statistics},\ }\href
	{https://doi.org/10.1103/PhysRevB.99.035437} {\bibfield  {journal} {\bibinfo
			{journal} {Phys. Rev. B}\ }\textbf {\bibinfo {volume} {99}},\ \bibinfo
		{pages} {035437} (\bibinfo {year} {2019})}\BibitemShut {NoStop}%
	\bibitem [{\citenamefont {Brandes}(2008)}]{Brandes:Waiting}%
	\BibitemOpen
	\bibfield  {author} {\bibinfo {author} {\bibfnamefont {T.}~\bibnamefont
			{Brandes}},\ }\bibfield  {title} {\bibinfo {title} {Waiting times and noise
			in single particle transport},\ }\href
	{https://doi.org/10.1002/andp.200810306} {\bibfield  {journal} {\bibinfo
			{journal} {Ann. Physik}\ }\textbf {\bibinfo {volume} {17}},\ \bibinfo {pages}
		{477} (\bibinfo {year} {2008})}\BibitemShut {NoStop}%
	\bibitem [{\citenamefont {Walldorf}\ \emph {et~al.}(2018)\citenamefont
		{Walldorf}, \citenamefont {Padurariu}, \citenamefont {Jauho},\ and\
		\citenamefont {Flindt}}]{Walldorf2018}%
	\BibitemOpen
	\bibfield  {author} {\bibinfo {author} {\bibfnamefont {N.}~\bibnamefont
			{Walldorf}}, \bibinfo {author} {\bibfnamefont {C.}~\bibnamefont {Padurariu}},
		\bibinfo {author} {\bibfnamefont {A.-P.}\ \bibnamefont {Jauho}},\ and\
		\bibinfo {author} {\bibfnamefont {C.}~\bibnamefont {Flindt}},\ }\bibfield
	{title} {\bibinfo {title} {Electron {W}aiting {T}imes of a {C}ooper {P}air
			{S}plitter},\ }\href {https://doi.org/10.1103/PhysRevLett.120.087701}
	{\bibfield  {journal} {\bibinfo  {journal} {Phys. Rev. Lett.}\ }\textbf
		{\bibinfo {volume} {120}},\ \bibinfo {pages} {087701} (\bibinfo {year}
		{2018})}\BibitemShut {NoStop}%
	\bibitem [{\citenamefont {Brange}\ \emph {et~al.}(2021)\citenamefont {Brange},
		\citenamefont {Prech},\ and\ \citenamefont {Flindt}}]{Brange:2021}%
	\BibitemOpen
	\bibfield  {author} {\bibinfo {author} {\bibfnamefont {F.}~\bibnamefont
			{Brange}}, \bibinfo {author} {\bibfnamefont {K.}~\bibnamefont {Prech}},\ and\
		\bibinfo {author} {\bibfnamefont {C.}~\bibnamefont {Flindt}},\ }\bibfield
	{title} {\bibinfo {title} {{Dynamic Cooper Pair Splitter}},\ }\href
	{https://doi.org/10.1103/PhysRevLett.127.237701} {\bibfield  {journal}
		{\bibinfo  {journal} {Phys. Rev. Lett.}\ }\textbf {\bibinfo {volume} {127}},\
		\bibinfo {pages} {237701} (\bibinfo {year} {2021})}\BibitemShut {NoStop}%
\end{thebibliography}%


\end{document}