\documentclass[11pt, reqno]{amsart} % AMS article style
\usepackage{amssymb,amscd,amsfonts,amsbsy}
\usepackage{latexsym}
\usepackage{exscale}
\usepackage{amsmath,amsthm,amsfonts}
\usepackage{mathrsfs}
\usepackage{xcolor} % Enables the creations of colors.
\usepackage[colorlinks=true, linkcolor=blue, citecolor=red, urlcolor=red, %backref=page
]{hyperref} % Add coloured links
\usepackage{esint} % For \fint symbol
\usepackage{stmaryrd}
\usepackage{pifont}

\usepackage[utf8]{inputenc}


\parskip=3pt

\setlength{\oddsidemargin}{0in}
\setlength{\evensidemargin}{0in}
\setlength{\evensidemargin}{0in}
\setlength{\textwidth}{6.2in}
\setlength{\textheight}{9in}
\setlength{\topmargin}{-0.50in}
\calclayout

\newcommand{\red}{\color{red}}
\newcommand{\blue}{\color{blue}}
\newcommand{\black}{\color{black}}

\allowdisplaybreaks%[3] % Allow equation break between pages


\makeatletter
\@namedef{subjclassname@2020}{%
  \textup{2020} Mathematics Subject Classification}
\makeatother


\theoremstyle{plain}
\newtheorem{theorem}[equation]{Theorem}
\newtheorem{proposition}[equation]{Proposition}
\newtheorem{lemma}[equation]{Lemma}
\newtheorem{corollary}[equation]{Corollary}

%%% Definition/remark type environments (which only differ
%%% from theorem type invironmants in the choices of fonts).  The
%%% numbering is still tied to the theorem counter.
\theoremstyle{definition}
\newtheorem{definition}[equation]{Definition}
\newtheorem{example}[equation]{Example}
\newtheorem{remark}[equation]{Remark}


\numberwithin{equation}{section} %numbers equations within sections.



\def\B{\mathbb{B}}
\def\C{\mathbb{C}}
\def\N{\mathbb{N}}
\def\D{\mathcal{D}}
\def\E{\mathcal{E}}
\def\F{\mathscr{F}}
\def\bQ{\mathbb{Q}}
\def\G{\mathscr{G}}
\def\H{\mathscr{H}}
\def\I{\mathscr{I}}
\def\II{\mathbb{I}}
\def\T{\mathcal{T}}
\def\R{\mathbb{R}}
\def\Rn{\mathbb{R}^n}
\def\Z{\mathcal{Z}}
\def\ZZ{\mathbb{Z}}
\def\J{\mathcal{J}}


\def\loc{\operatorname{loc}}
\def\diam{\operatorname{diam}}
\def\div{\operatorname{div}}
\def\supp{\operatorname{supp}}
\def\BMO{\operatorname{BMO}}
\def\CMO{\operatorname{CMO}}
\def\length{\operatorname{length}}
\def\Lip{\operatorname{Lip}}
\def\d{\operatorname{d}}
\def\rd{\operatorname{rd}}
\def\ird{\operatorname{ird}}
\def\rs{\operatorname{rs}}
\def\ch{\operatorname{ch}}
\def\sk{\operatorname{sk}}


\newcommand{\norm}[1]{\left\lVert#1\right\rVert}
\newcommand\restr[2]{\ensuremath{\left.#1\right|_{#2}}}
\DeclareMathOperator*{\esssup}{ess\,sup}
\DeclareMathOperator*{\essinf}{ess\,inf}


\renewcommand{\emptyset}{\text{\textup{\O}}}
\newcommand{\tinyemptyset}{\mbox{\tiny \textup{\O}}}


\begin{document}

\title[A compact $T1$ theorem]
{A compact $T1$ theorem for singular integrals associated with Zygmund dilations} 


\author[M. Cao]{Mingming Cao}
\address{Mingming Cao\\
Instituto de Ciencias Matem\'aticas CSIC-UAM-UC3M-UCM\\
Con\-se\-jo Superior de Investigaciones Cient{\'\i}ficas\\
C/ Nicol\'as Cabrera, 13-15\\
E-28049 Ma\-drid, Spain} \email{mingming.cao@icmat.es}



\author[J. Chen]{Jiao Chen}
\address{Jiao Chen \\
School of Mathematical Sciences\\ 
Chongqing Normal University\\ 
Chongqing 400000\\ 
People's Republic of China} \email{chenjiaobnu@163.com} 



\author[Z. Li]{Zhengyang Li}
\address{Zhengyang Li\\ 
School of Mathematics $\&$ Computational Science\\
Hunan University of Science and Technology\\
Xiangtan 411201\\ 
People's Republic of China} \email{zhengyli@hnust.edu.cn}



\author[F. Liao]{Fanghui Liao} 
\address{Fanghui Liao\\ 
School of Mathematics $\&$ Computational Science\\
Xiangtan University\\ 
Xiangtan 411105\\ 
People's Republic of China} \email{liaofh@xtu.edu.cn} 



\author[K. Yabuta]{K\^{o}z\^{o} Yabuta}
\address{K\^{o}z\^{o} Yabuta\\
Research Center for Mathematics and Data Science\\
Kwansei Gakuin University\\
Gakuen 2-1, Sanda 669-1337\\
Japan} \email{kyabuta3@kwansei.ac.jp}



\author[J. Zhang]{Juan Zhang}
\address{Juan Zhang\\
School of Science\\
Beijing Forestry University\\
Beijing, 100083 \\
People's Republic of China}\email{juanzhang@bjfu.edu.cn}




\thanks{The first author acknowledges financial support from Spanish Ministry of Science and Innovation through the Ram\'{o}n y Cajal  2021 (RYC2021-032600-I), through the ``Severo Ochoa Programme for Centres of Excellence in R\&D'' (CEX2019-000904-S), and through PID2019-107914GB-I00, and from the Spanish National Research Council through the ``Ayuda extraordinaria a Centros de Excelencia Severo Ochoa'' (20205CEX001). The third author was supported by National Natural Science Foundation of China (No. 12101222). The fourth author was supported Hunan Education Department Project, China(22B0155). The last author was supported by National Natural Science Foundation of China (No. 12101049). Part of this work was carried out while the first two authors were visiting the Hausdorff Center for Mathematics. The authors would like to express their gratitude to the HCM for their support.
}



\date{July 26, 2023}

\subjclass[2020]{42B20, 42B25}


%42B20 Singular and oscillatory integrals (Calderon-Zygmund, etc.)
%42B25 Maximal functions, Littlewood-Paley theory



\keywords{
Zygmund dilations, 
Singular integrals, 
Compactness,
$T1$ theorem, 
Dyadic analysis}


%%%%%%%%%%%%%%%%%%%%%% ABSTRACT ABSTRACT ABSTRACT %%%%%%%%%%%%%%%%%%%%%%%%
\begin{abstract}
We, for the first time, prove a compact version of $T1$ theorem for singular integrals of Zygmund type on $\mathbb{R}^3$. That is, if a singular integral operator $T$ associated with Zygmund dilations admits the compact full and partial kernel representations, and satisfies the weak compactness property and the cancellation condition, then $T$ can be extended to a compact operator on $L^p(\mathbb{R}^3)$ for all $p \in (1, \infty)$. Let $\theta \in (0, 1]$ be the kernel parameter, and let $A_{p, \mathcal{R}}$ and $A_{p, \mathcal{Z}}$ respectively denote the class of of strong $A_p$ weights and the class of $A_p$ weights adapted to Zygmund dilations. Under the same assumptions as above, we establish more general results: if $\theta \in (0, 1)$, $T$ is compact on $L^p(w)$ for all $p \in (1, \infty)$ and $w \in A_{p, \mathcal{R}}$; if $\theta=1$, $T$ is compact on $L^p(w)$ for all $p \in (1, \infty)$ and $w \in A_{p, \mathcal{Z}}$.
\end{abstract}
%%%%%%%%%%%%%%%%%%%%%%% ABSTRACT ABSTRACT ABSTRACT %%%%%%%%%%%%%%%%%%%%%%%


\maketitle
%\tableofcontents


%%%%%%%%%%%%%%%%%%%%%%%%%% SECTION SECTION SECTION %%%%%%%%%%%%%%%%%%%%%%
%%%%%%%%%%%%%%%%%%%%%%%%%% SECTION SECTION SECTION %%%%%%%%%%%%%%%%%%%%%%
\section{Introduction}


\subsection{Motivation and the main theorem} 
The exploration of singular integrals originated from the study of singular convolution operators, such as the Hilbert and Riesz transforms. Calder\'{o}n and Zygmund \cite{CZ} extended the concept of the Hilbert transform on $\R$ to the convolution type of singular integral operators on $\R^{n}$. The core of this theory lies in the invariance of regularity and cancellation conditions under the one-parameter dilation group on $\R^{n}$ defined by $\delta(x_1, \ldots, x_n) = (\delta x_1, \ldots, \delta x_n)$, $\delta>0$. Certainly, the classical singular integrals, maximal functions, and multipliers exhibit invariance under one-parameter dilations. 


However, the translation or dilation invariance is not essential for the theory. David and Journ\'{e} \cite{DJ} first formulated the famous $T1$ theorem, which gives a characterization of the $L^2$ boundedness for Calder\'{o}n-Zygmund operators, which is a class of operators of non-convolution type without either of the two invariances. Soon after, Journ\'{e} \cite{Jou} used the vector-valued Calder\'{o}n--Zygmund theory to  establish a bi-parameter $T1$ theorem for a class of non-convolution singular integral operators on product spaces. This generalized the multi-parameter theory due to Fefferman and Stein \cite{FS}, which studied convolution operators satisfying certain quantitative properties enjoyed by tensor products of operators of Calder\'{o}n-Zygmund type, as for instance the double Hilbert transform $\R^2$. Subsequently, Pott and Vilarroya \cite{PV} gave a new type of the $T1$ theorem on product spaces, where the vector-valued formulations are replaced by several new mixed type conditions. Interestingly, it was shown in \cite{G} that the vector-valued formulations of Journ\'{e} \cite{Jou} are equivalent to the full and partial kernel assumptions of Pott and Vilarroya \cite{PV} for bi-parameter singular integrals. Such result was extended to arbitrarily many parameter setting in \cite{Ou} using mixed type characterizing conditions. See \cite{CF80, CF85, Fef86, Fef87, Fef88, FL, MPTT, RS} for more developments about the multi-parameter theory. 


In terms of the multi-parameter theory, it is widely accepted that the Zygmund dilation %defined by 
\begin{align*}
(x_1, x_2, x_3) \mapsto (\delta_1 x_1, \delta_2 x_2, \delta_1 \delta_2 x_3), \qquad \delta_1, \delta_2 > 0, 
\end{align*}
is commonly acknowledged as the next simplest multi-parameter dilation group after the multi-parameter dilations. Stein \cite{Fef86} first established a correlation between the properties of maximal operators associated with Zygmund dilations and the boundary value problems pertaining to Poisson integrals on symmetric spaces. After that, Ricci and Stein \cite{RS} introduced a class of singular integrals with more general dilations and obtained the boundedness for multi-parameter singular integral operators. Fefferman and Pipher \cite{FP} further considered specific singular integral operators associated with Zygmund dilations and obtained their boundedness on weighted Lebesgue spaces. Recently, Han, et al. \cite{HLLT} introduced a particular category of singular integral operators associated with Zygmund dilations by providing appropriate versions of regularity conditions and cancellation conditions for the convolution kernel, which generalizes Nagel-Wainger singular integrals in \cite{NW} and  Fefferman-Pipher multipliers in \cite{FP}. In another work \cite{HLLTW}, they developed the multi-parameter theory on weighted Hardy spaces associated with Zygmund dilations, which, in particular, proved endpoint estimates for Ricci-Stein operators \cite{RS} and Fefferman-Pipher multipliers \cite{FP}. Until very recently, the authors \cite{HLMV} developed the dyadic multiresolution analysis and the related dyadic-probabilistic methods in the Zygmund dilation setting, which is quite useful to achieve weighted boundedness for singular integrals of Zygmund type. 


On the other hand, considering the compactness of singular integrals, Villarroya \cite{Vil} first gave a new $T1$ theorem to characterize the compactness for a class of singular integral operators. The study of the compactness of singular integrals has found the practical significance in the field of partial differential equations and geometric measure theory. For example, Fabes et al. \cite{FJR} applied the compactness of layer potentials to solve the Dirichlet and Neumann boundary value problems, while Hofmann et al. \cite[Section 4.6]{HMT} characterized regular SKT domains via the compactness of layer potentials and commutators. 


It has been an open problem in the last several years that how one can establish the $T1$ theorem to obtain the compactness of multi-parameter singular integrals. This problem was first solved by Cao, Yabuta, and Yang \cite{CYY} for bi-parameter singular integrals, which was further extended to arbitrarily many parameter in \cite{CLLYZ} and to the bilinear case in \cite{CLSY, CY}. The dyadic analysis plays a crucial part in the research of compactness. However, so far, there is no work about the compactness in the Zygmund dilation setting. Therefore, in this paper, our objective is to formulate a general framework encompassing a diverse set of singular integrals associated with Zygmund dilations and develop a comprehensive theory of compactness for these operators. 


To set the stage, let us give the definition of compactness of operators. Given normed spaces $\mathscr{X}$ and $\mathscr{Y}$, a linear operator $T: \mathscr{X} \to \mathscr{Y}$ is called \emph{compact} if for all bounded sets $A \subset \mathscr{X}$, the set $T(A)$ is relatively compact in $\mathscr{Y}$. Equivalently, $T$ is compact if for all bounded sequences $\{x_n\} \subset \mathscr{X}$, the sequence $\{T(x_n)\}$ has a convergent subsequence in $\mathscr{Y}$. More definitions and notation are presented in Sections \ref{sec:SIOZ} and \ref{sec:pre}. 


Our main result can be formulated as follows. 
%%%%%%%%%%%%%%%%%%%%%%% THEOREM THEOREM THEOREM %%%%%%%%%%%%%%%%%%%%%%%
\begin{theorem}\label{thm:compact} 
Let $T$ be a singular integral operator associated with Zygmund dilations, with parameter $\theta \in (0, 1]$ (cf. Definition \ref{def:SIO}). Assume that 
\begin{list}{\rm (\theenumi)}{\usecounter{enumi}\leftmargin=1.2cm \labelwidth=1cm \itemsep=0.2cm \topsep=.2cm \renewcommand{\theenumi}{H\arabic{enumi}}}

\item\label{list-1} $T$ admits the compact full kernel representations (cf. Definition \ref{def:SIO}), 

\item\label{list-2} $T$ admits the compact partial kernel representations (cf. Definition \ref{def:SIO}), 

\item\label{list-3} $T$ satisfies the weak compactness property (cf. Definition \ref{def:WCP}), 

\item\label{list-4} $T$ satisfies the cancellation condition (cf. Definition \ref{def:cancellation}). 
\end{list} 
Then the following statements hold: 
\begin{list}{\rm (\theenumi)}{\usecounter{enumi}\leftmargin=1.2cm \labelwidth=1cm \itemsep=0.2cm \topsep=.2cm \renewcommand{\theenumi}{\roman{enumi}}}

\item\label{list-01} If $\theta \in (0, 1)$, then $T$ is compact on $L^p(w)$ for all $p \in (1, \infty)$ and $w \in A_{p, \mathcal{R}}$. 

\item\label{list-02} If $\theta=1$, then $T$ is compact on $L^p(w)$ for all $p \in (1, \infty)$ and $w \in A_{p, \mathcal{Z}}$. 
\end{list} 
\end{theorem}
%%%%%%%%%%%%%%%%%%%%%%% THEOREM THEOREM THEOREM %%%%%%%%%%%%%%%%%%%%%%%%




%%%%%%%%%%%%%%%%%%%%% SUBSECTION SUBSECTION  SUBSECTION %%%%%%%%%%%%%%%%%%%%%
\subsection{Singular integrals of Zygmund type}\label{sec:SIOZ}
Having stated the main result, we present the precise definitions of all the previous concepts to understand what we are handling.  


%%%%%%%%%%%%%%%%%%%%%%% DEFINITION DEFINITION DEFINITION %%%%%%%%%%%%%%%%%%%%%
\begin{definition}\label{def:FF}
Let $\mathscr{F}$ consist of all triples $(F_1, F_2, F_3)$ of bounded functions $F_1, F_2, F_3: [0, \infty) \to [0, \infty)$ satisfying 
\begin{align*}
\lim_{t \to 0} F_1(t) 
=\lim_{t \to \infty} F_2(t) 
= \lim_{t \to \infty} F_3(t)
=0.  
\end{align*}
Let $\mathscr{F}_0$ be the collection of all bounded functions $F: \mathcal{I} \to [0, \infty)$ satisfying 
\begin{align*}
\lim_{\ell(I) \to 0} F(I)
= \lim_{\ell(I) \to \infty} F(I)
= \lim_{|c_I| \to \infty} F(I)
=0, 
\end{align*}
where $\mathcal{I}$ denotes the family of all intervals on $\R$. 
\end{definition}
%%%%%%%%%%%%%%%%%%%%%%% DEFINITION DEFINITION DEFINITION %%%%%%%%%%%%%%%%%%%%%



%Given a linear operator $T$ acting on functions defined on $\R^1 \times \R^2$, we define its \emph{adjoint} $T^*$ and its \emph{partial adjoint} $T_1^*$ and $T_{2, 3}^*$ by
%\begin{align*}
%\langle T(f_1 \otimes f_{2, 3}), g_1 \otimes g_{2, 3} \rangle
%&=\langle T^*(g_1 \otimes g_{2, 3}), f_1 \otimes f_{2, 3} \rangle 
%\\
%&=\langle T_1^*(g_1 \otimes f_{2, 3}), f_1 \otimes g_{2, 3} \rangle
%=\langle T_{2, 3}^*(f_1 \otimes g_{2, 3}), g_1 \otimes f_{2, 3} \rangle.
%\end{align*}



Throughout this article, we always assume that $\delta_1, \delta_{2, 3} \in (0, 1]$ are fixed numbers. Given a parameter $\theta \in (0, 1]$, let 
\begin{align*}
D_{\theta}(x) := \bigg(\frac{|x_1 x_2|}{|x_3|} + \frac{|x_3|}{|x_1 x_2|} \bigg)^{-\theta}, 
\end{align*}
for any $x =(x_1, x_2, x_3) \in \R^3 \setminus \{x: x_1 x_2 x_3 = 0 \}$. Next, we proceed to give the compact full and partial kernel representations, and then define singular integral operators associated with Zygmund dilations.  

%%%%%%%%%%%%%%%%%%%%%%% DEFINITION DEFINITION DEFINITION %%%%%%%%%%%%%%%%%%%%%
\begin{definition}\label{def:full}
A linear operator $T$ admits a \emph{compact full kernel representation on the parameters $\{1\}$ and $\{2, 3\}$} if the following hold. 
If $f = f_1 \otimes f_{2, 3}$ and $g = g_1 \otimes g_{2, 3}$ with $f_1, g_1 :\R \rightarrow \C$, $f_{2, 3}, g_{2, 3} :\R^2 \rightarrow \C$, 
$\supp(f_1) \cap \supp(g_1) = \emptyset$, and $\supp(f_{2, 3}) \cap \supp(g_{2, 3}) = \emptyset$, then 
\begin{align*}
\langle Tf, g \rangle 
= \int_{\R^3} \int_{\R^3} K(x, y) f(y) g(x) \, dx \, dy,
\end{align*}
where the kernel $K: (\R^3 \times \R^3) \setminus \big\{(x, y) \in \R^3 \times \R^3: x_1=y_1 \text{ or } x_2=y_2 \text{ or } x_3=y_3\big\} \rightarrow \C$ satisfies 
\begin{list}{\rm (\theenumi)}{\usecounter{enumi}\leftmargin=1.2cm \labelwidth=1cm \itemsep=0.2cm \topsep=.2cm \renewcommand{\theenumi}{\arabic{enumi}}} 

\item\label{full-1} the size condition
\begin{align*}
|K(x, y)| \leq D_{\theta}(x-y) \prod_{i=1}^3 \frac{F_i(x_i, y_i)}{|x_i - y_i|}. 
\end{align*}
 

\item\label{full-2} the H\"{o}lder conditions
\begin{align*}
&|K(x, y) - K((x_1, x'_2, x'_3), y) - K((x'_1, x_2, x_3), y) + K(x', y)| 
\\
&\quad\leq \bigg(\frac{|x_1 - x'_1|}{|x_1 - y_1|}\bigg)^{\delta_1} 
\bigg(\frac{|x_2 - x'_2|}{|x_2 - y_2|} + \frac{|x_3 - x'_3|}{|x_3 - y_3|}\bigg)^{\delta_{2, 3}}
D_{\theta}(x-y) \prod_{i=1}^3 \frac{F_i(x_i, y_i)}{|x_i - y_i|}, 
\end{align*}
whenever $|x_i - x'_i| \leq |x_i - y_i|/2$ for $i=1, 2, 3$, 
\begin{align*}
&|K(x, y) - K(x, (y_1, y'_2, y'_3)) - K(x, (y'_1, y_2, y_3)) + K(x, y')| 
\\
&\quad\leq \bigg(\frac{|y_1 - y'_1|}{|x_1 - y_1|}\bigg)^{\delta_1} 
\bigg(\frac{|y_2 - y'_2|}{|x_2 - y_2|} + \frac{|y_3 - y'_3|}{|x_3 - y_3|}\bigg)^{\delta_{2, 3}}
D_{\theta}(x-y) \prod_{i=1}^3 \frac{F_i(x_i, y_i)}{|x_i - y_i|}, 
\end{align*}
whenever $|y_i - y'_i| \leq |x_i - y_i|/2$ for $i=1, 2, 3$, 
\begin{align*}
&|K(x, y) - K((x_1, x'_2, x'_3), y) - K(x, (y'_1, y_2, y_3)) + K(x_1, x'_2, x'_3, y'_1, y_2, y_3)| 
\\
&\quad\leq \bigg(\frac{|y_1 - y'_1|}{|x_1 - y_1|}\bigg)^{\delta_1} 
\bigg(\frac{|x_2 - x'_2|}{|x_2 - y_2|} + \frac{|x_3 - x'_3|}{|x_3 - y_3|}\bigg)^{\delta_{2, 3}}
D_{\theta}(x-y) \prod_{i=1}^3 \frac{F_i(x_i, y_i)}{|x_i - y_i|}, 
\end{align*}
whenever $|y_1 - y'_1| \leq |x_1 - y_1|/2$ and $|x_i - x'_i| \leq |x_i - y_i|/2$ for $i=2, 3$, 
\begin{align*}
&|K(x, y) - K(x, (y_1, y'_2, y'_3)) - K((x'_1, x_2, x_3), y) + K((x'_1, x_2, x_3), (y_1, y'_2, y'_3))| 
\\
&\quad\leq \bigg(\frac{|x_1 - x'_1|}{|x_1 - y_1|}\bigg)^{\delta_1} 
\bigg(\frac{|y_2 - y'_2|}{|x_2 - y_2|} + \frac{|y_3 - y'_3|}{|x_3 - y_3|}\bigg)^{\delta_{2, 3}}
D_{\theta}(x-y) \prod_{i=1}^3 \frac{F_i(x_i, y_i)}{|x_i - y_i|}, 
\end{align*}
whenever $|x_1 - x'_1| \leq |x_1 - y_1|/2$ and $|y_i - y'_i| \leq |x_i - y_i|/2$ for $i=2, 3$.  

 
\item\label{full-3} the mixed size-H\"{o}lder conditions 
\begin{align*}
|K(x, y) - K((x'_1, x_2, x_3), y)|
\leq \bigg(\frac{|x_1 - x'_1|}{|x_1 - y_1|}\bigg)^{\delta_1} 
D_{\theta}(x-y) \prod_{i=1}^3 \frac{F_i(x_i, y_i)}{|x_i - y_i|}, 
\end{align*}
whenever $|x_1 - x'_1| \leq |x_1-y_1|/2$,  
\begin{align*}
|K(x, y) - K((x_1, x'_2, x'_3), y)|
\le \bigg(\frac{|x_2 - x'_2|}{|x_2 - y_2|} + \frac{|x_3 - x'_3|}{|x_3 - y_3|}\bigg)^{\delta_{2, 3}}
D_{\theta}(x-y) \prod_{i=1}^3 \frac{F_i(x_i, y_i)}{|x_i - y_i|}, 
\end{align*}
whenever $|x_2 - x'_2| \leq |x_2 - y_2|/2$ and $|x_3 - x'_3| \leq |x_3 - y_3|/2$, 
\begin{align*}
|K(x, y) - K(x, (y'_1, y_2, y_3))|
\leq \bigg(\frac{|y_1 - y'_1|}{|x_1 - y_1|}\bigg)^{\delta_1} 
D_{\theta}(x-y) \prod_{i=1}^3 \frac{F_i(x_i, y_i)}{|x_i - y_i|}, 
\end{align*}
whenever $|y_1 - y'_1| \leq |x_1-y_1|/2$, and 
\begin{align*}
|K(x, y) - K(x, (y_1, y'_2, y'_3))|
\le \bigg(\frac{|y_2 - y'_2|}{|x_2 - y_2|} + \frac{|y_3 - y'_3|}{|x_3 - y_3|}\bigg)^{\delta_{2, 3}}
D_{\theta}(x-y) \prod_{i=1}^3 \frac{F_i(x_i, y_i)}{|x_i - y_i|}, 
\end{align*}
whenever $|y_2 - y'_2| \leq |x_2 - y_2|/2$ and $|y_3 - y'_3| \leq |x_3 - y_3|/2$. 

\item\label{full-4} the function $F_i$ above is given by 
\begin{align*}
F_i(x_i, y_i) := F_{i, 1}(|x_i - y_i|) F_{i, 2}(|x_i - y_i|) F_{i, 3}(|x_i + y_i|),  
\end{align*}
where $(F_{i, 1}, F_{i, 2}, F_{i, 3}) \in \mathscr{F}$, $i=1, 2, 3$.
\end{list}
\end{definition}
%%%%%%%%%%%%%%%%%%%%%%% DEFINITION DEFINITION DEFINITION %%%%%%%%%%%%%%%%%%%%%


Similarly, one can define the \emph{compact full kernel representation on the parameters $\{2\}$ and $\{1, 3\}$}. 


%%%%%%%%%%%%%%%%%%%%%%% DEFINITION DEFINITION DEFINITION %%%%%%%%%%%%%%%%%%%%%
\begin{definition}\label{def:partial-1}
A linear operator $T$ admits a \emph{compact partial kernel representation on the parameter $\{1\}$} if the following hold. 
If $f = f_1 \otimes f_{2, 3}$ and $g = g_1 \otimes g_{2, 3}$ with $\supp(f_1) \cap \supp(g_1) = \emptyset$, then 
\begin{align*}
\langle Tf, g \rangle 
= \int_{\R} \int_{\R} K_{f_{2, 3}, g_{2, 3}}(x_1, y_1) f_1(y_1) g_1(x_1) \, dx_1 \, dy_1, 
\end{align*}
where the kernel $K_{f_{2, 3}, g_{2, 3}}: (\R \times \R) \setminus \big\{(x_1, y_1) \in \R \times \R: x_1 = y_1 \big\} \rightarrow \C$ satisfies  
\begin{list}{\rm (\theenumi)}{\usecounter{enumi}\leftmargin=1.2cm \labelwidth=1cm \itemsep=0.2cm \topsep=.2cm \renewcommand{\theenumi}{\arabic{enumi}}} 

\item\label{partial-11} the size condition
\begin{align*}
|K_{f_{2, 3}, g_{2, 3}}(x_1,y_1)| 
\leq C(f_{2, 3}, g_{2, 3}) \frac{F_1(x_1, y_1)}{|x_1-y_1|}. 
\end{align*}

\item\label{partial-12} the H\"{o}lder conditions
\begin{align*}
|K_{f_{2, 3}, g_{2, 3}}(x_1, y_1) - K_{f_{2, 3}, g_{2, 3}}(x'_1, y_1)|
\leq C(f_{2, 3}, g_{2, 3}) F_1(x_1, y_1) \frac{|x_1-x'_1|^{\delta_1}}{|x_1-y_1|^{1+\delta_1}}
\end{align*}
whenever $|x_1-x'_1| \leq |x_1-y_1|/2$, and 
\begin{align*}
|K_{f_{2, 3}, g_{2, 3}}(x_1, y_1) - K_{f_{2, 3}, g_{2, 3}}(x_1, y'_1)|
\leq C(f_{2, 3}, g_{2, 3}) F_1(x_1, y_1) \frac{|y_1-y'_1|^{\delta_1}}{|x_1-y_1|^{1+\delta_1}}
\end{align*}
whenever $|y_1-y'_1| \leq |x_1-y_1|/2$.

\item\label{partial-13} the function $F_1$ above is given by 
\begin{align*}
F_1(x_1, y_1) := F_{1, 1}(|x_1 - y_1|) F_{1, 2}(|x_1 - y_1|) F_{1, 3}(|x_1 + y_1|), 
\end{align*}
where $(F_{1, 1}, F_{1, 2}, F_{1, 3}) \in \mathscr{F}$.

\item\label{partial-14} the bound $C(f_{2, 3}, g_{2, 3})$ above verifies  
\begin{align*}
C(\mathbf{1}_{I_{2, 3}}, \mathbf{1}_{I_{2, 3}}) 
+ C(\mathbf{1}_{I_{2, 3}}, a_{I_{2, 3}}) 
+ C(a_{I_{2, 3}}, \mathbf{1}_{I_{2, 3}}) 
\le F_2(I_2) \, |I_2| \, F_3(I_3) \, |I_3|,  
\end{align*}
for all rectangles $I_{2, 3} \subset \R^2$ and all functions $a_{I_{2, 3}}$ satisfying $\supp(a_{I_{2, 3}}) \subset I_{2, 3}$, $|a_{I_{2, 3}}| \le 1$, and $\int a_{I_{2, 3}} \, dx_{2, 3}=0$, where $F_2, F_3 \in \mathscr{F}_0$. 
\end{list} 
\end{definition}
%%%%%%%%%%%%%%%%%%%%%%% DEFINITION DEFINITION DEFINITION %%%%%%%%%%%%%%%%%%%%%


Analogously, one can define the \emph{compact partial kernel representation on the parameter $\{2\}$}. 


%%%%%%%%%%%%%%%%%%%%%%% DEFINITION DEFINITION DEFINITION %%%%%%%%%%%%%%%%%%%%%
\begin{definition}\label{def:partial-2}
A linear operator $T$ admits a \emph{compact partial kernel representation on the parameter $\{2, 3\}$} if the following hold. 
If $f = f_1 \otimes f_{2, 3}$ and $g = g_1 \otimes g_{2, 3}$ with $\supp(f_{2, 3}) \cap \supp(g_{2, 3}) = \emptyset$, then 
\begin{align*}
\langle Tf, g \rangle 
= \int_{\R^2} \int_{\R^2} K_{f_1, g_1}(x_{2, 3}, y_{2, 3}) f_{2, 3}(y_{2, 3}) g_{2, 3}(x_{2, 3}) \, dx_{2, 3} \, dy_{2, 3}, 
\end{align*}
where the kernel $K_{f_1, g_1}: (\R^2 \times \R^2) \setminus \big\{(x_{2, 3}, y_{2, 3}) \in \R^2 \times \R^2: x_2 = y_2 \text{ or } x_3=y_3 \big\} \rightarrow \C$ satisfies  
\begin{list}{\rm (\theenumi)}{\usecounter{enumi}\leftmargin=1.2cm \labelwidth=1cm \itemsep=0.2cm \topsep=.2cm \renewcommand{\theenumi}{\arabic{enumi}}} 

\item\label{partial-21} the size condition
\begin{align*}
|K_{f_1, g_1}(x_{2, 3}, y_{2, 3})| 
\leq C(f_1, g_1) D_{\theta}(t, x_{2, 3} - y_{2, 3}) 
\prod_{i=2}^3 \frac{F_i(x_i, y_i)}{|x_i-y_i|}, 
\end{align*}
where $t:= |\supp f_1 \cup \supp g_1|$.

\item\label{partial-22} the H\"{o}lder conditions
\begin{align*}
&|K_{f_1, g_1}(x_{2, 3}, y_{2, 3}) - K_{f_1, g_1}(x'_{2, 3}, y_{2, 3})|
\\ 
&\quad\leq C(f_1, g_1) \bigg(\frac{|x_2 - x'_2|}{|x_2 - y_2|} + \frac{|x_3 - x'_3|}{|x_3 - y_3|}\bigg)^{\delta_{2, 3}} 
D_{\theta}(t, x_{2, 3} - y_{2, 3}) \prod_{i=2}^3 \frac{F_i(x_i, y_i)}{|x_i-y_i|} 
\end{align*}
whenever $x'_{2, 3} =(x'_2, x'_3)$ satisfies $|x_2-x'_2| \leq |x_2-y_2|/2$ and $|x_3-x'_3| \leq |x_3-y_3|/2$, and 
\begin{align*}
&|K_{f_1, g_1}(x_{2, 3}, y_{2, 3}) - K_{f_1, g_1}(x_{2, 3}, y'_{2, 3})|
\\ 
&\quad\leq C(f_1, g_1) \bigg(\frac{|y_2 - y'_2|}{|x_2 - y_2|} + \frac{|y_3 - y'_3|}{|x_3 - y_3|}\bigg)^{\delta_{2, 3}} 
D_{\theta}(t, x_{2, 3} - y_{2, 3}) \prod_{i=2}^3 \frac{F_i(x_i, y_i)}{|x_i-y_i|} 
\end{align*}
whenever $y'_{2, 3} =(y'_2, y'_3)$ satisfies $|y_2-y'_2| \leq |x_2-y_2|/2$ and $|y_3-y'_3| \leq |x_3-y_3|/2$. 

\item\label{partial-23} the function $F_1$ above is given by 
\begin{align*}
F_1(x_1, y_1) := F_{1, 1}(|x_1 - y_1|) F_{1, 2}(|x_1 - y_1|) F_{1, 3}(|x_1 + y_1|), 
\end{align*}
where $(F_{1, 1}, F_{1, 2}, F_{1, 3}) \in \mathscr{F}$.

\item\label{partial-24} the bound $C(f_1, g_1)$ above verifies  
\begin{align*}
C(\mathbf{1}_{I_1}, \mathbf{1}_{I_1}) 
+ C(\mathbf{1}_{I_1}, a_{I_1}) 
+ C(a_{I_1}, \mathbf{1}_{I_1}) 
\le F_1(I_1) \, |I_1|, 
\end{align*}
for all intervals $I_1 \subset \R$ and all functions $a_{I_1}$ satisfying $\supp(a_{I_1}) \subset I_1$, $|a_{I_1}| \le 1$, and $\int a_{I_1} \, dx_1=0$, where $F_1 \in \mathscr{F}_0$. 
\end{list} 
\end{definition}
%%%%%%%%%%%%%%%%%%%%%%% DEFINITION DEFINITION DEFINITION %%%%%%%%%%%%%%%%%%%%%


Similarly, one can define the \emph{compact partial kernel representation on the parameter $\{1, 3\}$}. 


%%%%%%%%%%%%%%%%%%%%%% DEFINITION DEFINITION DEFINITION %%%%%%%%%%%%%%%%%%%%%%
\begin{definition}\label{def:SIO}
Let $T$ be a linear operator. 
\begin{itemize}

\item We say that $T$ admits the \emph{compact full kernel representation} if it admits the compact full kernel representations on the parameters $\{1\}$, $\{2, 3\}$, $\{2\}$, and $\{1, 3\}$. 

\item We say that $T$ admits the \emph{compact partial kernel representation} if it admits the compact partial kernel representation on the parameters $\{1\}$, $\{2, 3\}$, $\{2\}$, and $\{1, 3\}$. 

\item We say that $T$ admits the \emph{full kernel representation} if both $F_1$ and $F_2$ in the compact full kernel representation are replaced by a uniform constant $C \ge 1$. 

\item We say that $T$ admits the \emph{partial kernel representation} if both $F_1$ and $F_2$ in the compact partial kernel representation are replaced by a uniform constant $C \ge 1$. 

\item $T$ is called a \emph{singular integral operator associated with Zygmund dilations} if $T$ admits the full and partial kernel representations. 
\end{itemize}
\end{definition}
%%%%%%%%%%%%%%%%%%%%%% DEFINITION DEFINITION DEFINITION %%%%%%%%%%%%%%%%%%%%%%

Next, let us give compactness and cancellation assumptions for singular integrals associated with Zygmund dilations. 

%%%%%%%%%%%%%%%%%%%%%% DEFINITION DEFINITION DEFINITION %%%%%%%%%%%%%%%%%%%%%%
\begin{definition}\label{def:WCP}
We say that $T$ satisfies the \emph{weak compactness property} if 
\begin{align*}
|\langle T\mathbf{1}_I, \mathbf{1}_I \rangle|
\leq F_1(I_1) F_2(I_2) F_3(I_3) \, |I|, 
\end{align*}
for all Zygmund rectangles $I=I_1 \times I_2 \times I_3$, where $F_1, F_2, F_3 \in \mathscr{F}_0$. If $F_1$, $F_2$, and $F_3$ above are replaced by a uniform constant $C \ge 1$, we say that $T$ satisfies the \emph{weak boundedness property}.
\end{definition}
%%%%%%%%%%%%%%%%%%%%%% DEFINITION DEFINITION DEFINITION %%%%%%%%%%%%%%%%%%%%%%



%%%%%%%%%%%%%%%%%%%%%% DEFINITION DEFINITION DEFINITION %%%%%%%%%%%%%%%%%%%%%%
\begin{definition}\label{def:cancellation}
We say that $T$ satisfies the \emph{cancellation condition} if   
\begin{align*}
\langle &T(1 \otimes \mathbf{1}_{I_{2, 3}}), h_{J_1} \otimes \mathbf{1}_{J_{2, 3}} \rangle
=\langle T(h_{I_1} \otimes \mathbf{1}_{I_{2, 3}}), 1 \otimes \mathbf{1}_{J_{2, 3}} \rangle
\\
&=\langle T(\mathbf{1}_{I_1} \otimes 1), \mathbf{1}_{J_1} \otimes h_{J_{2, 3}} \rangle
=\langle T(\mathbf{1}_{I_1} \otimes h_{I_{2, 3}}), \mathbf{1}_{J_1} \otimes 1 \rangle
\\
&=\langle T(1 \otimes \mathbf{1}_{I_{1, 3}}), h_{J_2} \otimes \mathbf{1}_{J_{1, 3}} \rangle
=\langle T(h_{I_2} \otimes \mathbf{1}_{I_{1, 3}}), 1 \otimes \mathbf{1}_{J_{1, 3}} \rangle
\\
&=\langle T(\mathbf{1}_{I_2} \otimes 1), \mathbf{1}_{J_2} \otimes h_{J_{1, 3}} \rangle
=\langle T(\mathbf{1}_{I_2} \otimes h_{I_{1, 3}}), \mathbf{1}_{J_2} \otimes 1 \rangle
=0
\end{align*}
for all rectangles $I=I_1 \times I_2 \times I_3=: I_1 \times I_{2, 3} =: I_2 \times I_{1, 3}$ and $J=J_1 \times J_2 \times J_3=: J_1 \times J_{2, 3} =: J_2 \times J_{1, 3}$.
\end{definition}
%%%%%%%%%%%%%%%%%%%%%% DEFINITION DEFINITION DEFINITION %%%%%%%%%%%%%%%%%%%%%%





This paper is organized as follows. Section \ref{sec:pre} contains some preliminaries, notation, and definitions. Some auxiliary results are presented in Section \ref{sec:aux} to show Theorem \ref{thm:compact}. Section \ref{sec:reduction} is devoted to giving crucial reductions of Theorem \ref{thm:compact}. In Section \ref{sec:proof}, we give main estimates for matrix elements involved in the expansion of $\langle Tf, g \rangle$.   






%%%%%%%%%%%%%%%%%%%%%%%%%%%% SECTION SECTION SECTION %%%%%%%%%%%%%%%%%%%%%
%%%%%%%%%%%%%%%%%%%%%%%%%%%% SECTION SECTION SECTION %%%%%%%%%%%%%%%%%%%%%
\section{Preliminaries}\label{sec:pre}

%%%%%%%%%%%%%%%%%%%%%%%% SUBSECTION SUBSECTION SUBSECTION %%%%%%%%%%%%%%%%%%
\subsection{Notation}
\begin{itemize}
\item For convenience, we always use the $\ell^{\infty}$ metric on $\Rn$.

\item Given a measurable set $E \subset \Rn$, let $|E|$ denote the Lebesgue measure of $E$. 

\item Given a cube $I \subset \Rn$, we denote its center by $c_I$ and its side length by $\ell(I)$. For any $\lambda>0$, we denote by $\lambda I$ the cube with the center $c_I$ and side length $\lambda \ell(I)$. 

\item Write $\II := [-\frac12, \frac12]$ and $\lambda \II := [-\frac{\lambda}{2}, \frac{\lambda}{2}]$. 

\item We will use the symbol $\uplus$ to denote a union comprised of pairwise disjoint sets. 

\item Given a dyadic grid $\D$ on $\Rn$, $I \in \D$, and $k \in \N$, we set $\ch(I):=\{I' \in \D: I' \subset I, \ell(I')=\ell(I)/2\}$ and let $I^{(k)}$ denote the unique dyadic cube $I'' \in \D$ so that $I \subset I''$ and $\ell(I'')=2^k \ell(I)$. 

\item Given cubes $I, J \subset \Rn$, we denote by $I \vee J$ the cube with minimal sidelength containing $I \cup J$.  If there is more than one cube satisfying this condition, we will simply select one. 

\item Given nonempty subsets $E, F \subset \Rn$, we denote the distance between them by $\d(E, F) := \inf\{|x-y|: x \in E, y \in F\}$.

\item Given $f \in L^1_{\loc}(\Rn)$ and a measurable set $E \subset \Rn$ with $0<|E|<\infty$, we write $\langle f \rangle_E := \frac{1}{|E|} \int_E f \, dx$. 

\item The relative size between cubes $I$ and $J$ in $\Rn$ is defined as 
\begin{align*}
\rs(I, J) := \frac{\min\{\ell(I), \ell(J)\}}{\max\{\ell(I), \ell(J)\}}.
\end{align*}

\item Given rectangles $I=I_1 \times I_2 \times I_3$ and $J=J_1 \times J_2 \times J_3$ in $\R^3$, we respectively define the relative distance between them by 
\begin{align*}
&\rd(I_i, J_i) := \frac{\d(I_i, J_i)}{\max\{\ell(I_i), \, \ell(J_i)\}}, \quad i=1, 2, 3,
\\
&\rd(I_{2, 3}, J_{2, 3}) :=\max\{\rd(I_2, J_2), \rd(I_3, J_3)\}.
\end{align*}

\item Let $\mathcal{R}$ denote the collection of all rectangles in $\R^3$ with sides parallel to the coordinate axes. 

\item Let $\Z=\{I \in \mathcal{R}: \ell(I_3) = \ell(I_1) \ell(I_2)\}$ denote the collection of all Zygmund rectangles. 


\item We shall use $a\lesssim b$ and $a \simeq b$ to mean, respectively, that $a\leq C b$ and $0<c\leq a/b\leq C$, where the constants $c$ and $C$ are harmless positive constants, not necessarily the same at each occurrence, which depend only on dimension and the constants appearing in the hypotheses of theorems. 
\end{itemize} 



\subsection{Haar functions}




Given a rectangle $I = I_1 \times \cdots \times I_n \subset \Rn$ and $\eta=(\eta_1, \ldots, \eta_n) \in \{0, 1\}^n$, the \emph{Haar function} $h_I^{\eta}$ is defined by $h_I^{\eta} = h_{I_1}^{\eta_1} \otimes \cdots \otimes h_{I_n}^{\eta_n}$, where $h_{I_i}^0 = |I_i|^{-\frac12} \mathbf{1}_{I_i}$ and $h_{I_i}^1 = |I_i|^{-\frac12}(\mathbf{1}_{I_i^-} - \mathbf{1}_{I_i^+})$ for every $i = 1, \ldots, n$. Here $I_i^-$ and $I_i^+$ are the left and right halves of the interval $I_i$ respectively. Note that for any $\eta \neq 0$, the Haar function is cancellative : $\int_{\Rn} h_I^{\eta} \, dx = 0$. We usually suppress the presence of $\eta$ and simply write $h_I$ for some $h_I^{\eta}$, $\eta \neq 0$. 

If we view $\Rn$ as the bi-parameter product space $\Rn = \R^{n_1} \times \R^{n_2}$, and $I_1 \times I_2 \subset \R^{n_1} \times \R^{n_2}$, then $h_{I_1 \times I_2}$ still means just the one-parameter Haar function: $h_{I_1 \times I_2} = h^{\eta}_{I_1 \times I_2}$ for some $\eta \in \{0, 1\}^n \setminus \{0\}$. It is not the bi-parameter Haar function on the rectangle $I_1 \times I_2$, which should have the different form $h^{\alpha}_{I_1} \otimes h^{\beta}_{I_2}$ for some $\alpha \in \{0, 1\}^{n_1} \setminus \{0\}$ and $\beta \in \{0, 1\}^{n_2} \setminus \{0\}$. In what follows, when we need a product object, we write it explicitly with this tensor.

Given a dyadic grid $\D$ on $\Rn$ and $I \in \D$, we define the \emph{averaging operator} $E_I$ and the \emph{martingale difference} $\Delta_I$ by 
\begin{align*}
E_I f := \langle f \rangle_I \mathbf{1}_I
\quad\text{ and }\quad 
\Delta_I f := \sum_{I' \in \ch(I)} \big(\langle f \rangle_{I'} - \langle f \rangle_I\big) \mathbf{1}_{I'}.
\end{align*}
Then one can see that 
\begin{align}\label{eq:EDF}
E_I f = \langle f, h_I^0 \rangle h_I^0 
\quad\text{ and }\quad 
\Delta_I f = \sum_{\eta \in \{0, 1\}^n \setminus \{0\}} \langle f, h_I^{\eta} \rangle h_I^{\eta}
=: \langle f, \, h_I \rangle \, h_I. 
\end{align}
where we suppress the finite $\eta$ summation in the last step. Since all the cancellative Haar functions on $\Rn$ form an orthonormal basis of $L^2(\Rn)$, for each $f \in L^2(\Rn)$, we may write
\begin{align}\label{eq:fhh} 
f = \sum_{I \in \D}  \langle f, h_I \rangle \, h_I
=\sum_{I \in \D} \Delta_I f.
\end{align}



\subsection{Resolution of functions on $\R^3$} 
We are working on $\R^3$ in this subsection. Given dyadic grids $\D^1$, $\D^2$, and $\D^3$ on $\R$, we define $\D := \D^1 \times \D^2 \times \D^3$ and $\D_{\Z} := \{I=I_1 \times I_2 \times I_3 \in \D: \ell(I_3) = \ell(I_1) \ell(I_2)\}$. Given $I=I_1 \times I_2 \times I_3 =: I_1 \times I_{2, 3}\in \D_{\Z}$, we define the \emph{Zygmund martingale difference} $\Delta_{I, \Z} := \Delta_{I_1} \Delta_{I_{2, 3}} f$, where 
\begin{align*}
\Delta_{I_{2, 3}} f
:= \sum_{I'_2 \in \ch(I_2) \atop I'_3 \in \ch(I_3)} 
\big(\langle f \rangle_{I'_2 \times I'_3} - \langle f \rangle_{I_2 \times I_3} \big) \mathbf{1}_{I'_2 \times I'_3} .
\end{align*}
Note that $\Delta_{2, 3}$ is the one-parameter martingale difference operator on the rectangle $I_{2, 3}$, which is quite different from the bi-parameter martingale difference operator $\Delta_{I_2} \Delta_{I_3}$. In addition, it is easy to check that for all $I, J \in \D_{\Z}$, 
\begin{align*}
\Delta_{I, \Z} \Delta_{J, \Z} = \Delta_{I, \Z} \mathbf{1}_{\{I=J\}}, \qquad 
\int_{\R} \Delta_{I, \Z} f \, dx_1 = 0 
= \int_{\R^2} \Delta_{I, \Z} f \, dx_2 \, dx_3, 
\end{align*}
and 
\begin{align*}
\Delta_{I, \Z} f 
=\Delta_{I_1} \Delta_{I_{2, 3}} f 
= \langle f, h_{I_1} \otimes h_{I_{2, 3}} \rangle h_{I_1} \otimes h_{I_{2, 3}}
=: \langle f, h_{I, \Z} \rangle h_{I, \Z}. 
\end{align*}
For any $\lambda=2^k$ with $k \in \mathbb{Z}$, we define the dilated grid  
\begin{align*}
\D_{\lambda}^{2, 3} 
:= \{I_{2, 3} \in \D^{2, 3} = \D^2 \times \D^3: \ell(I_3) = \lambda \ell(I_2)\}.
\end{align*}
By means of these notation and \eqref{eq:fhh}, we may expand a function $f \in L^2(\R^3)$ as 
\begin{align}\label{eq:expand}
f = \sum_{I_1 \in \D^1} \Delta_{I_1} f 
= \sum_{I_1 \in \D^1} \sum_{I_{2, 3} \in \D^{2, 3}_{\ell(I_1)}} \Delta_{I_1} f  \Delta_{I_{2, 3}} f 
=\sum_{I \in \D_{\Z}} \Delta_{I, \Z} f. 
\end{align}

 


\subsection{Weights}
A measurable function $w$ on $\R^{n_1} \times \R^{n_2}$ is called a weight if it is a.e. positive and locally integrable. Given $p \in (1, \infty)$, we say that a weight $w$ on $\R^{n_1} \times \R^{n_2}$ belongs to the bi-parameter weight class $A_p(\R^{n_1} \times \R^{n_2})$ if
\begin{align*}
[w]_{A_p(\R^{n_1} \times \R^{n_2})} 
:= \sup_{R} \bigg(\fint_R w \, dx \bigg) \bigg(\fint_R w^{-\frac{1}{p-1}} dx \bigg)^{p-1} < \infty,  
\end{align*}
where the supremum is taken over rectangles $R=I_1 \times I_2$, where $I_i$ is a cube in $\R^{n_i}$, $i=1, 2$. 


Let us record an interpolation theorem for compact operators below, which is a particular case of \cite[Theorem 9]{CK}. 

%%%%%%%%%%%%%%%%%%%%%%% THEOREM THEOREM THEOREM %%%%%%%%%%%%%%%%%%%%%%%%
\begin{theorem}\label{thm:inter}
Let $1<p_i<\infty$ and $w_i \in A_{p_i}(\R^3)$, $i=0, 1$. Assume that $T$ is a linear operator such that $T$ is bounded on $L^{p_0}(w_0)$ and compact on $L^{p_1}(w_1)$. Then $T$ is compact on $L^p(w)$, where $\frac1p = \frac{1-\theta}{p_0} + \frac{\theta}{p_1}$, $w^{\frac1p} = w_0^{\frac{1-\theta}{p_0}}  w_1^{\frac{\theta}{p_1}}$, and $0<\theta<1$.  
\end{theorem}
%%%%%%%%%%%%%%%%%%%%%%% THEOREM THEOREM THEOREM %%%%%%%%%%%%%%%%%%%%%%%%




\section{Auxiliary results}\label{sec:aux} 
The goal of this section is to present some useful estimates for integrals, which will appear in the expansion of $\langle Tf, g \rangle$.


\begin{lemma}\label{lem:JJA}
Let $A \ge 1$ and $B \ge 0$. Let $\delta_1, \delta_{2, 3} \in (0, 1]$, $0<\theta < \min\{\delta_1, \delta_{2, 3}\}$, and $\alpha>0$. Write   $\delta_0 :=\min\{\theta, \, \delta_1 - \theta, \, \delta_{2, 3} - \theta\}$. Then for every $k=2, 3$,  
\begin{align}
\label{JJA-1} \mathscr{I}_k(A, B) 
&:= \sum_{\substack{k_2 \ge k_3 \ge 0 \\ j_2, j_3 \ge 1 \\ j_k \ge A, \, k_2 \ge B}} 2^{-(k_2-k_3) \alpha} j_2^{-1} j_3^{-1}
\frac{(2^{-k_2} j_2^{-1} + 2^{-k_3} j_3^{-1})^{\delta_{2, 3}}}{(j_2 j_3^{-1} + j_2^{-1} j_3)^{\theta}} 
\nonumber \\
&\lesssim A^{-\delta_0} 2^{-B \min\{\delta_0, \alpha\}} (B+1), 
\end{align}
and for every $i=1, 2, 3$, 
\begin{align}
\label{JJA-2} \mathscr{J}_i(A)  
:= \sum_{\substack{k_2 \ge k_1 \ge 0 \\ j_1, j_2, j_3 \ge 1 \\ j_i \ge A}} 
2^{-k_1 \delta_1} j_1^{-1-\delta_1}  j_2^{-1} j_3^{-1}  
\frac{(2^{-k_2} j_2^{-1} + 2^{k_1-k_2} j_3^{-1})^{\delta_{2, 3}}}
{\big(2^{k_1} j_1 j_2 j_3^{-1} +  2^{-k_1} j_1^{-1} j_2^{-1} j_3 \big)^{\theta}}  
\lesssim A^{-\delta_0}. 
\end{align}
\end{lemma}


\begin{proof}
We begin with two elementary inequalities: for any $\delta>0$, $a >-\delta$, and $b>0$, 
\begin{align}
\label{kkk-1} \sum_{k_2 \ge k_3 \ge 0 \atop k_2 \ge B} 
2^{-k_2 \delta} 2^{(k_2-k_3) a}
&\lesssim 2^{-B(\delta + \min\{0, \, a\})} (B+1), 
\\
\label{kkk-2} \sum_{k_2 \ge k_3 \ge 0 \atop k_2 \ge B} 
2^{-k_3 \delta} 2^{(k_2-k_3) b}
&\lesssim 2^{-B \min\{\delta, \, b\}} (B+1), 
\end{align}
for which we omit the proof. For \eqref{kkk-1}, since the proof of $\mathscr{I}_2$ is similar to that of $\mathscr{I}_3$, we only show the latter. The estimates \eqref{kkk-1} and \eqref{kkk-2} imply 
\begin{align*}
\mathscr{I}_3(A, B) 
&\lesssim \sum_{\substack{k_2 \ge k_3 \ge 0 \\ j_2 \ge 1, \, j_3 \ge A \\ k_2 \ge B}} 
2^{-(k_2-k_3) \alpha} j_2^{-1} j_3^{-1} (2^{-k_2} j_2^{-1})^{\delta_{2, 3}} (j_2^{-1} j_3)^{-\theta}
\\
&\quad+ \sum_{\substack{k_2 \ge k_3 \ge 0 \\ j_2 \ge 1, \, j_3 \ge A \\ k_2 \ge B}} 
2^{-(k_2-k_3) \alpha} j_2^{-1} j_3^{-1}  (2^{-k_3} j_3^{-1})^{\delta_{2, 3}} (j_2 j_3^{-1})^{-\theta}
\\
&= \sum_{\substack{k_2 \ge k_3 \ge 0 \\ k_2 \ge B}} 
2^{-k_2 \delta_{2, 3}} 2^{-(k_2-k_3) \alpha} 
\sum_{j_2 \ge 1, \, j_3 \ge A} j_2^{-1- (\delta_{2, 3} - \theta)} j_3^{-1-\theta}
\\
&\quad + \sum_{\substack{k_2 \ge k_3 \ge 0 \\ k_2 \ge B}} 
2^{-k_3 \delta_{2, 3}} 2^{-(k_2-k_3) \alpha} 
\sum_{j_2 \ge 1, \, j_3 \ge A} j_2^{-1-\theta} j_3^{-1-(\delta_{2, 3} - \theta)} 
\\
&\lesssim A^{-\theta} 2^{-B \delta_{2, 3}} (B+1) 
+ A^{-(\delta_{2, 3} - \theta)} 2^{-B \min\{\delta_{2, 3}, \alpha\}} (B+1)
\\
&\lesssim A^{-\delta_0} 2^{-B \min\{\delta_0, \alpha\}} (B+1), 
\end{align*}
provided that $0<\delta_0 \le \min\{\theta, \delta_{2, 3} - \theta\}$.

To proceed, for $i=1, 2, 3$, let 
\begin{align*}
\mathscr{J}_i^1(A)
&:= \sum_{\substack{k_2 \ge k_1 \ge 0 \\ j_1, j_2, j_3 \ge 1 \\ j_i \ge A \\ j_3 \ge 2^{k_1} j_2}} 
2^{-k_1 \delta_1} j_1^{-1-\delta_1}  j_2^{-1} j_3^{-1} 
\frac{(2^{-k_2} j_2^{-1} + 2^{k_1-k_2} j_3^{-1})^{\delta_{2, 3}}}
{(2^{k_1} j_1 j_2 j_3^{-1} +  2^{-k_1} j_1^{-1} j_2^{-1} j_3)^{\theta}}, 
\\
\mathscr{J}_i^2(A)
&:= \sum_{\substack{k_2 \ge k_1 \ge 0 \\ j_1, j_2, j_3 \ge 1 \\ j_i \ge A \\ j_3 < 2^{k_1} j_2}} 
2^{-k_1 \delta_1} j_1^{-1-\delta_1}  j_2^{-1} j_3^{-1} 
\frac{(2^{-k_2} j_2^{-1} + 2^{k_1-k_2} j_3^{-1})^{\delta_{2, 3}}}
{(2^{k_1} j_1 j_2 j_3^{-1} +  2^{-k_1} j_1^{-1} j_2^{-1} j_3)^{\theta}}. 
\end{align*}  
The condition $0<\theta<\min\{\delta_1, \, \delta_{2, 3}\}$ gives  
\begin{align*}
\mathscr{J}_2^1(A)
&\lesssim \sum_{\substack{k_1 \ge 0 \\ j_1 \ge 1}} 2^{-k_1 \delta_1} j_1^{-1-(\delta_1-\theta)}  
\sum_{\substack{k_2 \ge 0 \\ j_2 \ge A}}  2^{-k_2 \delta_{2, 3}} j_2^{-1-\delta_{2, 3}} 
\sum_{j_3 \ge 2^{k_1} j_2} (2^{k_1} j_2)^{\theta} j_3^{-1-\theta} 
\nonumber \\
&\lesssim \sum_{\substack{k_1 \ge 0 \\ j_1 \ge 1}} 2^{-k_1 \delta_1} j_1^{-1-(\delta_1-\theta)}  
\sum_{\substack{k_2 \ge 0 \\ j_2 \ge A}}  2^{-k_2 \delta_{2, 3}} j_2^{-1-\delta_{2, 3}}   
\lesssim A^{-\delta_{2, 3}} 
\le A^{-\delta_0}, 
\end{align*}
and
\begin{align*}
\mathscr{J}_2^2(A)
&\lesssim \sum_{\substack{k_1 \ge 0 \\ j_1 \ge 1}} 2^{-k_1 \delta_1} j_1^{-1-\delta_1}  
\sum_{\substack{k_2 \ge k_1 \\ j_3 \ge 1}} 2^{-(k_2-k_1) \delta_{2, 3}} j_3^{-1-\delta_{2, 3}} 
\sum_{j_2 \ge \max\{A, \, 2^{-k_1} j_3\}}  (2^{-k_1} j_3)^{\theta} j_2^{-1-\theta} 
\nonumber \\
&\lesssim \sum_{\substack{k_1 \ge 0 \\ j_1 \ge 1}} 2^{-k_1 \delta_1} j_1^{-1-\delta_1}  
\sum_{\substack{k_2 \ge 0 \\ j_3 \ge 1}} 2^{-k_2 \delta_{2, 3}} j_3^{-1-\delta_{2, 3}} 
\min\{1, \, 2^{-k_1} j_3 A^{-1} \}^{\theta} 
\nonumber \\
&= \sum_{\substack{k_1 \ge 0 \\ j_1 \ge 1}} 2^{-k_1 \delta_1} j_1^{-1-\delta_1}  
\sum_{k_2 \ge 0} 2^{-k_2 \delta_{2, 3}} \sum_{j_3 \ge 2^{k_1} A} j_3^{-1-\delta_{2, 3}} 
\nonumber \\
&\quad+ A^{-\theta} \sum_{\substack{k_1 \ge 0 \\ j_1 \ge 1}} 2^{-k_1 (\delta_1 + \theta)} j_1^{-1-\delta_1}  
\sum_{k_2 \ge 0} 2^{-k_2 \delta_{2, 3}} \sum_{j_3 < 2^{k_1} A} j_3^{-1-(\delta_{2, 3}-\theta)} 
\nonumber \\
&\lesssim A^{-\delta_{2, 3}} \sum_{\substack{k_1 \ge 0 \\ j_1 \ge 1}} 
2^{-k_1 (\delta_1 + \delta_{2, 3})} j_1^{-1-\delta_1}  
\sum_{k_2 \ge 0} 2^{-k_2 \delta_{2, 3}}  
\nonumber \\
&\quad+ A^{-\theta} \sum_{\substack{k_1 \ge 0 \\ j_1 \ge 1}} 
2^{-k_1 (\delta_1 + \theta)} j_1^{-1-\delta_1}  
\sum_{k_2 \ge 0} 2^{-k_2 \delta_{2, 3}} 
\nonumber \\
&\lesssim A^{-\delta_0}.  
\end{align*}
Thus, $\mathscr{J}_2(A) = \mathscr{J}_2^1(A) + \mathscr{J}_2^2(A) \lesssim A^{-\delta_0}$. Following the proof above, one can see that $\mathscr{J}_1(A) \lesssim A^{-(\delta_1 - \theta)} + A^{-\delta_1} \lesssim A^{-\delta_0}$. Moreover, 
\begin{align*}
\mathscr{J}_3^1(A)
&\lesssim \sum_{\substack{k_2 \ge k_1 \ge 0 \\ j_1, j_2 \ge 1}} 
2^{-k_1 \delta_1} j_1^{-1-(\delta_1-\theta)}  
2^{-k_2 \delta_{2, 3}} j_2^{-1-\delta_{2, 3}} 
\sum_{j_3 \ge \max\{A, 2^{k_1} j_2\}} (2^{k_1} j_2)^{\theta} j_3^{-1-\theta} 
\\
&\lesssim \sum_{\substack{k_2 \ge k_1 \ge 0 \\ j_2 \ge 1}} 
2^{-k_1 \delta_1} 2^{-k_2 \delta_{2, 3}} j_2^{-1-\delta_{2, 3}} 
\min\{1, 2^{k_1} j_2 A^{-1}\}^{\theta}
\\
&= \sum_{k_2 \ge k_1 \ge 0} 2^{-k_1 \delta_1}  2^{-k_2 \delta_{2, 3}} 
\sum_{j_2 \ge \max\{1, \, 2^{-k_1} A\}} j_2^{-1-\delta_{2, 3}}   
\\
&\quad +  \sum_{k_2 \ge k_1 \ge 0 \atop 1 \le j_2 < 2^{-k_1} A} 
2^{-k_1 \delta_1}  2^{-k_2 \delta_{2, 3}} j_2^{-1-\delta_{2, 3}} 
(2^{k_1} j_2 A^{-1})^{\theta}
\\
&\lesssim \sum_{\substack{k_2 \ge k_1 \ge 0 \\ 2^{k_1} \ge A}} 
2^{-k_1 \delta_1} 2^{-k_2 \delta_{2, 3}}   
+ \sum_{\substack{k_2 \ge k_1 \ge 0 \\ 2^{k_1} < A}} 
2^{-k_1 \delta_1} 2^{-k_2 \delta_{2, 3}} (2^{-k_1} A)^{-\delta_{2, 3}}
\\
&\quad + A^{- \theta}  \sum_{k_2 \ge k_1 \ge 0 \atop  j_2 \ge 1} 
2^{-k_1 (\delta_1 - \theta)}  2^{-k_2 \delta_{2, 3}} j_2^{-1-(\delta_{2, 3}-\theta)}   
\\
&\lesssim A^{-\delta_1} + A^{-\delta_{2, 3}} + A^{- \theta}
\lesssim A^{-\delta_0}, 
\end{align*}
and 
\begin{align*}
\mathscr{J}_3^2(A)
&\lesssim \sum_{\substack{k_2 \ge k_1 \ge 0 \\ j_1 \ge 1, \, j_3 \ge A}} 
2^{-k_1 \delta_1} j_1^{-1-\delta_1}   
2^{-(k_2-k_1) \delta_{2, 3}} j_3^{-1-\delta_{2, 3}} 
\sum_{j_2 \ge \max\{1, \, 2^{-k_1} j_3\}} (2^{-k_1} j_3)^{\theta} j_2^{-1-\theta}
\\
&\lesssim \sum_{\substack{k_2 \ge k_1 \ge 0 \\ j_3 \ge A}} 
2^{-k_1 \delta_1}  2^{-(k_2-k_1) \delta_{2, 3}} j_3^{-1-\delta_{2, 3}} 
\lesssim A^{-\delta_{2, 3}} 
\lesssim A^{-\delta_0}. 
\end{align*}
Hence, $\mathscr{J}_3(A) = \mathscr{J}_3^1(A) + \mathscr{J}_3^2(A) \lesssim A^{-\delta_0}$. This completes the proof of  \eqref{JJA-2}.  
\end{proof}


Let us recall a technical lemma from \cite[Lemma 2.3]{CYY}.
 
%%%%%%%%%%%%%%%%%%%%%%%%%% LEMMA LEMMA LEMMA %%%%%%%%%%%%%%%%%%%%%%%%%%
\begin{lemma}\label{lem:improve}
Given $(F_{i, 1}, F_{i, 2}, F_{i, 3}) \in \F$, denote 
\begin{align*}
F_i(x_i, y_i) := F_{i, 1}(|x_i - y_i|) F_{i, 2}(|x_i - y_i|) F_{i, 3}(|x_i + y_i|),  \quad i=1, 2. 
\end{align*}
\begin{list}{\rm (\theenumi)}{\usecounter{enumi}\leftmargin=1.2cm \labelwidth=1cm \itemsep=0.2cm \topsep=.2cm \renewcommand{\theenumi}{\roman{enumi}}} 

\item\label{size-imp} For each $i=1, 2$, there exists a triple $(F'_{i, 1}, F'_{i, 2}, F'_{i, 3}) \in \F$ such that $F'_{i, 1}$ is monotone increasing, $F'_{i, 2}$ and $F'_{i, 3}$ are monotone decreasing, and 
\begin{align*}
F_i(x_i, y_i) 
\le F'_i(x_i, y_i) 
:= F'_{i, 1}(|x_i-y_i|) F'_{i, 2}(|x_i-y_i|) F'_{i, 3} \bigg(1 + \frac{|x_i+y_i|}{1+|x_i-y_i|}\bigg). 
\end{align*}

\item\label{Holder-imp}
Assume that $F_{1, 1}$ and $F_{2, 1}$ are monotone increasing. Then for each $i=1, 2$, there exist $\delta'_i \in (0, \delta_i)$ and $(F'_{i, 1}, F'_{i, 2}, F'_{i, 3}) \in \F$ such that $F'_{i, 1}$ is monotone increasing, $F'_{i, 2}$ and $F'_{i, 3}$ are monotone decreasing, and 
\begin{align*}
F_i(x_i, y_i) \frac{|y_i-y'_i|^{\delta_i}}{|x_i-y_i|^{n_i+\delta_i}} 
\le F'_i(x_i, y_i) \frac{|y_i-y'_i|^{\delta'_i}}{|x_i-y_i|^{n_i+\delta'_i}}, 
\end{align*}
whenever $|y_i-y'_i| \leq |x_i-y_i|/2$, where 
\begin{align*}
F'_i(x_i, y_i) 
:= F'_{i, 1}(|y_i-y'_i|) F'_{i, 2}(|x_i-y_i|) F'_{i, 3} \bigg(1 + \frac{|x_i+y_i|}{1+|x_i-y_i|}\bigg). 
\end{align*}
\end{list}
\end{lemma}
%%%%%%%%%%%%%%%%%%%%%%%%%% LEMMA LEMMA LEMMA %%%%%%%%%%%%%%%%%%%%%%%%%%


\subsection{Single integrals}

The following result is essentially contained in the proof of \cite[Lemma 2.12]{CYY}. 
 
\begin{lemma}\label{lem:diag}
Let $r>1$ and $1<s<2$. Assume that $F$ is monotone decreasing. Then exists a bounded and decreasing function $\widetilde{F}: [0, \infty) \to [0, \infty)$ satisfying $\lim\limits_{t \to \infty} \widetilde{F}(t)=0$ such that for all intervals $I, J \subset \R$ with $I \cap J = \emptyset$ and $\d(I, J)=0$, 
\begin{align}
\label{eq:diag-1} 
& \bigg(\fint_I \fint_J F(|x - y|)^r \, dx \, dy \bigg)^{\frac1r} 
\lesssim \min\big\{\widetilde{F}(\ell(I)), \, \widetilde{F}(\ell(J)) \big\},  
\\
\label{eq:diag-2} 
&\bigg(\fint_I \fint_J \frac{1}{|x - y|^s} \, dx \, dy  \bigg)^{\frac1s} 
\lesssim \min\big\{|I|^{-\frac1s} |J|^{\frac1s -1}, \, |I|^{\frac1s -1} |J|^{-\frac1s} \big\}. 
\end{align}
\end{lemma}


Next, let us present some useful estimates, which will be used frequently below.  

\begin{lemma}\label{lem:IJD} 
Let $F$ be a monotone decreasing function such that $\lim\limits_{t \to \infty} F(t)=0$. Then there exists a bounded and decreasing function $\widetilde{F}: [0, \infty) \to [0, \infty)$ satisfying $\lim\limits_{t \to \infty} \widetilde{F}(t)=0$ such that for all dyadic intervals $I, J \subset \R$ with $\rd(I, J)<1$ and for any $t>0$, if we define 
\begin{align*}
\mathscr{I}_{\theta}^t(I, J) 
:= \fint_I \fint_J \bigg(t |x-y| + \frac{1}{t|x-y|}\bigg)^{-\theta} \frac{F(|x-y|)}{|x-y|} dx \, dy,  
\end{align*}
then the following hold: 
\begin{enumerate} 
\item If $\theta=0$ and $I \cap J = \emptyset$, then for any $r>2$, 
\begin{align}\label{IIJ-1}
\mathscr{I}_{\theta}^t(I, J) 
\lesssim \min\big\{\widetilde{F}(\ell(I)), \, \widetilde{F}(\ell(J)) \big\} 
\min\big\{|I|^{\frac1r - 1} |J|^{-\frac1r}, \, |I|^{-\frac1r} |J|^{\frac1r - 1} \big\}. 
\end{align}

\item If $\theta \in (0, 1)$, then for any $r>\frac{2}{1-\theta}$, 
\begin{multline}\label{IIJ-2}
\mathscr{I}_{\theta}^t(I, J) 
\lesssim  \min\big\{\widetilde{F}(\ell(I)), \, \widetilde{F}(\ell(J)) \big\} 
\\
\times \min\Big\{|I|^{\frac1r - 1} |J|^{-\frac1r} (t|J| + t^{-1} |J|^{-1})^{-\theta}, \, 
|I|^{-\frac1r} |J|^{\frac1r - 1} (t|I| + t^{-1} |I|^{-1})^{-\theta} \Big\},  
\end{multline}
where the implicit constant is independent of $t$, $I$, and $J$. 
\end{enumerate} 
\end{lemma}


\begin{proof}
Since the proof of \eqref{IIJ-1} is contained in that of \eqref{IIJ-2}, we only show the latter. Let $\theta \in (0, 1)$,  $I, J \subset \R$ be dyadic intervals with $\rd(I, J)<1$. We begin with the case $I \cap J = \emptyset$. We may assume that $\d(I, J)=0$, otherwise there exists some absolute constant $\kappa \ge 1$ so that $(\kappa I) \cap (\kappa J) = \emptyset$ and $\d(\kappa I, \kappa J)=0$. The fact that $\mathscr{I}_{\theta}^t(I, J) \le \kappa^2  \mathscr{I}_{\theta}^t(\kappa I, \kappa J)$ will give the general result. 

If $t|J| < 1$, we use the H\"{o}lder's inequality and the fact that $r'(1-\theta)<2$ to obtain 
\begin{align*}
\mathscr{I}_{\theta}^t(I, J) 
&\le t^{\theta} \fint_I \fint_J \frac{F(|x-y|)}{|x-y|^{1-\theta}} dx \, dy
\\
&\le t^{\theta} \bigg(\fint_I \fint_J F(|x-y|)^r dx \, dy \bigg)^{\frac1r}
\bigg(\fint_I \fint_J \frac{dx \, dy}{|x-y|^{r'(1-\theta)}} \bigg)^{\frac{1}{r'}}
\\
&\lesssim t^{\theta} \min\big\{\widetilde{F}(\ell(I)), \, \widetilde{F}(\ell(J)) \big\} 
|I|^{\frac1r - 1} |J|^{-\frac1r+\theta}
\\
&\simeq \min\big\{\widetilde{F}(\ell(I)), \, \widetilde{F}(\ell(J)) \big\} 
|I|^{\frac1r-1} |J|^{-\frac1r} (t|J| + t^{-1} |J|^{-1})^{-\theta}. 
\end{align*}
where we have used Lemma \ref{lem:diag} in the second-to-last step. Similarly, if $t|J| \ge 1$, noting that $r'(1+\theta)<2$, we apply the H\"{o}lder's inequality to arrive at 
\begin{align*}
\mathscr{I}_{\theta}^t(I, J) 
&\le t^{-\theta} \fint_I \fint_J \frac{F(|x-y|)}{|x-y|^{1+\theta}} dx \, dy
\\
&\le t^{-\theta} \bigg(\fint_I \fint_J F(|x-y|)^r dx \, dy \bigg)^{\frac1r}
\bigg(\fint_I \fint_J \frac{dx \, dy}{|x-y|^{r'(1+\theta)}} \bigg)^{\frac{1}{r'}}
\\
&\lesssim t^{-\theta} \min\big\{\widetilde{F}(\ell(I)), \, \widetilde{F}(\ell(J)) \big\} 
|I|^{\frac1r - 1} |J|^{-\frac1r-\theta}
\\
&\simeq \min\big\{\widetilde{F}(\ell(I)), \, \widetilde{F}(\ell(J)) \big\} 
|I|^{\frac1r - 1} |J|^{-\frac1r} (t|J| + t^{-1} |J|^{-1})^{-\theta}. 
\end{align*}

Now let $I \cap J \not=\emptyset$, hence, $I=J$. By a change of variables, we are reduced to showing  
\begin{align}\label{IJD-1}
t \fint_{tI} \fint_{tI} \frac{F(t^{-1} |x-y|)}{|x-y|^{1-\theta}} dx \, dy 
\lesssim |I|^{-1} (t|I| + t^{-1} |I|^{-1})^{-\theta} \widetilde{F}(\ell(I)), 
\end{align}
since $\mathscr{I}_t(I, I)$ is controlled by the left hand side of \eqref{IJD-1}. Note that the inequality \eqref{IJD-1} is invariant under the translation and dilation. Hence, we may assume that $t I =[0, 1]$ and it suffices to prove that 
\begin{align}\label{IJD-2}
\int_0^1 \int_0^1 \frac{F(s |x-y|)}{|x-y|^{1-\theta}} dx \, dy 
\lesssim \widetilde{F}(s), \quad \forall \, s>0. 
\end{align}
Indeed, fix $x \in (0, 1)$, it follows from the monotonicity of $F$ that 
\begin{align*}
\int_0^1 \frac{F(s |x-y|)}{|x-y|^{1-\theta}} \, dy 
&= \int_0^x \frac{F(s |x-y|)}{|x-y|^{1-\theta}} \, dy 
+ \int_x^1 \frac{F(s |x-y|)}{|x-y|^{1-\theta}} \, dy 
\\
&\le \sum_{k \ge 0 \atop 2^{-k+1} \le x} \int_{x-2^{-k+1}}^{x-2^{-k}} 
+ \sum_{k \ge 0 \atop 2^{-k+1} \le 1-x} \int_{x+2^{-k}}^{x+2^{-k+1}} 
\\
&\lesssim \sum_{k=0}^{\infty} F(2^{-k} s) \big[2^{(-k+1) \theta} - 2^{-k \theta} \big]
\\
&\lesssim \sum_{k=0}^{\infty} 2^{-k \theta} F(2^{-k} s) 
=: \widetilde{F}(s),  
\end{align*}
where the implicit constants are independent of $x$. This immediately implies \eqref{IJD-2}. It is easy to check that $\widetilde{F}$ is bounded and decreasing, and $\lim\limits_{t \to \infty} \widetilde{F}(t)=0$. The proof is complete. 
\end{proof}


Given $i \in \{1, 2, 3\}$, let $(F_{i, 1}, F_{i, 2}, F_{i, 3}) \in \mathscr{F}$ satisfy that $F_{i, 1}$ is increasing, $F_{i, 2}$ and $F_{i, 3}$ are decreasing. Define  
\begin{align}
\label{def:FF-1} F_i(I_i, J_i)
&:= F_{i, 1}(\ell(I_i)) F_{i, 2}(\ell(I_i)) F_{i, 3}(\rd(I_i \vee J_i, \II)), 
\\ 
\label{def:FF-2} \widetilde{F}_i(I_i, J_i)
&:= F_{i, 1}(\ell(I_i)) \widetilde{F}_{i, 2}(\ell(I_i)) \widetilde{F}_{i, 3}(I_i \vee J_i), 
\\
\label{def:FF-3} \widetilde{F}_i(I_i)
&:= F_{i, 1}(\ell(I_i)) \widetilde{F}_{i, 2}(\ell(I_i)) \widetilde{F}_{i, 3}(I_i), 
\\ 
\label{def:FF-4} \widehat{F}_i(I_i) 
&:=  F_i(I_i) +  \widetilde{F}_i(I_i), 
\end{align}
where $F_i(I_i)$ comes from Definitions \ref{def:partial-1}--\ref{def:partial-2}, 
\begin{align*}
\widetilde{F}_{i, 2}(t) := \sum_{k=0}^{\infty} 2^{-k \theta_2} F(2^{-k} t), 
\quad\text{ and }\quad 
\widetilde{F}_{i, 3}(I_i)
:=\sum_{k=0}^{\infty} 2^{-k \theta_3} F_{i, 3}(\rd(2^k I_i, \II)). 
\end{align*}
The auxiliary parameters $\theta_2, \theta_3 \in (0, 1)$ are harmless and small enough. The notation in \eqref{def:FF-1}--\eqref{def:FF-4} will be used throughout this paper. 



\begin{lemma}\label{lem:FF}
Fix $i \in \{1, 2, 3\}$ and define $F_i(I_i, J_i)$, $\widetilde{F}_i(I_i, J_i)$, $\widetilde{F}_i(I_i)$, and $\widehat{F}_i(I_i)$ as above. Then the following statements hold: 
\begin{list}{\rm (\theenumi)}{\usecounter{enumi}\leftmargin=1.2cm \labelwidth=1cm \itemsep=0.2cm \topsep=.2cm \renewcommand{\theenumi}{\alph{enumi}}} 

\item\label{list-FF1} For any $\varepsilon>0$, there exists $N_0 \ge 1$ so that for all $N \ge N_0$, $I_i \in \D^i$, and $J \not\in \D^i(2N)$ with $\ell(I_i) = \ell(J_i)$, there holds 
\begin{align*}
F_i(I_i, J_i) < \varepsilon 
\quad\text{ or }\quad 
\rd(I_i, J_i) \ge 2N^{\frac18}.
\end{align*}

\item\label{list-FF2} For any $\varepsilon>0$, there exists $N_0 \ge 1$ so that for all $N \ge N_0$, $I_i \in \D^i$, and $J \not\in \D^i(2N)$ with $\ell(I_i) = \ell(J_i)$ and $\rd(I_i, J_i)<1$, there holds 
\begin{align*}
\widetilde{F}_i(I_i, J_i) < \varepsilon.
\end{align*}

\item\label{list-FF3} One has 
\begin{align*}
\lim_{N \to \infty} \sup_{I_i \not\in \D^i(2N)} \widetilde{F}_i(I_i)
= \lim_{N \to \infty} \sup_{I_i \not\in \D^i(2N)} \widehat{F}_i(I_i) 
=0. 
\end{align*}
\end{list}
\end{lemma}


\begin{proof}
Parts \eqref{list-FF1} and \eqref{list-FF2} are essentially contained in \cite[Lemmas 4.7, 4.26]{CYY}, while part \eqref{list-FF3} just follows from the definition.  
\end{proof}


\begin{lemma}\label{lem:PP} 
Let $i \in \{1, 2, 3\}$, $(F_{i, 1}, F_{i, 2}, F_{i, 3}) \in \mathscr{F}$ satisfy that $F_{i, 1}$ is increasing, $F_{i, 2}$ and $F_{i, 3}$ are decreasing. Let $I_i, J_i \subset \R$ be dyadic intervals, $i=1, 2, 3$. Then the following hold: 
\begin{enumerate}

\item If $\rd(I_i, J_i) \ge 1$, then 
\begin{align}\label{def:P1}
\mathscr{P}_i(I_i, J_i)
:= \int_{I_i} \int_{J_i} \frac{F_i(x_i, y_i)}{|x_i - y_i|}  dx_i \, dy_i 
\lesssim \frac{F_i(I_i, J_i)}{\rd(I_i, J_i)} \min\{\ell(I_i), \ell(J_i) \},   
\end{align}
where 
\begin{align*}
F_i(x_i, y_i)
:= F_{i, 1}(|y_i - c_{I_i}|) F_{i, 2}(|x_i - y_i|) F_{i, 3}\bigg(1+\frac{|x_i+y_i|}{1+|x_i-y_i|}\bigg). 
\end{align*}

\item If $\rd(I_i, J_i) < 1$ and $I_i \cap J_i = \emptyset$, then 
\begin{align}\label{def:P2}
\mathscr{P}_i(I_i, J_i)
&:= \int_{I_i} \int_{J_i} \frac{F_i(x_i, y_i)}{|x_i - y_i|}  dx_i \, dy_i 
\nonumber \\
&\lesssim \widetilde{F}_i(I_i) \min\big\{\ell(I_i)^{\frac1r} \ell(J_i)^{1-\frac1r}, \, \ell(I_i)^{1-\frac1r} \ell(J_i)^{\frac1r} \big\} 
\nonumber \\
&\le \widetilde{F}_i(I_i, J_i) \min\big\{\ell(I_i)^{\frac1r} \ell(J_i)^{1-\frac1r}, \, \ell(I_i)^{1-\frac1r} \ell(J_i)^{\frac1r} \big\}, 
\end{align}
where 
\begin{align*}
F_i(x_i, y_i)
&:= F_{i, 1}(|x_i - y_i|) F_{i, 2}(|x_i - y_i|) F_{i, 3}\bigg(1+\frac{|x_i+y_i|}{1+|x_i-y_i|}\bigg). 
\end{align*}

\item If $\rd(I_i, J_i)<1$, then for any $t>0$ and $\theta \in (0, 1)$, 
\begin{align}\label{def:Q0}
\mathscr{Q}_i^{0, t}(I_i, J_i) 
&:= \int_{I_i} \int_{J_i} \bigg(t |x_i - y_i| + \frac{1}{t|x_i - y_i|}\bigg)^{-\theta} \frac{F_i(x_i, y_i)}{|x_i - y_i|} dx_i \, dy_i 
\nonumber \\
&\lesssim \min\big\{\ell(I_i)^{\frac1r} \ell(J_i)^{1-\frac1r} \big[t \ell(J_i) + t^{-1} \ell(J_i)^{-1} \big]^{-\theta}, 
\nonumber \\
&\qquad\qquad\ell(I_i)^{1-\frac1r} \ell(J_i)^{\frac1r} \big[t \ell(I_i) + t^{-1} \ell(I_i)^{-1} \big]^{-\theta} \big\}  
\widetilde{F}_i(I_i), 
\end{align}
where the implicit constant is independent of $t$, $I_i$, and $J_i$.


\item If $\rd(I_i, J_i)<1$, then for any $t>0$, $\delta_0 \in [0, 1)$, and $\theta \in (0, 1)$, 
\begin{align}\label{def:Q1}
\mathscr{Q}_i^{1, t}(I_i, J_i) 
&:= \int_{I_i} \int_{J_i \setminus 3I_i} 
\bigg(\frac{\ell(I_i)}{|x_i - y_i|}\bigg)^{\delta_1}
\bigg(t |x_i - y_i| + \frac{1}{t|x_i - y_i|}\bigg)^{-\theta} \frac{F_i(x_i, y_i)}{|x_i - y_i|} dx_i \, dy_i 
\nonumber \\
&\lesssim \frac{\widetilde{F}_i(I_i, J_i)}{\ird(I_i, J_i)^{\delta_0}} |I_i| \, 
\big[t |I_i| + t^{-1} |I_i|^{-1} \big]^{-\theta}, 
\end{align}
where the implicit constant is independent of $t$, $I_i$, and $J_i$.


\item For any $t>0$ and $\theta \in (0, 1)$, 
\begin{align}\label{def:Q2}
\mathscr{Q}_i^{2, t}(I_i) 
&:= \int_{I_i} \int_{(3I_i)^c} 
\bigg(t |x_i - y_i| + \frac{1}{t|x_i - y_i|}\bigg)^{-\theta} \frac{F_i(x_i, y_i)}{|x_i - y_i|} dx_i \, dy_i 
\nonumber \\
&\lesssim \widetilde{F}_i(I_i) |I_i| \, (t |I_i|)^{-\theta}, 
\end{align}
where the implicit constant is independent of $t$ and $I_i$. 
\end{enumerate}

\end{lemma}


\begin{proof}
First, suppose that $\rd(I_i, J_i) \ge 1$. For any $x_i \in J_i$ and $y_i \in I_i$,
\begin{align}\label{eq:PP-1}
|x_i - y_i| \le \ell(I_i \vee J_i) 
\le \d(I_i, J_i) + \ell(I_i) + \ell(J_i) 
\le 3\d(I_i, J_i) \le 3 |x_i - y_i|. 
\end{align}
Note that $\frac12(x_i+y_i) \in I_i \vee J_i$ and
\begin{align*}
1+\frac{|x_i + y_i|}{1+|x_i - y_i|}
\ge \frac{2\d(I_i \vee J_i, \II)}{1+ \ell(I_i \vee J_i)}
\ge \frac{\d(I_i \vee J_i, \II)}{\max\{\ell(I_i \vee J_i), 1\}}
=\rd(I_i \vee J_i, \II), 
\end{align*}
which gives 
\begin{align}\label{eq:PP-2}
F_i(x_i, y_i)
\le F_{i, 1}(\ell(I_i)) F_{i, 2}(\ell(I_i \vee J_i)) F_{i, 3}(\rd(I_i \vee J_i, \II))
= F_i(I_i, J_i).
\end{align}
Hence, it follows from \eqref{eq:PP-1} and \eqref{eq:PP-2} that 
\begin{align*}
\mathscr{P}_i(I_i, J_i)
\lesssim \frac{F_i(I_i, J_i)}{\d(I_i, J_i)} \ell(I_i) \ell(J_i) 
= \frac{F_i(I_i, J_i)}{\rd(I_i, J_i)} \min\{\ell(I_i), \ell(J_i) \}.  
\end{align*}
This shows \eqref{def:P1}. 


Next, to prove \eqref{def:P2}, let $\rd(I_i, J_i)<1$, $x_i \in J_i$, and $y_i \in I_i$. Then, $J_i \subset 5I_i \setminus I_i$, $|x_i - y_i| \le 5 \ell(I_i)$, and
\begin{align*}
1+\frac{|x_i + y_i|}{1 + |x_i - y_i|}
&\ge \frac{|x_i + y_i| + |x_i - y_i|}{1 + |x_i - y_i|}
\\
&\ge \frac{2|y_i|}{1+5\ell(I_i)}
\ge \frac{\d(I_i, \II)}{3\max\{\ell(I_i), 1\}}
= \frac13 \rd(I_i, \II),
\end{align*}
which imply that
\begin{align}\label{eq:PP-3}
F_i(x_i, y_i)
\le F_{i, 1}(\ell(I_i)) F_{i, 2}(|x_i - y_i|)  F_{i, 3}(\rd(I_i, \II)).
\end{align}
Thus, \eqref{eq:PP-3} and \eqref{IIJ-1} yield 
\begin{align*}
\mathscr{P}_i(I_i, J_i)
&\lesssim F_{i, 1}(\ell(I_i)) F_{i, 3}(\rd(I_i, \II))
\int_{I_i} \int_{J_i} \frac{F_{i, 2}(|x_i - y_i|)}{|x_i - y_i|}  dx_i \, dy_i
\\
&\lesssim F_{i, 1}(\ell(I_i)) \widetilde{F}_{i, 2}(\ell(I_i)) F_{i, 3}(\rd(I_i, \II)) 
\min\big\{\ell(I_i)^{\frac1r} \ell(J_i)^{1-\frac1r}, \, \ell(I_i)^{1-\frac1r} \ell(J_i)^{\frac1r} \big\} 
\\
&\le F_{i, 1}(\ell(I_i)) \widetilde{F}_{i, 2}(\ell(I_i)) \widetilde{F}_{i, 3}(I_i) 
\min\big\{\ell(I_i)^{\frac1r} \ell(J_i)^{1-\frac1r}, \, \ell(I_i)^{1-\frac1r} \ell(J_i)^{\frac1r} \big\} 
\\
&= \widetilde{F}_i(I_i) \min\big\{\ell(I_i)^{\frac1r} \ell(J_i)^{1-\frac1r}, \, \ell(I_i)^{1-\frac1r} \ell(J_i)^{\frac1r} \big\}  
\\
&\le \widetilde{F}_i(I_i, J_i) \min\big\{\ell(I_i)^{\frac1r} \ell(J_i)^{1-\frac1r}, \, \ell(I_i)^{1-\frac1r} \ell(J_i)^{\frac1r} \big\},
\end{align*}
provided $\rd(I_i, \II) \ge \rd(I_i \vee J_i, \II)$. Similarly, invoking \eqref{eq:PP-3} and \eqref{IIJ-2}, we conclude \eqref{def:Q0} as desired. 


Finally, let us turn to the proof of \eqref{def:Q1}. Let $x_i \in J_i \setminus 3I_i$ and $y_i \in I_i$. Then $|x_i - y_i| \ge \ell(I_i)$ and $|x_i - c_{I_i}| > \ell(I_i) \ge 2|y_i - c_{I_i}|$. It follows from $\rd(I_i, J_i)<1$ that $\d(I_i, J_i)<\ell(J_i)$ and
\begin{align*}
|x_i - y_i|
\le \ell(J_i) + \d(I_i, J_i) + \ell(I_i)
\le 3\ell(J_i).
\end{align*}
Note that $|y_i| \ge \d(I_i, \II) \ge \d(I_i \vee J_i, \II)$. Hence, we obtain
\begin{align*}
1+ \frac{|x_i + y_i|}{1 + |x_i - y_i|}
&\ge \frac{1+2|y_i|}{1 + |x_i - y_i|}
\ge \frac{2\d(I_i \vee J_i, \II)}{1 + 3\ell(J_i)}
\\
&=\frac{\max\{\ell(I_i \vee J_i), 1\}}{1+3\ell(J_i)} \rd(I_i \vee J_i, \II)
\ge \frac14 \rd(I_i \vee J_i, \II),
\end{align*}
which implies that
\begin{align}\label{FXY-1}
F_i(x_i, y_i)
&:= F_{i, 1}(|y_i - c_{I_i}|) F_{i, 2}(|x_i - y_i|) F_{i, 3}\bigg(1 + \frac{|x_i + y_i|}{1 + |x_i - y_i|}\bigg)
\nonumber \\
&\lesssim F_{i, 1}(\ell(I_i)) F_{i, 2}(\ell(I_i)) F_{i, 3}(\rd(I_i \vee J_i, \II))
\nonumber \\
&\le F_{i, 1}(\ell(I_i)) \widetilde{F}_{i, 2}(\ell(I_i)) \widetilde{F}_{i, 3}(I_i \vee J_i)
= \widetilde{F}_i(I_i, J_i).
\end{align}
In addition,
\begin{align}\label{FXY-2}
|x_i - c_{I_i}|
\ge \ell(I_i)/2 + \d(I_i, J_i)
\ge \ell(I_i) \ird(I_i, J_i)/2.
\end{align}
Let
\begin{align*}
R_i^{k_i} := \big\{x_i \in \R: 2^{k_i} \ell(I_i) < |x_i - c_{I_i}| \le 2^{k_i+1} \ell(I_i)\big\}, \quad k_i \ge k_i^0,
\end{align*}
where $k_i^0 := \max\{0, \, \log_2 (\ird(I_i, J_i)/2)\}$. If $t|I_i| \ge 1$, then we invoke \eqref{FXY-1} and \eqref{FXY-2} to conclude
\begin{align*}
\mathscr{Q}_i^{1, t}(I_i, J_i)
&\le t^{-\theta} \int_{I_i} \int_{J_i \setminus 3I_i}  
\bigg(\frac{\ell(I_i)}{|x_i - y_i|}\bigg)^{\delta_1} 
\frac{F_i(x_i, y_i)}{|x_i - y_i|^{1 + \theta}} \, dx_i \, dy_i 
\\
&\lesssim t^{-\theta} \sum_{k_i=k_i^0}^{\infty} \int_{I_i} \int_{R_i^{k_i}}  2^{-k_i \delta_1} 
\frac{\widetilde{F}_i(I_i, J_i)}{(2^{k_i} \ell(I_i))^{1+\theta}} \, dx_i \, dy_i 
\\
&\lesssim t^{-\theta} \ell(I_i)^{-\theta} \sum_{k_i=k_i^0}^{\infty} 2^{-k_i (\delta_1+\theta)}  
\widetilde{F}_i(I_i, J_i) |I_i|
\\
&\lesssim \frac{\widetilde{F}_i(I_i, J_i)}{\ird(I_i, J_i)^{\delta_1 + \theta}} |I_i| 
\big[t |I_i| + t^{-1} |I_i|^{-1} \big]^{-\theta}.
\end{align*}
Similarly, if $t|I_i| \le 1$, we have 
\begin{align*}
\mathscr{Q}_i^{1, t}(I_i, J_i)
&\lesssim t^{\theta} \sum_{k_i=k_i^0}^{\infty} \int_{I_i} \int_{R_i^{k_i}}  2^{-k_i \delta_1} 
\frac{\widetilde{F}_i(I_i, J_i)}{(2^{k_i} \ell(I_i))^{1-\theta}} \, dx_i \, dy_i 
\\
&\lesssim t^{\theta} \ell(I_i)^{\theta} \sum_{k_i=k_i^0}^{\infty} 2^{-k_i (\delta_1-\theta)}  
\widetilde{F}_i(I_i, J_i) |I_i|
\\
&\lesssim \frac{\widetilde{F}_i(I_i, J_i)}{\ird(I_i, J_i)^{\delta_1 - \theta}} |I_i| 
\big[t |I_i| + t^{-1} |I_i|^{-1} \big]^{-\theta}.
\end{align*}
This completes the proof.
\end{proof} 




\subsection{Double integrals}

\begin{lemma}\label{lem:doulbe} 
Let $\theta \in (0, 1)$ and $t>0$. Given $i \in \{1, 2, 3\}$, let $(F_{i, 1}, F_{i, 2}, F_{i, 3}) \in \mathscr{F}$ satisfy that $F_{i, 1}$ is increasing, $F_{i, 2}$ and $F_{i, 3}$ are decreasing. Let $I=I_1 \times I_2 \times I_3$ and $J=J_1 \times J_2 \times J_3$ be dyadic Zygmund rectangles so that $\ell(I_1) \le \ell(J_1)$, $\ell(I_2) \ge \ell(J_2)$, and $\ell(I_3) \ge \ell(J_3)$. Then the following hold:
\begin{enumerate} 
\item If $\rd(I_1, J_1)<1$ and $\rd(I_2, J_2)<1$, then 
\begin{align}\label{def:R1200}
\mathscr{R}_{1, 2}^{0, 0, t}(I_{1, 2}, J_{1, 2}) 
&:= \int_{I_{1, 2}} \int_{J_{1, 2}} D_{\theta}(x_{1, 2}-y_{1, 2}, t)
\prod_{i=1}^2 \frac{F_i(x_i, y_i)}{|x_i - y_i|} \, dx_{1, 2} \, dy_{1, 2}
\nonumber \\
&\lesssim \widetilde{F}_1(I_1) |I_1|^{1-\frac1r} |J_1|^{\frac1r} \, 
\widetilde{F}_2(J_2) |I_2|^{\frac1r} |J_2|^{1-\frac1r} \,
\bigg[\frac{|I_1| |J_2|}{t} + \frac{t}{|I_1| |J_2|} \bigg]^{-\theta}. 
\end{align}


\item If $\rd(I_1, J_1)<1$ and $\rd(I_3, J_3)<1$, then 
\begin{align}\label{def:R1300}
\mathscr{R}_{1, 3}^{0, 0, t}(I_{1, 3}, J_{1, 3}) 
&:= \int_{I_{1, 3}} \int_{J_{1, 3}} D_{\theta}(x_1 - y_1, t, x_3 - y_3)
\prod_{i=1, 3} \frac{F_i(x_i, y_i)}{|x_i - y_i|} \, dx_{1, 3} \, dy_{1, 3}
\nonumber \\
&\lesssim \widetilde{F}_1(I_1) |I_1|^{1-\frac1r} |J_1|^{\frac1r} \, 
\widetilde{F}_3(J_3) |I_3|^{\frac1r} |J_3|^{1-\frac1r} 
\bigg[\frac{t |I_1|}{|J_3|} + \frac{|J_3|}{t |I_1|} \bigg]^{-\theta}. 
\end{align}



\item If $\rd(I_2, J_2)<1$ and $\rd(I_3, J_3)<1$, then 
\begin{align}\label{def:R2300}
\mathscr{R}_{2, 3}^{0, 0, t}(I_{2, 3}, J_{2, 3}) 
&:= \int_{I_{2, 3}} \int_{J_{2, 3}} D_{\theta}(t, x_{2, 3}-y_{2, 3})
\prod_{i=2}^3 \frac{F_i(x_i, y_i)}{|x_i - y_i|} \, dx_{2, 3} \, dy_{2, 3}
\nonumber \\
&\lesssim \widetilde{F}_2(J_2) |I_2|^{\frac1r} |J_2|^{1-\frac1r} \, 
\widetilde{F}_3(J_3) |I_3|^{\frac1r} |J_3|^{1-\frac1r} \, 
\bigg[\frac{t |J_2|}{|J_3|} + \frac{|J_3|}{t |J_2|} \bigg]^{-\theta}. 
\end{align}


\item If $\rd(I_1, J_1)<1$ and $\rd(I_3, J_3)<1$, then 
\begin{align}\label{def:R1310}
\mathscr{R}_{1, 3}^{1, 0, t}(I_{1, 3}, J_{1, 3}) 
&:= \int_{I_1 \times I_3} \int_{(J_1 \setminus 3I_1) \times J_3} 
\bigg(\frac{\ell(I_1)}{|x_1 - y_1|}\bigg)^{\delta_1} 
\nonumber \\
&\quad\times D_{\theta}(x_1 - y_1, t, x_3 - y_3) 
\prod_{i=1, 3} \frac{F_i(x_i, y_i)}{|x_i - y_i|} \, dx_{1, 3} \, dy_{1, 3}
\nonumber \\
&\lesssim \frac{\widetilde{F}_1(I_1, J_1)}{\ird(I_1, J_1)^{\delta_0}} |I_1| \, 
\widetilde{F}_3(J_3) |I_3|^{\frac1r} |J_3|^{1-\frac1r} 
\bigg[\frac{t |I_1|}{|J_3|} + \frac{|J_3|}{t |I_1|} \bigg]^{-\theta}. 
\end{align}


\item If $\rd(I_2, J_2)<1$ and $\rd(I_3, J_3)<1$, then 
\begin{align}\label{def:R2301}
\mathscr{R}_{2, 3}^{0, 1, t}(I_{2, 3}, J_{2, 3}) 
&:= \int_{(3J_2) \times (I_3 \setminus 3J_3)} \int_{J_{2, 3}} D_{\theta}(t, x_{2, 3}-y_{2, 3})
\prod_{i=2}^3 \frac{F_i(x_i, y_i)}{|x_i - y_i|} \, dx_{2, 3} \, dy_{2, 3}
\nonumber \\
&\lesssim \widetilde{F}_2(J_2) |J_2| \, 
\frac{\widetilde{F}_3(I_3, J_3)}{\ird(I_3, J_3)^{\delta_0}} |J_3| \, 
\bigg[\frac{t |J_2|}{|J_3|} + \frac{|J_3|}{t |J_2|} \bigg]^{-\theta}. 
\end{align}



\item If $\rd(I_2, J_2)<1$ and $\rd(I_3, J_3)<1$, then 
\begin{align}\label{def:R2310}
\mathscr{R}_{2, 3}^{1, 0, t}(I_{2, 3}, J_{2, 3}) 
&:= \int_{(I_2 \setminus 3J_2) \times 3J_3} \int_{J_{2, 3}} D_{\theta}(t, x_{2, 3}-y_{2, 3})
\prod_{i=2}^3 \frac{F_i(x_i, y_i)}{|x_i - y_i|} \, dx_{2, 3} \, dy_{2, 3}
\nonumber \\
&\lesssim \frac{\widetilde{F}_2(I_2, J_2)}{\ird(I_2, J_2)^{\delta_0}} |J_2| \, 
\widetilde{F}_3(J_3) |J_3| \, 
\bigg[\frac{t |J_2|}{|J_3|} + \frac{|J_3|}{t |J_2|} \bigg]^{-\theta}. 
\end{align}


\item If $\rd(I_2, J_2)<1$ and $\rd(I_3, J_3)<1$, then 
\begin{align}\label{def:R2311}
\mathscr{R}_{2, 3}^{1, 1, t}(I_{2, 3}, J_{2, 3}) 
&:= \int_{(I_2 \setminus 3J_2) \times (I_3 \setminus 3J_3)} \int_{J_{2, 3}} 
\bigg(\frac{\ell(J_2)}{|x_2 - y_2|} + \frac{\ell(J_3)}{|x_3 - y_3|}\bigg)^{\delta_{2, 3}}
\nonumber \\
&\qquad\times 
D_{\theta}(t, x_{2, 3}-y_{2, 3})
\prod_{i=2}^3 \frac{F_i(x_i, y_i)}{|x_i - y_i|} \, dx_{2, 3} \, dy_{2, 3}
\nonumber \\
&\lesssim \frac{\widetilde{F}_2(I_2, J_2)}{\ird(I_2, J_2)^{\delta_0}} |J_2| \, 
\frac{\widetilde{F}_3(I_3, J_3)}{\ird(I_3, J_3)^{\delta_0}} |J_3| \, 
\bigg[\frac{t |J_2|}{|J_3|} + \frac{|J_3|}{t |J_2|} \bigg]^{-\theta}. 
\end{align}


\item If $\rd(I_2, J_2)<1$ and $\rd(I_3, J_3)<1$, then 
\begin{align}\label{def:R2302}
\mathscr{R}_{2, 3}^{0, 2, t}(I_{2, 3}, J_{2, 3})
&:= \int_{J_2 \times (3J_3)^c} \int_{J_{2, 3}}
D_{\theta}(t, x_{2, 3} - y_{2, 3}) \prod_{i=2}^3 \frac{F_i(x_i, y_i)}{|x_i - y_i|} dx_{2, 3} \, dy_{2, 3}
\nonumber \\ 
&\lesssim \widetilde{F}_2(J_2) |J_2| \widetilde{F}_3(J_3) |J_3|
[t|J_2| |J_3|^{-1}]^{\theta}. 
\end{align}


\item If $\rd(I_2, J_2)<1$ and $\rd(I_3, J_3)<1$, then 
\begin{align}\label{def:R2322}
\mathscr{R}_{2, 3}^{2, 2, t}(I_{2, 3}, J_{2, 3})
&:= \int_{(3J_2)^c \times (3J_3)^c} \int_{J_{2, 3}}
D_{\theta}(t, x_{2, 3} - y_{2, 3}) \prod_{i=2}^3 \frac{F_i(x_i, y_i)}{|x_i - y_i|} dx_{2, 3} \, dy_{2, 3}
\nonumber \\ 
&\lesssim \widetilde{F}_2(J_2) |J_2| \widetilde{F}_3(J_3) |J_3| 
\bigg[\frac{t |J_2|}{|J_3|} + \frac{|J_3|}{t |J_2|} \bigg]^{-\theta}. 
\end{align}
\end{enumerate}
\end{lemma}


\begin{proof}
We begin with showing \eqref{def:R1200}. Let $\rd(I_1, J_1)<1$ and $\rd(I_2, J_2)<1$. Fix $x_1 \in J_1$ and $y_1 \in I_1$. Writing $t_2 := \frac{|x_1-y_1|}{t}$ and using \eqref{def:Q0}, we obtain 
\begin{align*}
\mathscr{Q}_2^{0, t_2}(I_2, J_2) 
&\lesssim \widetilde{F}_2(J_2) |I_2|^{\frac1r} |J_2|^{1-\frac1r} 
\big[t_2 |J_2| + t_2^{-1} |J_2|^{-1} \big]^{-\theta} 
\\
&= \widetilde{F}_2(J_2) |I_2|^{\frac1r} |J_2|^{1-\frac1r}
\bigg[t_1 |x_1-y_1| + \frac{1}{t_1 |x_1-y_1|}\bigg]^{-\theta}, 
\end{align*}
where $t_1 := \frac{|J_2|}{t}$. Then, \eqref{def:Q0} implies 
\begin{align*}
\mathscr{R}_{1, 2}^t(I_{1, 2}, J_{1, 2}) 
&=\int_{I_1} \int_{J_1} \mathscr{Q}_2^{0, t_2}(I_2, J_2) \frac{F_1(x_1, y_1)}{|x_1 - y_1|} dx_1 \, dy_1 
\\
&\lesssim \mathscr{Q}_1^{0, t_1}(I_1, J_1) \widetilde{F}_2(J_2) |I_2|^{\frac1r} |J_2|^{1-\frac1r}
\\
&\lesssim \widetilde{F}_1(I_1) |I_1|^{1-\frac1r} |J_1|^{\frac1r} 
\widetilde{F}_2(J_2) |I_2|^{\frac1r} |J_2|^{1-\frac1r}
\bigg[\frac{|I_1| |J_2|}{t} + \frac{t}{|I_1| |J_2|}\bigg]^{-\theta}.
\end{align*}
This justifies \eqref{def:R1200}. Much as in the same way, one can prove \eqref{def:R1300} and \eqref{def:R2300}.  

To show \eqref{def:R2301}, observe that $3J_3=J_3^0 \uplus J_3^1 \uplus J_3^2$, where $J_3^0 = J_3$ and $J_3^i \in \D$ with $\ell(J_3^i) = \ell(J_3)$ for each $i=1, 2$. Then, $\rd(J_2, J_3^i) \le 2$ for each $i=0, 1, 2$, which together with $t_2 := \frac{t}{|x_3-y_3|}$, $t_3 := \frac{1}{t |J_2|}$, and \eqref{def:Q0}--\eqref{def:Q1} gives 
\begin{align*}
\mathscr{R}_{2, 3}^{0, 1, t}(I_{2, 3}, J_{2, 3}) 
&= \sum_{i=0}^2 \int_{I_3 \setminus 3J_3} \int_{J_3} \mathscr{Q}_2^{0, t_2}(J_2, J_3^i) 
\frac{F_3(x_3, y_3)}{|x_3 - y_3|} dx_3 \, dy_3 
\\
&\lesssim \int_{I_3 \setminus 3J_3} \int_{J_3} \widetilde{F}_2(J_2) |J_2| 
\big[t_2 |J_2| + t_2^{-1} |J_2|^{-1}\big]^{-\theta} 
\frac{F_3(x_3, y_3)}{|x_3 - y_3|} dx_3 \, dy_3 
\\
&=\widetilde{F}_2(J_2) |J_2| \, \mathscr{Q}_3^{1, t_3}(I_3, J_3) 
\\
&\lesssim \widetilde{F}_2(J_2) |J_2| 
\frac{\widetilde{F}_3(I_3, J_3)}{\ird(I_3, J_3)^{\delta_0}} |J_3| 
\bigg[\frac{t|J_2|}{|J_3|} + \frac{|J_3|}{t|J_2|}\bigg]^{-\theta}. 
\end{align*}
Similarly, \eqref{def:R2310} holds and 
\begin{align*}
\mathscr{R}_{2, 3}^{0, 2, t}(I_{2, 3}, J_{2, 3}) 
&\lesssim \int_{(3J_3)^c} \int_{J_3} \mathscr{Q}_2^{0, t_2}(J_2, J_2) 
\frac{F_3(x_3, y_3)}{|x_3 - y_3|} dx_3 \, dy_3 
\\
&\lesssim \widetilde{F}_2(J_2) |J_2| \, \mathscr{Q}_3^{2, t_3}(J_3) 
\lesssim \widetilde{F}_2(J_2) |J_2| \widetilde{F}_3(J_3) |J_3| 
[t|J_2| |J_3|^{-1}]^{\theta}, 
\end{align*}
which shows \eqref{def:R2302}. 


To proceed, for $i=2, 3$, let
\begin{align*}
R_i^{k_i} := \big\{x_i \in \R: 2^{k_i} \ell(J_i) < |x_i - c_{J_i}| \le 2^{k_i+1} \ell(J_i)\big\}, \quad k_i \ge k_i^0,
\end{align*}
where $k_i^0 := \max\{0, \, \log_2 (\ird(I_i, J_i)/2)\}$. Write $s := \frac{t |J_2|}{|J_3|}$. As shown in \eqref{FXY-1}, there holds $F_i(x_i, y_i) \lesssim \widetilde{F}_i(I_i, J_i)$. Hence, 
\begin{align*}
\mathscr{R}_{2, 3}^{1, 1, t}(I_{2, 3}, J_{2, 3}) 
&\lesssim \sum_{k_2 \ge k_2^0 \atop k_3 \ge k_3^0} 
\int_{R_2^{k_2} \times R_3^{k_3}} \int_{J_2 \times J_3} 
\frac{(2^{-k_2} + 2^{-k_3})^{\delta_{2, 3}}}{(2^{k_2-k_3} s + 2^{-k_2+k_3}s^{-1})^{\theta}}
\prod_{i=2}^3 \frac{\widetilde{F}_i(I_i, J_i)}{2^{k_i} \ell(J_i)} \, dx_{2, 3} \, dy_{2, 3}
\nonumber \\
&\lesssim \frac{\widetilde{F}_2(I_2, J_2)}{\ird(I_2, J_2)^{\delta_0}} |J_2| \, 
\frac{\widetilde{F}_3(I_3, J_3)}{\ird(I_3, J_3)^{\delta_0}} |J_3| \, 
\bigg[\frac{t |J_2|}{|J_3|} + \frac{|J_3|}{t |J_2|} \bigg]^{-\theta},  
\end{align*}
where we have used the following eatimate
\begin{align*}
&\sum_{k_2 \ge k_2^0 \atop k_3 \ge k_3^0} 
\frac{(2^{-k_2} + 2^{-k_3})^{\delta_{2, 3}}}{(2^{k_2-k_3} s + 2^{-k_2+k_3} s^{-1})^{\theta}}
\\
&\lesssim \sum_{\substack{k_2 \ge k_2^0 \\ k_3 \ge k_3^0 \\ k_2 \le k_3}} 
\frac{2^{-k_2\delta_{2, 3}}}{(2^{-k_2+k_3} s^{-1})^{\theta}}
+ \sum_{\substack{k_2 \ge k_2^0 \\ k_3 \ge k_3^0 \\ k_2 > k_3}} 
\frac{2^{-k_3\delta_{2, 3}}}{(2^{k_2-k_3} s)^{\theta}}
\\
&\le s^{\theta} \sum_{\substack{k_2 \ge k_2^0 \\ k_3 \ge k_3^0}} 
2^{-k_2(\delta_{2, 3} - \theta)} 2^{-k_3 \theta}
+ s^{-\theta} \sum_{\substack{k_2 \ge k_2^0 \\ k_3 \ge k_3^0}} 
2^{-k_2 \theta} 2^{-k_3(\delta_{2, 3} - \theta)} 
\\
&\lesssim s^{\theta} 2^{-k_2^0(\delta_{2, 3} - \theta)} 2^{-k_3^0 \theta}
+ s^{-\theta} 2^{-k_2^0 \theta} 2^{-k_3^0(\delta_{2, 3} - \theta)} 
\\
&\lesssim (s^{\theta} + s^{-\theta}) 2^{-k_2^0 \delta_0} 2^{-k_3^0 \delta_0}. 
\end{align*}
This shows \eqref{def:R2311}. Similarly, one can get \eqref{def:R2322}. The proof is complete. 
\end{proof}





\subsection{Triple integrals}

\begin{lemma}\label{lem:triple}
Let $\theta \in (0, 1)$ and $t>0$. For each $i \in \{1, 2, 3\}$, let $(F_{i, 1}, F_{i, 2}, F_{i, 3}) \in \mathscr{F}$ satisfy that $F_{i, 1}$ is increasing, $F_{i, 2}$ and $F_{i, 3}$ are decreasing. Let $I=I_1 \times I_2 \times I_3$ and $J=J_1 \times J_2 \times J_3$ be dyadic Zygmund rectangles so that $\ell(I_1) \le \ell(J_1)$, $\ell(I_2) \ge \ell(J_2)$, and $\ell(I_3) \ge \ell(J_3)$. Then the following hold:
\begin{enumerate}
\item If $\rd(I_i, J_i)<1$ for each $i=1, 2, 3$, then 
\begin{align}\label{def:S000}
\mathscr{S}_{0, 0, 0}(I, J)
&:= \int_{I_1 \times I_2 \times I_3} \int_{J_1 \times J_2 \times J_3}  
D_{\theta}(x-y) \prod_{i=1}^3 \frac{F_i(x_i, y_i)}{|x_i - y_i|} \, dx \, dy
\nonumber \\
&\lesssim \widetilde{F}_1(I_1) |I_1|^{1-\frac1r} |J_1|^{\frac1r} \, 
\widetilde{F}_2(J_2) |I_2|^{\frac1r} |J_2|^{1-\frac1r} 
\widetilde{F}_3(J_3) |I_3|^{\frac1r} |J_3|^{1-\frac1r}. 
\end{align}

\item If $\rd(I_3, J_3)<1$, then 
\begin{align}\label{def:S001}
\mathscr{S}_{0, 0, 1}(I, J)
&:= \int_{I_1 \times 3J_2 \times (I_3 \setminus 3J_3)} \int_{3I_1 \times J_2 \times J_3} 
D_{\theta}(x-y) \prod_{i=1}^3 \frac{F_i(x_i, y_i)}{|x_i - y_i|} \, dx \, dy
\nonumber \\ 
&\lesssim \widetilde{F}_1(I_1) |I_1|^{1-\frac1r} |J_1|^{\frac1r} \, 
\widetilde{F}_2(J_2) |I_2|^{\frac1r} |J_2|^{1-\frac1r} \, 
\widetilde{F}_3(I_3, J_3) |J_3| \frac{\rs(I_1, J_1)^{\theta}}{\ird(I_3, J_3)^{\theta}}. 
\end{align}


\item There holds 
\begin{align}\label{def:S002}
\mathscr{S}_{0, 0, 2}(I, J)
&:= \int_{I_1 \times 3J_2 \times (3J_3)^c} \int_{3I_1 \times J_2 \times J_3} 
D_{\theta}(x-y) \prod_{i=1}^3 \frac{F_i(x_i, y_i)}{|x_i - y_i|} \, dx \, dy
\nonumber \\ 
&\lesssim \widetilde{F}_1(I_1) |I_1|^{1-\frac1r} |J_1|^{\frac1r} \, 
\widetilde{F}_2(J_2) |I_2|^{\frac1r} |J_2|^{1-\frac1r} \, 
\widetilde{F}_3(J_3) |J_3|. 
\end{align}



\item If $\rd(I_i, J_i)<1$ for each $i=2, 3$, then 
\begin{align}\label{def:S011}
\mathscr{S}_{0, 1, 1}(I, J)
&:= \int_{I_1 \times (I_2 \setminus 3J_2) \times (I_3 \setminus 3J_3)} \int_{3I_1 \times J_2 \times J_3} 
\bigg(\frac{|J_2|}{|x_2-y_2|} + \frac{|J_3|}{|x_3-y_3|}\bigg)^{\delta_{2, 3}}
\nonumber \\ 
&\qquad\times D_{\theta}(x-y) \prod_{i=1}^3 \frac{F_i(x_i, y_i)}{|x_i - y_i|} \, dx \, dy
\nonumber \\ 
&\lesssim \widetilde{F}_1(I_1) |I_1|^{1-\frac1r} |J_1|^{\frac1r} \, 
\frac{\widetilde{F}_2(I_2, J_2)}{\ird(I_2, J_2)^{\delta_0}} |J_2| \, 
\frac{\widetilde{F}_3(I_3, J_3)}{\ird(I_3, J_3)^{\delta_0}} |J_3| \,  
\rs(I_1, J_1)^{-\theta}. 
\end{align}

\item If $\rd(I_i, J_i)<1$ for each $i=1, 2, 3$, then 
\begin{align}\label{def:S111}
\mathscr{S}_{1, 1, 1}(I, J)
&:= \int_{I_1 \times (I_2 \setminus 3J_2) \times (I_3 \setminus 3J_3)} 
\int_{(J_1 \setminus 3I_1) \times J_2 \times J_3} 
\bigg(\frac{|I_1|}{|x_1-y_1|}\bigg)^{\delta_1}
\nonumber \\ 
&\qquad\times \bigg(\frac{|J_2|}{|x_2-y_2|} + \frac{|J_3|}{|x_3-y_3|}\bigg)^{\delta_{2, 3}} 
D_{\theta}(x-y) \prod_{i=1}^3 \frac{F_i(x_i, y_i)}{|x_i - y_i|} \, dx \, dy
\nonumber \\ 
&\lesssim \frac{\widetilde{F}_1(I_1, J_1)}{\ird(I_1, J_1)^{\delta_0}} |I_1| \, 
\frac{\widetilde{F}_2(I_2, J_2)}{\ird(I_2, J_2)^{\delta_0}} |J_2| \, 
\frac{\widetilde{F}_3(I_3, J_3)}{\ird(I_3, J_3)^{\delta_0}} |J_3| \,  
\rs(I_1, J_1)^{-\theta}. 
\end{align}


\item If $\rd(I_i, J_i)<1$ for each $i=2, 3$, then 
\begin{align}\label{def:S211}
\mathscr{S}_{2, 1, 1}(I, J)
&:= \int_{I_1 \times (I_2 \setminus 3J_2) \times (I_3 \setminus 3J_3)} \int_{(3I_1)^c \times J_2 \times J_3} 
\bigg(\frac{|I_1|}{|x_1-y_1|}\bigg)^{\delta_1}
\nonumber \\ 
&\qquad\times \bigg(\frac{|J_2|}{|x_2-y_2|} + \frac{|J_3|}{|x_3-y_3|}\bigg)^{\delta_{2, 3}} 
D_{\theta}(x-y) \prod_{i=1}^3 \frac{F_i(x_i, y_i)}{|x_i - y_i|} \, dx \, dy
\nonumber \\ 
&\lesssim \widetilde{F}_1(I_1) |I_1| \rs(I_1, J_1)^{-\theta} \, 
\frac{\widetilde{F}_2(I_2, J_2)}{\ird(I_2, J_2)^{\delta_0}} |J_2|
\frac{\widetilde{F}_3(I_3, J_3)}{\ird(I_3, J_3)^{\delta_0}} |J_3|. 
\end{align}


\item There holds 
\begin{align}\label{def:Sd0d}
\mathscr{S}_{\dagger, 0, \dagger}(I, J)
&:= \int_{I_1 \times J_2 \times (3J_3 \setminus J_3)} 
\int_{(3I_1 \setminus I_1) \times J_2 \times J_3} D_{\theta}(x-y)
\prod_{i=1}^3 \frac{F_i(x_i, y_i)}{|x_i - y_i|} \, dx \, dy
\nonumber \\
&\lesssim \widetilde{F}_1(I_1) |I_1| \, 
\widetilde{F}_2(J_2) |J_2| \, 
\widetilde{F}_3(J_3) |J_3|.   
\end{align}


\item If $\rd(I_1, J_1)<1$, then 
\begin{align}\label{def:S10d}
\mathscr{S}_{1, 0, \dagger}(I, J)
&:= \int_{I_1 \times J_2 \times (3J_3 \setminus J_3)} 
\int_{(J_1 \setminus 3I_1) \times J_2 \times J_3} 
\frac{|I_1|^{\delta_1}}{|x_1-y_1|^{\delta_1}}
D_{\theta}(x-y) \prod_{i=1}^3 \frac{F_i(x_i, y_i)}{|x_i - y_i|} \, dx \, dy
\nonumber \\
&\lesssim \frac{\widetilde{F}_1(I_1, J_1)}{\ird(I_1, J_1)^{\delta_0}} |I_1| \,  
\widetilde{F}_2(J_2) |J_2| \, \widetilde{F}_3(J_3) |J_3|.  
\end{align}


\item There holds  
\begin{align}\label{def:S20d}
\mathscr{S}_{2, 0, \dagger}(I, J)
&:= \int_{I_1 \times J_2 \times (3J_3 \setminus J_3)} 
\int_{(3I_1)^c \times J_2 \times J_3} 
\frac{|I_1|^{\delta_1}}{|x_1-y_1|^{\delta_1}}
D_{\theta}(x-y) \prod_{i=1}^3 \frac{F_i(x_i, y_i)}{|x_i - y_i|} \, dx \, dy
\nonumber \\
&\lesssim \widetilde{F}_1(I_1) |I_1| \,  
\widetilde{F}_2(J_2) |J_2| \, \widetilde{F}_3(J_3) |J_3|.  
\end{align}



\item If $\rd(I_3, J_3)<1$, then 
\begin{align}\label{def:S201}
\mathscr{S}_{2, 0, 1}(I, J)
&:= \int_{I_1 \times 3J_2 \times (I_3 \setminus 3J_3)} \int_{(3I_1)^c \times J_2 \times J_3} 
\frac{|I_1|^{\delta_1}}{|x_1-y_1|^{\delta_1}} 
D_{\theta}(x-y) \prod_{i=1}^3 \frac{F_i(x_i, y_i)}{|x_i - y_i|} \, dx \, dy
\nonumber \\ 
&\lesssim \widetilde{F}_1(I_1) |I_1| \rs(I_1, J_1)^{-\theta} \, 
\widetilde{F}_2(J_2) |J_2| \, 
\frac{\widetilde{F}_3(I_3, J_3)}{\ird(I_3, J_3)^{\delta_0}} |J_3|.
\end{align}


\item There holds  
\begin{align}\label{def:Sd22}
\mathscr{S}_{\dagger, 2, 2}(I, J)
&:= \int_{I_1 \times (3J_2)^c \times (3J_3)^c}  
\int_{(3I_1 \setminus I_1) \times J_2 \times J_3} 
\bigg(\frac{|J_2|}{|x_2 - y_2|} + \frac{|J_3|}{|x_3 - y_2|}\bigg)^{\delta_{2, 3}}
\nonumber \\
&\qquad\times D_{\theta}(x-y) \prod_{i=1}^3 \frac{F_i(x_i, y_i)}{|x_i - y_i|} \, dx \, dy
\nonumber \\
&\lesssim \widetilde{F}_1(I_1) |I_1| \,  
\widetilde{F}_2(J_2) |J_2| \, \widetilde{F}_3(J_3) |J_3| \, 
\rs(I_1, J_1)^{-\theta}.  
\end{align}



\item If $\rd(I_1, J_1)<1$, then 
\begin{align}\label{def:S122}
\mathscr{S}_{1, 2, 2}(I, J)
&:= \int_{I_1 \times (3J_2)^c \times (3J_3)^c}  
\int_{(J_1 \setminus 3I_1) \times J_2 \times J_3} 
\bigg(\frac{|J_2|}{|x_2 - y_2|} + \frac{|J_3|}{|x_3 - y_2|}\bigg)^{\delta_{2, 3}}
\nonumber \\
&\qquad\times D_{\theta}(x-y) \prod_{i=1}^3 \frac{F_i(x_i, y_i)}{|x_i - y_i|} \, dx \, dy
\nonumber \\
&\lesssim \widetilde{F}_1(I_1, J_1) |I_1| \,  
\widetilde{F}_2(J_2) |J_2| \, \widetilde{F}_3(J_3) |J_3| 
\rs(I_1, J_1)^{-\theta}.   
\end{align}


\item There holds  
\begin{align}\label{def:S222}
\mathscr{S}_{2, 2, 2}(I, J)
&:= \int_{I_1 \times (3J_2)^c \times (3J_3)^c} \int_{(3I_1)^c \times J_2 \times J_3} 
\bigg(\frac{|I_1|}{|x_1-y_1|}\bigg)^{\delta_1}
\nonumber \\ 
&\qquad\times \bigg(\frac{|J_2|}{|x_2-y_2|} + \frac{|J_3|}{|x_3-y_3|}\bigg)^{\delta_{2, 3}} 
D_{\theta}(x-y) \prod_{i=1}^3 \frac{F_i(x_i, y_i)}{|x_i - y_i|} \, dx \, dy
\nonumber \\ 
&\lesssim \widetilde{F}_1(I_1) |I_1| \,  
\widetilde{F}_2(J_2) |J_2| \, \widetilde{F}_3(J_3) |J_3| 
\rs(I_1, J_1)^{-\theta}.  
\end{align}



\item There holds  
\begin{align}\label{def:Sdd2}
\mathscr{S}_{\dagger, \dagger, 2}(I, J) 
&:= \int_{I_1 \times (3J_2 \setminus J_2) \times (3J_3)^c} 
\int_{(3I_1 \setminus I_1) \times J_2 \times J_3} 
D_{\theta}(x-y) \prod_{i=1}^3 \frac{F_i(x_i, y_i)}{|x_i - y_i|} \, dx \, dy
\nonumber \\ 
&\lesssim \widetilde{F}_1(I_1) |I_1|\, 
\widetilde{F}_2(J_2) |J_2| \, 
\widetilde{F}_3(J_3) |J_3|. 
\end{align}

\end{enumerate}
\end{lemma}

\begin{proof}
Note that 
\begin{align}\label{33IJ}
3I_1 = I_1^0 \uplus I_1^1 \uplus I_1^2, 
\qquad
3J_2 = J_2^0 \uplus J_2^1 \uplus J_2^2, 
\quad\text{ and }\quad 
3J_3 = J_3^0 \uplus J_3^1 \uplus J_3^2, 
\end{align}
where $I_1^0 = I_1$, $J_2^0 = J_2$, $J_3^0 = J_3$, and $I_1^i, J_2^i, J_3^i \in \D$ with $|I_1^i| = |I_1|$, $|J_2^i| = |J_2|$, and $|J_3^i| = |J_3|$ for each $i=1, 2$. We begin with the proof of \eqref{def:S000}. Letting $t=|x_3-y_3|$ and $t_3=\frac{1}{|I_1| |J_2|}$, we apply \eqref{def:R1200} and \eqref{def:Q0} to deduce that 
\begin{align*}
\mathscr{S}_{0, 0, 0}(I, J)
&= \int_{I_3} \int_{J_3} \mathscr{R}_{1, 2}^{0, 0, t}(I_{1, 2}, J_{1, 2}) 
\frac{F_3(x_3, y_3)}{|x_3-y_3|} \, dx_3 \, dy_3 
\\ 
&\lesssim \widetilde{F}_1(I_1) |I_1|^{1-\frac1r} |J_1|^{\frac1r} \, 
\widetilde{F}_2(J_2) |I_2|^{\frac1r} |J_2|^{1-\frac1r} 
\\
&\quad\times \int_{I_3} \int_{J_3} \bigg[t_3 |x_3-y_3| + \frac{1}{t_3 |x_3-y_3|} \bigg]^{-\theta} 
\frac{F_3(x_3, y_3)}{|x_3-y_3|} \, dx_3 \, dy_3
\\
&= \widetilde{F}_1(I_1) |I_1|^{1-\frac1r} |J_1|^{\frac1r} \, 
\widetilde{F}_2(J_2) |I_2|^{\frac1r} |J_2|^{1-\frac1r} 
\mathscr{Q}_3^{0, t_3}(I_3, J_3) 
\\
&\lesssim \widetilde{F}_1(I_1) |I_1|^{1-\frac1r} |J_1|^{\frac1r} \, 
\widetilde{F}_2(J_2) |I_2|^{\frac1r} |J_2|^{1-\frac1r} 
\widetilde{F}_3(J_3) |I_3|^{\frac1r} |J_3|^{1-\frac1r}. 
\end{align*}
In view of \eqref{33IJ}, the estimates \eqref{def:R1200} and \eqref{def:Q1} imply 
\begin{align*}
\mathscr{S}_{0, 0, 1}(I, J)
&= \sum_{i, j=0}^2 \int_{I_3 \setminus 3J_3} \int_{J_3} 
\mathscr{R}_{1, 2}^{0, 0, t}(I_1 \times J_2^i, I_1^j \times J_2) 
\frac{F_3(x_3, y_3)}{|x_3-y_3|} \, dx_3 \, dy_3 
\\ 
&\lesssim \widetilde{F}_1(I_1) |I_1|^{1-\frac1r} |J_1|^{\frac1r} \, 
\widetilde{F}_2(J_2) |I_2|^{\frac1r} |J_2|^{1-\frac1r} 
\mathscr{Q}_3^{1, t_3}(I_3, J_3) 
\\
&\lesssim \widetilde{F}_1(I_1) |I_1|^{1-\frac1r} |J_1|^{\frac1r} \, 
\widetilde{F}_2(J_2) |I_2|^{\frac1r} |J_2|^{1-\frac1r} 
\widetilde{F}_3(I_3, J_3) |J_3| \frac{\rs(I_1, J_1)^{\theta}}{\ird(I_3, J_3)^{\theta}},  
\end{align*}
which gives \eqref{def:S001}. Analogously, it follows from \eqref{def:R1200} and \eqref{def:Q2} that 
\begin{align*}
\mathscr{S}_{0, 0, 2}(I, J)
&\lesssim \widetilde{F}_1(I_1) |I_1|^{1-\frac1r} |J_1|^{\frac1r} \, 
\widetilde{F}_2(J_2) |I_2|^{\frac1r} |J_2|^{1-\frac1r} 
\mathscr{Q}_3^{2, t_3}(J_3) 
\\
&\lesssim \widetilde{F}_1(I_1) |I_1|^{1-\frac1r} |J_1|^{\frac1r} \, 
\widetilde{F}_2(J_2) |I_2|^{\frac1r} |J_2|^{1-\frac1r} 
\widetilde{F}_3(J_3) |J_3|, 
\end{align*}
which shows \eqref{def:S002}. Symmetrically to $\mathscr{S}_{0, 0, 2}(I, J)$, one can prove \eqref{def:S200}.

To proceed, write $t=|I_1|$ and $t_1 = \frac{|x_2-y_2|}{|x_3-y_3|}$. Then, \eqref{def:Q0} and \eqref{def:R2311} yield 
\begin{align*}
\mathscr{S}_{0, 1, 1}(I, J)
&= \int_{(I_2 \setminus 3J_2) \times (I_3 \setminus 3J_3)} \int_{J_2 \times J_3} 
\mathscr{Q}_1^{0, t_1}(I_1, J_1) 
\bigg(\frac{|x_2 - c_{J_2}|}{|x_2-y_2|} + \frac{|x_3 - c_{J_3}|}{|x_3-y_3|}\bigg)^{\delta_{2, 3}}
\nonumber \\ 
&\qquad\times D_{\theta}(x-y) \prod_{i=1}^3 \frac{F_i(x_i, y_i)}{|x_i - y_i|} \, dx \, dy
\nonumber \\ 
&\lesssim \widetilde{F}_1(I_1) |I_1|^{1-\frac1r} |J_1|^{\frac1r} \, 
\mathscr{R}_{2, 3}^{1, 1, t}(I_{2, 3}, J_{2, 3}) 
\nonumber \\ 
&\lesssim \widetilde{F}_1(I_1) |I_1|^{1-\frac1r} |J_1|^{\frac1r} \, 
\frac{\widetilde{F}_2(I_2, J_2)}{\ird(I_2, J_2)^{\delta_0}} |J_2| \, 
\frac{\widetilde{F}_3(I_3, J_3)}{\ird(I_3, J_3)^{\delta_0}} |J_3| \,  
\rs(I_1, J_1)^{-\theta}. 
\end{align*}
Similarly, replacing $\mathscr{Q}_1^{0, t_1}(I_1, J_1) $ by $\mathscr{Q}_1^{1, t_1}(I_1, J_1) $ and $\mathscr{Q}_1^{2, t_1}(I_1, J_1) $ respectively, we apply \eqref{def:Q1} and \eqref{def:Q2} to obtain \eqref{def:S111} and \eqref{def:S211}. Moreover, \eqref{def:Sd0d} is just a consequence of \eqref{def:S000} and \eqref{33IJ}.  

Next, letting $t=|x_1 - y_1|$ and $t_1 = \frac{|J_2|}{|J_3|}$, we invoke \eqref{def:R2300} and \eqref{def:Q1} to obtain 
\begin{align*}
\mathscr{S}_{1, 0, \dagger}(I, J)
&= \int_{I_1} \int_{J_1 \setminus 3I_1} 
\mathscr{R}_{2, 3}^{0, 0, t}(I_{2, 3}, J_{2, 3}) 
\bigg(\frac{|I_1|}{|x_1 - y_1|}\bigg)^{\delta_1} 
\frac{F_1(x_1, y_1)}{|x_1 - y_1|} \, dx_1 \, dy_1 
\nonumber \\ 
&\lesssim \mathscr{Q}_1^{1, t_1}(I_1, J_1) \, 
\widetilde{F}_2(J_2) |J_2| \, \widetilde{F}_3(J_3) |J_3| 
\nonumber \\ 
&\lesssim \frac{\widetilde{F}_1(I_1, J_1)}{\ird(I_1, J_1)^{\delta_0}} |I_1| \, 
\widetilde{F}_2(J_2) |J_2| \, \widetilde{F}_3(J_3) |J_3|. 
\end{align*}
As above, \eqref{def:Q2} gives \eqref{def:S20d}. In the same way, replacing $\mathscr{R}_{2, 3}^{0, 0, t}(I_{2, 3}, J_{2, 3})$ and $\mathscr{Q}_1^{1, t_1}(I_1)$ by $\mathscr{R}_{2, 3}^{0, 1, t}(I_{2, 3}, J_{2, 3})$ and $\mathscr{Q}_1^{2, t_1}(I_1)$ respectively, we obtain \eqref{def:S201}.

Set $t=|I_1|$ and $t_1 = \frac{|x_2-y_2|}{|x_3-y_3|}$. Then, \eqref{def:Q0} and \eqref{def:R2322} give 
\begin{align*}
\mathscr{S}_{\dagger, 2, 2}(I, J)
&= \int_{(3J_2)^c \times (3J_3)^c}  \int_{J_2 \times J_3} 
\big[\mathscr{Q}_1^{0, t_1}(I_1, I_1^1) + \mathscr{Q}_1^{0, t_1}(I_1, I_1^2) \big] 
\nonumber \\
&\quad\times 
\bigg(\frac{|J_2|}{|x_2 - y_2|} + \frac{|J_3|}{|x_3 - y_3|}\bigg)^{\delta_{2, 3}} 
D_{\theta}(x-y) \prod_{i=2}^3 \frac{F_i(x_i, y_i)}{|x_i - y_i|} \, dx_{2, 3} \, dy_{2, 3}
\nonumber \\
&\lesssim \widetilde{F}_1(I_1) |I_1| \, \mathscr{R}_{2, 3}^{2, 2, t}(I_{2, 3}, J_{2, 3}) 
\nonumber \\
&\lesssim \widetilde{F}_1(I_1) |I_1| \rs(I_1, J_1)^{-\theta} \,  
\widetilde{F}_2(J_2) |J_2| \, \widetilde{F}_3(J_3) |J_3|.  
\end{align*}
In addition, replacing $\mathscr{Q}_1^{0, t_1}(I_1, I_1^1) + \mathscr{Q}_1^{0, t_1}(I_1, I_1^2)$ by $\mathscr{Q}_1^{1, t_1}(I_1, J_1)$ and $\mathscr{Q}_1^{2, t_1}(I_1)$ respectively, we invoke \eqref{def:Q1} and \eqref{def:Q2} to deduce \eqref{def:S122} and \eqref{def:S222}.

Finally, let $t=|x_3 - y_3|$ and $t_3 = \frac{1}{|I_1| |J_2|}$. It follows from \eqref{33IJ}, \eqref{def:R2311}, and \eqref{def:Q2} that 
\begin{align*}
\mathscr{S}_{\dagger, \dagger, 2}(I, J) 
&= \sum_{i, j=1}^2 \int_{(3J_3)^c} \int_{J_3} \mathscr{R}_{2, 3}^{1, 1, t}(I_1 \times J_2^j, I_1^i \times J_2) 
\frac{F_3(x_3, y_3)}{|x_3 - y_3|} \, dx_3 \, dy_3  
\\ 
&\lesssim \widetilde{F}_1(I_1) |I_1| \, 
\widetilde{F}_2(J_2) |J_2| \, 
\mathscr{Q}_3^{2, t_3}(J_3)  
\\ 
&\lesssim \widetilde{F}_1(I_1) |I_1| \, 
\widetilde{F}_2(J_2) |J_2| \, 
\widetilde{F}_3(J_3) |J_3|,  
\end{align*} 
which shows \eqref{def:Sdd2}. The proof is complete. 
\end{proof}







%%%%%%%%%%%%%%%%%%%%%%%%%%%% SECTION SECTION SECTION %%%%%%%%%%%%%%%%%%%%%
%%%%%%%%%%%%%%%%%%%%%%%%%%%% SECTION SECTION SECTION %%%%%%%%%%%%%%%%%%%%%
\section{Reductions of the proof}\label{sec:reduction} 
Let $T$ be a Zygmund singular integral operator satisfying the assumptions \eqref{list-1}--\eqref{list-4}. We are going to prove that 
\begin{align}\label{reduction-1}
\text{$T$ is compact on $L^2(\R^3)$.}
\end{align}
Assuming \eqref{reduction-1} momentarily, we can conclude the proof of Theorem \ref{thm:compact} as follows. We only show part \eqref{list-02} because the proof of part \eqref{list-01} can be given in a similar way. 

By definition, $T$ verifies the weak boundedness property. Note that $\theta=1$ in the current scenario. Thus, \cite[Theorem 1.9]{HLMV} implies that 
\begin{align}\label{T-Lp}
\text{$T$ is bounded on $L^q(v)$ for all $q \in (1, \infty)$ and $v \in A_{q, \Z}$}.
\end{align}  
Fix $p \in (1, \infty)$, $w \in A_{p, \Z}$, $p_1=2$, and $w_1 \equiv 1$. We claim that there exist $\eta \in (0, 1)$, $p_0=p_0(\eta) \in (1, \infty)$, and $w_0=w_0(\eta) \in A_{p_0, \Z}$ such that 
\begin{align}\label{wpp}
\frac1p=\frac{1}{p_0} + \frac{1}{p_1}
\qquad\text{ and }\qquad 
w^{\frac1p} = w_0^{\frac{1-\eta}{p_0}} w_1^{\frac{\eta}{p_1}}. 
\end{align}
Indeed, \eqref{wpp} can be shown as in the proof of \cite[Lemma 4.1]{COY}. Although the latter is given for cubes, the same argument still holds for Zygmund rectangles because the proof heavily depends on the reverse H\"{o}lder inequality for $A_{p, \Z}$ weights, which is true. In fact, given $p \in (1, \infty)$, $w \in A_{p, \Z}$, and a Zygmund rectangle $I=I_1 \times I_{2, 3}$, we have $[w(\cdot, x_2, x_3)]_{A_p(\R)} \le [w]_{A_{p, \Z}}$ uniformly on $x_2, x_3 \in \R$ and $[\langle w \rangle_{I_1}]_{A_p(\R^2)} \le [w]_{A_{p, \Z}}$ uniformly on $I_1$. Then, by the standard one-parameter reverse H\"{o}lder inequality there exist $r_1, r_2 > 1$ such that 
\begin{align}\label{eq:weta-1}
\bigg(\fint_{I_1} w^{r_1} \, dx_1 \bigg)^{\frac{1}{r_1}} 
\lesssim \fint_{I_1} w \, dx_1 
\qquad \text{uniformly on } x_2, x_3 \in \R
\end{align}
and 
\begin{align}\label{eq:weta-2}
\bigg(\fint_{I_{2, 3}} \langle w \rangle_{I_1}^{r_2} \, dx_2 dx_3 \bigg)^{\frac{1}{r_2}} 
\lesssim \fint_{I_{2, 3}} \langle w \rangle_{I_1} \, dx_2 dx_3 
\qquad\text{uniformly on } I_1. 
\end{align}
Picking $r := \min\{r_1, r_2\}$, we use Jensen's inequality and \eqref{eq:weta-1}--\eqref{eq:weta-2} to obtain that 
\begin{align*}
\fint_I w^r \, dx
&=\fint_{I_{2, 3}} \bigg(\fint_{I_1} w^r \, dx_1\bigg) \, dx_2 \, dx_3
\le \fint_{I_{2, 3}} \bigg(\fint_{I_1} w^{r_1} \, dx_1\bigg)^{\frac{r}{r_1}} \, dx_2 \, dx_3
\\
&\lesssim \fint_{I_{2, 3}} \bigg(\fint_{I_1} w \, dx_1\bigg)^r \, dx_2 \, dx_3
\le \bigg(\fint_{I_{2, 3}} \bigg(\fint_{I_1} w \, dx_1\bigg)^{r_2} \, dx_2 \, dx_3 \bigg)^{\frac{r}{r_2}}
\\
&\lesssim \bigg(\fint_{I_{2, 3}} \fint_{I_1} w \, dx_1 \, dx_2 \, dx_3\bigg)^r
=\bigg(\fint_I w \, dx\bigg)^r, 
\end{align*}
which shows the reverse H\"{o}lder inequality for $A_{p, \Z}$ weights. More details about the proof of \eqref{wpp} are left to the reader. 

Observe that \eqref{T-Lp} applied to $w_0 \in A_{p_0, \Z}$ yields that 
\begin{align}\label{TL-1}
\text{$T$ is bounded on $L^{p_0}(w_0)$}.
\end{align}  
Additionally, \eqref{reduction-1} is equivalent to that  
\begin{align}\label{TL-2}
\text{$T$ is compact on $L^{p_1}(w_1)$.}
\end{align}
As a consequence of Theorem \ref{thm:inter}, \eqref{wpp}, and \eqref{TL-1}--\eqref{TL-2}, we obtain that $T$ is compact on $L^p(w)$. This justifies Theorem \ref{thm:compact} part \eqref{list-02}. 


It remains to show \eqref{reduction-1}. For any $N \in \N_+$ and $\lambda=2^k$ with $k \in \mathbb{Z}$, let 
\begin{align}\label{def:DZN}
\D_{\Z}(N) 
:= \big\{I \in \D_{\Z}:  \, 
&2^{-N} \le \ell(I_1) \le 2^N, \, \rd(I_1, \II_1) \le N, 
\nonumber \\ 
&2^{-N} \le \ell(I_2) \le 2^N, \, \rd(I_{2, 3}, \II_{2, 3}) \le N \big\},  
\end{align}
where $\II_1 := \II$ and $\II_{2, 3} := \II \times \II$. Given $N \in \N_+$, we define the projection operator as 
\begin{align*}
P_N f := \sum_{J \in \D_{\Z}(N)} \langle f, h_{J_1} \otimes h_{J_{2, 3}} \rangle \, h_{J_1} \otimes h_{J_{2, 3}}, 
\end{align*}
and the orthogonal projection operator as 
\begin{align*}
P_N^{\perp} f 
:= \sum_{J \not\in \D_{\Z}(N)} \langle f, h_{J_1} \otimes h_{J_{2, 3}} \rangle \, h_{J_1} \otimes h_{J_{2, 3}}, 
\end{align*}
where $J \not\in \D_{\Z}(N)$ means that $J \in \D_{\Z} \setminus \D_{\Z}(N)$. Note that $P_N$ is a finite-dimensional operator, which along with \cite[Proposition 15.1]{FHHMZ} implies that $P_N$ is a compact operator. In view of this fact and that $P_N^{\perp} \circ T=T - P_N \circ T$, to obtain \eqref{reduction-1}, it suffices to demonstrate 
\begin{align}\label{reduction-2}
\lim_{N \to \infty} \|P_N^{\perp} \circ T\|_{L^2(\R^3) \to L^2(\R^3)} =0. 
\end{align}

Let $f, g \in L^2(\R^3)$. By definition and \eqref{eq:expand}, we rewrite 
\begin{align}\label{eq:TDD}
\langle P_{2N}^{\perp} (Tf), g \rangle 
= \langle  Tf, P_{2N}^{\perp} g \rangle 
=\sum_{\substack{I  \in \D_{\Z} \\ J \not\in \D_{\Z}(2N)}} 
\G_{I, J} \, f_I \, g_J, 
\end{align}
where 
\begin{align*}
\G_{I, J} := \langle  T(h_{I_1} \otimes h_{I_{2, 3}}), \, h_{J_1} \otimes h_{J_{2, 3}} \rangle, \quad  
f_I := \langle f, h_{I_1} \otimes h_{I_{2, 3}} \rangle, \quad 
g_J := \langle g, h_{J_1} \otimes h_{J_{2, 3}} \rangle. 
\end{align*}
Given $I, J \in \D_{\Z}$, the size relationship between them must satisfy one of the following: 
\begin{equation}\label{sidelength}
\begin{aligned}
&\text{(1) \quad $\ell(I_1) \le \ell(J_1)$, $\ell(I_2) \le \ell(J_2)$ (hence, $\ell(I_3) \le \ell(J_3)$);} 
\\ 
&\text{(2) \quad $\ell(I_1) \le \ell(J_1)$, $\ell(I_2) > \ell(J_2)$, and $\ell(I_3) \le \ell(J_3)$;}
\\
&\text{(3) \quad $\ell(I_1) \le \ell(J_1)$, $\ell(I_2) > \ell(J_2)$, and $\ell(I_3) > \ell(J_3)$;}
\\
&\text{(4) \quad $\ell(I_1) > \ell(J_1)$, $\ell(I_2) \le \ell(J_2)$, and $\ell(I_3) \le \ell(J_3)$;} 
\\
&\text{(5) \quad $\ell(I_1) > \ell(J_1)$, $\ell(I_2) \le \ell(J_2)$, and $\ell(I_3) > \ell(J_3)$;} 
\\
&\text{(6) \quad $\ell(I_1) > \ell(J_1)$, $\ell(I_2) > \ell(J_2)$ (hence, $\ell(I_3) > \ell(J_3)$).}
\end{aligned}
\end{equation}


\subsection{The parameter grouping $\{1\}$, $\{2, 3\}$}\label{sec:123}
First, let us treat the cases (1), (3), (4), and (6), which have structure in the parameters $\{1\}$ and $\{2, 3\}$. In what follows, by symmetry and similarity, we focus on the case $\ell(I_1) \le \ell(J_1)$, $\ell(I_2) \ge \ell(J_2)$, and $\ell(I_3) \ge \ell(J_3)$ in order to reveal the general ideas and techniques.  

Let $I, J \in \D_{\Z}$ with $\ell(I_1) \le \ell(J_1)$, $\ell(I_2) \ge \ell(J_2)$, and $\ell(I_3) \ge \ell(J_3)$. According to relative distance between $I_1$ and $J_1$, there are three cases: 
\begin{align*}
(1) \, \rd(I_1, J_1) \ge 1;  \quad 
(2) \, \rd(I_1, J_1) < 1 \text{ and } I_1 \cap J_1 = \emptyset;  \quad 
(3) \, I_1 \subset J_1. 
\end{align*}
Analogously, the relationship between $I_{2, 3}$ and $J_{2, 3}$ satisfies either of the following 
\begin{align*}
(1') \, \rd(I_{2, 3}, J_{2, 3}) \ge 1;  \quad 
(2') \, \rd(I_{2, 3}, J_{2, 3}) < 1 \text{ and } I_{2, 3} \cap J_{2, 3} = \emptyset;  \quad 
(3') \, I_{2, 3} \supset J_{2, 3}. 
\end{align*}
Throughout this paper, we use $\sum'$ to indicate the summation with the restriction $\ell(I_1) \le \ell(J_1)$, $\ell(I_2) \ge \ell(J_2)$, and $\ell(I_3) \ge \ell(J_3)$. Therefore, based on the arguments above, to justify \eqref{reduction-2}, we are reduced to proving that given $\varepsilon>0$, there exists $N_0=N_0(\varepsilon) >1$ such that 
\begin{align}\label{reduction-3}
\max_{1 \le j \le 9} |\I^N_j|
\lesssim \varepsilon \|f\|_{L^2(\R^3)} \|g\|_{L^2(\R^3)}  
\qquad\text{ for all } N>N_0, 
\end{align}
where the implicit constant is independent of $\varepsilon$, $N$, $f$, and $g$, and $\I^N_j$ is defined by  
\begin{align*}
&\I^N_1 := \sideset{}{'}\sum_{\substack{I  \in \D_{\Z} \\ J \not\in \D_{\Z}(2N) \\ 
\rd(I_1, J_1) \ge 1 \\ \rd(I_{2, 3}, J_{2, 3}) \ge 1}} 
\G_{I, J} \, f_I \, g_J, 
\qquad 
\I^N_2 := \sideset{}{'}\sum_{\substack{I \in \D_{\Z} \\ J \not\in \D_{\Z}(2N)\\ 
\rd(I_1, J_1) \ge 1 \\ \rd(I_{2, 3}, J_{2, 3}) < 1 \\ I_{2, 3} \cap J_{2, 3} = \emptyset}} 
\G_{I, J} \, f_I \, g_J, 
\\
&\I^N_3 := \sideset{}{'}\sum_{\substack{I  \in \D_{\Z} \\ J \not\in \D_{\Z}(2N) \\ 
\rd(I_1, J_1) \ge 1 \\ I_{2, 3} \supset J_{2, 3}}} 
\G_{I, J} \, f_I \, g_J, 
\qquad\quad  
\I^N_4 := \sideset{}{'}\sum_{\substack{I  \in \D_{\Z} \\ J \not\in \D_{\Z}(2N)\\ 
\rd(I_1, J_1) < 1 \\ I_1 \cap J_1 = \emptyset \\ \rd(I_{2, 3}, J_{2, 3}) \ge 1}} 
\G_{I, J} \, f_I \, g_J, 
\\
&\I^N_5 := \sideset{}{'}\sum_{\substack{I  \in \D_{\Z} \\ J \not\in \D_{\Z}(2N) \\ 
\rd(I_1, J_1) < 1 \\ I_1 \cap J_1 = \emptyset \\ \rd(I_{2, 3}, J_{2, 3}) < 1 \\ I_{2, 3} \cap J_{2, 3} = \emptyset}} 
\G_{I, J} \, f_I \, g_J, 
\qquad
\I^N_6 := \sideset{}{'}\sum_{\substack{I  \in \D_{\Z} \\ J \not\in \D_{\Z}(2N) \\ 
\rd(I_1, J_1) < 1 \\ I_1 \cap J_1 = \emptyset \\ I_{2, 3} \supset J_{2, 3}}} 
\G_{I, J} \, f_I \, g_J, 
\\
&\I^N_7 := \sideset{}{'}\sum_{\substack{I  \in \D_{\Z} \\ J \not\in \D_{\Z}(2N) \\ 
I_1 \subset J_1 \\ \rd(I_{2, 3}, J_{2, 3}) \ge 1}}
\G_{I, J} \, f_I \, g_J, 
\qquad 
\I^N_8 := \sideset{}{'}\sum_{\substack{I  \in \D_{\Z} \\ J \not\in \D_{\Z}(2N) \\ 
I_1 \subset J_1 \\ \rd(I_{2, 3}, J_{2, 3}) < 1 \\ I_{2, 3} \cap J_{2, 3} = \emptyset}} 
\G_{I, J} \, f_I \, g_J, 
\\
&\quad\text{ and } \quad \I^N_9 := \sideset{}{'}\sum_{\substack{I  \in \D_{\Z} \\ J \not\in \D_{\Z}(2N) \\ 
I_1 \subset J_1 \\ I_{2, 3} \supset J_{2, 3}}} 
\G_{I, J} \, f_I \, g_J. 
\end{align*}
In addition, the condition $\rd(I_{2, 3}, J_{2, 3})=\max\{\rd(I_2, J_2), \rd(I_3, J_3)\} \ge 1$ implies one of the cases holds: (i) $\rd(I_2, J_2) \ge 1$ and $\rd(I_3, J_3) \ge 1$; (ii) $\rd(I_2, J_2) \ge 1$ and $\rd(I_3, J_3) < 1$; (iii) $\rd(I_2, J_2) < 1$ and $\rd(I_3, J_3) \ge 1$. Consequently, for each $j=1, 4, 7$, denoting by $\I^N_{j, i}$ the resulting terms, $i=1, 2, 3$, we have  
\begin{align}\label{NNN}
\I^N_j = \sum_{i=1}^3 \I^N_{j, i}, \qquad j=1, 4, 7. 
\end{align}
To control $\I^N_j$ for $j=1, 4, 7$, we will analyze each term $\I^N_{j, i}$ separately. 




\subsection{The parameter grouping $\{2\}$, $\{1, 3\}$}\label{sec:213}
Next, we deal with the cases (2) and (5), which have structure in the parameters $\{2\}$ and $\{1, 3\}$. 

Observe that $J_1 \times J_2 \times J_3 \in \D_{\Z}$ is equivalent to $J_2 \times J_1 \times J_3 \in \D_{\Z}$, and $J_1 \times J_2 \times J_3 \in \D_{\Z}(2N)$ is equivalent to $J_2 \times J_1 \times J_3 \in \D_{\Z}(2N)$. Hence,  
\begin{align*}
\sum_{\substack{I  \in \D_{\Z} \\ J \not\in \D_{\Z}(2N)}} \G_{I, J} \, f_I \, g_J
=\sum_{\substack{I_2 \times I_1 \times I_3  \in \D_{\Z} \\ J_2 \times J_1 \times J_3 \not\in \D_{\Z}(2N)}} \G'_{I, J} \, f'_I \, g'_J, 
\end{align*}
where 
\begin{align*}
\G'_{I, J} := \langle  T(h_{I_2} \otimes h_{I_{1, 3}}), \, h_{J_2} \otimes h_{J_{1, 3}} \rangle, \quad  
f'_I := \langle f, h_{I_2} \otimes h_{I_{1, 3}} \rangle, \quad 
g'_J := \langle g, h_{J_2} \otimes h_{J_{1, 3}} \rangle. 
\end{align*}
This, together with the assumptions with respect to the parameters $\{2\}$ and $\{1, 3\}$, means that the cases (2) and (5) can be addressed as in Section \ref{sec:123}. Therefore, it suffices to prove \eqref{reduction-3}. 




%%%%%%%%%%%%%%%%%%%%%%%%%%%% SECTION SECTION SECTION %%%%%%%%%%%%%%%%%%%%
%%%%%%%%%%%%%%%%%%%%%%%%%%%% SECTION SECTION SECTION %%%%%%%%%%%%%%%%%%%%
\section{The main estimates}\label{sec:proof} 

In what follows, we work with the parameter grouping $\{1\}$, $\{2, 3\}$ and focus on the case $\ell(I_1) \le \ell(J_1)$, $\ell(I_2) \ge \ell(J_2)$, and $\ell(I_3) \ge \ell(J_3)$. Before estimating each term $\I^N_j$, $j=1, \ldots, 9$, we may assume some extra properties of functions appearing the hypotheses \eqref{list-1}--\eqref{list-4}. These will be quite useful in our proof. 


\begin{list}{\rm (\theenumi)}{\usecounter{enumi}\leftmargin=1.2cm \labelwidth=1cm \itemsep=0.2cm \topsep=.2cm \renewcommand{\theenumi}{P\arabic{enumi}}} 

\item\label{list:P1} Assume that for each $i=1, 2$ and $j=1, 2, 3$, the function $F_{i, j}$ in Definitions \ref{def:full} is the same as the one in Definitions \ref{def:partial-1}--\ref{def:partial-2}. Indeed, given $(F_{i, 1}^1, F_{i, 2}^1, F_{i, 3}^1) \in \F$ in Definition \ref{def:full}, $(F_{i, 1}^2, F_{i, 2}^2, F_{i, 3}^2) \in \F$ in Definition \ref{def:partial-1}, and $(F_{i, 1}^3, F_{i, 2}^3, F_{i, 3}^3)  \in \F$ in Definition \ref{def:partial-2}, if we define 
\begin{align*}
F_{i, j} := \max\{F_{i, j}^1, F_{i, j}^2, F_{i, j}^3\}, \qquad j=1, 2, 3, 
\end{align*}
then the compact full and partial kernel representations hold for $(F_{i, 1}, F_{i, 2}, F_{i, 3}) \in \F$. 


\item\label{list:P2} For each $i=1,2$ and $(F_{i, 1}, F_{i, 2}, F_{i, 3}) \in \F$ in Definitions \ref{def:full}--\ref{def:partial-2}, we may assume that $F_{i, 1}$ is monotone increasing while $F_{i, 2}$ and $F_{i, 3}$ are monotone decreasing. Indeed, it suffices to define 
\begin{align*}
F_{i, 1}^*(t) := \sup_{0 \le s \le t} F_{i, 1}(s), \quad 
F_{i, 2}^*(t) := \sup_{s \ge t} F_{i, 2}(s), \quad\text{and}\quad 
F_{i, 3}^*(t) := \sup_{s \ge t} F_{i, 3}(s).
\end{align*}
Then it is easy to verify that the compact full and partial kernel representations hold for $(F_{i, 1}^*, F_{i, 2}^*, F_{i, 3}^*) \in \F$, where $F^*_{i, 1}$ is monotone increasing, $F^*_{i, 2}$ and $F^*_{i, 3}$ are monotone decreasing. 

\item\label{list:P3} Since any dilation of functions in $\F$ and $\F_0$ still belong to the original space, we will often omit all universal constants appearing in the argument involving these functions.  

\item\label{list:P4} In light of Lemma \ref{lem:improve}, we will mostly use alternative estimates for kernels in Definitions \ref{def:full}--\ref{def:partial-2}. For example, in Definition \ref{def:full}, the size condition can be replaced by 
\begin{align*}
|K(x, y)| \leq D_{\theta}(x-y) \prod_{i=1}^3 \frac{F_i(x_i, y_i)}{|x_i - y_i|},  
\end{align*}
where 
\begin{align*}
F_i(x_i, y_i) := F_{i, 1}(|x_i - y_i|) F_{i, 2}(|x_i - y_i|) F_{i, 3} \bigg(1+ \frac{|x_i + y_i|}{1+|x_i - y_i|} \bigg),  
\end{align*}
while the first H\"{o}lder condition can be replaced by 
\begin{align*}
&|K(x, y) - K((x_1, x'_2, x'_3), y) - K((x'_1, x_2, x_3), y) + K(x', y)| 
\\
&\quad\leq \bigg(\frac{|x_1 - x'_1|}{|x_1 - y_1|}\bigg)^{\delta_1} 
\bigg(\frac{|x_2 - x'_2|}{|x_2 - y_2|} + \frac{|x_3 - x'_3|}{|x_3 - y_3|}\bigg)^{\delta_{2, 3}}
D_{\theta}(x-y) \prod_{i=1}^3 \frac{F_i(x_i, y_i)}{|x_i - y_i|}, 
\end{align*}
whenever $|x_i - x'_i| \leq |x_i - y_i|/2$ for $i=1, 2, 3$, where 
\begin{align*}
F_i(x_i, y_i) := F_{i, 1}(|x_i - x'_i|) F_{i, 2}(|x_i - y_i|) F_{i, 3} \bigg(1+ \frac{|x_i + y_i|}{1+|x_i - y_i|} \bigg). 
\end{align*} 
Other conditions can be formulated in a similar way. 
\end{list} 



\subsection{$\rd(I_1, J_1) \ge 1$ and $\rd(I_{2, 3}, J_{2, 3}) \ge 1$}
In view of \eqref{NNN}, we are going to estimate $\I^N_{1, i}$, $i=1, 2, 3$,  sequentially. 


\subsubsection{\bf Case 1: $\rd(I_i, J_i) \ge 1$, $i=1, 2, 3$} 
We begin with the follows estimate. 

%%%%%%%%%%%%%%%%%%%%%%%%%%%%% LEMMA LEMMA LEMMA %%%%%%%%%%%%%%%%%%%%%%%%
\begin{lemma}\label{lem:HH-1}
Let $I, J \in \D_{\Z}$ with $\ell(I_1) \le \ell(J_1)$, $\ell(I_2) \ge \ell(J_2)$, and $\ell(I_3) \ge \ell(J_3)$ so that $\rd(I_i, J_i) \ge 1$, $i=1, 2, 3$. Then
\begin{align*}
|\mathscr{G}_{I, J}|
\lesssim \bigg[\frac{\rs(I_1, J_1)}{\rd(I_1, J_1)}\bigg]^{\delta_1} 
\bigg[\frac{\rs(I_2, J_2)}{\rd(I_2, J_2)} + \frac{\rs(I_3, J_3)}{\rd(I_3, J_3)}\bigg]^{\delta_{2, 3}}
\frac{1}{D_{\theta}(I, J)} \prod_{i=1}^3 F_i(I_i, J_i) \frac{\rs(I_i, J_i)^{\frac12}}{\rd(I_i, J_i)}, 
\end{align*}
where 
\begin{align*}
D_{\theta}(I, J) := \bigg[\frac{\rd(I_1, J_1) \rd(I_2, J_2)}{\rs(I_1, J_1) \rd(I_3, J_3)} 
+ \frac{\rs(I_1, J_1) \rd(I_3, J_3)}{\rd(I_1, J_1) \rd(I_2, J_2)} \bigg]^{\theta}. 
\end{align*}
\end{lemma}
%%%%%%%%%%%%%%%%%%%%%%%%%%%%% LEMMA LEMMA LEMMA %%%%%%%%%%%%%%%%%%%%%%%%


%%%%%%%%%%%%%%%%%%%%%%%%%%%%%% PROOF PROOF PROOF %%%%%%%%%%%%%%%%%%%%%%%
\begin{proof}
Let $x_i \in J_i$ and $y_i \in I_i$, $i=1, 2, 3$. Then, for each $i=1, 2, 3$, 
\begin{align*}
|x_i - y_i| \le \ell(I_i \vee J_i) 
\le \d(I_i, J_i) + \ell(I_i) + \ell(J_i) 
\le 3\d(I_i, J_i) \le 3|x_i - y_i|, 
\end{align*}
which in turn implies 
\begin{align}
\label{DT-1} \frac{|y_1 - c_{I_1}|}{|x_1 - y_1|} 
&\lesssim \frac{\ell(I_1)}{\d(I_1, J_1)} 
=\frac{\rs(I_1, J_1)}{\rd(I_1, J_1)}, 
\\
\label{DT-2} \frac{|x_i - c_{J_i}|}{|x_i - y_i|} 
&\lesssim \frac{\ell(J_i)}{\d(I_i, J_i)} 
=\frac{\rs(I_i, J_i)}{\rd(I_i, J_i)}, \qquad i=2, 3, 
\end{align}
and 
\begin{align}\label{DT-3}
D_{\theta}(x-y) 
&= \Bigg( \frac{\frac{|x_1 - y_1|}{\ell(I_1)} \frac{|x_2 - y_2|}{\ell(I_2)}}{\frac{|x_3 - y_3|}{\ell(I_3)}} 
+ \frac{\frac{|x_3 - y_3|}{\ell(I_3)}}{\frac{|x_1 - y_1|}{\ell(I_1)} \frac{|x_2 - y_2|}{\ell(I_2)}} \Bigg)^{-\theta}
\nonumber \\
&\simeq \Bigg( \frac{\frac{\d(I_1, J_1)}{\ell(J_1)} \frac{\d(I_2, J_2)}{\ell(I_2)}}{\frac{\ell(I_1)}{\ell(J_1)} \frac{\d(I_3, J_3)}{\ell(I_3)}} 
+ \frac{\frac{\ell(I_1)}{\ell(J_1)} \frac{\d(I_3, J_3)}{\ell(I_3)}}{\frac{\d(I_1, J_1)}{\ell(J_1)} \frac{\d(I_2, J_2)}{\ell(I_2)}} \Bigg)^{-\theta}
\nonumber \\
&= \Bigg( \frac{\rd(I_1, J_1) \rd(I_2, J_2)}{\rs(I_1, J_1) \rd(I_3, J_3)} 
+ \frac{\rs(I_1, J_1) \rd(I_3, J_3)}{\rd(I_1, J_1) \rd(I_2, J_2)} \Bigg)^{-\theta}
= D_{\theta}(I, J)^{-1}. 
\end{align}
In light of \eqref{DT-1}--\eqref{DT-3}, we use the compact full kernel representation, the cancellation of $h_{I_1}$ and $h_{J_{2, 3}}$, and the H\"{o}lder condition of $K$ to deduce that 
\begin{align*}
|\mathscr{G}_{I, J}| 
&\lesssim \int_{I_1 \times I_{2, 3}} \int_{J_1 \times J_{2, 3}} 
\bigg(\frac{|y_1 - c_{I_1}|}{|x_1 - y_1|}\bigg)^{\delta_1} 
\bigg(\frac{|x_2 - c_{J_2}|}{|x_2 - y_2|} + \frac{|x_3 - c_{J_3}|}{|x_3 - y_3|}\bigg)^{\delta_{2, 3}}
\\
&\quad\times D_{\theta}(x-y) \prod_{i=1}^3 \frac{F_i(x_i, y_i)}{|x_i - y_i|} 
|h_{I_1} \otimes h_{I_{2, 3}}(y)| |h_{J_1} \otimes h_{J_{2, 3}}(x)| \, dx \, dy
\\
&\lesssim \bigg[\frac{\rs(I_1, J_1)}{\rd(I_1, J_1)}\bigg]^{\delta_1} 
\bigg[\frac{\rs(I_2, J_2)}{\rd(I_2, J_2)} + \frac{\rs(I_3, J_3)}{\rd(I_3, J_3)}\bigg]^{\delta_{2, 3}}
\frac{1}{D_{\theta}(I, J)} \prod_{i=1}^3 \mathscr{P}_i(I_i, J_i) |I_i|^{-\frac12} |J_i|^{-\frac12}
\\ 
&\lesssim \bigg[\frac{\rs(I_1, J_1)}{\rd(I_1, J_1)}\bigg]^{\delta_1} 
\bigg[\frac{\rs(I_2, J_2)}{\rd(I_2, J_2)} + \frac{\rs(I_3, J_3)}{\rd(I_3, J_3)}\bigg]^{\delta_{2, 3}}
\frac{1}{D_{\theta}(I, J)} \prod_{i=1}^3 F_i(I_i, J_i) \frac{\rs(I_i, J_i)^{\frac12}}{\rd(I_i, J_i)}, 
\end{align*}
where \eqref{def:P1} was used in the last inequality.
\end{proof}
%%%%%%%%%%%%%%%%%%%%%%%%%%%%%% END END END PROOF %%%%%%%%%%%%%%%%%%%%%%%




Let us turn to the estimate for $\I^N_{1, 1}$. Given $i=1, 2, 3$, $k_i \in \ZZ$, $j_i \ge 1$, and $I_i, J_i \in \D_i$, let
\begin{align*}
J_i(k_i, j_i) &:= \big\{I_i \in \D_i: \, \ell(I_i) = 2^{-k_i} \ell(J_i), \, j_i \le \rd(I_i, J_i) < j_i +1\big\}, 
\\
I_i(k_i, j_i) &:= \big\{J_i \in \D_i: \, \ell(J_i) = 2^{k_i} \ell(I_i), \, j_i \le \rd(I_i, J_i) < j_i +1\big\}. 
\end{align*}
Note that for each $i=1, 2, 3$, $I_i \in J_i(k_i, j_i)$ if and only if $J_i \in I_i(k_i, j_i)$,
\begin{align}\label{car-IJ}
&\# J_i(k_i, j_i) \lesssim 2^{\max\{k_i, \, 0\}},
\quad\text{ and }\quad
\# I_i(k_i, j_i) \lesssim 2^{\max\{-k_i, \, 0\}}.
\end{align}
By definition, it is not hard to see that 
\begin{align}\label{JDD}
J \not\in \D_{\Z}(2N) \, \Longrightarrow \, 
J_1 \not\in \D(2N), \text{ or } \, J_2 \not\in \D(2N), \text{ or }\, J_3 \not\in \D(2N).
\end{align}
Hence, by Lemma \ref{lem:HH-1}, \eqref{JDD}, and Lemma \ref{lem:FF} part \eqref{list-FF1}, there exists $N_0>1$ so that for all $N \ge N_0$,
\begin{align*}
|\I^N_{1, 1}|
\le \sum_{k_2 \ge k_1 \ge 0 \atop j_1, j_2, j_3 \ge 1} \sum_{J \not\in \D_{\Z}(2N)} 
\sideset{}{'}\sum_{\substack{I \in \D_{\Z} \\ I_1 \in J_1(k_1, j_1) \\ I_2 \in J_2(-k_2, j_2) \\ I_3 \in J_3(k_1-k_2, j_3)}} 
|\G_{I, J}| \, |f_I| |g_J|
\lesssim \sum_{i=0}^3 \I^{N, i}_{1, 1},
\end{align*}
where
\begin{align*}
\I^{N, 0}_{1, 1}
&:= \varepsilon \sum_{\substack{k_2 \ge k_1 \ge 0 \\ j_1, j_2, j_3 \ge 1}} 
\Gamma(k, j) \, 2^{-\frac{k_1}{2}} \, 2^{-\frac{k_1}{2}} \, 2^{-\frac{k_2-k_1}{2}}  
\sideset{}{'}\sum_{\substack{I, J \in \D_{\Z} \\ I_1 \in J_1(k_1, j_1) \\ I_2 \in J_2(-k_2, j_2) \\ I_3 \in J_3(k_1-k_2, j_3)}} \, |f_I| |g_J|,
\\
\I^{N, i}_{1, 1}
&:= \sum_{\substack{k_2 \ge k_1 \ge 0 \\ j_1, j_2, j_3 \ge 1 \\ j_i \ge N^{1/8}}}
\Gamma(k, j) \, 2^{-\frac{k_1}{2}} \, 2^{-\frac{k_1}{2}} \, 2^{-\frac{k_2-k_1}{2}}  
\sideset{}{'}\sum_{\substack{I, J \in \D_{\Z} \\ I_1 \in J_1(k_1, j_1) \\ I_2 \in J_2(-k_2, j_2) \\ I_3 \in J_3(k_1-k_2, j_3)}} \, |f_I| |g_J|,
\end{align*}
for $i=1, 2, 3$, with the term 
\begin{align*}
\Gamma(k, j) := 2^{-k_1 \delta_1} j_1^{-1-\delta_1} 
\frac{(2^{-k_2} j_2^{-1} + 2^{k_1-k_2} j_3^{-1})^{\delta_{2, 3}}}{(2^{k_1} j_1 j_2 j_3^{-1} +  2^{-k_1} j_1^{-1} j_2^{-1} j_3)^{\theta}}. 
\end{align*} 
It follows from the Cauchy--Schwarz inequality, \eqref{car-IJ}, and \eqref{JJA-2} that
\begin{align*}
\I^{N, 0}_{1, 1}
&\le \varepsilon \sum_{\substack{k_2 \ge k_1 \ge 0 \\ j_1, j_2, j_3 \ge 1}} \Gamma(k, j) \, 
2^{-\frac{k_1}{2}} 2^{-\frac{k_1}{2}}  2^{-\frac{k_2-k_3}{2}}   
\\
&\quad\times 
\bigg(\sum_{I \in \D_{\Z}} 
\sum_{\substack{J \in \D_{\Z} \\ J_1 \in I_1(k_1, j_1) \\ J_2 \in I_2(-k_2, j_2) \\ J_3 \in I_3(k_1-k_2, j_3)}} \, |f_I|^2 \bigg)^{\frac12}
\bigg(\sum_{J \in \D_{\Z}} 
\sum_{\substack{I \in \D_{\Z} \\ I_1 \in J_1(k_1, j_1) \\ I_2 \in J_2(-k_2, j_2) \\ I_3 \in I_3(k_1-k_2, j_3)}} \, |g_J|^2 \bigg)^{\frac12}
\\
&\lesssim \varepsilon \sum_{\substack{k_2 \ge k_1 \ge 0 \\ j_1, j_2, j_3 \ge 1}} \Gamma(k, j) 
\bigg(\sum_{I \in \D_{\Z}} |f_I|^2 \bigg)^{\frac12}
\bigg(\sum_{J \in \D_{\Z}} |g_J|^2 \bigg)^{\frac12}
\\
&\lesssim \varepsilon \|f\|_{L^2(\R^3)} \|g\|_{L^2(\R^3)}. 
\end{align*}
Analogously, for each $i=1, 2, 3$, 
\begin{align*}
\I^{N, i}_{1, 1}
&\lesssim \sum_{\substack{k_2 \ge k_1 \ge 0 \\ j_1, j_2, j_3 \ge 1 \\ j_i \ge N^{1/8}}} \Gamma(k, j) 
\bigg(\sum_{I \in \D_{\Z}} |f_I|^2 \bigg)^{\frac12}
\bigg(\sum_{J \in \D_{\Z}} |g_J|^2 \bigg)^{\frac12}
\\
&\lesssim N^{-\delta_0/8} \|f\|_{L^2(\R^3)} \|g\|_{L^2(\R^3)} 
\le \varepsilon \|f\|_{L^2(\R^3)} \|g\|_{L^2(\R^3)}, 
\end{align*}
provide $N>N_0$ large enough. Therefore, one has 
\begin{align}\label{SN11}
|\I^N_{1, 1}|
\lesssim \sum_{i=0}^3 \I^{N, i}_{1, 1} 
\lesssim \varepsilon \|f\|_{L^2(\R^3)} \|g\|_{L^2(\R^3)}. 
\end{align}




\subsubsection{\bf Case 2: $\rd(I_1, J_1) \ge 1$, $\rd(I_2, J_2) \ge 1$, and $\rd(I_3, J_3) < 1$} 



\begin{lemma}\label{lem:HH-2}
Let $I, J \in \D_{\Z}$ with $\ell(I_1) \le \ell(J_1)$, $\ell(I_2) \ge \ell(J_2)$, and $\ell(I_3) \ge \ell(J_3)$ so that $\rd(I_1, J_1) \ge 1$, $\rd(I_2, J_2) \ge 1$, and $\rd(I_3, J_3) < 1$. Then
\begin{align*}
|\mathscr{G}_{I, J}| 
\lesssim F_1(I_1, J_1) \frac{\rs(I_1, J_1)^{\frac12 + \delta_1}}{\rd(I_1, J_1)^{1+\delta_1}} 
F_2(I_2, J_2) \frac{\rs(I_2, J_2)^{\frac12}}{\rd(I_2, J_2)} 
\widetilde{F}_3(I_3, J_3) \frac{\rs(I_3, J_3)^{\frac12 - \frac1r}}{D_{\theta}^{1, 2}(I, J)},
\end{align*}
where 
\begin{align*}
D_{\theta}^{1, 2}(I, J)
:= \bigg(\frac{\rd(I_1, J_1) \rd(I_2, J_2)}{\rs(I_2, J_2)}  + \frac{\rs(I_2, J_2)}{\rd(I_1, J_1) \rd(I_2, J_2)} \bigg)^{\theta}. 
\end{align*}
\end{lemma}

\begin{proof}
Considering that $\rd(I_1, J_1) \ge 1$ and $\rd(I_2, J_2) \ge 1$, we have 
\begin{align}\label{HH21}
\frac{|y_1-c_{I_1}|}{|x_1-y_1|} \lesssim \frac{\rs(I_1, J_1)}{\rd(I_1, J_1)}
\quad \text{ and }\quad 
|x_i-y_i| \simeq \d(I_i, J_i), \quad i=1, 2, 
\end{align}
which together with $\ell(J_1) \ell(I_2) = \frac{\ell(J_3)}{\rs(I_2, J_2)}$ yields  
\begin{align}\label{HH22}
D_{\theta}(x-y) 
&= \Bigg( \frac{\frac{|x_1 - y_1|}{\ell(J_1)} \frac{|x_2 - y_2|}{\ell(I_2)}}{\frac{\rs(I_2, J_2)}{\ell(J_3)} |x_3 - y_3|} 
+ \frac{\frac{\rs(I_2, J_2)}{\ell(J_3)} |x_3 - y_3|}{\frac{|x_1 - y_1|}{\ell(J_1)} \frac{|x_2 - y_2|}{\ell(I_2)}} \Bigg)^{-\theta}
\nonumber \\
&\simeq \Bigg( \frac{\frac{\d(I_1, J_1)}{\ell(J_1)} \frac{\d(I_2, J_2)}{\ell(I_2)}}{\frac{\rs(I_2, J_2)}{\ell(J_3)} |x_3 - y_3|} 
+ \frac{\frac{\rs(I_2, J_2)}{\ell(J_3)} |x_3 - y_3|}{\frac{\d(I_1, J_1)}{\ell(J_1)} \frac{\d(I_2, J_2)}{\ell(I_2)}} \Bigg)^{-\theta}
\nonumber \\
&= \bigg( \frac{\rd(I_1, J_1) \rd(I_2, J_2) \ell(J_3)}{\rs(I_2, J_2) |x_3 - y_3|} 
+ \frac{\rs(I_2, J_2) |x_3 - y_3|}{\rd(I_1, J_1) \rd(I_2, J_2) \ell(J_3)}  \bigg)^{-\theta}. 
\end{align}
If we set $t_3 := \frac{\rs(I_2, J_2)}{\rd(I_1, J_1) \rd(I_2, J_2) \ell(J_3)}$, then by the compact full kernel representation, the cancellation of $h_{I_1}$, the mixed size-H\"{o}lder condition of $K$, and \eqref{HH21}--\eqref{HH22}, we conclude 
\begin{align}\label{HH23}
|\mathscr{G}_{I, J}| 
&\lesssim \int_{I_1 \times I_{2, 3}} \int_{J_1 \times J_{2, 3}} 
\bigg(\frac{|y_1 - c_{I_1}|}{|x_1 - y_1|}\bigg)^{\delta_1} D_{\theta}(x-y) 
\nonumber \\
&\quad\times \prod_{i=1}^3 \frac{F_i(x_i, y_i)}{|x_i - y_i|} 
|h_{I_1} \otimes h_{I_{2, 3}}(y)| |h_{J_1} \otimes h_{J_{2, 3}}(x)| \, dx \, dy
\nonumber \\
&\lesssim \frac{\rs(I_1, J_1)^{\delta_1}}{\rd(I_1, J_1)^{\delta_1}} 
\mathscr{P}_1(I_1, J_1) \mathscr{P}_2(I_2, J_2) \mathscr{Q}_3^{0, t_3}(I_3, J_3)
\prod_{i=1}^3 |I_i|^{-\frac12} |J_i|^{-\frac12}  
\nonumber \\
&\lesssim F_1(I_1, J_1) \frac{\rs(I_1, J_1)^{\frac12 + \delta_1}}{\rd(I_1, J_1)^{1+\delta_1}} 
F_2(I_2, J_2) \frac{\rs(I_2, J_2)^{\frac12}}{\rd(I_2, J_2)} 
\widetilde{F}_3(I_3, J_3) \frac{\rs(I_3, J_3)^{\frac12 - \frac1r}}{D_{\theta}^{1, 2}(I, J)}, 
\end{align}
where \eqref{def:P1} and \eqref{def:Q0} were used in the last step. 
\end{proof}


Observe that for any $k_0 \in \ZZ$ and $K \in \D$, 
\begin{align}
\label{rdd-1} 
& \#\big\{K' \in \D: 2^{-k_0} = \ell(K') \le \ell(K), \, \rd(K', K) < 1 \big\} \lesssim 2^{k_0} \ell(K), 
\\
\label{rdd-2} 
& \#\big\{K' \in \D: 2^{-k_0} = \ell(K') > \ell(K), \, \rd(K', K) < 1 \big\} \lesssim 1.  
\end{align}
Let $k_1, k_2 \ge 0$ and $I, J \in \D_{\Z}$ with $J_1 \in I_1(k_1, j_1)$, $J_2 \in I_2(-k_2, j_2)$, and $\rd(I_3, J_3) < 1$. Then 
\begin{align*}
\ell(I_3) \ge \ell(J_3) = \ell(J_1) \ell(J_2) 
=2^{k_1} \ell(I_1) 2^{-k_2} \ell(I_2)
=2^{k_1-k_2} \ell(I_3), 
\end{align*}
which together with \eqref{rdd-1} and \eqref{rdd-2} implies 
\begin{align}
\label{Car-331} & \#\big\{J_3 \in \D: \ell(J_3) \le \ell(I_3), \rd(I_3, J_3) < 1\big\} \lesssim 2^{k_2 - k_1},
\\
\label{Car-332} & \#\big\{I_3 \in \D: \ell(I_3) \ge \ell(I_3), \rd(I_3, J_3) < 1\big\} \lesssim 1. 
\end{align}
In light of Lemma \ref{lem:HH-2}, \eqref{JDD}, and Lemma \ref{lem:FF} parts \eqref{list-FF1}--\eqref{list-FF2}, there exists $N_0>1$ so that for all $N \ge N_0$,
\begin{align*}
|\I^N_{1, 2}| 
\le \sum_{k_2 \ge k_1 \ge 0 \atop j_1, j_2 \ge 1}  \sum_{J \not\in \D_{\Z}(2N)}
\sum_{\substack{I \in \D_{\Z} \\ I_1 \in J_1(k_1, j_1) \\ I_2 \in J_2(-k_2, j_2) \\ \rd(I_3, J_3) < 1}} |\G_{I, J}| \, |f_I| |g_J|
\lesssim \sum_{i=0}^2 \I^{N, i}_{1, 2},
\end{align*}
where
\begin{align*}
\I^{N, 0}_{1, 2}
&:= \varepsilon \sum_{k_2 \ge k_1 \ge 0 \atop j_1, j_2 \ge 1} 
2^{-k_1(\frac12 + \delta_1)} j_1^{-1-\delta_1} 2^{-\frac{k_2}{2}} j_2^{-1} 
\frac{2^{-(k_2-k_1)(\frac12 - \frac1r)}}{(2^{k_2} j_1 j_2)^{\theta}}  
\sum_{\substack{I, J \in \D_{\Z} \\ I_1 \in J_1(k_1, j_1) \\ I_2 \in J_2(-k_2, j_2) \\ \rd(I_3, J_3) < 1}} |f_I| |g_J|,
\\
\I^{N, i}_{1, 2}
&:= \sum_{\substack{k_2 \ge k_1 \ge 0 \\ j_1, j_2 \ge 1 \\ j_i \ge N^{1/8}}}
2^{-k_1(\frac12 + \delta_1)} j_1^{-1-\delta_1} 2^{-\frac{k_2}{2}} j_2^{-1} 
\frac{2^{-(k_2-k_1)(\frac12 - \frac1r)}}{(2^{k_2} j_1 j_2)^{\theta}}  
\sum_{\substack{I, J \in \D_{\Z} \\ I_1 \in J_1(k_1, j_1) \\ I_2 \in J_2(-k_2, j_2) \\ \rd(I_3, J_3) < 1}} |f_I| |g_J|,
\end{align*}
for $i=1, 2$. By the Cauchy--Schwarz inequality, \eqref{car-IJ}, and \eqref{Car-331}--\eqref{Car-332}, it yields 
\begin{align*}
\I^{N, 0}_{1, 2}
&\le \varepsilon \sum_{k_2 \ge k_1 \ge 0 \atop j_1, j_2 \ge 1} 
2^{-k_1 (\frac12 + \delta_1)} j_1^{-1-\theta} 2^{-k_2 (\frac12 + \theta)} j_2^{-1-\theta} 
2^{-(k_2-k_1)(\frac12 - \frac1r)}  
\\
&\quad\times 
\bigg(\sum_{I \in \D_{\Z}} \sum_{\substack{J \in \D_{\Z} \\ J_1 \in I_1(k_1, j_1) \\ J_2 \in I_2(-k_2, j_2) \\ \rd(I_3, J_3) < 1}} |f_I|^2 \bigg)^{\frac12}
\bigg(\sum_{J \in \D_{\Z}} \sum_{\substack{I \in \D_{\Z} \\ I_1 \in J_1(k_1, j_1) \\ I_2 \in J_2(-k_2, j_2) \\ \rd(I_3, J_3) < 1}} |g_J|^2 \bigg)^{\frac12}
\\
&\lesssim \varepsilon \sum_{k_1, k_2 \ge 0 \atop j_1, j_2 \ge 1} 
2^{-k_1 \delta_1} j_1^{-1-\theta} 
2^{-k_2(\theta-\frac1r)} j_2^{-1-\theta} 
\bigg(\sum_{I \in \D_{\Z}} |f_I|^2 \bigg)^{\frac12}
\bigg(\sum_{J \in \D_{\Z}} |g_J|^2 \bigg)^{\frac12}
\\
&\lesssim \varepsilon \|f\|_{L^2(\R^3)} \|g\|_{L^2(\R^3)}, 
\end{align*}
and for each $i=1, 2$, 
\begin{align*}
\I^{N, i}_{1, 2}
&\lesssim \sum_{\substack{k_1, k_2 \ge 0 \\ j_1, j_2 \ge 1 \\ j_i \ge N^{1/8}}} 
2^{-k_1 \delta_1} j_1^{-1-\theta} 
2^{-k_2(\theta-\frac1r)} j_2^{-1-\theta} 
\bigg(\sum_{I \in \D_{\Z}} |f_I|^2 \bigg)^{\frac12}
\bigg(\sum_{J \in \D_{\Z}} |g_J|^2 \bigg)^{\frac12}
\\
&\lesssim N^{-\theta/8} \|f\|_{L^2(\R^3)} \|g\|_{L^2(\R^3)}
\le \varepsilon \|f\|_{L^2(\R^3)} \|g\|_{L^2(\R^3)},  
\end{align*}
provided $N>1$ sufficiently large. Collecting the estimates above, we conclude  
\begin{align}\label{SN12}
|\I^N_{1, 2}|
\lesssim \sum_{i=0}^2 \I^{N, i}_{1, 2} 
\lesssim \varepsilon \|f\|_{L^2(\R^3)} \|g\|_{L^2(\R^3)}. 
\end{align}





\subsubsection{\bf Case 3: $\rd(I_1, J_1) \ge 1$, $\rd(I_2, J_2) < 1$, and $\rd(I_3, J_3) \ge 1$} 



\begin{lemma}\label{lem:HH-3}
Let $I, J \in \D_{\Z}$ with $\ell(I) = \ell(J)$ so that $\rd(I_1, J_1) \ge 1$, $\rd(I_2, J_2) < 1$, and $\rd(I_3, J_3) \ge 1$. Then
\begin{align*}
|\mathscr{G}_{I, J}| 
\lesssim F_1(I_1, J_1) \frac{\rs(I_1, J_1)^{\frac12 + \delta_1}}{\rd(I_1, J_1)^{1+\delta_1}} 
\widetilde{F}_2(I_2, J_2) \frac{\rs(I_2, J_2)^{\frac12 - \frac1r}}{D_{\theta}^{1, 3}(I, J)}
F_3(I_3, J_3) \frac{\rs(I_3, J_3)^{\frac12}}{\rd(I_3, J_3)},
\end{align*}
where 
\begin{align*}
D_{\theta}^{1, 3}(I, J)
:= \bigg( \frac{\rd(I_1, J_1) \rs(I_3, J_3)}{\rd(I_3, J_3)}  + \frac{\rd(I_3, J_3)}{\rd(I_1, J_1) \rs(I_3, J_3)} \bigg)^{\theta}. 
\end{align*}
\end{lemma}


\begin{proof}
The condition $\rd(I_1, J_1) \ge 1$ and $\rd(I_3, J_3) \ge 1$ implies 
\begin{align}\label{HH31}
\frac{|y_1-c_{I_1}|}{|x_1-y_1|} \lesssim \frac{\rs(I_1, J_1)}{\rd(I_1, J_1)}
\quad \text{ and }\quad 
|x_i-y_i| \simeq \d(I_i, J_i), \quad i=1, 3, 
\end{align}
which along with $\frac{\ell(J_1)}{\ell(I_3)} = \frac{\rs(I_3, J_3)}{\ell(J_2)}$ gives 
\begin{align}\label{HH32}
D_{\theta}(x-y) 
&= \Bigg( \frac{\frac{|x_1 - y_1|}{\ell(J_1)} \frac{\rs(I_3, J_3)}{\ell(J_2)} |x_2 - y_2|}{\frac{|x_3 - y_3|}{\ell(I_3)}}  
+ \frac{\frac{|x_3 - y_3|}{\ell(I_3)}}{\frac{|x_1 - y_1|}{\ell(J_1)} \frac{\rs(I_3, J_3)}{\ell(J_2)} |x_2 - y_2|} \Bigg)^{-\theta}
\nonumber \\
&\simeq \bigg(t_2 |x_2 - y_2| + \frac{1}{t_2 |x_2 - y_2|} \bigg)^{-\theta}, 
\end{align}
where $t_2 := \frac{\rd(I_1, J_1) \rs(I_3, J_3)}{\ell(J_2) \rd(I_3, J_3)}$. In view of \eqref{HH31}--\eqref{HH32}, similarly to \eqref{HH23}, we invoke \eqref{def:P1} and \eqref{def:Q0} to obtain 
\begin{align*}
|\mathscr{G}_{I, J}| 
&\lesssim \frac{\rs(I_1, J_1)^{\delta_1}}{\rd(I_1, J_1)^{\delta_1}}  
\mathscr{P}_1(I_1, J_1) \mathscr{Q}_2^{0, t_2}(I_2, J_2) \mathscr{P}_3(I_3, J_3) 
\prod_{i=1}^3  |I_i|^{-\frac12} |J_i|^{-\frac12} 
\\ 
&\lesssim F_1(I_1, J_1) \frac{\rs(I_1, J_1)^{\frac12 + \delta_1}}{\rd(I_1, J_1)^{1+\delta_1}} 
\widetilde{F}_2(I_2, J_2) \frac{\rs(I_2, J_2)^{\frac12 - \frac1r}}{D_{\theta}^{1, 3}(I, J)}
F_3(I_3, J_3) \frac{\rs(I_3, J_3)^{\frac12}}{\rd(I_3, J_3)}. 
\end{align*}
The proof is complete. 
\end{proof}


Let $k_1, k_3 \ge 0$ and $I, J \in \D_{\Z}$ with $I_1 \in J_1(k_1, j_1)$, $\rd(I_2, J_2) < 1$, and $I_3 \in J_3(-k_3, j_3)$. One has  
\begin{align*}
\ell(I_2) = \ell(I_1)^{-1} \ell(I_3)
=(2^{-k_1} \ell(J_1))^{-1} (2^{k_3} \ell(J_3))
=2^{k_1+k_3} \ell(J_2), 
\end{align*}
and much as in \eqref{Car-331}--\eqref{Car-332},  
\begin{align}
\label{Car-221} & \#\big\{J_2 \in \D: \ell(J_2) \le \ell(I_2), \rd(I_2, J_2) < 1\big\} \lesssim 2^{k_1 + k_3},
\\
\label{Car-222} & \#\big\{I_2 \in \D: \ell(I_2) \ge \ell(I_2), \rd(I_2, J_2) < 1\big\} \lesssim 1. 
\end{align}
By Lemma \ref{lem:HH-3} and Lemma \ref{lem:FF} parts \eqref{list-FF1}--\eqref{list-FF2}, there exists $N_0>1$ so that for all $N \ge N_0$,
\begin{align*}
|\I^N_{1, 3}| 
\le \sum_{k_1, k_3 \ge 0 \atop j_1, j_3 \ge 1}  \sum_{J \not\in \D_{\Z}(2N)}
\sum_{\substack{I \in \D_{\Z} \\ I_1 \in J_1(k_1, j_1) \\ \rd(I_2, J_2) < 1 \\ I_3 \in J_3(-k_3, j_3)}} |\G_{I, J}| \, |f_I| |g_J|
\lesssim \sum_{i=0}^2 \I^{N, i}_{1, 3},
\end{align*}
where
\begin{align*}
\I^{N, 0}_{1, 3}
&:= \varepsilon \sum_{k_1, k_3 \ge 0 \atop j_1, j_3 \ge 1} 
2^{-k_1 (\frac12 + \delta_1)} j_1^{-1-\delta_1} 
\frac{2^{-(k_1+k_3)(\frac12 - \frac1r)}}{(j_1^{-1} j_3 2^{k_3})^{\theta}} 
2^{-\frac{k_3}{2}} j_3^{-1} 
\sum_{\substack{I, J \in \D_{\Z} \\ I_1 \in J_1(k_1, j_1) \\ \rd(I_2, J_2) < 1 \\ I_3 \in J_3(-k_3, j_3)}} |f_I| |g_J|,
\\
\I^{N, i}_{1, 3}
&:= \sum_{\substack{k_1, k_3 \ge 0 \\ j_1, j_3 \ge 1 \\ j_{2i-1} \ge N^{1/8}}} 
2^{-k_1 (\frac12 + \delta_1)} j_1^{-1-\delta_1} 
\frac{2^{-(k_1+k_3)(\frac12 - \frac1r)}}{(j_1^{-1} j_3 2^{k_3})^{\theta}} 
2^{-\frac{k_3}{2}} j_3^{-1}   
\sum_{\substack{I, J \in \D_{\Z} \\ I_1 \in J_1(k_1, j_1) \\ \rd(I_2, J_2) < 1 \\ I_3 \in J_3(-k_3, j_3)}} |f_I| |g_J|,
\end{align*}
for $i=1, 2$. Applying the Cauchy--Schwarz inequality, \eqref{car-IJ}, and \eqref{Car-221}--\eqref{Car-222}, we deduce that
\begin{align*}
\I^{N, 0}_{1, 3}
&\le \varepsilon \sum_{k_1, k_3 \ge 0 \atop j_1, j_3 \ge 1} 
2^{-k_1 (\delta_1 - \frac1r)} j_1^{-1-(\delta_1-\theta)} 
2^{-k_3(\theta-\frac1r)} j_3^{-1-\theta} 2^{-(k_1+k_3)} 
\\
&\quad\times 
\bigg(\sum_{I \in \D_{\Z} } \sum_{\substack{J \in \D_{\Z} \\ J_1 \in I_1(k_1, j_1) \\ \rd(I_2, J_2) < 1 \\ J_3 \in I_3(-k_3, j_3)}} |f_I|^2 \bigg)^{\frac12}
\bigg(\sum_{J \in \D_{\Z} } \sum_{\substack{I \in \D_{\Z} \\ I_1 \in J_1(k_1, j_1) \\ \rd(I_2, J_2) < 1 \\ I_3 \in J_3(-k_3, j_3)}} |g_J|^2 \bigg)^{\frac12}
\\
&\lesssim \varepsilon \sum_{k_1, k_3 \ge 0 \atop j_1, j_3 \ge 1} 
2^{-k_1 (\delta_1 - \frac1r)} j_1^{-1-(\delta_1-\theta)} 
2^{-k_3(\theta-\frac1r)} j_3^{-1-\theta} 
\\
&\quad\times 
\bigg(\sum_{I \in \D_{\Z} } |f_I|^2 \bigg)^{\frac12}
\bigg(\sum_{J \in \D_{\Z} } |g_J|^2 \bigg)^{\frac12}
\\
&\lesssim \varepsilon \|f\|_{L^2(\R^3)} \|g\|_{L^2(\R^3)}, 
\end{align*}
and for each $i=1, 2$, 
\begin{align*}
\I^{N, i}_{1, 3}
&\lesssim \varepsilon \sum_{\substack{k_1, k_3 \ge 0 \\ j_1, j_3 \ge 1 \\ j_{2i-1} \ge N^{1/8}}} 
2^{-k_1 (\delta_1 - \frac1r)} j_1^{-1-(\delta_1-\theta)} 
2^{-k_3(\theta-\frac1r)} j_3^{-1-\theta} 
\|f\|_{L^2(\R^3)} \|g\|_{L^2(\R^3)} 
\\
&\lesssim \big[N^{-(\delta_1 - \theta)/8} + N^{-\theta/8} \big] \|f\|_{L^2(\R^3)} \|g\|_{L^2(\R^3)}
\le \varepsilon \|f\|_{L^2(\R^3)} \|g\|_{L^2(\R^3)}, 
\end{align*}
provided $N>N_0$ large enough. These estimates imply 
\begin{align*}
|\I^N_{1, 3}| 
\lesssim \sum_{i=0}^2 \I^{N, i}_{1, 3} 
\lesssim \varepsilon \|f\|_{L^2(\R^3)} \|g\|_{L^2(\R^3)}, 
\end{align*}
which along with \eqref{SN11} and \eqref{SN12} gives that for all $N>N_0$ sufficiently large, 
\begin{align*}
|\I^N_1|
\le \sum_{j=1}^3 |\I^N_{1, j}| 
\lesssim \varepsilon \|f\|_{L^2(\R^3)} \|g\|_{L^2(\R^3)}. 
\end{align*}



\subsection{$\rd(I_1, J_1) \ge 1$, $\rd(I_{2, 3}, J_{2, 3}) < 1$, and $I_{2, 3} \cap J_{2, 3} = \emptyset$}\label{sec:Sep-Near} 


\begin{lemma}\label{lem:HH-4}
Let $I, J \in \D_{\Z}$ with $\rd(I_1, J_1) \ge 1$, $\rd(I_{2, 3}, J_{2, 3}) < 1$, and $I_{2, 3} \cap J_{2, 3} = \emptyset$. Then
\begin{align*}
|\mathscr{G}_{I, J}|
&\lesssim F_1(I_1, J_1) \frac{\rs(I_1, J_1)^{\frac12 + \delta_0}}{\rd(I_1, J_1)^{1+\delta_0}} 
\widetilde{F}_2(I_2, J_2) \frac{\rs(I_2, J_2)^{\frac12}}{\ird(I_2, J_2)^{\delta_0}} 
\widetilde{F}_3(I_3, J_3) \frac{\rs(I_3, J_3)^{\frac12}}{\ird(I_3, J_3)^{\delta_0}}
\end{align*}
\end{lemma}


\begin{proof}
It is not hard to see that for any $x_1 \in J_1$ and $y_1 \in I_1$, 
\begin{align}\label{HH4}
|x_1-y_1| \simeq \d(I_1, J_1) \quad\text{and}\quad 
D_{\theta}(x-y) \simeq D_{\theta}(t, x_{2, 3} - y_{2, 3}), 
\end{align}
where $t := \d(I_1, J_1)$. Write $\G_{I, J} = \sum_{i=1}^4 \mathscr{H}_i$, where 
\begin{equation}\label{GHH}
\begin{aligned}
\mathscr{H}_1 
& := \langle T(h_{I_1} \otimes (h_{I_{2, 3}} \mathbf{1}_{(3J_2) \times (3J_3)})), 
h_{J_1} \otimes h_{J_{2, 3}} \rangle, 
\\
\mathscr{H}_2 
&:= \langle T(h_{I_1} \otimes (h_{I_{2, 3}} \mathbf{1}_{(3J_2) \times (3 J_3)^c})), 
h_{J_1} \otimes h_{J_{2, 3}} \rangle, 
\\
\mathscr{H}_3 
&:= \langle T(h_{I_1} \otimes (h_{I_{2, 3}} \mathbf{1}_{(3J_2)^c \times (3J_3)})), 
h_{J_1} \otimes h_{J_{2, 3}} \rangle, 
\\
\mathscr{H}_4  
&:= \langle T(h_{I_1} \otimes (h_{I_{2, 3}} \mathbf{1}_{(3J_2)^c \times (3J_3)^c})), 
h_{J_1} \otimes h_{J_{2, 3}} \rangle.
\end{aligned}
\end{equation}
Note that if $\ird(I_i, J_i)=1+\d(I_i, J_i)/\ell(J_i)>2$ for some $i=2, 3$, then $I_i \cap 3J_i=\emptyset$, and hence, $\mathscr{H}_1=0$. 
Thus, to bound $\mathscr{H}_1$, it suffices to consider the case $\ird(I_i, J_i) \le 2$ for each $i=2, 3$. We utilize the compact full kernel representation, the cancellation of $h_{I_1}$, the size-H\"{o}lder condition of $K$, and \eqref{HH4} to deduce that 
\begin{align*}
|\mathscr{H}_1| 
&\lesssim \int_{I_1 \times (3J_2) \times (3J_3)} \int_{J_1 \times J_{2, 3}} 
\bigg(\frac{|y_1 - c_{I_1}|}{|x_1 - y_1|}\bigg)^{\delta_1} D_{\theta}(x-y) 
\\
&\quad\times \prod_{i=1}^3 \frac{F_i(x_i, y_i)}{|x_i - y_i|} 
|h_{I_1} \otimes h_{I_{2, 3}}(y)| |h_{J_1} \otimes h_{J_{2, 3}}(x)| \, dx \, dy
\\
&\lesssim \frac{\rs(I_1, J_1)^{\delta_1}}{\rd(I_1, J_1)^{\delta_1}} \mathscr{P}_1(I_1, J_1) 
\mathscr{R}_{2, 3}^{0, 0, t}(I_{2, 3}, J_{2, 3}) \prod_{i=1}^3 |I_i|^{-\frac12} |J_i|^{-\frac12} 
\\
&\lesssim F_1(I_1, J_1) \frac{\rs(I_1, J_1)^{\frac12 + \delta_1}}{\rd(I_1, J_1)^{1+\delta_1}}  
\widetilde{F}_2(I_2, J_2) \rs(I_2, J_2)^{\frac12} 
\widetilde{F}_3(I_3, J_3) \rs(I_3, J_3)^{\frac12}
\\
&\lesssim F_1(I_1, J_1) \frac{\rs(I_1, J_1)^{\frac12 + \delta_0}}{\rd(I_1, J_1)^{1+\delta_0}}  
\widetilde{F}_2(I_2, J_2) \frac{\rs(I_2, J_2)^{\frac12}}{\ird(I_2, J_2)^{\delta_0}}
\widetilde{F}_3(I_3, J_3) \frac{\rs(I_3, J_3)^{\frac12}}{\ird(I_3, J_3)^{\delta_0}}
\end{align*}
where we have used \eqref{def:P1}, \eqref{def:R2300}, and that $\ird(I_i, J_i) \le 2$ for each $i=2, 3$. Similarly, 
\begin{align*}
|\mathscr{H}_2| 
&\lesssim \frac{\rs(I_1, J_1)^{\delta_1}}{\rd(I_1, J_1)^{\delta_1}} \mathscr{P}_1(I_1, J_1) 
\mathscr{R}_{2, 3}^{0, 1, t}(I_{2, 3}, J_{2, 3}) \prod_{i=1}^3 |I_i|^{-\frac12} |J_i|^{-\frac12} 
\\
&\lesssim F_1(I_1, J_1) \frac{\rs(I_1, J_1)^{\frac12 + \delta_0}}{\rd(I_1, J_1)^{1+\delta_0}}  
\widetilde{F}_2(I_2, J_2) \frac{\rs(I_2, J_2)^{\frac12}}{\ird(I_2, J_2)^{\delta_0}}
\widetilde{F}_3(I_3, J_3) \frac{\rs(I_3, J_3)^{\frac12}}{\ird(I_3, J_3)^{\delta_0}}, 
\end{align*}
where \eqref{def:P1} and \eqref{def:R2301} were used in the second-to-last inequality. Symmetrically, $\mathscr{H}_3$ has the same bound. 

To proceed, using the compact full kernel representation, the cancellation of $h_{I_1}$ and $h_{J_{2, 3}}$, the H\"{o}lder condition of $K$, and \eqref{HH4}, we deduce that 
\begin{align*}
|\mathscr{H}_4| 
&\lesssim \int_{I_1 \times (I_2 \setminus 3J_2) \times (I_3 \setminus 3J_3)} \int_{J_1 \times J_{2, 3}} 
\bigg(\frac{|y_1 - c_{I_1}|}{|x_1 - y_1|}\bigg)^{\delta_1} 
\bigg(\frac{|x_2 - c_{J_2}|}{|x_2 - y_2|} + \frac{|x_3 - c_{J_3}|}{|x_3 - y_3|}\bigg)^{\delta_{2, 3}} 
\\
&\quad\times D_{\theta}(x-y)  \prod_{i=1}^3 \frac{F_i(x_i, y_i)}{|x_i - y_i|} 
|h_{I_1} \otimes h_{I_{2, 3}}(y)| |h_{J_1} \otimes h_{J_{2, 3}}(x)| \, dx \, dy
\\
&\lesssim \frac{\rs(I_1, J_1)^{\delta_1}}{\rd(I_1, J_1)^{\delta_1}} \mathscr{P}_1(I_1, J_1) 
\mathscr{R}_{2, 3}^{1, 1, t}(I_{2, 3}, J_{2, 3}) \prod_{i=1}^3 |I_i|^{-\frac12} |J_i|^{-\frac12} 
\\
&\lesssim F_1(I_1, J_1) \frac{\rs(I_1, J_1)^{\frac12 + \delta_0}}{\rd(I_1, J_1)^{1+\delta_0}}  
\widetilde{F}_2(I_2, J_2) \frac{\rs(I_2, J_2)^{\frac12}}{\ird(I_2, J_2)^{\delta_0}}
\widetilde{F}_3(I_3, J_3) \frac{\rs(I_3, J_3)^{\frac12}}{\ird(I_2, J_2)^{\delta_0}}, 
\end{align*}
where \eqref{def:P1} and \eqref{def:R2311} were used in the second-to-last inequality. 
\end{proof}


To estimate $\I^N_2$, we introduce some notation. Given $i=1, 2, 3$, $k_i \in \ZZ$, $m_i \ge 1$, and $I_i, J_i \in \D$, let 
\begin{align*}
J_i(k_i, 0, m_i) &:= \big\{I_i \in \D: \, \ell(I_i) = 2^{-k_i} \ell(J_i), \, \rd(I_i, J_i) < 1, \, m_i \le \ird(I_i, J_i) < m_i+1\big\}, 
\\
I_i(k_i, 0, m_i) &:= \big\{J_i \in \D: \, \ell(J_i) = 2^{k_i} \ell(I_i), \, \rd(I_i, J_i) < 1, \, m_i \le \ird(I_i, J_i) < m_i+1\big\}.
\end{align*}
Then it is not hard to check that $I_i \in J_i(k_i, 0, m_i)$ if and only if $J_i \in I_i(k_i, 0, m_i)$,
\begin{align}\label{car-IJK}
\#J_i(k_i, 0, m_i) \lesssim 1, \quad\text{ and }\quad
 \# \bigg(\bigcup_{m_i=1}^{2^{k_i}} I_i(k_i, 0, m_i) \bigg) \lesssim 1.
\end{align}
Now by Lemmas \ref{lem:HH-4}, \eqref{JDD}, and Lemma \ref{lem:FF} parts \eqref{list-FF1}--\eqref{list-FF2}, there exists $N_0>1$ so that for all $N \ge N_0$,
\begin{align*}
|\I^N_2|
&\le \sum_{k_2 \ge k_1 \ge 0 \atop j_1 \ge 1} 
\sum_{1 \le m_2 \le 2^{k_2} \atop 1 \le m_3 \le 2^{k_2 - k_1}} 
\sum_{J \not\in \D_{\Z}(2N)} 
\sideset{}{'}\sum_{\substack{I \in \D_{\Z} \\ I_1 \in J_1(k_1, j_1) \\ I_2 \in J_2(-k_2, 0, m_2) \\ I_3 \in J_3(k_1-k_2, 0, m_3)}} 
|\G_{I, J}| \, |f_I| \, |g_J|
\lesssim \sum_{i=1}^3 \I^N_{2, i}, 
\end{align*}
where 
\begin{align*}
\I^N_{2, 1} 
&:= \varepsilon \sum_{k_2 \ge k_1 \ge 0 \atop j_1 \ge 1} 
\sum_{1 \le m_2 \le 2^{k_2} \atop 1 \le m_3 \le 2^{k_2 - k_1}} 
2^{-k_1(\frac12 + \delta_0)} j_1^{-1-\delta_0} 
2^{-\frac{k_2}{2}} 2^{-\frac{k_2-k_1}{2}} m_2^{-\delta_0} m_3^{-\delta_0}
\sideset{}{'}\sum_{\substack{I, J \in \D_{\Z} \\ I_1 \in J_1(k_1, j_1) \\ I_2 \in J_2(-k_2, 0, m_2) \\ I_3 \in J_3(k_1-k_2, 0, m_3)}} 
|f_I| \, |g_J|, 
\\
\I^N_{2, 2} 
&:= \sum_{k_2 \ge k_1 \ge 0 \atop j_1 \ge N^{1/8}} 
\sum_{1 \le m_2 \le 2^{k_2} \atop 1 \le m_3 \le 2^{k_2 - k_1}} 
2^{-k_1(\frac12 + \delta_0)} j_1^{-1-\delta_0} 2^{-\frac{k_2}{2}} 2^{-\frac{k_2-k_1}{2}} m_2^{-\delta_0} m_3^{-\delta_0}
\sideset{}{'}\sum_{\substack{I, J \in \D_{\Z} \\ I_1 \in J_1(k_1, j_1) \\ I_2 \in J_2(-k_2, 0, m_2) \\ I_3 \in J_3(k_1-k_2, 0, m_3)}} 
|f_I| \, |g_J|, 
\\
\I^N_{2, 3} 
&:= \sum_{\substack{k_2 \ge k_1 \ge 0 \\ j_1 \ge 1 \\ k_2 \ge N}} 
\sum_{1 \le m_2 \le 2^{k_2} \atop 1 \le m_3 \le 2^{k_2 - k_1}} 
2^{-k_1(\frac12 + \delta_0)} j_1^{-1-\delta_0} 2^{-\frac{k_2}{2}} 2^{-\frac{k_2-k_1}{2}} m_2^{-\delta_0} m_3^{-\delta_0}
\sideset{}{'}\sum_{\substack{I, J \in \D_{\Z} \\ I_1 \in J_1(k_1, j_1) \\ I_2 \in J_2(-k_2, 0, m_2) \\ I_3 \in J_3(k_1-k_2, 0, m_3)}} 
|f_I| \, |g_J|. 
\end{align*}
Note that for all $k \ge 0$ and $\alpha, \beta \in (0, 1)$ with $\beta \le \frac{1}{1+2\alpha}$, 
\begin{align}\label{MDT}
2^{-k} \sum_{1 \le m \le 2^k} m^{-2\alpha}
&=2^{-k} \sum_{1 \le m \le 2^{k \beta}} m^{-2 \alpha} 
+ 2^{-k} \sum_{2^{k \beta} < m \le 2^k} m^{-2 \alpha}
\nonumber \\
&\le 2^{-k}  2^{k\beta} + 2^{-k} 2^{k-2k \alpha \beta}
\le 2 \cdot 2^{-2k \alpha \beta}.
\end{align}
Then, from the Cauchy-Schwarz inequality, \eqref{car-IJ}, \eqref{car-IJK}--\eqref{MDT}, it follows that 
\begin{align*}
\I^N_{2, 1} 
&\le \varepsilon \sum_{k_2 \ge k_1 \ge 0 \atop j_1 \ge 1} 
2^{-k_1(\frac12 + \delta_0)} j_1^{-1-\delta_0} 2^{-\frac{k_2}{2}} 2^{-\frac{k_2-k_1}{2}} 
\\
&\quad\times 
\bigg(\sum_{I \in \D_{\Z}} \sum_{1 \le m_2 \le 2^{k_2} \atop 1 \le m_3 \le 2^{k_2 - k_1}} 
m_2^{-2\delta_0} m_3^{-2\delta_0}
\sideset{}{'}\sum_{\substack{J \in \D_{\Z} \\ J_1 \in I_1(k_1, j_1) \\ 
J_2 \in I_2(-k_2, 0, m_2) \\ J_3 \in J_3(k_1-k_2, 0, m_3)}} |f_I|^2 \bigg)^{\frac12}
\\
&\quad\times 
\bigg(\sum_{J \in \D_{\Z}} \sum_{1 \le m_2 \le 2^{k_2} \atop 1 \le m_3 \le 2^{k_2 - k_1}} 
\sideset{}{'}\sum_{\substack{I \in \D_{\Z} \\ I_1 \in J_1(k_1, j_1) \\ 
I_2 \in J_2(-k_2, 0, m_2) \\ I_3 \in J_3(k_1-k_2, 0, m_3)}} |g_J|^2 \bigg)^{\frac12}
\\ 
&\lesssim \varepsilon \sum_{k_2 \ge k_1 \ge 0 \atop j_1 \ge 1} 
2^{-k_1 \delta_0} j_1^{-1-\delta_0} 
\bigg(2^{-k_2} \sum_{1 \le m_2 \le 2^{k_2}} m_2^{-2\delta_0}  \bigg)^{\frac12}
\\
&\quad\times \bigg(2^{-(k_2-k_1)} \sum_{1 \le m_3 \le 2^{k_2 - k_1}} m_3^{-2\delta_0} \bigg)^{\frac12}
\bigg(\sum_{I \in \D_{\Z}} |f_I|^2 \bigg)^{\frac12} 
\bigg(\sum_{J \in \D_{\Z}} |g_J|^2 \bigg)^{\frac12}
\\ 
&\lesssim \varepsilon \sum_{j_1 \ge 1} j_1^{-1-\delta_0} 
\sum_{k_2 \ge k_1 \ge 0} 2^{-k_1 \delta_0} 2^{-k_2 \delta_0 \beta} 2^{-(k_2-k_1) \delta_0 \beta}
\|f\|_{L^2(\R^3)} \|g\|_{L^2(\R^3)}
\\ 
&\lesssim \varepsilon \|f\|_{L^2(\R^3)} \|g\|_{L^2(\R^3)},  
\end{align*}
where $\beta \in (0, 1)$ small enough. Analogously, for any $N > N_0$ sufficiently large, one has 
\begin{align*}
\I^N_{2, 2} 
&\lesssim \sum_{j_1 \ge N^{1/8}} j_1^{-1-\delta_0} 
\sum_{k_2 \ge k_1 \ge 0} 2^{-k_1 \delta_0} 2^{-k_2 \delta_0 \beta} 2^{-(k_2-k_1) \delta_0 \beta}
\|f\|_{L^2(\R^3)} \|g\|_{L^2(\R^3)}
\\ 
&\lesssim N^{-\delta_0/8} \|f\|_{L^2(\R^3)} \|g\|_{L^2(\R^3)}
\le \varepsilon \|f\|_{L^2(\R^3)} \|g\|_{L^2(\R^3)}, 
\end{align*}
and 
\begin{align*}
\I^N_{2, 3} 
&\lesssim \sum_{j_1 \ge 1} j_1^{-1-\delta_0} 
\sum_{k_1 \ge 0 \atop k_2 \ge \max\{k_1, N\}} 2^{-k_1 \delta_0} 2^{-k_2 \delta_0 \beta} 2^{-(k_2-k_1) \delta_0 \beta}
\|f\|_{L^2(\R^3)} \|g\|_{L^2(\R^3)}
\\ 
&\lesssim \big(2^{-N \delta_1} + 2^{-N\delta_0 \beta} \big) \|f\|_{L^2(\R^3)} \|g\|_{L^2(\R^3)}
\le \varepsilon \|f\|_{L^2(\R^3)} \|g\|_{L^2(\R^3)}, 
\end{align*}
provided the following estimate: for all $\alpha_1, \alpha_2>0$ and $N \ge 0$, 
\begin{align}\label{kaka}
&\sum_{k_1 \ge 0 \atop k_2 \ge \max\{k_1, N\}} 
2^{-k_1 \alpha_1} \, 2^{-k_2 \alpha_2} \, 2^{-(k_2-k_1) \alpha_2} 
\nonumber \\
&\quad\le \sum_{k_2 \ge k_1 \ge N} 2^{-k_1 \alpha_1} 2^{-2(k_2-k_1) \alpha_2} 
+ \sum_{k_2 \ge N > k_1} 2^{-k_1 \alpha_1} 2^{- k_2 \alpha_2} 
\nonumber \\
&\quad\le \sum_{k_1 \ge N \atop k_2 \ge 0} 2^{-k_1 \alpha_1} 2^{-2 k_2 \alpha_2} 
+ \sum_{k_1 \ge 0 \atop k_2 \ge N} 2^{-k_1 \alpha_1} 2^{- k_2 \alpha_2} 
\nonumber \\ 
&\quad\lesssim 2^{-N \alpha_1} + 2^{-N \alpha_2}.
\end{align}
Hence, gathering the estimates above, we obtain that for any $N > N_0$ large enough, 
\begin{align*}
|\I^N_2|
\lesssim \sum_{i=1}^3 \I^N_{2, i} 
\lesssim \varepsilon \|f\|_{L^2(\R^3)} \|g\|_{L^2(\R^3)}. 
\end{align*}


\subsection{$\rd(I_1, J_1) \ge 1$ and $I_{2, 3} \supset J_{2, 3}$} 


\begin{lemma}\label{lem:HH-5}
Let $I, J \in \D_{\Z}$ with $\ell(I_1) \le \ell(J_1)$, $\ell(I_2) \ge \ell(J_2)$, and $\ell(I_3) \ge \ell(J_3)$ so that $\rd(I_1, J_1) \ge 1$ and  $I_{2, 3} \supset J_{2, 3}$. Then
\begin{align*}
|\mathscr{G}_{I, J}|
\lesssim F_1(I_1, J_1) \frac{\rs(I_1, J_1)^{\frac12 + \delta_0}}{\rd(I_1, J_1)^{1+\delta_0}} 
\widehat{F}_2(J_2) \rs(I_2, J_2)^{\frac12} \widehat{F}_3(J_3) \rs(I_3, J_3)^{\frac12}. 
\end{align*}
\end{lemma}


\begin{proof}
We begin with the case $I_{2, 3} \supsetneq J_{2, 3}$. By the fact that $\mathbf{1}=\mathbf{1}_J + \mathbf{1}_{3J \setminus J} + \mathbf{1}_{(3J)^c}$ for any $J \in \D$, we perform the decomposition: $\G_{I, J} = \sum_{i=1}^9 \mathscr{L}_i$, where 
\begin{equation}\label{GLL}
\begin{aligned}
\mathscr{L}_1
& := \langle T(h_{I_1} \otimes (h_{I_{2, 3}} \mathbf{1}_{J_2 \times J_3})), 
h_{J_1} \otimes h_{J_{2, 3}} \rangle, 
\\ 
\mathscr{L}_2
&:= \langle T(h_{I_1} \otimes (h_{I_{2, 3}} \mathbf{1}_{J_2 \times (3J_3 \setminus J_3)})), 
h_{J_1} \otimes h_{J_{2, 3}} \rangle, 
\\ 
\mathscr{L}_3
&:= \langle T(h_{I_1} \otimes (h_{I_{2, 3}} \mathbf{1}_{J_2 \times (3J_3)^c})), 
h_{J_1} \otimes h_{J_{2, 3}} \rangle, 
\\
\mathscr{L}_4
&:= \langle T(h_{I_1} \otimes (h_{I_{2, 3}} \mathbf{1}_{(3J_2 \setminus J_2) \times J_3})), 
h_{J_1} \otimes h_{J_{2, 3}} \rangle, 
\\
\mathscr{L}_5
&:= \langle T(h_{I_1} \otimes (h_{I_{2, 3}} \mathbf{1}_{(3J_2 \setminus J_2) \times (3J_3 \setminus J_3)})), 
h_{J_1} \otimes h_{J_{2, 3}} \rangle, 
\\
\mathscr{L}_6
&:= \langle T(h_{I_1} \otimes (h_{I_{2, 3}} \mathbf{1}_{(3J_2 \setminus J_2) \times (3J_3)^c})), 
h_{J_1} \otimes h_{J_{2, 3}} \rangle, 
\\
\mathscr{L}_7
&:= \langle T(h_{I_1} \otimes (h_{I_{2, 3}} \mathbf{1}_{(3J_2)^c \times J_3})), 
h_{J_1} \otimes h_{J_{2, 3}} \rangle, 
\\
\mathscr{L}_8
&:= \langle T(h_{I_1} \otimes (h_{I_{2, 3}} \mathbf{1}_{(3J_2)^c \times (3J_3 \setminus J_3)})), 
h_{J_1} \otimes h_{J_{2, 3}} \rangle, 
\\
\mathscr{L}_9
&:= \langle T(h_{I_1} \otimes (h_{I_{2, 3}} \mathbf{1}_{(3J_2)^c \times (3J_3)^c})), 
h_{J_1} \otimes h_{J_{2, 3}} \rangle.
\end{aligned}
\end{equation}
Observe that $\mathscr{L}_4$, $\mathscr{L}_7$, and $\mathscr{L}_8$ are symmetrical to $\mathscr{L}_2$, $\mathscr{L}_3$, and $\mathscr{L}_6$, respectively, while $\mathscr{L}_6$ is similar to $\mathscr{L}_3$ because of $\mathscr{R}_{2, 3}^{1, 0, t}(I_{2, 3}, J_{2, 3})$ and that $3J_2 \setminus J_2$ can be written as the union of two disjoint dyadic cubes with the side-length $\ell(J_2)$. Hence, all of them have the same bound and it suffices to treat $\mathscr{L}_i$ for $i=1, 2, 3, 5, 9$. 

Let
\begin{align}\label{def:HIJ}
\widetilde{h}_{J_{2, 3}} := |J_{2, 3}|^{\frac12} h_{J_{2, 3}}
\quad\text{ and }\quad 
\widetilde{h}_{I_{2, 3}} := |I_{2, 3}|^{\frac12} h_{I_{2, 3}} \mathbf{1}_{J_{2, 3}}. 
\end{align}
Note that $\supp(\widetilde{h}_{J_{2, 3}}) \subset J_{2, 3}$, $\|\widetilde{h}_{I_{2, 3}}\|_{L^{\infty}} \le 1$, $\int_{\R^2} \widetilde{h}_{J_{2, 3}} \, dx_{2, 3} = 0$, $\supp(\widetilde{h}_{I_{2, 3}}) \subset J_{2, 3}$, and $\|\widetilde{h}_{I_{2, 3}}\|_{L^{\infty}} \le 1$. If $I_2 \supsetneq J_2$ and $I_3 \supsetneq J_3$, then $\widetilde{h}_{I_{2, 3}} = |I_{2, 3}|^{\frac12} \langle h_{I_{2, 3}} \rangle_{J_{2, 3}} \mathbf{1}_{J_{2, 3}}$. If $I_2=J_2$ and $I_3 \supsetneq J_3$, then $\widetilde{h}_{I_{2, 3}} = |I_{2, 3}|^{\frac12} \langle h_{I_3} \rangle_{J_3} \, h_{J_2} \otimes \mathbf{1}_{J_3}$ and $I_2 \cap (3J_2 \setminus J_2) = \emptyset = I_2 \cap (3J_2)^c$, where the former owns the cancellation property on $\R^2$ and the later implies $\mathscr{L}_i=0$ for each $i=4, \ldots, 9$. If $I_2 \supsetneq J_2$ and $I_3 = J_3$, then $\widetilde{h}_{I_{2, 3}} = |I_{2, 3}|^{\frac12} \langle h_{I_2} \rangle_{J_2} \, \mathbf{1}_{J_2} \otimes h_{J_3}$ and $I_3 \cap (3J_3 \setminus J_3) = \emptyset = I_3 \cap (3J_3)^c$, where the former owns the cancellation property on $\R^2$ and the later implies $\mathscr{L}_i=0$ for each $i=2, 3, 5, 6, 8, 9$. By the compact partial kernel representation, the cancellation of $h_{I_1}$, and the H\"{o}lder condition of $K_{\widetilde{h}_{I_{2, 3}}, \widetilde{h}_{J_{2, 3}}}$, one has
\begin{align}
|\mathscr{L}_1|
&= |I_{2, 3}|^{-\frac12} | |J_{2, 3}|^{-\frac12} 
|\langle T(h_{I_1} \otimes \widetilde{h}_{I_{2, 3}}), h_{J_1} \otimes \widetilde{h}_{J_{2, 3}} \rangle|
\nonumber \\
&\lesssim  \frac{\rs(I_1, J_1)^{\delta_1}}{\rd(I_1, J_1)^{\delta_1}} \mathscr{P}_1(I_1, J_1)
C(\widetilde{h}_{I_{2, 3}}, \widetilde{h}_{J_{2, 3}})
\prod_{i=1}^3 |I_i|^{-\frac12} |J_i|^{-\frac12}
\nonumber \\
&\lesssim F_1(I_1, J_1) \frac{\rs(I_1, J_1)^{\delta_1}}{\rd(I_1, J_1)^{1 + \delta_1}} |I_1| \, 
F_2(J_2) |J_2| \, F_2(J_3) |J_3| \prod_{i=1}^3 |I_i|^{-\frac12} |J_i|^{-\frac12}
\nonumber \\
&\le F_1(I_1, J_1) \frac{\rs(I_1, J_1)^{\frac12 + \delta_0}}{\rd(I_1, J_1)^{1 + \delta_0}}
F_2(J_2) \rs(I_2, J_2)^{\frac12} \, F_3(J_3) \rs(I_3, J_3)^{\frac12},
\end{align}
where we have used \eqref{def:P1} and that $0<\delta_0 \le \delta_1$. Similarly to $\mathscr{H}_1$ in Section \ref{sec:Sep-Near}, we see that both $\mathscr{L}_2$ and $\mathscr{L}_5$ have the same bound as above. Denote $t:= \d(I_1, J_1)$, hence $\frac{t |J_2|}{|J_3|} = \rd(I_1, J_1)$. Analogously to $\mathscr{H}_2$ and $\mathscr{H}_4$ respectively in Section \ref{sec:Sep-Near}, \eqref{def:P1}, \eqref{def:R2302}, and \eqref{def:R2322} imply 
\begin{align}
|\mathscr{L}_3| + |\mathscr{L}_9| 
&\lesssim \frac{\rs(I_1, J_1)^{\delta_1}}{\rd(I_1, J_1)^{\delta_1}} \mathscr{P}_1(I_1, J_1) 
\mathscr{R}_{2, 3}^{0, 2, t}(I_{2, 3}, J_{2, 3}) \prod_{i=1}^3 |I_i|^{-\frac12} |J_i|^{-\frac12} 
\nonumber \\
&\quad+ \frac{\rs(I_1, J_1)^{\delta_1}}{\rd(I_1, J_1)^{\delta_1}} \mathscr{P}_1(I_1, J_1) 
\mathscr{R}_{2, 3}^{2, 2, t}(I_{2, 3}, J_{2, 3}) \prod_{i=1}^3 |I_i|^{-\frac12} |J_i|^{-\frac12} 
\nonumber \\
&\lesssim F_1(I_1, J_1) \frac{\rs(I_1, J_1)^{\frac12 + \delta_0}}{\rd(I_1, J_1)^{1+\delta_0}}  
\widetilde{F}_2(J_2) \rs(I_2, J_2)^{\frac12} 
\widetilde{F}_3(J_3) \rs(I_3, J_3)^{\frac12}. 
\end{align}



Next, let us deal with the case $I_{2, 3} = J_{2, 3}$. Setting $\ch(J_{2, 3}) := \ch(J_2) \times \ch(J_3)$, we rewrite 
\begin{align}\label{HH50}
\G_{I, J}
&= \sum_{J'_{2, 3}, \, J''_{2, 3} \in \ch(J_{2, 3})} 
\langle h_{J_{2, 3}} \rangle_{J'_{2, 3}} \langle h_{J_{2, 3}} \rangle_{J''_{2, 3}}
\langle T(h_{I_1} \otimes \mathbf{1}_{J'_{2, 3}}), h_{J_1} \otimes \mathbf{1}_{J''_{2, 3}} \rangle.
\end{align}
In the case $J'_{2, 3} \neq J''_{2, 3}$, we observe that $\rd(J'_{2, 3}, J''_{2, 3})<1$ and $J'_{2, 3} \cap J''_{2, 3} = \emptyset$, which is similar to the situation of Lemma \ref{lem:HH-4}. Hence, it follows  
\begin{align}\label{HH52}
|\langle T(h_{I_1} \otimes \mathbf{1}_{J'_{2, 3}}), h_{J_1} \otimes \mathbf{1}_{J''_{2, 3}} \rangle| 
&\lesssim F_1(I_1, J_1) \frac{\rs(I_1, J_1)^{\frac12 + \delta_0}}{\rd(I_1, J_1)^{1+\delta_0}}
\widetilde{F}_2(J'_2) |J'_2| \, \widetilde{F}_3(J'_3) |J'_3|
\nonumber \\
&\lesssim F_1(I_1, J_1) \frac{\rs(I_1, J_1)^{\frac12 + \delta_0}}{\rd(I_1, J_1)^{1+\delta_0}}
\widetilde{F}_2(J_2) |J_2| \, \widetilde{F}_3(J_3) |J_3|. 
\end{align}
To handle the case $J'_{2, 3}=J''_{2, 3}$, invoking the compact partial kernel representation, the cancellation of $h_{I_1}$, the H\"{o}lder condition of $K_{I'_{2, 3}, I'_{2, 3}}$, and \eqref{def:P1}, we obtain  
\begin{align}\label{HH51}
|\langle T(h_{I_1} \otimes \mathbf{1}_{J'_{2, 3}}), h_{J_1} \otimes \mathbf{1}_{J'_{2, 3}} \rangle| 
&\lesssim C(\mathbf{1}_{J'_{2, 3}}, \mathbf{1}_{J'_{2, 3}}) 
\frac{\rs(I_1, J_1)^{\delta_1}}{\rd(I_1, J_1)^{\delta_1}} \mathscr{P}_1(I_1, J_1) |I_1|^{-\frac12} |J_1|^{-\frac12} 
\nonumber \\
&\lesssim F_1(I_1, J_1) \frac{\rs(I_1, J_1)^{\frac12 + \delta_1}}{\rd(I_1, J_1)^{1+\delta_1}} F_2(J'_2) |J'_2| F_3(J'_3) |J'_3|
\nonumber \\
&\lesssim F_1(I_1, J_1) \frac{\rs(I_1, J_1)^{\frac12 + \delta_0}}{\rd(I_1, J_1)^{1+\delta_0}} F_2(J_2) |J_2| F_3(J_3) |J_3|. 
\end{align}
Therefore, the desired conclusion follows from \eqref{GLL}--\eqref{HH51}.
\end{proof}




Now applying Lemmas \ref{lem:HH-5},  \eqref{JDD}, and Lemma \ref{lem:FF} parts \eqref{list-FF1} and \eqref{list-FF3}, we obtain that there exists $N_0>1$ so that for all $N \ge N_0$,
\begin{align*}
|\I^N_3| 
\le \sum_{\substack{k_2 \ge k_1 \ge 0 \\ j_1 \ge 1}} 
\sum_{J \not\in \D_{\Z}(2N)}
\sum_{\substack{I \in \D_{\Z} \\ I_1 \in J_1(k_1, j_1) \\ I_2 = J_2^{(k_2)} \\ I_3 = J_3^{(k_2-k_1)}}} |\G_{I, J}| \, |f_I| |g_J|
\lesssim \sum_{i=1}^3 \I^N_{3, i},
\end{align*}
where
\begin{align*}
\I^N_{3, 1}
&:= \varepsilon \sum_{\substack{k_2 \ge k_1 \ge 0 \\ j_1 \ge 1}} 
2^{-k_1(\frac12 + \delta_1)} j_1^{-1-\delta_1} 2^{-\frac{k_2}{2}} 2^{-\frac{k_2-k_1}{2}} 
\sum_{\substack{I, J \in \D_{\Z} \\ I_1 \in J_1(k_1, j_1) \\ I_2 = J_2^{(k_2)} \\ I_3 = J_3^{(k_2-k_1)}}} |f_I| |g_J|,
\\
\I^N_{3, 2}
&:= \sum_{\substack{k_2 \ge k_1 \ge 0 \\ j_1 \ge N^{1/8}}} 
2^{-k_1(\frac12 + \delta_1)} j_1^{-1-\delta_1} 2^{-\frac{k_2}{2}} 2^{-\frac{k_2-k_1}{2}} 
\sum_{\substack{I, J \in \D_{\Z} \\ I_1 \in J_1(k_1, j_1) \\ I_2 = J_2^{(k_2)} \\ I_3 = J_3^{(k_2-k_1)}}} |f_I| |g_J|,
\\
\I^N_{3, 3}
&:= \sum_{\substack{k_2 \ge k_1 \ge 0 \\ j_1 \ge 1 \\ k_2 \ge N}} 
2^{-k_1(\frac12 + \delta_1)} j_1^{-1-\delta_1} 2^{-\frac{k_2}{2}} 2^{-\frac{k_2-k_1}{2}} 
\sum_{\substack{I, J \in \D_{\Z} \\ I_1 \in J_1(k_1, j_1) \\ I_2 = J_2^{(k_2)} \\ I_3 = J_3^{(k_2-k_1)}}} |f_I| |g_J|. 
\end{align*}
Then, it follows from the Cauchy--Schwarz inequality and \eqref{car-IJ} that
\begin{align*}
\I^N_{3, 1}
&\le \varepsilon \sum_{\substack{k_2 \ge k_1 \ge 0 \\ j_1 \ge 1}} 
2^{-k_1(\frac12 + \delta_1)} j_1^{-1-\delta_1} 2^{-\frac{k_2}{2}} 2^{-\frac{k_2-k_1}{2}} 
\\
&\quad\times 
\bigg(\sum_{\substack{I \in \D_{\Z} \\ I_2= J_2^{(k_2)} \\ I_3 = J_3^{(k_2-k_1)}}} 
\sum_{J_1 \in I_1(k_1, j_1)} |f_I|^2 \bigg)^{\frac12}
\bigg(\sum_{J \in \D_{\Z}} \sum_{I_1 \in J_1(k_1, j_1)} |g_J|^2 \bigg)^{\frac12}
\\
&\lesssim \varepsilon \sum_{\substack{k_2 \ge k_1 \ge 0 \\ j_1 \ge 1}} 
2^{-k_1 \delta_1} j_1^{-1-\delta_1} 2^{-\frac{k_2}{2}} 2^{-\frac{k_2-k_1}{2}} 
\|f\|_{L^2(\R^3)} \|g\|_{L^2(\R^3)}
\\
&\lesssim \varepsilon \|f\|_{L^2(\R^3)} \|g\|_{L^2(\R^3)}. 
\end{align*}
Analogously,  
\begin{align*}
\I^N_{3, 2}
&\lesssim \sum_{\substack{k_2 \ge k_1 \ge 0 \\ j_1 \ge N^{1/8}}} 
2^{-k_1 \delta_1} j_1^{-1-\delta_1} 2^{-\frac{k_2}{2}} 2^{-\frac{k_2-k_1}{2}} 
\|f\|_{L^2(\R^3)} \|g\|_{L^2(\R^3)}
\\
&\lesssim N^{-\delta_1/8} \|f\|_{L^2(\R^3)} \|g\|_{L^2(\R^3)}
\le \varepsilon \|f\|_{L^2(\R^3)} \|g\|_{L^2(\R^3)},  
\end{align*}
and \eqref{kaka} implies 
\begin{align*}
\I^N_{3, 3}
&\lesssim \varepsilon \sum_{\substack{k_2 \ge k_1 \ge 0 \\ j_1 \ge 1 \\ k_2 \ge N}} 
2^{-k_1 \delta_1} j_1^{-1-\delta_1} 2^{-\frac{k_2}{2}} 2^{-\frac{k_2-k_1}{2}} 
\|f\|_{L^2(\R^3)} \|g\|_{L^2(\R^3)}
\\
&\lesssim \big(2^{-N \delta_1} + 2^{-N/2}\big) \|f\|_{L^2(\R^3)} \|g\|_{L^2(\R^3)}
\le \varepsilon \|f\|_{L^2(\R^3)} \|g\|_{L^2(\R^3)},  
\end{align*}
whenever $N>N_0$ large enough. Consequently, one has 
\begin{align*}
|\I^N_3| 
\lesssim \sum_{i=1}^3 \I^N_{3, i}
\lesssim \varepsilon \|f\|_{L^2(\R^3)} \|g\|_{L^2(\R^3)}. 
\end{align*}


\subsection{$\rd(I_1, J_1) < 1$, $I_1 \cap J_1 = \emptyset$, $\rd(I_{2, 3}, J_{2, 3}) \ge 1$} 
To control $\I^N_4$, by \eqref{NNN}, we have to consider three cases as aforementioned. In each case, we split $\G_{I, J} = \mathscr{J}_1 + \mathscr{J}_2$, where 
\begin{equation}\label{GJJ}
\begin{aligned}
\mathscr{J}_1
&:= \langle T(h_{I_1} \otimes h_{I_{2, 3}}), 
(h_{J_1} \mathbf{1}_{3I_1}) \otimes h_{J_{2, 3}} \rangle, 
\\
\mathscr{J}_2 
&:= \langle T(h_{I_1} \otimes h_{I_{2, 3}}), 
(h_{J_1} \mathbf{1}_{(3I_1)^c}) \otimes h_{J_{2, 3}} \rangle. 
\end{aligned}
\end{equation}

\subsubsection{$\rd(I_1, J_1) < 1$, $I_1 \cap J_1 = \emptyset$, $\rd(I_2, J_2) \ge 1$, and $\rd(I_3, J_3) \ge 1$} 


\begin{lemma}\label{lem:HH-6}
Let $I, J \in \D_{\Z}$ with $\ell(I_1) \le \ell(J_1)$, $\ell(I_2) \ge \ell(J_2)$, and $\ell(I_3) \ge \ell(J_3)$ so that $\rd(I_1, J_1) < 1$, $I_1 \cap J_1 = \emptyset$, $\rd(I_2, J_2) \ge 1$, and $\rd(I_3, J_3) \ge 1$. Then
\begin{align*}
|\mathscr{G}_{I, J}|
&\le \frac{\big[\frac{\rs(I_2, J_2)}{\rd(I_2, J_2)} 
+ \frac{\rs(I_3, J_3)}{\rd(I_3, J_3)}\big]^{\delta_{2, 3}}}{D_{\theta}^{2, 3}(I, J)} 
\widetilde{F}_1(I_1, J_1) \frac{\rs(I_1, J_1)^{\frac12}}{\ird(I_1, J_1)^{\delta_0}}
\prod_{i=2}^3 F_i(I_i, J_i) \frac{\rs(I_i, J_i)^{\frac12}}{\rd(I_i, J_i)},
\end{align*}
where 
\begin{align*}
D_{\theta}^{2, 3}(I, J) 
:= \bigg(\frac{\rd(I_2, J_2)}{\rd(I_3, J_3)} + \frac{\rd(I_3, J_3)}{\rd(I_2, J_2)}\bigg)^{\theta}.
\end{align*}
\end{lemma}


\begin{proof}
In the current scenario, for each $i=2, 3$ and for all $x_i \in J_i$ and $y_i \in I_i$, we have 
\begin{align}\label{HH61}
|x_i - y_i| \simeq \d(I_i, J_i), \qquad 
\frac{|y_i - c_{I_i}|}{|x_i - y_i|} \lesssim \frac{\rs(I_i, J_i)}{\rd(I_i, J_i)}, 
\end{align}
and 
\begin{align}\label{HH62}
D_{\theta}(x-y) 
&= \Bigg(\frac{\frac{|x_1 - y_1|}{\ell(I_1)} \frac{|x_2 - y_2|}{\ell(I_2)}}{\frac{|x_3 - y_3|}{\ell(I_3)}} 
+ \frac{\frac{|x_3 - y_3|}{\ell(I_3)}}{\frac{|x_1 - y_1|}{\ell(I_1)} \frac{|x_2 - y_2|}{\ell(I_2)}} \Bigg)^{-\theta}
\nonumber \\
&\simeq \bigg(\frac{|x_1 - y_1|}{\ell(I_1)} \frac{\rd(I_2, J_2)}{\rd(I_3, J_3)}
+ \frac{\ell(I_1)}{|x_1 - y_1|} \frac{\rd(I_3, J_3)}{\rd(I_2, J_2)} \bigg)^{-\theta}. 
\end{align}
Write $t := \frac{\rd(I_2, J_2)}{\ell(I_1) \rd(I_3, J_3)}$. In light of \eqref{HH61}--\eqref{HH62}, we use the compact full kernel representation, the cancellation of $h_{J_{2, 3}}$, and the mixed size-H\"{o}lder condition of $K$ to deduce that 
\begin{align*}
|\mathscr{J}_1|
&\lesssim \int_{I_1 \times I_{2, 3}} \int_{(3I_1 \setminus I_1) \times J_{2, 3}} 
\bigg(\frac{|x_2 - c_{J_2}|}{|x_2 - y_2|} + \frac{|x_3 - c_{J_3}|}{|x_3 - y_3|}\bigg)^{\delta_{2, 3}} D_{\theta}(x-y) 
\\
&\quad\times \prod_{i=1}^3 \frac{F_i(x_i, y_i)}{|x_i - y_i|} 
|h_{I_1} \otimes h_{I_{2, 3}}(y)| |h_{J_1} \otimes h_{J_{2, 3}}(x)| \, dx \, dy
\\
&\lesssim \bigg[\frac{\rs(I_2, J_2)}{\rd(I_2, J_2)} + \frac{\rs(I_3, J_3)}{\rd(I_3, J_3)}\bigg]^{\delta_{2, 3}} 
\mathscr{Q}_1^{0, t}(I_1, J_1) \mathscr{P}_2(I_2, J_2) \mathscr{P}_3(I_3, J_3) \black 
\prod_{i=1}^3 |I_i|^{-\frac12} |J_i|^{-\frac12}
\\ 
&\lesssim \frac{\big[\frac{\rs(I_2, J_2)}{\rd(I_2, J_2)} + \frac{\rs(I_3, J_3)}{\rd(I_3, J_3)}\big]^{\delta_{2, 3}}}{D_{\theta}^{2, 3}(I, J)} 
\widetilde{F}_1(I_1, J_1) \frac{\rs(I_1, J_1)^{\frac12 - \frac1r}}{\ird(I_1, J_1)^{\delta_0}} 
\prod_{i=2}^3 F_i(I_i, J_i) \frac{\rs(I_i, J_i)^{\frac12}}{\rd(I_i, J_i)},
\end{align*}
where we have used \eqref{def:P1}, \eqref{def:Q0}, and that $\ird(I_1, J_1) \le 2$ (otherwise, $3I_1 \cap J_1 = \emptyset$). Similarly, the compact full kernel representation, the cancellation of $h_{I_1}$ and $h_{J_{2, 3}}$, the H\"{o}lder condition of $K$, \eqref{def:P1}, and \eqref{def:Q1} imply 
\begin{align*}
|\mathscr{J}_2|
&\lesssim \int_{I_1 \times I_{2, 3}} \int_{(J_1 \setminus 3I_1) \times J_{2, 3}} 
\bigg(\frac{|y_1 - c_{I_1}|}{|x_1 - y_1|}\bigg)^{\delta_1} 
\bigg(\frac{|x_2 - c_{J_2}|}{|x_2 - y_2|} + \frac{|x_3 - c_{J_3}|}{|x_3 - y_3|}\bigg)^{\delta_{2, 3}}  
\\
&\quad\times D_{\theta}(x-y) \prod_{i=1}^3 \frac{F_i(x_i, y_i)}{|x_i - y_i|} 
|h_{I_1} \otimes h_{I_{2, 3}}(y)| |h_{J_1} \otimes h_{J_{2, 3}}(x)| \, dx \, dy
\\
&\lesssim \bigg[\frac{\rs(I_2, J_2)}{\rd(I_2, J_2)} + \frac{\rs(I_3, J_3)}{\rd(I_3, J_3)}\bigg]^{\delta_{2, 3}} 
\mathscr{Q}_1^{1, t}(I_1, J_1) \mathscr{P}_2(I_2, J_2) \mathscr{P}_3(I_3, J_3) \black 
\prod_{i=1}^3 |I_i|^{-\frac12} |J_i|^{-\frac12}
\\ 
&\lesssim \frac{\big[\frac{\rs(I_2, J_2)}{\rd(I_2, J_2)} + \frac{\rs(I_3, J_3)}{\rd(I_3, J_3)}\big]^{\delta_{2, 3}}}{D_{\theta}^{2, 3}(I, J)} 
\widetilde{F}_1(I_1, J_1) \frac{\rs(I_1, J_1)^{\frac12}}{\ird(I_1, J_1)^{\delta_0}} \prod_{i=2}^3 F_i(I_i, J_i) \frac{\rs(I_i, J_i)^{\frac12}}{\rd(I_i, J_i)}. 
\end{align*}
These estimates along with \eqref{GJJ} give the inequality as desired.  
\end{proof}




It follows from Lemmas \ref{lem:HH-3}, \eqref{JDD}, and Lemma \ref{lem:FF} parts \eqref{list-FF1}--\eqref{list-FF2} that there exists $N_0>1$ so that for all $N \ge N_0$,
\begin{align*}
|\I^N_{4, 1}| 
\le \sum_{k_2 \ge k_3 \ge 0 \atop j_2, j_3 \ge 1} \sum_{1 \le m_1 \le 2^{k_2-k_3}} \sum_{J \not\in \D_{\Z}(2N)} 
\sum_{\substack{I  \in \D_{\Z} \\ I_1 \in J_1(k_2-k_3, 0, m_1) \\ I_2 \in J_2(-k_2, j_2) \\ I_3 \in J_3(-k_3, j_3)}} 
|\G_{I, J}| \, |f_I| |g_J|
\lesssim \sum_{i=0}^3 \I^{N, i}_{4, 1}, 
\end{align*}
where 
\begin{align*}
&\I^{N, 0}_{4, 1}
:= \varepsilon \sum_{k_2 \ge k_3 \ge 0 \atop j_2, j_3 \ge 1} 
\Gamma(k, j) \, 2^{-\frac{k_2-k_3}{2}} 2^{-\frac{k_2}{2}}  2^{-\frac{k_3}{2}}
\sum_{1 \le m_1 \le 2^{k_2-k_3}}  m_1^{-\delta_0}
\sum_{\substack{I, J  \in \D_{\Z} \\ I_1 \in J_1(k_2-k_3, 0, m_1) \\ I_2 \in J_2(-k_2, j_2) \\ I_3 \in J_3(-k_3, j_3)}} 
|f_I| |g_J|,
\\
&\I^{N, 1}_{4, 1}
:= \sum_{\substack{k_2 \ge k_3 \ge 0 \\ j_2, j_3 \ge 1 \\ k_2 \ge N}} 
\Gamma(k, j) \, 2^{-\frac{k_2-k_3}{2}} 2^{-\frac{k_2}{2}}  2^{-\frac{k_3}{2}}
\sum_{1 \le m_1 \le 2^{k_2-k_3}}  m_1^{-\delta_0}
\sum_{\substack{I, J  \in \D_{\Z} \\ I_1 \in J_1(k_2-k_3, 0, m_1) \\ I_2 \in J_2(-k_2, j_2) \\ I_3 \in J_3(-k_3, j_3)}} 
|f_I| |g_J|,
\\
&\I^{N, i}_{4, 1}
:= \sum_{\substack{k_2 \ge k_3 \ge 0 \\ j_2, j_3 \ge 1 \\ j_i \ge N^{1/8}}}  
\Gamma(k, j) \, 2^{-\frac{k_2-k_3}{2}} 2^{-\frac{k_2}{2}}  2^{-\frac{k_3}{2}}
\sum_{1 \le m_1 \le 2^{k_2-k_3}}  m_1^{-\delta_0}
\sum_{\substack{I, J  \in \D_{\Z} \\ I_1 \in J_1(k_2-k_3, 0, m_1) \\ I_2 \in J_2(-k_2, j_2) \\ I_3 \in J_3(-k_3, j_3)}} 
|f_I| |g_J|,
\end{align*}
for $i=2, 3$, with the term 
\begin{align*}
\Gamma(k, j) 
:= \frac{(2^{-k_2} j_2^{-1} + 2^{-k_3} j_3^{-1})^{\delta_{2, 3}}}{(j_2 j_3^{-1} + j_2^{-1} j_3)^{\theta}} j_2^{-1} j_3^{-1}. 
\end{align*}
Then, invoking the Cauchy--Schwarz inequality, \eqref{car-IJ}, \eqref{car-IJK}--\eqref{MDT}, and \eqref{JJA-1} applied to $A=1$ and $B=0$, we conclude 
\begin{align*}
\I^{N, 0}_{4, 1}
&\le \varepsilon \sum_{k_2 \ge k_3 \ge 0 \atop j_2, j_3 \ge 1} 
\Gamma(k, j) \, 2^{-\frac{k_2-k_3}{2}} 2^{-\frac{k_2}{2}}  2^{-\frac{k_3}{2}} 
\\
&\quad\times \bigg(\sum_{I \in \D_{\Z}} \sum_{1 \le m_1 \le 2^{k_1-k_3}}  
\sum_{\substack{J  \in \D_{\Z} \\ J_1 \in I_1(k_2-k_3, 0, m_1) \\ J_2 \in I_2(-k_2, j_2) \\ J_3 \in I_3(-k_3, j_3)}} 
|f_I|^2 \bigg)^{\frac12}
\\
&\quad\times \bigg(\sum_{J \in \D_{\Z}} \sum_{1 \le m_1 \le 2^{k_1-k_3}}  m_1^{-2\delta_0} 
\sum_{\substack{I  \in \D_{\Z} \\ I_1 \in J_1(k_2-k_3, 0, m_1) \\ I_2 \in J_2(-k_2, j_2) \\ I_3 \in J_3(-k_3, j_3)}} 
|g_J|^2 \bigg)^{\frac12}
\\
&\lesssim \varepsilon \sum_{k_2 \ge k_3 \ge 0 \atop j_2, j_3 \ge 1} \Gamma(k, j)
\bigg(\sum_{I \in \D_{\Z}} |f_I|^2 \bigg)^{\frac12}
\\
&\quad\times \bigg(2^{-(k_2-k_3)} \sum_{1 \le m_1 \le 2^{k_1-k_3}}  m_1^{-2\delta_0} \bigg)^{\frac12}
\bigg(\sum_{J \in \D_{\Z}} |g_J|^2 \bigg)^{\frac12}
\\
&\lesssim \varepsilon \sum_{k_2 \ge k_3 \ge 0 \atop j_2, j_3 \ge 1} 
2^{-(k_2-k_3) \delta_0 \beta} \Gamma(k, j) 
\|f\|_{L^2(\R^3)} \|g\|_{L^2(\R^3)} 
\\
&\lesssim \varepsilon \|f\|_{L^2(\R^3)} \|g\|_{L^2(\R^3)}. 
\end{align*}
In the same way, one has 
\begin{align*}
\I^{N, 1}_{4, 1}
&\lesssim \varepsilon \sum_{\substack{k_2 \ge k_3 \ge 0 \\ j_2, j_3 \ge 1 \\ k_2 \ge N}} 
2^{-(k_2-k_3) \delta_0 \beta} \Gamma(k, j) 
\|f\|_{L^2(\R^3)} \|g\|_{L^2(\R^3)} 
\\
&\lesssim 2^{-N\delta_0 \beta} N \|f\|_{L^2(\R^3)} \|g\|_{L^2(\R^3)} 
\le \varepsilon \|f\|_{L^2(\R^3)} \|g\|_{L^2(\R^3)}. 
\end{align*}
and for each $i=2, 3$, 
\begin{align*}
\I^{N, i}_{4, 1}
&\lesssim \varepsilon \sum_{\substack{k_2 \ge k_3 \ge 0 \\ j_2, j_3 \ge 1 \\ j_i \ge N^{1/8}}} 
2^{-(k_2-k_3) \delta_0 \beta} \Gamma(k, j) 
\|f\|_{L^2(\R^3)} \|g\|_{L^2(\R^3)} 
\\
&\lesssim N^{-\delta_0/8} \|f\|_{L^2(\R^3)} \|g\|_{L^2(\R^3)} 
\le \varepsilon \|f\|_{L^2(\R^3)} \|g\|_{L^2(\R^3)}. 
\end{align*}
Now gathering the estimates above, we conclude that 
\begin{align}\label{SN41}
|\I^N_{4, 1}| 
\lesssim \sum_{i=0}^3 \I^{N, i}_{4, 1}
\lesssim \varepsilon \|f\|_{L^2(\R^3)} \|g\|_{L^2(\R^3)}. 
\end{align}


\subsubsection{$\rd(I_1, J_1) < 1$, $I_1 \cap J_1 = \emptyset$, $\rd(I_2, J_2) \ge 1$, and $\rd(I_3, J_3) <1$} \label{sec:NSN}


\begin{lemma}\label{lem:HH-7}
Let $I, J \in \D_{\Z}$ with $\ell(I_1) \le \ell(J_1)$, $\ell(I_2) \ge \ell(J_2)$, and $\ell(I_3) \ge \ell(J_3)$ so that $\rd(I_1, J_1) < 1$, $I_1 \cap J_1 = \emptyset$, $\rd(I_2, J_2) \ge 1$, and $\rd(I_3, J_3) < 1$. Then
\begin{align*}
|\mathscr{G}_{I, J}|
&\lesssim \widetilde{F}_1(I_1) \frac{\rs(I_1, J_1)^{\frac12}}{\ird(I_1, J_1)^{\delta_0}} 
F_2(I_2, J_2) \frac{\rs(I_2, J_2)^{\frac12}}{\rd(I_2, J_2)^{1+\theta}} 
\widetilde{F}_3(J_3) \rs(I_3, J_3)^{\frac12 - \frac1r + \theta}. 
\end{align*}
\end{lemma}


\begin{proof}
Note that for any $x_2 \in J_2$ and $y_2 \in I_2$, 
\begin{align}\label{HH7}
|x_2 - y_2| \simeq \d(I_2, J_2) =: t 
\quad\text{ and }\quad 
D_{\theta}(x-y) 
\simeq \bigg(\frac{t |x_1 - y_1|}{|x_3 - y_3|} 
+ \frac{|x_3 - y_3|}{t |x_1 - y_1|} \bigg)^{-\theta}. 
\end{align}
Then in view of the compact full kernel representation and the size condition of $K$, the estimate \eqref{HH7} yields 
\begin{align*}
|\mathscr{J}_1|
&\lesssim \mathscr{P}_2(I_2, J_2) 
\mathscr{R}_{1, 3}^{0, 0, t}(I_{1, 3}, J_{1, 3})  
\prod_{i=1}^3 |I_i|^{-\frac12} |J_i|^{-\frac12}
\\ 
&\lesssim \widetilde{F}_1(I_1) \rs(I_1, J_1)^{\frac12}
F_2(I_2, J_2) \frac{\rs(I_2, J_2)^{\frac12}}{\rd(I_2, J_2)^{1+\theta}} 
\widetilde{F}_3(I_3, J_3) \rs(I_3, J_3)^{\frac12 - \frac1r + \theta}, 
\end{align*}
where we have used \eqref{def:P1}, \eqref{def:R1300}, and that $\frac{t|I_1|}{|J_3|} = \frac{\rd(I_2, J_2)}{\rs(I_3, J_3)}$.  Moreover,  the compact full kernel representation and the H\"{o}lder condition of $K$ imply 
\begin{align*}
|\mathscr{J}_2|
&\lesssim \int_{I_1 \times I_{2, 3}} \int_{(J_1 \setminus 3I_1) \times J_{2, 3}} 
\bigg(\frac{|y_1 - c_{I_1}|}{|x_1 - y_1|}\bigg)^{\delta_1} D_{\theta}(x-y) 
\prod_{i=1}^3 \frac{F_i(x_i, y_i)}{|x_i - y_i|} |I_i|^{-\frac12} |J_i|^{-\frac12} \, dx \, dy
\\
&\lesssim \mathscr{P}_2(I_2, J_2) 
\mathscr{R}_{1, 3}^{1, 0, t}(I_{1, 3}, J_{1, 3})  
\prod_{i=1}^3 |I_i|^{-\frac12} |J_i|^{-\frac12}
\\ 
&\lesssim \widetilde{F}_1(I_1, J_1) \frac{\rs(I_1, J_1)^{\frac12}}{\ird(I_1, J_1)^{\delta_0}}
F_2(I_2, J_2) \frac{\rs(I_2, J_2)^{\frac12}}{\rd(I_2, J_2)^{1+\theta}} 
\widetilde{F}_3(I_3, J_3) \rs(I_3, J_3)^{\frac12 - \frac1r + \theta}. 
\end{align*}
The desired estimate follows from \eqref{GJJ}. 
\end{proof}




By Lemmas \ref{lem:HH-3},  \eqref{JDD}, and Lemma \ref{lem:FF} parts \eqref{list-FF1}--\eqref{list-FF2}, there exists $N_0>1$ so that for all $N \ge N_0$,
\begin{align*}
|\I^N_{4, 2}| 
\le \sum_{k_2 \ge k_1 \ge 0 \atop j_2 \ge 1} \sum_{1 \le m_1 \le 2^{k_1}} 
\sum_{J \not\in \D_{\Z}(2N)} 
\sum_{\substack{I \in \D_{\Z} \\ I_1 \in J_1(k_1, 0, m_1) \\ I_2 \in J_2(-k_2, j_2) \\ \rd(I_3, J_3) < 1}}
|\G_{I, J}| \, |f_I| |g_J|
\lesssim \sum_{i=1}^3 \I^{N, i}_{4, 2},
\end{align*}
where
\begin{align*}
\I^{N, 1}_{4, 2}
&:= \varepsilon \sum_{k_2 \ge k_1 \ge 0 \atop j_2 \ge 1} \sum_{1 \le m_1 \le 2^{k_1}} 
2^{-\frac{k_1}{2}} m_1^{-\delta_0} 2^{-\frac{k_2}{2}} j_2^{-1-\theta} 2^{-(k_2-k_1)(\frac12-\frac1r+\theta)}
\sum_{\substack{I, J \in \D_{\Z} \\ I_1 \in J_1(k_1, 0, m_1) \\ I_2 \in J_2(-k_2, j_2) \\ \rd(I_3, J_3) < 1}}
|f_I| |g_J|,
\\
\I^{N, 2}_{4, 2}
&:= \sum_{k_2 \ge k_1 \ge N \atop j_2 \ge 1} \sum_{1 \le m_1 \le 2^{k_1}} 
2^{-\frac{k_1}{2}} m_1^{-\delta_0} 2^{-\frac{k_2}{2}} j_2^{-1-\theta} 2^{-(k_2-k_1)(\frac12-\frac1r+\theta)}
\sum_{\substack{I, J \in \D_{\Z} \\ I_1 \in J_1(k_1, 0, m_1) \\ I_2 \in J_2(-k_2, j_2) \\ \rd(I_3, J_3) < 1}}
|f_I| |g_J|, 
\\
\I^{N, 3}_{4, 2}
&:= \sum_{k_2 \ge k_1 \ge 0 \atop j_2 \ge N^{1/8}} \sum_{1 \le m_1 \le 2^{k_1}} 
2^{-\frac{k_1}{2}} m_1^{-\delta_0} 2^{-\frac{k_2}{2}} j_2^{-1-\theta} 2^{-(k_2-k_1)(\frac12-\frac1r+\theta)}
\sum_{\substack{I, J \in \D_{\Z} \\ I_1 \in J_1(k_1, 0, m_1) \\ I_2 \in J_2(-k_2, j_2) \\ \rd(I_3, J_3) < 1}}
|f_I| |g_J|. 
\end{align*}
Then, it follows from the Cauchy--Schwarz inequality, \eqref{car-IJ}, and \eqref{car-IJK}--\eqref{MDT} that
\begin{align*}
\I^{N, 1}_{4, 2}
&\le \varepsilon \sum_{k_2 \ge k_1 \ge 0 \atop j_2 \ge 1} 
2^{-\frac{k_1}{2}} 2^{-\frac{k_2}{2}} j_2^{-1-\theta} 2^{-(k_2-k_1)(\frac12-\frac1r+\theta)}
\\
&\quad\times \bigg(\sum_{I \in \D_{\Z}} \sum_{1 \le m_1 \le 2^{k_1}}  
\sum_{\substack{J \in \D_{\Z} \\ J_1 \in I_1(k_1, 0, m_1) \\ J_2 \in I_2(-k_2, j_2) \\ \rd(I_3, J_3) < 1}}
|f_I|^2 \bigg)^{\frac12}
\\
&\quad\times \bigg(\sum_{J \in \D_{\Z}} \sum_{1 \le m_1 \le 2^{k_1}} m_1^{-2\delta_0}
\sum_{\substack{I \in \D_{\Z} \\ I_1 \in J_1(k_1, 0, m_1) \\ I_2 \in J_2(-k_2, j_2) \\ \rd(I_3, J_3) < 1}}
|g_J|^2 \bigg)^{\frac12}
\\ 
&\lesssim \varepsilon \sum_{k_2 \ge k_1 \ge 0 \atop j_2 \ge 1} 
j_2^{-1-\theta} 2^{-(k_2-k_1)(\theta-\frac1r)}
\bigg(2^{-k_1} \sum_{1 \le m_1 \le 2^{k_1}} m_1^{-2\delta_0} \bigg)^{\frac12}
\\
&\quad\times \bigg(\sum_{I \in \D_{\Z}} |f_I|^2 \bigg)^{\frac12}
\bigg(\sum_{J \in \D_{\Z}} |g_J|^2 \bigg)^{\frac12}
\\
&\lesssim \varepsilon \sum_{k_2 \ge k_1 \ge 0 \atop j_2 \ge 1} 
j_2^{-1-\theta} 2^{-k_1(\delta_0 \beta - \frac1r)} 2^{-(k_2-k_1) \theta}
\|f\|_{L^2(\R^3)} \|g\|_{L^2(\R^3)}
\\
&\lesssim \varepsilon 
\|f\|_{L^2(\R^3)} \|g\|_{L^2(\R^3)}. 
\end{align*}
Similarly, 
\begin{align*}
\I^{N, 2}_{4, 2}
&\lesssim \sum_{k_2 \ge k_1 \ge N \atop j_2 \ge 1} 
j_2^{-1-\theta} 2^{-k_1(\delta_0 \beta - \frac1r)} 2^{-(k_2-k_1) \theta}
\|f\|_{L^2(\R^3)} \|g\|_{L^2(\R^3)}
\\
&\lesssim 2^{-N(\delta_0 \beta - \frac1r)} \|f\|_{L^2(\R^3)} \|g\|_{L^2(\R^3)}
\le \varepsilon \|f\|_{L^2(\R^3)} \|g\|_{L^2(\R^3)},  
\end{align*}
and 
\begin{align*}
\I^{N, 3}_{4, 2}
&\lesssim \sum_{k_2 \ge k_1 \ge 0 \atop j_2 \ge N^{1/8}} 
j_2^{-1-\theta} 2^{-k_1(\delta_0 \beta - \frac1r)} 2^{-(k_2-k_1) \theta}
\|f\|_{L^2(\R^3)} \|g\|_{L^2(\R^3)}
\\
&\lesssim N^{-\theta/8} \|f\|_{L^2(\R^3)} \|g\|_{L^2(\R^3)} 
\le \varepsilon \|f\|_{L^2(\R^3)} \|g\|_{L^2(\R^3)}. 
\end{align*}
Thus, there holds 
\begin{align}\label{SN42}
|\I^N_{4, 2}| 
\lesssim \sum_{i=1}^3 \I^{N, i}_{4, 2} 
\lesssim \varepsilon \|f\|_{L^2(\R^3)} \|g\|_{L^2(\R^3)}. 
\end{align}


\subsubsection{$\rd(I_1, J_1) < 1$, $I_1 \cap J_1 = \emptyset$, $\rd(I_2, J_2) < 1$, and $\rd(I_3, J_3) \ge 1$} 


\begin{lemma}\label{lem:HH-8}
Let $I, J \in \D_{\Z}$ with $\ell(I_1) \le \ell(J_1)$, $\ell(I_2) \ge \ell(J_2)$, and $\ell(I_3) \ge \ell(J_3)$ so that $\rd(I_1, J_1) < 1$, $I_1 \cap J_1 = \emptyset$, $\rd(I_2, J_2) < 1$, and $\rd(I_3, J_3) \ge 1$. Then
\begin{align*}
|\mathscr{G}_{I, J}| 
&\lesssim \widetilde{F}_1(I_1) \frac{\rs(I_1, J_1)^{\frac12}}{\ird(I_1, J_1)^{\delta_0}}
\widetilde{F}_2(I_2, J_2) \rs(I_2, J_2)^{\frac12 - \frac1r + \theta} 
F_3(I_3, J_3) \frac{\rs(I_3, J_3)^{\frac12}}{\rd(I_3, J_3)^{1+\theta}}. 
\end{align*}
\end{lemma}


\begin{proof}
The condition $\rd(I_3, J_3) \ge 1$ implies that for all $x_3 \in J_3$ and $y_3 \in I_3$, 
\begin{align*}
|x_3 - y_3| \simeq \d(I_3, J_3) =: t 
\quad\text{ and }\quad
D_{\theta}(x-y) 
\simeq D_{\theta}(x_{1, 2} - y_{1, 2}, t) . 
\end{align*}
The proof can be given as in Section \ref{sec:NSN}, where $t_3$ and $t_1$ are respectively replaced by $t_2 := \frac{|x_1 - y_1|}{t}$ and $t_1 := \frac{|J_2|}{t}$. In this scenario, $t_1 |I_1| = \frac{\rs(I_2, J_2)}{\rd(I_3, J_3)}$. Details are left to the reader. 
\end{proof}


Following the estimates for $\I^N_{4, 2}$, we use Lemma \ref{lem:HH-8} to deduce that for any $N>1$ sufficiently large, 
\begin{align*}
|\I^N_{4, 3}| 
\lesssim \varepsilon \|f\|_{L^2(\R^3)} \|g\|_{L^2(\R^3)}, 
\end{align*}
which along with \eqref{SN41} and \eqref{SN42} gives 
\begin{align*}
|\I^N_4|
\lesssim \sum_{i=1}^3 |\I^N_{4, i}| 
\lesssim \varepsilon \|f\|_{L^2(\R^3)} \|g\|_{L^2(\R^3)}. 
\end{align*}


\subsection{$\rd(I_1, J_1) < 1$, $I_1 \cap J_1 = \emptyset$, $\rd(I_{2, 3}, J_{2, 3}) < 1$, and $I_{2, 3} \cap J_{2, 3} = \emptyset$}\label{sec:near-near} 

\begin{lemma}\label{lem:HH-9}
Let $I, J \in \D_{\Z}$ with $\ell(I_1) \le \ell(J_1)$, $\ell(I_2) \ge \ell(J_2)$, and $\ell(I_3) \ge \ell(J_3)$ so that $\rd(I_1, J_1) < 1$, $I_1 \cap J_1 = \emptyset$, $\rd(I_{2, 3}, J_{2, 3}) < 1$, and $I_{2, 3} \cap J_{2, 3} = \emptyset$. Then
\begin{align*}
|\mathscr{G}_{I, J}|
&\lesssim \prod_{i=1}^3 \widetilde{F}_i(I_i, J_i) 
\frac{\rs(I_i, J_i)^{\frac12}}{\ird(I_i, J_i)^{\delta_0}}. 
\end{align*}
\end{lemma}

\begin{proof} 
Observing that $h_{J_1} = h_{J_1} \mathbf{1}_{3I_1} + h_{J_1} \mathbf{1}_{(3I_1)^c}$, we further split each $\mathscr{H}_i$ in \eqref{GHH} into two terms $\mathscr{H}_{i, 1}$ and $\mathscr{H}_{i, 2}$. By symmetry, it suffices to estimate $\mathscr{H}_{i, j}$ for $i=1, 2, 4$ and $j=1, 2$. 

Recall that $\ell(I_1) \le \ell(J_1)$ and $\ell(I_i) \ge \ell(J_i)$ for $i=2, 3$. Then, $\ird(I_1, J_1)>2$ implies $3I_1 \cap J_1 = \emptyset$, and $\ird(I_i, J_i)>2$ implies $I_i \cap 3 J_i = \emptyset$ for $i=2, 3$. This leads the corresponding terms vanish. Hence, in this sequel, for those terms, we will implicitly use the condition $\ird(I_i, J_i) \le 2$ for some (or for all) $i=1, 2, 3$. 

Considering $I_1 \cap J_1 = \emptyset$ and $I_{2, 3} \cap J_{2, 3} = \emptyset$, we will always utilize the compact full kernel representation. In view of the size condition of $K$, \eqref{def:S000} and \eqref{33IJ} imply 
\begin{align*}
|\mathscr{H}_{1, 1}| 
&\lesssim \sum_{i, j, k=0}^2 \mathscr{S}_{0, 0, 0}(I_1 \times J_2^j \times J_3^k, I_1^i \times J_2 \times J_3) 
\prod_{i=1}^3 |I_i|^{-\frac12} |J_i|^{-\frac12}
\\ 
&\lesssim \widetilde{F}_1(I_1) \rs(I_1, J_1)^{\frac12} \, 
\widetilde{F}_2(J_2) \rs(I_2, J_2)^{\frac12} \, 
\widetilde{F}_3(J_3) \rs(I_3, J_3)^{\frac12}
\\ 
&\lesssim \widetilde{F}_1(I_1) \frac{\rs(I_1, J_1)^{\frac12}}{\ird(I_1, J_1)^{\delta_0}} \, 
\widetilde{F}_2(J_2) \frac{\rs(I_2, J_2)^{\frac12}}{\ird(I_2, J_2)^{\delta_0}} \, 
\widetilde{F}_3(J_3) \frac{\rs(I_3, J_3)^{\frac12}}{\ird(I_2, J_3)^{\delta_0}}, 
\end{align*}
provided that $\ird(I_i, J_i) \le 2$ for all $i=1, 2, 3$, and \eqref{def:S001} gives 
\begin{align*}
|\mathscr{H}_{2, 1}| 
&\lesssim \mathscr{S}_{0, 0, 1}(I, J) \prod_{i=1}^3 |I_i|^{-\frac12} |J_i|^{-\frac12}
\\ 
&\lesssim \widetilde{F}_1(I_1) \rs(I_1, J_1)^{\frac12} \, \widetilde{F}_2(J_2) \rs(I_2, J_2)^{\frac12} \, 
\widetilde{F}_3(I_3, J_3) \frac{\rs(I_3, J_3)^{\frac12}}{\ird(I_3, J_3)^{\theta}} 
\\ 
&\lesssim \widetilde{F}_1(I_1) \frac{\rs(I_1, J_1)^{\frac12}}{\ird(I_1, J_1)^{\delta_0}} \, 
\widetilde{F}_2(J_2) \frac{\rs(I_2, J_2)^{\frac12}}{\ird(I_2, J_2)^{\delta_0}} \, 
\widetilde{F}_3(J_3) \frac{\rs(I_3, J_3)^{\frac12}}{\ird(I_2, J_3)^{\delta_0}}, 
\end{align*}
provided that $\ird(I_i, J_i) \le 2$ for each $i=1, 2$. To control $\mathscr{H}_{4, 1}$, we apply the cancellation of $h_{J_{2, 3}}$, the mixed size-H\"{o}lder condition of $K$, and \eqref{def:S011} to deduce 
\begin{align*}
|\mathscr{H}_{4, 1}| 
&\lesssim \mathscr{S}_{0, 1, 1}(I, J) \prod_{i=1}^3 |I_i|^{-\frac12} |J_i|^{-\frac12} 
\\
&\lesssim \widetilde{F}_1(I_1) \rs(I_1, J_1)^{\frac12 - \theta} \, 
\widetilde{F}_2(I_2, J_2) \frac{\rs(I_2, J_2)^{\frac12}}{\ird(I_2, J_2)^{\delta_0}}  \, 
\widetilde{F}_3(I_3, J_3) \frac{\rs(I_3, J_3)^{\frac12}}{\ird(I_3, J_3)^{\delta_0}}
\end{align*}

Moreover, by the cancellation of $h_{I_1}$ and the mixed size-H\"{o}lder condition of $K$, an estimate similar to \eqref{def:S001} gives 
\begin{align*}
|\mathscr{H}_{1, 2}| 
&\lesssim \int_{I_1 \times  3J_2 \times 3J_3} \int_{(J_1 \setminus 3I_1) \times J_2 \times J_3} 
\bigg(\frac{|x_1 -c_{I_1}|}{|x_1 - y_1|}\bigg)^{\delta_1} D_{\theta}(x-y) 
\\
&\quad\times \prod_{i=1}^3 \frac{F_i(x_i, y_i)}{|x_i - y_i|} |I_i|^{-\frac12} |J_i|^{-\frac12} \, dx \, dy
\\
&\lesssim \widetilde{F}_1(I_1) \frac{\rs(I_1, J_1)^{\frac12}}{\ird(I_1, J_1)^{\delta_0}} 
\widetilde{F}_2(I_2, J_2) \frac{\rs(I_2, J_2)^{\frac12}}{\ird(I_2, J_2)^{\delta_0}}  
\widetilde{F}_3(I_3, J_3) \frac{\rs(I_3, J_3)^{\frac12}}{\ird(I_3, J_3)^{\delta_0}}, 
\end{align*}
provided that $\ird(I_i, J_i) \le 2$ for each $i=2, 3$, and an estimate similar to \eqref{def:S011} yields 
\begin{align*}
|\mathscr{H}_{2, 2}| 
&\lesssim \int_{I_1 \times (3J_2) \times (I_3 \setminus 3J_3)} \int_{(J_1 \setminus 3I_1) \times J_2 \times J_3} 
\bigg(\frac{|x_1 -c_{I_1}|}{|x_1 - y_1|}\bigg)^{\delta_1} D_{\theta}(x-y) 
\\
&\quad\times \prod_{i=1}^3 \frac{F_i(x_i, y_i)}{|x_i - y_i|} |I_i|^{-\frac12} |J_i|^{-\frac12} \, dx \, dy
\\
&\lesssim \widetilde{F}_1(I_1) \frac{\rs(I_1, J_1)^{\frac12}}{\ird(I_1, J_1)^{\delta_0}} 
\widetilde{F}_2(I_2, J_2) \frac{\rs(I_2, J_2)^{\frac12}}{\ird(I_2, J_2)^{\delta_0}}  
\widetilde{F}_3(I_3, J_3) \frac{\rs(I_3, J_3)^{\frac12}}{\ird(I_3, J_3)^{\delta_0}}, 
\end{align*}
where $\rd(I_2, J_2) \le 2$ was used. For $\mathscr{H}_{4, 2}$, using the cancellation of $h_{I_1}$ and $h_{J_{2, 3}}$, the H\"{o}lder condition of $K$, and \eqref{def:S111},  one has 
\begin{align*}
|\mathscr{H}_{4, 2}| 
&\lesssim \mathscr{S}_{1, 1, 1}(I, J) \prod_{i=1}^3 |I_i|^{-\frac12} |J_i|^{-\frac12} 
\\
&\lesssim \widetilde{F}_1(I_1, J_1) \frac{\rs(I_1, J_1)^{\frac12}}{\ird(I_1, J_1)^{\delta_0}} 
\widetilde{F}_2(I_2, J_2) \frac{\rs(I_2, J_2)^{\frac12}}{\ird(I_2, J_2)^{\delta_0}}  
\widetilde{F}_3(I_3, J_3) \frac{\rs(I_3, J_3)^{\frac12}}{\ird(I_3, J_3)^{\delta_0}} 
\bigg(\frac{|J_1|}{|I_1|}\bigg)^{\theta}. 
\end{align*}
This completes the proof. 
\end{proof}



Invoking Lemma \ref{lem:HH-9}, \eqref{JDD}, and Lemma \ref{lem:FF} part \eqref{list-FF2}, we conclude  
\begin{align*}
|\I^N_5| 
\le \sum_{k_2 \ge k_1 \ge 0} 
\sum_{\substack{1 \le m_1 \le 2^{k_1} \\ 1 \le m_2 \le 2^{k_2} \\ 1 \le m_3 \le 2^{k_2-k_1}}} 
\sum_{J \not\in \D_{\Z}(2N)} \sum_{\substack{I \in \D_{\Z} \\ 
I_1 \in J_1(k_1, 0, m_1) \\ I_2 \in J_2(-k_2, 0, m_2) \\ I_3 \in J_3(k_1-k_2, 0, m_3)}} 
|\G_{I, J}| \, |f_I| \, |g_J| 
\lesssim \sum_{i=0}^2 \I^N_{5, i}, 
\end{align*}
where 
\begin{align*}
\I^N_{5, 0}
&:= \varepsilon \sum_{k_2 \ge k_1 \ge 0} 
\sum_{\substack{1 \le m_1 \le 2^{k_1} \\ 1 \le m_2 \le 2^{k_2} \\ 1 \le m_3 \le 2^{k_2-k_1}}} 
\frac{2^{-\frac{k_1}{2}} 2^{-\frac{k_2}{2}} 2^{-\frac{k_2-k_1}{2}}}{m_1^{\delta_0} \, m_2^{\delta_0} \, m_3^{\delta_0}}
\sum_{\substack{I, J \in \D_{\Z} \\ I_1 \in J_1(k_1, 0, m_1) \\ I_2 \in J_2(-k_2, 0, m_2) \\ I_3 \in J_3(k_1-k_2, 0, m_3)}} 
|f_I| \, |g_J|,  
\\
\I^N_{5, i}
&:= \sum_{k_2 \ge k_1 \ge 0 \atop k_i \ge N} 
\sum_{\substack{1 \le m_1 \le 2^{k_1} \\ 1 \le m_2 \le 2^{k_2} \\ 1 \le m_3 \le 2^{k_2-k_1}}} 
\frac{2^{-\frac{k_1}{2}} 2^{-\frac{k_2}{2}} 2^{-\frac{k_2-k_1}{2}}}{m_1^{\delta_0} \, m_2^{\delta_0} \, m_3^{\delta_0}}
\sum_{\substack{I, J \in \D_{\Z} \\ I_1 \in J_1(k_1, 0, m_1) \\ I_2 \in J_2(-k_2, 0, m_2) \\ I_3 \in J_3(k_1-k_2, 0, m_3)}} 
|f_I| \, |g_J|, 
\end{align*}
for $i=1, 2$. The Cauchy--Schwarz inequality and \eqref{car-IJK}--\eqref{kaka} yield  
\begin{align*}
\I^N_{5, 0}
&\le \varepsilon \sum_{k_2 \ge k_1 \ge 0} 
2^{-\frac{k_1}{2}} \, 2^{-\frac{k_2}{2}} \, 2^{-\frac{k_2-k_1}{2}}
\\
&\quad\times \Bigg(\sum_{I \in \D_{\Z}} 
\sum_{\substack{1 \le m_1 \le 2^{k_1} \\ 1 \le m_2 \le 2^{k_2} \\ 1 \le m_3 \le 2^{k_2-k_1}}} m_1^{-2\delta_0}
\sum_{\substack{J \in \D_{\Z} \\ J_1 \in I_1(k_1, 0, m_1) \\ J_2 \in I_2(-k_2, 0, m_2) \\ J_3 \in I_3(k_1-k_2, 0, m_3)}} 
|f_I|^2 \Bigg)^{\frac12}
\\
&\quad\times \Bigg(\sum_{J \in \D_{\Z}} 
\sum_{\substack{1 \le m_1 \le 2^{k_1} \\ 1 \le m_2 \le 2^{k_2} \\ 1 \le m_3 \le 2^{k_2-k_1}}}  m_2^{-2\delta_0} m_3^{-2\delta_0}
\sum_{\substack{I \in \D_{\Z} \\ I_1 \in J_1(k_1, 0, m_1) \\ I_2 \in J_2(-k_2, 0, m_2) \\ I_3 \in J_3(k_1-k_2, 0, m_3)}} 
|g_J|^2 \Bigg)^{\frac12}
\\
&\lesssim \varepsilon \sum_{k_2 \ge k_1 \ge 0} 
\bigg(2^{-k_1} \sum_{1 \le m_1 \le 2^{k_1}} m_1^{-2\delta_0} \bigg)^{\frac12} 
\bigg(2^{-k_2} \sum_{1 \le m_2 \le 2^{k_2}} m_2^{-2\delta_0} \bigg)^{\frac12} 
\\
&\quad\times \bigg(2^{-(k_2-k_1)} \sum_{1 \le m_3 \le 2^{k_2-k_1}} m_3^{-2\delta_0} \bigg)^{\frac12} 
\bigg(\sum_{I \in \D_{\Z}} |f_I|^2 \bigg)^{\frac12}
\bigg(\sum_{J \in \D_{\Z}} |g_J|^2 \bigg)^{\frac12}
\\
&\lesssim \varepsilon \sum_{k_2 \ge k_1 \ge 0} 
2^{-k_1 \delta_0 \beta} \, 2^{-k_2 \delta_0 \beta} \, 2^{-(k_2-k_1) \delta_0 \beta} \, 
\|f\|_{L^2(\R^3)} \|g\|_{L^2(\R^3)}
\\
&\lesssim \varepsilon \|f\|_{L^2(\R^3)} \|g\|_{L^2(\R^3)}. 
\end{align*}
Similarly, for each $i=1, 2$,  
\begin{align*}
\I^N_{5, i}
&\lesssim \sum_{k_2 \ge k_1 \ge 0 \atop k_i \ge N} 
2^{-k_1 \delta_0 \beta} \, 2^{-k_2 \delta_0 \beta} \, 2^{-(k_2-k_1) \delta_0 \beta} \, 
\|f\|_{L^2(\R^3)} \|g\|_{L^2(\R^3)}
\\
&\lesssim 2^{-N \delta_0 \beta} \|f\|_{L^2(\R^3)} \|g\|_{L^2(\R^3)}
\le \varepsilon \|f\|_{L^2(\R^3)} \|g\|_{L^2(\R^3)}, 
\end{align*}
provided $N > 1$ sufficiently large. As a consequence, all the estimates above imply 
\begin{align*}
|\I^N_5| 
\lesssim \sum_{i=0}^2 \I^N_{5, i}
\lesssim \varepsilon \|f\|_{L^2(\R^3)} \|g\|_{L^2(\R^3)}. 
\end{align*}




\subsection{$\rd(I_1, J_1) < 1$, $I_1 \cap J_1 = \emptyset$, and $I_{2, 3} \supset J_{2, 3}$}\label{sec:near-inside} 




\begin{lemma}\label{lem:HH-10}
Let $I, J \in \D_{\Z}$ with $\ell(I_1) \le \ell(J_1)$, $\ell(I_2) \ge \ell(J_2)$, and $\ell(I_3) \ge \ell(J_3)$ so that $\rd(I_1, J_1) < 1$ and $I_1 \cap J_1 = \emptyset$. Then
\begin{align}\label{HH10}
|\G_{I, J}|
\lesssim \widetilde{F}_1(I_1, J_1) \frac{\rs(I_1, J_1)^{\frac12}}{\ird(I_1, J_1)^{\delta_0}}
\widehat{F}_2(J_2) \rs(I_2, J_2)^{\frac12} \, \widehat{F}_3(J_3) \rs(I_3, J_3)^{\frac12}. 
\end{align}
\end{lemma}


\begin{proof}
First, we consider the case $I_{2, 3} \supsetneq J_{2, 3}$. By the fact that $h_{J_1} = h_{J_1} \mathbf{1}_{3I_1} + h_{J_1} \mathbf{1}_{(3I_1)^c}$, each $\mathscr{L}_i$ in \eqref{GLL} can split into two terms $\mathscr{L}_{i, 1}$ and $\mathscr{L}_{i, 2}$. As argued there, it suffices to estimate $\mathscr{L}_{i, j}$ for $i=1, 2, 3, 5, 9$ and $j=1, 2$. We still use the notation in \eqref{def:HIJ}. The compact partial kernel representation, the size condition of $K_{\widetilde{h}_{I_{2, 3}}, \widetilde{h}_{J_{2, 3}}}$, \eqref{def:P2}, and $\ird(I_1, J_1) \le 2$ give 
\begin{align*}
|\mathscr{L}_{1, 1}|
&\lesssim \mathscr{P}_1(I_1, 3I_1 \setminus I_1) \, 
C(\widetilde{h}_{I_{2, 3}}, \widetilde{h}_{J_{2, 3}})  
\prod_{i=1}^3 |I_i|^{-\frac12} |J_i|^{-\frac12}
\\
&\lesssim \widetilde{F}_1(I_1, J_1) |I_1| \, F_2(I_2) |I_2| \, F_3(J_3) |J_3| \, 
\prod_{i=1}^3 |I_i|^{-\frac12} |J_i|^{-\frac12}
\\
&\lesssim \widetilde{F}_1(I_1, J_1) \frac{\rs(I_1, J_1)^{\frac12}}{\ird(I_1, J_1)^{\delta_0}} 
F_2(I_2) \rs(I_2, J_2)^{\frac12} \, F_3(J_3) \rs(I_3, J_3)^{\frac12}.  
\end{align*}
The compact partial kernel representation, the H\"{o}lder condition of $K_{\widetilde{h}_{I_{2, 3}}, \widetilde{h}_{J_{2, 3}}}$, and \eqref{def:Q1} applied to $\theta=0$ imply
\begin{align*}
|\mathscr{L}_{1, 2}|
&\lesssim \mathscr{Q}_1^{1, t}(I_1, J_1) \, 
C(\widetilde{h}_{I_{2, 3}}, \widetilde{h}_{J_{2, 3}}) 
\prod_{i=1}^3 |I_i|^{-\frac12} |J_i|^{-\frac12}
\\
&\lesssim \widetilde{F}_1(I_1, J_1) \frac{|I_1|}{\ird(I_1, J_1)^{\delta_1}} F_2(I_2) |I_2| \, F_3(J_3) |J_3| \, 
\prod_{i=1}^3 |I_i|^{-\frac12} |J_i|^{-\frac12}
\\
&\lesssim \widetilde{F}_1(I_1, J_1) \frac{\rs(I_1, J_1)^{\frac12}}{\ird(I_1, J_1)^{\delta_0}} 
F_2(I_2) \rs(I_2, J_2)^{\frac12} \, F_3(J_3) \rs(I_3, J_3)^{\frac12}. 
\end{align*}
By the compact full kernel representation, the size condition of $K$, \eqref{def:Sd0d}, and that $\ird(I_1, J_1) \le 2$  give 
\begin{align*}
|\mathscr{L}_{2, 1}|
&\lesssim \mathscr{S}_{\dagger, 0, \dagger}(I, J) \prod_{i=1}^3 |I_i|^{-\frac12} |J_i|^{-\frac12} 
\\
&\lesssim \widetilde{F}_1(I_1, J_1) \frac{\rs(I_1, J_1)^{\frac12}}{\ird(I_1, J_1)^{\delta_0}} 
F_2(I_2) \rs(I_2, J_2)^{\frac12} \, F_3(J_3) \rs(I_3, J_3)^{\frac12},  
\end{align*}
and the mixed size-H\"{o}lder condition of $K$ and \eqref{def:S10d} yield 
\begin{align*}
|\mathscr{L}_{2, 2}|
&\lesssim \mathscr{S}_{1, 0, \dagger}(I, J) 
\prod_{i=1}^3 |I_i|^{-\frac12} |J_i|^{-\frac12} 
\\
&\lesssim \widetilde{F}_1(I_1, J_1) \frac{\rs(I_1, J_1)^{\frac12}}{\ird(I_1, J_1)^{\delta_0}} 
F_2(I_2) \rs(I_2, J_2)^{\frac12} \, F_3(J_3) \rs(I_3, J_3)^{\frac12}.   
\end{align*}
In addition, in view of the compact full kernel representation, the mixed size-H\"{o}lder condition of $K$ and \eqref{def:Sd22} imply 
\begin{align*}
|\mathscr{L}_{9, 1}|
&\lesssim \mathscr{S}_{\dagger, 2, 2}(I, J) 
\prod_{i=1}^3 |I_i|^{-\frac12} |J_i|^{-\frac12} 
\\
&\lesssim \widetilde{F}_1(I_1, J_1) \frac{\rs(I_1, J_1)^{\frac12}}{\ird(I_1, J_1)^{\delta_0}} 
F_2(I_2) \rs(I_2, J_2)^{\frac12} \, F_3(J_3) \rs(I_3, J_3)^{\frac12},  
\end{align*}
and the H\"{o}lder condition of $K$ and \eqref{def:S122} give 
\begin{align*}
|\mathscr{L}_{9, 2}|
&\lesssim \mathscr{S}_{1, 2, 2}(I, J) 
\prod_{i=1}^3 |I_i|^{-\frac12} |J_i|^{-\frac12} 
\\
&\lesssim \widetilde{F}_1(I_1, J_1) \frac{\rs(I_1, J_1)^{\frac12}}{\ird(I_1, J_1)^{\delta_0}} 
F_2(I_2) \rs(I_2, J_2)^{\frac12} \, F_3(J_3) \rs(I_3, J_3)^{\frac12}. 
\end{align*}
Hence, these estimates yield \eqref{HH10} in the case $I_{2, 3} \supsetneq J_{2, 3}$.



Next, we invoke \eqref{HH50} to treat the case $I_{2, 3} = J_{2, 3}$. If $J'_{2, 3}=J''_{2, 3}$, the the compact partial kernel representation, the size condition of $K_{\mathbf{1}_{J'_{2, 3}}, \mathbf{1}_{J'_{2, 3}}}$, and \eqref{def:P2} imply 
\begin{align}\label{HH101}
|\langle &T(h_{I_1} \otimes \mathbf{1}_{J'_{2, 3}}), h_{J_1} \otimes \mathbf{1}_{J''_{2, 3}} \rangle|
\nonumber \\
&\lesssim C(\mathbf{1}_{J'_{2, 3}}, \mathbf{1}_{J'_{2, 3}}) 
\mathscr{P}_1(I_1, J_1) |I_1|^{-\frac12} |J_1|^{-\frac12}
\nonumber \\
&\lesssim \widetilde{F}_1(I_1, J_1) \rs(I_1, J_1)^{\frac12} \, F_2(J'_2) |J'_2| \, F_3(J'_3) |J'_3| 
\nonumber \\
&\lesssim \widetilde{F}_1(I_1, J_1) \rs(I_1, J_1)^{\frac12} \, F_2(J_2) |J_2| \, F_3(J_3) |J_3|. 
\end{align}
If $J'_{2, 3} \neq J''_{2, 3}$, we see that $\rd(I_1, J_1)<1$, $I_1 \cap J_1 = \emptyset$, $\rd(J'_{2, 3}, J''_{2, 3}) < 1$, and $J'_{2, 3} \cap J''_{2, 3} = \emptyset$. This is similar to the case in Section \ref{sec:near-near}, which along with the fact that $\ird(J'_2, J''_2) = \ird(J'_3, J''_3) = 1$ gives 
\begin{align}\label{HH102}
|\langle T(h_{I_1} \otimes \mathbf{1}_{J'_{2, 3}}), h_{J_1} \otimes \mathbf{1}_{J''_{2, 3}} \rangle|
&\lesssim \widetilde{F}_1(I_1, J_1) \frac{\rs(I_1, J_1)^{\frac12}}{\ird(I_1, J_1)^{\delta_0}} 
\widetilde{F}_2(J'_2) |J'_2| \, \widetilde{F}_3(J'_3) |J'_3|
\nonumber \\
&\lesssim \widetilde{F}_1(I_1, J_1) \frac{\rs(I_1, J_1)^{\frac12}}{\ird(I_1, J_1)^{\delta_0}} 
\widetilde{F}_2(J_2) |J_2| \, \widetilde{F}_3(J_3) |J_3|. 
\end{align}
Thus, \eqref{HH10} in the case $I_{2, 3} = J_{2, 3}$ is a consequence of \eqref{HH101}--\eqref{HH102}. 
\end{proof}


Now it follows from Lemma \ref{lem:HH-10}, \eqref{JDD}, and Lemma \ref{lem:FF} parts \eqref{list-FF2}--\eqref{list-FF3} that
\begin{align*}
|\I^N_6| 
\le \sum_{k_2 \ge k_1 \ge 0} \sum_{1 \le m_1 \le 2^{k_1}} 
\sum_{J \not\in \D_{\Z}(2N)} 
\sum_{\substack{I \in \D_{\Z} \\ I_1 \in J_1(k_1, 0, m_1) \\ I_2=J_2^{(k_2)} \\ I_3 = J_3^{(k_2-k_1)}}} 
|\G_{I, J}| \, |f_I| \, |g_J|
\lesssim \sum_{i=0}^2 \I^N_{6, i}, 
\end{align*}
where 
\begin{align*}
\I^N_{6, 0} 
&:= \varepsilon \sum_{k_2 \ge k_1 \ge 0} 
2^{-\frac{k_1}{2}} 2^{-\frac{k_2}{2}} 2^{-\frac{k_2-k_1}{2}}
\sum_{1 \le m_1 \le 2^{k_1}} m_1^{-\delta_0}
\sum_{\substack{I, J \in \D_{\Z} \\ I_1 \in J_1(k_1, 0, m_1) \\ I_2=J_2^{(k_2)} \\ I_3 = J_3^{(k_2-k_1)}}} 
\, |f_I| \, |g_J|, 
\\
\I^N_{6, i} 
&:= \varepsilon \sum_{k_2 \ge k_1 \ge 0 \atop k_i \ge N} 
2^{-\frac{k_1}{2}} 2^{-\frac{k_2}{2}} 2^{-\frac{k_2-k_1}{2}} 
\sum_{1 \le m_1 \le 2^{k_1}} m_1^{-\delta_0} 
\sum_{\substack{I, J \in \D_{\Z} \\ I_1 \in J_1(k_1, 0, m_1) \\ I_2=J_2^{(k_2)} \\ I_3 = J_3^{(k_2-k_1)}}} 
\, |f_I| \, |g_J|, 
\end{align*}
for $i=1, 2$. 
Then by the Cauchy-Schwarz inequality and \eqref{car-IJK}--\eqref{kaka}, we have 
\begin{align*}
\I^N_{6, 0} 
&\le \varepsilon \sum_{k_2 \ge k_1 \ge 0} 
2^{-\frac{k_1}{2}} 2^{-\frac{k_2}{2}} 2^{-\frac{k_2-k_1}{2}} 
\bigg(\sum_{\substack{I \in \D_{\Z} \\ I_2=J_2^{(k_2)} \\ I_3 = J_3^{(k_2-k_1)}}} 
\sum_{1 \le m_1 \le 2^{k_1}} \sum_{J_1 \in I_1(k_1, 0, m_1)} |f_I|^2 \bigg)^{\frac12}
\\
&\quad\times 
\bigg(\sum_{I \in \D_{\Z}} \sum_{1 \le m_1 \le 2^{k_1}} m_1^{-2\delta_0} 
\sum_{\substack{I, J \in \D_{\Z} \\ I_1 \in J_1(k_1, 0, m_1) \\ I_2=J_2^{(k_2)} \\ I_3 = J_3^{(k_2-k_1)}}} 
|g_J|^2 \bigg)^{\frac12}
\\
&\lesssim \varepsilon \sum_{k_2 \ge k_1 \ge 0} 
2^{-\frac{k_1}{2}} 2^{-\frac{k_2-k_1}{2}} 
\bigg(2^{-k_1} \sum_{1 \le m_1 \le 2^{k_1}} m_1^{-2\delta_0} \bigg)^{\frac12}
\|f\|_{L^2(\R^3)} \|g\|_{L^2(\R^3)}
\\
&\lesssim \varepsilon \sum_{k_2 \ge k_1 \ge 0} 
2^{-k_1 \delta_0 \beta} 2^{-\frac{k_2}{2}} 2^{-\frac{k_2-k_1}{2}} 
\|f\|_{L^2(\R^3)} \|g\|_{L^2(\R^3)}
\\
&\lesssim \varepsilon \|f\|_{L^2(\R^3)} \|g\|_{L^2(\R^3)}, 
\end{align*}
and for each $i=1, 2$, 
\begin{align*}
\I^N_{6, i} 
&\lesssim \sum_{k_2 \ge k_1 \ge 0 \atop k_i \ge N} 
2^{-k_1 \delta_0 \beta} 2^{-\frac{k_2}{2}} 2^{-\frac{k_2-k_1}{2}} 
\|f\|_{L^2(\R^3)} \|g\|_{L^2(\R^3)}
\\
&\lesssim (2^{-N\delta_0 \beta} + 2^{-N/2}) \|f\|_{L^2(\R^3)} \|g\|_{L^2(\R^3)}
\le \varepsilon \|f\|_{L^2(\R^3)} \|g\|_{L^2(\R^3)}. 
\end{align*}
Hence, we achieve 
\begin{align*}
|\I^N_6| 
\lesssim \sum_{i=0}^2 \I^N_{6, i} 
\lesssim \varepsilon \|f\|_{L^2(\R^3)} \|g\|_{L^2(\R^3)}. 
\end{align*}



\subsection{$I_1 \subset J_1$ and $\rd(I_{2, 3}, J_{2, 3}) \ge 1$} 
To control each term $\I^N_{7, i}$, $i=1, 2, 3$, we denote $\widetilde{h}_{I_1} := |I_1|^{\frac12} h_{I_1}$ and rewrite $\G_{I, J} = \mathscr{J}_0 + \mathscr{J}_1 + \mathscr{J}_2$, where 
\begin{equation}\label{GJJJ}
\begin{aligned}
\mathscr{J}_0  
&:= \langle T(h_{I_1} \otimes h_{I_{2, 3}}), (h_{J_1} \mathbf{1}_{I_1}) \otimes h_{J_{2, 3}} \rangle, 
\\
\mathscr{J}_1 
&:= \langle T(h_{I_1} \otimes h_{I_{2, 3}}), (h_{J_1} \mathbf{1}_{3I_1 \setminus I_1}) \otimes h_{J_{2, 3}} \rangle, 
\\
\mathscr{J}_2 
&:= \langle T(h_{I_1} \otimes h_{I_{2, 3}}), (h_{J_1} \mathbf{1}_{(3I_1)^c}) \otimes h_{J_{2, 3}} \rangle. 
\end{aligned}
\end{equation}



\subsubsection{$I_1 \subset J_1$, $\rd(I_2, J_2) \ge 1$, and $\rd(I_3, J_3) \ge 1$} 


\begin{lemma}\label{lem:HH-11}
Let $I, J \in \D_{\Z}$ with $\ell(I_1) \le \ell(J_1)$, $\ell(I_2) \ge \ell(J_2)$, and $\ell(I_3) \ge \ell(J_3)$ so that $I_1 \subset J_1$, $\rd(I_2, J_2) \ge 1$, and $\rd(I_3, J_3) \ge 1$. Then 
\begin{align*}
|\G_{I, J}|
\lesssim \widehat{F}_1(I_1) \frac{\rs(I_1, J_1)^{\frac12}}{D_{\theta}^{2, 3}(I, J)}  
\bigg[\frac{\rs(I_2, J_2)}{\rd(I_2, J_2)} 
+ \frac{\rs(I_3, J_3)}{\rd(I_3, J_3)} \bigg]^{\delta_{2, 3}} 
\prod_{i=2}^3 F_i(I_i, J_i) \frac{\rs(I_i, J_i)^{\frac12}}{\rd(I_i, J_i)}. 
\end{align*}
\end{lemma}


\begin{proof}
Observe that for all $x_{2, 3} \in J_{2, 3}$ and $y_{2, 3} \in I_{2, 3}$, 
\begin{align}\label{HH112}
|x_i - y_i| \simeq \d(I_i, J_i), \qquad 
\frac{|x_i - c_{J_i}|}{|x_i - y_i|} \lesssim \frac{\rs(I_i, J_i)}{\rd(I_i, J_i)}, \quad i=2, 3, 
\end{align}
and 
\begin{align}\label{HH113}
D_{\theta}(\ell(I_1), x_{2, 3} - y_{2, 3}) 
\simeq \bigg(\frac{\rd(I_2, J_2)}{\rd(I_3, J_3)} 
+ \frac{\rd(I_3, J_3)}{\rd(I_2, J_2)} \bigg)^{-\theta}
=D_{\theta}^{2, 3}(I, J)^{-1}. 
\end{align}
Let us first deal with the case $I_1 \subsetneq J_1$. The compact partial kernel representation, the cancellation of $h_{J_{2, 3}}$,  the H\"{o}lder condition of $K_{\widetilde{h}_{I_1}, \mathbf{1}_{I_1}}$, and \eqref{HH112}--\eqref{HH113} imply 
\begin{align*}
|\mathscr{J}_0| 
&\lesssim |I_1|^{-\frac12} |J_1|^{-\frac12} C(\widetilde{h}_{I_1}, \mathbf{1}_{I_1})  
\int_{I_{2, 3}} \int_{J_{2, 3}} 
\bigg(\frac{|y_2 - c_{I_2}|}{|x_2 - y_2|} + \frac{|y_3 - c_{I_3}|}{|x_3 - y_3|}\bigg)^{\delta_{2, 3}} 
\\
&\quad\times D_{\theta}(\ell(I_1), x_{2, 3} - y_{2, 3}) 
\prod_{i=2}^3 \frac{F_i(x_i, y_i)}{|x_i - y_i|} |I_i|^{-\frac12} |J_i|^{-\frac12} \, dx_{2, 3} \, dy_{2, 3} 
\\
&\lesssim F_1(I_1) \frac{\rs(I_1, J_1)^{\frac12}}{D_{\theta}^{2, 3}(I, J)} 
\bigg[\frac{\rs(I_2, J_2)}{\rd(I_2, J_2)} + \frac{\rs(I_3, J_3)}{\rd(I_3, J_3)}\bigg]^{\delta_{2, 3}} 
\prod_{i=2}^3 \mathscr{P}_i(I_i, J_i) |I_i|^{-\frac12} |J_i|^{-\frac12}  
\\
&\lesssim F_1(I_1) \frac{\rs(I_1, J_1)^{\frac12}}{D_{\theta}^{2, 3}(I, J)} 
\bigg[\frac{\rs(I_2, J_2)}{\rd(I_2, J_2)} + \frac{\rs(I_3, J_3)}{\rd(I_3, J_3)}\bigg]^{\delta_{2, 3}} 
\prod_{i=2}^3 F_i(I_i, J_i) \frac{\rs(I_i, J_i)^{\frac12}}{\rd(I_i, J_i)}, 
\end{align*}
where \eqref{def:P1} was used in the last inequality. Denote $t_1 := \frac{\d(I_2, J_2)}{\d(I_3, J_3)}$. Then from the compact partial kernel representation, the cancellation of $h_{J_{2, 3}}$, and the mixed size-H\"{o}lder condition of $K$, it follows 
\begin{align*}
|\mathscr{J}_1|
&\lesssim \mathscr{Q}_1^{0, t_1}(I_1) 
\mathscr{P}_2(I_2, J_2) \mathscr{P}_3(I_3, J_3)  
\bigg[\frac{\rs(I_2, J_2)}{\rd(I_2, J_2)} + \frac{\rs(I_3, J_3)}{\rd(I_3, J_3)}\bigg]^{\delta_{2, 3}} 
\prod_{i=1}^3  |I_i|^{-\frac12} |J_i|^{-\frac12}  
\\
&\lesssim \widetilde{F}_1(I_1) \frac{\rs(I_1, J_1)^{\frac12}}{D_{\theta}^{2, 3}(I, J)} 
\bigg[\frac{\rs(I_2, J_2)}{\rd(I_2, J_2)} + \frac{\rs(I_3, J_3)}{\rd(I_3, J_3)}\bigg]^{\delta_{2, 3}} 
\prod_{i=2}^3 F_i(I_i, J_i) \frac{\rs(I_i, J_i)^{\frac12}}{\rd(I_i, J_i)}. 
\end{align*}
In addition, we use the compact partial kernel representation, the cancellation of $h_{I_1}$ and $h_{J_{2, 3}}$, and the H\"{o}lder condition of $K$ to conclude 
\begin{align*}
|\mathscr{J}_2| 
&:= |\langle T(h_{I_1} \otimes h_{I_{2, 3}}), (h_{J_1} \mathbf{1}_{(3I_1)^c}) \otimes h_{J_{2, 3}} \rangle|
\\
&\lesssim \mathscr{Q}_1^{t_1}(I_1) \mathscr{P}_2(I_2, J_2) \mathscr{P}_3(I_3, J_3)
\bigg[\frac{\rs(I_2, J_2)}{\rd(I_2, J_2)} + \frac{\rs(I_3, J_3)}{\rd(I_3, J_3)}\bigg]^{\delta_{2, 3}} 
\prod_{i=1}^3 |I_i|^{-\frac12} |J_i|^{-\frac12}  
\\
&\lesssim \widetilde{F}_1(I_1) \frac{\rs(I_1, J_1)^{\frac12}}{D_{\theta}^{2, 3}(I, J)} 
\bigg[\frac{\rs(I_2, J_2)}{\rd(I_2, J_2)} + \frac{\rs(I_3, J_3)}{\rd(I_3, J_3)}\bigg]^{\delta_{2, 3}} 
\prod_{i=2}^3 F_i(I_i, J_i) \frac{\rs(I_i, J_i)^{\frac12}}{\rd(I_i, J_i)}. 
\end{align*}
Then, \eqref{GJJJ} gives the desired estimate.


Next, we treat the case $I_1=J_1$. Write 
\begin{align}\label{thh-1}
\langle T(h_{I_1} \otimes h_{I_{2, 3}}), h_{I_1} \otimes h_{J_{2, 3}} \rangle
= \sum_{I'_1, \, I''_1 \in \ch(I_1)} \langle h_{I_1} \rangle_{I'_1} \langle h_{I_1} \rangle_{I''_1}
\langle T(\mathbf{1}_{I'_1} \otimes h_{I_{2, 3}}), \mathbf{1}_{I''_1} \otimes h_{J_{2, 3}} \rangle.
\end{align}
If $I'_1 \neq I''_1$, then $\rd(I'_1, I''_1) < 1$, $I'_1 \cap I''_1 = \emptyset$, $\rd(I_2, J_2) \ge 1$, and $\rd(I_3, J_3) \ge 1$. This is similar to the case of Lemma \ref{lem:HH-6}. Hence, 
\begin{multline}\label{thh-2}
|\langle T(\mathbf{1}_{I'_1} \otimes h_{I_{2, 3}}), \mathbf{1}_{I''_1} \otimes h_{J_{2, 3}} \rangle|
\\ 
\lesssim \bigg[\frac{\rs(I_2, J_2)}{\rd(I_2, J_2)} 
+ \frac{\rs(I_3, J_3)}{\rd(I_3, J_3)} \bigg]^{\delta_{2, 3}} 
\frac{\widetilde{F}_1(I_1) |I_1|}{D_{\theta}^{2, 3}(I, J)} 
\prod_{i=2}^3 F_i(I_i, J_i) \frac{\rs(I_i, J_i)^{\frac12}}{\rd(I_i, J_i)}. 
\end{multline}
To handle the case $I'_1=I''_1$, the compact partial kernel representation, the cancellation of $h_{J_{2, 3}}$,  the H\"{o}lder condition of $K_{\mathbf{1}_{I'_1}, \mathbf{1}_{I'_1}}$, \eqref{HH112}--\eqref{HH113}, and \eqref{def:P1} imply 
\begin{align}\label{thh-3}
|\langle &T(\mathbf{1}_{I'_1} \otimes h_{I_{2, 3}}), \mathbf{1}_{I'_1} \otimes h_{J_{2, 3}} \rangle|
\nonumber \\
&\lesssim C(\mathbf{1}_{I'_1}, \mathbf{1}_{I'_1})  
\int_{I_{2, 3}} \int_{J_{2, 3}} 
\bigg(\frac{|y_2 - c_{I_2}|}{|x_2 - y_2|} + \frac{|y_3 - c_{I_3}|}{|x_3 - y_3|}\bigg)^{\delta_{2, 3}} 
\nonumber \\
&\quad\quad\times D_{\theta}(\ell(I_1), x_{2, 3} - y_{2, 3}) 
\prod_{i=2}^3 \frac{F_i(x_i, y_i)}{|x_i - y_i|} |I_i|^{-\frac12} |J_i|^{-\frac12} \, dx_{2, 3} \, dy_{2, 3} 
\nonumber \\
&\lesssim \frac{F_1(I'_1) |I'_1|}{D_{\theta}^{2, 3}(I, J)}  
\bigg[\frac{\rs(I_2, J_2)}{\rd(I_2, J_2)} + \frac{\rs(I_3, J_3)}{\rd(I_3, J_3)}\bigg]^{\delta_{2, 3}} 
\prod_{i=2}^3 \mathscr{P}_i(I_i, J_i) |I_i|^{-\frac12} |J_i|^{-\frac12}  
\nonumber \\
&\lesssim \frac{F_1(I_1) |I_1|}{D_{\theta}^{2, 3}(I, J)}
\bigg[\frac{\rs(I_2, J_2)}{\rd(I_2, J_2)} + \frac{\rs(I_3, J_3)}{\rd(I_3, J_3)}\bigg]^{\delta_{2, 3}} 
\prod_{i=2}^3 F_i(I_i, J_i) \frac{\rs(I_i, J_i)^{\frac12}}{\rd(I_i, J_i)}. 
\end{align}
Consequently, \eqref{thh-1}--\eqref{thh-3} yield the inequality as desired. 
\end{proof}


By Lemma \ref{lem:HH-11}, \eqref{JDD}, and Lemma \ref{lem:FF} parts \eqref{list-FF1} and \eqref{list-FF3}, there exists $N_0>1$ so that for all $N \ge N_0$,
\begin{align*}
|\I^N_{7, 1}| 
\le \sum_{k_2 \ge k_3 \ge 0 \atop j_2, j_3 \ge 1}  \sum_{J \not\in \D_{\Z}(2N)} 
\sideset{}{'}\sum_{\substack{I \in \D_{\Z} \\ J_1 = I_1^{(k_2-k_3)} \\ I_2 \in J_2(-k_2, j_2) \\ I_3 \in J_3(-k_3, j_3)}} 
|\G_{I, J}| \, |f_I| \, |g_J|  
\lesssim \sum_{i=0}^3 \I^{N, i}_{7, 1},
\end{align*}
where
\begin{align*}
\I^{N, 0}_{7, 1}
&:= \varepsilon \sum_{\substack{k_2 \ge k_3 \ge 0 \\ j_2, j_3 \ge 1}} 
\Gamma(k, j) \, 2^{-\frac{k_2}{2}} \, 2^{-\frac{k_3}{2}} 
\sum_{\substack{I, J \in \D_{\Z} \\ J_1 = I_1^{(k_2-k_3)} \\ I_2 \in J_2(-k_2, j_2) \\ I_3 \in J_3(-k_3, j_3)}} 
|f_I| |g_J|,
\\
\I^{N, 1}_{7, 1}
&:= \sum_{\substack{k_2 \ge k_3 \ge 0 \\ j_2, j_3 \ge 1 \\ k_2 \ge N}} 
\Gamma(k, j) \, 2^{-\frac{k_2}{2}} \, 2^{-\frac{k_3}{2}} 
\sum_{\substack{I, J \in \D_{\Z} \\ J_1 = I_1^{(k_2-k_3)} \\ I_2 \in J_2(-k_2, j_2) \\ I_3 \in J_3(-k_3, j_3)}} 
|f_I| |g_J|,
\\
\I^{N, i}_{7, 1}
&:= \sum_{\substack{k_2 \ge k_3 \ge 0 \\ j_2, j_3 \ge 1 \\  j_i \ge N^{1/8}}}  
\Gamma(k, j) \, 2^{-\frac{k_2}{2}} \, 2^{-\frac{k_3}{2}} 
\sum_{\substack{I, J \in \D_{\Z} \\ J_1 = I_1^{(k_2-k_3)} \\ I_2 \in J_2(-k_2, j_2) \\ I_3 \in J_3(-k_3, j_3)}} 
|f_I| |g_J|,
\end{align*}
for $i=2, 3$, with the term 
\begin{align*}
\Gamma(k, j) 
:= 2^{-\frac{k_2-k_3}{2}} j_2^{-1} j_3^{-1} 
\frac{(2^{-k_2} j_2^{-1} + 2^{-k_3} j_3^{-1})^{\delta_{2, 3}}}{(j_2 j_3^{-1} + j_2^{-1} j_3)^{\theta}}.
\end{align*}
Then, it follows from the Cauchy--Schwarz inequality, \eqref{car-IJ}, and \eqref{JJA-1} that for any $N>1$ sufficiently large, 

\begin{align*}
\I^{N, 0}_{7, 1}
&\le \varepsilon \sum_{\substack{k_2 \ge k_3 \ge 0 \\ j_2, j_3 \ge 1}} 
\Gamma(k, j) \, 2^{-\frac{k_2}{2}} \, 2^{-\frac{k_3}{2}}
\bigg(\sum_{\substack{I, J \in \D_{\Z} \\ J_1 = I_1^{(k_2-k_3)} \\ I_2 \in J_2(-k_2, j_2) \\ I_3 \in J_3(-k_3, j_3)}} 
|f_I|^2 \bigg)^{\frac12}
\bigg(\sum_{\substack{I, J \in \D_{\Z} \\ J_1 = I_1^{(k_2-k_3)} \\ I_2 \in J_2(-k_2, j_2) \\ I_3 \in J_3(-k_3, j_3)}} 
|g_J|^2 \bigg)^{\frac12}
\\
&\lesssim \varepsilon \sum_{\substack{k_2 \ge k_3 \ge 0 \\ j_2, j_3 \ge 1}} 
\Gamma(k, j) 
\bigg(\sum_{I \in \D_{\Z}} |f_I|^2 \bigg)^{\frac12}
\bigg(\sum_{J \in \D_{\Z}} |g_J|^2 \bigg)^{\frac12}
\\
&\lesssim \varepsilon \|f\|_{L^2(\R^3)} \|g\|_{L^2(\R^3)}
\end{align*}
Similarly, we have 
\begin{align*}
\I^{N, 1}_{7, 1}
&\lesssim \sum_{\substack{k_2 \ge k_3 \ge 0 \\ j_2, j_3 \ge 1 \\ k_i \ge N}} 
\Gamma(k, j) \|f\|_{L^2(\R^3)} \|g\|_{L^2(\R^3)} 
\\ 
&\lesssim 2^{-N \min\{\delta_0, \frac12\}} N \|f\|_{L^2(\R^3)} \|g\|_{L^2(\R^3)}
\le \varepsilon \|f\|_{L^2(\R^3)} \|g\|_{L^2(\R^3)}
\end{align*}
and for each $i=1, 2$, 
\begin{align*}
\I^{N, i}_{7, 1}
&\lesssim \sum_{\substack{k_2 \ge k_3 \ge 0 \\ j_2, j_3 \ge 1 \\ j_i \ge N^{1/8}}} 
\Gamma(k, j) \|f\|_{L^2(\R^3)} \|g\|_{L^2(\R^3)} 
\\ 
&\lesssim N^{-\delta_0/8} \|f\|_{L^2(\R^3)} \|g\|_{L^2(\R^3)}
\le \varepsilon \|f\|_{L^2(\R^3)} \|g\|_{L^2(\R^3)}. 
\end{align*} 
Now collecting the estimates above, we obtain 
\begin{align}\label{SN171}
|\I^N_{7, 1}| 
\lesssim \sum_{i=0}^4 \I^{N, i}_{7, 1} 
\lesssim \varepsilon \|f\|_{L^2(\R^3)} \|g\|_{L^2(\R^3)}. 
\end{align}


\subsubsection{$I_1 \subset J_1$, $\rd(I_2, J_2) \ge 1$, and $\rd(I_3, J_3) <1$} 


\begin{lemma}\label{lem:HH-12}
Let $I, J \in \D_{\Z}$ with $\ell(I_1) \le \ell(J_1)$, $\ell(I_2) \ge \ell(J_2)$, and $\ell(I_3) \ge \ell(J_3)$ so that $\rd(I_2, J_2) \ge 1$ and $\rd(I_3, J_3) < 1$. Then 
\begin{align}\label{HH12}
|\G_{I, J}|
\lesssim \widehat{F}_1(I_1) \rs(I_2, J_2)^{\frac12} 
F_2(I_2, J_2) \frac{\rs(I_2, J_2)^{\frac12}}{\rd(I_2, J_2)^{1+\theta}} 
\widetilde{F}_3(I_3, J_3) \rs(I_3, J_3)^{\frac12 + \theta - \frac1r}. 
\end{align}
\end{lemma}


\begin{proof}
Note that for all $x_2 \in J_2$ and $y_2 \in I_2$, 
\begin{align}\label{H672}
|x_2 - y_2| \simeq \d(I_2, J_2) 
\quad\text{ and }\quad 
D_{\theta}(\ell(I_1), x_{2, 3} - y_{2, 3}) 
\simeq \bigg(t_3 |x_3 - y_3| + \frac{1}{t_3 |x_3 - y_3|}\bigg)^{-\theta}, 
\end{align}
where $t_3 := \frac{1}{\ell(I_1) \d(I_2, J_2)}$. Let us first consider the case $I_1 \subsetneq J_1$. By the compact partial kernel representation and the size condition of $K_{\widetilde{h}_{I_1}, \mathbf{1}_{I_1}}$, the inequality \eqref{H672} gives 
\begin{align*}
|\mathscr{J}_0| 
&\lesssim |I_1|^{-\frac12} |J_1|^{-\frac12} C(\widetilde{h}_{I_1}, \mathbf{1}_{I_1})  
\int_{I_{2, 3}} \int_{J_{2, 3}} D_{\theta}(\ell(I_1), x_{2, 3} - y_{2, 3}) 
\nonumber \\
&\quad\times  
\prod_{i=2}^3 \frac{F_i(x_i, y_i)}{|x_i - y_i|} |I_i|^{-\frac12} |J_i|^{-\frac12} \, dx_{2, 3} \, dy_{2, 3} 
\nonumber \\
&\lesssim F_1(I_1) |I_1| \,
\mathscr{P}_2(I_2, J_2) \mathscr{Q}_3^{0, t_3}(I_3, J_3) 
\prod_{i=1}^3 |I_i|^{-\frac12} |J_i|^{-\frac12}  
\nonumber \\
&\lesssim F_1(I_1) \rs(I_1, J_1)^{\frac12} 
F_2(I_2, J_2) \frac{\rs(I_2, J_2)^{\frac12}}{\rd(I_2, J_2)^{1+\theta}} 
\widetilde{F}_3(I_3, J_3) \rs(I_3, J_3)^{\frac12 + \theta - \frac1r}, 
\end{align*}
where we have used \eqref{def:P1}, \eqref{def:Q0}, and that $t_3 |J_3| = \frac{\rs(I_3, J_3)}{\rd(I_2, J_2)}$. It follows from the compact full kernel representation and the size condition of $K$ that 
\begin{align*}
|\mathscr{J}_1| 
&\lesssim \mathscr{R}_{1, 3}^{0, 0, t}(I_1 \times I_3, I'_1 \times J_3) 
\mathscr{P}_2(I_2, J_2) \prod_{i=1}^3 |I_i|^{-\frac12} |J_i|^{-\frac12}  
\nonumber \\
&\lesssim \widetilde{F}_1(I_1) \rs(I_1, J_1)^{\frac12} 
F_2(I_2, J_2)  \frac{\rs(I_2, J_2)^{\frac12}}{\rd(I_2, J_2)^{1+\theta}} 
\widetilde{F}_3(J_3) \rs(I_3, J_3)^{\frac12 + \theta - \frac1r}, 
\end{align*}
where we have used \eqref{def:R1300}, \eqref{def:P2}, and that $\frac{t|I_1|}{|J_3|} = \frac{\rd(I_2, J_2)}{\rs(I_3, J_3)}$. Moreover, the mixed size-H\"{o}lder condition of $K$ implies 
\begin{align*}
|\mathscr{J}_2|
&\lesssim \mathscr{R}_{1, 3}^{2, 0, t}(I_{1, 3}, J_3) 
\mathscr{P}_2(I_2, J_2)  \prod_{i=1}^3 |I_i|^{-\frac12} |J_i|^{-\frac12}  
\nonumber \\
&\lesssim \widetilde{F}_1(I_1) \rs(I_1, J_1)^{\frac12} 
F_2(I_2, J_2) \frac{\rs(I_2, J_2)^{\frac12}}{\rd(I_2, J_2)^{1+\theta}}
\widetilde{F}_3(J_3) \rs(I_3, J_3)^{\frac12 + \theta - \frac1r}. 
\end{align*}


Next, to consider the case $I_1=J_1$, we use \eqref{thh-1}. If $I'_1 \neq I''_1$, then $\rd(I'_1, I''_1) < 1$, $I'_1 \cap I''_1 = \emptyset$, $\rd(I_2, J_2) \ge 1$, and $\rd(I_3, J_3) < 1$. This is similar to the case of Lemma \ref{lem:HH-7}. Thus, there holds 
\begin{align}\label{H721}
|\langle T(\mathbf{1}_{I'_1} \otimes h_{I_{2, 3}}), \mathbf{1}_{I''_1} \otimes h_{J_{2, 3}} \rangle|
\lesssim \widetilde{F}_1(I_1) |I_1| 
\frac{F_2(I_2, J_2)}{\rd(I_2, J_2)^{1+\theta}} 
\widetilde{F}_3(I_3, J_3). 
\end{align}
In the case $I'_1 = I''_1$, we use the compact partial kernel representation and the size condition of $K_{\mathbf{1}_{I'_1}, \mathbf{1}_{I'_1}}$ to arrive at  
\begin{align}\label{H722}
|\langle T&(\mathbf{1}_{I'_1} \otimes h_{I_{2, 3}}), \mathbf{1}_{I'_1} \otimes h_{J_{2, 3}} \rangle|
\nonumber \\
&\lesssim C(\mathbf{1}_{I'_1}, \mathbf{1}_{I'_1}) \int_{I_{2, 3}} \int_{J_{2, 3}} 
D_{\theta}(|I_1|, x_{2, 3} - y_{2, 3}) 
\prod_{i=2}^3 \frac{F_i(x_i, y_i)}{|x_i - y_i|} |I_i|^{-\frac12} |J_i|^{-\frac12} \, dx_{2, 3} \, dy_{2, 3} 
\nonumber \\
&\lesssim F_1(I'_1) |I'_1| \, \mathscr{P}_2(I_2, J_2) \mathscr{Q}_3^{0, t}(I_3, J_3) 
\prod_{i=2}^3 |I_i|^{-\frac12} |J_i|^{-\frac12}
\nonumber \\
&\lesssim F_1(I_1) |I_1| \, F_2(I_2, J_2) \frac{\rs(I_2, J_2)^{\frac12}}{\rd(I_2, J_2)^{1+\theta}} 
\widetilde{F}_3(I_3, J_3) \rs(I_3, J_3)^{\frac12 + \theta - \frac1r},  
\end{align}
where \eqref{def:P2}, \eqref{def:Q0}, and $t |J_3| = \frac{\rs(I_3, J_3)}{\rd(I_2, J_2)}$ were used in the last step. Hence, \eqref{H721} and \eqref{H722} imply \eqref{HH12}. 
\end{proof}


By Lemma \ref{lem:HH-12}, \eqref{JDD}, and Lemma \ref{lem:FF}, there exists $N_0>1$ so that for all $N \ge N_0$,
\begin{align*}
|\I^N_{7, 2}| 
\le \sum_{k_2 \ge k_1 \ge 0 \atop j_2 \ge 1} 
\sum_{J \not\in \D_{\Z}(2N)}
\sum_{\substack{I \in \D_{\Z} \\ J_1 = I_1^{(k_1)} \\ I_2 \in J_2(-k_2, j_2) \\ \rd(I_3, J_3) <1}} 
|\G_{I, J}| \, |f_I| \, |g_J|
\lesssim \sum_{i=0}^3 \I^{N, i}_{7, 2},
\end{align*}
where
\begin{align*}
\I^{N, 0}_{7, 2}
&:= \varepsilon \sum_{\substack{k_2 \ge k_1 \ge 0 \\ j_2 \ge 1}}   
2^{-\frac{k_1}{2}} j_2^{-1-\theta} 2^{-\frac{k_2}{2}} 2^{-(k_2-k_1)(\frac12 + \theta - \frac1r)} 
\sum_{\substack{I, J \in \D_{\Z} \\ J_1 = I_1^{(k_1)} \\ I_2 \in J_2(-k_2, j_2) \\ \rd(I_3, J_3) <1}} 
|f_I| |g_J|,
\\
\I^{N, i}_{7, 2}
&:= \sum_{\substack{k_2 \ge k_1 \ge 0 \\ j_2 \ge 1 \\ k_i \ge N}} 
2^{-\frac{k_1}{2}} j_2^{-1-\theta} 2^{-\frac{k_2}{2}} 2^{-(k_2-k_1)(\frac12 + \theta - \frac1r)} 
\sum_{\substack{I, J \in \D_{\Z} \\ J_1 = I_1^{(k_1)} \\ I_2 \in J_2(-k_2, j_2) \\ \rd(I_3, J_3) <1}}  
|f_I| |g_J|, 
\\
\I^{N, 3}_{7, 2}
&:= \sum_{\substack{k_2 \ge k_1 \ge 0 \\ j_2 \ge N^{1/8}}}  
2^{-\frac{k_1}{2}} j_2^{-1-\theta} 2^{-\frac{k_2}{2}} 2^{-(k_2-k_1)(\frac12 + \theta - \frac1r)} 
\sum_{\substack{I, J \in \D_{\Z} \\ J_1 = I_1^{(k_1)} \\ I_2 \in J_2(-k_2, j_2) \\ \rd(I_3, J_3) <1}} 
|f_I| |g_J|, 
\end{align*}
for $i=1, 2$. 
Then, it follows from the Cauchy--Schwarz inequality, \eqref{car-IJ}, and \eqref{Car-331}--\eqref{Car-332} that
\begin{align*}
\I^{N, 0}_{7, 2}
&\le \varepsilon \sum_{\substack{k_2 \ge k_1 \ge 0 \\ j_2 \ge 1}}   
2^{-\frac{k_1}{2}} j_2^{-1-\theta} 2^{-\frac{k_2}{2}} 
\bigg(\sum_{\substack{I, J \in \D_{\Z} \\ J_1 = I_1^{(k_1)} \\ I_2 \in J_2(-k_2, j_2) \\ \rd(I_3, J_3) <1}} 
|f_I|^2 \bigg)^{\frac12}
\\
&\quad\times 2^{-(k_2-k_1)(\frac12 + \theta - \frac1r)} 
\bigg(\sum_{\substack{I, J \in \D_{\Z} \\ J_1 = I_1^{(k_1)} \\ I_2 \in J_2(-k_2, j_2) \\ \rd(I_3, J_3) <1}}  
|g_J|^2 \bigg)^{\frac12}
\\
&\lesssim \varepsilon \sum_{\substack{k_2 \ge k_1 \ge 0 \\ j_2 \ge 1}}   
2^{-\frac{k_1}{2}} j_2^{-1-\theta} 2^{-(k_2-k_1)(\theta - \frac1r)}  
\|f\|_{L^2(\R^3)} \|g\|_{L^2(\R^3)}
\\
&\lesssim \varepsilon \|f\|_{L^2(\R^3)} \|g\|_{L^2(\R^3)}. 
\end{align*}
Similarly, for each $i=1, 2$, \eqref{kkk-1} gives 
\begin{align*}
\I^{N, i}_{7, 2}
&\lesssim \sum_{\substack{k_2 \ge k_1 \ge 0 \\ j_2 \ge 1 \\ k_i \ge N}}   
2^{-\frac{k_1}{2}} j_2^{-1-\theta} 2^{-(k_2-k_1)(\theta - \frac1r)}  
\|f\|_{L^2(\R^3)} \|g\|_{L^2(\R^3)}
\\
&\lesssim 2^{-N/2} \|f\|_{L^2(\R^3)} \|g\|_{L^2(\R^3)} 
\le \varepsilon \|f\|_{L^2(\R^3)} \|g\|_{L^2(\R^3)}, 
\end{align*}
and 
\begin{align*}
\I^{N, 3}_{7, 2}
&\lesssim \sum_{\substack{k_2 \ge k_1 \ge 0 \\ j_2 \ge N^{1/8}}}   
2^{-\frac{k_1}{2}} j_2^{-1-\theta} 2^{-(k_2-k_1)(\theta - \frac1r)}  
\|f\|_{L^2(\R^3)} \|g\|_{L^2(\R^3)}
\\
&\lesssim N^{-\theta/8} \|f\|_{L^2(\R^3)} \|g\|_{L^2(\R^3)}
\le \varepsilon \|f\|_{L^2(\R^3)} \|g\|_{L^2(\R^3)}. 
\end{align*}
Consequently, one has 
\begin{align}\label{SN172}
|\I^N_{7, 2}|
\lesssim \sum_{i=0}^3 \I^{N, i}_{7, 2} 
\lesssim \varepsilon \|f\|_{L^2(\R^3)} \|g\|_{L^2(\R^3)}. 
\end{align}


\subsubsection{$I_1 \subset J_1$, $\rd(I_2, J_2) < 1$, and $\rd(I_3, J_3) \ge 1$} 

\begin{lemma}\label{lem:HH-13}
Let $I, J \in \D_{\Z}$ with $\ell(I_1) \le \ell(J_1)$, $\ell(I_2) \ge \ell(J_2)$, and $\ell(I_3) \ge \ell(J_3)$ so that $I_1 \subset J_1$, $\rd(I_2, J_2) < 1$, and $\rd(I_3, J_3) \ge 1$. Then 
\begin{align}\label{HH13}
|\G_{I, J}| 
\lesssim \widehat{F}_1(I_1) \, \widetilde{F}_2(I_2, J_2) \rs(I_2, J_2)^{\frac12 + \theta - \frac1r}
F_3(I_3, J_3) \frac{\rs(I_3, J_3)^{\frac12}}{\rd(I_3, J_3)^{1+\theta}}. 
\end{align}
\end{lemma}


\begin{proof}
The proof is a minor modification of that of Lemma \ref{lem:HH-13}. The details are left to the reader. 
\end{proof}


With Lemma \ref{lem:HH-13} in hand, following the proof of \eqref{SN172}, one can show   
\begin{align}\label{SN173}
|\I^N_{7, 3}| 
\lesssim \varepsilon \|f\|_{L^2(\R^3)} \|g\|_{L^2(\R^3)}. 
\end{align}
Therefore, it follows from \eqref{SN171}, \eqref{SN172}, and \eqref{SN173} that 
\begin{align*}
|\I^N_7|
\le \sum_{j=1}^3 |\I^N_{7, j}| 
\lesssim \varepsilon \|f\|_{L^2(\R^3)} \|g\|_{L^2(\R^3)}. 
\end{align*}






\subsection{$I_1 \subset J_1$, $\rd(I_{2, 3}, J_{2, 3}) < 1$, and $I_{2, 3} \cap J_{2, 3} = \emptyset$}



\begin{lemma}\label{lem:HH-14}
Let $I, J \in \D_{\Z}$ with $\ell(I_1) \le \ell(J_1)$, $\ell(I_2) \ge \ell(J_2)$, and $\ell(I_3) \ge \ell(J_3)$ so that $I_1 \subset J_1$, $\rd(I_{2, 3}, J_{2, 3}) < 1$, and $I_{2, 3} \cap J_{2, 3} = \emptyset$. Then
\begin{align}\label{HH14}
|\G_{I, J}|
\lesssim \widehat{F}_1(I_1) \rs(I_1, J_1)^{\frac12} 
\widetilde{F}_2(I_2, J_2) \frac{\rs(I_2, J_2)^{\frac12}}{\ird(I_2, J_2)^{\delta_0}} 
\widetilde{F}_3(I_3, J_3) \frac{\rs(I_3, J_3)^{\frac12}}{\ird(I_3, J_3)^{\delta_0}}. 
\end{align}
\end{lemma}


\begin{proof}
Let us begin with the case $I_1 \subsetneq J_1$. Using $h_{J_1} = h_{J_1} \mathbf{1}_{I_1} + h_{J_1} \mathbf{1}_{3I_1 \setminus I_1} + h_{J_1} \mathbf{1}_{(3I_1)^c}$, each term $\mathscr{H}_i$ in \eqref{GHH} can be rewritten as the sum of three terms $\mathscr{H}_{i, j}$, $j=0, 1, 2$. In view of \eqref{def:R2300}, the compact partial kernel representation and the size condition of $K_{\widetilde{h}_{I_1}, \mathbf{1}_{I_1}}$ imply 
\begin{align}\label{HR10}
|\mathscr{H}_{1, 0}| 
& = |I_1|^{-\frac12} |\langle h_{J_1} \rangle_{I_1}| 
|\langle T(\widetilde{h}_{I_1} \otimes (h_{I_{2, 3}} \mathbf{1}_{(3J_2) \times (3J_3)})), 
\mathbf{1}_{I_1} \otimes h_{J_{2, 3}} \rangle|
\nonumber \\
&\lesssim C(\widetilde{h}_{I_1}, \mathbf{1}_{I_1}) 
\mathscr{R}_{2, 3}^{0, 0, |I_1|}(3J_{2, 3}, J_{2, 3})
\prod_{i=1}^3 |I_i|^{-\frac12} |J_i|^{-\frac12} 
\nonumber \\
&\lesssim F_1(I_1) |I_1| \, \widetilde{F}_2(J_2) |J_2| \, 
\widetilde{F}_3(J_3) |J_3| \prod_{i=1}^3 |I_i|^{-\frac12} |J_i|^{-\frac12} 
\\ \nonumber 
&\lesssim F_1(I_1) \rs(I_1, J_1)^{\frac12} 
\widetilde{F}_2(I_2, J_2) \frac{\rs(I_2, J_2)^{\frac12}}{\ird(I_2, J_2)^{\delta_0}} 
\widetilde{F}_3(I_3, J_3) \frac{\rs(I_3, J_3)^{\frac12}}{\ird(I_3, J_3)^{\delta_0}}, 
\end{align}
provided $\ird(I_i, J_i) \le 2$ for each $i=2, 3$. Note that $\mathscr{H}_{1, 1}$ is the same as that in Section \ref{sec:near-near}, and 
\begin{align}\label{def:S200}
|\mathscr{H}_{1, 2}| 
&\lesssim \int_{I_1 \times 3J_2 \times 3J_3} \int_{(3I_1)^c \times J_2 \times J_3} 
D_{\theta}(x-y) \prod_{i=1}^3 \frac{F_i(x_i, y_i)}{|x_i - y_i|} |I_i|^{-\frac12} |J_i|^{-\frac12} \, dx \, dy
\nonumber \\ 
&=: \mathscr{S}_{2, 0, 0}(I, J) \prod_{i=1}^3 |I_i|^{-\frac12} |J_i|^{-\frac12}
\nonumber \\ 
&\lesssim \widetilde{F}_1(I_1) |I_1| \, 
\widetilde{F}_2(J_2) |I_2|^{\frac1r} |J_2|^{1-\frac1r} \, 
\widetilde{F}_3(J_3) |I_3|^{\frac1r} |J_3|^{1-\frac1r}. 
\end{align}
which, along with $\ird(I_i, J_i) \le 2$ for each $i=2, 3$, gives the same bound as above. Here we point out that $\mathscr{S}_{2, 0, 0}(I, J)$ is symmetric to $\mathscr{S}_{0, 0, 2}(I, J)$ in \eqref{def:S002}. 

As shown for $\mathscr{H}_{1, 0}$, we use \eqref{def:R2301} and $\ird(I_2, J_2) \le 2$ to obtain   
\begin{align*}
|\mathscr{H}_{2, 0}| 
& = |I_1|^{-\frac12} |\langle h_{J_1} \rangle_{I_1}| 
|\langle T(\widetilde{h}_{I_1} \otimes (h_{I_{2, 3}} \mathbf{1}_{(3J_2) \times (3J_3)^c})), 
\mathbf{1}_{I_1} \otimes h_{J_{2, 3}} \rangle|
\\
&\lesssim C(\widetilde{h}_{I_1}, \mathbf{1}_{I_1}) 
\mathscr{R}_{2, 3}^{0, 1, |I_1|}(I_{2, 3}, J_{2, 3})
\prod_{i=1}^3 |I_i|^{-\frac12} |J_i|^{-\frac12} 
\\
&\lesssim F_1(I_1) |I_1| \, \widetilde{F}_2(J_2) |J_2| \, 
\widetilde{F}_3(I_3, J_3) \frac{|J_3|}{\ird(I_3, J_3)^{\delta_0}} 
\prod_{i=1}^3 |I_i|^{-\frac12} |J_i|^{-\frac12} 
\\
&\lesssim F_1(I_1) \rs(I_1, J_1)^{\frac12} 
\widetilde{F}_2(I_2, J_2) \frac{\rs(I_2, J_2)^{\frac12}}{\ird(I_2, J_2)^{\delta_0}} 
\widetilde{F}_3(I_3, J_3) \frac{\rs(I_3, J_3)^{\frac12}}{\ird(I_3, J_3)^{\delta_0}}. 
\end{align*}
Note that $\mathscr{H}_{2, 1}$ is the same as that in Section \ref{sec:near-near}, and 
\begin{align*}
|\mathscr{H}_{2, 2}| 
\lesssim \mathscr{S}_{2, 0, 1}(I, J) \prod_{i=1}^3 |I_i|^{-\frac12} |J_i|^{-\frac12},
\end{align*}
which along with \eqref{def:S201} and $\ird(I_2, J_2) \le 2$, gives the same bound as above. 

By symmetry, $\mathscr{H}_2$ and $\mathscr{H}_3$ have the same bound. Additionally, by the compact partial kernel representation, the H\"{o}lder condition of $K_{\widetilde{h}_{I_1}, \mathbf{1}_{I_1}}$, and \eqref{def:R2311}, we have 
\begin{align*}
|\mathscr{H}_{4, 0}| 
&=|I_1|^{-\frac12} |\langle h_{J_1} \rangle_{I_1}| 
|\langle T(\widetilde{h}_{I_1} \otimes (h_{I_{2, 3}} \mathbf{1}_{(3J_2)^c \times (3J_3)^c})), 
\mathbf{1}_{I_1} \otimes h_{J_{2, 3}} \rangle|
\\
&\lesssim C(\widetilde{h}_{I_1}, \mathbf{1}_{I_1}) 
\mathscr{R}_{2, 3}^{1, 1, |I_1|}(I_{2, 3}, J_{2, 3}) 
\prod_{i=1}^3 |I_i|^{-\frac12} |J_i|^{-\frac12} 
\\
&\lesssim F_1(I_1) |I_1| \, (|J_1|/|I_1|)^{\theta} 
\frac{\widetilde{F}_2(I_2, J_2) |J_2|}{\ird(I_2, J_2)^{\delta_0}} 
\frac{\widetilde{F}_3(I_3, J_3) |J_3|}{\ird(I_3, J_3)^{\delta_0}} 
\prod_{i=1}^3 |I_i|^{-\frac12} |J_i|^{-\frac12} 
\\
&\lesssim F_1(I_1) \rs(I_1, J_1)^{\frac12 - \theta} 
\widetilde{F}_2(I_2, J_2) \frac{\rs(I_2, J_2)^{\frac12}}{\ird(I_2, J_2)^{\delta_0}} 
\widetilde{F}_3(I_3, J_3) \frac{\rs(I_3, J_3)^{\frac12}}{\ird(I_3, J_3)^{\delta_0}}. 
\end{align*}
Note that $\mathscr{H}_{4, 1}$ coincides with the one in Section \ref{sec:near-near}, and 
\begin{align*}
|\mathscr{H}_{4, 2}| 
\lesssim \mathscr{S}_{2, 1, 1}(I, J) \prod_{i=1}^3 |I_i|^{-\frac12} |J_i|^{-\frac12}.
\end{align*} 
Hence, by \eqref{def:S211}, both of them have the same bound as above. Collecting these estimates, we deduce \eqref{HH14} in the case $I_1 \subsetneq J_1$. 

Next, we handle the case $I_1=J_1$. Recall \eqref{thh-1}. If $I'_1 \neq I''_1$, then $\rd(I'_1, I''_1) < 1$, $I'_1 \cap I''_1 =\emptyset$, $\rd(I_{2, 3}, J_{2, 3}) < 1$, and $I_{2, 3} \cap J_{2, 3} = \emptyset$. This is similar to the case in Section \ref{sec:near-near}. Thus, there holds 
\begin{align}\label{HH142}
|\langle T(\mathbf{1}_{I'_1} \otimes h_{I_{2, 3}}), \mathbf{1}_{I''_1} \otimes h_{J_{2, 3}} \rangle|
&\lesssim \widetilde{F}_1(I'_1) |I'_1| \, \widetilde{F}_2(I_2) \widetilde{F}_3(I_3)
\nonumber \\
&\lesssim \widetilde{F}_1(I_1) |I_1| \, \widetilde{F}_2(I_2, J_2) \widetilde{F}_3(I_3, J_3). 
\end{align}
If $I'_1 = I''_1$, then the compact partial kernel representation and the size condition of $K$ yield 
\begin{align}\label{HH143}
|\langle &T(\mathbf{1}_{I'_1} \otimes h_{I_{2, 3}}), \mathbf{1}_{I''_1} \otimes h_{J_{2, 3}} \rangle|
\lesssim C(\mathbf{1}_{I'_1}, \mathbf{1}_{I'_1}) 
\mathscr{R}_{2, 3}^{0, 0, |I'_1|}(I_{2, 3}, J_{2, 3}) |I_2|^{-1} |I_3|^{-1} 
\nonumber \\
&\lesssim F_1(I'_1) |I'_1| \, \widetilde{F}_2(I_2, J_2) \widetilde{F}_3(I_3, J_3) 
\lesssim F_1(I_1) |I_1| \, \widetilde{F}_2(I_2, J_2) \widetilde{F}_3(I_3, J_3), 
\end{align}
where \eqref{def:R2300} was used in the second-to-last inequality. Thus, we obtain \eqref{HH14} with $I_1=J_1$ from \eqref{HH142}--\eqref{HH143}.
\end{proof}



Applying Lemma \ref{lem:HH-14}, \eqref{JDD}, Lemma \ref{lem:FF} parts \eqref{list-FF2}--\eqref{list-FF3}, we have 
\begin{align*}
|\I^N_8| 
\le \sum_{k_2 \ge k_3 \ge 0} 
\sum_{1 \le m_2 \le 2^{k_2} \atop 1 \le m_3 \le 2^{k_3}} 
\sum_{J \not\in \D_{\Z}} \sum_{\substack{I \in \D_{\Z},\, J_1 = I_1^{(k_2-k_3)} \\ 
I_2 \in J_2(-k_2, 0, m_2) \\ I_3 \in J_3(-k_3, 0, m_3)}} 
|\G_{I, J}| \, |f_I| \, |g_J| 
\lesssim \sum_{i=1}^3 \I^N_{8, i},  
\end{align*}
where 
\begin{align*}
\I^N_{8, 1} 
&:= \varepsilon \sum_{k_2 \ge k_3 \ge 0} 
\sum_{1 \le m_2 \le 2^{k_2} \atop 1 \le m_3 \le 2^{k_3}} 
2^{-\frac{k_2-k_3}{2}} 2^{-\frac{k_2}{2}} 2^{-\frac{k_3}{2}} 
m_2^{-\delta_0} \, m_3^{-\delta_0}
\sum_{\substack{I, J \in \D_{\Z} \\ J_1 = I_1^{(k_2-k_3)} \\ 
I_2 \in J_2(-k_2, 0, m_2) \\ I_3 \in J_3(-k_3, 0, m_3)}} |f_I| \, |g_J|, 
\\
\I^N_{8, i} 
&:= \sum_{k_2 \ge k_3 \ge 0 \atop k_i \ge N} 
\sum_{1 \le m_2 \le 2^{k_2} \atop 1 \le m_3 \le 2^{k_3}} 
2^{-\frac{k_2-k_3}{2}} 2^{-\frac{k_2}{2}} 2^{-\frac{k_3}{2}} 
m_2^{-\delta_0} \, m_3^{-\delta_0}
\sum_{\substack{I, J \in \D_{\Z} \\ J_1 = I_1^{(k_2-k_3)} \\ 
I_2 \in J_2(-k_2, 0, m_2) \\ I_3 \in J_3(-k_3, 0, m_3)}} |f_I| \, |g_J|, 
\end{align*}
for $i=2, 3$. Then it follows from the Cauchy--Schwarz inequality and \eqref{car-IJK}--\eqref{kaka} that 
\begin{align*}
\I^N_{8, 0} 
&\le \varepsilon \sum_{k_2 \ge k_3 \ge 0} 2^{-\frac{k_2-k_3}{2}} 
\bigg(\sum_{1 \le m_2 \le 2^{k_2} \atop 1 \le m_3 \le 2^{k_3}} 
\sum_{\substack{I, J \in \D_{\Z} \\ J_1 = I_1^{(k_2-k_3)} \\ 
I_2 \in J_2(-k_2, 0, m_2) \\ I_3 \in J_3(-k_3, 0, m_3)}} |f_I|^2 \bigg)^{\frac12}
\\
&\quad\times  2^{-\frac{k_2}{2}} 2^{-\frac{k_3}{2}} 
\bigg(\sum_{1 \le m_2 \le 2^{k_2} \atop 1 \le m_3 \le 2^{k_3}} 
m_2^{-\delta_0} \, m_3^{-\delta_0}
\sum_{\substack{I, J \in \D_{\Z} \\ J_1 = I_1^{(k_2-k_3)} \\ 
I_2 \in J_2(-k_2, 0, m_2) \\ I_3 \in J_3(k_3, 0, m_3)}} |g_J|^2 \bigg)^{\frac12}
\\ 
&\lesssim \varepsilon \sum_{k_2 \ge k_3 \ge 0} 2^{-\frac{k_2-k_3}{2}} 
\bigg(\sum_{I \in \D_{\Z}} |f_I|^2 \bigg)^{\frac12} 
\bigg( \sum_{J \in \D_{\Z}} |g_J|^2 \bigg)^{\frac12}
\\
&\quad\times   
\bigg(2^{-k_2} 2^{-k_3} \sum_{1 \le m_2 \le 2^{k_2} \atop 1 \le m_3 \le 2^{k_3}} 
m_2^{-\delta_0} \, m_3^{-\delta_0}\bigg)^{\frac12} 
\\
&\lesssim  \varepsilon \sum_{k_2 \ge k_3 \ge 0} 2^{-\frac{k_2-k_3}{2}} \, 
2^{-k_2 \delta_0 \beta} \, 2^{-k_3 \delta_0 \beta}
\|f\|_{L^2(\R^3)} \|g\|_{L^2(\R^3)}
\\
&\lesssim  \varepsilon \|f\|_{L^2(\R^3)} \|g\|_{L^2(\R^3)}, 
\end{align*}
and for each $i=2, 3$, 
\begin{align*}
\I^N_{8, i} 
&\lesssim \sum_{k_2 \ge k_3 \ge 0 \atop k_i \ge N} 
2^{-\frac{k_2-k_3}{2}} \, 2^{-k_2 \delta_0 \beta} \, 2^{-k_3 \delta_0 \beta}
\|f\|_{L^2(\R^3)} \|g\|_{L^2(\R^3)}
\\
&\lesssim  \big(2^{-N/2} + 2^{-N \delta_0 \beta}\big) \|f\|_{L^2(\R^3)} \|g\|_{L^2(\R^3)}
\le  \varepsilon \|f\|_{L^2(\R^3)} \|g\|_{L^2(\R^3)}, 
\end{align*}
provided $N>1$ large enough. Thus, there holds 
\begin{align*}
|\I^N_8| 
\lesssim \sum_{i=1}^3 \I^N_{8, i} 
\lesssim \varepsilon \|f\|_{L^2(\R^3)} \|g\|_{L^2(\R^3)}. 
\end{align*}



\subsection{$I_1 \subset J_1$ and $I_{2, 3} \supset J_{2, 3}$} 



\begin{lemma}\label{lem:HH-15}
Let $I, J \in \D_{\Z}$ with $\ell(I_1) \le \ell(J_1)$, $\ell(I_2) \ge \ell(J_2)$, and $\ell(I_3) \ge \ell(J_3)$ so that $I_1 \subset J_1$ and $I_{2, 3} \supset J_{2, 3}$. Then there holds 
\begin{align}\label{HH15}
|\G_{I, J}|
\lesssim \mathscr{B}_{I, J} 
:= \widehat{F}_1(I_1) \rs(I_1, J_1)^{\frac12} \, 
\widehat{F}_2(J_2) \rs(I_2, J_2)^{\frac12} \,
\widehat{F}_3(J_3) \rs(I_3, J_3)^{\frac12}.
\end{align}
\end{lemma}


\begin{proof}
First, we treat the case $I_1 \subsetneq J_1$ and $I_{2, 3} \supsetneq J_{2, 3}$. By the fact that $h_{J_1} = h_{J_1} \mathbf{1}_{I_1} + h_{J_1} \mathbf{1}_{3I_1 \setminus I_1} + h_{J_1} \mathbf{1}_{(3I_1)^c}$, each term $\mathscr{L}_i$ in \eqref{GLL} can be split into three terms $\mathscr{L}_{i, j}$, $j=1, 2, 3$. We only present the proof of $\mathscr{L}_{i, j}$ for $i=2, 6, 9$ and $j=1, 2, 3$, which reveals the general strategy.  The estimate for $\mathscr{L}_{2, 1}$ is the same as in \eqref{HR10}, which is bounded by $\mathscr{B}_{I, J}$. It follows from the compact full kernel representation, the size condition of $K$, and \eqref{def:Sd0d} that 
\begin{align*}
|\mathscr{L}_{2, 2}|
\lesssim \mathscr{S}_{\dagger, 0, \dagger}(I, J) \prod_{i=1}^3 |I_i|^{-\frac12} |J_i|^{-\frac12}
\lesssim \mathscr{B}_{I, J}. 
\end{align*}
The compact full kernel representation, the size-H\"{o}lder condition of $K$, and \eqref{def:S20d} imply that 
\begin{align*}
|\mathscr{L}_{2, 3}|
\lesssim \mathscr{S}_{2, 0, \dagger}(I, J) \prod_{i=1}^3 |I_i|^{-\frac12} |J_i|^{-\frac12}
\lesssim \mathscr{B}_{I, J}. 
\end{align*}
Moreover, by the compact partial kernel representation, the size condition of $K_{\widetilde{h}_{I_1}, \mathbf{1}_{I_1}}$, and \eqref{def:R2302}, we have 
\begin{align*}
|\mathscr{L}_{6, 1}|
\lesssim C(\widetilde{h}_{I_1}, \mathbf{1}_{I_1}) 
\mathscr{R}_{2, 3}^{0, 2, |I_1|}(J_{2, 3}) 
\prod_{i=1}^3 |I_i|^{-\frac12} |J_i|^{-\frac12} 
\lesssim \mathscr{B}_{I, J}. 
\end{align*}
In light of \eqref{def:Sdd2}, the compact full kernel representation and the size condition of $K$ imply that 
\begin{align*}
|\mathscr{L}_{6, 2}|
\lesssim \mathscr{S}_{\dagger, \dagger, 2}(I, J) \prod_{i=1}^3 |I_i|^{-\frac12} |J_i|^{-\frac12}
\lesssim \mathscr{B}_{I, J}. 
\end{align*}
It follows from the compact full kernel representation and the size-H\"{o}lder condition of $K$ that 
\begin{align*}
|\mathscr{L}_{6, 3}|
&\lesssim \int_{I_1 \times (3J_2 \setminus J_2) \times (3J_3)^c} \int_{(3I_1)^c \times J_2 \times J_3} 
\frac{|I_1|^{\delta_1}}{|x_1-y_1|^{\delta_1}} 
D_{\theta}(x-y) \prod_{i=1}^3 \frac{F_i(x_i, y_i)}{|x_i - y_i|} \, dx \, dy
\\ 
&=: \mathscr{S}_{2, \dagger, 2}(I, J) \prod_{i=1}^3 |I_i|^{-\frac12} |J_i|^{-\frac12}
\lesssim \mathscr{B}_{I, J}, 
\end{align*}
where we have used a symmetric estimate to \eqref{def:Sd22}. 

On the other hand, by the compact partial kernel representation, the H\"{o}lder condition of $K_{\widetilde{h}_{I_1}, \mathbf{1}_{I_1}}$, and \eqref{def:R2322}, we arrive at  
\begin{align*}
|\mathscr{L}_{9, 1}|
\lesssim C(\widetilde{h}_{I_1}, \mathbf{1}_{I_1}) 
\mathscr{R}_{2, 3}^{2, 2, |I_1|}(J_{2, 3}) 
\prod_{i=1}^3 |I_i|^{-\frac12} |J_i|^{-\frac12} 
\lesssim \mathscr{B}_{I, J}. 
\end{align*}
The compact full kernel representation, the size-H\"{o}lder condition of $K$, and \eqref{def:Sd22} give  
\begin{align*}
|\mathscr{L}_{9, 2}|
\lesssim \mathscr{S}_{\dagger, 2, 2}(I, J) \prod_{i=1}^3 |I_i|^{-\frac12} |J_i|^{-\frac12}
\lesssim \mathscr{B}_{I, J}. 
\end{align*}
It follows from the compact full kernel representation, the H\"{o}lder condition of $K$, and \eqref{def:S222} that 
\begin{align*}
|\mathscr{L}_{9, 3}|
\lesssim \mathscr{S}_{2, 2, 2}(I, J) \prod_{i=1}^3 |I_i|^{-\frac12} |J_i|^{-\frac12}
\lesssim \mathscr{B}_{I, J}. 
\end{align*}
Now gathering these estimates, we conclude \eqref{HH15} in the case $I_1 \subsetneq J_1$ and $I_{2, 3} \supsetneq J_{2, 3}$. 

Next, considering the case $I_1 = J_1$ and $I_{2, 3} \subsetneq J_{2, 3}$, we use the identity \eqref{thh-1}. If $I'_1 \neq I''_1$, then $I'_1 \cap I''_1 = \emptyset$ and $\rd(I'_1, I''_1)<1$. This is similar to the case in Section \ref{sec:near-inside}, which immediately gives the desired bound. If $I'_1 = I''_1$, we apply the decomposition in \eqref{GLL} with both $h_{I_1}$ and $h_{J_1}$ replaced by $\mathbf{1}_{I'}$. As argued there, it suffices to analyze $\mathscr{L}_1$, $\mathscr{L}_3$, and $\mathscr{L}_9$. Invoking the cancellation condition $\langle T( \mathbf{1}_{I'_1} \otimes 1), \mathbf{1}_{I'_1} \otimes h_{J_{2, 3}}\rangle = 0$ and the compact partial kernel representation, we conclude that 
\begin{align}\label{HH052}
|\mathscr{L}_1| 
&= |\langle h_{I_{2, 3}} \rangle_{J_2 \times J_3}| 
|\langle T(\mathbf{1}_{I'_1} \otimes \mathbf{1}_{J_{2, 3}}), 
\mathbf{1}_{I'_1} \otimes h_{J_{2, 3}} \rangle| 
\nonumber 
\\
&\lesssim F_1(I_1) |I_1| \, \widetilde{F}_2(J_2) |J_2| \, 
\widetilde{F}_3(J_3) |J_3| \, |I_{2, 3}|^{-\frac12} |J_{2, 3}|^{-\frac12}
\end{align}
It follows from the compact partial kernel representation, the size condition, and \eqref{def:R2302} that  
\begin{align}\label{HH053}
|\mathscr{L}_3| 
&\lesssim C(\mathbf{1}_{I'_1}, \mathbf{1}_{I'_1})
\mathscr{R}_{2, 3}^{0, 2, |I'_1|}(J_{2, 3}) 
\, |I_{2, 3}|^{-\frac12} |J_{2, 3}|^{-\frac12} 
\nonumber \\ 
&\lesssim F_1(I_1) |I_1| \, \widetilde{F}_2(J_2) |J_2| \, 
\widetilde{F}_3(J_3) |J_3| \, |I_{2, 3}|^{-\frac12} |J_{2, 3}|^{-\frac12}
\end{align}
In addition, by the compact partial kernel representation, the H\"{o}lder condition, and \eqref{def:R2322},  we have 
\begin{align}\label{HH054}
|\mathscr{L}_9| 
&\lesssim C(\mathbf{1}_{I'_1}, \mathbf{1}_{I'_1})
\mathscr{R}_{2, 3}^{2, 2, |I'_1|}(J_{2, 3}) 
\, |I_{2, 3}|^{-\frac12} |J_{2, 3}|^{-\frac12} 
\nonumber \\ 
&\lesssim F_1(I_1) |I_1| \, \widetilde{F}_2(J_2) |J_2| \, 
\widetilde{F}_3(J_3) |J_3| \, |I_{2, 3}|^{-\frac12} |J_{2, 3}|^{-\frac12}
\end{align}
Thus, collecting \eqref{HH052}--\eqref{HH054}, we achieve \eqref{HH15} in the case $I_1 = J_1$ and $I_{2, 3} \subsetneq J_{2, 3}$. 

To address the case $I_1 \subsetneq J_1$ and $I_{2, 3} = J_{2, 3}$, we adopt a similar argument as that in the case $I_1 = J_1$ and $I_{2, 3} \subsetneq J_{2, 3}$. But now it needs to use the cancellation condition $\langle T(h_{I_1} \otimes \mathbf{1}_{J_{2, 3}}), 1 \otimes \mathbf{1}_{J_{2, 3}}\rangle = 0$. More details are left to the reader. 


Finally, let us deal with the case $I_1=J_1$ and $I_{2, 3} = J_{2, 3}$. As before, we write 
\begin{align}\label{HH151}
\langle &T(h_{I_1} \otimes h_{I_{2, 3}}), h_{I_1} \otimes h_{I_{2, 3}} \rangle
\nonumber \\
&= \sum_{I'_1, I''_1 \in \ch(I_1) \atop I'_{2, 3}, \, I''_{2, 3} \in \ch(I_{2, 3})} 
\langle h_{I_1} \rangle_{I'_1} \langle h_{I_1} \rangle_{I''_1}
\langle h_{I_{2, 3}} \rangle_{I'_{2, 3}} \langle h_{I_{2, 3}} \rangle_{I''_{2, 3}}
\langle T(\mathbf{1}_{I'_1} \otimes \mathbf{1}_{I'_{2, 3}}), \mathbf{1}_{I''_1} \otimes \mathbf{1}_{I''_{2, 3}} \rangle.
\end{align}
If $I'_1=I''_1$ and $I'_{2, 3}=I''_{2, 3}$, it follows from the weak compactness property that 
\begin{align}\label{HH152}
|\langle T(\mathbf{1}_{I'_1} \otimes \mathbf{1}_{I'_{2, 3}}), \mathbf{1}_{I'_1} \otimes \mathbf{1}_{I'_{2, 3}} \rangle|
\le \prod_{i=1}^3 F_i(I'_i) \, |I'_i| 
\lesssim \prod_{i=1}^3 F_i(I_i) \, |I_i|.  
\end{align}
If $I'_1=I''_1$ and $I'_{2, 3} \neq I''_{2, 3}$, we use the compact partial kernel representation, the size condition of $K_{\mathbf{1}_{I'_1}, \mathbf{1}_{I'_1}}$, and \eqref{def:R2300} to arrive at
\begin{align}\label{HH153}
|\langle &T(\mathbf{1}_{I'_1} \otimes \mathbf{1}_{I'_{2, 3}}),
\mathbf{1}_{I'_1} \otimes \mathbf{1}_{I''_{2, 3}} \rangle| 
\lesssim C(\mathbf{1}_{I'_1},  \mathbf{1}_{I'_1}) 
\mathscr{R}_{2, 3}^{0, 0, |I'_1|}(I'_{2, 3}, I''_{2, 3}) 
\nonumber \\
&\lesssim F_1(I'_1) |I'_1| \, \widetilde{F}_2(I'_2) |I'_2| \widetilde{F}_3(I'_3) |I'_3|
\lesssim F_1(I_1) |I_1| \, \widetilde{F}_2(I_2) |I_2| \widetilde{F}_3(I_3) |I_3|. 
\end{align}
Similarly, if $I'_1 \neq I''_1$ and $I'_{2, 3}=I''_{2, 3}$, it follows from \eqref{def:P2} that  
\begin{align}\label{HH154}
|\langle &T (\mathbf{1}_{I'_1} \otimes \mathbf{1}_{I'_{2, 3}}),
\mathbf{1}_{I''_1} \otimes \mathbf{1}_{I'_{2, 3}} \rangle|
\lesssim C(\mathbf{1}_{I'_{2, 3}}, \mathbf{1}_{I'_{2, 3}}) \mathscr{P}_1(I'_1, I''_1) 
\nonumber \\
&\lesssim \widetilde{F}_1(I'_1) |I'_1| \, F_2(I'_2) |I'_2| \, F_3(I'_3) |I'_3| 
\lesssim \widetilde{F}_1(I_1) |I_1| \, F_2(I_2) |I_2| \, F_3(I_3) |I_3|.
\end{align}
If $I'_1 \neq I''_1$ and $I'_{2, 3} \neq I''_{2, 3}$, then in light of \eqref{def:S000}, the compact full kernel representation and the size condition of $K$ imply
\begin{align}\label{HH155}
|\langle T (\mathbf{1}_{I'_1} \otimes \mathbf{1}_{I'_{2, 3}}),
\mathbf{1}_{I''_1} \otimes \mathbf{1}_{I''_{2, 3}} \rangle|
\lesssim \mathscr{S}_{0, 0, 0}(I', I'')  
\lesssim \prod_{i=1}^3 \widetilde{F}_i(I'_i) |I'_i|
\lesssim \prod_{i=1}^3 \widetilde{F}_i(I_i) |I_i|. 
\end{align}
Therefore, \eqref{HH15} is a consequence of \eqref{HH151}--\eqref{HH155}.
\end{proof}


Now by Lemma \ref{lem:HH-15}, \eqref{JDD}, and Lemma \ref{lem:FF} part \eqref{list-FF3}, we have for any $N>N_0$ large enough, 
\begin{align*}
|\I^N_9| 
&\le \sum_{k_2 \ge k_1 \ge 0} \sum_{J \not\in \D_{\Z}(2N)} 
\sum_{\substack{I \in \D_{\Z}, \, J_1 = I_1^{(k_1)} \\ I_2 = J_2^{(k_2)}, \, I_3 = J_3^{(k_2-k_1)}}}
|\G_{I, J}| \, |f_I| |g_J|
\lesssim \sum_{i=0}^2 \I^N_{9, i},
\end{align*}
where
\begin{align*}
\I^N_{9, 0}
&:= \varepsilon \sum_{k_2 \ge k_1 \ge 0} 2^{-\frac{k_1}{2}} 2^{-\frac{k_2}{2}} 2^{-\frac{k_2-k_1}{2}}
\sum_{\substack{I, J \in \D_{\Z}, \, J_1 = I_1^{(k_1)} \\ I_2 = J_2^{(k_2)}, \, I_3 = J_3^{(k_2-k_1)}}} 
\, |f_I| |g_J|, 
\\
\I^N_{9, i}
&:= \sum_{k_2 \ge k_1 \ge 0 \atop k_i \ge N} 2^{-\frac{k_1}{2}} 2^{-\frac{k_2}{2}} 2^{-\frac{k_2-k_1}{2}}
\sum_{\substack{I, J \in \D_{\Z}, \, J_1 = I_1^{(k_1)} \\ I_2 = J_2^{(k_2)}, \, I_3 = J_3^{(k_2-k_1)}}} 
\, |f_I| |g_J|,  
\end{align*}
for $i=1, 2$. The Cauchy--Schwarz inequality and \eqref{kaka} imply  
\begin{align*}
\I^N_{9, 0}
&\le \varepsilon \sum_{k_2 \ge k_1 \ge 0} 2^{-\frac{k_1}{2}} 2^{-\frac{k_2}{2}} 2^{-\frac{k_2-k_1}{2}}
\bigg(\sum_{\substack{I, J \in \D_{\Z}, \, J_1 = I_1^{(k_1)} \\ I_2 = J_2^{(k_2)}, \, I_3 = J_3^{(k_2-k_1)}}} 
\, |f_I|^2\bigg)^{\frac12}
\\
&\qquad\times 
\bigg(\sum_{\substack{I, J \in \D_{\Z}, \, J_1 = I_1^{(k_1)} \\ I_2 = J_2^{(k_2)}, \, I_3 = J_3^{(k_2-k_1)}}} 
\, |g_J|^2 \bigg)^{\frac12}
\\
&\lesssim \varepsilon \sum_{k_2 \ge k_1 \ge 0} 2^{-\frac{k_1}{2}} 2^{-\frac{k_2}{2}} 2^{-\frac{k_2-k_1}{2}} 
\|f\|_{L^2(\R^3)} \|g\|_{L^2(\R^3)}
\\
&\lesssim \varepsilon \|f\|_{L^2(\R^3)} \|g\|_{L^2(\R^3)},  
\end{align*}
and for each $i=1, 2$, 
\begin{align*}
\I^N_{9, i}
&\lesssim \sum_{k_2 \ge k_1 \ge 0 \atop k_i \ge N} 
2^{-\frac{k_1}{2}} 2^{-\frac{k_2}{2}} 2^{-\frac{k_2-k_1}{2}} 
\|f\|_{L^2(\R^3)} \|g\|_{L^2(\R^3)}
\\
&\lesssim 2^{-N/2} \|f\|_{L^2(\R^3)} \|g\|_{L^2(\R^3)}
\le \varepsilon \|f\|_{L^2(\R^3)} \|g\|_{L^2(\R^3)}, 
\end{align*}
provided $N>1$ sufficiently large. As a consequence, 
\begin{align*}
|\I^N_9| 
\lesssim \sum_{i=0}^2 \I^N_{9, i} 
\lesssim \varepsilon. 
\end{align*}
So far, we have proved \eqref{reduction-3}. 
\qed





%%%%%%%%%%%%%%%%%%% BIBLIGRAPHY BIBLIGRAPHY BIBLIGRAPHY %%%%%%%%%%%%%%%%%%
%%%%%%%%%%%%%%%%%%% BIBLIGRAPHY BIBLIGRAPHY BIBLIGRAPHY %%%%%%%%%%%%%%%%%%
\begin{thebibliography}{00}


\bibitem{CZ}A.P. Calder\'{o}n and A. Zygmund,
\emph{On the existence of certain singular integrals},
Acta Math. \textbf{88} (1952), 85--139.


\bibitem{CLSY}M. Cao, H. Liu, Z. Si, and K. Yabuta,
\emph{A compact bilinear $T1$ theorem},
in preparation.


\bibitem{CLLYZ}M. Cao, Z. Li, F. Liao, K. Yabuta, and J. Zhang, 
\emph{A characterization of compactness for multi-parameter singular integrals}, 
in preparation.


\bibitem{COY}M. Cao, A. Olivo, and K. Yabuta,
\emph{Extrapolation for multilinear compact operators and applications},
Trans. Amer. Math. Soc. \textbf{375} (2022), 5011--5070.


\bibitem{CY}M. Cao and K. Yabuta,
\emph{A compact multilinear $T1$ theorem on product spaces},
in preparation.


\bibitem{CYY}M. Cao, K. Yabuta, and D. Yang,
\emph{A compact extension of Journ\'{e}'s $T1$ theorem on product spaces},
\url{https://arxiv.org/abs/2303.10965}.


\bibitem{CF80}S.Y.A. Chang and R. Fefferman,
\emph{A continuous version of duality of $\mathrm{H}^1$ with $\BMO$ on the bidisc},
Ann. Math. \textbf{112} (1980), 179--201.


\bibitem{CF85}S.Y.A. Chang and R. Fefferman,
\emph{Some recent developments in Fourier analysis and $\mathrm{H}^p$ theory on product domains},
Bull. Amer. Math. Soc. \textbf{12} (1985), 1--43.


\bibitem{CK}M. Cwikel and N.J. Kalton,
\emph{Interpolation of compact operators by the methods of Calder\'{o}n and Gustavsson-Peetre},
Proc. Edin. Math. Soc. \textbf{38} (1995), 261--276.


\bibitem{DJ}G. David and J.-L. Journ\'{e},
\emph{A boundedness criterion for generalized Calder\'{o}n-Zygmund operators},
Ann. Math. \textbf{120} (1984),  371--397.


\bibitem{FJR}E. Fabes, M. Jodeit, and N. Rivi\'{e}re,
\emph{Potential techniques for boundary value problems on $C^1$-domains},
Acta Math. \textbf{141} (1978), 165--186.


\bibitem{FHHMZ}M. Fabian, P. Habala, P. H\'{a}jek, V. Montesinos, and V. Zizler,
\emph{Banach space theory. the basis for linear and nonlinear analysis},
CMS Books in Mathematics, Springer, New York, 2011.


\bibitem{Fef86}R. Fefferman,
\emph{Multi-parameter Fourier analysis },
Beijing Lectures in Harmonic Analysis. Annals of Mathematics Studies, vol. 112, pp. 47--130. 
Princeton University Press, Princeton, 1986.


\bibitem{Fef87}R. Fefferman,
\emph{Harmonic analysis on product spaces},
Ann. of Math. \textbf{126} (1987), 109--130.


\bibitem{Fef88}R. Fefferman,
\emph{$A_p$ weights and singular integrals},
Amer. J. Math. \textbf{110} (1988), 975--987.


\bibitem{FP}R. Fefferman and J. Pipher,
\emph{Multiparameter operators and sharp weighted inequalities},
Amer. J. Math. \textbf{11} (1997), 337--369.


\bibitem{FS}R. Fefferman and E. Stein,
\emph{Singular integrals on product spaces},
Adv. Math. \textbf{45} (1982), 117--143.


\bibitem{FL}S. Ferguson and M. Lacey,
\emph{A characterization of product $\BMO$ by commutators},
Acta Math. \textbf{189} (2002), 143--160.


\bibitem{G} A. Grau de la Herr\'{a}n,
\emph{Comparison of $T1$ conditions for multi-parameter operators}, 
Proc. Am. Math. Soc. \textbf{144} (2016),  2437--2443.



\bibitem{HLLT}Y. Han, J. Li, C.-C. Lin, and C. Tan,
\emph{Singular integrals associated with Zygmund dilations},
J. Geom. Anal. \textbf{29} (2019), 2410--2455.


\bibitem{HLLTW}Y. Han, J. Li, C.-C. Lin, C. Tan, and X. Wu,
\emph{Weighted endpoint estimates for singular integral operators associated with Zygmund dilations},
Taiwanese J. Math. \textbf{23} (2019), 375--408.


\bibitem{HMT}S. Hofmann, M. Mitrea, and M. Taylor,
\emph{Singular integrals and elliptic boundary problems on regular Semmes-Kenig-Toro domains},
Int. Math. Res. Not. (2010), 2567--2865.


\bibitem{HLMV}T. Hyt\"{o}nen, K. Li, H. Martikainen, and E. Vuorinen,
\emph{Multiresolution analysis and Zygmund dilations},
\url{https://arxiv.org/abs/2203.15777}.


\bibitem{Jou}J.-L. Journ\'{e},
\emph{Calder\'{o}n-Zygmund operators on product spaces},
Rev. Mat. Iberoam. \textbf{1} (1985), 55--91.


\bibitem{MPTT}C. Muscalu, J. Pipher, T. Tao, and C. Thiele,
\emph{Bi-parameter paraproducts},
Acta Math. \textbf{193} (2004), 269--296.


\bibitem{NW}A. Nagel and S. Wainger, 
\emph{$L^2$ boundedness of Hilbert transforms along surfaces and 
	  convolution operators homogeneous with respect to a multiple parameter group},  
Amer. J. Math. \textbf{99} (1977), 761--785. 


\bibitem{RS}F. Ricci and E.M. Stein,
\emph{Multiparameter singular integrals and maximal functions},
Ann. Inst. Fourier (Grenoble) \textbf{42} (1992), 637--670.


\bibitem{Ou}Y. Ou,
\emph{Multi-parameter singular integral operators and representation theorem},
Rev. Mat. Iberoam. \textbf{33} (2017), 325--350.


\bibitem{PV} S. Pott and P. Villarroya,
\emph{A $T(1)$ theorem on product spaces},
\url{https://arxiv.org/abs/1105.2516}.


\bibitem{Vil}P. Villarroya,
\emph{A characterization of compactness for singular integrals},
J. Math. Pures Appl. (9) \textbf{104} (2015), 485--532.


\end{thebibliography}


\end{document}



