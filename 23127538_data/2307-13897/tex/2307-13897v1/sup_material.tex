% This is samplepaper.tex, a sample chapter demonstrating the
% LLNCS macro package for Springer Computer Science proceedings;
% Version 2.21 of 2022/01/12
%
\documentclass[runningheads]{llncs}
%
\usepackage[T1]{fontenc}
% T1 fonts will be used to generate the final print and online PDFs,
% so please use T1 fonts in your manuscript whenever possible.
% Other font encondings may result in incorrect characters.
%
\usepackage{graphicx}
% Used for displaying a sample figure. If possible, figure files should
% be included in EPS format.
%
% If you use the hyperref package, please uncomment the following two lines
% to display URLs in blue roman font according to Springer's eBook style:
\usepackage{hyperref}
\usepackage{color}
\renewcommand\UrlFont{\color{blue}\rmfamily}
%
%--------  Siyi ------------------------------------
\newcommand\modelname{AViT}
\usepackage{multirow}
\usepackage{amsmath}
\usepackage{amssymb}
\usepackage{booktabs}
\usepackage{breqn}
\usepackage{array}
\newcolumntype{P}[1]{>{\centering\arraybackslash}p{#1}}
% ------- Siyi ------------------------------------
\begin{document}
%
\title{Supplementary Material: \modelname{}: Adapting Vision Transformers for Small Skin Lesion Segmentation Datasets}
%
\titlerunning{Supplementary Material: \modelname{}}
% If the paper title is too long for the running head, you can set
% an abbreviated paper title here
%
% \author{Paper ID: 14, Anonymous submission to ISICW 2023}
% %
% \authorrunning{Anonymous et al.}
\authorrunning{S. Du et al.}

\author{Siyi Du\inst{1}\orcidID{0000-0002-9961-4533} \and
Nourhan Bayasi\inst{1}\orcidID{0000-0003-4653-6081} \and
Ghassan Hamarneh\inst{2}\orcidID{0000-0001-5040-7448} \and
Rafeef Garbi\inst{1}\orcidID{0000-0001-6224-0876}}
% First names are abbreviated in the running head.
% If there are more than two authors, 'et al.' is used.
%
% \institute{Princeton University, Princeton NJ 08544, USA \and
% Springer Heidelberg, Tiergartenstr. 17, 69121 Heidelberg, Germany
% \email{lncs@springer.com}\\
% \url{http://www.springer.com/gp/computer-science/lncs} \and
% ABC Institute, Rupert-Karls-University Heidelberg, Heidelberg, Germany\\
% \email{\{abc,lncs\}@uni-heidelberg.de}}
% \institute{Anonymous Organization \\
% \email{abc@def.hij}}
\institute{University of British Columbia, Vancouver, British Columbia, CA \\
\email{\{siyi,nourhanb,rafeef\}@ece.ubc.ca} \\
\and Simon Fraser University, Burnaby, British Columbia, CA \\
\email{hamarneh@sfu.ca}} 
%
\maketitle              % typeset the header of the contribution
%



\begin{table*}
\centering
\caption{Skin lesion segmentation (SLS) results comparing BASE (\modelname{} w/o both adapters and the prompt generator and is fully fine-tuned), \modelname{}, and SOTA methods. We report the models' parameter count in millions (M). The 2nd column shows which pre-trained backbone the model used. R-34/50 represents ResNet-34/50.}
\label{table:SOTA}
\resizebox{\textwidth}{!}{
\begin{tabular}{|p{21mm}|P{21mm}|P{14mm}P{14mm}P{14mm}P{14mm}P{14mm}|P{14mm}P{14mm}P{14mm}P{14mm}P{14mm}|}
\hline 
\textbf{Model} & \textbf{Pre-}  & \multicolumn{10}{c|}{\textbf{Segmentation Results in Test Sets (\%)}} \\
\cline{3-12}
~ & \textbf{trained} & \multicolumn{5}{c|}{\textbf{Dice\tiny{\emph{$\pm$std}} \normalsize$\uparrow$}} & \multicolumn{5}{c|}{\textbf{IOU\tiny{\emph{$\pm$std}} \normalsize$\uparrow$}} \\
\cline{3-12}
~ & \textbf{backbone} &  ISIC & DMF & SCD & PH2 & Avg  & ISIC & DMF & SCD & PH2 & Avg\\
\hline 
\multicolumn{12}{|c|}{\textbf{(a) BASE \& Proposed Method}} \\ \hline
BASE	&	ViT-B	&	90.77\tiny{\emph{0.46}} 	&	91.69\tiny{\emph{0.79}} 	&	91.95\tiny{\emph{1.79}} 	&	95.64\tiny{\emph{0.73}} 	&	92.51\tiny{\emph{0.22}} 	&	83.71\tiny{\emph{0.70}} 	&	84.89\tiny{\emph{1.24}} 	&	85.42\tiny{\emph{2.82}} 	&	91.72\tiny{\emph{1.28}} 	&	86.43\tiny{\emph{0.34}} 	\\
\modelname{}	&	ViT-B	&	\underline{91.74}\tiny{\emph{0.64}} 	&	\underline{92.04}\tiny{\emph{0.67}} 	&	93.16\tiny{\emph{1.15}} 	&	95.66\tiny{\emph{0.56}} 	&	93.15\tiny{\emph{0.42}} 	&	\underline{85.22}\tiny{\emph{1.00}} 	&	\underline{85.47}\tiny{\emph{1.05}} 	&	87.39\tiny{\emph{1.94}} 	&	91.72\tiny{\emph{1.01}} 	&	87.45\tiny{\emph{0.70}} 	\\ \hline
\multicolumn{12}{|c|}{\textbf{(b) PEFT Methods}} \\ \hline
VPT	&	ViT-B	&	90.89\tiny{\emph{0.77}} 	&	91.26\tiny{\emph{0.58}} 	&	89.09\tiny{\emph{1.46}} 	&	93.14\tiny{\emph{0.85}} 	&	91.10\tiny{\emph{0.46}} 	&	83.83\tiny{\emph{1.17}} 	&	84.14\tiny{\emph{0.90}} 	&	80.76\tiny{\emph{2.23}} 	&	87.27\tiny{\emph{1.48}} 	&	84.00\tiny{\emph{0.74}} 	\\
AdaptFormer	&	ViT-B	&	91.12\tiny{\emph{0.74}} 	&	91.27\tiny{\emph{0.65}} 	&	89.65\tiny{\emph{0.75}} 	&	93.76\tiny{\emph{0.73}} 	&	91.45\tiny{\emph{0.42}} 	&	84.15\tiny{\emph{1.15}} 	&	84.18\tiny{\emph{1.00}} 	&	81.49\tiny{\emph{1.13}} 	&	88.33\tiny{\emph{1.30}} 	&	84.54\tiny{\emph{0.67}} 	\\ \hline

\multicolumn{12}{|c|}{\textbf{(c) SLS Methods w/o Pre-trained Backbones \& Trained From Scratch}} \\ \hline
SwinUnet	&	None	&	89.64\tiny{\emph{0.39}} 	&	90.67\tiny{\emph{0.73}} 	&	89.77\tiny{\emph{2.08}} 	&	94.24\tiny{\emph{0.93}} 	&	91.08\tiny{\emph{0.57}} 	&	81.94\tiny{\emph{0.62}} 	&	83.19\tiny{\emph{1.14}} 	&	82.07\tiny{\emph{3.02}}	&	89.24\tiny{\emph{1.59}} 	&	84.11\tiny{\emph{0.79}} 	\\
UNETR  &  None  & 89.60\tiny{\emph{0.61}} 	&	90.53\tiny{\emph{1.03}} 	&	88.13\tiny{\emph{3.67}} 	&	93.92\tiny{\emph{0.94}} 	&	90.55\tiny{\emph{0.87}} 	&	81.86\tiny{\emph{0.84}} 	&	83.02\tiny{\emph{1.60}} 	&	79.96\tiny{\emph{5.23}} 	&	88.68\tiny{\emph{1.58}} 	&	83.38\tiny{\emph{1.24}} 	  \\ 
UTNet	&	None	&	89.68\tiny{\emph{0.90}} 	&	89.87\tiny{\emph{0.58}} 	&	88.11\tiny{\emph{3.20}} 	&	93.29\tiny{\emph{1.08}} 	&	90.23\tiny{\emph{0.61}} 	&	81.99\tiny{\emph{1.25}} 	&	81.91\tiny{\emph{0.92}} 	&	79.71\tiny{\emph{4.34}} 	&	87.62\tiny{\emph{1.80}} 	&	82.81\tiny{\emph{0.77}} 	\\
MedFormer	&	None	&	90.47\tiny{\emph{0.68}} 	&	90.85\tiny{\emph{0.60}} 	&	90.60\tiny{\emph{2.56}} 	&	94.82\tiny{\emph{0.81}} 	&	91.68\tiny{\emph{0.74}} 	&	83.22\tiny{\emph{0.96}} 	&	83.52\tiny{\emph{0.95}} 	&	83.53\tiny{\emph{3.94}} 	&	90.23\tiny{\emph{1.44}} 	&	85.13\tiny{\emph{1.12}} 	\\
Swin UNETR	&	None	&	90.19\tiny{\emph{0.50}} 	&	91.00\tiny{\emph{0.72}} 	&	90.71\tiny{\emph{2.19}} 	&	94.54\tiny{\emph{0.77}} 	&	91.61\tiny{\emph{0.49}} 	&	82.78\tiny{\emph{0.76}} 	&	83.77\tiny{\emph{1.11}} 	&	83.54\tiny{\emph{3.45}} 	&	89.74\tiny{\emph{1.34}} 	&	84.96\tiny{\emph{0.74}} 	\\ \hline

\multicolumn{12}{|c|}{\textbf{(d) SLS Methods w/ Pre-trained Backbones \& Fully Fine-tuned}} \\ \hline
H2Former	&	R-34	&	91.17\tiny{\emph{0.72}} 	&	91.29\tiny{\emph{0.88}} 	&	92.76\tiny{\emph{1.81}} 	&	95.65\tiny{\emph{0.74}} 	&	92.72\tiny{\emph{0.63}} 	&	84.35\tiny{\emph{1.08}} 	&	84.22\tiny{\emph{1.38}} 	&	87.04\tiny{\emph{2.70}} 	&	91.77\tiny{\emph{1.30}} 	&	86.85\tiny{\emph{0.91}} 	\\
FAT-Net	&	R-34, DeiT-T	&	91.26\tiny{\emph{0.63}} 	&	91.32\tiny{\emph{0.73}} 	&	93.03\tiny{\emph{1.32}} 	&	96.07\tiny{\emph{0.52}} 	&	92.92\tiny{\emph{0.48}} 	&	84.42\tiny{\emph{0.98}} 	&	84.25\tiny{\emph{1.17}} 	&	87.23\tiny{\emph{2.19}} 	&	92.48\tiny{\emph{0.94}} 	&	87.10\tiny{\emph{0.80}} 	\\
BAT	&	R-50		&	91.33\tiny{\emph{0.68}} 	&	91.20\tiny{\emph{0.80}} 	&	92.95\tiny{\emph{1.40}} 	&	95.84\tiny{\emph{0.43}} 	&	92.83\tiny{\emph{0.46}} 	&	84.40\tiny{\emph{1.09}} 	&	84.03\tiny{\emph{1.29}} 	&	87.08\tiny{\emph{2.39}} 	&	92.04\tiny{\emph{0.78}} 	&	86.89\tiny{\emph{0.78}} 	\\
TransFuse	&	R-50, DeiT-B	&	91.73\tiny{\emph{0.51}} 	&	91.96\tiny{\emph{0.71}} 	&	\underline{94.11}\tiny{\emph{1.03}} 	&	\underline{96.18}\tiny{\emph{0.42}} 	&	\underline{93.50}\tiny{\emph{0.27}} 	&	\underline{85.22}\tiny{\emph{0.79}} 	&	85.33\tiny{\emph{1.15}} 	&	\underline{89.03}\tiny{\emph{1.75}} 	&	\underline{92.69}\tiny{\emph{0.76}} 	&	\underline{88.07}\tiny{\emph{0.47}} 	\\ \hline
\end{tabular}}
\end{table*}




\begin{table*}
\centering
\caption{Experiments using different pre-trained ViT backbones and ablation study of \modelname{}. $^*$ means the pre-trained backbone is frozen throughout training. $^{-P}$ or $^{-A}$ represent not using the prompt generator or adapters in \modelname{}.}
\label{table:ablation}
\resizebox{\textwidth}{!}{
\begin{tabular}{|p{15mm}|P{16mm}|P{14mm}P{14mm}P{14mm}P{14mm}P{14mm}|P{14mm}P{14mm}P{14mm}P{14mm}P{14mm}|}
\hline 
\textbf{Model} &  \textbf{Pre-} & \multicolumn{10}{c|}{\textbf{Segmentation Results in Test Sets (\%)}} \\
\cline{3-12}
~ & \textbf{trained} & \multicolumn{5}{c|}{\textbf{Dice\tiny{\emph{$\pm$std}} \normalsize$\uparrow$}} & \multicolumn{5}{c|}{\textbf{IOU\tiny{\emph{$\pm$std}} \normalsize$\uparrow$}} \\
\cline{3-12}
~ & \textbf{backbone} &  ISIC & DMF & SCD & PH2 & Avg  & ISIC & DMF & SCD & PH2 & Avg\\
\hline
\multicolumn{12}{|c|}{\textbf{(a) Applicability to Various Pre-trained ViT Backbones}} \\ \hline
BASE	&	Swin-B	&	91.63\tiny{\emph{0.71}} 	&	91.70\tiny{\emph{0.62}} 	&	92.71\tiny{\emph{0.99}} 	&	95.88\tiny{\emph{0.34}} 	&	92.98\tiny{\emph{0.37}} 	&	85.05\tiny{\emph{1.11}} 	&	84.89\tiny{\emph{0.99}} 	&	86.60\tiny{\emph{1.61}} 	&	92.13\tiny{\emph{0.61}} 	&	87.17\tiny{\emph{0.61}}  	\\
\modelname{}	&	Swin-B	&	91.54\tiny{\emph{0.60}} 	&	91.73\tiny{\emph{0.71}} 	&	93.60\tiny{\emph{0.95}} 	&	95.68\tiny{\emph{0.70}} 	&	93.14\tiny{\emph{0.39}}  	&	84.90\tiny{\emph{0.96}} 	&	84.94\tiny{\emph{1.12}} 	&	88.12\tiny{\emph{1.62}} 	&	91.77\tiny{\emph{1.26}} 	&	87.43\tiny{\emph{0.64}}  	\\
BASE	&	Swin-L	&	91.64\tiny{\emph{0.65}} 	&	91.69\tiny{\emph{0.65}} 	&	92.93\tiny{\emph{1.03}} 	&	95.83\tiny{\emph{0.46}} 	&	93.02\tiny{\emph{0.25}}  	&	85.08\tiny{\emph{1.05}} 	&	84.86\tiny{\emph{1.04}} 	&	86.97\tiny{\emph{1.66}} 	&	92.04\tiny{\emph{0.82}} 	&	87.24\tiny{\emph{0.43}} 	\\
\modelname{}	&	Swin-L	&	91.56\tiny{\emph{0.63}} 	&	91.91\tiny{\emph{0.56}} 	&	93.74\tiny{\emph{0.98}} 	&	96.07\tiny{\emph{0.50}} 	&	93.32\tiny{\emph{0.31}}  	&	84.93\tiny{\emph{1.00}} 	&	85.24\tiny{\emph{0.87}} 	&	88.38\tiny{\emph{1.69}} 	&	92.47\tiny{\emph{0.92}} 	&	87.76\tiny{\emph{0.50}}  	\\
BASE	&	ViT-L	&	91.37\tiny{\emph{0.78}} 	&	91.76\tiny{\emph{0.81}} 	&	93.23\tiny{\emph{1.04}} 	&	95.86\tiny{\emph{0.59}} 	&	93.06\tiny{\emph{0.29}}  	&	84.60\tiny{\emph{1.18}} 	&	84.99\tiny{\emph{1.29}} 	&	87.52\tiny{\emph{1.65}} 	&	92.09\tiny{\emph{1.06}} 	&	87.30\tiny{\emph{0.47}}  	\\
\modelname{}	&	ViT-L	&	91.54\tiny{\emph{0.59}} 	&	91.77\tiny{\emph{0.71}} 	&	93.48\tiny{\emph{1.16}} 	&	95.73\tiny{\emph{0.60}} 	&	93.13\tiny{\emph{0.48}}  	&	84.88\tiny{\emph{0.95}} 	&	85.01\tiny{\emph{1.11}} 	&	87.94\tiny{\emph{1.94}} 	&	91.85\tiny{\emph{1.08}} 	&	87.42\tiny{\emph{0.79}}  	\\  
BASE	&	DeiT-B	&		91.48\tiny{\emph{0.72}} 	&	91.82\tiny{\emph{0.85}} 	&	93.63\tiny{\emph{1.23}} 	&	95.83\tiny{\emph{0.45}} 	&	92.94\tiny{\emph{0.32}}  	&	84.77\tiny{\emph{1.10}}	&	85.10\tiny{\emph{1.36}} 	&	86.53\tiny{\emph{1.92}} 	&	92.04\tiny{\emph{0.81}} 	&	87.11\tiny{\emph{0.52}}  	\\
\modelname{}	&	DeiT-B	&		91.70\tiny{\emph{0.65}} 	&	91.85\tiny{\emph{0.82}} 	&	93.67\tiny{\emph{0.88}} 	&	95.97\tiny{\emph{0.46}} 	&	93.30\tiny{\emph{0.31}}  	&	85.14\tiny{\emph{1.05}} 	&	85.17\tiny{\emph{1.28}} 	&	88.22\tiny{\emph{1.51}} 	&	92.30\tiny{\emph{0.84}} 	&	87.71\tiny{\emph{0.51}}  	\\  
\hline

\multicolumn{12}{|c|}{\textbf{(b) Ablation Study}} \\ \hline
BASE$^*$	&	ViT-B	
&	87.18\tiny{\emph{0.84}} 	&	89.23\tiny{\emph{0.88}} 	&	86.24\tiny{\emph{1.55}} 	&	90.17\tiny{\emph{1.18}} 	&	88.20\tiny{\emph{0.46}}  	&	77.92\tiny{\emph{1.14}} 	&	80.81\tiny{\emph{1.31}} 	&	76.27\tiny{\emph{2.25}} 	&	82.30\tiny{\emph{1.87}} 	&	79.33\tiny{\emph{0.65}}  	\\
\modelname{}$^{-P}$	&	ViT-B	
&	91.47\tiny{\emph{0.63}} 	&	91.80\tiny{\emph{0.58}} 	&	91.18\tiny{\emph{0.79}} 	&	94.75\tiny{\emph{0.65}} 	&	92.30\tiny{\emph{0.31}}  	&	84.74\tiny{\emph{0.96}} 	&	85.04\tiny{\emph{0.90}} 	&	83.98\tiny{\emph{1.26}} 	&	90.09\tiny{\emph{1.16}} 	&	85.96\tiny{\emph{0.48}}  	\\
\modelname{}$^{-A}$	&	ViT-B	
&	90.87\tiny{\emph{0.72}} 	&	91.00\tiny{\emph{0.68}} 	&	89.09\tiny{\emph{3.62}} 	&	93.87\tiny{\emph{0.64}} 	&	91.21\tiny{\emph{0.83}}  	&	83.78\tiny{\emph{1.10}} 	&	83.72\tiny{\emph{1.06}} 	&	81.18\tiny{\emph{5.16}} 	&	88.53\tiny{\emph{1.14}} 	&	84.30\tiny{\emph{1.19}}  	\\
\modelname{}	&	ViT-B	&	\underline{91.74}\tiny{\emph{0.64}} 	&	\underline{92.04}\tiny{\emph{0.67}} 	&	\underline{93.16}\tiny{\emph{1.15}} 	&	\underline{95.66}\tiny{\emph{0.56}} 	&	\underline{93.15}\tiny{\emph{0.42}} 	&	\underline{85.22}\tiny{\emph{1.00}} 	&	\underline{85.47}\tiny{\emph{1.05}} 	&	\underline{87.39}\tiny{\emph{1.94}} 	&	\underline{91.72}\tiny{\emph{1.01}} 	&	\underline{87.45}\tiny{\emph{0.70}} 	\\ \hline
\end{tabular}}
\end{table*}








% \section{First Section}
% \subsection{A Subsection Sample}
% Please note that the first paragraph of a section or subsection is
% not indented. The first paragraph that follows a table, figure,
% equation etc. does not need an indent, either.

% Subsequent paragraphs, however, are indented.

% \subsubsection{Sample Heading (Third Level)} Only two levels of
% headings should be numbered. Lower level headings remain unnumbered;
% they are formatted as run-in headings.

% \paragraph{Sample Heading (Fourth Level)}
% The contribution should contain no more than four levels of
% headings. Table~\ref{tab1} gives a summary of all heading levels.

% \begin{table}
% \caption{Table captions should be placed above the
% tables.}\label{tab1}
% \begin{tabular}{|l|l|l|}
% \hline
% Heading level &  Example & Font size and style\\
% \hline
% Title (centered) &  {\Large\bfseries Lecture Notes} & 14 point, bold\\
% 1st-level heading &  {\large\bfseries 1 Introduction} & 12 point, bold\\
% 2nd-level heading & {\bfseries 2.1 Printing Area} & 10 point, bold\\
% 3rd-level heading & {\bfseries Run-in Heading in Bold.} Text follows & 10 point, bold\\
% 4th-level heading & {\itshape Lowest Level Heading.} Text follows & 10 point, italic\\
% \hline
% \end{tabular}
% \end{table}


% \noindent Displayed equations are centered and set on a separate
% line.
% \begin{equation}
% x + y = z
% \end{equation}
% Please try to avoid rasterized images for line-art diagrams and
% schemas. Whenever possible, use vector graphics instead (see
% Fig.~\ref{fig1}).

% % Figure environment removed

% \begin{theorem}
% This is a sample theorem. The run-in heading is set in bold, while
% the following text appears in italics. Definitions, lemmas,
% propositions, and corollaries are styled the same way.
% \end{theorem}
% %
% % the environments 'definition', 'lemma', 'proposition', 'corollary',
% % 'remark', and 'example' are defined in the LLNCS documentclass as well.
% %
% \begin{proof}
% Proofs, examples, and remarks have the initial word in italics,
% while the following text appears in normal font.
% \end{proof}
% For citations of references, we prefer the use of square brackets
% and consecutive numbers. Citations using labels or the author/year
% convention are also acceptable. The following bibliography provides
% a sample reference list with entries for journal
% articles~\cite{ref_article1}, an LNCS chapter~\cite{ref_lncs1}, a
% book~\cite{ref_book1}, proceedings without editors~\cite{ref_proc1},
% and a homepage~\cite{ref_url1}. Multiple citations are grouped
% \cite{ref_article1,ref_lncs1,ref_book1},
% \cite{ref_article1,ref_book1,ref_proc1,ref_url1}.

% \subsubsection{Acknowledgements} Please place your acknowledgments at
% the end of the paper, preceded by an unnumbered run-in heading (i.e.
% 3rd-level heading).

%
% ---- Bibliography ----
%
% BibTeX users should specify bibliography style 'splncs04'.
% References will then be sorted and formatted in the correct style.
%
% \bibliographystyle{splncs04}
% \bibliography{mybibliography}
%
% \begin{thebibliography}{8}
% \bibitem{ref_article1}
% Author, F.: Article title. Journal \textbf{2}(5), 99--110 (2016)

% \bibitem{ref_lncs1}
% Author, F., Author, S.: Title of a proceedings paper. In: Editor,
% F., Editor, S. (eds.) CONFERENCE 2016, LNCS, vol. 9999, pp. 1--13.
% Springer, Heidelberg (2016). \doi{10.10007/1234567890}

% \bibitem{ref_book1}
% Author, F., Author, S., Author, T.: Book title. 2nd edn. Publisher,
% Location (1999)

% \bibitem{ref_proc1}
% Author, A.-B.: Contribution title. In: 9th International Proceedings
% on Proceedings, pp. 1--2. Publisher, Location (2010)

% \bibitem{ref_url1}
% LNCS Homepage, \url{http://www.springer.com/lncs}. Last accessed 4
% Oct 2017
% \end{thebibliography}
\end{document}
