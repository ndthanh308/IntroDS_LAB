\documentclass[10pt,twocolumn,letterpaper]{article}

\usepackage{iccv}
\usepackage{times}
\usepackage{epsfig}
\usepackage{graphicx}
\usepackage{amsmath}
\usepackage{amssymb}
% Include other packages here, before hyperref.

\usepackage{graphicx}
\usepackage{amsmath}
\usepackage{amssymb}
\usepackage{booktabs}
\DeclareMathOperator*{\argmax}{\arg\!\max}
\usepackage[table]{xcolor}
\definecolor{bestcolor}{rgb}{1.0,1.0,1.0} %{1,0.5,0.25}
\definecolor{secondcolor}{rgb}{1.0,1.0,1.0} %{1,0.85,0.3}
\definecolor{thirdcolor}{rgb}{1.0,1.0,1.0} %{1.0,0.9,0.7}
\newcommand{\bone}{\cellcolor{bestcolor}}
\newcommand{\btwo}{\cellcolor{secondcolor}}
\newcommand{\bthird}{\cellcolor{thirdcolor}}
% \definecolor{initcolor}{rgb}{.830,.550,.400} %{1,0.85,0.3}
% \definecolor{optcolor}{rgb}{.400,0.490,.820} %{1.0,0.9,0.7}
\usepackage[pagebackref=true,breaklinks=true,letterpaper=true,colorlinks,bookmarks=false]{hyperref}
% If you comment hyperref and then uncomment it, you should delete
% egpaper.aux before re-running latex.  (Or just hit 'q' on the first latex
% run, let it finish, and you should be clear).

%\usepackage{subfigure}
\usepackage{url}            % simple URL typesetting
\usepackage{booktabs}       % professional-quality tables
\usepackage{amsfonts}       % blackboard math symbols
\usepackage{times}
\usepackage{epsfig}
\usepackage{amsmath}
\usepackage{amssymb}
\usepackage{multirow}
\usepackage{subcaption}
% \usepackage[linesnumbered,ruled]{algorithm2e}
% \usepackage{pbox}
% \usepackage[skip=2pt]{caption}
% \usepackage{makecell}
% \usepackage[export]{adjustbox}    
% \usepackage{diagbox}             % added
\usepackage{cuted}
\usepackage{changepage}
% \usepackage{wrapfig}
\usepackage{amsthm}
% \usepackage{geometry}
\usepackage{xcolor}
% \usepackage{setspace}
% \usepackage{mathtools}
% \usepackage{enumitem}
% \usepackage{capt-of}
% \usepackage{booktabs}
% \usepackage{microtype}      % microtypography
\usepackage{enumitem}
\usepackage{tikz}


% Support for easy cross-referencing
\usepackage[capitalize]{cleveref}
\crefname{section}{Sec.}{Secs.}
\Crefname{section}{Section}{Sections}
\Crefname{table}{Table}{Tables}
\crefname{table}{Tab.}{Tabs.}
\usepackage{appendix}

\newcommand{\methodname}{Strivec}



\iccvfinalcopy % *** Uncomment this line for the final submission

\def\iccvPaperID{3306} % *** Enter the ICCV Paper ID here
\def\httilde{\mbox{\tt\raisebox{-.5ex}{\symbol{126}}}}

% Pages are numbered in submission mode, and unnumbered in camera-ready
\ificcvfinal\pagestyle{empty}\fi

% Customized command
\newcommand{\boldstart}[1]{\noindent\textbf{#1}}
\newcommand{\boldstartspace}[1]{\vspace{0.05in}\noindent\textbf{#1}}

\newcommand{\zexiang}[1]{{\color{blue}{Zexiang: #1}}}
\newcommand{\qg}[1]{{\color{red}{Qiangeng: #1}}}
\newcommand{\qk}[1]{{\color{orange}{Quankai: #1}}}
\newcommand{\hs}[1]{{\color{magenta}{Hao: #1}}}

\newcommand{\first}{\tikz\draw[yellow,fill=yellow] (0,0) circle (0.7ex);}
\newcommand{\second}{\tikz\draw[lightgray,fill=lightgray] (0,0) circle (0.7ex);}
\newcommand{\third}{\tikz\draw[brown,fill=brown] (0,0) circle (0.7ex);}

\begin{document}

%%%%%%%%% TITLE
\title{Strivec: Sparse Tri-Vector Radiance Fields}

\author{Quankai Gao$^{*1}$ \qquad Qiangeng Xu$^{*1}$ \qquad Hao Su$^{2}$ \qquad Ulrich Neumann$^{1}$ \qquad Zexiang Xu$^{3}$  \\
    \hspace{0mm}$^1$University of Southern California \hspace{18mm} 
    $^2$UC San Diego\hspace{18mm}
    $^3$Adobe Research\\
    % \hspace{0mm}Los Angeles, California \hspace{28mm} San Diego, California\\
    {\tt\small \hspace{0mm}\{quankaig,qiangenx,uneumann\}@usc.edu}\hspace{5mm}
    {\tt\small haosu@ucsd.edu}\hspace{5mm}
    {\tt\small zexu@adobe.com}\qquad
}

\maketitle
% Remove page # from the first page of camera-ready.
\ificcvfinal\thispagestyle{empty}\fi
\maketitle
\let\thefootnote\relax\footnotetext{$^*$Equal contribution.\\
Code and results: \href{https://github.com/Zerg-Overmind/Strivec}{https://github.com/Zerg-Overmind/Strivec}}

%%%%%%%%% ABSTRACT
\begin{abstract}
   We propose \methodname{}, a novel neural representation that models a 3D scene as a radiance field with sparsely distributed and compactly factorized local tensor feature grids.
   Our approach leverages tensor decomposition, following the recent work TensoRF~\cite{chen2022tensorf}, to model the tensor grids.  In contrast to TensoRF which uses a global tensor and focuses on their vector-matrix decomposition,  we propose to utilize a cloud of local tensors and apply the classic CANDECOMP/PARAFAC (CP) decomposition \cite{carroll1970analysis} to factorize each tensor into triple vectors that express local feature distributions along spatial axes and compactly encode a local neural field.
   We also apply multi-scale tensor grids to discover the geometry and appearance commonalities and exploit spatial coherence with the tri-vector factorization at multiple local scales.  The final radiance field properties are regressed by aggregating neural features from multiple local tensors across all scales. Our tri-vector tensors are sparsely distributed around the actual scene surface, discovered by a fast coarse reconstruction, leveraging the sparsity of a 3D scene.
   We demonstrate that our model can achieve better rendering quality while using significantly fewer parameters than previous methods, including TensoRF and Instant-NGP~\cite{muller2022instant}.
   
   
\end{abstract}

%\vspace{-25pt}
% Figure environment removed

\section{Introduction}
Automatic 3D reconstruction of clothed humans using image inputs has gained increasing significance due to its potential applications in a wide array of AR/VR scenarios. High-fidelity reconstructions typically depend on sophisticated capture systems, which are developed with dense camera arrays~\cite{collet2015high,joo2015panoptic,joo2018total}, programmable light-stages~\cite{Vlasic2009, guo2019relightables}, and depth sensors~\cite{newcombe2011kinectfusion,DoubleFusion,BodyFusion,dou2016fusion4d,newcombe2015dynamicfusion}. However, stringent capture environments equipped with complex hardware pose significant challenges for consumer-level applications.


In this context, considerable research effort has been dedicated to developing methods that allow for more flexible capture configurations, such as utilizing a few RGB inputs. Among these works, learning implicit functions \cite{iccv2020PIFu, saito2020pifuhd, hong2021stereopifu} has proven effective in achieving highly detailed reconstructions by integrating the advancements of deep neural networks. These methods employ large multi-layer perceptrons (MLPs) to predict the occupancy probability or truncated signed distance function (TSDF) value of every queried 3D point based on its associated local feature, which is extracted from images. They can recover a continuous surface at arbitrary resolutions without topology restrictions.


However, in typical MLP-based implicit networks, the occupancy or TSDF value at each location is solved independently with planar image features, rendering them less capable of addressing challenging cases such as occlusions. Consequently, these methods suffer from generalization and robustness issues, particularly when tackling strong occlusions caused by large motion or multiple interacting humans. 
Some follow-up studies  \cite{zheng2021deepmulticap,zheng2021pamir,huang2020arch} utilize an extra geometric model, SMPL~\cite{Loper2015}, to improve robustness by introducing strong shape priors. 
Their success typically relies on the assumption of geometrical similarity \cite{huang2020arch} between the shape prior and target reconstruction, making them intractable for handling complex cases with loose clothes and sensitive to errors in SMPL model fitting.



%\ping{this paragraph sounds like `TSDF is better than MLP/SMPL, and we use TSDF to solve the problem'. But in Sec 3, we are telling a different story, saying `MLP needs a 3D convolutional encoder'. We need to make these two sections consistent.}\sicong{I think in this paragraph we claim that the TSDF}


%We opt for Trucated Signed Distance Funtion (TSDF) volumetric representations as they are naturally suitable for convolution operations, which have shown remarkable performance for learning hierarchical features on 2D visual perception tasks \cite{SunXLW19}. 
%Meanwhile, TSDF also describes the gradual geometry change around shape surface, which is not reflected by occupancy volume. 

We instead revisit the 3D volumetric representation and resort to 3D convolutional neural networks (CNNs) for feature learning, due to their impressive performance in feature learning and the ability to incorporate spatial context. However, volumetric methods and 3D convolution involve discretization, which might raise concerns regarding whether a discretized volume can preserve subtle geometric details as continuous representations learned in implicit functions. We investigate the relationship between volume resolution and quantization error on synthetic data by converting target mesh objects to TSDF volumes, as shown in Figure~\ref{fig:quantization_error}. We observe that the quantization errors are significantly reduced by increasing volume resolution and become nearly negligible when reaching a relatively high resolution (e.g., 512 or higher). In other words, achieving fine-detailed reconstruction is not supposed to be restricted by the use of volume representations as long as a proper volume resolution is utilized. Therefore, we present a method with high-resolution feature volumes, e.g., 256 and 512, while traditional volumetric methods \cite{varol18_bodynet,gilbert2018volumetric} are often limited to much lower resolutions, such as 32 or 128.



On the other hand, an increase in volume resolution may lead to a cubic growth of memory overhead \cite{8100085}. Reducing memory costs while guaranteeing the granularity of volumetric representations is necessary for pursuing high-quality reconstruction. Thus, we adopt a coarse-to-fine approach and cull away irrelevant voxels to build a sparse high-resolution feature volume. At the coarse level, the network computes an initial TSDF by applying a U-Net with sparse 3D CNN \cite{3DSemanticSegmentationWithSubmanifoldSparseConvNet} on the sparse feature volume, which is carved by a visual hull. Through our experiments, it turns out that more than 95\% of the volume grids are discarded by the visual hull culling, making the sparse 3D CNN efficient. At the fine level, the network focuses on a narrow band near the zero-level set of the initial TSDF and discretizes the narrow band with smaller voxels. By employing this narrow-band culling, we further shrink the sampling space, resulting in a relatively small range of grid numbers (usually 300K--500K in our experiments) even with a high volume resolution of 512. The remaining voxels in the narrow band are associated with features that fuse high-frequency information from the computed normal maps upon the low-frequency shape from the coarse level to compute the TSDF at high resolution. The final mesh is then extracted from the TSDF using the Marching-Cube algorithm ~\cite{Lorensen87marchingcubes}.
% Different from the u-net sturcture to preserve global topology context, we then apply a shallow 3dcnn to compute the final TSDF $D_{final}$ which contain more local geometry detail.




% \ping{this paragraph can be expanded. It is an important contribution and often ignored by other works. stress on the novel idea of regressing blending weights instead of colors}

In addition to geometry, high-quality mesh texture is also a crucial factor contributing to visual appearance. Directly computing a color field in 3D space, as in \cite{iccv2020PIFu}, struggles to capture high-frequency texture details, while the neural radiance field (NeRF) \cite{yu2020pixelnerf} or the DoubleField~\cite{shao2022doublefield} require expensive per-instance optimization and are often unstable for sparse input images. In contrast, we adopt an image-based rendering approach to compute a texture atlas map, which is efficient and widely supported in existing computer graphics tools. 
Specifically, we compute a blending weight at each 3D point on the mesh surface to determine its color as a weighted average of the colors at its image projections. The blending weights can be computed at a relatively coarse resolution, e.g., 512 volume resolution in our case, and leave texture details to the high-resolution images, such as 1K or 2K. Unlike previous methods that generate blurry texturing results under sparse input, our method generalizes well on both synthetic and real data with just a few input views. 
Figure~\ref{fig:teaser} shows two examples reconstructed by our method. Despite the challenging garment, pose, and occlusion, our method recovers faithful shape, normal, and texture on the right.

%with a wide variety of poses and clothing styles, and it is also adaptive to handle input image with arbitrary resolutions.
%\sicong{For this concern we claim that when the resolution of dicretized volume meets certain threshold (which is 256 in our experiment), the quantization error can be neglected.} 



In summary, the main contributions of this paper are as follows:
\begin{itemize}
\vspace{-0.1in}
  \item 
  We revisit the 3D volumetric representation and demonstrate that it can support clothed human reconstruction with equal or even better performance compared to implicit representation. 
  \item 
  We develop a memory and computation-efficient method for high-resolution volumetric reconstruction using sophisticated sparse 3D CNN, coarse-to-fine estimation, and voxel culling by visual hull and narrow bands. 
  \item 
  We introduce a novel method to compute a texture atlas map, which captures rich appearance details from high-resolution input images.
  \item 
  We achieve impressive results on standard benchmark datasets Twindom and MultiHuman, significantly reducing the point-2-surface (P2S) precision to approximately 0.2cm from just six input views, with more than $50\%$ error reduction compared to the state-of-the-art methods, including DoubleField~\cite{shao2022doublefield} and PIFuHD~\cite{saito2020pifuhd}.
\end{itemize}
\section{Related Work}
\label{appsec: related work}
Bayesian causal discovery literature has primarily focused on inference in linear models with closed-form posteriors or marginalized parameters. Early works considered sampling directed acyclic graphs (DAGs) for discrete~\cite{cooper1992bayesian, madigan1995bayesian, heckerman2006bayesian} and Gaussian random variables~\cite{friedman2003being, tong2001active} using Markov chain Monte Carlo (MCMC) in the DAG space. However, these approaches exhibit slow mixing and convergence~\cite{eaton2012bayesian,grzegorczyk2008improving}, often requiring restrictions on number of parents~\cite{kuipers2017partition}. %Alternative exact dynamic programming methods are limited to small settings~\cite{koivisto2012advances}. 

Recent advances in variational inference~\cite{zhang2018advances} have facilitated graph inference in DAG space, with gradient-based methods employing the NOTEARS DAG penalty \cite{zheng2018dags}.\cite{annadani2021variational} samples DAGs from autoregressive adjacency matrix distributions, while \cite{lorch2021dibs} utilizes Stein variational approach \cite{liu2016stein} for DAGs and causal model parameters. \cite{cundy2021bcd} proposed a variational inference framework on node orderings using the gumbel-sinkhorn gradient estimator \cite{mena2018learning}. \cite{deleu2022bayesian,nishikawa2022bayesian} employ the GFlowNet framework \cite{bengio2021gflownet} for inferring the DAG posterior. Most methods, except\cite{lorch2021dibs} are restricted to linear models, while \cite{lorch2021dibs} has high computational costs and lacks DAG generation guarantees compared to our method.
% at least quadratic scaling complexity, both with respect to the number of nodes (due to the DAG penalty) as well as number of posterior samples. Our proposed approach instead has linear complexity with respect to number of posterior samples and does not require any additional DAG penalty.     

In contrast, \emph{quasi-Bayesian} methods, such as DAG bootstrap \cite{friedman2013data}, demonstrate competitive performance. DAG bootstrap resamples data and estimates a single DAG using PC \cite{spirtes2000causation}, GES \cite{chickering2002optimal}, or similar algorithms, weighting the obtained DAGs by their unnormalized posterior probabilities. Recent neural network-based works employ variational inference to learn DAG distributions and point estimates for nonlinear model parameters \cite{charpentier2022differentiable,geffner2022deep}.
\section{Method} \label{method_hybridaugment}
In this section, we formally define the problem, motivate our work and then present our proposed techniques.


\subsection{Preliminaries}
Let $\mathcal{F}(x;W)$ be an image classification CNN trained on the training set $\mathcal{T}_\text{train} = (x_{i}, y_{i})^{N}_{i=1}$  with $N$ samples, where $x$ and $y$ correspond to images and labels. The clean accuracy (CA) of $\mathcal{F}(x;W)$ is formally defined as its accuracy over a clean test set $\mathcal{T}_\text{test} = (x_{j}, y_{j})^{M}_{j=1}$. Assume two operators ${A}(\cdot)$ and ${C}(c, s)$ that adversarially attacks or corrupts a given set of images with the corruption category $c$ and severity $s$, respectively.  Let $A\mathcal{T}_\text{test}$ and $C\mathcal{T}_\text{test}$ be the adversarially attacked and corrupted versions of $\mathcal{T}_\text{test}$, and let $\mathcal{F}(x;W)$ have a robust accuracy (RA) on $A\mathcal{T}_\text{test}$ and a corruption accuracy (CRA) on $C\mathcal{T}_\text{test}$. 
The aim is to fit $\mathcal{F}(x;W)$ such that the model gains robustness (\ie. increased RA and CRA compared its the baseline version), while retaining (or improving) the clean accuracy of its baseline version trained without robustness concerns.


\noindent \textbf{What we know.} Our work builds on the following crucial observations: i) CNNs favour high-frequency content \cite{wang2020high}, ii) adversaries and corruptions often reside in high-frequency \cite{wang2020towards}, iii) images are dominated by low-frequency \cite{Saikia_2021_ICCV} and iv) models relying on low-frequency components are more robust \cite{li2022robust,wang2020towards}. The robustness-accuracy trade-off is visible; low-frequency reliant models are more robust, but tend to miss out on clean accuracy brought by the high-frequency components. 

\subsection{HybridAugment}
We hypothesize that a \textit{sweet spot} in the robustness-accuracy trade-off can be found. Unlike the \textit{hard} approaches that completely rule out the reliance on high-frequency components (i.e. low-pass filters), we propose to \textit{reduce} the reliance on them. To this end, we adopt a data augmentation approach that aims to diversify $\mathcal{T}_\text{train}$ by an operation $\mathcal{HA(\cdot)}$. Keeping the strong relation intact between labels and low-frequency content (i.e. labels come from low-frequency-component image), we propose to swap high and low-frequency components of images in a batch on-the-fly. Unlike \cite{mukai2022improving}, we \textit{do not} restrict the images to belong to the same class; this diversifies the training distribution even further while preserving the image semantics. We call this basic version of our approach \textit{HybridAugment}, which corresponds to: 
%
\begin{equation} \label{hybrid_augment_paired}
    \mathcal{HA_{P}}(x_{i}, x_{j}) = \mathcal{LF}(x_{i}) + \mathcal{HF}(x_{j})
\end{equation}
%
where $x_{i}$ is the input image and $x_{j}$ is a randomly sampled image from the whole training set, which we simply sample from the mini batch at each training iteration in practice. $\mathcal{HF}$ and $\mathcal{LF}$ operators select the high and low-frequency components of an input image, for which we use:
%
\begin{equation} \label{eq:cutoff}
\begin{split}
    \mathcal{LF}(x) = GaussBlur(x) \\
    \mathcal{HF}(x) = x - \mathcal{LF}(x)
    \end{split}
\end{equation}
%
where $GaussBlur$ is used as a low-pass filter. Note that a similar outcome is possible by using Discrete Fourier Transforms (DFT), swapping the frequency bands and then applying Inverse DFT (IDFT). We find the gaussian blur operation to be faster and better in practice. 


Inspired from \cite{chen2021amplitude}, in addition to the image-pair scheme in Eq.~\ref{hybrid_augment_paired}, we propose a single image variant of \textit{HybridAugment}. In the single image variant, instead of combining two images, $x_i$ and $x_{j}$ are obtained by applying randomly sampled augmentations to a single image. The single image variant $\mathcal{HA_{S}}$ can therefore be defined as 
%
\begin{equation} \label{hybrid_augment_single}
    \mathcal{HA_{S}}(x_{i}) = \mathcal{LF}(Aug(x_{i})) + \mathcal{HF}(\hat{Aug}(x_{i}))
\end{equation}
%
where $Aug$ and $\hat{Aug}$ correspond to two sets of randomly sampled augmentation operations. Note that paired and single versions can work in tandem ($\mathcal{HA_{PS}}$), and actually outperform single or paired image versions. 


\subsection{HybridAugment++}


The frequency analysis is a vast literature, however, two core aspects often stand out; frequency-band analysis (i.e. low, high) and the decomposition of signals into amplitude and phase. \textit{HybridAugment} covers the former and shows competitive results in various benchmarks (see Section \ref{sec:exp_hybridaugment}). The latter is investigated in $\mathcal{APR}$ \cite{chen2021amplitude}, where phase is shown to be the more relevant component for correct classification, and training models based on their phase labels and swapping amplitude components of images randomly lead to more robust models. Note that frequency-band and phase/amplitude discussions are arguably orthogonal, since frequency, phase and amplitude provide distinct characterizations of a signal: intuitively speaking, frequency, phase and amplitude can be seen as the separation of visual patterns in terms of scale, location and significance. 


We hypothesize these two approaches can be complementary; a model reliant on low-frequency and spatial information (i.e. phase) can further improve robustness. Inspired by the successes of cascaded augmentation methods \cite{hendrycks2019augmix,wang2021augmax,calian2022defending}, we unify these two core aspects into a single, hierarchical augmentation method. We refer to this method as \textit{HybridAugment++} and define its paired version as:
%
\begin{equation}
  \mathcal{HA_{P}}^{++}(x_{i}, x_{j}, x_{z}) = \mathcal{APR_{P}}(\mathcal{LF}(x_{i}), x_{z}) + \mathcal{HF}(x_{j})
\end{equation}
%
where $x_{i}$, $x_{j}$ and $x_{z}$ are images sampled from the same batch. Here, $\mathcal{APR_{P}}$~\cite{chen2021amplitude} is defined as
\begin{equation}
    \mathcal{APR_{P}}(x_{i}, x_{z}) = \mathcal{IDFT}(A_{x_{z}} \otimes e^{i. P_{x_{i}}}) \\
\end{equation}
%
where $\otimes$ is element-wise multiplication, $A$ is the amplitude and $P$ is the phase component. Similar to $\mathcal{HA}$ and $\mathcal{APR}$, we also define a single-image version of \textit{HybridAugment++} as
%
\begin{equation}
 \mathcal{HA_{S}}^{++}(x_{i}) = \mathcal{APR_{S}}(\mathcal{LF}(Aug(x_{i}))) + \mathcal{HF}(\hat{Aug}(x_{i}))
\end{equation}
%
where $\mathcal{APR_{S}}$~\cite{chen2021amplitude} is defined as
%
\begin{equation}
\mathcal{APR_{S}}(x_{i}) = \mathcal{IDFT}\left(A_{\bar{Aug}(x_{i})} \otimes e^{i. P_{\overline{Aug}\left(x_{i}\right)}}\right)    
\end{equation}
%
where $Aug$, $\hat{Aug}$, $\bar{Aug}$ and $\overline{Aug}$ are different sets of randomly sampled augmentation operations. Note that we essentially propose a framework; one can use different single and paired image augmentations, either individually or together, and can still achieve competitive results (see ablations in Section \ref{sec:exp_hybridaugment}). There are also other alternatives, such as swapping phase/amplitude first and then performing $\mathcal{HA}$, but we observe poor performance in practice; dividing the phase component into frequency-bands is not interpretable as frequencies of the phase component are not well defined. The pseudo-code of our methods can be found in the supplementary material.




%-------------------------------------------------------------------------------
\section{Implementation} \label{imple}
%-------------------------------------------------------------------------------

% Figure environment removed

\sys's implementation consists of (\romannumerber{1}) a fully-functional \sys switch prototype which implements the overall in-fabric logic; and (\romannumerber{2}) a set of software APIs exposed to applications. Our prototype is built upon a FPGA-assisted commodity switch while a P4 programmable switch implementation is also provided.

%\sys's implementation consists of (\romannumerber{1}) a FPGA-based fully-functional prototype which implements the self-defined switch logic and (\romannumerber{2}) a set of software APIs exposed to multicast applications. \sys's switch logic can also be implemented on the Tofino P4 switch, as described in $\S$\ref{dis}.

\parab{FPGA-based prototype.} We implement the group registration, data packet duplication, header modification, and feedback aggregation logics on an FPGA board. The board is equipped with a commodity FPGA chip~\cite{ultrascale} and four 100Gbps Ethernet interfaces. The FPGA resource utilization is shown in Table~\ref{tab:overhead}. We build our testbed with the FPGA board, a commodity Ethernet switch, and four servers, as illustrated in Fig.~\ref{fig:fpgaprototype}. Each server is equipped with a commodity RNIC. The FPGA board and four RNICs are connected to the commodity switch through 100Gbps Ethernet interfaces. 

The commodity switch is configured by Access Control List (ACL) to route the servers' multicast traffic to the FPGA board. The FPGA board identifies the multicast data (ACK\footnote{In Fig.~\ref{fig:fpgaprototype}, we use ACK to represent all types of feedback.}) packets through the specific packet header by \textit{Parser} and \textit{Arbiter}. The data (ACK) packets will be duplicated (aggregated) by \textit{Duplicator} (\textit{ACK Aggregator}). The resulting packets will be pushed in \textit{Queue System}, waiting for the \textit{Multiplexer} to schedule in case for queue competition. Finally, the duplicated data (aggregated ACK) packets are sent back to the commodity switch. During processing, the \textit{Multicast Forwarding Table} is accessed when needed. %\todo{describe Fig. 8} Note those resulting packets' destination IP would be unicast IPs, so the switch routes them as normal unicast packets.
%

\parab{P4-based implementation.}
\sys in-fabric logic can be implemented on the P4 switch as well with some special handling. For the one-to-many data forwarding, P4 switch duplicates packets at the Traffic Manager (TM). The extended table states in Fig.~\ref{fig:table} are stored in the egress pipeline, indexed by <GroupID, EgressPort>. Because the lookup key in P4 switch is at most 32bits, we can use the least significant 24bits of GroupIP plus the 8bits port number as the real index. 

For the many-to-one feedback aggregation, there are two challenges due to the limited capability of P4 switch. Commodity P4 switch switch contains many stages, each with minimal computation capability and independent memory. Firstly, a single stage cannot support the \textit{wrapped-around} PSN comparison. To handle it, we simplify the standard PSN comparison to match the stage's capability, resulting in a tighter PSN space reduced from $2^{23}$ to $2^{22}$. Secondly, a single stage cannot iterate the entire table entries and find the minimum PSN. To handle it, we leverage multiple stages, each responsible for partial entries. Thus, the maximum number of entries supported in each multicast group is limited by the total stage number.

Besides, the P4 switch lacks the computation capability to recalculate the Invariant Cyclic Redundancy Checksum (ICRC) for the modified (aggregated) data (ACK) packets. As a result, those packets would violate the ICRC validation and be discarded at the receiver. This is why we choose the FPGA-based prototype to evaluate \sys in this work. However, a recent work shows that some RNICs provide the ability to bypass ICRC validation~\cite{switchML}. 
%Although this can work, it's a compromised method having security risks. 
%So we select the more integrated FPGA-based prototype to evaluate in $\S$\ref{eva}. Other operators can choose their preferred implementation based on their requirements.

%\begin{algorithm}[t]
%	\caption{Update PSN record and find PSN minimum in P4}\label{alg:psncomp}
%	\begin{algorithmic}[1]
%		%\Function{Generation}{}
%		\State $ack.psn, ack.port\gets $ the PSN and port of ACK packet
%		\State $rec.psn, rec.port\gets$ the PSN and port recorded
%		\State $isTrigger\gets$ whether the packet is a trigger packet
%		%\State $last\_ack\_psn\gets$ last aggregated ACK's psn
%		%\State $min\_port\gets$ port with minimum $ack\_psn$ last time
%		\State min.psn = ack.psn;
%		\State \textcolor{purple}{// every stage compare the PSN}
%		\If{$ack.psn > rec.psn$ or $(ack.psn <= 24'b3fffff$ and $rec.psn >= 24'b600000)$}
%			\State min.psn = rec.psn;
%			\If{rec.port == ack.port}
%				\State rec.psn = ack.psn;
%			\EndIf
%		\EndIf
%		\State \textcolor{purple}{// last stage write the min PSN back}
%		\If{isTrigger} 
%			\State ack.psn = min.psn;
%			\State Forward ACK.
%		\EndIf
%	\end{algorithmic}
%\end{algorithm}

\parab{Software APIs.} We provide various communication libraries and middleboxes for \sys multicast support. Take the commonly-used OpenMPI as an example, we modify the OpenMPI (v4.1.1)~\cite{openmpi} and UCX (v2.3)~\cite{ucx} to adapt to \sys's design, as shown in Fig.~\ref{fig:fpgaprototype}. Specifically, we add a new implementation of $MPI\_Bcast$ and modify UCX for multicast QPs creation and data transmission. When the new $MPI\_Bcast$ is called, the MPI process calls the UCX to establish QPs for multicast. Multicast members exchange their QPs information, and the handshake starts, as described in Appendix \ref{apx:regis}. Once the multicast group is successfully established, the UCX finally calls the RDMA primitives defined in the well-known \textit{libibverbs}~\cite{libibverbs} to transmit data. The software modifications at the end-host are transparent to the upper-layer applications and don't require any RNIC or RDMA driver modification.
%\parab{Coalescence of unicast and multicast}
%When we design \sys, there is a question in our mind: \textit{which is better, maintaining unicast and multicast transports separately at end-host, or utilizing the in-network support to enabling them to match the same transport?} Because of the long-standing resource limit in RNICs and the emeging trend of shifting appropriate computation task to programmability network, we believe the latter is the correct selection.

\begin{table}[t]
	\small
    \centering
%	\begin{center}
%    \begin{tabular}{l|c|c|c}
    \begin{tabular}{|p{0.2\linewidth}|p{0.18\linewidth}|p{0.18\linewidth}|p{0.18\linewidth}|}
    \hline
    \textbf{Resource} & \hfil \textbf{LUT} & \hfil \textbf{Register} & \hfil \textbf{BRAM} \\
    \hline
   	\textbf{Usage} & \hfil 53169 & \hfil 15391 & \hfil 188 \\
    \hline
    \end{tabular}
%    \end{center}
    \caption{Resource usage of the \sys in-fabric logic.}
    \label{tab:overhead}
    \vspace{-0.25cm}
\end{table}

%\parab{Resource overhead}  Note that the size of multicast forwarding table is determined by the number of ports of the switch and doesn't scale up with the multicast group size. 2.7MB memory can support upto 1K multicast groups, which is satisfied in datacenter. We provide a detailed calculation of the maxmum group support in Appendix \ref{apx:cal}. 
\begin{table}[t]
	\centering
	\caption{Preallocation strategy results with $3$ machines per tool group and $10$ operations per lot}
	\label{tab:table}
	\figspace\scriptsize
	%	\resizebox{15.5cm}{!}{
		\begin{tabular}{|l%r
				cl||rr|rr|rr|rr|}
			%			\hline
			%			&                    &                      & %        &
			%			 \multicolumn{8}{c}{\textbf{M = 9}} \\
			\hline
			& \multicolumn{1}{@{\hspace{-3mm}}c@{\hspace{-3mm}}}{\textbf{9 Machines}}                   &                      & % &
			\multicolumn{2}{r|}{\textbf{70 Operations}}                 & \multicolumn{2}{r|}{\textbf{80 Operations}}                 & \multicolumn{2}{r|}{\textbf{90 Operations}}                 & \multicolumn{2}{r|}{\textbf{100 Operations}}                 \\
			& Size % \multicolumn{2}{c}{\textbf{Parameters}}            
			&        &
			Lot                         & Step                        & Lot                         & Step                        & Lot          & Step         & Lot          & Step         \\
			%			& size              % & setup % idx
			%			                  &         & 0                           & 1                           & 0                           & 1                           & 0            & 1            & 0            & 1            \\
			%			&                    & setup                &         &                             &                             &                             &                             &              &              &              &              \\
			\hline\hline
			\multirow{3}{*}{\textbf{Fixed}}    & \multirow{3}{*}{1} & % \multirow{3}{*}{0/1} &
			Makespan    & 483                         & 428                         & 489                         & 440                         & 486          & 531          & 592          & 553         \\
			&                    & %                     &
			Setup/Batch & 6/12                        & 2/12                        & 5/14                        & 0/13                        & 5/14         & 3/12         & 3/12         & 0/16         \\
			&                    & %                     &
			1\ts{st}/2\ts{nd} Stage & 2/1                         & TO/27                          & 6/2                        & TO/13                          & 11/13         & TO           & TO/78           & TO           \\
			\midrule
			\multirow{6}{*}{\textbf{Flexible}} & \multirow{3}{*}{2} & % \multirow{6}{*}{0}   &
			Makespan    & 483                         & 475                         & 592                         & 592                         & 592          & 539          & 745          & 698          \\
			&                    & %                     &
			Setup/Batch & 2/8                        & 0/9                        & 1/8                        & 1/8                        & 1/10         & 0/11          & 0/12          & 0/15          \\
			&                    & %                     &
			1\ts{st}/2\ts{nd} Stage & 5/1                         & TO                          & TO/114                          & TO/1                          & TO/130           & TO           & TO           & TO          \\
			\cline{2-11}
			%			& & & & & & & & & & &   \\
			& \multirow{3}{*}{3} & %                     &
			Makespan    & 559                         & --                          & 815                         & --                          & 1357 & -- & 1486 & -- \\ % \multicolumn{4}{c|}{\multirow{3}{*}{Assignment issue}}     \\
			&                    & %                     &
			Setup/Batch & 0/8                         & --                          & 0/8                        & --                          & 0/10 & -- & 10/18 & -- \\ %\multicolumn{4}{c|}{}                                      \\
			&                    & %                     &
			1\ts{st}/2\ts{nd} Stage & TO                       & --                          & TO/140                          & --                          & TO/79 & -- & TO & -- \\ %\multicolumn{4}{c|}{}                                      \\
			\midrule
			\multirow{6}{*}{\textbf{Setup}}    & \multirow{3}{*}{2} & % \multirow{6}{*}{1}   &
			Makespan    & 483                         & 475                         & 592                         & 592                         & 592          & 536          & 745          & 683          \\
			&                    & %                     &
			Setup/Batch & 2/8                        & 0/9                        & 1/8                        & 1/8                        & 1/10         & 0/12          & 0/13          & 0/16          \\
			&                    & %                     &
			1\ts{st}/2\ts{nd} Stage & 2/1                        & TO                          & TO/21                          & TO/25                          & TO/22           & TO           & TO/76           & TO           \\
			%			& & & & & & & & & & &   \\
			\cline{2-11}
			& \multirow{3}{*}{3} & %                     &
			Makespan    & \textbf{334}                         & --                          & \textbf{345}                         & --                          & \textbf{434}          & --           & \textbf{555}          & --           \\
			&                    & %                     &
			Setup/Batch & 0/8                         & --                          & 0/8                         & --                          & 0/11          & --           & 0/12          & --           \\
			&                    & %                     &
			1\ts{st}/2\ts{nd} Stage & TO/20                       & --                          & TO/123                          & --                          & TO           & --           & TO/73           & --           \\
			\hline
		\end{tabular}
		%	}
\end{table}
%
We constructed a scalable set of benchmark instances, focusing on sub-routes of
$10$ production operations for two product types from the SMT2020 simulation scenario~\cite{kopp2020smt2020}.
The $10$ operations in both sub-routes are processed by machines
belonging to three tool groups and do thus involve re-entrant flow,
as a lot visits the same tool group multiple times.
Moreover, the operations incorporate batching and specific setups, and machines undergo periodic maintenance operations.
In the following, we concentrate on instances with $9$ machines, i.e., $3$ per
tool group, and gradually increasing number of lots.
Further smaller- and larger-scale instances along with our implementation are
available online.\footref{foo:online}

We ran our experiments with \clingodl\ (version 1.4.0) on an Intel® Core™i7-8650U CPU Dell Latitude 5590 machine under Windows 10, imposing two time limits per run:
the first stage for makespan minimization is aborted at $450$ seconds, in which case the best schedule found so far % (if any) 
is taken as upper bound on the makespan for proceeding to minimize setup and batch violations with 
another $150$ seconds time limit.

Table~\ref{tab:table} reports the quality of best schedules obtained within the time limits for both optimization stages, split into `Makespan' and `Setup/Batch'
values, while two runtimes or `TO' for a timeout, respectively, are given in the
`1\ts{st}/2\ts{nd} Stage' rows, only listing a single `TO' entry in case both stages timed out.
The `Size' column provides the value taken for the constant \lstinline{sub_size},
limiting the number of machines in subgroups to which the operations are preallocated.
For the latter, the `Lot' columns include results with value \lstinline{0} for the constant \lstinline{lot_step}, where a common subgroup takes all operations for a lot, or for value \lstinline{1} in the `Step' columns, leading to their distribution among subgroups.

The `Size' value 1 necessarily leads to a fixed machine assignment, for which the
quality indicators clearly show that the `Step' strategy yields better schedules,
although it incurs more timeouts and thus fewer certain optima because operations on different lots increase the flexibility of execution sequences and thus search complexity.
While flexibility within subgroups by setting their `Size' to 2 or 3 in principle allows for improved schedules, we observe a deterioration due to sharply increasing instantiation size and search effort, as already observed in \cite{ali2023flexible}.
The setup strategy to differentiate operations and machines within subgroups,
activated by changing the constant \lstinline{by_setup},
aims to cut down the scheduling complexity in line with the optimization objectives by reducing the need for setup changes.
This leads to significantly improved schedules with `Size' 3, where the
`Lot' and `Step' preallocation strategies are indifferent and redundant results for the latter are omitted, up to a critical size reached with $100$~operations.

With our preliminary approach~\cite{ali2023flexible}, using a more naive and less feature-rich encoding of either fixed or fully flexible machine assignments, the
threshold at which problem size and combinatorics get prohibitive was reached at less than $50$ operations already.
Despite gearing up to double that size, our benchmark instances still represent small excerpts of the large-scale semiconductor fabs with more than $100$ tool groups and from $242$ to $543$ production operations per lot modeled by~\cite{kopp2020smt2020}.
%
The elevated complexity in comparison to basic settings like the traditional FJSP is mainly due to sophisticated setup and maintenance operations, requiring a detailed analysis of execution sequences on machines for SMSP.
We conjecture that similar scalability limits would also be encountered with MIP or CP encodings, yet the first-order modeling language of ASP with difference logic facilitates rapid prototyping and experimentation.
In fact, our performance evaluation aims to explore the feasibility of search and optimization, in order to come up with strategies for breaking down large SMSP instances into more manageable portions, e.g., focusing on some bottleneck tool groups or re-entrant flow of operations.

% This section will show the experimental results performed by applying the machine assignment strategies mentioned before, with several instances ranging from $30$ to $130$ steps and $6$ to $12$ machines. All experiments are run using an Intel\textsuperscript{\textregistered} Core\texttrademark{} i7-8650U CPU Dell Latitude 5590 machine under Windows 10. Our timeout limit is $600$ seconds, splitted to $450$ seconds for the makespan and $150$ seconds for the setup and batching. 

% We considered three tool groups for all generated instances in which batch processing, time/counter-based maintenance, and setup are considered. For generating the instances, we started with a small instance containing $30$ steps and $6$ machines where each tool group has $2$ machines and then we generate the next instance by adding one more lot, which has $10$ steps. We kept the tool group size till the fixed machine assignment strategy could not reach the optimum within the time limit. We created $3$ parameters \textit{size, idx} and \textit{setup} to activate a specific machine assignment strategy. The size determines the size of a sub-group in each tool group. The $idx$ defines the Job/Step-based indexing of all steps in the same tool group where all steps of the same lot will have the same index if the $idx = 0$ and Hence, they are assigned to the same sub-group/machine. If $idx = 1$, then each step in the tool group will have an identical index. The last parameter setup is to activate the setup strategy or not. If the $setup = 1$, then the setup strategy is applied; if $setup = 0$ then it's not applied.

% % To continue tomorrow isA :)
% Table \ref{tab:table01} shows the results of the instances with $2$ machines in each toll group. The first column refers to the strategy applied for the machine assignment. The second and third columns show the parameters for selecting a particular strategy. The assignment is fully flexible if the \textit{size} is greater than or equal to the number of machines in a tool group. Otherwise, the assignment is partially flexible. In the fourth column, we list our optimization criteria and the time limit for the makespan and setup/batching represented by 1st/2nd call. Each following two consecutive columns illustrate the results of an instance when the Job/Step-based indexing is selected. From the \ref{tab:table01}, we observed that the best-obtained results were achieved by the full flexible assignment in the first three instances and for the last instance, the setup strategy was the best. The fixed/setup strategies terminated within the time limit except for only one case.

% \begin{table}[h]
% 	\centering
% 	\caption{Comparison between the allocation strategies with 2 machines per tool group}
% 	\label{tab:table01}
% %	\resizebox{15.5cm}{!}{
% 		\begin{tabular}{|l%r
% 			cl||rr|rr|rr|rr|}
% 			\hline
% %			&                    &                      &         & \multicolumn{8}{c}{\textbf{M = 6}} \\
% %			\hline
% 			& \textbf{M = 6}                   & %                     &
% 			  & \multicolumn{2}{r|}{\textbf{Instance 01}}                 & \multicolumn{2}{r|}{\textbf{Instance 02}}                 & \multicolumn{2}{r|}{\textbf{Instance 03}}                 & \multicolumn{2}{r|}{\textbf{Instance 04}}                 \\
% 			& Size % \multicolumn{2}{c}{\textbf{Parameters}}            
% 			 &			         & Job                         & Step                        & Job                         & Step                        & Job          & Step         & Job          & Step         \\
% 			\hline
% %			& size               & setup %idx
% %			                  &         & 0                           & 1                           & 0                           & 1                           & 0            & 1            & 0            & 1            \\
% %			&                    & setup                &         &                              &                             &                             &                             &              &              &              &              \\
% 			\hline
% 			\multirow{3}{*}{\textbf{Fixed}}    & \multirow{3}{*}{1} & % \multirow{3}{*}{0/1} &
% 			 Makespan    & 409                         & 353                         & 409                         & 409                         & 525          & 424          & 525          & 493          \\
% 			&                    & %                     &
% 			 Setup/Batch & 5/6                         & 4/6                         & 4/8                         & 4/8                         & 4/9          & 1/9          & 3/11          & 2/10          \\
% 			&                    & %                     &
% 			 1\ts{st}/2\ts{nd}-Call & \textless{}1/\textless{}1 & \textless{}1/\textless{}1 & \textless{}1/\textless{}1 & \textless{}1/\textless{}1 & 31/1         & 137/6        & 37/11          & TO/53           \\
% 			\midrule
% 			\multirow{3}{*}{\textbf{Flexible}} & \multirow{3}{*}{2} & % \multirow{3}{*}{0}   &
% 			 Makespan   & \textbf{233}                         & --                          & \textbf{281}                         & --                          & \textbf{365}          & --           & 587          & --           \\
% 			&                    & %                     &
% 			 Setup/Batch & 0/5                         & --                          & 0/6                         & --                          & 0/8          & --           & 3/9          & --           \\
% 			&                    & %                     &
% 			 1\ts{st}/2\ts{nd}-Call & 7/0                         & --                          & TO/6                          & --                          & TO/83           & --           & TO           & --           \\
% 			\midrule
% 			\multirow{3}{*}{\textbf{Setup}}    & \multirow{3}{*}{2} & % \multirow{3}{*}{1}   &
% 			 Makespan  & 277                         & --                          & 321                         & --                          & 381          & --           & \textbf{419}          & --           \\
% 			&                    & %                     &
% 			 Setup/Batch & 0/4                         & --                          & 0/6                         & --                          & 0/8          & --           & 0/9          & --           \\
% 			&                    & %                     &
% 			 1\ts{st}/2\ts{nd}-Call & \textless{}1/\textless{}1 & --                          & 25/1                         & --                          & TO/12        & --           & TO/122           & -- \\
% 			 \hline
% 		\end{tabular}
% %	}
% \end{table}

% Table~\ref{tab:table02} summarizes the results of the subsequent $4$ instances where each tool group has $3$ machines. In this instances group, we can split the machines into sub-group by setting the \textit{size} parameter to $2$; in that case, we have two sub-groups in each tool group. The experiments demonstrated that the fixed strategy has the same or better performance than the flexible. In addition, the flexible strategy could not find a feasible solution for instances $7$ and $8$ when all machines were in the same group. On the other hand, the setup strategy performed better than the other two strategies when all machines were in one group, in addition to reaching the optimal value of the setup for all instances. 

% \begin{table}[h]
% 	\centering
% 	\caption{Comparison between the allocation strategies with 3 machines per tool group}
% 	\label{tab:table02}
% %	\resizebox{15.5cm}{!}{
% 		\begin{tabular}{|l%r
% 			cl||rr|rr|rr|rr|}
% %			\hline
% %			&                    &                      & %        &
% %			 \multicolumn{8}{c}{\textbf{M = 9}} \\
% 			\hline
% 			& \textbf{M = 9}                   &                      & % &
% 			 \multicolumn{2}{r|}{\textbf{Instance 05}}                 & \multicolumn{2}{r|}{\textbf{Instance 06}}                 & \multicolumn{2}{r|}{\textbf{Instance 07}}                 & \multicolumn{2}{r|}{\textbf{Instance 08}}                 \\
% 			& Size % \multicolumn{2}{c}{\textbf{Parameters}}            
% 			&        &
% 			 Job                         & Step                        & Job                         & Step                        & Job          & Step         & Job          & Step         \\
% %			& size              % & setup % idx
% %			                  &         & 0                           & 1                           & 0                           & 1                           & 0            & 1            & 0            & 1            \\
% %			&                    & setup                &         &                             &                             &                             &                             &              &              &              &              \\
% 			\hline\hline
% 			\multirow{3}{*}{\textbf{Fixed}}    & \multirow{3}{*}{1} & % \multirow{3}{*}{0/1} &
% 			 Makespan    & 525                         & 433                         & 525                         & 452                         & 525          & 521          & 643          & \textbf{559}          \\
% 			&                    & %                     &
% 			 Setup/Batch & 6/13                        & 1/13                        & 5/15                        & 0/14                        & 5/16         & 6/16         & 6/12         & 3/12         \\
% 			&                    & %                     &
% 			 1\ts{st}/2\ts{nd}-Call & 30/3                         & TO/153                          & 24/8                        & TO/63                          & 231/81         & TO           & TO           & TO           \\
% 			\midrule
% 			\multirow{6}{*}{\textbf{Flexible}} & \multirow{3}{*}{2} & % \multirow{6}{*}{0}   &
% 			 Makespan    & 525                         & 475                         & 650                         & 650                         & 650          & 595          & 745          & 742          \\
% 			&                    & %                     &
% 			 Setup/Batch & 2/11                        & 0/11                        & 1/12                        & 1/12                        & 6/13         & 4/14          & 3/17          & n/a          \\
% 			&                    & %                     &
% 			 1\ts{st}/2\ts{nd}-Call & 26/7                         & TO                          & TO/12                          & TO                          & TO           & TO           & TO           & TO           \\
% 			\cline{2-11}
% %			& & & & & & & & & & &   \\
% 			& \multirow{3}{*}{3} & %                     &
% 			 Makespan    & 744                         & --                          & 1206                         & --                          & 1698 & -- & n/a & -- \\ % \multicolumn{4}{c|}{\multirow{3}{*}{Assignment issue}}     \\
% 			&                    & %                     &
% 			 Setup/Batch & 2/12                         & --                          & n/a                        & --                          & 8/15 & -- & n/a & -- \\ %\multicolumn{4}{c|}{}                                      \\
% 			&                    & %                     &
% 			 1\ts{st}/2\ts{nd}-Call & TO                       & --                          & TO                          & --                          & TO & -- & TO & -- \\ %\multicolumn{4}{c|}{}                                      \\
% 			\midrule
% 			\multirow{6}{*}{\textbf{Setup}}    & \multirow{3}{*}{2} & % \multirow{6}{*}{1}   &
% 			 Makespan    & 525                         & 475                         & 650                         & 650                         & 643          & 553          & 745          & 642          \\
% 			&                    & %                     &
% 			 Setup/Batch & 2/11                        & 0/11                        & 1/12                        & 1/12                        & 1/14         & 0/13          & 1/14          & 1/16          \\
% 			&                    & %                     &
% 			 1\ts{st}/2\ts{nd}-Call & 44/2                        & TO                          & TO/4                          & TO/2                          & TO           & TO/7           & TO           & TO           \\
% %			& & & & & & & & & & &   \\
% 			\cline{2-11}
% 			& \multirow{3}{*}{3} & %                     &
% 			 Makespan    & \textbf{346}                         & --                          & \textbf{373}                         & --                          & \textbf{429}          & --           & 820          & --           \\
% 			&                    & %                     &
% 			 Setup/Batch & n/a                         & --                          & n/a                         & --                          & n/a          & --           & n/a          & --           \\
% 			&                    & %                     &
% 			 1\ts{st}/2\ts{nd}-Call & TO                       & --                          & TO                          & --                          & TO           & --           & TO           & --           \\
% 			\hline
% 		\end{tabular}
% %	}
% \end{table}

% Table~\ref{tab:table03} considers $4$ machines in each tool group and the flexible strategy obtained the best result for the first instance. However, it had the same feasibility issue when all machines were in the same group. For the rest instances, the setup strategy dominated when the machines were equally distributed into sub-groups. 

% From the conducted experiments, we can conclude that 
% \begin{itemize}
% 	\item The flexible assignment performed well on the small-scale.
% 	\item While increasing the scale, the setup strategy dominates in the most cases
% 	\item Assigning the steps of the same lot independently with the fixed assignment leads to better performance
% 	\item The Setup strategy has a significant impact in minimizing the setup objective through all instances
% 	\item The full flexible assignment has an assignment issue while increasing the number of machines
% \end{itemize}

% \begin{table}[h]
% 	\centering
% 	\caption{Comparison between the allocation strategies with 4 machines per tool group}
% 	\label{tab:table03}
% %	\resizebox{15.5cm}{!}{%
% 		\begin{tabular}{|l%r
% 			cl||rr|rr|rr|rr|}
% 			\hline
% %			&                    &                      &  &  \multicolumn{8}{c}{\textbf{M = 12}} 
% %			\\ \hline
% 			& \textbf{M = 12}                   & %                     & 
% 			 & \multicolumn{2}{r|}{\textbf{Instance 09}}                 & \multicolumn{2}{r|}{\textbf{Instance 10}}                 & \multicolumn{2}{r|}{\textbf{Instance 11}}                 & \multicolumn{2}{r|}{\textbf{Instance 12}}                 \\
% 			& Size % \multicolumn{2}{l}{\textbf{Parameters}}            
% 			 &			 &			 Job                    & Step                   & Job                    & Step                   & Job                    & Step                   & Job                    & Step                   \\
% %			& Size               & setup % idx
% %			                  &  & 0                      & 1                      & 0                      & 1                      & 0                      & 1                      & 0                      & 1                      \\
% %			&                    & setup                &  &  &                        &                        &                        &                        &                        &                        &                                               \\
% 			\hline\hline
% 			\multirow{3}{*}{\textbf{Fixed}}    & \multirow{3}{*}{1} & % \multirow{3}{*}{0/1} &
% 			 Makespan                 & 525                    & 453                    & 525                    & 452                    & 525                    & 493                    & 643                    & 561                    \\
% 			&                    & %                     &
% 			 Setup/Batch              & 7/19                   & 3/20                   & 7/20                  & n/a                   & 6/22                   & 4/20                   & 4/22                   & n/a                   \\
% 			&                    & %                     &
% 			 1\ts{st}/2\ts{nd}-Call              & 124/5                 & TO & 25/17                 & TO & 25/53                 & TO/142 & TO & TO \\
% 			\midrule
% 			\multirow{9}{*}{\textbf{Flexible}} & \multirow{3}{*}{2} & % \multirow{9}{*}{0}   &
% 			 Makespan                 & \textbf{373}                    & 503                    & 491                    & 778                    & 569                    & 569                    & 765                    & 1673                   \\
% 			&                    & %                     &
% 			 Setup/Batch              & n/a                    & 6/17                    & n/a                   & n/a                    & n/a                    & n/a                   & n/a                    & 12/24                  \\
% 			&                    & %                     &
% 			 1\ts{st}/2\ts{nd}-Call              & TO & TO & TO & TO & TO & TO & TO & TO \\
% 			\cline{2-11}
% %			& & & & & & & & & & &   \\
% 			& \multirow{3}{*}{3} & %                     &
% 			 Makespan                 & 709                    & 688                    & 800                    & 907                    & 876                    & 876                    & 905                    & 1643                   \\
% 			&                    & %                     &
% 			 Setup/Batch              & 5/17                    & n/a                   & 3/18                   & 5/19                   & n/a                   & n/a                   & n/a                  & 15/24                    \\
% 			&                    & %                     &
% 			 1st/2nd              & TO & TO & TO & TO & TO & TO & TO & TO \\
% 			\cline{2-11}
% %			& & & & & & & & & & &   \\
% 			& \multirow{3}{*}{4} & %                     &
% 			 Makespan                 & n/a & -- & n/a & -- & n/a & -- & n/a & -- \\ %\multicolumn{8}{c|}{\multirow{3}{*}{Assignment issue}}                                                                                                                                                 \\
% 			&                    & %                     &
% 			 Setup/Batch              & n/a & -- & n/a & -- & n/a & -- & n/a & -- \\ %\multicolumn{8}{c|}{}                                                                                                                                                                                  \\
% 			&                    & %                     &
% 			 1\ts{st}/2\ts{nd}-Call              & TO & -- & TO & -- & TO & -- & TO & -- \\ %\multicolumn{8}{c|}{}                                                                                                                                                                                  \\
% 			\midrule
% 			\multirow{9}{*}{\textbf{Setup}}    & \multirow{3}{*}{2} & % \multirow{9}{*}{1}   &
% 			 Makespan                 & 401                    & 396                    & 419                    & \textbf{416}                    & \textbf{419}                    & \textbf{419}                    & \textbf{457}                    & 471                    \\
% 			&                    & %                     &
% 			 Setup/Batch              & 0/15                   & 0/14                   & 0/16                   & 0/16                   & n/a                   & n/a                   & 0/21                    & n/a                    \\
% 			&                    & %                     &
% 			 1\ts{st}/2\ts{nd}-Call              & TO & TO/92 & TO & TO & TO & TO & TO & TO \\
% 			\cline{2-11}
% %			& & & & & & & & & & &   \\
% 			& \multirow{3}{*}{3} & %                     &
% 			 Makespan                 & 706                    & 642                    & 792                    & 753                    & 942                    & 942                    & 939                    & 894                    \\
% 			&                    & %                     &
% 			 Setup/Batch              & 1/14                    & n/a                    & 2/16                    & n/a                   & n/a                   & n/a                    & n/a                    & 1/22                    \\
% 			&                    & %                     &
% 			 1\ts{st}/2\ts{nd}-Call              & TO & TO & TO & TO & TO & TO & TO & TO \\
% 			\cline{2-11}
% %			& & & & & & & & & & &   \\
% 			& \multirow{3}{*}{4} & %                     &
% 			 Makespan                 & 679                    & -- & 1725                    & -- & n/a                    & -- & n/a                    & -- \\
% 			&                    & %                     &
% 			 Setup/Batch              & n/a                   & -- & n/a                    & -- & n/a                   & -- & n/a                   & -- \\
% 			&                    & %                     &
% 			 1st/2nd              & TO & -- & TO & -- & TO & -- & TO & -- \\
% 			\hline
% 		\end{tabular}%
% %	}
% \end{table}

\section{Conclusion}
In this work, we have presented a novel approach for high-quality neural scene reconstruction and photo-realistic novel view synthesis.
We propose a novel tensor factorization-based scene representation, which leverages CP decomposition to compactly model a 3D scene as a sparse set of multi-scale tri-vector tensors that express local radiance fields.
Our representation leverages both sparsity and spatial local coherence, and leads to accurate and efficient modeling of complex scene geometry and appearance.
We demonstrate that the sparse tri-vector radiance fields can achieve superior rendering quality than previous state-of-the-art neural representations, including TensoRF and iNGP, while using significantly fewer parameters.

{\small
\bibliographystyle{ieee_fullname}
\bibliography{egbib}
}

\newpage
\begin{appendices}
%\title{Appendix}
    \twocolumn[{%
        \renewcommand\twocolumn[1][]{#1}%
        \begin{center}
            \centering
            \LARGE \textbf{\appendixname}
            \vspace{30pt}
        \end{center}%
    }]

%\author{First Author\\
%Institution1\\
%Institution1 address\\
%{\tt\small firstauthor@i1.org}

% For a paper whose authors are all at the same institution,
% omit the following lines up until the closing ``}''.
% Additional authors and addresses can be added with ``\and'',
% just like the second author.
% To save space, use either the email address or home page, not both
%\and
%Second Author\\
%Institution2\\
%First line of institution2 address\\
%{\tt\small secondauthor@i2.org}
%}

\maketitle
% Remove page # from the first page of camera-ready.
\ificcvfinal\thispagestyle{empty}\fi

% \begin{table}[hbt]
%         \begin{adjustwidth}{0pt}{0pt}  
%         \centering
%         \setlength\tabcolsep{3.0pt}
%             \small{
%                 \begin{tabular}{l|ccc}
%                 %& \multicolumn{5}{c}{\textbf{Average over Scene 101 and Scene 241}} & \\
%                 \hline %\\[-1.05em]
%                  VM-\#Comp & PSNR~$\uparrow$ & SSIM~$\uparrow$  &  \# Param.(M)~$\downarrow$ \\ \hline
%                  Single(0.3)-48  & 33.11 & 0.964 & 86.44   \\
%                  Single(0.3)-24 & 33.16 & 0.964 & 43.33 \\
%                  Single(0.3)-12 & 32.99 & 0.962 & 21.64 \\
%                  \hline
%                 \end{tabular}
%             }
%              \captionsetup{aboveskip=3pt}
%             \captionsetup{belowskip=-5pt}
%             \caption {Ablation of our method with replacing vector bases (CP) into vector-matrix (VM) bases under a different number of components settings on Synthetic-NeRF dataset. We select the number of components 48, 24, and 12 for comparison. Meanwhile, the scale is fixed at 0.3.}
%             % \KS{Add time/iterations for IBRNet. What are the different variants of Point-NeRF discussed here?}}
%             \label{tb:VM_comp_ablation} 
%         \end{adjustwidth}
%     \end{table}


\section{Ablation Studies on Tensor Factorization Strategies}
    \begin{table}[hbt]
        \begin{adjustwidth}{0pt}{0pt}  
        \centering
        \setlength\tabcolsep{3.0pt}
            \small{
                \begin{tabular}{l|c|ccc}
                %& \multicolumn{5}{c}{\textbf{Average over Scene 101 and Scene 241}} & \\
                \hline %\\[-1.05em]
                  & \# Comp & PSNR~$\uparrow$ & SSIM~$\uparrow$  &  \# Param.(M)~$\downarrow$ \\ \hline
                 Multi(0.6, 0.3, 0.15) & 24 & \textbf{33.24} & \textbf{0.963} & \textbf{7.07} \\
                 % Ours(mlt) & 48 & 33.55 & 0.965 & 13.52 \\
                 % Ours(mlt) & 96 & 33.59 & 0.965 & 21.07 \\
                 % TriVec-Cloud (0.3) & 48 & 32.73 & 0.961 & 4.15\\
                 Single(0.3) & 96 & 33.02 & 0.963 & 9.15 \\
                 VM-Cloud (0.3) & 6 & 32.59 & 0.959 & 11.36 \\
                 VM-Cloud (0.3) & 12 & 32.99 & 0.962 & 21.64\\
                 %VM-Single(0.3)$_{24}$ & 33.16 & 0.964 & 43.32 \\
                 %VM-Single(0.3)$_{48}$ & 33.11 & 0.964 & 86.44\\
                 \hline
                \end{tabular}
            }
             \captionsetup{aboveskip=3pt}
            \captionsetup{belowskip=-1pt}
            \caption {(a) Comparisons on our method pairing with different factorization strategies, e.g., CP decomposition and vector-matrix (VM) decomposition (row 2 vs 3,4). The local tensors' edge lengths are all set as 0.3. (b) We also compare a single-scale model with a multi-scale model (row 1 vs 2). We evaluate these settings on the NeRF Synthetic dataset~\cite{mildenhall2020nerf} and evaluate them with both rendering quality and model capacity (\#Param. denotes the number of parameters).}
            % \KS{Add time/iterations for IBRNet. What are the different variants of Point-NeRF discussed here?}}
            \label{tb:doubleAblation} 
        \end{adjustwidth}
    \end{table}
Other than CP decomposition, TensoRF \cite{chen2022tensorf} also proposes vector-matrix (VM) decomposition, which factorizes a 3D tensor as the summation of vector-matrix bases. Each basis is the outer product of a matrix along a plane, e.g., the XY plane, and a vector along an orthogonal direction, e.g., the Z axis. For comparison, we also explore to replace our tri-vector representation with the vector-matrix representation for each local tensor. Tab.~\ref{tb:doubleAblation} shows that the single-scale tri-vector cloud can outperform the vector-matrix cloud representation with less model capacity. 

It is not a surprise that our tri-vector cloud representation achieves more compactness. It applies more compression by factorizing each component of a 3D tensor, with a space complexity of $O(IJK)$, into three vectors, with a space complexity of $O(I+J+K)$. On the other hand, vector-matrix cloud representation factorizes it into three vectors and three matrices, which have a space complexity of $O(IJ+JK+IK)$. Even if we reduce the number of components, the vector-matrix clouds still require more space than our tri-vector representations.

In terms of quality, since our method exploits the spatial sparsity of natural scenes, we only need to factorize each local space independently instead of the entire scene together. The more compact tri-vector representation can benefit from the appearance coherence in local space and result in better performance. In TensoRF \cite{chen2022tensorf}, since the entire space is factorized all at once, the radiance information is, in general, less coherent across locations and the CP decomposition will lead to a shortage of rank. 


\section{Ablation Studies on Multi-scale Models}
In Tab.\ref{tb:doubleAblation}, we also compare our multi-scale tri-vector radiance fields with the single-scale strategy. In our default model, we have three scales, composed of tensors with lengths 0.15, 0.3, and 0.6, respectively. Similar to the findings in iNGP~\cite{muller2022instant}, our multi-scale models provide more smoothness and lead to a better rendering quality than their single-scale counterparts. The multi-scale model with 24 components (row 1) can already outperform the single-scale model (row 2), which has more parameters.

\section{Ablation Studies on the Number of Tensor Components}

We conduct experiments on the NeRF Synthetic dataset~\cite{mildenhall2020nerf} to show the relationship between rendering performance and the number of tensor components. In Tab.\ref{tb:ablation_nerfsynth}, we compare our multi-scale models with 12, 24, 48, and 96 appearance components, respectively. In general, more tensor components will lead to better performance. We also observe that the benefit of adding more components becomes marginal when the number reaches 48. We speculate that it is harder to learn high-frequency details even though the model's capacity can hold high-rank information. Improvement in this aspect can be a promising future direction.

%We also enable using less number of components in the local tensors of smaller scales because the space covered in small local tensors is also small and may not need many components to encode. And model size can be further decreased in this way.  

\begin{table*}%[hbt]
    \begin{adjustwidth}{0pt}{0pt}  
    \centering
    % \setlength\tabcolsep{5pt}
        \small{
            \begin{tabular}{c|ccccc }
            \hline %\\[-1.05em]
             & PSNR$\uparrow$ & SSIM$\uparrow$& LPIPS$_{Vgg}$ $\downarrow$ & LPIPS$_{Alex}$ $\downarrow$ & \# Param.(M)$\downarrow$\\ \hline
            Ours-12 & 32.94 & 0.961 & 0.049 & 0.028 & \textbf{4.87}\\
            %NSVF\cite{liu2020neural} & 4096 & 150k & 0.80-4.00 & - & 31.75 & 0.953 & - & 0.047\\
            Ours-24 & 33.24 & 0.963 & 0.046 & 0.026 & 7.07\\
            % Point-NeRF$_{20k}$\cite{xu2022point} & 4096 & 20k & 27.74 & 33.0m & 30.71 & 0.967 & 0.081& 0.050\\ 
            Ours-48 & 33.55 & 0.965 & 0.044 & 0.025 & 13.52 \\
            Ours-96 & \textbf{33.59} & \textbf{0.965} & \textbf{0.043} & \textbf{0.024} & 21.01  \\ \hline
            \end{tabular}
        }
        \captionsetup{aboveskip=5pt}
        \captionsetup{belowskip=-0pt}
        \caption {Ablation study on the number of tensor components. We use the same setting as our default model but only change the number of components in each variant. These variants are evaluated on the NeRF Synthetic dataset \cite{mildenhall2020nerf}.}
        % \KS{Add time/iterations for IBRNet. What are the different variants of Point-NeRF discussed here?}}
        \label{tb:ablation_nerfsynth} 
    \end{adjustwidth}
\end{table*} 

\section{Ablation Studies on Initial Geometry}
\label{initial_geometry}
We emphasize that our superior quality stems from our novel scene representation rather than the initial geometry.
The initial geometry is simply acquired from a low-res RGBA volume reconstruction, which is coarse and only used to roughly prune empty space.

We show in Fig.~\ref{init_geo} that our approach performs robustly with various choices of these geometry structures and consistently achieves high PSNRs, even with a much worse early-stopped RGBA reconstruction.
This showcases the key to our superior quality is our Strivec model itself.\\
In particular, the self-bootstrap geometry is generated purely from our own model with 8 coarse tri-vectors without existing modules in previous work. 
Moreover, we can also further prune unoccupied tensors during training but we find this leads to similar quality (0.03db difference) and unnecessary extra (+22\%) training time. 
We instead choose to use one single initial geometry to prune empty space in implementation for its simplicity and efficiency.

% Figure environment removed



\section{Speed v.s. Performance}
\label{speed}
Though speed is not our focus, here, if we reduce the number of scales from 3 to 2 and TopK from 4 to 2 (i.e., Multi(0.6, 0.3) with TopK=2), and Strivec becomes faster than CP and close to VM, while still having competitive quality (see Ours-48(fast) in Tab.\ref{tb:speed}). The fewer ranks of our tensor and the less number of TopK to be find for each sample point along a ray lead to less computation, and thus, acceleration. To conclude, Strivec is capable to improve quality, training time and compactness all together with proper hyper-parameters.
\vspace{-6pt}
\begin{table}[hbt]
      \centering
        \setlength\tabcolsep{2pt}
            \small{
                \begin{tabular}{l|cccc}
                \hline %\\[-1.05em]
                 & Train(s)$\downarrow$ & Inference(s/it)$\downarrow$ & \#Params.(M)$\downarrow$ & PSNR$\uparrow$ \\ \hline
                TensoRF-CP & 1914 & 2.01 & 0.98 & 31.56 \\ 
                TensoRF-VM & 915 & 1.60 & 17.95 & 33.14 \\
                %Ours-48(3scales) & 864 & 1.75 & \bf{33.87}\\
                Ours-48(fast) & 959  & 1.67 & 6.20 & 33.09\\
                \hline
                \end{tabular}
            }
            \captionsetup{aboveskip = 2pt}
            \captionsetup{belowskip = -15pt}
            \caption {Comparison on NeRF Synthetic dataset~\cite{mildenhall2020nerf}. We compare the average training time (s), inference time (s/it), the number of parameters (M) and PSNR.  }
            \label{tb:speed} 
    \end{table}




% \section{A. Per-scene Breakdown Results of the
% NeRF Synthetic Dataset}
\section{Per-scene Breakdown Results of the NeRF Synthetic Dataset}
    We show the per-scene detailed quantitative results of the comparisons on the NeRF Synthetic dataset \cite{mildenhall2020nerf} in Tab. \ref{tb:dt_nerfsynth} 
    and qualitative comparisons in our video. With compact model capacity, our method outperforms state-of-the-art methods \cite{mildenhall2020nerf,muller2022instant,xu2022point,chen2022tensorf} and achieves the best PSNRs, and LPIPSs in most of the scenes. %Please note the metric hyper-parameters of SSIM in PointNeRF's original version are different than others, they have updated their SSIM. 
    We report two versions of iNGP~\cite{muller2022instant}. Specifically, iNGP-dark$_{100k}$ is reported in the original paper. According to issue~\href{https://github.com/NVlabs/instant-ngp/discussions/745}{\#745} in iNGP's official repo, the method uses a random color background in training and dark background in testing. The number of iterations, 100k, is referenced to its initial code base release. We also refer to the results reported in \cite{factorfields} as iNGP-white$_{30k}$, since the authors use a white background in both training and testing for 30k iterations, which has the same setting as ours and many other compared methods. Please refer to issue \href{https://github.com/NVlabs/instant-ngp/discussions/745}{\#745} and \href{https://github.com/NVlabs/instant-ngp/issues/1266}{\#1266} in iNGP's official repo for more details.
\begin{table}[]
      \centering
        % \setlength\tabcolsep{5pt}
            \small{
                \begin{tabular}{l|cc|c}
                \hline %\\[-1.05em]
                 & garden & room & Model Size(avg) \\ \hline
                DVGO & 24.32 & 28.35 & 5.1GB \\ 
                Ours-48 & 24.13 & 28.11 & 12.6MB\\
                \hline
                \end{tabular}
            }
            \captionsetup{aboveskip = 2pt}
            \captionsetup{belowskip = -15pt}
            \caption {Results on the Mip-NeRF 360 dataset.}
            % \KS{Add time/iterations for IBRNet. What are the %different variants of Point-NeRF discussed here?}}
            \label{tb:360_tab} 
    \end{table}
    
    \begin{table*}[]
    %   \setlength\tabcolsep{4pt}
      \captionsetup{aboveskip=5pt}
      \centering
      \begin{tabular}{lcccccccc}
            \hline
            \multicolumn{9}{c}{NeRF Synthetic}                                                                                                                 \\
                       & Chair          & Drums          & Lego           & Mic            & Materials      & Ship           & Hotdog         & Ficus          \\ \hline
            \multicolumn{9}{c}{PSNR$\uparrow$}                                                                                                                           \\ \hline

            NeRF~\cite{mildenhall2020nerf}       & 33.00          & 25.01          & 32.54          & 32.91          & 29.62          & 28.65          & 36.18          & 30.13          \\
            NSVF~\cite{liu2020neural}       & 33.19          & 25.18          & 32.54          & 34.27          & \textbf{32.68} & 27.93          & 37.14 & 31.23          \\
            
            Point-NeRF$_{20k}$~\cite{xu2022point}  & 32.50 & 25.03 & 32.40 & 32.31 &  28.11 & 28.13 & 34.53 & 32.67          \\
            Point-NeRF$_{200k}$~\cite{xu2022point} & 35.40 & 26.06 & 35.04 & 35.95 & 29.61 & 30.97 & 37.30 & 36.13 \\ 
            iNGP-dark$_{100k}$~\cite{muller2022instant} & 35.00 & 26.02 & 36.39 & 36.22 & 29.78 & 31.10 & 37.40 & 33.51 \\
            iNGP-white$_{30k}$~\cite{muller2022instant,chen2023factor} & 35.42 & 24.24 & 34.82 & 35.98 & 28.99 & 30.72 & 37.45 & 32.09 \\ 
            TensoRF-CP~\cite{chen2022tensorf}-384$_{30k}$ & 33.60 & 25.17 & 34.05 & 33.77 & 30.10 & 28.84 & 36.24 &  30.72 \\ 
            TensoRF-VM~\cite{chen2022tensorf}-192$_{30k}$ & 35.76 &  26.01 & 36.46 & 34.61 & 30.12 & 30.77 & 37.41 &  33.99 \\ 
            Ours-12$_{30k}$ & 35.21	& 25.96 & 35.60 & 35.29 & 29.54 & 30.64 & 37.03 & 34.21 \\
            Ours-24$_{30k}$ & 35.60 & 26.16 & 36.05 & 35.81 & 29.79 & 30.89 & 37.24 & 34.37 \\
            Ours-48$_{30k}$ & \textbf{35.88} & \textbf{26.20} & \textbf{36.52} & \textbf{36.65} & 29.90 & \textbf{31.13} & \textbf{37.63} & 34.47 \\ \hline
            \multicolumn{9}{c}{SSIM$\uparrow$}                                                                                                                           \\ \hline
            
            NeRF       & 0.967          & 0.925          & 0.961          & 0.980          & 0.949          & 0.856          & 0.974          & 0.964          \\
            NSVF       & 0.968          & 0.931          & 0.960          & 0.987          & \textbf{0.973}          & 0.854          & 0.980          & 0.973          \\
            
            Point-NeRF$_{20k}$  & 0.981 & 0.944 & 0.980 &  0.986 & 0.959 & 0.916 & 0.983 & 0.986          \\
            Point-NeRF$_{200k}$ & \textbf{0.991} & \textbf{0.954} & \textbf{0.988} & \textbf{0.994} & 0.971 & \textbf{0.942} &  \textbf{0.991} & \textbf{0.993} \\
            
            %Point-NeRF$_{200k}$ (calibrated) & 0.984 & 0.935 & 0.978 & 0.990 & 0.948 & 0.892 & 0.982 & \textbf{0.987} \\ 
            iNGP-white$_{30k}$ & 0.985 & 0.924 & 0.979 & 0.990 & 0.945 & 0.892 & 0.982 & 0.977 \\
          
             TensoRF-CP-384$_{30k}$ & 0.973 & 0.921 &  0.971 & 0.983 & 0.950 &  0.857 &  0.975 &  0.965  \\
             TensoRF-VM-192$_{30k}$ & 0.985 & 0.937 & 0.983 & 0.988 & 0.952 & 0.895 & 0.982 & 0.982\\
             Ours-12$_{30k}$ & 0.983 & 0.937 & 0.980 & 0.989 & 0.948 & 0.888 & 0.981 & 0.983  \\
             Ours-24$_{30k}$ & 0.984 & 0.940 & 0.982 & 0.990 & 0.952 & 0.893 & 0.982 & 0.984 \\
             %Ours-48$_{30k}$ & \textbf{0.985} & \textbf{0.940} & \textbf{0.984} & \textbf{0.992} & 0.953 & \textbf{0.899} & \textbf{0.983} & 0.985 \\\hline
             Ours-48$_{30k}$ & 0.985 & 0.940 & 0.984 & 0.992 & 0.953 & 0.899 & 0.983 & 0.985 \\\hline
             
            \multicolumn{9}{c}{LPIPS$_{Vgg}\downarrow$}                                                                                                                       \\ \hline
           
            NeRF       & 0.046          & 0.091          & 0.050          & 0.028          & 0.063          & 0.206          & 0.121          & 0.044          \\
            
            Point-NeRF$_{20k}$  & 0.051 & 0.103 & 0.054 & 0.039 & 0.102 & 0.181 & 0.074 & 0.043         \\
            Point-NeRF$_{200k}$ & 0.023 & 0.078 & 0.024 &  0.014 & 0.072 &\textbf{0.124} & 0.037 & 0.022 \\
            iNGP-white$_{30k}$ &  0.022 & 0.092 & 0.025 & 0.017 & 0.069 & 0.137 & 0.037 & 0.026\\
            
            
            TensoRF-CP-384$_{30k}$ & 0.044 & 0.114 &  0.038 & 0.035 & 0.068 & 0.196 & 0.052 & 0.058 \\
            TensoRF-VM-192$_{30k}$ & 0.022 & 0.073 & 0.018 & 0.015 & 0.058 & 0.138 & 0.032 & 0.022\\
            Ours-12$_{30k}$ & 0.025 & 0.070 & 0.022 & 0.015 & 0.062 & 0.145 & 0.033 & 0.022 \\
            Ours-24$_{30k}$ &  0.022 & 0.067 & 0.020 & 0.013 & 0.058 & 0.141 & 0.031 & 0.021 \\
            Ours-48$_{30k}$ & \textbf{0.021} & \textbf{0.064} & \textbf{0.017} & \textbf{0.011} & \textbf{0.056} & 0.138 & \textbf{0.029} & \textbf{0.018} \\ \hline
            \multicolumn{9}{c}{LPIPS$_{Alex}\downarrow$}                                                                                                                      \\ \hline
            NSVF       & 0.043          & 0.069          & 0.029          & 0.010          & \textbf{0.021} & 0.162          & 0.025          & 0.017          \\
            Point-NeRF$_{20k}$  & 0.027 & 0.057 & 0.022 & 0.024 & 0.076& 0.127 & 0.044 & 0.022         \\
            Point-NeRF$_{200k}$ & 0.010 & 0.055 & 0.011 & 0.007 & 0.041 & \textbf{0.070} & 0.016 & \textbf{0.009} \\ 

            iNGP-white$_{30k}$ & 0.022 & 0.093 & 0.025 & 0.017 & 0.069 & 0.140 & 0.037 & 0.026 \\
            
            TensoRF-CP-384$_{30k}$ & 0.022 & 0.069 & 0.014 & 0.018 & 0.031 & 0.130 &  0.024 & 0.024 \\
            TensoRF-VM-192$_{30k}$ & 0.010 & 0.051 & 0.007 & 0.009 & 0.026 & 0.085 & 0.013 & 0.012 \\
            Ours-12$_{30k}$ & 0.011	& 0.051 & 0.009 & 0.007 & 0.027 & 0.092 & 0.015 & 0.013  \\
            Ours-24$_{30k}$ & 0.010 & 0.049 & 0.008 & 0.006 & 0.024 & 0.087 & 0.014 & 0.012  \\
            Ours-48$_{30k}$ & \textbf{0.009} & \textbf{0.048} & \textbf{0.007} & \textbf{0.005} & 0.023 & 0.086 & \textbf{0.012} & 0.011 \\ \hline
        \end{tabular}      
        \caption{Detailed breakdown of quantitative metrics on individual scenes in the NeRF Synthetic \cite{mildenhall2020nerf} for our method and baselines. All scores are averaged over the testing images. The subscripts are the number of iterations of the models. NeRF only \cite{mildenhall2020nerf} reports the LPIPS$_{Vgg}$~\cite{zhang2018perceptual} while NSVF only reports LPIPS$_{Alex}$. } %PointNeRF has updated their calibrated SSIM in arxiv since the metric hyper-parameters in the original version are different than others.
        \label{tb:dt_nerfsynth}
    \end{table*}
    


% Figure environment removed


\section{The Tanks and Temples Dataset}
We show the qualitative comparison between our Strivec and TensoRF-VM~\cite{chen2022tensorf} on the Tanks and Temples dataset~\cite{Knapitsch2017} in Fig.\ref{fig:tanks_comp}. Similar to the procedures on the NeRF Synthetic dataset, we build the coarse scene geometry within 30 seconds to place our local tensors. The quantitative results are reported in Tab.\ref{tb:tt}. %Note that there are many choices (e.g., meshes or point clouds from multi-view stereo) for the geometry prior and our results are expected to be better when the scene geometry has higher quality.
    
    \begin{table*}[hbt!]
    %   \setlength\tabcolsep{6pt}
      \centering
      \captionsetup{aboveskip=5pt}
        \begin{tabular}{ccccccc}
        \hline
        \multicolumn{7}{c}{Tanks \& Tamples}                                                                                                                \\
        \multicolumn{1}{l}{} & Ignatius             & Truck                & Barn      & Caterpillar                 & Family               & Mean                 \\ \hline
        \multicolumn{1}{l}{} & \multicolumn{1}{l}{} & \multicolumn{1}{l}{} & PSNR~$\uparrow$      & \multicolumn{1}{l}{} & \multicolumn{1}{l}{} & \multicolumn{1}{l}{} \\ \hline
        NV~\cite{lombardi2019neural}                   & 26.54                & 21.71                & 20.82     & 20.71                & 28.72                & 23.70                \\
        NeRF~\cite{mildenhall2020nerf}                 & 25.43                & 25.36                & 24.05     & 23.75                & 30.29                & 25.78                \\
        NSVF~\cite{liu2020neural}                 & 27.91                & 26.92                & 27.16     & 26.44                & 33.58                & 28.40                \\
        TensoRF-CP\cite{chen2022tensorf}    & 27.86                & 26.25                & 26.74     & 24.73                & 32.39                & 27.59 \\
        TensoRF-VM\cite{chen2022tensorf}  & 28.34 & 27.14 & 27.22 & 26.19 & \textbf{33.92} & 28.56 \\
        Ours-48 & \textbf{28.39} &	\textbf{27.32}	& \textbf{28.09} &	\textbf{26.58} &	33.13	& \textbf{28.70}
        \\ \hline
        \multicolumn{1}{l}{} & \multicolumn{1}{l}{} & \multicolumn{1}{l}{} & SSIM~$\uparrow$      & \multicolumn{1}{l}{} & \multicolumn{1}{l}{} &                      \\ \hline
        NV~\cite{lombardi2019neural}                   & 0.992                & 0.793                & 0.721     & 0.819                & 0.916                & 0.848                 \\
        NeRF~\cite{mildenhall2020nerf}                 & 0.920                & 0.860                & 0.750     & 0.860                & 0.932                & 0.864                 \\
        NSVF~\cite{liu2020neural}                 & 0.930                & 0.895                & 0.823     & 0.900                & 0.954                & 0.900                 \\
       TensoRF-CP\cite{chen2022tensorf}    &   0.934 & 0.885 & 0.839 &  0.879 & 0.948 & 0.897 \\
        TensoRF-VM\cite{chen2022tensorf}  &   0.948 & 0.914 &  0.864 & 0.912 & \textbf{0.965} & 0.920\\
        Ours-48 & \textbf{0.948} & \textbf{0.915} & \textbf{0.884} & \textbf{0.917} & 0.957 & \textbf{0.924} \\ \hline
        \multicolumn{1}{l}{} & \multicolumn{1}{l}{} & \multicolumn{1}{l}{} & LPIPS$_{Alex}\downarrow$ & \multicolumn{1}{l}{} & \multicolumn{1}{l}{} &                      \\ \hline
        NV~\cite{lombardi2019neural}                   & 0.117                & 0.312                & 0.479     & 0.280                & 0.111                & 0.260                 \\
        NeRF~\cite{mildenhall2020nerf}                 & 0.111                & 0.192                & 0.395     & 0.196                & 0.098                & 0.198                 \\
        NSVF~\cite{liu2020neural}                 & 0.106                & 0.148                & 0.307     & 0.141                & 0.063                & 0.153                 \\
        TensoRF-CP\cite{chen2022tensorf}  & 0.089 & 0.154 & 0.237 &  0.176 & 0.063 & 0.144\\
        TensoRF-VM\cite{chen2022tensorf}  &  \textbf{0.081} & 0.129 & 0.217 &  0.139 & \textbf{0.057} & 0.125\\
        Ours-48 & 0.083 & \textbf{0.123} & \textbf{0.167} & \textbf{0.125} & 0.065 & \textbf{0.113}\\ \hline
        \multicolumn{1}{l}{} & \multicolumn{1}{l}{} & \multicolumn{1}{l}{} & LPIPS$_{Vgg}\downarrow$  & \multicolumn{1}{l}{} & \multicolumn{1}{l}{} &                      \\ \hline
        TensoRF-CP\cite{chen2022tensorf}  & 0.106 & 0.202 & 0.283 & 0.227 & 0.088 & 0.181\\
        TensoRF-VM\cite{chen2022tensorf}  & \textbf{0.078} & \textbf{0.145} & 0.252 & 0.159 & \textbf{0.064} & 0.140\\
        Ours-48 & 0.083 & 0.150 & \textbf{0.216} & \textbf{0.154} & 0.078 & \textbf{0.136} \\ \hline
        \end{tabular}
        \caption{Quantity comparison on five scenes in the Tanks and Temples dataset \cite{Knapitsch2017} selected in NSVF \cite{liu2020neural}. NV, NeRF, and NSVF have not reported their  LPIPS$_{Vgg}$}
        \label{tb:tt}
    \end{table*}

\section{Mip-NeRF360 Dataset}
We evaluate our method on two scenes (one indoor scene and one outdoor scene) of Mip-NeRF360 dataset~\cite{barron2022mip}. Note that we only use the scene warping scheme the same as DVGO~\cite{sun2022direct} and Mip-NeRF360~\cite{barron2022mip} and keeping other components (i.e., positional encoding, point sampling, etc.) the same as TensoRF~\cite{chen2022tensorf}. The qualitative and quantitative results are shown in Fig.~\ref{fig:360_scene} and Tab.~\ref{tb:360_tab} , respectively. Here, we use only two scales in implementation to show our compactness and scalability.  

% Figure environment removed


% Figure environment removed

    
\end{appendices}






\end{document}