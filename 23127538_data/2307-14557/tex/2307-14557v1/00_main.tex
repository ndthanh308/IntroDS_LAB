
\documentclass[conference]{IEEEtran}
\usepackage{ifpdf}
\usepackage{cite}
\ifCLASSINFOpdf
 \usepackage[pdftex]{graphicx}
\else
 \usepackage[dvips]{graphicx}
\fi
\usepackage{amsmath}
\usepackage{amsfonts} 


\usepackage[ruled,vlined]{algorithm2e}

\newcommand{\algorithmicbreak}{\textbf{break}}
\newcommand{\Break}{\algorithmicbreak}
\usepackage{array}

\ifCLASSOPTIONcompsoc
   \usepackage[caption=false,font=normalsize,labelfont=sf,textfont=sf]{subfig}
\else
  \usepackage[caption=false,font=footnotesize]{subfig}
\fi
\usepackage{fixltx2e}
\usepackage{stfloats}
\usepackage{url}
\usepackage{xcolor}
% \usepackage{IEEEtrantools}

% \usepackage{authblk}


\newcommand{\SubItem}[1]{
    {\setlength\itemindent{15pt} \item[-] #1}
}
% *** Do not adjust lengths that control margins, column widths, etc. ***
% *** Do not use packages that alter fonts (such as pslatex).         ***
% There should be no need to do such things with IEEEtran.cls V1.6 and later.
% (Unless specifically asked to do so by the journal or conference you plan
% to submit to, of course. )
\usepackage{xcolor}
\usepackage{xspace}
% correct bad hyphenation here
\hyphenation{op-tical net-works semi-conduc-tor}
\newcommand\eat[1]{}
\newcommand{\cmtColor}[1]{\textcolor[rgb]{0.3, 0.8, 0.3}{#1}}

\newcommand{\mynote}[3]{%
  \ifthenelse{\boolean{showcomments}}{%
   \fbox{\bfseries\sffamily\scriptsize#1}%
   {\small$\blacktriangleright$\textsf{\emph{\color{#3}{#2}}}$\blacktriangleleft$}}%
  {%
   % these two lines ensure that there is no blank space inserted
   \@bsphack
   \@esphack
  }%
}
\newcommand{\mengyuan}[1]{\mynote{ML}{#1}{orange}}
\newcommand{\haoran}[1]{\mynote{HG}{#1}{cyan}}
\newcommand\notes[1]{\textcolor{black}{#1}}

\newcommand\final[1]{\textcolor{green}{#1}}

\newcommand{\edit}[1]{\textcolor{black}{#1}}
\newcommand{\name}{{X-Poly}\xspace}



\begin{document}
%
% paper title
% Titles are generally capitalized except for words such as a, an, and, as,
% at, but, by, for, in, nor, of, on, or, the, to and up, which are usually
% not capitalized unless they are the first or last word of the title.
% Linebreaks \\ can be used within to get better formatting as desired.
% Do not put math or special symbols in the title.
\title{Accelerating Polynomial Modular Multiplication with Crossbar-Based Compute-in-Memory}

% \author{\IEEEauthorblockN{Mengyuan Li\IEEEauthorrefmark{1}, Haoran Geng\IEEEauthorrefmark{1}, Michael Niemier, and Xiaobo Sharon Hu}
% \IEEEauthorblockA{Department of Computer Science and Engineering\\
% University of Notre Dame, Notre Dame, IN, USA, 46556\\
% E-mails: \{mli22, hgeng, mniemier, shu\}@nd\emaildot edu}
% \thanks{\IEEEauthorrefmark{1}These authors contributed equally to this work.}}



\author{\IEEEauthorblockN{Mengyuan Li*, Haoran Geng*, Michael Niemier,
Xiaobo Sharon Hu}
\IEEEauthorblockA{Department of Computer Science and Engineering\\
University of Notre Dame, Notre Dame, IN, USA, 46556 \\
E-mail: mli22@nd.edu  hgeng@nd.edu mniemier@nd.edu shu@nd.edu \\ * These authors contributed equally. } }



% make the title area
\maketitle
\pagestyle{plain}
% As a general rule, do not put math, special symbols or citations
% in the abstract
         

% no keywords


% For peer review papers, you can put extra information on the cover
% page as needed:
% \ifCLASSOPTIONpeerreview
% \begin{center} \bfseries EDICS Category: 3-BBND \end{center}
% \fi
%
% For peerreview papers, this IEEEtran command inserts a page break and
% creates the second title. It will be ignored for other modes.

%\IEEEpeerreviewmaketitle

\begin{abstract}

The Fast Reciprocal Square Root Algorithm is a well-established approximation technique consisting of two stages: first, a coarse approximation is obtained by manipulating the bit pattern of the floating point argument using integer instructions, and second, the coarse result is refined through one or more steps, traditionally using Newtonian iteration but alternatively using improved expressions with carefully chosen numerical constants found by other authors. The algorithm was widely used before microprocessors carried built-in hardware support for computing reciprocal square roots. At the time of writing, however, there is in general no hardware acceleration for computing other fixed fractional powers. This paper generalises the algorithm to cater to all rational powers, and to support any polynomial degree(s) in the refinement step(s), and under the assumption of unlimited floating point precision provides a procedure which automatically constructs provably optimal constants in all of these cases. It is also shown that, under certain assumptions, the use of monic refinement polynomials yields results which are much better placed with respect to the cost/accuracy tradeoff than those obtained using general polynomials. Further extensions are also analysed, and several new best approximations are given.

\end{abstract}

\section{Introduction}
\label{sec:intro}
\looseness -1
Gaining insight into population trends allows data analysts to make data-driven decisions to improve user experience.
Heavy hitter detection, or learning popular data points generated by users, plays an important role in learning about user behavior.
A well-known example of this is learning ``out-of-vocabulary" words typed on keyboard, which can then be used to improve next word prediction models. 
%The problem has been studied extensively in the setting where each user has a single data point, and one has no privacy concerns (see e.g. \cite{CHARIKAR20043, CORMODE200558}). However, t
This data is often sensitive and the privacy of users' data is paramount. When the data universe is small, one can obtain private solutions to this problem by directly using private histogram algorithms such as RAPPOR~\cite{erlingsson2014rappor}, and PI-RAPPOR~\cite{feldman2021lossless}, and reading off the heavy-hitters. However, when the data universe is large, as is the case with ``out-of-vocabulary" words, these solutions result in algorithms with either very high communication, or very high server side computation, or both.
Prefix-tree based iterative algorithms can lower communication and computation costs, while maintaining high utility by efficiently exploring the data universe for heavy hitters.
They also offer an additional advantage in the setting where users have multiple data points by refining the query in each iteration\longversion{ using the information learned thus far}, allowing each user to select amongst those data points which are more likely to be heavy hitters.

\looseness -1
In this work, we consider an iterative federated algorithm for heavy hitter detection in the aggregate model of differential privacy (DP) in the presence of computation or communication constraints. In this setting, each user has a private dataset on their device. In each round of the algorithm, the data analyst sends a query to the set of participating devices, and each participating device responds with a  \textit{response}, which is a random function of the private dataset of that user. These \textit{responses} are then summed using a secure aggregation protocol, and reported to the data analyst. The analyst can then choose a query for the next round adaptively, based on the aggregate results they have seen so far. The main DP guarantee is a user-level privacy guarantee on the outputs of the secure aggregator, accounting for the privacy cost of \emph{all} rounds of iteration. 
Our algorithm will additionally be DP in the local model of DP (with a larger privacy parameter)\footnote{A potential architecture for running iterative algorithms in this model of privacy is outlined in \cite{mcmillan2022private}.}.
\longversion{We do not assume that the set of participating devices is consistent between rounds. }

\looseness -1
In the central model of DP, there is a long line of work on adaptive algorithms for heavy hitter detection in data with a hierarchical structure such as learning popular $n$-grams~\cite{cormode2012differentially, Qardaji:2012, Song:2013, bagdasaryan2021towards, Kim2021DifferentiallyPN, mcmillan2022private}. These interactive algorithms all follow the same general structure. Each data point is represented as a sequence of data segments $d=a_1a_2\cdots a_r$ and the algorithm iteratively finds the popular values of the first segment $a_1$, then finds popular values of $a_1a_2$ where $a_1$ is restricted to only heavy hitters found in the previous iteration, and so on. 
This limits the domain of interest at each round, lowering communication and computation costs.
The method of finding the heavy hitters in each round of the algorithm varies in prior work, although is generally based on a DP frequency estimation subroutine. One should consider system constraints (communication, computation, number of participating devices, etc.) and the privacy model when choosing a frequency estimation subroutine. In this work, we will focus on using one-hot encoding with binary randomized response (inspired by RAPPOR~\cite{erlingsson2014rappor}) as our DP frequency estimation subroutine. Since we are primarily interested in algorithmic choices that affect the iterative algorithm, we believe our findings should be agnostic to the choice of frequency estimation subroutine used. 

We explore the effect on utility of different data selection schemes and algorithmic optimizations. 
We refer to our algorithm as \textit{Optimized Prefix Tree} (\textit{$\ouralgorithm$}). Our contributions are summarised below:

    \textbf{Adaptive Segmentation.} We propose an algorithm for adaptively choosing the segment length and the threshold for keeping popular prefixes. In contrast to prior works that treat the segment length as a hyperparameter, our algorithm chooses these parameters in response to user data from the previous iteration and attempts to maximize utility (measured as the fraction of the empirical probability distribution across all users captured by the returned heavy hitters), while satisfying any system constraints. We find that our method often results in the segment length varying across iterations, and outperforms the algorithm that uses a constant segment length. We also design a threshold selection algorithm that adaptively chooses the prefix list for the subsequent round. This allows us to control the false positive rate\longversion{ (the likelihood that a data point is falsely reported as a heavy hitter)}.
    
    \textbf{Analysis of the effect of on-device data selection mechanisms.} We explore the impact of interactivity in the setting where users have multiple data points. We observe empirically that when users have multiple data points, interactivity can improve utility, even in the absence of system constraints. In each round, users choose a single data point from their private data set to (privately) report to the server.
    The list of heavy hitters in the previous iteration provides a \emph{prefix list}, so users will only choose a data point with one of the allowed prefixes. If a user has several data points with allowed prefixes, then there are several selection rules they may use to choose which data point to report. Each user's private dataset defines an empirical distribution for that user. 
    We find that when users sample uniformly randomly from the support of their distribution (conditioned on the prefix list) then the algorithm is able to find more heavy hitters than when they sample from their empirical distribution (again conditioned on the prefix list). 

    \textbf{Analysis of the impact of inclusion of deny list.} Under the constraint of user-level differential privacy, each user is only able to communicate their most frequent data points, and less frequent data points are down weighted. We explore the use of a \emph{deny list} that asks users not to report data points that we already know are heavy hitters. In practice, a deny list may arise from an auxiliary data source, or from a prior run of the algorithm. Our analysis indicates even when the privacy budget is shared between multiple rounds of the algorithm, performing a second round equipped with a deny list improves performance.

The rest of the paper is organized as follows. In Section~\ref{sec:related} we discuss some of the prior works in privacy-preserving heavy hitters detection. Section~\ref{sec:privacy} explains the privacy primitives we used in this work. In Section~\ref{sec:alg} we elaborate the details of our prefix tree algorithm. Section~\ref{sec:post} explains the post-processing methods and the theoretical analysis behind it. Section~\ref{sec:expts} demonstrates the experimental results and in Section~\ref{sec:conclusion} we discuss the findings of our experiments. 


\section{Background}
\label{sec:background}
In this section, we discuss the role of PMM in cryptography, describe various PMM methods, review existing strategies for accelerating PMM, and review the concept of XBA.
\subsection{PMM in Cryptography}

The RLWE problem~\cite{RLWE}, foundational to lattice-based cryptography~\cite{lattice_theory}, and specifically to HE schemes~\cite{HEStandard}, leverages polynomials over a specific ring for its operations. HE, which enables arbitrary computations on encrypted data without prior decryption, ensures secure computation in untrusted environments while preserving data privacy. The primary computational bottleneck in HE arises from the need to perform polynomial arithmetic, particularly PMM~\cite{B/FV,BGV,CKKS}. Consequently, enhancing PMM's performance with respect to latency and energy consumption becomes critical in cryptography.


\subsection{PMM}\label{sec:pmm}

\edit{Polynomial modular multiplication (PMM) is a fundamental operation in various applications, including cryptography, error correction codes, and polynomial arithmetic. It involves multiplying two polynomials and reducing the result modulo a given polynomial, resulting in a polynomial of a lower degree. By performing PMM, it becomes possible to efficiently compute large polynomial expressions while maintaining the desired modulus properties.
}

\edit{PMM can be accomplished using various methods, including the Conv1D approach and more optimized solutions like NTT as shown in Fig.~\ref{fig:pmm}(a). The Conv1D approach for PMM follows a straightforward procedure (Fig.~\ref{fig:pmm}(a)(1)). Two polynomials $A(x)$ and $B(x)$, with polynomial degree $n$ and modulo $q$,  are multiplied by summing the corresponding terms, akin to Conv1D computation with time complexity of $O(n^2)$. Then, the product undergoes modular reduction by dividing it with a modulus polynomial. The remainder is extracted polynomial long division to get the final result $P(x)$.}

\edit{NTT, alternatively, is proposed to reduce the computational complexity of PMM, particularly when the modulus polynomial satisfies specific properties, such as being irreducible and having a specific degree~\cite{NTT}. As depicted in Fig.~\ref{fig:pmm}(a)(2), the NTT approach involves transforming the polynomials into a different domain through NTT. During the NTT transformation, butterfly computations are performed by combining pairs of coefficients and multiplying them with twiddle factors, which are complex values associated with the modulus polynomial, resulting in the frequency-domain representation of the polynomial~\cite{NTT}. The process has a time complexity of $O(n\log{n})$. Then in this transformed domain, element-wise multiplication is performed, followed by the inverse NTT (INTT) to convert the result back to the original domain to obtain the final polynomial $P(x)$. Modular reduction is applied after each domain transformation.}

\edit{The computational complexity of PMM in hardware is primarily influenced by two key factors: the polynomial degree $n$, which represents the number of coefficients in a polynomial, and the bitwidth $k$ of modulo $q$, which signifies the size of these coefficients. In real-world applications, such as HE in privacy-preserving machine learning inference, these parameters can be quite substantial. For instance, the polynomial degree $n$ in these applications can range from 256 to 8192, while the bitwidth $k$  can vary from 16 bits to 64 bits \cite{GAZELLE,cheetah}. The magnitude of these degrees and bitwidths significantly intensifies the computational complexity of a single PMM, presenting a considerable challenge in the field.}

% Figure environment removed

\subsection{Related Work}

In this section, we briefly review existing efforts to accelerate PMM. As existing work primarily employs NTT-based solutions, we focus our review accordingly, discussing both traditional ASIC and FPGA solutions, as well as CIM-based accelerators.

\subsubsection{ASIC and FPGA solutions}

 Nejatollahi, H., et al.~\cite{fpga_ntt} proposed an innovative FPGA solution by designing two high-throughput systolic array polynomial multipliers, one based on NTT and the other on convolution. Their sequential NTT-based multiplier yielded a 3$\times$ speedup over the SOTA FPGA implementation of the polynomial multiplier in the NewHope-Simple key exchange mechanism on an Artix7 FPGA \cite{newhope}. 

ASIC implementations of lattice-based cryptographic protocols have also been actively studied. LEIA~\cite{leia}, a high-performance lattice encryption instruction accelerator, and Sapphire~\cite{sapphire}, a configurable processor for low-power embedded devices, both demonstrate substantial performance improvements and energy efficiency compared to prior ASIC designs.

There are also a number of works that directly accelerate HE, inherently accelerating PMM~\cite{HE_F1,craterlake,BTS,cheetah,CIM_HE_SAC}. These works, which also typically use NTT, aim to create large-scale accelerators for privacy-preserving computations. Since this paper focuses on PMM, we will not compare it to these works. It suffices to say that an efficient PMM accelerator will directly help HE implementations. 



\subsubsection{Compute-in-Memory solutions}
\edit{Previous research has introduced a variety of CIM kernels, including crossbars and general-purpose CIM. Ranjan et al.~\cite{ranjan2019x} have demonstrated that XBAs excel at performing VMM. Reis et al.~\cite{reis2018computing} have discussed the general-purpose CIM enabling Boolean logic and arithmetic operations to be executed directly within the memory.  Additionally, ongoing researches focus on exploring different underlying technologies for implementing these CIM kernels, including CMOS, ReRAM, and Ferroelectric FET (FeFET)~\cite{ni2019ferroelectric}. These technologies are actively studied due to their potential to provide higher density and lower latency/energy overhead in CIM architecture.}
Several research efforts have explored the use of CIM architectures for the acceleration of the NTT, including CryptoPIM~\cite{CryptoPIM}, MENTT~\cite{MENTT}, RMNTT~\cite{rmntt} and BPNTT~\cite{bpntt}. We compare \name against these established researches, so we concisely introduce these approaches in the following discussion.

CryptoPIM, MENTT, and BPNTT proposed efficient NTT accelerators based on general-purpose CIM kernels. CryptoPIM \cite{CryptoPIM} and MENTT \cite{MENTT}, built on ReRAM and SRAM respectively, both introduced unique mapping strategies to streamline the data flow between NTT stages, leading to significant reductions in latency, energy, and area overheads. BPNTT presented an in-SRAM architecture using bit-parallel modular multiplication, significantly improving throughput-per-watt. 

RMNTT~\cite{rmntt} proposed an NTT accelerator using ReRAM-based XBAs. RMNTT stores the modified twiddle factor matrix in the XBAs and employs a modified Montgomery reduction algorithm to perform modular reduction on the VMM results. The evaluation results in~\cite{rmntt} show that RMNTT outperforms other NTT accelerators in terms of throughput but incurs a large area overhead.



\subsection{Crossbars} \label{sec:Crossbar}
Given the competitiveness of XBA-based NTT accelerators, we consider leveraging XBAs to accelerate PMM. We briefly review the XBA basics below. 

XBA~\cite{SWIPE} is one representative CIM kernel in which every input signal is connected to every output signal through their cross-points consisting of memory elements and selectors. XBAs can efficiently implement VMM and have been widely studied for CNNs. In particular, XBA implemented with nonvolatile memory (NVM) devices such as ReRAM~\cite{wu2018methodology} have gained popularity due to their high storage density, nonvolatility, and low energy consumption. \edit{However, XBAs face challenges stemming from the underlying memory devices and circuits. In-situ memory device nonidealities, e.g., non-linearity, thermal noise, and variations, impact computed accuracy.}

Fig.~\ref{fig:crossbar}(a) illustrates a general XBA structure. For each column, we adopt the current summing model as shown in Fig.~\ref{fig:crossbar}(b). In this work, both input voltage ($V_{j}$) and memory cell states ($G_{i,j}$) assume binary values, i.e., $I_{i} = \sum_{0}^{R-1}G_{ij}V_{j}$, where $V_{j}$ and $G_{ij}$ are either 0 or 1. Binary XBAs exhibit greater robustness to device and circuit nonidealities, and offer improved scalability.

% Figure environment removed
%%%%%%%%%%%%%%%%%%%%%%%%%%%%%%%%%%%%%%%%%%%%%%%%%%%%%%%%%%%%%%%%%%%

\section{NTT vs. Conv1D} \label{sec:discussion}
%%%%%%%%%%%%%%%%%%%%%%%%%%%%%%%%%%%%%%%%%%%%%%%%%%%%%%%%%%%%%%%%%%%

The choice of PMM algorithm is critical to achieving high performance in terms of speed, noise, and area in the context of the CIM computing paradigm as discussed in Sec.~\ref{sec:pmm}. Two commonly used methods for performing PMM are Conv1D and  NTT. Recent efforts utilizing XBAs for PMM have primarily focused on accelerating NTT-based methods~\cite{rmntt}~\cite{iedm_ntt}. However, there is no systematic comparative study of which method, Conv1D or NTT, is a better fit for leveraging XBAs to accelerate PMM. We fill this gap with an in-depth investigation below. Our study reveals three key insights which favor the Conv1D over the NTT-based approach. First,  data mapping complexity is higher when using NTT. Second, the Conv1D method potentially offers a better performance trade-off in terms of area and throughput, providing more opportunities for design scalability. Third, the noise growth is generally higher in the NTT approach than Conv1D, which can negatively impact the performance and accuracy of the system. Below, we elaborate on these insights.

\subsection{The Impact of Data Mapping to XBAs}

% Data mapping is a crucial aspect of computational-in-memory (CIM) as it significantly affects the efficiency and accuracy of the system. In previous NTT-type PMM implementations, a vector-matrix-multiplication (VMM) style mapping was required, as shown in \notes{Figure X}. This means that despite the use of NTT, the data mapping on XBA arrays still follows a VMM style, which is similar to the conventional Conv1D implementation.

To use XBAs for PMM, both NTT-based and Conv1D-based PMM approaches require converting their respective operands into matrices and performing VMM on XBAs\cite{rmntt}. In the NTT-based approach, the twiddle factor of NTT must be converted into a matrix. In the Conv1D approach, one of the polynomials is transformed into a matrix, while the other remains a vector, facilitating the execution of VMM. Data mapping to XBAs in NTT and Conv1D can be better visualized in Fig.\ref{fig:pmm}(b) and (c). The figures show that the same number of memory cells are needed for both methods; thus, NTT does not provide benefits over Conv1D in terms of the XBA area. Also, due to the butterfly computation involved in NTT, converting the twiddle factors into a matrix is significantly more complex than converting a polynomial into a matrix for Conv1D~\cite{rmntt}. 

The end-to-end computational complexity of NTT-based PMM on XBAs is actively higher than directly mapping Conv1D into XBAs. As depicted in Fig.~\ref{fig:pmm}(a), NTT-based PMM involves three main steps: NTT computation ($O(n\log n)$ complexity), element-wise multiplication ($O(n)$ complexity), and INTT computation ($O(n\log n)$ complexity). In contrast, Conv1D-based PMM has a complexity of $O(n^{2})$. However, when utilizing XBA acceleration, the complexity of Conv1D-based PMM can be reduced from $O(n^{2})$ to $O(1)$. By employing similar data mappings, NTT and Conv1D exhibit the same time complexity on XBA. Therefore, Conv1D-based PMM on XBA demonstrates a lower end-to-end complexity compared to NTT-based PMM, as it requires fewer operations—Conv1D only necessitates $O(1)$ operations, while NTT involves $O(1)$ + $O(n)$ + $O(1)$ operations.

\subsection{Performance Analysis}\label{sec:Performance_analysis}
% \notes{Table X} summarizes the area and operation requirements of Conv1D and NTT-type PolyM in XBA-type CIM applications. While NTT-type PolyM reduces the time complexity from $O(N^2)$ to $O(NlogN)$ on conventional computing platforms such as CPUs and GPUs, it does not offer significant speed or area benefits compared to the conventional Conv1D approach in XBA-type CIM implementations due to the similar data mapping resulting in similar operation requirements. Therefore, the choice between the two implementations depends on the specific requirements of the application, including the desired trade-offs between accuracy, efficiency, and complexity.

NTT-based PMM requires that the twiddle factors be stored for NTT and INTT in the XBAs~(See Fig~\ref{fig:pmm}(c)). The stored twiddle factors approach  necessitates either frequent updates to the twiddle factors stored in the XBAs or the use of additional XBAs to store all twiddle factors needed for NTT. As a result, this leads to either higher latency and energy consumption or increased area. Alternatively, Conv1D-based PMM has numerous identical values that, when stored in XBAs, can be reused repeatedly. This provides the opportunity to devise intelligent data reuse schemes (see Sec. \ref{sec:poly_mapping}), ultimately leading to more efficient and optimized solutions in terms of area and energy consumption. Therefore, Conv1D-based PMM can be a more promising method for accelerating PMM with XBAs.


\subsection{Noise}
% Noise is a significant challenge in CIM computing, especially for those built on non-volatile devices, as the device's non-idealities can cause noise in the computing output. Additionally, the noise can increase with more operations.

As discussed in Sec~\ref{sec:Crossbar}, XBAs are susceptible to accuracy degradation stemming from the intrinsic nonidealities of the memory cells, and the limitation of ADC precision. As a result, using XBAs inevitably introduces a certain amount of noise (i.e., error) in VMM results. When implemented on XBAs, Conv1D-based PMM incurs less noise than NTT-based PMM. The primary reason is that in Conv1D-based PMM, the entire computation can be completed in one step in XBAs, which helps control the magnitude of the noise. However, in NTT-based PMM, the NTT, element-wise multiplication, and INTT must be performed, which increases the noise introduced by XBAs multiplicatively (See Fig~\ref{fig:pmm}(a)). In applications such as HE, higher noise levels are not tolerable, making NTT-based XBA PMM unsuitable for such applications.

Based on the observations in this section, we believe that Conv1D-based PMM is a better approach for accelerating PMM with XBAs. We thus focus on the design and optimization of the XBA fabric to accelerate Conv1D-based PMM. 



\section{\name}\label{sec:design}
\eat{
% Figure environment removed}


% Figure environment removed

\edit{Design and optimization of Conv1D-based PMM on XBAs  for long polynomials must solve several key problems. These include mapping data to XBAs to efficiently  use the resources, enhancing memory utilization at the Processing Element (PE) level, and effectively implementing modular reduction strategies. We present \name for accelerating the Conv1D-based PMM and provide tailored solutions to address the aforementioned challenges.}

% \edit{In this section, we provide an overview of \name (Sec.~\ref{sec:overview}), the data mapping techniques (Sec.\ref{sec:mapping}), and details of our polynomial mapping strategies that aim to improve memory utilization  (Sec.\ref{sec:poly_mapping}). Lastly, we outline our proposed modular reduction strategies (Sec.\ref{sec:modular}) and conclude with a comprehensive breakdown of the entire computation flow of our XBA-based PMM accelerator design (Sec.~\ref{sec:flow}).}

% \name builds on the Conv1D implementation design and optimizes it for long polynomials. We first present an overview of our design approach in Sec.~\ref{sec:overview} and a detailed discussion of our novel data mapping techniques in Sec.~\ref{sec:mapping}. We then describe our PE-level optimization strategies to improve memory utilization in Sec.~\ref{sec:optim}. Lastly, we discuss our proposed modular reduction study in Sec.~\ref{sec:modular} and present the entire computation flow of our new XBA-based PMM accelerator design in Sec.~\ref{sec:flow}.


\subsection{Overview}\label{sec:overview}

The high-bitwidth long polynomials employed in cryptographic algorithms like HE propose challenges for the design of XBA-based architecture. One specific issue relates to the limited size of the XBA. For instance, an array with 128 rows and 128 columns falls short in accommodating high-bitwidth polynomials with a degree exceeding 256.

To address the challenge, \name utilizes a hierarchical approach to address computational complexity. Fig.~\ref{fig:overall} illustrates the overall structure and data mapping of \name, consisting of the tile, PEs, and XBAs. The tile (Fig.~\ref{fig:overall}(1)) contains multiple PEs, an accumulator, and a specifically designed reduction unit for modular reduction. Each PE holds one-bit weights and shares the same input. Thus, $k$ PEs can store $k$-bit polynomials from the most significant bit (MSB) to the least significant bit (LSB), working in parallel. 


The PE (Fig.~\ref{fig:overall}(2)) is composed of multiple XBAs working on different parts of the polynomials simultaneously, as well as an adder tree and a shifter. The XBAs (Fig.~\ref{fig:overall}(3)) are used for coefficient multiplication, while the adder tree and shifter within each PE accumulate partial results from each XBA and perform shift-add operations. 


\subsection{Bit Mapping}\label{sec:mapping}
% Description of the new XBA data mapping for high-bitwidth data
% Reduction of fine-grained shift-add operations

The high bitwidth and large polynomial degree required for cryptographic applications need a large number of shift-add operations, which may not be efficiently supported in a CIM architecture. Due to the limited precision of a memory cell in an XBA, we need to map the bits of weight into multiple memory cells. Fig.~\ref{fig:mapping}(a) illustrates the conventional approach for mapping the high bitwidth weight to multiple XBAs. All bits of weight are stored in multiple columns of the XBA. When input arrives at the XBA, each column conducts a multiplication operation. Immediately following this, shift-adders carry out the shift-add operations after the XBA computation. This XBA-level shift-add operation requires lots of shift-adders and is expensive in terms of both time and energy. 

As such, in this work, we propose a new bit mapping (BM) technique that groups the same bit of all weights together, as shown in Fig.~\ref{fig:mapping}(b). For example, in the case of 4x4 2-bit weights distributed among 2 PEs (4 XBAs per PE), each PE process one bit of each weight. After all PEs process one input bit, the shift operation is performed at the PE level, thereby avoiding a costly array-level shift-add operation. Comparing the conventional mapping (Fig.~\ref{fig:mapping}(a)) and the bit mapping (Fig.~\ref{fig:mapping}(b)) in the example, the number of shift-adders is reduced from 8 to 2. 


% Figure environment removed



As will be seen, this bit mapping strategy can significantly improve both the area and speed for processing high-bitwidth polynomial-based workloads in XBAs. In addition to its benefits for shift operations, the BM technique also simplifies the design of the PE. Since each PE handles a bit of each polynomial, the data patterns are captured at the polynomial coefficient level. We can simultaneously perform mapping optimization for all PEs. Thus, this technique can be easily extended to accommodate polynomials with different degrees or bitwidths, making it a flexible solution for performing polynomial operations in XBAs.


\eat{
% Figure environment removed
}



\subsection{Polynomial Mapping}\label{sec:poly_mapping}


In our PMM approach (utilizing VMM in XBAs), we first map polynomials into matrices to facilitate computation. Each polynomial is converted into a matrix by horizontally shifting the coefficients of the polynomial across each row, with any remaining gaps filled with zeros. This procedure results in a matrix structure that supports the critical shift-add operations intrinsic to PMM. 

Mapping the matrices into XBAs in our PMM approach using VMM is straightforward. However, in an effort to further optimize this mapping scheme, we noted that for any given polynomial degree $n$ and XBA row length $x$, there is a consistent pattern of repeated XBAs. Specifically, in every instance, we require $n/x$ identical XBAs to represent the polynomial matrix.

This aspect of our design stands in contrast with NTT-based XBA designs~\cite{rmntt}, which often find themselves confined to specific polynomial parameter settings. As such, \name offers a significant increase in flexibility. For example, consider two application scenarios for privacy-preserving machine learning (PPML) inference as shown in~\cite{GAZELLE}. On server-side inference, where performance is prioritized, and energy or area constraints are less critical, \name can leverage a larger count of XBAs for high-throughput PMM in HE of PPML. For edge-device inference, where area and energy efficiency are paramount, \name can efficiently handle a variety of large polynomial degrees and bitwidths with a smaller number of XBAs. A detailed study of the scalability of our design, referred to as \name, is provided in Sec. \ref{sec:memory_utilized}.




\subsection{Modular Reduction}\label{sec:modular}
Modular reduction is a crucial step in PMM, which ensures that the resulting polynomial remains within a specified degree and coefficient bounds. The essential steps in the reduction process include selecting an appropriate modulus for the ring and performing the modulo operation on the degree and coefficients of the resulting polynomial.

In \name, we utilize a variant of the Barrett reduction~\cite{Barrett} technique for efficient modular reduction. This method is known for its effectiveness in cryptographic applications and modular arithmetic, as it can compute the remainder of a division operation without performing the division itself. Notably, to minimize computation overhead in reduction, we strategically pre-compute specific parameters. This strategy transforms the complex, time-consuming multiplication and division operations into  shift operations, effectively reducing computation time and optimizing the overall reduction process.





\subsection{Computation Flow}\label{sec:flow}

Assuming polynomial $A$ is mapped onto XBAs in \name, PMM can be accomplished as follows. (1) \textbf{Input processing}: We begin by bit-slicing each element in the new polynomial B, separating it into its individual bits. (2) \textbf{PE computation:} Within each PE, different arrays handle distinct sections of the polynomial and perform multiplications with corresponding sections of the input. The results are then summed and shifted at the PE level. This bit-by-bit input process continues until all input bits have been addressed. (3) \textbf{Tile accumulation:} Afterward, the results from all PEs are accumulated at the tile level, and the partial results obtained from each PE are combined. (4) \textbf{Tile reduction:} Finally, a tile-level reduction operation is applied for efficient modular reduction. 

We also prioritize maximizing throughput in our design by incorporating a three-stage pipeline into the \name workflow to enhance the PMM process. This pipeline, which encompasses the PE computation, tile accumulation, and tile reduction stages, enables efficient synchronization and overlapping operations. 
%%%%%%%%%%%%%%%%%%%%%%%%%%%%%%%%%%%%%%%%%%%%%%%%%%%%%%%%%%%%%%%%%%%

\section{Evaluation}
\label{sec:evaluation}

In this section, we present the evaluation of \name. We begin by discussing our implementation setup and evaluation tools and follow with a comparison with both the CPU-based solutions as well as other hardware accelerators. We will quantitatively assess the performance benefits from \name. We then evaluate our bit mapping technique, with a focus on energy and area savings. Then we study the throughput per area performance of \name, demonstrating its superior performance over other SOTA CIM accelerators. Finally, we assess the scalability of \name and highlight its versatility in handling diverse polynomial degrees and bitwidths.

\subsection{Implementation Setup}
To verify the functionality of \name and evaluate performance characteristics such as latency, energy, and area, we have assembled a comprehensive evaluation framework. 

This framework considers the simulation of hardware components, including the modular reduction unit, shift-adders, accumulators and XBA arrays. We implemented the reduction unit, shift-adder, and accumulator using RTL, coded in Verilog and evaluated the energy consumption and area of these components using the RTL synthesis tool Cadence Encounter, paired with the 45nm CMOS predictive technology model (PTM) \cite{cao2011predictive}. We used Neurosim ~\cite{lu2021neurosim} to estimate the latency, energy, and area of the ReRAM-based XBAs, as well as successive-approximation-register (SAR) ADCs assuming the same 45nm technology node. The size of each XBA is 128 rows $\times$ 128 columns and one ADC is shared by 8 columns. 


We then incorporated the aftermentioned simulation-based results into our Python-based cycle-accurate simulator. This simulator tracks the pipeline stages for a given PMM operation and computes the cycle count and total energy consumption by emulating the operations of each hardware component on a cycle-by-cycle basis. This evaluation framework allows us to generate a holistic and precise assessment of the overall performance of a PMM in \name.


\begin{table*}[b]
%\vspace*{-2mm}
\centering
\caption{Comparison between \name and other SOTA solutions on a 256-point polynomial. Technology size: 45nm}
\label{tab:results}
\begin{tabular}{c|c|c|c|c|c|c|c|c|c c}
\hline 
\hline
& \multicolumn{2}{c|}{PMM Solutions} & \multicolumn{4}{c|}{NTT Solutions (CIM)} & \multicolumn{3}{c}{NTT Solutions (Non-CIM)} \\\hline

Design   & \textbf{\name} & \textbf{CPU} & \textbf{RMNTT} & \textbf{BPNTT} & \textbf{MENTT}& \textbf{CryptoPIM} & \textbf{FPGA}& \textbf{LEIA} & \textbf{Sapphire}\\ 
\hline
Device   &   ReRAM & CMOS & ReRAM & SRAM & SRAM & ReRAM & CMOS & CMOS  & CMOS\\ 
Frequency (MHz) & 400 & 2.5k & 400 & 3.8K & 218 & 909 & 164 & 267 & 64\\
Bit width & 16 & 16 & 14 & 16 & 14 & 16 & 16 & 14 & 14\\
\hline
Area ($mm^2$) & \textbf{0.27}  & - & $0.76^{*}$ & 0.063 & 0.173 & 0.152 & - & 1.77 & 0.354 \\
Latency ($us$) & \textbf{0.32} & 56 & 0.44 & 61.9 & 15.9 & 68.7 & 24.3 & 0.6 & 20.1\\
Energy ($nJ$) & 308.07 & - & 429.91 & 69.4 & 47.8 & 2.6k & 3.1k & 44.1 & 236.3\\
Throughput\$ (KOP$/s$) & $3.1k$ & - & $2.2k$  & $258.6$ & $62.8$ & $553.3$ & $41.2$ & $1.7k$ & $49.7$   \\
Throughput/Area\$ (KOP$/s/mm^2$) & \textbf{11.4k} & - & $2.9k$  & $4.1k$ & $364$ & $3.6k$ & - & $940.6$ & $140.1$  \\
\hline
\hline
  \multicolumn{11}{@{}p{\linewidth}@{}}{\small * We estimate the area for RMNTT based on the information reported in the paper. We utilize the same XBA area and peripheral components as \name for the sake of comparison.} \\
  \multicolumn{11}{@{}p{\linewidth}@{}}{\small \$ We evaluate the throughput based on the type of operations performed in the corresponding accelerators. We report the throughput of PMM for both X-Poly and CPU as well as the throughput of NTT for other accelerators as reported in the literature.}
\end{tabular}

%\vspace*{-5mm}
\end{table*} 
 
\subsection{Comparison with SOTA Solutions}
\subsubsection{Comparison with CPU} We first compared our \name implementation with a CPU implementation that performs PMM with a SOTA C++ library (Number Theory Library version 11.5.1 \cite{ntl}). An Intel(R) Xeon(R) CPU E5-2680 v3 operating at 2.50GHz was used for the CPU implementation. The results are shown in Table~\ref{tab:results} (col 3). The latency of the \name design is 200$\times$ better than the CPU implementation. Performance enhancement is primarily due to the parallel compute capability and fast multiplication inherent in the XBAs in our CIM-based architecture, allowing for a much more efficient PMM execution.

\subsubsection{Comparison with other accelerators} Next, we compared \name with other SOTA accelerators. As current accelerators for PMM only use NTT, we compare our approach to SOTA accelerators that support NTT given a polynomial degree of 256. That said, NTT solutions require additional multiplications and the INTT to obtain final PMM results. Compared to \name, this may increase overall latency and energy consumption by 2$\times$. Moreover, with \name, we can generate PMM results in a single step without the need for additional multiplication or INTT.

\textbf{XBA solutions:} We first compared our implementation with other CIM solutions, specifically with ReRAM implementations. \edit{We scaled the latency and energy of~\cite{rmntt} to 45nm for a fair comparison to \name, following the methodology outlined in~\cite{bpntt}. Given that the study in~\cite{rmntt} did not provide area results, we carried out an estimation using the mapping methodology introduced in their publication. We assume the same area for XBAs and peripheral components as \name.}  Although our design exhibited similar latency to RMNTT, our improved mapping technique results in a significantly reduced area. That is mainly because we reduce the footprint of shift-adders to just 20\% of the original area by using the proposed BM technique, thereby leading to a 3.9$\times$ improvement in the throughput-per-area ratio. %Our improved mapping technique allows us to utilize the available resources more efficiently, thus reducing the number of computational elements required. 

\textbf{Compute in SRAM solutions:} We also compared our implementation with in-SRAM solutions. The \name approach also improves throughput and throughput-per-area. Again, this can be attributed to both the parallel computing capability and the fast multiplication feature of our XBA-type mapping technique, which enables us to perform multiple computations simultaneously. More specifically, throughput is improved by 11$\times$, and throughput-per-area is improved by up to 3$\times$.

\textbf{Non-CIM solutions:} Finally, we compared \name with non-CIM solutions. The \name design is advantageous as it can store entire polynomial coefficients inside the XBAs. This feature eliminated the need for frequent access to on-chip memory for coefficients in long polynomials, which reduces data movement between the computing unit and the on-chip memory. This results in a reduction in both latency and energy consumption. Overall, \name outperformed ASIC and FPGA solutions in terms of throughput and energy efficiency. Compared to SOTA FPGA implementations, \name can achieve a remarkable 75$\times$ throughput improvement. When compared against SOTA ASIC implementations, \name can achieve a 2$\times$ throughput improvement.

\subsection{Bit Mapping Study}
We now evaluate the energy and area benefits of the proposed BM technique, discussed in Sec.\ref{sec:design}. To evaluate performance, we consider two scenarios: (1) conventional mapping as the implementation of RMNTT, the SOTA XBA-based NTT accelerator, and (2) our proposed BM technique. Fig.\ref{fig:MappingBenefit} illustrates the shift-adder area/energy and ADC area/energy given various polynomial degrees for each mapping. Results suggest that due to the large polynomial degrees and high bitwidths associated with the PMM, the peripherals (such as the shift-adders) in the design with conventional mapping consume a significant proportion of the energy and area. Moreover, this escalates with polynomial degrees. However, our proposed BM technique decreases the area for shift-add operations by 80\%, leading to an additional 3$\times$ reduction in overall area. Moreover, compared to conventional mapping, our design has lower latency and energy consumption. 

Fig.~\ref{fig:breakdown} illustrates the area and energy breakdown of our proposed design. This analysis further reveals that the majority of the energy consumption and area is spent on ADC operations, with the proposed mapping technique reducing the energy and area consumption for other peripherals significantly. 

% Figure environment removed

% Figure environment removed

\subsection{Throughput per Area Study}
Table~\ref{tab:results} shows that the XBA-based solutions (\name and \cite{rmntt}) achieve higher throughput but require a larger area than the in-SRAM solution~\cite{bpntt}. This is due to the inherent design of XBA-based solutions: they require a more expansive area to accommodate an increase in both polynomial degree and bitwidth~\cite{rmntt}. In-SRAM solutions can support larger parameter sizes within a similar area. However, this is accompanied by a substantial reduction in throughput. However, \name reduces XBA area  while maintaining its high throughput. 

To further understand the trade-off between throughput and area, we conducted an analysis of the throughput per area performance and compare the results with other SOTA CIM solutions. We consider a range of polynomial degrees and bitwidths, to generate a comprehensive perspective regarding the strengths of our design.

Fig.~\ref{fig:throughputarea} illustrates the throughput per area performance of our design, as well as the SOTA XBA design in \cite{rmntt} and the in-SRAM design in \cite{bpntt}. Our results show that \name can achieve significantly better throughput-per-area performance than both of these solutions, even as the parameter size increases. This highlights how \name  can lead to decreased area consumption of the XBA-based solution without compromising  the throughput. \eat{Our new data mapping technique reduces the peripheral cost and high memory density of the emerging ReRAM device.}

% Figure environment removed


\eat{
% Figure environment removed}

% Figure environment removed

\subsection{Scalibility of \name}
\label{sec:memory_utilized}

Modern applications like HE in privacy-preserving machine learning often choose polynomials with a large degree and bitwidth\cite{HE_F1,GAZELLE}. Storing these entirely within XBAs demands a high number of arrays, leading to significant area usage and energy consumption. \eat{As shown in Fig.~\ref{fig:latency}, we observed an increasing trend in area and energy as we increased both polynomial degree and bitwidth in our design.}

Our polynomial mapping scheme (Sec~\ref{sec:poly_mapping}) allows us to reuse arrays. This enables us to employ a smaller number of XBAs to accommodate larger polynomials. However, reusing XBAs could potentially affect our design's latency. To address this, we conducted an experiment where given a fixed number of XBA arrays, we assessed the capability of \name to adapt to various polynomial degrees and bitwidths. The objective here was to determine how we could optimize  \name to maximize design throughput under different polynomial degrees and bitwidths constraints. 


The left graph in Fig.~\ref{fig:throughput} depicts the maximum throughput of \name using different numbers of XBAs. We considered polynomial degrees ranging from 256 to 2048, with a fixed bitwidth of 16. The right graph demonstrates the maximum throughput for different bitwidths ranging from 8 to 64, while maintaining a constant polynomial degree of 512. Our experiments highlight that our design is capable of managing a wide array of polynomial degrees and bitwidths while maintaining a fixed number of XBAs. As anticipated, higher degrees and bitwidths require longer computation times due to the necessity for reuse of the same arrays within the pipeline. By modifying the number of XBAs, we can manage the balance between area and throughput. Overall, our design showcases robust scalability, effectively adapting to a broad spectrum of polynomial degrees and bitwidths.




\section{Conclusion}\label{sec:conclusion}

This paper presents our empirical domain knowledge distillation framework using ChatGPT and discusses our observations from the framework application experiments in the autonomous driving domain. The key finding is that: 1) with proper design of prompt engineering and execution flow, fully automated domain knowledge (in the ontology format) distillation is possible. However, due to the randomness in the response and the butterfly effect, the quality of fully automated distillation results is not guaranteed. To address this, we develop a web-based assistant to enable manual supervision and early intervention at runtime. We hope our findings and tools inspire future research toward revolutionizing the engineering processes of knowledge-based systems across domains.
\pagebreak

\bibliographystyle{IEEEtran}      
\bibliography{references}      

% that's all folks 
\end{document}