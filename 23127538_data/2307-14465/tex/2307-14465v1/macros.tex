% generally useful macros for writing
\newcommand{\TexDir}{/home3/aiken/tex}

% [[[There should really be a list environment for defining research questions, so that they can be given labels.]]]
\newcommand{\RQ}[1]{\textbf{RQ#1}\xspace}

% EMSE revision
\newcommand{\revision}[1]{\textcolor{blue}{#1}}

% These allow switching interline spacing; the change takes effect immediately:

\makeatletter
\newcommand{\singlespacing}{\let\CS=\@currsize\renewcommand{\baselinestretch}{1}\tiny\CS}
\newcommand{\oneandahalfspacing}{\let\CS=\@currsize\renewcommand{\baselinestretch}{1.25}\tiny\CS}
\newcommand{\doublespacing}{\let\CS=\@currsize\renewcommand{\baselinestretch}{1.5}\tiny\CS}
%setspacingto sets the interline spacing to the value of its argument
% e.g., \setspacingto{1.5} is the same as \doublespacing
\newcommand{\setspacingto}[1]{\let\CS=\@currsize\renewcommand{\baselinestretch}{#1}\tiny\CS}
\makeatother

% nonumber
\newcommand{\nn}{\nonumber}

% Tab for hand-formatting:
\newcommand{\tab}{\hspace*{2em}}

% Angle brackets:
\newcommand{\la}{\langle}
\newcommand{\ra}{\rangle}

% s.t.
\newcommand{\st}{\mbox{\ s.t.\ }}

% otherwise
\newcommand{\ow}{\m{\rm otherwise}}

% if
\newcommand{\mif}{\m{\rm if\ }}

% Harpoons
\newcommand{\rh}{\rightharpoonup}
\newcommand{\lh}{\leftharpoonup}

% Denotational-semantics-style brackets and bottom:
\newcommand{\lbk}{\lbrack\!\lbrack}
\newcommand{\rbk}{\rbrack\!\rbrack}
\newcommand{\bottom}{\perp}

% Projection operator
\newcommand{\proj}{\!\downarrow\!}

% macros for mbox combined with another style
% (useful for changing typefaces in math mode)
\newcommand {\mboxbf}[1]{\mbox{{\bf #1}}}
\newcommand {\mboxit}[1]{\mbox{{\it #1}}}
\newcommand {\mboxem}[1]{\mbox{{\em #1}}}
\newcommand {\m}{\mbox}
\newcommand {\ch}{\rm}
\newcommand{\change}[1]{{\color{red!60!white}\textbf{}~#1}\xspace}
	
% macro for creating a binary operator
%
% example:  \makebinop{\makebinop{\mybmod}{mod}
%	(duplicates the \bmod macro)
%
\def\makebinop#1#2{\def#1{\mskip-\medmuskip \mskip5mu
\mathbin{\rm #2} \penalty900 \mskip5mu \mskip-\medmuskip}}

% Environments for theorems, lemmas, etc.
%\newtheorem{theorem}{Theorem}[section]
%\newtheorem{lemma}[theorem]{Lemma}
%\newtheorem{corollary}[theorem]{Corollary}
%\newtheorem{definition}[theorem]{Definition}

%\newtheorem{observation}[theorem]{Observation}
%\newtheorem{fact}{Fact}[theorem]

%\newtheorem{proposition}[theorem]{Proposition}
%\newtheorem{example}[theorem]{Example}

%\newtheorem{constraint}[theorem]{Constraint}
%\newtheorem{axiom}[theorem]{Axiom}
%\newtheorem{law}[theorem]{Law}
%\newtheorem{algorithm}[theorem]{Algorithm}
%\newtheorem{invariant}[theorem]{Invariant}

% Definition of the proof-environment:
%\newenvironment{proof}{{\bf Proof:}\quad}{$\Box$}

% Definition of an eqnarray-like environment with an extra
% column for comments
\newcommand{\eeqnarray}[1]{\[\begin{array}{rcll} #1 \end{array}\]}

% Definition of a eqnarray-like environment for proofs of the
% form 
%     init     
%=>   step1   reason1
%=>   step2   reason2
%...
\newcommand{\peqnarray}[1]{\[\begin{array}{cll} #1 \end{array}\]}

%
% An environment for formatting programs. 
%
%	written by Hal Perkins
%	adapted by Charles Elkan	9/3/86
%	keyword macros by Anne Neirynck
%
% Usage :
%
% \begin{program}
% program text\\
% program text
% \end{program}
%
% The program environment is a tabbing environment with ten tab stops spaced
% evenly from the left of the page.  Initially the left margin is the second
% tab stop.  Use \+ to indent following lines one more tab stop, \- to undo
% the effect of \+, and \> at the beginning of a line to indent an extra tab.

\newlength{\pgmtab}          %  \pgmtab is the width of each tab in the
\setlength{\pgmtab}{2em}     %  program environment

% boxed program is like program, only boxed!
% This is useful for centering programs and preventing page breaks in programs.
% argument t or b is required.

\newenvironment{boxed-program}[1]{\begin{minipage}[#1]{9in}
\begin{tabbing}\hspace{\pgmtab}\=\hspace{\pgmtab}\=%
\hspace{\pgmtab}\=\hspace{\pgmtab}\=\hspace{\pgmtab}\=\hspace{\pgmtab}\=%
\hspace{\pgmtab}\=\hspace{\pgmtab}\=\hspace{\pgmtab}\=\hspace{\pgmtab}\=%
\hspace{\pgmtab}\=%
\kill}{\end{tabbing}\end{minipage}}

\newenvironment{program}{\begin{tabbing}\hspace{\pgmtab}\=\hspace{\pgmtab}\=%
\hspace{\pgmtab}\=\hspace{\pgmtab}\=\hspace{\pgmtab}\=\hspace{\pgmtab}\=%
\hspace{\pgmtab}\=\hspace{\pgmtab}\=\hspace{\pgmtab}\=\hspace{\pgmtab}\=%
\hspace{\pgmtab}\=%
\+\+\kill}{\end{tabbing}}

% The following commands should be used OUTSIDE math mode

\newcommand {\FUNCTION}{{\bf function\ }}
\newcommand {\BEGIN}{{\bf begin\ }}
\newcommand {\END}{{\bf end}}
\newcommand {\CASE}{{\bf case}}
\newcommand {\OF}{{\bf of}}
\newcommand {\SELECT}{{\bf select\ }}
\newcommand {\WHERE}{{\bf where\ }}
\newcommand {\DECLARE}{{\bf declare\ }}
\newcommand {\ARRAY}{{\bf array\ }}
\newcommand {\LET}{{\bf\ let\ }}
\newcommand {\IN}{{\bf\ in\ }}
\newcommand {\IF}{{\bf if\ }}
\newcommand {\THEN}{{\bf then\ }}
\newcommand {\ELSE}{{\bf else\ }}
\newcommand {\SKIP}{{\bf skip\ }}
\newcommand {\DO}{{\bf do\ }}
% OLD---don't use \OD
\newcommand {\OD}{{\bf od\ }}
\newcommand {\BY}{{\bf by\ }}
\newcommand {\LOOP}{{\bf loop\ }}
\newcommand {\WHILE}{{\bf while\ }}
\newcommand {\TO}{{\bf to\ }}
\newcommand {\DOWNTO}{{\bf down to\ }}
\newcommand {\FOR}{{\bf for\ }}
\newcommand {\FOREACH}{{\bf for each\ }}
\newcommand {\RETURN}{{\bf return\ }}
\newcommand {\REPEAT}{{\bf repeat\ }}
\newcommand {\UNTIL}{{\bf until\ }}
\newcommand {\LCOM}{$(\ast \;$}
\newcommand {\RCOM}{$\ast)$}
\newcommand {\GOTO}{{\bf goto\ }}

% The following commands are for use INSIDE math mode

\newcommand {\OP}[1]{\mbox{\sc #1}}
\newcommand {\op}[1]{\mbox{\sc #1}}
\newcommand {\mm}[1]{\mbox{\rm #1}\;}
\newcommand {\id}[1]{\mbox{\it #1\ }}

\newcommand {\ASSIGN}{\leftarrow }
\newcommand {\MIN}{\OP{min} }
\newcommand {\MOD}{\; {\bf{\rm mod}} \; } 
\newcommand {\LAMBDA}{{\bf \lambda\ }}
\newcommand {\FALSE}{{\em FALSE\ }}

% Miscellaneous notation

%\newcommand {\And}{\wedge}
\newcommand {\Or}{\vee}

\newcommand {\thus}{{\dot{. \: .}\;}}

\newcommand {\bigO}[1]{{\cal O}(#1)}

\newcommand {\app}{\!\!:\!}
\newcommand {\hastype}{::}
\newcommand{\hast}{:}
\newcommand{\qt}[1]{\mbox{``#1''}}
\newcommand{\TexComment}[1]{}

\newcommand{\dq}{\m{\tt "}}
\newcommand{\flatqt}[1]{\m{\dq #1 \dq}}
\newcommand{\seq}{\subseteq}
\newcommand{\derives}{\vdash}

% for writing grammars
\newcommand{\grammar}{::=}
\newcommand{\gor}{\,|\,}

% macros for writing inference rules
\newcommand{\infrule}[2]{\displaystyle{\displaystyle\strut{#1}} \over %
                                        {\displaystyle\strut {#2}}}
\newcommand{\cinfrule}[3]{\parbox{14cm}{\hfil$\infrule{#1}{#2}$\hfil}\parbox{4cm}{$\,#3$\hfil}}

% ASE'20 paper related
\newcommand{\SO}{\emph{SO}\xspace}
% [[[Should ML really be in italics? -- Gary]]]
\newcommand{\ML}{ML\xspace}
\newcommand{\tf}{\emph{TensorFlow}\xspace}
\newcommand{\scikit}{\emph{Scikit-learn}\xspace}
\newcommand{\keras}{\emph{Keras}\xspace}
\newcommand{\torch}{\emph{PyTorch}\xspace}
\newcommand{\etal}{\emph{et al.}\xspace}
\newcommandx{\improve}[2][1=]{\todo[linecolor=red,backgroundcolor=red!25,bordercolor=red,#1]{#2}}

%color
\definecolor{Gray}{gray}{0.85}
\definecolor{deepblue}{rgb}{0,0,0.5}
\definecolor{deepred}{rgb}{0.6,0,0}
\definecolor{deepgreen}{rgb}{0,0.5,0}
\definecolor{lightorange}{rgb}{1, 0.7804, 0.0078}
\definecolor{lightgreen}{rgb}{.1961,.7961,0}
\definecolor{lightblue}{rgb}{.4078,.7961,.8157}

% listing ML codes
\newsavebox{\mybox}
\definecolor{halfgray}{gray}{0.55}
\definecolor{ipython_frame}{RGB}{207, 207, 207}
\definecolor{ipython_bg}{RGB}{247, 247, 247}
\definecolor{ipython_red}{RGB}{186, 33, 33}
\definecolor{ipython_green}{RGB}{0, 128, 0}
\definecolor{ipython_cyan}{RGB}{64, 128, 128}
\definecolor{ipython_purple}{RGB}{170, 34, 255}
\lstset{
	breaklines=true,
	%
	extendedchars=true,
	literate=
	{á}{{\'a}}1 {é}{{\'e}}1 {í}{{\'i}}1 {ó}{{\'o}}1 {ú}{{\'u}}1
	{Á}{{\'A}}1 {É}{{\'E}}1 {Í}{{\'I}}1 {Ó}{{\'O}}1 {Ú}{{\'U}}1
	{à}{{\`a}}1 {è}{{\`e}}1 {ì}{{\`i}}1 {ò}{{\`o}}1 {ù}{{\`u}}1
	{À}{{\`A}}1 {È}{{\'E}}1 {Ì}{{\`I}}1 {Ò}{{\`O}}1 {Ù}{{\`U}}1
	{ä}{{\"a}}1 {ë}{{\"e}}1 {ï}{{\"i}}1 {ö}{{\"o}}1 {ü}{{\"u}}1
	{Ä}{{\"A}}1 {Ë}{{\"E}}1 {Ï}{{\"I}}1 {Ö}{{\"O}}1 {Ü}{{\"U}}1
	{â}{{\^a}}1 {ê}{{\^e}}1 {î}{{\^i}}1 {ô}{{\^o}}1 {û}{{\^u}}1
	{Â}{{\^A}}1 {Ê}{{\^E}}1 {Î}{{\^I}}1 {Ô}{{\^O}}1 {Û}{{\^U}}1
	{œ}{{\oe}}1 {Œ}{{\OE}}1 {æ}{{\ae}}1 {Æ}{{\AE}}1 {ß}{{\ss}}1
	{ç}{{\c c}}1 {Ç}{{\c C}}1 {ø}{{\o}}1 {å}{{\r a}}1 {Å}{{\r A}}1
	{€}{{\EUR}}1 {£}{{\pounds}}1
}

%%
%% Python definition (c) 1998 Michael Weber
%% Additional definitions (2013) Alexis Dimitriadis
%% modified by me (should not have empty lines)
%%
\lstdefinelanguage{iPython}{
	morekeywords={access,and,break,class,continue,def,del,elif,else,except,exec,finally,for,from,global,if,import,in,is,lambda,not,or,pass,print,raise,return,try,while,KerasRegressor,DataLoader,Dense},%
	%
	% Built-ins
	morekeywords=[2]{abs,all,any,basestring,bin,bool,bytearray,callable,chr,classmethod,cmp,compile,complex,delattr,dict,dir,divmod,enumerate,eval,execfile,file,filter,float,format,frozenset,getattr,globals,hasattr,hash,help,hex,id,input,int,isinstance,issubclass,iter,len,list,locals,long,map,max,memoryview,min,next,object,oct,open,ord,pow,property,range,raw_input,reduce,reload,repr,reversed,round,set,setattr,slice,sorted,staticmethod,str,sum,super,tuple,type,unichr,unicode,vars,xrange,zip,apply,buffer,coerce,intern},%
	%
	sensitive=true,%
	morecomment=[l]\#,%
	morestring=[b]',%
	morestring=[b]",%
	%
	morestring=[s]{'''}{'''},% used for documentation text (mulitiline strings)
	morestring=[s]{"""}{"""},% added by Philipp Matthias Hahn
	%
	morestring=[s]{r'}{'},% `raw' strings
	morestring=[s]{r"}{"},%
	morestring=[s]{r'''}{'''},%
	morestring=[s]{r"""}{"""},%
	morestring=[s]{u'}{'},% unicode strings
	morestring=[s]{u"}{"},%
	morestring=[s]{u'''}{'''},%
	morestring=[s]{u"""}{"""},%
	%
	% {replace}{replacement}{lenght of replace}
	% *{-}{-}{1} will not replace in comments and so on
	literate=
	*{+}{{{\color{ipython_purple}+}}}1
	{-}{{{\color{ipython_purple}-}}}1
	{*}{{{\color{ipython_purple}$^\ast$}}}1
	{/}{{{\color{ipython_purple}/}}}1
	{^}{{{\color{ipython_purple}\^{}}}}1
	{?}{{{\color{ipython_purple}?}}}1
	{!}{{{\color{ipython_purple}!}}}1
	{\%}{{{\color{ipython_purple}\%}}}1
	{<}{{{\color{ipython_purple}<}}}1
	{>}{{{\color{ipython_purple}>}}}1
	{|}{{{\color{ipython_purple}|}}}1
	{\&}{{{\color{ipython_purple}\&}}}1
	{~}{{{\color{ipython_purple}~}}}1
	%
	{==}{{{\color{ipython_purple}==}}}2
	{<=}{{{\color{ipython_purple}<=}}}2
	{>=}{{{\color{ipython_purple}>=}}}2
	%
	{+=}{{{+=}}}2
	{-=}{{{-=}}}2
	{*=}{{{$^\ast$=}}}2
	{/=}{{{/=}}}2,
	%
	literate=
	{á}{{\'a}}1 {é}{{\'e}}1 {í}{{\'i}}1 {ó}{{\'o}}1 {ú}{{\'u}}1
	{Á}{{\'A}}1 {É}{{\'E}}1 {Í}{{\'I}}1 {Ó}{{\'O}}1 {Ú}{{\'U}}1
	{à}{{\`a}}1 {è}{{\`e}}1 {ì}{{\`i}}1 {ò}{{\`o}}1 {ù}{{\`u}}1
	{À}{{\`A}}1 {È}{{\'E}}1 {Ì}{{\`I}}1 {Ò}{{\`O}}1 {Ù}{{\`U}}1
	{ä}{{\"a}}1 {ë}{{\"e}}1 {ï}{{\"i}}1 {ö}{{\"o}}1 {ü}{{\"u}}1
	{Ä}{{\"A}}1 {Ë}{{\"E}}1 {Ï}{{\"I}}1 {Ö}{{\"O}}1 {Ü}{{\"U}}1
	{â}{{\^a}}1 {ê}{{\^e}}1 {î}{{\^i}}1 {ô}{{\^o}}1 {û}{{\^u}}1
	{Â}{{\^A}}1 {Ê}{{\^E}}1 {Î}{{\^I}}1 {Ô}{{\^O}}1 {Û}{{\^U}}1
	{œ}{{\oe}}1 {Œ}{{\OE}}1 {æ}{{\ae}}1 {Æ}{{\AE}}1 {ß}{{\ss}}1
	{ç}{{\c c}}1 {Ç}{{\c C}}1 {ø}{{\o}}1 {å}{{\r a}}1 {Å}{{\r A}}1
	{€}{{\EUR}}1 {£}{{\pounds}}1,
	%
	%   identifierstyle=\color{red}\ttfamily,
	commentstyle=\color{ipython_cyan}\ttfamily,
	stringstyle=\color{ipython_red}\ttfamily,
	keepspaces=true,
	showspaces=false,
	showstringspaces=false,
	%
	rulecolor=\color{black},
	frame=single,
	frameround={t}{t}{t}{t},
	framexleftmargin=3.2mm,
	numbers=left,
	numberstyle=\tiny\color{halfgray},
	%
	%
	backgroundcolor=\color{ipython_bg},
	%   extendedchars=true,
	basicstyle=\scriptsize\ttfamily,
	keywordstyle=\color{ipython_green}\ttfamily,
}


%% print error message
\lstdefinelanguage{iPython2}{
	morekeywords={access,and,break,class,continue,def,del,elif,else,except,exec,finally,for,from,global,if,import,in,is,lambda,not,or,pass,print,raise,return,try,while,IndexError,Traceback,Sequential},%
	%
	% Built-ins
	morekeywords=[2]{abs,all,any,basestring,bin,bool,bytearray,callable,chr,classmethod,cmp,compile,complex,delattr,dict,dir,divmod,enumerate,eval,execfile,file,filter,float,format,frozenset,getattr,globals,hasattr,hash,help,hex,id,int,isinstance,issubclass,iter,len,locals,long,map,max,memoryview,min,next,object,oct,open,ord,pow,property,raw_input,reduce,reload,repr,reversed,round,set,setattr,slice,sorted,staticmethod,str,sum,super,tuple,type,unichr,unicode,vars,xrange,zip,apply,buffer,coerce,intern},%
	%
	sensitive=true,%
	morecomment=[l]\#,%
	morestring=[b]',%
	morestring=[b]",%
	%
	morestring=[s]{'''}{'''},% used for documentation text (mulitiline strings)
	morestring=[s]{"""}{"""},% added by Philipp Matthias Hahn
	%
	morestring=[s]{r'}{'},% `raw' strings
	morestring=[s]{r"}{"},%
	morestring=[s]{r'''}{'''},%
	morestring=[s]{r"""}{"""},%
	morestring=[s]{u'}{'},% unicode strings
	morestring=[s]{u"}{"},%
	morestring=[s]{u'''}{'''},%
	morestring=[s]{u"""}{"""},%
	%
	% {replace}{replacement}{lenght of replace}
	% *{-}{-}{1} will not replace in comments and so on
	literate=
	*{+}{{{\color{ipython_purple}+}}}1
	{-}{{{\color{ipython_purple}-}}}1
	{*}{{{\color{ipython_purple}$^\ast$}}}1
	{/}{{{\color{ipython_purple}/}}}1
	{^}{{{\color{ipython_purple}\^{}}}}1
	{?}{{{\color{ipython_purple}?}}}1
	{!}{{{\color{ipython_purple}!}}}1
	{\%}{{{\color{ipython_purple}\%}}}1
	{<}{{{\color{ipython_purple}<}}}1
	{>}{{{\color{ipython_purple}>}}}1
	{|}{{{\color{ipython_purple}|}}}1
	{\&}{{{\color{ipython_purple}\&}}}1
	{~}{{{\color{ipython_purple}~}}}1
	%
	{==}{{{\color{ipython_purple}==}}}2
	{<=}{{{\color{ipython_purple}<=}}}2
	{>=}{{{\color{ipython_purple}>=}}}2
	%
	{+=}{{{+=}}}2
	{-=}{{{-=}}}2
	{*=}{{{$^\ast$=}}}2
	{/=}{{{/=}}}2,
	%
	literate=
	{á}{{\'a}}1 {é}{{\'e}}1 {í}{{\'i}}1 {ó}{{\'o}}1 {ú}{{\'u}}1
	{Á}{{\'A}}1 {É}{{\'E}}1 {Í}{{\'I}}1 {Ó}{{\'O}}1 {Ú}{{\'U}}1
	{à}{{\`a}}1 {è}{{\`e}}1 {ì}{{\`i}}1 {ò}{{\`o}}1 {ù}{{\`u}}1
	{À}{{\`A}}1 {È}{{\'E}}1 {Ì}{{\`I}}1 {Ò}{{\`O}}1 {Ù}{{\`U}}1
	{ä}{{\"a}}1 {ë}{{\"e}}1 {ï}{{\"i}}1 {ö}{{\"o}}1 {ü}{{\"u}}1
	{Ä}{{\"A}}1 {Ë}{{\"E}}1 {Ï}{{\"I}}1 {Ö}{{\"O}}1 {Ü}{{\"U}}1
	{â}{{\^a}}1 {ê}{{\^e}}1 {î}{{\^i}}1 {ô}{{\^o}}1 {û}{{\^u}}1
	{Â}{{\^A}}1 {Ê}{{\^E}}1 {Î}{{\^I}}1 {Ô}{{\^O}}1 {Û}{{\^U}}1
	{œ}{{\oe}}1 {Œ}{{\OE}}1 {æ}{{\ae}}1 {Æ}{{\AE}}1 {ß}{{\ss}}1
	{ç}{{\c c}}1 {Ç}{{\c C}}1 {ø}{{\o}}1 {å}{{\r a}}1 {Å}{{\r A}}1
	{€}{{\EUR}}1 {£}{{\pounds}}1,
	%
	%   identifierstyle=\color{red}\ttfamily,
	commentstyle=\color{ipython_cyan}\ttfamily,
	stringstyle=\color{ipython_red}\ttfamily,
	keepspaces=true,
	showspaces=false,
	showstringspaces=false,
	%
	rulecolor=\color{black},
	frame=single,
	frameround={t}{t}{t}{t},
	framexleftmargin=1.2mm,
	framexrightmargin=2.2mm,
	numbers=none,
	%numberstyle=\tiny\color{halfgray},
	%
	%
	backgroundcolor=\color{ipython_bg},
	%   extendedchars=true,
	basicstyle=\scriptsize\ttfamily,
	keywordstyle=\color{ipython_green}\ttfamily,
}
% end listing ML codes

%edit tcolorbox
% new tcolorbox environment
% #1: tcolorbox options
% #2: color
% #3: box title
\newtcolorbox{mytbox}[3][]
{
	colframe = #2!25,
	colback  = #2!12,
	coltitle = #2!8!black,  
	title    = {#3},
	#1,
}

\newtcolorbox{mytbox2}[2][]
{
	colframe = green!5!white,
	colback  = green!75!black,
	fonttitle=\scriptsize\bfseries,
	colbacktitle = green!85!black,
	title    = {#2},
	#1,
}

\newcolumntype{g}{>{\columncolor{Gray}}l}

\definecolor{codegreen}{rgb}{0,0.6,0}
\definecolor{codegray}{rgb}{0.5,0.5,0.5}
\definecolor{codepurple}{rgb}{0.58,0,0.82}
\definecolor{backcolour}{rgb}{0.95,0.95,0.92}

\def\checkmark{\tikz\fill[scale=0.4](0,.35) -- (.25,0) -- (1,.7) -- (.25,.15) -- cycle;} 

%\lstdefinestyle{mystyle}{
%	backgroundcolor=\color{backcolour},   
%	commentstyle=\color{codegreen},
%	keywordstyle=\color{magenta},
%	numberstyle=\tiny\color{codegray},
%	stringstyle=\color{codepurple},
%	basicstyle=\scriptsize,
%	breakatwhitespace=false,         
%	breaklines=true,                 
%	captionpos=b,                    
%	keepspaces=true,                 
%	numbers=left,                    
%	numbersep=5pt,                  
%	showspaces=false,                
%	showstringspaces=false,
%	showtabs=false,                  
%	tabsize=2
%}
%\lstset{style=mystyle}

\newcounter{NumObservations}
\stepcounter{NumObservations}
\newcommand{\finding}[1]{
	\begin{bclogo}[couleur= orange!5, epBord= 0.8, arrondi=0.2, logo=\bccrayon,marge= 1.2, ombre=true, blur, couleurBord=brown!90, tailleOndu=3, sousTitre ={\em #1}]{~Finding \arabic{NumObservations} $:$ } 
		
	\end{bclogo}
	\stepcounter{NumObservations}
}


\setlength{\fboxsep}{1pt}

