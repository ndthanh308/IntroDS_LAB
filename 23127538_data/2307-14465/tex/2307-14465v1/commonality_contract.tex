\subsection{Common Patterns for  Contracts}

This section highlights common patterns of contracts in the dataset, i.e., we analyze the common patterns of ML contract violations observed in our study.
In section \S{\ref{subsec:ctvs}}, we noted that IC-1, F, MT are the most frequently occurring patterns across libraries. 
These contracts are \emph{atomic} in the sense that there is no dependency between behavioral and temporal contracts in these. We further investigated more complex contract patterns, including combinations of two or more atomic contracts, when answering \RQ{2}. These types of patterns belong to the high-level category \emph{hybrid} in our classification schema. Recall that hybrid contracts contain combinations, choices, or dependencies between the behavioral and temporal contracts. 

\input{level3pattern}

\finding{Eventually (F) related hybrid contracts are one of the most common patterns across ML libraries.}

\noindent
{\textbf{Patterns involving <F>.}} Our result shows that F contracts (about the method orderings at a certain point in history) are the places \ML API users struggle most, compared to G (always orderings); see Figure~\ref{fig:level3pattern}. For instance, we described that in \SO~post \ref{fig:post4}, the parameter choice for the \texttt{GridsearchCV()} API dictates whether it must be preceded by \texttt{preprocessing.scale()} API. In contrast, it is not always obvious for patterns involving F, e.g., \ML API users sometimes use a pooling layer after a convolution layer to downsample the feature collected in the previous layer. Thus, this order is not mandatory for all program points. But, if the order is used\footnote{https://stackoverflow.com/questions/34092850/}, then the API user should make sure the parameter \texttt{strides} of \texttt{tf.nn.conv2d()} is compatible with the  \texttt{ksize} and \texttt{strides} parameters of the pooling layer (e.g., \texttt{tf.nn.max\_pool()}) in \tf.~Violations of such hybrid patterns can be found using unit tests that capture variants of these patterns and testers should be aware of capturing these variants.

