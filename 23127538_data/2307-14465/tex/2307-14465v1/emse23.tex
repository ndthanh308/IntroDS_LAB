%%%%%%%%%%%%%%%%%%%%%%% file template.tex %%%%%%%%%%%%%%%%%%%%%%%%%
%
% This is a general template file for the LaTeX package SVJour3
% for Springer journals.          Springer Heidelberg 2010/09/16
%
% Copy it to a new file with a new name and use it as the basis
% for your article. Delete % signs as needed.
%
% This template includes a few options for different layouts and
% content for various journals. Please consult a previous issue of
% your journal as needed.
%
%%%%%%%%%%%%%%%%%%%%%%%%%%%%%%%%%%%%%%%%%%%%%%%%%%%%%%%%%%%%%%%%%%%
%
% First comes an example EPS file -- just ignore it and
% proceed on the \documentclass line
% your LaTeX will extract the file if required
% \begin{filecontents*}{example.eps}
% %!PS-Adobe-3.0 EPSF-3.0
% %%BoundingBox: 19 19 221 221
% %%CreationDate: Mon Sep 29 1997
% %%Creator: programmed by hand (JK)
% %%EndComments
% % gsave
% % newpath
% %   20 20 moveto
% %   20 220 lineto
% %   220 220 lineto
% %   220 20 lineto
% % closepath
% % 2 setlinewidth
% % gsave
% %   .4 setgray fill
% % grestore
% % stroke
% % grestore
% \end{filecontents*}
%
\RequirePackage{fix-cm}
%
%\documentclass{svjour3}                     % onecolumn (standard format)
%\documentclass[smallcondensed]{svjour3}     % onecolumn (ditto)
\documentclass[smallextended]{svjour3}       % onecolumn (second format)
%\documentclass[twocolumn]{svjour3}          % twocolumn
%
\smartqed  % flush right qed marks, e.g. at end of proof
%
\usepackage{graphicx}
%
% \usepackage{mathptmx}      % use Times fonts if available on your TeX system
%
% insert here the call for the packages your document requires
%\usepackage{latexsym}
% etc.
%
% please place your own definitions here and don't use \def but
% \newcommand{}{}
%
% Insert the name of "your journal" with
\journalname{Empirical Software Engineering}
%\usepackage[numbers]{natbib}
\usepackage[authoryear]{natbib}
\usepackage{multirow}
\usepackage{array}
\usepackage{enumitem}
\newcolumntype{C}[1]{>{\centering\arraybackslash}p{#1}}
\usepackage{colortbl}
%\usepackage{xcolor}
\usepackage{hyperref}% http://ctan.org/pkg/hyperref
\renewcommand{\sectionautorefname}{\S}
\usepackage{float} % force figure location
\usepackage{graphicx}
\usepackage{adjustbox}
\usepackage[font=small,labelfont=bf,tableposition=top]{caption}
\usepackage{newfloat}
\usepackage{balance}
% declare a new float for SO posts
\DeclareCaptionType{Stack Overflow post}
\usepackage[T1]{fontenc} 
\usepackage[utf8]{inputenc}
\usepackage{subfig}
\usepackage[tikz]{bclogo}
\usepackage{diagbox}
\usepackage{listings}
\usepackage{xargs}    
%\usepackage[usenames,dvipsnames]{xcolor}
%\usepackage[colorinlistoftodos,prependcaption,textsize=tiny]{todonotes}
\usepackage{xspace}
\usepackage{wrapfig}
\usepackage{tcolorbox} % text box
\usepackage{hhline}
\usepackage{tikz}
\usepackage{scalerel}

\newcommand{\tikzdiamond}[1][red,fill=red]{\scalerel*{\tikz \draw[rounded corners=0.06pt,#1] (-3pt,0)--++(45:3pt)--++(-45:3pt)--++(-180+45:3pt)--cycle;}{\diamond}}
\newcommand{\tikztriangleright}[1][red,fill=red]{\scalerel*{\tikz \draw[rounded corners=0.1pt,#1] (0,-2.5pt)--++(0,5pt)--++(-30:5pt)--cycle;}{\triangleright}}

\newcommand{\tikztriangleleft}[1][red,fill=red]{\scalerel*{\tikz \draw[rounded corners=0.1pt,#1] (0,-2.5pt)--++(0,5pt)--++(-180+30:5pt)--cycle;}{\triangleleft}}
% \usepackage{siunitx}
% \usetikzlibrary{shapes.geometric,shapes.symbols}
% \newcommand{\tikzcircle}[2][red,fill=red]{\tikz[baseline=-0.5ex]\draw[#1,radius=#2] (0,0) circle ;}%

\newcommand{\tikzcircle}[2][red,fill=red]{\tikz[baseline=-0.5ex]\draw[#1,radius=#2] (0,0) circle ;}%


% \newcommand{\tikzsymbol}[2][circle]{\tikz[baseline=-0.5ex]\node[inner
% sep=2pt,shape=#1,draw,#2]{};}%


\usepackage{amsmath,amssymb,amsfonts}
%\usepackage{algorithmic}
%\usepackage{graphicx}
%\usepackage{textcomp}
%\usepackage{xcolor}
%\def\BibTeX{{\rm B\kern-.05em{\sc i\kern-.025em b}\kern-.08em
%    T\kern-.1667em\lower.7ex\hbox{E}\kern-.125emX}}
\AtBeginDocument{%
	\providecommand\BibTeX{{%
			\normalfont B\kern-0.5em{\scshape i\kern-0.25em b}\kern-0.8em\TeX}}}

%\renewcommand{\baselinestretch}{0.97}
	

%
\begin{document}
\newcommand\calF{\mathcal{F}}
\newcommand\calG{\mathcal{G}}
\newcommand\calM{\mathcal{M}}
\newcommand\calV{\mathcal{V}}
\newcommand\calU{\mathcal{U}}
\newcommand\calW{\mathcal{W}}
\newcommand\calP{\mathcal{P}}
\newcommand\calD{\mathbb{D}}
%%%%%%%%%%%%%%%%%
%% macros introduced by Luke 
\newcommand\mydef[1]{{\bf\em #1}}
%%%%%%%%%%%%%%%%%

\newcommand{\numviparams}{{| \lambda |}}
\newcommand{\scoreaccvars}[1]{s_1^{#1}, \ldots, s_{\numviparams}^{#1}}
\newcommand{\scoreaccvar}[2]{s_{#1}^{#2}}
\newcommand{\isdeterm}[1]{\text{Deterministic}({#1})}


\newcommand{\expect}[1]{\mathbb{E}\left[{#1}\right]}
\newcommand{\var}[1]{\mathbb{V}\left[ {#1} \right]}
\newcommand{\expectdist}[2]{\mathbb{E}_{#1}\left[ {#2} \right]}
\newcommand{\vardist}[2]{\mathbb{V}_{#1}\left[ {#2} \right]}
\newcommand{\cov}[2]{\mathbb{C}\text{ov}[{#1}][{#2}]}
\newcommand{\covv}[1]{\mathbb{C}\text{ov}[{#1}]}
\newcommand{\corr}[1]{\mathbb{C}\text{orr}[{#1}]}

\newcommand{\fix}[1]{\mathit{fix}\left({#1}\right)}
\newcommand{\sbr}[1]{\left\llbracket {#1} \right\rrbracket}
\newcommand{\ctxtype}[3]{{#1} \cong_\text{ctx} {#2} : {#3}}
\newcommand{\bigstep}[3]{{#1} \Downarrow_{#2} {#3}}


% PCF types
\newcommand{\bool}{\mathit{bool}}
\newcommand{\nat}{\mathit{nat}}

\newcommand{\ctx}[1]{\mathcal{C}\left[ {#1}\right] }
\newcommand{\pcft}[1]{\text{PCF}_{#1}}

\newcommand{\nfl}{\mathbb{N}_\bot}
\newcommand{\bfl}{\mathbb{B}_\bot}

% PCF constructs
\newcommand{\succc}[1]{\mathbf{succ}({#1})}
\newcommand{\succcn}[2]{\mathbf{succ}^{#1}({#2})}
\newcommand{\zero}{\mathbf{0}}
\newcommand{\zerotest}[1]{\mathbf{zero}\left({#1}\right)}
\newcommand{\pred}[1]{\mathbf{pred}\left( {#1} \right)}
\newcommand{\predn}[2]{\mathbf{pred}^{#1}\left( {#2} \right)}
\def\solvable{\#}

\newcommand{\true}{\mathbf{true}}
\newcommand{\false}{\mathbf{false}}
\newcommand{\pcffix}[1]{\mathbf{fix}\left({#1}\right)}
\newcommand{\pcffn}[3]{\mathbf{fn}~{#1}:{#2}\mathpunct{.}{#3}}
\newcommand{\pairtype}[2]{{#1} * {#2}}
\newcommand{\pairexp}[2]{\mathbf{pair}({#1}, {#2})}
\newcommand{\leftexp}[1]{\mathbf{left}({#1})}
\newcommand{\rightexp}[1]{\mathbf{right}({#1})}

\newcommand{\RationalPos}{\mathbb{Q}^{+}}

\newcommand{\meas}[1]{\mathbb{M}\left( {#1} \right) }
\newcommand{\integ}[1]{\sbr{#1}_I}

\newcommand{\notbigstep}[2]{{#1}~\cancel{\Downarrow}_{#2}}
\newcommand{\subtrace}[3]{{#1}^{{#2} \ldots {#3}}}
\newcommand{\supp}[1]{\textsf{supp}\left({#1}\right)}
\newcommand{\dom}[1]{\textsf{Dom}\left({#1}\right)}
\newcommand{\suppk}[2]{\textsf{Supp}^{#1}\left({#2}\right)}
\newcommand{\tracespace}{\bigcup_{n \in \mathbb{N}}[0, 1]^n}
\newcommand{\generictracespace}{\mathbb{T}}
\newcommand{\nnreals}{\mathbb{R}_{\geq 0}}
\newcommand{\posreals}{\mathbb{R}_{> 0}}
\newcommand{\reals}{\mathbb{R}}

\newcommand{\unrollkM}[2]{\textsf{unroll}_{#1}\left({#2}\right)}
\newcommand{\nphmcint}[5]{\Psi_\textsf{NP}\left({#1}, {#2}, {#3}, {#4}, {#5}\right)}

%SPCF constructs
\newcommand{\spcfvalues}{\Lambda^0_v}

\newcommand{\prevalueM}[1]{\textsf{value}^{-1}_{#1}(\spcfvalues{})}
\newcommand{\num}[1]{\underline{#1}}

% \theoremstyle{definition}
% \newtheorem{thm}{Theorem}
% \newtheorem{lem}{Lemma}
% \newtheorem{defn}{Definition}
% \newtheorem{conj}{Conjecture}
% \newtheorem{prop}{Proposition}

%\theoremstyle{definition}
%\newtheorem{defn}{Definition}[section]
%\newtheorem{example}[defn]{Example}
%
%
%\theoremstyle{plain}
%\newtheorem{thm}{Theorem}[section]
%\newtheorem{lem}[thm]{Lemma}
%\newtheorem{cor}[thm]{Corollary}
%\newtheorem{conj}[thm]{Conjecture}
%\newtheorem{prop}[thm]{Proposition}
%\newtheorem{remark}[thm]{Remark}

%% Proofs
%\let\oldproof\proof
%\renewcommand{\proof}{\color{blue}\oldproof}


\definecolor{codegreen}{rgb}{0,0.6,0}
\definecolor{codegray}{rgb}{0.5,0.5,0.5}
\definecolor{codepurple}{rgb}{0.58,0,0.82}
\definecolor{backcolour}{rgb}{0.95,0.95,0.92}

\lstdefinestyle{myStyle}{
    belowcaptionskip=1\baselineskip,
    breaklines=true,
    frame=none,
    basicstyle=\footnotesize\ttfamily,
    keywordstyle=\bfseries\color{green!40!black},
    commentstyle=\itshape\color{purple!40!black},
    identifierstyle=\color{blue},
    backgroundcolor=\color{gray!10!white},
    %backgroundcolor=\color{backcolour}, 
    numberstyle=\tiny\color{codegray},
    stringstyle=\color{codepurple},
    breakatwhitespace=false,                          
    keepspaces=true,                 
    numbers=left,       
    numbersep=5pt,                  
    showspaces=false,                
    showstringspaces=false,
    showtabs=false,                  
    tabsize=2,
}

% argmin/argmax
\DeclareMathOperator*{\argmax}{arg\,max}
\DeclareMathOperator*{\argmin}{arg\,min}

% Concatenation of lists
\newcommand\doubleplus{+\kern-1.3ex+\kern0.8ex}

% Program configurations
\newcommand{\tuple}[1]{\ensuremath{\langle #1 \rangle}}
% Rule based definitions
\newcommand{\Rule}[4][]{\ensuremath{\inferrule*[lab={\hypertarget{#2}{(\TirName{#2})}},#1]{#3}{#4}}}

% Calligraphic symbols
\newcommand{\calI}{{\mathcal I}} 
\newcommand{\calT}{{\mathcal T}}

%  Macro for new Y operator.
\newcommand{\yBounded}[3]{\mu^{#1}_{#2}\rvert_{#3}}

%%%%%%%%%%%%%%%%%
 
%%%%%%%%%%%%%%%%%

\newcommand{\expv}{\mathbb{E}}

\newcommand{\combTr}[2]{\left[\begin{matrix}
		#1\\
		#2
	\end{matrix} \right]}

\newcommand{\exType}[2]{\left\{\begin{matrix}
		#1\\
		#2
	\end{matrix} \right\}}
\newcommand{\myint}[1]{ [#1]}
\newcommand{\Uniform}{\ensuremath{\mathrm{Uniform}}}
\newcommand{\Normal}{\ensuremath{\mathrm{normal}}}
\DeclareMathOperator{\abs}{abs}
\DeclareMathOperator{\pdf}{pdf}

\newcommand{\intConf}[1]{\lceil#1\rceil}
\newcommand{\tr}{\boldsymbol{t}}

\newcommand{\sample}{\tt{sample}}
%\newcommand{\fix}{\texttt{fix}}
%\newcommand{\num}[1]{\underline{#1}}
\newcommand{\myif}{\texttt{if}}
\newcommand{\mylet}{\texttt{let} \, }
\newcommand{\myin}{\, \texttt{in} \,}
\newcommand{\mythen}{\, \texttt{then} \,}
\newcommand{\myelse}{\, \texttt{else} \,}
\newcommand{\score}{\tt{score}}
\newcommand{\tick}{\tt{tick}}

\newcommand{\term}{\tt{term}}
\newcommand{\pv}{\mathbf{v}}
\newcommand{\rv}{\mathbf{r}}

\newcommand{\interval}{\mathfrak{I}}

\newcommand{\typeReal}{\textbf{\textsf{R}}}

\newcommand{\symbolInt}{\myint{\cdot}}

\newcommand{\LambdaInterval}{\Lambda_{\interval}}
\newcommand{\LambdaSymbolic}{\Lambda_{\text{sym}}}

\newcommand{\toIntervalTerm}[1]{#1^{2\interval}}

%Others
\newcommand{\Sset}{\mathbb{S}}
\newcommand{\Iset}{\mathbb{I}}
\newcommand{\Rset}{\mathbb{R}}
\newcommand{\Nset}{\mathbb{N}}
\newcommand{\Zset}{\mathbb{Z}}

\newcommand{\Term}{\mathbb{T}}
\newcommand{\prob}{\mathbb{P}}
\newcommand{\expt}{\mathbb{E}}


\newcommand{\Leb}{\tt{Leb}}
\newcommand{\Red}{\tt{Red}}
\newcommand{\cost}{\text{cost}}

%\newcommand{\intervalab}[2]{\underline{[#1,#2]}}
\newcommand{\intervalab}{\underline{[a,b]}}
\newcommand{\interI}{\mathcal{I}}
\newcommand{\trans}{\mathcal{T}}

\newcommand{\iv}{\mathbb{I}}

% Programming language constructs
\newcommand{\lit}[1]{\underline{#1}}
\newcommand{\letIn}[1]{\mathsf{let}\,{#1}\,\mathsf{in}\,}
\newcommand{\fixLam}[2]{\mu {#1} {#2}.}
\newcommand{\ifElse}[3]{\mathsf{if} (#1 \le \num{0}) \, {#2} \,\mathsf{else}\, {#3}}

%%Basic notions
\newcommand{\pspace}{(\Omega,\mathcal{F},\probm)}
\newcommand{\probm}{\mathbb{P}}
\newcommand{\condexpv}[2]{{\expt}{\left[{#1} \mid {#2}\right]}}

\newcommand{\stdConf}[1]{(#1)}
%\newcommand{\intConf}[1]{\lceil#1\rceil}
%\newcommand{\intConf}[1]{(#1)}
%\newcommand{\symConf}[1]{\langle\!\langle  #1 \rangle\!\rangle}
%\newcommand\symPath[1]{(#1)}
\newcommand{\symPath}[1]{\langle\!\langle  #1 \rangle\!\rangle}
\newcommand\symConf[1]{(#1)}

\newcommand{\ifSimple}[3]{\mathsf{if}(#1, #2, #3)}
%\newcommand{\ifElse}[3]{\mathsf{if} (#1 \le 0) \, \allowbreak {#2} \, \allowbreak \mathsf{else}\, {#3}}
%\newcommand{\ifElse}[3]{\ifSimple{#1}{#2}{#3}}

%\newcommand{\trace}{\mathsf{s}}
%
%\newcommand\defn[1]{{\bf \em #1}}
\newcommand{\traces}{\mathbb{T}}
%
%\newcommand{\stdConf}[1]{(#1)}
%%\newcommand{\intConf}[1]{\lceil#1\rceil}
%\newcommand{\intConf}[1]{(#1)}
%%\newcommand{\symConf}[1]{\langle\!\langle  #1 \rangle\!\rangle}
%%\newcommand\symPath[1]{(#1)}
%\newcommand{\symPath}[1]{\langle\!\langle  #1 \rangle\!\rangle}
%\newcommand\symConf[1]{(#1)}

\newcommand{\valueSem}[1]{\mathsf{val}_{#1}} % value (semantics)
\newcommand{\weightSem}[1]{\mathsf{wt}_{#1}} % weight (semantics)
\newcommand{\measureSem}[1]{\llbracket #1 \rrbracket}
\newcommand{\posterior}{\mathsf{posterior}}


%%%%%%%%%
% 
%%%%%%%%
\newcommand{\loc}{\ell}
\newcommand{\locs}{\mathit{L}}
\newcommand{\blocs}{\mathit{L}_{\mathrm{b}}}

\newcommand{\iflocs}{\mathit{L}_{\mathrm{if}}}
\newcommand{\looplocs}{\mathit{L}_{\mathrm{while}}}

\newcommand{\alocs}{\mathit{L}_{\mathrm{a}}}
\newcommand{\wlocs}{\mathit{L}_{\mathrm{w}}}
\newcommand{\rlocs}{\mathit{L}_{\mathrm{r}}}
\newcommand{\Alocs}[1]{\mathit{L}_{\mathrm{A}}^{\mathsf{#1}}}
\newcommand{\Dlocs}{\mathit{L}_{\mathrm{nd}}}
\newcommand{\transitions}{{\rightarrow}}

%%% 
\newcommand{\plocs}{\mathit{L}_{\mathrm{p}}}
\newcommand{\tlocs}{\mathit{L}_{\mathrm{t}}}

\newcommand{\lin}{\loc_\mathrm{init}}
\newcommand{\lout}{\loc_\mathrm{out}}
\newcommand{\val}[1]{\mbox{\sl Val}_{#1}}

\newcommand{\pvars}{V_\mathrm{p}}
\newcommand{\rvars}{V_{\mathrm{r}}}
\newcommand{\pre}{\mathrm{pre}}

\newcommand{\sle}{\sqsubseteq}
\newcommand{\sge}{\sqsupseteq}

\newcommand{\lfp}{\mathrm{lfp}}
\newcommand{\gfp}{\mathrm{gfp}}

\newcommand{\rdvarjdis}{\mathcal D}
\newcommand{\sampset}{\textit{supp}}

\newcommand{\upd}{\mbox{\sl upd}}
\newcommand{\wet}{\mbox{\sl wt}}
\newcommand{\transset}{\mathfrak T}
\newcommand{\valin}{\pv_{\mathrm{init}}}
\newcommand{\ret}{\mbox{\sl ret}}

\newcommand{\win}{w_{\mathrm{init}}}

\newcommand{\sampdpd}{\overline{\Upsilon}}

\newcommand{\outmap}{\text{O}}
\newcommand{\sat}[1]{\langle #1 \rangle}
\newcommand{\monoid}{\mbox{\sl Monoid}}
\newcommand{\handelmanformat}{(\dagger)}

\newcommand{\trunc}{\mathcal{B}}

\newcommand{\ewt}{\mbox{\sl ewt}}
\newcommand{\statemap}{\text{St}}

\newcommand{\valrd}{{\mathbf{r}}}
\newcommand{\frmloc}{\ell^{\mathrm{src}}}
\newcommand{\toloc}{\ell^{\mathrm{dst}}}

\newcommand{\monomials}{\mathbf{M}}

\title{What Kinds of Contracts Do ML APIs Need? 
	%\thanks{Grants or other notes
%about the article that should go on the front page should be
%placed here. General acknowledgments should be placed at the end of the article.}
}
%\subtitle{Do you have a subtitle?\\ If so, write it here}

%\titlerunning{Short form of title}        % if too long for running head

\author{Samantha Syeda Khairunnesa  \and 
	Shibbir Ahmed  \and 
	Sayem Mohammad Imtiaz \and 
	Hridesh Rajan \and 
	Gary T. Leavens
}

%\authorrunning{Short form of author list} % if too long for running head

\institute{Samantha Syeda Khairunnesa  \at
              Dept. of Computer Science and Information Systems, Bradley University \\
              \email{skhairunnesa@fsmail.bradley.edu}           %  \\
%             \emph{Present address:} of F. Author  %  if needed
           \and
           Shibbir Ahmed  \at
           Dept. of Computer Science, Iowa State University \\
           \email{shibbir@iastate.edu} 
           \and
           Sayem Mohammad Imtiaz  \at
           Dept. of Computer Science, Iowa State University \\
           \email{sayem@iastate.edu} 
           \and
           Hridesh Rajan  \at
           Dept. of Computer Science, Iowa State University \\
           \email{hridesh@iastate.edu} 
           \and
           Gary T. Leavens  \at
           Dept. of Computer Science, University of Central Florida \\
           \email{Leavens@ucf.edu} 
}

\date{Received: date / Accepted: date}
% The correct dates will be entered by the editor


\maketitle

\begin{abstract}
%\begin{abstract}

The Fast Reciprocal Square Root Algorithm is a well-established approximation technique consisting of two stages: first, a coarse approximation is obtained by manipulating the bit pattern of the floating point argument using integer instructions, and second, the coarse result is refined through one or more steps, traditionally using Newtonian iteration but alternatively using improved expressions with carefully chosen numerical constants found by other authors. The algorithm was widely used before microprocessors carried built-in hardware support for computing reciprocal square roots. At the time of writing, however, there is in general no hardware acceleration for computing other fixed fractional powers. This paper generalises the algorithm to cater to all rational powers, and to support any polynomial degree(s) in the refinement step(s), and under the assumption of unlimited floating point precision provides a procedure which automatically constructs provably optimal constants in all of these cases. It is also shown that, under certain assumptions, the use of monic refinement polynomials yields results which are much better placed with respect to the cost/accuracy tradeoff than those obtained using general polynomials. Further extensions are also analysed, and several new best approximations are given.

\end{abstract}
	
\begin{abstract}

The Fast Reciprocal Square Root Algorithm is a well-established approximation technique consisting of two stages: first, a coarse approximation is obtained by manipulating the bit pattern of the floating point argument using integer instructions, and second, the coarse result is refined through one or more steps, traditionally using Newtonian iteration but alternatively using improved expressions with carefully chosen numerical constants found by other authors. The algorithm was widely used before microprocessors carried built-in hardware support for computing reciprocal square roots. At the time of writing, however, there is in general no hardware acceleration for computing other fixed fractional powers. This paper generalises the algorithm to cater to all rational powers, and to support any polynomial degree(s) in the refinement step(s), and under the assumption of unlimited floating point precision provides a procedure which automatically constructs provably optimal constants in all of these cases. It is also shown that, under certain assumptions, the use of monic refinement polynomials yields results which are much better placed with respect to the cost/accuracy tradeoff than those obtained using general polynomials. Further extensions are also analysed, and several new best approximations are given.

\end{abstract}


\keywords{Machine Learning \and API contracts \and Empirical software engineering \and
	Software engineering for machine learning}
% \PACS{PACS code1 \and PACS code2 \and more}
% \subclass{MSC code1 \and MSC code2 \and more}
\end{abstract}

% Figure environment removed

\section{Introduction}
Automatic 3D reconstruction of clothed humans using image inputs has gained increasing significance due to its potential applications in a wide array of AR/VR scenarios. High-fidelity reconstructions typically depend on sophisticated capture systems, which are developed with dense camera arrays~\cite{collet2015high,joo2015panoptic,joo2018total}, programmable light-stages~\cite{Vlasic2009, guo2019relightables}, and depth sensors~\cite{newcombe2011kinectfusion,DoubleFusion,BodyFusion,dou2016fusion4d,newcombe2015dynamicfusion}. However, stringent capture environments equipped with complex hardware pose significant challenges for consumer-level applications.


In this context, considerable research effort has been dedicated to developing methods that allow for more flexible capture configurations, such as utilizing a few RGB inputs. Among these works, learning implicit functions \cite{iccv2020PIFu, saito2020pifuhd, hong2021stereopifu} has proven effective in achieving highly detailed reconstructions by integrating the advancements of deep neural networks. These methods employ large multi-layer perceptrons (MLPs) to predict the occupancy probability or truncated signed distance function (TSDF) value of every queried 3D point based on its associated local feature, which is extracted from images. They can recover a continuous surface at arbitrary resolutions without topology restrictions.


However, in typical MLP-based implicit networks, the occupancy or TSDF value at each location is solved independently with planar image features, rendering them less capable of addressing challenging cases such as occlusions. Consequently, these methods suffer from generalization and robustness issues, particularly when tackling strong occlusions caused by large motion or multiple interacting humans. 
Some follow-up studies  \cite{zheng2021deepmulticap,zheng2021pamir,huang2020arch} utilize an extra geometric model, SMPL~\cite{Loper2015}, to improve robustness by introducing strong shape priors. 
Their success typically relies on the assumption of geometrical similarity \cite{huang2020arch} between the shape prior and target reconstruction, making them intractable for handling complex cases with loose clothes and sensitive to errors in SMPL model fitting.



%\ping{this paragraph sounds like `TSDF is better than MLP/SMPL, and we use TSDF to solve the problem'. But in Sec 3, we are telling a different story, saying `MLP needs a 3D convolutional encoder'. We need to make these two sections consistent.}\sicong{I think in this paragraph we claim that the TSDF}


%We opt for Trucated Signed Distance Funtion (TSDF) volumetric representations as they are naturally suitable for convolution operations, which have shown remarkable performance for learning hierarchical features on 2D visual perception tasks \cite{SunXLW19}. 
%Meanwhile, TSDF also describes the gradual geometry change around shape surface, which is not reflected by occupancy volume. 

We instead revisit the 3D volumetric representation and resort to 3D convolutional neural networks (CNNs) for feature learning, due to their impressive performance in feature learning and the ability to incorporate spatial context. However, volumetric methods and 3D convolution involve discretization, which might raise concerns regarding whether a discretized volume can preserve subtle geometric details as continuous representations learned in implicit functions. We investigate the relationship between volume resolution and quantization error on synthetic data by converting target mesh objects to TSDF volumes, as shown in Figure~\ref{fig:quantization_error}. We observe that the quantization errors are significantly reduced by increasing volume resolution and become nearly negligible when reaching a relatively high resolution (e.g., 512 or higher). In other words, achieving fine-detailed reconstruction is not supposed to be restricted by the use of volume representations as long as a proper volume resolution is utilized. Therefore, we present a method with high-resolution feature volumes, e.g., 256 and 512, while traditional volumetric methods \cite{varol18_bodynet,gilbert2018volumetric} are often limited to much lower resolutions, such as 32 or 128.



On the other hand, an increase in volume resolution may lead to a cubic growth of memory overhead \cite{8100085}. Reducing memory costs while guaranteeing the granularity of volumetric representations is necessary for pursuing high-quality reconstruction. Thus, we adopt a coarse-to-fine approach and cull away irrelevant voxels to build a sparse high-resolution feature volume. At the coarse level, the network computes an initial TSDF by applying a U-Net with sparse 3D CNN \cite{3DSemanticSegmentationWithSubmanifoldSparseConvNet} on the sparse feature volume, which is carved by a visual hull. Through our experiments, it turns out that more than 95\% of the volume grids are discarded by the visual hull culling, making the sparse 3D CNN efficient. At the fine level, the network focuses on a narrow band near the zero-level set of the initial TSDF and discretizes the narrow band with smaller voxels. By employing this narrow-band culling, we further shrink the sampling space, resulting in a relatively small range of grid numbers (usually 300K--500K in our experiments) even with a high volume resolution of 512. The remaining voxels in the narrow band are associated with features that fuse high-frequency information from the computed normal maps upon the low-frequency shape from the coarse level to compute the TSDF at high resolution. The final mesh is then extracted from the TSDF using the Marching-Cube algorithm ~\cite{Lorensen87marchingcubes}.
% Different from the u-net sturcture to preserve global topology context, we then apply a shallow 3dcnn to compute the final TSDF $D_{final}$ which contain more local geometry detail.




% \ping{this paragraph can be expanded. It is an important contribution and often ignored by other works. stress on the novel idea of regressing blending weights instead of colors}

In addition to geometry, high-quality mesh texture is also a crucial factor contributing to visual appearance. Directly computing a color field in 3D space, as in \cite{iccv2020PIFu}, struggles to capture high-frequency texture details, while the neural radiance field (NeRF) \cite{yu2020pixelnerf} or the DoubleField~\cite{shao2022doublefield} require expensive per-instance optimization and are often unstable for sparse input images. In contrast, we adopt an image-based rendering approach to compute a texture atlas map, which is efficient and widely supported in existing computer graphics tools. 
Specifically, we compute a blending weight at each 3D point on the mesh surface to determine its color as a weighted average of the colors at its image projections. The blending weights can be computed at a relatively coarse resolution, e.g., 512 volume resolution in our case, and leave texture details to the high-resolution images, such as 1K or 2K. Unlike previous methods that generate blurry texturing results under sparse input, our method generalizes well on both synthetic and real data with just a few input views. 
Figure~\ref{fig:teaser} shows two examples reconstructed by our method. Despite the challenging garment, pose, and occlusion, our method recovers faithful shape, normal, and texture on the right.

%with a wide variety of poses and clothing styles, and it is also adaptive to handle input image with arbitrary resolutions.
%\sicong{For this concern we claim that when the resolution of dicretized volume meets certain threshold (which is 256 in our experiment), the quantization error can be neglected.} 



In summary, the main contributions of this paper are as follows:
\begin{itemize}
\vspace{-0.1in}
  \item 
  We revisit the 3D volumetric representation and demonstrate that it can support clothed human reconstruction with equal or even better performance compared to implicit representation. 
  \item 
  We develop a memory and computation-efficient method for high-resolution volumetric reconstruction using sophisticated sparse 3D CNN, coarse-to-fine estimation, and voxel culling by visual hull and narrow bands. 
  \item 
  We introduce a novel method to compute a texture atlas map, which captures rich appearance details from high-resolution input images.
  \item 
  We achieve impressive results on standard benchmark datasets Twindom and MultiHuman, significantly reducing the point-2-surface (P2S) precision to approximately 0.2cm from just six input views, with more than $50\%$ error reduction compared to the state-of-the-art methods, including DoubleField~\cite{shao2022doublefield} and PIFuHD~\cite{saito2020pifuhd}.
\end{itemize}
\section{Cross-Lingual Diffusion Language Model}
\label{sec:XDLMusion}

% In this section, we present our proposed language modeling objectives designed specifically for diffusion and the diffusion model applied for cross-lingual translation. These objectives cater to both monolingual and multilingual data, and they are situated within the diffusion model framework for facilitating cross-lingual translation.

In this section, we present the Cross-lingual Diffusion Language Model (XDLM), which incorporates a pre-training phase on cross-lingual data, utilizing diffusion techniques for the purpose of non-autoregressive machine translation, and a fine-tuning phase generating corresponding text from one language to another language based on the pre-trained model.

% \subsection{Preliminary}
% \subsubsection{Cross-lingual translation}
% (\irene{combine 3.1.1 and 3.1.2 as NAR machine translation, and, there is no such term called \textit{Cross-lingual translation}, all translation is cross-lingual, it should be either \textit{machine translation} or \textit{cross-lingual language model}})

% Cross-lingual translation typically involves generating an output sequence $Y=\{y_1, y_2,…, y_{|Y|}\}$ from a given input sequence $X=\{x_1,x_2,…,x_{|X|}\}$, with each sequence being in a different language. Three common generative paradigms exist for cross-lingual translation: AutoRegressive (AR) generation, Non-AutoRegressive (NAR) generation, and semi-NAR generation. Ordinarily, diffusion models employ the NAR approach for generation tasks.

% \subsubsection{Non-AutoRegressive(NAR) generation}
% The NAR generation follows the conditional probality: 
% $$
% p_{\theta}(Y|X)=\prod_{i=1}^{|Y|} p_{\theta}(y_i|X)
% $$

% Unlike AutoRegressive (AR) generation, all tokens $y_i$$(0\leq i \leq |Y|)$ in the generated sequence Y are predicted concurrently. The generation solely depends on the input sequence X, without any dependency on preceding tokens. This attribute presents a challenge in determining the length of the generated sequence. To address this issue, the prediction of the output sequence is introduced as an auxiliary task \cite{gu2017non}.

\textbf{Non-AutoRegressive (NAR) Machine Translation}
In machine translation, given the input sequence from a source language $X=\{x_1,x_2,…,x_{|X|}\}$, the task is to generate the output sequence of the translation in the target language $Y=\{y_1, y_2,…, y_{|Y|}\}$. In this work, we focus on the Non-AutoRegressive (NAR) translation setting with the diffusion model. Typically, it has the following conditional probability:  
$$
p_{\theta}(Y|X)=\prod_{i=1}^{|Y|} p_{\theta}(y_i|X).
$$

Unlike AutoRegressive (AR) text generation, all tokens $y_i$$(0\leq i \leq |Y|)$ in the generated sequence $Y$ are predicted concurrently. The generation solely depends on the input sequence $X$, without any dependency on preceding tokens. This attribute presents a challenge in determining the length of the generated sequence. To address this issue, the length prediction of the output sequence is introduced as an auxiliary task \cite{gu2017non}. And the training loss is defined as a weighted sum between the translation loss and the length prediction loss.

\textbf{Diffusion Models}
The Denoising Diffusion Probabilistic Model (DDPM) \cite{ho2020denoising} is a parametrized Markov chain, and it is trained using variational inference to generate samples that match the original input data. 
% a diffusion process for generative tasks was introduced by \cite{ho2020denoising}, yielding impressive results.
The diffusion process comprises a noise-adding forward process and a noise-removing backward process, both of which can be viewed as discrete-time Markov processes. During the forward process, the model gradually introduces random noise with different scheduled variance $\beta_1,...,\beta_t$, with the aim of generating a standard Gaussian noise $x_t$ after $t$ turns. This can be formalized as follows:
$$
q(x_{t+1}|x_t)=\mathcal{N}(x_{t+1};\sqrt{1-\beta_{t+1}}x_t,\beta_{t}\mathbf{I}).
$$

The backward process, the reverse of the forward process, attempts to reconstruct the target sequence from the standard noise. Like the forward process, this procedure is also applied incrementally and can be formalized as follows:

$$
    p(x_{t-1}|x_t)=\mathcal{N}(x_{t-1};\mu_{\theta}^{t-1},\sigma_{\theta}^{t-1}),
$$
$$
    \mu_{\theta}^{t-1}=\frac{1}{\sqrt{\alpha_{t}}}(x_t-\frac{\beta_{t}}{\sqrt{1-\overline(\alpha_{t})}}z_{\theta}(x_{t},t)), 
$$
$$
    \sigma_{\theta}^{{t-1}^2}=\frac{1-\overline{\alpha_{t-1}}}{1-\overline{\alpha_{t}}}\dot \beta_{t},
$$

where $\alpha_t=1-\beta_t, \overline{\alpha_{t}}=\prod_{i=1}^t \alpha_{i}$ and $z_\theta$ comes from the prediction of model parameterized by $\theta$. 
In this work, we apply discrete diffusion for text generating and cross-lingual translation. Based on \citet{zheng2023reparameterized}, we follow the proposed discrete diffusion model with the following routing mechanism.

$
    x_{t-1}, v_{t-1} \sim q(x_{t-1},v_{t-1}|x_t,x_0) \\
    q(v_{t-1}|x_t,x_0)=q(v_{t-1})=Bernoulli(\lambda) \\
    q(x_{t-1}|v_{t-1},x_t,x_0)= \\
    v_{t-1}x_t+(1-v^{(1)}_{t-1})q_{noise}, \quad if \quad x_t = x_0 \\
    v_{t-1}x_0+(1-v_{t-1}^{(2)})q_{noise} (x_t), \quad if \quad x_t \neq x_0 \\
$


Which models the joint distribution over both $x$ and $v$. The sampling process here also takes the reparameterized method, which improves flexibility and expressiveness compared to the original process.

% Figure environment removed
\textbf{Translation Diffusion Language Modeling (TDLM)}
% Contrary to previous language modeling objectives for diffusion models, which primarily focus on monolingual data and neglect the potential to harness cross-lingual modeling capabilities from parallel datasets, we propose a pretraining process for parallel language pairs along with a corresponding modeling objective.
Unlike previous diffusion model objectives for language modeling that primarily concentrate on monolingual data, we target to exploit cross-lingual modeling capabilities from parallel datasets. Consequently, we propose a pretraining process named Translation Diffusion Language Modeling (TDLM), aiming at enhancing cross-lingual pretraining with diffusion models. As illustrated in Figure 1, we first concatenate both source and target sentences and generate the corresponding language and position embedding sequences, and then stack them as the input to a diffusion model. 
% we select both source and target sentences, generate their corresponding language and position embedding series, and concatenate them to form the input text stream. 
In a similar vein to \citet{lin2023text}, we random mask 15\% of the tokens to the input as \cite{lample2019cross} designed, tasking the model with predicting the noise and its surrounding text based on the cross-lingual context. This denoising setting assists the model in grasping the cross-lingual context.


\subsection{Taxi drivers' labor supply behavior of and wage elasticity}
\added{While it is observed that the taxi market has been significantly affected by the rise of TNCs, we notice that taxi drivers' work hours do not change even though the income decreases. As taxi drivers are aware of the competition from the TNC sector, under the neo-classical scheme, it is reasonable to expect that taxi drivers will lower their work hours if their expected income decreases, which obviously contradicts the aforementioned observation. The underlying logic is that, with the increasing number of TNC trips, taxi market share is expected to decrease due to the increasing TNC competition, which may result in a lower income level and a lower expected income level. In light of this issue, we propose the research question two and three to investigate the reasons behind the dilemma. The second research question is: \emph{\textbf{does the driver decrease his expected wage along with the increasing number of TNC trips?}} On the other hand, with more TNC trips, taxi drivers are likely to encounter worse daily experiences where it is more difficult to reach a similar income level as before with the same amount of daily efforts (work hours). As discussed earlier, target work hours (reference income) will discourage the employee's motivation thus leading to lower productivity and eventually conducting a vicious circle for the industry. This idea gives rise to our suspicion that the taxi labor supply may now be better explained by the RDP behavior instead of the neoclassical one. On this basis, our interest is to shed the light on quantifying the RDP behavior in the taxi market to interpret the taxi drivers' non-intuitive response to the competition from the TNC sector. This question has its root in Crawford and Meng's~\cite{crawford2011new} study, where the taxi labor supply behavior is mainly driven by the income target. To this end, the labor supply estimation under both the RDP model and NS model can be analyzed from two aspects: wage proportion and wage elasticity. In this regard, we propose the third research question: \emph{\textbf{does the reference-dependent preference behavior present among taxi drivers with the increasing number of TNC trips? }}}


To test the second research question, we use the wage decomposition method by regressing the drivers' average daily expected wage on the natural log-transformation of monthly TNC trips. The results are given in Table~\ref{tab:expected wage1} and Table~\ref{tab:expected wage2}. 

% Please add the following required packages to your document preamble:
% \usepackage{multirow}
\begin{table}[!h]
\centering
\caption{Result of log-transformation of TNC trips: experiment \uppercase\expandafter{\romannumeral1}}
\label{tab:expected wage1}
\begin{tabular}{lcccc}
\toprule
\multirow{2}{*}{Month/Year}                                & \multicolumn{2}{c}{Yellow taxi} & \multicolumn{2}{c}{Green taxi} \\
                                                                    & Coefficient     & P value  & Coefficient     & P value  \\
                                                                    \hline
\multirow{1}{*}{01/13-06/15} & 0.0009 & 0.457 &-&- \\
\multirow{1}{*}{07/13-12/15} & 0.0006& 0.622&-&-\\
\multirow{1}{*}{01/14-06/16} & -0.0005 & 0.698&0.0044&0.046 *\\
\multirow{1}{*}{07/14-12/16} & -0.0031& 0.085 .&-0.0050&0.044 *\\
\multirow{1}{*}{01/15-06/17} & -0.0518& 0.006 **&-0.1167&4.16E-06 ***\\
\multirow{1}{*}{07/15-12/17} & -0.0816& 0.001 ***&-0.1537 & 2.85E-05 ***\\
\multirow{1}{*}{01/16-06/18} & -0.0656& 0.069.&-0.1113 &0.028 *\\
\multirow{1}{*}{07/16-12/18} & 0.0665& 0.068.&0.0556&0.188\\
                           \bottomrule
\end{tabular}

Note: *: P $\leq$ 0.05; **: P $\leq$ 0.01; ***: P $\leq$ 0.001.

\end{table}


\begin{table}[!h]
\centering
\caption{Result of log-transformation of monthly TNC trips: experiment \uppercase\expandafter{\romannumeral2}}
\label{tab:expected wage2}
\begin{tabular}{lcccc}
\toprule
\multirow{2}{*}{Month/Year}                                & \multicolumn{2}{c}{Yellow taxi} & \multicolumn{2}{c}{Green taxi} \\
                                                                    & Coefficient    & P value  & Coefficient    & P value  \\
                                                                    \hline
\multirow{1}{*}{01/13-12/14} &0.0013 &0.469 & -&-\\
\multirow{1}{*}{07/13-06/15} &0.0014 &0.293&-&-\\
\multirow{1}{*}{01/14-12/15} &0.0005 &0.761& 0.0045&0.073 .\\
\multirow{1}{*}{07/14-06/16} &-0.0020 &0.141&-0.0020&0.211 \\
\multirow{1}{*}{01/15-12/16} &-0.0650 &0.0002 ***&-0.0779&0.004 ** \\
\multirow{1}{*}{07/15-06/17} &-0.0401&0.0022 ** &-0.1447 & 0.003 **\\
\multirow{1}{*}{01/16-12/17} &-0.1218 & 0.005 **&-0.2065 &0.001 ***\\
\multirow{1}{*}{07/16-06/18} & 0.0323&0.512&0.0597&0.293\\
\multirow{1}{*}{01/17-12/18} & 0.1279&0.027 *&0.1900&0.003 **\\
                           \bottomrule
\end{tabular}

Note: *: P $\leq$ 0.05; **: P $\leq$ 0.01; ***: P $\leq$ 0.001.
\end{table}


Although monthly TNC trips are observed to be positively related to the yellow taxi drivers' expected wage from January 2013 to December 2014, the effect is non-significant. Meanwhile, green taxi drivers are found to increase their expected wage due to the increase of TNC trips, which implies that the taxi market is still under-supplied with TNC trips in the beginning stage. Besides, the TNC trips are also found to negatively impact the expected wage from January 2015 to December 2017 for both green and yellow taxi drivers, which means the taxi market gradually shifted into the over-supplied state with increasing competition between taxis and TNCs (at 0.01 significance level). However, the trend of decreasing expected wage changes in 2018 when the increase of TNC trips is found to be positively related to drivers' expected wage (at 0.05 significance level for both yellow and green taxi drivers in experiment \uppercase\expandafter{\romannumeral2}) due to an income rebound at individual-level as shown in Figure~\ref{fig:monthly fare  & TNC trips}. The income rebound is primarily due to the loss of total market supply (drivers quit the market), which is faster than the reduction in taxi ridership and total market revenue, as shown in Figure ~\ref{fig:proportion}. As a consequence, the results from the experiment indicate that the increase of TNC trips significantly decreases the taxi drivers' expected wage at most of the time. Besides, Table~\ref{tab:expected wage1} and Table~\ref{tab:expected wage2} also indicate the consistency of our results under different data compositions. 

% Figure environment removed

% Figure environment removed


To test the third research question, we measure the unanticipated and anticipated transitory wage variation based on the wage decomposition method and conduct the t-test on comparing the proportion of the unanticipated transitory wage in each group with its base year group. The proportion of the unanticipated transitory wage quantifies how much of the wage is unexpected to taxi drivers. The results from the t-test verify if there is a significant change in drivers' unexpected wage variation over time. 
\begin{table}[!h]
\centering
\caption{Results of wage variation decomposition: experiment \uppercase\expandafter{\romannumeral1}}
\label{tab:wage decomposition1}
\begin{tabular}{lcccccc}
\toprule
\multirow{2}{*}{Month/Year}                                & \multicolumn{3}{c}{Yellow taxi} & \multicolumn{3}{c}{Green taxi} \\
                                                                    & Fixed     & Anticipated  & Unanticipated   & Fixed     & Anticipated  & Unanticipated   \\
                                                                    \hline
\multirow{2}{*}{01/13-06/15}& 9.04E-06& 0.0021& 0.0002 &-&-&-\\
                           & (0.42\%)&(90.93\%)&(8.65\%)&-&-&-
\\

\multirow{2}{*}{07/13-12/15} & 1.46E-05& 0.0020& 0.0002&-&-&-
\\
                           & (0.65\%) & (90.05\%) &
                           (9.30\%) &-&-&-\\  
                           
\multirow{2}{*}{01/14-06/16}& 0.0001& 0.00198& 0.0002  &0.000412&0.0049&0.0004 \\
                           & (4.42\%)&(85.64\%)&(9.94\%)&(7.26\%)&(86.42\%)&(6.32\%)
\\
\multirow{2}{*}{07/14-12/16} & 0.0002& 0.0018& 0.0003& 0.0002&0.0040&0.0006 
\\
                           & (9.42\%) & (78.45\%) &
                           (12.13\%)& (3.22\%) & (84.36\%) &
                           (12.42\%)  \\  
\multirow{2}{*}{01/15-06/17} & 0.0002                                                       & 0.0029                                                          & 0.0004 &     3.16E-30&0.0070&0.0011                                                    \\
                           & (5.62\%)                                                      & (82.09\%)                                                         & (12.29\%)  & (0\%)                                                      & (85.98\%)                                                         & (14.02\%)                                                        \\
\multirow{2}{*}{07/15-12/17 } & 6.04E-05                                                       & 0.0025                                                          & 0.0005&7.89E-31& 0.0062&0.0091                                                    \\
                           & (1.95\%)                                                      & (80.65\%)                                                         & (17.40\%)  & (0\%)                                                      & (87.24\%)                                                         & (12.76\%)                                                         \\

\multirow{2}{*}{01/16-06/18} & 0.0002                                                      & 0.0025                                                           & 0.0005  &7.10E-30& 0.0055&0.0013                                                \\
                           & (6.76\%)                                                      & (77.28\%)                                                         & (15.96\%)  & (0\%)                                                      & (80.86\%)                                                         & (19.14\%)                                                         \\
\multirow{2}{*}{07/16-12/18} & 5.82E-05                                                       & 0.0021                                                          & 0.0004 &    3.16E-30&0.0028&0.0008                                                    \\
                           & (2.33\%)                                                      & (83.33\%)                                                         & (14.34\%)  & (0\%)                                                      & (78.03\%)                                                         & (21.97\%)                                                         \\
                           \bottomrule
\end{tabular}

Note: the proportion of variation of each part in the total variation is in the bracket.
\end{table}


\begin{table}[!h]
\centering
\caption{Results of wage variation decomposition: experiment \uppercase\expandafter{\romannumeral2}}
\label{tab:wage decomposition2}
\begin{tabular}{lcccccc}
\toprule
\multirow{2}{*}{Month/Year}                                & \multicolumn{3}{c}{Yellow taxi} & \multicolumn{3}{c}{Green taxi} \\
                                                                    & Fixed     & Anticipated  & Unanticipated   & Fixed     & Anticipated  & Unanticipated   \\
                                                                    \hline

\multirow{2}{*}{01/13-12/14 }& 9.6E-07& 0.0022& 0.0002 &-&-&-\\
                           & (0.04\%)&(92.84\%)&(7.12\%)&-&-&-
\\
\multirow{2}{*}{07/13-06/15} & 4.03E-05& 0.0021& 0.0002 &-&-&- \\
                           & (1.76\%)&(91.50\%)&(6.74\%)&-&-&-
\\
\multirow{2}{*}{01/14-12/15} & 4.65E-05& 0.0020& 0.0002&0.0005&0.0056&0.0003
\\
                           & (2.05\%) & (89.80\%) &
                           (8.15\%)& (8.29\%) & (87.17\%) &
                           (4.54\%)\\

\multirow{2}{*}{07/14-06/16} & 5.08E-05                                                       & 0.0015                                                           & 0.0002 &   7.15E-07&0.0018&0.0002                                                        \\
                           & (3.16\%)                                                      & (85.10\%)                                                         & (11.98\%) & (0.04\%)                                                      & (91.54\%)                                                         & (8.42\%)                                                           \\
\multirow{2}{*}{01/15-12/16} & 9.34E-05                                                       & 0.0020                                                           & 0.0003  & 3.16E-30&0.0047&0.0006                                                          \\
                           & (3.93\%)                                                      & (82.97\%)                                                         & (13.10\%) & (0\%)                                                      & (88.06\%)                                                         & (11.94\%)                                                          \\
\multirow{2}{*}{07/15-06/17} & 1.21E-05                                                       & 0.0024                                                          & 0.0005    & 7.89E-31&0.0066&0.0007                                                   \\
                           & (0.41\%)                                                      & (83.33\%)                                                         & (16.26\%)    & (0\%)                                                      & (89.84\%)                                                         & (10.16\%)                                                        \\
\multirow{2}{*}{01/16-12/17} & 8.97E-05                                                      & 0.0022                                                          & 0.0006    & 7.10E-30&0.0054&0.0012                                                       \\
                           & (3.10\%)                                                      & (76.90\%)                                                         & (20\%)& (0\%)                                                      & (82.05\%)                                                         & (17.95\%)                                                           \\
\multirow{2}{*}{07/16-06/18} & 7.05E-05                                                       & 0.0022&0.0005   &3.16E-05&0.0031&0.0008                                                                                                      \\
                           & (2.52\%)                                                      & (79.30\%)                                                         & (18.18\%)  & (0.80\%)                                                      & (78.68\%)                                                         & (20.52\%)                                                         \\
\multirow{2}{*}{01/17-12/18} & 0.0001                                                      & 0.0024                                                         & 0.0004    & 3.16E-30&0.0030&0.0009                                                 \\
                           & (4.35\%)                                                      & (81.33\%)                                                         & (14.32\%) & (0\%)                                                      & (76.55\%)                                                         & (23.45\%)                                                          \\
                                                  
                           \bottomrule
\end{tabular}

Note: the proportion of variation of each part in the total variation is in the bracket.
\end{table}

The wage decomposition results are presented in Table~\ref{tab:wage decomposition1}  and Table~\ref{tab:wage decomposition2}. The results quantify the fixed, the anticipated, and the unanticipated transitory variation of drivers' wage over time. For the yellow taxi drivers, we observe that the anticipated transitory variation dominates the total variation in both experiments. The proportion of unanticipated wage variation from January 2013 to December 2014 precisely matches the findings in Farber's study~\cite{farber2015you} when taxi market is still monopolistic. That is, when supply is not saturated, almost all of the taxi drivers' work behavior can be explained by the NS behavior, and the labor supply is positively correlated to their wage. However, the proportion of anticipated wage variation is observed to decrease as TNCs grow. And the unanticipated transitory variation gradually increases and reaches its \added{peak} at the end of 2017 when 20 \% of yellow taxi drivers' behavior can be explained by RDP as shown in Figure~\ref{fig:relationship}. This result matches well with our previous findings when drivers decrease their expected wage as shown in Table~\ref{tab:expected wage1} and Table~\ref{tab:expected wage2}, as well as when the market revenue and labor supply significantly decreased (see Table~\ref{tab:yellow taxi market time period variation}). For green taxi drivers, only a small proportion (4\%) of green taxi drivers shows RDP behavior in the beginning stage (from January 2014 to December 2015). With the increasing competition from TNCs(see Figure~\ref{overall}), green taxi drivers' expected wage decreases, which results in the reduction of green taxi drivers in their work hours and income from July 2015 to June 2017 (see Table~\ref{tab:green taxi market time period variation}). Therefore, green taxi drivers exhibit revenue optimizing behavior as suggested by the NS (drivers having RDP behavior also slightly decrease during this period from 14\% to 12\% in experiment \uppercase\expandafter{\romannumeral1} and 12\% to 10\% in experiment \uppercase\expandafter{\romannumeral2}). After the second half of 2017, green taxi drivers show increasing RDP behavior due to the increasing competition from TNCs (see Figure~\ref{fig:relationship}). Finally, we observe that RDP behavior captures over 20\% of the green taxi drivers in the current taxi market.

The results of the t-test are given in Table~\ref{tab:test hypothesis 2 for exp 1} and Table ~\ref{tab:test hypothesis 2 for exp 2}. The relationship between TNC trips and unanticipated transitory wage variation can be seen in Figure~\ref{fig:relationship}. The unanticipated transitory wage of yellow and green taxi drivers (both at 0.05 significance level) is significantly higher than those of the base year group. The results indicate that drivers are facing more uncertainty regarding daily wage and clearly illustrate the change of drivers' work behavior over time due to the increasing number of TNC trips. Therefore, \added{we conduct that RDP behavior presents among taxi drivers with more number of TNC trips}. Based on the results, we observe that yellow taxi drivers face much more serious competition than green taxi drivers before 2017. Moreover, such competition leads to an unsustainable state in the ride-sharing market and 20\% yellow taxi drivers having RDP behavior at the end of 2017, which means a high proportion of taxi drivers lose confidence about the taxi industry. Finally, \added{yellow taxi drivers} quit the taxi market and this leads to the decrease of RDP behavior among taxi drivers after 2017. On the other hand, the green taxi market benefits from the increase in demand at the beginning of TNCs' growth. Meanwhile, green taxi drivers present NS behavior with an increasing number of TNC trips before July 2017, which indicates that the green taxi market is more flexible than the yellow taxi market during this period. However, more green taxi drivers have RDP behavior after June 2017. And the unanticipated transitory wage variation of green taxi drivers is found to account for over 20\% of the total wage variation in both experiments at the end of 2018, which is over three times as compared to when the taxi market is still monopolistic. Consequently, the RDP behavior should not be ignored, and NS behavior is no longer suitable to interpret the total taxi drivers' work behavior in a competitive market. Instead, at least $20\%$ of green taxi drivers performs in a loss-aversion manner in the market rather than the revenue maximizing behavior, which is widely used when the taxi market is still monopolistic. This finding is aligned with the second explanation for the question that we raised to the OLS model, \added{which is that the driver has a specific reference target}. Moreover, based on the results from the OLS model, the monthly income of taxi drivers is found to be significantly decreased due to TNC competition while the individual labor supply is found to be barely unaffected (see Figure~\ref{fig:hours}),
which can be explained by neither the NS nor the RDP behavior. However, the trend of drivers' behavior shifting towards RDP suggests the reason for this observation. Furthermore, combining the results of the second and third research questions, we conclude that drivers decrease their income target and some of them even quit the market, so that the remaining drivers are observed to still serve the same amount of work hours. Meanwhile, the drop in medallion prices helps to partially offset their losses from the lowered income target. Nevertheless, it also points out the necessity to consider the regulation of TNCs for the sustainability of the taxi market. \added{This issue requires further investigation with related data sources. Finally, Table~\ref{tab:wage decomposition1} and Table~\ref{tab:wage decomposition2} again confirm the consistency of our results under different data compositions.}


\begin{table}[!h]
\centering
\caption{t-test for the change of unanticipated wage variation in experiment\uppercase\expandafter{\romannumeral 1} }
\label{tab:test hypothesis 2 for exp 1}
\begin{tabular}{lcc}
\toprule
Month/Year   & Yellow taxi& Green taxi \\  
                                                                    \hline

01/13-06/15&0.896&-
\\
07/13-12/15&0.853&-
\\
01/14-06/16&0.696&0.957
\\

07/14-12/16  &         0.228&0.032 *                                            \\
01/15-06/17           &0.118     &0.0007 ***   \\
07/15-12/17  &0.038*&0.0047 ** \\
01/16-06/18 & 0.023 * &0.002 **\\
07/16-12/18   &0.051 .&0.003 *  \\
\bottomrule

\end{tabular}
\end{table}


\begin{table}[!h]
\centering
\caption{t-test for the change of unanticipated wage variation in experiment \uppercase\expandafter{\romannumeral 2} }
\label{tab:test hypothesis 2 for exp 2}
\begin{tabular}{lcc}
\toprule
Month/Year   & Yellow taxi& Green taxi \\  
                                                                    \hline

07/13-06/15&0.6461&-
\\
01/14-12/15&0.922&-
\\

07/14-06/16 &         0.367&0.606                                         \\
01/15-12/16           &0.121     &0.019 *   \\
07/15-06/17   &0.0104*&0.015 *  \\
01/16-12/17 & 0.048 * &0.002 **\\
07/16-06/18   &0.138&0.046 *\\
01/17-12/18& 0.195 &0.006 **\\
\bottomrule

\end{tabular}

\end{table}

% Figure environment removed


% Figure environment removed


\begin{table}[!h]
\centering
\caption{Wage elasticity of taxi labor supply: experiment \uppercase\expandafter{\romannumeral 1}}
\label{tab:elast1}
\begin{tabular}{lcccc}
\toprule
\multirow{2}{*}{Month/Year}& \multicolumn{2}{c}{Yellow taxi}&\multicolumn{2}{c}{Green taxi}\\
     & Coefficient & Adj.R^2& Coefficient & Adj.R^2 \\
\hline
\multirow{2}{*}{01/13-06/15} & 0.6611      & 0.844& -&-\\
          & (5.00E-13 ***)     &   &-&        \\
\multirow{2}{*}{07/13-12/15} & 0.6526      & 0.844&-&-\\
          & (5.06E-13 ***)     &         &-&     \\
\multirow{2}{*}{01/14-06/16} & 0.6383      & 0.837&0.8006&0.92\\
          & (9.39E-13 ***)     &    &(4.55E-07 ***)&          \\
\multirow{2}{*}{07/14-12/16} & 0.5592      & 0.731&0.6228&0.953\\
          & (1.10E-09 ***)     &        &(5.96E-08 ***)&       \\
\multirow{2}{*}{01/15-06/17} & 0.5040      & 0.764&0.5822&0.903\\
          & (1.69E-10 ***)     &     &(7.51E-13 ***)&         \\
\multirow{2}{*}{07/15-12/17} & 0.4780      & 0.712&0.6103&0.909\\
          & (2.82E-09 ***)     &  &(2.48E-16 ***)&             \\
\multirow{2}{*}{01/16-06/18} & 0.4924      & 0.745&0.5770&0.863\\
          & (5.23E-10 ***)     &      &(8.34E-14 ***)&        \\
\multirow{2}{*}{07/16-12/18}& 0.4944      & 0.699&0.4776&0.541\\
          & (5.49E-09 ***)     &     &(2.24E-06 ***)&          \\
          \bottomrule
          
\end{tabular}

Note: the p-value of the estimate for log-transformation of monthly income per taxi driver is in the bracket (. : p $\leq$ 0.1; *: P $\leq$ 0.05; **: P $\leq$ 0.01; ***: P $\leq$ 0.001).
\end{table}

\begin{table}[!h]
\centering
\caption{Wage elasticity of taxi labor supply: experiment \uppercase\expandafter{\romannumeral 2}}
\label{tab:elast2}
\begin{tabular}{lcccc}
\toprule
\multirow{2}{*}{Month/Year}& \multicolumn{2}{c}{Yellow taxi}&\multicolumn{2}{c}{Green taxi}\\
     & Coefficient & Adj.R^2& Coefficient & 
     Adj.R^2 \\
\hline
\multirow{2}{*}{01/13-12/14}
& 0.6477      & 0.762&-&-\\
          & (5.68E-11 ***)     &  &-&          \\
\multirow{2}{*}{07/13-06/15} & 0.6582      & 0.854&-&-\\
          & (7.09E-11 ***)     &&-&            \\
\multirow{2}{*}{01/14-12/15} & 0.6436      & 0.827&0.8095&0.929\\
          & (4.76E-10 ***)     & &(2.35E-14 ***) &          \\
\multirow{2}{*}{07/14-06/16} & 0.6243      & 0.772&0.6834&0.762\\
          & (1.02E-08 ***)     & &(1.59E-08 ***) &            \\
\multirow{2}{*}{01/15-12/16}  & 0.5362      & 0.702&0.6087&0.862\\
          & (1.99E-07 ***)     & &(3.74E-11 ***)  &           \\
\multirow{2}{*}{07/15-06/17} & 0.5151      & 0.74&0.6379&0.933\\
          & (4.24E-08 ***)     &   &(1.30E-14 ***) &           \\
\multirow{2}{*}{01/16-12/17} & 0.4746      & 0.687&0.5960&0.893\\
          & (3.38E-07 ***)     &  &(2.33E-12 ***)            \\
\multirow{2}{*}{07/16-06/18} & 0.488      & 0.696 &0.5376&0.719\\
          & (2.45E-07 ***)     &&(1.02E-07 ***) &              \\
\multirow{2}{*}{01/17-12/18} &0.5243      & 0.798&0.4523&0.515\\
          & (2.59E-09 ***)     & &(4.81E-05 ***) &             \\
          \bottomrule
          
\end{tabular}

Note: the p-value of the estimate for log-transformation of monthly income per taxi driver is in the bracket (. : p $\leq$ 0.1; *: P $\leq$ 0.05; **: P $\leq$ 0.01; ***: P $\leq$ 0.001).

\end{table}

% Figure environment removed

Finally, although the increase in unanticipated transitory variation suggests that drivers' behavior may shift from NS to RDP, it is not equivalent to assert that the RDP should explain drivers' behavior. We further conduct the wage elasticity analysis to provide a better understanding of this issue. The results \added{are presented} in Table~\ref{tab:elast1} and Table~\ref{tab:elast2}. We observe that wage elasticity yields a similar trend as the change of unanticipated transitory variation proportion and the change of RDP behavior for both yellow and green taxi drivers (see Figure~\ref{fig:relationship}). The wage elasticity for both yellow and green taxi drivers remains positive, which is inconsistent with the elasticity of RDP models being -1. Although the wage elasticity is positive, it implies the yellow taxi drivers reached the lowest wage elasticity at the period of January of 2016 to December 2017 (see Figure~\ref{fig:elasticity}), when 1\% increase in drivers' wage leads to 0.47\% increase in monthly work hours. Besides, the wage elasticity of yellow taxi stayed above 0.6 before June 2016 but dropped rapidly until December 2017. This finding corresponds to when the transitory wage variation proportion has changed, as shown in Figure~\ref{fig:relationship}. For the wage elasticity of green taxi drivers, there is a rebound at the period of July 2015 to June 2017 (see Figure~\ref{fig:elasticity}), which confirms our insights that they show revenue-maximizing behavior along with the increasing of TNC trips during this period. Since then, the wage elasticity decreased. The similar results from the comparison of two experiments again verify the consistency of our results.

In conclusion, the results from both the wage variation decomposition and wage elasticity of labor supply are verify the conclusion that taxi drivers show RDP behavior. And the labor supply behavior of both yellow and green taxi drivers have changed since the increase of TNC trips at the beginning of 2015. In the current market, over 14\% yellow taxi and 20\% green taxi drivers' behavior can be explained by RDP. The RDP behavior among taxi drivers implies that more number of drivers will quit the taxi market due to the loss of confidence~\cite{eliaz2014reference}. And the slightly weakened RDP behavior among yellow taxis after 2018 is likely to support this claim, where the taxi market has lost a number of active taxi drivers so that the remaining drivers are less pessimistic about the market with less competition from the same sector. Finally, 1\% increase in TNC trips in the current market (based on the estimation from July 2016 to December 2018) will lead to 0.28\% decrease in yellow taxi monthly revenue and 0.68\% decrease in green taxi monthly revenue. Furthermore, 1\% increase in TNC trips in the
current market will result in 0.29\% reduction in yellow taxi monthly work hours (including these who quit from the market) and 0.75\% reduction in green taxi monthly work hours. The insights from both labor supply estimation and wage elasticity analyses suggest that taxi drivers show increasingly negative responses to the market over time. That is, along with the increase of TNC trips, more drivers show RDP behavior and tend to quit from the market if there exists no appropriate policy that regulates the market of FHVs and taxis. \added{And pricing regulation will serve as one fundamental tool in guiding drivers' behavior and reshaping market labor supply. Meanwhile, market entry regulation is also important to maintain the balance between taxis and FHVs so that neither sector will contribute to the growth of the RDP behavior in the other sector. In conclusion, to promote a sustainable and equitable environment in the competitive market of taxis and FHVs, it is important to reconsider the pricing regulation and entry restrictions by taking the behavior of labor supply into consideration.}








\section{Related Work}
\label{appsec: related work}
Bayesian causal discovery literature has primarily focused on inference in linear models with closed-form posteriors or marginalized parameters. Early works considered sampling directed acyclic graphs (DAGs) for discrete~\cite{cooper1992bayesian, madigan1995bayesian, heckerman2006bayesian} and Gaussian random variables~\cite{friedman2003being, tong2001active} using Markov chain Monte Carlo (MCMC) in the DAG space. However, these approaches exhibit slow mixing and convergence~\cite{eaton2012bayesian,grzegorczyk2008improving}, often requiring restrictions on number of parents~\cite{kuipers2017partition}. %Alternative exact dynamic programming methods are limited to small settings~\cite{koivisto2012advances}. 

Recent advances in variational inference~\cite{zhang2018advances} have facilitated graph inference in DAG space, with gradient-based methods employing the NOTEARS DAG penalty \cite{zheng2018dags}.\cite{annadani2021variational} samples DAGs from autoregressive adjacency matrix distributions, while \cite{lorch2021dibs} utilizes Stein variational approach \cite{liu2016stein} for DAGs and causal model parameters. \cite{cundy2021bcd} proposed a variational inference framework on node orderings using the gumbel-sinkhorn gradient estimator \cite{mena2018learning}. \cite{deleu2022bayesian,nishikawa2022bayesian} employ the GFlowNet framework \cite{bengio2021gflownet} for inferring the DAG posterior. Most methods, except\cite{lorch2021dibs} are restricted to linear models, while \cite{lorch2021dibs} has high computational costs and lacks DAG generation guarantees compared to our method.
% at least quadratic scaling complexity, both with respect to the number of nodes (due to the DAG penalty) as well as number of posterior samples. Our proposed approach instead has linear complexity with respect to number of posterior samples and does not require any additional DAG penalty.     

In contrast, \emph{quasi-Bayesian} methods, such as DAG bootstrap \cite{friedman2013data}, demonstrate competitive performance. DAG bootstrap resamples data and estimates a single DAG using PC \cite{spirtes2000causation}, GES \cite{chickering2002optimal}, or similar algorithms, weighting the obtained DAGs by their unnormalized posterior probabilities. Recent neural network-based works employ variational inference to learn DAG distributions and point estimates for nonlinear model parameters \cite{charpentier2022differentiable,geffner2022deep}.
\section{Conclusion and Future Work}
In this work, I design corruption-robust algorithms for the Lipschitz contextual search problem. I present the \emph{agnostic checking} technique and demonstrate its effectiveness in designing corruption-robust algorithms. There are several open problems for future research. First, in the algorithm I propose for pricing loss, the schedule for agnostic checks is fixed upfront. Can the learner design an adaptive checking schedule for the pricing loss? Second, this work assumes the learner has knowledge of the Lipschitz constant $L$. Can the learner design efficient no-regret algorithms without knowledge of $L$? 

\balance

%\section{Introduction}
%\label{intro}
%Your text comes here. Separate text sections with
%\section{Section title}
%\label{sec:1}
%Text with citations \cite{RefB} and \cite{RefJ}.
%\subsection{Subsection title}
%\label{sec:2}
%as required. Don't forget to give each section
%and subsection a unique label (see Sect.~\ref{sec:1}).
%\paragraph{Paragraph headings} Use paragraph headings as needed.
%\begin{equation}
%a^2+b^2=c^2
%\end{equation}
%
%% For one-column wide figures use
%% Figure environment removed
%%
%% For two-column wide figures use
%% Figure environment removed
%%
%% For tables use
%\begin{table}
%% table caption is above the table
%\caption{Please write your table caption here}
%\label{tab:1}       % Give a unique label
%% For LaTeX tables use
%\begin{tabular}{lll}
%\hline\noalign{\smallskip}
%first & second & third  \\
%\noalign{\smallskip}\hline\noalign{\smallskip}
%number & number & number \\
%number & number & number \\
%\noalign{\smallskip}\hline
%\end{tabular}
%\end{table}

\begin{acknowledgements}
This material is based upon work supported by the National Science Foundation under Grant CCF-15-18897, CNS-15-13263, CCF-17-23215, CCF-17-23432, CCF-19-34884, CNS-17-23198, CCF-22-23812. Any opinions, findings, and conclusions or recommendations expressed in this material are those of the authors and do not necessarily reflect the views of the National Science Foundation.
\end{acknowledgements}

\section*{Data Availability Statement}

The dataset (labeled \SO posts, queries, source codes, etc.) generated during the study are available in the \emph {figshare} repository, \url{https://figshare.com/s/c288c02598a417a434df}. 

% Authors must disclose all relationships or interests that 
% could have direct or potential influence or impart bias on 
% the work: 
%
\section*{Conflict of interest}

The authors declare conflict of interest with the people affiliated with Iowa State University, University of Central Florida, and Bradley University.



% BibTeX users please use one of
\bibliographystyle{spbasic}      % basic style, author-year citations
%\bibliographystyle{spmpsci}      % mathematics and physical sciences
%\bibliographystyle{spphys}       % APS-like style for physics
%\bibliography{}   % name your BibTeX data base
\bibliography{emse23.bib}
%\begin{thebibliography}{}
%%
%% and use \bibitem to create references. Consult the Instructions
%% for authors for reference list style.
%%
%\bibitem{RefJ}
%% Format for Journal Reference
%Author, Article title, Journal, Volume, page numbers (year)
%% Format for books
%\bibitem{RefB}
%Author, Book title, page numbers. Publisher, place (year)
%% etc
%\end{thebibliography}

\end{document}
% end of file template.tex

