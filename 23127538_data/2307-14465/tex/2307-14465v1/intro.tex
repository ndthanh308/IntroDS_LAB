\section{Introduction}
\label{sec:intro}

Software developers are increasingly integrating machine learning (\ML) 
into systems using ML libraries' application programming interfaces (APIs).
However, ML software is bug-prone ~\cite{10.1145/3213846.3213866, islam19, dfaults} and
like traditional software could benefit from adopting a design-by-contract methodology~\cite{islam19}.
Contracts can specify the expected behavior of an API and help client code use the API correctly,
e.g., a contract might require that the \texttt{fit} method be applied to a model before calling the \texttt{predict} method. Another example can be given using the \texttt{MaxPooling2D} method for retaining the most prominent features of the feature map in a convolutional neural network (CNN). There is a contract on the \texttt{MaxPooling2D} method's argument, \texttt{data\_format}, based on the shape of the input image. If the input image has the shape (N, C, H, W), then the value for the argument \texttt{data\_format} is set to \texttt{channels\_first}. If the input has the shape (N, H, W, C), then \texttt{data\_format} must be set to \texttt{channels\_last}. Here, the letters N, H, W, and C represent the following: the number of images in the batch, the height of the image, the width of the image, and the number of channels of the image.

There is a rich body of prior work 
that can be 
grouped into two categories: work on contracts for non-\ML software and work on \ML software.

The first category, contracts for non-\ML software,
can be further divided into two types: behavioral and temporal.
Behavioral contracts \cite{10.1145/363235.363259, Meyer88, 10.1109/ASE.2009.60, 10.1145/1858996.1859035, nguyen2014mining, khairunnesa2017exploiting}
specify acceptable program states, typically for calls to individual methods in an API. For instance, in the Java Development Kit (JDK) String class, the precondition `\texttt{beginIndex}$<=$\texttt{endIndex}' must be true before calling method \texttt{subString(beginIndex,endIndex)}. The contracts that belong to this category are preconditions (as in the example), or postconditions (constraints ensured by the execution of the call) for a method in question. There are also class invariants that capture the constraints for all methods in a particular class.
Temporal contracts \cite{Manna-Pnueli92, 10.1145/1831708.1831723, 10.1145/1595696.1595767, 10.1145/1287624.1287632} 
encode the correct ordering of calls, possibly among multiple APIs. For example, in Python, after creating a \texttt{threading.Lock} object, once a thread makes a call to \texttt{Lock.acquire()}, that thread should  eventually call \texttt{Lock.release()}. 

The notion of contracts in this study is similar to the kinds of contracts described just before this phrase. We have used the same definition of (behavioral and temporal) contracts in this study.
A contract specifies the correct usage of an API and an incorrect usage is a contract violation.


The second category is about \ML~software and its bugs \cite{10.1145/3213846.3213866, islam19, dfaults} 
and bug fixes \cite{8305957, islam20repairing}. 
These works study either the implementation of \ML library APIs or usage information about those APIs. \cite{10.1145/3213846.3213866} and \cite{dfaults} focused on understanding the defects in different ML libraries. The authors (\cite{10.1145/3213846.3213866}) noted that the defect might come from various sources, e.g., program code, execution environment, library framework itself, etc. In contrast, the focus of this study is to gain an understanding of ML API contracts.~\cite{islam19} reported on API misuse. API misuse can be 
detected if contract obligations are specified. \cite{8305957} investigated the issues in various ML libraries to understand the bug-fix patterns in these libraries, whereas \cite{islam20repairing} studied the deep neural network (DNN) models to understand the bug-fix patterns. In our study, we focused on ML API contracts and corresponding breaches. Suppose a user maintains a contract obligation for an ML API. In that case, if the API demonstrates exceptional behavior upon exiting, the issue may be present in the implementation of the API.

Our work focuses on investigating the kinds of contracts required to establish 
the correct usage of ML APIs. The main question is: 
{\em what are the kinds of contracts required to establish the correct usage of 
ML APIs?} 
We observe that
\ML~software is different from traditional software in several ways. 
In \ML~software, problem-solving is largely dependent on training 
data and subject to precise settings of hyper-parameters \cite{10.1145/3213846.3213866}. 
A prior work by \cite{dfaults} suggested that choice of loss function/optimizer, 
missing/redundant/wrong layers, etc. are distinctive bugs in \ML~software. 
Also, incorrect use of \ML~APIs may not always lead to crashes, but may instead lead to 
slower performance or statistically invalid results. In this study, we did not aim to check the reliability of the \ML~systems. Instead, we looked at the errors occurring in \ML~programs due to the incorrect usage of \ML~APIs.

We studied four popular \ML~libraries: 
\tf,~\scikit,~\keras and~\torch~and studied posts from the Q\&A forum 
\emph{Stack Overflow} (\SO)~that contain one of these libraries in a tag.
The dataset (labeled \SO posts, queries, source codes, etc.) generated during our study are available in the \emph {figshare} repository, \url{https://figshare.com/s/c288c02598a417a434df}. 
This dataset includes a total of 1565 posts, from which we manually curated posts that 
hold 413  contracts for relevant \ML~APIs. 
We use this data to answer the following research questions:
 
\noindent
{\textbf{RQ1 (Root Cause and Effect):}} What are the root causes and effects behind \ML~contract violations? \\
{\textbf{RQ2 (Patterns):}}  Are there common patterns of \ML~contract violations? 
\\
{\textbf{RQ3 (Contract Comprehension Challenges):}}  When does understanding \ML contracts require an advanced level of \ML software expertise? \\
{\textbf{RQ4 (Contract Violation Detection):}}  Can checking contracts at the API level help detect the violation in early \ML~pipeline stages?

These questions, and the data that support their answers, help to answer the main question, i.e., they enable researchers and practitioners to pinpoint where immediate support is required in terms of contracts for \ML~APIs. The key findings from our study are summarized in Table~\ref{tab:claim}.
\begin{table}[htbp!]
	\centering
	\caption{Findings and Insights}
	\setlength{\tabcolsep}{3.8pt}
	\scriptsize
	\label{tab:claim}
	\begin{tabular}{|l|p{52mm}||p{52mm}||}
		\hline
		{\cellcolor{Gray} \bfseries RQ} & {\cellcolor{Gray} \bfseries Findings} & {\cellcolor{Gray} \bfseries Actionable Insight} \\ \hline
		RQ1 & Most frequent contracts for \ML~APIs: (\S{\ref{par:IC-1} })
		\begin{enumerate}[leftmargin=*]
			\item Constraint check on single arguments of an API.
			\item Order of API calls that become a requirement eventually.
		\end{enumerate}  & This is a good news because 
		the software engineering (SE) community can employ some existing contract mining approaches 
		to also mine contracts for ML APIs; but there might be a need to combine
		behavioral and temporal contract mining approaches that have 
		been independently developed thus far. \\ \hline
			RQ4 & \ML~API contracts that are commonly violated occur in earlier ML pipeline stages (\S{\ref{par:eps}}). & 
		A verification system with \ML~contract knowledge can explain whether a bug 
		in the \ML~system that used those APIs stemmed from an API contract breach. \\ \hline %for those APIs. \\ \hline
		RQ3 & The absence of precise error messages (\S{\ref{par:Error}}) due to system failures 
		makes contract comprehension and violation detection more challenging. & 
		As domain experts can understand the challenging ML contracts (\S{\ref{sec:dcc}}), 
		this knowledge encoded as contracts can enable improved debugging mechanisms. \\ \hline
		RQ1 & \ML~APIs require several type checking contracts specific to \ML~(\S{\ref{par:ML}}) 
		and inter-dependency (Table \ref{tab:socontract1}) between behavioral and temporal contracts. & 
		Programming methodology and tools for design by contract
		should include sufficient expressiveness for these additional types of contracts seen in \ML~APIs. \\ \hline
	\end{tabular}
\end{table}

The contributions of our paper are the following. 
We provide a taxonomy for \ML~API contracts and corresponding root causes. 
This taxonomy (\S{\ref{subsec:con}}) added five new leaf node categories of 
contracts (with respect to the leaf categories observed in traditional behavioral and temporal contracts) observed in our study. 
The work also identified the stages of ML pipelines in which the violations occur (API contract violation locations) or affect the software and presented a dedicated 
classification (\S{\ref{subsec:vl}}). 
To our knowledge, this is the first work that attempts to understand the types 
of required contracts needed to prevent problems that may arise when using these \ML~APIs in 
software systems. 
In \S{\ref{sec:result}}, in addition to answering the research questions, 
we analyze the outcomes related to contract breaches. 
Finally, we provide recommendations to researchers, consumers, and producers of \ML~APIs based on the findings. 
