%%
% \documentclass[manuscript,screen,review,anonymous]{acmart}
%%% For arXiv
\documentclass[manuscript,nonacm]{acmart}

%% \BibTeX command to typeset BibTeX logo in the docs
\AtBeginDocument{%
  \providecommand\BibTeX{{%
    \normalfont B\kern-0.5em{\scshape i\kern-0.25em b}\kern-0.8em\TeX}}}

%% Rights management information. This information is sent to you
\setcopyright{acmlicensed}
\copyrightyear{2024}
\acmYear{2024}
\acmDOI{XXXXXXX.XXXXXXX}
%
\acmConference[FAccT '24]{2023 ACM Conference on Fairness, Accountability, and Transparency}{June 3--6, 2024}{Rio de Janeiro, Brazil}
%
%  Uncomment \acmBooktitle if th title of the proceedings is different
%  from ``Proceedings of ...''!
%
\acmBooktitle{2024 ACM Conference on Fairness, Accountability, and Transparency (FAccT '23), June 3--6, 2024, Rio de Janeiro, Brazil}

%% Submission ID. This information is sent to you
%%\acmSubmissionID{123-A56-BU3}

%% Packages
\usepackage{algorithm}
\usepackage[noend]{algpseudocode}
\usepackage{orcidlink}
\usepackage{dsfont}
\usepackage{bbold}

%% Other:
\newtheorem{remark}{Remark}
\newcommand{\sr}[1]{\textcolor{orange}{#1}}
\newcommand{\ja}[1]{\textcolor{blue}{#1}}
\newcommand{\am}[1]{\textcolor{purple}{#1}}
\newcommand{\candidatesset}{\mathcal{C}}
\newcommand{\candidatessubset}{\mathcal{D}}
\newcommand\restr[2]{{% we make the whole thing an ordinary symbol
  \left.\kern-\nulldelimiterspace % automatically resize the bar with \right
  #1 % the function
  \littletaller % pretend it's a little taller at normal size
  \right|_{#2} % this is the delimiter
  }}
\newcommand{\littletaller}{\mathchoice{\vphantom{\big|}}{}{}{}}
\newcommand{\besttext}{\text{\textit{best}}}
\newcommand{\goodtext}{\text{\textit{good}}}
\newcommand{\addtext}{\text{\textit{add}}}
\DeclareMathOperator*{\argmax}{arg\,max}
\DeclareMathOperator*{\argmin}{arg\,min}

\newcommand{\ci}{\mathrel{{\scalebox{1.07}{$\perp\mkern-10mu\perp$}}}}
%%
\begin{document}

%%
\title{The Initial Screening Order Problem}

%%
\author{Jose M. Alvarez \orcidlink{0000-0001-9412-9013}}
\email{jose.alvarez@sns.it}
% \orcid{0000-0001-9412-9013}
\affiliation{
  \institution{Scuola Normale Superiore, University of Pisa}
  \city{Pisa}
  \country{Italy}
}

\author{Antonio Mastropietro \orcidlink{0000-0002-8823-0163}}
% \orcid{0000-0002-8823-0163}
\affiliation{
  \institution{University of Pisa}
  \city{Pisa}
  \country{Italy}
}

\author{Salvatore Ruggieri \orcidlink{0000-0002-1917-6087}}
\email{salvatore.ruggieri@unipi.it}
% \orcid{0000-0002-1917-6087}
\affiliation{
  \institution{University of Pisa}
  \city{Pisa}
  \country{Italy}
}

%%
\renewcommand{\shortauthors}{Alvarez, Mastropietro, and Ruggieri}

%%
\begin{abstract}
    We investigate the role of the initial screening order (ISO) in candidate screening processes, such as hiring and academic admissions. ISO refers to the order in which the screener sorts the candidate pool before the evaluation. It has been largely overlooked in the literature, despite its potential impact on the optimality and fairness of the chosen set, especially under a human screener. We define two problem formulations: best-$k$, where the screener chooses the $k$ best candidates, and good-$k$, where the screener chooses the first $k$ good-enough candidates. To study the impact of ISO, we introduce a human-like screener and compare to its algorithmic counterpart. The human-like screener is conceived to be inconsistent over time due to fatigue. Our analysis shows that the ISO under a human-like screener hinders individual fairness despite meeting group level fairness. This is due to the position bias, where a candidate's evaluation is affected by its position within ISO. We report extensive simulated experiments exploring the parameters of the problem formulations both for algorithmic and human-like screeners. This work is motivated by a real world candidate screening problem studied in collaboration with a large European company. 
\end{abstract}

%% The code below is generated by the tool at http://dl.acm.org/ccs.cfm.
\begin{CCSXML}
<ccs2012>
   <concept>
       <concept_id>10003120.10003123.10011758</concept_id>
       <concept_desc>Human-centered computing~Interaction design theory, concepts and paradigms</concept_desc>
       <concept_significance>500</concept_significance>
       </concept>
   <concept>
       <concept_id>10002951.10003317.10003338</concept_id>
       <concept_desc>Information systems~Retrieval models and ranking</concept_desc>
       <concept_significance>500</concept_significance>
       </concept>
 </ccs2012>
\end{CCSXML}

\ccsdesc[500]{Information systems~Retrieval models and ranking}
\ccsdesc[500]{Human-centered computing~Interaction design theory, concepts and paradigms}

%%
\keywords{fair set selection, position bias, candidate screening, human-in-the-loop}

%%% For arXiv
%%
% \received{}
% \received[revised]{}
% \received[accepted]{}

%%
\maketitle

\section{Introduction}
%%%%%%%%%%%%%%%%%%%%%%%%%%%%%%%%%%%%%%%%%%%%%%%%%%%%%%%%%%%%%%%%%%%%%%%%%%%%%%%%
\section{Introduction}

Autonomous driving (AD) %with deep learning networks 
has shown promising achievements and is considered an important technological breakthrough that could revolutionize the future of transportation. Currently, ensuring the safety of autonomous driving systems has become a topic of extensive development.
% There has been much discussion on how to verify the safety of autonomous driving systems.
One traditional solution for safety tests is to exhaustively enumerate real scenarios for validation. Nevertheless, this process is not only labor-intensive and costly but also dangerous. Simulation has emerged as a robust, safe, and efficient alternative for training and evaluating AD software and algorithms~\cite{li2019aads, amini2020learning, amini2022vista}.

% Figure environment removed

Recently, neural radiance field (NeRF)~\cite{mildenhall2020nerf} has gained significant attention in AD simulation~\cite{drivesim}. This approach leverages multi-view images to construct a 3D scene and enable novel view synthesis for both indoor and outdoor applications. When it comes to constructing NeRF models in AD simulation, there are two options available: 1) collecting a large amount of data to cover as many viewpoints as possible, and constructing a fine-grained scene offline; 2) directly using log data from road tests to quickly create an environment and dynamically simulate driving scenarios. The first choice can deliver high-quality simulation~\cite{tancik2022block} by transforming the problem of view extrapolation into view interpolation through the use of large amounts of data. However, it is time- and cost-intensive, which makes it challenging to generalize. As for the second choice, the collected images from log data are usually similar to each other along the running trajectory, which may result in unsatisfactory outcomes, particularly when the camera pose is placed out-of-trajectory (see \figref{figSupportComp} as an example), semantic consistency cannot be guaranteed when synthesizing images from deviated views. We observe this problem under this data condition in all neural radiance approaches, and to the best of our knowledge, none of the existing work has solved this issue.
In our opinion, semantic consistency is crucial for AD simulation, and synthesizing on deviated views is unavoidable for scalability.

AD simulation usually involves map data for planning and control, which can be obtained from a prebuilt High-Definition Map (HD Map) or an online mapping module. While the map data may not be pixel-perfect, it can provide semantic-level information that is useful for enhancing the semantic consistency of the trained neural radiance field.
In this paper, we propose incorporating map priors into neural radiance fields to enhance the semantic consistency and rendering quality of deviated driving view synthesis. Firstly, we employ ground information from maps to supervise the density field of NeRF, providing a more reliable road base for semantic entities. Next, we propose sampling rays to simulate unseen views. Unlike most NeRF augmentation methods~\cite{zhang2022ray, chen2022geoaug}, we utilize ground and lane information in sampling computations to guide the radiance field. More importantly, we model the above two supervision methods as weak supervision by using an uncertainty parameter and propose an uncertainty tempering scheme to increase the uncertainty. This ensures that map priors only guide the training process rather than enforce it towards their absolute values. As a result, our proposed method not only improves the rendering quality of interpolated novel view synthesis quantitatively but also enhances the semantic consistency of deviated novel view synthesis. 
Our contributions can be summarized as follows:
% We summarize the contributions of this paper as follows.



% To overcome the limitations of the collected data, this paper proposes a novel approach that leverages map information to enhance the semantic consistency of the synthesized driving views. 

% Autonomous driving (AD) vehicles are being trained with the help of deep learning networks and have shown promising achievements. This technology is considered to be a breakthrough that could change the way of transportation in the near future. However, there are many discussions on how to verify or judge the safety of autonomous driving systems.
% A straightforward solution towards the safety tests is to exhaustively enumerate real scenarios for validation as many as possible. However, the process of implementing different real scenarios is not only labor-intensive and costly, but also dangerous. Simulation has been proved to be an alternative, which is robust, safe, efficient in training, and evaluating AD software and algorithms.
% Now, the emerging technology of neural radiance field (NeRF)~\cite{} leverages multi-view images to construct a 3D scene and enable novel view synthesis for many indoor and outdoor applications. For AD simulation, there are two choices for constructing NeRF models: 1) collect a large amount of data, such as LiDAR and camera data, similar to mapping, to construct a fine-grained scene offline; or 2) directly use the log file (typically in the format of ROS bag) to rapidly create an environment and then dynamically simulate the driving scenarios.
% The first choice can achieve high-quality simulation, but it is time-consuming and expensive, making it difficult to generalize to very large scales. On the other hand, the second option is fast but can lead to low-quality simulation due to the data being sparse and similar to each other in log data. This paper tackles the problem raised by choosing the latter option and attempts to improve the quality of out-of-trajectory driving view synthesis by incorporating map information. This approach is practical for many autonomous driving tests.
% In conclusion, the use of NeRF technology for AD simulation is a promising avenue for training and evaluating AD software and algorithms. While both options for constructing NeRF models have their pros and cons, this paper addresses the challenges of the second option and proposes a potential solution to improve the quality of simulation.

%There exist a few attempts to facilitate training a NeRF model for synthesizing out-of-trajectory (or called as extrapo trajectory) views.


\begin{itemize}
    \item We propose a novel method to incorporate commonly used map priors in AD scenes into neural radiance fields to improve the out-of-trajectory driving view synthesis.
    \item We explicitly model the uncertainty in map priors as a parameter and propose an uncertainty tempering scheme to guide the training process of the neural radiance field.
    \item Experiments demonstrated that the proposed method can improve the semantic consistency of out-of-trajectory views and the rendering quality of novel view trajectory interpolation.
\end{itemize}

Our proposed method is easy to implement, can be easily plugged into existing NeRF algorithms, and has the capability of extending to other formats of priors.

\section{Qualitative Background}
%
\label{sec:Generali}

The initial screening order problem is based on a collaborative work at an European Fortune Global 500. We refer to this company as G. This section summarizes our experience at G and how it motivated the problem. 

The goal of the collaboration was to study G's hiring process from an algorithmic fairness perspective. We worked closely with the teams of Advanced Analytics (AA) and Human Resources (HR), focusing on the candidate screening problem where an HR officer or screener selects a number of candidates based on their profiles. We mostly interviewed the HR officers to understand their tasks, constraints, and methodologies, often shadowing them during screening sessions. We concluded with a report to AA that formalized G's candidate screening problem as a ranking problem, evaluated the potential fairness implications, and assessed the risk and benefits of automation.

\subsection{The Hiring Pipeline}
\label{sec:Generali:HiringPipeline}

Hiring at company G consists broadly of three phases. In \textit{phase one}, HR builds a large enough candidate pool for a job opening. Here, ``large enough'' depended on the seniority and priority of the job opening. For a very senior position like Data Science Director, e.g., the candidate pool was often small (25 candidates max) as the amount of suitable candidates active in the job market was small throughout the year. The job opening includes a description of the ideal candidate provided by the hiring manager where candidates apply and/or are recruited accordingly. Candidates submit their CVs, complete relevant information in the form of multiple-choice questions, and occasionally provide a motivation letter. Sensitive information like a candidate's gender is always provided. The candidate pool is stored in Taleo\footnote{For more information, visit \url{https://www.oracle.com/human-capital-management/taleo/}.}, a database query platform for hiring by Oracle.

In \textit{phase two}, which covers the candidate screening problem, HR reduces the candidate pool into a smaller pool of suitable candidates to be invited for the interviews that take place in the next phase. The assigned HR officer or screener determines candidate suitability using a set of minimum basic requirements that all candidates must meet according to the hiring manager. Suitable candidates do not necessarily have to be ideal candidates. The existence of a third phase allows G to verify and update the information provided in each candidate's profile. The set of minimum basic requirements thus represents the non-negotiables for the hiring manager as well as a guide for flagging suitable candidates for the screener. 

Finally in \textit{phase three}, HR along with the hiring manager examine the subset of selected candidates from phase two across a series of interviews and aptitude tests. There is a first interview between HR and the candidates to asses the information on paper. If positive, candidates are interviewed and evaluated by members from the team offering the job opening. If positive, then the candidates at last meet the hiring manager (or partner) for the final call. At each interview, candidates are graded and their scores discussed among the interviewers. The grading is recorded in Taleo. The best candidates receive an offer, which they can accept or decline. If there is no match between the top candidates and the job opening, HR goes back to the runner-up candidates that remained in phase two and repeat phase three. 

\subsection{Candidate Screening}
\label{sec:Generali:CandidateScreening}

The choice to focus on phase two of the hiring pipeline was motivated by how appealing and dangerous it appeared for automation, offering an interesting tension between a time-consuming, repetitive task prone to human error and a high-risk, sensitive task requiring human oversight.\footnote{Automation seemed less appealing/reasonable for the other two phases: phase one was already streamlined with Taleo while phase three was clearly too human-dependent. This opinion was shared by G too.} When screening multiple candidates, HR officers clearly faced an overflow of information that had to be processed quickly. Phase two was crucial. It was not humanly possible, at least for a company of the size of G, to interview all candidates that applied, especially when we considered that HR ran multiple job openings at once. Given the nature of the task---that of extracting information from the candidate profiles and deciding whether it met a set of minimum basic requirements---the idea of developing a screening algorithm was, in principle, an appealing option to AA, HR, and us. 

Early on in the collaboration, however, it became clear to all stakeholders involved how human intensive and complex the candidate screening process was. It seemed unlikely that such process could be automated at all using the available data and available tools. It seemed even more unlikely when we considered the many unsuccessful AI-tools deployed in other companies (see, e.g., \cite{HiredByAlgoMITPodcast,TheAIWillSeeUMITPodcast}). Despite the known evidence on biased and inefficient human decision-making in similar settings to phase two (e.g., \cite{Kahneman2016Noise, Miller2018Bias, Kahneman2021Noise}), the goal of automation increasingly lost its appeal: the focus shifted away from the future algorithmic screener onto the present human screener. At best, we argued then and continue to do so, we could develop an algorithm to aid the human screener but never to replace it.  

This tension around automating phase two made us study closer the processes of the human screener. If the long-term goal was to create an algorithm that could aid the human screener, we first needed to identify instances where the human screener was likely to resort to biased (read, inconsistent) decision-making due to, say, cognitive heuristics (e.g., \cite{tversky_judgment_1974, Kahneman2011Thinking}). Hence, our focus on a \textit{human-like screener} throughout this paper. In Sections \ref{sec:ProblemFormulation} and \ref{sec:ISO} we formalize such screener and illustrate its role within the initial order screening problem.
We highlight three aspects that stood-out about phase two, which inspired the formulation of the initial screening order problem:
%
\begin{itemize}
    \item[(G1)] \textit{A varying initial order of the candidate pool.} As a default, Taleo sorted the candidate pool in alphabetical order. Each HR officer, however, was free to choose how to sort the candidate pool. The choice was restricted by the sorting fields offered by Taleo. One HR officer, e.g., preferred to sort the candidates using the submission date of the applications. The HR officers were consistent with their choice of initial screening order through multiple screening procedures.
    
    \item[(G2)] \textit{Multiple ways of searching the candidate pool.} Each HR officer decided how to search the candidate pool. What mattered is that they reached the quota of suitable candidates within a reasonable time. One HR officer, e.g., preferred to search the whole candidate pool while another stopped searching once meeting the quota. It took them one minute per profile. The HR officers were consistent with their choice of search procedure.
    
    \item[(G3)] \textit{Candidate meets minimum basic requirements.} Although the HR officer imposed an implicit order among candidates when screening the candidate pool, what mattered for those selected was that they met the set of minimum basic requirements as provided by the hiring manager. Again, the existence of the third phase allowed the HR officer to focus on getting enough suitable candidates on paper for an interview instead of finding the best possible candidates on paper for the job opening. This meant that order within the list of selected candidates was not necessarily important.
\end{itemize}
%

%
\begin{example}[Junior Data Scientist for Hire]
\label{RunningExample}
    As an illustrative example, suppose G wants to hire three junior data scientists. The hiring manager provides a list of qualifications describing the ideal candidate: a good command of English (B1 or above), a technical background (Engineering, Physics, or equivalent), Python programming skills (command at least of the \texttt{numpy}, \texttt{pandas}, and \texttt{scikit-learn} packages), and familiarity with EU banking regulation models (Basel II and III). The hiring manager sets as the minimum basic requirement that candidates meet two of these qualifications.
\end{example}
%

In job openings similar to that of Example~\ref{RunningExample}, a single HR officer usually had to select 15 to 30 candidates from a pool of 500 within a couple of hours. The HR officer would screen through each candidate profile, cheeking for the minimum basic requirements. A candidate with a B2 English and an Engineering degree but no programming skills in Python, e.g., would be considered suitable as candidate. A candidate with a B1 English but a Philosophy degree and no mention of Python skills would not be considered a suitable candidate. We stress that it is not about \textit{how suitable} the candidate is, but about whether the candidate is \textit{suitable or not}. 

\subsection{Fairness Goals}
\label{sec:Generali:Fairness}

We note that G, as it is common today with companies' efforts to have a diverse workforce, already had in place fairness goals in the form of \textit{representation quotas} for each job opening. These were, in fact, considered as key performance indicators, or KPIs, for the company. The main protected (or sensitive) attribute considered for such KPIs was gender. Throughout the hiring pipeline, when possible, HR would aim to reach a balanced list of candidates, meaning 50\% male and 50\% female candidates. 

In phase two, HR would aim to select a balanced subset of suitable candidates from the candidate pool. In practice, this meant imposing \textit{quotas} and \textit{threshold policies} with the goal of insuring some levels of female representation (or, in general, the under-represented group, which was always the protected group). HR, e.g., would apply different sets of minimum basic requirements to male and female candidates to meet certain representation goals specific to phase two. Such procedures are important, as we will see in the next sections, for the formulation of the initial screening order. We stress that G had well-established, working fairness policies already in place. Hence, what was of interest to us was understanding the role of the human-like screener in reaching these fairness goals.

%
% EOS
%


\section{Problem Formulation and Search Procedures}
%
\label{sec:ProblemFormulation}

We now describe our setup in its most basic form. The goal is to formulate the set selection problem, where a decision-maker selects a set of items from a population, by considering the initial order in which the items are presented to the decision-maker. Here, the candidates for a job position represent the items and the screener evaluating their profiles represents the decision-maker.  

\subsection{Setting}
\label{sec:ProblemFormulation.Setting}

% The candidate pool
We consider a \textit{candidate pool} $\mathcal{C}$ of $n$ candidates. 
Each \textit{candidate} $c$ is described by the \textit{vector of $p$ attributes} $\mathbf{X}_c \in \mathbb{R}^{p}$ and the \textit{protected attribute} $W_c$. 
To simplify our analysis, we assume that $W$ encodes the membership to the protected group and is, thus, binary: with $W_c = 1$ if $c$ belongs to the protected group and $W_c = 0$ otherwise. 
The underlying protected attribute can be discrete, continuous, or it may consists of multiple attributes.

% The screener
The candidates are evaluated by a \textit{screener} $h \in \mathcal{H}$, where $\mathcal{H}$ denotes the set of available screeners. 
The following variables refer to a specific $h$.
The screener is tasked with choosing $k$ candidates from $\mathcal{C}$ based on each candidate's application profile as summarized by the tuple $(\mathbf{X}_c, W_c)$. 
The goal of the screener is to obtain a \textit{set of $k$ selected candidates} $S^k \in [\mathcal{C}]^k$, where $[\mathcal{C}]^k$ denotes the set of $k$-subsets of $\mathcal{C}$, i.e.,~subsets with cardinality $k$.
We model candidate evaluation by assuming that the screener uses an \textit{individual scoring function} $s \colon \mathbb{R}^{p} \to [0, 1]$, such that, given a candidate $c$, $s(\mathbf{X}_c)$ returns the score of $c$ according to $h$. 
It is implied that the screener does not use the protected attribute when scoring candidates. 
The higher the score, the better the candidate fits the job position based on the screener's judgment. 

% All possible orders and the initial order
The screener explores the candidate pool $\mathcal{C}$ in a specific order.
% when evaluating the candidates. 
We denote the set of total ordering of candidates in $\candidatesset$ by $\Theta$. 
It represents all possible ways in which the $n$ candidates in $\candidatesset$ can be arranged. 
An \textit{order} $\sigma \in \Theta$ maps an integer $i \in \{1, \dots n\}$ to a candidate $c \in \candidatesset$, indicating that $c$ occupies the $i$-th position according to $\sigma$, with notation $\sigma(i) = c$ and vice-versa $\sigma^{-1}(c) = i$. 
We are interested in the \textit{initial order} $\theta \in \Theta$, which refers to the order chosen by the screener before starting the exploration of $\candidatesset$ and the evaluation of their scores, according to fact G1 in Section~\ref{sec:Generali}. 
The screener is not required to explore the entirety of the candidate pool $\mathcal{C}$, according to fact G2 in Section~\ref{sec:Generali}.
However, we assume that the screener \textit{respects} the initial order $\theta$. 
Formally:
%
\begin{equation}
\label{eq:order}
    \mbox{a candidate $c_1 \in \candidatesset$ is evaluated before $c_2 \in \candidatesset$ only if $\theta^{-1}(c_1) < \theta^{-1}(c_2)$}
\end{equation}
%

% On meaningfulness
%\textcolor{red}{
%We say that $\theta$ is non-informative to $h$. 
%We assume that only after choosing $\theta$, and subsequently having searched, evaluated, and sorted the entirety of $\mathcal{C}$ as prescribed by $\theta$, can $h$ be aware of any meaning behind $\theta$. 
%This assumption ensures that $h$ does not knowingly introduce bias by arranging $\candidatesset$ into a specific $\theta$. 
%}
% All initial orders are meaningless to the set of screeners $\mathcal{H}$.
% On m steps travelled 
%The screener is not required to explore the entirety of the candidate pool $\mathcal{C}$.\footnote{This condition is based on fact G2 in Section~\ref{sec:Generali}.} 
%We denote as $\candidatessubset \subseteq \candidatesset$, the \textit{subset of candidates whose profiles are screened by} $h$, with cardinality $m \leq n$, implying the case where $h$ prefers not to evaluate the complete $\candidatesset$.
%For $\theta \in \Theta$, we refer to $\theta^{(m)}$ as \textit{the total order restricted to first $m$ elements according to the initial order}, so that $\theta^{(m)}$ maps an integer $i \in \{1, \dots m\}$ to a candidate $c \in \candidatessubset$.
% For a given $\sigma \in \Theta$, we refer to $\sigma^{(m)}$ as the total order restricted to the first $m$ elements, so that $\sigma^{(m)}$ maps an integer $i \in \{1, \dots m\}$ to a candidate $c \in \candidatesset_h$.

\subsection{Group Fairness}
\label{sec:ProblemFormulation.Fairness}

We address the fairness goals of the screener $h$ in choosing the set of $k$ candidates by assuming a quota for the protected group $W = 1$. 
Let us introduce the fraction $f\big( S^k \big) \in [0, 1]$ of protected candidates in the selected set $S^k$:
%
\begin{equation}
\label{eq:fairness_function}
    f\big( S^k \big) = \frac{\left\vert\{c \in S^k \text{ s.t. } W_c = 1\}\right\vert}{k}
\end{equation}

% \begin{equation}
% \label{eq:fairness_function}
%     f\big( S^k \big) = \left\vert\{c \in S^k \text{ s.t. } W_c = 1\}\right\vert / k
% \end{equation}
%
Let $q \in [0, 1]$ denote the desired fraction of chosen protected candidates in $S^k$. 
The fair screener will then be constrained to meet the \textit{representational quota} $q$ when deriving $S^k$, i.e.,~to satisfy the condition $f\big( S^k \big) \geq q$. 
Notice that the unconstrained version, in which no requirement is assumed on the representativeness of the protected group in $S^k$, is simply achieved by considering $q=0$.

We view $q$ as a policy implemented by the screener to achieve a diverse set of selected candidates. Hence, it is 
% We stress that $q$ is
a statement on the composition of $S^k$, not a statement on the ordering of protected candidates within $S^k$, according to fact G4 in Section~\ref{sec:Generali}.
% \footnote{Here, $q$ is based on fact G4 in Section~\ref{sec:Generali} as G's HR officers were motivated by company diversity goals, not by the fairness literature.} 
For $k=10$ and $q=0.5$, e.g., the fair screener would need to derive $S^k$ with $50\%$ of protected candidates though in no particular order, such as requiring the first five candidates in $S^k$ to be protected candidates.


\subsection{Set Selection: Two Problem Formulations}
\label{sec:ProblemFormulation.Objectives}

We now proceed to formulate two set selection problems for the screener $h$ tasked with deriving the set $S^k \subseteq \candidatesset$, where 
we distinguish between best-$k$ and good-$k$. 
Under the \textit{best-}$k$ formulation, the set $S^k$ represents \textit{the fair best $k$ candidates} in $\candidatesset$ according to $h$. 
We denote it accordingly as $S^k_{\besttext}$. 
% This is the standard formulation in the fair ranking literature~\cite{Zehlike2023_FairRanking_P1, Zehlike2023_FairRanking_P2}. 
Under the \textit{good-}$k$ formulation, the set $S^k$ represents \textit{the fair first good-enough $k$ candidates} in $\candidatesset$ according to $h$. 
We denote it accordingly as $S^k_{\goodtext}$. 
% This formulation, to the best of our knowledge, has been overlooked by the fair ranking literature.
For both formulations, we define the objective of the screener in terms of achieving an optimal and fair selection of candidates. 
How we define optimality, as we will show, derives the best-$k$ and good-$k$ formulations; we already defined fairness in the previous subsection. 

\paragraph{Best-$k$.}
We first focus on the screener that finds the set of best $k$ candidates in the candidate pool $\candidatesset$ given the fairness constraint $q$
and while respecting the initial order $\theta$ (recall (\ref{eq:order})).
%Noteworthy, here $h$ has to search the whole $\candidatesset$, meaning $\am{\candidatessubset} = \candidatesset$ and $m=n$ in this setting.
Notice that, here, $h$ must evaluate the complete $\candidatesset$.
This is because $h$ must score all candidates according to the individual scoring function $s$ before choosing the ones with the highest scores and for which $q$ is satisfied.
%meaning $h$ evaluates each candidate and sorts them by their scores, in order to speak of the best candidates.
%
%% SR
%Formally, the ranking $\tau \in \Theta$ represents an ordering of $\candidatesset$ that follows the scoring function $s$, such that, if $i \leq j$, $s(\mathbf{X}_{\tau(i)}) \geq s(\mathbf{X}_{\tau(j)})$.
%Given $\tau$, the cut-off for being selected into $S^k$ is $s(X_{\tau(k)})$ with $\tau(k)$ denoting the $k^{th}$ position in the ranking $\tau$.
%We can then model the goal of the best-$k$ selection, without fairness requirements, as:
%
%\begin{equation}
%\label{eq:objective_best-k}
%    S_{\tau}^k = \{\tau(1), \dots, \tau(k)\}
%\end{equation}
% 
We view the goal in terms of maximizing a utility for the screener $h$. 
We define \textit{utility} as the benefit derived by $h$ from selecting $k$ suitable candidates  given $\theta$.
% We define \textit{utility} as the benefit derived by $h$ from selecting $k$ suitable candidates for the job while respecting $\theta$ (recall (\ref{eq:order})). 
Formally, utility is a function $U^k \colon [\mathcal{C}]^k \, \times \, \Theta \to \mathbb{R}$. 
%Note that, in general, the utility function evaluated on a subset $S^k$ of candidates is influenced by the initial order $\theta$ by which the same candidates are seen. 
The simplest expression for defining $U^k$ is to add the scores of the chosen candidates:
%
\begin{equation*}
\label{eq:Utility}
    U^k_{\addtext} \big( S^k, \theta \big) = \sum_{c \in S^{k}} s\big( \mathbf{X}_{c} \big)
\end{equation*}
%
rationalizing that $h$ will maximize its utility by selecting the $k$ most suitable candidates for the job position. 
%It is easy to see that $S_{\tau}^k$ maximizes the utility $U^k_{\text{add}}$.
Note that
% , in the above utility definition, 
the initial order $\theta$ in \eqref{eq:Utility} does not affect the evaluation of $S^k$ because of the commutative property of addition.
We emphasize that \eqref{eq:Utility} is not the only possible model for the utility of the screener. 
Alternative models, such as exposure discounting \cite{DBLP:conf/kdd/SinghJ18}, can be considered for the best-$k$ problem formulation.
We leave this for future work.
%The goal \eqref{eq:objective_best-k} and utility \eqref{eq:Utility} describe the best-$k$ set selection problem for $h$. 
%If the screener $h$ is fair, then it must meet the representational quota $q$ of the protected group \eqref{eq:fairness_function}. 
 
We define \textit{the fair best-$k$ set selection problem} (or, simply, the best-$k$ problem) as:
%
\begin{equation}
\label{eq:fair_objective_all_screener}
    \begin{aligned}
    \argmax_{S^k \in [\mathcal{C}]^k} & \quad U^k_{\addtext} \big(S^k, \theta\big) \\
    \textrm{s. t.} & \quad f(S^k) \geq q
    \end{aligned}
\end{equation}
%
which, as readers familiar with the literature might notice, describes the standard top-$k$ formulation in fair ranking problems with $q$ representing some group-level fairness quota \cite{Zehlike2023_FairRanking_P1, Zehlike2023_FairRanking_P2}. 
We denote the
% \footnote{In presence of ties in scores, the solution may not be unique. In such a case, we consider any solution.} 
solution of \eqref{eq:fair_objective_all_screener} as $S^k_{\besttext}$.
In presence of ties in scores, the solution may not be unique. 
In such a case, we consider any solution.

\paragraph{Good-$k$.}
We now focus on the screener $h$ that finds $k$ candidates in the candidate pool $\candidatesset$ that meet a \textit{minimum basic requirements} $\psi$ given the fairness constraint $q$ and while respecting the initial order $\theta$.
% \footnote{This setting is based on facts G2 and G3 in Section~\ref{sec:Generali}.}
Unlike the best-$k$ formulation, here $h$ is not required to evaluate the whole $\candidatesset$ as it is enough to find the first $k$ candidates that are good-enough according to $\psi$ and that satisfy $q$. 
%Hence, the goal \eqref{eq:objective_best-k} and utility \eqref{eq:Utility} from the standard set selection formulation are not suitable in this setting.
%
We represent $\psi$ as a minimum score, such that $h$ deems candidate $c \in \candidatesset$ as eligible, or good-enough, for being selected if $s(\mathbf{X}_c) \geq \psi$.
%In the setting without fairness, unlike the goal of best-$k$, $S_{\tau}^k$ \eqref{eq:objective_best-k}, we cannot speak of a cut-off for the goal of good-$k$, because there is no ranking $\tau$ to slice from.

Under the good-$k$ setting, clearly the initial order $\theta \in \Theta$ has a significant influence on the screening process. 
To observe this point, let $k=1$ and assume $s(\mathbf{X}_{c_1}) \geq \psi$ and $s(\mathbf{X}_{c_2}) \geq \psi$. An initial order such that $\theta^{-1}(c_1) = 1$ and $\theta^{-1}(c_2) = 2$ would imply by (\ref{eq:order}) that $c_1$ is considered eligible before even evaluating $c_2$. Conversely, the reverse initial order $\theta^{-1}(c_1) = 2$ and $\theta^{-1}(c_2) = 1$ would consider eligible $c_2$ before $c_1$. Since the screener will stop after finding $k=1$ good enough candidates, the initial order affects which candidates are selected.

%let us call $\mathcal{C}_{\psi}$ the set of eligible candidates according to the minimum basic requirements $\psi$. 
%Evidently, we need to assume that $|\mathcal{C}_{\psi}| \geq k$.
%It is useful to denote $\Theta_{\psi}$ as the set of total orderings of $\mathcal{C}_{\psi}$.
%Under this view, $\theta$ induces a total order $\theta_{\psi} \in \Theta_{\psi}$ such that, given two candidates $c', c'' \in \candidatesset_{\psi}$, then $\theta_{\psi}^{-1}(c') < \theta_{\psi}^{-1}(c'')$ if $\theta^{-1}(c') < \theta^{-1}(c'')$.
%In this case, without the fairness condition, it helps to indicate the subset of $\candidatesset_{\psi}$ containing the first $k$ suitable candidates according to $\theta_{\psi}$ as $S^k_{\psi}$.
%Formally:
%
%\begin{equation}
%\label{eq:S_k_psi}
%    S^k_{\psi} = \{\theta_{\psi}^{-1}(i) \mid i = 1, \dots, k\}
%\end{equation}
%
%It is possible for two screeners $h_1$ and $h_2$ that chose different initial orders $\theta_1$ and $\theta_2$, under the same $s$ and $\psi$, to obtain different selected sets. 
%
%For this last scenario to hold, though, it must be unappealing, utility-wise, for both $h_1$ and $h_2$ to search the entirety of $\candidatesset$ according to $\theta_1$ or $\theta_2$.
%The notion of any $h$ optimally choosing $k$ candidates acquires a different meaning in the good-$k$ setting since the utility function $U^k_{\addtext}$ as described in \eqref{eq:Utility} is inadequate to formalize a screener for whom a selected set of $k$ good enough candidates is sufficient. 
%We define the simplest expression for the utility $U^k_{\psi}$ motivating the screener in the good-$k$ setting:
We still view the goal in terms of maximizing a utility for the screener $h$.
As a possible utility function in the good-$k$ setting, we first consider:
%
\begin{equation*}
\label{eq:AlternativeUtility2}
    \hat{U}^k_{\psi}\big( S^k, \theta \big) = \left\{
    \begin{array}{ll}
        n - \max_{c \in S^k} \theta^{-1}(c) & \text{if} \  \forall c \in S^k \  s(\mathbf{X}_c) \geq \psi   \\
        0 & \text{otherwise.}
    \end{array} \right.
\end{equation*}
%
Intuitively, in the above definition, the screener wants to find as quickly as possible a set of $k$ eligible candidates.
Therefore, if $S^k$ contains only eligible candidates, the utility of $h$ selecting $S^k$ under $\theta$ is expressed by the number of candidates past the last one who was screened, i.e.,~the ``saved effort" of the screener $h$.

Despite the simplicity of the above definition, the utility function (\ref{eq:AlternativeUtility2}) is not suitable to properly model our intended problem.
To observe this point, let $n=3, k=2, q=0.5$. 
Assume three eligible candidates and $\theta(1) = c_1, \theta(2) = c_2, \theta(3) = c_3$ with $W_1 = W_2 = 0$ and $W_3 = 1$. It turns out that both $S' = \{c_1, c_3\}$ and $S'' = \{c_2, c_3\}$ maximize the utility and satisfy the fairness constraint. However, why should have been $c_2$ considered, and then returned in $S''$, if $c_1$ already meets the minimum basic requirement? 
A reason for doing that is a variant of our problem, in which the screener keeps evaluating non-protected candidates, even if their quota is reached but the one of protected candidates is not yet reached, for the purpose of keeping the best ones found so far. 
We do not consider such a variant in this paper.

Let us define the \textit{penalty function} $p(c, S^k, \theta) = \mathbb{1}(\exists\ c' \in \mathcal{C}\setminus S^k \mbox{s.t.}\ \theta^{-1}(c') < \theta^{-1}(c) \wedge s(\mathbf{X}_{c'}) \geq \psi \wedge W_{c'}=W_c)$ which, for a candidate $c$, looks for another candidate of the same group as $c$ and meeting the minimum basic requirement, who occurs before $c$ in the order $\theta$, but who has not been selected into $S^k$. 
Basically, the penalty function models the ``wasted effort" in choosing a candidate occurring after another one meeting all the same requirements. 
At worst, the are $k$ penalties, which leads to the following refined utility function:
%
\begin{equation*}
\label{eq:AlternativeUtility}
    U^k_{\psi}\big( S^k, \theta \big) = \left\{
    \begin{array}{ll}
        k - \sum_{c \in S^k} p(c, S^k, \theta) & \text{if} \  \forall c \in S^k \  s(\mathbf{X}_c) \geq \psi   \\
        0 & \text{otherwise.}
    \end{array} \right.
\end{equation*}
%
%
%index of the last candidate to be evaluated, according to $\theta$, to complete the set of $k$ eligible candidates.
%The fewer candidates $h$ are evaluated to obtain $S^k$, the more utility the screener gets from the selection.
%Otherwise, if $S^k$ contains at least one non-eligible candidate, the utility of selecting $S^k$ is a zero.
%Evidently, $S^k_{\psi}$ of \eqref{eq:S_k_psi} maximizes the above utility, without including the fairness requirement.
% We come back to this utility function at the end of this section.

%If we include the fairness constraint, we 
We define then \textit{the fair good-$k$ set selection problem} (or, simply, the good-$k$ problem) as:
%
\begin{equation}
\label{eq:fair_objective_U_psi}
    \begin{aligned}
    \argmax_{S^k \in [\mathcal{C}]^k} & \quad U^k_{\psi} \big(S^k, \theta\big) \\
    \textrm{s. t.} & \quad f(S^k) \geq q
    \end{aligned}
\end{equation}
%
We denote the solution of (\ref{eq:fair_objective_U_psi}) as $S_{\goodtext}^k(\psi)$ or, if there is no ambiguity on $\psi$, simply as $S_{\goodtext}^k$. 
Note that,
if the fairness constraint is strengthened to a fixed quota, i.e.,~$f(S^k) = q$, it can be shown that the solution is unique. 
In the general case, i.e.,~$f(S^k) \geq q$, there can be two solutions, but with different fractions of the protected group. 
%%%v
% JA: if needed, move example to Appendix
%%%
For example, consider $n=3, k=2, q=0.5$. Assume three eligible candidates and $\theta(1) = c_1, \theta(2) = c_2, \theta(3) = c_3$ with $W_1 = 0$ and $W_2 = W_3 = 1$. Both $S' = \{c_1, c_2\}$ and $S'' = \{c_2, c_3\}$ are solutions of (\ref{eq:fair_objective_U_psi}) with $U^k_{\psi} \big(S', \theta\big) = U^k_{\psi} \big(S'', \theta\big) = 2$. However, $f(S') = 0.5$ and $f(S'') = 1$. Intuitively, $S'$ is obtained by strictly iterating over $\theta$, which is the approach we take in Section~\ref{sec:ProblemFormulation.Algorithms}. 
%%%^
% JA: if needed, move example to Appendix
%%%

%$s(\mathbf{X}_{c_1}) \geq \psi$ and $s(\mathbf{X}_{c_2}) \geq \psi$. An initial order were $\theta^{-1}(c_1) = 1$ and $\theta^{-1}(c_2) = 2$ would imply by (\ref{eq:order}) that $c_1$ is considered eligible before even evaluating $c_2$. Conversely, the reverse initial order $\theta^{-1}(c_1) = 2$ and $\theta^{-1}(c_2) = 1$ would consider eligible $c_2$ before $c_1$. Since the screener will stop after finding $k=1$ good enough candidates, the initial order affects which candidates are selected.

%This fact is useful for unifying both good-$k$ and best-$k$ settings into a single problem formulation.
%To highlight how distinct this setting is relative to the standard best-$k$, let us formulate the good-$k$ setting without needing to resort to specifying a goal or utility function.
%We will return to the above utility function at the end of this section.

%We argue that the screener $h$ can settle for looking through a subset $\candidatessubset$ of cardinality $m \leq n$ as long as it contains enough $k$ eligible candidates so that it also holds $m \geq k$. 
%Moreover, $\candidatessubset$ is composed by the set of the first $m$ candidates of the initial order $\theta$, so we can state that $\candidatessubset = \{c \in \candidatesset \mid \theta^{-1}(c) \leq m\}$. 
%Here, $\theta$ induces an ordering of $\candidatessubset$ captured by the initial order restricted to its first $m$ elements $\theta^{(m)}$. 
%Once the selected set has cardinality $k$, the screener has no incentive to look for more eligible candidates according to $\psi$.
%Hence, without the fairness condition, the selected set is the set of the first $k$ eligible candidates in $\mathcal{C}_{\psi}$ following $\theta_{\psi}$.
%Further, if the screener $h$ is fair, it must meet the representational quota $q$ of the protected group that meets the minimum basic requirements.
%We can restate problem \eqref{eq:fair_objective_U_psi} as:
%
%\begin{equation}
%\label{eq:Alternativefair_objective_all_screener}
%\begin{aligned}
%    \min_{m \in [k, \ldots, n]} \quad & m \\
%    \textrm{s. t.} \quad & |\{ \theta_{\psi}(i) \mid i=1, \ldots, m\}| = k \\
%    \quad & f(S_{\goodtext}^k) \geq q
%\end{aligned}
%\end{equation}
%
%where the selected set is exactly $S_{\goodtext}^k = \{ \theta_{\psi}(i) \mid i=1, \ldots, m\}$, so that it necessarily has cardinality $k$. 

%\paragraph{A fair initial screening order problem formulation.}
%To summarize, we represent both best-$k$ and good-$k$ settings into one formulation for the set selection problem based on the initial order $\theta$:
%
%\begin{equation}
%\label{eq:GeneralISOFormulation}
%    \begin{aligned}
%        \max_{S^k \in \mathcal{S}^k} & \quad U^k(S^k, \theta) \\
%        \textrm{s. t.} & \quad f(S^k) \geq q
%    \end{aligned}
%\end{equation}
%
%where we obtain two solutions for the optimal and fair screener $h$ depending on the utility function specification. 
%Under $U^k_{\addtext}$ as by \eqref{eq:Utility}, the solution is $S_\besttext^k$.
%In this case, $\theta$ is redundant as it does not influence $U^k_{\addtext}$. 
%This is because $h$ will always explore all of $\candidatesset$ and sort it as the ranking $\tau \in \candidatesset$ regardless of $\theta$. 
%Still, $h$ explores $\candidatesset$ as prescribed by $\theta$.
%Under $U^k_{\psi}$ as by \eqref{eq:AlternativeUtility}, the solution is $S_\goodtext^k$. 
%In this case, $\theta$ is a key input to $U^k_{\psi}$, despite not being a parameter that can be optimized by $h$. 
%This is because $h$ can partially or fully explore $\candidatesset$ as prescribed by $\theta$ depending on when it meets its goal.
%With \eqref{eq:GeneralISOFormulation} we present the general initial screening order problem formulation.

% Antonio: I've commented out again as we don't use meaningulness before (I changed that also as you suggested)
% For a screener $h \in \mathcal{H}$, the ordering insider the chosen subset $\mathcal{S}^{k}$ can be meaningful. In principle, $S^{k}_{\besttext}$ leads to a meaningful candidate selection relative to $S^{k}_{\goodtext}$ as candidates are ordered by their scores. In practice, however, it depends on how the chosen subset is used later on in the hiring pipeline, meaning whether in the next phase the order within $\mathcal{S}^k$ carries any information (e.g., interviewing candidates as described by $\mathcal{S}^k$). Hence, the chosen set can be meaningful but it is not of interest here as its meaning is determined beyond the screener $h$.

\subsection{Two Search Algorithms}
\label{sec:ProblemFormulation.Algorithms}

We present two search procedures for the best-$k$ and the good-$k$ problems, respectively. 
% Their names refer to the click models literature discussed in Section~\ref{sec:RelatedWork}. 
% \cite{DBLP:conf/clef/GrotovCMSXR15}.

The \textit{ExaminationSearch} procedure, shown in Algorithm~\ref{algo:Examination}, solves the best-$k$ setting, returning $S^k_\besttext$ for given $n$ (candidates), initial order $\theta$, and parameters $k$ (subset size) and $q$ (minimum fraction of selected candidates from the protected group). 
First, line 2 calculates the minimum number $q^*$ of candidates from the protected group to be selected, and the maximum number of candidates $r^*$ not in that quota. 
Then, candidates are considered by descending scores, using the \texttt{argsortdesc} procedure (lines 2, 3). 
The loop in lines 5-13 iterates until $k$ candidates are found. The loop adds candidates to the sets $Q$ and $R$: $Q$ are candidates in the quota of the protected group; $R$ are candidates not in that quota (can be non-protected or protected). An non-protected candidate can be only added to the $R$ set, thus line 7 checks if there is still room in $R$ to do this. A protected candidate is added to the quota set $Q$ if there is room (lines 10-11), or to the other set $R$ otherwise (lines 12-13). 
Finally, the procedure returns the candidates in the quota set $Q$ or in the other set $R$. 
The result of the \textit{ExaminationSearch} procedure clearly maximize (\ref{eq:fair_objective_all_screener}), as candidates are added in decreasing score, while keeping the fairness constraint through the quota management.

The \textit{CascadeSearch} procedure, shown in Algorithm~\ref{algo:Cascade}, solves the good-$k$ setting, returning $S^k_\goodtext$ for given $n$, initial order $\theta$, and parameters $k$, $q$, and $\psi$ (minimum basic requirement). 
The difference with the \textit{ExaminationSearch} procedure consists in strictly following the initial order $\theta$ (line 4), and in checking the minimum basic requirement (line 8) before adding a candidate to the quota set $Q$ or to the other set $R$.
The result of the \textit{CascadeSearch} procedure maximizes (\ref{eq:fair_objective_U_psi}), as no penalty is accumulated in the loop. 
In fact, an non-protected candidate ($W_c=0$) is not added only because  there is no room in $R$ -- and $R$ never gets smaller to allow for more room later on. A protected candidate ($W_c=1$) is not added only if not meeting the minimum basic requirement (line 8), hence it cannot be counted for the penalty. Formally, $U^k_{\psi}\big(S^k_\goodtext, \theta\big) = k$ for the solution $S^k_\goodtext$ returned by \textit{CascadeSearch}.

%
%\begin{definition}[Examination Search]
%\label{def:ExaminationSearch}
%    Given $\theta$, $h$ explores all of $\candidatesset$, computing each $c$ candidate's score $s(\mathbf{X}_c)$, ranking them all into $\tau$, and choosing the first $k$-candidates accordingly while ensuring that the fairness representational quota $q$ is met. 
%    Algorithm~\ref{algo:Examination} implements this search procedure.
%\end{definition}
%

%
%\begin{definition}[Cascade Search]
%\label{def:CascadeSearch}
%    Given $\theta$, $h$ explores $\candidatesset$ up until reaching $k$-candidates that meet the minimum basic requirements $\psi$ in terms of each candidate's $c$ score $s(\mathbf{X}_c)$ while ensuring that the fairness representational quota $q$ is met. 
%    Algorithm~\ref{algo:Cascade} implements this search procedure.
%\end{definition}
%

%
% Figure environment removed
%

% Both Algorithms~\ref{algo:Examination}~and~\ref{algo:Cascade} implicitly assume there are enough suitable $m$ candidates in $\candidatesset$ according to the minimum basic requirements $\psi$, such that $k \geq m \geq n$. 
% Hence, both achieve the respective solution. 
% These are the simplest algorithms for the examination and cascade searches, which can easily be extended by allowing the screener to update, e.g., $\psi$ of $k$ depending on the size of $S^k$ during the search.

%We highlight the fairness goal of obtaining a representative selected set of $k$ candidates as denoted by $q$.
%In Algorithms~\ref{algo:Examination}~and~\ref{algo:Cascade}, $q^*$, 
% or \texttt{line 2} for both of them, 
%represents the minimum number of protected candidates ($W_c = 1$) to be selected. 
%Hence, $k - q^*$ represents the maximum number of non-protected candidates ($W_c = 0$) to be selected.
%Under \texttt{ExaminationSearch} (Algorithm~\ref{algo:Examination}), meeting this quota is based on searching the full candidate pool according to $\theta$, sorting it by the scores in two separate lists according to $W$, and choosing the $q^*$-best protected candidates from $\tau_0$ and the $(k-q^*)$-best non-protected candidates from $\tau_1$.
%Under \texttt{CascadeSearch} (Algorithm~\ref{algo:Cascade}), meeting this quota goes on a candidate by candidate basis, thus, not requiring to search the full candidate pool according to $\theta$. 
%Here, the algorithm prioritizes constructing $S^k$ with $q^*$ eligible protected candidates $l_1$ while keeping track of the eligible non-protected candidates $l_0$.
%Once the quota is met, if between $l_1$ and $l_0$ there are not enough $k$ candidates then the algorithm keeps searching though without considering group membership. Notice that $l_0$ cannot be greater than $k-q^*$.

%
% EOS
%


\section{The Human-Like Screener}
%
\label{sec:HumanScreener}

In this section, we want to model a screener $h$ prone to error when evaluating the candidate pool $\mathcal{C}$ given the initial order~$\theta$.
We do so to capture real world screening problems, such as the one in Section~\ref{sec:Generali}, involving human decision-makers and their reliance on digital technologies.    
Let us distinguish \textit{two kinds of screeners}.
We define a screener $h$ as \textit{algorithmic}, denoted by $h_a \in \mathcal{H}_a$, if it can consistently evaluate $\mathcal{C}$. 
The algorithmic screener is the implied screener in the fair ranking literature.
We also define a screener $h$ as \textit{human-like}, denoted by $h_h \in \mathcal{H}_h$, if it is conditioned by a fatigue component that hinders the consistency of its evaluation as it explores $\mathcal{C}$.
% SR
%As we will show here, it is under this human-like screener that the distinction between best-$k$ and good-$k$ becomes relevant.

% Time
Let us introduce a \textit{time component} to study these two screeners.
Let $t$ denote the discrete unit of time. 
It will be used to represent how long $h$ takes to evaluate a candidate $c \in \mathcal{C}$. We assume such a time to be constant, according to fact G5 in Section~\ref{sec:Generali}.
% \footnote{This assumption is based on fact G5 in Section~\ref{sec:Generali}.}  
This assumption implies that time itself cannot be optimized by the screener $h$.
We track time along the $\theta$ chosen by $h$ to evaluate $\candidatesset$, meaning that, at time $t=1$, $h$ evaluates the first candidate that appear in $\theta$ and so on. Thus time $t$ can range from $0$ (before start screening) to $n$ (after screening all candidates) at maximum.
%SR
%Hence, time $t$ follows the index $i$ in $\theta$ that identifies a candidate $c \in \mathcal{C}$.
% Fatigue
Time allows to capture what occurs in real life, especially with performing repetitive tasks, where the algorithmic screener $h_a$ is consistent in its evaluation of candidates while the human-like screener $h_h$ loses its consistency over time (see, e.g., \cite{Kahneman2021Noise}).
Formally, we introduce a \textit{fatigue component} $\phi(t)$ specific to $h_h$ as a function depending on time $t$ and 
model the \textit{accumulated fatigue} $\Phi \colon \{0, \ldots, n\} \to \mathbb{R}$, with $\Phi(0) = 0$. 
The discrete derivative of $\Phi$, that is, $\phi(i) = \Phi(i) - \Phi(i-1)$, defined for $t \geq 1$, is the effort of $h_h$ to examine the $t$-th candidate. 
%SR
% $c=\theta^{-1}(i)$.
%
What $\Phi$ means in practice is that a screener will evaluate the identical candidates $c_1$ and $c_2$ differently at times $t_1$ and $t_2$, as long as $\Phi(t_1) \neq \Phi(t_2)$.
We recognize that
fatigue can accumulate with different functional forms, 
meaning how we define $\Phi$ conditions the effects of fatigue on our analysis of $h_h$.
Here, we make the simplest modeling choice to define $\phi$ by assuming that fatigue accumulates linearly over time, or $\phi(t) = \lambda$ so that $\Phi(t) = \lambda \cdot t$.
This is one possible formulation based on our interpretation that $h_h$ will become tired at a constant pace over time.
We leave the study of other $\Phi$ formulations for future work.

\subsection{Fatigued Scores}
\label{sec:HumanScreener.BiasedScores}

We model the effect of fatigue in the evaluation of a candidate $c$ by $h_h$ through the \textit{fatigued score} $s_{h_h}(\mathbf{X}_c) + \epsilon$, where $\epsilon$ is a random variable depending on the fatigue $\Phi$ which quantifies the deviation from the \textit{truthful (or unbiased) score} $s_h(\mathbf{X}_c)$.
We can model $\epsilon$ in multiple ways. 
Here, we consider two ways.
The \textit{first modeling} is to see the error $\epsilon$ as a centered Gaussian variable, and the fatigue affecting only its variance. 
Formally, at given time $t$, 
$\epsilon$ is defined as $\epsilon_1 \sim \mathcal{N}(0, \, v(\Phi(t-1)))$, where $v \colon \mathbb{R} \to \mathbb{R}$ defines the variance of $\epsilon_1$ as an increasing function of the accumulated fatigue. 
The \textit{second modeling} is to assume that there is a negative bias in the evaluation of applicants.
Under this condition, the more fatigue, the more the screener $h_h$ tends to underscore the candidates.
Hence, we model $\epsilon$ as uncentered Gaussian, whose mean is a decreasing function of the fatigue.
Formally, $\epsilon$ is defined as $\epsilon_2 \sim \mathcal{N}(\mu(\Phi(t-1), \, v(\Phi(t-1))$, where $\mu \colon \mathbb{R} \to \mathbb{R}$ is a decreasing function rather than a constant.

We assume that $h_h$ is unaware of its fatigue as it goes over $\candidatesset$, representing an unconscious decision-making bias due, e.g.,~mental heuristics (see, e.g.,~\cite{tversky_judgment_1974}).
Hence, we are interested not by how $h_h$ minimizes its fatigue, which would require adding the fatigue to the problem definition, but by what is the impact of $h_h$'s fatigue on its screening process?
The answer depends on the objective of the set selection problem as discussed in Section~\ref{sec:ProblemFormulation.Objectives} and the initial order $\theta$. % chosen by the human-like screener $h_h$.

In terms of the search procedures proposed in Section~\ref{sec:ProblemFormulation.Algorithms}, Algorithms~\ref{algo:Examination} and \ref{algo:Cascade} represent the algorithmic screener $h_a$ as there is no notion of fatigue nor biased scores due to it. 
%SR
%Hence, we refer to such setting as the \textit{baseline model}.
To represent the human-like screener $h_h$, we track the accumulated fatigue $\Phi$ over the time and use it to draw $\epsilon$ so that $h_h$ computes the fatigued scores. %$s_{h_h}(\mathbf{X}_c) + \epsilon$. 
The main change is to line 2 in Algorithm~\ref{algo:Examination} and line 7 in Algorithm~\ref{algo:Cascade} where the score(s) computed for $c$ is biased by the $\epsilon$ at that point of time. 
We present the human-like versions of these two search procedures as Algorithms~\ref{algo:HumanExamination} and \ref{algo:HumanCascade} in Appendix~\ref{Appendix.HumanAlgorithms}.

\subsection{The Fairness of the Algorithmic and Human-like Screener}
\label{sec:HumanScreener.Analysis}

% Position Bias and the Implication of the Initial Order
We discuss the differences between an algorithmic $h_a$ or human-like $h_h$ screener from the fairness perspective.
Under the assumptions below, we can focus on $\theta$ and how, through position bias, it leads to individual fairness violations.
% Let us consider two assumptions.
%
% \textit{A1}: We assume that the initial order $\theta$ is independent of the protected attribute $W$, meaning the way in which candidates are sorted in $\theta$ contains no information about membership to either category in $W$.
% \textit{A2}: We also assume that the individual scoring functions $s$, for either screener, is able to evaluate candidate $c$ fairly and truthfully, meaning $s(\mathbf{X}_c)$ captures no information about $W_c$ (i.e., scores and the protected attribute are independent) and captures all information about the suitability of $c$ for the position. 
%
%
\begin{itemize}
    \item \textit{Assumption A1: We assume that the initial order $\theta$ is independent of the protected attribute $W$}, meaning the way in which candidates are sorted in $\theta$ contains no information about membership to either category in $W$.
    %
    \item \textit{Assumption A2: We also assume that the individual scoring functions $s$, for either screener, is able to evaluate candidate $c$ fairly and truthfully}, meaning $s(\mathbf{X}_c)$ captures no information about $W_c$ (i.e., scores and the protected attribute are independent) and captures all information about the suitability of $c$ for the position. 
\end{itemize}
%

We first focus on $h_a$ and the solutions of the best-$k$ and good-$k$ problems computed by Algorithms \ref{algo:Examination} and \ref{algo:Cascade}.
% SR
%We observe that imposing the condition of representational quota $f(S^k) \geq q$ changes the solution of the unconstrained problem, so that the achieved fair utility value is upper bounded by the maximum utility.
%
\textit{The $h_a$ reaches the optimal fair solution of both fair best-$k$ or good-$k$ problems.}
%
\textit{Moreover, $h_a$ guarantees individual fairness in both scenarios.}
%
%
Under assumptions \textit{A1} and \textit{A2}, this result for $h_a$ is intuitive for both Algorithms \ref{algo:Examination} and \ref{algo:Cascade}. 
Regarding group fairness, $h_a$ fulfills it by satisfying the representational quota $q$.
Regarding individual fairness, it is fulfilled as $h_a$ is able to consistently evaluate each candidate, treating similar candidates similarly. 
Therefore, $\theta$'s role here on fairness is trivial. 
These results hold under \textit{A2},
% \footnote{For instance, if $\mathbf{X}$ is a proxy for $W$, then $h_a$ can be consistently biased under $s$.} 
which we know it is likely not to be true in reality \cite{Zehlike2023_FairRanking_P1, Zehlike2023_FairRanking_P2} but we assume it here as our focus is on the human-like screener $h_h$ and its interaction with $\theta$.

% % Group fairness is fulfilled by satisfying the representational quota. 
% In particular, we highlight the assumption of the independent choice of the initial order $\theta$ of the protected attribute $W$.
% Individual fairness is guaranteed because two conditions are met: (i) the protected attribute does not directly affect the score evaluation;
% (ii) if the initial order $\theta$ is chosen independently from the protected attribute, then the computed score of similar individuals is the same. (recall \cite{DBLP:conf/innovations/DworkHPRZ12})

We now focus on $h_h$ and the solutions of the best-$k$ and good-$k$ problems computed by Algorithms~\ref{algo:HumanExamination} and \ref{algo:HumanCascade}.
Note that, under \textit{A1}, the error on the score does not affect the evaluation of the representational quota $q$ at a group level.
\textit{The $h_h$ guarantees, in both best-$k$ or good-$k$ problems, a fair group level solution despite the fatigue.}
Formally, the expected error $\mathbb{E}[\epsilon \mid W_c = 1, \theta] = \mathbb{E}[\epsilon \mid W_c = 0, \theta]$, whether $\epsilon$ is $\epsilon_1$ or $\epsilon_2$.
Intuitively, $h_h$ here simply needs to satisfy $q$. 
The fatigue and, thus, the fatigued scores are, on average, shared across protected and non-protected candidates.

However, \textit{individual fairness is not guaranteed under $h_h$.}
This result is intuitive.
Indeed, even if the protected attribute does not directly affect the score evaluation with error, the position of the candidate in $\theta$ influences the amount of error made by $h_h$ when evaluating that candidate. 
Therefore, similar candidates, because of their position in $\theta$ (i.e.,~no two candidate can occupy the same position), will not be evaluated similarly due to the accumulation of fatigue experienced by $h_h$.
For instance,
given two similar candidate $c_i = \theta^{-1}(i), c_j = \theta^{-1}(j)$, with $i < j$, their evaluation could be significantly different in the amount of error depending on the accumulated fatigue $\Phi(i) < \Phi(j)$. 
%In the case of $\epsilon_1$, $i$ has the advantage of being evaluated by a rested screener.
%Moreover, 
E.g.,~in the case of $\epsilon_2$, even if $\mathbf{X}_{c_j} = \mathbf{X}_{c_i}$, the score of $j$ is less, on average, than the one of $i$, so that $i$ has an unfair premium from $\theta$.
%Therefore the position bias of the human-like screener appears in both modeling of the error considered.

\subsection{Position Bias and the Initial Screening Order}
\label{sec:ProblemFormulation.PositionBias}

The implications of different modelings of the error $\epsilon$, or of other fairness requirements, need further discussion that we do not cover in this work.
Based on the previous section, though, our analysis already shows the overall risk of the position bias inherent to the initial order $\theta$ under a human-like screener.

In the case of the algorithmic screener $h_a$, we find trivial results, especially under the assumptions \textit{A1} and \textit{A2}.
The results for $h_a$ are trivial because of the nature of algorithms and their inherent consistency when it comes to decision making, which is, after all, a strong selling point for proponents of ADM systems (e.g., \cite{Miller2018Bias, Kahneman2016Noise}). 
Similarly, this is why position bias is treated as a technical bias in the fair ranking literature (recall, Section~\ref{sec:RelatedWork}) and focus is given on relaxing assumption \textit{A1} to learn a fair scoring function.

As our analysis shows, by considering the human-like screener $h_h$, however, that position bias and the role of $\theta$ acquire a significant meaning for individual fairness. 
We emphasize that the results in the previous subsection hold under assumptions \textit{A1} and \textit{A2}. If we relaxed either assumption, then the individual fairness violations should be even more concerning. 
Moving forward, we note that position bias as manifested in $\theta$ is only a function of where the candidates lie in $\theta$ when $h_h$ starts the candidate screening. 
We emphasize that
\textit{this fact is true whether $\theta$ is chosen by $h_h$ or provided to by, e.g., a (fair) ranking algorithm.}

%
% EOS
%


\section{Experiments}
\section{Experimental Results \& Discussion}\label{sec:experiments}
We wish to answer the following research questions experimentally:
\begin{description}
\item[\textbf{RQ1}] \textit{Is our proposed probabilistic method able to model exposure probabilities more accurately than existing methods?}
\item[\textbf{RQ2}] \textit{Can the model leverage contextual signals effectively?}
\item[\textbf{RQ3}] \textit{Are the obtained position biases useful for downstream tasks, such as unbiased offline evaluation?}
\end{description}

Naturally, position biases are heavily influenced by specific use-cases, platforms and interface choices.
The methods we propose in this work are motivated by a short-video feed recommendation use-case, and even though our proposed framework is generally applicable, we expect the Yule-Simon instantiation to only hold merit in similar use-cases.

In order to empirically validate the performance of both our and earlier proposed methods, we require \emph{interventional} data with logged views $V$, ranks $R$, and contexts $X$.
To the best of our knowledge and at the time of writing, we are unaware of any such datasets being publicly available.
Existing Learning-to-Rank (LTR) datasets do not contain rank interventions and deal with web search use-cases, which imply very different modalities to ours.
For this reason, we need to resort to proprietary datasets, but additionally release an open-source Jupyter notebook to reproduce the position bias curves visualised in Figure~\ref{fig:yulesimon} at \href{https://github.com/olivierjeunen/C-3PO-recsys-2023}{github.com/olivierjeunen/C-3PO-recsys-2023}.

\subsection{Estimating Exposure Probabilities (RQ1--2)}
We obtain a sample of 1 million sessions of feed view events on a social media platform, where rank interventions occurred following Fig.~\ref{fig:PGM}, collected over five days in February 2023.
We perform an 80-20\% train-test split, aiming to predict whether recommendations were viewed based on their rank and contextual information.

We compare several non-contextual variants: the standard DCG discount function as well as the logarithmic and exponential forms in Eq.~\ref{eq:dcg}, and the probabilistic method based on the Yule-Simon distribution introduced in Eq.~\ref{eq:prob}.
The latter three methods include a single parameter ($\alpha, \gamma, \rho$ respectively), which we learn to minimise NLL@$K$ on the training set, following the procedure laid out in \S\ref{sec:learning}.
We implement this in Python 3.9 with the SciPy library~\cite{Virtanen2020}.

As an additional baseline, we include a non-parametric method that predicts the empirical average from the training data.
This approach should be expected to outperform the aforementioned methods, but it requires a hard-coded probability at every rank instead of the single parameter that the logarithmic, exponential, or probabilistic forms require.
Additionally, this approach cannot easily be extended to incorporate contextual information $X$.

For the contextual case, we adopt a single continuous user-based feature describing users' past average scroll depth, as well as a single continuous context-based feature, describing average scroll depth at the time of day. 
We adopt a simple linear model to estimate the distribution parameter from this input $X$: $\rho_{\theta}(x) = \theta^{\intercal}x$.
The functional forms for the parameters $\alpha$ and $\gamma$ are analogous.
As such, the contextual and personalised methods consist of only \emph{three} parameters each (assuming $x$ includes a constant 1-feature, emulating a bias term in $\theta$).
Even in this simplistic scenario, the contextual and personalised methods significantly outperform those that do not consider this information, as shown in Table~\ref{tab:results}.
Our contextual, personalised, probabilistic position bias model \texttt{C-3PO} achieves the lowest NLL@$K$ for a wide range of $K$, whilst requiring a minimum of learnable parameters or computing resources.
This yields a desirable trade-off between parsimony and model expressiveness when compared to complex model classes like neural networks (which would typically require orders of magnitude more parameters). 
We observe that this additionally allows us to be sample-efficient, as our method already performs well with only $\mathcal{O}(10^{4})$ samples.
Indeed, instead of modelling the entire curve at every possible value of $r$, our proposed method outputs a single scalar which can be used to obtain position bias estimates for all natural numbers.
The inductive bias we enjoy from well-motivated mathematical models greatly improves the methods' real-world usability, when compared to neural-network based alternatives.

\begin{table}[t]
    \centering
    \begin{tabular}{lcccccc}
    \toprule
    \multirow{2}{*}{\textbf{Model}} & \multicolumn{5}{c}{\textbf{Negative Log-Likelihood (NLL)}}\\
     ~& \textbf{@5} & \textbf{@10} & \textbf{@25} & \textbf{@50} & \textbf{@100} \\
    \cline{2-6}

    $\widehat{\mathsf{P}}_{{\rm dcg}}(V|R)$ & 0.5453 & 0.6320 & 0.5998 & 0.4973 & 0.3763\\
    $\widehat{\mathsf{P}}_{\log}(V|R)$ & \underline{0.5159} & \underline{0.6001} & 0.5900 & 0.5036 & 0.3833\\
    $\widehat{\mathsf{P}}_{\exp}(V|R)$ & 0.5202 & 0.6158 & 0.6089 & 0.5101 & 0.3673\\
    $\widehat{\mathsf{P}}_{\rm prob}(V|R)$ & \underline{0.5162} & \underline{0.6002} &\underline{0.5873} & \underline{0.4891} & \underline{0.3495}\vspace{1ex}\\
    \cdashline{1-6}
    \vspace{-2ex}~\\
    $\widehat{\mathsf{P}}_{{\rm empirical}}(V|R)$ & 0.5157 & 0.5999 & 0.5843 & 0.4813 & 0.3369\vspace{1ex}\\
    \cdashline{1-6}
    \vspace{-2ex}~\\
    $\widehat{\mathsf{P}}_{\log}(V|R,X)$ & \textbf{0.4852} & \textbf{0.5620} & 0.5577 & 0.4806 & 0.3555\\ 
    $\widehat{\mathsf{P}}_{\exp}(V|R,X)$ & 0.4883 & 0.5761 & 0.5778 & 0.4959 & 0.3652\\ 
    $\widehat{\mathsf{P}}_{\rm prob}(V|R,X)$ & \textbf{0.4850} & \textbf{0.5620} &\textbf{0.5551} & \textbf{0.4651} & \textbf{0.3325}\\ 
\bottomrule
    \end{tabular}
    \caption{NLL for position bias models on observed data, lower is better.
    The top-group are independent of contextual information, the middle baseline is a non-parametric method that predicts a sample average, the bottom-group include three parameters that were optimised via linear regression. 
    Marked fields indicate stat. sig. improvements over other methods in the same group at a 99\% level.}
    \label{tab:results}
\end{table}

\subsection{Unbiased Offline Evaluation (RQ3)}
The main task position bias models need to perform, is to deliver offline estimates of online performance.
Given a dataset of logged impressions $\mathcal{D}\coloneqq \{(x_{i}, a_{i}, r_{i}, c_{i})_{i=1}^{N}\}$ (contexts, actions, ranks, rewards) , we wish to estimate the expected reward we would have obtained under some different ranking policy $\pi$.
This policy maps a context $X$ and set of \emph{candidate} items $\mathcal{A}_{x}$ to a ranked list.
We will denote with the shorthand notation $\pi(a|x)$ the rank that item $a$ will be placed at when $\pi$ is presented with context $x$ (assuming $\mathcal{A}_{x}$ given).
Note that this framing is easily extended to more general stochastic ranking policies~\cite{Oosterhuis2021}.
Then, a dataset $\mathcal{D}$ and position bias model $\widehat{\mathsf{P}}$ can be used to to obtain an unbiased estimate of the reward we would obtain under $\pi$:
\vspace{-1ex}
\begin{equation}\label{eq:unbiased_dcg}
    \mathop{\mathbb{E}}\limits_{r \sim \pi}[C] \stackrel{1}{\approx} {\rm DCG}_{\widehat{\mathsf{P}}}(\mathcal{D}, \pi) \stackrel{2}{\approx}
    \frac{1}{N}\sum_{i=1}^{N}
    c_{i} \cdot \frac{\widehat{\mathsf{P}}(V=1|R=\pi(a_{i}|x_{i}), X=x_{i})}{\widehat{\mathsf{P}}(V=1|R=r_{i}, X=x_{i})} .
\end{equation}
Here, the first approximation $\stackrel{1}{\approx}$ is due to the inherent assumptions of the DCG metric (compared to, e.g., cascade-based alternatives), whereas the second only exists because we resort to an empirical average over the observed data $\mathcal{D}$ and estimated position biases via $\widehat{\mathsf{P}}$.
Assuming unbiasedness of $\widehat{\mathsf{P}}$, the unbiasedness of the metric in Eq.~\ref{eq:unbiased_dcg} is easily recognised, as it is an application of importance sampling or IPS~\cite{Owen2013}.
As is typical for IPS-based methods, techniques like capping or introducing control variates can improve their finite-sample performance by reducing variance~\cite{Gilotte2018, Swaminathan2015snips}.
We do not consider such extensions in this short article, but remark that they are likely to further improve performance.

To validate the utility of these offline estimates, we perform an online experiment on a social media platform that operates a short-video recommendation feed.
Thus, we obtain samples from the reward distribution by sampling $\mathbb{E}_{r \sim \pi}[C]$ directly and taking an empirical average per day, for five days.
Then, for varying context-independent position bias models (optimised $@100$), we obtain offline estimates of online reward via Eq.~\ref{eq:unbiased_dcg}, and evaluate the offline estimates by Pearson's correlation coefficient between the ground truth and the offline estimate, over 5 days.

Table~\ref{tab:results2} shows relative improvements in correlation over the classical DCG formulation.
We observe that our probabilistically motivated position bias model is able to significantly improve the offline-online correlation compared to existing methods, and conjecture that the context-dependent variant can lead to further improvements.
This highlights the importance of a well-motivated position bias model, and is a strong argument in favour of our proposed methods.

\begin{table}[t]
    \vspace{-3ex}
    \centering
    \begin{tabular}{lllccccccc}
    \toprule
    \textbf{Position Bias Model} &~&~& $\widehat{\mathsf{P}}_{{\rm dcg}}(V|R)$ &~& $\widehat{\mathsf{P}}_{{\rm log}}(V|R)$ &~& $\widehat{\mathsf{P}}_{{\rm exp}}(V|R)$ &~& $\widehat{\mathsf{P}}_{{\rm prob}}(V|R)$ \\
         \cline{4-10}
\textbf{Relative Improvement} &~&~& 100\% &~& -3\% &~& -20\% &~& \textbf{+16\%} \\
    
\bottomrule
    \end{tabular}
    \caption{Relative correlation improvement over $\widehat{\mathsf{P}}_{{\rm dcg}}$ between DCG estimates and online metrics, higher is better.}
    \label{tab:results2}
\end{table}

\section{Conclusion}
\section{Discussion}
\label{sec:discussion}
To assist AI practitioners in navigating the rapidly evolving landscape of AI ethics, governance, and regulations, we have developed a method for generating actionable guidelines for responsible AI. This method enables easy updates of guidelines based on research papers and ISO standards, ensuring that the content remains relevant and up-to-date. We validated this method through a use case study at a large tech company, where we designed and evaluated a tool that uses our responsible AI guidelines. We conducted a formative study involving 10 AI practitioners to design the tool, and further evaluated it through an interview study with an additional 14 AI practitioners. The results indicate that the guidelines were perceived as practical and actionable, promoting self-reflection and enhancing understanding of the ethical considerations associated with AI during the early stages of development. In light of these results, we discuss how our method contributes to the idea of ``Responsible AI by Design'', that is, a design-first approach that considers responsible AI values throughout the development lifecycle and across business roles. We discuss the inherent problem of decontextualization in responsible AI toolkits, the concept of meta-responsibility, and provide practical recommendations for designing responsible AI toolkits with the aim of fostering collaboration and enabling organizational accountability.

\subsection{Theoretical Implications}

\subsubsection{Decontextualization}
The inherent challenge in responsible AI toolkits lies in their attempt to reconcile the tension between scalability and context specificity~\cite{wong2023seeing}. Traditional approaches to toolkit development have often favored a universal, top-down approach that assumes a one-size-fits-all solution~\cite{kelty_participatory_toolkit, mattern_toolkit}. However, participatory development, such as the methodology we followed in designing and populating a responsible AI toolkit with our guidelines, emphasizes the importance of tailoring responsible AI guidelines to specific contexts and job roles needs. It is crucial therefore to recognize that different AI practitioners, such as designers, developers, engineers, and advisors, have distinct requirements and considerations that cannot be treated as identical. This highlights the complexity of developing toolkits that cater to a diverse range of practitioners while accounting for their unique roles and settings---the problem of decontextualization in responsible AI toolkits~\cite{wong2023seeing}.

To tackle the problem of decontextualization, our proposed method incorporates two key elements: \emph{actionable guidelines} and \emph{follow-up questions}. Firstly, the integration of actionable guidelines, tailored to different roles and projects, provides practical steps and recommendations that technical practitioners can easily implement, or C-level executives can make informed decisions upon. These guidelines serve as a starting point for ethical decision-making throughout the AI lifecycle, contributing to the vision of responsible AI by design (borrowing from the idea of `privacy by design'\footnote{``Privacy by design'' is a standard practice for incorporating data protection into the design of technology. In other words, data protection is achieved when it is already integrated into the technology during its design and development~\cite{cavoukian2009privacy}.}). Secondly, the inclusion of follow-up questions enhances our toolkit's ability to capture the complexities of different social and organizational contexts. Expanding upon the concept that follow-up questions are an effective means of communication~\cite{weger2014relative}, as they help in gaining deeper insights, clarifying responses, and uncovering underlying meanings, AI practitioners can engage with these questions to explore the ethical considerations and challenges that are unique to their deployment context.

\subsubsection{Meta-responsibility} Scholars have long recognized the need for a socio-technical approach that considers the contextual factors governing the use of AI systems, including social, organizational, and cultural factors~\cite{tahaei2023toward}. In fact, Ackerman~\cite{ackerman2000intellectual} introduced the concept of socio-technical gap to highlight the disparity between human requirements in technology deployment contexts (socio-requirements) and the technical solutions. This gap arises due to the flexible and nuanced nature of human activity compared to the rigid and brittle nature of computational mechanisms, resulting from necessary formalization and abstraction. Along these lines, \citet{stahl2023embedding} introduced the concept of meta-responsibility to stress that AI systems should be viewed as systems of systems (ecosystems) rather than single entities. To establish a regime of meta-responsibility, Stahl argued for an adaptive governance structure to effectively respond to new insights and external influences (e.g., upcoming AI regulation), and for a knowledge base that equips AI stakeholders with technical, ethical, legal, and social understanding. By integrating ethical, legal, and social knowledge into the AI development process---what Stahl referred to as adaptive governance structure, and offering recommendations for areas that require additional attention (i.e., responsible AI blindspots), our work contribute to this line of research by providing empirical evidence to it and pushing the theoretical boundaries further.

\subsection{Practical Implications}

\subsubsection{Recommendations for designing responsible AI toolkits}
Our responsible AI guidelines, populated in a usable tool, leverage the concept of nudging to encourage users to consider the ethical implications of AI systems. Nudging has demonstrated effectiveness in various domains, such as mitigating the dissemination of misinformation on social media through the use of checklists~\cite{jahanbakhsh2021exploring}, or guiding users towards more private and secure choices~\cite{acquisti2017nudges, tahaei2021deciding}.

Nudges can be implemented in various ways. For instance, the \emph{confront} type of nudge incorporates elements of ``reminding consequences'' and ``providing multiple viewpoints,'' encouraging users to consider alternative directions and diverse perspectives~\cite{caraban2019ways}. In the case of our guidelines, these two concepts are utilized to remind AI developers about the ethical considerations of AI systems and to prompt them to think critically about alternative viewpoints, thus helping them avoid confirmation bias. Further research could explore additional types of nudges, such as incorporating visual cues (e.g., \emph{just-in-time nudges} within development tools), facilitating positive behavior (e.g., \emph{enabling social comparisons} by recognizing and appreciating developers who promote ethical values within the organization), or fostering empathy (e.g., \emph{instigating empathy} by presenting the environmental impact of an AI system through easily understandable animations).

While the format of our tool proved to be useful, it offers a starting point to explore other formats and interactions for populating and contextualizing the guidelines. For example, structuring the guidelines into a narrative might be useful to unpack the complexity of particularly complex guidelines, such as guideline \#15---\emph{ensuring compliance with agreements and legal requirements when handling data.} This guideline can be further sub-divided into sequential steps providing more context and explanations. Moreover, future responsible AI tools can incorporate configurable parameters or customization widgets to align with specific requirements of the developed AI systems or user preferences. Additionally, the use of Language Models (LLMs) can be explored to further customize and adapt the provided examples within the tool. Finally, more research can be done on exploring responsible AI tools as a method for artifact creation. This includes automatic generation of summary reports, model cards, or responsible AI certificates.

\subsubsection{Recommendations for fostering collaboration and enabling organizational accountability}
While individual adoption of responsible AI best practices is crucial, promoting collaboration among diverse AI stakeholders is equally important. Many existing responsible AI toolkits prioritize individual usage~\cite{wong2023seeing}. However, addressing complex ethical and societal challenges associated with AI systems requires collaborative actions. Our interactive tool populated with actionable guidelines addresses this need by offering features that facilitate collaboration. First, the tool stores users' inputs in a responsible AI knowledge base, enabling distributed teams to access and leverage this knowledge for a shared understanding of a particular AI system. This promotes collaboration and a collective approach to ethical considerations. Second, the tool automatically generates a report that summarizes the user's considerations. This report can be downloaded as a PDF and includes responsible AI blindspots, which are specific actions to be taken by individuals or shared among the development team. Highlighting these blindspots fosters awareness and prompts collective action towards responsible AI practices.

In addition to fostering collaboration, our interactive tool can be used to enable organizational accountability. Similar to Google's five-stage internal algorithmic auditing framework~\cite{raji2020closing}, our guidelines serve as a practical tool for closing the AI accountability gap. The automatically generated report plays a crucial role in this process by providing a summary of the guidelines that were effectively implemented, those that should be considered for future development, and the non-applicable ones. These reports establish an additional chain of accountability that can be shared with stakeholders at various levels, including managers, senior leadership, and AI engineers. By offering more oversight and the ability to troubleshoot if needed, these reports help mitigate unintentional harm. However, it is important to note that when an organization adopts our guidelines, it should establish clear ethical guidelines for their intended uses. Our tool is not intended to discourage developers from using it due to the fear of being held accountable for their responses. On the contrary, developers' responses, as documented in the report, provide an opportunity to identify potential ethical issues and address them early in the design stages. This proactive approach prevents the need for post-hoc fixes and repairs, aligning with the principle of addressing ethical considerations during the development process rather than as an afterthought~\cite{sambasivan2018toward}---the idea of \emph{Responsible AI by Design}.

\subsection{Limitations and Future Work} Our work has four main limitations that highlight the need for future research efforts. 

Firstly, although we followed a rigorous four-step process involving multiple stakeholders, the list of 22 guidelines may not be exhaustive. The rapidly evolving nature of AI ethics, governance, and regulations necessitates an ongoing effort to stay abreast of emerging developments. However, one of the strengths of our method lies in its modular design, which allows for ongoing refinement and expansion of the set of guidelines. This ensures that our responsible AI tool maintains its relevance and stays up to date in the ever-evolving landscape of AI ethics, governance, and regulations. As new ISOs are established, addressing specific aspects of AI systems such as functional safety (ISO 5469), data quality (ISO 5259), and explainability (ISO 6254), our tool can be readily extended to include these guidelines. Moreover, as the scientific community progresses in its understanding of ethical considerations in AI, our tool can incorporate new insights and recommendations to enhance its comprehensive coverage.

Secondly, it is important to consider the qualitative nature of our user study, which involved in-depth interviews and analysis of participants' responses. The findings from this study should be interpreted with caution, understanding that the reported frequency of themes should be viewed in a comparative context rather than taken at face value~\cite{fossey2002understanding}. This approach helps to avoid potential misinterpretation or overgeneralization of the results.

Thirdly, we need to acknowledge the limitations associated with the sample size and demographics of our user study. The study was conducted with a specific group of participants, and therefore, the findings may not fully represent the practices and perspectives of all AI practitioners. Our sample predominantly consisted of male participants, which aligns with the gender distribution reported in Stack Overflow's 2022 Developer Survey, where 92.85\% of professional developer respondents identified as male~\cite{stackoverflow2022survey}. Additionally, our participants were drawn from a large research-focused technology company. While the results may offer insights into practices within certain companies, they serve as a case study for future research. Furthermore, we did not explicitly consider participants' specific roles, despite their expertise spanning various domains and levels. Future studies could explore the considerations of ethical values in AI systems across organizations and different roles and areas of expertise. Previous research has indicated different understandings of responsible AI values between practitioners and the general public~\cite{maurice2022how}, suggesting the potential for similar research methods to be applied in this area.

Last but not least, our qualitative data suggests indicators of ease of use for AI practitioners but does not provide direct information on the actual effectiveness of the guidelines. Understanding the impact of guidelines (or other AI toolkits~\cite{wong2023seeing}) requires long-term studies that consider multiple projects, with some utilizing the toolkit and others not. One potential avenue, as suggested by clinical researchers developing deep learning tools for patient care~\cite{beede2020human}, is to conduct observational studies with users of the AI system to assess its performance. Another approach is to use proxies, such as measuring users' attitudes, beliefs, and mindset regarding ethical values before and after utilizing the guidelines. We intend to explore these directions in future research.

%%% For arXiv
% \newpage
% \section*{Research Ethics and Social Impact}
% % Authors should describe the ethical challenges they faced in their submission, ideally in a dedicated section, and how they addressed such challenges. In particular, submissions that (1) describe experiments with users and/or deployed systems (e.g., websites or apps), or (2) rely on sensitive user data (e.g., social network information), must adhere to precepts of ethical research and community norms. These include compliance with applicable laws and applicable professional ethical codes; respect for privacy; secure storage of sensitive data; voluntary and informed consent when appropriate; avoiding deceptive practices when not essential; beneficence and non-maleficence (maximizing the benefits to an individual or society while minimizing harm to the individual); risk mitigation; and post-hoc disclosure of audits.
% In this section, we describe the ethical challenges faced during this work as well as clarify any potential negative social impact this work might have in the future.

% \paragraph{Ethical considerations}
% % 1) a description of the ethical concerns the authors mitigated while conducting the work (as part of an ethical considerations statement)
% Given the nature of this work, which is mainly theoretical, to the best of our knowledge, we did not face any ethical challenges when drafting the paper. Our experiments are based on simulated data intended to illustrate our theoretical analysis. During our collaboration with company G, in particular, which occurred before the drafting of this paper, we followed G's strict ethical guidelines at all times. We concluded our collaboration with G with an internal report that we presented and discussed with all stakeholders. No sensitive data (or data at all) from company G was used for this work. At no point did we receive monetary compensation from G. Our research has been funded by public institutions. The views reflected in this paper are entirely our own.

% \paragraph{Researcher positionality}
% % 2) reflections on how their background and experiences inform or shape the work (as part of a researcher positionality statement)
% The authors come from a mixed of backgrounds, both personal and academic. To ensure anonymity, we do not disclose here further information about the authors, but we are committed to provide the necessary details if this submission is successful. We will provide a positionality statement per author. 
 
% \paragraph{Adverse impact}
% % 3) reflection on the adverse, unintended impact the work might have once published (as part of an adverse impact statement)
% We believe that this work shows the importance of considering the human user in the formulation of the candidate screening problem. We want to stress that our distinction between an algorithmic screener and a human-like screener was to show the importance of considering the latter kind and not to endorse the former kind. We strongly believe that candidate screening is a complex, human-dependent and human-centered process that should not be left only to ADM systems.

%%
\begin{acks}
    Jose M. Alvarez and Salvatore Ruggieri received funding from the European Union’s Horizon 2020 research and innovation program under Marie Sklodowska-Curie Actions (grant agreement number 860630) for the project "NoBIAS - Artificial Intelligence without Bias". 
    Antonio Mastropietro and Salvatore Ruggieri received funding from the European Union’s Horizon Europe (grant agreement number 101070212) for the FINDHR project.
    This work reflects only the authors' views and the European Research Executive Agency is not responsible for any use that may be made of the information it contains.
    No other funding was received by the authors.
\end{acks}

%%
% \newpage
\bibliographystyle{ACM-Reference-Format}
\bibliography{references}

%%
\newpage
\appendix
\appendices
\section{The Proof of Proposition \ref{prop2}}
\label{appa}
For the jointly Gaussian random vectors $\bm{x}$ and $\bm{y}$, we have
\begin{equation}
\begin{aligned}
&    \left[\begin{matrix}\bm{x}\\\bm{y}\\\end{matrix}\right] \sim \mathcal{N}\left(\left[\begin{matrix}\bm{\mu}_x\\\bm{\mu}_y\\\end{matrix}\right],\left[\begin{matrix}A&C\\C^T&B\\\end{matrix}\right]\right) \\
& = \mathcal{N}\left(\left[\begin{matrix}\bm{\mu}_x\\\bm{\mu}_y\\\end{matrix}\right],\left[\begin{matrix}\widetilde{A}&\widetilde{C}\\{\widetilde{C}}^T&B\\\end{matrix}\right]^{-1}\right)
\end{aligned}
\end{equation}
then the marginal and conditional distribution of $\bm{x}$ are shown as follows according to \cite{williams2006gaussian}.
\begin{equation}
    \bm{x} \sim \mathcal{N}\left(\bm{\mu}_x,A\right)
\end{equation}
% and
\begin{equation}
\label{app2-1}
    \bm{x}|\bm{y} \sim \mathcal{N}\left(\bm{\mu}_x+CB^{-1}\left(\bm{y}-\bm{\mu}_y\right),A-CB^{-1}C^T\right)
\end{equation}
% or
\begin{equation}
\label{app2-2}
    \bm{x}|\bm{y} \sim \mathcal{N}\left(\bm{\mu}_x-{\widetilde{A}}^{-1}\widetilde{C}\left(\bm{y}-\bm{\mu}_y\right),{\widetilde{A}}^{-1}\right)
\end{equation}

Thus, \textbf{Proposition \ref{prop2}} is proved.










\section{The Proof of Proposition \ref{prop3}}
\label{appb}
The product of two Gaussian distributions is represented as
\begin{equation}
\mathcal{N}\left(\bm{x}\middle|\bm{a},A\right)\mathcal{N}\left(\bm{x}\middle|\bm{b},B\right)=Z^{-1}\mathcal{N}\left(\bm{x}\middle|\bm{c},C\right)
\end{equation}
where
\begin{equation}
\label{app4}
    \bm{c}=C\left(A^{-1}\bm{a}+B^{-1}\bm{b}\right)
\end{equation}
\begin{equation}
\label{app5}
    C=\left(A^{-1}+B^{-1}\right)^{-1}
\end{equation}
\begin{equation}
\label{app6}
    Z^{-1}=\left(2\pi\right)^{-\frac{D}{2}}\left|A+B\right|^{-\frac{1}{2}}\exp{\left(-\frac{\left(\bm{a}-\bm{b}\right)^T\left(\bm{a}-\bm{b}\right)}{2\left(A+B\right)}\right)}
\end{equation}

Thus, through multiplying the cavity distribution by $t_i$ from (\ref{11}), \textbf{Proposition \ref{prop3}} is proved.


\section{The Proof of Proposition \ref{prop4}}
\label{appc}
Consider
\begin{equation}
\label{app7}
Z=\int_{-\infty}^{\infty}{\Phi\left(\frac{x-m}{v}\right)\mathcal{N}(x|\mu,\sigma^2)dx}
\end{equation}
% where
% \begin{equation}
%     \Phi\left(x\right)=\int_{-\infty}^{x}{\mathcal{N}\left(y\right)dy}
% \end{equation}
When $v>0$, by combining$ z=y-x+\mu-m$ and $w=x-\mu$ we can get
\begin{equation}
\begin{aligned}
& Z_{v>0}=\frac{\int_{-\infty}^{\infty}\int_{-\infty}^{x}\exp{\left(-\frac{\left(y-m\right)^2}{2v^2}-\frac{\left(x-\mu\right)^2}{2\sigma^2}\right)}}{2\pi\sigma v}dydx \\
& =\frac{\int_{-\infty}^{\mu-m}\int_{-\infty}^{\infty}\exp{\left(-\frac{\left(z+w\right)^2}{2v^2}-\frac{w^2}{2\sigma^2}\right)}}{2\pi\sigma v}dwdz
\end{aligned}
\end{equation}
% and
\begin{equation}
\begin{aligned}
& Z_{v>0} \\
& =\frac{\int_{-\infty}^{\mu-m}\int_{-\infty}^{\infty}\exp{\left(-\frac{1}{2}\left[\begin{matrix}w\\z\\\end{matrix}\right]^T\left[\begin{matrix}\frac{1}{v^2}+\frac{1}{\sigma^2}&\frac{1}{v^2}\\\frac{1}{v^2}&\frac{1}{v^2}\\\end{matrix}\right]\left[\begin{matrix}w\\z\\\end{matrix}\right]\right)}}{2\pi\sigma v}dwdz \\
& =\int_{-\infty}^{\mu-m}\int_{-\infty}^{\infty}\mathcal{N}\left(\left[\begin{matrix}w\\z\\\end{matrix}\right]|\mathbf{0},\left[\begin{matrix}\sigma^2&-\sigma^2\\-\sigma^2&v^2+\sigma^2\\\end{matrix}\right]\right)dwdz
\end{aligned}
\end{equation}
According to (\ref{app2-1}) and (\ref{app2-2}), we can get
\begin{equation}
\label{app11}
    Z_{v>0}=\frac{\int_{-\infty}^{\mu-m}\exp{\left(-\frac{z^2}{2\left(v^2+\sigma^2\right)}\right)}dz}{\sqrt{2\pi(v^2+\sigma^2)}}=\Phi\left(\frac{\mu-m}{\sqrt{v^2+\sigma^2}}\right)
\end{equation}
When $v<0$, by combining $\Phi\left(-z\right)=1-\Phi\left(z\right)$ and (\ref{app7}),
% we can obtain
\begin{equation}
\label{app12}
Z_{v<0}=1-\Phi\left(\frac{\mu-m}{\sqrt{v^2+\sigma^2}}\right)=\Phi\left(-\frac{\mu-m}{\sqrt{v^2+\sigma^2}}\right)
\end{equation}

By collecting (\ref{app11}) and (\ref{app12}), we can get
\begin{equation}
\label{app13}
Z=\int\Phi\left(\frac{x-m}{v}\right)\mathcal{N}\left(x\middle|\mu,\sigma^2\right)dx=\Phi\left(z\right)
\end{equation}
where $z=\frac{\mu-m}{v\sqrt{1+\sigma^2/v^2}} (v\neq0)$. 
% We aim to get the moments of
% \begin{equation}
% q\left(x\right)=Z^{-1}\Phi\left(\frac{x-m}{v}\right)\mathcal{N}\left(x\middle|\mu,\sigma^2\right)
% \end{equation}
By differentiating with respect to $\mu$ on (\ref{app13}), we can obtain
\begin{equation}
\begin{aligned}
& \frac{\partial Z}{\partial\mu}=\int{\frac{x-\mu}{\sigma^2}\Phi\left(\frac{x-m}{v}\right)}\mathcal{N}\left(x\middle|\mu,\sigma^2\right)dx =\frac{\partial}{\partial\mu}\Phi\left(z\right) \\
& \Longleftrightarrow \frac{1}{\sigma^2}\int x\Phi\left(\frac{x-m}{v}\right)\mathcal{N}\left(x\middle|\mu,\sigma^2\right)dx-\frac{\mu Z}{\sigma^2} \\
& =\frac{\mathcal{N}(z)}{v\sqrt{1+\sigma^2/v^2}}
\end{aligned}
\end{equation}
where $\partial\Phi\left(z\right)/\partial\mu=\mathcal{N}(z)\partial z/\partial\mu$ is utilized. Multiplying through by $\sigma^2/Z$, (\ref{app16}) is obtained.
\begin{equation}
\label{app16}
\mathbb{E}_q\left[x\right]=\mu+\frac{\sigma^2\mathcal{N}\left(z\right)}{\Phi\left(z\right)v\sqrt{1+\frac{\sigma^2}{v^2}}}
\end{equation}
Similarly, we can obtain the second moment as
\begin{equation}
\label{app17}
\begin{aligned}
 & \frac{\partial^2Z}{\partial\mu^2} \\
 & =\int{[\frac{x^2}{\sigma^4}-\frac{2\mu x}{\sigma^4}+\frac{\mu^2}{\sigma^4}-\frac{1}{\sigma^2}] \Phi\left(\frac{x-m}{v}\right)\mathcal{N}\left(x\middle|\mu,\sigma^2\right)} dx  \\
 & =-\frac{z\mathcal{N}(z)}{v^2+\sigma^2} \Longleftrightarrow \\
 & \mathbb{E}_q\left[x^2\right]=2\mu\mathbb{E}_q\left[x\right]-\mu^2+\sigma^2-\frac{\sigma^4z\mathcal{N}\left(z\right)}{\Phi\left(z\right)\left(v^2+\sigma^2\right)}
\end{aligned}
\end{equation}
By combining (\ref{app16}) and (\ref{app17}), we can get
\begin{equation}
\begin{aligned}
& \mathbb{E}_q\left[{(x-\mathbb{E}_q\left[x\right])}^2\right]=\mathbb{E}_q\left[x^2\right]-\mathbb{E}_q[x]^2 \\
& =\sigma^2-\frac{\sigma^4\mathcal{N}\left(z\right)}{\left(v^2+\sigma^2\right)\Phi\left(z\right)}\left(z+\frac{\mathcal{N}\left(z\right)}{\Phi\left(z\right)}\right)
\end{aligned}
\end{equation}

Thus, \textbf{Proposition \ref{prop4}} is proved.

\section{The Proof of Proposition \ref{prop5}}
\label{appd}
We can obtain (\ref{19-1}), (\ref{19-2}), and (\ref{19-3}) according to (\ref{app4}), (\ref{app5}), and (\ref{app6}). Hence, \textbf{Proposition \ref{prop5}} is proved.



\section{The Proof of Proposition \ref{prop6}}
\label{appe}
The approximated mean for $f_\ast$ can be denoted as
\begin{equation}
\begin{aligned}
& \mathbb{E}_q\left[f_\ast|X,\bm{y},\bm{x}_\ast\right]=\bm{k}_\ast^TK^{-1}\bm{\mu} \\
& =\bm{k}_\ast^TK^{-1}\left(K^{-1}+{\widetilde{\Sigma}}^{-1}\right)^{-1}{\widetilde{\Sigma}}^{-1}\widetilde{\bm{\mu}} \\
& =\bm{k}_\ast^T\left(K+\widetilde{\Sigma}\right)^{-1}\widetilde{\bm{\mu}}
\end{aligned}
\end{equation}

The variance of $f_\ast|(X,\bm{y})$ under the Gaussian approximation can be denoted as
\begin{equation}
\begin{aligned}
& \mathbb{V}_q\left[f_\ast\middle| X,\bm{y},\bm{x}_\ast\right] = \mathbb{E}_{p(f_\ast|X,\bm{x}_\ast,\bm{f})} {f_\ast-\mathbb{E}[f_\ast|X,\bm{x}_\ast,\bm{f}]}^2 \\
& =k\left(\bm{x}_\ast,\bm{x}_\ast\right)-\bm{k}_\ast^TK^{-1}\bm{k}_\ast+\bm{k}_\ast^TK^{-1}\left(K^{-1}+\widetilde{\Sigma}\right)^{-1}K^{-1}\bm{k}_\ast \\
& =k\left(\bm{x}_\ast,\bm{x}_\ast\right)-\bm{k}_\ast^T\left(K^{-1}+\widetilde{\Sigma}\right)^{-1}\bm{k}_\ast
\end{aligned}
\end{equation}

Then, we can obtain
\begin{equation}
\begin{aligned}
& q\left(y_\ast\middle| X,\bm{y},\bm{x}_\ast\right)=\mathbb{E}_q\left[\pi_\ast|X,\bm{y},\bm{x}_\ast\right] \\
& =\int\Phi\left(f_\ast\right)q\left(f_\ast\middle| X,\bm{y},\bm{x}_\ast\right)df_\ast
\end{aligned}
\end{equation}

According to (\ref{app11}), we can obtain
\begin{equation}
\label{app22}
\begin{aligned}
& q\left(y_\ast\middle| X,\bm{y},\bm{x}_\ast\right) \\
& =\Phi\left(\frac{\bm{k}_\ast^T\left(K+\widetilde{\Sigma}\right)^{-1}\widetilde{\bm{\mu}}}{\sqrt{1+k\left(\bm{x}_\ast,\bm{x}_\ast\right)-\bm{k}_\ast^T\left(K+\widetilde{\Sigma}\right)^{-1}\bm{k}_\ast}}\right)
\end{aligned}
\end{equation}

By combining (\ref{13}) and (\ref{app22}), \textbf{Proposition \ref{prop6}} is proved.




\section{The Proof of Proposition \ref{prop7}}
\label{appf}
Given $f_s$ and $f_\ast$, $y_s$ and $y_\ast$ are conditionally independent. Hence, $p\left(y_s,y_\ast\middle|\bm{x}_s,\bm{x}_\ast\right)$ can be represented as
\begin{equation}
\begin{aligned}
& p\left(y_s=1,y_\ast=1\middle|\bm{x}_s,\bm{x}_\ast\right) \\
& =\iint{\Phi\left(f_s\right)\Phi\left(f_\ast\right)\phi\left(f_s,f_\ast\middle|\mu_{s\ast},\Sigma_{s\ast}\right)}df_sdf_\ast \\
& =\iint{\Phi\left(f_\ast\right)\phi\left(f_\ast\middle|{\widetilde{\mu}}_\ast\left(f_s\right),{\widetilde{\sigma}}_{\ast\ast}\right)df_\ast\Phi\left(f_s\right)}\phi\left(f_s\middle|\mu_s,\sigma_{ss}\right)df_s \\
& =\int\Phi\left(\frac{{\widetilde{\mu}}_\ast\left(f_s\right)}{\sqrt{{\widetilde{\sigma}}_{\ast\ast}+1}}\right)\Phi\left(f_s\right)\phi\left(f_s\middle|\mu_s,\sigma_{ss}\right)df_s
\end{aligned}
\end{equation}

Hence, \textbf{Proposition \ref{prop7}} is proved.

% \section{The Proof of Lemma \ref{lem}}
% \label{appg}
% \begin{equation}
% \begin{aligned}
% & R_e=\frac{1}{N_a}\sum_{n=1}^{N_a}\mathbb{I}\left(\bm{L}_n \neq \bm{Y}_n\right) \\
% & =\displaystyle\frac{FA+FL}{TL+TA+FL+FA} \\
% & =\displaystyle\frac{1}{\displaystyle\frac{TL+TA+FL+FA}{FA+FL}} \\
% & =\displaystyle\frac{1}{1+\displaystyle\frac{TL+TA}{FA+FL}} \\
% & =\displaystyle\frac{1}{1+\displaystyle\frac{\displaystyle\frac{TL}{TA}+1}{\displaystyle\frac{FA}{TA}+\displaystyle\frac{FL}{TA}}} \\
% & =\frac{1}{1+\displaystyle\frac{\displaystyle\frac{TL}{TA}+1}{\displaystyle\frac{1}{P_{md}-1}+\displaystyle\frac{1}{P_{fa}-1}}}
% \end{aligned}
% \end{equation}

% Hence, \textbf{Lemma \ref{lem}} is proved.

%%
\end{document}

%
% EOF
%
