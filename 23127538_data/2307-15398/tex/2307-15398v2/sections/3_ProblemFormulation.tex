%
\label{sec:ProblemFormulation}

We now describe our setup in its most basic form. The goal is to formulate the set selection problem, where a decision-maker selects a set of items from a population, by considering the initial order in which the items are presented to the decision-maker. Here, the candidates for a job position represent the items and the screener evaluating their profiles represents the decision-maker.  

\subsection{Setting}
\label{sec:ProblemFormulation.Setting}

% The candidate pool
We consider a \textit{candidate pool} $\mathcal{C}$ of $n$ candidates. 
Each \textit{candidate} $c$ is described by the \textit{vector of $p$ attributes} $\mathbf{X}_c \in \mathbb{R}^{p}$ and the \textit{protected attribute} $W_c$. 
To simplify our analysis, we assume that $W$ encodes the membership to the protected group and is, thus, binary: with $W_c = 1$ if $c$ belongs to the protected group and $W_c = 0$ otherwise. 
The underlying protected attribute can be discrete, continuous, or it may consists of multiple attributes.

% The screener
The candidates are evaluated by a \textit{screener} $h \in \mathcal{H}$, where $\mathcal{H}$ denotes the set of available screeners. 
The following variables refer to a specific $h$.
The screener is tasked with choosing $k$ candidates from $\mathcal{C}$ based on each candidate's application profile as summarized by the tuple $(\mathbf{X}_c, W_c)$. 
The goal of the screener is to obtain a \textit{set of $k$ selected candidates} $S^k \in [\mathcal{C}]^k$, where $[\mathcal{C}]^k$ denotes the set of $k$-subsets of $\mathcal{C}$, i.e.,~subsets with cardinality $k$.
We model candidate evaluation by assuming that the screener uses an \textit{individual scoring function} $s \colon \mathbb{R}^{p} \to [0, 1]$, such that, given a candidate $c$, $s(\mathbf{X}_c)$ returns the score of $c$ according to $h$. 
It is implied that the screener does not use the protected attribute when scoring candidates. 
The higher the score, the better the candidate fits the job position based on the screener's judgment. 

% All possible orders and the initial order
The screener explores the candidate pool $\mathcal{C}$ in a specific order.
% when evaluating the candidates. 
We denote the set of total ordering of candidates in $\candidatesset$ by $\Theta$. 
It represents all possible ways in which the $n$ candidates in $\candidatesset$ can be arranged. 
An \textit{order} $\sigma \in \Theta$ maps an integer $i \in \{1, \dots n\}$ to a candidate $c \in \candidatesset$, indicating that $c$ occupies the $i$-th position according to $\sigma$, with notation $\sigma(i) = c$ and vice-versa $\sigma^{-1}(c) = i$. 
We are interested in the \textit{initial order} $\theta \in \Theta$, which refers to the order chosen by the screener before starting the exploration of $\candidatesset$ and the evaluation of their scores, according to fact G1 in Section~\ref{sec:Generali}. 
The screener is not required to explore the entirety of the candidate pool $\mathcal{C}$, according to fact G2 in Section~\ref{sec:Generali}.
However, we assume that the screener \textit{respects} the initial order $\theta$. 
Formally:
%
\begin{equation}
\label{eq:order}
    \mbox{a candidate $c_1 \in \candidatesset$ is evaluated before $c_2 \in \candidatesset$ only if $\theta^{-1}(c_1) < \theta^{-1}(c_2)$}
\end{equation}
%

% On meaningfulness
%\textcolor{red}{
%We say that $\theta$ is non-informative to $h$. 
%We assume that only after choosing $\theta$, and subsequently having searched, evaluated, and sorted the entirety of $\mathcal{C}$ as prescribed by $\theta$, can $h$ be aware of any meaning behind $\theta$. 
%This assumption ensures that $h$ does not knowingly introduce bias by arranging $\candidatesset$ into a specific $\theta$. 
%}
% All initial orders are meaningless to the set of screeners $\mathcal{H}$.
% On m steps travelled 
%The screener is not required to explore the entirety of the candidate pool $\mathcal{C}$.\footnote{This condition is based on fact G2 in Section~\ref{sec:Generali}.} 
%We denote as $\candidatessubset \subseteq \candidatesset$, the \textit{subset of candidates whose profiles are screened by} $h$, with cardinality $m \leq n$, implying the case where $h$ prefers not to evaluate the complete $\candidatesset$.
%For $\theta \in \Theta$, we refer to $\theta^{(m)}$ as \textit{the total order restricted to first $m$ elements according to the initial order}, so that $\theta^{(m)}$ maps an integer $i \in \{1, \dots m\}$ to a candidate $c \in \candidatessubset$.
% For a given $\sigma \in \Theta$, we refer to $\sigma^{(m)}$ as the total order restricted to the first $m$ elements, so that $\sigma^{(m)}$ maps an integer $i \in \{1, \dots m\}$ to a candidate $c \in \candidatesset_h$.

\subsection{Group Fairness}
\label{sec:ProblemFormulation.Fairness}

We address the fairness goals of the screener $h$ in choosing the set of $k$ candidates by assuming a quota for the protected group $W = 1$. 
Let us introduce the fraction $f\big( S^k \big) \in [0, 1]$ of protected candidates in the selected set $S^k$:
%
\begin{equation}
\label{eq:fairness_function}
    f\big( S^k \big) = \frac{\left\vert\{c \in S^k \text{ s.t. } W_c = 1\}\right\vert}{k}
\end{equation}

% \begin{equation}
% \label{eq:fairness_function}
%     f\big( S^k \big) = \left\vert\{c \in S^k \text{ s.t. } W_c = 1\}\right\vert / k
% \end{equation}
%
Let $q \in [0, 1]$ denote the desired fraction of chosen protected candidates in $S^k$. 
The fair screener will then be constrained to meet the \textit{representational quota} $q$ when deriving $S^k$, i.e.,~to satisfy the condition $f\big( S^k \big) \geq q$. 
Notice that the unconstrained version, in which no requirement is assumed on the representativeness of the protected group in $S^k$, is simply achieved by considering $q=0$.

We view $q$ as a policy implemented by the screener to achieve a diverse set of selected candidates. Hence, it is 
% We stress that $q$ is
a statement on the composition of $S^k$, not a statement on the ordering of protected candidates within $S^k$, according to fact G4 in Section~\ref{sec:Generali}.
% \footnote{Here, $q$ is based on fact G4 in Section~\ref{sec:Generali} as G's HR officers were motivated by company diversity goals, not by the fairness literature.} 
For $k=10$ and $q=0.5$, e.g., the fair screener would need to derive $S^k$ with $50\%$ of protected candidates though in no particular order, such as requiring the first five candidates in $S^k$ to be protected candidates.


\subsection{Set Selection: Two Problem Formulations}
\label{sec:ProblemFormulation.Objectives}

We now proceed to formulate two set selection problems for the screener $h$ tasked with deriving the set $S^k \subseteq \candidatesset$, where 
we distinguish between best-$k$ and good-$k$. 
Under the \textit{best-}$k$ formulation, the set $S^k$ represents \textit{the fair best $k$ candidates} in $\candidatesset$ according to $h$. 
We denote it accordingly as $S^k_{\besttext}$. 
% This is the standard formulation in the fair ranking literature~\cite{Zehlike2023_FairRanking_P1, Zehlike2023_FairRanking_P2}. 
Under the \textit{good-}$k$ formulation, the set $S^k$ represents \textit{the fair first good-enough $k$ candidates} in $\candidatesset$ according to $h$. 
We denote it accordingly as $S^k_{\goodtext}$. 
% This formulation, to the best of our knowledge, has been overlooked by the fair ranking literature.
For both formulations, we define the objective of the screener in terms of achieving an optimal and fair selection of candidates. 
How we define optimality, as we will show, derives the best-$k$ and good-$k$ formulations; we already defined fairness in the previous subsection. 

\paragraph{Best-$k$.}
We first focus on the screener that finds the set of best $k$ candidates in the candidate pool $\candidatesset$ given the fairness constraint $q$
and while respecting the initial order $\theta$ (recall (\ref{eq:order})).
%Noteworthy, here $h$ has to search the whole $\candidatesset$, meaning $\am{\candidatessubset} = \candidatesset$ and $m=n$ in this setting.
Notice that, here, $h$ must evaluate the complete $\candidatesset$.
This is because $h$ must score all candidates according to the individual scoring function $s$ before choosing the ones with the highest scores and for which $q$ is satisfied.
%meaning $h$ evaluates each candidate and sorts them by their scores, in order to speak of the best candidates.
%
%% SR
%Formally, the ranking $\tau \in \Theta$ represents an ordering of $\candidatesset$ that follows the scoring function $s$, such that, if $i \leq j$, $s(\mathbf{X}_{\tau(i)}) \geq s(\mathbf{X}_{\tau(j)})$.
%Given $\tau$, the cut-off for being selected into $S^k$ is $s(X_{\tau(k)})$ with $\tau(k)$ denoting the $k^{th}$ position in the ranking $\tau$.
%We can then model the goal of the best-$k$ selection, without fairness requirements, as:
%
%\begin{equation}
%\label{eq:objective_best-k}
%    S_{\tau}^k = \{\tau(1), \dots, \tau(k)\}
%\end{equation}
% 
We view the goal in terms of maximizing a utility for the screener $h$. 
We define \textit{utility} as the benefit derived by $h$ from selecting $k$ suitable candidates  given $\theta$.
% We define \textit{utility} as the benefit derived by $h$ from selecting $k$ suitable candidates for the job while respecting $\theta$ (recall (\ref{eq:order})). 
Formally, utility is a function $U^k \colon [\mathcal{C}]^k \, \times \, \Theta \to \mathbb{R}$. 
%Note that, in general, the utility function evaluated on a subset $S^k$ of candidates is influenced by the initial order $\theta$ by which the same candidates are seen. 
The simplest expression for defining $U^k$ is to add the scores of the chosen candidates:
%
\begin{equation*}
\label{eq:Utility}
    U^k_{\addtext} \big( S^k, \theta \big) = \sum_{c \in S^{k}} s\big( \mathbf{X}_{c} \big)
\end{equation*}
%
rationalizing that $h$ will maximize its utility by selecting the $k$ most suitable candidates for the job position. 
%It is easy to see that $S_{\tau}^k$ maximizes the utility $U^k_{\text{add}}$.
Note that
% , in the above utility definition, 
the initial order $\theta$ in \eqref{eq:Utility} does not affect the evaluation of $S^k$ because of the commutative property of addition.
We emphasize that \eqref{eq:Utility} is not the only possible model for the utility of the screener. 
Alternative models, such as exposure discounting \cite{DBLP:conf/kdd/SinghJ18}, can be considered for the best-$k$ problem formulation.
We leave this for future work.
%The goal \eqref{eq:objective_best-k} and utility \eqref{eq:Utility} describe the best-$k$ set selection problem for $h$. 
%If the screener $h$ is fair, then it must meet the representational quota $q$ of the protected group \eqref{eq:fairness_function}. 
 
We define \textit{the fair best-$k$ set selection problem} (or, simply, the best-$k$ problem) as:
%
\begin{equation}
\label{eq:fair_objective_all_screener}
    \begin{aligned}
    \argmax_{S^k \in [\mathcal{C}]^k} & \quad U^k_{\addtext} \big(S^k, \theta\big) \\
    \textrm{s. t.} & \quad f(S^k) \geq q
    \end{aligned}
\end{equation}
%
which, as readers familiar with the literature might notice, describes the standard top-$k$ formulation in fair ranking problems with $q$ representing some group-level fairness quota \cite{Zehlike2023_FairRanking_P1, Zehlike2023_FairRanking_P2}. 
We denote the
% \footnote{In presence of ties in scores, the solution may not be unique. In such a case, we consider any solution.} 
solution of \eqref{eq:fair_objective_all_screener} as $S^k_{\besttext}$.
In presence of ties in scores, the solution may not be unique. 
In such a case, we consider any solution.

\paragraph{Good-$k$.}
We now focus on the screener $h$ that finds $k$ candidates in the candidate pool $\candidatesset$ that meet a \textit{minimum basic requirements} $\psi$ given the fairness constraint $q$ and while respecting the initial order $\theta$.
% \footnote{This setting is based on facts G2 and G3 in Section~\ref{sec:Generali}.}
Unlike the best-$k$ formulation, here $h$ is not required to evaluate the whole $\candidatesset$ as it is enough to find the first $k$ candidates that are good-enough according to $\psi$ and that satisfy $q$. 
%Hence, the goal \eqref{eq:objective_best-k} and utility \eqref{eq:Utility} from the standard set selection formulation are not suitable in this setting.
%
We represent $\psi$ as a minimum score, such that $h$ deems candidate $c \in \candidatesset$ as eligible, or good-enough, for being selected if $s(\mathbf{X}_c) \geq \psi$.
%In the setting without fairness, unlike the goal of best-$k$, $S_{\tau}^k$ \eqref{eq:objective_best-k}, we cannot speak of a cut-off for the goal of good-$k$, because there is no ranking $\tau$ to slice from.

Under the good-$k$ setting, clearly the initial order $\theta \in \Theta$ has a significant influence on the screening process. 
To observe this point, let $k=1$ and assume $s(\mathbf{X}_{c_1}) \geq \psi$ and $s(\mathbf{X}_{c_2}) \geq \psi$. An initial order such that $\theta^{-1}(c_1) = 1$ and $\theta^{-1}(c_2) = 2$ would imply by (\ref{eq:order}) that $c_1$ is considered eligible before even evaluating $c_2$. Conversely, the reverse initial order $\theta^{-1}(c_1) = 2$ and $\theta^{-1}(c_2) = 1$ would consider eligible $c_2$ before $c_1$. Since the screener will stop after finding $k=1$ good enough candidates, the initial order affects which candidates are selected.

%let us call $\mathcal{C}_{\psi}$ the set of eligible candidates according to the minimum basic requirements $\psi$. 
%Evidently, we need to assume that $|\mathcal{C}_{\psi}| \geq k$.
%It is useful to denote $\Theta_{\psi}$ as the set of total orderings of $\mathcal{C}_{\psi}$.
%Under this view, $\theta$ induces a total order $\theta_{\psi} \in \Theta_{\psi}$ such that, given two candidates $c', c'' \in \candidatesset_{\psi}$, then $\theta_{\psi}^{-1}(c') < \theta_{\psi}^{-1}(c'')$ if $\theta^{-1}(c') < \theta^{-1}(c'')$.
%In this case, without the fairness condition, it helps to indicate the subset of $\candidatesset_{\psi}$ containing the first $k$ suitable candidates according to $\theta_{\psi}$ as $S^k_{\psi}$.
%Formally:
%
%\begin{equation}
%\label{eq:S_k_psi}
%    S^k_{\psi} = \{\theta_{\psi}^{-1}(i) \mid i = 1, \dots, k\}
%\end{equation}
%
%It is possible for two screeners $h_1$ and $h_2$ that chose different initial orders $\theta_1$ and $\theta_2$, under the same $s$ and $\psi$, to obtain different selected sets. 
%
%For this last scenario to hold, though, it must be unappealing, utility-wise, for both $h_1$ and $h_2$ to search the entirety of $\candidatesset$ according to $\theta_1$ or $\theta_2$.
%The notion of any $h$ optimally choosing $k$ candidates acquires a different meaning in the good-$k$ setting since the utility function $U^k_{\addtext}$ as described in \eqref{eq:Utility} is inadequate to formalize a screener for whom a selected set of $k$ good enough candidates is sufficient. 
%We define the simplest expression for the utility $U^k_{\psi}$ motivating the screener in the good-$k$ setting:
We still view the goal in terms of maximizing a utility for the screener $h$.
As a possible utility function in the good-$k$ setting, we first consider:
%
\begin{equation*}
\label{eq:AlternativeUtility2}
    \hat{U}^k_{\psi}\big( S^k, \theta \big) = \left\{
    \begin{array}{ll}
        n - \max_{c \in S^k} \theta^{-1}(c) & \text{if} \  \forall c \in S^k \  s(\mathbf{X}_c) \geq \psi   \\
        0 & \text{otherwise.}
    \end{array} \right.
\end{equation*}
%
Intuitively, in the above definition, the screener wants to find as quickly as possible a set of $k$ eligible candidates.
Therefore, if $S^k$ contains only eligible candidates, the utility of $h$ selecting $S^k$ under $\theta$ is expressed by the number of candidates past the last one who was screened, i.e.,~the ``saved effort" of the screener $h$.

Despite the simplicity of the above definition, the utility function (\ref{eq:AlternativeUtility2}) is not suitable to properly model our intended problem.
To observe this point, let $n=3, k=2, q=0.5$. 
Assume three eligible candidates and $\theta(1) = c_1, \theta(2) = c_2, \theta(3) = c_3$ with $W_1 = W_2 = 0$ and $W_3 = 1$. It turns out that both $S' = \{c_1, c_3\}$ and $S'' = \{c_2, c_3\}$ maximize the utility and satisfy the fairness constraint. However, why should have been $c_2$ considered, and then returned in $S''$, if $c_1$ already meets the minimum basic requirement? 
A reason for doing that is a variant of our problem, in which the screener keeps evaluating non-protected candidates, even if their quota is reached but the one of protected candidates is not yet reached, for the purpose of keeping the best ones found so far. 
We do not consider such a variant in this paper.

Let us define the \textit{penalty function} $p(c, S^k, \theta) = \mathbb{1}(\exists\ c' \in \mathcal{C}\setminus S^k \mbox{s.t.}\ \theta^{-1}(c') < \theta^{-1}(c) \wedge s(\mathbf{X}_{c'}) \geq \psi \wedge W_{c'}=W_c)$ which, for a candidate $c$, looks for another candidate of the same group as $c$ and meeting the minimum basic requirement, who occurs before $c$ in the order $\theta$, but who has not been selected into $S^k$. 
Basically, the penalty function models the ``wasted effort" in choosing a candidate occurring after another one meeting all the same requirements. 
At worst, the are $k$ penalties, which leads to the following refined utility function:
%
\begin{equation*}
\label{eq:AlternativeUtility}
    U^k_{\psi}\big( S^k, \theta \big) = \left\{
    \begin{array}{ll}
        k - \sum_{c \in S^k} p(c, S^k, \theta) & \text{if} \  \forall c \in S^k \  s(\mathbf{X}_c) \geq \psi   \\
        0 & \text{otherwise.}
    \end{array} \right.
\end{equation*}
%
%
%index of the last candidate to be evaluated, according to $\theta$, to complete the set of $k$ eligible candidates.
%The fewer candidates $h$ are evaluated to obtain $S^k$, the more utility the screener gets from the selection.
%Otherwise, if $S^k$ contains at least one non-eligible candidate, the utility of selecting $S^k$ is a zero.
%Evidently, $S^k_{\psi}$ of \eqref{eq:S_k_psi} maximizes the above utility, without including the fairness requirement.
% We come back to this utility function at the end of this section.

%If we include the fairness constraint, we 
We define then \textit{the fair good-$k$ set selection problem} (or, simply, the good-$k$ problem) as:
%
\begin{equation}
\label{eq:fair_objective_U_psi}
    \begin{aligned}
    \argmax_{S^k \in [\mathcal{C}]^k} & \quad U^k_{\psi} \big(S^k, \theta\big) \\
    \textrm{s. t.} & \quad f(S^k) \geq q
    \end{aligned}
\end{equation}
%
We denote the solution of (\ref{eq:fair_objective_U_psi}) as $S_{\goodtext}^k(\psi)$ or, if there is no ambiguity on $\psi$, simply as $S_{\goodtext}^k$. 
Note that,
if the fairness constraint is strengthened to a fixed quota, i.e.,~$f(S^k) = q$, it can be shown that the solution is unique. 
In the general case, i.e.,~$f(S^k) \geq q$, there can be two solutions, but with different fractions of the protected group. 
%%%v
% JA: if needed, move example to Appendix
%%%
For example, consider $n=3, k=2, q=0.5$. Assume three eligible candidates and $\theta(1) = c_1, \theta(2) = c_2, \theta(3) = c_3$ with $W_1 = 0$ and $W_2 = W_3 = 1$. Both $S' = \{c_1, c_2\}$ and $S'' = \{c_2, c_3\}$ are solutions of (\ref{eq:fair_objective_U_psi}) with $U^k_{\psi} \big(S', \theta\big) = U^k_{\psi} \big(S'', \theta\big) = 2$. However, $f(S') = 0.5$ and $f(S'') = 1$. Intuitively, $S'$ is obtained by strictly iterating over $\theta$, which is the approach we take in Section~\ref{sec:ProblemFormulation.Algorithms}. 
%%%^
% JA: if needed, move example to Appendix
%%%

%$s(\mathbf{X}_{c_1}) \geq \psi$ and $s(\mathbf{X}_{c_2}) \geq \psi$. An initial order were $\theta^{-1}(c_1) = 1$ and $\theta^{-1}(c_2) = 2$ would imply by (\ref{eq:order}) that $c_1$ is considered eligible before even evaluating $c_2$. Conversely, the reverse initial order $\theta^{-1}(c_1) = 2$ and $\theta^{-1}(c_2) = 1$ would consider eligible $c_2$ before $c_1$. Since the screener will stop after finding $k=1$ good enough candidates, the initial order affects which candidates are selected.

%This fact is useful for unifying both good-$k$ and best-$k$ settings into a single problem formulation.
%To highlight how distinct this setting is relative to the standard best-$k$, let us formulate the good-$k$ setting without needing to resort to specifying a goal or utility function.
%We will return to the above utility function at the end of this section.

%We argue that the screener $h$ can settle for looking through a subset $\candidatessubset$ of cardinality $m \leq n$ as long as it contains enough $k$ eligible candidates so that it also holds $m \geq k$. 
%Moreover, $\candidatessubset$ is composed by the set of the first $m$ candidates of the initial order $\theta$, so we can state that $\candidatessubset = \{c \in \candidatesset \mid \theta^{-1}(c) \leq m\}$. 
%Here, $\theta$ induces an ordering of $\candidatessubset$ captured by the initial order restricted to its first $m$ elements $\theta^{(m)}$. 
%Once the selected set has cardinality $k$, the screener has no incentive to look for more eligible candidates according to $\psi$.
%Hence, without the fairness condition, the selected set is the set of the first $k$ eligible candidates in $\mathcal{C}_{\psi}$ following $\theta_{\psi}$.
%Further, if the screener $h$ is fair, it must meet the representational quota $q$ of the protected group that meets the minimum basic requirements.
%We can restate problem \eqref{eq:fair_objective_U_psi} as:
%
%\begin{equation}
%\label{eq:Alternativefair_objective_all_screener}
%\begin{aligned}
%    \min_{m \in [k, \ldots, n]} \quad & m \\
%    \textrm{s. t.} \quad & |\{ \theta_{\psi}(i) \mid i=1, \ldots, m\}| = k \\
%    \quad & f(S_{\goodtext}^k) \geq q
%\end{aligned}
%\end{equation}
%
%where the selected set is exactly $S_{\goodtext}^k = \{ \theta_{\psi}(i) \mid i=1, \ldots, m\}$, so that it necessarily has cardinality $k$. 

%\paragraph{A fair initial screening order problem formulation.}
%To summarize, we represent both best-$k$ and good-$k$ settings into one formulation for the set selection problem based on the initial order $\theta$:
%
%\begin{equation}
%\label{eq:GeneralISOFormulation}
%    \begin{aligned}
%        \max_{S^k \in \mathcal{S}^k} & \quad U^k(S^k, \theta) \\
%        \textrm{s. t.} & \quad f(S^k) \geq q
%    \end{aligned}
%\end{equation}
%
%where we obtain two solutions for the optimal and fair screener $h$ depending on the utility function specification. 
%Under $U^k_{\addtext}$ as by \eqref{eq:Utility}, the solution is $S_\besttext^k$.
%In this case, $\theta$ is redundant as it does not influence $U^k_{\addtext}$. 
%This is because $h$ will always explore all of $\candidatesset$ and sort it as the ranking $\tau \in \candidatesset$ regardless of $\theta$. 
%Still, $h$ explores $\candidatesset$ as prescribed by $\theta$.
%Under $U^k_{\psi}$ as by \eqref{eq:AlternativeUtility}, the solution is $S_\goodtext^k$. 
%In this case, $\theta$ is a key input to $U^k_{\psi}$, despite not being a parameter that can be optimized by $h$. 
%This is because $h$ can partially or fully explore $\candidatesset$ as prescribed by $\theta$ depending on when it meets its goal.
%With \eqref{eq:GeneralISOFormulation} we present the general initial screening order problem formulation.

% Antonio: I've commented out again as we don't use meaningulness before (I changed that also as you suggested)
% For a screener $h \in \mathcal{H}$, the ordering insider the chosen subset $\mathcal{S}^{k}$ can be meaningful. In principle, $S^{k}_{\besttext}$ leads to a meaningful candidate selection relative to $S^{k}_{\goodtext}$ as candidates are ordered by their scores. In practice, however, it depends on how the chosen subset is used later on in the hiring pipeline, meaning whether in the next phase the order within $\mathcal{S}^k$ carries any information (e.g., interviewing candidates as described by $\mathcal{S}^k$). Hence, the chosen set can be meaningful but it is not of interest here as its meaning is determined beyond the screener $h$.

\subsection{Two Search Algorithms}
\label{sec:ProblemFormulation.Algorithms}

We present two search procedures for the best-$k$ and the good-$k$ problems, respectively. 
% Their names refer to the click models literature discussed in Section~\ref{sec:RelatedWork}. 
% \cite{DBLP:conf/clef/GrotovCMSXR15}.

The \textit{ExaminationSearch} procedure, shown in Algorithm~\ref{algo:Examination}, solves the best-$k$ setting, returning $S^k_\besttext$ for given $n$ (candidates), initial order $\theta$, and parameters $k$ (subset size) and $q$ (minimum fraction of selected candidates from the protected group). 
First, line 2 calculates the minimum number $q^*$ of candidates from the protected group to be selected, and the maximum number of candidates $r^*$ not in that quota. 
Then, candidates are considered by descending scores, using the \texttt{argsortdesc} procedure (lines 2, 3). 
The loop in lines 5-13 iterates until $k$ candidates are found. The loop adds candidates to the sets $Q$ and $R$: $Q$ are candidates in the quota of the protected group; $R$ are candidates not in that quota (can be non-protected or protected). An non-protected candidate can be only added to the $R$ set, thus line 7 checks if there is still room in $R$ to do this. A protected candidate is added to the quota set $Q$ if there is room (lines 10-11), or to the other set $R$ otherwise (lines 12-13). 
Finally, the procedure returns the candidates in the quota set $Q$ or in the other set $R$. 
The result of the \textit{ExaminationSearch} procedure clearly maximize (\ref{eq:fair_objective_all_screener}), as candidates are added in decreasing score, while keeping the fairness constraint through the quota management.

The \textit{CascadeSearch} procedure, shown in Algorithm~\ref{algo:Cascade}, solves the good-$k$ setting, returning $S^k_\goodtext$ for given $n$, initial order $\theta$, and parameters $k$, $q$, and $\psi$ (minimum basic requirement). 
The difference with the \textit{ExaminationSearch} procedure consists in strictly following the initial order $\theta$ (line 4), and in checking the minimum basic requirement (line 8) before adding a candidate to the quota set $Q$ or to the other set $R$.
The result of the \textit{CascadeSearch} procedure maximizes (\ref{eq:fair_objective_U_psi}), as no penalty is accumulated in the loop. 
In fact, an non-protected candidate ($W_c=0$) is not added only because  there is no room in $R$ -- and $R$ never gets smaller to allow for more room later on. A protected candidate ($W_c=1$) is not added only if not meeting the minimum basic requirement (line 8), hence it cannot be counted for the penalty. Formally, $U^k_{\psi}\big(S^k_\goodtext, \theta\big) = k$ for the solution $S^k_\goodtext$ returned by \textit{CascadeSearch}.

%
%\begin{definition}[Examination Search]
%\label{def:ExaminationSearch}
%    Given $\theta$, $h$ explores all of $\candidatesset$, computing each $c$ candidate's score $s(\mathbf{X}_c)$, ranking them all into $\tau$, and choosing the first $k$-candidates accordingly while ensuring that the fairness representational quota $q$ is met. 
%    Algorithm~\ref{algo:Examination} implements this search procedure.
%\end{definition}
%

%
%\begin{definition}[Cascade Search]
%\label{def:CascadeSearch}
%    Given $\theta$, $h$ explores $\candidatesset$ up until reaching $k$-candidates that meet the minimum basic requirements $\psi$ in terms of each candidate's $c$ score $s(\mathbf{X}_c)$ while ensuring that the fairness representational quota $q$ is met. 
%    Algorithm~\ref{algo:Cascade} implements this search procedure.
%\end{definition}
%

%
% Figure environment removed
%

% Both Algorithms~\ref{algo:Examination}~and~\ref{algo:Cascade} implicitly assume there are enough suitable $m$ candidates in $\candidatesset$ according to the minimum basic requirements $\psi$, such that $k \geq m \geq n$. 
% Hence, both achieve the respective solution. 
% These are the simplest algorithms for the examination and cascade searches, which can easily be extended by allowing the screener to update, e.g., $\psi$ of $k$ depending on the size of $S^k$ during the search.

%We highlight the fairness goal of obtaining a representative selected set of $k$ candidates as denoted by $q$.
%In Algorithms~\ref{algo:Examination}~and~\ref{algo:Cascade}, $q^*$, 
% or \texttt{line 2} for both of them, 
%represents the minimum number of protected candidates ($W_c = 1$) to be selected. 
%Hence, $k - q^*$ represents the maximum number of non-protected candidates ($W_c = 0$) to be selected.
%Under \texttt{ExaminationSearch} (Algorithm~\ref{algo:Examination}), meeting this quota is based on searching the full candidate pool according to $\theta$, sorting it by the scores in two separate lists according to $W$, and choosing the $q^*$-best protected candidates from $\tau_0$ and the $(k-q^*)$-best non-protected candidates from $\tau_1$.
%Under \texttt{CascadeSearch} (Algorithm~\ref{algo:Cascade}), meeting this quota goes on a candidate by candidate basis, thus, not requiring to search the full candidate pool according to $\theta$. 
%Here, the algorithm prioritizes constructing $S^k$ with $q^*$ eligible protected candidates $l_1$ while keeping track of the eligible non-protected candidates $l_0$.
%Once the quota is met, if between $l_1$ and $l_0$ there are not enough $k$ candidates then the algorithm keeps searching though without considering group membership. Notice that $l_0$ cannot be greater than $k-q^*$.

%
% EOS
%
