%
\label{sec:Discussion}

% Summary
We presented the initial screening order (ISO); defined two formulations of the set selection problem, best-$k$ and good-$k$; and introduced a human-like screener to study the effects of the ISO as though we were studying a human user for the candidate screening problem.
Our analysis confirms the fairness impact of the ISO, motivated by the risk of position bias, on a chosen set of $k$ candidates. Extensive simulations showed complex relations between best-$k$ and good-$k$, which heavily depend on the score distribution, on their correlation with the ISO, and on the modeling of fatigue.

% Limitations
% Our work is based on a collaboration with company G, and not on a field or observational study, such as \cite{Pisanelli2022_YourCV}, which is why we chose the term ``stylised facts'' to summarize our experience in Section~\ref{sec:Generali}. Our observations of G's candidate screening practices are limited to our interaction with G's AA and HT teams.
In this work, we have made functional assumptions on what defines a human-like screener. 
% The notions of utility and fatigue are open to discussion. 
% Future work should consider alternative, more complex formulations to better capture the human-like screener. 
In particualr, we reduced the ``humanity'' of the screener in terms of fatigue. 
This assumption allowed us to present a simple and intuitive setting to showcase the impact of the ISO. 
Future work should explore alternative formulation for fatigue as well as alternative ISO settings, such as a human-like screener that can rest while searching the candidate pool. We see recurrent survival models \cite{DBLP:conf/www/ChandarTMPSWCLJ22} well suited for this task.
Overall, defining a human-like screener is not limited to fatigue. Future work should explore, e.g., theories on human decision making (see \cite[Ch. 10]{DBLP:books/daglib/0033056} and \cite{DBLP:conf/chi/CarabanKGC19, DBLP:journals/isr/AdomaviciusBCZ13}). Moreover, the simulation procedures can be extended to account for additional parameters and for score distributions estimated from real data, and they can be made accessible through a user-friendly interface.

We also emphasize the role of the individual scoring function used for evaluating each candidate. 
We framed the algorithmic screener in terms of its consistency to evaluate similar candidates similarly.
This assumption allowed us to focus on the role of the ISO.
We are, though, aware that current work mainly focuses on defining fair scoring functions to obtain a fair ranking of candidates. 
Future work should explore fair ranking algorithms under the ISO problem. The goal should be to create an algorithm that provides a fair ISO specific to the human user.
This is because, as long as the fair ranking or fair platform is used by a human for candidate screening, the risk of the ISO problem is still present.

%
% EOS
%
