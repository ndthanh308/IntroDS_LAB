%
\label{sec:Generali}

Our work is based on a collaborative effort at an European Fortune Global 500. We refer to this company as G.
The purpose of that collaboration was to study G's hiring process from an algorithmic fairness perspective. For four months we worked closely with G's Advanced Analytics (AA) and HR teams, focusing on candidate screening.
% We mostly interviewed the HR officers to understand their tasks, constraints, and methodologies, often shadowing them during screening sessions. We concluded with a report to AA that formalized G's candidate screening process as a ranking problem, evaluated the potential fairness implications, and assessed the risk and benefits of automation.
We summarize our experience in the form of \textit{stylized facts}, which underpin the initial screening order problem studied in this paper.
Although this section is specific to G, it highlights salient aspects of a real-world candidate screening problem that are likely to hold for other companies of similar size and reputation as G.  

Hiring at G consists broadly of three phases. 
In \textit{phase one}, the HR builds a candidate pool for the job opening.
Candidates submit their CVs, complete a multiple-choice questioner, and write a motivation letter. Sensitive information, such as gender, ethnicity, and age, is also provided or it can be inferred. The candidate pool is stored in a database platform. %\footnote{G used the Taleo platform by Oracle: \url{https://www.oracle.com/human-capital-management/taleo/}.}
In \textit{phase two}, the HR officer reduces the candidate pool into a smaller pool of suitable candidates. The HR officer determines candidate suitability based on each candidate's profile using a set of minimum basic requirements. 
In \textit{phase three}, the chosen candidates are interviewed by HR and the team offering the job. 
% Candidates, depending on the job, are also evaluated via case scenarios. 
The best candidates receive an offer. If no candidates are hired, HR resorts to the runner-up candidates from phase two and repeat phase three.

The choice to focus on candidate screening, or phase two, was motivated by how it represented a time-consuming, repetitive task prone to human error \cite{Kahneman2011Thinking, Kahneman2021Noise} and a sensitive, high-risk task requiring human oversight \cite{bringas2022fairness, EU_AIAct}.
% HR officers tasked with screening the candidate pool had to process considerable amounts information within certain time constraints. 
The idea of automating candidate screening initially appealed to all stakeholders; however, it became less appealing over time given how complex and human-dependent candidate screening is as a process. 
% Automation became less appealing, especially giving the lack of successful tools used by other companies (see, e.g., \cite{HiredByAlgoMITPodcast,TheAIWillSeeUMITPodcast}).
Hence, our focus shifted from the future algorithmic screener onto the present human screener.
% In formalizing this \textit{human screener}, we derived the initial screening order problem. 
Below we present stylised facts from G's candidate screening process. 
% relevant to this and similar problems. 
We draw from these facts in Sections~\ref{sec:ProblemFormulation} and \ref{sec:HumanScreener}.
We formally refer to the HR officer as the screener.

%
\begin{itemize}
    \item[G1] \textit{A varying initial screening order for the candidate pool.} 
    Each screener, through the platform, chose how to order the candidate pool before screening the candidates.
    The choice was restricted by the sorting fields offered by the platform, such as by candidates' last name or by the date of arrival of the applications. 
        
    \item[G2] \textit{Two ways to search the candidate pool.} 
    Once the initial screening order was set, each screener chose how to search it.
    Two search practices became apparent: screeners either fully or partially searched the candidate pool. In the latter case, the screener would stop once it reached the desired number of suitable candidates. 
        
    \item[G3] \textit{Meeting the minimum basic requirements.} 
    Although screeners were able to differentiate candidates relative to each other, their focus was on finding candidates that met the minimum basic requirements for the job. 
    % This meant that order within the selected candidates was not necessarily important.

    \item[G4] \textit{A representative set of selected candidates.}
    G already had in place fairness goals in the form of representation quotas, often around gender, that were enforced by the screeners when selecting candidates. 
    % This meant screeners prioritized suitable candidates from the protected group when selecting candidates. 
    
    \item[G5] \textit{Consistent notion of time.}
    Given the amount of information to be processed for each candidate, screeners aimed at spending one minute per candidate.
\end{itemize}
%

%
% EOS
%
