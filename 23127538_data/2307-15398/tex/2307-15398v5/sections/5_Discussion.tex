%
\label{sec:Discussion}

In this work, we presented the \textit{initial screening order} (ISO) as a parameter of interest; 
defined two formulations under distinct utility models of the set selection problem, the best-$k$ and the good-$k$, with their corresponding algorithmic implementations; 
and introduced a human-like screener to study the effects of the ISO on a human user. 
We also provided a simulation framework flexible enough to study and model multiple screening scenarios.
Our analysis confirms the fairness and optimality impact of the ISO, motivated by the risk of position bias, on the set of $k$ selected candidates by an algorithmic or human-like screener via an IAS. 

The simulations show a complex relations between the best-$k$ and good-$k$ problems.
Our results are limited by the functional assumptions made for formulating the two problems and screeners, in particular, the human-like screener.
Future work should explore alternative utility models and fatigue terms, while still relying on the current experimental framework.
An alternative formulation to fatigue, e.g., could involve a human-like screener that rests while searching over the ISO. 
We see recurrent survival models \cite{DBLP:conf/www/ChandarTMPSWCLJ22} well suited for this task.
Future work should also explore theories on human decision-making (e.g., \cite{DBLP:books/daglib/0033056,DBLP:conf/chi/CarabanKGC19, DBLP:journals/isr/AdomaviciusBCZ13}), or the use of the simulations framework for testing for optimal parameters (e.g., deriving a minimum score $\psi$ for which best-$k$ and good-$k$ problems coincide). 
That said, given that it is costly and time-consuming to run real candidate screening experiments, especially at the same scale of Section~\ref{sec:Experiments},
we view our work as another example \cite{DBLP:conf/fat/IonescuHJ21, DBLP:journals/corr/abs-2006-09663, DBLP:conf/fat/BountouridisHMM19, Bokanyi2020_Understanding, Schelling1971_Dynamic} of how simulations can be useful tools for studying the fairness and optimality of real-world decision-making processes involving ML-based systems \cite{DBLP:journals/air/MosqueiraReyHABF23}.

We conclude with two takeaways from our analysis for user search behavior. 
First, defining the proper problem formulation is important for understanding the impact of the ISO on the selected candidates, which reflects on the search procedures of the screeners regardless of their kind.
Second, once the search procedure is clear, it is important to understand how the screener behaves as it searches over the ISO.
These takeaways have direct implications for practitioners.
They might seem obvious ex-post, though they are supported by an extensive analysis that accounts for several factors that only become clear under modeling and experiments.



%
% EOS
%
