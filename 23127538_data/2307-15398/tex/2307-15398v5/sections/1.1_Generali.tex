%
\label{sec:Generali}

This work borrows from a previous collaborative effort at a European Fortune Global 500.
We refer to this company as G.
The purpose was to study G's hiring process as an algorithmic fairness problem.
We worked closely with Human Resources (HR), focusing on candidate screening.
In this phase of G's hiring process, an HR officer reduces the candidate pool for a job opening into a smaller pool of suitable candidates based on each candidate's profile.
The candidate pool was stored in Oracle's Taleo, a hiring platform used by HR officers to, among other things, obtain an ISO.

The following \textit{five stylized facts} summarize key practices by the HR officers (henceforth, screeners) that motivated the ISO problem.
%
\textit{\textbf{G1}} \textit{Varying ISOs.} Screeners chose the ISO. The choice was restricted by the sorting fields of the hiring platform, such as using the candidates' last name.
%
\textit{\textbf{G2}} \textit{Two ways to search the candidate pool.} Two search practices became apparent: full or partial search of the candidate pool.
%
\textit{\textbf{G3}} \textit{Meeting the set of minimum basic requirements.} Screeners were able to differentiate candidates relative to each other, but their focus was on finding candidates that met these requirements. 
Order within the selected $k$ candidates was not necessarily important.
%
\textit{\textbf{G4}} \textit{Diverse suitable candidates.} Fairness goals already existed in the form of representation quotas, often around gender, that were enforced by the screeners. 
%
\textit{\textbf{G5}} \textit{A consistent notion of time.} Screeners aimed at spending one minute per candidate.

Although G1-5 are specific to G, they highlight salient aspects of real-world candidate screening problems likely to hold in similar settings involving humans searching a pool of candidates (see, e.g., \cite{DBLP:journals/jcmc/PanHJLGG07, DBLP:conf/clef/GrotovCMSXR15, DBLP:conf/www/RichardsonDR07, DBLP:conf/chi/EchterhoffYM22, PeiEAAMO2023}).
G4-5 are standard to the fair set selection problem, especially under an algorithmic screener, while G1-3 introduce new considerations to such problem formulation, especially under a human screener.
We come back to G1-5 in Sections \ref{sec:ProblemFormulation} and \ref{sec:HumanScreener}.

%
% EOS
%
