%
\label{sec:Introduction}

Candidate screening is a complex, human-dependent task. 
It consists of a decision-maker or user, which we refer to as the \textit{screener}, tasked with selecting $k$ candidates from a candidate pool. 
Common candidate screening processes include the evaluation of resumes for a job interview \cite{Pisanelli2022_YourCV} or application packages for college admission \cite{SukumarMH18_PeacanPie}. 
The screener usually evaluates the candidate pool using limited information and under strict time constraints.
Information access systems (IAS), such as online platforms like LinkedIn and database queries like Taleo, play a central role today in candidate screening by allowing screeners to search more efficiently the candidate pool.
Enabled by machine learning (ML) \cite{DBLP:journals/corr/abs-2307-03195}, IAS often present candidates according to an estimated relevance or, at least, according to a relevant characteristic chosen by the screener \cite{DBLP:books/aw/Baeza-YatesR99}.
An industry around algorithmic candidate screening has emerged in recent years \cite{DBLP:journals/air/WillKL23}, though poor fairness results \cite{DBLP:conf/fat/WilsonG0MBSTP21,Wehner20,Sonderling22,Raghavan2020AlgortihmicHiring,DBLP:journals/datamine/RheaMDSSSKS22,jintelligence9030046,DBLP:journals/patterns/SloaneMC22, Fabris2023_DBLP:journals/corr/abs-2309-13933} and calls to consider the behavior of the human user \cite{Carricco2018EUHumanCentred, DBLP:conf/aaai/Ruggieri0PST23, bringas2022fairness, DBLP:journals/ethicsit/AlvarezCEFFFGMPLRSSZR24} continue to drive research on the social impact of IAS.

In this paper, we investigate the role of the \textit{initial screening order} (ISO) on the set selection problem implicit in the task of candidate screening. 
The ISO refers to the order in which the candidates appear in the candidate pool.
It can be chosen by or provided to the screener enabled via an IAS \cite{DBLP:journals/ftir/Ekstrand0B022}.
We develop a utility-based framework to understand the search behavior of the screener when going over the ISO, and use it to implement extensive simulations that study the influence of the ISO.
We find that the ISO can impact the optimality (i.e., choosing the best candidates) and fairness (i.e., treating similar candidates similarly) of the selected $k$ candidates, especially when the screener is human.
This is mainly because of the \textit{position bias} inherent to the ISO.
Here, position bias refers to the penalty (or premium) a candidate experiences due to where it falls on the ISO, as humans are predisposed to favor the items placed at the top of a list \cite{DBLP:journals/cacm/Baeza-Yates18, 10.1093/qje/qjr028}.

We motivate the ISO problem further in Section \ref{sec:Generali} based on our collaboration with an European company.
In Section~\ref{sec:ProblemFormulation}, we describe how the screener searches the ISO for an optimal and fair set of $k$ candidates w.r.t. a representational quota $q$ of protected candidates.
We define two problem formulations. 
In the best-$k$ the screener selects the $k$ top candidates, whereas in the good-$k$ the screener selects the $k$ first good-enough candidates fitting some minimum candidate quality measure.
We devise algorithmic solutions for both problems.
The good-$k$ is noteworthy as it allows the screener to partially search the candidate pool; the set selection problem formulation often assumes a full search.
%
In Section~\ref{sec:HumanScreener}, we analyze the algorithmic and human-like screeners to understand the impact of the ISO when a human is involved. 
The former refers to a consistent screener; the latter refers to an inconsistent screener whose evaluation of candidates suffers over time due to the fatigue of performing a repetitive task.
%
In Section~\ref{sec:Experiments}, we enhance our analysis of these two screeners through simulations that mimic multiple screening settings.
Our results confirm the role of position bias inherent to the ISO and raise new fairness concerns.
For instance, we find that the human-like screener violates individual fairness by not evaluating similar candidates similarly \cite{DBLP:conf/innovations/DworkHPRZ12} while still meeting the quota $q$, and that the algorithmic screener misses the best candidate depending on its search procedure.
%
In Section~\ref{sec:Discussion}, we conclude by discussing the limitations and extensions of our work.\footnote{We provide additional material in the Appendix.}
% \footnote{We provide additional material in the Appendix of the arXiv version of this paper~\cite{DBLP:journals/corr/abs-2307-15398}.}

Our work is the first to formalize the ISO problem.
Our main contributions are threefold.
\textit{First}, we formalize the role of the ISO in two search behaviors of the screener with the best-$k$ and good-$k$ problems.
\textit{Second}, we introduce a human-like screener and compare it theoretically and experimentally to its algorithmic counterpart.
\textit{Third}, we provide a flexible simulation tool for studying the ISO problem able to inform practitioners without needing to run real-world screening scenarios.

\subsection{Qualitative Background}
%
\label{sec:Generali}

This work borrows from a previous collaborative effort at a European Fortune Global 500.
We refer to this company as G.
The purpose was to study G's hiring process as an algorithmic fairness problem.
We worked closely with Human Resources (HR), focusing on candidate screening.
In this phase of G's hiring process, an HR officer reduces the candidate pool for a job opening into a smaller pool of suitable candidates based on each candidate's profile.
The candidate pool was stored in Oracle's Taleo, a hiring platform used by HR officers to, among other things, obtain an ISO.

The following \textit{five stylized facts} summarize key practices by the HR officers (henceforth, screeners) that motivated the ISO problem.
%
\textit{\textbf{G1}} \textit{Varying ISOs.} Screeners chose the ISO. The choice was restricted by the sorting fields of the hiring platform, such as using the candidates' last name.
%
\textit{\textbf{G2}} \textit{Two ways to search the candidate pool.} Two search practices became apparent: full or partial search of the candidate pool.
%
\textit{\textbf{G3}} \textit{Meeting the set of minimum basic requirements.} Screeners were able to differentiate candidates relative to each other, but their focus was on finding candidates that met these requirements. 
Order within the selected $k$ candidates was not necessarily important.
%
\textit{\textbf{G4}} \textit{Diverse suitable candidates.} Fairness goals already existed in the form of representation quotas, often around gender, that were enforced by the screeners. 
%
\textit{\textbf{G5}} \textit{A consistent notion of time.} Screeners aimed at spending one minute per candidate.

Although G1-5 are specific to G, they highlight salient aspects of real-world candidate screening problems likely to hold in similar settings involving humans searching a pool of candidates (see, e.g., \cite{DBLP:journals/jcmc/PanHJLGG07, DBLP:conf/clef/GrotovCMSXR15, DBLP:conf/www/RichardsonDR07, DBLP:conf/chi/EchterhoffYM22, PeiEAAMO2023}).
G4-5 are standard to the fair set selection problem, especially under an algorithmic screener, while G1-3 introduce new considerations to such problem formulation, especially under a human screener.
We come back to G1-5 in Sections \ref{sec:ProblemFormulation} and \ref{sec:HumanScreener}.

%
% EOS
%


\subsection{Related Work}
%
\label{sec:RelatedWork}

In formalizing the ISO problem, our work revives the role of position bias within a screening process involving human screeners.
% Click models
Works on search engine click models were the first to formalize \cite{CraswellZTR08_ExperimentsClickPositionBias} and test \cite{DBLP:journals/jcmc/PanHJLGG07, DBLP:conf/clef/GrotovCMSXR15, DBLP:conf/www/RichardsonDR07} how users search over an ISO.
These works 
% from the early 2000s 
modeled the different clicking procedures observed in users and
% , using among others eye-tracking technology, 
provided experimental evidence for the existence of position bias.
% The name choice of the two search procedures in Section~\ref{sec:ProblemFormulation.Algorithms} is a reference to this line of work. 
Hence, it is not surprising that today product owners are willing to pay premiums to search engines and similar platforms for the first spots in the search results of a user \cite{10.1093/qje/qjr028}.
Different from these works, we consider a user that chooses (i.e., ``clicks'') more than one item. Further, we formalize such users under a utility-maximizing framework with fairness constraints, which relates to past set selections works \cite{stoyanovich2018online}.

% Overall fairness
Within algorithmic fairness, focus has been mainly on learning an algorithm that provides a fair ISO \cite{Zehlike2023_FairRanking_P1, Zehlike2023_FairRanking_P2, DBLP:journals/vldb/PitouraSK22}.
% Within this literature, position bias is viewed as a technical bias, meaning it is due to platform design or inherent to the human subject.
Works on probability-based \cite{DBLP:conf/ssdbm/YangS17, DBLP:conf/cikm/ZehlikeB0HMB17} and exposure-based \cite{DBLP:journals/cacm/Baeza-Yates18, DBLP:journals/sigir/JoachimsGPHG17} fairness tackle position bias by (re-)arranging the ISO so that is fair according to, respectively, some model for user attention and spot allocations. 
These works, however, avoid formalizing the user for which the fair ISO is meant, e.g., assuming a complete search of the (re-)arranged ISO by the user.
Based on the click models and our own experience with G, we argue that it is important to model the human user to fully grasp the ISO problem.
The fair set selection literature, including the fair ranking, is vast; we position our work within this literature in Section~\ref{sec:Add_RelatedWork} after we have formulated the ISO problem.

% Evidence of position bias
We highlight two recent works that provide evidence for position bias and the implicit role of the ISO.
\citet{DBLP:conf/chi/EchterhoffYM22} collaborate with a college to study anchoring bias \cite{Kahneman2011Thinking} in admissions;
anchoring bias \cite{tversky_judgment_1974} occurs when a candidate's evaluation is conditioned by the quality of the previously evaluated candidates.
This paper finds that the same candidate is better off if it is preceded by worst candidates than better candidates, which violates individual fairness \cite{DBLP:conf/innovations/DworkHPRZ12}, and
% It occurs because the human screener anchors its expectations on a lower reference point despite each candidate being independent from each other. 
proposes an algorithm for capturing and balancing anchoring bias.
% Moreover, 
\citet{PeiEAAMO2023} work with a college to study how the platforms used by professors for evaluating homework affects the students' grades. 
This paper runs a set of experiments varying the order in which the homeworks are presented by the platform. 
The experiments shows that the default sorting order, which is in alphabetical order, unfairly rewards students with the same work quality depending on their last names, which violates individual fairness \cite{DBLP:conf/innovations/DworkHPRZ12}. 

Both papers provide evidence of the ISO problem (w.r.t.~fairness) and, similar to our work, highlight the role of the human using the ISO.
However, we differ from these works
% and the few other works based on real world collaborations (e.g.,~\citet{SukumarMH18_PeacanPie}) 
in that we provide a simulations framework supported by our problem formulation flexible enough to capture multiple screening scenarios. 
Gathering experimental data or having access to the real process are costly and limited to stakeholders studying questions of fairness, which positions simulations as useful tools for answering these questions \cite{DBLP:conf/fat/IonescuHJ21, DBLP:journals/corr/abs-2006-09663, DBLP:conf/fat/BountouridisHMM19, Bokanyi2020_Understanding, Schelling1971_Dynamic}.
% Our work, e.g., can be useful to simulate different ISO problems to inform the human screener and diminish the role of the position bias. 
We come back to this point in Section~\ref{sec:Discussion}.

%
% EOS
%


%
% EOS
%
