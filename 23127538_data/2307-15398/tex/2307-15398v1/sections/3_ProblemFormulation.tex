%
\label{sec:ProblemFormulation}
 
In the initial order screening problem, a human-like screener selects a number of candidates from a candidate pool based on their profiles. Order within the selected candidates does not matter. What humanizes the screener is a tendency to become inconsistent in its decision-making over time. 

\subsection{Standard Set Selection}
\label{sec:Setting}

Consider the candidate pool $\mathcal{C}$ containing $n$ candidates. A screener $\mathbf{s} \in \mathcal{S}$, where $\mathcal{S}$ represents the set of screeners, selects $k$ candidates such that $k < n$. Based on a set of minimum basic requirements, the screener evaluates the candidates' attributes $X$ and derives the candidates' scores $Y$ using a scoring function $f$ such that $f(X)=Y$. The screener also has access to the candidates' protected attribute $A$ but cannot directly use it. The scoring function summarizes how well each candidate aligns with the set of minimum basic requirements.
The screener $\mathbf{s}$ sorts the candidate pool using the scores $Y$ to obtain the ranking $\tau$: 
%
\begin{equation}
\label{eq:RankingDef}
    Y_{\tau(1)} \geq Y_{\tau(2)} \geq \dots \geq Y_{\tau(k)} \geq \dots \geq Y_{\tau(n-1)} \geq Y_{\tau(n)} 
\end{equation}
%
where $\tau(i)$ returns the candidate $c \in \mathcal{C}$ at position $i \in \tau$ while $\tau^{-1}(c)$ returns the ranking of candidate $c$ in $\tau$. The higher the score, the better suited the candidate is for the job. Hence, $Y_{\tau(k)}$ denotes the cut-off point. We summarize the objective of the screener $\mathbf{s}$ as finding:
%
\begin{equation}
\label{eq:ObjectiveSelect-k}
    \text{\textit{select-}}k = \{ Y_{\tau(1)}, \dots, Y_{\tau(k)} \}
\end{equation}
%
a subset of the candidate pool where, clearly, $|\text{\textit{select-}}k| = k$. 
%
Further, as it is convenient for thinking about algorithmic fairness, we formulate objective of the screener in terms of achieving an optimal and fair selection of candidates.

To address optimality, it helps to view \eqref{eq:ObjectiveSelect-k} in terms of utility. Formally, utility is the benefit derived by $\mathbf{s}$ from sorting $\mathcal{C}$ into $\tau$ and selecting $k$ suitable candidates for the job. Let $U^k(\tau)$ denote this utility. The simplest method for computing $U^k(\tau)$ is to add the scores (which are assumed to be non-negative) of the selected candidates:
%
\begin{equation}
\label{eq:Utility}
    U^k(\tau) = \sum_{i=1}^{k} Y_{\tau(i)}
\end{equation}
%
where the optimal screener will maximize its utility by selecting the $k$ most suitable candidates. Note that this goal is already given by the fact that $\mathbf{s}$ sorts $\mathcal{C}$ in \eqref{eq:RankingDef} and uses $Y_{\tau(k)}$ in \eqref{eq:ObjectiveSelect-k} as the cutoff. Utility helps us rationalize this fact.
%
To address fairness, it helps to view \eqref{eq:ObjectiveSelect-k} in terms of representation. Let $\phi \in [0, 1]$ denote the desired proportion of candidates in \textit{select}-$k$ from a protected group $A=a$. The fair screener will then be constrained by meeting the representational goal, or quota, captured by $\phi$. Formally:
%
\begin{equation}
\label{eq:MaxU}
    U^* = \arg \max_{\tau} U^k(\tau) \;\; \text{s. t.} \;\; \phi
\end{equation}
%
where $U^*$ represents the ranking $\tau$ derived by the optimal and fair screener under a fixed $k$. We have, thus far, presented the standard (fair) \textit{set selection problem} (see, e.g., \cite{DBLP:conf/eaamo/BueningSBGD22}).

\subsection{Order within the Selected Set}
\label{sec:ProblemFormulation:Order}

Order within $\text{\textit{select-}}k$ is not important; otherwise, we would instead write about $\text{\textit{top-}}k$ in \eqref{eq:ObjectiveSelect-k}. The screener $\mathbf{s}$ is looking for candidates in $\mathcal{C}$ that, above all, meet the set of minimum basic requirements.\footnote{A practical interpretation of this approach by the screener, based on our time at company G, is that the screener might not be able to fully evaluate the knowledge of the candidates relative to, say, the hiring manager.} There can be better candidates relative to others in $\text{\textit{select-}}k$ in terms of $Y$, but all candidates in $\text{\textit{select-}}k$ share the minimum basic requirements.

How can we formalize this characteristic of the problem? Let us assume that the ranking $\tau$ in \eqref{eq:RankingDef} exists under some scoring function $f$. Under strict inequalities, there is one way to order the candidate pool $\mathcal{C}$, while under equalities there are $n!$ ways to order the candidate pool $\mathcal{C}$. We assume that the screener is \textit{aware} of the (many) order(s), meaning it is able to differentiate among candidates using $f$. In practice, e.g., this implies an HR officer being able to tell a very good CV from a good CV based on the content of each CV. 

The role of order within the \textit{select}-$k$ becomes less relevant under an algorithmic (as in non-human-like) screener. In fact, we argue, it is not relevant for our purposes to draw the distinction between \textit{top}-$k$ and \textit{select}-$k$ as said screener should be able to derive $\tau$ from $\mathcal{C}$. Any screener, as long as it is able to \textit{consistently search} the entire candidate pool, will select a set of $k$ candidates that preserves the order based on the scores. For an algorithmic screener, in principle, we take for granted the \textit{search, score, and sort} processes behind the derivation of $\tau$, which are well-understood in ranking problems under perfect information.

Now the situation changes under a human-like screener where decision-making becomes inconsistent over time. Said screener still can choose to search, score, and sort the candidate pool, but always at a cost. In other words, it is sometimes more appealing for said screener to \textit{select the $k$-first-suitable candidates rather than select the $k$-most-suitable candidates}. Hence, \eqref{eq:MaxU} does not capture this behaviour.\footnote{Obviously, this sort behaviour could be encoded into the utility function used by the algorithmic screener. Our point is that, under no threat of inconsistent decision-making, if one is to deploy a ranking algorithm, one might as well exploit it by making it explore the entirety of the candidate pool.} 

\subsubsection{Meeting the minimum basic requirements} 
Based on the candidate screening aspect G3 from Section~\ref{sec:Generali:CandidateScreening}, we extend the standard setting---still for a human or non-human like screener---by incorporating the set of minimum basic requirements (MBR). Let $\psi$ denote the \textit{screening threshold} capturing MBR. The screener uses it for judging the scores of the candidates and, thus, their suitability. Some candidates will be more suitable than others, captured by a higher score $Y$, but the screener is interested on the common denominator: what matters is that they all meet the MBR.

A candidate $c \in \mathcal{C}$ is considered an \textit{eligible candidate} if its score $Y_c \geq \psi$. It follows that all candidates in \textit{select}-$k$ have a score $Y$ equal or greater to $\psi$. This fact is what makes order within the selected $k$ candidates unimportant.\footnote{It helps to picture an additional \textit{selection function} $g$ that complements the scoring function $f$ such that $g(Y_c) = 1$ if $Y_c \geq \psi$; otherwise $g(Y_c) = 0$.  Hence, $\forall c \in$ \textit{select}-$k$, $g(Y_c)=1$.} As with $k$, the screening threshold $\psi$ is fixed.

\subsubsection{Lower-bound utility}
Further, compared to the standard setting, we introduce the notion of the lowest possible yet satisfactory utility that the screener can achieve from \textit{select}-$k$ \eqref{eq:ObjectiveSelect-k} under $U^k(\tau)$ \eqref{eq:Utility}:
%
\begin{equation}
\label{eq:LowerUtility}
    \underline{U}^k = k \cdot \psi
\end{equation}
%
denoting the case in which all candidates in \textit{select}-$k$ have a score $Y = \psi$. 

We view $\underline{U}^k$ as a reference point for the screener. This lower limit can be surpassed by selecting high-scoring candidates prior to achieving $k$, meaning that the screener still needs to continue screening more candidates. What this lower bound entails is that the utility of the screener ``flattens'' within the selected set: once $k$ is meet under $\psi$, the screener can choose to stop. Since the main goal is to find $k$ candidates, the screener can be indifferent from selecting one candidate with $Y_1$ and over one with $Y_2$ where $Y_1 > Y_2 \geq \psi$ as both are suitable to fill a spot in \textit{select}-$k$.

\subsection{The Human-like Screener}
\label{sec:SettingExtras}

To formalize inconsistency in the screening process, we extend further the standard setting by introducing new definitions that characterize the human-like screener. 

\subsubsection{Time}
Let $t$ denote the discrete unit of time. It represents how long the screener takes to evaluate a candidate $c \in \mathcal{C}$. For simplicity, we assume the screener takes the same amount of time per each candidate. This assumption aligns with what we observed at company G where HR officers aimed at spending around one minute per candidate profile. This assumption means that time itself cannot be optimized by the screener. It is convenient to track time along the search of the candidate pool, meaning that at time $t_1$ the screener evaluates the first candidate to appear in $\mathcal{C}$ and so on. Hence, for simplicity, we often track time $t$ using the corresponding  $c \in \mathcal{C}$.

\subsubsection{The initial order} 
\label{sec:InitialOrderDef}
The initial order represents the arrangement of the candidate pool prior to screening. The screener, as observed at company G where HR officers used from last name to application arrival date for sorting the candidate profiles, chooses the initial order. Under $|C|=n$, it is possible to arrange the candidate pool $n!$ ways before screening it. Let $\theta$ represent the initial order chosen by a screener and $\Theta$ the set of all possible initial orders such that $\theta \in \Theta$. This definition follows from aspect G1 from Section~\ref{sec:Generali:CandidateScreening}.
% The choice of initial order is what defines a screener. Two screeners $s, s' \in \mathcal{S}$ are the same if they choose the same $\theta$.

It follows that the ranking $\tau \in \Theta$, though the possibility of a screener choosing the optimal ranking (under strict inequalities in \eqref{eq:RankingDef}) are low: $1/n!$. Therefore, we introduce the notion of \textit{meaningful initial order} to refer to a $\theta$ that contains, in terms of the arrangement of the candidates, meaningful information for the screener. 

All initial orders are meaningless to the screener: even in the case where $\tau = \theta$, the screener will not be aware it has chosen the ranking until it has searched, evaluated, and sorted the entire candidate pool. This condition insures that the screener does not knowingly introduce bias by arranging $\mathcal{C}$ into $\theta$ in some meaningful way.

\subsubsection{Fatigue} 
To ``humanize'' the screener, we introduce the notion of fatigue denoted by the unit $\omega$. Fatigue accumulates over time, where we use $\Omega(t)$ to denote the cumulative fatigue of a screener. Fatigue, we assume, increases over time, meaning $\Omega(t) < \Omega(t+1)  \; \forall t$. As the name suggests, $\omega$ is meant to represent how a screener becomes tired over time $t$ and, in turn, becomes increasingly inconsistent. Fatigue can also represent the increasing cost of searching $\mathcal{C}$ for the screener as time moves along with each candidate being evaluated. 

What $\omega$ means in practice, for instance, is that a screener will evaluate the identical candidates $c_1$ and $c_2$ differently at times $t_1$ and $t_2$, as $|\Omega(t_1)| \neq |\Omega(t_2)|$. Therefore, in terms of its functional form, $\omega$ can affect directly (by, e.g., entering the utility function) or indirectly (by, e.g., affecting the scores entering the utility function) the screener's behavior. Further, we describe the screener as unaware of its fatigue as in it is a characteristic of the screener but not in the form of a parameter to be optimized by the screener. Hence, the question we are interested here is not how will the screener minimize its fatigue, but how tired will the screener be when it reaches its goal? The focus is mainly on how $\omega$ affects the resulting \textit{select}-$k$. 

\subsubsection{From partial to full search}
Finally, regarding aspect P1 from Section~\ref{sec:Generali:CandidateScreening}, the screener can opt to search partially or fully the candidate pool. Formally, this links to the notion of a lower-bound utility $\underline{U}^k$ as it is possible for $U^*$ to be higher, meaning the screener is content with stopping the search under a sub-optimal utility as it already reached $k$.
We explore the two main search procedures to the screener in the next section. Here, we rewrite \eqref{eq:MaxU} accordingly:
%
\begin{equation}
\label{eq:LowerMaxU}
    \underline{U}^* = \arg \max_{\underline{\tau}} U^k(\underline{\tau}) \;\; \text{s. t.} \;\; \phi, \; \underline{U}^K
\end{equation}
%
where $\underline{\tau} \subseteq \tau$. It essentially states that the screener will search $C_{\mathbf{s}}$ until it satisfies $\underline{U}^K$, resulting in the partial ranking $\underline{\tau}$. It follows that $\underline{U}^* \leq U^*$. 

Together, these elements based on G1, G2, and G3 in Section~\ref{sec:Generali:CandidateScreening} characterize the human-like screener. To summarize, both $k$ and $\psi$ are provided. The initial order $\theta$ comes with the screener $\mathbf{s}$. The screener also decides how to search the candidate pool (next section), which may result in a partial or full search of the candidate pool as formalized in \eqref{eq:LowerMaxU}.  

%
% EOS
%
