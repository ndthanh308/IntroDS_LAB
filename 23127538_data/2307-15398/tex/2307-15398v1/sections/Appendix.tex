%
\label{App}

\section{Proofs}
\label{App:Proofs}

%
\begin{proof}[Proof of Proposition~\ref{proposition:CascadeVsExamination}]
We focus on the expected accumulated fatigue $E[\Omega(t_f)]$ incurred by the screener at the time $f$ of the end of the search. For the Examination Search (Algorithm~\ref{alg:Exhaustive}), $t_f = n$ in all runs, and then $E[\Omega(t_f)] = n \cdot \omega$. For the Cascade Search (Algorithm~\ref{alg:Lazy}), $t_f$ is the time at which the $k^{th}$ eligible candidate is found. Clearly $t_f \in [k, n]$, with the case $t_f =n$ occurring when $m=k$ with probability $k/(n!)$. Thus, $k \cdot \omega \leq E[\Omega(t_f)] < n \cdot \omega$.
%
%To show an equal preference by the screener between a Cascade Search (Algorithm~\ref{alg:Lazy}) and an Examination Search (Algorithm~\ref{alg:Exhaustive}), consider the case where the number of eligible candidates in the candidate pool $\mathcal{C}$ equals the number of candidates the screener wishes to select: $m=k$. Further, assume that $|\mathcal{C}| = n > k$ and that all $m$ candidates fall in the bottom $m$-positions of the initial order $\theta$. In this scenario the screener will have to cover the entirety of $\theta$ under either search model to meet the goal of selecting $k$ candidates. Therefore, the screener finds \textit{select}-$k$ with the same accumulated fatigue $\Omega= n \cdot \omega$ under either search model. Here, the screener is indifferent between the two search models.
%
%To show a strict preference by the screener between a Cascade Search (Algorithm~\ref{alg:Lazy}) and an Examination Search (Algorithm~\ref{alg:Exhaustive}), consider the case where the same $m$ eligible candidates now appear at top $m$-positions of $\theta$. Under a Cascade Search, the screener finds \textit{select}-$k$ with an accumulated fatigue of $\Omega' = m \cdot \omega$ while, under an Examination search, the screener still needs to cover the remaining $n-m$ candidates and finds \textit{select}-$k$ with an accumulated fatigue of $\Omega'' = n \cdot \omega$. It follows that $\Omega' < \Omega''$, which makes the screener prefer the Cascade over the Examination search model. Therefore, the screener weakly prefers a Cascade Search (Algorithm~\ref{alg:Lazy}) from an Examination Search (Algorithm~\ref{alg:Exhaustive}).
%
% Note that under the trivial case where $n=m=k$, the screener is indifferent but then there is no selection problem to consider in this case.
\end{proof}
%

%
\begin{proof}[Proof of Proposition~\ref{proposition:UnfairUnbalancedData}]
Let us assume for simplicity that $r = n_{a'}/n_a > 1$ is a natural number. Consider a block of $r+1$ candidates, with one protected and $r$ non-protected. The fatique for examining the protected candidate is $(r+2)/2 \cdot \omega$ on average, based on the position of such candidate in the block. The fatique for examining the first non-protected candidate is $2 \cdot \omega$ with probability $1/(r+1)$ (when the protected candidate occurs in the first position) and $\omega$ with probability $r/(r+1)$ (when the protected candidate occurs in a later position), for an average fatigue of $(r+2)/(r+1) \cdot \omega$. This is lower than the one for the protected candidate, since $r>1$. 

The fatigue for examining the second protected candidate is at least $(v+2) \cdot \omega$, as such a candidate will be in the next block. If $r \geq 2$, the second non-protected candidate will be in the current block, and the fatigue for examining it will be  $3 \cdot \omega$ with probability $2/(r+1)$ (when the protected candidate occurs in the first or in the second position), and $2 \cdot \omega$ with probability $(r-1)/(r+1)$, for a total of $(5+r)/(r+1) \cdot \omega$. Again, the fatigues for the second non-protected candidate is lower  than the fatigue for the second protected candidate. This reasoning extends to all other order positions of protected and non-protected.

In summary, the burden of fatigue is unequally distributed among the first $h$ non-protected and the first $h$ protected candidates, where $h \in [1, \min\{n_a, n_{a'}\}]$. 
%
%Let $t_1$ and $t_2$ be the time of the screen come across the first non-protected and a protected candidate respectively. 
%    Let $A=a$ denote membership to the protected group and $A=a'$ to the non-protected group such that there are $n_a$ and $n_{a'}$ protected and non-protected individuals in the candidate pool $\mathcal{C}$. Assume $\mathcal{C}$ is unbalanced, meaning $n_{a} < n_{a'}$, and, consequently, so is the initial screening order $\theta$. Further, assume a Cascade Search (Algorithm~\ref{alg:Lazy}), fatigued scores (Section~\ref{sec:ISO:FatiguedScreener}), and a standard fair initial initial order (Def.~\ref{def:StandardFairIO}). Under this setting, it follows that the most valuable position to fall in for any candidate $c \in \theta$ is the first position, or $\theta(1)$. Therefore:
    %
%    \begin{equation*}
%        P(\theta(1) \wedge A=a) = \frac{n_a}{n} <  \frac{n_{a'}}{n} = P(\theta(1) \wedge A=a') 
%    \end{equation*}
    %
%    the probability for the screener to examine first a non-protected over a protected individual is higher. For simplicity, assume $\theta$ has an order that reflects the proportions $n_a/n$ and $n_{a'}/n$: e.g., if $n_a/n = 0.3$ and $n_{a'}/n = 0.7$ then every 10 positions should contain 3 protected and 7 non-protected candidates. Therefore:
        %
%    \begin{equation*}
%        P(\theta(i) \wedge A=a) = \frac{n_a}{n} <  \frac{n_{a'}}{n} = P(\theta(i) \wedge A=a') \; \forall i \in \theta
%    \end{equation*}
    %
%    meaning it is more likely for the screener to come across a non-protected (and over-represented) individual than a protected (and under-represented) individual when going through $\theta$.
    %
%    The accumulated fatigue $\Omega$ increases over $\theta$. Consider the first candidate to be evaluated by the screener from each group in $A$. As it is more likely for the screener to come across a non-protected candidate over a protected candidate, the screener will be more tired by the time it evaluates the first protected candidate than the first non-protected candidate. Since $\Omega(t_1) < \Omega(t_2)$ and so on, the burden of fatigue is unequally distributed in $\Theta$: it is more likely for the non-protected, over represented candidate to face a screener with $\Omega(t_1)$ than it is for a protected under-represents candidate. This argument extends to the rest of the initial order.
\end{proof}
%

%
\begin{proof}[Proof of Proposition~\ref{proposition:DifferentThresholds}]
Fairness is achieved if the number of candidates examined for $A=a$ and $A=a'$ are the same. Such numbers can be obtained by considering examination as drawing from an hypergeometric distribution. $\textit{HyperG}(n_a, \beta_a \cdot n_a, w_{a})$ (resp., $\textit{HyperG}(n_{a'}, \beta_{a'} \cdot m_{a'}, w_{a'})$) models the probability of finding a certain number of eligible candidates in $w_{a}$ (resp., $w_{a'}$) evaluations of protected (resp., unprotected) candidates from a pool of $n_a$ (resp., $n_{a'}$) candidates including $\beta_a \cdot n_a$ (resp., $\beta_{a'} \cdot m_{a'}$) eligible in total. The expectation of the distribution is $w_{a} \cdot \beta_a$ (resp., $w_{a'} \cdot \beta_{a'}$). The expectation is the mean number of eligible candidates found, i.e.:
\[w_{a} \cdot \beta_a = k_a \hspace{2cm} w_{a'} \cdot \beta_{a'} = k - k_a \]
Since we want $w_{a} = w_{a'}$, we have the solution:
\[ w_{a} = w_{a'} = \frac{k}{\beta_a + \beta_{a'}} \]
As a consequence, we must stop the search for each group after $w_{a} = w_{a'}$ evaluations of candidates, which may require to dynamically adapt the screening thresholds $\psi_a$ and $\psi_{a'}$  to ensure the quotas $k_a = \phi \cdot k$ and $k_{a'} = k - k_a$.


%    Suppose the inherent quality of candidates $Y^{*}$ in the population is independent from $A$, meaning that we have the same proportion of, say, bad, average, and good candidates across males ($A=a')$ and females ($A=a$) in our population. 
%    Assume we build $\mathcal{C}$ at random by drawing from both male and female populations, keeping an uneven gender balanced of $n_a < n_{a'}$. It follows that $Y^{*}$ and, consequently, the score $Y$ under an unbiased and consistent screener should be distributed evenly across $A$ in $\mathcal{C}$, though, by construction, the number of good male candidates ($g_{a'}$) will be higher than the number of good female candidates ($g_a$). 
%    If $g_{a'} \geq k_{a'}=k_a > g_a$, then we either reduce $k_{a'}$ until we reach $g_{a}$ and, in turn, increase $k_{a'}$ by that same amount, or we let $\psi_{a} < \psi_{a'}$ and proceed to complete the $k_{a}$ quota with average and even bad female candidates.
\end{proof}
%

%
% EOS
%

%\section{An Extension: The Dynamic Initial Screening Order Problem}
