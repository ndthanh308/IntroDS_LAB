%
\label{sec:Discussion}

We presented the initial screening order problem (Section~\ref{sec:ISO}), an extension of the set selection problem (Section~\ref{sec:ProblemFormulation} ) that involves a human-like screener with an objective of finding the first $k$ suitable candidates over the best $k$ candidates in a candidate pool given an initial screening order. This problem is based on a collaboration with a large company G (Section~\ref{sec:Generali}). Our main contribution is the formalization of this problem, which we argue opens the path for future extension of the current work (Section~\ref{sec:RelatedWork}) on ranking algorithms, fairness, and hiring.  

We showed how, under a human-like screener, the initial order $\theta$ poses a risk in terms of an optimal and fair \textit{select}-$k$ of candidates from a candidate pool $\mathcal{C}$. The optimality concerns, we argue, are admissible given the focus of the screener in finding the first $k$ suitable candidates rather then the $k$ best candidates in $\mathcal{C}$. A clear approach to this issue is to train a fair ranking algorithm (under, e.g., learning to rank methods) to provide the screener with $\theta$ that is reflective of the ranking $\tau$ in \eqref{eq:RankingDef}. Even then, though, there is always the risk of fatigue (or, conversely, lack of attention) when the human reads such ranking. Hence, it comes down to what we wish to assume regarding the units of fatigue $\omega$ and its cumulative effect $\Omega$.

The fairness concerns are more pressing as the accumulated fatigue under an unbalanced dataset can have an unequal effect on protected and non-protected individuals. Again, even under a $\theta$ derived now from an optimal and fair ranking algorithm (i.e., one that solves for \eqref{eq:MaxU}), we still need to consider the unwanted effects of fatigue and how it will affect the screener going over this ranking. This, of course, assuming that we \textit{can} and \textit{want} an optimal ranking over the initial order. What the initial screening order problem highlights is that we need to account for the interaction between the human and the algorithm. We essentially would want a procedure, read algorithm, that would provide an initial order $\theta^{*}$ such that it accounts for the human-like screener's fatigue. 

Future work should focus on modeling fatigue and developing a procedure to balance it out within the initial order, similar to how \cite{DBLP:conf/chi/EchterhoffYM22} did for anchoring bias. It should be possible to extend existing learning to rank procedures that use exposure based fairness definitions. Rather than assuming a distribution for the fatigue (or, conversely, attention), such as the geometric distribution, we could attempt to model it. Here, in particular, we see duration models (also known as survival models) \cite{DBLP:conf/www/ChandarTMPSWCLJ22} well suited for this task.

%
% EOS
%

% \subsection{Baseline Solution}
% The simplest solution we see if insuring, at least, that the fatigue is shared evenly across membership to the protected attribute $A$. By simple we also mean without having to train a ranking algorithm based on scores $Y$, which can be in reality subjective and open to discussion by all parties involved; instead, a simple algorithm with access to $A$ that can return $\theta^{*}$ with some notion of $\omega$ in mind.

% In particular, we picture an algorithm that places protected and non-protected individuals in $\theta^{*}$ from their proportional representation in $\mathcal{C}$ to an interspersed (or zigzag) pattern. For instance, if we have 20\% of $\mathcal{C}$ is female, then for every ten positions in $\theta^{*}$ there needs to be 2 females until all female candidates are accounted for. That would be one edge case. The other edge case would be allocating one male and one female (or vice-versa) throughout $\theta^{*}$ until all female candidates are accounted for. Since $A$ is available/provided by candidates, this solution should be attainable.

% \subsection{Modeling Fatigue}

% The baseline solution ignores fatigue as a problem parameter. Modeling it would motivate how protected and non-protected candidates are placed within $\theta^{*}$ in addition to the other parameters presented in Sections~\ref{sec:ProblemFormulation} and \ref{sec:ISO}. Here, the simplest modeling approach is to treat \textit{fatigue as deterministic:} assign some real value to the fatigue unit $\omega$, model an accumulation pattern over time $\Omega(t)$, and based on this look at protected and non-protected candidates and place them such that $\Omega(t)$ affects the pairs (in terms of appearance) similarly. We want, e.g., for the second male and second female to experience similar accumulated fatigue. We allocate them within $\theta^{*}$ such that this is the case. 

% Alternatively, we could model \textit{fatigue as probabilistic}. The notion of time in the initial screening problem, in particular, lends itself to modeling fatigue as a point process. We consider duration (or survival) models where the random variable realizes as a function of time like, e.g., time to failure or life expectancy. The question we are interested in \textit{how tired the screener is at time $t$}.
