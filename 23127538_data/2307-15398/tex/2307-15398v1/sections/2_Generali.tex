%
\label{sec:Generali}

The initial screening order problem is based on a collaborative work at an European Fortune Global 500. We refer to this company as G. This section summarizes our experience at G and how it motivated the problem. 

The goal of the collaboration was to study G's hiring process from an algorithmic fairness perspective. We worked closely with the teams of Advanced Analytics (AA) and Human Resources (HR), focusing on the candidate screening problem where an HR officer or screener selects a number of candidates based on their profiles. We mostly interviewed the HR officers to understand their tasks, constraints, and methodologies, often shadowing them during screening sessions. We concluded with a report to AA that formalized G's candidate screening problem as a ranking problem, evaluated the potential fairness implications, and assessed the risk and benefits of automation.

\subsection{The Hiring Pipeline}
\label{sec:Generali:HiringPipeline}

Hiring at company G consists broadly of three phases. In \textit{phase one}, HR builds a large enough candidate pool for a job opening. Here, ``large enough'' depended on the seniority and priority of the job opening. For a very senior position like Data Science Director, e.g., the candidate pool was often small (25 candidates max) as the amount of suitable candidates active in the job market was small throughout the year. The job opening includes a description of the ideal candidate provided by the hiring manager where candidates apply and/or are recruited accordingly. Candidates submit their CVs, complete relevant information in the form of multiple-choice questions, and occasionally provide a motivation letter. Sensitive information like a candidate's gender is always provided. The candidate pool is stored in Taleo\footnote{For more information, visit \url{https://www.oracle.com/human-capital-management/taleo/}.}, a database query platform for hiring by Oracle.

In \textit{phase two}, which covers the candidate screening problem, HR reduces the candidate pool into a smaller pool of suitable candidates to be invited for the interviews that take place in the next phase. The assigned HR officer or screener determines candidate suitability using a set of minimum basic requirements that all candidates must meet according to the hiring manager. Suitable candidates do not necessarily have to be ideal candidates. The existence of a third phase allows G to verify and update the information provided in each candidate's profile. The set of minimum basic requirements thus represents the non-negotiables for the hiring manager as well as a guide for flagging suitable candidates for the screener. 

Finally in \textit{phase three}, HR along with the hiring manager examine the subset of selected candidates from phase two across a series of interviews and aptitude tests. There is a first interview between HR and the candidates to asses the information on paper. If positive, candidates are interviewed and evaluated by members from the team offering the job opening. If positive, then the candidates at last meet the hiring manager (or partner) for the final call. At each interview, candidates are graded and their scores discussed among the interviewers. The grading is recorded in Taleo. The best candidates receive an offer, which they can accept or decline. If there is no match between the top candidates and the job opening, HR goes back to the runner-up candidates that remained in phase two and repeat phase three. 

\subsection{Candidate Screening}
\label{sec:Generali:CandidateScreening}

The choice to focus on phase two of the hiring pipeline was motivated by how appealing and dangerous it appeared for automation, offering an interesting tension between a time-consuming, repetitive task prone to human error and a high-risk, sensitive task requiring human oversight.\footnote{Automation seemed less appealing/reasonable for the other two phases: phase one was already streamlined with Taleo while phase three was clearly too human-dependent. This opinion was shared by G too.} When screening multiple candidates, HR officers clearly faced an overflow of information that had to be processed quickly. Phase two was crucial. It was not humanly possible, at least for a company of the size of G, to interview all candidates that applied, especially when we considered that HR ran multiple job openings at once. Given the nature of the task---that of extracting information from the candidate profiles and deciding whether it met a set of minimum basic requirements---the idea of developing a screening algorithm was, in principle, an appealing option to AA, HR, and us. 

Early on in the collaboration, however, it became clear to all stakeholders involved how human intensive and complex the candidate screening process was. It seemed unlikely that such process could be automated at all using the available data and available tools. It seemed even more unlikely when we considered the many unsuccessful AI-tools deployed in other companies (see, e.g., \cite{HiredByAlgoMITPodcast,TheAIWillSeeUMITPodcast}). Despite the known evidence on biased and inefficient human decision-making in similar settings to phase two (e.g., \cite{Kahneman2016Noise, Miller2018Bias, Kahneman2021Noise}), the goal of automation increasingly lost its appeal: the focus shifted away from the future algorithmic screener onto the present human screener. At best, we argued then and continue to do so, we could develop an algorithm to aid the human screener but never to replace it.  

This tension around automating phase two made us study closer the processes of the human screener. If the long-term goal was to create an algorithm that could aid the human screener, we first needed to identify instances where the human screener was likely to resort to biased (read, inconsistent) decision-making due to, say, cognitive heuristics (e.g., \cite{tversky_judgment_1974, Kahneman2011Thinking}). Hence, our focus on a \textit{human-like screener} throughout this paper. In Sections \ref{sec:ProblemFormulation} and \ref{sec:ISO} we formalize such screener and illustrate its role within the initial order screening problem.
We highlight three aspects that stood-out about phase two, which inspired the formulation of the initial screening order problem:
%
\begin{itemize}
    \item[(G1)] \textit{A varying initial order of the candidate pool.} As a default, Taleo sorted the candidate pool in alphabetical order. Each HR officer, however, was free to choose how to sort the candidate pool. The choice was restricted by the sorting fields offered by Taleo. One HR officer, e.g., preferred to sort the candidates using the submission date of the applications. The HR officers were consistent with their choice of initial screening order through multiple screening procedures.
    
    \item[(G2)] \textit{Multiple ways of searching the candidate pool.} Each HR officer decided how to search the candidate pool. What mattered is that they reached the quota of suitable candidates within a reasonable time. One HR officer, e.g., preferred to search the whole candidate pool while another stopped searching once meeting the quota. It took them one minute per profile. The HR officers were consistent with their choice of search procedure.
    
    \item[(G3)] \textit{Candidate meets minimum basic requirements.} Although the HR officer imposed an implicit order among candidates when screening the candidate pool, what mattered for those selected was that they met the set of minimum basic requirements as provided by the hiring manager. Again, the existence of the third phase allowed the HR officer to focus on getting enough suitable candidates on paper for an interview instead of finding the best possible candidates on paper for the job opening. This meant that order within the list of selected candidates was not necessarily important.
\end{itemize}
%

%
\begin{example}[Junior Data Scientist for Hire]
\label{RunningExample}
    As an illustrative example, suppose G wants to hire three junior data scientists. The hiring manager provides a list of qualifications describing the ideal candidate: a good command of English (B1 or above), a technical background (Engineering, Physics, or equivalent), Python programming skills (command at least of the \texttt{numpy}, \texttt{pandas}, and \texttt{scikit-learn} packages), and familiarity with EU banking regulation models (Basel II and III). The hiring manager sets as the minimum basic requirement that candidates meet two of these qualifications.
\end{example}
%

In job openings similar to that of Example~\ref{RunningExample}, a single HR officer usually had to select 15 to 30 candidates from a pool of 500 within a couple of hours. The HR officer would screen through each candidate profile, cheeking for the minimum basic requirements. A candidate with a B2 English and an Engineering degree but no programming skills in Python, e.g., would be considered suitable as candidate. A candidate with a B1 English but a Philosophy degree and no mention of Python skills would not be considered a suitable candidate. We stress that it is not about \textit{how suitable} the candidate is, but about whether the candidate is \textit{suitable or not}. 

\subsection{Fairness Goals}
\label{sec:Generali:Fairness}

We note that G, as it is common today with companies' efforts to have a diverse workforce, already had in place fairness goals in the form of \textit{representation quotas} for each job opening. These were, in fact, considered as key performance indicators, or KPIs, for the company. The main protected (or sensitive) attribute considered for such KPIs was gender. Throughout the hiring pipeline, when possible, HR would aim to reach a balanced list of candidates, meaning 50\% male and 50\% female candidates. 

In phase two, HR would aim to select a balanced subset of suitable candidates from the candidate pool. In practice, this meant imposing \textit{quotas} and \textit{threshold policies} with the goal of insuring some levels of female representation (or, in general, the under-represented group, which was always the protected group). HR, e.g., would apply different sets of minimum basic requirements to male and female candidates to meet certain representation goals specific to phase two. Such procedures are important, as we will see in the next sections, for the formulation of the initial screening order. We stress that G had well-established, working fairness policies already in place. Hence, what was of interest to us was understanding the role of the human-like screener in reaching these fairness goals.

%
% EOS
%
