%%
\documentclass[authorversion, nonacm]{acmart}

%%
%% \BibTeX command to typeset BibTeX logo in the docs
\AtBeginDocument{%
  \providecommand\BibTeX{{%
    \normalfont B\kern-0.5em{\scshape i\kern-0.25em b}\kern-0.8em\TeX}}}

%%
%% Rights management information
\setcopyright{acmcopyright}
\copyrightyear{2023}
\acmYear{2023}
\acmDOI{XXXXXXX.XXXXXXX}

%%
%% These commands are for a PROCEEDINGS abstract or paper
\acmConference[EAAMO'23]{ACM Conference on Equity and Access in Algorithms, Mechanisms, and Optimization}{October 30 - November 1, 2023}{Boston, MA}
%%
\acmBooktitle{EAAMO'23: ACM Conference on Equity and Access in Algorithms, Mechanisms, and Optimization, Boston, MA} 
\acmPrice{15.00}
\acmISBN{978-1-4503-XXXX-X/18/06}


%%
%% Submission ID
%%\acmSubmissionID{123-A56-BU3}

%%
%% MY STUFF
\usepackage{algorithm}
\usepackage{algpseudocode}
\usepackage{orcidlink}

%% Other:
\newtheorem{remark}{Remark}

%%
\begin{document}

%%
\title{The Initial Screening Order Problem}

%%
%%
\author{Jose M. Alvarez \orcidlink{0000-0001-9412-9013}}
\email{jose.alvarez@sns.it}
% \orcid{0000-0001-9412-9013}
\affiliation{
  \institution{Scuola Normale Superiore, University of Pisa}
  \city{Pisa}
  \country{Italy}
}

\author{Salvatore Ruggieri \orcidlink{0000-0002-1917-6087}}
\email{salvatore.ruggieri@unipi.it}
% \orcid{0000-0002-1917-6087}
\affiliation{
  \institution{University of Pisa}
  \city{Pisa}
  \country{Italy}
}

%%
\renewcommand{\shortauthors}{Alvarez and Ruggieri}

%%
\begin{abstract}
    In this paper we present the initial screening order problem, a crucial step within candidate screening. 
    It involves a human-like screener with an objective to find the first $k$ suitable candidates rather than the best $k$ suitable candidates in a candidate pool given an initial screening order.
    The initial screening order represents the way in which the human-like screener arranges the candidate pool prior to screening.
    The choice of initial screening order has considerable effects on the selected set of $k$ candidates. 
    We prove that under an unbalanced candidate pool (e.g., having more male than female candidates), the human-like screener can suffer from uneven efforts that hinder its decision-making over the protected, under-represented group relative to the non-protected, over-represented group. Other fairness results are proven under the human-like screener.
    This research is based on a collaboration with a large company to better understand its hiring process for potential automation.
    Our main contribution is the formalization of the initial screening order problem which, we argue, opens the path for future extensions of the current works on ranking algorithms, fairness, and automation for screening procedures. 
\end{abstract}

%%
%% The code below is generated by the tool at http://dl.acm.org/ccs.cfm.
\begin{CCSXML}
<ccs2012>
<concept>
<concept_id>10010147.10010257</concept_id>
<concept_desc>Computing methodologies~Machine learning</concept_desc>
<concept_significance>500</concept_significance>
</concept>
<concept>
<concept_id>10010405.10010481.10010484</concept_id>
<concept_desc>Applied computing~Decision analysis</concept_desc>
<concept_significance>500</concept_significance>
</concept>
</ccs2012>
\end{CCSXML}

\ccsdesc[500]{Computing methodologies~Machine learning}
\ccsdesc[500]{Applied computing~Decision analysis}

%%
\keywords{fair ranking algorithms, human resources, position bias, candidate screening}

\received{}
\received[revised]{}
\received[accepted]{}

%%
\maketitle

\section{Introduction}
%%%%%%%%%%%%%%%%%%%%%%%%%%%%%%%%%%%%%%%%%%%%%%%%%%%%%%%%%%%%%%%%%%%%%%%%%%%%%%%%
\section{Introduction}

Autonomous driving (AD) %with deep learning networks 
has shown promising achievements and is considered an important technological breakthrough that could revolutionize the future of transportation. Currently, ensuring the safety of autonomous driving systems has become a topic of extensive development.
% There has been much discussion on how to verify the safety of autonomous driving systems.
One traditional solution for safety tests is to exhaustively enumerate real scenarios for validation. Nevertheless, this process is not only labor-intensive and costly but also dangerous. Simulation has emerged as a robust, safe, and efficient alternative for training and evaluating AD software and algorithms~\cite{li2019aads, amini2020learning, amini2022vista}.

% Figure environment removed

Recently, neural radiance field (NeRF)~\cite{mildenhall2020nerf} has gained significant attention in AD simulation~\cite{drivesim}. This approach leverages multi-view images to construct a 3D scene and enable novel view synthesis for both indoor and outdoor applications. When it comes to constructing NeRF models in AD simulation, there are two options available: 1) collecting a large amount of data to cover as many viewpoints as possible, and constructing a fine-grained scene offline; 2) directly using log data from road tests to quickly create an environment and dynamically simulate driving scenarios. The first choice can deliver high-quality simulation~\cite{tancik2022block} by transforming the problem of view extrapolation into view interpolation through the use of large amounts of data. However, it is time- and cost-intensive, which makes it challenging to generalize. As for the second choice, the collected images from log data are usually similar to each other along the running trajectory, which may result in unsatisfactory outcomes, particularly when the camera pose is placed out-of-trajectory (see \figref{figSupportComp} as an example), semantic consistency cannot be guaranteed when synthesizing images from deviated views. We observe this problem under this data condition in all neural radiance approaches, and to the best of our knowledge, none of the existing work has solved this issue.
In our opinion, semantic consistency is crucial for AD simulation, and synthesizing on deviated views is unavoidable for scalability.

AD simulation usually involves map data for planning and control, which can be obtained from a prebuilt High-Definition Map (HD Map) or an online mapping module. While the map data may not be pixel-perfect, it can provide semantic-level information that is useful for enhancing the semantic consistency of the trained neural radiance field.
In this paper, we propose incorporating map priors into neural radiance fields to enhance the semantic consistency and rendering quality of deviated driving view synthesis. Firstly, we employ ground information from maps to supervise the density field of NeRF, providing a more reliable road base for semantic entities. Next, we propose sampling rays to simulate unseen views. Unlike most NeRF augmentation methods~\cite{zhang2022ray, chen2022geoaug}, we utilize ground and lane information in sampling computations to guide the radiance field. More importantly, we model the above two supervision methods as weak supervision by using an uncertainty parameter and propose an uncertainty tempering scheme to increase the uncertainty. This ensures that map priors only guide the training process rather than enforce it towards their absolute values. As a result, our proposed method not only improves the rendering quality of interpolated novel view synthesis quantitatively but also enhances the semantic consistency of deviated novel view synthesis. 
Our contributions can be summarized as follows:
% We summarize the contributions of this paper as follows.



% To overcome the limitations of the collected data, this paper proposes a novel approach that leverages map information to enhance the semantic consistency of the synthesized driving views. 

% Autonomous driving (AD) vehicles are being trained with the help of deep learning networks and have shown promising achievements. This technology is considered to be a breakthrough that could change the way of transportation in the near future. However, there are many discussions on how to verify or judge the safety of autonomous driving systems.
% A straightforward solution towards the safety tests is to exhaustively enumerate real scenarios for validation as many as possible. However, the process of implementing different real scenarios is not only labor-intensive and costly, but also dangerous. Simulation has been proved to be an alternative, which is robust, safe, efficient in training, and evaluating AD software and algorithms.
% Now, the emerging technology of neural radiance field (NeRF)~\cite{} leverages multi-view images to construct a 3D scene and enable novel view synthesis for many indoor and outdoor applications. For AD simulation, there are two choices for constructing NeRF models: 1) collect a large amount of data, such as LiDAR and camera data, similar to mapping, to construct a fine-grained scene offline; or 2) directly use the log file (typically in the format of ROS bag) to rapidly create an environment and then dynamically simulate the driving scenarios.
% The first choice can achieve high-quality simulation, but it is time-consuming and expensive, making it difficult to generalize to very large scales. On the other hand, the second option is fast but can lead to low-quality simulation due to the data being sparse and similar to each other in log data. This paper tackles the problem raised by choosing the latter option and attempts to improve the quality of out-of-trajectory driving view synthesis by incorporating map information. This approach is practical for many autonomous driving tests.
% In conclusion, the use of NeRF technology for AD simulation is a promising avenue for training and evaluating AD software and algorithms. While both options for constructing NeRF models have their pros and cons, this paper addresses the challenges of the second option and proposes a potential solution to improve the quality of simulation.

%There exist a few attempts to facilitate training a NeRF model for synthesizing out-of-trajectory (or called as extrapo trajectory) views.


\begin{itemize}
    \item We propose a novel method to incorporate commonly used map priors in AD scenes into neural radiance fields to improve the out-of-trajectory driving view synthesis.
    \item We explicitly model the uncertainty in map priors as a parameter and propose an uncertainty tempering scheme to guide the training process of the neural radiance field.
    \item Experiments demonstrated that the proposed method can improve the semantic consistency of out-of-trajectory views and the rendering quality of novel view trajectory interpolation.
\end{itemize}

Our proposed method is easy to implement, can be easily plugged into existing NeRF algorithms, and has the capability of extending to other formats of priors.

\section{Qualitative Background}
%
\label{sec:Generali}

The initial screening order problem is based on a collaborative work at an European Fortune Global 500. We refer to this company as G. This section summarizes our experience at G and how it motivated the problem. 

The goal of the collaboration was to study G's hiring process from an algorithmic fairness perspective. We worked closely with the teams of Advanced Analytics (AA) and Human Resources (HR), focusing on the candidate screening problem where an HR officer or screener selects a number of candidates based on their profiles. We mostly interviewed the HR officers to understand their tasks, constraints, and methodologies, often shadowing them during screening sessions. We concluded with a report to AA that formalized G's candidate screening problem as a ranking problem, evaluated the potential fairness implications, and assessed the risk and benefits of automation.

\subsection{The Hiring Pipeline}
\label{sec:Generali:HiringPipeline}

Hiring at company G consists broadly of three phases. In \textit{phase one}, HR builds a large enough candidate pool for a job opening. Here, ``large enough'' depended on the seniority and priority of the job opening. For a very senior position like Data Science Director, e.g., the candidate pool was often small (25 candidates max) as the amount of suitable candidates active in the job market was small throughout the year. The job opening includes a description of the ideal candidate provided by the hiring manager where candidates apply and/or are recruited accordingly. Candidates submit their CVs, complete relevant information in the form of multiple-choice questions, and occasionally provide a motivation letter. Sensitive information like a candidate's gender is always provided. The candidate pool is stored in Taleo\footnote{For more information, visit \url{https://www.oracle.com/human-capital-management/taleo/}.}, a database query platform for hiring by Oracle.

In \textit{phase two}, which covers the candidate screening problem, HR reduces the candidate pool into a smaller pool of suitable candidates to be invited for the interviews that take place in the next phase. The assigned HR officer or screener determines candidate suitability using a set of minimum basic requirements that all candidates must meet according to the hiring manager. Suitable candidates do not necessarily have to be ideal candidates. The existence of a third phase allows G to verify and update the information provided in each candidate's profile. The set of minimum basic requirements thus represents the non-negotiables for the hiring manager as well as a guide for flagging suitable candidates for the screener. 

Finally in \textit{phase three}, HR along with the hiring manager examine the subset of selected candidates from phase two across a series of interviews and aptitude tests. There is a first interview between HR and the candidates to asses the information on paper. If positive, candidates are interviewed and evaluated by members from the team offering the job opening. If positive, then the candidates at last meet the hiring manager (or partner) for the final call. At each interview, candidates are graded and their scores discussed among the interviewers. The grading is recorded in Taleo. The best candidates receive an offer, which they can accept or decline. If there is no match between the top candidates and the job opening, HR goes back to the runner-up candidates that remained in phase two and repeat phase three. 

\subsection{Candidate Screening}
\label{sec:Generali:CandidateScreening}

The choice to focus on phase two of the hiring pipeline was motivated by how appealing and dangerous it appeared for automation, offering an interesting tension between a time-consuming, repetitive task prone to human error and a high-risk, sensitive task requiring human oversight.\footnote{Automation seemed less appealing/reasonable for the other two phases: phase one was already streamlined with Taleo while phase three was clearly too human-dependent. This opinion was shared by G too.} When screening multiple candidates, HR officers clearly faced an overflow of information that had to be processed quickly. Phase two was crucial. It was not humanly possible, at least for a company of the size of G, to interview all candidates that applied, especially when we considered that HR ran multiple job openings at once. Given the nature of the task---that of extracting information from the candidate profiles and deciding whether it met a set of minimum basic requirements---the idea of developing a screening algorithm was, in principle, an appealing option to AA, HR, and us. 

Early on in the collaboration, however, it became clear to all stakeholders involved how human intensive and complex the candidate screening process was. It seemed unlikely that such process could be automated at all using the available data and available tools. It seemed even more unlikely when we considered the many unsuccessful AI-tools deployed in other companies (see, e.g., \cite{HiredByAlgoMITPodcast,TheAIWillSeeUMITPodcast}). Despite the known evidence on biased and inefficient human decision-making in similar settings to phase two (e.g., \cite{Kahneman2016Noise, Miller2018Bias, Kahneman2021Noise}), the goal of automation increasingly lost its appeal: the focus shifted away from the future algorithmic screener onto the present human screener. At best, we argued then and continue to do so, we could develop an algorithm to aid the human screener but never to replace it.  

This tension around automating phase two made us study closer the processes of the human screener. If the long-term goal was to create an algorithm that could aid the human screener, we first needed to identify instances where the human screener was likely to resort to biased (read, inconsistent) decision-making due to, say, cognitive heuristics (e.g., \cite{tversky_judgment_1974, Kahneman2011Thinking}). Hence, our focus on a \textit{human-like screener} throughout this paper. In Sections \ref{sec:ProblemFormulation} and \ref{sec:ISO} we formalize such screener and illustrate its role within the initial order screening problem.
We highlight three aspects that stood-out about phase two, which inspired the formulation of the initial screening order problem:
%
\begin{itemize}
    \item[(G1)] \textit{A varying initial order of the candidate pool.} As a default, Taleo sorted the candidate pool in alphabetical order. Each HR officer, however, was free to choose how to sort the candidate pool. The choice was restricted by the sorting fields offered by Taleo. One HR officer, e.g., preferred to sort the candidates using the submission date of the applications. The HR officers were consistent with their choice of initial screening order through multiple screening procedures.
    
    \item[(G2)] \textit{Multiple ways of searching the candidate pool.} Each HR officer decided how to search the candidate pool. What mattered is that they reached the quota of suitable candidates within a reasonable time. One HR officer, e.g., preferred to search the whole candidate pool while another stopped searching once meeting the quota. It took them one minute per profile. The HR officers were consistent with their choice of search procedure.
    
    \item[(G3)] \textit{Candidate meets minimum basic requirements.} Although the HR officer imposed an implicit order among candidates when screening the candidate pool, what mattered for those selected was that they met the set of minimum basic requirements as provided by the hiring manager. Again, the existence of the third phase allowed the HR officer to focus on getting enough suitable candidates on paper for an interview instead of finding the best possible candidates on paper for the job opening. This meant that order within the list of selected candidates was not necessarily important.
\end{itemize}
%

%
\begin{example}[Junior Data Scientist for Hire]
\label{RunningExample}
    As an illustrative example, suppose G wants to hire three junior data scientists. The hiring manager provides a list of qualifications describing the ideal candidate: a good command of English (B1 or above), a technical background (Engineering, Physics, or equivalent), Python programming skills (command at least of the \texttt{numpy}, \texttt{pandas}, and \texttt{scikit-learn} packages), and familiarity with EU banking regulation models (Basel II and III). The hiring manager sets as the minimum basic requirement that candidates meet two of these qualifications.
\end{example}
%

In job openings similar to that of Example~\ref{RunningExample}, a single HR officer usually had to select 15 to 30 candidates from a pool of 500 within a couple of hours. The HR officer would screen through each candidate profile, cheeking for the minimum basic requirements. A candidate with a B2 English and an Engineering degree but no programming skills in Python, e.g., would be considered suitable as candidate. A candidate with a B1 English but a Philosophy degree and no mention of Python skills would not be considered a suitable candidate. We stress that it is not about \textit{how suitable} the candidate is, but about whether the candidate is \textit{suitable or not}. 

\subsection{Fairness Goals}
\label{sec:Generali:Fairness}

We note that G, as it is common today with companies' efforts to have a diverse workforce, already had in place fairness goals in the form of \textit{representation quotas} for each job opening. These were, in fact, considered as key performance indicators, or KPIs, for the company. The main protected (or sensitive) attribute considered for such KPIs was gender. Throughout the hiring pipeline, when possible, HR would aim to reach a balanced list of candidates, meaning 50\% male and 50\% female candidates. 

In phase two, HR would aim to select a balanced subset of suitable candidates from the candidate pool. In practice, this meant imposing \textit{quotas} and \textit{threshold policies} with the goal of insuring some levels of female representation (or, in general, the under-represented group, which was always the protected group). HR, e.g., would apply different sets of minimum basic requirements to male and female candidates to meet certain representation goals specific to phase two. Such procedures are important, as we will see in the next sections, for the formulation of the initial screening order. We stress that G had well-established, working fairness policies already in place. Hence, what was of interest to us was understanding the role of the human-like screener in reaching these fairness goals.

%
% EOS
%


\section{Setting and Notation}
%
\label{sec:ProblemFormulation}

We now describe our setup in its most basic form. The goal is to formulate the set selection problem, where a decision-maker selects a set of items from a population, by considering the initial order in which the items are presented to the decision-maker. Here, the candidates for a job position represent the items and the screener evaluating their profiles represents the decision-maker.  

\subsection{Setting}
\label{sec:ProblemFormulation.Setting}

% The candidate pool
We consider a \textit{candidate pool} $\mathcal{C}$ of $n$ candidates. 
Each \textit{candidate} $c$ is described by the \textit{vector of $p$ attributes} $\mathbf{X}_c \in \mathbb{R}^{p}$ and the \textit{protected attribute} $W_c$. 
To simplify our analysis, we assume that $W$ encodes the membership to the protected group and is, thus, binary: with $W_c = 1$ if $c$ belongs to the protected group and $W_c = 0$ otherwise. 
The underlying protected attribute can be discrete, continuous, or it may consists of multiple attributes.

% The screener
The candidates are evaluated by a \textit{screener} $h \in \mathcal{H}$, where $\mathcal{H}$ denotes the set of available screeners. 
The following variables refer to a specific $h$.
The screener is tasked with choosing $k$ candidates from $\mathcal{C}$ based on each candidate's application profile as summarized by the tuple $(\mathbf{X}_c, W_c)$. 
The goal of the screener is to obtain a \textit{set of $k$ selected candidates} $S^k \in [\mathcal{C}]^k$, where $[\mathcal{C}]^k$ denotes the set of $k$-subsets of $\mathcal{C}$, i.e.,~subsets with cardinality $k$.
We model candidate evaluation by assuming that the screener uses an \textit{individual scoring function} $s \colon \mathbb{R}^{p} \to [0, 1]$, such that, given a candidate $c$, $s(\mathbf{X}_c)$ returns the score of $c$ according to $h$. 
It is implied that the screener does not use the protected attribute when scoring candidates. 
The higher the score, the better the candidate fits the job position based on the screener's judgment. 

% All possible orders and the initial order
The screener explores the candidate pool $\mathcal{C}$ in a specific order.
% when evaluating the candidates. 
We denote the set of total ordering of candidates in $\candidatesset$ by $\Theta$. 
It represents all possible ways in which the $n$ candidates in $\candidatesset$ can be arranged. 
An \textit{order} $\sigma \in \Theta$ maps an integer $i \in \{1, \dots n\}$ to a candidate $c \in \candidatesset$, indicating that $c$ occupies the $i$-th position according to $\sigma$, with notation $\sigma(i) = c$ and vice-versa $\sigma^{-1}(c) = i$. 
We are interested in the \textit{initial order} $\theta \in \Theta$, which refers to the order chosen by the screener before starting the exploration of $\candidatesset$ and the evaluation of their scores, according to fact G1 in Section~\ref{sec:Generali}. 
The screener is not required to explore the entirety of the candidate pool $\mathcal{C}$, according to fact G2 in Section~\ref{sec:Generali}.
However, we assume that the screener \textit{respects} the initial order $\theta$. 
Formally:
%
\begin{equation}
\label{eq:order}
    \mbox{a candidate $c_1 \in \candidatesset$ is evaluated before $c_2 \in \candidatesset$ only if $\theta^{-1}(c_1) < \theta^{-1}(c_2)$}
\end{equation}
%

% On meaningfulness
%\textcolor{red}{
%We say that $\theta$ is non-informative to $h$. 
%We assume that only after choosing $\theta$, and subsequently having searched, evaluated, and sorted the entirety of $\mathcal{C}$ as prescribed by $\theta$, can $h$ be aware of any meaning behind $\theta$. 
%This assumption ensures that $h$ does not knowingly introduce bias by arranging $\candidatesset$ into a specific $\theta$. 
%}
% All initial orders are meaningless to the set of screeners $\mathcal{H}$.
% On m steps travelled 
%The screener is not required to explore the entirety of the candidate pool $\mathcal{C}$.\footnote{This condition is based on fact G2 in Section~\ref{sec:Generali}.} 
%We denote as $\candidatessubset \subseteq \candidatesset$, the \textit{subset of candidates whose profiles are screened by} $h$, with cardinality $m \leq n$, implying the case where $h$ prefers not to evaluate the complete $\candidatesset$.
%For $\theta \in \Theta$, we refer to $\theta^{(m)}$ as \textit{the total order restricted to first $m$ elements according to the initial order}, so that $\theta^{(m)}$ maps an integer $i \in \{1, \dots m\}$ to a candidate $c \in \candidatessubset$.
% For a given $\sigma \in \Theta$, we refer to $\sigma^{(m)}$ as the total order restricted to the first $m$ elements, so that $\sigma^{(m)}$ maps an integer $i \in \{1, \dots m\}$ to a candidate $c \in \candidatesset_h$.

\subsection{Group Fairness}
\label{sec:ProblemFormulation.Fairness}

We address the fairness goals of the screener $h$ in choosing the set of $k$ candidates by assuming a quota for the protected group $W = 1$. 
Let us introduce the fraction $f\big( S^k \big) \in [0, 1]$ of protected candidates in the selected set $S^k$:
%
\begin{equation}
\label{eq:fairness_function}
    f\big( S^k \big) = \frac{\left\vert\{c \in S^k \text{ s.t. } W_c = 1\}\right\vert}{k}
\end{equation}

% \begin{equation}
% \label{eq:fairness_function}
%     f\big( S^k \big) = \left\vert\{c \in S^k \text{ s.t. } W_c = 1\}\right\vert / k
% \end{equation}
%
Let $q \in [0, 1]$ denote the desired fraction of chosen protected candidates in $S^k$. 
The fair screener will then be constrained to meet the \textit{representational quota} $q$ when deriving $S^k$, i.e.,~to satisfy the condition $f\big( S^k \big) \geq q$. 
Notice that the unconstrained version, in which no requirement is assumed on the representativeness of the protected group in $S^k$, is simply achieved by considering $q=0$.

We view $q$ as a policy implemented by the screener to achieve a diverse set of selected candidates. Hence, it is 
% We stress that $q$ is
a statement on the composition of $S^k$, not a statement on the ordering of protected candidates within $S^k$, according to fact G4 in Section~\ref{sec:Generali}.
% \footnote{Here, $q$ is based on fact G4 in Section~\ref{sec:Generali} as G's HR officers were motivated by company diversity goals, not by the fairness literature.} 
For $k=10$ and $q=0.5$, e.g., the fair screener would need to derive $S^k$ with $50\%$ of protected candidates though in no particular order, such as requiring the first five candidates in $S^k$ to be protected candidates.


\subsection{Set Selection: Two Problem Formulations}
\label{sec:ProblemFormulation.Objectives}

We now proceed to formulate two set selection problems for the screener $h$ tasked with deriving the set $S^k \subseteq \candidatesset$, where 
we distinguish between best-$k$ and good-$k$. 
Under the \textit{best-}$k$ formulation, the set $S^k$ represents \textit{the fair best $k$ candidates} in $\candidatesset$ according to $h$. 
We denote it accordingly as $S^k_{\besttext}$. 
% This is the standard formulation in the fair ranking literature~\cite{Zehlike2023_FairRanking_P1, Zehlike2023_FairRanking_P2}. 
Under the \textit{good-}$k$ formulation, the set $S^k$ represents \textit{the fair first good-enough $k$ candidates} in $\candidatesset$ according to $h$. 
We denote it accordingly as $S^k_{\goodtext}$. 
% This formulation, to the best of our knowledge, has been overlooked by the fair ranking literature.
For both formulations, we define the objective of the screener in terms of achieving an optimal and fair selection of candidates. 
How we define optimality, as we will show, derives the best-$k$ and good-$k$ formulations; we already defined fairness in the previous subsection. 

\paragraph{Best-$k$.}
We first focus on the screener that finds the set of best $k$ candidates in the candidate pool $\candidatesset$ given the fairness constraint $q$
and while respecting the initial order $\theta$ (recall (\ref{eq:order})).
%Noteworthy, here $h$ has to search the whole $\candidatesset$, meaning $\am{\candidatessubset} = \candidatesset$ and $m=n$ in this setting.
Notice that, here, $h$ must evaluate the complete $\candidatesset$.
This is because $h$ must score all candidates according to the individual scoring function $s$ before choosing the ones with the highest scores and for which $q$ is satisfied.
%meaning $h$ evaluates each candidate and sorts them by their scores, in order to speak of the best candidates.
%
%% SR
%Formally, the ranking $\tau \in \Theta$ represents an ordering of $\candidatesset$ that follows the scoring function $s$, such that, if $i \leq j$, $s(\mathbf{X}_{\tau(i)}) \geq s(\mathbf{X}_{\tau(j)})$.
%Given $\tau$, the cut-off for being selected into $S^k$ is $s(X_{\tau(k)})$ with $\tau(k)$ denoting the $k^{th}$ position in the ranking $\tau$.
%We can then model the goal of the best-$k$ selection, without fairness requirements, as:
%
%\begin{equation}
%\label{eq:objective_best-k}
%    S_{\tau}^k = \{\tau(1), \dots, \tau(k)\}
%\end{equation}
% 
We view the goal in terms of maximizing a utility for the screener $h$. 
We define \textit{utility} as the benefit derived by $h$ from selecting $k$ suitable candidates  given $\theta$.
% We define \textit{utility} as the benefit derived by $h$ from selecting $k$ suitable candidates for the job while respecting $\theta$ (recall (\ref{eq:order})). 
Formally, utility is a function $U^k \colon [\mathcal{C}]^k \, \times \, \Theta \to \mathbb{R}$. 
%Note that, in general, the utility function evaluated on a subset $S^k$ of candidates is influenced by the initial order $\theta$ by which the same candidates are seen. 
The simplest expression for defining $U^k$ is to add the scores of the chosen candidates:
%
\begin{equation*}
\label{eq:Utility}
    U^k_{\addtext} \big( S^k, \theta \big) = \sum_{c \in S^{k}} s\big( \mathbf{X}_{c} \big)
\end{equation*}
%
rationalizing that $h$ will maximize its utility by selecting the $k$ most suitable candidates for the job position. 
%It is easy to see that $S_{\tau}^k$ maximizes the utility $U^k_{\text{add}}$.
Note that
% , in the above utility definition, 
the initial order $\theta$ in \eqref{eq:Utility} does not affect the evaluation of $S^k$ because of the commutative property of addition.
We emphasize that \eqref{eq:Utility} is not the only possible model for the utility of the screener. 
Alternative models, such as exposure discounting \cite{DBLP:conf/kdd/SinghJ18}, can be considered for the best-$k$ problem formulation.
We leave this for future work.
%The goal \eqref{eq:objective_best-k} and utility \eqref{eq:Utility} describe the best-$k$ set selection problem for $h$. 
%If the screener $h$ is fair, then it must meet the representational quota $q$ of the protected group \eqref{eq:fairness_function}. 
 
We define \textit{the fair best-$k$ set selection problem} (or, simply, the best-$k$ problem) as:
%
\begin{equation}
\label{eq:fair_objective_all_screener}
    \begin{aligned}
    \argmax_{S^k \in [\mathcal{C}]^k} & \quad U^k_{\addtext} \big(S^k, \theta\big) \\
    \textrm{s. t.} & \quad f(S^k) \geq q
    \end{aligned}
\end{equation}
%
which, as readers familiar with the literature might notice, describes the standard top-$k$ formulation in fair ranking problems with $q$ representing some group-level fairness quota \cite{Zehlike2023_FairRanking_P1, Zehlike2023_FairRanking_P2}. 
We denote the
% \footnote{In presence of ties in scores, the solution may not be unique. In such a case, we consider any solution.} 
solution of \eqref{eq:fair_objective_all_screener} as $S^k_{\besttext}$.
In presence of ties in scores, the solution may not be unique. 
In such a case, we consider any solution.

\paragraph{Good-$k$.}
We now focus on the screener $h$ that finds $k$ candidates in the candidate pool $\candidatesset$ that meet a \textit{minimum basic requirements} $\psi$ given the fairness constraint $q$ and while respecting the initial order $\theta$.
% \footnote{This setting is based on facts G2 and G3 in Section~\ref{sec:Generali}.}
Unlike the best-$k$ formulation, here $h$ is not required to evaluate the whole $\candidatesset$ as it is enough to find the first $k$ candidates that are good-enough according to $\psi$ and that satisfy $q$. 
%Hence, the goal \eqref{eq:objective_best-k} and utility \eqref{eq:Utility} from the standard set selection formulation are not suitable in this setting.
%
We represent $\psi$ as a minimum score, such that $h$ deems candidate $c \in \candidatesset$ as eligible, or good-enough, for being selected if $s(\mathbf{X}_c) \geq \psi$.
%In the setting without fairness, unlike the goal of best-$k$, $S_{\tau}^k$ \eqref{eq:objective_best-k}, we cannot speak of a cut-off for the goal of good-$k$, because there is no ranking $\tau$ to slice from.

Under the good-$k$ setting, clearly the initial order $\theta \in \Theta$ has a significant influence on the screening process. 
To observe this point, let $k=1$ and assume $s(\mathbf{X}_{c_1}) \geq \psi$ and $s(\mathbf{X}_{c_2}) \geq \psi$. An initial order such that $\theta^{-1}(c_1) = 1$ and $\theta^{-1}(c_2) = 2$ would imply by (\ref{eq:order}) that $c_1$ is considered eligible before even evaluating $c_2$. Conversely, the reverse initial order $\theta^{-1}(c_1) = 2$ and $\theta^{-1}(c_2) = 1$ would consider eligible $c_2$ before $c_1$. Since the screener will stop after finding $k=1$ good enough candidates, the initial order affects which candidates are selected.

%let us call $\mathcal{C}_{\psi}$ the set of eligible candidates according to the minimum basic requirements $\psi$. 
%Evidently, we need to assume that $|\mathcal{C}_{\psi}| \geq k$.
%It is useful to denote $\Theta_{\psi}$ as the set of total orderings of $\mathcal{C}_{\psi}$.
%Under this view, $\theta$ induces a total order $\theta_{\psi} \in \Theta_{\psi}$ such that, given two candidates $c', c'' \in \candidatesset_{\psi}$, then $\theta_{\psi}^{-1}(c') < \theta_{\psi}^{-1}(c'')$ if $\theta^{-1}(c') < \theta^{-1}(c'')$.
%In this case, without the fairness condition, it helps to indicate the subset of $\candidatesset_{\psi}$ containing the first $k$ suitable candidates according to $\theta_{\psi}$ as $S^k_{\psi}$.
%Formally:
%
%\begin{equation}
%\label{eq:S_k_psi}
%    S^k_{\psi} = \{\theta_{\psi}^{-1}(i) \mid i = 1, \dots, k\}
%\end{equation}
%
%It is possible for two screeners $h_1$ and $h_2$ that chose different initial orders $\theta_1$ and $\theta_2$, under the same $s$ and $\psi$, to obtain different selected sets. 
%
%For this last scenario to hold, though, it must be unappealing, utility-wise, for both $h_1$ and $h_2$ to search the entirety of $\candidatesset$ according to $\theta_1$ or $\theta_2$.
%The notion of any $h$ optimally choosing $k$ candidates acquires a different meaning in the good-$k$ setting since the utility function $U^k_{\addtext}$ as described in \eqref{eq:Utility} is inadequate to formalize a screener for whom a selected set of $k$ good enough candidates is sufficient. 
%We define the simplest expression for the utility $U^k_{\psi}$ motivating the screener in the good-$k$ setting:
We still view the goal in terms of maximizing a utility for the screener $h$.
As a possible utility function in the good-$k$ setting, we first consider:
%
\begin{equation*}
\label{eq:AlternativeUtility2}
    \hat{U}^k_{\psi}\big( S^k, \theta \big) = \left\{
    \begin{array}{ll}
        n - \max_{c \in S^k} \theta^{-1}(c) & \text{if} \  \forall c \in S^k \  s(\mathbf{X}_c) \geq \psi   \\
        0 & \text{otherwise.}
    \end{array} \right.
\end{equation*}
%
Intuitively, in the above definition, the screener wants to find as quickly as possible a set of $k$ eligible candidates.
Therefore, if $S^k$ contains only eligible candidates, the utility of $h$ selecting $S^k$ under $\theta$ is expressed by the number of candidates past the last one who was screened, i.e.,~the ``saved effort" of the screener $h$.

Despite the simplicity of the above definition, the utility function (\ref{eq:AlternativeUtility2}) is not suitable to properly model our intended problem.
To observe this point, let $n=3, k=2, q=0.5$. 
Assume three eligible candidates and $\theta(1) = c_1, \theta(2) = c_2, \theta(3) = c_3$ with $W_1 = W_2 = 0$ and $W_3 = 1$. It turns out that both $S' = \{c_1, c_3\}$ and $S'' = \{c_2, c_3\}$ maximize the utility and satisfy the fairness constraint. However, why should have been $c_2$ considered, and then returned in $S''$, if $c_1$ already meets the minimum basic requirement? 
A reason for doing that is a variant of our problem, in which the screener keeps evaluating non-protected candidates, even if their quota is reached but the one of protected candidates is not yet reached, for the purpose of keeping the best ones found so far. 
We do not consider such a variant in this paper.

Let us define the \textit{penalty function} $p(c, S^k, \theta) = \mathbb{1}(\exists\ c' \in \mathcal{C}\setminus S^k \mbox{s.t.}\ \theta^{-1}(c') < \theta^{-1}(c) \wedge s(\mathbf{X}_{c'}) \geq \psi \wedge W_{c'}=W_c)$ which, for a candidate $c$, looks for another candidate of the same group as $c$ and meeting the minimum basic requirement, who occurs before $c$ in the order $\theta$, but who has not been selected into $S^k$. 
Basically, the penalty function models the ``wasted effort" in choosing a candidate occurring after another one meeting all the same requirements. 
At worst, the are $k$ penalties, which leads to the following refined utility function:
%
\begin{equation*}
\label{eq:AlternativeUtility}
    U^k_{\psi}\big( S^k, \theta \big) = \left\{
    \begin{array}{ll}
        k - \sum_{c \in S^k} p(c, S^k, \theta) & \text{if} \  \forall c \in S^k \  s(\mathbf{X}_c) \geq \psi   \\
        0 & \text{otherwise.}
    \end{array} \right.
\end{equation*}
%
%
%index of the last candidate to be evaluated, according to $\theta$, to complete the set of $k$ eligible candidates.
%The fewer candidates $h$ are evaluated to obtain $S^k$, the more utility the screener gets from the selection.
%Otherwise, if $S^k$ contains at least one non-eligible candidate, the utility of selecting $S^k$ is a zero.
%Evidently, $S^k_{\psi}$ of \eqref{eq:S_k_psi} maximizes the above utility, without including the fairness requirement.
% We come back to this utility function at the end of this section.

%If we include the fairness constraint, we 
We define then \textit{the fair good-$k$ set selection problem} (or, simply, the good-$k$ problem) as:
%
\begin{equation}
\label{eq:fair_objective_U_psi}
    \begin{aligned}
    \argmax_{S^k \in [\mathcal{C}]^k} & \quad U^k_{\psi} \big(S^k, \theta\big) \\
    \textrm{s. t.} & \quad f(S^k) \geq q
    \end{aligned}
\end{equation}
%
We denote the solution of (\ref{eq:fair_objective_U_psi}) as $S_{\goodtext}^k(\psi)$ or, if there is no ambiguity on $\psi$, simply as $S_{\goodtext}^k$. 
Note that,
if the fairness constraint is strengthened to a fixed quota, i.e.,~$f(S^k) = q$, it can be shown that the solution is unique. 
In the general case, i.e.,~$f(S^k) \geq q$, there can be two solutions, but with different fractions of the protected group. 
%%%v
% JA: if needed, move example to Appendix
%%%
For example, consider $n=3, k=2, q=0.5$. Assume three eligible candidates and $\theta(1) = c_1, \theta(2) = c_2, \theta(3) = c_3$ with $W_1 = 0$ and $W_2 = W_3 = 1$. Both $S' = \{c_1, c_2\}$ and $S'' = \{c_2, c_3\}$ are solutions of (\ref{eq:fair_objective_U_psi}) with $U^k_{\psi} \big(S', \theta\big) = U^k_{\psi} \big(S'', \theta\big) = 2$. However, $f(S') = 0.5$ and $f(S'') = 1$. Intuitively, $S'$ is obtained by strictly iterating over $\theta$, which is the approach we take in Section~\ref{sec:ProblemFormulation.Algorithms}. 
%%%^
% JA: if needed, move example to Appendix
%%%

%$s(\mathbf{X}_{c_1}) \geq \psi$ and $s(\mathbf{X}_{c_2}) \geq \psi$. An initial order were $\theta^{-1}(c_1) = 1$ and $\theta^{-1}(c_2) = 2$ would imply by (\ref{eq:order}) that $c_1$ is considered eligible before even evaluating $c_2$. Conversely, the reverse initial order $\theta^{-1}(c_1) = 2$ and $\theta^{-1}(c_2) = 1$ would consider eligible $c_2$ before $c_1$. Since the screener will stop after finding $k=1$ good enough candidates, the initial order affects which candidates are selected.

%This fact is useful for unifying both good-$k$ and best-$k$ settings into a single problem formulation.
%To highlight how distinct this setting is relative to the standard best-$k$, let us formulate the good-$k$ setting without needing to resort to specifying a goal or utility function.
%We will return to the above utility function at the end of this section.

%We argue that the screener $h$ can settle for looking through a subset $\candidatessubset$ of cardinality $m \leq n$ as long as it contains enough $k$ eligible candidates so that it also holds $m \geq k$. 
%Moreover, $\candidatessubset$ is composed by the set of the first $m$ candidates of the initial order $\theta$, so we can state that $\candidatessubset = \{c \in \candidatesset \mid \theta^{-1}(c) \leq m\}$. 
%Here, $\theta$ induces an ordering of $\candidatessubset$ captured by the initial order restricted to its first $m$ elements $\theta^{(m)}$. 
%Once the selected set has cardinality $k$, the screener has no incentive to look for more eligible candidates according to $\psi$.
%Hence, without the fairness condition, the selected set is the set of the first $k$ eligible candidates in $\mathcal{C}_{\psi}$ following $\theta_{\psi}$.
%Further, if the screener $h$ is fair, it must meet the representational quota $q$ of the protected group that meets the minimum basic requirements.
%We can restate problem \eqref{eq:fair_objective_U_psi} as:
%
%\begin{equation}
%\label{eq:Alternativefair_objective_all_screener}
%\begin{aligned}
%    \min_{m \in [k, \ldots, n]} \quad & m \\
%    \textrm{s. t.} \quad & |\{ \theta_{\psi}(i) \mid i=1, \ldots, m\}| = k \\
%    \quad & f(S_{\goodtext}^k) \geq q
%\end{aligned}
%\end{equation}
%
%where the selected set is exactly $S_{\goodtext}^k = \{ \theta_{\psi}(i) \mid i=1, \ldots, m\}$, so that it necessarily has cardinality $k$. 

%\paragraph{A fair initial screening order problem formulation.}
%To summarize, we represent both best-$k$ and good-$k$ settings into one formulation for the set selection problem based on the initial order $\theta$:
%
%\begin{equation}
%\label{eq:GeneralISOFormulation}
%    \begin{aligned}
%        \max_{S^k \in \mathcal{S}^k} & \quad U^k(S^k, \theta) \\
%        \textrm{s. t.} & \quad f(S^k) \geq q
%    \end{aligned}
%\end{equation}
%
%where we obtain two solutions for the optimal and fair screener $h$ depending on the utility function specification. 
%Under $U^k_{\addtext}$ as by \eqref{eq:Utility}, the solution is $S_\besttext^k$.
%In this case, $\theta$ is redundant as it does not influence $U^k_{\addtext}$. 
%This is because $h$ will always explore all of $\candidatesset$ and sort it as the ranking $\tau \in \candidatesset$ regardless of $\theta$. 
%Still, $h$ explores $\candidatesset$ as prescribed by $\theta$.
%Under $U^k_{\psi}$ as by \eqref{eq:AlternativeUtility}, the solution is $S_\goodtext^k$. 
%In this case, $\theta$ is a key input to $U^k_{\psi}$, despite not being a parameter that can be optimized by $h$. 
%This is because $h$ can partially or fully explore $\candidatesset$ as prescribed by $\theta$ depending on when it meets its goal.
%With \eqref{eq:GeneralISOFormulation} we present the general initial screening order problem formulation.

% Antonio: I've commented out again as we don't use meaningulness before (I changed that also as you suggested)
% For a screener $h \in \mathcal{H}$, the ordering insider the chosen subset $\mathcal{S}^{k}$ can be meaningful. In principle, $S^{k}_{\besttext}$ leads to a meaningful candidate selection relative to $S^{k}_{\goodtext}$ as candidates are ordered by their scores. In practice, however, it depends on how the chosen subset is used later on in the hiring pipeline, meaning whether in the next phase the order within $\mathcal{S}^k$ carries any information (e.g., interviewing candidates as described by $\mathcal{S}^k$). Hence, the chosen set can be meaningful but it is not of interest here as its meaning is determined beyond the screener $h$.

\subsection{Two Search Algorithms}
\label{sec:ProblemFormulation.Algorithms}

We present two search procedures for the best-$k$ and the good-$k$ problems, respectively. 
% Their names refer to the click models literature discussed in Section~\ref{sec:RelatedWork}. 
% \cite{DBLP:conf/clef/GrotovCMSXR15}.

The \textit{ExaminationSearch} procedure, shown in Algorithm~\ref{algo:Examination}, solves the best-$k$ setting, returning $S^k_\besttext$ for given $n$ (candidates), initial order $\theta$, and parameters $k$ (subset size) and $q$ (minimum fraction of selected candidates from the protected group). 
First, line 2 calculates the minimum number $q^*$ of candidates from the protected group to be selected, and the maximum number of candidates $r^*$ not in that quota. 
Then, candidates are considered by descending scores, using the \texttt{argsortdesc} procedure (lines 2, 3). 
The loop in lines 5-13 iterates until $k$ candidates are found. The loop adds candidates to the sets $Q$ and $R$: $Q$ are candidates in the quota of the protected group; $R$ are candidates not in that quota (can be non-protected or protected). An non-protected candidate can be only added to the $R$ set, thus line 7 checks if there is still room in $R$ to do this. A protected candidate is added to the quota set $Q$ if there is room (lines 10-11), or to the other set $R$ otherwise (lines 12-13). 
Finally, the procedure returns the candidates in the quota set $Q$ or in the other set $R$. 
The result of the \textit{ExaminationSearch} procedure clearly maximize (\ref{eq:fair_objective_all_screener}), as candidates are added in decreasing score, while keeping the fairness constraint through the quota management.

The \textit{CascadeSearch} procedure, shown in Algorithm~\ref{algo:Cascade}, solves the good-$k$ setting, returning $S^k_\goodtext$ for given $n$, initial order $\theta$, and parameters $k$, $q$, and $\psi$ (minimum basic requirement). 
The difference with the \textit{ExaminationSearch} procedure consists in strictly following the initial order $\theta$ (line 4), and in checking the minimum basic requirement (line 8) before adding a candidate to the quota set $Q$ or to the other set $R$.
The result of the \textit{CascadeSearch} procedure maximizes (\ref{eq:fair_objective_U_psi}), as no penalty is accumulated in the loop. 
In fact, an non-protected candidate ($W_c=0$) is not added only because  there is no room in $R$ -- and $R$ never gets smaller to allow for more room later on. A protected candidate ($W_c=1$) is not added only if not meeting the minimum basic requirement (line 8), hence it cannot be counted for the penalty. Formally, $U^k_{\psi}\big(S^k_\goodtext, \theta\big) = k$ for the solution $S^k_\goodtext$ returned by \textit{CascadeSearch}.

%
%\begin{definition}[Examination Search]
%\label{def:ExaminationSearch}
%    Given $\theta$, $h$ explores all of $\candidatesset$, computing each $c$ candidate's score $s(\mathbf{X}_c)$, ranking them all into $\tau$, and choosing the first $k$-candidates accordingly while ensuring that the fairness representational quota $q$ is met. 
%    Algorithm~\ref{algo:Examination} implements this search procedure.
%\end{definition}
%

%
%\begin{definition}[Cascade Search]
%\label{def:CascadeSearch}
%    Given $\theta$, $h$ explores $\candidatesset$ up until reaching $k$-candidates that meet the minimum basic requirements $\psi$ in terms of each candidate's $c$ score $s(\mathbf{X}_c)$ while ensuring that the fairness representational quota $q$ is met. 
%    Algorithm~\ref{algo:Cascade} implements this search procedure.
%\end{definition}
%

%
% Figure environment removed
%

% Both Algorithms~\ref{algo:Examination}~and~\ref{algo:Cascade} implicitly assume there are enough suitable $m$ candidates in $\candidatesset$ according to the minimum basic requirements $\psi$, such that $k \geq m \geq n$. 
% Hence, both achieve the respective solution. 
% These are the simplest algorithms for the examination and cascade searches, which can easily be extended by allowing the screener to update, e.g., $\psi$ of $k$ depending on the size of $S^k$ during the search.

%We highlight the fairness goal of obtaining a representative selected set of $k$ candidates as denoted by $q$.
%In Algorithms~\ref{algo:Examination}~and~\ref{algo:Cascade}, $q^*$, 
% or \texttt{line 2} for both of them, 
%represents the minimum number of protected candidates ($W_c = 1$) to be selected. 
%Hence, $k - q^*$ represents the maximum number of non-protected candidates ($W_c = 0$) to be selected.
%Under \texttt{ExaminationSearch} (Algorithm~\ref{algo:Examination}), meeting this quota is based on searching the full candidate pool according to $\theta$, sorting it by the scores in two separate lists according to $W$, and choosing the $q^*$-best protected candidates from $\tau_0$ and the $(k-q^*)$-best non-protected candidates from $\tau_1$.
%Under \texttt{CascadeSearch} (Algorithm~\ref{algo:Cascade}), meeting this quota goes on a candidate by candidate basis, thus, not requiring to search the full candidate pool according to $\theta$. 
%Here, the algorithm prioritizes constructing $S^k$ with $q^*$ eligible protected candidates $l_1$ while keeping track of the eligible non-protected candidates $l_0$.
%Once the quota is met, if between $l_1$ and $l_0$ there are not enough $k$ candidates then the algorithm keeps searching though without considering group membership. Notice that $l_0$ cannot be greater than $k-q^*$.

%
% EOS
%


\section{The Initial Screening Order Problem}
%
\label{sec:ISO}

In this section we show that the arrangement of the candidate pool prior to the screening, or the \textit{initial order} (Section~\ref{sec:InitialOrderDef}), matters for obtaining an optimal and fair selection of candidates under the human-like screener (henceforth, screener). The source of bias, as we show, comes from the position bias inherent in the initial order chosen by the screener. Let us explore the initial screening order problem.

\subsection{Two Search Algorithms}
\label{sec:ISO:SearchAlgorithms}

The screener can choose how to search the initial order $\theta$ (or $\mathcal{C}_{\mathbf{s}}$). We present the two search procedures available to the screener based on the HR officers at G. To focus only on the search procedures, we ignore the fairness representation goals here; we come back to this point in Section~\ref{sec:ISO:PositionBias}.

%
\begin{definition}[Cascade Search]
\label{def:Lazy}
    Under this search procedure, the screener $\mathbf{s}$ searches the initial screening order $\theta$ until it reaches $k$ candidates that satisfy the screening threshold $\psi$ based on the candidate scores $Y$ to derive \textit{select}-$k$. It ranges from a partial to a full search of the candidate pool $\mathcal{C}$, depending on how the suitable candidates are distributed in $\theta$. We summarize this search procedure in Algorithm~\ref{alg:Lazy}.
\end{definition}
%

%
\begin{definition}[Examination Search]
\label{def:Exhaustive}
    Under this search procedure, the screener $\mathbf{s}$ searches the initial screening order $\theta$, compute the candidate scores, sorts candidates based on the scores, and  then selects the first $k$ candidates to derive \textit{select}-$k$. It implies always a full search of the candidate pool $\mathcal{C}$. We summarize this search procedure in Algorithm~\ref{alg:Exhaustive}.
\end{definition}
%

The names of the search procedures are based on the clicking models that study position bias \cite{CraswellZTR08_ExperimentsClickPositionBias}. 
The cascade model explains the behavior of a user that stops or clicks on the first recommendation it finds useful, while the examination model explains the behavior of a user that examines multiple useful recommendations before stopping or clicking on the one it finds more useful. 
Definitions \ref{def:Lazy} and \ref{def:Exhaustive} extend these models into the screening setting where the user (i.e., screener) clicks on (i.e. evaluates) multiple recommendations (i.e. candidates). These clicking models, prior to the rise of fairness, were among the first to consider the implications of \textit{position bias} for the user. In the initial screening problem, position within $\theta$ is the source of bias. 
We explore this in detail in Section~\ref{sec:ISO:PositionBias}.

For both algorithms, we assume to have a sufficient number $m$ of eligible candidates within $\mathcal{C}$ under $\psi$. Formally, it occurs that $m = |\{ c \in \mathcal{C}\ |\ Y_c \geq \psi\}| \geq k$. In practice, however, it is possible for the screener to adjust its expectations by relaxing either $k$ or $\psi$ or both.

We also include in both algorithms the notion of fatigue, which accumulates over time into $\Omega$. Note that $\Omega$ is initiated as 0, or $\Omega(t_0)=0$, and increases by $\omega$ for each $c \in \theta$ candidate that is evaluated. If the screener evaluates $n' \leq n$ candidates, then $\Omega = n' \cdot \omega$. Here, $\Omega$ allows us to capture how tired the screener is after selecting the $k$ candidates. 

For illustrative purposes, fatigue appears as an implicit parameter: we can view $\Omega$ as a counter of something undesirable to the screener. In general, we could summarize the role of fatigue based on \eqref{eq:LowerMaxU} as $\underline{U}^* - \Omega(\underline{\tau})$, meaning the screener considers its fatigue after finding \textit{select}-$k$. In Section~\ref{sec:ISO:FatiguedScreener} we present one explicit form for the fatigue to enter into the screener's behaviour. Here, we use this general form to illustrate the overall role of fatigue as an undesirable thing to accumulate for the screener.

%
\begin{remark}[Baseline Model]
    \label{remark:Baseline}
    For illustrative purposes, we consider the scenario where the screener cannot get tired, $\omega = 0 \; \forall t$ in both Algorithms \ref{alg:Lazy} and \ref{alg:Exhaustive}. We refer to this scenario as the baseline model. Such screener still has the option to choose between a cascade or an examination search.
\end{remark}
%

%
\begin{algorithm}
\caption{CascadeSearch}\label{alg:Lazy}
\begin{algorithmic}[1]
%\Procedure{MyProcedure}{$x,y$}
\Require $\theta$, $k$, $\psi$, $\omega$ and $f$
\State $\text{select-}k \gets \text{list()}$
\State $\Omega \gets 0$
%\State $Y \gets \text{list()}$
%\While{$|\text{select-}k| < k$}
    \For{$c$ in $\theta$}
        \State $\mathbf{X}_c \gets \mathcal{C}[c]$
        \State $Y_c \gets f(\mathbf{X}_c)$
        \State $\Omega \; +\!= \; \omega$
        %\State $Y$\texttt{.append($(Y_c, c)$)}
        \If{$Y_c \geq \psi$}
            \State $\text{select-}k$\texttt{.append($c$)}
        \EndIf
        \If{$|\text{select-}k|=k$}
        \State break
    \EndIf
    \EndFor
%\EndWhile
\end{algorithmic}
\end{algorithm}
%

%
\begin{algorithm}
\caption{ExaminationSearch}\label{alg:Exhaustive}
\begin{algorithmic}[1]
%\Procedure{MyProcedure}{$x,y$}
\Require $\theta$, $k$, $\psi$, $\omega$, and $f$
%\State $\text{select-}k \gets \text{list()}$
\State $\text{select-}k \gets \text{list()}$
\State $\Omega \gets 0$
\State $\tau \gets \text{list()}$
\For{$c$ in $\theta$}
    \State $\mathbf{X}_c \gets \mathcal{C}[c]$
    \State $Y_c \gets f(\mathbf{X}_c)$ 
    \State $\Omega \; +\!= \; \omega$
    \State $\tau$\texttt{.append(($Y_c, c$))}
\EndFor
\State $s \gets $ sort $\tau$ w.r.t. second element 
\State $\text{select-}k \gets [ c$ for $Y_c, c$ in $s[0:k] ]$ 
%\State $Y$\texttt{.sort(key=lambda tuple: tuple[0])} 
% \Comment{Sort in decreasing order using the ranking scores}
%\State $\tau \gets$ \texttt{[i for i in $\tau$ if i $\geq \psi$]}
%\State $\tau \gets \tau[:k]$
%\State $\text{select-}k \gets$ \texttt{list(zip($^{*}\tau$))[1]} 
% \Comment{Extract the candidates from the sorted list of tuples}
\end{algorithmic}
\end{algorithm}
%

%
\begin{proposition}[The cascade search is weekly preferred over the examination search]
\label{proposition:CascadeVsExamination}
    Under sufficient $m \geq k$ eligible candidates in $\mathcal{C}$, with $|\mathcal{C}|=n > m$, 
    the human-like screener weakly prefers Algorithm~\ref{alg:Lazy} to Algorithm~\ref{alg:Exhaustive}.
\end{proposition}
%

We prove Proposition~\ref{proposition:CascadeVsExamination} in Appendix~\ref{App:Proofs}. We do so using the fatigue argument. Essentially, Algorithm~\ref{alg:Exhaustive} will always impose an accumulated fatigue of $\Omega=n \cdot \omega$ while Algorithm~\ref{alg:Lazy} will only require that much effort when the $k^{th}$ candidate is at the bottom of $\theta$. Otherwise, under Algorithm~\ref{alg:Lazy}, the screener will reach its goal of selecting $k$ suitable candidates with less fatigue.

\subsection{Fatigued Scores}
\label{sec:ISO:FatiguedScreener}

Now we consider a more explicit form in which fatigue affects the screener. Other functional forms are possible. We consider the specific scenario where the screener's fatigue directly affects its evaluation of each candidate, indirectly affecting its final utility. This formulation allows us to study how the screener's fatigue translates into potential inconsistent decision-making by the screener. Formally:
%
\begin{equation}
\label{eq:FatiguedScores}
    Y_c = f(X_c, \Omega(c))
\end{equation}
%
where $\Omega(c) = \tau^{-1}(c) \cdot \omega$ represents the accumulated fatigue of the screener after having evaluated up to the $\tau^{-1}(c)$-th candidate in $\theta$. It is natural to assume that $f(X_c) \geq f(X_c, \Omega(c))$, namely scores do not increase under fatigue. Also, for candidates $c_1, c_2$ with $\tau^{-1}(c_1) < \tau^{-1}(c_2)$ and $X_{c_1}=X_{c_2}$ we assume that $Y_{c_1} = f(X_{c_1}, \Omega(c_1)) \geq f(X_{c_2}, \Omega(c_2)) = Y_{c_2}$.

Proposition~\ref{proposition:CascadeVsExamination} still holds under the fatigued scores formulation \eqref{eq:FatiguedScores}. 
The fatigued scores allows us to be more precise on how fatigue overall affects the screening problem. Overall, all these formulations presented so far are attempts to characterize the fact that most screeners have no incentive(s) to explore the entire candidate pool for a job. 

\subsection{Position Bias}
\label{sec:ISO:PositionBias}

Position in the initial order $\theta$ matters. The screener, implicitly, sets a premium for the earlier positions in $\theta$ due to the fatigue term $\omega$ and lower-bound utility constraint $\underline{U}^k$. There is a \textit{position bias} in $\theta$. This bias is further intensified by the screener's weak preference for the Cascade Search over the Examination Search (Proposition~\ref{proposition:CascadeVsExamination}). In this section, we explore the optimality and fairness implications of the position bias in $\theta$. We focus on the Cascade Search (Algorithm~\ref{alg:Lazy}), though these results extend to the Examination Search (Algorithm~\ref{alg:Exhaustive}) too.

A clear consequence of this position bias is that the best suited candidate(s) might not be selected by the screener. Suppose we have $m$ suitable candidates in $\mathcal{C}$ where $|m| = k + 1$. Suppose further that the best suited candidate among these $m$ candidates appears at the end of the initial order $\theta$. Without even considering the screener's fatigue, under Algorithm~\ref{alg:Lazy}, such candidate will not be selected as the screener will stop once $k$ is met. The marginal gain of searching the initial order further does not compensate the screener when $\underline{U}^k$ is enough. Including fatigue into this setting makes searching further more costly. 

The position bias in $\theta$ also has fairness implications. Recall the fair representation constraint $\phi \in [0, 1]$ for the protected attribute $A$ that denotes membership ($A=a$) to the protected group. Defining $\phi=0.3$, for instance, would mean that 30\% of \textit{select}-$k$ should belong to $A=a$. While order within \textit{select}-$k$ is not important, the order of appearance in $\theta$ is important due to the position bias. Suppose $A$ contains two categories where $A=a'$ denotes no-membership to the protected group, then $n_{a} + n_{a'}=n$ in $\mathbf{C}$ and, respectively, in $\theta$. The top position of $\theta$ represents the best possible position for any candidate in $\theta$ given how the position bias increases over time for the screener. Therefore, the probability that the screener starts by evaluating a protected individual versus a non protected individual in $\theta$ is $n_{a}/n$ versus $n_{a'}/n$. 

%
\begin{definition}[Standard Fair Initial Order]
\label{def:StandardFairIO}
    We say the initial order $\theta$ is standard fair if it is independent from the protected attribute $A$. In other words, the choice of $\theta$ by the screener carries no information about membership to the protected group to ($A=a$) for all candidates in the candidate pool $\mathcal{C}$. 
\end{definition}
%

We assume Definition~\ref{def:StandardFairIO}. In practice, it means the screener does not use the choice of initial order $\theta$ to favor one group over the other, which is a realistic assumption. There are two consequences from this assumption. One, we cannot talk about any bias from $\theta$ beyond the inherent position bias already discussed. Two, we cannot talk about an unfair screener but instead about an unfair process. 

Regarding the fairness implications, in the initial screening order problem we are interested on how the burden of the position bias inherent to the initial screening order falls upon the groups in $A$. As the constraint $\phi$ translates into a quota for protected individuals in \textit{select}-$k$, let $k_a = \phi \cdot k$ represent the protected group quota and $k_{a'} = k - k_a$ the number of spaces available in \textit{select}-$k$ open to any candidate regardless of $A$. 

We rewrite Algorithm~\ref{alg:Lazy} to include the fairness constraint giving way to the Fair Cascade Search, or Algorithm~\ref{alg:FairLazy}. Besides including the $A$-specific quotas $k_a$ and $k_{a'}$, we incorporate the fatigued scores (Section~\ref{sec:ISO:FatiguedScreener}) as well as $A$-specific thresholds $\psi_{a}$ and $\psi_{a'}$. For now, let $\psi_{a} = \psi_{a'}$, meaning all candidates must meet the same screening threshold. Here, we let the fatigue enter as a linear term that lowers the score of the candidate. It, for instance, represents the inconsistency of the screener as it will fail to evaluate the same two identical candidates along $\theta$. As the fatigue accumulates, the more inconsistent the screener becomes. 

%
\begin{algorithm}
\caption{FairCascadeSearch}\label{alg:FairLazy}
\begin{algorithmic}[1]
%\Procedure{MyProcedure}{$x,y$}
\Require $\theta$, $k$, $\phi$, $\psi$, $\omega$, and $f$
\State $k_a \gets \phi \cdot k$
\State $k_{a'} \gets k - k_a$
\State $\text{count}_a \gets 0$; $\text{count}_{a'} \gets 0$
\State $\text{select-}k \gets \text{list()}$
\State $\Omega \gets 0$
%\State $Y \gets \text{list()}$
%\While{$|\text{select-}k| < k$}
    \For{$c$ in $\theta$}
        \State $\mathbf{X}_c, A_c \gets \mathcal{C}[c]$
        \State $Y_c \gets f(\mathbf{X}_c) - \Omega$
        \State $\Omega \; += \; \omega$
        \If{$A_c == a$}
            \If{$Y_c \geq \psi_a$ and $ \text{count}_{a} \leq k_{a}$}
                \State $\text{select-}k$\texttt{.append($c$)}
                \State $ \text{count}_{a}$ += 1
            \EndIf
        \Else
            \If{$Y_c \geq \psi_{a'}$ and $ \text{count}_{a'} \leq k_{a'}$}
                \State $\text{select-}k$\texttt{.append($c$)}
                \State $ \text{count}_{a'}$ += 1
            \EndIf
        \EndIf
        
        \If{$|\text{select-}k|=k$}
            \State break
        \EndIf
    \EndFor
%\EndWhile
\end{algorithmic}
\end{algorithm}
%

Under a consistent screener (i.e., one that cannot get tired), or the baseline model from Remark~\ref{remark:Baseline}, Algorithm~\ref{alg:FairLazy} meets the fairness goals summarized by the representational constraint $\phi$. In fact, such screener meets both group-level fairness (as prescribed by the composition of the \textit{select}-$k$) as well as individual-level fairness (as prescribed by the how the screener evaluates each candidate)\footnote{Here, in particular, the individual fairness definition often used is still that one by \citet{DBLP:conf/innovations/DworkHPRZ12}, where the same candidate should be treated similarly regardless of its position in $\theta$. \citet{DBLP:conf/chi/EchterhoffYM22}, e.g., phrase the anchoring bias in terms of this kind of individual fairness.}. This last point follows from Def.~\ref{def:StandardFairIO} and is also based on the assumption that there is no implicit bias in $\mathbf{X}$ nor in $f$ affecting the score $Y$ based on $A$. This is not true, however, under the (human-like) screener we consider here.

%
\begin{proposition}[Unfairness Implications of an Unbalanced Data]
\label{proposition:UnfairUnbalancedData}
    Let $\mathcal{C}$ be a candidate pool, and $\theta$  a standard fair initial screening order, such that $\mathcal{C}$ is unbalanced w.r.t. groups in $A$, i.e., $n_{a'} > n_{a}$ (more non-protected than protected individuals to be evaluated by the screener). Then the Cascade Search (Algorithm~\ref{alg:Lazy}) 
    introduces an unequal, cumulative burden of fatigue on the first $h$ examined candidates from the protected group compared to the first $h$ of the non-protected one, where $h \in [1, \min\{n_a, n_{a'}\}]$.  
\end{proposition}
%

We prove Proposition~\ref{proposition:UnfairUnbalancedData} in Appendix~\ref{App:Proofs}. The logic behind it is quite intuitive considering Algorithm~\ref{alg:FairLazy}. Under an unbalanced initial order $\theta$, the screener will come across more often with candidates from the non-protected and over-represented than the protected and under-represented group. If $\theta$ is non-informative of the protected attribute $A$ (Def.~\ref{def:StandardFairIO}), then we expect the non-protected and protected candidates to appear in $\theta$ based on their respective proportions $n_{a'}/n$ and $n_{a}/n$ in $\mathcal{C}$. As the screener accumulates fatigue in the form of $\Omega$ by going over $\theta$, it means that the screener be more tired when evaluating the first protected candidate than when evaluating the first non-protected candidate, and so on. 

An unbalanced $\mathcal{C}$ means that the protected, under-represented candidates have higher chances to face a more tired screener due to the accumulated fatigue from having evaluated mostly non-protected, over-represented candidates. This is the case for Algorithm~\ref{alg:FairLazy} despite the quota $k_a$. The burden of the screener's fatigue is only equally distributed among candidates in $A$ when $n_a = n_{a'}$.

Under an unbalanced candidate pool, the screener's fatigue affects disproportionally the under-represented group, which is in our case the protected group. In principle, this should not affect group-level fairness goals as captured by $\phi$ via the quota $k_a$. It does, however, affect individual fairness across and within the groups in $A$ as the screener becomes increasingly inconsistent over time. This was already the case in Algorithm~\ref{alg:Lazy}, but it can still be an issue for Algorithm~\ref{alg:FairLazy} despite the quotas. It is the (accumulated) fatigue of the screener that hinders fairness as well as optimality goals. Regarding fairness, in particular, the role of $\omega$ in Algorithm~\ref{alg:FairLazy} hinders the screener's capacity to evaluate similar candidates similarly, which violates individual fairness. It is not because the screener, the scoring function, or the provided information itself are unfair; it is the setup of the process. 

%
\begin{definition}[Fair Initial Order]
\label{def:FairIO}
    The initial order $\theta$ is fair for an algorithm $B$, if starting from $\theta$ as an input, $B$ distributes equally the burden of fatigue incurred by the screener over time across the protected and non-protected groups in $\theta$.
\end{definition}
%

With Def.~\ref{def:FairIO}, we shift the focus from considering the composition of the candidate pool $\mathcal{C}$ to how the protected and non-protected individuals are ordered prior to screening in the initial order $\theta$. What can seem as a seemingly uninteresting and fair (under Def.~\ref{def:StandardFairIO}) step, becomes a fairness matter when we factor in the potential inconsistencies. 

We also highlight the role of the $A$-specific screening thresholds in Algorithm~\ref{alg:FairLazy}. Ideally, the screener sets them equal to each other $\psi_{a} = \psi_{a'}$. However, this might not be possible under an unbalanced dataset as the protected and under-represented group might not have enough eligible candidates in $\mathcal{C}$. Moreover, setting a quota alone is not enough for achieving equality of fatigue .

%
\begin{proposition}[Fairness requires dynamically adapting screening thresholds]
\label{proposition:DifferentThresholds}
    Let $\mathcal{C}$ be a candidate pool, and $\theta$  a standard  initial screening order, such that $\mathcal{C}$ is unbalanced w.r.t. groups in $A$, i.e., $n_{a'} > n_{a}$.
    Let $\beta_a$ and $\beta_{a'}$ be the proportion of eligible candidates for the protected group 
    and the
    unprotected one respectively. To achieve equality of fatigue over the two groups in the Fair Cascade Search (Algorithm~\ref{alg:FairLazy}), the screening thresholds $\psi_a$ and $\psi_{a'}$ must be dynamically adapted. 
\end{proposition}
%

We sketch a proof for Proposition~\ref{proposition:DifferentThresholds} in Appendix~\ref{App:Proofs}. Intuitively, we view it as the consequence of the implicit population numbers. If we expect the protected and non-protected candidates to be distributed similarly in terms of aptitude for the job opening (measured by the screening threshold $\psi$), then we should expect similar candidates profiles across $A$ under a balanced candidate pool $\mathcal{C}$. When this does not occur, meaning the protected group is also the under-represented group, then $\mathcal{C}$ and in turn $\theta$ might be unable to provide the same number of eligible candidates across $A$. The screener then faces the choice to relax the the representation goals $\phi$ according to the available $\mathcal{C}$, or dynamically adapting the screening thresholds. In practice, it means the screener as it will come across more suitable non-protected candidates can be more strict with the minimum basic requirements for such candidates and, conversely, it can be less strict with the protected candidates knowing that the suitable candidates are scarce. 

These conclusions lead us to argue that the fairness goals, even under the ideal fair screener, are supply dependent. In other words, in practice, companies like G are from the beginning constraint by the candidate pool they have available for completing a job opening. All parameters considered in Algorithm~\ref{alg:FairLazy} are susceptible to the realities of $\mathcal{C}$. The initial order $\theta$ materializes these concerns through its inherent position bias. 

%
% EOS
%


\section{Related Work}
%
\label{sec:RelatedWork}

\paragraph{Estimating a scoring function}
The scoring function $f$ in Section~\ref{sec:ProblemFormulation} con be approximated using past screening data to obtain $\hat{f}$. In terms of the initial screening order, $\hat{f}$ represents the algorithmic screener. Two main approaches exist for obtaining $\hat{f}$: score-based ranking (SBR) \cite{Zehlike2023_FairRanking_P1} and learning-to-rank (LTR) \cite{Zehlike2023_FairRanking_P2}, where in the first a function is given to calculate the scores while in the second the function is learnt. The initial screening problem, we argue, requires elements of both approaches. 
What is clear to us from the interaction with the company G (Section~\ref{sec:Generali}), is that the human-like screener acts within the SBR framework (as underlined by $U^{*}$ in Section~\ref{sec:ProblemFormulation}). If it were (humanly) possible, such screener would go over all candidates in the candidate pool $\mathcal{C}$. The issue for obtaining $\hat{f}$ is not only that $f$ remains a mental process, but so do the scores $Y$: the \textit{selected}-$k$ is recorded, but never their individual scores.
In that regard, the LTR framework is appealing. Based on past candidate screenings and using LTR methods like RankNet~\cite{Burges2010ranknet} and ListNet~\cite{Cao2007learning}, we could learn $\hat{f}$ based on the characteristics of the candidates within and outside past \textit{selected}-$k$ sets. These methods are often used for providing suggestions to the user. The issue for obtaining $\hat{f}$, though, now becomes the lack of order within \textit{selected}-$k$. There is a difference in training a ranking algorithm that recommends the best item for the user over one that recommends $k$ items all interchangeable for the user (as underlined by $\underline{U}^{k}$ in Section~\ref{sec:ProblemFormulation}). 

We argue that the initial screening order problem, opens new formulations for both SBR and LTR (the fatigued scored \eqref{eq:FatiguedScores} in Section~\ref{sec:ISO:FatiguedScreener}, e.g., hints at this line of work). Estimating $\hat{f}$ to aid HR officers is an ultimate goal, but we need $\hat{f}$ to capture the role of fatigue, the influence of the initial order (and the fact that HR officers can alter it), and the objective of not always selecting the best possible candidates. With the initial screening order problem, we have highlighted a complex problem that, to the best of our knowledge, has not been tackled explicitly by the ranking literature.

\paragraph{Modeling position bias}
Position bias states that there is a premium for being, usually, at the top of a platform like a list or a website. Human are predisposed to favor those items, leading to biased decisions \cite{DBLP:journals/cacm/Baeza-Yates18}. Position bias falls under technical bias in the fair ranking literature \cite{DBLP:journals/vldb/PitouraSK22,Zehlike2023_FairRanking_P1,Zehlike2023_FairRanking_P2}. 
This behaviour was formalized (e.g., \cite{CraswellZTR08_ExperimentsClickPositionBias}) and tested (e.g., \cite{DBLP:journals/jcmc/PanHJLGG07, DBLP:conf/clef/GrotovCMSXR15, DBLP:conf/www/RichardsonDR07}) early on by the click model literature. The name choice of the two search procedures in Section~\ref{sec:ISO:SearchAlgorithms} is a reference to this line of work. Position bias, probably given its link to click models and current recommender system problems, is addressed within the LTR framework \cite{Zehlike2023_FairRanking_P2}. 

Fairness-wise, two lines of work---probability-based fairness (PBF) and exposure-based fairness (EBF)---tackle position bias. In PBF \cite{DBLP:conf/ssdbm/YangS17, DBLP:conf/cikm/ZehlikeB0HMB17}, we constraint the ranking algorithm to (re-)arrange the candidates under some fair distribution like a tossing a fair coin. In EBF \cite{DBLP:journals/cacm/Baeza-Yates18, DBLP:journals/sigir/JoachimsGPHG17}, we instead model attention, which decreases geometrically or logarithmically as the user faces a longer list of items, and constraint the ranking algorithm to (re-)arrange the candidates such that candidates receive similar exposure (e.g., \cite{DBLP:conf/kdd/SinghJ18}). The initial screening order, with its focus on fatigue, relates to the EBF works. We interpret fatigue as inversely proportional to attention: the more tired the screener, the less attention it gives to a candidate. Such link allows to use EBF to tackle our problem.

The initial screening problem, however, challenges current works on position bias. It is important to separate the bias itself from its effect on deriving a fair ranking. Position bias seems to be inherent to humans. EBF handles this by pre-emptily accounting for it and returning rankings that, e.g., are fair in exposure. In EBF problems we are returning rankings, meaning the user is expecting to go over a meaningful list of ordered items (like a Google search result or a recommendation of candidates to hire). This is not, however, the case for the initial screening order problem where the HR officer plans to make its own assessment of the list of candidates and, thus, attaches no meaning to it. That is way we make the distinction between the candidate pool $\mathcal{C}$ and the initial order $\theta$. EBF methods could be useful for providing a $\theta$ that accounts for fatigue, but it should do so without providing it in the form of recommendations. It is unclear from our experience with company G (Section~\ref{sec:Generali}) whether obtaining $\hat{f}$ is possible under such complex process; hence, it is unclear to us how the solutions proposed under EBF or even PBF can address the position bias when the goal is to return a fair initial order (Def.~\ref{def:FairIO}) instead of a fair ranking of candidates. 

We highlight recent work by \citet{DBLP:conf/chi/EchterhoffYM22} that focuses in capturing and balancing anchoring bias \cite{Kahneman2011Thinking} in sequential decision-making. Such bias occurs, e.g., when a given candidate is evaluated following, say, two very bad versus two very good previous candidates: that same candidate has a higher chance of a positive decision in the former scenario. This is because the screener anchors its expectations on a lower reference point despite each candidate being independent from each other. These researchers worked closely with a university to understand and model their admission process and proposed an algorithmic procedure to balance the anchoring bias. In that sense, our work also prioritizes the role of the user in the problem formulation and, like \cite{DBLP:conf/chi/EchterhoffYM22}, links future work between LTR, EBF, and theories on human decision making (see \cite[Ch10]{DBLP:books/daglib/0033056} and, e.g., \cite{DBLP:conf/chi/CarabanKGC19, DBLP:journals/isr/AdomaviciusBCZ13}).

\paragraph{Understanding candidate screening.}
With the exception of \cite{DBLP:conf/chi/EchterhoffYM22} and \cite{SukumarMH18_PeacanPie}, no other paper offers detailed insights on the process of candidate screening. Both of these papers focused on college admissions, with \cite{DBLP:conf/chi/EchterhoffYM22} proposing an algorithmic procedure for anchoring bias while \cite{SukumarMH18_PeacanPie} the use of visual tools to support the decision maker. We join these paper in stressing the complexity of candidate screening and share their view in using technology to aid the human decision maker. To the best of our knowledge, we offer the first formalization of hiring in candidate screening. 

%
% ESO
%

% \paragraph{Wed. 08/03.} In going over once again Meike's fair ranking survey, I came across the \textit{click model} literature. These ones were hidden under \textit{technical bias} and refer to practices that encourage bias. In particular, it refers to \textit{position bias}. Humans, for example, tend to read from top-to-bottom and, thus, associate items positioned at the top of a list as more important. Unsurprisingly, the ranking technologies reflect this bias: Google searches will put on top the most relevant search to your query; football leagues will order the classification board in terms of the leader, and so it goes. 

% The click model literature is interesting as it precedes fairness and the overall ML boom but still identifies the issue of bias. Based on user experiments (mostly done using eye-tracking technology), this literature assumes the position bias and tries to explain it via clicking models to address it. From ``In Google We Trust'', e.g., researchers found that humans will click on the top ranked item despite being the one with less relevance to the search. Both ``A Comparative Study of Click Models for Web Search'' and ``An Experimental Comparison of Click Position-Bias Models'', to name a few, propose/study several click models to see which one explains human behaviour better. The consensus seems to be: one, there is such a thing a position bias, two, no click model seems to outperform the other even when using several performance measures.

% Now, it is important to distinguish the setting from click models from the one studied in Generali. In the former, the user assumes that the search engine (or overall ranker) provides a ranking that is meaningful: i.e., the positions reflect the relevance of the item. In the latter, this is not the case: the HR platform offered many, non-meaningful ways to order the list of candidates but the HR screener was well aware that the first candidate on the list was not necessarily the better suited or more relevant one. The issue, of course, is whether this position bias can have unconscious effects. In that sense, the notion of fatigue, $\omega(t)$, is interesting as we just say the screener gets tired. The position bias arises because nobody wants to be reading a list of candidates all day... It is conceptually appealing to frame it as such. It also aligns with the notion of the \textit{decaying attention curve} used in click models where it is known that attention is scarce and it decays as the user goes over the ranked items. Hence, the position bias seems equivalent across these framings of the problem and, to an extent, unavoidable.

% Formally, say for candidate list $\mathcal{C}$ with $|\mathcal{C}|=n$ candidates, let $\mathcal{R}$ represent the set of all possible rankings of $\mathcal{C}$. In other words, it denotes all possible permutations. Clearly, $\mathcal{R}$ includes (assuming no tied items/candidates) the optimal initial ordering $r_{io}^*$, meaning the \textit{meaningful ordering for the screener} such that $r_{io}^*[1]$ represents the best suited candidates for the job or, in general, the most relevant item. $\mathcal{R}$ includes a given $r_{io}$ \textit{initial order ranking} that is not meaningful to the screener (though, it is possible for $r_{io} = r_{io}^*$ despite the screener being unable to know that: i.e., it does not matter to the problem setting. It is also very unlikely, with probability: $1/n!$.

% Therefore, in terms of related work, we need to \textit{(i)} consider the potential unconsciousness/unconscious ways the position bias materializes despite knowing that $r_{io}$ is not meaningful and \textit{(ii)} whether the human search algorithms (or models of behavior) relate to existing models like the clicking ones?

% Regarding \textit{(ii)}, I liked the models presented in ``An Experimental Comparison of Click Position-Bias Models''. It presents four models, though the mixed model would only apply if we allowed the screener to read the same $r_{io}$ more than once and also to be inconsistent in his/her search strategy (future work maybe?). We consider the following and link them to the IO problem:
% %
% \begin{itemize}
%     \item Baseline Model: there is no position bias or, equivalently, the screener never gets tired: $\forall t \omega(t)=0$. 
%     \item Examination Model: similar to the ExhaustiveSearch.
%     \item Cascade Model: similar to the LazySearch.
% \end{itemize}
% %
% where we must consider that these models are based on ``clicking'' once and stopping while we consider the case where we ``click'' $k$ times to reach \textit{select-k}. For instance, in the \textit{examination model}, a user looks over the ranking almost exhaustively and then decides where to click. Also, we must consider that these models are based on \textit{the user assuming that the initial order of the ranking is meaningful}. We, thus, should refer to them and, if we choose to use the same terminology, expand them according to our setting.

% Regarding \textit{(i)}, it seems it is a question on all possible $r_{io}$ rankings and, overall, a question \textit{sequential decision-making}: how can bias manifest itself when looking at many items sequentially with or without a meaningful order? In that sense, we need consider, one, these biases and, two, if there are studies on ExhaustiveSearch and LazySearch strategies.

% There are some interesting works in the FindHR folder, though I cannot find ones that relate directly to the setting studies here. We have a could on HR hiring and how AI could help/worsen fairness/bias. We also have one (``Sifting and Sorting'') that examines the hiring practices in a bank (though in 1997) and how personal contacts influence the sequential decisions. None of these seem to consider fatigue as a parameter (the one on HR tools to reduce bias do so indirectly). In that sense, the literature on judgement heuristics / nudging is closer to us in the sense that ``we known there is a bias due to mental processes and want th platform or ADM to help reduce this brisk of bias''. Still, in this line of work, I do not seem to find explicit formalizations on search strategies (as, e.g., the click model literature does for the WWW users). Maybe we have a found a bridge here?

% Similarly, we must be honest about what is implicit in our translation of the human screener: some degree of optimal decision making on the basis of rational thinking. At a minimum, we assume an agent that wants to reach a goal and minimize the task duration to some degree: time is limited; otherwise, the screener could spend as much as needed to build the \textit{select-k} set without getting tired.

% Along the lines of decision-making models, some of interest: \textit{the effort accuracy framework}, and \textit{the preference construction framework}. In the former, the agent knows what it wants and will try to find it while keeping in mind the accuracy-effort trade-off, meaning that the better suited candidate is not the optimal one if it lies at the bottom of the list... In the latter, instead, the agent does not know what it wants explicitly and forms its preferences as it starts searching, meaning the way the list is presented influences the final decisions. It seems that both frameworks can affect our screener in the Lazy and Exhaustive searches, meaning that it will depend on what we want to assume. Under the first framework, we are in a more algorithmic setting: try to minimize the risk that the desired candidates require a disproportionally high search effort. While under the second framework, we in a more cognitive setting: make sure that the platform or list nudges/helps the agent to make the right decision buy building the best preferences. For the Generali setting and ranking problem formulation, the effort accuracy framework comes more natural. The preference construction framework seems more natural for human-computer interaction works. Still, both frameworks are Related Work, no? Consider the Chapter 10 from the FindHR folder.

% Under preference construction, relevant to us are: \textit{primacy/recency effects}; \textit{priming}; \textit{defaults}... though I feel like, one, this is not much relative to the 1970s Judgment Heuristics, and, two, the nudging literature seems more robust than this. Also, the clicking models indirectly address some of these issues... its main default is that we assume one click, which may not translate to all decision-making scenarios. That said: we could frame the two searches in terms of \textit{personalities}. In particular, \textit{the maximizer} for the LazySearch and \textit{the satisficer} for the ExhaustiveSearch. Though, these are more psychology definitions... I have a preference for defining the agents in terms of fatigue or even some sort of \textit{memory parameter} to capture the effects of primacy/recency or priming or default. 


\section{Conclusion}
\section{Discussion}
\label{sec:discussion}
To assist AI practitioners in navigating the rapidly evolving landscape of AI ethics, governance, and regulations, we have developed a method for generating actionable guidelines for responsible AI. This method enables easy updates of guidelines based on research papers and ISO standards, ensuring that the content remains relevant and up-to-date. We validated this method through a use case study at a large tech company, where we designed and evaluated a tool that uses our responsible AI guidelines. We conducted a formative study involving 10 AI practitioners to design the tool, and further evaluated it through an interview study with an additional 14 AI practitioners. The results indicate that the guidelines were perceived as practical and actionable, promoting self-reflection and enhancing understanding of the ethical considerations associated with AI during the early stages of development. In light of these results, we discuss how our method contributes to the idea of ``Responsible AI by Design'', that is, a design-first approach that considers responsible AI values throughout the development lifecycle and across business roles. We discuss the inherent problem of decontextualization in responsible AI toolkits, the concept of meta-responsibility, and provide practical recommendations for designing responsible AI toolkits with the aim of fostering collaboration and enabling organizational accountability.

\subsection{Theoretical Implications}

\subsubsection{Decontextualization}
The inherent challenge in responsible AI toolkits lies in their attempt to reconcile the tension between scalability and context specificity~\cite{wong2023seeing}. Traditional approaches to toolkit development have often favored a universal, top-down approach that assumes a one-size-fits-all solution~\cite{kelty_participatory_toolkit, mattern_toolkit}. However, participatory development, such as the methodology we followed in designing and populating a responsible AI toolkit with our guidelines, emphasizes the importance of tailoring responsible AI guidelines to specific contexts and job roles needs. It is crucial therefore to recognize that different AI practitioners, such as designers, developers, engineers, and advisors, have distinct requirements and considerations that cannot be treated as identical. This highlights the complexity of developing toolkits that cater to a diverse range of practitioners while accounting for their unique roles and settings---the problem of decontextualization in responsible AI toolkits~\cite{wong2023seeing}.

To tackle the problem of decontextualization, our proposed method incorporates two key elements: \emph{actionable guidelines} and \emph{follow-up questions}. Firstly, the integration of actionable guidelines, tailored to different roles and projects, provides practical steps and recommendations that technical practitioners can easily implement, or C-level executives can make informed decisions upon. These guidelines serve as a starting point for ethical decision-making throughout the AI lifecycle, contributing to the vision of responsible AI by design (borrowing from the idea of `privacy by design'\footnote{``Privacy by design'' is a standard practice for incorporating data protection into the design of technology. In other words, data protection is achieved when it is already integrated into the technology during its design and development~\cite{cavoukian2009privacy}.}). Secondly, the inclusion of follow-up questions enhances our toolkit's ability to capture the complexities of different social and organizational contexts. Expanding upon the concept that follow-up questions are an effective means of communication~\cite{weger2014relative}, as they help in gaining deeper insights, clarifying responses, and uncovering underlying meanings, AI practitioners can engage with these questions to explore the ethical considerations and challenges that are unique to their deployment context.

\subsubsection{Meta-responsibility} Scholars have long recognized the need for a socio-technical approach that considers the contextual factors governing the use of AI systems, including social, organizational, and cultural factors~\cite{tahaei2023toward}. In fact, Ackerman~\cite{ackerman2000intellectual} introduced the concept of socio-technical gap to highlight the disparity between human requirements in technology deployment contexts (socio-requirements) and the technical solutions. This gap arises due to the flexible and nuanced nature of human activity compared to the rigid and brittle nature of computational mechanisms, resulting from necessary formalization and abstraction. Along these lines, \citet{stahl2023embedding} introduced the concept of meta-responsibility to stress that AI systems should be viewed as systems of systems (ecosystems) rather than single entities. To establish a regime of meta-responsibility, Stahl argued for an adaptive governance structure to effectively respond to new insights and external influences (e.g., upcoming AI regulation), and for a knowledge base that equips AI stakeholders with technical, ethical, legal, and social understanding. By integrating ethical, legal, and social knowledge into the AI development process---what Stahl referred to as adaptive governance structure, and offering recommendations for areas that require additional attention (i.e., responsible AI blindspots), our work contribute to this line of research by providing empirical evidence to it and pushing the theoretical boundaries further.

\subsection{Practical Implications}

\subsubsection{Recommendations for designing responsible AI toolkits}
Our responsible AI guidelines, populated in a usable tool, leverage the concept of nudging to encourage users to consider the ethical implications of AI systems. Nudging has demonstrated effectiveness in various domains, such as mitigating the dissemination of misinformation on social media through the use of checklists~\cite{jahanbakhsh2021exploring}, or guiding users towards more private and secure choices~\cite{acquisti2017nudges, tahaei2021deciding}.

Nudges can be implemented in various ways. For instance, the \emph{confront} type of nudge incorporates elements of ``reminding consequences'' and ``providing multiple viewpoints,'' encouraging users to consider alternative directions and diverse perspectives~\cite{caraban2019ways}. In the case of our guidelines, these two concepts are utilized to remind AI developers about the ethical considerations of AI systems and to prompt them to think critically about alternative viewpoints, thus helping them avoid confirmation bias. Further research could explore additional types of nudges, such as incorporating visual cues (e.g., \emph{just-in-time nudges} within development tools), facilitating positive behavior (e.g., \emph{enabling social comparisons} by recognizing and appreciating developers who promote ethical values within the organization), or fostering empathy (e.g., \emph{instigating empathy} by presenting the environmental impact of an AI system through easily understandable animations).

While the format of our tool proved to be useful, it offers a starting point to explore other formats and interactions for populating and contextualizing the guidelines. For example, structuring the guidelines into a narrative might be useful to unpack the complexity of particularly complex guidelines, such as guideline \#15---\emph{ensuring compliance with agreements and legal requirements when handling data.} This guideline can be further sub-divided into sequential steps providing more context and explanations. Moreover, future responsible AI tools can incorporate configurable parameters or customization widgets to align with specific requirements of the developed AI systems or user preferences. Additionally, the use of Language Models (LLMs) can be explored to further customize and adapt the provided examples within the tool. Finally, more research can be done on exploring responsible AI tools as a method for artifact creation. This includes automatic generation of summary reports, model cards, or responsible AI certificates.

\subsubsection{Recommendations for fostering collaboration and enabling organizational accountability}
While individual adoption of responsible AI best practices is crucial, promoting collaboration among diverse AI stakeholders is equally important. Many existing responsible AI toolkits prioritize individual usage~\cite{wong2023seeing}. However, addressing complex ethical and societal challenges associated with AI systems requires collaborative actions. Our interactive tool populated with actionable guidelines addresses this need by offering features that facilitate collaboration. First, the tool stores users' inputs in a responsible AI knowledge base, enabling distributed teams to access and leverage this knowledge for a shared understanding of a particular AI system. This promotes collaboration and a collective approach to ethical considerations. Second, the tool automatically generates a report that summarizes the user's considerations. This report can be downloaded as a PDF and includes responsible AI blindspots, which are specific actions to be taken by individuals or shared among the development team. Highlighting these blindspots fosters awareness and prompts collective action towards responsible AI practices.

In addition to fostering collaboration, our interactive tool can be used to enable organizational accountability. Similar to Google's five-stage internal algorithmic auditing framework~\cite{raji2020closing}, our guidelines serve as a practical tool for closing the AI accountability gap. The automatically generated report plays a crucial role in this process by providing a summary of the guidelines that were effectively implemented, those that should be considered for future development, and the non-applicable ones. These reports establish an additional chain of accountability that can be shared with stakeholders at various levels, including managers, senior leadership, and AI engineers. By offering more oversight and the ability to troubleshoot if needed, these reports help mitigate unintentional harm. However, it is important to note that when an organization adopts our guidelines, it should establish clear ethical guidelines for their intended uses. Our tool is not intended to discourage developers from using it due to the fear of being held accountable for their responses. On the contrary, developers' responses, as documented in the report, provide an opportunity to identify potential ethical issues and address them early in the design stages. This proactive approach prevents the need for post-hoc fixes and repairs, aligning with the principle of addressing ethical considerations during the development process rather than as an afterthought~\cite{sambasivan2018toward}---the idea of \emph{Responsible AI by Design}.

\subsection{Limitations and Future Work} Our work has four main limitations that highlight the need for future research efforts. 

Firstly, although we followed a rigorous four-step process involving multiple stakeholders, the list of 22 guidelines may not be exhaustive. The rapidly evolving nature of AI ethics, governance, and regulations necessitates an ongoing effort to stay abreast of emerging developments. However, one of the strengths of our method lies in its modular design, which allows for ongoing refinement and expansion of the set of guidelines. This ensures that our responsible AI tool maintains its relevance and stays up to date in the ever-evolving landscape of AI ethics, governance, and regulations. As new ISOs are established, addressing specific aspects of AI systems such as functional safety (ISO 5469), data quality (ISO 5259), and explainability (ISO 6254), our tool can be readily extended to include these guidelines. Moreover, as the scientific community progresses in its understanding of ethical considerations in AI, our tool can incorporate new insights and recommendations to enhance its comprehensive coverage.

Secondly, it is important to consider the qualitative nature of our user study, which involved in-depth interviews and analysis of participants' responses. The findings from this study should be interpreted with caution, understanding that the reported frequency of themes should be viewed in a comparative context rather than taken at face value~\cite{fossey2002understanding}. This approach helps to avoid potential misinterpretation or overgeneralization of the results.

Thirdly, we need to acknowledge the limitations associated with the sample size and demographics of our user study. The study was conducted with a specific group of participants, and therefore, the findings may not fully represent the practices and perspectives of all AI practitioners. Our sample predominantly consisted of male participants, which aligns with the gender distribution reported in Stack Overflow's 2022 Developer Survey, where 92.85\% of professional developer respondents identified as male~\cite{stackoverflow2022survey}. Additionally, our participants were drawn from a large research-focused technology company. While the results may offer insights into practices within certain companies, they serve as a case study for future research. Furthermore, we did not explicitly consider participants' specific roles, despite their expertise spanning various domains and levels. Future studies could explore the considerations of ethical values in AI systems across organizations and different roles and areas of expertise. Previous research has indicated different understandings of responsible AI values between practitioners and the general public~\cite{maurice2022how}, suggesting the potential for similar research methods to be applied in this area.

Last but not least, our qualitative data suggests indicators of ease of use for AI practitioners but does not provide direct information on the actual effectiveness of the guidelines. Understanding the impact of guidelines (or other AI toolkits~\cite{wong2023seeing}) requires long-term studies that consider multiple projects, with some utilizing the toolkit and others not. One potential avenue, as suggested by clinical researchers developing deep learning tools for patient care~\cite{beede2020human}, is to conduct observational studies with users of the AI system to assess its performance. Another approach is to use proxies, such as measuring users' attitudes, beliefs, and mindset regarding ethical values before and after utilizing the guidelines. We intend to explore these directions in future research.

%%
\begin{acks}
    This work has received funding from the European Union’s Horizon 2020 research and innovation program under Marie Sklodowska-Curie Actions (grant agreement number 860630) for the project "NoBIAS - Artificial Intelligence without Bias". This work reflects only the authors' views and the European Research Executive Agency (REA) is not responsible for any use that may be made of the information it contains.
\end{acks}

%%
\bibliographystyle{ACM-Reference-Format}
\bibliography{references}

%%
\newpage
\appendix
\appendices
\section{The Proof of Proposition \ref{prop2}}
\label{appa}
For the jointly Gaussian random vectors $\bm{x}$ and $\bm{y}$, we have
\begin{equation}
\begin{aligned}
&    \left[\begin{matrix}\bm{x}\\\bm{y}\\\end{matrix}\right] \sim \mathcal{N}\left(\left[\begin{matrix}\bm{\mu}_x\\\bm{\mu}_y\\\end{matrix}\right],\left[\begin{matrix}A&C\\C^T&B\\\end{matrix}\right]\right) \\
& = \mathcal{N}\left(\left[\begin{matrix}\bm{\mu}_x\\\bm{\mu}_y\\\end{matrix}\right],\left[\begin{matrix}\widetilde{A}&\widetilde{C}\\{\widetilde{C}}^T&B\\\end{matrix}\right]^{-1}\right)
\end{aligned}
\end{equation}
then the marginal and conditional distribution of $\bm{x}$ are shown as follows according to \cite{williams2006gaussian}.
\begin{equation}
    \bm{x} \sim \mathcal{N}\left(\bm{\mu}_x,A\right)
\end{equation}
% and
\begin{equation}
\label{app2-1}
    \bm{x}|\bm{y} \sim \mathcal{N}\left(\bm{\mu}_x+CB^{-1}\left(\bm{y}-\bm{\mu}_y\right),A-CB^{-1}C^T\right)
\end{equation}
% or
\begin{equation}
\label{app2-2}
    \bm{x}|\bm{y} \sim \mathcal{N}\left(\bm{\mu}_x-{\widetilde{A}}^{-1}\widetilde{C}\left(\bm{y}-\bm{\mu}_y\right),{\widetilde{A}}^{-1}\right)
\end{equation}

Thus, \textbf{Proposition \ref{prop2}} is proved.










\section{The Proof of Proposition \ref{prop3}}
\label{appb}
The product of two Gaussian distributions is represented as
\begin{equation}
\mathcal{N}\left(\bm{x}\middle|\bm{a},A\right)\mathcal{N}\left(\bm{x}\middle|\bm{b},B\right)=Z^{-1}\mathcal{N}\left(\bm{x}\middle|\bm{c},C\right)
\end{equation}
where
\begin{equation}
\label{app4}
    \bm{c}=C\left(A^{-1}\bm{a}+B^{-1}\bm{b}\right)
\end{equation}
\begin{equation}
\label{app5}
    C=\left(A^{-1}+B^{-1}\right)^{-1}
\end{equation}
\begin{equation}
\label{app6}
    Z^{-1}=\left(2\pi\right)^{-\frac{D}{2}}\left|A+B\right|^{-\frac{1}{2}}\exp{\left(-\frac{\left(\bm{a}-\bm{b}\right)^T\left(\bm{a}-\bm{b}\right)}{2\left(A+B\right)}\right)}
\end{equation}

Thus, through multiplying the cavity distribution by $t_i$ from (\ref{11}), \textbf{Proposition \ref{prop3}} is proved.


\section{The Proof of Proposition \ref{prop4}}
\label{appc}
Consider
\begin{equation}
\label{app7}
Z=\int_{-\infty}^{\infty}{\Phi\left(\frac{x-m}{v}\right)\mathcal{N}(x|\mu,\sigma^2)dx}
\end{equation}
% where
% \begin{equation}
%     \Phi\left(x\right)=\int_{-\infty}^{x}{\mathcal{N}\left(y\right)dy}
% \end{equation}
When $v>0$, by combining$ z=y-x+\mu-m$ and $w=x-\mu$ we can get
\begin{equation}
\begin{aligned}
& Z_{v>0}=\frac{\int_{-\infty}^{\infty}\int_{-\infty}^{x}\exp{\left(-\frac{\left(y-m\right)^2}{2v^2}-\frac{\left(x-\mu\right)^2}{2\sigma^2}\right)}}{2\pi\sigma v}dydx \\
& =\frac{\int_{-\infty}^{\mu-m}\int_{-\infty}^{\infty}\exp{\left(-\frac{\left(z+w\right)^2}{2v^2}-\frac{w^2}{2\sigma^2}\right)}}{2\pi\sigma v}dwdz
\end{aligned}
\end{equation}
% and
\begin{equation}
\begin{aligned}
& Z_{v>0} \\
& =\frac{\int_{-\infty}^{\mu-m}\int_{-\infty}^{\infty}\exp{\left(-\frac{1}{2}\left[\begin{matrix}w\\z\\\end{matrix}\right]^T\left[\begin{matrix}\frac{1}{v^2}+\frac{1}{\sigma^2}&\frac{1}{v^2}\\\frac{1}{v^2}&\frac{1}{v^2}\\\end{matrix}\right]\left[\begin{matrix}w\\z\\\end{matrix}\right]\right)}}{2\pi\sigma v}dwdz \\
& =\int_{-\infty}^{\mu-m}\int_{-\infty}^{\infty}\mathcal{N}\left(\left[\begin{matrix}w\\z\\\end{matrix}\right]|\mathbf{0},\left[\begin{matrix}\sigma^2&-\sigma^2\\-\sigma^2&v^2+\sigma^2\\\end{matrix}\right]\right)dwdz
\end{aligned}
\end{equation}
According to (\ref{app2-1}) and (\ref{app2-2}), we can get
\begin{equation}
\label{app11}
    Z_{v>0}=\frac{\int_{-\infty}^{\mu-m}\exp{\left(-\frac{z^2}{2\left(v^2+\sigma^2\right)}\right)}dz}{\sqrt{2\pi(v^2+\sigma^2)}}=\Phi\left(\frac{\mu-m}{\sqrt{v^2+\sigma^2}}\right)
\end{equation}
When $v<0$, by combining $\Phi\left(-z\right)=1-\Phi\left(z\right)$ and (\ref{app7}),
% we can obtain
\begin{equation}
\label{app12}
Z_{v<0}=1-\Phi\left(\frac{\mu-m}{\sqrt{v^2+\sigma^2}}\right)=\Phi\left(-\frac{\mu-m}{\sqrt{v^2+\sigma^2}}\right)
\end{equation}

By collecting (\ref{app11}) and (\ref{app12}), we can get
\begin{equation}
\label{app13}
Z=\int\Phi\left(\frac{x-m}{v}\right)\mathcal{N}\left(x\middle|\mu,\sigma^2\right)dx=\Phi\left(z\right)
\end{equation}
where $z=\frac{\mu-m}{v\sqrt{1+\sigma^2/v^2}} (v\neq0)$. 
% We aim to get the moments of
% \begin{equation}
% q\left(x\right)=Z^{-1}\Phi\left(\frac{x-m}{v}\right)\mathcal{N}\left(x\middle|\mu,\sigma^2\right)
% \end{equation}
By differentiating with respect to $\mu$ on (\ref{app13}), we can obtain
\begin{equation}
\begin{aligned}
& \frac{\partial Z}{\partial\mu}=\int{\frac{x-\mu}{\sigma^2}\Phi\left(\frac{x-m}{v}\right)}\mathcal{N}\left(x\middle|\mu,\sigma^2\right)dx =\frac{\partial}{\partial\mu}\Phi\left(z\right) \\
& \Longleftrightarrow \frac{1}{\sigma^2}\int x\Phi\left(\frac{x-m}{v}\right)\mathcal{N}\left(x\middle|\mu,\sigma^2\right)dx-\frac{\mu Z}{\sigma^2} \\
& =\frac{\mathcal{N}(z)}{v\sqrt{1+\sigma^2/v^2}}
\end{aligned}
\end{equation}
where $\partial\Phi\left(z\right)/\partial\mu=\mathcal{N}(z)\partial z/\partial\mu$ is utilized. Multiplying through by $\sigma^2/Z$, (\ref{app16}) is obtained.
\begin{equation}
\label{app16}
\mathbb{E}_q\left[x\right]=\mu+\frac{\sigma^2\mathcal{N}\left(z\right)}{\Phi\left(z\right)v\sqrt{1+\frac{\sigma^2}{v^2}}}
\end{equation}
Similarly, we can obtain the second moment as
\begin{equation}
\label{app17}
\begin{aligned}
 & \frac{\partial^2Z}{\partial\mu^2} \\
 & =\int{[\frac{x^2}{\sigma^4}-\frac{2\mu x}{\sigma^4}+\frac{\mu^2}{\sigma^4}-\frac{1}{\sigma^2}] \Phi\left(\frac{x-m}{v}\right)\mathcal{N}\left(x\middle|\mu,\sigma^2\right)} dx  \\
 & =-\frac{z\mathcal{N}(z)}{v^2+\sigma^2} \Longleftrightarrow \\
 & \mathbb{E}_q\left[x^2\right]=2\mu\mathbb{E}_q\left[x\right]-\mu^2+\sigma^2-\frac{\sigma^4z\mathcal{N}\left(z\right)}{\Phi\left(z\right)\left(v^2+\sigma^2\right)}
\end{aligned}
\end{equation}
By combining (\ref{app16}) and (\ref{app17}), we can get
\begin{equation}
\begin{aligned}
& \mathbb{E}_q\left[{(x-\mathbb{E}_q\left[x\right])}^2\right]=\mathbb{E}_q\left[x^2\right]-\mathbb{E}_q[x]^2 \\
& =\sigma^2-\frac{\sigma^4\mathcal{N}\left(z\right)}{\left(v^2+\sigma^2\right)\Phi\left(z\right)}\left(z+\frac{\mathcal{N}\left(z\right)}{\Phi\left(z\right)}\right)
\end{aligned}
\end{equation}

Thus, \textbf{Proposition \ref{prop4}} is proved.

\section{The Proof of Proposition \ref{prop5}}
\label{appd}
We can obtain (\ref{19-1}), (\ref{19-2}), and (\ref{19-3}) according to (\ref{app4}), (\ref{app5}), and (\ref{app6}). Hence, \textbf{Proposition \ref{prop5}} is proved.



\section{The Proof of Proposition \ref{prop6}}
\label{appe}
The approximated mean for $f_\ast$ can be denoted as
\begin{equation}
\begin{aligned}
& \mathbb{E}_q\left[f_\ast|X,\bm{y},\bm{x}_\ast\right]=\bm{k}_\ast^TK^{-1}\bm{\mu} \\
& =\bm{k}_\ast^TK^{-1}\left(K^{-1}+{\widetilde{\Sigma}}^{-1}\right)^{-1}{\widetilde{\Sigma}}^{-1}\widetilde{\bm{\mu}} \\
& =\bm{k}_\ast^T\left(K+\widetilde{\Sigma}\right)^{-1}\widetilde{\bm{\mu}}
\end{aligned}
\end{equation}

The variance of $f_\ast|(X,\bm{y})$ under the Gaussian approximation can be denoted as
\begin{equation}
\begin{aligned}
& \mathbb{V}_q\left[f_\ast\middle| X,\bm{y},\bm{x}_\ast\right] = \mathbb{E}_{p(f_\ast|X,\bm{x}_\ast,\bm{f})} {f_\ast-\mathbb{E}[f_\ast|X,\bm{x}_\ast,\bm{f}]}^2 \\
& =k\left(\bm{x}_\ast,\bm{x}_\ast\right)-\bm{k}_\ast^TK^{-1}\bm{k}_\ast+\bm{k}_\ast^TK^{-1}\left(K^{-1}+\widetilde{\Sigma}\right)^{-1}K^{-1}\bm{k}_\ast \\
& =k\left(\bm{x}_\ast,\bm{x}_\ast\right)-\bm{k}_\ast^T\left(K^{-1}+\widetilde{\Sigma}\right)^{-1}\bm{k}_\ast
\end{aligned}
\end{equation}

Then, we can obtain
\begin{equation}
\begin{aligned}
& q\left(y_\ast\middle| X,\bm{y},\bm{x}_\ast\right)=\mathbb{E}_q\left[\pi_\ast|X,\bm{y},\bm{x}_\ast\right] \\
& =\int\Phi\left(f_\ast\right)q\left(f_\ast\middle| X,\bm{y},\bm{x}_\ast\right)df_\ast
\end{aligned}
\end{equation}

According to (\ref{app11}), we can obtain
\begin{equation}
\label{app22}
\begin{aligned}
& q\left(y_\ast\middle| X,\bm{y},\bm{x}_\ast\right) \\
& =\Phi\left(\frac{\bm{k}_\ast^T\left(K+\widetilde{\Sigma}\right)^{-1}\widetilde{\bm{\mu}}}{\sqrt{1+k\left(\bm{x}_\ast,\bm{x}_\ast\right)-\bm{k}_\ast^T\left(K+\widetilde{\Sigma}\right)^{-1}\bm{k}_\ast}}\right)
\end{aligned}
\end{equation}

By combining (\ref{13}) and (\ref{app22}), \textbf{Proposition \ref{prop6}} is proved.




\section{The Proof of Proposition \ref{prop7}}
\label{appf}
Given $f_s$ and $f_\ast$, $y_s$ and $y_\ast$ are conditionally independent. Hence, $p\left(y_s,y_\ast\middle|\bm{x}_s,\bm{x}_\ast\right)$ can be represented as
\begin{equation}
\begin{aligned}
& p\left(y_s=1,y_\ast=1\middle|\bm{x}_s,\bm{x}_\ast\right) \\
& =\iint{\Phi\left(f_s\right)\Phi\left(f_\ast\right)\phi\left(f_s,f_\ast\middle|\mu_{s\ast},\Sigma_{s\ast}\right)}df_sdf_\ast \\
& =\iint{\Phi\left(f_\ast\right)\phi\left(f_\ast\middle|{\widetilde{\mu}}_\ast\left(f_s\right),{\widetilde{\sigma}}_{\ast\ast}\right)df_\ast\Phi\left(f_s\right)}\phi\left(f_s\middle|\mu_s,\sigma_{ss}\right)df_s \\
& =\int\Phi\left(\frac{{\widetilde{\mu}}_\ast\left(f_s\right)}{\sqrt{{\widetilde{\sigma}}_{\ast\ast}+1}}\right)\Phi\left(f_s\right)\phi\left(f_s\middle|\mu_s,\sigma_{ss}\right)df_s
\end{aligned}
\end{equation}

Hence, \textbf{Proposition \ref{prop7}} is proved.

% \section{The Proof of Lemma \ref{lem}}
% \label{appg}
% \begin{equation}
% \begin{aligned}
% & R_e=\frac{1}{N_a}\sum_{n=1}^{N_a}\mathbb{I}\left(\bm{L}_n \neq \bm{Y}_n\right) \\
% & =\displaystyle\frac{FA+FL}{TL+TA+FL+FA} \\
% & =\displaystyle\frac{1}{\displaystyle\frac{TL+TA+FL+FA}{FA+FL}} \\
% & =\displaystyle\frac{1}{1+\displaystyle\frac{TL+TA}{FA+FL}} \\
% & =\displaystyle\frac{1}{1+\displaystyle\frac{\displaystyle\frac{TL}{TA}+1}{\displaystyle\frac{FA}{TA}+\displaystyle\frac{FL}{TA}}} \\
% & =\frac{1}{1+\displaystyle\frac{\displaystyle\frac{TL}{TA}+1}{\displaystyle\frac{1}{P_{md}-1}+\displaystyle\frac{1}{P_{fa}-1}}}
% \end{aligned}
% \end{equation}

% Hence, \textbf{Lemma \ref{lem}} is proved.

\end{document}
%%
