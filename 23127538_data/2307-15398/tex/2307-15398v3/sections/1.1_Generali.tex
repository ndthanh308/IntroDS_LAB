%
\label{sec:Generali}

Our work borrows from a previous collaborative effort at a European Fortune Global 500.
We refer to this company as G.
The purpose was to study G's hiring process as an algorithmic fairness problem.
We worked closely with Advanced Analytics (AA) and Human Resources (HR) teams, focusing on candidate screening.
In this phase of G's hiring process, an HR officer reduces the candidate pool for a job opening into a smaller pool of suitable candidates, determining candidate suitability based on each candidate's profile using a set of minimum basic requirements.
The candidate pool is stored in Taleo by Oracle,
% \footnote{Visit \url{https://www.oracle.com/human-capital-management/taleo/}.} 
a platform specific to hiring.
For more details on our collaboration, see Appendix~\ref{Appendix.MoreOnG}.

Candidate screening at G represented a time-consuming, repetitive task prone to human error \cite{Kahneman2011Thinking} and a sensitive, high-risk task requiring human oversight \cite{bringas2022fairness}.
Therefore, the option of full automation was not possible. 
The focus was, thus, on understanding and modeling the search of the HR officer since the viable options involved either the presence or absence of an algorithmic aid.
In particular, under the realistic risk of position bias \cite{DBLP:journals/cacm/Baeza-Yates18} affecting candidate evaluations, as shown empirically in similar screening settings \cite{DBLP:journals/jcmc/PanHJLGG07, DBLP:conf/clef/GrotovCMSXR15, DBLP:conf/www/RichardsonDR07, DBLP:conf/chi/EchterhoffYM22, PeiEAAMO2023}, we wanted to study the influence of the ISO on the set of suitable candidates chosen by the HR officer.

The following \textit{stylized facts} summarize key practices by the HR officers (henceforth, screeners) that motivate the ISO problem in Sections \ref{sec:ProblemFormulation} and \ref{sec:HumanScreener}.
These are:
%
% \begin{itemize}
    % \item[G1] 
    \textit{\textbf{G1}} \textit{Varying ISOs.} 
    Screeners chose the ISO. 
    The choice was restricted by the sorting fields of the platform, such as using the candidates' last name. 
    %
    % \item[G2] 
    \textit{\textbf{G2}} \textit{Two ways to search the candidate pool.} 
    % Screeners chose how to search the ISO.
    Two search practices became apparent: full or partial search based on the desired number of suitable candidates. 
    %
    % \item[G3] 
    \textit{\textbf{G3}} \textit{Meeting the set of minimum basic requirements.} 
    Although screeners were able to differentiate candidates relative to each other, their focus was on finding candidates that met these requirements. 
    % This meant that order within the selected candidates was not necessarily important.
    %
    % \item[G4] 
    \textit{\textbf{G4}} \textit{Representative suitable candidates.}
    % G already had in place 
    Fairness goals already existed in the form of representation quotas, often around gender, that were enforced by the screeners. 
    % This meant screeners prioritized suitable candidates from the protected group when selecting candidates. 
    %
    % \item[G5] 
    \textit{\textbf{G5}} \textit{A consistent notion of time.}
    % Given the amount of information to be processed for each candidate, 
    Screeners aimed at spending one minute per candidate.
% \end{itemize}
%

Although G1-5 are specific to G, they highlight salient aspects of a real-world candidate screening problem that are, we argue, likely to hold in similar settings involving humans searching a pool of candidates.
We note that, while G4-5 are standard to the fair set selection problem, especially under an algorithmic screener (Section~\ref{sec:Add_RelatedWork}), G1-3 introduce new considerations to such problem formulation, especially under a human screener (Section~\ref{sec:RelatedWork}).

%
% EOS
%

% We summarize our experience in the form of \textit{stylized facts}, which underpin the initial screening order (ISO) problem studied in this paper.
% Although this section is specific to G, it highlights salient aspects of a real-world candidate screening problem that are likely to hold for other companies of similar size and reputation as G \ja{as well as, overall, screening settings involving human screeners}.  
% % \ja{\citet{PeiEAAMO2023}, e.g., finds similar screening practices by professors that use a platform to grade student homeworks.}

% Hiring at G consists of three phases. 
% In \textit{phase one}, the HR builds a candidate pool for the job opening.
% Candidates submit their CVs, complete a multiple-choice questioner, and write a motivation letter. Sensitive information, such as gender, ethnicity, and age, is also provided or it can be inferred. The candidate pool is stored in a database platform.
% In \textit{phase two}, the HR officer reduces the candidate pool into a smaller pool of suitable candidates. The HR officer determines candidate suitability based on each candidate's profile using a set of minimum basic requirements. 
% In \textit{phase three}, the chosen candidates are interviewed by HR and the team offering the job.
% The best candidates receive an offer. If no candidates are hired, HR resorts to the runner-up candidates from phase two and repeat phase three.

% The choice to focus on candidate screening, or phase two, was motivated by how it represented a time-consuming, repetitive task prone to human error \cite{Kahneman2011Thinking, Kahneman2021Noise} and a sensitive, high-risk task requiring human oversight \cite{bringas2022fairness, EU_AIAct}. 
% The idea of automating candidate screening initially appealed to all stakeholders; however, it became less appealing over time given how complex and human-dependent candidate screening is as a process.
% Hence, our focus shifted from the future algorithmic screener onto the present human screener.
% Below we present stylised facts from G's candidate screening process. 
% We draw from these facts in Sections~\ref{sec:ProblemFormulation} and \ref{sec:HumanScreener}.
% We formally refer to the HR officer as the screener.
% %
% \begin{itemize}
%     \item[G1] \textit{A varying initial screening order for the candidate pool.} 
%     Each screener, through the platform, chose how to order the candidate pool before screening the candidates.
%     The choice was restricted by the sorting fields offered by the platform, such as by candidates' last name or by the date of arrival of the applications. 
%     %
%     \item[G2] \textit{Two ways to search the candidate pool.} 
%     Once the initial screening order was set, each screener chose how to search it.
%     Two search practices became apparent: screeners either fully or partially searched the candidate pool. In the latter case, the screener would stop once it reached the desired number of suitable candidates. 
%     %
%     \item[G3] \textit{Meeting the minimum basic requirements.} 
%     Although screeners were able to differentiate candidates relative to each other, their focus was on finding candidates that met the minimum basic requirements for the job. 
%     % This meant that order within the selected candidates was not necessarily important.
%     %
%     \item[G4] \textit{A representative set of selected candidates.}
%     G already had in place fairness goals in the form of representation quotas, often around gender, that were enforced by the screeners when selecting candidates. 
%     % This meant screeners prioritized suitable candidates from the protected group when selecting candidates. 
%     %
%     \item[G5] \textit{A consistent notion of time.}
%     Given the amount of information to be processed for each candidate, screeners aimed at spending one minute per candidate.
% \end{itemize}
% %
