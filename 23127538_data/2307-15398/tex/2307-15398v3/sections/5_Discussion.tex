%
\label{sec:Discussion}

% Summary
In this work,
we presented the initial screening order (ISO) as a parameter of interest; 
defined two formulations under distinct utility models of the fair set selection problem, the best-$k$ and the good-$k$, with their corresponding algorithmic implementations; 
and introduced a human-like screener to study the effects of the ISO as though we were studying a human user in a candidate screening problem.
We also provided a simulation framework flexible enough to account and model for other screening scenarios.
Our analysis confirms the fairness and optimality impact of the ISO, motivated by the risk of position bias, on the set of $k$ candidates chosen by an algorithmic or human-like screener. 
Extensive simulations showed complex relations between best-$k$ and good-$k$, which heavily depend on the score distribution, on their correlation with the ISO, and on the fatigue modeling.

% Before discussing the implications of our work, 
We once again emphasize that the assumptions made--both in terms of functional forms and parameter choice--condition our analysis, in particular, our experimental results.
Some of these choices are probably less contested than others.
For instance, 
the utility models chosen in Section~\ref{sec:ProblemFormulation} for defining the best-$k$ and good-$k$ problems are intuitive and reflective of the screener's evaluation criteria in each setting and, overall, recall common modeling practices within the literature (e.g., \cite{Zehlike2023_FairRanking_P1, Zehlike2023_FairRanking_P2, stoyanovich2018online}).
Further, both problem formulations are not restricted to these specific utility models and future work should explore other models that ensure the desired screener behavior. 
% \am{NEXT SENTENCE IS UNCLEAR} 
Under such alternative utility models, though, Algorithms~\ref{algo:Examination} and \ref{algo:Cascade} and, in turn, Remark~\ref{remark:ISOandPS} should still hold, thus, illustrating the importance of distinguishing between these two problem formulations. 
% among which the good-$k$ represents a novel addition to the literature.

Other assumptions, like the choices, both function- (Section~\ref{sec:HumanScreener.BiasedScores}) and parameter-wise (Section~\ref{sec:Experiments.Setup}) regarding the human-like screener's fatigue are likely to be contested and limit the implications of our analysis.
We recognize this limitation, though we argue that we made such choices mainly to study the human-like screener (with the fatigue term) under an ISO chosen or provided.
We made the simplest modeling assumptions for the fatigue, but always keeping in mind what we observed at G and know from previous empirical works \cite{DBLP:conf/chi/EchterhoffYM22, PeiEAAMO2023}.
Indeed, under a different fatigue model, Remark~\ref{remark:HumanAndIF} may not hold and $h_h$ may not violate individual fairness; similarly, under different fatigue parameters, the experimental results in Section~\ref{sec:Experiments.Metrics.withFatigue} may motivate a different analysis.
% \am{NEXT SENTENCE IS UNCLEAR} 
% In both cases, though, 
Note, though, that the ISO would still play a central role in any variation of the current assumptions.
% though, we achieve our goal of motivating the ISO problem as the ISO still plays a central role in any variation of the current assumptions that would motivate additional analysis.

We stress that our modeling and simulation frameworks are flexible enough to account for other screening settings.
The experimental setup studied here is one of many possible screening scenarios that we could consider. 
Similarly, the framework could consider the role of the ISO under a different fatigue term.
What matters is studying the role of the ISO and, in particular, its effect on the human screener.
It is costly and time-consuming to run real experiments with human screeners and candidates, especially, at the same scale (10000 runs per setting) of Section~\ref{sec:Experiments}.
We view our work as another example \cite{DBLP:conf/fat/IonescuHJ21, DBLP:journals/corr/abs-2006-09663, DBLP:conf/fat/BountouridisHMM19, Bokanyi2020_Understanding, Schelling1971_Dynamic} of how simulations are useful tools for studying the fairness and optimality of real-world decision-making processes.

For instance, based on the experiments presented in Section~\ref{sec:Experiments}, a company like G could revisit the current screening practices of its HR officers.
Informed by the simulations, instead of running the sequential, partial search (Algorithm~\ref{algo:Cascade}) for the good-$k$, the company could, e.g., recommend its HR officers to take breaks after screening a certain number of candidates in the ISO to ensure that the fatigue does not accumulate. 
Similarly, such a new search procedure for the good-$k$ problem could be implemented and tested within our simulation framework as it would still be addressing the ISO problem. 
Overall, such an approach would be a better alternative for the company to experiment on real screeners and candidates.
Note that the company could carry out these analyses considering an ISO that is either chosen by or provided to the screener; in turn, it would allow for studying settings involving human screeners alone and human screeners relying on some form of algorithmic aid~\cite{DBLP:journals/air/MosqueiraReyHABF23}. %like a fair ranker or the sorting fields of a digital platform.

Thus, future work could explore alternative utility models and fatigue terms while relying on the current simulations framework.
For instance, an interesting alternative formulation for fatigue could involve a human-like screener that rests while searching the candidate pool under the ISO. 
We see recurrent survival models \cite{DBLP:conf/www/ChandarTMPSWCLJ22} well suited for this task.
Further, defining a human-like screener is not limited to fatigue. 
Future work could also explore theories on human decision-making (see \cite[Ch. 10]{DBLP:books/daglib/0033056} and \cite{DBLP:conf/chi/CarabanKGC19, DBLP:journals/isr/AdomaviciusBCZ13}).
Furthermore, the simulations framework could be extended to test for optimal parameters in the ISO problem. 
For example, finding an optimal minimum score $\psi$ for which the two problem formulations coincide and making the search procedure less influential.

%%% during flight
To conclude, we summarize three main takeaways from our analysis of the ISO problem. 
First, \textit{defining the adequate problem formulation
is important for understanding the impact of the ISO on the selected set of candidates} (Remark~\ref{remark:ISOandPS}), which reflects the search procedures of the screeners regardless of the kind.
Second, once the search procedure is clear, \textit{it is important to understand how the screener behaves as it searches over the ISO}.
This is most important for the human-like screener (with fatigue) and the fair individual treatment of each candidate in the pool (Remark~\ref{remark:HumanAndIF}).
Third, \textit{simulations are beneficial for understanding the implications of the interaction between the screener and the ISO on the selected candidates}.
For example, the composition of the candidate pool can amplify or diminish the influence of the position bias in the ISO (as in Sections~\ref{sec:Experiments.Metrics.outFatigue} and \ref{sec:Experiments.Metrics.withFatigue}).
All three takeaways have direct implications for practitioners.
These insights might seem obvious ex-post, though they are supported by an extensive analysis that accounts for several factors that only become clear under modeling and experiments.


% Let us take company G as an example here.
% If G's HR officers expect a pool of candidates with many good candidates, then based on our analysis
% they could opt for a partial search and still expect a set of selected candidates as though they had performed a full search.
% In other words, the simulations would inform the HR officers how to act under such a candidate pool.
% These are reminders that many problems, like the ISO problem, only become obvious after careful theoretical and experimental frameworks.

% We also emphasize the role of the individual scoring function used for evaluating each candidate. 
% We framed the algorithmic screener in terms of its consistency to evaluate similar candidates similarly.
% This assumption allowed us to focus on the role of the ISO.
% We are, though, aware that current work mainly focuses on defining fair scoring functions to obtain a fair ranking of candidates. 
% Future work should explore fair ranking algorithms under the ISO problem. The goal should be to create an algorithm that provides a fair ISO specific to the human user.
% This is because, as long as the fair ranking or fair platform is used by a human for candidate screening, the risk of the ISO problem is still present.

%
% EOS
%

% Limitations
% Our work is based on a collaboration with company G, and not on a field or observational study, such as \cite{Pisanelli2022_YourCV}, which is why we chose the term ``stylised facts'' to summarize our experience in Section~\ref{sec:Generali}. Our observations of G's candidate screening practices are limited to our interaction with G's AA and HT teams.
% In this work, we have made functional assumptions on what defines a human-like screener. 
% The notions of utility and fatigue are open to discussion. 
% Future work should consider alternative, more complex formulations to better capture the human-like screener. 
% In particular, we reduced the ``humanity'' of the screener in terms of fatigue. 
% This assumption allowed us to present a simple and intuitive setting to showcase the impact of the ISO.

% Moreover, the simulation procedures can be extended to account for additional parameters and for score distributions estimated from real data, and they can be made accessible through a user-friendly interface.