%
\label{sec:ProblemFormulation}
 
The goal is to formulate the fair set selection problem, where a decision-maker selects a set of items from a population, by considering the initial screening order (ISO) in which the items are presented to it. 
Here, the candidates for a job represent the items and the screener evaluating their profiles represents the decision-maker.  

\subsection{Setting}
\label{sec:ProblemFormulation.Setting}

Let us consider a \textit{candidate pool} $\mathcal{C}$ of $n$ candidates. 
Each \textit{candidate} $c$ is described by the \textit{vector of $p$ attributes} $\mathbf{X}_c \in \mathbb{R}^{p}$ and the \textit{protected attribute} $W_c$. 
To simplify our analysis, we assume that $W$ encodes the membership to the protected group and is binary with $W_c = 1$ if $c$ belongs to the protected group and $W_c = 0$ otherwise.
We can relax this assumption if needed.
% The underlying protected attribute can be discrete, continuous, or it may consist of multiple attributes.
The candidates are evaluated by a \textit{screener} $h \in \mathcal{H}$, where $\mathcal{H}$ denotes the set of available screeners. 
The following variables refer to a specific $h$.
% The screener
% The screener is tasked with \ja{selecting} $k$ candidates from $\mathcal{C}$ based on each candidate's application profile as summarized by the tuple $(\mathbf{X}_c, W_c)$. 
The goal of $h$ is to obtain a \textit{set of $k$ selected candidates} $S^k \in [\mathcal{C}]^k$, with $[\mathcal{C}]^k$ denoting the set of $k$-subsets of $\mathcal{C}$ (i.e., subsets with cardinality $k$),
based on each candidate's application profile as summarized by the tuple $(\mathbf{X}_c, W_c)$.
We model candidate evaluation by assuming that $h$ uses an \textit{individual scoring function} $s \colon \mathbb{R}^{p} \to [0, 1]$, such that
% , given a candidate $c$, 
$s(\mathbf{X}_c)$ returns the score of $c$ according to $h$. 
% It is implied that 
% The screener does not use the protected attribute when scoring candidates. 
Here, $h$ does not (or, equivalently, cannot) use $W_c$ when scoring any $c$. 
The higher the score, the better the candidate fits the job. 
% based on the screener's judgment. 

% All possible orders and the initial order
The screener $h$ explores the candidate pool $\mathcal{C}$ in a specific order.
% when evaluating the candidates. 
We denote the \textit{set of total orderings of candidates} in $\candidatesset$ by $\Theta$. 
It represents all the possible ways in which the $n$ candidates in $\candidatesset$ can be arranged. 
An \textit{order} $\sigma \in \Theta$ maps an integer $i \in \{1, \dots n\}$ to a candidate $c \in \candidatesset$, indicating that $c$ occupies the $i$-th position according to $\sigma$, with notation $\sigma(i) = c$ and vice-versa $\sigma^{-1}(c) = i$.

Importantly, the screener explores the candidate pool $\mathcal{C}$ under the \textit{initial screening order} $\theta \in \Theta$, which represents the order chosen by or, alternatively, provided to $h$ before starting the exploration of $\candidatesset$ 
(recall, G1 in Section~\ref{sec:Generali}). 
The screener is not required to explore the entirety of $\mathcal{C}$, meaning $h$ can either fully or partially explore $\candidatesset$ given $\theta$ (recall, G2 in Section~\ref{sec:Generali}).
% However, w
We assume that the screener \textit{respects} $\theta$, meaning: 
% Formally:
%
\begin{equation}
\label{eq:order}
    \mbox{a candidate $c_1 \in \candidatesset$ is evaluated before $c_2 \in \candidatesset$ only if $\theta^{-1}(c_1) < \theta^{-1}(c_2)$.}
\end{equation}
%

% All initial orders are meaningless to the set of screeners $\mathcal{H}$.
% On m steps travelled 
%The screener is not required to explore the entirety of the candidate pool $\mathcal{C}$.\footnote{This condition is based on fact G2 in Section~\ref{sec:Generali}.} 
%We denote as $\candidatessubset \subseteq \candidatesset$, the \textit{subset of candidates whose profiles are screened by} $h$, with cardinality $m \leq n$, implying the case where $h$ prefers not to evaluate the complete $\candidatesset$.
%For $\theta \in \Theta$, we refer to $\theta^{(m)}$ as \textit{the total order restricted to first $m$ elements according to the initial order}, so that $\theta^{(m)}$ maps an integer $i \in \{1, \dots m\}$ to a candidate $c \in \candidatessubset$.
% For a given $\sigma \in \Theta$, we refer to $\sigma^{(m)}$ as the total order restricted to the first $m$ elements, so that $\sigma^{(m)}$ maps an integer $i \in \{1, \dots m\}$ to a candidate $c \in \candidatesset_h$.

\subsection{Group Fairness}
\label{sec:ProblemFormulation.Fairness}

To address the fairness goal of the screener $h$ (recall, G4 in Section~\ref{sec:Generali}),
% in choosing the set of $k$ candidates 
we assume a quota $q$ for the protected group $W = 1$. 
We introduce the fraction $f\big( S^k \big) \in [0, 1]$ of protected candidates in the selected set $S^k$ as:
%
\begin{equation}
\label{eq:fairness_function}
    f\big( S^k \big) = \frac{\left\vert\{c \in S^k \text{ s.t. } W_c = 1\}\right\vert}{k}
\end{equation}
%
% \begin{equation}
% \label{eq:fairness_function}
%     f\big( S^k \big) = \left\vert\{c \in S^k \text{ s.t. } W_c = 1\}\right\vert / k
% \end{equation}
%
% which acts as a ``counter'' of protected candidates in $S^k$.
and we denote by $q \in [0, 1]$ the desired fraction of selected protected candidates in $S^k$. 
The fair $h$ will be constrained to meet the \textit{representational quota} $q$ when deriving $S^k$ by satisfying the condition $f\big( S^k \big) \geq q$. 
% Notice that t
The unconstrained version, in which no representativeness requirement is assumed,
% on the representativeness of the protected group in $S^k$, 
is achieved by considering $q=0$.
We view $q$ as a policy enforced by $h$ to achieve a diverse $S^k$. 
It is a statement on the composition of $S^k$, not a statement on the ordering of protected candidates within $S^k$.
For instance, for $k=10$ and $q=0.5$, the fair screener would need to derive $S^k$ with $50\%$ of protected candidates though in no particular order within $S^k$.
% , such as requiring the first five candidates in $S^k$ to be protected candidates.

\subsection{Two Problem Formulations}
\label{sec:ProblemFormulation.Objectives}

How might we account for the two ways $h$ can search $\mathcal{C}$ as stipulated in our setting?
We proceed to formulate two fair set selection problems for $h$, 
% tasked with deriving the set $S^k \subseteq \candidatesset$, 
in which we distinguish between best-$k$ and good-$k$. 
Under the \textit{best-}$k$ formulation, the set $S^k$ represents \textit{the fair best $k$ candidates} in $\candidatesset$ according to $h$. 
We denote it accordingly as $S^k_{\besttext}$. 
% This is the standard formulation in the fair ranking literature~\cite{Zehlike2023_FairRanking_P1, Zehlike2023_FairRanking_P2}. 
Under the \textit{good-}$k$ formulation, the set $S^k$ represents \textit{the fair first good-enough $k$ candidates} in $\candidatesset$ according to $h$. 
We denote it accordingly as $S^k_{\goodtext}$. 
% This formulation has been overlooked by the fair ranking\ja{/set selection literature}.
For both formulations we define the objective of $h$ in terms of achieving an optimal and fair selection of candidates. 
How we define optimality, as we show, derives the best-$k$ and good-$k$ problems; we already defined fairness in the previous subsection. 
The key difference between these two problem formulations is that the best-$k$ requires a full search of $\candidatesset$ while the good-$k$ allows for a partial search of $\candidatesset$ under an ISO $\theta$.

\subsubsection{Best-$k$.}
\label{sec:best-k}

We first focus on the screener $h$ that finds the set of best $k$ candidates in $\candidatesset$ given the fairness constraint $q$
and while respecting the ISO $\theta$ \eqref{eq:order}.
% (recall (\ref{eq:order})).
%Noteworthy, here $h$ has to search the whole $\candidatesset$, meaning $\am{\candidatessubset} = \candidatesset$ and $m=n$ in this setting.
Here, $h$ needs to evaluate the complete $\candidatesset$ since $h$ must score and rank all candidates according to the individual scoring function $s$ before choosing the ones with the highest scores and that satisfy $q$.

We view the goal in terms of maximizing a utility for $h$. 
We define \textit{utility} as the benefit derived by $h$ from selecting $k$ suitable candidates under $\theta$. 
Formally, utility is a function $U^k \colon [\mathcal{C}]^k \, \times \, \Theta \to \mathbb{R}$. 
The simplest expression for defining $U^k$ is to add the scores of the selected candidates:
%
\begin{equation}
\label{eq:Utility}
    U^k_{\addtext} \big( S^k, \theta \big) = \sum_{c \in S^{k}} s\big( \mathbf{X}_{c} \big)
\end{equation}
%
rationalizing that $h$ will maximize its utility by selecting the $k$ most suitable candidates for the job. 
%It is easy to see that $S_{\tau}^k$ maximizes the utility $U^k_{\text{add}}$.
Notice that $\theta$ in \eqref{eq:Utility} does not affect the evaluation of $S^k$ because of the commutative property of addition.
% We emphasize that \eqref{eq:Utility} is not the only possible model for the utility of the screener. Alternative models, such as exposure discounting \cite{DBLP:conf/kdd/SinghJ18}, can be considered for the best-$k$ problem formulation. We leave this for future work.
Under \eqref{eq:Utility}, we define \textit{the fair best-$k$ set selection problem} or, simply, \textbf{the best-\textit{k} problem} as:
%
\begin{equation}
\label{eq:fair_objective_all_screener}
    \begin{aligned}
    \argmax_{S^k \in [\mathcal{C}]^k} & \quad U^k_{\addtext} \big(S^k, \theta\big) \\
    \textrm{s. t.} & \quad f(S^k) \geq q
    \end{aligned}
\end{equation}
%
for which we denote its solution as $S^k_{\besttext}$. In the presence of tied scores, the solution may not be unique. In such a case, we consider any solution.
%
We emphasize that \eqref{eq:Utility} is not the only possible model for the utility of $h$ and alternative models, such as exposure discounting \cite{DBLP:conf/kdd/SinghJ18}, can be considered for \eqref{eq:fair_objective_all_screener}.
We leave this for future work.

\subsubsection{Good-$k$}
\label{sec:good-k}

We now focus on the screener $h$ that finds $k$ candidates in $\candidatesset$ that meet a set of \textit{minimum basic requirements} $\psi$ (recall, G3 in Section~\ref{sec:Generali}) given the fairness constraint $q$ and while respecting the ISO $\theta$ \eqref{eq:order}.
% Formally, w
We represent $\psi$ as a \textit{minimum score}, such that $h$ deems candidate $c \in \candidatesset$ as eligible for being selected if $s(\mathbf{X}_c) \geq \psi$.
Here, unlike the best-$k$ formulation, $h$ is not required to evaluate the whole $\candidatesset$ as it is enough to find the first $k$ candidates that are good enough according to $\psi$ and that satisfy $q$. 
%Hence, the goal \eqref{eq:objective_best-k} and utility \eqref{eq:Utility} from the standard set selection formulation are not suitable in this setting.

We still view the goal in terms of maximizing a utility for $h$. We need, however, to define an alternative utility function to \eqref{eq:Utility} that specifically ensures that $h$ stops searching $\candidatesset$ after finding the $k$-\textit{th} good-enough candidate according to $\psi$.
%%%
% JA: new alt u flow
%%%
We define the following expression for $U^k_\psi$:
%
\begin{equation}
\label{eq:AlternativeUtility}
    U^k_{\psi}\big( S^k, \theta \big) = \left\{
    \begin{array}{ll}
        k - \sum_{c \in S^k} p(c, S^k, \theta) & \text{if,} \  \forall c \in S^k, \  s(\mathbf{X}_c) \geq \psi   \\
        0 & \text{otherwise.}
    \end{array} \right.
\end{equation}
%
with the \textit{penalty function} $p(c, S^k, \theta) = \mathbb{1}(\exists\ c' \in \mathcal{C}\setminus S^k \, \mbox{s.t.}\ \, \theta^{-1}(c') < \theta^{-1}(c) \wedge s(\mathbf{X}_{c'}) \geq \psi \wedge W_{c'}=W_c)$.
Under \eqref{eq:AlternativeUtility}, $h$ wants to find as quick as possible the $k\text{-\textit{th}}$ suitable candidate without wanting to check whether the $k\text{-\textit{th}} + 1$ candidate is also suitable.
This is because, for a candidate $c$, $p(c, S^k, \theta)$, looks for another candidate of the same group as $c$ and meeting $\psi$, who occurs before $c$ under $\theta$ but who has not been selected into $S^k$.
The penalty function models the ``wasted effort" in choosing a candidate occurring after another one meeting all the same requirements. 
At worst, there are $k$ penalties.
%
Under \eqref{eq:AlternativeUtility}, we define \textit{the fair good-$k$ set selection problem} or, simply, \textbf{the good-\textit{k} problem} as:
%
\begin{equation}
\label{eq:fair_objective_U_psi}
    \begin{aligned}
    \argmax_{S^k \in [\mathcal{C}]^k} & \quad U^k_{\psi} \big(S^k, \theta\big) \\
    \textrm{s. t.} & \quad f(S^k) \geq q
    \end{aligned}
\end{equation}
%
for which we denote its solution
% We denote the solution of (\ref{eq:fair_objective_U_psi}) 
as $S_{\goodtext}^k(\psi)$ or, if there is no ambiguity on $\psi$, simply as $S_{\goodtext}^k$.
% Note that, 
If the fairness constraint is strengthened to a fixed quota, i.e.,~$f(S^k) = q$, it can be shown that the solution is unique. In the general case, i.e.,~$f(S^k) \geq q$, there can be two solutions but with different fractions of the protected group.
For this last point, see Example~\ref{ex:diff_fractions_prot} in Appendix~\ref{Appendix.NaiveUtilityGoodk}.
%
Further, we emphasize that \eqref{eq:AlternativeUtility} is not the only utility model for \eqref{eq:fair_objective_U_psi}. Other models are possible as long as they properly describe the partial search allowed within the good-$k$ problem. We leave this for future work.
For more details, also see Appendix~\ref{Appendix.NaiveUtilityGoodk} where we motivate \eqref{eq:AlternativeUtility} 
%%% Jose: Appendix?
by presenting a simpler utility model (read, without the penalty) and showing its failure to stop $h$ from evaluating more candidates.
%index of the last candidate to be evaluated, according to $\theta$, to complete the set of $k$ eligible candidates.
%The fewer candidates $h$ are evaluated to obtain $S^k$, the more utility the screener gets from the selection.
%Otherwise, if $S^k$ contains at least one non-eligible candidate, the utility of selecting $S^k$ is a zero.
%Evidently, $S^k_{\psi}$ of \eqref{eq:S_k_psi} maximizes the above utility, without including the fairness requirement.
% We come back to this utility function at the end of this section.

%%% Jose: Appenddix?
% %
% \begin{example}
% \label{ex:diff_fractions_prot}
%     % For example, c
%     Consider $n=3, k=2, q=0.5$. Assume three eligible candidates and $\theta(1) = c_1, \theta(2) = c_2, \theta(3) = c_3$ with $W_1 = 0$ and $W_2 = W_3 = 1$. Both $S' = \{c_1, c_2\}$ and $S'' = \{c_2, c_3\}$ are solutions of (\ref{eq:fair_objective_U_psi}) with $U^k_{\psi} \big(S', \theta\big) = U^k_{\psi} \big(S'', \theta\big) = 2$. However, $f(S') = 0.5$ and $f(S'') = 1$. Intuitively, $S'$ is obtained by strictly iterating over $\theta$.
%     % , which is the approach we take in Section~\ref{sec:ProblemFormulation.Algorithms} \ja{under the algorithmic solution of this problem}.
% \end{example}
% %

%%%%%%%%%%%%%%%%%%
%%%%%%%%%%%%%%%%%%
%%%%%%%%%%%%%%%%%%
% JA on 08/04/2024
%%% original flow for alt u 1 and alt u 2
% As a possible utility function in the good-$k$ setting, we first consider:
% %
% \begin{equation}
% \label{eq:AlternativeUtility2}
%     \hat{U}^k_{\psi}\big( S^k, \theta \big) = \left\{
%     \begin{array}{ll}
%         n - \max_{c \in S^k} \theta^{-1}(c) & \text{if} \  \forall c \in S^k \  s(\mathbf{X}_c) \geq \psi   \\
%         0 & \text{otherwise.}
%     \end{array} \right.
% \end{equation}
% %
% Intuitively, in the above definition, the screener wants to find as quickly as possible a set of $k$ eligible candidates.
% Therefore, if $S^k$ contains only eligible candidates, the utility of $h$ selecting $S^k$ under $\theta$ is expressed by the number of candidates past the last one who was screened, i.e.,~the ``saved effort" of the screener $h$.

% Despite the simplicity of the above definition, the utility function (\ref{eq:AlternativeUtility2}) is not suitable to properly model our intended problem.
% To observe this point, let $n=3, k=2, q=0.5$. 
% Assume three eligible candidates and $\theta(1) = c_1, \theta(2) = c_2, \theta(3) = c_3$ with $W_1 = W_2 = 0$ and $W_3 = 1$. It turns out that both $S' = \{c_1, c_3\}$ and $S'' = \{c_2, c_3\}$ maximize the utility and satisfy the fairness constraint. However, why should have been $c_2$ considered, and then returned in $S''$, if $c_1$ already meets the minimum basic requirement? 
% A reason for doing that is a variant of our problem, in which the screener keeps evaluating non-protected candidates, even if their quota is reached but the one of protected candidates is not yet reached, for the purpose of keeping the best ones found so far. 
% We do not consider such a variant in this paper.

% Let us define the \textit{penalty function} $p(c, S^k, \theta) = \mathbb{1}(\exists\ c' \in \mathcal{C}\setminus S^k \mbox{s.t.}\ \theta^{-1}(c') < \theta^{-1}(c) \wedge s(\mathbf{X}_{c'}) \geq \psi \wedge W_{c'}=W_c)$ which, for a candidate $c$, looks for another candidate of the same group as $c$ and meeting the minimum basic requirement, who occurs before $c$ in the order $\theta$, but who has not been selected into $S^k$. 
% Basically, the penalty function models the ``wasted effort" in choosing a candidate occurring after another one meeting all the same requirements. 
% At worst, the are $k$ penalties, which leads to the following refined utility function:
% %
% \begin{equation}
% \label{eq:AlternativeUtility}
%     U^k_{\psi}\big( S^k, \theta \big) = \left\{
%     \begin{array}{ll}
%         k - \sum_{c \in S^k} p(c, S^k, \theta) & \text{if} \  \forall c \in S^k \  s(\mathbf{X}_c) \geq \psi   \\
%         0 & \text{otherwise.}
%     \end{array} \right.
% \end{equation}
% %
%%%%%%%%%%%%%%%%%%
%%%%%%%%%%%%%%%%%%
%%%%%%%%%%%%%%%%%%

%
\begin{remark}
\label{remark:ISOandPS}
    The ISO $\theta$ can influence the screening process under the good-$k$ problem \eqref{eq:fair_objective_U_psi} due to the potential partial search of $\mathcal{C}$ by $h$. 
    This holds without assuming anything about $h$. 
    To observe this point, let $k=1$ and assume two candidates such that $s(\mathbf{X}_{c_1}) \geq \psi$ and $s(\mathbf{X}_{c_2}) \geq \psi$.
    A $\theta$ such that $\theta^{-1}(c_1) = 1$ and $\theta^{-1}(c_2) = 2$ would imply that $c_1$ is considered eligible before even evaluating $c_2$. 
    Conversely, a reverse $\theta$ such that $\theta^{-1}(c_1) = 2$ and $\theta^{-1}(c_2) = 1$ would imply eligible $c_2$ before $c_1$. Since $h$ will stop after finding $k=1$ good-enough candidates, $\theta$ affects which candidates are selected.
\end{remark}
%

%$s(\mathbf{X}_{c_1}) \geq \psi$ and $s(\mathbf{X}_{c_2}) \geq \psi$. An initial order were $\theta^{-1}(c_1) = 1$ and $\theta^{-1}(c_2) = 2$ would imply by (\ref{eq:order}) that $c_1$ is considered eligible before even evaluating $c_2$. Conversely, the reverse initial order $\theta^{-1}(c_1) = 2$ and $\theta^{-1}(c_2) = 1$ would consider eligible $c_2$ before $c_1$. Since the screener will stop after finding $k=1$ good enough candidates, the initial order affects which candidates are selected.

%This fact is useful for unifying both good-$k$ and best-$k$ settings into a single problem formulation.
%To highlight how distinct this setting is relative to the standard best-$k$, let us formulate the good-$k$ setting without needing to resort to specifying a goal or utility function.
%We will return to the above utility function at the end of this section.

%We argue that the screener $h$ can settle for looking through a subset $\candidatessubset$ of cardinality $m \leq n$ as long as it contains enough $k$ eligible candidates so that it also holds $m \geq k$. 
%Moreover, $\candidatessubset$ is composed by the set of the first $m$ candidates of the initial order $\theta$, so we can state that $\candidatessubset = \{c \in \candidatesset \mid \theta^{-1}(c) \leq m\}$. 
%Here, $\theta$ induces an ordering of $\candidatessubset$ captured by the initial order restricted to its first $m$ elements $\theta^{(m)}$. 
%Once the selected set has cardinality $k$, the screener has no incentive to look for more eligible candidates according to $\psi$.
%Hence, without the fairness condition, the selected set is the set of the first $k$ eligible candidates in $\mathcal{C}_{\psi}$ following $\theta_{\psi}$.
%Further, if the screener $h$ is fair, it must meet the representational quota $q$ of the protected group that meets the minimum basic requirements.
%We can restate problem \eqref{eq:fair_objective_U_psi} as:
%
%\begin{equation}
%\label{eq:Alternativefair_objective_all_screener}
%\begin{aligned}
%    \min_{m \in [k, \ldots, n]} \quad & m \\
%    \textrm{s. t.} \quad & |\{ \theta_{\psi}(i) \mid i=1, \ldots, m\}| = k \\
%    \quad & f(S_{\goodtext}^k) \geq q
%\end{aligned}
%\end{equation}
%
%where the selected set is exactly $S_{\goodtext}^k = \{ \theta_{\psi}(i) \mid i=1, \ldots, m\}$, so that it necessarily has cardinality $k$. 

%\paragraph{A fair initial screening order problem formulation.}
%To summarize, we represent both best-$k$ and good-$k$ settings into one formulation for the set selection problem based on the initial order $\theta$:
%
%\begin{equation}
%\label{eq:GeneralISOFormulation}
%    \begin{aligned}
%        \max_{S^k \in \mathcal{S}^k} & \quad U^k(S^k, \theta) \\
%        \textrm{s. t.} & \quad f(S^k) \geq q
%    \end{aligned}
%\end{equation}
%
%where we obtain two solutions for the optimal and fair screener $h$ depending on the utility function specification. 
%Under $U^k_{\addtext}$ as by \eqref{eq:Utility}, the solution is $S_\besttext^k$.
%In this case, $\theta$ is redundant as it does not influence $U^k_{\addtext}$. 
%This is because $h$ will always explore all of $\candidatesset$ and sort it as the ranking $\tau \in \candidatesset$ regardless of $\theta$. 
%Still, $h$ explores $\candidatesset$ as prescribed by $\theta$.
%Under $U^k_{\psi}$ as by \eqref{eq:AlternativeUtility}, the solution is $S_\goodtext^k$. 
%In this case, $\theta$ is a key input to $U^k_{\psi}$, despite not being a parameter that can be optimized by $h$. 
%This is because $h$ can partially or fully explore $\candidatesset$ as prescribed by $\theta$ depending on when it meets its goal.
%With \eqref{eq:GeneralISOFormulation} we present the general initial screening order problem formulation.

% Antonio: I've commented out again as we don't use meaningulness before (I changed that also as you suggested)
% For a screener $h \in \mathcal{H}$, the ordering insider the chosen subset $\mathcal{S}^{k}$ can be meaningful. In principle, $S^{k}_{\besttext}$ leads to a meaningful candidate selection relative to $S^{k}_{\goodtext}$ as candidates are ordered by their scores. In practice, however, it depends on how the chosen subset is used later on in the hiring pipeline, meaning whether in the next phase the order within $\mathcal{S}^k$ carries any information (e.g., interviewing candidates as described by $\mathcal{S}^k$). Hence, the chosen set can be meaningful but it is not of interest here as its meaning is determined beyond the screener $h$.

\subsection{Two Search Procedures}
\label{sec:ProblemFormulation.Algorithms}

We present two search procedures for the best-$k$ and good-$k$ problems.
These two algorithmic implementations, with the \textit{ExaminationSearch} solving for $S_{\besttext}^k$ and the \textit{CascadeSearch} solving for $S_{\goodtext}^k$, allow us to operationalize the two problem formulations given the ISO $\theta$ of candidates chosen by or provided to $h$.
%%% TBD
% \ja{Due to space limitations, we detail each algorithm in Appendix~\ref{Appendix.DiscussionAlgorithms}.}

The \textbf{\textit{ExaminationSearch}} procedure, Algorithm~\ref{algo:Examination}, solves the best-$k$ setting, returning $S^k_\besttext$ for given $n$ (candidates), ISO $\theta$, and parameters $k$ (subset size) and $q$ 
% (minimum fraction of selected candidates from the protected group)
(fairness group constraint). 
First, line 2 calculates the minimum number $q^*$ of candidates from the protected group to be selected, and the maximum number of candidates $r^*$ not in that quota. 
Then, candidates are considered by descending scores, using the \texttt{argsortdesc} procedure (lines 2, 3). 
The loop in lines 5-13 iterates until $k$ candidates are found. The loop adds candidates to the sets $Q$ and $R$: $Q$ are candidates in the quota of the protected group; $R$ are candidates not in that quota (can be non-protected or protected). An non-protected candidate can be only added to the $R$ set, thus line 7 checks if there is still room in $R$ to do this. A protected candidate is added to the quota set $Q$ if there is room (lines 10-11) or to the other set $R$ otherwise (lines 12-13). 
Finally, the procedure returns the candidates in the quota set $Q$ or in the other set $R$. 
The result of the \textit{ExaminationSearch} procedure maximizes (\ref{eq:fair_objective_all_screener}), as candidates are added in decreasing score, while keeping the fairness constraint through the quota management.

The \textbf{\textit{CascadeSearch}} procedure, Algorithm~\ref{algo:Cascade}, solves the good-$k$ setting, returning $S^k_\goodtext$ for given $n$, ISO $\theta$, and parameters $k$, $q$ and $\psi$ (minimum basic requirement). 
The difference with the \textit{ExaminationSearch} procedure consists in strictly following the ISO $\theta$ (line 4), and in checking $\psi$ (line 8) before adding a candidate to the quota set $Q$ or to the other set $R$.
The result of the \textit{CascadeSearch} procedure maximizes (\ref{eq:fair_objective_U_psi}), as no penalty is accumulated in the loop. 
In fact, a non-protected candidate ($W_c=0$) is not added only because there is no room in $R$ -- and $R$ never gets smaller to allow for more room later on. A protected candidate ($W_c=1$) is not added only if not meeting the minimum basic requirement (line 8), hence it cannot be counted for the penalty. Formally, $U^k_{\psi}\big(S^k_\goodtext, \theta\big) = k$ for the solution $S^k_\goodtext$ returned by \textit{CascadeSearch}.

%
% Figure environment removed
%

\subsection{Additional Related Work}
\label{sec:Add_RelatedWork}

We presented two formulations of fair set selection problems. 
Here, we can now compare our formulations with the fair ranking and set selection literature, where, in the latter, $h$ cares about the order within the chosen $S^k$.

The reference ranking problem is the top-$k$ selection \cite{fagin2001optimal} representing the standard setting in the fairness literature \cite{Zehlike2023_FairRanking_P1,Zehlike2023_FairRanking_P2}. 
The reference set selection problem is the Secretary Problem (SP), inspired by the hiring process \cite{ferguson1989solved}: the goal is to select the maximum valued element in a randomly ordered sequence of $n$, with the condition of committing irrevocably to the acceptance or rejection decision after each evaluation.
The past literature has already analyzed the fairness implication of the SP, even in the $k$-choice extension (e.g., \cite{stoyanovich2018online}).
Our best-$k$ and good-$k$ formulations, differently from the SP, focus on the screening process, not the interview phase. 
More generally, we assume an offline set selection, that is, $h$ evaluates the applicants individually according to the ISO and each decision is not irrevocable. 
Instead, the SP is online, i.e., a decision must be made as each applicant is presented.
Notably, past works focusing on the online setting assume the additive utility we employ in the best-$k$ formulation \cite{stoyanovich2018online, mehrotra2021mitigating}.
This fact shows how the peculiar good-$k$ set selection formulation, inspired by G's hiring process, differs from the previously analyzed settings.

We emphasize that our research question focuses on the impact of the ISO $\theta$
% choice, which can be potentially linked to the protected attribute, 
on fair screening practices. 
% in the industry. 
With few exceptions (recall, Section~\ref{sec:RelatedWork}), the impact of $\theta$ is neglected by the past literature to the best of our knowledge.
For this reason, we move from the optimality analysis of the two algorithms (e.g., \cite{fagin2001optimal}), toward modeling the behaviour of a screener solving the two problem formulations. 
For this reason, we do not aim to provide novel optimal algorithms for the two problem formulations.
The two algorithms we discuss are inspired by the click-model literature. 
They are inherently sequential and can be applied online because we aim to model a human-like screener, described in the next section, that operates sequentially.
Yet, in both settings, the applicants are disclosed according to $\theta$ and not to the score, differently from past works \cite{stoyanovich2018online}.
We add to this line of work by introducing $\theta$ as a parameter of interest. 

%
% EOS
%

%%% Old utility definitions...
%% SR
%Formally, the ranking $\tau \in \Theta$ represents an ordering of $\candidatesset$ that follows the scoring function $s$, such that, if $i \leq j$, $s(\mathbf{X}_{\tau(i)}) \geq s(\mathbf{X}_{\tau(j)})$.
%Given $\tau$, the cut-off for being selected into $S^k$ is $s(X_{\tau(k)})$ with $\tau(k)$ denoting the $k^{th}$ position in the ranking $\tau$.
%We can then model the goal of the best-$k$ selection, without fairness requirements, as:
%
%\begin{equation}
%\label{eq:objective_best-k}
%    S_{\tau}^k = \{\tau(1), \dots, \tau(k)\}
%\end{equation}
% 

%let us call $\mathcal{C}_{\psi}$ the set of eligible candidates according to the minimum basic requirements $\psi$. 
%Evidently, we need to assume that $|\mathcal{C}_{\psi}| \geq k$.
%It is useful to denote $\Theta_{\psi}$ as the set of total orderings of $\mathcal{C}_{\psi}$.
%Under this view, $\theta$ induces a total order $\theta_{\psi} \in \Theta_{\psi}$ such that, given two candidates $c', c'' \in \candidatesset_{\psi}$, then $\theta_{\psi}^{-1}(c') < \theta_{\psi}^{-1}(c'')$ if $\theta^{-1}(c') < \theta^{-1}(c'')$.
%In this case, without the fairness condition, it helps to indicate the subset of $\candidatesset_{\psi}$ containing the first $k$ suitable candidates according to $\theta_{\psi}$ as $S^k_{\psi}$.
%Formally:
%
%\begin{equation}
%\label{eq:S_k_psi}
%    S^k_{\psi} = \{\theta_{\psi}^{-1}(i) \mid i = 1, \dots, k\}
%\end{equation}
%
%It is possible for two screeners $h_1$ and $h_2$ that chose different initial orders $\theta_1$ and $\theta_2$, under the same $s$ and $\psi$, to obtain different selected sets. 
%
%For this last scenario to hold, though, it must be unappealing, utility-wise, for both $h_1$ and $h_2$ to search the entirety of $\candidatesset$ according to $\theta_1$ or $\theta_2$.
%The notion of any $h$ optimally choosing $k$ candidates acquires a different meaning in the good-$k$ setting since the utility function $U^k_{\addtext}$ as described in \eqref{eq:Utility} is inadequate to formalize a screener for whom a selected set of $k$ good enough candidates is sufficient. 
%We define the simplest expression for the utility $U^k_{\psi}$ motivating the screener in the good-$k$ setting:

%%% Old search procedures stuff

% Both Algorithms~\ref{algo:Examination}~and~\ref{algo:Cascade} implicitly assume there are enough suitable $m$ candidates in $\candidatesset$ according to the minimum basic requirements $\psi$, such that $k \geq m \geq n$. 
% Hence, both achieve the respective solution. 
% These are the simplest algorithms for the examination and cascade searches, which can easily be extended by allowing the screener to update, e.g., $\psi$ of $k$ depending on the size of $S^k$ during the search.

%We highlight the fairness goal of obtaining a representative selected set of $k$ candidates as denoted by $q$.
%In Algorithms~\ref{algo:Examination}~and~\ref{algo:Cascade}, $q^*$, 
% or \texttt{line 2} for both of them, 
%represents the minimum number of protected candidates ($W_c = 1$) to be selected. 
%Hence, $k - q^*$ represents the maximum number of non-protected candidates ($W_c = 0$) to be selected.
%Under \texttt{ExaminationSearch} (Algorithm~\ref{algo:Examination}), meeting this quota is based on searching the full candidate pool according to $\theta$, sorting it by the scores in two separate lists according to $W$, and choosing the $q^*$-best protected candidates from $\tau_0$ and the $(k-q^*)$-best non-protected candidates from $\tau_1$.
%Under \texttt{CascadeSearch} (Algorithm~\ref{algo:Cascade}), meeting this quota goes on a candidate by candidate basis, thus, not requiring to search the full candidate pool according to $\theta$. 
%Here, the algorithm prioritizes constructing $S^k$ with $q^*$ eligible protected candidates $l_1$ while keeping track of the eligible non-protected candidates $l_0$.
%Once the quota is met, if between $l_1$ and $l_0$ there are not enough $k$ candidates then the algorithm keeps searching though without considering group membership. Notice that $l_0$ cannot be greater than $k-q^*$.

%
%\begin{definition}[Examination Search]
%\label{def:ExaminationSearch}
%    Given $\theta$, $h$ explores all of $\candidatesset$, computing each $c$ candidate's score $s(\mathbf{X}_c)$, ranking them all into $\tau$, and choosing the first $k$-candidates accordingly while ensuring that the fairness representational quota $q$ is met. 
%    Algorithm~\ref{algo:Examination} implements this search procedure.
%\end{definition}
%

%
%\begin{definition}[Cascade Search]
%\label{def:CascadeSearch}
%    Given $\theta$, $h$ explores $\candidatesset$ up until reaching $k$-candidates that meet the minimum basic requirements $\psi$ in terms of each candidate's $c$ score $s(\mathbf{X}_c)$ while ensuring that the fairness representational quota $q$ is met. 
%    Algorithm~\ref{algo:Cascade} implements this search procedure.
%\end{definition}
%
