%
\label{sec:Introduction}

Candidate screening is a complex, human-dependent process. 
It consists of a decision-maker, which we refer to as the \textit{screener}, tasked with choosing $k$ suitable candidates from a candidate pool. 
Common candidate screening processes include the evaluation of resumes for a job interview \cite{Pisanelli2022_YourCV} or application packages for university admission \cite{SukumarMH18_PeacanPie}. 
The screener usually evaluates the candidate pool using limited information and under strict time constraints. 
Because of this, as with other repetitive tasks performed by humans \cite{Kahneman2011Thinking, Kahneman2021Noise}, candidate screening is prone to biased decision-making \cite{Pisanelli2022_YourCV, DBLP:conf/chi/EchterhoffYM22}. 
An industry around automating candidate screening using Machine Learning (ML) has emerged in recent years \cite{DBLP:journals/air/WillKL23},
though with limited results in terms of fairness \cite{DBLP:conf/fat/WilsonG0MBSTP21,Wehner20,Sonderling22,Raghavan2020AlgortihmicHiring,DBLP:journals/datamine/RheaMDSSSKS22,jintelligence9030046,DBLP:journals/patterns/SloaneMC22, Fabris2023_DBLP:journals/corr/abs-2309-13933}. 
The future of automated decision-making (ADM) for candidate screening is an open question as ADM systems for this task are increasingly required by stakeholders, 
such as regulators, 
to consider how humans interact with their output \cite{Carricco2018EUHumanCentred, DBLP:conf/aaai/Ruggieri0PST23}. 

In this paper, we investigate the role of the \textit{initial screening order} (ISO) in the fair set selection problem within candidate screening. 
The ISO refers to the order in which the candidates appear in the candidate pool.
It can be chosen by 
% (like sorting candidates by their last name) 
or provided to 
% (like using the ranking of candidates from a search engine) 
the screener.
We find that the ISO has an impact in terms of optimality (choosing the best candidates) and fairness (treating similar candidates similarly) on the selected set of candidates by the screener,
especially when the screener is human (e.g., \cite{PeiEAAMO2023, DBLP:conf/chi/EchterhoffYM22}).
This is mainly because of the \textit{position bias} inherent to the ISO.
Position bias refers to the penalty (or premium) each candidate experiences due to where it falls on the ISO \cite{DBLP:journals/cacm/Baeza-Yates18}.
Humans are predisposed to favor the items being placed at the top of a list as we tend to read from top to bottom \cite{DBLP:journals/cacm/Baeza-Yates18, 10.1093/qje/qjr028}.
% The ISO determines when the screener comes across a specific candidate;
Clearly, there is a benefit for those candidates that appear at the start of the ISO.
We further motivate the ISO problem in Section \ref{sec:Generali}.

Our results are based on a modeling framework for understanding how a screener interacts with an ISO along with a simulations framework that implements extensively the problem formulations for practical insights on the influence of the ISO.
We start with Section~\ref{sec:ProblemFormulation},
in which we describe how a screener explores the ISO in search of an optimal and fair (at the group-level w.r.t. a representational quota $q$) set of $k$ candidates.
We define two problem formulations.
In the best-$k$ the screener selects the $k$ top candidates, while
in the good-$k$ the screener selects the $k$ first good-enough candidates w.r.t. some minimum score for measuring candidate quality.
We devise algorithmic solutions for both problem formulations.
The good-$k$ is noteworthy as it allows the screener to partially search the candidate pool;
set selection problem formulations often assume a full search by the screener.
We discuss this further in Sections \ref{sec:RelatedWork} and \ref{sec:Add_RelatedWork}.
% To the best of our knowledge, this paper provides the first formalization for the good-$k$.

To better understand the impact of the ISO when a human screener is involved, 
in Section~\ref{sec:HumanScreener} we introduce and analyse the algorithmic and human-like screeners. 
The former refers to a consistent screener (like a ML model);
the latter refers to an inconsistent screener (like a human) whose evaluation of candidates suffers over time due to fatigue (see, e.g., \citet{Kahneman2011Thinking}).
We enhance our analysis of these two screeners in Section~\ref{sec:Experiments} through simulated experiments that mimic multiple screening settings and help us understand the impact of the ISO.
The results, although specific to our assumptions, confirm the role of position bias inherent to the ISO.
We find, for instance, that the human-like screener can violate individual fairness \cite{DBLP:conf/innovations/DworkHPRZ12} by not evaluating similar candidates similarly while still meeting $q$. 
We also find that the algorithmic screener can miss the best candidate depending on its search procedure.
In short, we find that where a candidate falls within the ISO can determine its chances of getting selected,
highlighting the influence of position bias.
We conclude in Section~\ref{sec:Discussion} by discussing the limitations of our work, future work, and key takeaways.
% Importantly, we argue that our simulations framework, which is open source, represents a tool for studying screening processes under the ISO.
% The code in-itself is a valuable contribution considering how difficult it can be to run real world screening seeings (see, e.g., \cite{PeiEAAMO2023, DBLP:conf/chi/EchterhoffYM22}). 

Our main contributions are threefold.
First, 
% we provide insights from a real-world candidate screening problem, summarized as stylized facts, that are important to the fair set selection literature.
% Second, 
we formalize the role of the ISO in two possible objectives of the screener with the best-$k$ and good-$k$ problems. 
% With the good-$k$ we also introduce a screener that can search partially the candidate pool.
% In particular, we point out that for good-$k$ the optimality can be reached by a partial search of the candidate pool
%
Second, we formulate a human-like screener and compare it to its algorithmic counterpart both theoretically and with experiments.
%
Third, with the simulations framework, we provide a useful, open source tool for studying the ISO problem and inform stakeholders without needing to run real world screening scenarios, which can be both costly and time consuming. 

\subsection{Qualitative Background}
%
\label{sec:Generali}

This work borrows from a previous collaborative effort at a European Fortune Global 500.
We refer to this company as G.
The purpose was to study G's hiring process as an algorithmic fairness problem.
We worked closely with Human Resources (HR), focusing on candidate screening.
In this phase of G's hiring process, an HR officer reduces the candidate pool for a job opening into a smaller pool of suitable candidates based on each candidate's profile.
The candidate pool was stored in Oracle's Taleo, a hiring platform used by HR officers to, among other things, obtain an ISO.

The following \textit{five stylized facts} summarize key practices by the HR officers (henceforth, screeners) that motivated the ISO problem.
%
\textit{\textbf{G1}} \textit{Varying ISOs.} Screeners chose the ISO. The choice was restricted by the sorting fields of the hiring platform, such as using the candidates' last name.
%
\textit{\textbf{G2}} \textit{Two ways to search the candidate pool.} Two search practices became apparent: full or partial search of the candidate pool.
%
\textit{\textbf{G3}} \textit{Meeting the set of minimum basic requirements.} Screeners were able to differentiate candidates relative to each other, but their focus was on finding candidates that met these requirements. 
Order within the selected $k$ candidates was not necessarily important.
%
\textit{\textbf{G4}} \textit{Diverse suitable candidates.} Fairness goals already existed in the form of representation quotas, often around gender, that were enforced by the screeners. 
%
\textit{\textbf{G5}} \textit{A consistent notion of time.} Screeners aimed at spending one minute per candidate.

Although G1-5 are specific to G, they highlight salient aspects of real-world candidate screening problems likely to hold in similar settings involving humans searching a pool of candidates (see, e.g., \cite{DBLP:journals/jcmc/PanHJLGG07, DBLP:conf/clef/GrotovCMSXR15, DBLP:conf/www/RichardsonDR07, DBLP:conf/chi/EchterhoffYM22, PeiEAAMO2023}).
G4-5 are standard to the fair set selection problem, especially under an algorithmic screener, while G1-3 introduce new considerations to such problem formulation, especially under a human screener.
We come back to G1-5 in Sections \ref{sec:ProblemFormulation} and \ref{sec:HumanScreener}.

%
% EOS
%


\subsection{Related Work}
%
\label{sec:RelatedWork}

In formalizing the ISO problem, our work revives the role of position bias within a screening process involving human screeners.
% Click models
Works on search engine click models were the first to formalize \cite{CraswellZTR08_ExperimentsClickPositionBias} and test \cite{DBLP:journals/jcmc/PanHJLGG07, DBLP:conf/clef/GrotovCMSXR15, DBLP:conf/www/RichardsonDR07} how users search over an ISO.
These works 
% from the early 2000s 
modeled the different clicking procedures observed in users and
% , using among others eye-tracking technology, 
provided experimental evidence for the existence of position bias.
% The name choice of the two search procedures in Section~\ref{sec:ProblemFormulation.Algorithms} is a reference to this line of work. 
Hence, it is not surprising that today product owners are willing to pay premiums to search engines and similar platforms for the first spots in the search results of a user \cite{10.1093/qje/qjr028}.
Different from these works, we consider a user that chooses (i.e., ``clicks'') more than one item. Further, we formalize such users under a utility-maximizing framework with fairness constraints, which relates to past set selections works \cite{stoyanovich2018online}.

% Overall fairness
Within algorithmic fairness, focus has been mainly on learning an algorithm that provides a fair ISO \cite{Zehlike2023_FairRanking_P1, Zehlike2023_FairRanking_P2, DBLP:journals/vldb/PitouraSK22}.
% Within this literature, position bias is viewed as a technical bias, meaning it is due to platform design or inherent to the human subject.
Works on probability-based \cite{DBLP:conf/ssdbm/YangS17, DBLP:conf/cikm/ZehlikeB0HMB17} and exposure-based \cite{DBLP:journals/cacm/Baeza-Yates18, DBLP:journals/sigir/JoachimsGPHG17} fairness tackle position bias by (re-)arranging the ISO so that is fair according to, respectively, some model for user attention and spot allocations. 
These works, however, avoid formalizing the user for which the fair ISO is meant, e.g., assuming a complete search of the (re-)arranged ISO by the user.
Based on the click models and our own experience with G, we argue that it is important to model the human user to fully grasp the ISO problem.
The fair set selection literature, including the fair ranking, is vast; we position our work within this literature in Section~\ref{sec:Add_RelatedWork} after we have formulated the ISO problem.

% Evidence of position bias
We highlight two recent works that provide evidence for position bias and the implicit role of the ISO.
\citet{DBLP:conf/chi/EchterhoffYM22} collaborate with a college to study anchoring bias \cite{Kahneman2011Thinking} in admissions;
anchoring bias \cite{tversky_judgment_1974} occurs when a candidate's evaluation is conditioned by the quality of the previously evaluated candidates.
This paper finds that the same candidate is better off if it is preceded by worst candidates than better candidates, which violates individual fairness \cite{DBLP:conf/innovations/DworkHPRZ12}, and
% It occurs because the human screener anchors its expectations on a lower reference point despite each candidate being independent from each other. 
proposes an algorithm for capturing and balancing anchoring bias.
% Moreover, 
\citet{PeiEAAMO2023} work with a college to study how the platforms used by professors for evaluating homework affects the students' grades. 
This paper runs a set of experiments varying the order in which the homeworks are presented by the platform. 
The experiments shows that the default sorting order, which is in alphabetical order, unfairly rewards students with the same work quality depending on their last names, which violates individual fairness \cite{DBLP:conf/innovations/DworkHPRZ12}. 

Both papers provide evidence of the ISO problem (w.r.t.~fairness) and, similar to our work, highlight the role of the human using the ISO.
However, we differ from these works
% and the few other works based on real world collaborations (e.g.,~\citet{SukumarMH18_PeacanPie}) 
in that we provide a simulations framework supported by our problem formulation flexible enough to capture multiple screening scenarios. 
Gathering experimental data or having access to the real process are costly and limited to stakeholders studying questions of fairness, which positions simulations as useful tools for answering these questions \cite{DBLP:conf/fat/IonescuHJ21, DBLP:journals/corr/abs-2006-09663, DBLP:conf/fat/BountouridisHMM19, Bokanyi2020_Understanding, Schelling1971_Dynamic}.
% Our work, e.g., can be useful to simulate different ISO problems to inform the human screener and diminish the role of the position bias. 
We come back to this point in Section~\ref{sec:Discussion}.

%
% EOS
%


%
% EOS
%
