%
\label{sec:RelatedWork}

This work focuses on the screener's search behavior given an ISO.
%
The creation of the ISO itself can be modeled as a fair set selection or, if order matters, fair ranking problem \cite{Zehlike2023_FairRanking_P1, Zehlike2023_FairRanking_P2, DBLP:journals/vldb/PitouraSK22} in which the goal is to learn a fair ISO from data using, e.g., probability-based \cite{DBLP:conf/ssdbm/YangS17, DBLP:conf/cikm/ZehlikeB0HMB17} and exposure-based \cite{DBLP:journals/cacm/Baeza-Yates18, DBLP:journals/sigir/JoachimsGPHG17} methods.
Similarly, other works, e.g., \cite{DBLP:conf/wsdm/KokkodisPI15, DBLP:conf/kdd/Kokkodis18, Hadass_2004_TheEffectInternetRecruiting} study how information access systems, like online job markets, enable the creation of candidate pools and, in turn, the ISO.
%
Instead, motivated by our experience at G, we are interested in how a screener, particularly a human one, searches the ISO.
With some exceptions \cite{DBLP:conf/chi/EchterhoffYM22, PeiEAAMO2023}, most fairness works avoid studying the user of the ISO, treating, e.g., position bias as a technical bias.
%
Our screener-centric approach is similar to web search click models that were the first to formalize \cite{CraswellZTR08_ExperimentsClickPositionBias} and test \cite{DBLP:journals/jcmc/PanHJLGG07, DBLP:conf/clef/GrotovCMSXR15, DBLP:conf/www/RichardsonDR07} how users search over an ISO.
Different from these works, we consider a user that ``clicks'' on more than one item, and formalize such user under a utility-maximizing framework with fairness constraints, relating the insights from these works to candidate screening as well as to past fair set selection works \cite{stoyanovich2018online}.

The works on click models provide empirical evidence for the position bias, though they precede considerably the fairness literature.
%
Notably, two works provide recent empirical evidence for position bias in candidate screening due the ISO with a focus on individual fairness \cite{DBLP:conf/innovations/DworkHPRZ12}.
%
\citet{DBLP:conf/chi/EchterhoffYM22} collaborate with a college to study anchoring bias \cite{tversky_judgment_1974} in admissions officers. 
They find that the same applicant is better off if it is preceded by worst rather than better applicants, and propose an algorithm that balances out the anchoring bias when presenting applications to the admissions officer.
%
\citet{PeiEAAMO2023} also collaborate with a college to study how the platforms used by professors when evaluating homeworks affects the students' grades. 
They run experiments varying the order in which the homeworks are presented by the platform, and show that the default alphabetical order unfairly rewards students with the same work quality due to their last names.
%
We differ from these two works by defining the ISO as a parameter in the problem formulation. We also do not work with empirical data, and instead provide a simulations framework flexible enough to capture multiple screening scenarios, including those in \cite{DBLP:conf/chi/EchterhoffYM22, PeiEAAMO2023}.

Our best-$k$ and good-$k$ formulations (Section~\ref{sec:ProblemFormulation.Objectives}) belong to the fair set selection literature \cite{DBLP:conf/eaamo/BueningSBGD22}. 
The reference problem is the secretary problem (SP) \cite{ferguson1989solved}
% , inspired by the hiring process \cite{ferguson1989solved}, 
where we select a candidate in a randomly ordered sequence 
% of $n$ 
% with the condition of 
committing irrevocably to the acceptance or rejection decision after each evaluation.\footnote{When order matters, the reference problem is the top-$k$ selection \cite{fagin2001optimal}.}
The past literature has already analyzed the fairness implication of the SP, even in the $k$-choice extension \cite{stoyanovich2018online}.
However, our best-$k$ and good-$k$ formulations, differently from the SP, focus on the screening process, not the interview phase. Formally, we assume an offline set selection, meaning each candidate is individually evaluated by the screener and each decision is not irrevocable. 
Such setting emphasizes the role of the ISO.
Notably, past works focusing on the SP online setting assume the additive utility we employ in the best-$k$ formulation \cite{stoyanovich2018online, mehrotra2021mitigating}.
This fact shows how peculiar the good-$k$ problem formulation is relative to the previously analyzed settings. 
% See Section~\ref{sec:ProblemFormulation.Objectives} for technical details.

%
% EOS
%
