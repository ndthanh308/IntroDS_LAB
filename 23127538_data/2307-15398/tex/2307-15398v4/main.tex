%%%
% \documentclass[sigconf,anonymous,review]{acmart}
%%% For arXiv
\documentclass[sigconf,screen,nonacm]{acmart}

%% \BibTeX command to typeset BibTeX logo in the docs
\AtBeginDocument{%
  \providecommand\BibTeX{{%
    \normalfont B\kern-0.5em{\scshape i\kern-0.25em b}\kern-0.8em\TeX}}}

%% Rights management information. This information is sent to you
\setcopyright{acmlicensed}
\copyrightyear{2025}
\acmYear{2025}
\acmDOI{XXXXXXX.XXXXXXX}
%
\acmConference[WSDM'25]{2025 International Conference on Web Search and Data Mining}{March 10--14, 2025}{Hannover, Germany}
%
%  Uncomment \acmBooktitle if th title of the proceedings is different
%  from ``Proceedings of ...''!
%
\acmBooktitle{2025 International Conference on Web Search and Data Mining (WSDM'25), March 10--14, 2025, Hannover, Germany}

%% Submission ID. This information is sent to you
%%\acmSubmissionID{123-A56-BU3}

%% Packages
\usepackage{algorithm}
\usepackage[noend]{algpseudocode}
\usepackage{orcidlink}
\usepackage{dsfont}
\usepackage{bbold}

%% Other:
\newtheorem{remark}{Remark}
\newcommand{\sr}[1]{\textcolor{orange}{#1}}
\newcommand{\ja}[1]{\textcolor{blue}{#1}}
\newcommand{\am}[1]{\textcolor{purple}{#1}}
\newcommand{\candidatesset}{\mathcal{C}}
\newcommand{\candidatessubset}{\mathcal{D}}
\newcommand\restr[2]{{% we make the whole thing an ordinary symbol
  \left.\kern-\nulldelimiterspace % automatically resize the bar with \right
  #1 % the function
  \littletaller % pretend it's a little taller at normal size
  \right|_{#2} % this is the delimiter
  }}
\newcommand{\littletaller}{\mathchoice{\vphantom{\big|}}{}{}{}}
\newcommand{\besttext}{\text{\textit{best}}}
\newcommand{\goodtext}{\text{\textit{good}}}
\newcommand{\addtext}{\text{\textit{add}}}
\DeclareMathOperator*{\argmax}{arg\,max}
\DeclareMathOperator*{\argmin}{arg\,min}

\newcommand{\ci}{\mathrel{{\scalebox{1.07}{$\perp\mkern-10mu\perp$}}}}
%%
\begin{document}

%%
\title{The Initial Screening Order Problem}

%%
\author{Jose M. Alvarez \orcidlink{0000-0001-9412-9013}}
% \email{jose.alvarez@sns.it}
% \orcid{0000-0001-9412-9013}
\affiliation{
  \institution{University of Pisa}
  \city{Pisa}
  \country{Italy}
}

\author{Antonio Mastropietro \orcidlink{0000-0002-8823-0163}}
% \orcid{0000-0002-8823-0163}
\affiliation{
  \institution{University of Pisa}
  \city{Pisa}
  \country{Italy}
}

\author{Salvatore Ruggieri \orcidlink{0000-0002-1917-6087}}
% \email{salvatore.ruggieri@unipi.it}
% \orcid{0000-0002-1917-6087}
\affiliation{
  \institution{University of Pisa}
  \city{Pisa}
  \country{Italy}
}

%%
\renewcommand{\shortauthors}{Alvarez, Mastropietro, and Ruggieri}

%%
\begin{abstract}
    We investigate the role of the initial screening order (ISO) in candidate screening tasks, such as employee hiring and academic admissions, in which a screener is tasked with selecting $k$ candidates from a candidate pool. The ISO refers to the order in which the screener searches the candidate pool. Today, it is common for the ISO to be the product of an information access system, such as an online platform or a database query. The ISO has been largely overlooked in the literature, despite its potential impact on the optimality and fairness of the chosen $k$ candidates, especially under a human screener. We define two problem formulations describing the search behavior of the screener under the ISO: the best-$k$, where the screener selects the $k$ best candidates; and the good-$k$, where the screener selects the $k$ first good-enough candidates. To study the impact of the ISO, we introduce a human-like screener and compare it to its algorithmic counterpart, where the human-like screener is conceived to be inconsistent over time due to fatigue. In particular, our analysis shows that the ISO, under a human-like screener solving for the good-$k$ problem, hinders individual fairness despite meeting group level fairness, and hampers the optimality of the selected $k$ candidates. This is due to position bias, where a candidate's evaluation is affected by its position within the ISO. We report extensive simulated experiments exploring the parameters of the best-$k$ and good-$k$ problems for the algorithmic and human-like screeners. The simulation framework is flexible enough to account for multiple screening settings, being an alternative to running real-world candidate screening procedures. This work is motivated by a real-world candidate screening problem studied in collaboration with an European company. 
\end{abstract}

%% The code below is generated by the tool at http://dl.acm.org/ccs.cfm.
\begin{CCSXML}
<ccs2012>
   <concept>
       <concept_id>10003120.10003123.10011758</concept_id>
       <concept_desc>Human-centered computing~Interaction design theory, concepts and paradigms</concept_desc>
       <concept_significance>500</concept_significance>
       </concept>
   <concept>
       <concept_id>10002951.10003317.10003338</concept_id>
       <concept_desc>Information systems~Retrieval models and ranking</concept_desc>
       <concept_significance>500</concept_significance>
       </concept>
 </ccs2012>
\end{CCSXML}

\ccsdesc[500]{Information systems~Retrieval models and ranking}
\ccsdesc[500]{Human-centered computing~Interaction design theory, concepts and paradigms}

%%
\keywords{Fair set selection, position bias, search user behavior}

%%
% \received{}
% \received[revised]{}
% \received[accepted]{}

%%
\maketitle

\newpage
\section{Introduction}
%%%%%%%%%%%%%%%%%%%%%%%%%%%%%%%%%%%%%%%%%%%%%%%%%%%%%%%%%%%%%%%%%%%%%%%%%%%%%%%%
\section{Introduction}

Autonomous driving (AD) %with deep learning networks 
has shown promising achievements and is considered an important technological breakthrough that could revolutionize the future of transportation. Currently, ensuring the safety of autonomous driving systems has become a topic of extensive development.
% There has been much discussion on how to verify the safety of autonomous driving systems.
One traditional solution for safety tests is to exhaustively enumerate real scenarios for validation. Nevertheless, this process is not only labor-intensive and costly but also dangerous. Simulation has emerged as a robust, safe, and efficient alternative for training and evaluating AD software and algorithms~\cite{li2019aads, amini2020learning, amini2022vista}.

% Figure environment removed

Recently, neural radiance field (NeRF)~\cite{mildenhall2020nerf} has gained significant attention in AD simulation~\cite{drivesim}. This approach leverages multi-view images to construct a 3D scene and enable novel view synthesis for both indoor and outdoor applications. When it comes to constructing NeRF models in AD simulation, there are two options available: 1) collecting a large amount of data to cover as many viewpoints as possible, and constructing a fine-grained scene offline; 2) directly using log data from road tests to quickly create an environment and dynamically simulate driving scenarios. The first choice can deliver high-quality simulation~\cite{tancik2022block} by transforming the problem of view extrapolation into view interpolation through the use of large amounts of data. However, it is time- and cost-intensive, which makes it challenging to generalize. As for the second choice, the collected images from log data are usually similar to each other along the running trajectory, which may result in unsatisfactory outcomes, particularly when the camera pose is placed out-of-trajectory (see \figref{figSupportComp} as an example), semantic consistency cannot be guaranteed when synthesizing images from deviated views. We observe this problem under this data condition in all neural radiance approaches, and to the best of our knowledge, none of the existing work has solved this issue.
In our opinion, semantic consistency is crucial for AD simulation, and synthesizing on deviated views is unavoidable for scalability.

AD simulation usually involves map data for planning and control, which can be obtained from a prebuilt High-Definition Map (HD Map) or an online mapping module. While the map data may not be pixel-perfect, it can provide semantic-level information that is useful for enhancing the semantic consistency of the trained neural radiance field.
In this paper, we propose incorporating map priors into neural radiance fields to enhance the semantic consistency and rendering quality of deviated driving view synthesis. Firstly, we employ ground information from maps to supervise the density field of NeRF, providing a more reliable road base for semantic entities. Next, we propose sampling rays to simulate unseen views. Unlike most NeRF augmentation methods~\cite{zhang2022ray, chen2022geoaug}, we utilize ground and lane information in sampling computations to guide the radiance field. More importantly, we model the above two supervision methods as weak supervision by using an uncertainty parameter and propose an uncertainty tempering scheme to increase the uncertainty. This ensures that map priors only guide the training process rather than enforce it towards their absolute values. As a result, our proposed method not only improves the rendering quality of interpolated novel view synthesis quantitatively but also enhances the semantic consistency of deviated novel view synthesis. 
Our contributions can be summarized as follows:
% We summarize the contributions of this paper as follows.



% To overcome the limitations of the collected data, this paper proposes a novel approach that leverages map information to enhance the semantic consistency of the synthesized driving views. 

% Autonomous driving (AD) vehicles are being trained with the help of deep learning networks and have shown promising achievements. This technology is considered to be a breakthrough that could change the way of transportation in the near future. However, there are many discussions on how to verify or judge the safety of autonomous driving systems.
% A straightforward solution towards the safety tests is to exhaustively enumerate real scenarios for validation as many as possible. However, the process of implementing different real scenarios is not only labor-intensive and costly, but also dangerous. Simulation has been proved to be an alternative, which is robust, safe, efficient in training, and evaluating AD software and algorithms.
% Now, the emerging technology of neural radiance field (NeRF)~\cite{} leverages multi-view images to construct a 3D scene and enable novel view synthesis for many indoor and outdoor applications. For AD simulation, there are two choices for constructing NeRF models: 1) collect a large amount of data, such as LiDAR and camera data, similar to mapping, to construct a fine-grained scene offline; or 2) directly use the log file (typically in the format of ROS bag) to rapidly create an environment and then dynamically simulate the driving scenarios.
% The first choice can achieve high-quality simulation, but it is time-consuming and expensive, making it difficult to generalize to very large scales. On the other hand, the second option is fast but can lead to low-quality simulation due to the data being sparse and similar to each other in log data. This paper tackles the problem raised by choosing the latter option and attempts to improve the quality of out-of-trajectory driving view synthesis by incorporating map information. This approach is practical for many autonomous driving tests.
% In conclusion, the use of NeRF technology for AD simulation is a promising avenue for training and evaluating AD software and algorithms. While both options for constructing NeRF models have their pros and cons, this paper addresses the challenges of the second option and proposes a potential solution to improve the quality of simulation.

%There exist a few attempts to facilitate training a NeRF model for synthesizing out-of-trajectory (or called as extrapo trajectory) views.


\begin{itemize}
    \item We propose a novel method to incorporate commonly used map priors in AD scenes into neural radiance fields to improve the out-of-trajectory driving view synthesis.
    \item We explicitly model the uncertainty in map priors as a parameter and propose an uncertainty tempering scheme to guide the training process of the neural radiance field.
    \item Experiments demonstrated that the proposed method can improve the semantic consistency of out-of-trajectory views and the rendering quality of novel view trajectory interpolation.
\end{itemize}

Our proposed method is easy to implement, can be easily plugged into existing NeRF algorithms, and has the capability of extending to other formats of priors.

\section{Searching the Pool of Candidates}
%
\label{sec:ProblemFormulation}

We formulate the set selection problem, in which a decision-maker selects a set of items from a population, given an ISO.
Here, the candidates for a job represent the items and the screener evaluating their profiles represents the decision-maker.
Let the ISO be the product of an IAS specific to hiring.

\subsection{Setting}
\label{sec:ProblemFormulation.Setting}

Let us consider a \textit{candidate pool} $\mathcal{C}$ of $n$ candidates, where each \textit{candidate} $c$ is described by the \textit{vector of $p$ attributes} $\mathbf{X}_c \in \mathbb{R}^{p}$ and the \textit{protected attribute} $W_c$.
We assume that $W$ is binary, such that $W_c = 1$ if $c$ belongs to the protected group and $W_c = 0$ otherwise; we can relax this assumption if needed.
The candidates are evaluated by a \textit{screener} $h \in \mathcal{H}$, where $\mathcal{H}$ denotes the set of available screeners. 
The following variables refer to a specific $h$. 
The goal of $h$ is to obtain a \textit{set of $k$ selected candidates} $S^k \in [\mathcal{C}]^k$, with $[\mathcal{C}]^k$ denoting the set of $k$-subsets of $\mathcal{C}$, 
% (i.e., subsets with cardinality $k$), 
based on each candidate's application profile as summarized by the tuple $(\mathbf{X}_c, W_c)$.
Candidate evaluation occurs when $h$ uses an \textit{individual scoring function} $s \colon \mathbb{R}^{p} \to [0, 1]$, such that $s(\mathbf{X}_c)$ returns the score of $c$, and $h$ cannot use $W_c$ when scoring $c$. 
The higher the score, the better $c$ fits the job.

The screener $h$ explores the candidate pool $\mathcal{C}$ in a specific order. 
We denote the \textit{set of total orderings of candidates} in $\candidatesset$ by $\Theta$.
An \textit{order} $\sigma \in \Theta$ maps an integer $i \in \{1, \dots n\}$ to a candidate $c \in \candidatesset$, indicating that $c$ occupies the $i$-th position according to $\sigma$, with notation $\sigma(i) = c$ and vice-versa $\sigma^{-1}(c) = i$.
Importantly, the screener explores $\mathcal{C}$ under the ISO $\theta \in \Theta$, which represents the order chosen by or, alternatively, provided to $h$ before searching $\candidatesset$ (recall, G1 in Section~\ref{sec:Generali}) via an IAS. 
The screener is not required to explore the entirety of $\mathcal{C}$, meaning $h$ can either fully or partially explore $\candidatesset$ given $\theta$ (recall, G2 in Section~\ref{sec:Generali}).
We assume that the screener \textit{respects} $\theta$, meaning: 
%
\begin{equation}
\label{eq:order}
    \mbox{$c_1 \in \candidatesset$ is evaluated before $c_2 \in \candidatesset$ only if $\theta^{-1}(c_1) < \theta^{-1}(c_2)$.}
\end{equation}
%

\subsection{Two Problem Formulations}
\label{sec:ProblemFormulation.Objectives}

We formulate two utility-based set selection problems for $h$ with the shared objective of achieving an optimal and fair set $S^k$.
Under the \textit{best-}$k$ formulation, $S^k$ represents \textit{the fair best $k$ candidates} in $\candidatesset$ according to $h$; we denote it as $S^k_{\besttext}$. 
Under the \textit{good-}$k$ formulation, $S^k$ represents \textit{the fair first good-enough $k$ candidates} in $\candidatesset$ according to $h$; we denote it as $S^k_{\goodtext}$.
The key difference between the two is that the best-$k$ requires a full search of $\candidatesset$ while the good-$k$ allows for a partial search of $\candidatesset$ under $\theta$.
% It is the standard formulation in the fair ranking literature.

How we define optimality, as shown in Sections~\ref{sec:best-k} and \ref{sec:good-k}, determines the best-$k$ and good-$k$ problems.
For fairness, we define the \textit{representational quota} $q \in [0, 1]$ as the desired fraction of protected candidates in $S^k$ and use the fraction $f\big( S^k \big) \in [0, 1]$:
%
\begin{equation}
\label{eq:fairness_function}
    f\big( S^k \big) = \frac{\left\vert\{c \in S^k \text{ s.t. } W_c = 1\}\right\vert}{k}
\end{equation}
%
for $h$ to meet $q$ when deriving $S^k$ by satisfying the condition $f\big( S^k \big) \geq q$.
The unconstrained version is achieved by $q=0$.
We view $q$ as a policy enforced by $h$ to achieve a diverse $S^k$ (recall, G4 in Section~\ref{sec:Generali}). 
It is a statement on the composition of $S^k$, not a statement on the ordering of protected candidates within $S^k$.\footnote{For $k=10$ and $q=0.5$, e.g., the fair screener would need to derive $S^k$ with $5$ protected candidates though in no particular order within $S^k$.}

\subsubsection{Best-$k$.}
\label{sec:best-k}

The screener $h$ finds the set of best $k$ candidates in $\candidatesset$ given $q$ while respecting the ISO \eqref{eq:order}.
Here, $h$ needs to evaluate the complete $\candidatesset$ since it must score and rank all candidates according to the individual scoring function $s$ before choosing the ones with the highest scores and that satisfy $q$.

We view the goal in terms of maximizing a utility for $h$. 
We define \textit{utility} as the benefit derived by $h$ from selecting $k$ candidates. 
Formally, utility is a function $U^k \colon [\mathcal{C}]^k \, \times \, \Theta \to \mathbb{R}$. 
The simplest expression for $U^k$ is to add the scores of the selected candidates:
%
\begin{equation}
\label{eq:Utility}
    U^k_{\addtext} \big( S^k, \theta \big) = \sum_{c \in S^{k}} s\big( \mathbf{X}_{c} \big)
\end{equation}
%
rationalizing that $h$ maximizes its utility by selecting the $k$ most suitable candidates given $\theta$. 
Notice that $\theta$ in \eqref{eq:Utility} does not affect the evaluation order of $S^k$ due to the commutative property of addition.
Under \eqref{eq:Utility}, we define  \textbf{the best-\textit{k} problem} as:
%
\begin{equation}
\label{eq:fair_objective_all_screener}
    \begin{aligned}
    \argmax_{S^k \in [\mathcal{C}]^k} & \quad U^k_{\addtext} \big(S^k, \theta\big) \\
    \textrm{s. t.} & \quad f(S^k) \geq q
    \end{aligned}
\end{equation}
%
with its solution as $S^k_{\besttext}$. In the presence of tied scores, $S^k_{\besttext}$ may not be unique. In such a case, we consider any solution.
%
We emphasize that \eqref{eq:Utility} is not the only possible model for the utility of $h$ and alternative models, such as exposure discounting \cite{DBLP:conf/kdd/SinghJ18}, can be considered for \eqref{eq:fair_objective_all_screener}.
We leave this for future work.

%
% Figure environment removed
%

\subsubsection{Good-$k$}
\label{sec:good-k}

The screener $h$ finds $k$ candidates in $\candidatesset$ that meet a set of \textit{minimum basic requirements} $\psi$ (recall, G3 in Section~\ref{sec:Generali}) given $q$ while respecting the ISO \eqref{eq:order}.
We represent $\psi$ as a \textit{minimum score}, such that $h$ deems a candidate $c \in \candidatesset$ as eligible for being selected if $s(\mathbf{X}_c) \geq \psi$.
Unlike the best-$k$ formulation, here $h$ is not required to evaluate the whole $\candidatesset$ as it is enough to find the first $k$ candidates that are good enough according to $\psi$ and that satisfy $q$. 

We still view the goal in terms of maximizing a utility for $h$. We need to, however, define an alternative utility function to \eqref{eq:Utility} that ensures $h$ stops searching $\candidatesset$ after finding the $k$-\textit{th} good-enough candidate according to $\psi$.
We define the following expression:
%
\begin{equation}
\label{eq:AlternativeUtility}
    U^k_{\psi}\big( S^k, \theta \big) = \left\{
    \begin{array}{ll}
        k - \sum_{c \in S^k} p(c, S^k, \theta) & \text{if,} \  \forall c \in S^k, \  s(\mathbf{X}_c) \geq \psi   \\
        0 & \text{otherwise.}
    \end{array} \right.
\end{equation}
%
with the \textit{penalty function} defined as:
%
\begin{equation}
\label{eq:Penalty}
\begin{aligned}
    p(c, S^k, \theta) = & \, \mathbb{1} \big \{ \exists\ c' \in \mathcal{C}\setminus S^k \, \\ & \mbox{s.t.}\ \, \theta^{-1}(c') < \theta^{-1}(c) \wedge s(\mathbf{X}_{c'}) \geq \psi \wedge W_{c'}=W_c \big \}.
\end{aligned}
\end{equation}
%
Under \eqref{eq:AlternativeUtility}, $h$ wants to find as quickly as possible the $k\text{-\textit{th}}$ suitable candidate without wanting to check whether the $(k+1)\text{-\textit{th}}$ candidate is also suitable.
This is because, for a candidate $c$, \eqref{eq:Penalty} looks for another candidate of the same group as $c$ and meeting $\psi$, who occurs before $c$ under $\theta$ but who has not been selected into $S^k$.
It models the ``wasted effort" in choosing a candidate occurring after another one meeting all the same requirements. 
At worst, there are $k$ penalties.
%
Under \eqref{eq:AlternativeUtility}, we define \textbf{the good-\textit{k} problem} as:
%
\begin{equation}
\label{eq:fair_objective_U_psi}
    \begin{aligned}
    \argmax_{S^k \in [\mathcal{C}]^k} & \quad U^k_{\psi} \big(S^k, \theta\big) \\
    \textrm{s. t.} & \quad f(S^k) \geq q
    \end{aligned}
\end{equation}
%
with its solution as $S_{\goodtext}^k(\psi)$ or, if there is no ambiguity on $\psi$, simply as $S_{\goodtext}^k$.
%
When the fairness constraint is strengthened to a fixed quota, $f(S^k) = q$, the solution is unique; in the general case, $f(S^k) \geq q$, there can be two solutions but with different fractions of the protected group.
% See Example~\ref{ex:diff_fractions_prot} in Appendix~\ref{Appendix.NaiveUtilityGoodk} for details.
Similarly to the best-\textit{k} problem, we emphasize that \eqref{eq:AlternativeUtility} is not the only utility model for \eqref{eq:fair_objective_U_psi}; other models are possible as long as they describe the partial search. 
% We leave this for future work.
% For more details, we motivate \eqref{eq:AlternativeUtility} in Appendix~\ref{Appendix.NaiveUtilityGoodk} by presenting a simpler utility model without \eqref{eq:Penalty} and showing its failure to stop $h$.

%
\begin{remark}
\label{remark:ISOandPS}
    $\theta$ influences the screening process under the good-$k$ problem \eqref{eq:fair_objective_U_psi} due to the potential partial search of $\mathcal{C}$ by $h$, affecting which $k$ candidates are selected.
\end{remark}
%
To observe Remark~\ref{remark:ISOandPS}, 
let $k=1$ and assume two candidates such that $s(\mathbf{X}_{c_1}) \geq \psi$ and $s(\mathbf{X}_{c_2}) \geq \psi$. A $\theta$ such that $\theta^{-1}(c_1) = 1$ and $\theta^{-1}(c_2) = 2$ would imply that $c_1$ is considered eligible and is selected before $h$ even evaluates $c_2$. 
Conversely, a reverse $\theta$ such that $\theta^{-1}(c_1) = 2$ and $\theta^{-1}(c_2) = 1$ would imply the opposite.
This remark holds without assuming anything about $h$.

\subsection{Two Search Procedures}
\label{sec:ProblemFormulation.Algorithms}

The \textit{ExaminationSearch} 
% procedure 
(Algorithm~\ref{algo:Examination}) solves the best-$k$ problem, returning $S^k_\besttext$ for given $n$ (candidates) and $\theta$ (ISO), and parameters $k$ (subset size) and $q$ (group fairness constraint).
First, line 2 calculates the minimum number $q^*$ of candidates from the protected group to be selected, and the maximum number of candidates $r^*$ not in that quota. 
Then, candidates are considered by descending scores, using the \texttt{argsortdesc} procedure (lines 2-3). 
The loop in lines 5-13 iterates until $k$ candidates are found. 
The loop adds candidates to the sets $Q$ and $R$: $Q$ are candidates in the quota of the protected group; $R$ are candidates not in that quota that can be non-protected or protected. 
A non-protected candidate can be only added to the $R$ set; thus, line 7 checks if there is still room in $R$ to do this. A protected candidate is added to the quota set $Q$ if there is room (lines 10-11) or to the other set $R$ otherwise (lines 12-13). 
Finally, the procedure returns the candidates in the quota set $Q$ or in the other set $R$. 
The result of the \textit{ExaminationSearch} procedure maximizes (\ref{eq:fair_objective_all_screener}), as candidates are added in decreasing score, while keeping the fairness constraint through the quota management.

The \textit{CascadeSearch} 
% procedure 
(Algorithm~\ref{algo:Cascade}) solves the good-$k$ setting, returning $S^k_\goodtext$ for given $n$ and $\theta$, and parameters $k$, $q$ and $\psi$ (min. basic requirement).
The difference with the \textit{ExaminationSearch} procedure consists in strictly following $\theta$ (line~4) and checking $\psi$ (line~8) before adding a candidate to the quota set $Q$ or to the other set $R$.
The result of the \textit{CascadeSearch} procedure maximizes (\ref{eq:fair_objective_U_psi}), as no penalty is accumulated in the loop. 
This is because, a non-protected candidate ($W_c=0$) is not added only if there is no room in $R$ (and $R$ never gets smaller to allow for more room later on), while a protected candidate ($W_c=1$) is not added only if it does not meet $\psi$ in line 8 and, thus it cannot be counted for the penalty. 

Our aim here is not to provide novel optimal algorithms, but to study the screener's search behavior of $\candidatesset$ under $\theta$.
Hence, we move away from an optimality analysis of the two algorithms (e.g., \cite{fagin2001optimal}), and focus on modeling the search behavior of $h$ when solving the two problems.
Both algorithms are inherently sequential and can be applied online because we aim to model a human-like screener (next section) that operates sequentially.
Yet, in both algorithms, the applicants are disclosed according to $\theta$ and not to the score, differently from past set selection works (e.g., \cite{stoyanovich2018online}).

%
% EOS
%


\section{The Human-Like Screener}
%
\label{sec:HumanScreener}

To study the human interaction 
% under these two problems 
with the \textit{initial screening order} (ISO), we distinguish two kinds of screeners $h$ based on the proneness to error when evaluating the candidate pool: 
$h$ is an \textit{algorithmic screener}, denoted by $h_a \in \mathcal{H}_a$, if it can consistently evaluate $\mathcal{C}$; 
% The algorithmic screener is the implied screener in the fair ranking literature. 
whereas $h$ is a \textit{human-like screener}, denoted by $h_h \in \mathcal{H}_h$, if its fatigue hinders the consistency of its evaluation of $\mathcal{C}$.

\subsection{Fatigue and Fatigued Scores}
\label{sec:HumanScreener.BiasedScores}

% Time
We first introduce a \textit{time component} to study these two screeners.
Let $t$ denote the discrete unit of time that represents how long $h$ takes to evaluate a candidate $c \in \mathcal{C}$. 
We assume that $t$ is constant (recall G5 Section~\ref{sec:Generali}), implying that time itself cannot be optimized by $h$.
We track time along $\theta$, meaning $h$ evaluates the first candidate that appears in $\theta$ at time $t=1$, and so on. 
Time $t$, thus, ranges from $0$ to $n$ at maximum.
% Fatigue
We then introduce a \textit{fatigue component} $\phi(t)$ specific to $h_h$ as a function of $t$ and model the \textit{accumulated fatigue} $\Phi \colon \{0, \ldots, n\} \to \mathbb{R}$, with $\Phi(0) = 0$. 
The discrete derivative of $\Phi$, that is, $\phi(i) = \Phi(i) - \Phi(i-1)$, defined for $t \geq 1$, is the effort of $h_h$ to examine the $t$-th candidate. 
%SR
% $c=\theta^{-1}(i)$.
How we define $\Phi$ conditions the effect of fatigue on our analysis of $h_h$.
We make the simplest modeling choice for $\phi$ by assuming that \textit{fatigue accumulates linearly over time}, or $\phi(t) = \lambda$ so that $\Phi(t) = \lambda \cdot t$,
% This $\Phi$ is one possible formulation based on our interpretation that 
meaning $h_h$ becomes tired over time at a constant pace.
% We leave the study of other $\Phi$ formulations for future work.

How does fatigue materialize for $h_h$?
We model the effect of fatigue on $h_h$ through the \textit{fatigued score}:
%
\begin{equation}
\label{eq:BiasedScoresForHh}
    s_{h_h}(\mathbf{X}_c) + \epsilon
\end{equation}
%
% $s_{h_h}(\mathbf{X}_c) + \epsilon$, 
where $\epsilon$ is a random variable dependent on $\Phi$ that quantifies the deviation from the \textit{truthful score} $s_h(\mathbf{X}_c)$.
We model $\epsilon$ using \emph{two modeling choices} at a given $t$.
% %
% \begin{itemize}
%     \item 
    \textit{\textbf{First modeling choice}:}
    $\epsilon_1$ is a centered Gaussian, and the fatigue affects only its variance. 
    Formally, $\epsilon_1 \sim \mathcal{N}(0, \, v(\Phi(t-1)))$, where $v \colon \mathbb{R} \to \mathbb{R}$ defines the variance of $\epsilon_1$ as an increasing function of $\Phi$.
    % %
    % \item 
    \textit{\textbf{Second modeling choice}:}
    $\epsilon_2$ as an uncentered Gaussian, whose mean is a decreasing function of the fatigue.
    % The second choice assumes that a negative bias can affect each applicant's evaluation.
    % Hence, we model $\epsilon_2$ as an uncentered Gaussian, whose mean is a decreasing function of the fatigue.
    Formally, $\epsilon_2 \sim \mathcal{N}(\mu(\Phi(t-1), \, v(\Phi(t-1))$, where $\mu \colon \mathbb{R} \to \mathbb{R}$ is a decreasing function rather than a constant of $\Phi$.
    % The more fatigue, the more the screener $h_h$ tends to underscore the candidates.
% \end{itemize}
% %

Intuitively, under $\epsilon_1$, $h_h$ tends to overscore or underscore candidates over time, introducing both negative and positive bias (i.e., fatigue as ``less attention'' when evaluating more candidates) over time; under $\epsilon_2$, instead, $h_h$ tends to underscore the candidates (i.e., fatigue as ``less effort'' when evaluating more candidates) over time, introducing always a negative bias. 
With both $\epsilon_1$ and $\epsilon_2$, we capture two realistic biased settings driven by the ISO $\theta$.
We assume that $h_h$ is unaware of its fatigue, representing an unconscious bias due to, e.g., performing a repetitive tasks over time \cite{Kahneman2011Thinking, Kahneman2021Noise}.

%
\begin{remark}
\label{remark:HumanAndIF}
    $\Phi$ implies that $h_h$ evaluates identical candidates $c_1$ and $c_2$ differently under $\theta$ at $t_1$ and $t_2$, as long as $\Phi(t_1) \neq \Phi(t_2)$ and regardless of whether $h_h$ is solving for the best-$k$ or good-$k$ problem.
\end{remark}
%

Algorithms~\ref{algo:Examination} and \ref{algo:Cascade} represent $h_a$ as there is no notion of biased scores.
To represent the human-like screener $h_h$, we must track $\Phi$ over time and draw $\epsilon_1$ (or $\epsilon_2$) to compute the fatigued scores \eqref{eq:BiasedScoresForHh} of $h_h$ at time $t$. 
The only changes are to line 2 in Algorithm~\ref{algo:Examination} and line 7 in Algorithm~\ref{algo:Cascade}, where the score computed for candidate $c$ is biased by $\epsilon_1$ (or $\epsilon_2$). 
We present the human-like versions of the search procedures in Appendix~\ref{Appendix.HumanAlgorithms} as Algorithms~\ref{algo:HumanExamination} and \ref{algo:HumanCascade};
%
though these two can be observed using Figure~\ref{fig:TheAlgos}.

\subsection{Position Bias Implications}
\label{sec:PositionBias}

No two candidates occupy the same position in the ISO $\theta$. With this fact in mind, we now analyse the fairness and optimality implications of the position bias implicit to $\theta$.
For concreteness, we make \textit{two assumptions}.
%
\textit{\textbf{A1}: We assume that $\theta$ is independent of the protected attribute $W$}, meaning that how candidates appear in $\theta$ contains no information about $W$.
%
\textit{\textbf{A2}: We assume that the individual scoring function $s$ is able to evaluate any candidate $c$ fairly and truthfully}, meaning $s(\mathbf{X}_c)$ captures no information about $W_c$ and only information about the suitability of $c$.
%
Under \textit{A1} and \textit{A2}, we can control for other biases, such as measurement error in $s$, and focus on the position bias coming from $\theta$.

We start with the fairness implications for both best-$k$ and good-$k$ problems. 
Given $\theta$, it is important to distinguish between the group fairness constraining $h$ (i.e., the quota $q$) and the individual fairness violation when $h$ fails to evaluate similar candidates similarly \cite{DBLP:conf/innovations/DworkHPRZ12}.
Regarding group-level fairness, both $h_a$ and $h_h$ are fair in solving for \eqref{eq:fair_objective_all_screener} and \eqref{eq:AlternativeUtility} by satisfying $f\big(S^k\big) \geq q$.
This point is clear for $h_a$ in Algorithms \ref{algo:Examination} and \ref{algo:Cascade} as there is no fatigue involved.
The same holds for $h_h$ in Algorithms \ref{algo:HumanExamination} and \ref{algo:HumanCascade} 
because the error on the score does not affect the evaluation of $q$.
Here, we have that the expected error $\mathbb{E}[\epsilon \mid W_c = 1, \theta] = \mathbb{E}[\epsilon \mid W_c = 0, \theta]$, regardless of $\epsilon_1$ or $\epsilon_2$ for $h_h$. 
The fatigue and, thus, the fatigued scores are, on average, shared across protected and non-protected candidates.

Distinguishing between $h_a$ and $h_h$ becomes important under individual-level fairness because $h_a$ ensures it, while $h_h$ violates individual fairness.
A candidate's position in $\theta$ influences the amount of error made by $h_h$ when evaluating that candidate. 
Similar candidates will not be evaluated similarly due to the unequal accumulation of fatigue experienced by $h_h$ when searching $\theta$.
% For instance, given two similar candidate $c_i = \theta^{-1}(i), c_j = \theta^{-1}(j)$, with $i < j$, their evaluation could be significantly different in the amount of error depending on the accumulated fatigue $\Phi(i) < \Phi(j)$. 
% In the case of $\epsilon_1$, $i$ has the advantage of being evaluated by a rested screener.
For $\epsilon_2$, e.g., even if $\mathbf{X}_{c_j} = \mathbf{X}_{c_i}$ but $j>i$, the score of $j$ is less, on average, than the one of $i$, and $i$ has an unfair premium over $j$ from $\theta$.

We now consider the optimality implications for both best-$k$ and good-$k$ problems. 
Recall that each problem, due to its own utility model, has different optimal solutions.
It follows that $h_a$ reaches the optimal solution in both problems as the absence of fatigue enables $h_a$ to consistently judge suitable candidates.
The opposite holds for $h_h$ due to the inconsistent scoring of candidates ascribed by the accumulated fatigue.
The biased scores not only violate individual fairness, but also lead $h_h$ to misjudge candidates, eventually choosing the wrong ones when searching $\theta$.

To summarize, \textit{$h_a$ reaches the optimal and fair solution for both best-$k$ and good-$k$ problems}.
Moreover, \textit{$h_a$ guarantees individual fairness in both problems}.
Instead, \textit{$h_h$ reaches the fair but sub-optimal solutions for both best-$k$ and good-$k$ problems}.
Moreover, \textit{$h_h$ does not guarantee individual fairness in both settings}. 
Significantly, under the $h_h$, the position of a candidate in $\theta$ matters. 
It impacts whether a candidate, depending on the search strategy, is evaluated or not (Remark~\ref{remark:ISOandPS}) and, if so, is evaluated fairly or not (Remark~\ref{remark:HumanAndIF}).
These results only worsen when relaxing \textit{A1} and \textit{A2}.
We explore these insights further by considering multiple hiring settings for $h_a$ and $h_h$ in the next section.

%
% Figure environment removed
%
%%%%%%%%%%%%%%%%%%%%%%%%%%%%%%
%%%%%%%%%%%%%%%%%%%%%%%%%%%%%%
%%% Moved to the Appendix
% %
% % Figure environment removed
% %\vspace{-3ex}
% %
%%%%%%%%%%%%%%%%%%%%%%%%%%%%%%
%%%%%%%%%%%%%%%%%%%%%%%%%%%%%%

%
% EOS
%


% \section{Experiments}
\section{An Experimental Framework}
%
\label{sec:Experiments}

We now explore the implications of the best-$k$ and good-$k$ problems through Monte Carlo experiments for both the algorithmic $h_a$ and human-like $h_h$ screeners.
Moreover, this section shows how our framework can explore different screening scenarios involving the initial screening order (ISO).
The algorithms~\ref{algo:Examination}--\ref{algo:HumanCascade} and simulation procedures are developed in R \cite{Rlang}.\footnote{See the GitHub repository for the code; \url{https://github.com/cc-jalvarez/initial-screening-order-problem/tree/main}.}
%
% The source code is provided in an anonymous repository: \url{https://anonymous.4open.science/r/initial-screening-order-problem-D078}.

\subsection{Experimental Setup}
\label{sec:Experiments.Setup}

% We consider one possible setup; naturally, these inputs can be changed to account for other screening settings.

\subsubsection{Generating the sample.}
We assume a sample consisting of $n$ triplets $\{ (\theta({c_i}), s(\mathbf{X}_{c_i}), W_{c_i}) \}_{i=1}^n$ 
% for $i = 1, \ldots, n$, 
drawn from a probability distribution with domain 
$\mathcal{G}_n \times \mathbb{R}^n \times \{0, 1\}^n$ where $\mathcal{G}_n$ is the set of all permutations of $\{1, \ldots, n\}$. 
Each sample represents a specific candidate pool $\candidatesset$ sorted according to an ISO $\theta$.
 
Regarding $s(\mathbf{X}_{c_i})$, we consider three possible distributions of candidate scores. 
All are based on the truncated normal distribution family $tN(\mu, \sigma)$ \cite{Botev2017} with values bounded in the interval $[0, 1]$.
Here, we wish to model scenarios in which very good candidates, as in those with top scores, occur with different probabilities.
All three scenarios are shown in Figure~\ref{fig:1} (left).
% %
% \begin{itemize}
    % \item 
    \textit{A symmetric distribution of scores} (in red) defined by $tN(0.5, 0.02)$ with mean/median of $0.5$ implies that top candidates occur with a very low probability. 
    % Hence, setting a large minimum basic requirement $\psi$ is highly selective of top candidates.
    %
    % \item 
    \textit{An asymmetric distribution of scores} (in blue) defined by $tN(0.8, 0.05)$ implies that top candidates occur with a higher probability and median value ($\approx 0.75$) compared to the previous scenario. 
    % Hence, setting a large $\psi$ is less selective of top candidates.
    %
    % \item 
    \textit{An increasing distribution of scores} (in green) defined by $tN(1, 0.05)$ implies that top candidates occur with an even higher probability and median value ($\approx 0.85$). 
    % In this scenario, there are many good candidates, and thus $\psi$ is not selective.
% \end{itemize}
% %

The score scenarios have implications, in particular, for the good-k problem where we must set the minimum score $\psi$ and the screener is not required to explore all of $\candidatesset$ under $\theta$ (recall Remark~\ref{remark:ISOandPS}).
Setting a large $\psi$ makes the screening process highly selective in the first setting, less selective in the second setting, and not selective at all in the third setting.
These settings represent a range of candidate pools with different candidate quality on average.
% For instance, in the ideal setting of having to choose from mostly top candidates (the third setting), the screener is not worried of choosing a high $\psi$.

Regarding $\theta$, we consider two possible variants for the ISO.
%
First: \textit{the ISO is randomly and independently generated from the candidate scores.} 
This setting models the case in which $\theta$ brings no information regarding candidate quality, and the screener prefers the alphabetical order or performs a random shuffle of candidates. 
We denote such a scenario as $\theta \ci s$.
%
Second: \textit{The ISO is randomly generated generated with a given Spearman's rank correlation $\rho$ with the candidate scores.}
Formally, $\rho$ is the Spearman's rank correlation of the pairs $\{ (\theta(c_i), s(\mathbf{X}_{c_i})) \}_{i=1}^k$. 
% for $i=1, \ldots, n$. 
It assesses how well $\theta$ monotonically relates to the scores of candidates.\footnote{To generate correlated initial ordering and scores, we rely on copulas -- see e.g., \cite[Section 3.4]{EMBRECHTS2003329}.}
We denote such scenario as $\theta \not\!\perp\!\!\!\perp s$.
In particular, for $\rho=-1$, it means that $\theta$ ranks candidates by descending scores similar to giving a ranked list of candidates to the screener.
%
With these two variants, we can consider a $\theta$ chosen by or provided to the screener, respectively.
% %
% \begin{itemize}
%     \item  $\theta$ is randomly generated independently from the scores. This setting models the case where the screener prefers the alphabetical order, or performs a random shuffle of candidates. We denote such a scenario as $\theta \ci s$.
% %
%     \item $\theta$ is randomly generated with a given Spearman's rank correlation $\rho$ with the scores.\footnote{Formally, $\rho$ is the Spearman's rank correlation of the pairs $(\theta(c_i), s(\mathbf{X}_{c_i})$ for $i=1, \ldots, n$. It assesses how well the initial order $\theta$ monotonically relates to the scores of candidates. To generate correlated initial ordering and scores, we rely on copulas -- see e.g., \cite[Section 3.4]{EMBRECHTS2003329}.} 
%     In particular, for $\rho=-1$, it means that $\theta$ ranks candidates by descending scores. %Correlation close to $-1$ models scenarios where the initial order is the result of some pre-evaluation of the candidates, including automatic ranking by ML models or manual ordering based on job-required skills. 
% \end{itemize}
% %

Regarding $W$, we initially consider 
% one variant for the protected attribute where 
the sample of candidates drawn from $Ber(\mathit{pr})$ such that $\mathit{pr}=0.2$ is the fraction of protected group candidates in the population.
The sample is independently drawn from both the scores and the ISO, according to the assumptions \textit{A1} and \textit{A2} from Section~\ref{sec:PositionBias}.
Here, we are determining the diversity of $\candidatesset$.
Later on, we increase $\mathit{pr}$ to study a more diverse $\candidatesset$ and its effect on the screener reaching the representational quota $q$.
% By setting a low $\mathit{pr}$, we picture a $\candidatesset$ where the protected individuals are under-presented, forcing the screener to search it longer to meet the representational quota $q$.
%For the former, it means that scores are not unfairly assigned by the screener. 
%For the the latter, it means that the initial order provided by the screener or by a ML ranking model is fair. Clearly, these two cases are also interesting to experiment with, but this is out of the scope of the paper.

Beyond the triplet $\big( \theta({c_i}), s(\mathbf{X}_{c_i}), W_{c_i} \big)$, 
% Finally, 
for the fatigued scores of $h_h$, we fix $\lambda=1$, hence $\Phi(t) = t$, and define:
% %
% \begin{itemize}
    % \item 
    $\epsilon_1 \sim \mathcal{N}(0, \, (0.005 \cdot (t-1))^2)$, hence with constant expectation and with standard deviation of $0.005 \cdot (t-1)$; and
    %
    % \item 
    $\epsilon_2 \sim \mathcal{N}(-0.005 \cdot (t-1), \, (0.001 \cdot (t-1))^2)$, hence with a decreasing expectation and with a smaller standard deviation than $\epsilon_1$.
% \end{itemize}
% %
Recall Section~\ref{sec:HumanScreener.BiasedScores} for details.

\subsubsection{Evaluation metrics.}
% For concreteness, 
We consider the solution $S^k_{\besttext}$ of the best-$k$ problem (\ref{eq:fair_objective_all_screener}) for $h_a$ (Algorithm~\ref{algo:Examination}) as the baseline solution. 
We compare it with the analysis of the solutions for the good-$k$ problem (\ref{eq:fair_objective_U_psi}) under $h_a$ (Algorithm~\ref{algo:Cascade}) and of the solutions for the best-$k$ and good-$k$ problems under $h_h$ (respectively, Algorithms \ref{algo:HumanExamination} and \ref{algo:HumanCascade}).
We introduce two comparison metrics that capture how close is the compared solution to the baseline solution.
   
The \textbf{\textit{ratio to baseline (RtB)}} is defined as the ratio of $U^k_{\addtext}$ between the compared solution and the baseline solution. 
For the solution $S_{\goodtext}^k$ under $h_a$, e.g., it is $U^k_{\addtext}(S_{\goodtext}^k) / U^k_{\addtext}(S^k_{\besttext})$. 
The closer the ratio is to $1$, the better the compared solution approximates the best-$k$ solution under $h_a$ in terms of $U^k_{\addtext}$ utility.
Here, when calculating the utility of $h_h$, we use the truthful scores, not the fatigued scores, to be able to compare w.r.t. the baseline.
%
The \textbf{\textit{Jaccard similarity (JdS)}} is defined as the proportion of candidates in both the compared and baseline solutions over those in at least one of the two. 
For the solution $S_{\goodtext}^k$ under $h_a$, e.g., it is $|S_{\goodtext}^k \cap S^k_{\besttext}|\, / \, |S_{\goodtext}^k \cup S^k_{\besttext}|$. 
Such a metric quantifies the share of candidates between the two solutions.
Essentially, the RtB metric captures whether the compared solution achieves the same utility as the baseline solution as measured by $U^k_{\addtext}$, while the JdS metric captures the overlap in candidates between the compared solution and the baseline solution.

\subsubsection{Simulations.}
For each setting of the parameters ($n$, $k$, $q$, $\rho$, $\psi$ or others), we run 10,000 times the experiments by randomly generating $n$ triplets at each run. 
The runs for which a solution of the problem does not exist are discarded. This
mainly occurs in the good-$k$ problem when there are not enough $k$ candidates with scores greater or equal than $\psi$.
% \footnote{This mainly occurs in the best-$k$ problem when there are not enough candidates with scores greater or equal than $\psi$.}
The plots report the mean output based on the evaluation metrics over all the runs.

\subsection{Experiments without Fatigue}
\label{sec:Experiments.Metrics.outFatigue}

%
% Figure environment removed
%
% %
% % Figure environment removed
% %\vspace{-3ex}
% %

We start by exploring the settings without fatigue, meaning we consider $h_a$, which allows for clarifying the relation between the best-$k$ (Algorithm~\ref{algo:Examination}) and good-$k$ (Algorithm~\ref{algo:Cascade}) solutions.
Given these two problem formulations, here we are mainly interested if their solutions differ in practice due to the ISO, especially since the best-$k$ requires a full search of $\candidatesset$ while the good-$k$ allows for a partial search of $\candidatesset$.

% First, w
We study the impact of the score distributions on the metrics at the variation of $\psi$.
We consider the case of $n=120$, $k=6$, $q=0.5$, and $\theta \ci s$. 
Note that these parameters are shared by both best-$k$ and good-$k$ problems. 
Instead, $\psi$ is specific to the good-$k$, which is why we focus on it.
We find that, as $\psi$ increases and screening becomes more selective, the good-$k$ approximates the best-$k$ when there is a low probability of having good candidates in $\candidatesset$.
%
Figure~\ref{fig:1}, under the RtB (center) and the JdS (right) metrics, illustrates this point for the three score distribution scenarios.
In particular, the symmetric distribution (in red) allows the good-$k$ to better approximate the best-$k$ for medium-to-high $\psi$ values.
This result is expected. 
Having few good candidates forces $h_a$ to explore more of $\candidatesset$ under $\theta$, especially as $\psi$ increases and the $k$ first good-enough candidates essentially become the $k$ top candidates.

The opposite holds for the other two distributions (in green and blue), which are more resilient to $\psi$ as each represents a higher concentration of good candidates.
It follows that having many good candidates makes it difficult for $h_a$ to select the $k$ top candidates under a partial search.
As the RtB and JdS metrics show in Figure~\ref{fig:1} (center, right), $h_a$ still achieves significant utilities under the other two distributions but is unlikely to derive the same selected set of candidates under a partial search w.r.t. a full search of $\candidatesset$.
These results are also expected, but worth emphasizing. 
Having many good candidates means that $h_a$ can still partially search $\candidatesset$ despite having a highly selective $\psi$: i.e., $h_a$ finds $k$ good-enough candidates with high-enough scores as the $k$ top candidates but not the same ones.

Figure~\ref{fig:1} illustrates how the two problems materialize differently when implemented due to the ISO $\theta$.
Clearly, as noted back in Remark~\ref{remark:ISOandPS}, where the $k$ top candidates appear in $\theta$ can determine if they are selected or not by $h_a$ under a partial search.
The position bias in the ISO becomes more prevalent under many good candidates as, e.g., even the $\candidatesset$'s best candidate may never be selected by $h_a$ under a partial search if it lies at the bottom of $\theta$.

We also study what occurs when $\theta \not\!\perp\!\!\!\perp s$. 
We present these results in Appendix~\ref{Appendix:Experiments.Metrics.outFatigue} for  $n=120$, $k=6$, and $q=0.5$.
Here, we briefly discuss these results as they further illustrate the role of $\theta$.
Recall that $\rho$ represents the correlation between $\theta$ and the scores, with a negative $\rho$ implying a descending order.
We find that the good-$k$ solution approximates quite well the best-$k$ solution already for $\rho=-0.5$; for $\rho=-1$, the two solutions are the same.
These results are expected as $\theta$ essentially represents the best-$k$ solution or an approximation of it depending on $\rho$'s strength. 
Under $\rho=-1$, e.g., the $\theta$ searched by $h_a$ is already sorted by the candidate scores and, in turn, the $k$ first good-enough candidates are also the $k$ best candidates in the candidate pool. 
See Figure~\ref{fig:2} (center, right) for details.

In Appendix~\ref{Appendix:Experiments.Metrics.outFatigue} we also study the impact of the number $n$ candidates in $\candidatesset$ and the number of $k$ candidates to be selected.
We find that under $\theta \ci s$, the ratio $k/n$ is positively correlated with the ability of best-$k$ to approximate good-$k$. Intuitively and unsurprisingly, it means that the more candidates we can select from $\candidatesset$, the better the chance to include top ones under a partial search.
Clearly, the influence of $\theta$ diminishes as $k/n$ increases.
See Figure~\ref{fig:2} (left) for details.
Similarly, we study the role of changing the representational quota $q$. Here, results are expected given our setup and underlying assumptions (\textit{A1} and \textit{A2} from Section~\ref{sec:PositionBias}), finding that under $\theta \ci s$, $q$ does not affect the relative strengths of best-$k$ and good-$k$ solutions.
See Figure~\ref{fig:3} (all) for details.

%%%
%%% Moved to Appendix
% Second, we consider the impact of the number $n$ candidates in $\candidatesset$ and the number of $k$ candidates to be selected. 
% We focus only on the symmetric distribution and the ratio to baseline, but the results are similar for the other two distributions and the Jaccard similarity. 
% Figure~\ref{fig:2} (left) compares the case $n=120, k=6$ considered earlier to two other scenarios. 
% The first scenario increases $k=20$ but leaves the ratio of selected $k/n = 0.05$ the same by also increasing $n=400$. 
% The second scenario, instead, leaves $k=6$ the same, but it increases $k/n = 0.2$ by decreasing $n=30$. 
% The plot shows that changes in the ratio $k/n$ affect the metric, in particular a larger ratio ($n=30$, $k=6$) leads good-$k$ to better approximate best-$k$ for a same $\psi$. The more candidates we can select from the pool, the better the chance to include top ones. 
% \textit{In summary, under $\theta \ci s$, the ratio $k/n$ is positively correlated with the ability of best-$k$ to approximate good-$k$}.

% Third, we now consider the impact of the correlation $\rho$ between the initial order $\theta$ and the scores. 
% Recall that $\rho = -1$ means that the candidates are ordered by descending scores. Under such a condition, the good-$k$ and best-$k$ procedures return the same solution. 
% This result is apparent in Figure~\ref{fig:2} (center, right) where we report the ratio to baseline for the symmetric (left) and the increasing (right) score distributions. 
% The plots show that even a moderate correlation of $\rho = -0.5$ leads the good-$k$ solution to approximate the best-$k$ one quite well. 
% For the increasing distribution (right plot), the ratio to baseline is around 95\%. \textit{In summary, initial orders that negatively correlate to the score greatly reduce the difference in utility between the good-$k$ and best-$k$ solutions}.

% %
% % Figure environment removed
% %\vspace{-3ex}
% %

% Let us now consider the quota parameter $q$, thus far set to $q=0.5$ over a population with a fraction of protected candidates set to $\mathit{pr}=0.2$. 
% Since we assumed that $W$ is independent from both scores and the ISO, the fraction of protected group in the solutions of best-$k$ and good-$k$ is, on average, $\mathit{min}\{q, \mathit{pr}\}$. 
% Figure~\ref{fig:3} (left) shows this result in the solution for good-$k$. 
% A less trivial question is whether $q$ is also not affecting the evaluation metrics: e.g., whether the quota $q$ changes the ratio to baseline? 
% Figure~\ref{fig:3} (center, right) show that this is not the case in two experimental settings. 
% Again, this result is theoretically implied by the independence of $W$ with scores and initial order. 
% In summary, under $\theta \ci s$, the quota parameter in the best-$k$ and good-$k$ problem does not affect the relative strengths of their solutions.
%%%
%%%

\subsection{Experiments with Fatigue}
\label{sec:Experiments.Metrics.withFatigue}

% We now consider the settings with fatigue, meaning we consider $h_h$.
We now focus on $h_h$.
First, we consider whether fatigue impacts utility w.r.t.~the baseline solution, namely the solution of best-$k$ without fatigue (Algorithm~\ref{algo:Examination}). 
We compare to such a baseline both the best-$k$ with fatigue (Algorithm.~\ref{algo:HumanExamination}) and good-$k$ with fatigue (Algorithm.~\ref{algo:HumanCascade}).
%See Appendix~\ref{Appendix.HumanAlgorithms} for implementation details on both of these algorithms.
%In the latter case, we readily extend the metric of ratio to baseline as the ratio of the utility for the solution of best-$k$ with fatigue over the one of best-$k$ without fatigue. 

%
% Figure environment removed
%
%
% Figure environment removed
% \vspace{-3ex}
%

Figure~\ref{fig:4} (left) shows the RtB metric for the three score distributions for the good-$k$ 
% (i.e., $h_h$ can perform a partial search) 
solution with fatigue based on $\epsilon_1$ for the fatigued scores. 
Based on Figure~\ref{fig:1} (center), for the asymmetric (in blue) and increasing (in green) distributions, 
which are the scenarios with many good candidates in $\candidatesset$,
there is not much difference w.r.t.~the case without fatigue.
For the symmetric distribution (in red), instead, there is a considerable decrease for high $\psi$ values. 
This can be attributed to the low number of top scores, for which the perturbation due the $\epsilon_1$'s has a large effect. 
For the other two distributions, instead, there are sufficiently many top scores, for which perturbation does not dramatically change the score distribution for top scores.
Intuitively, since the RtB metric captures achieving the utility of the baseline model, under a partial search $h_h$ is able to reach high utility solutions when $\candidatesset$ has many good candidates because $h_h$ can avoid evaluating all candidates.
This is unlikely to be the case when few good candidates are in $\candidatesset$. 
As $\psi$ increases and $h_h$ becomes more selective, it also becomes more tired under $\theta$ as it needs to evaluate more and more candidates to achieve $k$.

Figure~\ref{fig:4} (center) is analogous to (left), but considers the fatigued scores based on $\epsilon_2$. 
The effect for the symmetric distribution (in red) is not present in such a case, due to the lower standard deviation of $\epsilon_2$. 
The bias of $\epsilon_2$ does not impact too much, apart from high values of $\psi$ where it causes the problem not to have a solution as fatigued scores are smaller than scores of an already low number of top candidates. 
In summary, under $\theta \ci s$, variance appears more relevant than bias in the case of low probability of top scores.
This result illustrates the importance of how we define fatigue. It can also inform how the a human screener should behave in practice to diminish the role of position bias within $\theta$. Given these results, e.g., we would be interested in exploring under what settings would the human screener experience $\epsilon_1$ over $\epsilon_2$.

Figure~\ref{fig:4} (right) shows the RtB for the best-$k$ solution with fatigue at the variation of the quota $q$. There is a considerable and constant loss in utility under fatigue, which is more consistent for the symmetric distribution (in red). 
Interestingly enough, the RtB is lower than in the case of the good-$k$ with fatigue (see left) for $\psi \geq 0.5$. 
This result means that, for the symmetric distribution, the good-$k$ solution with fatigue has better utility than the best-$k$ solution with fatigue. 
This noteworthy result can inform the screening practice.

Finally, we consider the impact of the correlation $\rho$ on the ISO $\theta$ for the good-$k$ solution with fatigue.  
Figure~\ref{fig:5} (left) considers the symmetric distribution (in red). 
For the lower half of $\psi$'s, lines are similar to the analogous case without fatigue shown in Figure~\ref{fig:2} (center). 
For the higher half of $\psi$'s, instead, there is a decrease in the metric. 
This is, again, due to the low probability of top scores, for which the effects of the variability of $\epsilon_1$ is not counter-balanced by correlation of the scores and $\theta$.
Such an effect does not appear for $\epsilon_2$ nor for $\epsilon_1$ with the increasing distribution. 
In fact, the plots in Figure~\ref{fig:5} (center) and (right) closely resemble those in Figure~\ref{fig:2} (center) and (right) respectively. 
This is an interesting result on its own. 
It means that, in the presence of correlation, fatigue does not have an impact on utility of the good-$k$ solution if there are sufficiently many top scores or a sufficiently small variability of the fatigue.

Moreover, we believe this last result points at the importance in practice of providing a $\theta$ to the human screener that has some information about candidate quality. 
Intuitively, under a partial search procedure and the threat of position bias materializing through $\theta$, we would like to decrease $h_h$'s fatigue by minimizing its need to search more of $\candidatesset$.
A way to do is to already provide to $h_h$ a sorted $\theta$. 
Note that this point excludes the difficulty behind deriving such a sorted $\theta$ in the first place, which is the main goal of the fair set selection literature. 
This point, however, hints at an interesting line of future work focused on human screeners and their interactions with an algorithmic aid.

%
% EOS
%

% \subsection{Evaluation Metrics}
% \label{sec:Experiments.Metrics}

% The solution $S^k_{\besttext}$ of the fair best-$k$ problem (\ref{eq:fair_objective_all_screener}) for the algorithmic screener represents a baseline to compare with in the analysis of the solutions of the fair good-$k$ problem (\ref{eq:fair_objective_U_psi}) for the algorithmic screener, and of the solution of both fair best-$k$ and fair good-$k$ for the human-like screener. 
% We introduce two comparison metrics.
% %
% %\begin{itemize}
%     %\item 
    
%     The \textit{ratio to baseline} is defined as the ratio of $U^k_{\addtext}$-utility between the compared solution and the baseline solution. E.g.,~for the solution $S_{\goodtext}^k$ of the algorithmic screener, it is $U^k_{\addtext}(S_{\goodtext}^k) / U^k_{\addtext}(S^k_{\besttext})$. 
%     The closer the ratio is to $1$, the better the compared solution approximates the best-$k$ solution of the algorithmic screener in terms of $U^k_{\addtext}$ utility. 
    
%     %\item 
    
%     The \textit{Jaccard similarity} is defined as the proportion of candidates in both the compared and baseline solutions over those in at least one of the two. E.g., for the solution $S_{\goodtext}^k$ of the algorithmic screener, it is $|S_{\goodtext}^k \cap S^k_{\besttext}|/|S_{\goodtext}^k \cup S^k_{\besttext}|$. 
%     Such a metric quantifies the share of candidates between the two solutions. 
    %The closer the metric is to $1$, the more equivalent is the set of selected candidates between the problems.
%\end{itemize}
%
% SR
%We will experiment with the values of these two metrics at the variation of $\psi$ and other parameter settings.


% \section{Discussion}
\section{Conclusion}
%
\label{sec:Discussion}

% Summary
In this work,
we presented the initial screening order (ISO) as a parameter of interest; 
defined two formulations under distinct utility models of the fair set selection problem, the best-$k$ and the good-$k$, with their corresponding algorithmic implementations; 
and introduced a human-like screener to study the effects of the ISO as though we were studying a human user in a candidate screening problem.
We also provided a simulation framework flexible enough to account and model for other screening scenarios.
Our analysis confirms the fairness and optimality impact of the ISO, motivated by the risk of position bias, on the set of $k$ candidates chosen by an algorithmic or human-like screener. 
Extensive simulations showed complex relations between best-$k$ and good-$k$, which heavily depend on the score distribution, on their correlation with the ISO, and on the fatigue modeling.

% Before discussing the implications of our work, 
We once again emphasize that the assumptions made--both in terms of functional forms and parameter choice--condition our analysis, in particular, our experimental results.
Some of these choices are probably less contested than others.
For instance, 
the utility models chosen in Section~\ref{sec:ProblemFormulation} for defining the best-$k$ and good-$k$ problems are intuitive and reflective of the screener's evaluation criteria in each setting and, overall, recall common modeling practices within the literature (e.g., \cite{Zehlike2023_FairRanking_P1, Zehlike2023_FairRanking_P2, stoyanovich2018online}).
Further, both problem formulations are not restricted to these specific utility models and future work should explore other models that ensure the desired screener behavior. 
% \am{NEXT SENTENCE IS UNCLEAR} 
Under such alternative utility models, though, Algorithms~\ref{algo:Examination} and \ref{algo:Cascade} and, in turn, Remark~\ref{remark:ISOandPS} should still hold, thus, illustrating the importance of distinguishing between these two problem formulations. 
% among which the good-$k$ represents a novel addition to the literature.

Other assumptions, like the choices, both function- (Section~\ref{sec:HumanScreener.BiasedScores}) and parameter-wise (Section~\ref{sec:Experiments.Setup}) regarding the human-like screener's fatigue are likely to be contested and limit the implications of our analysis.
We recognize this limitation, though we argue that we made such choices mainly to study the human-like screener (with the fatigue term) under an ISO chosen or provided.
We made the simplest modeling assumptions for the fatigue, but always keeping in mind what we observed at G and know from previous empirical works \cite{DBLP:conf/chi/EchterhoffYM22, PeiEAAMO2023}.
Indeed, under a different fatigue model, Remark~\ref{remark:HumanAndIF} may not hold and $h_h$ may not violate individual fairness; similarly, under different fatigue parameters, the experimental results in Section~\ref{sec:Experiments.Metrics.withFatigue} may motivate a different analysis.
% \am{NEXT SENTENCE IS UNCLEAR} 
% In both cases, though, 
Note, though, that the ISO would still play a central role in any variation of the current assumptions.
% though, we achieve our goal of motivating the ISO problem as the ISO still plays a central role in any variation of the current assumptions that would motivate additional analysis.

We stress that our modeling and simulation frameworks are flexible enough to account for other screening settings.
The experimental setup studied here is one of many possible screening scenarios that we could consider. 
Similarly, the framework could consider the role of the ISO under a different fatigue term.
What matters is studying the role of the ISO and, in particular, its effect on the human screener.
It is costly and time-consuming to run real experiments with human screeners and candidates, especially, at the same scale (10000 runs per setting) of Section~\ref{sec:Experiments}.
We view our work as another example \cite{DBLP:conf/fat/IonescuHJ21, DBLP:journals/corr/abs-2006-09663, DBLP:conf/fat/BountouridisHMM19, Bokanyi2020_Understanding, Schelling1971_Dynamic} of how simulations are useful tools for studying the fairness and optimality of real-world decision-making processes.

For instance, based on the experiments presented in Section~\ref{sec:Experiments}, a company like G could revisit the current screening practices of its HR officers.
Informed by the simulations, instead of running the sequential, partial search (Algorithm~\ref{algo:Cascade}) for the good-$k$, the company could, e.g., recommend its HR officers to take breaks after screening a certain number of candidates in the ISO to ensure that the fatigue does not accumulate. 
Similarly, such a new search procedure for the good-$k$ problem could be implemented and tested within our simulation framework as it would still be addressing the ISO problem. 
Overall, such an approach would be a better alternative for the company to experiment on real screeners and candidates.
Note that the company could carry out these analyses considering an ISO that is either chosen by or provided to the screener; in turn, it would allow for studying settings involving human screeners alone and human screeners relying on some form of algorithmic aid~\cite{DBLP:journals/air/MosqueiraReyHABF23}. %like a fair ranker or the sorting fields of a digital platform.

Thus, future work could explore alternative utility models and fatigue terms while relying on the current simulations framework.
For instance, an interesting alternative formulation for fatigue could involve a human-like screener that rests while searching the candidate pool under the ISO. 
We see recurrent survival models \cite{DBLP:conf/www/ChandarTMPSWCLJ22} well suited for this task.
Further, defining a human-like screener is not limited to fatigue. 
Future work could also explore theories on human decision-making (see \cite[Ch. 10]{DBLP:books/daglib/0033056} and \cite{DBLP:conf/chi/CarabanKGC19, DBLP:journals/isr/AdomaviciusBCZ13}).
Furthermore, the simulations framework could be extended to test for optimal parameters in the ISO problem. 
For example, finding an optimal minimum score $\psi$ for which the two problem formulations coincide and making the search procedure less influential.

%%% during flight
To conclude, we summarize three main takeaways from our analysis of the ISO problem. 
First, \textit{defining the adequate problem formulation
is important for understanding the impact of the ISO on the selected set of candidates} (Remark~\ref{remark:ISOandPS}), which reflects the search procedures of the screeners regardless of the kind.
Second, once the search procedure is clear, \textit{it is important to understand how the screener behaves as it searches over the ISO}.
This is most important for the human-like screener (with fatigue) and the fair individual treatment of each candidate in the pool (Remark~\ref{remark:HumanAndIF}).
Third, \textit{simulations are beneficial for understanding the implications of the interaction between the screener and the ISO on the selected candidates}.
For example, the composition of the candidate pool can amplify or diminish the influence of the position bias in the ISO (as in Sections~\ref{sec:Experiments.Metrics.outFatigue} and \ref{sec:Experiments.Metrics.withFatigue}).
All three takeaways have direct implications for practitioners.
These insights might seem obvious ex-post, though they are supported by an extensive analysis that accounts for several factors that only become clear under modeling and experiments.


% Let us take company G as an example here.
% If G's HR officers expect a pool of candidates with many good candidates, then based on our analysis
% they could opt for a partial search and still expect a set of selected candidates as though they had performed a full search.
% In other words, the simulations would inform the HR officers how to act under such a candidate pool.
% These are reminders that many problems, like the ISO problem, only become obvious after careful theoretical and experimental frameworks.

% We also emphasize the role of the individual scoring function used for evaluating each candidate. 
% We framed the algorithmic screener in terms of its consistency to evaluate similar candidates similarly.
% This assumption allowed us to focus on the role of the ISO.
% We are, though, aware that current work mainly focuses on defining fair scoring functions to obtain a fair ranking of candidates. 
% Future work should explore fair ranking algorithms under the ISO problem. The goal should be to create an algorithm that provides a fair ISO specific to the human user.
% This is because, as long as the fair ranking or fair platform is used by a human for candidate screening, the risk of the ISO problem is still present.

%
% EOS
%

% Limitations
% Our work is based on a collaboration with company G, and not on a field or observational study, such as \cite{Pisanelli2022_YourCV}, which is why we chose the term ``stylised facts'' to summarize our experience in Section~\ref{sec:Generali}. Our observations of G's candidate screening practices are limited to our interaction with G's AA and HT teams.
% In this work, we have made functional assumptions on what defines a human-like screener. 
% The notions of utility and fatigue are open to discussion. 
% Future work should consider alternative, more complex formulations to better capture the human-like screener. 
% In particular, we reduced the ``humanity'' of the screener in terms of fatigue. 
% This assumption allowed us to present a simple and intuitive setting to showcase the impact of the ISO.

% Moreover, the simulation procedures can be extended to account for additional parameters and for score distributions estimated from real data, and they can be made accessible through a user-friendly interface.

%%
\begin{acks}
    Jose M. Alvarez and Salvatore Ruggieri received funding from the European Union’s Horizon 2020 research and innovation program under Marie Sklodowska-Curie Actions (grant agreement number 860630) for the project "NoBIAS - Artificial Intelligence without Bias". 
    Antonio Mastropietro and Salvatore Ruggieri received funding from the European Union’s Horizon Europe (grant agreement number 101070212) for the FINDHR project.
    This work reflects only the authors' views and the European Research Executive Agency is not responsible for any use that may be made of the information it contains.
    No other funding was received by the authors.
\end{acks}

%%
% \clearpage

\section*{Ethical Considerations}

% \paragraph{Ethical considerations}
% 1) a description of the ethical concerns the authors mitigated while conducting the work (as part of an ethical considerations statement)
We did not face any ethical challenges when drafting the paper.  
Results are based on simulated data intended to illustrate our theoretical analysis. During our collaboration with company G, in particular, which occurred before the drafting of this paper, we followed G's ethical guidelines at all times. We concluded our collaboration with G with an internal report that we presented and discussed with all stakeholders. No sensitive data (or data at all) from company G was used for this work. At no point did we receive monetary compensation from G. Our research has been funded by public institutions. The views reflected are entirely our own.

% \paragraph{Researcher positionality}
% % 2) reflections on how their background and experiences inform or shape the work (as part of a researcher positionality statement)
% The authors come from a mixed of backgrounds, both personal and academic. To ensure anonymity, we do not disclose here further information about the authors, but we are committed to provide the necessary details if this submission is successful. We will provide a positionality statement per author. 

% \paragraph{Adverse impact}
% % 3) reflection on the adverse, unintended impact the work might have once published (as part of an adverse impact statement)
We believe that this work shows the importance of considering the human user in the formulation of the candidate screening problem. We want to stress that our distinction between an algorithmic screener and a human-like screener was to show the importance of considering the latter kind and not to endorse the former kind. We strongly believe that candidate screening is a complex, human-dependent and human-centered process that should not be left as an automated decision-making problem.

%%
\bibliographystyle{ACM-Reference-Format}
\bibliography{references}

%%
% \clearpage

\appendix
\appendices
\section{The Proof of Proposition \ref{prop2}}
\label{appa}
For the jointly Gaussian random vectors $\bm{x}$ and $\bm{y}$, we have
\begin{equation}
\begin{aligned}
&    \left[\begin{matrix}\bm{x}\\\bm{y}\\\end{matrix}\right] \sim \mathcal{N}\left(\left[\begin{matrix}\bm{\mu}_x\\\bm{\mu}_y\\\end{matrix}\right],\left[\begin{matrix}A&C\\C^T&B\\\end{matrix}\right]\right) \\
& = \mathcal{N}\left(\left[\begin{matrix}\bm{\mu}_x\\\bm{\mu}_y\\\end{matrix}\right],\left[\begin{matrix}\widetilde{A}&\widetilde{C}\\{\widetilde{C}}^T&B\\\end{matrix}\right]^{-1}\right)
\end{aligned}
\end{equation}
then the marginal and conditional distribution of $\bm{x}$ are shown as follows according to \cite{williams2006gaussian}.
\begin{equation}
    \bm{x} \sim \mathcal{N}\left(\bm{\mu}_x,A\right)
\end{equation}
% and
\begin{equation}
\label{app2-1}
    \bm{x}|\bm{y} \sim \mathcal{N}\left(\bm{\mu}_x+CB^{-1}\left(\bm{y}-\bm{\mu}_y\right),A-CB^{-1}C^T\right)
\end{equation}
% or
\begin{equation}
\label{app2-2}
    \bm{x}|\bm{y} \sim \mathcal{N}\left(\bm{\mu}_x-{\widetilde{A}}^{-1}\widetilde{C}\left(\bm{y}-\bm{\mu}_y\right),{\widetilde{A}}^{-1}\right)
\end{equation}

Thus, \textbf{Proposition \ref{prop2}} is proved.










\section{The Proof of Proposition \ref{prop3}}
\label{appb}
The product of two Gaussian distributions is represented as
\begin{equation}
\mathcal{N}\left(\bm{x}\middle|\bm{a},A\right)\mathcal{N}\left(\bm{x}\middle|\bm{b},B\right)=Z^{-1}\mathcal{N}\left(\bm{x}\middle|\bm{c},C\right)
\end{equation}
where
\begin{equation}
\label{app4}
    \bm{c}=C\left(A^{-1}\bm{a}+B^{-1}\bm{b}\right)
\end{equation}
\begin{equation}
\label{app5}
    C=\left(A^{-1}+B^{-1}\right)^{-1}
\end{equation}
\begin{equation}
\label{app6}
    Z^{-1}=\left(2\pi\right)^{-\frac{D}{2}}\left|A+B\right|^{-\frac{1}{2}}\exp{\left(-\frac{\left(\bm{a}-\bm{b}\right)^T\left(\bm{a}-\bm{b}\right)}{2\left(A+B\right)}\right)}
\end{equation}

Thus, through multiplying the cavity distribution by $t_i$ from (\ref{11}), \textbf{Proposition \ref{prop3}} is proved.


\section{The Proof of Proposition \ref{prop4}}
\label{appc}
Consider
\begin{equation}
\label{app7}
Z=\int_{-\infty}^{\infty}{\Phi\left(\frac{x-m}{v}\right)\mathcal{N}(x|\mu,\sigma^2)dx}
\end{equation}
% where
% \begin{equation}
%     \Phi\left(x\right)=\int_{-\infty}^{x}{\mathcal{N}\left(y\right)dy}
% \end{equation}
When $v>0$, by combining$ z=y-x+\mu-m$ and $w=x-\mu$ we can get
\begin{equation}
\begin{aligned}
& Z_{v>0}=\frac{\int_{-\infty}^{\infty}\int_{-\infty}^{x}\exp{\left(-\frac{\left(y-m\right)^2}{2v^2}-\frac{\left(x-\mu\right)^2}{2\sigma^2}\right)}}{2\pi\sigma v}dydx \\
& =\frac{\int_{-\infty}^{\mu-m}\int_{-\infty}^{\infty}\exp{\left(-\frac{\left(z+w\right)^2}{2v^2}-\frac{w^2}{2\sigma^2}\right)}}{2\pi\sigma v}dwdz
\end{aligned}
\end{equation}
% and
\begin{equation}
\begin{aligned}
& Z_{v>0} \\
& =\frac{\int_{-\infty}^{\mu-m}\int_{-\infty}^{\infty}\exp{\left(-\frac{1}{2}\left[\begin{matrix}w\\z\\\end{matrix}\right]^T\left[\begin{matrix}\frac{1}{v^2}+\frac{1}{\sigma^2}&\frac{1}{v^2}\\\frac{1}{v^2}&\frac{1}{v^2}\\\end{matrix}\right]\left[\begin{matrix}w\\z\\\end{matrix}\right]\right)}}{2\pi\sigma v}dwdz \\
& =\int_{-\infty}^{\mu-m}\int_{-\infty}^{\infty}\mathcal{N}\left(\left[\begin{matrix}w\\z\\\end{matrix}\right]|\mathbf{0},\left[\begin{matrix}\sigma^2&-\sigma^2\\-\sigma^2&v^2+\sigma^2\\\end{matrix}\right]\right)dwdz
\end{aligned}
\end{equation}
According to (\ref{app2-1}) and (\ref{app2-2}), we can get
\begin{equation}
\label{app11}
    Z_{v>0}=\frac{\int_{-\infty}^{\mu-m}\exp{\left(-\frac{z^2}{2\left(v^2+\sigma^2\right)}\right)}dz}{\sqrt{2\pi(v^2+\sigma^2)}}=\Phi\left(\frac{\mu-m}{\sqrt{v^2+\sigma^2}}\right)
\end{equation}
When $v<0$, by combining $\Phi\left(-z\right)=1-\Phi\left(z\right)$ and (\ref{app7}),
% we can obtain
\begin{equation}
\label{app12}
Z_{v<0}=1-\Phi\left(\frac{\mu-m}{\sqrt{v^2+\sigma^2}}\right)=\Phi\left(-\frac{\mu-m}{\sqrt{v^2+\sigma^2}}\right)
\end{equation}

By collecting (\ref{app11}) and (\ref{app12}), we can get
\begin{equation}
\label{app13}
Z=\int\Phi\left(\frac{x-m}{v}\right)\mathcal{N}\left(x\middle|\mu,\sigma^2\right)dx=\Phi\left(z\right)
\end{equation}
where $z=\frac{\mu-m}{v\sqrt{1+\sigma^2/v^2}} (v\neq0)$. 
% We aim to get the moments of
% \begin{equation}
% q\left(x\right)=Z^{-1}\Phi\left(\frac{x-m}{v}\right)\mathcal{N}\left(x\middle|\mu,\sigma^2\right)
% \end{equation}
By differentiating with respect to $\mu$ on (\ref{app13}), we can obtain
\begin{equation}
\begin{aligned}
& \frac{\partial Z}{\partial\mu}=\int{\frac{x-\mu}{\sigma^2}\Phi\left(\frac{x-m}{v}\right)}\mathcal{N}\left(x\middle|\mu,\sigma^2\right)dx =\frac{\partial}{\partial\mu}\Phi\left(z\right) \\
& \Longleftrightarrow \frac{1}{\sigma^2}\int x\Phi\left(\frac{x-m}{v}\right)\mathcal{N}\left(x\middle|\mu,\sigma^2\right)dx-\frac{\mu Z}{\sigma^2} \\
& =\frac{\mathcal{N}(z)}{v\sqrt{1+\sigma^2/v^2}}
\end{aligned}
\end{equation}
where $\partial\Phi\left(z\right)/\partial\mu=\mathcal{N}(z)\partial z/\partial\mu$ is utilized. Multiplying through by $\sigma^2/Z$, (\ref{app16}) is obtained.
\begin{equation}
\label{app16}
\mathbb{E}_q\left[x\right]=\mu+\frac{\sigma^2\mathcal{N}\left(z\right)}{\Phi\left(z\right)v\sqrt{1+\frac{\sigma^2}{v^2}}}
\end{equation}
Similarly, we can obtain the second moment as
\begin{equation}
\label{app17}
\begin{aligned}
 & \frac{\partial^2Z}{\partial\mu^2} \\
 & =\int{[\frac{x^2}{\sigma^4}-\frac{2\mu x}{\sigma^4}+\frac{\mu^2}{\sigma^4}-\frac{1}{\sigma^2}] \Phi\left(\frac{x-m}{v}\right)\mathcal{N}\left(x\middle|\mu,\sigma^2\right)} dx  \\
 & =-\frac{z\mathcal{N}(z)}{v^2+\sigma^2} \Longleftrightarrow \\
 & \mathbb{E}_q\left[x^2\right]=2\mu\mathbb{E}_q\left[x\right]-\mu^2+\sigma^2-\frac{\sigma^4z\mathcal{N}\left(z\right)}{\Phi\left(z\right)\left(v^2+\sigma^2\right)}
\end{aligned}
\end{equation}
By combining (\ref{app16}) and (\ref{app17}), we can get
\begin{equation}
\begin{aligned}
& \mathbb{E}_q\left[{(x-\mathbb{E}_q\left[x\right])}^2\right]=\mathbb{E}_q\left[x^2\right]-\mathbb{E}_q[x]^2 \\
& =\sigma^2-\frac{\sigma^4\mathcal{N}\left(z\right)}{\left(v^2+\sigma^2\right)\Phi\left(z\right)}\left(z+\frac{\mathcal{N}\left(z\right)}{\Phi\left(z\right)}\right)
\end{aligned}
\end{equation}

Thus, \textbf{Proposition \ref{prop4}} is proved.

\section{The Proof of Proposition \ref{prop5}}
\label{appd}
We can obtain (\ref{19-1}), (\ref{19-2}), and (\ref{19-3}) according to (\ref{app4}), (\ref{app5}), and (\ref{app6}). Hence, \textbf{Proposition \ref{prop5}} is proved.



\section{The Proof of Proposition \ref{prop6}}
\label{appe}
The approximated mean for $f_\ast$ can be denoted as
\begin{equation}
\begin{aligned}
& \mathbb{E}_q\left[f_\ast|X,\bm{y},\bm{x}_\ast\right]=\bm{k}_\ast^TK^{-1}\bm{\mu} \\
& =\bm{k}_\ast^TK^{-1}\left(K^{-1}+{\widetilde{\Sigma}}^{-1}\right)^{-1}{\widetilde{\Sigma}}^{-1}\widetilde{\bm{\mu}} \\
& =\bm{k}_\ast^T\left(K+\widetilde{\Sigma}\right)^{-1}\widetilde{\bm{\mu}}
\end{aligned}
\end{equation}

The variance of $f_\ast|(X,\bm{y})$ under the Gaussian approximation can be denoted as
\begin{equation}
\begin{aligned}
& \mathbb{V}_q\left[f_\ast\middle| X,\bm{y},\bm{x}_\ast\right] = \mathbb{E}_{p(f_\ast|X,\bm{x}_\ast,\bm{f})} {f_\ast-\mathbb{E}[f_\ast|X,\bm{x}_\ast,\bm{f}]}^2 \\
& =k\left(\bm{x}_\ast,\bm{x}_\ast\right)-\bm{k}_\ast^TK^{-1}\bm{k}_\ast+\bm{k}_\ast^TK^{-1}\left(K^{-1}+\widetilde{\Sigma}\right)^{-1}K^{-1}\bm{k}_\ast \\
& =k\left(\bm{x}_\ast,\bm{x}_\ast\right)-\bm{k}_\ast^T\left(K^{-1}+\widetilde{\Sigma}\right)^{-1}\bm{k}_\ast
\end{aligned}
\end{equation}

Then, we can obtain
\begin{equation}
\begin{aligned}
& q\left(y_\ast\middle| X,\bm{y},\bm{x}_\ast\right)=\mathbb{E}_q\left[\pi_\ast|X,\bm{y},\bm{x}_\ast\right] \\
& =\int\Phi\left(f_\ast\right)q\left(f_\ast\middle| X,\bm{y},\bm{x}_\ast\right)df_\ast
\end{aligned}
\end{equation}

According to (\ref{app11}), we can obtain
\begin{equation}
\label{app22}
\begin{aligned}
& q\left(y_\ast\middle| X,\bm{y},\bm{x}_\ast\right) \\
& =\Phi\left(\frac{\bm{k}_\ast^T\left(K+\widetilde{\Sigma}\right)^{-1}\widetilde{\bm{\mu}}}{\sqrt{1+k\left(\bm{x}_\ast,\bm{x}_\ast\right)-\bm{k}_\ast^T\left(K+\widetilde{\Sigma}\right)^{-1}\bm{k}_\ast}}\right)
\end{aligned}
\end{equation}

By combining (\ref{13}) and (\ref{app22}), \textbf{Proposition \ref{prop6}} is proved.




\section{The Proof of Proposition \ref{prop7}}
\label{appf}
Given $f_s$ and $f_\ast$, $y_s$ and $y_\ast$ are conditionally independent. Hence, $p\left(y_s,y_\ast\middle|\bm{x}_s,\bm{x}_\ast\right)$ can be represented as
\begin{equation}
\begin{aligned}
& p\left(y_s=1,y_\ast=1\middle|\bm{x}_s,\bm{x}_\ast\right) \\
& =\iint{\Phi\left(f_s\right)\Phi\left(f_\ast\right)\phi\left(f_s,f_\ast\middle|\mu_{s\ast},\Sigma_{s\ast}\right)}df_sdf_\ast \\
& =\iint{\Phi\left(f_\ast\right)\phi\left(f_\ast\middle|{\widetilde{\mu}}_\ast\left(f_s\right),{\widetilde{\sigma}}_{\ast\ast}\right)df_\ast\Phi\left(f_s\right)}\phi\left(f_s\middle|\mu_s,\sigma_{ss}\right)df_s \\
& =\int\Phi\left(\frac{{\widetilde{\mu}}_\ast\left(f_s\right)}{\sqrt{{\widetilde{\sigma}}_{\ast\ast}+1}}\right)\Phi\left(f_s\right)\phi\left(f_s\middle|\mu_s,\sigma_{ss}\right)df_s
\end{aligned}
\end{equation}

Hence, \textbf{Proposition \ref{prop7}} is proved.

% \section{The Proof of Lemma \ref{lem}}
% \label{appg}
% \begin{equation}
% \begin{aligned}
% & R_e=\frac{1}{N_a}\sum_{n=1}^{N_a}\mathbb{I}\left(\bm{L}_n \neq \bm{Y}_n\right) \\
% & =\displaystyle\frac{FA+FL}{TL+TA+FL+FA} \\
% & =\displaystyle\frac{1}{\displaystyle\frac{TL+TA+FL+FA}{FA+FL}} \\
% & =\displaystyle\frac{1}{1+\displaystyle\frac{TL+TA}{FA+FL}} \\
% & =\displaystyle\frac{1}{1+\displaystyle\frac{\displaystyle\frac{TL}{TA}+1}{\displaystyle\frac{FA}{TA}+\displaystyle\frac{FL}{TA}}} \\
% & =\frac{1}{1+\displaystyle\frac{\displaystyle\frac{TL}{TA}+1}{\displaystyle\frac{1}{P_{md}-1}+\displaystyle\frac{1}{P_{fa}-1}}}
% \end{aligned}
% \end{equation}

% Hence, \textbf{Lemma \ref{lem}} is proved.

%%
\end{document}

%
% EOF
%
