%%%%%%%%%%%%%%%%%%%%%%%%%%%%%%%%%%%%%%%%%%%%%%%%%%%%%%%%%%%%%%%%%%%%%%%%%%%%%%%%
%2345678901234567890123456789012345678901234567890123456789012345678901234567890




\documentclass[letterpaper, 10 pt, conference]{ieeeconf}  % Comment this line out if you need a4paper

%\documentclass[a4paper, 10pt, conference]{ieeeconf}      % Use this line for a4 paper

\IEEEoverridecommandlockouts                              % This command is only needed if 
                                                          % you want to use the \thanks command

\overrideIEEEmargins                                      % Needed to meet printer requirements.

%In case you encounter the following error:
%Error 1010 The PDF file may be corrupt (unable to open PDF file) OR
%Error 1000 An error occurred while parsing a contents stream. Unable to analyze the PDF file.
%This is a known problem with pdfLaTeX conversion filter. The file cannot be opened with acrobat reader
%Please use one of the alternatives below to circumvent this error by uncommenting one or the other
%\pdfobjcompresslevel=0
%\pdfminorversion=4

% See the \addtolength command later in the file to balance the column lengths
% on the last page of the document

% The following packages can be found on http:\\www.ctan.org
%\usepackage{graphics} % for pdf, bitmapped graphics files
%\usepackage{epsfig} % for postscript graphics files
%\usepackage{mathptmx} % assumes new font selection scheme installed
%\usepackage{times} % assumes new font selection scheme installed
%\usepackage{amsmath} % assumes amsmath package installed
%\usepackage{amssymb}  % assumes amsmath package installed
\usepackage{algorithm}
\usepackage{algpseudocode}
\usepackage[dvipsnames]{xcolor}
\usepackage{annotate-equations}
\usepackage{cancel}
\usepackage{color}
\usepackage{hyperref}

\usepackage{booktabs}
\usepackage{amsfonts}
\usepackage{graphicx}
\usepackage{hyperref}
\usepackage{textcomp}
\usepackage{listings}
\usepackage{adjustbox}
\usepackage{xspace}    % sticks a sane space after a command
\usepackage{multirow}
\usepackage{multicol}
\usepackage{tcolorbox}
\usepackage{wrapfig}
\usepackage{diagbox}
\usepackage{subcaption}
\usepackage{amsmath}
\usepackage{pifont}
\usepackage{breakcites}
\usepackage{array}

\newcommand{\todo}[1]{{\color{red}#1}}	
\newcommand{\shb}[1]{{\color{orange}#1}}	
\newcommand{\zx}[1]{\textcolor{blue}{(zx: #1)}}
\newcommand{\yf}[1]{\textcolor{purple}{(yf: #1)}}
\newcommand{\ar}[1]{\textcolor{green}{(ar: #1)}}

\title{\LARGE \bf
 PIMbot: Policy and Incentive Manipulation for Multi-Robot Reinforcement Learning in Social Dilemmas  
}


\author{Shahab Nikkhoo$^{1}$, Zexin Li$^{1}$, Aritra Samanta$^{1}$, Yufei Li$^{1}$ and Cong Liu$^{1}$% <-this % stops a space
\thanks{$^{1}$University of California, Riverside}%
}


\begin{document}


\maketitle
\thispagestyle{empty}
\pagestyle{empty}


%%%%%%%%%%%%%%%%%%%%%%%%%%%%%%%%%%%%%%%%%%%%%%%%%%%%%%%%%%%%%%%%%%%%%%%%%%%%%%%%
\begin{abstract}
\begin{abstract}

The Fast Reciprocal Square Root Algorithm is a well-established approximation technique consisting of two stages: first, a coarse approximation is obtained by manipulating the bit pattern of the floating point argument using integer instructions, and second, the coarse result is refined through one or more steps, traditionally using Newtonian iteration but alternatively using improved expressions with carefully chosen numerical constants found by other authors. The algorithm was widely used before microprocessors carried built-in hardware support for computing reciprocal square roots. At the time of writing, however, there is in general no hardware acceleration for computing other fixed fractional powers. This paper generalises the algorithm to cater to all rational powers, and to support any polynomial degree(s) in the refinement step(s), and under the assumption of unlimited floating point precision provides a procedure which automatically constructs provably optimal constants in all of these cases. It is also shown that, under certain assumptions, the use of monic refinement polynomials yields results which are much better placed with respect to the cost/accuracy tradeoff than those obtained using general polynomials. Further extensions are also analysed, and several new best approximations are given.

\end{abstract}

\end{abstract}


%%%%%%%%%%%%%%%%%%%%%%%%%%%%%%%%%%%%%%%%%%%%%%%%%%%%%%%%%%%%%%%%%%%%%%%%%%%%%%%%
\section{INTRODUCTION}
%
%
%
%
%
Clinical depression, a prevalent mental health condition, is considered as one of the leading contributors to the global health-related burden \cite{greenberg2015economic, lepine2011increasing}, affecting millions of people worldwide \cite{vos_et_al_GBD2016,institute2021global}. As a mood disorder, it is characterised by a prolonged (> two weeks) feeling of sadness, worthlessness and hopelessness, a reduced interest and a loss of pleasure in normal daily life activities, sleep disturbances, tiredness and lack of energy. Depression can lead to suicide in extreme cases \cite{goldney2000suicidal} and is often linked to comorbidities such as anxiety disorders, substance abuse disorders, hypertensive diseases, metabolic diseases, and diabetes \cite{steffen_et_al_2020_BMCPsychiatry,campayo2011diabetes}. Although effective treatment options are available, diagnosing depression through self-report and clinical observations presents significant challenges due to the inherent subjectivity and biases involved.

Over the last decade, researchers from affective computing and psychology have focused on investigating objective measures that can aid clinicians in the initial diagnosis and monitoring of treatment progress of clinical depression \cite{cohn2018multimodal, pampouchidou_et_al_TAC_DepressionReview}. A key catalyst to this progress is the availability of relevant datasets, such as AVEC2013 and subsequent challenges~\cite{valstar2013avec}. In recent years, research on depression detection employing affective computing approaches has increasingly focused on leveraging non-verbal behavioural cues such as facial expressions \cite{bourke2010processing, de2019combining}, body gestures \cite{joshi2013relative}, eye gaze \cite{alghowinem2016multimodal}, head movements \cite{alghowinem2013head} and verbal features \cite{cummins2011investigation, huang2019investigation} extracted from multimedia data to develop distinctive features to classify individuals as depressed or healthy controls, or to estimate the severity of depression on a continuous scale. 

In this study, we examine the utility of inherently interpretable head motion units, referred to as \emph{kinemes} \cite{madan_gahalawat_guha_subramanian_ICMI2021_Kinemes}, for assessing depression. Initially, we utilise data from both healthy controls and depressed patients to discover a basis set of kinemes via the (\emph{pitch}, \emph{yaw}, and \emph{roll}) head pose angular data obtained from short overlapping time-segments (termed two-class kineme discovery or 2CKD). Further, we employ these kinemes to generate features based on the frequency of occurrence of distinctive, class-characteristic kinemes. Subsequently, we discover kineme patterns solely from head pose data corresponding to healthy controls (Healthy control kineme discovery or HCKD), and use them to represent both healthy and depressed class segments. A set of statistical features are then computed from the reconstruction errors between the raw and learned head-motion segments corresponding to both the depressed and control classes (see Figure ~\ref{fig:Depression_proposed_framework}). Using machine learning methodologies, we evaluate the performance of the features derived from the two approaches. Our results show that head motion patterns are effective behavioural cues for detecting depression. Additionally, explanatory class-specific kinemes patterns can be observed, in alignment with prior research.  

% Figure environment removed

This paper makes the following research contributions:
%
\begin{itemize}
    \item A study of head movements as a biomarker for clinical depression, which so far has been understudied.
    \item Proposing the \textit{kineme} representation of motion patterns as an effective and explanatory means for depression analysis.
    \item \begin{sloppypar} A detailed investigation of various classifiers for 2-class and 4-class categorisation on the AVEC2013 and BlackDog datasets. We obtain peak F1-scores of 0.79 and 0.82, respectively, on \textit{thin-slice} chunks for binary classification on the BlackDog and AVEC2013 datasets, which compare favorably to prior approaches. Also, a video-level F1-score of 0.72 is achieved for 4-class categorisation on AVEC2013.  \end{sloppypar}
\end{itemize}
%
The remainder of this paper is organised as follows. Section \ref{Sec:RW} provides an overview of related work. Section \ref{Sec:KF} describes the kineme formulation, followed by Section \ref{Sec:EKF} that details the explainable kineme features used as a representation of motion patterns. The methodology is presented in Section \ref{Sec:Meth}, while Section \ref{Sec:ER} provides details of the datasets, experimental settings, and classifiers used in this study. The experimental results are shown and discussed in Section \ref{sec:ResultsDiscussion}. Finally, the conclusions are drawn in Section \ref{Sec:DC}.


% Add the basics of kinemes, the approach used
% Add overview of the framework implemented
% Contribution







\section{BACKGROUND}
\vspacebeforesection
\section{Background}
\label{sec:background}

In this section, we provide the necessary background information to ensure a comprehensive understanding of the attack described in this paper. We start with a description of the Distributed Hash Table (DHT) used by IPFS, followed by its content resolution mechanisms. We also detail techniques for network size estimation, necessary for our attack detection and mitigation mechanisms.

\vspacebeforesection
\subsection{IPFS DHT}
\label{sec:kad_dht}

We review the features of the Kademlia DHT~\cite{maymounkov2002kademlia} and its \texttt{libp2p} implementation~\cite{libp2p_github} that are the most relevant to our attack.
To participate in the DHT, each peer generates a public/private key pair and derives an identity $\peerid \in \{0,1\}^{256}$ as the hash of its public key.
Ideally, each peer generates a random key pair and, therefore, peer IDs are distributed uniformly and independently over the space $\{0,1\}^{256}$.
While honest nodes follow this rule, malicious nodes may generate and choose from an arbitrary number of key pairs.
Each peer maintains a routing table consisting of $m=256$ buckets.
The $i$-th bucket contains the addresses of up to $k=20$ peers whose peer IDs share a common prefix of exactly $i$ bits with the peer's own peer ID. 

%
A new participant node joins the IPFS network by contacting one of the hardcoded bootstrap nodes. This bootstrap node provides the new node with some initial peers allowing it to join the DHT. The new node uses this information to perform a walk through the DHT towards its own peer ID.
The walk allows to: \textit{(i)}~make sure that there is no other node in the network with the same ID; \textit{(ii)}~discover new peers and fill the newcomer's DHT routing table. At the same time, the newcomer establishes \bitswap~\cite{de2021accelerating} connections to a subset of encountered peers (usually around 300 of them). The core role of the \bitswap protocol is to enable bilateral content transfer and to play the role of a cache for recently-accessed content.

The main DHT operation $\Call{GetClosestPeers}{\key}$ returns the $k=20$ closest peers to $\key$. 
%
In Kademlia, the distance between two keys $x$ and $y$ in the key space is given by $x \oplus y \in \{0,...,2^{256}-1\}$, where $\oplus$ denotes the bitwise XOR operation on the keys; the resulting binary string is interpreted as an integer.
%
When a client wants to find the peers with IDs closest to $\key$, it sends a request to the $\alpha=3$ peers in its routing table whose peer IDs are closest to $\key$. Each of these peers returns the $k$ closest peers to $\key$ in its own routing table and the addresses of these peers. 
%
The client again sends a request to the $\alpha$ peers closest to $\key$, among peers in its routing table and those whose addresses it just received. This process repeats until the client does not find any more peers closer to $\key$.
Due to network churn and imperfect routing tables, we observed in our experiments that successive calls to $\Call{GetClosestPeers}{\key}$ do not always return the same set of $k=20$ peers (we provide more details in \Cref{sec:evaluation}, \Cref{fig:20closest}). This is an important limitation affecting our attack.

\vspacebeforesection
\subsection{Content Resolution in IPFS}
\label{sec:ipfs}

IPFS is a content-centric network.
It allows its participant to request files without specifying their location. 
%
Content is indexed by content IDs $\cid \in \{0,1\}^{256}$ that are derived from a hash of that content.
Both peer IDs and CIDs are used as keys in the DHT.
Each node can play the role of a \provider, \downloader, or \resolver. 
The process of content advertisement and resolution is illustrated in \Cref{fig:add_get_provider}.

%
When a \provider wishes to publish content with a given $\cid$ on IPFS, it creates a \emph{provider record} that contains $cid$ and the \provider's address.
During a $\Call{Provide}{\cid}$ operation, the \provider first uses $\Call{GetClosestPeers}{\cid}$ to locate the $k=20$ peers with their peer IDs closest to $\cid$, 
%
and then sends them a $\mathsf{PutProvider}$ message including the provider record (\Cref{fig:add_get_provider}(a)).
We call the peers that hold provider records for $\cid$ the \emph{resolvers} for $\cid$.

Each CID can have several \providers. In fact, by default, each IPFS client becomes a provider for each piece of content it downloads for a fixed amount of time (12h, 24h, or 48h depending on the client version or custom configuration). As a result, the system provides an auto-scaling feature with supply automatically rising with demand.

%
When a \downloader wishes to fetch a piece of content, it first sends a request to all its \bitswap peers. If none of them has the content, the \downloader uses the DHT-based resolution system. We stress that the \bitswap protocol plays the supporting role of a cache in the dissemination of popular files. However, the mechanism does not provide reliable content resolution, in particular for new or less popular content. %

When \bitswap unstructured search fails, the \downloader resolves $\cid$ using $\Call{FindProviders}{\cid}$. This operation uses a DHT walk identical to that of $\Call{GetClosestPeers}{\cid}$ to find $k$ \resolvers but also queries encountered nodes for a provider record for $\cid$ (\Cref{fig:add_get_provider}(b)). The process terminates when either 20 \providers have been found, or all \resolvers have been asked. Querying all encountered nodes (\ie, not only the designated \resolvers) is useful because some of the encountered nodes may have a provider record in their cache.
%

Upon receiving a provider record, the client connects to the address specified in the provider record to retrieve the actual content (\Cref{fig:add_get_provider}(c)).
Provider records are not authenticated, and therefore malicious \providers may respond with incorrect provider records (or may not respond at all). However, the integrity of the content is preserved because the hash of the retrieved content can be verified against its $\cid$.
%


%

\input{img/add_get_provider.tex}

\vspacebeforesection
\subsection{Network Size Estimator}
\label{sec:netsize}

The number of nodes in a decentralized system is generally unknown due to the avoidance of centralized membership management.
This number is nonetheless useful for optimizations, deciding on individual node configurations, or security mechanisms.
Various methods were proposed for the decentralized estimation of unstructured and structured networks~\cite{eli-sohl-dht-size-estimation,kostoulas2005decentralized, manku2003symphony}.
We use in this work a mechanism developed initially by Protocol Labs as part of a mechanism for decreasing the latency of publishing content in IPFS~\cite{network-size-estimation-notion,network-size-estimation-github-pr}.

%
%
%
%
%
%
%
%
%
%

Each node in the DHT refreshes its routing table periodically (every $10$ minutes in \texttt{libp2p}). 
For this, the node samples $m$ random keys (one for each bucket of its routing table)
%
and queries the DHT to obtain the $k=20$ closest peer IDs to each key.
Using these, the node then computes the average distance between each one of these keys $\key_j$ for $j=1,\dots,m$ and their $i$-th closest peer ID for $i=1,...,k$ (with $m=256$ and $k=20$).
\begin{equation}
    \label{equ:avg-dist}
    \overline{D}_i = \frac{1}{m} \sum_{j=1}^m \operatorname{dist}(\key_j, \peerid_{j}^{(i)})
\end{equation}
where $\peerid_{j}^{(i)}$ is the $i$-th closest peer ID to $\key_j$.
With $N$ peers in the DHT and peer IDs uniformly distributed in the hash space, the expected distance between a $\key$ and its $i$-th closest peer ID is $\frac{2^{256}i}{N+1}$. The node then runs a least square regression to compute the value of $N$ for which the expected distances best fit the empirical average distances, \ie,
\begin{equation}
    \label{equ:netsize-least-squares}
    \hat{N} = \arg\min_{N} \sum_{i=1}^k \left(\overline{D}_i - \frac{2^{256}i}{N+1}\right)^2.
\end{equation}
The resulting estimate $\hat{N}$ can be computed in closed form.
%

When a node starts running, it must perform DHT queries for a few random keys to initialize its network size estimate. 
Since a larger number of queries will result in higher accuracy, making more queries than what is needed to initialize one's routing table is recommended.
Thereafter, keeping the estimate up-to-date does not require any excess DHT queries beyond what is already used for refreshing the routing table as this is done frequently (every 10 minutes).

While the network size estimate has a stochastic variance resulting from the probability distribution of the honest peer IDs, it is hard for an attacker to bias the estimate significantly. Since the estimator uses the density of peer IDs around keys chosen uniformly at random, the adversary would require numerous Sybil nodes (on the order of the whole network size) to significantly affect the peer ID density around those keys.



\section{DESIGN}
\section{Design}
The design philosophy of causal-learn is centered around building an open-source, modular, easily extensible and embeddable Python platform for learning causality from data and making use of causality for various purposes. Due to the different goals, assumptions, and techniques between causal learning and traditional learning tasks, newcomers to the field often find it hard to get a clear picture of the developments in modern causality research. Thus, we briefly introduce the algorithms and functionalities in causal-learn with a special focus on their use cases and suitable application scenarios.

\subsection{Search methods}
Causal-learn covers representative causal discovery methods across all major categories with official implementation of most algorithms. We briefly introduce the methods as follows. It is worth noting that we are actively updating the library to incorporate latest algorithms.
\begin{itemize}
    \item \textbf{Constraint-based causal discovery methods.} Current algorithms under that category are PC \citep{spirtes2000causation}, FCI \citep{spirtes1995causal}, and CD-NOD \citep{huang2020causal}. PC is a classical and widely-used algorithm with consistency guarantee under independent and identically distributed (i.i.d.) sampling assuming no latent confounders, the faithfulness assumption, and the causal Markov condition, which has been extensively applied in many fields. By continuously applying (conditional) independence tests on subsets of variables of increasing size in a smart way, its search procedure returns a Markov Equivalence Class (MEC), of which the graphical object consists of a mixture of directed and undirected edges, known as a Completed Partially Directed Acyclic Graph (CPDAG). PC is highly adaptable to various use cases, facilitated by the selection of an appropriate independence test; it can handle data with different assumptions, such as Fisher-Z test \citep{fisher1921014} for linear Gaussian data, Chi/G-squared test \citep{tsamardinos2006max} for discrete data, and Kernel-based Conditional Independence (KCI) test \citep{zhang2011kernel} for the nonparametric case. Moreover, causal-learn provides an extension, Missing-Value PC (MV-PC) \citep{tu2019causal}, to address issues of missing data. Furthermore, we have implemented FCI for causal structures that include hidden confounders (it indicates the possible existence of hidden confounders whenever the possibility cannot be excluded, but it cannot help determine possible relations among them), and causal discovery from nonstationary/heterogeneous data (CD-NOD). These constraint-based methods offer wide applicability as they can accommodate various types of data distributions and causal relations, provided that appropriate conditional independence testing methods are utilized. However, genenerally speaking, they may not be able to determine the complete causal graph uniquely and, accordingly, there usually exist some undirected edges in the returned CPDAGs.

    \item \textbf{Score-based causal discovery methods.} Different from the search style of constraint-bed methods, score-based methods find the causal structure by optimizing a properly defined score function. Greedy Equivalence Search (GES) \citep{chickering2002optimal} is a well-known two-phase procedure that directly searches over the space of equivalence classes. Similarly, exact search (e.g., A* \citep{yuan2013learning}, Dynamic Programming \citep{silander2006simple}), and permutation-based search (e.g., GRaSP \citep{lam2022greedy}) apply different search strategies to return a set of the sparsest Directed Acyclic Graphs (DAGs) that contains the true model under assumptions strictly weaker than faithfulness. These score-based methods are versatile, able to accommodate a wide array of data and causal relations by choosing suitable score functions, such as BIC \citep{schwarz1978estimating} for linear Guassian data, BDeu \citep{buntine1991theory} for discrete data, and Generalized Score \citep{huang2018generalized} for the nonparametric case. The choice of score function can be conveniently adjusted as a hyperparameter.

    \item \looseness=-1 \textbf{Causal discovery methods based on constrained functional causal models.} While constraint-based and score-based methods offer flexibility through the selection of an appropriate independence test or score function, they are limited to returning equivalence classes, yielding non-unique solutions where the causal direction between certain variable pairs remains indeterminate. In contrast, assuming specific Functional Causal Models (FCMs)--that is, functions in a particular functional class to specify how the effect is generated from its direct causes and noise--allows for the full determination of the causal structure, albeit at the cost of certain trade-offs. Causal-learn incorporates algorithms based on several FCM variants, capable of producing unique causal directions. Examples include the linear non-Gaussian acyclic model (LiNGAM) \citep{shimizu2006linear} and its variant, i.e., DirectLiNGAM \citep{shimizu2011directlingam}, which have been extensively applied for non-Gaussian noises with linear relations. VAR-LiNGAM \citep{hyvarinen2010estimation}, which combines LiNGAM with vector autoregressive models (VAR), to estimate both time-delayed and instantaneous causal relations from time series. RCD \citep{maeda2020rcd}, an extension of LiNGAM, allows for hidden confounders, while CAM-UV \citep{maeda2021causal} further extends this to the nonlinear additive noise case. In addition, the additive noise model (ANM) \citep{hoyer2008nonlinear} has been proven to be identifiable in the presence of nonlinearity and additive noises. Furthermore, we have also incorporated the post-nonlinear (PNL) causal model \citep{zhang2009identifiability}, a highly general form (with LiNGAM and ANM as special cases) that has been demonstrated to be identifiable in the generic case, barring five specific situations described in \citep{zhang2009identifiability}.

    \item \textbf{Causal representation learning: Finding causally related hidden variables.} Latent variables play an instrumental role in a multitude of real-world scenarios, often acting as hidden confounders that influence observed variables. Unfortunately, most existing methods may fail to produce convincing results in cases with latent variables (confounders). In causal-learn, we implement the Generalized Independent Noise (GIN) condition \citep{xie2020generalized} for estimating linear non-Gaussian latent variable causal model, which allows causal relationships between latent variables and multiple latent variables behind any two observed variables. This promises to improve the detection and understanding of the complex, often hidden, causal structures that govern real-world phenomena.

\end{itemize}

Besides, causal-learn also has Granger causality \citep{granger1969investigating, granger1980testing} implemented for statistical but not causal\footnote{As mentioned by Granger, Granger causality is not necessarily true causality. In fact, If one assumes 1) that there is no latent confounding process, 2) that the data are sampled at the right causal frequency, and 3) that there are no instantaneous causal influences, then Granger causality defined by Granger \citep{granger1980testing} can be seen as causal relations that can be discovered from stochastic processes with constraints-based methods such as PC. Of course, if those assumptions are violated, one may still apply Granger causal analysis, but the estimated relations may not be true causal influences.} time series analysis. Through the collective efforts of various teams and the contributions of the open-source community, causal-learn is always under active development to incorporate the most recent advancements in causal discovery and make them available to both practitioners and researchers.

\subsection{(Conditional) independence tests}

In addition to its comprehensive search methods, causal-learn also provides a variety of (conditional) independence tests as independent modules. Besides being an essential parts of several search methods, these tests can also be independently utilized and seamlessly integrated into existing statistical analysis pipelines. Currently,the library features a diverse array of such tests including Fisher-z test \citep{fisher1921014}, Missing-value Fisher-z test, Chi-Square test, Kernel-based conditional independence (KCI) test and independence test \citep{zhang2011kernel}, and G-Square test \citep{tsamardinos2006max}, each with distinct capabilities and benefits. The Fisher-z test is ideally suited for linear-Gaussian data, while the Missing-value Fisher-z test addresses the challenges of missing values by implementing a testwise-deletion approach. For categorical variables, the Chi-Square and G-Square tests are most effective. For users interested in a nonparametric test or the case with mixed categorical and continuous data types, the KCI test is an option. Overall, the range of tests offered by causal-learn underscores its versatility in handling diverse data types.

\subsection{Score functions}
\looseness=-1
Moreover, a diverse range of score functions is available in \textit{causal-learn}. These score functions quantify the goodness of fit of a model to the data, a crucial measure in score-based causal discovery methods, and can also be utilized independently for model selection in a broader range. Among these, the Bayesian Information Criterion (BIC) score \citep{schwarz1978estimating} is used extensively, offering a balance between model complexity and fit to the data. Another important score function is the Bayesian Dirichlet equivalent uniform (BDeu) score \citep{buntine1991theory}. This score function, especially beneficial for discrete data, incorporates a uniform prior over the set of Bayesian networks. Additionally, the Generalized Score \citep{huang2018generalized} is also available in causal-learn, which offers the flexibility to accommodate more complex scenarios and is beneficial for nonparametric cases where the true data-generating process does not align with the assumptions of BIC (linear Gaussian) or BDeu (discrete).


\subsection{Utilities}

Causal-learn further offers a suite of utilities designed to streamline the assembly of causal analysis pipelines. The package features a comprehensive range of graph operations encompassing transformations among various graphical objects integral to causal discovery. These include Directed Acyclic Graphs (DAGs), Completed Partially Directed Acyclic Graphs (CPDAGs), Partially Directed Acyclic Graphs (PDAGs), and Partially Ancestral Graphs (PAGs). Additionally, to enhance the convenience of experimental processes, \textit{causal-learn} features a set of commonly used evaluation metrics to appraise the quality of the causal graphs discovered. These metrics include precision and recall for arrow directions or adjacency matrices, along with the Structural Hamming Distance \citep{acid2003searching}.

\subsection{Demos, documentation, and benchmark datasets}

The \textit{causal-learn} package also contains extensive usage examples of all search methods, (conditional) independence tests, score functions, and utilities at 
\\ \centerline{ \url{https://github.com/py-why/causal-learn/tree/main/tests}.} 
\\
Furthermore, detailed documentation is available at \\
\centerline{\url{https://causal-learn.readthedocs.io/en/latest/}.} \\
It is worth noting that it also includes a collection of well-tested benchmark datasets--since ground-truth causal relations are often unknown for real data, evaluation of causal discovery methods has been notoriously known to be hard, and we hope the availability of such benchmark datasets can help alleviate this issue and inspire the collection of more real-world datasets with (at least partially) known causal relations. 





\section{EVALUATION}
\section{Evaluation}
\label{sec:eval}

In this section, we first present our evaluation methodology and then describe performance results.

\subsection{Evaluation Methodology}
\label{ssec:eval-method}

\para{Network traces:} To understand the performance of our video enhancement approaches under diverse scenarios, we collect network traces over QUIC from WiFi, 3G, 4G, and 5G networks, shown in Table~\ref{tab:network_traces}. \kj{We use \textit{net-export}~\cite{net-export} in Chrome to collect the QUIC-related packets while watching Youtube videos. We especially identify packet loss in QUIC by capturing \textit{LOSS\_RETRANSMISSION} and \textit{PTO\_RETRANSMISSION} on the transmission type of the packet. Meanwhile, we measure the downlink throughput using iperf from an Azure server located in the central U.S. to a local client over the Internet. The 3G, 4G and 5G traces include static and walking scenarios. We also move the local client randomly to add mobility to the WiFi traces.}

% \kj{We also accumulate network traces from WiFi, 3G, 4G, and 5G over QUIC to show that our system is also indispensable in modern protocols. We use \textit{net-export}~\cite{net-export} in Chrome to collect the QUIC-related packets. We especially identify packet loss in QUIC by capturing \textit{LOSS\_RETRANSMISSION} and \textit{PTO\_RETRANSMISSION} on the transmission type of the packet. We run iperf to measure TCP/UDP throughput in January 2023.}

% We use 3G, 4G, and 5G traces from the existing works~\cite{mao2017neural,5G-measurement2}. We collect WiFi traces by running iperf from an Azure server to a local client over the Internet. The Azure server is located in the central U.S. and we move the local client randomly to add mobility for WiFi traces.
% We collect LEO traces in the StarLink network. We have access to a StarLink RV ground station in the west coast of the U.S.

% Figure environment removed

\begin{table}
  \centering
  \resizebox{0.5\columnwidth}{!}{%
  \begin{tabular}{c|c|c|c|c}
    \toprule
    & 3G & 4G & 5G & WiFi \\
    \midrule
    Amount & 45 & 62 & 53 & 68 \\
    Avg. Duration (s) & 322 & 317 & 302 & 309 \\
    Avg. Throughput (Mbps) & 7.5 & 21.6 & 36.4 & 82.3 \\
    Avg. Packet loss rate (\%) & 0.9 & 1.3 & 1.6 & 0.5 \\
    % Source & ~\cite{mao2017neural} & ~\cite{5G-measurement2} & ~\cite{5G-measurement2} & self-collected \\
    \bottomrule
  \end{tabular}
  }
  \vspace{10pt}
  \caption{Network traces}
  \label{tab:network_traces}
  % \vspace{-10pt}
\end{table}

\para{Video datasets:} We use the video dataset from NEMO for the evaluation. We choose videos from the top ten popular categories~\cite{medium-report} on YouTube: 'Product review', 'How-to', 'Vlogs', 'Game play', 'Skit', 'Haul', 'Challenges', 'Favorite', 'Education', and 'Unboxing'. From each category, we select five videos from distinct creators which support 4K at 30fps and are at least 5 minutes long. Then, four of them are distributed to the training dataset and the other belongs to the testing dataset. For adaptive streaming, we transcode them into multiple bitrate versions using the VP9 codec as per Wowza’s recommendation~\cite{wowza-recommendation}: \{512, 1024, 1600, 2640, 4400\} kbps at \{240, 360, 480, 720, 1080\}p resolutions. The GOP size is 120 (4 sec). 

\para{Performance metrics:} We use raw 1080p videos as a reference for measuring PSNR. We quantify the quality of recovered and super-resolved video frames using two widely used video quality metrics: SSIM and PSNR. Higher SSIM and PSNR values indicate better video quality. We quantify the performance of our system using QoE. A higher QoE indicates better video streaming for users. 


\subsection{DNN Performance} 

First, we evaluate the DNN performance in terms of video quality.

\para{DNN performance of video recovery: } Figure~\ref{fig:rc_dnn_quality} compares the video quality of simply reusing the previous video frame, predicting the video frame without the binary point code, and predicting using our binary point code. We use these schemes to predict the next 5, 10, 20, and 50 frames and calculate the average video quality. As we can see, video recovery without the binary point code yields 4-9dB PSNR improvement and 0.03-0.13 SSIM improvement over simple frame reuse; and the binary point code further increases PSNR by 6-12dB and increases SSIM by 0.04-0.17. The result shows the effectiveness of our binary point code. As we increase the number of future frames to predict, the prediction quality  using our recovery model degrades gracefully. 
% \zhaoyuan{Additionally, we evaluate the performance on the Macbook Air, which conducts warping at 1080p resolution. As a result, it shows superior performance compared to the iPhone 12, with an improvement of XXdB in PSNR and XX in SSIM.}

Figure~\ref{fig:vis_recovery} further illustrates the visualization of our video recovery results. As we can see, our recovery model can learn the motion movement between two consecutive frames and the recovered frames can closely resemble the ground truth frames. In addition, there is often a large difference between the previews frame and the current frame, and it can also be seen in this visualization that our model generates very reasonable predictions in regions where no reference can be found.

% Figure environment removed

Figure~\ref{fig:rc_cl_dnn_quality} further shows the partial video recovery results. We receive and decode video frames in a WiFi network environment. In this setting, many frames can only be partially decoded and our recovery model can recover these corrupted frames. We fill the decoded part of the frame into the recovered frame for all of the schemes such that the overall video quality is higher than the whole frame prediction. As we can see, our recovery without the binary point code yields 0.6-5dB PSNR improvement and 0.01-0.04 SSIM improvement over reusing the previous frame. The binary point code further increases PSNR by 4-8.5dB and increases SSIM by 0.04-0.06, respectively. 

Moreover, the gap between our video recovery without the binary point code and reusing the previous frame becomes larger because  $I_{part}$ allows the network to get an accurate hint to infer the content of the current frame. Similarly, the gap between the performance of our recovery with the binary point code and the other algorithms increases a lot compared to Figure~\ref{fig:rc_dnn_quality} likely because the model learns a stronger association between RGB frame content and binary point code in the successfully decoded part and better utilizes the learned binary code to generate predictions in the missing part.

Figure~\ref{fig:vis_conceal} plots the visualization of our error concealment results (\ie, recovery from partially corrupted frames). The recovered frames are also very similar to the original video frames. These results demonstrate the effectiveness of our video recovery for both completely lost or partially corrupted frames. 

% Figure environment removed

\para{DNN Performance of video super-resolution: } Figure~\ref{fig:sr_dnn_quality} compares the performance of our video super-resolution with upsampling. As we can see, our SR improves the PSNR and SSIM by 1.2dB, 1.1dB, 1dB, and 1.3dB; 0.015, 0.01, 0.007, and 0.008 at 240p, 360p, 480p, and 720p, respectively. The lower resolution video frames yield a higher improvement, as expected. Figure~\ref{fig:vis_sr} plots the visualization of video super-resolution results. Our proposed super-resolution algorithm delivers stable video frame quality improvement at all resolutions. % This confirms that our proposed super-resolution algorithm, which can run in real time on a mobile device, can consistently support the ABR algorithm to make better video streaming strategies. % this doesn't use ABR yet



% Figure environment removed

% Figure environment removed

\subsection{System Performance}

In this section, we evaluate the system performance in terms of video QoE. Note that we downscale the throughput for all network traces so that their throughput falls into the range between the highest and lowest video bit rates.  %  because adaptive streaming does not deliver any benefits at very high throughput. 
The average downscaled throughput among all the network traces is around 1-2Mbps. % To evaluate the performance under a lossy network environment, we use tc-netem as a Linux tool to impose packet loss on the network traffic using a 2-state Gilbert loss model, where the loss probability of the next packet depends on the previous state. % To emulate packet burst losses, we use the probability model, $Prob_{n} = p * Prob_{n-1} + (1 - p) * Random$, where each successive probability depends on a probability $p$ on the last one.

% \subsubsection{Video Recovery}\mbox{}\\

\para{QoE performance of video recovery: }To evaluate the QoE performance of video recovery, we consider three schemes: (i) without recovery model, (ii) without recovery-aware ABR, and (iii) our approach. Note that (ii) means we still perform video recovery for lost or late frames but select the bitrate without taking into account the benefits and cost of video recovery.  

Figure~\ref{fig:rc_only_qoe} shows the QoE performance of recovery-only schemes across different network traces. We make the following observations. First, video recovery alone improves the average QoE by 6.3\%, 11.2\%, 14.2\%, and 9.6\% in 3G, 4G, 5G, and WiFi, respectively, because it can recover lost and late frames such that the rebuffering time can be effectively reduced. 

Second, our recovery-aware algorithm improves over without recovery by 8.6\%, 18.3\%, 22.8\%, and 14.5\% in 3G, 4G, 5G, and WiFi, respectively, and improves over recovery alone by 2.2\%, 6.4\%, 7.5\%, and 4.5\% in 3G, 4G, 5G, and WiFi, respectively because it is aware of the usage of recovery for the next frames such that the bitrate can be chosen more wisely to maximize the system QoE. 

Third, comparing the results across different types of networks, we observe 5G enjoys the largest improvement because more frames require video recovery as we will show next.

% Figure environment removed 

Figure~\ref{fig:throughput} shows the average downscaled throughput of different network traces. We see a large fluctuation in 5G traces. Figure~\ref{fig:rc_percentage} reports the percentage of recovered frames. As 5G has the largest throughput fluctuation, many video frames are not received in time and require video recovery. Meanwhile, even 4G and WiFi see close to 10\% or more video frames that require video recovery. These numbers suggest that video recovery is important due to challenging network conditions. Figure~\ref{fig:sample_thrp} further shows a sample time series of throughput. We find that the scheme without recovery cannot sustain a good QoE when the throughput varies a lot. Recovery alone has a more stable QoE but sometimes gets below 0 due to the rebuffering overhead. Our approach always chooses the bitrate that yields the best QoE. 

% Figure environment removed

Table~\ref{tab:qoe_rc_frames} reports the average QoE of the recovered video frames only. Video recovery alone improves the QoE for the recovered frames by 1.26 - 10.65. The improvement comes mostly from reduced rebuffering time. 
% [XXX: double check; add break down of the improvement from 3 terms in QoE] 
Incorporating recovery-aware ABR further increases the QoE by 0.25 - 1.4. 
% [XXX: do we select a higher or lower rate in RC-aware ABR]

\para{QoE performance without FEC under lossy networks: } Figure~\ref{fig:rc_lossy_qoe} shows the QoE performance of recovery-only schemes across different network traces. Under this setting, we do not enable FEC for loss recovery. For (i), we reuse the last frame when a video frame is late or lost. For (ii) and (iii), our recovery model recovers both lost frames and late frames. Under a lossy network environment, we observe that video recovery alone improves the average QoE by 58.9\%, 74.3\%, 82.7\%, and 70.6\% in 3G, 4G, 5G, and WiFi, respectively. Our approach improves over that without recovery by 71.8\%, 90.8\%, 110\%, and 84.3\% in 3G, 4G, 5G, and WiFi, respectively, and improves over recovery alone by 8.1\%, 9.5\%, 14.6\%, and 8\% in 3G, 4G, 5G, and WiFi, respectively. We find that the improvement of our approach over baselines increases a lot under the lossy network environment because reusing the previous frames is not effective under many consecutive lost/late frames (as shown in Figure~\ref{fig:rc_dnn_quality}), which is more likely under a lossy environment. However, our recovery model can recover many frames with little degradation so its QoE performance is much better. 

% Figure environment removed

\para{QoE performance with FEC under lossy networks: } So far, we disable FEC in our algorithm. Next we further jointly optimize FEC and video recovery. Figure~\ref{fig:rc_fec_lossy_qoe} compares our algorithm but disable FEC (w/o FEC) with all other schemes with FEC, where the amount of FEC is determined based on our lookup table. We offline build separate lookup tables that map the packet loss rates to desired FEC levels for different schemes. Our joint optimization yields 51\%, 68\%, 83\%, and 72\% improvement over no recovery in 3G, 4G, 5G, and WiFi, respectively. The corresponding improvements over recovery alone are 13\%, 41\%, 48\%, and 31\%, respectively. Also, it outperforms no FEC by 1.2, 1.15, 1.3, and 1.23 in QoE, respectively. These results show that (i) FEC plays an important role under lossy network conditions, (ii) the desired amount of FEC depends on the recovery and ABR algorithms, and (iii) each component in our recovery model (\ie, recovery alone, recovery-aware, and joint optimization of FEC and recovery) is beneficial. 

% Without FEC performs worst and acquires negative average QoEs under different network traces. 

% We further adaptively Our recovery model can still get benefits under lossy networks because we can recover the lost frames without retransmission to avoid much rebuffering overhead. Figure~\ref{fig:qoe_redundant} indicates that there is always a peek where we can get the best QoE with the combination of FEC and our recovery model. To this end, we build a lookup table to get the best FEC redundant ratio under different packet loss rates and different network traces. Figure~\ref{fig:rc_lossy_qoe} shows the QoE performance across different network traces. 

% \subsubsection{Video Super-Resolution}\mbox{}\\

\para{QoE performance of video super-resolution: }Figure~\ref{fig:sr_only_qoe} compares the QoE of our super-resolution with (i) without SR, (ii) SR alone using our model, and (iii) NEMO~\cite{yeo2020nemo}.  Our SR-aware approach significantly outperforms all the other algorithms. Its improvement over (i), (ii), and (iii) are 18\%, 21\%, 22\%, and 19\%; 4.5\%, 6.5\%, 7.1\%, and 4.5\%; 0.7\%, 3.8\%, 4.5\%, and 2.7\% in 3G, 4G, 5G, and WiFi, respectively. SR alone brings 12\%-14\% improvement, and SR-aware ABR further brings 4\%-7\% improvement, which shows the importance of joint design of ABR and SR. % It is interesting that the improvement of SR-aware ABR is so large. [XXX: zoom in what video rates are selected. do we select higher or lower rates?] Our approach out-performs NEMO, the state of the art because XXX. 

% Figure environment removed

% \subsubsection{Video Recovery and Super-Resolution}\mbox{}\\

\para{QoE performance of video recovery and super-resolution: } Figure~\ref{fig:sr_rc_qoe} compares the QoE of (i) without SR or recovery, (ii) SR and recovery alone, (iii) NEMO, and (iv) our final algorithm. Our algorithm out-performs (i), (ii), and (iii) by 23.7\%, 32.2\%, 37.1\%, and 29\%; 5.9\%, 10\%, 11.9\%, and 8.4\%; 4.7\%, 13.2\%, 17.4\%, and 10.9\% in 3G, 4G, 5G, and WiFi, respectively. It can be found that both SR and Recovery play a significant effect, and combined with our enhancement aware ABR strategy, our method achieves the best performance. It out-performs NEMO by 4.7\%-17.4\%, and even SR and Recovery alone out-perform NEMO in all cases except 3G, because NEMO does not have recovery and has to reuse the previous frames for late or lost video frames. 3G is better for NEMO due to fewer lost/late video frames.

\zhaoyuan{
\subsection{System Latency and Resource Usage}

\para{System latency: } At the start of video streaming, we establish TCP and QUIC transmission sessions. The binary code, with a size of 1KB, can be encapsulated into a single TCP packet. Consequently, the TCP latency for each frame is expected to be approximately equivalent to the round-trip time (RTT). Given that the decoding and model inference processes of a frame can occur simultaneously with the receiving process of subsequent frames, the total latency can be viewed as the sum of the decoding time and the duration required for neural enhancement or recovery. The decoding time of 240p, 360p, 480p, 720p, and 1080p videos is 1.8, 2.3, 2.9, 4.1, and 6.2ms on the iPhone 12, respectively. Our model adds an additional 22ms for both enhancement and recovery, regardless of the video resolution. This results in a total latency of under 33 ms, demonstrating real-time processing capability in our system.

\para{CPU usage and energy consumption: } We also measure the CPU utilization and energy consumption with and without our model. We only evaluate the neural video recovery because both video recovery and enhancement share a similar model structure and exhibit identical inference time. This similarity implies comparable CPU usage and energy consumption between the two models. Without DNN processing, iPhone 12's CPU utilization is 28\% and the energy consumption is 0.04J per frame. Under 20\% frame losses, the corresponding numbers are 37\% and 0.05J, and under 100\% frame losses, they are 68\% and 0.07J. Consequently, with each frame undergoing neural recovery or enhancement, the expected battery life decreases from 13.2 hours to 7.5 hours.

}


\section{DISCUSSION $\And$ FUTURE WORKS}
\section{Discussion}
\label{sec: discussion}
\kmsdelete{In this work} We study \kmsreplace{Fairness-Aware PAC learning}{Fair-ERM} in the malicious noise model, and  in some cases allow 
the learner to maintain optimal overall accuracy despite the signal in Group $B$ being almost entirely washed out.
%when we allow learners to use the
%$\PQ$ randomized expansion of the hypothesis class $\mathcal{H}$
In particular we show that different fairness constraints have fundamentally different behavior in the presence of Malicious Noise, in terms of the amount of accuracy loss that a given level of Malicious Noise could cause a fairness-constrained learner to incur. 
The key to achieving our results, which are more optimistic than those in \cite{lampert}, is allowing for improper learners using the (P,Q)-randomized expansions of the given class $\mathcal{H}$.
%We \kmsreplace{present a picture of the}{prove upper and lower bounds on}
%accuracy loss for a range of fairness notions, given \kmsreplace{this simple randomization step.}{learning over $\PQ$.
%In general our results indicate Fair-ERM (given learning over $\PQ$) is more robust than claimed in \cite{lampert}.
The type of smoothness we create by using $\PQ$ seems to be a natural property that is likely shared by many natural hypothesis classes.

Fairness notions are motivated as a response to learned disparities when there is \kmsdelete{data corruption or} systemic error affecting \kmsdelete{the data for}
one group. 
Fairness notions are supposed to mitigate this by ruling out classifiers that have worse performance on a sub-group. 
This can peg both classifiers at a lower level of performance \kmsdelete{(e.g that the lower subgroup)} in order to \emph{motivate} \cite{hardt16} improving the data collection or labelling process to obtain more reliable performance. 
%So in \kmsreplace{some}{a} sense, sensitivity of the fairness notion to poor sub-group performance caused by malicious noise is the \textit{point} of fairness constraints! 
However, it also desirable that fairness constraints perform gracefully when subject to Malicious Noise because fairness constraints will be used in contexts where the data is unreliable and noisy and this might not be known to the learner.
This tension, exposed by our work, motivates 
%a revisiting of fairness notions from first principles approach and trying to axiomatize the 
%desired properties of a fairness intervention a la cryptography and privacy. \footnote{Work in multi-calibration \cite{multicalib} is a viable direction for this problem but it is unclear how 
%that and related notions behave with unreliable data. }
on going work studying the sensitivity level of fairness constraints. 
%If we we are to take a view, if a classifier is deployed 

\section{CONCLUSIONS}
 \section{Conclusion and Future Work}
In this work, I design corruption-robust algorithms for the Lipschitz contextual search problem. I present the \emph{agnostic checking} technique and demonstrate its effectiveness in designing corruption-robust algorithms. There are several open problems for future research. First, in the algorithm I propose for pricing loss, the schedule for agnostic checks is fixed upfront. Can the learner design an adaptive checking schedule for the pricing loss? Second, this work assumes the learner has knowledge of the Lipschitz constant $L$. Can the learner design efficient no-regret algorithms without knowledge of $L$? 


\section{ACKNOWLEDGMENT}
    \section*{Acknowledgements}%

The authors express their gratitude and a fond thought to Hassan \ak, who
with Gabriella Pasi set out to define a fuzzy version of OSF logic. This
paper originates from their work on the definition of similarity-based
unification for OSF terms, extending the approach of
\cite{AitKaciPasi2020}.


% \addtolength{\textheight}{-12cm}   % This command serves to balance the column lengths
                                  % on the last page of the document manually. It shortens
                                  % the textheight of the last page by a suitable amount.
                                  % This command does not take effect until the next page
                                  % so it should come on the page before the last. Make
                                  % sure that you do not shorten the textheight too much.

%%%%%%%%%%%%%%%%%%%%%%%%%%%%%%%%%%%%%%%%%%%%%%%%%%%%%%%%%%%%%%%%%%%%%%%%%%%%%%%%



%%%%%%%%%%%%%%%%%%%%%%%%%%%%%%%%%%%%%%%%%%%%%%%%%%%%%%%%%%%%%%%%%%%%%%%%%%%%%%%%



%%%%%%%%%%%%%%%%%%%%%%%%%%%%%%%%%%%%%%%%%%%%%%%%%%%%%%%%%%%%%%%%%%%%%%%%%%%%%%%%
% \section*{APPENDIX}



% \section*{ACKNOWLEDGMENT}

\bibliographystyle{IEEEtran}
\bibliography{ref}

\end{document}
