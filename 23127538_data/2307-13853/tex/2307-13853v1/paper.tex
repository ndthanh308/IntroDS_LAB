%% Beginning of file 'sample631.tex'
%%
%% Modified 2022 May  
%%
%% This is a sample manuscript marked up using the
%% AASTeX v6.31 LaTeX 2e macros.
%%
%% AASTeX is now based on Alexey Vikhlinin's emulateapj.cls 
%% (Copyright 2000-2015).  See the classfile for details.

%% AASTeX requires revtex4-1.cls and other external packages such as
%% latexsym, graphicx, amssymb, longtable, and epsf.  Note that as of 
%% Oct 2020, APS now uses revtex4.2e for its journals but remembehttps://www.overleaf.com/project/63c1b0de7d91a0d26a68f06ar that 
%% AASTeX v6+ still uses v4.1. All of these external packages should 
%% already be present in the modern TeX distributions but not always.
%% For example, revtex4.1 seems to be missing in the linux version of
%% TexLive 2020. One should be able to get all packages from www.ctan.org.
%% In particular, revtex v4.1 can be found at 
%% https://www.ctan.org/pkg/revtex4-1.

%% The first piece of markup in an AASTeX v6.x document is the \documentclass
%% command. LaTeX will ignore any data that comes before this command. The 
%% documentclass can take an optional argument to modify the output style.
%% The command below calls the preprint style which will produce a tightly 
%% typeset, one-column, single-spaced document.  It is the default and thus
%% does not need to be explicitly stated.
%%
%% using aastex version 6.3
\documentclass{aastex631}

%% The default is a single spaced, 10 point font, single spaced article.
%% There are 5 other style options available via an optional argument. They
%% can be invoked like this:
%%
%% \documentclass[arguments]{aastex631}
%% 
%% where the layout options are:
%%
%%  twocolumn   : two text columns, 10 point font, single spaced article.
%%                This is the most compact and represent the final published
%%                derived PDF copy of the accepted manuscript from the publisher
%%  manuscript  : one text column, 12 point font, double spaced article.
%%  preprint    : one text column, 12 point font, single spaced article.  
%%  preprint2   : two text columns, 12 point font, single spaced article.
%%  modern      : a stylish, single text column, 12 point font, article with
%% 		  wider left and right margins. This uses the Daniel
%% 		  Foreman-Mackey and David Hogg design.
%%  RNAAS       : Supresses an abstract. Originally for RNAAS manuscripts 
%%                but now that abstracts are required this is obsolete for
%%                AAS Journals. Authors might need it for other reasons. DO NOT
%%                use \begin{abstract} and \end{abstract} with this style.
%%
%% Note that you can submit to the AAS Journals in any of these 6 styles.
%%
%% There are other optional arguments one can invoke to allow other stylistic
%% actions. The available options are:
%%
%%   astrosymb    : Loads Astrosymb font and define \astrocommands. 
%%   tighten      : Makes baselineskip slightly smaller, only works with 
%%                  the twocolumn substyle.
%%   times        : uses times font instead of the default
%%   linenumbers  : turn on lineno package.
%%   trackchanges : required to see the revision mark up and print its output
%%   longauthor   : Do not use the more compressed footnote style (default) for 
%%                  the author/collaboration/affiliations. Instead print all
%%                  affiliation information after each name. Creates a much 
%%                  longer author list but may be desirable for short 
%%                  author papers.
%% twocolappendix : make 2 column appendix.
%%   anonymous    : Do not show the authors, affiliations and acknowledgments 
%%                  for dual anonymous review.
%%
%% these can be used in any combination, e.g.
%%
%% \documentclass[twocolumn,linenumbers,trackchanges]{aastex631}
%%
%% AASTeX v6.* now includes \hyperref support. While we have built in specific
%% defaults into the classfile you can manually override them with the
%% \hypersetup command. For example,
%%
%% \hypersetup{linkcolor=red,citecolor=green,filecolor=cyan,urlcolor=magenta}
%%
%% will change the color of the internal links to red, the links to the
%% bibliography to green, the file links to cyan, and the external links to
%% magenta. Additional information on \hyperref options can be found here:
%% https://www.tug.org/applications/hyperref/manual.html#x1-40003
%%
%% Note that in v6.3 "bookmarks" has been changed to "true" in hyperref
%% to improve the accessibility of the compiled pdf file.
%%
%% If you want to create your own macros, you can do so
%% using \newcommand. Your macros should appear before
%% the \begin{document} command.
%%
\newcommand{\vdag}{(v)^\dagger}
\newcommand\aastex{AAS\TeX}
\newcommand\latex{La\TeX}
\usepackage{subfigure}
\usepackage{natbib}

%% Reintroduced the \received and \accepted commands from AASTeX v5.2
%\received{March 1, 2021}
%\revised{April 1, 2021}
%\accepted{\today}

%% Command to document which AAS Journal the manuscript was submitted to.
%% Adds "Submitted to " the argument.
%\submitjournal{PSJ}

%% For manuscript that include authors in collaborations, AASTeX v6.31
%% builds on the \collaboration command to allow greater freedom to 
%% keep the traditional author+affiliation information but only show
%% subsets. The \collaboration command now must appear AFTER the group
%% of authors in the collaboration and it takes TWO arguments. The last
%% is still the collaboration identifier. The text given in this
%% argument is what will be shown in the manuscript. The first argument
%% is the number of author above the \collaboration command to show with
%% the collaboration text. If there are authors that are not part of any
%% collaboration the \nocollaboration command is used. This command takes
%% one argument which is also the number of authors above to show. A
%% dashed line is shown to indicate no collaboration. This example manuscript
%% shows how these commands work to display specific set of authors 
%% on the front page.
%%
%% For manuscript without any need to use \collaboration the 
%% \AuthorCollaborationLimit command from v6.2 can still be used to 
%% show a subset of authors.
%
%\AuthorCollaborationLimit=2
%
%% will only show Schwarz & Muench on the front page of the manuscript
%% (assuming the \collaboration and \nocollaboration commands are
%% commented out).
%%
%% Note that all of the author will be shown in the published article.
%% This feature is meant to be used prior to acceptance to make the
%% front end of a long author article more manageable. Please do not use
%% this functionality for manuscripts with less than 20 authors. Conversely,
%% please do use this when the number of authors exceeds 40.
%%
%% Use \allauthors at the manuscript end to show the full author list.
%% This command should only be used with \AuthorCollaborationLimit is used.

%% The following command can be used to set the latex table counters.  It
%% is needed in this document because it uses a mix of latex tabular and
%% AASTeX deluxetables.  In general it should not be needed.
%\setcounter{table}{1}

%%%%%%%%%%%%%%%%%%%%%%%%%%%%%%%%%%%%%%%%%%%%%%%%%%%%%%%%%%%%%%%%%%%%%%%%%%%%%%%%
%%
%% The following section outlines numerous optional output that
%% can be displayed in the front matter or as running meta-data.
%%
%% If you wish, you may supply running head information, although
%% this information may be modified by the editorial offices.
%\shorttitle{AASTeX v6.3.1 Sample article}
%\shortauthors{Schwarz et al.}
%%
%% You can add a light gray and diagonal water-mark to the first page 
%% with this command:
%% \watermark{text}
%% where "text", e.g. DRAFT, is the text to appear.  If the text is 
%% long you can control the water-mark size with:
%% \setwatermarkfontsize{dimension}
%% where dimension is any recognized LaTeX dimension, e.g. pt, in, etc.
%%
%%%%%%%%%%%%%%%%%%%%%%%%%%%%%%%%%%%%%%%%%%%%%%%%%%%%%%%%%%%%%%%%%%%%%%%%%%%%%%%%
%\graphicspath{{./}{figures/}}
%% This is the end of the preamble.  Indicate the beginning of the
%% manuscript itself with \begin{document}.

\begin{document}

\title{ Assessing the Accuracy of TESS Asteroseismology with APOGEE }


\author[0009-0000-7811-0726]{Artemis Theano Theodoridis}
\affiliation{Department of Astronomy, University of Florida, USA }


\author[0000-0002-4818-7885]{Jamie Tayar}
\affiliation{Department of Astronomy, University of Florida,  USA }






%% Mark off the abstract in the ``abstract'' environment. 
\begin{abstract}

The recent NASA TESS mission has the potential to increase the available asteroseismic sample dramatically, but its precision and accuracy have yet to be confirmed. To date, NASA's Kepler mission has been considered the gold standard for asteroseismic samples, despite data only being available for a small portion of the sky. TESS’s observations cover the whole sky, and previous work has identified 158,000 potential red giant oscillators. Using APOGEE, which is calibrated to the asteroseismic scale of the Kepler data, we show that seismology from TESS is calibrated to the Kepler scale to better than 5\% for about 90\% of red giants, and has only slight trends with mass, metallicity, and surface gravity. We therefore conclude that current TESS seismic results can already be used for galactic archaeology, and future results are likely to be highly transformational to our understanding. 




\end{abstract}



\section{Introduction} \label{sec:intro}
There has been much work done to understand the history and evolution of our Milky Way galaxy. Despite these successes, there remains a need for exact stellar ages for the individual single red giants. However, most standard methods, including measuring isochrones and analyzing chemistry, are not sufficiently precise. Asteroseismology, the study of oscillations, has shown promise in providing both precise and accurate stellar ages \citep{Zinn2022}. 
Unfortunately, previous analyses have been limited to small fields in the sky. With TESS, an all-sky survey including millions of red giants, and at least 158,000 identified oscillators, limited sky coverage is no longer a great concern \citep{Hon2021}. As exciting as this survey is, it is unclear how accurate its measurements are due to the shorter data collection period and the inclusion of machine learning in the seismic identification. Therefore, it is imperative to review these previously unconfirmed measurements. To validate its accuracy, it must be compared to the seismic scale using data that has already been thoroughly reviewed. We use spectroscopic results from the APOGEE survey for this purpose. 


\section{Methods} \label{sec:style}
	We refer to the APOGEE (Apache Point Observatory Galactic Evolution Experiment) \citep{Majewski2017} Data Release 17 \citep{Abdurro'uf2022}, of the Sloan Digital Sky Survey \citep{Blanton2017} project, whose purpose is to identify an archaeological record of the Milky Way galaxy through the collection of chemical abundances and radial velocities \citep{Santana2021}. APOGEE collected near-infrared spectra using the APOGEE spectrographs \citep{Wilson_2019}. The spectra were processed with an automated pipeline  \citep{Garcia2016} using Synspec atmosphere grids. Results were calibrated using previous asteroseismic results, open clusters, and low extinction fields \citep{Jonsson2020}, and stars with poor results were flagged or eliminated. 
 The data we use from TESS comes from the asteroseismic analysis of \citet{Hon2021}. That study used machine learning to analyze long-cadence photometry from TESS taken at 30-minute intervals for one month's duration (27 days). That analysis identified potential oscillations in 158,000 red giant stars. For comparison, previous work has identified $\sim$20,000 red giant oscillators from Kepler (4-year duration), $\sim$20,000 red giants from K2 ($\sim$70-day duration), and 1800 Red giants in CoRot fields ($\sim$1-3 month duration). 
 
 Therefore, the TESS sample represents a significant potential increase in the available targets compared to previous work. 
Using the asteroseismic results, and the spectroscopic data, we created a file containing values from TESS and those from APOGEE in a similar fashion to APOKASC. Specifically, we used the correspondence between TIC\_ID and 2MASSID from TIC-v7 \citep{Stassun2019}, eliminating any stars that did not have a matching counterpart in the other dataset.  Following this process, we found 15018 matches. We added to our table calculations of seismic log(g) and mass in a similar fashion to \citet{Hon2021}, using GAIA DR2 radius \citep{Gaia2018}, APOGEE DR17 temperature, and $\nu_{\rm max}$. Once we calculated the seismic log(g) and mass, we used standard equations for error propagation to calculate errors for both seismic log(g) and mass. The catalog can be viewed at this link: \href{https://zenodo.org/record/7814297} {https://zenodo.org/record/7814297} %{APOGEE Verification}

%FIGURE Jamie Junk Figure---------------------------------------------------

% Figure environment removed
%FIGURE omgcomp---------------------------------------------------



\section{Analysis} \label{sec:Analysis}
Here we rely on the transitive property. We know that the APOGEE spectroscopic surface gravity was calibrated on the Kepler asteroseismic gravity scale. Therefore, we assume that if the TESS surface gravity matches the surface gravity from APOGEE, they must be consistent with the Kepler scale. In general, we found this to be the case; there is agreement to better than 5\%, signifying a very accurate calibration.
 We searched for possible dependencies of the offset between the asteroseismic and spectroscopic surface gravities on metallicity, [$\alpha/$M], log(g), temperature, and mass. (Figure \ref{Fig:NoTrends}). We found evidence for slight potential trends with mass, metallicity, and surface gravity. Previous work has identified significant inconsistencies in the asteroseismic results in surface gravity as a function of metallicity \citep{Epstein2014}; here, we see at most a slight trend. We note that the largest offsets that we discovered tended to be for the clump stars. In Figure \ref{Fig:NoTrends}, we show histograms of the difference between the TESS asteroseismic and APOGEE spectroscopic gravities scaled by the claimed uncertainty for each star. Given that this is significantly wider than the standard normal distribution, we assert that the predicted uncertainty is underestimated for the majority of stars. Comparisons between the Kepler and APOGEE data indicate that the spectroscopic uncertainties in DR17 may be underestimated by at least a factor of two (M. Pinsonneault et al., in prep.), but this would not be sufficient to match the observed differences in our sample, suggesting that there are additional uncertainties that are also underestimated. 
 %further underestimates of the uncertainties remain to be identified. 



 
\section{Conclusion} \label{sec:Conclusion}
Our analysis indicates that the asteroseismic results from \citet{Hon2021} are reliable. For more than 98\% of stars, the inferred asteroseismic gravity matches the spectroscopic gravities to better than 10\%, suggesting that true giants have been identified. The precision for $\nu_{\rm max}$ is also good -- for 90\% of stars, the agreement on the inferred log(g) is better than 5\% ($\sim 0.05$ dex in log(g)). We observe slight differences between the spectroscopic and asteroseismic gravities that correlate with mass, log(g), and [M/H], but no significant offsets. We do note that there is a slightly larger offset on average for clump stars, which we attribute to one of the following:
1. A slight failure rate in correction identifying the stars' evolutionary state using only spectroscopic information, which is required to precisely calibrate the spectra to the Kepler asteroseismic scale.
2. Slight evolutionary state-dependent errors in the correction factor applied to the APOGEE spectroscopic gravities
3. Evolutionary state-dependent challenges in correctly pinpointing $\nu_ {\rm max}$ in short data sets using only a neural network. 
%The average claimed spectroscopic uncertainties are 0.02 dex from apogee rx, which we believe to be underestimated, while the average propogated error on the seismic uncertainty is BLANK, which we believe is also underestimated. The amount to which either spectroscopic or seismic uncertainties affect the overall error in estimation is unknown but should be investigated in future works for clarity. 
Given these results, we encourage both future investigations into the precise details of TESS asteroseismology of red giants, as well as usage of the current TESS results for galactic archaeology purposes.




\begin{acknowledgements} \label{sec:Acknowledgements}

NASA grants 80NSSC23K0143 and 80NSSC23K0436. We utilize data from SDSS 

(https://www.sdss.org/collaboration/citing-sdss/)

A.T Thanks Georgios and Hatice Theodoridis for helpful discussions
\end{acknowledgements}

\bibliography{paper.bbl}{}
\bibliographystyle{aasjournal}


\end{document}









%% This command is needed to show the entire author+affiliation list when
%% the collaboration and author truncation commands are used.  It has to
%% go at the end of the manuscript.
%\allauthors

%% Include this line if you are using the \added, \replaced, \deleted
%% commands to see a summary list of all changes at the end of the article.
%\listofchanges

% End of file `sample631.tex'.
