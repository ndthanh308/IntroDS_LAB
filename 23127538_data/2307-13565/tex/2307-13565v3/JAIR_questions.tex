\documentclass[12pt]{article}
\usepackage[paper=letterpaper,margin=2cm]{geometry}
\usepackage{amssymb, enumitem} % for '\blacksquare' macro
\newenvironment{answer}{%
     \setlength\parindent{0pt}\par\medskip\textbf{Answer}\quad}{%
     \hfill\tiny$\blacksquare$\par\medskip}
\newlist{questions}{enumerate}{1}
\setlist[questions]{label=\arabic*., wide=0pt, font=\bfseries}
\let\question=\item
\begin{document}

\begin{questions}
 \question   Please state in no more than 150 words why your work is important to other researchers in Artificial Intelligence and how they could make use of your results.

    
\begin{answer}
    In our survey article, we offer a comprehensive review of decision-focused learning (DFL) that caters to diverse readers, including those unfamiliar with the topic.  
    DFL is an emerging paradigm in machine learning
which directly trains a model to optimize decisions, integrating prediction and optimization
in an end-to-end system. This paradigm holds the promise to revolutionize 
decision-making in many real-world applications which operate under uncertainty.
This review is designed to offer a clear understanding of DFL concepts, methodologies, and challenges, making it a valuable resource for both researchers and practitioners. 
To achieve this, we introduce foundational DFL concepts, review various DFL methodologies, propose a classification taxonomy, and benchmark the performances of available methodologies on distinct problems, identifying their relative strengths and weaknesses.  Through these efforts, we contribute to the dissemination of the newly emerged DFL paradigm while aligning perfectly with JAIR's mission to share new research and insights with the global AI community. 

\end{answer}
\question Please list 1 - 3 papers previously published in JAIR that are closest to your work, and explain in no more than 150 words how your work differs from those papers. If you consider no previous articles in JAIR to be sufficiently close to your work, please state this and instead list a previous JAIR publication that has a similar structure to your submission. Please note that articles with little similarity in content or structure to published JAIR articles have a high chance of rejection without review.




\begin{answer}
To the best of our knowledge, no published paper in JAIR has implemented DFL. This is because of the recent prominence of DFL as a topic, with a majority of DFL papers appearing in conference proceedings. DFL, a novel technique, has emerged to address the challenges of decision-making under uncertainty. Several papers in JAIR [1,2 to name a few] focus on decision-making under uncertainty by optimizing probabilistic performance measures, assuming a distribution for the data. In contrast, DFL estimates uncertain quantities through machine learning (ML). The work closest to DFL is presented in [3], where ML is utilized to estimate uncertain quantities before optimization. DFL takes a step further by integrating ML with optimization.

[1] De Filippo, Allegra, Michele Lombardi, and Michela Milano. "Integrated offline and online decision making under uncertainty." Journal of Artificial Intelligence Research 70 (2021)

[2] Song, Wen, et al. "A sampling approach for proactive project scheduling under generalized time-dependent workability uncertainty." Journal of Artificial Intelligence Research 64 (2019)

[3] Krivic, Senka, et al. "Using machine learning for decreasing state uncertainty in planning." Journal of Artificial Intelligence Research 69 (2020)
\end{answer}
\question If any part of this paper has been previously published or is under review, please state where and explain how the current paper differs. If not, please state "No" as the response. If this is a resubmission to JAIR, please also provide here the original submission number and the name of the Associate Editor who previously handled it. This will make it easier for us to use (whenever possible), the same AE and reviewers, to ensure consistency of the reviewing process.
\end{questions}
\begin{answer}
No
\end{answer}
\end{document}