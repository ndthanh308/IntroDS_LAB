
%% Beginning of file 'sample63.tex'
%%
%% Modified 2019 June
%%
%% This is a sample manuscript marked up using the
%% AASTeX v6.3 LaTeX 2e macros.
%%
%% AASTeX is now based on Alexey Vikhlinin's emulateapj.cls 
%% (Copyright 2000-2015).  See the classfile for details.

%% AASTeX requires revtex4-1.cls (http://publish.aps.org/revtex4/) and
%% other external packages (latexsym, graphicx, amssymb, longtable, and epsf).
%% All of these external packages should already be present in the modern TeX 
%% distributions.  If not they can also be obtained at www.ctan.org.

%% The first piece of markup in an AASTeX v6.x document is the \documentclass
%% command. LaTeX will ignore any data that comes before this command. The 
%% documentclass can take an optional argument to modify the output style.
%% The command below calls the preprint style which will produce a tightly 
%% typeset, one-column, single-spaced document.  It is the default and thus
%% does not need to be explicitly stated.
%%
%%
%% using aastex version 6.3
\documentclass[manuscript]{aastex63}
%\documentclass[twocolumn]{aastex63}

%% The default is a single spaced, 10 point font, single spaced article.
%% There are 5 other style options available via an optional argument. They
%% can be invoked like this:
%%
%% \documentclass[arguments]{aastex63}
%% 
%% where the layout options are:
%%
%%  twocolumn   : two text columns, 10 point font, single spaced article.
%%                This is the most compact and represent the final published
%%                derived PDF copy of the accepted manuscript from the publisher
%%  manuscript  : one text column, 12 point font, double spaced article.
%%  preprint    : one text column, 12 point font, single spaced article.  
%%  preprint2   : two text columns, 12 point font, single spaced article.
%%  modern      : a stylish, single text column, 12 point font, article with
%% 		  wider left and right margins. This uses the Daniel
%% 		  Foreman-Mackey and David Hogg design.
%%  RNAAS       : Preferred style for Research Notes which are by design 
%%                lacking an abstract and brief. DO NOT use \begin{abstract}
%%                and \end{abstract} with this style.
%%
%% Note that you can submit to the AAS Journals in any of these 6 styles.
%%
%% There are other optional arguments one can invoke to allow other stylistic
%% actions. The available options are:
%%
%%   astrosymb    : Loads Astrosymb font and define \astrocommands. 
%%   tighten      : Makes baselineskip slightly smaller, only works with 
%%                  the twocolumn substyle.
%%   times        : uses times font instead of the default
%%   linenumbers  : turn on lineno package.
%%   trackchanges : required to see the revision mark up and print its output
%%   longauthor   : Do not use the more compressed footnote style (default) for 
%%                  the author/collaboration/affiliations. Instead print all
%%                  affiliation information after each name. Creates a much 
%%                  longer author list but may be desirable for short 
%%                  author papers.
%% twocolappendix : make 2 column appendix.
%%   anonymous    : Do not show the authors, affiliations and acknowledgments 
%%                  for dual anonymous review.
%%
%% these can be used in any combination, e.g.
%%
%% \documentclass[twocolumn,linenumbers,trackchanges]{aastex63}
%%
%% AASTeX v6.* now includes \hyperref support. While we have built in specific
%% defaults into the classfile you can manually override them with the
%% \hypersetup command. For example,
%%
%% \hypersetup{linkcolor=red,citecolor=green,filecolor=cyan,urlcolor=magenta}
%%
%% will change the color of the internal links to red, the links to the
%% bibliography to green, the file links to cyan, and the external links to
%% magenta. Additional information on \hyperref options can be found here:
%% https://www.tug.org/applications/hyperref/manual.html#x1-40003
%%
%% Note that in v6.3 "bookmarks" has been changed to "true" in hyperref
%% to improve the accessibility of the compiled pdf file.
%%
%% If you want to create your own macros, you can do so
%% using \newcommand. Your macros should appear before
%% the \begin{document} command.
%%
%\newcommand{\vdag}{(v)^\dagger}
%\newcommand\aastex{AAS\TeX}
%\newcommand\latex{La\TeX}

%% Reintroduced the \received and \accepted commands from AASTeX v5.2
%\received{June 1, 2019}
%\revised{January 10, 2019}
%\accepted{\today}
%% Command to document which AAS Journal the manuscript was submitted to.
%% Adds "Submitted to " the argument.
\submitjournal{The Astrophysical Journal}

%% For manuscript that include authors in collaborations, AASTeX v6.3
%% builds on the \collaboration command to allow greater freedom to 
%% keep the traditional author+affiliation information but only show
%% subsets. The \collaboration command now must appear AFTER the group
%% of authors in the collaboration and it takes TWO arguments. The last
%% is still the collaboration identifier. The text given in this
%% argument is what will be shown in the manuscript. The first argument
%% is the number of author above the \collaboration command to show with
%% the collaboration text. If there are authors that are not part of any
%% collaboration the \nocollaboration command is used. This command takes
%% one argument which is also the number of authors above to show. A
%% dashed line is shown to indicate no collaboration. This example manuscript
%% shows how these commands work to display specific set of authors 
%% on the front page.
%%
%% For manuscript without any need to use \collaboration the 
%% \AuthorCollaborationLimit command from v6.2 can still be used to 
%% show a subset of authors.
%
%\AuthorCollaborationLimit=2
%
%% will only show Schwarz & Muench on the front page of the manuscript
%% (assuming the \collaboration and \nocollaboration commands are
%% commented out).
%%
%% Note that all of the author will be shown in the published article.
%% This feature is meant to be used prior to acceptance to make the
%% front end of a long author article more manageable. Please do not use
%% this functionality for manuscripts with less than 20 authors. Conversely,
%% please do use this when the number of authors exceeds 40.
%%
%% Use \allauthors at the manuscript end to show the full author list.
%% This command should only be used with \AuthorCollaborationLimit is used.

%% The following command can be used to set the latex table counters.  It
%% is needed in this document because it uses a mix of latex tabular and
%% AASTeX deluxetables.  In general it should not be needed.
%\setcounter{table}{1}

%%%%%%%%%%%%%%%%%%%%%%%%%%%%%%%%%%%%%%%%%%%%%%%%%%%%%%%%%%%%%%%%%%%%%%%%%%%%%%%%
%%
%% The following section outlines numerous optional output that
%% can be displayed in the front matter or as running meta-data.
%%
%% If you wish, you may supply running head information, although
%% this information may be modified by the editorial offices.
\shorttitle{Preferential acceleration of heavy ions in magnetic reconnection}
\shortauthors{Jain et al.}
%%
%% You can add a light gray and diagonal water-mark to the first page 
%% with this command:
%% \watermark{text}
%% where "text", e.g. DRAFT, is the text to appear.  If the text is 
%% long you can control the water-mark size with:
%% \setwatermarkfontsize{dimension}
%% where dimension is any recognized LaTeX dimension, e.g. pt, in, etc.
%%
%%%%%%%%%%%%%%%%%%%%%%%%%%%%%%%%%%%%%%%%%%%%%%%%%%%%%%%%%%%%%%%%%%%%%%%%%%%%%%%%

%% This is the end of the preamble.  Indicate the beginning of the
%% manuscript itself with \begin{document}.

\begin{document}

\title{Preferential acceleration of heavy ions in magnetic reconnection: Hybrid-kinetic simulations with electron inertia}

%% LaTeX will automatically break titles if they run longer than
%% one line. However, you may use \\ to force a line break if
%% you desire. In v6.3 you can include a footnote in the title.

%% A significant change from earlier AASTEX versions is in the structure for 
%% calling author and affiliations. The change was necessary to implement 
%% auto-indexing of affiliations which prior was a manual process that could 
%% easily be tedious in large author manuscripts.
%%
%% The \author command is the same as before except it now takes an optional
%% argument which is the 16 digit ORCID. The syntax is:
%% \author[xxxx-xxxx-xxxx-xxxx]{Author Name}
%%
%% This will hyperlink the author name to the author's ORCID page. Note that
%% during compilation, LaTeX will do some limited checking of the format of
%% the ID to make sure it is valid. If the "orcid-ID.png" image file is 
%% present or in the LaTeX pathway, the OrcID icon will appear next to
%% the authors name.
%%
%% Use \affiliation for affiliation information. The old \affil is now aliased
%% to \affiliation. AASTeX v6.3 will automatically index these in the header.
%% When a duplicate is found its index will be the same as its previous entry.
%%
%% Note that \altaffilmark and \altaffiltext have been removed and thus 
%% can not be used to document secondary affiliations. If they are used latex
%% will issue a specific error message and quit. Please use multiple 
%% \affiliation calls for to document more than one affiliation.
%%
%% The new \altaffiliation can be used to indicate some secondary information
%% such as fellowships. This command produces a non-numeric footnote that is
%% set away from the numeric \affiliation footnotes.  NOTE that if an
%% \altaffiliation command is used it must come BEFORE the \affiliation call,
%% right after the \author command, in order to place the footnotes in
%% the proper location.
%%
%% Use \email to set provide email addresses. Each \email will appear on its
%% own line so you can put multiple email address in one \email call. A new
%% \correspondingauthor command is available in V6.3 to identify the
%% corresponding author of the manuscript. It is the author's responsibility
%% to make sure this name is also in the author list.
%%
%% While authors can be grouped inside the same \author and \affiliation
%% commands it is better to have a single author for each. This allows for
%% one to exploit all the new benefits and should make book-keeping easier.
%%
%% If done correctly the peer review system will be able to
%% automatically put the author and affiliation information from the manuscript
%% and save the corresponding author the trouble of entering it by hand.

\correspondingauthor{Neeraj Jain}
\email{neeraj.jain@tu-berlin.de}

\author{Neeraj Jain}
\affiliation{Zentrum f\"ur Astronomie und Astrophysik, Technische Universit\"at Berlin, Hardenbergstr. 36, D-10623, Berlin, Germany}

\author{J\"org B\"uchner}
\affiliation{Zentrum f\"ur Astronomie und Astrophysik, Technische Universit\"at Berlin, Hardenbergstr. 36, D-10623, Berlin, Germany}

\author{Miroslav B\'arta}
\affiliation{Astronomical Institute of the Academy of Sciences of the Czech Republic, Fri\u{c}ova 298, Ond\u{r}ejov, 251 65, Czech Republic}

\author{Radoslav Bu\u{c}\'ik}
\affiliation{Southwest Research Institute, 6220 Culebra Road, San Antonio, TX 78238, USA}

%\collaboration{1}{(AAS Journals Data Scientists collaboration)}

%% \author{...}
%% \affiliation{...}

%% \author{....}
%% \affiliation{...}

%\collaboration{1}{(LaTeX collaboration)}


%\nocollaboration{2}

%% Note that the \and command from previous versions of AASTeX is now
%% depreciated in this version as it is no longer necessary. AASTeX 
%% automatically takes care of all commas and "and"s between authors names.

%% AASTeX 6.3 has the new \collaboration and \nocollaboration commands to
%% provide the collaboration status of a group of authors. These commands 
%% can be used either before or after the list of corresponding authors. The
%% argument for \collaboration is the collaboration identifier. Authors are
%% encouraged to surround collaboration identifiers with ()s. The 
%% \nocollaboration command takes no argument and exists to indicate that
%% the nearby authors are not part of surrounding collaborations.

%% Mark off the abstract in the ``abstract'' environment. 
\begin{abstract}
  Solar energetic particles (SEPs) in the energy range 10s KeV/nucleon - 100s MeV/nucleon originate  from Sun.  Their high flux near Earth may damage the space borne electronics and generate secondary radiations harmful for the life on Earth and thus understanding their energization on Sun is important for space weather prediction. Impulsive (or ${}^{3}$He-rich) SEP events are associated  with the acceleration of charge particles in solar flares by magnetic reconnection and related processes. The preferential acceleration of heavy ions  and the extra-ordinary abundance enhancement of ${}^3$He in the impulsive SEP events are not understood yet. In this paper, we study ion acceleration in magnetic reconnection by two dimensional hybrid-kinetic plasma simulations (kinetic ions and inertial electron fluid). All the ions species are treated self-consistently in our simulations. We find that heavy ions are preferentially accelerated to energies many times larger than their initial thermal energies by a variety of acceleration mechanisms operating in reconnection.  Most efficient acceleration takes place in the flux pileup regions of magnetic reconnection. Heavy ions with sufficiently small values of charge to mass ratio   ($Q/M$) can be accelerated by pickup mechanism in outflow regions even before any magnetic flux is piled up. The energy spectra of heavy ions develop  a shoulder like region, a non-thermal feature, as a result of the acceleration. The spectral index of the power law fit to the shoulder region of the spectra varies approximately as $(Q/M)^{-0.64}$. Abundance enhancement factor, defined as number of particles above a threshold energy normalized to total number of particles, scales as $(Q/M)^{-\alpha}$ where $\alpha$ increases with the energy threshold. We discuss our simulation results in the light of the SEP observations.
\end{abstract}

%% Keywords should appear after the \end{abstract} command. 
%% See the online documentation for the full list of available subject
%% keywords and the rules for their use.
\keywords{magnetic reconnection, heavy ion acceleration}

%% From the front matter, we move on to the body of the paper.
%% Sections are demarcated by \section and \subsection, respectively.
%% Observe the use of the LaTeX \label
%% command after the \subsection to give a symbolic KEY to the
%% subsection for cross-referencing in a \ref command.
%% You can use LaTeX's \ref and \label commands to keep track of
%% cross-references to sections, equations, tables, and figures.
%% That way, if you change the order of any elements, LaTeX will
%% automatically renumber them.
%%
%% We recommend that authors also use the natbib \citep
%% and \citet commands to identify citations.  The citations are
%% tied to the reference list via symbolic KEYs. The KEY corresponds
%% to the KEY in the \bibitem in the reference list below. 

\section{Introduction \label{sec:introduction}}
        %{\bf Notes from ISIS meeting 2022: Look at Phan 2021 (HCS/PSP/reconnection), firehose condition, Bale+22, Arnold+21 (Guide field scaling), Nature 2022 June (Gregory)}

        %{\bf Details of Runs: Run2:: $Ti>Tp$, Run3: Bx=Bz, Run5: with $\sigma_y$, Rnu6: without $\sigma_y$, Run7: long box}

        
Non-thermal acceleration of charged particles is a widespread process in space and astrophysical plasmas ranging from planetary magnetospheres to clusters of galaxies. The accelerated particles, mostly ions, are detected near and on Earth in a wide energy range , $10^4-10^{20}$ eV,  either directly by particle detectors on satellites and high altitude balloons or indirectly by detecting the secondary particles and electromagnetic radiation produced by their interaction with Earth's atmosphere. These particles are generally categorized as solar energetic particles (SEPs), galactic cosmic rays and extra-galactic cosmic rays  based on their source regions in our star the Sun, our galaxy Milky Way  and beyond our galaxy, respectively. Solar energetic particles (SEPs) in typical energy range of 10s KeV/nucleon - 100s MeV/nucleon are of particular interest because their flux near Earth is sufficiently high to have implications for space weather phenomena.

SEP events can be classified into impulsive (short duration $\leq$ 1 day, less intense --- relatively smaller particle fluxes with typical energies $< 10$ MeV/nucleon and numerous --- about 1000 per year) and gradual (long duration $\sim$ several days, orders of magnitude more intense --- large particle fluxes with energies $>$ 10 MeV/nucleon, less frequent $\sim$ 10 per year  ) events \citep{Reames1999,Reames2021_book}. %Events with mixed characteristic of the gradual and impulsive events have also been observed.
Gradual events are attributed to the acceleration of charged particles by Coronal Mass Ejection (CME) driven shocks while impulsive events to the acceleration in solar flares by magnetic reconnection and associated processes \citep{Reames2013,Desai2016_review,Bucik2020}. Impulsive events are electron rich and associated with type-III radio bursts  while gradual events are proton rich and associated with type-II radio bursts \citep{Cane1986,Reames2013}. Impulsive events show abundance enhancements of heavier ions relative to their abundances in the solar corona  at energies well above their average thermal energy ($\sim$ 100 eV) \citep{Reames2021_review}. The enhancement factor, defined as the ratio of the relative (to a reference element, usually Oxygen) abundances of an element X in impulsive SEP events and solar corona, exhibits power law dependence on ion's charge ($Q$) to mass ($M$) ratio as $(Q/M)^{-\alpha}$. Event-to-event variations in the value of $\alpha$ have been found \citep {Reames2014a} with mean values $\alpha=3.26$ (at 0.375 MeV/nucleon) \citep{Mason2004}, $\alpha=3.64 \pm 0.15$ (at 3-10 MeV/nucleon) \citep{Reames2014}, and $\alpha=3.53$ (at 160-226 KeV/nucleon) and $\alpha=3.31$ (at 320-453 KeV/nucleon) \citep{Bucik2021}. 
Enhancement factor for ${}^3$He isotope, however, does not obey the power law and can have very large values (up to $10^4$) compared to those calculated using the power law  \citep{Kocharov1984,Mason2007}. For this reason, impulsive SEP events are also called ${}^3$He-rich events. The enhancement of heavy ions is, however, not correlated with the extra-ordinary enhancement of ${}^3$He \citep{Mason1986,Reames1999}. Abundance enhancement of ${}^3$He is sometimes observed in gradual events as well \citep{Desai2016,Bucik2023}. These are associated either with the acceleration of the remnant material in the interplanetary space from earlier flares by CME-driven shocks or with the simultaneous  activity in the corona.

The preferential acceleration of heavy ions with small value of $Q/M$ from their thermal energies to SEP energies and the extra-ordinary abundance enhancement of ${}^3$He are not understood yet. Since these enhancements are un-correlated \citep{Mason1986,Reames1999}, attempts have been made to explain the preferential acceleration of heavy ions and ${}^3$He by separate mechanisms. 
Models of the preferential acceleration of heavy ions consider  resonant interaction of the ions with the waves in Alfv\'enic plasma turbulence \citep{Eichler2014,Kumar2017,Fu2020,Shi2022,Miller1998}. Turbulent energy decays with wave frequency, i.e., higher frequency waves have lower energy. Heavy ions with lower values of $Q/M$ have lower values of cyclotron frequency ($\propto Q/M$) and thus would resonate with lower frequency but higher power waves in turbulence, favoring preferential acceleration of heavy ions. The mechanisms proposed for the preferential acceleration of ${}^3$He consider absorption of some wave energy by cyclotron resonance of  ${}^3$He with the wave \citep{Fisk1978, Temerin1992, Liu2006}. These waves can be produced by electron beams, e.g., by electrons streaming along open magnetic field lines in solar corona, or via coupling with low-frequency Alfv\'en waves. In fact, efficient acceleration of ${}^3$He by ion-cyclotron resonance has been observed in nuclear fusion devices, which has implications for space plasmas as well \citep{Kazakov2017}.



%Acceleration by turbulent waves, however, could not explain the huge abundance enhancements of ${}^3$He$^{2+}$ due to its high cyclotron frequency.

Magnetic reconnection can also accelerate electrons and ions  in solar flares \citep{Zhou2015,Zhou2016,Barta2011a,Barta2011b}. It has also been considered to explain the preferential acceleration of heavy ions.
In 2.5-D magnetohydrodynamic (MHD) and test particle simulations \citep{Kramolis2022}, heavy ions were found to be preferentially accelerated  by first order Fermi process in cascading plasmoids  generated by spontaneous magnetic reconnection in a meso-scale current sheet. The ion energy spectra and abundance enhancement factors exhibit power-law profiles. The index of the power law for the abundance enhancement factor, however, was not in agreement with the observations. The authors suspected that the disagreement could be due to the limitation of the MHD model which lacks the kinetic physics essential for magnetic reconnection in collisionless plasmas.  
In 2-D fully kinetic simulations of magnetic reconnection, ions entering the reconnection exhaust can behave like pickup particles and gets accelerated if their $Q/M$ is below a threshold value, and thus preferential acceleration of heavy ions. An explanation for the power-law dependence of the enhancements on $Q/M$ was proposed based on the pickup mechanism \citep{Drake2009,Knizhnik2011}. These simulations, carried out for only for one ion species (${}^4$He$^{2+}$), did not verify the power-law dependence. It is, therefore, not clear if acceleration of heavy ions by pickup mechanism can provide the observed scaling of the abundance enhancement with $Q/M$.  

Magnetic reconnection in collisionless plasmas is a multi-scale process which occurs at electron kinetic scales and then couples to ion and even larger macro-scales. An ideal simulation of magnetic reconnection requires kinetic treatment of electrons and ions and covering scales from electron kinetic scales all the way up to large macro-scales well above the ion kinetic scales. Such fully kinetic simulations, e.g., using Particle-in-Cell (PIC) method, of magnetic reconnection in electron-proton plasma are computationally very demanding. Self-consistent inclusion of heavier ion species make the simulations even more demanding because now one has to use larger simulation box to accommodate the larger gyro-radii of the heavier species while at the same time resolve the electron scales. Computationally less demanding hybrid-kinetic simulations, which treats ions kinetically and electron as an inertial fluid, can be employed at the cost of electron kinetic physics which is acceptable as far as the study of the ion acceleration is the primary objective. Treatment of electrons as an inertial fluid allows to include the physics at electron inertial scale but relaxes the numerical requirement of resolving the Debye length in PIC method and therefore allowing larger simulation domains for the same number of grid points. 


In this paper, we employ a hybrid-kinetic plasma model (kinetic ions and inertial electron fluid) to simulate magnetic reconnection in a 2-D plane with the objective of studying ion acceleration in reconnection. All the ion species are treated self-consistently in our simulations. We use the term ``heavy ion'' to mean any ion species whose mass $M$ is larger than the proton's mass. The paper is organized as follows. Section \ref{sec:simulation_setup} presents the simulation setup.  Simulation results are presented in section \ref{sec:results} and discussed and concluded in section \ref{sec:conclusion}.
%------------------------------------------------
%% \section{Hybrid-kinetic plasma model with electron inertia}
%% Hybrid-kinetic plasma model treats ions as kinetic species 
%%            \begin{align*}
%%                  & \text{\alert{Electron fluid+Maxwell's equations:}}\\
%%   %\end{align*}
%%   %\begin{align*} \label{eq:e_ohm}
%%      & \vec{E}  = -\vec{u}_e\times \vec{B} -\frac{\nabla p_e}{en_e} - \frac{m_e}{e}\left(\frac{\partial \vec{u}_e}{\partial t}+(\vec{u}_e\cdot\vec{\nabla})\vec{u}_e\right)  + \eta \vec{\jmath}\\
%%   %\end{align*}
%%   %\begin{align*} \label{eq:curl_emom}
%%    &  \frac{\partial \overrightarrow{W}}{\partial t} = \vec{\nabla}\times\left [\vec{u}_e\times \overrightarrow{W}\right]-\vec{\nabla}\times\left(\frac{\vec{\nabla} p_e}{m_en}\right)- \vec{\nabla}\times\left(\frac{\nu\vec{\jmath}}{en}\right)\\
%%   %\end{align*}  
%%   %\begin{align*}\label{eq:elliptic_b}
%%    &  \frac{1}{\mu_0e}\vec{\nabla}\times\left(\frac{\vec{\nabla}\times\vec{B}}{n}\right)+\frac{e\vec{B}}{m_e}  = \vec{\nabla}\times\vec{u}_i-\overrightarrow{W}\\
%%   %\end{align*}
%%   %\begin{align*} \label{eq:eos}
%%     &  p_e = n_ek_BT_e\\
%%     &\\
%%     &  \text{\alert{Ion macro-particle equations:}}\\
%%         &  \frac{d\vec{x}_i^{\;p}}{dt}     =\vec{v}_i^{\;p}                \\
%%     %\,\,\,\,\,\,\,\,\,\,\,  
%%     &  m_i\frac{d\vec{v}_i^{\;p}}{dt}  =e(\vec{E}^{\;p} + \vec{v}_i^{\;p}\times \vec{B}^{\;p})
%%   \end{align*}

        
\section{Simulation setup\label{sec:simulation_setup}}
We carry out 2-D hybrid-kinetic simulations of magnetic reconnection with electron inertia using the hybrid-PIC code CHIEF (Code Hybrid with Inertial Electron Fluid) which is a 3-D code parallelized based on Message Passing Interface (MPI) for high performance computing \citep{Munoz2018,Jain2022,Munoz2023}. In the hybrid code CHIEF, electrons are treated as an inertial and isothermal fluid whose equations are coupled to Maxwell's equation to obtain the electric and magnetic fields. All the ions (protons and heavier ions) , on the other hand, are treated self-consistently as kinetic species  and their equations of motion in electric and magnetic fields are solved using the PIC method. The code CHIEF treats the inertial effects of the electron fluid without any of the approximations used by other electron-inertial hybrid-kinetic code \citep{Jain2022}.  The details of the hybrid-kinetic model used in CHIEF, its numerical implementation and parallelization are discussed in our other publications \citep{Munoz2018,Jain2022,Munoz2023}.


We initialize the 2-D simulations with two Harris current sheets, $\mathbf{B}=\hat{z}B_{z0}[\tanh\{(y+L_y/4)/L\}-\tanh\{(y-L_y/4)/L\}-1]+\hat{x}B_{x0}$, in a y-z plane and a guide magnetic field $B_{x0}=0.2\,B_{z0}$ perpendicular to the plane. The half-thickness $L=d_p$ of the current sheets is taken to be equal to a proton inertial length ($d_p$) and $L_y$ is the length of the simulation domain along y-direction. A small initial perturbation is added to the Harris equilibrium to form an X-point and an O-point in the current sheets centered at $y=-L_y/4$ and $y=L_y/4$, respectively. The plasma of Harris current sheets consists of quasi-neutral populations of proton particles and electron fluid with Harris equilibrium density  profile $n_e=n_p=n_0[\text{sech}^2\{(y+L_y/4)/L\}+\text{sech}^2\{(y-L_y/4)/L\}]$. Harris sheets are embedded in a uniform background plasma of density $0.2\,n_0$ and therefore peak density at their centers is $1.2\,n_0$. The background plasma consists of electron fluid and particle populations of  heavy ions (only one species) and protons with densities $n_{be}=0.2\,n_0$, $n_{bi}=0.01\,n_0$ (5\% of the background plasma density) and $n_{bp}=n_{be}-Z\,n_{bi}$, respectively,  where $Z$ is the charge state of heavy ions with charge of heavy ion species given by $Q=Z\,e$.
%We take electron and proton temperatures both in the Harris sheet and the background as $T_e=T_p=0.25\, m_p V_{Ap}^2$ where $m_p$ is proton mass  and $V_{Ap}=B_{z0}/\sqrt{\mu_0n_0m_p}$ is the proton Alfv\'en velocity based on $B_{z0}$ and $n_0$.
The initial velocity distribution of the protons and heavy ions is Maxwellian. The initial  temperatures of electrons, protons and heavy ions are the same, viz., $T_e=T_p=T_i=0.25\, m_p V_{Ap}^2$ where $m_p$ is proton mass  and $V_{Ap}=B_{z0}/\sqrt{\mu_0n_0m_p}$ is the proton Alfv\'en velocity based on $B_{z0}$ and $n_0$.

Each of our simulations consists of only one species of heavy ion. We consider the following four different species of heavy ions to be included (one at a time) in our simulations : ${}^4$He$^{2+}$, ${}^3$He$^{2+}$, ${}^{16}$O$^{7+}$ and ${}^{56}$Fe$^{14+}$.
%We carry out simulations corresponding to each of the following four different species of heavy ions: ${}^4$He$^{2+}$, ${}^3$He$^{2+}$, ${}^{16}$O$^{7+}$ and ${}^{56}$Fe$^{14+}$.
We take proton to electron mass ratio as $m_p/m_e=25$. The simulation domain $L_y\times L_z$ is $51.2\,d_p \times 102.4\,d_p$ resolved by a grid spacing of $0.1\, d_p$ in each direction. The time step is $\Delta t=0.0025\,\omega_{cp}^{-1}$, where $\omega_{cp}=eB_{z0}/m_p$ is the proton cyclotron frequency. We take 500 and 200 particles per cell for protons and heavy ions, respectively. Plasma resistivity is taken to be zero and electron inertia allows the magnetic reconnection. Boundary conditions are periodic. 



%------------------------------------------------

\section{Simulation results \label{sec:results}}
Fig. \ref{fig:jx_evolution} shows evolution of out-of-plane current density $J_x$ from initial to late phase. In the initial phase ($\omega_{cp}t=19.6$), an X-point forms in the lower current sheet ($y<0$) while an O-point forms in the upper current sheet ($y>0$), as per the  initialized perturbation. By $\omega_{cp}t=39.19$, the lower (also the upper) current sheet spontaneously develop magnetic islands as a result of magnetic reconnection at multiple sites, even though the initialized perturbation was chosen to initiate reconnection at a single site in the simulation domain. The magnetic islands on each of the current sheet grow in size with time by merging and/or pushing among themselves and simultaneously develop turbulence  inside them.
%The lower current sheet has a single magnetic island in the central region ($-20 < z/d_p < 20$) at $\omega_{cp}t=48.99$.
At $\omega_{cp}t=58.79$, the magnetic islands of the lower and upper current sheet grow to big enough size that the particles accelerated in the upper (lower) current sheet may cross over to the lower (upper) current sheet. 
%they interact and make the whole simulation domain turbulent with hardly any resemblance to magnetic reconnection in the initial two current sheets.

Our objective here is to study the ion acceleration without the influence of the periodic boundaries. Since the upper current sheet is affected by the periodic boundary conditions along z-direction from the beginning (the initial  X-points form at the z-boundaries), we focus on the lower current sheet for our studies. The lower current sheet is also likely to be affected by periodic boundaries along z-direction after a time $\sim 50 \omega_{cp}^{-1}$, Alfv\'en waves take to cross half the simulation domain.
In order to avoid particles crossing from the upper half to the lower half of the simulation domain and the influence of periodic boundaries,  we limit our analysis of results only up to the time $\omega_{cp}t=48.99$ and in the spatial region $-23 \leq y/d_p \leq -3$.


%We study the physics of ion acceleration only in the lower half  ($y<0$) of the simulation domain ignoring the upper half. This is reasonable as the upper half of the simulation domain is not expected to provide any information different from the lower half but rather would increase the demand for the computational resources. Note that periodic boundary conditions along z-direction would influence the results, specially the acceleration of particles near the z-boundaries. We, however, consider this influence to be part of the physics of interest as periodic boundaries mimic the identical reconnection regions beyond the boundaries at which plasma flows resulting from reconnection interacts. This does not necessarily represent a physical situation but is of interest from the point of view of studying the physics of ion acceleration.

%Also note that we consider the acceleration of ions in the late phase of simulations, at $\omega_{cp}t=78.39$ and at later times, due to the turbulence which engulfs the whole simulation domain. The acceleration in turbulence could be due to the resonance mechanisms and/or by magnetic reconnection in many small turbulent current sheets. 


% Figure environment removed


Figure \ref{fig:av_ke_evolution}a shows the evolution of fractional change in average kinetic energy from its initial value
%and maximum kinetic energy normalized to initial maximum energy
for different ion species. All the heavy ions gain energy first at slow rates  up to $\omega_{cp}t\approx 30$ after which they gain energy at much faster rates.    Note that magnetic islands have significantly developed by $\omega_{cp}t=30$. This can be seen in Fig. \ref{fig:av_ke_evolution}b which shows that $\langle B_y^2 (y=-12.5\,d_p, z)\rangle_z/B_{z0}^2$ (average of the magnetic energy in the normal component of the magnetic field along the central line of the lower current sheet --- a proxy for the magnetic island development) begins to grow at $\omega_{cp}t=20$ and, by $\omega_{cp}t=30$, has grown to $\sim$ 5\% of the asymptotic magnetic energy in the anti-parallel magnetic field. The development of magnetic islands, therefore, seem to be linked with the efficient energization of heavy ions. Proton energization, on the other hand, does not seem to be affected significantly by the formation of magnetic islands as it does not enhance significantly after $\omega_{cp}t\approx 30$.

% Figure environment removed

Heavier ions (${}^{16}$O$^{7+}$ and ${}^{56}$O$^{14+}$ in Fig. \ref{fig:av_ke_evolution}) get energized significantly in comparison to the lighter Helium ions even during early phase of acceleration ($0 < \omega_{cp}t < 30$). This suggests that acceleration mechanisms, different from the mechanisms after $\omega_{cp}t=30$, operate in the early phase only on the heavier ions, implying a threshold for the energization based on the heaviness of the ions.  In order to understand the threshold behavior and the acceleration mechanisms before and after $\omega_{cp}t=30$ (the time by which $\sim$5\% of the magnetic energy of the asymptotic anti-parallel magnetic field has contributed to the development of magnetic islands), we show  the locations of the energized particles in the reconnection region at $\omega_{cp}t$=19.6 and 48.99 in two simulations --- one with  ${}^4$He$^{2+}$ (Fig. \ref{fig:acceleration_sites_4He2}) and the other with ${}^{16}$O$^{7+}$ (Fig. \ref{fig:acceleration_sites_16O7}). At $\omega_{cp}t=19.6$, ${}^{16}$O$^{7+}$ ions are more energized than ${}^4$He$^{2+}$ ions and the locations of the energized ${}^{16}$O$^{7+}$ ions are in the outflow regions. The energization is, however, not uniformly distributed in the outflow regions: the energized particles are located near the upper (lower) separatrix in the left (right) outflow region. This observations combined with the threshold behavior of energization suggests the pick-up mechanism of the energization in which non-adiabatic ions are energized as they cross the reconnection separatrices \citep{Drake2009}. A threshold condition,   $m_i/Z_im_p > 5 \sqrt{2\beta_{p}}/\pi$, for non-adibaticity of ions crossing the sepratrix in guide field reconnection was obtained by Drake et al. \citep{Drake2009}, where $m_i$ and $Z_i=Q_i/e$ are the mass and charge state of ions and $\beta_p$ is the proton plasma beta based on asymptotic value of the anti-parallel magnetic field. Although this condition, which becomes $m_i/Z_im_p > 1.12$ for our simulation parameters, is satisfied by all the species of heavy ions we have considered, the  ${}^4$He$^{2+}$ and ${}^3$He$^{2+}$ ions with $m_i/Z_im_p$ = 2 and 1.5 were not energized during $\omega_{cp}t<30$.

Note that the threshold condition was obtained assuming that the ions cross the separatrix with a velocity equal to $0.1 v_{Ap}$ which is almost the upper bound on the inflow velocity in a fully developed steady state magnetic reconnection \citep{Liu2017}.  In our simulations, magnetic reconnection at $\omega_{cp}t=19.6$ is still developing and the ion inflow velocity has not yet reached its maximum value. Using $|u_{iy}| \approx 0.05\,v_{Ap}$, the value of inflow velocity at $\omega_{cp}t=19.6$ in our simulations, the threshold condition becomes $m_i/Z_im_p > 2.24$ which allows non-adiabatic behavior for ${}^{56}$Fe$^{14+}$ ($m_i/Z_im_p =4$) and only marginally for ${}^{16}$O$^{7+}$ ($m_i/Z_im_p = 2.28$) but not for ${}^4$He$^{2+}$ ($m_i/Z_im_p = 2$) and ${}^3$He$^{2+}$ ($m_i/Z_im_p = 1.5$). 
%The inflow velocity near the separatrices away from the X-point is usually  smaller than $0.1v_{Ap}$. 

The localization of energized ${}^{16}$O$^{7+}$ ions near the upper (lower) separatrix in the left (right) outflow region at $\omega_{cp}t=19.6$ is due to the asymmetries in the lower and upper separatrices in guide field magnetic reconnection \citep{Li2020,Pritchett2004}. The thickness of the upper (lower) separatrix in the left (right) outflow region is smaller than that of the lower (upper) separatrix. This limits the non-adiabaticity and thus energization of ions entering outflow regions from the thicker separatrix. 


The increased rate of energization for all the heavy ion species after $\omega_{cp}t=30$ is somehow linked to the growth of the normal component $B_y$ of the magnetic field in the current sheet (Fig. \ref{fig:av_ke_evolution}). The normal component $B_y$ in current sheet can grow large by non-steady reconnection with increasing rate and/or the compression along the current sheet of the reconnected magnetic field lines. The compression can occur  due to the pile up of the magnetic flux reconnected at two neighboring sites on the magnetic island between the two sites, contraction of magnetic islands and/or mutual pushing among magnetic islands. Several acceleration mechanisms may be associated with these scenarios: direct acceleration by inductive reconnection electric field  in the X-point regions, acceleration by motional electric field induced by Alfv\'enic outflow in the outflow regions,  Fermi-like acceleration in contracting magnetic islands, magnetic curvature and gradient drifts aligned with inductive/motional electric field in the flux pile-up regions and betatron acceleration by time dependent magnetic field in the flux compression regions.

Figures \ref{fig:acceleration_sites_4He2} and \ref{fig:acceleration_sites_16O7} show that, at $\omega_{cp}t=48.99$, energized ions are located in the X-point regions, inside magnetic islands and outflow exhaust regions. Most energized particles (black dots in Figures \ref{fig:acceleration_sites_4He2} and \ref{fig:acceleration_sites_16O7}) are mostly concentrated near the opening of the exhaust regions, where the reconnected magnetic field lines usually pile up.  
In the case of simulations with ${}^{16}$O$^{7+}$,
%a spatial gap in the locations of the most energized particles can be seen in the right outflow region, and
merging of two magnetic islands and presence of energized ions at $z\approx 0$ can also be seen  (Fig. \ref{fig:acceleration_sites_16O7}). Although the locations of the energetic particles at a given time are not necessarily in the neighborhood  of their acceleration sites, it seems from Figs. \ref{fig:acceleration_sites_4He2} and \ref{fig:acceleration_sites_16O7}  that a number of acceleration mechanisms out of those discussed above are operating simultaneously in different regions of reconnection.  A given particle might experience acceleration due to different mechanisms  it encounters on its trajectory in different reconnection regions. Disentangling these mechanisms require a fully history of particles' motion and therefore tracking of their trajectories and would be the subject of our future studies.


% Figure environment removed


% Figure environment removed

%% Distributions of ions' velocity components change with time as a result of the energization. The evolution of the distribution of the velocity components  is shown in Figure \ref{fig:v_spectra_evolution} for protons, ${}^{4}$He$^{2+}$ and ${}^{16}$O$^{7+}$. Note that  these distributions are not local velocity distributions as they are obtained by taking into account all the particles in the region of analysis ($-23 < y/d_p < -3 $ for all $z$). Proton's distribution in $v_{ix}$ and $v_{iy}$ barely change  by $\omega_{cp}t=19.6$ and only slightly change later. Proton's distribution in $v_{iz}$, on the other hand, changes significantly  by $\omega_{cp}t=19.6$ due to the development of the Alfv\'enic outflows as a result of magnetic reconnection and does not change much afterwards. 

%% The rate of the energization of the heavy ions is species-dependent, typically larger for the species with smaller $Q/M$.

Fig. \ref{fig:spectra_evolution} shows the evolution of the energy spectra for different ions species. Energy spectra for all the ion species broaden with time.  Energy spectra for protons did not develop any noticeable non-thermal feature up to the final simulation time $\omega_{cp}t=48.99$.  Thus energy transferred to protons only heat them. Non-thermal feature, a shoulder in the spectra in the intermediate energy range after which spectra falls off rapidly, develops in the spectra of ${}^{4}$He$^{2+}$ and ${}^{16}$O$^{7+}$ at $\omega_{cp}t=39.19$ and $\omega_{cp}t=19.6$, respectively. The early development of the spectral shoulder for heavier ions, ${}^{16}$O$^{7+}$ and ${}^{56}$Fe$^{14+}$ (Figure not shown), in comparison to the lighter ions, ${}^{4}$He$^{2+}$ and ${}^{4}$He$^{3+}$ (Figure not shown), is consistent with the early  energization of heavier ions by pickup mechanism. At $\omega_{cp}t=19.6$, the energy of the accelerated  ${}^{16}$O$^{7+}$  (see Fig. \ref{fig:acceleration_sites_16O7}) is in the range 1-10 $m_pv_{Ap}^2$ in which ${}^{16}$O$^{7+}$ spectra develop a shoulder. The spectral shoulder rises with time as well as extends to higher energies  as more and more low energy heavy ions are accelerated to higher energies.
%The energy transferred to heavy ions, on the other hand,  is partitioned between thermal and non-thermal energy. 

% Figure environment removed


% Figure environment removed

Figure \ref{fig:spectra_ions}a compares the energy-per-nucleon spectra of different ion species at $\omega_{cp}t=48.99$ --- the last time up to which our simulations are valid.
Note that, for the purpose of comparing the spectra of different ion species independent of their total mass, the spectra in Fig. \ref{fig:spectra_ions}a is shown as a function of energy-per-nucleon unlike the spectra in Fig.\ref{fig:spectra_evolution} which is shown as a function of energy. 
The injection energy $\sim 0.3 m_pv_{Ap}^2$, around which the spectral shoulder begins, is similar  for heavy ion species. The cutoff energy, up to which the shoulder extends, and the maximum gained energy per nucleon are larger for the lighter ion species. In Fig. \ref{fig:spectra_ions}b, we plot the same spectra (excluding proton's spectra) as in Fig. \ref{fig:spectra_ions}a but artificially shifted on y-axis for better visibility of the spectral shoulder and power law fits, $E^{\gamma}$, of the shoulder region of the spectra, where $E$ is the energy-per-nucleon. The spectral indices of the fits, $\gamma$=-1.59, -1.53 and -1.68 respectively for   ${}^{4}$He$^{2+}$,  ${}^{3}$He$^{2+}$ and ${}^{16}$O$^{7+}$ are quite similar. The spectra for ${}^{56}$Fe$^{14+}$, on the other hand, is somewhat softer with $\gamma=-3.0$. The dependence of $\gamma$ on $Q/M$, shown in Fig. \ref{fig:qDm_dependence}a, can be approximately fitted as $\gamma \propto (Q/M)^{-0.64}$. 

%The maximum energy per nucleon (normalized to proton maximum energy per nucleon) increases with $Q/M$, approximately as $(Q/M)^{1.4}$, as shown in Fig. \ref{fig:spectra_ions}.

% Figure environment removed

In ${}^{3}$He-rich SEP events, ion abundances are enhanced relative to their solar abundances at energies well above their thermal energy. We consider a proxy for abundance enhancement factor $F$ as number of particles above a threshold energy normalized to the total number of particles. Fig. \ref{fig:qDm_dependence}b shows this proxy of the abundance enhancement factor as a function of $Q/M$ at $\omega_{cp}t=48.99$ for three energy thresholds $E_t=10 \, \mathit{KE}_{th}, 25 \, \mathit{KE}_{th}$ and $50\, \mathit{KE}_{th}$ well above the initial thermal energy $\mathit{KE}_{th}=0.25 m_pv_{Ap}^2$ which is the same for all ion species in our simulations. The value of $E_t$ is so chosen that it is  in the tail region of the initial energy spectra which is the same for all the ion species and at the same time is not outside the shoulder regions of the energy spectra of the heavy ions at $\omega_{cp}t=48.99$. Although the value $E_t=50\, \mathit{KE}_{th}=12.5 \,m_pv_{Ap}^2$ is close to the cutoff energy of the shoulder region in the energy spectra of ${}^4$He$^{2+}$ at $\omega_{cp}t=48.99$ (see Fig. \ref{fig:spectra_evolution}b), the abundances will be dominated by the abundances in the shoulder region as the energy spectra falls off very rapidly beyond the shoulder region. Note that the threshold $E_t=12.5 \,m_pv_{Ap}^2$ is in the rapidly falling region of the proton's energy spectra at $\omega_{cp}t=48.99$. It is, however, acceptable as proton's energy spectra does not develop a shoulder.             

Fig. \ref{fig:qDm_dependence}b also shows the power law fit $F \propto (Q/M)^{-\alpha}$. The enhancement factor drops with increasing $Q/M$, i.e., heavier ions are preferentially accelerated. The drop of the enhancement factor becomes steeper (the value of $\alpha$ increases) with the increasing threshold energy. %The larger the threshold energy, more preferentially heavy ions are accelerated.
As the value of $E_t$ increases, the contribution of the shoulder region relative to the contribution of the rapidly falling region of the energy spectra towards the abundances decreases. This decrease in the contribution  is more prominent for larger values of $Q/M$ as the shoulder in their energy spectra extends to smaller energies. This results in the increase in the value of $\alpha$ with $E_t$.    

%\section{Theoretical estimates \label{sec:theory}}
\section{Discussion and conclusion  \label{sec:conclusion}}
We carried out hybrid-kinetic simulations (with electron inertia) to study the acceleration of heavy ions ($Q/M < 1$) by magnetic reconnection. We find that heavy ions can be accelerated to high energies many times larger than their initial energies by a variety of acceleration mechanisms. Heavier ions are preferentially accelerated in the sense that energy gain averaged over particles increases with decreasing $Q/M$.
They are primarily accelerated in magnetic islands and flux pile up regions near the opening of the outflow exhausts. Most efficient acceleration takes place in the flux pileup regions. Heavy ions, depending upon the smallness of $Q/M$ which allows them to be non-adiabatic while crossing from inflow to outflow regions, can also be accelerated by pickup mechanism in outflow regions even before any magnetic flux is piled up. As a result of acceleration, heavy ions develop a shoulder, a non-thermal feature, in their energy spectra. The spectral index obtained from the power law fit in the shoulder region of the spectra varies approximately as $(Q/M)^{-0.64}$. Abundance enhancement factor, defined as number of particles above a threshold energy normalized to total number of particles, scales as $(Q/M)^{-\alpha}$ where $\alpha$ increases with the energy threshold. 

%The maximum energy per nucleon, a measure of velocity, is larger for the lighter species at the final simulation time. %In this sense, heavy ions are not preferentially accelerated.

Energy spectra with a shoulder or in other words double power law with a break in the spectra  at  energy $\sim$ 1 MeV/nucleon has been in-situ observed  in space \citep{Mason2002,Bucik2018}. Our simulations show the break in the energy spectra in the energy range 2-6 $m_pv_{Ap}^2$ per nucleon depending upon the ion species.  The value of $v_{Ap}$ in the active regions of solar corona, where acceleration takes place, has been estimated to lie in the range 2000-9000 km/s \citep{Brooks2021}. For a typical value $v_{Ap}$=5000 km/s, $m_pv_{Ap}^2\approx $0.25 MeV and therefore break in the simulations occurs in the range 0.5-1.5 MeV/nucleon, consistent with the observations. Note that the energy range of the break depends on the value of $v_{Ap}$, estimates of which in solar corona vary significantly \citep{Brooks2021}. In the observations, the spectral index $\gamma$ of the power law before the break is in the range 1-3 for the ions of Helium-4, Helium-3, Oxygen and Iron. The values of $\gamma$ before the break is in the same range in our simulations as well. The values of $\gamma$ after the break are, however, much larger in our simulation in comparison to the observations.  

We defined  a proxy $F$ for the abundance enhancement factor to study its variation with $Q/M$. Although this proxy does not exactly correspond to the abundance enhancement factor used in observations, we point out some similarity and differences in the behavior of the two. In our simulations, the power law index $\alpha$ of the fit $F\propto (Q/M)^{-\alpha}$  increases with the threshold energy. Observation also show event to event variation in the value of $\alpha$ in the range $-5 < \alpha < 12$ with a mean value of $\alpha=$3.26 at 385 keV/nucleon \citep{Mason2004} and $2< \alpha < 8$ with a mean value of $\alpha=3.64$ at 3-5 MeV/nucleon \citep{Reames2014a,Reames2014}. The mean values in the observations are similar despite their energies being an order of magnitude apart. The mean of the three values of $\alpha$ obtained from power law fits in Fig. \ref{fig:qDm_dependence}b is approximately 3.66. In observations, steeper energy spectra tend to have steeper fall of the abundance enhancement with $Q/M$, i.e., large value of $\alpha$ \citep{Reames2014a}. This effect is similar to the increase in the value of $\alpha$ with $E_t$ in our simulations. Larger values of $E_t$ increases the contribution of the steeper part of the energy spectra in the calculations of the abundances. 




%The variations in the observational value of $\alpha$ are, however, not necessarily associated with the variation of energy at which the abundance enhancement factor are calculated. In fact variations in $\alpha$ can be seen  even at the similar energies.
%On the other hand, similar observational values of $\alpha$ are obtained at widely different energies: $\alpha=3.26$ at 385 keV/nucleon in ${}^3$He rich impulsive events  \citep{Mason2004} and $\alpha=3.64$ at 3-5 MeV/nucleon in Fe-rich impulsive events associated with coronal mass ejections \citep{Reames2014}. In our simulations, similar value of $\alpha=3.34$ is obtained for energies $> 25 KE_{th}\approx $ 1.5 MeV/nucleon (for $v_{Ap}$=5000 km/s), consistent with the latter observation.  

Note that the SEP events are detected in space far away from their acceleration sites on Sun. The observed scaling of abundance enhancement with $Q/M$ and other features of these events may, therefore, not necessarily be the effect of only acceleration but also of the transport of particles from the acceleration site to the detection site. In fact, 3-D test particle modeling of the inter-planetary transport of relativistic protons from the source region on Sun shows that the spectra of the particles is highly observer dependent and do not necessarily reflect the source spectra \citep{Dalla2020}. Such transport effects are possible for the spectra of heavy ions as well. Nevertheless, magnetic reconnection is a potential candidate for the preferential acceleration of heavy ions which may provide  a power law dependence of the abundance enhancement on $Q/M$, as suggested by the results presented here. More detailed studies on the relative roles of the different acceleration mechanisms operating in magnetic reconnection in the abundance enhancement will be presented in a future publication. Note that we did not find any extra-ordinary abundance enhancement of ${}^3$He$^{2+}$ in our simulations. This, however, does not rule out the role of magnetic reconnection in the abundance enhancement of ${}^3$He$^{2+}$ as our simulations are carried out only for limited range of parameters. 

%NASA's Parker Solar Probe (PSP) spacecraft recently made in situ measurements at a distance $<$ 1 AU from the Sun of 10-100 keV/nucleon supra-thermal protons and heavy ions  during crossings of reconnection exhaust connected to the reconnection X-line located $\sim 43000$ km sunward of the PSP \citep{Desai2022}. Thus transport effect in these observations of ions would be minimal as the source acceleration sites are local.   
%% \appendix

\begin{acknowledgments}
The authors thank Patricio Mu\~noz for fruitful discussion and his help. We gratefully acknowledge  the financial support by the German Science Foundation (DFG), projects JA 2680-2-1 and BU 777-17-1 as well as the Czech Republic  GA\^CR project  20-09922J M, personally Markus Rampp and Meisam Tabriz  of the Max Planck Computing and Data Facility (MPCDF) for their support of  the development  of the hybrid-kinetic code CHIEF used for the simulation studies  presented in this paper. The simulations were carried out on the MPS supercomputers at the MPCDF, Garching,  Germany.





%  NJ thanks Patricio Mu\~noz for fruitful discussion and his help.  We gratefully acknowledge  the financial support by the German Science Foundation (DFG), project JA 2680-2-1 in developing the hybrid-kinetic code CHIEF which was used for the simulation studies presented in this paper. Simulations were carried out on the supercomputers of the Max Planck Computing and Data Facility, Garching.
%of the Institute for  Mathematics at the TU Berlin


\end{acknowledgments}








%% For this sample we use BibTeX plus aasjournals.bst to generate the
%% the bibliography. The sample63.bib file was populated from ADS. To
%% get the citations to show in the compiled file do the following:
%%
%% pdflatex sample63.tex
%% bibtext sample63
%% pdflatex sample63.tex
%% pdflatex sample63.tex

\bibliography{references_HIA}{}
\bibliographystyle{aasjournal}

%% This command is needed to show the entire author+affiliation list when
%% the collaboration and author truncation commands are used.  It has to
%% go at the end of the manuscript.
%\allauthors

%% Include this line if you are using the \added, \replaced, \deleted
%% commands to see a summary list of all changes at the end of the article.
%\listofchanges

\end{document}

% End of file `sample63.tex'.
