
\documentclass[10pt,journal,compsoc]{IEEEtran}
%\documentclass[journal,10pt]{IEEEtran}

\usepackage{setspace} 
\usepackage{latexsym, graphicx, textcomp}
\usepackage{epsfig,upgreek}
\usepackage{amsfonts, amssymb, amsthm,makecell,enumerate}
\usepackage{float,color,dblfloatfix}
\usepackage{fancyhdr}
\usepackage{amsmath}
\usepackage{amsfonts}
\usepackage{mathrsfs}
\usepackage{pifont}
\usepackage{amssymb}
\usepackage{graphicx}
\usepackage{multicol}
\usepackage{cases}
\usepackage{epstopdf}
\usepackage{ltxtable, filecontents}
\usepackage{latexsym}
\usepackage{booktabs}
\usepackage{multirow}
\usepackage{booktabs}
\usepackage{threeparttable}
\usepackage{supertabular}
\usepackage[figuresright]{rotating}
\usepackage[tight,small]{subfigure}
\usepackage{tabularx}
\usepackage{cite}
\usepackage{dcolumn}
\usepackage{algorithmicx}
\usepackage[lined,boxed]{algorithm2e}
%\usepackage[linesnumbered,ruled,vlined]{algorithm2e}
\usepackage{algpseudocode}
 \usepackage{ragged2e}
 \usepackage{cuted}
\usepackage{graphicx} 
\usepackage{diagbox}
\usepackage{array}
\usepackage{amsfonts}
\usepackage{bbm}
\usepackage{enumitem}
\usepackage{hyperref}
\usepackage{bm}
%\usepackage{lettrine}
\usepackage{hyperref}
%\usepackage[colorlinks=false,linkcolor=black,anchorcolor=black,citecolor=black,CJKbookmarks=false]{hyperref} 

%\pdfstringdefDisableCommands{\let\bm=\relax}
\graphicspath{{figures/}}
%\hypersetup{CJKbookmarks=true}
\newtheorem{myDef}{Definition}
\newtheorem{myLam}{Lemma}
\newtheorem{myPro}{Proof}
\newtheorem{Defn}{Definition}
\newtheorem{lem}{Lemma}
\newtheorem{col}{Corollary}
\newtheorem{Prop}{Proposition}
\newtheorem{thm}{Theorem}
\newtheorem{rek}{Remark}

\linespread{0.975}
\begin{document}

%\title{DISCO: Double-Plan-Promoted Isomorphic Subgraph Search and Optimization for Graph Task Scheduling over Vehicular Clouds}
%\title{Achieving Low Latency and High Reliability in Scheduling of Graph-Structured Tasks over Vehicular Clouds}
%\title{DISCO: Achieving Low Latency and High Reliability in Scheduling of Graph-Structured Tasks over Mobile Vehicular Cloud}
%
%\title{DISCO: Accelerating Reliable Graph Task Scheduling over Mobile Vehicular Clouds}

%\title{Accelerating Graph Task Scheduling over Mobile Vehicular Clouds: a Multi-Stage Approach}
%\title{Towards Stage-Wise Decision-Making for Accelerating the Scheduling of Graph-Structured Tasks over Mobile Vehicular Clouds}
\title{\huge Unleashing the Potential of Stage-Wise Decision-Making in Scheduling of Graph-Structured Tasks \\ over Mobile Vehicular Clouds}


\author{Minghui Liwang, \IEEEmembership{Member}, \IEEEmembership{IEEE}, Bingshuo Guo, Zhanxi Ma, Yuhan Su, Jian Jin, 
\\Seyyedali Hosseinalipour,~\IEEEmembership{Member}, \IEEEmembership{IEEE}, Xianbin Wang,~\IEEEmembership{Fellow}, \IEEEmembership{IEEE}, 
Huaiyu Dai,~\IEEEmembership{Fellow}, \IEEEmembership{IEEE}

%\thanks{Minghui Liwang (minghuilw@xmu.edu.cn), Bingshuo Guo (guobingshuo@stu.xmu.edu.cn) are with the School of Informatics, Xiamen University, China. Zhanxi Ma is with the School of Electronic and Engineering, 
%Nanjing University, China. 
%Yuhan Su (ysu@xmu.edu.cn) is with School of Electronic Science and Engineering, Xiamen University, China. Jian Jin (jin.jian@caict.ac.cn) is with the Research Institute of Industrial Internet of Things, China Academy of Information and Communications Technology, Beijing 100095, China. 
%Seyyedali Hosseinalipour (alipour@buffalo.edu) is with Department of Electrical Engineering, University at Buffalo-SUNY, USA. Xianbin Wang (xianbin.wang@uwo.ca) is with the Department of Electrical and Computer Engineering, Western University, Canada. Huaiyu Dai (hdai@ncsu.edu) is with Department of Electrical and Computer Engineering, North Carolina State University, USA.

\thanks{Minghui Liwang (minghuilw@xmu.edu.cn), Bingshuo Guo (guobingshuo@stu.xmu.edu.cn) and Yuhan Su (ysu@xmu.edu.cn) are with Xiamen University, China. Zhanxi Ma is with Nanjing University, China. Jian Jin (jin.jian@caict.ac.cn) is with the Research Institute of Industrial Internet of Things, China Academy of Information and Communications Technology, China. 
Seyyedali Hosseinalipour (alipour@buffalo.edu) is with University at Buffalo-SUNY, USA. Xianbin Wang (xianbin.wang@uwo.ca) is with Western University, Canada. Huaiyu Dai (hdai@ncsu.edu) is with North Carolina State University, USA.

%Corresponding author: Yuhan Su, Jian Jin


}}

\IEEEtitleabstractindextext{
\begin{abstract}
\justifying
% \noindent
%Data processing over dynamic and distributed vehicles calls for innovative resource provisioning techniques to support real-time, cost-effective, and reliable computing services. This article investigates how computation-intensive tasks modeled by undirected weighted graphs can be efficiently scheduled over a vehicular cloud formed by multiple mobile vehicles, for which a novel \underline{d}ouble-plan-promoted \underline{i}somorphic \underline{s}ubgraph sear\underline{c}h and \underline{o}ptimization (DISCO) paradigm is introduced. We consider two complementary plans: \textit{i)} Plan A achieves an optimal mapping (called $\alpha$) between graph tasks and vehicular cloud in advance to future resource requirements by analyzing historical statistics of uncertain factors; and \textit{ii)} Plan B represents a backup plan which seeks for a feasible mapping (called $\beta$) when mapping $\alpha$ fails to support the completion of tasks during each practical task scheduling process, due to random and dynamic nature of the network. The timeline and key concerns of DISCO are analyzed, while a case study with comprehensive comparisons is conducted to illustrate the commendable performance of DISCO. A series of interesting research directions are carried out to offer references to future resource sharing in the internet of vehicles. 
To effectively process high volume of data across a fleet of dynamic and distributed vehicles, it is crucial to implement resource provisioning techniques that can provide reliable, cost-effective, and timely computing services. This article explores computation-intensive task scheduling over mobile vehicular clouds (MVCs). We use undirected weighted graphs (UWGs) to model both the execution of tasks and communication patterns among vehicles in an MVC. We then study reliable and timely scheduling of UWG tasks through a novel mechanism, operating on two complementary decision-making stages: Plan A and Plan B. 
Plan A entails a proactive decision-making approach, leveraging historical statistical data for the preemptive creation of an optimal mapping ($\alpha$) between tasks and the MVC prior to practical task scheduling. In contrast, Plan B explores a real-time decision-making paradigm, functioning as a reliable contingency plan. It seeks a viable mapping ($\beta$) if $\alpha$ encounters failures during task scheduling due to the unpredictable nature of the network. Furthermore, we provide an in-depth exploration of the procedural intricacies and key contributing factors that underpin the success of our mechanism.
Additionally, we present a case study showcasing the superior performance on time efficiency and computation overhead. We further discuss a series of open directions for future research.
\end{abstract}

\begin{IEEEkeywords}
Task scheduling, vehicular cloud, undirected graph task, stage-wise decision-making, time effectiveness
\end{IEEEkeywords}}

\maketitle

\IEEEpeerreviewmaketitle

%Sec 1
\section{Introduction}
%\IEEEPARstart{T}{he} innovative communication/computing technologies and the acceleration of smart vehicles facilitate the explosive growth of Internet of Vehicles (IoV) as well as computation- and data-intensive applications (also called tasks) such as autonomous driving and in-vehicle entertainment. Most of those promising applications are machine learning (ML)-based and require high computing capacity for data processing, which impose great difficulties to a single vehicle~\cite{1,2}. Although cloud computing offers a good solution, transmitting massive volume of data can lead to heavy burdens on the backhaul/backbone networks and thus unacceptable delay. To respond to this challenge, mobile edge computing (MEC) technique~\cite{3} has been emerged as a powerful solution that brings computing/communication/storage resources close to end-devices. %The original version

%\IEEEPARstart{T}{he} rapid growth of the Internet of Vehicles (IoV) and vehicular computation-intensive applications, e.g., autonomous driving and in-vehicle entertainment, has been made possible by innovative communication and computing technologies, as well as the increasing prevalence of smart vehicles. Many of these applications/tasks are machine learning (ML)-driven and require significant computing power for data processing, making them difficult for a single vehicle to handle~\cite{1,2}. While cloud computing can offer a solution, transmitting large amounts of data from vehicles to cloud servers can lead to congestion on the backhaul network and significant delays. To overcome this challenge, mobile edge computing (MEC)~\cite{3} has emerged that brings computing, communication, and storage resources to the network edge closer to vehicles. %Ali's version

\IEEEPARstart{T}{he} recent surge and rapid expansion of Internet-of-Vehicles (IoV), coupled with the proliferation of vehicular computation-intensive applications~\cite{15}, 
%such as autonomous driving and in-vehicle entertainment, 
can be attributed to the advancements in communication and computing technologies. %This growth is further fueled by the widespread adoption of smart vehicles.
Many modern vehicular applications/tasks rely on machine learning (ML), demanding a substantial amount of computing power for real-time data processing. Often, this exceeds the resource capacity of a single vehicle. 
%While cloud computing offers a solution for this challenge, transmitting large amounts of data from vehicles to cloud servers can lead to significant delays and congestion in backhaul networks. 
To overcome these issues, mobile edge computing (MEC)~\cite{1} has emerged as an alternative approach that brings computing, communication, and storage resources closer to vehicles by deploying them at the network edge.
%Leveraging the fast development of MEC and the functionality of smart vehicles embedded with powerful sensors and processors, collaborative vehicular computing is expected to build a universal computing platform upon offering efficient on-board storage/computation~\cite{4,5}, while such a vehicular cloud formed by mobile vehicles (servers) allows tasks to be scheduled and processed parallelly on different servers. 
%Combine the rapid advancement of MEC technology and the integration of smart vehicles equipped with high-powered sensors and processors into its architecture, collaborative vehicular computing is poised to create a universal computing platform that provides efficient on-board storage and computation~\cite{4,5}. This vehicular cloud (VC), consisting of mobile vehicles acting as servers, enables tasks to be scheduled and processed simultaneously across multiple vehicles. 
%The integration of smart vehicles equipped with advanced sensors and processors into the architecture of MEC technology is rapidly advancing. This integration is paving the way for a universal vehicular computing platform that will provide on-demand storage and computation for emerging vehicular applications~\cite{4,5}. 
In the meantime, we have witnessed the integration of smart vehicles equipped with growing on-board processing power and advanced sensors into the MEC architecture. This integration has paved the way for the migration from conventional MEC to more universal vehicular computing platforms that can provide on-demand storage and computation for emerging vehicular applications~\cite{2}. 
%A paramount example of these computing platforms is a vehicular cloud (VC), consisting of mobile vehicles acting as servers, which enables tasks to be scheduled and processed simultaneously across multiple vehicles. 
A paramount example of such platforms is a mobile vehicular cloud (MVC), comprising nearby moving vehicles acting as collaborating servers, enabling the distribution and simultaneous processing of tasks. MVC facilitates task processing across multiple vehicles through vehicle-to-vehicle (V2V) links, eliminating the need to transfer tasks to edge servers as in conventional MEC systems.

%f1
% Figure environment removed

%This article considers periodically generated computation-intensive tasks (e.g., every second) where a task is represented by an undirected weighted graph structure~\cite{6,7} with multiple subtasks (e.g., data processing units) and undirected edges (interdependency among subtasks). Specifically, a subtask can be assigned to a vehicle for processing, while an edge between two connected subtasks requires the vehicles that are handling them to communicate with each other via vehicle-to-vehicle (V2V) links, to support possible intermediate data exchange during subtask execution. Our key goal is to \textit{investigate how graph tasks can be scheduled over the considered vehicular cloud, namely, obtaining feasible mappings between subtasks and the dynamic vehicles under a responsive and cost-effective manner}.
\subsection{Preliminaries and Motivations}
This work studies the execution of computation-intensive tasks that are generated/collected periodically in an MVC. Particularly, a task is represented by an undirected weighted graph (UWG)~\cite{2,6,8} consisting of multiple subtasks, which are interconnected by weighted edges. An edge between two subtasks describes the required intermediate data exchange during task execution, while the edge weight encapsulates parameters such as the required contact duration. 
%Comparing to directed acyclic graph (DAG)-represented tasks with their subtasks processed sequentially, tasks modeled by UWGs encourage their subtasks to be transmitted and executed on different computing servers in parallel, which thus greatly differs DAG task scheduling in nature. More importantly, the parallelizability calls for strict constraints on keeping the communication links among servers, which imposes crucial challenges in mobile networks with moving features, e.g., IoV.
Comparing to directed acyclic graph (DAG)-structured tasks and their sequential subtask processing~\cite{4}, UWG tasks require concurrent transmission and execution of their subtasks across distinct computing servers. This makes the scheduling of UWG tasks significantly different from that of DAG tasks. Notably, the inherent parallelizability of UWG tasks necessitates stringent constraints on maintaining communication links among the computing servers, presenting pivotal challenges, particularly in MVCs with dynamic vehicles acting as servers.
Paramount examples of UWG tasks over vehicular clouds are multi-target detection, simultaneous localization and mapping~\cite{6}, and D2D-enabled federated learning, etc.

%In securing necessary distributive resources, existing efforts in graph task scheduling mainly focus on onsite mechanism design \cite{6,8,9} where the decisions are made based on the current network conditions. However, significant technical challenges should be considered: 
The common emphasis on graph task scheduling, involving the required resources for tasks and the distribution of subtasks across vehicles, is oriented towards the formulation of efficient \textit{on-site decision-making} mechanisms~\cite{3,6,9}. These mechanisms rely on the current network conditions, implying that task scheduling decisions are determined at the moments when tasks arrive at the system and are required to be scheduled. Nonetheless, on-site decision making faces a set of challenges. 

%\begin{itemize}[leftmargin=4mm] 
% \item The scheduling of graph tasks over multiple servers (vehicles) generally calls for addressing the \textit{subgraph isomorphism} problem which is NP-complete~\cite{10}. More importantly, network dynamics such as changing topology of vehicular cloud, varying channel quality, and uncertain resource supply can further complicate the problem. For example, the contact duration between two vehicles that process two connected subtasks should be larger than or at least equal to the time duration needed to finish the execution of at least one of these subtasks.
% \item In general, the performance evaluation on task scheduling relies on various evaluation indicators such as task completion time and energy consumption. Jointly optimizing different factors can always lead to a \textit{multi-objective optimization} problem where the corresponding pareto optimality is hard to obtain, or a \textit{non-linear integer programming} problem which is NP-hard~\cite{8}. 
%\end{itemize}
\begin{itemize}[leftmargin=4mm] 
 \item To schedule graph tasks over multiple servers (vehicles), finding feasible mappings between subtasks and vehicles is necessary. This requires solving the \textit{subgraph isomorphism} problem, which is NP-complete~\cite{8} (two middle subplots in Fig. 1). Additionally, network dynamics such as changing topology of vehicles, varying channel quality, and uncertainties in resource supply can complicate the problem. For instance, when two vehicles process connected subtasks, the contact duration between them should be at least equal to (or larger than) the time duration needed to finish one of the subtasks, to ensure reliable and seamless task processing.
 \item To evaluate task scheduling effectiveness, diverse metrics are utilized, encompassing the duration of task completion and energy consumption. When the optimization of multiple factors is undertaken concurrently, it gives rise to a complex problem formulation, either be a multi-objective optimization problem where achieving the pareto optimality is challenging, or a non-linear integer programming problem that is NP-hard~\cite{2}. 
\end{itemize}

As a consequence, on-site decision-making may encounter noticeable performance degradations~\cite{11,12}:
%\begin{itemize}[leftmargin=4mm] 
%\item \textit{Unguaranteed time efficiency:} Obtaining feasible mappings between subtasks and servers can always lead to excessive latency due to the complexity of the problem, which significantly decreases the available time for actual resource/data sharing, especially for mobile users, e.g., vehicles. For example, the topology of a vehicular cloud can change during the time of decision-making~\cite{11}, while the corresponding mapping may fail to protect the structure of graph tasks due to the disconnection of V2V communication links. 
%\item \textit{Excessive energy overhead:} A long-lasting decision-making process can also bring extra energy consumption, which may further result in air pollution and greenhouse effect from the long-term perspective.
%\item \textit{Unsatisfied experience of service provisioning:} Apparently, the non-negligible latency on decision-making will also impact the user experience since a certain amount of resources are reserved while waiting for the final mapping. For example, only partial of the vehicles can finally be assigned with subtasks while any resources that have been put aside for failed candidate vehicles can cause poor experience~\cite{12}. 
%\end{itemize}
\begin{itemize}[leftmargin=4mm] 
\item \textit{Additional latency and unguaranteed task completion:} Finding feasible mappings between subtasks and vehicles can result in long delays due to the problem hardness. This further reduces the amount of time for actual resource sharing. For instance, 
%the topology of a VC can change during decision-making\cite{11}, which may cause the mapping fails to maintain the structure of graph tasks due to the disconnection of V2V communication links.
during decision-making, the topology of an MVC may change, causing the obtained mapping to fail in maintaining the structure of graph tasks due to V2V links being disconnected, further result in the incompletion of tasks.

\item \textit{Excessive energy consumption:} Obtaining mappings between subtasks and vehicles is a complicated process that calls for significant computing resources, leading to high energy usage. This, in turn, can have a detrimental effect on the sustainability of the system, e.g., carbon footprint.
%as it leads to a notable carbon footprint. 

\item \textit{Low quality of experience:} Extended decision-making processes can adversely affect the vehicle experience within the MVC. This issue arises due to the fact that only a subset of vehicles is selected for task execution, leading to subsequent compensation, leaving others uncompensated.
%This occurs because only a subset of vehicles can get chosen for task execution (and subsequent compensation), while others remain uncompensated. 
%This disparity can significantly diminish the overall experience for the latter group. 
\end{itemize}

%Motivated by the above-mentioned challenges, this article makes the first attempt to introduce an interesting graph task scheduling paradigm over vehicular cloud, called \underline{D}ouble-Plan-Promoted \underline{I}somorphic \underline{S}ubgraph Sear\underline{c}h and \underline{O}ptimization (DISCO) that involves: 
%\textit{i) Plan A}, which seeks for the optimal mapping ahead of future practical task scheduling events via analyzing historical statistics of the network, which we refer to as mapping $\alpha$ to distinguish. Specifically, mapping $\alpha$ tries to guarantee the performance of graph tasks with a high probability; and \textit{ii) Plan B}, helping with obtaining a feasible mapping $\beta$ during each practical task scheduling event under the current network conditions if $\alpha$ fails. As an efficient backup, Plan B aims to keep the reliability of the resource provisioning network. 

\subsection{Overview and Contributions}
%To schedule graph tasks over the MVC, we allow each subtask of a graph task to be assigned to a vehicle for processing, while the vehicles handling connected subtasks should communicate with each other through vehicle-to-vehicle (V2V) links to exchange data during the task execution~\cite{2,6}. 
\textit{In this work, we aim to optimize the scheduling of UWG tasks in a MVC, with the ultimate objective of obtaining feasible mappings between subtasks and mobile vehicles, ensuring reliability, promptness, and cost-efficiency of task execution}. In light of above-mentioned challenges, we propose a novel task scheduling mechanism, which involves two complementary decision-making processes: \textit{(i)} Plan A seeks to identify the optimal mapping for future task scheduling events by analyzing historical network statistics (such a decision-making process takes place before practical task scheduling events). This mapping, referred to as $\alpha$, aims to increase the likelihood of executing future tasks with high performance. \textit{(ii)} In case $\alpha$ fails at practical/actual task scheduling events, Plan B obtains a feasible mapping ($\beta$) suitable for current network conditions (this decision-making process is triggered at the practical task scheduling events). 
Plan A contributes to the streamlining of real-time task scheduling via removing the need for on-site decision making as long as the obtained template $\alpha$ remains feasible, thereby enhancing the promptness of the overall scheduling process.
Furthermore, the incorporation of Plan B ensures the reliability of task scheduling, mitigating the risk of execution failures in real-time. \textit{This additional layer of preparedness (i.e., Plan A) differs our study from existing works through introducing a new degree of freedom in task scheduling process.}

%In this article, we present comprehensive insights on how the proposed paradigm achieves time-efficient, cost-effective, and reliable resource provisioning in IoV, where major contributions are highlighted below:
This article provides an in-depth study of the stage-wise paradigm and reveals how it accomplishes time-efficient, cost-effective, and reliable resource provisioning in an MVC. We highlight major contributions below: 
%\begin{itemize}[leftmargin=4mm] 
%\item A novel double-plan-promoted graph task scheduling paradigm is introduced in IoV, where the framework, timeline, and key issues are analyzed in detail. 
%\item A case study is investigated to demonstrate how the proposed paradigm can be implemented in practice.
%\item Simulation results illustrate that the proposed paradigm achieves commendable performance on critical indicators such as task completion time, data exchange cost, and time efficiency.
%\end{itemize}
\begin{itemize}[leftmargin=4mm] 
%\item We develop a novel approach called DISCO that envisions a new methodology for the scheduling of graph tasks in MVC. We delve into DISCO's timeline and the significant obstacles it overcomes.
\item We develop a novel mechanism for scheduling graph-structured 
tasks in MVCs, involving two complementary decision-making processes. The detailed procedure (e.g., timeline), key challenges, and design considerations are discussed. 
\item We conduct a case study to demonstrate how our mechanism can be practically implemented over an MVC.
\item Through simulation results, we demonstrate that the proposed stage-wise graph task scheduling paradigm performs good on key performance metrics, including task completion time, data exchange cost, and time efficiency. Noteworthy future research directions are also discussed.
\end{itemize}
%\vfill

\section{Stage-Wise Decision-Making for Graph Task Scheduling over MVCs}
\subsection{Framework and Timeline}
%We are interested in a certain region (e.g., Google park) consists of: \textit{i)} multiple mobile vehicles as computing service providers (resource supply); \textit{ii)} a platform (e.g., an edge server) that generates (or collects) graph tasks (resource demand) and coordinates the task scheduling procedure, and \textit{iii)} several access points (APs, e.g., road side units) that help vehicles interact with the platform through vehicle-to-infrastructure (V2I) links. Specifically, our proposed paradigm focuses on a fixed number of vehicles in the considered region during a certain period of time, where the dynamics can be reflected by the varying contact duration among vehicles (rather than the arrival and departure of vehicles, similar ideas can also be found in~\cite{6,7}). Moreover, graph tasks are announced by the platform periodically (e.g., a task per second), while during each task scheduling event, subtasks can be distributed over vehicles for processing. The corresponding framework is shown in Fig. 1, which considers a star graph-represented~\cite{8,14} task as an example. 
We are interested in an IoV region (e.g., Google park), consisting of three key participants: \textit{(i)} Multiple mobile vehicles that act as computing servers; \textit{(ii)} A platform (e.g., an edge server) that generates/collects task requirements and coordinates task scheduling; \textit{(iii)} Several access points (APs, e.g., road side units), that allow vehicles to interact with the platform through vehicle-to-infrastructure (V2I) communication links. 

The platform announces graph tasks periodically, e.g., a certain number of tasks per second, while during each task scheduling event, subtasks can be allocated to vehicles for processing. Fig. 1 illustrates an example of this framework, considering a star graph task~\cite{2,8} with 6 subtasks denoted by A, B, C, D, E, and F. 
%Specifically, obtaining feasible mappings calls for addressing \textit{subgraph isomophism problem} that extracts subgraphs with the same structure as the UWG task from the MVC topology, while maintaining proper weights, e.g., the contact duration between two vehicles handling two connected subtasks should be larger than the weight associated with these subtasks. Upon having a certain goal to describe our desired performance, e.g., achieving the minimization of delay and energy cost for task completion, a best isomorphic subgraph can be selected to guide how the subtasks should be assigned over a MVC (e.g., involving the vehicles with the most powerful processing capabilities). 
Scheduling the graph task over the MVC involves solving the \textit{subgraph isomorphism problem}, aiming to identify subgraphs within the MVC topology that mirror/replicate the structure of the UWG task. During this process, the appropriate weights are retained, ensuring, for example, that the contact duration between vehicles handling interconnected subtasks exceeds the weight associated with these subtasks. Once a specific performance goal is defined (e.g., minimizing delay and energy cost), the best isomorphic subgraph can be selected. This selection guides how subtasks are assigned across the MVC. 

%To capture the random and unpredictable nature of mobile networks\cite{12}, key uncertainties should be considered:
To ensure an accurate portrayal of IoV, it is essential to consider key uncertainties that capture the network dynamics and thereby pose challenges to the design of task scheduling mechanisms:
%\begin{itemize}[leftmargin=4mm] 
%\item \textit{Fluctuant service quality:} The on-board resource supply of vehicles can always be fluctuant. For example, a vehicle may have local tasks that require a certain amount of computing/storage resources, which thus impact the computing service provided to the assigned graph tasks.
%\item \textit{Dynamic contact duration among vehicles:} A V2V contact event can occur when two vehicles are within the communication coverage of each other. Specifically, the mobility of vehicles always leads to limited and dynamic contact duration among vehicles, which calls for feasible task scheduling mechanism design to protect task structures, e.g., to support the data exchange required by connected subtasks in a graph task.
%\item \textit{Varying channel quality:} The wireless channel qualities of V2I and V2V links are changing with time due to factors such as the mobility of vehicles, obstacles (e.g., buildings), and transmission power supply. The varying V2I channel quality mainly results in fluctuant data transmission rate between vehicles and APs, while the changing V2V channel quality can impose data exchange cost among vehicles, e.g., the intermediate data transmission between two vehicles that are handling two connected subtasks of a graph task can cause energy consumption.
%\end{itemize}
\begin{itemize}[leftmargin=4mm] 
\item \textit{Fluctuant resource supply:} 
%Vehicles' supply of resources can fluctuate over time, as they also have to cope with their local tasks (rather than the graph task assigned by the platform only), leading to uncertain service quality and cost. For example, a vehicle requires a specific amount of computing/storage resources for its local workload, which thus affecting the service provided to a graph task. Also, when coming to prioritize the completion of graph tasks, the service cost of a vehicle may raise since its own tasks should wait for the release of the occupied resources. Such fluctuations can bring risks to the scheduling performance of our considered graph tasks, and even the incompletion.
The availability of vehicles' resource supply is subject to fluctuations over time. This variability is not only influenced by the assigned graph task from the platform, but also by the vehicles' local workloads. Such fluctuations introduce uncertainty in service quality and costs. 
%For example, a vehicle may require certain computing and storage resources for its local workload, impacting the services it can provide for a graph task. 
Additionally, forcing the prioritization of the completion of graph tasks may require additional expenses to vehicles, as their own tasks may have to wait for the release of occupied resources. These fluctuations pose risks to graph task, potentially resulting in their incomplete execution. 

\item \textit{Dynamic contact duration among vehicles:} 
%When two vehicles are within communication range of each other, they can experience a V2V contact event. Since vehicles are often moving and have limited contact duration with each other, it is crutial to design a task scheduling mechanism that can protect graph task structures, which includes supporting data exchange through V2V links for connected subtasks in a graph task. For example, a practicable mechanism should offer timely scheduling decisions for subtask-vehicle assignment before the MVC topology changes.
When two vehicles come within their communication range, a V2V contact event occurs. Given the mobile nature of vehicles and their varying contact windows, designing a task scheduling mechanism requires safeguarding task structures. This includes facilitating data exchange through V2V links for interconnected subtasks within a graph task. Additionally, a practical mechanism needs to provide prompt scheduling decisions before the MVC topology changes.

\item \textit{Varying channel condition:} 
%Wireless channel quality for V2I and V2V links is subject to change over time due to numerous factors, e.g., vehicular mobility, obstacles, transmission power supply. This variability in channel quality can lead to inconsistent data transmission rates and further delay as well as energy consumed by data delivery. Such a concern will bring uncertain completion of graph tasks, e.g., a task may confront a failure for not be completed within its tolerant time. Besides, this also impose certain overhead for interactions among different players (e.g., energy consumed for data exchange under a poor V2V channel quality).
The wireless channel quality for both V2I and V2V links is susceptible to fluctuations over time due to various factors such as vehicle movement, obstacles, and variations in the interference of the environment. The variations in channel quality can lead to inconsistent data transmission rates, creating the potential for delays and increased energy consumption during data transmission. This variability becomes a significant concern for the successful completion of graph tasks, with the possibility of task failures if they exceed their allowable completion time. Additionally, this variability introduces overhead in interactions, manifesting as increased energy consumption for data exchange under unfavorable V2V channel conditions.
\end{itemize}

%Timeline for Proposed Graph Task Scheduling
%To better illustrate our proposed paradigm, the corresponding timeline is shown in Fig. 2, which is divided into two segments: Before practical task requests, Plan A helps with obtaining mapping $\alpha$ via analyzing historical statistics of the uncertain factors; while during each task scheduling event, Plan B first checks whether $\alpha$ works practically in the current network, if not, Plan B seeks for a feasible mapping $\beta$ under a timely manner. Specifically, in Event 3, a certain period of time $\tau$ has been consumed to obtain mapping $\beta$, so that the actual service delivery can only start at $t+\Delta t+\tau$. Correspondingly, improve the availability of $\alpha$ and accelerate the search of $\beta$ represent the key concerns in this paradigm. 
Our stage-wise task scheduling mechanism is designed to ensure reliable task completion amidst the uncertainties discussed above. It comprises two complementary decision-making plans operating on different time segments (see Fig. 2). Plan A involves analyzing historical statistics of uncertain factors, such as the historical local workload of vehicles, past information on contact duration between vehicle pairs, and the distribution of channel quality. This analysis aims to derive a mapping, denoted as $\alpha$, before practical task scheduling events. Subsequently, during each task scheduling event, Plan B assesses whether $\alpha$ is viable under the current network conditions (e.g., Events 1 and 2 in Fig. 2). In essence, Plan B examines whether the pre-determined vehicles in $\alpha$ can sustain sufficient contact duration and meet the required service quality of the task. This includes ensuring that the resources of selected vehicles and their V2V link durations can guarantee the smooth completion of their assigned subtasks. If $\alpha$ proves effective, it is executed promptly; otherwise, Plan B searches for a feasible alternative mapping, denoted as $\beta$ (e.g., Event 3), which involves selecting a set of vehicles to support task execution given the current network condition. Event 3 requires a certain period of time to obtain $\beta$, denoted by $\tau$, causing the actual service delivery to start at $t+2\Delta t+\tau$ in Fig. 2. Therefore, enhancing the feasibility of $\alpha$ and expediting the search for $\beta$ represent major challenges.
%of our proposed mechanism, which are detailed next.  

%f2
% Figure environment removed

\subsection{Key Challenges and Considerations}
%This section analyzes significant concerns when designing DISCO for graph task scheduling over mobile vehicles. 
In the following, we address noteworthy considerations in the development of our stage-wise mechanism.
% for scheduling graph tasks over mobile vehicles. 

%\noindent
%First, \textit{the optimality of mapping $\alpha$ represents the key concern in Plan A.} Our proposed Plan A aims to optimize the task scheduling performance from a long-term perspective, by achieving the optimal mapping $\alpha$ in advance of future task requirements, according to the historical statistics of uncertain factors in the network. During this time, accurate information extraction of these uncertain factors represents one of the most significant challenges, which leaves a direct impact on whether $\alpha$ works for future network environment. For example, inaccurate statistics of vehicular contact duration can lead to a damage of task structures. Then, possible risks can generally exist in terms of cost (e.g., delay and energy consumption) and structure preservation. For example, applying mapping $\alpha$ during task scheduling can face the risk where the task completion time exceeds its corresponding tolerance. Also, communications among vehicles may fail to support the data exchange required by connected subtasks. To this end, the rational estimation and control of risks represent another urgent and critical issue. The above discussions should be carefully considered to keep the optimality of mapping $\alpha$, and thus improve the corresponding availability.
\noindent
\textbullet~\textit{Mapping $\alpha$: achieving the optimality and risk awareness with long-term vision}. 
%The primary focus of Plan A is to optimize the efficiency and viability of implementing mapping $\alpha$ directly into practical scheduling. Achieving this goal involves devising a strategy to improve the long-term performance of task completion. This strategy anticipates and identifies the optimal mapping $\alpha$ by leveraging historical data and statistics related to uncertain network factors. For instance, by considering distributions derived from past information on these uncertain factors (e.g., the V2V contact duration usually obeys a certain exponential distribution), Plan A aims to determine the most favorable $\alpha$ that minimizes the expectation of delay and energy costs associated with task completion.
%However, the precision of gathered information on the uncertainties remains a key hurdle, as it directly impacts the efficacy of $\alpha$ for future network conditions. For example, if the statistics about how long vehicles are in contact with each other are inaccurate, task execution using mapping $\alpha$ can lead to violating the corresponding requirements since communications among vehicles may fail to support the data exchange between connected subtasks. 
%Thus, utilizing mapping $\alpha$ is associated with a series of risks, in terms of cost (e.g., delay and energy consumption), and task structure preservation (a key to the completion of a graph task). Further, mapping 
%$\alpha$ may result in the task taking longer to complete than its delay tolerance, further leading to a failure. Therefore, it is crucial to estimate and control risks effectively, while reaching the optimality of mapping 
%$\alpha$ and improving its applicability.
This plan proactively anticipates and identifies the optimal mapping $\alpha$ by leveraging historical data and statistics related to uncertain network factors. For instance, by considering probability distributions derived from past information on uncertain factors (e.g., the V2V contact duration typically follows exponential distribution~\cite{6}), Plan A strives to determine the most favorable $\alpha$ that minimizes the expected delay and energy costs for task completion.
However, the precision of gathered information on uncertainties presents a significant challenge, as it directly influences the efficacy of $\alpha$ for future network conditions. For example, inaccuracies in statistics about the V2V contact duration can result in task execution using $\alpha$ violating task completion requirements. This is because communications among vehicles may fail to support the data exchange between connected subtasks. Utilizing mapping $\alpha$ thus entails addressing a series of risks, such as the timeout of tasks and the damage of task structure. Additionally, mapping $\alpha$ may result in the task taking longer to complete than its delay tolerance, leading to failures. Therefore, effective estimation and control of risks are crucial while striving for the optimality of mapping $\alpha$ and enhancing its applicability.

%\noindent
%Then, \textit{the time-effectiveness of mapping $\beta$ represents the key concern in Plan B.} Our proposed Plan B considers a short-term perspective, which occurs in each practical task scheduling event. Specifically, Plan B first examines whether mapping $\alpha$ works smoothly in the current network condition, e.g., if $\alpha$ can ensure the successful completion of graph tasks. Otherwise, Plan B solves isomorphic subgraph search and optimization problems instantaneously, according to the current network information, which obtains a feasible mapping $\beta$. During this time, time-effectiveness represents the most important concern since factors such as the topology of vehicular cloud, on-board resources, and channel quality can change over time. For example, a long-lasting decision-making process (e.g., a long period of time spent on searching and optimizing mapping $\beta$) may bring a failure of task scheduling (and thus the uncompletion of graph tasks), due to the mobility of vehicles. Therefore, achieving an acceptable computational complexity of the scheduling algorithm design remains nonnegligible. 
\noindent
\textbullet~\textit{Mapping $\beta$: ensuring the usability and time effectiveness with short-term vision}. Our approach for Plan B revolves around short-term objectives during each task scheduling event. This plan starts with evaluating the functionality of mapping $\alpha$ in the current network conditions, ensuring the successful completion of tasks by the vehicles associated with $\alpha$. If not, Plan B promptly addresses isomorphic subgraph search and subtask allocation optimization using the latest network information to obtain a viable mapping $\beta$. Thus, the primary concern for Plan B is the prompt acquisition of a feasible $\beta$. In particular, time-effectiveness is paramount in this process, given that the topology of the MVC, on-board resources of vehicles, and wireless channel qualities can change over time. Prolonged decision-making processes, such as spending excessive time on searching and optimizing $\beta$, can lead to task failures.
%due to the mentioned dynamics. 

%All in all, the proposed paradigm aims to achieve an optimal mapping $\alpha$, and a feasible mapping $\beta$ under a timely manner, to support responsive, cost-effective and reliable graph task scheduling in dynamic IoV environment.
In summary, our mechanism aims to proactively establish an optimal and risk-aware mapping $\alpha$ in advance, increasing the likelihood of its successful implementation in future practical task scheduling. Additionally, it focuses on obtaining a feasible mapping $\beta$ when graph tasks are practically scheduled, serving as robust support in case $\alpha$ encounters failures. These complementary mappings contribute to prompt, cost-effective, and reliable scheduling of graph tasks in dynamic MVCs.

\section{Case Study}
%This section investigates a case study associated with our proposed DISCO for graph task scheduling over vehicular cloud, where the platform announces one graph task with multiple subtasks periodically, for analytical simplicity. 
This section explores a case study that examines the deployment of our proposed mechanism.
%We assume that the platform periodically announces one graph task with multiple subtasks to simplify analysis.

\subsection{Key Modelling}
\noindent
%\textbf{Model of graph task.} 
We consider a computation-intensive task modelled as an UWG $\bm{\mathcal{G}^{task}}$ with a subtask set, an edge set and an edge weight set. Each subtask has three attributes: tolerable completion time, data size, and the required computing resources; while each edge describes the interdependency between two subtasks~\cite{6,7}. Additionally, each edge is associated with a weight that indicates the required time for data exchange. For example, when two connected subtasks are completed on separate vehicles with completion times of 2 and 3 seconds, the edge weight is 2 seconds to ensure continuous connection during the execution. Also, the contact duration of two vehicles that handle these connected subtasks should be long enough (i.e., larger than or at least equal to the corresponding weight) to allow data transfer between subtasks.
%We consider a computation-intensive task modelled as an UWG $\bm{\mathcal{G}^{task}}$ with a subtask set $\bm{\mathcal{V}^{task}}$, an edge set $\bm{\mathcal{E}^{task}}$ and an edge weight set $\bm{\mathcal{W}^{task}}$. Each subtask in $\bm{\mathcal{V}^{task}}$ has three attributes: tolerable completion time, data size, and the required computing resources. An edge in $\bm{\mathcal{E}^{task}}$ describes the interdependency between two subtasks~\cite{6,7}. Additionally, each edge is associated with a weight (collected in $\bm{\mathcal{W}^{task}}$) that indicates the required time for data exchange, which depends on the execution order of connected subtasks. For example, when two connected subtasks are completed on separate vehicles, with completion times of 2 seconds and 3 seconds, the edge weight is 2 seconds to ensure continuous connection during the execution time. Also, the contact duration of two vehicles that handle these connected subtasks should be long enough (i.e., larger than or at least equal to the corresponding weight) to allow data transfer between the subtasks.

%\noindent
%\textbf{Model of MVC.} 
We consider the existence of a certain number of vehicles in the MVC region, where the changes in contact duration among them reflect their dynamics \cite{6,7}. Accordingly, we model the MVC as an undirected weighted graph $\bm{\mathcal{G}^{cloud}}$ consisting of a set of vehicles, an edge set and a weight set. To capture dynamic resource supply, each vehicle in the MVC has an attribute $f$ that describes its computing capability (e.g., its CPU frequency). Also, each edge indicates the possibility of one-hop V2V communication between two vehicles. Each edge weight has two attributes: anticipated V2V contact duration $t$ between vehicles, and data exchange cost $c$ for processing of connected subtasks, e.g., energy consumed by data sharing. These attributes will be modelled as random variables to describe the network dynamics.
%We model the MVC as an undirected weighted graph $\bm{\mathcal{G}^{cloud}}$ consisting of a set $\bm{\mathcal{V}^{cloud}}$ of vehicles, an edge set $\bm{\mathcal{E}^{cloud}}$ and a weight set $\bm{\mathcal{W}^{cloud}}$. To capture the dynamics of resource supply, each vehicle in $\bm{\mathcal{V}^{cloud}}$ has an attribute $f$ that describes its computing capability in terms of its CPU frequency. Also, each edge in $\bm{\mathcal{E}^{cloud}}$ indicates the possibility of one-hop V2V communication between two vehicles. Also, each edge weight in $\bm{\mathcal{W}^{cloud}}$ has two attributes: \textit{i)} anticipated V2V contact duration $t$ between vehicles and \textit{ii)} data exchange cost $c$ for processing of connected subtasks, e.g., energy consumption spent on data sharing. These two attributes are both modeled as random variables to capture vehicles' mobility.  

%In this case study, we assume that the IoV platform generates/collects graph tasks. A subtask can be offloaded to a feasible vehicle for processing via getting access to a nearby AP, while the data transmission rate of the corresponding V2I link is considered as a random variable $r$ to reflect the time-varying nature of channel quality. 
In this case study, we assume that the platform generates or collects graph tasks, where subtasks can be delegated to suitable vehicles for processing through a nearby AP. Note that the data transmission rates of V2I links are  represented as random variables, denoted as $r$, to capture the time-varying channel qualities. 

\subsection{How We Design the Two Plans}
%This case study aims to minimize both \textit{i)} the task completion time, which depends on the time of the latest finished subtask via considering data transmission and execution delay; and \textit{ii)} the overall data exchange cost, which quantizes the overhead (e.g., energy consumption) on intermediate data sharing among vehicles during subtask processing. Based on which, the cost function $\mathbb{C}$ is formulated as the weighted sum of the above two objectives. 
Our goal in this case study is to map $\bm{\mathcal{G}^{task}}$ to $\bm{\mathcal{G}^{cloud}}$, while minimizing the task completion time (data transmission and execution delays) and the overall data exchange costs among vehicles during task processing (when two vehicles are processing connected subtasks, cost $c$ is incurred). This is achieved by formulating a cost function $F(f,c,r)$, weighting both task completion time and data exchange cost by certain coefficients, and sums them together. 
%$\mathbb{F}(f,c,r)$

%Since the timeline of the proposed task scheduling is divided into two segments, our key goal in plan A is to minimize the expected value of cost function $\mathbb{C}$, which we refer to as $\overline{\mathbb{C}}$. Specifically, possible risks can exist during this time owing to the uncertainties (e.g., random variables $f$, $t$, $c$, and $r$).
%Thus, Plan A considers two key risks as probabilistic constraints via analyzing the historical statistics, e.g., distributions of $f$, $t$, $c$, and $r$: \textit{i)} the risk of an unacceptable task completion time, which is calculated by the probability that the task completion time exceeds the maximal tolerant time among subtasks; and \textit{ii)} the risk of structural damage of the graph task, which is expressed by the probability that any vehicle pair fails to support the data exchange time required by the corresponding connected subtasks. To this end, the optimization problem in Plan A is given in by the following (1), where risks should be controlled within a certain range. 

We partition the task scheduling timeline into two segments. In Plan A, our primary goal involves minimizing the expected cost function, denoted as 
$\textbf{E}[F]$. While developing Plan A, it is crucial to account for potential risks stemming from uncertainties.
%(i.e., random variables $f$, $t$, $c$, $r$). 
Hence, Plan A involves a thorough analysis of historical statistics of the above random variables, to formalize two key risks as probabilistic constraints. The \textit{first risk} pertains to the probability of task completion time surpassing the maximum tolerable execution time for subtasks, ensuring its timely completion, based on the statistical information of $f$ and $r$. For example, the probability that the completion time fails to catch the deadline of the task. The \textit{second risk} addresses the probability of structural damage to the graph task according to the distribution of $t$, quantified by the likelihood of any vehicle pair failing to facilitate the transmission of necessary data for their assigned subtasks. For instance, the probability that the contact duration $t$ between two vehicles that are handling connected subtasks can not support the required data exchange.
Therefore, Plan A formulates an optimization problem that minimizes $\textbf{E}[F]$ while imposing control constraints on these risks within specified ranges (e.g., $\varepsilon_1$ and $\varepsilon_2$ in Fig. 3 are set as 30$\%$). 
%For instance, we limit the probability of overtime task completion under 30$\%$.

%We divide the task scheduling timeline into two segments. In plan A, our main objective is to minimize the expected value of cost function, which we refer to as $\overline{\mathbb{C}}$. When developing Plan A, it is important to consider possible risks due to uncertainties (i.e., random variables $f$, $t$, $c$, and $r$). Therefore, Plan A puts efforts in analyzing historical statistics, e.g., distributions of $f$, $t$, $c$, and $r$, to formalize two key risks as probabilistic constraints. The first risk is the probability of task completion time exceeding the maximal tolerable execution time among subtasks, which ensures timely completion of graph tasks. The second risk is the probability of structural damage to the graph task, which is expressed by the probability of any vehicle pair failing to support the transmission of necessary data for their corresponding assigned subtasks. Plan A solves the optimization problem to minimize $\overline{\mathbb{C}}$, under risk control constrains, which  control the above-discussed risks within certain ranges. For example, the probability of an overtime task completion can be limited under 30$\%$.

During practical task scheduling, Plan B first checks the feasibility of mapping $\alpha$, e.g., whether the V2V connections and computing capabilities of vehicles associated with $\alpha$ can support the task structure and its tolerable completion. If $\alpha$ is unfeasible, Plan B looks for a feasible mapping $\beta$ that minimizes the practical cost function (i.e., $F$) while respecting the time and structure preservation constraints based on the current network condition (e.g., practical value of random variables). These constraints ensure the on-time task completion and data exchange between  connected subtasks to maintain the corresponding task structure.

%f3
% Figure environment removed

%\subsection{Solution design}
%The problem in both Plan A and Plan B refers to a non-linear integer programming problem which is NP-Hard. To this end, to solve the isomorphic subgraph search and optimization problem under probabilistic constraints, in Plan A, we first decouple the problem into two subproblems: \textit{i)} feasible mapping search, and \textit{ii)} optimal mapping $\alpha$ selection. In the former, we borrow the idea from~\cite{13} and investigate an efficient algorithm under risk analysis, which can achieve all the feasible mappings while providing acceptable computational complexity. Then, in the latter, the optimal mapping $\alpha$ can be obtained with the lowest expected value of cost function among given mappings. 
%During each practical task scheduling event, Plan B checks if $\alpha$ works in the current network, or look for a mapping $\beta$ when necessary, via utilizing a similar algorithm designed in Plan A.
 
%which starts with obtaining a pivot subtask and the corresponding candidate vehicles, via considering the degree, probabilistic constraints and neighborhood information in both task and VC graphs. Then, the concept of eccentricity is introduced to determine the candidate regions for each subtask, which is 

%Both Plan A and Plan B involve finding the allocation/mapping vectors between subtasks and vehicles. Thus, they face a intractable nonlinear integer programming problem with complicated probabilistic constraints, which is NP-hard. 
%Obtaining the optimal mapping in a timely manner is thereby a daunting challenge, particularly as the time required for solving the problem increases exponentially with growing vehicular density
%and the complexity of MVC/task topology. 

Both Plan A and Plan B center around determining the allocation and mapping vectors between subtasks and vehicles. Consequently, they confront a nonlinear integer programming problem characterized by intricate probabilistic constraints, which is NP-hard. As a result, achieving the optimal mapping within a reasonable timeframe poses a formidable challenge, especially considering that the computational time escalates exponentially with the expanding vehicular density and the intricate MVC/task topology.
To tackle this issue and optimize the subgraph search while considering probabilistic constraints, Plan A divides the problem into two subproblems. The first subproblem concerns finding feasible mappings, for which we are inspired by \cite{13}, and accomplish a risk-aware mapping search algorithm, guaranteeing that all feasible mappings adhere to reasonable risks. 
In the second subproblem, we select the optimal mapping, $\alpha$, by identifying the one with the lowest expected cost function value among all possible mappings. In practical task scheduling events, Plan B verifies if $\alpha$ is compatible with the current network. If necessary, it searches for a mapping $\beta$ using a similar algorithm to the one used in Plan A (since the time consumed in obtaining mappings in plan A is acceptable). Fig. 3 and Fig. 4 show the flow chart, and our design logic in this case study.
%A flow chart is shown by Fig. 3, while Fig. 4 shows our design logic of this case study.

%Inspired by\cite{13}

%algo
% Figure environment removed


%\begin{algorithm}[h!t]
%{\footnotesize
%\caption{Outline of our case study}
%%Plan A 
%\SetKwInOut{Input}{Input}\SetKwInOut{Output}{Output} \SetKwInOut{Stage}{Stage}
%%\hrulefill
%
%%\hline\hline
%~~~~~~~~~~~~~~~~~~~~~~~~~~~~~~~~\Stage{\textbf{Plan A}}
%
%\Input{$\bm{\mathcal{G}^{task}}$, $\bm{\mathcal{G}^{cloud}}$, statistic characteristics of $f$, $t$, $c$, $r$}
%\Output{Mapping $\alpha$}
%
%1. Obtain the degree information of all the subtasks in $\bm{\mathcal{G}^{task}}$
%
%2. For each subtask, determine the candidate vehicles and record them in a set $\rho$ 
%
%3. For each subtask, calculate its eccentricity denoted by $\gamma$, referring to the maximum path length from it to others,
%
%4. Select the subtask with the maximum value of $\gamma|\rho|$ as the root subtask
%	
%5. Check all the candidate vehicles of the root subtask, saved by set $\rho^{root}$
%	
%6. For the root subtask, center on each candidate vehicle in $\rho^{root}$, set $\gamma$ as radius, extract a subgraph from $\bm{\mathcal{G}^{cloud}}$, and check all the possible mappings for scheduling $\bm{\mathcal{G}^{task}}$
%
%7. After obtaining all the mappings, check the constraints on risks related to the distribution of $f$, $t$, $c$, $r$, and delete the unavailable ones
%
%8. Determine the optimal mapping with the minimum expected value of cost function, as mapping $\alpha$
%
%%\hrulefill
%~~~~~~~~~~~~~~~~~~~~~~~~~~~~~~~~\Stage{\textbf{Plan B}}
%
%\Input{$\bm{\mathcal{G}^{task}}$, $\bm{\mathcal{G}^{cloud}}$, practical values of $f$, $t$, $c$, $r$}
%\Output{Mapping $\beta$}
%
%1. Check if $\alpha$ is available in the current network condition
%
%2. If yes, $\beta\leftarrow\alpha$
%
%3. If not, look for $\beta$ by using an algorithm similar to Plan A, according to 
%the practical values of $f$, $t$, $c$, $r$
%
%%\hrulefill
%}
%\end{algorithm}

%5
\subsection{Evaluation}
We compare the performance of our proposed \textbf{s}tage-\textbf{w}ise gr\textbf{a}ph \textbf{t}ask \textbf{s}cheduling mechanism (SWATS) against multiple baselines~\cite{2,15}: \textit{i)} onsite task scheduling (Onsite), which only uses Plan B; \textit{ii)} random task scheduling (Random), which randomly assigns subtasks to available vehicles at each task scheduling event~\cite{2}; \textit{iii)} time-preferred (TimePref) and degree-preferred (DegreePref) task scheduling, where TimePref maps each subtask to the vehicle with the lowest execution time, where DegreePref assigns each subtask to the vehicle with the largest number of V2V connections with others; and \textit{iv)} exhaustive search (ExSearch), which examines all possible mappings and selects the best one with the lowest cost function value at each practical scheduling event. We assume: $f$ (GHz) obeys a gaussian distribution with its mean in $[2,4]$ and variance in $[0.04,0.07]$ for each vehicle, $t$ (second) follows an exponential distribution with its mean falling in $[5,16]$, $c$ follows a normal distribution with its mean in $[0.03,0.07]$ and variance of 0.001, while $r$ (Mb/s) obeys a gaussian distribution with its mean in $[5,7]$ and variance of 0.55. Note that practical values of the above variables can neither be negative nor be too large in real-world networks due to the vehicular hardware settings and communication standards. We thus constrain/clip them as: $f\in[1.5,4.5]~\text{GHz}$, $t\in[0,60]~\text{seconds}$, $c\in[0.025,0.075]$ and $r\in[4,8]~\text{Mb/s}$~\cite{6,7}. We set the
weight coefficients in cost function as 0.5 since the task completion time and data exchange cost are on the same order of magnitude, according to our parameter settings. 
\vfill

In Fig. 5, we compare the average running time (ART) between our SWATS and other methods, which reflects the time spent on practical task scheduling events on searching for the best mappings. We consider 100 simulations (100 task scheduling events) to evaluate ART, where our results demonstrate the time effectiveness of SWATS.  
% Specifically, we use logarithmic representation to highlight the gap among different methods, upon considering three task types given by Fig. 3. Apparently, with the increasing number of subtasks and vehicles/connections, the ART curve of each method shows a general raising tendency owing to the growing problem size. Specifically, ART of ExhaustiveS illustrates a sharp rise due to a large searching space, which is unacceptable in real-world dynamic IoV. For example, considering tadpole-based task (type 3) and 15 vehicles in Fig. 4, the value of ART even exceeds $10^{3}~\text{seconds}$. Since Onsite method applies our proposed Plan B with an available computational complexity, it outperforms ExhaustiveS but the corresponding ART still stays above other methods due to that Onsite aims to obtain all the feasible mappings and selects the optimal one among them. Although Random, TimeGreedy, and DegreeGreedy achieves low values of ART and sometimes outperform our proposed DISCO, which always fail to obtain all the feasible mappings and thus leads to larger average value of cost function as shown in the following Figs. 5-7. In other words, these three algorithms usually gets a locally optical solution rather the global one. Our proposed DISCO achieves commendable performance on ART since the prior obtained mapping $\alpha$ can work in most simulations, which greatly reduces the decision-making latency and thus suits well in dynamic wireless networks. 
We use logarithmic representation to evaluate different methods for three task types given by Fig. 1 
%(i.e., Type 1: Bull graph with 5 nodes; Type 2: Star graph with 6 nodes; Type 3: Tadpole graph with 7 nodes). 
As anticipated, the duration for each method to acquire feasible mappings increases with the larger number of subtasks and vehicles/connections, as indicated by the rising ART values. ExSearch has a particularly sharp increase in ART due to its large search space, which is not suitable for real-world dynamic IoV. Onsite method outperforms ExSearch by using a less complex approach, but its ART is still higher than others because it seeks to find all feasible mappings and chooses the best one. Although Random, TimePref, and DegreePref methods have lower ART values and sometimes perform better than SWATS, we will later show that they incur large costs. Our SWATS has commendable performance on ART because the prior obtained mapping $\alpha$ can be used in most practical scheduling events, making it suitable for dynamic MVCs. In Fig. 5, for tadpole task (type 3) with 15 vehicles, the value of ART can exceed $10^3$ seconds when using ExSearch, which is impractical. 

%f5
% Figure environment removed

%f6
% Figure environment removed

%%f6
%% Figure environment removed
%
%%f7
%% Figure environment removed

%The performance on the average value of cost function (AVCF) is demonstrate by Figs. 5-7, upon considering 100 simulations and different settings of vehicular cloud as well as task types. Note that the unacceptable time overhead of ExhaustiveS has stopped us to show its performance in Figs. 5-7. As can be seen in these figures, our proposed DISCO greatly outperforms the random- and greedy-based methods on AVCF thanks to the optimality of mappings $\alpha$ and $\beta$. Moreover, DISCO obtains a similar performance on AVCF with the Onsite method, while outperforming it in Fig. 4 on the value of ART. All in all, our proposed DISCO for graph task scheduling achieves commendable performance on significant evaluation indicators in comparison with existing methods, which is worthy of reference for future resource sharing markets. 
We next depict the performance of SWATS, in terms of the average value of cost function (AVCF) and compare that to other methods in Fig. 6 for different task topologies. The results are consistent and reveal that SWATS greatly outperforms random and greedy-based methods due to the complementary mappings $\alpha$ and $\beta$. Onsite and ExSearch methods achieve the same value of AVCF since they both obtain all the possible mappings, while ExSearch suffers from unacceptable time overhead as shown in Fig. 5. Interestingly, SWATS performs similarly to the Onsite method on AVCF, while outperforming it on the value of ART (see Fig. 5), which reveals the importance of exploiting mapping $\alpha$ to achieve a high cost efficiency.
%We did not include the ExhaustiveS method in these figures due to its unacceptable time overhead.

Overall, SWATS achieves a commendable performance on key evaluation indicators in comparison with existing methods, making the stage-wise decision-making paradigm a worthy reference for resource provisioning in future MVCs. 

\section{Future Directions}
%\noindent
%Our proposed DISCO for graph task scheduling introduces a series of open research directions, which we discuss below:  
In the following, we discuss several avenues of research.
%hat are motivated by SWATS.
\begin{itemize}[leftmargin=4mm] 
 \item \textit{Incentive design:} 
% One of the primary concerns in the implementation of resource sharing involves the transmission/storage/processing of large volume of data, which can impose additional costs on vehicles. A practical way is to offer monetary incentives to integrate distributed on-board resources, which calls for establishing a bridge between the IoV network and resource trading market. Several fundamental problems in such a market can be considered, e.g., service price determination, service contract negotiation. 
Considering resource provisioning over vehicles, a major issue is engaging them in the transmission, storage, and processing of data. One potential solution involves providing financial incentives to vehicles in exchange for their services. However, developing effective incentive mechanisms will require establishing a connection between resource provisioning methods, trading markets, and graph matching. Additionally, it will impose+ tackling nuanced challenges, e.g., establishing service prices to ensure maintaining the inherent task structure.

\item \textit{Overbooking-promoted data duplication:} 
%Due to the random and unpredictable nature of mobile networks, e.g., a promissory vehicle may be absent from the considered region, or the communications among vehicles may risk outage events during the current task scheduling procedure owing to vehicular mobility, which thus lead to fluctuations in resource supply. To this end, an interesting concept called ``overbooking'' is introduced, allowing each task (or subtask) to be mapped to more vehicles, in case that some of them may fail to offer services. In this case, to ensure the performance of computation-intensive tasks, the data of each subtask can be transmitted to multiple vehicles for processing, which, however, may also impose excessive overhead, e.g., extra delay and energy consumption as we as possible abuse of limited resources under a large overbooking rate. Thus, how to optimize the overbooking rate remains an urgent and critical topic. 
Mobile vehicular networks exhibit unpredictability of fluctuations in resource supply and potential V2V outages. An interesting concept of ``overbooking" can be introduced, wherein each subtask can be assigned to more than one vehicle to account for potential service failures. However, overbooking may introduce excessive overhead, encompassing delays, increased energy consumption, and the risk of resource abuse. Consequently, optimizing the overbooking procedure can be considered as a pivotal focus to ensure high-performance task scheduling.

\item \textit{Accurate and timely environment assessment:} 
%Generally, multiple uncertainties can exist in a wireless mobile network environment, e.g., dynamic resource supply/demand, varying channel quality, possible communication outage, uncertain vehicular willingness of offering services. However, inaccurate statistics and prediction can leave severe impacts on task scheduling performance. Thus, the perception and sharing of information of uncertain factors call for smart algorithm design to better understand the dynamic environment under a timely manner. 
In wireless networks, uncertainties such as resource availability, communication disruptions, and the willingness of service providers are common occurrences. Inaccurate predictions of these uncertainties can have adverse effects on task scheduling performance. Hence, it is crucial to employ artificial intelligent (AI)-driven algorithms, such as time series prediction and recurrent neural networks, capable of swiftly and accurately acquiring information about the uncertainties in the environment.

\item \textit{Intelligent risk prediction, quantization, and management:} 
%The proposed Plan A represents a coexistence between risks and opportunities, where proper estimation and control of possible risks can facilitate a stable and healthy resource trading market. Thus, it is critical to design risk prediction, quantization, and management mechanisms with intelligent features in the future, rather than those probability-based models considered in this article.
Our Plan A aims to uphold the promptness and cost-effectiveness of task scheduling by effectively balancing risks and opportunities. This calls for developing intelligent mechanisms for predicting, quantifying, and managing risks. Departing from traditional probabilistic models, the proposed mechanisms can leverage learning-based methods to model potential risks. Thus, future endeavors could explore establishing a connection between the domains of explainable artificial intelligence (XAI) and risk analysis to enhance the comprehensibility and interpretability of risk assessment.

\item \textit{Competition and cooperation among vehicles:} 
%Real-world vehicles often exhibit a blend of competition and cooperation. A compelling research avenue involves exploring mechanisms fostering cooperation and revenue sharing among vehicles, while considering the structural intricacies inherent in graph tasks. This exploration may open a new research domain centered on executing graph tasks via cooperative games. Additionally, an alternative perspective on vehicle competition lies in the realm of auction theory. Both approaches must intricately incorporate the topology of vehicles and tasks.
As computing service providers, vehicles can demonstrate a dynamic interplay of competition and cooperation. An intriguing research direction entails delving into mechanisms that promote constructive cooperation and revenue sharing among vehicles, all while taking into account the inherent structural complexities of graph tasks. This exploration carves out a novel research domain on executing graph tasks through cooperative games. Moreover, an alternative perspective on vehicle competition can be explored within the realm of auction theory. 

\item \textit{Addressing the curse of dimensionality:}
%Also, competition among vehicles can be considered as an auction market.
%In addition, we are interested in exploring the possible cooperation among subtasks, e.g., a large-scale graph task can be remodeled as a small-scale graph since some subtasks with similar properties can form clusters. In this case, the size of the considered optimization problem can be reduced to accelerate solution designs. 
To reduce the problem scale, one way is to look for similarities between subtasks within a large graph task. For instance, we can group subtasks with similar attributes to form clusters. Another option is to group vehicles with similar resources and contact durations, treating them as a single unit. This can accelerate the optimization process by reducing the problem size.
\end{itemize}

\section{Conclusion}
%This article focuses on the problem of task scheduling over mobile vehicles, where both tasks and vehicular cloud are modeled as undirected weighted graphs. Motivated by the drawbacks of onsite task scheduling such as excessive overhead on decision-making, we propose a novel paradigm for graph task scheduling over vehicular cloud called DISCO, which consists of Plan A and Plan B. The goal of Plan A is to obtain the optimal mapping $\alpha$ ahead of future practical task scheduling process by analyzing historical statistics of uncertain factors such as mobility of vehicles and varying channel quality. Under given $\alpha$, during each practical task scheduling event, Plan B is regarded as a backup which looks for a commendable mapping $\beta$ under a timely manner, when $\alpha$ fails to work in the current network. The timeline and key concerns of DISCO are analyzed in detail. In addition, a case study with comprehensive simulations is conducted to illustrate the commendable performance of DISCO. Interesting research directions are introduced to offer references to future resource sharing in IoV. 
We studied the allocation of computation-intensive tasks, modelled as undirected weighted graphs, over mobile vehicles within mobile vehicular clouds. To tackle the challenges of on-site scheduling in terms of excessive decision-making overhead, we introduced a novel stage-wise decision-making mechanism that encompasses two complementary plans for task scheduling: Plan A and Plan B. The objective of Plan A is to analyze historical statistics of uncertain factors like vehicle mobility and channel quality variations to obtain the best mapping $\alpha$ between subtasks and vehicles ahead of future task scheduling events. During each practical task scheduling event, Plan B is considered as a backup option to quickly identify a suitable mapping $\beta$ in case $\alpha$ fails. We investigated the procedure of our proposed paradigm and primary factors that contribute to its success, followed by a case study that includes simulations showcasing the performance evaluation in terms of cost and time efficiency. We also highlighted a set of research directions to further enhance resource provisioning over mobile vehicular clouds. 

%\section*{Acknowledgement}
%This work was supported in part by the National Natural Science Foundation of China under Grant no. 62271424, the Natural Science Foundation of Xiamen City (Grant nos. 3502Z20227002, 3502Z20227007), the Fundamental Research Funds for the Central Universities under Grant no. 20720230035, the Basic and Applied Basic Research Foundation of Guangdong Province under Grant no. 2022A1515110042. 

\ifCLASSOPTIONcaptionsoff
 \newpage
\fi

%\begin{thebibliography}{15}
%%K. Wang, L. Wang, C. Pan and H. Ren, “Deep Reinforcement Learning-Based Resource Management for Flexible Mobile Edge Computing: Architectures, Applications, and Research Issues,” \textit{IEEE Veh. Technol. Mag.}, vol. 17, no. 2, pp. 85--93, 2022.
%
%%\bibitem{1} H. Li, K. Ota, and M. Dong, “Learning IoV in 6G: Intelligent Edge Computing for Internet of Vehicles in 6G Wireless Communications,” \textit{IEEE Wireless Commun.}, pp.1-1, 2023.
%
%%\bibitem{1} S. Guan, and A. Boukerche, "Intelligent Edge-Based Service Provisioning Using Smart Cloudlets, Fog and Mobile Edges," \textit{IEEE Netw.}, vol. 36, no. 2, pp. 139-145, 2022.
%%\bibitem{1} S. M. Alamouti, F. Arjomandi and M. Burger, "Hybrid Edge Cloud: A Pragmatic Approach for Decentralized Cloud Computing," \textit{IEEE Commun. Mag.}, vol. 60, no. 9, pp. 16-29, 2022.
%
%\bibitem{1}G. Panek, I. Fajjari, H. Tarasiuk, A. Bousselmi, and T. Toukabri, "Application Relocation in an Edge-Enabled 5G System: Use Cases, Architecture, and Challenges,"~\textit{IEEE Commun. Mag.}, vol. 60, no. 8, pp. 28-34, 2022.
%
%%\bibitem{2} B. Gu and Z. Zhou, “Task Offloading in Vehicular Mobile Edge Computing: A Matching-Theoretic Framework,” \textit{IEEE Veh. Technol. Mag.}, vol. 14, no. 3, pp. 100--106, 2019.
%%\bibitem{2} Y. Deng, X. Chen, G. Zhu, Y. Fang, Z. Chen and X. Deng, “Actions at the Edge: Jointly Optimizing the Resources in Multi-Access Edge Computing,” \textit{IEEE Wireless Commun.}, vol. 29, no. 2, pp. 192-198, 2022.
%\bibitem{2} M. Liwang, Z. Gao, S. Hosseinalipour, Y. Su, X. Wang, and H. Dai, “Graph-Represented Computation-Intensive Task Scheduling Over Air-Ground Integrated Vehicular Networks,” \textit{IEEE Trans. Services Comput.}, vol. 16, no. 5, pp. 3397-3411, 2023.
%
%%\bibitem{3} G. Cui, Q. He, F. Chen, Y. Zhang, H. Jin and Y. Yang, “Interference-Aware Game-Theoretic Device Allocation for Mobile Edge Computing,”\textit{IEEE Trans. Mobile Comput.}, vol. 21, no. 11, pp. 4001-4012, 2022.
%\bibitem{3}W. Fan, Y. Su, J. Liu, S. Li, W. Huang, F. Wu, Y. Liu,   “Joint Task Offloading and Resource Allocation for Vehicular Edge Computing Based on V2I and V2V Modes,” \textit{IEEE Trans. Intell. Transp. Syst.}, vol. 24, no. 4, pp. 4277-4292, 2023.
%
%%\bibitem{4} L. Liu, J. Feng, X. Mu, Q. Pei, D. Lan and M. Xiao, “Asynchronous Deep Reinforcement Learning for Collaborative Task Computing and On-Demand Resource Allocation in Vehicular Edge Computing,” \textit{IEEE Trans. Intell. Transp. Syst.}, pp. 1-1, 2023.
%
%\bibitem{4} Y. Wu, W. Zhang, N. Guan and Y. Ma, “TDTA: Topology-Based Real-Time DAG Task Allocation on Identical Multiprocessor Platforms,” ~\textit{IEEE Trans. Parallel  Distrib. Syst.}, vol. 34, no. 11, pp. 2895-2909, 2023.
%
%\bibitem{5}  J. Liu, N. Liu, L. Liu, S. Li, H. Zhu and P. Zhang, “A Proactive Stable Scheme for Vehicular Collaborative Edge Computing,” \textit{IEEE Trans. Veh. Technol.}, pp. 1--1, 2023.
%
%\bibitem{6} Z. Gao, M. Liwang, S. Hosseinalipour, H. Dai and X. Wang, “A Truthful Auction for Graph Job Allocation in Vehicular Cloud-Assisted Networks,” \textit{IEEE Trans. Mobile Comput.}, vol. 21, no. 10, pp. 3455--3469, 2022.
%
%\bibitem{7} H. Zhu, M. Li, L. Fu, G. Xue, Y. Zhu and L. M. Ni, “Impact of Traffic Influxes: Revealing Exponential Intercontact Time in Urban VANETs,”~\textit{IEEE Trans. Parallel  Distrib. Syst.}, vol. 22, no. 8, pp. 1258-1266, 2011.
%
%\bibitem{8} S. Hosseinalipour, A. Nayak, and H. Dai, “Power-aware Allocation of Graph Jobs in Geo-Distributed Cloud Networks,” \textit{IEEE Trans. Parallel Distrib. Syst.}, vol. 31, no. 4, pp. 749--765, Apr. 2020.
%
%\bibitem{9} H. Liao, X. Li, D. Guo, W. Kang, and J. Li, “Dependency-Aware Application Assigning and Scheduling in Edge Computing,”~\textit{IEEE Internet of Things J.}, vol. 9, no. 6, pp. 4451-4463, 2022.
%
%\bibitem{10} S. Wi, S. Woo, J. J. Whang, and S. Son, “HiddenCPG: Large-Scale Vulnerable Clone Detection Using Subgraph Isomorphism of Code Property Graphs,”\textit{Proc. ACM Web Conf.}, Lyon, France, 2022, pp. 755--766.
%
%\bibitem{11}W. Feng, N. Zhang, S. Li, S. Lin, R. Ning, S. Yang, T. Gao, “Latency Minimization of Reverse Offloading in Vehicular Edge Computing,”\textit{IEEE Trans. Veh. Technol.}, vol. 71, no. 5, pp. 5343-5357, 2022.
%
%\bibitem{12}  S. Sheng, R. Chen, P. Chen, X. Wang, and L. Wu, “Futures-based Resource Trading and Fair Pricing in Real-Time IoT Networks,” \textit{IEEE Wireless Commun. Lett.}, vol. 9, no. 1, pp. 125--128, 2020.
%
%\bibitem{13} M. Abulaish, Z. A. Ansari, and Jahiruddin, “Subiso: A Scalable and Novel Approach for Subgraph Isomorphism Search in Large Graph,” \textit{IEEE Int. Conf. Commun. Syst. Netw.}, Bengaluru, India, pp. 102--109, 2019.
%
%\bibitem{14} Y. Liu, D. Li, R. Li, Z. Zhao, Y. Zhu, and H. Zhang, “Secure and Efficient Stigmergy-Empowered Blockchain Framework for Heterogeneous Collaborative Services in the Internet of Vehicles,”~\textit{IEEE Commun. Mag.}, vol. 61, no. 9, pp. 186-192, 2023.
%
%\bibitem{15} S. Luo, X. Chen, Q. Wu, Z. Zhou, and S. Yu, “HFEL: Joint Edge Association and Resource Allocation for Cost-Efficient Hierarchical Federated Edge Learning,”~\textit{IEEE Trans. Wireless Commun.}, vol. 19, no. 10, pp. 6535-6548, 2020.
%\end{thebibliography}


\begin{thebibliography}{15}
%K. Wang, L. Wang, C. Pan and H. Ren, “Deep Reinforcement Learning-Based Resource Management for Flexible Mobile Edge Computing: Architectures, Applications, and Research Issues,” \textit{IEEE Veh. Technol. Mag.}, vol. 17, no. 2, pp. 85--93, 2022.

%\bibitem{1} H. Li, K. Ota, and M. Dong, “Learning IoV in 6G: Intelligent Edge Computing for Internet of Vehicles in 6G Wireless Communications,” \textit{IEEE Wireless Commun.}, pp.1-1, 2023.

%\bibitem{1} S. Guan, and A. Boukerche, "Intelligent Edge-Based Service Provisioning Using Smart Cloudlets, Fog and Mobile Edges," \textit{IEEE Netw.}, vol. 36, no. 2, pp. 139-145, 2022.
%\bibitem{1} S. M. Alamouti, F. Arjomandi and M. Burger, "Hybrid Edge Cloud: A Pragmatic Approach for Decentralized Cloud Computing," \textit{IEEE Commun. Mag.}, vol. 60, no. 9, pp. 16-29, 2022.

\bibitem{1}G. Panek, et al., “Application Relocation in an Edge-Enabled 5G System: Use Cases, Architecture, and Challenges,”~\textit{IEEE Commun. Mag.}, vol. 60, no. 8, pp. 28-34, 2022.

%\bibitem{2} B. Gu and Z. Zhou, “Task Offloading in Vehicular Mobile Edge Computing: A Matching-Theoretic Framework,” \textit{IEEE Veh. Technol. Mag.}, vol. 14, no. 3, pp. 100--106, 2019.
%\bibitem{2} Y. Deng, X. Chen, G. Zhu, Y. Fang, Z. Chen and X. Deng, “Actions at the Edge: Jointly Optimizing the Resources in Multi-Access Edge Computing,” \textit{IEEE Wireless Commun.}, vol. 29, no. 2, pp. 192-198, 2022.
\bibitem{2} M. Liwang, et al., “Graph-Represented Computation-Intensive Task Scheduling Over Air-Ground Integrated Vehicular Networks,” \textit{IEEE Trans. Services Comput.}, vol. 16, no. 5, pp. 3397-3411, 2023.

%\bibitem{3} G. Cui, Q. He, F. Chen, Y. Zhang, H. Jin and Y. Yang, “Interference-Aware Game-Theoretic Device Allocation for Mobile Edge Computing,”\textit{IEEE Trans. Mobile Comput.}, vol. 21, no. 11, pp. 4001-4012, 2022.
\bibitem{3}W. Fan, et al., “Joint Task Offloading and Resource Allocation for Vehicular Edge Computing Based on V2I and V2V Modes,” \textit{IEEE Trans. Intell. Transp. Syst.}, vol. 24, no. 4, pp. 4277-4292, 2023.

%\bibitem{4} L. Liu, J. Feng, X. Mu, Q. Pei, D. Lan and M. Xiao, “Asynchronous Deep Reinforcement Learning for Collaborative Task Computing and On-Demand Resource Allocation in Vehicular Edge Computing,” \textit{IEEE Trans. Intell. Transp. Syst.}, pp. 1-1, 2023.

\bibitem{4} Y. Wu, et al., “TDTA: Topology-Based Real-Time DAG Task Allocation on Identical Multiprocessor Platforms,” ~\textit{IEEE Trans. Parallel  Distrib. Syst.}, vol. 34, no. 11, pp. 2895-2909, 2023.

\bibitem{5}  J. Liu, et al., “A Proactive Stable Scheme for Vehicular Collaborative Edge Computing,” \textit{IEEE Trans. Veh. Technol.}, pp. 1--1, 2023.

\bibitem{6} Z. Gao, et al., “A Truthful Auction for Graph Job Allocation in Vehicular Cloud-Assisted Networks,” \textit{IEEE Trans. Mobile Comput.}, vol. 21, no. 10, pp. 3455--3469, 2022.

\bibitem{7} H. Zhu, et al., “Impact of Traffic Influxes: Revealing Exponential Intercontact Time in Urban VANETs,”~\textit{IEEE Trans. Parallel  Distrib. Syst.}, vol. 22, no. 8, pp. 1258-1266, 2011.

\bibitem{8} S. Hosseinalipour, et al., “Power-aware Allocation of Graph Jobs in Geo-Distributed Cloud Networks,” \textit{IEEE Trans. Parallel Distrib. Syst.}, vol. 31, no. 4, pp. 749--765, Apr. 2020.

\bibitem{9} H. Liao, et al., “Dependency-Aware Application Assigning and Scheduling in Edge Computing,”~\textit{IEEE Internet of Things J.}, vol. 9, no. 6, pp. 4451-4463, 2022.

\bibitem{10} S. Wi, S. Woo, J. J. Whang, and S. Son, “HiddenCPG: Large-Scale Vulnerable Clone Detection Using Subgraph Isomorphism of Code Property Graphs,”\textit{Proc. ACM Web Conf.}, Lyon, France, 2022, pp. 755--766.

\bibitem{11}W. Feng, et al., “Latency Minimization of Reverse Offloading in Vehicular Edge Computing,”\textit{IEEE Trans. Veh. Technol.}, vol. 71, no. 5, pp. 5343-5357, 2022.

\bibitem{12}  S. Sheng, et al., “Futures-based Resource Trading and Fair Pricing in Real-Time IoT Networks,” \textit{IEEE Wireless Commun. Lett.}, vol. 9, no. 1, pp. 125--128, 2020.

\bibitem{13} M. Abulaish, et al., “Subiso: A Scalable and Novel Approach for Subgraph Isomorphism Search in Large Graph,” \textit{IEEE Int. Conf. Commun. Syst. Netw.}, Bengaluru, India, pp. 102--109, 2019.

\bibitem{14} Y. Liu, et al., “Secure and Efficient Stigmergy-Empowered Blockchain Framework for Heterogeneous Collaborative Services in the Internet of Vehicles,”~\textit{IEEE Commun. Mag.}, vol. 61, no. 9, pp. 186-192, 2023.

\bibitem{15} S. Luo, et al., “HFEL: Joint Edge Association and Resource Allocation for Cost-Efficient Hierarchical Federated Edge Learning,”~\textit{IEEE Trans. Wireless Commun.}, vol. 19, no. 10, pp. 6535-6548, 2020.
\end{thebibliography}



\section*{Biography}
\vspace{-1cm}
\begin{IEEEbiographynophoto}{Minghui Liwang} is an assistant professor in School of Informatics, Xiamen University, China.
\end{IEEEbiographynophoto}
\vspace{-1cm}
\begin{IEEEbiographynophoto}{Bingshuo Guo} a Master student in Communication Engineering, Xiamen University, 
China. 
%(guobingshuo@stu.xmu.edu.cn)
\end{IEEEbiographynophoto}
\vspace{-1cm}
\begin{IEEEbiographynophoto}{Zhanxi Ma} is a Ph.D student in School of Electronic and Engineering, Nanjing University, China.
\end{IEEEbiographynophoto}
\vspace{-1cm}
\begin{IEEEbiographynophoto}{Yuhan Su} is an assistant professor in School of Electronic Science and Engineering, Xiamen University, China. 
%(ysu@xmu.edu.cn) 
\end{IEEEbiographynophoto}
\vspace{-1cm}
\begin{IEEEbiographynophoto}{Jian Jin} is with Research Institute of Industrial Internet of Things, China Academy of Information and Communications Technology, China.
\end{IEEEbiographynophoto}
%(jin.jian@caict.ac.cn)
\vspace{-1cm}
\begin{IEEEbiographynophoto}{Seyyedali Hosseinalipour} is an assistant professor in Department of Electrical Engineering at University at Buffalo, SUNY.
\end{IEEEbiographynophoto}
%(alipour@buffalo.edu)
\vspace{-1cm}
\begin{IEEEbiographynophoto}{Xianbin Wang} is a professor in Department of Electrical and Computer Engineering, Western University, Canada.
\end{IEEEbiographynophoto}
%(xianbin.wang@uwo.ca)
\vspace{-1cm}
\begin{IEEEbiographynophoto}{Huaiyu Dai} is a
professor in Department of Electrical and Computer Engineering, NC State University, USA.
\end{IEEEbiographynophoto}
\vfill
\end{document}

%~~~~~~~~~~~~~~~~~~~~~~~~~~~~~~~~~~~~~~~~~~~~~~Biography
%\begin{IEEEbiography}[{% Figure removed}]
%{Minghui Liwang}[M'19] is currently an assistant professor in School of Informatics, Xiamen University, China. Her research interests include wireless communication systems, Internet of Things, cloud/edge/service computing, as well as economic models and applications in wireless communication networks.
%\end{IEEEbiography}
%
%%Bingshuo Guo
%\begin{IEEEbiography}[{% Figure removed}]
%{Bingshuo Guo} received his B.S degree in communication engineering from Yanshan University, China, in 2022. He is currently working toward the M.S. degree in School of Informatics, Xiamen University, China. His research interests include mobile crowdsensing networks, graph theory and cloud/edge/service computing.
%\end{IEEEbiography}
%%\space{-0.1cm}
%
%%Zhanxi Ma
%\begin{IEEEbiography}[{% Figure removed}]
%{Zhanxi Ma} recived his B.S. degree in School of
%Informatics, Xiamen University, China, in 2023. He
%is currently working toward the Ph.D. degree in School of Electronic and Engineering, 
%Nanjing University, China. His research interests
%include space-terrestrial decoupling, resource
%scheduling in IoV, and edge computing. 
%\end{IEEEbiography}
%
%%Yuhan Su
%\begin{IEEEbiography}[{% Figure removed}]
%{Yuhan Su} is currently an assistant professor with School of Electronic Science and Engineering, Xiamen University, Xiamen, China. His research interests include wireless communications, UAV networks, and machine learning.
%\end{IEEEbiography}
%
%%Seyyedali Hosseinalipour 
%\begin{IEEEbiography}[{% Figure removed}]
%{Seyyedali Hosseinalipour} [M’20] is currently an assistant professor with the Department of Electrical
%Engineering, University at Buffalo, SUNY, Buffalo, NY, USA. His research interests include
%6G, machine learning, federated learning, fog
%and edge computing, and network optimization.
%\end{IEEEbiography}
%
%%Xianbin Wang
%\begin{IEEEbiography}[{% Figure removed}]
%{Xianbin Wang}[S’98-M’99-SM’06-F’17] is a Professor and Tier-1 Canada Research Chair at Western University, Canada. 
%%He received his Ph.D. degree in electrical and computer engineering from the National University of Singapore in 2001. Prior to joining Western, he was with Communications Research Centre Canada (CRC) as a Research Scientist/Senior Research Scientist between July 2002 and Dec. 2007. From Jan. 2001 to July 2002, he was a system designer at STMicroelectronics. 
%His current research interests include 5G/6G technologies, Internet-of-Things, communications security, machine learning and intelligent communications. 
%%Dr. Wang has over 450 highly cited journal and conference papers, in addition to 30 granted and pending patents and several standard contributions. Dr. Wang is a Fellow of Canadian Academy of Engineering, a Fellow of Engineering Institute of Canada, a Fellow of IEEE and an IEEE Distinguished Lecturer. He has received many awards and recognitions, including Canada Research Chair, CRC President’s Excellence Award, Canadian Federal Government Public Service Award, Ontario Early Researcher Award and six IEEE Best Paper Awards. He currently serves/has served as an Editor-in-Chief, Associate Editor-in-Chief, Editor/Associate Editor for over 10 journals. He was involved in many IEEE conferences including GLOBECOM, ICC, VTC, PIMRC, WCNC and CWIT, in different roles such as symposium chair, tutorial instructor, track chair, session chair, TPC co-chair and keynote speaker. He has been nominated as an IEEE Distinguished Lecturer several times during the last ten years. Dr. Wang is currently serving as the Chair of IEEE London Section and the Chair of ComSoc Signal Processing and Computing for Communications (SPCC) Technical Committee.
%\end{IEEEbiography}
%
%%Huaiyu Dai
%\begin{IEEEbiography}[{% Figure removed}]{Huaiyu Dai}[F’17] He is currently
%a Professor of Electrical and Computer Engineering with NC State University, Raleigh. His current research
%focuses on networked information processing and crosslayer design
%in wireless networks, cognitive radio networks, network security, and
%associated information-theoretic and computation-theoretic analysis. 
%\end{IEEEbiography}
%~~~~~~~~~~~~~~~~~~~~~~~~~~~~~~~~~~~~~~~~~~~~~~Biography







