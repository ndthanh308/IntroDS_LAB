
\documentclass[10pt,journal,compsoc]{IEEEtran}
%\documentclass[journal,10pt]{IEEEtran}

\usepackage{setspace} 
\usepackage{latexsym, graphicx, textcomp}
\usepackage{epsfig,upgreek}
\usepackage{amsfonts, amssymb, amsthm,makecell,enumerate}
\usepackage{float,color,dblfloatfix}
\usepackage{fancyhdr}
\usepackage{amsmath}
\usepackage{amsfonts}
\usepackage{mathrsfs}
\usepackage{pifont}
\usepackage{amssymb}
\usepackage{graphicx}
\usepackage{multicol}
\usepackage{cases}
\usepackage{epstopdf}
\usepackage{ltxtable, filecontents}
\usepackage{latexsym}
\usepackage{booktabs}
\usepackage{multirow}
\usepackage{booktabs}
\usepackage{threeparttable}
\usepackage{supertabular}
\usepackage[figuresright]{rotating}
\usepackage[tight,small]{subfigure}
\usepackage{tabularx}
\usepackage{cite}
\usepackage{dcolumn}
\usepackage{algorithmicx}
\usepackage[linesnumbered,ruled,vlined]{algorithm2e}
\usepackage{algpseudocode}
 \usepackage{ragged2e}
 \usepackage{cuted}
\usepackage{graphicx} 
\usepackage{diagbox}
\usepackage{array}
\usepackage{amsfonts}
\usepackage{bbm}
\usepackage{enumitem}
\usepackage{hyperref}
\usepackage{bm}
%\usepackage{lettrine}
\usepackage{hyperref}
%\usepackage[colorlinks=false,linkcolor=black,anchorcolor=black,citecolor=black,CJKbookmarks=false]{hyperref} 

%\pdfstringdefDisableCommands{\let\bm=\relax}
\graphicspath{{figures/}}
%\hypersetup{CJKbookmarks=true}
\newtheorem{myDef}{Definition}
\newtheorem{myLam}{Lemma}
\newtheorem{myPro}{Proof}
\newtheorem{Defn}{Definition}
\newtheorem{lem}{Lemma}
\newtheorem{col}{Corollary}
\newtheorem{Prop}{Proposition}
\newtheorem{thm}{Theorem}
\newtheorem{rek}{Remark}
\linespread{0.999}
\begin{document}

\title{DISCO: Double-Plan-Promoted Isomorphic Subgraph Search and Optimization for Graph Task Scheduling over Vehicular Clouds}
%\title{Achieving Low Latency and High Reliability in Scheduling of Graph-Structured Tasks over Vehicular Clouds}
\title{DISCO: Achieving Low Latency and High Reliability in Scheduling of Graph-Structured Tasks over Mobile Vehicular Cloud}

\author{Minghui Liwang, \IEEEmembership{Member}, \IEEEmembership{IEEE}, Bingshuo Guo, Zhanxi Ma, Yuhan Su, Jian Jin, 
\\Seyyedali Hosseinalipour,~\IEEEmembership{Member}, \IEEEmembership{IEEE}, Xianbin Wang,~\IEEEmembership{Fellow}, \IEEEmembership{IEEE}, 
Huaiyu Dai,~\IEEEmembership{Fellow}, \IEEEmembership{IEEE}

%\thanks{Minghui Liwang (minghuilw@xmu.edu.cn), Bingshuo Guo (guobingshuo@stu.xmu.edu.cn) are with the School of Informatics, Xiamen University, China. Zhanxi Ma is with the School of Electronic and Engineering, 
%Nanjing University, China. 
%Yuhan Su (ysu@xmu.edu.cn) is with School of Electronic Science and Engineering, Xiamen University, China. Jian Jin (jin.jian@caict.ac.cn) is with the Research Institute of Industrial Internet of Things, China Academy of Information and Communications Technology, Beijing 100095, China. 
%Seyyedali Hosseinalipour (alipour@buffalo.edu) is with Department of Electrical Engineering, University at Buffalo-SUNY, USA. Xianbin Wang (xianbin.wang@uwo.ca) is with the Department of Electrical and Computer Engineering, Western University, Canada. Huaiyu Dai (hdai@ncsu.edu) is with Department of Electrical and Computer Engineering, North Carolina State University, USA.

\thanks{Minghui Liwang (minghuilw@xmu.edu.cn), Bingshuo Guo (guobingshuo@stu.xmu.edu.cn) and Yuhan Su (ysu@xmu.edu.cn) are with Xiamen University, China. Zhanxi Ma is with Nanjing University, China. Jian Jin (jin.jian@caict.ac.cn) is with the Research Institute of Industrial Internet of Things, China Academy of Information and Communications Technology, China. 
Seyyedali Hosseinalipour (alipour@buffalo.edu) is with University at Buffalo-SUNY, USA. Xianbin Wang (xianbin.wang@uwo.ca) is with Western University, Canada. Huaiyu Dai (hdai@ncsu.edu) is with North Carolina State University, USA.

%Corresponding author: Yuhan Su, Jian Jin


}}

\IEEEtitleabstractindextext{
\begin{abstract}
\justifying
% \noindent
%Data processing over dynamic and distributed vehicles calls for innovative resource provisioning techniques to support real-time, cost-effective, and reliable computing services. This article investigates how computation-intensive tasks modeled by undirected weighted graphs can be efficiently scheduled over a vehicular cloud formed by multiple mobile vehicles, for which a novel \underline{d}ouble-plan-promoted \underline{i}somorphic \underline{s}ubgraph sear\underline{c}h and \underline{o}ptimization (DISCO) paradigm is introduced. We consider two complementary plans: \textit{i)} Plan A achieves an optimal mapping (called $\alpha$) between graph tasks and vehicular cloud in advance to future resource requirements by analyzing historical statistics of uncertain factors; and \textit{ii)} Plan B represents a backup plan which seeks for a feasible mapping (called $\beta$) when mapping $\alpha$ fails to support the completion of tasks during each practical task scheduling process, due to random and dynamic nature of the network. The timeline and key concerns of DISCO are analyzed, while a case study with comprehensive comparisons is conducted to illustrate the commendable performance of DISCO. A series of interesting research directions are carried out to offer references to future resource sharing in the internet of vehicles. 
To effectively process data across a fleet of dynamic and distributed vehicles, it is crucial to implement resource provisioning techniques that provide reliable, cost-effective, and real-time computing services. This article explores resource provisioning for computation-intensive tasks over mobile vehicular clouds (MVCs). We use undirected weighted graphs (UWGs) to model both the execution of tasks and communication patterns among vehicles in a MVC. We then study low-latency and reliable scheduling of UWG tasks through a novel methodology named \underline{d}ouble-plan-promoted \underline{i}somorphic \underline{s}ubgraph sear\underline{c}h and \underline{o}ptimization (DISCO). In DISCO, two complementary plans are envisioned to ensure effective task completion: Plan A and Plan B. Plan A analyzes the past data to create an optimal mapping ($\alpha$) between tasks and the MVC in advance to the practical task scheduling. Plan B serves as a dependable backup, designed to find a feasible mapping ($\beta$) in case $\alpha$ fails during task scheduling due to unpredictable nature of the network. We delve into DISCO's procedure and key factors that contribute to its success. Additionally, we provide a case study to demonstrate DISCO's commendable performance in regards to time efficiency and overhead. We further discuss a series of open directions for future research.
\end{abstract}

\begin{IEEEkeywords}
Task scheduling, vehicular cloud, undirected graph task, subgraph isomorphism, time effectiveness
\end{IEEEkeywords}}

\maketitle

\IEEEpeerreviewmaketitle

%Sec 1
\section{Introduction}
%\IEEEPARstart{T}{he} innovative communication/computing technologies and the acceleration of smart vehicles facilitate the explosive growth of Internet of Vehicles (IoV) as well as computation- and data-intensive applications (also called tasks) such as autonomous driving and in-vehicle entertainment. Most of those promising applications are machine learning (ML)-based and require high computing capacity for data processing, which impose great difficulties to a single vehicle~\cite{1,2}. Although cloud computing offers a good solution, transmitting massive volume of data can lead to heavy burdens on the backhaul/backbone networks and thus unacceptable delay. To respond to this challenge, mobile edge computing (MEC) technique~\cite{3} has been emerged as a powerful solution that brings computing/communication/storage resources close to end-devices. %The original version

%\IEEEPARstart{T}{he} rapid growth of the Internet of Vehicles (IoV) and vehicular computation-intensive applications, e.g., autonomous driving and in-vehicle entertainment, has been made possible by innovative communication and computing technologies, as well as the increasing prevalence of smart vehicles. Many of these applications/tasks are machine learning (ML)-driven and require significant computing power for data processing, making them difficult for a single vehicle to handle~\cite{1,2}. While cloud computing can offer a solution, transmitting large amounts of data from vehicles to cloud servers can lead to congestion on the backhaul network and significant delays. To overcome this challenge, mobile edge computing (MEC)~\cite{3} has emerged that brings computing, communication, and storage resources to the network edge closer to vehicles. %Ali's version

\IEEEPARstart{T}{he} recent emergence and rapid growth of the Internet of Vehicles (IoV) and vehicular computation-intensive applications, e.g., autonomous driving and in-vehicle entertainment, have been driven by the advancement and integration of communication and computing technologies, as well as the increasing prevalence of smart vehicles. Many of these vehicular applications/tasks are based on machine learning (ML), requiring significant amount of computing power for real-time data processing, which is often beyond a single vehicle's resource capacity~\cite{1,2}. While cloud computing provides a solution, transmitting large amounts of data from vehicles to cloud servers can lead to significant delays and backhaul network congestion. To overcome this challenge, mobile edge computing (MEC)~\cite{3} has emerged as an alternative approach that brings closer computing, communication, and storage resources from the network edge to vehicles.

%Leveraging the fast development of MEC and the functionality of smart vehicles embedded with powerful sensors and processors, collaborative vehicular computing is expected to build a universal computing platform upon offering efficient on-board storage/computation~\cite{4,5}, while such a vehicular cloud formed by mobile vehicles (servers) allows tasks to be scheduled and processed parallelly on different servers. 

%Combine the rapid advancement of MEC technology and the integration of smart vehicles equipped with high-powered sensors and processors into its architecture, collaborative vehicular computing is poised to create a universal computing platform that provides efficient on-board storage and computation~\cite{4,5}. This vehicular cloud (VC), consisting of mobile vehicles acting as servers, enables tasks to be scheduled and processed simultaneously across multiple vehicles. 

%The integration of smart vehicles equipped with advanced sensors and processors into the architecture of MEC technology is rapidly advancing. This integration is paving the way for a universal vehicular computing platform that will provide on-demand storage and computation for emerging vehicular applications~\cite{4,5}. 
In the meantime, we have witnessed the integration of smart vehicles equipped with increasing onboard processing power and advanced sensors into the MEC architecture. This integration is paving the way for a universal vehicular computing platform that can provide on-demand storage and computation for emerging vehicular applications~\cite{4,5}. 
%A paramount example of these computing platforms is a vehicular cloud (VC), consisting of mobile vehicles acting as servers, which enables tasks to be scheduled and processed simultaneously across multiple vehicles. 
A paramount example of these computing platforms is a mobile vehicular cloud (MVC), consisting of nearby moving vehicles acting as collaborating servers, which enables tasks to be distributed and processed simultaneously across multiple vehicles.

%f1
% Figure environment removed

%This article considers periodically generated computation-intensive tasks (e.g., every second) where a task is represented by an undirected weighted graph structure~\cite{6,7} with multiple subtasks (e.g., data processing units) and undirected edges (interdependency among subtasks). Specifically, a subtask can be assigned to a vehicle for processing, while an edge between two connected subtasks requires the vehicles that are handling them to communicate with each other via vehicle-to-vehicle (V2V) links, to support possible intermediate data exchange during subtask execution. Our key goal is to \textit{investigate how graph tasks can be scheduled over the considered vehicular cloud, namely, obtaining feasible mappings between subtasks and the dynamic vehicles under a responsive and cost-effective manner}.
In this work, we study the execution of computation-intensive tasks that are generated/collected periodically over a MVC (the left subplot in Fig. 1). A computation-intensive task is represented by an undirected weighted graph (UWG) structure~\cite{6,7} with multiple subtasks, which are interconnected by weighted edges, where an edge between two subtasks describes the required intermediate data exchange during task execution, while the edge weight encapsulates parameters such as the size of transmit data, and the required contact duration between the subtasks, etc. In this article, we consider a common scenario in a MVC, where each subtask can be assigned to a vehicle for processing, and the vehicles handling connected subtasks must communicate with each other through vehicle-to-vehicle (V2V) links to exchange data during the task execution~\cite{6,7}. 
\textit{We aim to optimize the scheduling of UWG tasks in a MVC, with the ultimate objective of obtaining feasible mappings between subtasks and the agile vehicles, ensuring both promptness and cost-efficiency of task execution}. 

%In securing necessary distributive resources, existing efforts in graph task scheduling mainly focus on onsite mechanism design \cite{6,8,9} where the decisions are made based on the current network conditions. However, significant technical challenges should be considered: 
When it comes to getting the resources needed for task execution, the current focus on graph task scheduling is geared towards developing effective \textit{onsite decision-making} mechanisms~\cite{6,9} (i.e., scheduling decisions are made at the time the task needs to be scheduled) based on the current network conditions. Nevertheless, several significant technical hurdles that need to be addressed are detailed below.

%\begin{itemize}[leftmargin=4mm] 
% \item The scheduling of graph tasks over multiple servers (vehicles) generally calls for addressing the \textit{subgraph isomorphism} problem which is NP-complete~\cite{10}. More importantly, network dynamics such as changing topology of vehicular cloud, varying channel quality, and uncertain resource supply can further complicate the problem. For example, the contact duration between two vehicles that process two connected subtasks should be larger than or at least equal to the time duration needed to finish the execution of at least one of these subtasks.
% \item In general, the performance evaluation on task scheduling relies on various evaluation indicators such as task completion time and energy consumption. Jointly optimizing different factors can always lead to a \textit{multi-objective optimization} problem where the corresponding pareto optimality is hard to obtain, or a \textit{non-linear integer programming} problem which is NP-hard~\cite{8}. 
%\end{itemize}
\begin{itemize}[leftmargin=4mm] 
 \item To schedule graph tasks over multiple servers (vehicles), it is necessary to solve the \textit{subgraph isomorphism} problem, which is NP-complete~\cite{10} (two middle subplots in Fig. 1). Additionally, network dynamics such as changing topology of vehicles, varying channel quality, and uncertainties in resource supply can make the problem more complex. For instance, when two vehicles process connected subtasks, the contact duration between them must be at least equal to (or larger than) the time duration needed to finish one of the subtasks, to ensure reliable and seamless task processing.
 \item When assessing task scheduling effectiveness, various metrics are employed, including the duration of task completion and energy consumption. When various factors are optimized together, it can result in a complex problem. This can either be a multi-objective optimization problem where achieving the pareto optimality is challenging, or a non-linear integer programming problem that is NP-hard ~\cite{8}. 
\end{itemize}

As a result, onsite decision-making may cause noticeable performance degradations as follows:
%\begin{itemize}[leftmargin=4mm] 
%\item \textit{Unguaranteed time efficiency:} Obtaining feasible mappings between subtasks and servers can always lead to excessive latency due to the complexity of the problem, which significantly decreases the available time for actual resource/data sharing, especially for mobile users, e.g., vehicles. For example, the topology of a vehicular cloud can change during the time of decision-making~\cite{11}, while the corresponding mapping may fail to protect the structure of graph tasks due to the disconnection of V2V communication links. 
%\item \textit{Excessive energy overhead:} A long-lasting decision-making process can also bring extra energy consumption, which may further result in air pollution and greenhouse effect from the long-term perspective.
%\item \textit{Unsatisfied experience of service provisioning:} Apparently, the non-negligible latency on decision-making will also impact the user experience since a certain amount of resources are reserved while waiting for the final mapping. For example, only partial of the vehicles can finally be assigned with subtasks while any resources that have been put aside for failed candidate vehicles can cause poor experience~\cite{12}. 
%\end{itemize}
\begin{itemize}[leftmargin=4mm] 
\item \textit{Unguaranteed time efficiency:} Finding feasible mappings between the subtasks and vehicles often results in long delays due to the complexity of the problem. This can reduce the amount of time for actual sharing of resources. For instance, 
%the topology of a VC can change during decision-making\cite{11}, which may cause the mapping fails to maintain the structure of graph tasks due to the disconnection of V2V communication links.
during decision-making, the topology of a MVC may change~\cite{11}, causing the obtained mapping to fail in maintaining the structure of graph tasks due to V2V communication links being disconnected.

\item \textit{Excessive energy consumption:} Obtaining the mappings between the subtasks and vehicles is a complex process that calls for significant computing resources, leading to high energy usage. This, in turn, can have a detrimental effect on the sustainability of the system, as it leads to a notable carbon footprint. 

\item \textit{Low quality of experience:} Lengthy decision-making processes can leave a negative impact on the experience of vehicles contributing their computing resources to MVC. This is due to the fact that only some vehicles will be selected for task execution and compensated, while others will not receive any compensation, which can have a detrimental effect on their overall experience~\cite{12}. 
\end{itemize}

%Motivated by the above-mentioned challenges, this article makes the first attempt to introduce an interesting graph task scheduling paradigm over vehicular cloud, called \underline{D}ouble-Plan-Promoted \underline{I}somorphic \underline{S}ubgraph Sear\underline{c}h and \underline{O}ptimization (DISCO) that involves: 
%\textit{i) Plan A}, which seeks for the optimal mapping ahead of future practical task scheduling events via analyzing historical statistics of the network, which we refer to as mapping $\alpha$ to distinguish. Specifically, mapping $\alpha$ tries to guarantee the performance of graph tasks with a high probability; and \textit{ii) Plan B}, helping with obtaining a feasible mapping $\beta$ during each practical task scheduling event under the current network conditions if $\alpha$ fails. As an efficient backup, Plan B aims to keep the reliability of the resource provisioning network. 
In light of the challenges mentioned above, this article proposes a new paradigm to schedule graph tasks over the MVC. Our approach, called \underline{d}ouble-plan-promoted \underline{i}somorphic \underline{s}ubgraph sear\underline{c}h and \underline{o}ptimization (DISCO), involves two complementary plans to ensure seamless task execution. \textit{(i)} Plan A seeks to identify the optimal mapping for future task scheduling events by analyzing historical network statistics. This mapping, referred to as $\alpha$, aims to increase the likelihood of executing future tasks with high performance. \textit{(ii)} In case mapping $\alpha$ fails at practical/actual task scheduling events, Plan B obtains a feasible mapping (mapping $\beta$) suitable for current network conditions. Having Plan A can help ease the workload of scheduling tasks in real-time, while having Plan B can ensure a reliable and timely backup option for task execution. This additional layer of preparedness differs our study from existing works and helps to improve the overall dependability of the task scheduling process. 

%In this article, we present comprehensive insights on how the proposed paradigm achieves time-efficient, cost-effective, and reliable resource provisioning in IoV, where major contributions are highlighted below:
In this work, we provide an in-depth study of DISCO and reveal how it accomplishes time-efficient, cost-effective, and reliable resource provisioning in a MVC. We highlight our major contributions below: 
%\begin{itemize}[leftmargin=4mm] 
%\item A novel double-plan-promoted graph task scheduling paradigm is introduced in IoV, where the framework, timeline, and key issues are analyzed in detail. 
%\item A case study is investigated to demonstrate how the proposed paradigm can be implemented in practice.
%\item Simulation results illustrate that the proposed paradigm achieves commendable performance on critical indicators such as task completion time, data exchange cost, and time efficiency.
%\end{itemize}
\begin{itemize}[leftmargin=4mm] 
%\item We develop a novel approach called DISCO that envisions a new methodology for the scheduling of graph tasks in MVC. We delve into DISCO's timeline and the significant obstacles it overcomes.
\item We develop a novel approach called DISCO for the scheduling of graph
tasks in a MVC. The detailed procedure, key challenges, and design considerations of DISCO are discussed. 
\item We conduct a case study to demonstrate how DISCO can be practically implemented for use over MVCs.
\item Through simulation results, we demonstrate that DISCO performs well on key metrics, including task completion time, data exchange cost, and time efficiency.
\end{itemize}
%\vfill

\section{Overview and Key Concerns}
\subsection{Framework and Timeline}
%We are interested in a certain region (e.g., Google park) consists of: \textit{i)} multiple mobile vehicles as computing service providers (resource supply); \textit{ii)} a platform (e.g., an edge server) that generates (or collects) graph tasks (resource demand) and coordinates the task scheduling procedure, and \textit{iii)} several access points (APs, e.g., road side units) that help vehicles interact with the platform through vehicle-to-infrastructure (V2I) links. Specifically, our proposed paradigm focuses on a fixed number of vehicles in the considered region during a certain period of time, where the dynamics can be reflected by the varying contact duration among vehicles (rather than the arrival and departure of vehicles, similar ideas can also be found in~\cite{6,7}). Moreover, graph tasks are announced by the platform periodically (e.g., a task per second), while during each task scheduling event, subtasks can be distributed over vehicles for processing. The corresponding framework is shown in Fig. 1, which considers a star graph-represented~\cite{8,14} task as an example. 
We consider an Internet-of-Vehicle (IoV) network in a specific area, such as Google park, which consists of three components: \textit{(i)} Multiple mobile vehicles that act as computing servers (resource supply); \textit{(ii)} A platform (e.g., an edge server) that generates or collects task requirements (resource demand) and coordinates task scheduling; \textit{(iii)} Several access points (APs, e.g., base stations, roadside units), that allow vehicles to communicate with the platform through vehicle-to-infrastructure (V2I) links. 

Our approach involves analyzing the behavior of a certain number of vehicles in a given area over a specified period. Instead of  the arrivals/departures, the changes in contact duration among them reflect their dynamics (similar to \cite{6} and \cite{7}). The platform announces graph tasks periodically, e.g., one task per second. During each task scheduling event, subtasks can be allocated to vehicles for processing. Fig. 1 illustrates an example of this framework, considering a star graph-represented task~\cite{8,14} with 6 subtasks denoted by A, B, C, D, E, and F. Specifically, obtaining feasible mappings calls for addressing \textit{subgraph isomophism problem} that extracts subgraphs with the same structure as the UWG task from the MVC topology, while the isomorphic subgraph with the least execution time (e.g., involving the vehicles with the best processing capabilities) can be selected for task execution.

%To capture the random and unpredictable nature of mobile networks\cite{12}, key uncertainties should be considered:
To ensure an accurate portrayal of IoV networks, it is essential to consider the significant uncertainties that exist within them~\cite{12}~ detailed below. 
%\begin{itemize}[leftmargin=4mm] 
%\item \textit{Fluctuant service quality:} The on-board resource supply of vehicles can always be fluctuant. For example, a vehicle may have local tasks that require a certain amount of computing/storage resources, which thus impact the computing service provided to the assigned graph tasks.
%\item \textit{Dynamic contact duration among vehicles:} A V2V contact event can occur when two vehicles are within the communication coverage of each other. Specifically, the mobility of vehicles always leads to limited and dynamic contact duration among vehicles, which calls for feasible task scheduling mechanism design to protect task structures, e.g., to support the data exchange required by connected subtasks in a graph task.
%\item \textit{Varying channel quality:} The wireless channel qualities of V2I and V2V links are changing with time due to factors such as the mobility of vehicles, obstacles (e.g., buildings), and transmission power supply. The varying V2I channel quality mainly results in fluctuant data transmission rate between vehicles and APs, while the changing V2V channel quality can impose data exchange cost among vehicles, e.g., the intermediate data transmission between two vehicles that are handling two connected subtasks of a graph task can cause energy consumption.
%\end{itemize}
\begin{itemize}[leftmargin=4mm] 
\item \textit{Fluctuant service quality:} A vehicle's supply of resources can fluctuate over time. For example, a vehicle may have local tasks that require a specific amount of computing and storage resources, affecting the service provided to the graph tasks.
\item \textit{Dynamic contact duration:} When two vehicles are within communication range of each other, they can experience a V2V contact event. Since vehicles are often moving and have limited contact duration with each other, it is important to design a task scheduling mechanism that can protect graph task structures. This includes supporting data exchange through V2V links for connected subtasks in a graph task.
\item \textit{Varying channel condition:} Wireless channel quality for V2I and V2V links is subject to change over time due to numerous factors such as vehicle mobility, obstacles, and power supply. This variability in channel quality can lead to inconsistent data transmission rates.
\end{itemize}

%Timeline for Proposed Graph Task Scheduling
%To better illustrate our proposed paradigm, the corresponding timeline is shown in Fig. 2, which is divided into two segments: Before practical task requests, Plan A helps with obtaining mapping $\alpha$ via analyzing historical statistics of the uncertain factors; while during each task scheduling event, Plan B first checks whether $\alpha$ works practically in the current network, if not, Plan B seeks for a feasible mapping $\beta$ under a timely manner. Specifically, in Event 3, a certain period of time $\tau$ has been consumed to obtain mapping $\beta$, so that the actual service delivery can only start at $t+\Delta t+\tau$. Correspondingly, improve the availability of $\alpha$ and accelerate the search of $\beta$ represent the key concerns in this paradigm. 

In DISCO, we aim to ensure seamless task execution under the uncertainties mentioned above. DISCO consists of two complementary task execution plans operating on two different segments of time as depicted in Fig. 2. Plan A involves analyzing historical statistics of uncertain factors to obtain mapping $\alpha$ before practical/actual task scheduling events. During each task scheduling event, Plan B first checks if $\alpha$ works under the current network conditions (e.g., Events 1 and 2 in Fig. 2). If $\alpha$ works, it is readily executed while if it does not, Plan B seeks for a feasible mapping $\beta$ in a timely manner (e.g., Event 3). Event 3 requires a certain period of time to obtain $\beta$, denoted by $\tau$, causing the actual service delivery to start at $t+2\Delta t+\tau$. Therefore, improving the feasibility of $\alpha$ and accelerating the search for $\beta$ are the key challenges in DISCO. 

%f2
% Figure environment removed

\subsection{Key Challenges and Design Considerations}
%This section analyzes significant concerns when designing DISCO for graph task scheduling over mobile vehicles. 
In the following, we cover significant factors when developing DISCO for scheduling graph tasks over mobile vehicles. 

%\noindent
%First, \textit{the optimality of mapping $\alpha$ represents the key concern in Plan A.} Our proposed Plan A aims to optimize the task scheduling performance from a long-term perspective, by achieving the optimal mapping $\alpha$ in advance of future task requirements, according to the historical statistics of uncertain factors in the network. During this time, accurate information extraction of these uncertain factors represents one of the most significant challenges, which leaves a direct impact on whether $\alpha$ works for future network environment. For example, inaccurate statistics of vehicular contact duration can lead to a damage of task structures. Then, possible risks can generally exist in terms of cost (e.g., delay and energy consumption) and structure preservation. For example, applying mapping $\alpha$ during task scheduling can face the risk where the task completion time exceeds its corresponding tolerance. Also, communications among vehicles may fail to support the data exchange required by connected subtasks. To this end, the rational estimation and control of risks represent another urgent and critical issue. The above discussions should be carefully considered to keep the optimality of mapping $\alpha$, and thus improve the corresponding availability.
\noindent
\textbullet~\textit{Optimality of mapping $\alpha$}. The top priority in Plan A is to enhance the efficiency and feasibility of mapping $\alpha$. To accomplish this, we have to devise a strategy that strives to elevate the long-term task scheduling performance by preemptively identifying the optimal mapping $\alpha$, drawing on past data/statistics of uncertain network factors. However, the precision of gathered information on the uncertainties remains a key hurdle, as it directly impacts the efficacy of $\alpha$ for future network conditions. For example, if the statistics about how long vehicles are in contact with each other are inaccurate, task execution using mapping $\alpha$ can lead to violating the task execution requirements since communications among vehicles may fail to support the data exchange between connected subtasks. Thus, utilizing mapping $\alpha$ is associated with a set of risks in terms of cost (e.g., delay and energy consumption) and task structure preservation. Further, mapping $\alpha$ may result in the task taking longer to complete than its delay tolerance. Therefore, it is crucial to estimate and control risks effectively to maintain the optimality of mapping $\alpha$ and improve its applicability.

%\noindent
%Then, \textit{the time-effectiveness of mapping $\beta$ represents the key concern in Plan B.} Our proposed Plan B considers a short-term perspective, which occurs in each practical task scheduling event. Specifically, Plan B first examines whether mapping $\alpha$ works smoothly in the current network condition, e.g., if $\alpha$ can ensure the successful completion of graph tasks. Otherwise, Plan B solves isomorphic subgraph search and optimization problems instantaneously, according to the current network information, which obtains a feasible mapping $\beta$. During this time, time-effectiveness represents the most important concern since factors such as the topology of vehicular cloud, on-board resources, and channel quality can change over time. For example, a long-lasting decision-making process (e.g., a long period of time spent on searching and optimizing mapping $\beta$) may bring a failure of task scheduling (and thus the uncompletion of graph tasks), due to the mobility of vehicles. Therefore, achieving an acceptable computational complexity of the scheduling algorithm design remains nonnegligible. 
\noindent
\textbullet~\textit{Time-effectiveness of mapping $\beta$}. Our approach for Plan B is centered around short-term objectives for every task scheduling event. Our first step is to evaluate if mapping $\alpha$ is functioning in the current network conditions, ensuring that tasks can be successfully completed. If not, Plan B promptly tackles isomorphic subgraph search and subtask allocation optimization using the latest network information, so that we can obtain a viable mapping $\beta$. Thus, when it comes to Plan B, the most important consideration is how quickly $\beta$ can be obtained. Time-effectiveness is paramount during this process since the topology of the MVC, on-board resources, and wireless channel qualities can change over time. Prolonged decision-making processes, e.g., spending too much time on searching and optimizing $\beta$, can lead to task scheduling failure due to vehicles' mobility. In sum, it is essential to achieve a low computational complexity for the algorithm used in Plan B.

%All in all, the proposed paradigm aims to achieve an optimal mapping $\alpha$, and a feasible mapping $\beta$ under a timely manner, to support responsive, cost-effective and reliable graph task scheduling in dynamic IoV environment.
Overall, DISCO strives to achieve an optimal mapping for $\alpha$ and a feasible mapping for $\beta$ in a timely manner. This will support responsive, cost-effective, and reliable scheduling of graph tasks in dynamic IoV environments.

\section{Case Study}
%This section investigates a case study associated with our proposed DISCO for graph task scheduling over vehicular cloud, where the platform announces one graph task with multiple subtasks periodically, for analytical simplicity. 
In this section, we take a closer look at a case study related to DISCO for graph task scheduling over a MVC. 
%We assume that the platform periodically announces one graph task with multiple subtasks to simplify analysis.

\subsection{Preliminaries}
\noindent
\textbf{Model of graph task.} We consider a computation-intensive task modeled as an UWG $\bm{\mathcal{G}^{task}}$ with a subtask set $\bm{\mathcal{V}^{task}}$, an edge set $\bm{\mathcal{E}^{task}}$ and an edge weight set $\bm{\mathcal{W}^{task}}$. Each subtask in $\bm{\mathcal{V}^{task}}$ has three attributes: tolerable completion time, data size, and the required computing resources. An edge in $\bm{\mathcal{E}^{task}}$ describes the interdependency between two subtasks~\cite{6,7}. Additionally, each edge is associated with a weight (collected in $\bm{\mathcal{W}^{task}}$) that indicates the required time for data exchange, which depends on the execution order of connected subtasks. For example, when two connected subtasks are completed on separate vehicles, with completion times of 2 seconds and 3 seconds, the edge weight is 2 seconds to ensure continuous connection during the execution time. 
Also, the contact duration of two vehicles that handle these connected subtasks should be long enough (i.e., larger than or at least equal to the corresponding weight) to allow data transfer between the subtasks.

\noindent
\textbf{Model of MVC.} We model the MVC as an undirected weighted graph $\bm{\mathcal{G}^{cloud}}$ consisting of a set $\bm{\mathcal{V}^{cloud}}$ of vehicles, an edge set $\bm{\mathcal{E}^{cloud}}$ and a weight set $\bm{\mathcal{W}^{cloud}}$. To capture the dynamics of resource supply, each vehicle in $\bm{\mathcal{V}^{cloud}}$ has an attribute $f$ that describes its computing capability in terms of its CPU frequency. Also, each edge in $\bm{\mathcal{E}^{cloud}}$ indicates the possibility of one-hop V2V communication between two vehicles. Also, each edge weight in $\bm{\mathcal{W}^{cloud}}$ has two attributes: \textit{i)} anticipated V2V contact duration $t$ between vehicles and \textit{ii)} data exchange cost $c$ for processing of connected subtasks, e.g., energy consumption spent on data sharing. These two attributes are both modeled as random variables to capture vehicles' mobility.  

In this case study, we assume that the IoV platform generates/collects graph tasks. A subtask can be offloaded to a feasible vehicle for processing via getting access to a nearby AP, while the data transmission rate of the corresponding V2I link is considered as a random variable $r$ to describe the unpredictable nature of the network condition, e.g., physical obstacles and vehicular transmission power supply. 

\subsection{Problem Formulation and Solution Design}
%This case study aims to minimize both \textit{i)} the task completion time, which depends on the time of the latest finished subtask via considering data transmission and execution delay; and \textit{ii)} the overall data exchange cost, which quantizes the overhead (e.g., energy consumption) on intermediate data sharing among vehicles during subtask processing. Based on which, the cost function $\mathbb{C}$ is formulated as the weighted sum of the above two objectives. 
Our goal in this case study is to reduce task completion time by taking into account data transmission and execution delays, as well as to minimize overall data exchange costs among vehicles during subtask processing. This is achieved by formulating a cost function $\mathbb{C}$. In particular, $\mathbb{C}$ weights both the task completion time and the data exchange cost by 0.5, and then sums them together. 

%Since the timeline of the proposed task scheduling is divided into two segments, our key goal in plan A is to minimize the expected value of cost function $\mathbb{C}$, which we refer to as $\overline{\mathbb{C}}$. Specifically, possible risks can exist during this time owing to the uncertainties (e.g., random variables $f$, $t$, $c$, and $r$).
%Thus, Plan A considers two key risks as probabilistic constraints via analyzing the historical statistics, e.g., distributions of $f$, $t$, $c$, and $r$: \textit{i)} the risk of an unacceptable task completion time, which is calculated by the probability that the task completion time exceeds the maximal tolerant time among subtasks; and \textit{ii)} the risk of structural damage of the graph task, which is expressed by the probability that any vehicle pair fails to support the data exchange time required by the corresponding connected subtasks. To this end, the optimization problem in Plan A is given in by the following (1), where risks should be controlled within a certain range. 

Our proposed task scheduling timeline is divided into two segments. In plan A, our main objective is to minimize the expected value of cost function $\mathbb{C}$, which we refer to as $\overline{\mathbb{C}}$. When developing Plan A, it is important to consider possible risks due to uncertainties, such as random variables ($f$, $t$, $c$, and $r$). Therefore, Plan A focuses on analyzing historical statistics, such as the distributions of $f$, $t$, $c$, and $r$, to formalize two key risks as probabilistic constraints. The first risk is the probability of task completion time exceeding the maximal tolerable execution time among subtasks, which ensures timely completion of graph tasks. The second risk is the probability of structural damage to the graph task, which is expressed by the probability of any vehicle pair failing to support the transmission of necessary data for their corresponding assigned subtasks. Plan A solves the optimization problem to minimize $\overline{\mathbb{C}}$, under risk control constrains, which 
%%eq1
%\begin{align} 
%\label{eq1} 
%&~~~~~~~~~~~\alpha = \arg\min\overline{\mathbb{C}} \notag\\
%&\text{s.t.~Risk~control~constraints,}
%\end{align} 
constrain the above-discussed risks within certain ranges. For example, the probability of an overtime task completion can be limited under 30$\%$.

During practical task scheduling, Plan B first checks the feasiblity of mapping $\alpha$. If it is unfeasible, Plan B looks for a feasible mapping $\beta$ that minimizes $\mathbb{C}$'s practical value while respecting the time and structure preservation constraints based on the current network condition. These constraints ensure the on-time task completion and data exchange between  connected subtasks to maintain the corresponding task structure.
%\subsection{Solution design}
%The problem in both Plan A and Plan B refers to a non-linear integer programming problem which is NP-Hard. To this end, to solve the isomorphic subgraph search and optimization problem under probabilistic constraints, in Plan A, we first decouple the problem into two subproblems: \textit{i)} feasible mapping search, and \textit{ii)} optimal mapping $\alpha$ selection. In the former, we borrow the idea from~\cite{13} and investigate an efficient algorithm under risk analysis, which can achieve all the feasible mappings while providing acceptable computational complexity. Then, in the latter, the optimal mapping $\alpha$ can be obtained with the lowest expected value of cost function among given mappings. 
%During each practical task scheduling event, Plan B checks if $\alpha$ works in the current network, or look for a mapping $\beta$ when necessary, via utilizing a similar algorithm designed in Plan A.
 
%which starts with obtaining a pivot subtask and the corresponding candidate vehicles, via considering the degree, probabilistic constraints and neighborhood information in both task and VC graphs. Then, the concept of eccentricity is introduced to determine the candidate regions for each subtask, which is 

Both Plan A and Plan B involve finding the allocation/mapping vectors between subtasks and vehicles. Thus, they face a challenging nonlinear integer programming problem which is generally NP-hard. To tackle this issue and optimize the subgraph search while considering probabilistic constraints, Plan A divides the problem into two subproblems. The first subproblem concerns finding feasible mappings. Inspired by\cite{13}, we accomplish a risk-aware mapping search algorithm, which guarantees that all possible mappings are achievable under a reasonable computational complexity. 
In the second subproblem, we select the optimal mapping, $\alpha$, by identifying the one with the lowest expected cost function value among all possible mappings. In practical task scheduling events, Plan B verifies if $\alpha$ is compatible with the current network. If necessary, it searches for a mapping $\beta$ using a similar algorithm to the one used in Plan A.

%5
\subsection{Performance Evaluation}
In this case study, we compare the performance of DISCO against multiple baselines: \textit{i)} onsite task scheduling (Onsite), which only uses Plan B; \textit{ii)} random task scheduling (Random), which randomly assigns subtasks to available vehicles at each task scheduling event~\cite{14}; \textit{iii)} time-preferred (TimeGreedy) and degree-preferred (DegreeGreedy) task scheduling~\cite{15} which rely on greedy-based approaches to assign subtasks to vehicles at each task scheduling event: TimeGreedy maps each subtask to the vehicle with the lowest execution time, while DegreeGreedy assigns each subtask to the vehicle with the largest number of V2V connections with others; and \textit{iv)} exhaustive search (ExhaustiveS), which examines all possible mappings and selects the best one with the lowest cost function value at each practical scheduling event.
We assume that: $f$ (GHz) obeys a gaussian distribution with its mean falling in $[2,4]$ and variance in $[0.04,0.07]$ for each vehicle, $t$ (second) follows an exponential distribution with its mean falling in $[5,16]$, $c$ follows a normal distribution with its mean in $[0.03,0.07]$ and variance of 0.001, while $r$ (Mb/s) obeys a gaussian distribution with its mean in $[5,7]$ and variance of 0.55. Note that practical values of the above random variables can neither be negative nor be too large in real-world networks due to the vehicular hardware settings and communication standards. We thus constrain/clip them as: $f\in[1.5,4.5]~\text{GHz}$, $t\in[0,60]~\text{seconds}$, $c\in[0.025,0.075]$ and $r\in[4,8]~\text{Mb/s}$~\cite{6,7}.
\vfill

%%f3
%% Figure environment removed

In Fig. 3, we compare the average running time (ART) between our proposed DISCO and other methods, which reflects the time spent on searching for the best mappings between the subtasks and vehicles. We consider 100 simulations (100 task scheduling events) to evaluate the ART, where our results demonstrate the time effectiveness of DISCO.  
% Specifically, we use logarithmic representation to highlight the gap among different methods, upon considering three task types given by Fig. 3. Apparently, with the increasing number of subtasks and vehicles/connections, the ART curve of each method shows a general raising tendency owing to the growing problem size. Specifically, ART of ExhaustiveS illustrates a sharp rise due to a large searching space, which is unacceptable in real-world dynamic IoV. For example, considering tadpole-based task (type 3) and 15 vehicles in Fig. 4, the value of ART even exceeds $10^{3}~\text{seconds}$. Since Onsite method applies our proposed Plan B with an available computational complexity, it outperforms ExhaustiveS but the corresponding ART still stays above other methods due to that Onsite aims to obtain all the feasible mappings and selects the optimal one among them. Although Random, TimeGreedy, and DegreeGreedy achieves low values of ART and sometimes outperform our proposed DISCO, which always fail to obtain all the feasible mappings and thus leads to larger average value of cost function as shown in the following Figs. 5-7. In other words, these three algorithms usually gets a locally optical solution rather the global one. Our proposed DISCO achieves commendable performance on ART since the prior obtained mapping $\alpha$ can work in most simulations, which greatly reduces the decision-making latency and thus suits well in dynamic wireless networks. 
We use logarithmic representation to evaluate different methods for three types of tasks shown on the right side of Fig. 1 (i.e., Type 1: Bull graph with 5 nodes; Type 2: Star graph with 6 nodes; Type 3: Tadpole graph with 7 nodes). As expected, the larger the number of subtasks and vehicles/connections, the longer it takes for each method to obtain feasible mappings, as shown by the increasing ART values. ExhaustiveS has a particularly sharp increase in ART due to its large search space, which is not suitable for real-world dynamic IoV. Onsite method outperforms ExhaustiveS by using a less complex approach, but its ART is still higher than other methods because it seeks to find all feasible mappings and chooses the best one. Although Random, TimeGreedy, and DegreeGreedy methods have lower ART values and sometimes perform better than DISCO, we will later show that these methods incur large costs. Our proposed DISCO has commendable performance on ART because the prior obtained mapping $\alpha$ can be used in most practical scheduling events, significantly reducing decision-making latency and making it suitable for dynamic IoV networks. As can be seen from Fig. 3, for tadpole task (type 3) with 15 vehicles, the value of ART can exceed $10^3$ second when using ExhaustiveS, which is impractical. 

%f4
% Figure environment removed

%f5
% Figure environment removed

%f6
% Figure environment removed

%f7
% Figure environment removed

%The performance on the average value of cost function (AVCF) is demonstrate by Figs. 5-7, upon considering 100 simulations and different settings of vehicular cloud as well as task types. Note that the unacceptable time overhead of ExhaustiveS has stopped us to show its performance in Figs. 5-7. As can be seen in these figures, our proposed DISCO greatly outperforms the random- and greedy-based methods on AVCF thanks to the optimality of mappings $\alpha$ and $\beta$. Moreover, DISCO obtains a similar performance on AVCF with the Onsite method, while outperforming it in Fig. 4 on the value of ART. All in all, our proposed DISCO for graph task scheduling achieves commendable performance on significant evaluation indicators in comparison with existing methods, which is worthy of reference for future resource sharing markets. 
We depict the performance of our proposed solution, DISCO, in terms of the average value of cost function (AVCF) and compare that to other methods in Figs. 4-6 for different task topologies. The results are consistent and reveal that DISCO greatly outperforms random and greedy-based methods due to using the complementary optimal mappings $\alpha$ and $\beta$. Onsite and ExhaustiveS methods achieve the same value of AVCF since they both search for all the possible mappings, while ExhaustiveS suffers from unacceptable time overhead as shown in Fig. 3. Interestingly, DISCO performs similarly to the Onsite method on AVCF, while outperforming it on the value of ART (as shown in Fig. 3), which reveals the importance of exploiting mapping $\alpha$ to reduce the running time of ART, while achieving a high cost efficiency.
%We did not include the ExhaustiveS method in these figures due to its unacceptable time overhead.

Overall, DISCO achieves a commendable performance on key evaluation indicators in comparison with existing methods, making it a worthy reference for resource provisioning in IoV networks. 

\section{Future Directions}
%\noindent
%Our proposed DISCO for graph task scheduling introduces a series of open research directions, which we discuss below:  
In the following, we discuss several avenues of research that are motivated by DISCO.
\begin{itemize}[leftmargin=4mm] 
 \item \textit{Incentive design:} 
% One of the primary concerns in the implementation of resource sharing involves the transmission/storage/processing of large volume of data, which can impose additional costs on vehicles. A practical way is to offer monetary incentives to integrate distributed on-board resources, which calls for establishing a bridge between the IoV network and resource trading market. Several fundamental problems in such a market can be considered, e.g., service price determination, service contract negotiation. 
When it comes to distributed resource provisioning through vehicles, a major issue is engaging vehicles in the transmission, storage, and processing of data. A possible solution is to offer financial incentives to vehicles in exchange for their resources. To make this work, a connection between resource provisioning methods and resource trading markets needs to be established. To do so, there are several challenges to consider, such as determining prices for services and negotiating service contracts.
 
\item \textit{Overbooking-promoted subtask duplication:} 
%Due to the random and unpredictable nature of mobile networks, e.g., a promissory vehicle may be absent from the considered region, or the communications among vehicles may risk outage events during the current task scheduling procedure owing to vehicular mobility, which thus lead to fluctuations in resource supply. To this end, an interesting concept called ``overbooking'' is introduced, allowing each task (or subtask) to be mapped to more vehicles, in case that some of them may fail to offer services. In this case, to ensure the performance of computation-intensive tasks, the data of each subtask can be transmitted to multiple vehicles for processing, which, however, may also impose excessive overhead, e.g., extra delay and energy consumption as we as possible abuse of limited resources under a large overbooking rate. Thus, how to optimize the overbooking rate remains an urgent and critical topic. 
Mobile networks can be unpredictable, which may cause issues during task scheduling due to the mobility of vehicles. This can lead to resource supply fluctuations and potential outage events. To address this, the concept of ``overbooking'' can be introduced, allowing each subtask to be assigned to more than one vehicle in case some vehicles may fail to offer services. However, overbooking can impose excessive overhead, including delay, energy consumption, and possible resource abuse. Therefore, optimizing the overbooking procedure is a critical topic for ensuring high performance graph task scheduling. 

\item \textit{Accurate and timely environment assessment:} 
%Generally, multiple uncertainties can exist in a wireless mobile network environment, e.g., dynamic resource supply/demand, varying channel quality, possible communication outage, uncertain vehicular willingness of offering services. However, inaccurate statistics and prediction can leave severe impacts on task scheduling performance. Thus, the perception and sharing of information of uncertain factors call for smart algorithm design to better understand the dynamic environment under a timely manner. 
In wireless mobile networks, there are often various uncertainties such as resource availability, communication disruptions, and service providers' willingness. Inaccurate predictions of these uncertainties can negatively impact task scheduling performance. Therefore, it is crucial to utilize smart algorithms (e.g., time series prediction and recurrent neural networks) that can quickly and accurately perceive and obtain information about the environment's uncertainties, thereby improving the overall task scheduling performance.

\item \textit{Intelligent risk prediction, quantization, and management:} 
%The proposed Plan A represents a coexistence between risks and opportunities, where proper estimation and control of possible risks can facilitate a stable and healthy resource trading market. Thus, it is critical to design risk prediction, quantization, and management mechanisms with intelligent features in the future, rather than those probability-based models considered in this article.
The proposed Plan A aims to maintain a stable and healthy resource trading market by balancing risks and opportunities. To assess risks, intelligent mechanisms for risk prediction, quantification, and management need to be developed. Instead of using probabilistic-based models, learning-based methods can be utilized to model potential risks. In order to achieve this, future work can consider drawing a connection between the fields of explainable artificial intelligence (XAI) and risk analysis.

\item \textit{Competition and cooperation among vehicles:} 
%Competition and cooperation among servers can exist in a real-world resource trading market. For example, it is interesting to investigate possible cooperation and revenue transfer mechanisms among vehicles, while considering the structural characteristics of graph tasks, e.g., structure-preservation-based cooperative game. 
In a real-world VC, there can be both competition and cooperation among vehicles. One interesting area for research is investigating potential mechanisms for cooperation and revenue transfer among vehicles, while taking into account the structural characteristics of graph tasks. This could lead to a new research area on graph task execution via cooperative games. Another way to view competition among vehicles is through the lens of auction theory.

\item \textit{Addressing the curse of dimensionality:}
%Also, competition among vehicles can be considered as an auction market.
%In addition, we are interested in exploring the possible cooperation among subtasks, e.g., a large-scale graph task can be remodeled as a small-scale graph since some subtasks with similar properties can form clusters. In this case, the size of the considered optimization problem can be reduced to accelerate solution designs. 
To reduce the problem space size, one way is to look for similarities between subtasks within a large graph task. For instance, in a large graph task, we can group subtasks with similar attributes to form clusters. Another option is to group vehicles with similar resources and contact durations, treating them as a single unit. This can lead to accelerating the optimization process by reducing the size of the problem.
\end{itemize}

\section{Conclusion}
%This article focuses on the problem of task scheduling over mobile vehicles, where both tasks and vehicular cloud are modeled as undirected weighted graphs. Motivated by the drawbacks of onsite task scheduling such as excessive overhead on decision-making, we propose a novel paradigm for graph task scheduling over vehicular cloud called DISCO, which consists of Plan A and Plan B. The goal of Plan A is to obtain the optimal mapping $\alpha$ ahead of future practical task scheduling process by analyzing historical statistics of uncertain factors such as mobility of vehicles and varying channel quality. Under given $\alpha$, during each practical task scheduling event, Plan B is regarded as a backup which looks for a commendable mapping $\beta$ under a timely manner, when $\alpha$ fails to work in the current network. The timeline and key concerns of DISCO are analyzed in detail. In addition, a case study with comprehensive simulations is conducted to illustrate the commendable performance of DISCO. Interesting research directions are introduced to offer references to future resource sharing in IoV. 
This work optimizes the allocation of computation-intensive tasks, modeled as undirected weighted graphs, over mobile vehicles within mobile vehicular clouds. To tackle the challenges of onsite scheduling in terms of excessive decision-making overhead, we introduced a novel approach called DISCO that encompasses two complementary plans for task scheduling: Plan A and Plan B. The objective of Plan A is to analyze historical statistics of uncertain factors like vehicle mobility and channel quality variations to obtain the best mapping $\alpha$ ahead of future practical task scheduling. During each practical task scheduling event, Plan B is considered as a backup option to quickly identify a suitable mapping $\beta$ in case $\alpha$ fails to operate efficiently under the current network conditions. We investigated DISCO's procdure and primary factors that contribute to its success, followed by a case study that includes simulations showcasing the performance of DISCO in terms of cost and time efficiency. We also highlighted a set of research directions that can serve as a reference for developing more efficient resource provisioning methods for future IoV. 

%\section*{Acknowledgement}
%This work was supported in part by the National Natural Science Foundation of China under Grant no. 62271424, the Natural Science Foundation of Xiamen City (Grant nos. 3502Z20227002, 3502Z20227007), the Fundamental Research Funds for the Central Universities under Grant no. 20720230035, the Basic and Applied Basic Research Foundation of Guangdong Province under Grant no. 2022A1515110042. 

\ifCLASSOPTIONcaptionsoff
 \newpage
\fi

\begin{thebibliography}{15}
%
%K. Wang, L. Wang, C. Pan and H. Ren, “Deep Reinforcement Learning-Based Resource Management for Flexible Mobile Edge Computing: Architectures, Applications, and Research Issues,” \textit{IEEE Veh. Technol. Mag.}, vol. 17, no. 2, pp. 85--93, 2022.

\bibitem{1} H. Li, K. Ota, and M. Dong, “Learning IoV in 6G: Intelligent Edge Computing for Internet of Vehicles in 6G Wireless Communications,” \textit{IEEE Wireless Commun.}, pp.1-1, 2023.
%
%\bibitem{2} B. Gu and Z. Zhou, “Task Offloading in Vehicular Mobile Edge Computing: A Matching-Theoretic Framework,” \textit{IEEE Veh. Technol. Mag.}, vol. 14, no. 3, pp. 100--106, 2019.
\bibitem{2} Y. Deng, X. Chen, G. Zhu, Y. Fang, Z. Chen and X. Deng, “Actions at the Edge: Jointly Optimizing the Resources in Multi-Access Edge Computing,” \textit{IEEE Wireless Commun.}, vol. 29, no. 2, pp. 192-198, 2022.

\bibitem{3} G. Cui, Q. He, F. Chen, Y. Zhang, H. Jin and Y. Yang, “Interference-Aware Game-Theoretic Device Allocation for Mobile Edge Computing,”\textit{IEEE Trans. Mobile Comput.}, vol. 21, no. 11, pp. 4001-4012, 2022.

\bibitem{4} L. Liu, J. Feng, X. Mu, Q. Pei, D. Lan and M. Xiao, “Asynchronous Deep Reinforcement Learning for Collaborative Task Computing and On-Demand Resource Allocation in Vehicular Edge Computing,” \textit{IEEE Trans. Intell. Transp. Syst.}, pp. 1--1, 2023.

\bibitem{5}  J. Liu, N. Liu, L. Liu, S. Li, H. Zhu and P. Zhang, “A Proactive Stable Scheme for Vehicular Collaborative Edge Computing,” \textit{IEEE Trans. Veh. Technol.}, pp. 1--1, 2023.

\bibitem{6} Z. Gao, M. Liwang, S. Hosseinalipour, H. Dai and X. Wang, “A Truthful Auction for Graph Job Allocation in Vehicular Cloud-Assisted Networks,” \textit{IEEE Trans. Mobile Comput.}, vol. 21, no. 10, pp. 3455--3469, 2022.

\bibitem{7} M. Liwang, Z. Gao, S. Hosseinalipour, H. Dai and X. Wang, "Energy-aware Allocation of Graph Jobs in Vehicular Cloud Computing-enabled Software-defined IoV,” \textit{IEEE Conf. Computer Commun. Workshops (INFOCOM WKSHPS)}, Toronto, ON, Canada, 2020, pp. 604--609.

\bibitem{8} S. Hosseinalipour, A. Nayak, and H. Dai, “Power-aware Allocation of Graph Jobs in Geo-Distributed Cloud Networks,” \textit{IEEE Trans. Parallel Distrib. Syst.}, vol. 31, no. 4, pp. 749--765, Apr. 2020.

\bibitem{9} H. Liao, X. Li, D. Guo, W. Kang, and J. Li, “Dependency-Aware Application Assigning and Scheduling in Edge Computing,”\textit{IEEE Internet of Things J.}, vol. 9, no. 6, pp. 4451-4463, 2022.

\bibitem{10} S. Wi, S. Woo, J. J. Whang, and S. Son, “HiddenCPG: Large-Scale Vulnerable Clone Detection Using Subgraph Isomorphism of Code Property Graphs,”\textit{Proc. ACM Web Conf.}, Lyon, France, 2022, pp. 755--766.

\bibitem{11}W. Feng, N. Zhang, S. Li, S. Lin, R. Ning, S. Yang, T. Gao, “Latency Minimization of Reverse Offloading in Vehicular Edge Computing,"”\textit{IEEE Trans. Veh. Technol.}, vol. 71, no. 5, pp. 5343-5357, 2022.

\bibitem{12}  S. Sheng, R. Chen, P. Chen, X. Wang, and L. Wu, “Futures-based Resource Trading and Fair Pricing in Real-Time IoT Networks,” \textit{IEEE Wireless Commun. Lett.}, vol. 9, no. 1, pp. 125--128, 2020.

\bibitem{13} M. Abulaish, Z. A. Ansari, and Jahiruddin, “Subiso: A Scalable and Novel Approach for Subgraph Isomorphism Search in Large Graph,” \textit{IEEE Int. Conf. Commun. Syst. Netw.}, Bengaluru, India, pp. 102--109, 2019.

\bibitem{14} M. Liwang, Z. Gao, S. Hosseinalipour, Y. Su, X. Wang and H. Dai, “Graph-Represented Computation-Intensive Task Scheduling Over Air-Ground Integrated Vehicular Networks,” \textit{IEEE Trans. Services Comput.}, pp. 1--1, 2023.

\bibitem{15} S. Luo, X. Chen, Q. Wu, Z. Zhou, and S. Yu, “HFEL: Joint Edge Association and Resource Allocation for Cost-Efficient Hierarchical Federated Edge Learning,”\textit{IEEE Trans. Wireless Commun.}, vol. 19, no. 10, pp. 6535-6548, 2020.

\end{thebibliography}

%~~~~~~~~~~~~~~~~~~~~~~~~~~~~~~~~~~~~~~~~~~~~~~Biography
%\begin{IEEEbiography}[{% Figure removed}]
%{Minghui Liwang}[M'19] is currently an assistant professor in School of Informatics, Xiamen University, China. Her research interests include wireless communication systems, Internet of Things, cloud/edge/service computing, as well as economic models and applications in wireless communication networks.
%\end{IEEEbiography}
%
%%Bingshuo Guo
%\begin{IEEEbiography}[{% Figure removed}]
%{Bingshuo Guo} received his B.S degree in communication engineering from Yanshan University, China, in 2022. He is currently working toward the M.S. degree in School of Informatics, Xiamen University, China. His research interests include mobile crowdsensing networks, graph theory and cloud/edge/service computing.
%\end{IEEEbiography}
%%\space{-0.1cm}
%
%%Zhanxi Ma
%\begin{IEEEbiography}[{% Figure removed}]
%{Zhanxi Ma} recived his B.S. degree in School of
%Informatics, Xiamen University, China, in 2023. He
%is currently working toward the Ph.D. degree in School of Electronic and Engineering, 
%Nanjing University, China. His research interests
%include space-terrestrial decoupling, resource
%scheduling in IoV, and edge computing. 
%\end{IEEEbiography}
%
%%Yuhan Su
%\begin{IEEEbiography}[{% Figure removed}]
%{Yuhan Su} is currently an assistant professor with School of Electronic Science and Engineering, Xiamen University, Xiamen, China. His research interests include wireless communications, UAV networks, and machine learning.
%\end{IEEEbiography}
%
%%Seyyedali Hosseinalipour 
%\begin{IEEEbiography}[{% Figure removed}]
%{Seyyedali Hosseinalipour} [M’20] is currently an assistant professor with the Department of Electrical
%Engineering, University at Buffalo, SUNY, Buffalo, NY, USA. His research interests include
%6G, machine learning, federated learning, fog
%and edge computing, and network optimization.
%\end{IEEEbiography}
%
%%Xianbin Wang
%\begin{IEEEbiography}[{% Figure removed}]
%{Xianbin Wang}[S’98-M’99-SM’06-F’17] is a Professor and Tier-1 Canada Research Chair at Western University, Canada. 
%%He received his Ph.D. degree in electrical and computer engineering from the National University of Singapore in 2001. Prior to joining Western, he was with Communications Research Centre Canada (CRC) as a Research Scientist/Senior Research Scientist between July 2002 and Dec. 2007. From Jan. 2001 to July 2002, he was a system designer at STMicroelectronics. 
%His current research interests include 5G/6G technologies, Internet-of-Things, communications security, machine learning and intelligent communications. 
%%Dr. Wang has over 450 highly cited journal and conference papers, in addition to 30 granted and pending patents and several standard contributions. Dr. Wang is a Fellow of Canadian Academy of Engineering, a Fellow of Engineering Institute of Canada, a Fellow of IEEE and an IEEE Distinguished Lecturer. He has received many awards and recognitions, including Canada Research Chair, CRC President’s Excellence Award, Canadian Federal Government Public Service Award, Ontario Early Researcher Award and six IEEE Best Paper Awards. He currently serves/has served as an Editor-in-Chief, Associate Editor-in-Chief, Editor/Associate Editor for over 10 journals. He was involved in many IEEE conferences including GLOBECOM, ICC, VTC, PIMRC, WCNC and CWIT, in different roles such as symposium chair, tutorial instructor, track chair, session chair, TPC co-chair and keynote speaker. He has been nominated as an IEEE Distinguished Lecturer several times during the last ten years. Dr. Wang is currently serving as the Chair of IEEE London Section and the Chair of ComSoc Signal Processing and Computing for Communications (SPCC) Technical Committee.
%\end{IEEEbiography}
%
%%Huaiyu Dai
%\begin{IEEEbiography}[{% Figure removed}]{Huaiyu Dai}[F’17] He is currently
%a Professor of Electrical and Computer Engineering with NC State University, Raleigh. His current research
%focuses on networked information processing and crosslayer design
%in wireless networks, cognitive radio networks, network security, and
%associated information-theoretic and computation-theoretic analysis. 
%\end{IEEEbiography}
%~~~~~~~~~~~~~~~~~~~~~~~~~~~~~~~~~~~~~~~~~~~~~~Biography

\vfill

\end{document}





