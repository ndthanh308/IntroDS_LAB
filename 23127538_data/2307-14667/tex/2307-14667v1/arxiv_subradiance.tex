\documentclass[article,amsmath,amssymb]{revtex4-2}
\usepackage{graphicx}
\usepackage{dcolumn}
\usepackage[colorlinks]{hyperref}
\usepackage{bm}

\begin{document}

\title{Single Photon Superradiance and Subradiance as Collective Emission From Symmetric and Antisymmetric States}


\author{Nicola Piovella}

\affiliation{Dipartimento di Fisica "Aldo Pontremoli", Universit\`{a} degli Studi di Milano, Via Celoria 16, I-20133 Milano, Italy
\&
INFN Sezione di Milano, Via Celoria 16, I-20133 Milano, Italy}

\author{Stefano Olivares}

\affiliation{Dipartimento di Fisica "Aldo Pontremoli", Universit\`{a} degli Studi di Milano, Via Celoria 16, I-20133 Milano, Italy
\&
INFN Sezione di Milano, Via Celoria 16, I-20133 Milano, Italy}


\begin{abstract}
Recent works have shown that collective single photon spontaneous emission from an ensemble of $N$ resonant two-level atoms is a rich field of study. Superradiance describes emission from a completely symmetric state of $N$ atoms, with a single excited atom prepared with a given phase, for instance imprinted by an external laser. Instead, subradiance is associated with the emission from the remaining $N-1$ asymmetric states, with a collective decay rate less than the single-atom value. Here, we discuss the properties of the orthonormal basis of symmetric and asymmetric states and the entanglement properties of superradiant and subradiant states.
\end{abstract}

\maketitle
 
\section{Introduction}

Cooperative light scattering by a system of $N$ two-level atoms has been a topic studied since many years \cite{Lehmberg1970}.  
Many studies in the past have been focused on a diffusive regime dominated by multiple scattering \cite{Lagendijk1988}, where  light travels over distances much larger than the mean free path. More recently, it has been shown that light scattering in dilute systems  induces a dipole-dipole interaction between atom pairs, leading to a different regime dominated by  single scattering of photons by many atoms. The transition between single and multiple scattering is controlled by the optical thickness parameter $b(\Delta)=b_0/(1+4\Delta^2/\Gamma^2)$ \cite{labeyrie2003slow,guerin2017light}, where $b_0$ is the resonant optical thickness, $\Delta$ is the detuning of the laser frequency from the atomic resonance frequency and $\Gamma$ is the single atom decay rate. A different cooperative emission is provided by superradiance and subradiant, both originally predicted by Dicke in 1954 \cite{Dicke1954} in fully inverted system. Whereas Dicke superradiance is based on constructive interference between many emitted photons, subradiance is a destructive interference effect leading to the partial trapping of light in the system. Dicke states have been considered for an assembly of $N$ two-level systems, realized, e.g., by atoms \cite{gross1982} or quantum dots \cite{lodahl2004}. In contrast to an initially fully inverted system with $N$ photons stored by $N$ atoms, states with at most one single excitation have attracted increasing attention in the context of quantum information science \cite{brandes2005,karasik2007,pedersen2009}, where the accessible Hilbert space can be restricted to single excitations by using, e.g., the Rydberg blockade \cite{tong2004,urban2009,gaetan2009}. A particular kind of single-excitation superradiance has been proposed by Scully and coworkers \cite{Scully2006,Svidzinsky2008,Svidzinsky2010}, in a system of $N$ two-level atoms prepared by the absorption of a single photon (Timed Dicke state).
A link between this single-photon superradiance and the more classical process of cooperative scattering of an incident laser by $N$ atoms  has been proposed by a series of theoretical and experimental papers \cite{Courteille2010,Bienaime2010,Bienaime2013b,Chabe2014,Bachelard2016}. In such systems of driven cold atoms subradiance has been also predicted \cite{Bienaime2012} and then observed \cite{Guerin2016}, after that the laser is abruptly switched off and the emitted photons detected in a given direction. Subradiance, by itself, has attracted a large interest of its application in the quantum optics as a possible method to control the spontaneous emission, storing the excitation for a relatively long time.
A crucial point is to determine if such subradiant states are entangled or not, in view of a possible application as quantum memories.

The aim of this paper is to provide a mathematical description of the single-excitation states in terms of superradiant and subradiant states, i.e. separating the fully symmetric state by the remaining antisymmetric ones. Symmetric and subradiant excited states are distinguished by their decay rates, once populated by a classical external laser and observed after that the laser is switched off: the symmetric state has a superradiant decay rate proportional to $N\Gamma$, where $\Gamma$ is the single-atom decay, whereas the antisymmetric states have a decay rate slower than $\Gamma$. 
Once characterized the time evolution of these states, we will apply the criteria of the spin squeezing inequalities introduced by T\'{o}th \cite{toth2007} to detect entanglement in the superradiant and subradiant states. We outline that we limit our study to the linear regime, where the excitation amplitude is proportional to the driving incident electric field. In this linear regime, we must consider
the entanglement criteria which are independent on the value of the driving field, i.e. abandoning these criteria which lead to expressions which depend nonlinearly from the driving field, as it will be discussed in the following.

The paper is organized as follow. In Sec.~\ref{s:model} we present the Hamiltonian describing the dynamics of $N$ two-level atoms interacting with the driving field and write the equation of motion in the linear regime. Then we calculate the decay rate and the transition rates between different elements of the so-called Timed-Dicke basis with its symmetric and antisymmetric states. Section~\ref{s:entanglement} introduces the collective spin operator and the formalism of the spin-squeezing inequalities to assess entanglement. Conclusions are eventually drawn in Sec.~\ref{s:conclusions}.


\section{The model}\label{s:model}

We consider $N\gg 1$ two-level atoms with the same atomic transition frequency $\omega_a$, linewidth $\Gamma$ and dipole $d$ (polarization effects are neglected). The atoms are driven by a monochromatic plane wave with electric field $E_0$, frequency $\omega_0$ and  wave vector $\mathbf{k}_0$, detuned from the atomic transition by $\Delta_0=\omega_0-\omega_a$. If  $|g_j\rangle$ and $|e_j\rangle$are the ground and excited states, respectively, of the $j$-th atom, $j=1,\ldots,N$, which is placed at position $\mathbf{r}_j$ ,
the dynamics of the whole system is described by the effective Hamiltonian  $\hat H=\hat H_0-i\hat H_{\rm eff}$, where \cite{Bienaime2011}
\begin{align}
&\hat{H}_0 =-\hbar\Delta_0\sum_{j=1}^N {\hat\sigma_j}^{\dagger}{\hat\sigma_j}+ \frac{\hbar\Omega_0}{2}\sum_{j=1}^N\left(
    \hat\sigma_je^{-i\mathbf{k}_0\cdot\mathbf{r}_j}
    +{\hat\sigma_j}^{\dagger}e^{i\mathbf{k}_0\cdot\mathbf{r}_j}\right) \label{H0}\\
&\hat H_{\rm eff} =\frac{\hbar\Gamma}{2}\sum_{j,m}G_{jm}\,
    \hat{\sigma}_j^\dagger\hat\sigma_m.\label{Heff}
\end{align}
Here $\Omega_0=dE_0/\hbar$ is the Rabi frequency, $\hat\sigma_j=|g_j\rangle\langle e_j|$ and $\hat\sigma_j^\dagger=|e_j\rangle\langle g_j|$ are the lowering and raising operators and
%\begin{equation}%\label{gammajm}
%    G_{jm}
%    =\frac{e^{ik_0r_{jm}}}{ik_0r_{jm}}=\frac{\sin(k_0r_{jm})}{k_0r_{jm}}-i
%    \frac{\cos(k_0r_{jm})}{k_0r_{jm}}
%    =\Gamma_{jm}-i\Omega_{jm}
%\end{equation}
%for $j\neq m$, whereas  $\Gamma_{jj}=1$ and $\Omega_{jj}=0$ for $j=m$, and $r_{jm}=|\mathbf{r}_j-\mathbf{r}_m|$.
\begin{equation}\label{gammajm}
    G_{jm}=
   	\left\{
    \begin{array}{ll}
    \Gamma_{jm}-i\, \Omega_{jm} & \mbox{if}~j\neq m, \\[1ex]
    1 & \mbox{if}~j = m,
\end{array}
	\right.
\end{equation}
where
\begin{equation}\label{gammajm:bis}
    \Gamma_{jm} = \frac{\sin(k_0r_{jm})}{k_0r_{jm}}\quad \mbox{and} \quad
    \Omega_{jm} = \frac{\cos(k_0r_{jm})}{k_0r_{jm}}.
\end{equation}

In this paper we consider the case in which only one atom among $N$ can be found in the excited states, whereas all the others are in their ground state. Therefore, we have a $(N+1)$-dimensional Hilbert space and a generic state $|\Psi\rangle$ of the our system can be written as \cite{Scully2006,Scully2015}
\begin{equation}\label{psi}
    |\Psi\rangle=\alpha|g\rangle+\sum_{j=1}^N\beta_j e^{i\mathbf{k}_0\cdot\mathbf{r}_j}|j\rangle
\end{equation}
where $|g\rangle=|g_1,\dots,g_N\rangle$ and $|j\rangle=|g_1,\ldots,e_j,\ldots,g_N\rangle$, $j=1,\ldots,N$. 
Starting from the Schr\"{o}dinger equation
\begin{equation}
i\hbar \frac{\partial|\Psi\rangle}{\partial t}=(\hat H_0-i\, \hat H_{\rm eff})|\Psi\rangle.\label{eqS}
\end{equation}
and assuming $\alpha\approx 1$ in the linear regime of weak excitation, we obtain the following equations for the coefficients $\beta_j$ of the state (\ref{psi}):
\begin{align}\label{betaj}
%  \frac{d\beta_j}{dt}
  \dot \beta_j &=
  \left(i\Delta_0-\frac{\Gamma}{2}\right)\beta_j-\frac{i \Omega_0}{2}-\frac{\Gamma}{2}
  \sum_{m\neq j}\widetilde{G}_{jm}\, 
  \beta_m(t), \qquad (j=1,\ldots,N)
\end{align}
where $\dot \beta_j$ refers to the time derivative  of the coefficient $\beta_j$ and $\widetilde{G}_{jm}=G_{jm}e^{-i\mathbf{k}_0\cdot(\mathbf{r}_j-\mathbf{r}_m)}$.

The use of the bare basis $\{|g\rangle,|j\rangle\}$ has the advantage that Eq.~(\ref{betaj}) can be easily solved numerically. However, it does not distinguish between symmetric and anti-symmetric states, that play a relevant role in our investigation. For this reason we introduce the Timed-Dicke (TD) basis with a single excitation \cite{Svidzinsky2008, Scully2010}:
\begin{equation}
\left\{
 |g\rangle,
 |+\rangle_{\mathbf{k}_0},
 |1\rangle_{\mathbf{k}_0},
 \ldots,
 |N-1\rangle_{\mathbf{k}_0}
\right\},
\end{equation}
where $|g\rangle$ has been introduced above, whereas
\begin{equation}\label{STD}
    |+\rangle_{\mathbf{k}_0}=\frac{1}{\sqrt{N}}\sum_{j=1}^N e^{i\mathbf{k}_0\cdot \mathbf{r}_j}
    |j\rangle
\end{equation}
is the symmetric Timed--Dicke (STD) state and
\begin{equation}\label{sk0}
    |s\rangle_{\mathbf{k}_0}=
    \frac{1}{\sqrt{s(s+1)}}
    \left\{
    \sum_{j=1}^{s}
    e^{i\mathbf{k}_0\cdot \mathbf{r}_j}|j\rangle
    - s\, e^{i\mathbf{k}_0\cdot \mathbf{r}_{s+1}}|s+1\rangle
    \right\}, \qquad (s=1,\ldots,N-1)
\end{equation}
are the anti-symmetric ones. The advantage of using the TD basis is that $|s\rangle_{\mathbf{k}_0}$ are collective states involving $s+1$ atoms. 
It is easy to verify that, within the considered Hilbert space, the TD basis is complete, namely:
\begin{equation}\label{unity}
    |g\rangle\langle g|+|+\rangle_{\mathbf{k}_{0}}\langle +|+\sum_{s=1}^{N-1}
    |s\rangle_{\mathbf{k}_{0}}\langle s|=\hat{I}
\end{equation}
and orthonormal, since $\langle g |+\rangle_{\mathbf{k}_0}=\langle g| s\rangle_{\mathbf{k}_0}=
\,_{\mathbf{k}_0}\langle +|s\rangle_{\mathbf{k}_0}=0$, and
$\,_{\mathbf{k}_0}\langle s|s'\rangle_{\mathbf{k}_0}=\delta_{s,s'}$.

To highlight the physical meaning of the TD states, it is useful to evaluate the following transition rates between the basis elements. On the one hand we find
\begin{align}
\,_{\mathbf{k}_{0}}\langle+|\hat H_0|+\rangle_{\mathbf{k}_0} = -\hbar\Delta_0
\qquad \mbox{and} \qquad
\,_{\mathbf{k}_{0}}\langle +|\hat H_{\rm eff}|+\rangle_{\mathbf{k}_0}   =\frac{\hbar\Gamma_+}{2} - i\hbar\, \Omega_+
\end{align}
with $\hat H_0$ and $\hat H_{\rm eff}$ given in Eqs.~(\ref{H0}) and (\ref{Heff}), respectively, and
\begin{align}
  \Gamma_+ &=\Gamma\left(1+\frac{1}{N}
  \sum_{j=1}^N\sum_{m=1\atop (m\neq j)}^N\widetilde\Gamma_{jm}\right), \label{G+}\\[1ex]
  \Omega_+ &= \frac{\Gamma}{N}
  \sum_{j=1}^N\sum_{m=1\atop (m\neq j)}^N\widetilde\Omega_{jm} \label{W+}
\end{align}
where
\begin{equation}
\widetilde\Gamma_{jm} = \Gamma_{jm}\cos[\mathbf{k}_0\cdot(\mathbf{r}_j-\mathbf{r}_m)]
\qquad \mbox{and} \qquad
\widetilde\Omega_{jm} = \Omega_{jm}\cos[\mathbf{k}_0\cdot(\mathbf{r}_{j}-\mathbf{r}_{m})]
\end{equation}
and $ \Gamma_{jm}$ and $\Omega_{jm}$ have been introduced in Eq.~(\ref{gammajm:bis}). 
For a cloud of cold atoms with a Gaussian distribution with parameter $\sigma_r$, one can prove that $\Gamma_+\approx \Gamma(1+b_0/12)$ \cite{Bienaime2010,Bienaime2011}, where $b_0=3N/\sigma^2$ is the resonant optical thickness, with $\sigma=k_0\sigma_r$ .
Thus we can conclude that $|+\rangle_{\mathbf{k}_0}$ is a {\it superradiant} state for very large $b_0$ \cite{Svidzinsky2008,Svidzinsky2010}.

On the other hand, the transition rates for the states $|s\rangle_{\mathbf{k}_0}$, $s=1,\dots,N-1$, read
\begin{align}
\,_{\mathbf{k}_{0}}\langle s|\hat H_0|s\rangle_{\mathbf{k}_0} = -\hbar\Delta_0
\qquad \mbox{and} \qquad
\,_{\mathbf{k}_{0}}\langle s|\hat H_{\rm eff}|s\rangle_{\mathbf{k}_0} = \frac{\hbar\Gamma_s}{2} - i\hbar\, \Omega_s 
\end{align}
with
\begin{align}
 &\Gamma_s =\Gamma\left[
  1+\frac{1}{s+1}\left(
  \frac{1}{s}\sum_{j=1}^s\sum_{m=1\atop(m\neq j)}^s\tilde\Gamma_{jm}
  -2\sum_{j=1}^s
  \tilde\Gamma_{j,s+1}\right)
  \right],\label{Gs}\\[1ex]
  &\Omega_s =\frac{\Gamma}{2(s+1)}\left(
  \frac{1}{s}
  \sum_{j\neq m=1}^s\tilde\Omega_{jm}
  -2\sum_{j=1}^s
  \tilde\Omega_{j,s+1}
  \right).
\end{align}
In this case the decay rates $\Gamma_s$ are less than the single-atom decay rate $\Gamma$ and the states $|s\rangle_{\mathbf{k}_0}$ turn out to be {\it subradiant} \cite{Scully2015}.


Now we focus on the state (\ref{psi}), that in the TD basis reads
\begin{equation}\label{psi:TD}
    |\Psi\rangle=\alpha|g\rangle+
    \beta_{+}|+\rangle_{\mathbf{k}_0}+\sum_{s=1}^{N-1}\gamma_s|s\rangle_{\mathbf{k}_0},
\end{equation}
with
\begin{align}
  \beta_{+} &=  \,_{\mathbf{k}_0}\langle +|\Psi\rangle=\frac{1}{\sqrt{N}}\sum_{j=1}^N \beta_j,\\
  \gamma_s &= \,_{\mathbf{k}_{0}}\langle s|\Psi\rangle=
  \frac{1}{\sqrt{s(s+1)}}
    \left(
    \sum_{j=1}^{s}
    \beta_j-s\,\beta_{s+1}
    \right).
\end{align}
Thanks to the considerations made above about the properties of the TD states, we can easily find the probability to find our state in a superradiant, i.e.~STD, and subradiant state, that is:
\begin{equation}\label{sup:prob}
P_{+} = |\beta_{+}|^2
\end{equation}
and
\begin{equation}\label{sub:prob}
    P_{\rm sub} = \sum_{s=1}^{N-1}|\gamma_s|^2,
\end{equation}
respectively. Moreover, from the normalization of the state $|\Psi\rangle$ it follows that
\begin{align}
\sum_{j=1}^N|\beta_j|^2 &= |\beta_+|^2+\sum_{s=1}^{N-1}|\gamma_s|^2= P_{+} + P_{\rm sub}.\label{sum_beta2}
\end{align}
Finally, the superradiant fraction of excited atoms is given by
\begin{equation}
f_{\rm SR}=\frac{P_{+}}{P_{+} + P_{\rm sub}}
%\frac{|\beta_+|^2}{|\beta_+|^2+\sum_{s=1}^{N-1}|\gamma_s|^2}=\frac{1}{N}\frac{\left|\sum_j\beta_j\right|^2}
%{\sum_j|\beta_j|^2}
= \frac{\left|\overline{\beta}\right|^2}{~\overline{|\beta|^2}~}\label{fSR}
\end{equation}
where we introduced the mean quantities:
\begin{align}
   &\overline{|\beta|^2} = \frac{1}{N} \sum_{j=1}^N|\beta_j|^2\label{ave1}\\
   &\left|\overline{\beta}\right| = \frac{1}{N}\left|\sum_{j=1}^N\beta_j\right|.\label{ave2}
\end{align}
and, in turn, the subradiant fraction is $f_{\rm sub}=1-f_{\rm SR}$.

\section{Entanglement properties of the superradiant and subradiant collective states}\label{s:entanglement}

As the system we are investigating it this paper consists in a large number of atoms, $N\gg 1$, we cannot asses its entanglement properties by individually addressing the single particles. Nevertheless, it is known that spin squeezing can be used to create large scale entanglement \cite{kitagawa1993}. Therefore, here we consider suitable generalized spin squeezing inequalities \cite{toth2007} based only on collective quantities that are accessible and can be measured experimentally.

Given $N$ two-level atoms, we start defining the following collective angular momentum operators:
\begin{equation}\label{Force-operator}
    \hat J_k=\frac{1}{2}\sum_{j=1}^N \hat\sigma_{j}^{(k)},\qquad\qquad
    (k=x,y,z)
\end{equation}
where $\hat \sigma_j^{(k)}$ are the Pauli matrices associated with the $j$-th atom. If we assume  the state given by Eq.~(\ref{psi}), we can calculate the expectation values of the first and second moments of $\hat J_k$ using the result of the previous section (see Appendix~\ref{app:calculations} for the details): 
\begin{eqnarray}
% \langle\hat J_-\rangle&=&\sum_j\langle\tilde\sigma_j\rangle=\alpha^*\sum_j\beta_j\label{J-}\\
% \langle\hat J_+\rangle&=&\sum_j\langle\tilde\sigma_j^\dagger\rangle=\alpha\sum_j\beta_j^*\label{J+}\\ 
\langle\hat J_x\rangle&=\Re{\rm e}[\alpha^*\sum_{j=1}^N\beta_j],\\[1ex]
\langle\hat J_y\rangle&=-\Im{\rm m}[\alpha^*\sum_{j=1}^N\beta_j],\\[1ex]
 \langle\hat J_z\rangle&= - N \left(\displaystyle
 \frac12 - \overline{|\beta|^2}
 \right),
% \frac{1}{2}\sum_j\langle\hat\sigma_j^{(z)}\rangle=-\frac{N}{2}+\sum_j|\beta_j|^2\label{Jz}\\
\end{eqnarray}
and
\begin{align}
    &\langle\hat J_x^2\rangle=\langle\hat J_y^2\rangle=
    \frac{N}{2} \left(
    \frac12 + N \, \left|\overline{\beta}\right|^2 -  \overline{|\beta|^2}
    \right),\\[1ex]
& \langle \hat J_z^2\rangle = N\left[\frac{N}{4}
 - (N-1)\, \overline{|\beta|^2}
\right],
\end{align}
$\beta_j$ being the solutions of Eq.~(\ref{betaj}) and $\overline{|\beta|^2}$ and $\left|\overline{\beta}\right|$ are given in Eqs.~(\ref{ave1}) and (\ref{ave2}), respectively. One can also straightforwardly evaluate the corresponding variances $(\Delta\hat J_k)^2=\langle\hat J_k^2\rangle-\langle\hat J_k\rangle^2$, $k=x,y,z$.

In Ref.~\cite{toth2007}, G.~T\'oth and co-workers proved that a sufficient condition to have entanglement is the violation of suitable inequalities involving the first and second moments of the $\hat J_k$ operators evaluated above. Though they proposed four inequalities, in our case only one of them is relevant to our system, namely:
\begin{equation}
(\Delta\hat J_x)^2+(\Delta\hat J_y)^2+(\Delta\hat J_z)^2 \ge  \frac{N}{2},\label{SS2}
\end{equation}
since the other three:
\begin{subequations}
\begin{align}
\langle\hat J^2_x\rangle+\langle\hat J^2_y\rangle+\langle\hat J^2_z\rangle&\le  \frac{N(N+2)}{4},\label{SS1}\\[1ex]
\langle\hat J^2_k\rangle+\langle\hat J^2_l\rangle-\frac{N}{2}&\le  (N-1)(\Delta\hat J_m)^2,\label{SS3}\\[1ex]
(N-1)\left[(\Delta\hat J_k)^2+(\Delta\hat J_l)^2\right]&\ge  \langle\hat J^2_m\rangle+\frac{N(N-2)}{4}\label{SS4}
\end{align}
\end{subequations}
where $k,l,m$ take all the possible permutations of $x,y,z$, are not useful, as the first and the third are never violated, whereas the second one leads to a condition on the $\beta_j$ going beyond the linear regime assumed to solve Eq.~(\ref{eqS}), as shown in Appendix~\ref{app:ineq}.

The inequality (\ref{SS2}) is more interesting. Its l.h.s.~can be rewritten as a function of the superradiant fraction (\ref{fSR}), that is:
\begin{align}
  (\Delta\hat J_x)^2+(\Delta\hat J_y)^2+(\Delta\hat J_z)^2 &=
  \frac{N}{2} + N^2 \overline{|\beta|^2} \left(N\, |\overline{\beta}|^2 - \overline{|\beta|^2}\right)\\[1ex]
  &=\frac{N}{2} + N^2\left(\overline{|\beta|^2} \right)^2 \left(N\,f_{\rm SR} - 1 \right), \label{7b}
\end{align}
which can be solved within the linear regime assumed throughout the paper.
Hence, the Ineq.~(\ref{SS2}) is violated when the superradiant fraction is $f_{SR}<1/N$, then highlighting the entanglement of the collective atomic state. This will occur when the driving laser is cut and the superradiant component decays faster than the subradiant one, until the subradiant fraction becomes larger than $1-1/N$.

Figures \ref{fig1} and \ref{fig2} present a typical result. We evaluate numerically the values of $\beta_j$ solving the linear equations (\ref{betaj}) for $N=10^3$, $\Delta_0=10\Gamma$ and a Gaussian distribution with $\sigma=k_0\sigma_R=10$. The laser is cut off after $\Gamma t=20$. Fig.\ref{fig1} shows $P=(1/N)\sum_{j}|\beta_j|^2$ vs time. 
Fig.\ref{fig2} shows the superradiant and subradiant fractions, $f_{SR}$ (continuous blue line) and $f_{sub}$ (dashed red line), as defined by Eq.(\ref{fSR}). The dotted black line is the value $1/N$. We observe that when the laser is on, the subradiant fraction is only about the $3\%$ of the total excitation. As soon as the laser is cut off, the superradiant fraction decays fast and the subradiant fraction increases, becoming dominating for $\Gamma t> 20.5$. The atoms become entangled when $f_{SR}<1/N$, at time larger than $\Gamma t>24$.
    % Figure environment removed 
        % Figure environment removed 

\section{Conclusions}\label{s:conclusions}

We have discussed the symmetry properties of  the state of $N$ two-level atoms with positions $\mathbf{r}_j$ (with $j=1,\dots,N$), where only one is excited among $N$. The single-excitation Hilbert space can be spanned by a completely  symmetric, state (the 'Symmetric Timed Dicke state') and $N-1$ asymmetric ones, where the first has a superradiant decay rate proportional to $N$, while the others have a subradiant decay rates less than the single-atom value. Remarkably, the projection on the symmetric and asymmetric states allows to calculate the superradiant and subradiant fractions of the ensemble. 

One open question is whether the atoms in the superradiant and subradiant fraction are entangled or not. To address this problem, we studied the relevant case of an ensemble of $N$ atoms driven by an external, uniform laser. In the framework of the linear regime, the probability amplitude of excitation is proportional to the incident electric field, and the superradiant fraction is largely dominant, with only a small fraction of atoms in the subradiant states. However, when the laser is cut off, the superradiant fraction rapidly decays to zero, leaving only the subradiant one, as it has been observed in the experiments of Ref.\cite{Guerin2016}. 

In order to investigate the entanglement properties of the atomic ensemble, we exploited the spin squeezing inequalities introduced by G. T\'{o}th and cowokers \cite{toth2007}, based on the first and the second order moments of collective spin operators. We have found that one of these inequalities is violated when the superradiant fraction decreases below the value $1/N$. Therefore, to have entanglement the probability of finding $N$ atoms in the superradiant state must be less than the average probability per atom to be in the excited state. Conversely, when the superradiant fraction is larger than $1/N$ no entanglement can be detected by measuring the moments of the collective spin operators.
 
\appendix
\section[\appendixname~\thesection]{Calculation of the first and second moments of $\hat J_k$}\label{app:calculations}
%\subsection[\appendixname~\thesubsection]{}
In this appendix we explicitly calculate the first and second moments of the collective angular momentum operators used in the text, namely:
\begin{equation}
    \hat J_k=\frac{1}{2}\sum_{j=1}^N \hat\sigma_{j}^{(k)},\qquad\qquad
    (k=x,y,z)
\end{equation}
where $\hat \sigma_j^{(k)}$ are the Pauli matrices associated with the $j$-th atom. It is useful introducing the  identities $\hat\sigma_j^\dagger=\left(\hat\sigma_j^{(x)}+
i\hat\sigma_j^{(y)}\right)/2$ and $\hat\sigma_j=\left(\hat\sigma_j^{(x)}-
i\hat\sigma_j^{(y)}\right)/2$, that are the raising and lowering operators appearing in Eqs.~(\ref{H0}) and (\ref{Heff}), and $\sigma_{j}^{(z)}=|e_j\rangle\langle e_j|-|g_j\rangle\langle g_j|$.
We define
\begin{eqnarray}
\tilde\sigma_j^\dagger = e^{i\mathbf{k}_0\cdot \mathbf{r}_j}\hat\sigma_j^\dagger\,,\qquad
\tilde\sigma_j = e^{-i\mathbf{k}_0\cdot \mathbf{r}_j}\hat\sigma_j
\end{eqnarray}
and the collective operators:
\begin{eqnarray}
    \hat J_-=\sum_{j=1}^N \tilde\sigma_j\,,\quad
    \hat J_+=\sum_{j=1}^N \tilde\sigma_j^\dagger\quad\mbox{and}\quad
    \hat J_z=\frac{1}{2}\sum_{j=1}^N \hat\sigma_j^{(z)}\label{Jop}
\end{eqnarray}
with commutation relations:
\begin{eqnarray}
    [\hat J_+,\hat J_-]=2\hat J_z,\qquad
    [\hat J_z,\hat J_\pm]=\pm 2\hat J_\pm.
    \label{Jcom}
\end{eqnarray}
We can now apply the formalism of Ref.~\cite{toth2007}, calculating the first and second order moments
of the collective operators $\hat J_x=(\hat J_+ +\hat J_-)/2$ and $\hat J_y=(\hat J_+ -\hat J_-)/(2i)$.
Explicitly,
\begin{eqnarray}
    \langle\hat J_x\rangle&=&\frac{1}{2}\left\{ \langle\hat J_+\rangle+ \langle\hat J_-\rangle\right\},\\
    \langle\hat J_y\rangle&=&\frac{1}{2i}\left\{ \langle\hat J_+\rangle- \langle\hat J_-\rangle\right\},\\
    \langle\hat J_x^2\rangle&=&\frac{1}{4}\left\{\langle\hat J_+\hat J_-\rangle+ \langle\hat J_-\hat J_+\rangle
    +\langle\hat J_+^2\rangle+\langle\hat J_-^2\rangle\right\},\\
    \langle\hat J_y^2\rangle&=&\frac{1}{4}\left\{\langle\hat J_+\hat J_-\rangle+ \langle\hat J_-\hat J_+\rangle
    -\langle\hat J_+^2\rangle-\langle\hat J_-^2\rangle\right\},
\end{eqnarray}
where the expectations are evaluated using the $\beta_j$ as evaluated from the Eq.~(\ref{betaj}) (see Sec.~\ref{s:model}). Since:
\begin{align}
    \tilde\sigma_j|\Psi\rangle &=\beta_j|g\rangle\label{s1},\\[1ex]
    \tilde\sigma_j^\dagger|\Psi\rangle &= \alpha e^{i\mathbf{k}_0\cdot \mathbf{r}_j}|j\rangle\label{s2},\\[1ex]
    \hat\sigma_j^{(z)}|\Psi\rangle &=-\alpha |g\rangle+\beta_j e^{i\mathbf{k}_0\cdot \mathbf{r}_j}|j\rangle
    -\sum_{m=1}^N(1-\delta_{jm})\beta_m e^{i\mathbf{k}_0\cdot \mathbf{r}_m}|m\rangle\nonumber\\
    &=-|\Psi\rangle+2\beta_j
    e^{i\mathbf{k}_0\cdot \mathbf{r}_j}|j\rangle
    \label{s3},
\end{align}
we derive
\begin{align}
 \langle\hat J_-\rangle&=\sum_{j=1}^N\langle\tilde\sigma_j\rangle=\alpha^*\sum_{j=1}^N\beta_j\label{J-},\\[1ex]
 \langle\hat J_+\rangle&=\sum_{j=1}^N\langle\tilde\sigma_j^\dagger\rangle=\alpha\sum_{j=1}^N\beta_j^*\label{J+},\\[1ex] 
 \langle\hat J_z\rangle&=\frac{1}{2}\sum_j\langle\hat\sigma_j^{(z)}\rangle=-\frac{N}{2}+\sum_j|\beta_j|^2\label{Jz},\\[1ex]
    \langle\hat J_x\rangle&=\Re\textrm{e}\left[\alpha^*\sum_{j=1}^N\beta_j\right],\\[1ex]
    \langle\hat J_y\rangle&=-\Im\textrm{m}\left[\alpha^*\sum_{j=1}^N\beta_j\right].
\end{align}
Recalling that the system has a single excitation, we have $\tilde\sigma_j\tilde\sigma_m^\dagger|\Psi\rangle=0$ and $\tilde\sigma_j^\dagger\tilde\sigma_m^\dagger|\Psi\rangle=0$, and we get:
\begin{equation}\label{J2}
    \langle \hat J_-^2\rangle = \langle \hat J_+^2\rangle=0.
\end{equation}
Furthermore, 
\begin{align}
    \langle \hat J_+\hat J_-\rangle&=\sum_{j,m}\langle\tilde\sigma_j^\dagger\hat\sigma_j\rangle\nonumber\\
    &=\sum_{j,m}\beta_j^*\beta_m=
    \left|\sum_{j=1}^N\beta_j\right|^2 \label{J+-}
\end{align}
and, using the commutation rule (\ref{Jcom}) and Eqs.~(\ref{Jz}) and (\ref{J+-}), we find
\begin{align}
    \langle \hat J_-\hat J_+\rangle&=\langle \hat J_+\hat J_-\rangle-2\langle \hat J_z\rangle\nonumber\\[1ex]
    &=N+
    \left|\sum_{j=1}^N\beta_j\right|^2-2\sum_{j=1}^N|\beta_j|^2.\label{J-+}
\end{align}
From these results, we can calculate the expectations of $\langle\hat J_k^2\rangle$, $k=x,y,z$, namely:
\begin{align}
    \langle\hat J_x^2\rangle&=\langle\hat J_y^2\rangle=
    \frac{N}{4}+\frac{1}{2}\left|\sum_{j=1}^N\beta_j\right|^2-\frac{1}{2}\sum_{j=1}^N|\beta_j|^2,\\[1ex]
    \langle \hat J_z^2\rangle&=\frac{N^2}{4}-(N-1)\sum_{j=1}^N|\beta_j|^2.\label{z2}
\end{align}
It is remarkable that the second order moments do not depend on $\alpha$, which appears only in the expression for $\langle\hat J_{x,y}\rangle$.

We are now ready to evaluate also the variances of the collective angular momentum operators.
Defining $(\Delta\hat A)^2=\langle\hat A^2\rangle-\langle\hat A\rangle^2$, we have:
\begin{align}
    (\Delta\hat J_x)^2&=\frac{N}{4}+\frac{1}{2}\left|\sum_{j=1}^N\beta_j\right|^2-\frac{1}{2}\sum_{j=1}^N|\beta_j|^2
    -\left\{\Re\textrm{e}\left[\alpha^*\sum_{j=1}^N\beta_j\right]\right\}^2\label{varx}\\[1ex]
    (\Delta\hat J_y)^2&=\frac{N}{4}+\frac{1}{2}\left|\sum_{j=1}^N\beta_j\right|^2-\frac{1}{2}\sum_{j=1}^N|\beta_j|^2
    -\left\{\Im\textrm{m}\left[\alpha^*\sum_{j=1}^N\beta_j\right]\right\}^2\label{vary}\\[1ex]
    (\Delta\hat J_z)^2&=\left(\sum_{j=1}^N|\beta_j|^2\right)\left(
    1-\sum_{j=1}^N|\beta_j|^2\right).
    \label{varz}
\end{align}
Notice that:
\begin{align}
    (\Delta\hat J_x)^2+(\Delta\hat J_y)^2 = \frac{N}{2}-\left(\sum_{j=1}^N|\beta_j|^2\right)\left(1-\left|\sum_{j=1}^N\beta_j\right|^2\right)
    \label{varxy}
\end{align}
which is independent of $\alpha$.

All the previous results can be simplified introducing the mean quantities $\overline{|\beta|^2}$ and  $\left|\overline{\beta}\right|$ given in Eqs.~(\ref{ave1}) and (\ref{ave2}), respectively. The corresponding expressions are directly reported in Sec.~\ref{s:entanglement}.

\section[\appendixname~\thesection]{Considerations on Ineqs.~(\ref{SS3}) and (\ref{SS4})}\label{app:ineq}
%\subsection[\appendixname~\thesubsection]{}
In this appendix we show that, in our case, Ineqs.~(\ref{SS1}), (\ref{SS3}) and (\ref{SS4}) cannot lead to useful conclusions.

In our case the l.h.s.~of the Ineq.~(\ref{SS1}) reads:
\begin{equation}\label{7a}
   \langle \hat J_x^2\rangle+\langle \hat J_y^2\rangle+\langle \hat J_z^2\rangle = 
    \frac{N(N+2)}{4} - N^2\sigma_\beta^2
\end{equation}
where we defined the particle variance
\begin{align}\label{sigma}
    \sigma_\beta^2&=\overline{|\beta|^2}-\left|\overline{\beta}\right|^2,\\[1ex]
     &=\frac{1}{N}\sum_{s=1}^{N-1}|\gamma_s|^2.
\end{align}
As one may expect, the Ineq.~(\ref{SS1}) is true and the equality sign holds for all the symmetric states, with $\sigma_\beta=0$ or, equivalently, $f_{\rm SR} = 1$.

Concerning the Ineq.~(\ref{SS3}), if we study the following expression (similar results can be obtained for the other combination of the involved expectations):
\begin{equation}\label{7c}
    \langle \hat J_x^2\rangle+\langle \hat J_y^2\rangle-\frac{N}{2}\le
    (N-1)(\Delta\hat J_z)^2,
\end{equation}
we derive:
\begin{align}\label{7c3}
  \left|  \overline{\beta} \right|^2 &\le \overline{|\beta|^2} \left[
    1-(N-1)\overline{|\beta|^2} \right] %\\[1ex]
\end{align}
or, equivalently:
\begin{equation}\label{7c4}
\sigma_\beta\ge\sqrt{N-1}\left(|\overline{\beta}|^2+\sigma_\beta^2\right),
\end{equation}
where $\sigma_\beta^2=\overline{|\beta|^2}-\left|\overline{\beta}\right|^2$ is the particle variance and $\overline{|\beta|^2}$ and $\left|\overline{\beta}\right|$ given in Eqs.~(\ref{ave1}) and (\ref{ave2}), respectively. We can clearly see that these inequalities are nonlinear in $\overline{|\beta|^2}$. Recalling Eqs.~(\ref{ave1}) and (\ref{ave2}), we conclude that they cannot lead to useful conditions, since  the solution of (\ref{7c3}) or (\ref{7c4}) depends on the $\beta_j^2$ and, thus on $\Omega_0^2$, whereas the $\beta_j$ have been obtained in the framework of the linear regime with respect $\Omega_0$, see Eq.~(\ref{betaj}).

Ineq.~(\ref{SS4}) is never violated for our system. In fact, for instance, we have:
\begin{eqnarray}
(N-1)[(\Delta\hat J_x)^2+(\Delta\hat J_y)^2] \ge \langle\hat J^2_z\rangle+\frac{N(N-2)}{4}
\end{eqnarray}
which, using Eqs.(\ref{varxy}) and (\ref{z2}), yields
\begin{align}
N^2
\overline{|\beta|^2}\left|\overline{\beta}\right|^2 > 0.
\end{align}


\bibliography{Bibliography}

\end{document}


