\section{Introduction} \label{sec:intro}

Mathematical models for the description of fluids, plasmas, and many other physical phenomena, are typically given in terms of systems of hyperbolic conservation laws. These systems are characterized by a set of multi-dimensional partial differential equations (PDEs) that express the conservation of various physical quantities in terms of their respective fluxes. A prototypical example is provided by the Euler's equations for gas-dynamics describing the conservation of mass, momentum, and energy of a gas.

In this work we focus on one-dimensional systems of $m>1$ hyperbolic conservation laws:
\begin{equation} \label{eq:hyp:sys}
	\frac{\partial}{\partial t} \mathbf{u}(x,t) + \frac{\partial}{\partial x} \mathbf{f}(\mathbf{u}(x,t)) = \mathbf{0},
\end{equation}
where, $\mathbf{u}: \mathbb{R} \times \mathbb{R}^+_0 \to \mathbb{R}^m$ is the quantity of interest, and $\mathbf{f}: \mathbb{R}^m \to \mathbb{R}^m$ is the vector of the flux functions. System~\eqref{eq:hyp:sys} is hyperbolic when the eigenvalues $\{\lambda_j(\mathbf{u}(x,t))\}_{j=1}^m$ of the associated Jacobian matrix are real and determine a complete set of eigenvectors. The eigenvalues of~\eqref{eq:hyp:sys} provide the characteristic speeds, which describe the speed propagation of waves in the system. These waves can be either acoustic waves (shocks and rarefactions) or material waves (contact discontinuities). Requiring that the eigenvalues are real implies that the propagation of information through the system is finite.

Solving hyperbolic systems of conservation laws is a challenging task, both analytically and numerically, e.g.~due to the occurrence of singularities or the need of devising high order accurate non-oscillatory methods to avoid low-resolution approximations. Another source of numerical difficulty is represented by stiff problems that occur when the system is characterized by speeds spanning different orders of magnitude, namely when $\frac{\max_{j=1,\dots,m} \lambda_j(\mathbf{u})}{\min_{j=1,\dots,m} \lambda_j(\mathbf{u})} \gg 1$. This happens, for instance, in gas-dynamics when the material velocity, i.e.~the contact speed, is much less than the speed of the acoustic waves. In many applications the phenomenon of interest travels with a slow speed. An example is provided by low-Mach number problems occurring when the equations governing the flow become stiff due to the very low fluid velocity compared to the speed of sound in the fluid. In these situations, the compressibility effects of the fluid can be neglected, and the fluid can be assumed close to incompressibility. Then, if the interest is on the movement of the fluid, accuracy in the propagation of sound becomes irrelevant. For low-Mach problems we refer to~\cite{2010DellacherieLowMach,2011DegondTang,2017AbbateAllSpeed,2017DimarcoLoubere_LowMachAP,2018BoscarinoRussoScandurra,2017Tavelli_SemiImplicitAllMach}.

Numerical schemes used to solve hyperbolic problems need to be carefully designed to handle the stiff regime. In fact, it is well-known that explicit schemes are subject to the Courant-Friedrichs-Levy (CFL) stability condition that specifies a constraint on the numerical speed in relation to the maximum speed of information propagating in the system, in order to ensure that the numerical solution remains stable. More precisely, the CFL condition imposes that the numerical speed must be faster than the maximum speed of propagation to ensure that information does not travel too far between adjacent space cells during one time-step. For this reason, the stability request on the time-step of explicit schemes becomes very restrictive for stiff problems due to the presence of fast waves, thus limiting the computational efficiency of the scheme. In contrast, implicit schemes are not constrained by the CFL condition and can be employed with a time-step focusing on the phenomenon of interest, for instance, on the fluid speed, thus potentially allowing for larger time-step sizes. In addition, since the accuracy of a scheme depends on the difference between the numerical and the actual speed, it turns out that implicit schemes reduce accuracy on the (fast) acoustic waves, still resulting high accurate on the (slow) material waves. However, implicit methods are more computationally expensive than explicit ones since they require the solution of a system of equations, in general nonlinear, at each time-step. Therefore, the choice between explicit and implicit schemes depends on the specific problem being solved and the desired trade-off between computational efficiency and accuracy.

Here, we deal with an efficient formulation of implicit high order finite volume schemes~\cite{LeVeque:book}. In first order implicit schemes the only source of nonlinearity is due to the nonlinear flux function, namely to the physical structure of the model, which is therefore unavoidable; however they produce large dissipation errors. High order accurate implicit schemes require, as much as their explicit counterparts, nonlinear space-limiting procedures to prevent spurious oscillations (see e.g. \cite{PSV23:Quinpi} for a discussion of the TVD property of implicit schemes). Such space-limiting procedures introduce an additional source of nonlinearity which becomes computationally challenging when using implicit schemes. The novel idea of~\cite{PSV23:Quinpi} was to simplify considerably the implicit high order scheme by using a first order predictor to freeze the non-linearities of the space-limiting procedure. The implicit approach proposed in~\cite{PSV23:Quinpi}, named \emph{Quinpi}, was tailored to the third order implicit approximation of \emph{scalar} conservation laws. More specifically, third order accuracy was achieved by using a third order Diagonally Implicit Runge-Kutta (DIRK) for the time integration and a third order Central Weighted Essentially Non-Oscillatory (CWENO) reconstruction, cf.~\cite{CPSV:cweno}, for the space discretization. The first order implicit scheme was based on a composite backward Euler, evaluated at the abscissae of the DIRK, naturally combined with a piecewise constant (i.e.~linear in the data) reconstruction in space. This predictor was used to freeze the nonlinear weights of the CWENO reconstruction making the resulting third order implicit scheme nonlinear just because of the nonlinearity of the flux function.

As noted in~\cite{2020Arbogast,PSV23:Quinpi}, the appearance of spurious oscillations can be observed in implicit integration, especially for large Courant numbers, despite space-limiting being performed. A time-limiting procedure is required in order to control the oscillations arising in the implicit time integration, i.e.~when the large time-step size allows for propagation of waves crossing several adjacent space cells during a single time-step. The problem of the time-limiting has been discussed in several papers, which typically deal with limiting in space
and time simultaneously. For instance, we mention~\cite{2003DurasaisamyBaeder,2007DurasaisamyBaeder} for second-order schemes and~\cite{2020Arbogast} for a fully nonlinear third order implicit scheme. In~\cite{PSV23:Quinpi}, instead, the time-limiting was obtained as a cell-based a-posteriori nonlinear blending of the first order and third order solutions. Troubled cells were detected using a combination of space and time regularity indicators, in a WENO-like fashion. However, being based on the cells, the technique in~\cite{PSV23:Quinpi} has the drawback of being non-conservative, and, thus, it requires a conservative correction. Conservative flux-based time-limiting procedures, inspired by the Multi-dimensional Optimal Order Detection (MOOD) method introduced in~\cite{CDL11:MOOD,CDL12:MOOD}, were proposed in~\cite{VTSP23:Quinpi:Book,EBS22:Implicit:Networks} for the control of nonphysical oscillations of high order implicit numerical solutions. MOOD was originally developed in order to reduce the order of the space reconstruction on problematic cells with the help of several problem-dependent detectors to check whether extrema of the numerical solution are smooth, physical or spurious oscillations. MOOD was also extended to other contexts, as for instance in~\cite{SL:18:AMRMOOD,LDD:14,ZDLS:14}.

The contribution of the present work is twofold:
\begin{enumerate}
	\item We extend the Quinpi framework proposed in~\cite{PSV23:Quinpi} to the case of the numerical approximation of general stiff hyperbolic systems when the interest is on the movement of the fluid. This implicit approach is not tailored to the solution of a specific stiff problem, as it usually happens when devising numerical schemes for low-Mach problems. Moreover, the space-limiting exploits the novel CWENO reconstruction~\cite{STP23:cweno:boundary} which does not make use of ghost cells for boundary reconstructions;
	\item We introduce a flux-based conservative time-limiting procedure inspired by the MOOD technique, namely we replace the high order numerical fluxes at the cell interfaces of troubled cells with low order numerical fluxes. A crucial point is the choice of the troubling cells indicator; here we investigate the use of the numerical entropy production error~\cite{PS11:numerical:entropy,Puppo04:numerical:entropy} as indicator that signals non-smooth solutions instead of the typical MOOD detectors of oscillatory cells. 
\end{enumerate}

Finally, we mention the following approaches to the implicit integration of hyperbolic systems developed in the literature. A fifth order implicit WENO scheme was proposed in~\cite{2001Gottlieb}, where a predictor-corrector technique was also used. However, the predictor was based on an explicit first order scheme and therefore it allows to deal with small Courant numbers only. A fully nonlinear implicit scheme, based on a third order RADAU time integrator and a third order WENO reconstruction, can be found in~\cite{2020Arbogast}. Fully implicit, semi-implicit, implicit-explicit, local time-stepping and active flux treatments of stiff hyperbolic equations were also investigated, e.g., in~\cite{CNPT:09,CNPT:10,CNPT:10a,CPPT:06,FZ:22,FKRZ:22,ABIR:19,BB:23}.

The paper is organized as follows. In Section~\ref{sec:scheme} we introduce the fully implicit third order scheme of system~\eqref{eq:hyp:sys} which uses the CWENO reconstruction of~\cite{STP23:cweno:boundary} for the space approximation and a DIRK method for the time integration. Then, in Section~\ref{sec:quinpi}, we devise the Quinpi framework, which allows to overcome the nonlinearity of CWENO by freezing the nonlinear weights with a first order composite backward Euler approximation of the solution. The limiting in time is discussed, which employs a flux-based MOOD technique combined with the numerical entropy production error as detector of troubling cells. Numerical experiments focused on the Euler's equations for gas-dynamics are provided in Section~\ref{sec:numerics}, where we investigate the experimental order of convergence of the Quinpi scheme and its performance on stiff problems. Finally, we discuss results and perspectives in Section~\ref{sec:conclusion}.