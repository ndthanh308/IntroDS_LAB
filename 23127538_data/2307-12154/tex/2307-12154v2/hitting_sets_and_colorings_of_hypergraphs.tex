%\addtolength{\hoffset}{-0.5cm}
\documentclass[11pt,oneside]{amsart}


%\PassOptionsToPackage{defaults=prettiest, hyphenation=huhyphn, hyphenmins=22}{magyar.ldf}



\usepackage[utf8]{inputenc}
\usepackage[T1]{fontenc}
\usepackage{lmodern}
\usepackage[margin = 3cm]{geometry}
\usepackage{array}
\usepackage{fancyhdr}
\usepackage{amsmath}
\usepackage{amsfonts}
\usepackage{amssymb}
\usepackage{graphicx}
\usepackage{float}
\usepackage{caption}
\usepackage{pgf,tikz}
\usetikzlibrary{arrows}
\usepackage{tikzlings-addons}
\usepackage{tikzlings-penguins}[2021/08/06 version v0.8 Draw penguins in TikZ]
\usetikzlibrary{svg.path}
\captionsetup[figure]{font=small}
\usepackage{subcaption}
\usepackage{parskip}
\setlength{\parindent}{0.75 cm}
\setlength{\parskip}{0 cm}
\usepackage[export]{adjustbox}
%\renewcommand{\figurename}{ábra}
\renewcommand{\baselinestretch}{1.5} 
\usepackage{verbatim}
\usepackage{hyperref}

\definecolor{ffttww}{rgb}{1,0.2,0.4}
\definecolor{uuuuuu}{rgb}{0.27,0.27,0.27}
\definecolor{cqcqcq}{rgb}{0.75,0.75,0.75}
\definecolor{wwqqcc}{rgb}{0.4,0,0.8}

\newcommand{\R}{\mathbb{R}}
\newcommand{\N}{\mathbb{N}}
\newcommand{\HH}{\mathcal{H}}
\newcommand{\RR}{\mathcal{R}}
\newcommand{\A}{\mathcal{A}}
\newcommand{\B}{\mathcal{B}}
\newcommand{\est}{\varnothing}

\usepackage{tikzlings}
\usepackage{tikzducks}
\newtheorem{theorem}{Theorem}
\newtheorem{proposition}{Proposition}
\newtheorem{corollary}{Corollary}
\newtheorem{lemma}{Lemma}
\newtheorem{definition}{Definition}
\newtheorem{jel}{Notation}
\newtheorem{conj}{Conjecture}

\let\svthefootnote\thefootnote
\newcommand\freefootnote[1]{%
  \let\thefootnote\relax%
  \footnotetext{#1}%
  \let\thefootnote\svthefootnote%
}

\title[Hitting sets and colorings of hypergraphs]%
  {Hitting sets and colorings of hypergraphs}

%\author{Balázs Bursics, Bence Csonka, Luca Szepessy}

\author{Bal\'azs Bursics}
\email{bursicsb@student.elte.hu \vspace*{-0.5cm}}
\address{Eötvös Loránd University Faculty of Science, Pázmány Péter sétány 1/A, H-1117 Budapest, Hungary}

\author{Bence Csonka}
\email{csonkab@edu.bme.hu \vspace*{-0.5cm}}
\address{Budapest University of Technology and Economics, M\H{u}egyetem rkp. 3., H-1111 Budapest, Hungary}

\author{Luca Szepessy}
\email{szepessyluca@student.elte.hu \vspace*{-0.5cm}}
\address{Eötvös Loránd University Faculty of Science, Pázmány Péter sétány 1/A, H-1117 Budapest, Hungary}


\begin{document}




\maketitle

\thispagestyle{empty}




\begin{abstract}	

In this paper we study the minimal size of edges in hypergraph families that guarantees the existence of a polychromatic coloring, that is, a $k$-coloring of a vertex set such that every hyperedge contains a vertex of all $k$ color classes. We also investigate the connection of this problem with $c$-shallow hitting sets: sets of vertices that intersect each hyperedge in at least one and at most $c$ vertices.

We determine for some hypergraph families the minimal $c$ for which a $c$-shallow hitting set exists.

We also study this problem for a special hypergraph family, which is induced by arithmetic progressions with a difference from a given set. We show connections between some geometric hypergraph families and the latter, and prove relations between the set of differences and polychromatic colorability.




\end{abstract}


\section{Introduction}

%\subsection{Polychromatic $k$-coloring of geometric hypergraphs}

\freefootnote{MSC2020: 05C15, 05C65. Key words and phrases: geometric hypergraphs, polychromatic coloring, shallow hitting
sets, arithmetic progressions}

%A {\it hypergraph} is a couple $H=(V,\mathcal{E})$ with {\it vertex set} $V$ and {\it set of hyperedges } $\mathcal{E}\subseteq\mathcal{P}(V).$
A {\it polychromatic k-coloring} of a hypergraph $H$ is a $k$-coloring of its vertex set such that every hyperedge contains a vertex of all $k$ color classes. Observe that a polychromatic 2-coloring is the same as the usual proper 2-coloring of hypergraphs, where we require that no edge is monochromatic. By merging $j$ color classes of a polychromatic $k$-coloring we get a polychromatic $(k-j+1)$-coloring, so the condition of monochromatic $k$-colorability becomes stricter as $k$ increases. A trivial necessary condition for the existence of a polychromatic $k$-coloring is that all edges of $H$ must be of size at least $k$.


\begin{definition} Denote by $H_{\ge m}$ or simply $H_m$ the hypergraph obtained from $H$ by deleting all hyperedges of size smaller than $m$, and denote by $H_{=m}$ the hypergraph consisting of the hyperedges of $H$ with size exactly $m$. Similarly, for a hypergraph family $\mathcal{H}$ let $\HH_m=\HH_{\ge m}=\{H_{\ge m}:H\in\HH\}$ and $\HH_{=m}=\{H_{=m}:H\in\HH\}$.
\end{definition}

One can make statements about the existence of polychromatic $k$-colorings for every member of a hypergraph family $\HH$ using the following parameter:

\begin{definition} Let $\mathcal{H}$ be a hypergraph family. Denote by $m_\mathcal{H}(k)$ the smallest positive integer $m$ such that for every $H\in\mathcal{H}$ there exists a polychromatic $k$-coloring of the hypergraph $H_{\ge m}$. If there is no such $m$, set $m_{\mathcal{H}}(k)=\infty$.
\end{definition}

Determining or bounding $m_{\mathcal{H}}(k)$ is an interesting problem in itself for some hypergraph families. {\it Range capturing hypergraph families} are particularly well studied: Suppose $\mathcal{S}$ is a family of planar (or higher dimensional) sets, called the {\it range family}, consider the family of hypergraphs whose members have a finite vertex set $V$, and whose edge set consists of all subsets $e\subseteq V$ such that $e=V\cap S$ for some $S\in\mathcal{S}$, in which case we say that the range $S$ captures $e$. Much research has been done on polychromatic colorings and the parameter $m(k)$ of such hypergraph families, for example, when the range family consists of halfplanes \cite{felsik}, translates of a polygon \cite{varadarajan}, translates of a convex body \cite{damasdi2021three}, homothets of a polygon \cite{damasdi2022realizing}, translates of an octant \cite{ternyolcad}, axis-parallel rectangles \cite{teglalap}, or axis-parallel strips \cite{apstrip}. For a comprehensive summary, see the website \cite{cogezoo} maintained by Keszegh and Pálvölgyi.

Investigating this problem is also motivated by the connection between polychromatic colorings and the cover-decomposability of planar sets. Suppose that we have a planar polygon $P$, and some translates of $P$ are given in such a way that every point of the plane is covered at least $m$ times. A natural question is whether this cover can be decomposed into two sets such that both set of translates of $P$ covers the whole plane in itself. The method of dualization can be used to reduce this to the problem of polychromatic colorability where the vertex set is the set of centers of gravity of the translate polygons, and the hyperedges are the sets contained in any translate of $P$. This is described in detail in \cite{dekompozicio}, for some related results see e.g. \cite{pach2016unsplittable, palvolgyi2010convex, palvolgyi2010indecomposable, varadarajan, keszegh2012octants, kovacs2015indecomposable}.

Now we turn to shallow hitting sets, and their role in constructing polychromatic colorings under suitable conditions.


\begin{definition}
 Let $H=(V,E)$ be a hypergraph, an $U\subset V$ vertex set is a {\it c-shallow hitting set} for a positive integer $c$ if for every hyperedge $e\in E$: $1\le|e\cap U|\le c.$
\end{definition} 

\begin{definition}
 Let $H=(V,E)$ be a hypergraph, and $V'\subset V$. Define $E'=\{e'\subset V'| \exists e\in E: e'=e\cap E'\}$, we say that $E'=(V',E')$ is the induced subhypergraph of $H$ on $V'$.
\end{definition}


Suppose that $\HH$ is a hypergraph family such that for arbitrary $H=(V,E)\in\HH$ and $V'\subset V$ the induced subhypergraph $H'$ is a member of $\HH$. Then $c$-shallow hitting sets can be used to create polychromatic colorings, see e.g. \cite{felsik, pshalfplane}.

In these papers $c$-shallow hitting sets are applied through the implicit use of the following lemma, which we restate here (and which also follows from \cite[Theorem 2.11.]{pshalfplane}):

\begin{lemma}\label{prop1}
Suppose that the hypergraph family $\mathcal{H}$ satisfies the condition that for arbitrary $H=(V,E)\in\HH$ and $V'\subset V$, the induced subhypergraph of $H$ on $V'$ is an element of $\HH$, and that every hyperedge $e\in E$ contains a vertex $v\in e$ such that for $e'=e\setminus \{v\}$ and $E'=E\setminus \{e\}\cup\{e'\}$ we have $H'=(V,E')\in\HH.$
%replacing $e$ with $e'=e\setminus v$ results in an element of $\HH$. 
Suppose that for every $m\ge c$ and every element of $\mathcal{H}_{=m}$ there exists a $c$-shallow hitting set. Then $m_\mathcal{H}(k) \leq c \cdot (k-1) +1$.
\end{lemma}

For range capturing hypergraph families, the first condition of the above statement -- that taking an induced subhypergraph does not lead out of the family -- suffices automatically.

On the other hand, the existence of $c$-shallow hitting sets is not so evident, for example it is a frequent case for geometric hypergraph families that a hyperedge contains more than $c$ pairwise disjoint hyperedges. 
Thus we need to restrict to a {\it Sperner} subfamily of the original hypergraph family (that is, where no two edges are contained one in another), to have a chance for the existence of $c$-shallow hitting sets. However, in some cases, e.g., hypergraphs induced by bottomless rectangles (see below), this is not strong enough either \cite{pshalfplane}. 

Fortunately, Lemma \ref{prop1}. is about a special type of Sperner subfamily, the $m$-uniform members of the family, which gives even better chances to obtain $c$-shallow hitting sets. Therefore, for a given hypergraph family $\HH$, it is natural to ask whether there is a positive integer $c$ such that there exists a $c$-shallow hitting set for all $H_{=m}\in \HH_{=m}$.

%But if the first condition applies and we restrict to the {\it Sperner} members of the family (being the hypergraphs with no two edges contained one in another); or even more specifically the uniform hypergraphs, because for any member of $\HH$ we can replace edges by smaller ones to get a Sperner or uniform member of the family while the condition of polychromatic colorability only gets stronger. The Sperner condition is not strong enough in many cases, so we mostly study the uniform case.



%The question arises that for a given hypergraph family $\HH$ does there exists a positive integer $c$ such that there exist a $c$-shallow hitting set for all $H_{=m}\in \HH_{=m}$?

Since the resulting bound is linear in $k$, Lemma \ref{prop1}. is also loosely related to an important conjecture of the field:

\begin{conj}[\cite{dekompozicio}]
If $m_\HH(2)<\infty$ for a hypergraph family $\HH$, then $m_\HH(k)=O(k)$.
\end{conj}

It would also be interesting to see whether Lemma \ref{prop1}. is reversible, more precisely, does an upper bound of $m_\mathcal{H}(k)$, which is linear in $k$, guarantee a constant $c$ independent from $k$ such that there exists a $c$-shallow hitting set for every member of $\HH_{=m}$?      



\subsection{Bottomless rectangles}

\begin{definition} Denote by $\mathcal{B}$ the hypergraph family which consists of those hypergraphs $H=(V,E)$, for which $V\subset \R^2$ is finite, and the edges are induced by bottomless rectangles: every $e\in E$ edge can be written in the form $e=\{(x,y)\in V: x_0 < y < x_1, y<y_0\}$ for some $x_0, x_1, y_0 \in\R$. 
\end{definition}
% Figure environment removed

It is already known \cite{feneketlen} that $1.67k\le m_{\mathcal{B}}(k)\le 3k-2$, and that for an arbitrary $c$ there is a Sperner member of the family with no $c$-shallow hitting set \cite{pshalfplane}. The existence of 2-shallow hitting sets on $\B_{=m}$ would imply $m_{\mathcal{B}}(k)\le 2k-1$, and the existence of 3-shallow hitting sets would give another proof of the upper bound of $3k-2$. Keszegh and Pálvölgyi \cite{pshalfplane}, and also Chekan and Ueckerdt \cite{chekan2022polychromatic} asked whether this could be the case. However, our following result refutes these possibilities:

\begin{theorem}\label{bottomless}
Let $m \ge 12$, then there is a member of $\mathcal{B}_{=m}$ which does not have a 3-shallow hitting set.
\end{theorem}

We do not yet know whether there is a 4-shallow hitting set for any $\mathcal{B}_{=m}$ with $m$ large enough. However, Planken and Ueckerdt recently showed that there is a 10-shallow hitting set for any member of $\mathcal{B}_{=m}$ \cite{bless10}.

\subsection{Axis-parallel strips}

We also investigated this question on another geometric hypergraph family and its dual:


\begin{definition} Denote by $\A$ the hypergraph family which consists of those hypergraphs $H=(V,E)$ for which $V\subset \R^2$ is finite, and the edges are induced by axis-parallel strips: every $e\in E$ edge can be presented in the form $V\cap\{(x,y)\in\R^2: x_0<x<x_1\}$ or $V\cap\{(x,y)\in\R^2: y_0<y<y_1\}$ for some $x_0, x_1, y_0, y_1 \in\R$. Denote by $\A^*$ its dual: the family of such hypergraphs that the vertices are axis-parallel strips, and the edges are sets consisting of the strips containing a given point $(x,y)$. 
\end{definition}


The following results are presented in \cite{apstrip}: 
$$2\bigg\lceil\frac{3k}{4}\bigg\rceil-1 \le m_\A(k)\le 2k-1$$
$$2\bigg\lceil \frac{k}{2}\bigg\rceil +1\le m_{\A^*}(k)\le 2k-1$$

The lower bound for $m_{\A^*}(k)$ comes from the more general case of arbitrary dimension, however, in the plane we can improve on this bound:


\begin{theorem}\label{strip_dual}
For the hypergraph family $\A^*$ we have 
$2\left(\left\lceil \frac{3}{4}k\right\rceil -1\right) + 1 \leq m_{\A^*}(k)$.
\end{theorem}

The existence of 2-shallow hitting sets on the hypergraph family $\A$ would give an alternative proof for the upper bound of $m_\A(k)$ but this is not the case for this problem either.


\begin{theorem}\label{strip}
For sufficiently large $m$ there is an element in $\A_{=m}$ with no 2-shallow hitting set.
\end{theorem}

Note that for given $m$ and for $k= \lceil \frac{m}{2}\rceil$ colors, taking a color class in the coloring corresponding to the upper bound in \cite{apstrip} gives a 3-shallow hitting set, so we have determined the smallest possible $c$ with a $c$-shallow hitting set for every member of $\A_{=m}$. 

%Furthermore, it is easy to see that every member of $\A^*_{m}$ has a 4-shallow hitting set, and that there are hypergraphs in $\A^*_{m}$ with no 3-shallow hitting sets: To obtain a 4-shallow hitting set, one can choose independently a set of horizontal resp. vertical strips such that their union is the same as the union of all horizontal resp. vertical strips in the vertex set, and every point of the plane is covered at most twice by the selected horizontal strips, and at most twice by the vertical ones.  
%To obtain a hypergraph in with no 3-shallow hitting set, take $m$ copies of each of the strips $\{(x,y):0<x<2\}$, $\{(x,y):1<x<3\}$, $\{(x,y):0<y<2\}$, and $\{(x,y):1<y<3\}$, this results in a hypergraph $\mathcal{H}\in\A^*_m$ such that any hitting set of $\mathcal{H}$ contains two horizontal and two vertical strips covering $(1.5,1.5).$

This problem motivated the study of the following more general hypergraph families:


\begin{definition} Denote by $\A^+$ the hypergraph family which consists of such $H=(V,E)$ hypergraphs for which $V\subset \R^2$ is finite, and the edges are induced by the union of a horizontal and a vertical axis-parallel strip: every $e\in E$ edge can be presented in the form $V\cap\big(\{(x,y)\in\R^2: y_0<y<y_1\}\cup \{(x,y)\in\R^2: x_0<x<x_1\}\big)$ for some $x_0,x_1,y_0,y_1\in\R$. 
\end{definition}

\begin{definition} Denote by $\A_s$ the hypergraph family which consists of such $H=(V,E)$ hypergraphs for which $V\subset \R^2$ is finite, and the edges are induced by the union of $s$ axis-parallel strips: every $e\in E$ edge can be written in the form $V\cap \Big(\cup_{i=1}^s A_i\Big)$ where $A_1,\ldots, A_s$ are axis-parallel strips. 
\end{definition}

We have the following bounds about their polychromatic colorings:


\begin{theorem}\label{kereszt}
$$3\bigg\lceil \frac{3}{4}k\bigg\rceil -2\le m_{\A^+}(k)\le 4k-3.$$
\end{theorem}

We note that $m_{\A^+}(2)=5$, which implies that the upper bound is sharp in the case of two colors.

\begin{theorem}\label{s_strip}
$$m_{\A_s}(k)=s \cdot m_\A(k)-s+1.$$
\end{theorem}


\begin{corollary}
$2s(\lceil \frac{3}{4} k\rceil-1)+1\le m_{\A_s}(k)\le s(2k-2)+1$, in particular $m_{\A_s}(2)=2s+1$, and $m_{\A_s}(3)=4s+1.$
\end{corollary}

\subsection{Arithmetic progressions}

A famous result of van der Waerden states that for any finite coloring of $\mathbb{N}$ there exists an arbitrarily long monochromatic arithmetic progression, or in other words, if we take the hypergraph on $\mathbb{N}$ with finite arithmetic progressions as edges, then every finite coloring contains a monochromatic hyperedge of arbitrary size.

There are many well-studied related problems, one of them being how does a {\it ladder}, a set $L\subseteq\mathbb{N}$ such that all finite colorings of $\mathbb{N}$ contain arbitrarily long arithmetic progressions with difference $d\in L$ look like \cite{ladder1, ladder2}. Translated again to the hypergraph terminology, let $H^L$ be the hypergraph obtained from the previously mentioned hypergraph by keeping only those edges which are derived from arithmetic progressions with difference from some subset $L\subseteq \mathbb{N}$,  then $L$ is a ladder if there is no $m\in \N$ such that $(H^L)_m$ is properly colorable with finitely many colors. Another related and already examined question concerning polychromatic edges is the following: what is the smallest $r$ such that every $r$-coloring of $\{1,2,\ldots,n\}$ contains a polychromatic arithmetic progression of size $k$? This is usually referred to as the {\it anti-van der Waerden number} aw$([n],k)$, which is also studied in the case of Abelian groups instead of $[n]$, see e.g. \cite{aw1, aw2}.

We have considered the polychromatic colorability of these, and also of somewhat more general hypergraphs. An additional motivation for this is a connection between the family $\mathcal{B}$ of geometric hypergraphs induced by bottomless rectangles and a special case of the arithmetic progression hypergraphs (Theorem \ref{veges_sorozat}. (1), $M=\{0\}$), realized by Keszegh and Pálvölgyi \cite{temavezetes}. In this paper some more connections between geometric hypergraph families and hypergraph families induced by arithmetic progressions are shown, leading to bounds for the parameter $m_k$ of these hypergraphs.

We denote by $\mathbb{N}$ the set of all natural numbers including $0$ and $\mathbb{N}^+=\mathbb{N}\setminus\{0\}$.

\begin{definition} 
Let $D,M \subseteq \mathbb{N}$ and denote by $\mathcal{A}_{D,M}$ the family of hypergraphs $A$ such that $S:=V(A) \subseteq \mathbb{N}$ is a finite set, and 
$E(A) \subseteq \{\{a_n\}_{n=1}^{k} \cap S:\{a_n\}_{n=1}^{k} \hspace{0.2 cm}\text{is a finite arithmetic} \newline \text{progression with difference} \hspace{0.2 cm}d \in D\hspace{0.2 cm} \text{such that}\hspace{0.2 cm}\exists m\in \N: a_0-md\in M\}.$

Denote by $\mathcal{A}_{D,M}^{\infty}$ the family of hypergraphs $A$ such that $S:=V(A) \subseteq \mathbb{N}$ is a finite set and $E(A) \subseteq \{\{a_n\}_{n=1}^{\infty} \cap S:\{a_n\}_{n=1}^{\infty} \hspace{0.2 cm}\text{is an infinite arithmetic progression with difference} \hspace{0.2 cm}d \in D\hspace{0.2 cm} \text{such that}\hspace{0.2 cm}\exists m\in \N: a_0-md\in M \}.$
\end{definition}

For the sake of simplicity we use the notations
$m_{D,M}(k):=m_{\mathcal{A}_{D,M}}(k)$ and
$m_{D,M}^\infty(k):=m_{\mathcal{A}_{D,M}^{\infty}}(k)$. 



If $D$ can be written in the form $\{\prod_{i=1}^n t_i^{j_i}: j_1,\ldots,j_n\in \mathbb{N}\}$ for some $t_1,\ldots,t_n$ integers, we can determine in most cases whether $m_{D,M}(k)$ and $m_{D,M}^\infty(k)$ are finite or not. We do this with the help of the already defined family $\mathcal{B}$ and the following hypergraph families:


\begin{definition} Denote by $\RR$ the hypergraph family which consists hypergraphs $H=(V,E)$ such that $V\subset \R^2$ is finite and the edges are induced by axis-parallel rectangles: every $e\in E$ edge can be written in the form $e=V\cap\big(\{(x,y)\in\R^2: y_0\le y \le y_1, x_0\le x\le x_1, \}\big)$ for some $x_0,x_1,y_0,y_1\in\R$. 
\end{definition}



\begin{definition} 
Denote by $\mathcal{T}_3$ the hypergraph family which consists hypergraphs $H =(V,E)$ such that $V \subset \mathbb{R}^3$ is finite and the edges are induced by translates of octants: every $e\in E$ edge can be written in the form
\[
e = V \cap \{(x,y,z):x_0 \le x,y\le y_0, z\le z_0\}
\]
some $x_0,y_0,z_0 \in \mathbb{R}$. The point $(x_0,y_0,z_0)$ is called the corner of the octant.
\end{definition}

\begin{definition} 
Denote by $\mathcal{T}_4$ the hypergraph family which consists of hypergraphs $H =(V,E)$ such that $V \subset \mathbb{R}^4$ is finite, and the edges are induced by translates of hextants: every $e\in E$ edge can be written in the form
\[
e = V \cap \{(x,y,z,w):x_0 \le x,y\le y_0, z\le z_0, w\le w_0\}
\]
some $x_0,y_0,z_0,w_0 \in \mathbb{R}$.
\end{definition}

Keszegh and Pálvölgyi showed that $m_{\mathcal{T}_3}(k)\le O(k^{5,09})$ \cite{ternyolcad}, and it is known that $m_\RR(k)=\infty$ (in fact, there is not even an integer $m$ such that the elements of $\RR_{=m}$ can be properly $k$-colored \cite[Theorem 3.]{teglalap})
Using this and a remark of Cardinal, Keszegh and Pálvölgyi proved that there is no $m$ such that the elements of $(\mathcal{T}_4)_{=m}$ can all be $k$-colored \cite[Theorem 12.]{hextant}, which yields $m_{\mathcal{T}_4}(k)=\infty$.

These hypergraph families are connected to the hypergraph families defined by arithmetic progressions in the following way: for suitable $C,D$ the vertices of $H\in\A_{C,D}$ or $H\in\A_{C,D}^\infty$ can be put into correspondence with the points of the 2-, 3-, or 4 dimensional space in such a way that the resulting set of points contains all hyperedges as an edge of a geometric hypergraph on this set, or conversely, we define a vertex set $S\subset \mathbb{N}$ while preserving all the edges.
This gives the possibility to improve the colorings of geometric hypergraphs by using hypergraphs induced by arithmetic progressions, and vice versa.





Our results regarding hypergraphs induced by infinite arithmetic progressions are as follows:

\begin{theorem}\label{vegtelen_sorozat}

\begin{enumerate}
    \item 
    Let $D := \{t^i:i \in \mathbb{N}$ for some $t \in \mathbb{N} \setminus \{0,1\}\}$, then
    \[
    m_{D, \mathbb{N}}^{\infty}(k) \leq m_{\mathcal{T}_3}(k).
    \]
    \item 
    Let $p,q$ be positive coprime integers, and suppose $M$ contains at most one element of every residue class modulo $pq$, then
    \[
    m_{\{p^iq^j:i,j \in \mathbb{N}\}, M}^{\infty}(k) \leq |M|(m_{\mathcal{T}_3}(k) -1)+1.
    \]
    Moreover, if $M=\{0\}$, then equality holds.
    \item 
    Let $p_1, p_2, p_3$ be positive coprime integers, then
    \[
    m_{\left\{ p_1^{j_1}p_2^{j_2}p_3^{j_3}:j_1,j_2,j_3 \in \mathbb{N}\right\},\{0\}}^{\infty}(k) = \infty.
    \]

\end{enumerate}
\end{theorem}

For the case of finite arithmetic progression we have the following:

\begin{theorem}\label{veges_sorozat}
\begin{enumerate}
    \item 
    Suppose that $M$ contains at most one element of every residue class modulo $t$, then
    \[
    m_{\{t^i:i \in \mathbb{N}\}, M}(k) \leq |M|(m_{\mathcal{B}}(k) -1) + 1.
    \]
    \item 
    Let $p,q$ be positive coprime integers, then
    \[
    m_{\{p^iq^j:i,j \in \mathbb{N}\},\{0\}}(k) = \infty.
    \]
\textbf{\textbf{}}\end{enumerate}
\end{theorem}

The proof of case $M=\{0\}$ in $(1)$ is by Keszegh and Pálvölgyi \cite{temavezetes}.

\section{Proofs}

\noindent{\it Notation.}
Directions in the plane are denoted by north, west, south, and east.


\begin{proof} [Proof of Lemma \ref{prop1}.] 
Let $H\in\HH$ be arbitrary, we need to show that $H_{\ge c(k-1)+1}$ can be colored polychromatically with $k$ colors.

First shrink all edges of $H_{\ge c(k-1)+1}$ to be of size exactly $c(k-1)+1$ in such a way that the resulting graph is in $\HH$, we can do this by our assumptions.
Then choose a $c$-shallow hitting set from our hypergraph, color its points to one fixed color, and leave out these vertices from the graph. The resulting graph is still an element of the hypergraph family, and all of its edges can be shrinked to contain exactly $c(k-2)+1$ vertices, and so that we get a graph from $\HH_{=c(k-2)+1}$. Then take another $c$-shallow hitting set, color it to the second color, and so on. After determining this way the first $k-1$ color classes, we still have at least one remaining vertex in every edge, color the remaining vertices to the $k$th color to get a desired $k$-coloring.
\end{proof}

\begin{proof} [Proof of Theorem~\ref{bottomless}] 

Take $\B_{=m}$ with $m\ge 12$. We will show that this family has an element with no 3-shallow hitting set.
First take $m$ points, $\mathcal{P}=\{P_1,\ldots,P_m\}$ along a horizontal line, and above each of them, take the construction on Figure \ref{fig:haromsek}.

%% Figure environment removed

% Figure environment removed


Starting from $P_i$ we have $A_i$ on a diagonal line consisting of $m-2$ points and going from west to east and from south to north, to northwest from $A_i$ we have $B_i$ on a diagonal line which goes to the other direction and which consists of $m$ points. To the west from $A_i$, we have $m^2$ points in a diagonal in the same direction as $A_i$, which consists of $m$ smaller groups of size $m$ $C_{i,j}$ (for $j=1,\ldots,m$), with $C_{i,j}$ being vertically between the $j$th and $(j+1)$th vertex of $B_i$. Let $H$ be the hypergraph on this set which contains all $m$-sets contained in a bottemless rectangle as hyperedges. Now for the sake of contradiction suppose that this hypergraph has a 3-shallow hitting set.

There is a point in $\mathcal{P}$ which is in the hitting set, because these vertices form a hyperedge. Now we separate two cases:

Case 1. $A_i$ contains at least one further point from the hitting set besides $\mathcal{P}$. $B_i$ consists of $m$ points, and can be separated by a bottomless rectangle, so it has a point from the hitting set, say the $j$th. We can choose an axis-parallel rectangle that it contains exactly this $j$th point from $B_i$. Also in $C_{i,j}$ there must be a point from the hitting set, so we can choose the rectangle in a way that its top left corner contains exactly this point from $C_{i,j}$. Now we got a bottomless rectangle with $m$ points in it, $m-2$ from $A_i$ with at least two from the hitting set, and 1-1 in the top corners from the hitting set, one from $B_i$, one from $C_{i,j}$, a contradiction.

Case 2. $A_i$ does not contain any more points from the hitting set (apart from $P_i$). Now we can take a bottomless rectangle which contains the $m-3$ rightmost points of $A_i$ and any 3 neighbouring points from $B_i$. This implies that any 3 neighbouring points in $B_i$ contain a point from the hitting set, which means $B_i$ contains at least 4 points from the hitting set if $m\ge12$, which is a contradiction.
\end{proof}

\begin{proof}[Proof of Theorem~\ref{strip_dual}]

Take $\lceil \frac{3}{4}k\rceil-1$ copies each of the the following axis-parallel strips: $\{(x,y):0<x<2\}$, $\{(x,y):1<x<3\}$, $\{(x,y):0<y<2\}$, and $\{(x,y):1<y<3\}$. Notice that any two of these has a part of their intersection which is disjoint from the other two original strips. (If we do not want vertices derived from the same strip, we can perturb them a little.)

Take that $H$ hypergraph from $\A^*$ which has these $4(\lceil \frac{3}{4}k\rceil-1)$ strips as vertices and all possible hyperedges determined by points. In any coloring of the strips, for each of the 4 original strips there exist strictly more than $\frac{k}{4}$ colors which do not appear in the copies of that strip. By the pigeonhole principle there are two original strips and a color excluded from copies of these two strips, hence $H$ has a hyperedge of size $2(\lceil \frac{3}{4}k\rceil-1)$ which is not polychromatic, implying $2\lceil\frac{3}{4}k\rceil -1\le m_{\A^*}(k)$.
\end{proof}

\begin{proof}[Proof of Theorem~\ref{strip}]

We construct a hypergraph in $\A$ with no 2-shallow hitting sets. The main arrangement of some vertices can be seen on Figure \ref{gigabra1}.
% Figure environment removed

\begin{comment}
% Figure environment removed
\end{comment}

Let $m$ be large enough, let $a= \lceil \frac{m}{3} \rceil-1$, and $b$ be 8 for $m \equiv 0 (\mod 3)$, 4 for $m \equiv 1 (\mod 3)$, and 6 for $m \equiv 2 (\mod 3)$, this satisfies $m=3a+ \frac{b}{2}-1$. Later we will need that $m\ge 5b$.

First we take $\mathcal{X}=\{x_1,\ldots,x_m\},$ a set of $m$ points on a southeast to northwest diagonal, as in the middle of Figure \ref{gigabra1}. We will leave empty a vertical strip containing them, so a hitting set must contain one of them. Then we assign disjoint horizontal strips containing each $x_i$, and in all of these horizontal strips, we take 3 sets of points $X_{i,1},X_{i,2},X_{i,3}$ of size $2a$ as shown on Figure \ref{gigabra1}, reserving disjoint vertical strips containing each of these sets, and whith all points from $X_{i,2}$ and $a$ points from $X_{i,1}$ having smaller $y$-coordinates then $x_i$, and all points from $X_{i,3}$ and the other $a$ points from $X_{i,1}$ having larger $y$-coordinates then $x_i$. It is enough to show that we can place additional points in these vertical strips in such a way that forces the original set of $2a$ points to contain a point from the hitting set, because in this case there will be a horizontal strip with $3a+1\le m$ points which has at least 3 points from the hitting set.

For each $X\in\{X_{i,j}:i\le m, j\le3\}$, the construction of this arrangement in a vertical strip is shown on Figure \ref{gigasok}. These points are placed in separate horizontal strips for each $X_{i,j}$. We arrange the points in $B$ and $C$ diagonally, and all groups contain as many points as written next to them.


%% Figure environment removed

% Figure environment removed



Let $S$ be the hitting set in question, suppose indirectly that the grey set of $2a$ points, denoted by $X$, does not contain an element from $S$. Then $B$ must have a point from the hitting set, because it forms a vertical strip of size $m$ together with $X$. Take the leftmost point in $B$ from $S$, and take the next $m-2a$ points on the diagonal of $B$ and $C$. These are also contained in a vertical strip of size $m$ together with $X$, so there must be a point there from the hitting set. This implies that we have $m-2a+1$ points in the middle of the diagonal of $B$ and $C$, which has at least 2 points from the hitting set. This implies that $D\cap S=\est$.

Thus $E \cup G$, $E \cup H$ and $F \cup H$ must contain a point from $S$. Notice that $m-2a-b+1$ (the size of $E$ and $F$) is exactly $a-\frac{b}{2}$, so the size of $E\cup F\cup H$ is $2a$, which implies $\big|(E\cup F\cup H)\cap S\big|\le 1$, because otherwise $E \cup F \cup H \cup B$ would be a strip of size $m$ containing at least 3 points from the hitting set.

% This implies $E\cap S=\est$, hence $G$ and $H$ contain a point from S, and so $I\cap S=\est$. This means that $J$ and $K$ has a point from $S$, $L\cap S=\est$. So if $m\ge 6b$, then $I\cup L$ form a strip of size $2m-6b\ge m$ without a point from the hitting set, which is a contradiction.


 This implies $E\cap S=\est$, hence $G$ and $H$ contain a point from S, and so $I\cap S=\est$. This means that $J$ and $K$ has a point from $S$, so if $m\ge 5b$, then $G\cup J\cup K$ form a strip of size $5b\le m$ with at least 3 points from the hitting set, which is a contradiction.
\end{proof}

\begin{proof}[Proof of Theorem~\ref{kereszt}]

For proving the upper bound, color an arbitrary point set in the plane in such a way that all axis-parallel strips of size at least $2k-1$ be polychromatic, this is possible by Theorem 1. in \cite{apstrip}. Then any edge of size at least $4k-3$ contains a horizontal or vertical axis-parallel strip of size at least $2k-1$ and thus polychromatic, implying that the coloring is polychromatic.

For the construction to prove the lower bound, take 8 sets of $\lceil \frac{3}{4}k\rceil-1$ points in the arrangement shown on Figure \ref{keresztabra}.

% Figure environment removed

For an arbitrary coloring of the vertices each set out of the eight has strictly more than $\frac{k}{4}$ colors which does not appear in them, so by the pigeonhole principle we have 3 sets with a common missing color. Observe that any three sets can be separated from the others with the union of a horizontal and a vertical strip, which implies $m(k)>3(\lceil \frac{3}{4}k\rceil-1)$.
\end{proof}

\begin{proof}[Proof of Theorem~\ref{s_strip}]

For given $k\ge 2$ take a hypergraph from $\A$ which has a non-poly\-chromatic edge of size at least $m_\A(k)-1$ for any coloring. Take $k(s-1)+1$ copies of the point set that forms the vertices of this hypergraph, and place them along a diagonal line so that all edges of the original hypergraph would be still an edge. For any coloring of the resulting hypergraph all copy has $m_\A(k)-1$ points in it forming an edge and missing at least one color. By the pigeonhole principle we have $s$ of these with the same missing color, these points are contained in the union of $s$ axis-parallel strips, so $s \cdot m_\A(k) -s + 1 \le m_{\A_s}(k)$.

For the other inequality we have to show the existence of a coloring such that any edge of size at least $s \cdot m_\A(k) -s + 1$ contains all $k$ colors. The same coloring as for $m_\A(k)$ satisfies this, because any $s \cdot m_\A(k) -s + 1$ points which is in the union of $s$ strips has $m_\A(k)$ points in it contained in a single strip.
\end{proof}



\begin{proof}[Proof of Theorem~\ref{vegtelen_sorozat}. (1):]
    We will show that for every $A \in \mathcal{A}_{D,C}^{\infty}$ there exists $T_A \in \mathcal{T}_3$ and $\phi :V(A)\to V(T_A)$ injection such that for every hyperedge $E \subseteq V(A)$ the set $\phi [E]\subseteq V(T_A)$ is also a hyperedge of $T_A$. Therefore, if we can $k$-color $\left(T_A\right)_m$ properly for $m:=m_{\mathcal{T}_3}(k)$, we obtain a polychromatic $k$-coloring of $A_{m}$.
    
    The vertex set of $A \in \mathcal{A}_{D,C}^{\infty}$ contains finitely many natural numbers. We will assign points in $\mathbb{R}^3$ to them as the vertices of $T_A$ such that for every hyperedge $E \subseteq V(A)$ the set $\phi [E]\subseteq V(T_A)$ can be defined as the intersection of an octant and $V(T_A)$. Fix the difference set D and the set C. For the sake of simplicity, we will assign a point in $\mathbb{R}^3$ to every natural number, such that every subset of $\mathbb{N}$ which could be a hyperedge in a hypergraph $A \in \mathcal{A}_{D,C}^{\infty}$ can be defined by an octant in $\mathbb{R}^3$. 
    
    Firstly, let us discuss the case when $D = \{t^i: i \in \mathbb{N}\}$ for some $t \in \mathbb{N}^+$. We will place axis-parallel squares recursively into each other on the $y$-$z$ plane, each of them corresponding to a natural number. This natural number is placed in the bottom left corner of the square, see Figure \ref{negyzetek}.
    
    % Figure environment removed
    
    The $0$th generation of the recursion is a square $S_0$ in the $y$-$z$ plane, corresponding to $0$. Place a point $P_0$ into the southwest corner of it. For the first generation, consider the $t$-base form of every positive natural number. The smallest valued non-zero bit can be $1, ..., t-1$, and it can represent some value $k \cdot t^i$ where $k \in \{1,...,t-1\}$, $i \in \mathbb{N}$. In other words, take that natural numbers whose $t$-base form contains only one non-zero bit. Accordingly, the first generation consists of squares $S_1, S_2,...,S_{t-1}, S_{10}, S_{20}, ..., S_{[t-1]0}, S_{100}, S_{200}, ...$ corresponding to $1, 2, ..., t-1, t, 2t, ..., (t-1)t, t^2, 2t^2, ...$, respectively. We place them diagonally (from northwest to southeast) into $S_0$, according to Figure \ref{negyzetek}, and we place a point $P_a$ into the southwest corner of every $S_a$. The second generation of squares correspond to natural numbers whose $t$-base form contains two nonzero bits. The value and place of the smallest nonzero bit defines that first-generation square in which we put our second-generation square: in $S_a$ where the value of $a$ is $k_1 \cdot t^{i_1}$, we place the squares corresponding to the values $k_1 \cdot t^{i_1} + k_2 \cdot t^{i_2}$, in a diagonal, ordered by $i_2$ and then $k_2$ ($i_1<i_2$). And so on, the $k$-th generation consists of squares corresponding to numbers whose $t$-base form contains exactly $k$ nonzero bits. As in the 0th and first generation, into every square $S_a$ placed in the $k$-th generation we put a point $P_a$ to the southwest corner. 

    
    Now we have points in the $y$-$z$ plane assigned to every $n\in\mathbb{N}$, if $n$ has $k$ pieces of nonzero bits in its $t$-base form, we have placed a square for it in the $k$-th generation. Finally, we set the $x$-coordinate of $P_n$ to $n$, and let $\phi (n)=P_n.$
    
    We have to prove that for every set of form $E = \{a_0+ i \cdot t^j: i \in \mathbb{N}\}$, $a_0, j \in \mathbb{N}$ there exists an octant which contains exactly these points. Firstly, fix $a_0$ and $j$ such that $a_0<t^j$. If we project back our points to the $y$-$z$ plane, we can almost "cut out" the sequence with an axis-parallel quarter-plane, see Figure \ref{negyzetek}. Take the square $S_b$ corresponding to $a_0$. According to the construction, the square $S_c$ corresponding to $a_0 + t^j$ is contained in $S_b$. Let $e$ be the east bordering line of $S_c$ and $f$ be the north bordering line of $S_b$. The lines $e$ and $f$ define four quarter-planes, and let $Q$ be the one southwest to $y$ and $z$. $Q$ contains points corresponding to numbers which are at least $a_0$ and are divisible by $t^j$, or which are smaller than $a_0$. Notice that every number in form of $n=a_0+k \cdot t^j$, $k\in \mathbb{N}$ is contained in $Q$ since only $S_b$ contains numbers congruent to $a_0$ modulo $t^j$, and points above $f$ in $S_b$ correspond to numbers which are the sum of $a_0$ and $r_j$ where $r_j$ has a $t$-base form with $0$ in its $j^{th}$ bit. 
    
    For the case $a_0 \geq t^j$, we can define some $a_{-1}, a_{-2},...,a_{-k}$ such that $0\leq a_{-k} < t^j$, and $\{a_i\}_{i \in \{-k, ..., -1\} \cup \mathbb{N}}$ is still an arithmetic progression, and we can build the same construction for this extended arithmetic progression as in the case $a_0<t^j$. Take the axis-parallel lines $e$ and $f$ as in the previous case, and assume that they are defined by the equations $y=y_0$ and $z=z_0$, respectively. Then the octant in $\mathbb{R}^3$ corresponding to $E$ is the set $\{(x,y,z): x \geq a_0, y \leq y_0, z \leq z_0\}$. 
    
    The general case when $D = \{d_i: i, d_i \in \mathbb{N}^+, d_{i-1} | d_{i}\}$, $d_0:=1$ is very similar. We use almost the same construction, we just replace the $t$-base form of natural numbers with the following: Assume that for every $k<n$ we have defined a form $k = \sum\limits_{j=0}^{i-1} c_j \cdot d_j$. For $n \in \mathbb{N}$, let $i$ be the maximal index for which $d_i \leq n$. Let $c_i$ be the maximal natural number for which $c_i \cdot d_i \leq n$. Notice that $c_i<\frac{d_{i+1}}{d_i}$. If $k := n-c_i \cdot d_i = \sum\limits_{j=0}^{i-1} c_j \cdot d_j$, write $n$ in form $n = \sum\limits_{j=0}^{i} c_j \cdot d_j$. We build up the construction in the same way, we just replace the $t$-base form of numbers with the sequence $...c_2c_1c_0$.
    \end{proof}


    \begin{proof}[Proof of Theorem~\ref{vegtelen_sorozat}. (2)]
        Firstly, we show that if $M=\{0\}$ then 
$m_{\{p^iq^j:i,j \in \mathbb{N}\},M}^{\infty}(k) = m_{\mathcal{T}_3}(k)$. To prove this, we construct a $\phi:\mathbb{N}\to \mathbb{R}^3$ injection such that for every fixed $i,j$ the set $\phi[\{k \cdot p^i q^j: k \in \mathbb{N}^+\}]$ can be defined by an octant. If $n = k \cdot p^i q^j$ where $p,q$ are not divisors of $k$, let $g_1(n) := i, g_2(n):=j$. Since $p$ and $q$ are relative primes, the form above is unique for every $n$, thus $g_1(n)$ and $g_2(n)$ are well-defined. For $g_1(n), g_2(n) >0$ define $\phi(n) = (n, 1- \frac{1}{g_1(n)}, 1- \frac{1}{g_2(n)})$, and replace the corresponding coordinate by $1$ if $g_1(n)$ or $g_2(n)$ is 0. Let $\phi(0)=(0,1,1)$. The set $\phi[\{a_0+k \cdot p^iq^j: k \in \mathbb{N}, \text{ } p^iq^j|a_0\}]$ is the intersection of {\rm Im}$(\phi)$ and the octant  $\{(x,y,z): x \geq a_0, y \geq 1-\frac{1}{i}, z \geq 1-\frac{1}{j} \}$. 

Now we turn to the general case. Notice that in the previous construction {\rm Im}$(\phi)$ projected to the $y$-$z$ plane is contained in a square $S$ which has side length $1$. In this case, we build up $pq$ squares on the $y$-$z$ plane, diagonally. The square $S_r$ ($0\leq r < pq$) will contain numbers congruent to $r$ modulo $pq$.  

Define the function $\psi: \mathbb{N} \to \mathbb{R}^3$ such that for $n=k\cdot pq+r$, $0 \leq r<pq$, $pq \nmid k$ let $\psi(n):= (n, (1- \frac{1}{g_1(kpq)})+r, (1- \frac{1}{g_2(kpq)})-r)$, where $g_1,g_2$ are the functions from the definition of $\phi$. Let $\psi(0)=(0,1,1)$. Projecting back {\rm Im}$(\psi)$ to the $y$-$z$ plane, we can observe that the square $S_r=[r,r+1]\times [-r,-r+1]$ contains numbers congruent to $r$ modulo $pq$, and the construction in $S_0$ is almost the same as in $S$.  

Observe that every arithmetic progression with difference $d=p^iq^j$, $i, j>0$ is contained by $S_r$ for some $r$ on the $y$-$z$ plane. Similarly to the case $M=\{0\}$, the set $\{a_0 + k \cdot p^iq^j: k \in \mathbb{N}\}$ is the intersection of {\rm Im}$(\psi)$ and the octant  $\{(x,y,z): x \geq a_0, y \geq 1-\frac{1}{i}+r, z \geq 1-\frac{1}{j}-r \}$, where $r \equiv a_0 \mod pq$. 


We have yet to consider the case when $d\in\{1,p,q\}$. 
In this case, every hyperedge with at least $|M|\cdot(m_{\mathcal{T}_3}(k) -1)+1$ vertices 
contain at least $m_{\mathcal{T}_3}(k)$ vertices for some $r\in M$, according to the pigeonhole principle. The corresponding numbers of these vertices form an arithmetic progression with difference $d=pq$. 
We have already proved that this is sufficient for the existence of a polychromatic $k$-coloring.

To prove equality for the case $M=\{0\}$, we assign to every $T_A \in \mathcal{T}_3$ a hypergraph $A \in \mathcal{A}_{D,\{0\}}^{\infty}$, $D = \{p^iq^j:i,j \in \mathbb{N}\}$ by a function $\gamma: V(T_A) \to \mathbb{N}$ such that for every hyperedge (octant) $E\subseteq V(T_A)$ the vertex set (set of some natural numbers) $\gamma[E]$ is a hyperedge (arithmetic progression with proper difference) in $A$. We can assume that the vertices of $T_A$ have distinct $x$-, $y$- and also $z$-coordinates, since perturbing every point a little to obtain different coordinates results in a more extensive set of hyperedges. Order the elements of $V(T_A)$ in ascending order by their $x$-coordinate and denote by $x(P)$ the place of $P$ in the ordering for every $P \in V(T_A)$. Define $y(P)$ and $z(P)$ similarly. Let $\gamma(P) = p^{y(P)} \cdot q^{z(P)} \cdot k$ with some $k$ such that $\gamma(P)$ be the $x(P)^{th}$ largest value in {\rm Im}($\gamma$). Notice that $|V(T_A)|<\infty$. Now every vertex set in $T_A$ which is defined by an octant corresponds to a subset of an arithmetic progression with difference $p^iq^j$ and $a_0=0$.
\end{proof}

\begin{proof}[Proof of Theorem~\ref{vegtelen_sorozat}. (3)]
We use the fact that $m_{\mathcal{T}_4}(k) = \infty$. Again, we assign to every $T_A \in \mathcal{T}_4$ a hypergraph $A \in \mathcal{A}_{D,\{0\}}^{\infty}$, $D = \{p^iq^j:i,j \in \mathbb{N}\}$ by a function $\eta: V(T_A) \to \mathbb{N}$ such that for every hyperedge (hextant) $E\subseteq T_A$ the vertex set (set of some natural numbers) $\eta[E]$ is a hyperedge (arithmetic progression with proper difference) in $A$. It follows that if $m := m_{\left\{ p_1^{j_1}p_2^{j_2}p_3^{j_3}:j_1,j_2,j_3 \in \mathbb{N}\right\},\{0\}}^{\infty}(k)$ would be finite then we could polichromaticly $k$-color the hyperedges of an arbitrary $(T_A)_{\geq m} \in \mathcal{T}_4$. 

Now we execute the very same method as in the end of the proof of Theorem 6.2. We can assume again that the vertices of $T_A$ have distinct $x$-, $y$-, $z$- and also $w$-coordinates. Order the elements of $V(T_A)$ in ascending order by their $x$-coordinate and denote by $x(P)$ the place of $P$ in the ordering for every $P \in V(T_A)$. Define $y(P)$, $z(P)$ and $w(P)$ similarly. Let $\eta(P) = p_1^{y(P)} \cdot p_2^{z(P)} \cdot p_3^{w(P)} \cdot k$ with some $k$ such that $\eta(P)$ be the $x(P)^{th}$ largest value in {\rm Im}($\eta$). Now every vertex set in $T_A$ defined by a hextant corresponding to a subset of an arithmetic progression with difference $p^iq^j$ and $a_0=0$. 
\end{proof}

\begin{proof} [Proof of Theorem~\ref{veges_sorozat}. (1)]
Firstly, we show that $m_{\{t^i:i \in \mathbb{N}\}, \{0\}}(k) \leq m_{\mathcal{B}}(k)$. To prove this, we construct a $\phi:\mathbb{N}\to \mathbb{R}^2$ injection such that for every fixed $i, j_0$ and $j_{\text{max}}$ the set $\phi[\{j \cdot t^i: j \in \mathbb{N}^+, j_0 \leq j \leq j_{\text{max}}\}]$ can be defined by a bottomless rectangle. Denote the arithmetic sequence $j_0 \cdot t^i, \ldots, j_{\text{max}} \cdot t^i$ by $a_0, \ldots, a_{\text{max}}$. If $n = j \cdot t^i$ where $t$ is not a divisor of $j$, let $t(n) = i$. Define $\phi(n)= (1- \frac{1}{n}, \frac{1}{t(n)})$ and $\phi(0)=(0,0)$. Notice that the vertical strip $\{(x,y): 1-\frac{1}{a} \leq x \leq 1-\frac{1}{b}\}$ for some $a \leq b \in \mathbb{N}$ contains the images of numbers between $a$ and $b$, and the half-plane $\{(x,y): y \leq \frac{1}{i}\}$ for some $i \in \mathbb{N}$ contains the images of numbers which are divisible by $t^i$. Therefore, the set $\phi[\{a_0, \ldots, a_{\text{max}}\}]$ is the intersection of {\rm Im}$(\phi)$ and the bottomless rectangle $\{(x,y): 1-\frac {1}{a_0} \leq x \leq 1- \frac {1}{a_{\text{max}}}, \text{ } y \leq \frac{1}{i}\}$ if $j_0>0$, and $\{(x,y): 0 \leq x \leq 1- \frac {1}{a_{\text{max}}}, \text{ } y \leq \frac{1}{i}\}$ if $j_0=a_0=0$.

Notice that in this construction {\rm Im}$(\phi)$ was contained in the bottomless rectangle $\{(x,y): 0 \leq x \leq 1, \text{ } y \leq 1\}$. In the general case, we place $t$ bottomless rectangles such that the bottomless rectangle $B_r$ ($0 \leq r < t$) will contain numbers congruent to $r$ modulo $t$. 

Define the function $\psi: \mathbb{N} \to \mathbb{R}^2$ such that for $n=m\cdot t+r$, $0 \leq r<t$, let $\psi(n):= (2r+1-\frac{1}{n}, \frac{1}{t(n)})$ and $\psi(0)=(0,0)$. Observe that every arithmetic progression with difference $d=t^i$, $i>0$ is contained by $B_r$ for some $r$, where $B_r = \{(x,y): 2r \leq x \leq 2r+1, \text{ } y \leq 1\}$, since the $t-$residue of such a sequence is constant. Similarly to the case $M=\{0\}$, the set $\{m_r + j \cdot t^i: j \in \mathbb{N}, j \leq j_{\text{max}}\}$ is the intersection of {\rm Im}$(\psi)$ and the bottomless rectangle $\{(x,y): 2r+1- \frac {1}{a_0} \leq x \leq 2r+1- \frac {1}{a_{\text{max}}}, \text{ } y \leq \frac{1}{i}\}$ where $r \equiv m_r \mod t$, $a_0 = m_r + 0 \cdot t^i -r, a_1 = m_r + 1 \cdot t^i- r, \ldots, a_{\text{max}} = m_r + j_{\text{max}} \cdot t^i -r$, if $m_r \neq r$ (thus $a_0 \neq 0$). If $m_r=r$, then the corresponding bottomless rectangle is $\{(x,y): 2r \leq x \leq 2r+1- \frac {1}{a_{\text{max}}}, \text{ } y \leq \frac{1}{i}\}$. 

In the case $i=0$ (and consequently $d=1$), every hyperedge with at least $|M|\cdot(m_{\mathcal{B}}(k) -1)+1$ vertices  
contain at least $m_{\mathcal{B}}(k)$ vertices in $B_r$ for some $r\in M$, according to the pigeonhole principle. The corresponding numbers of these vertices form an arithmetic progression with difference $d=t$. We have already proved that this is sufficient for the existence of a polychromatic $k$-coloring.
\end{proof}

\begin{proof}[Proof of Theorem~\ref{veges_sorozat}. (2)]
Firstly, we define the hypergraph class $\mathcal{T}_{fin}$. Let $A \in \mathcal{T}_{fin}$ if $A$ is a hypergraph such that $V(A) \subset \mathbb{R}^3$, $|V(A)|< \infty$, and $$E(A) \subseteq \{V(A) \cap \{(x,y,z): x_1 \leq x \leq x_2, \text{ } y \leq y_0, \text { } z \leq z_0\}: \text{ } x_1,x_2,y_0,z_0 \in \mathbb{R}\}.$$
    \begin{lemma}
    $m_{\mathcal{T}_{fin}}(k) = \infty$.
    \end{lemma}
    \begin{proof}
        As stated previously, $m_{\mathcal{R}}(k) = \infty$.
        Take a plane $H_0$ in $\mathbb{R}^3$ whose normal vector is $(0,1,1)$. For every $R \in \mathcal{R}$ on $H_0$, there exists a hypergraph $A \in \mathcal{T}_{fin}$, such that $V(R) = V(A)$, and every hyperedge in $R$ is also a hyperedge in $A$, see Figure \ref{ferde}. Therefore, $m_{\mathcal{T}_{fin}}(k) \geq m_{\mathcal{R}}(k) = \infty$.
\end{proof}

% Figure environment removed

Our goal is to show that $m_{\{p^iq^j:i,j \in \mathbb{N}\},\{0\}}(k) \geq m_{\mathcal{T}_{fin}}(k)$. 

We have already shown in the proof of Theorem \ref{vegtelen_sorozat}. (2) that 
$m_{\{p^iq^j:i,j \in \mathbb{N}\},\{0\}}^{\infty}(k) \geq m_{\mathcal{T}_3}(k)$. To every $T_A \in \mathcal{T}_3$, we have assigned a hypergraph $A \in \mathcal{A}_{D,\{0\}}^{\infty}$, $D = \{p^iq^j:i,j \in \mathbb{N}\}$ by an injection $\gamma_A: V(T_A) \to \mathbb{N}$ such that for every hyperedge (octant) $E\subseteq V(T_A)$ the vertex set (set of some natural numbers) $\gamma_A[E]$ was a hyperedge (arithmetic progression with proper difference $d_E$) in $A$. Observe that every set $E_{x_1,x_2,y_0,z_0} = \{(x,y,z): x_1 \leq x \leq x_2, \text{ } y \leq y_0, \text { } z \leq z_0\}$ is the subset of an octant $O_{x_1,y_0,z_0} = \{(x,y,z): x_1 \leq x, \text{ } y \leq y_0, \text { } z \leq z_0\}$. 

Therefore, for every $T_{A'} \in \mathcal{T}_{fin}$ we have a $T_A \in \mathcal{T}_3$ such that $V(T_A) = V(T_{A'})$, and every hyperedge in $T_{A'}$ is the subset of a hyperedge of $T_A$. Use the same $\gamma_A$ injection to obtain a hypergraph $A' \in \mathcal{A}_{D,\{0\}}$. The image of every octant $O_{x_1,y_0,z_0} \cap V(T_A)$ is (the subset of) an arithmetic progresson with some $a_0$ and difference $d=p^iq^j$. Notice that $a_0$ is divisible by $d$. Take a hyperedge $E = E_{x_1,x_2,y_0,z_0} \cap V(T_{A'})$ of $T_{A'}$ with an arbitrary $x_2 \in \mathbb{R}$. The image $\gamma_A[E]$ is the prefix of the arithmetic progression above, with last element $a_n$. If $a_n = K \cdot d$ then $\gamma_A[E] = \{k \cdot d: k \in \mathbb{N}^+, k \leq K\} \cap V(A')$, which can be a hyperedge in $A'$.
Therefore, $m_{\{p^iq^j:i,j \in \mathbb{N}\},\{0\}}(k) \geq m_{\mathcal{T}_{fin}}(k) = \infty$.
\end{proof}

\section*{Acknowledgements}

The research was supported by the Lend\"ulet program of the Hungarian Academy of Sciences (MTA). BB was also supported by Lend\"ulet Grant no. 2022-58. of the Hungarian Academy of Sciences (MTA).

We are extremely grateful to Balázs Keszegh and Dömötör Pálvölgyi for their valuable remarks. We are also thankful to Sára Tóth for her collaboration throughout the research.

%\newpage

%\begin{thebibliography}{99}



%\bibitem{felsik}
%S. Smorodinsky, Y. Yuditsky, Polychromatic Coloring for Half-Planes, J. Comb. Theory, Ser. A 119:146-154, 2012
%\\http://arxiv.org/abs/1006.3191

%\bibitem{feneketlen}
%A. Asinowski, J. Cardinal, N. Cohen, S. Collette, T. Hackl, M. Hoffmann, K. Knauer, S. Langerman, M. Lason, P. Micek, G. Rote, T. Ueckerdt, Coloring hypergraphs induced by dynamic point sets and bottomless rectangles, Algorithms and Data Structures, LNCS 8037:73-84, 2013
%\\http://arxiv.org/abs/1302.2426

%\bibitem{apstrip}
%G. Aloupis, J. Cardinal, S. Collette, S. Imahori, M. Korman, S.Langerman, O. Schwartz, S. Smorodinsky, P. Taslakian, Colorful Strips, Graphs and Combinatorics 27:327-339, 2011
%\\http://arxiv.org/abs/0904.2115

%\bibitem{ternyolcad}
%B. Keszegh, D. Pálvölgyi, More on Decomposing Coverings by Octants, Journal of Computational Geometry 6:300-315, 2015
%\\http://arxiv.org/abs/1503.01669

%\bibitem{dekompozicio} J. Pach, D. Pálvölgyi, G. Tóth, Survey on Decomposition of Multiple Coverings, Bolyai Society Mathematical Studies 24:219-257, Springer-Verlag, 2014
%\\https://domotorp.web.elte.hu/cikkek/surveyfinal.pdf


%\bibitem{teglalap}
%X. Chen, J. Pach, M. Szegedy, G. Tardos, Delaunay graphs of point sets in the plane with respect to axis-parallel rectangles, Random Struct. Algorithms, 34:11-23, 2009
%\\https://www.math.nyu.edu/~pach/publications/rectangle120406.pdf

%\bibitem{pshalfplane}
%B. Keszegh, D. Pálvölgyi, An abstract approach to polychromatic coloring: shallow hitting sets in ABA-free hypergraphs and pseudohalfplanes, WG 2015, 266-280
%\\http://arxiv.org/abs/1410.0258

%\bibitem{hextant}
%B. Keszegh, D. Pálvölgyi, Proper Coloring of Geometric Hypergraphs, Discrete and Computational Geometry 62(3) (2019), 674-689
%\\http://arxiv.org/abs/1612.02158


%\bibitem{ladder1}
%J. Guerreiro, I. Z. Ruzsa, M. Silva, Monochromatic paths for the integers.
%European Journal of Combinatorics 58 (2016), 283–288.
%\\https://arxiv.org/abs/1606.00418

%\bibitem{ladder2}
%T. C. Brown, R. L. Graham, B. M. Landman, On the set of common differences in van der Waerden’s theorem on arithmetic progressions. Canadian Mathematical Bulletin 42, 1 (1999), 25–36.

%\bibitem{aw1}
%S. Butler, C. Erickson, L. Hogben, K. Hogenson, L. Kramer, R. Kramer, J. Lin, R. Martin,
%D. Stolee, N. Warnberg, and M. Young, Rainbow arithmetic progressions. arXiv preprint (2014).
%\\https://arxiv.org/pdf/1404.7232.pdf

%\bibitem{aw2}
%Z. Berikkyzy, A. Schulte, M. Young, Anti-van der Waerden numbers of 3-term arithmetic progressions, The Electronic Journal of Combinatorics 24(2), (2016)
%\\https://arxiv.org/abs/1604.08819



%\end{thebibliography}




\bibliographystyle{plain}
\bibliography{ref}

\medskip

\end{document}
