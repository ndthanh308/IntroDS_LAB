\section{Related Works}
\subsection{Conversational Recommendation}
 Conversational recommendation systems (CRS) \cite{lei2018sequicity, priyogi2019preference, xie2021comparison, zhou2020improving,zhao2023towards} aim to communicate with the user and recommend items based on the attributes explicitly asked during the conversation. Due to its ability to dynamically get the user's feedback, CRS has become an effective solution for capturing dynamic user preferences and solving the explainability problem. Various efforts have been conducted to explore the challenges in CRS which can mainly be categorized into two tasks: dialogue-biased CRS studies the dialogue understanding and generation \cite{li2018towards,chen2019towards,kang2020recommendation,liu2020towards}, and recommendation-biased CRS explores the strategy to consult and recommend \cite{christakopoulou2016towards,christakopoulou2018q,sun2018conversational,lei2020estimation}. This work focuses on the recommendation-biased CRS.

Early works on the recommendation-biased CRS \cite{christakopoulou2016towards, christakopoulou2018q, sun2018conversational} only consider the conversational recommendation under simplified settings. For example, Christakopoulou  \etal \cite{christakopoulou2016towards} consider the situation that CRS only needs to recommend without asking the user about his/her preferred attributes. The Q\&A work \cite{christakopoulou2018q} proposes to explore the situation that CRS jointly asks attributes and recommends items, but restricts the conversational recommendation to two turns: one to ask attributes and the other to recommend items. To explore a more realistic scenario of the recommendation-biased CRS, further efforts \cite{lei2020estimation,lei2020interactive} based on the reinforcement learning (RL) are conducted to explore the problem of multi-round conversational recommendation (MCR) which aims to strategically ask users binary questions towards attributes and recommend items in multiple rounds, achieving success in the limited turns. Zhang \etal \cite{zhang2022multiple} further explore the setting of multi-interest MCR (MMCR) where users have multiple interests in attribute combinations and allows CRS to ask multi-choice questions towards the user-preferred attributes.

 The main challenge of MCR is how to dynamically learn user preferences, and accordingly choose actions that satisfy user preferences. CRM \cite{sun2018conversational} and EAR \cite{lei2020estimation} learn user preferences with a factorization-based method under the pairwise Bayesian Personalized Ranking (BPR) framework \cite{rendle2009bpr}. SCPR \cite{lei2020interactive} learns user preferences by reasoning the path on the user-item-attribute graph and strictly chooses actions on the path. Unicorn \cite{deng2021unified} builds a weighted graph to model the dynamic relationship between the user and the candidate action space and proposes a graph-based Markov Decision Process (MDP) environment to learn dynamic user preferences and choose actions from the candidate action space. MCMIPL \cite{zhang2022multiple} further considers the multiple interests of the user and develops a multi-interest policy learning module that combines the graph-based MDP with the multi-attention mechanism. Despite effectiveness, previous works model user preferences with binary relations, while hardly capturing the multiplex relations and ignore the influence of social relations on user preferences which are important in modeling dynamic user preferences.

\subsection{Social Recommendation}
Social recommendation \cite{kautz1997referral,guo2015trustsvd,jiang2014scalable} aims to exploit social relations to enhance the recommender system. According to the social science theories \cite{anagnostopoulos2008influence,bond201261,mcpherson2001birds}, user decisions are influenced by their social relations, leading to similar preferences among social neighbors. Following this assumption, SoRec \cite{ma2009trust} jointly factorizes the user-item matrix and the user-user social relation matrix by sharing the same user preference latent factor. STE \cite{ma2009learning} learns user preferences by linearly combing the preference latent factor of the user and his/her social neighbors. SocialMF \cite{jamali2009trustwalker} forces the user preference latent factor to be similar to that of his/her social neighbors by adding regularization to the user-item matrix factorization. These works only leverage first-order social neighbors for recommendation and ignore the fact that the social influence could diffuse recursively through social networks.

To model the high-order social influence, graph neural networks (GNNs) \cite{kipf2016semi} are introduced to social recommendation due to their superiority in learning the graph structure. GraphRec \cite{fan2019graph} applies GNNs to capture the heterogeneous graph information from the user-item interactions and social relations. DiffNet \cite{wu2019neural} and its extension DiffNet++ \cite{wu2020diffnet++}  develop a layer-wise influence propagation structure to model the recursive social diffusion in social recommendation. These works model user preferences with pairwise relations and fail to capture the complex multiplex user relation patterns (\ie user-friend-item). MHCN \cite{yu2021self} constructs hypergraphs by unifying nodes that form specific triangular relations and applies hypergraph neural network \cite{feng2019hypergraph, xia2022hypergraph} to model user preferences with hypergraphs. Despite effectiveness, previous works treat social relations as static information, while ignoring the dynamic characteristic of user preferences and failing to dynamically choose helpful social information for the learning of user preferences.



