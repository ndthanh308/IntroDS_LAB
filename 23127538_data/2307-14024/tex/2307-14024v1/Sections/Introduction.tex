\section{Introduction}
% CRS MCR 简单介绍
 Recommendation systems \cite{zhao2022multi,wu2019session, wang2020disenhan,wei2023recommendation,ferrari2019we} are emerging as an efficient tool to help users find items of potential interest. They conventionally learn user preferences from their historical actions \cite{he2017neural,rendle2010factorization}, while hardly acquiring dynamic user preferences which often drift with time. To this end, conversational recommendation systems (CRS) \cite{lei2018sequicity} are proposed to dynamically acquire user preferences and accordingly make recommendations through interactive conversations. 
 % Figure environment removed
 Different settings  \cite{christakopoulou2016towards,christakopoulou2018q,sun2018conversational} of CRS are explored and we focus on the Multi-Interest Multi-round Conversational Recommendation (MMCR) \cite{zhang2022multiple} in which users could accept multiple items and CRS needs to strategically ask multi-choice questions about user-preferred attributes and accordingly recommend items, reaching success in the limited turns. 

Learning the dynamic user preferences for the candidate attributes and items accurately is of crucial importance for CRS. CRM \cite{sun2018conversational} and EAR \cite{lei2020estimation} develop factorization-based methods to learn user preferences from pairwise interactions, but they fail to capture multi-hop information from the connectivity graph. SCPR \cite{lei2020interactive} learns user preferences by reasoning the path on the user-item-attribute graph.  Unicorn \cite{deng2021unified} and MCMIPL \cite{zhang2022multiple} further apply graph neural networks to 
 learn user preferences from the graph structure that captures rich correlations among different types of nodes (\ie user, attribute, and item). 
 Despite effectiveness, previous works learn user preferences with pairwise relations from the interactive conversation (\ie user-item and user-attribute relations) and the item knowledge (\ie item-attribute relations), while largely ignoring the fact that factors for a relationship in CRS are multiplex. For the example in Fig.\ref{fig:motivation}, the user dislikes Switch because of its attribute named "game console" rather than its other attributes like "electronics". Moreover, social influence is also an important factor that affects user preferences towards the item, since people with social connections will influence each other, leading to similar interests \cite{cialdini2004social, guo2015trustsvd}.
 However, in the field of CRS, social information is seldom explored.
 Inspired by the advantage of hypergraph \cite{feng2019hypergraph, xia2022hypergraph} in modeling the multiplex relations (\ie relations that connect more than two nodes), we investigate the potential of hypergraph modeling with the integration of interactive conversation, item knowledge, and social influence for learning dynamic user preferences in CRS.


Actually, it's non-trivial to build a hypergraph for learning dynamic user preferences in CRS, due to three challenges: 1) The first challenge is the dynamic filtering and utilizing of social information. The social information conventionally contains all the historical interactions of the user's friends, which could be noisy for the dynamic user preferences in the current conversation, since only friend preferences that satisfy the current conversation are helpful for dynamic user preferences learning.
For the example in Fig.\ref{fig:motivation}, only the friends' preferences for "smartphone" are helpful for learning the dynamic user preferences. 
2) The second challenge is hypergraph formulation. In the scenario of CRS (as illustrated in Fig.\ref{fig:motivation}), there mainly remain three multiplex relation patterns, that is, the user likes/dislikes the items that satisfy some attribute (\emph{Like/Dislike view}) and the user shares the preferences for items with some friend (\emph{Social view}). Each relation pattern corresponds to a kind of hyperedges, which are successively generated during the interactive conversation.
3) The third challenge is the aggregation of user preferences learned from different views, which might obscure the inherent characteristics of preference distributions from different views and the correlation between them. Specifically, user preferences from the same views should be more similar than user preferences from different views. And the user preferences from \emph{Like View} should be similar to \emph{Social View} while different from \emph{Dislike View}. Contrastive learning \cite{wu2021self, velickovic2019deep, hassani2020contrastive}, one successful self-supervised learning paradigm, which aims to learn discriminative representations by contrasting positive and negative samples, paves a way to maintain the inherent characteristics and the correlation of user preferences learned from different views.

%The user preferences learned from different views are distributed independently, while user preference distributions should be correlated in the scenario of CRS. Specifically, the user preferences from the same views should be more similar than user preferences from different views. And the user preferences from (\emph{Like View}) should be similar with (\emph{Friend View}) when compared with (\emph{Dislike View}). %Inspired by the contrastive learning (CL) \cite{wu2021self, velickovic2019deep}, one successful self-supervised learning paradigm, which aims to learn discriminative representations by contrasting positive and negative samples, we contrast the user preferences learned from different views, for sufficiently integrating them to learn discriminative representations of user preferences.
% 解决问题:如何用社交关系、 以及超图
%To tackle the problem of information limitation, we could leverage social relations to enhance the learning of user preferences, since users tend to have similar preferences with their friends. Existing works for social recommendation \cite{wu2020diffnet++,yu2021self}  mainly treat social networks as static information, while ignoring the dynamic characteristic of user preferences. In the scenario of CRS, we should accordingly choose helpful social information (\eg the friends' preferences for "smartphone" as illustrated in Fig.\ref{fig:motivation}) to learn the dynamic user preferences. To learn the multiplex relations, hypergraph \cite{feng2019hypergraph, xia2022hypergraph} provides a solution by generalizing the concept of edge to make it connect more than two nodes. Nevertheless, the real scenarios of
%conversational recommendation are more complicated (as illustrated in Fig.\ref{fig:motivation}), the user likes the items that satisfy some attribute (\emph{Like view}), the user dislikes items that satisfy some attribute (\emph{Dislike view}) and the user shares the preferences for items with some friend (\emph{Friend view}). By integrating information from different types of multiplex relations, we could comprehensively obtain dynamic user preferences from different views. Consequently, it motivates us to design a socially-aware hypergraph neural network to dynamically learn user preferences from different views.

%我们的方案。

To this end, we propose a novel hypergraph-based model, namely \underline{M}ulti-view \underline{H}ypergraph \underline{C}ontrastive \underline{P}olicy \underline{L}earning (MHCPL). Spe-cifically, MHCPL dynamically filters social information according to the interactive conversation and builds a dynamic multi-view hypergraph with three types of multiplex relations from different views: the user likes/dislikes the items that satisfy some attribute (\emph{Like/Dislike view}) and the user shares the preferences for items with some friend (\emph{Social view}). The multiplex relations in each view are successively connected according to their generation order in the interactive conversation. 
A hierarchical hypergraph neural network is proposed to learn user preferences by integrating information of the graphical and sequential structure from the dynamic hypergraph.
Furthermore, a cross-view contrastive learning module is proposed with two terms to maintain the inherent characteristics and the correlations of user preferences from different views.
%Furthermore, a contrastive learning module comparing user preferences learned from different views is developed to help integrate user preferences from different views. 
Extensive experiments conducted on Yelp
and LastFM demonstrate that MHCPL
outperforms the state-of-the-art methods.

\textbf{Our contributions} of this work are summarized as follows:
\begin{itemize}
    \item \textbf{General Aspects:} We emphasize the importance of multiplex relations and investigate three views to integrate interactive conversation, item knowledge, and social influence for dynamic user preference learning in CRS.
    
    \item \textbf{Novel Methodologies:} We propose the model MHCPL to timely filters social information according to the interactive conversation and learns dynamic user preferences with three types of multiplex relations from different views. Moreover, a cross-view contrastive learning module is proposed to maintain the inherent characteristics and the correlations of user preferences from different views.
    
    \item \textbf{Multifaceted Experiments:} We conduct extensive experiments on two benchmark datasets. The results demonstrate the advantage of our MHCPL in better dynamic user preference learning, which shows the effectiveness of our MHCPL for conversational recommendation.
\end{itemize}
