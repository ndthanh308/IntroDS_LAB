\documentclass[sigconf,natbib=true]{acmart}

\usepackage{caption}
\usepackage{subcaption}
\usepackage{courier}  % DO NOT CHANGE THIS
\usepackage{algorithm}
\usepackage{algorithmic}
\usepackage{amsmath}
\usepackage{makecell}
\usepackage{tabularx}
\usepackage{threeparttable}
\usepackage{multirow}
\usepackage{booktabs}
\usepackage{marvosym}
\usepackage{bm}
\usepackage{mwe}
% \usepackage{titlesec}
\usepackage{titletoc}
\usepackage{booktabs}
% \usepackage{hyperref}
\usepackage{newfloat}
\usepackage{listings}
\usepackage{makecell}
\usepackage{threeparttable}
\usepackage{pdfpages}
\usepackage{import}
\usepackage[titletoc]{appendix}
% \usepackage{caption}
\usepackage{multirow}
\usepackage{graphicx}
 \usepackage{enumitem}
 
\newcommand{\eg}{\emph{e.g.,}\xspace}
\newcommand{\rf}{\emph{rf.}\xspace}
\newcommand{\wrt}{\emph{w.r.t.}\xspace}
\newcommand{\ie}{\emph{i.e.,}\xspace}
\newcommand{\etc}{\emph{etc.}\xspace}
\newcommand{\etal}{\emph{et al.}\xspace}
\newcommand{\aka}{\emph{a.k.a.,}\xspace}
\newcommand{\wo}{\emph{w / o }\xspace}
\let\oldhat\hat
\renewcommand{\vec}[1]{\mathbf{#1}}
\renewcommand{\hat}[1]{\oldhat{\mathbf{#1}}}
%%
%% \BibTeX command to typeset BibTeX logo in the docs
\AtBeginDocument{%
  \providecommand\BibTeX{{%
    \normalfont B\kern-0.5em{\scshape i\kern-0.25em b}\kern-0.8em\TeX}}}


%% values in them; it is your responsibility as an author to replace
%% the commands and values with those provided to you when you
%% complete the rights form.

\copyrightyear{2023}
\acmYear{2023}
\setcopyright{acmlicensed}\acmConference[SIGIR '23]{Proceedings of the 46th International ACM SIGIR Conference on Research and Development in Information Retrieval}{July 23--27, 2023}{Taipei, Taiwan}
\acmBooktitle{Proceedings of the 46th International ACM SIGIR Conference on Research and Development in Information Retrieval (SIGIR '23), July 23--27, 2023, Taipei, Taiwan}
\acmPrice{15.00}
\acmDOI{10.1145/3539618.3591737}
\acmISBN{978-1-4503-9408-6/23/07}

\begin{document}

\title{Multi-view Hypergraph Contrastive Policy Learning for Conversational Recommendation}


\author{Sen Zhao}
\affiliation{%
  \institution{CCIIP Laboratory, Huazhong
University of Science and Technology
Joint Laboratory of HUST and Pingan
Property \& Casualty Research (HPL)}
  %\city{Wuhan}
  \country{China}}
\email{senzhao@hust.edu.cn}

\author{Wei Wei*}
\thanks{*Corresponding author.}
\affiliation{%
  \institution{CCIIP Laboratory, Huazhong
University of Science and Technology
Joint Laboratory of HUST and Pingan
Property \& Casualty Research (HPL)}
  %\city{Wuhan}
  \country{China}}
\email{weiw@hust.edu.cn}

\author{Xian-Ling Mao}
\affiliation{%
  \institution{Beijing Institute of Technology}
  %\city{Wuhan}
  \country{China}}
\email{maoxl@bit.edu.cn}

\author{Shuai Zhu}
\affiliation{%
  \institution{Ant Group}
  %\city{Wuhan}
  \country{China}}
\email{zs261988@antgroup.com}

\author{Minghui Yang}
\affiliation{%
  \institution{Ant Group}
  %\city{Wuhan}
  \country{China}}
\email{minghui.ymh@antgroup.com}

\author{Zujie Wen}
\affiliation{%
  \institution{Ant Group}
  %\city{Wuhan}
  \country{China}}
\email{zujie.wzj@antgroup.com}

\author{Dangyang Chen}
\affiliation{%
  \institution{Ping An Property \& Casualty
Insurance Company of China, Ltd}
  %\city{Wuhan}
  \country{China}}
\email{chendangyang273@pingan.com.cn}

\author{Feida Zhu}
\affiliation{%
  \institution{Singapore Management University}
  %\city{Wuhan}
  \country{Singapore}}
\email{fdzhu@smu.edu.sg}
\renewcommand{\shortauthors}{Sen Zhao et al.}
\begin{abstract}
  Conversational recommendation systems (CRS) aim to interactively acquire user preferences and accordingly recommend items to users. Accurately learning the dynamic user preferences is of crucial importance for CRS. Previous works learn the user preferences with pairwise relations from the interactive conversation and item knowledge, while largely ignoring the fact that factors for a relationship in CRS are multiplex. Specifically, the user likes/dislikes the items that satisfy some attributes (\emph{Like/Dislike view}). Moreover social influence is another important factor that affects user preference towards the item (\emph{Social view}), while is largely ignored by previous works in CRS. The user preferences from these three views are inherently different but also correlated as a whole. The user preferences from the same views should be more similar than that from different views. The user preferences from \emph{Like View} should be similar to \emph{Social View} while different from \emph{Dislike View}.
  To this end, we propose a novel model, namely \underline{M}ulti-view \underline{H}ypergraph \underline{C}ontrastive \underline{P}olicy \underline{L}earning (MHCPL). Specifically, MHCPL timely chooses useful social information according to the interactive history and builds a dynamic hypergraph with three types of multiplex relations from different views. The multiplex relations in each view are successively connected according to their generation order in the interactive conversation.
A hierarchical hypergraph neural network is proposed to learn user preferences by integrating information of the graphical and sequential structure from the dynamic hypergraph.
A cross-view contrastive learning module is proposed to maintain the inherent characteristics and the correlations of user preferences from different views. Extensive experiments conducted on benchmark datasets demonstrate that MHCPL outperforms the state-of-the-art methods.
\end{abstract}

\begin{CCSXML}
<ccs2012>
    <concept>
        <concept_id>10002951.10003317.10003331</concept_id>
        <concept_desc>Information systems~Users and interactive retrieval</concept_desc>
        <concept_significance>500</concept_significance>
    </concept>
    <concept>
        <concept_id>10002951.10003317.10003347.10003350</concept_id>
        <concept_desc>Information systems~Recommender systems</concept_desc>
        <concept_significance>500</concept_significance>
    </concept>
</ccs2012>
\end{CCSXML}

\ccsdesc[500]{Information systems~Recommender systems}
%%
%% The code below is generated by the tool at http://dl.acm.org/ccs.cfm.
%% Please copy and paste the code instead of the example below.
%%

%%
%% Keywords. The author(s) should pick words that accurately describe
%% the work being presented. Separate the keywords with commas.
\keywords{Conversational Recommendation, Reinforcement Learning, Graph Representation Learning}

%% A "teaser" image appears between the author and affiliation
%% information and the body of the document, and typically spans the
%% page.

%%
%% This command processes the author and affiliation and title
%% information and builds the first part of the formatted document.
 \maketitle
The problem of the presence or absence of phase transition is central in statistical mechanics. To prove the existence of phase transition, the standard idea is to define a notion of contour and use \textit{Peierls' argument} \cite{Peierls.1936}. In the usual Ising model \cite{Ising_25}, particles of the system interact only with their nearest-neighbors. On ferromagnetic long-range Ising models \cite{Anderson_Yuval_69}, there is interaction between each pair of spins in the lattice. The Hamiltonian of the model is given formally by
\begin{equation*}
    H(\sigma) = - \sum_{x,y\in \Z^d}J_{xy}\sigma_x\sigma_y,
\end{equation*}
where $J_{xy}=J|x-y|^{-\alpha}$, $J>0$, $\alpha > d$. It is well-known that the Peierls' argument in dimension 2 implies phase transition for Ising models with nearest-neighbors or long-range interactions when $d\geq 2$, using correlation inequalities. For the unidimensional lattice, it was known that short-range models do not present phase transition. In the long-range case, a different behavior was expected depending on the exponent $\alpha$ (see \cite{Kac_Thompson_69}), but the problem was challenging since contours were first created as multidimensional objects.

In dimension $d=1$, phase transition was proved first in 1969 by Dyson \cite{Dyson.69}, for $\alpha \in (1,2)$, by proving phase transition in an auxiliary model and then using correlation inequalities. In 1982, Fr{\"o}hlich and Spencer \cite{Frohlich.Spencer.82} introduced a notion of one-dimensional contours and then applied the Peierls' argument to show phase transition for the critical value $\alpha = 2$. These contours were inspired by the multiscale techniques previously introduced to study the Berezinskii-Kosterlitz-Thouless transition in two-dimensional continuous spin systems \cite{FS81}. Later, Cassandro, Ferrari, Merola and Presutti  \cite{Cassandro.05} extended the contour argument previously available for $\alpha=2$ to exponents $\alpha\in (3-\frac{\ln 3}{\ln 2}, 2)$, with the additional restriction that the nearest-neighbor interaction is strong, i.e.,  ${J(1)\gg 1}$; this restriction was removed for a subclass of interactions in \cite{Bissacot.Endo.18}. Further results were obtained using contour arguments, such as the decay of correlations, cluster expansions, phase transition with random interactions, etc; some references with these results are \cite{ Cassandro.Merola.Picco.17, Cassandro.Merola.Picco.Rozikov.14, Imbrie.82, Imbrie.Newman.88, Johansson.91}. 

In the multidimensional setting ($d\geq 2$), Ginibre, Grossmann, and Ruelle, in \cite{Ginibre.Grossmann.Ruelle.66}, proved the phase transition for $\alpha > d+1$, using an enhanced version of Peierls' argument and the usual contours. Park proposed a different notion of contour for long-range systems in \cite{Park.88.I, Park.88.II}, extending the Pirogov-Sinai theory available for short-range interactions assuming $\alpha > 3d+1$, although he can also consider Potts models with his methods. Some results in the literature suggest that truly long-range effects appear only when $d < \alpha \leq d+1$, see for instance, \cite{Biskup_Chayes_Kivelson_07}. Recently, Affonso, Bissacot, Endo and Handa \cite{Affonso.2021}, inspired by the ideas from Fr{\"o}hlich and Spencer in \cite{FS81, Frohlich.Spencer.82}, introduced a version of multiscale multidimensional contour and proved phase transition by a contour argument in the whole region $\alpha > d$. They can consider long-range Ising models with deterministic decaying fields, first introduced in the context of nearest-neighbor interactions in \cite{Bissacot_Cioletti_10}. For these models, the lack of analyticity of the free energy does not imply phase transition since these models have the same free energy as the models with zero field. It is expected that fields decaying slowly imply uniqueness. In this setting, a contour argument is useful for proofs of phase transitions as well for uniqueness, some papers with models with deterministic decaying fields are \cite{Aoun_Ott_Velenik_23, Bissacot_Cass_Cio_Pres_15, Bissacot.Endo.18, Cioletti_Vila_2016}.

The Random Field Ising model (RFIM) \cite{Imry.Ma.75} is the nearest-neighbor Ising model with an additional external field acting on each site $(h_x)_{x\in\Z^d}$ that is a family of i.i.d. Gaussian random variable with mean 0 and variance 1. Formally, the Hamiltonian of the model is given by
\begin{equation*}
    H(\sigma) = - \sum_{\substack{x,y\in \Z^d \\|x-y|=1}}J\sigma_x\sigma_y  - \varepsilon\sum_{x\in\Z^d}h_x\sigma_x,
\end{equation*}
where $J>0$, $\varepsilon>0$, $\alpha > d$ and $d \geq 1$. A detailed account of the history of the phase transition problem for this model, as well as detailed proofs, was given in \cite{Bovier.06}. Here we present a brief overview.

During the 1980s, the question of the specific dimension where phase transition for the RFIM should happen attracted much attention and was a topic of heated debate. Two convincing arguments were dividing the physics community. One of them, due to Imry and Ma \cite{Imry.Ma.75}, was a non-rigorous application of the Peierls' argument together with the use of the isoperimetric inequality. The key idea of Peierls' argument is to define a notion of contour and calculate the energy cost of "erasing" each contour, i.e., the energy cost of flipping all spins inside the contour. When there is no external field, that energy necessary to flip the spins in a region $A\subset \Z^d$ is of the order of the boundary $|\partial A|$. When we add an external field, we get an extra cost depending on this field. Imry and Ma argued that this cost should be approximately $\sqrt{|A|}$, which is smaller than $|\partial A|$ for all regions only when $d\geq 3$, so this should be the region where phase transition occurs. The other argument, due to Parisi and Sourlas \cite{Parisi.Sourlas.79}, based on dimensional reduction, predicted that the $d$-dimensional RFIM would behave like the $d-2$-dimensional nearest-neighbor Ising model, therefore presenting phase transition only when $d\geq 4$. 

The question was settled by two celebrated papers showing that Imry and Ma's prediction was correct. First, in 1988, Bricmont and Kupiainen \cite{Bricmont.Kupiainen.88} showed that there is phase transition almost surely in $d\geq3$, for low temperatures and variance $\varepsilon$ small enough. Their proof uses a rigorous renormalization group analysis for the short-range case and it is considered involved. Still, they claimed that the result works for any model with a suitable contour representation and centered sub-gaussian external field. Later on, Aizenman and Wehr \cite{Aizenman.Wehr.90} proved uniqueness for $d\leq 2$. For detailed proofs of these results, we refer the reader to \cite{Bovier.06} (see also \cite{Berretti.85, Camia.18, Frohlich.Imbre.84,  Klein.Masooman.97} for more uniqueness results). 

Recently, Ding and Zhuang, see \cite{Ding2021}, provided a simpler proof of the phase transition, not using RGM. And in  \cite{Ding.Liu.Xia.22}, Ding, Liu and Xia proved that if $\beta_c(d)$ is the critical inverse of the temperature of the Ising model with no field, for all $\beta>\beta_c(d)$ there exists a critical value $\varepsilon_0(d, \beta)$ such that the RFIM with $\varepsilon \leq \varepsilon_0$ presents phase transition. 

In the present paper, we are considering a long-range Ising model with a random field, whose Hamiltonian is given formally by
\begin{equation*}
    H(\sigma) = - \sum_{x,y\in \Z^d}J_{xy}\sigma_x\sigma_y - \varepsilon\sum_{x\in\Z^d}h_x\sigma_x,
\end{equation*}
where $J_{xy}=J|x-y|^{-\alpha}$, $J, \varepsilon>0$, $\alpha > d$ and $h_x\in\mathbb{R}$, $d\geq 3$.
Until now, the only known result in the long-range setting is for the one-dimensional long-range Ising model with a random field, by Cassandro, Orlandi, and Picco \cite{Cassandro.Picco.09}. They used the contours of \cite{Cassandro.05} to show the phase transition for the model when $\alpha\in (3-\frac{\ln 3}{\ln 2}, \frac{3}{2})$, under the assumption $J(1) \gg 1$. We stress that, as remarked by Aizenman, Greenblatt, and Lebowitz \cite{Aizenman_Greenblatt_Lebowitz_2012}, although their argument does not work for the whole region for the exponent $\alpha$, the phase transition holds for values close to the critical value $\alpha=3/2$, since by the Aizenman-Wehr theorem we know that there is uniqueness for $\alpha>3/2$.

The argument from Ding and Zhuang in \cite{Ding2021}, for $d\geq3$, involves controlling the probability of a bad event, which is closely related to controlling the quantity $$\sup_{\substack{0\in A\subset\Z^d \\ A \text{ connected }}}\frac{\sum_{x\in A}h_x}{|\partial A|},$$ known as the greedy animal lattice normalized by the boundary. The greedy animal lattice normalized by the size, instead of the boundary, was extensively studied for general distributions of $(h_x)_{x\in\Z^d}$, see \cite{Cox_Gandolfi_Griffin_Kesten_93, Gandolfi_Kesten_94, Hammond_06, Martin_02}. When we normalize by the boundary, an argument by Fisher, Fr\"{o}hlich and Spencer \cite{FFS84} shows that the expected value of the greedy animal lattice is constant. In dimension $d=2$, the expected value is not finite, see \cite{Ding.Wirth.20}. The supremum is taken over connected regions containing the origin since the interiors of the usual Peierls contours are of this form.


For the long-range model, the interior of contours is not necessarily connected. In fact, long-range contours may have considerably large diameters with respect to their size, so their interiors can be very sparse. To avoid this, we define contours, strongly inspired by the $(M,a,r)$-partition in \cite{Affonso.2021}, using a multiscaled procedure that assures that the contours have no cluster with small density.  With them, we generalize the arguments by Fisher-Fr\"{o}hlich-Spencer \cite{FFS84}, and prove that the expected value of the greedy animal lattice is constant, even considering regions not necessarily connected in the supremum. Then, we prove the phase transition for $d\geq 3$. The main result of this paper is the following.
\begin{theorem*}Given $d\geq 3$, $\alpha>d$, there exists $\beta_c\coloneqq\beta(d, \alpha)$ and $\varepsilon_c\coloneqq\varepsilon(d, \alpha)$ such that, for $\beta >\beta_c$ and $\varepsilon\leq \varepsilon_c$, the extremal Gibbs measures $\mu_{\beta, \varepsilon}^+$ and $\mu_{\beta, \varepsilon}^-$ are distinct, that is, $\mu_{\beta, \varepsilon}^+ \neq \mu_{\beta, \varepsilon}^-$ $\mathbb{P}$-almost surely. Therefore the long-range random field Ising model presents phase transition.
\end{theorem*}

This paper is divided as follows. In Section 2, we define the model and the contours, and suitable generalizations to the constructions in \cite{Affonso.2021} are introduced.  In Section 3, we define two bad events of the external field and prove that they occur with a small probability.  In Section 4, we present the proof of the phase transition.
Previous work has shown that integration over learned functions is a powerful tool. It has been used in many disparate applications, from constraining monotonic functions \cite{monotonic} to computing the output of a network over a distribution of parameters \cite{bayesian}. The entire field of Neural ODEs is predicated on the ability to integrate a learned dynamics function over time \cite{neuralode, ffjord, graphode}. However, all of these methods rely on \textit{numerical} integration, which is computationally expensive and not exact.

Another work in the same vein as our method is Hamiltonian Neural Networks \cite{hamiltonian}, which parameterises a vector field which conserves energy by formulating it as the symplectic gradient of an energy function. However, this method is only used as a means of constraining the learned function, not as a technique for integration.
\section{Methodology}
\label{sec:method}
% Figure environment removed


%\vspace*{0.2cm}\noindent\textbf{Problem formulation.} 
% Let $O$ be the set of Objects,  $S$ be the set of
%  States and $\textit{I}$ the set of
%   Images which consists of the disjoint sets ${I^{S}}$ and ${I^{U}}$ that are used during the training and testing phase respectively. 
%    Each image $i_{k} \in \textit{I}$  contains an object $o_{i} \in O$ which  is situated in a state $s_{j} \in S$. The OSC task deals with the yielding  of a  predicted state label $sp_{j} \in S$ for an image $i_{k} \in {I^U}$ that has been given as an input. In the zero-shot variation of OSC, ${S^{S}} 
%   \not \supseteq {S^{U}}$, i.e. some of the states contained in the testing images do not appear in the training images. 
% Let $O$ denote a set of objects, 
% \textcolor{red}{$S^S$ as the set of known object states found in the training images, $S^U$ as the test, yet unknown, object state labels} and $I$ the set of images, which is partitioned into the training set $I^T$ and the test set $I^U$. 
% % Each image $i \in I$ contains an object $o \in O$ in a state $s \in S$. 
% \textcolor{red}{Each image $i \in I^T$ contains an object $o \in O$ in a state $s \in S^S$, while an image $i \in I^U$ contains an object $o \in O$ in any state $s \in S = \{S^S \cup S^U\}$. }
% The goal of OSC is to predict the state $s \in S$, given the object $o$ in $i \in I^U$. In the zero-shot variation of OSC, the set of states observed in the test images $S^U$ is not a subset of the set of states observed in the training images $S^S$, i.e., there exists some states in the test image set that do not appear in the training set. Furthermore, the task should be addressed in an object-agnostic manner, i.e. no information concerning the object classes is to be utilized explicitly.  However,  although the set of object classes does not directly affect the task of OaSC, its size is proportional to the complexity of the problem. 
% The workflow of the proposed method is shown in \autoref{fig:pipeline}.

Let $O$ denote a set of objects, $S$ denote the set of states and $I$ denote the set of images, which is partitioned into the training set $I^T$ and the testing set $I^U$. Each image $i \in I$ contains an object $o \in O$ in a state $s \in S$. 
The goal of OSC is to predict the state $s \in S$, given the object $o$ in $i \in I^U$. In the zero-shot variation of OSC, the set of states observed in the test images $S^U$ is not a subset of the set of states observed in the training images $S^S$, i.e., there exists some states in the test image set that do not appear in the training set. Furthermore, the task should be addressed in an object-agnostic manner, i.e. no information concerning the object classes is to be utilized explicitly.  However,  although the set of object classes does not directly affect the task of OaSC, its size is proportional to the complexity of the problem. 
The workflow of the proposed method is shown in \autoref{fig:pipeline}.




% and will be analyzed in the following sections.

% \vspace*{0.2cm}\noindent\textbf{Approach.}

\subsection{Overview}
% Our method is inspired by works that address the problem of zero-shot object classification \cite{}. The main idea behind this line of work is that the necessary information for the classification of the unseen classes can be found in a Knowledge Graph (KG) if processed appropriately by a Graph Neural Network (GNN). Obviously, the most crucial component of this approach lies in the combination of the visual information stemming from the training images and referring to the seen classes with the semantic information stemming from the KG  and referring to the unseen classes.

We are inspired by prior research on zero-shot object classification and leverage the potential of KGs and GNNs to classify previously unseen objects~\cite{Kampffmeyer2019,nayak:tmlr22}. 
The core idea is that semantic information that is stored in the KG can be used by GNNs to learn graph embeddings that can be utilized jointly with visual information extracted from training images. 
This enables the model to generalize to new object classes by leveraging the semantic and contextual information encoded in the graph embeddings of the KG.

% More in detail, the GNN architecture is adopted to the architecture of the Classifier  that is used for the training on seen classes, the GNN last layer has the same size  with the Classifier last layer. This way the GNN can produce semantic embedding features that correspond to all the classes, both seen and unseen, that will be encountered during the inference. These embedding features  replace the last layer of the Classifier. Holding this layer fixed, the body of the Classifier is then fine-tuned with the training images.

GNNs are designed to operate on graph-structured data, such as KGs~\cite{kipf2016semi,Monka2022}. KGs are typically represented as labeled multi-graphs, where nodes correspond to entities, and edges represent entity relationships. GNNs process this graph by iteratively aggregating information from neighboring nodes, using neural network-based operations.

At each iteration, a GNN receives a feature vector for each graph node, which is initially set to the node's embedding vector. Then, the GNN performs a message-passing step that aggregates information from neighboring nodes, based on the edge weights and the features of the nodes. This message-passing operation can be formulated as a neural network layer, which applies a learnable function to the features of the neighboring nodes and returns an aggregated message for each node. After the message-passing step, the GNN updates the node features by applying a learnable transformation that takes into account the original features of the node and the received messages from its neighbors. This updated feature vector is then passed to the next iteration of the message-passing step. The process continues until a fixed number of epochs or convergence.
%%%AAA: Endexetai na mas rethrown gia tis times aytwn twn parametrwn?
% KP edw anaferetai genika mia diadikasia GNN training. Na anaferoume edw times parametrwn h sto 4 - see implementation details ?

The proposed method leverages GNN training using a visual classifier that is trained on seen state classes as supervision. In particular, the last layer of the GNN is designed to have the same size as the last layer of the classifier. This enables the GNN to generate semantic embedding features that correspond to all classes, including both seen and unseen classes that will be encountered during inference. Subsequently, the semantic embedding features replace the last layer of the classifier while this layer is kept fixed. The body of the classifier is then fine-tuned with the training images to optimize the overall model for state recognition.

% \vspace*{0.2cm}\noindent\textbf{GNN Details.} 
Overall, we experimented with four different model architectures and opted for the Transformer Graph Convolutional network (Tr-GCN)~\cite{nayak:tmlr22}. Further details are provided in Section~\ref{sec:abl} and the supplementary material of this work. 
The Tr-GCN mode is capable of combining input sets non-linearly by utilizing multilayer perceptrons and self-attention. Tr-GCN refers to an inductive model that can learn node representations by aggregating local neighborhood features allowing the trained model to make predictions on new graph structures without retraining. 
We leverage the aforementioned property of the Tr-GCN to train a permutation invariant non-linear aggregator that captures the intricate structure of a common sense knowledge graph. 
% , rendering it well-suited for zero-shot learning. 
% It is worth noting that a similar network architecture has been effectively employed for zero-shot object classification~\cite{nayak:tmlr22}.

% A critical aspect of the proposed method involves calibrating the weights of the GNN in a manner that its predictions in the semantic space are useful for the classifier deployed in the visual space. To accomplish this, we adopt an approach based on prior research \cite{Kampffmeyer2019, Wang2018b, nayak:tmlr22} that involves learning the semantic class representations by minimizing the L2 distance between the learned class representations and the weights of a fully connected layer in a ResNet classifier pre-trained on the ILSVRC 2012 dataset \cite{russakovsky2015imagenet}. Once the class representations are learned, we fix them and fine-tune the ResNet backbone using the training images from the dataset.




% \vspace*{0.2cm}\noindent\textbf{Building of the KG.}
% The KG is created by the querying  of a common sense repository. The repositories that we are ConceptNet \cite{} and WordNet\cite{}. The procedure takes place as follows. Initially we create a set of nodes that correspond to the target stace classes. Subsequently, the repository is queried for each of these nodes and its neighbours in the repository of  added to the KG if  certain criteria are met (see ablation section for more details). This procedure is repeated for the newly added nodes and henceforth until a number of hops has been reached.  

\subsection{The proposed OaSC approach}
\label{sec:pipeline}
Overall, the proposed method consists of four stages, as shown in \autoref{fig:pipeline}: (1) construction of the KG, (2) GNN training and learning of semantic graph embeddings, (3) fine-tuning of the visual classifier and (4) deployment of the fine-tuned state classifier.

\vspace*{0.0cm}\noindent\textbf{Construction of the KG (Stage 1)}:
To create the KG, we query a common sense repository to compile a generic solution and to avoid the construction of a task-specific KG, tailored to the entities at hand and their relationships. First, a set of nodes that correspond to the words of the target state classes $S^U$ and $S^S$ is generated. Then, we query the repository for each of these nodes and add their neighbors in the KG, if they meet specific criteria (see also Section~\ref{sec:abl}). This process is repeated for the newly added nodes until a specified number of node hops is reached.

This technique for building a generic KG offers several advantages in comparison to other problem-specific approaches. First, it allows the same KG to be used for different variations of the task. It also enables transfer learning since KGs can be reused to tackle other related problems. Moreover, the construction of such a KG does not rely on expert knowledge. Besides, the structured representation of relationships between entities and concepts that KGs provide can be leveraged to generate robust embeddings for zero-shot learning.
% which is expensive and time-consuming.  
The trade-off is that such KGs are prone to noisy information in the used repositories. 

% In comparison, language models, such as BERT~\cite{devlin2018bert}, often rely on large amounts of unstructured text data to generate embeddings. While language models are highly effective at capturing semantic relationships between words and phrases, they can also be prone to create associations between concepts that are not actually related. This can lead to noisy or unreliable embeddings, which can in turn degrade the performance of zero-shot learning models. By contrast, the structured nature of KGs allows for more accurate and precise capture of relationships between entities and concepts, leading to more robust embeddings that can improve the accuracy and reliability of zero-shot learning models~\cite{brown2020language}.


\vspace*{0.0cm}\noindent\textbf{Computation of  Graph Embeddings (Stage 2)}:
% Given the KG constructed in Stage 1, a word features embedding matrix corresponding to the KG nodes is created by utilizing the pre-computed word features of GloVe~\cite{pennington2014glove}. 
% % Subsequently,  random walks are performed in the KG and a sample of neighbors for each node is obtained.
% By taking the word features embedding matrix, the KG topology, and a target node as inputs, the GNN estimates the node's embeddings: the features of the node and its neighbors are  aggregated and submitted to a series of convolutions and pooling operations before the  output is produced in the form of a feature vector, the length of which is tailored to be the same as the size dimension of the last layer of a ResNet-101 classifier. 
% This procedure is repeated for all KG nodes and results in the computation of the semantic embeddings for all target state classes with each embedding being a feature vector of length equal to 2048. By combining these embeddings for the \mathcal{d} target classes  a  $ d \times 2048$ features matrix is created which serves as the last layer of a CNN classifier that is utilized during Stages 3 and 4.
% which serves as the last layer of a CNN classifier that is utilized during Stages 3 and 4 .
% We employ an established approach~\cite{Kampffmeyer2019, Wang2018b} that involves training of a transformer-based Graph Convolutional Model using graph embeddings of a set of semantic entities acquired by a common sense repository by minimizing the L2 distance between the learned class representations and the weights of a fully connected layer in a ResNet classifier, pre-trained on the ILSVRC 2012 dataset~\cite{russakovsky2015imagenet}, ensuring that the semantic class representations are meaningfully embedded.
We employ an established approach~\cite{Kampffmeyer2019, Wang2018b} that involves the training of a transformer-based Graph Convolutional Network (GCN)
 \textcolor{black}{ that utilizes a KG as input  %Training is performed %using features of a set of semantic entities acquired by a common sense repository, \textcolor{red}{(e.g. the ConceptNet, CSKG, or other)}  
 and generates an embedding vector for each node of the  KG. %. For the production of the embeddings vectors the GCM employs a sequence of transformations to the semantic features that correspond to the concepts linked to each node.
This process defines pre-computed GloVe word, i.e. semantic features~\cite{pennington2014glove}, for the KG nodes with each node representing a concept class.
% To compute node embeddings, the GNN is applied to encode the KG topology and the word feature embedding matrix. 
The GNN  aggregates each node's and its neighbors' features through a sequence of convolutions and pooling operations. %This results in the generation of a feature vector having a length equal to the dimension of the last layer in a visual CNN-based classifier that is instantiated using a ResNet-101 model. 
%By pre-training the visual classifier in a set of target classes 
The visual classifier is pre-trained on a set of target classes and using the weights of its fully connected layer, the GCN learns to produce visual feature representations, i.e. visual embeddings,  corresponding to the concept classes of the KG`s nodes.}
\textcolor{black}{
Formally,  the training involves the minimization of the L2 distance   $\mathcal{L_G}$ between the generated visual embeddings and the ground truth visual embeddings stemming from the visual classifier.} 
\textcolor{black}{In notation, 
\begin{equation}
\mathcal{L_G} = \frac{1}{2N} \sum_{n \in N} \sum_{p \in P} (W_{n,p} - \tildea{W}_{n,p})^2,
 \end{equation}
where $\tildea{W} \in \mathbb{R}^{|N|xP}$ denotes the weights of the GCN for the set of known concept classes $N$ and the dimensionality $P$ of the weight vector. Similar to~\cite{Kampffmeyer2019}, the ground truth weights, denoted as $W \in \mathbb{R}^{|N|xP}$, are obtained by extracting the last layer weights of a pre-trained CNN.}
% This process is repeated for all KG nodes corresponding to $S^U$ and $S^S$, generating semantic graph embeddings for all target state classes. 
%Each embedding comes in the form of a feature vector of length 2048. 


%By combining these embeddings for the $d$ target classes, a  $d \times 2048$ features matrix is defined that is integrated as the final layer of the visual CNN-based classifier that is employed in Stages~3-4.
%A critical aspect of this process is adjusting the GNN weights to align its predictions with the semantic space. This ensures that the semantic embeddings effectively aid the classifier used in Stages 3 and 4, operating in the visual space. 



\textcolor{black}{ 
The KG  given as an input to the GCN model is a hierarchical graph created for the requirements of the   ILSVRC 2012 dataset~\cite{russakovsky2015imagenet} and represents the WordNet hierarchical structure of the $1,000$ classes comprising the dataset. These 1,000 concept labels constitute the set of classes upon which the visual classifier used for the extraction of the ground truth visual embeddings is pre-trained.
}
% A critical aspect of this process is adjusting the GNN weights to align its predictions with the semantic space. This ensures that the semantic embeddings effectively aid the classifier used in Stages 3 and 4, operating in the visual space. 
% The concepts  that are used for the training refer to a set of 1K object classes of the ILSVRC 2012 dataset~\cite{russakovsky2015imagenet}, while the pre-trained ResNet101-based classifier is used for supervision to ensure that the GNN outputs, thus the semantic object class representations, are meaningfully embedded into the visual feature space. 
After the training is completed, the GCN model is employed to process the KG (constructed in Stage 1) and generate visual embeddings for the KG nodes that correspond to the object state classes,  by taking as input the  KG that was constructed during Stage 1. Each embedding comes in the form of a feature vector of length 2048, i.e. dimension of the last layer of the  pre-trained visual CNN-based classifier.
By combining these embeddings for the $d$ target classes, a  $d \times 2048$ features matrix is defined that is integrated as the final layer of the visual CNN-based classifier that is employed in Stages~3-4.
% First, using their pre-computed GloVe word features~\cite{pennington2014glove}, a matrix of word, i.e. semantic, features embeddings is defined for each of the KG nodes.
% % To compute node embeddings, the GNN is applied to encode the KG topology and the word feature embedding matrix. 
% Subsequently,  the GNN takes as input every target node that corresponds to any class in $S^U$ and $S^S$ and aggregates the features about the node and its neighbors through a sequence of convolutions and pooling operations. This results in the generation of a feature vector having a length equal to the dimension of the last layer in the visual CNN-based classifier that is instantiated using a ResNet-101 model.
% % This process is repeated for all KG nodes corresponding to $S^U$ and $S^S$, generating semantic graph embeddings for all target state classes. 
% Each embedding comes in the form of a feature vector of length 2048. By combining these embeddings for the $d$ target classes, a  $d \times 2048$ features matrix is defined that is integrated as the final layer of the visual CNN-based classifier that is employed in Stages~3-4.
% A critical aspect of this process is adjusting the GNN weights to align its predictions with the semantic space. This ensures that the semantic embeddings effectively aid the classifier used in Stages 3 and 4, operating in the visual space. 

% A critical aspect of this procedure involves calibrating the weights of the GNN to embed its predictions in the semantic space, i.e. semantic embeddings, are useful for the classifier deployed in the visual space during Stages 3 and 4. To accomplish this, we adopt an approach based on prior research~\cite{Kampffmeyer2019, Wang2018b} that involves learning the semantic class representations by minimizing the L2 distance between the learned class representations and the weights of a fully connected layer in a ResNet classifier pre-trained on the ILSVRC 2012 dataset~\cite{russakovsky2015imagenet}.  

\vspace*{0.0cm}\noindent\textbf{Fine-tuning of the Visual Classifier (Stage 3)}:
The estimated semantic embeddings are integrated into a visual CNN classifier that relies on the ResNet backbone and is initially pre-trained for object classification. The embeddings serve as the final layer of the network, encapsulating the representations essential for predicting the train state classes $S^S$. To enable this adaptation, the visual classifier undergoes re-training, specifically tailored to the classification of the train classes. 
During this fine-tuning process, input images $I^T$ contain states sourced exclusively from the training set $S^S$, i.e. ``seen states''. The primary objective is to harness the classifier capabilities to classify these familiar states, accurately. Notably, fine-tuning involves keeping the weights of the last layer fixed, safeguarding the integrity of the acquired semantic representations from Stage 2. Consequently, adjustments are only applied to the weights of preceding layers to ensure they effectively match the ``frozen'' last-layer weights.
% Apart from this detail, the procedure takes place in the same manner as the training of a CNN classifier.
% in every training epoch a loss is computed the value of which guides the update of all layers weights except the last one. 
Following the notation introduced 
\textcolor{black}{in the beginning of Section~\ref{sec:method}, the loss function is defined as:}
\begin{equation}
% \mathcal{L} = -\sum_{i \in S^{S}} y_i \cdot \log(P(y=i|X))
\mathcal{L_V} = -\sum_{s \in S^S, i \in I^{T}} y_s \cdot \log(P(s|i)),
 \end{equation}
\textcolor{black}{for the predicted \textit{$y_s$} state label in the \textit{$S^S$} set of state labels. $P(s|i)$ denotes the probability of state label \textit{s} based on the softmax vector given an image \textit{i} from the $I^T$ training set.}

\noindent\textbf{Zero-shot OaSC (Stage 4)}:
Upon the completion of fine-tuning, the visual state classifier can be utilized for  prediction by choosing the most likely class
\begin{equation}
% \^y = \arg\max_{i \in S} \left( P(y=i|X) \right)
\hat{y} = \arg\max_{s \in S^U i \in I^{U}} \left( P(s|i) \right),
\end{equation}
\textcolor{black}{where $I^U$ denotes the test image set and $S^U$ the test state classes respectively.} 
We highlight that the classifier is well-suited for predicting either only unseen classes, i.e. zero-shot classification, or both seen and unseen classes, i.e. generalized zero-shot classification.
\vspace{-.15cm}
% \subsection{Pipeline}

% Overall, the pipeline of our method consists of four stages (\autoref{fig:pipeline}}). During \textbf{Stage 1}, the KG is constructed.

% \vspace*{0.2cm}\noindent\textbf{Construction of the KG (Stage 1)}:
% The KG creation process involves querying a common sense repository to enable generalization instead of creating a custom KG tailored to specific entities and relationships. Initially, nodes corresponding to the target state classes are generated. The repository is then queried for each node, and neighbors meeting specific criteria are added to the knowledge graph. This process continues for the newly added nodes until a specified number of hops is reached. More details can be found in the ablation section.


% \vspace*{0.2cm}\noindent\textbf{Computation of semantic embeddings (Stage 2)}:


% \vspace*{0.2cm}\noindent\textbf{Finetuning of the Classfier (Stage 3)}:

% \vspace*{0.2cm}\noindent\textbf{Deployment  (Stage 4)}:
\section{Experiments}
% In this section, we will detail the settings of our experiments and present the experimental results.
To fully demonstrate the superiority of MHCPL,
we conduct experiments\footnote{https://github.com/Snnzhao/MHCPL} on two public datasets to explore the following questions:
\begin{itemize}
    \item \textbf{RQ1:} How does MHCPL perform compared with the state-of-the-art methods?
    \item \textbf{RQ2:} How do different components (social influence, hypergraph based state encoder, and cross-view contrastive learning) affect the results of MHCPL?
    \item \textbf{RQ3:} How do parameters (the layer number of Hypergraph based State Encoder) influence the results of MHCPL?
    \item \textbf{RQ4:} Can our MHCPL effectively leverage the interactive conversation, item knowledge, and social influence to learn the dynamic user preferences?
\end{itemize}

\subsection{Datasets}\label{sec:standalone}
To evaluate the proposed method, we adapt two existing
MCR benchmark datasets, named Yelp and LastFM. The statistics of these datasets are presented in Table \ref{tab:data}.
\begin{itemize}
    \item {\textbf{LastFM}}~\cite{lei2020estimation}: LastFM dataset is the music listening dataset collected from Last.fm online music systems. As Zhang \etal \shortcite{zhang2022multiple}, We define the 33 coarse-grained groups as attribute types for the 8,438 attributes.
    \item{\textbf{Yelp}}~\cite{lei2020estimation}: Yelp dataset is adopted from the 2018 edition of the Yelp challenge. Following Zhang \etal \shortcite{zhang2022multiple}, we define the 29 first-layer categories as attribute types, and 590 second-layer categories as attributes.
\end{itemize}
Following Zhang \etal \shortcite{zhang2022multiple}, we sample two items with partially overlapped attributes as the user's acceptable items for each conversation episode.
\begin{table}[ht]
    \setlength{\tabcolsep}{5pt}
 \centering
 \small
    \begin{tabularx}{0.45\textwidth}{p{3cm}|X|X}
    \toprule
    \makecell[c]{\text{Dataset}}&\makecell[c]{\text{Yelp}}&\makecell[c]{\text{LastFM}}\cr
    \hline
    \hline
    \makecell[c]{\text{Users}}&\makecell[c]{27,675}&\makecell[c]{1,801}\cr

    \makecell[c]{\text{Items}}&\makecell[c]{70,311}&\makecell[c]{7,432}\cr
    \makecell[c]{\text{Attributes}}&\makecell[c]{590}&\makecell[c]{8,438}\cr
    \makecell[c]{\text{Attribute types}}&\makecell[c]{29}& \makecell[c]{34}\cr
    \hline
    \makecell[c]{\text{User-Item}}&\makecell[c]{1,368,606}&\makecell[c]{76,693}\cr
    \makecell[c]{\text{User-User}}&\makecell[c]{688,209}&\makecell[c]{23,958}\cr
    \makecell[c]{\text{Item-Attribute}}&\makecell[c]{477,012}&\makecell[c]{94,446}\cr
    \bottomrule[0.8pt]
    \end{tabularx}
    \caption{Statistics of two utilized datasets}
    \label{tab:data}
\end{table}
\subsection{Experiments Setup}

\subsubsection{User Simulator}
MMCR is a system that is trained and evaluated based on interactive conversations with users. Following the user simulator adopted in \cite{zhang2022multiple}, we simulate a interactive session for each user-item set interaction pair $(u, \mathcal{V}_u)$. Each item in the item set $v \in \mathcal{V}_u$ is treated as an acceptable item for the user. Each session is initialized with a user $u$ specifying an attribute $p_0 \in \mathcal{P}_{joint}$, where $\mathcal{P}_{joint}$ is the set of attributes that are shared by the items in $\mathcal{V}_u$. Then the session follows the process of "System Ask or Recommend, User response" \cite{zhang2022multiple} as described in Section \ref{sec:def}.

\begin{table*}[t]
    \centering
    \begin{tabular}{p{2.0cm}<{\centering}p{1cm}<{\centering}p{1cm}<{\centering}p{1cm}<{\centering}p{1cm}<{\centering}p{1cm}<{\centering}p{0.01cm}p{1cm}<{\centering}p{1cm}<{\centering}p{1cm}<{\centering}p{1cm}<{\centering}p{1cm}<{\centering}p{0.01cm}p{1cm}<{\centering}p{1cm}<{\centering}p{1cm}<{\centering}p{0.01cm}p{1cm}<{\centering}p{1cm}<{\centering}p{1cm}<{\centering}}
    \toprule
    \multirow{2}{*}{\bfseries Models }&\multicolumn{5}{c}{\bfseries Yelp }&&\multicolumn{5}{c}{\bfseries LastFM }\\
    \cline{2-6}
    \cline{8-12}
    &SR@5&SR@10&SR@15&AT&hDCG&&SR@5&SR@10&SR@15&AT&hDCG\\
    \midrule

    Abs Greedy &0.078&0.124&0.150&13.65&0.065&&0.292&0.436&0.512&10.10&0.237\\
    Max Entropy&0.046&0.200&0.390&12.97&0.117&&0.280&0.560&0.680&9.34&0.263\\
    CRM& 0.026&0.100&0.188&13.99&0.059&&0.092&0.240&0.372&12.56&0.130\\
    EAR& 0.120&0.198&0.240&12.91&0.094&&0.298&0.436&0.508&10.08&0.237\\
    SCPR &0.146&0.188&0.436&12.29&0.169&&0.322&0.630&0.764&8.47&0.322\\
    UNICORN &\underline{0.200}&0.338&0.430&11.33&0.175&&0.444&0.774&0.846&7.10&0.348\\
     MCMIPL&{0.162}&{0.366}&{0.522}&{11.25}&{0.184}&&\underline{0.448}&{0.809}&{0.884}&{6.87}&{0.353}\\
    \midrule
    S*-UNICORN &0.120&0.478&0.696&10.59&0.223&&0.412&0.850&0.912&6.69&0.363\\
     S*-MCMIPL&{0.126}&\underline{0.490}&\underline{0.722}&\underline{10.51}&\underline{0.230}&&{0.442}&\underline{0.872}&\underline{0.940}&\underline{6.43}&\underline{0.368}\\
    \midrule
    MHCPL&{0.142}&{\bfseries 0.592}&{\bfseries 0.854}&{\bfseries 9.96}&{\bfseries 0.261}&&{\bfseries 0.470}&{\bfseries 0.938}&{\bfseries 0.982}&{\bfseries 5.87}&{\bfseries 0.427}\\
    % \midrule
     Improv. &-&20.82$\%$&18.28$\%$&5.23$\%$&13.48$\%$&&4.91$\%$&7.57$\%$&4.47$\%$&8.71$\%$&16.03$\%$\\
    \bottomrule
    \end{tabular}
    \caption{Performance comparison of different models on the two datasets. The bold number represents the improvement of our model over baselines is statistically significant with p-value $< 0.01$. hDCG stands for hDCG@($15,10$).}
    \label{tab:results}
\end{table*}

\subsubsection{Baselines}
To demonstrate the effectiveness of the proposed MHCPL, the state-of-the-art methods are chosen for comparison : 
\begin{itemize}
    \item \textbf{Max Entropy.} This method employs a rule-based strategy to ask and recommend. It chooses to select an attribute with maximum entropy based on the current state, or recommends the top-ranked item with certain probabilities \cite{lei2020estimation}.
    \item \textbf{Greedy\cite{christakopoulou2016towards}.} This method only makes item recommendations and updates the model based on the feedback. It keeps recommending items until the successful recommendation is made or the pre-defined round is reached. 

    \item \textbf{CRM\cite{sun2018conversational}.} A reinforcement learning-based method that records the users' preferences into a belief tracker and learns the policy deciding when and which attributes to ask based on the belief tracker.
    \item \textbf{EAR\cite{lei2020estimation}.} This method proposes a three-stage solution to enhance the interaction between the conversational component and the recommendation component.
    \item \textbf{SCPR\cite{lei2020interactive}.} This method learns user preferences by reasoning the path on the user-item-attribute graph via the user’s feedback and accordingly chooses actions.
    \item \textbf{UNICORN\cite{deng2021unified}.} This work builds a weighted graph to model dynamic relationships between the user and the candidate action space, and proposes a graph-based Markov Decision Process (MDP) environment to learn dynamic user preferences and chooses actions from the candidate action space.
    \item \textbf{MCMIPL\cite{zhang2022multiple}.} This approach proposes a multi-interest policy learning framework that captures the multiple interests of the user to decide the next action.
    \item \textbf{S*-UNICORN and S*-MCMIPL.} For a more comprehensive and fair performance comparison, we adapt UNICORN and MCMIPL by timely selecting helpful social information and incorporating it into the weighted graph of the model. We name the two adapted methods S*-UNICORN and S*-MCMIPL.
\end{itemize}

\subsubsection{Parameters Setting}
Following \cite{zhang2022multiple}, we recommend top $K=10$ items or ask $K_a= 2$ attributes in each turn. We employ the Adam optimizer with a learning rate of $1e-4$. Discount factor $\gamma$ is set to be $0.999$. Following \cite{deng2021unified}, we adopt TransE \cite{TransE} via OpenKE \cite{OpenKE} to pretrain the node embeddings with 64 dimensions in the constructed KG with the training set. We make use of Nvidia Titan RTX graphics cards equipped with AMD r9-5900x CPU (32GB Memory).
For the action space, we select $K_p=10$ attributes and $K_v=10$ items.
To maintain a fair comparison, we adopt the same reward settings as previous works  \cite{lei2020estimation, lei2020interactive,deng2021unified,zhang2022multiple}: $r_{rec\_suc}=1, r_{rec\_fail}=-0.1, r_{ask\_suc}=0.01, r_{ask\_fail}=0.1, r_{quit}=-0.3$. For MHCPL, we select the number of layers from {1, 2, 3, 4}.

\subsubsection{Evaluation Metrics}
Following previous works \cite{lei2020estimation, lei2020interactive,deng2021unified}, we adopt success rate (SR@t) to measure the cumulative ratio of successful recommendations by the turn t, average turns (AT) to evaluate the average number of turns for all sessions, and hDCG@(T, K) to additionally evaluate the ranking performance of recommendations. 
Therefore, the higher SR@t and hDCG@(T, K) indicate better performance, while the lower AT means an higher efficiency.


\subsection{Performance Comparison (RQ1)}
\subsubsection{Overall Performance}
The comparison experimental results of the baseline models and our models are shown in Table \ref{tab:results}. 
We can summarize our observations as follows:

\begin{itemize}[leftmargin=*]
    %our对比正常
    \item \textbf{Our proposed MHCPL achieves the best performance.} MHCPL significantly outperforms all the baselines on the metrics of SR@15, AT and hDCG by over 4.47$\%$, 5.23$\%$ and 13.48$\%$, respectively.  We attribute the improvements to the following reasons: 1) The proposed dynamic multi-view hypergraph could effectively capture multiplex relations from three views. And the proposed hierarchical hypergraph neural network is able to well learn dynamic user preferences by integrating the information of graph structure and sequential modeling from the dynamic multi-view hypergraph; 
    2) MHCPL timely selects helpful social information and effectively integrates the interactive conversation, item knowledge, and social influence for better dynamic user preference learning; 3) MHCPL designs a cross-view contrastive learning method to help maintain the inherent characteristics and the correlations of user preferences from different views.
    
    \item \textbf{The learning of the dynamic user preferences is crucial for conversational recommendation.} The graph-based methods (MHCPL, MCMIPL, UNICORN, SCPR) outperforms the factorization-based methods (EAR, CRM) since they learn user preferences from the collaborative information in the graph. MCMIPL achieves the best performance among the graph-base baselines since it further considers the multiple interests of the user preferences. Our proposed MHCPL further outperforms these methods since we leverage multiplex relations to integrate interactive conversation, item knowledge, and social influence to help learn the dynamic user preferences.

    \item \textbf{Social influence is effective in helping learn dynamic user preferences for conversational recommendation when well filtered.} The socially adapted methods (\ie S*-UNICORN and S*-MCMIPL) outperform their original versions in the final performances. We attribute this to the reason that social influence is an important factor that affects user preferences and could help learn dynamic user preferences with friends' preferences that satisfy the interactive conversation. But the socially adapted methods perform worse than their original version in the early turns (\eg SR@5). This happens because the information in the interactive conversation is not sufficient to filter out the noise from the social information in the early turn of the conversation.
    
\end{itemize}

% Figure environment removed

\subsubsection{Comparison at Different Conversation Turns} Besides the performance in the final turn, we also present success rates at different turns in \autoref{fig:overall}.
In order to better observe the differences among different models, we use the relative success rate compared with the most competitive baseline $\text{S*-MCMIPL}$, where the blue line of $\text{S*-MCMIPL}$ is set to zero in the figures. From the \autoref{fig:overall}, we following observations:
\begin{itemize}[leftmargin=*]
    %our对比正achieve
    \item \textbf{} The proposed MHCPL outperforms these baseline methods across all the datasets and almost all the turns in the conversational recommendation. This is because our proposed MHCPL could better learn dynamic user preferences with multiplex relations that integrate interactive conversation, item knowledge, and social influence.
    
    \item \textbf{} The recommendation success rate of the proposed socially-aware methods (\ie MHCPL, S*-MCMIPL, and S*-UNICORN) could not surpass all the baselines in the early turns of the conversational recommendation, especially on the dataset Yelp with a larger candidate space of items and attributes. This is because the information in the interactive conversation is not sufficient to filter out the noise from the social information at the early turn of the conversation.
    Furthermore, socially-aware methods prefer to ask rather than recommend in the early turns when the user's preference is not certain enough. This will effectively reduce the action space and better learn user preferences, but lead to a lower recommendation success rate in the early turns. %This is because they may recommend items with unclear user preferences in the early turns. This can increase early turns' success rates but is not effective since it is helpless in learning user preferences.
\end{itemize}
\begin{table}[t]
    \small
    \centering
    \begin{tabular}{p{2.7cm}<{\centering}p{0.6cm}<{\centering}p{0.5cm}<{\centering}p{0.6cm}<{\centering}p{0.6cm}<{\centering}p{0.5cm}<{\centering}p{0.6cm}<{\centering}}
    \toprule
    % \hline
    \multirow{2}{*}{\bfseries Models }& \multicolumn{3}{c}{\textbf{Yelp}}& \multicolumn{3}{c}{\textbf{LastFM}}\\
    &SR@15&AT&hDCG&SR@15&AT&hDCG\\
    \midrule
    Ours&{\bfseries0.854}&{\bfseries9.96}&{\bfseries0.261}&{\bfseries0.982}&{\bfseries5.87}&{\bfseries0.427}\\
    \midrule
    -w/o social&0.592&10.80&0.208&0.908&6.63&0.365\\
    -w/o hypergraph&0.726&10.68&0.346&0.938&6.58&0.382\\
    -w/o contrastive&0.762&10.37&0.237&0.962&6.17&0.403\\
    \bottomrule
    \end{tabular}
    \caption{Results of the Ablation Study. }
    \label{tab:ablation_study}
\end{table}
\subsection{Ablation Studies (RQ2)}
To investigate the underline mechanism of MHCPL, we conduct ablation experiments on the Yelp and LastFM datasets with three ablated methods including: $\text{MHCPL}_{\wo social}$ that ablates the social influence, $\text{MHCPL}_{\wo hypergraph}$ that replaces the hypergraph neural networks with graph neural networks, and $\text{MHCPL}_{\wo contrastive}$ that ablates the cross-view contrastive learning. From results shown in Table~\ref{tab:ablation_study}, we have the following observations:
\begin{itemize}
    \item $\text{MHCPL}_{\wo social}$ is the least  competitive. This demonstrates the importance of social influence in alleviating the data sparsity problem and helping learn dynamic user preferences. And it is effective to accordingly choose helpful social information based on interactive conversation. $\text{MHCPL}_{\wo social}$ still outperforms all the baselines that ignore the social information in Table~\ref{tab:results}, which proves the effectiveness of MHCPL in learning dynamic user preferences with multiplex relations.
    
    \item MHCPL outperforms $\text{MHCPL}_{\wo hypergraph}$. 
    We contribute this to the importance of multiplex relations in learning dynamic user preferences. This also proves the effectiveness of our proposed multi-view hypergraph-based state encoder in learning user preferences by integrating the information of graph structure and sequential modeling from the dynamic multi-view hypergraph.
    \item MHCPL outperforms $\text{MHCPL}_{\wo contrastive}$. This demonstrates the effectiveness of the cross-view contrastive learning module in helping maintain the inherent characteristics and correlations of user preferences from different views. 
\end{itemize}

% Figure environment removed

\subsection{Hyper-parameter Sensitivity Analysis (RQ3)}
\subsubsection{Impact of Layer Number} The hypergraph-based state encoder learns dynamic user preferences from the multiplex relations in the hypergraph. By stacking more layers, collaborative information from multi-hop neighbors is distilled. We investigate how the layer number $L$ influences the performance of MHCPL. Specifically, we conduct experiments with $L$ in the range $\{1, 2, 3, 4\}$, and the results are shown in Figure \ref{fig:layers}. There are some observations:
\begin{itemize}
    \item Increasing the number of layers can improve the performance of our model. MHCPL-2 highly outperforms MHCPL-1. The reason is that MHCPL-1 only gains information from the one-hop neighbors and neglects high-order collaborative information. 
    \item When increasing the layer of number, the performance does not always improve. MHCPL-3 outperforms MHCPL-4 on data LastFM. This can be attributed to the noise which increases along with the hop of neighbors.
\end{itemize}

% Figure environment removed

\subsection{Case Study (RQ4)}
To show the effectiveness of our proposed MHCPL in leveraging multiplex relations to integrate interactive conversation, item knowledge, and social influence to learn dynamic user preferences, we present a case of conversational recommendation generated by our framework in \autoref{fig:case}. As illustrated in the figure, by integrating the information from the interactive conversation, item knowledge, and social information with multiplex relations from different views, MHCPL is able to effectively ask attributes and recommend user-preferred items, reaching success in five turns. Furthermore, the social information selected according to the interactive conversation is helpful in learning dynamic user preferences. With the help of selected social information, MHCPL could accurately select the target item when the information from the interactive history is limited in distinguishing user preferences towards the seventy candidate items.
\section{Conclusion}\label{sec:conclusion}

This paper presents our empirical domain knowledge distillation framework using ChatGPT and discusses our observations from the framework application experiments in the autonomous driving domain. The key finding is that: 1) with proper design of prompt engineering and execution flow, fully automated domain knowledge (in the ontology format) distillation is possible. However, due to the randomness in the response and the butterfly effect, the quality of fully automated distillation results is not guaranteed. To address this, we develop a web-based assistant to enable manual supervision and early intervention at runtime. We hope our findings and tools inspire future research toward revolutionizing the engineering processes of knowledge-based systems across domains.
%\normalem


\bibliographystyle{ACM-Reference-Format}
\bibliography{sigir2023}

\end{document}

