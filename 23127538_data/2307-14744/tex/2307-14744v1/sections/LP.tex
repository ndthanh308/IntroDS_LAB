As explained earlier, we traverse down Uruv to the correct leaf node and perform all operations on the linked-list in that leaf. Therefore, we discuss the LPs of the versioned linked-list.
\vspace{-0.2cm}\paragraph*{Insert.} There are two cases. If the key does not exist, we insert the key into the linked-list at line \ref{alg21:line264} or \ref{alg25:line314}. However, the key’s versioned timestamp is invalid, and it can only be made valid using \texttt{initTs} after the first successful execution of line \ref{alg22:line270}. This can be executed either by insert at line \ref{alg21:line265} or \ref{alg25:line312}, some concurrent read at line \ref{alg23:line272}, or some concurrent update at line \ref{alg24:line275} or \ref{alg25:init}.

If the key already exists, we update its value by atomically replacing its current $vhead$ with a new versioned node inside \texttt{vCAS} at line \ref{alg24:line280} or inside \texttt{wfVCAS} at line \ref{alg26:line321}. After successfully changing the $vhead$, the node’s timestamp is still invalid, and it can only be made valid using \texttt{initTs} after the first successful execution of line \ref{alg22:line270}. This can be executed either by \texttt{vCAS} at line \ref{alg24:line281} or by \texttt{wfVCAS} at line \ref{alg26:line319}, some concurrent read at line \ref{alg23:line272}, or some concurrent update at line \ref{alg24:line275} or  \ref{alg25:init}.

Line \ref{alg22:line270} is the LP for insert. Whenever we read any key’s value from its versioned node, we need to make sure the versioned node’s timestamp is valid. This ensures that a range query fetches all the relevant keys and their values present when the range query was invoked. Otherwise, if the LP is given to some line earlier than \ref{alg22:line270}, the key could have existed, but a concurrent range query can miss it, leading to inconsistency. 
\vspace{-0.3cm}\paragraph*{Delete.} There are two cases. If the key does not exist, then there is no need to delete the key as it does not exist. Therefore, the LP would be line \ref{alg20:line243}, where we last read a node from the linked-list. Instead, if the key exists, since we perform an update at line \ref{alg9:line132} or \ref{alg10:line151}, the LP would be the same as insert’s, which is line \ref{alg22:line270}.
\vspace{-0.2cm}\paragraph*{Search.} Regardless of whether the key exists or not, search's LP would be line \ref{alg20:line243} when we first read the node whose key is greater than the key we are searching for in the linked-list. If a concurrent insert/delete leads to a split/merge operation, then there is a chance that search will end up at a leaf node that is no longer a part of Uruv. In that case, search’s LP would have happened before insert/delete’s LP. Search’s LP remains the same as above.
\vspace{-0.2cm}\paragraph*{Range Query.} In a range query, we call the \texttt{addTimestamp} method, which updates the global timestamp and returns that. So the LP for range query would be the successful execution of line \ref{alg4:line83}. The range query’s LP will remain the same regardless of any other concurrent operation.
