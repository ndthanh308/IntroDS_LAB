\urv derives from a B$^+$Tree \cite{comer+:ACM1979}, a self-balancing data structure. However, to support \lble range search operations, they are equipped with additional components. 
The key-value pairs in \urv are stored in the \textit{key nodes}. A \textit{leaf node} of \urv is a sorted linked-list of key nodes. Thus, the leaf nodes of \urv differ from the array-based leaf nodes of a B$^+$Tree. The \textit{internal nodes} are implemented by arrays containing ordered set of keys and pointers to its descendant children, which facilitate traversal from the root to key nodes. A search path in \urv is shown in Figure \ref{fig:tree-design}.
 
 
%, bb=0 0 550 600
% Figure environment removed



%We perform the update operations on Uruv proactively. If any node violates the maximum/minimum threshold during traversal inside updates, we immediately perform a split/merge operation. Once the correct leaf node is found, we perform the required operation. If an update operation violates the threshold, a split/merge is performed on the leaf node and the operation is retried on the new leaf node created. A seach or range query operation does not modify Uruv. Refer to Figure \ref{fig:tree-design} for further details.


% \vspace{-10pt}