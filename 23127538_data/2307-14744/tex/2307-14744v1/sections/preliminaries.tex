We consider the standard shared-memory model with atomic \texttt{read}, \texttt{write}, \texttt{\faa} (\FAA),  and \texttt{\cas} (\CAS) instructions. \urv implements a key-value store $(\mathcal{K},\mathcal{V})$ of keys $K\in\mathcal{K}$ and their associated values $V\in\mathcal{V}$.

\textbf{The Abstract Data Type (ADT):} We consider an ADT $\mathcal{A}$ as a set of operations: $\mathcal{A}$ = $\{\insertADT(K,V)$, ~$\remove(K)$, ~$\search(K)$,  ~$\rangeQuery(K1,K2)\}$
\begin{enumerate}[topsep=0pt,itemsep=0pt,parsep=0pt,partopsep=0pt]
\item An $\insertADT(K, V)$ inserts the key $K$ and an associated value $V$ if $K\notin\mathcal{K}$.
\item A $\remove(K)$ deletes the key $K$ and its associated value if $K\in\mathcal{K}$.
\item A $\search(K)$ returns the associated value of key $K$ if $K\in\mathcal{K}$; otherwise, it returns $-1$. It does not modify $(\mathcal{K},\mathcal{V})$.
\item A $\rangeQuery(K_1,K_2)$ returns keys $\{K\in\mathcal{K}:K_1{\le}K{\le}K_2\}$, and associated values without modifying $(\mathcal{K},\mathcal{V})$; if no such key exists, it returns $-1$.
\end{enumerate}
\ignore{
\textbf{The data structure:} \urv derives from a B$^+$Tree \cite{comer+:ACM1979}, a self-balancing data structure. The index repeatedly divides the search space until it becomes small enough to use traditional search methods such as binary search. It provides logarithmic asymptotic guarantees for create, read, update and delete (CRUD) operations on the data it indexes. The nodes of the index, also called \textit{internal nodes}, are ordered set of keys and pointers to its descendant children, which facilitate traversal through the index to \textit{leaf nodes}.%The data in a B$^+$Tree is stored in the form of key, value pairs and the tree supports the operations in the ADT $\mathcal{A}$ described above.

%The tree contains two kinds of nodes - Internal and Leaf. An Internal node has an ordered set of keys and pointers to its descendant children. Pointers in the internal node facilitate traversal through the tree to a leaf node. The leaf node contains keys and maps them to their values. The keys present in internal nodes are redundant as compared with the the leaf nodes and are just used for traversal.

The key-value pairs are stored in the leaf nodes by ordered traversal of the set of keys. Any action must begin with a search for the leaf node holding the key on which you have to act. This search begins at the root node and proceeds to the next descendant node using a search over the sorted set of keys. This process continues until you reach a leaf node. Apart from the root, each node in the tree has a size between the lowest and maximum threshold. These limits ensure that the cost of traversing to the leaf node is logarithmic in tree size. All ADT actions are carried out after reaching the leaf node. The self-balancing nature of a B$^+$Tree ensures that all the leaf nodes should be at the same level and that only one path exists to traverse from the root to a leaf node. The leaf nodes are linked in a linked-list fashion, with each leaf node pointing to its next neighbouring leaf node for efficient rangequeries.

\textcolor{red}{If the maximum/minimum threshold is exceeded after an insertion/deletion at a leaf node, the tree will split/merge. The split operation divides the leaf node into two halves while adding a key to the parent node. If the parent is likewise full, the parent node will split and may cascade to the root. Similarly, the merge procedure joins two leaf nodes while removing a key from the parent node. If the parent node fails to meet the minimal threshold, it may merge with its sibling, cascading to the root.
An alternative to this is a \textit{proactive approach} which checks the cascading effect proactively. In this technique, if a node crosses either threshold while traversing, it preemptively performs the balancing. As a result, insertion or deletion at the leaf node will be determined only by its parent node. According to our knowledge, Uruv is the first concurrent tree-like structure that employs this proactive strategy.}
}