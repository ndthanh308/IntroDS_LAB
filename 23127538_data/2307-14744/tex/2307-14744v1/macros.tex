%\newcommand{\todo}[1]{\textcolor{red}{TODO: #1}\PackageWarning{TODO:}{#1!}}

% comment a region
\newcommand{\punt}[1]{}
\newcommand{\cmnt}[1]{}
\newcommand{\ignore}[1]{}

%\algnewcommand{\IIf}[1]{\State\algorithmicif\ #1\ \algorithmicthen}
%\algnewcommand{\EndIIf}{\unskip\ \algorithmicend\ \algorithmicif}

\newcommand{\tlowest}{\ensuremath{T_{lowest}\xspace}}
\newcommand{\tother}{\ensuremath{T_{other}\xspace}}

% a word should not be broken across lines
\newcommand{\nosplit}{\linebreak}

% no hyphenation
\def\nohyphens{\hyphenpenalty=10000\exhyphenpenalty=10000}

% ~ character
\newcommand{\tilda}{\symbol{126}}

% useful mathematical symbols

\newcommand{\ang}[1]{\langle #1 \rangle}
\newcommand{\Ang}[1]{\Big\langle #1 \Big\rangle}
\newcommand{\ceil}[1]{\lceil #1 \rceil}
\newcommand{\floor}[1]{\lfloor #1 \rfloor}

%%%%%%%%%%%%%%%%%%%%%%%%%%%%%%%%%%%%%%%%%%%%%%%%%%%%%%%%%%%%%%%%%%%%%%%%%%%%%%%%%%%%%%%%%%%%%%

% keywords for pseudocode

%% various theorem environments
%% Commented out because they are already defined in llncs file %%
%% Decommented out because they are not defined in sigplan class %%
\definecolor{darkblue}{rgb}{0.0, 0.0, 0.55}
\newcommand{\linecomment}[1]{{\scriptsize \textcolor{darkblue}{#1}}}
%\newtheorem{theorem}{Theorem}
%\newtheorem{thm}{Theorem}
%\renewcommand{\thetheorem}{\arabic{thm}}
%\newtheorem{lemma}[theorem]{Lemma}
%\newtheorem{corollary}[theorem]{Corollary}
%\newtheorem{proposition}[theorem]{Proposition}
%\newtheorem{property}[thm]{Property}
%\newtheorem{observation}[thm]{Observation}
%\newtheorem{remark}[theorem]{Remark}

%\newtheorem{definition}[theorem]{Definition}
%\renewcommand{\thedefinition}{\arabic{definition}}

\newcounter{history}
\newcommand{\hist}[1]{\refstepcounter{history} {#1}}

%%%%%%%%%%%%%%%%%%%%%%%%%%%%%%%%%%%%%%%%%%%%%%%%%%%%%%%%%%%%%%%%%%%%%%%%%%%%%%%%%%%%%%%%%%%%%%

%% Defined by me for Histories
%\newtheorem{history}{History}

%%%%%%%%%%%%%%%%%%%%%%%%%%%%%%%%%%%%%%%%%%%%%%%%%%%%%%%%%%%%%%%%%%%%%%%%%%%%%%%%%%%%%%%%%%%%%%
%-----------------Theorem Notations Defined by Petr --------------------------------------------------------------------------------------------------------
%% Defined by Petr

%\newtheorem{acknowledgement}[theorem]{Acknowledgement}
%\newtheorem{algorithm}[theorem]{Algorithm}
%\newtheorem{case}[theorem]{Case}
%\newtheorem{claim}[theorem]{Claim}
%\newtheorem{explanation}[theorem]{Explanation}
%\newtheorem{fact}[theorem]{Fact}
% \newtheorem{conclusion}[theorem]{Conclusion}
% \newtheorem{condition}[theorem]{Condition}
%\newtheorem{conjecture}[theorem]{Conjecture}
% \newtheorem{criterion}[theorem]{Criterion}
%\newtheorem{definition}[theorem]{Definition}
% \newtheorem{example}[theorem]{Example}
% \newtheorem{exercise}[theorem]{Exercise}
%\newtheorem{lemma}[theorem]{Lemma}
% \newtheorem{notation}[theorem]{Notation}
% \newtheorem{problem}[theorem]{Problem}
%\newtheorem{proposition}[theorem]{Proposition}
% \newtheorem{remark}[theorem]{Remark}
% \newtheorem{solution}[theorem]{Solution}
% \newtheorem{summary}[theorem]{Summary}
%\newtheorem{observation}[theorem]{Observation}

%\newenvironment{proof}[1][Proof]{\noindent\textbf{#1.} }{} %\rule{0.5em}{0.5em}\\}
%\newenvironment{proofsketch}[1][Proof Sketch]{\noindent\textbf{#1.} }{} 
%\newenvironment{proofsketch}[1][Proof Sketch]{\noindent#1: }{\hfill $\Box$\\[0.4mm]} %\rule{0.5em}{0.5em}\\}
%\newenvironment{reptheorem}[1][Theorem]{\noindent\textbf{#1}}{}

%------------------------------------------------------------------------------------------------------------------------------

% various references
\newcommand{\chapref}[1]{Chapter~\ref{chap:#1}}
\newcommand{\secref}[1]{Section~\ref{sec:#1}}
\newcommand{\figref}[1]{Figure~\ref{fig:#1}}
\newcommand{\tabref}[1]{Table~\ref{tab:#1}}
\newcommand{\stref}[1]{Step~\ref{step:#1}}
\newcommand{\csref}[1]{Case~\ref{case:#1}}
\newcommand{\thmref}[1]{Theorem~\ref{thm:#1}}
\newcommand{\lemref}[1]{Lemma~\ref{lem:#1}}
\newcommand{\corref}[1]{Corollary~\ref{cor:#1}}
\newcommand{\axmref}[1]{Proposition~\ref{axm:#1}}
\newcommand{\defref}[1]{Definition~\ref{def:#1}}
\newcommand{\eqnref}[1]{Eqn(\ref{eq:#1})}
\newcommand{\eqvref}[1]{Equivalence~(\ref{eqv:#1})}
\newcommand{\ineqref}[1]{Inequality~(\ref{ineq:#1})}
%\newcommand{\invref}[1]{(\ref{inv:#1})}
\newcommand{\exref}[1]{Example~\ref{ex:#1}}
\newcommand{\propref}[1]{Property~\ref{prop:#1}}
\newcommand{\obsref}[1]{Observation~\ref{obs:#1}}
\newcommand{\asmref}[1]{Assumption~\ref{asm:#1}}
\newcommand{\thref}[1]{Thread~\ref{th:#1}}
\newcommand{\trnref}[1]{Transaction~\ref{trn:#1}}
%\newcommand{\lineref}[1]{Line~\ref{lin:#1}}
\newcommand{\linref}[1]{Line~\ref{lin:#1}}
\newcommand{\algoref}[1]{Algo~\ref{algo:#1}}
\newcommand{\subsecref}[1]{SubSection{\ref{subsec:#1}}}

\newcommand{\histref}[1]{\ref{hist:#1}}

\newcommand{\apnref}[1]{Appendix~\ref{apn:#1}}
\newcommand{\invref}[1]{Invariant~\ref{inv:#1}}

\newcommand{\Chapref}[1]{Chapter~\ref{chap:#1}}
\newcommand{\Secref}[1]{Section~\ref{sec:#1}}
\newcommand{\Figref}[1]{Figure~\ref{fig:#1}}
\newcommand{\Tabref}[1]{Table~\ref{tab:#1}}
\newcommand{\Stref}[1]{Step~\ref{step:#1}}
\newcommand{\Thmref}[1]{Theorem~\ref{thm:#1}}
\newcommand{\Lemref}[1]{Lemma~\ref{lem:#1}}
\newcommand{\Corref}[1]{Corollary~\ref{cor:#1}}
\newcommand{\Axmref}[1]{Proposition~\ref{axm:#1}}
\newcommand{\Defref}[1]{Definition~\ref{def:#1}}
\newcommand{\Eqref}[1]{(\ref{eq:#1})}
\newcommand{\Eqvref}[1]{Equivalence~(\ref{eqv:#1})}
\newcommand{\Ineqref}[1]{Inequality~(\ref{ineq:#1})}
\newcommand{\Exref}[1]{Example~\ref{ex:#1}}
\newcommand{\Propref}[1]{Property~\ref{prop:#1}}
\newcommand{\Obsref}[1]{Observation~\ref{obs:#1}}
\newcommand{\Asmref}[1]{Assumption~\ref{asm:#1}}

%\newcommand{\Lineref}[1]{Line~\ref{lin:#1}}
\newcommand{\Algoref}[1]{Algo~ \ref{algo:#1}}

\newcommand{\Apnref}[1]{Section~\ref{apn:#1}}
\newcommand{\Invref}[1]{Invariant~\ref{inv:#1}}

% environment for writing a proof

%\newcommand{\proof}[1][]{\noindent{\bf\boldmath
%                          P\hspace{-0.25ex}roof{#1}\/:\unboldmath}\hspace*{0.5em}}


\newcommand{\theqed}{$\Box$}

%\newcommand{\qed}{\hspace*{\fill}\theqed\\\vspace*{-0.5em}}

% In case the proof is immediately followed by the start of a new section

\newcommand{\nsqed}{\hspace*{\fill} \theqed}


% Renews the footnote command
\renewcommand{\thefootnote}{\alph{footnote}}

%%%%%%%%%%%%%%%%%%%%%%%%%%%%%%%%%%%%%%%%%%%%%%%%%%%%%%%%%%%%%%%%
% Definitions for new Author Style
%%%%%%%%%%%%%%%%%%%%%%%%%%%%%%%%%%%%%%%%%%%%%%%%%%%%%%%%%%%%%%%%

\newcommand*{\affaddr}[1]{#1} % No op here. Customize it for different styles.
\newcommand*{\affmark}[1][*]{\textsuperscript{#1}}
%\newcommand*{\email}[1]{\texttt{#1}}

%=======================State with no line number=======================================
\def\Statenolinnum#1{{\def\alglinenumber##1{}\State #1}\addtocounter{ALG@line}{-1}}
%==============================================================
%%%%%%%%%%%%%%%%%%%%%%%%%%%%%%%%%%%%%%%%%%%%%%%%%%%%%%%%%%%%%%%%
% Definitions by Petr 
%%%%%%%%%%%%%%%%%%%%%%%%%%%%%%%%%%%%%%%%%%%%%%%%%%%%%%%%%%%%%%%%

%\newcommand{\ignore}[1]{}


%\newcommand{\para}[1]{\noindent\textbf{\itshape\large#1.}}
%\newcommand{\myparagraph}[1]{\noindent\textbf{#1.}}
%\algdef{SE}[DOWHILE]{Do}{doWhile}{\algorithmicdo}[1]{\algorithmicwhile\ #1}%
%\algdef{SE}[DOWHILE]{Do}{doWhile}{\algorithmicdo}[1]{\algorithmicwhile\ #1}

%------------------------------------------------------------------------------------------------------------------
% Definitions for Concurrent Structures
%------------------------------------------------------------------------------------------------------------------
%
\newcommand{\sq}{\hbox{\rlap{$\sqcap$}$\sqcup$}}
%\newcommand{\qed}{\hspace*{\fill}\sq}
%\newenvironment{proof}{\noindent {\bf Proof.}\ }{\qed\par\vskip 4mm\par}
%\newenvironment{proofof}[1]{\bigskip \noindent {\bf Proof of #1:}\quad } {\qed\par\vskip 4mm\par}


\newcommand{\fa}{\mbox{\em fetch\&add}}
\newcommand{\FA}{\mbox{\em FA}}
\newcommand{\MOD}{~\mbox{mod}~}

\newcommand{\re}{\mbox{\em enter-count}}
\newcommand{\rx}{\mbox{\em exit-count}}
\newcommand{\nr}{x_2}
\newcommand{\ncs}{y_2}

\newcommand{\hst}[1] {#1.hist\xspace}
\newcommand{\exec}[1] {#1.exec\xspace}
%\newcommand{\state}[1] {#1.state\xspace}


\newcommand{\op} {operation\xspace}
\newcommand{\mth} {method\xspace}
\newcommand{\cc} {correctness-criterion\xspace}
\newcommand{\ccs} {correctness-criteria\xspace}
\newcommand{\gen}[1] {gen(#1)}
\newcommand{\term} {term\text{-}op\xspace}
\newcommand{\termop} {terminal operation\xspace}
\newcommand{\ad} {\texttt{add}\xspace}
\newcommand{\rem} {\texttt{delete}\xspace}
\newcommand{\srch} {\texttt{search}\xspace}
\newcommand{\con} {\texttt{contains}\xspace}
%\newcommand{\sear} {\texttt{search}\xspace}
\newcommand{\rnq} {\texttt{rangeQuery}\xspace}
\newcommand{\ins}{\texttt{insert}\xspace}
\newcommand{\up}{\texttt{update}\xspace}

\newcommand{\cas} {\texttt{\textbf{CAS}}\xspace}
\newcommand{\faa}{\texttt{\textbf{FAA}}\xspace}
%\newcommand{\faa}{fetch-and-increment\xspace}
\newcommand{\FAA}{fetch-and-increment\xspace}
\newcommand{\CAS}{compare-and-swap\xspace}

\newcommand{\urv} {Uruv\xspace}
\newcommand{\cng} {concurrent graph\xspace}
\newcommand{\cgds} {concurrent graph data structure\xspace}
\newcommand{\pset} {partial-set\xspace}

\newcommand{\lble} {linearizable\xspace}
\newcommand{\lbty} {linearizability\xspace}
\newcommand{\rbty} {reachability\xspace}
\newcommand{\legality} {legality\xspace}

\newcommand{\eevts}[1] {#1.evts}
%\newcommand{\mth}[1] {#1.mths}
%\newcommand{\mth}[1] {#1.ops}
\newcommand{\inv} {inv\xspace}
\newcommand{\rsp} {resp\xspace}

\newcommand{\sspec} {sequential\text{-}specification\xspace}
\newcommand{\legal} {legal\xspace}
\newcommand{\lp} {LP\xspace}

\newcommand{\stfdm} {starvation-freedom\xspace}
\newcommand{\stf} {starvation-free\xspace}
\newcommand{\cmth} {commit-throughput\xspace}
\newcommand{\insertADT}{\textsc{Insert}\xspace}
\newcommand{\remove}{\textsc{Delete}\xspace}
\newcommand{\search}{\textsc{Search}\xspace}
\newcommand{\rangeQuery}{\textsc{RangeQuery}\xspace}
\newcommand{\keyADT}{\textsc{key}\xspace}
\newcommand{\keystart}{\textsc{key\_start}
\xspace}
\newcommand{\keyend}{\textsc{key\_end}\xspace}
\newcommand{\valueADT}{\textsc{value}\xspace}







\newcommand{\lazy}{lazy-list\xspace}
\newcommand{\hoh}{hoh-locking-list\xspace}

%------ Comments -------------------
\newcommand \spnote[1] {\todo[inline,size=\footnotesize,color=yellow!20]{Sathya: #1}}
\newcommand \bapicomment[1] {\todo[inline,size=\footnotesize,color=red!20]{Bapi: #1}}
\newcommand \gauravcomment[1] {\todo[inline,size=\footnotesize,color=red!20]{Gaurav: #1}}