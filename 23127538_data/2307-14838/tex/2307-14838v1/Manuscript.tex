 \documentclass[aps,prb,twocolumn,showpacs,10pt,superscriptaddress,floatfix,longbibliography]{revtex4-1}


\usepackage{newtxtext,newtxmath}
%\usepackage{mathptmx}
\usepackage{graphicx}
\usepackage{subfigure}
\usepackage{amsmath}
\usepackage{amsfonts}
%\usepackage{amssymb}
\usepackage{mathrsfs}
\usepackage{bbm}
\usepackage{subfigure}
\usepackage{xcolor}
\usepackage[colorlinks=true]{hyperref}
\usepackage{ulem}
\usepackage[T1]{fontenc}




\newcommand{\red}[1]{\textcolor{red}{#1}}
\newcommand{\blue}[1]{\textcolor{blue}{#1}}
\newcommand{\green}[1]{\textcolor{green}{#1}}
\newcommand{\gray}[1]{\textcolor{gray}{#1}}
\newcommand{\orange}[1]{\textcolor{orange}{#1}}
\newcommand{\purple}[1]{\textcolor{purple}{#1}}

\begin{document}

\title{Topological superconductivity with large Chern numbers in a ferromagnetic metal-superconductor heterostructure}	

\author{Ying-Wen Zhang}
\affiliation{School of Physics, Guangdong Provincial Key Laboratory of Magnetoelectric Physics and Devices, Sun Yat-Sen University, Guangzhou 510275, China}
\affiliation{State Key Laboratory of Optoelectronic Materials and Technologies, Sun Yat-Sen University, Guangzhou 510275, China} 


\author{Dao-Xin Yao}
\email{yaodaox@mail.sysu.edu.cn}
\affiliation{School of Physics, Guangdong Provincial Key Laboratory of Magnetoelectric Physics and Devices, Sun Yat-Sen University, Guangzhou 510275, China}
\affiliation{State Key Laboratory of Optoelectronic Materials and Technologies, Sun Yat-Sen University, Guangzhou 510275, China}
\affiliation{International Quantum Academy, Shenzhen 518048, China} 


\author{Zhi Wang}
\email{wangzh356@mail.sysu.edu.cn}
\affiliation{School of Physics, Guangdong Provincial Key Laboratory of Magnetoelectric Physics and Devices, Sun Yat-Sen University, Guangzhou 510275, China}
 
\begin{abstract}
     The ferromagnetic metal-superconductor heterostructure with interface Rashba spin-orbit hopping is a promising candidate for topological superconductivity. We study the interplay between the interface Rashba hopping and the intrinsic Dresselhaus spin-orbit coupling in this heterostructure, and demonstrate rich topological phases with five distinct Chern numbers.
     In particular, we find a topological state with a Chern number as large as four in the parameter space of the heterostructure. We calculate the Berry curvatures that construct the Chern numbers, and show that these Berry curvatures induce anomalous thermal Hall transport of the superconducting quasiparticles. We reveal chiral edge states in the topological phases, as well as helical edge states in the trivial phase, and show that the wave functions of these edge states mostly concentrate on the ferrometal layer of the heterostructure. 


\end{abstract}
\maketitle
\section{Introduction}
Topological superconductors with Majorana modes have attracted much interest recently due to their possible applications in topological quantum computation \cite{kitaev2003fault, qi2011RMP, beenakker2013, stern2013topological, lutchyn2018review, aguado2017, Sato2017}. The topology of two-dimensional superconducting systems can be characterized by the Chern number \cite{read2000paired}, which is the summation of the momentum-space Berry curvatures of the Bogoliubov–de Gennes Hamiltonian \cite{qi2010chiral,chiu2016classification,bernevig2013topological}. For superconductors with nonzero Chern numbers, the bulk-edge correspondence predicts Majorana chiral modes \cite{bernevig2013topological}. If the Chern number is odd, it has been shown that Majorana zero modes could exist in the presence of a superconducting vortex \cite{volovik1999fermion,qi2010chiral}. The search for superconductors with nontrivial Chern numbers is the one central tasks in the present study of topological superconductivity \cite{nayak2008non,sarma2015majorana,flensberg2021engineered}.

The early theories of topological superconductors with nontrivial Chern numbers are usually based on time-reversal symmetry breaking superconducting gap, such as the p+ip or d+id gap function\cite{read2000paired,alicea2012new,Black_Schaffer2014}. However, superconductors with intrinsic chiral gap functions are rare in bulk materials\cite{flensberg2021engineered}. As a result, recent experimental and theoretical efforts for topological superconductivity concentrate around designed heterostructures where conventional s-wave superconductors are in closed contact with metallic systems such as topological insulators\cite{fu2008superconducting,wang2012coexistence,xu2014artificial,Sun2016Majorana} and spin-orbit coupling semiconductors\cite{sau2010twodimension,sau2010generic,sau2010non}. On the interface of these heterostructures, the delicate combination of the s-wave superconductivity, the spin-orbit coupling, and the Zeeman energy due to external or intrinsic magnetization would bring topological superconducting states with nonzero Chern numbers\cite{alicea2012new,flensberg2021engineered}.

The spin-orbit coupling is the central ingredient in these heterostructures. Previous theoretical proposals for the required spin-orbit coupling fall into two major categories. 
Firstly, a number of theoretical proposals considered the heterostructures where the metal layer has intrinsic spin-orbit coupling, such as the surface states of three-dimensional topological insulators and the spin-orbit coupling semiconductor.
Then an externally applied magnetic field could turn the system into a topological state \cite{sau2010twodimension,sato2010topological,fu2008superconducting}. These proposals have been extensively explored in experiments.
An alternative proposal involves a half-metal without intrinsic spin-orbit coupling. Instead, a Rahsba spin-orbit hopping between the half-metal and the conventional superconductor may happen due to the inversion symmetry breaking at the interface \cite{eschrig2008triplet,chung2011topological,zhang2014topological}.
The topological superconducting phase was predicted to be robust for this scenario as long as the half-metal has the odd number of Fermi surfaces. Experimental progress in this direction is limited because the proximity of superconductivity to the half-metal system is difficult.



% Figure environment removed

Recently, there have been considerable progress in the experimental realization of layered or van der Waals (vdW) materials\cite{huang2017layer,song2018giant,gong2017discovery,wang2018electric,fei2018two,deng2018gate} and their heterostructures, such as the FGT/superconductor hybrid systems\cite{hu2023long}, $\rm{CrO_2}$/superconductor heterostructures\cite{keizer2006spin,zhang2020tunable} and $\rm{CrBr_3/NbSe_2}$ heterostructures\cite{kezilebieke2020topological,kezilebieke2021electronic,kezilebieke2022moire}. In these heterostructures, a strong superconducting proximity effect has been confirmed on the surface of the ferromagnetic metal, thereby raising the interesting possibility of topological superconducting surface states. Previous first principle calculations predict considerable intrinsic spin-orbit coupling in the ferromagnetic metal\cite{kim2018large,alghamdi2019highly}. Meanwhile, Rashba spin-orbit tunneling between the ferromagnetic metal and the superconductor is unavoidable due to the inversion symmetry breaking on the interface \cite{chung2011topological}. Therefore, these ferromagnetic metal-superconductor heterostructures provide an interesting playground for investigating the interplay between the intrinsic spin-orbit coupling and the Rashba spin-orbit hopping. In particular, one would expect competitions when there is an intrinsic Dresselhaus spin-orbit coupling in the ferromagnetic metal. 


In this work, we study a minimal model of ferromagnetic metal and s-wave superconductor heterostructure as shown in Fig. \ref{Fig:Model}. The ferromagnetic metal film has an intrinsic Dresselhaus spin-orbit coupling and there are electrons hopping across the interface, this model allows for both spin-conserving and spin-flip hoppings. Due to the inversion symmetry breaking, spin-flip hopping can be introduced by Rashba spin-orbit hopping.
We study the $\mu-V$ phase diagram under three different scenarios, where the intrinsic spin-orbit coupling dominates, the two spin-orbit coupling magnitudes are comparable, and the inter-layer spin-orbit coupling dominates, respectively, and provide some analytical expressions for the phase boundaries. We find that when the Fermi pocket of the superconductor is small, the phase diagram is very rich when considering the tight-binding Hamiltonian of the system, with multiple regions of Chern numbers -2, -1, 0, 1, and 2. In particular, if the superconducting Fermi pocket is chosen to be at the maximum Fermi pocket, an unstable region with a Chern number of 4 will also appear. We choose typical points corresponding to different Chern number regions with a small superconducting Fermi pocket and plot their Berry curvatures in momentum space and we study the thermal Hall conductivity curves corresponding to these Berry curvatures. Finally, we study the boundary states and the distribution of boundary wave functions corresponding to different Chern numbers. 

The remainder of this paper is organized as follows. In Sec. II, we introduce the ferromagnetic metal and s-wave superconductor two-layer model that is considered in the work, and we show the phase diagram which exhibits topological states with different Chern numbers. In Sec. III, we show the Berry curvatures in the momentum space which construct the Chern numbers and the thermal Hall conductivity curves corresponding to these Berry curvatures. In Sec. IV, we demonstrate the dispersion of the edge states with open boundary conditions and show the real-space distribution of the edge state wave functions. Finally, we give a summary and discussions in Sec. V.




\section{Model and phase diagram}
In the designs of realistic topological superconducting systems, one of the main theoretical proposals is the heterostructure of conventional superconductors and spin-orbit coupling semiconductors. In these designs, the central ingredient is the spin-orbit coupling. To be specific, there are two several different methods to incorporate the spin-orbit coupling into the system. One proposal is to consider a semiconductor with intrinsic spin-orbit coupling \cite{fu2008TSCvortex,sau2010twodimension,sato2010topological}, then the direct proximity of the superconducting pairing will induce topological superconductivity with proper Zeeman field. Another proposal involves a half-metal without intrinsic spin-orbit coupling. In contrast, the proximity of the superconducting pairing must include a Rashba spin-orbit inter-layer hopping which flips the spin of the Cooper pair\cite{chung2011topological}. Here, we consider a minimal model which takes account of these two possible processes. This minimal model is a double-layer tight-binding model of square-lattice, as shown in Fig. \ref{Fig:Model}. The ferromagnetic metal layer is described by the Hamiltonian
\begin{eqnarray}
H_{\textrm {FM}} &&=-t_1\sum_{\langle ij \rangle\alpha}c^{\dagger}_{i\alpha}c_{j\alpha}-\mu_1 \sum_{i\alpha}c^{\dagger}_{i\alpha}c_{i\sigma}\\
&&+i\beta \sum_{\langle ij \rangle\alpha\gamma}c^{\dagger}_{i\alpha}(\boldsymbol{\sigma}_{\alpha\gamma} \times \boldsymbol{d}_{ij})_zc_{j\gamma}+M_z\sum_{i}(c^{\dagger}_{i\uparrow}c_{i\downarrow}-c^{\dagger}_{i\downarrow}c_{i \uparrow}), \nonumber
    \label{eq:1}
  \end{eqnarray}
where $\langle ij \rangle$ represents the nearest-neighboring sites in lattice, $t_1$ is the nearest-neighbor hopping energy, $\mu_1$ is the chemical potential, $M_z$ is the effective Zeeman energy, $\beta$ represents to the strength of Dresselhaus spin-orbit coupling, $\boldsymbol{\sigma}_{\alpha\gamma}$ corresponds Pauli matrices in spin space, and $\boldsymbol{d}_{ij}$ is a unit vector which can be written as $(-(i_x-j_x), i_y-j_y, 0)$. The lattice constants are set to unity throughout this paper. 

The s-wave superconductor layer is described by the Hamiltonian
\begin{equation}
    \begin{aligned}
     H_{\textrm {SC}}= -t_2\sum_{\langle ij \rangle\alpha}f^{\dagger}_{i\alpha}f_{j\alpha}-\mu_2 \sum_{i\alpha}f^{\dagger}_{i\alpha}f_{i\alpha}+\Delta\sum_i f^{\dagger}_{i\uparrow}f^{\dagger}_{i\downarrow},
    \end{aligned}\label{eq:2}
  \end{equation}
where $f^{\dagger}_{i\alpha}$ creates an electron with spin $\alpha$ at site $i$, $t_2$ is the nearest-neighbor hopping energy, $\mu_2$ is the chemical potential, and $\Delta$ represents the superconducting order parameter.

This system also contains an additional term for the interaction between the two layers and the Hamiltonian can be written as
\begin{equation}
    \begin{aligned}    
    H_{\textrm {I}}={\alpha}_0\sum_{i\alpha}f^{\dagger}_{i\alpha}c_{i\alpha}+i\alpha_1 \sum_{\langle ij \rangle\alpha\gamma}c^{\dagger}_{i\alpha}(\boldsymbol{\sigma}_{\alpha\gamma} \times \boldsymbol{d}^{\prime}_{ij})_zf_{j\gamma},
    \end{aligned}\label{eq:3}
  \end{equation}
where $\alpha_0$ and $\alpha_1$ represent the hopping parameters for spin-flip hopping and spin-conserving hopping when electrons hopping across the interlayer, respectively, $\boldsymbol{d}^{\prime}_{ij}$ can be written as $(i_x-j_x, i_y-j_y, 0)$, is the unit projection vector from j to i in $x-y$ plane.  
The total Hamiltonian for this model in real space can be written as $H=H_{\rm{FM}}+H_{\rm{SC}}+H_{\rm{t}}$, which is a summation of these three terms.


The real-space tight-binding Hamiltonian can be transformed to the momentum-space when we consider periodic boundary condition. In the presence of superconductivity, it is more convenient to formulate the total Hamiltonian in the Bogoliubov–de Gennes (BdG) form, which is written as
\begin{equation}
H=\frac{1}{2}\sum_{\boldsymbol{k}} \psi_{\boldsymbol{k}}^{\dagger}H_{\rm{BdG}}(\boldsymbol{k})\psi_{\boldsymbol{k}},
\label{eq:4}
\end{equation}	
where we define the Nambu spinor operator $\psi_{\boldsymbol{k}}^{\dagger}=(c_{{\boldsymbol{k}}\uparrow}^{\dagger},c_{{\boldsymbol{k}}\downarrow}^{\dagger},f_{{\boldsymbol{k}}\uparrow}^{\dagger},f_{{\boldsymbol{k}}\downarrow}^{\dagger},c_{-{\boldsymbol{k}}\uparrow},c_{-{\boldsymbol{k}}\downarrow},f_{-{\boldsymbol{k}}\uparrow},f_{-{\boldsymbol{k}}\downarrow})$, and the mean-field BdG Hamiltonian is written as
\begin{equation}
\begin{aligned}
H_{\rm{BdG}}(\boldsymbol{k}) &= \frac{1}{2}\epsilon _1(\boldsymbol{k})\tau_z(s_0+s_z)\sigma_0+\frac{1}{2}\epsilon _2(\boldsymbol{k})\tau_z(s_0-s_z)\sigma_0\\
&+\frac{1}{2}\beta \sin k_x \tau_z(s_0+s_z)\sigma_y + \frac{1}{2}\beta \sin k_y \tau_0(s_0+s_z)\sigma_y  \\ 
&+\alpha_0 \tau_zs_x\sigma_0-\alpha_1 \sin k_x\tau_zs_z\sigma_y +\alpha_1 \sin k_y\tau_0s_z\sigma_x \\
&+\frac{1}{2}M_z\tau_z(s_0+s_z)\sigma_z-\frac{1}{2}\Delta \tau_y(s_0-s_z)\sigma_y,
\end{aligned}\label{eq:5}
\end{equation}
where $\epsilon _1(\boldsymbol{k})=-2t_1(\cos k_x+\cos k_y)-\mu_1$, $\epsilon _2(\boldsymbol{k})=-2t_2(\cos k_x+\cos k_y)-\mu_2$, and $\boldsymbol{\tau}, \boldsymbol{s}, \boldsymbol{\sigma}$  are Pauli matrices in the Numbu (particle-hole), interlayer, and spin degrees of freedom, respectively. 



The topology of this two-layer system is characterized by the BdG Hamiltonian $H_{\rm BdG}(\boldsymbol{k})$, which is effectively a single particle Hamiltonian with particle-hole symmetry. The system has Zeeman energy which breaks the time-reversal symmetry. As a result, the chiral symmetry of the system is also broken, and the system belongs to the D class \cite{schnyder2008classification,ryu2010topological,chiu2016classification} in the topological classification. The topological number is the Chern number which is the summation of the Berry curvatures in the momentum space
\begin{equation}
\begin{aligned}
\Omega_{\mu\upsilon}^n(\boldsymbol{R})=i\sum_{m\neq n}\frac{\left \langle m|\partial H/\partial R^{\mu}|n \right \rangle \left \langle n|\partial H/\partial R^{\upsilon}|m \right \rangle}{(E_m-E_n)^2}-(\mu \leftrightarrow \upsilon).
\end{aligned}\label{eq:6}
\end{equation}
We sum all of the $\Omega_{\mu\upsilon}^n$ with $E<0$ to calculate the Chern number in this system. 

 We calculate the Chern number of the BdG Hamiltonian and find that this two-layer system exhibits multiple topological phases with different Chern numbers. The complicated phase diagram comes from the competition between the Dresselhaus spin-orbit coupling and the Rashba spin-orbit inter-layer hopping. To reveal this competition, we demonstrate phase diagrams for the Chern number with four typical parameters in Fig. \ref{Fig:Phase}. We first demonstrate the scenario where the Dresselhaus spin-orbit coupling dominates the Rashba spin-orbit inter-layer hopping. As shown in Fig. \ref{Fig:Phase}(a), we find a phase diagram that qualitatively replicates the phase diagram of the single-layer models, which was studied in previous works\cite{sau2010twodimension,sato2010topological,sato2010non}. There are three topological regions with two distinct Chern numbers of one and minus two. The phase boundary also closely resembles the single-layer system. The gap of $H_{\rm{BdG}}(\boldsymbol{k})$ closes at momenta $\Gamma(0,0)$, $X(0,\pi),(0,-\pi),(\pi,0),(-\pi,0)$, $M(\pi,\pi),(\pi,-\pi),(-\pi,\pi),(-\pi,-\pi)$ points. For the case that gap closes at $\Gamma$, the intrinsic Dresslhaus spin-orbit coupling and interlayer Rashba spin-orbit hopping terms in $H_{\rm{BdG}}$ are equal to zero, we have $H_{\rm{BdG}}(\Gamma)\Psi=E\Psi$, and the secular equation of the eigenvalue E is $\det[H_{\rm{BdG}}(\Gamma)-E]=0$, so we can write down the phase transition between the topological trivial phase and the topological phase with Chern number equals to one (the case when the Fermi pocket is at the bottom of the ferromagnet metal' energy band) as
\begin{equation}
\begin{aligned}
M_{z1}=\sqrt{(a_1{\mu}_1^2+a_2{\mu}_1+a_3)/a_1},
\end{aligned}\label{eq:7}
\end{equation}  
where $a_1=16t_2^2-8t_2{\mu}_2+{\mu}_2^2+{\Delta}^2$, $a_2=-2\alpha_0^2({\mu}_2-4t_2)-8t_1a_1$, and $a_3=\alpha_0^4-8\alpha_0^2t_1(4t_2-{\mu}_2)+16t_1^2a_1$. Similarly, we can also write the phase transition boundary expression between the topological trivial phase and the topological phase with the Chern number equals to minus two(the gap closes at X points) as
\begin{equation}
\begin{aligned}
M_{z2}=\sqrt{(b_1{\mu}_1^2+b_2{\mu}_1+b_3)/b_1},
\end{aligned}\label{eq:8}
\end{equation}	
where $b_1={\mu}_2^2+{\Delta}^2$, $b_2=-2\alpha_0^2{\mu}_2$, and $b_3=\alpha_0^4$. We notice that this expression for the phase boundary resembles the formula given in Ref. [\onlinecite{sau2010twodimension}], although the parameters become much more complicated due to the two-layer nature of the model. In Fig. \ref{Fig:Phase}(b), we demonstrate the scenario where the Rashba spin-orbit inter-layer hopping dominates the Dresselhaus spin-orbit coupling. In this scenario, the intrinsic spin-orbit coupling of the ferromagnetic metal is negligible and we simply come back to the half-metal/superconductor model that was introduced in Ref. [\onlinecite{chung2011topological}]. We find that the phase diagrams are similar to Fig. \ref{Fig:Phase}(a), except that the Chern numbers have a sign reversal.


% Figure environment removed


% Figure environment removed

Then we consider the more interesting scenario where the impacts of the Dresselhaus spin-orbit coupling and the Rashba spin-orbit inter-layer hopping are comparable. In this case, the phase diagram exhibits complicated topological ordering as shown in Fig. \ref{Fig:Phase}(c). There are a number of distinct phase regions with five distinct Chern numbers of plus/minus one, plus/minus two, and zero. Looking at the boundary of the phase transition, we find that the phase transition between topological states and trivial states as shown in Eqs. (\ref{eq:7}) and (\ref{eq:8}) still exist. However, there are additional phase transitions that enrich the phase diagram. In particular, we find phase transitions between the topological states of $C=1$ and $C=-1$. This phase transition comes from the competition between the Dresselhaus spin-orbit coupling and the Rashba spin-orbit inter-layer hopping. Let us reveal this competition by looking closely at the single-particle Hamiltonian of the two-layer system
\begin{equation}       %开始数学环境
H = \left(                 %左括号
  \begin{array}{cc}   %该矩阵一共3列,每一列都居中放置
    (\epsilon _1(\boldsymbol{k})\sigma_0+M_z\sigma_z+\beta(\sin k_y\sigma_x+\sin k_x\sigma_y) &  c.p.\\  %第一行元素
    \alpha_0\sigma_0+\alpha_1(\sin k_y\sigma_x+\sin k_x\sigma_y) &\epsilon _2(\boldsymbol{k})\sigma_0 \\  %第二行元素
  \end{array}\label{eq:9}
\right).              %右括号
\end{equation}
This is a $4 \times 4$ matrix where the upper left $2 \times 2$ blocks describe the ferromagnetic metal and the lower right $2 \times 2$ block describes the conventional superconductor.
In this $4 \times 4$ matrix, we notice that the Dresselhaus spin-orbit coupling and the Rashba spin-orbit inter-layer hopping stay at different $2 \times 2$ blocks. However, they can be put together by diagonalizing the ferromagnetic metal block with a unitary transformation,
\begin{equation}       %开始数学环境
U=
\left(                 %左括号
  \begin{array}{cc}   %该矩阵一共3列,每一列都居中放置
    e^{\frac{i\theta}{2}\hat{n}\cdot \hat{\sigma}} &  0\\  %第一行元素
    0& \sigma_0 \\  %第二行元素
  \end{array}\label{eq:10}
\right)    ,             %右括号
\end{equation}
where $\theta=\arctan (\beta\sqrt{(\sin k_x^2+\sin k_y^2)}/M_z)$, and $\hat{n}=(-\sin k_x,\sin k_y,0)/\sqrt{\sin k_x^2+\sin k_y^2}$ is a unit vector. After this unitary transformation, the Hamiltonian is written as
\begin{equation}       %开始数学环境
H = \left(                 %左括号
  \begin{array}{cc}   %该矩阵一共3列,每一列都居中放置
    \epsilon _1(\boldsymbol{k})\sigma_0+\frac{M_z}{\cos \theta}\sigma_z & h_{12}\\  %第一行元素
    h_{21}& \epsilon _2(\boldsymbol{k}) \\  %第二行元素
  \end{array}\label{eq:11}
\right),                 %右括号
\end{equation}
where 
 \begin{equation}
\begin{aligned}
h_{12}
&=(\alpha_0\cos \frac{\theta}{2}-2i\alpha_1\sin \frac{\theta}{2}\sin k_x \sin k_y/\sqrt{\sin k_x^2+\sin k_y^2})\sigma_0\\
&+(\alpha_1\cos \frac{\theta}{2}\sin k_y-i\alpha_0sin\frac{\theta}{2}\sin k_x/\sqrt{\sin k_x^2+\sin k_y^2})\sigma_x\\
&-(\alpha_1\cos \frac{\theta}{2}\sin k_x-i\alpha_0sin\frac{\theta}{2}\sin k_y/\sqrt{\sin k_x^2+\sin k_y^2})\sigma_y \\
&-(\alpha_1\sin \frac{\theta}{2}(\sin ^2k_x-\sin ^2k_y)/\sqrt{\sin k_x^2+\sin k_y^2})\sigma_z  ,
\end{aligned} \label{eq:12}
\end{equation}	
we notice that both Dresselhaus spin-orbit coupling and the Rashba spin-orbit inter-layer hopping are now transformed to the off-diagonal $2\times 2$ block of the Hamiltonian. The model is now equivalent to the half-metal/superconductor system introduced by Ref. [\onlinecite{chung2011topological}], where the spin-orbit inter-layer hopping is a combination of the Dresselhaus and the Rashba terms. By simplifying the $h_{12}$ in Eq. (\ref{eq:11}), we can get the real and imaginary terms of the upper right part in $h_{12}$, respectively,

 \begin{equation}
\begin{aligned}
&Re (h_{12})=(\alpha_1\cos \frac{\theta}{2}+\alpha_0\sin \frac{\theta}{2}/\sqrt{\sin k_x^2+\sin k_y^2})\sin k_y \\
&Im (h_{12})=(\alpha_1\cos \frac{\theta}{2}-\alpha_0\sin \frac{\theta}{2}/\sqrt{\sin k_x^2+\sin k_y^2})\sin k_x, \\
\end{aligned}\label{eq:13}
\end{equation}  
when the off-diagonal elements of the $h_{12}$ term are $\sin k_x+i\sin k_y$ and $\sin k_x-i\sin k_y$, the system corresponds to effective $p+ip$ and $p-ip$ topological superconductivity, respectively, and their corresponding Chern numbers are opposite in sign. So we can get the phase transition condition: $tan\frac{\theta}{2}=\frac{\alpha_1}{\alpha_0}\sqrt{\sin k_x^2+\sin k_y^2}$, under which, there is a phase transition between $C=1$ and $C=-1$.


Finally, we demonstrate topological phases with a Chern number as large as four, shown in Fig. \ref{Fig:Phase}(d). This topological phase with a large Chern number requires a delicate balancing of the parameters. For example, the Fermi pockets of both the ferromagnetic metal and the superconductor have to overlap near the nesting position. This large Chern number is a result of the two-layer nature of the model. As we will show, it requires all four bands of the system to be topological. Therefore, it does not appear in the previous one-layer models\cite{sau2010twodimension}.


% Figure environment removed


\section{Berry curvature}
The topological phase diagram with different Chern numbers can be understood more clearly by checking the Berry curvatures in the momentum space.
For this purpose, we illustrate Berry curvatures for typical Chern numbers in Fig. \ref{Fig:Berry}. We first look at the Berry curvatures with $C= \pm 1$, where one Majorana chiral mode is expected at the edge, and Majorana zero mode would appear in the presence of a superconducting vortex. As shown in Fig. \ref{Fig:Berry}(a) and Fig. \ref{Fig:Berry}(b), The Berry curvatures peak at a circle in the Brillouin zone. In fact, these circles are exactly the Fermi pocket of the ferromagnetic metal. Due to the relatively large Zeeman energy, the two bands of the ferromagnetic metal split, and only one of them intersects with the Fermi energy. 
The Dresselhaus spin-orbit coupling or the Rashba spin-orbit inter-layer hopping modulates the Bogoliubov eigenstates around the Fermi pockets and induces Berry curvatures that sum to the Chern number $C= \pm 1$.
In Fig. \ref{Fig:Berry}(c) and Fig. \ref{Fig:Berry}(d), we show the Berry curvatures corresponding to the Chern numbers $C=\pm 2$. In these cases, the chemical potential is lifted so that it touches both the energy bands of the ferromagnetic metal, even though they energetically split by the Zeeman energy. In this regime, the Berry curvatures also concentrate around the two Fermi pockets of the ferromagnetic metal. We notice that the Fermi pockets of the ferromagnetic metal are near nesting, which resembles the results of the one-layer model. While roughly one Fermi pocket contributes a Chern number of plus/minus one, the total summation of the Berry curvatures provides a Chern number of plus/minus two. 
The Berry curvatures of the topological trivial state with $C=0$ are shown in Fig. \ref{Fig:Berry}(e). It is clear that the Berry curvatures are non-vanishing while the summation is zero.
Finally, we examine the topological states with the large Chern number of $C=4$. As shown in Fig. \ref{Fig:Berry}(f), the Berry curvatures peak at the four Fermi pockets of the ferromagnetic metal and the superconductor. These four Fermi pockets are all near nesting, which causes complicated inter-band coupling effects. We believe that this near-nesting Fermi pocket is the key ingredient to achieving the large Chern numbers in this system.
% Figure environment removed

The Berry curvatures are important to transport properties. Here, we show the temperature dependence of thermal Hall conductivity with different Chern numbers in Fig. \ref {Fig:ThermalHall}, the thermal Hall conductivity obtained using the particle-hole symmetric BdG bands is given by\cite{smrcka1977transport,vafek2001quasiparticle,cvetkovic2015berry}
 \begin{equation}
\begin{aligned}
\kappa_{xy}=\frac{1}{T}\sum_n\int[d\textbf{k}](\Omega_{n\textbf{k}})\int_{E_{n\textbf{k}}}^{\infty}d\eta\eta^2f'(\eta,T),
\end{aligned}\label{eq:14}
\end{equation}
where $f(E_{n\textbf{k}},T)=1/(e^{E_{n\textbf{k}}/T}+1)$ is the Fermi-Dirac distribution at temperature $T$, $f'$ is its derivative with respect to $E_{n\textbf{k}}$, $E_{n\textbf{k}}$ and $\Omega_{n\textbf{k}}$ are energy and the Berry curvature in momentum space with the energy band index $n$, respectively.
We find that the low-temperature limit of the thermal Hall conductivity is determined by the Chern number and the universal value\cite{PhysRevLett.126.187001} of $\kappa_0=\pi C_1k_B^2T/6\hbar$. In the high-temperature limit, the thermal Hall conductivity gradually decays to zero. We particularly note that the thermal Hall conductivity is non-vanishing even for zero Chern numbers. As shown in Fig. \ref {Fig:ThermalHall}(a), the thermal Hall conductivity is actually in the same order as those for nonzero Chern numbers at finite temperatures. The reason for the significant thermal Hall conductivity in the topological trivial regime is well understood from the Berry curvature distribution shown in Fig. \ref{Fig:Berry}(e). The Berry curvatures are comparable to the topological nontrivial regions even though their summation gives a zero Chern number. These transport signals are experimentally measurable and would provide information for both the Chern number and the Berry curvatures of topological superconducting systems.

\section{Chiral Majorana Edge states}
For open boundary systems, the bulk-edge correspondence predicts the chiral Majorana modes in the class D topological states with nonzero Chern numbers \cite{PhysRevB.82.184516, PhysRevB.92.064520}. To explicitly demonstrate the chiral Majorana edge modes, we consider the two-dimensional lattice with periodic boundary condition in the x direction and open boundary condition (OBC) along the y direction. With this boundary condition, the momentum in the x direction is still a valid quantum number while the momentum in the y direction is not.
We numerically solve the Bogoliubov–de Gennes equation for this open boundary system, and show the energy spectra as a function of the momentum in the x direction, as shown in Fig. \ref{Fig:edgestate}.
For the topological states with odd Chern numbers of $C=\pm1$, we find one Majorana edge mode which propagates along the opposite direction at the different edges. 
In Fig. \ref {Fig:edgestate}(a), the blue edge localizes at the left boundary moving downwards with velocity $v_y<0$, while the red edge state localizes at the right boundary moving upwards with velocity $v_y>0$. In Fig. \ref {Fig:edgestate}(b), the chirality of the Majorana modes flips when the Chern number changes sign.


For larger Chern numbers of $C = 2$, we find two chiral Majorana edge states propagating in the same direction, as shown in Fig. \ref {Fig:edgestate}(c). These two chiral Majorana modes are energetically split, therefore can not be simply understood as part of the same chiral Dirac mode. In Fig. \ref {Fig:edgestate}(d), we show the edge states for $C=-2$, it is obvious that the propagation direction of edge states is reversed, consistent with expectation from the Chern number.


Finally, we show the results of topological trivial states with zero Chern number. For most of the trivial states, we find the absence of any edge states as shown in Fig. \ref {Fig:edgestate}(e). However, 
In some cases, we find unexpected edge states even in the trivial region. As shown in Fig. \ref {Fig:edgestate}(f), two chiral edge modes are propagating along opposite directions. These edge states are obviously unrelated to the Chern number. Instead, they are topologically protected by a winding number. To study this topological winding number, we examine the one-dimensional Hamiltonian $H(k_x)$ by setting $k_y=0$, 
\begin{equation}
\begin{aligned}
&H(k_x)=\frac{1}{2}\epsilon _1(k_x)\tau_z(s_0+s_z)\sigma_0+\frac{1}{2}\epsilon _2(k_x)\tau_z(s_0-s_z)\sigma_0\\
&+\frac{1}{2}\beta \sin k_x \tau_z(s_0+s_z)\sigma_y + \alpha_0 \tau_zs_x\sigma_0-\alpha_0 \sin k_x\tau_zs_z\sigma_y\\
&+\frac{1}{2}V\tau_z(s_0+s_z)\sigma_z-\frac{1}{2}\Delta \tau_y(s_0-s_z)\sigma_y,
\end{aligned}\label{eq:151}
\end{equation}
where $\boldsymbol{\tau}$, $\boldsymbol{s}$, $\boldsymbol{\sigma}$ are Pauli matrices that act in the Nambu (particle-hole), layer, and spin spaces. This one-dimensional Hamiltonian has both particle-hole symmetry and chiral symmetry. We can define particle-hole operator $\hat{P}=\tau_x s_0 \sigma_0 \hat{\kappa}$, where $\hat{\kappa}$ is the complex conjugation operator, and the particle-hole symmetry is explicitly written as $H(k_x)=-\hat{P}H(-k_x)\hat{P}^{\dagger}$. The chiral operator is defined as $\hat{C}=\tau_x s_0 \sigma_0$, and the Chiral symmetry is written as $H(k_x)=-\hat{C}H(k_x)\hat{C}^{-1}$. The combination of these two symmetry operations gives $H(k_x)=H^{\top}(-k_x)$, which suggests that the one dimensional Hamiltonian
$H(k_x)$ belongs to class BDI \cite{ryu2010topological} in the tenfold way of Altland-Zirnbauer classification. Then the winding number can be defined as\cite{sato2011topology}
 \begin{equation}
\begin{aligned}
w=\frac{1}{4\pi i}\int_{-\pi}^{\pi}dk_x Tr[\hat{C}H^{-1}\partial_xH].
\end{aligned}\label{eq:16}
\end{equation}
We calculate the winding number for Fig. \ref{Fig:edgestate}(f), and find that $w = 2$ which is consistent with the number of edge states. Similarly, we can calculate the one-dimensional Hamiltonian $H(k_y)$ when studying the open boundary condition in the x-direction.

% Figure environment removed

Since we are studying the two-layer system, we would like to examine the distribution of the wave functions of the edge states. In Fig. \ref{Fig:Psi}, we show the real-space distribution of the wave functions corresponding to the Chern number $C=-1$. The red line represents the wave functions in the ferromagnetic metal, while the blue line corresponds to the wave functions of the superconductor. It can be seen that the wave functions are mainly distributed in the ferromagnetic layer.


\section{conclusion}
In summary, we have studied a double-layer model consisting of spin-orbit coupling ferromagnetic metal and s-wave superconductor. The ferromagnetic metal layer has the intrinsic Dresselhaus spin-orbit coupling, while the two layers have the Rahsba spin-orbit interlayer hopping.
We calculated the Chern numbers of the system and demonstrated the phase diagrams. We found topological phases with five different Chern numbers. In particular, we found that the Chern number can be as large as four if the parameters of the systems are well controlled. We illustrated the Berry curvatures and showed that there are non-vanishing Berry curvature distributions in the momentum space even in the topological trivial regime. We calculated the thermal Hall conductivity governed by the Berry curvatures. We revealed the chiral Majorana edge states protected by Chern numbers and by the winding numbers. We found that the wave functions of these edge states mostly distribute in the ferromagnetic metal layer.




\textit{Acknowledgments.---} We thank Zhongbo Yan, Jun He, and Dingyong Zhong for the valuable discussions. This project is supported by NKRDPC-2022YFA1402802, NKRDPC-2018YFA0306001, NSFC-92165204, NSFC-11974432, NSFC-12174453 and Shenzhen Institute for Quantum Science and Engineering.  

\bibliographystyle{apsrev4-2} 
\bibliography{Feynman}



\end{document}