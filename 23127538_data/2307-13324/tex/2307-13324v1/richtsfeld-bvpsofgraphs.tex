\documentclass[11pt]{article}
\usepackage[a4paper, total={5.5in, 8in}]{geometry}

\usepackage[utf8]{inputenc}
\usepackage[english]{babel}

\usepackage{amsmath}
\usepackage{amssymb}
\usepackage{amsfonts}
\usepackage{amsthm}
\usepackage{mathtools}
\usepackage[all,cmtip]{xy}
\usepackage{csquotes}

\usepackage{authblk}
\usepackage{hyperref}


\usepackage{floatrow}

\swapnumbers
\theoremstyle{plain}

\newtheorem{thm}{Theorem}
\numberwithin{thm}{section} 
\newtheorem{prop}[thm]{Proposition}
%\numberwithin{prop}{subsection} 
\newtheorem{lem}[thm]{Lemma}
%\numberwithin{lem}{subsection} 
\newtheorem{cor}[thm]{Corollary}
%\numberwithin{cor}{subsection} 
\newtheorem{defin}[thm]{Definition}
%\numberwithin{defin}{subsection} 

\theoremstyle{definition}
\newtheorem{exa}[thm]{Example}
%\numberwithin{exa}{subsection} 

\theoremstyle{remark}
\newtheorem*{rmk}{Remark}

\DeclareMathOperator{\dom}{dom}
\DeclareMathOperator{\tr}{tr}
\DeclareMathOperator{\dt}{dt}
\DeclareMathOperator{\ds}{ds}
\DeclareMathOperator{\R}{Re}
\DeclareMathOperator{\I}{Im}
\DeclareMathOperator{\spanc}{span}
\DeclareMathOperator{\ind}{ind}
\DeclareMathOperator{\coker}{coker}
\DeclareMathOperator{\id}{id}
\DeclareMathOperator{\im}{im}
\DeclareMathOperator{\diag}{diag}
\DeclareMathOperator{\pr}{pr}
\DeclareMathOperator{\inc}{i}
\DeclareMathOperator{\aut}{Aut_{\mathbb{C}}}
\DeclareMathOperator{\ran}{ran}
\DeclareMathOperator{\vol}{dvol}
\DeclareMathOperator{\supp}{supp}
\DeclareMathOperator{\dmu}{d\mu}
\DeclareMathOperator{\dx}{dx}
\DeclareMathOperator{\dy}{dy}
\DeclareMathOperator{\dr}{dr}
\DeclareMathOperator{\dth}{d\theta}
\DeclareMathOperator{\ad}{ad}
\DeclareMathOperator{\sgn}{sgn}


\def\MakeUppercaseUnsupportedInPdfStrings{\scshape}


\newcommand*{\email}[1]{%
	\normalsize\href{mailto:#1}{#1}\par
}

\title{Boundary Value Problems for Dirac Operators on Graphs}
\author{Alberto Richtsfeld}
\affil{Institut für Mathematik,\\
	Universität Potsdam, \\
	D-14476, Potsdam, Germany\\ \email{richtsfeld@uni-potsdam.de}}
\date{}


\begin{document}

\maketitle
	
\begin{abstract}
	  We carry the index theory for manifolds with boundary of Bär and Ballmann over to first order differential operators on metric graphs. This results in an elegant proof for the index of such operators. Then the self-adjoint extensions and the spectrum of the Dirac operator on the complex line bundle are studied. We also introduce two types of boundary conditions for the Dirac operator, whose spectrum encodes information of the underlying topology of the graph.
\end{abstract}
	
\section{Introduction}

	Schrödinger operators on metric graphs, known as quantum graphs, have a long history of study with applications in many fields, including mathematics, physics and engineering. They provide toy models for quantum mechanics, but also describe the motion of particles through thin metallic wires called nanowires. We will mainly focus on the following mathematical aspects of the theory of quantum graphs, namely self-adjoint extensions, which are realised by imposing boundary conditions at each vertex, and the index theory of these operators. A full treatment of self-adjoint extensions can be found in \cite{FullingKuchmentWilson, Harmer, KostrykinSchrader, Kuchment2014}, which is summarised by Kuchment \cite{kuchment2008quantum}, the take-away being that for the Schrödinger operator self-adjoint extensions are given by a combination of Dirichlet, von Neumann and Robin boundary conditions. Index theory for quantum graphs was treated by Fulling, Kuchment and Wilson \cite{FullingKuchmentWilson}, relating the index to heat kernel asymptotics.  A nice overview of the study of quantum graphs is given for example by Kuchment \cite{kuchment2008quantum}.
	
In order to study the effect of spin on the quantum mechanics of graphs, the Dirac operator has to be considered. 
The first ones to study a two-dimensional, quantum-mechanical Dirac operator on graphs were Bulla and Trenkler \cite{bulla}. They studied self-adjoint extensions of the Dirac operator on directed graphs and computed its spectrum on a junction of three wires, a graph for which the Schrödinger operator had previously been considered by Exner and \v{S}eba \cite{exner}. Bolte and Harrison \cite{bolte} considered this Dirac operator in a similar way to the approach taken in this paper: the graph under consideration is metric in the sense that each edge has an associated length, and its boundary conditions are given by a unitary matrix describing complex transition probability amplitudes. Post \cite{post} introduced a discrete notion of the Dirac operator and computed the index of the discrete and metric Dirac operators, by finding that it is equal to the Euler-characteristic of the graph. For more information on the study of Dirac operators on graphs, see the survey article \cite{harrisonsurv} written by Harrison.

The point of view in this paper is as follows: A metric graph can be reinterpreted as a one-dimensional Riemannian manifold with boundary, where each connected component is isomorphic to a closed interval. We then consider the Dirac operator on this one-dimensional manifold, subject to boundary conditions that respect the underlying graph structure. The aim is to study the self-adjoint extensions and the index of this operator, and to find interesting boundary conditions.

The index theory of Dirac operators on manifolds with boundary was initiated by the seminal paper of Atiyah, Patodi and Singer \cite{aps}, whose main achievement was the discovery of a non-local boundary condition, formulated in terms of the spectrum of the boundary operator.
The theory centred on boundary value problems for Dirac operators was then simplified and unified by Bär and Ballmann \cite{bb, bb2}, in which they characterised all extensions of the Dirac operators with linear subspaces of the czech-space, a Banach space formed from the eigenspaces of the boundary operator. These results were then extended by Bär and Bandara \cite{lashi} to general first-order differential operators. We will work in the spirit of the Bär-Ballmann framework, but the theory does not apply directly to the one-dimensional case, since there is no notion of a boundary operator in this case, since the boundary is zero-dimensional.

 The paper is structured as follows: First, we consider general first-order differential operators over metric graphs with coefficients in a general bundle, without specifying the boundary conditions. Using techniques developed by Bär and Ballmann, we quickly obtain a general index formula for such operators. We then turn our attention to the scalar-valued Dirac operator and explore properties of its spectrum. In the last part we discuss two types of boundary conditions. The first is given by certain permutation matrices which encode decompositions of the underlying graph into directed trails, the spectrum of the Dirac operator is then explicitly calculable and is fully determined by the lengths of the trails appearing in the decomposition. The second boundary condition is given by the directed adjacency matrix, which distributes the inflow at a vertex evenly among the outgoing edges. This boundary conditions is in general no longer self-adjoint, but it links the spectrum of the Dirac operator to the appearance of cycles in the graph. Thus, both types of boundary conditions reveal properties of the cycles and trails present in the graph, linking its topology to the spectrum of the Dirac operator.
\newpage
\section{Preliminaries}
A metric graph $G=(V,E,l)$ is given by a directed multigraph with vertex set $V$ and edge set $E$ and a length function $l: E\rightarrow (0,\infty)$, $e\mapsto l_e$. For each edge $e\in E$ we denote by $\partial_-e$ the tail of $e$ and by $\partial_+e$ the head of $e$. We then interprete $G$ as the following topological space
\[G = \bigsqcup_{e\in E}[0,l_e]/_\sim,\]
where $x\sim y$ iff $\psi(x)=\psi(y)$ with 
\[\psi : \bigsqcup_{e\in E}\{0,l_e\}\rightarrow V\qquad 0_e\mapsto \partial_-e, \quad l_e\mapsto \partial_+ e.\]
We denote the coordinate on the egde $e$ by $x_e$. For our purposes, we will always assume that our metric graph is finite and has no isolated vertices.

We define now the following function spaces on $G$:
\[L^2(G,\mathbb{C}^r)= \bigoplus_{e\in E} L^2([0,l_e], \mathbb{C}^r),\quad H^1(G,\mathbb{C}^r)=\bigoplus_{e\in E}H^1([0,l_e],\mathbb{C}^r),\] 
\[ C^\infty(G,\mathbb{C}^r)=\bigoplus_{e\in E} C^\infty ([0,l_e],\mathbb{C}^r).\]


Standard Sobolev estimates now tell us that each function in $H^1(G,\mathbb{C}^r)$ is continuous when restricted to a single edge and that there is a well-defined bounded operator 
\[\tr : H^1(G,\mathbb{C}^r)\rightarrow \mathbb{C}^r[G]\qquad f\mapsto \sum_{e\in E}f(0_e)e_-+f(l_e)e_+\]
where $\mathbb{C}^r[G]\cong \mathbb{C}^{r\lvert E\rvert}\oplus \mathbb{C}^{r\lvert E\rvert}$. This allows us to introduce boundary conditions on the graph: A boundary condition is a linear subspace $B\subseteq \mathbb{C}^r[G]$, we then denote the space of $H^1$-functions which fulfill the boundary condition $B$ by
\[H^1_B(G,\mathbb{C}^r)=\{f\in H^1(G,\mathbb{C}^r)\mid \tr(f)\in B\}.\]
If $B=\{0\}$, we write $H^1_0(G,\mathbb{C}^r)$ instead.
Letting $$\mathbb{C}^{r, V}:=\left\{\sum_{v\in V}c_v\cdot v\mid c_v\in \mathbb{C}^r\right\},$$ 
for a vertex $v\in V$, we can define $\mathbb{C}_v^r[G]:= \psi_G^{-1}(\spanc(v))$, where $\psi_G:\mathbb{C}^r[G]\rightarrow\mathbb{C}^V$ is the linear map mapping $e_-\mapsto \partial_-e$ and $e_+\mapsto \partial_+e$. We now say that a boundary condition $B$ is \textit{local} iff there are linear subspaces $B_v\subseteq \mathbb{C}_v^r[G]$, such that 
\[B=\bigoplus_{v\in V}B_v.\]
Furthermore, we define the following spaces:
\begin{align*}
	\mathbb{C}^{r,G}&:= \left\{\sum c_e \cdot e \mid c_e \in\mathbb{C}^r \right\}\\
	\mathbb{C}_+^r(G,v)&:= \spanc(\{e\mid e \text{ ends at }v\})\subseteq \mathbb{C}^{r,G}\\
	\mathbb{C}_-^r(G,v)&:= \spanc(\{e\mid e \text{ starts at }v\})\subseteq \mathbb{C}^{r,G}
\end{align*}
In the case that \(r=1\), we omit \(r\) in the above definitions. Note that there are canonical inclusions \(\iota_\pm:\mathbb{C}^{r,G}\rightarrow \mathbb{C}^r[G]\) sending \(e\) to \(e_\pm\) and that \(\mathbb{C}^{r,V}\) is equal to the direct sum \(\iota_+(\mathbb{C}_+^r(G,v))\oplus \iota_-(\mathbb{C}_-^r(G,v))\). We equip all so far defined vector spaces with the canonical inner products making any two elements of the generating sets orthonormal to each other. 


Given  two endomorphism fields $\sigma, A\in C^\infty(G,\mathbb{C}^{r\times r})$ we can define a first order differential operator 
\[D: C^\infty(G,\mathbb{C}^r)\rightarrow C^\infty (G,\mathbb{C}^r), \qquad f\mapsto \sigma f' +Af.\]
We will always assume that the differential operator $D$ is elliptic, i.e. that $\sigma$ is non-singular at every point. $D$ naturally extends to a bounded operator $$D_{\max}:H^1(G,\mathbb{C}^r)\rightarrow L^2(G,\mathbb{C}^r)$$ and consequently for a boundary condition $B$ we can restrict $D_{\max}$ to $H^1_B(G,\mathbb{C}^r)$, yielding the operator $D_B$.  If $B=\{0\}$, write $D_{\min}$ instead. These operators will be the center of our study. 


It is straightforward to check that the unbounded operator $$D_B: H^1_B(G,\mathbb{C}^r)\subseteq L^2(G,\mathbb{C}^r)\rightarrow L^2(G,\mathbb{C}^r)$$ is closed, densely defined and a Fredholm operator. Furthermore, we have the following Green's formula:
\begin{lem}
	For $f,g\in H^1(G,\mathbb{C}^r)$, we have that 
	\begin{equation*}
		\langle D_{\max}f,g\rangle - \langle f, D_{\max}^\dagger g\rangle = \langle \sigma_0 \tr(f),\tr(g)\rangle,
	\end{equation*}
	where $D^\dagger$ is the formal adjoint of $D$ and 
	\[\sigma_0:\mathbb{C}^r[G]\rightarrow \mathbb{C}^r[G], \qquad \sum_{e\in E} a_e e_- + b_e e_+\mapsto \sum_{e\in E} -\sigma(0_e)a_e e_- + \sigma(l_e)b_e e_+ .\]
\end{lem}

We directly obtain 
\begin{prop}\label{prop:sa}
	Let $B$ be a boundary condition, define the adjoint boundary condition by
	\[B^{\mathrm{ad}}:=\{y\in \mathbb{C}^r[G]\mid \forall b\in B: \langle \sigma_0 b, y\rangle =0\}.\]
	Then, the adjoint of $D_B$ is given by $D_{B^{\mathrm{ad}}}^\dagger$. In particular, $D_B$ is self-adjoint iff $D$ is formally self-adjoint and $B=B^{\mathrm{ad}}$.
\end{prop}
\begin{cor}
	The adjoint of $D_{\max}$ is $D^\dagger_{\min}$ and the adjoint of $D_{\min}$ is $D^\dagger_{\max}$.
\end{cor}

\section{The index of $D_B$}

By elementary ODE-theory we can solve the equation
\[D_{max}f=g\]
on each edge separetely for any $g\in L^2(G,\mathbb{C}^r)$ . Hence $D_{\max}$ is surjective. Furthermore, it is easily seen that the dimension of the solution space of the equation 
$$D_{\max}f=0$$
is equal to $r$ on each single edge, and since the kernel of $D_{\max}$ is just the direct sum of the kernels of the single edges, we obtain 
\[\dim\ker D_{\max} = r\lvert E\rvert.\]
We have therefore proven 
\begin{lem}
	The following equations hold:
	\begin{equation*}
		\ind D_{\max}  = r\lvert E\rvert, \qquad \ind D_{\min}= - r\lvert E\rvert
	\end{equation*}
\end{lem}

In order to prove the general index formula, we pursue the approach developed by Bär and Ballmann. We need the following proposition:

\begin{prop}\label{prop:bb}
	Let $H$ be a Hilbert space, $E$ and $F$ Banach spaces and let $L:H\rightarrow E$ and $P:H\rightarrow F$ be bounded linear maps. Assume that $P:H\rightarrow F$ is onto. Then, $L|_{\ker P}:\ker P\rightarrow E$ is Fredholm of index $k$ if and only if $L\oplus P:H\rightarrow E\oplus F$ is Fredholm of index $k$.
\end{prop}
\begin{proof}
	See \cite[Proposition A.1]{bb}.
\end{proof}


Analogous to \cite[Corollary 8.7]{bb}, we have:

\begin{lem}\label{cor:fred}
	Let $B$ be a subspace of $\mathbb{C}^r[G]$, $B^\perp$ be the orthogonal complement to $B$ and let $P:\mathbb{C}^r[G]\rightarrow\mathbb{C}^r[G]$ be the orthogonal projection onto $B^\perp$. Then, 
	\begin{align*}
		L:H^1(G,\mathbb{C}^r)&\rightarrow L^2(G,\mathbb{C}^r)\oplus B^\perp\\
		g&\mapsto (D_{\max}g, P\tr g)
	\end{align*}
	is a Fredholm operator with the same index as $D_B$.
\end{lem}
\begin{proof}
	The kernel of $P\tr$ is equal to $H^1_B(G,\mathbb{C}^r)$. The statement then follows directly from Proposition \ref{prop:bb}.
\end{proof}

As in \cite[Corollary 8.8]{bb}, we go on to obtain:

\begin{lem}\label{lem:deform}
	Let $B_1\subseteq B_2 \subseteq \mathbb{C}^r[G]$ be boundary conditions. Then,
	$$\ind D_{B_2}=\ind D_{B_1}+ \dim B_2/B_1.$$
\end{lem}
\begin{proof}
	Since $B_1$ is a subspace of $B_2$,
	the orthogonal complement $B_2^\perp$ is a subspace of $B_1^\perp$. Denote the inclusion of $B_2^\perp$ into $B_1^\perp$ by $\iota$. The diagram 
	\[
	\xymatrix{
	 & L^2(G,\mathbb{C}^r)\oplus B_2^\perp \ar@{^{(}->}[dd]^{\id \oplus \iota} \\
	H^1(G,\mathbb{C}^r)  \ar[ru]^{(D,Q_2)} \ar[rd]^{(D,Q_1)} \\
		&  L^2(G,\mathbb{C}^r)\oplus B_1^\perp
	}
	\]
	commutes, where, as in Lemma \ref{cor:fred}, the operator $Q_i$ is the composition of $\tr$ with the orthogonal projection onto $B_i^\perp$. Clearly, $\id\oplus\iota$ is a Fredholm operator with index
	$$\ind (\id\oplus\iota)=\ind (\iota) = -\dim(B^\perp_1/B^\perp_2) = -\dim(B_2/B_1).$$
	Since the index is additive, we have
	\begin{align*}
		\ind D_{B_1}&= \ind (D,Q_1)\\
		&= \ind (D,Q_2) + \ind \id \oplus i\\
		&= \ind D_{B_2} -\dim B_2/B_1.\qedhere
	\end{align*}
\end{proof}

\begin{thm}\label{thm:ind}
	For any boundary condition $B$, the following formula for the index of $D_B$ holds:
	\begin{equation}\label{eq:index}
		\ind(D_B)=\dim B -r|E|.
	\end{equation}
\end{thm}

\begin{proof}
	Applying Lemma \ref{lem:deform}, the index of $D_B$ is given by
	\begin{align*}
		\dim D_B &= \dim B + \ind D_0\\
		&= \dim B + \ind D_{\min}\\
		&= \dim B -r|E|.\qedhere
	\end{align*}
\end{proof}
\newpage
\section{The scalar Dirac operator}

For the rest of the paper, we will focus on the scalar Dirac operator, that is, we take \(r=1\) and set \(D= i\frac{d}{dx}\).
We will first study its self-adjoint extensions:

\subsection{Self-adjoint extensions}

Let \(A: \mathbb{C}^G\rightarrow \mathbb{C}^G\) be a linear map, we can then define an associated boundary condition 
\[\Gamma(A):=\left\{\iota_+(x)+\iota_-(Ax)\mid x\in\mathbb{C}^G\right\}\]
which we call the \textit{graph of }\(A\).  We say that \(A\) is a \(G\)-endomorphism, if \(\Gamma(A)\) is a local boundary condition. In this case for each \(v\in V\), \(A\) restricts to a homomorphism \(A_v :\mathbb{C}_+(G,v)\rightarrow \mathbb{C}_-(G,v)\).

\begin{thm}\label{thm:unit}
	Let \(B\) be a local boundary condition on \(G\). Then, \(D_B\) is self-adjoint iff there exists a unitary  \(G\)-endomorphism \(A\), such that \(B=\Gamma(A)\).
\end{thm}
\begin{proof}
	Let $B$ be such that $D_B$ is self-adjoint. Since the index of a self-adjoint operator is zero, by Theorem \ref{thm:ind}, the dimension of $B$ is equal to the amount of edges, denoted by $n$. Choosing a basis $b_1,...,b_n$ of $B$, let 
	$$b_i^\pm=\pi_\pm(b_i)\in \mathbb{C}^G$$
	for $i=1,...,n$, where \(\pi_\pm: \mathbb{C}[G]\rightarrow \mathbb{C}^G\) is the map sending \(e_\pm\) to \(e\) and \(e_\mp\) to \(0\). Now, assume that there exist $\lambda_1,...,\lambda_n$, such that
	$$\sum_{i=1}^n \lambda_i b^+_i =0.$$
	Letting
	$$b=\sum_{i=1}^n \lambda_i b_i,$$
	we arrive at $\pi_+(b)=0$. Since $B$ is equal to its adjoint boundary condition, by propositon \ref{prop:sa} we have
	\begin{align*}
		0= \langle \sigma_0(b),b\rangle=i \langle \pi_{-}(b),\pi_{-}(b)\rangle -i\langle \pi_{+}(b),\pi_{+}(b)\rangle = i \langle \pi_{-}(b),\pi_{-}(b)\rangle.
	\end{align*}
	Therefore, $\pi_-(b)$ is also zero, which implies that $b$ is equal to zero. Since $b_1,..,b_n$ is a basis of $B$,
	$$\lambda_1=...=\lambda_n=0$$
	follows. Consequently, the vectors $b^+_1,...,b^+_n$ are linearly independent.
	Considering that the dimension of $\mathbb{C}^G$ is equal to $n$, it follows that $b^+_1,...,b^+_n$ forms a basis of $\mathbb{C}^G$. W.l.o.g., $b_1,...,b_n$ can be chosen such that $b^+_1,...,b^+_n$ is an orthonormal basis of $\mathbb{C}^G$. Define $A$ as the linear map sending $b^+_i$ to $b^-_i$ for $i\in\{1,...,n\}$. Since $B=B^{\ad}$, we have 
	$$\langle \sigma_0(b_i),b_j\rangle=0$$
	for $i,j\in\{1,...,n\}$, hence
	$$ i \langle b^-_i,b^-_j\rangle -i\langle b^+_i,b^+_j\rangle=0.$$
	Rearranging this equation and using the definition of $A$, this yields 
	\begin{align*}
		\langle b^+_i,b^+_j\rangle=\langle Ab^+_i,Ab^+_j\rangle,
	\end{align*}
	which proves that $A$ is unitary.
	It is easily checked that $A$ is a $G$-endomorphism. Therefore,
	each $b_i$ is contained in $\Gamma(A)$ for $i\in\{1,...,n\}$ and, consequently, the same can be said for $B$. Equality, then, holds by equality of dimensions.
	
	Conversely, it is easily checked that for a given unitary \(G\)-endomorphism $A$, the graph $\Gamma(A)$ is a local boundary condition for $G$ such that $\Gamma(A)^{\ad}=\Gamma(A)$.
\end{proof}

Unfortunately, the existence of self-adjoint boundary conditions poses quite some restrictions on the graph itself:

\begin{prop}\label{prop: s.a. implies eulerian}
	If there exists a boundary condition $B$ on $G$ such that $D_B$ is self-adjoint, then every connected component of $G$ is a Eulerian digraph.
\end{prop}

\begin{proof}
	By Theorem \ref{thm:unit}, there exists a unitary \(G\)-endomorphism $A$ such that $B$ is the graph of $A$. Restricting $A$ to each vertex space, we then obtain isometries $A_v$ from
	$\mathbb{C}_{+}(G,v)$ to $\mathbb{C}_{-}(G,v)$,  and hence the dimensions of the two spaces must be equal, implying that each vertex has as many ingoing as outgoing edges.
\end{proof}

\subsection{The spectrum of the Dirac operator}

We proceed with the study of the spectrum \(\sigma(D_B)\) of $D_B$ and start with some general properties:

\begin{prop}\label{prop:ptspctrm}
	Let $B$ be a boundary condition on $G$. If the dimension of $B$ is not equal to the number of edges \(n\) of $G$, then the spectrum of $D_B$ is equal to $\mathbb{C}$. Otherwise, the spectrum of $D_B$ is equal to the point spectrum, i.e. $\sigma(D_B)=\sigma_p(D_B)$.
\end{prop}

\begin{proof}
	Assume that $\lambda$ is an element of the resolvent set of \(D_B\). Then the operator $D-\lambda I$ is bijective, and hence the index of $(D-\lambda I)_B= D_B -\lambda I$ must be zero. Theorem \ref{thm:ind} implies that the dimension of $B$ is equal to $n$. Consequently, if the dimension of $B$ is not equal to $n$, the resolvent set of $D_B$ is empty.
	
	Now, let the dimension of $B$ be equal to $n$. We have to show that injectivity of $D_B-\lambda I$ implies surjectivity. Assuming that $D_B-\lambda I$ is injective,
	by Theorem \ref{thm:ind}, the codimension of $ \ran D_B -\lambda I $ is zero. Furthermore, $D_B-\lambda I$ has closed range since it is Fredholm. Together, this implies that $D_B-\lambda I$ is surjective.
\end{proof}

We now turn to boundary conditions given by $G$-endomorphisms. The spectrum of the Dirac operator under such boundary conditions is closely related to the eigenvalues of the defining $G$-endomorphism. However, the different lengths of the edges distort the powers of the variables in the characteristic polynomial, such that it is necessary to introduce a multi-dimensional version:

\begin{defin}
	For a \(G\)-endomorphism $A$ define the multi-dimensional characteristic polynomial $P_A$ of $A$ to be the polynomial in the variables $\{x_e\mid e\in E(G)\}$ given by
	$$P_A(x)=\det\left(\diag(x)-A\right),$$
	where \(diag(x)\) is the linear map on \(\mathbb{C}^G\) given by \(\sum_{e\in E} x_e e\otimes e^\ast\).
	The characteristic function $P_{A,l}:\mathbb{C}\rightarrow\mathbb{C}$ with respect to the length function $l$ is given by $$P_{A,l}(\lambda)=P_A(\exp(i\lambda l)),$$
	where \(\exp(i\lambda l) =\sum_{e\in E} \exp(i\lambda l_e)e\). 
\end{defin}

The following Lemma allows us to delete vertices that have exactly one in-coming and one outgoing edge when dealing with the multi-dimensional characteristic polynomial. This will facilitate future arguments and calculations regarding the spectrum of the Dirac operator.

\begin{lem}\label{lem:reduction}
	Let $A$ be a $G$-endomorphism and let $v\in V(G)$ be a vertex of $G$ with both in- and out-degrees equal to one, such that the ingoing edge is different from the outgoing edge. Enumerate the edges of $G$ by $e_1,e_2,...,e_n$ such that $e_1$ is the ingoing and $e_2$ is the outgoing edge at $v$, defining a basis of $\mathbb{C}^G$. Let $\tilde{G}$ be the graph obtained out of $G$ by replacing $e_1$ and $e_2$ with a single edge $e$ running from the tail of $e_1$ to the head of $e_2$. The transformation matrix of $A$ with respect to the basis is then given by
	\[
	A=\scalebox{0.9}{$
		\left(\begin{array}{@{}c c@{}}
			\begin{matrix}
				0 & \alpha_{12} \\
				\alpha & 0
			\end{matrix}
			&
			\begin{matrix}
				\ldots  &\alpha_{1n}\\ 
				\ldots & 0
			\end{matrix}
			\\
			\begin{matrix}
				0 & \alpha_{32}\\
				\vdots & \vdots\\
				0 & \alpha_{n2}
			\end{matrix}
			&
			\hat{A}
		\end{array}\right)$}.
	\]
	Revising $A$, we arrive at the following matrix $\tilde{A}$, which defines a $\tilde{G}$-automorphism with respect to the basis $e,e_3,...,e_n$:
	\[
	\tilde{A}=
	\left(\begin{array}{@{}c c@{}}
		\alpha\alpha_{12}
		&
		\begin{matrix}
			\ldots  &\alpha\alpha_{1n}
		\end{matrix}
		\\
		\begin{matrix}
			\alpha_{32}\\
			\vdots \\
			\alpha_{n2}
		\end{matrix}
		&
		\hat{A}
	\end{array}\right)
	\]
	For the multi-dimensional characteristic polynomials, then
	$$P_A(x_1,x_2,...,x_n)=P_{\tilde{A}}(x_1 x_2,...,x_n)$$
	holds.
	In particular, if the length of $e$ is the sum of the lengths of $e_1$ and $e_2$, then $P_{A,l}$ is equal to $P_{\tilde{A},\tilde{l}}$, where $\tilde{l}$ is the induced length function on $\tilde{G}$.
\end{lem}


\begin{proof}
	The lemma follows from a quick calculation using the Laplace expansion along the first column and the multilinearity of determinants:
	
	\begin{align*}
		P_A(x_1,x_2,...,x_n)&=\det \scalebox{0.9}{$
			\left(\begin{array}{@{}c c@{}}
				\begin{matrix}
					x_1 & -\alpha_{12} \\
					-\alpha & x_2
				\end{matrix}
				&
				\begin{matrix}
					\ldots  &-\alpha_{1n}\\ 
					\ldots & 0
				\end{matrix}
				\\
				\begin{matrix}
					0 & -\alpha_{32}\\
					\vdots & \vdots\\
					0 & -\alpha_{n2}
				\end{matrix}
				&
				\diag(\hat{x})-\hat{A}
			\end{array}\right)$}\\
		&=x_1
		\left(\begin{array}{@{}c c@{}}
			x_2
			&
			\begin{matrix}
				0 & \ldots  & 0
			\end{matrix}
			\\
			\begin{matrix}
				-\alpha_{32}\\
				\vdots\\
				-\alpha_{n2}
			\end{matrix}
			&
			\diag(\hat{x})-\hat{A}
		\end{array}\right)
		+\alpha
		\left(\begin{array}{@{}c c@{}}
			-\alpha_{12}
			&
			\begin{matrix}
				\ldots  & -\alpha_{1n}
			\end{matrix}
			\\
			\begin{matrix}
				-\alpha_{32}\\
				\vdots\\
				-\alpha_{n2}
			\end{matrix}
			&
			\diag(\hat{x})-\hat{A}
		\end{array}\right)\\
		&= 
		\left(\begin{array}{@{}c c@{}}
			x_1 x_2 -\alpha\alpha_{12}
			&
			\begin{matrix}
				\ldots  & -\alpha\alpha_{1n}
			\end{matrix}
			\\
			\begin{matrix}
				-\alpha_{32}\\
				\vdots\\
				-\alpha_{n2}
			\end{matrix}
			&
			\diag(\hat{x})-\hat{A}
		\end{array}\right)=P_{\tilde{A}}(x_1 x_2, ...,x_n).
	\end{align*}
	It is easily checked that $\tilde{A}$ indeed defines a $\tilde{G}$-automorphism. The equality of the characteristic functions follows from 
	$$P_{A,l}(\lambda)=P_A(e^{i\lambda l_1},e^{i\lambda l_2},...,e^{i\lambda l_n})=P_{\tilde{A}}(e^{i\lambda (l_1+l_2)},...,e^{i\lambda l_n})=P_{\tilde{A},\tilde{l}}(\lambda).$$
\end{proof}


The next proposition relates the eigenvalues of the Dirac operator to the zeroes of the characteristic function of $A$. However, if the lengths of the edges are incommensurable, i.e. they have no common rational divisor, the characteristic function can no longer be related to a polynomial, which makes the solution of this problem very difficult.

\begin{prop}\label{prop:spec}
	Let \(A\) be a $G$-endomorphism. Then the spectrum of $D_{\Gamma(A)}$ is equal to the set of zeroes of $P_{A,l}$, i.e.
	\begin{equation*}
		\sigma(D_{\Gamma(A)})=P_{A,l}^{-1}(\{0\}).
	\end{equation*}
	The multiplicity of an eigenvalue $\lambda$ is given by), which will be also referred to as the canonical Dirac operator
	\begin{equation*}
		m_{\Gamma(A)}(\lambda)=\dim\ker(\diag(\exp(i\lambda l))-A).
	\end{equation*}
\end{prop}


\begin{rmk}
	Note that $P_{A,l}(\lambda)$ does not capture the kernel of $A$. For instance, if $A=0$, then the spectrum of $D_{\Gamma(0)}$ is empty.
\end{rmk}


\begin{proof}
	Let \(\lambda\in \sigma(D_{\Gamma(A)})\). Since the dimension of \(\Gamma(A)\) is equal to the number of edges of \(G\), by Proposition \ref{prop:ptspctrm}, there exists an eigenvector \(\phi\in C^\infty(G)\) to the eigenvalue \(\lambda\), i.e. \(\phi\) fulfills 
	\[D\phi =\lambda \phi, \qquad \tr \phi \in \Gamma(A).\] 
	The first condition implies that on each edge \(\phi\) is of the form \(w_e \exp(-i\lambda x_e)\) for some \(w_e\in \mathbb{C}\) and together with the second condition we obtain the equation
	\[A\sum w_e \exp(-i\lambda l_e)e= \sum w_e e,\]
	which is equivalent to \(P_{A,l}(\lambda) = 0\). It is easy to see that any element of \(\ker(\diag(\exp(i\lambda l))-A)\) defines a eigenfunction to the eigenvalue \(\lambda\). This then implies the equivalence of the first statement and proves the assertion on the multiplicity.
\end{proof}

If all edges of the digraph have the same length, the spectrum of $D_{\Gamma(A)}$ is particularly easy to compute:

\begin{cor}\label{cor:char}
	Let all edges have the same length \(l\) and \(A\) be a \(G\)-endomorphism. Let $\mu_1,...,\mu_m$ be the non-zero eigenvalues of $A$. Then, there
	exist unique $\alpha_1,...,\alpha_m\in\mathbb{R}$ and $\varphi_1,...,\varphi_m\in[0,2\pi)$ such that
	$$\mu_j=\exp(\alpha_j+i\varphi_j).$$
	The spectrum of $D_{\Gamma(A)}$ is then given by 
	\begin{equation*}
		\sigma(D_{\Gamma(A)})=\left\{(\varphi_j-i\alpha_j+2\pi k)/l \mid j\in\{1,...,m\}\text{, } k\in\mathbb{Z} \right\}.
	\end{equation*}
\end{cor}

\begin{proof}
	By Proposition \ref{prop:spec}, $\lambda$ is an eigenvalue of $D_{\Gamma(A)}$ if and only if $e^{i\lambda l}\in\mathbb{C}$ is an eigenvalue of $A$. This proves the corollary.
\end{proof}

A matrix $A$ is reducible if there exists a permutation matrix $P$ such that $P^{-1}AP$ is of the form
$$\left(
\begin{array}{c c} 
	X & 0\\ 
	Y & Z\\ 
\end{array}\right),$$
where $X$ and $Z$ are square matrices. Otherwise, $A$ is said to be irreducible \cite[p.18]{Cvet}.
We transfer this definition to linear maps on $G$:

\begin{defin}
	A \(G\)-endomorphism $A$ is said to be reducible with respect to a proper subset $\Delta$ of $E(G)$ if \(A\) maps $\mathbb{C}^{\Delta}:=\{\sum_{e\in \Delta}c_e \cdot e \mid c_e\in \mathbb{C}\}$ into itself. If no such $\Delta$ exists, $A$ is said to be irreducible.
\end{defin}

This is always the case when $G$ is disconnected. If $G^\prime$ is a connected component of $G$, then all \(G\)-endomorphisms $A$ are reducible with respect to $E(G^\prime)$.

Given $\Delta\subseteq E(G)$, we denote its complement by $\Delta^c$ and by \(G[\Delta]\) the subgraph of \(G\) induced by \(\Delta\).

\begin{prop}\label{prop:decomp}
	Let \(A\) be a \(G\)-endomorphism. If $A$ is reducible with respect to $\Delta\subsetneq E(G)$, then the spectrum of $D_{\Gamma(A)}$ is given by
	\begin{equation*}
		\sigma(D_{\Gamma(A)})=\sigma(D_{\Gamma(A_1)}^1)\cup\sigma(D_{\Gamma(A_2)}^2),
	\end{equation*}
	where $A_1$ is the restriction of $A$ to $ \mathbb{C}^{\Delta} $ and $A_2$ is the restriction of $A$ to $ \mathbb{C}^{\Delta^c} $. The operators $D^1$ and $D^2$ are the Dirac operators on $G[\Delta]$ and $G[\Delta^c]$, respectively. 
\end{prop}

\begin{proof}
	Enumerate the edges $e_1,...,e_n$ of $G$ such that $\{e_1,...,e_k\}=\Delta$. Then, the transformation matrix of $A$ with respect to this basis is of the form 
	$$\left(
	\begin{array}{c c} 
		A_1 & C\\ 
		0 & A_2\\ 
	\end{array}\right),$$
	where $A_1$ and $A_2$ are square matrices. The matrix $A_1$ represents the restriction of $A$ to $ \mathbb{C}^{\Delta} $ and $A_2$ represents $A$ restricted to $ \mathbb{C}^{\Delta^c} $. These matrices give boundary conditions on $G[\Delta]$ and $G[\Delta^c]$, respectively. Denote the induced length function on $G[\Delta]$ and $G[\Delta^c]$ by $l_1$ and $l_2$, respectively. We then have 
	\begin{align*}
		P_{A,l}(\lambda)&=
		\det\left(
		\begin{array}{c c} 
			\diag(\exp(i\lambda l_1))-A_1 & -C\\ 
			0 & \diag(\exp(i\lambda l_2))-A_2\\ 
		\end{array}\right)\\
		&=P_{A_1,l_1}(\lambda)\cdot P_{A_2,l_2}(\lambda).
	\end{align*}
	Therefore, $P_{A,l}(\lambda)$ is zero if either $P_{A_1,l_1}(\lambda)$ or 
	$P_{A_2,l_2}(\lambda)$ is zero and, hence, 
	$$ \sigma(D_A)=\sigma(D_{\Gamma(A_1)}^1)\cup\sigma(D_{\Gamma (A_2)}^2). $$
\end{proof}


\begin{exa}\label{exa:cycle}
	Let $C_n$ be the directed cycle with $n$ edges and let $A$ be a unitary $C_n$-endomorphism. Enumerating the edges $e_1,...,e_n$ in cyclic order, it is easily seen that there exists $\varphi_1,...,\varphi_n\in[0,2\pi)$ such that the transformation matrix with respect to this basis is of the form
	\[
	A=\left(
	\begin{array}{ l l l l } 
		0&  &  &e^{i\varphi_n}\\ 
		e^{i\varphi_1}&0 &  & \\ 
		& &\ddots &\\
		&  & e^{i\varphi_{n-1}} &0
	\end{array}\right).
	\]
	By removing all vertices but one, according to Lemma \ref{lem:reduction}, the multi-dimensional characteristic polynomial is of the form 
	\begin{align*}
		P_A(x_1,...,x_n)&=x_1\cdot...\cdot x_n-\exp(i\varphi_1)\cdot...\cdot \exp(i\varphi_n)\\
		&=\prod_{i=1}^n x_i - \exp\left(i\sum_{i=1}^n\varphi_i\right).
	\end{align*}
	Hence, the characteristic function is given by 
	\begin{align*}
		P_{A,l}(\lambda)
		= \exp\left(i\lambda\sum_{j=1}^n l_j\right)-\exp\left(i\sum_{j=1}^n\varphi_j\right).
	\end{align*}
	This implies that the spectrum of the canonical Dirac operator with respect to the boundary condition $\Gamma(A)$ is given by 
	\begin{equation}\label{eq:cycleev}
		\lambda_k=\frac{1}{L}\left(\sum_{j=1}^n\varphi_j +2\pi k\right) \text{ for }k\in\mathbb{Z},
	\end{equation}
	where $L=\sum_{j=1}^n l_j$. By deleting the last column of $ \diag(e^{i\lambda_k l})-A $, we obtain a matrix of the form
	\begin{equation*}
		\tilde{A}_k=\left(
		\begin{array}{c c c c} 
			e^{i\lambda_k l_1} &  & &\\ 
			-e^{i\varphi_1}& e^{i\lambda_k l_2} &  &\\ 
			&-e^{i\varphi_2}  &\ddots& \\ 
			& & & e^{i\lambda_k l_{n-1}}\\
			& &  & -e^{i\varphi_{n-1}} 
		\end{array}\right).
	\end{equation*}
	Clearly, $ \tilde{A}_k $ has rank $n-1$. Thus, the kernel of $ \diag(e^{i\lambda_k l})-A $ can only be one-dimensional, which implies that the multiplicity of $\lambda_k$ is one.
\end{exa}

\section{Boundary conditions}

\subsection{Decompositions of Eulerian Digraphs}\label{section:decomp}

Recall that a directed walk is an alternating sequence of edges 
$$W:=e_1,e_2,\cdots, e_k$$
where $\partial_+ e_{i-1}= \partial_- e_i$. Denote by $E(W)$ the set of edges appearing in $W$. A directed walk is said to be closed if $ \partial_+ e_k= \partial_- e_0$. A trail is a walk in which all edges are distinct. A closed directed trail that does not pass a vertex twice, except for $\partial_- e_0$, is called a cycle. A cycle/trail decomposition of $G$ is a family 
$$S=\{C_1,...,C_m\}$$
of cycles/trails such that every edge of $G$ appears in a unique cycle/trail of $S$.

\begin{defin}
	Let $G$ be a digraph. We call a bijective map $P:E(G)\rightarrow E(G)$ a permutation of the edges of $G$, the set of permutations of its edges are denoted by $\mathfrak{S}(E(G))$. If $P$ satisfies that $P(e)=e^\prime$ implies $\partial_+e= \partial_- e^\prime$, then $P$ is called a $G$-permutation. We denote the set of all \(G\)-permutations by $\mathfrak{S}(G)$. 
\end{defin}

\begin{rmk}
	We have $\mathfrak{S}(G)\subseteq \mathfrak{S}(E(G))$. By enumerating $E(G)$, we assign to every permutation $P\in \mathfrak{S}(E(G))$ a corresponding element in the symmetric group $\mathfrak{S}_n$, where $n$ is the amount of edges of $G$.  
\end{rmk}

\begin{lem}
	A permutation $P\in \mathfrak{S}(G)$ of $G$ induces for each vertex $v$ a bijection $P_v$ from the ingoing edges of $v$ to the outgoing edges of $v$. In particular, $\mathfrak{S}(G)$ is empty if $G$ is not a Eulerian digraph.
\end{lem}

\begin{proof}
	Fix $v\in V(G)$. Let $e$ be an ingoing vertex at $v$. By the definition of $G$-permutations, $P(e)$ is then an outgoing vertex of $v$. Therefore, $P_v$ is well-defined. Injectivity of $P$ implies injectivity of $P_v$ and, since $P$ is surjective, for every outgoing edge $e^\prime$ at $v$, there exists an edge $e$ of $G$ mapped by $P$ onto $e^\prime$. This implies that the head of $e$ is the tail of $e^\prime$. Therefore, $e$ is an ingoing edge of $v$ and $P_v$ maps $e$ onto $e^\prime$, which proves that $P_v$ is bijective. 
\end{proof}

It is clear that every \(G\)-permutation \(P\in\mathfrak{S}(G)\) induces a unitary \(G\)-endomorphism, and hence, a local self adjoint boundary condition. We will now investigate the spectrum of the Dirac operator subject to such a boundary condition. 

To this end, recall the following facts concerning permutations: for every permutation $\pi\in \mathfrak{S}_n$ and $x\in\{1,...,n\}$ there exists a $k\in \mathbb{N}$ such that $\pi^k(x)=x$. If $k$ is the smallest such integer, then $x,\pi(x),\pi^2(x),...,\pi^{k-1}(x)$ is said to be a $k$-cycle in $\pi$. Every permutation can be decomposed into a disjoint union of its cycles.

\begin{thm}\label{thm:perm}
	Let $G$ be a Eulerian digraph. Every decomposition of the digraph into directed closed trails corresponds to a $G$-permutation $P$ and vice versa. Each trail of length $k$ then corresponds to a $k$-cycle of $P$, i.e. the trail containing the edge $e$ is given by
	\begin{equation*}
		e,P(e),P^2(e),...,P^{k-1}(e).
	\end{equation*}
\end{thm}

\begin{proof}
	Given a decomposition of the digraph into closed directed trails, define a map $P:E(G)\rightarrow E(G)$ by $P(e)=e^\prime$, where $e^\prime$ is the edge that comes after $e$ in the closed trail containing $e$. In order to show that $P$ is a permutation, it remains to be shown that it is injective. Hence, assume that 
	\begin{equation*}
		P(e)=e^\prime=P(\tilde{e})
	\end{equation*}
	and, therefore, there is a trail $C$ that contains $e$ and a trail $\tilde{C}$ that contains $\tilde{e}$ such that $e^\prime$ is the subsequent edge in both trails. Hence, $C$ and $\tilde{C}$ both contain $e^\prime$ and, since the decomposition demands the trails be disjoint, we get $C=\tilde{C}$. A trail cannot contain an edge twice and, therefore, $e$ is equal to $\tilde{e}$. Consequently, injectivity of $P$ follows.
	
	Now, let \(P\) be a $G$-permutation. For an edge $e$, define a directed walk
	\begin{equation}\label{eq:trail}\tag{\(\heartsuit\)}
		T(e)=e,P(e),P^2(e),...,P^{k-1}(e),
	\end{equation}
	where $k$ is the minimum number such that $P^{k}(e)=e$. The fact that $P$ is a permutation of the set of edges of $G$ ensures that such a $k$ exists. It is, indeed, a directed walk since, by definition of $P$, the head of $P^i(e)$ is the tail of $P^{i+1}(e)$. It is closed, since the head of $P^{k-1}(e)$ is the tail of $P^{k}(e)=e$. It remains to be shown that it is a trail, i.e. that there exists no $l< j$ in $\{0,...,k-1\}$ such that $P^l(e)=P^j(e)$. This would imply $P^{j-l}(e)=e$, which would contradict the definition of $k$, since $j-l<k$.
	
	Hence, equation \eqref{eq:trail} defines for every edge a directed closed trail that contains this edge. To show that these trails are a decomposition of the digraph, it remains to be proven that two distinct trails do not share any edges. So, let 
	\begin{align*}
		T_1&=e,P(e),P^2(e),...,P^{k_1}(e)\\
		T_2&=e^\prime,P(e^\prime),P^2(e^\prime),...,P^{k_2}(e^\prime)
	\end{align*}
	be two trails that share an edge, i.e. $P^{m_1}(e)=P^{m_2}(e^\prime)$ for some $1\leq m_i\leq k_i$. W.l.o.g., let $m_1\geq m_2$. We then arrive at $P^{m_1-m_2}(e)=e^\prime$. Applying $P^{k_1}$, which commutes with $P^{m_1-m_2}$, to both sides of this equation, we see that
	$$P^{m_1-m_2}(e)=P^{k_1}(e^\prime).$$
	Therefore, $P^{k_1}(e^\prime)$ is equal to $e^\prime$, which implies that $k_1\geq k_2$. The inequality $k_2\geq k_1$ follows analogously, which shows that $k_1=k_2$. 
	Therefore, $P^{m_1-m_2}$ gives a bijection between the edges of $T_1$ and $T_2$. Therefore, they are the same trail.
	
	It is clear that these constructions are inverse to each other and, hence, this correspondence is one to one.
\end{proof}

\begin{prop}
	Let $P$ be a $G$-permutation. Then, the trace of $P$ is equal to the number of loops in the decomposition of $G$ that corresponds to $P$.
\end{prop}
\begin{proof}
	The trace of $P$ is equal to
	\begin{equation*}
		\tr P = \sum_{e\in E(G)} \langle e, P(e)\rangle
	\end{equation*}
	and further 
	\[\langle e, P(e)\rangle = \begin{cases}
		1\quad \text{if } P(e)=e \\ 0 \quad \text{else}
	\end{cases}\]
	Hence, the trace of \(P\) is equal to the number of edges such that \(P(e)=e\). For this to hold, \(e\) must be a loop, and furthermore by the proof of Theorem \ref{thm:perm}, it is the closed trail containing \(e\).
\end{proof}

Given a closed trail $T=e_1,...,e_k$, let its length by
$$L_T=\sum_{i=1}^k l_{e_i}.$$

\begin{lem}\label{lem:cycle}
	Let $P$ be a $G$-permutation. Then, for each closed trail $T$ in the decomposition defined by $P$, $P$ is reducible with respect to $E(T)$.
\end{lem}
\begin{proof}
	Let $T$ be a closed trail in the decomposition defined by $P$ and let $e$ be an edge that appears in $T$. Then, by definition of $T$, $P(e)$ is again an edge that appears in $T$, which shows that $P$ is reducible with respect to $E(T)$.
\end{proof}

We say that $\alpha\in \mathbb{R}$ is divisible by $\beta\in \mathbb{R}$ if there exists a $k\in\mathbb{Z}$ such that $ \alpha=k\cdot\beta $. 

\begin{thm}\label{thm:perm2}
	Let $G$ be a Eulerian digraph and let $P$ be a linear $G$-permutation. If $T_1,...,T_m$ are the closed trails of the decomposition defined by $P$, then the spectrum of $D_{\Gamma(P)}$ is given by
	\begin{equation*}
		\lambda_{k,i} = \frac{2\pi k}{L_i}\text{ with } i\in\{1,...,m\}\text{, }k\in \mathbb{Z},
	\end{equation*}
	where $L_j$ is the length of $T_j$. The multiplicity $m(\lambda_{k,i})$ is, for $k\neq 0$, equal to the amount of cycles whose length is divisible by $L_i/k$. Furthermore, $m(0)$ is equal to the total amount of closed trails in the decomposition defined by $P$.
\end{thm}
\begin{rmk}
	If we choose all edges to have unit length, then $L_j$ is the actual amount of edges appearing in the closed trail $T_j$.
\end{rmk}

\begin{proof}
	By Lemma \ref{lem:cycle}, $P$ is reducible with respect to each $T_i$ and, hence, by Proposition \ref{prop:decomp}, we have that
	\begin{equation*}
		\sigma(D_{\Gamma(P)})=\bigcup_{i=1}^n\sigma(D_{\Gamma(P_i)}),
	\end{equation*}
	where $P_i$ is the restriction of $P$ to $E(T_i)$. Hence we can assume that we have only one trail \(T\). 
	
	Enumerate the edges of \(T\) in such a way that \(P(e_i)= e_{i +1 \mod L_T }\).
	The transformation matrix of $P$ with respect to the basis $e_{0},...,e_{L_T-1}$ is equal to
	\begin{equation*}
		\left(
		\begin{array}{c c c c c} 
			0 &  & & &1\\ 
			1& 0 & & &\\ 
			& 1  & 0 & &\\ 
			& &\ddots &\ddots &\\
			& & & 1 & 0
		\end{array}\right).
	\end{equation*}
	As in Example \ref{exa:cycle}, the spectrum of $P$ can be easily calculated to be
	\begin{equation*}
		\sigma(D_{\Gamma(P)})=\left\{\frac{2\pi k}{L}\mid k\in \mathbb{Z}\right\},
	\end{equation*}
	with each eigenvalue having multiplicity one. 
\end{proof}

\begin{cor}\label{cor:perm}
	Let $P$ be a $G$-permutation. Let $\lambda_0$ be the smallest positive eigenvalue of $D_{\Gamma(P)}$. Then, 
	\begin{equation*}
		L=\frac{2\pi}{\lambda_0}
	\end{equation*}
	is the length of the longest cycle in the decomposition associated with $P$. The multiplicity of $\lambda$ is equal to the amount of cycles of length $L$.
\end{cor}
\begin{proof}
	Let $\lambda_0$ be the smallest positive eigenvalue of $D_{\Gamma(P)}$. By Theorem \ref{thm:perm2}, it is of the form
	\begin{equation*}
		\lambda_0= \frac{2\pi k}{L},
	\end{equation*}
	where $k$ is an integer and $L$ is the length of a closed trail appearing in the decomposition defined by $P$. In order for $\lambda_0$ to be the smallest positive eigenvalue, $k$ has to be one and $L$ must be the maximum of the lengths of the closed trails.
\end{proof}

\subsection{The directed adjacency matrix}

As we have seen, the spectrum of the Dirac operator is closely tied to the boundary condition it is subjected to. Boundary conditions involve choices which are a priori not canonical. In order to obtain a theory that ties the spectrum of the Dirac operator to properties of the underlying graph, we need to come up with a canonical boundary condition. By Proposition \ref{prop: s.a. implies eulerian}, this boundary condition will not be self adjoint. However, canonical matrices on graphs are given by directed adjacency matrices:

\begin{defin}
	The directed adjacency matrix $\mathcal{A}(G)$ of a digraph $G$ is given by
	\begin{align*}
		\mathcal{A}(G) &= \sum_{e, e^\prime \in E} I_{e, e^\prime} e\otimes (e^\prime)^\ast \in \mathbb{C}^G\otimes(\mathbb{C}^G)^\ast = \mathrm{End}(\mathbb{C}^G), \\
		I_{e, e^\prime} &=\left\{ \substack{1 \quad \text{if }\partial_+ e^\prime = \partial_- e\\ 0 \quad \text{otherwise}}\right.
	\end{align*}
\end{defin}

\begin{lem}\label{lem:inciauto}
	The directed adjacency matrix $\mathcal{A}(G)$ is a $G$-endomorphism.
\end{lem}
\begin{proof}
	Let $v\in V$ be a vertex of $G$ and $e^\prime$ be an edge in $\mathbb{C}_+(G,v)$. Therefore, $\partial_+ e^\prime = v$, and
	$$\mathcal{A}(G)e^\prime=\sum_{e\in E(G)}I_{e,e^\prime}e=\sum_{\substack{e \\ \partial_- e= v}}e \in \mathbb{C}_-(G,v),$$
	which proves the lemma, since $\mathbb{C}_+(G,v)$ is spanned by \(\{ e^\prime \mid \partial_+ e^\prime = v\}\).
\end{proof}

Lemma \ref{lem:inciauto} shows that the adjacency matrix defines a boundary condition for the Dirac operator. We denote the Dirac operator subject to the boundary condition $\Gamma(\mathcal{A}(G))$ by $D_G$. As we will see, the spectrum of $D_G$ highly depends on the directed cycles present in $G$. We use the notation $G^\prime\subseteq G$ to say that $G^\prime$ is a subgraph of $G$ with no isolated vertices.

\begin{defin}
	A disjoint collection of cycles is a digraph $C$, whose connected components consist of directed cycles. The size $\alpha(C)$ of $C$ is the amount of connected components, the total length $\eta(C)$ of $C$ is the amount of edges of $C$. For a digraph $G$, define
	$$\mathcal{C}(G):=\{C\subseteq G\mid C \text{ is a disjoint collection of cycles} \}.$$
\end{defin}


\begin{rmk}
	The empty subgraph $\varnothing$ with no vertices and edges is always an element of $\mathcal{C}(G)$. Its size and length equal zero.
\end{rmk}


The following lemma gives a characterization of disjoint collections of cycles:

\begin{lem}\label{lem:idod}
	If $G$ is a digraph such that for every vertex $v\in V(G)$ the in-degree $id(v)$ and the out-degree $od(v)$ are both equal to one, then $G$ is a disjoint collection of cycles.
\end{lem}


\begin{proof}
	It is easily checked that, in this case, $\mathcal{A}(G)$ is a $G$-permutation. By Theorem \ref{thm:perm}, the directed adjacency matrix defines a decomposition $S$ of $G$ into directed trails. However, if a trail in $S$ passes a vertex twice or two distinct trails intersect at a vertex, then the in- and out-degrees of the vertex would be greater than one, which would contradict the assumption. Consequently, $G$ is a disjoint collection of cycles.
\end{proof}

\begin{prop}\label{prop:adjacency}
	The directed adjacency matrix $\mathcal{A}(G)$ of $G$ is non-singular if and only if $G$ is a disjoint collection of directed cycles. In this case, the determinant of $\mathcal{A}(G)$ is given by the formula
	$$\det \mathcal{A}(G) = (-1)^{\alpha(G)-\eta(G)}.$$
\end{prop}

\begin{proof}
	Let $v$ be a vertex of $G$. If the in-degree/out-degree of $v$ is equal to zero, then the out-degree/in-degree must be unequal zero as $G$ has no isolated vertices. Then, for every edge $e$ starting/ending in $v$, the row/column of $\mathcal{A}(G)$ that corresponds to $e$ is equal to zero, which implies that $\mathcal{A}(G)$ is singular. Therefore, in order for $\mathcal{A}(G)$ to be non-singular, the in-degree and the out-degree of every vertex of $G$ must be greater or equal to one.
	
	Now, assume that the in-degree/out-degree of a vertex $v$ is greater than or equal to two and let $e_1$ and $e_2$ be two edges ending/starting in $v$. Then, the columns/rows of $\mathcal{A}(G)$ that correspond to $e_1$ and $e_2$ are equal, which implies that $\mathcal{A}(G)$ is singular. Consequently, $\mathcal{A}(G)$ can only be non-singular if the in-degree and the out-degree of all vertices are equal to one.
	
	Lemma \ref{lem:idod} implies that $G$ is a disjoint collection of cycles and $\mathcal{A}(G)$ is a $G$-permutation, where each cycle of $G$ corresponds to a cycle of $\mathcal{A}(G)$ of the same length. Writing $G$ as the disjoint union of cycles
	$$G=C_{k_1}\sqcup...\sqcup C_{k_s},$$
	where $\alpha(G)=s$ and $\eta(G)=k_1+...+k_s$, it follows that $\mathcal{A}(G)$ is the product of cycles of length $k_1,...,k_s$. Since the signature of a cycle of length $k$ is given by $(-1)^{k-1}$, the signature $\sgn(\mathcal{A}(G))$ of $\mathcal{A}(G)$ is given by $-1$ to the power of
	$$\sum_{i=1}^s k_i -1=\eta(G)-\alpha(G).$$
	The determinant of a permutation matrix is equal to the signature of the underlying permutation, which implies that
	$$\det \mathcal{A}(G) = \sgn(\mathcal{A}(G)) = (-1)^{\eta(G)-\alpha(G)}=(-1)^{\alpha(G)-\eta(G)},$$
	and proves the proposition.
\end{proof}

For a subgraph $H$ of $G$, define a monomial in the polynomial ring over the variables $\{x_e\mid e\in E(G)\}$ by 
$$x^H:=\prod_{e\not\in E(H)}x_e.$$
It seems counter-intuitive to multiply over the edges that are not contained in $E(H)$, however, it is the most convenient notation for the next theorem:

\begin{thm}
	The multi-dimensional characteristic polynomial $P_{\mathcal{A}(G)}$ of the directed adjacency matrix $\mathcal{A}(G)$ is given by
	\begin{equation}\label{eq:charpol}
		P_{\mathcal{A}(G)}(x)=\sum_{C\in \mathcal{C}(G)}(-1)^{\alpha(C)}x^C.
	\end{equation}
\end{thm}

\begin{proof}
	First, assume that $G$ has no loops. The multi-dimensional characteristic polynomial is given by
	\begin{align*}
		P_{\mathcal{A}(G)}(x)&= \det(\diag(x)-\mathcal{A}(G)) \\
		&= \sum_{\sigma\in \mathfrak{S}(E(G))}\sgn(\sigma)\prod_{e\in E(G)}(\diag(x)-\mathcal{A}(G))_{\sigma(e),e}.
	\end{align*}
	For a subgraph $H\subseteq G$, let $\mathfrak{S}_H$ be the subset of $\mathfrak{S}(E(G))$, which consists of all permutation whose fixed point set is equal to $E(G)\setminus E(H)$. We can regard $\mathfrak{S}_H$ as the subset of $\mathfrak{S}(E(H))$ of permutations without fixed points. We can rewrite $P_{\mathcal{A}(G)}$ as
	$$P_{\mathcal{A}(G)}(x)=\sum_{H\subseteq G}\prod_{e\notin E(H)}(x_e-I_{e,e})\sum_{\sigma\in \mathfrak{S}_H}\sgn(\sigma)\prod_{e\in E(H)}(-I_{\sigma(e),e}),$$
	where we sum over all induced subgraphs of $G$. Since $G$ has no loops, $I_{e,e}$ is always zero, which simplifies the above formula to
	$$P_{\mathcal{A}(G)}(x)=\sum_{H\subseteq G}(-1)^{|E(H)|}x^H\sum_{\sigma\in \mathfrak{S}_H}\sgn(\sigma)\prod_{e\in E(H)}\mathcal{A}(G)_{\sigma(e),e}.$$
	Moreover, whenever a permutation $\sigma\in \mathfrak{S}(E(H))$ has a fixed point, the product $\prod\limits_{e\in E(H)}\mathcal{A}(G)_{\sigma(e),e}$ is equal to zero. Therefore, we can sum over all $\sigma\in \mathfrak{S}(E(H))$ instead of $\mathfrak{S}_H$. Further, the restriction of $\mathcal{A}(G)$ to the edges of $H$ gives us the directed adjacency matrix of $H$, i.e.
	$$I(H)=\sum_{e,e^\prime\in E(H)}I_{e,e^\prime}e\otimes e^\prime.$$
	Altogether, this implies that
	$$P_{\mathcal{A}(G)}=\sum_{H\subseteq G}(-1)^{|E(H)|}x^H\det I(H).$$
	Using Proposition \ref{prop:adjacency}, the multi-dimensional characteristic polynomial takes the form
	$$P_{\mathcal{A}(G)}=\sum_{C\in\mathcal{C}(G)}(-1)^{\eta(C)}x^C(-1)^{\alpha(C)-\eta(C)}=\sum_{C\in\mathcal{C}(G)}(-1)^{\alpha(C)}x^C,$$
	which proves the theorem for the case in which $G$ has no loops.
	
	Now, assume that $G$ has exactly one loop $e$. Split the loop into two edges $e_1,e_2$ by adding a vertex $v$ to the loop to obtain a loopless graph $H$, see Figure \ref{figure:vertexadded}. We want to use Lemma \ref{lem:reduction}: note that the graph $\tilde{H}$ obtained by removing $v$ is equal to $G$ and, moreover, the reduction of $I(H)$ to $G$ is equal to the directed adjacency matrix of $G$. We therefore have that 
	$$P_{\mathcal{A}(G)}(x_e,x_3,...,x_n)=P_{I(H)}(x_e,1,x_3,...,x_n)=\sum_{\hat{C}\in\mathcal{C}(H)}(-1)^{\alpha(\hat{C})}(x_e,1,x_3,...,x_n)^{\hat{C}}.$$
	By replacing the cycle $e_1,e_2$ with $e$, it is clear that every collection of disjoint cycles $\hat{C}\in\mathcal{C}(H)$ corresponds to a unique collection $C\in\mathcal{C}(G)$ of same size such that $e_1,e_2$ is a cycle in $\hat{C}$ if and only if $e$ is a cycle in $C$.
	% Figure environment removed
	
	Then, the monomial $(x_e,x_3,...,x_n)^C$ is equal to $(x_e,1,x_3,...,x_n)^{\hat{C}}$ and, consequently, we conclude that
	\begin{align*}
		P_{\mathcal{A}(G)}(x_e,x_3,...,x_n)&=\sum_{\hat{C}\in\mathcal{C}(H)}(-1)^{\alpha(\hat{C})}(x_e,1,x_3,...,x_n)^{\hat{C}}\\
		&= \sum_{C\in\mathcal{C}(G)}(-1)^{\alpha(C)}(x_e,x_3,...,x_n)^C.
	\end{align*}
	If $G$ has multiple loops, by repeating this argument inductively, one can prove the theorem for the most general case.
\end{proof}

We continue by examining the consequences of formula \eqref{eq:charpol} for the spectrum of $D_G$. Unfortunately, a complete characterization of $\sigma(D_G)$ cannot be given. However, it only depends on the structure of $\mathcal{C}(G)$ and the lengths of the cycles present in $G$. The total length of a subgraph $H$ of $G$ is given by the sum of the lengths of the edges of $H$, i.e.
$$L_H=\sum_{e\in E(H)}l_e.$$

\begin{cor}
	The characteristic function $P_{\mathcal{A}(G),l}$ of the directed adjacency matrix is given by 
	$$P_{\mathcal{A}(G),l}(\lambda)=\sum_{C\in\mathcal{C}(G)}(-1)^{\alpha(C)}\exp(i\lambda(L_G-L_C)).$$
	The spectrum of $D_G$ is given by the solutions to the equation
	$$\sum_{C\in\mathcal{C}(G)}(-1)^{\alpha(C)}\exp(-i\lambda L_C)=0.$$
\end{cor}

\begin{proof}
	The first statement follows from a short calculation that uses formula \eqref{eq:charpol}:
	\begin{align*}
		P_{\mathcal{A}(G),l}(\lambda)&=\sum_{C\in\mathcal{C}(G)}(-1)^{\alpha(C)}\prod_{e\notin E(C)}\exp(i\lambda l_e)\\
		&=\sum_{C\in\mathcal{C}(G)}(-1)^{\alpha(C)}\exp(i\lambda(L_G-L_C)).
	\end{align*}
	The second statement then follows by multiplying the equation $P_{\mathcal{A}(G),l}(\lambda)=0$ by $\exp(-i\lambda L_G)$.
\end{proof}

If all edges have length $1$ the characteristic function of the directed adjacency matrix is determined by the characteristic polynomial $p_{\mathcal{A}(G)}$ of $\mathcal{A}(G)$ by $$P_{\mathcal{A}(G),1}(\lambda)= p_{\mathcal{A}(G)}(e^{i\lambda})$$ and we can obtain the characteristic polynomial from the multi-dimensional characteristic polynomial by
$$p_{\mathcal{A}(G)}(t)=P_{\mathcal{A}(G)}(t,...,t).$$
This is a well-known object in spectral graph theory (see e.g. \cite{rigo}), we will prove a few basic facts to illustrate the connection of the characteristic polynomial and hence the spectrum of the Dirac operator to the topology of the graph.

\begin{cor}\label{cor:charpol}
	Let $G$ be a digraph with $n$ edges.
	The characteristic polynomial $p_{\mathcal{A}(G)}$ of the directed adjacency matrix is given by
	$$p_{\mathcal{A}(G)}(t)=\sum_{C\in\mathcal{C}(G)}(-1)^{\alpha(C)}t^{n-\eta(C)}.$$
\end{cor}

\begin{proof}
	The corollary follows directly from formula \eqref{eq:charpol}.
\end{proof}

By expressing the characteristic polynomial in the usual form 
$$p_{\mathcal{A}(G)}(t)=\sum_{k=0}^n a_k t^k,$$
Corollary \ref{cor:charpol} implies that the $k$-th coefficient of $p_{\mathcal{A}(G)}(t)$ is given by
$$a_k=\sum_{\substack{C\in\mathcal{C}(G) \\ \eta(C)=n-k}}(-1)^{\alpha(C)}.$$

Recall that the girth $g(G)$ is the length of the shortest directed cycle in $G$ and that the edge-connectivity of $G$ is equal to $k$ if $k$ is the least number of edges that have to be removed such that the remaining graph is disconnected or consists of an isolated vertex. In this case, we say that $G$ is $k$-connected.

\begin{prop}
	Let $G$ be a digraph with $n$ edges that contains at least one cycle.
	The girth of $G$ is given by
	$$g(G)=\min\{l\in\{1,...,n\}\mid a_{n-l}\neq 0\}.$$
	Moreover, $-a_{n-l}$ is equal to the number of cycles of length $l$ in $G$ for $0\leq l < 2g(G)$. Furthermore, if $G$ is $k$-connected, then, for $0\leq l <k$, the coefficient $a_l$ is the negative of the amount of cycles of length $n-l$ in $G$. 
\end{prop}

\begin{proof}
	Let $l<g(G)$ and assume that $a_{n-l}$ is not equal to zero. Then, there exists a collection of disjoint cycles $C\in \mathcal{C}(G)$ of length $l$. Consequently, $C$ contains a cycle whose length is less than $g(G)$, which is a contradiction. Therefore, $a_{n-l}$ must be zero.
	
	If a disjoint collection of cycles $C\in \mathcal{C}(G)$ is of length $l$, where $l$ satisfies $0\leq l < 2g(G)$, then the size of $C$ must be equal to one, otherwise it would contain a cycle of length smaller than $g(G)$. Consequently,
	$$a_{n-l}=\sum_{\substack{C\in\mathcal{C}(G) \\ \eta(C)=l}}-1 = -\#\{\text{cycles of length }l\}.$$
	Now, let $G$ be $k$-connected and let $0\leq l <k$. If $C\in \mathcal{C}(G)$ is of length $n-l$, then $C$ cannot be disconnected, since only $l$ edges have been removed from $G$ in order to obtain $C$. Therefore, $C$ is a cycle and 
	$$a_l=\sum_{\substack{C\in\mathcal{C}(G) \\ \eta(C)=n-l}}-1 = -\#\{\text{cycles of length }n-l\}.$$
\end{proof}

We now characterize some high-order coefficients of the characteristic polynomial.

\begin{cor}
	Let $G$ be a graph with $n$ edges.
	For the following coefficients of $p_{\mathcal{A}(G)}$, the following holds:
	\begin{itemize}
		\item $a_n=1$
		\item $a_{n-1}=-\#\{\text{loops in }G\}$
		\item $a_{n-2}=-\#\{2\text{-cycles}\}+\#\{\text{disjoint pairs of loops}\}$
		\item $a_{n-3}=-\#\{3\text{-cycles}\} + \#\{\text{disjoint pairs of loops and }2\text{-cycles}\}\\-\#\{\text{disjoint triples of loops}\}$
	\end{itemize}
\end{cor}
\begin{proof}
	This follows directly from the above considerations.
\end{proof}


\begin{exa}
	To conclude this chapter, we give some examples of characteristic polynomials of graphs:
	\begin{itemize}
		%\item A tree $T$ is a graph that contains no cycles. If $n$ is the number of edges of $T$, then the characteristic polynomial of $I(T)$ is given by $p_{I(T)}=t^n$.
		\item The characteristic polynomial of the $n$-rose $R_n$ is given by $p_{\mathcal{A}(R_n)}=t^n-nt^{n-1}$.
		\item The characteristic polynomial of the $n$-cycle $C_n$ is given by $p_{\mathcal{A}(C_n)}=t^n-1$.
		\item Let $G_1$ be the graph given in Figure \ref{figure:G1}. The characteristic polynomial of $G_1$ is given by 
		$p_{\mathcal{A}(G_1)}=t^4-2t^3$. The reason that the second-order coefficient is equal to zero is due to the fact that the cycle of length $2$ in the middle cancels out the pair of peripheric loops.
		% Figure environment removed
		\item Let $G_2$ be the graph given in Figure \ref{figure:G2}. The characteristic polynomial of $G_2$ is given by 
		$p_{\mathcal{A}(G_2)}=t^4-2t^3$ and, therefore, is equal to the characteristic polynomial of $G_1$.
		% Figure environment removed
		\item Let $G_3$ be the graph given in Figure \ref{figure:G3}. The characteristic polynomial of $G_3$ is given by
		$p_{\mathcal{A}(G_3)}=t^6-3t^4-2t^3$, which, since no two cycles of $G_3$ are disjoint, represents the cycles present in $G_3$, i.e. we have three cycles of length $2$ and two cycles of length $3$.
		% Figure environment removed
	\end{itemize}
\end{exa}

\newpage

\bibliography{bib2}{}
\bibliographystyle{plain}
\end{document}