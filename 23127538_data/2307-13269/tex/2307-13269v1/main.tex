
\documentclass{article} %
\usepackage{iclr2023_conference,times}

%%%%% NEW MATH DEFINITIONS %%%%%
\newtheorem{property}{Property}
\newtheorem{definition}{Definition}
\newtheorem{theorem}{Theorem}
\newtheorem{lemma}{Lemma}
\newtheorem{corollary}{Corollary}
\DeclarePairedDelimiter\abs{\lvert}{\rvert}
\DeclarePairedDelimiter\norm{\lVert}{\rVert}
\makeatletter
\let\oldabs\abs
\def\abs{\@ifstar{\oldabs}{\oldabs*}}
\let\oldnorm\norm
\def\norm{\@ifstar{\oldnorm}{\oldnorm*}}
\makeatother

% Mark sections of captions for referring to divisions of figures
\newcommand{\figleft}{{\em (Left) }}
\newcommand{\figcenter}{{\em (Center) }}
\newcommand{\figright}{{\em (Right)}}
\newcommand{\figtop}{{\em (Top) }}
\newcommand{\figbottom}{{\em (Bottom) }}
\newcommand{\captiona}{{\em (a) }}
\newcommand{\captionb}{{\em (b) }}
\newcommand{\captionc}{{\em (c) }}
\newcommand{\captiond}{{\em (d) }}

% Highlight a newly defined term
\newcommand{\newterm}[1]{{\bf #1}}


\def\figref#1{figure~\ref{#1}}
\def\Figref#1{Figure~\ref{#1}}
\def\twofigref#1#2{figures \ref{#1} and \ref{#2}}
\def\quadfigref#1#2#3#4{figures \ref{#1}, \ref{#2}, \ref{#3} and \ref{#4}}
\def\secref#1{section~\ref{#1}}
\def\Secref#1{Section~\ref{#1}}
\def\twosecrefs#1#2{sections \ref{#1} and \ref{#2}}
\def\secrefs#1#2#3{sections \ref{#1}, \ref{#2} and \ref{#3}}
\def\eqref#1{equation~\ref{#1}}
\def\Eqref#1{Equation~\ref{#1}}
% A raw reference to an equation---avoid using if possible
\def\plaineqref#1{\ref{#1}}
% Reference to a chapter, lower-case.
\def\chapref#1{chapter~\ref{#1}}
% Reference to an equation, upper case.
\def\Chapref#1{Chapter~\ref{#1}}
% Reference to a range of chapters
\def\rangechapref#1#2{chapters\ref{#1}--\ref{#2}}
% Reference to an algorithm, lower-case.
\def\algref#1{algorithm~\ref{#1}}
% Reference to an algorithm, upper case.
\def\Algref#1{Algorithm~\ref{#1}}
\def\twoalgref#1#2{algorithms \ref{#1} and \ref{#2}}
\def\Twoalgref#1#2{Algorithms \ref{#1} and \ref{#2}}
% Reference to a part, lower case
\def\partref#1{part~\ref{#1}}
% Reference to a part, upper case
\def\Partref#1{Part~\ref{#1}}
\def\twopartref#1#2{parts \ref{#1} and \ref{#2}}

% Random variables
\def\reta{{\textnormal{$\eta$}}}
\def\ra{{\textnormal{a}}}

% Random vectors
\def\rvepsilon{{\mathbf{\epsilon}}}
\def\rvtheta{{\mathbf{\theta}}}
\def\rva{{\mathbf{a}}}

% Elements of random vectors
\def\erva{{\textnormal{a}}}
\def\ervb{{\textnormal{b}}}

% Random matrices
\def\rmA{{\mathbf{A}}}
\def\rmB{{\mathbf{B}}}

% Elements of random matrices
\def\ermA{{\textnormal{A}}}
\def\ermB{{\textnormal{B}}}

\def\fvec{{\mathbf{f}}}
\def\bff{{\mathbf{f}}}
\def\bfg{{\mathbf{g}}}
% Vectors
\def\vzero{{\bm{0}}}
\def\vone{{\bm{1}}}
\def\vmu{{\bm{\mu}}}
\def\vtheta{{\bm{\theta}}}
\def\va{{\bm{a}}}
\def\vb{{\bm{b}}}
\def\vc{{\bm{c}}}
\def\vd{{\bm{d}}}
\def\ve{{\bm{e}}}
\def\vf{{\bm{f}}}
\def\vg{{\bm{g}}}
\def\vh{{\bm{h}}}
\def\vi{{\bm{i}}}
\def\vj{{\bm{j}}}
\def\vk{{\bm{k}}}
\def\vl{{\bm{l}}}
\def\vm{{\bm{m}}}
\def\vn{{\bm{n}}}
\def\vo{{\bm{o}}}
\def\vp{{\bm{p}}}
\def\vq{{\bm{q}}}
\def\vr{{\bm{r}}}
\def\vs{{\bm{s}}}
\def\vt{{\bm{t}}}
\def\vu{{\bm{u}}}
\def\vv{{\bm{v}}}
\def\vw{{\bm{w}}}
\def\vx{{\bm{x}}}
\def\vy{{\bm{y}}}
\def\vz{{\bm{z}}}

% Matrix
\def\mA{{\bm{A}}}

% Tensor
\DeclareMathAlphabet{\mathsfit}{\encodingdefault}{\sfdefault}{m}{sl}
\SetMathAlphabet{\mathsfit}{bold}{\encodingdefault}{\sfdefault}{bx}{n}
\newcommand{\tens}[1]{\bm{\mathsfit{#1}}}
\def\tA{{\tens{A}}}
\def\tB{{\tens{B}}}
\def\tC{{\tens{C}}}
\def\tD{{\tens{D}}}
\def\tE{{\tens{E}}}
\def\tF{{\tens{F}}}
\def\tG{{\tens{G}}}
\def\tH{{\tens{H}}}
\def\tI{{\tens{I}}}
\def\tJ{{\tens{J}}}
\def\tK{{\tens{K}}}
\def\tL{{\tens{L}}}
\def\tM{{\tens{M}}}
\def\tN{{\tens{N}}}
\def\tO{{\tens{O}}}
\def\tP{{\tens{P}}}
\def\tQ{{\tens{Q}}}
\def\tR{{\tens{R}}}
\def\tS{{\tens{S}}}
\def\tT{{\tens{T}}}
\def\tU{{\tens{U}}}
\def\tV{{\tens{V}}}
\def\tW{{\tens{W}}}
\def\tX{{\tens{X}}}
\def\tY{{\tens{Y}}}
\def\tZ{{\tens{Z}}}


% Graph
\def\gA{{\mathcal{A}}}
\def\gB{{\mathcal{B}}}
\def\gC{{\mathcal{C}}}
\def\dataset{{\mathcal{D}}}
\def\gE{{\mathcal{E}}}
\def\gF{{\mathcal{F}}}
\def\fourier{{\mathcal{F}}}
\def\gG{{\mathcal{G}}}
\def\gH{{\mathcal{H}}}
\def\gI{{\mathcal{I}}}
\def\gJ{{\mathcal{J}}}
\def\gK{{\mathcal{K}}}
\def\gL{{\mathcal{L}}}
\def\loss{{\mathcal{L}}}
\def\gM{{\mathcal{M}}}
\def\gN{{\mathcal{N}}}
\def\normal{{\mathcal{N}}}
\def\gaussian{{\mathcal{N}}}
\def\gO{{\mathcal{O}}}
\def\gP{{\mathcal{P}}}
\def\gQ{{\mathcal{Q}}}
\def\gR{{\mathcal{R}}}
\def\gS{{\mathcal{S}}}
\def\gT{{\mathcal{T}}}
\def\gU{{\mathcal{U}}}
\def\uniform{{\mathcal{U}}}
\def\gV{{\mathcal{V}}}
\def\gW{{\mathcal{W}}}
\def\gX{{\mathcal{X}}}
\def\gY{{\mathcal{Y}}}
\def\gZ{{\mathcal{Z}}}

\def\algebra{{\mathscr{A}}}
\def\borel{{\mathscr{B}}}
\def\manifold{{\mathscr{M}}}

% Sets
\def\sA{{\mathbb{A}}}
\def\sB{{\mathbb{B}}}
\def\complex{{\mathbb{C}}}
\def\sD{{\mathbb{D}}}
\def\expectation{{\mathbb{E}}}
\newcommand{\E}{\mathbb{E}}
\def\sF{{\mathbb{F}}}
\def\sG{{\mathbb{G}}}
\def\sH{{\mathbb{H}}}
\def\sI{{\mathbb{I}}}
\def\sJ{{\mathbb{J}}}
\def\sK{{\mathbb{K}}}
\def\sL{{\mathbb{L}}}
\def\sM{{\mathbb{M}}}
\def\natural{{\mathbb{N}}}
\def\sO{{\mathbb{O}}}
\def\sP{{\mathbb{P}}}
\def\rational{{\mathbb{Q}}}
\def\real{{\mathbb{R}}}
\newcommand{\R}{\mathbb{R}}
\def\sS{{\mathbb{S}}}
\def\sphere{{\mathbb{S}}}
\def\sT{{\mathbb{T}}}
\def\sU{{\mathbb{U}}}
\def\sV{{\mathbb{V}}}
\def\sW{{\mathbb{W}}}
\def\sX{{\mathbb{X}}}
\def\sY{{\mathbb{Y}}}
\def\integer{{\mathbb{Z}}}
\def\indicator{{\mathbbm{1}}}

% Entries of a matrix
\def\emLambda{{\Lambda}}
\def\emA{{A}}
\def\emB{{B}}
\def\emC{{C}}
\def\emD{{D}}
\def\emE{{E}}
\def\emF{{F}}
\def\emG{{G}}
\def\emH{{H}}
\def\emI{{I}}
\def\emJ{{J}}
\def\emK{{K}}
\def\emL{{L}}
\def\emM{{M}}
\def\emN{{N}}
\def\emO{{O}}
\def\emP{{P}}
\def\emQ{{Q}}
\def\emR{{R}}
\def\emS{{S}}
\def\emT{{T}}
\def\emU{{U}}
\def\emV{{V}}
\def\emW{{W}}
\def\emX{{X}}
\def\emY{{Y}}
\def\emZ{{Z}}
\def\emSigma{{\Sigma}}

% entries of a tensor
% Same font as tensor, without \bm wrapper
\newcommand{\etens}[1]{\mathsfit{#1}}
\def\etLambda{{\etens{\Lambda}}}
\def\etA{{\etens{A}}}
\def\etB{{\etens{B}}}
\def\etC{{\etens{C}}}
\def\etD{{\etens{D}}}
\def\etE{{\etens{E}}}
\def\etF{{\etens{F}}}
\def\etG{{\etens{G}}}
\def\etH{{\etens{H}}}
\def\etI{{\etens{I}}}
\def\etJ{{\etens{J}}}
\def\etK{{\etens{K}}}
\def\etL{{\etens{L}}}
\def\etM{{\etens{M}}}
\def\etN{{\etens{N}}}
\def\etO{{\etens{O}}}
\def\etP{{\etens{P}}}
\def\etQ{{\etens{Q}}}
\def\etR{{\etens{R}}}
\def\etS{{\etens{S}}}
\def\etT{{\etens{T}}}
\def\etU{{\etens{U}}}
\def\etV{{\etens{V}}}
\def\etW{{\etens{W}}}
\def\etX{{\etens{X}}}
\def\etY{{\etens{Y}}}
\def\etZ{{\etens{Z}}}

\def\ceil#1{\lceil #1 \rceil}
\def\floor#1{\lfloor #1 \rfloor}
\def\eps{{\epsilon}}

\newcommand{\pder}[1]{\frac{\partial}{\partial #1}}

\newcommand{\half}{\frac{1}{2}}
\newcommand{\limNinf}{\lim_{N \to \infty}}
\newcommand{\limTzero}{\lim_{\tau \to 0}}


\newcommand{\cmark}{\ding{51}}
\newcommand{\xmark}{\ding{55}}

\newcommand{\layer}{\mathcal{H}}
\newcommand{\defeq}{\triangleq}
%\newcommand{\defeq}{vcentcolon=}
\newcommand{\domain}{\Omega}
\newcommand{\grad}{\nabla}

\newcommand{\cin}{c_{\rm{in}}}
\newcommand{\cout}{c_{\rm{out}}}
\newcommand{\intdomain}{\int_{\domain}}
\newcommand{\network}{\gT}
\newcommand{\subnet}{\gK}
\newcommand{\map}{\gR} %\gR

\newcommand{\innerproduct}[2]{\langle #1, #2 \rangle}
\newcommand{\mcsum}[1][j]{\frac{1}{N}\sum_{#1=1}^N}

\newcommand{\inrspace}[1][c]{\gF_{#1}}

\DeclareMathOperator*{\argmax}{arg\,max}
\DeclareMathOperator*{\argmin}{arg\,min}

\let\ab\allowbreak


\usepackage[colorlinks=true,citecolor=.]{hyperref}
\usepackage{url}
\usepackage{inconsolata}
\usepackage{eurosym}
\usepackage{todonotes} 
\usepackage{subcaption}
\usepackage{amssymb}
\usepackage{amsmath}
\usepackage{enumitem}
\usepackage{array}
\usepackage{longtable}
\usepackage{makecell}
\usepackage{xspace}
\usepackage{xcolor,colortbl}
\usepackage{tcolorbox}
\usepackage{arydshln}
\usepackage{adjustbox}
\newcommand{\dline}{\hdashline[0.5pt/1pt]}
\usepackage{tikz}
\usepackage[tikz]{bclogo}
\usepackage{pgfplots}
\pgfplotsset{width=1.0\columnwidth}
\usepackage{multirow}

\usepackage{makecell}
\usepackage{pifont}
\usepackage{bbm}
\usepackage{rotating}
\usepackage{tablefootnote}
\usepackage{soul}
\usepackage{booktabs}
\usepackage{tikz}
\usepackage[tikz]{bclogo}
\usepackage{pgfplotstable}
\usepackage{mdframed}
\usepackage{graphicx}

\newcommand{\yuchen}[1]{\textcolor{red}{[Yuchen: #1]}} 
\newcommand{\lorahub}{LoraHub\xspace}
\newcommand{\flan}[0]{FLAN-T5\xspace}
\newcommand\scale[1]{{\fontfamily{mathtt}\selectfont {#1}}\xspace}
\newcommand{\icon}{\raisebox{-1pt}{% Figure removed}\xspace}


\newmdenv[
  backgroundcolor=purple!10,
  skipabove=1em,
  skipbelow=0em,
  leftline=true,
  topline=false,
  bottomline=false,
  rightline=false,
  linecolor=purple!88,
  linewidth=4pt
]{githubquote}


\title{\icon \lorahub{}: Efficient Cross-Task Generalization via Dynamic LoRA  Composition}



\newcommand*{\affaddr}[1]{#1}
\newcommand*{\affmark}[1][*]{\textsuperscript{#1}}
\newcommand*{\email}[1]{\tt{#1}}

\author{
\makecell{Chengsong Huang\affmark[\textdagger\S]{\thanks{The first three authors contributed equally to this work. Correspondence to Qian Liu at \href{mailto:liuqian@sea.com}{\texttt{liuqian@sea.com}}.}}~~, Qian Liu\affmark[\textdagger]$^*$, Bill Yuchen Lin\affmark[$\lozenge$]$^*$, Tianyu Pang\affmark[\textdagger], Chao Du\affmark[\textdagger], Min Lin\affmark[\textdagger]}
\\
\centerline{\affaddr{\affmark[\textdagger]Sea AI Lab, Singapore}}\\
\centerline{\affaddr{\affmark[\S]Washington University in St. Louis, MO, USA}}\\
\centerline{\affaddr{\affmark[$\lozenge$]Allen Institute for AI, Seattle, WA, USA}}
}


\newcommand{\fix}{\marginpar{FIX}}
\newcommand{\new}{\marginpar{NEW}}

\iclrfinalcopy %
\begin{document}


\maketitle

\begin{abstract}



Low-rank adaptations (LoRA) are often employed to fine-tune large language models (LLMs) for new tasks. This paper investigates LoRA composability for cross-task generalization and introduces \lorahub, a strategic framework devised for the purposive assembly of LoRA modules trained on diverse given tasks, with the objective of achieving adaptable performance on unseen tasks. With just a few examples from a novel task, \lorahub enables the fluid combination of multiple LoRA modules, eradicating the need for human expertise. Notably, the composition requires neither additional model parameters nor gradients. Our empirical results, derived from the Big-Bench Hard (BBH) benchmark, suggest that \lorahub can effectively mimic the performance of in-context learning in few-shot scenarios, excluding the necessity of in-context examples alongside each inference input. A significant contribution of our research is the fostering of a community for LoRA, where users can share their trained LoRA modules, thereby facilitating their application to new tasks. We anticipate this resource will widen access to and spur advancements in general intelligence as well as LLMs in production.
Code will be available at \href{https://github.com/sail-sg/lorahub}{\texttt{github.com/sail-sg/lorahub}}.
\end{abstract}

% Figure environment removed


\section{Introduction}

\section{Introduction}

% Figure environment removed

Reinforcement Learning from Human Feedback (RLHF) has recently been used to great effect to align pretrained large language models (LLMs) to human preferences, optimizing for desirable qualities like harmlessness and helpfulness~\citep{bai2022training} and achieving state-of-the-art results across a variety of natural language tasks~\citep{openai2023gpt4}. %RLHF approaches fundamentally rely on collecting pairs of LLM outputs $(o_1, o_2)$ from a shared prompt $p$, with a human indicating which output in each pair is better on a specified attribute.
% A fundamental component of RLHF is a preference model derived from human labels, typically formatted as pairs of LLM outputs $(o_1, o_2)$ generated from a shared prompt $p$.

A standard RLHF procedure fine-tunes an initial unaligned LLM using an RL algorithm such as PPO~\citep{schulman2017proximal}, optimizing the LLM to align with human preferences. %\violet{not sure whether we need to provide this detail in the intro, especially this has nothing to do with our contribution.} % i feel like this context is useful later when e.g. explaining that context distillation is SFT
RLHF is thus critically dependent on a reward model derived from human-labeled preferences, typically \textit{pairwise preferences} on LLM outputs $(o_1, o_2)$ generated from a shared prompt $p$. % and labeled by humans. 

However, collecting human pairwise preference data, especially high-quality data, may be expensive and time consuming at scale. To address this problem, approaches have been proposed to obtain labels without human annotation, such as Reinforcement Learning from AI Feedback (RLAIF) and context distillation. 

\iffalse
raising the question of whether we can generate high-quality data for RLHF without using human labeling. %accurately-labeled preference pairs $(o_1, o_2)$
%, motivating model alignment approaches that aim to generate accurately-labeled preference pairs $(o_1, o_2)$ without human involvement. 
Two major categories of such approaches are . 
\fi

RLAIF approaches (e.g.,~\citet{bai2022constitutional}) simulate human pairwise preferences by scoring $o_1$ and $o_2$ with an LLM (Figure \ref{fig:rlcd_differences} center); the scoring LLM is often the same as the one used to generate the original pairs $(o_1, o_2)$. Of course, the resulting LLM pairwise preferences will be somewhat noisier compared to human labels. However, this problem is exacerbated by using the same prompt $p$ to generate both $o_1$ and $o_2$, causing $o_1$ and $o_2$ to often be of very similar quality and thus hard to differentiate (e.g., Table~\ref{tab:rlaif_bad_example}). Consequently, training signal can be overwhelmed by label noise, yielding lower-quality preference data. 

% While it avoids human labeling efforts, it has weakness. First, LLM preference labels will naturally be somewhat noisier compared to human labels. Furthermore, since the same prompt $p$ is used to generate both $o_1$ and $o_2$, their quality is often very similar and hard to differentiate (See Table~\ref{tab:rlaif_bad_example}). As a result, training signals can be overwhelmed by label noise, yielding lower-quality preference data. 

Meanwhile, context distillation methods (e.g., \citet{sun2023principle}) create more training signal by modifying the initial prompt $p$. 
%to create more significant training signal. 
The modified prompt $p_+$ typically contains additional context encouraging a \textit{directional attribute change} in the output $o_+$ (Figure \ref{fig:rlcd_differences} right). However, context distillation methods only generate a single output $o_+$ per prompt $p_+$, which is then used for supervised fine-tuning, losing the pairwise preferences which help RLHF-style approaches to 
%rather than using a RLHF-style preference model to 
derive signal from the contrast between outputs. 
Multiple works have observed that RL approaches using preference models for pairwise preferences can substantially improve over supervised fine-tuning by itself when aligning LLMs~\citep{ouyang2022training,dubois2023alpacafarm}. 

% conduct alignment by running supervised fine-tuning on model outputs $o_+$ generated from a modified prompt $p_+$. $p_+$ typically contains additional context encouraging desirable attributes (Figure \ref{fig:rlcd_differences} right), such as in \citet{sun2023principle}. However, multiple works have observed that RLHF-style approaches can substantially improve over supervised fine-tuning by itself when aligning LLMs~\citep{ouyang2022training,dubois2023alpacafarm}. 

Therefore, while both RLAIF and context distillation approaches have already been successfully applied in practice to align language models, we posit that it may be even more effective to combine the key advantages of both. That is, we will use RL with \textit{pairwise preferences}, while also using modified prompts to encourage \textit{directional attribute change} in outputs. %In particular, we will adapt the RLAIF data generation process with two different prompts rather than a single $p$, modifying both prompts similarly to context distillation. %\violet{this motivation is a little unexciting. I think we can more specifically discuss the potential benefits of our approach, like the benefits from RL: exploration/data generation; benefits from contrast. I don't think we get too much benefits from context distillation since we switched to the RL framework.} 

Concretely, we propose \oursfull{} (\ours{}). 
\ours{} generates preference data as follows. Rather than producing two i.i.d.\ model outputs $(o_1, o_2)$ from the same prompt $p$ as in RLAIF, \ours{} creates two variations of $p$: a \textit{positive prompt} $p_+$ similar to context distillation which encourages directional change toward a desired attribute, and a \textit{negative prompt} $p_-$ which encourages directional change \textit{against} it (Figure \ref{fig:rlcd_differences} left). We then generate model outputs $(o_+, o_-)$ respectively, and automatically label $o_+$ as preferred---that is, \ours{} automatically ``generates'' pairwise preference labels by construction. %, without further post hoc labeling.\violet{should make it clearer that our approach `generates' labels by construction} 
We then follow the standard RL pipeline of training a preference model followed by PPO. 

Compared to RLAIF-generated preference pairs $(o_1, o_2)$ from the same input prompt $p$, there is typically a clearer difference in the quality of $o_+$ and $o_-$ generated using \ours{}'s directional prompts $p_+$ and $p_-$, which may result in less label noise. %which may result in better training signal for the preference model. 
That is, intuitively, \ours{} exchanges having examples be \textit{closer to the classification boundary} for much more \textit{accurate labels} on average. Compared to standard context distillation methods, on top of leveraging pairwise preferences for RL training, \ours{} can derive signal not only from the positive prompt $p_+$ which improves output quality, but also from the negative prompt $p_-$ which degrades it. %\ours{} is not learning to imitate $o_+$, but to distill the \textit{contrast} between $o_+$ and $o_-$. 
Positive outputs $o_+$ don't need to be perfect; they only need to contrast with $o_-$ on the desired attribute while otherwise following a similar style.

% \todo{discuss our method and why intuitively it may be better.}

We evaluate the practical effectiveness of \ours{} through both human and automatic evaluations on three tasks, aiming to improve the ability of LLaMA-7B~\citep{touvron2023llama} to generate harmless outputs, helpful outputs, and high-quality story outlines. %\ours{} outperforms both RLAIF and context distillation baselines in pairwise comparisons on 
As shown in Sec. \ref{sec:experiments}, \ours{} substantially outperforms both RLAIF and context distillation baselines in pairwise comparisons when simulating preference data with LLaMA-7B, while still performing equal or better when simulating with LLaMA-30B. 
%On all three tasks, \ours{} substantially outperforms both RLAIF and context distillation baselines in pairwise comparisons---by a margin of at least 9\% and often more than 30\%---validating our method's efficacy. 
We will release all code at a later date, although in any case \ours{} is fairly easy to implement by modifying any reference RLAIF codebase. %We release all code at \todo{github link}.



\section{Problem Statement}

\section{Problem Statement}

\textbf{Large Language Models}\;\; We assume that a large language model $M_\theta$ is based on Transformer architecture~\citep{transformer_paper} and has been pre-trained on a large-scale text corpus. The model architecture can be either encoder-decoder~\citep{t5_paper} or decoder-only~\citep{gpt3}. Also, $M_\theta$ could also have been fine-tuned with a large set of instruction-following datasets such as Flan Colleciton~\citep{longpre2023flan} and PromptSource~\citep{bach2022promptsource}.

\textbf{Cross-Task Generalization}\;\; In real-world situations, users often desire an LLM to perform novel tasks that it has not encountered before — an ability widely known as \textit{cross-task generalization}.
Generally, cross-task generalization falls into two categories: zero-shot learning~\citep{mishra-etal-2022-cross,t0_paper,flan,chatgpt_paper, recross}, which necessitates no labeled examples of the new task, and few-shot learning~\citep{ye-etal-2021-crossfit,min-etal-2022-metaicl} which demands a handful of labeled examples.
Assume we have $N$ distinct \textit{upstream tasks} that the LLM has been trained on, denoted as $\mathbb{T}=\{\mathcal{T}_1, ..., \mathcal{T}_N\}$.
Our paper primarily focuses on the latter category, where for an unseen target task $\mathcal{T}' \notin \mathbb{T}$, users can only provide a limited set of labeled examples, $Q$.
Our aim is to modify the model $M_\theta$ to adapt it to task $\mathcal{T}'$ using only $Q$.
An intuitive method would be to fine-tune the weights of $M_\theta$ based on $Q$, yielding an updated model $M_\phi$ with enhanced performance on $\mathcal{T}'$. However, this approach is inefficient, time-consuming, and unstable when $Q$ is small.

\textbf{LoRA Tuning}\;\; LoRA~\citep{hu2022lora}, a parameter-efficient fine-tuning method, facilitates the adaptation of LLMs using lightweight modules, eliminating the need for fine-tuning the entire weights.
LoRA tuning involves keeping the original model weights frozen while introducing trainable low-rank decomposition matrices as adapter modules into each layer of the model.
Compared to the base LLM, this module possesses significantly fewer trainable parameters, paving the way for rapid adaptation using minimal examples.
As such, LoRA tuning presents a resource-efficient technique to quickly adapt LLMs for new tasks with restricted training data.
However, traditional LoRA methods primarily concentrate on training and testing within the same tasks~\citep{Gema2023ParameterEfficientFO}, rather than venturing into few-shot cross-task generalization.


\section{Methodology}
\section{Methodology}
\label{sec:method}

\subsection{Overview}
\label{sec:method_fmwk}

As shown in~\cref{fig:method_fmwk}, the proposed unsupervised MOT framework is trained with the widely-used contrastive learning technique~\cite{chen2020simple,he2020momentum}. 
\lk{Specifically, for multi-object tracking}, objects within the tracklet ($\boldsymbol{k}_{+}$) should be pulled together and different tracklets ($\boldsymbol{k}_{-}$) should be separated. It can be mathematically formulated as:

\begin{equation}
% \begin{split}
    \mathcal{L}_{cl}( \boldsymbol{q}; \boldsymbol{k}_{+}; \boldsymbol{k}_{-} )= 
    - \log \frac{\exp(\boldsymbol{q} \cdot \boldsymbol{k}_{+} / \epsilon)}{\sum_{i}\exp(\boldsymbol{q} \cdot \boldsymbol{k}_{i} / \epsilon)}
  \label{eq:method_nce}
% \end{split}  
\end{equation}

\noindent where $\mathcal{L}_{cl}$ denotes the InfoNCE~\cite{oord2018representation} loss function, and $\epsilon$ is the temperature hyper-parameter~\cite{wu2018unsupervised}. 
Within a video, following the unsupervised tracking fashion~\cite{liu2022online,shuai2022id}, the positive and negative keys mainly come from two sources, \ie pseudo-labeled historical frame and self-augmented frame. 

\lk{However, two issues occur: (1) the uncertainty reduces the accuracy of pseudo-tracklets and (2) the randomly augmented samples fail to learn the inter-frame consistency.} 
We argue the above issues are not independent. 
\lk{By leveraging the uncertainty in turn,} the accurate pseudo-tracklets can guide the qualified positive and negative augmentations.

To address these two issues, we propose an uncertainty-aware pseudo-tracklet labeling strategy in \cref{sec:method_uoap}, which integrates a verification-and-rectification mechanism into the tracklet generation. Our method significantly improves the accuracy of pseudo-identities, especially in long-term interval. 
Then we propose a tracklet-guided augmentation strategy in \cref{sec:method_ada_aug}, which brings the temporary information into spatial augmentation. The augmented samples simulates the objects' motion. A hierarchical uncertainty-based sampling strategy is proposed for hard sample mining. More details are described in the following section.


\subsection{Uncertainty-aware Tracklet-Labeling}
\label{sec:method_uoap}

Accurate pseudo tracklet is critical in \liuk{learning feature consistency}. 
However, without manual annotation, \lk{the aggravated uncertainty makes} the tracklet-labeling a huge challenge due to various interference factors, including similar appearance among objects, frequent object cross and occlusions, \etc. 
\lk{In fact, the uncertainty can also be leveraged to improve the pseudo-accuracy in turn.}
In this section, we propose an \textbf{U}ncertainty-aware \textbf{T}racklet-\textbf{L}abeling (\textbf{UTL}) strategy for better pseudo-tracklets.

Given an input video sequence $V = \{I^{1}, I^{2}, \cdots, I^{N}\}$, each frame $I^{t}$ is annotated with the bounding boxes $B^{t} = \{b_{1}^{t}, b_{2}^{t}, \cdots, b_{M^{t}}^{t}\}$ of $M^{t}$ objects in $t_{th}$ frame, where $b_{i}^{t} = (cx_{i}^{t}, cy_{i}^{t}, w_{i}^{t}, h_{i}^{t})$ is the center coordinate and shape of the $i_{th}$ object $o_{i}^{t}$. As shown in~\cref{fig:method_fmwk}, \mywork~generates accurate pseudo-tracklets in four main steps:

1) \textbf{Association}. For a certain object $o_{i}^{t}$ in frame $I^{t}$, the $\ell_2$-normalized representation $\boldsymbol{f}_{i}^{t}$ can be expressed as $\boldsymbol{f}_{i}^{t} = {\phi}(I^{t}, b_{i}^{t})$, 
% \begin{equation}
%   \boldsymbol{f}_{i}^{t} = {\phi}(I^{t}, b_{i}^{t})
%   % / {\Vert {\phi}(I^{t}, b_{i}^{t}) \Vert}_{2}
%   \label{eq:method_feat}
% \end{equation}
where the embedding encoder is denoted as $\phi$.

To associate the objects in frame $I^{t}$ with the objects or trajectories in previous $I^{t \minus 1}$, a similarity matrix is constructed with their appearance embeddings:

\begin{equation}
  \boldsymbol{C} \in \mathbb{R}^{M^{t} \times M^{t \minus 1}}, \;
  c_{i,j} = {\boldsymbol{f}_{i}^{t}} \cdot  \boldsymbol{f}_{j}^{t \minus 1}
  \label{eq:method_matrix}
\end{equation}

\noindent where $c_{i,j}$ represents the cosine similarity between the $i_{th}$ object in frame $I^{t}$ and the $j_{th}$ object (or trajectory) in frame $I^{t \minus 1}$. Then the Hungarian algorithm~\cite{kuhn1955hungarian} is adopted to generate the identity association results.

2) \textbf{Verification}. However, the appearance representations are sometimes unreliable, especially in the unsupervised scenario. To solve this issue, an uncertainty metric is proposed to evaluate the association after the first stage.

% For an object $o_{i}^{t}$ in frame $I^{t}$, the similarities against the $M^{t \minus 1}$ objects in the previous frame can be expressed as:

% \begin{equation}
%   \boldsymbol{s}_{i} = \boldsymbol{C}_{i} = [c_{i,1}, c_{i,2}, \cdots, c_{i,M^{t \minus 1}}]^T
%   \label{eq:method_svec}
% \end{equation}

% Inspired by margin-based OOD detection~\cite{hendrycks2016baseline}, we assume that the assignment ($o_{i}^{t} \!\sim\! o_{j}^{t \minus 1}$) in the association stage is not convincing under the following circumstances:

% \begin{itemize}
%     \setlength{\itemsep}{0pt}
%     \item The assigned similarity between $o_{i}^{t}$ and $o_{j}^{t \minus 1}$ is relatively low (\ie, $c_{i,j} < m_1$).
%     \item The second-highest similarity with others ($c_{i,j_{2}}$) is close to the assigned $o_{j}^{t \minus 1}$ (\ie, $c_{i,j} - c_{i,j_{2}} < m_2$).
% \end{itemize}

% Based on these assumptions, we define an association-level uncertainty metric, which is formulated as:



Object association can be viewed as multi-category classification.
And confidence-score has been proved efficient and effective on detecting mis-classified examples~\cite{hendrycks2016baseline}.
Inspired by this, we propose to detect the mis-associated objects through the similarity-scores.


Given an object $o_{i}^{t}$ associated with $o_{j}^{t \minus 1}$ in the previous frame based on \cref{eq:method_matrix}, the association ($o_{i}^{t} \!\sim\! o_{j}^{t \minus 1}$) is unconvincing in two cases: 
1) the assigned similarity $c_{i,j}$ is relatively low (\eg, partial occlusion or motion blur) and 
2) there are other objects whose similarities are close to the assigned $c_{i,j}$ (\eg, similar appearance or indistinguishable embedding).
It can be formulated as:

\begin{equation}
  c_{i,j} < m_1; \quad c_{i,j_{2}} > c_{i,j} - m_2
  \label{eq:method_margin}
\end{equation}


\noindent 
where $m_1,m_2$ are constant margins. Only the second-highest similarity with others ($c_{i,j_{2}}$) is considered for simplicity.
In an ideal association, $c_{i,j}$ should be close to 1 and $c_{i,j_{2}}$ close to 0.
We thus proposed to estimate the association \lk{risk} as:

% \sigma_{i,j} = - \left( 
% \log c_{i,j} + \log \left( 1 - c_{i,j_{2}} \right)
% + \overline{\log \left( 1 - c_{i,l} \right) }
% \right)  
\begin{equation}
  \sigma_{i,j} = - \log c_{i,j} - \log \left( 1 - c_{i,j_{2}} \right)
  \label{eq:method_energy}
\end{equation}

Detailed derivation process refers to the supplementary materials.
Combining with \cref{eq:method_margin} and \cref{eq:method_energy} , an adaptive threshold is proposed:

\begin{equation}
  % \gamma_{i,j} = -\log \left( 1 + m_2 - c_{i,j} \right) -\log m_1 \left( 1 - m_3 \right)
  \gamma_{i,j} =  -\log m_1 - \log \left( 1 + m_2 - c_{i,j} \right)
  \label{eq:method_border}
\end{equation}

As shown in~\cref{fig:method_verify}, when the \lk{risk} $\sigma_{i,j}$ is higher than the threshold $\gamma_{i,j}$, the assignment ($o_{i}^{t} \!\sim\! o_{j}^{t \minus 1}$) should be re-considered. 
\lk{The \textbf{association uncertainty} is quantified as:}

\begin{equation}
  \delta_{i,j} = \sigma_{i,j} - \gamma_{i,j}
  \label{eq:method_uncertain}
\end{equation}

The results are not sensitive to the exact margins. We set $m_1 = 0.5$ and $m_2 = 0.05$ for a slightly better performance.
% More experimental details are shown in the supplementary materials.

The uncertain pairs after the verification stage and unmatched objects after the association stage are gathered as uncertain candidates for the rectification stage.


3) \textbf{Rectification}. 
The rectification stage is performed among the uncertain candidate. The similarities between two adjacent frames are no longer convincing.
% due to irregular motion, severe occlusion, and so on. 
More information should be taken into account, including motion \lk{estimation} and appearance \lk{variation} within a tracklet. 
% Specifically, intersection-over-union (IoU)~\cite{bewley2016simple} is the widely-used motion metric. At the same time, the tracklet embeddings can provide complementary appearance information.

For the uncertain candidates, \mywork~constructs another similarity matrix for the secondary rectification. 
First, \lk{the motion constraints should be relaxed}, so the association shares overlap \lk{higher than} $\beta$ 
% in adjacent frames 
\lk{are preserved}. 
Second, \lk{the appearance should not vary extremely fast}, so we adopt the averaged similarity between object $o_{i}^{t}$ and tracklet $trk_{j} = \{o_{j}^{t \minus K}, \cdots, o_{j}^{t \minus 1}\}$ within previous $K$ frames. 
In this stage, we solve the sub-problem of global identity assignments, which can be formulated as:

\begin{equation}
\begin{split}
  \boldsymbol{C}^\prime \in \mathbb{R}^{{M^{t}}^\prime \times {M^{t \minus 1}}^\prime} & \\
  c^\prime_{i,j} = \left( \frac{1}{K} \sum_{\hat{t} = t \minus K}^{t \minus 1} {\boldsymbol{f}_{i}^{t}} \cdot  \boldsymbol{f}_{j}^{\hat{t}} \right) 
            \times \mathbb{I} & \left( \text{IoU} \left( b_{i}^{t}, b_{j}^{t \minus 1} \right) > \beta \right) 
  \label{eq:method_recti}
\end{split}
\end{equation}

\noindent where $\mathbb{I}(*)$ is the indicator function. Then the match set is updated based on the Hungarian algorithm.

\lk{
\textit{Remark.} Our core contribution is the uncertainty-based verification mechanism, rather than the specific rectification, which shall be adjusted in practice. Empirically we set $\beta=0.1$ and $K=5$.
}

% Figure environment removed

4) \textbf{Propagation}. The pseudo-tracklets are propagated frame-by-frame. As shown in~\cref{fig:method_reidacc}, our strategy brings \lk{consistently} accurate pseudo-identities, \lk{\eg, reaching 97\% accuracy across 100 frames}.
% The pseudo-tracklets are progressively updated during the training process.
The long-term intra-tracklet consistency is successfully maintained.
% by the accurate pseudo-identities.

It is worth mentioning that the \lk{verification and rectification} stages can be naturally applied to the inference process to boost the performance, \lk{which does not conflict with existing association methods}.

\subsection{Tracklet-Guided Augmentation}
\label{sec:method_ada_aug}

The accurate pseudo-tracklets can guide the sample augmentation in the unsupervised MOT framework.
To learn the \liuk{inter-frame consistency}~\cite{chen2020simple,zhang2021fairmot}, good training samples should be diverse and \liuk{temporal-aware}. 
However, as illustrated in~\cref{fig:method_ada_aug}, existing methods usually treat augmentation and multi-object tracking as two isolated tasks, leading to ineffective augmentations. 
Instead, this paper utilizes the tracklet's spatial displacements to guide the augmentation process. 
According to a properly selected anchor pair, the proposed strategy makes the augmented frames aligned to the historical frames, simulating realistic tracklet movements. The proposed method concurrently focuses on the hard negative samples.
Details \lk{of the \textbf{T}racklet-\textbf{G}uided \textbf{A}ugmentation (TGA)} are given below.

% Figure environment removed

We introduce the temporal information into spatial transformation. 
First, given a current frame $I^{t}$ with $M^{t}$ objects, we select a source-anchor object $o_{a}^{t}$, whose bounding box is denoted as $b_{a}^{t} = (cx_{a}^{t}, cy_{a}^{t}, w_{a}^{t}, h_{a}^{t})$. Then, we choose a target-anchor $o_{a}^{t \minus \tau}$ in $(t \minus \tau)_{th}$  historical frame from the pseudo-tracklet $trk_{a} = \{o_{a}^{t_0}, o_{a}^{t_1}, \cdots, o_{a}^{t}\}$. 
Finally, to augment the current $I^{t}$ to align with historical $I^{t \minus \tau}$,  a tracklet-guided affine transformation can be expressed as:

\begin{equation}
  \begin{bmatrix}
      x^{t \minus \tau} \\ y^{t \minus \tau} \\ 1
  \end{bmatrix}
  =
  \boldsymbol{M}_{t}^{t \minus \tau}
  \begin{bmatrix}
      x^{t} \\ y^{t} \\ 1
  \end{bmatrix}
  =
  \begin{bmatrix}
      m_{11} & m_{12} & m_{13} \\
      m_{21} & m_{22} & m_{23} \\
      0      & 0      & 1
  \end{bmatrix}
  \begin{bmatrix}
      x^{t} \\ y^{t} \\ 1
  \end{bmatrix}
  \label{eq:method_affine}
\end{equation}

\noindent where $x^*,y^*$ are spatial coordinates, and $\boldsymbol{M}_{t}^{t \minus \tau}$ can be solved by direct linear transform (DLT) algorithm ~\cite{detone2016deep}. 
% with the corner locations of the anchor pair $(o_{a}^{t} \!\sim\! o_{a}^{t \minus \tau})$. 
Then an augmented frame $\tilde{I}^{t}$ is generated based on the tracklet-guided affine transformation with perspective jitter, which can be expressed as $\tilde{I}^{t} = \mathcal{T}\left(I^{t}, M_{t}^{t \minus \tau} \right)$.
% \begin{equation}
%   \tilde{I}^{t} = \mathcal{T}\left(I^{t}, M_{t}^{t \minus \tau} \right)
%   \label{eq:method_aug}
% \end{equation}

Intuitively, a proper anchor-selection is vitally important for our augmentation strategy. 

First, the identity accuracy of anchor pair $(o_{a}^{t} \!\sim\! o_{a}^{t \minus \tau})$ is important. In other words, the consistency of anchor tracklet $trk_{a}$ should be guaranteed. We thus design a tracklet-level uncertain metric based on the propagated association-level uncertainty defined in \cref{eq:method_uncertain}, which is formulated as:

\begin{equation}
  \Omega_{i} = \frac{1}{n} \sum_{s=t_0}^{t} \exp (\delta_{i}^{s})
  % \Omega_{i} = \sqrt[n]{\sigma_{i}^{t_0} \cdot \sigma_{i}^{t_1} \cdots \sigma_{i}^{t}}
  \label{eq:method_tenergy}
\end{equation}

\noindent where $\Omega_{i}$ represents the uncertainty of tracklet $trk_{i}$, \lk{and $n$ is the tracklet length}.
An uncertainty-based sampling strategy is designed to select the source anchor $o_{a}^{t}$ (along with the anchor $trk_{a}$) from the $M^{t}$ objects in frame $I^{t}$, which can be formulated as:

\begin{equation}
  p\left(a=i \mid t \right) 
  % = softmax\left(-\Omega_{i}\right)
  = \frac{\exp{\left(-\Omega_{i}\right)}}{\sum_{\hat{i}=1}^{M^{t}}\exp{\left(-\Omega_{\hat{i}}\right)}}
  \label{eq:method_sel_an_src}
\end{equation}

\noindent where $p\left(a=i \mid t \right)$ represents the probability to choose the $i_{th}$ tracklet $trk_{i}$ as the anchor $trk_{a}$.
The uncertain tracklet with high $\Omega$ is less likely to be selected, avoiding dramatic augmentations from erroneous pseudo-tracklets.

Second, hard negative samples matters in discriminablity learning. We tend to choose an indistinguishable (or, high uncertain) target anchor $o_{a}^{t \minus \tau}$ along the tracklet $trk_{i}$. The selection probability can be formulated as:

\begin{equation}
  p\left(\pi=t \minus \tau \mid a \right) 
  = \frac{\exp{\left(\delta_{a}^{t \minus \tau}\right)}}{\sum_{\hat{\tau}=t_0}^{t-1}\exp{\left(\delta_{a}^{t-\hat{\tau}}\right)}}
  \label{eq:method_sel_an_tgt}
\end{equation}

\lk{A visualization example are displayed in the supplementary material to illustrate the hierarchical sampling process.}

Compared with conventional random transformation, the proposed tracklet-guided augmentation is well-directed and tracking-related. 
\lk{Together with accurate pseudo-tracklets, \mywork~successfully improves the inter-frame consistency, as shown in \cref{fig:method_disc_vis}. }


% Figure environment removed

% \subsection{Momentum Memory Dictionary}
% \label{sec:method_md}


%To reuse the encoded samples from the intermediate mini-batches, we maintain a queue for each video in the memory dictionary by enqueueing the $M^{t}$ objects in the current frame and removing the oldest samples.
%In representation learning, high-quality negative samples play an essential role~\cite{chen2020simple,he2020momentum}. However, existing unsupervised trackers only take negative samples from adjacent frames, augmented frames, and the current frame itself. The lack of negative sample diversity prevents trackers from learning discriminative representations. \mywork~adopts a momentum dictionary mechanism to alleviate this problem.

%As shown in~\cref{fig:method_fmwk}, we build a memory dictionary for each \textit{independent} video input during training. Given an input image $I^{t}$ from video $V$, we randomly sample a number of negative object samples from other videos in the memory dictionary, so as to enrich the negative sample diversity. To reuse the encoded samples from the intermediate mini-batches, we maintain a queue for each video in the memory dictionary by enqueueing the $M^{t}$ objects in the current frame and removing the oldest samples.


   










\section{Evaluation}
In this section, we provide details on our main experiments. First, we give an overview of the experimental setup and implementation details. Next, we present our findings along with the results.

\subsection{Experimental setup}

\paragraph{Large Language Model} In our main experiments, we employ \flan{}~\citep{flan}, particularly \flan{}-large, as the base LLM. 
The model has shown impressive abilities to perform zero-shot and few-shot learning.

\paragraph{Candidate LoRA Modules} Our methodology requires a compendium of LoRA modules trained on preceding tasks. For parity with FLAN, we adopt the tasks utilized to instruct \flan{}, thereby incorporating nearly $200$ distinct tasks and their corresponding instructions~\footnote{We released used the dataset at \href{https://huggingface.co/datasets/lorahub/flanv2}{\texttt{huggingface.co/datasets/lorahub/flanv2}}.}. Following this, we trained several LoRA modules as potential candidates.
During each experimental sequence, we randomly select $20$ LoRA modules from them as the candidate for our \lorahub{} learning. 

\paragraph{Dataset and evaluation}

Our method is evaluated using the Big-Bench Hard (BBH) benchmark, a well-established standard that consists of multiple-choice questions from a variety of domains.
The benchmark consists of $27$ different tasks, which are regarded to be challenging for language models.
For all tasks, we employ the exact match (EM) as our evaluation metric.

\begin{table}[t]
\centering
\small

\caption{Experimental results of zero-shot learning (Zero), few-shot in-context learning (ICL), IA3 fine-tuning (IA3), LoRA tuning (LoRA), full fine-tuning (FFT) and our proposed few-shot \lorahub{} learning (\lorahub{}) on the BBH benchmark with \flan{}-large as the base LLM. We denote algorithmic tasks with the superscript $\S$ following previous work~\citep{DBLP:journals/corr/abs-2303-17564}. Note that we employ three runs, each leveraging different $5$-shot examples per task, as demonstrations for all few-shot methods. The average performance of all methods is reported below, and the best performance of each few-shot method can be found in the Appendix~\ref{sec:maximum_appendix}.}
\label{tab:performance}
\begin{tabular}{lccccccc}
\toprule
Task & Zero & ICL$_{\rm avg}$ & IA3$_{\rm avg}$ & LoRA$_{\rm avg}$ & FFT$_{\rm avg}$ & \lorahub{}$_{\rm avg}$ \\
\midrule
Boolean Expressions & 54.0 & 59.6 & 56.2 & 56.0 & 62.2 & 55.5 \\
Causal Judgement & 57.5 & 59.4 & 60.2 & 55.6 & 57.5 & 54.3 \\
Date Understanding & 15.3 & 20.4 & 20.0 & 35.8 & 59.3 & 32.9 \\
Disambiguation & 0.0 & 69.1 & 0.0 & 68.0 & 68.2 & 45.2 \\
Dyck Languages & 1.3 & 0.9 & 4.2 & 22.2 & 19.5 & 1.0 \\
Formal Fallacies & 51.3 & 55.3 & 51.5 & 53.6 & 54.0 & 52.8 \\
Geometric Shapes & 6.7 & 19.6 & 14.7 & 24 & 31.1 & 7.4 \\
Hyperbaton & 6.7 & 71.8 & 49.3 & 55.3 & 77.3 & 62.8 \\
\begin{tabular}[c]{@{}l@{}}Logical Deduction$^\S$ \\ {\small ~~~~~~~~~~~~~(five objects)}\end{tabular}  & 21.3 & 39.1 & 32.7 & 40.0 & 42.2 & 36.1 \\
\begin{tabular}[c]{@{}l@{}}Logical Deduction$^\S$ \\ {\small ~~~~~~~~~~~~~(seven objects)}\end{tabular} & 12.7 & 40.7 & 33.8 & 37.3 & 44.9 & 36.8 \\
\begin{tabular}[c]{@{}l@{}}Logical Deduction$^\S$ \\ {\small ~~~~~~~~~~~~~(three objects)}\end{tabular} & 0.0 & 51.6 & 8.5 & 53.6 & 52.9 & 45.7 \\
Movie Recommendation & 62.7 & 55.8 & 61.8 & 51.5 & 66.0 & 55.3 \\
Multistep Arithmetic & 0.7 & 0.7 & 0.7 & 0.2 & 0.0 & 0.4 \\
Navigate & 47.3 & 45.3 & 46.2 & 48.0 & 48.0 & 47.1 \\
Object Counting & 34.7 & 32.4 & 35.1 & 38.7 & 35.6 & 33.7 \\
Penguins in a Table & 43.5 & 41.3 & 45.0 & 36.2 & 31.9 & 35.9 \\
Reasoning about Colored Objects & 32.0 & 40.2 & 40.7 & 39.6 & 37.6 & 40.0 \\
Ruin Names & 23.3 & 19.3 & 24.4 & 37.8 & 61.3 & 24.4 \\
Salient Translation Error Detection & 37.3 & 47.3 & 37.1 & 16.0 & 16.2 & 36.0 \\
Snarks & 50.0 & 54.2 & 53.9 & 55.6 & 66.7 & 56.9 \\
Sports Understanding & 56.0 & 54.7 & 55.1 & 56.5 & 54.0 & 56.7 \\
Temporal Sequences & 16.7 & 25.1 & 18.2 & 25.1 & 37.8 & 18.2 \\
\begin{tabular}[c]{@{}l@{}}Tracking Shuffled Objects$^\S$ \\ {\small ~~~~~~~~~~~~~\quad\quad\quad(five objects)}\end{tabular} & 12.0 & 12.0 & 12.0 & 13.8 & 16.9 & 12.3 \\
\begin{tabular}[c]{@{}l@{}}Tracking Shuffled Objects$^\S$ \\ {\small ~~~~~~~~~~~~~\quad\quad\quad(seven objects)}\end{tabular} & 6.7 & 6.7 & 6.7 & 10.0 & 9.8 & 7.7 \\
\begin{tabular}[c]{@{}l@{}}Tracking Shuffled Objects$^\S$ \\ {\small ~~~~~~~~~~~~~\quad\quad\quad(three objects)}\end{tabular} & 24.7 & 31.1 & 30.7 & 30.9 & 32.0 & 29.2 \\
Web of Lies & 54.0 & 53.8 & 54.2 & 52.7 & 48.2 & 50.1 \\
Word Sorting & 1.3 & 0.5 & 1.3 & 4.9 & 4.9 & 1.1 \\
\midrule
Avg Performance Per Task & 27.0 & 37.3 & 31.6 & 37.7 & 42.1 & 34.7 \\
Avg Tokens Per Example & 111.6 & 597.8 & 111.6 & 111.6 & 111.6 & 111.6 \\
Gradient-based Training & No & No & Yes & Yes & Yes & No \\
\bottomrule
\end{tabular}
\end{table}


\paragraph{Baseline Setup}

To enhance the demonstration of our method's performance, we expanded our comparisons beyond the zero-shot and in-context learning settings. We specifically chose three representative gradient-based methods for comparison: full fine-tuning (FFT), LoRA tuning (LoRA)~\citep{hu2022lora}, and IA3 fine-tuning (IA3)~\citep{Liu2022FewShotPF}.
For all gradient-based methods, for a fair comparsion, we train for $40$ epochs on the same three runs of $5$ examples employed in our methods.
In the case of FFT, a learning rate of 3e-5 is employed, whereas for IA3 and LoRA, we adopt a learning rate of 2e-4.
We report the performance of each method on the test set at the end of training (averaged over three runs) without any model selection to avoid potential selection bias.

\subsection{Main results} 

As shown in Table~\ref{tab:performance}, our experimental results demonstarte the superior efficacy of our method in comparison to zero-shot learning while closely resembling the performance of in-context learning (ICL) in few-shot scenarios. This observation is derived from an average performance of three runs, each leveraging different few-shot examples.
Importantly, our model utilizes an equivalent number of tokens as the zero-shot method, notably fewer than the count used by ICL. 
Although occasional performance fluctuations, our method consistently outperforms zero-shot learning in most tasks.
In the era of LLMs, the input length is directly proportional to the inference cost, and thus \lorahub's ability to economize on input tokens while approaching the peak performance grows increasingly significant.
Moreover, as shown in Appendix Table~\ref{tab:max_perf}, the upper bound performance of our method across these runs can surpass ICL on $18$ tasks, demonstrating its potential for future development.

Even when compared to certain gradient-based optimization methods, our approach consistently demonstrates competitive performance. For example, as depicted in Table~\ref{tab:performance}, our method exhibits a notable improvement of $3.1\%$ on average in contrast to the promising IA3 method. Nevertheless, we acknowledge that our approach still falls behind LoRA tuning and full fine-tuning, especially in tasks that exhibit significant deviation from the upstream task. Taking Dyck Languages as an example, both \lorahub and ICL achieve only an average performance of nearly $1.0\%$ on these tasks, while LoRA and FFT methods showcase impressive results with only $5$ examples.

\subsection{Discussion}

LoraHub addresses the challenge of reducing inference costs by eliminating the need for processing additional tokens, resulting in a noticeable reduction in overall inference expenses. However, it introduces an inherent cost during the \textsc{Adapt} stage, necessitating extra inference steps, such as the $40$ steps employed in our experiments. This introduces a trade-off between choosing the ICL approach and LoraHub, with the decision typically hinging on the nature of the situation.

For one-time ad-hoc tasks, the ICL approach should be more pragmatic due to LoraHub's additional inference step costs. In such scenarios, where immediate, single-use solutions are preferred, the simplicity and efficiency of ICL might outweigh the benefits of potential savings offered by LoraHub. Conversely, for recurring or similar tasks, LoraHub emerges as a compelling option. Despite the added inference step cost, LoraHub's ability to efficiently handle repetitive tasks, often occurring thousands of times, while concurrently reducing overall expenses, positions it as a viable option in such kind of situations.

In summary, our intention is not to replace ICL, but to present LoraHub as a complementary strategy with performance-efficiency trade-offs. Thus, we encourage a careful consideration of specific use cases and requirements when choosing between ICL and LoraHub, recognizing that the optimal solution may vary based on the nature and frequency of the tasks at hand.



\section{Analysis}
\section{Analysis Results}
\label{sec:experiment}
%To develop a more comprehensive understanding of Python security commits, we conduct a series of analysis: (1) Dataset Characteristics, (2) Security Commits Categories to reveal what types of vulnerabilities have been fixed, (3) Security Commits Distribution over Repositories, (4) Security Commits Complexity, (5) Security Commits Locality and (6) Fix Patterns Summarization to assist in automated program repair projects. 

After constructing our datasets, we frame our evaluation into four research questions, as outlined below. 
\begin{itemize}[leftmargin=*]

\item \textbf{RQ1:} Can the graph learning-based method help improve the data collection efficiency?
\item \textbf{RQ2:} How various and representative are the collected security commits? 
\item \textbf{RQ3:} What are the unique patterns of security commits in Python? 
\item \textbf{RQ4:} How do the wild commit samples help improve \gnn{} model for downstream security commit detection? 
\end{itemize}

\subsection{Dataset Construction (RQ1)}\label{results:efficiency}

After keyword filtering and graph-based identification with humans in the loop, we collect 1,258 security commits in total. Specifically, as shown in Table~\ref{tab: dataset}, there are 729, 400, and 129 security commits in the base, pilot, and augmented datasets, respectively. Also, 2,791 non-security commits are manually labeled during the collection process.

% \begin{table}[h]
% \centering
%     \caption{The statistical information of \db{}.}
%     \setlength{\tabcolsep}{3.4mm}{
%     \begin{tabular}{c|c|c|c|c}
%     \toprule
%     {} & \multirow{2}{*}{\shortstack{\bf Base\\\bf Dataset}} & \multirow{2}{*}{\shortstack{\bf Pilot\\\bf Dataset}} & \multirow{2}{*}{\shortstack{\bf Augmented\\\bf Dataset}} & \multirow{2}{*}{\bf Total} \\
%     {} & {} & {} & {} & {} \\
%      % & \textbf{Base} & \textbf{Pilot} & \textbf{Augmented} & \multirow{2}{*}{\textbf{Total}} \\
%      % & \textbf{Dataset} & \textbf{Dataset} & \textbf{Dataset} & \\
%      % & Base Dataset & Pilot Dataset & Augmented Dataset & Total \\
%                          % & \begin{tabular}[c]{@{}c@{}}Base \\ Dataset\end{tabular} & \begin{tabular}[c]{@{}c@{}}Pilot \\ Dataset\end{tabular} & \begin{tabular}[c]{@{}c@{}}Augmented \\ Dataset\end{tabular}  & Total \\ \hline
%     \midrule
%     \multirow{2}{*}{\shortstack{\bf Security\\\bf Commits}} & \multirow{2}{*}{729} & \multirow{2}{*}{400} & \multirow{2}{*}{129} & \multirow{2}{*}{1258} \\
%     {} & {} & {} & {} & {} \\
%     \midrule
%     \multirow{2}{*}{\shortstack{\bf Non-security\\\bf Commits}} & \multirow{2}{*}{2134} & \multirow{2}{*}{535} & \multirow{2}{*}{122} & \multirow{2}{*}{2791} \\
%     {} & {} & {} & {} & {} \\
%     \bottomrule
%     \end{tabular}
%     }
%     \label{tab: dataset}
% \end{table}

\begin{table}[h]
\vspace{-0.05in}
\centering
    \caption{The composition of \db{}.}
    \setlength{\tabcolsep}{1.7mm}{
    \begin{tabular}{c|p{1.0cm}<{\centering}|p{1.0cm}<{\centering}|p{1.4cm}<{\centering}|p{1.0cm}<{\centering}}
    \toprule
    {\diagbox{\bf Commit}{\bf Dataset}} & {\bf Base} & {\bf Pilot} & {\bf Augmented} & {\bf Total} \\
     % & \textbf{Base} & \textbf{Pilot} & \textbf{Augmented} & \multirow{2}{*}{\textbf{Total}} \\
     % & \textbf{Dataset} & \textbf{Dataset} & \textbf{Dataset} & \\
     % & Base Dataset & Pilot Dataset & Augmented Dataset & Total \\
                         % & \begin{tabular}[c]{@{}c@{}}Base \\ Dataset\end{tabular} & \begin{tabular}[c]{@{}c@{}}Pilot \\ Dataset\end{tabular} & \begin{tabular}[c]{@{}c@{}}Augmented \\ Dataset\end{tabular}  & Total \\ \hline
    \midrule
    {\bf Security} & {729} &  {400} &  {129} & {1258} \\
    \midrule
    {\bf Non-Security} & {2134} & {535} & {122} &{2791} \\
    \bottomrule
    \end{tabular}
    }
    \label{tab: dataset}
\vspace{-0.05in}
\end{table}

Table~\ref{tab:spr} lists the augmentation efficiency of random selection, keyword filtering, and \gnn{}.
Compared with identifying security commits from scratch, the keyword filtering mechanism improves the collecting efficiency by over 30 percentage points and \gnn{} improves the efficiency by 40 percentage points. %It is to be noted that we only test our data augmentation methods on a small portion of commits, which has already shown effectiveness.



\begin{table}[ht]
\vspace{-0.05in}
\centering
\caption{Efficiency of keyword filtering and \gnn{}.}
\setlength{\tabcolsep}{4mm}{
\begin{tabular}{c|c|c|c}
\toprule
% \multirow{2}{*}{\textbf{Methods}} & \multirow{2}{*}{\textbf{Candidates}} & \multirow{2}{*}{\shortstack{\bf Verified\\ \bf Security Commits}} & \multirow{2}{*}{\textbf{Ratio}} \\
% {} & {} & {} & {} \\
%  \midrule
\textbf{Method}      & \textbf{\# Candidates} & \textbf{\# Verified SC$^{*}$} & \textbf{Ratio} \\
 \midrule
% Methods      & Candidates & \begin{tabular}[c]{@{}c@{}}Verified \\ Security Commits\end{tabular} & Ratio   \\ \hline
{Random~\cite{wang2021patchdb}} & {-} & {-} & {6-10\%}    \\ 
\midrule
{Keywords} & {935} & {400} & {42.70\%} \\ 
\midrule
{\gnn{}} & {251} & {129} & {51.39\%} \\ 
\bottomrule

\end{tabular}
}
\begin{tablenotes}[flushleft]
    \footnotesize
    \item $^{*}$ SC = Security Commits.
\end{tablenotes}
% \vspace{-0.1in}
\label{tab:spr}
\vspace{-0.15in}
\end{table}

\begin{table}[]
\centering
\caption{Top 5 repositories by number of security commits.}
\label{tab:repo}
\setlength{\tabcolsep}{5.4mm}{
\begin{tabular}{c|c|c}
\toprule
\textbf{Repository} & \textbf{\#SecurityCommits} & \textbf{\textbf{Proportion}} \\
\midrule
django      & 166  & 13.20\%   \\ \midrule
twisted     & 87   & 6.91\%   \\ \midrule
glance      & 54   & 4.29\%     \\ \midrule
pillow     & 41   & 3.26\%     \\ \midrule
numpy       & 39   & 3.10\%        \\ \midrule
\rowcolor{gray!10}\textbf{Total of Top 5}                   & \textbf{387}   &   \textbf{30.76\%} \\
\bottomrule
\end{tabular}
}
\vspace{-0.1in}
\end{table}


\subsection{Security Commits Categorization and Distribution (RQ2)}
% \XD{should show broad coverage and variety of DB}}

%The Common Weakness Enumeration (CWE)
NVD CWE slice~\cite{CWE_slice} associated classification taxonomy serves to identify and describe security vulnerabilities.
% in terms of CWEs. 
To understand the purpose of these commits, we investigate the CWE types associated with the CVE reports and plot the distribution of the CWE types that have been explicitly documented. Among the 729 security commits linked to 556 CVEs, due to the limited number of MITRE human analysts, only 312 (56.1\%) CVEs have been assigned CWEs. %Since the CWE taxonomy is a hierarchical structure, a CVE can be assigned with more than one CWE. 
Even so, there are already 119 distinct CWEs associated with our security commits in the base dataset, which means our \db{} contains at least 119 types of security commits in terms of corresponding vulnerabilities.
Figure~\ref{fig:cwe} enumerates the most common CWEs, including frequent security problems such as cross-site scripting (CWE-79), path traversal (CWE-22), etc. Note that we do not directly assign CWE type to security samples in the remaining base, pilot, and augmented dataset since the MITRE CWE team has its own internal process. However, based on our observation and our data collection approaches that are able to introduce wild security commits with more variance (as discussed in~\ref{exp:variance}), \db{} can encompass a broad range of security concerns with various kinds of security commits, including but not limited to above 119 CWEs. 

% % Figure environment removed

% Figure environment removed


Our collected security commits distribute among 351 popular GitHub repositories unevenly. Among them, 69 repositories provide more than two security commits, bringing a certain amount of variety. In Table~\ref{tab:repo}, the top five repositories that have the most occurrence in our dataset are django~\cite{django_2023}, twisted~\cite{twisted_2023}, glance~\cite{openstack_2023}, pillow~\cite{python-pillow_2023}, and numpy~\cite{numpy_2023}, implying that the samples in \db{} align with the popularity trend of security issue in practice.
%react to security issues on time.




% \subsection{Security Commits Complexity}

% \subsection{Security Commits Locality}

\subsection{Patch Patterns (RQ3)}
\label{rq3}

We manually go through the whole \db{} dataset, 
% we manually explore the full security commit dataset. 
% samples that have less than 200 code lines.
% We find 1,027 samples in total, which take up 81.7\% of the security commits.
% After the comprehensive analysis, 
and discover four common security fix patterns (taking up 85.85\% of all security commit samples) that may benefit software maintenance, i.e., adding or updating sanity checks,  updating APIs, updating regular expressions, and updating security properties, as listed in Table~\ref{tab:pattern}.


\begin{table}[]
\centering
%\scriptsize
\caption{The pattern types of security commits in \db{}.}
\label{tab:pattern}
\setlength{\tabcolsep}{4.0mm}{
\begin{tabular}{l|c|c}
\toprule
    \textbf{Pattern} & \textbf{\#Commits} & \textbf{Proportion} \\ 
    \midrule
    {1) Add or Update Sanity Checks} & {416} & {37.12\%} \\ 
    \midrule
    {2) Update API Usage} & {241} & {19.16\%} \\ 
    \midrule
    {3) Update Regular Expressions} & {189} & {15.02\%} \\ 
    \midrule
    {4) Restrict Security Properties} & {183} & {14.55\%} \\ 
    \midrule
    {5) Others} & {178} & {14.15\%} \\ 
    \midrule 
    \rowcolor{gray!10}{\bf Total} & {\bf 1258} & {\bf 100\%} \\
    \bottomrule
\end{tabular}
}
\vspace{-0.1in}
\end{table}


%re 168 + 21
%api 214 +27
%if 420 + 47
%property 167 +16
%others 159 + 19

%re 151
%api 190
%if 369
%property 146
%others 43


\subsubsection{Add or Update Sanity Checks}
A sanity check is a basic method to quickly evaluate if a claim or a calculation result can be true, which has been extensively applied to multiple scenarios, e.g., authentication property verification, access control, HTTP request checking~\cite{wang2020machine}. 
We summarize three representative patterns that fix the vulnerabilities via adding or updating sanity checks, which are presented by 37.12\% of security commits in \db{}.

\noindent{\bf Authentication.} Authentication is the act of proving an assertion, e.g., we need to compare the identity with the system data to verify a system user. The authentication-related vulnerabilities
% occurs when the authentication is performed improperly, which 
provide attackers the opportunities to masquerade as legitimate users. To defend them, an effective solution is to perform the additional authentication by adding more check requirements or making existing conditions more restrictive. List~\ref{lst:auth} presents an example of fixing an authentication vulnerability by narrowing down an existing restriction from \texttt{\small True} (i.e., all possible return values except \texttt{\small False}) to \texttt{\small "on"} only.


\lstdefinestyle{lst}{
    float=th,
    floatplacement=tbp,
    % abovecaptionskip=0.01in,
    numbers=left, 
    numberstyle=\scriptsize, 
    numbersep = 5pt,
    framexleftmargin = 0in,
    framexrightmargin = 0in,
    breaklines = true,
    xleftmargin = 0.18in,
    xrightmargin = 0.1in,
    basicstyle=\ttfamily\scriptsize, 
    frame=lines,
    showtabs=true,
    showspaces=true,
    showstringspaces=false,
    literate={\ }{{\ }}1,
    aboveskip=-0.00in,
    belowskip=-0.15in,
}

\begin{lstlisting}[
language=diff, 
style=lst,
caption=An example of security commit to fix authentication vulnerability (CVE-2022-0273).,
label={lst:auth},
mathescape=true
]
 $\textbf{commit 0c0313f375bed7b035c8c0482bbb09599e16bfcf}$ 
 diff --git a/cps/shelf.py b/cps/shelf.py
 @@ -248,7 +248,7 @@ def create_edit_shelf(shelf,
 ...
         $\textbf{return}$ redirect(url_for('web.index'))
-    is_public = 1 if to_save.get("is_public") else 0
+    is_public = 1 if to_save.get("is_public") == "on" else 0
     $\textbf{if}$ config.config_kobo_sync:
 ...
\end{lstlisting}

\noindent{\bf Authorization.} Authorization refers to the process of granting or denying access to certain data or actions within a system.
Authorization comes after authentication and is achieved by an access control list (ACL).
The ACL is used to check the user identity with a list of authorized operations and determine which actions a user is allowed to take, e.g., file and data permission.
% determining  to perform and which are restricted, including but not limited to file permission and data permission. 
Unrestricted authorization may lead to improper resource consumption since attackers could bypass the system to access high-security level data. List~\ref{lst:access control1} is an example that fixes an authorization bypass exploit by requiring the value of \texttt{\small os.environ.get('GITHUB\_ACTIONS')} to be \texttt{\small true}.

\lstdefinestyle{lst}{
    float=th,
    floatplacement=tbp,
    % abovecaptionskip=0.01in,
    numbers=left, 
    numberstyle=\scriptsize, 
    numbersep = 5pt,
    framexleftmargin = 0in,
    framexrightmargin = 0in,
    breaklines = true,
    xleftmargin = 0.18in,
    xrightmargin = 0.1in,
    basicstyle=\ttfamily\scriptsize, 
    frame=lines,
    showtabs=true,
    showspaces=true,
    showstringspaces=false,
    literate={\ }{{\ }}1,
    aboveskip=+0.10in,
    belowskip=-0.30in,
}

\begin{lstlisting}[
language=diff, 
style=lst,
caption=An example of security commit that fixes an authorization bypass exploit vulnerability (CVE-2022-46179).,
label={lst:access control1},
mathescape=true
]
 $\textbf{commit c658b4f3e57258acf5f6207a90c2f2169698ae22}$  
 diff --git a/core.py b/core.py
 @@ -112,7 +112,7 @@ def actualsys() :
     $\textbf{if}$ attemps == 6:
         ## Brute force protection
         $\textbf{raise}$ Exception("Too many password attempts.")
-    if os.environ.get('GITHUB_ACTIONS') != "":
+    if os.environ.get('GITHUB_ACTIONS') == "true":
         logging.warning("Running on Github Actions")
         actualsys()
     $\textbf{elif}$ uname == cred.name and pwdhash == cred.pass:
\end{lstlisting}


\noindent{\bf HTTP Request.} If the interpretation of Content-Length and/or Transfer-Encoding headers between HTTP servers are inconsistent, the attackers may take advantage of this issue and send malicious requests to the servers, i.e., HTTP request smuggling. 
A good solution is to maintain the same interpretation methods in both front-end and back-end servers. 
In this way, an effective coding practice is to add consistent sanity checks on request interpretation for both servers. 
List~\ref{lst:http} adds such a sanity check on \texttt{\small data} to determine if all characters are digits.% to avoid such exploits.

\lstdefinestyle{lst}{
    float=th,
    floatplacement=tbp,
    % abovecaptionskip=0.01in,
    numbers=left, 
    numberstyle=\scriptsize, 
    numbersep = 5pt,
    framexleftmargin = 0in,
    framexrightmargin = 0in,
    breaklines = true,
    xleftmargin = 0.18in,
    xrightmargin = 0.1in,
    basicstyle=\ttfamily\scriptsize, 
    frame=lines,
    showtabs=true,
    showspaces=true,
    showstringspaces=false,
    literate={\ }{{\ }}1,
    aboveskip=-0.00in,
    belowskip=-0.15in,
}

\begin{lstlisting}[
language=diff, 
style=lst,
caption=An example of security commit that fixes an HTTP request smuggling vulnerability (CVE-2022-24801).,
label={lst:http},
mathescape=true
]
 $\textbf{commit 8ebfa8f6577431226e109ff98ba48f5152a2c416}$ 
 diff --git a/src/twisted/web/http.py b/src/twisted/web/http.py
 @@ -2274,6 +2274,8 @@ def fail():
     $\textbf{if}$ header == b"content-length":
+        if not data.isdigit():
+            return fail()
         $\textbf{try}$:
             length = int(data)
         $\textbf{except}$ ValueError:
\end{lstlisting}


\subsubsection{Update API Usage}% Packages}
Compared with implementing the fixes from scratch, there are abundant well-formulated packages that can be adopted to realize the intended functionalities and help enforce security restrictions. 
We notice that a large number (19.16\%) of Python security commits fix vulnerabilities by imposing or substituting APIs. %packages, which is different from other languages like C/C++.
%This pattern differs from the fix patterns in other languages, e.g., C/C++.
We further categorize such security fixes % into different types 
according to their application scenarios. %scopes.

\noindent{\bf General Purpose.} There is a set of security-related modifications on built-in packages shared by applications for various purposes. For instance, \texttt{\small re.escape} is an API to escape non-alphanumerics that are not part of regular expression syntax, to avoid OS command injection, code injection, and regular expression injection. List~\ref{lst:re} is a commit example to fix regular expression injection vulnerability, which demonstrates the application of \texttt{\small re.escape} on \texttt{\small user} and \texttt{\small collection\_url}.

\lstdefinestyle{lst}{
    float=th,
    floatplacement=tbp,
    % abovecaptionskip=0.01in,
    numbers=left, 
    numberstyle=\scriptsize, 
    numbersep = 5pt,
    framexleftmargin = 0in,
    framexrightmargin = 0in,
    breaklines = true,
    xleftmargin = 0.18in,
    xrightmargin = 0.1in,
    basicstyle=\ttfamily\scriptsize, 
    frame=lines,
    showtabs=true,
    showspaces=true,
    showstringspaces=false,
    literate={\ }{{\ }}1,
    aboveskip=-0.00in,
    belowskip=-0.15in,
}

\begin{lstlisting}[
language=diff, 
style=lst,
caption=An example of security commit that fixes a regular expression injection vulnerability (CVE-2015-8748).,
label={lst:re},
mathescape=true
]
 $\textbf{commit 4bfe7c9f7991d534c8b9fbe153af9d341f925f98}$ 
 diff --git a/radicale/rights/regex.py b/radicale/rights/regex.py
 @@ -65,7 +65,10 @@ def _read_from_sections(user, collection_url, permission):
 ...
-    regex = ConfigParser({"login": user, "path": collection_url})
+    # Prevent "regex injection"
+    user_escaped = re.escape(user)
+    collection_url_escaped = re.escape(collection_url)
+    regex = ConfigParser({"login": user_escaped, "path": collection_url_escaped})
 ...
\end{lstlisting}

\noindent{\bf Web Applications.} To properly process the inputs of web applications, security commits can adopt %the existing APIs %, e.g., \texttt{\small escape\_html}, 
APIs in third-party packages for Python
(e.g., \texttt{\small parser.quote}, \texttt{\small request.server.escape}, \texttt{\small django.utils.html.escape}, and \texttt{\small html.unescape}) to escape ampersands, brackets, and quotes to the HTML/XML entities or HTTP requests for defeating cross-site scripting (XSS) and HTTP Smuggling. 
List~\ref{lst:xss2} is an example that fixes an XSS vulnerability by using the API \texttt{\small django.utils.html.escape}.

\lstdefinestyle{lst}{
    float=th,
    floatplacement=tbp,
    % abovecaptionskip=0.01in,
    numbers=left, 
    numberstyle=\scriptsize, 
    numbersep = 5pt,
    framexleftmargin = 0in,
    framexrightmargin = 0in,
    breaklines = true,
    xleftmargin = 0.18in,
    xrightmargin = 0.1in,
    basicstyle=\ttfamily\scriptsize, 
    frame=lines,
    showtabs=true,
    showspaces=true,
    showstringspaces=false,
    literate={\ }{{\ }}1,
    aboveskip=+0.00in,
    belowskip=-0.15in,
}

\begin{lstlisting}[
language=diff, 
style=lst,
caption=An example of security commit that fixes an XSS vulnerability (CVE-2022-24710).,
label={lst:xss2},
mathescape=true
]
 $\textbf{commit f6753a1a1c63fade6ad418fbda827c6750ab0bda }$
 diff --git a/weblate/trans/forms.py b/weblate/trans/forms.py
 @@ -37,6 +37,7 @@
 ...
+from django.utils.html import escape
 ...
-    label = str(unit.translation.language)
+    label = escape(unit.translation.language)
 ...
\end{lstlisting}


\noindent{\bf Shell Commands.} To handle the shell commands securely, security fixes can adopt \texttt{\small shlex.quote} and \texttt{\small subprocess} to load or execute the commands. 
With the \texttt{\small shlex.quote} API, we can have an escaped version of shell inputs, which can be safely used as tokens in a command line to avoid shell command injection.
List~\ref{lst:shell} is an example that shows the usage of \texttt{\small shlex.quote} to fix a shell injection vulnerability. 

\lstdefinestyle{lst}{
    float=th,
    floatplacement=tbp,
    % abovecaptionskip=0.01in,
    numbers=left, 
    numberstyle=\scriptsize, 
    numbersep = 5pt,
    framexleftmargin = 0in,
    framexrightmargin = 0in,
    breaklines = true,
    xleftmargin = 0.18in,
    xrightmargin = 0.1in,
    basicstyle=\ttfamily\scriptsize, 
    frame=lines,
    showtabs=true,
    showspaces=true,
    showstringspaces=false,
    literate={\ }{{\ }}1,
    aboveskip=-0.00in,
    belowskip=-0.15in,
}

\begin{lstlisting}[
language=diff, 
style=lst,
caption=An example of security commit that fixes a shell injection vulnerability (CVE-2013-7416).,
label={lst:shell},
mathescape=true
]
 $\textbf{commit 2817869f98c54975f31e2dd674c1aefa70749cca }$
 diff --git a/canto_curses/guibase.py b/canto_curses/guibase.py
 @@ -156,6 +156,11 @@ def _fork(self, path, href, text, fetch=False):
 ...
+    href = shlex.quote(href)
 ...
\end{lstlisting}


\noindent{\bf Path Name.} 
If a path name is improperly neutralized, attackers may access the files and directories outside of the restricted location. 
This vulnerability can occur by using absolute file paths or manipulating the path variables where the reference files contain ``dot-dot-slash (../)" sequences or variations.
To effectively escape such unsafe sequences, Python security commits usually adopt the secure APIs, e.g., \texttt{\small werkzeug.utils.safe\_join}, \texttt{\small yaml.safe\_load}, and \texttt{\small werkzeug.utils.secure\_filename}, to prevent the files or directories from being accessed by malicious users. 
List~\ref{lst:path traversal} is a commit example that fixes a path traversal via using the API \texttt{\small werkzeug.utils.secure\_filename}.

\lstdefinestyle{lst}{
    float=th,
    floatplacement=tbp,
    % abovecaptionskip=0.01in,
    numbers=left, 
    numberstyle=\scriptsize, 
    numbersep = 5pt,
    framexleftmargin = 0in,
    framexrightmargin = 0in,
    breaklines = true,
    xleftmargin = 0.18in,
    xrightmargin = 0.1in,
    basicstyle=\ttfamily\scriptsize, 
    frame=lines,
    showtabs=true,
    showspaces=true,
    showstringspaces=false,
    literate={\ }{{\ }}1,
    aboveskip=+0.10in,
    belowskip=-0.25in,
}

\begin{lstlisting}[
language=diff, 
style=lst,
caption=An example of security commit that fixes a path traversal vulnerability (CVE-2022-23609).,
label={lst:path traversal},
mathescape=true
]
 $\textbf{commit 1eb1e5428f0926b2829a0bbbb65b0d946e608593}$ 
 diff --git a/upload/server.py b/upload/server.py
 @@ -5,7 +5,7 @@
-
+import werkzeug.utils
 @@ -189,7 +189,7 @@ def uploadimage():
     filename = all_files[0][1] + all_files[0][2]
-    remove(filename)
+    remove(werkzeug.utils.secure_filename(filename))
     $\textbf{del}$ all_files[0]
     length = len(all_files)
\end{lstlisting}


\subsubsection{Update Regular Expressions}
Python has become a popular choice for back-end web development, and it is usually combined with some other front-end languages~\cite{python_app}. For this reason, we observe there are 15.02\% fixes that modify the regular expressions to avoid XSS, SQL injection, and open redirect vulnerabilities. 
% Python inserts itself in web development as a back-end language, and it is usually combined with some other front-end language (e.g., javascript) to build a whole website.
% We observe 15.02\% 
The regular expression patterns are tailored to match specific strings within the given text, including SQL commands, URLs, and other scripts.

\noindent{\bf SQL Commands.} The improper neutralization of SQL commands may lead to SQL injection vulnerabilities, which allow attackers to manipulate the backend database and access the information not intended to be displayed.
The corresponding fixes need to escape the unsafe characters. 
List~\ref{lst:sql} is a fixed example of SQL injection vulnerability, which substitutes the matched single and double quote characters (i.e., \texttt{\small '} and \texttt{\small "}) in the string \texttt{\small self.queueid}.

\lstdefinestyle{lst}{
    float=th,
    floatplacement=tbp,
    % abovecaptionskip=0.01in,
    numbers=left, 
    numberstyle=\scriptsize, 
    numbersep = 5pt,
    framexleftmargin = 0in,
    framexrightmargin = 0in,
    breaklines = true,
    xleftmargin = 0.18in,
    xrightmargin = 0.1in,
    basicstyle=\ttfamily\scriptsize, 
    frame=lines,
    showtabs=true,
    showspaces=true,
    showstringspaces=false,
    literate={\ }{{\ }}1,
    aboveskip=+0.0in,
    belowskip=-0.15in,
}

\begin{lstlisting}[
language=diff, 
style=lst,
caption=An example of security commit that fixes a SQL injection vulnerability (CVE-2014-125082).,
label={lst:sql},
mathescape=true
]
 $\textbf{commit fc2c1ea1b8d795094abb15ac73cab90830534e04}$
 diff --git a/.../model.py b/.../model.py
 @@ -772,13 +772,13 @@ def _get_filter(self):
 $\textbf{if}$ self.queueid:
-    ... = '%s'" % (self.queueid)
+    ... = '%s'" % (re.sub("[\"']", "", self.queueid))
\end{lstlisting}


\noindent{\bf URLs.} The improper neutralization of URLs may lead to open redirect vulnerability, which redirects an unsuspecting victim from a legitimate domain to an attacker’s phishing site. 
Effective mitigation is to replace the dangerous special characters with trusted symbols. List~\ref{lst:redirect} is an example of an open redirect vulnerability, which replaces the explicit backslash with an encoded backslash to circumvent the dangerous redirect.

\lstdefinestyle{lst}{
    float=th,
    floatplacement=tbp,
    % abovecaptionskip=0.01in,
    numbers=left, 
    numberstyle=\scriptsize, 
    numbersep = 5pt,
    framexleftmargin = 0in,
    framexrightmargin = 0in,
    breaklines = true,
    xleftmargin = 0.18in,
    xrightmargin = 0.1in,
    basicstyle=\ttfamily\scriptsize, 
    frame=lines,
    showtabs=true,
    showspaces=true,
    showstringspaces=false,
    literate={\ }{{\ }}1,
    aboveskip=-0.00in,
    belowskip=-0.15in,
}

\begin{lstlisting}[
language=diff, 
style=lst,
caption=An example of security commit that fixes an open redirect vulnerability (CVE-2019-10255).,
label={lst:redirect},
mathescape=true
]
 $\textbf{commit 08c4c898182edbe97aadef1815cce50448f975cb}$ 
 diff --git a/auth/login.py b/auth/login.py
 @@ -39,6 +39,10 @@ def _redirect_safe(self, url, ...):
+    url = url.replace("\\", "%5C")
     parsed = urlparse(url)
     $\textbf{if}$ parsed.netloc $\textbf{or not}$ (parsed.path + '/').startswith(self.base_url):
\end{lstlisting}

\noindent{\bf Scripts.} The improper input validation and encoding during web page generation may lead to XSS, which is able to reveal the cookies, session tokens, or other sensitive information retained by the browser to the attackers. A straightforward solution is to validate the matched characters of a pre-defined pattern. List~\ref{lst:xss} is an example to fix the XSS vulnerability by re-matching the characters between parentheses instead of the characters between square brackets and validating the matched pattern one by one.

\lstdefinestyle{lst}{
    float=th,
    floatplacement=tbp,
    % abovecaptionskip=0.01in,
    numbers=left, 
    numberstyle=\scriptsize, 
    numbersep = 5pt,
    framexleftmargin = 0in,
    framexrightmargin = 0in,
    breaklines = true,
    xleftmargin = 0.18in,
    xrightmargin = 0.1in,
    basicstyle=\ttfamily\scriptsize, 
    frame=lines,
    showtabs=true,
    showspaces=true,
    showstringspaces=false,
    literate={\ }{{\ }}1,
    aboveskip=+0.10in,
    belowskip=-0.25in,
}

\begin{lstlisting}[
language=diff, 
style=lst,
caption=An example of security commit that fixes an XSS vulnerability (CVE-2021-3994).,
label={lst:xss},
mathescape=true
]
 $\textbf{commit a22eb0673fe0b7784f99c6b5fd343b64a6700f06}$ 
 diff --git a/helpdesk/models.py b/helpdesk/models.py
 @@ -238 +238 @@ def cvesForCPE(cpe,
     $\textbf{if not}$ text:
         $\textbf{return}$ ""
-    pattern = fr'([\[\s\S\]]*?)\(([\s\S]*?):([\[\s\S\]]*?)\)'
+    pattern = fr'([\[\s\S\]]*?)\(([\s\S]*?):([\s\S]*?)\)'
     # Regex check
     $\textbf{if}$ re.match(pattern, text):
         # get get value of group regex
\end{lstlisting}



\subsubsection{Restrict Security Properties} 
The exploits often result from improper settings of security properties. 
14.55\% security commits in \db{} fix improper settings by updating boolean flags from \texttt{\small True} to \texttt{\small False} or vice versa, adding more arguments to methods, or adding security decorators.


\noindent{\bf Update Security Flags.} 
Security flags perform restrictions on the methods that may have access to sensitive objects. 
Improper restrictions on such flags may expose users to a risky environment and/or lead to sensitive information leakage. 
List~\ref{lst:flag} changes the flag from \texttt{\small False} to \texttt{\small True} to fix a vulnerability, where a sensitive cookie does not have a `HttpOnly' flag.


\lstdefinestyle{lst}{
    float=th,
    floatplacement=tbp,
    % abovecaptionskip=0.01in,
    numbers=left, 
    numberstyle=\scriptsize, 
    numbersep = 5pt,
    framexleftmargin = 0in,
    framexrightmargin = 0in,
    breaklines = true,
    xleftmargin = 0.18in,
    xrightmargin = 0.1in,
    basicstyle=\ttfamily\scriptsize, 
    frame=lines,
    showtabs=true,
    showspaces=true,
    showstringspaces=false,
    literate={\ }{{\ }}1,
    aboveskip=-0.00in,
    belowskip=-0.15in,
}

\begin{lstlisting}[
language=diff, 
style=lst,
caption=An example of security commit that fixes a vulnerability where the sensitive cookie does not have a `HttpOnly' flag (CVE-2019-25091).,
label={lst:flag},
mathescape=true
]
 $\textbf{commit 60a3fe559c453bc36b0ec3e5dd39c1303640a59a}$ 
 diff --git a/src/nsupdate/settings/base.py b/src/nsupdate/settings/base.py
 @@ -283,7 +283,7 @@
 ...
-CSRF_COOKIE_HTTPONLY = False
+CSRF_COOKIE_HTTPONLY = True
 ...
\end{lstlisting}

\noindent{\bf Add Restriction Arguments.} Some restriction arguments will be passed to the functions during execution. Improper argument settings may lead to a variety of mishandling. As shown in List~\ref{lst:arg}, the \texttt{\small formaction} is added to restrict the attributes of a variable to avoid XSS vulnerability.


\lstdefinestyle{lst}{
    float=th,
    floatplacement=tbp,
    % abovecaptionskip=0.01in,
    numbers=left, 
    numberstyle=\scriptsize, 
    numbersep = 5pt,
    framexleftmargin = 0in,
    framexrightmargin = 0in,
    breaklines = true,
    xleftmargin = 0.18in,
    xrightmargin = 0.1in,
    basicstyle=\ttfamily\scriptsize, 
    frame=lines,
    showtabs=true,
    showspaces=true,
    showstringspaces=false,
    literate={\ }{{\ }}1,
    aboveskip=-0.00in,
    belowskip=-0.15in,
}

\begin{lstlisting}[
language=diff, 
style=lst,
caption=An example of security commit that fixes a cross-site-scripting (XSS) vulnerability (CVE-2021-28957).,
label={lst:arg},
mathescape=true
]
 $\textbf{commit 10ec1b4e9f93713513a3264ed6158af22492f270}$ 
 diff --git a/src/lxml/html/defs.py b/src/lxml/html/defs.py
 @@ -23,6 +23,8 @@
 ...
+    # HTML5 formaction
+    'formaction'
     ])
 ...
\end{lstlisting}

\noindent{\bf Add Security Decorators.} A decorator is a function that takes another function and extends the behavior of the function without explicit modification. This mechanism has been widely adopted by security commits to add more detailed security restrictions on existing methods. List~\ref{lst:access control2} shows a security commit that fixes an access control vulnerability by adding decorator \texttt{\small security.private} to function \texttt{\small enumerateRoles}.

\lstdefinestyle{lst}{
    float=th,
    floatplacement=tbp,
    % abovecaptionskip=0.01in,
    numbers=left, 
    numberstyle=\scriptsize, 
    numbersep = 5pt,
    framexleftmargin = 0in,
    framexrightmargin = 0in,
    breaklines = true,
    xleftmargin = 0.18in,
    xrightmargin = 0.1in,
    basicstyle=\ttfamily\scriptsize, 
    frame=lines,
    showtabs=true,
    showspaces=true,
    showstringspaces=false,
    literate={\ }{{\ }}1,
    aboveskip=+0.10in,
    belowskip=-0.25in,
}

\begin{lstlisting}[
language=diff, 
style=lst,
caption=An example of security commit that fixes an access control vulnerability (CVE-2021-21336).,
label={lst:access control2},
mathescape=true
]
 $\textbf{commit 2dad81128250cb2e5d950cddc9d3c0314a80b4bb}$ 
 diff --git a/src/Products/plugins/ZODBRoleManager.py b/src/Products/plugins/ZODBRoleManager.py
 @@ -112,6 +112,7 @@ def getRolesForPrincipal(self, principal, request=None):
     #   IRoleEnumerationPlugin implementation
+    @security.private
     $\textbf{def}$ enumerateRoles(self, id=None, exact_match=False, sort_by=None, max_results=None, **kw):
         """ See IRoleEnumerationPlugin.
\end{lstlisting}



\subsection{Unique Patterns Captured from the Wild (RQ4)}\label{exp:variance}

Recall that we construct pilot and augmented datasets because the base dataset provides a limited number of security commits samples. Here, we further show the examples captured by our security commit collection approaches that introduce more variety in syntax and semantics of security-related code changes, enabling wider applications of \db{} in solving real-world Python-related security issues.

\subsubsection{Data Variety Introduced by Pilot Dataset}
We study the contribution of involving the pilot dataset for \gnn{} by comparing the model trained only on the base dataset and the model trained on the combination of the base and pilot datasets.
% first training on the base dataset and then training on the combination of the base dataset and the pilot dataset.
%Then, we analyze the samples that have not been identified by the first model but have been identified by the second model. 
We find that the pilot dataset helps the latter model to be able to identify more wild security commits. For instance, the latter \gnn{} can detect more subtle changes. % after expanding the training set with the pilot dataset. 
In List~\ref{lst:pilot}, the \texttt{\small '\%s'} has been changed to \texttt{\small ?} in a SQL query, protecting the database from being injected. 
% the condition refines the value of \texttt{\small GITHUB\_ACTIONS} from not null to true, protecting the authentication from being bypassed. 
The capability of detecting such minor changes is enabled by similar samples in the pilot dataset but not existed in the base dataset.

\lstdefinestyle{lst}{
    float=th,
    floatplacement=tbp,
    % abovecaptionskip=0.01in,
    numbers=left, 
    numberstyle=\scriptsize, 
    numbersep = 5pt,
    framexleftmargin = 0in,
    framexrightmargin = 0in,
    breaklines = true,
    xleftmargin = 0.18in,
    xrightmargin = 0.1in,
    basicstyle=\ttfamily\scriptsize, 
    frame=lines,
    showtabs=true,
    showspaces=true,
    showstringspaces=false,
    literate={\ }{{\ }}1,
    aboveskip=-0.00in,
    belowskip=-0.15in,
}

\begin{lstlisting}[
language=diff, 
style=lst,
caption=An example of security commit detected by \gnn{} trained on the base and pilot datasets.,
label={lst:pilot},
mathescape=true
]
 $\textbf{commit 9d8adbc07c384ba51c2583ce0819c9abb77dc648}$ 
 diff --git .../__init__.py .../__init__.py
 @@ -71,7 +71,7 @@ def klauen(self,
-    a = u"name == '%s' AND item =='%s'" % (name, item)
+    a = u"name == ? AND item ==?", (name, item)
\end{lstlisting}
%  $\textbf{commit c658b4f3e57258acf5f6207a90c2f2169698ae22}$ 
%  diff --git a/core.py b/core.py
%  @@ -112,7 +112,7 @@ def actualsys() :
% -    if os.environ.get('GITHUB_ACTIONS') != "":
% +    if os.environ.get('GITHUB_ACTIONS') == "true":
%          logging.warning("Running on Github Actions")
%          actualsys()

\subsubsection{Variance Introduced by Augmented Dataset}
We further evaluate to show that our augmented dataset can help train a model that is able to identify more various security commits from the wild. For example, after introducing augmented dataset into the training phase, the model detects a new escape pattern. As shown in List~\ref{lst:augmented}, the characters \texttt{\small <}, \texttt{\small >}, and \texttt{\small \&} have been escaped by being translated into Unicode, which prevents cross-site-scripting crafted with a partial JSON-serializable object. Compared with the escape expressions in Section~\ref{rq3} that only include ASCII characters, the augmented dataset help \gnn{} generalize the escapes to Unicode.


%46e95f5

\lstdefinestyle{lst}{
    float=th,
    floatplacement=tbp,
    % abovecaptionskip=0.01in,
    numbers=left, 
    numberstyle=\scriptsize, 
    numbersep = 5pt,
    framexleftmargin = 0in,
    framexrightmargin = 0in,
    breaklines = true,
    xleftmargin = 0.18in,
    xrightmargin = 0.1in,
    basicstyle=\ttfamily\scriptsize, 
    frame=lines,
    showtabs=true,
    showspaces=true,
    showstringspaces=false,
    literate={\ }{{\ }}1,
    aboveskip=-0.00in,
    belowskip=-0.15in,
}

\begin{lstlisting}[
language=diff, 
style=lst,
caption=A security commit example detected by the \gnn{} trained on the base{,} pilot{,} and augmented datasets.,
label={lst:augmented},
mathescape=true
]
 $\textbf{commit d3e428a6f7bc4c04d100b06e663c071fdc1717d9}$ 
 diff --git a/.../djblets_js.py b/.../djblets_js.py 
 @@ -28,11 +28,18 @@
+_safe_js_escapes = {
+    ord('&'): u'\\u0026',
+    ord('<'): u'\\u003C',
+    ord('>'): u'\\u003E',
+}
\end{lstlisting}



















\section{Related Work}







\paragraph{Model Merging}

Our method substantially draws on the concept of LoRA module composition, and thus, aligns with the significant thread of research in model merging. This research focus is broadly categorized based on the ultimate objectives of model merging.

 

The first category focuses on merging entire models, and the goal is to combine individually trained models to approximate the performance benefits of model ensembling or multi-task learning.
Prior works such as \cite{Matena2021MergingMW} and \cite{jin2023dataless} operated under the assumption of shared model architectures. \cite{Matena2021MergingMW} amalgamates models by approximating Gaussian posterior distributions garnered from Fisher information, while \cite{jin2023dataless} merges models steered by weights that minimize the differences in prediction.
Another approach is merging models with different architectures.
For instance, \cite{ainsworth2023git} configures weights of different models prior to their merger. Following this objective, \cite{stoica2023zipit} merges models operating on varying tasks by identifying common features, without requiring additional training.
Unlike these works, our work focuses on merging models to enable cross-task generalization.

The second category most closely aligns with our research, stemming from a shared motivation of module composition. Various scholars have made advances in this line of research: \cite{Kingetsu2021NeuralNM} decomposes and recomposes modules on the basis of their functionality;
\cite{Ilharco2022EditingMW} proposes modulating model behavior using task vectors;
\cite{Lv2023ParameterefficientWE} amalgamates parameter-efficient modules weighted according to task similarity;
\cite{Zhang2023ComposingPM} crafts modules by employing specific arithmetic operations; 
\cite{sun2023multitask} improves few-shot performance of unseen tasks by multi-task pre-training of prompts;
\cite{Chronopoulou2023AdapterSoupWA} averages adapter weights intended for transfer;
\cite{Ponti2023CombiningPM} focuses on jointly learning adapters and a routing function that allocates skills to each task;
and \cite{Muqeeth2023SoftMO} concentrates on amalgamating experts in mixture of experts models;
However, these methods generally necessitate multi-task training or human prior on module selection for the downstream task.
In contrast, our method does not impose any special training requirements and simply employs vanilla LoRA tuning. 
Additionally, the module selection for downstream tasks is entirely data-driven without human prior knowledge.
This design gives the advantage of easily adding new LoRA modules for reuse, allowing our method to flexibly scale up the number of potential LoRA module candidates in the future.



 


\paragraph{Mixture of Experts}
The Mixture of Experts (MoE) is an ensemble method, often visualized as a collection of sub-modules, or 'experts', each specializing in processing different types of input data. Each expert in this system is controlled by a unique gating network, activated based on the distinct nature of the input data. For every token in these input sequences, this network identifies and engages the most suitable experts to process the data. As a result, the performance is superior compared to relying on a single, generic model for all types of input. This technique has proven instrumental in numerous domains, such as natural language processing and computer vision~\citep{Jacobs1991AdaptiveMO,Shazeer2017OutrageouslyLN, Du2021GLaMES,zhang2022skillnetnlu}. Our methodology displays similarities to MoE, wherein upstream-trained LoRA modules can be aligned with MoE's expert design. A noteworthy distinguishing factor is that our approach mechanism does not require any specialized manipulation of LoRAs during training while facilitating dynamic LoRA module assembly at any scale, each pre-tuned to different tasks. In contrast, MoE mandates a predetermined count of experts during both the training and testing phases. Recent studies on the interrelation between MoE and instruction tuning have demonstrated that the simultaneous application of both approaches enhances the effectiveness of each individually~\citep{shen2023mixtureofexperts}.

\paragraph{Cross-Task Generalization}
Recent advancements like CrossFit~\citep{ye-etal-2021-crossfit}, ExT5~\citep{aribandi2022ext}, FLAN~\citep{Wei2021FinetunedLM}, T0~\citep{sanh2021t0}, InstructGPT~\citep{InstructGPT}, and ReCross~\citep{recross} have been striving to foster a vastly multi-task model's generalization across different tasks, very much aligned with the objectives of our research. Among this cohort, the connections of CrossFit and ReCross with \lorahub are particularly noteworthy. 
The CrossFit framework \citep{ye-etal-2021-crossfit} mandates a minimal number of labeled examples of the target task for few-shot fine-tuning. However, its limitation lies in the application of task names as hard prefixes in templates, posing challenges in the task's generalization. On the other hand, while ReCross mitigates the need for labels in few-shot examples for retrieval, it necessitates a fine-tuning process using the retrieved data. This procedure appears time-consuming when compared to \lorahub's approach. Through the deployment of few-shot labeled examples and a gradient-free optimization process, \lorahub facilitates an iterative update of weights to compose the LoRA modules. The resultant method is more efficient and cost-effective relative to previous work. Overall, \lorahub offers a more practical and viable solution to the optimization process.




\section{Conclusion}


In this work, we have introduced \lorahub, a strategic framework for composing LoRA modules trained on diverse tasks in order to achieve adaptable performance on new tasks. Our approach enables the fluid combination of multiple LoRA modules using just a few examples from a novel task, without requiring additional model parameters or human expertise. The empirical results on the BBH benchmark demonstrate that \lorahub can effectively match the performance of in-context learning in few-shot scenarios, removing the need for in-context examples during inference.
Overall, our work shows the promise of strategic LoRA composability for rapidly adapting LLMs to diverse tasks. 
By fostering reuse and combination of LoRA modules, we can work towards more general and adaptable LLMs while minimizing training costs.
 
 




\section{Limitations \& Future Work}\label{sec:limitation}

\paragraph{Pre-Filtering of LoRA Module Candidates}
While our method is successful in identifying and weighting relevant aspects from seen tasks to enhance unseen task performance, relying entirely on the model to perform this search can lead to increased computational demands and potentially unstable results. Incorporating a pre-filtering step to select only pertinent LoRA modules could expedite and refine performance. Identifying an effective selection strategy warrants further study.

\paragraph{Method Applicability to Decoder-Only Models}
All experiments for this study were executed using the encoder-decoder architecture. We aspire to extrapolate this method to decoder-only models such as GPT~\citep{gpt3}, aiming to determine its applicability in such contexts.

\paragraph{Exploring Superior Optimization Methods}
The use of a genetic algorithm for optimization in this study raises the question of whether better optimization approaches exist that could provide superior gradient-free optimization with limited examples. Although the current method has shown adequate performance, there is still room for improvement.

\bibliography{iclr2023_conference}
\bibliographystyle{iclr2023_conference}


\end{document}
