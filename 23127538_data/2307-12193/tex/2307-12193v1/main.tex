\documentclass[reprint, amsmath,amssymb, aps, superscriptaddress]{revtex4-2}
\usepackage{b26_style}
\usepackage[yyyymmdd,hhmmss]{datetime}

\newcommand{\var}[1]{\langle #1^2 \rangle}
\newcommand{\remark}[1]{\textcolor{blue}{#1}}
\newcommand{\fix}[1]{\textcolor{red}{#1}}
\newcommand{\confirm}[1]{\textcolor{green}{#1}}
\newcommand{\suggestion}[1]{\textcolor{blue}{#1}}
\newcommand{\CnNOTe}{\textrm{C}_\textrm{n} \textrm{NOT}_\textrm{e}}
\newcommand{\fifteenN}{^{15}\textrm{N}}
\newcommand{\thirteenC}{^{13}\textrm{C}}

% toggle (1) to add the supplementary to this document, (0) not to 
\def\suppl{0}
\DeclareUnicodeCharacter{2009}{\,}
\DeclareSIUnit\gauss{G}

\begin{document}

\title{
Programmable Quantum Processors based on Spin Qubits with Mechanically-Mediated Interactions and Transport}

\author{F. Fung}
\affiliation{Department of Physics, Harvard University, Cambridge, Massachusetts, 02138, USA}

\author{E. Rosenfeld}
\thanks{Currently at AWS Center for Quantum Computing, Pasadena, California, 91106, USA}
\affiliation{Department of Physics, Harvard University, Cambridge, Massachusetts, 02138, USA}

\author{J. D. Schaefer}
\affiliation{Department of Physics, Harvard University, Cambridge, Massachusetts, 02138, USA}

\author{A. Kabcenell}
\affiliation{Department of Physics, Harvard University, Cambridge, Massachusetts, 02138, USA}

\author{J. Gieseler}
\affiliation{Department of Physics, Harvard University, Cambridge, Massachusetts, 02138, USA}
\thanks{Current address: IAV GmbH DigitalLab.}

\author{\\ T. X. Zhou}
\thanks{Currently at Northrop Grumman Mission Systems, Linthicum, Maryland, 21090, USA}
\affiliation{Department of Physics, Harvard University, Cambridge, Massachusetts, 02138, USA}
\affiliation{Harvard John A. Paulson School of Engineering and Applied Sciences, Harvard University, Cambridge, Massachusetts, 02138, USA}
\affiliation{Massachusetts Institute of Technology, Cambridge, Massachusetts, 02139, USA}

\author{T. Madhavan}
\affiliation{Harvard John A. Paulson School of Engineering and Applied Sciences, Harvard University, Cambridge, Massachusetts, 02138, USA}

\author{N. Aslam}
\affiliation{Department of Physics, Harvard University, Cambridge, Massachusetts, 02138, USA}
\affiliation{Institute of Condensed Matter Physics, Technische Universit\"{a}t Braunschweig, Braunschweig, Germany}

\author{A. Yacoby}
\affiliation{Department of Physics, Harvard University, Cambridge, Massachusetts, 02138, USA}

\author{M. D. Lukin}
\email{lukin@physics.harvard.edu}
\affiliation{Department of Physics, Harvard University, Cambridge, Massachusetts, 02138, USA}

%\date{\today~at~\currenttime}

\begin{abstract} 
Solid state spin qubits are promising candidates for quantum information processing, but controlled interactions and entanglement in large, multi-qubit systems are currently difficult to achieve. We describe a method for programmable control of multi-qubit spin systems, in which individual nitrogen-vacancy (NV) centers in diamond nanopillars are coupled to magnetically functionalized silicon nitride mechanical resonators in a scanning probe configuration. Qubits can be entangled via interactions with nanomechanical resonators while programmable connectivity is realized via mechanical transport of qubits in nanopillars. To demonstrate the feasibility of this approach, we characterize both the mechanical properties and the magnetic field gradients around the micromagnet placed on the nanobeam resonator. Furthermore, we show coherent manipulation and mechanical transport of a proximal spin qubit by utilizing nuclear spin memory, and use the NV center to detect the time-varying magnetic field from the oscillating micromagnet, extracting a spin-mechanical coupling of \SI{7.7(9)}{\Hz}. With realistic improvements the high-cooperativity regime can be reached, offering a new avenue towards scalable quantum information processing with spin qubits.

\end{abstract}

\maketitle

%%%%%%%%%%%%%%%%%%%%%%%%%%%%%%%%%%%%%%%%%%%%%%%%%%
\paragraph*{Introduction.}
%%%%%%%%%%%%%%%%%%%%%%%%%%%%%%%%%%%%%%%%%%%%%%%%%%

Isolated spin defects in the solid state, such as nitrogen vacancy (NV) centers in diamond, have long been considered as promising candidates for quantum information processing, owing to their extended coherence times even at elevated temperatures \cite{balasubramanian2009ultralong, maurer2012room,widmann2015coherent,anderson2022five,stas2022robust}. While small spin registers have been realized using coupled electronic and 
%nitrogen vacancy (NV) centers 
nuclear spins \cite{dutt2007quantum, abobeih2018one, bradley2019ten}, such demonstrations rely on magnetic dipole-dipole interactions, which limit the distance between spins to tens of nanometers. The short-range nature of these interactions and imprecision of defect fabrication at these length scales make it challenging to control systems containing large arrays of spin qubits. 

Several approaches are currently being explored to address this challenge, including long-range entanglement based on photonic \cite{bhaskar2020experimental,hermans2022qubit} and mechanical systems \cite{rabl_quantum_2010,kuzyk2018scaling}. In particular, nanomechanical resonators have been proposed as a mesoscopic interface between distant and otherwise isolated spin qubits. Such a hybrid quantum system can be realized by combining electronic spins with magnetically functionalized mechanical resonators \cite{arcizet_single_2011, bennett_measuring_2012, kolkowitz_coherent_2012, nichol_nanomechanical_2012, rugar_single_2004, teissier_strain_2014, pigeau_observation_2015, lee_strain_2016, meesala_enhanced_2016, wrachtrup_cantilever_2020, gieseler_single-spin_2020, maity_coherent_2020}. Mechanical resonators can be engineered to have very high quality factors with flexible, compact geometric realizations, and feature low crosstalk relative to their electromagnetic counterparts \cite{chu2020perspective}. Using mechanical modes as a quantum transducer, distant spin qubits can be entangled deterministically, even when the mechanical mode is in a thermal, highly excited state \cite{rabl_quantum_2010, schuetz_high-fidelity_2017, rosenfeld_efficient_2020}. Furthermore,  spin qubits can be used to cool the mechanical resonator to its ground state \cite{rabl2009strong,rabl_cooling_2010} and subsequently prepare non-Gaussian states of motion \cite{oconnell_quantum_2010}. 
Despite these intriguing proposals, realizing the necessary strong coupling between mechanical systems and individual spin qubits is a challenging task, requiring deterministic positioning of spin qubits in close proximity to magnetized mechanical resonators. Moreover, even though transducers extend the spin-spin interaction range, the system connectivity remains local, limiting its programmability and scalability. 

% Figure environment removed

In this Letter, we introduce a novel platform for realizing programmable interactions between distant spin qubits. The key idea of our architecture is illustrated in Fig.~\ref{fig:setup}(a). In our approach, individual NV centers in diamond nanopillars are coupled to silicon nitride nanobeam mechanical resonators in a scanning probe geometry. A micromagnet attached to the nanobeam provides the magnetic field gradient for the spin-mechanical coupling. Ultra-high quality factors $Q>10^9$ have been demonstrated in silicon nitride mechanical resonators through a combination of techniques such as soft-clamping, dissipation dilution, and strain engineering \cite{tsaturyan_ultracoherent_2017, ghadimi_elastic_2018, bereyhi_clamp-tapering_2019, grob_fractal_2022,kippenberg_fractal_2021}. At the same time, the nanoscale footprint of the diamond nanopillar enables small separations between the NV center and the micromagnet \cite{zhou_scanning_2017, maletinsky_robust_2012}, providing access to high magnetic field gradients required for large spin-mechanical coupling. Remarkably, spin qubits confined in nanopillars can be moved mechanically in and out of the near-field of the magnetized resonators. Moreover, they can be transported across relatively long (10-100 \SI{}{\micro\meter}) distances, enabling non-local connectivity between distant qubits \cite{dolev_transport, dhordjevic2021entanglement,mandel2003coherent,pino2021demonstration,monroe2014large,cirac2000scalable}. Further improvements in coherence time can be obtained by making use of a nuclear spin quantum memory. Since the latter is less sensitive to magnetic fields, such storage can be used for long-distant qubit transport even in the presence of proximal magnetic gradients, enabling reconfigurable quantum processing architecture similar to that demonstrated recently for neutral atom array qubits \cite{dolev_transport}.

To demonstrate the feasibility of this approach, we first perform scanning magnetometry with the nanopillar to characterize the magnetic field and field gradients around a micromagnet. Subsequently, as a proof-of-principle demonstration of coherent transport, we store coherent information inside the NV center's intrinsic $\fifteenN$ nuclear spin, and show that the spin coherence is not affected by movement over $\SI{1.7(2)}{\micro\meter}$ near the micromagnet. Finally, by measuring the mechanical motion interferometrically and using the NV center to detect the time-varying magnetic field from the oscillating micromagnet, we extract the single-phonon spin-mechanical coupling strength of $\SI{7.7(9)}{\hertz}$. With realistic improvements to our quality factors and further reduction in the magnet-NV center distance, the coherent coupling regime is within reach.

%%%%%%%%%%%%%%%%%%%%%%%%%%%%%%%%%%%%%%%%%%%%%%%%%%
\paragraph*{Experimental setup and static magnetic field characterization.}
%%%%%%%%%%%%%%%%%%%%%%%%%%%%%%%%%%%%%%%%%%%%%%%%%%

Our experimental platform consists of a scanning probe setup, where a diamond nanopillar containing a single NV center is positioned near a micromagnet placed at the center of the nanobeam (Fig.~\ref{fig:setup}(b, c)). In addition to improving the optical collection efficiency, the nanopillar has a small surface area at its apex, allowing for nanoscale magnet-NV center distances \cite{zhou_scanning_2017, maletinsky_robust_2012,angle_etch}.

% Figure environment removed

The presence of a magnetic field perpendicular to the NV center quantization axis limits NV spin readout contrast, photoluminesence intensity, and coherence time $T_{2,e}$ in a natural-abundance $\thirteenC$ diamond \cite{stanwix_coherence_2010}. In order to align the magnetic field and characterize the field distribution, we scan the micromagnet with respect to the diamond nanopillar with a 3-axis stack of piezoelectric nanopositioners. At each position, we optically measure the electron spin resonance (ESR) and calculate the magnetic field along the NV center quantization axis \cite{supp}; an example map of the magnetic field around a micromagnet is shown in Fig.~\ref{fig:field_scan}(a). With our current smallest micromagnet-NV distance of $\sim\SI{1.0}{\micro\meter}$, we estimate gradients exceeding \SI{1.5e4}{\tesla/\meter}, corresponding to an expected single-phonon spin-mechanical coupling strength of $\lambda/2\pi \sim 5\,\SI{}{\hertz}$ (Fig. \ref{fig:field_scan}(b)). 

\paragraph*{Preservation of spin coherence while moving in a magnetic field gradient.}

% Figure environment removed

Next, to investigate whether the spin coherence can be maintained during qubit transport, we perform a proof-of-principle experiment in which we move the micromagnet \SI{1.7(2)}{\micro\meter} away from the diamond nanopillar and then back to its original position. Pulsed ESR measurements at different times during the movement sequence (Fig.~\ref{fig:coherence} (a)) reveal a significant change in the magnetic field environment, as evidenced by a shift of $\SI{9.8(1)}{\mega\hertz}$ in the ESR frequency \cite{supp}.

We demonstrate the preservation of spin coherence while moving in a magnetic field gradient, using the pulse sequence in Fig.~\ref{fig:coherence}(b) synchronized with the movement sequence in Fig.~\ref{fig:coherence}(a). Since the total movement time of \SI{1.7}{\milli\second} is significantly longer than the electronic spin coherence time of \SI{0.95(4)}{\milli\second} \cite{supp}, we use 
the NV center's intrinsic $\fifteenN$ nuclear spin as a quantum memory 
\cite{pfender2017nonvolatile, zaiser2016enhancing} to enable the mechanical qubit transport. 

Specifically, in our demonstration the electron and $\fifteenN$ nuclear spin are first initialized in a two-qubit register $\ket{-1}_e \otimes \ket{\downarrow}_n$ \cite{supp}, followed by a $\pi/2$-pulse which puts the $\fifteenN$ nuclear spin in a superposition $\ket{-1}_e \otimes \left( \ket{\downarrow} + \ket{\uparrow} \right )_n$. 
Subsequently, we apply a $\CnNOTe$ gate which fully entangles the electron-nucleus pair
$-\ket{0}_e \ket{\downarrow}_n + \ket{-1}_e \ket{\uparrow}_n$.

During the subsequent free evolution time $\tau$, the entangled electron-nucleus pair accumulates a phase $\phi(\tau)$. For the particular NV center in our measurements, hyperfine interactions with a nearby $\thirteenC$ nuclear spin lead to phase accumulation at a rate of $\sim \SI{0.9}{\mega\hertz}$. A second $\CnNOTe$ gate then disentangles the electron-nuclear pair and the phase information $\phi(\tau)$ is now entirely stored in the $^{15}\textrm{N}$ nuclear spin
$\ket{-1}_e \otimes \left( -\ket{\downarrow} + e^{i \phi(\tau)} \ket{\uparrow} \right )_n$.


As shown in Fig.~\ref{fig:coherence}(a), the field at the nanopillar changes significantly during the movement sequence, leading to an additional phase accumulation on the $\fifteenN$. We eliminate this additional phase by applying a $\pi$-pulse on the $^{15}\textrm{N}$ near the middle of the movement sequence \cite{supp}. Finally, a $\pi/2$-pulse at the end of the movement sequence converts the stored phase information $\phi(\tau)$ into the probability of finding the $\fifteenN$ in either $\ket{\downarrow}_n$ or $\ket{\uparrow}_n$, which can be measured with a boost in signal-to-noise ratio (SNR) using repetitive readout \cite{jiang2009repetitive}.

By fixing $\tau = 900\,\SI{}{\nano\second} < T_{2,e}^*$ and varying the rotation axis angle of the final $\pi/2$-pulse, we can quantify the spin coherence preservation. The results, shown in \ref{fig:coherence}(c) demonstrate that the normalized contrasts for cases where the micromagnet is moved (orange) and kept stationary (blue) are $0.61(3)$ and $0.57(3)$ respectively, indicating that the nuclear spin coherence is unaffected by the significant change in magnetic field.


\paragraph*{Mechanical motion and single-phonon coupling strength.}

Finally, we characterize the spin-mechanical coupling by exciting the nanobeam and characterizing its mechanical motion via independent measurements with both an interferometer and the nearby NV center. To take advantage of higher quality factors at low temperatures, we use the scanning probe setup in a helium cryostat. The interferometric measurements, shown in Fig.~\ref{fig:coupling}(a,b), reveal a quality factor of $\num{8.25(6)e5}$ (Fig.~\ref{fig:coupling}(b)), demonstrating that the quality factor remains high despite magnetic functionalization \cite{supp}. The mechanical frequency $\omega_r \sim \SI{1.4}{\mega\hertz}$ (Fig.~\ref{fig:coupling}(a)) corresponds to a period of \SI{0.7}{\micro\second}. As a result, the mechanical resonator can undergo multiple oscillations during the spin coherence time $T_{2,e}$, which is around several microseconds. The readily accessible high mechanical frequency of the nanobeam compares favorably to other spin-mechanical platforms, 
%with low mechanical frequencies in the \SI{}{\kilo\hertz} range, 
such as those featuring cantilevers, nanowires, and magnetic levitation \cite{rugar_single_2004, arcizet_single_2011, gieseler_single-spin_2020}.
% Figure environment removed

A displacement of the mechanical mode by the zero-point fluctuation $z_p$ shifts the NV center spin resonance by $\lambda / 2\pi$ via the Zeeman effect, resulting in the single-phonon coupling strength $\lambda = \gamma_e z_p \nabla_z$, where $\gamma_e$ is the NV center electronic spin gyromagnetic ratio, and $\nabla_z$ is the magnetic field gradient along the NV center quantization axis. To quantify the spin-mechanical coupling strength, we excite the nanobeam with an external broadband drive and detect the field from the oscillating micromagnet with the nearby NV center. We use a Hahn echo pulse sequence, which results in frequency-dependent detection of the magnetic spin environment \cite{supp}. Sweeping the time $\tau$ between the $\pi$ pulses and assuming a Gaussian distribution of the mechanical state, the spin contrast can be approximated as
\begin{equation}\label{signal}
S(\tau, \lambda, \Delta_x) = \alpha e^{- 8 \Delta_x^2 \lambda^2 \sin^4{(\omega_r \tau /2)}/\omega_r^2 z_p^2}e^{-\chi(\tau)}
\end{equation}
where $\Delta_x$ is the root-mean-squared amplitude of motion, $\alpha$ is the spin readout contrast, and $\chi(\tau)$ describes the coherence decay from other noise sources in the diamond, such as the bath of $^{13}$C nuclear spins \cite{stanwix_coherence_2010} \cite{supp}. 

To determine $\lambda$, we first independently quantify $\Delta_x$ by integrating the interferometer signal of the mechanical response from the wideband drive, and assign $\omega_r$ to the center frequency. For the data corresponding to figure \ref{fig:coupling}(a), we find that $\Delta_x = 1.86(1)$ nm. We then fit the Hahn echo data, normalized to a baseline Hahn echo measurement to compensate for intrinsic NV decoherence $e^{-\chi(\tau)}$ (Fig. \ref{fig:coupling}(c), black dots). For the fit (blue line), $\omega_r$ and $\Delta_x$ are fixed, leaving $\lambda$ as a free parameter. We find that $\lambda/2\pi = \SI{7.7(9)}{\hertz}$ (Fig. \ref{fig:coupling}(b), corresponding to a gradient of \SI{2.4(1)e4}{\tesla/\meter}, similar to the gradients from the static field imaging of the same magnet (Fig. \ref{fig:field_scan}(b)).



\paragraph*{Discussion and Outlook.}
Our experiments demonstrate the feasibility of the proposed architecture for programmable mechanically-mediated interactions between distant spins. Specifically, we showed that the NV center's intrinsic nuclear spin memory is not degraded by movement inside a field gradient, if the proper decoupling pulse sequences are applied. The demonstrated movement distance of $\SI{1.7(2)}{\micro\meter}$ in Fig.~\ref{fig:coherence}(a-c) significantly exceeds the range of magnetic dipole-dipole interactions between spins, and is limited by the moving speed of $\SI{1}{\milli\meter/\second}$ and nuclear spin coherence time $T_{2,n} \sim 5\,\SI{}{\milli\second}$ \cite{supp}. The speed can be increased by using a nanopositioner with a higher bandwidth and minimizing residual vibrations caused by the scanning motion. By decoupling the nuclear spin from its local environment or cooling to cryogenic temperatures, $T_{2,n}$ can be extended to $>1\,\SI{}{\second}$ \cite{maurer2012room, pfender2017protecting}, which would extend the possible distance to $> 1\,\SI{}{\milli\meter}$ even with the current speed.


At the same time, achieving quantum coherent spin-mechanical coupling \cite{rabl_cooling_2010, bennett_measuring_2012, vinante_upper_2016, van_wezel_nanoscale_2012, schuetz_high-fidelity_2017, rosenfeld_efficient_2020} requires increasing the coupling strength while minimizing noise. Specifically, the onset of coherent quantum phenomena is generally marked by the spin-mechanical cooperativity $C \equiv \frac{\lambda ^2}{\Gamma \kappa n_{th}} \gtrsim 1$, which compares the coherent coupling rate $\lambda$ to the dissipation rates $\Gamma, \kappa n_{th}$ of the spin and mechanical mode respectively. While the cooperativity of our present experiment exceeds previous spin-mechanical platforms involving NV centers \cite{supp}, it remains far below the coherent coupling regime. However, significant improvements can be made. Drift of the NV-magnet distance causes significant variations of the ESR frequency at high magnetic field gradients, limiting our current gradient to $\SI{2.4e4}{\tesla/\meter}$ at a distance of \SI{1.0}{\micro \meter}. With improvements to the setup stability and the use of atomic-force microscopy (AFM) feedback \cite{jayich_afm,zhou_scanning_2017}, positioning the NV center at a reduced distance of \SI{50}{\nano \meter} from the surface of a \SI{1}{\micro\meter}-diameter micromagnet should yield gradients $\sim\SI{1.4e6}{\tesla/\meter}$, or a spin-mechanical coupling of $\lambda/2\pi \sim \SI{800}{\hertz}$. The doubly clamped nanobeam can be replaced with recent designs that utilize strain engineering and soft-clamping, which have demonstrated $Q \sim 10^9$ at MHz frequencies \cite{tsaturyan_ultracoherent_2017, ghadimi_elastic_2018, kippenberg_fractal_2021, grob_fractal_2022, bereyhi2022perimeter}. Even higher quality factors have been demonstrated or predicted, by replacing silicon nitride with crystalline materials such as silicon and diamond \cite{beccari2022strained,sementilli2022nanomechanical}. For a coupling strength of $\lambda/2\pi = \SI{800}{\hertz}$, an NV center electronic spin coherence time $T_{2,e}$ of \SI{10}{\milli\second} \cite{abobeih2018one,bar-gill_solid-state_2013}, and a quality factor of $10^9$ at \SI{4}{\kelvin}, the coherent coupling regime is possible with $C \sim 75$. Under such conditions, mechanics-mediated entanglement of electronic spins with fidelity exceeding \SI{95}{\percent} should be feasible according to the proposal in \cite{rosenfeld_efficient_2020}. Although we expect $T_{2,e}$ to improve with larger NV implantation depth, further investigation into diamond fabrication and surface termination might be required to increase $T_{2,e}$ to the \SI{10}{\milli\second} regime for NV centers in diamond nanopillars \cite{luan_decoherence_nodate, sangtawesin_origins_2019}.

Compared to previous work involving on-chip, circuit-based hybrid quantum systems \cite{rabl_quantum_2010, oconnell_quantum_2010}, a spin-mechanical architecture featuring dynamical qubit transport has the advantage of being able to generate programmable, non-local interactions, similar to reconfigurable platforms based on neutral atoms and trapped ions. The long coherence time of the nuclear spin allows multiple distant spins to be dynamically transported to interact with the same mechanical bus. While Fig.~\ref{fig:setup} only shows one mechanical resonator for clarity, the architecture can also be parallelized, with multiple mechanical resonators simultaneously mediating interactions within large arrays of spins. We also note that, unlike most other hybrid quantum systems \cite{gieseler_single-spin_2020, oconnell_quantum_2010}, both the mechanical and spin components of our platform are highly coherent even at room temperature. With a nanomagnet diameter of \SI{0.3}{\micro\meter}, an NV-magnet separation of \SI{20}{\nano \meter}, spin coherence time of $T_{2,e} = \SI{2}{\milli\second}$ \cite{herbschleb2019ultra}, and $Q=\num{1e9}$, reaching the coherent-coupling regime at room temperature appears feasible. The above considerations indicate that with realistic improvements, our platform can enable programmable interactions between distant spins, opening up a new avenue towards scalable quantum information processing with solid-state spin qubits. Finally, the present approach can be extended to realize other hybrid systems by coupling spin qubits to  quantum systems such as superconducting qubits and optical photons \cite{clerk2020hybrid,chu2020perspective,barzanjeh2022optomechanics}.

\vspace{10pt}
\paragraph*{Acknowledgements} We thank M. Markham and Element Six for providing diamond samples, R. Walsworth and M. Turner for annealing diamonds, K. Van Kirk, D. Bluvstein, A. Mulski, and E. Polzik for helpful discussions, B. Machielse for assistance with fabrication, R. Riedinger for comments on the manuscript, S. Y. F. Zhou and A. Cui for assistance with sample magnetization, and A. S. Zibrov and J. MacArthur for technical assistance. This work was supported by NSF, Center for Ultracold Atoms, and DOE Quantum Systems Accelerator Center (contract no. 7568717). E. R. acknowledges support from the NSF Graduate Research Fellowship Program. A. K. acknowledges support from the DOD through the NDSEG Fellowship Program. J. G. acknowledges support by the European Union (SEQOO, H2020-MSCA-IF-2014, No. 655369). Diamond and resonator sample fabrication were performed at the Center for Nanoscale Systems (CNS), a member of the National Nanotechnology Coordinated Infrastructure, which
is supported by the NSF under award no. ECCS -1541959. CNS is part of Harvard University. T. X. Z is an affiliate to Center for Quantum Network (CQN) established by NSF to facilitate collaboration within CQN to advance quantum research. T.M. acknowledges support from the NSF Graduate Research Fellowship Program (grant 2140743). N.A. acknowledges support from the Alexander von Humboldt Foundation and by the Federal Ministry of Education and Research (BMBF, project ``13N16297'').

%\bibliographystyle{apsrev4-1}
\bibliography{strings} 

%\onecolumngrid
%\clearpage
%\documentclass[report, amsmath,amssymb, aps, superscriptaddress]{revtex4-2}
\usepackage{b26_style}
\usepackage[yyyymmdd,hhmmss]{datetime}

\newcommand{\var}[1]{\langle #1^2 \rangle}
\newcommand{\remark}[1]{\textcolor{blue}{#1}}
\newcommand{\fix}[1]{\textcolor{red}{#1}}
\newcommand{\confirm}[1]{\textcolor{green}{#1}}
\newcommand{\suggestion}[1]{\textcolor{blue}{#1}}
\newcommand{\CnNOTe}{\textrm{C}_\textrm{n} \textrm{NOT}_\textrm{e}}
\newcommand{\fifteenN}{^{15}\textrm{N}}
\newcommand{\thirteenC}{^{13}\textrm{C}}

\DeclareUnicodeCharacter{2009}{\,}
\DeclareSIUnit\gauss{G}

\begin{document}
% New commands
\newcommand{\incr}{\,\mathrm{d}}
\newcommand{\set}[1]{\{#1\}}
\newcommand{\diff}[2]{\frac{\mathrm{d}{#1}}{\mathrm{d}{#2}}}
\newcommand{\pdiff}[2]{\frac{\partial{#1}}{\partial{#2}}}
\newcommand{\ndiff}[3][]{\frac{\mathrm{d}^{#1}{#2}}{\mathrm{d}{#3}^{#1}}}
\newcommand{\npdiff}[3][]{\frac{\partial^{#1}{#2}}{\partial{#3}^{#1}}}
\newcommand{\R}{\mathbb R}

% New environments
\newtheorem{remark}{Remark}

\title{\Supplementary Information for \\
Programmable Quantum Processors based on Spin Qubits with Mechanically-Mediated Interactions and Transport}
\author{}
\maketitle

\tableofcontents

\section{Experimental setup}
The experiment is performed inside a modified Janis ST-500 continuous flow cryostat, which also serves as a vacuum chamber for room temperature measurements shown in Fig.~2 and 3. The AFM chip containing the diamond nanopillar is attached to a glass slide which is then mounted on a fixed copper block. The sample containing the nanobeams is mounted on a 3-axis Attocube stack (ANPz101 and ANPx101), which allows for fine control over the position of the diamond nanopillar relative to nanobeam.

We use a home-built confocal microscope to perform NV measurements. A Mach-Zehnder interferometer (Fig.~\ref{fig:interferometer}) integrated into the optical path is used to characterize the mechanical properties of the nanobeams.

% Figure environment removed

\section{Sample fabrication}
We start with a 4-inch, \SI{525}{\micro\meter}-thick, high resistivity ($\geq 10^4 ~\Omega \cdot \mathrm{cm}$) $\mathrm{\langle 100 \rangle}$ silicon wafer obtained from Silicon Valley Microelectronics (SVMI). These wafers have \SI{150}{\milli\meter} of high stress ($\sim 1$ GPa) silicon nitride deposited on one side using LPCVD. The backside is also scored $\sim \SI{200}{\nano\meter}$ deep according to a \SI{1}{\centi\meter} square grid.

First, we fabricate the coplanar waveguide (CPW) using optical lithography, electron-beam evaporation, and a liftoff process. The CPW consists of \SI{200}{\nano\meter} of gold evaporated on a \SI{7}{\nano\meter} sticking layer of titanium. Next, the mechanical resonators are defined with electron-beam lithography and the corresponding areas are etched with reactive ion etching, resulting in 
nanobeams that are \SI{145}{\micro\meter} long and \SI{1}{\micro\meter} wide.  Then the resonators are released using either a wet KOH etch or a dry $\mathrm{XeF_2}$ isotropic etch. Finally, the sample chip is wire-bonded to a printed circuit board (PCB) for microwave delivery.

We magnetically functionalize the nanobeam using a home-built micromanipulator setup. The sample is placed on a 3-axis translation stage, while a tungsten tip (75960-02, Electron Microscopy Sciences) is mounted on a nearby micromanipulator. This allows us to move both the sample and tip independently under a microscope objective (100X Mitutoyo Plan Apo SL). We first deposit a small volume of UV epoxy (Loctite 349) on the center of the beam using the tungsten tip. Then we use another tungsten tip to pick up a NdFeB micromagnet, put it on the glue, and cure the epoxy with UV light. Since we have observed that the glue remains tacky even after prolonged UV exposure, we put the sample on a hot plate at $\SI{80}{\celsius}$ for $8$ hours to eliminate any ``tackiness''. A comparison of the quality factors of a functionalized and bare resonator is shown in Fig.~\ref{fig:functionalized_Q}.

% Figure environment removed


Details on fabrication of the diamond nanopillars can be found in references \cite{angle_etch,zhou_scanning_2017}.


\section{Magnetic field imaging}
The scan in Fig.~2(a) in the main text was done in air at room temperature without an external field. At each position in the scan, we perform an electron spin resonance (ESR) measurement by sweeping a microwave frequency and observing changes in the NV center photoluminescence. Since we are not driving the mechanical resonator and the measurement time is much longer than the mechanical period, we can neglect the mechanics. The spin Hamiltonian is given by:

\begin{equation}
\mathcal{H}/\hbar = D S_z^2 + \gamma_e (S_z \cdot B_z + S_x \cdot B_x)
\end{equation}
where $D = \SI{2.8707}{\giga\hertz}$ is the zero-field splitting and $\gamma_e = \SI{2.8}{\mega\hertz/\gauss}$ is the NV gyromagnetic ratio. By diagonalizing $\mathcal{H}$, we can extract the magnetic field component $B_z$ from the micromagnet that is parallel to the NV center quantization axis.

Two pixels are missing because the ESR frequencies exceeded the frequency sweep range. We do not consider the missing pixels in the fit to the dipole model. For display in the figure, we fill in these pixels by interpolating from nearby pixels.

\section{Analytical form of contrast in the presence of mechanical motion}

For our analysis we consider only the $\ket{S_z = +1}$ and $\ket{S_z = 0}$ eigenstates, and model our spin-mechanical system using the following Hamiltonian:

\begin{equation}\label{Ham}
\mathcal{H}/\hbar = \frac{\omega_s}{2} \sigma_z + \omega_r a^\dagger a + \lambda 
\sigma_z (a + a^\dagger)
\end{equation}
where $a$ is the annihilation operator for the mechanical mode, $S_z = \frac{1}{2} \sigma_z$ is the spin-$\frac{1}{2}$ operator, and $\lambda = 2\pi \gamma_e z_p G$ is the spin-mechanical coupling strength for gradient $G$ and zero-point motion $z_p$. For our system, we find typical values of $z_p \sim \SI{10}{\femto\meter}$.

Considering a Hahn echo pulse sequence, we calculate the signal in the limit of no intrinsic spin decoherence. Since $\lambda$ is small, we use a semi-classical approximation, where the resonator quadrature $x = z_p (a + a^\dagger)$ is independent of the spin state.

After initializing the spin into $\ket{0}$, we put it into a coherent superposition with a $\pi/2$ pulse. We then let it evolve over time $\tau$, resulting in the following spin state, up to a global phase:

\begin{equation}
    \ket{\psi} = \frac{1}{\sqrt{2}}\left( \ket{0} + e^{-i\int_0^\tau \frac{\lambda}{z_p} x(t) dt}\ket{1} \right)
\end{equation}

After the $\pi$ pulse and evolution for another time $\tau$, the spin state becomes
\begin{equation}
\ket{\psi} = \frac{1}{\sqrt{2}}\left(\ket{0} + e^{-i \phi (\tau)}\ket{1} \right)
\end{equation}
where $ \phi(\tau) \equiv \int_\tau^{2\tau} \frac{\lambda}{z_p} x(t) dt - \int_0^\tau \frac{\lambda}{z_p} x(t) dt$ is the phase difference accumulated due to the mechanical motion. After the last $\pi/2$ pulse to convert the coherence into population, the measured contrast is $\alpha \cos{(\phi(\tau))}$, where $\alpha$ is a constant determined by the spin-dependent optical initialization and readout, as well as the background fluorescence. We assume that the quality factor is high, $Q \gg 1$, such that the signal is highly correlated during the pulse sequence, and can be given by a specific point in phase space with amplitude $x_0$ and phase $\phi_0$, such that $x(t) = x_0 \cos{(\omega_r t + \phi_0)}$. For a given run of the pulse sequence with phase $\phi_0$ and amplitude $x_0$ the accumulated phase on the spin is: 
\begin{equation}
    \phi(\tau) = \frac{\lambda}{z_p \omega_r} x_0 \left(\sin{(2\omega_r \tau + \phi_0)} - 2\sin{(\omega_r \tau + \phi_0)} + \sin{(\phi_0)} \right)
\end{equation}

The resonator frequency drifts over the duration of the Hahn echo measurement, so we apply a wide-band drive to ensure that the resonator is always driven on resonance. Because the wide-band drive effectively raises the temperature of the mechanical mode and the measurement is longer than the dissipation time,  we incoherently average over the statistical distribution for $x(t)$. From the statistics of thermal states, we assume that $\phi_0$ is uniformly distributed, and that the energy $E$ of the resonator follows a Boltzmann distribution:
\begin{equation}
p(E) dE = \frac{1}{kT} e^{\frac{E}{kT}} dE
\end{equation}
where $E = \frac{1}{2} m_{\mathrm{eff}} \omega_r^2 x_0^2$ and $\frac{1}{2} k_B T = \frac{1}{2} m_{\mathrm{eff}} \omega_r^2 \Delta_x^2$ according to the equipartition theorem; $m_{\mathrm{eff}}$ is the effective mass of the resonator, $k_B$ is the Botlzmann constant, $T$ is the effective temperature of the mechanical mode, and $\Delta_x^2$ is the variance of the mechanical motion.

%the square of the amplitude follows a Boltzmann distribution with mean $2\Delta_x^2$ (multiplied by two for each of the quadratures). %square of the amplitude $\mathcal{E}$ follows the Boltzmann exponential distribution, with mean given by $m \omega_r^2 \Delta_x^2$. 

Averaging over the phase $\phi_0$ first, we find the equivalent signal under a coherent drive with amplitude $x_0$:
\begin{equation}
    \alpha \langle \cos{(\phi(\tau))} \rangle_{\phi_0} = \alpha \mathcal{J}_0 \left(\frac{2\lambda x_0}{z_p\omega_r} \left(\cos{(\omega_r t)}-1\right)\right),
\end{equation}
where $\mathcal{J}_0$ is the zeroth order Bessel function of the first kind. Note that this would be the form of the signal under a coherent state assumption. Averaging over the amplitudes:
\begin{equation}
      \alpha \langle \cos{(\phi(\tau))} \rangle_{A, \phi_0} =\alpha \int_0^\infty \frac{1}{\Delta_x^2} e^{-x_0^2/2\Delta_x^2} \langle \cos{(\phi(\tau))} \rangle_{\phi_0} x_0 \text{ d}x_0 = \alpha e^{-q(\tau)}
\end{equation}
where $q(\tau) = 8 \Delta_x^2 \lambda^2 \sin^4{(\omega_r \tau/2)}/(\omega_r^2 z_p^2)$ as reported in the main text \cite{bennett_measuring_2012}. To account for spin decoherence, we include an additional phenomenological term $e^{\chi(\tau)}$ and factor it out with a baseline Hahn echo measurement in the absence of mechanical driving.

%As a consistency check, integrating over Gaussian distributions of the quadratures $x_0$ and $p_0 = m\omega_r \dot{x}_0$ yields the same result.

%\subsection{Consistency check 1}
%We can arrive at the same result if instead we model the position of the oscillator as $x(t) = x_0 \cos{(\omega_r t)} + \frac{p_0}{\omega m} \sin{(\omega_r t)}$ and take Gaussian averages over $x_0$ and $p_0$. YOU GET THE SAME RESULT IN THE END :)

%\subsection{Consistency check 2} COMPARING TO BENNETT ET AL: THEY INCLUDE THE EFFECT OF FINITE Q SINCE THEY DO LONGER DYNAMICAL DECOUPLING AND HAVE WORSE Q. THIS RESULTS IN NUMERICS. IN A LIMIT THEY GET AN EXPONENTIAL DECAY FORM LIKE WE DO, AND THEIR CURVES LOOK QUALITATIVELY THE SAME AS US.

\section{Preservation of coherent information while moving inside a field gradient}

\subsection{Experimental setup}
The measurements in Fig.~3 in the main text were performed in air at room temperature. An external field of about \SI{180}{\gauss} from a bar magnet is aligned within $1^{\circ}$ to the quantization axis of the NV center. To drive radiofrequency (RF) pulses on nuclear spins, we use an external coil consisting of magnet wire wound in two layers with $7$ turns each.

The sample is placed on a 3-axis Attocube stack, which in turn is placed on top of a single-axis flexure stage (Thorlabs NFL5DP20S). To move the sample containing the mechanical resonators horizontally relative to the nanopillar during the movement sequence, we drive the flexure stage with a signal from an arbitrary function generator (Tektronix AFG3021) amplified by a piezo driver (Thorlabs BPC301). The signal is chosen to be a simple sinusoid to avoid exciting higher harmonics of the setup (e.g. resonances of the Attocube stack).

\subsection{Nuclear and electronic spin control}
The RF signals are synthesized with an arbitrary function generator (Tektronix AFG3022) and amplified (Mini Circuits LZY-22+). With the RF coil placed between the nanopillar and long working distance objective (LU Plan 100X/0.90), we obtain minimum $\pi$-pulse durations of $\sim\SI{20}{\micro \second}$.

The same on-chip coplanar waveguide (CPW) in Fig.~1 of the main text is used to drive microwave (MW) transitions of the electronic spin. We generate two MW signals at the fixed frequencies $\mathrm{MW}_\uparrow$ and $\mathrm{MW}_{\downarrow}$, which correspond to the upper and lower $\fifteenN$ hyperfine transitions (Fig.~\ref{fig:pulsed_esr_stationary}). The two MW signals are combined using a power combiner, gated with a switch, and then amplified (Mini Circuits ZHL 16W 43+).

% Figure environment removed

During the movement sequence, the micromagnet begins near the NV center, moves away, and finally returns to its original position. The pulse sequences for the coherence measurements are synchronized with the movement sequence. While the hyperfine frequencies shift significantly over the entire movement sequence, we expect from the sinusoidal driving signal that the shift will be insignificant during the beginning and end, which are the only times where $\CnNOTe$ pulses are applied. We estimate from our field profile measurement in Fig.~3(a) that during the time between the two entangling/disentangling gates, the hyperfine frequencies do not deviate by more than \SI{15}{\kilo\hertz}, corresponding to an estimated phase accumulation of less than $0.01\,\pi$ over a maximum accumulation time of $\SI{4.5}{\micro\second}$.

\subsection{Pulse sequence for ESR measurements during movement sequence}

To monitor the field experienced by the NV center during the movement of the micromagnet (Fig.~3(a) in main text), we take pulsed ESR measurements at different times and MW frequencies during the movement sequence (Fig.~\ref{fig:pulsed_esr_moving}). By initializing the NV center electronic spin and then applying a $\pi$ pulse, we expect to see a drop in the NV photoluminescence if the MW frequency coincides with the ESR frequency. To obtain a better signal-to-noise ratio, we repeat for $n = 11$ times the initialization and $\pi$-pulse pair, for each time segment ($\sim \SI{46}{\micro\second}$) of the entire movement duration of \SI{1.7}{\milli\second}.

Due to the long read times of our data acquisition card, we use a single long readout window for each time segment, which results in faster data acquisition at the expense of lower photoluminescence contrast, as the readout window now includes times where the NV center is being reinitialized.

% Figure environment removed

\subsection{Pulse sequence for coherence measurement}
The detailed version of the pulse sequence in Fig.~3(b) in the main text is shown in Fig.~\ref{fig:full_pulse_sequence}. At the beginning of the pulse sequence, we polarize the $\fifteenN$ nuclear spin by first polarizing the electron spin and then transferring the polarization to the nuclear spin via a SWAP operation \cite{jiang2009repetitive}. A pulsed ESR measurement after $\fifteenN$ polarization is shown in Fig.~\ref{fig:nuclear_polarization}, where the polarization is $\sim\SI{78}{\percent}$.

% Figure environment removed


To normalize the measurement in Fig.~3(c) in the main text, we use the same pulse sequence, but without the two entangling and disentangling $\CnNOTe$ gates. Setting the rotation axis angle of the final nuclear $\pi/2$ pulse to be $\theta = 0$ and $\pi$, we obtain the upper and lower bounds for the fluorescence levels, which we use to normalize the measurements where we vary $\theta$.

% Figure environment removed

Despite the small gyromagnetic ratio of the the $\fifteenN$ nuclear spin, the field change during the movement sequence is large enough to cause significant phase accumulation ($>2 \pi$) on the nuclear spin. To cancel out this phase, we apply a $\pi$-pulse on the $\fifteenN$ nuclear spin. Since the field changes symmetrically during the movement sequence, the optimal time for this $\pi$-pulse is around the middle of the movement sequence, when the micromagnet is the furthest away from the NV center.

One can calculate the optimal $T_\pi$ by measuring the ESR frequencies of the electronic spin during the movement sequence (Fig.~3(a) in the main text). As the micromagnet moves away from the diamond nanopillar, the ESR frequencies become positively detuned relative to their initial values due to the Zeeman interaction; as the micromagnet moves back, the detunings of the ESR frequencies decrease back to $0$. The nuclear magnetic resonance (NMR) frequency of the $\fifteenN$ nuclear spin also experiences the same detuning as the electron spin, scaled by the ratio of their respective gyromagnetic ratios $\gamma_{\mathrm{n}}/\gamma_e$. We then solve for $T_\pi$ such that the total phase accumulated $\phi_{\mathrm{n}}$ is $0$:

\begin{equation}
\phi_{\mathrm{n}} = \int^{T_\pi}_0 \gamma_{\mathrm{n}}\delta_{\mathrm{n}}(t) dt - \gamma_{\mathrm{n}} \int^{T_{\mathrm{move}}}_{T_\pi} \gamma_{\mathrm{n}} \delta_{\mathrm{n}}(t) dt = 0
\label{eqn:phase_accum}
\end{equation}
where $T_{\mathrm{move}} = \SI{1.7}{\milli\second}$ is the duration of the entire movement sequence; $\delta_n(t)$ is the time-dependent detuning of the $\fifteenN$ NMR frequency from its frequency at the beginning of the movement sequence.

To experimentally determine $T_\pi$, we use the pulse sequence in Fig.~\ref{fig:vary_pi_time}, where we vary the $\pi$-pulse time $T_\pi$ and observe the resulting change in readout photoluminescence. As expected, as we vary $T_\pi$, the total phase accumulation also changes, resulting in the fringes in Fig.~\ref{fig:vary_pi_time}. The maxima of the fringe correspond to the times when the total phase accumulation is $0$. The maximum at $\sim \SI{853}{\micro\second}$ is roughly half the time of the movement sequence and agrees with the calculated result using Equation~\ref{eqn:phase_accum}.

% Figure environment removed

\subsection{Phase information $\phi(\tau)$}
In fig.~3(c) of the main text, we quantify the preservation of spin coherence by fixing $\tau = \SI{900}{\nano\second}$ and varying the rotation axis angle of the final $\pi/2$ pulse. We also show in Fig.~\ref{fig:phase_vary_tau} a measurement where we keep the rotation axis angle of the final $\pi/2$ pulse same as all the other nuclear pulses, and vary $\tau$ instead. As expected, we observe Ramsey fringes due to the $\CnNOTe$ gate having an equal detuning from the two $\thirteenC$ transitions. 

% Figure environment removed

\section{NV center electronic and nuclear spin coherence times}

In Fig.~\ref{fig:t1_t2}, we show measurements of the electron spin coherence time ($T_{2,e}$), lifetime ($T_{1,e}$), and the $\fifteenN$ nuclear spin coherence time ($T_{2,n}$). These measurements are performed in air at room temperature in the presence of an external bias magnetic field of $\sim \SI{180}{\gauss}$. With an XY8-4 sequence, $T_{2,e}$ exceeds \SI{0.95(4)}{\milli\second} (middle). We expect that NV centers implanted with more energy than the 6 keV used in this diamond nanopillar would have improved coherence times. Furthermore, we measure the nuclear spin coherence time $T_{2,n} \sim \SI{5}{\milli\second}$ (right), similar to $T_{1,e}$, suggesting that it is limited by the electron spin lifetime.

% Figure environment removed

\section{Comparisons of Cooperativity for Previous Works}
In Table~\ref{table:cooperativities} we compare the spin-mechanical cooperativity of our current system with those of previous works involving NV centers. We note that some of the previous works do not explicitly report parameters necessary to calculate the cooperativity; we give optimistic estimates (highlighted) where appropriate.

\begin{table}[]
\centering
\begin{tabular}{lcccccc}
\hline
Work &
  \begin{tabular}[c]{@{}c@{}}$\omega/(2\pi)$ \\ (Hz)\end{tabular} &
  \begin{tabular}[c]{@{}c@{}}$g/(2\pi)$\\ (Hz)\end{tabular} &
  \begin{tabular}[c]{@{}c@{}}T\\ (K)\end{tabular} &
  \begin{tabular}[c]{@{}c@{}}$n \kappa/(2\pi)$\\ (Hz)\end{tabular} &
  \begin{tabular}[c]{@{}c@{}}$T_2$\\ (s)\end{tabular} &
  $C$ \\ \hline
This work        & $\num{1.4e6}$ & $\num{7.7}$    & 20  & $\num{5e5}$  & $\num{8.8e-4}$ & $\num{1.0e-7}$  \\
Gieseler \textit{et al.} \cite{gieseler_single-spin_2020} & $\num{1.4e2}$ & $\num{4.8e-2}$ & \colorbox{yellow}{4}   & $\num{8e4}$  & $\colorbox{yellow}{\num{1.0e-2}}$ & $\num{2.8e-10}$ \\
Arcizet \textit{et al.} \cite{arcizet_single_2011} & $\num{1.0e6}$ & $\num{1.4e2}$  & 300 & $\num{4e10}$ & $\num{4.0e-4}$ & $\num{1.8e-10}$ \\
Oeckinghaus \textit{et al.} \cite{wrachtrup_cantilever_2020} & $\num{8.6e5}$ & $\num{7.7}$    & 5   & $\num{2e5}$  & $\num{1.2e-6}$ & $\num{2.8e-10}$ \\ \hline
\end{tabular}
\caption{Comparison of spin-mechanical cooperativities with previous works. Optimistic estimates (highlighted) are given for unreported values. We also note the large coupling strength ($> \SI{700}{\hertz}$ reported in a related work where a cantilever is coupled to unpaired electron spins ($\sim \SI{700}{\hertz})$ \cite{rugar_single_2004}; however, the spin coherence time and quality factor were not reported.}
\label{table:cooperativities}
\end{table}

%\bibliographystyle{unsrt}
\bibliography{SI_bib} 
\end{document}

\end{document}