%Version 2.1 April 2023
% See section 11 of the User Manual for version history
%
%%%%%%%%%%%%%%%%%%%%%%%%%%%%%%%%%%%%%%%%%%%%%%%%%%%%%%%%%%%%%%%%%%%%%%
%%                                                                 %%
%% Please do not use \input{...} to include other tex files.       %%
%% Submit your LaTeX manuscript as one .tex document.              %%
%%                                                                 %%
%% All additional figures and files should be attached             %%
%% separately and not embedded in the \TeX\ document itself.       %%
%%                                                                 %%
%%%%%%%%%%%%%%%%%%%%%%%%%%%%%%%%%%%%%%%%%%%%%%%%%%%%%%%%%%%%%%%%%%%%%

%%\documentclass[referee,sn-basic]{sn-jnl}% referee option is meant for double line spacing

%%=======================================================%%
%% to print line numbers in the margin use lineno option %%
%%=======================================================%%

%%\documentclass[lineno,sn-basic]{sn-jnl}% Basic Springer Nature Reference Style/Chemistry Reference Style

%%======================================================%%
%% to compile with pdflatex/xelatex use pdflatex option %%
%%======================================================%%

%%\documentclass[pdflatex,sn-basic]{sn-jnl}% Basic Springer Nature Reference Style/Chemistry Reference Style


%%Note: the following reference styles support Namedate and Numbered referencing. By default the style follows the most common style. To switch between the options you can add or remove �Numbered� in the optional parenthesis. 
%%The option is available for: sn-basic.bst, sn-vancouver.bst, sn-chicago.bst, sn-mathphys.bst. %  
 
\documentclass[sn-mathphys,Numbered,iicol]{sn-jnl}% Style for submissions to Nature Portfolio journals
%%\documentclass[sn-basic]{sn-jnl}% Basic Springer Nature Reference Style/Chemistry Reference Style
% \documentclass[sn-mathphys,Numbered]{sn-jnl}% Math and Physical Sciences Reference Style
%%\documentclass[sn-aps]{sn-jnl}% American Physical Society (APS) Reference Style
%%\documentclass[sn-vancouver,Numbered]{sn-jnl}% Vancouver Reference Style
%%\documentclass[sn-apa]{sn-jnl}% APA Reference Style 
%%\documentclass[sn-chicago]{sn-jnl}% Chicago-based Humanities Reference Style
%%\documentclass[default]{sn-jnl}% Default
%%\documentclass[default,iicol]{sn-jnl}% Default with double column layout

%%%% Standard Packages
%%<additional latex packages if required can be included here>

\usepackage{graphicx}%
\usepackage{multirow}%
\usepackage{amsmath,amssymb,amsfonts}%
\usepackage{amsthm}%
\usepackage{mathrsfs}%
\usepackage[title]{appendix}%
\usepackage{xcolor}%
\usepackage{textcomp}%
\usepackage{manyfoot}%
\usepackage{booktabs}%
\usepackage{algorithm}%
\usepackage{algorithmicx}%
\usepackage{algpseudocode}%
\usepackage{listings}%
\usepackage{subfigure}
\usepackage{bm}
\usepackage{multicol}

%%%%

%%%%%=============================================================================%%%%
%%%%  Remarks: This template is provided to aid authors with the preparation
%%%%  of original research articles intended for submission to journals published 
%%%%  by Springer Nature. The guidance has been prepared in partnership with 
%%%%  production teams to conform to Springer Nature technical requirements. 
%%%%  Editorial and presentation requirements differ among journal portfolios and 
%%%%  research disciplines. You may find sections in this template are irrelevant 
%%%%  to your work and are empowered to omit any such section if allowed by the 
%%%%  journal you intend to submit to. The submission guidelines and policies 
%%%%  of the journal take precedence. A detailed User Manual is available in the 
%%%%  template package for technical guidance.
%%%%%=============================================================================%%%%

%\jyear{2021}%

%% as per the requirement new theorem styles can be included as shown below
\theoremstyle{thmstyleone}%
\newtheorem{theorem}{Theorem}%  meant for continuous numbers
%%\newtheorem{theorem}{Theorem}[section]% meant for sectionwise numbers
%% optional argument [theorem] produces theorem numbering sequence instead of independent numbers for Proposition
\newtheorem{proposition}[theorem]{Proposition}% 
%%\newtheorem{proposition}{Proposition}% to get separate numbers for theorem and proposition etc.

\theoremstyle{thmstyletwo}%
\newtheorem{example}{Example}%
\newtheorem{remark}{Remark}%

\theoremstyle{thmstylethree}%
\newtheorem{definition}{Definition}%

\raggedbottom
%%\unnumbered% uncomment this for unnumbered level heads

\begin{document}

\title[Article Title]{Non-Markovian Quantum Gate Set Tomography}

%%=============================================================%%
%% Prefix	-> \pfx{Dr}
%% GivenName	-> \fnm{Joergen W.}
%% Particle	-> \spfx{van der} -> surname prefix
%% FamilyName	-> \sur{Ploeg}
%% Suffix	-> \sfx{IV}
%% NatureName	-> \tanm{Poet Laureate} -> Title after name
%% Degrees	-> \dgr{MSc, PhD}
%% \author*[1,2]{\pfx{Dr} \fnm{Joergen W.} \spfx{van der} \sur{Ploeg} \sfx{IV} \tanm{Poet Laureate} 
%%                 \dgr{MSc, PhD}}\email{iauthor@gmail.com}
%%=============================================================%%

\author[1,3,4]{\fnm{Ze-Tong} \sur{Li}}

\author[1,3,4]{\fnm{Cong-Cong} \sur{Zheng}}

\author[5]{\fnm{Fan-Xu} \sur{Meng}}

\author[2,3,4,6]{\fnm{Zai-Chen} \sur{Zhang}}

\author*[1,3,4,6]{\fnm{Xu-Tao} \sur{Yu}}\email{yuxutao@seu.edu.cn}

% \equalcont{These authors contributed equally to this work.}
\affil[1]{\orgdiv{State Key Laboratory of Millimeter Waves}, \orgname{Southeast University}, \orgaddress{\city{Nanjing}, \postcode{210096}, \country{China}}}

\affil[2]{\orgdiv{National Mobile Communications Research Laboratory}, \orgname{Southeast University}, \orgaddress{\city{Nanjing}, \postcode{210096}, \country{China}}}

\affil[3]{\orgdiv{Frontiers Science Center for Mobile Information Communication and Security}, \orgname{Southeast University}, \orgaddress{\city{Nanjing}, \postcode{210096}, \country{China}}}

\affil[4]{\orgdiv{Quantum Information Center}, \orgname{Southeast University}, \orgaddress{\city{Nanjing}, \postcode{210096}, \country{China}}}

\affil[5]{\orgdiv{College of Artificial Intelligence}, \orgname{Nanjing Tech University,}, \orgaddress{\city{Nanjing}, \postcode{211800}, \country{China}}}

\affil[6]{\orgname{Purple Mountain Lab}, \orgaddress{\city{Nanjing}, \postcode{211111}, \country{China}}}


%%==================================%%
%% sample for unstructured abstract %%
%%==================================%%

\abstract{Engineering quantum devices requires reliable characterization of the quantum system including qubits, quantum operations (aka instruments) and the quantum noise. Recently, quantum gate set tomography (GST) has emerged as a promissing technique to self-consistently describe the quantum states, gates and measurements. However, non-Markovian correlations between the quantum system and environment cause the reliability regression of GST. It is essential to simultaneously describe the gate set and non-Markovian correlations. To this end, we first propose a self-consistent operational method, named instrument set tomography (IST), for non-Markovian GST. Based on the stochastic quantum process, the instrument set is defined to describe instruments, the initial state, and non-Markovian system-environment (SE) correlations. First, we propose a linear inversion IST (LIST) to detect and describe the disharmony of linear relationship of instruments and SE correlations with gauge freedom. 
However, LIST cannot always determine physical implementable instrument set because of the absence of constraints. Then, a physically constrained statistical method based on the miximum likelihood estimation for IST (MLE-IST) is proposed with polynomial number of parameters with respect to the Markovian order. It shows significant flexibility that suit for different types of device, e.g. noisy intermediate-scale quantum (NISQ) devices, by adjusting the model and constraints. The experimental results show the effectiveness of describing instruments and the non-Markovian quantum system. As a result, the IST provides an essential method for benchmarking and developing quantum devices in the aspect of instrument set.}

%%================================%%
%% Sample for structured abstract %%
%%================================%%

% \abstract{\textbf{Purpose:} The abstract serves both as a general introduction to the topic and as a brief, non-technical summary of the main results and their implications. The abstract must not include subheadings (unless expressly permitted in the journal's Instructions to Authors), equations or citations. As a guide the abstract should not exceed 200 words. Most journals do not set a hard limit however authors are advised to check the author instructions for the journal they are submitting to.
% 
% \textbf{Methods:} The abstract serves both as a general introduction to the topic and as a brief, non-technical summary of the main results and their implications. The abstract must not include subheadings (unless expressly permitted in the journal's Instructions to Authors), equations or citations. As a guide the abstract should not exceed 200 words. Most journals do not set a hard limit however authors are advised to check the author instructions for the journal they are submitting to.
% 
% \textbf{Results:} The abstract serves both as a general introduction to the topic and as a brief, non-technical summary of the main results and their implications. The abstract must not include subheadings (unless expressly permitted in the journal's Instructions to Authors), equations or citations. As a guide the abstract should not exceed 200 words. Most journals do not set a hard limit however authors are advised to check the author instructions for the journal they are submitting to.
% 
% \textbf{Conclusion:} The abstract serves both as a general introduction to the topic and as a brief, non-technical summary of the main results and their implications. The abstract must not include subheadings (unless expressly permitted in the journal's Instructions to Authors), equations or citations. As a guide the abstract should not exceed 200 words. Most journals do not set a hard limit however authors are advised to check the author instructions for the journal they are submitting to.}

\keywords{non-Markovian correlation, gate set tomography, quantum tomography}

%%\pacs[JEL Classification]{D8, H51}

%%\pacs[MSC Classification]{35A01, 65L10, 65L12, 65L20, 65L70}

\maketitle

\section{Introduction}
Quantum computing requires engineering reliable and controllable quantum devices that manipulate the quantum states with high fidelity. However, recent quantum devices suffer the non-ignorable quantum noise introduced by the imperfect implementations of quantum gates and the system-environment (SE) correlations \cite{papic2023Error}. Characterization of qubits, operations, and entire processors to analyse the influence of quantum noise plays a significant role in the quantum characterization, verification, and validation (QCVV) and offers basic information for the device manufacturing and calibration. 

Based on different assumptions, many protocols have been proposed for this task under the common skeleton of quantum tomography \cite{banaszek2013Focus,smolin2012Efficient,blume-kohout2010Optimal,koutny2022Neuralnetwork,riebe2006Process,mohseni2008Quantumprocess,surawy-stepney2022Projected,greenbaum2015Introduction,nielsen2021Gate}: (1) prepare a set of experiments described by quantum states, circuits and measurements; (2) gather data by executing the prepared experiments; (3) yeild the target result of quantum states, processes and/or measurements by performing estimation algorithms. Among these tomographic methods, gate set tomography (GST) \cite{greenbaum2015Introduction,nielsen2021Gate} is the most powerful and comprehensive method to operationally and self-consistently characterize quantum gates, 
state preparations and measurements (SPAM) without assuming any component of the experiments to be known previously, while the quantum state tomography (QST) \cite{banaszek2013Focus,smolin2012Efficient,blume-kohout2010Optimal,koutny2022Neuralnetwork} and quantum process tomography (QPT) \cite{riebe2006Process,mohseni2008Quantumprocess,surawy-stepney2022Projected} generally require the full knowledge of not-target parts in the experiments. The GST successfully describes two-time noisy quantum gates by completely positive trace-preserving (CPTP) maps under the Markovian assumption. However, no system is isolated \cite{pollock2018NonMarkovian}. There is sufficient evidence that the non-Markovian multiple time correlation nonnegligibly impacts current generation quantum devices \cite{blume-kohout2017Demonstration,proctor2022Measuring,white2020Demonstration,sarovar2020Detecting}. It not only disturbs the tomography under Markovian model that operations in the past influence the behavior of current operation and result in the theoretical violation of CPTP constraints \cite{proctor2022Measuring,milz2021Quantum}. Moreover, the effectiveness of quantum error-correcting codes can degrade or vanish with the appearance of the non-Markovian correlation \cite{nickerson2019Analysing,clader2021Impact}. Therefore, Markovian two-time CPTP maps are not sufficient to describe entire dynamics of the quantum device. Correlations across multiple time scales should be considered while characterizing the device. 

Based on the quantum stochastic process \cite{milz2021Quantum} representing the multiple time correlation, the non-Markovian system dynamics can be modeled by the system, environment, instruments act on the system, and unitaries act on the system and environment simultaneously. For an experimenter, the only accessible part is the instruments representing interventions on the system including quantum gates and measurements. Hence, an instrument can be represented by completely positive trace-non-increasing (CPTNI) maps. Aiming at operationally describing the time-dependent SE correlations, process tensor tomography (PTT) \cite{pollock2018NonMarkovian,guo2022Reconstructing,milz2018Reconstructing,white2022NonMarkovian} relaxes the Markovian constraint to perform the non-Markovian quantum process tomography. It construct well defined CP process tensor with unit trace by interventions of known instruments. However, the differences between the knowledge and the practical performance of instruments may disturb the reconstruction of the process tensor \cite{white2022NonMarkovian}. A simple example is that PTT may generate inconsistent two process tensors using two set of faulty state-informationally complete instruments (that are sufficient to span the space of quantum state). Consequently, the characterization of real quantum devices requires a self-consistent method to tomographically describe the non-Markovian SE correlation and faulty instruments, which directly motivate this work.

To tackle these issues, we first propose a self-consistent method to perform GST under non-Markovian situation. We call the method instrument set tomography (IST). We first propose the linear inversion IST (LIST) a simple, closed-form algorithm to estimate the instruments as well as the SE correlations represented by the process tensor. Unsurprisingly, the IST still exhibits the gauge freedom as GST. Hence the gauge optimization is required at the end of LIST. Although the estimated result may not satisfy the physical constraints since we introduce no constraint in the gauge optimization, it is consistent to the probability measurement data.
Then, we propose a statistical IST method based on the maximum likelihood estimation (MLE) trying to extract more information from overcomplete measurement data. The MLE-IST models the instruments and SE correlations via a flexible way that can suite for different assumption. By introduce constraints, the results are guaranteed to be physical. It also enables the explicit estimation of unitaries representing the non-Markovian SE correlation and evolution instead of the process tensor. Particularly, we also demonstrate how to implement IST on the current noisy quantum intermediate-scale quantum (NISQ) devices. The experimental results show the effectiveness of characterization of instruments, initial states, and non-Markovian correlations. As a result, the IST provides an essential, self-consistent, and reliable method for benchmarking and developing a quantum device under non-Markovian situation in the aspect of instrument set.

\section{Result}

\subsection{Quantum Stochastic Process and Instrument Set}
Before moveing on to present the IST, we first recall the quantum stochastic process representing the non-Markovian quantum correlation and give definitions for instrument set. For a $d$-dimensional quantum system with non-Markovian correlations, the experimenter intervenes the quantum system at $k$ time steps by CPTNI instruments from 
\begin{align}
  \mathcal{J}^{(t)} := \left\{\mathcal{A}^{(t)}_{0}, \mathcal{A}^{(t)}_{1},\dots, \mathcal{A}^{(t)}_{m_t-1}\right\},
\end{align}
where $t=1,\dots,k$ and $m_t$ is the number of valid instruments at time step $t$. Each intervention of the instrument output a value and transform the quantum state for the next time step. The available instruments at different time steps may be different. Then, the operational open quantum process can be described by a $d$-dimensional system and a $d$-dimensional environment with interventions of instruments on the system at $k$ time steps and SE unitaries evolutions between time steps as depicted in Fig.~\ref{fig:qsp} \cite{pollock2018NonMarkovian}. Note that there is a boundary between the accessible and inaccessible parts of the open quantum dynamics to an experimenter. Specifically, an experimenter can not access the quantum state directly. All information of the quantum state the experimenter obtained should with the help of output values of interventions of instruments. The probability to get a sequence of output values $\bm{x}$ is
\begin{equation}\label{eq:se_evo_prob_tr}
  p_{\bm{x}}=\mathrm{Tr}\left[\mathcal{A}^{(k)}_{x_{k-1}}\bigcirc_{t=0}^{k-2}\left(\mathcal{U}_{t:t+1}\mathcal{A}^{(t)}_{x_t}\right)\left(\rho_{SE}^{(0)}\right)\right],
\end{equation}
where $x_t$ is the output value at time step $t$. Without loss of generality, we refer the output value $x_t$ to be the indexes of intruments instead of the actual output value in the following text. Besides, we use the note $\mathcal{A}^{(t)}_{x_t}$ instead of $\mathcal{A}^{(t)}_{x_t}\otimes \mathcal{I}$ for simplicity without confusing. 

From Eq.~\eqref{eq:se_evo_prob_tr}, it is quite clear that the probability can be determined when the instruments, the SE unitary dynamics, and the initial state are given. Therefore, the instrument set describing the operational open quantum dynamics of the quantum device can be defined as
\begin{align}\label{eq:instrument_set_full}
  \mathfrak{I}_{\mathrm{full}} := \left\{\mathcal{J}, \mathcal{U},\rho^{(0)}_{SE}\right\},
\end{align}
where $ \mathcal{J}:=\left\{\mathcal{J}^{(t)}\right\}_{t=0}^{k-1}$ and $\mathcal{U}:=\left\{\mathcal{U}_{t:t+1}\right\}_{t=0}^{k-2}$.
This full definition explicitly depends on the inaccessible initial state and the SE unitary evolution between time steps in which the experimenter may be insterested. However, explicitly characterization of the initial state and the SE unitarie is difficult.

Benifit from the process tensor $\mathcal{T}$ representing the inaccessible parts \cite{pollock2018NonMarkovian,milz2021Quantum,white2022NonMarkovian}, the probability to get $\bm{x}$ can be described as 
\begin{align}\label{eq:pt_def}
  p_{\bm{x}} = \mathcal{T}\left(\mathcal{A}^{(0)}_{x_0},\dots,\mathcal{A}^{(k-1)}_{x_{k-1}}\right),
\end{align}
implying the sufficiency to determine the measurement probability by given $\mathcal{T}$ and $\mathcal{J}$. Therefore, the reduced instrument set can be defined as 
\begin{align}\label{eq:instrument_set_reduced}
  \mathfrak{I}_{\mathrm{reduced}}:=\left\{\mathcal{J}, \mathcal{T}\right\}.
\end{align}

These two definitions of instrument set will be used to propose the IST with clear declaration. In the following, we always use the pauli transfer matrix (PTM) representation to describe instruments, quantum states and process tensors. Particularly, notations $A$ and $\vert \rho\rangle\!\rangle$ are used to indicate the instrument $\mathcal{A}$ and the quantum state $\rho$. Moreover, the PTM representation of the process tensor $\Upsilon_{\mathcal{T}}$ is defined as
\begin{align}\label{eq:se_evo_prob_pt}
  p_{\bm{x}} = \mathrm{Tr}\left[\Upsilon_{\mathcal{T}}^\dagger\left(\begin{matrix}A^{(0)}_{x_0}\\\vdots\\A^{(k-1)}_{x_{k-1}}\end{matrix}\right)\right],
\end{align}
where terms in parentheses are defined as
\begin{align}
  \left(\begin{matrix}X_1\\ \vdots \\X_n\end{matrix}\right) = \left(\begin{matrix}X_1, \dots, X_n\end{matrix}\right):=X_1\otimes\dots\otimes X_n
\end{align}
for clearness and simplicity, instead of directly applying the Choi-Jamiołkowski isomorphism (CJI) representation for the notation consistency. It is easy to verify that the PTM and CJI representation of process tensor is equivalent.

% $\chi^{(t)}_{x_t}$ and $\Upsilon_{\mathcal{T}}$ are the Choi-Jamiołkowski isomorphism (CJI) of $\mathcal{A}^{(t)}_{x_t}$ and $\mathcal{T}$ respectively. Therefore, the reduced instrument set can be defined as 
% \begin{align}\label{eq:instrument_set_reduced}
%   \mathfrak{I}_{\mathrm{reduced}}=\left\{\mathcal{J}, \mathcal{T}\right\}.
% \end{align}
% The reduced instrument set is sufficient to describe the non-Markovian quantum dynamics with the interventions of instruments as the full instrument set, since the process tensor are sufficient to describe the initial state and SE unitary evolution that 
% \begin{align}\label{eq:pt_u_convert}
%   \Upsilon_{\mathcal{T}} = \mathrm{Tr}_E \left[\mu_{k-2:k-1}\star \dots \star \mu_{0:1} \star \rho_{SE}^{(0)}\right],
% \end{align}
% where $\mu_{t:t+1}$ is the CJI of $\mathcal{U}_{t:t+1}$ and $\star$ is the link product defined in \cite{chiribella2009Theoretical}. These two definitions of instrument set will be used to propose the IST with clear declaration. 

% Figure environment removed

\subsection{Linear Inversion IST}\label{sec:LIST}
We first propose the linear inversion IST (LIST) based on the reduced instrument set as defined in Eq.~\eqref{eq:instrument_set_reduced}. Focusing on the time step $t$, the measurement probability can be described as
\begin{equation}\label{eq:inst_decomp_tr_basis}
  p^{(t)}_{\alpha,{x_t}} = \mathrm{Tr}\left[B_\alpha^{(t)} A^{(t)}_{x_t}\right],
\end{equation}
where $A^{(t)}_{x_t}$ is the PTM a $d^2\times d^2$ matrix that completely represent the instrument $\mathcal{A}^{(t)}_{x_t}$, $B_\alpha^{(t)}$ is a $d^2\times d^2$ real basis matrix indexed by $\alpha$. Let $\bm{x}^+$ and $\bm{x}^-$ denote the output values before and after time step $t$ in a $k$-time step non-Markovian experiment, respectively. The LIST constructs a bijection $\alpha = f(\bm{x}^+, \bm{x}^-)$ between integer $\alpha$ and the concatenation of vectors $(\bm{x}^+,\bm{x}^-)$ by the adjustment of $\bm{x}^+$ and $\bm{x}^-$ such that $\mathbb{B}^{(t)} = \{B_0^{(t)}, B_1^{(t)},\dots, B_{d^4-1}^{(t)}\}$ is a linear independent basis set. See Method for detail. 

This implies the decomposition of $A^{(t)}_{x_t}$ on the non-orthogonal process-informationally complete basis $\mathbb{B}^{(t)}$, 
\begin{equation}\label{eq:inst_decomp_vec}
  \bm{p}_{x_t}^{(t)} = \begin{bmatrix}
    (\bm{b}_0^{(t)})^\dagger\\
    (\bm{b}_1^{(t)})^\dagger\\
    \vdots\\
    (\bm{b}_{d^4-1}^{(t)})^\dagger\\ 
  \end{bmatrix}\bm{a}^{(t)}_{x_t} = B^{(t)}\bm{a}^{(t)}_{x_t},
\end{equation}
where $\bm{a}^{(t)}_{x_t}$ and $\bm{b}_\alpha^{(t)}$ represent the vectorization of the $A^{(t)}_{x_t}$ and $B_\alpha^{(t)}$, respectively. Note that instruments at time step $t$ share the same $B^{(t)}$. If $B^{(t)}$ is invertible, we can get instruments
\begin{gather}\label{eq:list_recover}
    \Xi^{(t)} =\left(B^{(t)}\right)^{-1} \Gamma^{(t)},
\end{gather}
where $\Xi^{(t)} = [\bm{a}^{(t)}_{0},\bm{a}^{(t)}_{1}, \dots, \bm{a}^{(t)}_{m_t-1}]$ and $\Gamma^{(t)} = [\bm{p}_{0}^{(t)},\bm{p}_{1}^{(t)},\dots,\bm{p}_{m_t-1}^{(t)}]$. PTMs of instruments can be recovered by devectorization of determined $\bm{a}^{(t)}_{x_t}$. 

The instruments are reconstructed by repeating this for each time step. Then, we choose the maximum linear independent set of the instruments at each time step to formulate the process tensor
\begin{align}
  \Upsilon_\mathcal{T} =\sum_{\bm{x}}p_{\bm{x}}\left(\begin{matrix}D^{(0)}_{x_0}\\\vdots\\D^{(k-1)}_{x_{k-1}}\end{matrix}\right),
\end{align}
where $\left\{D^{(t)}_{x_{t}}\right\}$ is the dual set of maximum linear independent set $\left\{A^{(t)}_{x_t}\right\}$ such that $\mathrm{Tr}\left[\left(D^{(t)}_{i}\right)^\dagger A^{(t)}_{j}\right] = \delta_{ij}$. 

The tomography of the instrument set shows gauge freedom up to a set of invertible matrices $\{B^{(t)}\}$ because of the inaccessible initial state and SE unitaries. We can not distinguish the quantum operations up to $\{B^{(t)}\}$ by the probability measurement, because we can obtain a set of instruments and process tensor without violations of measurement probabilities $p_{\bm{x}}$ for each given set of gauge matrices $\{B^{(t)}\}$. See Method for detail.

The gauge optimization is required to provide a reasonable gauge matrices set to determine the tomographic result of instrument set. We assume that the quantum instruments are implemented well that are close to the ideal instruments. Then, the gauge matrix can be optimized by
\begin{equation}\label{eq:gauge_opt_obj_fn}
  B^{(t)} = \arg\min_X \sum_{t} \left\|X\Gamma^{(t)} - \Xi^{(t)}_{\mathrm{knowledge}}\right\|_F,
\end{equation}
where $\Xi^{(t)}_{\mathrm{knowledge}}$ is the knowledge of instruments to the experimenter. Consequently, the tomographic result is
\begin{gather}
  \hat{\mathfrak{I}}=\left\{\hat{\mathcal{J}}, \hat{\mathcal{T}}\right\},\label{eq:result_list}\\
  \hat{\mathcal{J}}=\left\{\left\{A^{(t)}_{0},\dots,A^{(t)}_{m_t-1}\right\}\right\}_{t=0}^{k-1},\\
  \hat{\mathcal{T}}=\Upsilon_\mathcal{T}.
\end{gather}

A few points are worth mentioning. First, it can be seen from Eq.\eqref{eq:list_recover} and Eq.\eqref{eq:gauge_opt_obj_fn} that the tomographic result of instruments is always the prior knowledge at the time step the instruments are linear independent and not overcomplete. In this case, the LIST degrades to the linear inversion PTT and all imperfect implementation of instruments are represented by the process tensor. The LIST shows the power detecting the disharmony of linear relationship when the instruments at a time step are not linear independent. Moreover, the linear inverse method actually determines a self-consistent tomographic result of instruments and the process tensor, but they may not physically implementable. These characteristics result from the absence of constraints in the gauge optimization. We actually can constrain each $B_\alpha^{(t)}$ and/or $A_{x_t}^{(t)}$ to be CPTNI\footnote{The constraints to the instruments are corresponding to the completely positive trace non-increasing assumption of instruments. This can be adjusted along with the instruments' assumptions (CPTP at intermediate time steps on NISQ devices, for example).}. Nevertheless, this will increase computational complexity. Instead, we optimize $B^{(t)}$ over the entire group of real, invertible matrices to strenuously fit the data. This is similar to the linear inverse GST (LGST) \cite{greenbaum2015Introduction} under Markovian situation.

Second, the objective function in Eq.\eqref{eq:gauge_opt_obj_fn} is not convex and may has nonunique global minima especially the instruments at time step $t$ are not process-informationally complete. This indeterminacy is generic in quantum tomography, and appears in QST, QPT and GST as well. Therefore, we adopt the reduced definition of instrument set using the process tensor to avoid explicit introducing of the SPAM gauge freedom at each time step. See Method for detail. However, this gauge freedom objectively exists that we cannot determine the initial state and actual SE unitaries by the LIST but a set of consistent ones. In this case, fixing the gauge provides no additional information about the initial state and S-E unitaries. Therefore, the LIST also derives the consequence that a initialization error can not be distinguished from a faulty measurement (as described in GST) at each time step $t$ \cite{greenbaum2015Introduction}. Moreover, the non-Markovian SE correlation before the intervention of the instrument can not be distinguished from the non-Markovian SE correlation after.

Third, the LIST at each time step requires the a process-informationally complete basis by combining the instrument at other time steps rather than that the system state and the measurement simultaneously and respectively form state-informationally complete basis. This is because the environment carries the information by non-Markovian SE evolutions. See Method for detail. However, it is difficult to confirm that the entanglements before and after a time step are enough to carry the information such that the specified set of time steps has the ability to construct process-informationally complete basis for tomography at the time step, because the SE dynamics are inaccessible for experimenters. In other word, the non-Markovian effect may not so severe that has high possibility to satisfy the condition of composing process-informationally complete basis. Therefore, we still recommend constructing state-informationally complete basis before and after the time step, respectively. Note that this challenge becomes intractable when conducting tomography at time steps closed to the time edge, especially $0$ and $k$, that the former or the later instruments can not form a state-informationally complete basis. The proposed LIST do not tackle this problem. However, the tomographic result are still compatible with the measurement probabilities.

\subsection{Maximum Likelihood Estimation based IST}

LIST provide a quick method to estimate the instruments and the non-Markovian quantum system. However, it may not always give a physical result and is incompatible of working with overcomplete data for constructing basis of decomposition, which could be used to improve the estimate. Moreover, experimenters may interested in more characteristics, for example, the S-E evolutions themself, requiring high flexibility of the model. To tackle these issue, we propose a statistical framework for IST via maximum likelihood estimation (MLE-IST). As a result, the likelihood function of instrument set is derived as
\begin{align}
  l(\hat{\mathfrak{I}})=\sum_{\bm{x}}{\left(\tilde{p}_{\bm{x}}-\hat{p}_{\bm{x}}\right)^2}/{\sigma_{\bm{x}}^2},
\end{align}
where $\tilde{p}_{\bm{x}}$ denote the measurement probability of getting $\bm{x}$ obtained by the experiment, $\sigma_{\bm{x}}^2$ is the sampling variance of $\tilde{p}_{\bm{x}}$, and $\hat{p}_{\bm{x}}$ is the estimator of measurement probability which is modeled by parameters. The MLE-IST exhibit high flexibility estimating the instrument set with physical constraints based on the various assumptions, such as CPTNI for generality or CPTP on NISQ.

Based on the full definition of instrument set in Eq.\eqref{eq:instrument_set_full}, each instrument $\mathcal{A}_{x_t}^{(t)}$ is modeled by a real matrix $\hat{R}_{x_t}^{(t)}\in[-1,1]^{d^2\times d^2}$ as the PTM representation with the CPTNI constraints. More specifically, the CP requires the Choi state of $\hat{R}_{x_t}^{(t)}$ to be positive semidefinite as 
\begin{align}
  \hat{\rho}_{x_t} = \frac{1}{d^2}\sum_{i,j=0}^{d^2-1}[\hat{R}_{x_t}^{(t)}]_{i,j} \left(\begin{matrix}P_j^T\\ P_i\end{matrix}\right) \succcurlyeq 0,
\end{align}
where $P_i$ represents the $i$-th Pauli matrix. The TNI requires the first entry to be $0\le[\hat{R}_{x_t}^{(t)}]_{0,0} \le 1$. The initial S-E state $\vert\hat{\rho}^{(0)}_{SE}\rangle\!\rangle\in[-1,1]^{d^2\times 1}$ is modeled as a real vector with CP and unit-trace constraints. In other word, the corresponding density matrix is positive semidefinite, and the first entry of $\vert\hat{\rho}^{(0)}_{SE}\rangle\!\rangle$ is $1/\sqrt{d}$. Without loss of generality, we assume that the S-E evolutions do not include the operation on the system dimension, which means any evolution on the system only are absorbed into the instruments. Hence, we can use $\bm{\alpha}^{(t:t+1)} \in [-\pi,\pi]^{d^4(d^4-1)}$ to model each $U_{t}$ corresponding to rotation angles of Pauli operators in $\mathbb{P} = \{P^{S}_i \otimes P^{E}_j|i=1,2,\dots,d^4-1,~j=0,1,\dots,d^4-1\}$, where $P_0 = I$ is the identity matrix. Letting $\bm{\sigma}$ represent the vector of Pauli operators, the recovered unitary $\hat{V}_{t:t+1}(\bm{\alpha})$ is defined as the PTM of unitary $\exp\left(\iota (\bm{\alpha}^{(t:t+1)})^T\bm{\sigma}\right)$, where $\iota^2 = -1$. We use the notation $\hat{V}_{{t:t+1}}$ to indicate $\hat{V}_{{t:t+1}}(\bm{\alpha}^{(t:t+1)})$ in the following for simplicity. Hence, the estimator of the probability is given by
\begin{align}
  \hat{p}_{\bm{x}}=\langle\!\langle 0_{SE}\rvert\!\left(\!\begin{matrix}\hat{R}^{(k-1)}_{x_{k-1}}\\I\end{matrix}\!\right) \!\!\prod_{t=0}^{k-2} \!{\hat{V}_{t:t+1}\left(\!\begin{matrix}\hat{R}^{(t)}_{x_t}\\ I\end{matrix}\right)} \!\lvert\hat{\rho}^{(0)}_{SE}\rangle\!\rangle.
\end{align}

Then, the optimization problem describing MLE-IST based on the full definition of instrument set is given by
\begin{align}\label{eq:mle_ist_opt_prob_full}
  \min&_{\substack{\lvert \hat{\rho}^{(0)}_{SE}\rangle\!\rangle,\hat{R}^{(t)}_{x_t}, \bm{\alpha}^{(t:t+1)},
  \forall x_t, t}}~l(\hat{\mathfrak{I}}),\\
  s.t.~& \hat{\rho}_{x_t} = \frac{1}{d^2}\sum_{i,j=0}^{d^2-1}[\hat{R}_{x_t}^{(t)}]_{i,j}\left(\begin{matrix}P_j^T\\ P_i\end{matrix}\right) \succcurlyeq 0, \forall x_t,\tag{C1} \label{eq:full_def_cp_costraint}\\
  &0\le[\hat{R}_{x_t}^{(t)}]_{1,1} \le 1, \forall x_t,\tag{C2} \label{eq:full_def_tni_costraint}\\
  &-1\le[\hat{R}_{x_t}^{(t)}]_{i,j} \le 1, \forall x_t,i,j,\tag{C3} \label{eq:full_def_ptm_val_constraint}\\
  &\hat{\rho}_{SE}^{(0)} = \frac{1}{\sqrt{d}} \sum_{i=0}^{d^4-1} \langle\!\langle i \vert\hat{\rho}^{(0)}_{SE}\rangle\!\rangle P_i \succcurlyeq 0, \tag{C4} \label{eq:full_def_init_state_cp_constraint}\\
  &[\vert\hat{\rho}^{(0)}_{SE}\rangle\!\rangle]_0 = 1/\sqrt{d}, \tag{C5} \label{eq:full_def_init_state_tr1_constraint}\\
  &-\pi \le [\bm{\alpha}^{(t)}]_i \le \pi, \tag{C6} \label{eq:full_def_upval_constraint}
\end{align}
where \eqref{eq:full_def_cp_costraint} and \eqref{eq:full_def_tni_costraint} constraint the instruments to be CP and TNI, respectively, \eqref{eq:full_def_ptm_val_constraint} defines the range of PTM entries, \eqref{eq:full_def_init_state_cp_constraint} and \eqref{eq:full_def_init_state_tr1_constraint} restrict the initial state to be CP and with unit trace, respectively, and \eqref{eq:full_def_upval_constraint} limits the range of paramters of SE unitaries. Consequently, the MLE-IST estimate the insturment set as 
\begin{gather}\label{eq:result_mleist_full}
  \hat{\mathfrak{I}}:= \left\{\hat{\mathcal{J}}, \hat{\mathcal{U}},\vert\hat{\rho}^{(0)}_{SE}\rangle\!\rangle\}\right\}\\
  \hat{\mathcal{J}}= \left\{\left\{\hat{R}^{(t)}_{0},\dots, \hat{R}^{(t)}_{m_t-1}\right\}\right\}_{t=0}^{k-1},\\
  \hat{U} = \left\{\hat{V}_{t:t+1}\right\}_{t=0}^{k-2}
\end{gather}
with $\sum_{t=0}^{k-1}m_t d^4 + (k-1)d^4(d^4-1) + d^4-1$ parameters which is linear with respect to the non-Markovian order $k$.

The model described above makes an isolation of the instruments and SE unitary dynamics that the S-E unitaries include nothing act on the system dimension only. All evolutions on the local system dimension are absorbed into the instruments. Therefore, the result data explicitly link the instruments and the transformaton of the state on system dimension. This may helps the calibration of quantum operations. Moreover, the models of instruments and the SE unitaries are flexible to be manipulated depending on assumptions the experimenter takes and the characteristics of the instrument set the experimenter interested in. For example, the constraints of instruments can be replaced by the CPTP for quantum gates on NISQ devices, while the instruments of measurements are assumed to be vectors. The SE unitary can also be modeled as a CPTP real orthonormal matrix.

It is obvious that the optimization problem is non-convex and may have multiple global optima, because each estimator $\hat{p}_{\bm{x}}$ consists of mulplications of variable matrices resulting in $(k+2)$-order of polynomial with $k$-order of exponential parameters. Hence, a reasonable initialization of parameters is significant for the optimization. We recommend conducting the LIST (or regular MLE-GST under the Markovian assumption if the LIST generates a nonphysical result) for the initialization of the MLE-IST with identity initialization of $\hat{V}_t$.

Additionally, the MLE-IST can also work with reduced instrument set. However, it requires $\mathcal{O}(d^{4k})$ parameters which is exponential with respect to the non-Markovian order $k$. It is intractable to solve the problem with exponentially increasing number of parameters. Therefore, we propose the reduced instrument set MLE-IST framework but do not implement it for simulations and experiments.

% \subsubsection*{MLE-IST with reduced instrument set}

% While performing MLE-IST with reduced instrument set (reduced MLE-IST), an instrument $\mathcal{A}_{x_t}^{(t)}$ is modeled as it in the full instrument set MLE-IST with CPTNI constraints. The process tensor is modeled by $\hat{\Upsilon}_\mathcal{T} \in [-1,1]^{d^{2k}\times d^{2k}}$ with CP and casuality constraints. Specifically, the casuality requires $\langle\!\langle \hat{\Upsilon} \vert 0\rangle\!\rangle = 1$ and
% \begin{gather}
%   \langle\!\langle \hat{\Upsilon} \vert P_{\mathrm{ban}}\rangle\!\rangle = 0, \\
%   \forall P_{\mathrm{ban}} := I^{\otimes{2t+1}} \otimes \left(\begin{matrix} \tilde{Q}_{2t+2} \\Q_{2t+3}\\\vdots\\ Q_{2k-1}\end{matrix}\right), \forall t,\\
%   \tilde{Q} \in \left\{P_1,\dots,P_{d^2-1}\right\},\\
%   Q \in \left\{P_0,P_1,\dots,P_{d^2-1}\right\}.
% \end{gather}
% Then, the estimator is modeled by
% \begin{align}
%   \hat{p}_{\bm{x}}=\langle\!\langle \hat{\Upsilon} \vert\left(\begin{matrix}\vert\hat{\chi}^{(0)}_{x_0}\rangle\!\rangle\\\vdots\\\vert\hat{\chi}^{(k-1)}_{x_{k-1}}\rangle\!\rangle\end{matrix}\right).
% \end{align}

% Consequently, the optimization problem of the reduced MLE-IST is given by
% \begin{align}\label{eq:mle_ist_opt_prob_reduced}
%   \min&_{\substack{\langle\!\langle \hat{\Upsilon} \vert, \vert\hat{\chi}^{(t)}_{x_t}\rangle\!\rangle,\forall x_t, t
%   }}~l(\hat{\mathfrak{I}})\\
%   s.t.~& \hat{\chi}^{(t)}_{x_t} = \frac{1}{\sqrt{d}} \sum_{i=0}^{d^4-1} \langle\!\langle i \vert\hat{\chi}^{(t)}_{x_t}\rangle\!\rangle P_i \succcurlyeq 0, ~ \forall{t,x_t},\tag{C7} \label{eq:reduced_def_inst_cp}\\
%   & 0 \le [\vert\hat{\chi}^{(t)}_{x_t}\rangle\!\rangle]_0 \le 1, ~\forall{t,x_t} \tag{C8} \label{eq:reduced_def_inst_tni}\\
%   & \langle\!\langle \hat{\Upsilon} \vert P_{\mathrm{ban}}\rangle\!\rangle = 0, ~\forall P_{\mathrm{ban}},\tag{C9} \label{eq:reduced_def_casuality_1}\\
%   & \langle\!\langle \hat{\Upsilon} \vert 0\rangle\!\rangle = 1, \tag{C10} \label{eq:reduced_def_casuality_2}
% \end{align}
% where \eqref{eq:reduced_def_inst_cp} and \eqref{eq:reduced_def_inst_tni} constraint the CPTNI of instruments, \eqref{eq:reduced_def_casuality_1} and \eqref{eq:reduced_def_casuality_2} guarantee the casuality of process tensor. See Method for detail. The reduced MLE-IST requires $\sum_{t=0}^{k-1}m_t d^4 + d^{4k} - \frac{d^{2k}-d^2}{d+1}$ parameters to estimate the instrument set as described in Eq.~\eqref{eq:result_list} after the devectorization of instruments.

% Obviously, this formulation constraints the instrument set in a quite simple form that all inaccessible initial state and SE unitary dynamics are modeled in a vector represented process tensor. However, it requires $\mathcal{O}(d^{4k})$ parameters which is exponential with respect to the non-Markovian order $k$. It is intractable to solve the problem with exponentially increasing number of parameters. Therefore, we propose the reduced instrument set MLE-IST framework but do not implement it for simulations and experiments. The flexibility to adjust constraints to fit the various assumptions still holds. Moreover, this optimization problem is still non-convex and may have multiple global optima, since each estimator consists of $(k+1)$-order of polynomial parameters.

\subsection{Performing IST on NISQ Devices}
A typical NISQ device execute a given quantum circuit consisting of CPTP intermidiate operations and a measurement at the end. Hence, instruments at time step $t$ consist of a set of CPTP operations and a set of measurements,
\begin{align}\label{eq:nisq_inst}
  \mathcal{J}^{(t)} := \left\{\left\{\mathcal{A}^{(t)}_{x_t}\right\}, \left\{\mathcal{M}_{x_t}^{(t)}\right\}\right\}.
\end{align}

Hence, the measurement probability for a $k$-time step non-Markovian quantum circuit is given by
\begin{align}\label{eq:nisq_exp}
  p_{\bm{x}} = \left(\begin{matrix}\langle\!\langle M^{(k-1)}_{x_{k-1}}\vert\\\langle\!\langle 0_E\vert\end{matrix}\right)\prod_{t=0}^{k-2} {U_{t:t+1}\left(\begin{matrix}A^{(t)}_{x_t}\\ I\end{matrix}\right)} \vert\rho^{(0)}_{SE}\rangle\!\rangle.
\end{align}
Associating with Eq.\eqref{eq:nisq_exp} and Eq.\eqref{eq:se_evo_prob_tr}, the measurement $\langle\!\langle M_{x_t}^{(t)}\rvert$ is the first row of the PTM of a CP and trace decreasing (TD) instrument with other entries $0$ at the end. Therefore, LIST can be performed as regular non-Markovian situation by representing measurements as regular instruments at the last time step and takes the first row as the result, when the non-Markovian correlation is sufficient to construct process-informationally complete basis by justing the instruments before the time step. However, it is intractable to perform ordinary LIST in the other case. This results from that the measurement should be the last instrument of the circuit. Hence, every time step are considered as the last time step in LIST.

Based on the observation that the PTM matrix of a (CPTD) measurement is always linear independent with the CPTP maps, the LIST can be performed by separately conducting the LIST subroutine for CPTP maps and measurements. The first row of CPTP maps are omitted in the vectorization in Eq.~\eqref{eq:inst_decomp_vec}, leading to the requirement of $d^2(d^2-1)$ measured probabilities per CPTP map and $d^2(d^2-1)\times d^2(d^2-1)$ dimensional gauge matrix. Then, the tomography of measurements are conducted by $d^2$ measured probabilities per measurement and $d^2\times d^2$ dimensional gauge matrix. Probability data and gauge matrix are measured and optimized in the two subroutine independently. Other steps are the same as ordinary LIST.

As for MLE-IST, the model can be simplified to enhance the efficiency. Each measurement can be modeled by a $d^2$ dimensional real row vector $\langle\!\langle \hat{E}_{x_t}\vert\in[-1, 1]^{1\times d^2}$ with positive constraints, i.e., both the matrix $\hat{E}_{x_t}$ the $\langle\!\langle \hat{E}_{x_t}\vert$ represents and $I-\hat{E}_{x_t}$
are positive semidefinite. Each intermediate instrument can be modeled by $d^2\times (d^2-1)$ parameters with CP constraint, since the TP constraint implies that the first row of $\hat{R}_{x_t}^{(t)}$ is $[1,0,0,\dots,0]$. Then, the estimator of probability is given by
\begin{align}\label{eq:nisq_estimator}
  \hat{p}_{\bm{x}} = \left(\begin{matrix}\langle\!\langle \hat{E}^{(k-1)}_{x_{k-1}}\vert\\\langle\!\langle 0_E\vert\end{matrix}\right)\prod_{t=0}^{k-2} {\hat{V}_{t:t+1}\left(\begin{matrix}\hat{R}^{(t)}_{x_t}\\ I\end{matrix}\right)} \vert\hat{\rho}^{(0)}_{SE}\rangle\!\rangle.
\end{align}

Consequently, the optimization problem for MLE-IST on NISQ devices can be described as
\begin{align}
  \min&_{\substack{\lvert \rho^{(0)}_{SE}\rangle\!\rangle, \langle\!\langle E_{x_t}^{(t)} \vert,R^{(t)}_{x_t},V_{t:t+1}, 
  \forall x_t, t}}~l(\hat{\mathcal{I}}),\\
  s.t.~& \eqref{eq:full_def_cp_costraint}, \eqref{eq:full_def_ptm_val_constraint},\eqref{eq:full_def_init_state_cp_constraint},\eqref{eq:full_def_init_state_tr1_constraint}, \eqref{eq:full_def_upval_constraint}\notag\\
  &[R_{x_t}^{(t)}]_{0,i} = \delta_{0,i}, \forall x_t, i,\tag{C7}\label{eq:cptp_constraint}\\
  &\hat{E}^{(t)}= \frac{1}{\sqrt{d}} \sum_{i=0}^{d^2-1} \langle\!\langle E_{x_t}^{(t)} \vert i\rangle\!\rangle P_i \succcurlyeq 0,~ \forall t,\tag{C8}\label{eq:mea_positive}\\
  &I-\hat{E}^{(t)} \succcurlyeq 0,~ \forall t,\tag{C9}\label{eq:comp_mea_positive}
\end{align}
where Eq.~\eqref{eq:cptp_constraint} is the CPTP constraint, Eq.~\eqref{eq:mea_positive} and Eq.~\eqref{eq:comp_mea_positive} are positive constraints of measurements.

\subsection{Experiment Result}

We first conduct a $5$-time-step single-qubit LIST simulation with overcomplete instruments and SE unitaries $R_{ZZ}(0.2)$ for all time step. Details of instruments and SE unitaries are given in the Method and depicted in Fig.~\ref{fig:ideal_inst_set}. Note that the knowledge and the basic implementation of $\mathcal{A}_4$ and $\mathcal{A}_5$ are not identical. 

The tomographic result of CPTP maps are depicted in the Fig.~\ref{fig:u1_list_ptm}. The difference between knowledge and implementation of $\mathcal{A}_4$ and $\mathcal{A}_5$ are detected. However, the disharmony not only influence the tomographic results of $\mathcal{A}_4$ and $\mathcal{A}_5$, since the LIST cannot distinguish which instrument is correctly implemented. Besides, the nonunique global optima leads to the results consistent with the probability but different from corresponding PTMs in Fig.~\ref{fig:ideal_inst_set}. 

% Figure environment removed

% Figure environment removed


Then, a $5$-time-step single-qubit MLE-IST is simulated with perfect and imperfect implemented complete instruments and SE unitaries $R_{ZZ}(0.2)$ for all time step. As depicted in Fig.~\ref{fig:perfect_inst_set} and Fig.~\ref{fig:imperfect_inst_set}, the tomographic results shows that the IST methods effectively reconstruct the instrument set. However, there are non-ignorable differences between the setups and the results of unitaries, initial state and measurement at the start and the end time step in the imperfect scenario. The sufficiency of constructing process-informationally complete decomposition basis influences the tomographic result in a subtle way. There are more results of unitary evolutions and initial quantum states that fit the probability data when the sufficiency is not satisfied. As a consequence, the output may not meet the experimenter's expectation, but is loyal to the data. 

Moreover, we demonstrate the instrument set of $4$-time-step single-qubit MLE-IST on the real quantum device of IBM Quantum Experience (QX) with complete instruments in Fig~\ref{fig:ibm_lima_inst_set}. Each time step consists of 10 time slots of single qubit gate depending on the quantum hardware. The result shows the potentiality guiding the quantum device engineering.

% Figure environment removed
% Figure environment removed
% Figure environment removed

\section{Method}

\subsection{Decomposition of Instruments for LIST}

By Pauli transfer matrix (PTM) representation defined in \cite{greenbaum2015Introduction}, the probability of getting $\bm{x}$ as described in Eq.\eqref{eq:se_evo_prob_tr} can be reformed as
\begin{align}
  p_{\bm{x}} =& \langle\!\langle0_{SE}\vert \!\left(\!\!\begin{matrix}A^{(k-1)}_{x_{k-1}}\\ I\!\end{matrix}\right)\! \prod_{t=0}^{k-2}U_{t:t+1}\!\left(\begin{matrix}\!\!A^{(t)}_{x_t}\\ I\end{matrix}\right) \!\vert \rho^{(0)}_{SE}\rangle\!\rangle,
  \label{eq:nm_exp_ptm}
\end{align}
where $\vert \bullet \rangle\!\rangle$ and $\langle\!\langle \bullet\vert$ are the superoperators of a quantum state and a positive operator-valued measurement (POVM) operator, $A^{(t)}_{x_t}$ and $U_{t:t+1}$ are the PTM representations of $\mathcal{A}^{(t)}_{x_t}$ and $\mathcal{U}_{t:t+1}$, respectively, and $I$ is the identity. When performing tomography at time step $t$, the probability of getting $x_t$ with $\bm{x}^{+}$ and $\bm{x}^{-}$ is
\begin{align}
  &p_{\bm{x^+}\bm{x^-}}(x_t)\\
  &\begin{aligned}=&\langle\!\langle 0_{SE}\vert \left[\prod_{i=1}^{k-t-2}\left(\begin{matrix}A^{(t+i)}_{x^{+}_i}\\ I\end{matrix}\right)U_{t+i-1:t+i}\right]  \\
    &\left(\begin{matrix}A^{(t)}_{x_t}\\ I\end{matrix}\right)\left[\prod_{j=1}^{t-1}U_{j:j+1}(A^{(j)}_{x^{-}_j}\otimes I)\right] \vert \rho^{(0)}_{SE}\rangle\!\rangle\end{aligned}\\
  =&\langle\!\langle 0_{SE}\vert F_{\bm{x}^+} \left(\begin{matrix}A^{(t)}_{x_t}\\ I\end{matrix}\right)F_{\bm{x}^-}\vert \rho^{(0)}_{SE}\rangle\!\rangle\\
  =&\langle\!\langle F_{\bm{x}^+}^{SE}\vert  \left(\begin{matrix}A^{(t)}_{x_t}\\ I\end{matrix}\right)\vert F_{\bm{x}^-}^{SE}\rangle\!\rangle\\
  =&\sum_{ij}\langle\!\langle F^S_{\bm{x}^+,i}\vert A^{(t)}_{x_t}\vert F^S_{\bm{x}^-,j}\rangle\!\rangle \langle\!\langle F^E_{\bm{x}^+,i}\vert F^E_{\bm{x}^-,j}\rangle\!\rangle\\
  =&\mathrm{Tr}\left[\!\sum_{ij}\langle\!\langle F^E_{\bm{x}^+\!,i}\vert F^E_{\bm{x}^-\!,j}\rangle\!\rangle \vert F^S_{\bm{x}^-\!,j}\rangle\!\rangle\langle\!\langle F^S_{\bm{x}^+\!,i}\vert A^{(t)}_{x_t}\!\right],\label{eq:inst_decomp_detail}
\end{align}
which corresponds to Eq.~\eqref{eq:inst_decomp_tr_basis} implying the decomposition of $A^{(t)}_{x_t}$ on the non-orthogonal basis $\mathbb{B}^{(t)} = \left\{B_{f(\bm{x}^+,\bm{x}^-)} := \sum_{ij}\langle\!\langle F^E_{\bm{x}^+,i}\vert F^E_{\bm{x}^-,j}\rangle\!\rangle \vert F^S_{\bm{x}^-,j}\rangle\!\rangle\langle\!\langle F^S_{\bm{x}^+,i}\vert\right\}$. Reconstruction of $A^{(t)}_{x_t}$ requires $\mathbb{B}^{(t)}$ to be process-informationally complete that there exist at least $d^4$ linear independent basis matrices in it. Then we can obtain the decomposition in Eq.\eqref{eq:inst_decomp_vec} via the vectorization of the matrices. 

% The vectorization method is specified as the superoperator of the Choi state of the operand without loss of generality for the simplicity of process tensor representation. Specifically, let 
% \begin{align}
%   \vert \chi_{x_t}^{(t)}\rangle\!\rangle = \sum_i s_i A_{x_t}^{(t)} \vert i \rangle\!\rangle \otimes \vert i \rangle\!\rangle,\\
%   \langle\!\langle B_{\alpha}^{(t)}\vert = \sum_j s_j \langle\!\langle j \vert B_{\alpha}^{(t)} \otimes \langle\!\langle j\vert,
% \end{align}
% where $\sum_i s_i \vert i \rangle\!\rangle \otimes \vert i \rangle\!\rangle$ is the superoperator of maximum entangled state, $s_i \in \{-1, 1\}$. Then, the measurement probability is represented by 
% \begin{align}
%   \left(\vert p_{x_t}^{(t)}\rangle\!\rangle\right)_\alpha =& \langle\!\langle B_{\alpha}^{(t)}\vert \chi_{x_t}^{(t)}\rangle\!\rangle \\
%   =& \sum_{i,j} s_i s_j \langle\!\langle j \vert B_{\alpha}^{(t)} A_{x_t}^{(t)} \vert i \rangle\!\rangle \otimes \langle\!\langle j\vert i \rangle\!\rangle\\
%   =& \sum_{i} \langle\!\langle i \vert B_{\alpha}^{(t)} A_{x_t}^{(t)} \vert i \rangle\!\rangle\\
%   =& \mathrm{Tr}\left[B_{\alpha}^{(t)} A_{x_t}^{(t)}\right] = p_{\alpha,x_t}^{(t)},
% \end{align}
% which brings about the Eq.\eqref{eq:inst_decomp_ptm_choi}.


\subsection{Gauge Freedom}
The tomography of the instrument set shows gauge freedom up to a set of invertible matrices $\{B^{(t)}\}$ because of the inaccessible initial state and SE unitaries. We can not distinguish the quantum operations by probability measurement up to $\{B^{(t)}\}$. This is because the probability is given by
\begin{align}
  p_{\bm{x}} &= \mathrm{Tr}\left[\Upsilon_{\mathcal{T}}^\dagger\left(\begin{matrix}A^{(0)}_{x_0}\\\vdots\\A^{(k-1)}_{x_{k-1}}\end{matrix}\right)\right]\\
  &=\sum_{\bm{x}}p_{\bm{x}}\prod_{t=0}^{k-1}\mathrm{Tr}\left[D^{(t)\dagger}_{x_t}A^{(t)}_{x_t}\right]\\
  &=\sum_{\bm{x}}p_{\bm{x}}\prod_{t=0}^{k-1}{\bm{d}}^{(t)\dagger}_{x_t}\bm{a}^{(t)}_{x_t}\\
  &=\sum_{\bm{x}}p_{\bm{x}}\prod_{t=0}^{k-1}{\bm{q}}^{(t)\dagger}_{x_t}\left(B^{(t)}\right)^{-1} B^{(t)}\bm{p}^{(t)}_{x_t},
\end{align}
where $\left\{ \bm{q}^{(t)}_{x_{t}}\right\}$ is the dual set of $\left\{\bm{p}^{(t)}_{x_{t}}\right\}$ corresponding to $\left\{A^{(t)}_{x_{t}}\right\}$. This indicates that, for any gauge $\{B^{(t)}\}$, we can obtain a set of instruments and process tensor without violations of measurement probabilities $p_{\bm{x}}$.

Another kind of gauge freedom is the indeterminacy of SE unitaries and initial states. Specifically, we can not distinguish $\langle\!\langle 0_{SE}\vert F_{\bm{x}^+} U(A^{(t)}_{x_t}, I)F_{\bm{x}^-}\vert\rho^{(0)}_{SE}\rangle\!\rangle$ and $\langle\!\langle 0_{SE}\vert F_{\bm{x}^+} ( A^{(t)}_{x_t}, I ) F_{\bm{x}^-} U\vert\rho^{(0)}_{SE}\rangle\!\rangle$, where U is an arbitrary operation that commutes with $(A, I)$. For example, the depolarizing noise on the system with an arbitrary operation on the environment. This generally results in different sets $\{F_{\bm{x}^+}, F_{\bm{x}^-}, \vert\rho^{(0)}_{SE}\rangle\!\rangle\}$ that consistent with the data. We adopt the reduced definition of instrument set using process tensor instead of discussing $F_{\bm{x}^+}$, $F_{\bm{x}^-}$ and $\vert\rho^{(0)}_{SE}\rangle\!\rangle$ themselves to avoid explicit introducing of this type of gauge freedom.

\subsection{Likelihood Function}

Specifying a sequence $\bm{x}$, the probability defined in Eq.~\eqref{eq:se_evo_prob_tr} is measured by repeating the experiment $n_s$ times and record $n_{\bm{x}}$ how many times the desired outputs occur. Therefore, we use the general likelihood function of instrument set
\begin{align}
  \mathcal{L}(\hat{\mathfrak{I}}) = \prod_{\bm{x}}(\hat{p}_{\bm{x}})^{n_{\bm{x}}}(1-\hat{p}_{\bm{x}})^{n_{s}-n_{\bm{x}}},
\end{align}
where $\hat{p}_{\bm{x}}$ is the probability estimator modeled by parameters.

By exploiting the central limit theorem, each term of the likelihood can be rewritten as a normal distribution,
\begin{align}
  \mathcal{L}(\hat{\mathfrak{I}}) = \prod_{\bm{x}}\exp\left[-\frac{\left(\tilde{p}_{\bm{x}}-\hat{p}_{\bm{x}}\right)^2}{\sigma_{\bm{x}}^2}\right],
\end{align}
where $\tilde{p}_{\bm{x}}=n_{bm{x}}/n_s$ represents the measured probability, $\sigma_{\bm{x}}^2=\tilde{p}_{\bm{x}}(1-\tilde{p}_{\bm{x}})/n_s$ is the sampling variance in the measurement $m_{\bm{x}}$. Exploiting the the monotonic logarithm function, maximizing $\mathcal{L}$ is equivalent to minimizing the weighted mean square error (MSE)
\begin{align}
  l(\hat{\mathfrak{I}})=& -\log(\mathcal{L}(\hat{\mathfrak{I}})) = \sum_{\bm{x}}\frac{\left(\tilde{p}_{\bm{x}}-\hat{p}_{\bm{x}}\right)^2}{\sigma_{\bm{x}}^2}.
\end{align}

\subsection{Detail of Numerical Simulation}
We conduct the numerical simulations on classical computers by Python with Pennylane package. The probability $\tilde{p}$ is analytically computed without sampling error appling the model as shown in Fig.~\ref{fig:qsp}. We specify the number of qubits to simulating the environment is the same as the system. Moreover, we introduce the $U_{\textnormal{-}1:0}$ to generate the initial state as $\vert\rho^{(0)}_{SE}\rangle\!\rangle=U_{\textnormal{-}1:0}\vert \rho^{(\textnormal{-}1)}_{SE}\rangle\!\rangle$, where $\vert \rho^{(\textnormal{-}1)}_{SE}\rangle\!\rangle$ is the superoperator of $\vert0_{SE}\rangle\langle0_{SE}\vert$.

Instruments consist of a measurement $\mathcal{M}$ and CPTP operations $\mathcal{A}_i$, $i=0,1,\dots,5$. 
The knowledge of measurements and CPTP maps known to the experimenter is defined as
\begin{align}
  \mathcal{M} &:=\vert 0 \rangle\langle 0\vert,~ \mathcal{A}_0 := I,~ \mathcal{A}_1 := X,\\
  \mathcal{A}_{2} &:=\sqrt{X}R_Z\left(\frac{\pi}{2}\right),~
  \mathcal{A}_3 := R_Z\left(\frac{\pi}{2}\right)\sqrt{X},\\
  \mathcal{A}_4 &:=\sqrt{X}R_Z\left(\frac{\pi}{3}\right),~ \mathcal{A}_5 := R_Z\left(\frac{\pi}{3}\right)\sqrt{X}.
\end{align}
However, the basic implementation of the $\mathcal{A}_4$ and $\mathcal{A}_5$ are $\sqrt{X}R_Z\left(\frac{\pi}{4}\right)$ and $R_Z\left(\frac{\pi}{4}\right)\sqrt{X}$, respectively.

The perfect implementations of instruments are the basic implementations described above. For the imperfect implementations, there is a depolarizing channel and an amplitude damping channel after the basic implementation of each CPTP operation. The parameters of depolarizing and amplitude damping are set increasingly with respect to the time step as $0.05(t+1)$. 

Simulations uses both the complete instruments and overcomplete instruments, where the complete instruments are given by
\begin{equation}\label{eq:complete_instruments}
  \mathcal{J}^{(t)} := \left\{\begin{aligned}&\left\{\mathcal{A}_0, ..., \mathcal{A}_3, \mathcal{M}\right\}, & t\ne k-1,\\
    &\left\{\mathcal{M}\right\}, & t=k-1,\end{aligned}\right.\\
\end{equation}
and the overcomplete instruments are defined as
\begin{equation}\label{eq:overcomplete_instruments}
  \mathcal{J}^{(t)} := \left\{\begin{aligned}&\left\{\mathcal{A}_0, ..., \mathcal{A}_5, \mathcal{M}\right\}, & t\ne k-1,\\
    &\left\{\mathcal{M}\right\}. & t=k-1.\end{aligned}\right.\\
\end{equation}


% For all simulations, SE unitary evolutions are selected from unitaries defined as
% \begin{align}
%   U_0 &:= R_{ZZ}(0.2)R_{YY}(0.2)R_{XX}(0.2),\\
%   U_1 &:= R_{ZZ}(0.2),\\
%   U_2 &:= R_{IX}(0.2)R_{ZZ}(0.2),\\
%   U_3 &:=  R_{IX}(0.2)R_{ZZ}(0.2)R_{XI}(0.2),
% \end{align}
% where $R_{P^SP^E}(\theta):=\exp(\frac{-\iota \theta P^SP^E}{2})$. Specifically, $U_1$, $U_2$, and $U_3$ introduce the non-Markovian quantum correlations that are insufficient to construct process-informationally complete decomposition basis for the first and last time steps but sufficient for intermediate time steps. On the contrary, the non-Markovian correlations of $U_0$ are sufficient to construct the basis for all time steps. Besides, there is no component of $U_0$ and $U_1$ on the system or the environment only. $U_2$ and $U_3$ have the components on the environment and both the system and environment, respectively.


\section{Discussion}\label{sec:discussion}

In this paper, we proposed a framework the instrument set tomography (IST) for quantum gate set tomography under the non-Markovian situation. Based on the quantum stochastic process operationally representing the non-Markovian quantum correlation and evolution, the instrument set is defined in the full and reduced formation. We first proposed a quick linear inversion method based on the reduced instrument set for IST, aka LIST. Consequently, both the disharmony of linear relationship of instruments and the non-Markovian quantum correlations are detected and described with gauge freedom by LIST. However, because of the absence of constraints in the gauge optimization, the result of linear independent instruments is always the prior knowledge when the probability matrix is full rank. Moreover, the result of LIST is not guaranteed to be physical implementable. Then, a statistical method based on the miximum likelihood estimation for IST is proposed as MLE-IST with the ability utilizing overcomplete data. Based on the full instrument set, the MLE-IST tries to explicitly describe the detail of SE correlations with polynomial number of parameters with respect to the Markovian order. The results of MLE-IST is guaranteed to be physical implementable with constraints based on the assumptions of the quantum device. Specifically, we demonstrate how to implement IST on the current noisy quantum intermediate-scale quantum (NISQ) devices. The results of simulations and experiments shows the effectiveness of describing instruments and the non-Markovian quantum system including the initial state and the SE correlations. The IST provide an essential method for benchmarking and developing a quantum device under non-Markovian situation in the aspect of instrument set.
\backmatter

\bmhead{Acknowledgments}

This work is supported by NSFC projects 61960206005 and the Fundamental Research Funds for the Central Universities 2242022k60001, and in part by the National Science Foundation of China under Grant 61871111.


%%===========================================================================================%%
%% If you are submitting to one of the Nature Portfolio journals, using the eJP submission   %%
%% system, please include the references within the manuscript file itself. You may do this  %%
%% by copying the reference list from your .bbl file, paste it into the main manuscript .tex %%
%% file, and delete the associated \verb+\bibliography+ commands.                            %%
%%===========================================================================================%%

\bibliography{citations}% common bib file
%% if required, the content of .bbl file can be included here once bbl is generated
%%\input sn-article.bbl


\end{document}
