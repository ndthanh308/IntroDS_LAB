The set of configurations of the long-range Ising model is, as usual, $\Omega \coloneqq \{-1,1\}^{\Z^d}$. However, each spin interacts with all others, not only its neighbors, so the interaction $\{J_{xy}\}_{x,y\in\Z^d}$ is defined as

\begin{equation}\label{Long-Range Interaction}
    J_{xy} = \begin{cases}
                   \frac{J}{|x-y|^\alpha} &\text{ if }x\neq y,\\
                   0                &\text{otherwise}. 
              \end{cases}
\end{equation}
where $J >0$, $\alpha>d$ and the distance is given by the $\ell_1$-norm. Our results work for more general interactions $\bm{J} = \{J_{xy}\}_{x,y\in\Z^d}$ that are translation invariant and satisfy
\begin{equation*}
    \sum_{\substack{x\in\Z^d \\ |x>1|}} |x_i|J_{0,x} < J_{0,e_i}
\end{equation*}
for every $i=1,\dots, d$, where $e_i$ is a base vector in the $i$-th direction and $x_i$ is the $i$-th coordinate of $x$. The external field is a family $\{h_x\}_{x\in\Z^d}$ of i.i.d. random variables in $(\Tilde{\Omega}, \mathcal{A}, \mathbb{P})$, and every $h_x$ has a standard normal distribution. Our results also hold for more general distributions of $h_x$, see Remark \ref{Rmk: More.general.h_x}. 

We write $\Lambda\Subset \Z^d$ to denote a finite subset of $\Z^d$, and $\mathcal{F}$ is the set of finite subsets. Fixed such $\Lambda$, the \textit{local configurations} are $\Omega_\Lambda\coloneqq \{-1,1\}^\Lambda$. Moreover, given ${\eta\in\Omega}$, the set of local configurations with $\eta$ boundary conditions is ${\Omega_\Lambda^\eta\coloneqq \{\omega\in\Omega_\Lambda : \omega_x=\eta_x, \text{ }\forall x\in\Lambda^c\}}$. The \textit{local Hamiltonian of the random field long-range Ising model} in $\Lambda\Subset\Z^d$ with $\eta$-boundary condition is $H_{\Lambda, \varepsilon}^{\eta}:  \Omega_\Lambda^\eta \to \mathbb{R}$, given by 

\begin{equation}
    H_{\Lambda; \varepsilon h}^{\eta}(\sigma)= -\sum_{x,y\in\Lambda} J_{x,y}\sigma_x\sigma_y - \sum_{x\in \Lambda, y\in\Lambda^c} J_{x,y}\sigma_x\eta_y - \sum_{x\in\Lambda} \varepsilon h_x\sigma_x,
\end{equation}
where $\varepsilon >0$ is a parameter that controls the variance of the external field. Given $\Lambda\Subset\Z^d$, consider $\mathscr{F}_\Lambda$ the $\sigma$-algebra generated by the cylinders supported in $\Lambda$. The main object of study in classical statistical mechanics are the \textit{finite volume Gibbs measures}, which are probability measures in $(\Omega, \mathscr{F}_\Lambda)$, given by 
    \begin{equation}
        \mu_{\Lambda;\beta, \varepsilon h}^\eta(\sigma) = \mathbbm{1}_{\Omega_\Lambda^\eta}(\sigma)\frac{e^{-\beta H_{\Lambda, \varepsilon h}^{\eta}(\sigma)}}{Z_{\Lambda; \beta, \varepsilon}^{\eta}(h)}.
    \end{equation}

Here, $\beta>0$ is the inverse temperature and $Z_{\Lambda; \beta, \varepsilon}^{\eta}$ is the so called \textit{partition function}, defined as 

\begin{equation}
    Z_{\Lambda; \beta, \varepsilon}^{\eta}(h)\coloneqq \sum_{\sigma\in\Omega_\Lambda^\eta} e^{-\beta H_{\Lambda, \varepsilon h}^{\eta}(\sigma)}.
\end{equation}
One important remark is that, since the external field is random, the Gibbs measures are random variables. To explicit the dependence of $\mu_{\Lambda;\beta, \varepsilon h}^\eta$ on $\tilde{\Omega}$, we will often write $\mu_{\Lambda;\beta, \varepsilon h}^\eta[\omega]$, with $\omega$ being a general element of $\tilde{\Omega}$. Two particularly important boundary conditions are given by the configurations $\eta_{+} \equiv +1$ and $\eta_{-} \equiv -1$, and are called $+$ and $-$ boundary conditions, respectively. For these boundary conditions, we can $\mathds{P}$-almost surely define the infinite volume measures by taking the weak*-limit
\begin{equation}
    \mu_{\beta,\varepsilon h}^{\pm}[\omega] = \lim_{n\to\infty} \mu_{\Lambda_n;\beta, \varepsilon h}^{\pm}[\omega],
\end{equation}
where $(\Lambda_n)_{n\in\mathbb{N}}$ is any sequence invading $\Z^d$, that is, for any subset $\Lambda\Subset\mathbb{Z}^d$, there exists $N=N(\Lambda)>0$ such that $\Lambda\subset\Lambda_n$ for every $n>N$.
To  have more than one Gibbs measure, it is enough that $\mu_{\beta,\varepsilon h}^{+}[\omega]\neq  \mu_{\beta,\varepsilon h}^{-}[\omega]$, with $\mathbb{P}$-probability 1, see \cite[Theorem 7.2.2]{Bovier.06}.

Peiers' argument, used to prove phase transition, is based on the idea of erasing contours. Contours are geometric objects in the dual lattice $\mathbb{Z}^d_*$ defined as: denoting $C_x$ the closed unit cube in $\mathbb{R}^d$ centered in $x$, $\mathbb{Z}^d_*$ is the union of all faces $C_x\cap C_y$ with $|x-y| = 1$. Given a configuration, its contours are the maximal connected components of the union of the faces $C_x\cap C_y$ satisfying $\sigma_x \neq \sigma_y$. The set of contours of $\sigma$ is denoted by $\Gamma(\sigma)$, and $\gamma$ denotes a generic element of $\Gamma(\sigma)$. The set of all contours is $\Gamma$ and the \textit{interior} of a contour $\gamma$, denoted $\I(\gamma)$, is the set of points connected to $\infty$ only by paths crossing $\gamma$. Given $n\in\mathbb{N}$, take
\begin{equation*}
    \Gamma_0(n) \coloneqq \{\gamma\in\Gamma \ : \ 0\in\I(\gamma), \  |\gamma| = n\}
\end{equation*}
and $\Gamma_0 = \cup_{n\geq 1}\Gamma_0(n)$. The operation $\tau_{\gamma}$ used to remove a contour $\gamma\in\Gamma(\sigma)$ can be written as a particular case of the following one: given $A\subset\Z^d$, take $\tau_A:\mathbb{R}^{\Z^d} \xrightarrow{} \mathbb{R}^{\Z^d}$ as 
\begin{equation}
    (\tau_A(\sigma))_i \coloneqq \begin{cases}
                        -\sigma_i &\text{if }i\in A,\\
                        \sigma_i   &\text{otherwise},
                      \end{cases}
\end{equation}
for every $i\in\Z^d$. The transformation that erases a contour $\gamma$ is $\tau_\gamma(\sigma) \coloneqq \tau_{\I(\gamma)}(\sigma)$. Following \cite{Ginibre.Grossmann.Ruelle.66} we can bound the difference in the Hamiltonian after erasing a contours, when there is no external field.
\begin{proposition}
    For the long-range Ising model with $\alpha > d+1$ and inverse temperature $\beta>0$, there is a constant $c_1(\alpha)>0$ such that
    \begin{equation}
        H_{\Lambda, 0}^{+}(\tau_{\gamma}(\sigma)) - H_{\Lambda, 0}^{+}(\sigma) \leq - J c_1(\alpha) |\gamma|.
    \end{equation}
\end{proposition}
\begin{proof}
    A straightforward computation shows that
    \begin{align*}
        H_{\Lambda, 0}^{+}(\tau_\gamma(\sigma)) &= -\sum_{\substack{x,y\in \I(\gamma)}} J_{xy}\tau_\gamma(\sigma)_x\tau_\gamma(\sigma)_y - \sum_{\substack{x,y\in \I(\gamma)^c}} J_{xy}\tau_\gamma(\sigma)_x\tau_\gamma(\sigma)_y - \sum_{\substack{x\in \I(\gamma)\\ y\in \I(\gamma)^c}} J_{xy}\tau_\gamma(\sigma)_x\tau_\gamma(\sigma)_y \\
        %
        &= -\sum_{\substack{x,y\in \I(\gamma)}} J_{xy}\sigma_x\sigma_y -\sum_{\substack{x,y\in \I(\gamma)^c}} J_{xy}\sigma_x\sigma_y + \sum_{\substack{x\in \I(\gamma)\\ y\in \I(\gamma)^c}} J_{xy}\sigma_x\sigma_y\\
        & = H_{\Lambda, 0}^{+}(\sigma) + 2\sum_{\substack{x\in \I(\gamma)\\ y\in \I(\gamma)^c}} J_{xy}\sigma_x\sigma_y
    \end{align*}
    Therefore, the difference can be bounded in the following way 
    \begin{align*}
        \frac{1}{2}(H_{\Lambda, 0}^{+}(\tau_{\gamma}(\sigma)) - H_{\Lambda, 0}^{+}(\sigma)) &= \sum_{\substack{x\in \I(\gamma)\\ y\in \I(\gamma)^c}} J_{xy}\sigma_x\sigma_y
        %
        =\sum_{\substack{x\in \fint\I(\gamma), \\   y\in \fext\I(\gamma)^c}} J\sigma_x\sigma_y + \sum_{\substack{x\in \I(\gamma), \\  y\in \I(\gamma)^c \\ |x-y|\geq 2}} J_{xy}\sigma_x\sigma_y\\
        %
        &\leq -J|\gamma| + \sum_{\substack{x\in \I(\gamma) \\  y\in \I(\gamma)^c \\ |x-y|\geq 2}} J_{xy} \\
        &= -J|\gamma| + \sum_{\substack{k\in \Z^d \\ |k|\geq 2}} J_{0,k}|\{\{x,y\} : x\in \I(\gamma), \  y\in \I(\gamma)^c, x-y = k\}|.
    \end{align*}
    For $i=1,\dots, d$, let $\gamma_i$ be the faces of $\gamma$ perpendicular to the direction $e_i$. Using that $$|{\{x,y\} : x\in \I(\gamma), \  y\in \I(\gamma)^c, x-y = k}| \leq \sum_{i=1}^d |k_i||\gamma_i|,$$ 
we get     
\begin{align*}
    H_{\Lambda, 0}^{+}(\tau_{\gamma}(\sigma)) - H_{\Lambda, 0}^{+}(\sigma) &\leq  -2J|\gamma| + 2\sum_{\substack{k\in \Z^d \\ |k|\geq 2}} \frac{J}{|k|^{\alpha+1}}|\gamma|. \\
\end{align*}
Taking $c_1(\alpha)=2(1- \sum_{\substack{k\in \Z^d \\ |k|\geq 2}}\frac{1}{|k|^{\alpha+1}})$, we conclude our proof by noticing that $c_1(\alpha)>0$ if and only if $\alpha > d+1$.
\end{proof}
