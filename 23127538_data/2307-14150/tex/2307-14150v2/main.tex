
\documentclass[12pt,a4paper]{article}
\usepackage[english]{babel}
\usepackage{float}
\usepackage{amsmath, amssymb, amsthm, epsfig, color, mathtools, bbm, dsfont, mathrsfs, bm}
\usepackage[latin1]{inputenc}
\usepackage[T1]{fontenc}
\usepackage{indentfirst}
\usepackage[shortlabels]{enumitem}
\usepackage[title]{appendix}
\usepackage{lmodern}       
\usepackage{comment}  
   % alternative Computer Modern-fonts for computer 
\usepackage{textcomp, upgreek}
\usepackage{setspace}
\usepackage{soul}
\usepackage{accents}
\usepackage{hyperref} %url nas referencias 
\usepackage{geometry}
\usepackage{xcolor}

\usepackage{xr}    %para usar labels de outros documentos 
\usepackage{graphicx} %para inserir imagens
\usepackage[export]{adjustbox} %posiciona imagens
\usepackage{subcaption} %posiciona imagens
\usepackage{mathtools} %usa multiline dentro do align
\usepackage{siunitx}
\usepackage{bm} %for bold in math mode 
\usepackage{comment}


\usepackage{addfont}
\addfont{OT1}{rsfs10}{\rsfs}

%----------------

\usepackage{tikz, tkz-euclide}
\usetikzlibrary{patterns}
\usetikzlibrary{arrows.meta}
\usetikzlibrary{calc}
\usetikzlibrary{decorations.markings}
\usepackage{fancybox}

%-------------------

\usepackage{graphicx}
\graphicspath{ {./images/} }
\newcommand{\Z}{\mathbb{Z}}
\newcommand{\bbr}{\mathbb{R}}
\newcommand{\bbn}{\mathbb{N}}
\newcommand{\caln}{\mathcal{N}}
\newcommand{\cals}{\mathcal{S}}
\newcommand{\lab}{\mathrm{lab}}     
\newcommand{\supp}{\text{supp}\;}
\newcommand{\din}{\partial_{\mathrm{in}}}
\newcommand{\dout}{\partial_{\mathrm{out}}}
\newcommand{\diam}{\mathrm{diam}}
\newcommand{\dis}{\mathrm{dist}}
\newcommand{\bfr}{\mathbf{r}}
\newcommand{\bfq}{\mathbf{q}}
\newcommand{\I}{\mathrm{I}}
\newcommand{\Sp}{\mathrm{sp}}
\newcommand{\f}{\mathrm{f}}
\newcommand{\R}{\mathcal{R}}
\newcommand{\fint}{\partial_{\mathrm{in}}}
\newcommand{\fext}{\partial_{\mathrm{ex}}}
\newcommand{\C}{\mathscr{C}}
\newcommand{\calP}{\mathcal{P}}

\newcommand{\ssum}[1]{
\sum_{\mathclap{\substack{#1}}}
}
\newcommand{\subscript}[2]{$#1 _ #2$}

\def\supp{\mathop{\textrm{\rm supp}}\nolimits}            %Support
\def\d{\mathop{\textrm{\rm d}}\nolimits}                  %integral d
\def\Ext{\mathop{\textrm{\rm Ext}}\nolimits}                  %Exterior
\def\sign{\mathop{\textrm{\rm sign}}\nolimits}                  %sign function
\def\sf{\mathop{\textrm{\rm sf}}\nolimits}            %sf


\newcommand{\be}{\begin{equation}}
\newcommand{\ee}{\end{equation}}


%----------------
\numberwithin{equation}{section}


\addtolength{\hoffset}{-1.5cm} \addtolength{\textwidth}{2cm}
\addtolength{\voffset}{-1.5cm} \addtolength{\textheight}{2cm}
  \newcounter{dummy} \numberwithin{dummy}{section}
  \theoremstyle{plain}
  \newtheorem*{theorem*}        {Theorem}
	\newtheorem*{conjecture*}   {Conjecture}
  \newtheorem{theorem}[dummy]          {Theorem}
  \newtheorem{lemma}[dummy]              {Lemma}
  \newtheorem*{lemma*}          {Lemma}
    \newtheorem{claim}[dummy]         {Claim}
  \newtheorem{corollary}[dummy]           {Corollary}
  \newtheorem{proposition}[dummy]       {Proposition}
  \newtheorem{question}[dummy]           {Question} 
  \newtheorem{remark}[dummy]           {Remark}
  \newtheorem{notation}[dummy]           {Notation}
  \theoremstyle{remark}
    %\newtheorem{notation}[dummy]           {Notation}
  \theoremstyle{definition}
   \newtheorem{definition}[dummy]          {Definition}
%------------------- long left right arrow with word above
\makeatletter

\newcommand\longleftrightarrowfill@{%
  \arrowfill@\leftarrow\relbar\rightarrow}
\makeatother

%% Comman to edit collors
%
%
\definecolor{Red}{cmyk}{0,1,1,0}
\def\red{\color{Red}}
\definecolor{Blue}{cmyk}{1,1,0,0}
\def\blue{\color{Blue}}
\definecolor{DarkBlue}{rgb}{0.1,0.1,0.5}
\definecolor{Red}{rgb}{0.9,0.0,0.1}
\definecolor{DarkGreen}{rgb}{0.10,0.50,0.10}
\definecolor{DarkRed}{rgb}{0.50,0.10,0.10}
\definecolor{bleu}{RGB}{0,140,189}%
\newcommand\bfblue[1]{\textcolor{blue}{\textbf{#1}}}
\newcommand\bfred[1]{\textcolor{red}{\textbf{#1}}}

%-----------------Margem e Formatação---------------

\newgeometry{vmargin={15mm}, hmargin={12mm,17mm}}

%--------------------------------------------------


\newcommand{\eqdef}{\overset{\mathrm{def}}{=\joinrel=}}

\definecolor{vermelho}{RGB}{208,2,27}    %cores dos graficos
\definecolor{verde}{RGB}{126,211,33} %cores dos graficos

\DeclareMathOperator{\arctanh}{arctanh}
\DeclareMathOperator{\h}{\bm{h}}
\DeclareMathOperator{\dist}{\mathrm{d}}
\DeclareMathOperator{\s}{\mathrm{sp}}

\DeclarePairedDelimiter\ceil{\lceil}{\rceil} %\ceil*{}
\DeclarePairedDelimiter\floor{\lfloor}{\rfloor} %\floor*{}




\begin{document}


\begin{center}
{\LARGE Phase Transition for Long-Range Random Field Ising Model in Higher Dimensions}
\vskip.5cm
Lucas Affonso$^{1}$, Rodrigo Bissacot$^{1,2}$, Jo{\~a}o Vitor Maia$^{1}$
\vskip.3cm
\begin{footnotesize}
$^{1}$Institute of Mathematics and Statistics (IME-USP), University of S\~{a}o Paulo, Brazil\\
$^{2}$ Faculty of Mathematics and Computer Science, Nicolaus Copernicus University, Poland\\  
\end{footnotesize}
\vskip.1cm
\begin{scriptsize}
emails: lucas.affonso.pereira@gmail.com, rodrigo.bissacot@gmail.com, joao.vitor.maia@usp.br
\end{scriptsize}

\end{center}

\begin{abstract}
We extend the recent argument from Ding and Zhuang from the nearest-neighbor to long-range interactions and prove the phase transition in the class of ferromagnetic random field Ising models. Our arguments use a proof of phase-transition introduced in \cite{Ginibre.Grossmann.Ruelle.66} and the coarse-graining idea from Fisher, Fr\"ohlich and Spencer. The result shows that the Ding-Zhuang strategy is also useful, in particular, for interactions $J_{xy}=|x-y|^{- \alpha}$ when $\alpha > d+1$ in dimension $d\geq 3$. We can consider i.i.d. random fields with Gaussian or Bernoulli distributions.
\end{abstract}

\section{Introduction}The problem of the presence or absence of phase transition is central in statistical mechanics. To prove the existence of phase transition, the standard idea is to define a notion of contour and use \textit{Peierls' argument} \cite{Peierls.1936}. In the usual Ising model \cite{Ising_25}, particles of the system interact only with their nearest-neighbors. On ferromagnetic long-range Ising models \cite{Anderson_Yuval_69}, there is interaction between each pair of spins in the lattice. The Hamiltonian of the model is given formally by
\begin{equation*}
    H(\sigma) = - \sum_{x,y\in \Z^d}J_{xy}\sigma_x\sigma_y,
\end{equation*}
where $J_{xy}=J|x-y|^{-\alpha}$, $J>0$, $\alpha > d$. It is well-known that the Peierls' argument in dimension 2 implies phase transition for Ising models with nearest-neighbors or long-range interactions when $d\geq 2$, using correlation inequalities. For the unidimensional lattice, it was known that short-range models do not present phase transition. In the long-range case, a different behavior was expected depending on the exponent $\alpha$ (see \cite{Kac_Thompson_69}), but the problem was challenging since contours were first created as multidimensional objects.

In dimension $d=1$, phase transition was proved first in 1969 by Dyson \cite{Dyson.69}, for $\alpha \in (1,2)$, by proving phase transition in an auxiliary model and then using correlation inequalities. In 1982, Fr{\"o}hlich and Spencer \cite{Frohlich.Spencer.82} introduced a notion of one-dimensional contours and then applied the Peierls' argument to show phase transition for the critical value $\alpha = 2$. These contours were inspired by the multiscale techniques previously introduced to study the Berezinskii-Kosterlitz-Thouless transition in two-dimensional continuous spin systems \cite{FS81}. Later, Cassandro, Ferrari, Merola and Presutti  \cite{Cassandro.05} extended the contour argument previously available for $\alpha=2$ to exponents $\alpha\in (3-\frac{\ln 3}{\ln 2}, 2)$, with the additional restriction that the nearest-neighbor interaction is strong, i.e.,  ${J(1)\gg 1}$; this restriction was removed for a subclass of interactions in \cite{Bissacot.Endo.18}. Further results were obtained using contour arguments, such as the decay of correlations, cluster expansions, phase transition with random interactions, etc; some references with these results are \cite{ Cassandro.Merola.Picco.17, Cassandro.Merola.Picco.Rozikov.14, Imbrie.82, Imbrie.Newman.88, Johansson.91}. 

In the multidimensional setting ($d\geq 2$), Ginibre, Grossmann, and Ruelle, in \cite{Ginibre.Grossmann.Ruelle.66}, proved the phase transition for $\alpha > d+1$, using an enhanced version of Peierls' argument and the usual contours. Park proposed a different notion of contour for long-range systems in \cite{Park.88.I, Park.88.II}, extending the Pirogov-Sinai theory available for short-range interactions assuming $\alpha > 3d+1$, although he can also consider Potts models with his methods. Some results in the literature suggest that truly long-range effects appear only when $d < \alpha \leq d+1$, see for instance, \cite{Biskup_Chayes_Kivelson_07}. Recently, Affonso, Bissacot, Endo and Handa \cite{Affonso.2021}, inspired by the ideas from Fr{\"o}hlich and Spencer in \cite{FS81, Frohlich.Spencer.82}, introduced a version of multiscale multidimensional contour and proved phase transition by a contour argument in the whole region $\alpha > d$. They can consider long-range Ising models with deterministic decaying fields, first introduced in the context of nearest-neighbor interactions in \cite{Bissacot_Cioletti_10}. For these models, the lack of analyticity of the free energy does not imply phase transition since these models have the same free energy as the models with zero field. It is expected that fields decaying slowly imply uniqueness. In this setting, a contour argument is useful for proofs of phase transitions as well for uniqueness, some papers with models with deterministic decaying fields are \cite{Aoun_Ott_Velenik_23, Bissacot_Cass_Cio_Pres_15, Bissacot.Endo.18, Cioletti_Vila_2016}.

The Random Field Ising model (RFIM) \cite{Imry.Ma.75} is the nearest-neighbor Ising model with an additional external field acting on each site $(h_x)_{x\in\Z^d}$ that is a family of i.i.d. Gaussian random variable with mean 0 and variance 1. Formally, the Hamiltonian of the model is given by
\begin{equation*}
    H(\sigma) = - \sum_{\substack{x,y\in \Z^d \\|x-y|=1}}J\sigma_x\sigma_y  - \varepsilon\sum_{x\in\Z^d}h_x\sigma_x,
\end{equation*}
where $J>0$, $\varepsilon>0$, $\alpha > d$ and $d \geq 1$. A detailed account of the history of the phase transition problem for this model, as well as detailed proofs, was given in \cite{Bovier.06}. Here we present a brief overview.

During the 1980s, the question of the specific dimension where phase transition for the RFIM should happen attracted much attention and was a topic of heated debate. Two convincing arguments were dividing the physics community. One of them, due to Imry and Ma \cite{Imry.Ma.75}, was a non-rigorous application of the Peierls' argument together with the use of the isoperimetric inequality. The key idea of Peierls' argument is to define a notion of contour and calculate the energy cost of "erasing" each contour, i.e., the energy cost of flipping all spins inside the contour. When there is no external field, that energy necessary to flip the spins in a region $A\subset \Z^d$ is of the order of the boundary $|\partial A|$. When we add an external field, we get an extra cost depending on this field. Imry and Ma argued that this cost should be approximately $\sqrt{|A|}$, which is smaller than $|\partial A|$ for all regions only when $d\geq 3$, so this should be the region where phase transition occurs. The other argument, due to Parisi and Sourlas \cite{Parisi.Sourlas.79}, based on dimensional reduction, predicted that the $d$-dimensional RFIM would behave like the $d-2$-dimensional nearest-neighbor Ising model, therefore presenting phase transition only when $d\geq 4$. 

The question was settled by two celebrated papers showing that Imry and Ma's prediction was correct. First, in 1988, Bricmont and Kupiainen \cite{Bricmont.Kupiainen.88} showed that there is phase transition almost surely in $d\geq3$, for low temperatures and variance $\varepsilon$ small enough. Their proof uses a rigorous renormalization group analysis for the short-range case and it is considered involved. Still, they claimed that the result works for any model with a suitable contour representation and centered sub-gaussian external field. Later on, Aizenman and Wehr \cite{Aizenman.Wehr.90} proved uniqueness for $d\leq 2$. For detailed proofs of these results, we refer the reader to \cite{Bovier.06} (see also \cite{Berretti.85, Camia.18, Frohlich.Imbre.84,  Klein.Masooman.97} for more uniqueness results). 

Recently, Ding and Zhuang, see \cite{Ding2021}, provided a simpler proof of the phase transition, not using RGM. And in  \cite{Ding.Liu.Xia.22}, Ding, Liu and Xia proved that if $\beta_c(d)$ is the critical inverse of the temperature of the Ising model with no field, for all $\beta>\beta_c(d)$ there exists a critical value $\varepsilon_0(d, \beta)$ such that the RFIM with $\varepsilon \leq \varepsilon_0$ presents phase transition. 

In the present paper, we are considering a long-range Ising model with a random field, whose Hamiltonian is given formally by
\begin{equation*}
    H(\sigma) = - \sum_{x,y\in \Z^d}J_{xy}\sigma_x\sigma_y - \varepsilon\sum_{x\in\Z^d}h_x\sigma_x,
\end{equation*}
where $J_{xy}=J|x-y|^{-\alpha}$, $J, \varepsilon>0$, $\alpha > d$ and $h_x\in\mathbb{R}$, $d\geq 3$.
Until now, the only known result in the long-range setting is for the one-dimensional long-range Ising model with a random field, by Cassandro, Orlandi, and Picco \cite{Cassandro.Picco.09}. They used the contours of \cite{Cassandro.05} to show the phase transition for the model when $\alpha\in (3-\frac{\ln 3}{\ln 2}, \frac{3}{2})$, under the assumption $J(1) \gg 1$. We stress that, as remarked by Aizenman, Greenblatt, and Lebowitz \cite{Aizenman_Greenblatt_Lebowitz_2012}, although their argument does not work for the whole region for the exponent $\alpha$, the phase transition holds for values close to the critical value $\alpha=3/2$, since by the Aizenman-Wehr theorem we know that there is uniqueness for $\alpha>3/2$.

The argument from Ding and Zhuang in \cite{Ding2021}, for $d\geq3$, involves controlling the probability of a bad event, which is closely related to controlling the quantity $$\sup_{\substack{0\in A\subset\Z^d \\ A \text{ connected }}}\frac{\sum_{x\in A}h_x}{|\partial A|},$$ known as the greedy animal lattice normalized by the boundary. The greedy animal lattice normalized by the size, instead of the boundary, was extensively studied for general distributions of $(h_x)_{x\in\Z^d}$, see \cite{Cox_Gandolfi_Griffin_Kesten_93, Gandolfi_Kesten_94, Hammond_06, Martin_02}. When we normalize by the boundary, an argument by Fisher, Fr\"{o}hlich and Spencer \cite{FFS84} shows that the expected value of the greedy animal lattice is constant. In dimension $d=2$, the expected value is not finite, see \cite{Ding.Wirth.20}. The supremum is taken over connected regions containing the origin since the interiors of the usual Peierls contours are of this form.


For the long-range model, the interior of contours is not necessarily connected. In fact, long-range contours may have considerably large diameters with respect to their size, so their interiors can be very sparse. To avoid this, we define contours, strongly inspired by the $(M,a,r)$-partition in \cite{Affonso.2021}, using a multiscaled procedure that assures that the contours have no cluster with small density.  With them, we generalize the arguments by Fisher-Fr\"{o}hlich-Spencer \cite{FFS84}, and prove that the expected value of the greedy animal lattice is constant, even considering regions not necessarily connected in the supremum. Then, we prove the phase transition for $d\geq 3$. The main result of this paper is the following.
\begin{theorem*}Given $d\geq 3$, $\alpha>d$, there exists $\beta_c\coloneqq\beta(d, \alpha)$ and $\varepsilon_c\coloneqq\varepsilon(d, \alpha)$ such that, for $\beta >\beta_c$ and $\varepsilon\leq \varepsilon_c$, the extremal Gibbs measures $\mu_{\beta, \varepsilon}^+$ and $\mu_{\beta, \varepsilon}^-$ are distinct, that is, $\mu_{\beta, \varepsilon}^+ \neq \mu_{\beta, \varepsilon}^-$ $\mathbb{P}$-almost surely. Therefore the long-range random field Ising model presents phase transition.
\end{theorem*}

This paper is divided as follows. In Section 2, we define the model and the contours, and suitable generalizations to the constructions in \cite{Affonso.2021} are introduced.  In Section 3, we define two bad events of the external field and prove that they occur with a small probability.  In Section 4, we present the proof of the phase transition.


\section{The model and Contours}  The set of configurations of the long-range Ising model is, as usual, $\Omega \coloneqq \{-1,1\}^{\Z^d}$. However, each spin interacts with all others, not only its neighbors, so the interaction $\{J_{xy}\}_{x,y\in\Z^d}$ is defined as

\begin{equation}\label{Long-Range Interaction}
    J_{xy} = \begin{cases}
                   \frac{J}{|x-y|^\alpha} &\text{ if }x\neq y,\\
                   0                &\text{otherwise}. 
              \end{cases}
\end{equation}
where $J >0$, $\alpha>d$ and the distance is given by the $\ell_1$-norm. Our results work for more general interactions $\bm{J} = \{J_{xy}\}_{x,y\in\Z^d}$ that are translation invariant and satisfy
\begin{equation*}
    \sum_{\substack{x\in\Z^d \\ |x>1|}} |x_i|J_{0,x} < J_{0,e_i}
\end{equation*}
for every $i=1,\dots, d$, where $e_i$ is a base vector in the $i$-th direction and $x_i$ is the $i$-th coordinate of $x$. The external field is a family $\{h_x\}_{x\in\Z^d}$ of i.i.d. random variables in $(\Tilde{\Omega}, \mathcal{A}, \mathbb{P})$, and every $h_x$ has a standard normal distribution. Our results also hold for more general distributions of $h_x$, see Remark \ref{Rmk: More.general.h_x}. 

We write $\Lambda\Subset \Z^d$ to denote a finite subset of $\Z^d$, and $\mathcal{F}$ is the set of finite subsets. Fixed such $\Lambda$, the \textit{local configurations} are $\Omega_\Lambda\coloneqq \{-1,1\}^\Lambda$. Moreover, given ${\eta\in\Omega}$, the set of local configurations with $\eta$ boundary conditions is ${\Omega_\Lambda^\eta\coloneqq \{\omega\in\Omega_\Lambda : \omega_x=\eta_x, \text{ }\forall x\in\Lambda^c\}}$. The \textit{local Hamiltonian of the random field long-range Ising model} in $\Lambda\Subset\Z^d$ with $\eta$-boundary condition is $H_{\Lambda, \varepsilon}^{\eta}:  \Omega_\Lambda^\eta \to \mathbb{R}$, given by 

\begin{equation}
    H_{\Lambda; \varepsilon h}^{\eta}(\sigma)= -\sum_{x,y\in\Lambda} J_{x,y}\sigma_x\sigma_y - \sum_{x\in \Lambda, y\in\Lambda^c} J_{x,y}\sigma_x\eta_y - \sum_{x\in\Lambda} \varepsilon h_x\sigma_x,
\end{equation}
where $\varepsilon >0$ is a parameter that controls the variance of the external field. Given $\Lambda\Subset\Z^d$, consider $\mathscr{F}_\Lambda$ the $\sigma$-algebra generated by the cylinders supported in $\Lambda$. The main object of study in classical statistical mechanics are the \textit{finite volume Gibbs measures}, which are probability measures in $(\Omega, \mathscr{F}_\Lambda)$, given by 
    \begin{equation}
        \mu_{\Lambda;\beta, \varepsilon h}^\eta(\sigma) = \mathbbm{1}_{\Omega_\Lambda^\eta}(\sigma)\frac{e^{-\beta H_{\Lambda, \varepsilon h}^{\eta}(\sigma)}}{Z_{\Lambda; \beta, \varepsilon}^{\eta}(h)}.
    \end{equation}

Here, $\beta>0$ is the inverse temperature and $Z_{\Lambda; \beta, \varepsilon}^{\eta}$ is the so called \textit{partition function}, defined as 

\begin{equation}
    Z_{\Lambda; \beta, \varepsilon}^{\eta}(h)\coloneqq \sum_{\sigma\in\Omega_\Lambda^\eta} e^{-\beta H_{\Lambda, \varepsilon h}^{\eta}(\sigma)}.
\end{equation}
One important remark is that, since the external field is random, the Gibbs measures are random variables. To explicit the dependence of $\mu_{\Lambda;\beta, \varepsilon h}^\eta$ on $\tilde{\Omega}$, we will often write $\mu_{\Lambda;\beta, \varepsilon h}^\eta[\omega]$, with $\omega$ being a general element of $\tilde{\Omega}$. Two particularly important boundary conditions are given by the configurations $\eta_{+} \equiv +1$ and $\eta_{-} \equiv -1$, and are called $+$ and $-$ boundary conditions, respectively. For these boundary conditions, we can $\mathds{P}$-almost surely define the infinite volume measures by taking the weak*-limit
\begin{equation}
    \mu_{\beta,\varepsilon h}^{\pm}[\omega] = \lim_{n\to\infty} \mu_{\Lambda_n;\beta, \varepsilon h}^{\pm}[\omega],
\end{equation}
where $(\Lambda_n)_{n\in\mathbb{N}}$ is any sequence invading $\Z^d$, that is, for any subset $\Lambda\Subset\mathbb{Z}^d$, there exists $N=N(\Lambda)>0$ such that $\Lambda\subset\Lambda_n$ for every $n>N$.
To  have more than one Gibbs measure, it is enough that $\mu_{\beta,\varepsilon h}^{+}[\omega]\neq  \mu_{\beta,\varepsilon h}^{-}[\omega]$, with $\mathbb{P}$-probability 1, see \cite[Theorem 7.2.2]{Bovier.06}.

Peiers' argument, used to prove phase transition, is based on the idea of erasing contours. Contours are geometric objects in the dual lattice $\mathbb{Z}^d_*$ defined as: denoting $C_x$ the closed unit cube in $\mathbb{R}^d$ centered in $x$, $\mathbb{Z}^d_*$ is the union of all faces $C_x\cap C_y$ with $|x-y| = 1$. Given a configuration, its contours are the maximal connected components of the union of the faces $C_x\cap C_y$ satisfying $\sigma_x \neq \sigma_y$. The set of contours of $\sigma$ is denoted by $\Gamma(\sigma)$, and $\gamma$ denotes a generic element of $\Gamma(\sigma)$. The set of all contours is $\Gamma$ and the \textit{interior} of a contour $\gamma$, denoted $\I(\gamma)$, is the set of points connected to $\infty$ only by paths crossing $\gamma$. Given $n\in\mathbb{N}$, take
\begin{equation*}
    \Gamma_0(n) \coloneqq \{\gamma\in\Gamma \ : \ 0\in\I(\gamma), \  |\gamma| = n\}
\end{equation*}
and $\Gamma_0 = \cup_{n\geq 1}\Gamma_0(n)$. The operation $\tau_{\gamma}$ used to remove a contour $\gamma\in\Gamma(\sigma)$ can be written as a particular case of the following one: given $A\subset\Z^d$, take $\tau_A:\mathbb{R}^{\Z^d} \xrightarrow{} \mathbb{R}^{\Z^d}$ as 
\begin{equation}
    (\tau_A(\sigma))_i \coloneqq \begin{cases}
                        -\sigma_i &\text{if }i\in A,\\
                        \sigma_i   &\text{otherwise},
                      \end{cases}
\end{equation}
for every $i\in\Z^d$. The transformation that erases a contour $\gamma$ is $\tau_\gamma(\sigma) \coloneqq \tau_{\I(\gamma)}(\sigma)$. Following \cite{Ginibre.Grossmann.Ruelle.66} we can bound the difference in the Hamiltonian after erasing a contours, when there is no external field.
\begin{proposition}
    For the long-range Ising model with $\alpha > d+1$ and inverse temperature $\beta>0$, there is a constant $c_1(\alpha)>0$ such that
    \begin{equation}
        H_{\Lambda, 0}^{+}(\tau_{\gamma}(\sigma)) - H_{\Lambda, 0}^{+}(\sigma) \leq - J c_1(\alpha) |\gamma|.
    \end{equation}
\end{proposition}
\begin{proof}
    A straightforward computation shows that
    \begin{align*}
        H_{\Lambda, 0}^{+}(\tau_\gamma(\sigma)) &= -\sum_{\substack{x,y\in \I(\gamma)}} J_{xy}\tau_\gamma(\sigma)_x\tau_\gamma(\sigma)_y - \sum_{\substack{x,y\in \I(\gamma)^c}} J_{xy}\tau_\gamma(\sigma)_x\tau_\gamma(\sigma)_y - \sum_{\substack{x\in \I(\gamma)\\ y\in \I(\gamma)^c}} J_{xy}\tau_\gamma(\sigma)_x\tau_\gamma(\sigma)_y \\
        %
        &= -\sum_{\substack{x,y\in \I(\gamma)}} J_{xy}\sigma_x\sigma_y -\sum_{\substack{x,y\in \I(\gamma)^c}} J_{xy}\sigma_x\sigma_y + \sum_{\substack{x\in \I(\gamma)\\ y\in \I(\gamma)^c}} J_{xy}\sigma_x\sigma_y\\
        & = H_{\Lambda, 0}^{+}(\sigma) + 2\sum_{\substack{x\in \I(\gamma)\\ y\in \I(\gamma)^c}} J_{xy}\sigma_x\sigma_y
    \end{align*}
    Therefore, the difference can be bounded in the following way 
    \begin{align*}
        \frac{1}{2}(H_{\Lambda, 0}^{+}(\tau_{\gamma}(\sigma)) - H_{\Lambda, 0}^{+}(\sigma)) &= \sum_{\substack{x\in \I(\gamma)\\ y\in \I(\gamma)^c}} J_{xy}\sigma_x\sigma_y
        %
        =\sum_{\substack{x\in \fint\I(\gamma), \\   y\in \fext\I(\gamma)^c}} J\sigma_x\sigma_y + \sum_{\substack{x\in \I(\gamma), \\  y\in \I(\gamma)^c \\ |x-y|\geq 2}} J_{xy}\sigma_x\sigma_y\\
        %
        &\leq -J|\gamma| + \sum_{\substack{x\in \I(\gamma) \\  y\in \I(\gamma)^c \\ |x-y|\geq 2}} J_{xy} \\
        &= -J|\gamma| + \sum_{\substack{k\in \Z^d \\ |k|\geq 2}} J_{0,k}|\{\{x,y\} : x\in \I(\gamma), \  y\in \I(\gamma)^c, x-y = k\}|.
    \end{align*}
    For $i=1,\dots, d$, let $\gamma_i$ be the faces of $\gamma$ perpendicular to the direction $e_i$. Using that $$|{\{x,y\} : x\in \I(\gamma), \  y\in \I(\gamma)^c, x-y = k}| \leq \sum_{i=1}^d |k_i||\gamma_i|,$$ 
we get     
\begin{align*}
    H_{\Lambda, 0}^{+}(\tau_{\gamma}(\sigma)) - H_{\Lambda, 0}^{+}(\sigma) &\leq  -2J|\gamma| + 2\sum_{\substack{k\in \Z^d \\ |k|\geq 2}} \frac{J}{|k|^{\alpha+1}}|\gamma|. \\
\end{align*}
Taking $c_1(\alpha)=2(1- \sum_{\substack{k\in \Z^d \\ |k|\geq 2}}\frac{1}{|k|^{\alpha+1}})$, we conclude our proof by noticing that $c_1(\alpha)>0$ if and only if $\alpha > d+1$.
\end{proof}


\section{Ding and Zhuang approach} The main idea used in Ding and Zhuang's proof of phase transition in \cite{Ding2021} is to make the Peierls' argument on the joint space of the configurations and the external field, and when erasing a contour, perform in the external field the same flips you do in the configuration. Doing this, the part on the Hamiltonian that depends on the external field does not change, but the partition function does. The complication of this method is to control such difference.

In the short-range case, the spins that need to be flipped to erase a contour are precisely the ones in the interior of it. This is not the case for the long-range model, so we make a slight modification in the argument, and instead of performing the same flips in both spaces, we flip the external field only on $\I_-(\gamma)$. Doing this, not only does the partition function change but we also get an extra cost when comparing the original energy with the energy after performing such transformation. The extra term depends only on the external field in $\Sp{(\gamma)}$.  

In this section, we define the measure in the joint space and show that, with high probability, both the change of partition function and the extra energy cost resulting from such flipping are upper-bounded by the size of the support $|\gamma|$. 
    \subsection{Joint measure and bad events}       Given $\Lambda\subset\Z^d$, define the local joint measure for $(\sigma, h)$ as
\begin{equation*}
    \mathbb{Q}_{\Lambda; \beta, \varepsilon}^+(\sigma \in A, h\in B) = \int_{B} \mu_{\Lambda;\beta, \varepsilon h}^+(A) d\mathbb{P}(h),
\end{equation*}
for $A\subset\Omega$ measurable and $B\subset \mathbb{R}^{\Lambda}$ borelian. Since $\beta, \varepsilon $ and $\Lambda$ are fixed, we will omit then from the notation. 
This measure $\mathbb{Q}$ has density
\begin{equation*}
    g_{\Lambda; \beta, \varepsilon}^+(\sigma, h) = \prod_{u\in\Lambda}\frac{1}{\sqrt{2\pi}}e^{-\frac{1}{2}h_u^2} \times \mu_{\Lambda;\beta, \varepsilon h}^+(\sigma).
\end{equation*}

The main idea used in the proof of phase transition in \cite{Ding2021} is to make the Peierls' argument on the measure $\mathbb{Q}$, and perform in the external field the same flips you do in the configuration when erasing a contour. Formally, in \cite{Ding2021} they compare the density $g_{\Lambda; \beta, \varepsilon}^+(\sigma, h)$ with the density after erasing a contour $\gamma\in\Gamma(\sigma)$, and performing the same flips on the external field, getting

\begin{align}\label{Eq: quotient.of.gs}
    \frac{g_{\Lambda; \beta, \varepsilon}^+(\sigma, h)}{g_{\Lambda; \beta, \varepsilon}^+(\tau_{\gamma}(\sigma),\tau_{\gamma}(h))} \nonumber
    %
    &= \exp{\{\beta H_{\Lambda, 0}^{+}(\tau_{\gamma}(\sigma)) - \beta H_{\Lambda, 0}^{+}(\sigma)\}}\frac{Z_{\Lambda; \beta, \varepsilon}^{+}(\tau_{\gamma}(h))}{Z_{\Lambda; \beta, \varepsilon}^{+}(h)}.  \nonumber \\ 
\end{align}

For some realizations of the external field, the quotient of the partition functions can be bigger than the exponential term. Denoting
\begin{equation}
\Delta_A(h) \coloneqq -\frac{1}{\beta}\ln{\frac{Z_{\Lambda; \beta, \varepsilon}^{+}(h)}{Z_{\Lambda; \beta, \varepsilon}^{+}(\tau_{A}(h))}}
\end{equation}
 for every $A\subset \Z^d$, the bad event is
$$\mathcal{E}^c\coloneqq \left\{\sup_{\substack{\gamma\in\Gamma_0}} \frac{|\Delta_{\I(\gamma)}(h)|}{c_1|\gamma|} > \frac{1}{4}\right\}.$$
To control the probability of this bad event, we need a concentration result for Gaussian random variables. The following one is due to M. Ledoux and M. Talagrand, and a proof can be found in \cite{Ledoux.Talagrand.91}.

\begin{theorem}\label{Theo: Gaussian.concentration}
    Let $f:\mathbb{R}^M \xrightarrow[]{} \mathbb{R}$ be a uniform Lipschitz continuous function with constant $C_{Lip}$, that is, for any $X,Y\in\mathbb{R}^M$, $$|f(X) - f(Y)| \leq C_{Lip} || X - Y ||_2 .$$ 
    
    Then, if $X_1,\dots, X_M$ are i.i.d. Gaussian random variables with variance 1,
    \begin{equation}\label{Eq: Tail.concentration}
        \mathbb{P}\left(|f(X_1,\dots, X_M) - \mathbb{E}(f(X_1, \dots, X_M))|\geq z\right) \leq 2\exp{\left\{\frac{-z^2}{2C_{Lip}^2}\right\}}.
    \end{equation}
\end{theorem}

\begin{remark}\label{Rmk: MVT.Lipschitz}
    If $f$ is differentiable and $||\nabla f(\cdot)||_2$ is bounded, the mean value theorem guarantees that $\sup_{Z\in\mathbb{R}^M}||\nabla f(Z)||_2$ is a uniform Lipschitz constant for $f$. 
\end{remark}
\begin{remark}
    If $f$ has a compact support and convex level sets, an equation similar to \eqref{Eq: Tail.concentration} holds, with some adjustments on the constants and replacing the mean by the median, see \cite[Theorem 7.1.3]{Bovier.06}. Therefore, our results hold when $h_i$ has a Bernoulli distribution $\mathbb{P}(h_i=+1) =\mathbb{P}(h_i=-1)= \frac{1}{2}$. 
\end{remark}

Given $A\subset\Z^d$, $h_A\coloneqq (h_x)_{x\in A}$ denotes the restriction of the external field to the subset $A$. The next Lemma was proved in \cite{Ding2021} and is a direct consequence of the previous theorem. 

\begin{lemma}\label{Lemma: Concentration.for.Delta.General}
    For any $A, A^\prime \Subset \mathbb{Z}^d$ and $\lambda>0$, we have 
\begin{equation}\label{Eq: Tail.of.Delta_A}
    \mathbb{P}\left(|\Delta_A(h)| \geq \lambda \vert h_{A^c}\right) \leq2e^{\frac{-\lambda^2}{8\varepsilon^2 |A|}},
\end{equation}
and 
\begin{equation}\label{Eq: Tail.of.the.diff.of.Deltas}
     \mathbb{P}(|\Delta_{A}(h) - \Delta_{A^\prime}(h)|>\lambda|h_{{A \cup A^\prime}^c}) \leq  2e^{-\frac{{\lambda^2}}{{8\varepsilon^2|A \Delta A^\prime|}}},
\end{equation}
where $A\Delta A^\prime$ is the symmetric difference.
\end{lemma}

\begin{remark}\label{Rmk: More.general.h_x}
    This lemma holds whenever $h=(h_x)_{x\in\Z^d}$ satisfy equation \eqref{Eq: Tail.concentration}.As a consequence, our results can be stated for more general external fields. 
\end{remark}

    \subsection{Coarse-graining procedure}      To control the probability of $\mathcal{E}^c$ we use a multi-scale analysis method presented in \cite{FFS84}. This subsection is dedicated to prove

\begin{proposition}\label{Prop: Bound.bad.event.1}       
    There exists $C_1\coloneqq C_1(\alpha, d)$ such that $\mathbb{P}(\mathcal{E}^c)\leq e^{-\frac{C_1}{\varepsilon^2}}$. 
\end{proposition}

As pointed out by \cite{Ding2021}, the proof presented in \cite{FFS84}, despite being self-contained, is an indirect application of Dudley's entropy bound. Here we adapt the proof presented in \cite{FFS84} using this entropy bound. For the detailed argument of the original proof, see \cite{Bovier.06}. 

First, we need to introduce some probability tools. Consider $(T,d)$ a metric space and a process $(X_t)_{t\in T}$ such that, for every $\lambda>0$ and $t,s\in T$,
\begin{equation}\label{Eq: Sub_Gaussian_r.v.}
    \mathbb{P}\left( |X_t - X_s| \geq \lambda \right) \leq 2\exp{-\frac{\lambda^2}{2d(s,t)^2}}.
\end{equation}
Assume also that $\mathbb{E}\left(X_t\right) = 0$ for every $t\in T$. One example of such process is $(|\Delta_{\I_-(\gamma)}|)_{\gamma\in\Gamma_0}$ with the distance $\d_2(A,A^\prime) = 2\varepsilon |A\Delta A^\prime|^{\frac{1}{2}}$. Denote $\sigma^2_X \coloneqq \sup_{t\in T} \text{Var}(X_t)$. 

\begin{definition}
    Given a set $T$, a sequence $(\mathcal{A}_n)_{n\geq 0}$ of partitions of $T$ is \textit{admissible} when $|\mathcal{A}_0|=1$, $|\mathcal{A}_n|\leq 2^{2^n}$ for $n\geq 0$, and $(\mathcal{A}_n)_{n\geq 0}$ is increasing, that is, every set of $\mathcal{A}_{n+1}$ is contained in a set of $\mathcal{A}_n$.
\end{definition}

Given $t\in T$ and an admissible sequence $(\mathcal{A}_n)_{n\geq 0}$, $A_n(t)$ denotes the element of $\mathcal{A}_n$ that contains $t$. 

\begin{definition}
    Given $\theta \geq 0$ and a metric space $(T,d)$, we define
    \begin{equation*}
        \gamma_\theta(T,d) \coloneqq \inf_{(\mathcal{A}_n)_{n\geq 0}}\sup_{t\in T}\sum_{n\geq 0}2^{\frac{n}{\theta}}\diam(A_n(t)),
    \end{equation*}
where the infimum is taken over all admissible sequences. 
\end{definition}
The next theorem shows that the functional $\gamma_2(T,d)$ controls the expected value of the supremum of $(X_t)_{t\in T}$. A proof can be fund in \cite[Theorem 2.4.1]{Talagrand_14}
\begin{theorem}[Majorizing measure theorem] Given a metric space $(T,d)$ and a family $(X_t)_{t\in T}$ satisfying \eqref{Eq: Sub_Gaussian_r.v.} and $\mathbb{E}(X_t)=0$ for every $t\in T$, there is a universal constant $L>0$ such that
\begin{equation*}
    \frac{1}{L}\gamma_2(T,d) \leq \mathbb{E}\left( \sup_{t\in T} X_t \right) \leq L\gamma_2(T,d).
\end{equation*}
\end{theorem}

Given $\epsilon>0$, let $N(T,\d, \epsilon)$ be the minimal number of ball with radius $\epsilon$ necessary to cover $T$, using the distance $d$. 
\begin{proposition}[Dudley's entropy bound \cite{Dudley67}]
    Let $(X_t)_{t\in T}$ be a family of random variables satisfying \eqref{Eq: Sub_Gaussian_r.v.} for some distance $\d$. Then there exists a constant $L>0$ such that 
    \begin{equation*}
        \mathbb{E}\left[\sup_{t\in T}X_t\right]\leq L\int_{0}^\infty \sqrt{\log N(T,\d,\epsilon)}d\epsilon.
    \end{equation*}
\end{proposition}

Dudley's entropy bound together with the Majorizing measure Theorem yields that, for a constant $L>0$, $\gamma_2(T,\d)\leq \int_{0}^\infty \sqrt{\log N(T,\d,\epsilon)}d\epsilon$.  We only need one last inequality, see \cite[Theorem 2.2.27]{Talagrand_14}

\begin{theorem}\label{Theo: Theo_2.2.27_Talagrand} Given a metric space $(T,\d)$ and a family $(X_t)_{t\in T}$ satisfying \eqref{Eq: Sub_Gaussian_r.v.}, there is a universal constant $L>0$ such that, for any $u>0$,
\begin{equation*}
\mathbb{P}\left( \sup_{t\in T}X_t > L(\gamma_2(T,\d) + u\diam(T)) \right)\leq e^{-{u^2}},
\end{equation*}
where the $\diam(T)$ is the diameter taken with respect to the distance $\d$
\end{theorem}

We will apply this lemmas for the family $(|\Delta_{\I_-(\gamma)}|)_{\gamma\in\Gamma_0(n)}$. To construct the covering by balls in Dudley's entropy bound, we use the coarse-graining idea introduced in \cite{FFS84}. For each $0<\ell$ and each contour ${\gamma\in\Gamma_0}$, we will associate a region $B_\ell(\gamma)$ that approximates the interior $\I(\gamma)$ in a scaled lattice, with the scale growing with $\ell$. This is done in a way that two contours that have the same representation are in a ball with fixed radius, depending on $\ell$.

For any $x\in\Z^d$ and $m\geq 0$,
\begin{equation}
    C_{m}(x) \coloneqq \left(\prod_{i=1}^d{\left[2^{m}x_i - 2^{m-1}, 2^{m}x_i + 2^{m-1} \right)}\right)\cap \Z^d
\end{equation}
is the cube of $\mathbb{Z}^d$ centered in $2^{m}x$ with side length $2^{m} -1$. Any such cube is called an $m$-cube, and we refer an arbitrary $m$-cube by $C_m$. An arbitrary collection of $m$-cubes will be denoted $\mathscr{C}_m$ and $B_{\mathscr{C}_m}\coloneqq \cup_{C\in\mathscr{C}_m}C$ is the region covered by $\mathscr{C}_m$. We  denote by $\mathscr{C}_m(\Lambda)$ the covering of $\Lambda\Subset\Z^d$ with the smallest possible number  of $m$-cubes.

\input{Figura.0}

Fix $n \in \mathbb{N}$, $\gamma\in \Gamma_0(n)$, and $\ell\in\{0, 1, \dots, k\}$. An $\ell$-cube $C_{\ell}$ is \textit{admissible} if it more than a  half of its points are inside $\I(\gamma)$. Thus, the set of admissible cubes is
\begin{equation*}
    \mathfrak{C}_\ell(\gamma) \coloneqq \{C_{\ell} : |C_{\ell}\cap \I(\gamma)| \geq \frac{1}{2}|C_{\ell}|\}.
\end{equation*}
We choose $B_\ell(\gamma) \coloneqq B_{\mathfrak{C}_{\ell}(\gamma)}$, the region covered by the admissible cubes.
Notice that $B_\ell(\gamma)$ is uniquely determined by $\partial B_\ell(\gamma)$. Moreover, $\partial B_\ell(\gamma)$ is uniquely determined by
$$
\partial \mathfrak{C}_\ell(\gamma) \coloneqq \{ \{C_{\ell}, C^\prime_{\ell}\} : C_{\ell} \in \mathfrak{C}_\ell(\gamma), \ C_\ell^\prime \notin \mathfrak{C}_\ell, \  C_\ell^\prime \text{ shares a face with }C_\ell\}.
$$ 
We will now control the number of cubes in $\mathfrak{C}_\ell(\gamma)$ by proving a proposition similar to \cite[Proposition 2]{FFS84}. This proposition was written for $d=3$ and $\gamma$ simply connected, but it can clearly be extended to $d\geq 2$ with no restriction in $\gamma$, see \cite{Bovier.06}. As we could not find a detailed proof anywhere, we provide one here.

Given a rectangle $\mathcal{R} = [1,r_1]\times[1,r_2]\times\dots\times[1,r_d]$, consider $\R_i\coloneqq\{x\in \R : x_i=1\}$ the face of $\R$ that is perpendicular to the direction $e_i$, for $i=1,\dots,d$. The line that connects a point $x\in \R_i$ to a point in the opposite face of $\R_i$ is $\ell_x^i \coloneqq \{ x + ke_i : 1\leq k\leq r_i\}$. Given $A\subset \Z^d$, the projection of $A\cap \R$ into the face $\R_i$ is
\begin{equation*}
    \calP_i(A\cap\R) \coloneqq \{x\in\R_i : \ell_x^i \cap A \neq \emptyset\}.
\end{equation*}
\input{Figura.4.1}
In many situations, we will split the projections into \textit{good} and \textit{bad} points. The set of good points is $\calP_i(A\cap\R)^{G} \coloneqq \{x\in \calP_i(A\cap \R) : \ell_x^i \cap (\R\setminus A) \neq \emptyset\}$, that is, there exist a point in $\ell_x^i\cap \R$ that is not in $A$.  The bad points are defined as $\calP^{B}_i(A\cap\R) \coloneqq \calP_i(A\cap\R)\setminus \calP_i^G(A\cap\R)$.

\input{Figura.5}
Given $x\in \calP_i(A\cap\R)^{G}$, by definition of the projection, there exists a point in $\ell_x^i\cap A$. Therefore, there exists a point $p\in \ell_x^i$ such that $p\in\fext A \cap \R$. As all lines are disjoint, we conclude that 
\begin{equation}\label{Eq: upper.bound.good.points}
     |\calP_i^{G}(A\cap\R)|\leq |\fext A \cap \R|.
\end{equation}
 We now prove two auxiliary lemmas.
 
\begin{lemma}\label{Lemma: Geo.discreta.1}
    Given $d\geq 2$, for any family of positive integers $\bm{r}=(r_i)_{i=1}^d$ with $R\leq r_i \leq 2R$ for some $R\geq 2$, $0<\lambda < 1$ and $A\subset\Z^d$, there exists a constant $c\coloneqq c(d, \lambda)$ such that, if 
    \begin{equation}\label{Eq: hypothesis.lemma.1}
         |\calP_i(A\cap \R)| \leq \lambda|\R_i|
    \end{equation}
    for all $i= 1,\dots, d$, then 
    \begin{equation*}
        \sum_{i=1}^d |\calP_{i}(A\cap \R)|\leq c|\fext A\cap \R|,
    \end{equation*}
    where $\R=[1,r_1]\times\dots\times [1,r_d]$.
\end{lemma}

\begin{proof}
The proof will be done by induction on the dimension. For $d=2$, take a rectangle ${\R=[1,r_1]\times[1,r_2]}$. If there is no bad points in $\calP_1(A\cap\R)$, then 
\begin{align}\label{Eq: Bound.1.on.P.1}
    |\calP_1(A\cap\R)| = |\calP_1^G(A\cap \R)| \leq |\fext A \cap \R|.
\end{align}

If there is a bad point $p=(1,p_2)\in \calP_1^B(A\cap\R)$, $\ell_p^1\subset A\cap \R$  by definition of bad point. As $|\calP_1(A\cap \R)| \leq \lambda|\R_1| < |\R_1|$, there is a point $p^\prime = (1,p_2^\prime)\in \R_1\setminus \calP_1(A\cap \R)$ that is in the face $\R_1$ but not in the projection. By definition of the projection, $\ell_{p^\prime}^1\in A^c\cap \R$. Therefore, for any $1\leq k\leq r_1$, $(k,p_2)\in  A\cap \R$ and $(k,p^\prime_2)\in  A^c\cap \R$, we can find a point $p^k=(k, p^k_2) \in \fext A \cap \R$. Since $p^{k_1}\neq p^{k_2}$ for every $k_1\neq k_2$, we have $r_1 \leq |\fext A \cap \R|$, hence
\begin{equation}\label{Eq: Bound.2.on.P.1}
   |P_1(A\cap \R)| \leq  |\R_1| = {r_2}\leq  2R \leq 2r_1 \leq  2|\fext A \cap \R|.
\end{equation}

A completely analogous argument can be done to bound $|P_2(A\cap \R)|$, and we conclude that
\begin{equation*}
    \sum_{i=1}^2|\calP_i(A\cap \R)|\leq 4|\fext A \cap \R|,
\end{equation*}
and take $c(2,\lambda)=4$. Suppose the lemma holds for $d-1$ and fix a rectangle $\R=[1,r_1]\times\dots\times[1,r_d]$. We split $\R$ into layers $L_k = \{x\in\Z^d : x_d = k\}$, for $k=1,\dots, r_d$. We can then partition the projection and write
\begin{equation*}
|\calP_i(A\cap \R)| = \sum_{k=1}^{r_d} |\calP_i(A\cap \R)\cap L_k|,    
\end{equation*}
for any $i\in\{1,\dots, d-1\}$. This yields
\begin{align}\label{Eq: Partition.sum.proj.}
    \sum_{i=1}^d|\calP_i(A\cap \R)| &= \sum_{i=1}^{d-1}\sum_{k=1}^{r_d}|\calP_i(A\cap \R)\cap L_k| + |\calP_d(A\cap \R)| \nonumber \\
    &=  \sum_{k=1}^{r_d}\sum_{i=1}^{d-1}|\calP_i(A\cap \R)\cap L_k| + |\calP_d(A\cap \R)|.
\end{align}

Notice now that $\calP_i(A\cap \R)\cap L_k = \calP_i(A\cap (\R\cap L_k))$. Defining the rectangle $\R^k \coloneqq \R\cap L_k$, for every point $p\in \calP_j^B(A\cap \R^k)$, $\ell_p^j \subset A\cap \R^k$. Moreover, we can associate every point $x\in \ell_p^j$ in the line with a point $x^\prime\in \calP_d(A\cap\R)$ by taking $x_m^\prime = x_m$ for $m \leq d-1$ and $x_d^\prime = 1$, therefore

\begin{equation*}
    r_j|\calP_j^B(A\cap \R^k)| = \sum_{p\in \calP_j^B(A\cap \R^k)}|\ell_p^j| \leq |\calP_d(A\cap\R)|.
\end{equation*}
\input{Imagem.6}
Using the hypothesis \eqref{Eq: hypothesis.lemma.1} we conclude that

\begin{equation}\label{Eq: upper.bound.projection.i.bad.points}
     |\calP_j^B(A\cap \R^k)| \leq \lambda\frac{|\R_d|}{r_j} = \lambda \frac{ \prod_{q\neq d}r_q}{r_j} =  \lambda \prod_{q\neq j,d}r_q = \lambda |(\R^k)_j|.
\end{equation}
We consider know two cases:
    \begin{itemize}
        \item[(a)] If $|\calP_i(A\cap \R^k)| \leq \frac{\lambda +1}{2}|(\R^k)_i|$, for all $i\leq d-1$, then we are in the hypothesis of the lemma in $d-1$ and therefore
\begin{equation}\label{Eq: Primeiro.bound.soma.projecoes}
    \sum_{i=1}^{d-1} |\calP_i(A\cap \R^k)| \leq c\left(d-1, \frac{\lambda + 1}{2}\right)|\fext A\cap \R^k|.
\end{equation}
    \item [(b)] If there exists $j\in\{1,\dots,d-1\}$ satisfying $|\calP_j(A\cap \R^k)| > \frac{\lambda +1}{2}|(\R^k)_j|$, by \eqref{Eq: upper.bound.projection.i.bad.points} we have $|\calP_j^G(A\cap \R^k)| = |\calP_j(A\cap \R^k)| - |\calP_j^B(A\cap \R^k)| \geq \frac{1-\lambda}{2}|(\R^k)_j|$, hence
\begin{equation*}
    |(\R^k)_j| \leq  \frac{2}{1 - \lambda}|\fext A \cap \R^k|.
\end{equation*}
    Using that $|(\R^k)_i| \leq (2R)^{d-2} \leq 2^{d-2}|(\R^k)_j|$ for every $i\in\{1,\dots,d\}$, we conclude that 
\begin{equation}\label{Eq: Segundo.bound.soma.projecoes}
    \sum_{i=1}^{d-1} |P_i(A\cap\R^k)| \leq \sum_{i=1}^{d-1} |(\R^k)_i| \leq (d-1)2^{d-2}|(\R^k)_j| \leq \frac{(d-1)2^{d-1}}{1 - \lambda}|\fext A \cap \R^k|.
\end{equation}
\end{itemize}


In both cases we were able to bound the sum of projections by a constant times the size of the boundary of $A$ in $\R^k$. Applying \eqref{Eq: Primeiro.bound.soma.projecoes} and \eqref{Eq: Segundo.bound.soma.projecoes} back in \eqref{Eq: Partition.sum.proj.} we get
\begin{align*}
    \sum_{i=1}^d|\calP_i(A\cap \R)| &\leq
    \sum_{k=1}^{r_d}\left[c\left(d-1, \frac{\lambda + 1}{2}\right)+ \frac{(d-1)2^{d-1}}{1 - \lambda}\right]|\fext A \cap \R \cap L_k| + |\calP_d(A\cap \R)|\\
    %
    &=\left[c\left(d-1, \frac{\lambda + 1}{2}\right)+ \frac{(d-1)2^{d-1}}{1 - \lambda}\right]|\fext A \cap \R| + |\calP_d(A\cap \R)|.
\end{align*}
We finish the proof by noticing that we can repeat this exact same argument but now splitting $\R$ into layers $L_k = \{x\in \R : x_j = k\}$. By doing so, we have that  
\begin{equation*}
    \sum_{i=1}^d|\calP_i(A\cap \R)| \leq
     \left[c\left(d-1, \frac{\lambda + 1}{2}\right)+ \frac{(d-1)2^{d-1}}{1 - \lambda}\right]|\fext A \cap \R| + |\calP_j(A\cap \R)|
\end{equation*}
for any $j\in\{1,\dots, d\}$. Summing both sides in $j$ we conclude

\begin{equation}
    \sum_{i=1}^d|\calP_i(A\cap \R)| \leq \frac{d}{d-1}\left[c\left(d-1, \frac{\lambda + 1}{2}\right)+ \frac{(d-1)2^{d-1}}{1 - \lambda}\right]|\fext A \cap \R|,
\end{equation}
which proves our claim if we take $c(d,\lambda) \coloneqq \frac{d}{d-1}\left[c(d-1, \frac{\lambda + 1}{2})+ \frac{(d-1)2^{d-1}}{1 - \lambda}\right] = 2d + \frac{(d-2)d2^{d-1}}{1-\lambda}$.
\end{proof}

\begin{remark}
This lemma can be proved when $R\leq r_i \leq \kappa R$ for any $\kappa>1$. When applying the lemma, we will choose $\lambda=\frac{7}{8}$ to simplify the notation. All the proofs work as long as we choose $\lambda> \frac{3}{4}$.
\end{remark}

\begin{lemma}\label{Lemma: Proposicao1.Aux1}
    Given $A\subset \Z^d$, $\ell\geq 0$ and $U= C_{\ell}\cup C_{\ell}^\prime$ with $C_{\ell}$ and $C_{\ell}^\prime$ being two $\ell$-cubes sharing a face, there exists a constant $b\coloneqq b(d)$ such that, if 
    
\begin{align}\label{Eq. U.condition}
    \frac{1}{2}|C_{\ell}| \leq |C_{\ell}\cap A| \qquad \text{and} \qquad |C_{\ell}^\prime\cap A|< \frac{1}{2}|C_{\ell}^\prime|
\end{align}
then $2^{\ell(d-1)}\leq b|\fext A\cap U|$.
\end{lemma}


\begin{proof}
For $\ell=0$, \eqref{Eq. U.condition} guarantees that $C_{\ell} = \{x\} \subset A$ and $C_{\ell}^\prime = \{y\}\subset A^c$, hence $|\fext A\cap \{x,y\}| = 1$ and it is enough to take $b\geq 1$. For $\ell \geq 1$, \eqref{Eq. U.condition} yields
\begin{equation}\label{Eq. A.cap.U.volume}
    \frac{1}{2}2^{\ell d} \leq |A\cap U| \leq \frac{3}{2}2^{\ell d}.
\end{equation}

To simplify the notation, we can assume wlog that ${U=[1,2^{\ell}]^{d-1}\times [1, 2^{\ell+1}]}$. As discussed before, for each point $p\in\calP^B_j(A\cap U)$ in the projection, $\ell_p^j\subset A\cap U$ and the lines are disjoint. Moreover, the size of the lines  is constant $r_j\coloneqq |\ell_p^j|$, hence $|\calP_{j}^B(A\cap U)|r_j = \sum_{p\in\calP_{j}^B(A\cap U)} |\ell_p^j| \leq |A\cap U|$. Together with the upper bound \eqref{Eq. A.cap.U.volume}, this yields 
\begin{equation}\label{Eq: Upper.bound.bad.points}
    |\calP_{j}^B(A\cap U)| \leq \frac{3}{2}2^{\ell d}r_j^{-1}.
\end{equation}
Using the isometric inequality, the lower bound on \eqref{Eq. A.cap.U.volume} yields $d2^{\frac{1}{d}}2^{\ell(d-1)}\leq |\fext (A\cap U)|$. As 
%
\begin{align*}
\frac{1}{2d}|\fext (A\cap U)| &\leq |\fint (A\cap U)| = |\fint(A\cap U) \cap \fint U| + |\fint(A\cap U) \cap(U\setminus \fint U)|\\
%
&\leq 2\sum_{i=1}^d |\calP_{i}(A\cap U)| + |\fint A\cap U| \leq 2\sum_{i=1}^d |\calP_{i}(A\cap U)| + |\fext A\cap U|,
\end{align*}
we get
\begin{equation}\label{Eq: Lemma.geo.discreta.3}
    2^{\frac{1}{d}-1}2^{\ell(d-1)}\leq 2\sum_{i=1}^d |\calP_{i}(A\cap U)| + |\fext A\cap U|
\end{equation}

We again consider two cases:
\begin{itemize}
    \item[(a)] If $|\calP_{j}(A\cap U)|> \frac{7}{8}|U_j|$ for some $j=1,\dots, d$, by \eqref{Eq: Upper.bound.bad.points} and \eqref{Eq: upper.bound.good.points} we get
    \begin{align*}
    \frac{7}{8}|U_j| < |\calP_{j}(A\cap U)| \leq |\fext A \cap U| + \frac{3}{2}2^{\ell d}r_j^{-1}.
    \end{align*}
    A simple calculation shows that $\frac{1}{8}2^{\ell(d-1)}\leq \frac{7}{8}|U_j| - \frac{3}{2}2^{\ell d}r_j^{-1}$, therefore 
    \begin{equation}\label{Eq: upper.bound.big.projections.1}
        \frac{1}{8}2^{\ell(d-1)} \leq |\fext A \cap U|.
    \end{equation}

    \begin{align*}
        \frac{7}{8}|U_j| - \frac{3}{2}2^{\ell d}r_j^{-1} &= \frac{1}{8r_j}( 7|U| - 12 \times 2^{\ell d})\\
        %
        &=\frac{1}{8r_j}2^{\ell(d-1)}(7\times 2^{\ell + 1} - 12\times 2^{\ell}) = \frac{1}{4}2^{\ell(d-1)}\frac{(2^{\ell})}{r_j} \geq \frac{1}{8}2^{\ell(d-1)},
    \end{align*}

    \item[(b)] If $|\calP_{i}(A\cap U)|\leq \frac{7}{8} |U_i|$ for all $i$, by Lemma \ref{Lemma: Geo.discreta.1}, there is a constant $c= c(d)$ such that
    \begin{equation}\label{Eq: Lemma.geo.discreta.2}
        \sum_{i=1}^d |\calP_{i}(A\cap U)|\leq c|\fext A\cap U|.
    \end{equation}
    Together with \eqref{Eq: Lemma.geo.discreta.3}, this yields
    \begin{equation}\label{Eq: upper.bound.big.projections.2}
        2^{\ell(d-1)}\leq \frac{2c+1}{2^{\frac{1}{d}-1}}|\fext A\cap U|.
    \end{equation}
\end{itemize}

Equations \eqref{Eq: upper.bound.big.projections.1} and \eqref{Eq: upper.bound.big.projections.2} shows the desired results taking $b\coloneqq \max \{8, {(2c+1)}{2^{1-\frac{1}{d}}}\}$.
\end{proof}


\begin{proposition}\label{Proposition1}For the functions $B_0,\dots,B_k$ defined above, there exists constants $b_1,b_2$ depending only on $d$ and $r$ such that 
\begin{equation}\label{Eq: Prop.1.FFS.i}
    |\partial\mathfrak{C}_\ell(\gamma)| \leq b_1\frac{|\fext \I(\gamma)|}{2^{\ell(d-1)}} \leq b_1 \frac{|\gamma|}{2^{\ell(d-1)}}
\end{equation}
    and 
\begin{equation}\label{Eq: Prop.1.FFS.ii}
    |B_\ell(\gamma)\Delta B_{\ell+1}(\gamma)| \leq b_2 2^{\ell} |\gamma|
\end{equation}
for every $\ell\in\{0,\dots,k\}$ and $\gamma\in\Gamma_0(n)$.
\end{proposition}

    \begin{proof} Fix $\ell\in\{0,\dots,k\}$. Given a pair $(C_{\ell}, C_\ell^\prime)$, we will write $C_{\ell} \sim C_{\ell}^\prime$ when $(C_{\ell}, C_{\ell}^\prime) \in  \partial\mathfrak{C}_\ell(\gamma)$, and the union we be denoted by $U = C_{\ell} \cup C_{\ell}^\prime$. Then, 
   defining $\overline{\mathscr{C}}_{\ell} = \{C_{\ell} \in \partial \mathfrak{C}_\ell(\gamma): C_{\ell} \sim C_{\ell}^\prime \text{ for some } C_{\ell}^prime \notin  \mathfrak{C}_\ell(\gamma)\}$, for any $A\Subset\Z^d$, 
    
    \begin{align*}
        \sum_{\substack{(C_{\ell}, C^\prime_\ell)\in \partial \mathfrak{C}_\ell(\gamma)}}|A \cap \{C_{\ell} \cup C_{\ell}^\prime\}| &\leq \sum_{\substack{C_{\ell}\in \overline{\mathscr{C}}_{\ell}}}\sum_{\substack{C_{\ell}^\prime\notin  \mathfrak{C}_\ell(\gamma)\\ C_{\ell} \sim C_{\ell}^\prime}}  \left(|A \cap C_{\ell}| + |A \cap C_{\ell}^\prime|\right)\\
        %
        &\leq \sum_{\substack{C_{\ell}\in\overline{\mathscr{C}}_{\ell}}} 2d |A \cap C_{\ell}| + \sum_{C_{\ell}^\prime\notin \mathfrak{C}_\ell(\gamma)} 2d|A \cap C_{\ell}^\prime| \leq  2d |A|\\
    \end{align*}

 Any pair of cubes $C_{\ell}\sim C_{\ell}^\prime$ are in the hypothesis of Lemma \ref{Lemma: Proposicao1.Aux1}, hence $ b 2^{\ell(d-1)} \leq |\fext \I(\gamma) \cap U|$. Applying equation above for $A=\fext\I(\gamma)$ we get that
    \begin{equation*}
       \frac{b}{2d} 2^{\ell(d-1)}| \partial \mathfrak{C}_\ell(\gamma)| \leq \frac{1}{2d} \sum_{\substack{(C_{\ell}, C^\prime_\ell)\in \partial \mathfrak{C}_\ell(\gamma)}} |\fext \I(\gamma) \cap \{C_{\ell} \cup C_{\ell}^\prime\}| \leq |\fext \I(\gamma)|,
    \end{equation*}
that concludes \eqref{Eq: Prop.1.FFS.i} for $b_1\coloneqq 2d/b$.

Given $C_{(\ell+1)}\in \mathscr{C}_{(\ell+1)}(B_{\ell+1}(\gamma)\setminus B_{\ell}(\gamma))$, there is a $\ell$-cube $C_{\ell}^\prime\subset C_{r(\ell + 1)}$ with $C_{\ell}^\prime\notin \mathfrak{C}_\ell(\gamma)$, otherwise  $(B_{\ell+1}(\gamma)\setminus B_{\ell}(\gamma))\cap C_{(\ell+1)} = \emptyset$. There is also a $\ell$-cube $C_{\ell}\subset C_{r(\ell + 1)}$ with $C_{\ell}\in \mathfrak{C}_\ell(\gamma)$, otherwise we would have 
\begin{align*}
    |\I(\gamma)\cap C_{(\ell+1)}| &= \sum_{C_{\ell}\subset C_{(\ell+1)}} |\I(\gamma)\cap C_{\ell}| \leq \frac{1}{2} |C_{(\ell+1)}|.
\end{align*}

Moreover, we can assume that $C_{\ell}$ and $C_{\ell}^\prime$ share a face. Again, we use Lemma \ref{Lemma: Proposicao1.Aux1} to get,
\begin{align}\label{Eq: bound.on.c.bar}
    |B_{\ell+1}(\gamma)\setminus B_\ell(\gamma)\cap C_{(\ell+1)}| &\leq |C_{(\ell+1)}| =2^{d}2^{\ell}2^{\ell(d-1)} \nonumber\\
                %
                &\leq 2^{d}2^{\ell} b|\fext \I(\gamma) \cap \{C_{\ell} \cup C_{\ell}^\prime\}| \nonumber\\
                %
                &\leq 2^{d}2^{\ell}b|\fext \I(\gamma) \cap C_{(\ell+1)}|.
\end{align}
Therefore, 
\begin{align*}
    |B_{\ell+1}(\gamma)\setminus B_\ell(\gamma)| &= \sum_{C_{(\ell+1)}\in \mathscr{C}_{(\ell+1)}(B_{\ell+1}(\gamma)\setminus B_\ell(\gamma))} |B_{\ell+1}(\gamma)\setminus B_\ell(\gamma)\cap C_{(\ell+1)}| \\
    %
    &\leq \sum_{C_{(\ell+1)}\in \mathscr{C}_{(\ell+1)}(B_{\ell+1}(\gamma)\setminus B_\ell(\gamma))} 2^{d}2^{\ell}b|\fext \I(\gamma) \cap C_{(\ell+1)}| \leq  \frac{b_2}{2}2^{\ell}|\fext \I(\gamma)|.
\end{align*}
with $b_2=b2^{d+1}$. To get the same bound for $|B_{\ell}(\gamma)\setminus B_{\ell+1}(\gamma)|$ we repeat a similar argument, covering $B_{\ell}(\gamma)\setminus B_{\ell+1}(\gamma)$ with $(\ell+1)$-cubes. 
    \end{proof}

\begin{corollary}
    For any $\ell>0$ and any two contours $\gamma_1,\gamma_2 \in \Gamma_0(n)$ such that $B_\ell(\gamma_1)=B_{\ell}(\gamma_2)$, there exists a constant $b_3>0$ such that 
    \begin{equation*}
        \d_2(\gamma_1,\gamma_2)\leq 4 \varepsilon b_3 2^{\frac{\ell}{2}} n^{\frac{1}{2}}. 
    \end{equation*} 
\end{corollary}

\begin{proof}
    This is a simple application of the triangular inequality, since $d_2(\gamma_1,\gamma_2) \leq d_2(\gamma_1,B_\ell(\gamma_1)) + d_2(\gamma_2,B_\ell(\gamma_2))$ and 
    \begin{align*}
        d_2(\gamma_1,B_\ell(\gamma_1)) &\leq \sum_{i=1}^\ell d_2(B_i(\gamma_1),B_{i-1}(\gamma_1)) = \sum_{i=1}^\ell 2\varepsilon\sqrt{B_i(\gamma_1)\Delta B_{i-1}(\gamma_1)} \\
        %
        & \leq \sum_{i=1}^\ell 2\varepsilon\sqrt{b_2} 2^{\frac{i}{2}} \sqrt{n}  \leq 2\varepsilon\sqrt{b_2}(\sqrt{2}+1)2^{\frac{\ell}{2}} \sqrt{n} 
    \end{align*}
    where in the second to last equation used \eqref{Eq: Prop.1.FFS.ii}. As the same bound holds for $d_2(\gamma_2,B_\ell(\gamma_2))$, the corollary is proved by taking ${b_3 = \sqrt{b_2}(\sqrt{2}+1)}$
\end{proof}

\begin{remark}\label{Rmk: Bounding_N_by_B_ell}
    This corollary shows that we can create a covering of $\Gamma_0(n)$, indexed by $B_\ell(\Gamma_0(n))$, of ball with radius $4 \varepsilon b_3 2^{\frac{\ell}{2}} n^{\frac{1}{2}}$. Therefore $N(\Gamma_0(n), \d_2, 4\varepsilon b_3 2^{\frac{\ell}{2}} n^{\frac{1}{2}}) \leq |B_\ell(\Gamma_0(n))|$. 
\end{remark}
In the next proposition we bound $|B_\ell(\Gamma_0(n))|$, again following \cite{FFS84}.

\begin{proposition}\label{Prop: Proposition_2_FFS}
    There exists a constant $b_4\coloneqq b_4(d)$ such that, for any $n\in\mathbb{N}$,
    \begin{equation}\label{Eq: Proposition_2_FFS}
        |B_\ell(\Gamma_0(n))|\leq \exp{\frac{b_4\ell n}{2^{\ell(d-1)}}},
    \end{equation}
    that is, the number of coarse-grained contours in $B_\ell(\Gamma_0(n))$ is bounded above by an exponential term. 
\end{proposition}

\begin{proof}
Start by noticing that $|B_\ell(\Gamma_0(n))| = |\partial B_\ell(\Gamma_0(n))|$, and to each $B_\ell(\gamma)$ we can associate a contour $\xi_\ell(\gamma)$ with $\I(\xi_\ell) = B_{\ell}(\gamma)$. Given $\xi_\ell\in\xi_\ell(\Gamma_0(n))$ for $\ell\in\{1,\dots,k\}$, let $\{\xi_\ell^{(1)}, \xi_\ell^{(2)},\dots,\xi_\ell^{(m)}\}$ be the connected components of $\xi_\ell$. To connected component $\xi_\ell^{(i)}$ we can uniquely associate a pair $(C_\ell,C^\prime_\ell)\in \partial\mathfrak{C}_\ell(\gamma)$ with $B_{C_\ell}\subset\I(\xi_\ell^{(i)})$. By Lemma \ref{Lemma: Proposicao1.Aux1}, there are at least $b^{-1}2^{\ell (d-1)}$ points of $\fext\I(\gamma)$ in $C_\ell\cup C_\ell^\prime$. Hence $\xi_\ell$ has at most $\frac{b n}{2^{\ell(d-1)}}$ connected components and we take $M_n \coloneqq \frac{b n}{2^{\ell(d-1)}}$.
Moreover, by Lemma \ref{Proposition1}, $ |\xi_\ell| = \sum_{i=1}^{M_n} |\xi_\ell^{(i)}| = 2^{\ell(d-1)}|\partial\mathfrak{C}_\ell(\gamma)| \leq b_1 n$.

Fixed $x_1,x_2,\dots, x_{M_n}\in 2^\ell\Z^d$, and $s^1,s^2,\dots, s^{M_n}\in 2^{\ell (d-1)}\mathbb{N}$, if $\Gamma(\{x_i\}_{i=1}^{M_n},\{s_i\}_{i=1}^{M_n})$ is the number of coarse-grained contours with $x_i\in \I(\xi_\ell^{(i)})$ and $|\xi_\ell^{(i)}| = s^i$ for every $i$, then
\begin{align}\label{Eq: Proposition_2_FFS_Aux_1}
        \Gamma(\{x_i\}_{i=1}^{M_n},\{s_i\}_{i=1}^{M_n}) \leq \exp{\left( \ln{(4d)} \sum_{i=1}^{M_n} \frac{s^i}{2^{\ell (d-1)}}\right)} \leq \exp{\left( b_1\ln{(4d)}   \frac{n}{2^{\ell (d-1)}}\right)}.
\end{align}
The number of choices of $s^1,s^2,\dots, s^1\in 2^{\ell d}\mathbb{N}$ with $\sum_{i=1}^q s^i \leq b_1 n$ is less then $2^{\frac{b_1 n}{2^{\ell(d-1)}}}$. This is a simple bound on the number of ways of putting up to $\frac{b_1 n}{2^{\ell(d-1)}}$ balls on $M_n$ spaces. 

It remains know to bound the number of choices for $({x_i})_{i=1}^{M_n}$. Set $d_1 = |x_1|$ and $d_i = |x_i - x_{i-1}|$ for $i=2,\dots, M_n$. For every $i=1,\dots, M_n$, we choose $y_1,\dots,y_{M_n}\in \I({\gamma})$ such that $|x_i - y_i| = d(x_i, \I({\gamma}))$. With this choice we get $d(x_i,y_i) \leq d2^{\ell}$. Since all the cubes are not connected, $d(y_i,y_{i-1})>2^\ell$, hence
\begin{equation*}
    d(x_i,y_i) \leq d |y_i - y_{i-1}|.
\end{equation*}
As all $y_i$ are in $\I(\gamma)$ and $y_0=x_0=0$, we can reorder the terms to minimize the sum of distances, getting 
\begin{equation*}
    \sum_{i=1}^{M_n} d(y_i, y_{i-1}) \leq 2d |\fext\I(\gamma)| = 2dn.
\end{equation*}
This yields 
\begin{equation}\label{Eq: bound.sum.of.d_i}
    \sum_{i=1}^{M_n} d_i \leq 2\sum_{i=1}^{M_n} d(x_i,y_i) + \sum_{i=1}^{M_n} d(y_i,y_{i-1}) \leq (2d + 1)\sum_{i=1}^{M_n} d(y_i, y_{i-1}) \leq (2d + 1)^2 n
\end{equation}
 Fixing $d_1, d_2,\dots, d_q$, the number of ways of choosing $x_1,\dots,x_q$ is bounded by $\prod_{i=1}^{M_n} (2d_i)^{d}$. The maximum of this quantity is reached when all the distances are the same. Assuming $d_1=\dots =d_q=d^*$, equation \eqref{Eq: bound.sum.of.d_i} yields 
 \begin{equation*}
     d^* \leq \frac{(2d + 1)^2 n}{M_n} = \frac{(2d+1)^2 2^{\ell(d-1)}}{b},
 \end{equation*}
so we have at most 
\begin{equation}\label{Eq: Proposition_2_FFS_Aux_2}
    (\frac{(2d+1)^2 2^{\ell(d-1)}}{b})^{\frac{d b n}{2^{\ell(d-1)}}} \leq \exp\left\{(2d(d-1)\ln{(2)}b\ln{(\frac{(2d+1)}{b})}) \frac{\ell n}{2^{\ell(d-1)}}\right\}
\end{equation}
 ways of choose $x_1,\dots,x_{M_n}$ given $d_1,\dots,d_{M_n}$. The number of solutions $(d_1,\dots,d_{M_n})$ to $\sum_{i=1}^{M_n} d_i =N $ is $\binom{N-1}{M_n}$. As $\binom{N-1}{M_n} < \frac{N^{M_n}}{M_n!}\leq (\frac{eN}{M_n})^{M_n}$, the number of solutions of \eqref{Eq: bound.sum.of.d_i} is bounded by 
 \begin{align}\label{Eq: Proposition_2_FFS_Aux_3}
     \sum_{N=1}^{(2d+1)^2n}\left(\frac{eN}{M_n}\right)^{M_n} &\leq\int_{0}^{(2d+1)^2n +1} \left(\frac{ex}{M_n}\right)^{M_n}dx =  e^{M_n}\frac{(2d+1)^2n +1}{M_n + 1}\left(\frac{(2d+1)^2n +1}{M_n} \right)^{M_n} \nonumber \\
     %
     &\leq \left(\frac{2e(2d+1)^2 2^{\ell(d-1)}}{b} \right)^{\frac{b n}{2^{\ell(d-1)}}} \leq \exp{\left\{4eb^{-1}\ln(2)(2d+1)^2 \frac{\ell n}{2^{\ell(d-1)}} \right\}}.
 \end{align}
 Equations \eqref{Eq: Proposition_2_FFS_Aux_1}, \eqref{Eq: Proposition_2_FFS_Aux_2} and \eqref{Eq: Proposition_2_FFS_Aux_3} proves the proposition for $b_4=\max{\{ b_1\ln{(4d)}, 2d(d-1)\ln{(2)}b\ln{(\frac{(2d+1)}{b})}, eb^{-1}\ln(2)(2d+1)^2 \}}$.
\end{proof}

We are ready to prove the main proposition.

\textit{Proof of Proposition \ref{Prop: Bound.bad.event.1}:} As $N(\Gamma_0(n), \d_2, \epsilon)$ is decreasing in $\epsilon$, we can use Dudley's integral bound to bound
    \begin{align}\label{Eq: Prop_bound_Ec_1}
        {\mathbb{E}\left[\sup_{\gamma\in\Gamma_0(n)}{\Delta_{\I(\gamma)}(h)}\right]} &\leq \int_{0}^\infty \sqrt{\log N(\Gamma_0(n), \d_2, \epsilon)}d\epsilon \leq \sum_{\ell=0}^\infty\sqrt{\log N(\Gamma_0(n), \d_2, \ell)}\nonumber\\
        %
        &\leq 2\varepsilon b_3 n^{\frac{1}{2}}\sum_{\ell=1}^\infty (2^{\frac{\ell}{2}} - 2^{\frac{\ell-1}{2}})\sqrt{\log N(\Gamma_0(n), \d_2,\varepsilon b_3 2^{\frac{\ell}{2}}n^{\frac{1}{2}})}.
    \end{align}

Remember that, as discussed in Remark \ref{Rmk: Bounding_N_by_B_ell}, $N(\Gamma_0(n), \d_2,\varepsilon b_3 2^{\frac{\ell}{2}}n^{\frac{1}{2}})\leq |B_{\ell}(\Gamma_0(n))|$. Therefore, by Proposition \ref{Prop: Proposition_2_FFS},
\begin{align}\label{Eq: Prop_bound_Ec_2}
    \sum_{\ell=1}^\infty (2^{\frac{\ell}{2}} - 2^{\frac{\ell-1}{2}})\sqrt{\log N(\Gamma_0(n), \d_2,\varepsilon b_3 2^{\frac{\ell}{2}}n^{\frac{1}{2}})} &\leq \sum_{\ell=1}^{\infty} (2^{\frac{\ell}{2}} - 2^{\frac{\ell-1}{2}}) \sqrt{\frac{b_4\ell n}{2^{\ell(d-1)}}} \nonumber \\
    %
    &\leq \sqrt{b_4}n^{\frac{1}{2}}(1 - \frac{\sqrt{2}}{2})\sum_{\ell=1}^{\infty}\sqrt{\frac{\ell}{2^{\ell(d-2)}}}.
\end{align}
Denoting $\tau(d) = \sum_{\ell=1}^{\infty}\sqrt{\frac{\ell}{2^{\ell(d-2)}}}$, and $b_5 = 2\tau(d)b_3\sqrt{b_4}(1 - \frac{\sqrt{2}}{2})$, equation \eqref{Eq: Prop_bound_Ec_1} and \eqref{Eq: Prop_bound_Ec_2} yields
\begin{equation*}
    {\mathbb{E}\left[\sup_{\gamma\in\Gamma_0(n)}{\Delta_{\I(\gamma)}(h)}\right]} \leq b_5 \varepsilon n.
\end{equation*}
To apply Theorem \ref{Theo: Theo_2.2.27_Talagrand}, notice that, by the isoperimetric inequality $\diam(\Gamma_0(n)) = \sup_{\gamma_1,\gamma_2\in\Gamma_0(n)} \leq 4\varepsilon\sqrt{|\I(n)|}\leq 4\varepsilon n^{\frac{1}{2}(1 + \frac{1}{d-1})}$. Hence, for $\varepsilon$ small enough
\begin{align*}
    \mathbb{P}\left( \sup_{\gamma}\Delta_{\I(\gamma)}(h) \geq \frac{c_2}{2}n \right) \leq \mathbb{P}\left( \sup_{\gamma}\Delta_{\I(\gamma)}(h) \geq L(b_5 \varepsilon n + n^{\frac{1}{2}(1 - \frac{1}{d-1})}n^{\frac{1}{2}(1 + \frac{1}{d-1})}) \right)\leq e^{-\frac{n^{(1 - \frac{1}{d-1})}}{\varepsilon^2}}.
\end{align*}
The union bound yields $\mathbb{P}(\mathcal{E}^c)\leq e^{-\frac{C}{\varepsilon^2}}$ for $C>0$ large enough.


\section{Phase transition}      \begin{theorem}
For $d\geq 3$ and $\alpha>d+1$, there exists a constant $C\coloneqq C(d,\alpha)$ such that, for all $\beta>0$, $e\leq C$ and $N\geq 1$, the event 
    \begin{equation}\label{Eq: PTLR}
        \mu_{\Lambda; \beta, \varepsilon h}^+(\sigma_0 = -1) \leq e^{-C\beta} + e^{-C/\varepsilon^2} 
    \end{equation}
    has $\mathbb{P}$-probability bigger then $1 - e^{-C\beta} - e^{-C/\varepsilon^2}$.\\
    
In particular, for $\beta>\beta_c$ and $\varepsilon$ small enough, there is phase transition for the long-range Ising model.  
\end{theorem}

\begin{proof}
        The proof is an application of the Peierls' argument, but now on the joint measure $\mathbb{Q}$. By Proposition \ref{Prop: Bound.bad.event.1}, we have
        \begin{align}\label{Eq: Upper.bound.on.Q.1}
            \mathbb{Q}_{\Lambda; \beta, \varepsilon}^+(\sigma_0 = -1) &=  \mathbb{Q}_{\Lambda; \beta, \varepsilon}^+(\sigma_0 = -1 \cap \mathcal{E}) + \mathbb{Q}_{\Lambda; \beta, \varepsilon}^+(\sigma_0 = -1\cap \mathcal{E}^c) \nonumber \\
            %
            & \leq \mathbb{Q}_{\Lambda; \beta, \varepsilon}^+(\sigma_0 = -1 \cap \mathcal{E}) +  e^{-C_1/\varepsilon^2} \nonumber \\
            %
        \end{align}
since $\mathbb{Q}_{\Lambda; \beta, \varepsilon}^+(\sigma_0 = -1\cap \mathcal{E}^c) \leq \mathbb{Q}_{\Lambda; \beta, \varepsilon}^+(\mathcal{E}^c) = \mathbb{P}(\mathcal{E}^c)$.  When $\sigma_0 = -1$, there must exist a contour $\gamma$ with $0\in V(\gamma)$, hence
\begin{equation*}
    \mu_{\Lambda; \beta, \varepsilon h}^+(\sigma_0 = -1) \leq \sum_{\gamma \in \mathcal{C}_0}\mu_{\Lambda; \beta, \varepsilon h}^+(\Omega(\gamma)),
\end{equation*}
where $\Omega(\gamma) \coloneqq \{\sigma\in\Omega : \gamma \subset \Gamma(\sigma)\}$. So we can write

\begin{align}\label{Eq: Upper.bound.on.Q.2}
    \mathbb{Q}_{\Lambda; \beta, \varepsilon}^+(\sigma_0 = -1 \cap \mathcal{E}) &= \int_{\mathcal{E}}\sum_{\sigma : \sigma_0 = -1}g_{\Lambda; \beta, \varepsilon}^+(\sigma, h)dh \nonumber \\
    %
    &\leq  \sum_{\mathcal{C}_0} \int_{\mathcal{E}}\sum_{\gamma\in \sigma\in\Omega(\gamma)}g_{\Lambda; \beta, \varepsilon}^+(\sigma, h)dh \nonumber \\
    %
    &\leq  \sum_{\gamma \in \mathcal{C}_0} \frac{2^{|\gamma|}\int_{\mathcal{E}}\sum_{\sigma\in\Omega(\gamma)}g_{\Lambda; \beta, \varepsilon}^+(\sigma, h)dh}{\int_{\mathcal{E}}\sum_{\sigma\in\Omega(\gamma)}g_{\Lambda; \beta, \varepsilon}^+(\tau_{\gamma, \sigma}(\sigma), \tau_{\I_-(\gamma)}(h))dh} \nonumber \\
    %
    & \leq \sum_{\gamma\in\mathcal{C}_0}2^{|\gamma|} \sup_{\substack{h\in\mathcal{E}\\ \sigma\in\Omega(\gamma)}}\frac{g_{\Lambda; \beta, \varepsilon}^+(\sigma, h)}{g_{\Lambda; \beta, \varepsilon}^+(\tau_{\gamma, \sigma}(\sigma), \tau_{\I_-(\gamma)}(h))}.
\end{align}

In the third equation, we used that $\int_{\mathcal{E}}\sum_{\sigma\in\Omega(\gamma)}g_{\Lambda; \beta, \varepsilon}^+(\tau_{\gamma, \sigma}(\sigma), \tau_{\I_-(\gamma)}(h))dh \leq 2^{|\gamma|}$, since the number of configurations that are incorrect in $\Sp(\gamma)$ are bounded by $2^{|\gamma|}$. By \eqref{Eq: quotient.of.gs} and the definition of the event $\mathcal{E}$, 
\begin{align}\label{Eq: Upper.bound.on.Q.3}
    \sup_{\substack{h\in\mathcal{E}\\ \sigma\in\Omega(\gamma)}}\frac{g_{\Lambda; \beta, \varepsilon}^+(\sigma, h)}{g_{\Lambda; \beta, \varepsilon}^+(\tau_{\gamma, \sigma}(\sigma), \tau_{\I_-(\gamma)}(h))} &\leq \sup_{\substack{h\in\mathcal{E}\\ \sigma\in\Omega(\gamma)}}  \exp{\{{- \beta c_2 |\gamma|}\}}\frac{Z_{\Lambda; \beta, \varepsilon}^{+}(\tau_{\I_-(\gamma)}(h))}{Z_{\Lambda; \beta, \varepsilon}^{+}(h)} \nonumber\\
    %
    &= \sup_{\substack{h\in\mathcal{E}\\ \sigma\in\Omega(\gamma)}}  \exp{\{{- \beta c_2 |\gamma| + \beta \Delta_{\gamma}(h)}\}} \nonumber\\
    %
    &\leq  \exp{\{{- \beta \frac{c_2}{2} |\gamma| }\}},
\end{align}
since $\Delta_{\gamma}(h) \leq \frac{1}{2}(c_2|\gamma|)$, for all $h\in\mathcal{E}_1$. Equations \eqref{Eq: Upper.bound.on.Q.1}, \eqref{Eq: Upper.bound.on.Q.2} and \eqref{Eq: Upper.bound.on.Q.3} yields
\begin{align*}
     \mathbb{Q}_{\Lambda; \beta, \varepsilon}^+(\sigma_0 = -1) &\leq  \sum_{\substack{\gamma\in \mathcal{E}_\Lambda^+\\ 0\in V(\gamma)}} 2^{|\gamma|}\exp{\{{- \beta \frac{c_2}{2} |\gamma| }\}} + e^{-c_0/\varepsilon^2}\\
     &\leq \sum_{n\geq 1}\sum_{\substack{\gamma\in \mathcal{E}_\Lambda^+, |\gamma|=n \\ 0\in V(\gamma)}} \exp{\{{(-\beta \frac{c_2}{2} + \ln2)n}\}} + e^{-c_0/\varepsilon^2}\\
     %
     &\leq \sum_{n\geq 1}|\mathcal{C}_0(n)| \exp{\{{(-\beta \frac{c_2}{2} +\ln2)n}\}} + e^{-c_1/\varepsilon^2} \leq \sum_{n\geq 1} e^{(c_1 -\beta \frac{c_2}{2} +\ln2)n} + e^{-c_0/\varepsilon^2}. \\
\end{align*}
When $\beta$ is large enough, the sum above converges and there exists a constant $C$ such that   
\begin{equation*}
    \mathbb{Q}_{\Lambda; \beta, \varepsilon}^+(\sigma_0 = -1) \leq e^{-\beta 2C} + e^{-2C / \varepsilon^2}.
\end{equation*}
The Markov Inequality finally yields
\begin{align*}
    \mathbb{P}\left( \mu_{\Lambda; \beta, \varepsilon h}^+(\sigma_0 = -1) \geq e^{-C\beta} + e^{-C/\varepsilon^2}\right) &\leq \frac{\mathbb{Q}_{\Lambda; \beta, \varepsilon}^+(\sigma_0 = -1)}{e^{-C\beta} - e^{-C/\varepsilon^2}} \\
    %
    &\leq \frac{e^{-\beta 2C} + e^{-2C / \varepsilon^2}}{e^{-C\beta} - e^{-C/\varepsilon^2}} \leq e^{-C\beta} - e^{-C/\varepsilon^2},
\end{align*}
what proves our claim.
\end{proof}

The natural question that comes to mind is if there is phase transition for $d<\alpha<d+1$. In this region, a recent paper \cite{Affonso.2021} introduces a new notion of contour, based in a construction by Frohlich and Spencer \cite{Frohlich.Spencer.82}, and with them they prove phase transition. The key difference in these contours is that they are no longer connected, so the count argument on Proposition \ref{Prop: Proposition_2_FFS} does not hold. Nevertheless, using another counting method and this new contours, it should be possible to prove:

\begin{conjecture*}
    Let $\Gamma_0$ be the set of contours with 0 in its volume, as defined in \cite{Affonso.2021}. Then, for any $\alpha>d$, $d\geq 3$, there exists $C_2\coloneqq C_2(\alpha, d)$ such that $\mathbb{P}(\left\{\sup_{\substack{\gamma\in\Gamma_0}} \frac{|\Delta_{\I_-(\gamma)}(h)|}{|\gamma|} > 1\right\})\leq e^{-\frac{C_2}{\varepsilon^2}}$. 

    In particular, for $\beta>\beta_c$ and $\varepsilon$ small enough, there is phase transition for the long-range Ising model in $d<\alpha\leq d+1$ and $d\geq 3$.  
\end{conjecture*}

\begin{conjecture*}
    For $d\geq 3$ and $d<\alpha \leq d+1$, there exists a constant $C^\prime\coloneqq C^\prime(d,\alpha)$ such that, for all $\beta>0$, $e\leq C^\prime$ and $N\geq 1$, the event 
    \begin{equation}
        \mu_{\Lambda; \beta, \varepsilon h}^+(\sigma_0 = -1) \leq e^{-C^\prime\beta} + e^{-C^\prime/\varepsilon^2} 
    \end{equation}
    has $\mathbb{P}$-probability bigger then $1 - e^{-C^\prime\beta} - e^{-C^\prime/\varepsilon^2}$.\\
    
In particular, for $\beta>\beta_c$ and $\varepsilon$ small enough, there is phase transition for the long-range Ising model.  
\end{conjecture*}  


\section{Concluding Remarks}

In this paper, we proved phase transition for the long-range Ising model in $d\geq 3$ and $\alpha >d+1$, by following a new method of proving phase transition introduced by Ding and Zhuang \cite{Ding2021}, and using the contours argument of \cite{Ginibre.Grossmann.Ruelle.66}. We expect that this me can be extended to other models with a contour system, in particular for the long-range model in the region $d<\alpha\leq d+1$, using the contours introduced in \cite{Affonso.2021}.

The results presented by Bricmont and Kupiainen \cite{Bricmont.Kupiainen.88} are more general than ours, since they only need the external field to be symmetric around zero and have a sub-Gaussian tail. In \cite{Ding2021}, Ding and Zhuang claim that it should be possible, with more care, to extend their results to an external field in the same generality.

Another natural question is to investigate the smaller dimensions $d=1, 2$. Aizenman and Wehr, in \cite{Aizenman.Wehr.90}, proved that for $d\leq 2$ there is uniqueness when $\alpha > 3d/2$. This shows that the bound on $\mathbb{P}(\mathcal{E}_1^c)$ should depend on $\alpha$. %Moreover, in \cite{Ding.Wirth.20}, it is shown that in $d=2$, $\mathbb{E}(\sup_{\substack{A\Subset\Z^d\\ A \textrm{ connected}}})$


\section*{Acknowledgements}

LA is supported by FAPESP Grant 2017/18152-2 and 2020/14563-0. RB is supported by CNPq grants 312294/2018-2 and 408851/2018-0, by FAPESP grant 16/25053-8, and by the University Center of Excellence \textquotedblleft Dynamics, Mathematical Analysis and Artificial Intelligence\textquotedblright, at the Nicolaus Copernicus University; JM is supported by FAPESP Grant 2018/26698-8 and 2016/25053-8. 

	
%%%%%%%% References

\bibliographystyle{habbrv} 
\bibliography{main} 





\end{document}
