\begin{theorem}
For $d\geq 3$ and $\alpha>d$, there exists a constant $C\coloneqq C(d,\alpha)$ such that, for all $\beta>0$ and $e\leq C$, the event 
    \begin{equation}\label{Eq: PTLR}
        \nu_{\Lambda; \beta, \varepsilon h}^+(\sigma_0 = -1) \leq e^{-C\beta} + e^{-C/\varepsilon^2} 
    \end{equation}
    has $\mathbb{P}$-probability bigger then $1 - e^{-C\beta} - e^{-C/\varepsilon^2}$.\\
    
In particular, for $\beta>\beta_c$ and $\varepsilon$ small enough, there is phase transition for the long-range Ising model.  
\end{theorem}

\begin{proof}
        The proof is an application of the Peierls' argument, but now on the joint measure $\mathbb{Q}$. Define $\mathcal{E} = \mathcal{E}_0 \cap \mathcal{E}_1$. By Proposition \ref{Prop: Bound.bad.event.0}, we have
        \begin{align}\label{Eq: Upper.bound.on.Q.1}
            \mathbb{Q}_{\Lambda; \beta, \varepsilon}^+(\sigma_0 = -1) &=  \mathbb{Q}_{\Lambda; \beta, \varepsilon}^+(\sigma_0 = -1 \cap \mathcal{E}_0) + \mathbb{Q}_{\Lambda; \beta, \varepsilon}^+(\sigma_0 = -1\cap \mathcal{E}_0^c) \nonumber \\
            %
            & \leq \mathbb{Q}_{\Lambda; \beta, \varepsilon}^+(\sigma_0 = -1 \cap \mathcal{E}_0) +  e^{-C_0/\varepsilon^2} \nonumber \\
            %
            & \leq \mathbb{Q}_{\Lambda; \beta, \varepsilon}^+(\sigma_0 = -1 \cap \mathcal{E}) + \mathbb{Q}_{\Lambda; \beta, \varepsilon}^+(\sigma_0 = -1 \cap \mathcal{E}_0 \cap \mathcal{E}_{1}^c)  + e^{-C_0/\varepsilon^2} \nonumber \\
            %
            & \leq \mathbb{Q}_{\Lambda; \beta, \varepsilon}^+(\sigma_0 = -1 \cap \mathcal{E}) + e^{-C_1/\varepsilon^2}  + e^{-C_0/\varepsilon^2},
        \end{align}
since $\mathbb{Q}_{\Lambda; \beta, \varepsilon}^+(\sigma_0 = -1\cap \mathcal{E}_0^c) \leq \mathbb{Q}_{\Lambda; \beta, \varepsilon}^+(\mathcal{E}_0^c) = \mathbb{P}(\mathcal{E}_0^c)$ and, analogously, ${\mathbb{Q}_{\Lambda; \beta, \varepsilon}^+(\sigma_0 = -1 \cap \mathcal{E}_0 \cap \mathcal{E}_{1}^c) \leq \mathbb{P}(\mathcal{E}_1^c)}$.  When $\sigma_0 = -1$, there must exist a contour $\gamma$ with $0\in V(\gamma)$, hence
\begin{equation*}
    \nu_{\Lambda; \beta, \varepsilon h}^+(\sigma_0 = -1) \leq \sum_{\gamma \in \mathcal{C}_0}\nu_{\Lambda; \beta, \varepsilon h}^+(\Omega(\gamma)),
\end{equation*}
where $\Omega(\gamma) \coloneqq \{\sigma\in\Omega : \gamma \subset \Gamma(\sigma)\}$. So we can write

\begin{align}\label{Eq: Upper.bound.on.Q.2}
    \mathbb{Q}_{\Lambda; \beta, \varepsilon}^+(\sigma_0 = -1 \cap \mathcal{E}) &= \int_{\mathcal{E}}\sum_{\sigma : \sigma_0 = -1}g_{\Lambda; \beta, \varepsilon}^+(\sigma, h)dh \nonumber \\
    %
    &\leq  \sum_{\gamma\in\mathcal{C}_0} \int_{\mathcal{E}}\sum_{\sigma\in\Omega(\gamma)}g_{\Lambda; \beta, \varepsilon}^+(\sigma, h)dh \nonumber \\
    %
    &\leq  \sum_{\gamma \in \mathcal{C}_0} \frac{2^{|\gamma|}\int_{\mathcal{E}}\sum_{\sigma\in\Omega(\gamma)}g_{\Lambda; \beta, \varepsilon}^+(\sigma, h)dh}{\int_{\mathcal{E}}\sum_{\sigma\in\Omega(\gamma)}g_{\Lambda; \beta, \varepsilon}^+(\tau_{\gamma}(\sigma), \tau_{\I_-(\gamma)}(h))dh} \nonumber \\
    %
    & \leq \sum_{\gamma\in\mathcal{C}_0}2^{|\gamma|} \sup_{\substack{h\in\mathcal{E}\\ \sigma\in\Omega(\gamma)}}\frac{g_{\Lambda; \beta, \varepsilon}^+(\sigma, h)}{g_{\Lambda; \beta, \varepsilon}^+(\tau_{\gamma}(\sigma), \tau_{\I_-(\gamma)}(h))}. 
\end{align}

In the third equation, we used that $\int_{\mathcal{E}}\sum_{\sigma\in\Omega(\gamma)}g_{\Lambda; \beta, \varepsilon}^+(\tau_{\gamma, \sigma}(\sigma), \tau_{\I_-(\gamma)}(h))dh \leq 2^{|\gamma|}$, since the number of configurations that are incorrect in $\Sp(\gamma)$ are bounded by $2^{|\gamma|}$. By \eqref{Eq: quotient.of.gs} and the definition of the event $\mathcal{E}$, 
\begin{align}\label{Eq: Upper.bound.on.Q.3}
    \sup_{\substack{h\in\mathcal{E}\\ \sigma\in\Omega(\gamma)}}\frac{g_{\Lambda; \beta, \varepsilon}^+(\sigma, h)}{g_{\Lambda; \beta, \varepsilon}^+(\tau_{\gamma, \sigma}(\sigma), \tau_{\I_-(\gamma)}(h))} &\leq \sup_{\substack{h\in\mathcal{E}\\ \sigma\in\Omega(\gamma)}}  \exp{\{{- \beta c_2 |\gamma| -2\beta\sum_{x\in \Sp^-(\gamma)}\varepsilon h_x}\}}\frac{Z_{\Lambda; \beta, \varepsilon}^{+}(\tau_{\I_-(\gamma)}(h))}{Z_{\Lambda; \beta, \varepsilon}^{+}(h)} \nonumber\\
    %
    &= \sup_{\substack{h\in\mathcal{E}\\ \sigma\in\Omega(\gamma)}}  \exp{\{{- \beta c_2 |\gamma| -2\beta\sum_{x\in \Sp^-(\gamma)}\varepsilon h_x + \beta \Delta_{\gamma}(h)}\}} \nonumber\\
    %
    &\leq  \exp{\left\{{- \beta \frac{c_2}{2} |\gamma| }\right\}},
\end{align}
since $\Delta_{\gamma}(h) -2\beta\sum_{x\in \Sp^-(\gamma)}\varepsilon h_x \leq \frac{c_2}{2}|\gamma|$, for all $h\in\mathcal{E}$. Equations \eqref{Eq: Upper.bound.on.Q.1}, \eqref{Eq: Upper.bound.on.Q.2} and \eqref{Eq: Upper.bound.on.Q.3} yields
\begin{align*}
     \mathbb{Q}_{\Lambda; \beta, \varepsilon}^+(\sigma_0 = -1) &\leq  \sum_{\substack{\gamma\in \mathcal{E}_\Lambda^+\\ 0\in V(\gamma)}} 2^{|\gamma|}e^{{- \beta \frac{c_2}{2} |\gamma| }} + e^{-C_1/\varepsilon^2} + e^{-C_0/\varepsilon^2}\\
     &\leq \sum_{n\geq 1}\sum_{\substack{\gamma\in \mathcal{E}_\Lambda^+, |\gamma|=n \\ 0\in V(\gamma)}} e^{{(-\beta \frac{c_2}{2} + \ln2)n}} + e^{-C_1/\varepsilon^2} + e^{-C_0/\varepsilon^2}\\
     %
     &\leq \sum_{n\geq 1}|\mathcal{C}_0(n)| e^{{(-\beta \frac{c_2}{2} +\ln2)n}} + e^{-C_1/\varepsilon^2} + e^{-C_0/\varepsilon^2}\leq \sum_{n\geq 1} e^{(c_1 -\beta \frac{c_2}{2} +\ln2)n} + e^{-C_1/\varepsilon^2} + e^{-C_0/\varepsilon^2}. \\
\end{align*}
When $\beta$ is large enough, the sum above converges and there exists a constant $C$ such that   
\begin{equation*}
    \mathbb{Q}_{\Lambda; \beta, \varepsilon}^+(\sigma_0 = -1) \leq e^{-\beta 2C} + e^{-2C / \varepsilon^2}.
\end{equation*}
The Markov Inequality finally yields
\begin{align*}
    \mathbb{P}\left( \nu_{\Lambda; \beta, \varepsilon h}^+(\sigma_0 = -1) \geq e^{-C\beta} + e^{-C/\varepsilon^2}\right) &\leq \frac{\mathbb{Q}_{\Lambda; \beta, \varepsilon}^+(\sigma_0 = -1)}{e^{-C\beta} - e^{-C/\varepsilon^2}} \\
    %
    &\leq \frac{e^{-\beta 2C} + e^{-2C / \varepsilon^2}}{e^{-C\beta} - e^{-C/\varepsilon^2}} \leq e^{-C\beta} - e^{-C/\varepsilon^2},
\end{align*}
what proves our claim.
\end{proof}