The set of configurations of the long-range Ising model is, as usual, $\Omega \coloneqq \{-1,1\}^{\Z^d}$. However, each spin interacts with all others, not only its neighbors, so the interaction $\{J_{xy}\}_{x,y\in\Z^d}$ is defined as

\begin{equation}\label{Long-Range Interaction}
    J_{xy} = \begin{cases}
                   \frac{J}{|x-y|^\alpha} &\text{ if }x\neq y,\\
                   0                &\text{otherwise}. 
              \end{cases}
\end{equation}
where $J >0$, $\alpha>d$ and the distance $|x-y|$ is given by the $\ell_1$-norm. The external field is a family $\{h_x\}_{x\in\Z^d}$ of i.i.d. random variables in $(\widetilde{\Omega}, \mathcal{A}, \mathbb{P})$, and every $h_x$ has a standard normal distribution. Our results also hold for more general distributions of $h_x$, see Remarks \ref{Rmk: Bernoulli_external_field} and \ref{Rmk: More.general.h_x}. 

We write $\Lambda\Subset \Z^d$ to denote a finite subset of $\Z^d$. Fixed such $\Lambda$, the \textit{local configurations} are $\Omega_\Lambda\coloneqq \{-1,1\}^\Lambda$. Moreover, given ${\eta\in\Omega}$, the set of local configurations with $\eta$ boundary conditions is ${\Omega_\Lambda^\eta\coloneqq \{\sigma\in\Omega : \sigma_x=\eta_x, \text{ }\forall x\in\Lambda^c\}}$. The \textit{local Hamiltonian of the random field long-range Ising model} in $\Lambda\Subset\Z^d$ with $\eta$-boundary condition is $H_{\Lambda, \varepsilon h}^{\eta}:  \Omega_\Lambda^\eta \to \mathbb{R}$, given by 

\begin{equation}
    H_{\Lambda; \varepsilon h}^{\eta}(\sigma)= -\sum_{x,y\in\Lambda} J_{x,y}\sigma_x\sigma_y - \sum_{x\in \Lambda, y\in\Lambda^c} J_{x,y}\sigma_x\eta_y - \sum_{x\in\Lambda} \varepsilon h_x\sigma_x,
\end{equation}
where $\varepsilon >0$ is a parameter that controls the variance of the external field. Given $\Lambda\Subset\Z^d$, consider $\mathscr{F}_\Lambda$ the $\sigma$-algebra generated by the cylinders sets supported in $\Lambda$ and $\mathscr{F}$ the $\sigma$-algebra generated by finite union of cylinders. One of the main objects of study in classical statistical mechanics is the \textit{finite volume Gibbs measures}, which are probability measures in $(\Omega, \mathscr{F})$, given by 
    \begin{equation}
        \mu_{\Lambda;\beta, \varepsilon h}^\eta(\sigma) = \mathbbm{1}_{\Omega_\Lambda^\eta}(\sigma)\frac{e^{-\beta H_{\Lambda, \varepsilon h}^{\eta}(\sigma)}}{Z_{\Lambda; \beta, \varepsilon}^{\eta}(h)},
    \end{equation}
where $\beta>0$ is the inverse temperature and $Z_{\Lambda; \beta, \varepsilon}^{\eta}$ is called \textit{partition function}, defined as 

\begin{equation}
    Z_{\Lambda; \beta, \varepsilon}^{\eta}(h)\coloneqq \sum_{\sigma\in\Omega_\Lambda^\eta} e^{-\beta H_{\Lambda, \varepsilon h}^{\eta}(\sigma)}.
\end{equation}
One important remark is that, since the external field is random, the Gibbs measures are random variables. To explicit the dependence of $\mu_{\Lambda;\beta, \varepsilon h}^\eta$ on $\widetilde{\Omega}$, we will often write $\mu_{\Lambda;\beta, \varepsilon h}^\eta[\omega]$, with $\omega$ being a general element of $\widetilde{\Omega}$. Two particularly important boundary conditions are given by the configurations $\eta_{+} \equiv +1$ and $\eta_{-} \equiv -1$, and are called $+$ and $-$ boundary conditions, respectively. For these boundary conditions, we can $\mathds{P}$-almost surely define the infinite volume measures by taking the weak*-limit
\begin{equation}
    \mu_{\beta,\varepsilon h}^{\pm}[\omega] = \lim_{n\to\infty} \mu_{\Lambda_n;\beta, \varepsilon h}^{\pm}[\omega],
\end{equation}
where $(\Lambda_n)_{n\in\mathbb{N}}$ is any sequence invading $\Z^d$, that is, for any subset $\Lambda\Subset\mathbb{Z}^d$, there exists $N=N(\Lambda)>0$ such that $\Lambda\subset\Lambda_n$ for every $n>N$.
To have more than one Gibbs measure, it is enough to show that $\mu_{\beta,\varepsilon h}^{+}[\omega]\neq  \mu_{\beta,\varepsilon h}^{-}[\omega]$, with $\mathbb{P}$-probability 1, see \cite[Theorem 7.2.2]{Bovier.06}.

