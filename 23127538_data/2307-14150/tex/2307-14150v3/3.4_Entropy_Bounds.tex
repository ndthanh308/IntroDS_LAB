As we discussed before, in the definition of admissibility, $|B_\ell(\mathcal{C}_0(n,j))| = |\partial \mathfrak{C}_{r\ell}(\mathcal{C}_0(n,j))|$. In the short-range case, a key ingredient to count the admissible cubes is that despite $B_\ell(\gamma)$ not being connected, all cubes are inside a connected region with size $|\gamma|$. As the contours now are not connected, we need to change the strategy: we choose a suitable scale $L(\ell)$ and count how many $rL(\ell)$-coverings of $\gamma$ there are. That is, we first control $|\C_{rL(\ell)}(\mathcal{C}_0(n,j))|$. Once the $rL(\ell)$-covering is fixed, we choose which $r\ell$-cubes inside this covering will be admissible. At last, we choose the scale $L(\ell)$ in a suitable way. This has to be done with some care, since the behavior of $|\C_{rL(\ell)}(\mathcal{C}_0(n,j))|$ depends if $L(\ell)<j$ or not.

The first step is to bound $|\C_{rL}(\mathcal{C}_0(n,j))|$, for $L>0$. For $n,m\geq 0$, we say that $\mathscr{C}_n$ is \textit{subordinated} to $\C_m$, denoted by $\C_n\preceq \C_m$, if $\C_m = \C_m(B_{\C_n})$. Moreover, define 
\begin{equation*}
    N(\C_m, n, V) \coloneqq\{\C_n : \C_n\preceq \C_m, |\C_n|=V\},
\end{equation*}
the number of collections of $n$-cubes $\C_n$ subordinated to a fixed collection $\C_m$ and with $|\C_n|=V$. Notice that every $m$-cube contains $2^d$ $(m-1)$-cubes, all of them being disjoint. Therefore, the number of $n$-cubes inside a $m$-cube is $2^{(m-n)d}$ and we have $N(\C_m, n, V) = \binom{2^{(m-n)d} |\C_{m}|}{V}$.
In particular, the bound on the binomial $\binom{n}{k}\leq \left(\frac{en}{k}\right)^k$ yields
\begin{equation}\label{Eq: Bound.on.N}
    N(\C_{r(\ell+1)}, r\ell, V) = \binom{2^{rd}|\C_{r(\ell+1)}|}{V} \leq \left(\frac{2^{rd}e|\C_{r(\ell+1)}|}{V}\right)^{V}.
\end{equation}
For any subset $\Lambda \Subset \Z^d$, define
\begin{equation*}
    V_r^\ell(\Lambda)\coloneqq \sum_{n=\ell}^{n_r(\Lambda)} |\C_{rn}(\Lambda)|,
\end{equation*}
where $n_r(\Lambda)\coloneqq \ceil{\log_{2^r}(\diam (\Lambda))}$. To control $V_r^\ell$ we control the number of coverings at a fixed step $L>0$.

\begin{proposition}\label{Prop. partition.a.graph}
Let $k\geq 1$ and $G$ be a finite, non-empty, connected simple graph with vertex set $v(G)$. Then, $G$ can be covered by $\ceil*{|v(G)|/k}$ connected sub-graphs of size at most $2k$.
\end{proposition}
We omit the proof since it is the same as in \cite{Affonso.2021}. Remember that, for $A\Subset \Z^d$ and $j\geq 1$, $\Gamma^r_j(A)$ are the elements of the partition removed at step $j$, in the construction presented in Section 2. Using this construction we can prove the following lemma.

\begin{lemma}\label{Lemma: Big.clusters_2}
    Let $A\Subset \Z^d$, $\gamma\in\Gamma^r(A)$ and $j \geq 1$ be such that $\gamma\in \Gamma^r_j(A)$. Then, for any $\ell < j$ and $G_{r\ell}\in \mathscr{G}_{r\ell}(\gamma)$,
    \begin{equation}\label{Eq: Lower_bound_on_the_covering_of_gamma_G}
        2^{r(1-\frac{1}{d})\ell} \leq |\C_{r\ell}(\gamma_{G_{r\ell}})| 
    \end{equation}
\end{lemma}
\begin{proof}
        Given $G_{r\ell}\in \mathscr{G}_{r\ell}(\gamma)$, by our construction of the contour, $2^{r(d+1)\ell}\leq |V(\gamma_{G_{r\ell}})|$. A trivial bound gives us $|V(\gamma_{G_{r\ell}})| \leq 2^{r\ell d}|\C_{r\ell}(V(\gamma_{G_{r\ell}}))|$. Associating each cube $C_m(x)$ to its center $x$, we get a one-to-one correspondence between $m$-cubes and lattice points that preserves neighbors, that is, two m-cubes $C_m(x)$ and $C_m(y)$ share a face if and only if $|x-y|=1$. We can therefore apply the isoperimetric inequality to get $|\C_{r\ell}(V(\gamma_{G_{r\ell}}))| \leq |\fint \C_{r\ell}(V(\gamma_{G_{r\ell}}))|^{\frac{d}{d-1}}\leq |\C_{r\ell}(\gamma_{G_{r\ell}})|^{\frac{d}{d-1}}$, where in the last equation we are using that every cube in the boundary of cubes must cover at least one point of $\gamma_{G_{r\ell}}$. We conclude that $2^{r(d+1)\ell} \leq 2^{r\ell d}|\C_{r\ell}(\gamma_{G_{r\ell}})|^{\frac{d}{d-1}}$, and \eqref{Eq: Lower_bound_on_the_covering_of_gamma_G} follows.
\end{proof}

As a corollary, we can recuperate a key lemma of \cite{Affonso.2021}, which is the following.
\begin{lemma}\label{Lemma: Big.clusters_1}
    Given $A\Subset \Z^d$, $n> 1$ and $\gamma\in\Gamma(A)$, if $|\mathscr{G}_{rn}(\gamma)|\geq 2$ then $|v(G_{rn}(\gamma))| \geq 2^r$ for every $G_{rn}(\gamma)\in \mathscr{G}_{rn}(\gamma)$ 
\end{lemma}


   The next proposition bounds the partial volume.
\begin{proposition}\label{Prop. Bound.on.V_r^l(gamma)}
    There exists a constant $b_3 \coloneqq b_3(d, M, r)$ such that, for any $A\Subset \Z^d$, $\gamma\in\Gamma(A)$ and $0 \leq \ell$,
    
     \begin{equation*}
        V_r^\ell(\gamma)\leq b_3 (\ell\wedge 1)^{\frac{r-d-1}{\log_2(a)}} |\mathscr{C}_{r\ell}|. 
    \end{equation*}

\end{proposition}

    
\begin{proof}
Fix $\xi\in \Gamma(A)$ with $\xi=B_{\C_{r\ell}}$. Let's first assume $\ell\geq 2$. Define $g : \mathbb{N} \xrightarrow{} \Z$ by
\begin{equation}
    g(n)\coloneqq \floor*{\frac{n - 2 - \log_{2^r}(2M)}{a}}.
\end{equation}
It was proved in \cite[Proposition 3.13]{Affonso.2021} that 
\begin{equation}\label{Eq: Bound_c_n_by_C_g(n)}
    |\C_{rn}(\xi)| \leq \frac{1}{2^{r-d-1}}|\C_{rg(n)}(\xi)|,
\end{equation}
whenever $g(n)>0$, and every connected component of $G_{rg(n)}(\xi)$ has more than $2^r -1$ vertices. This is equivalent,  by Lemma \ref{Lemma: Big.clusters_1}, to $|\mathscr{G}_{rg(n)}(\xi)|\geq 2$ or $|\mathscr{G}_{rg(n)}(\xi)|=1$ with $|v(G_{rg(n)}(\xi))| \geq 2^r$. Consider then the auxiliary quantities
\begin{align*}
    &l_1(n)\coloneqq\max\{m : g^m(n)\geq \ell\} &\text{and} &&l_2(n)\coloneqq\max\{ m : |\mathscr{G}_{rg^m(n)}(\xi)| = 1 \text{ and } |v(G_{rg^m(n)})|\leq 2^r-1\}.
\end{align*}
Notice that, to cover a $r\ell$-cube, we need $2^{rd}$ $r(\ell-1)$-cubes. As $\xi$ is the region covered by $r\ell$-cubes, this implies that we need at least $2^{rd}$ $r(\ell-1)$-cubes to cover $\xi$, hence $\ell-1 < g^{l_2(n)}(n)$ and therefore $l_2(n)\leq l_1(n)$. Knowing that $|\C_{k}(\xi)|\leq |\C_j(\xi)|$, for all $j\leq k$, we get 
\begin{equation}\label{Eq: bound.on.rn.covering}
    |\C_{rn}(\xi)|\leq |\C_{rg^{l_2(n)}(n)}(\xi)|\leq \frac{1}{2^{(r-d-1)(l_1(n)-l_2(n))}}|\C_{r\ell}(\xi)|.
\end{equation}
We claim that
\begin{equation}\label{Eq: lower.bound.on.l1}
    l_1(n) \geq \begin{cases}
                        0, &\text{ if }n\leq \overline{b}+\ell\\ 
                        \left\lfloor\frac{\log_2(n) - \log_2(\overline{b} + \ell)}{\log_2(a)}\right\rfloor, & \text{ if }n > \overline{b} + \ell, 
                \end{cases}
\end{equation}
where $\overline{b} = (a+2 + \log_{2^r}(2M))(a-1)^{-1}$. Given $n > \overline{b} + \ell$, consider
\begin{equation*}
    \Tilde{g}(n) = \frac{n - 2 - \log_{2^r}(2M)}{a} - 1.
\end{equation*}
It is clear that $g(n)\geq \Tilde{g}(n)$ and both functions are increasing, therefore $g^m(n)\geq \Tilde{g}^m(n)$ for every $m\geq 0$. As
\begin{equation*}
    \Tilde{g}^m(n) = \frac{n}{a^m} - b^\prime\frac{a^m - 1}{a^{m-1}(a-1)},
\end{equation*}
with $b^\prime = (a+2 + \log_{2^r}(2M))a^{-1}$, it is sufficient to have
\begin{equation*}
    \frac{n}{a^m} -\frac{a b^\prime}{(a-1)}\geq \ell.
\end{equation*}
We get the desired bound by applying the logarithm with base two in the equation above. We now need an upper bound on $l_2(n)$. For any $m\leq l_2(n)$, all cubes in $\C_{rg^m(n)}(\xi)$ are distant at most $M2^{arg^m(n)}$ and $|\C_{rg^m(n)(\xi)}|\leq 2^{r}-1$, therefore
\begin{equation*}
    \diam(\xi)\leq \diam(B_{\C_{rg^m(n)(\xi)}})\leq (d2^{rg^m(n)} + Md^a2^{arg^m(n)})|\C_{rg^m(n)(\xi)}|\leq 2Md^a2^{arg^m(n)+r}.
\end{equation*}
Applying the logarithm with respect to base $2^{r}$ we get
\begin{equation*}
    \log_{2^r}(\diam(\xi)) \leq \log_{2^r}(2M) + ag^m(n)+1 \leq \log_{2^r}(2M) + \frac{n}{a^{m-1}} + 1
\end{equation*}
Assuming $\diam(\xi)>2^{2r + 1}M$, we can isolate the term depending on $m$ in the equation above and take to logarithm in both sides to get
\begin{equation*}
    m \leq 1 + \frac{\log_2(n) - \log_2(\log_{2^r}(\diam(\xi)) - \log_{2^r}(2M) - 1)}{\log_2(a)}.
\end{equation*}
Equation above holds for any element of $\{m : |\mathscr{G}_{rg^m(n)}(A)| = 1, |v(G_{rg^m(n)})|\leq 2^r-1\}$ thus it also holds for $l_2(n)$. Together with \eqref{Eq: lower.bound.on.l1}, this yields 

\begin{equation}
    l_1(n) - l_2(n)\geq \frac{\log_2[\log_{2^r}(\diam(\xi)) -\log_{2^r}(2M) - 1] - \log_2(\overline{b} + \ell)}{\log_2(a)} - 2.
\end{equation}
Applying this back in equation \eqref{Eq: bound.on.rn.covering}, we get
\begin{align*}
    V_r^\ell(\xi) &\leq (\overline{b} + 1)|\C_{r\ell}(\xi)| + |\C_{r\ell}(\xi)|\frac{2^{2(r-d-1)} (\overline{b} +\ell)^{\frac{r-d-1}{\log_2(a)}}n_r(\xi)}{[\log_{2^r}(\diam(\xi)) - \log_{2^r}(2M) - 1]^{\frac{r-d-1}{\log_2(a)}}} \\
    %
    &\leq [\overline{b} +1 + 2^{2(r-d-1)} (\overline{b} +\ell)^{\frac{r-d-1}{\log_2(a)}}(\log_{2^r}(2M) +3)]|\C_{r\ell}(\xi)|\\
    %
    &\leq [\overline{b} +1 + 2^{2(r-d-1)} (\overline{b} +1)^{\frac{r-d-1}{\log_2(a)}}(\log_{2^r}(2M) +3)]\ell^{\frac{r-d-1}{\log_2(a)}}|\C_{r\ell}(\xi)|
\end{align*}
where in the second equation we used that $(x/(x-w))\leq 1 + w$ for any $x\geq w + 1$. If $\diam(\xi)\leq 2^{2r + 1}M$, we have
\begin{equation*}
     V_r^\ell(\xi) \leq (n_r(\xi) - \ell+1)|\C_{r\ell}(\xi)| \leq (3 + \log_{2^r}(2M))|\C_{r\ell}(\xi)|.
\end{equation*}
Taking $b_3^\prime\coloneqq \overline{b} +1 + 2^{2(r-d-1)} (\overline{b} +1)^{\frac{r-d-1}{\log_2(a)}}(\log_{2^r}(2M) +3)$ we get the desired bound when $\ell\geq 2$. For $\ell=0$, a trivial bound yields  $V_r^0(\gamma) = 2|\gamma| + V_r^2(\gamma)\leq (2 + b_3^\prime 2^{\frac{r-d-1}{\log_2(a)}})|\gamma|$. Similarly, for $\ell =1$,  $V_r^1(\gamma) = |\C_{r}(\gamma)| + V_r^2(\gamma)\leq (1+ b_3^\prime 2^{\frac{r-d-1}{\log_2(a)}})|\mathscr{C}_{r}(\gamma)|$ and we conclude the proof by taking $b_3 \coloneqq 2 + b_3^\prime 2^{\frac{r-d-1}{\log_2(a)}}$.
\end{proof}

We then need to bound the number of $r\ell$-cubes to cover a contour. Using only Lemma \ref{Lemma: Big.clusters_1}, we can prove the next proposition, in the same steps as in \cite[Proposition 3.13]{Affonso.2021}.

\begin{proposition}\label{Prop. Bound.on.C_rl(gamma)_Lucas}
    There exists a constant $b_4^\prime\coloneqq b_4^\prime(\alpha, d)$ such that for any $A\Subset \Z^d$, $\gamma\in\Gamma(A)$ and $\ell\geq 1$, 
     \begin{equation*}
       |\C_{r\ell}(\gamma)|\leq b_4^\prime\frac{|\gamma|}{\ell^{\frac{r-d-1}{\log_2(a)}}}.
    \end{equation*}
\end{proposition}

Using our construction, we can improve this for steps scales smaller than the step the contour was removed. This is done in the next proposition. 
\begin{proposition}\label{Prop. Bound.on.C_rl(gamma)}
    There exists a constant $b_4\coloneqq b_4(\alpha, d)$ such that for any $A\Subset \Z^d$, $\gamma\in\Gamma^r_j(A)$ and $0 \leq \ell<j$,
    
     \begin{equation*}
       |\C_{r\ell}(\gamma)|\leq b_4\frac{(\ell\wedge 1)^{\frac{d+1}{a + (1-\frac{1}{d})}}}{2^{ra^\prime\ell}}|\gamma|,
    \end{equation*}
    where $a^\prime \coloneqq \frac{(1-\frac{1}{d})}{a-1 + (1-\frac{1}{d})}$
\end{proposition}

\begin{proof}
     Define $f : \mathbb{N} \xrightarrow{} \Z$ by
\begin{equation}
    f(\ell)\coloneqq \floor*{\frac{\ell - \log_{2^r}(2M) - 1}{a + (1-\frac{1}{d})}}.
\end{equation}
Following the proof of \eqref{Eq: Bound_c_n_by_C_g(n)} in \cite[Proposition 3.13]{Affonso.2021}, we can show that 
\begin{equation}\label{Eq: Bound_C_l_by_C_f(l)}
    |\C_{r\ell}(\gamma)| \leq \frac{2^{d+1}}{2^{r(1-\frac{1}{d})f(\ell)}}|\C_{rf(\ell)}(\gamma)|.
\end{equation}
 By definition, $\mathscr{G}_{rf(\ell)}(\gamma)$ is the set of all connected components of $G_{rf(\ell)}(\gamma)$, hence
    \begin{equation}\label{Eq: 3.16.Lucas}
        |\C_{rf(\ell)}(\gamma)| = 2^{r(1-\frac{1}{d})f(\ell)}\sum_{G\in \mathscr{G}_{rf(\ell)}(\gamma)}\frac{|v(G)|}{2^{r(1-\frac{1}{d})f(\ell)}}.
    \end{equation}
    Proposition \ref{Prop. partition.a.graph} guarantees that we can split $G$ into sub-graphs $G_i$, with $1\leq i\leq \ceil{v(G)/2^{r(1-\frac{1}{d})f(\ell)}}$ and $|v(G_i)|\leq 2^{r(1-\frac{1}{d})f(\ell)+1}$. For any $\Lambda,\Lambda^\prime\Subset \Z^d$, 
    \begin{equation*}
        \diam(\Lambda\cup \Lambda) \leq \diam(\Lambda) + \diam(\Lambda^\prime) + \d(\Lambda,\Lambda^\prime),
    \end{equation*}
    and we can always extract a vertex from a connected graph in a way that the induced sub-graph is still connected, by removing a leaf of a spanning tree. Using this we can bound 
    \begin{align*}
        \diam(B_{v(G_i)}) &\leq \sum_{C_{rf(\ell)}\in v(G_i)} \diam(C_{rf(\ell)}) + |v(G_i)|M2^{arf(\ell)}\\
        %
        &\leq |v(G_i)|(d2^{rf(\ell)} + M2^{arf(\ell)}) \leq 2M2^{r[f(\ell)((1-\frac{1}{d})) + a] + 1}\\
        %
        &\leq 2^{r\ell}.
    \end{align*}
    
    The last inequality holds since $M,a,r\geq 1$. This shows that every $G_i$ can be covered by a cube with center in $\Z^d$ and side length $2^{r\ell}$. Every such cube can be covered by at most $2^d$ $r\ell$-cubes. Indeed, it is enough to consider the simpler case when the cube is of the form
    \begin{equation}\label{Cube.Q}
        \prod_{i=1}^d[q_i - 2^{r\ell - 1}, q_i + 2^{r\ell -1})\cap\Z^d,
    \end{equation}
    with $q_i\in\{0, 1, \dots, 2^{r\ell} - 1\}$, for $1\leq i \leq d$. It is easy to see that \begin{equation*}
        [q_i - 2^{r\ell - 1}, q_i + 2^{r\ell -1})\subset [-2^{r\ell-1}, 2^{r\ell-1})\cup [2^{r\ell-1},2^{r\ell} + 2^{r\ell -1}). 
    \end{equation*} 
    Taking the products for all $1\leq i\leq d$, we get $2^d$ $r\ell$-cubes that covers \eqref{Cube.Q}. 
    We conclude that, to cover a connected component $G\in \mathscr{G}_{rf(\ell)}$, we need at most $2^d\ceil{|v(G)|/2^{r(1-\frac{1}{d})f(\ell)}}$ $rf(\ell)$-cubes, yielding us
    \begin{equation}\label{Eq: 3.18.Lucas}
        |\C_{r\ell}(\gamma)|\leq |\C_{r\ell}( B_{\C_{rf(\ell)}(\gamma)})| \leq \sum_{G\in \mathscr{G}_{rf(\ell)}} |\C_{r\ell}(v(G))| \leq \sum_{G\in \mathscr{G}_{rf(\ell)}} 2^d\left\lceil{\frac{|v(G)|}{2^{r(1-\frac{1}{d})f(\ell)}}}\right\rceil. 
    \end{equation}
    When every connected component of $G_{rf(\ell)}(\gamma)$ has more than $2^{r(1-\frac{1}{d})f(\ell)}$ vertices, we can bound 
    \begin{equation*}
        \frac{1}{2}\left\lceil{\frac{|v(G)|}{2^{r(1-\frac{1}{d})f(\ell)}}}\right\rceil \leq {\frac{|v(G)|}{2^{r(1-\frac{1}{d})f(\ell)}}}.
    \end{equation*}
    Together with inequalities \eqref{Eq: 3.16.Lucas}  and \eqref{Eq: 3.18.Lucas}, this yields
    \begin{equation}\label{Eq: Bound_C_rl_by_gamma_with_f(l)}
         |\C_{r\ell}(\gamma)| \leq \sum_{G\in \mathscr{G}_{rf(\ell)}} 2^{d+1} {\frac{|v(G)|}{2^{r(1-\frac{1}{d})f(\ell)}}}= \frac{2^{d+1}}{2^{r(1-\frac{1}{d})f(\ell)}}|\C_{rf(\ell)}(\gamma)|.
    \end{equation}

    Equation \eqref{Eq: Bound_C_l_by_C_f(l)} can be iterated as long as the radius is positive. Considering then the auxiliary quantity
    \begin{equation*}
        m(\ell)\coloneqq\max\{m : f^m(\ell)\geq 0\},  
    \end{equation*}
we have 
\begin{equation*}
    |\C_{r\ell}(\gamma)| \leq \frac{2^{(d+1)m(\ell)}}{2^{r(\delta - d)(1-\frac{1}{d})[\sum_{i=1}^{m(\ell)}f^i(\ell)]}}|\gamma|,
\end{equation*}
so we need upper and lower estimates for $m(\ell)$.
We claim that
\begin{equation}\label{Eq: lower.bound.on.m}
    m(\ell) \geq \begin{cases}
                        0, &\text{ if }n\leq \overline{b}\\ 
                        \left\lfloor\frac{\log_2(n) - \log_2(\overline{b})}{\log_2(a + (1-\frac{1}{d}))}\right\rfloor, & \text{ if }n > \overline{b}, 
                \end{cases}
\end{equation}
where $\overline{b} = (\overline{a}+1 + \log_{2^r}(2M))(\overline{a}-1)^{-1}$ and $\overline{a}\coloneqq a + (1-\frac{1}{d})$. Given $\ell > \overline{b}$, consider
\begin{equation*}
     \overline{f}(\ell) = \frac{\ell - 1 - \log_{2^r}(2M)}{a + (1 - \frac{1}{d})} - 1.
\end{equation*}
It is clear that $f(n)\geq \overline{f}(n)$ and both functions are increasing, therefore $f(\ell)^m\geq \overline{f}^m(\ell)$ for every $m\geq 0$.  As
\begin{equation*}
       \overline{f}^m(\ell) = \frac{\ell}{\overline{a}^m} - b^\prime\frac{\overline{a}^m - 1}{\overline{a}^{m-1}(\overline{a}-1)},
\end{equation*}
with $b^\prime = (\overline{a}+1 + \log_{2^r}(2M))\overline{a}^{-1}$, it is sufficient to have
\begin{equation*}
    \frac{n}{\overline{a}^m} -\frac{\overline{a} b^\prime}{(\overline{a}-1)}\geq 0.
\end{equation*}
We get the desired bound by applying the logarithm with base two in the equation above. Moreover, we can bound
\begin{align*}
    \sum_{i=1}^{m(\ell)}f^{i}(\ell) & \geq  \sum_{i=1}^{m(\ell)}\frac{\ell}{\overline{a}^m} - m(\ell)\frac{\overline{a}b^\prime}{\overline{a} - 1} = \frac{1}{\overline{a}}(\frac{1-\frac{1}{\overline{a}^{m(\ell)}}}{{1-\frac{1}{\overline{a}}}})\ell - m(\ell)\overline{b} \\
    %
    &\geq \frac{1}{\overline{a}-1}(1-\frac{1}{\overline{a}^{m(\ell)}}) - m(\ell)\overline{b} \geq \frac{1}{\overline{a}-1}(1-\frac{\overline{a}\overline{b}}{\ell}) - m(\ell)\overline{b}
\end{align*}

For the upper bound on $m(\ell)$, take $\Tilde{f}(\ell) \coloneqq \frac{\ell}{a + (1-\frac{1}{d})}$. As $f(\ell)\leq \Tilde{f}(\ell)$ and $\Tilde{f}$ is increasing,  $f^m(\ell)\leq \Tilde{f}^m(\ell)$ for every $m\geq 0$. Notice that, if $\Tilde{f}^m(\ell)\leq 1$, $f^{m+1}(\ell)<0$ and therefore $m+1>m(\ell)$. As $\Tilde{f}^m(\ell)\leq 1$ if and only if $\ell \leq [a + (1-\frac{1}{d})]^m$, we have $\frac{\log_2(\ell)}{a + (1-\frac{1}{d})} + 1 > m(\ell)$.
Applying this bound on \eqref{Eq: Bound_C_rl_by_gamma_with_f(l)} we conclude that 
\begin{equation}\label{Eq: bound_on_C_ell_covering_l_geq_ab}
     |\C_{r\ell}(\gamma)| \leq \frac{2^{d+1 + \overline{a}\overline{b}}\ell^{\frac{d+1}{a + (1-\frac{1}{d})}}}{2^{r(1-\frac{1}{d})\frac{1}{\overline{a}-1}\ell}}|\gamma|,
\end{equation}
for $\ell>\overline{a}\overline{b}$. When $\ell\leq \overline{a}\overline{b}$, we can take $\overline{b}_4\coloneqq \min\{{ j^{\frac{d+1}{a + (1-\frac{1}{d})}}{2^{-r(1-\frac{1}{d})\frac{1}{\overline{a}-1}j}}} : 0\leq j \leq \overline{a}\overline{b}\}$ and then 
\begin{equation*}
       |\C_{r\ell}(\gamma)| \leq |\gamma|\leq \frac{1}{\overline{b}_4}\frac{\ell^{\frac{d+1}{a + (1-\frac{1}{d})}}}{2^{r(1-\frac{1}{d})\frac{1}{\overline{a}-1}\ell}}|\gamma|.
\end{equation*}
This, together with equation \eqref{Eq: bound_on_C_ell_covering_l_geq_ab}, concludes the proposition with $b_4^\prime \coloneqq \max\{ 2^{d+1 + \overline{a}\overline{b}}, \overline{b}_4^{-1}\}$.
\end{proof}


For any non-negative $V, M, a, r$, define
\begin{equation*}
    \mathcal{F}_{V}^\ell\coloneqq \{ \C_{r\ell} : V_r^\ell(B_{\C_{r\ell}}) = V, B_{\C_{r\ell}}\subset C_{2rn_r(B_{\C_{r\ell}})}(0)\}.
\end{equation*}

Using equation \eqref{Eq: Bound.on.N}, in the same steps as \cite[Proposition 3.11]{Affonso.2021}, we can show that the number of collections in $\mathcal{F}_V$ is exponentially bounded by $V+\ell$.

\begin{proposition}\label{Prop. Bound.on.Fv}
    There exists $b_5\coloneqq b_5(d,r)$ such that
    \begin{equation}\label{Eq: Bound.on.F_V}
        |\mathcal{F}^\ell_V| \leq e^{b_5(V+\ell)}.
    \end{equation}
\end{proposition}

\begin{proof}
We start by splitting $\mathcal{F}^\ell_V$ into $\mathcal{F}^\ell_{V,m} \coloneqq \{ \C_{r\ell}\in \mathcal{F}^\ell_V : n_r(B_{\C_{r\ell}})=m \}$. Since $\ell\leq n_r(\C_{r\ell}) \leq V_r^\ell(B_{\C_{r\ell}}) +\ell$, we get

\begin{equation}
    |\mathcal{F}^\ell_V| \leq \sum_{m=\ell}^{V+\ell} |\mathcal{F}^\ell_{V,m}|. 
\end{equation}
Taking $b^{\prime \prime} =  2$ and denoting $(V_{rn})_{n=\ell}^{m}$ an arbitrary family of natural numbers satisfying 
\begin{equation}\label{Eq: sum.of.V_rn}
    \sum_{n=\ell}^{m} V_{rn}\leq  V,
\end{equation}
and $V_{rn}\leq V_{r(n-1)}$, we can bound
\begin{align}\label{Eq: bound.FVL}
    |\mathcal{F}^\ell_{V,m}| \leq \sum_{(V_{rn})_{n=\ell}^{m}} |\{\C_{r\ell} : B_{\C_{r\ell}} \subset  C_{b^{\prime \prime}rm}(0), |\C_{rn}(B_{\C_{r\ell}})| = V_{rn}, \text{ for every } \ell \leq n\leq m, n_r(B_{\C_{r\ell}}) = m\}|.
\end{align}
By our choice of $n_r$, every collection $\C_{r\ell}$ satisfying $n_r(B_{\C_{r\ell}})=m$ can be covered by a cube centered in $\Z^d$ and side length $2^{rm}$. Every such cube can be covered by at most $2^d$ $rm$-cubes. Indeed, it is enough to consider the simpler case when the cube is of the form
    \begin{equation}\label{Cube.Q_2}
        \prod_{i=1}^d[q_i - 2^{rm - 1}, q_i + 2^{rm -1})\cap\Z^d,
    \end{equation}
    with $q_i\in\{0, 1, \dots, 2^{rm} - 1\}$, for $1\leq i \leq d$. It is easy to see that \begin{equation*}
        [q_i - 2^{rm - 1}, q_i + 2^{rm -1})\subset [-2^{rm-1}, 2^{rm-1})\cup [2^{rm-1},2^{rm} + 2^{rm -1}). 
    \end{equation*} 
    Taking the products for all $1\leq i\leq d$, we get $2^d$ $rm$-cubes that covers \eqref{Cube.Q_2}. Let $\mathcal{C}^m$ be the set of collections $\C_{rm}$ such that $|\C_{rm}|=V_{rm}$, $B_{\C_{rm}}\subset C_{b^{\prime\prime}rm}(0)$  and there exists a cube $C$ centered in $\Z^d$ with side length $2^{rm}$ such that, for all $C_{rm}\in \C_{rm}$, $C_{rm}\cap C\neq \emptyset$. For every collection $\C_{r\ell}$ in the set on the RHS of \eqref{Eq: bound.FVL}, $\C_{rm}(B_{\C_{r\ell}}) \in \mathcal{C}^m$. We have that $B_{\C_{rm}(B_{\C_{r\ell}})}\subset C_{b^{\prime \prime}rm}(0)$ since given two cubes $C_{a}, C_{b}$, with $a\leq b$, either $C_{a}\subset C_{b}$ or $C_{a}\subset C_{b}^c$.   Moreover, we can bound 
\begin{align*}
    |\mathcal{C}^m| &\leq |\{C \text{ cube with side length }2^{rm} \text{ centered in }\Z^d\cap C_{rm} \text{ with }C_{rm}\subset C_{b^{\prime\prime}rm}(0) \}|\binom{2^d}{V_{rm}}\\
    %
    &\leq 2^{rdm}|\{C_{rm} : C_{rm} \subset C_{b^{\prime\prime}rm}(0)\}|\binom{2^d}{V_{rm}} \leq 2^{rdm}2^{d(b^{\prime \prime}-1)rm}(e2^d)^{V_m}
\end{align*}
We can count the RHS of equation \eqref{Eq: bound.FVL} by counting the number of families $(\C_{rn})_{n=\ell}^m$ such that $\C_{rn}\preceq \C_{r(n+1)}$, for $n<m$, and $\C_{rm}\in \mathcal{C}^m$, yielding us 
\begin{align*}
    |\mathcal{F}^\ell_{V,m}| 
    &\leq \sum_{(V_{rn})_{n=\ell}^{m-1}} |\{ (\C_{rn})_{n=\ell}^{m} : |\C_{rn}|=V_{rn}, \C_{rn}\preceq \C_{r(n+1)}, \C_{rm}\in \mathcal{C}^m\}| \\
    %
    & \leq  \sum_{(V_{rn})_{n=\ell}^{m-1}} \sum_{\C_{rm}\in \mathcal{C}^m}\sum_{\substack{\C_{r(m-1)} \\ |\C_{r(m-1)}|=V_{r(m-1)}\\ \C_{r(m-1)\preceq \C_{rm}}}} \cdots \sum_{\substack{\C_{r(\ell+1)} \\ |\C_{r(\ell+1)}|=V_{r(\ell+1)}\\ \C_{r(\ell+1)\preceq \C_{r(\ell+2)}}}} N(\C_{r(\ell+1)}, r\ell, V_{r\ell}).           
\end{align*}
Iterating equation \eqref{Eq: Bound.on.N} we get that
\begin{align*}
    |\mathcal{F}^\ell_{V,m}| &\leq \sum_{(V_{rn})_{n=\ell}^{m}}\sum_{\C_{rm}\in \mathcal{C}^m}\prod_{n=\ell}^{m-1}\left( \frac{2^{rd}e V_{r(n+1)}}{V_{rn}}\right)^{V_{rn}}\\
    %
    &\leq  \sum_{(V_{rn})_{n=\ell}^{m-1}}|\mathcal{C}^m|\prod_{n=\ell}^{m-1}e^{(rd\ln(2) +1)V_{rn}} \leq 2^{db^{\prime \prime}rm}\sum_{(V_{rn})_{n=\ell}^{m}}e^{(rd\ln(2)+1)V}.
\end{align*}



As $\sum_{m=\ell}^{V+\ell}2^{db^{\prime \prime}rm}\leq 2^{(db^{\prime \prime}r)(\ell+V) + 1}$ and the number of solutions of \eqref{Eq: sum.of.V_rn} is bounded by $2^V$, we conclude that 
\begin{equation*}
     |\mathcal{F}^\ell_{V}| \leq \sum_{m=\ell}^{V+\ell}|\mathcal{F}_{V,m}^\ell| \leq  2^{(db^{\prime \prime}r)(\ell+V) +1}2^Ve^{(rd\ln(2) + 1)V},
\end{equation*}
therefore equation \eqref{Eq: Bound.on.F_V} holds for $b_5\coloneqq [db^{\prime \prime}r + rd +2]\ln(2) + 1$.
\end{proof}
    
\begin{lemma}\label{Prop: Counting_spamming_trees}
    Given $\ell>0$, consider the graph $G=V(\C_{r\ell}(\Z^d), E)$, with two vertices $C,C^\prime$ being connected if and only if $d(C,C^\prime)\leq M2^{ra\ell}$. There exists a constant $b_5^\prime \coloneqq b_5^\prime(d,\alpha)$ such that
    \begin{equation}
        |\{{\C_{r\ell}}: C_{r\ell}(0)\in \C_{r\ell}, \ \C_{r\ell} \emph{ is connected }, |\C_{r\ell}|=N\}| \leq e^{b_5^\prime\ell N}.
    \end{equation}
\end{lemma}

\begin{proof}
    To count $|\{{\C_{r\ell}}: C_{r\ell}(0)\in \C_{r\ell}, \ \C_{r\ell} \emph{ is connected }, |\C_{r\ell}|=N\}|$, it is enough to count the number spanning trees containing $C_{r\ell}(0)$ with $N$ vertices. Let $\mathcal{T}_0$ be the set of all such trees Fixed $T\in\mathcal{T}_0$, for each $C_{r\ell}\in v(T)$, let $\d_T(C_{r\ell})$ be the degree of $C_{r\ell}$. As $T$ is a tree, $\sum_{C_{r\ell}\in v(T)}\d_T(C_{r\ell}) = 2(N-1)$. Moreover, as there are at most $2^{rd(a\ell + \log_{2^r}M - \ell)}$ $r\ell$-cubes inside a $r(a\ell + \log_{2^r}M)$-cube, each cube $C_{r\ell}\in T$ has at most $2^{rd(a +\log_{2^r}M -1)\ell}$ neighbours. Let $(d_i)_{i=1}^N$ denote a general solution to 
    \begin{equation}\label{Eq: sum_d_i}
        \sum_{i=1}^{N}d_i = 2(N-1),
    \end{equation}
    with $d_i\leq 2^{rd(a +\log_{2^r}M -1)\ell}$ for all $i=1,\dots, N$. Then 
    \begin{align*}
        |\{{\C_{r\ell}}: C_{r\ell}(0)\in \C_{r\ell}, \ \C_{r\ell} \emph{ is connected }, |\C_{r\ell}|=N\}| \leq \sum_{(d_i)_{i=1}^N} |\{T \in\mathcal{T}_0: d_T(C^i) = d_i\}|.
    \end{align*}
    In the set above, $\{C^1, C^2, \dots, C^N\}$ is any ordering of $v(T)$ with $C^1 = C_{r\ell}(0)$. Therefore, 
    \begin{equation*}
        |\{T \in\mathcal{T}_0: d_T(C^i) = d_i\}| \leq \prod_{i=1}^N\binom{2^{rd(a +\log_{2^r}M -1)\ell}}{d_i}\leq (e2^{rd(a +\log_{2^r}M -1)\ell})^{N}. 
    \end{equation*}
    As the number of solutions to \eqref{Eq: sum_d_i} is bounded by $2^N$, we conclude that
    \begin{align*}
        \{{\C_{r\ell}}: C_{r\ell}(0)\in \C_{r\ell}, \ \C_{r\ell} \emph{ is connected }, |\C_{r\ell}|=N\}| \leq  2^Ne^N2^{rd(a +\log_{2^r}M -1)\ell N} \leq e^{b_5^\prime \ell N}, 
    \end{align*}
    with $b_5^\prime \coloneqq \ln{2} + 1 + {rd(a +\log_{2^r}M -1)\ln{2}}$.
\end{proof}

\begin{proposition}\label{Prop: Bound_on_rl_coverings}
    Let $n,j\geq 0$, $\Lambda\Subset\Z^d$. There exists a constant $b_6\coloneqq b_6(a,d)>0$ such that if $0\leq \ell\leq j$,
    \begin{equation*}
        |\C_{r\ell}(\mathcal{C}_0(n,j))|\leq \exp{\left\{ b_6 \left[\frac{(\ell\wedge 1)^{\kappa}n}{2^{ra^\prime\ell}} + \ell \right]   \right\}},
    \end{equation*}
    where $\kappa \coloneqq \kappa(\alpha, d) = 1 + \frac{2}{\log_2(a)} + \frac{d+1}{a + (1-\frac{1}{d})}$. Moreover, if $\ell >j$, 
    \begin{equation*}
        |\C_{r\ell}(\mathcal{C}_0(n,j))|\leq \exp{\{b_6\frac{|\gamma|}{\ell^{\frac{r-d-1}{\log_2(a)} - 1}}\}}.
    \end{equation*}
\end{proposition}
\begin{proof}
    For $1\leq \ell\leq j$, Proposition \ref{Prop. Bound.on.V_r^l(gamma)} together with Proposition \ref{Prop. Bound.on.C_rl(gamma)} yields, 
    \begin{equation}\label{Eq: Bound_partial_volume_l_between_1_and_j}
        V_r^\ell(\gamma) = V_r^\ell(B_{\C_{r\ell}(\gamma)}) \leq b_3b_4\frac{\ell^{\kappa}|\gamma|}{2^{ra^\prime\ell}}.
    \end{equation}
    Therefore, 
    \begin{equation*}
        \{\C_{r\ell} :\C_{r\ell}=\C_{r\ell}(\gamma) \ \textrm{for some }\gamma\in\mathcal{C}_0(n,j)\}\subset \{ \C_{r\ell} : V_r^\ell(B_{\C_{r\ell}}) \leq b_3b_4\frac{\ell^{\kappa}|\gamma|}{2^{ra^\prime\ell}} \ , B_{\C_{r\ell}}\subset C_{2rn_r(B_{\C_{r\ell}})}(0)\}.
    \end{equation*}
    Proposition \ref{Prop. Bound.on.Fv} yields 
    \begin{multline*}
        |\{ \C_{r\ell} : V_r^\ell(B_{\C_{r\ell}}) \leq b_3b_4\frac{\ell^{\kappa}|\gamma|}{2^{ra^\prime\ell}}, B_{\C_{r\ell}}\subset C_{2rn_r(B_{\C_{r\ell}})}(0)\}| \leq \sum_{V=1}^{\ceil{b_3b_4\frac{\ell^{\kappa}|\gamma|}{2^{ra^\prime\ell}}}}|\mathcal{F}^\ell_V|  \leq \exp{\left\{ b_5b_4b_3\frac{\ell^{\kappa}n}{2^{ra^\prime\ell}} + b_5\ell  \right\}}.
    \end{multline*}
    When $\ell=0$, the same argument holds replacing the RHS of \eqref{Eq: Bound_partial_volume_l_between_1_and_j} by $b_3 b_4|\gamma|$. When $\ell>j$, for any $\gamma\in \mathcal{C}_0(n,j)$, $|\G_{r\ell}(\gamma)|=1$. Moreover, by Proposition \ref{Prop. Bound.on.C_rl(gamma)_Lucas},  
    \begin{equation*}
        |\C_{r\ell}(\gamma)|\leq b_4^\prime\frac{|\gamma|}{\ell^{\frac{r-d-1}{\log_2(a)}}}.
    \end{equation*}
    As $0\in V(\gamma)$, we now there is a cube $C_{r\ell}\in\C_{r\ell}(\gamma)$ that intersects the axis $e_1$ and $d(C_{r\ell}, 0)<|\C_{r\ell}(\gamma)|2^{r\ell}$. Therefore, there exists $|\C_{r\ell}(\gamma)|\leq b_4^\prime\frac{|\gamma|}{\ell^{\frac{r-d-1}{\log_2(a)}}}$ possible positions for $C_{r\ell}$. Using Lemma \ref{Prop: Counting_spamming_trees} we conclude that 
    \begin{equation}
        |\C_{r\ell}(\mathcal{C}_0(n,j))| \leq b_4^\prime\frac{|\gamma|}{\ell^{\frac{r-d-1}{\log_2(a)}}}\exp{\{{b_5^\prime b_4^\prime\ell\frac{|\gamma|}{\ell^{\frac{r-d-1}{\log_2(a)}}}}\}} \leq \exp{\{{2b_5^\prime b_4^\prime\ell\frac{|\gamma|}{\ell^{\frac{r-d-1}{\log_2(a)}}}}\}} 
    \end{equation} 
    what concludes the proof for $b_6\coloneqq \max{\{b_5b_4b_3, 2b_5^\prime b_4^\prime\}}$.
\end{proof}
A consequence of this Proposition is that we get an exponential bound on the number of contours with a fixed size.
\begin{corollary}\label{Cor: Bound_on_C_0_n}
	Let $n\geq 1$, $d\ge 2$, and $\Lambda\Subset \mathbb{Z}^d$. There exists $c_1\coloneqq c_1(d,M,r)>0$ such that
	\begin{equation}\label{Eq: exp.bound.contours}
	|\mathcal{C}_0(n)| \leq e^{c_1 n}.	    
	\end{equation}
\end{corollary}
\begin{proof} For any $j\geq 1$, Proposition  \ref{Prop: Bound_on_rl_coverings} applied to $\ell=0$ yields
     $|\mathcal{C}_0(n,j)|\leq e^{ b_6n}$. Remember that, in the proof of Proposition \eqref{Prop: Bound.bad.event.1} we showed that  $j\leq \frac{d}{d^2-1}\log_{2^r}n + 1$. So we can bound 
     \begin{align*}
        |\mathcal{C}_0(n)| \leq \sum_{j=1}^{\frac{d}{d^2-1}\log_{2^r}n + 1}|\mathcal{C}_0(n,j)| \leq e^{c_1n}
    \end{align*}
     with $c_1\coloneqq 2 b_6 + \ln{(\frac{d}{d^2-1} + 1)}$.
\end{proof}

\begin{proposition}\label{Prop: Bound_on_boundary_of_admissible_sets}
    Let $n,j\geq 0$, $\Lambda\Subset\Z^d$ and $\ell\geq 0$. There exists a constant $c_4\coloneqq c_4(\alpha, d)$ such that,
    \begin{equation}\label{Eq: Bound_on_boundary_of_admissible_sets}
        |B_\ell(\mathcal{C}_0(n,j))|\leq \exp{\left\{c_4 \ell^{\kappa}\left[\frac{n}{2^{r\ell(d-1-\frac{2\log_2(a)}{r-d-1-\log_2(a)})}} + \frac{n}{2^{2r\ell}} +1 \right]   \right\}}.
    \end{equation}
\end{proposition}

\begin{proof}
    Remember that $|B_\ell(\mathcal{C}_0(n,j))|=|\partial\mathfrak{C}_\ell(\mathcal{C}_0(n,j))|$. Moreover, given $\{C_{r\ell},C_{r\ell}^\prime\}\in\partial\mathfrak{C}_\ell(\gamma)$, either $C_{r\ell}\in\fint\mathfrak{C}_{\ell}$ or $C_{r\ell}^\prime\in\fint\mathfrak{C}_{\ell}$. Using that $\sum_{k=1}^p\binom{p}{k} = 2^{p}$, we have
    \begin{equation}\label{Eq: Replacing_edge_by_inner_boundary}
       \begin{split} |\partial\mathfrak{C}_\ell(\mathcal{C}_0(n,j))| &\leq \sum_{\fint\C_{r\ell}\in \fint\mathscr{C}_{\ell}(\mathcal{C}_0(n,j))} |\{\partial\C_{r\ell}^\prime : \fint\C_{r\ell}^\prime = \fint\C_{r\ell}\}| \\
        %
        &\leq \sum_{\fint\C_{r\ell}\in \fint\mathscr{C}_{\ell}(\mathcal{C}_0(n,j))} \sum_{k=1}^{2d|\fint\C_{r\ell}|}\binom{2d|\fint\C_{r\ell}|}{k}  \\
        %
        &=\sum_{\fint\C_{r\ell}\in \fint\mathscr{C}_{\ell}(\mathcal{C}_0(n,j))} 2^{2d|\fint\C_{r\ell}|}\leq |\fint\mathfrak{C}_\ell(\mathcal{C}_0(n,j))|e^{\ln(2)2db_1\frac{n}{2^{r\ell(d-1)}}},
        \end{split}
    \end{equation}
     where in the last inequality we applied Proposition \ref{Proposition1}. For every $L\geq \ell$ and an arbitrary collection $\C_{rL}$, define $\overline{\C_{rL}} = \C_{rL}\cup \{C_{rL}^\prime : \exists C_{rL}\in\C_{rL} \text{ such that } C_{rL}^\prime \text{ shares a face with } C_{rL}\}$. 
     
     Given $C_{r\ell}\in \fint \mathfrak{C}_\ell(\ell)$, either $C_{r\ell}$ or one of its neighbouring cubes intersects $\gamma$. Hence, for any $L\geq \ell$, $\fint \mathfrak{C}_{\ell}(\gamma)\preceq \overline{\C_{rL}(\gamma)}$. Moreover, the number of $r\ell$-cubes inside a collection $\overline{\C_{rL}(\gamma)}$ of $rL$-cubes is bound by $|\overline{\C_{rL}(\gamma)}|2^{rd(L-\ell)} \leq 2d|\C_{rL}(\gamma)|2^{rd(L-\ell)}$. Using again Proposition \ref{Proposition1}, we can bound 
    \begin{equation}\label{Eq: Bound_on_internal_boundary}
    \begin{split}
           |\fint \mathfrak{C}_{r\ell}(\mathcal{C}_0(n,j))| &\leq  \sum_{\substack{\C_{rL} \in \C_{rL}(\mathcal{C}_0(n,j))}}\sum_{k=1}^{\frac{b_1n}{2^{r\ell(d-1)}}}\binom{2d|\C_{rL}|2^{rd(L-\ell)}}{k} \\
           %
           &\leq \sum_{\substack{\C_{rL} \in \C_{rL}(\mathcal{C}_0(n,j))}}\left(\frac{e2d|\C_{rL}|2^{rdL}}{{b_1n}{2^{r\ell}}}\right)^{\frac{b_1n}{2^{r\ell(d-1)}}},
           \end{split}
    \end{equation}
    where in the last equation we used that, for any $0<M\leq N$, $\sum_{p=1}^{M}\binom{N}{p}\leq \left(\frac{eN}{M}\right)^{M}$. 
    
    Now we need to choose $L(\ell)$ according to $\ell$. When $\ell > (\frac{r-d-1}{\log_2 (a)}-1)(\frac{\log_2 (j)}{2r} + 1)$, we take $L(\ell) = 2^{2r\left\lfloor{\frac{\log_2(a)\ell}{r-d-1 - \log_2 (a)}}\right\rfloor}$. Then, $L>j$ and  Proposition \ref{Prop. Bound.on.C_rl(gamma)_Lucas} yields $e2d|\C_{rL(\ell)}(\gamma)|2^{rdL(\ell)}\leq e2db^\prime_4n 2^{rd2^{r\frac{2\log_2(a)\ell}{r-d-1-\log_2(a)}}}$. Therefore, for any $\C_{rL} \in \C_{rL}(\mathcal{C}_0(n,j))$,
  
\begin{align*}
    \left(\frac{e2d|\C_{rL}|2^{rdL}}{{b_1n}{2^{r\ell}}}\right)^{b_1n2^{-r\ell(d-1)}} &\leq \exp{\left\{{c_4^\prime\frac{n}{2^{r\ell(d-1 - \frac{2\log_2(a)}{r-d-1 -\log_2(a)})}}}\right\}}
\end{align*}
with $c_4^\prime = \ln(2)r(d+\log_{2^r}(\frac{e2db^\prime_4}{b_1}))b_1$.
With our choice of $L(\ell)$, Proposition \ref{Prop: Bound_on_rl_coverings} yields

\begin{align*}
   |\C_{rL(\ell)}(\mathcal{C}_0(n,j))| &\leq \exp{\left\{ b_6  \frac{n}{2^{2r\ell}}{2^{2r\frac{r-d-1}{\log_2(a)}}} \right\}}.
\end{align*}

Applying both inequalities above back in \eqref{Eq: Bound_on_internal_boundary} we get 

\begin{equation}\label{Eq: Bound_on_inner_admissible_cubes_large_l}
    |\fint \mathfrak{C}_{r\ell}(\mathcal{C}_0(n,j))| \leq \exp{\left\{ \left(c_4^\prime +  b_6 {2^{2r\frac{r-d-1}{\log_2(a)}}}\right) \left(\frac{n}{2^{r\ell(d-1-\frac{2\log_2(a)}{r-d-1-\log_2(a)})}} + \frac{n}{2^{2r\ell}}\right) \right\}}.
\end{equation}

For $\ell \leq (\frac{r-d-1}{\log_2 (a)}-1)(\frac{\log_2 (j)}{2r} + 1)$ and $j>\frac{2}{a^\prime}\left[\frac{r-d-1}{\log_2(a)} -1\right]\left[\frac{\log_2(j)}{2r} + 1\right]$, we take $L(\ell) = \floor{\frac{2\ell}{a^\prime}}$. Then $L(\ell)< j$ and Proposition \ref{Prop. Bound.on.C_rl(gamma)} yields $|\C_{rL(\ell)}(\gamma)|2^{rdL(\ell)} \leq  b_4 n{\left(\frac{2\ell}{a^\prime}\right)^{\frac{d+1}{a + 1 - \frac{1}{d}}}}{2^{r(d-a^\prime)\frac{2}{a^\prime} \ell}}$. Therefore, for any $\C_{rL} \in \C_{rL}(\mathcal{C}_0(n,j))$,
\begin{align*}%\label{Eq: Bound_on_choices_add_cubes_small_l}
\left(\frac{e2d|\C_{rL}|2^{rdL}}{{b_1n}{2^{r\ell}}}\right)^{\frac{b_1n}{2^{r\ell(d-1)}}} &\leq \exp{\left\{c_4^{\prime\prime}\frac{\ell n}{2^{r\ell(d-1)}}\right\}},
\end{align*}

for $c_4^{\prime\prime}\coloneqq \ln(2)r(d - a^\prime + \log_{2^r}(\frac{e2db_4}{b_1} + \frac{d+1}{a+1-\frac{1}{d}}\log_{2^r}(\frac{2}{a^\prime})))\frac{2}{a^\prime}b_1$. With our choice of $L(\ell)$, Proposition \ref{Prop: Bound_on_rl_coverings} yields
\begin{align*}
    |\C_{rL(\ell)}(\mathcal{C}_0(n,j))| \leq \exp{\left\{b_6\left(\frac{2\ell}{a^\prime}\right)^{\kappa}2^{ra^\prime}\left(\frac{n}{2^{2r\ell}} + 1 \right)\right\}}.
\end{align*}

Applying both inequalities above back in \eqref{Eq: Bound_on_internal_boundary} we get 
\begin{equation}\label{Eq: Bound_on_inner_admissible_cubes_small_l}
    |\fint \mathfrak{C}_{r\ell}(\mathcal{C}_0(n,j))| \leq \exp{\left\{ \left(c_4^{\prime\prime} + b_62^{ra^\prime}\left(\frac{2}{a^\prime}\right)^{\kappa}\right)\ell^{\kappa}\left(\frac{n}{2^{r\ell(d-1)}} + \frac{n}{2^{2r\ell}} + 1 \right) \right\}}.
\end{equation}

We are left to consider the case when $j\leq\frac{2}{a^\prime}\left[\frac{r-d-1}{\log_2(a)} -1\right]\left[\frac{\log_2(j)}{2r} + 1\right]$. This happens only if $j\leq \overline{c}_4$, for a constant $\overline{c}_4 \coloneqq \overline{c}_4(\alpha,d)$. Then, $n\leq 2^{r(d+1)j}\leq 2^{r(d+1)\overline{c}_4}$ and therefore $|B_\ell(\mathcal{C}_0(n,j))|\leq e^{c_4^{\prime\prime\prime}}$ for a suitable constant $c_4^{\prime\prime\prime}\coloneqq c_4^{\prime\prime\prime}(\alpha,d)$.

Taking $c_4 \coloneqq c_4^{\prime\prime\prime} + c_4^{\prime\prime} + c_4^{\prime} +  b_6(2^{ra^\prime}\left(\frac{2}{a^\prime}\right)^{\kappa} + 2^{2r\frac{r-d-1}{\log_2(a)}}) +\ln(2)2db_1$, the proposition follows from \eqref{Eq: Replacing_edge_by_inner_boundary}, \eqref{Eq: Bound_on_inner_admissible_cubes_large_l} and \eqref{Eq: Bound_on_inner_admissible_cubes_small_l}.
\end{proof}


\begin{proof}[Proof of Proposition \ref{Prop: Bound.gamma_2}.]
 As $N(\mathcal{C}_0(n,j), \d_2, \epsilon)$ is decreasing in $\epsilon$, we can use Dudley's entropy bound to get
    \begin{align*}
        {\mathbb{E}\left[\sup_{\gamma\in\mathcal{C}_0(n,j)}{\Delta_{\I_-(\gamma)}(h)}\right]} &\leq \int_{0}^\infty \sqrt{\log N(\mathcal{C}_0(n,j), \d_2, \epsilon)}d\epsilon \leq \sum_{\ell=0}^\infty\sqrt{\log N(\mathcal{C}_0(n,j), \d_2, \ell)}\nonumber\\
        %
        &\leq 2\varepsilon b_3 n^{\frac{1}{2}}\sum_{\ell=1}^\infty (2^{\frac{r\ell}{2}} - 2^{\frac{r(\ell-1)}{2}})\sqrt{\log N(\mathcal{C}_0(n,j), \d_2,\varepsilon b_3 2^{\frac{r\ell}{2}}n^{\frac{1}{2}})}.
    \end{align*}
Since $\d_2(\gamma_1,\gamma_2)\leq 2\varepsilon\sqrt{|\I_-(\gamma_1)| + |\I_-(\gamma_2)|}\leq 2\sqrt{2}\varepsilon n^{\frac{1}{2} + \frac{1}{2(d-1)}}$ for any $\gamma_1,\gamma_2\in\mathcal{C}_0(n,j)$, when $\varepsilon b_3 2^{\frac{r\ell}{2}}n^{\frac{1}{2}}\geq 2\sqrt{2}\varepsilon n^{\frac{1}{2} + \frac{1}{2(d-1)}}$, only one ball covers all contours, hence all the terms in the sum above with $\ell\geq k(n)\coloneqq 2\ceil{\log_{2^r}(2\sqrt{2})+\frac{\log_{2^r}(n)}{2(d-1)}}$ are zero. As $N(\mathcal{C}_0(n,j), \d_2,\varepsilon b_3 2^{\frac{r\ell}{2}}n^{\frac{1}{2}})\leq |B_{\ell}(\mathcal{C}_0(n,j))|$, see Remark \ref{Rmk: Bounding_N_by_B_ell}, using Proposition \ref{Prop: Bound_on_boundary_of_admissible_sets} we get
\begin{align}\label{Eq: Prop_bound_Ec_1}
        {\mathbb{E}\left[\sup_{\gamma\in\mathcal{C}_0(n,j)}{\Delta_{\I(\gamma)}(h)}\right]} &\leq 2\varepsilon b_3\sqrt{c_4} n^{\frac{1}{2}}\sum_{\ell=1}^{k(n)}2^{\frac{r\ell}{2}}\sqrt{\frac{n\ell^\kappa}{2^{r\ell(d-1-\frac{2\log_2(a)}{r-d-1-\log_2(a)})}} + \frac{n\ell^\kappa}{2^{2r\ell}}+ \ell^\kappa }\nonumber\\
        %
        &\leq  2\varepsilon b_3\sqrt{c_4}\sum_{\ell=1}^{\infty}\left[\frac{\ell^\frac{\kappa}{2}}{2^{\frac{r\ell}{2}(d-2-\frac{2\log_2(a)}{r-d-1-\log_2(a)})}} + \frac{\ell^{\frac{\kappa}{2}}}{2^{r\ell}}\right]n + 2\varepsilon b_3\sqrt{c_4}n^{\frac{1}{2}}\sum_{\ell=1}^{k(n)}2^{\frac{r\ell}{2}}\ell^\frac{\kappa}{2}.
\end{align}

By our choice of $r$, $d-2-\frac{2\log_2(a)}{r-d-1-\log_2(a)}>0$ so the series above converges. Moreover, we can bound
\begin{align*}
    \sum_{\ell=1}^{k(n)}2^{\frac{r\ell}{2}}\ell^\frac{\kappa}{2} &\leq k(n)^{\frac{\kappa}{2}}\frac{2^{\frac{rk(n)}{2}}}{\sqrt{2^r}-1} \\
    %
    &\leq 2^\kappa(\log_{2^r}(2\sqrt{2})+\frac{\log_{2^r}(n)}{2(d-1)} + 1)^\kappa 2^{r(\log_{2^r}(2\sqrt{2})+\frac{\log_{2^r}(n)}{2(d-1)} +1)}\\
    %
    &\leq  2^{\kappa + r}2\sqrt{2}(\log_{2^r}(2\sqrt{2})+\frac{1}{2(d-1)} + 1)^\kappa \log_{2^r}(n)^\kappa n^{\frac{1}{2(d-1)}} \\
    %
    &\leq  2^{\kappa + r + \frac{3}{2}}(\log_{2^r}(2\sqrt{2})+\frac{1}{2(d-1)} + 1)^\kappa n^{\frac{1}{2}}.
\end{align*}
Applying this back in \eqref{Eq: Prop_bound_Ec_1} we conclude that 
\begin{equation*}
       {\mathbb{E}\left[\sup_{\gamma\in\mathcal{C}_0(n,j)}{\Delta_{\I_-(\gamma)}(h)}\right]} \leq \varepsilon L_1^\prime n,
\end{equation*}
with $L_1^\prime\coloneqq   2\varepsilon b_3\sqrt{c_4}\left[ 2^{\kappa + r + \frac{3}{2}}(\log_{2^r}(2\sqrt{2})+\frac{1}{2(d-1)} + 1)^\kappa + \sum\limits_{\ell= 1}^\infty\left(\frac{\ell^\frac{\kappa}{2}}{2^{\frac{r\ell}{2}(d-2-\frac{2\log_2(a)}{r-d-1-\log_2(a)})}} + \frac{\ell^{\frac{\kappa}{2}}}{2^{r\ell}}\right)\right]$. Inequality \eqref{Eq: gamma_2_bounded_by_Dudley_integral} yields the desired result with $L_1\coloneqq L^\prime L_1^\prime$.
\end{proof}
