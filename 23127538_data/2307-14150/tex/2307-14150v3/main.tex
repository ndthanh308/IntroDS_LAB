\documentclass[12pt,a4paper]{article}
\usepackage[english]{babel}
\usepackage{float}
\usepackage{amsmath, amssymb, amsthm, epsfig, color, mathtools, bbm, dsfont, mathrsfs, bm}
\usepackage[latin1]{inputenc}
\usepackage[T1]{fontenc}
\usepackage{indentfirst}
\usepackage[shortlabels]{enumitem}
\usepackage[title]{appendix}
\usepackage{lmodern}       
\usepackage{comment}  
   % alternative Computer Modern-fonts for computer 
\usepackage{textcomp, upgreek}
\usepackage{setspace}
\usepackage{soul}
\usepackage{accents}
\usepackage{hyperref} %url nas referencias 
\usepackage{geometry}
\usepackage{xcolor}

\usepackage{xr}    %para usar labels de outros documentos 
\usepackage{graphicx} %para inserir imagens
\usepackage[export]{adjustbox} %posiciona imagens
\usepackage{subcaption} %posiciona imagens
\usepackage{mathtools} %usa multiline dentro do align
\usepackage{siunitx}
\usepackage{bm} %for bold in math mode 


\usepackage{addfont}
\addfont{OT1}{rsfs10}{\rsfs}

%----------------

\usepackage{tikz, tkz-euclide}
\usetikzlibrary{patterns}
\usetikzlibrary{arrows.meta}
\usetikzlibrary{calc}
\usetikzlibrary{decorations.markings}
\usepackage{fancybox}

%-------------------

\usepackage{graphicx}
\graphicspath{ {./images/} }
\newcommand{\Z}{\mathbb{Z}}
\newcommand{\bbr}{\mathbb{R}}
\newcommand{\bbn}{\mathbb{N}}
\newcommand{\caln}{\mathcal{N}}
\newcommand{\cals}{\mathcal{S}}
\newcommand{\lab}{\mathrm{lab}}     
\newcommand{\supp}{\text{supp}\;}
\newcommand{\din}{\partial_{\mathrm{in}}}
\newcommand{\dout}{\partial_{\mathrm{out}}}
\newcommand{\diam}{\mathrm{diam}}
\newcommand{\dis}{\mathrm{dist}}
\newcommand{\bfr}{\mathbf{r}}
\newcommand{\bfq}{\mathbf{q}}
\newcommand{\I}{\mathrm{I}}
\newcommand{\Sp}{\mathrm{sp}}
\newcommand{\f}{\mathrm{f}}
\newcommand{\R}{\mathcal{R}}
\newcommand{\fint}{\partial_{\mathrm{in}}}
\newcommand{\fext}{\partial_{\mathrm{ex}}}
\newcommand{\C}{\mathscr{C}}
\newcommand{\G}{\mathscr{G}}
\newcommand{\calP}{\mathcal{P}}

\newcommand{\ssum}[1]{
\sum_{\mathclap{\substack{#1}}}
}
\newcommand{\subscript}[2]{$#1 _ #2$}

\def\supp{\mathop{\textrm{\rm supp}}\nolimits}            %Support
\def\d{\mathop{\textrm{\rm d}}\nolimits}                  %integral d
\def\Int{\mathop{\textrm{\rm Int}}\nolimits}                %Interior
\def\Ext{\mathop{\textrm{\rm Ext}}\nolimits}                  %Exterior
\def\sign{\mathop{\textrm{\rm sign}}\nolimits}                  %sign function
\def\sf{\mathop{\textrm{\rm sf}}\nolimits}            %sf


\newcommand{\be}{\begin{equation}}
\newcommand{\ee}{\end{equation}}


%----------------
\numberwithin{equation}{section}


\addtolength{\hoffset}{-1.5cm} \addtolength{\textwidth}{2cm}
\addtolength{\voffset}{-1.5cm} \addtolength{\textheight}{2cm}
  \newcounter{dummy} \numberwithin{dummy}{section}
  \theoremstyle{plain}
  \newtheorem*{theorem*}        {Theorem}
	\newtheorem*{conjecture*}   {Conjecture}
  \newtheorem{theorem}[dummy]          {Theorem}
  \newtheorem{lemma}[dummy]              {Lemma}
  \newtheorem*{lemma*}          {Lemma}
    \newtheorem{claim}[dummy]         {Claim}
  \newtheorem{corollary}[dummy]           {Corollary}
  \newtheorem{proposition}[dummy]       {Proposition}
  \newtheorem{question}[dummy]           {Question} 
  \newtheorem{remark}[dummy]           {Remark}
  \newtheorem{notation}[dummy]           {Notation}
  \theoremstyle{remark}
    %\newtheorem{notation}[dummy]           {Notation}
  \theoremstyle{definition}
   \newtheorem{definition}[dummy]          {Definition}
%------------------- long left-right arrow with the word above
\makeatletter

\newcommand\longleftrightarrowfill@{%
  \arrowfill@\leftarrow\relbar\rightarrow}
\makeatother

%% Comman to edit colors
%
%
\definecolor{Red}{cmyk}{0,1,1,0}
\def\red{\color{Red}}
\definecolor{Blue}{cmyk}{1,1,0,0}
\def\blue{\color{Blue}}
\definecolor{DarkBlue}{rgb}{0.1,0.1,0.5}
\definecolor{Red}{rgb}{0.9,0.0,0.1}
\definecolor{DarkGreen}{rgb}{0.10,0.50,0.10}
\definecolor{DarkRed}{rgb}{0.50,0.10,0.10}
\definecolor{bleu}{RGB}{0,140,189}%
\newcommand\bfblue[1]{\textcolor{blue}{\textbf{#1}}}
\newcommand\bfred[1]{\textcolor{red}{\textbf{#1}}}

%-----------------Margem e Formatação---------------

\newgeometry{vmargin={15mm}, hmargin={12mm,17mm}}

%--------------------------------------------------


\newcommand{\eqdef}{\overset{\mathrm{def}}{=\joinrel=}}

\definecolor{vermelho}{RGB}{208,2,27}    %cores dos graficos
\definecolor{verde}{RGB}{126,211,33} %cores dos graficos

\DeclareMathOperator{\arctanh}{arctanh}
\DeclareMathOperator{\h}{\bm{h}}
\DeclareMathOperator{\dist}{\mathrm{d}}
\DeclareMathOperator{\s}{\mathrm{sp}}

\DeclarePairedDelimiter\ceil{\lceil}{\rceil} %\ceil*{}
\DeclarePairedDelimiter\floor{\lfloor}{\rfloor} %\floor*{}




\begin{document}


\begin{center}
{\LARGE Phase Transitions in Long-Range Random Field Ising Models in Higher Dimensions}
\vskip.5cm
Lucas Affonso$^{1}$, Rodrigo Bissacot$^{1,2}$, Jo{\~a}o Maia$^{1}$
\vskip.3cm
\begin{footnotesize}
$^{1}$Institute of Mathematics and Statistics (IME-USP), University of S\~{a}o Paulo, Brazil\\
$^{2}$ Faculty of Mathematics and Computer Science, Nicolaus Copernicus University, Poland\\  
\end{footnotesize}
\vskip.1cm
\begin{scriptsize}
emails: lucas.affonso.pereira@gmail.com, rodrigo.bissacot@gmail.com, maia.joaovt@gmail.com
\end{scriptsize}

\end{center}

\begin{abstract}
We extend the recent argument by Ding and Zhuang from nearest-neighbor to long-range interactions and prove the phase transition in the class of ferromagnetic random field Ising mo-dels. Our proof combines a generalization of Fr\"ohlich-Spencer contours to the multidimensional setting, proposed by two of us, with the coarse-graining procedure introduced by Fisher, Fr\"ohlich and Spencer. The result shows that the Ding-Zhuang strategy is also useful for interactions $J_{xy}=|x-y|^{- \alpha}$ when $\alpha > d$ in dimension $d\geq 3$ if we have a suitable system of contours. We can consider i.i.d. random fields with Gaussian or Bernoulli distributions. Our main result is an alternative proof that does not use the Renormalization Group Method (RGM), since Bricmont and Kupiainen claimed that the RGM should also work on this generality.
\end{abstract}

\section{Introduction}The problem of the presence or absence of phase transition is central in statistical mechanics. To prove the existence of phase transition, the standard idea is to define a notion of contour and use \textit{Peierls' argument} \cite{Peierls.1936}. In the usual Ising model \cite{Ising_25}, particles of the system interact only with their nearest-neighbors. On ferromagnetic long-range Ising models \cite{Anderson_Yuval_69}, there is interaction between each pair of spins in the lattice. The Hamiltonian of the model is given formally by
\begin{equation*}
    H(\sigma) = - \sum_{x,y\in \Z^d}J_{xy}\sigma_x\sigma_y,
\end{equation*}
where $J_{xy}=J|x-y|^{-\alpha}$, $J>0$, $\alpha > d$. It is well-known that the Peierls' argument in dimension 2 implies phase transition for Ising models with nearest-neighbors or long-range interactions when $d\geq 2$, using correlation inequalities. For the unidimensional lattice, it was known that short-range models do not present phase transition. In the long-range case, a different behavior was expected depending on the exponent $\alpha$ (see \cite{Kac_Thompson_69}), but the problem was challenging since contours were first created as multidimensional objects.

In dimension $d=1$, phase transition was proved first in 1969 by Dyson \cite{Dyson.69}, for $\alpha \in (1,2)$, by proving phase transition in an auxiliary model and then using correlation inequalities. In 1982, Fr{\"o}hlich and Spencer \cite{Frohlich.Spencer.82} introduced a notion of one-dimensional contours and then applied the Peierls' argument to show phase transition for the critical value $\alpha = 2$. These contours were inspired by the multiscale techniques previously introduced to study the Berezinskii-Kosterlitz-Thouless transition in two-dimensional continuous spin systems \cite{FS81}. Later, Cassandro, Ferrari, Merola and Presutti  \cite{Cassandro.05} extended the contour argument previously available for $\alpha=2$ to exponents $\alpha\in (3-\frac{\ln 3}{\ln 2}, 2)$, with the additional restriction that the nearest-neighbor interaction is strong, i.e.,  ${J(1)\gg 1}$; this restriction was removed for a subclass of interactions in \cite{Bissacot.Endo.18}. Further results were obtained using contour arguments, such as the decay of correlations, cluster expansions, phase transition with random interactions, etc; some references with these results are \cite{ Cassandro.Merola.Picco.17, Cassandro.Merola.Picco.Rozikov.14, Imbrie.82, Imbrie.Newman.88, Johansson.91}. 

In the multidimensional setting ($d\geq 2$), Ginibre, Grossmann, and Ruelle, in \cite{Ginibre.Grossmann.Ruelle.66}, proved the phase transition for $\alpha > d+1$, using an enhanced version of Peierls' argument and the usual contours. Park proposed a different notion of contour for long-range systems in \cite{Park.88.I, Park.88.II}, extending the Pirogov-Sinai theory available for short-range interactions assuming $\alpha > 3d+1$, although he can also consider Potts models with his methods. Some results in the literature suggest that truly long-range effects appear only when $d < \alpha \leq d+1$, see for instance, \cite{Biskup_Chayes_Kivelson_07}. Recently, Affonso, Bissacot, Endo and Handa \cite{Affonso.2021}, inspired by the ideas from Fr{\"o}hlich and Spencer in \cite{FS81, Frohlich.Spencer.82}, introduced a version of multiscale multidimensional contour and proved phase transition by a contour argument in the whole region $\alpha > d$. They can consider long-range Ising models with deterministic decaying fields, first introduced in the context of nearest-neighbor interactions in \cite{Bissacot_Cioletti_10}. For these models, the lack of analyticity of the free energy does not imply phase transition since these models have the same free energy as the models with zero field. It is expected that fields decaying slowly imply uniqueness. In this setting, a contour argument is useful for proofs of phase transitions as well for uniqueness, some papers with models with deterministic decaying fields are \cite{Aoun_Ott_Velenik_23, Bissacot_Cass_Cio_Pres_15, Bissacot.Endo.18, Cioletti_Vila_2016}.

The Random Field Ising model (RFIM) \cite{Imry.Ma.75} is the nearest-neighbor Ising model with an additional external field acting on each site $(h_x)_{x\in\Z^d}$ that is a family of i.i.d. Gaussian random variable with mean 0 and variance 1. Formally, the Hamiltonian of the model is given by
\begin{equation*}
    H(\sigma) = - \sum_{\substack{x,y\in \Z^d \\|x-y|=1}}J\sigma_x\sigma_y  - \varepsilon\sum_{x\in\Z^d}h_x\sigma_x,
\end{equation*}
where $J>0$, $\varepsilon>0$, $\alpha > d$ and $d \geq 1$. A detailed account of the history of the phase transition problem for this model, as well as detailed proofs, was given in \cite{Bovier.06}. Here we present a brief overview.

During the 1980s, the question of the specific dimension where phase transition for the RFIM should happen attracted much attention and was a topic of heated debate. Two convincing arguments were dividing the physics community. One of them, due to Imry and Ma \cite{Imry.Ma.75}, was a non-rigorous application of the Peierls' argument together with the use of the isoperimetric inequality. The key idea of Peierls' argument is to define a notion of contour and calculate the energy cost of "erasing" each contour, i.e., the energy cost of flipping all spins inside the contour. When there is no external field, that energy necessary to flip the spins in a region $A\subset \Z^d$ is of the order of the boundary $|\partial A|$. When we add an external field, we get an extra cost depending on this field. Imry and Ma argued that this cost should be approximately $\sqrt{|A|}$, which is smaller than $|\partial A|$ for all regions only when $d\geq 3$, so this should be the region where phase transition occurs. The other argument, due to Parisi and Sourlas \cite{Parisi.Sourlas.79}, based on dimensional reduction, predicted that the $d$-dimensional RFIM would behave like the $d-2$-dimensional nearest-neighbor Ising model, therefore presenting phase transition only when $d\geq 4$. 

The question was settled by two celebrated papers showing that Imry and Ma's prediction was correct. First, in 1988, Bricmont and Kupiainen \cite{Bricmont.Kupiainen.88} showed that there is phase transition almost surely in $d\geq3$, for low temperatures and variance $\varepsilon$ small enough. Their proof uses a rigorous renormalization group analysis for the short-range case and it is considered involved. Still, they claimed that the result works for any model with a suitable contour representation and centered sub-gaussian external field. Later on, Aizenman and Wehr \cite{Aizenman.Wehr.90} proved uniqueness for $d\leq 2$. For detailed proofs of these results, we refer the reader to \cite{Bovier.06} (see also \cite{Berretti.85, Camia.18, Frohlich.Imbre.84,  Klein.Masooman.97} for more uniqueness results). 

Recently, Ding and Zhuang, see \cite{Ding2021}, provided a simpler proof of the phase transition, not using RGM. And in  \cite{Ding.Liu.Xia.22}, Ding, Liu and Xia proved that if $\beta_c(d)$ is the critical inverse of the temperature of the Ising model with no field, for all $\beta>\beta_c(d)$ there exists a critical value $\varepsilon_0(d, \beta)$ such that the RFIM with $\varepsilon \leq \varepsilon_0$ presents phase transition. 

In the present paper, we are considering a long-range Ising model with a random field, whose Hamiltonian is given formally by
\begin{equation*}
    H(\sigma) = - \sum_{x,y\in \Z^d}J_{xy}\sigma_x\sigma_y - \varepsilon\sum_{x\in\Z^d}h_x\sigma_x,
\end{equation*}
where $J_{xy}=J|x-y|^{-\alpha}$, $J, \varepsilon>0$, $\alpha > d$ and $h_x\in\mathbb{R}$, $d\geq 3$.
Until now, the only known result in the long-range setting is for the one-dimensional long-range Ising model with a random field, by Cassandro, Orlandi, and Picco \cite{Cassandro.Picco.09}. They used the contours of \cite{Cassandro.05} to show the phase transition for the model when $\alpha\in (3-\frac{\ln 3}{\ln 2}, \frac{3}{2})$, under the assumption $J(1) \gg 1$. We stress that, as remarked by Aizenman, Greenblatt, and Lebowitz \cite{Aizenman_Greenblatt_Lebowitz_2012}, although their argument does not work for the whole region for the exponent $\alpha$, the phase transition holds for values close to the critical value $\alpha=3/2$, since by the Aizenman-Wehr theorem we know that there is uniqueness for $\alpha>3/2$.

The argument from Ding and Zhuang in \cite{Ding2021}, for $d\geq3$, involves controlling the probability of a bad event, which is closely related to controlling the quantity $$\sup_{\substack{0\in A\subset\Z^d \\ A \text{ connected }}}\frac{\sum_{x\in A}h_x}{|\partial A|},$$ known as the greedy animal lattice normalized by the boundary. The greedy animal lattice normalized by the size, instead of the boundary, was extensively studied for general distributions of $(h_x)_{x\in\Z^d}$, see \cite{Cox_Gandolfi_Griffin_Kesten_93, Gandolfi_Kesten_94, Hammond_06, Martin_02}. When we normalize by the boundary, an argument by Fisher, Fr\"{o}hlich and Spencer \cite{FFS84} shows that the expected value of the greedy animal lattice is constant. In dimension $d=2$, the expected value is not finite, see \cite{Ding.Wirth.20}. The supremum is taken over connected regions containing the origin since the interiors of the usual Peierls contours are of this form.


For the long-range model, the interior of contours is not necessarily connected. In fact, long-range contours may have considerably large diameters with respect to their size, so their interiors can be very sparse. To avoid this, we define contours, strongly inspired by the $(M,a,r)$-partition in \cite{Affonso.2021}, using a multiscaled procedure that assures that the contours have no cluster with small density.  With them, we generalize the arguments by Fisher-Fr\"{o}hlich-Spencer \cite{FFS84}, and prove that the expected value of the greedy animal lattice is constant, even considering regions not necessarily connected in the supremum. Then, we prove the phase transition for $d\geq 3$. The main result of this paper is the following.
\begin{theorem*}Given $d\geq 3$, $\alpha>d$, there exists $\beta_c\coloneqq\beta(d, \alpha)$ and $\varepsilon_c\coloneqq\varepsilon(d, \alpha)$ such that, for $\beta >\beta_c$ and $\varepsilon\leq \varepsilon_c$, the extremal Gibbs measures $\mu_{\beta, \varepsilon}^+$ and $\mu_{\beta, \varepsilon}^-$ are distinct, that is, $\mu_{\beta, \varepsilon}^+ \neq \mu_{\beta, \varepsilon}^-$ $\mathbb{P}$-almost surely. Therefore the long-range random field Ising model presents phase transition.
\end{theorem*}

This paper is divided as follows. In Section 2, we define the model and the contours, and suitable generalizations to the constructions in \cite{Affonso.2021} are introduced.  In Section 3, we define two bad events of the external field and prove that they occur with a small probability.  In Section 4, we present the proof of the phase transition.


\section{Preliminaries}  
    \subsection{The model}The set of configurations of the long-range Ising model is, as usual, $\Omega \coloneqq \{-1,1\}^{\Z^d}$. However, each spin interacts with all others, not only its neighbors, so the interaction $\{J_{xy}\}_{x,y\in\Z^d}$ is defined as

\begin{equation}\label{Long-Range Interaction}
    J_{xy} = \begin{cases}
                   \frac{J}{|x-y|^\alpha} &\text{ if }x\neq y,\\
                   0                &\text{otherwise}. 
              \end{cases}
\end{equation}
where $J >0$, $\alpha>d$ and the distance $|x-y|$ is given by the $\ell_1$-norm. The external field is a family $\{h_x\}_{x\in\Z^d}$ of i.i.d. random variables in $(\widetilde{\Omega}, \mathcal{A}, \mathbb{P})$, and every $h_x$ has a standard normal distribution. Our results also hold for more general distributions of $h_x$, see Remarks \ref{Rmk: Bernoulli_external_field} and \ref{Rmk: More.general.h_x}. 

We write $\Lambda\Subset \Z^d$ to denote a finite subset of $\Z^d$. Fixed such $\Lambda$, the \textit{local configurations} are $\Omega_\Lambda\coloneqq \{-1,1\}^\Lambda$. Moreover, given ${\eta\in\Omega}$, the set of local configurations with $\eta$ boundary conditions is ${\Omega_\Lambda^\eta\coloneqq \{\sigma\in\Omega : \sigma_x=\eta_x, \text{ }\forall x\in\Lambda^c\}}$. The \textit{local Hamiltonian of the random field long-range Ising model} in $\Lambda\Subset\Z^d$ with $\eta$-boundary condition is $H_{\Lambda, \varepsilon h}^{\eta}:  \Omega_\Lambda^\eta \to \mathbb{R}$, given by 

\begin{equation}
    H_{\Lambda; \varepsilon h}^{\eta}(\sigma)= -\sum_{x,y\in\Lambda} J_{x,y}\sigma_x\sigma_y - \sum_{x\in \Lambda, y\in\Lambda^c} J_{x,y}\sigma_x\eta_y - \sum_{x\in\Lambda} \varepsilon h_x\sigma_x,
\end{equation}
where $\varepsilon >0$ is a parameter that controls the variance of the external field. Given $\Lambda\Subset\Z^d$, consider $\mathscr{F}_\Lambda$ the $\sigma$-algebra generated by the cylinders sets supported in $\Lambda$ and $\mathscr{F}$ the $\sigma$-algebra generated by finite union of cylinders. One of the main objects of study in classical statistical mechanics is the \textit{finite volume Gibbs measures}, which are probability measures in $(\Omega, \mathscr{F})$, given by 
    \begin{equation}
        \mu_{\Lambda;\beta, \varepsilon h}^\eta(\sigma) = \mathbbm{1}_{\Omega_\Lambda^\eta}(\sigma)\frac{e^{-\beta H_{\Lambda, \varepsilon h}^{\eta}(\sigma)}}{Z_{\Lambda; \beta, \varepsilon}^{\eta}(h)},
    \end{equation}
where $\beta>0$ is the inverse temperature and $Z_{\Lambda; \beta, \varepsilon}^{\eta}$ is called \textit{partition function}, defined as 

\begin{equation}
    Z_{\Lambda; \beta, \varepsilon}^{\eta}(h)\coloneqq \sum_{\sigma\in\Omega_\Lambda^\eta} e^{-\beta H_{\Lambda, \varepsilon h}^{\eta}(\sigma)}.
\end{equation}
One important remark is that, since the external field is random, the Gibbs measures are random variables. To explicit the dependence of $\mu_{\Lambda;\beta, \varepsilon h}^\eta$ on $\widetilde{\Omega}$, we will often write $\mu_{\Lambda;\beta, \varepsilon h}^\eta[\omega]$, with $\omega$ being a general element of $\widetilde{\Omega}$. Two particularly important boundary conditions are given by the configurations $\eta_{+} \equiv +1$ and $\eta_{-} \equiv -1$, and are called $+$ and $-$ boundary conditions, respectively. For these boundary conditions, we can $\mathds{P}$-almost surely define the infinite volume measures by taking the weak*-limit
\begin{equation}
    \mu_{\beta,\varepsilon h}^{\pm}[\omega] = \lim_{n\to\infty} \mu_{\Lambda_n;\beta, \varepsilon h}^{\pm}[\omega],
\end{equation}
where $(\Lambda_n)_{n\in\mathbb{N}}$ is any sequence invading $\Z^d$, that is, for any subset $\Lambda\Subset\mathbb{Z}^d$, there exists $N=N(\Lambda)>0$ such that $\Lambda\subset\Lambda_n$ for every $n>N$.
To have more than one Gibbs measure, it is enough to show that $\mu_{\beta,\varepsilon h}^{+}[\omega]\neq  \mu_{\beta,\varepsilon h}^{-}[\omega]$, with $\mathbb{P}$-probability 1, see \cite[Theorem 7.2.2]{Bovier.06}.


    \subsection{The contours} Contours were first defined in the seminal paper of R. Peierls \cite{Peierls.1936}, where he introduced these geometrical objects to prove phase transition in the Ising model for $d\geq 2$. This technique is known nowadays as the \textit{Peierls' Argument}. Several extensions of this argument were made for different systems. The most successful of them was made by S. Pirogov and Y. Sinai \cite{Pirogov.Sinai.75}, and extended by Zahradnik \cite{Zahradnik.84}. This is known as the \textit{Pirogov-Sinai} Theory, which can be used in models with short-range interactions and finite state spaces, even without symmetries. The Pirogov-Sinai Theory was one of the achievements cited when Yakov Sinai received the Abel Prize \cite{Sinai_Abel_Prize}.

For long-range models, using the usual Peierls' contours with plaquettes of dimension $d-1$, Ginibre, Grossman, and Ruelle, in \cite{Ginibre.Grossmann.Ruelle.66}, proved phase transition for $\alpha > d+1$.  Park, in \cite{Park.88.I,Park.88.II}, considered systems with two-body interactions satisfying $|J_{xy}|\leq |x-y|^{-\alpha}$ for $\alpha > 3d+1$, and extended the Pirogov-Sinai theory for this class of models.  Fr{\"o}hlich and Spencer, in \cite{Frohlich.Spencer.82}, proposed a different contour definition for the one-dimensional long-range Ising models. Roughly speaking, collections of intervals will be the new contours. When they are sufficiently far apart, the collections are different contours, while collections of intervals close enough are considered a single contour. Note that this definition drastically changes the notion of contour since now they are not necessarily connected objects. This fact implies that the control of the number of contours for a fixed size could be much more challenging. Inspired by such contours, Affonso, Bissacot, Endo, and Handa proposed a definition of contour extending the contours of Fr{\"o}hlich and Spencer to any dimension $d\geq 2$, see \cite{Affonso.2021}. With these contours, they were able to use Peierls' argument to show phase transition in the whole region $\alpha>d$, with $d\geq 2$. However, such contours can be very sparse, in the sense that its diameter can be much larger than its size. This fact makes them not suitable to deal with the random external field. We modify the contour definition for our proposals, adding a restriction that splits clusters with small volumes into different contours, see Proposition \ref{Prop:Construction_(M,a,delta)_partition}. In this section, we describe our contours.

\begin{definition}\label{def1}
	Given $\sigma \in \Omega$, a point $x \in \Z^d$ is called \emph{+ (or - resp.)} \emph{correct} if $\sigma_y = +1$, (or $-1$, resp.) for all points $y$ in $B_1(x)$. The \emph{boundary} of $\sigma$, denoted by $\partial \sigma$, is the set of all points in $\Z^d$ that are neither $+$ nor $-$ correct.
\end{definition}

Here, $B_R(x)$ is the ball with radius $R$ in the $\ell_1$-norm. The boundary of a configuration is not finite in general, it can even be the whole lattice $\Z^d$. To avoid this problem, we will restrict our attention to configurations with finite boundaries. Such configurations, by definition of incorrectness, satisfy $\sigma \in \Omega^+_\Lambda$ or $\sigma \in \Omega^-_\Lambda$ for some $\Lambda\Subset \Z^d$. For each $\Lambda\Subset \Z^d$, $\Lambda^{(0)}$ is the unique unbounded connected component of $\Lambda^c$. The \textit{volume} of $\Lambda$ is defined as $V(\Lambda)\coloneqq \Z^d\setminus \Lambda^{(0)}$. The \textit{interior} of $\Lambda$ is $\I(\Lambda)\coloneqq \Lambda^c\setminus \Lambda^{(0)}$.


The usual definition of contours in Pirogov-Sinai theory considers only the connected subsets of the boundary $\partial \sigma$. We have to proceed differently for long-range models since every point in the lattice interacts with all the others. The definition below, which is strongly inspired in \cite{Affonso.2021}, allows contours to be disconnected (as in unidimensional long-range models); in return, we can control the interaction between two contours.

\begin{definition}\label{Def: delta-partiton}
    Let $M>0$ and $a,\delta >d$. For each $A\Subset\Z^d$, a set $\Gamma(A) \coloneqq \{\overline{\gamma} : \overline{\gamma} \subset A\}$ is called an $(M,a,\delta)$-\emph{partition} when the following conditions are satisfied:
	\begin{enumerate}[label=\textbf{(\Alph*)}, series=l_after] 
		\item They form a partition of $A$, i.e.,  $\bigcup_{\overline{\gamma} \in \Gamma(A)}\overline{\gamma}=A$ and $\overline{\gamma} \cap \overline{\gamma}' = \emptyset$ for distinct elements of $\Gamma(A)$. Moreover, each $\overline{\gamma}'$ is contained in only one connected component of $(\overline{\gamma})^c$. 
		
		\item For all $\overline{\gamma}, \overline{\gamma}^\prime \in \Gamma(A)$ 
			\be\label{B_distance_2}
			\dis(\overline{\gamma},\overline{\gamma}') > M\min\left \{|V(\overline{\gamma})|,|V(\overline{\gamma}')|\right\}^\frac{a}{\delta}.
			\ee
	\end{enumerate}
\end{definition}

The existence of a $(M,a)$-partition for any $A\Subset\Z^d$ does not depend on the choice of $M,a>0$. However, to guarantee the existence of phase transition, we have to choose particular values for these parameters. From now on we fix $a \coloneqq a(\alpha,d,\epsilon) = \max\left\{\frac{2(d+1)+\epsilon}{\alpha-d}, 2(d+1)+\epsilon\right\}$, for any $\epsilon>0$ fixed. Later on, in Proposition \ref{Prop: Cost_erasing_contour}, $M$ will be taken large enough.

\begin{remark}
    Our construction and the control of the energy works for any $d<\delta<\frac{a(\alpha - d)}{2}$. To simplify the calculations, we will take $\delta = d+1$ from now on, so a $(M,a,\delta)$-partition will be called $(M,a)$-partition.
\end{remark}

\begin{remark}
A similar partition was used to define the contours in \cite{Affonso.2021}, relying on multiscale methods which we adapted in our construction. Their definition is enough to control the energy of erasing a contour and also its entropy. These are the two main ingredients of the Peierls' argument in the deterministic case. Unfortunately, these contours can be formed by clusters of thin sets, each cluster being far away from each other. In the random field case, we need to control the probability of a bad event that correlates with all the interiors of the contours, thus such contours may correlate very distant parts with large fields on the lattice. To avoid this problem, we introduced the new Definition \ref{Def: delta-partiton}, that separates all subsets with small densities into different contours since we use the volume instead of the diameter.
\end{remark}

We write $\Gamma(\sigma) \coloneqq \Gamma(\partial\sigma)$ for a $(M,a)$-partition of $\partial\sigma$. By condition \textbf{(A)}, $\overline{\gamma}^\prime$ is contained in the unbounded component of $\overline{\gamma}^c$ if and only if $V(\overline{\gamma})\cap V(\overline{\gamma}^\prime) = \emptyset$. In general, there is more than one $(M,a)$-partition for each region $A\in\Z^d$. Given two partitions $\Gamma$ and $\Gamma^\prime$ of a set $A$, we say that $\Gamma$ \emph{is finer than} $\Gamma'$, and denote $\Gamma\preceq\Gamma^{\prime}$, if for every $\overline{\gamma} \in \Gamma$ there is $\overline{\gamma}' \in \Gamma'$ with $\overline{\gamma} \subseteq \overline{\gamma}'$. From now on, when taking a $(M,a)$-partition $\Gamma(A)$, we will always assume it is the finest. The next proposition shows that such a partition always exists.

\input{Figure_1}

 \begin{proposition}
     For every $A\Subset\Z^d$, there is a finest $(M,a)$-partition.
 \end{proposition}
\begin{proof}
    Given any two $(M,a)$-partitions $\Gamma(A)$ and $\Gamma^\prime(A)$, consider
\begin{equation*}
    \Gamma\cap\Gamma^\prime\coloneqq \{\overline{\gamma}\cap\overline{\gamma}^\prime : \overline{\gamma} \in \Gamma(A), \ \overline{\gamma}^\prime\in\Gamma^\prime(A), \ \overline{\gamma}\cap\overline{\gamma}^\prime\neq \emptyset\}.
\end{equation*}

Then, $\Gamma\cap\Gamma^\prime$ is a $(M,a)$-partition of $A$ finer than $\Gamma(A)$ and $\Gamma^\prime(A)$. Indeed, given $\overline{\gamma}_1\cap\overline{\gamma}_1^\prime, \overline{\gamma}_2\cap\overline{\gamma}_2^\prime\in \Gamma\cap\Gamma^\prime$, by condition \textbf{(A)}, there exists a connected component $A$ of $\overline{\gamma}_2^c$ that contains $\overline{\gamma}_1$. Hence
\begin{equation*}
    \overline{\gamma}_1\cap\overline{\gamma}_1^\prime\subset A \subset (\overline{\gamma}_2\cap\overline{\gamma}_2^\prime)^c.
\end{equation*}

For condition \textbf{(B)}, just notice that
\begin{align*}
    \d(\overline{\gamma}_1\cap\overline{\gamma}_1^\prime, \overline{\gamma}_2\cap\overline{\gamma}_2^\prime)\geq \d(\overline{\gamma}_1, \overline{\gamma}_2) \geq  M\min{\{|V(\overline{\gamma}_1\cap\overline{\gamma}_1^\prime)|, |V(\overline{\gamma}_2\cap \overline{\gamma}_2^\prime)|\}}^{\frac{a}{d+1}}.
\end{align*}
As the number of $(M,a)$-partitions is finite, we construct the finest one by intersecting all of them. 
\end{proof}

Counting the number of contours surrounding zero using the finest $(M,a)$-partition may be troublesome since we have very little information on the geometry of these objects. To get around this, we establish a multiscaled procedure, depending on a parameter $r$, that creates a $(M,a)$-partition of any given set. To define this procedure, we introduce some notation. 

 For any $x\in\Z^d$ and $m\geq 0$,
\begin{equation}
    C_{m}(x) \coloneqq \left(\prod_{i=1}^d{\left[2^{m}x_i , \ 2^{m}(x_i+1) \right)}\right)\cap \Z^d,
\end{equation}
is the cube of $\mathbb{Z}^d$ centered at $2^{m}x + 2^{m-1} - \frac{1}{2}$ with side length $2^{m} -1$. Any such cube is called an $m$-cube. As all cubes in this paper are of this form, with centers $2^{m}x + 2^{m-1} - \frac{1}{2}$ and $x \in \mathbb{Z}^d$, we will often omit the center in what follows, writing $C_m$ for an $m$-cube instead of $C_m(x)$. An arbitrary collection of $m$-cubes will be denoted $\mathscr{C}_m$ and $B_{\mathscr{C}_m}\coloneqq \cup_{C\in\mathscr{C}_m}C$ is the region covered by $\mathscr{C}_m$. We denote by $\mathscr{C}_m(\Lambda)$ the covering of $\Lambda\Subset\Z^d$ with the smallest possible number of $m$-cubes.

\input{Figure_0}

For each $n\geq 0$, define the graph $G_n(\Lambda) = (V_n(\Lambda), E_n(\Lambda))$ with vertex set $V_n(\Lambda) = \mathscr{C}_n(\Lambda)$ and $E_n(\Lambda) = \{ (C_n, C_n^\prime) : d(C_n,C_n^\prime) \leq M2^{an}\}$. Let $\mathscr{G}_n(\Lambda)$ be the connected components of $G_n(\Lambda)$. Given ${G = (V,E) \in \mathscr{G}_n(\Lambda)}$, we denote $\Lambda^G \coloneqq \Lambda \cap B_V$ the area of $\Lambda$ covered by $G$. 

\begin{definition}\label{Prop:Construction_(M,a,delta)_partition}
   Given $r> 0$ and $A\Subset\Z^d$, $\Gamma^r(A)$ is the partition of A created by the following procedure: in the first step we consider $A_1 \coloneqq A$ and we take the connected components of $G\in \mathscr{G}_r(A_1)$ such that $A_1^G$ have small density, that is, consider
\begin{equation*}
   \mathscr{P}_1 = \{G \in  \mathscr{G}_r(A_1) : |V(A_1^G)|\leq {2^{r(d+1)}}\}.
\end{equation*}
    Then, the subsets to be removed in the first step are $\Gamma_1^r(A) = \{A_1^G : G\in \mathscr{P}_1\}$ and the set left to partition is $A_2 = A_1\setminus \{\gamma \in \Gamma_1^r(A)\}$. We can repeat this inductively by taking 
\begin{equation*}
   \mathscr{P}_n = \{G \in  \mathscr{G}_{rn}(A_{n}) : |V(A_n^G)|\leq {2^{rn{(d+1)}}}\},
\end{equation*}
then define $\Gamma_n^r(A) = \{A_n^G : G \in \mathscr{P}_n\}$ and $A_{n+1} = A_n \setminus \{ \gamma\in \Gamma_n^r(A)\}$. As the cubes invade the lattice, this procedure stops, in the sense that for some $N$ large enough, $\mathscr{P}_n=\emptyset$ for all $n\geq N$. We then define $\Gamma^r(A) \coloneqq \cup_{n\geq 0} \Gamma_n^r(A)$.
\end{definition}
The next proposition shows that these partitions are indeed $(M,a)$-partitions.
\begin{proposition}
    For any $r> 0$ and $A\Subset\Z^d$, $\Gamma^r(A)$ is a $(M,a)$-partition.
\end{proposition}

\begin{proof}
 The proof that $\Gamma^r(A)$ satisfies condition  \textbf{(A)} is identical to the proof in \cite{Affonso.2021} that their partitions satisfy condition  \textbf{(A)}. To show condition \textbf{(B)}, take $\overline{\gamma},\overline{\gamma}^\prime\in \Gamma^r(A)$. Let $m\geq n\geq 1$ be such that $\overline{\gamma}\in\Gamma_n^r(A)$ and $\overline{\gamma}^\prime\in\Gamma_m^r(A)$. Then, 
\begin{equation*}
\d(\overline{\gamma},\overline{\gamma}^\prime)\geq M2^{rna}\geq M\left(2^{rn(d+1)}\right)^{\frac{a}{d+1}} \geq M|V(\overline{\gamma})|^{\frac{a}{d+1}}.    
\end{equation*}
If $m=n$, we the same inequality holds for $|V(\overline{\gamma}^\prime)|$ and condition \textbf{(B)} holds. When $m>n$, $\overline{\gamma}^\prime$ was not removed at step $n$, so $|V(\overline{\gamma}^\prime)|> 2^{rn(d+1)} \geq |V(\overline{\gamma})|$, so $|V(\overline{\gamma})| = \min\{|V(\overline{\gamma})|,|V(\overline{\gamma}^\prime)|\}$ and again we get condition \textbf{(B)}. 
\end{proof}

Our construction works for any $r>0$, but we need to take $r$ large enough so all the computations work. So we fix $r\coloneqq 4\lceil\log_2(a+1) \rceil + d +1$, where $\lceil x \rceil$ is the smallest integer greater than or equal to $x$. This $r$ is taken larger than the one in \cite{Affonso.2021} to simplify some calculations. All our computations should work with the previous choice of $r$, with some adaptation. Next, we define the label of a contour. 
 
\begin{definition}
    For $\Lambda\subset\Z^d$, the \textit{edge boundary} of $\Lambda$ is $\partial\Lambda = \{\{x,y\} \subset \Z^d: |x-y|=1, x \in \Lambda, y \in \Lambda^c\}$. The \textit{inner boundary} of $\Lambda$ is $\fint\Lambda\coloneqq\{x\in\Lambda : \exists y\in \Lambda^c \text{ such that }|x-y|=1\}$ and the \textit{external boundary} is $\fext\Lambda\coloneqq\{x\in\Lambda^c : \exists y\in \Lambda \text{ such that }|x-y|=1\}$
\end{definition}

\begin{remark}
    The usual isoperimetric inequality states that $2d|\Lambda|^{\frac{d-1}{d}}\leq |\partial \Lambda|$. The inner boundary and the edge are related by $|\fint \Lambda|\leq |\partial \Lambda|\leq 2d|\fint \Lambda|$, so we can write the inequality as $|\Lambda|^{\frac{d-1}{d}}\leq |\fint \Lambda|$.
\end{remark}

To define the label of a contour, the most naive definition would be to take the sign of the inner boundary of the set $\overline{\gamma}$. However, this cannot be done since this inner boundary may have different signs, see Figure \ref{fig: Figura2}.

\input{Figure_2}

For any $\Lambda\Subset\Z^d$, its connected components are denoted $\Lambda^{(1)}, \dots, \Lambda^{(n)}$. Given $\overline{\gamma} \in\Gamma(\sigma)$, a connected component $\overline{\gamma}^{(k)}$ is \textit{external} if $V(\overline{\gamma}^{(j)})\subset V(\overline{\gamma}^{(k)})$, for all other connected components $\overline{\gamma}^{(j)}$ satisfying $V(\overline{\gamma}^{(j)})\cap V(\overline{\gamma}^{(k)}) \neq \emptyset$. Denoting 
\begin{equation*}
    \overline{\gamma}_\mathrm{ext} = \hspace{-0.5cm}\bigcup_{\substack{k\geq 1 \\ \overline{\gamma}^{(k)} \text{ is external}}}\hspace{-0.5cm}\overline{\gamma}^{(k)},
\end{equation*} 
it is shown in \cite[Lemma 3.8]{Affonso.2021} that the sign of $\sigma$ is constant in $\fint V(\overline{\gamma}_{\mathrm{ext}})$. The \textit{label} of $\overline{\gamma}$ is the function $\lab_{\overline{\gamma}} :\{(\overline{\gamma})^{(0)}, \I(\overline{\gamma})^{(1)}\dots, \I(\overline{\gamma})^{(n)}\} \rightarrow \{-1,+1\}$ defined as: $\lab_{\overline{\gamma}}(\I(\overline{\gamma})^{(k)})$ is the sign of the configuration $\sigma$ in $\fint V(\I(\overline{\gamma})^{(k)})$, for $k\geq 1$, and $\lab_{\overline{\gamma}}((\overline{\gamma})^{(0)})$ is the sign of $\sigma$ in $\fint V(\overline{\gamma}_\mathrm{ext})$.  We then define the contours.
\begin{definition}
Given a configuration $\sigma$ with finite boundary, its \emph{contours} $\gamma$ are pairs $(\overline{\gamma},\lab_{\overline{\gamma}})$,  where $\overline{\gamma} \in \Gamma(\sigma)$ and $\lab_{\overline{\gamma}}$ is the label of $\overline{\gamma}$ as defined above. The \emph{support of the contour}  $\gamma$ is defined as $\Sp(\gamma)\coloneqq \overline{\gamma}$ and its \emph{size} is given by $|\gamma| \coloneqq |\Sp(\gamma)|$.
\end{definition}

Another important definition is of the \textit{interior} of a contour $\gamma$, given by $\I(\gamma) \coloneqq V(\Sp(\gamma)) \setminus \Sp(\gamma)$. We also split the interior in

\begin{equation*}
    \I_\pm(\gamma) = \hspace{-0.7cm}\bigcup_{\substack{k \geq 1, \\ \lab_{\overline{\gamma}}(\I(\gamma)^{(k)})=\pm 1}}\hspace{-0.7cm}\I(\gamma)^{(k)}.
\end{equation*}
To simplify the notation, we write $V(\gamma)\coloneqq V(\Sp(\gamma))$. Different from Pirogov-Sinai theory, where the interiors of contours are a union of simply connected sets, the interior $\I(\gamma)$ is at most the union of connected sets, that is, they may have holes. 

Moreover, there is no bijection between families of contours $\Gamma = \{\gamma_1,\dots,\gamma_n\}$ and configurations. Usually, more than one configuration can have the same boundary. First, $\Gamma$ may not even form a $(M,a)$-partition. Even so, their labels may not be compatible. We say that $\Gamma$ is \textit{compatible} when there exists a configuration $\sigma$ with contours precisely $\Gamma$.

\input{Figure_3}

A contour $\gamma$ in $\Gamma$ is \textit{external} if its external connected components are not contained in any other $V(\gamma')$, for $\gamma' \in \Gamma\setminus\{\gamma\}$. Taking $V(\Gamma)\coloneqq\cup_{\gamma\in\Gamma}V(\gamma)$, for each $\Lambda\Subset\Z^d$ we consider the sets\\
\begin{equation*}
\mathcal{E}^\pm_\Lambda \coloneqq\{\Gamma= \{\gamma_1, \ldots, \gamma_n\}: \Gamma \text{ is compatible,} \gamma_i \text{ is external}, \lab_{\gamma_i}((\gamma_i)^{(0)})=\pm1, V(\Gamma) \subset \Lambda\},
\end{equation*}
\\of all external compatible families of contours with external label $\pm$ contained in $\Lambda$.  When we write $\gamma \in \mathcal{E}^\pm_\Lambda$ we mean $\{\gamma\} \in \mathcal{E}^\pm_\Lambda$. Most of the time the set $\Lambda$ will play no hole, so we will often omit the subscript. 

The first step for a Peierl's-type argument to hold is to control the number of contours with a fixed size. Consider 
    \[
	\mathcal{C}_0(n) \coloneqq \{\gamma \in \mathcal{E}^+_\Lambda: 0 \in V(\gamma), |\gamma|=n\},
	\]
the set of contours with fixed size with the origin in its volume, and  $\mathcal{C}_0 \coloneqq \cup_{n\geq 1}\mathcal{C}_0(n)$. As we are taking the finest $(M,a)$-partition, for any $\gamma\in\mathcal{C}_0(n)$, $\Gamma(\gamma)=\{\gamma\}$. Hence, $\Gamma^r(\gamma) = \{\gamma\}$ and we can take, for $j\geq 1$,
\[
	\mathcal{C}_0(n,j) = \{\gamma \in \mathcal{C}_0(n): \gamma\in\Gamma_j^r(\gamma)\},
\]
the set of contours with fixed size and removed at step $j$. We will later show in Corollary \ref{Cor: Bound_on_C_0_n} that the size of the set $C_0(n)$ is exponentially bounded depending on $n$. 

A key step of a Peierls-type argument is to control the energy cost of erasing a contour. Given $\Gamma\in \mathcal{E}^+$, the configurations compatible with $\Gamma$ are $\Omega(\Gamma)\coloneqq \{\sigma\in\Omega^+_\Lambda : \Gamma\subset \Gamma(\sigma)\}$. The map $\tau_\Gamma:\Omega(\Gamma) \rightarrow \Omega_{\Lambda}^+$ defined as 
\be
\tau_\Gamma(\sigma)_x = 
\begin{cases}
	\;\;\;\sigma_x &\text{ if } x \in \I_+(\Gamma)\cup V(\Gamma)^c, \\
	-\sigma_x &\text{ if } x \in \I_-(\Gamma),\\
	+1 &\text{ if } x \in \Sp(\Gamma),
\end{cases}
\ee
erases a family of compatible contours, since the spin-flip preserves incorrect points but transforms $-$-correct points into $+$-correct points. Define, for $B\Subset\Z^d$, the interaction
\begin{equation*}
    F_B\coloneqq \sum_{\substack{x\in B \\ y\in  B^c}}J_{xy}.
\end{equation*}

Given $B\subset\Z^d$ and $\sigma\in\Omega$ with $\partial\sigma$ finite, let $\Gamma_{\Int}(\sigma, B)$ be the contours $\gamma^\prime \in \Gamma(\sigma)$ enclosed by $B$, that is, $\gamma^\prime\subset B$. Define also $\Gamma_{\Ext}(\sigma, B)$ as the contours $\gamma^\prime \in \Gamma(\sigma)$ outside $B$, that is, $\gamma^\prime\subset B^c$. 

\begin{lemma}\label{Lemma: Aux_1}
Given $\sigma\in\Omega$ with $\partial \sigma$ finite and $\gamma\in \Gamma(\sigma)$ , there is a constant $\kappa^{(1)}_\alpha\coloneqq \kappa^{(1)}_\alpha(\alpha, d)$, such that, for  $B = \Sp(\gamma)$ or $B=\I_-(\gamma)$ we have 

\begin{equation}\label{Eq: Lemma_aux_2}
    \sum_{\substack{x\in B \\ y\in V(\Gamma_{\Ext}(\sigma, B)\setminus\{\gamma\})}} J_{xy} \leq \kappa^{(1)}_\alpha \left[ \frac{|B|}{M^{\alpha - d}}|V(\gamma)|^{\frac{a}{d+1}(d-\alpha)} + \frac{F_B}{M} \right],  
\end{equation}

\end{lemma}

\begin{proof}

Fixed $\sigma$ and $B$, we drop them from the notation, so $\Gamma_{\Ext} \coloneqq \Gamma_{\Ext}(\sigma, B)$.  Splitting $\Gamma_{\Ext}\setminus\{\gamma\}$ into $\Upsilon_1 \coloneqq \{\gamma^\prime \in \Gamma_{\Ext} : |V(\gamma^\prime)| \geq |V(\gamma)|\}\setminus\{\gamma\}$ and $\Upsilon_2 = \Gamma_{\Ext} \setminus (\Upsilon_1\cup \gamma)$ we get 
\begin{equation*}
     \sum_{\substack{x\in B \\ y\in V(\Gamma_{\Ext}\setminus\{\gamma\})}} J_{xy} \leq  \sum_{\substack{x\in B \\ y\in V(\Upsilon_1)}} J_{xy} +  \sum_{\substack{x\in B \\ y\in V(\Upsilon_2)}} J_{xy}.
\end{equation*}
For any  $\gamma^\prime \in \Upsilon_1$, $\d(\gamma,\gamma^\prime)> M |V(\gamma)|^{\frac{a}{d+1}}$, hence 
\begin{equation}\label{Eq: Lemma_aux_2_1}
    \sum_{\substack{x\in B \\ y\in V(\Upsilon_1)}} J_{xy} \leq \sum_{\substack{x\in B \\ y : |y-x| > R}} J_{xy} = |B|\sum_{y : |y|>R}J_{0y},
\end{equation}
with $R\coloneqq M |V(\gamma)|^{\frac{a}{d+1}}$. 

Defining $s(n) \coloneqq |\{x\in \Z^d : |x|=n \}|$, it is known that $s(n)\leq 2^{2d - 1}e^{d-1}n^{d-1}$, see for example \cite[Lemma 4.2]{Affonso.2021}. Using the integral bound of the sum, we can show that
\begin{align}\label{Eq: Interaction_outside_ball}
 \sum_{y : |y|>R}J_{0y} &= J \sum_{n>R} \frac{s(n)}{n^\alpha} \leq J2^{2d - 1}e^{d-1} \sum_{n>R} \frac{1}{n^{\alpha-d+1}} \nonumber\\
 %
 &\leq J2^{2d - 1}e^{d-1} \int_{\floor{R}}^{\infty}\frac{1}{x^{\alpha - d +1}}dx \leq \frac{J2^{2d - 1}e^{d-1}}{(\alpha - d)} \floor{R}^{d-\alpha}\leq \frac{J2^{d - 1 + \alpha}e^{d-1}}{(\alpha - d)} {R}^{d-\alpha}.
\end{align}
Together with \eqref{Eq: Lemma_aux_2_1}, this yields
\begin{align}\label{Eq: Lemma_aux_2.Part1}
    \sum_{\substack{x\in B \\ y\in V(\Upsilon_1)}} J_{xy} &\leq |B| \frac{J2^{d-1+\alpha}e^{d-1}}{(\alpha - d)}\left[ M |V(\gamma)|^{\frac{a}{d+1}} \right]^{d-\alpha} \nonumber \\
    %
    &\leq \frac{J2^{d-1 + \alpha }e^{d-1}}{(\alpha - d)}\frac{|B|}{M^{\alpha - d}}|V(\gamma)|^{\frac{a}{d+1}(d-\alpha)}.
\end{align}
To bound the other term, split $\Upsilon_2$ into layers $\Upsilon_{2,m} \coloneqq \{ \gamma^\prime \in \Upsilon_2 : |V(\gamma^\prime)|=m\}$, for $1\leq m\leq |V(\gamma)|-1$. Denoting  $y_{\gamma^\prime,x} \in \gamma^\prime$ the point satisfying $\d(x, \gamma^\prime) = \d(x, y_{\gamma^\prime, x})$, we can bound 
\begin{align*}
   \sum_{\substack{x\in B \\ y\in V(\Upsilon_{2,m})}} J_{xy} &\leq  \sum_{\substack{x\in B \\ \gamma^\prime \in \Upsilon_{2,m}}} |V(\gamma^\prime)| J_{x,y_{\gamma^\prime, x}} \leq m \sum_{\substack{x\in B \\ \gamma^\prime \in \Upsilon_{2,m}}} J_{x,y_{\gamma^\prime, x}}.
\end{align*}
Since $\d(x,y_{\gamma^\prime, x}) \geq \d(\gamma, \gamma^\prime)>Mm^{\frac{a}{d+1}}$, and $\d(y_{\gamma^\prime, x}, y_{\gamma^{\prime\prime}, x}) \geq \d(\gamma^\prime, \gamma^{\prime\prime})>Mm^{\frac{a}{d+1}}$ for any $\gamma^\prime, \gamma^{\prime\prime}\in\Upsilon_{2,m}$, the balls with radius $\frac{M}{3}m^{\frac{a}{d+1}}$ centered in $y_{\gamma^\prime, x}$, for all $\gamma^\prime\in \Upsilon_{2,m}$ are disjoint and are contained in $B^c$. Hence, we can bound
\begin{equation*}
    \sum_{\substack{x\in B \\ \gamma^\prime \in\Upsilon_{2,m}}} J_{x,y_{\gamma^\prime, x}} \leq \frac{3}{Mm^{\frac{a}{d+1}}}F_B.
\end{equation*}
That gives us
\begin{align}\label{Eq: Lemma_aux_2.Part2}
     \sum_{\substack{x\in B \\ y\in V(\Upsilon_{2})}} J_{xy} \leq \sum_{m=1}^{|V(\gamma)|-1} \frac{3}{Mm^{\frac{a}{d+1}-1}}F_B\leq \frac{3\zeta({\frac{a}{d+1}-1})}{M}F_B,
\end{align}
what concludes the proof for $\kappa_\alpha^{(1)}\coloneqq \frac{J2^{d-1 + \alpha}e^{d-1}}{(\alpha - d)} + {3\zeta({\frac{a}{d+1}-1})}$.
\end{proof}

\begin{corollary}\label{Cor: Corrolary_of_Lemma_aux}
For any configuration $\sigma\in\Omega$ and $\gamma\in\Gamma(\sigma)$, 

\begin{equation}
  \sum_{\substack{x\in \Sp(\gamma) \\ y\in V(\Gamma(\sigma)\setminus\{\gamma\})}} J_{xy} \leq  2\kappa^{(1)}_\alpha\frac{F_{\Sp(\gamma)}}{M^{(\alpha - d)\wedge 1}},  \\
\end{equation}
 
\begin{equation}
    \sum_{\substack{x\in \I_-(\gamma) \\ y\in V(\Gamma_{\Ext}(\sigma, \I_-(\gamma))\setminus\{\gamma\})}} J_{xy}  \leq 2\kappa^{(1)}_\alpha\frac{F_{\I_-(\gamma)}}{M^{(\alpha - d)\wedge 1}},
\end{equation}
and 
\begin{equation}
    \sum_{\substack{x\in \I_-(\gamma)^c \\ y\in V(\Gamma_{\Int}(\sigma, \I_-(\gamma)))}} J_{xy} \leq  \kappa^{(1)}_\alpha\frac{F_{\I_-(\gamma)}}{M}
\end{equation}
\end{corollary}

\begin{proof}
    The first inequality is a direct application of the Lemma for $B=\Sp(\gamma)$, once we note that $\Gamma_{\Ext}(\sigma, B) = \Gamma(\sigma)\setminus\{\gamma\}$ and, our choice of $a$,  $\frac{|\gamma|}{|V(\gamma)|^{\frac{a}{d+1}(\alpha-d)} } \leq \frac{|\gamma|}{|V(\gamma)|}\leq 1$. The second inequality is likewise a direct application of the lemma for $B=V(\gamma)$, since  $\frac{|V(\gamma)|}{|V(\gamma)|^{\frac{a}{d+1}(\alpha-d)} } \leq \frac{|V(\gamma)|}{|V(\gamma)|}\leq 1$.

For the last inequality, we cannot apply the lemma directly. However, the proof works in the same steps when we take $B = \I_-(\gamma)^c$. Moreover, notice that $V(\Gamma_{\Int}(\sigma, \I_-(\gamma))) = V(\Gamma_{\Ext}(\sigma, \I_-(\gamma)^c))$ and, for all $\gamma^\prime\in \Gamma_{\Int}(\sigma, \I_-(\gamma))$, $|V(\gamma^\prime)| < |V(\gamma)|$.  In the notation of the proof of Lemma \ref{Lemma: Aux_1}, this means that $\Upsilon_{2} = \Gamma_{\Ext}(\sigma, \I_-(\gamma)^c)$, so equation \eqref{Eq: Lemma_aux_2.Part2} yields
\begin{align*}
      \sum_{\substack{x\in \I_-(\gamma)^c \\ y\in V(\Gamma_{\Ext}(\sigma, \I_-(\gamma)^c))}} J_{xy} \leq \zeta({\frac{a}{d+1}-1})\frac{F_{\I_-(\gamma)^c}}{M}.
\end{align*}
Since $F_{\I_-(\gamma)^c} = F_{\I_-(\gamma)}$ and $\zeta({\frac{a}{d+1}-1})\leq \kappa^{(1)}_\alpha  $, we get the desired bound.
\end{proof}

We are ready to prove the main proposition of this section:

\begin{proposition}\label{Prop: Cost_erasing_contour}
	For $M$ large enough, there exists a constant $c_2\coloneqq c_2(\alpha,d)>0$, such that for  any $\Lambda \Subset \Z^d$, $\gamma\in \mathcal{E}^+_\Lambda$, and $\sigma \in \Omega(\gamma)$ it holds that
	\be
	H_{\Lambda; 0}^+(\sigma)- H_{\Lambda;0}^+(\tau_{\gamma}(\sigma))\geq c_2|\gamma|.
	\ee
\end{proposition}

\begin{proof}
   To simplify the notation, we denote $\tau\coloneqq \tau_\gamma(\sigma)$. Taking $B(\gamma) \coloneqq \I_+(\gamma)\cup V(\gamma)^c$ and redoing the steps of the proof of \cite[Proposition 4.5]{Affonso.2021}, we can write 
\begin{align}\label{Eq: Difference_of_Hamiltonians_1}
    H_\Lambda^+(\sigma) - H_\Lambda^+(\tau) &= \sum_{\substack{x \in \Sp(\gamma) \\ y \in \Z^d}} J_{xy}\mathbbm{1}_{ \{ \sigma_x \neq \sigma_y \}} +       \sum_{\substack{x \in \Sp(\gamma) \\ y \in \Sp(\gamma)^c}} J_{xy}\mathbbm{1}_{ \{ \sigma_x \neq \sigma_y \}} - 2\sum_{\substack{x \in \I_-(\gamma) \\ y \in B(\gamma)}} J_{xy}\sigma_x\sigma_y \nonumber \\
    %
    & - 2\sum_{\substack{x \in \Sp(\gamma) \\ y \in B(\gamma)}} J_{xy}\mathbbm{1}_{ \{ \sigma_y = -1 \}} - 2\sum_{\substack{x \in \Sp(\gamma) \\ y \in \I_-(\gamma)}} J_{xy}\mathbbm{1}_{ \{ \sigma_y = +1\}}. 
\end{align}


We start by analyzing the last two negative terms. It holds that 
\begin{equation*}
    \sum_{\substack{x \in \Sp(\gamma) \\ y \in B(\gamma)}} J_{xy}\mathbbm{1}_{ \{ \sigma_y = -1 \}} + \sum_{\substack{x \in \Sp(\gamma) \\ y \in \I_-(\gamma)}} J_{xy}\mathbbm{1}_{ \{ \sigma_y = +1\}} \leq \sum_{\substack{x \in \Sp(\gamma) \\ y \in V(\Gamma(\sigma)\setminus\{\gamma\})}} J_{xy},
\end{equation*}
since the characteristic function can only be non-zero in the volume of other contours. By Corollary \ref{Cor: Corrolary_of_Lemma_aux},
\begin{equation}\label{Eq: Aux_1_Diferenca_de_Hamiltonianos}
    \sum_{\substack{x \in \Sp(\gamma) \\ y \in V(\Gamma(\sigma)\setminus\{\gamma\})}} J_{xy}  \leq 2{\kappa^{(1)}_\alpha}\frac{F_{\Sp(\gamma)}}{M^{(\alpha - d)\wedge 1}}.
\end{equation}

For the negative term left, taking $\Gamma^\prime$ the contours inside $\I_-(\gamma)$ and $\Gamma^{\prime\prime} = \Gamma(\sigma)\setminus {(\Gamma^\prime\cup\gamma)}$ we can write
\begin{multline}\label{Eq: Interaction_between_interior_and_B}
     \sum_{\substack{x \in \I_-(\gamma) \\ y \in B(\gamma)}} J_{xy}\sigma_x\sigma_y = \sum_{\substack{x \in V(\Gamma^{\prime}) \\ y \in  V(\Gamma^{\prime\prime}) }} J_{xy} + \sum_{\substack{x \in \I_-(\gamma)\setminus V(\Gamma^{\prime}) \\ y \in  V(\Gamma^{\prime\prime}) }} 2J_{xy}\mathbbm{1}_{\{\sigma_y=-1\}} + \sum_{\substack{x \in V(\Gamma^{\prime}) \\ y \in  B(\gamma)\setminus V(\Gamma^{\prime\prime}) }} 2J_{xy}\mathbbm{1}_{\{\sigma_x=+1\}} \\
     %
       - \sum_{\substack{x \in \I_-(\gamma)\setminus V(\Gamma^{\prime}) \\ y \in  V(\Gamma^{\prime\prime}) }}  J_{xy}  - \sum_{\substack{x \in V(\Gamma^{\prime}) \\ y \in  V(\Gamma^{\prime\prime}) }} 2J_{xy}\mathbbm{1}_{\{\sigma_x\neq \sigma_y\}} -  \sum_{\substack{x \in \I_-(\gamma) \\ y \in  B(\gamma)\setminus V(\Gamma^{\prime\prime}) }}J_{xy}.
\end{multline}

We can bound the first two terms by
\begin{equation}\label{Eq: Aux_1_Interaction_between_interior_and_B}
 \sum_{\substack{x \in V(\Gamma^{\prime}) \\ y \in  V(\Gamma^{\prime\prime}) }} J_{xy} +  \sum_{\substack{x \in \I_-(\gamma)\setminus V(\Gamma^{\prime}) \\ y \in  V(\Gamma^{\prime\prime}) }} 2J_{xy}\mathbbm{1}_{\{\sigma_y=-1\}}  \leq  2\sum_{\substack{x \in \I_-(\gamma) \\ y \in V(\Gamma^{\prime\prime}) }} J_{xy} \leq 4\kappa^{(1)}_\alpha\frac{F_{\I_-(\gamma)}}{M^{(\alpha - d)\wedge 1}}.
 \end{equation}
In the second inequality we are using that $ \Gamma^{\prime\prime} =  \Gamma_{\Ext}(\sigma, \I_-(\gamma))\setminus \{\gamma\}$ so we can applying Corollary \ref{Cor: Corrolary_of_Lemma_aux}. For the next term, since $B(\gamma)\setminus V(\Gamma^{\prime\prime})\subset \I_-(\gamma)^c$, we can bound
\begin{equation}\label{Eq: Aux_2_Interaction_between_interior_and_B}
    \sum_{\substack{x \in V(\Gamma^{\prime}) \\ y \in  B(\gamma)\setminus V(\Gamma^{\prime\prime}) }} 2J_{xy}\mathbbm{1}_{\{\sigma_y=+1\}}\leq \sum_{\substack{x \in V(\Gamma^{\prime}) \\ y \in  \I_-(\gamma)^c}} 2J_{xy}\leq 2\kappa_\alpha^{(1)}\frac{F_{\I_-(\gamma)}}{M}.
\end{equation}
 In the last inequality we are again applying Corollary \ref{Cor: Corrolary_of_Lemma_aux}, since $\Gamma^{\prime} = \Gamma_{\Int}(\sigma, \I_-(\gamma))$.

For the negative terms in \eqref{Eq: Interaction_between_interior_and_B}, we bound the term containing $\mathbbm{1}_{\{\sigma_x\neq\sigma_x\}}$ by $0$ and control the remaining term using the second inequality of \eqref{Eq: Aux_1_Interaction_between_interior_and_B}, getting
\begin{align}\label{Eq: Aux_4_Interaction_between_interior_and_B}
  \sum_{\substack{x \in \I_-(\gamma)\setminus V(\Gamma^{\prime}) \\ y \in  V(\Gamma^{\prime\prime}) }} J_{xy} + \sum_{\substack{x \in \I_-(\gamma) \\ y \in  B(\gamma)\setminus V(\Gamma^{\prime\prime}) }} J_{xy} 
    %
    &= F_{\I_-(\gamma)} - \sum_{\substack{x \in V(\Gamma^{\prime}) \\ y \in  V(\Gamma^{\prime\prime})}} J_{xy} \nonumber \\
    %
    &\geq F_{\I_-(\gamma)} -  2\kappa^{(1)}_\alpha\frac{F_{\I_-(\gamma)}}{M^{(\alpha - d)\wedge 1}}.
\end{align}
 Plugging inequalities \eqref{Eq: Aux_1_Interaction_between_interior_and_B}, \eqref{Eq: Aux_2_Interaction_between_interior_and_B} and \eqref{Eq: Aux_4_Interaction_between_interior_and_B} back in \eqref{Eq: Interaction_between_interior_and_B} we get
 \begin{equation}\label{Eq: Aux_2_Diferenca_de_Hamiltonianos}
    \sum_{\substack{x \in \I_-(\gamma) \\ y \in B(\gamma)}} J_{xy}\sigma_x\sigma_y \leq 6\kappa^{(1)}_\alpha\frac{F_{\I_-(\gamma)}}{M^{(\alpha - d)\wedge 1}} +  2\kappa_\alpha^{(1)}\frac{F_{\I_-(\gamma)}}{M}  \nonumber - F_{\I_-(\gamma)}.
\end{equation}

For the positive terms in \eqref{Eq: Difference_of_Hamiltonians_1}, we use the triangular inequality to get 
\begin{equation*}
    J_{xy} \geq \frac{1}{(2d+1)2^\alpha}\sum_{|x-x^\prime|\leq 1}J_{x^\prime y},
\end{equation*}
and therefore 
\begin{equation}\label{Eq: Aux_3_Diferenca_de_Hamiltonianos}
    \sum_{\substack{x \in \Sp(\gamma) \\ y \in \Z^d}} J_{xy}\mathbbm{1}_{ \{ \sigma_x \neq \sigma_y \}} +       \sum_{\substack{x \in \Sp(\gamma) \\ y \in \Sp(\gamma)^c}} J_{xy}\mathbbm{1}_{ \{ \sigma_x \neq \sigma_y \}} \geq \frac{1}{(2d+1)2^\alpha}\left(Jc_\alpha |\gamma| + F_{\Sp(\gamma)}\right),
\end{equation}
with $c_\alpha = \sum_{\substack{y\in\Z^d\setminus{0}}}|y|^{-\alpha}$. Plugging \eqref{Eq: Aux_1_Diferenca_de_Hamiltonianos}, \eqref{Eq: Aux_2_Diferenca_de_Hamiltonianos} and \eqref{Eq: Aux_3_Diferenca_de_Hamiltonianos} back in \eqref{Eq: Difference_of_Hamiltonians_1} we get
\begin{align*}
     H_\Lambda^+(\sigma) - H_\Lambda^+(\tau) &\geq \frac{1}{(2d+1)2^\alpha}\left(Jc_\alpha |\gamma| + F_{\Sp(\gamma)}\right) - 8\kappa^{(1)}_\alpha\frac{F_{\I_-(\gamma)}}{M^{(\alpha - d)\wedge 1}} + F_{\I_-(\gamma)} -2{\kappa^{(1)}_\alpha}\frac{F_{\Sp(\gamma)}}{M^{(\alpha - d)\wedge 1}}\\
     %
     &\geq \frac{Jc_\alpha }{(2d+1)2^\alpha} |\gamma| + \left(1 -  \frac{8\kappa^{(1)}}{M^{(\alpha - d)\wedge 1}}\right)F_{\I_-(\gamma)} + \left( \frac{1 }{(2d+1)2^\alpha} -  \frac{2\kappa^{(1)}_\alpha}{M^{(\alpha - d)\wedge 1}}\right)F_{\Sp(\gamma)},
\end{align*}
what proves the proposition for $M>\max{\{8\kappa^{(1)}, 2^{\alpha + 1}(2d+1)\kappa^{(1)}\}}$. 
\end{proof}

    

\section{Ding and Zhuang approach}  The main idea used in Ding and Zhuang's proof of phase transition in \cite{Ding2021} is to make the Peierls' argument on the joint space of the configurations and the external field, and when erasing a contour, perform in the external field the same flips you do in the configuration. Doing this, the part on the Hamiltonian that depends on the external field does not change, but the partition function does. The complication of this method is to control such difference.

In the short-range case, the spins that need to be flipped to erase a contour are precisely the ones in the interior of it. This is not the case for the long-range model, so we make a slight modification in the argument, and instead of performing the same flips in both spaces, we flip the external field only on $\I_-(\gamma)$. Doing this, not only does the partition function change but we also get an extra cost when comparing the original energy with the energy after performing such transformation. The extra term depends only on the external field in $\Sp{(\gamma)}$.  

In this section, we define the measure in the joint space and show that, with high probability, both the change of partition function and the extra energy cost resulting from such flipping are upper-bounded by the size of the support $|\gamma|$. 
    \subsection{Joint measure and bad events}       Given $\Lambda\subset\Z^d$, define the local joint measure for $(\sigma, h)$ as
\begin{equation*}
    \mathbb{Q}_{\Lambda; \beta, \varepsilon}^+(\sigma \in A, h\in B) = \int_{B} \mu_{\Lambda;\beta, \varepsilon h}^+(A) d\mathbb{P}(h),
\end{equation*}
for $A\subset\Omega$ measurable and $B\subset \mathbb{R}^{\Lambda}$ borelian. Since $\beta, \varepsilon $ and $\Lambda$ are fixed, we will omit then from the notation. 
This measure $\mathbb{Q}$ has density
\begin{equation*}
    g_{\Lambda; \beta, \varepsilon}^+(\sigma, h) = \prod_{u\in\Lambda}\frac{1}{\sqrt{2\pi}}e^{-\frac{1}{2}h_u^2} \times \mu_{\Lambda;\beta, \varepsilon h}^+(\sigma).
\end{equation*}

The main idea used in the proof of phase transition in \cite{Ding2021} is to make the Peierls' argument on the measure $\mathbb{Q}$, and perform in the external field the same flips you do in the configuration when erasing a contour. Formally, in \cite{Ding2021} they compare the density $g_{\Lambda; \beta, \varepsilon}^+(\sigma, h)$ with the density after erasing a contour $\gamma\in\Gamma(\sigma)$, and performing the same flips on the external field, getting

\begin{align}\label{Eq: quotient.of.gs}
    \frac{g_{\Lambda; \beta, \varepsilon}^+(\sigma, h)}{g_{\Lambda; \beta, \varepsilon}^+(\tau_{\gamma}(\sigma),\tau_{\gamma}(h))} \nonumber
    %
    &= \exp{\{\beta H_{\Lambda, 0}^{+}(\tau_{\gamma}(\sigma)) - \beta H_{\Lambda, 0}^{+}(\sigma)\}}\frac{Z_{\Lambda; \beta, \varepsilon}^{+}(\tau_{\gamma}(h))}{Z_{\Lambda; \beta, \varepsilon}^{+}(h)}.  \nonumber \\ 
\end{align}

For some realizations of the external field, the quotient of the partition functions can be bigger than the exponential term. Denoting
\begin{equation}
\Delta_A(h) \coloneqq -\frac{1}{\beta}\ln{\frac{Z_{\Lambda; \beta, \varepsilon}^{+}(h)}{Z_{\Lambda; \beta, \varepsilon}^{+}(\tau_{A}(h))}}
\end{equation}
 for every $A\subset \Z^d$, the bad event is
$$\mathcal{E}^c\coloneqq \left\{\sup_{\substack{\gamma\in\Gamma_0}} \frac{|\Delta_{\I(\gamma)}(h)|}{c_1|\gamma|} > \frac{1}{4}\right\}.$$
To control the probability of this bad event, we need a concentration result for Gaussian random variables. The following one is due to M. Ledoux and M. Talagrand, and a proof can be found in \cite{Ledoux.Talagrand.91}.

\begin{theorem}\label{Theo: Gaussian.concentration}
    Let $f:\mathbb{R}^M \xrightarrow[]{} \mathbb{R}$ be a uniform Lipschitz continuous function with constant $C_{Lip}$, that is, for any $X,Y\in\mathbb{R}^M$, $$|f(X) - f(Y)| \leq C_{Lip} || X - Y ||_2 .$$ 
    
    Then, if $X_1,\dots, X_M$ are i.i.d. Gaussian random variables with variance 1,
    \begin{equation}\label{Eq: Tail.concentration}
        \mathbb{P}\left(|f(X_1,\dots, X_M) - \mathbb{E}(f(X_1, \dots, X_M))|\geq z\right) \leq 2\exp{\left\{\frac{-z^2}{2C_{Lip}^2}\right\}}.
    \end{equation}
\end{theorem}

\begin{remark}\label{Rmk: MVT.Lipschitz}
    If $f$ is differentiable and $||\nabla f(\cdot)||_2$ is bounded, the mean value theorem guarantees that $\sup_{Z\in\mathbb{R}^M}||\nabla f(Z)||_2$ is a uniform Lipschitz constant for $f$. 
\end{remark}
\begin{remark}
    If $f$ has a compact support and convex level sets, an equation similar to \eqref{Eq: Tail.concentration} holds, with some adjustments on the constants and replacing the mean by the median, see \cite[Theorem 7.1.3]{Bovier.06}. Therefore, our results hold when $h_i$ has a Bernoulli distribution $\mathbb{P}(h_i=+1) =\mathbb{P}(h_i=-1)= \frac{1}{2}$. 
\end{remark}

Given $A\subset\Z^d$, $h_A\coloneqq (h_x)_{x\in A}$ denotes the restriction of the external field to the subset $A$. The next Lemma was proved in \cite{Ding2021} and is a direct consequence of the previous theorem. 

\begin{lemma}\label{Lemma: Concentration.for.Delta.General}
    For any $A, A^\prime \Subset \mathbb{Z}^d$ and $\lambda>0$, we have 
\begin{equation}\label{Eq: Tail.of.Delta_A}
    \mathbb{P}\left(|\Delta_A(h)| \geq \lambda \vert h_{A^c}\right) \leq2e^{\frac{-\lambda^2}{8\varepsilon^2 |A|}},
\end{equation}
and 
\begin{equation}\label{Eq: Tail.of.the.diff.of.Deltas}
     \mathbb{P}(|\Delta_{A}(h) - \Delta_{A^\prime}(h)|>\lambda|h_{{A \cup A^\prime}^c}) \leq  2e^{-\frac{{\lambda^2}}{{8\varepsilon^2|A \Delta A^\prime|}}},
\end{equation}
where $A\Delta A^\prime$ is the symmetric difference.
\end{lemma}

\begin{remark}\label{Rmk: More.general.h_x}
    This lemma holds whenever $h=(h_x)_{x\in\Z^d}$ satisfy equation \eqref{Eq: Tail.concentration}.As a consequence, our results can be stated for more general external fields. 
\end{remark}

    \subsection{Probability Results}   To control the probability of $\mathcal{E}_1^c$, we use some results on Majorising measure. For an extensive overview, we refer to \cite{Talagrand_14}. Consider $(T,d)$ a metric space and a process $(X_t)_{t\in T}$ such that, for every $\lambda>0$ and $t,s\in T$,
\begin{equation}\label{Eq: Sub_gaussian_def}
    \mathbb{P}\left( |X_t - X_s| \geq \lambda \right) \leq 2\exp{\frac{-\lambda^2}{2d(s,t)^2}}.
\end{equation}
Assume also that $\mathbb{E}\left(X_t\right) = 0$ for every $t\in T$. One example of such process is $( |\Delta_{\I_-(\gamma)}|)_{\gamma\in\mathcal{C}_0}$ with the distance $\d_2(A,A^\prime) = 2\varepsilon |A\Delta A^\prime|^{\frac{1}{2}}$. For $n\in \mathbb{N}$, consider the quantities $N_n = 2^{2^n}$ and $N_0=1$. 

\begin{definition}
    Given a set $T$, a sequence $(\mathcal{A}_n)_{n\geq 0}$ of partitions of $T$ is \textit{admissible} when $|\mathcal{A}_n|\leq N_n$ and $\mathcal{A}_{n+1}\preceq \mathcal{A}_n$ for all $n\geq 0$.
\end{definition}

Given $t\in T$ and an admissible sequence $(\mathcal{A}_n)_{n\geq 0}$, $A_n(t)$ denotes the element of $\mathcal{A}_n$ that contains $t$. 

\begin{definition}
    Given $\theta \geq 0$ and a metric space $(T,d)$, we define
    \begin{equation*}
        \gamma_\theta(T,d) \coloneqq \inf_{(\mathcal{A}_n)_{n\geq 0}}\sup_{t\in T}\sum_{n\geq 0}2^{\frac{n}{\theta}}\diam(A_n(t)),
    \end{equation*}
where the infimum is taken over all admissible sequences. 
\end{definition}

\begin{theorem}[Majorizing measure theorem] There is a universal constant $L>0$ such that
\begin{equation*}
    \frac{1}{L}\gamma_2(T,d) \leq \mathbb{E}\left( \sup_{t\in T} X_t \right) \leq L\gamma_2(T,d).
\end{equation*}
\end{theorem}

    Given $\epsilon>0$, let $N(T,\d, \epsilon)$ be the minimal number of balls with radius $\epsilon$ necessary to cover $T$, using the distance $d$. 
\begin{proposition}[Dudley's entropy bound \cite{Dudley67}]
    Let $(X_t)_{t\in T}$ be a family of random variables satisfying \eqref{Eq: Sub_gaussian_def} for some distance $\d$. Then there exists a constant $L>0$ such that 
    \begin{equation*}
        \mathbb{E}\left[\sup_{t\in T}X_t\right]\leq L\int_{0}^\infty \sqrt{\log N(T,\d,\epsilon)}d\epsilon.
    \end{equation*}
\end{proposition}

Dudley's entropy bound together with the Majorizing measure theorem yields that there is a constant $L^\prime>0$ such that,
\begin{equation}\label{Eq: gamma_2_bounded_by_Dudley_integral}   
\gamma_2(T,\d)\leq L^\prime\int_{0}^\infty \sqrt{\log N(T,\d,\epsilon)}d\epsilon.
\end{equation}
We only need one last result.
\begin{theorem}\label{Theo: Theo_2.2.27_Talagrand} Given a metric space $(T,\d)$ and a family $(X_t)_{t\in T}$ of centered random variables satisfying \eqref{Eq: Sub_gaussian_def}, there is a universal constant $L>0$ such that, for any $u>0$,
\begin{equation*}
\mathbb{P}\left( \sup_{t\in T}X_t > L(\gamma_2(T,\d) + u\diam(T)) \right)\leq e^{-{u^2}},
\end{equation*}
where the $\diam(T)$ is the diameter taken with respect to the distance $\d$
\end{theorem} 
A proof can be found in  \cite[Theorem 2.2.27]{Talagrand_14}. Using these results, the bound on the bad event $\mathcal{E}_1$ follows from the following proposition.
\begin{proposition}\label{Prop: Bound.gamma_2}
    Given $n,j\geq 0$ and $\alpha > d$, there are large enough constants $L_1>0$  such that, for $d \geq 3$, $$\gamma_2(\mathcal{C}_0(n,j),\d_2) \leq \varepsilon L_1 n.$$
\end{proposition}
As a direct consequence of this Proposition, we can control the probability of the bad event.
\begin{proposition}\label{Prop: Bound.bad.event.1}     
    There exists $C_1\coloneqq C_1(\alpha, d)$ such that $\mathbb{P}(\mathcal{E}^c)\leq e^{-\frac{C_1}{\varepsilon^2}}$. 
\end{proposition}

\begin{proof}
   Given $\gamma\in\mathcal{C}_0(n,j)$, by the construction of the contours and the isoperimetric inequality, we have $2^{r(d+1)(j-1)}\leq |V(\gamma)|\leq n^{1 + \frac{1}{d-1}}$. Taking $N\coloneqq \frac{d}{d^2-1}\log_{2^r}n + 1$, we have $j\leq N$ and the union bound yields
\begin{align}\label{Eq: Union_bound_bad_event}
    \mathbb{P}\left({\sup_{\substack{\gamma\in\mathcal{C}_0}} \frac{|\Delta_{\I_-(\gamma)}(h)|}{c_2|\gamma|} > \frac{1}{4}}\right) \leq \sum_{n=1}^\infty \sum_{j=1}^{N}\mathbb{P}\left({\sup_{\substack{\gamma\in\mathcal{C}_0(n,j)}} |\Delta_{\I_-(\gamma)}(h)| > \frac{c_2}{4}}|\gamma|\right). 
\end{align}
Let $\gamma,\gamma^\prime\in \mathcal{C}_0(n,j)$ two contours satisfying $\diam(\mathcal{C}_0(n,j)) = \d_2(\gamma,\gamma^\prime)$, where the diameter is in the $\d_2$ distance. By the isoperimetric inequality,
\begin{equation*}
    \diam(\mathcal{C}_0(n,j))\leq 2\varepsilon{|\I_-(\gamma)\Delta \I_-(\gamma^\prime)|}^{\frac{1}{2}} = 2\sqrt{2}\varepsilon n^{(\frac{d}{d-1})\frac{1}{2}} = 2\sqrt{2}\varepsilon n^{(\frac{1}{2} + \frac{1}{2(d-1)})}.
\end{equation*}
Together with Proposition \ref{Prop: Bound.gamma_2}, this yields
\begin{align*}
  \frac{c_2}{4}|\gamma| &= L\left[\varepsilon L_1 n + \varepsilon L_1 \left(\frac{c_2}{4\varepsilon L_1 L} - 1\right)n\right]\\
    %
    &\geq  L\left[\gamma_2(\mathcal{C}_0(n,j),\d_2) +  \frac{L_1}{2\sqrt{2}} \left(\frac{c_2}{4\varepsilon L_1 L} - 1\right)n^{\frac{1}{2} - \frac{1}{2(d-1)}}\diam(\mathcal{C}_0(n,j))\right]\\
    %
    &\geq   L\left[\gamma_2(\mathcal{C}_0(n,j),\d_2) +  \frac{C_1^\prime}{\varepsilon}n^{\frac{1}{2} - \frac{1}{2(d-1)}}\diam(\mathcal{C}_0(n,j))\right],
\end{align*}
with $C_1^\prime = \frac{c_2}{16\sqrt{2}L_1L}$ and $\varepsilon> \frac{c_2}{8L_1L}$. Applying Theorem \ref{Theo: Theo_2.2.27_Talagrand} with $u = \frac{C_1^\prime}{\varepsilon}n^{\frac{1}{2} - \frac{1}{2(d-1)}}$, we have
\begin{align*}
    \mathbb{P}\left({\sup_{\substack{\gamma\in\mathcal{C}_0(n,j)}} |\Delta_{\I_-(\gamma)}(h)| > \frac{c_2}{4}}|\gamma|\right) &\leq \mathbb{P}\left({\sup_{\substack{\gamma\in\mathcal{C}_0(n,j)}} |\Delta_{\I_-(\gamma)}(h)| >    L\left[\gamma_2(\mathcal{C}_0(n,j),\d_2) +  \frac{C_1^\prime}{\varepsilon}n^{\frac{1}{2} - \frac{1}{2(d-1)}}\diam(\mathcal{C}_0(n,j))\right]}\right) \\ 
    %
    &\leq \exp{\left\{ - \frac{C_1^{\prime2}n^{1 - \frac{1}{(d-1)}}}{\varepsilon^2}\right\}} 
\end{align*}
Using this back in equation \eqref{Eq: Union_bound_bad_event}, we conclude that 
\begin{equation*}
       \mathbb{P}\left({\sup_{\substack{\gamma\in\mathcal{C}_0}} \frac{|\Delta_{\I_-(\gamma)}(h)|}{c_2|\gamma|} > \frac{1}{4}}\right) \leq \sum_{n=1}^\infty  \left(\frac{d}{d^2-1}\log_{2^r}n +1\right) \exp{\left\{ - \frac{C_1^{\prime2}n^{1 - \frac{1}{(d-1)}}}{\varepsilon^2}\right\}} \leq e^{-\frac{C_1}{\varepsilon^2}},
\end{equation*}
for a suitable constant $C_1> \frac{c_2}{8L_1L}$ and $\varepsilon> C_1$.
\end{proof}
The rest of this section is dedicated to proving Proposition \ref{Prop: Bound.gamma_2}.
    \subsection{Coarse-graining and entropy bounds} We will apply this probability estimations for the family $(|\Delta_{\I_-(\gamma)}|)_{\gamma\in\mathcal{C}_0(n,j)}$. To construct the covering by balls in Dudley's entropy bound, we use the coarse-graining idea introduced in \cite{FFS84}.  For each $\ell>0$ and each contour ${\gamma\in\mathcal{C}(n,j)}$, we will associate a region $B_\ell(\gamma)$ that approximates the interior $\I(\gamma)$ in a scaled lattice, with the scale growing with $\ell$. This is done in a way that two contours that are approximated by the same region are in a ball in distance $\d_2$ with a fixed radius, depending on $\ell$.

An $r\ell$-cube $C_{r\ell}$ is \textit{admissible} if more than a  half of its points are inside $\I_-(\gamma)$. Thus, the set of admissible cubes is
\begin{equation*}
    \mathfrak{C}_\ell(\gamma) \coloneqq \{C_{r\ell} : |C_{r\ell}\cap \I_-(\gamma)| \geq \frac{1}{2}|C_{r\ell}|\}.
\end{equation*}
With this notion of admissibility, two contours with the same admissible cubes should be close in distance $d_2$. Consider functions $B_\ell:\mathcal{E}^+_\Lambda \xrightarrow[]{} \mathcal{P}(\Lambda)$ that takes contours $\gamma$ to $B_\ell(\gamma) \coloneqq B_{\mathfrak{C}_{\ell}(\gamma)}$, the region covered by the admissible cubes. We will be interest in counting  the image of $B_\ell$ by $\mathcal{C}_0(n,m,j)$, that is, $|B_\ell(\mathcal{C}_{0}(n,m,j))| = |\{ B : B = B_\ell(\gamma) \text{ for some }\gamma \in \mathcal{C}_0(n,m,j)\}|$. Notice that $B_\ell(\gamma)$ is uniquely determined by $\partial B_\ell(\gamma)$. Given any collection $\C_{m}$, we define the \textit{edge boundary of } $\C_m$ as 
$$
\partial \C_m(\gamma) \coloneqq \{ \{C_{m}, C^\prime_{m}\} : C_{m} \in \C_m, \ C_m^\prime \notin \C_m \textrm{ and} \  C_m^\prime \text{ shares a face with }C_m\}.
$$ 
We also define the \textit{inner boundary of }$\C_m$ as
$$
\fint \C_m(\gamma) \coloneqq \{ C_{m}\in \C_m : \exists C_m^\prime \notin \C_m \textrm{ such that }  \{C_m,C_m^\prime\}\in\partial \C_m\}.
$$ 
With this definition, it is clear that $\partial B_\ell(\gamma)$ is uniquely determined by $\partial \mathfrak{C}_\ell(\gamma)$. We will now control the number of cubes in $\mathfrak{C}_\ell(\gamma)$. This proposition was written for $d=3$ and $\I_-(\gamma)$ simply connected for all contours, but it can be extended to $d\geq 2$ with no restriction on the interiors, see \cite{Bovier.06}. As we could not find a detailed proof anywhere, we provide one here.

Given a rectangle $\mathcal{R} = [1,r_1]\times[1,r_2]\times\dots\times[1,r_d]$, consider $\R_i\coloneqq\{x\in \R : x_i=1\}$ the face of $\R$ that is perpendicular to the direction $e_i$, for $i=1,\dots,d$. The line that connects a point $x\in \R_i$ to a point in the opposite face of $\R_i$ is $\ell_x^i \coloneqq \{ x + ke_i : 1\leq k\leq r_i\}$. Given $A\subset \Z^d$, the projection of $A\cap \R$ into the face $\R_i$ is
\begin{equation*}
    \calP_i(A\cap\R) \coloneqq \{x\in\R_i : \ell_x^i \cap A \neq \emptyset\}.
\end{equation*}
\input{Figure_4.1}
In many situations, we will split the projections into \textit{good} and \textit{bad} points. The set of good points is $\calP_i(A\cap\R)^{G} \coloneqq \{x\in \calP_i(A\cap \R) : \ell_x^i \cap (\R\setminus A) \neq \emptyset\}$, that is, there exist a point in $\ell_x^i\cap \R$ that is not in $A$.  The bad points are defined as $\calP^{B}_i(A\cap\R) \coloneqq \calP_i(A\cap\R)\setminus \calP_i^G(A\cap\R)$.
\input{Figure_5}
Given $x\in \calP_i(A\cap\R)^{G}$, by definition of the projection, there exists a point in $\ell_x^i\cap A$. Therefore, there exists a point $p\in \ell_x^i$ such that $p\in\fext A \cap \R$. As all lines are disjoint, we conclude that 
\begin{equation}\label{Eq: upper.bound.good.points}
     |\calP_i^{G}(A\cap\R)|\leq |\fext A \cap \R|.
\end{equation}
 We now prove two auxiliary lemmas.
 
\begin{lemma}\label{Lemma: Geo.discreta.1}
    Given $d\geq 2$, for any family of positive integers $\bm{r}=(r_i)_{i=1}^d$ with $R\leq r_i \leq 2R$ for some $R\geq 2$, $0<\lambda < 1$ and $A\subset\Z^d$, there exists a constant $c\coloneqq c(d, \lambda)$ such that, if 
    \begin{equation}\label{Eq: hypothesis.lemma.1}
         |\calP_i(A\cap \R)| \leq \lambda|\R_i|
    \end{equation}
    for all $i= 1,\dots, d$, then 
    \begin{equation*}
        \sum_{i=1}^d |\calP_{i}(A\cap \R)|\leq c|\fext A\cap \R|,
    \end{equation*}
    where $\R=[1,r_1]\times\dots\times [1,r_d]$.
\end{lemma}

\begin{proof}
The proof will be done by induction on the dimension. For $d=2$, take a rectangle ${\R=[1,r_1]\times[1,r_2]}$. If there is no bad points in $\calP_1(A\cap\R)$, then 
\begin{align}\label{Eq: Bound.1.on.P.1}
    |\calP_1(A\cap\R)| = |\calP_1^G(A\cap \R)| \leq |\fext A \cap \R|.
\end{align}

If there is a bad point $p=(1,p_2)\in \calP_1^B(A\cap\R)$, $\ell_p^1\subset A\cap \R$  by definition of bad point. As $|\calP_1(A\cap \R)| \leq \lambda|\R_1| < |\R_1|$, there is a point $p^\prime = (1,p_2^\prime)\in \R_1\setminus \calP_1(A\cap \R)$ that is in the face $\R_1$ but not in the projection. By definition of the projection, $\ell_{p^\prime}^1\in A^c\cap \R$. Therefore, for any $1\leq k\leq r_1$, $(k,p_2)\in  A\cap \R$ and $(k,p^\prime_2)\in  A^c\cap \R$, we can find a point $p^k=(k, p^k_2) \in \fext A \cap \R$. Since $p^{k_1}\neq p^{k_2}$ for every $k_1\neq k_2$, we have $r_1 \leq |\fext A \cap \R|$, hence
\begin{equation}\label{Eq: Bound.2.on.P.1}
   |P_1(A\cap \R)| \leq  |\R_1| = {r_2}\leq  2R \leq 2r_1 \leq  2|\fext A \cap \R|.
\end{equation}
A completely analogous argument can be done to bound $|P_2(A\cap \R)|$, and we conclude that
\begin{equation*}
    \sum_{i=1}^2|\calP_i(A\cap \R)|\leq 4|\fext A \cap \R|,
\end{equation*}
and take $c(2,\lambda)=4$. Suppose the lemma holds for $d-1$ and fix a rectangle $\R=[1,r_1]\times\dots\times[1,r_d]$. We split $\R$ into layers $L_k = \{x\in\Z^d : x_d = k\}$, for $k=1,\dots, r_d$. We can then partition the projection and write
\begin{equation*}
|\calP_i(A\cap \R)| = \sum_{k=1}^{r_d} |\calP_i(A\cap \R)\cap L_k|,    
\end{equation*}
for any $i\in\{1,\dots, d-1\}$. This yields
\begin{align}\label{Eq: Partition.sum.proj.}
    \sum_{i=1}^d|\calP_i(A\cap \R)| &= \sum_{i=1}^{d-1}\sum_{k=1}^{r_d}|\calP_i(A\cap \R)\cap L_k| + |\calP_d(A\cap \R)| \nonumber \\
    &=  \sum_{k=1}^{r_d}\sum_{i=1}^{d-1}|\calP_i(A\cap \R)\cap L_k| + |\calP_d(A\cap \R)|.
\end{align}

Notice now that $\calP_i(A\cap \R)\cap L_k = \calP_i(A\cap (\R\cap L_k))$. Defining the rectangle $\R^k \coloneqq \R\cap L_k$, for every point $p\in \calP_j^B(A\cap \R^k)$, $\ell_p^j \subset A\cap \R^k$. Moreover, we can associate every point $x\in \ell_p^j$ in the line with a point $x^\prime\in \calP_d(A\cap\R)$ by taking $x_m^\prime = x_m$ for $m \leq d-1$ and $x_d^\prime = 1$, therefore

\begin{equation*}
    r_j|\calP_j^B(A\cap \R^k)| = \sum_{p\in \calP_j^B(A\cap \R^k)}|\ell_p^j| \leq |\calP_d(A\cap\R)|.
\end{equation*}
\input{Figure_6}
Using the hypothesis \eqref{Eq: hypothesis.lemma.1} we conclude that

\begin{equation}\label{Eq: upper.bound.projection.i.bad.points}
     |\calP_j^B(A\cap \R^k)| \leq \lambda\frac{|\R_d|}{r_j} = \lambda \frac{ \prod_{q\neq d}r_q}{r_j} =  \lambda \prod_{q\neq j,d}r_q = \lambda |(\R^k)_j|.
\end{equation}
We consider two cases:
    \begin{itemize}
        \item[(a)] If $|\calP_i(A\cap \R^k)| \leq \frac{\lambda +1}{2}|(\R^k)_i|$, for all $i\leq d-1$, then we are in the hypothesis of the lemma in $d-1$ and therefore
\begin{equation}\label{Eq: Primeiro.bound.soma.projecoes}
    \sum_{i=1}^{d-1} |\calP_i(A\cap \R^k)| \leq c\left(d-1, \frac{\lambda + 1}{2}\right)|\fext A\cap \R^k|.
\end{equation}
    \item [(b)] If there exists $j\in\{1,\dots,d-1\}$ satisfying $|\calP_j(A\cap \R^k)| > \frac{\lambda +1}{2}|(\R^k)_j|$, by \eqref{Eq: upper.bound.projection.i.bad.points} we have $|\calP_j^G(A\cap \R^k)| = |\calP_j(A\cap \R^k)| - |\calP_j^B(A\cap \R^k)| \geq \frac{1-\lambda}{2}|(\R^k)_j|$, hence
\begin{equation*}
    |(\R^k)_j| \leq  \frac{2}{1 - \lambda}|\fext A \cap \R^k|.
\end{equation*}
    Using that $|(\R^k)_i| \leq (2R)^{d-2} \leq 2^{d-2}|(\R^k)_j|$ for every $i\in\{1,\dots,d\}$, we conclude that 
\begin{equation}\label{Eq: Segundo.bound.soma.projecoes}
    \sum_{i=1}^{d-1} |P_i(A\cap\R^k)| \leq \sum_{i=1}^{d-1} |(\R^k)_i| \leq (d-1)2^{d-2}|(\R^k)_j| \leq \frac{(d-1)2^{d-1}}{1 - \lambda}|\fext A \cap \R^k|.
\end{equation}
\end{itemize}


In both cases, we were able to bound the sum of projections by a constant times the size of the boundary of $A$ in $\R^k$. Applying \eqref{Eq: Primeiro.bound.soma.projecoes} and \eqref{Eq: Segundo.bound.soma.projecoes} back in \eqref{Eq: Partition.sum.proj.} we get
\begin{align*}
    \sum_{i=1}^d|\calP_i(A\cap \R)| &\leq
    \sum_{k=1}^{r_d}\left[c\left(d-1, \frac{\lambda + 1}{2}\right)+ \frac{(d-1)2^{d-1}}{1 - \lambda}\right]|\fext A \cap \R \cap L_k| + |\calP_d(A\cap \R)|\\
    %
    &=\left[c\left(d-1, \frac{\lambda + 1}{2}\right)+ \frac{(d-1)2^{d-1}}{1 - \lambda}\right]|\fext A \cap \R| + |\calP_d(A\cap \R)|.
\end{align*}
We finish the proof by noticing that we can repeat this same argument but now splitting $\R$ into layers $L_k = \{x\in \R : x_j = k\}$. Doing so, we have that  
\begin{equation*}
    \sum_{i=1}^d|\calP_i(A\cap \R)| \leq
     \left[c\left(d-1, \frac{\lambda + 1}{2}\right)+ \frac{(d-1)2^{d-1}}{1 - \lambda}\right]|\fext A \cap \R| + |\calP_j(A\cap \R)|
\end{equation*}
for any $j\in\{1,\dots, d\}$. Summing both sides in $j$ we conclude

\begin{equation}
    \sum_{i=1}^d|\calP_i(A\cap \R)| \leq \frac{d}{d-1}\left[c\left(d-1, \frac{\lambda + 1}{2}\right)+ \frac{(d-1)2^{d-1}}{1 - \lambda}\right]|\fext A \cap \R|,
\end{equation}
which proves our claim if we take $c(d,\lambda) \coloneqq \frac{d}{d-1}\left[c(d-1, \frac{\lambda + 1}{2})+ \frac{(d-1)2^{d-1}}{1 - \lambda}\right] = 2d + \frac{(d-2)d2^{d-1}}{1-\lambda}$.
\end{proof}

\begin{remark}
This lemma can be proved when $R\leq r_i \leq \kappa R$ for any $\kappa>1$. When applying the lemma, we will choose $\lambda=\frac{7}{8}$ to simplify the notation. All the proofs work as long as we choose $\lambda> \frac{3}{4}$.
\end{remark}

\begin{lemma}\label{Lemma: Proposicao1.Aux1}
    Given $A\subset \Z^d$, $\ell\geq 0$ and $U= C_{r\ell}\cup C_{r\ell}^\prime$ with $C_{r\ell}$ and $C_{r\ell}^\prime$ being two $r\ell$-cubes sharing a face, there exists a constant $b\coloneqq b(d)$ such that, if 
    
\begin{align}\label{Eq. U.condition}
    \frac{2^{r\ell d}}{2} \leq |C_{r\ell}\cap A| \qquad \text{and} \qquad |C_{r\ell}^\prime\cap A|< \frac{2^{r\ell d}}{2}
\end{align}
then $2^{r\ell(d-1)}\leq b|\fext A\cap U|$.
\end{lemma}


\begin{proof}
For $\ell=0$, \eqref{Eq. U.condition} guarantees that $C_{r\ell} = \{x\} \subset A$ and $C_{r\ell}^\prime = \{y\}\subset A^c$, hence $|\fext A\cap \{x,y\}| = 1$ and it is enough to take $b\geq 1$. For $\ell \geq 1$, \eqref{Eq. U.condition} yields
\begin{equation}\label{Eq. A.cap.U.volume}
    \frac{1}{2}2^{r\ell d} \leq |A\cap U| \leq \frac{3}{2}2^{r\ell d}.
\end{equation}

To simplify the notation, we can assume wlog that ${U=[1,2^{r\ell}]^{d-1}\times [1, 2^{r\ell+1}]}$. As discussed before, for each point $p\in\calP^B_j(A\cap U)$ in the projection, $\ell_p^j\subset A\cap U$ and the lines are disjoint. Moreover, the size of the lines  is constant $r_j\coloneqq |\ell_p^j|$, hence $|\calP_{j}^B(A\cap U)|r_j = \sum_{p\in\calP_{j}^B(A\cap U)} |\ell_p^j| \leq |A\cap U|$. Together with the upper bound \eqref{Eq. A.cap.U.volume}, this yields 
\begin{equation}\label{Eq: Upper.bound.bad.points}
    |\calP_{j}^B(A\cap U)| \leq \frac{3}{2}2^{r\ell d}r_j^{-1}.
\end{equation}
Using the isometric inequality, the lower bound on \eqref{Eq. A.cap.U.volume} yields $d2^{\frac{1}{d}}2^{r\ell(d-1)}\leq |\fext (A\cap U)|$. As %
%
\begin{align*}
\frac{1}{2d}|\fext (A\cap U)| &\leq |\fint (A\cap U)| = |\fint(A\cap U) \cap \fint U| + |\fint(A\cap U) \cap(U\setminus \fint U)|\\
%
&\leq 2\sum_{i=1}^d |\calP_{i}(A\cap U)| + |\fint A\cap U| \leq 2\sum_{i=1}^d |\calP_{i}(A\cap U)| + |\fext A\cap U|,
\end{align*}
we get
\begin{equation}\label{Eq: Lemma.geo.discreta.3}
    2^{\frac{1}{d}-1}2^{r\ell(d-1)}\leq 2\sum_{i=1}^d |\calP_{i}(A\cap U)| + |\fext A\cap U|
\end{equation}

We again consider two cases:
\begin{itemize}
    \item[(a)] If $|\calP_{j}(A\cap U)|> \frac{7}{8}|U_j|$ for some $j=1,\dots, d$, by \eqref{Eq: Upper.bound.bad.points} and \eqref{Eq: upper.bound.good.points} we get
    \begin{align*}
    \frac{7}{8}|U_j| < |\calP_{j}(A\cap U)| \leq |\fext A \cap U| + \frac{3}{2}2^{\ell d}r_j^{-1}.
    \end{align*}
    A simple calculation shows that $\frac{1}{8}2^{\ell(d-1)}\leq \frac{7}{8}|U_j| - \frac{3}{2}2^{\ell d}r_j^{-1}$, therefore 
    \begin{equation}\label{Eq: upper.bound.big.projections.1}
        \frac{1}{8}2^{\ell(d-1)} \leq |\fext A \cap U|.
    \end{equation}
    \item[(b)] If $|\calP_{i}(A\cap U)|\leq \frac{7}{8} |U_i|$ for all $i$, by Lemma \ref{Lemma: Geo.discreta.1}, there is a constant $c= c(d)$ such that
    \begin{equation}\label{Eq: Lemma.geo.discreta.2}
        \sum_{i=1}^d |\calP_{i}(A\cap U)|\leq c|\fext A\cap U|.
    \end{equation}
    Together with \eqref{Eq: Lemma.geo.discreta.3}, this yields
    \begin{equation}\label{Eq: upper.bound.big.projections.2}
        2^{\ell(d-1)}\leq \frac{2c+1}{2^{\frac{1}{d}-1}}|\fext A\cap U|.
    \end{equation}
\end{itemize}

Equations \eqref{Eq: upper.bound.big.projections.1} and \eqref{Eq: upper.bound.big.projections.2} shows the desired results taking $b\coloneqq \max \{8, {(2c+1)}{2^{1-\frac{1}{d}}}\}$.
\end{proof}

\begin{proposition}\label{Proposition1}For the functions $B_0,\dots,B_k$ defined above, there exists constants $b_1,b_2$ depending only on $d$ and $r$ such that 
\begin{equation}\label{Eq: Prop.1.FFS.i}
    |\partial\mathfrak{C}_\ell(\gamma)| \leq b_1\frac{|\fext \I_-(\gamma)|}{2^{r\ell(d-1)}} \leq b_1 \frac{|\gamma|}{2^{r\ell(d-1)}}
\end{equation}
    and 
\begin{equation}\label{Eq: Prop.1.FFS.ii}
    |B_\ell(\gamma)\Delta B_{\ell+1}(\gamma)| \leq b_2 2^{r\ell} |\gamma|
\end{equation}
for every $\ell\in\{0,\dots,k\}$ and $\gamma\in\mathcal{C}_0(n)$.
\end{proposition}

    \begin{proof} Fix $\ell\in\{0,\dots,k\}$. To each cube $C_{r\ell}\in  \partial \mathfrak{C}_\ell(\gamma)$ there is an $r\ell$-cube $C_{r\ell}^\prime \not\in  \mathfrak{C}_\ell(\gamma)$ not admissible, sharing a face with  $C_{r\ell}$. We denote this relation by $C_{r\ell} \sim C_{r\ell}^\prime$, and the union $U = C_{r\ell} \cup C_{r\ell}^\prime$.  Considering the collection of $r\ell$-cubes $\overline{\mathscr{C}}_{r\ell} = \{C_{r\ell} : C_{r\ell}\in \partial \mathfrak{C}_\ell(\gamma) \text{ or } C_{r\ell}\notin  \mathfrak{C}_\ell(\gamma)\}$ and $A\Subset\Z^d$, 
    
    \begin{align*}
        \sum_{\substack{C_{r\ell}\in \partial \mathfrak{C}_\ell(\gamma)}}\sum_{\substack{C_{r\ell}^\prime\notin  \mathfrak{C}_\ell(\gamma)\\ C_{r\ell} \sim C_{r\ell}^\prime}} |A \cap \{C_{r\ell} \cup C_{r\ell}^\prime\}| &\leq \sum_{\substack{C_{r\ell}\in \partial \mathfrak{C}_\ell(\gamma)}}\sum_{\substack{C_{r\ell}^\prime\notin  \mathfrak{C}_\ell(\gamma)\\ C_{r\ell} \sim C_{r\ell}^\prime}}  \left(|A \cap C_{r\ell}| + |A \cap C_{r\ell}^\prime|\right)\\
        %
        &\leq \sum_{\substack{C_{r\ell}\in \partial \mathfrak{C}_\ell(\gamma)}} 2d |A \cap C_{r\ell}| + \sum_{C_{r\ell}^\prime\notin \mathfrak{C}_\ell(\gamma)} 2d|A \cap C_{r\ell}^\prime| \\
        %
        &= 2d \sum_{C\in\overline{\mathscr{C}}_{r\ell}} |A\cap C| = 2d|A\cap B_{\overline{\mathscr{C}}_{r\ell}}| \leq 2d |A|
    \end{align*}
    
Any pair of cubes $C_{r\ell}\sim C_{r\ell}^\prime$ are in the hypothesis of Lemma \ref{Lemma: Proposicao1.Aux1}, hence $ b 2^{r\ell(d-1)} \leq |\fext \I_-(\gamma) \cap U|$. Applying equation above for $A=\fext\I_-(\gamma)$ we get that
    \begin{equation*}
       \frac{b}{2d} 2^{r\ell(d-1)}| \partial \mathfrak{C}_\ell(\gamma)| \leq \frac{1}{2d}\sum_{\substack{C_{r\ell}\in \partial \mathfrak{C}_\ell(\gamma)}}\sum_{\substack{C_{r\ell}^\prime\notin  \mathfrak{C}_\ell(\gamma)\\ C_{r\ell} \sim C_{r\ell}^\prime}}  |\fext \I_-(\gamma) \cap \{C_{r\ell} \cup C_{r\ell}^\prime\}| \leq |\fext \I_-(\gamma)|,
    \end{equation*}
that concludes \eqref{Eq: Prop.1.FFS.i} for $b_1\coloneqq 2d/b$.

Given $C_{r(\ell+1)}\in \mathscr{C}_{r(\ell+1)}(B_{\ell+1}(\gamma)\setminus B_{\ell}(\gamma))$, there is a $r\ell$-cube $C_{r\ell}^\prime\subset C_{r(\ell + 1)}$ with $C_{r\ell}^\prime\notin \mathfrak{C}_\ell(\gamma)$, otherwise  $(B_{\ell+1}(\gamma)\setminus B_{\ell}(\gamma))\cap C_{r(\ell+1)} = \emptyset$. There is also a $r\ell$-cube $C_{r\ell}\subset C_{r(\ell + 1)}$ with $C_{r\ell}\in \mathfrak{C}_\ell(\gamma)$, otherwise we would have 
\begin{align*}
    |\I_-(\gamma)\cap C_{r(\ell+1)}| &= \sum_{C_{r\ell}\subset C_{r(\ell+1)}} |\I_-(\gamma)\cap C_{r\ell}| \leq \frac{1}{2} |C_{r(\ell+1)}|.
\end{align*}

Moreover, we can assume that $C_{r\ell}$ and $C_{r\ell}^\prime$ share a face. Again, we use Lemma \ref{Lemma: Proposicao1.Aux1} to get,
\begin{align}\label{Eq: bound.on.c.bar}
    |B_{\ell+1}(\gamma)\setminus B_\ell(\gamma)\cap C_{r(\ell+1)}| &\leq |C_{r(\ell+1)}| =2^{rd}2^{r\ell}2^{r\ell(d-1)} \nonumber\\
                %
                &\leq 2^{rd}2^{r\ell} b|\fext \I_-(\gamma) \cap \{C_{r\ell} \cup C_{r\ell}^\prime\}| \nonumber\\
                %
                &\leq 2^{rd}2^{r\ell}b|\fext \I_-(\gamma) \cap C_{r(\ell+1)}|.
\end{align}
Therefore, 
\begin{align*}
    |B_{\ell+1}(\gamma)\setminus B_\ell(\gamma)| &= \sum_{C_{r(\ell+1)}\in \mathscr{C}_{r(\ell+1)}(B_{\ell+1}(\gamma)\setminus B_\ell(\gamma))} |B_{\ell+1}(\gamma)\setminus B_\ell(\gamma)\cap C_{r(\ell+1)}| \\
    %
    &\leq \sum_{C_{r(\ell+1)}\in \mathscr{C}_{r(\ell+1)}(B_{\ell+1}(\gamma)\setminus B_\ell(\gamma))} 2^{rd}2^{r\ell}b|\fext \I_-(\gamma) \cap C_{r(\ell+1)}| \leq  \frac{b_2}{2}2^{r\ell}|\fext \I_-(\gamma)|.
\end{align*}
with $b_2=b2^{rd+1}$. To get the same bound for $|B_{\ell}(\gamma)\setminus B_{\ell+1}(\gamma)|$ we repeat a similar argument, covering $B_{\ell}(\gamma)\setminus B_{\ell+1}(\gamma)$ with $r(\ell+1)$-cubes. 
    \end{proof}

\begin{corollary}
    For any $\ell>0$ and any two contours $\gamma_1,\gamma_2 \in \mathcal{C}_0(n)$ such that $B_\ell(\gamma_1)=B_{\ell}(\gamma_2)$, there exists a constant $b_3>0$ such that 
    \begin{equation*}
        \d_2(\gamma_1,\gamma_2)\leq 4 \varepsilon b_3 2^{\frac{\ell}{2}} n^{\frac{1}{2}}. 
    \end{equation*} 
\end{corollary}

\begin{proof}
    This is a simple application of the triangular inequality, since $d_2(\gamma_1,\gamma_2) \leq d_2(\gamma_1,B_\ell(\gamma_1)) + d_2(\gamma_2,B_\ell(\gamma_2))$ and 
    \begin{align*}
        d_2(\gamma_1,B_\ell(\gamma_1)) &\leq \sum_{i=1}^\ell d_2(B_i(\gamma_1),B_{i-1}(\gamma_1)) = \sum_{i=1}^\ell 2\varepsilon\sqrt{B_i(\gamma_1)\Delta B_{i-1}(\gamma_1)} \\
        %
        & \leq \sum_{i=1}^\ell 2\varepsilon{b_2}^{\frac{r}{2}} 2^{\frac{ri}{2}} \sqrt{n}  \leq 2\varepsilon{b_2}^{\frac{r}{2}}(2^{\frac{r}{2}}+1)2^{\frac{r\ell}{2}} \sqrt{n} 
    \end{align*}
    where in the second to last equation used \eqref{Eq: Prop.1.FFS.ii}. As the same bound holds for $d_2(\gamma_2,B_\ell(\gamma_2))$, the corollary is proved by taking ${b_3 = {b_2}^{\frac{r}{2}}(2^{\frac{r}{2}}+1)}$.
\end{proof}

\begin{remark}\label{Rmk: Bounding_N_by_B_ell}
    This corollary shows that we can create a covering of $\mathcal{C}_0(n)$, indexed by $B_\ell(\mathcal{C}_0(n))$, of balls with radius $4 \varepsilon b_3 2^{\frac{r\ell}{2}} n^{\frac{1}{2}}$. Therefore $N(\mathcal{C}_0(n), \d_2, 4\varepsilon b_3 2^{\frac{r\ell}{2}} n^{\frac{1}{2}}) \leq |B_\ell(\mathcal{C}_0(n))|$. 
\end{remark}
In the next section we we bound $|B_\ell(\mathcal{C}_0(n,j))|$, using a method similar to the one used in \cite{Affonso.2021} to count $|\mathcal{C}_0(n)|$.


    \subsection{Entropy Bounds} As we discussed before, in the definition of admissibility, $|B_\ell(\mathcal{C}_0(n,j))| = |\partial \mathfrak{C}_{r\ell}(\mathcal{C}_0(n,j))|$. In the short-range case, a key ingredient to count the admissible cubes is that despite $B_\ell(\gamma)$ not being connected, all cubes are inside a connected region with size $|\gamma|$. As the contours now are not connected, we need to change the strategy: we choose a suitable scale $L(\ell)$ and count how many $rL(\ell)$-coverings of $\gamma$ there are. That is, we first control $|\C_{rL(\ell)}(\mathcal{C}_0(n,j))|$. Once the $rL(\ell)$-covering is fixed, we choose which $r\ell$-cubes inside this covering will be admissible. At last, we choose the scale $L(\ell)$ in a suitable way. This has to be done with some care, since the behavior of $|\C_{rL(\ell)}(\mathcal{C}_0(n,j))|$ depends if $L(\ell)<j$ or not.

The first step is to bound $|\C_{rL}(\mathcal{C}_0(n,j))|$, for $L>0$. For $n,m\geq 0$, we say that $\mathscr{C}_n$ is \textit{subordinated} to $\C_m$, denoted by $\C_n\preceq \C_m$, if $\C_m = \C_m(B_{\C_n})$. Moreover, define 
\begin{equation*}
    N(\C_m, n, V) \coloneqq\{\C_n : \C_n\preceq \C_m, |\C_n|=V\},
\end{equation*}
the number of collections of $n$-cubes $\C_n$ subordinated to a fixed collection $\C_m$ and with $|\C_n|=V$. Notice that every $m$-cube contains $2^d$ $(m-1)$-cubes, all of them being disjoint. Therefore, the number of $n$-cubes inside a $m$-cube is $2^{(m-n)d}$ and we have $N(\C_m, n, V) = \binom{2^{(m-n)d} |\C_{m}|}{V}$.
In particular, the bound on the binomial $\binom{n}{k}\leq \left(\frac{en}{k}\right)^k$ yields
\begin{equation}\label{Eq: Bound.on.N}
    N(\C_{r(\ell+1)}, r\ell, V) = \binom{2^{rd}|\C_{r(\ell+1)}|}{V} \leq \left(\frac{2^{rd}e|\C_{r(\ell+1)}|}{V}\right)^{V}.
\end{equation}
For any subset $\Lambda \Subset \Z^d$, define
\begin{equation*}
    V_r^\ell(\Lambda)\coloneqq \sum_{n=\ell}^{n_r(\Lambda)} |\C_{rn}(\Lambda)|,
\end{equation*}
where $n_r(\Lambda)\coloneqq \ceil{\log_{2^r}(\diam (\Lambda))}$. To control $V_r^\ell$ we control the number of coverings at a fixed step $L>0$.

\begin{proposition}\label{Prop. partition.a.graph}
Let $k\geq 1$ and $G$ be a finite, non-empty, connected simple graph with vertex set $v(G)$. Then, $G$ can be covered by $\ceil*{|v(G)|/k}$ connected sub-graphs of size at most $2k$.
\end{proposition}
We omit the proof since it is the same as in \cite{Affonso.2021}. Remember that, for $A\Subset \Z^d$ and $j\geq 1$, $\Gamma^r_j(A)$ are the elements of the partition removed at step $j$, in the construction presented in Section 2. Using this construction we can prove the following lemma.

\begin{lemma}\label{Lemma: Big.clusters_2}
    Let $A\Subset \Z^d$, $\gamma\in\Gamma^r(A)$ and $j \geq 1$ be such that $\gamma\in \Gamma^r_j(A)$. Then, for any $\ell < j$ and $G_{r\ell}\in \mathscr{G}_{r\ell}(\gamma)$,
    \begin{equation}\label{Eq: Lower_bound_on_the_covering_of_gamma_G}
        2^{r(1-\frac{1}{d})\ell} \leq |\C_{r\ell}(\gamma_{G_{r\ell}})| 
    \end{equation}
\end{lemma}
\begin{proof}
        Given $G_{r\ell}\in \mathscr{G}_{r\ell}(\gamma)$, by our construction of the contour, $2^{r(d+1)\ell}\leq |V(\gamma_{G_{r\ell}})|$. A trivial bound gives us $|V(\gamma_{G_{r\ell}})| \leq 2^{r\ell d}|\C_{r\ell}(V(\gamma_{G_{r\ell}}))|$. Associating each cube $C_m(x)$ to its center $x$, we get a one-to-one correspondence between $m$-cubes and lattice points that preserves neighbors, that is, two m-cubes $C_m(x)$ and $C_m(y)$ share a face if and only if $|x-y|=1$. We can therefore apply the isoperimetric inequality to get $|\C_{r\ell}(V(\gamma_{G_{r\ell}}))| \leq |\fint \C_{r\ell}(V(\gamma_{G_{r\ell}}))|^{\frac{d}{d-1}}\leq |\C_{r\ell}(\gamma_{G_{r\ell}})|^{\frac{d}{d-1}}$, where in the last equation we are using that every cube in the boundary of cubes must cover at least one point of $\gamma_{G_{r\ell}}$. We conclude that $2^{r(d+1)\ell} \leq 2^{r\ell d}|\C_{r\ell}(\gamma_{G_{r\ell}})|^{\frac{d}{d-1}}$, and \eqref{Eq: Lower_bound_on_the_covering_of_gamma_G} follows.
\end{proof}

As a corollary, we can recuperate a key lemma of \cite{Affonso.2021}, which is the following.
\begin{lemma}\label{Lemma: Big.clusters_1}
    Given $A\Subset \Z^d$, $n> 1$ and $\gamma\in\Gamma(A)$, if $|\mathscr{G}_{rn}(\gamma)|\geq 2$ then $|v(G_{rn}(\gamma))| \geq 2^r$ for every $G_{rn}(\gamma)\in \mathscr{G}_{rn}(\gamma)$ 
\end{lemma}


   The next proposition bounds the partial volume.
\begin{proposition}\label{Prop. Bound.on.V_r^l(gamma)}
    There exists a constant $b_3 \coloneqq b_3(d, M, r)$ such that, for any $A\Subset \Z^d$, $\gamma\in\Gamma(A)$ and $0 \leq \ell$,
    
     \begin{equation*}
        V_r^\ell(\gamma)\leq b_3 (\ell\wedge 1)^{\frac{r-d-1}{\log_2(a)}} |\mathscr{C}_{r\ell}|. 
    \end{equation*}

\end{proposition}

    
\begin{proof}
Fix $\xi\in \Gamma(A)$ with $\xi=B_{\C_{r\ell}}$. Let's first assume $\ell\geq 2$. Define $g : \mathbb{N} \xrightarrow{} \Z$ by
\begin{equation}
    g(n)\coloneqq \floor*{\frac{n - 2 - \log_{2^r}(2M)}{a}}.
\end{equation}
It was proved in \cite[Proposition 3.13]{Affonso.2021} that 
\begin{equation}\label{Eq: Bound_c_n_by_C_g(n)}
    |\C_{rn}(\xi)| \leq \frac{1}{2^{r-d-1}}|\C_{rg(n)}(\xi)|,
\end{equation}
whenever $g(n)>0$, and every connected component of $G_{rg(n)}(\xi)$ has more than $2^r -1$ vertices. This is equivalent,  by Lemma \ref{Lemma: Big.clusters_1}, to $|\mathscr{G}_{rg(n)}(\xi)|\geq 2$ or $|\mathscr{G}_{rg(n)}(\xi)|=1$ with $|v(G_{rg(n)}(\xi))| \geq 2^r$. Consider then the auxiliary quantities
\begin{align*}
    &l_1(n)\coloneqq\max\{m : g^m(n)\geq \ell\} &\text{and} &&l_2(n)\coloneqq\max\{ m : |\mathscr{G}_{rg^m(n)}(\xi)| = 1 \text{ and } |v(G_{rg^m(n)})|\leq 2^r-1\}.
\end{align*}
Notice that, to cover a $r\ell$-cube, we need $2^{rd}$ $r(\ell-1)$-cubes. As $\xi$ is the region covered by $r\ell$-cubes, this implies that we need at least $2^{rd}$ $r(\ell-1)$-cubes to cover $\xi$, hence $\ell-1 < g^{l_2(n)}(n)$ and therefore $l_2(n)\leq l_1(n)$. Knowing that $|\C_{k}(\xi)|\leq |\C_j(\xi)|$, for all $j\leq k$, we get 
\begin{equation}\label{Eq: bound.on.rn.covering}
    |\C_{rn}(\xi)|\leq |\C_{rg^{l_2(n)}(n)}(\xi)|\leq \frac{1}{2^{(r-d-1)(l_1(n)-l_2(n))}}|\C_{r\ell}(\xi)|.
\end{equation}
We claim that
\begin{equation}\label{Eq: lower.bound.on.l1}
    l_1(n) \geq \begin{cases}
                        0, &\text{ if }n\leq \overline{b}+\ell\\ 
                        \left\lfloor\frac{\log_2(n) - \log_2(\overline{b} + \ell)}{\log_2(a)}\right\rfloor, & \text{ if }n > \overline{b} + \ell, 
                \end{cases}
\end{equation}
where $\overline{b} = (a+2 + \log_{2^r}(2M))(a-1)^{-1}$. Given $n > \overline{b} + \ell$, consider
\begin{equation*}
    \Tilde{g}(n) = \frac{n - 2 - \log_{2^r}(2M)}{a} - 1.
\end{equation*}
It is clear that $g(n)\geq \Tilde{g}(n)$ and both functions are increasing, therefore $g^m(n)\geq \Tilde{g}^m(n)$ for every $m\geq 0$. As
\begin{equation*}
    \Tilde{g}^m(n) = \frac{n}{a^m} - b^\prime\frac{a^m - 1}{a^{m-1}(a-1)},
\end{equation*}
with $b^\prime = (a+2 + \log_{2^r}(2M))a^{-1}$, it is sufficient to have
\begin{equation*}
    \frac{n}{a^m} -\frac{a b^\prime}{(a-1)}\geq \ell.
\end{equation*}
We get the desired bound by applying the logarithm with base two in the equation above. We now need an upper bound on $l_2(n)$. For any $m\leq l_2(n)$, all cubes in $\C_{rg^m(n)}(\xi)$ are distant at most $M2^{arg^m(n)}$ and $|\C_{rg^m(n)(\xi)}|\leq 2^{r}-1$, therefore
\begin{equation*}
    \diam(\xi)\leq \diam(B_{\C_{rg^m(n)(\xi)}})\leq (d2^{rg^m(n)} + Md^a2^{arg^m(n)})|\C_{rg^m(n)(\xi)}|\leq 2Md^a2^{arg^m(n)+r}.
\end{equation*}
Applying the logarithm with respect to base $2^{r}$ we get
\begin{equation*}
    \log_{2^r}(\diam(\xi)) \leq \log_{2^r}(2M) + ag^m(n)+1 \leq \log_{2^r}(2M) + \frac{n}{a^{m-1}} + 1
\end{equation*}
Assuming $\diam(\xi)>2^{2r + 1}M$, we can isolate the term depending on $m$ in the equation above and take to logarithm in both sides to get
\begin{equation*}
    m \leq 1 + \frac{\log_2(n) - \log_2(\log_{2^r}(\diam(\xi)) - \log_{2^r}(2M) - 1)}{\log_2(a)}.
\end{equation*}
Equation above holds for any element of $\{m : |\mathscr{G}_{rg^m(n)}(A)| = 1, |v(G_{rg^m(n)})|\leq 2^r-1\}$ thus it also holds for $l_2(n)$. Together with \eqref{Eq: lower.bound.on.l1}, this yields 

\begin{equation}
    l_1(n) - l_2(n)\geq \frac{\log_2[\log_{2^r}(\diam(\xi)) -\log_{2^r}(2M) - 1] - \log_2(\overline{b} + \ell)}{\log_2(a)} - 2.
\end{equation}
Applying this back in equation \eqref{Eq: bound.on.rn.covering}, we get
\begin{align*}
    V_r^\ell(\xi) &\leq (\overline{b} + 1)|\C_{r\ell}(\xi)| + |\C_{r\ell}(\xi)|\frac{2^{2(r-d-1)} (\overline{b} +\ell)^{\frac{r-d-1}{\log_2(a)}}n_r(\xi)}{[\log_{2^r}(\diam(\xi)) - \log_{2^r}(2M) - 1]^{\frac{r-d-1}{\log_2(a)}}} \\
    %
    &\leq [\overline{b} +1 + 2^{2(r-d-1)} (\overline{b} +\ell)^{\frac{r-d-1}{\log_2(a)}}(\log_{2^r}(2M) +3)]|\C_{r\ell}(\xi)|\\
    %
    &\leq [\overline{b} +1 + 2^{2(r-d-1)} (\overline{b} +1)^{\frac{r-d-1}{\log_2(a)}}(\log_{2^r}(2M) +3)]\ell^{\frac{r-d-1}{\log_2(a)}}|\C_{r\ell}(\xi)|
\end{align*}
where in the second equation we used that $(x/(x-w))\leq 1 + w$ for any $x\geq w + 1$. If $\diam(\xi)\leq 2^{2r + 1}M$, we have
\begin{equation*}
     V_r^\ell(\xi) \leq (n_r(\xi) - \ell+1)|\C_{r\ell}(\xi)| \leq (3 + \log_{2^r}(2M))|\C_{r\ell}(\xi)|.
\end{equation*}
Taking $b_3^\prime\coloneqq \overline{b} +1 + 2^{2(r-d-1)} (\overline{b} +1)^{\frac{r-d-1}{\log_2(a)}}(\log_{2^r}(2M) +3)$ we get the desired bound when $\ell\geq 2$. For $\ell=0$, a trivial bound yields  $V_r^0(\gamma) = 2|\gamma| + V_r^2(\gamma)\leq (2 + b_3^\prime 2^{\frac{r-d-1}{\log_2(a)}})|\gamma|$. Similarly, for $\ell =1$,  $V_r^1(\gamma) = |\C_{r}(\gamma)| + V_r^2(\gamma)\leq (1+ b_3^\prime 2^{\frac{r-d-1}{\log_2(a)}})|\mathscr{C}_{r}(\gamma)|$ and we conclude the proof by taking $b_3 \coloneqq 2 + b_3^\prime 2^{\frac{r-d-1}{\log_2(a)}}$.
\end{proof}

We then need to bound the number of $r\ell$-cubes to cover a contour. Using only Lemma \ref{Lemma: Big.clusters_1}, we can prove the next proposition, in the same steps as in \cite[Proposition 3.13]{Affonso.2021}.

\begin{proposition}\label{Prop. Bound.on.C_rl(gamma)_Lucas}
    There exists a constant $b_4^\prime\coloneqq b_4^\prime(\alpha, d)$ such that for any $A\Subset \Z^d$, $\gamma\in\Gamma(A)$ and $\ell\geq 1$, 
     \begin{equation*}
       |\C_{r\ell}(\gamma)|\leq b_4^\prime\frac{|\gamma|}{\ell^{\frac{r-d-1}{\log_2(a)}}}.
    \end{equation*}
\end{proposition}

Using our construction, we can improve this for steps scales smaller than the step the contour was removed. This is done in the next proposition. 
\begin{proposition}\label{Prop. Bound.on.C_rl(gamma)}
    There exists a constant $b_4\coloneqq b_4(\alpha, d)$ such that for any $A\Subset \Z^d$, $\gamma\in\Gamma^r_j(A)$ and $0 \leq \ell<j$,
    
     \begin{equation*}
       |\C_{r\ell}(\gamma)|\leq b_4\frac{(\ell\wedge 1)^{\frac{d+1}{a + (1-\frac{1}{d})}}}{2^{ra^\prime\ell}}|\gamma|,
    \end{equation*}
    where $a^\prime \coloneqq \frac{(1-\frac{1}{d})}{a-1 + (1-\frac{1}{d})}$
\end{proposition}

\begin{proof}
     Define $f : \mathbb{N} \xrightarrow{} \Z$ by
\begin{equation}
    f(\ell)\coloneqq \floor*{\frac{\ell - \log_{2^r}(2M) - 1}{a + (1-\frac{1}{d})}}.
\end{equation}
Following the proof of \eqref{Eq: Bound_c_n_by_C_g(n)} in \cite[Proposition 3.13]{Affonso.2021}, we can show that 
\begin{equation}\label{Eq: Bound_C_l_by_C_f(l)}
    |\C_{r\ell}(\gamma)| \leq \frac{2^{d+1}}{2^{r(1-\frac{1}{d})f(\ell)}}|\C_{rf(\ell)}(\gamma)|.
\end{equation}
 By definition, $\mathscr{G}_{rf(\ell)}(\gamma)$ is the set of all connected components of $G_{rf(\ell)}(\gamma)$, hence
    \begin{equation}\label{Eq: 3.16.Lucas}
        |\C_{rf(\ell)}(\gamma)| = 2^{r(1-\frac{1}{d})f(\ell)}\sum_{G\in \mathscr{G}_{rf(\ell)}(\gamma)}\frac{|v(G)|}{2^{r(1-\frac{1}{d})f(\ell)}}.
    \end{equation}
    Proposition \ref{Prop. partition.a.graph} guarantees that we can split $G$ into sub-graphs $G_i$, with $1\leq i\leq \ceil{v(G)/2^{r(1-\frac{1}{d})f(\ell)}}$ and $|v(G_i)|\leq 2^{r(1-\frac{1}{d})f(\ell)+1}$. For any $\Lambda,\Lambda^\prime\Subset \Z^d$, 
    \begin{equation*}
        \diam(\Lambda\cup \Lambda) \leq \diam(\Lambda) + \diam(\Lambda^\prime) + \d(\Lambda,\Lambda^\prime),
    \end{equation*}
    and we can always extract a vertex from a connected graph in a way that the induced sub-graph is still connected, by removing a leaf of a spanning tree. Using this we can bound 
    \begin{align*}
        \diam(B_{v(G_i)}) &\leq \sum_{C_{rf(\ell)}\in v(G_i)} \diam(C_{rf(\ell)}) + |v(G_i)|M2^{arf(\ell)}\\
        %
        &\leq |v(G_i)|(d2^{rf(\ell)} + M2^{arf(\ell)}) \leq 2M2^{r[f(\ell)((1-\frac{1}{d})) + a] + 1}\\
        %
        &\leq 2^{r\ell}.
    \end{align*}
    
    The last inequality holds since $M,a,r\geq 1$. This shows that every $G_i$ can be covered by a cube with center in $\Z^d$ and side length $2^{r\ell}$. Every such cube can be covered by at most $2^d$ $r\ell$-cubes. Indeed, it is enough to consider the simpler case when the cube is of the form
    \begin{equation}\label{Cube.Q}
        \prod_{i=1}^d[q_i - 2^{r\ell - 1}, q_i + 2^{r\ell -1})\cap\Z^d,
    \end{equation}
    with $q_i\in\{0, 1, \dots, 2^{r\ell} - 1\}$, for $1\leq i \leq d$. It is easy to see that \begin{equation*}
        [q_i - 2^{r\ell - 1}, q_i + 2^{r\ell -1})\subset [-2^{r\ell-1}, 2^{r\ell-1})\cup [2^{r\ell-1},2^{r\ell} + 2^{r\ell -1}). 
    \end{equation*} 
    Taking the products for all $1\leq i\leq d$, we get $2^d$ $r\ell$-cubes that covers \eqref{Cube.Q}. 
    We conclude that, to cover a connected component $G\in \mathscr{G}_{rf(\ell)}$, we need at most $2^d\ceil{|v(G)|/2^{r(1-\frac{1}{d})f(\ell)}}$ $rf(\ell)$-cubes, yielding us
    \begin{equation}\label{Eq: 3.18.Lucas}
        |\C_{r\ell}(\gamma)|\leq |\C_{r\ell}( B_{\C_{rf(\ell)}(\gamma)})| \leq \sum_{G\in \mathscr{G}_{rf(\ell)}} |\C_{r\ell}(v(G))| \leq \sum_{G\in \mathscr{G}_{rf(\ell)}} 2^d\left\lceil{\frac{|v(G)|}{2^{r(1-\frac{1}{d})f(\ell)}}}\right\rceil. 
    \end{equation}
    When every connected component of $G_{rf(\ell)}(\gamma)$ has more than $2^{r(1-\frac{1}{d})f(\ell)}$ vertices, we can bound 
    \begin{equation*}
        \frac{1}{2}\left\lceil{\frac{|v(G)|}{2^{r(1-\frac{1}{d})f(\ell)}}}\right\rceil \leq {\frac{|v(G)|}{2^{r(1-\frac{1}{d})f(\ell)}}}.
    \end{equation*}
    Together with inequalities \eqref{Eq: 3.16.Lucas}  and \eqref{Eq: 3.18.Lucas}, this yields
    \begin{equation}\label{Eq: Bound_C_rl_by_gamma_with_f(l)}
         |\C_{r\ell}(\gamma)| \leq \sum_{G\in \mathscr{G}_{rf(\ell)}} 2^{d+1} {\frac{|v(G)|}{2^{r(1-\frac{1}{d})f(\ell)}}}= \frac{2^{d+1}}{2^{r(1-\frac{1}{d})f(\ell)}}|\C_{rf(\ell)}(\gamma)|.
    \end{equation}

    Equation \eqref{Eq: Bound_C_l_by_C_f(l)} can be iterated as long as the radius is positive. Considering then the auxiliary quantity
    \begin{equation*}
        m(\ell)\coloneqq\max\{m : f^m(\ell)\geq 0\},  
    \end{equation*}
we have 
\begin{equation*}
    |\C_{r\ell}(\gamma)| \leq \frac{2^{(d+1)m(\ell)}}{2^{r(\delta - d)(1-\frac{1}{d})[\sum_{i=1}^{m(\ell)}f^i(\ell)]}}|\gamma|,
\end{equation*}
so we need upper and lower estimates for $m(\ell)$.
We claim that
\begin{equation}\label{Eq: lower.bound.on.m}
    m(\ell) \geq \begin{cases}
                        0, &\text{ if }n\leq \overline{b}\\ 
                        \left\lfloor\frac{\log_2(n) - \log_2(\overline{b})}{\log_2(a + (1-\frac{1}{d}))}\right\rfloor, & \text{ if }n > \overline{b}, 
                \end{cases}
\end{equation}
where $\overline{b} = (\overline{a}+1 + \log_{2^r}(2M))(\overline{a}-1)^{-1}$ and $\overline{a}\coloneqq a + (1-\frac{1}{d})$. Given $\ell > \overline{b}$, consider
\begin{equation*}
     \overline{f}(\ell) = \frac{\ell - 1 - \log_{2^r}(2M)}{a + (1 - \frac{1}{d})} - 1.
\end{equation*}
It is clear that $f(n)\geq \overline{f}(n)$ and both functions are increasing, therefore $f(\ell)^m\geq \overline{f}^m(\ell)$ for every $m\geq 0$.  As
\begin{equation*}
       \overline{f}^m(\ell) = \frac{\ell}{\overline{a}^m} - b^\prime\frac{\overline{a}^m - 1}{\overline{a}^{m-1}(\overline{a}-1)},
\end{equation*}
with $b^\prime = (\overline{a}+1 + \log_{2^r}(2M))\overline{a}^{-1}$, it is sufficient to have
\begin{equation*}
    \frac{n}{\overline{a}^m} -\frac{\overline{a} b^\prime}{(\overline{a}-1)}\geq 0.
\end{equation*}
We get the desired bound by applying the logarithm with base two in the equation above. Moreover, we can bound
\begin{align*}
    \sum_{i=1}^{m(\ell)}f^{i}(\ell) & \geq  \sum_{i=1}^{m(\ell)}\frac{\ell}{\overline{a}^m} - m(\ell)\frac{\overline{a}b^\prime}{\overline{a} - 1} = \frac{1}{\overline{a}}(\frac{1-\frac{1}{\overline{a}^{m(\ell)}}}{{1-\frac{1}{\overline{a}}}})\ell - m(\ell)\overline{b} \\
    %
    &\geq \frac{1}{\overline{a}-1}(1-\frac{1}{\overline{a}^{m(\ell)}}) - m(\ell)\overline{b} \geq \frac{1}{\overline{a}-1}(1-\frac{\overline{a}\overline{b}}{\ell}) - m(\ell)\overline{b}
\end{align*}

For the upper bound on $m(\ell)$, take $\Tilde{f}(\ell) \coloneqq \frac{\ell}{a + (1-\frac{1}{d})}$. As $f(\ell)\leq \Tilde{f}(\ell)$ and $\Tilde{f}$ is increasing,  $f^m(\ell)\leq \Tilde{f}^m(\ell)$ for every $m\geq 0$. Notice that, if $\Tilde{f}^m(\ell)\leq 1$, $f^{m+1}(\ell)<0$ and therefore $m+1>m(\ell)$. As $\Tilde{f}^m(\ell)\leq 1$ if and only if $\ell \leq [a + (1-\frac{1}{d})]^m$, we have $\frac{\log_2(\ell)}{a + (1-\frac{1}{d})} + 1 > m(\ell)$.
Applying this bound on \eqref{Eq: Bound_C_rl_by_gamma_with_f(l)} we conclude that 
\begin{equation}\label{Eq: bound_on_C_ell_covering_l_geq_ab}
     |\C_{r\ell}(\gamma)| \leq \frac{2^{d+1 + \overline{a}\overline{b}}\ell^{\frac{d+1}{a + (1-\frac{1}{d})}}}{2^{r(1-\frac{1}{d})\frac{1}{\overline{a}-1}\ell}}|\gamma|,
\end{equation}
for $\ell>\overline{a}\overline{b}$. When $\ell\leq \overline{a}\overline{b}$, we can take $\overline{b}_4\coloneqq \min\{{ j^{\frac{d+1}{a + (1-\frac{1}{d})}}{2^{-r(1-\frac{1}{d})\frac{1}{\overline{a}-1}j}}} : 0\leq j \leq \overline{a}\overline{b}\}$ and then 
\begin{equation*}
       |\C_{r\ell}(\gamma)| \leq |\gamma|\leq \frac{1}{\overline{b}_4}\frac{\ell^{\frac{d+1}{a + (1-\frac{1}{d})}}}{2^{r(1-\frac{1}{d})\frac{1}{\overline{a}-1}\ell}}|\gamma|.
\end{equation*}
This, together with equation \eqref{Eq: bound_on_C_ell_covering_l_geq_ab}, concludes the proposition with $b_4^\prime \coloneqq \max\{ 2^{d+1 + \overline{a}\overline{b}}, \overline{b}_4^{-1}\}$.
\end{proof}


For any non-negative $V, M, a, r$, define
\begin{equation*}
    \mathcal{F}_{V}^\ell\coloneqq \{ \C_{r\ell} : V_r^\ell(B_{\C_{r\ell}}) = V, B_{\C_{r\ell}}\subset C_{2rn_r(B_{\C_{r\ell}})}(0)\}.
\end{equation*}

Using equation \eqref{Eq: Bound.on.N}, in the same steps as \cite[Proposition 3.11]{Affonso.2021}, we can show that the number of collections in $\mathcal{F}_V$ is exponentially bounded by $V+\ell$.

\begin{proposition}\label{Prop. Bound.on.Fv}
    There exists $b_5\coloneqq b_5(d,r)$ such that
    \begin{equation}\label{Eq: Bound.on.F_V}
        |\mathcal{F}^\ell_V| \leq e^{b_5(V+\ell)}.
    \end{equation}
\end{proposition}

\begin{proof}
We start by splitting $\mathcal{F}^\ell_V$ into $\mathcal{F}^\ell_{V,m} \coloneqq \{ \C_{r\ell}\in \mathcal{F}^\ell_V : n_r(B_{\C_{r\ell}})=m \}$. Since $\ell\leq n_r(\C_{r\ell}) \leq V_r^\ell(B_{\C_{r\ell}}) +\ell$, we get

\begin{equation}
    |\mathcal{F}^\ell_V| \leq \sum_{m=\ell}^{V+\ell} |\mathcal{F}^\ell_{V,m}|. 
\end{equation}
Taking $b^{\prime \prime} =  2$ and denoting $(V_{rn})_{n=\ell}^{m}$ an arbitrary family of natural numbers satisfying 
\begin{equation}\label{Eq: sum.of.V_rn}
    \sum_{n=\ell}^{m} V_{rn}\leq  V,
\end{equation}
and $V_{rn}\leq V_{r(n-1)}$, we can bound
\begin{align}\label{Eq: bound.FVL}
    |\mathcal{F}^\ell_{V,m}| \leq \sum_{(V_{rn})_{n=\ell}^{m}} |\{\C_{r\ell} : B_{\C_{r\ell}} \subset  C_{b^{\prime \prime}rm}(0), |\C_{rn}(B_{\C_{r\ell}})| = V_{rn}, \text{ for every } \ell \leq n\leq m, n_r(B_{\C_{r\ell}}) = m\}|.
\end{align}
By our choice of $n_r$, every collection $\C_{r\ell}$ satisfying $n_r(B_{\C_{r\ell}})=m$ can be covered by a cube centered in $\Z^d$ and side length $2^{rm}$. Every such cube can be covered by at most $2^d$ $rm$-cubes. Indeed, it is enough to consider the simpler case when the cube is of the form
    \begin{equation}\label{Cube.Q_2}
        \prod_{i=1}^d[q_i - 2^{rm - 1}, q_i + 2^{rm -1})\cap\Z^d,
    \end{equation}
    with $q_i\in\{0, 1, \dots, 2^{rm} - 1\}$, for $1\leq i \leq d$. It is easy to see that \begin{equation*}
        [q_i - 2^{rm - 1}, q_i + 2^{rm -1})\subset [-2^{rm-1}, 2^{rm-1})\cup [2^{rm-1},2^{rm} + 2^{rm -1}). 
    \end{equation*} 
    Taking the products for all $1\leq i\leq d$, we get $2^d$ $rm$-cubes that covers \eqref{Cube.Q_2}. Let $\mathcal{C}^m$ be the set of collections $\C_{rm}$ such that $|\C_{rm}|=V_{rm}$, $B_{\C_{rm}}\subset C_{b^{\prime\prime}rm}(0)$  and there exists a cube $C$ centered in $\Z^d$ with side length $2^{rm}$ such that, for all $C_{rm}\in \C_{rm}$, $C_{rm}\cap C\neq \emptyset$. For every collection $\C_{r\ell}$ in the set on the RHS of \eqref{Eq: bound.FVL}, $\C_{rm}(B_{\C_{r\ell}}) \in \mathcal{C}^m$. We have that $B_{\C_{rm}(B_{\C_{r\ell}})}\subset C_{b^{\prime \prime}rm}(0)$ since given two cubes $C_{a}, C_{b}$, with $a\leq b$, either $C_{a}\subset C_{b}$ or $C_{a}\subset C_{b}^c$.   Moreover, we can bound 
\begin{align*}
    |\mathcal{C}^m| &\leq |\{C \text{ cube with side length }2^{rm} \text{ centered in }\Z^d\cap C_{rm} \text{ with }C_{rm}\subset C_{b^{\prime\prime}rm}(0) \}|\binom{2^d}{V_{rm}}\\
    %
    &\leq 2^{rdm}|\{C_{rm} : C_{rm} \subset C_{b^{\prime\prime}rm}(0)\}|\binom{2^d}{V_{rm}} \leq 2^{rdm}2^{d(b^{\prime \prime}-1)rm}(e2^d)^{V_m}
\end{align*}
We can count the RHS of equation \eqref{Eq: bound.FVL} by counting the number of families $(\C_{rn})_{n=\ell}^m$ such that $\C_{rn}\preceq \C_{r(n+1)}$, for $n<m$, and $\C_{rm}\in \mathcal{C}^m$, yielding us 
\begin{align*}
    |\mathcal{F}^\ell_{V,m}| 
    &\leq \sum_{(V_{rn})_{n=\ell}^{m-1}} |\{ (\C_{rn})_{n=\ell}^{m} : |\C_{rn}|=V_{rn}, \C_{rn}\preceq \C_{r(n+1)}, \C_{rm}\in \mathcal{C}^m\}| \\
    %
    & \leq  \sum_{(V_{rn})_{n=\ell}^{m-1}} \sum_{\C_{rm}\in \mathcal{C}^m}\sum_{\substack{\C_{r(m-1)} \\ |\C_{r(m-1)}|=V_{r(m-1)}\\ \C_{r(m-1)\preceq \C_{rm}}}} \cdots \sum_{\substack{\C_{r(\ell+1)} \\ |\C_{r(\ell+1)}|=V_{r(\ell+1)}\\ \C_{r(\ell+1)\preceq \C_{r(\ell+2)}}}} N(\C_{r(\ell+1)}, r\ell, V_{r\ell}).           
\end{align*}
Iterating equation \eqref{Eq: Bound.on.N} we get that
\begin{align*}
    |\mathcal{F}^\ell_{V,m}| &\leq \sum_{(V_{rn})_{n=\ell}^{m}}\sum_{\C_{rm}\in \mathcal{C}^m}\prod_{n=\ell}^{m-1}\left( \frac{2^{rd}e V_{r(n+1)}}{V_{rn}}\right)^{V_{rn}}\\
    %
    &\leq  \sum_{(V_{rn})_{n=\ell}^{m-1}}|\mathcal{C}^m|\prod_{n=\ell}^{m-1}e^{(rd\ln(2) +1)V_{rn}} \leq 2^{db^{\prime \prime}rm}\sum_{(V_{rn})_{n=\ell}^{m}}e^{(rd\ln(2)+1)V}.
\end{align*}



As $\sum_{m=\ell}^{V+\ell}2^{db^{\prime \prime}rm}\leq 2^{(db^{\prime \prime}r)(\ell+V) + 1}$ and the number of solutions of \eqref{Eq: sum.of.V_rn} is bounded by $2^V$, we conclude that 
\begin{equation*}
     |\mathcal{F}^\ell_{V}| \leq \sum_{m=\ell}^{V+\ell}|\mathcal{F}_{V,m}^\ell| \leq  2^{(db^{\prime \prime}r)(\ell+V) +1}2^Ve^{(rd\ln(2) + 1)V},
\end{equation*}
therefore equation \eqref{Eq: Bound.on.F_V} holds for $b_5\coloneqq [db^{\prime \prime}r + rd +2]\ln(2) + 1$.
\end{proof}
    
\begin{lemma}\label{Prop: Counting_spamming_trees}
    Given $\ell>0$, consider the graph $G=V(\C_{r\ell}(\Z^d), E)$, with two vertices $C,C^\prime$ being connected if and only if $d(C,C^\prime)\leq M2^{ra\ell}$. There exists a constant $b_5^\prime \coloneqq b_5^\prime(d,\alpha)$ such that
    \begin{equation}
        |\{{\C_{r\ell}}: C_{r\ell}(0)\in \C_{r\ell}, \ \C_{r\ell} \emph{ is connected }, |\C_{r\ell}|=N\}| \leq e^{b_5^\prime\ell N}.
    \end{equation}
\end{lemma}

\begin{proof}
    To count $|\{{\C_{r\ell}}: C_{r\ell}(0)\in \C_{r\ell}, \ \C_{r\ell} \emph{ is connected }, |\C_{r\ell}|=N\}|$, it is enough to count the number spanning trees containing $C_{r\ell}(0)$ with $N$ vertices. Let $\mathcal{T}_0$ be the set of all such trees Fixed $T\in\mathcal{T}_0$, for each $C_{r\ell}\in v(T)$, let $\d_T(C_{r\ell})$ be the degree of $C_{r\ell}$. As $T$ is a tree, $\sum_{C_{r\ell}\in v(T)}\d_T(C_{r\ell}) = 2(N-1)$. Moreover, as there are at most $2^{rd(a\ell + \log_{2^r}M - \ell)}$ $r\ell$-cubes inside a $r(a\ell + \log_{2^r}M)$-cube, each cube $C_{r\ell}\in T$ has at most $2^{rd(a +\log_{2^r}M -1)\ell}$ neighbours. Let $(d_i)_{i=1}^N$ denote a general solution to 
    \begin{equation}\label{Eq: sum_d_i}
        \sum_{i=1}^{N}d_i = 2(N-1),
    \end{equation}
    with $d_i\leq 2^{rd(a +\log_{2^r}M -1)\ell}$ for all $i=1,\dots, N$. Then 
    \begin{align*}
        |\{{\C_{r\ell}}: C_{r\ell}(0)\in \C_{r\ell}, \ \C_{r\ell} \emph{ is connected }, |\C_{r\ell}|=N\}| \leq \sum_{(d_i)_{i=1}^N} |\{T \in\mathcal{T}_0: d_T(C^i) = d_i\}|.
    \end{align*}
    In the set above, $\{C^1, C^2, \dots, C^N\}$ is any ordering of $v(T)$ with $C^1 = C_{r\ell}(0)$. Therefore, 
    \begin{equation*}
        |\{T \in\mathcal{T}_0: d_T(C^i) = d_i\}| \leq \prod_{i=1}^N\binom{2^{rd(a +\log_{2^r}M -1)\ell}}{d_i}\leq (e2^{rd(a +\log_{2^r}M -1)\ell})^{N}. 
    \end{equation*}
    As the number of solutions to \eqref{Eq: sum_d_i} is bounded by $2^N$, we conclude that
    \begin{align*}
        \{{\C_{r\ell}}: C_{r\ell}(0)\in \C_{r\ell}, \ \C_{r\ell} \emph{ is connected }, |\C_{r\ell}|=N\}| \leq  2^Ne^N2^{rd(a +\log_{2^r}M -1)\ell N} \leq e^{b_5^\prime \ell N}, 
    \end{align*}
    with $b_5^\prime \coloneqq \ln{2} + 1 + {rd(a +\log_{2^r}M -1)\ln{2}}$.
\end{proof}

\begin{proposition}\label{Prop: Bound_on_rl_coverings}
    Let $n,j\geq 0$, $\Lambda\Subset\Z^d$. There exists a constant $b_6\coloneqq b_6(a,d)>0$ such that if $0\leq \ell\leq j$,
    \begin{equation*}
        |\C_{r\ell}(\mathcal{C}_0(n,j))|\leq \exp{\left\{ b_6 \left[\frac{(\ell\wedge 1)^{\kappa}n}{2^{ra^\prime\ell}} + \ell \right]   \right\}},
    \end{equation*}
    where $\kappa \coloneqq \kappa(\alpha, d) = 1 + \frac{2}{\log_2(a)} + \frac{d+1}{a + (1-\frac{1}{d})}$. Moreover, if $\ell >j$, 
    \begin{equation*}
        |\C_{r\ell}(\mathcal{C}_0(n,j))|\leq \exp{\{b_6\frac{|\gamma|}{\ell^{\frac{r-d-1}{\log_2(a)} - 1}}\}}.
    \end{equation*}
\end{proposition}
\begin{proof}
    For $1\leq \ell\leq j$, Proposition \ref{Prop. Bound.on.V_r^l(gamma)} together with Proposition \ref{Prop. Bound.on.C_rl(gamma)} yields, 
    \begin{equation}\label{Eq: Bound_partial_volume_l_between_1_and_j}
        V_r^\ell(\gamma) = V_r^\ell(B_{\C_{r\ell}(\gamma)}) \leq b_3b_4\frac{\ell^{\kappa}|\gamma|}{2^{ra^\prime\ell}}.
    \end{equation}
    Therefore, 
    \begin{equation*}
        \{\C_{r\ell} :\C_{r\ell}=\C_{r\ell}(\gamma) \ \textrm{for some }\gamma\in\mathcal{C}_0(n,j)\}\subset \{ \C_{r\ell} : V_r^\ell(B_{\C_{r\ell}}) \leq b_3b_4\frac{\ell^{\kappa}|\gamma|}{2^{ra^\prime\ell}} \ , B_{\C_{r\ell}}\subset C_{2rn_r(B_{\C_{r\ell}})}(0)\}.
    \end{equation*}
    Proposition \ref{Prop. Bound.on.Fv} yields 
    \begin{multline*}
        |\{ \C_{r\ell} : V_r^\ell(B_{\C_{r\ell}}) \leq b_3b_4\frac{\ell^{\kappa}|\gamma|}{2^{ra^\prime\ell}}, B_{\C_{r\ell}}\subset C_{2rn_r(B_{\C_{r\ell}})}(0)\}| \leq \sum_{V=1}^{\ceil{b_3b_4\frac{\ell^{\kappa}|\gamma|}{2^{ra^\prime\ell}}}}|\mathcal{F}^\ell_V|  \leq \exp{\left\{ b_5b_4b_3\frac{\ell^{\kappa}n}{2^{ra^\prime\ell}} + b_5\ell  \right\}}.
    \end{multline*}
    When $\ell=0$, the same argument holds replacing the RHS of \eqref{Eq: Bound_partial_volume_l_between_1_and_j} by $b_3 b_4|\gamma|$. When $\ell>j$, for any $\gamma\in \mathcal{C}_0(n,j)$, $|\G_{r\ell}(\gamma)|=1$. Moreover, by Proposition \ref{Prop. Bound.on.C_rl(gamma)_Lucas},  
    \begin{equation*}
        |\C_{r\ell}(\gamma)|\leq b_4^\prime\frac{|\gamma|}{\ell^{\frac{r-d-1}{\log_2(a)}}}.
    \end{equation*}
    As $0\in V(\gamma)$, we now there is a cube $C_{r\ell}\in\C_{r\ell}(\gamma)$ that intersects the axis $e_1$ and $d(C_{r\ell}, 0)<|\C_{r\ell}(\gamma)|2^{r\ell}$. Therefore, there exists $|\C_{r\ell}(\gamma)|\leq b_4^\prime\frac{|\gamma|}{\ell^{\frac{r-d-1}{\log_2(a)}}}$ possible positions for $C_{r\ell}$. Using Lemma \ref{Prop: Counting_spamming_trees} we conclude that 
    \begin{equation}
        |\C_{r\ell}(\mathcal{C}_0(n,j))| \leq b_4^\prime\frac{|\gamma|}{\ell^{\frac{r-d-1}{\log_2(a)}}}\exp{\{{b_5^\prime b_4^\prime\ell\frac{|\gamma|}{\ell^{\frac{r-d-1}{\log_2(a)}}}}\}} \leq \exp{\{{2b_5^\prime b_4^\prime\ell\frac{|\gamma|}{\ell^{\frac{r-d-1}{\log_2(a)}}}}\}} 
    \end{equation} 
    what concludes the proof for $b_6\coloneqq \max{\{b_5b_4b_3, 2b_5^\prime b_4^\prime\}}$.
\end{proof}
A consequence of this Proposition is that we get an exponential bound on the number of contours with a fixed size.
\begin{corollary}\label{Cor: Bound_on_C_0_n}
	Let $n\geq 1$, $d\ge 2$, and $\Lambda\Subset \mathbb{Z}^d$. There exists $c_1\coloneqq c_1(d,M,r)>0$ such that
	\begin{equation}\label{Eq: exp.bound.contours}
	|\mathcal{C}_0(n)| \leq e^{c_1 n}.	    
	\end{equation}
\end{corollary}
\begin{proof} For any $j\geq 1$, Proposition  \ref{Prop: Bound_on_rl_coverings} applied to $\ell=0$ yields
     $|\mathcal{C}_0(n,j)|\leq e^{ b_6n}$. Remember that, in the proof of Proposition \eqref{Prop: Bound.bad.event.1} we showed that  $j\leq \frac{d}{d^2-1}\log_{2^r}n + 1$. So we can bound 
     \begin{align*}
        |\mathcal{C}_0(n)| \leq \sum_{j=1}^{\frac{d}{d^2-1}\log_{2^r}n + 1}|\mathcal{C}_0(n,j)| \leq e^{c_1n}
    \end{align*}
     with $c_1\coloneqq 2 b_6 + \ln{(\frac{d}{d^2-1} + 1)}$.
\end{proof}

\begin{proposition}\label{Prop: Bound_on_boundary_of_admissible_sets}
    Let $n,j\geq 0$, $\Lambda\Subset\Z^d$ and $\ell\geq 0$. There exists a constant $c_4\coloneqq c_4(\alpha, d)$ such that,
    \begin{equation}\label{Eq: Bound_on_boundary_of_admissible_sets}
        |B_\ell(\mathcal{C}_0(n,j))|\leq \exp{\left\{c_4 \ell^{\kappa}\left[\frac{n}{2^{r\ell(d-1-\frac{2\log_2(a)}{r-d-1-\log_2(a)})}} + \frac{n}{2^{2r\ell}} +1 \right]   \right\}}.
    \end{equation}
\end{proposition}

\begin{proof}
    Remember that $|B_\ell(\mathcal{C}_0(n,j))|=|\partial\mathfrak{C}_\ell(\mathcal{C}_0(n,j))|$. Moreover, given $\{C_{r\ell},C_{r\ell}^\prime\}\in\partial\mathfrak{C}_\ell(\gamma)$, either $C_{r\ell}\in\fint\mathfrak{C}_{\ell}$ or $C_{r\ell}^\prime\in\fint\mathfrak{C}_{\ell}$. Using that $\sum_{k=1}^p\binom{p}{k} = 2^{p}$, we have
    \begin{equation}\label{Eq: Replacing_edge_by_inner_boundary}
       \begin{split} |\partial\mathfrak{C}_\ell(\mathcal{C}_0(n,j))| &\leq \sum_{\fint\C_{r\ell}\in \fint\mathscr{C}_{\ell}(\mathcal{C}_0(n,j))} |\{\partial\C_{r\ell}^\prime : \fint\C_{r\ell}^\prime = \fint\C_{r\ell}\}| \\
        %
        &\leq \sum_{\fint\C_{r\ell}\in \fint\mathscr{C}_{\ell}(\mathcal{C}_0(n,j))} \sum_{k=1}^{2d|\fint\C_{r\ell}|}\binom{2d|\fint\C_{r\ell}|}{k}  \\
        %
        &=\sum_{\fint\C_{r\ell}\in \fint\mathscr{C}_{\ell}(\mathcal{C}_0(n,j))} 2^{2d|\fint\C_{r\ell}|}\leq |\fint\mathfrak{C}_\ell(\mathcal{C}_0(n,j))|e^{\ln(2)2db_1\frac{n}{2^{r\ell(d-1)}}},
        \end{split}
    \end{equation}
     where in the last inequality we applied Proposition \ref{Proposition1}. For every $L\geq \ell$ and an arbitrary collection $\C_{rL}$, define $\overline{\C_{rL}} = \C_{rL}\cup \{C_{rL}^\prime : \exists C_{rL}\in\C_{rL} \text{ such that } C_{rL}^\prime \text{ shares a face with } C_{rL}\}$. 
     
     Given $C_{r\ell}\in \fint \mathfrak{C}_\ell(\ell)$, either $C_{r\ell}$ or one of its neighbouring cubes intersects $\gamma$. Hence, for any $L\geq \ell$, $\fint \mathfrak{C}_{\ell}(\gamma)\preceq \overline{\C_{rL}(\gamma)}$. Moreover, the number of $r\ell$-cubes inside a collection $\overline{\C_{rL}(\gamma)}$ of $rL$-cubes is bound by $|\overline{\C_{rL}(\gamma)}|2^{rd(L-\ell)} \leq 2d|\C_{rL}(\gamma)|2^{rd(L-\ell)}$. Using again Proposition \ref{Proposition1}, we can bound 
    \begin{equation}\label{Eq: Bound_on_internal_boundary}
    \begin{split}
           |\fint \mathfrak{C}_{r\ell}(\mathcal{C}_0(n,j))| &\leq  \sum_{\substack{\C_{rL} \in \C_{rL}(\mathcal{C}_0(n,j))}}\sum_{k=1}^{\frac{b_1n}{2^{r\ell(d-1)}}}\binom{2d|\C_{rL}|2^{rd(L-\ell)}}{k} \\
           %
           &\leq \sum_{\substack{\C_{rL} \in \C_{rL}(\mathcal{C}_0(n,j))}}\left(\frac{e2d|\C_{rL}|2^{rdL}}{{b_1n}{2^{r\ell}}}\right)^{\frac{b_1n}{2^{r\ell(d-1)}}},
           \end{split}
    \end{equation}
    where in the last equation we used that, for any $0<M\leq N$, $\sum_{p=1}^{M}\binom{N}{p}\leq \left(\frac{eN}{M}\right)^{M}$. 
    
    Now we need to choose $L(\ell)$ according to $\ell$. When $\ell > (\frac{r-d-1}{\log_2 (a)}-1)(\frac{\log_2 (j)}{2r} + 1)$, we take $L(\ell) = 2^{2r\left\lfloor{\frac{\log_2(a)\ell}{r-d-1 - \log_2 (a)}}\right\rfloor}$. Then, $L>j$ and  Proposition \ref{Prop. Bound.on.C_rl(gamma)_Lucas} yields $e2d|\C_{rL(\ell)}(\gamma)|2^{rdL(\ell)}\leq e2db^\prime_4n 2^{rd2^{r\frac{2\log_2(a)\ell}{r-d-1-\log_2(a)}}}$. Therefore, for any $\C_{rL} \in \C_{rL}(\mathcal{C}_0(n,j))$,
  
\begin{align*}
    \left(\frac{e2d|\C_{rL}|2^{rdL}}{{b_1n}{2^{r\ell}}}\right)^{b_1n2^{-r\ell(d-1)}} &\leq \exp{\left\{{c_4^\prime\frac{n}{2^{r\ell(d-1 - \frac{2\log_2(a)}{r-d-1 -\log_2(a)})}}}\right\}}
\end{align*}
with $c_4^\prime = \ln(2)r(d+\log_{2^r}(\frac{e2db^\prime_4}{b_1}))b_1$.
With our choice of $L(\ell)$, Proposition \ref{Prop: Bound_on_rl_coverings} yields

\begin{align*}
   |\C_{rL(\ell)}(\mathcal{C}_0(n,j))| &\leq \exp{\left\{ b_6  \frac{n}{2^{2r\ell}}{2^{2r\frac{r-d-1}{\log_2(a)}}} \right\}}.
\end{align*}

Applying both inequalities above back in \eqref{Eq: Bound_on_internal_boundary} we get 

\begin{equation}\label{Eq: Bound_on_inner_admissible_cubes_large_l}
    |\fint \mathfrak{C}_{r\ell}(\mathcal{C}_0(n,j))| \leq \exp{\left\{ \left(c_4^\prime +  b_6 {2^{2r\frac{r-d-1}{\log_2(a)}}}\right) \left(\frac{n}{2^{r\ell(d-1-\frac{2\log_2(a)}{r-d-1-\log_2(a)})}} + \frac{n}{2^{2r\ell}}\right) \right\}}.
\end{equation}

For $\ell \leq (\frac{r-d-1}{\log_2 (a)}-1)(\frac{\log_2 (j)}{2r} + 1)$ and $j>\frac{2}{a^\prime}\left[\frac{r-d-1}{\log_2(a)} -1\right]\left[\frac{\log_2(j)}{2r} + 1\right]$, we take $L(\ell) = \floor{\frac{2\ell}{a^\prime}}$. Then $L(\ell)< j$ and Proposition \ref{Prop. Bound.on.C_rl(gamma)} yields $|\C_{rL(\ell)}(\gamma)|2^{rdL(\ell)} \leq  b_4 n{\left(\frac{2\ell}{a^\prime}\right)^{\frac{d+1}{a + 1 - \frac{1}{d}}}}{2^{r(d-a^\prime)\frac{2}{a^\prime} \ell}}$. Therefore, for any $\C_{rL} \in \C_{rL}(\mathcal{C}_0(n,j))$,
\begin{align*}%\label{Eq: Bound_on_choices_add_cubes_small_l}
\left(\frac{e2d|\C_{rL}|2^{rdL}}{{b_1n}{2^{r\ell}}}\right)^{\frac{b_1n}{2^{r\ell(d-1)}}} &\leq \exp{\left\{c_4^{\prime\prime}\frac{\ell n}{2^{r\ell(d-1)}}\right\}},
\end{align*}

for $c_4^{\prime\prime}\coloneqq \ln(2)r(d - a^\prime + \log_{2^r}(\frac{e2db_4}{b_1} + \frac{d+1}{a+1-\frac{1}{d}}\log_{2^r}(\frac{2}{a^\prime})))\frac{2}{a^\prime}b_1$. With our choice of $L(\ell)$, Proposition \ref{Prop: Bound_on_rl_coverings} yields
\begin{align*}
    |\C_{rL(\ell)}(\mathcal{C}_0(n,j))| \leq \exp{\left\{b_6\left(\frac{2\ell}{a^\prime}\right)^{\kappa}2^{ra^\prime}\left(\frac{n}{2^{2r\ell}} + 1 \right)\right\}}.
\end{align*}

Applying both inequalities above back in \eqref{Eq: Bound_on_internal_boundary} we get 
\begin{equation}\label{Eq: Bound_on_inner_admissible_cubes_small_l}
    |\fint \mathfrak{C}_{r\ell}(\mathcal{C}_0(n,j))| \leq \exp{\left\{ \left(c_4^{\prime\prime} + b_62^{ra^\prime}\left(\frac{2}{a^\prime}\right)^{\kappa}\right)\ell^{\kappa}\left(\frac{n}{2^{r\ell(d-1)}} + \frac{n}{2^{2r\ell}} + 1 \right) \right\}}.
\end{equation}

We are left to consider the case when $j\leq\frac{2}{a^\prime}\left[\frac{r-d-1}{\log_2(a)} -1\right]\left[\frac{\log_2(j)}{2r} + 1\right]$. This happens only if $j\leq \overline{c}_4$, for a constant $\overline{c}_4 \coloneqq \overline{c}_4(\alpha,d)$. Then, $n\leq 2^{r(d+1)j}\leq 2^{r(d+1)\overline{c}_4}$ and therefore $|B_\ell(\mathcal{C}_0(n,j))|\leq e^{c_4^{\prime\prime\prime}}$ for a suitable constant $c_4^{\prime\prime\prime}\coloneqq c_4^{\prime\prime\prime}(\alpha,d)$.

Taking $c_4 \coloneqq c_4^{\prime\prime\prime} + c_4^{\prime\prime} + c_4^{\prime} +  b_6(2^{ra^\prime}\left(\frac{2}{a^\prime}\right)^{\kappa} + 2^{2r\frac{r-d-1}{\log_2(a)}}) +\ln(2)2db_1$, the proposition follows from \eqref{Eq: Replacing_edge_by_inner_boundary}, \eqref{Eq: Bound_on_inner_admissible_cubes_large_l} and \eqref{Eq: Bound_on_inner_admissible_cubes_small_l}.
\end{proof}


\begin{proof}[Proof of Proposition \ref{Prop: Bound.gamma_2}.]
 As $N(\mathcal{C}_0(n,j), \d_2, \epsilon)$ is decreasing in $\epsilon$, we can use Dudley's entropy bound to get
    \begin{align*}
        {\mathbb{E}\left[\sup_{\gamma\in\mathcal{C}_0(n,j)}{\Delta_{\I_-(\gamma)}(h)}\right]} &\leq \int_{0}^\infty \sqrt{\log N(\mathcal{C}_0(n,j), \d_2, \epsilon)}d\epsilon \leq \sum_{\ell=0}^\infty\sqrt{\log N(\mathcal{C}_0(n,j), \d_2, \ell)}\nonumber\\
        %
        &\leq 2\varepsilon b_3 n^{\frac{1}{2}}\sum_{\ell=1}^\infty (2^{\frac{r\ell}{2}} - 2^{\frac{r(\ell-1)}{2}})\sqrt{\log N(\mathcal{C}_0(n,j), \d_2,\varepsilon b_3 2^{\frac{r\ell}{2}}n^{\frac{1}{2}})}.
    \end{align*}
Since $\d_2(\gamma_1,\gamma_2)\leq 2\varepsilon\sqrt{|\I_-(\gamma_1)| + |\I_-(\gamma_2)|}\leq 2\sqrt{2}\varepsilon n^{\frac{1}{2} + \frac{1}{2(d-1)}}$ for any $\gamma_1,\gamma_2\in\mathcal{C}_0(n,j)$, when $\varepsilon b_3 2^{\frac{r\ell}{2}}n^{\frac{1}{2}}\geq 2\sqrt{2}\varepsilon n^{\frac{1}{2} + \frac{1}{2(d-1)}}$, only one ball covers all contours, hence all the terms in the sum above with $\ell\geq k(n)\coloneqq 2\ceil{\log_{2^r}(2\sqrt{2})+\frac{\log_{2^r}(n)}{2(d-1)}}$ are zero. As $N(\mathcal{C}_0(n,j), \d_2,\varepsilon b_3 2^{\frac{r\ell}{2}}n^{\frac{1}{2}})\leq |B_{\ell}(\mathcal{C}_0(n,j))|$, see Remark \ref{Rmk: Bounding_N_by_B_ell}, using Proposition \ref{Prop: Bound_on_boundary_of_admissible_sets} we get
\begin{align}\label{Eq: Prop_bound_Ec_1}
        {\mathbb{E}\left[\sup_{\gamma\in\mathcal{C}_0(n,j)}{\Delta_{\I(\gamma)}(h)}\right]} &\leq 2\varepsilon b_3\sqrt{c_4} n^{\frac{1}{2}}\sum_{\ell=1}^{k(n)}2^{\frac{r\ell}{2}}\sqrt{\frac{n\ell^\kappa}{2^{r\ell(d-1-\frac{2\log_2(a)}{r-d-1-\log_2(a)})}} + \frac{n\ell^\kappa}{2^{2r\ell}}+ \ell^\kappa }\nonumber\\
        %
        &\leq  2\varepsilon b_3\sqrt{c_4}\sum_{\ell=1}^{\infty}\left[\frac{\ell^\frac{\kappa}{2}}{2^{\frac{r\ell}{2}(d-2-\frac{2\log_2(a)}{r-d-1-\log_2(a)})}} + \frac{\ell^{\frac{\kappa}{2}}}{2^{r\ell}}\right]n + 2\varepsilon b_3\sqrt{c_4}n^{\frac{1}{2}}\sum_{\ell=1}^{k(n)}2^{\frac{r\ell}{2}}\ell^\frac{\kappa}{2}.
\end{align}

By our choice of $r$, $d-2-\frac{2\log_2(a)}{r-d-1-\log_2(a)}>0$ so the series above converges. Moreover, we can bound
\begin{align*}
    \sum_{\ell=1}^{k(n)}2^{\frac{r\ell}{2}}\ell^\frac{\kappa}{2} &\leq k(n)^{\frac{\kappa}{2}}\frac{2^{\frac{rk(n)}{2}}}{\sqrt{2^r}-1} \\
    %
    &\leq 2^\kappa(\log_{2^r}(2\sqrt{2})+\frac{\log_{2^r}(n)}{2(d-1)} + 1)^\kappa 2^{r(\log_{2^r}(2\sqrt{2})+\frac{\log_{2^r}(n)}{2(d-1)} +1)}\\
    %
    &\leq  2^{\kappa + r}2\sqrt{2}(\log_{2^r}(2\sqrt{2})+\frac{1}{2(d-1)} + 1)^\kappa \log_{2^r}(n)^\kappa n^{\frac{1}{2(d-1)}} \\
    %
    &\leq  2^{\kappa + r + \frac{3}{2}}(\log_{2^r}(2\sqrt{2})+\frac{1}{2(d-1)} + 1)^\kappa n^{\frac{1}{2}}.
\end{align*}
Applying this back in \eqref{Eq: Prop_bound_Ec_1} we conclude that 
\begin{equation*}
       {\mathbb{E}\left[\sup_{\gamma\in\mathcal{C}_0(n,j)}{\Delta_{\I_-(\gamma)}(h)}\right]} \leq \varepsilon L_1^\prime n,
\end{equation*}
with $L_1^\prime\coloneqq   2\varepsilon b_3\sqrt{c_4}\left[ 2^{\kappa + r + \frac{3}{2}}(\log_{2^r}(2\sqrt{2})+\frac{1}{2(d-1)} + 1)^\kappa + \sum\limits_{\ell= 1}^\infty\left(\frac{\ell^\frac{\kappa}{2}}{2^{\frac{r\ell}{2}(d-2-\frac{2\log_2(a)}{r-d-1-\log_2(a)})}} + \frac{\ell^{\frac{\kappa}{2}}}{2^{r\ell}}\right)\right]$. Inequality \eqref{Eq: gamma_2_bounded_by_Dudley_integral} yields the desired result with $L_1\coloneqq L^\prime L_1^\prime$.
\end{proof}


\section{Phase transition}      \begin{theorem}
For $d\geq 3$ and $\alpha>d$, there exists a constant $C\coloneqq C(d,\alpha)$ such that, for all $\beta>0$ and $e\leq C$, the event 
    \begin{equation}\label{Eq: PTLR}
        \nu_{\Lambda; \beta, \varepsilon h}^+(\sigma_0 = -1) \leq e^{-C\beta} + e^{-C/\varepsilon^2} 
    \end{equation}
    has $\mathbb{P}$-probability bigger then $1 - e^{-C\beta} - e^{-C/\varepsilon^2}$.\\
    
In particular, for $\beta>\beta_c$ and $\varepsilon$ small enough, there is phase transition for the long-range Ising model.  
\end{theorem}

\begin{proof}
        The proof is an application of the Peierls' argument, but now on the joint measure $\mathbb{Q}$. Define $\mathcal{E} = \mathcal{E}_0 \cap \mathcal{E}_1$. By Proposition \ref{Prop: Bound.bad.event.0}, we have
        \begin{align}\label{Eq: Upper.bound.on.Q.1}
            \mathbb{Q}_{\Lambda; \beta, \varepsilon}^+(\sigma_0 = -1) &=  \mathbb{Q}_{\Lambda; \beta, \varepsilon}^+(\sigma_0 = -1 \cap \mathcal{E}_0) + \mathbb{Q}_{\Lambda; \beta, \varepsilon}^+(\sigma_0 = -1\cap \mathcal{E}_0^c) \nonumber \\
            %
            & \leq \mathbb{Q}_{\Lambda; \beta, \varepsilon}^+(\sigma_0 = -1 \cap \mathcal{E}_0) +  e^{-C_0/\varepsilon^2} \nonumber \\
            %
            & \leq \mathbb{Q}_{\Lambda; \beta, \varepsilon}^+(\sigma_0 = -1 \cap \mathcal{E}) + \mathbb{Q}_{\Lambda; \beta, \varepsilon}^+(\sigma_0 = -1 \cap \mathcal{E}_0 \cap \mathcal{E}_{1}^c)  + e^{-C_0/\varepsilon^2} \nonumber \\
            %
            & \leq \mathbb{Q}_{\Lambda; \beta, \varepsilon}^+(\sigma_0 = -1 \cap \mathcal{E}) + e^{-C_1/\varepsilon^2}  + e^{-C_0/\varepsilon^2},
        \end{align}
since $\mathbb{Q}_{\Lambda; \beta, \varepsilon}^+(\sigma_0 = -1\cap \mathcal{E}_0^c) \leq \mathbb{Q}_{\Lambda; \beta, \varepsilon}^+(\mathcal{E}_0^c) = \mathbb{P}(\mathcal{E}_0^c)$ and, analogously, ${\mathbb{Q}_{\Lambda; \beta, \varepsilon}^+(\sigma_0 = -1 \cap \mathcal{E}_0 \cap \mathcal{E}_{1}^c) \leq \mathbb{P}(\mathcal{E}_1^c)}$.  When $\sigma_0 = -1$, there must exist a contour $\gamma$ with $0\in V(\gamma)$, hence
\begin{equation*}
    \nu_{\Lambda; \beta, \varepsilon h}^+(\sigma_0 = -1) \leq \sum_{\gamma \in \mathcal{C}_0}\nu_{\Lambda; \beta, \varepsilon h}^+(\Omega(\gamma)),
\end{equation*}
where $\Omega(\gamma) \coloneqq \{\sigma\in\Omega : \gamma \subset \Gamma(\sigma)\}$. So we can write

\begin{align}\label{Eq: Upper.bound.on.Q.2}
    \mathbb{Q}_{\Lambda; \beta, \varepsilon}^+(\sigma_0 = -1 \cap \mathcal{E}) &= \int_{\mathcal{E}}\sum_{\sigma : \sigma_0 = -1}g_{\Lambda; \beta, \varepsilon}^+(\sigma, h)dh \nonumber \\
    %
    &\leq  \sum_{\gamma\in\mathcal{C}_0} \int_{\mathcal{E}}\sum_{\sigma\in\Omega(\gamma)}g_{\Lambda; \beta, \varepsilon}^+(\sigma, h)dh \nonumber \\
    %
    &\leq  \sum_{\gamma \in \mathcal{C}_0} \frac{2^{|\gamma|}\int_{\mathcal{E}}\sum_{\sigma\in\Omega(\gamma)}g_{\Lambda; \beta, \varepsilon}^+(\sigma, h)dh}{\int_{\mathcal{E}}\sum_{\sigma\in\Omega(\gamma)}g_{\Lambda; \beta, \varepsilon}^+(\tau_{\gamma}(\sigma), \tau_{\I_-(\gamma)}(h))dh} \nonumber \\
    %
    & \leq \sum_{\gamma\in\mathcal{C}_0}2^{|\gamma|} \sup_{\substack{h\in\mathcal{E}\\ \sigma\in\Omega(\gamma)}}\frac{g_{\Lambda; \beta, \varepsilon}^+(\sigma, h)}{g_{\Lambda; \beta, \varepsilon}^+(\tau_{\gamma}(\sigma), \tau_{\I_-(\gamma)}(h))}. 
\end{align}

In the third equation, we used that $\int_{\mathcal{E}}\sum_{\sigma\in\Omega(\gamma)}g_{\Lambda; \beta, \varepsilon}^+(\tau_{\gamma, \sigma}(\sigma), \tau_{\I_-(\gamma)}(h))dh \leq 2^{|\gamma|}$, since the number of configurations that are incorrect in $\Sp(\gamma)$ are bounded by $2^{|\gamma|}$. By \eqref{Eq: quotient.of.gs} and the definition of the event $\mathcal{E}$, 
\begin{align}\label{Eq: Upper.bound.on.Q.3}
    \sup_{\substack{h\in\mathcal{E}\\ \sigma\in\Omega(\gamma)}}\frac{g_{\Lambda; \beta, \varepsilon}^+(\sigma, h)}{g_{\Lambda; \beta, \varepsilon}^+(\tau_{\gamma, \sigma}(\sigma), \tau_{\I_-(\gamma)}(h))} &\leq \sup_{\substack{h\in\mathcal{E}\\ \sigma\in\Omega(\gamma)}}  \exp{\{{- \beta c_2 |\gamma| -2\beta\sum_{x\in \Sp^-(\gamma)}\varepsilon h_x}\}}\frac{Z_{\Lambda; \beta, \varepsilon}^{+}(\tau_{\I_-(\gamma)}(h))}{Z_{\Lambda; \beta, \varepsilon}^{+}(h)} \nonumber\\
    %
    &= \sup_{\substack{h\in\mathcal{E}\\ \sigma\in\Omega(\gamma)}}  \exp{\{{- \beta c_2 |\gamma| -2\beta\sum_{x\in \Sp^-(\gamma)}\varepsilon h_x + \beta \Delta_{\gamma}(h)}\}} \nonumber\\
    %
    &\leq  \exp{\left\{{- \beta \frac{c_2}{2} |\gamma| }\right\}},
\end{align}
since $\Delta_{\gamma}(h) -2\beta\sum_{x\in \Sp^-(\gamma)}\varepsilon h_x \leq \frac{c_2}{2}|\gamma|$, for all $h\in\mathcal{E}$. Equations \eqref{Eq: Upper.bound.on.Q.1}, \eqref{Eq: Upper.bound.on.Q.2} and \eqref{Eq: Upper.bound.on.Q.3} yields
\begin{align*}
     \mathbb{Q}_{\Lambda; \beta, \varepsilon}^+(\sigma_0 = -1) &\leq  \sum_{\substack{\gamma\in \mathcal{E}_\Lambda^+\\ 0\in V(\gamma)}} 2^{|\gamma|}e^{{- \beta \frac{c_2}{2} |\gamma| }} + e^{-C_1/\varepsilon^2} + e^{-C_0/\varepsilon^2}\\
     &\leq \sum_{n\geq 1}\sum_{\substack{\gamma\in \mathcal{E}_\Lambda^+, |\gamma|=n \\ 0\in V(\gamma)}} e^{{(-\beta \frac{c_2}{2} + \ln2)n}} + e^{-C_1/\varepsilon^2} + e^{-C_0/\varepsilon^2}\\
     %
     &\leq \sum_{n\geq 1}|\mathcal{C}_0(n)| e^{{(-\beta \frac{c_2}{2} +\ln2)n}} + e^{-C_1/\varepsilon^2} + e^{-C_0/\varepsilon^2}\leq \sum_{n\geq 1} e^{(c_1 -\beta \frac{c_2}{2} +\ln2)n} + e^{-C_1/\varepsilon^2} + e^{-C_0/\varepsilon^2}. \\
\end{align*}
When $\beta$ is large enough, the sum above converges and there exists a constant $C$ such that   
\begin{equation*}
    \mathbb{Q}_{\Lambda; \beta, \varepsilon}^+(\sigma_0 = -1) \leq e^{-\beta 2C} + e^{-2C / \varepsilon^2}.
\end{equation*}
The Markov Inequality finally yields
\begin{align*}
    \mathbb{P}\left( \nu_{\Lambda; \beta, \varepsilon h}^+(\sigma_0 = -1) \geq e^{-C\beta} + e^{-C/\varepsilon^2}\right) &\leq \frac{\mathbb{Q}_{\Lambda; \beta, \varepsilon}^+(\sigma_0 = -1)}{e^{-C\beta} - e^{-C/\varepsilon^2}} \\
    %
    &\leq \frac{e^{-\beta 2C} + e^{-2C / \varepsilon^2}}{e^{-C\beta} - e^{-C/\varepsilon^2}} \leq e^{-C\beta} - e^{-C/\varepsilon^2},
\end{align*}
what proves our claim.
\end{proof}  


\section{Concluding Remarks}

In this paper, we proved phase transition for the long-range Ising model in $d\geq 3$ and $\alpha >d$, by following a new method of proving phase transition introduced by Ding and Zhuang \cite{Ding2021}, and using the contours defined in \cite{Affonso.2021}. The key part of the argument was to extend the results of \cite{FFS84} to contours that are not necessarily connected. This proof can be extended to other models with a contour system, as long as the probability of the event $\mathcal{E}_1^c$ decreases to zero for large $\varepsilon$.

The results presented by Bricmont and Kupiainen \cite{Bricmont.Kupiainen.88} are more general than ours since they only need the external field to be symmetric around zero and have a sub-Gaussian tail. In \cite{Ding2021}, Ding and Zhuang claim that it should be possible, with more care, to extend their results to an external field in the same generality.

The natural question is to investigate the smaller dimensions $d=1, 2$. Aizenman and Wehr, in \cite{Aizenman.Wehr.90}, proved that for $d\leq 2$ there is uniqueness when $\alpha > 3d/2$. This shows that the bound on $\mathbb{P}(\mathcal{E}_1^c)$ should depend on $\alpha$.


\section*{Acknowledgements}

LA is supported by FAPESP Grants 2017/18152-2, 2020/14563-0, and 2023/00854-1. RB is supported by CNPq grants 312294/2018-2 and 408851/2018-0, by FAPESP grant 16/25053-8, and by the University Center of Excellence \textquotedblleft Dynamics, Mathematical Analysis and Artificial Intelligence\textquotedblright, at the Nicolaus Copernicus University; JM is supported by FAPESP Grant 2018/26698-8 and 2016/25053-8.  The authors thank Kelvyn Welsch and Jo\~ao Rodrigues for the careful reading of the previous versions of the paper and Abel Klein for discussions about earlier results in the literature concerning multiscale analysis. JM and LA are very grateful to Eric Endo for the support and hospitality during their first visit to China and NYU-Shanghai; they also thank Weijun Xu, as well as Jian Ding, for the support and hospitality during their visit to Peking University, especially to Professor Ding for fruitful discussions.
	

\bibliographystyle{habbrv} 
\bibliography{bib}

\end{document}
