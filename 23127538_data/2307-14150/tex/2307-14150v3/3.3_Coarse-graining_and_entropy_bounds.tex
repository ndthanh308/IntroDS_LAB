We will apply this probability estimations for the family $(|\Delta_{\I_-(\gamma)}|)_{\gamma\in\mathcal{C}_0(n,j)}$. To construct the covering by balls in Dudley's entropy bound, we use the coarse-graining idea introduced in \cite{FFS84}.  For each $\ell>0$ and each contour ${\gamma\in\mathcal{C}(n,j)}$, we will associate a region $B_\ell(\gamma)$ that approximates the interior $\I(\gamma)$ in a scaled lattice, with the scale growing with $\ell$. This is done in a way that two contours that are approximated by the same region are in a ball in distance $\d_2$ with a fixed radius, depending on $\ell$.

An $r\ell$-cube $C_{r\ell}$ is \textit{admissible} if more than a  half of its points are inside $\I_-(\gamma)$. Thus, the set of admissible cubes is
\begin{equation*}
    \mathfrak{C}_\ell(\gamma) \coloneqq \{C_{r\ell} : |C_{r\ell}\cap \I_-(\gamma)| \geq \frac{1}{2}|C_{r\ell}|\}.
\end{equation*}
With this notion of admissibility, two contours with the same admissible cubes should be close in distance $d_2$. Consider functions $B_\ell:\mathcal{E}^+_\Lambda \xrightarrow[]{} \mathcal{P}(\Lambda)$ that takes contours $\gamma$ to $B_\ell(\gamma) \coloneqq B_{\mathfrak{C}_{\ell}(\gamma)}$, the region covered by the admissible cubes. We will be interest in counting  the image of $B_\ell$ by $\mathcal{C}_0(n,m,j)$, that is, $|B_\ell(\mathcal{C}_{0}(n,m,j))| = |\{ B : B = B_\ell(\gamma) \text{ for some }\gamma \in \mathcal{C}_0(n,m,j)\}|$. Notice that $B_\ell(\gamma)$ is uniquely determined by $\partial B_\ell(\gamma)$. Given any collection $\C_{m}$, we define the \textit{edge boundary of } $\C_m$ as 
$$
\partial \C_m(\gamma) \coloneqq \{ \{C_{m}, C^\prime_{m}\} : C_{m} \in \C_m, \ C_m^\prime \notin \C_m \textrm{ and} \  C_m^\prime \text{ shares a face with }C_m\}.
$$ 
We also define the \textit{inner boundary of }$\C_m$ as
$$
\fint \C_m(\gamma) \coloneqq \{ C_{m}\in \C_m : \exists C_m^\prime \notin \C_m \textrm{ such that }  \{C_m,C_m^\prime\}\in\partial \C_m\}.
$$ 
With this definition, it is clear that $\partial B_\ell(\gamma)$ is uniquely determined by $\partial \mathfrak{C}_\ell(\gamma)$. We will now control the number of cubes in $\mathfrak{C}_\ell(\gamma)$. This proposition was written for $d=3$ and $\I_-(\gamma)$ simply connected for all contours, but it can be extended to $d\geq 2$ with no restriction on the interiors, see \cite{Bovier.06}. As we could not find a detailed proof anywhere, we provide one here.

Given a rectangle $\mathcal{R} = [1,r_1]\times[1,r_2]\times\dots\times[1,r_d]$, consider $\R_i\coloneqq\{x\in \R : x_i=1\}$ the face of $\R$ that is perpendicular to the direction $e_i$, for $i=1,\dots,d$. The line that connects a point $x\in \R_i$ to a point in the opposite face of $\R_i$ is $\ell_x^i \coloneqq \{ x + ke_i : 1\leq k\leq r_i\}$. Given $A\subset \Z^d$, the projection of $A\cap \R$ into the face $\R_i$ is
\begin{equation*}
    \calP_i(A\cap\R) \coloneqq \{x\in\R_i : \ell_x^i \cap A \neq \emptyset\}.
\end{equation*}
\input{Figure_4.1}
In many situations, we will split the projections into \textit{good} and \textit{bad} points. The set of good points is $\calP_i(A\cap\R)^{G} \coloneqq \{x\in \calP_i(A\cap \R) : \ell_x^i \cap (\R\setminus A) \neq \emptyset\}$, that is, there exist a point in $\ell_x^i\cap \R$ that is not in $A$.  The bad points are defined as $\calP^{B}_i(A\cap\R) \coloneqq \calP_i(A\cap\R)\setminus \calP_i^G(A\cap\R)$.
\input{Figure_5}
Given $x\in \calP_i(A\cap\R)^{G}$, by definition of the projection, there exists a point in $\ell_x^i\cap A$. Therefore, there exists a point $p\in \ell_x^i$ such that $p\in\fext A \cap \R$. As all lines are disjoint, we conclude that 
\begin{equation}\label{Eq: upper.bound.good.points}
     |\calP_i^{G}(A\cap\R)|\leq |\fext A \cap \R|.
\end{equation}
 We now prove two auxiliary lemmas.
 
\begin{lemma}\label{Lemma: Geo.discreta.1}
    Given $d\geq 2$, for any family of positive integers $\bm{r}=(r_i)_{i=1}^d$ with $R\leq r_i \leq 2R$ for some $R\geq 2$, $0<\lambda < 1$ and $A\subset\Z^d$, there exists a constant $c\coloneqq c(d, \lambda)$ such that, if 
    \begin{equation}\label{Eq: hypothesis.lemma.1}
         |\calP_i(A\cap \R)| \leq \lambda|\R_i|
    \end{equation}
    for all $i= 1,\dots, d$, then 
    \begin{equation*}
        \sum_{i=1}^d |\calP_{i}(A\cap \R)|\leq c|\fext A\cap \R|,
    \end{equation*}
    where $\R=[1,r_1]\times\dots\times [1,r_d]$.
\end{lemma}

\begin{proof}
The proof will be done by induction on the dimension. For $d=2$, take a rectangle ${\R=[1,r_1]\times[1,r_2]}$. If there is no bad points in $\calP_1(A\cap\R)$, then 
\begin{align}\label{Eq: Bound.1.on.P.1}
    |\calP_1(A\cap\R)| = |\calP_1^G(A\cap \R)| \leq |\fext A \cap \R|.
\end{align}

If there is a bad point $p=(1,p_2)\in \calP_1^B(A\cap\R)$, $\ell_p^1\subset A\cap \R$  by definition of bad point. As $|\calP_1(A\cap \R)| \leq \lambda|\R_1| < |\R_1|$, there is a point $p^\prime = (1,p_2^\prime)\in \R_1\setminus \calP_1(A\cap \R)$ that is in the face $\R_1$ but not in the projection. By definition of the projection, $\ell_{p^\prime}^1\in A^c\cap \R$. Therefore, for any $1\leq k\leq r_1$, $(k,p_2)\in  A\cap \R$ and $(k,p^\prime_2)\in  A^c\cap \R$, we can find a point $p^k=(k, p^k_2) \in \fext A \cap \R$. Since $p^{k_1}\neq p^{k_2}$ for every $k_1\neq k_2$, we have $r_1 \leq |\fext A \cap \R|$, hence
\begin{equation}\label{Eq: Bound.2.on.P.1}
   |P_1(A\cap \R)| \leq  |\R_1| = {r_2}\leq  2R \leq 2r_1 \leq  2|\fext A \cap \R|.
\end{equation}
A completely analogous argument can be done to bound $|P_2(A\cap \R)|$, and we conclude that
\begin{equation*}
    \sum_{i=1}^2|\calP_i(A\cap \R)|\leq 4|\fext A \cap \R|,
\end{equation*}
and take $c(2,\lambda)=4$. Suppose the lemma holds for $d-1$ and fix a rectangle $\R=[1,r_1]\times\dots\times[1,r_d]$. We split $\R$ into layers $L_k = \{x\in\Z^d : x_d = k\}$, for $k=1,\dots, r_d$. We can then partition the projection and write
\begin{equation*}
|\calP_i(A\cap \R)| = \sum_{k=1}^{r_d} |\calP_i(A\cap \R)\cap L_k|,    
\end{equation*}
for any $i\in\{1,\dots, d-1\}$. This yields
\begin{align}\label{Eq: Partition.sum.proj.}
    \sum_{i=1}^d|\calP_i(A\cap \R)| &= \sum_{i=1}^{d-1}\sum_{k=1}^{r_d}|\calP_i(A\cap \R)\cap L_k| + |\calP_d(A\cap \R)| \nonumber \\
    &=  \sum_{k=1}^{r_d}\sum_{i=1}^{d-1}|\calP_i(A\cap \R)\cap L_k| + |\calP_d(A\cap \R)|.
\end{align}

Notice now that $\calP_i(A\cap \R)\cap L_k = \calP_i(A\cap (\R\cap L_k))$. Defining the rectangle $\R^k \coloneqq \R\cap L_k$, for every point $p\in \calP_j^B(A\cap \R^k)$, $\ell_p^j \subset A\cap \R^k$. Moreover, we can associate every point $x\in \ell_p^j$ in the line with a point $x^\prime\in \calP_d(A\cap\R)$ by taking $x_m^\prime = x_m$ for $m \leq d-1$ and $x_d^\prime = 1$, therefore

\begin{equation*}
    r_j|\calP_j^B(A\cap \R^k)| = \sum_{p\in \calP_j^B(A\cap \R^k)}|\ell_p^j| \leq |\calP_d(A\cap\R)|.
\end{equation*}
\input{Figure_6}
Using the hypothesis \eqref{Eq: hypothesis.lemma.1} we conclude that

\begin{equation}\label{Eq: upper.bound.projection.i.bad.points}
     |\calP_j^B(A\cap \R^k)| \leq \lambda\frac{|\R_d|}{r_j} = \lambda \frac{ \prod_{q\neq d}r_q}{r_j} =  \lambda \prod_{q\neq j,d}r_q = \lambda |(\R^k)_j|.
\end{equation}
We consider two cases:
    \begin{itemize}
        \item[(a)] If $|\calP_i(A\cap \R^k)| \leq \frac{\lambda +1}{2}|(\R^k)_i|$, for all $i\leq d-1$, then we are in the hypothesis of the lemma in $d-1$ and therefore
\begin{equation}\label{Eq: Primeiro.bound.soma.projecoes}
    \sum_{i=1}^{d-1} |\calP_i(A\cap \R^k)| \leq c\left(d-1, \frac{\lambda + 1}{2}\right)|\fext A\cap \R^k|.
\end{equation}
    \item [(b)] If there exists $j\in\{1,\dots,d-1\}$ satisfying $|\calP_j(A\cap \R^k)| > \frac{\lambda +1}{2}|(\R^k)_j|$, by \eqref{Eq: upper.bound.projection.i.bad.points} we have $|\calP_j^G(A\cap \R^k)| = |\calP_j(A\cap \R^k)| - |\calP_j^B(A\cap \R^k)| \geq \frac{1-\lambda}{2}|(\R^k)_j|$, hence
\begin{equation*}
    |(\R^k)_j| \leq  \frac{2}{1 - \lambda}|\fext A \cap \R^k|.
\end{equation*}
    Using that $|(\R^k)_i| \leq (2R)^{d-2} \leq 2^{d-2}|(\R^k)_j|$ for every $i\in\{1,\dots,d\}$, we conclude that 
\begin{equation}\label{Eq: Segundo.bound.soma.projecoes}
    \sum_{i=1}^{d-1} |P_i(A\cap\R^k)| \leq \sum_{i=1}^{d-1} |(\R^k)_i| \leq (d-1)2^{d-2}|(\R^k)_j| \leq \frac{(d-1)2^{d-1}}{1 - \lambda}|\fext A \cap \R^k|.
\end{equation}
\end{itemize}


In both cases, we were able to bound the sum of projections by a constant times the size of the boundary of $A$ in $\R^k$. Applying \eqref{Eq: Primeiro.bound.soma.projecoes} and \eqref{Eq: Segundo.bound.soma.projecoes} back in \eqref{Eq: Partition.sum.proj.} we get
\begin{align*}
    \sum_{i=1}^d|\calP_i(A\cap \R)| &\leq
    \sum_{k=1}^{r_d}\left[c\left(d-1, \frac{\lambda + 1}{2}\right)+ \frac{(d-1)2^{d-1}}{1 - \lambda}\right]|\fext A \cap \R \cap L_k| + |\calP_d(A\cap \R)|\\
    %
    &=\left[c\left(d-1, \frac{\lambda + 1}{2}\right)+ \frac{(d-1)2^{d-1}}{1 - \lambda}\right]|\fext A \cap \R| + |\calP_d(A\cap \R)|.
\end{align*}
We finish the proof by noticing that we can repeat this same argument but now splitting $\R$ into layers $L_k = \{x\in \R : x_j = k\}$. Doing so, we have that  
\begin{equation*}
    \sum_{i=1}^d|\calP_i(A\cap \R)| \leq
     \left[c\left(d-1, \frac{\lambda + 1}{2}\right)+ \frac{(d-1)2^{d-1}}{1 - \lambda}\right]|\fext A \cap \R| + |\calP_j(A\cap \R)|
\end{equation*}
for any $j\in\{1,\dots, d\}$. Summing both sides in $j$ we conclude

\begin{equation}
    \sum_{i=1}^d|\calP_i(A\cap \R)| \leq \frac{d}{d-1}\left[c\left(d-1, \frac{\lambda + 1}{2}\right)+ \frac{(d-1)2^{d-1}}{1 - \lambda}\right]|\fext A \cap \R|,
\end{equation}
which proves our claim if we take $c(d,\lambda) \coloneqq \frac{d}{d-1}\left[c(d-1, \frac{\lambda + 1}{2})+ \frac{(d-1)2^{d-1}}{1 - \lambda}\right] = 2d + \frac{(d-2)d2^{d-1}}{1-\lambda}$.
\end{proof}

\begin{remark}
This lemma can be proved when $R\leq r_i \leq \kappa R$ for any $\kappa>1$. When applying the lemma, we will choose $\lambda=\frac{7}{8}$ to simplify the notation. All the proofs work as long as we choose $\lambda> \frac{3}{4}$.
\end{remark}

\begin{lemma}\label{Lemma: Proposicao1.Aux1}
    Given $A\subset \Z^d$, $\ell\geq 0$ and $U= C_{r\ell}\cup C_{r\ell}^\prime$ with $C_{r\ell}$ and $C_{r\ell}^\prime$ being two $r\ell$-cubes sharing a face, there exists a constant $b\coloneqq b(d)$ such that, if 
    
\begin{align}\label{Eq. U.condition}
    \frac{2^{r\ell d}}{2} \leq |C_{r\ell}\cap A| \qquad \text{and} \qquad |C_{r\ell}^\prime\cap A|< \frac{2^{r\ell d}}{2}
\end{align}
then $2^{r\ell(d-1)}\leq b|\fext A\cap U|$.
\end{lemma}


\begin{proof}
For $\ell=0$, \eqref{Eq. U.condition} guarantees that $C_{r\ell} = \{x\} \subset A$ and $C_{r\ell}^\prime = \{y\}\subset A^c$, hence $|\fext A\cap \{x,y\}| = 1$ and it is enough to take $b\geq 1$. For $\ell \geq 1$, \eqref{Eq. U.condition} yields
\begin{equation}\label{Eq. A.cap.U.volume}
    \frac{1}{2}2^{r\ell d} \leq |A\cap U| \leq \frac{3}{2}2^{r\ell d}.
\end{equation}

To simplify the notation, we can assume wlog that ${U=[1,2^{r\ell}]^{d-1}\times [1, 2^{r\ell+1}]}$. As discussed before, for each point $p\in\calP^B_j(A\cap U)$ in the projection, $\ell_p^j\subset A\cap U$ and the lines are disjoint. Moreover, the size of the lines  is constant $r_j\coloneqq |\ell_p^j|$, hence $|\calP_{j}^B(A\cap U)|r_j = \sum_{p\in\calP_{j}^B(A\cap U)} |\ell_p^j| \leq |A\cap U|$. Together with the upper bound \eqref{Eq. A.cap.U.volume}, this yields 
\begin{equation}\label{Eq: Upper.bound.bad.points}
    |\calP_{j}^B(A\cap U)| \leq \frac{3}{2}2^{r\ell d}r_j^{-1}.
\end{equation}
Using the isometric inequality, the lower bound on \eqref{Eq. A.cap.U.volume} yields $d2^{\frac{1}{d}}2^{r\ell(d-1)}\leq |\fext (A\cap U)|$. As %
%
\begin{align*}
\frac{1}{2d}|\fext (A\cap U)| &\leq |\fint (A\cap U)| = |\fint(A\cap U) \cap \fint U| + |\fint(A\cap U) \cap(U\setminus \fint U)|\\
%
&\leq 2\sum_{i=1}^d |\calP_{i}(A\cap U)| + |\fint A\cap U| \leq 2\sum_{i=1}^d |\calP_{i}(A\cap U)| + |\fext A\cap U|,
\end{align*}
we get
\begin{equation}\label{Eq: Lemma.geo.discreta.3}
    2^{\frac{1}{d}-1}2^{r\ell(d-1)}\leq 2\sum_{i=1}^d |\calP_{i}(A\cap U)| + |\fext A\cap U|
\end{equation}

We again consider two cases:
\begin{itemize}
    \item[(a)] If $|\calP_{j}(A\cap U)|> \frac{7}{8}|U_j|$ for some $j=1,\dots, d$, by \eqref{Eq: Upper.bound.bad.points} and \eqref{Eq: upper.bound.good.points} we get
    \begin{align*}
    \frac{7}{8}|U_j| < |\calP_{j}(A\cap U)| \leq |\fext A \cap U| + \frac{3}{2}2^{\ell d}r_j^{-1}.
    \end{align*}
    A simple calculation shows that $\frac{1}{8}2^{\ell(d-1)}\leq \frac{7}{8}|U_j| - \frac{3}{2}2^{\ell d}r_j^{-1}$, therefore 
    \begin{equation}\label{Eq: upper.bound.big.projections.1}
        \frac{1}{8}2^{\ell(d-1)} \leq |\fext A \cap U|.
    \end{equation}
    \item[(b)] If $|\calP_{i}(A\cap U)|\leq \frac{7}{8} |U_i|$ for all $i$, by Lemma \ref{Lemma: Geo.discreta.1}, there is a constant $c= c(d)$ such that
    \begin{equation}\label{Eq: Lemma.geo.discreta.2}
        \sum_{i=1}^d |\calP_{i}(A\cap U)|\leq c|\fext A\cap U|.
    \end{equation}
    Together with \eqref{Eq: Lemma.geo.discreta.3}, this yields
    \begin{equation}\label{Eq: upper.bound.big.projections.2}
        2^{\ell(d-1)}\leq \frac{2c+1}{2^{\frac{1}{d}-1}}|\fext A\cap U|.
    \end{equation}
\end{itemize}

Equations \eqref{Eq: upper.bound.big.projections.1} and \eqref{Eq: upper.bound.big.projections.2} shows the desired results taking $b\coloneqq \max \{8, {(2c+1)}{2^{1-\frac{1}{d}}}\}$.
\end{proof}

\begin{proposition}\label{Proposition1}For the functions $B_0,\dots,B_k$ defined above, there exists constants $b_1,b_2$ depending only on $d$ and $r$ such that 
\begin{equation}\label{Eq: Prop.1.FFS.i}
    |\partial\mathfrak{C}_\ell(\gamma)| \leq b_1\frac{|\fext \I_-(\gamma)|}{2^{r\ell(d-1)}} \leq b_1 \frac{|\gamma|}{2^{r\ell(d-1)}}
\end{equation}
    and 
\begin{equation}\label{Eq: Prop.1.FFS.ii}
    |B_\ell(\gamma)\Delta B_{\ell+1}(\gamma)| \leq b_2 2^{r\ell} |\gamma|
\end{equation}
for every $\ell\in\{0,\dots,k\}$ and $\gamma\in\mathcal{C}_0(n)$.
\end{proposition}

    \begin{proof} Fix $\ell\in\{0,\dots,k\}$. To each cube $C_{r\ell}\in  \partial \mathfrak{C}_\ell(\gamma)$ there is an $r\ell$-cube $C_{r\ell}^\prime \not\in  \mathfrak{C}_\ell(\gamma)$ not admissible, sharing a face with  $C_{r\ell}$. We denote this relation by $C_{r\ell} \sim C_{r\ell}^\prime$, and the union $U = C_{r\ell} \cup C_{r\ell}^\prime$.  Considering the collection of $r\ell$-cubes $\overline{\mathscr{C}}_{r\ell} = \{C_{r\ell} : C_{r\ell}\in \partial \mathfrak{C}_\ell(\gamma) \text{ or } C_{r\ell}\notin  \mathfrak{C}_\ell(\gamma)\}$ and $A\Subset\Z^d$, 
    
    \begin{align*}
        \sum_{\substack{C_{r\ell}\in \partial \mathfrak{C}_\ell(\gamma)}}\sum_{\substack{C_{r\ell}^\prime\notin  \mathfrak{C}_\ell(\gamma)\\ C_{r\ell} \sim C_{r\ell}^\prime}} |A \cap \{C_{r\ell} \cup C_{r\ell}^\prime\}| &\leq \sum_{\substack{C_{r\ell}\in \partial \mathfrak{C}_\ell(\gamma)}}\sum_{\substack{C_{r\ell}^\prime\notin  \mathfrak{C}_\ell(\gamma)\\ C_{r\ell} \sim C_{r\ell}^\prime}}  \left(|A \cap C_{r\ell}| + |A \cap C_{r\ell}^\prime|\right)\\
        %
        &\leq \sum_{\substack{C_{r\ell}\in \partial \mathfrak{C}_\ell(\gamma)}} 2d |A \cap C_{r\ell}| + \sum_{C_{r\ell}^\prime\notin \mathfrak{C}_\ell(\gamma)} 2d|A \cap C_{r\ell}^\prime| \\
        %
        &= 2d \sum_{C\in\overline{\mathscr{C}}_{r\ell}} |A\cap C| = 2d|A\cap B_{\overline{\mathscr{C}}_{r\ell}}| \leq 2d |A|
    \end{align*}
    
Any pair of cubes $C_{r\ell}\sim C_{r\ell}^\prime$ are in the hypothesis of Lemma \ref{Lemma: Proposicao1.Aux1}, hence $ b 2^{r\ell(d-1)} \leq |\fext \I_-(\gamma) \cap U|$. Applying equation above for $A=\fext\I_-(\gamma)$ we get that
    \begin{equation*}
       \frac{b}{2d} 2^{r\ell(d-1)}| \partial \mathfrak{C}_\ell(\gamma)| \leq \frac{1}{2d}\sum_{\substack{C_{r\ell}\in \partial \mathfrak{C}_\ell(\gamma)}}\sum_{\substack{C_{r\ell}^\prime\notin  \mathfrak{C}_\ell(\gamma)\\ C_{r\ell} \sim C_{r\ell}^\prime}}  |\fext \I_-(\gamma) \cap \{C_{r\ell} \cup C_{r\ell}^\prime\}| \leq |\fext \I_-(\gamma)|,
    \end{equation*}
that concludes \eqref{Eq: Prop.1.FFS.i} for $b_1\coloneqq 2d/b$.

Given $C_{r(\ell+1)}\in \mathscr{C}_{r(\ell+1)}(B_{\ell+1}(\gamma)\setminus B_{\ell}(\gamma))$, there is a $r\ell$-cube $C_{r\ell}^\prime\subset C_{r(\ell + 1)}$ with $C_{r\ell}^\prime\notin \mathfrak{C}_\ell(\gamma)$, otherwise  $(B_{\ell+1}(\gamma)\setminus B_{\ell}(\gamma))\cap C_{r(\ell+1)} = \emptyset$. There is also a $r\ell$-cube $C_{r\ell}\subset C_{r(\ell + 1)}$ with $C_{r\ell}\in \mathfrak{C}_\ell(\gamma)$, otherwise we would have 
\begin{align*}
    |\I_-(\gamma)\cap C_{r(\ell+1)}| &= \sum_{C_{r\ell}\subset C_{r(\ell+1)}} |\I_-(\gamma)\cap C_{r\ell}| \leq \frac{1}{2} |C_{r(\ell+1)}|.
\end{align*}

Moreover, we can assume that $C_{r\ell}$ and $C_{r\ell}^\prime$ share a face. Again, we use Lemma \ref{Lemma: Proposicao1.Aux1} to get,
\begin{align}\label{Eq: bound.on.c.bar}
    |B_{\ell+1}(\gamma)\setminus B_\ell(\gamma)\cap C_{r(\ell+1)}| &\leq |C_{r(\ell+1)}| =2^{rd}2^{r\ell}2^{r\ell(d-1)} \nonumber\\
                %
                &\leq 2^{rd}2^{r\ell} b|\fext \I_-(\gamma) \cap \{C_{r\ell} \cup C_{r\ell}^\prime\}| \nonumber\\
                %
                &\leq 2^{rd}2^{r\ell}b|\fext \I_-(\gamma) \cap C_{r(\ell+1)}|.
\end{align}
Therefore, 
\begin{align*}
    |B_{\ell+1}(\gamma)\setminus B_\ell(\gamma)| &= \sum_{C_{r(\ell+1)}\in \mathscr{C}_{r(\ell+1)}(B_{\ell+1}(\gamma)\setminus B_\ell(\gamma))} |B_{\ell+1}(\gamma)\setminus B_\ell(\gamma)\cap C_{r(\ell+1)}| \\
    %
    &\leq \sum_{C_{r(\ell+1)}\in \mathscr{C}_{r(\ell+1)}(B_{\ell+1}(\gamma)\setminus B_\ell(\gamma))} 2^{rd}2^{r\ell}b|\fext \I_-(\gamma) \cap C_{r(\ell+1)}| \leq  \frac{b_2}{2}2^{r\ell}|\fext \I_-(\gamma)|.
\end{align*}
with $b_2=b2^{rd+1}$. To get the same bound for $|B_{\ell}(\gamma)\setminus B_{\ell+1}(\gamma)|$ we repeat a similar argument, covering $B_{\ell}(\gamma)\setminus B_{\ell+1}(\gamma)$ with $r(\ell+1)$-cubes. 
    \end{proof}

\begin{corollary}
    For any $\ell>0$ and any two contours $\gamma_1,\gamma_2 \in \mathcal{C}_0(n)$ such that $B_\ell(\gamma_1)=B_{\ell}(\gamma_2)$, there exists a constant $b_3>0$ such that 
    \begin{equation*}
        \d_2(\gamma_1,\gamma_2)\leq 4 \varepsilon b_3 2^{\frac{\ell}{2}} n^{\frac{1}{2}}. 
    \end{equation*} 
\end{corollary}

\begin{proof}
    This is a simple application of the triangular inequality, since $d_2(\gamma_1,\gamma_2) \leq d_2(\gamma_1,B_\ell(\gamma_1)) + d_2(\gamma_2,B_\ell(\gamma_2))$ and 
    \begin{align*}
        d_2(\gamma_1,B_\ell(\gamma_1)) &\leq \sum_{i=1}^\ell d_2(B_i(\gamma_1),B_{i-1}(\gamma_1)) = \sum_{i=1}^\ell 2\varepsilon\sqrt{B_i(\gamma_1)\Delta B_{i-1}(\gamma_1)} \\
        %
        & \leq \sum_{i=1}^\ell 2\varepsilon{b_2}^{\frac{r}{2}} 2^{\frac{ri}{2}} \sqrt{n}  \leq 2\varepsilon{b_2}^{\frac{r}{2}}(2^{\frac{r}{2}}+1)2^{\frac{r\ell}{2}} \sqrt{n} 
    \end{align*}
    where in the second to last equation used \eqref{Eq: Prop.1.FFS.ii}. As the same bound holds for $d_2(\gamma_2,B_\ell(\gamma_2))$, the corollary is proved by taking ${b_3 = {b_2}^{\frac{r}{2}}(2^{\frac{r}{2}}+1)}$.
\end{proof}

\begin{remark}\label{Rmk: Bounding_N_by_B_ell}
    This corollary shows that we can create a covering of $\mathcal{C}_0(n)$, indexed by $B_\ell(\mathcal{C}_0(n))$, of balls with radius $4 \varepsilon b_3 2^{\frac{r\ell}{2}} n^{\frac{1}{2}}$. Therefore $N(\mathcal{C}_0(n), \d_2, 4\varepsilon b_3 2^{\frac{r\ell}{2}} n^{\frac{1}{2}}) \leq |B_\ell(\mathcal{C}_0(n))|$. 
\end{remark}
In the next section we we bound $|B_\ell(\mathcal{C}_0(n,j))|$, using a method similar to the one used in \cite{Affonso.2021} to count $|\mathcal{C}_0(n)|$.

