To control the probability of $\mathcal{E}_1^c$, we use some results on Majorising measure. For an extensive overview, we refer to \cite{Talagrand_14}. Consider $(T,d)$ a metric space and a process $(X_t)_{t\in T}$ such that, for every $\lambda>0$ and $t,s\in T$,
\begin{equation}\label{Eq: Sub_gaussian_def}
    \mathbb{P}\left( |X_t - X_s| \geq \lambda \right) \leq 2\exp{\frac{-\lambda^2}{2d(s,t)^2}}.
\end{equation}
Assume also that $\mathbb{E}\left(X_t\right) = 0$ for every $t\in T$. One example of such process is $( |\Delta_{\I_-(\gamma)}|)_{\gamma\in\mathcal{C}_0}$ with the distance $\d_2(A,A^\prime) = 2\varepsilon |A\Delta A^\prime|^{\frac{1}{2}}$. For $n\in \mathbb{N}$, consider the quantities $N_n = 2^{2^n}$ and $N_0=1$. 

\begin{definition}
    Given a set $T$, a sequence $(\mathcal{A}_n)_{n\geq 0}$ of partitions of $T$ is \textit{admissible} when $|\mathcal{A}_n|\leq N_n$ and $\mathcal{A}_{n+1}\preceq \mathcal{A}_n$ for all $n\geq 0$.
\end{definition}

Given $t\in T$ and an admissible sequence $(\mathcal{A}_n)_{n\geq 0}$, $A_n(t)$ denotes the element of $\mathcal{A}_n$ that contains $t$. 

\begin{definition}
    Given $\theta \geq 0$ and a metric space $(T,d)$, we define
    \begin{equation*}
        \gamma_\theta(T,d) \coloneqq \inf_{(\mathcal{A}_n)_{n\geq 0}}\sup_{t\in T}\sum_{n\geq 0}2^{\frac{n}{\theta}}\diam(A_n(t)),
    \end{equation*}
where the infimum is taken over all admissible sequences. 
\end{definition}

\begin{theorem}[Majorizing measure theorem] There is a universal constant $L>0$ such that
\begin{equation*}
    \frac{1}{L}\gamma_2(T,d) \leq \mathbb{E}\left( \sup_{t\in T} X_t \right) \leq L\gamma_2(T,d).
\end{equation*}
\end{theorem}

    Given $\epsilon>0$, let $N(T,\d, \epsilon)$ be the minimal number of balls with radius $\epsilon$ necessary to cover $T$, using the distance $d$. 
\begin{proposition}[Dudley's entropy bound \cite{Dudley67}]
    Let $(X_t)_{t\in T}$ be a family of random variables satisfying \eqref{Eq: Sub_gaussian_def} for some distance $\d$. Then there exists a constant $L>0$ such that 
    \begin{equation*}
        \mathbb{E}\left[\sup_{t\in T}X_t\right]\leq L\int_{0}^\infty \sqrt{\log N(T,\d,\epsilon)}d\epsilon.
    \end{equation*}
\end{proposition}

Dudley's entropy bound together with the Majorizing measure theorem yields that there is a constant $L^\prime>0$ such that,
\begin{equation}\label{Eq: gamma_2_bounded_by_Dudley_integral}   
\gamma_2(T,\d)\leq L^\prime\int_{0}^\infty \sqrt{\log N(T,\d,\epsilon)}d\epsilon.
\end{equation}
We only need one last result.
\begin{theorem}\label{Theo: Theo_2.2.27_Talagrand} Given a metric space $(T,\d)$ and a family $(X_t)_{t\in T}$ of centered random variables satisfying \eqref{Eq: Sub_gaussian_def}, there is a universal constant $L>0$ such that, for any $u>0$,
\begin{equation*}
\mathbb{P}\left( \sup_{t\in T}X_t > L(\gamma_2(T,\d) + u\diam(T)) \right)\leq e^{-{u^2}},
\end{equation*}
where the $\diam(T)$ is the diameter taken with respect to the distance $\d$
\end{theorem} 
A proof can be found in  \cite[Theorem 2.2.27]{Talagrand_14}. Using these results, the bound on the bad event $\mathcal{E}_1$ follows from the following proposition.
\begin{proposition}\label{Prop: Bound.gamma_2}
    Given $n,j\geq 0$ and $\alpha > d$, there are large enough constants $L_1>0$  such that, for $d \geq 3$, $$\gamma_2(\mathcal{C}_0(n,j),\d_2) \leq \varepsilon L_1 n.$$
\end{proposition}
As a direct consequence of this Proposition, we can control the probability of the bad event.
\begin{proposition}\label{Prop: Bound.bad.event.1}     
    There exists $C_1\coloneqq C_1(\alpha, d)$ such that $\mathbb{P}(\mathcal{E}^c)\leq e^{-\frac{C_1}{\varepsilon^2}}$. 
\end{proposition}

\begin{proof}
   Given $\gamma\in\mathcal{C}_0(n,j)$, by the construction of the contours and the isoperimetric inequality, we have $2^{r(d+1)(j-1)}\leq |V(\gamma)|\leq n^{1 + \frac{1}{d-1}}$. Taking $N\coloneqq \frac{d}{d^2-1}\log_{2^r}n + 1$, we have $j\leq N$ and the union bound yields
\begin{align}\label{Eq: Union_bound_bad_event}
    \mathbb{P}\left({\sup_{\substack{\gamma\in\mathcal{C}_0}} \frac{|\Delta_{\I_-(\gamma)}(h)|}{c_2|\gamma|} > \frac{1}{4}}\right) \leq \sum_{n=1}^\infty \sum_{j=1}^{N}\mathbb{P}\left({\sup_{\substack{\gamma\in\mathcal{C}_0(n,j)}} |\Delta_{\I_-(\gamma)}(h)| > \frac{c_2}{4}}|\gamma|\right). 
\end{align}
Let $\gamma,\gamma^\prime\in \mathcal{C}_0(n,j)$ two contours satisfying $\diam(\mathcal{C}_0(n,j)) = \d_2(\gamma,\gamma^\prime)$, where the diameter is in the $\d_2$ distance. By the isoperimetric inequality,
\begin{equation*}
    \diam(\mathcal{C}_0(n,j))\leq 2\varepsilon{|\I_-(\gamma)\Delta \I_-(\gamma^\prime)|}^{\frac{1}{2}} = 2\sqrt{2}\varepsilon n^{(\frac{d}{d-1})\frac{1}{2}} = 2\sqrt{2}\varepsilon n^{(\frac{1}{2} + \frac{1}{2(d-1)})}.
\end{equation*}
Together with Proposition \ref{Prop: Bound.gamma_2}, this yields
\begin{align*}
  \frac{c_2}{4}|\gamma| &= L\left[\varepsilon L_1 n + \varepsilon L_1 \left(\frac{c_2}{4\varepsilon L_1 L} - 1\right)n\right]\\
    %
    &\geq  L\left[\gamma_2(\mathcal{C}_0(n,j),\d_2) +  \frac{L_1}{2\sqrt{2}} \left(\frac{c_2}{4\varepsilon L_1 L} - 1\right)n^{\frac{1}{2} - \frac{1}{2(d-1)}}\diam(\mathcal{C}_0(n,j))\right]\\
    %
    &\geq   L\left[\gamma_2(\mathcal{C}_0(n,j),\d_2) +  \frac{C_1^\prime}{\varepsilon}n^{\frac{1}{2} - \frac{1}{2(d-1)}}\diam(\mathcal{C}_0(n,j))\right],
\end{align*}
with $C_1^\prime = \frac{c_2}{16\sqrt{2}L_1L}$ and $\varepsilon> \frac{c_2}{8L_1L}$. Applying Theorem \ref{Theo: Theo_2.2.27_Talagrand} with $u = \frac{C_1^\prime}{\varepsilon}n^{\frac{1}{2} - \frac{1}{2(d-1)}}$, we have
\begin{align*}
    \mathbb{P}\left({\sup_{\substack{\gamma\in\mathcal{C}_0(n,j)}} |\Delta_{\I_-(\gamma)}(h)| > \frac{c_2}{4}}|\gamma|\right) &\leq \mathbb{P}\left({\sup_{\substack{\gamma\in\mathcal{C}_0(n,j)}} |\Delta_{\I_-(\gamma)}(h)| >    L\left[\gamma_2(\mathcal{C}_0(n,j),\d_2) +  \frac{C_1^\prime}{\varepsilon}n^{\frac{1}{2} - \frac{1}{2(d-1)}}\diam(\mathcal{C}_0(n,j))\right]}\right) \\ 
    %
    &\leq \exp{\left\{ - \frac{C_1^{\prime2}n^{1 - \frac{1}{(d-1)}}}{\varepsilon^2}\right\}} 
\end{align*}
Using this back in equation \eqref{Eq: Union_bound_bad_event}, we conclude that 
\begin{equation*}
       \mathbb{P}\left({\sup_{\substack{\gamma\in\mathcal{C}_0}} \frac{|\Delta_{\I_-(\gamma)}(h)|}{c_2|\gamma|} > \frac{1}{4}}\right) \leq \sum_{n=1}^\infty  \left(\frac{d}{d^2-1}\log_{2^r}n +1\right) \exp{\left\{ - \frac{C_1^{\prime2}n^{1 - \frac{1}{(d-1)}}}{\varepsilon^2}\right\}} \leq e^{-\frac{C_1}{\varepsilon^2}},
\end{equation*}
for a suitable constant $C_1> \frac{c_2}{8L_1L}$ and $\varepsilon> C_1$.
\end{proof}
The rest of this section is dedicated to proving Proposition \ref{Prop: Bound.gamma_2}.