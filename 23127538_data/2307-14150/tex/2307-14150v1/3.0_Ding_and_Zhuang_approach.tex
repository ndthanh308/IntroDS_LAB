The main idea used in Ding and Zhuang's proof of phase transition in \cite{Ding2021} is to make the Peierls' argument on the joint space of the configurations and the external field, and when erasing a contour, perform in the external field the same flips you do in the configuration. Doing this, the part on the Hamiltonian that depends on the external field does not change, but the partition function does. The complication of this method is to control such differences.

As the contours are the same as in the short-range case, the spins that need to be flipped to erase a contour are precisely the ones in the interior of it. In this section, we define the measure in the joint space and show that with high probability, both the change of partition function and the extra energy cost resulting from such flipping are upper-bounded by the size of the support $|\gamma|$. 

