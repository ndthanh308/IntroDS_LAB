Given $\Lambda\subset\Z^d$, define the local joint measure for $(\sigma, h)$ as
\begin{equation*}
    \mathbb{Q}_{\Lambda; \beta, \varepsilon}^+(\sigma \in A, h\in B) = \int_{B} \mu_{\Lambda;\beta, \varepsilon h}^+(A) d\mathbb{P}(h),
\end{equation*}
for $A\subset\Omega$ measurable and $B\subset \mathbb{R}^{\Lambda}$ borelian. Since $\beta, \varepsilon $ and $\Lambda$ are fixed, we will omit then from the notation. 
This measure $\mathbb{Q}$ has density
\begin{equation*}
    g_{\Lambda; \beta, \varepsilon}^+(\sigma, h) = \prod_{u\in\Lambda}\frac{1}{\sqrt{2\pi}}e^{-\frac{1}{2}h_u^2} \times \mu_{\Lambda;\beta, \varepsilon h}^+(\sigma).
\end{equation*}

The main idea used in the proof of phase transition in \cite{Ding2021} is to make the Peierls' argument on the measure $\mathbb{Q}$, and perform in the external field the same flips you do in the configuration when erasing a contour. Formally, in \cite{Ding2021} they compare the density $g_{\Lambda; \beta, \varepsilon}^+(\sigma, h)$ with the density after erasing a contour $\gamma\in\Gamma(\sigma)$, and performing the same flips on the external field, getting

\begin{align}\label{Eq: quotient.of.gs}
    \frac{g_{\Lambda; \beta, \varepsilon}^+(\sigma, h)}{g_{\Lambda; \beta, \varepsilon}^+(\tau_{\gamma}(\sigma),\tau_{\gamma}(h))} \nonumber
    %
    &= \exp{\{\beta H_{\Lambda, 0}^{+}(\tau_{\gamma}(\sigma)) - \beta H_{\Lambda, 0}^{+}(\sigma)\}}\frac{Z_{\Lambda; \beta, \varepsilon}^{+}(\tau_{\gamma}(h))}{Z_{\Lambda; \beta, \varepsilon}^{+}(h)}.  \nonumber \\ 
\end{align}

For some realizations of the external field, the quotient of the partition functions can be bigger than the exponential term. Denoting
\begin{equation}
\Delta_A(h) \coloneqq -\frac{1}{\beta}\ln{\frac{Z_{\Lambda; \beta, \varepsilon}^{+}(h)}{Z_{\Lambda; \beta, \varepsilon}^{+}(\tau_{A}(h))}}
\end{equation}
 for every $A\subset \Z^d$, the bad event is
$$\mathcal{E}^c\coloneqq \left\{\sup_{\substack{\gamma\in\Gamma_0}} \frac{|\Delta_{\I(\gamma)}(h)|}{c_1|\gamma|} > \frac{1}{4}\right\}.$$
To control the probability of this bad event, we need a concentration result for Gaussian random variables. The following one is due to M. Ledoux and M. Talagrand, and a proof can be found in \cite{Ledoux.Talagrand.91}.

\begin{theorem}\label{Theo: Gaussian.concentration}
    Let $f:\mathbb{R}^M \xrightarrow[]{} \mathbb{R}$ be a uniform Lipschitz continuous function with constant $C_{Lip}$, that is, for any $X,Y\in\mathbb{R}^M$, $$|f(X) - f(Y)| \leq C_{Lip} || X - Y ||_2 .$$ 
    
    Then, if $X_1,\dots, X_M$ are i.i.d. Gaussian random variables with variance 1,
    \begin{equation}\label{Eq: Tail.concentration}
        \mathbb{P}\left(|f(X_1,\dots, X_M) - \mathbb{E}(f(X_1, \dots, X_M))|\geq z\right) \leq 2\exp{\left\{\frac{-z^2}{2C_{Lip}^2}\right\}}.
    \end{equation}
\end{theorem}

\begin{remark}\label{Rmk: MVT.Lipschitz}
    If $f$ is differentiable and $||\nabla f(\cdot)||_2$ is bounded, the mean value theorem guarantees that $\sup_{Z\in\mathbb{R}^M}||\nabla f(Z)||_2$ is a uniform Lipschitz constant for $f$. 
\end{remark}
\begin{remark}
    If $f$ has a compact support and convex level sets, an equation similar to \eqref{Eq: Tail.concentration} holds, with some adjustments on the constants and replacing the mean by the median, see \cite[Theorem 7.1.3]{Bovier.06}. Therefore, our results hold when $h_i$ has a Bernoulli distribution $\mathbb{P}(h_i=+1) =\mathbb{P}(h_i=-1)= \frac{1}{2}$. 
\end{remark}

Given $A\subset\Z^d$, $h_A\coloneqq (h_x)_{x\in A}$ denotes the restriction of the external field to the subset $A$. The next Lemma was proved in \cite{Ding2021} and is a direct consequence of the previous theorem. 

\begin{lemma}\label{Lemma: Concentration.for.Delta.General}
    For any $A, A^\prime \Subset \mathbb{Z}^d$ and $\lambda>0$, we have 
\begin{equation}\label{Eq: Tail.of.Delta_A}
    \mathbb{P}\left(|\Delta_A(h)| \geq \lambda \vert h_{A^c}\right) \leq2e^{\frac{-\lambda^2}{8\varepsilon^2 |A|}},
\end{equation}
and 
\begin{equation}\label{Eq: Tail.of.the.diff.of.Deltas}
     \mathbb{P}(|\Delta_{A}(h) - \Delta_{A^\prime}(h)|>\lambda|h_{{A \cup A^\prime}^c}) \leq  2e^{-\frac{{\lambda^2}}{{8\varepsilon^2|A \Delta A^\prime|}}},
\end{equation}
where $A\Delta A^\prime$ is the symmetric difference.
\end{lemma}

\begin{remark}\label{Rmk: More.general.h_x}
    This lemma holds whenever $h=(h_x)_{x\in\Z^d}$ satisfy equation \eqref{Eq: Tail.concentration}.As a consequence, our results can be stated for more general external fields. 
\end{remark}
