\begin{theorem}
For $d\geq 3$ and $\alpha>d+1$, there exists a constant $C\coloneqq C(d,\alpha)$ such that, for all $\beta>0$, $e\leq C$ and $N\geq 1$, the event 
    \begin{equation}\label{Eq: PTLR}
        \mu_{\Lambda; \beta, \varepsilon h}^+(\sigma_0 = -1) \leq e^{-C\beta} + e^{-C/\varepsilon^2} 
    \end{equation}
    has $\mathbb{P}$-probability bigger then $1 - e^{-C\beta} - e^{-C/\varepsilon^2}$.\\
    
In particular, for $\beta>\beta_c$ and $\varepsilon$ small enough, there is phase transition for the long-range Ising model.  
\end{theorem}

\begin{proof}
        The proof is an application of the Peierls' argument, but now on the joint measure $\mathbb{Q}$. By Proposition \ref{Prop: Bound.bad.event.1}, we have
        \begin{align}\label{Eq: Upper.bound.on.Q.1}
            \mathbb{Q}_{\Lambda; \beta, \varepsilon}^+(\sigma_0 = -1) &=  \mathbb{Q}_{\Lambda; \beta, \varepsilon}^+(\sigma_0 = -1 \cap \mathcal{E}) + \mathbb{Q}_{\Lambda; \beta, \varepsilon}^+(\sigma_0 = -1\cap \mathcal{E}^c) \nonumber \\
            %
            & \leq \mathbb{Q}_{\Lambda; \beta, \varepsilon}^+(\sigma_0 = -1 \cap \mathcal{E}) +  e^{-C_1/\varepsilon^2} \nonumber \\
            %
        \end{align}
since $\mathbb{Q}_{\Lambda; \beta, \varepsilon}^+(\sigma_0 = -1\cap \mathcal{E}^c) \leq \mathbb{Q}_{\Lambda; \beta, \varepsilon}^+(\mathcal{E}^c) = \mathbb{P}(\mathcal{E}^c)$.  When $\sigma_0 = -1$, there must exist a contour $\gamma$ with $0\in V(\gamma)$, hence
\begin{equation*}
    \mu_{\Lambda; \beta, \varepsilon h}^+(\sigma_0 = -1) \leq \sum_{\gamma \in \mathcal{C}_0}\mu_{\Lambda; \beta, \varepsilon h}^+(\Omega(\gamma)),
\end{equation*}
where $\Omega(\gamma) \coloneqq \{\sigma\in\Omega : \gamma \subset \Gamma(\sigma)\}$. So we can write

\begin{align}\label{Eq: Upper.bound.on.Q.2}
    \mathbb{Q}_{\Lambda; \beta, \varepsilon}^+(\sigma_0 = -1 \cap \mathcal{E}) &= \int_{\mathcal{E}}\sum_{\sigma : \sigma_0 = -1}g_{\Lambda; \beta, \varepsilon}^+(\sigma, h)dh \nonumber \\
    %
    &\leq  \sum_{\mathcal{C}_0} \int_{\mathcal{E}}\sum_{\gamma\in \sigma\in\Omega(\gamma)}g_{\Lambda; \beta, \varepsilon}^+(\sigma, h)dh \nonumber \\
    %
    &\leq  \sum_{\gamma \in \mathcal{C}_0} \frac{2^{|\gamma|}\int_{\mathcal{E}}\sum_{\sigma\in\Omega(\gamma)}g_{\Lambda; \beta, \varepsilon}^+(\sigma, h)dh}{\int_{\mathcal{E}}\sum_{\sigma\in\Omega(\gamma)}g_{\Lambda; \beta, \varepsilon}^+(\tau_{\gamma, \sigma}(\sigma), \tau_{\I_-(\gamma)}(h))dh} \nonumber \\
    %
    & \leq \sum_{\gamma\in\mathcal{C}_0}2^{|\gamma|} \sup_{\substack{h\in\mathcal{E}\\ \sigma\in\Omega(\gamma)}}\frac{g_{\Lambda; \beta, \varepsilon}^+(\sigma, h)}{g_{\Lambda; \beta, \varepsilon}^+(\tau_{\gamma, \sigma}(\sigma), \tau_{\I_-(\gamma)}(h))}.
\end{align}

In the third equation, we used that $\int_{\mathcal{E}}\sum_{\sigma\in\Omega(\gamma)}g_{\Lambda; \beta, \varepsilon}^+(\tau_{\gamma, \sigma}(\sigma), \tau_{\I_-(\gamma)}(h))dh \leq 2^{|\gamma|}$, since the number of configurations that are incorrect in $\Sp(\gamma)$ are bounded by $2^{|\gamma|}$. By \eqref{Eq: quotient.of.gs} and the definition of the event $\mathcal{E}$, 
\begin{align}\label{Eq: Upper.bound.on.Q.3}
    \sup_{\substack{h\in\mathcal{E}\\ \sigma\in\Omega(\gamma)}}\frac{g_{\Lambda; \beta, \varepsilon}^+(\sigma, h)}{g_{\Lambda; \beta, \varepsilon}^+(\tau_{\gamma, \sigma}(\sigma), \tau_{\I_-(\gamma)}(h))} &\leq \sup_{\substack{h\in\mathcal{E}\\ \sigma\in\Omega(\gamma)}}  \exp{\{{- \beta c_2 |\gamma|}\}}\frac{Z_{\Lambda; \beta, \varepsilon}^{+}(\tau_{\I_-(\gamma)}(h))}{Z_{\Lambda; \beta, \varepsilon}^{+}(h)} \nonumber\\
    %
    &= \sup_{\substack{h\in\mathcal{E}\\ \sigma\in\Omega(\gamma)}}  \exp{\{{- \beta c_2 |\gamma| + \beta \Delta_{\gamma}(h)}\}} \nonumber\\
    %
    &\leq  \exp{\{{- \beta \frac{c_2}{2} |\gamma| }\}},
\end{align}
since $\Delta_{\gamma}(h) \leq \frac{1}{2}(c_2|\gamma|)$, for all $h\in\mathcal{E}_1$. Equations \eqref{Eq: Upper.bound.on.Q.1}, \eqref{Eq: Upper.bound.on.Q.2} and \eqref{Eq: Upper.bound.on.Q.3} yields
\begin{align*}
     \mathbb{Q}_{\Lambda; \beta, \varepsilon}^+(\sigma_0 = -1) &\leq  \sum_{\substack{\gamma\in \mathcal{E}_\Lambda^+\\ 0\in V(\gamma)}} 2^{|\gamma|}\exp{\{{- \beta \frac{c_2}{2} |\gamma| }\}} + e^{-c_0/\varepsilon^2}\\
     &\leq \sum_{n\geq 1}\sum_{\substack{\gamma\in \mathcal{E}_\Lambda^+, |\gamma|=n \\ 0\in V(\gamma)}} \exp{\{{(-\beta \frac{c_2}{2} + \ln2)n}\}} + e^{-c_0/\varepsilon^2}\\
     %
     &\leq \sum_{n\geq 1}|\mathcal{C}_0(n)| \exp{\{{(-\beta \frac{c_2}{2} +\ln2)n}\}} + e^{-c_1/\varepsilon^2} \leq \sum_{n\geq 1} e^{(c_1 -\beta \frac{c_2}{2} +\ln2)n} + e^{-c_0/\varepsilon^2}. \\
\end{align*}
When $\beta$ is large enough, the sum above converges and there exists a constant $C$ such that   
\begin{equation*}
    \mathbb{Q}_{\Lambda; \beta, \varepsilon}^+(\sigma_0 = -1) \leq e^{-\beta 2C} + e^{-2C / \varepsilon^2}.
\end{equation*}
The Markov Inequality finally yields
\begin{align*}
    \mathbb{P}\left( \mu_{\Lambda; \beta, \varepsilon h}^+(\sigma_0 = -1) \geq e^{-C\beta} + e^{-C/\varepsilon^2}\right) &\leq \frac{\mathbb{Q}_{\Lambda; \beta, \varepsilon}^+(\sigma_0 = -1)}{e^{-C\beta} - e^{-C/\varepsilon^2}} \\
    %
    &\leq \frac{e^{-\beta 2C} + e^{-2C / \varepsilon^2}}{e^{-C\beta} - e^{-C/\varepsilon^2}} \leq e^{-C\beta} - e^{-C/\varepsilon^2},
\end{align*}
what proves our claim.
\end{proof}

The natural question that comes to mind is if there is phase transition for $d<\alpha<d+1$. In this region, a recent paper \cite{Affonso.2021} introduces a new notion of contour, based in a construction by Frohlich and Spencer \cite{Frohlich.Spencer.82}, and with them they prove phase transition. The key difference in these contours is that they are no longer connected, so the count argument on Proposition \ref{Prop: Proposition_2_FFS} does not hold. Nevertheless, using another counting method and this new contours, it should be possible to prove:

\begin{conjecture*}
    Let $\Gamma_0$ be the set of contours with 0 in its volume, as defined in \cite{Affonso.2021}. Then, for any $\alpha>d$, $d\geq 3$, there exists $C_2\coloneqq C_2(\alpha, d)$ such that $\mathbb{P}(\left\{\sup_{\substack{\gamma\in\Gamma_0}} \frac{|\Delta_{\I_-(\gamma)}(h)|}{|\gamma|} > 1\right\})\leq e^{-\frac{C_2}{\varepsilon^2}}$. 

    In particular, for $\beta>\beta_c$ and $\varepsilon$ small enough, there is phase transition for the long-range Ising model in $d<\alpha\leq d+1$ and $d\geq 3$.  
\end{conjecture*}

\begin{conjecture*}
    For $d\geq 3$ and $d<\alpha \leq d+1$, there exists a constant $C^\prime\coloneqq C^\prime(d,\alpha)$ such that, for all $\beta>0$, $e\leq C^\prime$ and $N\geq 1$, the event 
    \begin{equation}
        \mu_{\Lambda; \beta, \varepsilon h}^+(\sigma_0 = -1) \leq e^{-C^\prime\beta} + e^{-C^\prime/\varepsilon^2} 
    \end{equation}
    has $\mathbb{P}$-probability bigger then $1 - e^{-C^\prime\beta} - e^{-C^\prime/\varepsilon^2}$.\\
    
In particular, for $\beta>\beta_c$ and $\varepsilon$ small enough, there is phase transition for the long-range Ising model.  
\end{conjecture*}