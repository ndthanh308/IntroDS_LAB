The Random Field Ising Model (RFIM) is the Ising model with short-range interactions where the external field $(h_x)_{x\in\Z^d}$ is a family of i.i.d. Gaussian random variable with mean 0 and variance 1. The Hamiltonian of the model is given formally by
\begin{equation*}
    H(\sigma) = - \sum_{x,y\in \Z^d}J_{xy}\sigma_x\sigma_y - \varepsilon\sum_{x\in\Z^d}h_x\sigma_x,
\end{equation*}
where $J_{xy}=J|x-y|^{-\alpha}$, $J, \varepsilon>0$, $\alpha > d$ and $h_x\in\mathbb{R}$.

A detailed description of the history of the phase transition problem for this model, as well as detailed proof of key results, was given in \cite{Bovier.06}. Here we present a brief overview. Phase transition for this model was a hot topic of discussion in the 1980s when two arguments divided the physics community. One of them, due to Imry and Ma \cite{Imry.Ma.75}, was a non-rigorous application of the Peierls' argument \cite{Peierls.1936}. The key idea of Peierls' argument is to define a notion of contour and calculate the energy cost of "erasing" each contour, i.e., the energy cost of flipping all spins inside the contour. When there is no external field, that energy necessary to flip the spins in a region $A\subset \Z^d$ has the order of the boundary $|\partial A|$. When we add an external field, we get an extra cost depending on this field. Imry and Ma predicted that this cost would be like $\sqrt{|A|}$, which is smaller than $|\partial A|$ when $d\geq 3$, and this should be the region where phase transition occurs. The second one, due to Parisi and Sourlas \cite{Parisi.Sourlas.79}, predicted that the RFIM in dimension $d$ would behave like the usual Ising model in $d-2$, therefore it would present phase transition only on $d\geq 4$. Two celebrated papers showed that Imry and Ma's prediction was correct. First, in 1988, Bricmont and Kupiainen \cite{Bricmont.Kupiainen.88} showed that there is more than one Gibbs measure in $d\geq3$ for high temperature and variance $\varepsilon$ small enough. Later on, Aizenman and Wehr \cite{Aizenman.Wehr.90} proved uniqueness for $d\leq 2$. For details of these proofs, we again refer to \cite{Bovier.06}. For uniqueness results see \cite{Frohlich.Imbre.84, Berretti.85, Camia.18, Klein.Masooman.97}.

To prove phase transition, Bricmont and Kupiainen used a renormalization group analysis. Their results work as long for any model with a contour representation and sub-gaussian external field centered in zero, but the proof is intricate. More recently, in \cite{Ding2021}, Ding and Zhuang provided a simpler proof, not using renormalization. The idea of the proof is to perform in the external field the same flips you make in the configuration when erasing a contour, and then to use Peierls' argument on a joint measure on the state space and the external field probability. Afterward, in \cite{Ding.Liu.Xia.22}, Ding, Liu, and Xi extended that result and proved that, if $\beta_c(d)$ is the critical inverse of the temperature of the Ising model with no field, for all $\beta<\beta_c(d)$ there exist a critical value $\varepsilon_0(d, \beta)$ such that the RFIM with $\epsilon \leq \varepsilon_0$ presents phase transition.

For the one-dimensional long-range model with a random field, Cassandro, Orlandi and Picco in \cite{Cassandro.Picco.09} used the contours of \cite{Cassandro.05} to show that, under the assumption $J(1)>>1$, the model presents phase transition when $\alpha\in (3-\frac{\ln 3}{\ln 2}, \frac{3}{2})$. In this paper, we use \cite{Ginibre.Grossmann.Ruelle.66} and \cite{Ding2021} to prove phase transition for the long-range Ising model. The main result of this paper is the following.
\begin{theorem*}Given $d\geq 3$, $\alpha> d + 1$, there exists $\beta_c\coloneqq\beta(d, \alpha)$ and $\varepsilon_c\coloneqq\varepsilon(d, \alpha)$ such that, for $\beta\geq \beta_c$ and $\varepsilon\leq \varepsilon_c$, the extremal Gibbs measures $\mu_{\beta, \varepsilon}^+$ and $\mu_{\beta, \varepsilon}^-$ are distinct, that is, $\mu_{\beta, \varepsilon}^+ \neq \mu_{\beta, \varepsilon}^-$ $\mathbb{P}$-almost surely. Therefore the long-range random field Ising model presents phase transition.
\end{theorem*}

The overall strategy to prove this result is: first we prove that, when there is no external field, the difference between the Hamiltonian with a configuration $\sigma$ and with the configuration after erasing a contour $\gamma$ is bounded by $c|\gamma|$, for some constant $c>0$. Then, we show that, in the joint measure, the energy cost of erasing a contour and flipping the external field on the interior of the contour is bounded by $-c|\gamma|$ plus an error $\Delta_\gamma(h)$. Finally, we take the bad event of existing a contour with $0$ in its interior such that $\Delta_\gamma(h)>\frac{c}{2}|\gamma|$ and show that it has probability smaller than $e^{-\frac{C}{\varepsilon^2}}$, for a suitable constant $C>0$. This allow us to perform a Peierls' argument on the joint space and prove phase transition. 

This paper is divided as follows. In Section 2, we define the model, the contours and calculate the energy cost of erasing a contour.  In section 3, we define the bad event of the external field and prove that they occur with a small probability.  In Section 4, we present the proof of the phase transition.