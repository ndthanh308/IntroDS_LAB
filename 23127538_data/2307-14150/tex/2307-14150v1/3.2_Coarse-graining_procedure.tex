To control the probability of $\mathcal{E}^c$ we use a multi-scale analysis method presented in \cite{FFS84}. This subsection is dedicated to prove

\begin{proposition}\label{Prop: Bound.bad.event.1}       
    There exists $C_1\coloneqq C_1(\alpha, d)$ such that $\mathbb{P}(\mathcal{E}^c)\leq e^{-\frac{C_1}{\varepsilon^2}}$. 
\end{proposition}

As pointed out by \cite{Ding2021}, the proof presented in \cite{FFS84}, despite being self-contained, is an indirect application of Dudley's entropy bound. Here we adapt the proof presented in \cite{FFS84} using this entropy bound. For the detailed argument of the original proof, see \cite{Bovier.06}. 

First, we need to introduce some probability tools. Consider $(T,d)$ a metric space and a process $(X_t)_{t\in T}$ such that, for every $\lambda>0$ and $t,s\in T$,
\begin{equation}\label{Eq: Sub_Gaussian_r.v.}
    \mathbb{P}\left( |X_t - X_s| \geq \lambda \right) \leq 2\exp{-\frac{\lambda^2}{2d(s,t)^2}}.
\end{equation}
Assume also that $\mathbb{E}\left(X_t\right) = 0$ for every $t\in T$. One example of such process is $(|\Delta_{\I_-(\gamma)}|)_{\gamma\in\Gamma_0}$ with the distance $\d_2(A,A^\prime) = 2\varepsilon |A\Delta A^\prime|^{\frac{1}{2}}$. Denote $\sigma^2_X \coloneqq \sup_{t\in T} \text{Var}(X_t)$. 

\begin{definition}
    Given a set $T$, a sequence $(\mathcal{A}_n)_{n\geq 0}$ of partitions of $T$ is \textit{admissible} when $|\mathcal{A}_0|=1$, $|\mathcal{A}_n|\leq 2^{2^n}$ for $n\geq 0$, and $(\mathcal{A}_n)_{n\geq 0}$ is increasing, that is, every set of $\mathcal{A}_{n+1}$ is contained in a set of $\mathcal{A}_n$.
\end{definition}

Given $t\in T$ and an admissible sequence $(\mathcal{A}_n)_{n\geq 0}$, $A_n(t)$ denotes the element of $\mathcal{A}_n$ that contains $t$. 

\begin{definition}
    Given $\theta \geq 0$ and a metric space $(T,d)$, we define
    \begin{equation*}
        \gamma_\theta(T,d) \coloneqq \inf_{(\mathcal{A}_n)_{n\geq 0}}\sup_{t\in T}\sum_{n\geq 0}2^{\frac{n}{\theta}}\diam(A_n(t)),
    \end{equation*}
where the infimum is taken over all admissible sequences. 
\end{definition}
The next theorem shows that the functional $\gamma_2(T,d)$ controls the expected value of the supremum of $(X_t)_{t\in T}$. A proof can be fund in \cite[Theorem 2.4.1]{Talagrand_14}
\begin{theorem}[Majorizing measure theorem] Given a metric space $(T,d)$ and a family $(X_t)_{t\in T}$ satisfying \eqref{Eq: Sub_Gaussian_r.v.} and $\mathbb{E}(X_t)=0$ for every $t\in T$, there is a universal constant $L>0$ such that
\begin{equation*}
    \frac{1}{L}\gamma_2(T,d) \leq \mathbb{E}\left( \sup_{t\in T} X_t \right) \leq L\gamma_2(T,d).
\end{equation*}
\end{theorem}

Given $\epsilon>0$, let $N(T,\d, \epsilon)$ be the minimal number of ball with radius $\epsilon$ necessary to cover $T$, using the distance $d$. 
\begin{proposition}[Dudley's entropy bound \cite{Dudley67}]
    Let $(X_t)_{t\in T}$ be a family of random variables satisfying \eqref{Eq: Sub_Gaussian_r.v.} for some distance $\d$. Then there exists a constant $L>0$ such that 
    \begin{equation*}
        \mathbb{E}\left[\sup_{t\in T}X_t\right]\leq L\int_{0}^\infty \sqrt{\log N(T,\d,\epsilon)}d\epsilon.
    \end{equation*}
\end{proposition}

Dudley's entropy bound together with the Majorizing measure Theorem yields that, for a constant $L>0$, $\gamma_2(T,\d)\leq \int_{0}^\infty \sqrt{\log N(T,\d,\epsilon)}d\epsilon$.  We only need one last inequality, see \cite[Theorem 2.2.27]{Talagrand_14}

\begin{theorem}\label{Theo: Theo_2.2.27_Talagrand} Given a metric space $(T,\d)$ and a family $(X_t)_{t\in T}$ satisfying \eqref{Eq: Sub_Gaussian_r.v.}, there is a universal constant $L>0$ such that, for any $u>0$,
\begin{equation*}
\mathbb{P}\left( \sup_{t\in T}X_t > L(\gamma_2(T,\d) + u\diam(T)) \right)\leq e^{-{u^2}},
\end{equation*}
where the $\diam(T)$ is the diameter taken with respect to the distance $\d$
\end{theorem}

We will apply this lemmas for the family $(|\Delta_{\I_-(\gamma)}|)_{\gamma\in\Gamma_0(n)}$. To construct the covering by balls in Dudley's entropy bound, we use the coarse-graining idea introduced in \cite{FFS84}. For each $0<\ell$ and each contour ${\gamma\in\Gamma_0}$, we will associate a region $B_\ell(\gamma)$ that approximates the interior $\I(\gamma)$ in a scaled lattice, with the scale growing with $\ell$. This is done in a way that two contours that have the same representation are in a ball with fixed radius, depending on $\ell$.

For any $x\in\Z^d$ and $m\geq 0$,
\begin{equation}
    C_{m}(x) \coloneqq \left(\prod_{i=1}^d{\left[2^{m}x_i - 2^{m-1}, 2^{m}x_i + 2^{m-1} \right)}\right)\cap \Z^d
\end{equation}
is the cube of $\mathbb{Z}^d$ centered in $2^{m}x$ with side length $2^{m} -1$. Any such cube is called an $m$-cube, and we refer an arbitrary $m$-cube by $C_m$. An arbitrary collection of $m$-cubes will be denoted $\mathscr{C}_m$ and $B_{\mathscr{C}_m}\coloneqq \cup_{C\in\mathscr{C}_m}C$ is the region covered by $\mathscr{C}_m$. We  denote by $\mathscr{C}_m(\Lambda)$ the covering of $\Lambda\Subset\Z^d$ with the smallest possible number  of $m$-cubes.

\input{Figura.0}

Fix $n \in \mathbb{N}$, $\gamma\in \Gamma_0(n)$, and $\ell\in\{0, 1, \dots, k\}$. An $\ell$-cube $C_{\ell}$ is \textit{admissible} if it more than a  half of its points are inside $\I(\gamma)$. Thus, the set of admissible cubes is
\begin{equation*}
    \mathfrak{C}_\ell(\gamma) \coloneqq \{C_{\ell} : |C_{\ell}\cap \I(\gamma)| \geq \frac{1}{2}|C_{\ell}|\}.
\end{equation*}
We choose $B_\ell(\gamma) \coloneqq B_{\mathfrak{C}_{\ell}(\gamma)}$, the region covered by the admissible cubes.
Notice that $B_\ell(\gamma)$ is uniquely determined by $\partial B_\ell(\gamma)$. Moreover, $\partial B_\ell(\gamma)$ is uniquely determined by
$$
\partial \mathfrak{C}_\ell(\gamma) \coloneqq \{ \{C_{\ell}, C^\prime_{\ell}\} : C_{\ell} \in \mathfrak{C}_\ell(\gamma), \ C_\ell^\prime \notin \mathfrak{C}_\ell, \  C_\ell^\prime \text{ shares a face with }C_\ell\}.
$$ 
We will now control the number of cubes in $\mathfrak{C}_\ell(\gamma)$ by proving a proposition similar to \cite[Proposition 2]{FFS84}. This proposition was written for $d=3$ and $\gamma$ simply connected, but it can clearly be extended to $d\geq 2$ with no restriction in $\gamma$, see \cite{Bovier.06}. As we could not find a detailed proof anywhere, we provide one here.

Given a rectangle $\mathcal{R} = [1,r_1]\times[1,r_2]\times\dots\times[1,r_d]$, consider $\R_i\coloneqq\{x\in \R : x_i=1\}$ the face of $\R$ that is perpendicular to the direction $e_i$, for $i=1,\dots,d$. The line that connects a point $x\in \R_i$ to a point in the opposite face of $\R_i$ is $\ell_x^i \coloneqq \{ x + ke_i : 1\leq k\leq r_i\}$. Given $A\subset \Z^d$, the projection of $A\cap \R$ into the face $\R_i$ is
\begin{equation*}
    \calP_i(A\cap\R) \coloneqq \{x\in\R_i : \ell_x^i \cap A \neq \emptyset\}.
\end{equation*}
\input{Figura.4.1}
In many situations, we will split the projections into \textit{good} and \textit{bad} points. The set of good points is $\calP_i(A\cap\R)^{G} \coloneqq \{x\in \calP_i(A\cap \R) : \ell_x^i \cap (\R\setminus A) \neq \emptyset\}$, that is, there exist a point in $\ell_x^i\cap \R$ that is not in $A$.  The bad points are defined as $\calP^{B}_i(A\cap\R) \coloneqq \calP_i(A\cap\R)\setminus \calP_i^G(A\cap\R)$.

\input{Figura.5}
Given $x\in \calP_i(A\cap\R)^{G}$, by definition of the projection, there exists a point in $\ell_x^i\cap A$. Therefore, there exists a point $p\in \ell_x^i$ such that $p\in\fext A \cap \R$. As all lines are disjoint, we conclude that 
\begin{equation}\label{Eq: upper.bound.good.points}
     |\calP_i^{G}(A\cap\R)|\leq |\fext A \cap \R|.
\end{equation}
 We now prove two auxiliary lemmas.
 
\begin{lemma}\label{Lemma: Geo.discreta.1}
    Given $d\geq 2$, for any family of positive integers $\bm{r}=(r_i)_{i=1}^d$ with $R\leq r_i \leq 2R$ for some $R\geq 2$, $0<\lambda < 1$ and $A\subset\Z^d$, there exists a constant $c\coloneqq c(d, \lambda)$ such that, if 
    \begin{equation}\label{Eq: hypothesis.lemma.1}
         |\calP_i(A\cap \R)| \leq \lambda|\R_i|
    \end{equation}
    for all $i= 1,\dots, d$, then 
    \begin{equation*}
        \sum_{i=1}^d |\calP_{i}(A\cap \R)|\leq c|\fext A\cap \R|,
    \end{equation*}
    where $\R=[1,r_1]\times\dots\times [1,r_d]$.
\end{lemma}

\begin{proof}
The proof will be done by induction on the dimension. For $d=2$, take a rectangle ${\R=[1,r_1]\times[1,r_2]}$. If there is no bad points in $\calP_1(A\cap\R)$, then 
\begin{align}\label{Eq: Bound.1.on.P.1}
    |\calP_1(A\cap\R)| = |\calP_1^G(A\cap \R)| \leq |\fext A \cap \R|.
\end{align}

If there is a bad point $p=(1,p_2)\in \calP_1^B(A\cap\R)$, $\ell_p^1\subset A\cap \R$  by definition of bad point. As $|\calP_1(A\cap \R)| \leq \lambda|\R_1| < |\R_1|$, there is a point $p^\prime = (1,p_2^\prime)\in \R_1\setminus \calP_1(A\cap \R)$ that is in the face $\R_1$ but not in the projection. By definition of the projection, $\ell_{p^\prime}^1\in A^c\cap \R$. Therefore, for any $1\leq k\leq r_1$, $(k,p_2)\in  A\cap \R$ and $(k,p^\prime_2)\in  A^c\cap \R$, we can find a point $p^k=(k, p^k_2) \in \fext A \cap \R$. Since $p^{k_1}\neq p^{k_2}$ for every $k_1\neq k_2$, we have $r_1 \leq |\fext A \cap \R|$, hence
\begin{equation}\label{Eq: Bound.2.on.P.1}
   |P_1(A\cap \R)| \leq  |\R_1| = {r_2}\leq  2R \leq 2r_1 \leq  2|\fext A \cap \R|.
\end{equation}

A completely analogous argument can be done to bound $|P_2(A\cap \R)|$, and we conclude that
\begin{equation*}
    \sum_{i=1}^2|\calP_i(A\cap \R)|\leq 4|\fext A \cap \R|,
\end{equation*}
and take $c(2,\lambda)=4$. Suppose the lemma holds for $d-1$ and fix a rectangle $\R=[1,r_1]\times\dots\times[1,r_d]$. We split $\R$ into layers $L_k = \{x\in\Z^d : x_d = k\}$, for $k=1,\dots, r_d$. We can then partition the projection and write
\begin{equation*}
|\calP_i(A\cap \R)| = \sum_{k=1}^{r_d} |\calP_i(A\cap \R)\cap L_k|,    
\end{equation*}
for any $i\in\{1,\dots, d-1\}$. This yields
\begin{align}\label{Eq: Partition.sum.proj.}
    \sum_{i=1}^d|\calP_i(A\cap \R)| &= \sum_{i=1}^{d-1}\sum_{k=1}^{r_d}|\calP_i(A\cap \R)\cap L_k| + |\calP_d(A\cap \R)| \nonumber \\
    &=  \sum_{k=1}^{r_d}\sum_{i=1}^{d-1}|\calP_i(A\cap \R)\cap L_k| + |\calP_d(A\cap \R)|.
\end{align}

Notice now that $\calP_i(A\cap \R)\cap L_k = \calP_i(A\cap (\R\cap L_k))$. Defining the rectangle $\R^k \coloneqq \R\cap L_k$, for every point $p\in \calP_j^B(A\cap \R^k)$, $\ell_p^j \subset A\cap \R^k$. Moreover, we can associate every point $x\in \ell_p^j$ in the line with a point $x^\prime\in \calP_d(A\cap\R)$ by taking $x_m^\prime = x_m$ for $m \leq d-1$ and $x_d^\prime = 1$, therefore

\begin{equation*}
    r_j|\calP_j^B(A\cap \R^k)| = \sum_{p\in \calP_j^B(A\cap \R^k)}|\ell_p^j| \leq |\calP_d(A\cap\R)|.
\end{equation*}
\input{Imagem.6}
Using the hypothesis \eqref{Eq: hypothesis.lemma.1} we conclude that

\begin{equation}\label{Eq: upper.bound.projection.i.bad.points}
     |\calP_j^B(A\cap \R^k)| \leq \lambda\frac{|\R_d|}{r_j} = \lambda \frac{ \prod_{q\neq d}r_q}{r_j} =  \lambda \prod_{q\neq j,d}r_q = \lambda |(\R^k)_j|.
\end{equation}
We consider know two cases:
    \begin{itemize}
        \item[(a)] If $|\calP_i(A\cap \R^k)| \leq \frac{\lambda +1}{2}|(\R^k)_i|$, for all $i\leq d-1$, then we are in the hypothesis of the lemma in $d-1$ and therefore
\begin{equation}\label{Eq: Primeiro.bound.soma.projecoes}
    \sum_{i=1}^{d-1} |\calP_i(A\cap \R^k)| \leq c\left(d-1, \frac{\lambda + 1}{2}\right)|\fext A\cap \R^k|.
\end{equation}
    \item [(b)] If there exists $j\in\{1,\dots,d-1\}$ satisfying $|\calP_j(A\cap \R^k)| > \frac{\lambda +1}{2}|(\R^k)_j|$, by \eqref{Eq: upper.bound.projection.i.bad.points} we have $|\calP_j^G(A\cap \R^k)| = |\calP_j(A\cap \R^k)| - |\calP_j^B(A\cap \R^k)| \geq \frac{1-\lambda}{2}|(\R^k)_j|$, hence
\begin{equation*}
    |(\R^k)_j| \leq  \frac{2}{1 - \lambda}|\fext A \cap \R^k|.
\end{equation*}
    Using that $|(\R^k)_i| \leq (2R)^{d-2} \leq 2^{d-2}|(\R^k)_j|$ for every $i\in\{1,\dots,d\}$, we conclude that 
\begin{equation}\label{Eq: Segundo.bound.soma.projecoes}
    \sum_{i=1}^{d-1} |P_i(A\cap\R^k)| \leq \sum_{i=1}^{d-1} |(\R^k)_i| \leq (d-1)2^{d-2}|(\R^k)_j| \leq \frac{(d-1)2^{d-1}}{1 - \lambda}|\fext A \cap \R^k|.
\end{equation}
\end{itemize}


In both cases we were able to bound the sum of projections by a constant times the size of the boundary of $A$ in $\R^k$. Applying \eqref{Eq: Primeiro.bound.soma.projecoes} and \eqref{Eq: Segundo.bound.soma.projecoes} back in \eqref{Eq: Partition.sum.proj.} we get
\begin{align*}
    \sum_{i=1}^d|\calP_i(A\cap \R)| &\leq
    \sum_{k=1}^{r_d}\left[c\left(d-1, \frac{\lambda + 1}{2}\right)+ \frac{(d-1)2^{d-1}}{1 - \lambda}\right]|\fext A \cap \R \cap L_k| + |\calP_d(A\cap \R)|\\
    %
    &=\left[c\left(d-1, \frac{\lambda + 1}{2}\right)+ \frac{(d-1)2^{d-1}}{1 - \lambda}\right]|\fext A \cap \R| + |\calP_d(A\cap \R)|.
\end{align*}
We finish the proof by noticing that we can repeat this exact same argument but now splitting $\R$ into layers $L_k = \{x\in \R : x_j = k\}$. By doing so, we have that  
\begin{equation*}
    \sum_{i=1}^d|\calP_i(A\cap \R)| \leq
     \left[c\left(d-1, \frac{\lambda + 1}{2}\right)+ \frac{(d-1)2^{d-1}}{1 - \lambda}\right]|\fext A \cap \R| + |\calP_j(A\cap \R)|
\end{equation*}
for any $j\in\{1,\dots, d\}$. Summing both sides in $j$ we conclude

\begin{equation}
    \sum_{i=1}^d|\calP_i(A\cap \R)| \leq \frac{d}{d-1}\left[c\left(d-1, \frac{\lambda + 1}{2}\right)+ \frac{(d-1)2^{d-1}}{1 - \lambda}\right]|\fext A \cap \R|,
\end{equation}
which proves our claim if we take $c(d,\lambda) \coloneqq \frac{d}{d-1}\left[c(d-1, \frac{\lambda + 1}{2})+ \frac{(d-1)2^{d-1}}{1 - \lambda}\right] = 2d + \frac{(d-2)d2^{d-1}}{1-\lambda}$.
\end{proof}

\begin{remark}
This lemma can be proved when $R\leq r_i \leq \kappa R$ for any $\kappa>1$. When applying the lemma, we will choose $\lambda=\frac{7}{8}$ to simplify the notation. All the proofs work as long as we choose $\lambda> \frac{3}{4}$.
\end{remark}

\begin{lemma}\label{Lemma: Proposicao1.Aux1}
    Given $A\subset \Z^d$, $\ell\geq 0$ and $U= C_{\ell}\cup C_{\ell}^\prime$ with $C_{\ell}$ and $C_{\ell}^\prime$ being two $\ell$-cubes sharing a face, there exists a constant $b\coloneqq b(d)$ such that, if 
    
\begin{align}\label{Eq. U.condition}
    \frac{1}{2}|C_{\ell}| \leq |C_{\ell}\cap A| \qquad \text{and} \qquad |C_{\ell}^\prime\cap A|< \frac{1}{2}|C_{\ell}^\prime|
\end{align}
then $2^{\ell(d-1)}\leq b|\fext A\cap U|$.
\end{lemma}


\begin{proof}
For $\ell=0$, \eqref{Eq. U.condition} guarantees that $C_{\ell} = \{x\} \subset A$ and $C_{\ell}^\prime = \{y\}\subset A^c$, hence $|\fext A\cap \{x,y\}| = 1$ and it is enough to take $b\geq 1$. For $\ell \geq 1$, \eqref{Eq. U.condition} yields
\begin{equation}\label{Eq. A.cap.U.volume}
    \frac{1}{2}2^{\ell d} \leq |A\cap U| \leq \frac{3}{2}2^{\ell d}.
\end{equation}

To simplify the notation, we can assume wlog that ${U=[1,2^{\ell}]^{d-1}\times [1, 2^{\ell+1}]}$. As discussed before, for each point $p\in\calP^B_j(A\cap U)$ in the projection, $\ell_p^j\subset A\cap U$ and the lines are disjoint. Moreover, the size of the lines  is constant $r_j\coloneqq |\ell_p^j|$, hence $|\calP_{j}^B(A\cap U)|r_j = \sum_{p\in\calP_{j}^B(A\cap U)} |\ell_p^j| \leq |A\cap U|$. Together with the upper bound \eqref{Eq. A.cap.U.volume}, this yields 
\begin{equation}\label{Eq: Upper.bound.bad.points}
    |\calP_{j}^B(A\cap U)| \leq \frac{3}{2}2^{\ell d}r_j^{-1}.
\end{equation}
Using the isometric inequality, the lower bound on \eqref{Eq. A.cap.U.volume} yields $d2^{\frac{1}{d}}2^{\ell(d-1)}\leq |\fext (A\cap U)|$. As 
%
\begin{align*}
\frac{1}{2d}|\fext (A\cap U)| &\leq |\fint (A\cap U)| = |\fint(A\cap U) \cap \fint U| + |\fint(A\cap U) \cap(U\setminus \fint U)|\\
%
&\leq 2\sum_{i=1}^d |\calP_{i}(A\cap U)| + |\fint A\cap U| \leq 2\sum_{i=1}^d |\calP_{i}(A\cap U)| + |\fext A\cap U|,
\end{align*}
we get
\begin{equation}\label{Eq: Lemma.geo.discreta.3}
    2^{\frac{1}{d}-1}2^{\ell(d-1)}\leq 2\sum_{i=1}^d |\calP_{i}(A\cap U)| + |\fext A\cap U|
\end{equation}

We again consider two cases:
\begin{itemize}
    \item[(a)] If $|\calP_{j}(A\cap U)|> \frac{7}{8}|U_j|$ for some $j=1,\dots, d$, by \eqref{Eq: Upper.bound.bad.points} and \eqref{Eq: upper.bound.good.points} we get
    \begin{align*}
    \frac{7}{8}|U_j| < |\calP_{j}(A\cap U)| \leq |\fext A \cap U| + \frac{3}{2}2^{\ell d}r_j^{-1}.
    \end{align*}
    A simple calculation shows that $\frac{1}{8}2^{\ell(d-1)}\leq \frac{7}{8}|U_j| - \frac{3}{2}2^{\ell d}r_j^{-1}$, therefore 
    \begin{equation}\label{Eq: upper.bound.big.projections.1}
        \frac{1}{8}2^{\ell(d-1)} \leq |\fext A \cap U|.
    \end{equation}

    \begin{align*}
        \frac{7}{8}|U_j| - \frac{3}{2}2^{\ell d}r_j^{-1} &= \frac{1}{8r_j}( 7|U| - 12 \times 2^{\ell d})\\
        %
        &=\frac{1}{8r_j}2^{\ell(d-1)}(7\times 2^{\ell + 1} - 12\times 2^{\ell}) = \frac{1}{4}2^{\ell(d-1)}\frac{(2^{\ell})}{r_j} \geq \frac{1}{8}2^{\ell(d-1)},
    \end{align*}

    \item[(b)] If $|\calP_{i}(A\cap U)|\leq \frac{7}{8} |U_i|$ for all $i$, by Lemma \ref{Lemma: Geo.discreta.1}, there is a constant $c= c(d)$ such that
    \begin{equation}\label{Eq: Lemma.geo.discreta.2}
        \sum_{i=1}^d |\calP_{i}(A\cap U)|\leq c|\fext A\cap U|.
    \end{equation}
    Together with \eqref{Eq: Lemma.geo.discreta.3}, this yields
    \begin{equation}\label{Eq: upper.bound.big.projections.2}
        2^{\ell(d-1)}\leq \frac{2c+1}{2^{\frac{1}{d}-1}}|\fext A\cap U|.
    \end{equation}
\end{itemize}

Equations \eqref{Eq: upper.bound.big.projections.1} and \eqref{Eq: upper.bound.big.projections.2} shows the desired results taking $b\coloneqq \max \{8, {(2c+1)}{2^{1-\frac{1}{d}}}\}$.
\end{proof}


\begin{proposition}\label{Proposition1}For the functions $B_0,\dots,B_k$ defined above, there exists constants $b_1,b_2$ depending only on $d$ and $r$ such that 
\begin{equation}\label{Eq: Prop.1.FFS.i}
    |\partial\mathfrak{C}_\ell(\gamma)| \leq b_1\frac{|\fext \I(\gamma)|}{2^{\ell(d-1)}} \leq b_1 \frac{|\gamma|}{2^{\ell(d-1)}}
\end{equation}
    and 
\begin{equation}\label{Eq: Prop.1.FFS.ii}
    |B_\ell(\gamma)\Delta B_{\ell+1}(\gamma)| \leq b_2 2^{\ell} |\gamma|
\end{equation}
for every $\ell\in\{0,\dots,k\}$ and $\gamma\in\Gamma_0(n)$.
\end{proposition}

    \begin{proof} Fix $\ell\in\{0,\dots,k\}$. Given a pair $(C_{\ell}, C_\ell^\prime)$, we will write $C_{\ell} \sim C_{\ell}^\prime$ when $(C_{\ell}, C_{\ell}^\prime) \in  \partial\mathfrak{C}_\ell(\gamma)$, and the union we be denoted by $U = C_{\ell} \cup C_{\ell}^\prime$. Then, 
   defining $\overline{\mathscr{C}}_{\ell} = \{C_{\ell} \in \partial \mathfrak{C}_\ell(\gamma): C_{\ell} \sim C_{\ell}^\prime \text{ for some } C_{\ell}^prime \notin  \mathfrak{C}_\ell(\gamma)\}$, for any $A\Subset\Z^d$, 
    
    \begin{align*}
        \sum_{\substack{(C_{\ell}, C^\prime_\ell)\in \partial \mathfrak{C}_\ell(\gamma)}}|A \cap \{C_{\ell} \cup C_{\ell}^\prime\}| &\leq \sum_{\substack{C_{\ell}\in \overline{\mathscr{C}}_{\ell}}}\sum_{\substack{C_{\ell}^\prime\notin  \mathfrak{C}_\ell(\gamma)\\ C_{\ell} \sim C_{\ell}^\prime}}  \left(|A \cap C_{\ell}| + |A \cap C_{\ell}^\prime|\right)\\
        %
        &\leq \sum_{\substack{C_{\ell}\in\overline{\mathscr{C}}_{\ell}}} 2d |A \cap C_{\ell}| + \sum_{C_{\ell}^\prime\notin \mathfrak{C}_\ell(\gamma)} 2d|A \cap C_{\ell}^\prime| \leq  2d |A|\\
    \end{align*}

 Any pair of cubes $C_{\ell}\sim C_{\ell}^\prime$ are in the hypothesis of Lemma \ref{Lemma: Proposicao1.Aux1}, hence $ b 2^{\ell(d-1)} \leq |\fext \I(\gamma) \cap U|$. Applying equation above for $A=\fext\I(\gamma)$ we get that
    \begin{equation*}
       \frac{b}{2d} 2^{\ell(d-1)}| \partial \mathfrak{C}_\ell(\gamma)| \leq \frac{1}{2d} \sum_{\substack{(C_{\ell}, C^\prime_\ell)\in \partial \mathfrak{C}_\ell(\gamma)}} |\fext \I(\gamma) \cap \{C_{\ell} \cup C_{\ell}^\prime\}| \leq |\fext \I(\gamma)|,
    \end{equation*}
that concludes \eqref{Eq: Prop.1.FFS.i} for $b_1\coloneqq 2d/b$.

Given $C_{(\ell+1)}\in \mathscr{C}_{(\ell+1)}(B_{\ell+1}(\gamma)\setminus B_{\ell}(\gamma))$, there is a $\ell$-cube $C_{\ell}^\prime\subset C_{r(\ell + 1)}$ with $C_{\ell}^\prime\notin \mathfrak{C}_\ell(\gamma)$, otherwise  $(B_{\ell+1}(\gamma)\setminus B_{\ell}(\gamma))\cap C_{(\ell+1)} = \emptyset$. There is also a $\ell$-cube $C_{\ell}\subset C_{r(\ell + 1)}$ with $C_{\ell}\in \mathfrak{C}_\ell(\gamma)$, otherwise we would have 
\begin{align*}
    |\I(\gamma)\cap C_{(\ell+1)}| &= \sum_{C_{\ell}\subset C_{(\ell+1)}} |\I(\gamma)\cap C_{\ell}| \leq \frac{1}{2} |C_{(\ell+1)}|.
\end{align*}

Moreover, we can assume that $C_{\ell}$ and $C_{\ell}^\prime$ share a face. Again, we use Lemma \ref{Lemma: Proposicao1.Aux1} to get,
\begin{align}\label{Eq: bound.on.c.bar}
    |B_{\ell+1}(\gamma)\setminus B_\ell(\gamma)\cap C_{(\ell+1)}| &\leq |C_{(\ell+1)}| =2^{d}2^{\ell}2^{\ell(d-1)} \nonumber\\
                %
                &\leq 2^{d}2^{\ell} b|\fext \I(\gamma) \cap \{C_{\ell} \cup C_{\ell}^\prime\}| \nonumber\\
                %
                &\leq 2^{d}2^{\ell}b|\fext \I(\gamma) \cap C_{(\ell+1)}|.
\end{align}
Therefore, 
\begin{align*}
    |B_{\ell+1}(\gamma)\setminus B_\ell(\gamma)| &= \sum_{C_{(\ell+1)}\in \mathscr{C}_{(\ell+1)}(B_{\ell+1}(\gamma)\setminus B_\ell(\gamma))} |B_{\ell+1}(\gamma)\setminus B_\ell(\gamma)\cap C_{(\ell+1)}| \\
    %
    &\leq \sum_{C_{(\ell+1)}\in \mathscr{C}_{(\ell+1)}(B_{\ell+1}(\gamma)\setminus B_\ell(\gamma))} 2^{d}2^{\ell}b|\fext \I(\gamma) \cap C_{(\ell+1)}| \leq  \frac{b_2}{2}2^{\ell}|\fext \I(\gamma)|.
\end{align*}
with $b_2=b2^{d+1}$. To get the same bound for $|B_{\ell}(\gamma)\setminus B_{\ell+1}(\gamma)|$ we repeat a similar argument, covering $B_{\ell}(\gamma)\setminus B_{\ell+1}(\gamma)$ with $(\ell+1)$-cubes. 
    \end{proof}

\begin{corollary}
    For any $\ell>0$ and any two contours $\gamma_1,\gamma_2 \in \Gamma_0(n)$ such that $B_\ell(\gamma_1)=B_{\ell}(\gamma_2)$, there exists a constant $b_3>0$ such that 
    \begin{equation*}
        \d_2(\gamma_1,\gamma_2)\leq 4 \varepsilon b_3 2^{\frac{\ell}{2}} n^{\frac{1}{2}}. 
    \end{equation*} 
\end{corollary}

\begin{proof}
    This is a simple application of the triangular inequality, since $d_2(\gamma_1,\gamma_2) \leq d_2(\gamma_1,B_\ell(\gamma_1)) + d_2(\gamma_2,B_\ell(\gamma_2))$ and 
    \begin{align*}
        d_2(\gamma_1,B_\ell(\gamma_1)) &\leq \sum_{i=1}^\ell d_2(B_i(\gamma_1),B_{i-1}(\gamma_1)) = \sum_{i=1}^\ell 2\varepsilon\sqrt{B_i(\gamma_1)\Delta B_{i-1}(\gamma_1)} \\
        %
        & \leq \sum_{i=1}^\ell 2\varepsilon\sqrt{b_2} 2^{\frac{i}{2}} \sqrt{n}  \leq 2\varepsilon\sqrt{b_2}(\sqrt{2}+1)2^{\frac{\ell}{2}} \sqrt{n} 
    \end{align*}
    where in the second to last equation used \eqref{Eq: Prop.1.FFS.ii}. As the same bound holds for $d_2(\gamma_2,B_\ell(\gamma_2))$, the corollary is proved by taking ${b_3 = \sqrt{b_2}(\sqrt{2}+1)}$
\end{proof}

\begin{remark}\label{Rmk: Bounding_N_by_B_ell}
    This corollary shows that we can create a covering of $\Gamma_0(n)$, indexed by $B_\ell(\Gamma_0(n))$, of ball with radius $4 \varepsilon b_3 2^{\frac{\ell}{2}} n^{\frac{1}{2}}$. Therefore $N(\Gamma_0(n), \d_2, 4\varepsilon b_3 2^{\frac{\ell}{2}} n^{\frac{1}{2}}) \leq |B_\ell(\Gamma_0(n))|$. 
\end{remark}
In the next proposition we bound $|B_\ell(\Gamma_0(n))|$, again following \cite{FFS84}.

\begin{proposition}\label{Prop: Proposition_2_FFS}
    There exists a constant $b_4\coloneqq b_4(d)$ such that, for any $n\in\mathbb{N}$,
    \begin{equation}\label{Eq: Proposition_2_FFS}
        |B_\ell(\Gamma_0(n))|\leq \exp{\frac{b_4\ell n}{2^{\ell(d-1)}}},
    \end{equation}
    that is, the number of coarse-grained contours in $B_\ell(\Gamma_0(n))$ is bounded above by an exponential term. 
\end{proposition}

\begin{proof}
Start by noticing that $|B_\ell(\Gamma_0(n))| = |\partial B_\ell(\Gamma_0(n))|$, and to each $B_\ell(\gamma)$ we can associate a contour $\xi_\ell(\gamma)$ with $\I(\xi_\ell) = B_{\ell}(\gamma)$. Given $\xi_\ell\in\xi_\ell(\Gamma_0(n))$ for $\ell\in\{1,\dots,k\}$, let $\{\xi_\ell^{(1)}, \xi_\ell^{(2)},\dots,\xi_\ell^{(m)}\}$ be the connected components of $\xi_\ell$. To connected component $\xi_\ell^{(i)}$ we can uniquely associate a pair $(C_\ell,C^\prime_\ell)\in \partial\mathfrak{C}_\ell(\gamma)$ with $B_{C_\ell}\subset\I(\xi_\ell^{(i)})$. By Lemma \ref{Lemma: Proposicao1.Aux1}, there are at least $b^{-1}2^{\ell (d-1)}$ points of $\fext\I(\gamma)$ in $C_\ell\cup C_\ell^\prime$. Hence $\xi_\ell$ has at most $\frac{b n}{2^{\ell(d-1)}}$ connected components and we take $M_n \coloneqq \frac{b n}{2^{\ell(d-1)}}$.
Moreover, by Lemma \ref{Proposition1}, $ |\xi_\ell| = \sum_{i=1}^{M_n} |\xi_\ell^{(i)}| = 2^{\ell(d-1)}|\partial\mathfrak{C}_\ell(\gamma)| \leq b_1 n$.

Fixed $x_1,x_2,\dots, x_{M_n}\in 2^\ell\Z^d$, and $s^1,s^2,\dots, s^{M_n}\in 2^{\ell (d-1)}\mathbb{N}$, if $\Gamma(\{x_i\}_{i=1}^{M_n},\{s_i\}_{i=1}^{M_n})$ is the number of coarse-grained contours with $x_i\in \I(\xi_\ell^{(i)})$ and $|\xi_\ell^{(i)}| = s^i$ for every $i$, then
\begin{align}\label{Eq: Proposition_2_FFS_Aux_1}
        \Gamma(\{x_i\}_{i=1}^{M_n},\{s_i\}_{i=1}^{M_n}) \leq \exp{\left( \ln{(4d)} \sum_{i=1}^{M_n} \frac{s^i}{2^{\ell (d-1)}}\right)} \leq \exp{\left( b_1\ln{(4d)}   \frac{n}{2^{\ell (d-1)}}\right)}.
\end{align}
The number of choices of $s^1,s^2,\dots, s^1\in 2^{\ell d}\mathbb{N}$ with $\sum_{i=1}^q s^i \leq b_1 n$ is less then $2^{\frac{b_1 n}{2^{\ell(d-1)}}}$. This is a simple bound on the number of ways of putting up to $\frac{b_1 n}{2^{\ell(d-1)}}$ balls on $M_n$ spaces. 

It remains know to bound the number of choices for $({x_i})_{i=1}^{M_n}$. Set $d_1 = |x_1|$ and $d_i = |x_i - x_{i-1}|$ for $i=2,\dots, M_n$. For every $i=1,\dots, M_n$, we choose $y_1,\dots,y_{M_n}\in \I({\gamma})$ such that $|x_i - y_i| = d(x_i, \I({\gamma}))$. With this choice we get $d(x_i,y_i) \leq d2^{\ell}$. Since all the cubes are not connected, $d(y_i,y_{i-1})>2^\ell$, hence
\begin{equation*}
    d(x_i,y_i) \leq d |y_i - y_{i-1}|.
\end{equation*}
As all $y_i$ are in $\I(\gamma)$ and $y_0=x_0=0$, we can reorder the terms to minimize the sum of distances, getting 
\begin{equation*}
    \sum_{i=1}^{M_n} d(y_i, y_{i-1}) \leq 2d |\fext\I(\gamma)| = 2dn.
\end{equation*}
This yields 
\begin{equation}\label{Eq: bound.sum.of.d_i}
    \sum_{i=1}^{M_n} d_i \leq 2\sum_{i=1}^{M_n} d(x_i,y_i) + \sum_{i=1}^{M_n} d(y_i,y_{i-1}) \leq (2d + 1)\sum_{i=1}^{M_n} d(y_i, y_{i-1}) \leq (2d + 1)^2 n
\end{equation}
 Fixing $d_1, d_2,\dots, d_q$, the number of ways of choosing $x_1,\dots,x_q$ is bounded by $\prod_{i=1}^{M_n} (2d_i)^{d}$. The maximum of this quantity is reached when all the distances are the same. Assuming $d_1=\dots =d_q=d^*$, equation \eqref{Eq: bound.sum.of.d_i} yields 
 \begin{equation*}
     d^* \leq \frac{(2d + 1)^2 n}{M_n} = \frac{(2d+1)^2 2^{\ell(d-1)}}{b},
 \end{equation*}
so we have at most 
\begin{equation}\label{Eq: Proposition_2_FFS_Aux_2}
    (\frac{(2d+1)^2 2^{\ell(d-1)}}{b})^{\frac{d b n}{2^{\ell(d-1)}}} \leq \exp\left\{(2d(d-1)\ln{(2)}b\ln{(\frac{(2d+1)}{b})}) \frac{\ell n}{2^{\ell(d-1)}}\right\}
\end{equation}
 ways of choose $x_1,\dots,x_{M_n}$ given $d_1,\dots,d_{M_n}$. The number of solutions $(d_1,\dots,d_{M_n})$ to $\sum_{i=1}^{M_n} d_i =N $ is $\binom{N-1}{M_n}$. As $\binom{N-1}{M_n} < \frac{N^{M_n}}{M_n!}\leq (\frac{eN}{M_n})^{M_n}$, the number of solutions of \eqref{Eq: bound.sum.of.d_i} is bounded by 
 \begin{align}\label{Eq: Proposition_2_FFS_Aux_3}
     \sum_{N=1}^{(2d+1)^2n}\left(\frac{eN}{M_n}\right)^{M_n} &\leq\int_{0}^{(2d+1)^2n +1} \left(\frac{ex}{M_n}\right)^{M_n}dx =  e^{M_n}\frac{(2d+1)^2n +1}{M_n + 1}\left(\frac{(2d+1)^2n +1}{M_n} \right)^{M_n} \nonumber \\
     %
     &\leq \left(\frac{2e(2d+1)^2 2^{\ell(d-1)}}{b} \right)^{\frac{b n}{2^{\ell(d-1)}}} \leq \exp{\left\{4eb^{-1}\ln(2)(2d+1)^2 \frac{\ell n}{2^{\ell(d-1)}} \right\}}.
 \end{align}
 Equations \eqref{Eq: Proposition_2_FFS_Aux_1}, \eqref{Eq: Proposition_2_FFS_Aux_2} and \eqref{Eq: Proposition_2_FFS_Aux_3} proves the proposition for $b_4=\max{\{ b_1\ln{(4d)}, 2d(d-1)\ln{(2)}b\ln{(\frac{(2d+1)}{b})}, eb^{-1}\ln(2)(2d+1)^2 \}}$.
\end{proof}

We are ready to prove the main proposition.

\textit{Proof of Proposition \ref{Prop: Bound.bad.event.1}:} As $N(\Gamma_0(n), \d_2, \epsilon)$ is decreasing in $\epsilon$, we can use Dudley's integral bound to bound
    \begin{align}\label{Eq: Prop_bound_Ec_1}
        {\mathbb{E}\left[\sup_{\gamma\in\Gamma_0(n)}{\Delta_{\I(\gamma)}(h)}\right]} &\leq \int_{0}^\infty \sqrt{\log N(\Gamma_0(n), \d_2, \epsilon)}d\epsilon \leq \sum_{\ell=0}^\infty\sqrt{\log N(\Gamma_0(n), \d_2, \ell)}\nonumber\\
        %
        &\leq 2\varepsilon b_3 n^{\frac{1}{2}}\sum_{\ell=1}^\infty (2^{\frac{\ell}{2}} - 2^{\frac{\ell-1}{2}})\sqrt{\log N(\Gamma_0(n), \d_2,\varepsilon b_3 2^{\frac{\ell}{2}}n^{\frac{1}{2}})}.
    \end{align}

Remember that, as discussed in Remark \ref{Rmk: Bounding_N_by_B_ell}, $N(\Gamma_0(n), \d_2,\varepsilon b_3 2^{\frac{\ell}{2}}n^{\frac{1}{2}})\leq |B_{\ell}(\Gamma_0(n))|$. Therefore, by Proposition \ref{Prop: Proposition_2_FFS},
\begin{align}\label{Eq: Prop_bound_Ec_2}
    \sum_{\ell=1}^\infty (2^{\frac{\ell}{2}} - 2^{\frac{\ell-1}{2}})\sqrt{\log N(\Gamma_0(n), \d_2,\varepsilon b_3 2^{\frac{\ell}{2}}n^{\frac{1}{2}})} &\leq \sum_{\ell=1}^{\infty} (2^{\frac{\ell}{2}} - 2^{\frac{\ell-1}{2}}) \sqrt{\frac{b_4\ell n}{2^{\ell(d-1)}}} \nonumber \\
    %
    &\leq \sqrt{b_4}n^{\frac{1}{2}}(1 - \frac{\sqrt{2}}{2})\sum_{\ell=1}^{\infty}\sqrt{\frac{\ell}{2^{\ell(d-2)}}}.
\end{align}
Denoting $\tau(d) = \sum_{\ell=1}^{\infty}\sqrt{\frac{\ell}{2^{\ell(d-2)}}}$, and $b_5 = 2\tau(d)b_3\sqrt{b_4}(1 - \frac{\sqrt{2}}{2})$, equation \eqref{Eq: Prop_bound_Ec_1} and \eqref{Eq: Prop_bound_Ec_2} yields
\begin{equation*}
    {\mathbb{E}\left[\sup_{\gamma\in\Gamma_0(n)}{\Delta_{\I(\gamma)}(h)}\right]} \leq b_5 \varepsilon n.
\end{equation*}
To apply Theorem \ref{Theo: Theo_2.2.27_Talagrand}, notice that, by the isoperimetric inequality $\diam(\Gamma_0(n)) = \sup_{\gamma_1,\gamma_2\in\Gamma_0(n)} \leq 4\varepsilon\sqrt{|\I(n)|}\leq 4\varepsilon n^{\frac{1}{2}(1 + \frac{1}{d-1})}$. Hence, for $\varepsilon$ small enough
\begin{align*}
    \mathbb{P}\left( \sup_{\gamma}\Delta_{\I(\gamma)}(h) \geq \frac{c_2}{2}n \right) \leq \mathbb{P}\left( \sup_{\gamma}\Delta_{\I(\gamma)}(h) \geq L(b_5 \varepsilon n + n^{\frac{1}{2}(1 - \frac{1}{d-1})}n^{\frac{1}{2}(1 + \frac{1}{d-1})}) \right)\leq e^{-\frac{n^{(1 - \frac{1}{d-1})}}{\varepsilon^2}}.
\end{align*}
The union bound yields $\mathbb{P}(\mathcal{E}^c)\leq e^{-\frac{C}{\varepsilon^2}}$ for $C>0$ large enough.
