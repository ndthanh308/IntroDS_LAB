%% bare_adv.tex
%% V1.4b
%% 2015/08/26
%% by Michael Shell
%% See: 
%% http://www.michaelshell.org/
%% for current contact information.
%%
%% This is a skeleton file demonstrating the advanced use of IEEEtran.cls
%% (requires IEEEtran.cls version 1.8b or later) with an IEEE Computer
%% Society journal paper.
%%
%% Support sites:
%% http://www.michaelshell.org/tex/ieeetran/
%% http://www.ctan.org/pkg/ieeetran
%% and
%% http://www.ieee.org/

%%*************************************************************************
%% Legal Notice:
%% This code is offered as-is without any warranty either expressed or
%% implied; without even the implied warranty of MERCHANTABILITY or
%% FITNESS FOR A PARTICULAR PURPOSE! 
%% User assumes all risk.
%% In no event shall the IEEE or any contributor to this code be liable for
%% any damages or losses, including, but not limited to, incidental,
%% consequential, or any other damages, resulting from the use or misuse
%% of any information contained here.
%%
%% All comments are the opinions of their respective authors and are not
%% necessarily endorsed by the IEEE.
%%
%% This work is distributed under the LaTeX Project Public License (LPPL)
%% ( http://www.latex-project.org/ ) version 1.3, and may be freely used,
%% distributed and modified. A copy of the LPPL, version 1.3, is included
%% in the base LaTeX documentation of all distributions of LaTeX released
%% 2003/12/01 or later.
%% Retain all contribution notices and credits.
%% ** Modified files should be clearly indicated as such, including  **
%% ** renaming them and changing author support contact information. **
%%*************************************************************************


% *** Authors should verify (and, if needed, correct) their LaTeX system  ***
% *** with the testflow diagnostic prior to trusting their LaTeX platform ***
% *** with production work. The IEEE's font choices and paper sizes can   ***
% *** trigger bugs that do not appear when using other class files.       ***                          ***
% The testflow support page is at:
% http://www.michaelshell.org/tex/testflow/


% IEEEtran V1.7 and later provides for these CLASSINPUT macros to allow the
% user to reprogram some IEEEtran.cls defaults if needed. These settings
% override the internal defaults of IEEEtran.cls regardless of which class
% options are used. Do not use these unless you have good reason to do so as
% they can result in nonIEEE compliant documents. User beware. ;)
%
%\newcommand{\CLASSINPUTbaselinestretch}{1.0} % baselinestretch
%\newcommand{\CLASSINPUTinnersidemargin}{1in} % inner side margin
%\newcommand{\CLASSINPUToutersidemargin}{1in} % outer side margin
%\newcommand{\CLASSINPUTtoptextmargin}{1in}   % top text margin
%\newcommand{\CLASSINPUTbottomtextmargin}{1in}% bottom text margin




%
\documentclass[10pt,journal,compsoc]{IEEEtran}

\usepackage{amsmath,amsfonts,amssymb}
\usepackage{algorithmic}
\usepackage{algorithm}
\usepackage{array}
\usepackage[caption=false,font=normalsize,labelfont=sf,textfont=sf]{subfig}
\usepackage{textcomp}
\usepackage{stfloats}
\usepackage{url}
\usepackage{verbatim}
\usepackage{graphicx}
\usepackage{caption}
\usepackage{hyperref}

\def\sig{\text{sign\,}}
\def\diag{\text{diag\,}}
\def\exp{\text{exp\,}}
\def\eg{\emph{e.g.}\,}
\def\ie{\emph{i.e.}\,}

\usepackage{tikz}
\usepackage{comment}
\usepackage{color}

\usepackage{multirow}
\usepackage{booktabs}
\usepackage{arydshln}
\usepackage{marvosym}
\newcommand{\ra}[1]{\renewcommand{\arraystretch}{#1}}

% If IEEEtran.cls has not been installed into the LaTeX system files,
% manually specify the path to it like:
% \documentclass[10pt,journal,compsoc]{../sty/IEEEtran}


% For Computer Society journals, IEEEtran defaults to the use of 
% Palatino/Palladio as is done in IEEE Computer Society journals.
% To go back to Times Roman, you can use this code:
%\renewcommand{\rmdefault}{ptm}\selectfont





% Some very useful LaTeX packages include:
% (uncomment the ones you want to load)



% *** MISC UTILITY PACKAGES ***
%
%\usepackage{ifpdf}
% Heiko Oberdiek's ifpdf.sty is very useful if you need conditional
% compilation based on whether the output is pdf or dvi.
% usage:
% \ifpdf
%   % pdf code
% \else
%   % dvi code
% \fi
% The latest version of ifpdf.sty can be obtained from:
% http://www.ctan.org/pkg/ifpdf
% Also, note that IEEEtran.cls V1.7 and later provides a builtin
% \ifCLASSINFOpdf conditional that works the same way.
% When switching from latex to pdflatex and vice-versa, the compiler may
% have to be run twice to clear warning/error messages.






% *** CITATION PACKAGES ***
%
\ifCLASSOPTIONcompsoc
  % The IEEE Computer Society needs nocompress option
  % requires cite.sty v4.0 or later (November 2003)
  \usepackage[nocompress]{cite}
\else
  % normal IEEE
  \usepackage{cite}
\fi
% cite.sty was written by Donald Arseneau
% V1.6 and later of IEEEtran pre-defines the format of the cite.sty package
% \cite{} output to follow that of the IEEE. Loading the cite package will
% result in citation numbers being automatically sorted and properly
% "compressed/ranged". e.g., [1], [9], [2], [7], [5], [6] without using
% cite.sty will become [1], [2], [5]--[7], [9] using cite.sty. cite.sty's
% \cite will automatically add leading space, if needed. Use cite.sty's
% noadjust option (cite.sty V3.8 and later) if you want to turn this off
% such as if a citation ever needs to be enclosed in parenthesis.
% cite.sty is already installed on most LaTeX systems. Be sure and use
% version 5.0 (2009-03-20) and later if using hyperref.sty.
% The latest version can be obtained at:
% http://www.ctan.org/pkg/cite
% The documentation is contained in the cite.sty file itself.
%
% Note that some packages require special options to format as the Computer
% Society requires. In particular, Computer Society  papers do not use
% compressed citation ranges as is done in typical IEEE papers
% (e.g., [1]-[4]). Instead, they list every citation separately in order
% (e.g., [1], [2], [3], [4]). To get the latter we need to load the cite
% package with the nocompress option which is supported by cite.sty v4.0
% and later.





% *** GRAPHICS RELATED PACKAGES ***
%
\ifCLASSINFOpdf
  % \usepackage[pdftex]{graphicx}
  % declare the path(s) where your graphic files are
  % \graphicspath{{../pdf/}{../jpeg/}}
  % and their extensions so you won't have to specify these with
  % every instance of \includegraphics
  % \DeclareGraphicsExtensions{.pdf,.jpeg,.png}
\else
  % or other class option (dvipsone, dvipdf, if not using dvips). graphicx
  % will default to the driver specified in the system graphics.cfg if no
  % driver is specified.
  % \usepackage[dvips]{graphicx}
  % declare the path(s) where your graphic files are
  % \graphicspath{{../eps/}}
  % and their extensions so you won't have to specify these with
  % every instance of \includegraphics
  % \DeclareGraphicsExtensions{.eps}
\fi
% graphicx was written by David Carlisle and Sebastian Rahtz. It is
% required if you want graphics, photos, etc. graphicx.sty is already
% installed on most LaTeX systems. The latest version and documentation
% can be obtained at: 
% http://www.ctan.org/pkg/graphicx
% Another good source of documentation is "Using Imported Graphics in
% LaTeX2e" by Keith Reckdahl which can be found at:
% http://www.ctan.org/pkg/epslatex
%
% latex, and pdflatex in dvi mode, support graphics in encapsulated
% postscript (.eps) format. pdflatex in pdf mode supports graphics
% in .pdf, .jpeg, .png and .mps (metapost) formats. Users should ensure
% that all non-photo figures use a vector format (.eps, .pdf, .mps) and
% not a bitmapped formats (.jpeg, .png). The IEEE frowns on bitmapped formats
% which can result in "jaggedy"/blurry rendering of lines and letters as
% well as large increases in file sizes.
%
% You can find documentation about the pdfTeX application at:
% http://www.tug.org/applications/pdftex





% *** MATH PACKAGES ***
%
%\usepackage{amsmath}
% A popular package from the American Mathematical Society that provides
% many useful and powerful commands for dealing with mathematics.
%
% Note that the amsmath package sets \interdisplaylinepenalty to 10000
% thus preventing page breaks from occurring within multiline equations. Use:
%\interdisplaylinepenalty=2500
% after loading amsmath to restore such page breaks as IEEEtran.cls normally
% does. amsmath.sty is already installed on most LaTeX systems. The latest
% version and documentation can be obtained at:
% http://www.ctan.org/pkg/amsmath





% *** SPECIALIZED LIST PACKAGES ***
%\usepackage{acronym}
% acronym.sty was written by Tobias Oetiker. This package provides tools for
% managing documents with large numbers of acronyms. (You don't *have* to
% use this package - unless you have a lot of acronyms, you may feel that
% such package management of them is bit of an overkill.)
% Do note that the acronym environment (which lists acronyms) will have a
% problem when used under IEEEtran.cls because acronym.sty relies on the
% description list environment - which IEEEtran.cls has customized for
% producing IEEE style lists. A workaround is to declared the longest
% label width via the IEEEtran.cls \IEEEiedlistdecl global control:
%
% \renewcommand{\IEEEiedlistdecl}{\IEEEsetlabelwidth{SONET}}
% \begin{acronym}
%
% \end{acronym}
% \renewcommand{\IEEEiedlistdecl}{\relax}% remember to reset \IEEEiedlistdecl
%
% instead of using the acronym environment's optional argument.
% The latest version and documentation can be obtained at:
% http://www.ctan.org/pkg/acronym


%\usepackage{algorithmic}
% algorithmic.sty was written by Peter Williams and Rogerio Brito.
% This package provides an algorithmic environment fo describing algorithms.
% You can use the algorithmic environment in-text or within a figure
% environment to provide for a floating algorithm. Do NOT use the algorithm
% floating environment provided by algorithm.sty (by the same authors) or
% algorithm2e.sty (by Christophe Fiorio) as the IEEE does not use dedicated
% algorithm float types and packages that provide these will not provide
% correct IEEE style captions. The latest version and documentation of
% algorithmic.sty can be obtained at:
% http://www.ctan.org/pkg/algorithms
% Also of interest may be the (relatively newer and more customizable)
% algorithmicx.sty package by Szasz Janos:
% http://www.ctan.org/pkg/algorithmicx




% *** ALIGNMENT PACKAGES ***
%
%\usepackage{array}
% Frank Mittelbach's and David Carlisle's array.sty patches and improves
% the standard LaTeX2e array and tabular environments to provide better
% appearance and additional user controls. As the default LaTeX2e table
% generation code is lacking to the point of almost being broken with
% respect to the quality of the end results, all users are strongly
% advised to use an enhanced (at the very least that provided by array.sty)
% set of table tools. array.sty is already installed on most systems. The
% latest version and documentation can be obtained at:
% http://www.ctan.org/pkg/array


%\usepackage{mdwmath}
%\usepackage{mdwtab}
% Also highly recommended is Mark Wooding's extremely powerful MDW tools,
% especially mdwmath.sty and mdwtab.sty which are used to format equations
% and tables, respectively. The MDWtools set is already installed on most
% LaTeX systems. The lastest version and documentation is available at:
% http://www.ctan.org/pkg/mdwtools


% IEEEtran contains the IEEEeqnarray family of commands that can be used to
% generate multiline equations as well as matrices, tables, etc., of high
% quality.


%\usepackage{eqparbox}
% Also of notable interest is Scott Pakin's eqparbox package for creating
% (automatically sized) equal width boxes - aka "natural width parboxes".
% Available at:
% http://www.ctan.org/pkg/eqparbox




% *** SUBFIGURE PACKAGES ***
%\ifCLASSOPTIONcompsoc
%  \usepackage[caption=false,font=footnotesize,labelfont=sf,textfont=sf]{subfig}
%\else
%  \usepackage[caption=false,font=footnotesize]{subfig}
%\fi
% subfig.sty, written by Steven Douglas Cochran, is the modern replacement
% for subfigure.sty, the latter of which is no longer maintained and is
% incompatible with some LaTeX packages including fixltx2e. However,
% subfig.sty requires and automatically loads Axel Sommerfeldt's caption.sty
% which will override IEEEtran.cls' handling of captions and this will result
% in non-IEEE style figure/table captions. To prevent this problem, be sure
% and invoke subfig.sty's "caption=false" package option (available since
% subfig.sty version 1.3, 2005/06/28) as this is will preserve IEEEtran.cls
% handling of captions.
% Note that the Computer Society format requires a sans serif font rather
% than the serif font used in traditional IEEE formatting and thus the need
% to invoke different subfig.sty package options depending on whether
% compsoc mode has been enabled.
%
% The latest version and documentation of subfig.sty can be obtained at:
% http://www.ctan.org/pkg/subfig




% *** FLOAT PACKAGES ***
%
%\usepackage{fixltx2e}
% fixltx2e, the successor to the earlier fix2col.sty, was written by
% Frank Mittelbach and David Carlisle. This package corrects a few problems
% in the LaTeX2e kernel, the most notable of which is that in current
% LaTeX2e releases, the ordering of single and double column floats is not
% guaranteed to be preserved. Thus, an unpatched LaTeX2e can allow a
% single column figure to be placed prior to an earlier double column
% figure.
% Be aware that LaTeX2e kernels dated 2015 and later have fixltx2e.sty's
% corrections already built into the system in which case a warning will
% be issued if an attempt is made to load fixltx2e.sty as it is no longer
% needed.
% The latest version and documentation can be found at:
% http://www.ctan.org/pkg/fixltx2e


%\usepackage{stfloats}
% stfloats.sty was written by Sigitas Tolusis. This package gives LaTeX2e
% the ability to do double column floats at the bottom of the page as well
% as the top. (e.g., "% Figure environment removed

%The overview of the proposed MGP-STR method is depicted in Fig.~\ref{fig:overview}, which is mainly built upon the original Vision Transformer (ViT) model~\cite{dosovitskiy2020image}. We propose a tailored Adaptive Addressing and Aggregation (A$^3$) module to select a meaningful combination of tokens from ViT and integrate them into one output token corresponding to a specific character, denoted as Character A$^3$ module.Moreover, subword classification heads based on BPE A$^3$ module and WordPiece A$^3$ module are devised for subword predictions, so that the language information can be implicitly modeled. Finally, these multi-granularity predictions are merged via a simple and effective fusion strategy. 

The schematic overview of the proposed MGP-STR method is depicted in Fig.~\ref{fig:overview}, which is mainly built upon the original Vision Transformer (ViT) model~\cite{dosovitskiy2020image} (Sec.~\ref{Sec:ViT}). We propose a tailored Adaptive Addressing and Aggregation (A$^3$) module to select a meaningful combination of tokens from ViT and integrate them into one output token corresponding to a specific character, denoted as Character A$^3$ module (Sec.~\ref{sec:a3modules}). Moreover, subword classification heads based on BPE A$^3$ module and WordPiece A$^3$ module are devised for subword predictions, such that the language information can be implicitly modeled (Sec.~\ref{sec:MGP}). Finally, these multi-granularity predictions are merged via a fusion stage, which can be realized with two specially designed fusion strategies: Confidence-based Fusion Strategy (Sec.~\ref{sec:fuse}) and Learnable Fusion Strategy (Sec.~\ref{sec:learnable}).

\subsection{Vision Transformer Backbone}
\label{Sec:ViT}
The fundamental architecture of MGP-STR is Vision Transformer~\cite{dosovitskiy2020image,deit}, where the original image patches are directly utilized for image feature extraction by linear projection. 
As shown in Fig.~\ref{fig:overview}, an input RGB image  $ \mathbf{x} \in  \mathbb{R}^{H \times W \times C} $  is split into non-overlapping patches.
Concretely, the image is reshaped  into a sequence of flattened 2D patches $ \mathbf{x}_p \in  \mathbb{R}^{N \times (P^2 C)} $,
where $(P \times P) $ is the resolution of each image patch and $(P^2 C)$ is the number of feature channels of $ \mathbf{x}_p$. 
In this way, a 2D image is represented as a sequence with $N = HW/P^2$ tokens, which serve as the effective input sequence of Transformer blocks.
Then, these tokens of $ \mathbf{x}_p$  are linearly transcribed into $D$ dimension patch embeddings. 
Similar to the original ViT~\cite{dosovitskiy2020image} backbone, a learnable  $[class]$ token embedding with $D$ dimension is introduced into patch embeddings.
And position embeddings are also added to each patch embedding to retain the positional information,
where the standard learnable 1$D$ position embedding is employed.
Thus, the generation of patch embedding vector is formulated as follows:
\begin{equation}
\begin{split}
\mathbf{z}_0=[\mathbf{x}_{class}; \mathbf{x}^1_p\mathbf{E}; \mathbf{x}^2_p\mathbf{E}; \ldots; \mathbf{x}^N_p\mathbf{E}] + \mathbf{E}_{pos},
\end{split}
\end{equation}
where $\mathbf{x}_{class} \in \mathbb{R}^{ 1 \times D}$ is the $[class]$ embedding,  $\mathbf{E}  \in \mathbb{R}^{ (P^2 C) \times D } $ is a linear projection matrix and  $ \mathbf{E}_{pos} \in \mathbb{R}^{ (N+1) \times D }   $ is the position embedding.

The resultant feature sequence $\mathbf{z}_0  \in \mathbb{R}^{ (N+1) \times D} $ serves as the input of Transformer encoder blocks~\cite{dosovitskiy2020image},
which are mainly composed of Multi-head Self-Attention (MSA), Layer Normalization (LN), Multilayer Perceptron (MLP) and residual connection as in Fig.\ref{fig:overview}. The Transformer encoder block is formulated as:
\begin{equation}
\begin{split}
&\mathbf{z}_{l}^{\prime}=\text{MSA} (LN(\mathbf{z}_{l-1}))+ \mathbf{z}_{l-1}   \\
&\mathbf{z}_{l}=\text{MLP} (LN(\mathbf{z}_{l}^{\prime}))+ \mathbf{z}_{l}^{\prime}.
\end{split}
\end{equation}
Here, $l$ is the depth of Transformer block and  $ l=1 \ldots L $.
The MLP consists of two linear layers with GELU activation.
Finally, the output embedding $ \mathbf{z}_{L} \in \mathbb{R}^{ (N+1) \times D }$ of Transformer is utilized for subsequent text recognition.


\subsection{Adaptive Addressing and Aggregation Modules} 
\label{sec:a3modules}

Traditional Vision Transformers~\cite{dosovitskiy2020image,deit} usually append a learnable $\mathbf{x}_{class}$ token to the sequence of patch embeddings, which directly collects and aggregates the meaningful information and serves as the image representation for the classification of the whole image.
While the task of scene text recognition aims to produce a sequence of character predictions, where each character is only related to a small patch of the image.
Thus, the global image representation $ \mathbf{z}_{L}^0 \in \mathbb{R}^{ D } $  is inadequate for the text recognizing task.
ViTSTR~\cite{ViTSTR} directly employs the first $T$ tokens of  $ \mathbf{z}_{L} $ for text recognition, where  $T$ is the maximum text length. Unfortunately, the rest tokens of $ \mathbf{z}_{L} $ are not fully utilized.

% Figure environment removed

In order to take full advantage of the rich information of the sequence $ \mathbf{z}_{L} $ for text sequence prediction, 
we propose a tailored Adaptive Addressing and Aggregation (A$^3$)  module to select a  meaningful combination of tokens $ \mathbf{z}_{L} $ and integrate them into one token corresponding to a specific character.
Specifically, we manage to learn $T$ tokens $\mathbf{Y} = [\mathbf{y}_i]_{i=1}^{T}$ from the sequence $ \mathbf{z}_{L} $ for the subsequent text recognizing task.
An aggregation function is, thus, formulated as $\mathbf{y}_i = A_i(\mathbf{z}_{L})$, which converts the input  $ \mathbf{z}_{L} $ to a token vector $\mathbf{y}_i: \mathbb{R}^{ (N+1) \times D}  \mapsto  \mathbb{R}^{ 1 \times D } $. And  such $ T $ functions are constructed for the sequential output of text recognition.
Typically, the aggregation function $A_i(\mathbf{z}_{L})$ is implemented via a spatial attention mechanism~\cite{tokenlearner} to adaptively select the tokens from $ \mathbf{z}_{L} $ corresponding  to $i_{th}$ character.
Here, we employ function $\alpha_i(\mathbf{z}_{L})$ and softmax function to generate precise spatial attention 
mask $\mathbf{m}_i \in  \mathbb{R}^{ (N+1) \times 1}  $ from $\mathbf{z}_{L} \in \mathbb{R}^{ (N+1) \times D}$.
Thus, each output token $\mathbf{y}_i $ of A$^3$ module is produced by 
\begin{equation}
\begin{split}
&\mathbf{z}_{L} = LN(\mathbf{z}_{L}) \\
&\mathbf{y}_{i}= A_i(\mathbf{z}_{L}) = \mathbf{m}_i^T \tilde{\mathbf{z}}_{L}  = \text{softmax}(\alpha_i(\mathbf{z}_{L}))^T (\mathbf{z}_{L}\mathbf{U}) \\
&\mathbf{y}_{i} = LN(\mathbf{y}_{i}). \\
\end{split}
\end{equation}
Here, $\alpha_i$(·) is implemented by group convolution with one $1 \times 1$ kernel and $A_i$ represents the $i_{th}$ Addressing function to generate attention mask $\mathbf{m}_i \in  \mathbb{R}^{ (N+1) \times 1}$. And $ \mathbf{U}\in \mathbb{R}^{ D \times D}$  is a linear mapping matrix for learning feature $ \tilde{\mathbf{z}}_{L}$. Therefore, the resulting tokens of different aggregation functions are gathered together to form the final output tensor as follows:
\begin{equation}
\label{eq:Y}
\begin{split}
&\mathbf{Y}= [\mathbf{y}_{1}\mathbf{y}_{2};\ldots;\mathbf{y}_{T}] = [A_1(\mathbf{z}_{L}); A_2(\mathbf{z}_{L}); \ldots ;A_T(\mathbf{z}_{L}) ].\\
\end{split}
\end{equation}

Owing to the effective and efficient  A$^3$ module, the ultimate
representation of the text sequence is denoted as $\mathbf{Y} \in \mathbb{R}^{ T \times D}$ in Eq.~(\ref{eq:Y}). Then, a character classification head is built by $ \mathbf{G} = \mathbf{YW}^T \in \mathbb{R}^{ T \times K} $ for text recognition, where $\mathbf{W} \in \mathbb{R}^{ K \times D}$ is a linear mapping matrix, $K$ is the number of categories and $ \mathbf{G} $ is the classification logits. 
We regard this as the Character A$^3$ module for character-level prediction, of which the detailed structure is illustrated in Fig.~\ref{fig:token} (a). 

\subsection{Multi-Granularity Predictions}
\label{sec:MGP}
Character tokenization that simply splits text into characters is commonly used in scene text recognition methods.
However, this naive and standard way ignores the language information of the text. 
In order to effectively resort to linguistic information for scene text recognition, we incorporate subword~\cite{subword} tokenization mechanism in NLP~\cite{BERT} into the text recognition method. Subword tokenization algorithms aim to decompose rare words into meaningful subwords and remain frequently used words, so that the grammatical information of word has already been captured in the subwords. Meanwhile, since A$^3$ module is independent of Transformer encoder backbone, we can directly add extra parallel subword A$^3$ modules for subword predictions. 
In such a way, the language information can be implicitly injected into model learning for better performance. Notably, previous methods, \ie, SRN~\cite{SRN} and ABINet~\cite{ABInet}, design an explicit transformer module for language modeling, while we cast linguistic information encoding problem as a character and subword prediction task without an explicit language model. 

Specifically, we employ two subword tokenization algorithms Byte-Pair Encoding (BPE)~\cite{BPE} and WordPiece~\cite{wordpiece}
\footnote{Considering the potential out-of-vocabulary (OOV) issue in the inference phase, we did not directly predict whole words.} 
to produce various combinations, as shown in Fig.~\ref{fig:token}. 
%For instance, ``coffee'' is tokenized into ``co'' and ``ffee'' by BPE, while it remains the whole word ``coffee'' via WordPiece. 
Thus, the BPE A$^3$ module and WordPiece A$^3$ module are proposed for subword attention.
And two subword-level classification heads are used for subword predictions.
%Finally, these multi-granularity predictions (character and subword) are merged via a simple and effective fusion strategy.
Since subwords could be whole words (such as ``coffee'' in WordPiece), subword-level and even word-level predictions can be generated by the BPE and WordPiece classification heads. 
Along with the original character-level prediction, we denote these various outputs as multi-granularity predictions for text recognition. In this way, character-level prediction guarantees the fundamental recognition accuracy, and subword-level or word-level predictions can serve as complementary results for noised images via linguistic information. 

Technically, the architecture of BPE or WordPiece A$^3$ module is the same as Character one. They are independent of each other with different parameters.
And the numbers of categories are different for different classification heads, which depend on the vocabulary size of each tokenization method.
The cross-entropy loss is employed for classification.
Additionally, the mask $\mathbf{m}_{i}$ precisely indicates the attention location of the $i_{th}$ character in Character A$^3$ module, 
while it roughly shows the $i_{th}$ subword region of the image in subword A$^3$ modules, due to the higher complexity and uncertainty of learning subword splitting. 

% Figure environment removed

\subsection{Confidence-based Fusion Strategy}
\label{sec:fuse}
Multi-granularity predictions (Character, BPE and WordPiece) are generated by different A$^3$ modules and classification heads. Thus, a fusion strategy is required to merge these results. At the beginning, we attempt to fuse multi-granularity information by aggregating text features  $\mathbf{Y}$ of the output of different A$^3$ modules at the feature level. However, since these features are from different granularities, the $i_{th}$  token $\mathbf{y}_i^{char}$ of character level is not aligned with the $i_{th}$ token $\mathbf{y}_i^{bpe}$ (or $\mathbf{y}_i^{wp}$) of BPE level (or WordPiece level), so that these features cannot be directly added for fusion. Meanwhile, even if we concatenate features by [$\mathbf{Y}^{char}, \mathbf{Y}^{bpe}, \mathbf{Y}^{wp}$], only one character-level head can be used for final prediction. The subword information will be greatly impaired in this way, resulting in less improvement.

Instead, a decision-level fusion strategy is employed in our method. However, perfectly fusing these predictions is still a challenging problem~\cite{rrlrgu}.
We, therefore, propose a compromised but efficient Confidence-based Fusion Strategy (CFS) based on the prediction confidences of the three branches.
Specifically, the recognition confidence of each character or subword branch can be obtained by the corresponding classification head.
Then, we use two score functions $f(\cdot)$ to produce the final recognition score based on the atomic confidences:
\begin{equation}
\label{eq:mean}
f_{Mean}([c_1, c_2, \ldots, c_{eos}])= \frac{1}{n} \sum_{i=1}^{eos} c_i, \\
\end{equation}
\begin{equation}
\label{eq:prod}
f_{Cumprod}([c_1, c_2, \ldots, c_{eos}])= \prod_{i=1}^{eos}\ c_i.
\end{equation}

We only consider the confidence of valid characters or subwords and ending symbol $eos$, and ignore padding symbol $pad$.
``Mean'' recognition score is generated by the mean value function as in Eq.~(\ref{eq:mean}), while ``Cumprod'' represents the score produced by the cumulative product function.
Then, three recognition scores of three classification heads for one image can be obtained by $f(\cdot)$.
CFS simply picks the one with the highest fusion score as the final predicted result.


\subsection{Learnable Fusion Strategy} \label{sec:learnable}

%Confidence-based Fusion Strategy (CFS) is efficient and effective for decision-level fusion. However, since the raw confidence scores are produced from three different classification tasks at different levels, they are theoretically incomparable among each others, potentially limiting the effectiveness of CFS. Meanwhile, ``Mean'' function might reduce the impact of individual character errors, and ``Cumprod'' function may be skewed towards  shorter character sequences. One predefined fusion function can not indeed reflect the recognition quality of multi-granularity results. Therefore, we propose a Learnable Fusion Strategy (LFS) to directly measure the similarities between text and images, where the confidences of each classification heads are not considered.
Confidence-based Fusion Strategy (CFS) is both efficient and effective. However, since the raw confidence scores are produced from three different classification tasks at different levels, they are theoretically incomparable among each other, potentially limiting the effectiveness of CFS. Meanwhile, the ``Mean'' function might reduce the impact of individual character errors, and the ``Cumprod'' function may be skewed towards shorter character sequences. A predefined fusion function may not fully reflect the recognition quality of multi-granularity results. Therefore, we propose a Learnable Fusion Strategy (LFS) to directly measure the similarities between the predicted words and the original images, rather than relying on the raw confidence scores outputted by the three classification heads.

The main idea and module architecture of the proposed LFS are depicted in Fig.~\ref{fig:fusion}, where the image embedding $\mathbf{I} \in \mathbb{R}^{ 1 \times D} $ of the whole image is generated by the fusion image encoder
and the text embeddings $\mathbf{T}  \in \mathbb{R}^{ 1 \times D}$ are produced by the fusion text encoder.
Drawing inspiration from CLIP~\cite{CLIP}, we use contrastive learning to learn the alignment of text-image pairs.
The similarity scores between the image and three text sequences (words) produced by the three branches (Character, BPE and WordPiece) can be used for fusing the multi-granularity predictions. 
Noted that the ViT backbone and the three A$^3$ modules are frozen for generating the original multi-granularity predictions, while the new LFS module will be solely trained to fuse these results.
The details of the LFS module will be elaborated in the following subsections.

\subsubsection{Fusion Image Encoder}
The image embeddings are mainly derived from the ViT backbone as mentioned in Sec.~\ref{Sec:ViT}.
Specifically, we attach a trainable ``Adapter'' consisting of $H$ Transformer encoder blocks and a new Image A$^3$ module shown in Fig.~\ref{fig:fusion} to the $(L-1)_{th}$ layer of ViT backbone as in Fig.~\ref{fig:overview} for the whole image representation, denoted as $\mathbf{I}= A_I(\mathbf{\hat{z}})$,
where $\mathbf{\hat{z}}$ is produced by the output embeddings $\mathbf{{z}}_{L-1}$ of the $(L-1)_{th}$ frozen layers of ViT backbone within extra trainable Transformer encoder blocks.
% \begin{equation}
% \begin{split}
% &\mathbf{\hat{z}}_{L}^{\prime}=\text{MSA} (LN(\mathbf{z}_{L-1}))+ \mathbf{z}_{L-1}   \\
% &\mathbf{\hat{z}}_{L}=\text{MLP} (LN(\mathbf{\hat{z}}_{L}^{\prime}))+ \mathbf{\hat{z}}_{L}^{\prime} \\
% &\mathbf{I}= A_I(\mathbf{\hat{z}}_{L}). \\
% \end{split}
% \end{equation}
Here, we do not directly utilize the outputs of the last layer $\mathbf{z}_{L}$ for image embedding extraction,
since $\mathbf{z}_{L}$ is customized for multi-granularity text recognition.
We deem that the outputs of the penultimate block $\mathbf{z}_{L-1}$ are more general features for computing the whole image representation.
Different from the Character A$^3$ module that produces $T$ tokens $\mathbf{Y} \in \mathbb{R}^{ T \times D}$ in Sec.\ref{sec:a3modules}, the Image A$^3$ module $A_I$ generates only one token $\mathbf{I} \in \mathbb{R}^{ 1 \times D}$ to represent the whole image.
Finally, the image feature is layer normalized and linearly projected into the multi-modal embedding space for contrastive learning.

\subsubsection{Fusion Text Encoder}

We adopt a text encoder with $J$ Transformer blocks to transcribe words into text embeddings based on character level tokenization.
All the characters in the words are modeled uniformly, meaning that various words including frequently used words, combinations of characters and numbers, or even random strings are encoded consistently and treated equally, avoiding potential 
imbalance issues.
% words with a little charatcter errors can be modeled uniformly.
% but the subword tokenization outputs of them might be unpredictable and unstable.
Specifically, the text encoder is a Transformer with the architecture of the one in CLIP~\cite{gpt2,CLIP}.
The max input sequence length is $T$ and causal attention mask is used.
The character sequence of text is bracketed with $bos$ and $eos$ tokens and the activations of the last layer of the transformer at the $eos$
token are treated as the text feature representation $\mathbf{T} \in \mathbb{R}^{ 1 \times D}$.
Similar to the image features, layer normalization and linear projection are utilized.

\subsubsection{Training}
Given a batch of $M$ (\textit{text}, \textit{image}) pairs, CLIP aims to learn a multi-modal metric space by maximizing the similarities of the image and text embeddings of the $M$ true pairs in the batch, while minimizing the similarities of the
embeddings of the rest $M^2 - M$ incorrect pairings.
However, the target of LFS is to distinguish the subtle differences between characters within similar words referring to the image as shown in Fig.~\ref{fig:fusion} (1).
Thus, given a (\textit{text}, \textit{image}) pair, in the context of LFS, the negative text is constructed with \textit{\textbf{intentionally noised text variations}}, rather than the text from other pairs.
Concretely, we define $4$ single-character perturbations to delete, replace, repeat or insert at every character position in a word.
And we also design multi-character perturbations where we first randomly select $2$ or $3$ character positions, and then perform one random single-character perturbation at each selected position.
Notably, the noised text variations must be different from the ground-truth text.
Consequently, $N_s$ noised text is produced for one reference image.
Next, the contrastive representation learning with InfoNCE loss~\cite{infoNCE} is adapted for our image and text encoder learning,
which is computed as follows:
\begin{equation}
\begin{split}
& p_{\theta}( \mathbf{I}_i, \mathbf{T}_{i} )=\dfrac{\exp( S(\mathbf{I}_{i}, \mathbf{T}_{i}^{*}))}{\exp(S(\mathbf{I}_{i}, \mathbf{T}_{i}^{*})) + \sum_{k\in{{N}_{s}}}\exp(S(\mathbf{I}_{i},\mathbf{T}_{i}^{k}))}  \\
& L_{fuse} (\theta) = -\dfrac{1}{N_I} \sum_{i=1} \log \; p_{\theta}( \mathbf{I}_{i}, \mathbf{T}_{i}). \\
\end{split}
\end{equation}

Here, $S(\mathbf{I},\mathbf{T})$ represents the cosine similarity between image and text embeddings.
$\mathbf{T}_{i}^{*}$ is the embedding of the ground-truth text and $\mathbf{T}_{i}^{k}$ is the one of the noised text for the $i_{th}$ image.
${N_I}$ is the number of images.
$\theta$ represents the trainable parameters of LFS as illustrated in Fig.~\ref{fig:fusion} (1).

\subsubsection{Inference}

The inference process of LFS is shown in Fig.~\ref{fig:fusion}(2).
Given an input image $\mathbf{x}$, multi-granularity predictions are firstly produced by the A$^3$ modules as mentioned in Sec.~\ref{sec:MGP}. 
Then, the text embeddings of these $3$ predictions are generated by the fusion text encoder and the image embedding is produced by the fusion image encoder. 
Lastly, the cosine similarities between the image embedding and the $3$ text embeddings are computed, and the text with the highest similarity score will be chosen as the final result.

\begin{table*}[t]\centering
\setlength{\tabcolsep}{5pt}
\ra{1.2}
\caption{The ablation study of the proposed vision STR model and the accuracy comparisons with previous SOTA STR methods based on only vision information.}
\label{tab:char}
\begin{tabular}{|l|c|c|c|c|c|c|c|c|c|c|}
\hline
\textbf{Methods} & \textbf{Backbone} & \textbf{Image Size (Patch)} &\textbf{IC13} &\textbf{SVT}  &\textbf{IIIT}   & \textbf{IC15} & \textbf{SVTP} &\textbf{CUTE}  &\textbf{AVG} \\
\hline
MASTER~\cite{MASTER} &   \multirow{3}{*}{CNN}  & - &95.3 &90.60 &95.0 &79.4 &84.5 &87.5 &89.5	 \\
SRN$_V$~\cite{SRN}  &  & - &93.2 &88.1 &92.3 &77.5 &79.4 &84.7 &86.9	 \\
ABINet$_V$~\cite{ABInet}   &  & - &94.9 &90.4 &94.6 &81.7 &84.2 &86.5 &89.8 \\
\hline
MGP-STR$_{P=16}$ &  \multirow{3}{*}{ViT}   & $ 224 \times 224 (16 \times 16) $ &95.68	&91.96	&95.13	&83.88	&85.74	&90.28	&91.07	 \\
MGP-STR$_{P=4}$ &  & $ 32 \times 128 (4 \times 4) $ &96.62	&92.27	&95.40	&84.76	&86.98	&88.54	&91.58	 \\
% MGP-STR$_{Vision}$  &  & $  32 \times 128 (4 \times 4) $ &96.62 &93.05 &96.43 &85.81 &89.61 &90.28 &92.65 \\
MGP-STR$_{Vision}$  &  & $  32 \times 128 (4 \times 4)$ &96.50 &93.20 &{96.37} &86.25 &89.46 &{90.63} &92.73 \\
\hline
\end{tabular}
\end{table*}


%-------------------------------------------------------------------------
\section{Experiment} \label{Sec:Experiment}

In this section, we present qualitative examples, quantitative comparisons, ablation studies and 
in-depth analyses, to prove the effectiveness and advantages of the proposed MGP-STR method.

\subsection{Datasets}
\label{Sec:data}
%The comparison with different methods must be based on a unified framework, which includes the use of the same training dataset and test dataset. 
For fair comparisons, we use MJSynth~\cite{MJ1,MJ2} and SynthText~\cite{ST} as the training data. MJSynth contains $9M$ synthetic text images and SynthText includes $7M$ synthetic text images, respectively. The test data consists of ``regular'' and ``irregular'' datasets. 
The ``regular'' dataset is mainly composed of horizontally aligned text images. 
IIIT 5K-word (IIIT)~\cite{IIIT} consists of 3,000 images collected on the website. 
Street View Text (SVT)~\cite{SVT} contains 647 test images. 
ICDAR 2013 (IC13)~\cite{IC13} includes 1,095 images cropped from mall pictures, and has two versions for evaluation (\ie, IC13(857) and IC13(1015)).
% but we eventually evaluate on 857 images, discarding images that contain non-alphanumeric characters or less than three characters. 
The text instances in the ``irregular'' dataset are mostly curved or distorted. 
ICDAR 2015 (IC15)~\cite{IC15} includes 2,077 images collected from Google Eyes, and also has two versions for evaluation (\ie, IC15(1811) and IC15(2077)).
% but we use 1,811 images without some extremely distorted images. 
Street View Text-Perspective (SVTP)~\cite{SVTP} has 639 images collected from Google Street View. 
CUTE80 (CUTE)~\cite{CUTE} consists of 288 images with curved text.
Typically, IC13 and IC15 represent IC13(857) and IC15(1811) in the following experiments, unless otherwise stated.

We also study the performance of  MGP-STR trained on real data. Following~\cite{PARSeq}, we use COCO-Text (COCO)~\cite{COCO}, RCTW17~\cite{RCTW}, Uber-Text (Uber)~\cite{Uber}, ArT~\cite{ArT}, LSVT~\cite{LSVT}, MLT19~\cite{MLT}, ReCTS~\cite{ReCTS}, TextOCR~\cite{TextOCR} and OpenVINO~\cite{OpenVINO} as real training datasets. Please refer to \cite{realdata,PARSeq} for the comprehensive description of the real data.
Meanwhile, we also evaluate MGP-STR on the test sets of $3$ challenging datasets used in~\cite{PARSeq} for a more comprehensive comparison.
Specifically, COCO-Text (COCO)~\cite{COCO} includes 9,825 samples with low-resolution and occluded text.
ArT~\cite{ArT} contains 35,149 images with curved and rotated text.
Uber-Text (Uber)~\cite{Uber} has 80,418 images with vertical and rotated text instances.

To validate the universality of the proposed MGP-STR algorithm, we also conduct experiments on handwritten text datasets with English (IAM~\cite{IAM} and CVL~\cite{CVL}) and French (RIMES~\cite{RIMES}). In accordance with the methodologies employed in DiG~\cite{DiG}, we gather a training dataset of 146,805 images sourced from the training sets of CVL and IAM, and two test sets of 12,012 images and 13,752 images are assigned to CVL and IAM, respectively. For the RIMES dataset, we gather 51,737 images for training and 7,776 images for testing.


% \subsection{Implementation Details}
%   \textbf{Model Configuration} Our backbone network was introduced in 3.1. Like ViTSTR~\cite{ViTSTR}, the embedding dim in MGP is also set to 768, but the patch size is set to 4. The Vision Transformer consists of 12 stacked units, where the number of heads is 12 and the number of hidden units is 768. The size of the character-level dictionary is 38, the WordPiece-level is 30522, and the BPE-level is 50257. The maximum length of the output sequence is set to 27. The height and width of the input image are 32 x 128, and we applied the image enhancement methods provided in ViTSTR~\cite{ViTSTR} designed for STR, including various image transformations such as invert, bend, blur, noise, warp, rotate, stretch /compression, perspective, shrinkage, etc. But no matter how you change the image, it doesn't change the meaning of the text in it.

% \noindent\textbf{Model Training} We use basically the same training configuration as ViTSTR~\cite{ViTSTR}, except that the input size is resized to 32x128 and the patch size is 4. DeiT's~\cite{deit} pretrained weights are automatically downloaded before training, and we used 2 NVIDIA Tesla V100 GPUs to train our model with a batch size of 128. The Adadelta~\cite{Adadelta} optimizer was used with an initial learning rate of 1, the learning rate decay strategy is Cosine Annealing LR~\cite{cosinlr}, and the training epoch is 10 epochs.


\subsection{Implementation Details}

\subsubsection{Model Configuration} 

In the default setting, MGP-STR is built upon the ViT model~\cite{dosovitskiy2020image}, which is composed of $12$ stacked Transformer blocks.
For each layer, the number of heads is $12$ and the embedding dimension $D$ is $768$. 
Notably, square $ 224 \times 224$ images~\cite{dosovitskiy2020image,deit,ViTSTR} are not adopted in our method.
The standard height $H$ and width $W$ of the input images are set to $32$ and $128$.
The patch size $P$ is set to 4 and thus $N=8 \times 32 =256$ plus one $[class]$ tokens $\mathbf{z}_L \in \mathbb{R}^{257\times768}$ will be produced.
The maximum length $T$ of the output sequence $\mathbf{Y}$ of A$^3$ module is set to $27$. 
The vocabulary size $K$ of the Character classification head is set to 38, including $0-9$, $a-z$, $pad$ for the padding symbol and $eos$ for the ending symbol.
The vocabulary sizes of BPE and WordPiece heads are $50,257$ and $30,522$, respectively.
In particular, since RIMES includes special characters of French, the vocabulary size $K$ is set to $50$, and we use a special WordPiece tokenization that supports French, which has $28,996$ subwords.
The evaluation on the handwritten text dataset is case-insensitive and excludes punctuations.

For the LFS module, an extra $H=1$ Transformer block is attached to the $11_{th}$ layer of ViT backbone as mentioned above for the image representation in fusion image encoder.
$J=2$ Transformer blocks are used for text feature extraction in the Fusion Text Encoder.
The maximum input text length $T$ is set to $27$.
$L2$ normalization is performed before the computation of cosine similarity.
$N_s =256$ noised text variations are generated by repeated $4$ single-character perturbations within $T / L_w * 5$ times and about $150$ multi-character perturbations, where $L_w$ is the length of the ground-truth text.

%\vspace{-4mm}

\subsubsection{Model Training} 

In the default setting, the pre-trained weights of the DeiT-base~\cite{deit} model are loaded as the initial weights for the ViT backbone of MGP-STR, except for the patch embedding module, due to inconsistent patch sizes.
Common data augmentation methods~\cite{randaug} for text images, such as perspective distortion, affine distortion, blur, noise and rotation, are applied.
We use $4$ NVIDIA A100 GPUs to train our model with a batch size of $100$. Adadelta~\cite{Adadelta} optimizer is employed with an initial learning rate of $1$.
The learning rate decay strategy is Cosine Annealing LR~\cite{cosinlr} and the training lasts about $100$ million iterations.

The trainable weights of LFS are randomly initialized.
LFS is trained with a batch size of $100$ on $2$ NVIDIA A100 GPUs. 
Adadelta~\cite{Adadelta} optimizer is used with an initial learning rate of $0.5$.
Besides the common data augmentation mentioned above, random earasing~\cite{erasing} data augmentation is introduced and $80\%$ images will be augmented.
Cosine Annealing LR is also used for learning rate decay and the training lasts about $200,000$ iterations.

\begin{table}[t]\centering
\setlength{\tabcolsep}{4pt}
\ra{1.2}
\caption{The accuracies of MGP-STR$_{CFS}$ with different confidence-based fusion strategies}
\label{tab:fuse}
\begin{tabular}{|l|c|c|c|c|c|c|c|c|}
\hline
\textbf{Mode} &\textbf{IC13} &\textbf{SVT}  &\textbf{IIIT}   & \textbf{IC15} & \textbf{SVTP} &\textbf{CUTE}  &\textbf{AVG} \\
\hline
Char &96.49	&93.66	&96.1	&86.14	&88.83	&89.58	&92.53	 \\
\hline
Mean &97.31	&94.28	&96.60	&86.97	&90.23	&90.97	&93.28	 \\
Cumprod &97.32	&94.74	&96.40	&87.24	&91.01	&90.28	&93.35	 \\
\hline
\end{tabular}
\vspace{-3mm}
\end{table}

\begin{table*}[t]\centering
\setlength{\tabcolsep}{5pt}
\ra{1.2}
\caption{The results of the four variants of MGP-STR model. ``Char'', ``BPE'' and ``WP'' at the ``Output'' column represent predictions of Character, BPE and WordPiece classification head  in each model, respectively. ``CFS'' represents the fused results via confidence-based fusion strategies.}
\label{tab:CBW}
\begin{tabular}{|l|c|c|c|c|c|c|c|c|c|c|c|c|}
\hline
\textbf{Methods} & \textbf{Char} & \textbf{BPE} & \textbf{WP} &\textbf{Output} &\textbf{IC13} &\textbf{SVT}  &\textbf{IIIT}   & \textbf{IC15} & \textbf{SVTP} &\textbf{CUTE}  &\textbf{AVG} \\
\hline
% MGP-STR$_{Vision}$ &\checkmark & $ \times $ & $ \times $ &  Char &96.50	&93.51	&96.13	&86.14	&89.30	&91.32	&92.65	 \\
MGP-STR$_{Vision}$ &\checkmark & $ \times $ & $ \times $ &  Char &96.50 &93.20 &{96.37} &86.25 &89.46 &{90.63} &92.73 \\
\hline
\multirow{3}{*}{MGP-STR$_{C+B}$} &  \multirow{3}{*}{\checkmark} & \multirow{3}{*}{ \checkmark} &  \multirow{3}{*}{$ \times $} &  Char &97.43	&93.82	&96.53	&85.92	&89.15	&90.28	&92.84	 \\
% \cline{6-12}
& & & &  BPE &97.78	&94.13	&90.00	&81.12	&88.37	&82.64	&88.63	 \\
%\cline{6-12}
& & & &  CFS &97.67 &94.47	&96.73	&86.97	&88.99	&89.93	&93.24	 \\
\hline
\multirow{3}{*}{MGP-STR$_{C+W}$} &  \multirow{3}{*}{\checkmark} &  \multirow{3}{*}{$ \times $} & \multirow{3}{*}{ \checkmark}  &  Char &96.97	&93.97	&96.30	&86.20	&90.39	&89.93	&92.87	 \\
%\cline{6-12}
& & & &  WP &95.92	&93.35	&87.70	&78.74	&89.30	&80.21	&86.78	 \\
%\cline{6-12}
& & & &  CFS &97.32	&93.82	&96.60	&86.91	&90.54	&90.97	&93.25	 \\
\hline
\multirow{4}{*}{MGP-STR$_{CFS}$} &  \multirow{4}{*}{\checkmark} &  \multirow{4}{*}{\checkmark} & \multirow{4}{*}{ \checkmark}  &  Char &96.49	&93.66	&96.10	&86.14	&88.83	&89.58	&92.53	 \\
%\cline{6-12}
& & & &  BPE &95.56	&93.66	&88.73	&79.84	&89.76	&83.33	&87.63	 \\
%\cline{6-12}
& & & &  WP &95.79	&94.59	&86.37	&77.36	&89.61	&79.86	&85.99	 \\
%\cline{6-12}
& & & &  CFS &97.32	&94.74	&96.40	&87.24	&91.01	&90.28	&93.35	 \\
\hline
\end{tabular}
\end{table*}

\begin{table}[t]\centering
\setlength{\tabcolsep}{2pt}
\ra{1.5}
\caption{The results of MGP-STR$_{Vision}$ equipped with BCN and MGP-STR$_{CFS}$. ``V'' represents the results of the pure vision output. ``V+L'' represents the results based on both vision and language parts. $*$ represents the upper bound of MGP-STR$_{CFS}$. }
\label{tab:BCN}
\begin{tabular}{|l|c|c|c|c|c|c|c|c|c|}
\hline
\textbf{Methods} & \textbf{Mode} &\textbf{IC13} &\textbf{SVT}  &\textbf{IIIT}   & \textbf{IC15} & \textbf{SVTP} &\textbf{CUTE}  &\textbf{AVG} \\
\hline
MGP-STR$_{Vision}$  &  V &96.97	&93.82	&95.90	&85.53	&89.15	&89.58	&92.40	 \\
+BCN & V+L &97.32	&95.36	&95.97	&86.69	&91.78	&89.93	&93.14	 \\
\hline
MGP-STR$_{CFS}$   & V+L &97.32	&94.74	&96.40	&87.24	&91.01	&90.28	&93.35	 \\
MGP-STR$_{CFS}^*$   & V+L &97.66	&96.29	&96.97	&89.06	&92.09	&92.01	&94.38	 \\
\hline
\end{tabular}
\vspace{-3mm}
\end{table}

\subsection{Discussions on ViT and A$^3$ Modules} 
We analyze the influence of the patch size of Vision Transformer and the effectiveness of A$^3$ module in the proposed MGP-STR method (shown in Tab.~\ref{tab:char}). MGP-STR$_{P=16}$ represents the model that simply uses the first $T$ tokens of $\mathbf{z}_L$ for text recognition as in ViTSTR~\cite{ViTSTR}, where the input image is reshaped to $224 \times 224$ and the patch size is set to $16 \times 16 $. In order to retain the significant information of the original text image, $32 \times 128 $ images with $4 \times 4 $ patches are employed in MGP-STR$_{P=4}$. MGP-STR$_{P=4}$ outperforms MGP-STR$_{P=16}$, which indicates that the standard image size of ViT~\cite{dosovitskiy2020image,deit} is incompatible with the task of text recognition. Thus, $32 \times 128 $ images with $4 \times 4 $ patches are used in MGP-STR.

When the Character A$^3$ module is introduced into MGP-STR, denoted as MGP-STR$_{Vision}$, the recognition performance will be further improved. MGP-STR$_{P=16}$ and MGP-STR$_{P=4}$ cannot fully learn and utilize all the tokens, while the Character A$^3$ module can adaptively aggregate features of the last layer, resulting in more sufficient learning and higher accuracy. Meanwhile, compared with previous SOTA text recognition methods with CNN feature extractors, the proposed MGP-STR$_{Vision}$ method achieves substantial performance improvement. %These results demonstrate that MGP-STR$_{Vision}$ is a simple yet powerful baseline for text recognition.

\subsection{Discussions on Multi-Granularity Predictions }

\subsubsection{Effect of Confidence-based Fusion Strategy }

Since the subwords generated by subword tokenization methods carry statistical and even grammatical information, we directly employ subwords as the targets of our model to capture the linguistic prior implicitly. As described in Sec.~\ref{sec:a3modules}, two different subword tokenizations BPE and WordPiece are employed for complementary multi-granularity predictions. 
Besides the character prediction, we propose two confidence-based fusion strategies to further merge the results from the three branches (corresponding to Character, BPE and WordPiece), denoted as ``Mean’’ and ``Cumprod’’ as mentioned in Sec.~\ref{sec:fuse}. We denote this method that merges the three results as MGP-STR$_{CFS}$, and the accuracies of MGP-STR$_{CFS}$ with different CFS fusion strategies are listed in Tab.~\ref{tab:fuse}. Additionally, the first line ``Char'' in Tab.~\ref{tab:fuse} records the result of character classification head in MGP-STR$_{CFS}$. As can be seen, both ``Mean’’ and ``Cumprod’’ fusion strategies can significantly improve the final recognition accuracy over that of single character-level results. Due to the better performance of ``Cumprod’’ strategy, we employ it as the default choice of Confidence-based Fusion Strategy (CFS) in the following experiments.

\subsubsection{Effect of Subword Representations}
%To further explore the effect of different subword tokenization, 
We evaluate four variants of the MGP-STR model to verify the effect of subword representations.
The performances of these four variants are elaborately reported in Tab.~\ref{tab:CBW}, including the fused results and the results of each single classification head. Specifically, MGP-STR$_{Vision}$ with only Character A$^3$ module has already obtained promising results. MGP-STR$_{C+B}$ and MGP-STR$_{C+W}$ incorporate Character A$^3$ module with BPE A$^3$ module and WordPiece A$^3$ module, respectively. No matter which subword tokenization is used alone, the accuracy of ``CFS’’ can exceed that of ``Char’’ in both MGP-STR$_{C+B}$ and MGP-STR$_{C+W}$ methods, respectively. Notably, the performance of the classification head of ``BPE’’ or ``WP’’ could be better than that of ``Char’’ on the SVP and SVTP datasets in the same model. These results show that subword predictions can 
boost text recognition performance by implicitly introducing language information. Thus, MGP-STR$_{CFS}$ with three A$^3$ modules can produce complementary multi-granularity predictions. By fusing these multi-granularity results, MGP-STR$_{CFS}$ obtains better performance than single ones. 


\subsubsection{Comparison with Bidirectional Cloze Network}
\label{sec:bcn}
Bidirectional Cloze Network (BCN) is designed in ABINet~\cite{ABInet} for explicit language modeling, and it leads to favorable improvement over the pure vision model. We equip MGP-STR$_{Vision}$ with BCN as a competitor of MGP-STR$_{CFS}$ to verify the advantage of the Multi-Granularity Prediction strategy. Concretely, we first reduce the dimension $768$ of feature $\mathbf{Y}$ to $512$ for aligning with the output of BCN. Following the training settings in~\cite{ABInet}, the results of this hybrid model are reported in Tab.~\ref{tab:BCN}. Apparently, the accuracy of ``V+L'' is further improved over the pure vision prediction ``V'' in MGP-STR$_{Vision}$+BCN, and better than the original ABINet~\cite{ABInet}. However, the performance of MGP-STR$_{Vision}$+BCN is a little worse than that of MGP-STR$_{CFS}$.

In addition, we provide the upper bound of the performance of MGP-STR$_{CFS}$, denoted as MGP-STR$_{CFS}^*$ in Tab. 4. If one of the three predictions (``Char'', ``BPE'' and ``WP'') is right, the final prediction is considered to be correct. 
The highest score of MGP-STR$_{CFS}^*$ demonstrates the good potential of multi-granularity predictions.
Moreover, MGP-STR$_{CFS}$ only requires two new subword prediction heads, rather than the design of a specific and explicit language model as in~\cite{ABInet,SRN}. 
These experiments demonstrate the superiority of multi-granularity predictions over BCN.

\subsection{Discussions on ViT Backbones}
\label{sec:ViTBackbones}

\subsubsection{Results with Different Sizes of ViT} 
All of the proposed MGP-STR models mentioned earlier are based on DeiT-Base~\cite{deit}. We also introduce two smaller models, namely DeiT-Small and DeiT-Tiny as presented in~\cite{deit} to further evaluate the effectiveness of MGP-STR$_{CFS}$. Specifically, the embedding dimensions of DeiT-Small and DeiT-Tiny are reduced to $384$ and $192$, respectively. Tab.~\ref{tab:y0} records the results of each prediction head of MGP-STR$_{CFS}$ with different ViT backbones. In general, fusing multi-granularity predictions can still improve the performance of pure character-level prediction with different backbones, and bigger models can achieve higher performances with the same classification heads. More importantly, the results of ``Char’’ with DeiT-Small and even DeiT-Tiny have already surpassed the SOTA pure CNN-based vision models, referring to Tab.~\ref{tab:char}. Therefore, MGP-STR$_{Vision}$ with small or tiny ViT backbones is also a competitive vision model and multi-granularity predictions can also work well with ViT backbones of different sizes, showing the good adaptability of the proposed MGP-STR method.

\begin{table}[t]\centering
\setlength{\tabcolsep}{2.5pt}
\ra{1.2}
\caption{The results of MGP-STR$_{CFS}$ with different sizes of ViT backbones.}
\label{tab:y0}
\begin{tabular}{|l|c|c|c|c|c|c|c|c|c|}
\hline
\textbf{Backbone} & \textbf{Output} &\textbf{IC13} &\textbf{SVT}  &\textbf{IIIT}   & \textbf{IC15} & \textbf{SVTP} &\textbf{CUTE}  &\textbf{AVG} \\
\hline
\multirow{4}{*}{DeiT-Tiny}  &  Char &93.47	&90.57	&93.93	&82.94	&81.71	&84.38	&89.36	 \\
&  BPE &87.40	&84.39	&83.17	&73.72	&77.83	&71.53	&80.48	 \\
&  WP &53.79	&45.44	&60.07	&52.57	&42.79	&42.71	&53.92	 \\
&  CFS &94.05	&91.19	&94.30	&83.38	&83.57	&84.38	&89.91	 \\
\hline
\multirow{4}{*}{DeiT-Small}  &  Char &95.92	&91.04	&94.97	&84.59	&85.89	&86.81	&91.01	 \\
&  BPE &96.27	&93.35	&89.37	&79.74	&86.67	&82.29	&87.61	 \\
&  WP &75.50	&70.48	&74.70	&66.81	&68.06	&62.15	&71.36	 \\
&  CFS &96.38	&93.51	&95.30	&86.09	&87.29	&87.85	&91.96	 \\
\hline
\multirow{4}{*}{DeiT-Base}  &  Char &96.49	&93.66	&96.10	&86.14	&88.83	&89.58	&92.53	 \\
&  BPE &95.56	&93.66	&88.73	&79.84	&89.76	&83.33	&87.63	 \\
&  WP  &95.79	&94.59	&86.37	&77.36	&89.61	&79.86	&85.99	 \\
&  CFS &97.32	&94.74	&96.40	&87.24	&91.01	&90.28	&93.35	 \\
\hline
\end{tabular}
\end{table}

\subsubsection{Results with Different Initial Weights} 
Besides DeiT-Base, we also initialize MGP-STR with various recent pre-trained ViT backbone models (\ie, DINO~\cite{DINO}, DINOv2~\cite{DINOv2}, MAE~\cite{mae} CLIP~\cite{CLIP} and BLIP~\cite{BLIP}) to verify the effectiveness of our method. We train MGP-STR with more iterations ($200$ million) to fully exploit the ability of the models, and the results with different initial weights are listed in Tab.~\ref{tab:clip}.
With more training iterations, the result of MGP-STR$_{CFS}$ (Deit-Base) will be improved.
Additionally, the pre-training weights of both the pure image pre-trained ViTs (\ie, DINO, DINOv2 and MAE) and the image encoder ViTs of vision-language models (\ie{CLIP and BLIP}) are compatible with MGP-STR.
%Owing to the sufficient training iterations, the difference of the results with different initial weights are slight, demonstrating the good robustness of the initial weight.
As can be observed from Tab.~\ref{tab:clip}, once sufficiently trained, the ViT backbones from different pre-trained models almost work equally well with MGP-STR.
In the subsequent experiments, the DeiT-Base model with $200$ million iterations is used, due to its good performance.

\subsection{Discussions on Learnable Fusion Strategy}

\subsubsection{Effect of Learnable Fusion Strategy}
We train the LFS module based on MGP-STR$^{\dagger}_{CFS}$,
and the results of the model with LFS are detailedly depicted in Tab.~\ref{tab:learnable}.
Referring to Tab.~\ref{tab:y0}, the results of ``BPE'' and ``WP'' in MGP-STR$^{\dagger}$ are improved by a large margin than MGP-STR. We argue that the difficulty of subword recognition is higher than that of character recognition.
Due to the sufficient training time, the performances of ``BPE'' and ``WP'' heads can be improved further, resulting in a better-fused accuracy in MGP-STR$^{\dagger}_{CFS}$.
Moreover, it is worth noting that the upper bound (as explained in Sec.~\ref{sec:bcn}) of MGP-STR$^{\dagger}$ is $94.67 \%$.
Compared with MGP-STR$^{\dagger}_{CFS}$ ($93.70 \%$), MGP-STR$^{\dagger}_{LFS}$ ($94.05 \%$) unleashes the potential of the multi-granularity predictions further and achieves substantial improvement ($0.3\%$).
These results have proved the advantage of LFS over CFS.

\begin{table}[t]\centering
\setlength{\tabcolsep}{2.5pt}
\ra{1.2}
\caption{The results of MGP-STR$_{CFS}$ with different initial weights. The models with $\dagger$ are trained with $200$ million iterations. ``V'' represents the image pre-training models. ``V+L'' represents the vision-language pre-training models.}
\label{tab:clip}
\begin{tabular}{|l|c|c|c|c|c|c|c|c|c|}
\hline
\textbf{Backbone} & \textbf{Mode} &\textbf{IC13} &\textbf{SVT}  &\textbf{IIIT}   & \textbf{IC15} & \textbf{SVTP} &\textbf{CUTE}  &\textbf{AVG} \\
\hline
DeiT & V  &97.32  &94.74 &96.40 &87.24 &91.01	&90.28 &93.35 \\
DeiT$^\dagger$ &  V &97.43	&95.52	&96.77	&87.80	&90.23	&91.32	&93.70	 \\
\hline
MAE$^\dagger$ & V &97.32 &95.36 & 96.43 &87.96 &90.54 & 91.32 & 93.60	 \\
DINO$^\dagger$ & V &97.78 &95.21	&96.27	&87.58	&90.70	&90.97 & 93.47	 \\
DINOv2$^\dagger$ & V &97.55 &95.05 &96.53 &87.63 &90.54 &92.36 & 93.59	 \\
\hline
CLIP$^\dagger$  & V+L &97.67 &94.59	&96.20	&87.91	&91.63	&92.36 & 93.60	 \\
BLIP$^\dagger$ & V+L &97.67 &95.05 &96.03 &87.85 &91.78 &91.32 	&93.53 \\
\hline
\end{tabular}
\end{table}

\begin{table}[t]\centering
\setlength{\tabcolsep}{3pt}
\ra{1.3}
\caption{The results of MGP-STR$^\dagger$ with ``CFS'', and ``LFS''. $\dagger$ represents $200$ million training iterations, $*$ represents the upper bound of MGP-STR$^{\dagger}$. }
\label{tab:learnable}
\begin{tabular}{|l|c|c|c|c|c|c|c|c|c|}
\hline
\textbf{Methods} &\textbf{IC13} &\textbf{SVT}  &\textbf{IIIT}   & \textbf{IC15} & \textbf{SVTP} &\textbf{CUTE}  &\textbf{AVG} \\
\hline
% \multirow{5}{*}{MGP-STR~\cite{mgp}}  &  Char &96.49	&93.66	&96.10	&86.14	&88.83	&89.58	&92.53	 \\
% %\cline{6-12}
% &   BPE &95.56	&93.66	&88.73	&79.84	&89.76	&83.33	&87.63	 \\
% %\cline{6-12}
% &   WP &95.79	&94.59	&86.37	&77.36	&89.61	&79.86	&85.99	 \\
% \cline{2-9}
% &  CFS &97.32	&94.74	&96.40	&87.24	&91.01	&90.28	&93.35	 \\
% &  LFS &97.32	&94.74	&96.40	&87.24	&91.01	&90.28	&93.35	 \\
% \hline
MGP-STR$^\dagger_{Char}$  &96.97	&93.97	&96.53	&86.53	&88.22	&89.93	&92.85	 \\
MGP-STR$^\dagger_{BPE}$  &96.97	&94.90	&90.27	&81.67	&89.77	&85.07	&89.07	 \\
MGP-STR$^\dagger_{WP}$  &97.20	&95.36	&89.23	&79.90	&91.63	&82.64	&88.34	 \\
\cline{1-9}
MGP-STR$^\dagger_{CFS}$  &97.43	&95.52	&96.77	&87.80	&90.23	&91.32	&93.70	 \\
MGP-STR$^\dagger_{LFS}$  &97.78	&96.29	&96.77	&88.13	&92.09	&91.32	&94.05	 \\
\cline{1-9}
MGP-STR$^\dagger_{CFS^*}$  &98.13 &96.45 &97.23  &89.29 &93.02  &91.67  &94.67   \\
\hline
\end{tabular}
\end{table}

\subsubsection{Effect of the Number of Layers} 
We construct $4$ variants of LFS with different numbers of layers, and the results are recorded in Tab.~\ref{tab:depth}.
Both fusion image encoder and fusion text encoder are based on Transformer layers, and the number of layers in Tab.~\ref{tab:depth} means the trainable layers as described in Sec.~\ref{sec:learnable}.
We can see that all the models, even model (a) with a single layer, can achieve similar performances.
Meanwhile, the latency of LFS is negligible (about $3$ ms).
We claim that the image information has already been encoded in the ViT backbone and $1$ more extra layer is enough for computing the whole image representation. 
And the word information with characters can be captured with $2$ transformer layers.
In summary, the proposed LFS module is an effective alternative for better fusion results and the costs of training and inference brought by LFS are quite limited.
Therefore, we chose model (c) as the default setting of the proposed LFS, due to the good result and high efficiency.

\begin{table}[t]\centering
\setlength{\tabcolsep}{3pt}
\ra{1.2}
\caption{The results of the four variants of the LFS module. ``Text'' and ``Image'' represent the number of layers of fusion text encoder and fusion image encoder, respectively. Additionally, the results include the inference time and model size of LFS.}
\label{tab:depth}
\begin{tabular}{|c|c|c|c|c|c|c|c|c|}
\hline
\textbf{\multirow{2}{*}{Model}} &\textbf{\multirow{2}{*}{Text}} &\textbf{\multirow{2}{*}{Image} }&\textbf{IC13} &\textbf{SVT}  &\textbf{IIIT}   & \textbf{\multirow{2}{*}{AVG}}  & \textbf{Time} &   \textbf{Params}\\
& & & \textbf{IC15} & \textbf{SVTP} &\textbf{CUTE}  & & \textbf{(ms)} &  \textbf{($ \times 10^6$)} \\
\hline
\multirow{2}{*}{(a)}  & \multirow{2}{*}{1} & \multirow{2}{*}{1} &98.02	&96.29	&96.57	&\multirow{2}{*}{94.03} & \multirow{2}{*}{2.48} & \multirow{2}{*}{15.6}  \\
& & &88.02	&92.56	&91.67	&	&	&	 \\
\hline
\multirow{2}{*}{(b)}  & \multirow{2}{*}{1} & \multirow{2}{*}{2} &97.90	&96.29	&96.50	&\multirow{2}{*}{93.97}	& \multirow{2}{*}{3.02}	& 	\multirow{2}{*}{22.6}  \\
& & &88.13	&92.09	&91.67	&	& &	 \\
\hline
\multirow{2}{*}{(c)}  & \multirow{2}{*}{2} & \multirow{2}{*}{1}  &97.78	&96.29	&96.77	&\multirow{2}{*}{94.05} 	& \multirow{2}{*}{3.05}	&  \multirow{2}{*}{22.6}	\\
& & &88.13	&92.09	&91.32	&	&	&	 \\
\hline
\multirow{2}{*}{(d)}  & \multirow{2}{*}{2} & \multirow{2}{*}{2} &98.02	&96.29	&96.57	&\multirow{2}{*}{94.00} 	& \multirow{2}{*}{3.65}	& \multirow{2}{*}{29.7} \\
& & &88.13	&92.09	&91.32	&	&	&	 \\
\hline
\end{tabular}
\vspace{-3mm}
\end{table}

\begin{table}[t]\centering
\setlength{\tabcolsep}{2.5pt}
\ra{1.2}
\caption{The results of MGP-STR$^{\dagger}_{LFS}$ trained with different numbers of noised text variants.}
\label{tab:neg}
\begin{tabular}{|c|c|c|c|c|c|c|c|c|c|}
\hline
\textbf{Models} &\textbf{Number} &\textbf{IC13} &\textbf{SVT}  &\textbf{IIIT}   & \textbf{IC15} & \textbf{SVTP} &\textbf{CUTE}  &\textbf{AVG} \\
\hline
CFS &  - &97.43	&95.52	&96.77	&87.80	&90.23	&91.32	&93.70	 \\
\hline
\multirow{4}{*}{LFS} &  8 &97.55	&95.67	&96.50	&87.74	&91.32	&90.63	&93.67 \\
&  64 &97.78	&95.98	&96.53	&87.85	&91.94	&90.63	&93.82	 \\
&  128 &97.90	&96.29	&96.57	&88.02	&92.25	&91.32	&93.97	 \\
&  256 &97.78	&96.29	&96.77	&88.13	&92.09	&91.32	&94.05	 \\
\hline
\end{tabular}
\vspace{-3mm}
\end{table}

\subsubsection{Effect of the Number of Noised Text} 
We construct experiments to investigate the effect of the number of negative pairs, which is essential in contrastive learning.
The results of models trained with $4$ different numbers of noised text are listed in Tab.~\ref{tab:neg}.
In general, the more noised text, the relatively higher the fused accuracies.
Specifically, the fusion text encoder can not be sufficiently learned with less noised text variants(\ie, $8$ or $64$).
Thus, MGP-STR$^{\dagger}_{LFS}$ with $8$ or $16$ noised text can not obtain obvious advantage over MGP-STR$^{\dagger}_{CFS}$.
However, when the number of noised text instances increases to $128$ or $256$, MGP-STR$^{\dagger}_{LFS}$ can achieve appealing improvements ($0.3 \%$) over MGP-STR$^{\dagger}_{CFS}$.
Due to the training efficiency, we set the number of noised text instances to $256$ in the proposed LFS module.

\subsection{Inside Details of MGP-STR}  
\subsubsection{Details of Multi-Granularity Predictions of CFS}  
\label{Sec:fa}

We show the detailed prediction process of the proposed MGP-STR$_{CFS}$ method on $6$ test images from standard datasets in Tab.~\ref{tab:fa}. In the first three images, the results of character-level prediction are incorrect, due to irregular font, motion blur and curved shape, respectively. The scores of character prediction are very low, since the images are difficult to recognize and one character is wrong in each image. However, ``BPE'' and ``WP'' heads can recognize ``table'' image with high scores. And ``BPE'' can make correct predictions with two subwords on ``dvisory'' and ``watercourse'' images, while ``WP''  is wrong for the ``watercourse'' image. After fusion, the mistakes can be corrected. From the rest three images, interesting phenomena can be observed. The predictions of ``Char'' and ``BPE''  conform to the images. The predictions of ``WP'', however, attempt to produce strings with more linguistic content, like ``today'' and ``guide''.
Generally, ``Char'' aims to produce characters one by one, while ``BPE'' usually generates n-gram segments related to images, and ``WP'' tends to directly predict whole words that are linguistically meaningful. These prove that the predictions of different granularities convey text information in different aspects and are indeed complementary.

\begin{table*}[t]\centering
\setlength{\tabcolsep}{5pt}
\ra{1.2}
\caption{The details of multi-granularity predictions of MGP-STR$_{CFS}$, including the CFS recognition scores (Score) of each prediction head, the intermediate multi-granularity (Gra.) results and the predictions (Pred.). (Best viewed in color.)}
\label{tab:fa}
\begin{tabular}{|c|c|c|c|c|c|c|c|}
\hline
\textbf{Images} & \textbf{GT} &\textbf{Output} &\textbf{Char}  &\textbf{BPE}   & \textbf{WP}  & \textbf{CFS} \\
\hline
    \multirow{3}{*}{
    \begin{minipage}[b]{0.15\columnwidth}
		\centering
		\raisebox{-.5\height}{% Figure removed}
	\end{minipage}
	}
	& \multirow{3}{*}{table}
	&Score & 0.1643 & 0.9813 & 0.9521 & 0.9813\\
	\cline{3-7}
	& &Gra. & tabbe & \textcolor{orange}{table} & \textcolor{orange}{table}  & -\\
	\cline{3-7}
	& &Pred. & tabbe & table &  table & table \\
\hline
    \multirow{3}{*}{
    \begin{minipage}[b]{0.15\columnwidth}
		\centering
		\raisebox{-.5\height}{% Figure removed}
	\end{minipage}
	}
	& \multirow{3}{*}{dvisory}
	&Score &0.0316 & 0.8218 & 0.2574 & 0.8218\\
	\cline{3-7}
	& &Gra. &divsoory &  \textcolor{red}{d}  \textcolor{blue}{visory} &  \textcolor{orange}{dvisory} & - \\
	\cline{3-7}
	& &Pred. &divsoory & dvisory  & dvisory & dvisory \\
	\hline
    \multirow{3}{*}{
    \begin{minipage}[b]{0.15\columnwidth}
		\centering
		\raisebox{-.5\height}{% Figure removed}
	\end{minipage}
	}
	& \multirow{3}{*}{watercourse}
	&Score &0.1565 & 0.8295 & 0.632 & 0.8295\\
	\cline{3-7}
	& &Gra. & watercourss &  \textcolor{red}{water}  \textcolor{blue}{course} &  \textcolor{orange}{waterco} & - \\
	\cline{3-7}
	& &Pred. &watercourss & watercourse  & waterco & watercourse \\
		\hline
    \multirow{3}{*}{
    \begin{minipage}[b]{0.15\columnwidth}
		\centering
		\raisebox{-.5\height}{% Figure removed}
	\end{minipage}
	}
	& \multirow{3}{*}{1869}
	&Score &0.9999 & 0.9207 & 0.0354 & 0.9999\\
	\cline{3-7}
	& &Gra. & 1869 &  \textcolor{red}{18}  \textcolor{blue}{69} &  \textcolor{orange}{18} & - \\
	\cline{3-7}
	& &Pred. &1869 & 1869  & 18 & 1869 \\
	\hline
    \multirow{3}{*}{
    \begin{minipage}[b]{0.12\columnwidth}
		\centering
		\raisebox{-.5\height}{% Figure removed}
	\end{minipage}
	}
	& \multirow{3}{*}{thday}
	&Score &0.9998 & 0.5983 & 0.7638 & 0.9998 \\
	\cline{3-7}
	& &Gra. & thday &  \textcolor{red}{th}  \textcolor{blue}{day} &  \textcolor{orange}{today} & - \\
	\cline{3-7}
	& &Pred. &thday & thday  & today & thday \\
				\hline
    \multirow{3}{*}{
    \begin{minipage}[b]{0.12\columnwidth}
		\centering
		\raisebox{-.5\height}{% Figure removed}
	\end{minipage}
	}
	& \multirow{3}{*}{guide}
	&Score &0.9675 & 0.6959 & 0.1131 & 0.9675 \\
	\cline{3-7}
	& &Gra. & guice &  \textcolor{red}{gu}  \textcolor{blue}{ice} &  \textcolor{orange}{guide} & - \\
	\cline{3-7}
	& &Pred. &guice & guice  & guide & guice \\
	\hline
\end{tabular}
\end{table*}

% \begin{table*}[t]\centering
% \setlength{\tabcolsep}{5pt}
% \ra{1.2}
% \caption{The details of multi-granularity prediction of MGP-STR$_{CFS}$, including the CFS recognition scores (Score) of each prediction head, the intermediate multi-granularity (Gra.) results and the final prediction (Pred.). Best viewed in color.   }
% \label{tab:fa}
% \begin{tabular}{|c|c|c|c|c|c|c|c|}
% \hline
% \textbf{Images} & \textbf{GT} &\textbf{Output} &\textbf{Char}  &\textbf{BPE}   & \textbf{WP}  & \textbf{CFS} \\
% \hline
%     \multirow{3}{*}{
%     \begin{minipage}[b]{0.15\columnwidth}
% 		\centering
% 		\raisebox{-.5\height}{% Figure removed}
% 	\end{minipage}
% 	}
% 	& \multirow{3}{*}{table}
% 	&Score & 0.1643 & 0.9813 & 0.9521 & 0.9813\\
% 	\cline{3-7}
% 	& &Gra. & tabbe & \textcolor{green}{table} & \textcolor{green}{table}  & -\\
% 	\cline{3-7}
% 	& &Pred. & tabbe & table &  table & table \\
% \hline
%     \multirow{3}{*}{
%     \begin{minipage}[b]{0.15\columnwidth}
% 		\centering
% 		\raisebox{-.5\height}{% Figure removed}
% 	\end{minipage}
% 	}
% 	& \multirow{3}{*}{dvisory}
% 	&Score &0.0316 & 0.8218 & 0.2574 & 0.8218\\
% 	\cline{3-7}
% 	& &Gra. &divsoory &  \textcolor{red}{d}  \textcolor{blue}{visory} &  \textcolor{green}{dvisory} & - \\
% 	\cline{3-7}
% 	& &Pred. &divsoory & dvisory  & dvisory & dvisory \\
% 	\hline
%     \multirow{3}{*}{
%     \begin{minipage}[b]{0.15\columnwidth}
% 		\centering
% 		\raisebox{-.5\height}{% Figure removed}
% 	\end{minipage}
% 	}
% 	& \multirow{3}{*}{watercourse}
% 	&Score &0.1565 & 0.8295 & 0.632 & 0.8295\\
% 	\cline{3-7}
% 	& &Gra. & watercourss &  \textcolor{red}{water}  \textcolor{blue}{course} &  \textcolor{green}{waterco} & - \\
% 	\cline{3-7}
% 	& &Pred. &watercourss & watercourse  & waterco & watercourse \\
% 		\hline
%     \multirow{3}{*}{
%     \begin{minipage}[b]{0.15\columnwidth}
% 		\centering
% 		\raisebox{-.5\height}{% Figure removed}
% 	\end{minipage}
% 	}
% 	& \multirow{3}{*}{1869}
% 	&Score &0.9999 & 0.9207 & 0.0354 & 0.9999\\
% 	\cline{3-7}
% 	& &Gra. & 1869 &  \textcolor{red}{18}  \textcolor{blue}{69} &  \textcolor{green}{18} & - \\
% 	\cline{3-7}
% 	& &Pred. &1869 & 1869  & 18 & 1869 \\
% 	\hline
%     \multirow{3}{*}{
%     \begin{minipage}[b]{0.12\columnwidth}
% 		\centering
% 		\raisebox{-.5\height}{% Figure removed}
% 	\end{minipage}
% 	}
% 	& \multirow{3}{*}{thday}
% 	&Score &0.9998 & 0.5983 & 0.7638 & 0.9998 \\
% 	\cline{3-7}
% 	& &Gra. & thday &  \textcolor{red}{th}  \textcolor{blue}{day} &  \textcolor{green}{today} & - \\
% 	\cline{3-7}
% 	& &Pred. &thday & thday  & today & thday \\
% 				\hline
%     \multirow{3}{*}{
%     \begin{minipage}[b]{0.12\columnwidth}
% 		\centering
% 		\raisebox{-.5\height}{% Figure removed}
% 	\end{minipage}
% 	}
% 	& \multirow{3}{*}{guide}
% 	&Score &0.9675 & 0.6959 & 0.1131 & 0.9675 \\
% 	\cline{3-7}
% 	& &Gra. & guice &  \textcolor{red}{gu}  \textcolor{blue}{ice} &  \textcolor{green}{guide} & - \\
% 	\cline{3-7}
% 	& &Pred. &guice & guice  & guide & guice \\
% 	\hline
% \end{tabular}
% \end{table*}



\begin{table*}[t]\centering
\setlength{\tabcolsep}{2.5pt}
\ra{1.2}
\caption{The details of MGP-STR$_{LFS}$, including the CFS recognition scores (Score) and the final prediction (Pred.) of each prediction head. The similarities between each type of prediction and input images given by LFS are listed at (Sim.). }
\label{tab:lfs}
\begin{tabular}{|c|c|c|c|c|c|c|c|}
\hline
\textbf{Images} & \textbf{GT} &\textbf{Output} &\textbf{Char}  &\textbf{BPE}   & \textbf{WP}  & \textbf{CFS} & \textbf{LFS} \\
\hline
    \multirow{3}{*}{
    \begin{minipage}[b]{0.15\columnwidth}
		\centering
		\raisebox{-.5\height}{% Figure removed}
	\end{minipage}
    }
    & \multirow{3}{*}{cafe} &Score & 0.33 & 0.81 & 0.28 & 0.81 & - \\
    \cline{3-8}
    & & Sim. & 0.05 & 0.02 & 0.15 & - & 0.15 \\
    \cline{3-8}
    & & Pred. & caf & c & cafe & c & cafe \\
   \hline

    \multirow{3}{*}{
    \begin{minipage}[b]{0.23\columnwidth}
		\centering
		\raisebox{-.5\height}{% Figure removed}
	\end{minipage}
    }
    & \multirow{3}{*}{hotel} &Score & 0.88 & 0.69 & 0.79 & 0.88 & - \\
    \cline{3-8}
    & &Sim. & 0.28 & 0.17 & 0.33 & - & 0.33 \\
    \cline{3-8}
    & & Pred. & horel & hoel & hotel & horel & hotel \\
    \hline

    \multirow{3}{*}{
    \begin{minipage}[b]{0.15\columnwidth}
		\centering
		\raisebox{-.5\height}{% Figure removed}
	\end{minipage}
    }
    & \multirow{3}{*}{hard} &Score & 0.45 & 0.26 & 0.44 & 0.45 & - \\
    \cline{3-8}
    & &Sim. & 0.07 & 0.07 & 0.08 & - & 0.08 \\
    \cline{3-8}
    & & Pred. & herd & herd & hard & herd & hard \\
    \hline

    \multirow{3}{*}{
    \begin{minipage}[b]{0.23\columnwidth}
		\centering
		\raisebox{-.5\height}{% Figure removed}
	\end{minipage}
    }
    & \multirow{3}{*}{honeybee} &Score & 0.75 & 0.95 & 0.33 & 0.95 & - \\
    \cline{3-8}
    & &Sim. & 0.11 & -0.02 & 0.21 & - & 0.21 \\
    \cline{3-8}
    & & Pred. & honeybec & honey & honeybee & honey & honeybee \\
    \hline

    \multirow{3}{*}{
    \begin{minipage}[b]{0.25\columnwidth}
		\centering
		\raisebox{-.5\height}{% Figure removed}
	\end{minipage}
    }
    & \multirow{3}{*}{restaurants} &Score & 0.01 & 0.35 & 0.07 & 0.35 & - \\
    \cline{3-8}
    & &Sim. & -0.24 & -0.06 & 0.03 & - & 0.03 \\
    \cline{3-8}
    & & Pred. & restuuannts & resturants & restaurants & resturants & restaurants \\
	\hline

    \multirow{3}{*}{
    \begin{minipage}[b]{0.28\columnwidth}
		\centering
		\raisebox{-.5\height}{% Figure removed}
	\end{minipage}
    }
    & \multirow{3}{*}{cinemasterpieces} &Score & 0.90 & 0.85 & 0.07 & 0.90 & - \\
    \cline{3-8}
    & &Sim. & 0.33 & 0.38 & 0.23 & - & 0.38\\
    \cline{3-8}
    & & Pred. & cinemasterpiecss & cinemasterpieces & cinemaspiece & cinemasterpiecss & cinemasterpieces\\
    \hline
\end{tabular}
\end{table*}


\subsubsection{Details of Fusion Process of LFS}  \label{Sec:explfs}

We show several typical cases where the results of MGP-STR$_{CFS}$ are wrong, but MGP-STR$_{LFS}$ make correct predictions, to analyze the working mechanism of LFS.
The detailed predictions and similarities produced by MGP-STR$_{LFS}$ on $6$ test images are illustrated in Tab.\ref{tab:lfs}.
Generally, all ``Char'' predictions in Tab.\ref{tab:lfs} are wrong, due to irregular fonts, distortions or long text.
And the correct results of ``BPE'' and ``WP'' heads may be with lower CFS recognition scores, leading to wrong fusion results for MGP-STR$_{CFS}$.
In contrast, the similarities generated by LFS can be utilized to correct these errors.
Key observations from Tab.\ref{tab:lfs} are: 
(1) The similarities between the image and the wrong text with unmatched length are obviously lower than ones with matched length ( \eg, ``cafe'' $>$ ``caf''  $>$ ``c'', ``restaurants'' $>$ ``resturants''  $>$ ``restuuannts'',  and ``honeybee'' $>$ ``honeybec''  $>$ ``honey'').
(2) Even if only one character is incorrect, the similarity value will decrease drastically, \eg, $10\%$ of ``honeybee'' over ``honeybec'' and $5\%$ of ``cinemasterpieces'' over ``cinemasterpiecss''.
These qualitative analyses demonstrate that the similarity values of LFS can faithfully reflect the matching degrees between the images and the text sequences (even with subtle distinctions). In conclusion, LFS is a more effective way for fusing multi-granularity predictions than CFS.

\begin{table*}[t]
\setlength{\tabcolsep}{5pt}
\ra{1.3}
\centering
\caption{The comparisons with other STR methods on $8$ public standard benchmarks. The models with $\dagger$ are trained with $200$ million iterations. For training data, ``S'' represents synthetic training datasets (MJSynth and SynthText), S$^*$ means external text datasets are used for the language model, and ``R'' represents real datasets provided by~\cite{PARSeq}. The \textbf{bold} and \underline{underline} numbers represent the best and the second-best results, respectively. ``AVG'' is the average accuracy except those of IC13 (1015) and IC15 (2077).}
\label{tab:sota}
\begin{tabular}{|l|c|c|c|c|c|c|c|c|c|c|c|}
\hline
\textbf{\multirow{3}{*}{Methods}}  &\textbf{\multirow{3}{*}{Year}} &  &\multicolumn{4}{c|}{\textbf{Regular Text}} &\multicolumn{4}{c|}{\textbf{Irregular Text}} & \textbf{\multirow{2}{*}{AVG} }\\
\cline{4-11}
& &\textbf{Train} &\textbf{IC13} &\textbf{IC13} &\textbf{SVT}  &\textbf{IIIT}   & \textbf{IC15} & \textbf{IC15} & \textbf{SVTP} &\textbf{CUTE} &  \\
\cline{4-12}
& &\textbf{data} &\textbf{857} &\textbf{1015} &\textbf{647}  &\textbf{3000}   & \textbf{1811}  & \textbf{2077} & \textbf{645} &\textbf{288} & \textbf{7248} \\
\hline
TBRA~\cite{deep} &ICCV (2019) &S &93.6 &92.3 &87.5 &87.9 &77.6 & 71.8 &79.2  & 74.0  & 84.6 \\
ViTSTR-Base~\cite{ViTSTR} &ICDAR (2021) &S &93.2 &92.4 & 87.7 & 88.4 &78.5 &72.6 &81.8 &81.3 &85.6 \\
ESIR~\cite{ESIR} &CVPR (2019) &S  &-  &91.3 &90.2 &93.3 &- &76.9 &79.6 &83.3  &-  \\
DAN~\cite{DAN} &AAAI (2020) &S &-  &93.9 &89.2 &94.3  &- &74.5  &80.0 &84.4   & -  \\
%SAM~\cite{TextSpotter}     &95.3 &90.6 &93.9 &77.3 &82.2 &87.8  &  88.34  \\
SE-ASTER~\cite{SEED} &CVPR (2020) &S  &- &92.8 &89.6 &93.8 &80.0 & &81.4 &83.6  & 88.3  \\
TrOCR-Base~\cite{TrOCR} &AAAI (20223) &S &97.3	&96.3 &91.0	&90.1	&81.1 &75.0	&90.7 &86.8 & 88.7 \\
RobustScanner~\cite{RobustScanner}  &ECCV (2020) &S  &- &94.8 & 88.1  & 95.3 &- & 77.1 & 79.5 &  90.3 & - \\
TextScanner~\cite{TextScanner} &AAAI (2020) &S &- &92.9 &90.1 &93.9 &79.4 & &84.3 &83.3  & - \\
%ScRN~\cite{ScRN}  &93.9  & 88.9   &94.4    &78.7 & 80.8   & 87.5 & 88.44 \\
SATRN~\cite{SATRN} &CVPRW (2020)  &S  &- &94.1 &91.3 &92.8  &- &79.0 &86.5 &87.8 &- \\
MASTER~\cite{MASTER} &PR (2021) &S &-	&95.3 &90.6	&95.0	&- &79.4 &84.5	&87.5 & - \\
% \hline
SRN~\cite{SRN} &CVPR (2020)  &S  & 95.5 & - & 91.5  & 94.8  &82.7 & - &85.1   &87.8 & 90.4 \\
VisionLAN~\cite{vlan}  &ICCV (2021) &S &95.7 & - &91.7& 95.8 &83.7 & - &86.0 &88.5 &91.2 \\
PARSeq~\cite{PARSeq} &ECCV (2022) &S &96.3	&95.5 &92.6	&95.7	&85.1 &81.4	&87.9	&91.4 & 92.0 \\
ABINet++~\cite{ABInetv2}  &TPAMI (2023) &S$^*$  &97.4 & 95.7 &93.5 &96.2 &86.0 &\textbf{85.1} &89.3 &89.2  & 92.6 \\
LevOCR~\cite{matrn} &ECCV (2022) &S$^*$ &96.9 &- &92.9 &96.6	&86.4 &-	&88.1	&91.7 & 92.8 \\
% \hline
% PARSeq$_A$~\cite{PARSeq} &ECCV(2022) &S &97.0	&96.2 &93.6	&\textbf{97.0}	&86.5 &82.9	&88.9	&\underline{92.2} & 93.2 \\
PTIE~\cite{ptie} &ECCV (2022) &S &-	&\textbf{97.2} &94.9	&96.3	&87.8 &\underline{84.3} &90.1	&\underline{91.7} &- \\
MATRN~\cite{matrn} &ECCV (2022) &S$^*$ &\textbf{97.9}	&95.8 &95.0	&96.6	&86.6 &82.8	&90.6	&\textbf{93.5} & 93.5 \\
% \hline
MGP-STR$_{CFS}$~\cite{mgp} &ECCV (2022) &S  &{97.32} &96.55 &94.74 &96.40	&87.24 & 83.78	&\underline{91.01}	&90.28	&93.35 \\
\hline
MGP-STR$^{\dagger}_{CFS}$ &Ours &S &97.43	& 96.65 &\underline{95.52}	&\underline{96.77}	&\underline{87.80}	& 83.78 &90.23	&91.32	&\underline{93.70} \\
MGP-STR$^{\dagger}_{LFS}$ &Ours &S &\underline{97.78}	&\underline{96.95} &\textbf{96.29}	&\textbf{96.77}	&\textbf{88.13}	&84.11 &\textbf{92.09}	&91.32	&\textbf{94.05}	 \\
\hline
PARSeq~\cite{PARSeq} &ECCV (2022) &R &98.0	&98.1 &97.5	&98.3	&89.6 &88.4	&94.6	&97.7 &95.7 \\
ABINet++~\cite{ABInetv2}  &TPAMI (2023) &R  &98.0 &97.8 &\underline{97.8} &\textbf{98.6} &90.2 &88.5 &93.9 &97.7  &95.9 \\
MGP-STR$^{\dagger}_{CFS}$ &Ours &R  & \underline{98.37} & \underline{98.13} & 97.68  & 97.93  &\underline{90.89} & \underline{89.55} &\underline{95.35}   &\underline{97.92} & \underline{95.97} \\
MGP-STR$^{\dagger}_{LFS}$ &Ours &R  & \textbf{98.60} & \textbf{98.42} & \textbf{98.30}  &\underline{98.40}  &\textbf{91.06} & \textbf{89.79} &\textbf{96.59}   &\textbf{97.92} & \textbf{96.40} \\
\hline
\end{tabular}
\end{table*}

\subsection{Comparisons with Previous State-of-the-Arts} 
\label{Sec:sota}

\subsubsection{Results on Standard Benchmarks}

We compare the proposed MGP-STR$^{\dagger}_{CFS}$ and MGP-STR$^{\dagger}_{LFS}$ methods with previous state-of-the-art scene text recognition methods, and the results on $8$ standard benchmarks are summarized in Tab.~\ref{tab:sota}. All of the compared methods and ours are trained on synthetic datasets MJ and ST for a fair evaluation. And the results are obtained without any lexicon-based post-processing. Generally, language-augmented methods (\ie, SRN~\cite{SRN}, VisionLAN~\cite{vlan}, PARSeq~\cite{PARSeq}, ABINet++~\cite{ABInetv2}, LevOCR~\cite{levocr}, MATRN~\cite{matrn} and MGP-STR) perform better than language-free methods, showing the significance of linguistic information. PTIE~\cite{ptie}, which utilizes a transformer-only model with multiple patch resolutions, also achieves good results. 

Notably, owing to the multi-granularity predictions, MGP-STR$^{\dagger}_{CFS}$ has already outperformed the recent state-of-the-art method MATRN. Furthermore, MGP-STR$^{\dagger}_{LFS}$ obtains more impressive results by resorting to the proposed LFS over MGP-STR$^{\dagger}_{CFS}$.
Particularly, MGP-STR$^{\dagger}_{LFS}$ achieves the best results on $4$ out of $8$ benchmarks, outperforming the second best method MATRN by $1.3 \%$ on SVT, $0.17 \%$ on IIIT, $1.5 \%$ on IC15 (1811) and $2.3 \%$ on SVTP.
Moreover,  MGP-STR$^{\dagger}_{LFS}$ can attain the second best performance on IC13(857), IC13(1015) and IIIT. 
And MGP-STR$^{\dagger}_{LFS}$ also achieves comparable results on IC15(2077) and CUTE.
Consequently, MGP-STR$^{\dagger}_{LFS}$ achieves a substantial improvement of $0.7\%$ over MATRN in terms of average accuracy.

Following PARSeq~\cite{PARSeq}, we also train MGP-STR with real data to further study the potential of our method.
The models trained with real data as shown in Tab.~\ref{tab:sota} utilize the training and testing datasets provided by ~\cite{PARSeq} for fair comparisons.
Obviously, using real data can significantly boost the recognition accuracy of the evaluated models.
MGP-STR$^{\dagger}_{LFS}$ can also obtain consistent improvement ($0.43 \%$) over MGP-STR$^{\dagger}_{CFS}$, and achieve the best results on $7$ out of $8$ benchmarks.
Thus, MGP-STR$^{\dagger}_{LFS}$ establishes new SOTA results.
These comparisons clearly demonstrate the effectiveness of the proposed MGP and LFS strategies.

\begin{table}[t]\centering
\setlength{\tabcolsep}{2.5pt}
\ra{1.3}
\caption{The comparisons with other STR methods on $3$ challenging datasets. The models with $\dagger$ are trained with $200$ million iterations. For training data, ``S'' represents synthetic training datasets (MJSynth and SynthText), and ``R'' represents real training data provided by~\cite{PARSeq}.}
\label{tab:art}
\begin{tabular}{|l|c|c|c|c|c|c|c|}
\hline
\textbf{\multirow{2}{*}{Methods} } & \textbf{\multirow{2}{*}{Year}}  &\textbf{Train} &\textbf{ArT} &\textbf{COCO}  &\textbf{Uber}   &\textbf{AVG} \\
\cline{4-7}
&  &\textbf{data} &\textbf{35149} &\textbf{9825}  &\textbf{80418}   &\textbf{125392} \\
\hline
ABINet++~\cite{ABInetv2}  & TPAMI (2023) &S &65.4 &57.1 &34.9	&45.2 \\
% PARSeq$_A$~\cite{PARSeq} & ECCV(2022) &S &70.7 &64.0	&42.0	&51.8 \\
PARSeq~\cite{PARSeq} & ECCV (2022) &S &69.1 &60.2	&39.9	&49.7 \\
MATRN~\cite{matrn} & ECCV (2022) &S &68.9	&64.0	&40.1	&50.0 \\
% \hline
% MGP-STR$^{\dagger}_{Char}$ & Ours &S &97.32 &94.74 &96.40	&87.24 \\
% MGP-STR$^{\dagger}_{BPE}$ & Ours &S &97.32 &94.74 &96.40	&87.24 \\
% MGP-STR$^{\dagger}_{WP}$ & Ours &S &97.32 &94.74 &96.40	&87.24 \\
% \hline
MGP-STR$^{\dagger}_{CFS}$ & Ours &S &70.38 &66.20 &42.18	&51.96 \\
MGP-STR$^{\dagger}_{LFS}$ & Ours &S &70.67 &66.94 &42.59	&52.37 \\
\hline
ABINet++~\cite{ABInetv2}  & TPAMI (2023) &R &81.2 &76.4 &71.5	&74.6 \\
PARSeq~\cite{PARSeq} & ECCV (2022) &R &83.0 &77.0	&82.4	&82.1 \\
MGP-STR$^{\dagger}_{CFS}$ & Ours &R &83.48 &78.43 &82.51	&82.46 \\
MGP-STR$^{\dagger}_{LFS}$ & Ours &R &84.10 &79.81 &83.32	&83.26 \\
\hline
\end{tabular}
\vspace{-4mm}
\end{table}

\subsubsection{Results on More Challenging Datasets} 

To further verify the superiority of MGP-STR in dealing with 
challenging scenarios, we also evaluate it on $3$ large-scale datasets with real-world complexity (\ie, ArT~\cite{ArT}, COCO-Text~\cite{COCO} and Uber-Text~\cite{Uber}).
We construct $2$ types of experiments:
(1) The evaluated models are trained on synthetic training datasets (\ie, MJSynth and SynthText), and tested on the three challenging datasets (\ie, ArT, COCO-Text, and Uber-Text);
(2) The models are directly trained on the real training data as mentioned in Sec.~\ref{Sec:data}, and also evaluated on the three datasets.
The results with these $2$ different experimental settings are shown in Tab.~\ref{tab:art}.

For the experiments training on synthetic data,
the word level recognition accuracies of all methods on these challenging datasets are obviously lower than those of the standard benchmarks in Tab.~\ref{tab:sota}. This indicates that for STR methods there is still a significant gap between training on synthetic data and real data.
Typically, MGP-STR$^{\dagger}_{LFS}$ can also surpass MGP-STR$^{\dagger}_{CFS}$ and achieve the best performance with synthetic data training.
Note that the model weights of MGP-STR in Tab.~\ref{tab:art} are the same as those in Tab.~\ref{tab:sota}.
These experiments show that MGP-STR has excellent robustness and generalization ability.

Compared with training on synthetic data, when training on real datasets, the overall accuracy of all methods has been greatly improved (about $30 \%$ absolute gain in performance) on challenging datasets, while only about $2 \%$ improvements on standard benchmarks in Tab.~\ref{tab:sota}.
This proves that using real data can mitigate the impact of different data distributions, and attain much higher accuracy for real challenging images.
Moreover, MGP-STR$^{\dagger}_{LFS}$ consistently achieves the best results, compared with the recent STR methods, confirming that MGP-STR can adapt well to scene text images with real-world challenges.

\begin{table}[t]\centering
\setlength{\tabcolsep}{4pt}
\ra{1.3}
\caption{The comparisons with other STR methods on three handwritten text benchmarks. LFS$^\#$ indicates that the LFS module is trained with MJSynth and SynthText, besides the handwritten text datasets. }
\label{tab:hand}
\begin{tabular}{|l|c|c|c|c|c|}
\hline
\textbf{\multirow{2}{*}{Methods} }& \textbf{\multirow{2}{*}{Year}} &\textbf{IAM} &\textbf{CVL}  &\textbf{RIMES}  \\
\cline{3-6}
&  &\textbf{13752} &\textbf{12012}  &\textbf{7776}  \\
\hline
SeqCLR~\cite{seqclr}  & CVPR (2021) &79.9 &77.8 &92.4	\\
% SLOGAN~\cite{SLOGAN} & TNNLS (2022) &87.1 &-	&91.2  \\
TextAdaIN~\cite{TextAdaIN} & ECCV (2022) &87.3 &78.2 &94.4 \\
% \hline
PerSec-ViT~\cite{liu2022perceiving} & AAAI (2022) &83.7	&82.9	&-	 \\
DiG-ViT-Base~\cite{DiG} & MM (2022) &87.0 &91.3	&-	\\
\hline
MGP-STR$_{Char}$ & Ours &90.55 &89.02  &93.80	\\
MGP-STR$_{BPE}$&  Ours &86.95 &89.98 &92.40	\\
MGP-STR$_{WP}$ & Ours &84.16 &89.24 &91.78 \\
\cline{1-5}
MGP-STR$_{CFS}$   & Ours  &91.89 &90.84  &94.87 \\
MGP-STR$_{LFS}$ & Ours &90.21 &91.24 &94.30 \\
% \cline{1-5}
MGP-STR$_{LFS^\#}$& Ours &92.93 &91.96 &95.20 \\
\hline
\end{tabular}
\vspace{-3mm}
\end{table}

\begin{table}[t]\centering
%\vspace{-2mm}
\setlength{\tabcolsep}{2.5pt}
\ra{1.2}
\caption{Comparisons on inference time and model size.}
\label{tab:speed}
%\vspace{-2mm}
\begin{tabular}{|l|l|l|}
\hline
%\textbf{Model}  & \begin{tabular}[c]{@{}l@{}}\textbf{Speed}\\(msec/image)\end{tabular} & \begin{tabular}[c]{@{}l@{}}\textbf{Parameters}\\(1 x 10$^6$)\end{tabular}  \\
\textbf{Methods} & \textbf{Time (ms)} & \textbf{Params ($\times 10^6$)}   \\
\hline
ABINet-S-iter1/iter2/iter3 & 13.7/18.6/24.3 & 32.8 \\
ABINet-L-iter1/iter2/iter3 & 16.1/21.4/26.8 & 36.7 \\
MATRN-iter1/iter2/iter3 & 17.9/26.5/35.1 & 44.2 \\
\hline
MGP-STR$_{Vision}$-tiny/small/base  & 10.6/10.8/10.9  & 5.4/21.4/85.5 \\
MGP-STR$_{CFS}$-tiny/small/base  & 12.0/12.2/12.3  & 21.0/52.6/148.0 \\
MGP-STR$_{LFS}$-base  & 15.3  & 170.6 \\
\hline
\end{tabular}
%\vspace{-3mm}
\end{table}

\subsubsection{Results on Handwritten Text Datasets} 

In order to validate the adaptability of MGP-STR, we carry out experiments and comparisons on $3$ widely-used handwritten text benchmarks, and the word-level accuracies are reported in Tab.~\ref{tab:hand}.
SeqCLR~\cite{seqclr} and TextAdaIN~\cite{TextAdaIN} are pre-trained and fine-tuned with only handwritten text images.
Since the number of images in the handwritten dataset is relatively small, PerSec~\cite{liu2022perceiving} and DiG~\cite{DiG} utilize large-scale unlabeled text images for model pre-training, resulting better performance.

For better results of handwritten text recognition, we directly fine-tune the model of MGP-STR$^{\dagger}_{CFS}$ trained on synthetic datasets in Tab.~\ref{tab:sota} with $300,000$ iterations.
LFS modules are further trained on handwritten datasets.
In general, MGP-STR$_{CFS}$ can effectively fuse the multi-granularity predictions, and outperform any single classification heads.
MGP-STR$_{LFS}$ can surpass MGP-STR$_{CFS}$ on CVL, but obtain worse results on IAM and RIMES.
We suspect that since there are only a limited number of words in the handwritten text datasets, the learning of textual information is insufficient.
Thus, besides handwritten data, we add synthetic data (\ie, MJSynth and SynthText) into the LFS learning.
The resultant model, MGP-STR$_{LFS^\#}$, can achieve considerable improvements over MGP-STR$_{LFS}$ ($2.7 \%$ on IAM, $0.7 \%$ on CVL and $0.9 \%$ on RIMES).
Finally, MGP-STR$_{LFS^\#}$ achieves SOTA performances on benchmarks for handwritten text recognition, verifying the effectiveness and generalization ability of our method.

\subsubsection{Comparisons of Model Size and Inference Time} 

The model sizes and latencies of the proposed MGP-STR with different settings as well as those of ABINet, MATRN are depicted in Tab.~\ref{tab:speed}~\footnote{All the evaluations are conducted on a NVIDIA V100 GPU.}. Since MGP-STR is equipped with a regular Vision Transformer (ViT) and involves no iterative refinement, the inference speed of MGP-STR is very fast. With the ViT-Base backbone, MGP-STR$_{CFS}$ and MGP-STR$_{LFS}$ only take $12.3$ ms and $15.3$ ms, respectively. Compared with ABINet and MATRN, MGP-STR runs much faster ($15.3$ms vs. $26.8$ms), and obtains higher performance. The model size of MGP-STR is relatively large. However, a large portion of the model parameter is from the BPE and WordPiece branches. Meanwhile, the actual GPU memory usage of MGP-STR$_{LFS}$-Base is small (about $2$ GB), which allows for the deployment on commonly-used devices. For scenarios that are sensitive to model size or with limited GPU memory, MGP-STR$_{Vision}$ is an excellent choice; for scenarios where resources (compute and memory) are sufficient, MGP-STR$_{LFS}$ is recommended.
%The performance of MGP-STR$_{Vision}$ is higher than that of previous SOTA methods, while its model size is significantly reduced (since there are no BPE and WordPiece branches) and the running speed is further boosted. 
%}

\subsection{Qualitative Results}  
\label{Sec:Qualitative}

% Figure environment removed

% Figure environment removed

% Figure environment removed

We conduct qualitative comparisons with two representative STR methods (\ie ABINet~\cite{ABInet} and MATRN~\cite{matrn}) on typical images from standard benchmarks.
Fig.~\ref{fig:vis_matrn} shows the examplar images on which both ABINet and MATRN fail but MGP-STR$_{LFS}$ succeeds.
Clearly, MGP-STR$_{LFS}$ can produce correct results on vertical text, curved text, occluded text, text with irregular font, and handwritten text.
Moreover, MGP-STR$_{LFS}$ is more robust to disturbances from the background, while ABINet and MATRN tend to predict redundant characters, such as the extra ``k'' and ``1'' in the ``41km'' image.

We also illustrate typical failure cases of MGP-STR in Fig.~\ref{fig:vis_bad}. As can be seen, in the first row, MGP-STR$_{CFS}$ is not good at dealing with extremely long and heavily curved text (with complex background).
This is probably because they are rare in the training set so that the A$^3$ modules are not sufficiently learned.
Moreover, extremely vague text and words with artistic styles may pose challenges.
For the first three images in the second row, MGP-STR with LFS gave wrong predictions, due to the ambiguous visual clues in the images.
For the last image with long digits, MGP-STR$_{CFS}$ can provide correct results, while MGP-STR$_{LFS}$ mode wrong fusion decision, probably due to the weak linguistic prior in combinations of pure digits.

We also show the qualitative results of handwritten text images in Fig.~\ref{fig:vis_hand}.
Compared with scene text, handwritten text is more difficult, due to flexible writing styles, cursive writing, and special characters.
Specifically, images with regular text in the first column can be successfully recognized.
In the second column, the results of MGP-STR$_{CFS}$ are correct, while MGP-STR$_{LFS}$ are wrong.
In the third column, MGP-STR$_{LFS}$ can produce right results while MGP-STR$_{CFS}$ failed, revealing that the latter is not good at predicting correct text lengths for challenging cases.
For the last column, since the images are with ambiguous text instances, MGP-STR gave incorrect results.

% Figure environment removed

\subsection{Visualization of Attention Maps of A$^3$ Modules} 

Exemplar attention maps $\mathbf{m}_i $ of the Character, BPE, WordPiece, and Image A$^3$ modules are shown in Fig.~\ref{fig:viscase}.
The Character A$^3$ module shows extremely precise addressing ability on a variety of text images.
Specifically, for the ``7'' image with one character, the attention mask seems like the ``7'' shape.
For the ``day'' and ``bar'' images with three characters, the attention masks of the middle character ``a'' are completely different.
In addition, the character attention maps can perform well on curved, hand-written, and even long-curved images.
These qualitative analyses verify the capability of the Character A$^3$ module for adaptive addressing and aggregation.

The BPE A$^3$ module tends to generate short segments, as depicted in Fig.\ref{fig:mgp} and in Tab.~\ref{tab:fa}. The attention masks of BPE are spilt into $2$ or $3$ areas as shown in ``leaves'' and ``academy'' images.
Due to the fact that performing subword splitting and character addressing simultaneously may be difficult, the attention masks of the BPE A$^3$ module are not as such precise as those of the Character A$^3$ module.
Moreover, the WordPiece A$^3$ module often produces a whole word, and the attention maps are expected to cover the whole feature map. However, in reality, they are usually sparse, because the softmax function is used to these attention maps.
These results are consistent with those of Tab.~\ref{tab:CBW}, where the accuracies of ``BPE'' and ``WP'' are relatively lower than ``Char'', due to the difficulty of precise subword prediction. 

The attention maps of the Image A$^3$ module in LFS are also sparse. However, the activations appear at certain character areas that might play an important role in computing text-image similarities. Therefore, we think the Image A$^3$ module can be regarded as an attention pooling module for the whole image representation.
% These masks are different from those in WordPiece A$^3$ module.
% We suspect that the feature learnings of cross entropy loss and contrastive loss may be focus on different 

\section{Discussion}
\label{sec:discussion}
%\subsection{Analysis on MGP-STR}
%We list some underlying reasons of why MGP-STR works well from the following aspects. (1) Tokenization scheme that can deal with an infinite potential vocabulary via a finite list of known subwords has been pivotal in the recent methods of NLP (\ie, Transformer architecture~\cite{trans,BERT} and the BERT language model~\cite{BERT}) for the aid of ``reading'' text. Previous methods~\cite{cnn-n-gram,TrOCR} that directly produce wordpiece units as the recognized text show the potential of subword-level predictions in OCR. (2) For ViT-based STR models, small patch shape ($4 \times 4$) is more compatible with text recognition, rather than $16 \times 16$ patches of the standard image. Recent methods PARseq~\cite{PARSeq} and PTIE~\cite{ptie} also utilize $4 \times 8$ patches for ViT backbone. Owing to the powerful capability of ViT structure, customized patch shape and sufficient synthetic training data, ViT-based STR models can establish significant results. (3) Multi-granularity predictions can be regarded as $3$ independently distinct recognizers based on the shared basic features, which can be adaptively transcribed into specific features further by the presented flexible A$^3$ modules for different granularities.Thus, model ensemble via the proposed fusion strategy is an effective way for better results. (4) Recent Language-Image pre-training method, CLIP~\cite{CLIP}, have demonstrated promising transferability between natural language and visual concepts, which enables zero-shot transfer of the model to STR task. Therefore, using the contrastive learning between language and image is a perfect fit for the learnable fusion strategy of MGP-STR. Conclusively, the impressive performance can be achieved by MGP-STR$_{LFS}$.

\subsection{Reflections}

In this section, we will discuss possible reasons for the effectiveness of the proposed Multi-Granularity Prediction (MGP) strategy and the Learnable Fusion strategy (LFS).

As demonstrated in Sec.~\ref{Sec:sota}, the proposed MGP-STR, even without an explicit language model, can outperform previous SOTA STR methods as well as a ViT-based baseline. We conjecture that: (1) One reason might be that the three output branches (Character, BPE and WordPiece), realized via multi-task learning, can capture the features from different aspects and are highly complementary. (2) Another reason, which we think is more essential, is that \textit{\textbf{language, as a kind of compositional objects, possesses the nature of multi-granularity}} and the Multi-Granularity Prediction strategy conforms well to this nature, thus leading to higher text recognition accuracy. (3) Last but not least, the subword branches (BPE and WordPiece), working at a more macroscopic level, are much robuster to difficult cases where blur, deformation and occlusion occur, compared with the conventional character-level representation.

Regarding the fusion part, the proposed LFS actually brings in a ``look-back’’ mechanism, \ie, taking a \textit{backward} pathway to verify the matching degrees between the predicted text (words) and the input image after the \textit{forward} pass, which makes the proposed algorithm go beyond a pure discriminative model, to certain extent partially similar to the ``analysis-by-synthesis’’ paradigm~\cite{yuille2006vision,bever2010analysis}.

\subsection{Limitations and Future Works}
%Although MGP-STR can achieve promising results on challenging benchmarks with scene text and handwritten text, it still has some imperfections as follows. First, there are some fixed configurations in MGP-STR and other STR methods (\eg, ABINet~\cite{ABInet}). Specifically, the max predicted length should be pre-defined at the training and testing phrases. Since the images with extremely long text are rare in both synthetic data and real data, MGP-STR could obtain wrong results on them and can not handle the text with more than $T$ characters. Moreover, since the image is resized into a fixed shape (32 $\times$ 128), it is difficult to process the heavily cured and vertical text. Meanwhile, MGP-STR heavily resorts to pre-defined tokenization in English, and so that it can not generate multi-granularity predictions on Chinese. Second, both CFS and LFS are decision-level fusion strategies. How to effectively fuse visual features and multi-granularity textual features to further improve MGP-STR performance remains a challenge.
%It is noted that the technique of multi-granularity prediction is not restricted on the English text, and it can also be applied to the other languages with available tokenizations. MGP-STR employs parallel decoder for the trade-off of cost and accuracy, while recent STR methods~\cite{ptie,PARSeq} use autoregressive (AR) decoder for better results with more resource consumption. Naturally, MGP-STR can be equipped with AR decoders, where characters or subwords should be generated by different AR decoders for multi-granularity predictions. Moreover, STR methods with recent vision-language pre-training techniques deserve a further study.

Though introduced new strategies and achieved competitive performance, MGP-STR still has several drawbacks which need to be addressed in the future: (1) Limited by the maximum sequence length $T$ (which is set to $27$ currently), MGP-STR can only correctly read text shorter than this threshold. We will extend MGP-STR to handle longer words. (2) Even running quite fast, the model sizes of the MGP-STR series are relatively large, hindering the applicability range of MGP-STR (especially for mobile devices and embedded systems). Reducing the model parameters while keeping the recognition accuracy will be another direction for improvement. (3) Due to the adopted tokenizers and training data, MGP-STR cannot recognize text other than English and Arabic numerals. In the future, we will construct a unified text recognizer, which is able to support more types of languages (such as Chinese, Spanish, Arabic and German).

\section{Conclusion}

In this paper, we first presented a ViT-based pure vision STR model, which shows its superiority in recognition accuracy. To improve this baseline model, we then proposed a Multi-Granularity Prediction strategy to implicitly make use of linguistic knowledge. In particular, subword representations (BPE and WordPiece) are utilized to induce linguistic information in an implicit manner through multi-task learning and optimization. The proposed algorithm (named MGP-STR) with Confidence-based Fusion Strategy (CFS) is a conceptually simple yet powerful STR model without an independent language model. In addition, we propose a Learnable Fusion Strategy (LFS) to directly measure the similarities between words and images to unlock the potential of the multi-granularity predictions, further boosting the performance. MGP-STR with LFS pushes the recognition accuracy on standard datasets to a new height.

Extensive experiments are conducted to demonstrate the superiority of MGP-STR. Concretely, MGP-STR achieves state-of-the-art performances on six widely-used benchmarks (IC13~\cite{IC13}, SVT~\cite{SVT}, IIIT~\cite{IIIT}, IC15~\cite{IC15}, SVTP~\cite{SVTP} and CUTE~\cite{CUTE}) and standard handwritten text benchmarks (IAM~\cite{IAM}, CVL~\cite{CVL} and RIMES~\cite{RIMES}). Besides, when using synthetic data or real data for training, MGP-STR can also obtain state-of-the-art accuracy on recent challenging scene text datasets (ArT~\cite{ArT}, COCO-Text~\cite{COCO} and Uber-Text~\cite{Uber}). In the future, we will extend the idea of multi-granularity prediction with learnable fusion to broader domains and more tasks.

% \appendices
% \section{Proof of the First Zonklar Equation}
% Appendix one text goes here.

% % you can choose not to have a title for an appendix
% % if you want by leaving the argument blank
% \section{}
% Appendix two text goes here.


% use section* for acknowledgment
\ifCLASSOPTIONcompsoc
  % The Computer Society usually uses the plural form
  \section*{Acknowledgments}
\else
  % regular IEEE prefers the singular form
  \section*{Acknowledgment}
\fi

The authors would like to thank the reviewers for their hard work and valuable suggestions.


% Can use something like this to put references on a page
% by themselves when using endfloat and the captionsoff option.
\ifCLASSOPTIONcaptionsoff
  \newpage
\fi



\bibliographystyle{IEEEtran}
% argument is your BibTeX string definitions and bibliography database(s)
\bibliography{MGP-STR}


\begin{IEEEbiography}[{% Figure removed}]{Cheng Da}
received the B.S. degree in computer science from Sichuan University, Chengdu, China, in 2014, and the Ph.D. degree in National Laboratory of Pattern Recognition, Institute of Automation, Chinese Academy of Sciences, Beijing, China, in 2019. He is currently an algorithm researcher at Alibaba DAMO Academy, Beijing, China. His current research interests include scene text recognition and computer vision.
\end{IEEEbiography}

% if you will not have a photo at all:
\begin{IEEEbiography}
[{% Figure removed}]{Peng Wang}
received the B.S. degree from the National Key Laboratory of Science and Technology on Communications at Beijing University of Posts and Telecommunications, Beijing China, in 2016. He is currently an algorithm researcher at Alibaba DAMO Academy, Beijing, China. His current research interests include computer vision and scene text recognition.
\end{IEEEbiography}

% insert where needed to balance the two columns on the last page with
% biographies
%\newpage

\begin{IEEEbiography}[{% Figure removed}]{Cong Yao} is currently with Alibaba DAMO Academy,
Beijing, China. He received the B.S. and Ph.D. degrees in electronics and information engineering from the Huazhong University of Science and Technology (HUST), Wuhan, China, in 2008 and 2014, respectively. He was a research intern at Microsoft Research Asia (MSRA), Beijing, China, from 2011 to 2012. He was a Visiting Research Scholar with Temple University,
Philadelphia, PA, USA, in 2013. His research has focused on computer vision and deep learning, in particular, the areas of scene text reading and document understanding.
\end{IEEEbiography}

% You can push biographies down or up by placing
% a \vfill before or after them. The appropriate
% use of \vfill depends on what kind of text is
% on the last page and whether or not the columns
% are being equalized.

%\vfill

% Can be used to pull up biographies so that the bottom of the last one
% is flush with the other column.
%\enlargethispage{-5in}



% that's all folks
\end{document}


