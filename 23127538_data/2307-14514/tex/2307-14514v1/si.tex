\documentclass[aip,jcp,reprint,amsmath,amssymb,floatfix,citeautoscript,nofootinbib]{revtex4-2}
\usepackage{cancel}
\usepackage{amsmath}
\usepackage{physics}
\usepackage{graphicx}
\usepackage{amssymb}
\usepackage[english]{babel}
\usepackage{amsthm}
\usepackage{bm}
\usepackage{booktabs}
\usepackage{threeparttable}
\usepackage{dcolumn}
\newcommand{\thead}[1]{\multicolumn{1}{c}{#1}}
\newcolumntype{x}[1]{D{.}{.}{#1}}
\usepackage{ragged2e}
\usepackage{txfonts}
\allowdisplaybreaks
\usepackage{mathtools}
\usepackage{color}
\usepackage[usenames,dvipsnames]{xcolor}
\definecolor{myblue}{rgb}{0,0,1}
\usepackage[breaklinks=true,colorlinks=true,linkcolor=myblue,urlcolor=myblue,citecolor=myblue]{hyperref}
\DeclarePairedDelimiterX\pint[2]{(}{)}{#1 \delimsize\vert #2}
\newcommand{\red}[1]{{\color[rgb]{0.6,0,0}{#1}}}
\newcommand{\green}[1]{{\color[rgb]{0,0.6,0}{#1}}}
\newcommand{\blackpink}[1]{{\color[rgb]{0.8,0.25,0.8}{#1}}}

\newcommand{\vk}{{\bm{k}}}
\newcommand{\vG}{{\bm{G}}}

\newcommand{\hzy}[1]{\blackpink{[HZY: #1]}}
\newcommand{\tcb}[1]{\green{[TCB: #1]}}

\renewcommand{\thefigure}{S\arabic{figure}}
\renewcommand{\theequation}{S\arabic{equation}}
\renewcommand{\thetable}{S\arabic{table}}

\begin{document}
\title{Supporting Information for:
Can spin-component scaled MP2 achieve kJ/mol accuracy for cohesive energies of molecular crystals?
}

\author{Yu Hsuan Liang}
\affiliation{Department of Chemistry, Columbia University, New York, NY 10027 USA}
\author{Hong-Zhou Ye}
\email{hzyechem@gmail.com}
\affiliation{Department of Chemistry, Columbia University, New York, NY 10027 USA}
\author{Timothy C. Berkelbach}
\email{t.berkelbach@columbia.edu}
\affiliation{Department of Chemistry, Columbia University, New York, NY 10027 USA}


\maketitle
Detailed data can be found in the \href{https://github.com/welltemperedpaprika/x23_mp2}{GitHub repository}.


\section{$k$-point sampling mesh pairs for TDL extrapolation}
The most computationally efficient pair of $k$-point sampling meshes for extrapolation is obtained by screening a range of ``viable'' $k$-point meshes. 
We define ``viable'' meshes as those for which the number of $k$-points along each axis is roughly inversely proportional to the corresponding lattice constants of the unit cell. 
All such exploratory calculations are done with a minimal basis set, and the most efficient choice of $k$-point mesh pairs for production calculations is determined from the smallest $k$-point mesh pair with an extrapolation error smaller than 2~kJ/mol with respect to a dense $k$-point mesh pair.
An example for ethyl carbamate is shown in figure~\ref{fig:kptsamp}. 
% Figure environment removed
\section{Composite Correction}


As discussed in the main text, when the calculation at the larger $k$-point mesh with $N_{k,2}$ points is not possible in the QZ basis, we apply the composite correction in equation~(5). 
The error of this approximation can be assessed on molecular crystals for which the large calculation is possible (13 crystals in total). 
In the main text figure 2, we showed an example for ammonia including the estimated TDL/CBS cohesive energy using the composite corrected $N_{k,2}$/QZ value. 
The error in the cohesive energy due to this composite correction is shown in figure~\ref{fig:comp_corr_all} for all 13 crystals for which it is accessible.


% Figure environment removed

\newpage

\section{Basis Set Convergence}

Results reported in the main text were obtained using TZ/QZ extrapolation.
In figure~\ref{fig:basis}, we show basis set convergence for the ammonia crystal at two different $k$-point mesh sizes.
At both mesh sizes, we see that TZ/QZ extrapolation has an error of less than 0.5~kJ/mol compared to QZ/5Z extrapolation.
We performed similar analysis (i.e., using small $k$-point meshes) for all 23
molecular crystals and concluded that TZ/QZ extrapolation has a mean error of
0.9~kJ/mol compared to QZ/5Z extrapolation, justifying our use of the former in
the main text.

% Figure environment removed

\widetext

\section{MP2, SCS-MP2, SOS-MP2, SCS(MI)-MP2 Results}

Numerical values of the cohesive energies of all 23 molecular crystals at all discussed levels
of theory (plus experiment) are given in table~\ref{tab:ecoh}.

\begin{table*}[!h]
    \centering
    \label{tab:ecoh}
    \caption{Cohesive energies (kJ/mol) at TDL and CBS limit for all X23 molecular crystals.}
    \begin{tabular}{lrrrrrrr}
        \toprule
        {} &     HF &     MP2 &  SCS-MP2 &  SOS-MP2 &  SCS(MI) &  SCS(X23) &    Exp \\
        molecule             &        &         &          &          &         &          &        \\
        \midrule
        1,4-cyclohexanedione &     1.45 &   99.54 &    74.96 &    62.67 &   86.07 &    87.73 &   90.0 \\
        acetic acid          &    16.64 &   72.50 &    57.14 &    49.47 &   66.22 &    66.13 &   73.6 \\
        adamantane           & $-$52.88 &   74.10 &    73.05 &    72.52 &   25.07 &    50.64 &   71.8 \\
        ammonia              &     4.76 &   38.34 &    30.33 &    26.33 &   33.31 &    34.19 &   38.7 \\
        anthracene           & $-$66.15 &  148.84 &   142.64 &   139.54 &   70.37 &   110.30 &  110.4 \\
        benzene              & $-$24.23 &   76.57 &    59.68 &    51.23 &   54.13 &    62.21 &   54.8 \\
        carbon dioxide       &     6.88 &   27.81 &    21.92 &    18.98 &   25.59 &    25.46 &   29.4 \\
        cyanamide            &    26.94 &   91.13 &    73.67 &    64.94 &   83.74 &    83.77 &   81.5 \\
        cytosine             &    47.20 &  185.02 &   151.89 &   135.33 &  164.65 &   168.05 &  163.5 \\
        ethyl carbamate      &    16.61 &   93.05 &    74.56 &    65.32 &   81.87 &    83.67 &   88.2 \\
        formamide            &    29.78 &   82.82 &    69.48 &    62.82 &   75.58 &    76.44 &   81.1 \\
        hexamine             & $-$17.73 &   99.23 &    74.06 &    61.48 &   78.93 &    84.05 &   84.1 \\
        imidazole            &    12.87 &  107.61 &    86.19 &    75.48 &   92.21 &    95.58 &   90.4 \\
        naphthalene          & $-$42.26 &  122.42 &    86.03 &    67.83 &   94.80 &   101.29 &   81.3 \\
        oxalic acid $\alpha$ &    35.07 &  102.51 &    82.23 &    72.09 &   96.73 &    95.29 &   98.8 \\
        oxalic acid $\beta$  &    30.39 &  104.73 &    84.33 &    74.12 &   96.35 &    96.25 &   96.8 \\
        pyrazine             & $-$18.44 &   91.75 &    67.92 &    56.01 &   72.74 &    77.48 &   64.3 \\
        pyrazole             &     6.27 &   64.61 &    48.67 &    40.69 &   57.97 &    57.94 &   78.8 \\
        succinic acid        &    20.23 &  130.77 &   101.94 &    87.52 &  116.76 &   117.76 &  130.1 \\
        triazine             &  $-$3.33 &   79.67 &    60.91 &    51.53 &   66.19 &    69.14 &   62.6 \\
        trioxane             & $-$10.10 &   59.73 &    44.66 &    37.12 &   47.65 &    50.68 &   64.6 \\
        uracil               &    39.69 &  163.62 &   134.26 &   119.58 &  144.86 &   148.24 &  136.2 \\
        urea                 &    45.19 &  108.66 &    91.33 &    82.66 &  101.42 &   101.40 &  102.1 \\
        \bottomrule
        \end{tabular}        
\end{table*}

\newpage

\section{Effect of spin scaling on cohesive energy}

Let $\gamma = E_{\mathrm{os}}^{(2)} / E_{\mathrm{ss}}^{(2)}$.
The MP2 correlation energy can be written as
\begin{equation}
    E^{(2)}
        = E_{\mathrm{ss}}^{(2)} + E_{\mathrm{os}}^{(2)}
        = ( 1 + \gamma ) E_{\mathrm{ss}}^{(2)}
\end{equation}
Let $X$ denote spin scaling prescription specified by coefficients $(c_{\mathrm{ss}}^{X}, c_{\mathrm{os}}^{X})$, which are listed in table~\ref{tab:css_cos} for the three prescriptions used in this work.

\begin{table}[!h]
    \centering
    \caption{Spin scaling coefficients for different spin scaling prescriptions.}
    \label{tab:css_cos}
    \begin{tabular}{lll}
        \toprule
        {}  & $c_{\mathrm{ss}}$ & $c_{\mathrm{os}}$ \\
        \midrule
        SOS & $0$ & $1.3$   \\
        SCS & $0.333$ & $1.2$   \\
        SCS(MI) & $1.29$ & $0.4$   \\
        SCS(MC) & $0.99$ & $0.76$   \\
        \bottomrule
    \end{tabular}
\end{table}

The correlation energy after spin scaling can be written as
\begin{equation}
\begin{split}
    E^{(2),X}
        = c_{\mathrm{ss}}^{X} E_{\mathrm{ss}}^{(2)} +
        c_{\mathrm{os}}^{X} E_{\mathrm{os}}^{(2)}
        = (c_{\mathrm{ss}}^{X} + c_{\mathrm{os}}^{X} \gamma) E_{\mathrm{ss}}^{(2)}
        = \underbrace{
            \frac{c_{\mathrm{ss}}^{X} + c_{\mathrm{os}}^{X} \gamma}{1 + \gamma}
        }_{\alpha^{X}(\gamma)}
        E^{(2)}
        = \alpha^{X}(\gamma) E^{(2)}.
\end{split}
\end{equation}
The spin scaling correction to the cohesive energy reads
\begin{equation}
    \Delta E_{\mathrm{coh}}^{X}
        = E^{X}_{\mathrm{coh}} - E_{\mathrm{coh}}
        \approx \left[
            \alpha^{X}(\gamma_{\mathrm{mol}}) E_{\mathrm{mol}}^{(2)} -
            \alpha^{X}(\gamma_{\mathrm{cell}}) \frac{E_{\mathrm{cell}}^{(2)}}{N_{\mathrm{mol}}}
        \right] -
        \left[
            E_{\mathrm{mol}}^{(2)} - \frac{E_{\mathrm{cell}}^{(2)}}{N_{\mathrm{mol}}}
        \right]
\end{equation}
where we have discarded the relaxation energy whose contribution to $\Delta E_{\mathrm{coh}}^{X}$ is small.

The values of $\gamma_{\mathrm{mol}}$ and $\gamma_{\mathrm{cell}}$ for the entire X23 set are plotted in figure~\ref{fig:gdab_all}a.
We see that typically $\gamma_{\mathrm{mol}} \approx \gamma_{\mathrm{cell}} \approx 3$, motivating the introduction of the following two parameters
\begin{equation}
    \bar{\gamma}
        = \frac{\gamma_{\mathrm{mol}} + \gamma_{\mathrm{cell}}}{2},
    \qquad{}
    \delta
        = \gamma_{\mathrm{mol}} - \gamma_{\mathrm{cell}}
\end{equation}
The parameter $\delta$ is small but positive for the entire X23 set (figure~\ref{fig:gdab_all}b), which justifies a linear expansion of $\Delta E_{\mathrm{coh}}^{X}$ around $\bar{\gamma}$
\begin{equation}    \label{eq:ecoh_decomp}
    \Delta E_{\mathrm{coh}}^{X}
        \approx \underbrace{
            \left[ \alpha^{X}(\bar{\gamma}) - 1 \right] E^{(2)}_{\mathrm{coh}}
        }_{\Delta E_{\mathrm{coh},\alpha}^{X}}
        + \underbrace{
            \beta^{X}(\bar{\gamma}) \frac{E^{(2)}_{\mathrm{cell}}}{N_{\mathrm{mol}}} \delta
        }_{\Delta E_{\mathrm{coh},\beta}^{X}}
        + O(\delta^2)
        \approx \Delta E_{\mathrm{coh},\alpha}^{X} + \Delta E_{\mathrm{coh},\beta}^{X}
\end{equation}
where
\begin{equation}
    \beta^{X}(\gamma)
        = \frac{
            c_{\mathrm{ss}}^{X} - c_{\mathrm{os}}^{X}
        }{
            (1 + \gamma)^2
        }.
\end{equation}
The first term $\Delta E_{\mathrm{coh},\alpha}^{X}$ thus reflects how much the spin scaled correlation energy deviates from the bare MP2, while the second term is proportional to $(c_{\mathrm{ss}}^{X} - c_{\mathrm{os}}^{X}) \delta$.
From figure~\ref{fig:gdab_all}(c) we see that $\alpha^{\mathrm{SOS}} \approx \alpha^{\mathrm{SCS}} \approx 1$, consistent with the two methods approximately conserving the MP2 correlation energy, while $\alpha^{\mathrm{SCS(MI)}} \approx 0.6$.
As a result, $\Delta E_{\mathrm{coh},\alpha}^{X}$ vanishes for both SOS and SCS-MP2, while being a large negative value for SCS(MI)-MP2.
The second term, on the other hand follows a trend
\begin{equation}
    \Delta E_{\mathrm{coh},\beta}^{\mathrm{SOS}} >
        \Delta E_{\mathrm{coh},\beta}^{\mathrm{SCS}}
        \approx -\Delta E_{\mathrm{coh},\beta}^{\mathrm{SCS(MI)}}
        > 0
\end{equation}
where the sign change for SCS(MI)-MP2 is due to the flipped spin scaling coefficients (table~\ref{tab:css_cos}).
Both the two components, $\Delta E_{\mathrm{coh},\alpha}^{X}$ and $\Delta E_{\mathrm{coh},\beta}^{X}$, and the total spin scaling correction $\Delta E_{\mathrm{coh}}^{X}$ are plotted in figure~\ref{fig:ecoh_decomp_all}.
As already analyzed in the main text, for most systems where $\delta \approx 0.1$, the cancellation of the two terms in SCS(MI)-MP2 renders its correction the smallest among the three, while for the two outliers, adamantane and anthracene, where $\delta$ vanishes (accidentally), the correction from SOS and SCS-MP2 vanishes accordingly while that from SCS(MI)-MP2 is uncompensated and abnormally large.

% Figure environment removed

% Figure environment removed


\end{document}
