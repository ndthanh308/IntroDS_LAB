\documentclass[conference]{IEEEtran}
\IEEEoverridecommandlockouts
% The preceding line is only needed to identify funding in the first footnote. If that is unneeded, please comment it out.
\usepackage{cite}
\usepackage{amsmath,amssymb,amsfonts}
\usepackage{algorithmic}
\usepackage{graphicx}
\usepackage{textcomp}
%\usepackage{xcolor}

\usepackage{soul}

\usepackage{subfigure}
\usepackage{multirow}

%\usepackage{color}%Use this to enable colors
\usepackage[monochrome]{color} %Use this to make complete text in black and white

%\raggedbottom

%%code for supressing strikthrough
\makeatletter
\renewcommand\st[1]{\@bsphack\@esphack}%
\makeatother

\newcommand{\tabincell}[2]{
\begin{tabular}{@{}#1@{}}#2\end{tabular}
}

\def\BibTeX{{\rm B\kern-.05em{\sc i\kern-.025em b}\kern-.08em
    T\kern-.1667em\lower.7ex\hbox{E}\kern-.125emX}}
\begin{document}

\title{Dead-time Compensation Method for Bus-clamping Modulated Voltage Source Inverter\\

% {\footnotesize \textsuperscript{*}Note: Sub-titles are not captured in Xplore and should not be used}

}



\author{
    \IEEEauthorblockN{Reza Asrar Ghaderloo, \textit{Student Member, IEEE}, Yidi Shen, \textit{Student Member, IEEE}, \\
    Chanaka Singhabahu, \textit{Student Member, IEEE}, 
    Rakesh Resalayyan, \textit{Member, IEEE}\\
    and Alireza Khaligh, \textit{Senior Member, IEEE}
    }
	\IEEEauthorblockA{%\IEEEauthorrefmark{1}
		Maryland Power Electronics Laboratory, Department of Electrical and Computer Engineering \\{and The Institute for Systems Research}, University of Maryland, College Park, MD 20742 USA\\
	}	
	E-mail: rezasrar@umd.edu, yidishen@umd.edu, chanaka@umd.edu,\\ rakeshr@umd.edu, khaligh@umd.edu; URL: https://khaligh.umd.edu/	
}



\maketitle

\begin{abstract}
Bus-clamping Pulse Width Modulation (PWM) is an effective method to reduce the switching loss in a three-phase voltage source inverter (VSI). In bus-clamping PWM scheme, the phase legs are switched using high frequency PWM signals for two-third of the line cycle, while for the remaining duration of cycle, the pole voltage is clamped to either positive or negative rail of the DC bus. In PWM operation of a half bridge, a dead-time is applied between the gate signals of complementary switches to ensure safe and reliable operation. However, introduction of dead-time leads to poor power quality, increased Total Harmonic Distortion (THD) and variation in actual voltage compared to the intended pole voltage. Moreover, when the bus-clamping technique is used, the PWM has both high frequency switching region and clamped region in a line cycle, and consequently, the undesired effects of dead-time are further aggravated. Therefore, in order to enhance the quality of output voltage, this paper presents a dead-time compensation strategy for a VSI operating with bus-clamping PWM. The proposed method calculates the required compensation term to be added on the modulation signal considering wide range of operating conditions. Additionally, the compensation includes a new strategy for low current conditions near zero-crossing to avoid distortion. The proposed method is verified by simulation and experiments in a three-phase VSI with a switching frequency of 100 kHz and a fundamental frequency of 60Hz.


\end{abstract}

\begin{IEEEkeywords}
	Voltage Source Inverter, Pulse-width Modulation, Dead-time, Bus-clamping
\end{IEEEkeywords}

\section{Introduction}
\label{sec:Intro}
Two-level voltage source inverter (VSI) is a widely used topology to generate three phase sinusoidal output voltage due to its low device count, ease of modulation and well-established literature \cite{one,omid_rezaei}.
Three phase VSI consists of three half-bridge legs, with each leg being modulated with complimentary pulses with a dead time between the gate signals of the switches\cite{two,moj_salehi}. While dead-time is a necessity for VSI, it introduces undesired distortion at the output of VSI \cite{dashtaki,Ali3}.
Although the use of wide band-gap devices with fast transition speed can minimize the required dead-time\cite{Ali2}, operation at high switching frequencies still accumulates the dead-time effect and distorts the output voltage. Hence, the effects of dead-time should be investigated in the case of a practical VSI.


There are several existing methods for dead-time compensation \cite{two,dashtaki,five_four,six,seven,eight}. In \cite{two}, the dead-time effect is compensated directly by a closed-loop controller design for output waveform. Since the dead-time will affect the output by introducing odd harmonics, the feedback controller is designed to regulate the induced harmonics in \cite{dashtaki} and \cite{five_four}. The main constraint of this method is the demand for high gain and high bandwidth control. In feed-forward type dead-time compensation method, the compensation voltage is derived based on current value and direction, then added to the modulation signal\cite{six}. To eliminate the sampling delays in current sensing and to improve the dead-time compensation accuracy, predicted phase current is used in \cite{seven}, however, the precise parameters of inverter load are required in this method. 


Emergence of wide band-gap semiconductors has pushed the switching frequencies from a few kilo-Hertz to 100kHz-500kHz for VSIs with power rating of 1kW to 3kW \cite{five_four,Ali1}. Although the higher switching frequencies bring up benefits in lower converter volume and faster dynamic response, it imposes new challenges including aforementioned dead-time effect \cite{six}. Another known challenge in higher switching frequency is increased switching losses. Researchers have investigated various methods to reduce the switching loss in three phase VSI such as bus-clamping modulation \cite{eight}. The above-mentioned works have studied dead-time compensation in regular sine-wave modulated VSI, whereas the dead-time compensation in the case of bus-clamping PWM has not been well-investigated.


In this paper, a novel method is proposed to compensate the dead-time effect in bus-clamping modulated VSI based on sensed load current. Section II includes the analysis of dead time effect and calculation of compensation voltage based on load current. In Section III, the dead time compensation method is extended to bus-clamping modulation considering the sampling delay for current sensing. Subsequently in Section IV, experimental results are presented to validate the proposed method. Finally, the Section V concludes the paper.

% Figure environment removed

% Figure environment removed



\vspace{2 mm}

\section{The Analysis of Dead-Time Effect}
\label{chap:design_VSI}


The basic structure of a VSI half-bridge leg is shown in Fig. \ref{fig:topology} where an inductor represents either the output filter or an inductive load. The six operating modes of a half bridge in one switching cycle assuming positive direction for inductor current (phase current drawn out of the pole) is depicted in Fig. \ref{fig:topology}. Fig. \ref{fig:Op_wave} illustrates gating and pole voltage waveforms of the half bridge, showing the effect of dead-time and drain-to-source parasitic capacitance ($C_{ds,1}$,$C_{ds,2}$). In this analysis, $C_{ds,1}$ and $C_{ds,2}$ are assumed to be constant. In Fig. \ref{fig:Op_wave}, the $Q_{1,gate}^*$ and $Q_{2,gate}^*$ are ideal gate signals and $V_{a}^*$ is the ideal pole voltage without dead-time. While, $Q_{1,gate}$ and $Q_{2,gate}$ are the gate signals and $V_{a}$ is the actual pole voltage after applying the dead-time. $V_{error}$ is the error voltage which is added to the ideal pole voltage due to dead-time and $C_{ds}$ parasitic capacitor effects, and $t_{dT}$ is the time duration of dead-time.


% Figure environment removed

% Figure environment removed

The description for each mode of operation is given below:

$ Mode\ 1 \ (t_{0}-t_{1})$: In the first mode, the $Q_{1}$ switch is ON and the load current is flowing through its channel. In this mode, the pole voltage is equal to $V_{dc}$. The voltage across $C_{ds,1}$ is zero and voltage across $C_{ds,2}$ capacitor is $V_{dc}$. This mode ends by turning off the gate of switch $Q_{1}$ at $t_{1}$.


$ Mode\ 2 \ (t_{1}-t_{2})$: In this mode, the current in $Q_{1}$ channel is ceased and the load current is comprised of current through $C_{ds,1}$ and $C_{ds,2}$ capacitors (as shown in Fig. \ref{fig:Op_wave}). During this mode, the $C_{ds,1}$ capacitor is gradually charged to $V_{dc}$ and $C_{ds,2}$ capacitor is discharged to zero. Hence, in this interval, the pole voltage $V_{a}$ is gradually decreased from $V_{dc}$ to zero. This mode ends at $t_{2}$ when pole voltage reaches zero. It should be noted that, if the load current magnitude is larger or $C_{ds}$ capacitor value is smaller, the time interval for this mode ($t_{1}$-$t_{2}$) would be shorter. In Fig. \ref{fig:Op_wave}, the magnitude of load current is assumed to be high enough to completely charge $C_{ds,1}$  and discharge $C_{ds,2}$ before triggering $Q_{2}$.


$ Mode\ 3 \ (t_{2}-t_{3})$: At $t_{2}$, the $C_{ds,1}$ is charged to Vdc and $C_{ds,2}$ capacitor is completely discharged, thus, the load current flows through the anti-parallel diode of switch $Q_{2}$. This mode ends at $t=t_{3}$ when the $Q_{2}$ switch is triggered. It should be noted that the time interval between $t_{1}$ and $t_{3}$ is when the dead-time is applied. 

$ Mode\ 4 \ (t_{3}-t_{4})$: At $t=t_{3}$, the gate signal is applied to $Q_{2}$ switch and load current transfers from anti-parallel diode of $Q_{2}$ to its channel. This mode continues until gate signal is removed from $Q_{2}$ switch. 


$ Mode\ 5 \ (t_{4}-t_{5})$: Once the gate signal is removed from $Q_{2}$ switch, the load current flows through $Q_{2}$ anti-parallel diode once more. This mode ends at $t=t_{5}$ when the $Q_{1}$ switch is triggered. 


$ Mode\ 6 \ (t_{5}-t_{6})$: At $t=t_{5}$, the $Q_{1}$ switch turns on. Once $Q_{1}$  starts to conduct, the capacitor $C_{ds,1}$ discharges into $Q_{1}$  channel. Concurrently, current passes through $Q_{1}$ channel towards capacitor $C_{ds,2}$ to charge it to $V_{dc}$. Since the current discharging $C_{ds,1}$ and charging $C_{ds,2}$ capacitor is high in magnitude, the time interval for mode 6 is assumed to be negligible. During this mode, the $Q_{1}$ channel provides the load current, discharge current of $C_{ds,1}$ capacitor and charge current of $C_{ds,2}$ capacitor.

As shown in Fig. \ref{fig:Op_wave}, for positive current direction of inductor, there is a rectangle-shape (red shadowed) negative voltage error due to dead-time effect and a triangle-shape (blue shadowed) positive voltage error due to $C_{ds}$ capacitors. Hence, in each switching cycle, an error voltage is introduced to the intended pole voltage which results in distortion and voltage drop. 

Based on the analysis of Mode 2 operation above, since the load current is charging and discharging the $C_{ds}$ capacitors, the time duration of Mode 2, $\Delta  t_{Mode 2}$, is calculated in (\ref{eq:1}) and (\ref{eq:2}).


\begin{equation}
\label{eq:1}
\begin{aligned}
I_{L} & = 2C_{ds}\frac{V_{dc}}{\Delta t_{Mode 2} }
&
\end{aligned}
\end{equation}

\begin{equation}
\label{eq:2}
\begin{aligned}
\Delta t_{Mode 2} & =t_{2} -t_{1} & = \frac{2C_{ds}V_{dc}}{I_{L} }
&
\end{aligned}
\end{equation}

In (\ref{eq:1}) and (\ref{eq:2}),  $V_{DC}$ is the DC-link voltage and $I_{L}$ is the inductor current. Based on (\ref{eq:2}), $\Delta t_{Mode 2}$ is inversely proportional to inductor current. From the mode-wise operation analysis above, the maximum value of $\Delta t_{Mode 2}$ is equal to dead-time duration, $t_{dT}$. Hence, based on the magnitude of inductor current, three types of error voltage waveforms are obtained for positive current direction as shown in Fig. \ref{fig:pos_cur}.


In Fig. \ref{fig:pos_cur}(a), referring to mode 2 of half bridge operation analysis, the inductor current is large enough (above certain threshold current $I_{th}$) to complete charging of $C_{ds,1}$ capacitor and discharging of $C_{ds,2}$ capacitor before triggering $Q_{2}$ switch, thus, $ \Delta t_{Mode 2}$ is less than dead-time and it can be obtained from (\ref{eq:2}). In Fig. \ref{fig:pos_cur} (b), the process of charging the $C_{ds,1}$ capacitor and discharging the $C_{ds,2}$ capacitor is completed at the instant of triggering the $Q_{2}$ switch. Hence, $ \Delta t_{Mode 2}$ is equal to dead-time. Based on (\ref{eq:2}), this threshold current $I_{th}$ is calculated in (\ref{eq:3}).
\begin{equation}
\label{eq:3}
\begin{aligned}
I_{th} & = \frac{2C_{ds}V_{dc}}{ t_{dT} }
&
\end{aligned}
\end{equation}

In Fig. \ref{fig:pos_cur} (c), the magnitude of inductor current is smaller than $I_{th}$, thus, it will take longer than $t_{dT}$ to fully charge the $C_{ds,1}$ capacitor and discharge the $C_{ds,2}$ capacitor in dead-time interval. Therefore, the error voltage due to $C_{ds}$ capacitors effect has a trapezoidal shape.


With similar analysis, possible error voltage waveforms for negative direction of current are shown in Fig. \ref{fig:neg_cur} . 


Based on possible error voltage waveforms for positive and negative current directions, the average of error voltage in one switching cycle can be determined as (\ref{eq:4}). 
\begin{equation}
\label{eq:4}
  \overline{V_{error}}=\begin{cases}
  \vspace{3mm}
  -\frac{V_{DC}t_{dT}}{ T_{s} }+\frac{C_{ds}V_{DC}^2}{ I_{L}T_{s} } , & \text{ $I_{th}\le I_{L}$ }\\
    \vspace{3mm}
    -\frac{t_{dT}^2I_{L}}{ 4C_{ds}T_{s} }, & \text{$ -I_{th}< I_{L} < I_{th} $}\\
    \vspace{3mm}
    \frac{V_{DC}t_{dT}}{ T_{s} } - \frac{C_{ds}V_{DC}^2}{ I_{L}T_{s} } , & \text{ $I_{L}\le I_{th}$ }
  \end{cases}
\end{equation}

According to (\ref{eq:4}), the average of error voltage ($\overline{V_{error}} $) due to dead-time and $C_{ds}$ capacitors effects, which is added to ideal pole voltage is plotted in Fig. \ref{fig:error}. 


% Figure environment removed


It is evident from Fig. \ref{fig:error} that $\overline{V_{error}}$ is smooth in the area of current zero-crossing point, thus, the compensation of error will not result in distortion by sudden changes in voltage. However, if the $C_{ds}$ capacitor effect is not considered, sudden changes near current zero-crossing can exist, as in some of the conventional methods \cite{seven}.


Assuming a sinusoidal switching cycle average pole voltage with lagging power factor, the ideal and actual switching cycle average pole voltage waveforms including dead-time and $C_{ds}$ capacitors effects are shown in Fig. \ref{fig:sine_compensate}.

% Figure environment removed


According to Fig. \ref{fig:sine_compensate}, in positive half cycle of inductor current, the error voltage is negative, whereas, during the negative half cycle of inductor current, the error voltage is positive.



\section{Dead-Time Compensation Method for Bus-Clamping PWM}


In bus-clamping PWM, each leg will stop switching for one-third of the line cycle\cite{seven}, hence, during this clamped interval, the error voltage due to dead-time and $C_{ds}$ capacitors effects is zero. Fig. \ref{fig:bus_compensate} illustrates the $\overline{V_{error}}$ for a VSI half bridge leg which is controlled with bus-clamping modulation.

% Figure environment removed


To compensate the effects of dead-time and $C_{ds}$ capacitors, a signal corresponding to $-\overline{V_{error}}$ is added to the bus-clamping modulation signal. It can be concluded from Fig. \ref{fig:bus_compensate} that the accuracy of compensation is highly dependent on inductor current direction. However, in practical implementations, there is a propagation delay ($T_{delay}$) between the sampled and the actual current which leads to reduced accuracy of compensation approach. As shown in (\ref{eq:5}), this can be addressed with simple current prediction method by adding the $\omega T_{delay}$ to the phase of sampled current \cite{seven}. 


\begin{equation}
\label{eq:5}
\begin{cases}
  \vspace{3mm}
    i_{a}^{'} = I_{P} \sin(\theta_{i}+\omega T_{delay}) \\
    \vspace{3mm}
    i_{b}^{'} = I_{P} \sin(\theta_{i}-120^{\circ}+\omega T_{delay}) \\
    \vspace{3mm}
    i_{c}^{'} = I_{P} \sin(\theta_{i}+120^{\circ}+\omega T_{delay}) 
  \end{cases}
\end{equation}

In (\ref{eq:5}), $I_{P}$ is the magnitude, $\theta _{i}$ is the phase and $ \omega  $ is the angular frequency of current which can be calculated according to sampled three-phase current. The $ i_{a}^ {'} $, $ i_{b}^ {'}$ and $ i_{c} ^ {'}$ are the predicted values of current. The $ T_{delay} $ can be estimated based on current sensing circuit.


\begin{table}[h]
    \caption{The characteristics of simulated VSI}
    \label{table 1}
   \footnotesize
\centering
    \renewcommand{\arraystretch}{1.6}	
    \begin{tabular}{cc}
        \hline\hline
       \textbf{Design Variable} & \textbf{Value}   \\
        \hline
          DC-link voltage     & $ V_{DC}=400V $  \\
        \hline
        Switching frequency  & 100 kHz \\
        \hline
        Dead-time  & $t_{dT}=200 ns$ \\
        \hline
        Switch $C_{ds}$ capacitor & 100 $pF$ \\
        \hline\hline
    \end{tabular}
\end{table}

The proposed method is simulated for a three phase VSI controlled with bus-clamping modulation. The specifications of the VSI are tabulated in Table \ref{table 1}. 
The simulated switching cycle average of the pole voltage with and without dead-time is shown in Fig. \ref{fig:pole_uncomp}. The voltage distortion is evident from  Fig. \ref{fig:pole_uncomp} reflecting the importance of compensating the dead-time and $C_{ds}$ capacitors effects. 

% Figure environment removed

% Figure environment removed


% Figure environment removed


In Fig. \ref{fig:comp_signals}, phase current, error voltage $\overline{V_{error}}$ for sine PWM and bus-clamping PWM, and bus-clamping modulation signal are shown. The $\overline{V_{error}}$ for bus-clamping PWM is set to zero when the modulation signal is clamped to +1 or -1 and is calculated based on (\ref{eq:4}) when not clamped. The proposed method of compensating the dead-time and $C_{ds}$ capacitors effects is activated by adding $-\overline{V_{error}}/V_{DC}$ to the modulation signal. The resultant average of pole voltage is shown in  Fig. \ref{fig:pole_comp}, where the actual voltage matches the intended voltage by activation of the proposed compensation.


\section{Experimental Results}

To verify the operation of proposed method in compensating the effects of dead-time and $C_{ds}$ capacitors, a three-phase VSI hardware prototype is built, as shown in Fig. \ref{fig:hard}. The key parameters of prototype circuit are tabulated in Table \ref{table 2}. To highlight the effect of dead-time on output voltage, the dead-time is set to 300ns.

% Figure environment removed

\begin{table}[t!]
    \caption{The parameters of prototype VSI circuit}
    \label{table 2}
   \footnotesize
\centering
    \renewcommand{\arraystretch}{1.6}	
    \begin{tabular}{cc}
        \hline\hline
       \textbf{Design Variable} & \textbf{Value}   \\
        \hline
          DC-link voltage     & $ V_{DC}=350V $  \\
        \hline
        Output phase voltage  & $ V_{ph}=100Vrms $ \\
        \hline
        Output line-to-line voltage  & $ V_{ph}=173Vrms $ \\
        \hline
        Switching frequency  & $f_{sw}=100 kHz$ \\
        \hline
        Output line frequency  & $f_{line}=60 Hz$ \\
           \hline
        Modulation index & $m=0.808$ \\
        \hline
        Dead-time  & $t_{dT}=300 ns$ \\
        \hline
        Switch $C_{ds}$ capacitor & $C_{ds}=68 pF$ \\
        \hline
        Switch part number & UJ4C075033K4S \\
        \hline\hline
    \end{tabular}
\end{table}
Based on commanded bus-clamping modulation index of m=0.808 to VSI, the desired output line-to-line voltage is expected to be 173Vrms ($V_{ph}$=100 Vrms) as shown in (\ref{eq:6}). However, as shown in Fig. \ref{fig:dropped}(a) , the output voltage is dropped to 167Vrms due to the effects of dead-time and $C_{ds}$ capacitors.
\begin{equation}
\label{eq:6}
\begin{aligned}
V_{ph,rms} & = \frac{mV_{dc}}{ 2\sqrt{2} }& =\frac{0.808\times{350}}{2\sqrt{2}} & = 100V
&
\end{aligned}
\end{equation}

The resulted voltage drop matches the voltage drop which is obtained from (\ref{eq:4}). Similar to Fig. \ref{fig:bus_compensate}, a compensating signal which corresponds to error voltage is generated and added to bus-clamping modulation signal in order to effectively compensate for the voltage drop.

% Figure environment removed

 To have accurate compensation strategy, the sampling delay of current sensing circuit is incorporated in dead-time compensation. The resulted output voltage after activation of dead-time compensation is illustrated in Fig. \ref{fig:dropped} (b) which reflects 173 Vrms line-to-line voltage that is matching the desired output voltage. Hence, the dead-time effect is accurately estimated using (\ref{eq:4}) and compensated according to Fig. \ref{fig:bus_compensate}.  
 
\section{Conclusion}

This paper presented a method to compensate the error between actual and desired output voltage due to the effect of dead-time in bus-clamping PWM modulated VSI circuit. As well as dead-time, the effect of drain-to-source capacitors $C_{ds}$ of switches are considered in this analysis. First, the operation principle for switching transition of half bridge is analyzed to indicate the error between actual and intended output voltage which is introduced due to dead-time effect under different current directions. Subsequently, the error voltage is analyzed for a VSI which is modulated with bus-clamping technique to reduce the switching loss. The simulation results are provided to prove the effects of dead-time and $C_{ds}$ capacitors on the output voltage drop. In order to verify the operation of compensation method experimentally, a three-phase VSI prototype is implemented. Based on experimental results, the presence dead-time and parasitic $C_{ds}$ capacitors cause  drop in desired output voltage RMS value. By activating the dead-time compensation method, the voltage drop is compensated and the output voltage matches the desired voltage for the applied modulation index value.

\vskip 3 mm

\bibliographystyle{./Biblio/IEEE_tran}
\bibliography{Biblio/Mybiblio}

\end{document}
