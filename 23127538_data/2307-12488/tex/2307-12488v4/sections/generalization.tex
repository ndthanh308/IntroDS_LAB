
\vspace{-0.5em}\subsection{Generalization Ability}

We have tried the ChatGPT to perform code semantic anlaysis, vulnerability analysis, and input generation.
It possesses the capability to generate syntactically correct code and offer valuable insights and suggestions in certain scenarios. However, \ding{189} the extent to which its generated content relies solely on the training dataset remains uncertain, considering the massive amount of data it was trained on.
Generalization ability is a measurement of DL model's performance on unseen data samples.
To inveagate ChatGPT's generalization capabilities, we proceed to test it on diverse set of unseen code snippets in the subsequent section.

\label{sec:gene}


% In this section, we aim to investigate whether ChatGPT can effectively analyze and comprehend code that was not included in its training dataset.
%\fixme{(how can we ensure leetcode is not used in the dataset?) ZL: Leecode is in ChatGPT's training set, the obfuscated program is not.}.
To evaluate this, we randomly select several code snippets from LeetCode~\footnote{The source code is available at: \url{https://github.com/Moirai7/ChatGPTforSourceCode}}, representing varying levels of difficulty, and assess ChatGPT's performance on the original code snippets and the corresponding obfuscated code snippets.
% Table \ref{table:cases} presents the LeetCode problems we choose and the corresponding solutions, which will serve as the dataset for our evaluation.
% By analyzing ChatGPT's responses to these unseen code snippets, we seek to gain insights into its generalization capabilities and understand how effectively it can apply its learned knowledge to new code scenarios.
% Our findings shed light on the robustness of ChatGPT's source code analysis ability and provide valuable information for its potential applications in real-world scenarios beyond the confines of its training data.


Our experiments show that ChatGPT can correctly understand the original code snippets.
Taking the Supper Egg Drop problem ({\bf \ref{appendix:Q14}}) as an example, it explains the purpose of the code snippet, the logic and complexity of the algorithm, and provides suggestions to improve the code. 
Then, we manually obfuscated the code snippets so that the code is unseen in its trainingset.
% by renaming the variables and function names and insert dummy code that will not have an impact on its results.
The obfuscated code is shown in~\autoref{code:obfuscated}.
ChatGPT can not provide useful review of the obfuscated code ({\bf \ref{appendix:Q15}}).

In essence, ChatGPT learns the literal features of code snippets that exist within its training data and generates responses based on statistical patterns involving characters, keywords, and syntactic structures.
This knowledge encompasses various aspects like indentation, syntax, and common coding conventions.
However, it lacks a profound comprehension of code semantics, logic, or context beyond the patterns and examples present in its training data.
This limitation means that ChatGPT may not possess the ability to precisely review or assess the correctness, efficiency, or security of code without a more comprehensive understanding of its underlying semantics.
