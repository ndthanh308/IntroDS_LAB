% \section{Background}
\section{Code Analysis}
\label{sec:back}
In the realm of cybersecurity, certain critical problem-solving processes, such as reverse engineering, vulnerability analysis, and bug fixing, still rely on heavy manual effort.
% For instance, when reverse engineering one malware, engineers often employ numerous tools to aid their investigations.
In response to this challenge, there is a strong industry drive to leverage AI and deep learning technologies to reduce manual effort and boost cybersecurity practices.
The advent of deep learning opens up exciting possibilities, enabling accurate detection and prediction of cyber threats, improved agility, cost reduction, enhanced job performance for human analysts, and increased levels of automation for security teams~\cite{CLS20,aisecurity,wang2021spotting,wang2021identifying,Xin2018,liu2018alphadiff,pewny2014leveraging,zou2021deep,Tobiyama2016,Raff2017,Ispoglou2016}.

When it comes to security-oriented program analysis, deep learning offers several advantages over traditional rule-based methods.
% \textcolor{red}{(what are the traditional methods?)}  
Traditionally, analysts need to firstly analyze the code manually, 
summary the patterns of code into heuristic rules, and then develop an automatic tools to solve the analytic tasks.
Firstly, deep learning requires less domain-specific knowledge, making it more accessible to a broader range of researchers and practitioners. Secondly, the representations learned by a deep learning model can be applied to various downstream tasks, providing versatility and efficiency in the analysis process. 
Researchers have explored the applications of deep learning in program analysis, categorizing them into two main groups:

\noindent{\bf Deep learning for source code anlaysis.}
As software inevitably contains bugs, security researchers seek to identify and fix these bugs to preclude potential attackers from exploiting them. Source code level analysis and protection have emerged as one of the most popular approaches to achieving these objectives, primarily because of the rich semantic information from the source code.
Source code analysis finds diverse applications in security, encompassing vulnerability discovery, analysis, fixing, and the implementation of security strategies.
Leveraging the comprehensive semantic details encoded in the source code, security professionals can efficiently identify potential weaknesses and take appropriate measures to enhance the software's resilience against threats.



\noindent{\bf Deep learning for binary code anlaysis.} 
% \textcolor{red}{(Binary code vs Assembly code? - Binary code)}
The primary purpose of security-oriented binary code analysis is to extract semantic information from binary files, enabling the resolution of security issues such as malware analysis.
Given that many commercial and malicious software do not provide access to their source code, binary code analysis becomes crucial in the realm of cybersecurity.
For instance, when securing a binary-only program with security strategies like Control Flow Integrity (CFI)~\cite{abadi2009control}, binary code analysis is required to unveil the hierarchical structures of the program, including code sections, functions, and basic blocks.

Binary code analysis finds various applications in cybersecurity. For instance, in the analysis of ransomware and the quest for cryptographic keys, analysts must first generate flow graphs of potentially obfuscated program. Previous research has utilized Deep Learning (DL) methods for a variety of binary analysis tasks~\cite{chua2017neural,shin2015recognizing,LQY21,wang2021spotting,wang2021identifying}.
The primary focus of these works is to learn effective embeddings from binary instructions or raw bytes, subsequently employing a classification output layer to predict the label for the target task.
By leveraging DL techniques, researchers aim to improve the accuracy and efficiency of binary code analysis, enabling better detection and understanding of security threats hidden within binary-only programs.

