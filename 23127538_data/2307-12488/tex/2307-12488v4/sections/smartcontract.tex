\vspace{-0.5em}\subsection{Security Analysis and Audit of Smart Contract}
\label{sec:smart}

In this section, we endeavor to cover not only traditional program-level/implementation-level bugs but also logical vulnerabilities (cross-function/cross-contract vulnerability).
Therefore, we aim to investigate \ding{188} the capabilities of ChatGPT to analyze and audit smart contracts, which are commonly deployed in the Web3 ecosystem. 
% Totally, three parts of analysis are included: 
% 1) analysis of typical and classical smart contracts vulnerable cases; 2) analysis of most recent smart contract vulnerable cases; 3) analysis of vulnerable cross-function/cross-contract programs.

First, we tested different types of vulnerabilities (({\bf \ref{appendix:Q8}})) that happened before ChatGPT launched so that they could potentially manifest within the ChatGPT system (\ie, seen programs).
According to the answer, ChatGPT adeptly identifies potential vulnerabilities and offers appropriate patches. Furthermore, after we select a full list of smart contracts \cite{Crytic,runtimeverified} across various categories including reentrancy attack and variable shadowing, ChatGPT consistently discerns vulnerabilities and provides accurate solutions.

Secondly, we conducted analyses on a selection of recently emerged vulnerable smart contracts (({\bf \ref{appendix:Q9}})), occurring within the temporal scope of 2023, to evaluate their efficacy in terms of generalization. This examination was predicated upon a real-world incursion targeting Midas Capital (as detailed in \cite{midascapital}) in June 2023.
The actual root cause of this attack is due to round issues, however, based on the answer that ChatGPT gives us, it is not aware of the risk in function \texttt{redeemFresh}.
We also test several recent attack cases \cite{Web3Sec}, and ChatGPT can correctly identify none of them.

Thirdly, we additionally assess the efficacy of ChatGPT in the realm of cross-contract vulnerability detection (({\bf \ref{appendix:Q10}})). 
To this end, we present a real-world attack scenario exemplified by an incident occurring within the Euler Finance ecosystem \cite{eulerdefi}.
According to the findings of the analysis, ChatGPT is unable to accurately discern vulnerabilities, opting instead to provide seemingly correct responses arbitrarily, devoid of substantive remedial recommendations. Consequently, it is still difficult for chatGPT to grapple with the intricacies in the analysis of cross-contract vulnerabilities.