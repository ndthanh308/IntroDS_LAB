\documentclass[10pt,conference,compsocconf,romanappendices]{IEEEtran}% 
% \documentclass[a4paper,12pt]{article} 
\usepackage{times}
% \usepackage{geometry}
% \geometry{portrait, margin=1in}
\usepackage{authblk}
\usepackage{mathrsfs}
\usepackage{amsmath,amsthm,amssymb,amsfonts,latexsym}
%\let\Bbbk\relax
\usepackage{tablefootnote}
\usepackage[numbers]{natbib}
\let\Bbbk\relax
\usepackage{algorithmic,algorithm}
\usepackage{adjustbox}
\usepackage{multirow}
% \usepackage{pgfplots}
\usepackage{threeparttable}
\usepackage{array}
\usepackage{graphicx,color}
\usepackage{listings}
\usepackage[labelfont=bf,font=small,skip=5pt]{caption}
\usepackage{pifont}
\usepackage{url}
\usepackage{array}
\usepackage{float}
\usepackage{framed}
\usepackage{soul}
\usepackage[table]{xcolor}

% \usepackage{pgf}
\usepackage{fancyvrb}
\let\labelindent\relax
\usepackage{enumitem}
\usepackage{mathtools}
\usepackage{cuted}
\usepackage{stackengine}
\usepackage{booktabs}
\usepackage{diagbox}
\usepackage{tikz}
\usetikzlibrary{positioning}
\usetikzlibrary{decorations.pathreplacing}
\usetikzlibrary{patterns}
\usepackage{xspace}
\usepackage[utf8]{inputenc}
\usepackage{pifont}% http://ctan.org/pkg/pifont
\usepackage{sidecap}
\usepackage{subcaption}
\usepackage{fancybox}
\usepackage{balance}
\usepackage{hyperref}
\newcommand*\circled[1]{\tikz[baseline=(char.base)]{
            \node[shape=circle,draw,inner sep=0.3pt] (char) {#1};}}

\usepackage{xparse}
\usepackage{lipsum}

%% Adjust text width to suit
\tikzset{chatstyle/.style={text width=3.2in,rounded corners=2pt}}

\definecolor{mygreen}{HTML}{5fedb7}
\definecolor{lightgray}{HTML}{b6b8b7}
\definecolor{shadecolor}{gray}{0.9}

% %% |=====8><-----| %% New solution:


\pagestyle{plain}
%\newcommand{\mypara}[1]{\vspace{.3em}\noindent\textbf{{#1. }}}
\newcommand{\qpara}[1]{\vspace{.5em}\noindent\textbf{\textit{#1 }}}

\usepackage{xstring}
\newcommand{\mypara}[1]{
\vspace{.3em}
\noindent{\bf \IfEndWith{#1}{.}{#1}{\IfEndWith{#1}{?}{#1}{#1.}}}
}

\newcommand{\xdashrightarrow}[2][]{\ext@arrow 0359\rightarrowfill@@{#1}{#2}}
\newcounter{question}[section]
\setcounter{question}{0}
\newenvironment{question}[1][]
{\refstepcounter{question}\par
\textbf{\cc{Q\thequestion.#1}} \rmfamily}

\newcounter{mydefinition}[section]
\setcounter{mydefinition}{0}
\newenvironment{mydefinition}[1][]
{\refstepcounter{mydefinition}\par\medskip
   \textit{Definition~\themydefinition. #1} \rmfamily}{\medskip}

\newcommand{\etal}{\textit{et al}.\xspace}
\newcommand{\ie}{\textit{i}.\textit{e}.}
\newcommand{\eg}{\textit{e}.\textit{g}.}



\newcommand{\fixme}[1]{\textcolor{red}{#1}}
\newcommand{\zl}[1]{{\color{green} Zhilong:} {\color{purple} #1}}
\newcommand{\replace}[2]{\st{#1}$\Rightarrow${\color{blue} #2}}

\newcommand{\redu}[1]{{#1}} % redundancy sentence.

\include{solidity} % for solidity highlight

\renewcommand{\sectionautorefname}{Section}
\renewcommand{\subsectionautorefname}{Section}
\renewcommand{\subsubsectionautorefname}{Section}

\setlength{\textfloatsep}{10pt}     % distance between floats on the top or the bottom and the text
\setlength{\floatsep}{10pt}         % distance between two floats
\setlength{\dbltextfloatsep}{10pt}  % distance between a float spanning both columns and the text
\setlength{\dblfloatsep}{10pt}      % distance between two floats spanning both columns
\definecolor{bittersweet}{rgb}{1.0, 0.44, 0.37}
\definecolor{bleudefrance}{rgb}{0.19, 0.55, 0.91}

\makeatletter
\newcommand{\setword}[2]{%
  \phantomsection
  #1\def\@currentlabel{\unexpanded{#1}}\label{#2}%
}

\usepackage{relsize}

\renewcommand{\ttdefault}{pxtt}

\newcommand{\cc}[1]{\mbox{\smaller[0.5]\texttt{#1}}}

\newcommand{\kenali}{\mbox{\textsc{Kenali}}\xspace}
\newcommand{\xmp}{\mbox{\textsc{xMP}}\xspace}
\newcommand{\dynpta}{\mbox{\textsc{DynPTA}}\xspace}

\definecolor{dkgreen}{rgb}{0,0.6,0}
\definecolor{gray}{rgb}{0.5,0.5,0.5}
\definecolor{mauve}{rgb}{0.58,0,0.82}
\definecolor{mygray}{gray}{0.9}
\colorlet{lightblue}{blue!70}
\colorlet{lightred}{red!70}

\lstset{
  language=C,
  frame=tb,
  basicstyle={\scriptsize \ttfamily}, %\scriptsize,\ttfamily,%
  tabsize=3,
  breaklines=true,
  % breakatwhitespace=false,
  showstringspaces=false,
  % columns=fullflexible,
  numbers=left,
  numbersep=-8pt,                     % where to put the line-numbers
  numberstyle=\tiny\color{darkgray},
  escapeinside={(*}{*)},
  xleftmargin=2pt,
  stringstyle=\color{mauve},
  keywordstyle=\color{blue},
  commentstyle=\color{dkgreen} \textit,%\scriptsize \textit,
  %directivestyle={\color{black}},
  %emph={int,char,double,float,unsigned, static, const, if, return, goto},
  emphstyle={\color{red}},
}

\newcommand{\squishlist}{
\begin{itemize}[noitemsep,nolistsep]
  \setlength{\itemsep}{-0pt}
}
\newcommand{\squishend}{
  \end{itemize}
}

\usepackage{url}
\def\UrlBreaks{\do\A\do\B\do\C\do\D\do\E\do\F\do\G\do\H\do\I\do\J
\do\K\do\L\do\M\do\N\do\O\do\P\do\Q\do\R\do\S\do\T\do\U\do\V
\do\W\do\X\do\Y\do\Z\do\[\do\\\do\]\do\^\do\_\do\`\do\a\do\b
\do\c\do\d\do\e\do\f\do\g\do\h\do\i\do\j\do\k\do\l\do\m\do\n
\do\o\do\p\do\q\do\r\do\s\do\t\do\u\do\v\do\w\do\x\do\y\do\z
\do\.\do\@\do\\\do\/\do\!\do\_\do\|\do\;\do\>\do\]\do\)\do\,
\do\?\do\'\do+\do\=\do\#}

\theoremstyle{definition}
\newtheorem{definition}{Definition}

\theoremstyle{definition}
\newtheorem*{explanation}{Explanation}

\theoremstyle{definition}
\newtheorem*{problem}{Problem}

\theoremstyle{definition}
\newtheorem{challenge}{Challenge}

\theoremstyle{definition}
\newtheorem{insight}{Insight}


\date{}
% \title{ChatGPT for Security: A Case Study of ChatGPT's Potentials and Limitations in Security Applications} 
% \title{Exploring the Security-Oriented Program Analysis Capabilities of ChatGPT: A Case Study of ChatGPT for Security}
\title{\LARGE A Case Study of Large Language Models (ChatGPT and CodeBERT) for Security-Oriented Code Analysis}
\author[1]{Zhilong Wang\thanks{Corresponding author:~zzw169@psu.edu}}
\author[1]{Lan Zhang}
\author[1]{Chen Cao}
\author[1]{Nanqing Luo}
\author[1]{Peng Liu}
\affil[1]{The Pennsylvania State University, United States}

% \providecommand{\keywords}[1]
% {
%   \small	
%   \textbf{\textit{Keywords---}} #1
% }


\begin{document}

\maketitle

\begin{abstract}

The Fast Reciprocal Square Root Algorithm is a well-established approximation technique consisting of two stages: first, a coarse approximation is obtained by manipulating the bit pattern of the floating point argument using integer instructions, and second, the coarse result is refined through one or more steps, traditionally using Newtonian iteration but alternatively using improved expressions with carefully chosen numerical constants found by other authors. The algorithm was widely used before microprocessors carried built-in hardware support for computing reciprocal square roots. At the time of writing, however, there is in general no hardware acceleration for computing other fixed fractional powers. This paper generalises the algorithm to cater to all rational powers, and to support any polynomial degree(s) in the refinement step(s), and under the assumption of unlimited floating point precision provides a procedure which automatically constructs provably optimal constants in all of these cases. It is also shown that, under certain assumptions, the use of monic refinement polynomials yields results which are much better placed with respect to the cost/accuracy tradeoff than those obtained using general polynomials. Further extensions are also analysed, and several new best approximations are given.

\end{abstract}


% \begin{IEEEkeywords}
% encryption detection, ransomware, loop detection, k complexity, data flow analysis, graph neural network
% \end{IEEEkeywords}
% no keywords
% \keywords{large language model, ChatGPT, CodeBERT, code analysis, security analysis}
\section{Introduction}
Current quantum hardware is unable to carry out universal quantum computations due to the buildup of errors that occur during the computation. 
The magnitude of the individual error is currently above the value that the Threshold Theorem requires in order to kick-start quantum error correction and fault-tolerant quantum computation~\cite[Section 10.6]{nielsen_chuang_2010}. 
Although the experimentally achieved fidelity rates are promising and the error bounds are inching closer to the required threshold, we will have to work for the foreseeable future with quantum hardware with errors that build-up during the computation.  This implies that we can only do a limited number of steps before the output of the computation has become completely uncorrelated with the intended one.

For fault-tolerant quantum computing, we repeat four steps: 
1) We apply a number of single and two-qubit quantum gates, in parallel whenever possible; 
2) We perform a syndrome measurement on a subset of the qubits; 
3) We perform fast classical computations to determine which errors have occurred and how to correct them; 
and, 4) We apply correction terms based on the classical computations.
We then repeat these four steps with a next sequence of gates. 
These four steps are essential to fault-tolerant quantum computing. 


The starting point of this work is to use the four steps outlined above, not to carry out error correction and fault-tolerant computation, but to enhance short, constant-depth, {\em uncorrected} quantum circuits that perform single qubit gates and {\em nearest-neighbor} two qubit gates. 
Since in the long run we will have to implement error-correction and fault-tolerant computation anyhow, and this is done by such a four-step process, why not make other use of this architecture? Moreover, on some of the quantum hardware platforms, these operations are already in place.
Embracing this idea we naturally arrive at the question: what is the computational power of \textit{low-depth} quantum-classical circuits organized as in the four steps outlined above? 
We thus investigate circuits that execute a small, ideally constant, number of stages, where at each stage we may apply, in parallel, single qubit gates and {\em nearest-neighbor} two qubit gates, followed by measurements, followed by low-depth classical computations of which the outcome can control quantum gates in later stages. 
It is not clear, at first, whether such circuits, especially with constant depth, can do anything remotely useful. 
But we will see that this is indeed the case: many quantum computations can be done by such circuits in constant depth. 
By parallelizing quantum computations in this way, we improve the overall computational capabilities of these circuits, as we do not incur errors on qubits that are idle, simply because qubits are not idle for a very long time. 
Furthermore, reducing the depth of quantum circuits, at the cost of increasing width, allows the circuit to be run faster even if errors occur.

The first usage of such a four-step layout, not to do error correction, but to perform computations, can be found in the paradigm of measurement-based quantum computing~\cite{gottesman1999demonstrating,raussendorf2001one,jozsa2006introduction,clark2007generalised}: 
A universal form of quantum computing where a quantum state is prepared and operations are performed by measuring qubits in different bases, depending on previous measurements and intermediate measurements.

\citeauthor{PhamSvore2013} were the first to formalize the four-step protocol for performing computations~\cite{PhamSvore2013}. They included specific hardware topologies by considering two-dimensional graphs for imposing constraints on qubit interactions. In their model, they develop circuits for particularly useful multi-qubit gates, including specifying costs in the width, number of qubits, depth, number of concurrent time steps, size, and total number of non-Identity operations.
As a result, they find an algorithm that factors integers in polylogarithmic depth.
\citeauthor{Browne:2011} showed that the main tool in the work by \citeauthor{PhamSvore2013}, the fan-out gate, can also be replaced by additional log-depth classical computations in the measurement-based quantum computing setting~\cite{Browne:2011}.

More recently, \citeauthor{Cirac:2021} introduced a scheme to implement unitary operations involving quantum circuits combined with Local Operations and Classical Communication ($\mathsf{LOCC}$) channels: $\mathsf{LOCC}$-assisted quantum circuits~\cite{Cirac:2021}. Similarly to the four-step scheme we just described, they allow for a short depth circuit to be run on the qubits, followed by one round of $\mathsf{LOCC}$, in which ancilla qubits are measured and local unitaries are applied based on the measurement outcomes. They show that in this model any 1D transitionally invariant matrix-product state (MPS) with fixed bond dimension is in the same phase of matter as the trivial state. Similar ideas can be found in~\cite{TVV_NonAbelianTopologicalOrder_2022, tantivasadakarn2021long}.

In this work, we introduce a new model, called \textit{Local Alternating Quantum-Classical Computations} ($\LAQCC$). In this model we alternate between running quantum circuits (constrained by locality), ending in the measurement of a subset of qubits, and fast classical computations based on the measurement results. The outcome of the classical computations are then used to control future quantum circuits. We allow for flexibility in this model, by giving different constraints to the power of both the quantum circuits and the classical circuits as well as the number of alternations between them. 
Most attention will be given to $\LAQCC$ containing quantum circuits of constant depth, classical circuits of logarithmic depth and at most a constant number of alternations between them. 
Any circuit constructed in this model is considered to be of constant depth. 
We restrict ourselves to logarithmic depth classical computations, as this is the first natural and non-trivial extension beyond constant-depth classical computations. 
Constant-depth classical computations do however also have an equivalent constant-depth quantum implementation.

The definition of $\LAQCC$ sharpens the original definition of \citeauthor{PhamSvore2013} by adding constraints to the intermediate classical computations. This allows us to bound the power of $\LAQCC$ from above. 

The main result of \citeauthor{Cirac:2021}, that 1D translational invariant MPS with fixed bond dimension can be prepared by $\mathsf{LOCC}$-assisted circuits, relies on local symmetries of the MPS. These symmetries allow them to prepare local states (on a constant number of qubits) and glue them together by doing one round of the appropriate entangling measurement and corrections, after which they run a round of local unitaries to get the desired result. This general scheme for preparing states that exhibit an MPS description with the appropriate local symmetries requires only geometrically local unitaries and one round of measurement and corrections an therefore is accessible in $\LAQCC$. Studying different local symmetries, known as Symmetry Protected Topological (SPT) phases of matter, to find measurement-based constant depth circuits for states is a broad ongoing field of research~\cite{TVV_NonAbelianTopologicalOrder_2022, tantivasadakarn2021long, smith2023deterministic}. 
All these schemes have a $\LAQCC$ implementation.

%$\LAQCC$-circuits also exist for general schemes of preparing local states, based on the local tensors, and gluing them together using one round of entangled measurement and corrections, based on the local symmetry. 
%The main result of \citeauthor{Cirac:2021}, that 1D translational invariant MPS with fixed bond dimension can be prepared by $\mathsf{LOCC}$-assisted circuits, relies heavily on local symmetries of the MPS and as a result also has an equivalent $\LAQCC$ implementation. 
%The corrections applied after the measurement round are local unitaries depending on the local symmetries of the MPS. 

 

%This general scheme of preparing local states, based on the local tensors, and gluing it together by doing one round of entangled measurement and corrections, based on the local symmetry, is accessible in $\LAQCC$.
Note however that \citeauthor{Cirac:2021} also suggest a circuit for the $W$-state.
This circuit uses sequentially and dependent measurement-based corrections of the ancilla qubits. 
These dependent measurements translate to sequential alternations between the quantum and classical circuits and therefore increase the total depth to linear depth, exceeding the constant-depth constraints imposed by $\LAQCC$-circuits. 

We study the power of the $\LAQCC$ model with respect to state preparation, showing that even with only constant quantum-depth and logarithmic classical depth it remains possible to prepare states with long-range entanglement.
Another surprising result is that it is unlikely that $\LAQCC$ circuits are classically simulatable. We show that any instantaneous quantum polynomial-time (IQP) circuit~\cite{Bremner2010,Shepherd2009} has an $\LAQCC$ implementation.
Classical simulation of IQP circuits implies the collapse of the polynomial hierarchy to the third level, which is not believed to be true~\cite{Bremner2017}. Therefore, we expect that $\LAQCC$ circuits are unlikely to be classically simulatable. We bound the power of $\LAQCC$ by showing that it is contained in $\QNC^1$, the class of polynomial-size, log-depth circuits.

Next, we also study the power that intermediate classical calculations can add to quantum computations, by considering a new model that alternates between polynomially many polynomial-depth quantum circuits and unbounded classical computations
We study this model by doing a complexity theoretical analysis, where we draw inspiration from the notions of complexity given by \citeauthor{RosenthalYuen:2022}, \citeauthor{MetgerYuen:2023}, and \citeauthor{Aaronson:2004}.
All three complexity notions are based on the notion of state preparation, instead of more traditional definition of complexity such as the decidability of a computational problem. 
The first two consider classes based on sequences of quantum states preparable by a polynomial-sized quantum circuit, where the circuits are uniformly generated by a computational class, for instance, the class $\mathsf{PSPACE}$, which results in the complexity class $\mathsf{StatePSPACE}$~\cite{RosenthalYuen:2022,MetgerYuen:2023}.
The third notion considers a relative complexity, where the complexity is measured between two given states, and is measured by the number of gates, from a given gate-set, required to transform one state in another state~\cite{Aaronson:2004}. 
For our definition of state preparation complexity, we drop the uniformity constraint from~\cite{RosenthalYuen:2022,MetgerYuen:2023} and define a class as $\mathsf{StateX}$, which refers to states preparable by circuits of type $\mathsf{X}$. 
As an example, if $\mathsf{X} = \QNC^0$, this results in the class $\mathsf{StateQNC^0}$, which is the set of states preparable from the $\ket{0}^n$ state by poly-size constant-depth circuits. 
This notion is similar to the relative complexity from~\cite{Aaronson:2004}, where one state is the  $\ket{0}^n$ state and instead of counting the number of gates we consider the set of states preparable by a fixed number of gates. Using this notion of complexity we show that any state preparable by an $\LAQCC^*$ circuit is also preparable by a $\mathsf{PostQPoly}$ circuit, the class of circuits of polynomial depth with an additional post-selection gate. 

All Clifford circuits have a constant-depth $\LAQCC$ implementation, implying that any stabilizer state can be implemented by a constant-depth $\LAQCC$ circuit, see Section~\ref{sec:clifford_circuits} for a proof of this statement. 
Efficient circuits for stabilizer states have been known already through measurement-based quantum computing. Therefore this paper focuses on the preparation of non-stabilizer states, and as a surprising result we find novel constant-depth protocols for four very natural classes of non-stabilizer states.
Despite the extensive research into these four classes of non-stabilizer states and the many applications of them, no efficient constant- or low-depth state preparation protocols are known yet. We specifically consider these four classes as they are all often used as initial states in other algorithms.

The first state is a uniform superposition over an arbitrary number of states. 
This state finds applications in many quantum algorithms, as they often start with a uniform superposition over multiple states. 
This superposition is often achieved by applying Hadamard gates to every qubit due to its simplicity to prepare. 
Yet, the analysis of many algorithms, such as Shor's algorithm~\cite{Shor:1997}, would benefit from a different initial superposition. 
The circuit to prepare the uniform superposition over an arbitrary number of states uses an exact version of Grover search as a subroutine, that turns a probabilistic circuit, with a known constant probability of success, into a deterministic circuit. 
We use the circuit for preparing a uniform superposition over an arbitrary number of states as a subroutine in the next two quantum state preparation protocols. 

The second state is the $W$-state, the uniform superposition over all computational basis states of Hamming-weight~$1$, a natural long-ranged entangled state that displays a fundamentally nonequivalent type of entanglement from the Greenberger–Horne–Zeilinger state~\cite{WState:2000}, for which $\LAQCC$-type constant-depth circuits were previously known~\cite{PhamSvore2013, Cirac:2021}. 
The $W$-state is often used as benchmark for new quantum hardware~\cite{Haffner2005,Neeley2010,GarciaPerez:2021}. 
A novel way to prepare the $W$-state therefore gives a new way to benchmark different quantum devices with each other. 
A circuit for preparing the $W$-state was given in~\cite{Cirac:2021}, but this implementation requires sequentially alternating measurements followed by local unitaries, which in the $\LAQCC$ model is not considered to be of constant depth. 
We improve this protocol by giving an $\LAQCC$ implementation of the $W$-state, based on a compress-uncompress method that links the one-hot and binary encoding of integers.

The third state considered is the Dicke state, a generalization of the $W$-state, a superposition over all computational basis states with Hamming-weight $k$~\cite{Dicke:1954}. 
Dicke states have relevance in various practical settings.
For instance, for quantum game theory~\cite{zdemir2007}, quantum storage~\cite{Bacon_Compress:2006,Plesch:2010}, quantum error correction~\cite{ouyang2014permutation}, quantum metrology~\cite{toth2012multipartite}, and quantum networking~\cite{prevedel2009experimental}. 
Dicke states have been used as a starting state for variational optimization algorithms, most notably Quantum Alternating Operator Ansatz (QAOA)~\cite{Hadfield2019}, to find solutions to problems such as Maximum k-vertex Cover~\cite{Brandhofer2022,cook2020quantum}.
The ground states of physical Hamiltonians describing one-dimensional chains tend to show a resemblance to Dicke states such as states resulting from the Bethe ansatz, making them an ideal starting state when investigating the ground state behavior of these Hamiltonians~\cite{TDL_BetheAnsatzDerivation:2010,B_ExcitedStateQuantumPhaseTransitions:2013,DickeTransitions:2021}. 
For instance, the algorithm by \citeauthor{van2021preparing}, who give an algorithm to prepare the Bethe ansatz eigenstates of the spin-1/2 XXZ spin chain, starts by first preparing a Dicke state~\cite{van2021preparing}. 
A Dicke-state preparation protocol based on the compress-uncompress methodology used in the $W$-state furthermore finds applications in entanglement distillation, where the entanglement of a large state is concentrated on only a few qubits. 
Efficient deterministic circuits for preparing Dicke states have been proposed by \citeauthor{bartschi2019deterministic}~\cite{bartschi2019deterministic, bartschi2022deterministic_short_depth}. 
They provide a quantum circuit of depth $\mathO(k \log(\frac{n}{k}))$, allowing arbitrary connectivity, to prepare a Dicke state, which they conjecture to be optimal when $k$ is constant. 
In this work, we provide a constant-depth $\LAQCC$ circuit below their conjectured bound already for constant $k$. 
However, this does not directly disprove their conjecture, as we allow for intermediate measurements and classical computations. 
More significantly, we even construct constant-depth $\LAQCC$ circuits for $k = \mathO(\sqrt{n})$ greatly improving their bound.
This construction extends the compress-uncompress method for the $W$-state combined with additional subroutines. 

We continue with a log-depth state preparation protocol for the Dicke-state for arbitrary $k$. 
This protocol implements an efficient transformation between the factoradic number representation and the combinatorial number representation of a positive integer. 
The combinatorial number representation relates directly to the Dicke state. 
The provided efficient transformation between number representation systems might be of independent interest. 

We conclude by modifying our protocol for preparing a Dicke-state to a protocol that prepares quantum many-body scar states in constant-depth. 
These states have low entanglement and longer coherence times than states with similar energy density.
These characteristics make many-body scar states interesting to analyze and relevant within physics.
Many-body scar states appear for instance in the AKLT model~\cite{AKLT:1987,MRBAR:2018,MRB:2018} and different spin models~\cite{SI:2019,MOBFR:2020}.
Known methods for preparing these states have polynomial-depth~\cite{Gustafson:2023}, whereas our circuit has constant depth. 

% We conclude by studying the power that intermediate classical calculations can add to quantum computations. 
% In this study, we define a new model that relaxes constant-depth quantum circuits to polynomial depth quantum circuits, log-depth classical calculations to unbounded classical computations and a constant number of alternations to a polynomial number of alternations. 
% We call this model $\LAQCC^*$. 
% We study this model by doing a complexity theoretical analysis, where we draw inspiration from the notions of complexity given by \citeauthor{RosenthalYuen:2022}, \citeauthor{MetgerYuen:2023}, and \citeauthor{Aaronson:2004}.
% All three complexity notions are based on the notion of state preparation, instead of more traditional definition of complexity such as the decidability of a computational problem. 
% The first two consider classes based on sequences of quantum states preparable by a polynomial-sized quantum circuit, where the circuits are uniformly generated by a computational class, for instance, the class $\mathsf{PSPACE}$, which results in the complexity class $\mathsf{StatePSPACE}$~\cite{RosenthalYuen:2022,MetgerYuen:2023}.
% The third notion considers a relative complexity, where the complexity is measured between two given states, and is measured by the number of gates, from a given gate-set, required to transform one state in another state~\cite{Aaronson:2004}. 
% For our definition of state preparation complexity, we drop the uniformity constraint from~\cite{RosenthalYuen:2022,MetgerYuen:2023} and define a class as $\mathsf{StateX}$, which refers to states preparable by circuits of type $\mathsf{X}$. 
% As an example, if $\mathsf{X} = \QNC^0$, this results in the class $\mathsf{StateQNC^0}$, which is the set of states preparable from the $\ket{0}^n$ state by poly-size constant-depth circuits. 
% This notion is similar to the relative complexity from~\cite{Aaronson:2004}, where one state is the  $\ket{0}^n$ state and instead of counting the number of gates we consider the set of states preparable by a fixed number of gates. Using this notion of complexity we show that any state preparable by an $\LAQCC^*$ circuit is also preparable by a $\mathsf{PostQPoly}$ circuit, the class of circuits of polynomial depth with an additional post-selection gate. 

\paragraph{Summary of results}
\begin{itemize}
    \item We give a new definition of a computational model that captures the power of the four step process: applying a constant number of layers of one- and two-qubit gates; performing a syndrome measurement; perform a fast classical computation determining corrections; apply corrections. We call this model \emph{Local Alternating Quantum Classical Computations}, or $\LAQCC$ for short. In this model we bound the allowed quantum operations, intermediate classical calculations, and number of rounds separately. In Section~\ref{sec:LAQCC_model} we define this model and give a list of operations based on results from literature contained in this computational model. In some of these operations we explicitly use that we allow for multiple, but at most constant, rounds  of corrections.
    \item  We show show that there exist $\LAQCC$ circuits that can not be weakly simulated in Section~\ref{sec:IQP_in_LAQCC}. We further show that for every $\LAQCC$ circuit there exists a $\QNC^1$ circuit simulating it perfectly, in Section~\ref{sec:LAQCC_in_QNC1}.
    \item We introduce a new type computational complexity for preparing states and show that the extension of $\LAQCC$ where we allow a polynomial number of rounds and unbounded classical computation, is contained in $\mathsf{PostQPoly}$, the class of polynomial circuits with post-selection, in Section~\ref{sec:Complexity results}.
    \item We show a protocol to prepare the uniform superposition state of size $q$ in $\LAQCC$ using $\mathO(\ceil{\log_2(q)}^2)$ qubits in Section~\ref{sec:superposition_modulo_q}. 
    \item We show a protocol to prepare the $W_n$ state in $\LAQCC$ using $\mathO(n\log(n))$ qubits in Section~\ref{sec:W_state_in_LAQCC}.
    \item We show two ways of preparing the Dicke-$(n,k)$ state. The first method is in $\LAQCC$, works up to $k = \mathO(\sqrt{n})$, uses $\mathO(n^2\log(n))$ qubits, and is found in Section~\ref{sec:dicke:small_k}. The second method is in $\LAQCC\text{-}\mathsf{LOG}$ (an extension of $\LAQCC$ allowing for logarithmic number of alterations instead of constant), works for any $k$, uses $\mathO(\text{poly}(n))$ qubits, and is found in Section~\ref{sec:Dicke_in_LAQCC_LOG}. 
    \item We extend on our $\LAQCC$ method of generating Dicke-$(n,k)$ states for $k = \mathO(\sqrt{n})$ and show a protocol to generate many-body scar states for a particular Hamiltonian in $\LAQCC$ (Section~\ref{sec:many_body_scar}). 
\end{itemize}
Summarized in a table, we provide the following state generation protocols:
\begin{table}[htb]
\centering
\begin{tabular}{l|l|l|l}
\textbf{State description} & \textbf{Width} & \textbf{Depth} & \textbf{Implementation}\\
\hline 
Uniform superposition mod $q$: $\frac{1}{\sqrt{q}} \sum_{i = 0}^{q-1}\ket{i}$ & $\mathO(\ceil{\log^2 q})$ & $\mathO(1)$ & Section~\ref{sec:superposition_modulo_q}\\

$W$-state: $\frac{1}{\sqrt{n}}\sum_{i = 0}^{n-1}\ket{e_i}$ & $\mathO(n \log n)$ & $\mathO(1)$ & Section~\ref{sec:W_state_in_LAQCC}\\

Dicke-$(n,k)$, $k = \mathO(\sqrt{n})$: $\binom{n}{k}^{-1/2}\sum_{x \in \{0,1\}^n: |x| = k} \ket{x}$ &  $\mathO(n^2\log n)$ & $\mathO(1)$ 
&Section~\ref{sec:dicke:small_k}\\

Dicke-$(n,k)$: $\binom{n}{k}^{-1/2}\sum_{x \in \{0,1\}^n: |x| = k} \ket{x}$ & $\mathO(\text{poly}(n))$ & $\mathO(\log n)$ &Section~\ref{sec:Dicke_in_LAQCC_LOG}\\

QMBS: $\ket{S_k} = \frac{1}{k! \sqrt{\mathcal N(n,k)}}(Q^\dagger)^k \ket{\Omega}$ &  $\mathO(n^2\log n)$ & $\mathO(1)$  &  Section~\ref{sec:many_body_scar}
\end{tabular}
\caption{Summary of state preparation protocols given in this paper.}
\label{tab:sate_prep}
\end{table}
In the entry for the quantum many-body scar state $Q$ denotes the raising operator and $\mathcal N(n,k)=\binom{n-k-1}{k}$. 
Section~\ref{sec:many_body_scar} will provide more details on the variables and the implementation. 

\paragraph{Organization of the paper}
\noindent We first introduce relevant preliminaries in Section~\ref{sec:preliminaries}. 
In Section~\ref{sec:LAQCC_model} we formally define the class of Local Alternating Quantum-Classical Computations ($\LAQCC$). We also show that any Clifford circuit can be implemented in constant depth $\LAQCC$ (a result based on a result from measurement-based quantum computing~\cite{jozsa2006introduction}). 
This result allows us to give many useful multi-qubit gates and routines in Section~\ref{sec:gates_created_in_LAQCC}. 
Beyond that we show that constant depth $\LAQCC$ circuits are contained in $\QNC^1$ and that any $\mathsf{IQP}$ circuit has an $\LAQCC$ implementation.
We conclude this section with an analysis of a more powerful instantiation of $\LAQCC$ and show an inclusion with respect to the class $\mathsf{PostQPoly}$, which is the class of circuits of polynomial depth with one additional post-selection gate. 
In Section~\ref{sec:state_prep_in_LAQCC} we give $\LAQCC$ circuit implementations for preparing the uniform superposition over an arbitrary number of states, the $W$-state and the Dicke state up to $k = \mathO(\sqrt{n})$. We furthermore give a log-depth circuit implementation for preparing the Dicke state for any $k$. We conclude by showing a $\LAQCC$ circuit for generating many body scar states of a particular type of Hamiltonian.


% \vspacebeforesection
\section{Background}
\label{sec:background}

In this section, we provide the necessary background information to ensure a comprehensive understanding of the attack described in this paper. We start with a description of the Distributed Hash Table (DHT) used by IPFS, followed by its content resolution mechanisms. We also detail techniques for network size estimation, necessary for our attack detection and mitigation mechanisms.

\vspacebeforesection
\subsection{IPFS DHT}
\label{sec:kad_dht}

We review the features of the Kademlia DHT~\cite{maymounkov2002kademlia} and its \texttt{libp2p} implementation~\cite{libp2p_github} that are the most relevant to our attack.
To participate in the DHT, each peer generates a public/private key pair and derives an identity $\peerid \in \{0,1\}^{256}$ as the hash of its public key.
Ideally, each peer generates a random key pair and, therefore, peer IDs are distributed uniformly and independently over the space $\{0,1\}^{256}$.
While honest nodes follow this rule, malicious nodes may generate and choose from an arbitrary number of key pairs.
Each peer maintains a routing table consisting of $m=256$ buckets.
The $i$-th bucket contains the addresses of up to $k=20$ peers whose peer IDs share a common prefix of exactly $i$ bits with the peer's own peer ID. 

%
A new participant node joins the IPFS network by contacting one of the hardcoded bootstrap nodes. This bootstrap node provides the new node with some initial peers allowing it to join the DHT. The new node uses this information to perform a walk through the DHT towards its own peer ID.
The walk allows to: \textit{(i)}~make sure that there is no other node in the network with the same ID; \textit{(ii)}~discover new peers and fill the newcomer's DHT routing table. At the same time, the newcomer establishes \bitswap~\cite{de2021accelerating} connections to a subset of encountered peers (usually around 300 of them). The core role of the \bitswap protocol is to enable bilateral content transfer and to play the role of a cache for recently-accessed content.

The main DHT operation $\Call{GetClosestPeers}{\key}$ returns the $k=20$ closest peers to $\key$. 
%
In Kademlia, the distance between two keys $x$ and $y$ in the key space is given by $x \oplus y \in \{0,...,2^{256}-1\}$, where $\oplus$ denotes the bitwise XOR operation on the keys; the resulting binary string is interpreted as an integer.
%
When a client wants to find the peers with IDs closest to $\key$, it sends a request to the $\alpha=3$ peers in its routing table whose peer IDs are closest to $\key$. Each of these peers returns the $k$ closest peers to $\key$ in its own routing table and the addresses of these peers. 
%
The client again sends a request to the $\alpha$ peers closest to $\key$, among peers in its routing table and those whose addresses it just received. This process repeats until the client does not find any more peers closer to $\key$.
Due to network churn and imperfect routing tables, we observed in our experiments that successive calls to $\Call{GetClosestPeers}{\key}$ do not always return the same set of $k=20$ peers (we provide more details in \Cref{sec:evaluation}, \Cref{fig:20closest}). This is an important limitation affecting our attack.

\vspacebeforesection
\subsection{Content Resolution in IPFS}
\label{sec:ipfs}

IPFS is a content-centric network.
It allows its participant to request files without specifying their location. 
%
Content is indexed by content IDs $\cid \in \{0,1\}^{256}$ that are derived from a hash of that content.
Both peer IDs and CIDs are used as keys in the DHT.
Each node can play the role of a \provider, \downloader, or \resolver. 
The process of content advertisement and resolution is illustrated in \Cref{fig:add_get_provider}.

%
When a \provider wishes to publish content with a given $\cid$ on IPFS, it creates a \emph{provider record} that contains $cid$ and the \provider's address.
During a $\Call{Provide}{\cid}$ operation, the \provider first uses $\Call{GetClosestPeers}{\cid}$ to locate the $k=20$ peers with their peer IDs closest to $\cid$, 
%
and then sends them a $\mathsf{PutProvider}$ message including the provider record (\Cref{fig:add_get_provider}(a)).
We call the peers that hold provider records for $\cid$ the \emph{resolvers} for $\cid$.

Each CID can have several \providers. In fact, by default, each IPFS client becomes a provider for each piece of content it downloads for a fixed amount of time (12h, 24h, or 48h depending on the client version or custom configuration). As a result, the system provides an auto-scaling feature with supply automatically rising with demand.

%
When a \downloader wishes to fetch a piece of content, it first sends a request to all its \bitswap peers. If none of them has the content, the \downloader uses the DHT-based resolution system. We stress that the \bitswap protocol plays the supporting role of a cache in the dissemination of popular files. However, the mechanism does not provide reliable content resolution, in particular for new or less popular content. %

When \bitswap unstructured search fails, the \downloader resolves $\cid$ using $\Call{FindProviders}{\cid}$. This operation uses a DHT walk identical to that of $\Call{GetClosestPeers}{\cid}$ to find $k$ \resolvers but also queries encountered nodes for a provider record for $\cid$ (\Cref{fig:add_get_provider}(b)). The process terminates when either 20 \providers have been found, or all \resolvers have been asked. Querying all encountered nodes (\ie, not only the designated \resolvers) is useful because some of the encountered nodes may have a provider record in their cache.
%

Upon receiving a provider record, the client connects to the address specified in the provider record to retrieve the actual content (\Cref{fig:add_get_provider}(c)).
Provider records are not authenticated, and therefore malicious \providers may respond with incorrect provider records (or may not respond at all). However, the integrity of the content is preserved because the hash of the retrieved content can be verified against its $\cid$.
%


%

\input{img/add_get_provider.tex}

\vspacebeforesection
\subsection{Network Size Estimator}
\label{sec:netsize}

The number of nodes in a decentralized system is generally unknown due to the avoidance of centralized membership management.
This number is nonetheless useful for optimizations, deciding on individual node configurations, or security mechanisms.
Various methods were proposed for the decentralized estimation of unstructured and structured networks~\cite{eli-sohl-dht-size-estimation,kostoulas2005decentralized, manku2003symphony}.
We use in this work a mechanism developed initially by Protocol Labs as part of a mechanism for decreasing the latency of publishing content in IPFS~\cite{network-size-estimation-notion,network-size-estimation-github-pr}.

%
%
%
%
%
%
%
%
%
%

Each node in the DHT refreshes its routing table periodically (every $10$ minutes in \texttt{libp2p}). 
For this, the node samples $m$ random keys (one for each bucket of its routing table)
%
and queries the DHT to obtain the $k=20$ closest peer IDs to each key.
Using these, the node then computes the average distance between each one of these keys $\key_j$ for $j=1,\dots,m$ and their $i$-th closest peer ID for $i=1,...,k$ (with $m=256$ and $k=20$).
\begin{equation}
    \label{equ:avg-dist}
    \overline{D}_i = \frac{1}{m} \sum_{j=1}^m \operatorname{dist}(\key_j, \peerid_{j}^{(i)})
\end{equation}
where $\peerid_{j}^{(i)}$ is the $i$-th closest peer ID to $\key_j$.
With $N$ peers in the DHT and peer IDs uniformly distributed in the hash space, the expected distance between a $\key$ and its $i$-th closest peer ID is $\frac{2^{256}i}{N+1}$. The node then runs a least square regression to compute the value of $N$ for which the expected distances best fit the empirical average distances, \ie,
\begin{equation}
    \label{equ:netsize-least-squares}
    \hat{N} = \arg\min_{N} \sum_{i=1}^k \left(\overline{D}_i - \frac{2^{256}i}{N+1}\right)^2.
\end{equation}
The resulting estimate $\hat{N}$ can be computed in closed form.
%

When a node starts running, it must perform DHT queries for a few random keys to initialize its network size estimate. 
Since a larger number of queries will result in higher accuracy, making more queries than what is needed to initialize one's routing table is recommended.
Thereafter, keeping the estimate up-to-date does not require any excess DHT queries beyond what is already used for refreshing the routing table as this is done frequently (every 10 minutes).

While the network size estimate has a stochastic variance resulting from the probability distribution of the honest peer IDs, it is hard for an attacker to bias the estimate significantly. Since the estimator uses the density of peer IDs around keys chosen uniformly at random, the adversary would require numerous Sybil nodes (on the order of the whole network size) to significantly affect the peer ID density around those keys.

%\section{Preliminaries}\label{sec: preliminary}
This section introduces the fundamentals of multi-agent market model.  
We first overview the market model in Section~\ref{subsec: ham_sys},
and discuss more details about the order book and trading agent in Section~\ref{subsec: ham_var} and Section~\ref{subsec: agent} respectively. 

\subsection{Multi-agent Market Model}\label{subsec: ham_sys}
Multi-agent market model is a complex and dynamic trading simulation system
that generates order flow by modeling the interaction of multiple elements of stock market. 
The element of stock market has different flavors in economics~\cite{lebaron2006agent}. 
In this paper, we follow one of the mainstream viewpoints that 
a market is dominated by different trading strategies (modeled as agent variable) and the fundamental values of the asset~\cite{chiarella2009impact}. 
In other words, the stock price of an asset jointly depends on its fundamental value as well as the traders' behaviors/strategies. 

As such, a multi-agent market model can be regarded as a parameterized generation model, which is shown as 
\begin{equation}
    Pr(x|a, w), 
\end{equation}
where an order flow, denoted as $x$, is generated from a conditional distribution
that depends on the vector of agent variable, denoted as $a$, and the asset's fundamental value, denoted as $w$. 

Each trading agent has a set of variables to determine its ordering decisions.
The decisions of the agents collectively influence the generated order flow. 
The details of agent variable and their trading strategies will be discussed in Section~\ref{subsec: agent}. 

An asset's fundamental value refers to the common sense of the asset's pricing in market. 
In reality, it is usually from the reports of noted financial experts and institutions. 
Since we do not focus on stock pricing or prediction, 
we use the mid-price (see Equation~(\ref{eq: order_book})) upon the target order flow to calculate the fundamental value. 

The multi-agent model has randomness due to agent irrationality. 
In practice, traders are not completely rational such that
they may not strictly follow their trading strategies~\cite{chiarella2009impact, mathew2012genetic}.  
To model this, the agents' trading decisions are designed to be partially rational, 
making the market model a probabilistic model. 

\subsection{Order Book}\label{subsec: ham_var}

% Figure environment removed 

In a multi-agent market model, 
agents interact and trade through an order book, which is illustrated in Figure~\ref{fig: orderbook}. 
Agents making an order to an order book is analogous to placing an item on shelves, 
where the levels of the shelf stands for the asset prices, 
and weight of the item represents the order size (the number of shares). 
The order book matchmakes a pair of bid and ask orders once they make a deal on the asset price. 

We regard the mean of the highest bid and lowest ask prices on an order book as the asset's current price at time instant~$t$, the so-called mid-price $P_t$. 
Centered at the mid-price, the shape of an order book is represented as a vector of the order sizes at numbers of price levels, 
ranging from~$P-n\delta$ to $P+n\delta$, where~$\delta$ is the minimal variation unit of price, and $n$ is often called the depth of an order book. 
Given the above, the state of the order book at time instant $t$, denoted as $x_t$, can be represented as
\begin{equation}
    x_t=\left( P_t, Q_t \right)
    \label{eq: order_book}
\end{equation}
where $Q$ is the shape of the order book, and the dimension of Q is $2n+1$. 
The state of the order book is the result of order flow. 
Therefore, we use the order book's state to represent the order flow. 
For simple discussion, we do not consider short sale, and  
market order will be considered as a special case of limit order~\cite{gould2013limit}. 


\subsection{Trading Agent}\label{subsec: agent}

The main difference among trading agents' trading strategies lies in the evaluation of asset price, the so-called expected price.  
The expected price of an agent directly determines the agent's order, specifically the ordering price and size.  
Intuitively, a trader should be willing to buy/sell more stock shares when the stock price deviates more from its expected price. 

To introduce how the agents evaluate the price of an asset, 
we exemplify the taxonomy in~\cite{chiarella2006asset},
where the expected price of an agent is determined by the three components of fundamentalist ($f$), chartist ($c$) and irrational trader ($n$). 
Formally, denoting $\alpha$ as the proportion of the components and $\hat{P}$ as the expected price of the component, 
the expected price of an agent, indexed by $i$, is given as 
% which regards each agent as a composition of three components that determines how it evaluates the expected price:
\begin{equation}
    \hat{P}_t^{(i)} = \frac{\alpha_w^{(i)}\hat{P}_t^{(w)}+\alpha_c^{(i)}\hat{P}_t^{(c)}+\alpha_n^{(i)}\hat{P}_t^{(n)}   }{\alpha_w^{(i)}+\alpha_c^{(i)}+\alpha_n^{(i)}}.   
\end{equation}


A pure fundamentalist always agrees with the fundamental value of the asset, whose pricing is shown as 
\begin{equation}
    \hat{P}_t^{(w)}=w_t.
\end{equation}
On the other side, a pure chartist estimates price by looking at the historical prices on the market, which is given as 
\begin{equation}
    \hat{P}_t^{(c)}=g\left(P_{t-1}, P_{t-2}, ..., P_{t-\tau_i}\right), 
\end{equation}
where $g(\cdot)$ is usually a regression function, and $\tau_i$ is the horizon of the $i$th agent. 
The irrational trader characterizes the irrationality of real trader, 
whose expected price is sampled from a certain statistical distribution. 
Take Gaussian distribution as an example, the expected price of a pure irrational trader is given as
\begin{equation}
    \hat{P}_t^{(n)}\sim \mathcal{N}\left(P; P_{t-1}, \sigma_n^2\right), 
\end{equation}
where $\sigma_n$ quantifies the degree of irrationality. 

% Figure environment removed 

Given the expected price, 
an agent would make an order that specifies the price, denoted as $p_t$,  and size, denoted as $v_t$, of an order (we omit the index $i$ for conciseness). 
Commonly in behavioral economics, a trader tends to make a larger size of order 
when the order price deviates more from the expected price, 
because a large deviation means a large profit, the so-called CARA utility~\cite{babcock1993risk}. 
Based on this, the relationship between order price and size can be modeled as
\begin{equation}
    v_t=\frac{1}{\beta\Delta P_tp_t}\log\left(\frac{\hat{P}_t}{p_t} \right),
    \label{eq: risk}
\end{equation}
where $\beta$ is the degree of risk aversion, and $\Delta P_t$ is the variance of historical prices~\cite{vyetrenko2020get}. 
In this example, the agent's ordering depends on the proportion of 
the components~$\alpha_i^{(w)}, \alpha_i^{(c)}, \alpha_i^{(n)}$, irrationality $\sigma_n$, risk aversion $\beta_i$, and horizon $\tau_i$. 
These are the variables of each individual agent. 

Multi-agent market model needs the heterogeneity of agents to ensure the volatility of the market~\cite{iori2012agent, liu2020semiglobal}. 
Conversely, if all trading agents behave the same, no deal would be made on the order book, and the simulation becomes pointless.  
On the other hand, since a market model usually involves hundreds of agents and more, 
it is unrealistic to calibrate the variables of each agent. 
Therefore, literature summarizes the variables of all agents in the market model as multiple statistical distributions, 
upon which the variable of each agent is sampled. 
Accordingly, the set of prior parameters of these distributions is called the agent variable of the market model.  
For example, the risk aversion is sampled from a Gaussian distribution as
\begin{equation}
    \beta_i\sim \mathcal{N}\left(\beta; a_\beta, \sigma^2 \right),
\end{equation}
where $a_\beta$ is an element of the agent variable of the market model. 
% Essentially, the adoption of distribution introduces system randomness, 
% which is necessary for the simulation system to provide agent heterogeneity.

Overall, we summarize the agent variable of the model as a vector, which is written as
\begin{equation}
\begin{aligned}
    a&=\left[a_w, a_c, a_n, \sigma_n, a_\beta, a_\tau \right]\\
    &=\left[a_w, a_c, \sigma_n, a_\beta, a_\tau \right], 
\end{aligned}
\end{equation}
where the equality holds by normalizing the proportions of the three components as one. 
Note that, despite we introducing the above agent as an example,  
\sysname{} is also applicable to other types of rule-based agents.  



\iffalse
\subsection{Fintech Multi-agent System}\label{sec: ham_sys}
% Definition
Fintech multi-agent System, specifically referred to as heterogeneous multi-agent system~\cite{lebaron2006agent} in this paper, is a computational model that simulates the actions and interactions of different types of traders.
Even with simple agents, 
it can exhibit complex behavior patterns and provide valuable information about the dynamics of the real-world system which they emulate~\cite{iori2012agent}. 
Such a simulation system usually consists of an orderbook~\cite{gould2013limit} and a group of basic agents. Heterogeneous agents are created by combining the components of basic agents using linear combinations.


The basic agents mainly differ in how they evaluate the price of the asset. 
There are some common types of basic agents:
\begin{itemize}
    % zero-intelligence agent
    \item \emph{Fundamentalist}  (or value-based agent) estimates asset price according to fundamental price, which is an exogenous information that conveys the value of the asset in a moment:
    \begin{equation}
        \hat{P}_f^{(t)}=P_f^{(t)},
    \label{eq: fundamental}
    \end{equation}
    where $P_f$ is the fundamental price at time $t$.

    
    \item \emph{Chartist} estimates asset prices based on historical data on the orderbook.
    For example, chartist in~\cite{chiarella2006asset} evaluates asset price by
    \begin{equation}
        \hat{P}^{(t+1)}_c=P^{(t)}+\frac{1}{\tau}\sum_{T=t-\tau}^{t} \left( P^{(T)}-P^{(T-1)} \right),
    \label{eq: chartist}
    \end{equation}
    where $\tau$ is the investment horizon of the agent and $P^{(t)}$ is the mid-price at time $t$.

    \item \emph{Noise trader} models the irrationality of the market,
    where the estimated price is sampled from a distribution.
    The noise trader in~\cite{chiarella2006asset} samples prices from Gaussian distribution
    \begin{equation}
        \hat{P}_n^{(t)}\sim \mathcal{N}(P^{(t-1)}, \sigma^2),
        \label{eq: noise_trader}
    \end{equation}
    where $\sigma$ is the pre-determined standard deviation. 
\end{itemize}
More types of agents can be found in~\cite{chiarella2009impact}.


The estimated price of a heterogenous agent is calculated by
\begin{equation}
\hat{P}_{t+\tau}=\frac{g_f\hat{P}_f^{(t)}+g_c\hat{P}_c^{(t)}+g_n\hat{P}_n^{(t)}}{g_f+g_c+g_n},
\label{eq: price_est}
\end{equation}
where $f$, $c$, and $n$ refer to fundamentalist, chartist, and noise trader.
The agents differ by the component of being these three characters, i.e.~$g_f$, $g_c,$ and $g_n$.
In this paper, the component of each agent is separately sampled from Laplacian distributions.

% make orders
Agents make buy/sell orders decision according to their esitimated asset prices.
A buy or sell order consists of price and size (a market order is regarded as a limited order with the current price).
In this paper,
each agent calculates the price and size based on CARA utility function~\cite{babcock1993risk}:
\begin{equation}
\pi(P)=\frac{\log(\hat{P}_{t+\tau^{(i)}}/P)}{\alpha^{(i)} V_t P},
\label{eq: price_size}
\end{equation}
where order size $\pi(P)$ is determined by estimated price $\hat{P}_{t+\tau}$, current price $P$, 
risk aversion of the agent $\alpha^{(i)}$, and historical price variance $V_t$.
The $\tau^{(i)}$ denotes the investment horizon of the agent, 
where its order will be canceled after $\tau$ time slots without being executed.
Basically, 
Equation~\ref{eq: price_size} means to buy more at a lower price, and vice versa.
And the price can be sampled from a range that the agent's account can take.
We differentiate the institutional trader~\cite{} from a regular trader by initial cash and stocks in their accounts.
The investment horizon of agent $i$ is
\begin{equation}
\tau^{(i)}=\tau\frac{1+g_f^{(i)}}{1+g_c^{(i)}},
\label{eq: horizon}
\end{equation}
and its risk aversion is calculated by 
\begin{equation}
\alpha^{(i)}=\alpha\frac{1+g_f^{(i)}}{1+g_c^{(i)}},
\label{eq: risk_aversion}
\end{equation}
where $\tau$ and $\alpha$ are the base parameters. The initial cash and account are sampled from the uniform distribution.
More details can be found in~\cite{chiarella2006asset, chiarella2009impact}.



\subsection{Macroscopic Market Information}
% \chang{introduce CPI, PPI, ..., and why they can represent market state, how they impact the calibration, interpret why we need different strategies for differen market state}

Macroscopic market information usually reflects the running condition of the whole market,
effectively characterizing the overall behaviors of investors~\cite{levy1995microscopic}. 
Therefore, it will be useful to integrate such information in calibrating multi-agent simulation systems. 
Here, we list some indices that are commonly adopted to describe market states:

\begin{itemize}
    \item CPI stands for Consumer Price Index, 
    which is a measure of the average change in prices of goods and services over time. 
    A high CPI indicates severe inflation, where investors may become concerned about the future profitability of companies and may sell their stocks, 
    causing the stock market to decline;
    \item PPI measures the average change in prices that producers receive for their goods and services. 
    It is used to track inflation in the early stages of the production process, which can provide insight into the future direction of inflation. 
    PPI can impact the stock market and the behaviors of stock investors because changes in producer prices can affect the profitability of companies;
    \item PMI stands for Purchasing Managers' Index, which is a measure of the economic activity in the manufacturing sector. 
    % It is based on a survey of purchasing managers who provide information on factors such as new orders, production levels, and employment.    
    When PMI is high, it indicates that the manufacturing sector is expanding, which can lead to increased production and potentially higher profits for companies. This can cause investors to become more optimistic and may lead to a rise in the stock market;
    \item Market trend is a technical indicator of market state. In a market whose price forms a strong trend, investor strategies tend to resort to the technical side. In this paper, we calculate market trend by
    \begin{equation}
        MT=\frac{P_t-P_N}{ATR_t},
    \end{equation}
    where~$P_N$ is the averaged asset price of the recent N days and the average true range (ATE) is calculated by 
    \begin{equation}
        ATE_t=\frac{ATE_{t-1}\times (n-1)+TR_t}{n}.
    \end{equation}
    Here, the true range~$TR=\max\{(P_{high}-P_{low}), abs(P_{high}-P_{close}), abs(P_{low}-P_{close})\}$;
    \item Market noise is a technical indicator of price uncertainty. 
    It is usually represented by efficiency ratio
    \begin{equation}
        ER_t=\frac{P_t-P_{t-n}}{\sum_{i=t-n}^t P_i-P_{i-1}}.
    \end{equation}
    
\end{itemize}
More details of macroscopic market information can be found in~\cite{AKShareZhaiQuanShuJuAKShare}.

To guarantee diversity in market states, 
we select five monthly-updated indicators in the experiments:
consumer price index~(CPI), producer price index~(PPI), purchasing manager's index~(PMI),
market trend, and market noise. 
The historical records of CPI, PPI, and PMI can be found in~\cite{AKShareZhaiQuanShuJuAKShare}, 
and the market trend and noise are two well-known indicators that can be calculated separately based on 
average true range~(ATR) and efficiency ratio~(ER) using daily close price. 

\fi

% \vspace{-0.5em}
\section{Case Study}
To verify our findings, we conducted case studies on ChatGPT to further investigate the details. ChatGPT is a state-of-the-art LLM designed to understand and generate human-like text based on the input it receives. Its ability to answer questions, provide explanations, and engage in conversations is helpful for us to understand the nuances of success or failure in our case study. Similar to CodeBERT, ChatGPT has also been adopted to perform source code level analysis and address security-related issues, such as vulnerability discovery and fixing~\cite{surameery2023use,cheshkov2023evaluation}.

In this section, we conduct case studies on ChatGPT with a series of analytical tasks. Specifically, we first assess ChatGPT's capability to comprehend source code. Subsequently, we evaluate ChatGPT's proficiency in tackling specific security challenges and its generalization ability.

\subsection{Semantic Comprehension}


\label{sec:gene}
To evaluate whether ChatGPT has a better ability to understand the logical flows in the code without the influence of naming, we randomly select several code snippets from LeetCode~\footnote{The source code is available at: \url{https://github.com/Moirai7/ChatGPTforSourceCode}}, representing varying levels of difficulty, and assess ChatGPT's performance on the original code snippets and the corresponding obfuscated code snippets.





Our experiments show that ChatGPT can correctly understand the original code snippets.
Taking the Supper Egg Drop problem ({\bf \ref{appendix:Q1}}) as an example, it explains the purpose of the code snippet, the logic and complexity of the algorithm, and provides suggestions to improve the code. 
Then, we manually obfuscated the code snippets by renaming the variables and function names and insert dummy code that will not have an impact on its results.
The obfuscated code is shown in~\autoref{code:obfuscated}.
ChatGPT can not provide useful review of the obfuscated code ({\bf \ref{appendix:Q2}}).

Based on the responses from ChatGPT, we can conjecture that ChatGPT learns the literal features of code snippets that exist within its training data and generates responses based on statistical patterns involving characters, keywords, and syntactic structures. This knowledge encompasses various aspects such as indentation, syntax, and common coding conventions. However, it lacks a profound comprehension of code semantics, logic, or context beyond the patterns and examples present in its training data. Consequently, it struggles to comprehend obfuscated code, as it relies heavily on literal features. This issue is exacerbated by the fact that ChatGPT is trained on a general corpus rather than a code-focused corpus like CodeBERT, making it more susceptible to misinterpretation due to name changes.

\subsection{Vulnerabilities Analysis}
\label{sec:smart}
Though GPT relies on literal features to understand code logic, several works demonstrate its ability to understand normal code (without obfuscation)~\cite{sun2024gptscan,sagodi2024reality,xiao2023supporting}.
The following section tests its ability in software testing, including vulnerability identification, exploitation development, remediation, and real-world security audits.

\subsubsection{Synthetic Programs and Classic Vulnerability}
We adopt a piece of code ({\bf \ref{appendix:Q3}}) that contains buffer overflow vulnerability and then task ChatGPT to find the vulnerabilities.
Then, we replace the \texttt{strcpy} function with a homemade string copy function ({\bf \ref{appendix:Q4}}) and challenge ChatGPT with modified code.
The analysis conducted by ChatGPT on the vulnerability in the code was impressively accurate, showing its potential to identify vulnerabilities in the source code effectively.
However, in an effort to explore its understanding of potential security risks, we posed another question:
\ding{182} Whether ChatGPT can exploit a vulnerability to launch an attack.

Through experiment ({\bf \ref{appendix:Q5}}), we observed that the exploit proposed by ChatGPT, although not directly usable to launch attacks, was conceptually correct.
This suggests that ChatGPT possesses the ability to assist attackers in generating potential exploits.

Then, we investigate \ding{183} whether ChatGPT can also provide solutions to fix the identified vulnerability.
Our evaluation ({\bf \ref{appendix:Q6}}) has demonstrated that the proposed patching effectively addresses the vulnerability in the code, providing an effective fix. As a result, we believe that ChatGPT holds the potential to assist developers and security analysts in identifying and resolving vulnerabilities in normal programs.

Finally, we assess \ding{184} whether it can suggest protection schemes that thwart potential exploits by attackers.
By leveraging ChatGPT's analysis, we aim to explore its ability to propose robust security measures that can safeguard against known vulnerabilities and potential attack vectors.

In the experiment (({\bf \ref{appendix:Q7}})), the proposed protection scheme suggested by ChatGPT demonstrates a conceptually correct approach to detect buffer overflows through the use of a ``canary''.
However, upon closer inspection, we identified some implementation errors that may impact its effectiveness.
The placement of the canary at the beginning of the buffer, instead of at the end, renders it unable to detect buffer overflows that overwrite adjacent variables located at higher memory addresses.
Additionally, the revised code introduces the ``canary'' into the buffer without correspondingly increasing the buffer size, limiting its capacity to handle strings as long as the original implementation.

While ChatGPT appears to understand the concept of using a ``canary'' for buffer overflow detection, it seems to struggle in correctly applying this knowledge to unseen code snippets. These issues raise concerns regarding ChatGPT's soundness.
That means GPT can provide answer/solution that are correct in the high level, but incorrect in some details. 
As the software protection require higher soundness, which poses challenges to adopt GPT to protect software. 

\subsubsection{Real-world Program and Logic Bugs}
The last subsection discussed GPT's ability with synthetic programs containing popular classic bugs. In this section, we test its ability to reason about logic bugs (cross-function vulnerabilities) in real-world applications. Specifically, our objective is to investigate the capabilities of ChatGPT to analyze and audit smart contracts, which are commonly deployed in the Web3 ecosystem.

First, we tested different types of vulnerabilities (({\bf \ref{appendix:Q8}})) that happened before ChatGPT launched so that they could potentially manifest within the ChatGPT system (\ie, seen programs).
According to the answer, ChatGPT adeptly identifies potential vulnerabilities and offers appropriate patches. Furthermore, after selecting a full list of smart contracts \cite{Crytic,runtimeverified} across various categories including reentrancy attack and variable shadowing, ChatGPT consistently discerns vulnerabilities and provides accurate solutions.

Secondly, we conducted analyses on a selection of recently emerged vulnerable smart contracts (({\bf \ref{appendix:Q9}})), occurring within the temporal scope of 2023, to evaluate their efficacy in terms of generalization. This examination was predicated upon a real-world incursion targeting Midas Capital (as detailed in \cite{midascapital}) in June 2023.
The actual root cause of this attack is due to round issues, however, based on the answer that ChatGPT gives us, it is not aware of the risk in function \texttt{redeemFresh}.
We also test several recent attack cases \cite{Web3Sec}, and ChatGPT can correctly identify none of them.

Third, we additionally assess the efficacy of ChatGPT in the realm of cross-contract vulnerability detection (({\bf \ref{appendix:Q10}})). 
To this end, we present a real-world attack scenario exemplified by an incident occurring within the Euler Finance ecosystem \cite{eulerdefi}.
According to the findings of the analysis, ChatGPT is unable to accurately discern vulnerabilities, opting instead to provide seemingly correct responses arbitrarily, devoid of substantive remedial recommendations. Consequently, it is still difficult for chatGPT to grapple with the intricacies in the analysis of cross-contract vulnerabilities.

% \vspace{-0.5em}\subsection{Input Generation}
Input generation is wide used in software testing, which aim to achieve a high coverage by triggering different program pathes.
This section, we aim to answer \ding{188} whether ChatGPT can offer valuable and effective seed inputs after understanding the source code.
Specifically, we test the ChatGPT's input generatation ability to achieve code coverage, input mutations, and  exeuction simulation.

Upon analyzing the responses ({\bf \ref{appendix:Q11}, \ref{appendix:Q12}, \ref{appendix:Q13}}) from ChatGPT, we observed that the provided seeds were comparable to those generated by AFL (American Fuzzy Lop) \cite{aflrefurl} using seed mutation strategies. However, it became apparent that ChatGPT did not propose highly effective seeds (capable of triggering the vulnerability). This observation suggests that ChatGPT may face challenges in leveraging the program semantics it learned from the code to generate such effective seeds.
There could be two possible reasons for this:

\begin{enumerate}
\item The prompts we use may not have been designed to specifically encourage ChatGPT to produce the most useful and targeted solutions.\lan{remove this. if it is true, we could say our previous case studies are not correct because of prompts. this is out-of-scope}
\item ChatGPT's responses might be too general, primarily relying on the information present in the corpus it was trained on, rather than effectively applying the learned program semantics to new situations.
\end{enumerate}

\lan{I vote to remove this. then we only have to kinds of case study: semantic and vunlerbiltiy. it would be easier to explain the storyline}
%Understanding these limitations is crucial for utilizing ChatGPT effectively in vulnerability discovery and other security-oriented tasks. Further research is needed to refine the prompts and explore strategies to enhance ChatGPT's ability to provide more precise and context-aware responses in security-related scenarios.



\subsection{Observations}
% This section presents a series of experiments and empirical analyses aimed at elucidating ChatGPT's capacity for solving security-oriented program analysis.
% Firstly, it demonstrates that ChatGPT, along with other large language models, opens up a novel avenue for security-oriented source code analysis, proving to be an efficient method for learning high-level semantics from well-named source code. 



The conclusion of this case study aligns with the findings from our main experiments: users should expect degraded model performance when applying LLMs to analyze source code lacking sufficient information in variable, function, or class names. For instance, code generated from decompilation~\cite{katz2018using} or code that does not adhere to standard naming conventions~\cite{butler2015investigating} may yield lower accuracy, as demonstrated in questions {\bf \ref{appendix:Q1}} and {\bf \ref{appendix:Q2}}.
Furthermore, additional findings from the case study indicate that for very specific questions involving implementation-level details, logical reasoning, or systematic solutions, ChatGPT's proposed solutions may be conceptually correct at a high level but may not fully address the problem.
% \vspace{-0.5em}
\section{Quantitative Analysis on CodeBert}
% The previous section has explored the ChatGPT's strengths, potential, and limitation in solve some code analysis tasks.
% In this section, we shift our attention from the ChatGPT to the CodeBert. 
CodeBert and GraphCodeBERT~\cite{feng2020codebert,guo2020graphcodebert} are based on Transformer architecture and learn code representations through self-supervised training tasks, such as masked language modeling and structure-aware tasks, using a large-scale unlabeled corpus.
Specifically, CodeBERT is pre-trained over six programming languages (Python, Java, JavaScript, PHP, Ruby, Go) and is designed to accomplish three main tasks: masked language modeling, code structure edge prediction, and representation alignment.
By mastering these tasks, CodeBERT can comprehend and represent the underlying semantics of code, empowering security researchers with a powerful tool for source code analysis and protection.

Since the pre-trained CodeBert and GraphCodeBert and datasets~\cite{trainedcodebert} for the downstream tasks are publicly available for us to download and use, we are able to conduct focused experiments to evaluate them. 
Currently, GraphCodeBert takes code snippets of functions or class methods as data samples. 
It tokenized keywords, operators, and developer-defined names from the code pieces. It utilizes Transformer to learn the data flow of the source code in the pre-training stage.
Inside a function or a method, we can group the developer-defined names into three categories: 1) variable name. 2) method name. 3) method invocation name. 
Program logic is not affected if we map these developer-defined names with other strings in the same namespace.
To evaluate whether the model could learn the code semantics, we design experiments to anonymize three types of literal features: variable names, method names, and method/function invocation names.
For each group of experiments, we anonymize certain categories of developer-defined names. The last group anonymizes all three kinds of literal features.

%\begin{enumerate}
%    \item In the first group of experiments, we anonymize the variable names. %An example is the change from \texttt{it\_end} to \texttt{var3} and \texttt{finished} to \texttt{var4} between \autoref{code:goodname} and \autoref{code:badname}. 
%    \item In the second group of experiments, we anonymize the method names. %An example is the change from \texttt{bubble\_sort} to \texttt{fun1} between \autoref{code:goodname} and \autoref{code:badname}. 
%    \item In the third group of experiments, we anonymize the method/function invocation names. %An example is the change from \texttt{swap} to \texttt{fun2} between \autoref{code:goodname} and \autoref{code:badname}. 
%    \item The last group of experiments are a combination of the first three experiments, which anonymize all three kinds of developer-defined names. 
%\end{enumerate}

\begin{table}[t!]%htpb!
    \setlength{\tabcolsep}{4pt}
  \centering
  \captionsetup{justification=centering}
  \scriptsize
  \caption{Results on Code Search.}
  \label{tab:search}
  \begin{tabular}{lcccccc}
  \toprule
  {\bf Lang-} & \multirow{2}{*}{\bf Orig.}  & \multirow{2}{*}{\bf Anon.} &  {\bf w/o } & {\bf w/o }  & {\bf w/o} & \multirow{2}{*}{\bf All} \\
  {\bf uage}& & & {\bf Variable} & {\bf Method Def.} & {\bf Method Inv.} & \\
  \midrule
  \multirow{2}{0.5cm}{Java}  & \multirow{2}{0.5cm}{70.36\%} & Random  & 67.73\% & 60.89\% & 69.84\% & 17.42\% \\
  & &  Meaningful & 67.14\%	& 58.36\% &	69.84\% &	17.03\% \\
  \midrule
  \multirow{2}{0.5cm}{Python}  & \multirow{2}{0.5cm}{68.17\%}  & Random  &  59.8\% & 55.43\% & 65.61\% & 24.09\% \\
  & & Meaningful& 59.78\% &	55.65\% &	65.61\% &	23.73\%  \\
  \bottomrule
\end{tabular}
\end{table}

\begin{table}[t!]%htpb!
  \setlength{\tabcolsep}{4pt}
  \centering
  \captionsetup{justification=centering}
  \scriptsize
  \caption{Results on Clone Detection.}
  \label{tab:clone}
  \begin{tabular}{ccccccc}
  \toprule
  {\bf Lang-} & \multirow{2}{*}{\bf Orig.}  & \multirow{2}{*}{\bf Anon.} &  {\bf w/o } & {\bf w/o }  & {\bf w/o} & \multirow{2}{*}{\bf All} \\
  {\bf uage}& & & {\bf Variable} & {\bf Method Def.} & {\bf Method Inv.} & \\
  \midrule
  \multirow{2}{0.5cm}{Java}  & \multirow{2}{0.5cm}{94.87\%} & Random & 92.64\% & 93.97\% & 94.72\% & 86.77\% \\
  & & Meaningful & 92.52\% &	94.27\% &	93.67\% &	84.76\% \\
  \bottomrule
\end{tabular}
\end{table}
Besides, we adopt two strategies to anonymize the name:
The first strategy called ``randomly-generated'' randomly generates strings (\eg, ``oe4yqk4cit2maq7t'') without any literal meaning. 
The second strategy called ``meaningfully-generated'' generates strings with a literal meaning.
However, the literal meaning does not reflect the intention of the variable/function/invocation.
For example, this strategy could replace ``bubble\_sort'' with ``aes\_encryption''.

% (replacing a name with a meaningful-string or randomly-generated-sting without any literal meaning)

Based on the four types of name sets to replace and two replacing strategies, we eventually generated 8 variants of the original dataset from \cite{guo2020graphcodebert}. 
Then, we retrain the existing models and evaluate their performance on the existing 2 downstream tasks: natural language code search, and clone detection. 
%We did not choose the other downstream tasks because code refinement adopts an abstraction representation and code translation involves two programming languages, which introduce challenges to uniformly anonymized user-defined names. 
%Since our experiments are focusing on the learning ability of pre-trained models on literal features, and other semantic features. And the downstream tasks are only used to evaluate the effectiveness of the pre-trained model. Therefore, we believe that any reasonable downstream should be fine. 


\subsection{Experiment Results}
Table~\ref{tab:search} and Table~\ref{tab:clone} show experiment results (accuracy) on the downstream task of code search and code clone detection, respectively.
The second column shows the module performance reported by the original paper~\cite{guo2020graphcodebert}.
The fourth, fifth, and sixth columns show the module performance when we anonymize the variable name, method definition name, and method invocation name, respectively.
The last column shows the model performance after we remove all three developer-defined names.





The results show that the anonymization of the variable names, method definition names, and method invocation names will result in a huge downgrade in model performance no matter if we replace developer-defined names with ``randomly-generated" strings or ``meaningfully-generated" strings.
Also, on average the dataset with ``meaningfully-generated" strings performs worse than the dataset with randomly-generated strings, which indicates that  ``meaningfully-generated" strings could mislead the models.

Overall, our experiments reflect that current source-code level representation learning methods still largely rely on the \textit{literal feature} and ignore the \textit{logic feature}.
However, the \textit{literal feature} is not always reliable as mentioned in \autoref{sec:intro}.
The current model still cannot effectively learn the logic feature in the source code.
Attackers can simply randomize the \textit{literal feature} to mislead the CodeBert and GraphCodeBERT models.
%An adversarial machine learning could be trained to further exploit the weakness of the CodeBert. %\fixme{(How? more details?)}
%\fixme{}

%The \textit{logic feature} hidden in the source code contains all the information of a piece of the program. But current models still cannot effectively learn them.  


%\subsection{Observations}
%\fixme{(remove this section?)}
%Through a set of experiments and empirical analysis, this section tries to explain the learning ability of current BERT-based source code representation learning schemes.
%The results show that CodeBERT and GraphCodeBERT are efficient to learn \textit{literal features} but less efficient to learn \textit{logic features}. 
\section{Discussion}
\label{sec: discussion}
\kmsdelete{In this work} We study \kmsreplace{Fairness-Aware PAC learning}{Fair-ERM} in the malicious noise model, and  in some cases allow 
the learner to maintain optimal overall accuracy despite the signal in Group $B$ being almost entirely washed out.
%when we allow learners to use the
%$\PQ$ randomized expansion of the hypothesis class $\mathcal{H}$
In particular we show that different fairness constraints have fundamentally different behavior in the presence of Malicious Noise, in terms of the amount of accuracy loss that a given level of Malicious Noise could cause a fairness-constrained learner to incur. 
The key to achieving our results, which are more optimistic than those in \cite{lampert}, is allowing for improper learners using the (P,Q)-randomized expansions of the given class $\mathcal{H}$.
%We \kmsreplace{present a picture of the}{prove upper and lower bounds on}
%accuracy loss for a range of fairness notions, given \kmsreplace{this simple randomization step.}{learning over $\PQ$.
%In general our results indicate Fair-ERM (given learning over $\PQ$) is more robust than claimed in \cite{lampert}.
The type of smoothness we create by using $\PQ$ seems to be a natural property that is likely shared by many natural hypothesis classes.

Fairness notions are motivated as a response to learned disparities when there is \kmsdelete{data corruption or} systemic error affecting \kmsdelete{the data for}
one group. 
Fairness notions are supposed to mitigate this by ruling out classifiers that have worse performance on a sub-group. 
This can peg both classifiers at a lower level of performance \kmsdelete{(e.g that the lower subgroup)} in order to \emph{motivate} \cite{hardt16} improving the data collection or labelling process to obtain more reliable performance. 
%So in \kmsreplace{some}{a} sense, sensitivity of the fairness notion to poor sub-group performance caused by malicious noise is the \textit{point} of fairness constraints! 
However, it also desirable that fairness constraints perform gracefully when subject to Malicious Noise because fairness constraints will be used in contexts where the data is unreliable and noisy and this might not be known to the learner.
This tension, exposed by our work, motivates 
%a revisiting of fairness notions from first principles approach and trying to axiomatize the 
%desired properties of a fairness intervention a la cryptography and privacy. \footnote{Work in multi-calibration \cite{multicalib} is a viable direction for this problem but it is unclear how 
%that and related notions behave with unreliable data. }
on going work studying the sensitivity level of fairness constraints. 
%If we we are to take a view, if a classifier is deployed 





% \balance
\bibliographystyle{ACM-Reference-Format}
\bibliography{main}

\label{sec:apdx}
\begin{comment}
\section{System Architecture}
\label{appendix:architecture}
\system has a novel modularized system architecture with three key components: 
\emph{StreamManager}, 
\emph{TxnManager} and \emph{TxnScheduler}. 
These components are instantiated in each thread locally.
The execution outline of \system is presented in Algorithm~\ref{alg:algo}.
Transactional stream processing is continuous and potentially never ends (Line 1$\sim$8).
The dependency resolution and execution of state transactions are separated into two non-overlapping phases by punctuations~\cite{Tucker:2003:EPS:776752.776780} (Line 2 and 5), which guarantees that no subsequent input event will have a smaller timestamp. 
Effectively, a batch of state transactions is collected during the first phase, and processed during the second phase.

In the first phase (i.e., stream processing phase), 
the \emph{StreamManager} conducts preprocessing for every input event ($e$). Similar to some prior works~\cite{tstream}, state transactions may be issued but not immediately processed during preprocessing (Line 3).
The \emph{pre\_processing} and \emph{post\_processing} functions are exposed as APIs to users.
The \emph{TxnManager} handles dependency resolution (Line 4) among state transactions and insert decomposed operations to construct a \tpg. We discuss the detailed two-phase \tpg construction process in Section~\ref{subsec:construction}.

In the second phase  (i.e., transaction processing phase), 
the \emph{TxnManager} is first involved again to refine (Line 6) the constructed \tpg with further dependency resolution.
The \emph{TxnScheduler} 
schedules operations for concurrent execution based on the constructed \tpg according to the three dimensions of scheduling decisions (Line 7). 
In particular, a scheduling decision model $M$ is instantiated based on the constructed \tpg (Line 14).
\textbf{\circled{1}} Guided by $M$, execution threads adopt an exploration strategy (Section~\ref{subsec:explore}) to explore the constructed \tpg for operations available to be scheduled constrained by dependencies. 
\textbf{\circled{2}} 
During exploration, one or multiple operations may be treated as the 
% basic 
unit of scheduling (Section~\ref{subsec:granularity}). 
Subsequently, \textbf{\circled{3}} every thread executes operation(s) in the unit of scheduling with various abort handling mechanisms (Section~\ref{subsec:abort_handling}).
Only when state transactions are processed (i.e., committed or aborted) can the associated input events be postprocessed (Line 8) by the \emph{StreamManager} based on transaction processing results.
\end{comment}

\begin{comment}
\begin{algorithm}
\footnotesize
    \KwData{$e$ \tcp{Input event}}
    \KwData{$txn_{ts}$ \tcp{State transaction}}
    \KwData{$G$ \tcp{The currently constructed TPG}}
    \While{!finish processing of input streams}{
        \eIf(\tcp*[h]{Phase 1}){\text{$e$ is not a $punctuation$}}{
                $txn_{ts}$ $\gets$ PRE\_Processing($e$)\;
                \textbf{TPG\_Construction}($G$, $txn_{ts}$)\; 
          }(\tcp*[h]{Phase 2}){
                \textbf{TPG\_Refinement}($G$)\; 
                \textbf{TXN\_Scheduling}($G$)\; 
                POST\_Processing()\;
          }
    }
    
    \SetKwFunction{FMain}{TPG\_Construction}
    \SetKwProg{Fn}{Function}{:}{}
    \Fn{\FMain{$G$, $txn_{ts}$}}{
        $O_{1..k}$ $\gets$ \textbf{Partition} $txn_{ts}$\;
        \ForEach{\text{operation $O_{i}$ $\in$ $O_{1..k}$}}{
            \textbf{Identify} its \ld\;
            $G$ $\gets$ $G$ + $O_{i}$ \;
        }
    }
    \SetKwFunction{FMain}{TPG\_Refinement}
    \SetKwProg{Fn}{Function}{:}{}
    \Fn{\FMain{$G$}}{
        \ForEach{\text{vertex $e_{i}$ $\in$ $G$}}{
            \textbf{Identify} its \td, \pd\;
        }
    }
    
    \SetKwFunction{FMain}{TXN\_Scheduling}
    \SetKwProg{Fn}{Function}{:}{}
    \Fn{\FMain{$G$}}{
        $M$ $\gets$ Instantiated with $G$;\tcp{A decision model}
        \While{!finish scheduling of $G$
        }{
          \textbf{\circled{2}} $Scheduling Unit$ $\gets$ \textbf{\circled{1}} \emph{Explore}($G$, $M$)\; 
            \textbf{\circled{3}} \emph{Execute with Abort Handling} ($Scheduling Unit$)\; 
        }
    }
  \caption{Execution Outline of \system}
  \label{alg:algo}
\end{algorithm}
\end{comment}

\end{document} 
