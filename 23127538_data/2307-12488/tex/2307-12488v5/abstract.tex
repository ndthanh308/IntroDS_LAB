
\begin{abstract}
The Large Language Models (LLMs), such as GPT and BERT, were proposed for natural language processing (NLP) and have shown promising results as general-purpose language models. An increasing number of industry professionals and researchers are adopting LLMs for program analysis tasks. However, one significant difference between programming languages and natural languages is that a programmer has the flexibility to assign any names to variables, methods, and functions in the program, whereas a natural language writer does not. Intuitively, the quality of naming in a program affects the performance of LLMs in program analysis tasks. This paper investigates how naming affects LLMs on code analysis tasks. Specifically, we create a set of datasets with code containing nonsense or misleading names for variables, methods, and functions, respectively. We then use well-trained models (CodeBERT) to perform code analysis tasks on these datasets. The experimental results show that naming has a significant impact on the performance of code analysis tasks based on LLMs, indicating that code representation learning based on LLMs heavily relies on well-defined names in code. Additionally, we conduct a case study on some special code analysis tasks using GPT, providing further insights.
\end{abstract}
