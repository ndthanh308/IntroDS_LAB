\section{Background}
\label{sec:back}
In the realm of cybersecurity, certain critical problem-solving processes, such as reverse engineering, vulnerability analysis, and bug fixing, still heavily rely on manual efforts from professionals and engineers. For instance, when reverse engineering malware code, engineers often employ numerous tools to aid their investigations. In response to these challenges, there is a strong industry drive to leverage AI and deep learning technologies to reduce manual workload and enhance cybersecurity practices. The advent of deep learning opens up exciting possibilities, enabling accurate detection and prediction of cyber threats, improved agility, cost reduction, enhanced job performance for human analysts, and increased levels of automation for security teams~\cite{CLS20,Xin2018,liu2018alphadiff,pewny2014leveraging,zou2021deep,Tobiyama2016,Raff2017,Ispoglou2016}.

When it comes to security-oriented program analysis, deep learning offers several advantages over traditional methods. Firstly, deep learning requires less domain-specific knowledge, making it more accessible to a broader range of researchers and practitioners. Secondly, the representations learned by a deep learning model can be applied to various downstream tasks, providing versatility and efficiency in the analysis process. Researchers have explored the applications of deep learning in program analysis, categorizing them into two main groups:

\noindent{\bf Deep learning for source code analysis.}
As software inevitably contains bugs, security researchers seek to identify and fix these vulnerabilities or protect programs against potential attackers who may exploit these bugs. Source code level analysis and protection have emerged as one of the most popular approaches to achieve these objectives, primarily because source code contains rich semantic information. Source code analysis finds diverse applications in security, encompassing vulnerability discovery, analysis, fixing, and the implementation of security strategies. Leveraging the comprehensive semantic details encoded in the source code, security professionals can efficiently identify potential weaknesses and take appropriate measures to enhance the software's resilience against threats.

To facilitate source code analysis, pre-trained models like CodeBert and GraphCodeBERT~\cite{feng2020codebert,guo2020graphcodebert} have been developed. These models are based on Transformer architecture and learn code representations through self-supervised training tasks, such as masked language modeling and structure-aware tasks, using a large-scale unlabeled corpus.
Specifically, CodeBERT is pre-trained over six programming languages and is designed to accomplish three main tasks: masked language modeling, code structure edge prediction, and representation alignment. By mastering these tasks, CodeBERT can effectively comprehend and represent the underlying semantics of code, empowering security researchers with a powerful tool for source code analysis and protection.


\noindent{\bf Deep learning for binary analysis.} 
The primary purpose of security-oriented binary code analysis is to extract semantic information from binary files, enabling the resolution of security issues such as malware analysis. Given that many commercial and malicious software do not provide access to their source code, binary code analysis becomes crucial in the realm of cybersecurity. For instance, when dealing with binary-only programs, security strategies like Control Flow Integrity (CFI)~\cite{abadi2009control} may need to be deployed. Binary code analysis is employed in such cases to unveil the hierarchical structures of the binary file, including code sections, functions, and basic blocks.

Binary code analysis finds various applications in cybersecurity. For instance, in the analysis of ransomware and the quest for cryptographic keys, analysts must first generate flow graphs of potentially obfuscated ransomware code. Previous research has utilized Deep Learning (DL) methods for a variety of binary analysis tasks~\cite{chua2017neural,shin2015recognizing}. The primary focus of these works is to learn effective embeddings from binary instructions or raw bytes, subsequently employing a classification output layer to predict the label for the target task. By leveraging DL techniques, researchers aim to improve the accuracy and efficiency of binary code analysis, enabling better detection and understanding of security threats hidden within binary files.

