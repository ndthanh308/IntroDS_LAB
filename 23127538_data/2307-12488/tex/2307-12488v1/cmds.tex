\usepackage{relsize}

\renewcommand{\ttdefault}{pxtt}

\newcommand{\cc}[1]{\mbox{\smaller[0.5]\texttt{#1}}}

\newcommand{\kenali}{\mbox{\textsc{Kenali}}\xspace}
\newcommand{\xmp}{\mbox{\textsc{xMP}}\xspace}
\newcommand{\dynpta}{\mbox{\textsc{DynPTA}}\xspace}

\definecolor{dkgreen}{rgb}{0,0.6,0}
\definecolor{gray}{rgb}{0.5,0.5,0.5}
\definecolor{mauve}{rgb}{0.58,0,0.82}
\definecolor{mygray}{gray}{0.9}
\colorlet{lightblue}{blue!70}
\colorlet{lightred}{red!70}

\lstset{
  language=C,
  frame=tb,
  basicstyle={\scriptsize \ttfamily}, %\scriptsize,\ttfamily,%
  tabsize=3,
  breaklines=true,
  % breakatwhitespace=false,
  showstringspaces=false,
  % columns=fullflexible,
  numbers=left,
  numbersep=-8pt,                     % where to put the line-numbers
  numberstyle=\tiny\color{darkgray},
  escapeinside={(*}{*)},
  xleftmargin=2pt,
  stringstyle=\color{mauve},
  keywordstyle=\color{blue},
  commentstyle=\color{dkgreen} \textit,%\scriptsize \textit,
  %directivestyle={\color{black}},
  %emph={int,char,double,float,unsigned, static, const, if, return, goto},
  emphstyle={\color{red}},
}

\newcommand{\squishlist}{
\begin{itemize}[noitemsep,nolistsep]
  \setlength{\itemsep}{-0pt}
}
\newcommand{\squishend}{
  \end{itemize}
}

\usepackage{url}
\def\UrlBreaks{\do\A\do\B\do\C\do\D\do\E\do\F\do\G\do\H\do\I\do\J
\do\K\do\L\do\M\do\N\do\O\do\P\do\Q\do\R\do\S\do\T\do\U\do\V
\do\W\do\X\do\Y\do\Z\do\[\do\\\do\]\do\^\do\_\do\`\do\a\do\b
\do\c\do\d\do\e\do\f\do\g\do\h\do\i\do\j\do\k\do\l\do\m\do\n
\do\o\do\p\do\q\do\r\do\s\do\t\do\u\do\v\do\w\do\x\do\y\do\z
\do\.\do\@\do\\\do\/\do\!\do\_\do\|\do\;\do\>\do\]\do\)\do\,
\do\?\do\'\do+\do\=\do\#}

\theoremstyle{definition}
\newtheorem{definition}{Definition}

\theoremstyle{definition}
\newtheorem*{explanation}{Explanation}

\theoremstyle{definition}
\newtheorem*{problem}{Problem}

\theoremstyle{definition}
\newtheorem{challenge}{Challenge}

\theoremstyle{definition}
\newtheorem{insight}{Insight}
