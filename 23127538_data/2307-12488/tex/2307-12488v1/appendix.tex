\appendix
% \section{Detail of the Ransomware Mitagation Related Works}
% File analysis includes analyzing the file entropy \cite{scaife2016cryptolock, mehnaz2018rwguard, lee2019machine}, monitoring honeyfiles \cite{gomez2018rlocker}, monitoring I/O operations \cite{mehnaz2018rwguard}, etc. 
% % Essentially a crypto ransomware will encrypt and modify user files, performing a sequence of I/O operations and creating new files containing data with high entropy. 
% Though file analysis based ransomware prevention is often considered to be to late, it is still very effective.

% % Another common strategy is using 
% Information from the system can also be used to detect ransomware, such as program activities, system logs, registry activities, etc. The log of programs' activities can be analyzed by methods such as TF-IDF (term frequency-inverse document frequency) to measure whether a sequence of activities is related to ransomware.
% % For example, \cite{chen2017automated} collects the logs of program activities as documents, and use TF-IDF (term frequency-inverse document frequency) to measure whether a sequence of activities is related to ransomware.

% Similar to system information, API calls are also useful when mitigating the ransomware \cite{qin2020api, javaheri2018detection, kok2019prevention}. Precisely, the sequence of the API calls are useful to detect ransomware. Machine learning is often used here because the patterns in such sequences are usually implicit.
% % It is noteworthy that the API call sequences are often analyzed using machine learning models, as the patterns in such sequences are usually implicit.

% % Another strategy that also involves machine learning is the instruction sequence. 
% Some works \cite{baldwin2018leveraging, khan2020digital} directly analyze the instruction sequence extracted either statically (if possible) or dynamically from the ransomware. Similar to the API call sequences, instruction sequences contain implicit patterns, which usually require machine learning techniques to do effective analysis.

% Finally, it is also common to focus on the network traffic, since ransomware usually need to communicate with the C\&C server. There are works \cite{almashhadani2019multi} detect ransomware by checking the packets, whereas others \cite{cabaj2016using} try to hinder the communication between ransomware and the C\&C servers.
% % For example, \cite{almashhadani2019multi} detect ransomware by analyzing the packets and their flows. \cite{cabaj2016using} designed a system to hinder the communication between the ransomware and the C\&C servers.