\section{Qualitative Analysis on ChatGPT}
ChatGPT is potential to perform source code level analysis and address security-related issues, such as vulnerability discovery and fixing~\cite{surameery2023use,cheshkov2023evaluation}.
Since it is still in the early stage, relevant dataset is not available for a large scale analysis.
We conduct case studies on the ChatGPT with a series of analytical tasks.
Specifically, we first assess ChatGPT's capability to comprehend source code.
Subsequently, we evaluate ChatGPT's proficiency in tackling specific security challenges.

\subsection{Code Semantic Inference}
Accurately analyzing source code not only requires the analyzer to comprehend the literal meaning of variable and function names but also to understand the underlying program logic.
Gaining a precise understanding of the logic poses several challenges to analytical tasks.
Firstly, the analyzer must be capable of tracking the data flow of variables within the code.
Secondly, it needs to possess inter-procedural analysis abilities to comprehend code snippets that involve more than one function.

To assess ChatGPT's performance on source code semantic inference, we conduct experiments where ChatGPT analyzes various code snippets and answers questions related to the program's logic.
Initially, we provided a C/C++ implementation of the Bubble Sort algorithm and inquired about its functionality.
ChatGPT accurately analyzed the program and correctly identified its type, which aligned with our expectations.

Next, we challenged ChatGPT by obfuscating the variable and function names, replacing them with strings devoid of any specific semantic meaning.
This aimed to test whether ChatGPT could still accurately interpret the code's logic despite the lack of meaningful names as shown in Q2.

\colorlet{shadecolor}{red!5}
\begin{shaded*}
\noindent{\bf Q2:} \textit{Please analyze the following algorithm and infer the algorithm type:}  
  {
  \begin{lstlisting}[language=C,
    emph={abc},
    belowskip=-1.2 \baselineskip,
    caption={{\bf Anonymized source code of bubble sorting algirithm.}},label=code:bubble,captionpos=b]
    void abc(int a[], int b){
        int c, d;
        for (c = 0; c < b - 1; c++)
            for (d = 0; d < b - c - 1; d++)
                if (a[d] > a[d + 1]){
                    int e = a[d];
                    a[d] = a[d + 1];
                    a[d + 1] = e;
                }
    }
   \end{lstlisting}
  }
\end{shaded*}
\vspace{-2.5em}
\colorlet{shadecolor}{green!5}
\begin{shaded*}
\setlength{\parindent}{0em}
{\footnotesize 
  \noindent{\bf Reply from ChatGPT:} The given algorithm is a sorting algorithm, and specifically, it is the Bubble Sort algorithm. This algorithm works by repeatedly swapping adjacent elements that are in the wrong order until the array is sorted.

    The algorithm begins by iterating through the array from the first element to the second to last element. For each iteration, it compares each element with the adjacent element to its right and swaps them if they are in the wrong order. This process continues until the end of the array is reached, and the largest element is placed in its correct position. The algorithm then repeats the process, iterating through the array again from the first element to the second to last element and swapping adjacent elements that are in the wrong order until the entire array is sorted.
  }
\end{shaded*} 



\autoref{code:bubble} is anonymized on function name and variable names in case ChatGPT can infer the algorithm from literal meaning of the names.
The response from ChatGPT in this case is accurate, indicating its ability to learn the logic features from the provided source code. Encouraged by this result, we proceeded to challenge ChatGPT further by splitting the algorithm into three functions and prompting it to infer the overall algorithm.
Specifically, we deliberately test ChatGPT's inter-procedural analysis capabilities.
This evaluation aimed to determine if ChatGPT could effectively understand and connect the logic across multiple functions within the code.

\colorlet{shadecolor}{red!5}
\begin{shaded*}
\noindent{\bf Q3:} \textit{Please analyze the following algorithm and infer the algorithm type:}  
  {
  \begin{lstlisting}[language=C++,
    emph={lmn,efg,abc},
    belowskip=-1.2 \baselineskip,
    caption={{\bf Source code of bubble sorting algirithm.} We split the algorithm into three functions.},label=code:split,captionpos=b]
    void lmn(int * a, int *b){
        if ( *a > *b){
            int c = *a;
            *a = *b;
            *b = c;
        }
    }
    void efg(int a[], int c, int b){
        int d;
        for (d = 0; d < b - c - 1; d++){
            lmn( & a[d], & a[d+1]); 
        }        
    }
    void abc(int a[], int b)
    {
        int c, d;
        for (c = 0; c < b - 1; c++){
            efg(a, c, b);
        }
    }
\end{lstlisting}
  }
\end{shaded*}
\vspace{-2.5em}
\colorlet{shadecolor}{green!5}
\begin{shaded*}
\setlength{\parindent}{0em}
{\footnotesize 
  \noindent{\bf Reply from ChatGPT:} The given algorithm is implementing the Bubble Sort algorithm. It has three functions, lmn, efg, and abc.

  The lmn function swaps the values of two integers if the first integer is greater than the second. It takes two pointers to integers as input and returns void.
  
  The efg function is responsible for performing one pass of the bubble sort algorithm. It takes an array a, starting index c, and ending index b as inputs. Within this function, it calls the lmn function to compare and swap adjacent elements in the subarray from index c to b-1.
  
  The abc function is the main function implementing the bubble sort algorithm. It takes an array a and its length b as inputs. It calls the efg function b-1 times with different starting and ending indices in each pass to sort the entire array.
  
  Therefore, the algorithm type is Bubble Sort.
}
\end{shaded*} 

Upon careful evaluation of ChatGPT's response, we can confidently conclude that all details provided are accurate.
The answer demonstrates ChatGPT's capability to analyze the data flow across multiple functions, showing its potential for inter-procedural analysis.
This promising result suggests that ChatGPT has the ability to perform complex tasks involving the understanding and inference of program logic across different functions.

\subsection{Vulnerabilities Analysis and Bug Fixing}
In the preceding section, we highlight ChatGPT's proficiency in analyzing program semantics and logic.
In this section, our attention shifts to several specific security problems, where we put ChatGPT's problem-solving ability to the test.
We aim to assess how effectively ChatGPT can address these security challenges and leverage its potential in security-oriented program analysis.

\colorlet{shadecolor}{red!5}
\begin{shaded*}
\noindent{\bf Q4:} \textit{Does this piece of code have vulnerability?} 
  {
  \begin{lstlisting}[language=C++,
    emph={strcpy,printf,main},
    belowskip=-1.2 \baselineskip,
    caption={A piece of code that contains a buffer overflow vulnerability.},label=code:overflow,captionpos=b]
    int main(int argc, char *argv[])
    {
        char buffer[5];  
        strcpy(buffer, argv[1]);
        printf("buffer content= %s\n", buffer);
        return 0;
    }
\end{lstlisting}
  }
\end{shaded*}
\vspace{-2.5em}
\colorlet{shadecolor}{green!5}
\begin{shaded*}
\setlength{\parindent}{0em}
{\footnotesize 
  \noindent{\bf Reply from ChatGPT:} Yes, this piece of code has a vulnerability known as a buffer overflow.

    The variable buffer is defined as an array of 5 characters, which means it can hold up to 4 characters plus the null terminator. However, the program copies the content of argv[1] into buffer using the \texttt{strcpy} function, which does not check if the source string is larger than the destination buffer.
    
    If argv[1] contains more than 4 characters (or the null terminator), \texttt{strcpy} will write past the end of buffer, potentially overwriting other important data on the stack. This can cause unpredictable behavior, crashes, and even security vulnerabilities like code injection.
  }\end{shaded*} 


The analysis provided by ChatGPT has been remarkably accurate thus far.
However, in order to further challenge its capabilities, we decided to replace the \texttt{strcpy} function with a home-made string copy function.
Subsequently, we pose the same question to ChatGPT to observe how well it can handle this modified scenario.

\colorlet{shadecolor}{red!5}
\begin{shaded*}
\noindent{\bf Q5:} \textit{Does this piece of code have vulnerability?} 
  {\begin{lstlisting}[language=C++,
    emph={abc,printf,main},
    belowskip=-1.2 \baselineskip,
    caption={A piece of code that contains a buffer overflow vulnerability.},label=code:overflow2,captionpos=b]
    char* abc(char* def, const char* hij)
    {
        if (def == NULL) {
            return NULL;
        }
        char *ptr = def;
        while ( *hij != 0)
        {
            *def = *hij;
            def++;
            hij++;
        }
        *def = 0;
        return ptr;
    }
    int main(int argc, char *argv[])
    {
        char xyz[5];  
        abc(xyz, argv[1]);
        printf("xyz content= %s\n", xyz);
        return 0;
    }
\end{lstlisting}
  }
\end{shaded*}
\vspace{-2.5em}
\colorlet{shadecolor}{green!5}
\begin{shaded*}
\setlength{\parindent}{0em}
{\footnotesize 
  \noindent{\bf Reply from ChatGPT:} 
    Yes, the provided code contains a vulnerability related to the use of the abc function and the potential for buffer overflow. The vulnerable code is in the abc function.

    The issue arises from the fact that there is no check on the size of the destination buffer (\texttt{def}) before copying data from the source (\texttt{hij}). The function uses a simple loop to copy characters from hij to def until it reaches the null-terminator (`0'). However, it does not verify whether the destination buffer has enough space to hold the entire source string.

    In the main function, the \texttt{xyz} buffer is declared with a size of 5 characters. If the user provides a command-line argument (argv[1]) longer than 4 characters (since one byte is reserved for the null-terminator), the abc function will copy the characters beyond the bounds of the xyz buffer, leading to a buffer overflow. This behavior results in undefined behavior, potentially overwriting adjacent memory or causing segmentation faults.

    To fix this vulnerability, you should modify the abc function to take the size of the destination buffer as an additional argument and ensure that it does not exceed the buffer size during the copying process. Alternatively, you can use safer string functions like 
    \texttt{strncpy} or \texttt{memcpy} that allow specifying the size of the destination buffer to avoid buffer overflows.
  }\end{shaded*} 

The analysis conducted by ChatGPT on the vulnerability in the code was impressively accurate and detailed, showing its potential to identify vulnerabilities in source code effectively.
However, in an effort to explore its understanding of potential security risks, we posed another question, asking ChatGPT how one could exploit this vulnerability to launch an attack.

\colorlet{shadecolor}{red!5}
\begin{shaded*}
\noindent{\bf Q6:} \textit{How to leverage this vulnerability to launch attack? }
\end{shaded*}
\vspace{-2.5em}
\colorlet{shadecolor}{green!5}
\begin{shaded*}
\setlength{\parindent}{0em}
{\footnotesize 
  \noindent{\bf Reply from ChatGPT:} 
    This program has a buffer overflow vulnerability, which can be exploited by an attacker to gain control of the program execution and possibly execute malicious code. Here's an example of how an attacker could leverage this vulnerability to launch an attack: 
    
    The attacker crafts a malicious input string that is longer than 5 bytes, such as 
    \begin{lstlisting}[language=C++,
        emph={strcpy,strlen,main,printf},
        label=code:bit11,captionpos=b, numbers=none, frame=none]
"AAAAAAAAAAAAAAAAAAAAAAAAAAAAAAAABBBB"
    \end{lstlisting}
    The attacker runs the program with the malicious input as a command-line argument, like this: 
    \begin{lstlisting}[language=C++,
        emph={strcpy,strlen,main,printf},
        label=code:bit12,captionpos=b, numbers=none, frame=none]
"./program AAAAAAAAAAAAAAAAAAAAAAAAAAAAAAAABBBB" 
    \end{lstlisting}
    When the program executes the \texttt{strcpy} function with the malicious input, it copies the entire string (including the "BBBB" part) into the buffer, overwriting the return address on the stack. The attacker's code in the "BBBB" part of the input string gets executed instead of the intended code, giving the attacker control of the program execution and potentially allowing them to execute arbitrary code or launch further attacks.
  }\end{shaded*} 


The exploit proposed by ChatGPT, although not directly usable to launch attacks, was conceptually correct.
This suggests that ChatGPT possesses the ability to assist attackers in generating potential exploits.
In the subsequent step, we investigate whether ChatGPT can also provide solutions to fix the identified vulnerability.


\colorlet{shadecolor}{red!5}
\begin{shaded*}
\noindent{\bf Q7:} \textit{Can you rewrite the code to avoid the vulnerability?}
  { }
  \end{shaded*}
\vspace{-2.5em}
\colorlet{shadecolor}{green!5}
\begin{shaded*}
\setlength{\parindent}{0em}
{\footnotesize 
  \noindent{\bf Reply from ChatGPT:} 
Yes, we can rewrite the code to avoid the buffer overflow vulnerability. Here's an example:
\begin{lstlisting}[language=C++,
    emph={strcpy,strlen,main,printf},
    caption={{\bf Source code of bubble sorting algirithm.} We not only anonymized the name of function and variable name but also split the algorithm into three functions.},label=code:bit13,captionpos=b]
    int main(int argc, char *argv[]) {
        char buffer[BUFFER_SIZE];
        int len;
        if (argc < 2) {
            printf("Usage: %s <string>\n", argv[0]);
            return 1;
        }
        len = strlen(argv[1]);
        if (len >= BUFFER_SIZE) {
            printf("Input string is too long\n");
            return 1;
        }
        strcpy(buffer, argv[1]);
        printf("buffer content = %s\n", buffer);
        return 0;
    }
\end{lstlisting}

In this version of the code, we have defined a constant BUFFER\_SIZE to set the maximum size of the buffer. Before copying the input string to the buffer, we check its length to ensure it will not overflow the buffer. If the input string is too long, we print an error message and exit the program. This way, we ensure that the buffer will never be overflowed.
  }\end{shaded*} 



% \noindent{\bf Q7:} \textit{Can you rewrite the code to avoid the vulnerability?}
% \begin{shaded*}
%     \footnotesize
% \noindent{\bf Reply from ChatGPT:} 
% Yes, we can rewrite the code to avoid the buffer overflow vulnerability. Here's an example:
% \begin{lstlisting}[language=C++,
%     emph={strcpy,strlen,main,printf},
%     caption={Source code of bubble sorting algirithm. We not only anonymized the name of function and variable name but also split the algorithm into three functions.},label=code:bit,captionpos=b]
%     int main(int argc, char *argv[]) {
%         char buffer[BUFFER_SIZE];
%         int len;
%         if (argc < 2) {
%             printf("Usage: %s <string>\n", argv[0]);
%             return 1;
%         }
%         len = strlen(argv[1]);
%         if (len >= BUFFER_SIZE) {
%             printf("Input string is too long\n");
%             return 1;
%         }
%         strcpy(buffer, argv[1]);
%         printf("buffer content = %s\n", buffer);
%         return 0;
%     }
% \end{lstlisting}

% In this version of the code, we have defined a constant BUFFER\_SIZE to set the maximum size of the buffer. Before copying the input string to the buffer, we check its length to ensure it will not overflow the buffer. If the input string is too long, we print an error message and exit the program. This way, we ensure that the buffer will never be overflowed.
% \end{shaded*}


Our evaluation has demonstrated that the proposed patching effectively addresses the vulnerability in the code, providing an effective fix.
As a result, we believe that ChatGPT holds the potential to assist developers and security analysts in identifying and resolving vulnerabilities in programs.

In the subsequent step, we assess its capability to suggest protection schemes that can thwart potential exploits by attackers.
By leveraging ChatGPT's insights and analysis, we aim to explore its ability to propose robust security measures that can safeguard against known vulnerabilities and potential attack vectors.


\colorlet{shadecolor}{red!5}
\begin{shaded*}
\noindent{\bf Q8:} \textit{Can you propose a method to detect the vulnerability at program running time?}
  {}
  \end{shaded*}
\vspace{-2.5em}
\colorlet{shadecolor}{green!5}
\begin{shaded*}
\setlength{\parindent}{0em}
{\footnotesize 
  \noindent{\bf Reply from ChatGPT:} 
One method to detect the vulnerability at program running time is to use a technique called ``canary values'' or ``stack cookie''. This technique involves adding a random value (the ``canary'') to the stack before the buffer, and checking that the canary has not been overwritten when the function returns.

To implement this technique, we can modify the original code as follows:
\begin{lstlisting}[language=C++,
    emph={memcpy,strlen,main,printf},
    label=code:canary,captionpos=b]
    int main(int argc, char *argv[]){
        char buffer[5];
        int canary = rand();
        printf("Canary value: %d\n", canary);
        memcpy(buffer, &canary, sizeof(canary));
        memcpy(buffer + sizeof(canary), argv[1], strlen(argv[1]) + 1);
        if (canary != *(int*)buffer) {
            printf("Buffer overflow detected!");
        } else {
            printf("buffer content = %s\n", buffer + sizeof(canary));
        }
        return 0;
    }
\end{lstlisting}
In this modified code, we add a random integer canary value to the stack before the buffer. We then check if the canary value has been modified after the user input is copied into the buffer. If the canary value has been modified, it indicates that a buffer overflow has occurred, and we can take appropriate action, such as terminating the program or alerting the user.

Note that this technique is not foolproof and can be bypassed in certain cases. It is still important to ensure that buffers are properly sized and user input is properly validated to avoid buffer overflow vulnerabilities in the first place.
  }\end{shaded*} 





% \noindent{\bf Q8:} \textit{Can you propose a method to detect the vulnerability at program running time?}

% \begin{shaded*}
%     \footnotesize
% \noindent{\bf Reply from ChatGPT:} 
% One method to detect the vulnerability at program running time is to use a technique called ``canary values'' or ``stack cookie''. This technique involves adding a random value (the ``canary'') to the stack before the buffer, and checking that the canary has not been overwritten when the function returns.

% To implement this technique, we can modify the original code as follows:
% \begin{lstlisting}[language=C++,
%     emph={memcpy,strlen,main,printf},
%     label=code:canary,captionpos=b]
%     int main(int argc, char *argv[]){
%         char buffer[5];
%         int canary = rand();
%         printf("Canary value: %d\n", canary);
%         memcpy(buffer, &canary, sizeof(canary));
%         memcpy(buffer + sizeof(canary), argv[1], strlen(argv[1]) + 1);
%         if (canary != *(int*)buffer) {
%             printf("Buffer overflow detected!");
%         } else {
%             printf("buffer content = %s\n", buffer + sizeof(canary));
%         }
%         return 0;
%     }
% \end{lstlisting}
% In this modified code, we add a random integer canary value to the stack before the buffer. We then check if the canary value has been modified after the user input is copied into the buffer. If the canary value has been modified, it indicates that a buffer overflow has occurred, and we can take appropriate action, such as terminating the program or alerting the user.

% Note that this technique is not foolproof and can be bypassed in certain cases. It is still important to ensure that buffers are properly sized and user input is properly validated to avoid buffer overflow vulnerabilities in the first place.

% \end{shaded*}



The proposed protection scheme suggested by ChatGPT demonstrates a conceptually correct approach in detecting buffer overflows through the use of a ``canary''.
However, upon closer inspection, we identify some implementation errors that may impact its effectiveness.
The placement of the canary at the beginning of the buffer, instead of at the end, renders it unable to detect buffer overflows that overwrite adjacent variables located at higher memory addresses.
Additionally, the revised code introduces the ``canary'' into the buffer without correspondingly increasing the buffer size, limiting its capacity to handle strings as long as the original implementation.

These issues raise concerns regarding ChatGPT's generalization ability.
While ChatGPT appears to understand the concept of using a ``canary'' for buffer overflow detection, it seems to struggle in correctly applying this knowledge to specific code snippets that were not encountered during its training.
This observation leads us to question whether ChatGPT has effectively learned the intricacies of canary-based detection schemes from its training corpus, yet lacks the capability to seamlessly apply this knowledge to novel scenarios.

To further investigate ChatGPT's generalization capabilities, we proceed to test its performance on a diverse set of unseen code snippets in the subsequent section.
This evaluation aims to shed light on the extent to which ChatGPT can effectively apply its knowledge to previously unseen code and discern its ability to adapt its understanding across different contexts.
Understanding ChatGPT's strengths and limitations in generalization is crucial in maximizing its potential for security-oriented program analysis and generating reliable solutions in a broader range of scenarios.

\subsection{Generalization Ability in Code Review}
\label{sec:gene}

Some programmers could have adopt ChatGPT to perform code review, code optimization, vulnerability mining complexity analysis, and code rewriting.
% ~\fixme{(Do we have some references? If so, the above assessment duplicates the work in these references?)}.
It possesses the capability to generate syntactically correct code and offer valuable insights and suggestions in certain scenarios.
However, the extent to which its generated content relies solely on the training dataset remains uncertain, considering the massive amount of data it was trained on.

In this section, we aim to investigate whether ChatGPT can effectively analyze and comprehend code that was not included in its training dataset.
%\fixme{(how can we ensure leetcode is not used in the dataset?) ZL: Leecode is in ChatGPT's training set, the obfuscated program is not.}.
To evaluate this, we randomly select four code snippets from LeetCode, representing varying levels of difficulty, and assess ChatGPT's performance on the original code snippets and the corresponding obfuscated code snippets.

Table \ref{table:cases} presents the LeetCode problems we choose and the corresponding solutions, which will serve as the dataset for our evaluation.
By analyzing ChatGPT's responses to these unseen code snippets, we seek to gain insights into its generalization capabilities and understand how effectively it can apply its learned knowledge to new code scenarios.
Our findings shed light on the robustness of ChatGPT's source code analysis ability and provide valuable information for its potential applications in real-world scenarios beyond the confines of its training data.

\begin{table}[htbp]
\centering
\footnotesize
\caption{Selected code snippets.}
\label{table:cases}
% \begin{adjustbox}{max width=\columnwidth}
\begin{threeparttable}
\begin{tabular}{c|c|c|c|c}
\toprule
Leetcode ID  & Code~\tnote{1}  & Type                & Difficulty & Language  \\ \midrule
\href{https://leetcode.com/problems/super-egg-drop/}{887} & \href{https://github.com/Moirai7/ChatGPTforSourceCode/blob/main/SupperEggDrop.cpp}{SupperEggDrop} & Dynamic Programming & Hard       & C++  \\ \midrule
  \href{https://leetcode.com/problems/disconnect-path-in-a-binary-matrix-by-at-most-one-flip/}{2556}           &       \href{https://github.com/Moirai7/ChatGPTforSourceCode/blob/main/DisconnectPath.cpp}{isPossibleToCutPath}        &       Depth-First Search              &     Medium       &    C++         \\ \midrule
\href{https://leetcode.com/problems/find-subsequence-of-length-k-with-the-largest-sum/}{2099} & \href{https://github.com/Moirai7/ChatGPTforSourceCode/blob/main/maxSubsequence.java}{maxSubsequence} & Heap & Easy       & Java \\ \midrule
\href{https://leetcode.com/problems/car-fleet/description/}{853} & \href{https://github.com/Moirai7/ChatGPTforSourceCode/blob/main/carFleet.py}{carFleet} & Array & Medium       & Python  \\ 
\bottomrule
\end{tabular}
\begin{tablenotes}
    \scriptsize
    \item[1] The source code is available at: \url{https://github.com/Moirai7/ChatGPTforSourceCode}.
  \end{tablenotes}
\end{threeparttable}
% \end{adjustbox}
\end{table}

ChatGPT can correctly understand the original code snippets.
Taking the Supper Egg Drop problem as an example, it explains the purpose of the code snippet, the logic and complexity of the algorithm, and provides suggestions to improve the code. 
% \fixme{(where is the bellowing code in Q9?)}


\colorlet{shadecolor}{red!5}
\begin{shaded*}
\noindent{\bf Q9: Could you review the bellowing code?}
  {\begin{lstlisting}[language=C++,
    emph={abc,printf,main},
    % abovecaptionskip=0pt,
    % belowcaptionskip=0pt,
    belowskip=-1.2 \baselineskip,
    caption={A solution to \texttt{Supper Egg Drop} problem from LeetCode.},
    label=code:eggoriginal,captionpos=b]
    int superEggDrop(int K, int N) {
        vector<vector<int>> dp(N + 1, vector<int>(K + 1, 0));
        int m = 0;
        while (dp[m][K] < N) {
            m++;
            for (int k = 1; k <= K; ++k)
                dp[m][k] = dp[m - 1][k - 1] + dp[m - 1][k] + 1;
        }
        return m;
    }
  \end{lstlisting}
  }
  \end{shaded*}
\vspace{-2.5em}
\colorlet{shadecolor}{green!5}
\begin{shaded*}
\setlength{\parindent}{0em}
{\footnotesize 
  \noindent{\bf Reply from ChatGPT:} 
The code appears to be a dynamic programming solution to the  ``egg drop'' problem, where one has $K$ eggs and a building with $N$ floors, and the objective is to find the minimum number of drops required to determine the floor from which the egg breaks.

The code creates a 2D vector dp with dimensions $(N+1) \times (K+1)$ and initializes all elements to $0$. It then iterates through values of $m$ (starting from $1$) until the value of $dp[m][K]$ is greater than or equal to $N$. For each value of $m$, it updates the values of $dp[m][k]$ for each egg $(k)$, based on the recurrence relation $dp[m][k] = dp[m-1][k-1] + dp[m-1][k] + 1$. Finally, it returns the value of $m$.

Overall, the code looks correct and efficient. It correctly computes the minimum number of drops required to determine the floor from which the egg breaks. The use of dynamic programming allows the algorithm to avoid unnecessary computations and runs in $O(K \times N)$ time, which is optimal for this problem.

However, it should be noted that this solution is not the only one to the egg drop problem, and there are other approaches that can be taken to solve it. Also, the code could benefit from some comments to explain the purpose and logic of the algorithm.
  }\end{shaded*} 

% \noindent{\bf Q9: Could you review the bellowing code?} 
% % \textit{Please analyze the algorithm and infer the program logic.}
% \begin{shaded*}
%     \footnotesize
% \noindent{\bf Reply from ChatGPT:} 
% The code appears to be a dynamic programming solution to the  ``egg drop'' problem, where one has $K$ eggs and a building with $N$ floors, and the objective is to find the minimum number of drops required to determine the floor from which the egg breaks.

% The code creates a 2D vector dp with dimensions $(N+1) \times (K+1)$ and initializes all elements to $0$. It then iterates through values of $m$ (starting from $1$) until the value of $dp[m][K]$ is greater than or equal to $N$. For each value of $m$, it updates the values of $dp[m][k]$ for each egg $(k)$, based on the recurrence relation $dp[m][k] = dp[m-1][k-1] + dp[m-1][k] + 1$. Finally, it returns the value of $m$.

% Overall, the code looks correct and efficient. It correctly computes the minimum number of drops required to determine the floor from which the egg breaks. The use of dynamic programming allows the algorithm to avoid unnecessary computations and runs in $O(K \times N)$ time, which is optimal for this problem.

% However, it should be noted that this solution is not the only one to the egg drop problem, and there are other approaches that can be taken to solve it. Also, the code could benefit from some comments to explain the purpose and logic of the algorithm.
% \end{shaded*}
{
  \begin{lstlisting}[language=C++,
      emph={abc,printf,main},
      caption={\textbf{Obfuscated code based on \autoref{code:eggoriginal}}. We renaming the variables and function names and insert dummy code that will not have a effect on its results.},label=code:obfuscated,captionpos=b]
  // obfuscated code snippet
  int gyjdfgadfsfvcvjhyt(int kdsfjaldfie55qwer, int kdsfja) {
      vector<vector<int>> kdasd35sfja(kdsfja + 1, vector<int>(kdsfjaldfie55qwer + 1, 0));
      int kdasd35sfjada = 0;
      int af567856dfg = kdasd35sfja.size(), abf567856dfg = kdasd35sfja[0].size(),
          abf5678s56dfg = 0, abf5678sss56dfg = abf567856dfg - 1;
          int abfss556dfg =0;

      while (kdasd35sfja[kdasd35sfjada][kdsfjaldfie55qwer] < kdsfja) {
          while (abf5678s56dfg < af567856dfg && abf5678sss56dfg > -1) {
              int abf556dfg = kdasd35sfja[abf5678s56dfg][abf5678sss56dfg];
              if (abf556dfg == abfss556dfg) break;
              if (abfss556dfg > abf556dfg) abf5678s56dfg++;
              else abf5678sss56dfg--;
          }
          kdasd35sfjada++;
          for (int sdf23423 = 1; sdf23423 <= kdsfjaldfie55qwer; ++sdf23423)
              kdasd35sfja[kdasd35sfjada][sdf23423] = kdasd35sfja[kdasd35sfjada - 1][sdf23423 - 1] + kdasd35sfja[kdasd35sfjada - 1][sdf23423] + 1;
      }
      return kdasd35sfjada;
  }
  \end{lstlisting}
}

Then, we manually obfuscated the code snippets by renaming the variables and function names and insert dummy code that will not have an impact on its results.
The obfuscated code is shown in \autoref{code:obfuscated}.
ChatGPT can not provide useful review of the obfuscated code.
The bellowing is the generated review with the obfuscated code.

\colorlet{shadecolor}{red!5}
\begin{shaded*}
\noindent{\bf Q10:} \textit{Could you review~\ref{code:obfuscated}?}
  {}
  \end{shaded*}
\vspace{-2.5em}
\colorlet{shadecolor}{green!5}
\begin{shaded*}
\setlength{\parindent}{0em}
{\footnotesize 
  \noindent{\bf Reply from ChatGPT:} 
The code you have provided appears to be written in C++ and defines a function called ``gyjdfgadfsfvcvjhyt''. Unfortunately, the function name and parameters do not give any indication of what the function is supposed to do. It would be helpful to have some context or explanation of the intended purpose of this function.

Regarding the code itself, there are some issues with readability and style that could be improved. For example, the use of single-letter variable names (\eg, ``kdsfjaldfie55qwer'', ``kdasd35sfja'') can make it difficult for someone reading the code to understand what is going on. Additionally, the inconsistent use of whitespace and indentation can make the code more difficult to read.

It's also worth noting that the code appears to be using nested while loops, which can be difficult to follow and potentially lead to performance issues if the loops are executing a large number of times. Without more context, it's difficult to say whether this is a problem in this particular case.

Finally, there are no comments or documentation explaining what the code is doing, which could make it difficult for other developers to understand and modify the code in the future.

Overall, I would recommend improving the readability and style of the code, adding comments and documentation, and providing more context and explanation of the intended purpose of the function.
  }\end{shaded*} 


% \noindent{\bf Q10:} \textit{Could you review the bellowing code?}
% \begin{shaded*}
%     \footnotesize
% \noindent{\bf Reply from ChatGPT:} 
% The code you have provided appears to be written in C++ and defines a function called ``gyjdfgadfsfvcvjhyt''. Unfortunately, the function name and parameters do not give any indication of what the function is supposed to do. It would be helpful to have some context or explanation of the intended purpose of this function.

% Regarding the code itself, there are some issues with readability and style that could be improved. For example, the use of single-letter variable names (\eg, ``kdsfjaldfie55qwer'', ``kdasd35sfja'') can make it difficult for someone reading the code to understand what is going on. Additionally, the inconsistent use of whitespace and indentation can make the code more difficult to read.

% It's also worth noting that the code appears to be using nested while loops, which can be difficult to follow and potentially lead to performance issues if the loops are executing a large number of times. Without more context, it's difficult to say whether this is a problem in this particular case.

% Finally, there are no comments or documentation explaining what the code is doing, which could make it difficult for other developers to understand and modify the code in the future.

% Overall, I would recommend improving the readability and style of the code, adding comments and documentation, and providing more context and explanation of the intended purpose of the function.
% \end{shaded*}

In essence, ChatGPT learns the literal features of code snippets that exist within its training data and generates responses based on statistical patterns involving characters, keywords, and syntactic structures.
This knowledge encompasses various aspects like indentation, syntax, and common coding conventions.
However, it lacks a profound comprehension of code semantics, logic, or context beyond the patterns and examples present in its training data.
This limitation means that ChatGPT may not possess the ability to precisely review or assess the correctness, efficiency, or security of code without a more comprehensive understanding of its underlying semantics.


\vspace{-0.5em}\subsection{Input Generation}
Input generation is wide used in software testing, which aim to achieve a high coverage by triggering different program pathes.
This section, we aim to answer \ding{188} whether ChatGPT can offer valuable and effective seed inputs after understanding the source code.
Specifically, we test the ChatGPT's input generatation ability to achieve code coverage, input mutations, and  exeuction simulation.

Upon analyzing the responses ({\bf \ref{appendix:Q11}, \ref{appendix:Q12}, \ref{appendix:Q13}}) from ChatGPT, we observed that the provided seeds were comparable to those generated by AFL (American Fuzzy Lop) \cite{aflrefurl} using seed mutation strategies. However, it became apparent that ChatGPT did not propose highly effective seeds (capable of triggering the vulnerability). This observation suggests that ChatGPT may face challenges in leveraging the program semantics it learned from the code to generate such effective seeds.
There could be two possible reasons for this:

\begin{enumerate}
\item The prompts we use may not have been designed to specifically encourage ChatGPT to produce the most useful and targeted solutions.\lan{remove this. if it is true, we could say our previous case studies are not correct because of prompts. this is out-of-scope}
\item ChatGPT's responses might be too general, primarily relying on the information present in the corpus it was trained on, rather than effectively applying the learned program semantics to new situations.
\end{enumerate}

\lan{I vote to remove this. then we only have to kinds of case study: semantic and vunlerbiltiy. it would be easier to explain the storyline}
%Understanding these limitations is crucial for utilizing ChatGPT effectively in vulnerability discovery and other security-oriented tasks. Further research is needed to refine the prompts and explore strategies to enhance ChatGPT's ability to provide more precise and context-aware responses in security-related scenarios.



\subsection{Assemble Code Analysis}
In this section, we investigate the ChatGPT's ability to perform the binary code analysis.
We first test ChatGPT's ability on binary disassembling, and then ask questions about the semantic of the code.  


\lstset{language=[x64]Assembler}

\colorlet{shadecolor}{red!5}
\begin{shaded*}
\noindent{\bf Q14:} \textit{The disassemble code of \texttt{`c7 45 cc 0a 00 00 00 c7 45 c8 14 00 00 00 8b 75 cc 03 75 c8 89 75 c4'} in Intel 64 instruction set?}
  { }
  \end{shaded*}
\vspace{-2.5em}
\colorlet{shadecolor}{green!5}
\begin{shaded*}
\setlength{\parindent}{0em}
{\footnotesize 
  \noindent{\bf Reply from ChatGPT:} 
\begin{lstlisting}[emph={},
	label=code:assem,captionpos=b, numbers=none, frame=none]
    c7 45 cc 0a 00 00 00     mov     DWORD PTR [rbp-0x34], 0xa
    c7 45 c8 14 00 00 00     mov     DWORD PTR [rbp-0x38], 0x14
    8b 75 cc                 mov     esi, DWORD PTR [rbp-0x34]
    03 75 c8                 add     esi, DWORD PTR [rbp-0x38]
    89 75 c4                 mov     DWORD PTR [rbp-0x3c], esi
    
\end{lstlisting}
This is the disassembled code in Intel 64 instruction set:
\begin{enumerate}
\item \texttt{mov DWORD PTR [rbp-0x34], 0xa} ~\textendash~ Moves the immediate value \texttt{0xa} into the double-word (4 bytes) memory location at \texttt{[rbp-0x34]}.

\item \texttt{mov DWORD PTR [rbp-0x38], 0x14} ~\textendash~ Moves the immediate value \texttt{0x14} into the double-word memory location at \texttt{[rbp-0x38]}.

\item \texttt{mov esi, DWORD PTR [rbp-0x34]} ~\textendash~ Moves the double-word value from memory location \texttt{[rbp-0x34]} into the \texttt{esi} register.

\item \texttt{add esi, DWORD PTR [rbp-0x38]} ~\textendash~ Adds the double-word value from memory location \texttt{[rbp-0x38]} to the \texttt{esi} register.

\item \texttt{mov DWORD PTR [rbp-0x3c], esi} ~\textendash~ Moves the value in the esi register into the double-word memory location at \texttt{[rbp-0x3c]}.
\end{enumerate}
  }\end{shaded*} 

% \noindent{\bf Q14:} \textit{The disassemble code of \texttt{`c7 45 cc 0a 00 00 00 c7 45 c8 14 00 00 00 8b 75 cc 03 75 c8 89 75 c4'} in Intel 64 instruction set?}

% \begin{shaded*}
%     \footnotesize
% \noindent{\bf Reply from ChatGPT:} 
% \begin{lstlisting}[emph={},
% 	label=code:assem,captionpos=b, numbers=none, frame=none]
%     c7 45 cc 0a 00 00 00     mov     DWORD PTR [rbp-0x34], 0xa
%     c7 45 c8 14 00 00 00     mov     DWORD PTR [rbp-0x38], 0x14
%     8b 75 cc                 mov     esi, DWORD PTR [rbp-0x34]
%     03 75 c8                 add     esi, DWORD PTR [rbp-0x38]
%     89 75 c4                 mov     DWORD PTR [rbp-0x3c], esi
    
% \end{lstlisting}
% This is the disassembled code in Intel 64 instruction set:
% \begin{enumerate}
% \item \texttt{mov DWORD PTR [rbp-0x34], 0xa} ~\textendash~ Moves the immediate value \texttt{0xa} into the double-word (4 bytes) memory location at \texttt{[rbp-0x34]}.

% \item \texttt{mov DWORD PTR [rbp-0x38], 0x14} ~\textendash~ Moves the immediate value \texttt{0x14} into the double-word memory location at \texttt{[rbp-0x38]}.

% \item \texttt{mov esi, DWORD PTR [rbp-0x34]} ~\textendash~ Moves the double-word value from memory location \texttt{[rbp-0x34]} into the \texttt{esi} register.

% \item \texttt{add esi, DWORD PTR [rbp-0x38]} ~\textendash~ Adds the double-word value from memory location \texttt{[rbp-0x38]} to the \texttt{esi} register.

% \item \texttt{mov DWORD PTR [rbp-0x3c], esi} ~\textendash~ Moves the value in the esi register into the double-word memory location at \texttt{[rbp-0x3c]}.
% \end{enumerate}
% \end{shaded*}

As shown in Q14, ChatGPT can correctly disassemble Intel x64 binary code. 
In the next question, we ask about the semantic information about this assemble code. 


\colorlet{shadecolor}{red!5}
\begin{shaded*}
\noindent{\bf Q15:} \textit{
After the execution of intel64 machine code \texttt{`c7 45 cc 0a 00 00 00 c7 45 c8 14 00 00 00 8b 75 cc 03 75 c8 89 75 c4'}. What is the value in register \texttt{`\%esi'}, memory \texttt{`rbp-0x34'} and memory \texttt{`rbp-0x3c'}?}
  { }
  \end{shaded*}
\vspace{-2.5em}
\colorlet{shadecolor}{green!5}
\begin{shaded*}
\setlength{\parindent}{0em}
{\footnotesize 
  \noindent{\bf Reply from ChatGPT:} 
Let's simulate the execution of the provided Intel 64 machine code step by step to determine the values in register \texttt{\%esi}, memory location \texttt{rbp-0x34}, and memory location \texttt{rbp-0x3c} after the execution.

Assuming that the initial values of the relevant registers and memory locations are 0, here's the execution:
\begin{enumerate}
\item \texttt{mov DWORD PTR [rbp-0x34], 0xa} ~\textendash~ Moves the immediate value \texttt{0xa} into the memory location \texttt{[rbp-0x34]}.

Value at \texttt{[rbp-0x34]: 0x0000000a}
\item \texttt{mov DWORD PTR [rbp-0x38], 0x14} ~\textendash~ Moves the immediate value \texttt{0x14} into the memory location [rbp-0x38].
Value at \texttt{[rbp-0x38]: 0x00000014}
\item \texttt{mov esi, DWORD PTR [rbp-0x34]} ~\textendash~ Moves the double-word value from memory location \texttt{[rbp-0x34]} into the \texttt{\%esi} register.

Value in \texttt{\%esi: 0x0000000a}
\item \texttt{add esi, DWORD PTR [rbp-0x38]} ~\textendash~ Adds the double-word value from memory location [rbp-0x38] to the \texttt{\%esi} register.

Value in \texttt{\%esi: 0x0000001e (0x0a + 0x14 = 0x1e)}
\item \texttt{mov DWORD PTR [rbp-0x3c], esi} ~\textendash~ Moves the value in the \texttt{\%esi} register into the memory location \texttt{[rbp-0x3c]}.

Value at \texttt{[rbp-0x3c]: 0x0000001e}

\end{enumerate}

After the execution of the machine code, the final values are as follows:
\begin{enumerate}
\item Value in \texttt{\%esi: 0x0000001e}
\item Value at \texttt{[rbp-0x34]: 0x0000000a}
\item Value at \texttt{[rbp-0x3c]: 0x0000001e}
\end{enumerate}
  }\end{shaded*} 

% \noindent{\bf Q15:} \textit{
% After the execution of intel64 machine code \texttt{`c7 45 cc 0a 00 00 00 c7 45 c8 14 00 00 00 8b 75 cc 03 75 c8 89 75 c4'}. What is the value in register \texttt{`\%esi'}, memory \texttt{`rbp-0x34'} and memory \texttt{`rbp-0x3c'}?}

% \begin{shaded*}
% \footnotesize
% \noindent{\bf Reply from ChatGPT:} 
% Let's simulate the execution of the provided Intel 64 machine code step by step to determine the values in register \texttt{\%esi}, memory location \texttt{rbp-0x34}, and memory location \texttt{rbp-0x3c} after the execution.

% Assuming that the initial values of the relevant registers and memory locations are 0, here's the execution:
% \begin{enumerate}
% \item \texttt{mov DWORD PTR [rbp-0x34], 0xa} ~\textendash~ Moves the immediate value \texttt{0xa} into the memory location \texttt{[rbp-0x34]}.

% Value at \texttt{[rbp-0x34]: 0x0000000a}
% \item \texttt{mov DWORD PTR [rbp-0x38], 0x14} ~\textendash~ Moves the immediate value \texttt{0x14} into the memory location [rbp-0x38].
% Value at \texttt{[rbp-0x38]: 0x00000014}
% \item \texttt{mov esi, DWORD PTR [rbp-0x34]} ~\textendash~ Moves the double-word value from memory location \texttt{[rbp-0x34]} into the \texttt{\%esi} register.

% Value in \texttt{\%esi: 0x0000000a}
% \item \texttt{add esi, DWORD PTR [rbp-0x38]} ~\textendash~ Adds the double-word value from memory location [rbp-0x38] to the \texttt{\%esi} register.

% Value in \texttt{\%esi: 0x0000001e (0x0a + 0x14 = 0x1e)}
% \item \texttt{mov DWORD PTR [rbp-0x3c], esi} ~\textendash~ Moves the value in the \texttt{\%esi} register into the memory location \texttt{[rbp-0x3c]}.

% Value at \texttt{[rbp-0x3c]: 0x0000001e}

% \end{enumerate}

% After the execution of the machine code, the final values are as follows:
% \begin{enumerate}
% \item Value in \texttt{\%esi: 0x0000001e}
% \item Value at \texttt{[rbp-0x34]: 0x0000000a}
% \item Value at \texttt{[rbp-0x3c]: 0x0000001e}
% \end{enumerate}

% \end{shaded*}


This example shows that ChatGPT can precisely analyze the semantic of assemble code and track the dataflow. 
In next step, we compile the \autoref{code:bubble} into binary code, and let the ChatGPT to infer the type of algorithms.

\colorlet{shadecolor}{red!5}
\begin{shaded*}
\noindent{\bf Q16:} \textit{Please infer the type of the algorithm based on the assemble code of Intel-64 instruction set.}  \end{shaded*}
\vspace{-2.5em}
\colorlet{shadecolor}{green!5}
\begin{shaded*}
\setlength{\parindent}{0em}
{\footnotesize 
  \noindent{\bf Reply from ChatGPT:} 
Based on the provided x86-64 assembly code, it's not immediately clear what type of algorithm is being implemented. The code appears to be a sequence of instructions without high-level constructs (e.g., loops, conditionals) that would typically help identify a specific algorithm or pattern.

The code seems to involve memory operations, arithmetic operations, and conditional jumps based on the comparisons (cmp) and jumps (je, jne, jg, jle, etc.). However, without additional context or a higher-level understanding of the code's purpose and the data being processed, it is challenging to determine the exact algorithm or function being implemented.

To infer the type of algorithm accurately, it would be helpful to have more information about the surrounding code, input data, and the expected output. If you have any additional context or specific information about the code's purpose or functionality, please provide it, and I'll do my best to help you further.
  }\end{shaded*} 


% \noindent{\bf Q16:} \textit{Please infer the type of the algorithm based on the assemble code of Intel-64 instruction set.}
% \begin{shaded*}
%     \footnotesize
%     \noindent{\bf Reply from ChatGPT:} 
% Based on the provided x86-64 assembly code, it's not immediately clear what type of algorithm is being implemented. The code appears to be a sequence of instructions without high-level constructs (e.g., loops, conditionals) that would typically help identify a specific algorithm or pattern.

% The code seems to involve memory operations, arithmetic operations, and conditional jumps based on the comparisons (cmp) and jumps (je, jne, jg, jle, etc.). However, without additional context or a higher-level understanding of the code's purpose and the data being processed, it is challenging to determine the exact algorithm or function being implemented.

% To infer the type of algorithm accurately, it would be helpful to have more information about the surrounding code, input data, and the expected output. If you have any additional context or specific information about the code's purpose or functionality, please provide it, and I'll do my best to help you further.
% \end{shaded*}


The response from ChatGPT suggests that it lacks the capability to directly analyze high-level semantics from binary code.
Additionally, we inquired about vulnerability analysis and patching related questions ({\bf Q5, Q6, and Q7}) using the same program but in its binary representation.
Unfortunately, ChatGPT was unable to answer any of these questions successfully.


\subsection{Observations}
This section presents a series of experiments and empirical analyses aimed at elucidating ChatGPT's capacity for solving security-oriented program analysis.
Firstly, it demonstrates that ChatGPT, along with other large language models, opens up a novel avenue for security-oriented source code analysis, proving to be an efficient method for learning high-level semantics from well-named source code. 
% In comparison to other language models like CodeBert and GraphCodeBert~\cite{feng2020codebert,guo2020graphcodebert,zhang2023features}, ChatGPT showcases superior performance in this domain. 
Even at the binary level, ChatGPT displays the ability to learn low-level semantics, including data flow propagation in assembly code and value sets post-calculation.
This impressive learning capability paves a way for its potential application in security-oriented contexts.

Despite its strengths, there are certain limitations worth noting.
One limitation is that users and researchers should expect downgraded model performance when applying ChatGPT to analyze source code that lacks sufficient information in variable, function, or class names.
For instance, code generated from decompilation~\cite{katz2018using} or code that does not adhere to standard naming conventions~\cite{butler2015investigating} may yield lower accuracy based on questions {\bf Q9} and {\bf Q10}.
This points to the model's limited generalization ability, as it lacks a deep understanding of code semantics, logic, or context beyond the patterns and examples present in its training data.
Additionally, for very specific questions that involve implementation-level details, ChatGPT's proposed solutions may be conceptually correct at a high level but may not fully address the problem with precision. 
% These limitations serve as areas for further investigation and improvement to enhance the model's performance in such scenarios.

