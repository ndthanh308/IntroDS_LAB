% \documentclass[letterpaper,twocolumn,10pt]{article}
% \usepackage{usenix-2020-09}
\documentclass[a4paper,12pt]{article} 
\usepackage{times}
\usepackage{geometry}
\geometry{portrait, margin=1in}
\usepackage{authblk}
\usepackage{mathrsfs}
\usepackage{amsmath,amsthm}
%\let\Bbbk\relax
\usepackage{tablefootnote}
\usepackage[numbers]{natbib}
\usepackage{amssymb}
\let\Bbbk\relax
\usepackage{algorithmic,algorithm}
\usepackage{adjustbox}
\usepackage{multirow}
\usepackage{pgfplots}
\usepackage{threeparttable}
\usepackage{array}
\usepackage{graphicx,color}
\usepackage{listings}
\usepackage[labelfont=bf,font=small,skip=5pt]{caption}
\usepackage{pifont}
\usepackage{url}
\usepackage{array}
\usepackage{amsfonts}
\usepackage{float}
\usepackage{framed}
\usepackage{soul}
\usepackage{xcolor}
\usepackage{tikz}
\usepackage{pgf}
\usepackage{fancyvrb}
\let\labelindent\relax
\usepackage{enumitem}
\usepackage{mathtools,amsthm}
\usepackage{cuted}
\usepackage{stackengine}
\usepackage{booktabs}
\usepackage{diagbox}
\usetikzlibrary{positioning}
\usetikzlibrary{decorations.pathreplacing}
\usepackage{xspace}
\usepackage[utf8]{inputenc}
\usepackage{amsmath, amssymb, latexsym}
\usepackage{pifont}% http://ctan.org/pkg/pifont
\usepackage{sidecap}
\usepackage{subcaption}
\usepackage{tikz}
\usepackage{fancybox}
\usetikzlibrary{decorations.pathreplacing}
\usepackage{balance}
\usetikzlibrary{patterns}
\usepackage{hyperref}
\newcommand*\circled[1]{\tikz[baseline=(char.base)]{
            \node[shape=circle,draw,inner sep=0.3pt] (char) {#1};}}

\usepackage{xparse}
% \usepackage{footnote}
\usepackage{threeparttable}
\newcounter{chatlinenum}

%% Adjust text width to suit
\tikzset{chatstyle/.style={text width=3.2in,rounded corners=2pt}}

\definecolor{mygreen}{HTML}{5fedb7}
\definecolor{lightgray}{HTML}{b6b8b7}
\definecolor{shadecolor}{gray}{0.9}

% %% |=====8><-----| %% New solution:

%% Alter colors to suit
\begingroup
    \lccode`~=`\^^M
    \lowercase{%
\endgroup
    \def\newchatline#1~{%
        \stepcounter{chatlinenum}%
        \ifodd\thechatlinenum
            \tikz[]{\node[fill=lightgray,chatstyle]{\strut#1\strut};}%
        \else
            \hfill
            \tikz[]{\node[fill=mygreen,chatstyle,align=left]{\strut#1\strut};}%
        \fi
        ~
        \smallskip
    }%
}

\NewDocumentEnvironment{newchat}{}{%
    \setcounter{chatlinenum}{0}
    \begin{minipage}{3.4in}
        \obeylines
        \everypar={\newchatline}
}{%
    \end{minipage}
}


\usepackage{lipsum,xcolor}
\newcommand\dunderline[3][-1pt]{{%
  \sbox0{#3}%
  \ooalign{\copy0\cr\rule[\dimexpr#1-#2\relax]{\wd0}{#2}}}}


\pagestyle{plain}
%\newcommand{\mypara}[1]{\vspace{.3em}\noindent\textbf{{#1. }}}
\newcommand{\qpara}[1]{\vspace{.5em}\noindent\textbf{\textit{#1 }}}

\usepackage{xstring}
\newcommand{\mypara}[1]{
\vspace{.3em}
\noindent{\bf \IfEndWith{#1}{.}{#1}{\IfEndWith{#1}{?}{#1}{#1.}}}
}

\newcommand{\xdashrightarrow}[2][]{\ext@arrow 0359\rightarrowfill@@{#1}{#2}}
\newcounter{question}[section]
\setcounter{question}{0}
\newenvironment{question}[1][]
{\refstepcounter{question}\par
\textbf{\cc{Q\thequestion.#1}} \rmfamily}

\newcounter{mydefinition}[section]
\setcounter{mydefinition}{0}
\newenvironment{mydefinition}[1][]
{\refstepcounter{mydefinition}\par\medskip
   \textit{Definition~\themydefinition. #1} \rmfamily}{\medskip}

\newcommand{\etal}{\textit{et al}.\xspace}
\newcommand{\ie}{\textit{i}.\textit{e}.}
\newcommand{\eg}{\textit{e}.\textit{g}.}


\definecolor{mygray}{gray}{0.2}
\newcommand{\fixme}[1]{\textcolor{red}{#1}}
\newcommand{\zl}[1]{{\color{green} Zhilong:} {\color{purple} #1}}
\newcommand{\hu}[1]{{\textbf{\color{red} Hong:}} {\color{red} #1}}
\newcommand{\haizhou}[1]{{\color{green} Haizhou:} {\color{violet} #1}}
\newcommand{\replace}[2]{\st{#1}$\Rightarrow${\color{blue} #2}}

\newcommand{\redu}[1]{{#1}} % redundancy setence.

% \definecolor{dkgreen}{rgb}{0,0.6,0}
% \definecolor{gray}{rgb}{0.5,0.5,0.5}
% \definecolor{mauve}{rgb}{0.58,0,0.82}
\definecolor{tablegray}{gray}{0.9}
\lstdefinelanguage
    [x32]{Assembler}     % add a "x64" dialect of Assembler
    [x86masm]{Assembler} % based on the "x86masm" dialect
    {morekeywords={movl,addl,cmpl,CMPXCHG16B,JRCXZ,LODSQ,MOVSXD, %
                  POPFQ,PUSHFQ,SCASQ,STOSQ,IRETQ,RDTSCP,SWAPGS,CALLQ,LEAVEQ,RETQ, %
                  rax,rdx,rcx,rbx,rsi,rdi,rsp,rbp, %
                  r8,r8d,r8w,r8b,r9,r9d,r9w,r9b, %
                  r10,r10d,r10w,r10b,r11,r11d,r11w,r11b, %
                  r12,r12d,r12w,r12b,r13,r13d,r13w,r13b, %
                  r14,r14d,r14w,r14b,r15,r15d,r15w,r15b}} % etc.

\lstset{
%language=[x64]Assembler,
language=C,
captionpos=b,
frame=shadowbox,
rulesepcolor=\color{mygray},
backgroundcolor=\color{white},
numbers=left,
stepnumber=1,
xleftmargin=0pt, %5\columnwidth, 
xrightmargin=0pt, %1\columnwidth,
}
\renewcommand{\lstlistingname}{Code}



\renewcommand{\sectionautorefname}{Section}
\renewcommand{\subsectionautorefname}{Section}
\renewcommand{\subsubsectionautorefname}{Section}

\setlength{\textfloatsep}{10pt}     % distance between floats on the top or the bottom and the text
\setlength{\floatsep}{10pt}         % distance between two floats
\setlength{\dbltextfloatsep}{10pt}  % distance between a float spanning both columns and the text
\setlength{\dblfloatsep}{10pt}      % distance between two floats spanning both columns
\definecolor{bittersweet}{rgb}{1.0, 0.44, 0.37}
\definecolor{bleudefrance}{rgb}{0.19, 0.55, 0.91}
\usepackage{relsize}

\renewcommand{\ttdefault}{pxtt}

\newcommand{\cc}[1]{\mbox{\smaller[0.5]\texttt{#1}}}

\newcommand{\kenali}{\mbox{\textsc{Kenali}}\xspace}
\newcommand{\xmp}{\mbox{\textsc{xMP}}\xspace}
\newcommand{\dynpta}{\mbox{\textsc{DynPTA}}\xspace}

\definecolor{dkgreen}{rgb}{0,0.6,0}
\definecolor{gray}{rgb}{0.5,0.5,0.5}
\definecolor{mauve}{rgb}{0.58,0,0.82}
\definecolor{mygray}{gray}{0.9}
\colorlet{lightblue}{blue!70}
\colorlet{lightred}{red!70}

\lstset{
  language=C,
  frame=tb,
  basicstyle={\scriptsize \ttfamily}, %\scriptsize,\ttfamily,%
  tabsize=3,
  breaklines=true,
  % breakatwhitespace=false,
  showstringspaces=false,
  % columns=fullflexible,
  numbers=left,
  numbersep=-8pt,                     % where to put the line-numbers
  numberstyle=\tiny\color{darkgray},
  escapeinside={(*}{*)},
  xleftmargin=2pt,
  stringstyle=\color{mauve},
  keywordstyle=\color{blue},
  commentstyle=\color{dkgreen} \textit,%\scriptsize \textit,
  %directivestyle={\color{black}},
  %emph={int,char,double,float,unsigned, static, const, if, return, goto},
  emphstyle={\color{red}},
}

\newcommand{\squishlist}{
\begin{itemize}[noitemsep,nolistsep]
  \setlength{\itemsep}{-0pt}
}
\newcommand{\squishend}{
  \end{itemize}
}

\usepackage{url}
\def\UrlBreaks{\do\A\do\B\do\C\do\D\do\E\do\F\do\G\do\H\do\I\do\J
\do\K\do\L\do\M\do\N\do\O\do\P\do\Q\do\R\do\S\do\T\do\U\do\V
\do\W\do\X\do\Y\do\Z\do\[\do\\\do\]\do\^\do\_\do\`\do\a\do\b
\do\c\do\d\do\e\do\f\do\g\do\h\do\i\do\j\do\k\do\l\do\m\do\n
\do\o\do\p\do\q\do\r\do\s\do\t\do\u\do\v\do\w\do\x\do\y\do\z
\do\.\do\@\do\\\do\/\do\!\do\_\do\|\do\;\do\>\do\]\do\)\do\,
\do\?\do\'\do+\do\=\do\#}

\theoremstyle{definition}
\newtheorem{definition}{Definition}

\theoremstyle{definition}
\newtheorem*{explanation}{Explanation}

\theoremstyle{definition}
\newtheorem*{problem}{Problem}

\theoremstyle{definition}
\newtheorem{challenge}{Challenge}

\theoremstyle{definition}
\newtheorem{insight}{Insight}

\date{}
\title{ChatGPT for Software Security: \\ Exploring the Strengths and Limitations of ChatGPT in the Security Applications}
% \title{Exploring the Security-Oriented Program Analysis Capabilities of ChatGPT: A Case Study of ChatGPT for Security}
\author[1]{Zhilong Wang\thanks{Corresponding author:~zzw169@psu.edu}}
\author[1]{Lan Zhang}
\author[1]{Peng Liu}
\affil[1]{The Pennsylvania State University, United States}

\begin{document}

\maketitle

\begin{abstract}

The Fast Reciprocal Square Root Algorithm is a well-established approximation technique consisting of two stages: first, a coarse approximation is obtained by manipulating the bit pattern of the floating point argument using integer instructions, and second, the coarse result is refined through one or more steps, traditionally using Newtonian iteration but alternatively using improved expressions with carefully chosen numerical constants found by other authors. The algorithm was widely used before microprocessors carried built-in hardware support for computing reciprocal square roots. At the time of writing, however, there is in general no hardware acceleration for computing other fixed fractional powers. This paper generalises the algorithm to cater to all rational powers, and to support any polynomial degree(s) in the refinement step(s), and under the assumption of unlimited floating point precision provides a procedure which automatically constructs provably optimal constants in all of these cases. It is also shown that, under certain assumptions, the use of monic refinement polynomials yields results which are much better placed with respect to the cost/accuracy tradeoff than those obtained using general polynomials. Further extensions are also analysed, and several new best approximations are given.

\end{abstract}


% \begin{IEEEkeywords}
% encryption detection, ransomware, loop detection, k complexity, data flow analysis, graph neural network
% \end{IEEEkeywords}
% no keywords
\section{Introduction}
Current quantum hardware is unable to carry out universal quantum computations due to the buildup of errors that occur during the computation. 
The magnitude of the individual error is currently above the value that the Threshold Theorem requires in order to kick-start quantum error correction and fault-tolerant quantum computation~\cite[Section 10.6]{nielsen_chuang_2010}. 
Although the experimentally achieved fidelity rates are promising and the error bounds are inching closer to the required threshold, we will have to work for the foreseeable future with quantum hardware with errors that build-up during the computation.  This implies that we can only do a limited number of steps before the output of the computation has become completely uncorrelated with the intended one.

For fault-tolerant quantum computing, we repeat four steps: 
1) We apply a number of single and two-qubit quantum gates, in parallel whenever possible; 
2) We perform a syndrome measurement on a subset of the qubits; 
3) We perform fast classical computations to determine which errors have occurred and how to correct them; 
and, 4) We apply correction terms based on the classical computations.
We then repeat these four steps with a next sequence of gates. 
These four steps are essential to fault-tolerant quantum computing. 


The starting point of this work is to use the four steps outlined above, not to carry out error correction and fault-tolerant computation, but to enhance short, constant-depth, {\em uncorrected} quantum circuits that perform single qubit gates and {\em nearest-neighbor} two qubit gates. 
Since in the long run we will have to implement error-correction and fault-tolerant computation anyhow, and this is done by such a four-step process, why not make other use of this architecture? Moreover, on some of the quantum hardware platforms, these operations are already in place.
Embracing this idea we naturally arrive at the question: what is the computational power of \textit{low-depth} quantum-classical circuits organized as in the four steps outlined above? 
We thus investigate circuits that execute a small, ideally constant, number of stages, where at each stage we may apply, in parallel, single qubit gates and {\em nearest-neighbor} two qubit gates, followed by measurements, followed by low-depth classical computations of which the outcome can control quantum gates in later stages. 
It is not clear, at first, whether such circuits, especially with constant depth, can do anything remotely useful. 
But we will see that this is indeed the case: many quantum computations can be done by such circuits in constant depth. 
By parallelizing quantum computations in this way, we improve the overall computational capabilities of these circuits, as we do not incur errors on qubits that are idle, simply because qubits are not idle for a very long time. 
Furthermore, reducing the depth of quantum circuits, at the cost of increasing width, allows the circuit to be run faster even if errors occur.

The first usage of such a four-step layout, not to do error correction, but to perform computations, can be found in the paradigm of measurement-based quantum computing~\cite{gottesman1999demonstrating,raussendorf2001one,jozsa2006introduction,clark2007generalised}: 
A universal form of quantum computing where a quantum state is prepared and operations are performed by measuring qubits in different bases, depending on previous measurements and intermediate measurements.

\citeauthor{PhamSvore2013} were the first to formalize the four-step protocol for performing computations~\cite{PhamSvore2013}. They included specific hardware topologies by considering two-dimensional graphs for imposing constraints on qubit interactions. In their model, they develop circuits for particularly useful multi-qubit gates, including specifying costs in the width, number of qubits, depth, number of concurrent time steps, size, and total number of non-Identity operations.
As a result, they find an algorithm that factors integers in polylogarithmic depth.
\citeauthor{Browne:2011} showed that the main tool in the work by \citeauthor{PhamSvore2013}, the fan-out gate, can also be replaced by additional log-depth classical computations in the measurement-based quantum computing setting~\cite{Browne:2011}.

More recently, \citeauthor{Cirac:2021} introduced a scheme to implement unitary operations involving quantum circuits combined with Local Operations and Classical Communication ($\mathsf{LOCC}$) channels: $\mathsf{LOCC}$-assisted quantum circuits~\cite{Cirac:2021}. Similarly to the four-step scheme we just described, they allow for a short depth circuit to be run on the qubits, followed by one round of $\mathsf{LOCC}$, in which ancilla qubits are measured and local unitaries are applied based on the measurement outcomes. They show that in this model any 1D transitionally invariant matrix-product state (MPS) with fixed bond dimension is in the same phase of matter as the trivial state. Similar ideas can be found in~\cite{TVV_NonAbelianTopologicalOrder_2022, tantivasadakarn2021long}.

In this work, we introduce a new model, called \textit{Local Alternating Quantum-Classical Computations} ($\LAQCC$). In this model we alternate between running quantum circuits (constrained by locality), ending in the measurement of a subset of qubits, and fast classical computations based on the measurement results. The outcome of the classical computations are then used to control future quantum circuits. We allow for flexibility in this model, by giving different constraints to the power of both the quantum circuits and the classical circuits as well as the number of alternations between them. 
Most attention will be given to $\LAQCC$ containing quantum circuits of constant depth, classical circuits of logarithmic depth and at most a constant number of alternations between them. 
Any circuit constructed in this model is considered to be of constant depth. 
We restrict ourselves to logarithmic depth classical computations, as this is the first natural and non-trivial extension beyond constant-depth classical computations. 
Constant-depth classical computations do however also have an equivalent constant-depth quantum implementation.

The definition of $\LAQCC$ sharpens the original definition of \citeauthor{PhamSvore2013} by adding constraints to the intermediate classical computations. This allows us to bound the power of $\LAQCC$ from above. 

The main result of \citeauthor{Cirac:2021}, that 1D translational invariant MPS with fixed bond dimension can be prepared by $\mathsf{LOCC}$-assisted circuits, relies on local symmetries of the MPS. These symmetries allow them to prepare local states (on a constant number of qubits) and glue them together by doing one round of the appropriate entangling measurement and corrections, after which they run a round of local unitaries to get the desired result. This general scheme for preparing states that exhibit an MPS description with the appropriate local symmetries requires only geometrically local unitaries and one round of measurement and corrections an therefore is accessible in $\LAQCC$. Studying different local symmetries, known as Symmetry Protected Topological (SPT) phases of matter, to find measurement-based constant depth circuits for states is a broad ongoing field of research~\cite{TVV_NonAbelianTopologicalOrder_2022, tantivasadakarn2021long, smith2023deterministic}. 
All these schemes have a $\LAQCC$ implementation.

%$\LAQCC$-circuits also exist for general schemes of preparing local states, based on the local tensors, and gluing them together using one round of entangled measurement and corrections, based on the local symmetry. 
%The main result of \citeauthor{Cirac:2021}, that 1D translational invariant MPS with fixed bond dimension can be prepared by $\mathsf{LOCC}$-assisted circuits, relies heavily on local symmetries of the MPS and as a result also has an equivalent $\LAQCC$ implementation. 
%The corrections applied after the measurement round are local unitaries depending on the local symmetries of the MPS. 

 

%This general scheme of preparing local states, based on the local tensors, and gluing it together by doing one round of entangled measurement and corrections, based on the local symmetry, is accessible in $\LAQCC$.
Note however that \citeauthor{Cirac:2021} also suggest a circuit for the $W$-state.
This circuit uses sequentially and dependent measurement-based corrections of the ancilla qubits. 
These dependent measurements translate to sequential alternations between the quantum and classical circuits and therefore increase the total depth to linear depth, exceeding the constant-depth constraints imposed by $\LAQCC$-circuits. 

We study the power of the $\LAQCC$ model with respect to state preparation, showing that even with only constant quantum-depth and logarithmic classical depth it remains possible to prepare states with long-range entanglement.
Another surprising result is that it is unlikely that $\LAQCC$ circuits are classically simulatable. We show that any instantaneous quantum polynomial-time (IQP) circuit~\cite{Bremner2010,Shepherd2009} has an $\LAQCC$ implementation.
Classical simulation of IQP circuits implies the collapse of the polynomial hierarchy to the third level, which is not believed to be true~\cite{Bremner2017}. Therefore, we expect that $\LAQCC$ circuits are unlikely to be classically simulatable. We bound the power of $\LAQCC$ by showing that it is contained in $\QNC^1$, the class of polynomial-size, log-depth circuits.

Next, we also study the power that intermediate classical calculations can add to quantum computations, by considering a new model that alternates between polynomially many polynomial-depth quantum circuits and unbounded classical computations
We study this model by doing a complexity theoretical analysis, where we draw inspiration from the notions of complexity given by \citeauthor{RosenthalYuen:2022}, \citeauthor{MetgerYuen:2023}, and \citeauthor{Aaronson:2004}.
All three complexity notions are based on the notion of state preparation, instead of more traditional definition of complexity such as the decidability of a computational problem. 
The first two consider classes based on sequences of quantum states preparable by a polynomial-sized quantum circuit, where the circuits are uniformly generated by a computational class, for instance, the class $\mathsf{PSPACE}$, which results in the complexity class $\mathsf{StatePSPACE}$~\cite{RosenthalYuen:2022,MetgerYuen:2023}.
The third notion considers a relative complexity, where the complexity is measured between two given states, and is measured by the number of gates, from a given gate-set, required to transform one state in another state~\cite{Aaronson:2004}. 
For our definition of state preparation complexity, we drop the uniformity constraint from~\cite{RosenthalYuen:2022,MetgerYuen:2023} and define a class as $\mathsf{StateX}$, which refers to states preparable by circuits of type $\mathsf{X}$. 
As an example, if $\mathsf{X} = \QNC^0$, this results in the class $\mathsf{StateQNC^0}$, which is the set of states preparable from the $\ket{0}^n$ state by poly-size constant-depth circuits. 
This notion is similar to the relative complexity from~\cite{Aaronson:2004}, where one state is the  $\ket{0}^n$ state and instead of counting the number of gates we consider the set of states preparable by a fixed number of gates. Using this notion of complexity we show that any state preparable by an $\LAQCC^*$ circuit is also preparable by a $\mathsf{PostQPoly}$ circuit, the class of circuits of polynomial depth with an additional post-selection gate. 

All Clifford circuits have a constant-depth $\LAQCC$ implementation, implying that any stabilizer state can be implemented by a constant-depth $\LAQCC$ circuit, see Section~\ref{sec:clifford_circuits} for a proof of this statement. 
Efficient circuits for stabilizer states have been known already through measurement-based quantum computing. Therefore this paper focuses on the preparation of non-stabilizer states, and as a surprising result we find novel constant-depth protocols for four very natural classes of non-stabilizer states.
Despite the extensive research into these four classes of non-stabilizer states and the many applications of them, no efficient constant- or low-depth state preparation protocols are known yet. We specifically consider these four classes as they are all often used as initial states in other algorithms.

The first state is a uniform superposition over an arbitrary number of states. 
This state finds applications in many quantum algorithms, as they often start with a uniform superposition over multiple states. 
This superposition is often achieved by applying Hadamard gates to every qubit due to its simplicity to prepare. 
Yet, the analysis of many algorithms, such as Shor's algorithm~\cite{Shor:1997}, would benefit from a different initial superposition. 
The circuit to prepare the uniform superposition over an arbitrary number of states uses an exact version of Grover search as a subroutine, that turns a probabilistic circuit, with a known constant probability of success, into a deterministic circuit. 
We use the circuit for preparing a uniform superposition over an arbitrary number of states as a subroutine in the next two quantum state preparation protocols. 

The second state is the $W$-state, the uniform superposition over all computational basis states of Hamming-weight~$1$, a natural long-ranged entangled state that displays a fundamentally nonequivalent type of entanglement from the Greenberger–Horne–Zeilinger state~\cite{WState:2000}, for which $\LAQCC$-type constant-depth circuits were previously known~\cite{PhamSvore2013, Cirac:2021}. 
The $W$-state is often used as benchmark for new quantum hardware~\cite{Haffner2005,Neeley2010,GarciaPerez:2021}. 
A novel way to prepare the $W$-state therefore gives a new way to benchmark different quantum devices with each other. 
A circuit for preparing the $W$-state was given in~\cite{Cirac:2021}, but this implementation requires sequentially alternating measurements followed by local unitaries, which in the $\LAQCC$ model is not considered to be of constant depth. 
We improve this protocol by giving an $\LAQCC$ implementation of the $W$-state, based on a compress-uncompress method that links the one-hot and binary encoding of integers.

The third state considered is the Dicke state, a generalization of the $W$-state, a superposition over all computational basis states with Hamming-weight $k$~\cite{Dicke:1954}. 
Dicke states have relevance in various practical settings.
For instance, for quantum game theory~\cite{zdemir2007}, quantum storage~\cite{Bacon_Compress:2006,Plesch:2010}, quantum error correction~\cite{ouyang2014permutation}, quantum metrology~\cite{toth2012multipartite}, and quantum networking~\cite{prevedel2009experimental}. 
Dicke states have been used as a starting state for variational optimization algorithms, most notably Quantum Alternating Operator Ansatz (QAOA)~\cite{Hadfield2019}, to find solutions to problems such as Maximum k-vertex Cover~\cite{Brandhofer2022,cook2020quantum}.
The ground states of physical Hamiltonians describing one-dimensional chains tend to show a resemblance to Dicke states such as states resulting from the Bethe ansatz, making them an ideal starting state when investigating the ground state behavior of these Hamiltonians~\cite{TDL_BetheAnsatzDerivation:2010,B_ExcitedStateQuantumPhaseTransitions:2013,DickeTransitions:2021}. 
For instance, the algorithm by \citeauthor{van2021preparing}, who give an algorithm to prepare the Bethe ansatz eigenstates of the spin-1/2 XXZ spin chain, starts by first preparing a Dicke state~\cite{van2021preparing}. 
A Dicke-state preparation protocol based on the compress-uncompress methodology used in the $W$-state furthermore finds applications in entanglement distillation, where the entanglement of a large state is concentrated on only a few qubits. 
Efficient deterministic circuits for preparing Dicke states have been proposed by \citeauthor{bartschi2019deterministic}~\cite{bartschi2019deterministic, bartschi2022deterministic_short_depth}. 
They provide a quantum circuit of depth $\mathO(k \log(\frac{n}{k}))$, allowing arbitrary connectivity, to prepare a Dicke state, which they conjecture to be optimal when $k$ is constant. 
In this work, we provide a constant-depth $\LAQCC$ circuit below their conjectured bound already for constant $k$. 
However, this does not directly disprove their conjecture, as we allow for intermediate measurements and classical computations. 
More significantly, we even construct constant-depth $\LAQCC$ circuits for $k = \mathO(\sqrt{n})$ greatly improving their bound.
This construction extends the compress-uncompress method for the $W$-state combined with additional subroutines. 

We continue with a log-depth state preparation protocol for the Dicke-state for arbitrary $k$. 
This protocol implements an efficient transformation between the factoradic number representation and the combinatorial number representation of a positive integer. 
The combinatorial number representation relates directly to the Dicke state. 
The provided efficient transformation between number representation systems might be of independent interest. 

We conclude by modifying our protocol for preparing a Dicke-state to a protocol that prepares quantum many-body scar states in constant-depth. 
These states have low entanglement and longer coherence times than states with similar energy density.
These characteristics make many-body scar states interesting to analyze and relevant within physics.
Many-body scar states appear for instance in the AKLT model~\cite{AKLT:1987,MRBAR:2018,MRB:2018} and different spin models~\cite{SI:2019,MOBFR:2020}.
Known methods for preparing these states have polynomial-depth~\cite{Gustafson:2023}, whereas our circuit has constant depth. 

% We conclude by studying the power that intermediate classical calculations can add to quantum computations. 
% In this study, we define a new model that relaxes constant-depth quantum circuits to polynomial depth quantum circuits, log-depth classical calculations to unbounded classical computations and a constant number of alternations to a polynomial number of alternations. 
% We call this model $\LAQCC^*$. 
% We study this model by doing a complexity theoretical analysis, where we draw inspiration from the notions of complexity given by \citeauthor{RosenthalYuen:2022}, \citeauthor{MetgerYuen:2023}, and \citeauthor{Aaronson:2004}.
% All three complexity notions are based on the notion of state preparation, instead of more traditional definition of complexity such as the decidability of a computational problem. 
% The first two consider classes based on sequences of quantum states preparable by a polynomial-sized quantum circuit, where the circuits are uniformly generated by a computational class, for instance, the class $\mathsf{PSPACE}$, which results in the complexity class $\mathsf{StatePSPACE}$~\cite{RosenthalYuen:2022,MetgerYuen:2023}.
% The third notion considers a relative complexity, where the complexity is measured between two given states, and is measured by the number of gates, from a given gate-set, required to transform one state in another state~\cite{Aaronson:2004}. 
% For our definition of state preparation complexity, we drop the uniformity constraint from~\cite{RosenthalYuen:2022,MetgerYuen:2023} and define a class as $\mathsf{StateX}$, which refers to states preparable by circuits of type $\mathsf{X}$. 
% As an example, if $\mathsf{X} = \QNC^0$, this results in the class $\mathsf{StateQNC^0}$, which is the set of states preparable from the $\ket{0}^n$ state by poly-size constant-depth circuits. 
% This notion is similar to the relative complexity from~\cite{Aaronson:2004}, where one state is the  $\ket{0}^n$ state and instead of counting the number of gates we consider the set of states preparable by a fixed number of gates. Using this notion of complexity we show that any state preparable by an $\LAQCC^*$ circuit is also preparable by a $\mathsf{PostQPoly}$ circuit, the class of circuits of polynomial depth with an additional post-selection gate. 

\paragraph{Summary of results}
\begin{itemize}
    \item We give a new definition of a computational model that captures the power of the four step process: applying a constant number of layers of one- and two-qubit gates; performing a syndrome measurement; perform a fast classical computation determining corrections; apply corrections. We call this model \emph{Local Alternating Quantum Classical Computations}, or $\LAQCC$ for short. In this model we bound the allowed quantum operations, intermediate classical calculations, and number of rounds separately. In Section~\ref{sec:LAQCC_model} we define this model and give a list of operations based on results from literature contained in this computational model. In some of these operations we explicitly use that we allow for multiple, but at most constant, rounds  of corrections.
    \item  We show show that there exist $\LAQCC$ circuits that can not be weakly simulated in Section~\ref{sec:IQP_in_LAQCC}. We further show that for every $\LAQCC$ circuit there exists a $\QNC^1$ circuit simulating it perfectly, in Section~\ref{sec:LAQCC_in_QNC1}.
    \item We introduce a new type computational complexity for preparing states and show that the extension of $\LAQCC$ where we allow a polynomial number of rounds and unbounded classical computation, is contained in $\mathsf{PostQPoly}$, the class of polynomial circuits with post-selection, in Section~\ref{sec:Complexity results}.
    \item We show a protocol to prepare the uniform superposition state of size $q$ in $\LAQCC$ using $\mathO(\ceil{\log_2(q)}^2)$ qubits in Section~\ref{sec:superposition_modulo_q}. 
    \item We show a protocol to prepare the $W_n$ state in $\LAQCC$ using $\mathO(n\log(n))$ qubits in Section~\ref{sec:W_state_in_LAQCC}.
    \item We show two ways of preparing the Dicke-$(n,k)$ state. The first method is in $\LAQCC$, works up to $k = \mathO(\sqrt{n})$, uses $\mathO(n^2\log(n))$ qubits, and is found in Section~\ref{sec:dicke:small_k}. The second method is in $\LAQCC\text{-}\mathsf{LOG}$ (an extension of $\LAQCC$ allowing for logarithmic number of alterations instead of constant), works for any $k$, uses $\mathO(\text{poly}(n))$ qubits, and is found in Section~\ref{sec:Dicke_in_LAQCC_LOG}. 
    \item We extend on our $\LAQCC$ method of generating Dicke-$(n,k)$ states for $k = \mathO(\sqrt{n})$ and show a protocol to generate many-body scar states for a particular Hamiltonian in $\LAQCC$ (Section~\ref{sec:many_body_scar}). 
\end{itemize}
Summarized in a table, we provide the following state generation protocols:
\begin{table}[htb]
\centering
\begin{tabular}{l|l|l|l}
\textbf{State description} & \textbf{Width} & \textbf{Depth} & \textbf{Implementation}\\
\hline 
Uniform superposition mod $q$: $\frac{1}{\sqrt{q}} \sum_{i = 0}^{q-1}\ket{i}$ & $\mathO(\ceil{\log^2 q})$ & $\mathO(1)$ & Section~\ref{sec:superposition_modulo_q}\\

$W$-state: $\frac{1}{\sqrt{n}}\sum_{i = 0}^{n-1}\ket{e_i}$ & $\mathO(n \log n)$ & $\mathO(1)$ & Section~\ref{sec:W_state_in_LAQCC}\\

Dicke-$(n,k)$, $k = \mathO(\sqrt{n})$: $\binom{n}{k}^{-1/2}\sum_{x \in \{0,1\}^n: |x| = k} \ket{x}$ &  $\mathO(n^2\log n)$ & $\mathO(1)$ 
&Section~\ref{sec:dicke:small_k}\\

Dicke-$(n,k)$: $\binom{n}{k}^{-1/2}\sum_{x \in \{0,1\}^n: |x| = k} \ket{x}$ & $\mathO(\text{poly}(n))$ & $\mathO(\log n)$ &Section~\ref{sec:Dicke_in_LAQCC_LOG}\\

QMBS: $\ket{S_k} = \frac{1}{k! \sqrt{\mathcal N(n,k)}}(Q^\dagger)^k \ket{\Omega}$ &  $\mathO(n^2\log n)$ & $\mathO(1)$  &  Section~\ref{sec:many_body_scar}
\end{tabular}
\caption{Summary of state preparation protocols given in this paper.}
\label{tab:sate_prep}
\end{table}
In the entry for the quantum many-body scar state $Q$ denotes the raising operator and $\mathcal N(n,k)=\binom{n-k-1}{k}$. 
Section~\ref{sec:many_body_scar} will provide more details on the variables and the implementation. 

\paragraph{Organization of the paper}
\noindent We first introduce relevant preliminaries in Section~\ref{sec:preliminaries}. 
In Section~\ref{sec:LAQCC_model} we formally define the class of Local Alternating Quantum-Classical Computations ($\LAQCC$). We also show that any Clifford circuit can be implemented in constant depth $\LAQCC$ (a result based on a result from measurement-based quantum computing~\cite{jozsa2006introduction}). 
This result allows us to give many useful multi-qubit gates and routines in Section~\ref{sec:gates_created_in_LAQCC}. 
Beyond that we show that constant depth $\LAQCC$ circuits are contained in $\QNC^1$ and that any $\mathsf{IQP}$ circuit has an $\LAQCC$ implementation.
We conclude this section with an analysis of a more powerful instantiation of $\LAQCC$ and show an inclusion with respect to the class $\mathsf{PostQPoly}$, which is the class of circuits of polynomial depth with one additional post-selection gate. 
In Section~\ref{sec:state_prep_in_LAQCC} we give $\LAQCC$ circuit implementations for preparing the uniform superposition over an arbitrary number of states, the $W$-state and the Dicke state up to $k = \mathO(\sqrt{n})$. We furthermore give a log-depth circuit implementation for preparing the Dicke state for any $k$. We conclude by showing a $\LAQCC$ circuit for generating many body scar states of a particular type of Hamiltonian.


\vspacebeforesection
\section{Background}
\label{sec:background}

In this section, we provide the necessary background information to ensure a comprehensive understanding of the attack described in this paper. We start with a description of the Distributed Hash Table (DHT) used by IPFS, followed by its content resolution mechanisms. We also detail techniques for network size estimation, necessary for our attack detection and mitigation mechanisms.

\vspacebeforesection
\subsection{IPFS DHT}
\label{sec:kad_dht}

We review the features of the Kademlia DHT~\cite{maymounkov2002kademlia} and its \texttt{libp2p} implementation~\cite{libp2p_github} that are the most relevant to our attack.
To participate in the DHT, each peer generates a public/private key pair and derives an identity $\peerid \in \{0,1\}^{256}$ as the hash of its public key.
Ideally, each peer generates a random key pair and, therefore, peer IDs are distributed uniformly and independently over the space $\{0,1\}^{256}$.
While honest nodes follow this rule, malicious nodes may generate and choose from an arbitrary number of key pairs.
Each peer maintains a routing table consisting of $m=256$ buckets.
The $i$-th bucket contains the addresses of up to $k=20$ peers whose peer IDs share a common prefix of exactly $i$ bits with the peer's own peer ID. 

%
A new participant node joins the IPFS network by contacting one of the hardcoded bootstrap nodes. This bootstrap node provides the new node with some initial peers allowing it to join the DHT. The new node uses this information to perform a walk through the DHT towards its own peer ID.
The walk allows to: \textit{(i)}~make sure that there is no other node in the network with the same ID; \textit{(ii)}~discover new peers and fill the newcomer's DHT routing table. At the same time, the newcomer establishes \bitswap~\cite{de2021accelerating} connections to a subset of encountered peers (usually around 300 of them). The core role of the \bitswap protocol is to enable bilateral content transfer and to play the role of a cache for recently-accessed content.

The main DHT operation $\Call{GetClosestPeers}{\key}$ returns the $k=20$ closest peers to $\key$. 
%
In Kademlia, the distance between two keys $x$ and $y$ in the key space is given by $x \oplus y \in \{0,...,2^{256}-1\}$, where $\oplus$ denotes the bitwise XOR operation on the keys; the resulting binary string is interpreted as an integer.
%
When a client wants to find the peers with IDs closest to $\key$, it sends a request to the $\alpha=3$ peers in its routing table whose peer IDs are closest to $\key$. Each of these peers returns the $k$ closest peers to $\key$ in its own routing table and the addresses of these peers. 
%
The client again sends a request to the $\alpha$ peers closest to $\key$, among peers in its routing table and those whose addresses it just received. This process repeats until the client does not find any more peers closer to $\key$.
Due to network churn and imperfect routing tables, we observed in our experiments that successive calls to $\Call{GetClosestPeers}{\key}$ do not always return the same set of $k=20$ peers (we provide more details in \Cref{sec:evaluation}, \Cref{fig:20closest}). This is an important limitation affecting our attack.

\vspacebeforesection
\subsection{Content Resolution in IPFS}
\label{sec:ipfs}

IPFS is a content-centric network.
It allows its participant to request files without specifying their location. 
%
Content is indexed by content IDs $\cid \in \{0,1\}^{256}$ that are derived from a hash of that content.
Both peer IDs and CIDs are used as keys in the DHT.
Each node can play the role of a \provider, \downloader, or \resolver. 
The process of content advertisement and resolution is illustrated in \Cref{fig:add_get_provider}.

%
When a \provider wishes to publish content with a given $\cid$ on IPFS, it creates a \emph{provider record} that contains $cid$ and the \provider's address.
During a $\Call{Provide}{\cid}$ operation, the \provider first uses $\Call{GetClosestPeers}{\cid}$ to locate the $k=20$ peers with their peer IDs closest to $\cid$, 
%
and then sends them a $\mathsf{PutProvider}$ message including the provider record (\Cref{fig:add_get_provider}(a)).
We call the peers that hold provider records for $\cid$ the \emph{resolvers} for $\cid$.

Each CID can have several \providers. In fact, by default, each IPFS client becomes a provider for each piece of content it downloads for a fixed amount of time (12h, 24h, or 48h depending on the client version or custom configuration). As a result, the system provides an auto-scaling feature with supply automatically rising with demand.

%
When a \downloader wishes to fetch a piece of content, it first sends a request to all its \bitswap peers. If none of them has the content, the \downloader uses the DHT-based resolution system. We stress that the \bitswap protocol plays the supporting role of a cache in the dissemination of popular files. However, the mechanism does not provide reliable content resolution, in particular for new or less popular content. %

When \bitswap unstructured search fails, the \downloader resolves $\cid$ using $\Call{FindProviders}{\cid}$. This operation uses a DHT walk identical to that of $\Call{GetClosestPeers}{\cid}$ to find $k$ \resolvers but also queries encountered nodes for a provider record for $\cid$ (\Cref{fig:add_get_provider}(b)). The process terminates when either 20 \providers have been found, or all \resolvers have been asked. Querying all encountered nodes (\ie, not only the designated \resolvers) is useful because some of the encountered nodes may have a provider record in their cache.
%

Upon receiving a provider record, the client connects to the address specified in the provider record to retrieve the actual content (\Cref{fig:add_get_provider}(c)).
Provider records are not authenticated, and therefore malicious \providers may respond with incorrect provider records (or may not respond at all). However, the integrity of the content is preserved because the hash of the retrieved content can be verified against its $\cid$.
%


%

\input{img/add_get_provider.tex}

\vspacebeforesection
\subsection{Network Size Estimator}
\label{sec:netsize}

The number of nodes in a decentralized system is generally unknown due to the avoidance of centralized membership management.
This number is nonetheless useful for optimizations, deciding on individual node configurations, or security mechanisms.
Various methods were proposed for the decentralized estimation of unstructured and structured networks~\cite{eli-sohl-dht-size-estimation,kostoulas2005decentralized, manku2003symphony}.
We use in this work a mechanism developed initially by Protocol Labs as part of a mechanism for decreasing the latency of publishing content in IPFS~\cite{network-size-estimation-notion,network-size-estimation-github-pr}.

%
%
%
%
%
%
%
%
%
%

Each node in the DHT refreshes its routing table periodically (every $10$ minutes in \texttt{libp2p}). 
For this, the node samples $m$ random keys (one for each bucket of its routing table)
%
and queries the DHT to obtain the $k=20$ closest peer IDs to each key.
Using these, the node then computes the average distance between each one of these keys $\key_j$ for $j=1,\dots,m$ and their $i$-th closest peer ID for $i=1,...,k$ (with $m=256$ and $k=20$).
\begin{equation}
    \label{equ:avg-dist}
    \overline{D}_i = \frac{1}{m} \sum_{j=1}^m \operatorname{dist}(\key_j, \peerid_{j}^{(i)})
\end{equation}
where $\peerid_{j}^{(i)}$ is the $i$-th closest peer ID to $\key_j$.
With $N$ peers in the DHT and peer IDs uniformly distributed in the hash space, the expected distance between a $\key$ and its $i$-th closest peer ID is $\frac{2^{256}i}{N+1}$. The node then runs a least square regression to compute the value of $N$ for which the expected distances best fit the empirical average distances, \ie,
\begin{equation}
    \label{equ:netsize-least-squares}
    \hat{N} = \arg\min_{N} \sum_{i=1}^k \left(\overline{D}_i - \frac{2^{256}i}{N+1}\right)^2.
\end{equation}
The resulting estimate $\hat{N}$ can be computed in closed form.
%

When a node starts running, it must perform DHT queries for a few random keys to initialize its network size estimate. 
Since a larger number of queries will result in higher accuracy, making more queries than what is needed to initialize one's routing table is recommended.
Thereafter, keeping the estimate up-to-date does not require any excess DHT queries beyond what is already used for refreshing the routing table as this is done frequently (every 10 minutes).

While the network size estimate has a stochastic variance resulting from the probability distribution of the honest peer IDs, it is hard for an attacker to bias the estimate significantly. Since the estimator uses the density of peer IDs around keys chosen uniformly at random, the adversary would require numerous Sybil nodes (on the order of the whole network size) to significantly affect the peer ID density around those keys.

\section{Source Code Analysis}
ChatGPT can be adopted to perform source code level analysis and address security-related issues, such as vulnerability discovery and fixing.
In this section, we first assess CodeBert's capability to comprehend pieces of code. Subsequently, we evaluate ChatGPT's proficiency in tackling specific security challenges.

\subsection{Code Semantic Inference}
``A source code can be understood in two ways: literal analysis, and logic analysis. 2) The literal analysis makes a conclusion based on the name of variables and functions, which is easier to analyze but is not always reliable. 3) The logic analysis requires a high-level understanding of the code, which is more reliable but hard to analyze.''~\cite{zhang2023features} 
Accurately analyzing source code not only requires the analyzer to comprehend the literal meaning of variable and function names but also to understand the underlying code logic. Gaining a precise understanding of the logic poses several challenges to analytical tasks. Firstly, the analyzer must be capable of tracking the data flow of variables within the code. Secondly, it needs to possess inter-procedural analysis abilities to comprehend code pieces that involve more than one function.

To assess ChatGPT's performance in source code semantic inference, we conducted experiments where ChatGPT analyzed various code pieces and answered questions related to the program's logic. Initially, we provided a C/C++ implementation of the Bubble Sort algorithm and inquired about its functionality. ChatGPT accurately analyzed the program and correctly identified its type, which aligned with our expectations.

Next, we challenged ChatGPT by obfuscating the variable and function names, replacing them with strings devoid of any specific semantic meaning. This aimed to test whether ChatGPT could still accurately interpret the code's logic despite the lack of meaningful names.

Furthermore, we deliberately broke down the algorithm into three functions to test ChatGPT's inter-procedural analysis capabilities. This evaluation aimed to determine if ChatGPT could effectively understand and connect the logic across multiple functions within the code.

\noindent{\bf Q2:} \textit{Please analyze the following algorithm and infer the algorithm type:}
\begin{lstlisting}[language=C,
    emph={abc},
    caption={{\bf Anonymized Source code of bubble sorting algirithm.} We anonymized the name of function and variable name incase the ChatGPT can infer the algorighm from literal meaning of the names.},label=code:bubble,captionpos=b]
    void abc(int a[], int b){
        int c, d;
        for (c = 0; c < b - 1; c++)
            for (d = 0; d < b - c - 1; d++)
                if (a[d] > a[d + 1]){
                    int e = a[d];
                    a[d] = a[d + 1];
                    a[d + 1] = e;
                }
    }
\end{lstlisting}

\begin{shaded*}
    \footnotesize
    \noindent{\bf Reply from ChatGPT:} The given algorithm is a sorting algorithm, and specifically, it is the Bubble Sort algorithm. This algorithm works by repeatedly swapping adjacent elements that are in the wrong order until the array is sorted.

    The algorithm begins by iterating through the array from the first element to the second to last element. For each iteration, it compares each element with the adjacent element to its right and swaps them if they are in the wrong order. This process continues until the end of the array is reached, and the largest element is placed in its correct position. The algorithm then repeats the process, iterating through the array again from the first element to the second to last element and swapping adjacent elements that are in the wrong order until the entire array is sorted.
\end{shaded*}
The response from ChatGPT in this case is accurate, indicating its ability to learn the logic features from the provided source code. Encouraged by this result, we proceeded to challenge ChatGPT further by splitting the algorithm into three functions and prompting it to infer the overall algorithm.

\noindent{\bf Q3:} \textit{Please analyze the following algorithm and infer the algorithm type:}
\begin{lstlisting}[language=C++,
    emph={lmn,efg,abc},
    caption={{\bf Source code of bubble sorting algirithm.} We split the algorithm into three functions.},label=code:split,captionpos=b]
    void lmn(int * a, int *b){
        if ( *a > *b){
            int c = *a;
            *a = *b;
            *b = c;
        }
    }
    void efg(int a[], int c, int b){
        int d;
        for (d = 0; d < b - c - 1; d++){
            lmn( & a[d], & a[d+1]); 
        }        
    }
    void abc(int a[], int b)
    {
        int c, d;
        for (c = 0; c < b - 1; c++){
            efg(a, c, b);
        }
    }
\end{lstlisting}
\begin{shaded*}
    \footnotesize
    \noindent{\bf Reply from ChatGPT:} The given algorithm is implementing the Bubble Sort algorithm. It has three functions, lmn, efg, and abc.

    The lmn function swaps the values of two integers if the first integer is greater than the second. It takes two pointers to integers as input and returns void.
    
    The efg function is responsible for performing one pass of the bubble sort algorithm. It takes an array a, starting index c, and ending index b as inputs. Within this function, it calls the lmn function to compare and swap adjacent elements in the subarray from index c to b-1.
    
    The abc function is the main function implementing the bubble sort algorithm. It takes an array a and its length b as inputs. It calls the efg function b-1 times with different starting and ending indices in each pass to sort the entire array.
    
    Therefore, the algorithm type is Bubble Sort.
\end{shaded*}
Upon careful evaluation of ChatGPT's response, we can confidently conclude that all the details provided are accurate. The answer demonstrates ChatGPT's capability to analyze the data flow across multiple functions, showcasing its potential for inter-procedural analysis. This promising result suggests that ChatGPT has the ability to perform complex tasks involving the understanding and inference of program logic across different functions.

\subsection{Vulnerabilities Analysis and Bug Fixing}
In the preceding section, we highlighted ChatGPT's proficiency in analyzing program semantics and logic. Now, in this section, our attention shifts to several specific security problems, where we put ChatGPT's problem-solving ability to the test. We aim to assess how effectively ChatGPT can address these security challenges and leverage its potential in security-oriented program analysis.

\noindent{\bf Q4:} \textit{Does this piece of code have vulnerability?}
\begin{lstlisting}[language=C++,
    emph={strcpy,printf,main},
    caption={{\bf A piece of code that contains a buffer overflow vulnerability.}},label=code:overflow,captionpos=b]
    int main(int argc, char *argv[])
    {
        char buffer[5];  
        strcpy(buffer, argv[1]);
        printf("buffer content= %s\n", buffer);
        return 0;
    }
\end{lstlisting}

\begin{shaded*}
    \footnotesize
    \noindent{\bf Reply from ChatGPT:} Yes, this piece of code has a vulnerability known as a buffer overflow.

    The variable buffer is defined as an array of 5 characters, which means it can hold up to 4 characters plus the null terminator. However, the program copies the content of argv[1] into buffer using the \texttt{strcpy} function, which does not check if the source string is larger than the destination buffer.
    
    If argv[1] contains more than 4 characters (or the null terminator), \texttt{strcpy} will write past the end of buffer, potentially overwriting other important data on the stack. This can cause unpredictable behavior, crashes, and even security vulnerabilities like code injection.
\end{shaded*}
The analysis provided by ChatGPT has been remarkably accurate thus far. However, in order to further challenge its capabilities, we decided to replace the \texttt{strcpy} function with a home-made string copy function. Subsequently, we posed the same question to ChatGPT to observe how well it can handle this modified scenario.

\noindent{\bf Q5:} \textit{Does this piece of code have vulnerability?}
\begin{lstlisting}[language=C++,
    emph={abc,printf,main},
    caption={{\bf A piece of code that contains a buffer overflow vulnerability.} We adopt a home-made  string copy function~\textendash~\texttt{abc}.},label=code:overflow,captionpos=b]
    char* abc(char* def, const char* hij)
    {
        if (def == NULL) {
            return NULL;
        }
        char *ptr = def;
        while ( *hij != 0)
        {
            *def = *hij;
            def++;
            hij++;
        }
        *def = 0;
        return ptr;
    }
    int main(int argc, char *argv[])
    {
        char xyz[5];  
        abc(xyz, argv[1]);
        printf("xyz content= %s\n", xyz);
        return 0;
    }
\end{lstlisting}
\begin{shaded*}
    \footnotesize
    \noindent{\bf Reply from ChatGPT:} 
    Yes, the provided code contains a vulnerability related to the use of the abc function and the potential for buffer overflow. The vulnerable code is in the abc function.

    The issue arises from the fact that there is no check on the size of the destination buffer (\texttt{def}) before copying data from the source (\texttt{hij}). The function uses a simple loop to copy characters from hij to def until it reaches the null-terminator (`0'). However, it does not verify whether the destination buffer has enough space to hold the entire source string.

    In the main function, the \texttt{xyz} buffer is declared with a size of 5 characters. If the user provides a command-line argument (argv[1]) longer than 4 characters (since one byte is reserved for the null-terminator), the abc function will copy the characters beyond the bounds of the xyz buffer, leading to a buffer overflow. This behavior results in undefined behavior, potentially overwriting adjacent memory or causing segmentation faults.

    To fix this vulnerability, you should modify the abc function to take the size of the destination buffer as an additional argument and ensure that it does not exceed the buffer size during the copying process. Alternatively, you can use safer string functions like 
    \texttt{strncpy} or \texttt{memcpy} that allow specifying the size of the destination buffer to avoid buffer overflows.
\end{shaded*}
The analysis conducted by ChatGPT on the vulnerability in the code was impressively accurate and detailed, showcasing its potential to identify vulnerabilities in source code effectively. However, in an effort to explore its understanding of potential security risks, we posed another question, asking ChatGPT how one could exploit this vulnerability to launch an attack.

\noindent{\bf Q6:} \textit{How to leverage this vulnerability to launch attack? }
\begin{shaded*}
    \footnotesize
    \noindent{\bf Reply from ChatGPT:} 
    This program has a buffer overflow vulnerability, which can be exploited by an attacker to gain control of the program execution and possibly execute malicious code. Here's an example of how an attacker could leverage this vulnerability to launch an attack: 
    
    The attacker crafts a malicious input string that is longer than 5 bytes, such as 
    \begin{lstlisting}[language=C++,
        emph={strcpy,strlen,main,printf},
        label=code:bit,captionpos=b, numbers=none, frame=none]
"AAAAAAAAAAAAAAAAAAAAAAAAAAAAAAAABBBB"
    \end{lstlisting}
    The attacker runs the program with the malicious input as a command-line argument, like this: 
    \begin{lstlisting}[language=C++,
        emph={strcpy,strlen,main,printf},
        label=code:bit,captionpos=b, numbers=none, frame=none]
"./program AAAAAAAAAAAAAAAAAAAAAAAAAAAAAAAABBBB" 
    \end{lstlisting}
    When the program executes the \texttt{strcpy} function with the malicious input, it copies the entire string (including the "BBBB" part) into the buffer, overwriting the return address on the stack. The attacker's code in the "BBBB" part of the input string gets executed instead of the intended code, giving the attacker control of the program execution and potentially allowing them to execute arbitrary code or launch further attacks.
\end{shaded*}
The exploit proposed by ChatGPT, although not directly usable to launch attacks, was conceptually correct. This suggests that ChatGPT possesses the ability to assist attackers in generating potential exploits. In the subsequent step, we decided to investigate whether ChatGPT could also provide solutions to fix the identified vulnerability.

\noindent{\bf Q7:} \textit{Can you rewrite the code to avoid the vulnerability?}
\begin{shaded*}
    \footnotesize
\noindent{\bf Reply from ChatGPT:} 
Yes, we can rewrite the code to avoid the buffer overflow vulnerability. Here's an example:
\begin{lstlisting}[language=C++,
    emph={strcpy,strlen,main,printf},
    caption={{\bf Rewritten code by ChatGPT to fix the buffer overflow vulnerability.}},label=code:bit,captionpos=b]
    int main(int argc, char *argv[]) {
        char buffer[BUFFER_SIZE];
        int len;
        if (argc < 2) {
            printf("Usage: %s <string>\n", argv[0]);
            return 1;
        }
        len = strlen(argv[1]);
        if (len >= BUFFER_SIZE) {
            printf("Input string is too long\n");
            return 1;
        }
        strcpy(buffer, argv[1]);
        printf("buffer content = %s\n", buffer);
        return 0;
    }
\end{lstlisting}

In this version of the code, we have defined a constant BUFFER\_SIZE to set the maximum size of the buffer. Before copying the input string to the buffer, we check its length to ensure it will not overflow the buffer. If the input string is too long, we print an error message and exit the program. This way, we ensure that the buffer will never be overflowed.
\end{shaded*}
Our human evaluation has demonstrated that the proposed patching effectively addresses the vulnerability in the code, providing an effective fix. As a result, we believe that ChatGPT holds the potential to assist developers and security analysts in identifying and resolving vulnerabilities in programs.

In the subsequent step, we assess its capability to suggest protection schemes that can thwart potential exploits by attackers. By leveraging ChatGPT's insights and analysis, we aim to explore its ability to propose robust security measures that can safeguard against known vulnerabilities and potential attack vectors.

\noindent{\bf Q8:} \textit{Can you propose a method to detect the vulnerability at program running time?}

\begin{shaded*}
    \footnotesize
\noindent{\bf Reply from ChatGPT:} 
One method to detect the vulnerability at program running time is to use a technique called ``canary values'' or ``stack cookie''. This technique involves adding a random value (the ``canary'') to the stack before the buffer, and checking that the canary has not been overwritten when the function returns.

To implement this technique, we can modify the original code as follows:
\begin{lstlisting}[language=C++,
    emph={memcpy,strlen,main,printf},
    label=code:canary,captionpos=b]
    int main(int argc, char *argv[]){
        char buffer[5];
        int canary = rand();
        printf("Canary value: %d\n", canary);
        memcpy(buffer, &canary, sizeof(canary));
        memcpy(buffer + sizeof(canary), argv[1], strlen(argv[1]) + 1);
        if (canary != *(int*)buffer) {
            printf("Buffer overflow detected!");
        } else {
            printf("buffer content = %s\n", buffer + sizeof(canary));
        }
        return 0;
    }
\end{lstlisting}
In this modified code, we add a random integer canary value to the stack before the buffer. We then check if the canary value has been modified after the user input is copied into the buffer. If the canary value has been modified, it indicates that a buffer overflow has occurred, and we can take appropriate action, such as terminating the program or alerting the user.

Note that this technique is not foolproof and can be bypassed in certain cases. It is still important to ensure that buffers are properly sized and user input is properly validated to avoid buffer overflow vulnerabilities in the first place.

\end{shaded*}
The proposed protection scheme suggested by ChatGPT demonstrates a conceptually correct approach in detecting buffer overflows through the use of a ``canary''. However, upon closer inspection, we identify some implementation errors that may impact its effectiveness. The placement of the canary at the beginning of the buffer, instead of at the end, renders it unable to detect buffer overflows that overwrite adjacent variables located at higher memory addresses. Additionally, the revised code introduces the ``canary'' into the buffer without correspondingly increasing the buffer size, limiting its capacity to handle strings as long as the original implementation.

These issues raise concerns regarding ChatGPT's generalization ability. While ChatGPT appears to understand the concept of using a ``canary'' for buffer overflow detection, it seems to struggle in correctly applying this knowledge to specific code pieces that were not encountered during its training. This observation leads us to question whether ChatGPT has effectively learned the intricacies of canary-based detection schemes from its training corpus, yet lacks the capability to seamlessly apply this knowledge to novel scenarios.

To further investigate ChatGPT's generalization capabilities, we proceed to test its performance on a diverse set of unseen code pieces in the subsequent section. This evaluation aims to shed light on the extent to which ChatGPT can effectively apply its learned knowledge to previously unseen code and discern its ability to adapt its understanding across different contexts. Understanding ChatGPT's strengths and limitations in generalization will be crucial in maximizing its potential for security-oriented program analysis and generating reliable solutions in a broader range of scenarios.

\subsection{Generalization Ability in Code Review}
\label{sec:gene}
ChatGPT has demonstrated its utility in various source code analysis tasks, including code review, code optimization, vulnerability mining, complexity analysis, and code rewriting. It possesses the capability to generate syntactically correct code and offer valuable insights and suggestions. However, the extent to which its generated content relies solely on the training dataset remains uncertain, considering the massive amount of data it was trained on.

In this section, we aim to investigate whether ChatGPT can effectively analyze and comprehend code that was not included in its training dataset. To evaluate this, we randomly selected four code snippets from LeetCode, representing varying levels of difficulty, and assessed ChatGPT's performance on three distinct source code analysis tasks.

\begin{table}[t]
    \centering
    \footnotesize
    \caption{Selected code snippets from the LeetCode.}
    \label{table:cases}
    % \begin{adjustbox}{max width=\columnwidth}
    \begin{threeparttable}
    \begin{tabular}{c|c|c|c|c}
    \toprule
    Leetcode ID  & Code~\tnote{1}  & Type                & Difficulty & Language  \\ \midrule
    \href{https://leetcode.com/problems/super-egg-drop/}{887} & \href{https://github.com/Moirai7/ChatGPTforSourceCode/blob/main/SupperEggDrop.cpp}{SupperEggDrop} & Dynamic Programming & Hard       & C++  \\ \midrule
      \href{https://leetcode.com/problems/disconnect-path-in-a-binary-matrix-by-at-most-one-flip/}{2556}           &       \href{https://github.com/Moirai7/ChatGPTforSourceCode/blob/main/DisconnectPath.cpp}{isPossibleToCutPath}        &       Depth-First Search              &     Medium       &    C++         \\ \midrule
    \href{https://leetcode.com/problems/find-subsequence-of-length-k-with-the-largest-sum/}{2099} & \href{https://github.com/Moirai7/ChatGPTforSourceCode/blob/main/maxSubsequence.java}{maxSubsequence} & Heap & Easy       & Java \\ \midrule
    \href{https://leetcode.com/problems/car-fleet/description/}{853} & \href{https://github.com/Moirai7/ChatGPTforSourceCode/blob/main/carFleet.py}{carFleet} & Array & Medium       & Python  \\ 
    \bottomrule
    \end{tabular}
    \begin{tablenotes}
        \scriptsize
        \item[1] The source code is available at: \url{https://github.com/Moirai7/ChatGPTforSourceCode}.
      \end{tablenotes}
    \end{threeparttable}
    % \end{adjustbox}
    \end{table}

Table \ref{table:cases} presents the LeetCode problems we chose and the corresponding solutions, which will serve as the dataset for our evaluation. By analyzing ChatGPT's responses to these unseen code snippets, we seek to gain insights into its generalization capabilities and understand how effectively it can apply its learned knowledge to novel code scenarios. 
% Our findings will shed light on the robustness of ChatGPT's source code analysis abilities and provide valuable information for its potential applications in real-world scenarios beyond the confines of its training data.


\begin{lstlisting}[language=C++,
    emph={abc,printf,main},
    caption={A solution to \texttt{Supper Egg Drop} problem from LeetCode.},label=code:original,captionpos=b]
    int superEggDrop(int K, int N) {
        vector<vector<int>> dp(N + 1, vector<int>(K + 1, 0));
        int m = 0;
        while (dp[m][K] < N) {
            m++;
            for (int k = 1; k <= K; ++k)
                dp[m][k] = dp[m - 1][k - 1] + dp[m - 1][k] + 1;
        }
        return m;
    }
\end{lstlisting}
ChatGPT can provide correctly review the original code snippets.
Taking the Supper Egg Drop problem in \autoref{code:original} as an example, it explains the purpose of the code snippet, the logic and complexity of the algorithm, and provides suggestions to improve the code.


\noindent{\bf Q9: Could you review \autoref{code:original}?} 
% \textit{Please analyze the algorithm and infer the program logic.}
\begin{shaded*}
    \footnotesize
\noindent{\bf Reply from ChatGPT:} 
The code appears to be a dynamic programming solution to the  ``egg drop'' problem, where one has $K$ eggs and a building with $N$ floors, and the objective is to find the minimum number of drops required to determine the floor from which the egg breaks.

The code creates a 2D vector dp with dimensions $(N+1) \times (K+1)$ and initializes all elements to $0$. It then iterates through values of $m$ (starting from $1$) until the value of $dp[m][K]$ is greater than or equal to $N$. For each value of $m$, it updates the values of $dp[m][k]$ for each egg $(k)$, based on the recurrence relation $dp[m][k] = dp[m-1][k-1] + dp[m-1][k] + 1$. Finally, it returns the value of $m$.

Overall, the code looks correct and efficient. It correctly computes the minimum number of drops required to determine the floor from which the egg breaks. The use of dynamic programming allows the algorithm to avoid unnecessary computations and runs in $O(K \times N)$ time, which is optimal for this problem.

However, it should be noted that this solution is not the only one to the egg drop problem, and there are other approaches that can be taken to solve it. Also, the code could benefit from some comments to explain the purpose and logic of the algorithm.
\end{shaded*}

Then, we manually obfuscated the code snippets by renaming the variables and function names and insert dummy code that will not have a effect on its results. The obfuscated code is shown in \autoref{code:obfuscated}.

\begin{lstlisting}[language=C++,
    emph={abc,printf,main},
    caption={\textbf{Obfuscated code based on \autoref{code:original}}. We renaming the variables and function names and insert dummy code that will not have a effect on its results.},label=code:obfuscated,captionpos=b]
    // obfuscated code snippet
    int gyjdfgadfsfvcvjhyt(int kdsfjaldfie55qwer, int kdsfja) {
        vector<vector<int>> kdasd35sfja(kdsfja + 1, vector<int>(kdsfjaldfie55qwer + 1, 0));
        int kdasd35sfjada = 0;
        int af567856dfg = kdasd35sfja.size(), abf567856dfg = kdasd35sfja[0].size(),
            abf5678s56dfg = 0, abf5678sss56dfg = abf567856dfg - 1;
            int abfss556dfg =0;

        while (kdasd35sfja[kdasd35sfjada][kdsfjaldfie55qwer] < kdsfja) {
            while (abf5678s56dfg < af567856dfg && abf5678sss56dfg > -1) {
                int abf556dfg = kdasd35sfja[abf5678s56dfg][abf5678sss56dfg];
                if (abf556dfg == abfss556dfg) break;
                if (abfss556dfg > abf556dfg) abf5678s56dfg++;
                else abf5678sss56dfg--;
            }
            kdasd35sfjada++;
            for (int sdf23423 = 1; sdf23423 <= kdsfjaldfie55qwer; ++sdf23423)
                kdasd35sfja[kdasd35sfjada][sdf23423] = kdasd35sfja[kdasd35sfjada - 1][sdf23423 - 1] + kdasd35sfja[kdasd35sfjada - 1][sdf23423] + 1;
        }
        return kdasd35sfjada;
    }
\end{lstlisting}
ChatGPT can not provide useful review of the obfuscated code. The bellowing is the generated review with the obfuscated code.

\noindent{\bf Q10:} \textit{Could you review \autoref{code:obfuscated}?}
\begin{shaded*}
    \footnotesize
\noindent{\bf Reply from ChatGPT:} 
The code you have provided appears to be written in C++ and defines a function called ``gyjdfgadfsfvcvjhyt''. Unfortunately, the function name and parameters do not give any indication of what the function is supposed to do. It would be helpful to have some context or explanation of the intended purpose of this function.

Regarding the code itself, there are some issues with readability and style that could be improved. For example, the use of single-letter variable names (\eg, ``kdsfjaldfie55qwer'', ``kdasd35sfja'') can make it difficult for someone reading the code to understand what is going on. Additionally, the inconsistent use of whitespace and indentation can make the code more difficult to read.

It's also worth noting that the code appears to be using nested while loops, which can be difficult to follow and potentially lead to performance issues if the loops are executing a large number of times. Without more context, it's difficult to say whether this is a problem in this particular case.

Finally, there are no comments or documentation explaining what the code is doing, which could make it difficult for other developers to understand and modify the code in the future.

Overall, I would recommend improving the readability and style of the code, adding comments and documentation, and providing more context and explanation of the intended purpose of the function.
\end{shaded*}

In essence, ChatGPT learns the literal features of code snippets that exist within its training data and generates responses based on statistical patterns involving characters, keywords, and syntactic structures. This knowledge encompasses various aspects like indentation, syntax, and common coding conventions. However, it lacks a profound comprehension of code semantics, logic, or context beyond the patterns and examples present in its training data. This limitation means that ChatGPT may not possess the ability to precisely review or assess the correctness, efficiency, or security of code without a more comprehensive understanding of its underlying semantics.
\vspace{-0.5em}\subsection{Input Generation}
Input generation is wide used in software testing, which aim to achieve a high coverage by triggering different program pathes.
This section, we aim to answer \ding{188} whether ChatGPT can offer valuable and effective seed inputs after understanding the source code.
Specifically, we test the ChatGPT's input generatation ability to achieve code coverage, input mutations, and  exeuction simulation.

Upon analyzing the responses ({\bf \ref{appendix:Q11}, \ref{appendix:Q12}, \ref{appendix:Q13}}) from ChatGPT, we observed that the provided seeds were comparable to those generated by AFL (American Fuzzy Lop) \cite{aflrefurl} using seed mutation strategies. However, it became apparent that ChatGPT did not propose highly effective seeds (capable of triggering the vulnerability). This observation suggests that ChatGPT may face challenges in leveraging the program semantics it learned from the code to generate such effective seeds.
There could be two possible reasons for this:

\begin{enumerate}
\item The prompts we use may not have been designed to specifically encourage ChatGPT to produce the most useful and targeted solutions.\lan{remove this. if it is true, we could say our previous case studies are not correct because of prompts. this is out-of-scope}
\item ChatGPT's responses might be too general, primarily relying on the information present in the corpus it was trained on, rather than effectively applying the learned program semantics to new situations.
\end{enumerate}

\lan{I vote to remove this. then we only have to kinds of case study: semantic and vunlerbiltiy. it would be easier to explain the storyline}
%Understanding these limitations is crucial for utilizing ChatGPT effectively in vulnerability discovery and other security-oriented tasks. Further research is needed to refine the prompts and explore strategies to enhance ChatGPT's ability to provide more precise and context-aware responses in security-related scenarios.



\section{Assemble Code Analysis}
In this section, we investigate the ChatGPT's ability to perform the binary code analysis. We first test ChatGPT's ability on the binary disassembling, and then ask questions about the semantic of the code.  

\noindent{\bf Q14:} \textit{The disassemble code of \texttt{`c7 45 cc 0a 00 00 00 c7 45 c8 14 00 00 00 8b 75 cc 03 75 c8 89 75 c4'} in Intel 64 instruction set?}
\lstdefinelanguage
   [x64]{Assembler}     % add a "x64" dialect of Assembler
   [x86masm]{Assembler} % based on the "x86masm" dialect
   % with these extra keywords:
   {morekeywords={CDQE,CQO,CMPSQ,CMPXCHG16B,JRCXZ,LODSQ,MOVSXD, %
                  POPFQ,PUSHFQ,SCASQ,STOSQ,IRETQ,RDTSCP,SWAPGS, %
                  rax,rdx,rcx,rbx,rsi,rdi,rsp,rbp, %
                  r8,r8d,r8w,r8b,r9,r9d,r9w,r9b, %
                  r10,r10d,r10w,r10b,r11,r11d,r11w,r11b, %
                  r12,r12d,r12w,r12b,r13,r13d,r13w,r13b, %
                  r14,r14d,r14w,r14b,r15,r15d,r15w,r15b}} % etc.

\lstset{language=[x64]Assembler}
\begin{shaded*}
    \footnotesize
\noindent{\bf Reply from ChatGPT:} 
\begin{lstlisting}[emph={},
	label=code:assem,captionpos=b, numbers=none, frame=none]
    c7 45 cc 0a 00 00 00     mov     DWORD PTR [rbp-0x34], 0xa
    c7 45 c8 14 00 00 00     mov     DWORD PTR [rbp-0x38], 0x14
    8b 75 cc                 mov     esi, DWORD PTR [rbp-0x34]
    03 75 c8                 add     esi, DWORD PTR [rbp-0x38]
    89 75 c4                 mov     DWORD PTR [rbp-0x3c], esi
    
\end{lstlisting}
This is the disassembled code in Intel 64 instruction set:
\begin{enumerate}
\item \texttt{mov DWORD PTR [rbp-0x34], 0xa} ~\textendash~ Moves the immediate value \texttt{0xa} into the double-word (4 bytes) memory location at \texttt{[rbp-0x34]}.

\item \texttt{mov DWORD PTR [rbp-0x38], 0x14} ~\textendash~ Moves the immediate value \texttt{0x14} into the double-word memory location at \texttt{[rbp-0x38]}.

\item \texttt{mov esi, DWORD PTR [rbp-0x34]} ~\textendash~ Moves the double-word value from memory location \texttt{[rbp-0x34]} into the \texttt{esi} register.

\item \texttt{add esi, DWORD PTR [rbp-0x38]} ~\textendash~ Adds the double-word value from memory location \texttt{[rbp-0x38]} to the \texttt{esi} register.

\item \texttt{mov DWORD PTR [rbp-0x3c], esi} ~\textendash~ Moves the value in the esi register into the double-word memory location at \texttt{[rbp-0x3c]}.
\end{enumerate}
\end{shaded*}
Our evaluation shows that that ChatGPT can disassemble the binary code correctly. 
In the next question, we ask about the semantic information about assemble code. 

\noindent{\bf Q15:} \textit{
After the execution of intel64 machine code \texttt{`c7 45 cc 0a 00 00 00 c7 45 c8 14 00 00 00 8b 75 cc 03 75 c8 89 75 c4'}. What is the value in register \texttt{`\%esi'}, memory \texttt{`rbp-0x34'} and memory \texttt{`rbp-0x3c'}?}

\begin{shaded*}
\footnotesize
\noindent{\bf Reply from ChatGPT:} 
Let's simulate the execution of the provided Intel 64 machine code step by step to determine the values in register \texttt{\%esi}, memory location \texttt{rbp-0x34}, and memory location \texttt{rbp-0x3c} after the execution.

Assuming that the initial values of the relevant registers and memory locations are 0, here's the execution:
\begin{enumerate}
\item \texttt{mov DWORD PTR [rbp-0x34], 0xa} ~\textendash~ Moves the immediate value \texttt{0xa} into the memory location \texttt{[rbp-0x34]}.

Value at \texttt{[rbp-0x34]: 0x0000000a}
\item \texttt{mov DWORD PTR [rbp-0x38], 0x14} ~\textendash~ Moves the immediate value \texttt{0x14} into the memory location [rbp-0x38].
Value at \texttt{[rbp-0x38]: 0x00000014}
\item \texttt{mov esi, DWORD PTR [rbp-0x34]} ~\textendash~ Moves the double-word value from memory location \texttt{[rbp-0x34]} into the \texttt{\%esi} register.

Value in \texttt{\%esi: 0x0000000a}
\item \texttt{add esi, DWORD PTR [rbp-0x38]} ~\textendash~ Adds the double-word value from memory location [rbp-0x38] to the \texttt{\%esi} register.

Value in \texttt{\%esi: 0x0000001e (0x0a + 0x14 = 0x1e)}
\item \texttt{mov DWORD PTR [rbp-0x3c], esi} ~\textendash~ Moves the value in the \texttt{\%esi} register into the memory location \texttt{[rbp-0x3c]}.

Value at \texttt{[rbp-0x3c]: 0x0000001e}

\end{enumerate}

After the execution of the machine code, the final values are as follows:
\begin{enumerate}
\item Value in \texttt{\%esi: 0x0000001e}
\item Value at \texttt{[rbp-0x34]: 0x0000000a}
\item Value at \texttt{[rbp-0x3c]: 0x0000001e}
\end{enumerate}

\end{shaded*}
This example shows that ChatGPT can precise analyze the semantic of assemble code and precisely track the dataflow. 
In next step, 
we compile the \autoref{code:bubble} in to binary code, and let the ChatGPT to infer the type of algorithms.

\noindent{\bf Q16:} \textit{Please infer the type of the algorithm based on the assemble code of Intel-64 instruction set.}
\begin{shaded*}
    \footnotesize
    \noindent{\bf Reply from ChatGPT:} 
Based on the provided x86-64 assembly code, it's not immediately clear what type of algorithm is being implemented. The code appears to be a sequence of instructions without high-level constructs (e.g., loops, conditionals) that would typically help identify a specific algorithm or pattern.

The code seems to involve memory operations, arithmetic operations, and conditional jumps based on the comparisons (cmp) and jumps (je, jne, jg, jle, etc.). However, without additional context or a higher-level understanding of the code's purpose and the data being processed, it is challenging to determine the exact algorithm or function being implemented.

To infer the type of algorithm accurately, it would be helpful to have more information about the surrounding code, input data, and the expected output. If you have any additional context or specific information about the code's purpose or functionality, please provide it, and I'll do my best to help you further.
\end{shaded*}
The response from ChatGPT suggests that it lacks the capability to directly analyze high-level semantics from binary code. Additionally, we inquired about vulnerability analysis and patching related questions ({\bf Q5, Q6, and Q7}) using the same program but in its binary representation. Unfortunately, ChatGPT was unable to answer any of these questions successfully.

\section{Observations and Discussion}
This paper presents a series of experiments and empirical analyses aimed at elucidating ChatGPT's capacity for solving security-oriented program analysis. The insights derived from this research offer valuable guidance for future researchers and users in security area. Firstly, it demonstrates that ChatGPT, along with other large language models, opens up a novel avenue for security-oriented source code analysis, proving to be an efficient method for learning high-level semantics from well-named source code. In comparison to other language models like CodeBert and GraphCodeBert~\cite{feng2020codebert,guo2020graphcodebert,zhang2023features}, ChatGPT showcases superior performance in this domain. Even at the binary level, ChatGPT displays the ability to learn low-level semantics, including data flow propagation in assembly code and value sets calculation. This impressive learning capability paves the way for its potential application in security-oriented contexts.

Despite its strengths, there are certain limitations worth noting. One such limitation is that users and researchers should expect reduced model performance when applying ChatGPT to analyze source code that lacks sufficient information in variable, method, and function names. For instance, code generated from decompilation~\cite{katz2018using} or code that does not adhere to standard naming conventions~\cite{butler2015investigating} may yield lower accuracy based on questions {\bf Q9} and {\bf Q10}. This points to the model's limited generalization ability, as it lacks a deep understanding of code semantics, logic, or context beyond the patterns and examples present in its training data. Additionally, for very specific questions that involve implementation-level details, ChatGPT's proposed solutions may be conceptually correct at a high level but may not fully address the problem with precision. These limitations serve as areas for further investigation and improvement to enhance the model's performance in such scenarios.


% \begin{enumerate}
%     \item ChatGPT is able to track the data flow across the functions\cite{}.
%     \item Newly generated code in the LeeCode
%     \item Some complex context, the ChatGPT.
%     \item Read the eviction of Banned question.
%     \item ChatGPT plugin.
% \end{enumerate}



% \section{Code Reviewing}


% \section{Code Translation}
% \begin{tikzpicture}
% \node[align=left, text width=3.2in,fill=lightgray, rounded corners=2pt, anchor=north west] at (0,0) {
% \begin{minipage}{3.4in}
%  \obeylines
% Can you translate the following code to python: 
% \begin{verbatim}
% template<class T>
% inline T unaligned_load(const void* p) {
%     T ret;
%     memcpy(&ret, p, sizeof(T));
%     return ret;
% }
% \end{verbatim}
% \end{minipage}};
% \end{tikzpicture}
% %
% \begin{flushright}
% \begin{tikzpicture}
% \node[align=left, text width=3.2in,fill=mygreen, rounded corners=2pt, anchor=north west] at (2,0) {
% Sure! Here's the Python equivalent code:
% \begin{minipage}{3.4in}
%  \obeylines
% \begin{verbatim}
% import struct

% def unaligned_load(p):
%     return struct.unpack('!{}'.format(type(p).__name__), p)[0]
    
% \end{verbatim}
% \end{minipage}

% Note: This Python code assumes that the input p is a bytes-like object (e.g. bytes, bytearray, etc.). If p is a pointer to some memory location (as in the original C++ code), you will need to modify the code accordingly to read the data from that location.
% };
% \end{tikzpicture}
% \end{flushright}

% This translation is ok, given there is no context of how to use this function.

% \begin{tikzpicture}
% \node[align=left, text width=3.2in,fill=lightgray, rounded corners=2pt, anchor=north west] at (0,0) {
% \begin{minipage}{3.4in}
%  \obeylines
% Can you translate the following code to python:
% \begin{verbatim}
% template<class T>
% inline T unaligned_load(const void* p) {
%     T ret;
%     memcpy(&ret, p, sizeof(T));
%     return ret;
% }

% u_int32_t copy_then_shift(const u_int32_t *data) {
%     u_int32_t copy = unaligned_load<u_int32_t>(data);
%     copy = copy << 5;
%     return copy;
% }
% \end{verbatim}
% \end{minipage}};
% \end{tikzpicture}
% %
% \begin{flushright}
% \begin{tikzpicture}
% \node[align=left, text width=3.2in,fill=mygreen, rounded corners=2pt, anchor=north west] at (2,0) {
% Here's the Python translation of the provided code:
% \begin{minipage}{3.4in}
%  \obeylines
% \begin{verbatim}
% import struct

% def unaligned_load(p):
%     return struct.unpack("<I", p[:4])[0]

% def copy_then_shift(data):
%     copy = unaligned_load(data)
%     copy = copy << 5
%     return copy
    
% \end{verbatim}
% \end{minipage}
% Note that the Python implementation uses the struct module to unpack the binary data instead of memcpy. The unaligned\_load function is modified to unpack the data from little-endian byte order and to return the first value in the resulting tuple.

% };
% \end{tikzpicture}
% \end{flushright}





% \balance
\bibliographystyle{ACM-Reference-Format}
\bibliography{main}

\label{sec:apdx}
\begin{comment}
\section{System Architecture}
\label{appendix:architecture}
\system has a novel modularized system architecture with three key components: 
\emph{StreamManager}, 
\emph{TxnManager} and \emph{TxnScheduler}. 
These components are instantiated in each thread locally.
The execution outline of \system is presented in Algorithm~\ref{alg:algo}.
Transactional stream processing is continuous and potentially never ends (Line 1$\sim$8).
The dependency resolution and execution of state transactions are separated into two non-overlapping phases by punctuations~\cite{Tucker:2003:EPS:776752.776780} (Line 2 and 5), which guarantees that no subsequent input event will have a smaller timestamp. 
Effectively, a batch of state transactions is collected during the first phase, and processed during the second phase.

In the first phase (i.e., stream processing phase), 
the \emph{StreamManager} conducts preprocessing for every input event ($e$). Similar to some prior works~\cite{tstream}, state transactions may be issued but not immediately processed during preprocessing (Line 3).
The \emph{pre\_processing} and \emph{post\_processing} functions are exposed as APIs to users.
The \emph{TxnManager} handles dependency resolution (Line 4) among state transactions and insert decomposed operations to construct a \tpg. We discuss the detailed two-phase \tpg construction process in Section~\ref{subsec:construction}.

In the second phase  (i.e., transaction processing phase), 
the \emph{TxnManager} is first involved again to refine (Line 6) the constructed \tpg with further dependency resolution.
The \emph{TxnScheduler} 
schedules operations for concurrent execution based on the constructed \tpg according to the three dimensions of scheduling decisions (Line 7). 
In particular, a scheduling decision model $M$ is instantiated based on the constructed \tpg (Line 14).
\textbf{\circled{1}} Guided by $M$, execution threads adopt an exploration strategy (Section~\ref{subsec:explore}) to explore the constructed \tpg for operations available to be scheduled constrained by dependencies. 
\textbf{\circled{2}} 
During exploration, one or multiple operations may be treated as the 
% basic 
unit of scheduling (Section~\ref{subsec:granularity}). 
Subsequently, \textbf{\circled{3}} every thread executes operation(s) in the unit of scheduling with various abort handling mechanisms (Section~\ref{subsec:abort_handling}).
Only when state transactions are processed (i.e., committed or aborted) can the associated input events be postprocessed (Line 8) by the \emph{StreamManager} based on transaction processing results.
\end{comment}

\begin{comment}
\begin{algorithm}
\footnotesize
    \KwData{$e$ \tcp{Input event}}
    \KwData{$txn_{ts}$ \tcp{State transaction}}
    \KwData{$G$ \tcp{The currently constructed TPG}}
    \While{!finish processing of input streams}{
        \eIf(\tcp*[h]{Phase 1}){\text{$e$ is not a $punctuation$}}{
                $txn_{ts}$ $\gets$ PRE\_Processing($e$)\;
                \textbf{TPG\_Construction}($G$, $txn_{ts}$)\; 
          }(\tcp*[h]{Phase 2}){
                \textbf{TPG\_Refinement}($G$)\; 
                \textbf{TXN\_Scheduling}($G$)\; 
                POST\_Processing()\;
          }
    }
    
    \SetKwFunction{FMain}{TPG\_Construction}
    \SetKwProg{Fn}{Function}{:}{}
    \Fn{\FMain{$G$, $txn_{ts}$}}{
        $O_{1..k}$ $\gets$ \textbf{Partition} $txn_{ts}$\;
        \ForEach{\text{operation $O_{i}$ $\in$ $O_{1..k}$}}{
            \textbf{Identify} its \ld\;
            $G$ $\gets$ $G$ + $O_{i}$ \;
        }
    }
    \SetKwFunction{FMain}{TPG\_Refinement}
    \SetKwProg{Fn}{Function}{:}{}
    \Fn{\FMain{$G$}}{
        \ForEach{\text{vertex $e_{i}$ $\in$ $G$}}{
            \textbf{Identify} its \td, \pd\;
        }
    }
    
    \SetKwFunction{FMain}{TXN\_Scheduling}
    \SetKwProg{Fn}{Function}{:}{}
    \Fn{\FMain{$G$}}{
        $M$ $\gets$ Instantiated with $G$;\tcp{A decision model}
        \While{!finish scheduling of $G$
        }{
          \textbf{\circled{2}} $Scheduling Unit$ $\gets$ \textbf{\circled{1}} \emph{Explore}($G$, $M$)\; 
            \textbf{\circled{3}} \emph{Execute with Abort Handling} ($Scheduling Unit$)\; 
        }
    }
  \caption{Execution Outline of \system}
  \label{alg:algo}
\end{algorithm}
\end{comment}

\end{document} 
