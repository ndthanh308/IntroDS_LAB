\subsubsection{\textsc{StackelUCB} as a Special Case}
\label{app:special}
In \cref{fig:analysis} and \cref{sec:analysis} we give a comparison between \gpmw and \alg, and see that \gpmw can be seen as a special case of \alg with a one-node causal graph. \textsc{StackelUCB} \cite{sessa2020learning} is another online learning algorithm that can be seen as a special case of \alg. We show the graph in \cref{fig:stackel}. In \textsc{StackelUCB} an agent plays an unknown Stackelberg game against a changing opponent which at time $t$ has representation $a_{0,t}'$. After the agent selects an action $a_{0, t}$, the opponent sees this action and responds based upon a fixed but unknown response mechanism for that opponent: their response is $X_{0, t} = f_0(a_{0,t}', a_{0, t})$. Then, the response, game outcome $Y_t$, and opponent identity are revealed to the agent. The sequence of opponent identities, i.e. $\{a_{0,t}'\}_{t=1}^T$, can be selected potentially adversarially based on knowledge of the agent's strategy.

% Figure environment removed