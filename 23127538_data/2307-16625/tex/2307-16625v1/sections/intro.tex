\section{Introduction}

How can a scientist efficiently optimize an unknown function that is expensive to evaluate? This problem arises in automated machine learning, drug discovery and agriculture. \emph{Bayesian optimization} (BO) encompasses an array of algorithms for sequentially optimizing unknown functions \citep{movckus1975}. Classical BO algorithms treat the unknown function mostly as a black box and make minimal structural assumptions. By incorporating more domain knowledge about the unknown function into the algorithm, one can hope to optimize the function using fewer evaluations. 

A recent line of work on causal Bayesian optimization (CBO) \citep{agliettiCausalBayesianOptimization2020} attempts to integrate use of a structural causal model (SCM) into BO methods. It is assumed that actions are interventions on some set of observed variables, which are causally related to each other and a reward variable through a known causal graph (\cref{fig:overview}b), but unknown mechanisms. Many important BO problems might take such a shape.  
For example, managing supply in a Shared Mobility System (SMS) involves intervening on the distribution of bikes and scooters across the city (we study this further in our experiments).  
Importantly, \citet{sussex2022model} show that a BO approach leveraging the additional structure of CBO can achieve exponentially lower regret in terms of the action space size.\looseness=-1 

Most CBO methods to date assume that the system is completely stationary across interactions and that only one agent interacts with the system. However, in several examples it would be desirable to incorporate the influence of external events. For example, the effect of changing weather in crop management, demand patterns in a SMS, or the actions of other agents in multi-agent settings. In all such cases, we would like an algorithm that \emph{adapts} to the variability in these external events. %Existing CBO approaches won't achieve this adaptiveness because they can only model such events with the limited \iid noise model. 

In this work, we incorporate external events into CBO by introducing a novel \emph{adversarial} CBO (ACBO) setup, illustrated in \cref{fig:overview}c. Crucially, in ACBO the downstream reward is explicitly influenced by adversarial interventions on certain nodes in the causal graph (identified using dashed nodes in~\cref{fig:overview}c) that can only be observed a-posteriori. For this general setting, we propose a novel algorithm -- CBO with multiplicative weights (\alg) -- and prove a regret guarantee using a stronger (but natural) notion of regret than the one used in existing CBO works. For settings where the number of intervention targets is large, we propose a distributed version of \alg which is computationally efficient and can achieve approximate regret guarantees under some additional submodularity assumptions on the reward.
Finally, we find empirical evidence that \alg outperforms relevant prior work in adversarial versions of previously studied CBO benchmarks and in learning to re-balance units in an SMS simulation based upon real data. \looseness=-1

\section{Secure Design of \puma}\label{sec:design}
In this section, we first present an overview of \puma, and present the protocols for secure $\gelu$ , $\softmax$, embedding, and $\layernorm$ used by \puma. Note that the linear layers such as matrix multiplication are straightforward in replicated secret sharing, so we mainly describe our protocols for non-linear layers in this manuscript.

\subsection{Overview of \puma}\label{sec:overview}
To achieve secure inference of Transformer models, \puma\ defines three kinds of roles: one model owner, one client, and three computing parties. The model owner and the client  provide their models or inputs to the computing parties (i.e., $P_0$, $P_1$, and $P_2$) in a secret-shared form, then the computing parties execute the MPC protocols and send the results back to the client. Note that the model owner and client can also act as one of the computing party, we describe them separately for generality. \eg, when the model owner acts as $P_0$, the client acts as  $P_1$, a third-party dealer acts as $P_2$, the system model becomes the same with \mpcformer~\citep{li2023mpcformer}.

During the secure inference process, a key invariant is maintained: For any layer, the computing parties always start with 2-out-of-3 replicated secret shares of the previous layer's output and the model weights, and end with 2-out-of-3 replicated secret shares of this layer's output. As the shares do not leak any information to each party, this ensures that the layers can be sequentially combined for arbitrary depths to obtain a secure computation scheme for any Transformer-based model.
%The main focus of \puma\ is to reduce the computation and communication costs between the computing parties while maintaining the desired level of security. 



\iffalse
\textbf{Threat Model.}
Following previous works~\citep{aby3,li2023mpcformer},
\puma\ resists a semi-honest (a.k.a., honest-but-curious) adversary in honest-majority~\citep{lindell2009proof}, where the adversary passively corrupts no more than one computing party. Such an adversary follows the protocol specification exactly, but may try to learn more information than permitted. Please note that \puma\ cannot protect against the extraction of information from the inference results, and the examination of mitigating solutions (\eg, differential privacy~\citep{abadi2016deep}) falls outside the scope of this study.
\fi 

\subsection{Protocol for Secure GeLU}\label{sec:gelu}
Most of the current approaches view the $\gelu$ function as a composition of smaller functions and try to optimize each piece of them, making them to miss the
chance of optimizing the private $\gelu$ as a whole. Given the $\gelu$ function:
\begin{equation}\label{eq:gelu}
\begin{split}
    \gelu(x) &= \frac{x}{2} \cdot \left(1 + \tanh \left( \sqrt{\frac{2}{\pi}} \cdot \left(x + 0.044715 \cdot x^3 \right) \right) \right)\\
    &\approx x\cdot \mathsf{sigmoid}(0.071355\cdot x^3 + 1.595769\cdot x) 
\end{split},
\end{equation}
these approaches~\citep{hao2022iron,characmpctranformer} focus either on designing efficient protocols for function $\tanh$
or using the existing MPC protocols of exponentiation and reciprocal for $\mathsf{sigmoid}$. 

However, none of current approaches have utilized the fact that $\gelu$ function is almost linear on the two sides (\ie, $\gelu(x)\approx 0$ for $x<-4$ and $\gelu(x)\approx x$ for $x>3$). 
Within the short interval $[-4,3]$ of $\gelu$,
we suggest a piece-wise approximation of low-degree polynomials is a more efficient and easy-to-implement choice for its secure protocol. Concretely, our piece-wise low-degree polynomials are shown as equation~(\ref{eq:geluapprox}):
\begin{equation}\label{eq:geluapprox}
\gelu(x)=
\begin{cases}
0, & x<-4 \\
F_0(x), & -4 \le x < -1.95 \\
F_1(x), & -1.95 \le x \le 3 \\
x, & x >3
\end{cases},
\end{equation}
where polynomials $F_0()$ and $F_1()$ are computed by library $\mathsf{numpy.ployfit}$\footnote{\url{https://numpy.org/doc/stable/reference/generated/numpy.polyfit.html}} as equation~(\ref{eq:f0f1}). Surprsingly, the above simple poly fit works very well and our $\mathsf{max\ error}< 0.01403$, $\mathsf{median\ error}< 4.41e-05$, and $\mathsf{mean\ error}< 0.00168$.
\begin{equation}\label{eq:f0f1}
\begin{cases}
F_0(x) &= -0.011034134030615728 x^3 -0.11807612951181953 x^2 \\
&- 0.42226581151983866 x -0.5054031199708174\\
F_1(x) &= 0.0018067462606141187x^6 -0.037688200365904236 x^4 \\
&+ 0.3603292692789629x^2 + 0.5x + 0.008526321541038084
\end{cases}
\end{equation}

Formally, given secret input $\share{x}$, our secure $\gelu$ protocol $\Pi_{\gelu}$ is constructed as algorithm~\ref{protocol:gelu}. 
\iffalse
\begin{itemize}
    \item The parties jointly compute
$\share{b_0}^2 = \Pi_{\mathsf{LT}}(\share{x}, 4)$,
$\share{b_1}^2 = \Pi_{\mathsf{LT}}(\share{x}, -1.95)$, and
$\share{b_2}^2 = \Pi_{\mathsf{LT}}(3, \share{x})$.

\item  Then, each $P_i$ locally compute
$\share{b_3}^2 = \share{b_1}^2 \oplus \share{b_2}^ \oplus 1$ and
$\share{b_4}^2 = \share{b_0}^2 \oplus \share{b_1}^2$

\item Finally, the parties compute and return 
$\share{b_2}^2 \cdot \share{x} + \share{b_4}^2 \cdot F_0(\share{x}) + \share{b_3}^2 \cdot F_1(\share{x})$, where polynomials $(F_0, F_1)$ can be computed easily using secure addition and multiplication (and its variants, \eg, secure square)~\citep{spu}. 
\end{itemize}
\fi 

\begin{algorithm}[tp]
\caption{Secure $\gelu$ Protocol $\Pi_{\mathsf{GeLU}}$}\label{protocol:gelu}
\begin{algorithmic}[1]
\REQUIRE
$P_i$ holds the 2-out-of-3 replicate secret share $\share{x}_i$ for $i\in \{0,1,2\}$ 
\ENSURE
$P_i$ gets the 2-out-of-3 replicate secret share $\share{y}_i$ for $i\in \{0,1,2\}$, where $y=\gelu(x)$.

\STATE $P_0$, $P_1$, and $P_2$ jointly compute
\begin{equation*}
\begin{split}
&\shareb{b_0} = \Pi_{\mathsf{LT}}(\share{x}, -4),~~~\vartriangleright b_0 = 1\{x<-4\}\\
&\shareb{b_1} = \Pi_{\mathsf{LT}}(\share{x}, -1.95),~~~\vartriangleright b_1 = 1\{x<-1.95\} \\
&\shareb{b_2} = \Pi_{\mathsf{LT}}(3, \share{x}),~~~~~~\vartriangleright b_2 = 1\{3<x\}
\end{split}
\end{equation*}
and compute 
$\shareb{z_0} = \shareb{b_0} \oplus \shareb{b_1}$,
$\shareb{z_1} = \shareb{b_1} \oplus \shareb{b_2} \oplus 1$, and $\shareb{z_2}=\shareb{b_2}$. Note that $z_0 = 1\{-4\le x < -1.95\}$, $z_1 = 1\{-1.95\le x\le 3\}$, and $z_2 = 1\{x>3\}$.

\STATE Jointly compute $\share{x^2} = \Pi_{\mathsf{Square}}(\share{x})$, $\share{x^3} = \Pi_{\mathsf{Mul}}(\share{x}, \share{x^2})$, $\share{x^4} = \Pi_{\mathsf{Square}}(\share{x^2})$, and $\share{x^6} = \Pi_{\mathsf{Square}}(\share{x^3})$.

\STATE Computing polynomials $\share{F_0(x)}$ and $\share{F_1(x)}$ based on $\{\share{x}, \share{x^2}, \share{x^3}, \share{x^4}, \share{x^6}\}$ as equation~(\ref{eq:geluapprox}) securely.


\RETURN$\share{y} = \Pi_{\mathsf{Mul_{BA}}}(\shareb{z_0}, \share{F_0(x)}) + \Pi_{\mathsf{Mul_{BA}}}(\shareb{z_1}, \share{F_1(x)})+\Pi_{\mathsf{Mul_{BA}}}(\shareb{z_2}, \share{x})$.

\end{algorithmic}
\end{algorithm}



\subsection{Protocol for Secure Softmax}\label{sec:secureatten}

In the function $\attention(\Q,\K,\V)=
\softmax(\Q \cdot \K^\mathsf{T} + \M) \cdot \V$, where $\M$ can be viewed as a bias matrix, the key challenge is computing function $\softmax$. For the sake of numerical stability, the $\softmax$ function is computed as
\begin{equation}\label{eq:softmax}
    \softmax(\x)[i]=\frac{\exp(\x[i] - \bar{x} - \epsilon)}{\sum_i \exp(\x[i] - \bar{x} - \epsilon)},
\end{equation}
where $\bar{x}$ is the maximum element of the input vector $\x$. 
For the normal plaintext softmax, $\epsilon=0$. For a two-dimension matrix, we apply equation~(\ref{eq:softmax}) to each of its row vector.

Formally, our detailed secure protocol  $\Pi_{\softmax}$ is illustrated in algorithm~\ref{protocol:softmax}, where we propose two optimizations:
\begin{itemize}
\item 
For the first optimization, we set $\epsilon$ in equation~\ref{eq:softmax} to a tiny and positive
value, e.g., $\epsilon =
10^{-6}$, so that the inputs to exponentiation
in equation~\ref{eq:softmax} are all negative. We exploit the negative operands
for acceleration. Particularly, we compute the exponentiation using the Taylor series~\citep{tan2021cryptgpu} with a simple clipping
\begin{equation}\label{eq:negexp}
\mathsf{negExp}(x) = \begin{cases}
    0, &x < T_{\exp} \\
    (1+\frac{x}{2^t})^{2^t}, &x\in [T_{\exp},0].
\end{cases}
\end{equation}
Indeed, we apply the less-than for the branch $x < T_{\exp}$
The division by $2^t$ can be achieved using
$\Pi_{\mathsf{Trunc}}^t$ since the input is already negative. Also, we can
compute the power-of-$2^t$ using $t$-step sequences of square function $\Pi_{\mathsf{square}}$ and $\Pi_{\mathsf{Trunc}}^f$. Suppose our MPC program uses
$18$-bit fixed-point precision. Then we set $T_{\exp}=-14$ given $\exp(-14) < 2^{-18}$, and empirically set $t = 5$.


\item 
Our second optimization is to reduce the number of divisions, which ultimately saves computation and communication costs.
To achieve this, for a vector $\x$ of size $n$, we have replaced the operation $\mathsf{Div}(\x, \mathsf{Broadcast}(y))$ with $\x \cdot  \mathsf{Broadcast}(\frac{1}{y})$, where $y=\sum_{i=1}^n\x[i]$. By making this replacement, we effectively reduce $n$ divisions to just one reciprocal operation and $n$ multiplications.
This optimization is particularly beneficial in the case of the $\softmax$ operation. The $\frac{1}{y}$ in the $\softmax$ operation is still large enough to maintain sufficient accuracy under fixed-point values. As a result, this optimization can significantly reduce the computational and communication costs while still providing accurate results.
\end{itemize}

\begin{algorithm}[tp]
\caption{Secure $\softmax$ Protocol $\Pi_{\softmax}$}\label{protocol:softmax}
\begin{algorithmic}[1]
\REQUIRE
$P_i$ holds the 2-out-of-3 replicate secret share $\share{\x}_i$ for $i\in \{0,1,2\}$, and $\x$ is a vector of size $n$. 
\ENSURE
$P_i$ gets the 2-out-of-3 replicate secret share $\share{\y}_i$ for $i\in \{0,1,2\}$, where $\y=\softmax(\x)$.

\STATE $P_0$, $P_1$, and $P_2$ jointly compute
$\shareb{\mathbf{b}} = \Pi_{\mathsf{LT}}(T_{\exp}, \share{\x})$ and the maximum $\share{\bar{x}} = \Pi_{\mathsf{Max}}(\share{\x})$.

\STATE Parties locally computes $\share{\hat{\x}} = \share{\x} - \share{\bar{x}} - \epsilon$, and jointly compute $\share{\z_0} = 1+  \Pi_{\mathsf{Trunc}}^t(\share{\hat{\x}})$.

\FOR{$j=1,2,\dots, t$}
\STATE $\share{\z_j} = \Pi_{\mathsf{Square}}(\share{\z_{j-1}})$.
\ENDFOR

\STATE Parties locally compute $\share{z} = \sum_{i=1}^n \share{\z[i]}$ and jointly compute $\share{1/z} = \Pi_{\mathsf{Recip}}(\share{z})$.

\STATE Parties jointly compute $\share{\z / z} = \Pi_{\mathsf{Mul}}(\share{\z}, \share{1/z})$

\RETURN $\share{\y} = \Pi_{\mathsf{Mul}_{\mathsf{BA}}}( \shareb{\mathbf{b}}, \share{\z / z})$.

\end{algorithmic}
\end{algorithm}

\subsection{Protocol for Secure Embedding}\label{sec:embed}


The current secure embedding procedure described in~\citep{li2023mpcformer} necessitates the client to  generate a one-hot vector using the token $\tokenid$ locally. This deviates from a plaintext Transformer workflow where the one-hot vector is generated inside the model. As a result, they have to carefully strip off the one-hot step from the pre-trained models, and add the step to the client side, which could be an obstacle for deployment. 



To address this issue, we propose a secure embedding design as follows. Assuming that the token $\tokenid\in [n]$ and all embedding vectors are denoted by $\E= (\e_1^T, \e_2^T, \dots, \e_n^T)$, the embedding can be formulated as $\e_{\tokenid} = \mathbf{E}[\tokenid]$. Given $(\tokenid, \E)$ are in secret-shared fashion, our secure embedding protocol $\Pi_{\mathsf{Embed}}$ works as follows:
\begin{itemize}
    \item The computing parties securely compute the one-hot vector $\shareb{\mathbf{o}}$ after receiving $\share{\tokenid}$ from the client. Specifically, $\shareb{\mathbf{o}[i]}=\Pi_{\mathsf{Eq}}(i,\share{\tokenid})$ for $i\in [n]$.
    \item The parties can compute the embedded vector via $\share{\e_{\tokenid}} = \Pi_{\mathsf{Mul_{BA}}}(\share{\E}, \shareb{\mathbf{o}})$, where  does not require secure truncation.
\end{itemize}
In this way, our $\Pi_{\mathsf{Embed}}$ does not require explicit modification of the workflow of plaintext Transformer models, at the cost of more $\Pi_{\mathsf{Eq}}$ and $\Pi_{\mathsf{Mul_{BA}}}$ operations. 



\subsection{Protocol for Secure LayerNorm}\label{sec:seclayernorm}
Recall that given a vector $\x$ of size $n$, $\layernorm(\x)[i] =  \gamma \cdot \frac{\x[i]-\mu}{\sqrt{\sigma}} + \beta$, where $(\gamma, \beta)$ are trained parameters, $\mu = \frac{\sum_{i=1}^n \x[i]}{n}$, and $\sigma = \sum_{i=1}^n (\x[i] - \mu)^2$. In MPC, the key challenge is the evaluation of the divide-square-root $\frac{\x[i]-\mu}{\sqrt{\sigma}}$ formula. To securely evaluate this formula, CrypTen sequentially executes the MPC protocols of square-root, reciprocal, and multiplication. However, we observe that $\frac{\x[i]-\mu}{\sqrt{\sigma}}$ is equal to $(\x[i]-\mu)\cdot \sigma^{-1/2}$. And in the MPC side, the costs of computing the inverse-square-root $\sigma^{-1/2}$ is similar to that of the square-root operation~\citep{rSqrt}. Besides, inspired by the second optimization of \S~\ref{sec:secureatten}, we can first compute $\sigma^{-1/2}$ and then $\mathsf{Broadcast}(\sigma^{-1/2})$ to support fast and secure $\layernorm(\x)$. And our formal protocol $\Pi_{\layernorm}$ is shown in algorithm~\ref{protocol:layernorm}.

\begin{algorithm}[tp]
\caption{Secure $\mathsf{LayerNorm}$ Protocol $\Pi_{\mathsf{LayerNorm}}$}\label{protocol:layernorm}
\begin{algorithmic}[1]
\REQUIRE
$P_i$ holds the 2-out-of-3 replicate secret share $\share{\x}_i$ for $i\in \{0,1,2\}$, and $\x$ is a vector of size $n$. 
\ENSURE
$P_i$ gets the 2-out-of-3 replicate secret share $\share{\y}_i$ for $i\in \{0,1,2\}$, where $\y=\mathsf{LayerNorm}(\x)$.

\STATE $P_0$, $P_1$, and $P_2$ compute $\share{\mu} = \frac{1}{n}\cdot \sum_{i=1}^n\share{\x[i]}$ and $\share{\sigma} = \sum_{i=1}^n \Pi_{\mathsf{Square}}(\share{\x} - \share{\mu})[i]$.

\STATE Parties jointly compute $\share{\sigma^{-1/2}} = \Pi_{\mathsf{rSqrt}}(\share{\sigma})$.

\STATE Parties jointly compute $\share{\mathbf{c}} = \Pi_{\mathsf{Mul}}((\share{\x} - \share{\mu}), \share{\sigma^{-1/2}})$

\RETURN $\share{\y} = \Pi_{\mathsf{Mul}}(\share{\gamma}, \share{\mathbf{c}}) + \share{\beta}$.

\end{algorithmic}
\end{algorithm}
