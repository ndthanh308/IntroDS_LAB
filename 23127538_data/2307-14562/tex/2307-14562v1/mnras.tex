% mnras_template.tex 
%
% LaTeX template for creating an MNRAS paper
%
% v3.0 released 14 May 2015
% (version numbers match those of mnras.cls)
%
% Copyright (C) Royal Astronomical Society 2015
% Authors:
% Keith T. Smith (Royal Astronomical Society)

% Change log
%
% v3.0 May 2015
%    Renamed to match the new package name
%    Version number matches mnras.cls
%    A few minor tweaks to wording
% v1.0 September 2013
%    Beta testing only - never publicly released
%    First version: a simple (ish) template for creating an MNRAS paper

%%%%%%%%%%%%%%%%%%%%%%%%%%%%%%%%%%%%%%%%%%%%%%%%%%
% Basic setup. Most papers should leave these options alone.
\documentclass[fleqn,usenatbib]{mnras}

% MNRAS is set in Times font. If you don't have this installed (most LaTeX
% installations will be fine) or prefer the old Computer Modern fonts, comment
% out the following line
\usepackage{newtxtext,newtxmath}
% Depending on your LaTeX fonts installation, you might get better results with one of these:
%\usepackage{mathptmx}
%\usepackage{txfonts}
\usepackage{mathtools}
\usepackage{xspace}

% Use vector fonts, so it zooms properly in on-screen viewing software
% Don't change these lines unless you know what you are doing
\usepackage[T1]{fontenc}

% Allow "Thomas van Noord" and "Simon de Laguarde" and alike to be sorted by "N" and "L" etc. in the bibliography.
% Write the name in the bibliography as "\VAN{Noord}{Van}{van} Noord, Thomas"
\DeclareRobustCommand{\VAN}[3]{#2}
\let\VANthebibliography\thebibliography
\def\thebibliography{\DeclareRobustCommand{\VAN}[3]{##3}\VANthebibliography}


%%%%% AUTHORS - PLACE YOUR OWN PACKAGES HERE %%%%%

% Only include extra packages if you really need them. Common packages are:
\usepackage{graphicx}	% Including figure files
\usepackage{amsmath}	% Advanced maths commands
% \usepackage{amssymb}	% Extra maths symbols
\usepackage{threeparttable} % For tablenote

%%%%%%%%%%%%%%%%%%%%%%%%%%%%%%%%%%%%%%%%%%%%%%%%%%

%%%%% AUTHORS - PLACE YOUR OWN COMMANDS HERE %%%%%

% Please keep new commands to a minimum, and use \newcommand not \def to avoid

% FORCE ROMAN FONTS IN SUBSCRIPTS
% Sam Geen - I added this to save you using "\rm" in every subscript. Remove it if you like.
\begingroup
\catcode`\_=\active
\gdef_#1{\ensuremath{\sb{\mathrm{#1}}}}
\endgroup
\mathcode`\_=\string"8000
\catcode`\_=12

% overwriting existing commands. Example:
\newcommand{\ie}{{\it i.e.}}     
\newcommand{\eg}{{\it e.g.}}  
\newcommand{\cc}{cm$^{-3}$}                          % cm^-3
\newcommand{\hcc}{H~cm$^{-3}$}                       % H/cm^3
\newcommand{\HII}{H\,{\sc ii}\xspace}
\newcommand{\hm}{H$_2$}                              % H_2
\newcommand{\jwst}{\textit{JWST}\xspace}
\newcommand{\msun}{{\rm M}_{\odot}}                  % solar masses
\newcommand{\msyr}{{\rm M}_{\odot}~{\rm yr}^{-1}}    % Msun/yr
\newcommand{\nh}{n_{H}}                          % n_H
\newcommand{\ramses}{{\sc ramses}}                % RAMSES
\newcommand{\ramsesrt}{{\sc ramses-rt}}           % RAMSES-RT
\newcommand{\subi}{\textit{i}}
\newcommand{\subj}{\textit{j}}
\newcommand{\highlight}[1]{\textcolor{red}{#1}}
\newcommand{\MR}[1]{\textcolor{magenta}{#1}}
\newcommand{\outward}{Fiducial-Outward\xspace}      
\newcommand{\inward}{Fiducial-Inward\xspace}      
\newcommand{\non}{Fiducial-Neutral\xspace}      

%%%%%%%%%%%%%%%%%%%%%%%%%%%%%%%%%%%%%%%%%%%%%%%%%%

%%%%%%%%%%%%%%%%%%% TITLE PAGE %%%%%%%%%%%%%%%%%%%

% Title of the paper, and the short title which is used in the headers.
% Keep the title short and informative.
\title[Outward migration of Population~III stars]{On the origin of outward migration of Population~III stars}

% The list of authors, and the short list which is used in the headers.
% If you need two or more lines of authors, add an extra line using \newauthor
\author[J. Park, M. Ricotti and K. Sugimura]{
Jongwon Park,$^{1}$\thanks{E-mail: jwpark@umd.edu} 
Massimo Ricotti,$^{1}$\thanks{E-mail: ricotti@umd.edu}
and Kazuyuki Sugimura$^{2}$
\\
% List of institutions
$^{1}$Department of Astronomy, University of Maryland, College Park, MD 20742, USA\\
$^{2}$Faculty of Science, Hokkaido University,
Sapporo, Hokkaido 060-0810, Japan\\
}

% These dates will be filled out by the publisher
\date{Accepted XXX. Received YYY; in original form ZZZ}

% Enter the current year, for the copyright statements etc.
\pubyear{2023}

% Don't change these lines
\begin{document}
\label{firstpage}
\pagerange{\pageref{firstpage}--\pageref{lastpage}}
\maketitle

% Abstract of the paper
\begin{abstract}
Outward migration of massive binary stars or black holes in their circumbinary disc is often observed in simulations and it is key to the formation of wide black hole binaries. Using numerical simulations of Population~III (Pop~III) star formation, we study the angular momentum of Pop~III binaries and the torques between stars and gas discs to understand the origin of outward migration and high ellipticity. The outward migration of protostars is produced by gravitational torques exerted on them by their circumstellar minidiscs. The minidiscs, on the other hand, migrate outward mainly by gaining angular momentum by accreting gas from the circumbinary disc. The angular momentum transfer is most efficient for rapidly accreting equal-mass binaries, and weaker when the secondary mass is small or the massive companion evaporates the gas disc via radiative feedback. We conclude that outward migration and the formation of wide equal-mass massive binaries is common in metal-free/metal-poor star formation, mainly driven by their large accretion rates. We expect that the lower gas temperature and accretion rates in metal-enriched circumstellar discs would lead more often to inward migration and closer binary separations. We also observe inward migration for smaller mass Pop~III protostars/fragments, leading to the rapid merging of sink particles and likely the formation of close binary black holes that, however, reach separations below the resolution of our simulations. We discuss the implications that Pop~III separations and ellipticity may have on the interpretation that gravitational wave signals from merging intermediate-mass black holes come from Pop~III remnants.
\end{abstract}

% Select between one and six entries from the list of approved keywords.
% Don't make up new ones.
\begin{keywords}
gravitational waves -- binaries: general -- stars: formation -- stars: kinematics and dynamics -- stars: Population III -- dark ages, reionization, first stars
\end{keywords}

%%%%%%%%%%%%%%%%%%%%%%%%%%%%%%%%%%%%%%%%%%%%%%%%%%
%%%%%%%%%%%%%%%%% BODY OF PAPER %%%%%%%%%%%%%%%%%%




%%%%%%%%%%%%%%%%%%%%%%%%%%%%%%%%%%%%%%%%%%%%%%%%%%
%%%%%%%%%%%% SECTION 1 - INTRODUCTION %%%%%%%%%%%%
%%%%%%%%%%%%%%%%%%%%%%%%%%%%%%%%%%%%%%%%%%%%%%%%%%

\section{Introduction}

In recent years understanding the formation and properties of metal-free first stars (Population~III or Pop~III stars) has gained renewed impetus thanks to the discovery of gravitational wave (GW) emission from merging intermediate-mass black holes (IMBHs) and the successful launch of the \textit{James Webb Space Telescope} (\jwst). The detection by VIRGO and LIGO of signals from merging black hole binaries (BHBs) with individual masses $\sim 30~\msun$ \citep{abbott2016,abbott2017} indicate that they might be the remnants of metal-poor or metal-free star formation \citep{hartwig2016,liu2020}. The \textit{Hubble Space Telescope} and the \jwst have detected a few high-$z$ lensed stellar objects at $z\sim 5-6$ \citep[e.g.,][]{welch_highly_2022, welch_jwst_2022}, thereby suggesting detection of Pop~III stars is possible at high magnification. The search of Pop~III stars with \jwst is actively ongoing and strongly-lensed extremely metal-poor small mass star clusters ($10^4$~M$_\odot$ with $Z \sim 10^{-3}$~Z$_\odot$, possibly with a top-heavy IMF) at $z\sim 6$ have already been observed \citep{Vanzella:2023}. Remarkably, compact objects with similar properties have been predicted to exist in simulations of the first galaxies \citep{garcia2023}. Finally, massive Pop~III stars may explode as pair-instability supernovae \citep{heger2002}, and these extreme events may be detected by the \textit{JWST} and in the near future by the \textit{Nancy Grace Roman Space Telescope} \citep{whalen2014}. Given this context, understanding the formation and properties of Pop~III stars and more generally the fragmentation and evolution of metal-poor gas clouds and circumstellar discs is especially timely and well-motivated.

Among the many properties of Pop~III stars, their multiplicity is still poorly understood, despite its importance. There is a general consensus that the fragmentation of a metal-free gas disc and the subsequent formation of Pop~III binaries is a common occurrence (\citealp{machida2008}; \citealp*{stacy2010}; \citealp{clark2011,susa2013}; \citealp*{susa2014}; \citealp{hosokawa2016,sugimura2020}). Once they form, these binaries can migrate either inward or outward. Inward migration may lead to the formation of close BHBs or to mergers between Pop~III protostars. In the latter case, this enhances the growth of the primary star \citep{greif2012,hosokawa2016} and reduces the multiplicity of the Pop~III star system, thereby reshaping the Pop~III initial mass function (IMF). Inward migrations appear commonly in several numerical simulations \citep{greif2012,stacy2013,hirano2017,chon2019} but outward migrations have also been observed \citep{greif2011,greif2012,stacy2013,chon2019,sugimura2020}. 

A series of papers \citep[][hereafter, Papar~I, II, and III, respectively]{park2021a,park2021b,park2023} found that the formation of wide hierarchical binaries via outward migration, often with relatively large eccentricities, is the most typical outcome for massive Pop~III stars (semi-major axis, $a \sim$ a few 1,000~AU to 10,000~AU). This is the fourth paper in the series focusing on understanding the physical mechanisms driving outward migration and exploring the origin of high eccentricities.
In Paper~III, we underscored their importance in forming wide Pop~III BHBs because even these may lead to BH mergers and GW emission through two different dynamical channels: dynamical hardening \citep{liu2020} or orbital excitation \citep{michaely2019,michaely2020}. 

Due to its ubiquity and importance, there have been several previous studies aimed at understanding the origin of outward migration in circumbinary discs. Several authors studied ejections via N-body processes \citep{greif2011,greif2012,stacy2013} or gas accretion \citep[][CH19 hereafter]{chon2019}. This latter study focused on the evolution at small scales (tens of AUs) while Pop~III binaries typically reach separations up to 10,000~AU over a timescale of 50-100~kyr (\citealp{sugimura2020}; Paper~II; Paper~III). These previous works suggested the accretion of high angular momentum gas as the dominant mechanism for outward migration. In Paper~II, although we lacked a quantitative analysis, we discussed the possible connection between outward migration and gas accretion. The migration rate and separation of wide binaries were found to depend on the intensity of an external X-ray background in the following way. An X-ray background enhances the \hm\ cooling and thus lowers the gas temperature. The reduced sound speed $c_{s}$ leads to a lower accretion rate due to its $c_{s}^3$ dependence. In an X-ray, therefore, protostars grow more slowly and accrete less high-angular momentum gas, and thus stars have lower masses and the extent of their outward migration is reduced. For this reason, the maximum separation of wide binaries tends to be smaller in an intense X-ray radiation background (see Fig.~8 of Paper~II). This correlation, however, was less evident in Paper~III which differs from our previous work in that it included a full treatment of radiative feedback from accreting protostars. 

In this work, we analyse the simulations presented in Paper~III to study in detail the physical mechanism that induces outward migrations and ultimately produces the large separations of hierarchical Pop~III binaries. As discussed above, previous works on stars and black hole binaries had some disagreement on whether the main mechanism for migration is gravitational torques or the accretion of high angular momentum gas. We aim at clarifying this issue and analyse the role of radiative feedback that is likely responsible for the somewhat different results between Paper~II and Paper~III regarding migration. The ultimate goal is to determine what are the unique conditions in Pop~III protostellar discs that lead to migration and eccentric orbits, and why this behaviour is not observed in simulations of binaries formed in protostellar discs with solar metallicity \citep{he2023}.

This paper is organised as follows. In Section~\ref{sec:method} we summarise the method and simulation focusing on the features relevant to this work. In Section~\ref{sec:result1}, we explore the origin of outward migration. In Section~\ref{sec:result2}, we discuss binaries without outward migration to better understand the outward cases. In Section~\ref{sec:discussion} and \ref{sec:summary}, we discuss the implication of our findings and summarise the main conclusions.

%%%%%%%%%%%%%%%%%%%%%%%%%%%%%%%%%%%%%%%%%%%%%%%%%%
%%%%%%%%%%%%% SECTION 2 - SIMULATION %%%%%%%%%%%%%
%%%%%%%%%%%%%%%%%%%%%%%%%%%%%%%%%%%%%%%%%%%%%%%%%%

\section{Method}
\label{sec:method}

\subsection{Simulations}
\label{sec:sim}
\subsubsection{Overview}
We make use of Adaptive Mesh Refinement (AMR) code \ramsesrt\ \citep{teyssier2002,rosdahl2013} to simulate the formation of Pop~III stars in metal-free gas discs extracted from minihaloes in cosmological simulations. The non-viscous gas motion is described by solving the Euler equations using a second-order Godunov method and an approximate solution of the radiative transfer equations for UV radiation emitted from massive protostars is included and coupled to primordial chemistry (\citealp{rosdahl2013}; Paper~I; Paper~III). In addition to radiative feedback from protostars, we include external radiation backgrounds in the Lyman-Werner (LW) bands and X-rays. Time-dependent primordial gas chemistry and relevant cooling/heating processes for gas densities up to $10^{14}$~cm$^{-3}$ are included. In this paper we analysed the same simulations run in Paper~III, hence we refer to that paper for details on the simulation methods and convergence tests (see Appendix in Paper~III). Here, for completeness, we only summarise some basic information on the simulations and the parameters of different runs.

We extract the central regions ($2$~pc) at the centre of two minihaloes in cosmological zoom-in simulations (see Fig.~2 of Paper~III) to create the initial conditions. These initial conditions include a metal-free gas cloud that has been irradiated by various X-ray and LW backgrounds. We performed 8 simulations and identified 14 binaries in them. Table~\ref{tab:sim} summarises the set of simulations and some properties of the binaries. Once a simulation begins, the gas cloud with peak hydrogen number density $\nh \sim 10^7$~\hcc\ contracts and flattens out to form a disc. As the cloud contracts and the central density increases, cells are refined from AMR level 7 to 15 with the Jeans refinement criterion $\lambda_{\subj} < N_{\subj} \Delta x$. Here, $\lambda_{\subj}$ is the local Jeans length in each gas cell, $\Delta x$ is the cell size, and $N_{\subj} = 16$ is the number of cells into which we resolve the Jeans length. The smallest cell size or the maximum spatial resolution is $\Delta x_{min} = 2~{\rm pc}/2^{15} = 12.6$~AU.
\begin{table*}
    \caption{Summary of simulations. From left to light, we show 1) binary name, 2) formation time, 3) final time, 4) separation at formation, 5) final separation, 6) eccentricity, 7) mass ratio, 8-9) mass of individual stars, and 10) type of migration.}
    \footnotesize
    \begin{threeparttable}
    	\centering
    	\label{tab:sim}
    	\begin{tabular}{ | l | c | c | c | c | c | c | c | c | c | }
		    \hline
            Label\tnote{a,b} & $t_{form}$ [kyr] & $t_{final}$ [kyr] \tnote{c} & $d_{form}$ [AU] & $d_{final}$ [AU] & $e$\tnote{d} & $q$\tnote{e} & $M_1 (\msun)$\tnote{f} & $M_2 (\msun)\tnote{f}$ & Migration\tnote{g} \\
		    \hline

            Run~A & & & & & & & \\
            S01-S02\tnote{$\star$} & $0.38$ & $68$ & $727$ & $7637$ & $0.79$ & $0.847$ & $183$ & $215$ & neutral \\
            \hline

            Run~B & & & & & & & \\
            S01-S04 & $0.30$ & $42$ & $1528$ & $1594$ & $0.46$ & $0.971$ & $88$ & $85$ & neutral \\
            \hline

            Run~C & & & & & & &\\
            S01-S02 & $0.22$ & $92$ & $584$ & $4927$ & $0.22$ & $0.943$ & $247$ & $249$ & outward \\
            - S01-S06 & $29$ & $92$ & $1544$ & $705$ & $0.63$ & $0.205$ & $247$ & $51$ & inward \\
            - S02-S12\tnote{\dag} & $51$ & $92$ & $1149$ & $484$ & $0.33$ & $0.132$ & $249$ & $33$ & inward \\
            \hline

            Run~D & & & & & & &\\
            S01-S02 & $0.31$ & $67$ & $492$ & $2910$ & $0.19$ & $0.901$ & $74$ & $82$ & outward \\
            \hline

            Run~E & & & & & & &\\
            S01-S02 & $0.30$ & $103$ & $498$ & $7623$ & $0.28$ & $0.971$ & $49$ & $51$ & outward \\
            \hline

            Run~F & & & & & & &\\
            S01-S03 & $0.83$ & $108$ & $664$ & $40982$ & $0.16$ & $0.800$ & $83$ & $67$ & outward \\
            - S01-S10\tnote{$\ast$} & $25$ & $108$ & $519$ & $5244$ & $0.34$ & $0.870$ & $83$ & $71$ & outward \\
            - S03-S05 & $15$ & $108$ & $426$ & $4240$ & $0.32$ & $0.840$ & $67$ & $56$ & outward \\
            \hline

            Run~G & & & & & & &\\
            S01-S03 & $1.6$ & $95$ & $1537$ & $43310$ & $0.27$ & $0.505$ & $73$ & $41$ & outward \\
            - S01-S02 & $0.31$ & $95$ & $558$ & $4846$ & $0.40$ & $0.971$ & $73$ & $75$ & outward \\
            - S03-S04 & $21$ & $95$ & $520$ & $6711$ & $0.61$ & $0.833$ & $41$ & $34$ & outward \\
            \hline

            Run~H & & & & & & &\\
            S01-S03 & $0.33$ & $33$ & $1799$ & $4405$ & $0.53$ & $0.775$ & $89$ & $114$ & outward \\
            \hline
	    \end{tabular}
	    \begin{tablenotes}
	        \item[a] Following the labels in Fig.~1 in Paper~III.
            \item[b] A hyphen before the name indicates the binary belongs to a hierarchical binary.
            \item[c] Final time of the simulation.
            \item[d] Maximum eccentricity.
            \item[e] $q=M_2/M_1$ if $M_1 \geq M_2$; $q=M_1/M_2$ if $M_1 < M_2$.
            \item[f] At $t_{final}$.
            \item[g] Outward if $d_{form} < d_{final}$; inward if $d_{form} > d_{final}$; neutral if $d_{form} \sim d_{final}$. An exception is Run~A where the orbit is kept highly eccentric.
            \item[$\star$] Representative case of non-migration binaries. Throughout the paper, it is referred to as ``\non''.
            \item[\dag] Representative case of inward migration binaries. Throughout the paper, it is referred to as ``\inward''.
            \item[$\ast$] Representative case of outward migration binaries. Throughout the paper, it is referred to as ``\outward''.
        \end{tablenotes}
    \end{threeparttable}
\end{table*}

\subsubsection{Sink prescription}
The flattened disc fragments and multiple blobs form in the disc or spiral structures. Using the built-in halo finder \citep{bleuler2015} we identify these dense blobs at every coarse timestep ($\sim 13$~yr) and assign a sink particle at the density peak of each blob if,
\begin{enumerate}
    \item $\nh$ at the peak exceeds $n_{sink} = 10^{12}$~\hcc,

    \item there is no other sink particle within $2r_{sink}$,

    \item the velocity field in the blob has a net negative divergence.
\end{enumerate}
A sink particle has the radius of $r_{sink} = 8\Delta x_{min} = 101$~AU. This resolution of sink is insufficient to resolve a close binary and therefore a single sink particle may represent multiple Pop~III stars. In this work, however, we assume that each sink particle describes a Pop~III star and use the terms `sink particle' and `star' interchangeably. If two sink particles are closer than $2r_{sink} = 202$~AU, they merge into one. The new sink particle is positioned at the centre-of-mass (CoM) of the two and momentum is conserved. The acceleration of a sink particle by the gas or other sinks is smoothed by the smoothing length $\varepsilon = 4\Delta x_{min} = 50.4$~AU. The accretion of gas onto the sinks is performed by checking the gas density of each cell within $r_{sink}$. If a hydrogen number density $n_{H}$ exceeds the threshold value, the excess mass $(n-n_{sink})\Delta x^3$ is added to the sink and subtracted from the cell. If a sink particle mass exceeds $1~\msun$, it emits UV radiation. The luminosity is determined by the sink mass and accretion rate and is computed by interpolating the tabulated model of massive protostar by \citet{hosokawa2009} and \citet*{hosokawa2010}. We record the masses, positions, and velocities of the existing sink particles at every coarse timestep ($\sim 13$~yr).

\subsection{Torque}
\label{sec:torque}
To study the migration of stars we estimate their orbital angular momentum. The total orbital angular momentum of a binary is,
\begin{equation}
    \boldsymbol{J} = \sum_{\subi=1}^{2} \boldsymbol{J}_{\subi} = \sum_{\subi=1}^{2} m_{\subi} \boldsymbol{r}_{\subi} \times \boldsymbol{v}_{\subi},
\end{equation}
where the subscript $i$ indicates the index of a sink in the binary, $\boldsymbol{J}_{\subi}$ is each angular momentum, $m_{\subi}$ is the mass, $\boldsymbol{r}_{\subi}$ is its position from the CoM of the binary, and $\boldsymbol{v}_{\subi}$ is the velocity relative to the CoM. The rate of change of $\boldsymbol{J}$ is,
\begin{equation}
    \label{eq:djdt}
    \frac{\mathrm{d} \boldsymbol{J}}{\mathrm{d}t} = \sum_{\subi=1}^{2} \left( \frac{\mathrm{d}m_{\subi}}{\mathrm{d}t} \boldsymbol{r}_{\subi} \times \boldsymbol{v}_{\subi} + m_{\subi} \frac{\mathrm{d} \boldsymbol{r}_{\subi}}{\mathrm{d}t} \times \boldsymbol{v}_{\subi} + m_{\subi} \boldsymbol{r}_{\subi} \times \frac{\mathrm{d}\boldsymbol{v}_{\subi}}{\mathrm{d}t} \right)
\end{equation}
% Figure environment removed

The first term on the right-hand side of the equation describes the change in the angular momentum due to the mass growth of the binary. The change of $J$ caused by the third term is due to external torques exerted on the binary. When gas is accreted smoothly, the second term is zero since $d\boldsymbol{r}_{\subi}/dt \times \boldsymbol{v}_{\subi} = \boldsymbol{v}_{\subi} \times \boldsymbol{v}_{\subi}=0$. This does not hold, however, when a merger with a third body offsets the position of Sink $i$. In this case, the second term is treated as a merger torque (see equation~(\ref{eq:torque_merger})). We can arrange the various terms contributing to the angular momentum change as follows:
\begin{equation}
    \label{eq:total_torque}
    \frac{\mathrm{d} \boldsymbol{J}}{\mathrm{d}t} - \left( \sum_{\subi=1}^{2} \frac{\mathrm{d}m_{\subi}}{\mathrm{d}t} \boldsymbol{r}_{\subi} \times \boldsymbol{v}_{\subi} \right) = \boldsymbol{\tau}_{total} = \boldsymbol{\tau}_{gas} + \boldsymbol{\tau}_{sink} + \boldsymbol{\tau}_{merge} + \boldsymbol{\tau}_{acc}.
\end{equation}
Here, $\boldsymbol{\tau}_{gas}$ is the gravitational torque exerted by the gas,
\begin{equation}
    \begin{split}
        \boldsymbol{\tau}_{gas} &= \sum_{\subi=1}^{2}  \boldsymbol{r}_{\subi} \times \boldsymbol{F}_{gas} \\
        &= \sum_{\subi=1}^{2}  \boldsymbol{r}_{\subi} \times \left( \sum_{\subj=cell} \frac{G m_{\subi} m_{\subj}}{(s^2+\varepsilon^2)^{3/2}} \boldsymbol{s}_{\subj} \right),
    \end{split}
    \label{eq:tau_gas}
\end{equation}
where $\boldsymbol{F}$ is the gravitational force on Sink $i$, $G$ is the gravitational constant, $m_{\subj}$ is the mass of gas cell $j$, $\boldsymbol{s}_{\subj}$ is the displacement vector from the sink to the cell, and $s=||\boldsymbol{s}_{\subj}||$ is the distance between them. The gravitational torque exerted by other sinks is calculated similarly,
\begin{equation}
    \begin{split}
        \boldsymbol{\tau}_{sink} &= \sum_{\subi=1}^{2} \boldsymbol{r}_{\subi} \times \boldsymbol{F}_{sink} \\
        &= \sum_{\subi=1}^{2} \boldsymbol{r}_{\subi} \times \left( \sum_{\substack{ {j = \rm sink} \\ j \neq i }} \frac{G m_{\subi} m_{\subj}}{(s^2+\varepsilon^2)^{3/2}} \boldsymbol{s}_{\subj} \right),
    \end{split}
    \label{eq:tau_star}
\end{equation}
where we sum the gravitational torques exerted on Sink $i$ from all the other sinks in the simulation. $\boldsymbol{\tau}_{merge}$ is the change in angular momentum due to mergers. If a third body merges with Sink $i$, then it has a new position and velocity. The rate of change is,
\begin{equation}
    \boldsymbol{\tau}_{merge} = \frac{1}{\Delta t} \sum_{\subi=1}^{2} m_{\subi,old} \left( \boldsymbol{r}_{\subi,new} \times \boldsymbol{v}_{\subi,new} - \boldsymbol{r}_{\subi,old} \times \boldsymbol{v}_{\subi,old} \right),
    \label{eq:torque_merger}
\end{equation}
where the subscripts "new" and "old" refer to the position and velocity vectors with respect to the CoM of Sink $i$ after and before the merger with a third body, respectively. Finally, we calculate the torque due to the accretion of gas onto the stars 
following \citet*{tang2017}.
\begin{equation}
    \boldsymbol{\tau}_{acc} = \sum_{\subi=1}^{2} \sum_{\substack{\rm \textit{j}=cell \\ s \leq r_{sink} \\ \Delta m_{\subj}>0}} \frac{\Delta m_{\subj}}{\Delta t} \boldsymbol{r}_{\subi} \times (\boldsymbol{v}_{\subj} - \boldsymbol{v}_{\subi}).
    \label{eq:tau_acc}
\end{equation}
Note that this term plus $(\mathrm{d}m_{\subi}/\mathrm{d}t) \boldsymbol{r}_{\subi} \times \boldsymbol{v}_{\subi}$ (left-hand side of equation~(\ref{eq:total_torque})) is equal, neglecting numerical errors, to the rate of angular momentum accretion onto the sinks. The latter, however, does not directly contribute to the change in the orbit, and thus only the former is treated as torque in equation~(\ref{eq:tau_acc}). When a gas cell inside the sink radius and its number density exceeds the sink density threshold ($n \geq 10^{12}$~\cc), the linear momentum of the extra mass $\left[\frac{\Delta m_{\subj}}{\Delta t}(\boldsymbol{v}_{\subj}-\boldsymbol{v}_{\subi})\right]$ is dumped to the central sink particle where $\Delta t$ is the time-step. The time evolution of the angular momentum due to each of the torques is calculated by integrating in time each torque,
\begin{equation}
    \label{eq:int_tau}
    \boldsymbol{J}_{\subi} = \int \boldsymbol{\tau}_{\subi} \mathrm{d}t \approx \sum \boldsymbol{\tau}_{\subi} \mathrm{d}t,
\end{equation}
where the subscript $i$ denotes tot, acc, merge, star, or gas. We also consider the specific angular momentum of the binary,
\begin{equation}
    \label{eq:js}
    \boldsymbol{j} = \sum_{\subi=1}^{2} \frac{\boldsymbol{J}_{\subi}}{m_{\subi}} = \sum_{\subi=1}^{2} ~ \boldsymbol{j}_{\subi} = \sum_{\subi=1}^{2} \boldsymbol{r}_{\subi} \times \boldsymbol{v}_{\subi}.
\end{equation}
Note that $\boldsymbol{J}/M =(m_1 \boldsymbol{j}_1+m_1 \boldsymbol{j}_2)/M \not= \boldsymbol{j}$, where we have defined the total mass $M\equiv m_1 + m_2$.
% Figure environment removed
%%%%%%%%%%%%%%%%%%%%%%%%%%%%%%%%%%%%%%%%%%%%%%%%%%
%%%% SECTION 3 - Origin of outward migration %%%%%
%%%%%%%%%%%%%%%%%%%%%%%%%%%%%%%%%%%%%%%%%%%%%%%%%%

\section{Origin of outward migration}
\label{sec:result1}

\subsection{Typical formation scenario}
\label{sec:general}
In this subsection, we briefly summarise the general scenario of Pop~III star formation in a primordial disc according to the results of our previous papers in the series. For a more detailed picture, see Section 3.1 and Fig.~4 of Paper~III. A gas cloud with a nearly isothermal profile and peak density $\nh \sim 10^7$~\hcc\ contracts and flattens out to form a protostellar disc due to the angular momentum conservation. This gas disc becomes gravitationally unstable and fragments \citep*{kimura2021}, and multiple sink particles form. Some of the sinks migrate inward in a few kyr and merge with others while the others survive and migrate outward. The survivors possess a circumstellar minidisc and these minidiscs may fragment to form multiple stars. Some stars migrate inward while others migrate outward repeating the initial fragmentation time evolution but on the smaller scale of the circumstellar disc (rather than the larger circumbinary disc). In this scenario, the most common outcome is a hierarchical binary (a binary of binaries, Fig.~\ref{fig:binary}), but dynamically unstable systems \citep[e.g., single-triple pair,][]{sugimura2020} may appear. Each of the binaries generally also migrates outward, but there are a few exceptions (Section~\ref{sec:result2}). At late times, in some systems stars can form at Lagrange points L4/L5 of the main binary. This scenario, however, depends on the intensity of the X-ray background that regulates the gas accretion rate onto the disc and stars via enhanced cooling. Strong X-ray irradiation typically lowers the multiplicity and masses of Pop~III stars (see Paper~II and Paper~III).

In Fig.~\ref{fig:orbit} we plot the trajectories of long-lived sink particles (shown as coloured circles at the time of formation and crosses when they merge) for the three runs in which our fiducial binary cases are found. The top panels show the star orbits in a system of reference centred on the CoM, while in the bottom panels the system of reference is centred on one of the binary stars. Sinks that are short-lived are not shown here: they migrate inward on a timescale of a few kyr before merging with other sinks. In general, the orbits of long-lived sinks are eccentric and expand with time. This can be clearly seen in the orbit of the \outward case (red and blue symbols in the bottom left panel). The top left panel also shows the outward migration of the individual binaries. The orbit of this binary expands faster at a later time due to the gravitational torque by S11 (purple) and the merger with it (purple cross). 

In the middle panels, we plot the trajectories of the stars in Run~C (top) and the \inward binary (bottom). With S02 (green) fixed at the centre of the frame of reference (bottom middle panel), its companion (S12, blue) forms at a distance of $\sim 1200$~AU and it migrates inward down to $\sim 400$~AU. Later, the size of the orbit remains nearly constant without further migrating or merging with S02. In the right panels, we show a case (\non: red and green) in which the orbit does not expand significantly but it is highly eccentric ($e \sim 0.8$). In this run at later times S09 (blue) merges with S02 while other stars (S08 and S14) are ejected via three-body interactions.
% Figure environment removed
% Figure environment removed
% Figure environment removed

\subsection{Source of angular momentum of sinks}
\label{sec:outward}
Conservation of angular momentum imposes that, in the absence of external torques and gas accretion from outside the system, the binary orbital parameters remain constant. The migration of the stars, therefore, is caused by the transfer of angular momentum between the binary and other parts of the system via torques and/or gas accretion (\citealp{tang2017,munoz2019,moody2019}; CH19; \citealp{tiede2020}). The first step to understanding the dominant physical mechanism causing migration, therefore, is to estimate the torques existing in the system. Note that we often loosely refer to the accretion of angular momentum as a torque, $\tau_{acc}$, as defined in equation~(\ref{eq:tau_acc}). 

Fig.~\ref{fig:torque_out} shows an example of such calculation for our prototype binary system showing outward migration (\outward). This system is the binary S01-S10 in the minidisc around star S01 in Run F (see also top left panel of Fig.~\ref{fig:orbit}). S10 forms at $t\sim 25$~kyr from the fragmentation of the circumstellar disc around S01 ($M \sim 50~\msun$). It has an initial separation from S01 of $\sim 500$~AU and a mass $q\sim 0.1$ times that of S01. The separation increases by a factor of 4 ($\sim$ 2,000~AU) before S10 merges with S11 at $t \sim 69$~kyr. After this merger, this initially unequal mass binary becomes of nearly equal mass (top panel) and the separation increases by a factor of two. Here, however, we focus on understanding the effect of other torques (i.e., accretion and gravitational) before the merger. 

To understand the main mechanism for the outward migration we compare the increase of angular momentum by separating the effect of various torques as a function of time (see bottom panel). Before the merger, the gravitational torque from the disc (light blue line) and accretion of angular momentum (purple line) are the two dominant sources of angular momentum (see bottom panel). However, they work in opposite directions: the gas accretion torque is negative, hence reducing the binary angular momentum and acting as a viscous or drag term. The gravitational torque is instead positive (producing outward migration) and it dominates over the negative torque. The net change in $\boldsymbol{J}$ from the sum of all torques (dashed line) is positive. When compared to the actual orbital angular momentum of the binary (black solid line) the agreement is quite good but not perfect because the calculations are somewhat uncertain due to the finite time steps of our outputs ($\sim 1-2$~kyr). The result shown in Fig.~\ref{fig:torque_out} is representative of most simulations in that the  gravitational torque is the dominant term and the accretion torques have a negative sign, somewhat reducing the outward migration effect. When stars are observed to migrate outward, both the solid and dashed lines are increasing nearly in all binaries we analysed, meaning the gas disc plays the dominant role in producing the binary migration via gravitational torque.

Now that we established that the outward migration of stars is caused by the gravitational torque by the gas disc, we need to understand why the sign is positive (i.e., the star is accelerated in the direction of its rotation by some gas overdensity in front of it) and which part of the gas disc contributes most to the torque (the spiral structure or the minidisc). Different answers to this question have been provided in various contexts in the literature. In the context of binary SMBHs, \citet{tang2017} argued that it is the gravitational torque by the minidiscs around sink particles (SMBHs) to drive inward migration. CH19 instead argues that disc spiral structures extract the angular momentum from the binary. In our simulations, as observed in CH19, a spiral structure often appears when newly formed sinks migrate inward to merge with the central one. Once this initial merger phase is over, the remaining stars have circumstellar minidiscs and spiral arms extending outward and connecting the minidiscs (see Fig~\ref{fig:binary}). For this reason, we hypothesise these non-axis-symmetric structures are the dominant contribution to the torque and outward migration of the binary. To test this hypothesis, we construct maps showing the contribution to the gravitational torque on the binary from each gas cell on the disc (shown face-on). To visualise these maps we choose a logarithmic polar grid around each sink particle and calculate the total torque on the binary by each cell on this grid. This approach imposes the cell size $r^2 \mathrm{d}\theta \mathrm{d}\ln{r}$ and thus eliminates $r^{-2}$ dependence of gravity which might lead to visually underestimating the effect of outer spiral arms despite their large extent and mass. The parameters of the logarithmic grid are provided in Table~\ref{tab:polar}. Note that the colour maps show the total torque in each cell, not torque density like in previous works \citep[e.g.,][]{tang2017}: in this work the torque density would also lead to visually underestimating the role of the outer structures because of the increasing cell size with radial distance. The contribution to the torque from different parts of the gas disc is visualised in the left panel of Fig.~\ref{fig:torque_map_sink} for \outward binary. Although the torque by the minidiscs is large, spiral arms also exert significant torques on the binary, although the sign of the torque varies between positive and negative values making it difficult to determine if the net torque is positive or negative. For quantitative analysis, we fold the torque map twice along the x-axis and vertical line crossing the midpoint of two sinks (dotted-dashed lines). In this way, torques in the entire domain are added to the 1st quadrant and it is easier to determine the relative importance of the positive and negative regions of the torque and their location. As can be seen in the middle panel, the torque is positive near the sink radius (solid line) and outside ($\sim 200$~pc, top left of the solid line). This region corresponds to the front side of the minidiscs and therefore this result implies that the sinks are accelerated in the directions of the binary orbital velocity. The azimuthally averaged radial profile, shown in the right panel, also shows that the gas just outside the sinks dominates the contribution to the net positive torques and therefore is the cause of the outward migration.

\begin{table}
    \caption{Parameters of the logarithmic polar grid.}
    \footnotesize
    \begin{threeparttable}
        \centering
        \label{tab:polar}
        \begin{tabular}{ | l | l | l | }
		\hline
            Parameter & Value & Description \\
		\hline
            $r_{min}$ & $10$~AU & inner boundary \\
            \hline
            $r_{max}$ & $7071$~AU & outer boundary \\
            \hline
            $N_{r}$ & $100$ & number of grids in the radial direction \\
            \hline
            $N_{\theta}$ & $60$ & number of grids in the azimuthal direction \\
            \hline
            ${\rm d}\ln{r}$ & $0.0656$ & radial resolution \\
            \hline
            ${\rm d}\theta$ & $0.105\pi$ & azimuthal resolution\\
            \hline

	\end{tabular}
    \end{threeparttable}
\end{table}

Fig.~\ref{fig:torque_map_sink} illustrates the importance of the minidiscs in driving the outward migration of the binary at a specific time. However, it does not show the overall effect integrated over time. As seen in Fig.~\ref{fig:orbit} and Fig.~\ref{fig:torque_out} the stars in the binary migrate continuously. This means that the net positive torque has to be kept during this period. This is visualised in Fig.~\ref{fig:torque_map_sink_total}, showing the same torque map integrated over time (i.e., showing the angular momentum, equation~(\ref{eq:int_tau})). In the figure, the bright yellow feature (top panel) and the cumulative torque (blue line, bottom panel) indicate the gas near the sink (minidiscs) exert net positive torque consistently. This result, therefore, implies the minidiscs play the most significant roles in producing the migration of the stars as found in \citet{tang2017}, although contrary to the BH binary case in their work, in our case the sign of the torque is positive and the migration is outward.

\subsection{Source of angular momentum of disc}
\label{sec:disc_out}
As mentioned in the previous subsection, the minidiscs are the dominant source of the gravitational torque on the stars. This means that they inevitably lose some of their angular momentum by dragging their central stars with them. For the circumstellar minidiscs and their central stars to keep migrating outward, therefore, there must be an external source of angular momentum acting on the minidiscs. An obvious candidate is the accretion of high angular momentum gas from the circumbinary envelope. This idea was first suggested in \citet{sugimura2020} and in Paper~II, without a quantitative analysis of this effect. However, Paper~II provided several hints that outward migration is connected to the gas accretion rate. The global rate of gas accretion from larger scales to the disc center ($\dot{M}$) is proportional to $c_{s}^3$, where $c_s$ is the gas sound speed. Even though it seems counter-intuitive, the sound speed is reduced if the gas is irradiated by a strong/moderate X-ray background because of the enhanced formation and cooling by \hm. The reduction in $\dot{M}$ slows down the accretion of high angular momentum gas onto the circumstellar discs and stars. We therefore found that, when irradiated by an X-ray background, Pop~III protostars tend to have smaller masses and the rate of outward migration is reduced, producing binaries with smaller separations (Fig.~8 of Paper~II). We concluded that the rate of gas accretion plays an important role in the angular momentum supply and outward migration of Pop~III binary stars.

To test the idea that minidiscs gain angular momentum by accreting high angular momentum gas from the outer parts of the disc, we estimate the external gas gravitational torque and angular momentum accretion rate of the discs. To carry out this calculation, we first define a cylinder with a radius $r_{cyl} = 300$~AU and height of $h_{cyl} = 600$~AU around each sink particle. We assume that this cylindrical boundary contains the minidiscs around each sink particle. The gravitational torque is calculated as explained in Section~\ref{sec:outward} for the case of the sinks, using equation~(\ref{eq:tau_gas}). In the equation, however, the subscript $i$ runs over all gas cells within the cylinder and $j$ runs over gas cells outside the cylinder. The change in angular momentum over time due to this effect is shown in light blue in Fig.~\ref{fig:accJ}. To calculate the accretion rate of angular momentum, we identify the cells near the cylinder walls ($r_{cyl}-\Delta x_{min}/2 \leq r < r_{cyl}+\Delta x_{min}/2$), where $r$ is the radial distance in the cylindrical coordinates. The radius of the cylinders is kept fixed for simplicity. The total accretion rate through the walls is calculated as follows:
\begin{equation}
    \frac{ \mathrm{d}J_{\textit{z}} }{\mathrm{d}t} = \sum_{\subi=1}^{2} \sum_{\subj=cell} \frac{J_{\subj,\textit{z}}}{V_{\subj}} v_{\subj,\textit{r}} \Delta x_{\subj}^2.
\end{equation}
The subscripts $i$ and $j$ indicate the sinks and gas cells, respectively. $V_{\subj}$ is the volume of cell $j$ and therefore $(J_{\subj,\textit{z}}/V)$ is its angular momentum density. $v_{\subj,\textit{r}}$ is the radial velocity in the cylindrical coordinate. As seen in Fig.~\ref{fig:accJ}, the change in angular momentum due to gas accretion (black) is dominant over the gravitational torque (light blue) on the disc. Other binaries with outward migration follow a similar trend with a more pronounced difference. Another result that can be noticed inspecting the accretion of angular momentum as a function of time in Fig.~\ref{fig:accJ}, is that the angular momentum accretion has periodic oscillations with peaks happening just after the stars reach the apocentre of the elliptical orbit. This aspect is more clearly seen in Fig.~\ref{fig:disc_sink}, showing that the disc angular momentum (red lines in panel a) peaks near the maximum binary separation (vertical dotted lines). This is because stars are further from the CoM and therefore they can accrete high angular momentum gas more easily. These accretion peaks are pronounced when the secondary star passes the outer spiral arm. At pericentres, on the other hand, the discs lose angular momentum to the sinks, but the net effect integrated over time is toward an increase of the angular momentum.

% Figure environment removed

% Figure environment removed


\subsection{Efficiency of angular momentum accretion}
\label{sec:eff}
For binaries to gain angular momentum and migrate outward by gas accretion, the specific angular momentum of the accreted gas must be higher than that of the stars. The difference between these two values, in addition to the mass accretion rate, may explain why some stars migrate outward efficiently while others do not. The answer to this question requires an understanding of the distribution of the angular momentum of gas and stars. 
The specific angular momentum of the stars in a binary is,
\begin{equation}
    \begin{split}
        \label{eq:jstar}
        j_{star} &= j_{star,1} + j_{star,2}=r_1^2 \omega_{kep} + r_2^2 \omega_{kep}  \\
        &= r_1^2 \sqrt{GM/r^3} + r_2^2\sqrt{GM/r^3} \\
        &= \left(\frac{m_2}{M}\right)^2 r^2\omega_{kep} + \left(\frac{m_1}{M}\right)^2 r^2\omega_{kep},
    \end{split}
\end{equation}
where $\omega_{kep} =2\pi/P=\sqrt{GM/r^3}$ is the angular velocity of the binary, with $r=r_1+r_2$, $r_1/r=m_2/M$ and $r_2/r=m_1/M$. Without loss of generality, we assume that $m_2 \le m_1$ (or $q\equiv m_2/m_1 \le 1$), hence the orbit of (Star 2) is further from the CoM of the system. The radial profile of the specific angular momentum of circumbinary disc outside the orbit of the binary stars is nearly Keplerian because the disc mass is typically negligible with respect to that of the Pop~III stars ($M_{sink}/M_{disc} \sim 4-5$) as seen in Panel c of Fig.~\ref{fig:disc_sink}. Hence, the specific angular momentum of the gas beyond the outer star orbit (i.e., at a distance $r_2$ from the CoM) is approximately,
\begin{equation}
    j_{gas, kep}(r_2) \approx r_2 v_{kep}(r_2) = \sqrt{GMr_2},
\end{equation}
where $v_{kep} = \sqrt{GM/r_2}$ is the Keplerian velocity. If we compare the specific angular momentum of the gas at the same distance of the outer star orbit ($r_2$) we have:
\begin{equation}
    j_{gas, kep}(r_2)/j_{star,2} \approx (r/r_2)^{3/2} = (M/m_1)^{3/2} = (1+q)^{3/2}\ge 1.
\end{equation}
This ratio approaches unity when the mass ratio of the binary is $q=m_2/m_1 \ll 1$ and reaches a maximum of $2^{3/2}$ for equal mass binaries (i.e., $q=1$). In this latter case, both stars in the binary contribute equally to the accretion of higher angular momentum gas. We assume the angular momentum vectors are perpendicular to the orbital plane and thus the quantities in the equations above are scalars. The actual distribution of angular momentum is shown in Fig.~\ref{fig:J}. The solid line, two dotted lines, and a dashed line in each panel indicate $j_{gas}$ (measured), $j_{star,1~or~2}$, and $j_{gas, kep}=\sqrt{GMr}$, respectively. As it can be seen in the figure, the actual measured values for the sinks (circles) agree with having Keplerian orbits (dotted lines). The figure illustrates how the specific angular momentum distribution varies with the mass ratio. When $q$ is small (top panel), the difference between $j_{gas}$ (solid line) and $j_{star,2}$ (dotted line) is relatively small. When the two stars have comparable masses (bottom panel), on the other hand, individual stars have smaller angular momentum (dotted lines below dashed and solid lines) and thus have a greater difference. Consequently, a binary obtains angular momentum more easily in an equal-mass binary.

% Figure environment removed

The gas motion deviates from the perfect Keplerian motion due to the presence of gas. This deviation is greatest right after the initial fragmentation and the formation of the first binary as shown in Fig.~\ref{fig:J_init} (Run~F at $t=1.3$~kyr). Since a larger amount of gas is available in the beginning, the gas angular momentum scales linearly rather than with $\sqrt{r}$ (Keplerian). Because the infalling gas has a higher specific angular momentum than the sinks, the stars that survive mergers migrate outward efficiently. 

This mechanism also explains why the mass ratio of the binaries often approaches $q=1$ for several of the binaries in this work (see Panel b of Fig.~\ref{fig:torque_out}) and in previously published works \citep{bate2000,munoz2020}.  When $q$ is low, the outer star intercepts most of the gas from the circumbinary disc and it grows faster than the primary star. By doing so, $q$ approaches unity and the binary can migrate outward more rapidly.

To summarise, in this section, we explore the origin of the outward migration of binary stars. The outward migration takes place in a two-step process. First, the minidiscs around stars gain angular momentum by accreting high angular momentum gas. Then the stars embedded in the circumstellar minidiscs gain angular momentum from them via gravitational torque and migrate outward. 

% Figure environment removed

\subsection{Origin of eccentric orbits}
Many Pop~III binaries have eccentric orbits. This is interesting because the eccentricity often gives rise to a periodic variability of Pop~III stars' luminosity and spectral type as discussed in Paper~III. In addition, the existence of highly eccentric orbits (e.g., \non with $e=0.8$) suggests the possible relevance of a new channel for GW emission: binary black hole mergers via dynamical excitation \citep{michaely2019,michaely2020}. These reasons motivate us to further explore the origin of eccentricity. Fig.~\ref{fig:ecc} shows the orbital eccentricity since the formation of each binary in the three migration groups (from top to bottom). We highlight the fiducial cases with thick solid lines. The binaries are generally born eccentric through 3-body or N-body interactions/mergers with other sink particles or gravitational torque by non-axisymmetric structures. Then, for the cases with outward migration or no migration, the eccentricities remain nearly constant in time. Note that outward migration happens continuously over a long period of time (see Fig.~\ref{fig:torque_out}). This implies that the relationship between torques, outward migration, and orbital (de-)excitation is insignificant. Interestingly, the periodic forcing from the accretion of gas of high-angular momentum at the apocentre (driving the outward migration) should circularise the orbit. However, we speculate that this does not happen because the periodic forcing happens only on the minidiscs while the forcing on the sinks by the gravitational torque is rather continuous over time (see Figure~\ref{fig:disc_sink}). 

Finally, the inward-migrating binaries (middle panels) show a behaviour more in line with what is observed in normal protostellar discs and/or protoplanetary discs: the orbits that are initially eccentric circularise while the stars migrate inward, due to the loss of angular momentum.
% Figure environment removed

% Figure environment removed

%%%%%%%%%%%%%%%%%%%%%%%%%%%%%%%%%%%%%%%%%%%%%%%%%%
%%%%%%%%%% SECTION 4 - Inward migration %%%%%%%%%%
%%%%%%%%%%%%%%%%%%%%%%%%%%%%%%%%%%%%%%%%%%%%%%%%%%

\section{Cases without outward migration}
\label{sec:result2}
Most binaries migrate outward in our simulations but there are a few exceptions (see runs labeled as "neutral" or "inward" in Table~\ref{tab:sim}). To better understand the conditions necessary for outward migration, we select some cases in which it does not happen and we make a comparative analysis. Fig.~\ref{fig:torque_in} shows the time evolution of the mass, mass ratio, separation, and angular momentum and torques for the case \inward, and can be compared directly, having the same format, to Fig.~\ref{fig:torque_out} for our \outward case. In this binary, similarly to the \outward case, the minidisc of the primary (S02) fragments to form the secondary star (S12). However, for this binary, the initial separation between the stars is initially about 2-3 times than for the \outward case ($1200$~AU) but decreases down to $400$~AU during the first $\sim 10$~kyr as S12 migrates inward (third panel). This inward migration is caused by the accretion (purple), gravitational torque from the gas disc (light blue line), and other stars in the system (red line, Panel~d). Similarly to the \outward case, the accretion torque is negative but, unlike in the outward case, the gravitational torque is also negative. During the inward migration, we notice that a prominent spiral structure develops and it disappears nearly when the migration stops. We speculate, therefore, it is this spiral structure that drives the inward migration as found in CH19.
In Section~\ref{sec:disc_out}, on the other hand, we argued that the fundamental reason for the outward migration is the accretion of high angular momentum gas by the circumstellar discs that migrate outward and drag the stars with them by gravitational torques. This implies that this latter process is weakened or suppressed in this binary. A possible explanation for this difference is the small mass ratio of the binary stars (Fig.~\ref{fig:torque_in}, Panel b). The mass ratio for this binary remains small ($q \lesssim 0.01$) for the first $\sim 10$~kyr when the inward migration takes place, also implying a lower gas accretion rate with respect to the \outward case. On the other hand, in the outward migration case in Fig.~\ref{fig:torque_out}, the secondary star grows quickly and retains mass comparable to the primary. This difference in mass ratio and accretion rate is critical to the direction of migration for the following reasons. First, as discussed in Section~\ref{sec:eff} and Fig.~\ref{fig:J}, the mass ratio of the stars determines the efficiency of the accretion of high angular momentum gas for a fixed mass accretion rate. With a small mass ratio, the secondary, which accretes a gas at the outer orbit, has nearly Keplerian angular momentum. In this case, the change in angular momentum through accretion is reduced and the binary cannot overcome the torque from the spiral structure. Second, if the mass ratio is large enough, the binary opens a gap more easily then the migration can turn into an outward one by accreting gas (CH19). On the other hand, if the secondary star fails to gain a large mass and the mass ratio remains small, it cannot open a gap to halt the inward migration. In conclusion, the mass ratio of binary plays a crucial role in the direction of migration because it affects the efficiency of angular momentum accretion and gap opening. Note that the binaries with inward migration have large initial separations and small mass ratios (Table~\ref{tab:sim}). The secondary star cannot open a gap and does not migrate outward and grow in mass. If a binary, on the other hand, forms at the centre of a barred spiral structure, stars have a small initial separation and typically mass ratios closer to unity, thereby they can more easily start the outward migration that is self-sustained by continuously accreting higher angular momentum gas from the circumbinary disc.

% Figure environment removed

Finally, we discuss another consequence of the mass ratio that is likely relevant: radiative feedback from the central massive stars. To motivate this effect, we briefly introduce one of the main results of Paper~II, which did not include radiative feedback from the protostars. According to simulation results in that paper, there is a correlation between the accretion rate, that is proportional to the gas temperature of the disc and regulated by the external irradiation from X-rays, and the maximum distance of the stars from the centre. This result suggests the high accretion rate causes the outward migration of the stars. In Paper~III, however, this trend was less clear. For instance, some binary stars in a strong X-ray background were found to expand with time to large distances ($d_{max} \sim $~8,000~AU, Run~E in Table~\ref{tab:sim}) while binaries in a moderate X-ray background (e.g., Run~B and Run~D) have smaller separations ($d \sim $~2,000~AU), even though the global accretion rate is lower in the former. Our interpretation is that this discrepancy between the results in Paper~II and Paper~III occurs because the radiative feedback from massive stars weakens the correlation between the accretion rate and outward migration found in Paper~II. In the feedback model used in Paper~III \citep{hosokawa2009,hosokawa2010}, the luminosity of a star is not a linear function of the mass and accretion rate. For this reason, the local accretion rate of the stars deviates from the global accretion trend regulated by an X-ray background. In addition, under strong protostellar feedback, the gas disc is evaporated and remains small compared to the stars (compare Panel c of Fig.~\ref{fig:disc_sink} and  Fig.~\ref{fig:disc_sink_inward}). The result is that the secondary star lacks a significant circumstellar minidisc which accretes high angular momentum gas and drags the star in the direction of the motion. With the lack of a minidisc, stars may not overcome the angular momentum loss by the inner spiral arms (CH19). The process is also visualised in Fig.~\ref{fig:snapshot}. In \inward case, the primary star (green circle) creates an intense UV radiation field (bottom panel) and the secondary (blue circle) lacks a prominent circumstellar minidisc. 

As seen in Table~\ref{tab:sim} we have four binaries without outward migration (inward and non-migrating). In these binaries, the primary stars are either born massive (\non) or the secondary star forms so late that the primary has enough time to grow in mass by accretion (\inward). Then the massive stars evaporate the gas discs and suppress gas accretion from the envelope. In this condition, it becomes harder for the small secondary to halt the inward migration and initiate the outward migration by accreting high angular momentum gas.

% Figure environment removed

%%%%%%%%%%%%%%%%%%%%%%%%%%%%%%%%%%%%%%%%%%%%%%%%%%
%%%%%%%%%%%%% SECTION 5 - DISCUSSION %%%%%%%%%%%%%
%%%%%%%%%%%%%%%%%%%%%%%%%%%%%%%%%%%%%%%%%%%%%%%%%%

\section{Discussion}
\label{sec:discussion}

We find that most massive Pop~III binary stars form at relatively close separations ($\sim 500$~AU) but migrate outward to form wide Pop~III binaries (5,000$-$10,000~AU). If these binaries survive with such orbits for the next $\sim 2$~Myr, when the stars explode, they become wide black hole binaries. Thanks to their wide separation, they have greater opportunities of being perturbed by field stars than close binaries. When perturbed, the orbits of the wide binaries are dynamically hardened \citep{liu2020}. Furthermore, if the orbits are eccentric (e.g., \non), the orbits may get excited and become more eccentric reducing their pericentre distance. If the distance at pericentre becomes sufficiently small, the emission of gravitational waves becomes important in removing angular momentum eventually leading to BHB merger and the emission of a strong GW signal within a Hubble time \citep{michaely2019,michaely2020}. 

The role of radiative feedback in migration has been studied by many authors. This has been considered insignificant when stars migrate inward because the migration time scale is shorter than the Kelvin-Helmholtz time scale after which stars become luminous in the UV \citep{stacy2010,inayoshi2014,latif2015,hirano2017}. Radiative feedback, however, may contribute to migration in particular situations as discussed in Section~\ref{sec:result2}. The main difference between this work and others is the fact that binaries can form at late times. If the UV photons from the primary star are blocked in the disc plane, the minidisc of the primary is not fully evaporated and the secondary can form. At the same time, the radiative feedback from the primary is strong enough to prevent the secondary from growing. If the secondary is kept small in size, the gap opening is delayed and the star is found in a smaller orbit. Furthermore, the secondary competing with the massive companion cannot accrete high angular momentum gas and thus fails to migrate outward. We note, however, that the result is sensitive to the feedback prescription \citep{jaura2022}. We artificially reduce the amount of radiation in the disc plane (See Section~2 of Paper~III). If the feedback is strong enough to evaporate the disc quickly ($\sim 10$~kyr), outward migration would occur less frequently due to the lack of high angular momentum gas.

An interesting discussion is a trend in migration across different metalicities. \citet{he2023} reported that metal-rich stars preferentially migrate inward. The difference from our work is that these systems are dominated by the central stars while the other stars have relatively low masses (that is, low mass ratio $q$). In this case, as mentioned in Section~\ref{sec:eff}, the angular momentum difference between the secondary star and the gas is reduced. Since the star accretes less angular momentum, the outward migration becomes less efficient. In our simulations, on the other hand, stars survive if they have masses comparable to the primary and migrate outward efficiently. 

Differences in gas metallicities may result in different behaviour of migration in the two populations of stars. A primordial gas cloud has a high temperature due to inefficient \hm\ cooling. Since the accretion rate is proportional to $c_{s}^3$, Pop~III star formation happens in a gas cloud with a high accretion rate. As seen in Paper~II, however, a disc in this environment is more Toomre-unstable and thus fragmentation is more active. Stars forming in this disc are likely to have similar masses and migrate outward in the end. On the contrary, the opposite happens in a metal-rich gas cloud. The accretion onto each prestellar core remains low and thus the disc is Toomre-stable and smooth as seen in \citet{he2023}. Once stars form, however, the evolution of the system is dominated by the massive central star and smaller secondary easily migrate inward as discussed in Section~\ref{sec:result2}. An interesting question is whether there is a critical metallicity above which stellar migration transitions from outward to inward migration. Since it is the accretion rate determining this transition, and the accretion rate is also responsible for the typical high-masses of Pop~III stars and their top-heavy IMF, it is likely that the critical metallicity is the same determining the transition from top-heavy to Salpeter IMF \citep[e.g.,][]{chon2021}.

%%%%%%%%%%%%%%%%%%%%%%%%%%%%%%%%%%%%%%%%%%%%%%%%%%
%%%%%%%%%%%%%% SECTION 6 - SUMMARY %%%%%%%%%%%%%%%
%%%%%%%%%%%%%%%%%%%%%%%%%%%%%%%%%%%%%%%%%%%%%%%%%%

\section{Summary}
\label{sec:summary}
We exploit radiative hydrodynamics simulations of star formation at the centre of metal-free minihaloes to explore the mechanisms leading to the outward migration of Pop~III protostar binaries. The initial conditions are created by extracting the central regions of two minihaloes irradiated by LW/X-ray backgrounds of various intensities as described in Paper~III. In each simulation, the typical outcome is the formation of hierarchical binaries or more generally multiple-star systems with a top-heavy IMF. The orbits of the binaries are typically elliptical and have a wide separation that increases with time, i.e., outward migration is a common occurrence. We investigate the time variation of the orbital angular momentum of the stars and the different torques and/or accretion of angular momentum leading to its increase. Below we summarise the key findings:
\begin{enumerate}
    \item Multiple stars form out of disc fragmentation. Some of them migrate inward and merge with the primary star on relatively short timescales, often producing an initial ellipticity significantly larger than zero of the stellar orbits of surviving stars. Of the stars that survive without merging for a longer time, however, most of them form binaries that migrate outward (for 10 out of 14 binary systems) producing wide binaries with separations up to 5,000$-$10,000~AU and elliptical orbits. The circumstellar minidiscs around each of the stars in this wide binary often fragment forming a quadruple hierarchical system in which even the binary stars in the minidiscs migrate outward.  We conclude that outward migration and the formation of wide Pop~III binaries with elliptical orbits is the most common outcome for Pop~III stars.

    \item Pop~III protostars obtain orbital angular momentum from the gravitational torque of their circumstellar minidiscs. These minidiscs, lose angular momentum to the stars but gain angular momentum by accreting high angular momentum gas from the circumbinary envelope. The accretion of angular momentum per unit mass is most effective in an equal-mass binary because their orbital velocity is significantly lower than that of the Keplerian disc at the same radial distance from the CoM. Then the angular momentum in the minidisc is transferred to the stars by gravitational torque and therefore the binary expands with time. On the other hand, inward migration happens in a binary that forms with a wider separation and with a small mass ratio ($q=m_2/m_1 \lesssim 0.1$), because in this case, the angular momentum accretion is less effective.

    \item Outward migration happens with the following periodic cycle. At the apocentre, the minidiscs grow in mass and migrate outward by accreting high angular momentum gas from the circumbinary disc. At pericentre, the accreted gas stored in the minidisc is accreted onto the stars. This cycle repeats until the density of the circumstellar minidiscs are low and the accretion rate of high angular momentum gas becomes negligible (at separations of 5,000$-$10,000~AU).
    
    \item Unlike previous studies, we find radiative feedback from protostars may affect the migration of stars in certain conditions. When a binary forms via late-time fragmentation, the massive primary star blows out the gas in the disc, suppressing the accretion rate on the secondary star and preventing it from gaining angular momentum by gas accretion. In these cases, the gravitational torque from spiral structures produces a net negative torque and therefore the binary star spirals inward.

    \item The lack of efficient cooling in a gas of primordial composition is the ultimate reason for the top-heavy IMF of Pop~III stars and their outward migration, which is not observed as frequently in Pop~II star formation \citet{he2023}. As the accretion rate is proportional to $c_{s}^3$, metal-free discs are relatively massive and Toomre-unstable, fragmenting rapidly. In these discs nearly equal-mass Pop~III binaries easily form in barred spiral structures at the centre of the disc, producing a configuration where disc gas with higher angular momentum than the stars is accreted most effectively. Pop~II star formation, on the other hand, is dominated by the central massive star and a disc with significantly less mass and Toomre-stable. According to \citet{he2023} the fragments in the disc are pre-existing and simply accreted into the disc from outside. In this case, the angular momentum accretion is less effective and therefore Pop~II stars tend to migrate inward and often get ejected by 3-body interactions.

    \item Binaries are typically born eccentric through 3-body encounters/mergers or non-axisymmetric potential. For outward migrating or non-migrating binaries the orbital eccentricity remains constant unless perturbed by mergers, while the orbits circularize for the fewer inward migrating cases. This implies that the periodic forcing at apocentre from torques producing outward migration do not excite or circularise the orbit. We speculate this is because the periodic force is applied onto the circumstellar discs while gravitational torque from the mini-disc to the star remains relatively constant over an orbital period.
\end{enumerate}

\section*{Acknowledgements}
The authors thank Dr. Takashi Hosokawa for kindly sharing the radiative feedback model of protostars. We make use of Deepthought2 and Zaratan operated by the University of Maryland (http://hpcc.umd.edu) to perform and analyse the simulations in this work.

%%%%%%%%%%%%%%%%%%%%%%%%%%%%%%%%%%%%%%%%%%%%%%%%%%
\section*{Data Availability}

The data underlying this article will be shared on reasonable request to the corresponding author.




%%%%%%%%%%%%%%%%%%%% REFERENCES %%%%%%%%%%%%%%%%%%

% The best way to enter references is to use BibTeX:

\bibliographystyle{mnras}
\bibliography{reference} % if your bibtex file is called example.bib

%%%%%%%%%%%%%%%%%%%%%%%%%%%%%%%%%%%%%%%%%%%%%%%%%%

%%%%%%%%%%%%%%%%% APPENDICES %%%%%%%%%%%%%%%%%%%%%

%\appendix

%\section{SOME POSSIBLE SECTIONS}

%%%%%%%%%%%%%%%%%%%%%%%%%%%%%%%%%%%%%%%%%%%%%%%%%%


% Don't change these lines
\bsp	% typesetting comment
\label{lastpage}
\end{document}

% End of mnras_template.tex
