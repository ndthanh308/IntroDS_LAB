\documentclass{article}


% if you need to pass options to natbib, use, e.g.:
%     \PassOptionsToPackage{numbers, compress}{natbib}
% before loading neurips_2023


% ready for submission
%\usepackage{neurips_2023}

\input{arxiv_style}
\usepackage{amsmath}
\usepackage{amssymb}
\usepackage{amsthm}
\usepackage{mathtools}
\usepackage{comment}
\usepackage{todonotes}
\usepackage{float}
\usepackage[algo2e,ruled,linesnumbered]{algorithm2e}

% for restatable
\usepackage{thmtools,thm-restate}
\usepackage{physics}
% %% >> for restatable links
% \usepackage{xpatch}
% \usepackage{xcolor}
% \usepackage{scalerel}

% % a flag to turn on and off
% \newif\ifmarginprooflinks
%     \marginprooflinkstrue
%     % \marginprooflinksfalse


% %% STEP 1: patch restatable so there are backward links on recall
% \makeatletter
% \xpatchcmd{\thmt@restatable}% Edit \thmt@restatable
%    {\csname #2\@xa\endcsname\ifx\@nx#1\@nx\else[{#1}]\fi}% Replace this code
%    {\ifthmt@thisistheone%
%     \csname #2\@xa\endcsname\ifx\@nx#1\@nx\else[{#1}]\fi% same as before
%     %except with also marginparbox
%    \else\fi} {}{\typeout{FIRST PATCH TO THM RESTATE FAILED}}
% \xpatchcmd{\thmt@restatable}% A second edit to \thmt@restatable
%    {\csname end#2\endcsname}
%    {\ifthmt@thisistheone\csname end#2\endcsname\else\fi}
%    {}{\typeout{FAILED SECOND THMT RESTATE PATCH}}


% \newcommand{\recall}[1]{\medskip\par\noindent{\bf \Cref{thmt@@#1}.} \begingroup\em \noindent
%    \expandafter\csname#1\endcsname* \endgroup\par\smallskip}

% %% STEP 2: make forward links to restatable.
% \setlength\marginparwidth{1.55cm}
% \let\oldmarginpar\marginpar
% \renewcommand{\marginpar}[1]{%
%     \leavevmode%
%     \oldmarginpar{#1}%
%     \ignorespacesafterend\ignorespaces}
% \newsavebox\marginprooflinkbox
% \newenvironment{linked}[3][]{%
%     \def\linkedproof{#3}%
%     \def\linkedtype{#2}%
%     \ifmarginprooflinks%
%     \sbox\marginprooflinkbox{%
%         \centering%
%         \hyperref[proof:\linkedproof]{%
%         \color{blue!30!white}%
%         \scaleleftright{$\Big[$}{\,\mbox{\footnotesize\centering\tt\begin{tabular}{@{}c@{}}
%             link to\\[-0.15em]
%             proof
%         \end{tabular}}\,}{$\Big]$}}~}
%     \fi
%         \restatable[#1]{#2}{#2:#3}\label{#2:#3}%
%     \reversemarginpar	\ifmarginprooflinks\marginpar{\vspace{-1ex}\usebox\marginprooflinkbox}\fi
%     }%
%     {\sbox\marginprooflinkbox{}\endrestatable}
% \newcounter{proofcntr}
% \newenvironment{lproof}{\begin{proof}\refstepcounter{proofcntr}}{\end{proof}}

\newcommand{\vect}[1]{\ensuremath{\mathbf{#1}}}

%% Useful
\newcommand{\p}[1]{\left( #1 \right)}
\newcommand{\br}[1]{\left[ #1 \right]}


%\newcommand{\ev}[1]{\mathbb{E}\left[{#1}\right]}
\newcommand{\evd}[2]{\mathbb{E}_{#1}\left[{#2}\right]}

%% Algortihm notations
\newcommand{\bigO}[1]{O \left( #1 \right )}

%% Calibration
\newcommand{\I}[1]{\mathbb{I}\left[#1\right]}       % Indicator
\newcommand{\calerr}{\mathrm{calerr}}   % Calibration error

\newcommand{\A}{\mathcal{A}}    % Algorithm
\newcommand{\Ber}{\mathrm{Ber}}
\newcommand{\Ecover}{\event^{\textrm{cover}}}   % Event that covered epochs exist
\newcommand{\Enegl}{\event^{\textrm{negl}}}     % Event that negligible epochs exist
\newcommand{\Epoch}{\mathsf{Epoch}}
\newcommand{\eps}{\epsilon}     % epsilon
\newcommand{\Etruth}{\event^{\textrm{truth}}}   % Event that all epochs are truthful
\newcommand{\event}{\mathcal{E}}    % Events
\newcommand{\Ex}[2]{\operatorname*{\mathbb{E}}_{#1}\left[#2\right]}  % Expectation
\newcommand{\Int}{\mathcal{I}}      % Interval
\newcommand{\poly}{\operatorname*{poly}}    % Polynomial
\newcommand{\pr}[1]{\Pr\left[#1\right]}     % Probability
%\newcommand{\red}[1]{{\color{red} #1}}

\newcommand{\red}[1]{\textcolor{red}{#1}}
\newcommand{\blue}[1]{\textcolor{blue}{#1}}

\newcommand{\SPinner}{\mathsf{SP}^{\textrm{inner}}}
\newcommand{\SPouter}{\mathsf{SP}^{\textrm{outer}}}
\newcommand{\Tact}{T^{\mathrm{actual}}}     % Actual stopping time
\newcommand{\prodspace}{\mathcal{X}\times A \times \mathcal{Y}}


%% Fair ERM Notation
\newcommand{\error}[1]{ \left| \mathbb{E}_{(x,y) \sim \mathcal{D}} \ [\one (#1(x) \neq y)] - \ \mathbb{E}_{(x,y) \sim \mathcal{D}} \ [\one (h^*(x) \neq y)] \right|}
\newcommand{\htilde}{\tilde{h}}
\newcommand{\hhat}{\hat{h}}
\newcommand{\hstar}{h^*}
\newcommand{\hclass}{\mathcal{H}}
\newcommand{\posrate}[1]{ P_{(x,y) \sim \DA} [#1 (x)=1]}

\newcommand{\DAC}{\widetilde{\mathcal{D}}_A}
\newcommand{\DBC}{\widetilde{\mathcal{D}}_B}
\newcommand{\RA}{P_{(x,y) \sim \dist } [x \in A]}
\newcommand{\RB}{P_{(x,y) \sim \dist} [x \in B]}
\newcommand{\normalF}{F}
\newcommand{\corruptF}{\widetilde{F}}

\usepackage{mkolar_definitions}
\usepackage{multirow}
\usepackage{amsmath}
\usepackage{comment}

\newcommand{\dist}{\mathcal{D}}
\newcommand{\DA}{\mathcal{D}_A}
\newcommand{\DB}{\mathcal{D}_B}
\newcommand{\closure}{cl(\mathcal{H})}
\newcommand{\Bern}{\text{Bernoulli}}

%\usepackage{color-edits}%[suppress] % use suppress below instead to implement all the edits and remove comments
\usepackage[suppress]{color-edits} 
\addauthor{pco}{purple}
\addauthor{kms}{orange}
\addauthor{as}{cyan}
\usepackage{alg}
\usepackage{multirow}
\usepackage{graphicx}


%\usepackage[demo]{graphicx}
%\usepackage{subfig}

% to compile a preprint version, e.g., for submission to arXiv, add add the
% [preprint] option:
%     \usepackage[preprint]{neurips_2023}


% to compile a camera-ready version, add the [final] option, e.g.:
%     \usepackage[final]{neurips_2023}


% to avoid loading the natbib package, add option nonatbib:
%\usepackage[nonatbib]{neurips_2023}


\usepackage[utf8]{inputenc} % allow utf-8 input
\usepackage{amsmath}
\usepackage{amsfonts}
\usepackage{amssymb}
\usepackage{bbm}


\usepackage[T1]{fontenc}    % use 8-bit T1 fonts
\usepackage{hyperref}       % hyperlinks
\usepackage{url}            % simple URL typesetting
\usepackage{booktabs}       % professional-quality tables
\usepackage{amsfonts}       % blackboard math symbols
\usepackage{nicefrac}       % compact symbols for 1/2, etc.
\usepackage{microtype}      % microtypography
\usepackage{xcolor}         % colors
%%%%%%
%Best title


\title{On the Vulnerability of Fairness Constrained Learning to Malicious Noise}


%%%%%%%%%%%%%
%\title{Malicious Noise and an Improper Fair ERM: Not Impossible Fairness Aware Learning from Corrupted Data}

%\title{ Fairness Aware Learning from Corrupted Data is not Impossible}

%%%GOOD titles 
%\title{On Fairness Constrained Learning and Malicious Noise}
%\title{On Fairness and Malicious Noise }********
%\title{Fair Learning with Malicious Noise Is Possible, with Randomized Classifiers }

%\title{On Fairness, Malicious Noise, and Improper Learning} %%% SABA LIKES THIS
%\title{Fair Classification and Malicious Noise}

%\title{On Fair Learning, Malicious Noise, and Randomization}
%\title{On the vulnerability of Fair Learning to Malicious Noise}


%\title{Fairness Through Randomization despite Malicious Noise}
%\title{Fairness and Malicious Noise}


%\title{On The Vulnerability of Fairness-Constrained Learning to Data Poisoning and Malicious Noise} 

%------
% The \author macro works with any number of authors. There are two commands
% used to separate the names and addresses of multiple authors: \And and \AND.
%
% Using \And between authors leaves it to LaTeX to determine where to break the
% lines. Using \AND forces a line break at that point. So, if LaTeX puts 3 of 4
% authors names on the first line, and the last on the second line, try using
% \AND instead of \And before the third author name.

\author{%
 Avrim Blum \\
TTIC \footnote{Toyota Technological Institute at Chicago  } \\
avrim@ttic.edu
 \and
 Princewill  Okoroafor \\
 Cornell University \\
 pco9@cornell.edu
 \and
 Aadirupa Saha \\
 TTIC \\
 aadirupa@ttic.edu
 \and
 Kevin Stangl \\
TTIC  \\
 kevin@ttic.edu
}
% \author{%
%   David S.~Hippocampus\thanks{Use footnote for providing further information
%     about author (webpage, alternative address)---\emph{not} for acknowledging
%     funding agencies.} \\
%   Department of Computer Science\\
%   Cranberry-Lemon University\\
%   Pittsburgh, PA 15213 \\
%   \texttt{hippo@cs.cranberry-lemon.edu} \\
%   % examples of more authors
%   % \And
%   % Coauthor \\
%   % Affiliation \\
%   % Address \\
%   % \texttt{email} \\
%   % \AND
%   % Coauthor \\
%   % Affiliation \\
%   % Address \\
%   % \texttt{email} \\
%   % \And
%   % Coauthor \\
%   % Affiliation \\
%   % Address \\
%   % \texttt{email} \\
%   % \And
%   % Coauthor \\
%   % Affiliation \\
%   % Address \\
%   % \texttt{email} \\
% }

\begin{document}


\maketitle


\begin{abstract}

The Fast Reciprocal Square Root Algorithm is a well-established approximation technique consisting of two stages: first, a coarse approximation is obtained by manipulating the bit pattern of the floating point argument using integer instructions, and second, the coarse result is refined through one or more steps, traditionally using Newtonian iteration but alternatively using improved expressions with carefully chosen numerical constants found by other authors. The algorithm was widely used before microprocessors carried built-in hardware support for computing reciprocal square roots. At the time of writing, however, there is in general no hardware acceleration for computing other fixed fractional powers. This paper generalises the algorithm to cater to all rational powers, and to support any polynomial degree(s) in the refinement step(s), and under the assumption of unlimited floating point precision provides a procedure which automatically constructs provably optimal constants in all of these cases. It is also shown that, under certain assumptions, the use of monic refinement polynomials yields results which are much better placed with respect to the cost/accuracy tradeoff than those obtained using general polynomials. Further extensions are also analysed, and several new best approximations are given.

\end{abstract}


% Figure environment removed

\section{Introduction}
Automatic 3D reconstruction of clothed humans using image inputs has gained increasing significance due to its potential applications in a wide array of AR/VR scenarios. High-fidelity reconstructions typically depend on sophisticated capture systems, which are developed with dense camera arrays~\cite{collet2015high,joo2015panoptic,joo2018total}, programmable light-stages~\cite{Vlasic2009, guo2019relightables}, and depth sensors~\cite{newcombe2011kinectfusion,DoubleFusion,BodyFusion,dou2016fusion4d,newcombe2015dynamicfusion}. However, stringent capture environments equipped with complex hardware pose significant challenges for consumer-level applications.


In this context, considerable research effort has been dedicated to developing methods that allow for more flexible capture configurations, such as utilizing a few RGB inputs. Among these works, learning implicit functions \cite{iccv2020PIFu, saito2020pifuhd, hong2021stereopifu} has proven effective in achieving highly detailed reconstructions by integrating the advancements of deep neural networks. These methods employ large multi-layer perceptrons (MLPs) to predict the occupancy probability or truncated signed distance function (TSDF) value of every queried 3D point based on its associated local feature, which is extracted from images. They can recover a continuous surface at arbitrary resolutions without topology restrictions.


However, in typical MLP-based implicit networks, the occupancy or TSDF value at each location is solved independently with planar image features, rendering them less capable of addressing challenging cases such as occlusions. Consequently, these methods suffer from generalization and robustness issues, particularly when tackling strong occlusions caused by large motion or multiple interacting humans. 
Some follow-up studies  \cite{zheng2021deepmulticap,zheng2021pamir,huang2020arch} utilize an extra geometric model, SMPL~\cite{Loper2015}, to improve robustness by introducing strong shape priors. 
Their success typically relies on the assumption of geometrical similarity \cite{huang2020arch} between the shape prior and target reconstruction, making them intractable for handling complex cases with loose clothes and sensitive to errors in SMPL model fitting.



%\ping{this paragraph sounds like `TSDF is better than MLP/SMPL, and we use TSDF to solve the problem'. But in Sec 3, we are telling a different story, saying `MLP needs a 3D convolutional encoder'. We need to make these two sections consistent.}\sicong{I think in this paragraph we claim that the TSDF}


%We opt for Trucated Signed Distance Funtion (TSDF) volumetric representations as they are naturally suitable for convolution operations, which have shown remarkable performance for learning hierarchical features on 2D visual perception tasks \cite{SunXLW19}. 
%Meanwhile, TSDF also describes the gradual geometry change around shape surface, which is not reflected by occupancy volume. 

We instead revisit the 3D volumetric representation and resort to 3D convolutional neural networks (CNNs) for feature learning, due to their impressive performance in feature learning and the ability to incorporate spatial context. However, volumetric methods and 3D convolution involve discretization, which might raise concerns regarding whether a discretized volume can preserve subtle geometric details as continuous representations learned in implicit functions. We investigate the relationship between volume resolution and quantization error on synthetic data by converting target mesh objects to TSDF volumes, as shown in Figure~\ref{fig:quantization_error}. We observe that the quantization errors are significantly reduced by increasing volume resolution and become nearly negligible when reaching a relatively high resolution (e.g., 512 or higher). In other words, achieving fine-detailed reconstruction is not supposed to be restricted by the use of volume representations as long as a proper volume resolution is utilized. Therefore, we present a method with high-resolution feature volumes, e.g., 256 and 512, while traditional volumetric methods \cite{varol18_bodynet,gilbert2018volumetric} are often limited to much lower resolutions, such as 32 or 128.



On the other hand, an increase in volume resolution may lead to a cubic growth of memory overhead \cite{8100085}. Reducing memory costs while guaranteeing the granularity of volumetric representations is necessary for pursuing high-quality reconstruction. Thus, we adopt a coarse-to-fine approach and cull away irrelevant voxels to build a sparse high-resolution feature volume. At the coarse level, the network computes an initial TSDF by applying a U-Net with sparse 3D CNN \cite{3DSemanticSegmentationWithSubmanifoldSparseConvNet} on the sparse feature volume, which is carved by a visual hull. Through our experiments, it turns out that more than 95\% of the volume grids are discarded by the visual hull culling, making the sparse 3D CNN efficient. At the fine level, the network focuses on a narrow band near the zero-level set of the initial TSDF and discretizes the narrow band with smaller voxels. By employing this narrow-band culling, we further shrink the sampling space, resulting in a relatively small range of grid numbers (usually 300K--500K in our experiments) even with a high volume resolution of 512. The remaining voxels in the narrow band are associated with features that fuse high-frequency information from the computed normal maps upon the low-frequency shape from the coarse level to compute the TSDF at high resolution. The final mesh is then extracted from the TSDF using the Marching-Cube algorithm ~\cite{Lorensen87marchingcubes}.
% Different from the u-net sturcture to preserve global topology context, we then apply a shallow 3dcnn to compute the final TSDF $D_{final}$ which contain more local geometry detail.




% \ping{this paragraph can be expanded. It is an important contribution and often ignored by other works. stress on the novel idea of regressing blending weights instead of colors}

In addition to geometry, high-quality mesh texture is also a crucial factor contributing to visual appearance. Directly computing a color field in 3D space, as in \cite{iccv2020PIFu}, struggles to capture high-frequency texture details, while the neural radiance field (NeRF) \cite{yu2020pixelnerf} or the DoubleField~\cite{shao2022doublefield} require expensive per-instance optimization and are often unstable for sparse input images. In contrast, we adopt an image-based rendering approach to compute a texture atlas map, which is efficient and widely supported in existing computer graphics tools. 
Specifically, we compute a blending weight at each 3D point on the mesh surface to determine its color as a weighted average of the colors at its image projections. The blending weights can be computed at a relatively coarse resolution, e.g., 512 volume resolution in our case, and leave texture details to the high-resolution images, such as 1K or 2K. Unlike previous methods that generate blurry texturing results under sparse input, our method generalizes well on both synthetic and real data with just a few input views. 
Figure~\ref{fig:teaser} shows two examples reconstructed by our method. Despite the challenging garment, pose, and occlusion, our method recovers faithful shape, normal, and texture on the right.

%with a wide variety of poses and clothing styles, and it is also adaptive to handle input image with arbitrary resolutions.
%\sicong{For this concern we claim that when the resolution of dicretized volume meets certain threshold (which is 256 in our experiment), the quantization error can be neglected.} 



In summary, the main contributions of this paper are as follows:
\begin{itemize}
\vspace{-0.1in}
  \item 
  We revisit the 3D volumetric representation and demonstrate that it can support clothed human reconstruction with equal or even better performance compared to implicit representation. 
  \item 
  We develop a memory and computation-efficient method for high-resolution volumetric reconstruction using sophisticated sparse 3D CNN, coarse-to-fine estimation, and voxel culling by visual hull and narrow bands. 
  \item 
  We introduce a novel method to compute a texture atlas map, which captures rich appearance details from high-resolution input images.
  \item 
  We achieve impressive results on standard benchmark datasets Twindom and MultiHuman, significantly reducing the point-2-surface (P2S) precision to approximately 0.2cm from just six input views, with more than $50\%$ error reduction compared to the state-of-the-art methods, including DoubleField~\cite{shao2022doublefield} and PIFuHD~\cite{saito2020pifuhd}.
\end{itemize}

\section{Preliminaries}
In this section, we describe the necessary background for automated planning and the significance of the International Planning Competition. 

% \subsection{Ontology}
% A formal ontology is typically represented as a set of concepts, relations, and axioms. A concept represents a set of objects or entities that share common properties, while a relation represents a connection or association between two or more concepts. Axioms are statements that define the relationships between concepts and relations. It is a formal representation of knowledge that is designed to facilitate automated reasoning and information processing. It acts as a structured vocabulary that describes a domain and promotes interoperability, data integration, and communication between humans and machines. Formally, an ontology $O$ can be represented as a tuple $(C, R, A)$, where $C$ is the set of concepts, $R$ is the set of relations, and $A$ is the set of axioms. Each concept \textit{c} $\in$ $C$ can be represented as a set of attributes, denoted as $Att(c)$. Similarly, each relation \textit{r} $\in$ $R$ can be represented as a set of attributes, denoted as $Att(r)$.

% Ontology is a branch of philosophy that deals with the nature of existence and being. In the field of computer science, however, ontology refers to a formal representation of knowledge that is designed to facilitate automated reasoning and information processing. It is a structured vocabulary that describes a domain and promotes interoperability, data integration, and communication between humans and machines. Various tools and methodologies, including Protege and ontology editors, are available for ontology creation. Ontologies are increasingly important in artificial intelligence, knowledge engineering, and the semantic web, and researchers are exploring their potential in diverse domains and applications.

% Figure environment removed

\subsection{Automated Planning}

Automated planning, also known as AI planning, is the process of finding a sequence of actions that will transform an initial state of the world into a desired goal state \cite{ghallab2004automated}. It involves constructing a plan or a sequence of actions that will achieve a specified objective while respecting any constraints or limitations that may be present. Formally, automated planning can be defined as a tuple $(S, A, T, I, G)$, where:
\begin{itemize}
    \item $S$ is the set of possible states of the world
    \item $A$ is the set of possible actions that can be taken
    \item $T$ is the transition function that describes the effects of taking an action on the current state of the world
    \item $I$ is the initial state of the world
    \item $G$ is the desired goal state
\end{itemize}
Using this notation, the problem of automated planning can be framed as finding a sequence of actions $\prec a_1, a_2, ..., a_k\succ$ that will transform the initial state $I$ into the goal state $G$, while respecting any constraints or limitations on the actions. 
 % In automated planning, 
 A problem is defined in terms of a domain and a problem instance. The domain defines the possible actions that can be taken and the effects of each action, while the problem instance specifies the initial state of the world and the desired goal state. 
Various techniques can be used to solve the planning problem, such as search algorithms, constraint-based reasoning, and optimization methods. These techniques involve exploring the space of possible plans and selecting the one that satisfies the objective and any constraints. Figure \ref{fig:planning_bw} illustrates an automated planning scenario for the blocksworld domain, where an initial state can be transformed into a goal state by executing a sequence of actions.

% \noindent \textbf{Attributes modeled about a domain.}
%   %\noindent \textbf{Attributes modeled in a domain file}
%  \begin{enumerate}
%      \item \textbf{Requirements:} A list of requirements that the planner must satisfy in order to solve the domain. Requirements include durative actions, conditional effects, or negative preconditions. For example, in blocksworld domain with types involved, one of the requirements is \emph{typing}.
%     \item \textbf{Predicates:} Predicates are fundamental elements in the planning domain that define the properties of the world. They are used to describe the initial and goal states, as well as the preconditions and effects of actions. Predicates are usually defined as logical expressions over a set of variables, where each variable can take on a finite number of values. In the context of planning, predicates are typically used to represent facts about the world that can be true or false, such as the location of an object or the status of a machine. For example, in blocksworld domain, the predicate \verb|(on b1 b2)| could indicate that block 'b2' is on top of block 'b1'.
%      \item \textbf{Actions:} Actions are the basic units of change in the planning domain. They represent atomic operations that can be performed to transform the world from one state to another. Each action has a name, a set of parameters, preconditions that must be satisfied before the action can be executed, and effects that describe the changes that the action makes to the world. Actions can be used to model a wide variety of operations, ranging from simple movements or transformations to complex processes such as planning or decision-making. For example, in blocksworld domain, the action \verb|unstack b2 b1| can be used to unstack block 'b2' from block 'b1'. 
     
%      \item \textbf{Preconditions:} Preconditions are the conditions that must be true before an action can be executed. They are usually defined using predicates and can involve multiple variables. Preconditions can also be negative, which means that a certain condition must not be true for an action to be executed. In planning, preconditions ensure that actions are only executed when the necessary conditions have been met, such as ensuring that a machine is turned off before it is serviced. For example, in blocksworld domain, the action \verb|unstack b2 b1| has a precondition of \verb|(on b1 b2)|, meaning that for the action to be valid, the block 'b2' should be on top of block 'b1'.
     
%      \item \textbf{Effects:} Effects describe the changes that an action makes to the world. They are usually defined using predicates and can involve multiple variables. Effects can be positive, which means that a certain condition becomes true after the action is executed, or negative, which means that a certain condition becomes false after the action is executed. In the context of planning, effects are used to model the changes that result from executing an action, such as moving an object from one location to another or turning a machine on. For example, in blocksworld domain, when the action \verb|unstack b2 b1| is executed, one of its effect is \verb|(not (on b1 b2))|, indicating that block 'b2' is no longer on top of block 'b1'.
     
%      \item \textbf{Constants:} Constants are values that are fixed and do not change during the execution of the planning problem. They are used to represent objects or entities in the world that have a fixed value, such as the speed limit on a road. Constants can be used to simplify the planning problem by reducing the number of variables that need to be considered and by providing a fixed set of values that can be used in predicates and actions. For example, in blocksworld domain, the constant \emph{table} could represent the surface on which the blocks are initially placed.
     
%      \item \textbf{Types:} Types are used to classify objects or entities in the world based on their attributes or properties. They are used to define the domain of values that a variable can take on and can be used to constrain the values that are assigned to variables. In the context of planning, types are typically used to group related objects or entities together, such as cars or bicycles, and to specify the properties that are common to all members of a type, such as their color or size. For example, in blocksworld domain with types involved, one can represent the predicate as \verb|(on ?x - block ?y - block)| stating that the parameters in the predicate are of type \emph{block}.

%  \end{enumerate}


% ######### Shorter version for AI Planning preliminaries
% \subsection{Automated Planning}

% Automated planning, also known as AI planning, finds actions transforming an initial world state into a goal state \cite{ghallab2004automated}. It involves creating a plan, respecting constraints, defined as $(S, A, T, I, G)$ where $S$ is the world states set, $A$ is the actions set, $T$ is the state transition function, $I$ is the initial state, and $G$ is the goal state. The challenge is to find actions $\prec a_1, a_2, ..., a_k\succ$ converting $I$ to $G$ under constraints. 

% A problem has a domain (defining actions and effects) and an instance (specifying initial and goal states). Various techniques can be used to solve the planning problem, such as search algorithms, constraint-based reasoning, and optimization methods. These techniques involve exploring the space of possible plans and selecting the one that satisfies the objective and any constraints. Figure \ref{fig:planning_bw} illustrates an automated planning scenario for the blocksworld domain, where an initial state can be transformed into a goal state by executing a sequence of actions.

\noindent \textbf{Attributes modeled about a domain.}
 \begin{enumerate}
     \item \textbf{Requirements:} A list of requirements that the planner must satisfy to solve the given domain, e.g., \emph{typing} in blocksworld with types.
     \item \textbf{Predicates:} Define world properties, e.g., \verb|(on b1 b2)| in blocksworld.
     \item \textbf{Actions:} Units of change with preconditions and effects, e.g., \verb|unstack b2 b1| in blocksworld.
     \item \textbf{Preconditions:} Conditions for action execution, e.g., \verb|(on b1 b2)| for \\ \verb|unstack b2 b1|.
     \item \textbf{Effects:} Post-action world changes, e.g., \verb|(not (on b1 b2))| after \\ \verb|unstack b2 b1|.
     \item \textbf{Constants:} Fixed values, e.g., \emph{table} in blocksworld.
     \item \textbf{Types:} Classifications based on attributes, e.g., \\ \verb|(on ?x - block ?y - block)| in typed blocksworld.
 \end{enumerate}

\noindent \textbf{Attributes modeled about a problem instance from a domain.}
\begin{enumerate}
    \item \textbf{Name:} The name of the planning problem.
    \item \textbf{Domain:} The name of the planning domain that the problem belongs to.
    \item \textbf{Objects:} A list of objects that are present in the planning problem. Objects are typically defined in terms of their type and name. In the example shown in Figure \ref{fig:planning_bw}, objects are b1, b2, and b3.
    \item \textbf{Initial State:} A description of the initial state of the world, including the values of all relevant predicates. Figure \ref{fig:planning_bw} represents an example initial state.
    \item \textbf{Goal State:} A description of the desired goal state of the world, including the values of all relevant predicates. Figure \ref{fig:planning_bw} represents an example goal state.
\end{enumerate}

% \vspace{2cm}
\subsection{International Planning Competition (IPC)}

% IPC serves as a significant means of assessing and comparing various planning systems. By presenting new planners and benchmark problems each year, the competitions aim to stimulate the advancement of new planning methodologies and reflect current trends and challenges in the field. The competition comprises multiple tracks, each covering various planning problems such as classical, temporal, and probabilistic planning. These tracks include benchmark problems that evaluate the performance of planners concerning parameters such as plan quality, plan length, and run time. The results of these competitions provide insights into the current state-of-the-art in planning and help identify the strengths and weaknesses of different planning systems. IPC can serve as an excellent starting point for building a planning-related ontology as the benchmark problems used in these competitions can provide a comprehensive overview of the domain and the types of problems that planners need to solve. 

IPC is pivotal for evaluating and contrasting planning systems. Introducing new planners and benchmarks, it promotes innovative planning methodologies and reflects the field's evolving challenges. The competition has multiple tracks, such as classical and probabilistic planning, with benchmarks assessing plan quality, length, and run time. IPC results offer a glimpse into the latest in planning, highlighting system pros and cons. The benchmarks from IPC are ideal for crafting a planning-related ontology, encapsulating the domain's breadth and planners' challenges.


\subsection{ Multiple Groups}
\kmsmargincomment{I think this is all true but I need to think more closely---low priority since it can be done in the camera ready version}
In this paper, for ease of presentation we will focus on the disjoint case of binary groups i.e. two disjoint groups. 
Our core theorems have natural extensions to the case of an integer number of groups, where the adversary will typically attack the \emph{smallest group}.
%If the groups have non-empty intersection, this will not change the lower/upper bounds in this paper with the appropriate modification which we discuss in Appendix \ref{multgroup}.
\pcomargincomment{Is this actually true? It's not obvious to me that this holds. If a point lies in the intersection of A and B, how would the randomized classifier behave? which biased coin would they toss, the one for A or for B}
%In the case of group overlap, we can observe that our lower bounds depend on the size of the smallest group. 
The elegant and contemporary approach in \cite{multicalib} of defining relevant groups with a computationally rich class $C$ handles the issue of group and sub-group 
overlap \pcodelete{naturally}, %two naturals next to each other
but a natural research question is how that framework interacts with malicious noise and our closure model.
We briefly discuss this in Section \ref{sec:calib} as a direction for future research.
%but a full treatment of this important problem will be left for future research.


\section{Results for Human Rating}

Before the theoretical comparison with the preference-based approach, let us first establish some theoretical results for our more general rating model. In particular, we analyze the suboptimality of the LCB algorithm under our more practical rating model. These results can provide some theoretical explanation for how human bias and uncertainty could adversely affect policy learning.

In the case of human rating, we are given an offline dataset $\cD = \{(s_i,a_i, \widetilde{r}_i)\}_{i=1}^{n}$. The state-action pairs in $\cD$ are generated in an i.i.d. fashion according to a sampling distribution over the state-action space. The sampling probability of the state-action pair $(s,a)$ is denoted with $d(s,a)$. For each $(s_i,a_i)$, the human annotator provides a \textit{rating sample} $\widetilde{r}_i$ following the rating model \eqref{eq:rating-h-model} based on the true reward $r(s_i,a_i)$.

Let us also make a brief review of the standard LCB approach for offline policy learning \citep{jin2021pessimism,rashidinejad2021bridging,yin2021towards}. In the existing literature, it is common to assume the knowledge of a reasonable upper bound on the variance of reward observations. Similarly, we assume there exists an upper bound on the variance $\Var_\epsilon(h(r,\epsilon))$ for all $r\in[0,R]$, which we denote with $V_{R, \sigma}^2$ and can depend on $R$ and $\sigma$. Recall that the learner has no knowledge of the transformation $h$, but let us assume the learner can make a reasonable estimate $\widetilde{V}_{R,\sigma}^2$ for the true variance $V_{R, \sigma}^2$ such that $\widetilde{V}_{R, \sigma}^2 = c_VV_{R, \sigma}^2$, where $c_V > 0$ is an absolute constant. To learn the optimal policy with at least $1-\delta$ success probability, the standard LCB algorithm (Algorithm \ref{alg:LCB}) uses a penalty in the form of
\begin{equation}\label{eq:standard-LCB}
    b_m = c_b \sqrt{\frac{\widetilde{V}_{R,\sigma}^2\log\frac{SA}{\delta}}{m}}
\end{equation}
with an appropriately chosen constant $c_b$. 

To understand the effects of human bias and uncertainty on policy learning under our more realistic rating model, let us establish the lower bound on the suboptimality of the LCB algorithm. We will consider two scenarios with different coverage assumptions for the offline dataset $\cD$.

\begin{algorithm}[t]
    \caption{LCB for contextual bandits} \label{alg:LCB}
    \begin{algorithmic}[1]
        \STATE \textbf{Input:} Offline dataset $\cD$, confidence level $\delta\in(0,1)$.

        \FOR{all $(s,a)\in\cS\times\cA$}
            \STATE Set $n_{(s,a)} = \sum_{i=1}^{n}\ind\{(s_i,a_i) = (s,a)\}$;

            \STATE Set $\widetilde{r}(s,a) = \frac{1}{n}\sum_{i=1}^{n} \widetilde{r}_{i}\ind\{(s_i,a_i) = (s,a)\}$;

            \STATE Set $\widehat{r}(s,a) = \max\{\widetilde{r}(s,a) - b_{n_{(s,a)}}, 0\}$;
        \ENDFOR

        \RETURN $\widehat{\pi}_{\rm LCB}(\cdot) = \arg\max_{a\in\cA}\widehat{r}(\cdot,a)$.
    \end{algorithmic}
\end{algorithm}

\subsection{Lower Bound under Partial Coverage}

As \cite{rashidinejad2021bridging,yin2021towards} have shown, to learn the optimal policy in the offline setting, it is sufficient for the sampling distribution of the offline dataset to cover the state-action pairs that the optimal policy can reach. Concretely, this assumption can be written as follows:
\begin{assumption}\label{assumption:Cstar}
    \textit{There exists an optimal policy $\pi^\star$ such that $d(s,a) > 0$ whenever $d^{\pi^\star}_\rho(s,a) > 0$ for any $(s,a)\in\cS\times\cA$.}
\end{assumption}
Under this assumption, it makes sense to define a concentrability coefficient $C^\star$ as follows:
\begin{equation}
    C^\star := \max_{(s,a)\in\cX}\frac{d^{\pi^\star}_\rho(s,a)}{d(s,a)},
\end{equation}
where the set $\cX$ is the set of all state-action pairs that the sampling distribution of $\cD$ can cover, i.e., $\cX := \{(s,a)\in\cS\times\cA ~:~ d(s,a)> 0\}$. Under Assumption \ref{assumption:Cstar}, if the reward can be observed exactly or with only additive sub-gaussian noise, the LCB algorithm (Algorithm \ref{alg:LCB}) with penalty \eqref{eq:standard-LCB} is guaranteed to converge to the optimal policy \citep{rashidinejad2021bridging,yin2021towards}. However, theory suggests it does not converge in the worst case when the reward function is engineered from human rating. In particular, let us consider the setting beyond the standard additive sub-gaussian noise, which has been well-studied in the existing literature. That is, let us consider a more practical model in the form of \eqref{eq:rating-h-model} with $q \ge 2$. We can prove that even when the rating model preserves the correct reward ordering in expectation and keeps the policy learning problem consistent, it is possible that the LCB algorithm does not converge to the optimal policy and must suffer constant suboptimality.

\begin{theorem}\label{thm:Cstar}
    \textit{For any fixed constant $0 < \delta < 1$, there exists a contextual bandit instance with initial state distribution $\rho$ such that if one samples a dataset $\cD$ of size $n \ge c(\delta,c_b,c_V,q,\sigma,R)$ using a sampling distribution $d$ satisfying Assumption \ref{assumption:Cstar} with $C^\star = 2$ and runs Algorithm \ref{alg:LCB} on $\cD$, the output policy $\widehat{\pi}_{\rm LCB}$ must suffer constant suboptimality, i.e., 
    \begin{equation}
        \EE_\cD[\SubOpt(\widehat{\pi}_{\rm LCB})] = c_0 R,
    \end{equation}
    where $c_0$ is a universal constant and $c(\delta,c_b,c_V,q,\sigma,R)$ is a constant depending on $\delta,c_b,c_V,q,\sigma,R$.}
\end{theorem}

This result is reminiscent of Proposition 1 in \cite{rashidinejad2021bridging}, which constructs a bandit and shows the empirically best policy chooses a suboptimal action with constant probability under Assumption \ref{assumption:Cstar}. The very same work also shows that by adding a pessimism penalty, the LCB algorithm (Algorithm \ref{alg:LCB}) can converge to the optimal policy under the same data coverage assumption. In contrast, our theorem shows that even when we make pessimistic estimates and penalize less-observed state-action pairs in human rating data, a constant suboptimality can still ensue. This shows a disadvantage of using human rating as reward samples directly: although the estimation problem induced by human rating is still consistent, using LCB with only the knowledge of variance is not sufficient for convergence. Instead, the learner needs to know the shape of the noise distribution, but it is unrealistic to model the human uncertainty accurately in practice. 

\textbf{Proof sketch} In a bandit instance with special reward design, we first find the lower bound for the probability that suboptimal actions are only observed for a very small number of times in the offline dataset. Such state-action pairs can have huge fluctuation in their empirical reward average and mislead the algorithm. Then, we find the lower bound on the probability that a state-action pair $(s,a)$ such that $\widehat{r}(s,a) > \widehat{r}(s,a^\star)$ exists, which can cause the algorithm to always select the suboptimal action $a$ and suffer suboptimality. Different from Proposition 1 in \cite{rashidinejad2021bridging}, in which the reward noise for suboptimal actions is defined with two Dirac delta functions, the noise under our rating model is unbounded and can be viewed as a Gaussian chaos, so we compute this probability using a method from the corresponding literature. Moreover, in the same paper, a bandit instance is sufficient to induce constant suboptimality as long as its action space is designed large. In our case, since the pessimism penalty in Algorithm \ref{alg:LCB} accounts for the bandit size and larger bandit instances are penalized more, it requires a careful balance in the design of our bandit instance.

For concreteness, let us also provide a corollary under the example rating model in \eqref{eq:h-example} as follows.

\begin{corollary}\label{cor:Cstar}
    \textit{For any fixed constant $0 < \delta < 1$, there exists a contextual bandit instance with initial state distribution $\rho$ such that if one samples a dataset $\cD$ of size $n \ge c(\delta,c_b,c_V,\sigma,R)$ using a sampling distribution $d$ satisfying Assumption \ref{assumption:Cstar} with $C^\star = 2$ and runs Algorithm \ref{alg:LCB} on $\cD$, the output policy $\widehat{\pi}_{\rm LCB}$ must suffer constant suboptimality, i.e., 
    \begin{equation}
        \EE_\cD[\SubOpt(\widehat{\pi}_{\rm LCB})] = c_0,
    \end{equation}
    where $c_0$ is a universal constant and $c(\delta,c_b,c_V,\sigma,R)$ is a constant depending on $\delta,c_b,c_V,\sigma,R$.}
\end{corollary}

\subsection{Lower Bound under Full Coverage}

Uniform coverage is another popular coverage assumption for offline policy learning \citep{yin2021near,uehara2021finite,hao2021bootstrapping}. It can be written as follows:
\begin{assumption}\label{assumption:uniform}
    \textit{The sampling distribution satisfies $d(s,a) > 0$ for any $(s,a)\in\cS\times\cA$.}
\end{assumption}
This coverage assumption is much stronger than Assumption \ref{assumption:Cstar} and makes the offline policy learning problem much easier. Under Assumption \ref{assumption:uniform}, many algorithms without the pessimism principle can also be shown to provably converge to the optimal policy \citep{chen2019information,xie2020q}. Moreover, \cite{jin2021pessimism} showed that the suboptimality of algorithms with pessimism can decay faster when the data are well-explored. In this setting, we establish a lower bound on the suboptimality of Algorithm \ref{alg:LCB} under Assumption \ref{assumption:uniform}. 

\begin{theorem}\label{thm:uniform}
    \textit{For any fixed constant $0 < \delta < 1$, there exists a contextual bandit instance with initial state distribution $\rho$ such that if one samples a dataset $\cD$ of size $n \ge \max\{48\sigma^4, 60\}$ using a sampling distribution $d$ satisfying Assumption \ref{assumption:uniform} with $d(s,a) = \frac{1}{SA}$ for every $s\in\cS$ and $a\in\cA$ and runs Algorithm \ref{alg:LCB} on $\cD$, the output policy $\widehat{\pi}_{\rm LCB}$ must suffer suboptimality at least
    \begin{equation*}
        \EE_\cD[\SubOpt(\widehat{\pi}_{\rm LCB})] = c_0\cdot \bar{h}^{-1}\left(\sqrt{\frac{V_{R,\sigma}^2}{n}}\right),
    \end{equation*}
    where $c_0$ is a constant that depends on $q$.}
\end{theorem}

In fact, under uniform data coverage as in Theorem \ref{thm:uniform}, pessimism becomes unnecessary and this result holds no matter what penalty $b_n$ is used in the algorithm. This theorem demonstrates another disadvantage of human rating: even when the data covers the entire state-action space and learning is no longer impeded by the lack of knowledge of human uncertainty, the suboptimality is still bottlenecked by human bias.

We also provide a corollary corresponding to the example model in \eqref{eq:h-example}, which shows the suboptimality can decay more slowly under the influence of annotator bias.

\begin{corollary}\label{cor:uniform}
    \textit{For any fixed constant $0 < \delta < 1$, there exists a contextual bandit instance with initial state distribution $\rho$ such that if one samples a dataset $\cD$ of size $n \ge \max\{48\sigma^4, 60\}$ using a sampling distribution $d$ satisfying Assumption \ref{assumption:uniform} with $d(s,a) = \frac{1}{SA}$ for every $s\in\cS$ and $a\in\cA$ and runs Algorithm \ref{alg:LCB} on $\cD$, the output policy $\widehat{\pi}_{\rm LCB}$ must suffer suboptimality at least
    \begin{equation*}
        \EE_\cD[\SubOpt(\widehat{\pi}_{\rm LCB})] = \frac{c_0\sigma}{n^{1/4}},
    \end{equation*}
    where $c_0$ is a universal constant.}
\end{corollary}

\subsection{Upper Bound with Knowledge of Noise Distribution}

To compare with the negative results for human rating under partial coverage (Assumption \ref{assumption:Cstar}), we prove an upper bound on the suboptimality of the LCB algorithm (Algorithm \ref{alg:LCB}) in the most benign case that the learner has full knowledge of the uncertainty noise distribution of the rating model and design the LCB penalty $b_n$ accordingly. This assumes the learner is able to find the confidence interval with any $\delta$, which is equivalent to knowing the cumulative density function of the distribution and can be unrealistic for real human feedback data in practice. This upper bound result provides a more direct comparison with the preference-based approach and demonstrates how human bias can affect the suboptimality when the uncertainty noise can be coped with.

\begin{theorem}\label{thm:upper-bound}
    Suppose Assumption \ref{assumption:Cstar} holds. For any fixed constant $0 < \delta < 1$, if one runs Algorithm \ref{alg:LCB} with 
    \begin{equation*}
        b_m = c_b\sqrt{\frac{V^2_{R,\sigma}\log^q\frac{SA}{\delta}}{m}}
    \end{equation*}
    and an appropriately chosen universal constant $c_b$, as soon as $n > 8\log\frac{2SA}{\delta}/\bar{d}$, where $\bar{d} := \min_{(s,a)\in\cX}d(s,a)$ and $\cX := \{(s,a)\in\cS\times\cA ~:~ d(s,a)> 0\}$, with probability $1-\delta$, the suboptimality of the output policy $\widehat{\pi}_{\rm LCB}$ satisfies
    \begin{equation*}
        \SubOpt(\widehat{\pi}_{\rm LCB}) \le c_0\sum_{(s,a)\in\cX}d_\rho^{\pi^\star}(s,a)\cdot \bar{h}^{-1}\left(\sqrt{\frac{V_{R,\sigma}^2\log^q\frac{SA}{\delta}}{n\cdot d(s,a)}}\right),
    \end{equation*}
    where $c_0$ is a constant that depends on $q$.
\end{theorem}

This theorem shows that even when the algorithm has full knowledge of the human uncertainty in the rating model, human bias can still influence the suboptimality of $\widehat{\pi}_{\rm LCB}$ negatively. It demonstrates the effect of bias on the suboptimality is truly unavoidable when using human rating directly. This can be further illustrated with our example model \eqref{eq:h-example} as follows, which shows the suboptimality still decays more slowly because of the quadratic human bias.

\begin{corollary}\label{cor:upper-bound}
    Suppose Assumption \ref{assumption:Cstar} holds. For any fixed constant $0 < \delta < 1$, if one runs Algorithm \ref{alg:LCB} with 
    \begin{equation*}
        b_m = \sqrt{\frac{64\sigma^4\log\frac{SA}{\delta}}{m}} + \frac{8\sigma^2\log\frac{SA}{\delta}}{m},
    \end{equation*}
    as soon as $n > 8\log\frac{2SA}{\delta}/\bar{d}$, where $\bar{d} := \min_{(s,a)\in\cX}d(s,a)$ and $\cX := \{(s,a)\in\cS\times\cA ~:~ d(s,a)> 0\}$, with probability $1-\delta$, the suboptimality of the output policy $\widehat{\pi}_{\rm LCB}$ satisfies
    \begin{equation*}
        \SubOpt(\widehat{\pi}_{\rm LCB}) \le 2\sum_{(s,a)\in\cX}d_\rho^{\pi^\star}(s,a)\left(\frac{256\sigma^4\log\frac{SA}{\delta}}{n\cdot d(s,a)}\right)^{1/4} + d_\rho^{\pi^\star}(s,a)\sqrt{\frac{32\sigma^2\log\frac{SA}{\delta}}{n\cdot d(s,a)}}.
    \end{equation*}
\end{corollary}

\section{Comparison with Preference-based Methods}

In contrast to rating, the preference-based approach relies on models that characterize how a human annotator would rank a group of subjects by reward. In this case, the feedback is simply the most preferred subject to the human annotator within the group. Such feedback actually contains less information than rating. Preference data are also incompatible with standard bandit algorithms and require special adaptation to use \citep{wang2023rlhf}. However, the preference-based approach has received much attention recently because some have found it easier and more accurate for human to make preferences than rating \citep{novikova2018rankme,Tarlow21Reliable,yannakakis2015ratings}. In this section, we compare the human rating approach with the preference-based approach. 

\subsection{Human Preference under BTL}

Let us consider the most basic case of human preference called pairwise comparison, which involves the ranking between a pair of state-action pairs based on their rewards. This is predominantly modeled with the Bradley-Terry-Luce (BTL) model \citep{Bradley1952RankAO}, under which a human annotator gives a binary response $y = \{0,1\}$ following a Bernoulli distribution when asked to compare two state-action pairs $(s,a^0)$ and $(s,a^1)$ with $a^0 \neq a^1$:
\begin{equation}\label{eq:BTL}
    P(y|s,a,a') = \frac{\exp(r(s,a^y))}{\exp(r(s,a^0))+\exp(r(s,a^1))}.
\end{equation} 

Like our rating model in \eqref{eq:rating-h-model}, the BTL model admits a consistent statistical problem. The learner is given a dataset $\cD' = \{(s_i,a_i^0,a_i^1,y_i)\}_{i=1}^{n}$, which contains i.i.d. human preference samples from some sampling distribution. $y_i$ is the binary human preference feedback for the comparison between $(s_i,a_i^0)$ and $(s_i,a_i^1)$. We denote the sampling probability of the state-action-action triplet $(s,a^0,a^1)$ with $d(s,a^0,a^1)$.

\begin{algorithm}[t]
    \caption{Pessimistic MLE for contextual bandits} \label{alg:LCB-comp}
    \begin{algorithmic}[1]
        \STATE \textbf{Input:} Offline dataset $\cD'$, confidence level $\delta\in(0,1)$.

        \STATE Construct the reward function set
        \begin{equation*}
            \cF := \{v\in\RR^{SA} :\mathbf{1}^\top v = 0, \norm{v}_\infty \le R\};
        \end{equation*}
    
        \STATE Set 
        \begin{equation*}
            \widetilde{r} = \arg\max_{f\in\cF}\sum_{i=1}^{n}\log\left(\frac{\ind\{y_i=1\}\exp(f(s_i,a_{i}^{1}))}{\exp(f(s_i,a_{i}^{0})) + \exp(f(s_i,a_{i}^{1}))} + \frac{\ind\{y_i=0\}\exp(f(s_i,a_{i}^{0}))}{\exp(f(s_i,a_{i}^{0})) + \exp(f(s_i,a_{i}^{1}))}\right);
        \end{equation*}

        \STATE Construct empirical covariance matrix
        \begin{equation*}
            \widehat{\Sigma} = \frac{1}{n}\sum_{i=1}^n \left(\mathbf{1}_{(s_i,a_{i}^{0})} - \mathbf{1}_{(s_i,a_{i}^{1})}\right)\left(\mathbf{1}_{(s_i,a_{i}^{0})} - \mathbf{1}_{(s_i,a_{i}^{1})}\right)^\top;
        \end{equation*}

        \STATE Construct the pessimistic reward function set
        \begin{equation*}
            \cF_{\mathrm{CR}}(\widetilde{r}) = \left\{f\in \cF ~:~ \sqrt{(f - \widetilde{r})^\top \widehat{\Sigma}(f - \widetilde{r})} \le b_n'\right\};
        \end{equation*}
        
        \RETURN $\widehat{\pi}_{\rm PMLE} = \arg\max_{\pi}\min_{\widehat{r}\in\cF_{\mathrm{CR}}(\widetilde{r})}\EE_{s\sim\rho}[\widehat{r}(s,\pi(s))]$. \label{alg-line:comp-r-hat}
    \end{algorithmic}
\end{algorithm}

To find the optimal policy with human preference data, we can use pessimistic MLE \citep{zhu2023principled}, which first computes a reward function by MLE and then outputs the optimal policy corresponding to a pessimistic version of this MLE reward (Algorithm \ref{alg:LCB-comp}). The data coverage assumption is similar to Assumption \ref{assumption:Cstar}, which essentially requires the sampling distribution to covers the state-actions pairs that optimal policy can reach. In the tabular case, this assumption can be written as follows:

\begin{assumption}\label{assumption:Cstar-comp}
    \textit{There exists an optimal policy $\pi^\star$ such that the pairwise concentrability coefficient 
    \begin{equation}\label{eq:C-dagger}
        C^{\dagger} := \sqrt{\sup_{v\in[-1,1]^{SA}:\mathbf{1}^\top v = 0}\frac{\Big(\sum_{(s,a)} d_\rho^{\pi^\star}(s,a) v(s,a)\Big)^2}{\sum_{(s,a^0,a^1)} d(s,a^0, a^1) \big(v(s,a^0) - v(s,a^1)\big)^2}}
    \end{equation}
    is bounded.}  
\end{assumption}

\cite{zhu2023principled} proved the convergence of pessimistic MLE in the linear bandit setting. The following theorem is a special case of Theorem 3.2 from \cite{zhu2023principled} with some modification, which expresses everything in the tabular setting. This shows when we assume human preference follows the BTL model, pessimistic MLE can provably converge to the optimal policy under the mild data coverage assumption of Assumption \ref{assumption:Cstar-comp} and its suboptimality decays at a fast rate of $O(1/\sqrt{n})$. This result marks a clear distinction from the negative results for human rating.

\begin{theorem}\label{thm:comp}
    \textit{Denote $\gamma = \frac{1}{2+\exp(R\sqrt{SA})+\exp(-R\sqrt{SA})}$. Suppose Assumption \ref{assumption:Cstar-comp} holds. For any fixed constant $0 < \delta < 1$, if one runs Algorithm \ref{alg:LCB-comp} with 
    \begin{equation*}
        b_m' = c_b'\sqrt{\frac{SA + \log\frac{1}{\delta}}{\gamma^2 m}},
    \end{equation*}
    where $c_b'$ is an appropriately chosen universal constant, with probability $1-\delta$, the suboptimality of the output policy $\widehat{\pi}_{\rm PMLE}$ satisfies
    \begin{equation*}
        \SubOpt(\widehat{\pi}_{\rm PMLE}) \le c_0C^\dagger R\left(\sqrt{\frac{SA + \log\frac{1}{\delta}}{\gamma^2 n}} + \sqrt{\frac{S^2A^2\log\frac{n}{\delta}}{n}}\right),
    \end{equation*}
    where $c_0$ is a universal constant.}
\end{theorem}

We can compare the suboptimality in this theorem with the results for the rating-based approach. A comparison with Theorem \ref{thm:Cstar} shows the uncertainty in human ratings may require the data to have stronger coverage in order to converge to the optimal policy. A comparison with Theorem \ref{thm:uniform} shows when the bias in human ratings distorts the reward function and makes it more extreme and drastic (less smooth in the Lipschitz sense), the $\bar{h}^{-1}(\cdot)$ can slow down the suboptimality's decay with respect to the sample size. In fact, we can observe that the preference-based approach enjoys faster suboptimality decay because preference feedback contains no bias and mild uncertainty noise according to the BTL model. While such modeling is justified by empirical evidences, it makes one wonder whether the advantage of preference-based methods mostly comes from the modeling aspect. To delve into this further, let us make another theoretical analysis for the case when preference data are affected by human bias.

\subsection{Human Preference under Biased BTL}

Let us introduce a new model for human preference called the biased BTL model. This model considers the case when human preferences are also subject to bias just like the rating model \eqref{eq:rating-h-model} and the feedback is generated with respect to the biased reward. In particular, the binary feedback $\widetilde{y} = \{0,1\}$ for $(s,a^0)$ and $(s,a^1)$ follows: 
\begin{equation}\label{eq:BTL-biased}
    P(\widetilde{y}|s,a,a') = \frac{\exp(\bar{h}(r(s,a^{\widetilde{y}})))}{\exp(\bar{h}(r(s,a^0)))+\exp(\bar{h}(r(s,a^1)))},
\end{equation} 
where $\bar{h}$ is the expected bias function from \eqref{eq:rating-h-model}.

We consider the performance of pessimistic MLE (Algorithm \ref{alg:LCB-comp}) again with human preference data generated under this model. While the data are generated under human bias this time, we still run pessimistic MLE on the new data as before. Different from the suboptimality results in the previous section, we focus on the sample complexity for learning the optimal policy. We take a gap-dependent approach in our analysis to consider the case when human bias closes the biased optimality gap $r(s,\pi^\star(s)) - r(s,a)$ and the actual optimality gap $\bar{h}(r(s,\pi^\star(s))) - \bar{h}(r(s,a))$ remains big, where $a$ is the second best action at $s$. This echoes with the type of undesirable bias we considered in the last comparison, which is true when human annotators have more extreme standards at heart. In a simple bandit instance, we can obtain the following result and notice the samples needed to find the optimal policy with the preference-based approach is no less than the samples needed for the rating-based approach.

\begin{theorem}\label{thm:both-bias}
    \textit{Consider any single-state bandit instance with $\cA=\{a_1,a_2\}$ and $0 \le \bar{h}(r(a_1)) < \bar{h}(r(a_2)) \le 1$. For any fixed constant $0 < \delta < 1$, let $n_{\text{rate}}$ be the total number of samples needed to learn the optimal action with probability at least $1-\delta$ in the human rating setting under observation model \eqref{eq:rating-h-model} with additive sub-gaussian uncertainty noise and uniform data coverage $n_{a_1} = n_{a_2}$, and let $n_{\text{pref}}$ be the number of samples needed to learn the optimal action with probability at least $1-\delta$ in the human preference setting with observation model \eqref{eq:BTL-biased}. It always holds that
    \begin{equation}
        \frac{n_{\text{rate}}}{n_{\text{pref}}} < 0.25\sigma^2. 
    \end{equation}}
\end{theorem}

We can see that when the variance proxy of the uncertainty noise $\sigma^2$ is no larger than $4$ in human rating (the expected reward is bounded in $[0,1]$), the samples needed in the rating-based approach is always fewer than the preference-based approach. This shows if one assumes a similar amount of human bias and uncertainty in both types of human feedback, the preference-based approach is no more sample-efficient. This actually contradicts with the empirical observations in the existing literature, which suggests preference-based methods have superior performance. Hence, our theory shows the bias-free modeling plays a great role in the lower sample complexity of preference-based methods, and our theoretical results can conversely confirm the standard BTL modeling of human preference feedback---it is reasonable to believe human preference data is indeed subject to less bias and uncertainty in practice. 


%\corruptparity*

\begin{proof}[Proof of Proposition~\ref{prop:corruptparity}]\label{proof:corruptparity}
We want to bound the change in the proportion of positive labels assigned by $h$ when we move from the original distribution $\mathcal{D}$ to the corrupted distribution $\widetilde{\mathcal{D}}$. For a fixed group $A$, we can express the proportion of positive labels assigned by $h$ in $\widetilde{\mathcal{D}}$ in terms of the proportion of positive labels assigned by $h$ in $\mathcal{D}$ as follows:

\begin{equation}
    P_{(x,y) \sim \DAC} [h(x)=1] = \frac{(1-\alpha) P_{(x,y) \sim \DA} [h(x)=1] \cdot \RA + E_A}{(1-\alpha) \RA + \alpha_A}
\end{equation}

where $\alpha_A$ is the proportion of the data set that is corrupted and in group $A$ and $E_A$ is the proportion of the data set that is corrupted, in group $A$ and positively labeled by $h$.

Our goal is to obtain an upper bound on the difference between $P_{(x,y) \sim \DAC} [h(x)=1]$ and $P_{(x,y) \sim \DA} [h(x)=1]$.
We use the fact that $E_A \leq \alpha$ and $\alpha_A \leq \alpha$ to obtain the following upper bound:

\begin{equation}
    \left| P_{(x,y) \sim \DAC} [h(x)=1] - P_{(x,y) \sim \DA} [h(x)=1] \right| = \left| \frac{E_A - \alpha_A P_{(x,y) \sim \DA} [h(x)=1] }{(1-\alpha) \RA + \alpha_A} \right| \leq \frac{\alpha}{(1-\alpha) r_A + \alpha  }
\end{equation}
\end{proof}

\mainparity*

\begin{proof}[Proof of Theorem~\ref{thm:mainparity}]\label{proof:mainparity}
For $z \in \{A, B \}$, let $\normalF_z(h)$ and $\corruptF_z(h)$ denote the proportions of positive labels assigned by $h$ in group $z$ in the original and corrupted distributions respectively. That is, for group $A$, $\normalF_A(h) = P_{(x,y) \sim \DA} [ h (x)=1]$ and $\corruptF_A(h) = P_{(x,y) \sim \DAC} [ h (x)=1]$.
    It suffices to show that there exists $h \in \closure$ that satisfies the guarantees above. 
    % \pcocomment{Might need to add a lemma before this where we show that this is sufficient.}
    Consider $\hstar \in \hclass$. By the realizability assumption 
    % \pcocomment{there are two realizability assumptions here, one where $h^*$ satisfies the violation up to $\delta$ and one where it's exact equality. We'll use the one assuming equality and we'll add a lemma showing that things work fine for the $\delta$ violation one.}
    , $\hstar$ satisfies the parity constraint i.e $\normalF_A(h^*) = \normalF_B(h^*)$. 
    After the corruption, the parity violation of $h^*$, $|\corruptF_A(h^*) - \corruptF_B(h^*)|$ may increase. Now we define the following parameters ($p_z$ and $q_z$) for $z \in \{A, B \}$.
    \begin{equation}
        p_z = \begin{cases}
            \frac{\normalF_z(h^*) - \corruptF_z(h^*)}{1 - \corruptF_z(h^*)} & \text{if} \ \normalF_z(h^*) \geq \corruptF_z(h^*)\\
            \frac{\corruptF_z(h^*) - \normalF_z(h^*)}{\corruptF_z(h^*)} & \text{otherwise}\\
        \end{cases} \quad
        q_z = \begin{cases}
            1 & \text{if} \ \normalF_z(h^*) \geq \corruptF_z(h^*)\\
            0 & \text{otherwise}\\
        \end{cases}
    \end{equation}
    Now consider a hypothesis $\hhat$ that behaves as follows: Given a sample $x$:
    \begin{itemize}
        \item  If $x \in A$, with probability $p_A$, return label $q_A$. Otherwise return $h^* (x)$
        \item Similarly, if $x \in B$, with probability $p_B$, return label $q_B$. Otherwise return $h^* (x)$
    \end{itemize}
    $\hhat \in \PQ$ since it follows the definition of our closure model. We will now show that $\hhat$ satisfies the parity constraint in the corrupted distribution (i.e $\corruptF_A(\hhat) = \corruptF_B(\hhat)$). First, observe that for $z \in \{A, B \} $, if $\normalF_z(h^*) \geq \corruptF_z(h^*)$, then $\corruptF_z(\hhat) = \normalF_z(h^*)$. This is because
    \begin{align*}
        \corruptF_z(\hhat) 
        &= (1 - p_z) \corruptF_z(h^*) + p_z q_z \\
        &= \corruptF_z(h^*) + p_z(1 - \corruptF_z(h^*)) \\
        &= \corruptF_z(h^*) + \normalF_z(h^*) - \corruptF_z(h^*) \\
        &= \normalF_z(h^*)
    \end{align*}
    Similarly, if $\normalF_z(h^*) < \corruptF_z(h^*)$, then $\corruptF_z(\hhat) = \normalF_z(h^*)$. This is because
    \begin{align*}
        \corruptF_z(\hhat) 
        &= (1 - p_z) \corruptF_z(h^*) + p_z q_z \\
        &= \corruptF_z(h^*) + p_z(0 - \corruptF_z(h^*)) \\
        &= \corruptF_z(h^*) + \normalF_z(h^*) - \corruptF_z(h^*) \\
        &= \normalF_z(h^*)
    \end{align*}
    Thus, $\corruptF_A(\hhat) = \normalF_A(h^*) = \normalF_B(h^*) = \corruptF_B(\hhat)$. Therefore $\hhat$ satisfies the parity constraint in the corrupted distribution.
    
    We will now show that $\error{\hhat} \leq O(\alpha) $. Since $\hhat$ deviates from $\hstar$ with probability $p_A$ on samples from $A$, and with probability $p_B$ on samples from $B$, we only need to show that the proportion of samples such that $\hhat (x) \neq h^* (x)$ is small. Fix a group $z \in \{A, B\}$. If $\normalF_z (h^*) \geq \corruptF_z (h^*)$, then with probability $p_z = \frac{\normalF_z (h^*) -\corruptF_z (h^*)}{1 - \corruptF_z (h^*)}$, $\hhat$ returns a positive label for samples in group $z$. Thus, the expected proportion of samples in group $z$ such that $\hhat (x) \neq h^* (x)$ is $p_z$ times the proportion of negative labelled samples (by $h^*$) in group $z$ (since those get flipped to positive).
    \begin{align*}
        \EE_{x \in z} [\one (\hhat (x) \neq h^* (x))] &= p_z \cdot P_{(x,y) \sim \dist } [x \in z] (1 - \corruptF_z (h^*) ) \\
        &= \frac{\normalF_z (h^*) -\corruptF_z (h^*)}{1 - \corruptF_z (h^*)} \cdot P_{(x,y) \sim \dist } [x \in z] (1 - \corruptF_z (h^*) ) \\
        &= (\normalF_z (h^*) -\corruptF_z (h^*)) \cdot P_{(x,y) \sim \dist } [x \in z] 
    \end{align*}
    Similarly, if $\corruptF_z (h^*) > \normalF_z (h^*)$, then with probability $p_z = \frac{\corruptF_z (h^*) -\normalF_z (h^*)}{\corruptF_z (h^*)}$, $\hhat$ returns a negative label. Thus, the expected proportion of samples in group $z$ such that $\hhat (x) \neq h^* (x)$ is $p_z$ times the proportion of positively labelled samples (by $h^*$) in group $z$ (since those get flipped to negative).
    \begin{align*}
        \EE_{x \in z} [\one(\hhat (x) \neq h^* (x))] &= p_z \cdot P_{(x,y) \sim \dist } [x \in z] \cdot \corruptF_z (h^*) \\
        &= \frac{\corruptF_z (h^*) -\normalF_z (h^*)}{\corruptF_z (h^*)} \cdot P_{(x,y) \sim \dist } [x \in z] \cdot \corruptF_z (h^*)  \\
        &= (\corruptF_z (h^*) -\normalF_z (h^*)) \cdot P_{(x,y) \sim \dist } [x \in z] 
    \end{align*}
    Therefore, the expected total number of samples such that $\hhat (x) \neq h^* (x)$ across the entire distribution is bounded as follows:
    \begin{align*}
        \mathbb{E}_{(x,y) \sim \mathcal{D}} \ [\one (\hhat (x) \neq h^* (x))] 
        &= \sum_{z \in \{ A, B \}} |\corruptF_z (h^*) -\normalF_z (h^*)| \cdot P_{(x,y) \sim \dist } [x \in z] \\ 
        &\leq \sum_{z \in \{ A, B \}} \frac{\alpha}{(1- \alpha) P_{(x,y) \sim \dist } [x \in z] + \alpha} \cdot P_{(x,y) \sim \dist } [x \in z] \\ \intertext{by proposition \ref{prop:corruptparity}} 
        &\leq \frac{2\alpha}{(1- \alpha)}
    \end{align*}
    Note that even though the adversary can choose a different distribution at each timestep, we can wlog assume the adversary chooses the same distribution $\widetilde{D}$ where the quantity $|\corruptF_z (h^*) -\normalF_z (h^*)|$ is maximized at every timestep, as in Proposition \ref{prop:corruptparity}.
    Although the model in \cite{kearns1988learning} is slightly weaker than \cite{lampert}, this theorem holds in full generality for both models where we replace the difference $|\corruptF_z (h^*) -\normalF_z (h^*)|$ with the bounds from Lemma 2 of \cite{lampert}. The dependence on $\alpha$ remains the same in both cases.
    % \pcocomment{Because of the way I define $z$, the entire proof/construction should work for any number of groups but we would have $n \times \alpha$ in the numerator for accuracy loss. I wonder if that's avoidable} \pcocomment{Update: I think it's avoidable. The bound in proposition 1 could be improved to make the adversary's changes across all groups sum up to $\alpha$}
\end{proof}

%\subsection{Equal Opportunity}

% \begin{theorem}\label{thm:regimesEopp}
% There are three regimes depending on the size of $|B|$ ( where group $B$ is the smaller group) and we typically think of it as disadvantaged.


% Consider an initial case where $TPR_{A}=TPR_{B}=100 \%$ (think this follow for the other directions. 
% In each of

% To wit, the regimes are 
% \begin{enumerate}
%     \item $|B| = O( \alpha)$. Excess accuracy loss is $O(\alpha)$. Observe that the learner can simply exhibit the all positive classifier on Group $B$ and thus still satisfy Equal Opportunity. 
%      \item $|A|=|B|$. Excess Accuracy loss is still $O(\alpha)$. The adversary can only push the TPR rate of Group $B$ down by at most $O(\alpha)$. Then we can do a randomized closure of $h_{A}^{*}$ (e.g. if $h_{A}^{*}=1$, report $-1$ with probability $\alpha$. 
%     \item if $|B| \sim \sqrt{\alpha}$ the accuracy loss is $O(\sqrt{\alpha})$ [Hardest case]
% \end{enumerate}
% \end{theorem}

\begin{comment}
\begin{align*}
    & Error(Closure) = \min \{ \frac{\alpha}{r_{B}} (1-r_B), r_B \} \\
    & LHS = \text{ERROR when rejecting some of the true positives in Group A}  \\
    & p_{A}= P(y=1 \cap x \in A), p_{B}= P(y=1 \cap x \in B)  \\
    & r_{B} = P(x \in B) \\
    & p_{B} \in [ r_{B}^{+},1], r_{B}^{+}>0, \text{ imagine like $25 \%$} \\
    & RHS =\text{error when accepting everyone in Group B} \\
    &\frac{\alpha}{r_{B}^{+}}  r_A p_{A}= r_B (1-p_{B}) \\
      &\frac{\alpha}{r_{B}} r_A p_{A}= r_B (1-p_{B}) \\
    &\frac{\alpha}{r_{B}} (1-r_B) p_{A}= r_B (1-p_{B}) \\
    & \frac{\alpha}{r_{B}} (1-r_B) p_{A}= r_B (1-p_{B})  \quad \quad \text{dropping base rates} \\
    & \text{ approx equality for } r_B = \sqrt{\alpha} \\
    & \frac{\alpha}{\sqrt{\alpha}} (1-\sqrt{\alpha}) \sim \sqrt{\alpha} = r_B
\end{align*}
\end{comment}

% \begin{proposition}[TPR after corruption]\label{prop:fixedhypo}
%     Let $\widetilde{\mathcal{D}}$ be the corrupted distribution, and $h$ be a fixed hypothesis in $\mathcal{H}$. For a fixed group $A$, the following inequality bounds the change in True Positive Rate of $h$:
%     \begin{equation}
%         \left| \text{TPR}_A(h, \widetilde{\mathcal{D}}) - \text{TPR}_A(h, \dist) \right| \leq \frac{\alpha}{(1-\alpha) r_A^+ + \alpha }
%     \end{equation}
%     where $\text{TPR}_A(h, \dist) = P_{(x,y) \sim \DA} [h(x)=1 | y = 1]$ and $r_A^+ = P_{(x,y) \sim \DA} [y = 1]$
% \end{proposition}


\corrupttpr*

\begin{proof}[Proof of Proposition~\ref{prop:corrupttpr}]\label{proof:corrupttpr}
For a fixed group $A$, the TPR of $h$ in $\widetilde{\mathcal{D}}$ can be expressed in terms of the TPR of $h$ in the original distribution $\mathcal{D}$ as follows:
\begin{equation}
        \text{TPR}_A(h, \widetilde{\mathcal{D}}) = \frac{(1-\alpha) \text{TPR}_A(h, \mathcal{D}) \cdot \RA + E_A^+}{(1-\alpha) \RA + \alpha_A^+}
\end{equation}
where $\alpha_A$ is the proportion of the data set that is corrupted and in group $A$ and $E_A^+$ is the proportion of the data set that is corrupted, in group $A$, positively labeled and . Thus,
\begin{equation}
    \left| \text{TPR}_A(h, \widetilde{\mathcal{D}}) - \text{TPR}_A(h, \mathcal{D}) \right| = \left| \frac{E_A - \alpha_A \text{TPR}_A(h, \mathcal{D}) }{(1-\alpha) \RA + \alpha_A} \right| \leq \frac{\alpha}{(1-\alpha) r_A^+ + \alpha }
\end{equation}
since $E_A \leq \alpha$ and $\alpha_A \leq \alpha$
\end{proof}

% \begin{theorem}
%     For any hypothesis class $\mathcal{H}$ and distribution $ \dist = (\DA, \DB)$, a robust fair-ERM learner for the equal opportunity constraint in the Malicious Adversarial Model returns a hypothesis $\hhat$ such that 
%     \begin{equation*}
%         \error{\hhat} \leq O(\sqrt{\alpha})
%     \end{equation*}
% \end{theorem}

\maineopp*

\begin{proof} [Proof of Theorem~\ref{thm:maineopp}]\label{proof:maineopp}
It suffices to show that there exists $h \in \PQ$ that satisfies the guarantees above. 
Consider $\hstar \in \hclass$. By the realizability assumption, $\hstar$ satisfies the equal opportunity constraint i.e $\text{TPR}_A(h^*, \mathcal{D}) = \text{TPR}_B(h^*, \mathcal{D})$. 
After the corruption, the equal opportunity violation of $h^*$, $|\text{TPR}_A(h^*, \widetilde{\mathcal{D}}) - \text{TPR}_B(h^*, \widetilde{\mathcal{D}})|$ may increase. Now we define the following parameters ($p_z^i$ and $q_z^i$) for $i, z \in \{A, B \}$. 
\begin{equation}\label{eq:tpr-prob}
    p_z^i = \begin{cases}
        \frac{\corruptF_i(h^*) - \corruptF_z(h^*)}{1 - \corruptF_z(h^*)} & \text{if} \ \corruptF_i(h^*) \geq \corruptF_z(h^*)\\
        \frac{\corruptF_z(h^*) - \corruptF_i(h^*)}{\corruptF_z(h^*)} & \text{otherwise}\\
    \end{cases} \quad
    q_z^i = \begin{cases}
        1 & \text{if} \ \corruptF_i(h^*) \geq \corruptF_z(h^*)\\
        0 & \text{otherwise}\\
    \end{cases}
\end{equation}
One can think of the parameter $p_z^i$ as the proportion of samples in group $z$ whose outcomes needs to be changed in order to match the true positivity rate of group $i$,
\pcocomment{The parameters should be clipped so that they are in the $[0,1]$ interval. will fix later} 
    Now consider two hypothesis $\hhat_i$ for $i \in \{ A, B\}$ that behaves as follows: Given a sample $x$:
    \begin{itemize}
        \item  If $x \in A$, with probability $p_A^i$, return label $q_A^i$. Otherwise return $h^* (x)$
        \item Similarly, if $x \in B$, with probability $p_B^i$, return label $q_B^i$. Otherwise return $h^* (x)$
    \end{itemize}
One can think of $\hhat_i$ as a hypothesis that deviates from $h^*$ on every other group to make their true positive rate on the corrupted distribution match that of group $i$.
$\hhat_i \in \PQ$ for $i \in \{A, B\}$ since it follows the definition of our closure model. We will now show that $\hhat_i$ for $i \in \{ A, B\}$ satisfies the true positivity rate constraint in the corrupted distribution (i.e $\corruptF_A(\hhat_i) = \corruptF_B(\hhat_i)$ for fixed $i \in \{ A, B\}$). First, observe that for $z \in \{A, B \} $, if $\corruptF_i(h^*) \geq \corruptF_z(h^*)$, then $\corruptF_z(\hhat_i) = \corruptF_i(h^*)$. This is because
    \begin{align*}
        \corruptF_z(\hhat_i) 
        &= (1 - p_z) \corruptF_z(h^*) + p_z q_z \\
        &= \corruptF_z(h^*) + p_z(1 - \corruptF_z(h^*)) \\
        &= \corruptF_z(h^*) + \corruptF_i(h^*) - \corruptF_z(h^*) \\
        &= \corruptF_i(h^*)
    \end{align*}
    Similarly, if $\corruptF_i(h^*) < \corruptF_z(h^*)$, then $\corruptF_z(\hhat) = \corruptF_i(h^*)$. This is because
    \begin{align*}
        \corruptF_z(\hhat) 
        &= (1 - p_z) \corruptF_z(h^*) + p_z q_z \\
        &= \corruptF_z(h^*) + p_z(0 - \corruptF_z(h^*)) \\
        &= \corruptF_z(h^*) + \corruptF_i(h^*) - \corruptF_z(h^*) \\
        &= \corruptF_i(h^*)
    \end{align*}
    Thus, $\corruptF_A(\hhat_i) = \corruptF_i(h^*) = \corruptF_B(\hhat_i)$. Therefore $\hhat_i$ for $i \in \{ A, B\}$ satisfies the equal opportunity constraint in the corrupted distribution.
    
    We will now show that for at least one $\hhat_i$ for $i \in \{A, B\}$ satisfies $\error{\hhat} \leq O(\sqrt{\alpha})$. Since $\hhat_i$ deviates from $\hstar$ with probability $p_A^i$ on samples from $A$, and with probability $p_B^i$ on samples from $B$, it suffices to show that $p_A^i \cdot r_A + p_B^i \cdot r_B$ is $O(\sqrt{\alpha})$ for one some $i \in \{A, B\}$. 
    We consider the following cases:
\begin{enumerate}
    \item Suppose wlog $r_B \leq \frac{\sqrt{\alpha}}{1 - \sqrt{\alpha}}$. Then $\hat{h}_{B}$ satisfies the guarantee. This is because $p_B^B = 0$ (by equation~\ref{eq:tpr-prob} ) and $p_A^B \leq 1$. Thus, $p_A^B \cdot r_A + p_B^B \cdot r_B$ is $O(\sqrt{\alpha})$

    \item If instead $\min (r_A, r_B) > \frac{\sqrt{\alpha}}{1 - \sqrt{\alpha}}$. wlog let $B$ be a group with the highest true positive rate greater than 0.5 or the smallest true positive rate less than 0.5. At least one group must satisfy this constraint. If $B$ has the highest true positive rate greater than 0.5, then 
    \begin{align*}
    p_B^A &= \frac{\corruptF_B (h^*) - \corruptF_A (h^*)}{\corruptF_B (h^*)} \\
    &\leq \frac{\corruptF_B (h^*) - \corruptF_B (h^*) + \corruptF_A (h^*) - \corruptF_A (h^*)}{0.5} \intertext{since $\corruptF_B (h^*) \geq 0.5$ and by realizability assumption $\normalF_B (h^*) = \normalF_A (h^*)$}
    &\leq 2 |\corruptF_B (h^*) - \corruptF_B (h^*)| + 2 |\normalF_A (h^*) - \corruptF_A (h^*)| \\ \intertext{by proposition~\ref{prop:corrupttpr}}
    &\leq O (\sqrt{\alpha})
    \end{align*} 
    Thus, $p_A^A \cdot r_A + p_B^A \cdot r_B$ is at most $O(\sqrt{\alpha})$
    The case where $B$ has the smallest true positive rate follows similarly.
\end{enumerate}
The argument above can be extended to derive a bound of $\sqrt{(n-1)\alpha}$ in the general case of $n$ groups. It is important to note that this bound has a different dependency on the number of groups compared to the parity case, where there is no dependence on $n$.
% To extend, consider all the groups with size greater than alpha, then transform every group's tpr to that of the smallest tpr greater than 0.5 or largest less than 0.5
    
% Fix a group $i \in \{A, B\}$ and let $j$ be the other group. If $\corruptF_i (h^*) \geq \corruptF_j (h^*)$, then with probability $p_j = \frac{\normalF_i (h^*) -\corruptF_j (h^*)}{1 - \corruptF_j (h^*)}$, $\hhat_i$ returns a positive label for samples in group $j$ and samples in group $i$ remain unchanged. Thus, 
% \begin{align*}
%     \EE [\mathbbm{1} (\hhat_i (x) \neq h^* (x))] &\leq p_A^i \cdot r_A + p_B^i \cdot r_B \\
%     &\leq p_j^i \cdot P_{(x,y) \sim \dist } [x \in j] \\
%     &= \frac{\corruptF_i (h^*) -\corruptF_j (h^*)}{1 - \corruptF_j (h^*)} \cdot P_{(x,y) \sim \dist } [x \in j]
% \end{align*}
% Similarly, if $\corruptF_j (h^*) > \corruptF_i (h^*)$, then with probability $p_j = \frac{\corruptF_j (h^*) -\corruptF_i (h^*)}{\corruptF_j (h^*)}$, $\hhat$ returns a negative label and samples in group $i$ remain unchanged.
% \begin{align*}
%     \EE_ [\mathbbm{1}(\hhat (x) \neq h^* (x))] 
%     &\leq p_j^i \cdot P_{(x,y) \sim \dist } [x \in j] \cdot \corruptF_z (h^*) \\
%     &= \frac{\corruptF_j (h^*) -\corruptF_i (h^*)}{\corruptF_j (h^*)} \cdot P_{(x,y) \sim \dist } [x \in j] 
% \end{align*}
% Observe that the term $|\corruptF_i (h^*) -\corruptF_j (h^*)|$ can be bounded as follows:
% \begin{align*}
%     |\corruptF_i (h^*) -\corruptF_j (h^*)|
%     &= |\corruptF_i (h^*) - \normalF_i (h^*) + \normalF_j (h^*) - \corruptF_j (h^*)| \\ \intertext{ since $\normalF_i (h^*) = \normalF_j (h^*)$}
%     &= |\corruptF_i (h^*) - \normalF_i (h^*)| + |\normalF_j (h^*) - \corruptF_j (h^*)| \\
%     &= \frac{\alpha}{(1- }
% \end{align*}
% Therefore, the expected total number of samples such that $\hhat (x) \neq h^* (x)$ across the entire distribution is bounded as follows:
% \begin{align*}
%     \mathbb{E}_{(x,y) \sim \mathcal{D}} \ [\mathbbm{1} (\hhat (x) \neq h^* (x))] 
%     &= \sum_{z \in \{ A, B \}} |\corruptF_z (h^*) -\normalF_z (h^*)| \cdot P_{(x,y) \sim \dist } [x \in z] \\ 
%     &\leq \sum_{z \in \{ A, B \}} \frac{\alpha}{(1- \alpha) P_{(x,y) \sim \dist } [x \in z] + \alpha} \cdot P_{(x,y) \sim \dist } [x \in z] \\ \intertext{by proposition \ref{prop:fixedh-parity}}
%     &\leq 
% \end{align*}
\end{proof}

%\lowereopp*
%\begin{proof} [Proof of Theorem~\ref{thm:lowereopp}]\label{proof:lowereopp}
%Suppose group A is of size $1 - \sqrt{a}$ and B is of size $\sqrt{a}$. Suppose the best classifier in the hypothesis class can only attain $(1 - \sqrt{\alpha})\%$ percent TPR on both groups. An adversary with $\alpha\%$ can corrupt the distribution so that this classifier has $100 \%$ percent on corrupted distribution for group $B$. Fix a classifier $h$ returned by a learner in this setting. In order to satisfy the perceived fairness constraint of the ERM solver i.e tpr of $h$ must be the same for both groups in the corrupted distribution, then  

%\end{proof}

\lowereopp*

\begin{proof} [Proof of Theorem~\ref{thm:lowereopp}]\label{proof:lowereopp}
We will show a distribution and a malicious adversary of power $\alpha$ such that any hypothesis returned by a learner incurs at least $\sqrt{\alpha}$ expected error.
The distribution $\mathcal{D}$ will be such that $P_{x \sim \mathcal{D}}[x \in B]= \Omega(\sqrt{\alpha})$. This distribution will be supported on exactly four points $x_1 \in A, x_2 \in A, x_3 \in B, x_4 \in B$ with labels $y_1 = +, y_2 = -, y_3 = +, y_4 = -$. We also have that 
$$P_{x,y \sim \mathcal{D}}[x = x_1, y = +] = P_{x, y \sim \mathcal{D}}[x = x_2, y = -] = \frac{1 - \sqrt{\alpha}}{2}$$ 
and 
$$P_{x, y \sim \mathcal{D}}[x = x_3, y = +] = P_{x,y \sim \mathcal{D}}[x = x_4, y = -] = \frac{\sqrt{\alpha}}{2}$$
That is, each group has equal proportion of positives and negatives. 

The adversary commits to a poisoning strategy that places positive examples from Group $B$ into the negative region of the optimal classifier. That is, the adversary changes the original distribution $\mathcal{D}$ so that 
$$P_{x, y \sim \mathcal{D}}[x = x_1, y = +] = P_{x, y \sim \mathcal{D}}[x = x_2, y = -] = \frac{(1-\alpha)(1 - \sqrt{\alpha})}{2}$$
$$P_{x, y \sim \mathcal{D}}[x = x_3, y = +] =
P_{x, y \sim \mathcal{D}}[x = x_4, y = -] = \frac{(1-\alpha)\sqrt{\alpha}}{2}$$ and $P_{x, y \sim \mathcal{D}}[x = x_4, y = +] = \alpha$

We assume the perfect classifier is in the hypothesis class.
Now fix a classifier $h$ returned by a learner. This classifier must satisfy equal opportunity. Let $p_1, p_2, p_3, p_4$ be the probability that 
$h$ classifies $x_1, x_2, x_3, x_4$  as positive, respectively. 
Observe that $\widetilde{\text{TPR}}(h_A) = p_1$ and $\widetilde{\text{TPR}}(h_B) = 1 - (1 - p_4) \alpha' - (1-p_3)(1 - \alpha')$ where $\alpha' = \frac{2\sqrt{\alpha}}{(1 - \alpha) + 2\sqrt{\alpha}}$. The latter is due to the samples $(x_4, +)$ which the adversary added to the distribution. 
The adversary added an $\alpha$ amount which turned out to be an $\alpha'$ proportion of the positives in $B$. 
Since this classifier satisfies equal opportunity on the corrupted distribution, it must be the case that $p_1 = 1 - (1 - p_4) \alpha' - (1-p_3)(1 - \alpha')$. Thus, $(1 - p_1) \geq (1 - p_4) \alpha'$.
The error of $h$ on the original distribution is therefore
\begin{align*}
& (1 - p_1 + p_2) \frac{(1 - \sqrt{\alpha})}{2} + (1 - p_3 + p_4) \frac{\sqrt{\alpha}}{2} \\
\geq & \ (1 - p_1) \frac{(1 - \sqrt{\alpha})}{2} + p_4 \frac{\sqrt{\alpha}}{2} \\ \intertext{by the equal opportunity constraint}
\geq & \ (1 - p_4) \alpha' \frac{(1 - \sqrt{\alpha})}{2} + p_4 \frac{\sqrt{\alpha}}{2} \\
= & \ (1 - p_4) \cdot \frac{2\sqrt{\alpha}}{(1 - \alpha) + 2\sqrt{\alpha}} \cdot \frac{(1 - \sqrt{\alpha})}{2} + p_4 \frac{\sqrt{\alpha}}{2} \\
\geq & \ (1 - p_4) \frac{\sqrt{\alpha}}{2} + p_4 \frac{\sqrt{\alpha}}{2}  \geq \Omega (\sqrt{\alpha}) 
\end{align*}
% This decreases $TPR(h_{B}^{*})$.
% Informally we can think of the original distribution as consisting of four points, positive points from Group $A$, negative points from Group $A$,
% positive points from Group $B$, and negative points from Group $B$.
% The adversary places $\alpha$ positive points on top of the negative points from Group $B$. 

% In order to balance the True Positive Rates on the biased data, without loss of generality, we will need to correctly classify at least some of those malicious positive points in Group $B$ and  intentionally mis-classify some natural positives from Group $A$.


% Because of the corruption, an $\frac{\alpha}{\sqrt{\alpha} + \alpha} = \Omega(\sqrt{\alpha})$ of the positive points in Group $B$ are misclassified by $h^{*}$ and 
% we will need to modify $h^{*}_{B}$ to classify those points as positive. 

% %we need to balance the True Positive Rate, and we do this by increasing the True Positive Rate by some amount. 
% %we will need to exhibit a classifier that classifies some of the mal
% %Given whatever parameters $p,q$ we choose, the True Positive Rate for Group $B$ on the biased data is
% Assume we pick any $h_B \in \PQ$ such that $TPR(h_B)=1-\sqrt{\alpha}(1-\hat{p}_A)$ for some $\hat{p}_A$.
% $\hat{p}_A$ is implicitly determined by $p_A,q_A$. 

% Then the True Positive Rate equality (Equal Opportunity) requires that;
% \begin{align*}
% & TPR(h_{A}) = 1-\hat{p}_B= 1-\sqrt{\alpha}(1-\hat{p}_A) = TPR(h_B) \\
% & \hat{p}_B =\sqrt{\alpha}(1-\hat{p}_A)  
% \end{align*}
% $\hat{p}_B$ is an amount we deviate from $h_{B}^{*}$ by intentionally mis-classifying positive points in Group $A$.

% The excess error compared to $h^{*}$ is
% \begin{align*}
% & =  \hat{p}_A \sqrt{\alpha} + \hat{p}_{B}(1-\sqrt{\alpha}) \\
% & = \hat{p}_A \sqrt{\alpha}+  \sqrt{\alpha}(1-\hat{p}_A ) (1-\sqrt{\alpha}) \\
% & = \sqrt{\alpha} - \alpha - p \sqrt{\alpha} + p \alpha + p \sqrt{\alpha} = \sqrt{\alpha} - \alpha -  + p \alpha \\
% &  \geq \sqrt{\alpha} - \alpha = \Omega(\sqrt{\alpha})
% \end{align*}
% [Recall we need to weight by group size and the only deviation from $h^{*}$ of corruption is given by $\hat{p}_B$ and $1-\hat{p}_B$.
\end{proof}




%\section{Equalized Odds}
\label{sec:eoddsproof}
Now we will consider Equalized Odds.
%which will have a similar lower bound to Calibration in Theorem \ref{thm:calib}. 

\begin{proof}[Equalized Odds Proof  of $\Omega(1)$ accuracy loss:] \label{proof:eodds}
%To show that for Equalized Odds requires $\Omega(1)$ error when the adversary has corruption budget $\alpha$, even with our hypothesis class $\PQ$, 
it suffices to exhibit a `bad' distribution and matching corruption strategy; which we exhibit below.

\begin{enumerate}
\item Say Group A has $1-\alpha$ of the probability mass i.e. $P_{x \sim \calD}[x \in A] \geq 1-\alpha$ and thus $P_{x \sim \calD}[x \in B] \leq \alpha$.
\item The positive fraction for each group under distribution $\mathcal{D}$ is $P_{x \sim \DA}[y=1]=P_{x \sim D_B}[y=1]=\frac{1}{2}$
\item Since $P_{x \sim \calD}[x \in B] \leq \alpha$, the adversary has sufficient corruption budget such that they can inject a duplicate copy of each example in B but with the opposite label.   
That is, for each example x in Group B in the training set, the adversary adds another identical example but with the opposite label.
\end{enumerate}

This adversarial data ensures that on Group $B$, any hypothesis $h$ (of any form) will now satisfy 
\[ P_{x \sim \hat{\mathcal{D}}_{B}}[h(x)=1 | y=1] = P_{x \sim   \hat{\mathcal{D}}_{B}}[h(x)=1  | y=0] = p  \] for some value $p \in [0,1]$
due to the indistinguishable duplicated examples; i.e. the hypothesis can choose how often to accept examples but it cannot distinguish positive/negative examples anyway in Group $B$.\footnote{Since these distributions are evenly balanced by class, $P_{x \sim \DB}[h(x)=1]=p$.}

Note that we can select $p$ using some arbitrary \pcoreplace{$h \in \PQ$}{h} but that randomness does not help us.
Observe that similarly, the True Negative/False Negative Rates on $B$ must be $1-p$.
%So, for group A, to satisfy equalized odds, both thes terms must  must also equal $1/2$

Since $A$ is evenly split among positive and negative and we must satisfy Equalized Odds, 
this means that our error rate on group A is
\begin{align*}
& P_{(x,y) \sim \DA}[ h(x) \neq y ] =  P_{(x,y) \sim \DA}[ h(x) \neq y \cap y=1] +  P_{(x,y) \sim \DA}[ h(x) \neq y \cap y=0] \\
& =P_{(x,y) \sim \DA}[ h(x) \neq 1 | y =1  ]P[y=1] + P_{(x,y) \sim \DA}[ h(x) \neq 0 | y = 0  ]P[y=0]  \\
& = P_{(x,y) \sim \DA}[ h(x) \neq 0 | y =1  ]P[y=1] + P_{(x,y) \sim \DA}[ h(x) = 1 | y = 0  ]P[y=0] \\
& = (1-TPR_{A}) \frac{1}{2} + FPR_{A} \frac{1}{2} \\
& =(1-p)(\frac{1}{2}) + p(\frac{1}{2})= \frac{1}{2}
\end{align*}
So, the adversary has forced us to have $50 \%$ error on group A.
\end{proof}

\section{Main Results: Calibration}
\label{sec:calib}
In this section, we explore various notions of calibration \cite{dawid} for our model.
Calibration is a desirable property typically considered for classifiers, where predicted label probabilities should correspond to observed frequencies in the long run. For example, in weather forecasting, a well-calibrated predictor should have approximately 60\% of days with rain when it forecasts a 60\% chance of rain. This calibration requirement should hold for every predicted probability value output by the model.

Calibration has important fairness implications \cite{flores2016false, chouldechova2017fair, faircalib,multicalib} because a mis-calibrated predictor can lead to harmful actions in high-stakes settings, such as over-incarceration \cite{compassgender}. 
We show that varying the exact calibration requirements can substantially impact the model's accuracy loss when malicious noise is present in the training data.
%These results may be of independent interest to the calibration literature.

In this section, we align closely  with \cite{faircalib}, where the learner seeks to maximize accuracy while ensuring the classifier is perfectly calibrated.
Throughout this paper, we have focused on binary classifiers, so in Section \ref{subsec: predparity} we consider a related notion called Predictive Parity \cite{chouldechova2017fair, flores2016false}, before considering calibration notions for hypotheses with output in $[0,1]$.


\begin{comment}
To recall, as before the learning problem is 

\begin{align}
\min_{h\in\mathcal{H}} & ~~\mathbb{E}_{(X,Y,Z)\sim\mathcal{D}} \left[\mathbbm{1}(h(X) \neq Y)\right] ~~~~\ \\
\text{subject to} & |K(z)-K(z')| \leq \delta \qquad \forall z,z'\in \mathcal{Z}. \label{FairnessConstraint}
\end{align}
where $K: h \rightarrow \mathbbm{R}$ is some notion of calibration error for $h$ for group $z$ given the true labels $y$.

Typically calibration requirements are most natural for regression problems where predictor $h$ provides 
\emph{fine-grained} scores that corresponds to the underlying 
probability of some outcome. 
\end{comment}

%Throughout these sections we consider a property titled \emph{shared range}.
%Namely that even though the hypotheses are fine tuned for each group, these calibrated classifiers
%share the same range. 

\subsection{ Predictive Parity Lower Bound}
\label{subsec: predparity}
\begin{definition}[Predictive Parity \cite{chouldechova2017fair}]
A binary classifier $h: \mathcal{X} \rightarrow \{0,1\}$ satisfies predictive parity if for groups A and B, $P_{x \sim \DA}[h(x)=1]>0$, $P_{x \sim \DB}[h(x)=1]>0$ and
% \footnote{This mild technical remark is explained a in the Appendix} 
\[P_{(x,y) \sim \DA} [y=1 | h(x)=1] = P_{(x,y) \sim \DB} [y=1 | h(x)=1] \]
\end{definition}
In later sections we consider other calibration notions.
Here we consider an adversary who is attacking a learner constrained by equal predictive parity when group sizes are \emph{imbalanced}.

\begin{comment}
\begin{theorem}
With probability $1-(1-n)^{\alpha}$, there exists a FAIR ERM learner constrained learner with $O(\alpha)$
excess error. 
\end{theorem}
\end{comment}


\begin{theorem}
\label{thm:predparity}
    For a malicious adversary with corruption fraction $\alpha$, for Fair-ERM constrained to satisfy Predictive Parity, then there is no $h \in \PQ$ with less than $\Omega(1)$ error. 
\end{theorem}

The intuition for this statement is that imbalanced group size will allow the adversary to change the conditional mean substantially.
%In expectation, 
Below, we have an informal proof:
\begin{proof}[Proof Sketch:]
Suppose $P(x \in A)=1-\alpha$ and $P(x \in B)= \alpha$.
Observe that whatever the initial value of $P_{(x,y) \sim \DB} [y=1 | h(x)=1]$, the adversary can drive this value $P_{(x,y) \sim \mathcal\DBC} [y=1 | h(x)=1]$ to $50\%$ or below
by adding a duplicate copy of every natural example in group $B$ with the opposite label.

Since all of these points are information-theoretically indistinguishable, any hypothesis for group $B$ that makes any positive predictions incurs at least $50\%$ error and $1/2=P_{(x,y) \sim \mathcal\DBC} [y=1 | h(x)=1]$ calibration error.
%will have to do the same for
Any classifier for group $A$ satisfying Predictive Parity will have to do the same, yielding our $\Omega(1)$ error.
%The full proof in Section \ref{proof:predparity}.
%\begin{align*}
 %   blehp
%\end{align*}
%Observe that this attack 
\end{proof}

\subsection{Extension to Finer Grained Hypothesis Classes}
\label{subsec:calib}
A criticism of this lower bound might be that these calibration notions are very coarse and calibration is intended for fine-grained predictors, meaning those that have a finer grained discretization of the probabilities in $[0,1]$.
%and inappropriate for a binary classifier that in effect has two bins. 
%While a diversion from the rest of the paper where we tend to focus on binary classifiers, 
We now provide extensions for these lower bounds to real valued $\mathcal{H}$. 
Interestingly, we show if the learner can modify their `binning strategy', the learner can `decouple' the classifiers for the groups in the population and 
thus only suffer $O(\alpha)$ accuracy loss.
%Rather than being an algorithmic trick, this attack is fundamental as it seems to occur in the wild
%organically 
%as a type of red-lining. 
%This is because absent further constraints, calibration is a weak notion of mere self-consistency.
%Attacks of this type motivate more constrained notions of calibration like 
We adopt the version of calibration from \cite{faircalib}.
\begin{definition}[Calibration] \label{def:calib}
A classifier $h: \mathcal{X} \rightarrow [0,1]$ is Calibrated with respect to distribution $\mathcal{D}$ if 
\[\forall r \in [0,1], r= \mathbb{E}_{(x,y) \sim \mathcal{D} }[y=1| h(x)=r]\]
We will primarily focus on the discretized version of this definition where the classifier assigns every data point to one of $R$ bins, each with a corresponding label $r$, that partition $[0,1]$ dis-jointly. 
We will refer to this partition as $[R]$ with $r \in [R]$ corresponding to the prediction of a bin. 
\[ \forall r \in [R], r= \mathbb{E}_{(x,y) \sim \mathcal{D} }[y=1| h(x)=r] \]
\end{definition}
%Observe that nothing in this initial definition references groups. 
%The natural generalization to the above definition with 
Calibration as a fairness requirements with demographic groups requires that the classifier $h$ is calibrated with 
respect to the group distributions $\DA$ and $\DB$ simultaneously. 
In the sections that follow when we say `calibrated' this always refers to calibration with respect to $\DA$ and $\DB$. 

\begin{theorem}
\label{thm:calib}
    The learner wants to maximize accuracy subject to using a calibrated classifier, $h: \mathcal{X} \rightarrow [R]$ where $[R]$ is a partition of $[0,1]$ into bins.%^labelled bins with each label.
    
    The learner may modify the binning strategy after the adversary commits to a corruption strategy.
    Then an adversary with corruption fraction $\alpha$ can force at most $O(\alpha)$ excess accuracy loss over the non-corrupted optimal
    classifier. 
\end{theorem}

\newpage


\subsection{Parity Calibration}
Motivated by Theorem \ref{thm:calib}, we introduce a \emph{novel} fairness notion we call \emph{Parity Calibration}\footnote{We would note that this is initial discussion of a novel fairness constraint that arose naturally from considering Theorem \ref{thm:calib}. The idea is in some cases it might be more desirable to have a more sensitive calibration notion, hence we define Parity Calibration. This notion requires further study and analysis before deployment in sensitive contexts.}
% \footnote{This is a strong fairness constraint and should be thought of as a strong prior that while conditional label distribution $\mathcal{D}_{y|x}$ can be different among groups, how much of each group falls in each risk category is the same.}.
Informally, this notion is a generalization of Statistical/Demographic parity \cite{dwork2012fairness} for the case of classifier with 
$R$ bins partitioning $[0,1]$.
\begin{definition}[Parity Calibration]
\label{def:paritycalib}
Classifier $h: \mathcal{X} \rightarrow [R]$, where $[R]$ is a partition of $[0,1]$ into labelled bins, satisfies
\emph{Parity Calibration} if the classifier is Calibrated (Definition \ref{def:calib}) \emph{and}
\begin{align*}
\forall r \in [R], P_{(x,y) \sim \DA} [h(x)=r] =  P_{(x,y) \sim \DB} [h(x)=r]
\end{align*} 
\end{definition}


%These lower bounds still hold for stronger notions of calibration error, namely $K_1(h, \mathcal{D})$ and $K_2(h, \mathcal{D})$ 
%which are average calibration error for the $l_1$ and $l_2$ norms respectively.
\begin{theorem}
\label{thm:paritycalib} 
Consider a learner maximizing accuracy subject to satisfying Parity Calibration.
%$h: \mathcal{X} \rightarrow [R]$ where $[R]$ is a partition of $[0,1]$ into labelled bins with each label.
    The learner may modify the binning strategy after the adversary commits to a corruption strategy.
    Then an adversary with corruption fraction $\alpha$ can force $\Omega(1)$ excess accuracy loss over the non-corrupted optimal
    classifier. 
\end{theorem}

%We defer the proof of this statement to the appendix, but the intuition is a follows.

If the size of Group $B$ is $O(\alpha)$, then following a similar duplication strategy for Predictive Parity Theorem \ref{thm:predparity},
then the adversary can force Group $B$ to have an expected label of $50\%$, i.e.
$\forall x \in B, \mathbb{E}_{x \sim \DB}[y|x]=50\%$.
Thus, any classifier that is calibrated must assign all of Group $B$ to a $50\%$ bucket.
In order to satisfy \emph{Parity Calibration}, the classifier must do the same to Group $A$, yielding $50\%$ error on Group $A$.

% \kmsdelete{\subsection{Discussion}
% In general these results are consistent with the observed behavior of Calibration in other parts of theoretical computer science.
% If the learner/society really only cares about accuracy, then the insensitivity in Section \ref{subsec:calib} is somewhat of a feature, not a bug, 
% especially if the unreliability of data in Group $B$ optimistically could be transient?
% %However, advocates for stronger notions calibration would instead note that in \ref{thm: calib} 
% In general, when thinking about accuracy loss and malicious in the context of fair ERM; what is the appropriate amount of sensitivity in
% the learning process? We shall discuss this somewhat more in Section \ref{sec: discussion}.}
% %We would observe that are substantial 


\begin{comment}
\begin{definition}[Average Calibration Error]
The avergae calibration error of a predictor $h$ (with $h: \mathcal{X} \rightarrow [0,1]$) on distribution $\mathcal{D}$ is:
\[ K_1(f, \mathcal{D}) = \sum_{v \in R(h) } P_{(x,y) \sim D} [h(x)=v]|v-\mathbbm{E}_{(x,y) \sim \mathcal{D}}[y|h(x)=v] |\]
where $R(h)$ is the range of $h$. 

Similarly, the average squared calibration error is 
    \[ K_1(f, \mathcal{D}) = \sum_{v \in R(h) } P_{(x,y) \sim D} [h(x)=v]|(v-\mathbbm{E}_{(x,y) \sim \mathcal{D}}[y|h(x)=v])^2\]\end{definition}

\begin{theorem}
    
\end{theorem}
\end{comment}




%
\section{Minimax Fairness}
\label{subsec:minimaxfair}
In this Section, we will consider Minimax Fairness.
Introduced in \cite{minimaxfair} this notion  optimizes for a different objective. 


Using their notation ($\epsilon_k = \mathbb{E}_{(x,y) \sim \mathcal{D}_k}[h(x) \neq y]$ or group-wise error),
\[ h^{*} = \argmin_{h \in \Delta{H}} \quad \{ \max_{ \leq k \leq K}  \epsilon_{k} (h) \} \]
Letting $OPT$ refer to the value of solution of the optimization problem, the learning goal is to find an $h$ that is $\epsilon$-approximately optimal for the mini-max objective, meaning that $h$ satisfies: 
\[ max_{k} \epsilon_{k}(h) \leq OPT + \epsilon\]

Observe that if the goal of the learner is compete with the value of $OPT$ on the unmodified data, in our malicious noise model this objective is
ineffective since if one group is of size $O(\alpha)$, the adversary can always drive the error rate on that group $\Omega(1)$.

This model seems incompatible with malicious noise due to the sensitivity of minimax fairness to small groups. 

Observe that the minimax fairness framework includes Equalized Error
rates as a special case.


%\begin{align*}
%& \min_{h \in \simplex{H}} err(h) \\
%& \text{subject to} err_{k}(h) \leq \gamma, k=1, \dots K
%\end{align*}

%Observe that when one sub-group is $O(\alpha)$ size, the adversary can always drive t

\begin{comment}
\section{Sufficient Properties for Robust Fair-ERM, Discussion from 4/18/23} 
%This isn't ready for prime time unfortunately-follow up work
In the previous section we exhibited a randomized classifier derived from a base class $\mathcal{H}$ that satisfied 
the fairness constraints on the biased data \emph{and} suffered a minimum accuracy loss.

However, there are obstacles to using this type of randomized classifier in practice.
Primarily that in fairness contexts, test time randomness feels `unfair' or in some sense procedural inappropriate in high stakes settings.
This observation motives us to try to abstract away the joint properties 
of the classifier and the distribution that would allow us to implement a deterministic classifier with the same properties as the randomized closure. 

\begin{example}
blah blah I am an example
\end{example}
\end{comment}
\section{Discussion}
\label{sec: discussion}
\kmsdelete{In this work} We study \kmsreplace{Fairness-Aware PAC learning}{Fair-ERM} in the malicious noise model, and  in some cases allow 
the learner to maintain optimal overall accuracy despite the signal in Group $B$ being almost entirely washed out.
%when we allow learners to use the
%$\PQ$ randomized expansion of the hypothesis class $\mathcal{H}$
In particular we show that different fairness constraints have fundamentally different behavior in the presence of Malicious Noise, in terms of the amount of accuracy loss that a given level of Malicious Noise could cause a fairness-constrained learner to incur. 
The key to achieving our results, which are more optimistic than those in \cite{lampert}, is allowing for improper learners using the (P,Q)-randomized expansions of the given class $\mathcal{H}$.
%We \kmsreplace{present a picture of the}{prove upper and lower bounds on}
%accuracy loss for a range of fairness notions, given \kmsreplace{this simple randomization step.}{learning over $\PQ$.
%In general our results indicate Fair-ERM (given learning over $\PQ$) is more robust than claimed in \cite{lampert}.
The type of smoothness we create by using $\PQ$ seems to be a natural property that is likely shared by many natural hypothesis classes.

Fairness notions are motivated as a response to learned disparities when there is \kmsdelete{data corruption or} systemic error affecting \kmsdelete{the data for}
one group. 
Fairness notions are supposed to mitigate this by ruling out classifiers that have worse performance on a sub-group. 
This can peg both classifiers at a lower level of performance \kmsdelete{(e.g that the lower subgroup)} in order to \emph{motivate} \cite{hardt16} improving the data collection or labelling process to obtain more reliable performance. 
%So in \kmsreplace{some}{a} sense, sensitivity of the fairness notion to poor sub-group performance caused by malicious noise is the \textit{point} of fairness constraints! 
However, it also desirable that fairness constraints perform gracefully when subject to Malicious Noise because fairness constraints will be used in contexts where the data is unreliable and noisy and this might not be known to the learner.
This tension, exposed by our work, motivates 
%a revisiting of fairness notions from first principles approach and trying to axiomatize the 
%desired properties of a fairness intervention a la cryptography and privacy. \footnote{Work in multi-calibration \cite{multicalib} is a viable direction for this problem but it is unclear how 
%that and related notions behave with unreliable data. }
on going work studying the sensitivity level of fairness constraints. 
%If we we are to take a view, if a classifier is deployed 


%\newpage

% We choose the "plain" reference style
%\bibliographystyle{plain}
\bibliography{ref}

\newpage

\begin{comment}
\section{System Architecture}
\label{appendix:architecture}
\system has a novel modularized system architecture with three key components: 
\emph{StreamManager}, 
\emph{TxnManager} and \emph{TxnScheduler}. 
These components are instantiated in each thread locally.
The execution outline of \system is presented in Algorithm~\ref{alg:algo}.
Transactional stream processing is continuous and potentially never ends (Line 1$\sim$8).
The dependency resolution and execution of state transactions are separated into two non-overlapping phases by punctuations~\cite{Tucker:2003:EPS:776752.776780} (Line 2 and 5), which guarantees that no subsequent input event will have a smaller timestamp. 
Effectively, a batch of state transactions is collected during the first phase, and processed during the second phase.

In the first phase (i.e., stream processing phase), 
the \emph{StreamManager} conducts preprocessing for every input event ($e$). Similar to some prior works~\cite{tstream}, state transactions may be issued but not immediately processed during preprocessing (Line 3).
The \emph{pre\_processing} and \emph{post\_processing} functions are exposed as APIs to users.
The \emph{TxnManager} handles dependency resolution (Line 4) among state transactions and insert decomposed operations to construct a \tpg. We discuss the detailed two-phase \tpg construction process in Section~\ref{subsec:construction}.

In the second phase  (i.e., transaction processing phase), 
the \emph{TxnManager} is first involved again to refine (Line 6) the constructed \tpg with further dependency resolution.
The \emph{TxnScheduler} 
schedules operations for concurrent execution based on the constructed \tpg according to the three dimensions of scheduling decisions (Line 7). 
In particular, a scheduling decision model $M$ is instantiated based on the constructed \tpg (Line 14).
\textbf{\circled{1}} Guided by $M$, execution threads adopt an exploration strategy (Section~\ref{subsec:explore}) to explore the constructed \tpg for operations available to be scheduled constrained by dependencies. 
\textbf{\circled{2}} 
During exploration, one or multiple operations may be treated as the 
% basic 
unit of scheduling (Section~\ref{subsec:granularity}). 
Subsequently, \textbf{\circled{3}} every thread executes operation(s) in the unit of scheduling with various abort handling mechanisms (Section~\ref{subsec:abort_handling}).
Only when state transactions are processed (i.e., committed or aborted) can the associated input events be postprocessed (Line 8) by the \emph{StreamManager} based on transaction processing results.
\end{comment}

\begin{comment}
\begin{algorithm}
\footnotesize
    \KwData{$e$ \tcp{Input event}}
    \KwData{$txn_{ts}$ \tcp{State transaction}}
    \KwData{$G$ \tcp{The currently constructed TPG}}
    \While{!finish processing of input streams}{
        \eIf(\tcp*[h]{Phase 1}){\text{$e$ is not a $punctuation$}}{
                $txn_{ts}$ $\gets$ PRE\_Processing($e$)\;
                \textbf{TPG\_Construction}($G$, $txn_{ts}$)\; 
          }(\tcp*[h]{Phase 2}){
                \textbf{TPG\_Refinement}($G$)\; 
                \textbf{TXN\_Scheduling}($G$)\; 
                POST\_Processing()\;
          }
    }
    
    \SetKwFunction{FMain}{TPG\_Construction}
    \SetKwProg{Fn}{Function}{:}{}
    \Fn{\FMain{$G$, $txn_{ts}$}}{
        $O_{1..k}$ $\gets$ \textbf{Partition} $txn_{ts}$\;
        \ForEach{\text{operation $O_{i}$ $\in$ $O_{1..k}$}}{
            \textbf{Identify} its \ld\;
            $G$ $\gets$ $G$ + $O_{i}$ \;
        }
    }
    \SetKwFunction{FMain}{TPG\_Refinement}
    \SetKwProg{Fn}{Function}{:}{}
    \Fn{\FMain{$G$}}{
        \ForEach{\text{vertex $e_{i}$ $\in$ $G$}}{
            \textbf{Identify} its \td, \pd\;
        }
    }
    
    \SetKwFunction{FMain}{TXN\_Scheduling}
    \SetKwProg{Fn}{Function}{:}{}
    \Fn{\FMain{$G$}}{
        $M$ $\gets$ Instantiated with $G$;\tcp{A decision model}
        \While{!finish scheduling of $G$
        }{
          \textbf{\circled{2}} $Scheduling Unit$ $\gets$ \textbf{\circled{1}} \emph{Explore}($G$, $M$)\; 
            \textbf{\circled{3}} \emph{Execute with Abort Handling} ($Scheduling Unit$)\; 
        }
    }
  \caption{Execution Outline of \system}
  \label{alg:algo}
\end{algorithm}
\end{comment}

\end{document}