\section{Equalized Odds}
\label{sec:eoddsproof}
Now we will consider Equalized Odds.
%which will have a similar lower bound to Calibration in Theorem \ref{thm:calib}. 

\begin{proof}[Equalized Odds Proof  of $\Omega(1)$ accuracy loss:] \label{proof:eodds}
%To show that for Equalized Odds requires $\Omega(1)$ error when the adversary has corruption budget $\alpha$, even with our hypothesis class $\PQ$, 
it suffices to exhibit a `bad' distribution and matching corruption strategy; which we exhibit below.

\begin{enumerate}
\item Say Group A has $1-\alpha$ of the probability mass i.e. $P_{x \sim \calD}[x \in A] \geq 1-\alpha$ and thus $P_{x \sim \calD}[x \in B] \leq \alpha$.
\item The positive fraction for each group under distribution $\mathcal{D}$ is $P_{x \sim \DA}[y=1]=P_{x \sim D_B}[y=1]=\frac{1}{2}$
\item Since $P_{x \sim \calD}[x \in B] \leq \alpha$, the adversary has sufficient corruption budget such that they can inject a duplicate copy of each example in B but with the opposite label.   
That is, for each example x in Group B in the training set, the adversary adds another identical example but with the opposite label.
\end{enumerate}

This adversarial data ensures that on Group $B$, any hypothesis $h$ (of any form) will now satisfy 
\[ P_{x \sim \hat{\mathcal{D}}_{B}}[h(x)=1 | y=1] = P_{x \sim   \hat{\mathcal{D}}_{B}}[h(x)=1  | y=0] = p  \] for some value $p \in [0,1]$
due to the indistinguishable duplicated examples; i.e. the hypothesis can choose how often to accept examples but it cannot distinguish positive/negative examples anyway in Group $B$.\footnote{Since these distributions are evenly balanced by class, $P_{x \sim \DB}[h(x)=1]=p$.}

Note that we can select $p$ using some arbitrary \pcoreplace{$h \in \PQ$}{h} but that randomness does not help us.
Observe that similarly, the True Negative/False Negative Rates on $B$ must be $1-p$.
%So, for group A, to satisfy equalized odds, both thes terms must  must also equal $1/2$

Since $A$ is evenly split among positive and negative and we must satisfy Equalized Odds, 
this means that our error rate on group A is
\begin{align*}
& P_{(x,y) \sim \DA}[ h(x) \neq y ] =  P_{(x,y) \sim \DA}[ h(x) \neq y \cap y=1] +  P_{(x,y) \sim \DA}[ h(x) \neq y \cap y=0] \\
& =P_{(x,y) \sim \DA}[ h(x) \neq 1 | y =1  ]P[y=1] + P_{(x,y) \sim \DA}[ h(x) \neq 0 | y = 0  ]P[y=0]  \\
& = P_{(x,y) \sim \DA}[ h(x) \neq 0 | y =1  ]P[y=1] + P_{(x,y) \sim \DA}[ h(x) = 1 | y = 0  ]P[y=0] \\
& = (1-TPR_{A}) \frac{1}{2} + FPR_{A} \frac{1}{2} \\
& =(1-p)(\frac{1}{2}) + p(\frac{1}{2})= \frac{1}{2}
\end{align*}
So, the adversary has forced us to have $50 \%$ error on group A.
\end{proof}