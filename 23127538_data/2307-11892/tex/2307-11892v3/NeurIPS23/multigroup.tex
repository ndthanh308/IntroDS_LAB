\subsection{ Multiple Groups}
\kmsmargincomment{I think this is all true but I need to think more closely---low priority since it can be done in the camera ready version}
In this paper, for ease of presentation we will focus on the disjoint case of binary groups i.e. two disjoint groups. 
Our core theorems have natural extensions to the case of an integer number of groups, where the adversary will typically attack the \emph{smallest group}.
%If the groups have non-empty intersection, this will not change the lower/upper bounds in this paper with the appropriate modification which we discuss in Appendix \ref{multgroup}.
\pcomargincomment{Is this actually true? It's not obvious to me that this holds. If a point lies in the intersection of A and B, how would the randomized classifier behave? which biased coin would they toss, the one for A or for B}
%In the case of group overlap, we can observe that our lower bounds depend on the size of the smallest group. 
The elegant and contemporary approach in \cite{multicalib} of defining relevant groups with a computationally rich class $C$ handles the issue of group and sub-group 
overlap \pcodelete{naturally}, %two naturals next to each other
but a natural research question is how that framework interacts with malicious noise and our closure model.
We briefly discuss this in Section \ref{sec:calib} as a direction for future research.
%but a full treatment of this important problem will be left for future research.
