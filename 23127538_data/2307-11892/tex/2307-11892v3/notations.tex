\usepackage{amsmath}
\usepackage{amssymb}
\usepackage{amsthm}
\usepackage{mathtools}
\usepackage{comment}
\usepackage{todonotes}
\usepackage{float}
\usepackage[algo2e,ruled,linesnumbered]{algorithm2e}

% for restatable
\usepackage{thmtools,thm-restate}
\usepackage{physics}
% %% >> for restatable links
% \usepackage{xpatch}
% \usepackage{xcolor}
% \usepackage{scalerel}

% % a flag to turn on and off
% \newif\ifmarginprooflinks
%     \marginprooflinkstrue
%     % \marginprooflinksfalse


% %% STEP 1: patch restatable so there are backward links on recall
% \makeatletter
% \xpatchcmd{\thmt@restatable}% Edit \thmt@restatable
%    {\csname #2\@xa\endcsname\ifx\@nx#1\@nx\else[{#1}]\fi}% Replace this code
%    {\ifthmt@thisistheone%
%     \csname #2\@xa\endcsname\ifx\@nx#1\@nx\else[{#1}]\fi% same as before
%     %except with also marginparbox
%    \else\fi} {}{\typeout{FIRST PATCH TO THM RESTATE FAILED}}
% \xpatchcmd{\thmt@restatable}% A second edit to \thmt@restatable
%    {\csname end#2\endcsname}
%    {\ifthmt@thisistheone\csname end#2\endcsname\else\fi}
%    {}{\typeout{FAILED SECOND THMT RESTATE PATCH}}


% \newcommand{\recall}[1]{\medskip\par\noindent{\bf \Cref{thmt@@#1}.} \begingroup\em \noindent
%    \expandafter\csname#1\endcsname* \endgroup\par\smallskip}

% %% STEP 2: make forward links to restatable.
% \setlength\marginparwidth{1.55cm}
% \let\oldmarginpar\marginpar
% \renewcommand{\marginpar}[1]{%
%     \leavevmode%
%     \oldmarginpar{#1}%
%     \ignorespacesafterend\ignorespaces}
% \newsavebox\marginprooflinkbox
% \newenvironment{linked}[3][]{%
%     \def\linkedproof{#3}%
%     \def\linkedtype{#2}%
%     \ifmarginprooflinks%
%     \sbox\marginprooflinkbox{%
%         \centering%
%         \hyperref[proof:\linkedproof]{%
%         \color{blue!30!white}%
%         \scaleleftright{$\Big[$}{\,\mbox{\footnotesize\centering\tt\begin{tabular}{@{}c@{}}
%             link to\\[-0.15em]
%             proof
%         \end{tabular}}\,}{$\Big]$}}~}
%     \fi
%         \restatable[#1]{#2}{#2:#3}\label{#2:#3}%
%     \reversemarginpar	\ifmarginprooflinks\marginpar{\vspace{-1ex}\usebox\marginprooflinkbox}\fi
%     }%
%     {\sbox\marginprooflinkbox{}\endrestatable}
% \newcounter{proofcntr}
% \newenvironment{lproof}{\begin{proof}\refstepcounter{proofcntr}}{\end{proof}}

\newcommand{\vect}[1]{\ensuremath{\mathbf{#1}}}

%% Useful
\newcommand{\p}[1]{\left( #1 \right)}
\newcommand{\br}[1]{\left[ #1 \right]}


%\newcommand{\ev}[1]{\mathbb{E}\left[{#1}\right]}
\newcommand{\evd}[2]{\mathbb{E}_{#1}\left[{#2}\right]}

%% Algortihm notations
\newcommand{\bigO}[1]{O \left( #1 \right )}

%% Calibration
\newcommand{\I}[1]{\mathbb{I}\left[#1\right]}       % Indicator
\newcommand{\calerr}{\mathrm{calerr}}   % Calibration error

\newcommand{\A}{\mathcal{A}}    % Algorithm
\newcommand{\Ber}{\mathrm{Ber}}
\newcommand{\Ecover}{\event^{\textrm{cover}}}   % Event that covered epochs exist
\newcommand{\Enegl}{\event^{\textrm{negl}}}     % Event that negligible epochs exist
\newcommand{\Epoch}{\mathsf{Epoch}}
\newcommand{\eps}{\epsilon}     % epsilon
\newcommand{\Etruth}{\event^{\textrm{truth}}}   % Event that all epochs are truthful
\newcommand{\event}{\mathcal{E}}    % Events
\newcommand{\Ex}[2]{\operatorname*{\mathbb{E}}_{#1}\left[#2\right]}  % Expectation
\newcommand{\Int}{\mathcal{I}}      % Interval
\newcommand{\poly}{\operatorname*{poly}}    % Polynomial
\newcommand{\pr}[1]{\Pr\left[#1\right]}     % Probability
%\newcommand{\red}[1]{{\color{red} #1}}

\newcommand{\red}[1]{\textcolor{red}{#1}}
\newcommand{\blue}[1]{\textcolor{blue}{#1}}

\newcommand{\SPinner}{\mathsf{SP}^{\textrm{inner}}}
\newcommand{\SPouter}{\mathsf{SP}^{\textrm{outer}}}
\newcommand{\Tact}{T^{\mathrm{actual}}}     % Actual stopping time
\newcommand{\prodspace}{\mathcal{X}\times A \times \mathcal{Y}}


%% Fair ERM Notation
\newcommand{\error}[1]{ \left| \mathbb{E}_{(x,y) \sim \mathcal{D}} \ [\one (#1(x) \neq y)] - \ \mathbb{E}_{(x,y) \sim \mathcal{D}} \ [\one (h^*(x) \neq y)] \right|}
\newcommand{\htilde}{\tilde{h}}
\newcommand{\hhat}{\hat{h}}
\newcommand{\hstar}{h^*}
\newcommand{\hclass}{\mathcal{H}}
\newcommand{\posrate}[1]{ P_{(x,y) \sim \DA} [#1 (x)=1]}

\newcommand{\DAC}{\widetilde{\mathcal{D}}_A}
\newcommand{\DBC}{\widetilde{\mathcal{D}}_B}
\newcommand{\RA}{P_{(x,y) \sim \dist } [x \in A]}
\newcommand{\RB}{P_{(x,y) \sim \dist} [x \in B]}
\newcommand{\normalF}{F}
\newcommand{\corruptF}{\widetilde{F}}
