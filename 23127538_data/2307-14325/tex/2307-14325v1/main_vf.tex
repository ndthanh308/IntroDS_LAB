\documentclass[aps,pra,amsfonts,floatfix,superscriptaddress,twocolumn,longbibliography,nofootinbib]{revtex4-1}

\usepackage{graphicx}% Include figure files
\usepackage{dcolumn}% Align table columns on decimal point
\usepackage{bm}% bold math
\setlength\parindent{.25in}
\usepackage{amsmath}
\usepackage{amssymb}
\usepackage{bbold}
\usepackage{physics}
\usepackage{float}
\usepackage{mathtools}
\usepackage{multirow}
\usepackage{tikz}
\usetikzlibrary{quantikz}
\usepackage{adjustbox}
\usepackage{soul,color}
\usepackage{xcolor}
\usepackage{hyperref}

\begin{document}
\preprint{APS/123-QED}
\title{Simulation of Open Quantum Systems via Low-Depth Convex Unitary Evolutions}

\author{Joseph Peetz}
 \email{peetz@ucla.edu}
\affiliation{Department of Physics and Astronomy, University of California, Los Angeles, California 90095}

\author{Scott E. Smart}
\affiliation{College of Letters and Science, University of California, Los Angeles, California 90095}

\author{Spyros Tserkis}
\affiliation{College of Letters and Science, University of California, Los Angeles, California 90095}

\author{Prineha Narang}
\email{prineha@ucla.edu}
\affiliation{College of Letters and Science, University of California, Los Angeles, California 90095}

\date{\today}

\begin{abstract}
Simulating physical systems on quantum devices is one of the most promising applications of quantum technology. Current quantum approaches to simulating open quantum systems are still practically challenging on NISQ-era devices, because they typically require ancilla qubits and extensive controlled sequences. In this work, we propose a hybrid quantum-classical approach for simulating a class of open system dynamics called random-unitary channels. These channels naturally decompose into a series of convex unitary evolutions, which can then be efficiently sampled and run as independent circuits. The method does not require deep ancilla frameworks and thus can be implemented with lower noise costs. We implement simulations of open quantum systems up to dozens of qubits and with large channel rank.
\end{abstract}

\maketitle


\section{Introduction} \label{section: intro}

System-environment interactions play an important role in the dynamics of quantum systems, giving rise to phenomena such as dissipation, decoherence, relaxation, and particle or energy transfer processes \cite{breuer_theory_2007,Head-Marsden2020}. Ideal simulations of open quantum systems on classical computers suffer from exponentially scaling memory and runtime requirements, leaving complex physical systems largely out of reach without major approximations \cite{brown_using_2010}. Using quantum computers, several proposed techniques for simulating open systems exist, with a common theme being to encode the challenging non-unitary dynamics on a larger, dilated Hilbert space \cite{paulsen_completely_2003}. Numerous techniques make use of Stinespring dilation \cite{stinespring_positive_1955, hu_quantum_2020}, Sz.-Nagy dilation \cite{langer_b_1972,Schlimgen2021}, and linear combination of unitaries \cite{childs_hamiltonian_2012, suri_two-unitary_2022,head-marsden_capturing_2021}. In practice, this dilation leads to long controlled gate sequences and large ancilla frameworks, both of which are practically at odds with NISQ-oriented applications.

An important subset of open system dynamics is the random-unitary channels, which admit an operator-sum decomposition \cite{sudarshan_stochastic_1961, hellwig_operations_1970} of the form:
\begin{equation} \label{eq:randomunitary}
\mathcal{E}(\rho)=\sum_i p_i \left(U_i \rho U_i^{\dagger}\right),
\end{equation}
where $p_i$ are probabilities and $U_i$ are unitary operators \cite{mendl_unital_2009}. Random-unitary channels, also known as mixed-unitary, naturally represent many probabilistic processes, including the well-known Pauli channels, and have found widespread utility across quantum information science. In fact, general Markovian processes can be mapped onto effective random-unitary channels through Pauli twirling \cite{dur_standard_2005, cai_constructing_2019} and randomized compiling techniques \cite{wallman_noise_2016}. They are particularly useful for noise modeling of quantum devices \cite{moueddene_realistic_2020, suzuki_qulacs_2021}, as well as within error correction \cite{terhal_quantum_2015} and mitigation \cite{temme_error_2017} schemes. They have also been implemented in the context of quantum simulations of thermal relaxation \cite{rost_simulation_2020, tolunayHamiltonianSimulationQuantum2023}. As quantum computers continue to grow in size, efficient simulation of random-unitary channels thus becomes increasingly imperative.

To address this need, we propose a hybrid quantum-classical method for simulating random-unitary channels. Our approach involves classically sampling a channel's probability distribution $\{p_i\}$, and then running a series of corresponding circuits on a quantum computer. Compared to ancilla-based methods, this reduces both the number of qubits and the depth of circuits required, thereby widening the class of simulations feasible on near-term quantum computers. By outlining a practical and efficient algorithm, we seek to enhance both the understanding and utility of random-unitary channels.

% Figure environment removed 

\section{Random-Unitary Estimation} \label{section: results}

We present a hybrid quantum-classical method of simulating random-unitary channels for open quantum systems. %The unitary channels are sampled randomly from a classical distribution, and then are implemented on the quantum device with the associated probabilities. 
For an ideal state evolved under a random-unitary channel of the form ~\eqref{eq:randomunitary}, $\rho \rightarrow \tilde{\rho} = \mathcal{E}(\rho)$, the expectation value of an observable $\hat{O}$ is:
\begin{align} \label{eq:wsum}
    \expval{\hat{O}}_{\tilde{\rho}} &= \operatorname{Tr} \left[ \hat{O} \tilde{\rho} \right] \\ &= \sum_i p_i \operatorname{Tr} \left[ \hat{O} U_i \rho U_i^{\dagger}\right].
\end{align}
Typically, ancilla-based methods attempt to faithfully produce $\tilde{\rho}$, which can be accessed directly via measurement. 


%\onecolumngrid
%\twocolumngrid
%Importantly, instead of mapping the full channel $\mathcal{E}(\rho)$ onto a circuit, we envision simulating the simple unitary evolutions $\rho \rightarrow \tilde{\rho_i} = U_i \rho U_i^{\dagger}$ and combining their expectation values according to a weighted probability. 

In our approach, we obtain the above expectation via an estimator of the random-unitary channel, following the formalism of Arrasmith et al. \cite{arrasmith_operator_2020}. We generate a multinomial distribution $S \sim \operatorname{Mult}(N,\{p_i\})$, with $N$ total number of shots,  and $S = (\hat{s}_1,\hat{s}_2,\cdot\cdot\cdot \hat{s}_k )$. The $\hat{s}_i$ outcomes correspond to quantum circuits measuring the operator $U_i^{} \hat{O} U^\dagger_i $. Then, the estimator of the expectation values has the form:
\begin{equation}\label{estimator}
    \hat{E} = \frac{1}{N} \sum_i \hat{s}_i \operatorname{Tr} \left[ \hat{O} U_i^{} \rho U^\dagger_i  \right],
\end{equation}
where $\mathbb{E}[\hat{s}_i] = p_i N$. This matches the form of Eq.~\eqref{eq:wsum} and is an unbiased estimator. Figure \ref{big_fig} represents this scheme pictorially. The variance of the estimator does not depend on the complexity or size of the random unitary channel. Rather, it scales with $O(||\hat{O}||/N)$ and is thus efficient up to the operator norm.

We note that a similar heuristic approach was tried for low-dimensional systems, but associated with exponential scaling problems \cite{rost_simulation_2020} and was limited to single-qubit applications \cite{tolunayHamiltonianSimulationQuantum2023}. Crucially, this work demonstrates the feasibility and scalability of this method even for exponentially sized probability distributions $\{p_i\}$, as long as this distribution can be sampled efficiently. We demonstrate these ideas for simple open systems which, even in lieu of error mitigation, show constant scaling over an exponential classical problem.


\section{Demonstrations of Scalable Random-Unitary Simulations} \label{depol_results}

To highlight our approach, we present two cases, both of which lead to exponentially scaling problems. We first look at examples of the depolarizing channel, a Pauli channel which grows exponentially with the number of qubits. We then look at a discretized time-evolution process, which grows exponentially in time. In the first instance, due to the simplicity of the method and the lack of ancilla costs required, we are able to easily demonstrate this approach on superconducting transmon qubit devices. The second example allows for a straightforward demonstration on quantum devices as well, although it can also be classically simulated via exact storage of the two-qubit density matrix. 
\subsection{Depolarizing Channel }

Let $P_i^{(n)}$ denote an $n$-qubit Pauli string, i.e. $P_i^{(n)} \in\{I, X, Y, Z\}^{\otimes n}$. Then the $n$-qubit depolarizing channel \cite{nielsen_quantum_2010} can be written as:
\begin{equation} \label{depolarizing_channel}
\mathcal{E}(\rho)=(1-p) \rho+\frac{p}{4^n-1} \sum_{i=1}^{4^n-1} P_i^{(n)} \rho P_i^{(n)}.
\end{equation}
For $n \in [1,27]$ (with ibmq\_montreal having $27$ qubits), we simulate the depolarizing channel on ibmq\_montreal with strength parameter $p = 0.5$ and an initial state of $|0^n \rangle$. For $n \in [1,3]$, we also demonstrate an ancilla-based approach, using a simple linear combination of unitaries. While many such methods exist, we compare simply against the most common example \cite{childs_hamiltonian_2012}, as generally all of these methods require ancilla circuits. The general resource scaling of the two methods are compared in Table \ref{table:resources}. We measure the set of single-Z observables $\langle Z^{(i)} \rangle$, e.g. $Z \otimes I$ for two qubits or $I \otimes Z \otimes I$ for three, and compare to the following analytic result:
\begin{equation} \label{z_analytic}
  \langle \hat{Z}^{(i)} \rangle = 1-\left(\frac{4^n}{4^n-1}\right) p.  
\end{equation}
We plot the mean squared error, ${\rm MSE} = \frac{1}{n} \sum_i^n (\langle Z^{(i)} \rangle - \langle \hat{Z}^{(i)} \rangle)^2$, against the number of qubits for both the ancilla and low-depth methods.

% Figure environment removed 

The probabilities associated with each unitary of the depolarizing channel are able to be sampled efficiently. However, due to each of these operations belonging to the Clifford group \cite{gottesman_theory_1998}, this simulation can be classically simulated in polynomial time via the Gottesman-Knill theorem \cite{gottesman_heisenberg_1998}. We also include an instance of a simple non-Clifford state, where we prepare the qubits pairwise in the entangled state $\frac{1}{\sqrt{2}} \ket{00} + \frac{e^{i \pi / 4}}{\sqrt{2}} \ket{11}$ and simulate the depolarizing channel for even values of $n$. We measure the set of pairwise observables $\left\langle\hat{X}^{(2i)}\hat{X}^{(2i+1)}\right\rangle$ and compare to the following analytic result:
\begin{equation} \label{xx_analytic}
  \left\langle\hat{X}^{(2i)}\hat{X}^{(2i+1)}\right\rangle=\frac{1}{\sqrt{2}}\left(1-\left(\frac{4^n}{4^n-1}\right) p\right) .
\end{equation}
We plot the mean squared error against the number of simulated qubits using our low-depth hybrid method. All results are combined in Figure \ref{depol_figure}. As shown, our hybrid sampling method clearly outperforms the traditional ancilla-based approach. It also demonstrates consistently reliable results up to the maximum number of qubits. In contrast, circuits to run the multiplexing operations for the ancilla-based method become infeasible after 3 qubits.

Despite the high numbers of qubits and the prevalence of errors, the simulations here still manage to faithfully capture the simulated state dynamics. In Figure \ref{hamming_figure}, we plot the Hamming weight distribution of the 27-qubit measurements alongside analytic expectation values. Expected bit-flip errors across each qubits skew the results away from the all zero state, yet not in a way that corrupts the overall distribution. We model the overall distribution as the depolarizing channel coupled with bit-flip errors of $p_X=4.7\%$ per qubit.

% Figure environment removed

\renewcommand{\arraystretch}{1.5}
\begin{table}[]
    \centering
    \begin{tabular}{c|c|c}
        \multirow{2}{*}{{\bf Method} }  &   \textbf{Stinespring} &  \textbf{Hybrid}   \\[-5pt]
        & {\bf Dilation} & {\bf Sampling} \\ \hline 
      {\bf Qubits}  & $n+\lceil\log (m)\rceil $ & $n$ \\
      {\bf Depth} & $\Omega(m)$ & $1$ \\
      {\bf Circuits} & $1$ & $\min (m, N)$ \\
      {\bf Variance} & $\frac{1}{N}\left(\left\langle O^2\right\rangle_{\tilde{\rho}} - \left\langle O\right\rangle^2_{\tilde{\rho}}  \right)$  & $\frac{1}{N}\left(\sum_i p_i\left\langle O^2\right\rangle_{\tilde{\rho_i}} - \langle O \rangle_{\tilde{\rho}}^2\right)$ 
    \end{tabular}
    \caption{Comparison of resources used in simulating $n$-qubit quantum channels, comprised of a convex sum of $m$ unitary evolutions.}
    \label{table:resources}
\end{table}


\subsection{Noisy Hamiltonian Evolution} \label{section:tfim}
To look at the performance under a different exponentially propagating scheme, we simulate the two-qubit transverse-field Ising model (TFIM) under a noise channel which occurs at each discretized step. 

The TFIM Hamiltonian is 
\begin{equation}
H=-J \sigma_z^{(1)} \sigma_z^{(2)}-h\left(\sigma_x^{(1)}+\sigma_x^{(2)}\right),
\end{equation}
where $J$ is the exchange interaction parameter and $h$ quantifies the strength of the transverse magnetic field. For small time steps $\Delta t$, we evolve this system as the Hamiltonian propagator composed with random unitary channel $\mathcal{S}$. The resulting composed operator is itself a random-unitary channel, denoted as $\tilde{\mathcal{S}}$. We generate each time step recursively, by sampling a distribution as in Eq.~\eqref{estimator}:
\begin{equation}
\tilde{\mathcal{S}}_{t+1} = \tilde{\mathcal{S} }\circ \tilde{\mathcal{S}}_{t}.
\end{equation}
We simulate these dynamics on the ibmq\_kolkata device, with a depolarizing strength of $p=0.05$ per time step, and with TFIM parameters $J=1$ and $h=1$. Figure \ref{tfim_figure} shows the populations over time, both in the standard computational basis and in the eigenbasis of the Hamiltonian.
Fitting exponential decay functions to the eigenstate population curves yields an estimated decoherence time of $T_1 = 4.84 \pm 0.46$ (a.u.), encompassing the predicted time of $T_1 = 5.25$ (a.u.).

In this example, the possible number of circuits to sample from grows exponentially with the number of time steps. For our simulations, after 25 time steps there are $16^{25}$ circuit permutations to consider, a seemingly intractable sample space. However, due to the constant norm of the probability distribution vector, the measurement statistics are agnostic to the complexity of the sample space. This enables our results to accurately capture the system-environment dynamics while using only $10^3$ shots at each time point.

% Figure environment removed


\section{Discussion and Conclusion}

These results demonstrate that for a certain class of quantum channels, we can obtain a significant benefit via the proposed hybrid quantum-classical approach. By delegating unitary transformations to the quantum device, exponentially complex processes can potentially be handled efficiently and are combined through classical sampling and processing. 

This advantage is clearly seen against the ancilla-based method, although there is at least one limitation. Even though our method is more efficient, it does not prepare a coherent quantum state as an output. This means that any observables obtained via further evolutions of the state must also be processed through an estimator. For instance, if one wishes to use a mixed state as input to some broader algorithm, then any desired observables at the end of this algorithm must themselves be sampled using an estimator of the form in Equation \ref{estimator}. However, if the channel's Stinespring circuit has an exponential number of compiled gates, implementation via any quantum processor will likely be significantly challenging as well. This problem is demonstrated in Figure \ref{depol_figure}, where simulating 3 qubits with dilation techniques requires nearly $10^5$ CX gates.

We point out that our method joins a prolific space of stochastic techniques within quantum simulation, and open quantum systems in particular. This is evidenced by many of the aforementioned works \cite{wallman_noise_2016, moueddene_realistic_2020, suzuki_qulacs_2021, terhal_quantum_2015, temme_error_2017, rost_simulation_2020, tolunayHamiltonianSimulationQuantum2023}. Stochastic compilation methods have also notably been used to construct first-order dynamics of Hamiltonian evolution \cite{campbell_random_2019} and twirled channels \cite{kimScalableErrorMitigation2023}, and as a tool in implementing certain shallow channels \cite{rost_simulation_2020}. 

Other difficulties exist in practically implementing these simulations on quantum computers. For instance, random distributions that cannot be sampled in constant or polynomial time could themselves result in memory issues. Another problem exists for $U_i$ that cannot be efficiently implemented as products of two-qubit unitaries, i.e. unitaries with exponential complexity in their generators, although this does not exclude their generation via more sophisticated quantum simulators. These problems would also plague ancilla-based approaches. Finally, while the generality of the method in the context of the entire operator space is quite limited, for random-unitary channels the proposed method offers a straightforward approach for simulating complex dynamics. 

% Conclusion 
Modeling environmental effects are essential in understanding the dynamics of many physical systems. The current work proposes a straightforward way to simulate random-unitary noise channels via classical probabilistic sampling of low-depth unitaries. For quantum simulations of random-unitary processes, the proposed hybrid approach has clear advantages over ancilla-based techniques. While the scope is limited to random-unitary channels, these represent an important class of quantum channels, opening the door for a variety of other dynamics to be simulated on near-term quantum devices. 

\section{Acknowledgements}
This work is supported by the NSF RAISE-QAC-QSA, Grant No. DMR-2037783; the Department of Energy, Office of Basic Energy Sciences, Grant No. DE-SC0019215; the NSF DGE NRT-QISE, Grant No. 2125924; the NSF ECCS CAREER, Grant No. 1944085; and the NSF CNS, Grant No. 2247007.

The authors acknowledge the use of IBM Quantum services for this work. The views expressed are those of the authors, and do not reflect the official policy or position of IBM or the IBM Quantum team.

\section{Appendix} \label{section: appendix}
\subsection{Shot Allocation}
While the hybrid simulation method laid out in Section \ref{section: intro} is robust to different sampling schemes, we optimize the use of valuable computing resources with shot-frugal allocation methods. Practically, some of the operators within a random-unitary channel demand higher priority due to their different weightings $p_i$ in Equation \ref{eq:wsum}. Throughout these results, we use weighted random sampling of the probability distributions $\{p_i\}$ to allocate shots, thus optimizing the use of computing time \cite{arrasmith_operator_2020}. More precisely, we sample the operator space $\{U_i\}$ with corresponding probabilities $\{p_i\}$ to determine the circuit run for each shot. This means that for a simulation using a total of $N$ shots, on average $p_i N$ shots will be allocated to each operator $U_i$ within the sampled space.

To compute the variance of our estimator in Equation \ref{estimator}, we follow the weighted random sampling calculations of Arrasmith et al. \cite{arrasmith_operator_2020}. This gives the following variance:
\begin{equation} \label{hybrid_variance}
\operatorname{Var}(\hat{E})=\frac{1}{N}\left(\sum_i p_i\left\langle O^2\right\rangle_{\tilde{\rho_i}} - \langle O \rangle_{\tilde{\rho}}^2\right),
\end{equation}
where $\tilde{\rho_i} := U_i \rho U_i^{\dagger}$. In comparison, when using Stinespring dilation, the variance of the estimator $\hat{E}_S$ is simply $\frac{1}{N}$ times the variance of the expectation value $\expval{O}_{\tilde{\rho}}$:
\begin{equation}
\operatorname{Var}(\hat{E}_S) = \frac{1}{N} \left(\left\langle O^2\right\rangle_{\tilde{\rho}} - \left\langle O\right\rangle^2_{\tilde{\rho}}  \right).
\end{equation}
Analytically, these expressions are equivalent: Applying Equation \ref{eq:wsum} to $O^2$ yields $\expval{O^2}_{\tilde{\rho}} = \sum_i p_i\left\langle O^2\right\rangle_{\tilde{\rho_i}}$. However, our hybrid sampling can only provide access to the individual expectation values $\left\langle O^2\right\rangle_{\tilde{\rho_i}}$, making Equation \ref{hybrid_variance} the practical means of calculating the variance.

% Figure environment removed

\subsection{Analytic Expectation Values}
The $n$-qubit depolarizing channel in Equation \ref{depolarizing_channel} can equivalently be written as follows \cite{nielsen_quantum_2010}:
$$\mathcal{E}(\rho)=(1-\lambda) \rho + \lambda\frac{I}{2^n},$$
where $\lambda = (\frac{4^n}{4^n-1}) p$. This form greatly simplifies analytic calculations, and in fact the expectation value of a general observable $\hat{O}$ evolved under $\mathcal{E}$ can be calculated as follows:
\begin{equation}
    \expval{\hat{O}}_{\tilde{\rho}} = (1-\lambda) \operatorname{Tr}\left[O  \rho \right] + \frac{\lambda}{2^n} \operatorname{Tr}\left[O \right].
\end{equation}
From this form, we computed the analytic expectation values $\langle \hat{Z}^{(i)} \rangle$ and $\left\langle\hat{X}^{(2i)}\hat{X}^{(2i+1)}\right\rangle$ in Equations \ref{z_analytic} and \ref{xx_analytic} respectively.




\subsection{Ancilla-Based Algorithm}
For the depolarizing channel simulations, we compare with results from an ancilla-based simulation algorithm \cite{childs_hamiltonian_2012}, described as follows. Consider the random-unitary channel $\mathcal{E}(\rho)=\sum_i^m p_i\left(U_i \rho U_i^{\dagger}\right)$. First, encode the probabilities $p_i$ as amplitudes of $\lceil\log (\mathrm{m})\rceil$-many ancilla qubits. Suppose a gate $P(\bm{\theta})$ exists such that:
\begin{equation}
    P(\bm{\theta}) \ket{0^{\lceil\log (\mathrm{m})\rceil}} = \sum_i^m \sqrt{p_i} \ket{i}.
\end{equation}
Then, apply each unitary evolution $U_i$ to the target state $\ket{\psi}$ conditioned on the ancilla state $\ket{i}$. This accomplishes the desired effect of applying each unitary $U_i$ with probability $p_i$. The overall algorithm is displayed in circuit-form in Figure \ref{fig:ancillas}.

%merlin.mbs apsrev4-1.bst 2010-07-25 4.21a (PWD, AO, DPC) hacked
%Control: key (0)
%Control: author (0) dotless jnrlst
%Control: editor formatted (1) identically to author
%Control: production of article title (0) allowed
%Control: page (1) range
%Control: year (0) verbatim
%Control: production of eprint (0) enabled
\begin{thebibliography}{29}%
\makeatletter
\providecommand \@ifxundefined [1]{%
 \@ifx{#1\undefined}
}%
\providecommand \@ifnum [1]{%
 \ifnum #1\expandafter \@firstoftwo
 \else \expandafter \@secondoftwo
 \fi
}%
\providecommand \@ifx [1]{%
 \ifx #1\expandafter \@firstoftwo
 \else \expandafter \@secondoftwo
 \fi
}%
\providecommand \natexlab [1]{#1}%
\providecommand \enquote  [1]{``#1''}%
\providecommand \bibnamefont  [1]{#1}%
\providecommand \bibfnamefont [1]{#1}%
\providecommand \citenamefont [1]{#1}%
\providecommand \href@noop [0]{\@secondoftwo}%
\providecommand \href [0]{\begingroup \@sanitize@url \@href}%
\providecommand \@href[1]{\@@startlink{#1}\@@href}%
\providecommand \@@href[1]{\endgroup#1\@@endlink}%
\providecommand \@sanitize@url [0]{\catcode `\\12\catcode `\$12\catcode
  `\&12\catcode `\#12\catcode `\^12\catcode `\_12\catcode `\%12\relax}%
\providecommand \@@startlink[1]{}%
\providecommand \@@endlink[0]{}%
\providecommand \url  [0]{\begingroup\@sanitize@url \@url }%
\providecommand \@url [1]{\endgroup\@href {#1}{\urlprefix }}%
\providecommand \urlprefix  [0]{URL }%
\providecommand \Eprint [0]{\href }%
\providecommand \doibase [0]{http://dx.doi.org/}%
\providecommand \selectlanguage [0]{\@gobble}%
\providecommand \bibinfo  [0]{\@secondoftwo}%
\providecommand \bibfield  [0]{\@secondoftwo}%
\providecommand \translation [1]{[#1]}%
\providecommand \BibitemOpen [0]{}%
\providecommand \bibitemStop [0]{}%
\providecommand \bibitemNoStop [0]{.\EOS\space}%
\providecommand \EOS [0]{\spacefactor3000\relax}%
\providecommand \BibitemShut  [1]{\csname bibitem#1\endcsname}%
\let\auto@bib@innerbib\@empty
%</preamble>
\bibitem [{\citenamefont {Breuer}\ and\ \citenamefont
  {Petruccione}(2007)}]{breuer_theory_2007}%
  \BibitemOpen
  \bibfield  {author} {\bibinfo {author} {\bibfnamefont {Heinz-Peter}\
  \bibnamefont {Breuer}}\ and\ \bibinfo {author} {\bibfnamefont {Francesco}\
  \bibnamefont {Petruccione}},\ }\href@noop {} {\emph {\bibinfo {title} {The
  {Theory} of {Open} {Quantum} {Systems}}}}\ (\bibinfo  {publisher} {Oxford
  University Press},\ \bibinfo {address} {Oxford, New York},\ \bibinfo {year}
  {2007})\BibitemShut {NoStop}%
\bibitem [{\citenamefont {{Head-Marsden}}\ \emph {et~al.}(2020)\citenamefont
  {{Head-Marsden}}, \citenamefont {Flick}, \citenamefont {Ciccarino},\ and\
  \citenamefont {Narang}}]{Head-Marsden2020}%
  \BibitemOpen
  \bibfield  {author} {\bibinfo {author} {\bibfnamefont {Kade}\ \bibnamefont
  {{Head-Marsden}}}, \bibinfo {author} {\bibfnamefont {Johannes}\ \bibnamefont
  {Flick}}, \bibinfo {author} {\bibfnamefont {Christopher~J.}\ \bibnamefont
  {Ciccarino}}, \ and\ \bibinfo {author} {\bibfnamefont {Prineha}\ \bibnamefont
  {Narang}},\ }\bibfield  {title} {\enquote {\bibinfo {title} {Quantum
  {{Information}} and {{Algorithms}} for {{Correlated Quantum Matter}}},}\
  }\href {\doibase 10.1021/acs.chemrev.0c00620} {\bibfield  {journal} {\bibinfo
   {journal} {Chemical Reviews}\ } (\bibinfo {year} {2020}),\
  10.1021/acs.chemrev.0c00620}\BibitemShut {NoStop}%
\bibitem [{\citenamefont {Brown}\ \emph {et~al.}(2010)\citenamefont {Brown},
  \citenamefont {Munro},\ and\ \citenamefont {Kendon}}]{brown_using_2010}%
  \BibitemOpen
  \bibfield  {author} {\bibinfo {author} {\bibfnamefont {Katherine~L.}\
  \bibnamefont {Brown}}, \bibinfo {author} {\bibfnamefont {William~J.}\
  \bibnamefont {Munro}}, \ and\ \bibinfo {author} {\bibfnamefont {Vivien~M.}\
  \bibnamefont {Kendon}},\ }\bibfield  {title} {\enquote {\bibinfo {title}
  {Using {Quantum} {Computers} for {Quantum} {Simulation}},}\ }\href {\doibase
  10.3390/e12112268} {\bibfield  {journal} {\bibinfo  {journal} {Entropy}\
  }\textbf {\bibinfo {volume} {12}},\ \bibinfo {pages} {2268--2307} (\bibinfo
  {year} {2010})}\BibitemShut {NoStop}%
\bibitem [{\citenamefont {Paulsen}(2003)}]{paulsen_completely_2003}%
  \BibitemOpen
  \bibfield  {author} {\bibinfo {author} {\bibfnamefont {Vern}\ \bibnamefont
  {Paulsen}},\ }\href {\doibase 10.1017/CBO9780511546631} {\emph {\bibinfo
  {title} {Completely {Bounded} {Maps} and {Operator} {Algebras}}}},\ Cambridge
  {Studies} in {Advanced} {Mathematics}\ (\bibinfo  {publisher} {Cambridge
  University Press},\ \bibinfo {address} {Cambridge},\ \bibinfo {year}
  {2003})\BibitemShut {NoStop}%
\bibitem [{\citenamefont {Stinespring}(1955)}]{stinespring_positive_1955}%
  \BibitemOpen
  \bibfield  {author} {\bibinfo {author} {\bibfnamefont {W.~Forrest}\
  \bibnamefont {Stinespring}},\ }\bibfield  {title} {\enquote {\bibinfo {title}
  {Positive functions on c*-algebras},}\ }in\ \href {\doibase
  10.1090/S0002-9939-1955-0069403-4} {\emph {\bibinfo {booktitle} {Proceedings
  of the {American} {Mathematical} {Society}}}},\ Vol.~\bibinfo {volume} {6}\
  (\bibinfo {year} {1955})\ pp.\ \bibinfo {pages} {211--216}\BibitemShut
  {NoStop}%
\bibitem [{\citenamefont {Hu}\ \emph {et~al.}(2020)\citenamefont {Hu},
  \citenamefont {Xia},\ and\ \citenamefont {Kais}}]{hu_quantum_2020}%
  \BibitemOpen
  \bibfield  {author} {\bibinfo {author} {\bibfnamefont {Zixuan}\ \bibnamefont
  {Hu}}, \bibinfo {author} {\bibfnamefont {Rongxin}\ \bibnamefont {Xia}}, \
  and\ \bibinfo {author} {\bibfnamefont {Sabre}\ \bibnamefont {Kais}},\
  }\bibfield  {title} {\enquote {\bibinfo {title} {A quantum algorithm for
  evolving open quantum dynamics on quantum computing devices},}\ }\href
  {\doibase 10.1038/s41598-020-60321-x} {\bibfield  {journal} {\bibinfo
  {journal} {Scientific Reports}\ }\textbf {\bibinfo {volume} {10}},\ \bibinfo
  {pages} {3301} (\bibinfo {year} {2020})}\BibitemShut {NoStop}%
\bibitem [{\citenamefont {Langer}(1972)}]{langer_b_1972}%
  \BibitemOpen
  \bibfield  {author} {\bibinfo {author} {\bibfnamefont {H.}~\bibnamefont
  {Langer}},\ }\bibfield  {title} {\enquote {\bibinfo {title} {B. {Sz}.-{Nagy}
  and {C}. {Foias}, {Harmonic} {Analysis} of {Operators} on {Hilbert} {Space}.
  {VIII} + 387 {S}. {Budapest}/{Amsterdam}/{London} 1970. {Akadémiai}
  {Kiadó}/{North}-{Holland} {Publishing} {Company}},}\ }\href {\doibase
  10.1002/zamm.19720520821} {\bibfield  {journal} {\bibinfo  {journal} {ZAMM -
  Zeitschrift für Angewandte Mathematik und Mechanik}\ }\textbf {\bibinfo
  {volume} {52}},\ \bibinfo {pages} {501--501} (\bibinfo {year}
  {1972})}\BibitemShut {NoStop}%
\bibitem [{\citenamefont {Schlimgen}\ \emph {et~al.}(2021)\citenamefont
  {Schlimgen}, \citenamefont {{Head-Marsden}}, \citenamefont {Sager},
  \citenamefont {Narang},\ and\ \citenamefont {Mazziotti}}]{Schlimgen2021}%
  \BibitemOpen
  \bibfield  {author} {\bibinfo {author} {\bibfnamefont {Anthony~W.}\
  \bibnamefont {Schlimgen}}, \bibinfo {author} {\bibfnamefont {Kade}\
  \bibnamefont {{Head-Marsden}}}, \bibinfo {author} {\bibfnamefont {Leeann~M.}\
  \bibnamefont {Sager}}, \bibinfo {author} {\bibfnamefont {Prineha}\
  \bibnamefont {Narang}}, \ and\ \bibinfo {author} {\bibfnamefont {David~A.}\
  \bibnamefont {Mazziotti}},\ }\bibfield  {title} {\enquote {\bibinfo {title}
  {Quantum {{Simulation}} of {{Open Quantum Systems Using}} a {{Unitary
  Decomposition}} of {{Operators}}},}\ }\href {\doibase
  10.1103/PhysRevLett.127.270503} {\bibfield  {journal} {\bibinfo  {journal}
  {Physical Review Letters}\ }\textbf {\bibinfo {volume} {127}},\ \bibinfo
  {pages} {270503} (\bibinfo {year} {2021})},\ \Eprint
  {http://arxiv.org/abs/2106.12588} {arxiv:2106.12588} \BibitemShut {NoStop}%
\bibitem [{\citenamefont {Childs}\ and\ \citenamefont
  {Wiebe}(2012)}]{childs_hamiltonian_2012}%
  \BibitemOpen
  \bibfield  {author} {\bibinfo {author} {\bibfnamefont {Andrew~M.}\
  \bibnamefont {Childs}}\ and\ \bibinfo {author} {\bibfnamefont {Nathan}\
  \bibnamefont {Wiebe}},\ }\bibfield  {title} {\enquote {\bibinfo {title}
  {Hamiltonian simulation using linear combinations of unitary operations},}\
  }\href@noop {} {\bibfield  {journal} {\bibinfo  {journal} {Quantum
  Information \& Computation}\ }\textbf {\bibinfo {volume} {12}},\ \bibinfo
  {pages} {901--924} (\bibinfo {year} {2012})}\BibitemShut {NoStop}%
\bibitem [{\citenamefont {Suri}\ \emph {et~al.}(2022)\citenamefont {Suri},
  \citenamefont {Barreto}, \citenamefont {Hadfield}, \citenamefont {Wiebe},
  \citenamefont {Wudarski},\ and\ \citenamefont
  {Marshall}}]{suri_two-unitary_2022}%
  \BibitemOpen
  \bibfield  {author} {\bibinfo {author} {\bibfnamefont {Nishchay}\
  \bibnamefont {Suri}}, \bibinfo {author} {\bibfnamefont {Joseph}\ \bibnamefont
  {Barreto}}, \bibinfo {author} {\bibfnamefont {Stuart}\ \bibnamefont
  {Hadfield}}, \bibinfo {author} {\bibfnamefont {Nathan}\ \bibnamefont
  {Wiebe}}, \bibinfo {author} {\bibfnamefont {Filip}\ \bibnamefont {Wudarski}},
  \ and\ \bibinfo {author} {\bibfnamefont {Jeffrey}\ \bibnamefont {Marshall}},\
  }\href {\doibase 10.48550/arXiv.2207.10007} {\enquote {\bibinfo {title}
  {Two-{Unitary} {Decomposition} {Algorithm} and {Open} {Quantum} {System}
  {Simulation}},}\ } (\bibinfo {year} {2022}),\ \bibinfo {note}
  {arXiv:2207.10007 [physics, physics:quant-ph]}\BibitemShut {NoStop}%
\bibitem [{\citenamefont {Head-Marsden}\ \emph {et~al.}(2021)\citenamefont
  {Head-Marsden}, \citenamefont {Krastanov}, \citenamefont {Mazziotti},\ and\
  \citenamefont {Narang}}]{head-marsden_capturing_2021}%
  \BibitemOpen
  \bibfield  {author} {\bibinfo {author} {\bibfnamefont {Kade}\ \bibnamefont
  {Head-Marsden}}, \bibinfo {author} {\bibfnamefont {Stefan}\ \bibnamefont
  {Krastanov}}, \bibinfo {author} {\bibfnamefont {David~A.}\ \bibnamefont
  {Mazziotti}}, \ and\ \bibinfo {author} {\bibfnamefont {Prineha}\ \bibnamefont
  {Narang}},\ }\bibfield  {title} {\enquote {\bibinfo {title} {Capturing
  non-{Markovian} dynamics on near-term quantum computers},}\ }\href {\doibase
  10.1103/PhysRevResearch.3.013182} {\bibfield  {journal} {\bibinfo  {journal}
  {Physical Review Research}\ }\textbf {\bibinfo {volume} {3}},\ \bibinfo
  {pages} {013182} (\bibinfo {year} {2021})}\BibitemShut {NoStop}%
\bibitem [{\citenamefont {Sudarshan}\ \emph {et~al.}(1961)\citenamefont
  {Sudarshan}, \citenamefont {Mathews},\ and\ \citenamefont
  {Rau}}]{sudarshan_stochastic_1961}%
  \BibitemOpen
  \bibfield  {author} {\bibinfo {author} {\bibfnamefont {E.~C.~G.}\
  \bibnamefont {Sudarshan}}, \bibinfo {author} {\bibfnamefont {P.~M.}\
  \bibnamefont {Mathews}}, \ and\ \bibinfo {author} {\bibfnamefont
  {Jayaseetha}\ \bibnamefont {Rau}},\ }\bibfield  {title} {\enquote {\bibinfo
  {title} {Stochastic {Dynamics} of {Quantum}-{Mechanical} {Systems}},}\ }\href
  {\doibase 10.1103/PhysRev.121.920} {\bibfield  {journal} {\bibinfo  {journal}
  {Physical Review}\ }\textbf {\bibinfo {volume} {121}},\ \bibinfo {pages}
  {920--924} (\bibinfo {year} {1961})}\BibitemShut {NoStop}%
\bibitem [{\citenamefont {Hellwig}\ and\ \citenamefont
  {Kraus}(1970)}]{hellwig_operations_1970}%
  \BibitemOpen
  \bibfield  {author} {\bibinfo {author} {\bibfnamefont {K.~E.}\ \bibnamefont
  {Hellwig}}\ and\ \bibinfo {author} {\bibfnamefont {K.}~\bibnamefont
  {Kraus}},\ }\bibfield  {title} {\enquote {\bibinfo {title} {Operations and
  measurements. {II}},}\ }\href {\doibase 10.1007/BF01646620} {\bibfield
  {journal} {\bibinfo  {journal} {Communications in Mathematical Physics}\
  }\textbf {\bibinfo {volume} {16}},\ \bibinfo {pages} {142--147} (\bibinfo
  {year} {1970})}\BibitemShut {NoStop}%
\bibitem [{\citenamefont {Mendl}\ and\ \citenamefont
  {Wolf}(2009)}]{mendl_unital_2009}%
  \BibitemOpen
  \bibfield  {author} {\bibinfo {author} {\bibfnamefont {Christian~B.}\
  \bibnamefont {Mendl}}\ and\ \bibinfo {author} {\bibfnamefont {Michael~M.}\
  \bibnamefont {Wolf}},\ }\bibfield  {title} {\enquote {\bibinfo {title}
  {Unital {Quantum} {Channels} – {Convex} {Structure} and {Revivals} of
  {Birkhoff}’s {Theorem}},}\ }\href {\doibase 10.1007/s00220-009-0824-2}
  {\bibfield  {journal} {\bibinfo  {journal} {Communications in Mathematical
  Physics}\ }\textbf {\bibinfo {volume} {289}},\ \bibinfo {pages} {1057--1086}
  (\bibinfo {year} {2009})}\BibitemShut {NoStop}%
\bibitem [{\citenamefont {Dür}\ \emph {et~al.}(2005)\citenamefont {Dür},
  \citenamefont {Hein}, \citenamefont {Cirac},\ and\ \citenamefont
  {Briegel}}]{dur_standard_2005}%
  \BibitemOpen
  \bibfield  {author} {\bibinfo {author} {\bibfnamefont {W.}~\bibnamefont
  {Dür}}, \bibinfo {author} {\bibfnamefont {M.}~\bibnamefont {Hein}}, \bibinfo
  {author} {\bibfnamefont {J.~I.}\ \bibnamefont {Cirac}}, \ and\ \bibinfo
  {author} {\bibfnamefont {H.-J.}\ \bibnamefont {Briegel}},\ }\bibfield
  {title} {\enquote {\bibinfo {title} {Standard forms of noisy quantum
  operations via depolarization},}\ }\href {\doibase
  10.1103/PhysRevA.72.052326} {\bibfield  {journal} {\bibinfo  {journal}
  {Physical Review A}\ }\textbf {\bibinfo {volume} {72}},\ \bibinfo {pages}
  {052326} (\bibinfo {year} {2005})}\BibitemShut {NoStop}%
\bibitem [{\citenamefont {Cai}\ and\ \citenamefont
  {Benjamin}(2019)}]{cai_constructing_2019}%
  \BibitemOpen
  \bibfield  {author} {\bibinfo {author} {\bibfnamefont {Zhenyu}\ \bibnamefont
  {Cai}}\ and\ \bibinfo {author} {\bibfnamefont {Simon~C.}\ \bibnamefont
  {Benjamin}},\ }\bibfield  {title} {\enquote {\bibinfo {title} {Constructing
  {Smaller} {Pauli} {Twirling} {Sets} for {Arbitrary} {Error} {Channels}},}\
  }\href {\doibase 10.1038/s41598-019-46722-7} {\bibfield  {journal} {\bibinfo
  {journal} {Scientific Reports}\ }\textbf {\bibinfo {volume} {9}},\ \bibinfo
  {pages} {11281} (\bibinfo {year} {2019})}\BibitemShut {NoStop}%
\bibitem [{\citenamefont {Wallman}\ and\ \citenamefont
  {Emerson}(2016)}]{wallman_noise_2016}%
  \BibitemOpen
  \bibfield  {author} {\bibinfo {author} {\bibfnamefont {Joel~J.}\ \bibnamefont
  {Wallman}}\ and\ \bibinfo {author} {\bibfnamefont {Joseph}\ \bibnamefont
  {Emerson}},\ }\bibfield  {title} {\enquote {\bibinfo {title} {Noise tailoring
  for scalable quantum computation via randomized compiling},}\ }\href
  {\doibase 10.1103/PhysRevA.94.052325} {\bibfield  {journal} {\bibinfo
  {journal} {Physical Review A}\ }\textbf {\bibinfo {volume} {94}},\ \bibinfo
  {pages} {052325} (\bibinfo {year} {2016})}\BibitemShut {NoStop}%
\bibitem [{\citenamefont {Moueddene}\ \emph {et~al.}(2020)\citenamefont
  {Moueddene}, \citenamefont {Khammassi}, \citenamefont {Bertels},\ and\
  \citenamefont {Almudever}}]{moueddene_realistic_2020}%
  \BibitemOpen
  \bibfield  {author} {\bibinfo {author} {\bibfnamefont {Ahmed~Abid}\
  \bibnamefont {Moueddene}}, \bibinfo {author} {\bibfnamefont {Nader}\
  \bibnamefont {Khammassi}}, \bibinfo {author} {\bibfnamefont {Koen}\
  \bibnamefont {Bertels}}, \ and\ \bibinfo {author} {\bibfnamefont {Carmen~G.}\
  \bibnamefont {Almudever}},\ }\bibfield  {title} {\enquote {\bibinfo {title}
  {Realistic simulation of quantum computation using unitary and measurement
  channels},}\ }\href {\doibase 10.1103/PhysRevA.102.052608} {\bibfield
  {journal} {\bibinfo  {journal} {Physical Review A}\ }\textbf {\bibinfo
  {volume} {102}},\ \bibinfo {pages} {052608} (\bibinfo {year}
  {2020})}\BibitemShut {NoStop}%
\bibitem [{\citenamefont {Suzuki}\ \emph {et~al.}(2021)\citenamefont {Suzuki},
  \citenamefont {Kawase}, \citenamefont {Masumura}, \citenamefont {Hiraga},
  \citenamefont {Nakadai}, \citenamefont {Chen}, \citenamefont {Nakanishi},
  \citenamefont {Mitarai}, \citenamefont {Imai}, \citenamefont {Tamiya},
  \citenamefont {Yamamoto}, \citenamefont {Yan}, \citenamefont {Kawakubo},
  \citenamefont {Nakagawa}, \citenamefont {Ibe}, \citenamefont {Zhang},
  \citenamefont {Yamashita}, \citenamefont {Yoshimura}, \citenamefont
  {Hayashi},\ and\ \citenamefont {Fujii}}]{suzuki_qulacs_2021}%
  \BibitemOpen
  \bibfield  {author} {\bibinfo {author} {\bibfnamefont {Yasunari}\
  \bibnamefont {Suzuki}}, \bibinfo {author} {\bibfnamefont {Yoshiaki}\
  \bibnamefont {Kawase}}, \bibinfo {author} {\bibfnamefont {Yuya}\ \bibnamefont
  {Masumura}}, \bibinfo {author} {\bibfnamefont {Yuria}\ \bibnamefont
  {Hiraga}}, \bibinfo {author} {\bibfnamefont {Masahiro}\ \bibnamefont
  {Nakadai}}, \bibinfo {author} {\bibfnamefont {Jiabao}\ \bibnamefont {Chen}},
  \bibinfo {author} {\bibfnamefont {Ken~M.}\ \bibnamefont {Nakanishi}},
  \bibinfo {author} {\bibfnamefont {Kosuke}\ \bibnamefont {Mitarai}}, \bibinfo
  {author} {\bibfnamefont {Ryosuke}\ \bibnamefont {Imai}}, \bibinfo {author}
  {\bibfnamefont {Shiro}\ \bibnamefont {Tamiya}}, \bibinfo {author}
  {\bibfnamefont {Takahiro}\ \bibnamefont {Yamamoto}}, \bibinfo {author}
  {\bibfnamefont {Tennin}\ \bibnamefont {Yan}}, \bibinfo {author}
  {\bibfnamefont {Toru}\ \bibnamefont {Kawakubo}}, \bibinfo {author}
  {\bibfnamefont {Yuya~O.}\ \bibnamefont {Nakagawa}}, \bibinfo {author}
  {\bibfnamefont {Yohei}\ \bibnamefont {Ibe}}, \bibinfo {author} {\bibfnamefont
  {Youyuan}\ \bibnamefont {Zhang}}, \bibinfo {author} {\bibfnamefont
  {Hirotsugu}\ \bibnamefont {Yamashita}}, \bibinfo {author} {\bibfnamefont
  {Hikaru}\ \bibnamefont {Yoshimura}}, \bibinfo {author} {\bibfnamefont
  {Akihiro}\ \bibnamefont {Hayashi}}, \ and\ \bibinfo {author} {\bibfnamefont
  {Keisuke}\ \bibnamefont {Fujii}},\ }\bibfield  {title} {\enquote {\bibinfo
  {title} {Qulacs: a fast and versatile quantum circuit simulator for research
  purpose},}\ }\href {\doibase 10.22331/q-2021-10-06-559} {\bibfield  {journal}
  {\bibinfo  {journal} {Quantum}\ }\textbf {\bibinfo {volume} {5}},\ \bibinfo
  {pages} {559} (\bibinfo {year} {2021})},\ \bibinfo {note} {arXiv:2011.13524
  [physics, physics:quant-ph]}\BibitemShut {NoStop}%
\bibitem [{\citenamefont {Terhal}(2015)}]{terhal_quantum_2015}%
  \BibitemOpen
  \bibfield  {author} {\bibinfo {author} {\bibfnamefont {Barbara~M.}\
  \bibnamefont {Terhal}},\ }\bibfield  {title} {\enquote {\bibinfo {title}
  {Quantum {Error} {Correction} for {Quantum} {Memories}},}\ }\href {\doibase
  10.1103/RevModPhys.87.307} {\bibfield  {journal} {\bibinfo  {journal}
  {Reviews of Modern Physics}\ }\textbf {\bibinfo {volume} {87}},\ \bibinfo
  {pages} {307--346} (\bibinfo {year} {2015})},\ \bibinfo {note}
  {arXiv:1302.3428 [quant-ph]}\BibitemShut {NoStop}%
\bibitem [{\citenamefont {Temme}\ \emph {et~al.}(2017)\citenamefont {Temme},
  \citenamefont {Bravyi},\ and\ \citenamefont {Gambetta}}]{temme_error_2017}%
  \BibitemOpen
  \bibfield  {author} {\bibinfo {author} {\bibfnamefont {Kristan}\ \bibnamefont
  {Temme}}, \bibinfo {author} {\bibfnamefont {Sergey}\ \bibnamefont {Bravyi}},
  \ and\ \bibinfo {author} {\bibfnamefont {Jay~M.}\ \bibnamefont {Gambetta}},\
  }\bibfield  {title} {\enquote {\bibinfo {title} {Error {Mitigation} for
  {Short}-{Depth} {Quantum} {Circuits}},}\ }\href {\doibase
  10.1103/PhysRevLett.119.180509} {\bibfield  {journal} {\bibinfo  {journal}
  {Physical Review Letters}\ }\textbf {\bibinfo {volume} {119}},\ \bibinfo
  {pages} {180509} (\bibinfo {year} {2017})}\BibitemShut {NoStop}%
\bibitem [{\citenamefont {Rost}\ \emph {et~al.}(2020)\citenamefont {Rost},
  \citenamefont {Jones}, \citenamefont {Vyushkova}, \citenamefont {Ali},
  \citenamefont {Cullip}, \citenamefont {Vyushkov},\ and\ \citenamefont
  {Nabrzyski}}]{rost_simulation_2020}%
  \BibitemOpen
  \bibfield  {author} {\bibinfo {author} {\bibfnamefont {Brian}\ \bibnamefont
  {Rost}}, \bibinfo {author} {\bibfnamefont {Barbara}\ \bibnamefont {Jones}},
  \bibinfo {author} {\bibfnamefont {Mariya}\ \bibnamefont {Vyushkova}},
  \bibinfo {author} {\bibfnamefont {Aaila}\ \bibnamefont {Ali}}, \bibinfo
  {author} {\bibfnamefont {Charlotte}\ \bibnamefont {Cullip}}, \bibinfo
  {author} {\bibfnamefont {Alexander}\ \bibnamefont {Vyushkov}}, \ and\
  \bibinfo {author} {\bibfnamefont {Jarek}\ \bibnamefont {Nabrzyski}},\ }\href
  {http://arxiv.org/abs/2001.00794} {\enquote {\bibinfo {title} {Simulation of
  {Thermal} {Relaxation} in {Spin} {Chemistry} {Systems} on a {Quantum}
  {Computer} {Using} {Inherent} {Qubit} {Decoherence}},}\ } (\bibinfo {year}
  {2020}),\ \bibinfo {note} {arXiv:2001.00794 [physics,
  physics:quant-ph]}\BibitemShut {NoStop}%
\bibitem [{\citenamefont {Tolunay}\ \emph {et~al.}(2023)\citenamefont
  {Tolunay}, \citenamefont {Liepuoniute}, \citenamefont {Vyushkova},\ and\
  \citenamefont {Jones}}]{tolunayHamiltonianSimulationQuantum2023}%
  \BibitemOpen
  \bibfield  {author} {\bibinfo {author} {\bibfnamefont {Meltem}\ \bibnamefont
  {Tolunay}}, \bibinfo {author} {\bibfnamefont {Ieva}\ \bibnamefont
  {Liepuoniute}}, \bibinfo {author} {\bibfnamefont {Mariya}\ \bibnamefont
  {Vyushkova}}, \ and\ \bibinfo {author} {\bibfnamefont {Barbara~A.}\
  \bibnamefont {Jones}},\ }\bibfield  {title} {\enquote {\bibinfo {title}
  {Hamiltonian simulation of quantum beats in radical pairs undergoing thermal
  relaxation on near-term quantum computers},}\ }\href {\doibase
  10.1039/D3CP00276D} {\bibfield  {journal} {\bibinfo  {journal} {Phys. Chem.
  Chem. Phys.}\ }\textbf {\bibinfo {volume} {25}},\ \bibinfo {pages}
  {15115--15134} (\bibinfo {year} {2023})}\BibitemShut {NoStop}%
\bibitem [{\citenamefont {Arrasmith}\ \emph {et~al.}(2020)\citenamefont
  {Arrasmith}, \citenamefont {Cincio}, \citenamefont {Somma},\ and\
  \citenamefont {Coles}}]{arrasmith_operator_2020}%
  \BibitemOpen
  \bibfield  {author} {\bibinfo {author} {\bibfnamefont {Andrew}\ \bibnamefont
  {Arrasmith}}, \bibinfo {author} {\bibfnamefont {Lukasz}\ \bibnamefont
  {Cincio}}, \bibinfo {author} {\bibfnamefont {Rolando~D.}\ \bibnamefont
  {Somma}}, \ and\ \bibinfo {author} {\bibfnamefont {Patrick~J.}\ \bibnamefont
  {Coles}},\ }\href {\doibase 10.48550/arXiv.2004.06252} {\enquote {\bibinfo
  {title} {Operator {Sampling} for {Shot}-frugal {Optimization} in
  {Variational} {Algorithms}},}\ } (\bibinfo {year} {2020}),\ \bibinfo {note}
  {arXiv:2004.06252 [quant-ph]}\BibitemShut {NoStop}%
\bibitem [{\citenamefont {Nielsen}\ and\ \citenamefont
  {Chuang}(2010)}]{nielsen_quantum_2010}%
  \BibitemOpen
  \bibfield  {author} {\bibinfo {author} {\bibfnamefont {Michael~A.}\
  \bibnamefont {Nielsen}}\ and\ \bibinfo {author} {\bibfnamefont {Isaac~L.}\
  \bibnamefont {Chuang}},\ }\href {\doibase 10.1017/CBO9780511976667} {\enquote
  {\bibinfo {title} {Quantum {Computation} and {Quantum} {Information}: 10th
  {Anniversary} {Edition}},}\ } (\bibinfo {year} {2010})\BibitemShut {NoStop}%
\bibitem [{\citenamefont
  {Gottesman}(1998{\natexlab{a}})}]{gottesman_theory_1998}%
  \BibitemOpen
  \bibfield  {author} {\bibinfo {author} {\bibfnamefont {Daniel}\ \bibnamefont
  {Gottesman}},\ }\bibfield  {title} {\enquote {\bibinfo {title} {Theory of
  fault-tolerant quantum computation},}\ }\href {\doibase
  10.1103/PhysRevA.57.127} {\bibfield  {journal} {\bibinfo  {journal} {Physical
  Review A}\ }\textbf {\bibinfo {volume} {57}},\ \bibinfo {pages} {127--137}
  (\bibinfo {year} {1998}{\natexlab{a}})}\BibitemShut {NoStop}%
\bibitem [{\citenamefont
  {Gottesman}(1998{\natexlab{b}})}]{gottesman_heisenberg_1998}%
  \BibitemOpen
  \bibfield  {author} {\bibinfo {author} {\bibfnamefont {Daniel}\ \bibnamefont
  {Gottesman}},\ }\href {\doibase 10.48550/arXiv.quant-ph/9807006} {\enquote
  {\bibinfo {title} {The {Heisenberg} {Representation} of {Quantum}
  {Computers}},}\ } (\bibinfo {year} {1998}{\natexlab{b}}),\ \bibinfo {note}
  {arXiv:quant-ph/9807006}\BibitemShut {NoStop}%
\bibitem [{\citenamefont {Campbell}(2019)}]{campbell_random_2019}%
  \BibitemOpen
  \bibfield  {author} {\bibinfo {author} {\bibfnamefont {Earl}\ \bibnamefont
  {Campbell}},\ }\bibfield  {title} {\enquote {\bibinfo {title} {Random
  {Compiler} for {Fast} {Hamiltonian} {Simulation}},}\ }\href {\doibase
  10.1103/PhysRevLett.123.070503} {\bibfield  {journal} {\bibinfo  {journal}
  {Physical Review Letters}\ }\textbf {\bibinfo {volume} {123}},\ \bibinfo
  {pages} {070503} (\bibinfo {year} {2019})}\BibitemShut {NoStop}%
\bibitem [{\citenamefont {Kim}\ \emph {et~al.}(2023)\citenamefont {Kim},
  \citenamefont {Wood}, \citenamefont {Yoder}, \citenamefont {Merkel},
  \citenamefont {Gambetta}, \citenamefont {Temme},\ and\ \citenamefont
  {Kandala}}]{kimScalableErrorMitigation2023}%
  \BibitemOpen
  \bibfield  {author} {\bibinfo {author} {\bibfnamefont {Youngseok}\
  \bibnamefont {Kim}}, \bibinfo {author} {\bibfnamefont {Christopher~J.}\
  \bibnamefont {Wood}}, \bibinfo {author} {\bibfnamefont {Theodore~J.}\
  \bibnamefont {Yoder}}, \bibinfo {author} {\bibfnamefont {Seth~T.}\
  \bibnamefont {Merkel}}, \bibinfo {author} {\bibfnamefont {Jay~M.}\
  \bibnamefont {Gambetta}}, \bibinfo {author} {\bibfnamefont {Kristan}\
  \bibnamefont {Temme}}, \ and\ \bibinfo {author} {\bibfnamefont {Abhinav}\
  \bibnamefont {Kandala}},\ }\bibfield  {title} {\enquote {\bibinfo {title}
  {Scalable error mitigation for noisy quantum circuits produces competitive
  expectation values},}\ }\href {\doibase 10.1038/s41567-022-01914-3}
  {\bibfield  {journal} {\bibinfo  {journal} {Nat. Phys.}\ }\textbf {\bibinfo
  {volume} {19}},\ \bibinfo {pages} {752--759} (\bibinfo {year}
  {2023})}\BibitemShut {NoStop}%
\end{thebibliography}%
\end{document}
