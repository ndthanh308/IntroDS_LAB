\newtheorem{thm}{Theorem}%[section]
\newtheorem{cor}[thm]{Corollary}
\newtheorem{lem}[thm]{Lemma}
\newtheorem{claim}[thm]{Claim}
\newtheorem{prop}[thm]{Proposition}
\newtheorem{defin}[thm]{Definition}
\newtheorem{defins}[thm]{Definitions}
\newtheorem{probdef}[thm]{Problem Definition}
\newtheorem{assume}[thm]{Assumption}

%\theoremstyle{remark}
\theoremstyle{definition}
\newtheorem{rem}[thm]{Remark}
\newtheorem{xlist}[thm]{List}



\newcommand{\RR}{ \mathbb{R} }
\newcommand{\NN}{\mathbb{N}}
\newcommand{\CC}{\mathbb{C}}
\newcommand{\ZZ}{\mathbb{Z}}

\newcommand{\oneover}[1]{\frac{1}{#1}}
\newcommand{\spaceo}{\hspace{2 mm}}
%\newcommand{\setsep}{ \spaceo | \spaceo}
\newcommand{\half}{\frac{1}{2}}
\newcommand{\textgoth}[1]{\mathcal{#1}}
\newcommand{\Prob}[1]{\mathbb{P}\left( #1 \right)}
\newcommand{\Probb}[1]{\mathbb{P}\left[ #1 \right]}
\newcommand{\Probu}[2]{\mathbb{P}_{#1}\left( #2 \right)}
\newcommand{\Probuu}[3]{\mathbb{P}_{#1}^{#2}\left( #3 \right)}
\newcommand{\Probsubidx}[2]{\mathbb{P}_{#1}\left( #2 \right)}
\newcommand{\sym}{~}
\newcommand{\argmax}{\operatornamewithlimits{argmax}}

\newcommand{\Abs}[1]{\left| #1 \right|}
\newcommand{\Set}[1]{\left\{ #1 \right\}}
\newcommand{\Brack}[1]{\left( #1 \right)}
\newcommand{\BBrack}[1]{\left\{ #1 \right\}}
\newcommand{\SqBrack}[1]{\left[ #1 \right]}
\newcommand{\inner}[2]{\left< #1 , #2 \right>}
\newcommand{\sqinner}[2]{\left[ #1 , #2 \right]}
\newcommand{\tens}{\otimes}
\newcommand{\Exp}[1]{ \mathbb{E} #1}
\newcommand{\Expsubidx}[2]{ \mathbb{E}_{#1} #2}
\newcommand{\ExpB}[1]{ \mathbb{E} \Brack{#1}}
\newcommand{\ExpsubidxB}[2]{ \mathbb{E}_{#1} \Brack{#2}}


\newcommand{\norm}[1]{\left\|#1\right\|}
\newcommand{\norms}[1]{\left|#1\right|}
\newcommand{\enorm}[1]{\left\|#1\right\|_2}

\newcommand{\normsup}[1]{\norm{#1}_{\infty}}
\newcommand{\normssup}[1]{\norms{#1}_{\infty}}
\newcommand{\normLtwo}[1]{\norm{#1}_{L_2}}

\newcommand{\normop}[1]{\left\|#1\right\|_{op}}
\newcommand{\normopnuc}[1]{\left\|#1\right\|_{nuc}}
\newcommand{\normophs}[1]{\left\|#1\right\|_{HS}}

\newcommand{\tr}{tr}
\newcommand{\Ind}[1]{ \mathbbm{1}_{\Set{#1}} }
\newcommand{\Indnp}{ \mathbbm{1} }
\newcommand{\Indnb}[1]{ \mathbbm{1}_{#1} }
\newcommand{\eps}{\varepsilon}
\newcommand{\Sbi}{S^{\backslash i}}

\DeclareMathOperator*{\argmin}{\arg\!\min} 

\newcommand{\quater}{\frac{1}{4}}

\newcommand{\groundX}{\mathcal{X}}
\newcommand{\groundD}{\mathcal{D}}

\newcommand{\disjcup}{\mathop{\dot{\bigcup}}}


\newlength{\dhatheight}
\newcommand{\doublehat}[1]{%
    \settoheight{\dhatheight}{\ensuremath{\hat{#1}}}%
    \addtolength{\dhatheight}{-0.35ex}%
    \hat{\vphantom{\rule{1pt}{\dhatheight}}%
    \smash{\hat{#1}}}}

\newcommand{\mcF}{\mathcal{F}}
\newcommand{\mc}[1]{\mathcal{#1}}

\newcommand{\bsigma}{\bar{\sigma}}
\newcommand{\expzz}[1]{\exp\Brack{#1}}
\newcommand{\expzn}[1]{e^{#1}}


\newcommand\givenbase[1][]{\:#1\vert\:}
\newcommand{\given}{\vert}
\newcommand{\givenx}{\givenbase[\Big]}

\newcommand{\setsep}{ \spaceo \vert \spaceo}
\newcommand{\setsepx}{ \spaceo \givenbase[\Big] \spaceo}

\DeclarePairedDelimiter{\ceil}{\lceil}{\rceil}

\newcommand{\eRad}[1]{\mathfrak{R}\left(#1\right)}
\newcommand{\eRadidx}[2]{\mathfrak{R}_{#1}\left(#2\right)}
\newcommand{\eRadn}[1]{\eRadidx{n}{#1}}

\newcommand{\lspan}[1]{span\Set{#1}}
\newcommand{\ltr}[1]{tr{#1}}

\newcommand{\grad}{\nabla}


\newif\ifimagesshow
%\imagesshowtrue
\imagesshowfalse

