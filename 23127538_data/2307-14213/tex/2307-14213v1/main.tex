%\documentclass[a4paper, 10 pt, conference]{ieeeconf} 
%\IEEEoverridecommandlockouts                             
%\usepackage{color}
%\overrideIEEEmargins
%\newcommand\nsu[1]{\textcolor{cyan}{#1}}
%\newcommand\edits[1]{\textcolor{black}{#1}}
%\overrideIEEEmargins
%\usepackage{graphicx} % for pdf, bitmapped graphics files
%\usepackage{mathtools} 
%\usepackage{siunitx}
%\addtolength{\topmargin}{50pt}
%\usepackage[backend=biber, style=ieee,maxbibnames=99]{biblatex}
\documentclass[a4paper, 10pt, conference]{ieeeconf}
\IEEEoverridecommandlockouts                             
\usepackage{color}
\overrideIEEEmargins
\usepackage{graphicx} % for pdf, bitmapped graphics files
\usepackage{mathtools} 
\usepackage{siunitx}
\addtolength{\topmargin}{50pt}

\title{\LARGE \bf Soft Air Pocket Force Sensors for Large Scale Flexible Robots}

\author{Michael R. Mitchell, Ciera McFarland, and Margaret M. Coad
\thanks{The authors are with the Department of Aerospace and Mechanical Engineering, University of Notre Dame, Notre Dame, IN 46556, USA. {\tt\small \{mmitch23,cmcfarl2,mcoad\}@nd.edu}}%
}

\begin{document}

\maketitle
\thispagestyle{empty}
\pagestyle{empty}

\begin{abstract}
Flexible robots have advantages over rigid robots in their ability to conform physically to their environment and to form a wide variety of shapes. Sensing the force applied by or to flexible robots is useful for both navigation and manipulation tasks, but it is challenging due to the need for the sensors to withstand the robots' shape change without encumbering their functionality. Also, for robots with long or large bodies, the number of sensors required to cover the entire surface area of the robot body can be prohibitive due to high cost and complexity. We present a novel soft air pocket force sensor that is highly flexible, lightweight, relatively inexpensive, and easily scalable to various sizes. Our sensor produces a change in internal pressure that is linear with the applied force. We present results of experimental testing of how uncontrollable factors (contact location and contact area) and controllable factors (initial internal pressure, thickness, size, and number of interior seals) affect the sensitivity. We demonstrate our sensor applied to a vine robot---a soft inflatable robot that ``grows" from the tip via eversion---and we show that the robot can successfully grow and steer towards an object with which it senses contact. 
\end{abstract}

\section{Introduction}
Soft and flexible robots have advantages over traditional rigid robots in applications where the ability to conform physically to the environment and undergo large shape changes is critical. For example, a flexible continuum robot can conform to the shape of the nasal cavity to reach the brain and resect a tumor~\cite{burgner2013telerobotic}, and a soft inflatable truss robot can change its shape from a deployable package to locomote and manipulate objects~\cite{usevitch2020untethered}. While rigid robots typically require touch sensing to achieve safe and effective physical interaction, the passive or ``embodied" softness of flexible robots often allows them to interact physically with their environment in a desirable way without requiring the use of touch sensors~\cite{hao2016universal}. Nevertheless, the ability to sense the magnitude and direction of contacts along a flexible robot's surface is useful for tasks such as active adjustment of the force applied by the robot during manipulation, updates to the robot's control based on the contact state during navigation, or measurement of environment properties~\cite{shih2020electronic}.

The integration of touch sensors into soft and flexible robots is challenging, due to the need for the sensors to withstand large shape changes and sense information from an infinite-degree-of-freedom contact state. Estimated contact forces can be determined by comparing the measured robot shape to a model of the expected robot shape given known actuation inputs~\cite{thuruthel2019soft}, but the accuracy of this approach depends on having good models and data processing algorithms. Various touch sensing skins that sense force directly have been developed for soft and flexible robots, using working principles such as resistance change of liquid metal traces upon contact~\cite{dickey2017stretchable}, but they tend to be challenging to fabricate and integrate over large areas. Air pressure sensing is a promising technique for touch sensing of soft and flexible robots, because it allows the creation of lightweight sensors that provide a straightforward relationship between contact force and sensed pressure and can be integrated over large areas. Soft air pressure sensors have been demonstrated for rigid robot fingertips~\cite{kim20153d, alspach2018design}, arm modules~\cite{ alspach2018design, kim2018soft}, and skins~\cite{gruebele2021stretchable}, as well as for single-degree-of-freedom sensor-actuator modules~\cite{buso2020soft, JonesIROS2022}, but they have not yet been adapted for touch sensing in a large scale flexible robot.

% Figure environment removed

Vine robots~\cite{blumenschein_design_2020}, soft inflatable robots that ``grow" from the tip, are one type of soft flexible robot with a large surface area over which touch sensing is desirable. These robots are made of a tube of thin, flexible fluid-tight material that is folded inside of itself in its pre-grown state. To achieve growth, air pressure is typically used to inflate the robot and turn its body inside-out at the tip (called ``eversion"). These robots can extend to at least 72~m long~\cite{hawkes_soft_2017} and have been used to navigate confined spaces such as rocky tunnels in an archeological site~\cite{coad2019vine}. Adding touch sensing would allow these robots to navigate spaces more intelligently, manipulate objects with their whole body, and sense information about the stiffness properties of their environment. Thus far, distributed orientation, temperature, and humidity sensors have been integrated onto a vine robot using flexible sensor bands~\cite{GruebeleRoboSoft2021}, and tactile perception has been achieved using resistive curvature sensors to measure the robot's shape and calculate the applied force~\cite{BryantIROS2022}, but direct force sensing has not been implemented on a vine robot.

In this work, we explore the capabilities of air pressure-based sensing to enable touch sensing for large scale flexible robots such as vine robots (Fig.~\ref{glamorshot}). We present a novel soft force sensor made of a sealed air pocket containing an internal air pressure sensor. Our force sensor is fabricated by sealing a largely non-stretchable thermoplastic membrane at the edges using an impulse heat sealer, with partial seals throughout its body to increase its flexibility. Because it is made of a thin membrane filled with air, the sensor is lightweight and feasible for integration onto inflatable robots. The sensor's cost is determined primarily by the cost of the internal air pressure sensor, making it relatively inexpensive. By adjusting the dimensions of the membrane, the sensor can be scaled up or down as desired to cover a large scale robot with the desired resolution. In the sections that follow, we describe the design and fabrication of the sensor, the results of experiments showing the effect of various factors on the sensitivity, and a demonstration of the sensors sensing contact of a vine robot while it grows and steers.

% Enabling robots to easily and accurately sense contact with their environment is critical for intelligent navigation through a space. This is especially important for soft robots, as they are able to contact the environment with minimal damage to themselves and the environment. Adding contact sensing would enhance their ability to autonomously navigate a space. This is based on the movement of plants in response to touch \cite{braam2005touch}. 
% Previously, this idea has been applied to traditional rigid robots, where a soft material and contact sensor covering a rigid robot forced the rigid link to either stop or move away from the point of contact if object contact was detected \cite{kim2018soft}. 
% A similar idea was applied to ensure safe interactions between a rigid robot and an object during manipulation tasks \cite{gruebele2021stretchable},\cite{kim20153d}. 
% Soft robots have also utilized soft sensors in order to have safer interactions with their environments and objects within them \cite{buso2020soft},\cite{alspach2018design}. For force sensing, soft, fluid-filled sensors have been previously developed to respond to pressure \cite{JonesIROS2022}. 
% We want to use the versatility in soft sensors and apply them to large-scale, flexible robots. In this paper, we use vine robots as an example of a large-scale flexible robot that our sensors can be used for.

% Vine robots are soft, pneumatic robots that lengthen from their tips in order to navigate a space \cite{blumenschein_design_2020}. As they are inflatable and do not slide relative to the ground, they are particularly skilled at traversing a space with minimal damage and fitting through small gaps \cite{hawkes_soft_2017}. Vine robots steer using series pouch motors which shorten when inflated and cause the vine robot to turn in the direction of the inflated series pouch motor \cite{coad2019vine}. We envision the vine robot moving through a space and using contact sensing to move either toward or away from that object based on a given task. This would allow them to draw support from the environment where necessary while also avoiding harmful contact with fragile objects. 

% The primary sensor that has been applied to vine robots in the past is a tip-mounted camera \cite{hawkes_soft_2017}, and a variety of mounts have been used to do so \cite{jeong2020tip}. Vine robots can be tele-operated \cite{coad2019vine}, which has previously made the camera a crucial sensor for functionality. Contact sensing would make the robot more autonomous and require less direct intervention from a human operator. Aside from visual feedback, one design used sensor bands to detect temperature, humidity, and shape data along the length of a vine robot \cite{GruebeleRoboSoft2021}. These flexible bands were able to provide information at discrete points along the robot, but, as with any non-soft component, they were subject to curvature restrictions. In addition, they did not implement any force sensing in these bands.


% Regarding contact sensing, there has been recent research into tactile sensing on the vine robot \cite{BryantIROS2022}. This design measured force using pockets containing flexible curvature sensors, which increase the bending stiffness of the robot. This could limit the environments the robot is able to access. This system also needs at least a few sensor measurements to be accurate, which might not always be possible given the size of the object. Another design used pressurized pockets to provide haptic feedback \cite{young2019bellowband}. This is an example of how simple pneumatic pockets can provide meaningful information on quantities that might otherwise be challenging to measure, such as force. We aim to build on these ideas by developing a soft contact sensing method along the length of a vine robot. Our main difference is in directly quantifying the amount of force the object touching the vine robot applies by looking at the sensed pressure.

% Here, we present our design for a soft force sensing pocket for large scale flexible robot. Our sensor can be applied to a vine robot and allows the vine robot to grow and steer while providing a pressure response to forces exerted from the environment. If contact is significant enough, the robot will grow in the direction of the contacted object, as demonstrated in Fig.~\ref{glamorshot}. We show how to calibrate the sensor to get an equation the translate change in internal pressure to applied force. We conduct experiments to determine the ideal design for our sensor pocket based on certain parameters. We also describe how our sensor's measurements can be interpreted in light of factors that cannot be designed for. Our design gives flexible robots like the vine robot a means of intelligently interacting with the environment and furthers their ability to navigate complex spaces. 

\section{Design} \label{Design}

In this section, we discuss our design objectives and the design and fabrication of our sensors, as well as how we integrated them on a vine robot.
% for adding a force sensing system to a flexible robot, and we present the design and manufacturing process of the force sensor, as well as its integration onto a vine robot. %We then show the design of our force sensor and how multiple sensors can be applied to a vine robot. Finally, we show the fabrication process for our sensors.

\subsection{Design Objectives} \label{Design Objective}
We considered several design objectives in creating our sensing system. First, the sensors should be able to be attached to the body of a flexible robot and detect forces applied along its entire surface. Second, the sensors should have as high accuracy as possible and a spatial resolution that can be chosen depending on the desired application (higher resolution improves localization of applied forces but requires more individual sensors). Third, the sensors should be able to withstand the forces the robot exerts and the shape change the robot undergoes without inhibiting their function.
% The sensors should be able to detect force applied to the exterior of the robot, with enough accuracy for the particular application.  so it will ideally be flexible and limit the number and size of rigid parts. 
% Additionally, the sensing system must be durable to handle the forces the robot exerts.
Fourth, the sensors should not hinder the robot's ability to move and apply forces; for use with vine robots, this means that the sensing system should be able to evert with the vine robot body and undergo the curvatures that the robot achieves as it steers.

% We decided to make a new sensor because a new sensor could be made specifically with the intent of functioning with a focus on these design requirements.

\subsection{Design Overview} \label{Sensing System Design}

Our sensing system is shown in Fig.~\ref{pocket}. The force sensor (Fig.~\ref{pocket}(a)) is comprised of a pressurized, airtight pocket that has exterior full seals on its edges and interior partial seals throughout its body to increase its flexibility while still allowing internal airflow. An off-the-shelf air pressure sensor is embedded inside the pocket, such that forces applied anywhere on the surface of the pocket cause an increase in the pocket's internal air pressure, which can be read by the sensor. An optional tube with a removable cap can be attached to the sensor for adjusting the initial internal pressure; otherwise, the pocket can be permanently sealed after inflation to an initial internal pressure.

% Figure environment removed

As one potential use case, an array of force sensors could be attached on the exterior of a vine robot's body, with sensors arranged radially around its circumference (Fig.~\ref{pocket}(b)) and distributed along its length (Fig.~\ref{pocket}(c)) with the desired resolution. One method of steering a vine robot involves the attachment of pneumatic artificial muscle actuators along the outside of the main growing tube~\cite{coad2019vine, greer2019soft, naclerio2020simple}; at least three actuators must be arranged around the robot's circumference in order to control the 3D position of the robot tip. For a vine robot steered in this way, to ensure that the force sensors will be the first point of contact with the environment, they could be attached on top of the actuators, and if additional circumferential resolution is desired, also between the actuators.

% The flexibility of the sensors, due to both their material properties and specific design objectives, allows the sensors to bend and evert/invert with the vine robot body and actuators. In particular, the interior seals allow for the sensor to bend at the seals, allowing for a flexible movement without compromising force sensing.

\subsection{Sensor Fabrication} \label{Pocket Manufacturing}

% Figure environment removed

The fabrication of the force sensor is shown in Fig.~\ref{manufacturing}. The body of the sensor is made of a tube of low-density polyethylene (LDPE) plastic with 10.2~cm lay flat diameter and a different thickness and length depending on the sensor, as discussed later. The other sensor components include the air pressure sensor (MPRLS, Adafruit New York, NY) with attached wires, the air tube made of polyurethane plastic with 0.64~cm outer diameter, and an airtight push-to-connect tube cap with 0.64~cm inner diameter. 
% The tools required for fabrication are scissors, a hot glue gun, and an impulse heat sealer.
% We created each pocket using 10.2-cm lay flat diameter LDPE tubing of 0.10~mm thickness (or 0.05~mm thickness for a thinner pocket), polyurethane tubing of 0.64 cm outer diameter, the pressure sensor, wires, airtight push-to-connect tube caps of 0.64 cm diameter, hot glue, scissors, and a heat sealer.
% Embedded into the sensor's body is an air pressure sensor (MPRLS, Adafruit New York, NY) and the optional air tube for adjusting the initial pressure, made of polyurethane tubing of 0.64~cm outer diameter. Wires connect the air pressure sensor to a microcontroller (Arduino Uno, Arduino, Turin, Italy) external to the sensor, and an airtight push-push-to-connect tube cap of 0.64~cm inner diameter can be used to cover the , hot glue, and the tools used are scissors, a hot glue gun, and a heat sealer.
% The sensors are sensitive to small forces on their exterior and flexible enough to evert and invert with a vine robot. The sensors can contain interior seals, which are partial seals in the interior of the sensor body. 
% The manufacturing of these interior seals is discussed in more detail in Section~\ref{Pocket Manufacturing}}
% We made several different sensors, but the overall manufacturing process is similar for all sensors. We created each pocket using 10.2-cm lay flat diameter LDPE tubing of 0.10~mm thickness (or 0.05~mm thickness for a thinner pocket), polyurethane tubing of 0.64 cm outer diameter, the pressure sensor, wires, airtight push-to-connect tube caps of 0.64 cm diameter, hot glue, scissors, and a heat sealer.
To fabricate the sensor, we first marked the locations on the LDPE tubing where we planned to make the interior and exterior seals and the incisions for the sensor wires and air tube (Fig.~\ref{manufacturing}(a)). We then used scissors to cut the incisions for the sensor wires and the air tube (Fig.~\ref{manufacturing}(b)). Next, we used an impulse heat sealer with a heating element width of 5~mm to make the interior partial seals (Fig.~\ref{manufacturing}(c)). 
% For all seals, we set the heat sealer to 5.25~seconds. 
We placed paper with width approximately one third of the pocket's lay flat diameter inside the LDPE tube, such that the impulse heat sealer would seal together only the plastic not covered by the paper; after sealing, we removed the paper. We then placed the sensor and air tube inside the pocket (Fig.~\ref{manufacturing}(d)). We used a hot glue gun on its low temperature setting to glue the wires and air tube to the LDPE tube from both the interior and exterior of the incisions (Fig.~\ref{manufacturing}(e)). Finally, we used the impulse heat sealer to create the exterior seals (Fig.~\ref{manufacturing}(f)), and we pressurized the pocket using the air tube and airtight cap. 

We placed the incisions approximately 3~cm from the exterior seals---close enough to minimally obstruct the body of the sensor but far enough to fit the components without interfering with the sealing process. After adding extra space in the end pockets for the sensor and the tubing, we spaced the partial seals equally along the length of the pocket to make it a uniform height; more seals with closer spacing gives the inflated pocket a lower profile and higher flexibility, but it slows internal airflow within the sensor, so we used three partial seals. 

We fabricated four test sensor geometries (Fig.~\ref{alltest}), which we used to study the effect of various parameters on the sensor's change in internal air pressure when force is applied (Sec.~\ref{Experimental Exploration of Sensor Calibration}). Our control pocket (Fig.~\ref{alltest}(a)), used as a baseline to compare the other pockets against, has a thickness of 0.10~mm and a pre-inflated length (measured between the centers of the two external seals) of 27.5~cm. Our small pocket (Fig.~\ref{alltest}(b)) has a thickness equal to that of the control pocket and a shorter pre-inflated length of 15~cm. Our thin pocket (Fig.~\ref{alltest}(c)) has a lower thickness of 0.05~mm and a pre-inflated length equal to that of the control pocket. Our sealed pocket (Fig.~\ref{alltest}(d)), has a thickness and pre-inflated length equal to that of the control pocket, and the partial internal seals create four subpockets with pre-inflated lengths of 8~cm, 5.75~cm, 5.75~cm, and 8~cm.

% Figure environment removed

% The sealed pocket has three seals. We chose this number because it increased the flexibility of our sensor while also decreasing the height of the sensor, making the sensor more low profile. Using more than three seals made it more challenging to manufacture and inflate. We placed the central seal in the middle of the pocket. We chose the other seals to be 5.75~cm away from the central seal on either side of it. We chose 5.75 cm because we decided to give the outside sub-pockets a little extra size to fit the sensor and the tubing. This created four sub-pockets in the sealed pocket, with two sub-pockets having 5.75~cm length pre-inflation and two sub-pockets having 8~cm of length pre-inflation.

\subsection{Integration onto a Vine Robot}

To demonstrate the use of our force sensors in robot control (Sec.~\ref{Demonstration}), we integrated an array of six of the sealed pocket sensors onto a vine robot (Figs.~\ref{glamorshot} and~\ref{demo}). We made the vine robot's main body out of an LDPE tube with 20.3~cm lay flat diameter and 0.10~mm thickness. On the left and right sides of the main body tube, we used double-sided tape (MD 9000, Marker-Tape, Mico, Texas) to attach two series pouch motor actuators fabricated as in~\cite{coad2019vine} out of LDPE tubes with 10.2~cm lay flat diameter and 0.10~mm thickness. We also used double-sided tape to attach three force sensors along the length of each actuator.

To control the speed of growth of the vine robot, we attached the main tube of the robot body to a base as in~\cite{coad2019vine}, i.e., a rigid pressure vessel with an air inlet. Inside the base, the pre-everted vine robot wraps around a spool. To evert the vine robot, the base is pressurized through the air inlet to a pressure higher than that needed to begin growth~\cite{hawkes_soft_2017}, and the spool is controlled using a motor with an encoder to unroll the robot body material. The air pressure inside the base and main body tube is controlled using a closed-loop air pressure regulator (QB3, Proportion-Air, McCordsville, IN) connected to a compressed air source. 
% The regulator allows more air to flow into the robot's main tube as its volume increases with the growth of the vine robot. 
To steer the robot left and right as it navigates along a tabletop, the air pressures inside the actuators on the left and right sides of the robot body are also controlled by the same type of closed-loop pressure regulators.

All of the sensors and actuators are connected to a single microcontroller (Uno, Arduino, Turin, Italy) on which the robot's control code runs. The air pressure sensors inside our force sensors communicate via I2C, and an I2C multiplexer (TCA9548A, Adafruit, New York, NY) allows the Arduino to read all six sensors in sequence. By using up to nine multiplexers, up to 64 air pressure sensors can be read at once.


% The pockets were designed to be attached on top of the actuators. We attached six sealed sensors to the exterior of the actuators (three on each one). The sealed pocket not only has very effective force sensing, but is also able to move more functionally with the vine robot and bend without causing unwanted side effects, such as possible pressure responses. 


% The air pressure sensor communicates via I2C to a microcontroller (Uno, Arduino, Turin, Italy). Using an I2C multiplexer (TCA9548A, Adafruit New York, NY), up to 64 sensors can be placed along the surface of a flexible robot and communicate with a single Arduino microcontroller. 

\section{Experiments and Results} \label{Experimental Exploration of Sensor Calibration}

In this section, we discuss experiments and results studying the relationship between the force applied to our sensors and their measured internal pressure when the parameters of either the contact or the sensors themselves are varied.

% our goal of the experimental characterization is to find the relationship between the applied force on the pocket and the change in measured internal pressure so we can interpret the sensor readings. We also  discuss how we can calibrate the sensors for use.% and show how we use the calibrating system to test a pocket for consistency when facing uncontrollable factors in the environment and how its performance changes when we change certain controllable factors in the design of the pocket.

\subsection{Experimental Procedure} \label{Testing}

% We wanted to determine the relationship between the force applied to the exterior of the pocket and the pressure change reported by the pressure sensor. We realized that modeling would be an unrealistic goal for the purposes of this design due to the challenge of creating an analytical model. There are many factors involved, including but not limited to pocket dimensions, initial pocket pressure, material of the pocket, location of the force, and surface area of the force. %Finite Element Analysis (FEA) could potentially generate an estimate of the force response, but we wanted to begin by experimentally characterizing it to ensure that our models best matched reality.

% Experimental characterization was shown to be successful in \cite{young2019bellowband}, so we aimed to do the same with our pocket sensor. 
Fig.~\ref{experimentalsetup} shows our experimental setup. We placed the sensor to be tested on a table and measured the initial internal pressure using the embedded air pressure sensor. We then placed a low-mass circular acrylic disk on top of the sensor to provide a desired contact area. On top of the disk, we placed the desired weights, and after approximately 5 seconds, we noted the resulting internal pressure. We subtracted off the initial internal pressure to determine the change in pressure for each applied force.

For each experiment, we used weights with masses of 150~g, 300~g, and 450~g, and we conducted three trials for each scenario. We also periodically measured the change in internal pressure with no weight on the pocket, and it was always approximately 0~kPa during our experiments. We chose these particular weights because they allow us to understand the general trends of the sensitivity within a force range that a vine robot might undergo when navigating an environment and contacting obstacles to wrap around.

For most trials, we used a medium-sized disk with contact area 12.5~cm$^2$ and mass 8.75~g, but to study the effect of varying contact area, we used two additional disks: one smaller disk with contact area 6.9~cm$^2$ and mass 4.36~g, and one larger disk with contact area 25~cm$^2$ and mass 17.6~g. Note that in plotting our data, we added the weight of the disk to the weight of the added mass to determine the total force applied to the sensor.

For most trials, we used the control pocket, but to study the effect of varying the sensor length, varying the membrane thickness, or adding the internal partial seals on the sensitivity, we compared the sensitivities of the control pocket with that of the other three pockets described in Section~\ref{Pocket Manufacturing}.

For most trials with the control pocket and all trials with the thin and small pockets, we applied the weight in the middle of the top surface of the sensor. However, to study the effect of varying the contact location, we applied the weight at varying locations. For the control pocket, we also applied the weight on the top of the sensor at 7~cm, 13~cm, and 19~cm from the end containing the pressure sensor, and we also rotated the sensor 90$^\circ$ circumferentially and applied the weight in the middle of the side of the sensor (along the line between the ends of the two end seals). For the sealed pocket, we usually applied the weight to the second sub-pocket from the end closest to the embedded pressure sensor, but we also compared this location with applying the weight to the third sub-pocket.

For most experiments, we used an initial internal pressure of 0.4~kPa above atmospheric. To study the effect of varying the initial internal pressure in the control pocket, we also tested pressures of 0.7~kPa and 1.0~kPa, which we achieved by inflating the control pocket through the air tube to the desired amount and then sealing it with the airtight cap.

% The weights varied from 50 to 500~grams in 50~gram increments as shown in Fig.~\ref{experimentalsetup}.
% As a note, the largest disk had a mass of 17.6~g and a contact area of 25~cm$^2$, the middle disk had a mass of 8.75~g and a contact area of 12.5~cm$^2$, and the small disk had a mass of 4.36~g and a contact area of 6.9~cm$^2$. 
% We then measured the accompanying change in pressure after 5-10 seconds. 
% This test is an extremely quick procedure to ready the sensors, as it takes under five minutes to perform and generates data by which future pressures can be compared against. 
% A maximum mass of 500 grams (4.9 N) was chosen for this process because for the purposes of the vine robot we were using, it would be unreasonable to expect it to exceed this amount of force.

% Figure environment removed

After plotting the data for each experiment, we determined the best fit line relating the applied force to the change in pressure. The slope of the line $s$ represents the sensitivity of the sensor and is given by 

\begin{equation}
    s = \frac{P_{sensed}-P_{initial}}{F_{applied}},
\label{pressureleft}    
\end{equation}
where $P_{sensed}$ is the sensed pressure, $P_{initial}$ is the initial pressure, and $F_{applied}$ is the applied force. 

The process of finding $s$ allows us to calibrate the sensor. This calibration can then be used for that sensor to map a sensed pressure change to an applied force. This mapping is given by 

\begin{equation}
    F_{applied} = \frac{P_{sensed}-P_{initial}}{s}.
\label{forceleft}
\end{equation}

% This equation can be used to generate a fairly accurate relationship between the change in pressure sensed and how much force is being applied. By calibrating the sensor, we can know what forces we are sensing simply by reading the change in pressure. Ideally, the weights ranging from 50 to 500 grams would be used on a medium disk in the middle of the pocket before the initial use of the sensor to calibrate it.

% We used testing not only to calibrate sensors for use, but also to learn more about the sensors, such as how certain factors, including contact location of the force, contact area of the force, initial pressure, pocket thickness, pocket size, and the number of interior seals, impact the performance of the pressure sensor. We used the four test pockets as described in Section~\ref{Pocket Manufacturing}.

% For our tests, we used the same general experimental setup with minor changes based on the test we were doing. We will describe the general setup here and include any adjustments to this procedure in the descriptions of the specific tests. In the tests, we placed masses of 150~g, 300~g, and 450~g onto our disk of mass 8.75~g, which we positioned on the vine robot. We then measured the change in pressure. We conducted three trials for each scenario. We also measured the change in pressure with no mass on the pocket. As a note, this was always essentially 0 kPa. Because we were more interested in trends and relationships than the specific calibration process, we used four data points instead of the usual eleven that would be required for calibration. 
The results of these experiments are shown in Fig.~\ref{controllablefactors}.

% Figure environment removed


\subsection{Results---Uncontrollable Factors} \label{Uncontrollable Factors}
In this subsection, we present results showing the effect of varying uncontrollable factors---that is, factors that are the properties of contact, not the sensor itself---on the sensitivity of the sensor. Our goal is to create a sensor that behaves consistently when the same force is applied, regardless of these factors.

% Through experimental testing, we determined the impact uncontrollable factors have on the sensed pressure change. In particular, we tested how the change in pressure varied when force was applied to different locations and over different contact areas. Our goal is to create a sensor that is consistent and predictable with consistent forces, regardless of contact location or contact area.

\subsubsection{Contact Location, Lengthwise} \label{Contact Location: Axial}

% We first began by looking at the response to force along the length of the pocket. We want our pocket to have the same pressure response to equal amounts of force along the entire length of the pocket. For this test, we used the control pocket because we were interested in the general response.

% For the axial test, we made marks 7~cm, 13~cm, and 19~cm from the exterior seal of the pocket. We used the testing method as described above. Note, small variations in internal pressure were present between tests, but we do not expect this to cause significant differences. 

As shown in Fig.~\ref{controllablefactors}(a), the sensitivities for the three contact locations along the length of the control pocket are the same to two significant digits with a value of $s$~=~0.31~kPa/N. There is no detectable relationship between location along the length and sensed pressure, which is desirable. 
% The slopes of the first-order linear approximation at the three locations are the same to two significant digits with a value of $s$~=~0.31. 
% This is very promising, as the pocket is consistent no matter where the contact force is applied.

\subsubsection{Contact Location, Radial} \label{Contact Location: Radial}

As shown in Fig.~\ref{controllablefactors}(b), the sensitivity when force is applied to the side of the control pocket is significantly higher than when force is applied to the top. The slope when force is applied to the top is $s$~=~0.33~kPa/N, which is very similar to the lengthwise location tests. Differences can be attributed to slight differences in initial pressure, which will be discussed in Section~\ref{Internal Pressure}. The slope when force is applied to the side is $s$~=~0.51~kPa/N. This significant pressure difference is likely due to how the material folds. When force is applied to the top, the inflated LDPE holds its shape, but when force is applied along the crease on the side, the inflated LDPE crumples, significantly decreasing the volume of the pocket, and resulting in significantly different sensed pressures, especially at higher forces. To counteract this difference in sensitivity, we suggest attaching the bottom of the pocket to the surface of the flexible robot, so that it is less likely that forces will be applied to the side.

% Additionally, we also want to consider the response to force at different locations around the pocket. Ideally, there would be no relationship between radial location along the pocket and pressure response. However, this is a factor that can be somewhat controlled through the placement of the sensor on the vine robot, so it is not necessary that there is no relationship. Again, we used the control pocket for this test.

% For the radial test, we measured the pressure response when we placed the masses on the top of the pocket and on the side of the pocket at the crease near the 13~cm mark. Similar to the axial tests, we also did these tests at three slightly varying pressures.
% The results of these test are shown in Fig.~\ref{controllablefactors}(b). 
% There is a significant difference in how much pressure changes along the crease. The slope when force is applied to the top is $s$~=~0.33, which is very similar to the axial location tests. Differences can be attributed to slight differences in pressure from the tests, which will be discussed in Section~\ref{Internal Pressure}. The slope when force is applied to the side is $s$~=~0.51. This significant pressure difference is due to how the material folds. When force is applied to the top, the inflated LDPE holds its shape, but when force is applied along the crease, the inflated LDPE crumples, significantly decreasing the volume of the pocket. This results in significantly different sensed pressures, especially at higher forces. To counteract this difference in force sensing, we suggest attaching the top of the pocket tangent to the surface of the vine robot. Efforts should be made to ensure the crease is not able to be contacted or inaccurate forces could be reported.

\subsubsection{Contact Location, Seals} \label{Contact Location: Seals} 

As shown in Fig.~\ref{controllablefactors}(c), the sensitivity when force is applied to the third subpocket is slightly higher than when force is applied to the second subpocket. For the second sub-pocket, the slope is $s$~=~0.38~kPa/N, while for the third sub-pocket, the slope is $s$~=~0.39~kPa/N. The increased $s$ for both sub-pockets over the previous lengthwise tests is due to the increased sensitivity seen in sealed pockets, which will be described in Section~\ref{Seals}. The minor difference in sensitivities at the two sub-pockets is likely due to slight differences in the sub-pocket shape as a result of the manufacturing process.

% Finally, we performed a location test using the sealed pocket. We did this by applying force to sub-pocket 2 and sub-pocket 3 of the sealed pocket. Sub-pocket 2 is adjacent to the sub-pocket containing the sensor, and sub-pocket 3 is adjacent to sub-pocket 2. We used the testing method as described in Section~\ref{Testing} on both sub-pockets. 
% From the results in Fig.~\ref{controllablefactors}(c), it is clear there is a slight difference in the change in pressure in the two sub-pockets. For sub-pocket 2, the slope is $s$~=~0.38, while for sub-pocket 3, the slope is $s$~=~0.39. The increased $s$ for both sub-pockets over the previous axial tests is due to the increased performance seen in sealed pockets, which will be described in Section~\ref{Seals}. The slopes at the two different sub-pockets are not significantly different from each other. This minor difference is likely due to slight differences in the sub-pocket shape that is a result of the manufacturing process.


% % Figure environment removed

\subsubsection{Contact Area} \label{Contact Area} 

% Another important factor we wanted to look into was how the pocket responds to the surface area of the force. Ideally, the surface area of the force should have no effect on the sensed pressure. We performed tests on the control pocket using three low-mass circular disks of varying surface areas. As a reminder, the largest disk had a contact area of 25~cm$^2$, the middle disk had a contact area of 12.5~cm$^2$, and the small disk had a contact area of 6.9~cm$^2$. We placed the same weights on the disks as described above. The disk weight was factored into our data.

As shown in Fig.~\ref{controllablefactors}(d), the sensitivity for smaller contact areas is slightly higher than that for larger contact areas. For the small disk, the slope is $s$~=~0.34~kPa/N. For the medium disk, the slope is $s$~=~0.31~kPa/N. For the large disk, the slope is $s$~=~0.30~kPa/N. This difference seems to be due to the larger pocket deformation effects the smaller contact areas have. Future work should aim to decrease the effect of contact area on the sensitivity of the pressurized pocket.

\subsubsection{Contact Area, Seals} \label{Contact Area: Seals}

% We also considered the effect of contact area for the sealed pocket. We used the same testing method as described above for the control pocket contact area tests and applied them to pocket 2 of the sealed pocket.
As shown in Fig.~\ref{controllablefactors}(e), the sensitivities for the three contact areas on the sealed pocket follow the same trend as those for the control pocket. For the small disk, the slope is $s$~=~0.41~kPa/N. For the medium disk, the slope is $s$~=~0.38~kPa/N. For the large disk, the slope is $s$~=~0.34~kPa/N. The change in slope with changes in contact area is slightly larger for the sealed pocket than the control pocket, likely because the effect of the larger deformation caused by smaller contact areas is more significant on the sealed pocket.
% For the sealed pocket, the difference in contact area has a significantly greater response than for the unsealed pocket of the same size. This difference in slopes seems to be due to the greater deformation on the pocket a smaller contact area has. The slopes are greater than the contact area test on the control pocket because sealed pockets tend to have a greater sensitivity than the control pocket.

% % Figure environment removed


\subsection{Results---Controllable Factors} \label{Controllable Factors}

In this subsection, we present results showing the effect of varying controllable factors---that is, factors that are properties of the sensor, not the contact---on the sensitivity of the sensor. Our goal is to choose these properties strategically so as to increase the sensitivity of the sensor.
% We also used experimental testing to determine which sensors had a more useful relationship between force applied and change in pressure. To do this, we compared a control pocket against the small pocket, the thin pocket, and the sealed pocket. A higher $s$ is better for force sensing as it shows a more robust force-sensing ability that is better able to deal with any variation in the data. The controllable factors we studied were initial pressure inside the pocket, pocket material thickness, pocket length, the presence of seals.

\subsubsection{Initial Pressure} \label{Internal Pressure}
As shown in Fig.~\ref{controllablefactors}(f), a lower initial pressure results in a higher sensitivity than a higher initial pressure. At 0.4~kPa, the slope is $s$~=~0.31~kPa/N. At 0.7~kPa, the slope is $s$~=~0.28~kPa/N. At 1.0~kPa, the slope is $s$~=~0.24~kPa/N. We do not recommend using an initial pressure of 0~kPa, as the pressure response tends to be unreliable, but a pressure around 0.4~kPa has a good combination of sensitivity and reliability.
% While not plotted, the change in internal pressure for a pocket at 0~kPa is significantly lower than any change in internal pressure for a pocket above atmospheric, and it displays greater unreliability. When the 450~g mass is applied, the change in pressure ranges from 1.05 to 1.18~kPa, which is a greater range than at any internal pressure above atmospheric. We therefore recommend using a pressure around 0.4~kPa for better sensitivity while still maintaining reliability.

% To start, we varied the internal pressure for the control pocket. We used internal pressures of around 1.05~kPa~$\pm$~0.1~kPa above atmospheric pressure, 0.7~kPa~$\pm$~0.1~kPa above atmospheric pressure, and  0.4~kPa~$\pm$~0.1~kPa above atmospheric pressure. This is with a recorded atmospheric pressure of about 99.9~$\pm$~0.1~kPa. There was occasionally some greater variation in the atmospheric pressure, but, for the majority of trials, atmospheric pressure was in the range stated. In addition, the sensitivity for sensors kept at atmospheric pressure is small and unreliable, so it is wise to keep sensors above atmospheric pressure. For the experiment, we conducted three trials for each starting internal pressure in the same manner in which all other trials were done. They were completed at the 13~cm mark on the control pocket. 

% The results of the experiment are shown in Fig.~\ref{controllablefactors}(f) below. As can be seen, a lower pressure has a significantly better sensitivity than a higher pressure. At 0.4~kPa~$\pm$~0.1~kPa above atmospheric pressure, the slope was $s$~=~0.31. At 0.7~kPa~$\pm$~0.1~kPa above atmospheric pressure, the slope was $s$~=~0.28. At 1.0~kPa~$\pm$~0.1~kPa above atmospheric pressure, the slope was $s$~=~0.24. We did not feel the need to show the atmospheric tests in the graph, but the change in internal pressure for a pocket at atmospheric pressure was significantly lower than any change in internal pressure for a pocket above atmospheric. In addition, the change of internal pressure has greater unreliability. When the 450~g mass was applied, the change in pressure ranges from 1.05 to 1.18~kPa. A 0.13~kPa range is greater than at any internal pressure above atmospheric.

% We therefore recommend using a lower pressure around 0.4~kPa above atmospheric pressure for a better sensitivity. It is important to have the internal pressure low, but not too close to atmospheric pressure, so that the response remains accurate.

\subsubsection{Thickness} \label{Thickness}
As shown in Fig.~\ref{controllablefactors}(g), the thin pocket is slightly more sensitive than the control pocket. The slope for the control pocket is $s$~=~0.31~kPa/N, while the slope for the thin pocket is $s$~=~0.32~kPa/N. The thin pocket is more prone to tearing and melting under the hot glue, so we recommend using a thicker pocket when using LDPE plastic.  

\subsubsection{Size} \label{Size}
As shown in Fig.~\ref{controllablefactors}(h), the small pocket is significantly more sensitive than the control pocket. The slope for the small pocket is $s$~=~0.42~kPa/N, while the slope for the control pocket is $s$~=~0.31~kPa/N. This is likely because the change in volume caused by the application of the force is a larger percentage of a small pocket's volume than for a larger pocket, yielding a larger change in pressure. While a larger pocket has more space with which to find forces, a smaller pocket has a greater sensitivity to differentiate between forces. The size best suited for a given robot is dependent on the task.

\subsubsection{Seals} \label{Seals}
As shown in Fig.~\ref{controllablefactors}(i), the sealed pocket is more sensitive than the control pocket. The slope for the sealed pocket is $s$~=~0.38~kPa/N, while the slope for the control pocket is $s$~=~0.31~kPa/N. Thus, in addition to being more flexible and lower profile for integration onto a vine robot, the pocket with seals is also preferable in its sensitivity.



% We have found that the pocket with seals is preferable both in sensitivity and also in ability to move with the vine robot. The seals allow the pocket to move more naturally when the vine robot grows or steers.


\section{Demonstration} \label{Demonstration}

We conducted a demonstration to showcase the use of our force sensors to enable new capabilities for a vine robot. Our goals were to show that the vine robot is able to grow and steer with the sensors and that the sensors can allow the robot to respond to force in the environment to inform its control. This demonstration is inspired by the motion of natural vines, which search for and wrap around objects in their environment to support themselves and extend their reach. With the addition of force sensing, a vine robot should be able to do the same, provided that the object is large enough that the robot has a tight enough turning radius to continuously wrap around it. 

In our demonstration, the robot is instructed to sweep out a path while searching for contact on its left and right sides. If contact occurs, the robot grows and steers towards the sensed object, attempting to wrap around it. If contact is sustained as the robot grows and subsequent sensors continue to detect contact, the robot continues to wrap around the object. Otherwise, the object is deemed too small, and the robot continues searching for contact elsewhere. 
% Fig.~\ref{alg} shows the state diagram operating on the robot for this demonstration.


% We based our demonstration on the idea that a vine robot could use large objects to help support itself in the environment, similar to a natural vine. If the vine robot sensed an object, it would steer toward it and grow, following its path. However, if the object was too small based on the vine robot's turning radius, and subsequent sensors did not sense anything, it would no longer steer towards the object. A simplified algorithm for the demonstration is shown in Fig.~\ref{alg}.

%For the demonstration, we used a vine robot with two actuators (one on the left side and one on the right side), and we attached six sealed sensors to the exterior of the actuators (three on each one). The sealed pocket not only allows for very effective force sensing, but the seals also allow the pocket to move more functionally with the vine robot and to bend without causing unwanted side effects, such as possible sensitivities. 
% We placed six pressurized pockets 180$^{\circ}$ from each other, with three pockets lining each side of the vine robot. 

%% Base info here

%% Pressure regulator: (QB3, Proportion-Air, McCordsville, IN)

% Control of the vine robot was achieved through an Arduino code relaying messages to an Arduino Uno. The Arduino Uno was connected to both the base of the vine robot and to a multiplexer which collected data from each of the six pocket sensors. 


% Figure environment removed

% We instructed the vine robot to steer back and forth for a length of time. If it did not sense anything, it would grow the length of one pocket at a constant speed and continue to steer back and forth, searching for an object. Once the pocket at the tip of the vine robot came into contact with and sensed an object, it would maintain its position and stop steering. It would then continue to grow while holding its steered position until the sensor at the tip no longer sensed pressure of a certain amount. Once the vine robot was no longer sensing pressure, it would continue its regular motion, searching for an object large and dense enough to support the vine robot. 

The robot control uses a state machine (Fig.~\ref{alg}) to achieve this task. There are five states: Growing Straight, Searching Left, Searching Right, Growing Left, and Growing Right. In Growing Straight, the vine robot grows with no steering until a time out of 12~s is reached, which corresponds to growing the length of one pocket. In Searching Left and Searching Right, the robot steers to the left or the right with no growth. During the searching states, the robot checks for contact on the front sensor on the side that it is steering towards. If the threshold pressure value of 1.01~kPa is reached before a time out of 15~s, contact is detected and robot begins to grow towards the sensed contact. Otherwise, the robot searches for contact elsewhere. In Growing Left and Growing Right, the robot steers to the left or the right while it grows the length of one pocket (12~s). Then, it returns to the appropriate sensing state and checks for contact with the newly grown front sensors.

Fig.~\ref{demo} shows the successful demonstration. The vine robot was able to steer left and right with the sensors and grow to expose more sensors. It stopped steering once a sensor sensed contact and was able to respond to that contact by growing while maintaining the current steering pressure. Once the vine robot grew the length of another pocket, the new front sensor did not sense any contact, so the vine robot stopped steering and continued its search for a large enough object.

% Figure environment removed

\section{Conclusion and Future Work} \label{Conclusion and Future Work}

We presented a novel force sensor made of an airtight pressurized pocket containing an embedded air pressure sensor. This force sensor is soft and can attach to the exterior of a flexible robot such as a vine robot while still allowing the vine robot to grow and steer. We demonstrated that this sensor can be used to adjust the control of a vine robot in response to contact.
We measured the effect of both uncontrollable (contact location, contact area) and controllable (initial pressure, thickness, size, and number of seals) factors on the sensitivity of the sensor. 
% These factors were split up into uncontrollable factors and controllable factors. We used uncontrollable factor tests to determine the consistency of a pocket's sensitivities to the same force when force was applied at different contact locations and areas. We used controllable factor tests to determine the best initial pressure, thickness, size, and number of seals to increase the sensitivity of the sensors.
We found that the location of contact along the robot's body length has a fairly low impact on a pocket, while radial location has the greatest impact. Contact area also has a noticeable but smaller impact. For the controllable factors, a lower initial pocket pressure, a thinner pocket, a smaller pocket, and a pocket with seals all provide higher sensitivities to varying degrees, with pocket size having the biggest impact. 

Future work will quantify the accuracy of our force sensor in detecting applied forces under various conditions, such as loading under different directions while attached to a vine robot. We will quantify and explore how to reduce both the effect of the addition of these sensors on the ability of a vine robot to grow and steer and the effect of the growth and steering of a vine robot on the function of the sensors. We will also investigate whether hysteresis and relaxation are present in the sensor's sensitivity and explore how to reduce them, and we will explore adjustments to the sensor design to reduce the impact of the uncontrollable factors on the sensitivity. Finally, we will explore the use of these sensors in more scenarios, towards our goal of improving the ability of robots to interact intelligently with the physical world.


\bibliographystyle{IEEEtran}
\bibliography{References}

\end{document}
