\documentclass[12pt, final]{l4dc2023}


%%%%% NEW MATH DEFINITIONS %%%%%
\newtheorem{property}{Property}
\newtheorem{definition}{Definition}
\newtheorem{theorem}{Theorem}
\newtheorem{lemma}{Lemma}
\newtheorem{corollary}{Corollary}
\DeclarePairedDelimiter\abs{\lvert}{\rvert}
\DeclarePairedDelimiter\norm{\lVert}{\rVert}
\makeatletter
\let\oldabs\abs
\def\abs{\@ifstar{\oldabs}{\oldabs*}}
\let\oldnorm\norm
\def\norm{\@ifstar{\oldnorm}{\oldnorm*}}
\makeatother

% Mark sections of captions for referring to divisions of figures
\newcommand{\figleft}{{\em (Left) }}
\newcommand{\figcenter}{{\em (Center) }}
\newcommand{\figright}{{\em (Right)}}
\newcommand{\figtop}{{\em (Top) }}
\newcommand{\figbottom}{{\em (Bottom) }}
\newcommand{\captiona}{{\em (a) }}
\newcommand{\captionb}{{\em (b) }}
\newcommand{\captionc}{{\em (c) }}
\newcommand{\captiond}{{\em (d) }}

% Highlight a newly defined term
\newcommand{\newterm}[1]{{\bf #1}}


\def\figref#1{figure~\ref{#1}}
\def\Figref#1{Figure~\ref{#1}}
\def\twofigref#1#2{figures \ref{#1} and \ref{#2}}
\def\quadfigref#1#2#3#4{figures \ref{#1}, \ref{#2}, \ref{#3} and \ref{#4}}
\def\secref#1{section~\ref{#1}}
\def\Secref#1{Section~\ref{#1}}
\def\twosecrefs#1#2{sections \ref{#1} and \ref{#2}}
\def\secrefs#1#2#3{sections \ref{#1}, \ref{#2} and \ref{#3}}
\def\eqref#1{equation~\ref{#1}}
\def\Eqref#1{Equation~\ref{#1}}
% A raw reference to an equation---avoid using if possible
\def\plaineqref#1{\ref{#1}}
% Reference to a chapter, lower-case.
\def\chapref#1{chapter~\ref{#1}}
% Reference to an equation, upper case.
\def\Chapref#1{Chapter~\ref{#1}}
% Reference to a range of chapters
\def\rangechapref#1#2{chapters\ref{#1}--\ref{#2}}
% Reference to an algorithm, lower-case.
\def\algref#1{algorithm~\ref{#1}}
% Reference to an algorithm, upper case.
\def\Algref#1{Algorithm~\ref{#1}}
\def\twoalgref#1#2{algorithms \ref{#1} and \ref{#2}}
\def\Twoalgref#1#2{Algorithms \ref{#1} and \ref{#2}}
% Reference to a part, lower case
\def\partref#1{part~\ref{#1}}
% Reference to a part, upper case
\def\Partref#1{Part~\ref{#1}}
\def\twopartref#1#2{parts \ref{#1} and \ref{#2}}

% Random variables
\def\reta{{\textnormal{$\eta$}}}
\def\ra{{\textnormal{a}}}

% Random vectors
\def\rvepsilon{{\mathbf{\epsilon}}}
\def\rvtheta{{\mathbf{\theta}}}
\def\rva{{\mathbf{a}}}

% Elements of random vectors
\def\erva{{\textnormal{a}}}
\def\ervb{{\textnormal{b}}}

% Random matrices
\def\rmA{{\mathbf{A}}}
\def\rmB{{\mathbf{B}}}

% Elements of random matrices
\def\ermA{{\textnormal{A}}}
\def\ermB{{\textnormal{B}}}

\def\fvec{{\mathbf{f}}}
\def\bff{{\mathbf{f}}}
\def\bfg{{\mathbf{g}}}
% Vectors
\def\vzero{{\bm{0}}}
\def\vone{{\bm{1}}}
\def\vmu{{\bm{\mu}}}
\def\vtheta{{\bm{\theta}}}
\def\va{{\bm{a}}}
\def\vb{{\bm{b}}}
\def\vc{{\bm{c}}}
\def\vd{{\bm{d}}}
\def\ve{{\bm{e}}}
\def\vf{{\bm{f}}}
\def\vg{{\bm{g}}}
\def\vh{{\bm{h}}}
\def\vi{{\bm{i}}}
\def\vj{{\bm{j}}}
\def\vk{{\bm{k}}}
\def\vl{{\bm{l}}}
\def\vm{{\bm{m}}}
\def\vn{{\bm{n}}}
\def\vo{{\bm{o}}}
\def\vp{{\bm{p}}}
\def\vq{{\bm{q}}}
\def\vr{{\bm{r}}}
\def\vs{{\bm{s}}}
\def\vt{{\bm{t}}}
\def\vu{{\bm{u}}}
\def\vv{{\bm{v}}}
\def\vw{{\bm{w}}}
\def\vx{{\bm{x}}}
\def\vy{{\bm{y}}}
\def\vz{{\bm{z}}}

% Matrix
\def\mA{{\bm{A}}}

% Tensor
\DeclareMathAlphabet{\mathsfit}{\encodingdefault}{\sfdefault}{m}{sl}
\SetMathAlphabet{\mathsfit}{bold}{\encodingdefault}{\sfdefault}{bx}{n}
\newcommand{\tens}[1]{\bm{\mathsfit{#1}}}
\def\tA{{\tens{A}}}
\def\tB{{\tens{B}}}
\def\tC{{\tens{C}}}
\def\tD{{\tens{D}}}
\def\tE{{\tens{E}}}
\def\tF{{\tens{F}}}
\def\tG{{\tens{G}}}
\def\tH{{\tens{H}}}
\def\tI{{\tens{I}}}
\def\tJ{{\tens{J}}}
\def\tK{{\tens{K}}}
\def\tL{{\tens{L}}}
\def\tM{{\tens{M}}}
\def\tN{{\tens{N}}}
\def\tO{{\tens{O}}}
\def\tP{{\tens{P}}}
\def\tQ{{\tens{Q}}}
\def\tR{{\tens{R}}}
\def\tS{{\tens{S}}}
\def\tT{{\tens{T}}}
\def\tU{{\tens{U}}}
\def\tV{{\tens{V}}}
\def\tW{{\tens{W}}}
\def\tX{{\tens{X}}}
\def\tY{{\tens{Y}}}
\def\tZ{{\tens{Z}}}


% Graph
\def\gA{{\mathcal{A}}}
\def\gB{{\mathcal{B}}}
\def\gC{{\mathcal{C}}}
\def\dataset{{\mathcal{D}}}
\def\gE{{\mathcal{E}}}
\def\gF{{\mathcal{F}}}
\def\fourier{{\mathcal{F}}}
\def\gG{{\mathcal{G}}}
\def\gH{{\mathcal{H}}}
\def\gI{{\mathcal{I}}}
\def\gJ{{\mathcal{J}}}
\def\gK{{\mathcal{K}}}
\def\gL{{\mathcal{L}}}
\def\loss{{\mathcal{L}}}
\def\gM{{\mathcal{M}}}
\def\gN{{\mathcal{N}}}
\def\normal{{\mathcal{N}}}
\def\gaussian{{\mathcal{N}}}
\def\gO{{\mathcal{O}}}
\def\gP{{\mathcal{P}}}
\def\gQ{{\mathcal{Q}}}
\def\gR{{\mathcal{R}}}
\def\gS{{\mathcal{S}}}
\def\gT{{\mathcal{T}}}
\def\gU{{\mathcal{U}}}
\def\uniform{{\mathcal{U}}}
\def\gV{{\mathcal{V}}}
\def\gW{{\mathcal{W}}}
\def\gX{{\mathcal{X}}}
\def\gY{{\mathcal{Y}}}
\def\gZ{{\mathcal{Z}}}

\def\algebra{{\mathscr{A}}}
\def\borel{{\mathscr{B}}}
\def\manifold{{\mathscr{M}}}

% Sets
\def\sA{{\mathbb{A}}}
\def\sB{{\mathbb{B}}}
\def\complex{{\mathbb{C}}}
\def\sD{{\mathbb{D}}}
\def\expectation{{\mathbb{E}}}
\newcommand{\E}{\mathbb{E}}
\def\sF{{\mathbb{F}}}
\def\sG{{\mathbb{G}}}
\def\sH{{\mathbb{H}}}
\def\sI{{\mathbb{I}}}
\def\sJ{{\mathbb{J}}}
\def\sK{{\mathbb{K}}}
\def\sL{{\mathbb{L}}}
\def\sM{{\mathbb{M}}}
\def\natural{{\mathbb{N}}}
\def\sO{{\mathbb{O}}}
\def\sP{{\mathbb{P}}}
\def\rational{{\mathbb{Q}}}
\def\real{{\mathbb{R}}}
\newcommand{\R}{\mathbb{R}}
\def\sS{{\mathbb{S}}}
\def\sphere{{\mathbb{S}}}
\def\sT{{\mathbb{T}}}
\def\sU{{\mathbb{U}}}
\def\sV{{\mathbb{V}}}
\def\sW{{\mathbb{W}}}
\def\sX{{\mathbb{X}}}
\def\sY{{\mathbb{Y}}}
\def\integer{{\mathbb{Z}}}
\def\indicator{{\mathbbm{1}}}

% Entries of a matrix
\def\emLambda{{\Lambda}}
\def\emA{{A}}
\def\emB{{B}}
\def\emC{{C}}
\def\emD{{D}}
\def\emE{{E}}
\def\emF{{F}}
\def\emG{{G}}
\def\emH{{H}}
\def\emI{{I}}
\def\emJ{{J}}
\def\emK{{K}}
\def\emL{{L}}
\def\emM{{M}}
\def\emN{{N}}
\def\emO{{O}}
\def\emP{{P}}
\def\emQ{{Q}}
\def\emR{{R}}
\def\emS{{S}}
\def\emT{{T}}
\def\emU{{U}}
\def\emV{{V}}
\def\emW{{W}}
\def\emX{{X}}
\def\emY{{Y}}
\def\emZ{{Z}}
\def\emSigma{{\Sigma}}

% entries of a tensor
% Same font as tensor, without \bm wrapper
\newcommand{\etens}[1]{\mathsfit{#1}}
\def\etLambda{{\etens{\Lambda}}}
\def\etA{{\etens{A}}}
\def\etB{{\etens{B}}}
\def\etC{{\etens{C}}}
\def\etD{{\etens{D}}}
\def\etE{{\etens{E}}}
\def\etF{{\etens{F}}}
\def\etG{{\etens{G}}}
\def\etH{{\etens{H}}}
\def\etI{{\etens{I}}}
\def\etJ{{\etens{J}}}
\def\etK{{\etens{K}}}
\def\etL{{\etens{L}}}
\def\etM{{\etens{M}}}
\def\etN{{\etens{N}}}
\def\etO{{\etens{O}}}
\def\etP{{\etens{P}}}
\def\etQ{{\etens{Q}}}
\def\etR{{\etens{R}}}
\def\etS{{\etens{S}}}
\def\etT{{\etens{T}}}
\def\etU{{\etens{U}}}
\def\etV{{\etens{V}}}
\def\etW{{\etens{W}}}
\def\etX{{\etens{X}}}
\def\etY{{\etens{Y}}}
\def\etZ{{\etens{Z}}}

\def\ceil#1{\lceil #1 \rceil}
\def\floor#1{\lfloor #1 \rfloor}
\def\eps{{\epsilon}}

\newcommand{\pder}[1]{\frac{\partial}{\partial #1}}

\newcommand{\half}{\frac{1}{2}}
\newcommand{\limNinf}{\lim_{N \to \infty}}
\newcommand{\limTzero}{\lim_{\tau \to 0}}


\newcommand{\cmark}{\ding{51}}
\newcommand{\xmark}{\ding{55}}

\newcommand{\layer}{\mathcal{H}}
\newcommand{\defeq}{\triangleq}
%\newcommand{\defeq}{vcentcolon=}
\newcommand{\domain}{\Omega}
\newcommand{\grad}{\nabla}

\newcommand{\cin}{c_{\rm{in}}}
\newcommand{\cout}{c_{\rm{out}}}
\newcommand{\intdomain}{\int_{\domain}}
\newcommand{\network}{\gT}
\newcommand{\subnet}{\gK}
\newcommand{\map}{\gR} %\gR

\newcommand{\innerproduct}[2]{\langle #1, #2 \rangle}
\newcommand{\mcsum}[1][j]{\frac{1}{N}\sum_{#1=1}^N}

\newcommand{\inrspace}[1][c]{\gF_{#1}}

\DeclareMathOperator*{\argmax}{arg\,max}
\DeclareMathOperator*{\argmin}{arg\,min}

\let\ab\allowbreak


\usepackage{hyperref}
\usepackage{url}
\usepackage{amssymb}
\usepackage{bbm}
\usepackage{algorithm}
\usepackage{algpseudocode}
\usepackage{wrapfig}
\usepackage{graphicx}
\usepackage{caption}
\usepackage{enumitem}

\newtheorem{assumption}{Assumption}

\title[Contrastive Example-Based Control]{Contrastive Example-Based Control}
\usepackage{times}



\author{%
 \Name{Kyle Hatch}~$^{1}$ \Email{khatch@stanford.edu}\\
 \Name{Benjamin Eysenbach}~$^{2}$ \Email{beysenba@cs.cmu.edu}\\
 \Name{Rafael Rafailov}~$^1$ \Email{rafailov@stanford.edu}\\
 \Name{Tianhe Yu}~$^{1}$ \Email{tianheyu@cs.stanford.edu}\\
 \Name{Ruslan Salakhutdinov}~$^2$ 
 \Email{rsalakhu@cs.cmu.edu}\\
 \Name{Sergey Levine}~$^3$ 
 \Email{svlevine@eecs.berkeley.edu}\\
 \Name{Chelsea Finn}~$^1$ \Email{cbfinn@cs.stanford.edu}\\
 \addr $^1$Department of Computer Science, Stanford University \\
 \addr $^2$Machine Learning Department, Carnegie Mellon University \\
 \addr $^3$Department of Electrical Engineering and Computer Sciences, UC Berkeley
}
\begin{document}

\maketitle

\begin{abstract}%






While many real-world problems that might benefit from reinforcement learning, these problems rarely fit into the MDP mold: interacting with the environment is often expensive and specifying reward functions is challenging. Motivated by these challenges, prior work has developed data-driven approaches that learn entirely from samples from the transition dynamics and examples of high-return states. These methods typically learn a reward function from high-return states, use that reward function to label the transitions, and then apply an offline RL algorithm to these transitions. While these methods can achieve good results on many tasks, 
they can be complex, often requiring regularization and temporal difference updates.
In this paper, we propose a method for offline, example-based control that learns an implicit model of multi-step transitions, rather than a reward function.
We show that this implicit model can represent the Q-values for the example-based control problem.
Across a range of state-based and image-based offline control tasks, our method outperforms baselines that use learned reward functions; additional experiments demonstrate improved robustness and scaling with dataset size.%
\footnote{Videos of our method are available on the project website: 
\url{https://sites.google.com/view/laeo-rl}.
Code is released at: 
\url{https://github.com/khatch31/laeo}.}

\end{abstract}

\begin{keywords}%
  reinforcement learning, offline RL, robot learning, reward learning, contrastive learning, model-based reinforcement learning, example-based control, reward-free learning %
\end{keywords}



% Figure environment removed

\section{Introduction}
Automatic 3D reconstruction of clothed humans using image inputs has gained increasing significance due to its potential applications in a wide array of AR/VR scenarios. High-fidelity reconstructions typically depend on sophisticated capture systems, which are developed with dense camera arrays~\cite{collet2015high,joo2015panoptic,joo2018total}, programmable light-stages~\cite{Vlasic2009, guo2019relightables}, and depth sensors~\cite{newcombe2011kinectfusion,DoubleFusion,BodyFusion,dou2016fusion4d,newcombe2015dynamicfusion}. However, stringent capture environments equipped with complex hardware pose significant challenges for consumer-level applications.


In this context, considerable research effort has been dedicated to developing methods that allow for more flexible capture configurations, such as utilizing a few RGB inputs. Among these works, learning implicit functions \cite{iccv2020PIFu, saito2020pifuhd, hong2021stereopifu} has proven effective in achieving highly detailed reconstructions by integrating the advancements of deep neural networks. These methods employ large multi-layer perceptrons (MLPs) to predict the occupancy probability or truncated signed distance function (TSDF) value of every queried 3D point based on its associated local feature, which is extracted from images. They can recover a continuous surface at arbitrary resolutions without topology restrictions.


However, in typical MLP-based implicit networks, the occupancy or TSDF value at each location is solved independently with planar image features, rendering them less capable of addressing challenging cases such as occlusions. Consequently, these methods suffer from generalization and robustness issues, particularly when tackling strong occlusions caused by large motion or multiple interacting humans. 
Some follow-up studies  \cite{zheng2021deepmulticap,zheng2021pamir,huang2020arch} utilize an extra geometric model, SMPL~\cite{Loper2015}, to improve robustness by introducing strong shape priors. 
Their success typically relies on the assumption of geometrical similarity \cite{huang2020arch} between the shape prior and target reconstruction, making them intractable for handling complex cases with loose clothes and sensitive to errors in SMPL model fitting.



%\ping{this paragraph sounds like `TSDF is better than MLP/SMPL, and we use TSDF to solve the problem'. But in Sec 3, we are telling a different story, saying `MLP needs a 3D convolutional encoder'. We need to make these two sections consistent.}\sicong{I think in this paragraph we claim that the TSDF}


%We opt for Trucated Signed Distance Funtion (TSDF) volumetric representations as they are naturally suitable for convolution operations, which have shown remarkable performance for learning hierarchical features on 2D visual perception tasks \cite{SunXLW19}. 
%Meanwhile, TSDF also describes the gradual geometry change around shape surface, which is not reflected by occupancy volume. 

We instead revisit the 3D volumetric representation and resort to 3D convolutional neural networks (CNNs) for feature learning, due to their impressive performance in feature learning and the ability to incorporate spatial context. However, volumetric methods and 3D convolution involve discretization, which might raise concerns regarding whether a discretized volume can preserve subtle geometric details as continuous representations learned in implicit functions. We investigate the relationship between volume resolution and quantization error on synthetic data by converting target mesh objects to TSDF volumes, as shown in Figure~\ref{fig:quantization_error}. We observe that the quantization errors are significantly reduced by increasing volume resolution and become nearly negligible when reaching a relatively high resolution (e.g., 512 or higher). In other words, achieving fine-detailed reconstruction is not supposed to be restricted by the use of volume representations as long as a proper volume resolution is utilized. Therefore, we present a method with high-resolution feature volumes, e.g., 256 and 512, while traditional volumetric methods \cite{varol18_bodynet,gilbert2018volumetric} are often limited to much lower resolutions, such as 32 or 128.



On the other hand, an increase in volume resolution may lead to a cubic growth of memory overhead \cite{8100085}. Reducing memory costs while guaranteeing the granularity of volumetric representations is necessary for pursuing high-quality reconstruction. Thus, we adopt a coarse-to-fine approach and cull away irrelevant voxels to build a sparse high-resolution feature volume. At the coarse level, the network computes an initial TSDF by applying a U-Net with sparse 3D CNN \cite{3DSemanticSegmentationWithSubmanifoldSparseConvNet} on the sparse feature volume, which is carved by a visual hull. Through our experiments, it turns out that more than 95\% of the volume grids are discarded by the visual hull culling, making the sparse 3D CNN efficient. At the fine level, the network focuses on a narrow band near the zero-level set of the initial TSDF and discretizes the narrow band with smaller voxels. By employing this narrow-band culling, we further shrink the sampling space, resulting in a relatively small range of grid numbers (usually 300K--500K in our experiments) even with a high volume resolution of 512. The remaining voxels in the narrow band are associated with features that fuse high-frequency information from the computed normal maps upon the low-frequency shape from the coarse level to compute the TSDF at high resolution. The final mesh is then extracted from the TSDF using the Marching-Cube algorithm ~\cite{Lorensen87marchingcubes}.
% Different from the u-net sturcture to preserve global topology context, we then apply a shallow 3dcnn to compute the final TSDF $D_{final}$ which contain more local geometry detail.




% \ping{this paragraph can be expanded. It is an important contribution and often ignored by other works. stress on the novel idea of regressing blending weights instead of colors}

In addition to geometry, high-quality mesh texture is also a crucial factor contributing to visual appearance. Directly computing a color field in 3D space, as in \cite{iccv2020PIFu}, struggles to capture high-frequency texture details, while the neural radiance field (NeRF) \cite{yu2020pixelnerf} or the DoubleField~\cite{shao2022doublefield} require expensive per-instance optimization and are often unstable for sparse input images. In contrast, we adopt an image-based rendering approach to compute a texture atlas map, which is efficient and widely supported in existing computer graphics tools. 
Specifically, we compute a blending weight at each 3D point on the mesh surface to determine its color as a weighted average of the colors at its image projections. The blending weights can be computed at a relatively coarse resolution, e.g., 512 volume resolution in our case, and leave texture details to the high-resolution images, such as 1K or 2K. Unlike previous methods that generate blurry texturing results under sparse input, our method generalizes well on both synthetic and real data with just a few input views. 
Figure~\ref{fig:teaser} shows two examples reconstructed by our method. Despite the challenging garment, pose, and occlusion, our method recovers faithful shape, normal, and texture on the right.

%with a wide variety of poses and clothing styles, and it is also adaptive to handle input image with arbitrary resolutions.
%\sicong{For this concern we claim that when the resolution of dicretized volume meets certain threshold (which is 256 in our experiment), the quantization error can be neglected.} 



In summary, the main contributions of this paper are as follows:
\begin{itemize}
\vspace{-0.1in}
  \item 
  We revisit the 3D volumetric representation and demonstrate that it can support clothed human reconstruction with equal or even better performance compared to implicit representation. 
  \item 
  We develop a memory and computation-efficient method for high-resolution volumetric reconstruction using sophisticated sparse 3D CNN, coarse-to-fine estimation, and voxel culling by visual hull and narrow bands. 
  \item 
  We introduce a novel method to compute a texture atlas map, which captures rich appearance details from high-resolution input images.
  \item 
  We achieve impressive results on standard benchmark datasets Twindom and MultiHuman, significantly reducing the point-2-surface (P2S) precision to approximately 0.2cm from just six input views, with more than $50\%$ error reduction compared to the state-of-the-art methods, including DoubleField~\cite{shao2022doublefield} and PIFuHD~\cite{saito2020pifuhd}.
\end{itemize}

\section{Related Work}
%\subsection{Cost Volume based Deep Stereo Matching}
%Stereo matching is a typical problem that has been studied for decades and a well-known four-step pipeline \cite{scharstein2002taxonomy} has been established, where cost volume construction is an indispensable step. Current state-of-the-art stereo matching methods are all cost volume based methods and they can be categorized into two types. Typically, a cost volume is a 4D tensor of height, width, disparity, and features. The first category just uses a full correlation to generate a single-feature cost volume. Such methods are usually efficient but lose much information because of the decimation of feature channels. Many previous work, including Dispnet \cite{dispnet}, MADNet \cite{madnet}, IResNet \cite{iresnet} and AANet \cite{aanet}, belong to this category. The second category usually uses concatenation \cite{gcnet} or group-wise correlation \cite{gwcnet} to generate a multi-feature 4D cost volume. Such a method can achieve better performance while requiring higher computational complexity and memory consumption. Actually, a majority of the top-performing networks in public leaderboards belong to this category, such as GANet \cite{ganet}, CSPN \cite{cspn} and ACFNet \cite{acfnet}. These methods generally employ multiple 3D convolution layers to constantly regularize the 4D cost volume and then apply softmax over the disparity dimension to produce a discrete disparity probability distribution. The final predicted disparity is obtained by softly weighting indices according to their probability, which is also called soft argmin in GCNet \cite{gcnet}. However, soft argmin leaves the output susceptible to multi-modal disparity probability distributions. ACFNet \cite{acfnet} observes this problem and proposes to directly supervise the cost volume with unimodal ground truth distributions. In contrast, we define an uncertainty estimation to quantify the degree to which the cost volume tends to be multi-modal distribution, higher implies the higher possibility of estimation error.

\subsection{Multi-scale Cost Volume based Stereo Matching}
Cost volume construction is an indispensable step in the well-known four-step pipeline for stereo matching \cite{scharstein2002taxonomy, pamisurvey1, pamisurvey2}. Typically, current state-of-the-art stereo matching methods can be categorized into two types of cost volume-based methods, where the cost volume is a 4D tensor of height, width, disparity, and features. The first category usually uses the single-feature 3D cost volume generated by full correlation, which is efficient while losing much information due to the decimation of feature channels. Many real-time methods, such as Dispnet \cite{dispnet}, MADNet \cite{madnet, madnet_pami} and AANet \cite{aanet}, belongs to the category. Moreover, two-stage refinement \cite{mcvmfc} and pyramidal towers \cite{madnet} are commonly applied in the single-feature cost volume based network to construct multi-scale cost volume. The second category usually uses the multi-feature 4D cost volume generated by concatenation \cite{gcnet} or group-wise correlation \cite{gwcnet}, which can achieve better performance with higher computational complexity and memory consumption. Most top-performing networks, including GANet \cite{ganet}, CSPN \cite{cspn} and ACFNet \cite{acfnet} belong to this category. 
% In these methods, the 4D cost volume is constantly regularized by multiple 3D convolution layers and then a discrete disparity probability distribution can be produced by softmax. Next, the final predicted disparity can be obtained by softly weighting indices according to their probability \cite{gcnet}. However, such output is susceptible to multimodal disparity probability distributions and ACFNet \cite{acfnet} gives a solution by directly supervising the cost volume with unimodal ground truth distributions to alleviate this problem. 
Recently, to alleviate the high computational complexity and memory consumption when employing multi-feature 4D cost volumes, \cite{cvpmvsnet, cascade, uscnet} propose to use cascade cost volume representation in multi-view stereo. These methods usually first predict an initial disparity at the coarsest resolution of the image and then gradually refine the disparity by narrowing down the disparity search space. More closely related to our approach is Casstereo \cite{cascade}, which first extended such representation to stereo matching. It selected to uniform sample a pre-defined range to generate the next stage’s disparity search range. Instead, we employ pixel-level uncertainty estimation to adaptively adjust the next stage disparity searching range and generate pseudo-labels for subsequent domain adaptation. Our method also shares similarities with UCSNet \cite{uscnet}, which constructs uncertainty-aware cost volume in multi-view stereo while it doesn’t employ uncertainty estimation to generate pseudo-labels.

%\subsection{Multi-scale Cost Volume based Deep Stereo Matching} 
% \subsection{Multi-scale Cost Volume based Stereo Matching} 
%Multi-scale cost volume firstly was applied in the single-feature cost volume based network with the form of two-stage refinement \cite{mcvmfc} and pyramidal towers \cite{madnet}. Recently, cascade cost volume representation \cite{cvpmvsnet, cascade, uscnet} was proposed in multi-view stereo to alleviate the high computational complexity and memory consumption when employing multi-feature 4D cost volumes. These methods generally predict an initial disparity at the coarsest resolution of the image. Then, they will narrow down the disparity search space and gradually refine the disparity. More closely related to our approach is Casstereo \cite{cascade}, which first extended such representation to stereo matching. It selected to uniform sample a pre-defined range to generate the next stage’s disparity search range. Instead, we employ uncertainty estimation to adaptively adjust the next stage pixel-level disparity searching range and push the next stage's cost volume to be predominantly unimodal.

% The single-feature cost volume based network with the form of two-stage refinement \cite{mcvmfc} and pyramidal towers \cite{madnet} first employ multi-scale cost volume for stereo matching. Recently, to alleviate the high computational complexity and memory consumption when employing multi-feature 4D cost volumes, \cite{cvpmvsnet, cascade, uscnet} propose to use cascade cost volume representation in multi-view stereo, which generally predict an initial disparity at the coarsest resolution of the image. Then, the disparity search space is narrowed down and the disparity is gradually refined. More closely related to our approach is Casstereo \cite{cascade}, which first extended such representation to stereo matching. It selected to uniform sample a pre-defined range to generate the next stage’s disparity search range. Instead, we employ uncertainty estimation to adaptively adjust the next stage pixel-level disparity searching range and push the next stage's cost volume to be predominantly unimodal.

% Figure environment removed

\subsection{Robust Stereo Matching} 
There exist three categories of generalization definitions for robust stereo matching. 1) Cross-domain Generalization: the network’s ability to perform well on unseen scenes (cannot see the image pairs of the target domain in advance). Towards this end, Jia et al \cite{sungeneralizaiton} propose to incorporate scene geometry priors into an end-to-end network. Zhang et al \cite{dsmnet} introduce a domain normalization and a trainable non-local graph-based filter to construct a domain-invariant stereo matching network. 2) Adapt Generalization: the network’s ability to adapt pre-trained models to the new domain with unlabeled target data. Previous work usually pre-trains the models on synthetic data and then adapts it to new target domains with Graph Laplacian regularization \cite{zoom}, non-adversarial progressive color transfer \cite{adastereo}, and Knowledge Reverse Distillation \cite{aohnet}. More closely related to our approach are \cite{aohnet, unsuperviseddomainadaptation} in stereo matching and Monoresmatch \cite{monoresmatch} in monocular depth estimation, which also proposes to generate a pseudo-label for domain adaptation. However, these methods all select to employ classical stereo matching methods \cite{sgm} alongside with confidence estimators, e.g., left-right consistency check to generate pseudo-labels. That is all these methods need an independent method to generate corresponding pseudo-labels. Instead, the proposed method is an end-to-end network that can generate the predicted disparity map, corresponding uncertainty map and pseudo-labels jointly, which is a more simple, yet efficient way. 
% Instead, our proposed method can employ pixel-level and area-level uncertainty estimation to self-distill the predicted disparity maps of our pre-training model and generate sparse while reliable pseudo-labels to align the domain gap, which is a more simple, yet efficient way. 
3) Joint Generalization: the network’s ability to perform well on a variety of datasets with the same model parameters. MCV-MFC \cite{mcvmfc} introduces a two-stage finetuning scheme to achieve a good trade-off between generalization and fitting capability on multiple datasets. However, it doesn’t touch the inner difference between diverse datasets, e.g, the unbalanced disparity distribution. To further address this problem, we propose a cascade cost volume to adaptively the next stage disparity searching space, where the pixel-level uncertainty estimation is at the core.

% \subsection{Monocular Depth Estimation}
% Monocular depth estimation aims to estimate depth values from a single image, instead of stereo images or multiple frames in a video. This problem is ill-posed because of the ambiguity of object sizes. However, humans could estimate the depth from a single image with prior knowledge of the scenes. Recently, learning based methods were explored to learn depth values by supervised or unsupervised learning. Eigen et al. first employed Convolutional Neural Networks (CNN) to predict depth in a coarse-to-fine manner and further improved its performance by multi-task learning. Liu et al. presented deep convolutional neural fields model by combining deep model with continuous CRF. Li et al. [22] refined deep CNN outputs with a hierarchical CRF. Multi-scale continuous CRF was formulated into a deep sequential network by Xu et al. [45] to refine depth estimation. Unsupervised methods tried to train monocular depth estimation with stereo
% image pairs or image sequences and test on single images. Garg et al. [9] used novel image view synthesis loss to train a depth estimation network in an unsupervised way. Godard et al. [11] introduced left-right consistency regularization to improve the performance of view synthesis loss. Recently, some work also propose to use the stereo matching network as a proxy to learn depth from synthetic data or directly employ traditional stereo matching methods to distill proxies labels from the target domain, which proves the feasibility of distilling stereo matching networks to learn monocular depth estimation.




\section{Method} \label{method_hybridaugment}
In this section, we formally define the problem, motivate our work and then present our proposed techniques.


\subsection{Preliminaries}
Let $\mathcal{F}(x;W)$ be an image classification CNN trained on the training set $\mathcal{T}_\text{train} = (x_{i}, y_{i})^{N}_{i=1}$  with $N$ samples, where $x$ and $y$ correspond to images and labels. The clean accuracy (CA) of $\mathcal{F}(x;W)$ is formally defined as its accuracy over a clean test set $\mathcal{T}_\text{test} = (x_{j}, y_{j})^{M}_{j=1}$. Assume two operators ${A}(\cdot)$ and ${C}(c, s)$ that adversarially attacks or corrupts a given set of images with the corruption category $c$ and severity $s$, respectively.  Let $A\mathcal{T}_\text{test}$ and $C\mathcal{T}_\text{test}$ be the adversarially attacked and corrupted versions of $\mathcal{T}_\text{test}$, and let $\mathcal{F}(x;W)$ have a robust accuracy (RA) on $A\mathcal{T}_\text{test}$ and a corruption accuracy (CRA) on $C\mathcal{T}_\text{test}$. 
The aim is to fit $\mathcal{F}(x;W)$ such that the model gains robustness (\ie. increased RA and CRA compared its the baseline version), while retaining (or improving) the clean accuracy of its baseline version trained without robustness concerns.


\noindent \textbf{What we know.} Our work builds on the following crucial observations: i) CNNs favour high-frequency content \cite{wang2020high}, ii) adversaries and corruptions often reside in high-frequency \cite{wang2020towards}, iii) images are dominated by low-frequency \cite{Saikia_2021_ICCV} and iv) models relying on low-frequency components are more robust \cite{li2022robust,wang2020towards}. The robustness-accuracy trade-off is visible; low-frequency reliant models are more robust, but tend to miss out on clean accuracy brought by the high-frequency components. 

\subsection{HybridAugment}
We hypothesize that a \textit{sweet spot} in the robustness-accuracy trade-off can be found. Unlike the \textit{hard} approaches that completely rule out the reliance on high-frequency components (i.e. low-pass filters), we propose to \textit{reduce} the reliance on them. To this end, we adopt a data augmentation approach that aims to diversify $\mathcal{T}_\text{train}$ by an operation $\mathcal{HA(\cdot)}$. Keeping the strong relation intact between labels and low-frequency content (i.e. labels come from low-frequency-component image), we propose to swap high and low-frequency components of images in a batch on-the-fly. Unlike \cite{mukai2022improving}, we \textit{do not} restrict the images to belong to the same class; this diversifies the training distribution even further while preserving the image semantics. We call this basic version of our approach \textit{HybridAugment}, which corresponds to: 
%
\begin{equation} \label{hybrid_augment_paired}
    \mathcal{HA_{P}}(x_{i}, x_{j}) = \mathcal{LF}(x_{i}) + \mathcal{HF}(x_{j})
\end{equation}
%
where $x_{i}$ is the input image and $x_{j}$ is a randomly sampled image from the whole training set, which we simply sample from the mini batch at each training iteration in practice. $\mathcal{HF}$ and $\mathcal{LF}$ operators select the high and low-frequency components of an input image, for which we use:
%
\begin{equation} \label{eq:cutoff}
\begin{split}
    \mathcal{LF}(x) = GaussBlur(x) \\
    \mathcal{HF}(x) = x - \mathcal{LF}(x)
    \end{split}
\end{equation}
%
where $GaussBlur$ is used as a low-pass filter. Note that a similar outcome is possible by using Discrete Fourier Transforms (DFT), swapping the frequency bands and then applying Inverse DFT (IDFT). We find the gaussian blur operation to be faster and better in practice. 


Inspired from \cite{chen2021amplitude}, in addition to the image-pair scheme in Eq.~\ref{hybrid_augment_paired}, we propose a single image variant of \textit{HybridAugment}. In the single image variant, instead of combining two images, $x_i$ and $x_{j}$ are obtained by applying randomly sampled augmentations to a single image. The single image variant $\mathcal{HA_{S}}$ can therefore be defined as 
%
\begin{equation} \label{hybrid_augment_single}
    \mathcal{HA_{S}}(x_{i}) = \mathcal{LF}(Aug(x_{i})) + \mathcal{HF}(\hat{Aug}(x_{i}))
\end{equation}
%
where $Aug$ and $\hat{Aug}$ correspond to two sets of randomly sampled augmentation operations. Note that paired and single versions can work in tandem ($\mathcal{HA_{PS}}$), and actually outperform single or paired image versions. 


\subsection{HybridAugment++}


The frequency analysis is a vast literature, however, two core aspects often stand out; frequency-band analysis (i.e. low, high) and the decomposition of signals into amplitude and phase. \textit{HybridAugment} covers the former and shows competitive results in various benchmarks (see Section \ref{sec:exp_hybridaugment}). The latter is investigated in $\mathcal{APR}$ \cite{chen2021amplitude}, where phase is shown to be the more relevant component for correct classification, and training models based on their phase labels and swapping amplitude components of images randomly lead to more robust models. Note that frequency-band and phase/amplitude discussions are arguably orthogonal, since frequency, phase and amplitude provide distinct characterizations of a signal: intuitively speaking, frequency, phase and amplitude can be seen as the separation of visual patterns in terms of scale, location and significance. 


We hypothesize these two approaches can be complementary; a model reliant on low-frequency and spatial information (i.e. phase) can further improve robustness. Inspired by the successes of cascaded augmentation methods \cite{hendrycks2019augmix,wang2021augmax,calian2022defending}, we unify these two core aspects into a single, hierarchical augmentation method. We refer to this method as \textit{HybridAugment++} and define its paired version as:
%
\begin{equation}
  \mathcal{HA_{P}}^{++}(x_{i}, x_{j}, x_{z}) = \mathcal{APR_{P}}(\mathcal{LF}(x_{i}), x_{z}) + \mathcal{HF}(x_{j})
\end{equation}
%
where $x_{i}$, $x_{j}$ and $x_{z}$ are images sampled from the same batch. Here, $\mathcal{APR_{P}}$~\cite{chen2021amplitude} is defined as
\begin{equation}
    \mathcal{APR_{P}}(x_{i}, x_{z}) = \mathcal{IDFT}(A_{x_{z}} \otimes e^{i. P_{x_{i}}}) \\
\end{equation}
%
where $\otimes$ is element-wise multiplication, $A$ is the amplitude and $P$ is the phase component. Similar to $\mathcal{HA}$ and $\mathcal{APR}$, we also define a single-image version of \textit{HybridAugment++} as
%
\begin{equation}
 \mathcal{HA_{S}}^{++}(x_{i}) = \mathcal{APR_{S}}(\mathcal{LF}(Aug(x_{i}))) + \mathcal{HF}(\hat{Aug}(x_{i}))
\end{equation}
%
where $\mathcal{APR_{S}}$~\cite{chen2021amplitude} is defined as
%
\begin{equation}
\mathcal{APR_{S}}(x_{i}) = \mathcal{IDFT}\left(A_{\bar{Aug}(x_{i})} \otimes e^{i. P_{\overline{Aug}\left(x_{i}\right)}}\right)    
\end{equation}
%
where $Aug$, $\hat{Aug}$, $\bar{Aug}$ and $\overline{Aug}$ are different sets of randomly sampled augmentation operations. Note that we essentially propose a framework; one can use different single and paired image augmentations, either individually or together, and can still achieve competitive results (see ablations in Section \ref{sec:exp_hybridaugment}). There are also other alternatives, such as swapping phase/amplitude first and then performing $\mathcal{HA}$, but we observe poor performance in practice; dividing the phase component into frequency-bands is not interpretable as frequencies of the phase component are not well defined. The pseudo-code of our methods can be found in the supplementary material.





\section{Experiments}
% \haizhou{Follow the same way of introduction as we did in Section2.}
% \noindent In this section, we will introduce datasets and experimental setups that we used. Then we evaluate our method, other self-supervised methods, and supervised methods under different distribution shifts (\ie, concept shifts and covariate shifts) under common settings (\ie, transductive, inductive settings). It has to note that we focus on node-level tasks (\eg, node classification) in this work. As for graph-level tasks, we leave it as our future work and some simple experiments can be found in Appendix~\ref{app:graph_classification}. 
In this section, we first introduce the experimental setup including datasets, training, and evaluation protocol in Section~\ref{sec:dataset}~and~\ref{sec:unsupervised}. 
% Next, we present our experimental setup and conduct extensive experiments to evaluate our method in Section~\ref{sec:unsupervised}. 
We then perform an ablation study to demonstrate the effectiveness of each proposed component in Section~\ref{sec:ablation}. 
Additionally, we analyze the impact of important hyper-parameters in Section~\ref{sec:sensitivity}. 
Subsequently, we integrate our method with various encoding models, showcasing the model-agnostic nature of our recipe in Section~\ref{sec:other_models}. 
Finally, we provide some qualitative results such as feature visualization in Section~\ref{sec:vis}.
It is important to note that we focus on node-level tasks (\eg, node classification) in this work. As for graph-level tasks, we leave it as our future work, while some simple experiments are also provided in Appendix~\ref{app:graph_classification}.

\subsection{Datasets}\label{sec:dataset}
There exist some benchmarks for evaluating graph out-of-distribution generalization~\cite{good,ji2022drugood,gds}. 
Among them, GOOD~\cite{good} is the most representative and comprehensive benchmark that curates more diverse graph datasets with diverse tasks, including single/multi-task graph classification, graph regression, and node classification involving more distribution shifts (\ie, concept shifts and covariate shifts). Hence in this work, we follow the evaluation protocol proposed in \cite{good}. Furthermore, we validate the effectiveness of our method in the datasets (\ie, Amazon-Photo, Elliptic) that are used in EERM~\cite{eerm}. The statistics and detailed introduction to these datasets can be found in Table~\ref{tab:dataset} and Appendix~\ref{app:datasets}.

\begin{table*}[htp]
\caption{The descriptions of datasets. ``Domain-Level'' means splitting by graphs, ``Time-Aware'' denotes splitting according to chronological order.``Word'' and ``Degree'' represent splitting according to word diversity and node degree respectively. ``Language'' means splitting by user language, suggesting the prediction should not be impacted by the language the user use. ``University'' denotes splitting according to the domain university, implying that the prediction of webpages should be based on word contents and link connections rather than university features. ``Color'' means that nodes are split according to node differences in covariate shift and color-label correlations in concept shift.}
\label{tab:dataset}
\centering
\begin{tabular}{cccccccc}
\toprule
Datasets     & Network Type        & \#Nodes & \#Edges & \#Attributes &\#Classes& Train/Val/Test Split     & Metric   \\
% Cora         & Artificial Transformation & 2,703   &         &              &         &                      & Accuracy \\
Amazon-Photo\footnotemark
             & Co-purchasing network      & 7,650   & 119,081   & 755          & 10      & Domain-Level         & Accuracy \\
Elliptic\footnotemark  
             & Bitcoin transactions       & 203,769 & 234,355   & 165          & 2       & Time-Aware           & F1-Score \\
GOOD-Cora    & Scientific publications    & 19,793  & 126,842   & 8,710         & 70      & Word/Degree          & Accuracy \\
% GOOD-Arxiv   & arXiv papers               & 169,343 & 2,315,598 & 128          & 40      & Time/Degree          & Accuracy \\
GOOD-Twitch  & Gamer network              & 34,120  & 892,346   & 128          & 2       & Language             & ROC-AUC  \\
GOOD-CBAS    & A BA-house graph           & 700     & 3,962     & 4             & 4       & Color                & Accuracy \\
GOOD-WebKB   & Webpage network            & 617     & 1,138     & 1,703         & 5       & University           & Accuracy \\
\bottomrule
\end{tabular}
\end{table*}
\footnotetext[5]{This dataset is adopted from~\cite{yang2016revisiting}. \cite{eerm} constructs ten graphs with different environment id’s for each graph.} 
\footnotetext[6]{The original is available on \hyperlink{https://www.kaggle.com/ellipticco/elliptic-data-set}{https://www.kaggle.com/ellipticco/elliptic-data-set}}

\subsection{Unsupervised Representation Learning}\label{sec:unsupervised}
\subsubsection{Transductive Setting}~\label{sec:trans}
% \noindent\textbf{Baselines.}\quad We conduct experiments with 12 baselines which consist of three categories: supervised methods and self-supervised generative methods, self-supervised contrastive methods. Specifically, we compare with three supervised baselines: empirical risk minimization~(ERM)~\cite{erm}, invariant risk minimization (IRM)~\cite{irm}, and a recent proposed graph OOD method dubbed EERM~\cite{eerm}. We also compare various unsupervised node-level representation learning methods: three self-supervised generative methods including GAE~\cite{gae}, VGAE~\cite{gae}, GraphMAE~\cite{gmae} and seven self-supervised contrastive methods: DGI~\cite{dgi}, MVGRL~\cite{mvgrl}, GRACE~\cite{grace}, RoSA~\cite{rosa}, BGRL~\cite{bgrl}, COSTA~\cite{costa}, SwAV~\cite{swav}. The descriptions of these methods can be found in Appendix~\ref{app:baselines}.
In this subsection, we focus on validating our proposed algorithm under the transductive setting, where the test nodes will participate in message passing~\cite{gilmer2017neural} during training following~\cite{good}. 

\noindent\textbf{Baselines.} We conduct experiments with 12 baselines from three categories: (i)~supervised methods, including empirical risk minimization~(\textbf{ERM})~\cite{erm}, invariant risk minimization (\textbf{IRM})~\cite{irm}, and a recent proposed graph OOD method \textbf{EERM}~\cite{eerm}; (ii)~self-supervised generative methods including Graph Autoencoder (\textbf{GAE})~\cite{gae}, Variational Graph Autoencoder (\textbf{VGAE})~\cite{gae}, Self-Supervised Masked Graph Autoencoders (\textbf{GraphMAE})~\cite{gmae}; (iii)~self-supervised contrastive methods including Deep Graph Infomax (\textbf{DGI})~\cite{dgi}, Contrastive Multi-View Representation Learning on Graphs (\textbf{MVGRL})~\cite{mvgrl}, Deep Graph Contrastive Representation Learning (\textbf{GRACE})~\cite{grace}, A Robust Self-Aligned Framework for Node-Node Graph Contrastive Learning (\textbf{RoSA})~\cite{rosa}, Bootstrapped Representation Learning on Graphs (\textbf{BGRL})~\cite{bgrl}, Covariance-Preserving Feature Augmentation for Graph Contrastive Learning (\textbf{COSTA})~\cite{costa}, Unsupervised Learning of Visual Features by Contrasting Cluster Assignments (\textbf{SwAV})~\cite{swav}. The detailed descriptions of these baselines can be found in Appendix~\ref{app:baselines}.

\noindent\textbf{Experimental setup.} We use the same graph encoder across different datasets for a fair comparison following~\cite{good}. We use grid search to find other hyper-parameters (\eg, learning rate, epochs) for different methods. For all experiments, we select the best checkpoints for ID and OOD tests according to results on ID and OOD validation sets following~\cite{good}, respectively. Experimental details and hyper-parameter selections are provided in Appendix~\ref{app:hyper}. For evaluating unsupervised methods, a linear classifier will be built on the frozen trained encoder after finishing pre-training. The reported results are the mean performance with standard deviation after 10 runs following~\cite{good}.

\noindent\textbf{Analysis.}\quad Based on the experimental results listed in Table~\ref{tab:trans_concept} and \ref{tab:trans_covariate}, we can draw the following conclusions: firstly, we find strong self-supervised methods (\eg, GRACE, BGRL, COSTA) are more robust to distribution shifts (concept shift in Table~\ref{tab:trans_concept} and covariate shift in Table~\ref{tab:trans_covariate}) compared to supervised methods. For instance, on GOOD-CBAS and GOOD-WebKB datasets, GRACE surpasses the best supervised method by large margins (over 6\% absolute improvement). Interestingly, we find the methods designed for OOD generalization (\ie, IRM) and graph OOD generalization (\ie, EERM) do not attain superior performance than the standard ERM on most of the datasets. For example, EERM shows superior OOD performance compared to ERM in only one experiment, and IRM outperforms ERM in four out of ten experiments across the conducted evaluations. This phenomenon is also observed in \cite{good,ahuja2020empirical,rosenfeld2021risks}, showcasing the challenge of achieving invariant prediction in non-Euclidean graph settings. 

Furthermore, our method surpasses other SOTA self-supervised methods on the OOD test set of all datasets by a considerable margin while achieving comparable performance in the in-distribution test set. For instance, on small datasets such as GOOD-CBAS and GOOD-WebKB, our method outperforms GRACE\footnote{MARIO is built up on GRACE according to our recipe. So, we make a comparison with GRACE here.} by over 2\% absolute accuracy on the OOD test set. On larger datasets such as GOOD-Cora and GOOD-Twitch, our method still outperforms other methods which shows its superiority. For instance, under covariate shift, MARIO surpasses other methods by over 7\% absolute accuracy on the GOOD-Twitch OOD test set. These statistics prove the effectiveness of our design.


\begin{table*}[htp]
\caption{Experimental results of all methods under concept shift. The bold font means the top-1 performance and the underline represents the second performance across the unsupervised methods. 'ID' represents in-distribution test performance and 'OOD' means out-of-distribution test performance. (OOM: out-of-memory on a GPU with 24GB memory)}
\label{tab:trans_concept}
\centering
\scalebox{0.95}{
\begin{tabular}{l|cc|cc|cc|cc|cc}
\toprule
\toprule
\multirow{3}{*}{concept shift} & \multicolumn{4}{c|}{GOOD-Cora}                   & \multicolumn{2}{c|}{GOOD-CBAS} & \multicolumn{2}{c|}{GOOD-Twitch} & \multicolumn{2}{c}{GOOD-WebKB} \\
                           & \multicolumn{2}{c}{word} & \multicolumn{2}{c|}{degree}& \multicolumn{2}{c|}{color}    & \multicolumn{2}{c|}{language}   & \multicolumn{2}{c}{university} \\
                           & ID         & OOD         & ID          & OOD          & ID            & OOD           & ID             & OOD            & ID            & OOD            \\
\midrule
ERM                        & 66.38±0.45 & 64.44±0.18  & 68.60±0.40  & 60.76±0.34   & 89.79±1.39    & 83.43±1.19    & 80.80±1.00     & 56.92±0.92     & 62.67±1.53    & 26.33±1.09     \\
IRM                        & 66.42±0.41 & 64.29±0.31  & 68.57±0.35  & 61.45±0.24   & 89.64±1.21    & 82.29±1.14    & 78.87±1.04     & 59.30±1.79     & 62.67±1.10    & 26.88±1.42     \\
EERM                       & 65.10±0.44 & 62.45±0.19  & 66.95±0.44  & 56.58±0.25   & 79.07±2.12    & 64.50±1.01    & OOM            & OOM            & 62.50±2.01    & 28.07±3.23      \\
\midrule
% Random-Init                & 37.53±1.74 & 32.12±1.24  & 37.82±1.71  & 27.74±1.14   &               &               &                &                & 60.33±2.21    & 27.07±1.70     \\
GAE                        & 60.65±0.89 & 58.00±0.55  & 62.59±1.11  & 53.44±0.80   & 75.28±1.36    & 68.07±2.05    & 81.25±0.81     & 51.51±1.05     & 62.17±3.34    & 25.78±1.85     \\
VGAE                       & 63.19±0.53 & 60.35±0.47  & 61.65±0.66  & 54.28±0.28   & 76.50±0.50    & 59.07±0.56    & 80.46±0.53     & 55.56±4.53     & 62.50±2.38    & 24.40±2.57     \\
GraphMAE                   & \underline{66.44±0.46} & \underline{64.87±0.30}  & 67.95±0.46  & 59.41±0.39   & 89.14±0.89    & 82.93±0.93    & 80.05±0.64     & 59.38±1.49     & 61.83±3.37    & 29.27±2.15     \\
DGI                        & 63.33±0.56 & 60.71±0.49  & 65.93±1.02  & 55.83±0.53   & 91.22±1.47    & 85.00±1.66    & 80.05±0.87     & 59.16±1.88     & 61.83±2.83    & 28.63±1.92      \\
MVGRL                      & OOM        & OOM         & OOM         & OOM          & 88.57±1.15    & 76.50±1.17    & OOM            & OOM            & 62.00±3.79    & 28.26±4.20     \\
GRACE                      & 65.61±0.61 & 63.92±0.44  & \textbf{68.59±0.35}  & 60.15±0.45   & 92.00±1.39    & 88.64±0.67    & \textbf{83.43±0.63}     & \underline{60.45±1.46}     & 64.00±3.43    & \underline{34.86±3.43}  \\
RoSA                       & 64.06±0.67 & 62.44±0.39  & 67.07±0.65  & 57.68±0.44   & 90.78±2.27    & 85.93±2.14    & 82.39±0.42     & 57.45±2.16     & 64.17±4.10    & 32.20±2.15     \\
BGRL                       & 65.18±0.43 & 63.43±0.45  & 66.83±0.80  & 59.63±0.38   & 92.36±1.16    & 87.14±1.60    & 82.52±0.60     & 55.48±1.48     & 63.67±2.33    & 31.47±3.43     \\
COSTA                      & 65.05±0.80 & 62.37±0.45  & 66.76±0.87  & 55.73±0.36   & \underline{93.50±2.62}    & \underline{89.29±3.11}    & 83.15±0.30 & 55.03±3.22     & 61.66±2.58    & 32.39±2.13 \\
% ArCL                       &            &             & 67.64±0.57  & 59.71±0.44   &               &               &                &                & 65.00±3.94    & 35.41±1.97 \\      
SwAV                       & 62.22±0.53 & 59.79±0.53  & 64.65±0.94  & 55.06±0.39   & 89.00±0.79    & 81.72±0.66    & \underline{83.32±0.15}     & 59.69±1.97     & \underline{65.17±3.76}    & 29.36±2.01    \\
\midrule
MARIO                       & \textbf{67.11±0.46} & \textbf{65.28±0.34}  & \underline{68.46±0.40}  & \textbf{61.30±0.28}   & \textbf{94.36±1.21}    & \textbf{91.28±1.10}    & 82.31±0.54     & \textbf{63.33±1.72}     & \textbf{65.67±2.81}    & \textbf{37.15±2.37}     \\
\bottomrule
\end{tabular}}
\end{table*}

\begin{table*}[htp]
\caption{Experimental results of all methods under covariate shift. The bold font means the top-1 performance and the underline represents the second performance across the unsupervised methods. 'ID' represents in-distribution test performance and 'OOD' means out-of-distribution test performance. (OOM: out-of-memory on a GPU with 24GB memory)}
\label{tab:trans_covariate}
\centering
\scalebox{0.95}{
\begin{tabular}{l|cc|cc|cc|cc|cc}
\toprule
\toprule
\multirow{3}{*}{covariate shift} & \multicolumn{4}{c|}{GOOD-Cora}                                   & \multicolumn{2}{c|}{GOOD-CBAS} & \multicolumn{2}{c|}{GOOD-Twitch} & \multicolumn{2}{c}{GOOD-WebKB} \\
                           & \multicolumn{2}{c}{word} & \multicolumn{2}{c|}{degree}& \multicolumn{2}{c|}{color}    & \multicolumn{2}{c|}{language}   & \multicolumn{2}{c}{university} \\
                           & ID         & OOD         & ID          & OOD          & ID            & OOD           & ID             & OOD            & ID            & OOD            \\
\midrule
ERM                        & 70.50±0.41 & 64.69±0.33  & 72.46±0.49  & 55.53±0.50   & 92.00±3.08    & 77.57±1.29    & 70.98±0.41     & 49.35±5.09     & 39.34±1.79    & 14.52±3.14   \\
IRM                        & 70.48±0.26 & 64.53±0.57  & 71.98±0.34  & 53.72±0.46   & 90.86±2.41    & 78.86±1.67    & 69.81±0.95     & 49.11±2.82     & 38.52±3.30    & 13.97±2.80     \\
EERM                       & OOM        & OOM         & OOM         & OOM          & 65.00±2.57    & 57.43±3.60    & OOM            & OOM            & 46.07±4.55    & 27.40±7.65     \\
\midrule
GAE                        & 56.63±0.79 & 48.93±0.93  & 66.30±0.88  & 34.01±0.87   & 73.00±2.16    & 60.86±3.01    & 67.24±1.23     & 47.65±2.49     & 45.08±6.32    & 28.02±6.29    \\
VGAE                       & 62.02±0.66 & 54.12±0.86  & 69.41±0.57  & 44.20±1.29   & 62.29±2.04    & 63.29±1.11    & 66.99±1.43     & \underline{50.48±4.58}     & 48.85±4.68    & 20.87±6.69     \\
GraphMAE                   & 68.14±0.43 & 64.00±0.33  & \textbf{73.36±0.56}  & 53.75±0.55   & 67.28±3.03    & 67.28±1.49    & 68.84±1.20     & 48.02±2.79     & 48.03±4.34    & 30.00±8.09     \\
DGI                        & 60.85±0.75 & 57.03±0.67  & 68.97±0.41  & 41.75±0.88   & 69.57±4.09    & 59.71±3.43    & 68.43±1.05     & 44.83±1.61     & 48.52±5.04    & 21.11±7.50     \\
MVGRL                      & OOM        & OOM         & OOM         & OOM          & 65.00±1.94    & 64.15±0.77    & OOM            & OOM           & \textbf{54.10±5.39}    & 16.59±6.51     \\
GRACE                      & \underline{68.77±0.33} & \underline{64.21±0.41}  & 72.69±0.34  & \underline{56.10±0.63}   & \underline{93.57±1.83}    & \underline{89.29±3.40}    & \underline{71.12±0.87} & 46.21±1.54 & 49.67±5.82    & 28.10±4.68    \\
RoSA                       & 68.19±0.56 & 62.48±0.61  & 71.04±0.62  & 52.72±0.79   & 84.71±4.14    &79.14±3.51     & 70.58±0.36     & 45.83±1.72     & 52.30±4.24    & \underline{34.24±7.92}     \\
BGRL                       & 67.23±0.43 & 61.33±0.36  & 72.11±0.39  & 49.15±0.73   & 89.00±2.56    & 79.86±3.29    & \textbf{71.43±0.53}     & 43.86±0.94     & 51.80±5.55    & 30.32±7.61    \\
COSTA                      & 65.28±0.60 & 60.33±0.53  & 70.65±0.62  & 54.03±0.28   & 92.29±1.59    & 82.71±2.74    & 69.29±1.37     & 49.07±2.13     & 50.49±3.01    & 29.84±4.75   \\
SwAV                       & 63.29±1.01 & 56.98±0.94  & 70.27±0.73  & 43.00±0.52   & 89.57±1.12    & 81.43±1.69    & 69.19±0.93     & 49.37±2.96     & 49.84±4.82    & 30.55±6.72   \\
\midrule
MARIO                       & \textbf{69.99±0.54} & \textbf{65.06±0.34}  & \underline{72.73±0.43}  & \textbf{57.73±0.45}  & \textbf{94.57±2.46}    & \textbf{91.00±2.48}     & 68.31±0.78 & \textbf{57.37±1.37}     & \underline{53.94±3.23}    & \textbf{35.24±4.98}   \\
\bottomrule
\end{tabular}}

\end{table*}

\subsubsection{Inductive Setting}
In this subsection, we conduct experiments under the inductive settings, where the test nodes are kept unseen during training. This setting is more suitable for domain generalization.
% But we think it is more convincing that conduct experiments under inductive settings which means test nodes are unseen during training. This setting is more appropriate for domain generalization.

\noindent\textbf{Baselines:} For GOOD-WebKB and GOOD-CBAS datasets, we adopt ERM, IRM, GraphMAE, and GRACE as our baselines. And for Amazon-Photo and Elliptic datasets, we select ERM, EERM, and GRACE as our baselines.

\noindent\textbf{Experimental setup:} For GOOD-WebKB and GOOD-CBAS datasets, we use the same model configuration in Section~\ref{sec:trans}.
% Besides, we add experiments on Amazon-Photo dataset~\cite{yang2016revisiting} and Elliptic~\cite{elliptic} dataset in this subsection. 
For Amazon-Photo dataset~\cite{yang2016revisiting} and Elliptic~\cite{elliptic} dataset, they consist of many snapshots (training data and testing data use different snapshots) which are naturally inductive. For Amazon-Photo dataset, we use 2-layer GCN~\cite{gcn} as the encoder and for elliptic dataset, we use 5-layer GraphSAGE~\cite{sage} as encoder following~\cite{eerm}.

% Figure environment removed

\noindent\textbf{Analysis:}
According to Figure~\ref{fig:amazon},\ref{fig:elliptic},\ref{fig:ind_con},\ref{fig:ind_cov}, we can draw following conclusions:
firstly, based on Figure~\ref{fig:amazon}, it is evident that our method outperforms other representative supervised and self-supervised methods on all test graphs (T1$\sim$T8). This superiority is reflected in the larger median value of our method compared to others. For instance, MARIO achieves over a 3\% absolute improvement compared to ERM in terms of the mean value of eight median values. Additionally, our method demonstrates higher stability across different random initializations, as indicated by the closer proximity of the first and third quartile values to the median value~(\eg, the difference of first and third quartile values of ERM, EERM, GRACE and MARIO are 4.2, 3.3, 6.7 and 1.0 on T8 respectively which indicates MARIO is much more stable than other methods). Furthermore, our method exhibits consistent performance across different graphs (\eg, The standard deviation of median values on T1$\sim$T8 for ERM, EERM, GRACE, and MARIO are 0.4, 1.1, 1.2, and 0.3, respectively.), indicating its robustness to environmental variations and its ability to extract invariant features: $g(G^e) \approx g(G^{e'})$ for all $e, e' \in \mathcal{E}^\text{train}$. In summary, our method showcases enhanced OOD generalization capabilities.
% $g(G^e)g(G^e^\prime)$ where $any e, e^\prime in \mathcal{E}^{train}$

Secondly, from the results presented in Figure~\ref{fig:elliptic}, we can observe that our method averagely harvests 10.9\% absolute improvement over GRACE and 12.5\% absolute improvement over EERM in terms of F1 scores on Elliptic dataset. This demonstrates the effectiveness of our method in handling distribution shifts and improving performance compared to existing approaches. It is worth noting that GRACE's performance worsens over time, indicating its inability to handle distribution shifts effectively. In contrast, our method consistently achieves better F1 scores, except for T9, which is caused by the dark market shutdown occurred after T7~\cite{elliptic}. The emergence of such an event introduces significant variations in data distributions, which subsequently results in performance degradation for all methods. Indeed, this event serves as an unpredictable external factor that introduces significant challenges for models trained on limited training data. The results indicate that the performance heavily depends on available training data. Nonetheless, our approach outperforms other methods even in such an extreme case. This highlights the effectiveness of our method in addressing distribution shifts and improving generalization performance.

Finally, based on the observations from Figure~\ref{fig:ind_con} and Figure~\ref{fig:ind_cov} MARIO demonstrates the best performances on both ID and OOD test sets for GOOD-WebKB and GOOD-CBAS datasets, under both concept shift and covariate shift. Notably, MARIO outperforms other methods by more than 3\% and 10\% absolute improvement on GOOD-WebKB and GOOD-CBAS, respectively, under covariate shift. We can draw similar conclusions as discussed in Section~\ref{sec:trans}. Even under the inductive setting, our method continues to demonstrate excellent OOD generalization capabilities and achieves comparable or even improved in-distribution test performance. These statistical results further validate the effectiveness of our method in handling distribution shifts and enhancing generalization performance.

Overall, the observations we have made provide strong evidence of the great capacity of our method for handling distribution shifts, validating its effectiveness and potential for real-world applications.



% Figure environment removed

% Figure environment removed


% Figure environment removed


\subsection{Ablation Studies}\label{sec:ablation}
\noindent Table~\ref{tab:aba} provides a detailed analysis of the effect of each component according to our proposed recipe for improving OOD generalization in graph contrastive learning. Let's examine the different variants of our method and their impact on performance.
Specifically, MARIO~(w/o ad) represents MARIO without  adversarial augmentation. MARIO~(w/o cmi) denotes we only maximize the mutual information between positive pairs without considering conditional mutual information. MARIO~(w/o cmi, ad) means a vanilla graph contrastive method that is similar to GRACE. 

From Table~\ref{tab:aba}, we can find MARIO~(w/o cmi) lags far behind MARIO on OOD test set which demonstrates appropriately minimizing the redundant information (\ie, conditional mutual information) is essential to improve OOD generalization of GCL methods. And adversarial augmentation can also boost OOD generalization because it can approximately serve as a supermum operator to learn more invariant features  discussed in Section~\ref{sec:aug}. Based on the analysis of these variants, it is evident that the proposed improvements on data augmentation and contrastive loss in the recipe are both effective in enhancing graph OOD generalization. Each component contributes to the overall performance improvement, and their combination leads to a stronger self-supervised graph learner in terms of graph OOD generalization. 

In short, the findings from Table~\ref{tab:aba} support the rationale behind your proposed recipe and provide empirical evidence of the effectiveness of each proposed component. By incorporating these enhancements, our method achieves superior performance in handling distribution shifts and improving graph OOD generalization in graph contrastive learning.
\begin{table*}[htp]
\caption{Ablation studies for MARIO by masking each component.}
\label{tab:aba}
\centering
\scalebox{0.9}{
\begin{tabular}{l|cc|cc|cc|cc|cc}
\toprule
\toprule
\multirow{3}{*}{concept shift} & \multicolumn{4}{c|}{GOOD-Cora}                       & \multicolumn{2}{c|}{GOOD-CBAS} & \multicolumn{2}{c|}{GOOD-Twitch} & \multicolumn{2}{c}{GOOD-WebKB} \\
                           & \multicolumn{2}{c}{word} & \multicolumn{2}{c|}{degree}& \multicolumn{2}{c|}{color}    & \multicolumn{2}{c|}{language}   & \multicolumn{2}{c}{university} \\
                           & ID         & OOD         & ID          & OOD          & ID            & OOD           & ID             & OOD            & ID            & OOD            \\
\midrule
MARIO                      & \textbf{67.11±0.46} & \textbf{65.28±0.34}  & \textbf{68.46±0.40}  & \textbf{61.30±0.28}      & \textbf{94.36±1.21}  & \textbf{91.28±1.10}    & 82.31±0.54     & \textbf{63.33±1.72}     & \textbf{65.67±2.81}    & \textbf{37.15±2.37}     \\
MARIO(w/o ad)              & 66.23±0.53 & 64.02±0.18  & 67.88±0.38  & 60.46±0.29   & 93.21±1.25    & 90.29±0.91    & 82.42±0.73     & 60.50±1.02     & 64.83±2.83    & 36.51±3.25    \\
MARIO(w/o cmi)             & 65.32±0.60 & 63.51±0.32  & 68.14±0.32  & 61.19±0.34   & 94.15±1.23    & 90.57±1.96    & \textbf{82.51±0.56}     & 61.41±2.63     & 64.50±4.35    & 35.78±2.53     \\
MARIO(w/o cmi, ad)         & 64.67±0.55 & 63.11±0.32  & 67.95±0.65  & 60.01±0.57   & 93.36±1.66    & 89.64±1.73    & 81.90±0.75     & 60.12±1.60     & 64.17±3.67    & 34.13±2.38     \\
\bottomrule
\end{tabular}}
\end{table*}
% & 65.32±0.60 & 63.51±0.32 exchange 64.67±0.55 & 63.11±0.32
% 68.14±0.32       id ood test: 60.95±0.43       ood ood test: 61.19±0.34


\subsection{Sensitivity Analysis}\label{sec:sensitivity}
\noindent In this subsection, we will analyze some important hyper-parameters of our method. We conduct sensitivity analysis on GOOD-WebKB dataset with concept shift, we chose two sensitive hyper-parameters (\ie, the coefficient $\gamma$ of condition mutual information in Equation~\ref{equ:cmi} and the number of prototypes $|C|$ in Equation~\ref{equ:pq}). The coefficient of CMI range in $[0.001, 0.01, 0.1, 0.5, 1]$ and the number of prototypes $|C|$ ranges in $[10, 50, 100, 200, 300]$. From Figure~\ref{fig:sensitivity}, we can observe that $\gamma$ reaches 0.1 and $|C|$ reaches 100 or 200 can achieve the best OOD test accuracy. Both higher and lower values of $\gamma$ result in suboptimal performance. This finding aligns with previous research such as DIB~\cite{dib}, indicating that an appropriate compression level is crucial for achieving optimal performance. Extremely high or low compression values are not ideal. 

Regarding the number of prototypes $|C|$, based on the results shown in Figure~\ref{fig:sensitivity}, it is found that setting $|C|=100$ leads to the best performance in terms of OOD test accuracy. This choice provides a moderate number of pseudo labels, which is beneficial for the learning process. 

Based on the sensitivity analysis, we determined that setting $\gamma=0.1$ and $|C|=100$ on most datasets. These hyperparameter values strike a balance between compression level and the number of prototypes, resulting in improved graph OOD generalization.
% Figure environment removed


\subsection{Integrated with Other Models}\label{sec:other_models}
% Figure environment removed

\begin{table}[htp]
\caption{Results of different learning approaches with different encoding models (\ie, GCN, GraphSAGE, GAT).}
\label{tab:others}
\centering
\scalebox{0.9}{
\begin{tabular}{cc|cc|cc}
\toprule
\toprule
\multirow{3}{*}{Model}& \multirow{3}{*}{Method} & \multicolumn{2}{c|}{GOOD-CBAS} & \multicolumn{2}{c}{GOOD-WebKB} \\
                & & \multicolumn{2}{c|}{color}    & \multicolumn{2}{c}{university} \\
                &   & ID          & OOD         & ID          & OOD            \\
\midrule
\multirow{3}{*}{GCN} 
&ERM               & 89.79±1.39 & 83.43±1.19  &  62.67±1.53 & 26.33±1.09         \\
&GRACE             & 92.00±1.39 & 88.64±0.67  &  64.00±3.43 & 34.86±3.43        \\
&MARIO             & 94.36±1.21 & 91.28±1.10  &  65.67±2.81 & 37.15±2.37        \\ \bottomrule
\multirow{3}{*}{SAGE} 
&ERM               & 95.07±1.51 & 75.14±1.19  & 73.67±2.08  & 46.33±3.42       \\
&GRACE             & 95.29±1.11 & 74.43±2.36  & 70.50±5.06  & 49.54±3.83        \\
&MARIO             & 96.00±1.07 & 76.29±3.01  & 71.00±3.82  & 51.74±4.63        \\ \bottomrule
\multirow{3}{*}{GAT} 
&ERM               & 78.64±3.63 & 72.93±2.64  & 61.33±3.71  & 28.99±2.63        \\
&GRACE             & 84.57±1.79 & 78.36±1.60  & 59.50±2.36  & 35.78±3.26        \\
&MARIO             & 84.93±1.95 & 80.43±1.89  & 62.17±4.78  & 38.17±3.10        \\
\bottomrule
\end{tabular}}
\end{table}



\noindent In the subsection, we demonstrate the model-agnostic nature of the recipe by integrating it with various graph neural network (GNN) models, including GCN, GraphSAGE, and GAT.

From Table~\ref{tab:others}, it can be observed that regardless of the specific GNN model used as the encoder, our method consistently achieves the best performance on the OOD test set. This indicates the effectiveness and robustness of our method across different GNN models.
By achieving superior performance across different GNN models, MARIO demonstrates its versatility and ability to improve the OOD generalization of various graph neural models. This highlights the broad applicability and effectiveness of our recipe in enhancing the performance of different GNN encoders.

Furthermore, we integrate our recipe with other GCL methods in Appendix~\ref{app:other_methods}. The results demonstrate our recipe can boost the OOD generalization ability of various GCL methods which means our recipe can serve as a plug-in for many current classical GCL methods.

% Figure environment removed

\subsection{Visualization}\label{sec:vis}
\subsubsection{Metric Score Curves}
We present metric score curves for ERM and MARIO, including training, ID validation, ID testing, OOD validation, and OOD testing accuracy, in Figure~\ref{fig:curve2}. Notably, MARIO demonstrates superior convergence with approximately 10\% absolute improvement on the OOD test set compared to ERM. Furthermore, MARIO effectively narrows the performance gap between in-distribution and out-of-distribution performance, showcasing its efficacy in enhancing OOD generalization for graph data. More metric score curves can be found in Appendix~\ref{app:curves}.


\subsubsection{Feature Visualization}
In order to assess the quality of learned embeddings, we adopt t-SNE~\cite{tsne} to visualize the node embedding on GOOD-Cora dataset (concept shift in word domain) using random-init of GCN, EERM, GRACE, and MARIO, where different classes have different colors in Figure~\ref{fig:vis}. For clarity, we select eight classes with the largest number of nodes to enhance the informativeness and interpretability of the visualization. We can observe that the 2D projection of node embeddings learned by MARIO has a better separation of clusters, which indicates the model can help learn representative features for downstream tasks. It has to note that we depict both ID nodes and OOD nodes in the same figure. 

Besides, we also separately visualize ID nodes and OOD nodes in the different figures in the Appendix~\ref{app:feature}. And we can find MARIO performs a clearer separation of clusters whether on ID nodes or OOD nodes compared to other methods.




\section{Conclusion and Future Work}
In this work, I design corruption-robust algorithms for the Lipschitz contextual search problem. I present the \emph{agnostic checking} technique and demonstrate its effectiveness in designing corruption-robust algorithms. There are several open problems for future research. First, in the algorithm I propose for pricing loss, the schedule for agnostic checks is fixed upfront. Can the learner design an adaptive checking schedule for the pricing loss? Second, this work assumes the learner has knowledge of the Lipschitz constant $L$. Can the learner design efficient no-regret algorithms without knowledge of $L$? 

This work was partially supported by CONICET (Argentina) (PUE 0015-2016), by the Santa Fe province (Argentina) Government under Grant PEICID-2021-170, by the Spanish Government under Grants PGC2018-096367-B-I00 and
PID2021-127685NB-I00 and by the Aragon Government under Grant DGA T45 17R/FSE.




% \bibliography{reference}
\begin{thebibliography}{10}

\bibitem{eilers2021product}Eilers, M., Meier, S. \& Müller, P. Product Programs in the Wild: Retrofitting Program Verifiers to Check Information Flow Security. {\em Computer Aided Verification (CAV)}. (2021)
\bibitem{tiwari2009complete}Tiwari, M., Wassel, H., Mazloom, B., Mysore, S., Chong, F. \& Sherwood, T. Complete information flow tracking from the gates up. {\em Proceedings Of The 14th International Conference On Architectural Support For Programming Languages And Operating Systems}. pp. 109-120 (2009)

\bibitem{tiwari2009execution}Tiwari, M., Li, X., Wassel, H., Chong, F. \& Sherwood, T. Execution leases: A hardware-supported mechanism for enforcing strong non-interference. {\em Proceedings Of The 42nd Annual IEEE/ACM International Symposium On Microarchitecture}. pp. 493-504 (2009)

\bibitem{jin2012proof}Jin, Y. \& Makris, Y. Proof carrying-based information flow tracking for data secrecy protection and hardware trust. {\em 2012 IEEE 30th VLSI Test Symposium (VTS)}. pp. 252-257 (2012)

\bibitem{li2011caisson}Li, X., Tiwari, M., Oberg, J., Kashyap, V., Chong, F., Sherwood, T. \& Hardekopf, B. Caisson: A Hardware Description Language for Secure Information Flow. {\em Proceedings Of The 32Nd ACM SIGPLAN Conference On Programming Language Design And Implementation}. pp. 109-120 (2011), http://doi.acm.org/10.1145/1993498.1993512

\bibitem{li2014sapper}Li, X., Kashyap, V., Oberg, J., Tiwari, M., Rajarathinam, V., Kastner, R., Sherwood, T., Hardekopf, B. \& Chong, F. Sapper: A Language for Hardware-level Security Policy Enforcement. {\em Proceedings Of The 19th International Conference On Architectural Support For Programming Languages And Operating Systems}. pp. 97-112 (2014), http://doi.acm.org/10.1145/2541940.2541947

\bibitem{zhang2015secverilog}Zhang, D., Wang, Y., Suh, G. \& Myers, A. A Hardware Design Language for Timing-Sensitive Information-Flow Security. {\em Proceedings Of The Twentieth International Conference On Architectural Support For Programming Languages And Operating Systems}. pp. 503-516 (2015), http://doi.acm.org/10.1145/2694344.2694372

\bibitem{bidmeshki2015vericoq}Bidmeshki, M. \& Makris, Y. VeriCoq: A Verilog-to-Coq converter for proof-carrying hardware automation. {\em 2015 IEEE International Symposium On Circuits And Systems (ISCAS)}. pp. 29-32 (2015)

\bibitem{hu2016detecting}Hu, W., Mao, B., Oberg, J. \& Kastner, R. Detecting hardware trojans with gate-level information-flow tracking. {\em Computer}. \textbf{49}, 44-52 (2016)

\bibitem{kong2017using}Kong, S., Shen, Y. \& Zhou, H. Using security invariant to verify confidentiality in hardware design. {\em Proceedings Of The On Great Lakes Symposium On VLSI 2017}. pp. 487-490 (2017)

\bibitem{ardeshiricham2017register}Ardeshiricham, A., Hu, W., Marxen, J. \& Kastner, R. Register transfer level information flow tracking for provably secure hardware design. {\em Proceedings Of The Conference On Design, Automation \& Test In Europe (DATE)}. pp. 1695-1700 (2017), http://dl.acm.org/citation.cfm?id=3130379.3130775

\bibitem{ardeshiricham2017clepsydra}Ardeshiricham, A., Hu, W. \& Kastner, R. Clepsydra: Modeling timing flows in hardware designs. {\em 2017 IEEE/ACM International Conference On Computer-Aided Design (ICCAD)}. pp. 147-154 (2017)

\bibitem{deng2017secchisel}Deng, S., Gümüşoğlu, D., Xiong, W., Gener, Y., Demir, O. \& Szefer, J. SecChisel: language and tool for practical and scalable security verification of security-aware hardware architectures. {\em Cryptology EPrint Archive}. (2017)

\bibitem{bidmeshki2017information}Bidmeshki, M., Antonopoulos, A. \& Makris, Y. Information flow tracking in analog/mixed-signal designs through proof-carrying hardware IP. {\em Design, Automation \& Test In Europe Conference \& Exhibition (DATE), 2017}. pp. 1703-1708 (2017)

\bibitem{boraten2018securing}Boraten, T. \& Kodi, A. Securing NoCs against timing attacks with non-interference based adaptive routing. {\em 2018 Twelfth IEEE/ACM International Symposium On Networks-on-Chip (NOCS)}. pp. 1-8 (2018)

\bibitem{pilato2018tainthls}Pilato, C., Wu, K., Garg, S., Karri, R. \& Regazzoni, F. Tainthls: High-level synthesis for dynamic information flow tracking. {\em IEEE Transactions On Computer-Aided Design Of Integrated Circuits And Systems}. \textbf{38}, 798-808 (2018)

\bibitem{zagieboylo2019using}Zagieboylo, D., Suh, G. \& Myers, A. Using information flow to design an ISA that controls timing channels. {\em 2019 IEEE 32nd Computer Security Foundations Symposium (CSF)}. pp. 272-27215 (2019)

\bibitem{pieper2020dynamic}Pieper, P., Herdt, V., Große, D. \& Drechsler, R. Dynamic information flow tracking for embedded binaries using SystemC-based virtual prototypes. {\em 2020 57th ACM/IEEE Design Automation Conference (DAC)}. pp. 1-6 (2020)

\bibitem{restuccia2021aker}Restuccia, F., Meza, A. \& Kastner, R. AKER: A design and verification framework for safe and secure soc access control. {\em IEEE/ACM International Conference On Computer Aided Design (ICCAD)}. (2021), https://par.nsf.gov/servlets/purl/10298115

\bibitem{restuccia2022framework}Restuccia, F., Meza, A., Kastner, R. \& Oberg, J. A Framework for Design, Verification, and Management of SoC Access Control Systems. {\em IEEE Transactions On Computers}. (2022), https://kastner.ucsd.edu/wp-content/uploads/2022/10/admin/tcomputer-aker22.pdf

\bibitem{cherupalli2017software}Cherupalli, H., Duwe, H., Ye, W., Kumar, R. \& Sartori, J. Software-based gate-level information flow security for IoT systems. {\em 50th IEEE/ACM International Symposium On Microarchitecture}. (2017), https://dl.acm.org/doi/pdf/10.1145/3123939.3123955

\bibitem{fadiheh2023exhaustive}Fadiheh, M., Wezel, A., Muller, J., Bormann, J., Ray, S., Fung, J., Mitra, S., Stoffel, D. \& Kunz, W. An Exhaustive Approach to Detecting Transient Execution Side Channels in RTL Designs of Processors. {\em IEEE Transactions On Computers}. \textbf{72}, 222-235 (2023,1)

\bibitem{wu2022exert}Wu, J., Fowze, F. \& Forte, D. EXERT: EXhaustive Integrity Analysis for Information Flow Security. {\em Asian Hardware Oriented Security And Trust Symposium (AsianHOST)}. (2022), https://dforte.ece.ufl.edu/wp-content/uploads/sites/65/2022/09/EXERT%5C_AsianHost.pdf

\bibitem{athalye2022knox}Athalye, A., Kaashoek, M. \& Zeldovich, N. Verifying Hardware Security Modules with Information-Preserving Refinement. {\em OSDI}. (2022)

\bibitem{fowze2022eisec}Fowze, F., Choudhury, M. \& Forte, D. EISec: Exhaustive Information Flow Security of Hardware Intellectual Property Utilizing Symbolic Execution. {\em Asian Hardware Oriented Security And Trust Symposium (AsianHOST)}. (2022)

\bibitem{athalye2019notary}Athalye, A., Belay, A., Kaashoek, M., Morris, R. \& Zeldovich, N. Notary: A Device for Secure Transaction Approval. {\em 27th Symposium On Operating Systems Principles (SOSP)}. (2019), https://doi.org/10.1145/3341301.3359661

\bibitem{meza2023hyperflowgraph}Meza, A. \& Kastner, R. Information Flow Coverage Metrics for Hardware Security Verification.  (2023), arXiv 2304.08263

\bibitem{ryan2023countering}Ryan, K. \& Sturton, C. Countering the Path Explosion Problem in the Symbolic Execution of Hardware Designs.  (2023), arXiv 2304.05445 

\bibitem{dorsey2020intel}Dorsey, V. \& Morhardt, C. Intel Security Development Lifecycle. (Intel,2020)

\bibitem{he2015model}He, S., Roe, N., Wood, E., Nachtigal, N. \& Helms, J. Model of the Product Development Lifecycle. (Sandia National Laboratories,2015)

\bibitem{YangSP2016}Yang, K., Hicks, M., Dong, Q., Austin, T. \& Sylvester, D. A2: Analog Malicious Hardware. {\em 2016 IEEE Symposium On Security And Privacy (SP)}. pp. 18-37 (2016)

\bibitem{or1200}. OpenRISC 1200 Implementation. , https://github.com/openrisc/or1200

\bibitem{msp430}. openMSP430. , https://opencores.org/projects/openmsp430

\bibitem{farzana2019soc}Farzana, N., Rahman, F., Tehranipoor, M. \& Farahmandi, F. SoC Security Verification using Property Checking. {\em 2019 IEEE International Test Conference (ITC)}. pp. 1-10 (2019)

\bibitem{TrustHub2}Farzana, N., Farahmandi, F. \& Tehranipoor, M. SoC Security Properties and Rules. {\em IACR Cryptol. EPrint Arch.}. \textbf{2021} pp. 1014 (2021)

\bibitem{hicks2015specs}Hicks, M., Sturton, C., King, S. \& Smith, J. SPECS: A Lightweight Runtime Mechanism for Protecting Software from Security-Critical Processor Bugs. {\em ASPLOS}. pp. 517-529 (2015)

\bibitem{bilzor2011security}Bilzor, M., Huffmire, T., Irvine, C. \& Levin, T. Security Checkers: Detecting processor malicious inclusions at runtime. {\em HOST}. (2011)

\bibitem{zhang2017scifinder}Zhang, R., Stanley, N., Griggs, C., Chi, A. \& Sturton, C. Identifying Security Critical Properties for the Dynamic Verification of a Processor. {\em ASPLOS}. pp. 541-554 (2017)

\bibitem{zhang2020transys}Zhang, R. \& Sturton, C. Transys: Leveraging Common Security Properties Across Hardware Designs. {\em Proceedings Of The Symposium On Security And Privacy (S\&P)}. (2020)

\bibitem{trippel2020ICAS}Trippel, T., Shin, K., Bush, K. \& Hicks, M. ICAS: an Extensible Framework for Estimating the Susceptibility of IC Layouts to Additive Trojans. {\em 2020 IEEE Symposium On Security And Privacy (SP)}. pp. 1742-1759 (2020)

\bibitem{Deutschbein2022JCEN}Deutschbein, C., Meza, A., Restuccia, F., Kastner, R. \& Sturton, C. Isadora: Automated Information Flow Property Generation for Hardware Security Verification. {\em Journal Of Cryptographic Engineering (JCEN)}. (2022)

\bibitem{zhang2021sidechannel}Zhang, T., Park, J., Tehranipoor, M. \& Farahmandi, F. PSC-TG: RTL Power Side-Channel Leakage Assessment with Test Pattern Generation. {\em 2021 58th ACM/IEEE Design Automation Conference (DAC)}. pp. 709-714 (2021)

\bibitem{torlak2014rosette}Torlak, E. \& Bodik, R. A Lightweight Symbolic Virtual Machine for Solver-Aided Host Languages. {\em Proceedings Of The 35th ACM SIGPLAN Conference On Programming Language Design And Implementation}. pp. 530-541 (2014), https://doi.org/10.1145/2594291.2594340

\bibitem{cha2012mayhem}Cha, S., Avgerinos, T., Rebert, A. \& Brumley, D. Unleashing Mayhem on Binary Code. {\em Proceedings Of The 2012 IEEE Symposium On Security And Privacy}. pp. 380-394 (2012)

\bibitem{bao2021symbolic}Bao, Q., Wang, Z., Li, X., Larus, J. \& Wu, D. Abacus: Precise side-channel analysis. {\em International Conference On Software Engineering (ICSE)}. pp. 797-809 (2021)

\bibitem{wang2017cached}Wang, S., Wang, P., Liu, X., Zhang, D. \& Wu, D. CacheD: Identifying cache-based timing channels in production software. {\em USENIX Security Symposium}. pp. 235-252 (2017)

\bibitem{wang2019identifying}Wang, S., Bao, Y., Liu, X., Wang, P., Zhang, D. \& Wu, D. Identifying Cache-Based Side Channels through Secret-Augmented Abstract Interpretation. {\em 28th USENIX Security Symposium (USENIX Security 19)}. pp. 657-674 (2019,8), https://www.usenix.org/conference/usenixsecurity19/presentation/wang-shuai

\bibitem{brotzman2019casym}Brotzman, R., Liu, S., Zhang, D., Tan, G. \& Kandemir, M. Casym: Cache aware symbolic execution for side channel detection and mitigation. {\em Symposium On Security And Privacy (SP)}. (2019)

\bibitem{guarnier2020spectector}Guarnieri, M., Köpf, B., Morales, J., Reineke, J. \& Sánchez, A. Spectector: Principled Detection of Speculative Information Flows. {\em 2020 IEEE Symposium On Security And Privacy (SP)}. pp. 1-19 (2020)

\bibitem{avgerinos2014automatic}Avgerinos, T., Cha, S., Rebert, A., Schwartz, E., Woo, M. \& Brumley, D. Automatic exploit generation. {\em Communications Of The ACM}. \textbf{57}, 74-84 (2014)

\bibitem{avgerinos2011automatic}Avgerinos, T., Hao, B. \& Brumley, D. Automatic exploit generation. {\em Network And Distributed System Security Symposium (NDSS)}. (2011)

\bibitem{renzelmann2012symdrive}Renzelmann, M., Kadav, A. \& Swift, M. SymDrive: Testing Drivers without Devices. {\em 10th USENIX Symposium On Operating Systems Design And Implementation}. (2012), https://www.usenix.org/conference/osdi12/technical-sessions/presentation/renzelmann

\bibitem{zhang2018end}Zhang, R., Deutschbein, C., Huang, P. \& Sturton, C. End-to-End Automated Exploit Generation for Validating the Security of Processor Designs. {\em Proceedings Of The International Symposium On Microarchitecture (MICRO)}. (2018)

\bibitem{Shen2018SymbolicEB}Shen, L., Mu, D., Cao, G., Qin, M., Blackstone, J. \& Kastner, R. Symbolic execution based test-patterns generation algorithm for hardware Trojan detection. {\em Comput. Secur.}. \textbf{78} pp. 267-280 (2018)

\bibitem{clarkson2010hyperproperties}Clarkson, M. \& Schneider, F. Hyperproperties. {\em J. Comput. Secur.}. \textbf{18}, 1157-1210 (2010,9), http://dl.acm.org/citation.cfm?id=1891823.1891830

\bibitem{Kozyri2022expressing}Kozyri, E., Chong, S. \& Myers, A. Expressing Information Flow Properties. {\em Foundations And Trends® In Privacy And Security}. \textbf{3}, 1-102 (2022), http://dx.doi.org/10.1561/3300000008

\bibitem{meza2022safety}Meza, A., Restuccia, F., Kastner, R. \& Oberg, J. Safety verification of third-party hardware modules via information flow tracking. {\em 1st Real-Time Intelligent Edge Computing Workshop (RAGE)}. (2022), https://kastner.ucsd.edu/wp-content/uploads/2022/08/admin/rage22-safety.pdf

\bibitem{Deutschbein2021Isadora}Deutschbein, C., Meza, A., Restuccia, F., Kastner, R. \& Sturton, C. Isadora: Automated Information Flow Property Generation for Hardware Designs. {\em Proceedings Of The Workshop On Attacks And Solutions In Hardware Security (ASHES)}. (2021)

\bibitem{ferraiuolo2017secverilog}Ferraiuolo, A., Xu, R., Zhang, D., Myers, A. \& Suh, G. Verification of a Practical Hardware Security Architecture Through Static Information Flow Analysis. {\em Proceedings Of The Twenty-Second International Conference On Architectural Support For Programming Languages And Operating Systems}. pp. 555-568 (2017), http://doi.acm.org/10.1145/3037697.3037739

\bibitem{ardeshiricham2019verisketch}Ardeshiricham, A., Takashima, Y., Gao, S. \& Kastner, R. VeriSketch: Synthesizing Secure Hardware Designs with Timing-Sensitive Information Flow Properties. {\em Proceedings Of The 2019 ACM SIGSAC Conference On Computer And Communications Security}. pp. 1623-1638 (2019)


\end{thebibliography}


\newpage
\appendix
\begin{comment}
\section{System Architecture}
\label{appendix:architecture}
\system has a novel modularized system architecture with three key components: 
\emph{StreamManager}, 
\emph{TxnManager} and \emph{TxnScheduler}. 
These components are instantiated in each thread locally.
The execution outline of \system is presented in Algorithm~\ref{alg:algo}.
Transactional stream processing is continuous and potentially never ends (Line 1$\sim$8).
The dependency resolution and execution of state transactions are separated into two non-overlapping phases by punctuations~\cite{Tucker:2003:EPS:776752.776780} (Line 2 and 5), which guarantees that no subsequent input event will have a smaller timestamp. 
Effectively, a batch of state transactions is collected during the first phase, and processed during the second phase.

In the first phase (i.e., stream processing phase), 
the \emph{StreamManager} conducts preprocessing for every input event ($e$). Similar to some prior works~\cite{tstream}, state transactions may be issued but not immediately processed during preprocessing (Line 3).
The \emph{pre\_processing} and \emph{post\_processing} functions are exposed as APIs to users.
The \emph{TxnManager} handles dependency resolution (Line 4) among state transactions and insert decomposed operations to construct a \tpg. We discuss the detailed two-phase \tpg construction process in Section~\ref{subsec:construction}.

In the second phase  (i.e., transaction processing phase), 
the \emph{TxnManager} is first involved again to refine (Line 6) the constructed \tpg with further dependency resolution.
The \emph{TxnScheduler} 
schedules operations for concurrent execution based on the constructed \tpg according to the three dimensions of scheduling decisions (Line 7). 
In particular, a scheduling decision model $M$ is instantiated based on the constructed \tpg (Line 14).
\textbf{\circled{1}} Guided by $M$, execution threads adopt an exploration strategy (Section~\ref{subsec:explore}) to explore the constructed \tpg for operations available to be scheduled constrained by dependencies. 
\textbf{\circled{2}} 
During exploration, one or multiple operations may be treated as the 
% basic 
unit of scheduling (Section~\ref{subsec:granularity}). 
Subsequently, \textbf{\circled{3}} every thread executes operation(s) in the unit of scheduling with various abort handling mechanisms (Section~\ref{subsec:abort_handling}).
Only when state transactions are processed (i.e., committed or aborted) can the associated input events be postprocessed (Line 8) by the \emph{StreamManager} based on transaction processing results.
\end{comment}

\begin{comment}
\begin{algorithm}
\footnotesize
    \KwData{$e$ \tcp{Input event}}
    \KwData{$txn_{ts}$ \tcp{State transaction}}
    \KwData{$G$ \tcp{The currently constructed TPG}}
    \While{!finish processing of input streams}{
        \eIf(\tcp*[h]{Phase 1}){\text{$e$ is not a $punctuation$}}{
                $txn_{ts}$ $\gets$ PRE\_Processing($e$)\;
                \textbf{TPG\_Construction}($G$, $txn_{ts}$)\; 
          }(\tcp*[h]{Phase 2}){
                \textbf{TPG\_Refinement}($G$)\; 
                \textbf{TXN\_Scheduling}($G$)\; 
                POST\_Processing()\;
          }
    }
    
    \SetKwFunction{FMain}{TPG\_Construction}
    \SetKwProg{Fn}{Function}{:}{}
    \Fn{\FMain{$G$, $txn_{ts}$}}{
        $O_{1..k}$ $\gets$ \textbf{Partition} $txn_{ts}$\;
        \ForEach{\text{operation $O_{i}$ $\in$ $O_{1..k}$}}{
            \textbf{Identify} its \ld\;
            $G$ $\gets$ $G$ + $O_{i}$ \;
        }
    }
    \SetKwFunction{FMain}{TPG\_Refinement}
    \SetKwProg{Fn}{Function}{:}{}
    \Fn{\FMain{$G$}}{
        \ForEach{\text{vertex $e_{i}$ $\in$ $G$}}{
            \textbf{Identify} its \td, \pd\;
        }
    }
    
    \SetKwFunction{FMain}{TXN\_Scheduling}
    \SetKwProg{Fn}{Function}{:}{}
    \Fn{\FMain{$G$}}{
        $M$ $\gets$ Instantiated with $G$;\tcp{A decision model}
        \While{!finish scheduling of $G$
        }{
          \textbf{\circled{2}} $Scheduling Unit$ $\gets$ \textbf{\circled{1}} \emph{Explore}($G$, $M$)\; 
            \textbf{\circled{3}} \emph{Execute with Abort Handling} ($Scheduling Unit$)\; 
        }
    }
  \caption{Execution Outline of \system}
  \label{alg:algo}
\end{algorithm}
\end{comment}

\end{document}
