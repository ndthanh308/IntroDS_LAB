\begin{table}[t]
\centering

\begin{subtable}{0.48\textwidth}
\centering
\renewcommand{\arraystretch}{1.5}
\begin{tabular}{l|ccccc}
\toprule[1mm]
Dimension & C & 2C & 3C & 4C & 5C \\ \hdashline[1pt/1pt]
mIoU & 70.55 & \color[HTML]{FF0000} \textbf{72.87} & 71.52 & 71.93 & 70.57 \\ \hline
\end{tabular}
\caption{Exploration of the embedding dimension in Global Rectification. The \textit{Dimension} denotes embedding dimension.}
\label{subtab:tableB}
\end{subtable}

\hfill

\begin{subtable}{0.48\textwidth}
\centering
\renewcommand{\arraystretch}{1.5}
\begin{tabular}{l|cccccc}
\toprule[1mm] %\midrule[1.5pt]
Layers & 0 & 1 & 2 & 3 & 4 & P \\ \hdashline[1pt/1pt]
mIoU & 72.87 & 73.13 & 73.06 & \color[HTML]{FF0000} \textbf{73.39} & 71.49 & 61.49\\ 
\hline
% \multicolumn{3}{c|}{Parallel} & \multicolumn{3}{c}{61.49} \\
% \hline
\end{tabular}
\caption{Exploration of Local Rectification.~\textbf{P} denotes the parallel addition of features from both groups.}
%
\label{subtab:tableA}
\end{subtable}

\caption{Ablation Study of the Central Angular Rectification Module~(CARM). \textit{Dimension} denotes the embedding dimension in Global Operation, and \textit{Layers} presents the number of 3D convolutions within Local Operation. mIoU (\%) is reported. The best results are highlighted in red.}
\label{tab:twosubtables}
\end{table}
{To rigorously assess the efficacy of our proposed method for rectifying asymmetric features, we embarked on a comprehensive suite of ablation studies centered around the innovative CARM framework.}~
As depicted in Table \ref{subtab:tableB}, we remove the local rectification module and increase the embedding dimension of the global module from $C$ to $4C$ to assess the optimal manner. Subsequently, upon projecting the dimensions to $2C$, our network achieves the highest score of $72.87\%$. Besides, excessively large dimensions (${>}2C$) might hinder the capacity to parse the features. To address the issue, we further explore the number of layers of 3D convolution in Local Rectification to evaluate the impact.
As in Table~\ref{subtab:tableA}, the employment of 3D convolutions proves advantageous for feature rectification when the number of convolutional layers is ${<}4$, and the best mIoU peaks at $73.39\%$. However, when utilizing four layers of 3D convolutions, the interconnections between different features are significantly disrupted, leading to a diminished mIoU score of $71.49\%$. 

%
{Subsequently, after exploring the combination of the Local Rectification Module and the Global Rectification Module through the parallel addition of features from both, it was observed that there was a degradation in the network’s discriminative capability. This unexpected outcome also proves the effectiveness of our network.}
