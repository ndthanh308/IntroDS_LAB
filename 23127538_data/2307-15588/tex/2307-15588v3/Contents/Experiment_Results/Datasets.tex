\begin{table}[t]
\centering
\setlength{\tabcolsep}{10pt}
\renewcommand{\arraystretch}{1.5}
\begin{adjustbox}{width=0.48\textwidth}
\begin{tabular}{c|ccc|c}
\toprule[1mm]
\textbf{Dataset}           & \textbf{Train} & \textbf{Val} & \textbf{Test} & \textbf{Disparity Range} \\ \midrule[1.5pt] \hline

UrbanLF-Real      & 580   & 80  & 164  & [-0.47,1.55] \\ %
UrbanLF-Syn       & 172   & 28  & 50   & [-0.47,1.55] \\ %
UrbanLF-RealE     & 780   & 80  & 164  & [-0.47,1.55] \\ %
UrbanLF-Syn-Big   & 280   & 40  & 80   & [-7.39,7.07] \\ \hline %
\end{tabular}
\end{adjustbox}
\caption{
%
\textbf{Statistic information of {LF} semantic segmentation datasets.} UrbanLF-RealE denotes UrbanLF-Real with an extension of synthetic samples for training.}
\label{tab:dataset}
\end{table}

Our experiments are based on the UrbanLF datasets~\cite{sheng2022urbanlf}.
\revised{These datasets comprises} $14$ categories for urban semantic scene understanding.
Each sample in the datasets consists of $81$ SAIs with an angular resolution of $9{\times}9$.
\revised{In our experiments, we have followed the protocols proposed in UrbanLF~\cite{sheng2022urbanlf} and LFIENet++\cite{cong2024end} conducting sets of experiments: Urban-Syn, Urban-Real, UrbanLF-RealE, and UrbanLF-Syn-Big. The UrbanLF-RealE is an extension of all real and synthetic data. UrbanLF-Syn-Big has a large disparity range compared with other datasets.}
Table~\ref{tab:dataset} presents the number of images used for training, validation (quantitative and qualitative analysis), \revised{and testing} (qualitative analysis, visual comparison without ground truth), along with the disparity range.




