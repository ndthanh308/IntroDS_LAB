\begin{table}[t]
\centering
\renewcommand{\arraystretch}{1.3}
\begin{adjustbox}{width=0.48\textwidth}
\begin{tabular}{l|cc}
\toprule[1mm]
\multicolumn{1}{c|}{\textbf{Model}} & \textbf{\#Params(M)} & \textbf{mIoU(\%)} \\ \midrule[1.5pt] \hline
\textbf{OAFuser (ours)} & 79.2 & 77.18 \\ \hdashline[1pt/1pt]
\textbf{-~Without CARM} & 65.0~(-14.2) &75.01~(-2.17) \\ \hdashline[1pt/1pt]
%
\textbf{-~SAIs only at Stage One} & 65.0~(-14.2) & 73.50~(-3.68) \\
\hdashline[1pt/1pt]
\textbf{-~SAIs only at Stage Four} & 65.0~(-14.2)& 73.46~(-3.72) \\
\hdashline[1pt/1pt]
\textbf{-~Baseline Method~\cite{zhang2022cmx}} & 65.0~(-14.2)& 73.25~(-3.93) \\ \hdashline[1pt/1pt]
\textbf{-~Without FFM} & 58.4~(-20.8) &70.21~(-6.93) \\ \hline
\end{tabular}
\end{adjustbox}
\caption{{Ablation study of the OAFuser framework: ``SAIs'' indicates the sub-aperture features. ``SAIs only at Stage One'' denotes that the sub-aperture features are fused in the first stage only. ``SAIs only at Stage Four'' means the SAI features are calculated but not fused and fed into the Transformer block for stages one, two, and three. The fusion (the second step \revised{in the} SAFM) occurs only in the fourth stage.}}
\label{tab:ablationstudy for model}
\end{table}
% Figure environment removed

As shown in Table~\ref{tab:ablationstudy for model}, we gradually ablate the proposed OAFuser structure.
When the CARM is replaced with a feature rectification module~\cite{zhang2022cmx}, accuracy dramatically decreases by $2.17\%$.
Although the number of parameters is also reduced, the CARM module is essential to overcome challenges such as image mismatching and out-of-focus issues.~{\revised{Furthermore, we progressively ablate the second component of SAFM,} \ie, {using SAIs in the first or last stage, and without SAIs. \revised{The performance} of these three variants decreases significantly ($2{\sim}3\%$ drops \revised{compared with OAFuser}). }
%
Especially in these variants, the number of parameters remains unchanged, \revised{because angle information} is obtained through pixel-level feature aggregation with shared weights and \revised{addition operations.}
This also validates \revised{the claim} that the introduction of the SAFM efficiently achieves the selection and fusion of rich information from the {LF} camera, enabling the proposed network to handle arbitrary SAIs. Note that the baseline method, CMX~\cite{zhang2022cmx}, adopts a dual-pipeline SegFormer structure. Therefore, we maintain a dual-branch structure and eliminate the addition of angular features (the second step \revised{in the} SAFM) at certain stages.}
{In addition, the FFM~\cite{zhang2022cmx} is also essential because combining complementary features is crucial in the LF semantic segmentation task.}~
%
In addition, we conduct visualization comparisons in this ablation study. Fig.~\ref{fig:ARlation} illustrates the difference maps of the cropped region. The visual result indicates the effectiveness of using the CARM and SAFM for LF semantic segmentation. % for the cropped area.

