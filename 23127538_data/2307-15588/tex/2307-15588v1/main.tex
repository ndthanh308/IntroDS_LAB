\documentclass[journal]{IEEEtran}
%
    
\usepackage{blindtext}

%-----------------------

\usepackage{amsmath,amssymb,amsfonts,bm}
\usepackage{lipsum}
\usepackage{times}
\usepackage{epsfig}

\usepackage{stfloats}
\usepackage{multicol}
\usepackage[inline]{enumitem}
\usepackage{algorithmic}
\usepackage{graphicx,subcaption}
\usepackage{textcomp}

\usepackage{booktabs}
\usepackage{multirow}

\usepackage[utf8]{inputenc}
\usepackage[T1]{fontenc}
\usepackage{float}
\usepackage{adjustbox}
\usepackage{array}
\usepackage{dsfont}
%\usepackage{subfig}

\usepackage{pifont}% for marks
\newcommand{\cmark}{\ding{51}}%
\newcommand{\xmark}{\ding{55}}%
\usepackage{paralist}
\usepackage[ruled,linesnumbered]{algorithm2e}

%%%%%%%%%%%%%%%%%%%%%%%%%%%%%  extra package  TF
\usepackage{arydshln}
\newcolumntype{C}[1]{>{\centering\arraybackslash}p{#1}}
\newcommand{\TF}[1]{\textcolor{blue}{#1}}
\usepackage{pifont}
%%%%%%%%%%%%%%%%%%%%%%%%%%%%%  end extra package  TF


% \usepackage{flushend}
% \usepackage{orcidlink}

%
\usepackage[accsupp]{axessibility}  % Improves PDF readability for those with disabilities.

\usepackage{tabulary}
\usepackage[table]{xcolor}
\newcommand\ver[1]{\rotatebox[origin=c]{90}{#1}}
\newcommand{\fl}[1]{\multicolumn{1}{c}{#1}}
\definecolor{gray}{rgb}{0.3,0.3,0.3}
\definecolor{blue}{rgb}{0,0.5,1}
\definecolor{mask_red}{rgb}{1,0,0.8}
\definecolor{green}{rgb}{0.2,1,0.2}
\definecolor{rblue}{rgb}{0,0,1}
\newcommand{\gray}[1]{\textcolor{gray}{#1}}
\newcommand{\green}[1]{\textcolor[RGB]{96,177,87}{#1}}
\newcommand{\fn}[1]{\footnotesize{#1}}
\newcommand{\gbf}[1]{\green{\bf{\fn{(#1)}}}}
\newcommand{\rbf}[1]{\gray{\bf{\fn{(#1)}}}}

\usepackage{hyperref}
\hypersetup{colorlinks=true,linkcolor=red,citecolor=blue}

\newcommand{\YKL}[1]{\textcolor{orange}{#1}}
\newcommand{\PKY}[1]{\textcolor{purple}{#1}}
\newcommand{\revised}[1]{\textcolor{orange}{#1}}


% Add a period to the end of an abbreviation unless there's one
% already, then \xspace.
\makeatletter
\DeclareRobustCommand\onedot{\futurelet\@let@token\@onedot}
\def\@onedot{\ifx\@let@token.\else.\null\fi\xspace}

\def\eg{\emph{e.g}\onedot} \def\Eg{\emph{E.g}\onedot}
\def\ie{\emph{i.e}\onedot} \def\Ie{\emph{I.e}\onedot}
\def\cf{\emph{c.f}\onedot} \def\Cf{\emph{C.f}\onedot}
\def\etc{\emph{etc}\onedot} \def\vs{\emph{vs}\onedot}
\def\wrt{w.r.t\onedot} \def\dof{d.o.f\onedot}
\def\etal{\emph{et al}\onedot}

\begin{document}

\title{OAFuser: Towards Omni-Aperture Fusion for Light Field Semantic Segmentation of Road Scenes}
\author{Fei Teng\IEEEauthorrefmark{1}, Jiaming Zhang\IEEEauthorrefmark{1}, Kunyu Peng, Kailun Yang\IEEEauthorrefmark{2}, Yaonan Wang, and Rainer Stiefelhagen%
\IEEEcompsocitemizethanks{
\IEEEcompsocthanksitem 
This work was supported in part by the Ministry of Science, Research and the Arts of Baden-Württemberg (MWK) through the Cooperative Graduate School Accessibility through AI-based Assistive Technology (KATE) under Grant BW6-03, in part by the University of Excellence through the ``KIT Future Fields'' project, in part by the Helmholtz Association Initiative and Networking Fund on the HAICORE@KIT partition, and in part by Hangzhou SurImage Technology Company Ltd.
\IEEEcompsocthanksitem F. Teng, J. Zhang, K. Peng, and R. Stiefelhagen are with the Institute for Anthropomatics and Robotics, Karlsruhe Institute of Technology, 76131 Karlsruhe, Germany.
\IEEEcompsocthanksitem K. Yang and Y. Wang are with the School of Robotics and the National Engineering Research Center of Robot Visual Perception and Control Technology, Hunan University, Changsha 410082, China.
%(E-Mail: kailun.yang@hnu.edu.cn.)
\IEEEcompsocthanksitem \IEEEauthorrefmark{1}Equal contribution.
\IEEEcompsocthanksitem \IEEEauthorrefmark{2}Corresponding author (E-Mail: kailun.yang@hnu.edu.cn.).
}%
}

\maketitle
\bstctlcite{IEEEexample:BSTcontrol}
%
%%%%%%%%% ABSTRACT
\begin{abstract}
\label{sec:abstract}
With the advent of decentralised digital currencies powered by blockchain technology, a new era of peer-to-peer transactions has commenced. The rapid growth of the cryptocurrency economy has led to increased use of transaction-enabling wallets, making them a focal point for security risks. As the frequency of wallet-related incidents rises, there is a critical need for a systematic approach to measure and evaluate these attacks, drawing lessons from past incidents to enhance wallet security.

In response, we introduce a multi-dimensional design taxonomy for existing and novel wallets with various design decisions. We classify existing industry wallets based on this taxonomy, identify previously occurring vulnerabilities and discuss the security implications of design decisions. We also systematise threats to the wallet mechanism and analyse the adversary's goals, capabilities and required knowledge. We present a multi-layered attack framework and investigate 84 incidents between 2012 and 2024, accounting for \$5.4B. Following this, we classify defence implementations for these attacks on the precautionary and remedial axes. We map the mechanism and design decisions to vulnerabilities, attacks, and possible defence methods to discuss various insights. 


\end{abstract}

%
\begin{IEEEkeywords} 
Semantic segmentation, light field, scene parsing, vision transformers, scene understanding.
\end{IEEEkeywords}

\IEEEpeerreviewmaketitle
%%%%%%%%% BODY TEXT
\section{Introduction}
The problem of the presence or absence of phase transition is central in statistical mechanics. To prove the existence of phase transition, the standard idea is to define a notion of contour and use \textit{Peierls' argument} \cite{Peierls.1936}. In the usual Ising model \cite{Ising_25}, particles of the system interact only with their nearest-neighbors. On ferromagnetic long-range Ising models \cite{Anderson_Yuval_69}, there is interaction between each pair of spins in the lattice. The Hamiltonian of the model is given formally by
\begin{equation*}
    H(\sigma) = - \sum_{x,y\in \Z^d}J_{xy}\sigma_x\sigma_y,
\end{equation*}
where $J_{xy}=J|x-y|^{-\alpha}$, $J>0$, $\alpha > d$. It is well-known that the Peierls' argument in dimension 2 implies phase transition for Ising models with nearest-neighbors or long-range interactions when $d\geq 2$, using correlation inequalities. For the unidimensional lattice, it was known that short-range models do not present phase transition. In the long-range case, a different behavior was expected depending on the exponent $\alpha$ (see \cite{Kac_Thompson_69}), but the problem was challenging since contours were first created as multidimensional objects.

In dimension $d=1$, phase transition was proved first in 1969 by Dyson \cite{Dyson.69}, for $\alpha \in (1,2)$, by proving phase transition in an auxiliary model and then using correlation inequalities. In 1982, Fr{\"o}hlich and Spencer \cite{Frohlich.Spencer.82} introduced a notion of one-dimensional contours and then applied the Peierls' argument to show phase transition for the critical value $\alpha = 2$. These contours were inspired by the multiscale techniques previously introduced to study the Berezinskii-Kosterlitz-Thouless transition in two-dimensional continuous spin systems \cite{FS81}. Later, Cassandro, Ferrari, Merola and Presutti  \cite{Cassandro.05} extended the contour argument previously available for $\alpha=2$ to exponents $\alpha\in (3-\frac{\ln 3}{\ln 2}, 2)$, with the additional restriction that the nearest-neighbor interaction is strong, i.e.,  ${J(1)\gg 1}$; this restriction was removed for a subclass of interactions in \cite{Bissacot.Endo.18}. Further results were obtained using contour arguments, such as the decay of correlations, cluster expansions, phase transition with random interactions, etc; some references with these results are \cite{ Cassandro.Merola.Picco.17, Cassandro.Merola.Picco.Rozikov.14, Imbrie.82, Imbrie.Newman.88, Johansson.91}. 

In the multidimensional setting ($d\geq 2$), Ginibre, Grossmann, and Ruelle, in \cite{Ginibre.Grossmann.Ruelle.66}, proved the phase transition for $\alpha > d+1$, using an enhanced version of Peierls' argument and the usual contours. Park proposed a different notion of contour for long-range systems in \cite{Park.88.I, Park.88.II}, extending the Pirogov-Sinai theory available for short-range interactions assuming $\alpha > 3d+1$, although he can also consider Potts models with his methods. Some results in the literature suggest that truly long-range effects appear only when $d < \alpha \leq d+1$, see for instance, \cite{Biskup_Chayes_Kivelson_07}. Recently, Affonso, Bissacot, Endo and Handa \cite{Affonso.2021}, inspired by the ideas from Fr{\"o}hlich and Spencer in \cite{FS81, Frohlich.Spencer.82}, introduced a version of multiscale multidimensional contour and proved phase transition by a contour argument in the whole region $\alpha > d$. They can consider long-range Ising models with deterministic decaying fields, first introduced in the context of nearest-neighbor interactions in \cite{Bissacot_Cioletti_10}. For these models, the lack of analyticity of the free energy does not imply phase transition since these models have the same free energy as the models with zero field. It is expected that fields decaying slowly imply uniqueness. In this setting, a contour argument is useful for proofs of phase transitions as well for uniqueness, some papers with models with deterministic decaying fields are \cite{Aoun_Ott_Velenik_23, Bissacot_Cass_Cio_Pres_15, Bissacot.Endo.18, Cioletti_Vila_2016}.

The Random Field Ising model (RFIM) \cite{Imry.Ma.75} is the nearest-neighbor Ising model with an additional external field acting on each site $(h_x)_{x\in\Z^d}$ that is a family of i.i.d. Gaussian random variable with mean 0 and variance 1. Formally, the Hamiltonian of the model is given by
\begin{equation*}
    H(\sigma) = - \sum_{\substack{x,y\in \Z^d \\|x-y|=1}}J\sigma_x\sigma_y  - \varepsilon\sum_{x\in\Z^d}h_x\sigma_x,
\end{equation*}
where $J>0$, $\varepsilon>0$, $\alpha > d$ and $d \geq 1$. A detailed account of the history of the phase transition problem for this model, as well as detailed proofs, was given in \cite{Bovier.06}. Here we present a brief overview.

During the 1980s, the question of the specific dimension where phase transition for the RFIM should happen attracted much attention and was a topic of heated debate. Two convincing arguments were dividing the physics community. One of them, due to Imry and Ma \cite{Imry.Ma.75}, was a non-rigorous application of the Peierls' argument together with the use of the isoperimetric inequality. The key idea of Peierls' argument is to define a notion of contour and calculate the energy cost of "erasing" each contour, i.e., the energy cost of flipping all spins inside the contour. When there is no external field, that energy necessary to flip the spins in a region $A\subset \Z^d$ is of the order of the boundary $|\partial A|$. When we add an external field, we get an extra cost depending on this field. Imry and Ma argued that this cost should be approximately $\sqrt{|A|}$, which is smaller than $|\partial A|$ for all regions only when $d\geq 3$, so this should be the region where phase transition occurs. The other argument, due to Parisi and Sourlas \cite{Parisi.Sourlas.79}, based on dimensional reduction, predicted that the $d$-dimensional RFIM would behave like the $d-2$-dimensional nearest-neighbor Ising model, therefore presenting phase transition only when $d\geq 4$. 

The question was settled by two celebrated papers showing that Imry and Ma's prediction was correct. First, in 1988, Bricmont and Kupiainen \cite{Bricmont.Kupiainen.88} showed that there is phase transition almost surely in $d\geq3$, for low temperatures and variance $\varepsilon$ small enough. Their proof uses a rigorous renormalization group analysis for the short-range case and it is considered involved. Still, they claimed that the result works for any model with a suitable contour representation and centered sub-gaussian external field. Later on, Aizenman and Wehr \cite{Aizenman.Wehr.90} proved uniqueness for $d\leq 2$. For detailed proofs of these results, we refer the reader to \cite{Bovier.06} (see also \cite{Berretti.85, Camia.18, Frohlich.Imbre.84,  Klein.Masooman.97} for more uniqueness results). 

Recently, Ding and Zhuang, see \cite{Ding2021}, provided a simpler proof of the phase transition, not using RGM. And in  \cite{Ding.Liu.Xia.22}, Ding, Liu and Xia proved that if $\beta_c(d)$ is the critical inverse of the temperature of the Ising model with no field, for all $\beta>\beta_c(d)$ there exists a critical value $\varepsilon_0(d, \beta)$ such that the RFIM with $\varepsilon \leq \varepsilon_0$ presents phase transition. 

In the present paper, we are considering a long-range Ising model with a random field, whose Hamiltonian is given formally by
\begin{equation*}
    H(\sigma) = - \sum_{x,y\in \Z^d}J_{xy}\sigma_x\sigma_y - \varepsilon\sum_{x\in\Z^d}h_x\sigma_x,
\end{equation*}
where $J_{xy}=J|x-y|^{-\alpha}$, $J, \varepsilon>0$, $\alpha > d$ and $h_x\in\mathbb{R}$, $d\geq 3$.
Until now, the only known result in the long-range setting is for the one-dimensional long-range Ising model with a random field, by Cassandro, Orlandi, and Picco \cite{Cassandro.Picco.09}. They used the contours of \cite{Cassandro.05} to show the phase transition for the model when $\alpha\in (3-\frac{\ln 3}{\ln 2}, \frac{3}{2})$, under the assumption $J(1) \gg 1$. We stress that, as remarked by Aizenman, Greenblatt, and Lebowitz \cite{Aizenman_Greenblatt_Lebowitz_2012}, although their argument does not work for the whole region for the exponent $\alpha$, the phase transition holds for values close to the critical value $\alpha=3/2$, since by the Aizenman-Wehr theorem we know that there is uniqueness for $\alpha>3/2$.

The argument from Ding and Zhuang in \cite{Ding2021}, for $d\geq3$, involves controlling the probability of a bad event, which is closely related to controlling the quantity $$\sup_{\substack{0\in A\subset\Z^d \\ A \text{ connected }}}\frac{\sum_{x\in A}h_x}{|\partial A|},$$ known as the greedy animal lattice normalized by the boundary. The greedy animal lattice normalized by the size, instead of the boundary, was extensively studied for general distributions of $(h_x)_{x\in\Z^d}$, see \cite{Cox_Gandolfi_Griffin_Kesten_93, Gandolfi_Kesten_94, Hammond_06, Martin_02}. When we normalize by the boundary, an argument by Fisher, Fr\"{o}hlich and Spencer \cite{FFS84} shows that the expected value of the greedy animal lattice is constant. In dimension $d=2$, the expected value is not finite, see \cite{Ding.Wirth.20}. The supremum is taken over connected regions containing the origin since the interiors of the usual Peierls contours are of this form.


For the long-range model, the interior of contours is not necessarily connected. In fact, long-range contours may have considerably large diameters with respect to their size, so their interiors can be very sparse. To avoid this, we define contours, strongly inspired by the $(M,a,r)$-partition in \cite{Affonso.2021}, using a multiscaled procedure that assures that the contours have no cluster with small density.  With them, we generalize the arguments by Fisher-Fr\"{o}hlich-Spencer \cite{FFS84}, and prove that the expected value of the greedy animal lattice is constant, even considering regions not necessarily connected in the supremum. Then, we prove the phase transition for $d\geq 3$. The main result of this paper is the following.
\begin{theorem*}Given $d\geq 3$, $\alpha>d$, there exists $\beta_c\coloneqq\beta(d, \alpha)$ and $\varepsilon_c\coloneqq\varepsilon(d, \alpha)$ such that, for $\beta >\beta_c$ and $\varepsilon\leq \varepsilon_c$, the extremal Gibbs measures $\mu_{\beta, \varepsilon}^+$ and $\mu_{\beta, \varepsilon}^-$ are distinct, that is, $\mu_{\beta, \varepsilon}^+ \neq \mu_{\beta, \varepsilon}^-$ $\mathbb{P}$-almost surely. Therefore the long-range random field Ising model presents phase transition.
\end{theorem*}

This paper is divided as follows. In Section 2, we define the model and the contours, and suitable generalizations to the constructions in \cite{Affonso.2021} are introduced.  In Section 3, we define two bad events of the external field and prove that they occur with a small probability.  In Section 4, we present the proof of the phase transition.


\section{Related Work}

In this section, the overview of semantic segmentation is introduced in Sec.~\ref{sec:2_a_ss}. Since the asymmetric feature
representation from the light field camera, various multi-
modal semantic segmentation works are presented in Sec.~\ref{sec:2_b_mm}.
Furthermore, to boost the angular information expression of
light field cameras, several applications in other areas are also
introduced in Sec.~\ref{sec:2_c_lf}

\subsection{Semantic Segmentation} \label{sec:2_a_ss}
Semantic scene segmentation, as a fundamental task of computer vision, plays a crucial role in scene understanding tasks, such as autonomous driving and intelligent transportation systems~\cite{zhou2022mtanet,zhang2022trans4trans}, by assigning a category to each pixel.
Since FCN~\cite{long2015fully} pioneers the use of convolutional neural networks to replace fully connected networks to propose an end-to-end framework, many segmentation works have emerged based on this approach, and the efficiency and accuracy of segmentation have been largely improved. 
For instance, \cite{chen2018encoder,badrinarayanan2017segnet} adopt an encoder-decoder structure to capture contextual information and local details.
Then, \cite{chen2018encoder,yang2018denseaspp} introduce dilated convolution to increase the receptive field.
To enhance the global context representation, \cite{ding2018context,he2019adaptive} adopt a pyramidal hierarchy in the encoding path.
Furthermore, enhancing prior contextual information~\cite{lin2017refinenet,fu2019dual,wang2021exploring} contributes to improving segmentation results.
Since the introduction of the self-attention mechanism in vision tasks~\cite{dosovitskiy2020image}, many following works~\cite{xie2021segformer,zhang2022vsa,zhang2022trans4trans} propose dense prediction, attention-based models.
Meanwhile, some other works introduce lightweight backbones to speed up the inference~\cite{sandler2018mobilenetv2,tan2019efficientnet,zhang2018shufflenet}.

Although those works demonstrate excellent results in handling dense prediction tasks, they still suffer from the limitation of image quality and lead to performance degeneration in handling complete areas, such as shallow or out-of-fuse areas in real-world self-driving scene understanding scenarios. 

\subsection{Multi-Modal Semantic Segmentation}\label{sec:2_b_mm}
The multiple sub-aperture images captured by the light field camera can be considered as various RGB modalities with inherent relationships. Therefore, the research on multi-modal semantic segmentation is essential for exploring the potential of light field cameras.
ACNet~\cite{hu2019acnet} and EDCNet~\cite{zhang2021exploring} leverage attention connections for facilitating cross-modal interactions in RGB-Depth and RGB-Event semantic segmentation, respectively.
MMFNet~\cite{chen2020mmfnet} enables the fusion of multiple medical images through the aggregation of different features in spatial and channel domains.
NestedFormer~\cite{xing2022nestedformer} proposed a feature aggregation module to fulfill multimodal medical image segmentation.
Furthermore, ESANet~\cite{seichter2021efficient} and SA-Gate~\cite{chen2020bi} utilize depth maps and RGB images to achieve high-accuracy semantic segmentation by employing uniquely designed fusion modules.
PGSNet~\cite{mei2022glass}, which introduces a dynamic integration module, achieves glass segmentation.
Additionally, \cite{sun2019rtfnet,zhou2021gmnet,ha2017mfnet} adopt RGB-thermal image fusion.
Especially, the works of CMX~\cite{zhang2022cmx,zhang2023delivering} present an arbitrary-modal fusion network, which can handle RGB and any other modality, such as depth, thermal, polarization, event, or LiDAR data.
HRFuser~\cite{broedermann2022hrfuser} realizes the fusion of an arbitrary number of additional modalities as supplementary information into RGB images by introducing multi-window cross-attention. 

Different from these methods, which are focused on handling symmetrical modalities, our proposed OAFuser focuses on the utilization of the angular information diversity in light-field images and considers the mismatching from images captured by all sub-apertures of light-field cameras.

\subsection{Light Field Scene Understanding}\label{sec:2_c_lf}
While light-field cameras are still under-explored in semantic segmentation, they have found wide applications in various areas, such as saliency detection~\cite{wang2017two,zhang2017saliency}, depth estimation~\cite{honauer2017dataset,peng2020zero}, and super-resolution~\cite{wang2018lfnet,jin2020light}, due to their rich visual information.
Several works have leveraged the potential of light-field cameras.
FES~\cite{chen2023fusion} achieves sub-aperture feature fusion via spatial and channel attention.
NoiseLF~\cite{feng2022learning} utilizes the all-focus central-view image and its corresponding focal stack, with a unique-designed forgetting matrix and confidence re-weighting strategy, to achieve supervised saliency detection under noisy labels.

Furthermore, several works~\cite{wang2020spatial,liang2023learning,zhang2019residual} employ sub-aperture images, macro-pixel images, epipolar images, or a combination of some of those images to generate high-resolution light field images.
Moreover, AIFLFNet~\cite{zhou2023aif} utilized light field images to estimate depth information. Additionally, SAA-Net~\cite{wu2021spatial} introduces spatial-angular attention modules for light field image reconstruction.
Especially, a design from~\cite{wang2022disentangling} proposes a unified block for handling macro-pixel images, which can be used for both super-resolution and disparity estimation.
For light-field semantic segmentation, the work of~\cite{sheng2022urbanlf} utilizes stacks of sub-aperture images from certain directions achieving the segmentation of central-view images.
LFIE-Net~\cite{cong2023combining} introduces an explicit branch to generate disparity maps within the network and cooperates with the sub-aperture images to achieve dense semantic segmentation.

Unlike existing works which are limited by memory costs when processing light-field images, our proposed OAFuser has the capability to handle an arbitrary number of sub-aperture images without parameter demands for light-field road-scene semantic understanding.
%


%%%%%%%%%%%%%%%%%%%%%%%%%%%%%%%%%%%%%%%%%%%%%%%%%%%%%%%%%%%%%%%%%%%%%%%%%%%%%%%%%%%%%%%%%%%%%%%%%%%%%%%%%%%%%%%%%%%%%%%%% Methodology %%%%%%%%%%%%%%%%%%%%%%%%%%%%%%%%%%%

\section{Methodology}


This section provides a detailed introduction to \revised{the proposed network}, \ie, \emph{Omni-Aperture Fuser (OAFuser)}, which is tailored to {LF} semantic segmentation. The overall OAFuser architecture is presented in Sec.~\ref{sec:3_A_oafuser}. The \emph{Sub-Apeture Fusion Module (SAFM)} for \revised{LF} feature aggregation is introduced in Sec.~\ref{sec:3_B_sf}. The \emph{Center Aperture Rectification Module (CARM)} \revised{is presented} in Sec.~\ref{sec:3_c_ro}.  

\subsection{Proposed OAFuser Architecture}\label{sec:3_A_oafuser}

As shown in Fig.~\ref{fig:Overall}, the proposed OAFuser \revised{comprises} a four-stage encoder and a decoder. The encoder consists of the proposed SAFM and CARM to handle the 
%
feature fusion, feature embedding, and feature rectification, respectively.
For simplicity, the following description is based on stage one, which is \revised{the same as that for the other three stages.}
Especially, the arbitrary number of {LF} images is \revised{denoted as sub-aperture images} $F_{s_i} {\in} \mathbb{R}^{H \times W \times 3}$ and central view image $F_c{\in}\mathbb{R}^{H \times W \times 3}$, where $s_i$ \revised{denotes} the $i$-th sub-aperture image in range $[1, N]$. All \revised{images} are fed into \revised{the} SAFM to embed angular feature $F_{agl}{\in}\mathbb{R}^{\frac{H}{8} \times \frac{W}{8} \times 64}$ that contains rich angular information and spatial feature $F_{spl}{\in}\mathbb{R}^{\frac{H}{8} \times \frac{W}{8} \times 64}$ which focus on spatial information for the central view. By applying two different transformer blocks following~\cite{zhang2022cmx}, both \revised{features} are transformed into $F_{agl}^*{\in}\mathbb{R}^{\frac{H}{8} \times \frac{W}{8} \times 64}$ and $F_{spl}^*{\in}\mathbb{R}^{\frac{H}{8} \times \frac{W}{8} \times 64}$.~ {One notable aspect of our design is that the sub-aperture features are fed into subsequent stages for further processing, which \revised{differs} from other early-fusion or late-fusion methods.} {\revised{Subsequently, the angular and spatial features are concatenated and fed onto the CARM}, which includes the Horizontal Operation and the Vertical Operation for feature rectification along the horizon and vertical \revised{directions} to eliminate misalignments.} 
\revised{In particular}, the concatenation of $F_{spl}^*$ and $F_{agl}^*$ along the horizon direction is applied to obtain $F_{c1}{\in}\mathbb{R}^{\frac{H}{8} \times \frac{W}{4} \times 64}$. After the Global Rectification and the Local Rectification, the horizontally rectified feature $F^{H}{\in}\mathbb{R}^{2 \times \frac{H}{8} \times \frac{W}{8} \times 64}$ is obtained. Then, given the feature $F^{H}$, the feature $F_{c2}{\in}\mathbb{R}^{\frac{H}{4} \times \frac{W}{8} \times 64}$ is obtained by {splitting} and concatenation \revised{operations along the vertical direction}. Especially, $F^{H}{\in}\mathbb{R}^{2 \times \frac{H}{8} \times \frac{W}{8} \times 64}$ is disentangled into $F^{H}_{agl}{\in}\mathbb{R}^{\frac{H}{8} \times \frac{W}{8} \times 64}$ and $F^{H}_{spl}{\in}\mathbb{R}^{\frac{H}{8} \times \frac{W}{8} \times 64}$. $F^{H}_{agl}$ and $F^{H}_{spl}$ are further concatenated along vertical direction. After applying \revised{the Global Rectification and the Local Rectification in the Vertical Operation}, the rectified features $F^{G}{\in}\mathbb{R}^{2 \times \frac{H}{8} \times \frac{W}{8} \times 64}$. $F^{G}_{agl}{\in}\mathbb{R}^{\frac{H}{8} \times \frac{W}{8} \times 64}$ and $F^{G}_{spl}{\in}\mathbb{R}^{\frac{H}{8} \times \frac{W}{8} \times 64}$ is obtained by {split operation} of $F^{G}$. (Notably, $F^{G}_{spl}{\in}\mathbb{R}^{\frac{H}{8} \times \frac{W}{8} \times 64}$ functions \revised{serve as both a spatial feature for subsequent feature fusion and the spatial feature input for stage two.}) Furthermore, \revised{the two rectified features}, \ie,  $F^G_{agl}$ and $F^G_{spl}$ are fused by using the FFM module~\cite{zhang2022cmx}, which integrates two different features, to produce the final feature $f_1$ for this stage. The weights between \revised{the Horizontal Operation and the Vertical Operation} are shared.

Note that one of our crucial designs is \revised{that} {\textbf{all sub-aperture information is fed into each stage}, which allows our network to effectively consider all the SAIs {throughout the entire process}}. The pyramidal features $f_1, f_2, f_3, f_4$ are obtained via the four-stage encoder in a dimension of $\{64,128,320,512\}$. Subsequently, the multi-stage features are further fed into an MLP Decoder~\cite{xie2021segformer} for final prediction.



\subsection{Sub-Aperture Fusion Module}\label{sec:3_B_sf}
% Figure environment removed
To retrieve the rich angular and spatial information from {LF} images, we introduce the SAFM module, which embeds \revised{this} information into angular and spatial features. As shown in Fig.~\ref{fig:SF Module}, the SAFM \revised{comprises} of two steps. \revised{The first step extracts} features from all the {LF} images. All images are firstly divided into patches $P_{(m, n)}^{s_i}$, $P_{(m, n)}^c$, correspondence to SAIs $F_{s_i}$ and $F_c$, and further fed into the convolutional layer to extract features, resulting in $\hat{P}_{(m, n)}^{s_i}$ and $\hat{P}_{(m, n)}^c$, where $m, n$ denotes the \revised{relative patch position} in a single feature map. The patch size follows~\cite {zhang2022cmx}. Especially, this work aims to segment the central view image. Thus, the weights for the center view are \revised{calculated individually}, which ensures that the spatial information of the central view remains independent. The weights \revised{SAIs} are shared among different SAIs. The calculations are presented in Eq.~(\ref{equ:feature_extraction1}) and Eq.~(\ref{equ:feature_extraction2}):
\begin{align}
\hat{P}_{(m, n)}^{s_i} &= \text{Conv}^{sub}(C_{\text{in}}, C_{\text{out}})(P_{(m, n)}^{s_i}), \label{equ:feature_extraction1} \\ 
\hat{P}_{(m, n)}^c &= \text{Conv}^{cen}(C_{\text{in}}, C_{\text{out}})(P_{(m, n)}^c), \label{equ:feature_extraction2}
\end{align}
where $\text{Conv}(C_{in}, C_{out}) $ \revised{denotes} the convolutional layer with input dimension $C_{in}$ and output dimension $C_{out}$. For the first {step}, $C_{in}$ is $3$ and $C_{out}$ is $64$. \revised{Subsequently}, the central image features $F_{spl}$ are obtained by combining different patches $\hat{P}_{(m,n)}^c$ and fed into the Spatial Transformer Block~(STB). \revised{The patches of the central view are also fed into the next step to cooperate with patches from the SAIs to obtain the angular features.} 

Afterward, the second step of \revised{the} SAFM is illustrated in Fig.~\ref{fig:SF Module}, which \revised{is represented by} the yellow gear symbol.

{For each patch $\hat{P}_{(m, n)}^{s_i}$, the pixel score $e{_{{(m,n)}}^{s_i}}$ is calculated by obtaining the \revised{Euclidean distance} between each patch from the sub-aperture feature and the corresponding patch from the center view feature, as \revised{expressed} in Eq.~(\ref{e}):
\begin{align}
e{_{{(m,n)}}^{s_i}} &= {\text{Abs}(\hat{P}^c_{{(m, n)}}} -\hat{P}_{{(m, n)}}^{s_i}), \label{e} 
\end{align}
where $\text{Abs}{(\cdot)}$ denotes \revised{the} absolute value between two patches. Given the $e{_{{(m,n)}}^{s_i}}$, the mask score $t_{(m,n)}^{s_i}$ for $\hat{P}_{(m, n)}^{s_i}$ is obtained by mapping into the range of [1,0] and squaring them to enhance the discrimination, as shown in Eq.~(\ref{mp}).
\begin{align}
t_{{(m,n)}}^{s_i} = {(\Theta\{{e{_{{(m,n)}}^{s_i}}}}\})^2,\label{mp}
\end{align}
where $\Theta\{\cdot\}$ denotes \revised{the mapping} operation. After obtaining the mask scores, \revised{a certain patch} $P_{{(m,n)}}$ of \revised{angular features} can be calculated. \revised{Specially}, the patch $P_{{(m,n)}}$ in \revised{the angular} feature $F_{agl}$ at position $(m,n)$ can be calculated by summarizing the corresponding central view patch $\hat{P}^c_{{(m, n)}}$ with masked sub-aperture patches $\hat{P}_{{(m, n)}}^{s_i}$, as \revised{shown} in Eq.~(\ref{p_final}).
\begin{align}
P_{(m,n)} &= \hat{P}^c_{{(m, n)}} + \sum_{i=1}^{N} {t{_{{(m,n)}}^{s_i}} \cdot \hat{P}_{{(m, n)}}^{s_i}}.  \label{p_final}
\end{align}
Finally, the patch of \revised{the} angular feature $P_{(m,n)}$ is obtained. The angular feature $F_{agl}$ is filled by $P_{(m,n)}$ and further fed into \revised{the} Angular Transformer Block~(AFB).}



\subsection{Central Angular Rectification Module}\label{sec:3_c_ro}
To eliminate information asymmetry, {the CARM is a crucial design to rectify both spatial and angular features while simultaneously rearranging the features captured from different viewpoints.} As shown in Fig.~\ref{fig:Rectification}, two features from different transformer blocks are fed into CARM, which includes Horizontal and Vertical Operations. Since the Horizontal and Vertical Operations are symmetrical, only the Horizontal Operation is introduced in detail.
In each operation, Global Rectification and Local Rectification are applied to rectify {feature cues} in different regions. At the end of CARM, the two features $F_{agl}^G$ and $F_{spl}^G$, which correspond to $F_{agl}^*$ and $F_{spl}^*$, are further fed into FFM (Feature Fusion Module). 
% Figure environment removed
Specifically, in the Horizontal Operation, $F_{agl}^*$ and $F_{spl}^*$ is first concatenated along horizontal direction into feature $F_{c1} \in \mathbb{R}^{\frac{H}{8}H \times \frac{W}{4} \times 64}$. {To intuitively represent the dimension change between different steps,} we use $F_{c1} \in \mathbb{R}^{H \times 2W \times C_1}$ as the concatenated feature. $F_{c1}$ is firstly fed into the \textbf{Global Rectification} stage. By applying an embedding process, $F_{c1}$ is projected into $F_{c1}^* \in \mathbb{R}^{H \times 2W \times C_2}$, which contain a set of tokens ${T}^*_{c1} \in \mathbb{R}^{2W \times C_2}$, with the projection matrix $M_{in} \in \mathbb{R}^{C_1 \times C_2}$, where $C_1$ denotes the input dimension of features, and $C_2$ denotes the embedding dimension for each token, 
%
the number of tokens is $2W$. To mitigate the covariate shift, $\dot{F}^*_{c1} \in \mathbb{R}^{H \times 2W \times C_2}$  with tokens $\dot{T}^*_{c1} \in \mathbb{R}^{2W \times C_2}$ is obtained by applying normalization of $F_{c1}^*$, following \cite{dosovitskiy2020image}. Then, the tokens, which represent the feature cues in a row, are utilized to generate query ($Q$), key ($K$), and value ($V$). To be more specific, $Q \in \mathbb{R}^{2W \times C_2}$ and $K \in \mathbb{R}^{2W \times C_2}$ are obtained by multiplication of tokens $\dot{T}^*_{c1}$ with matrices $M_q \in \mathbb{R}^{C_2 \times C_2}$ and $M_k \in \mathbb{R}^{C_2 \times C_2}$, respectively. The matrix $M_v \in \mathbb{R}^{C_2 \times C_2}$ multiplies the $T^*_{c1}$ to generate $V \in \mathbb{R}^{2W \times C_2}$. Furthermore, $Q$, $K$, and $V$ are divided along the channel dimension and fed into eight heads. The similarity of each head can be calculated by dot product and followed by a Softmax function to obtain the similarity scores in each head between those tokens, as in Eq.~(\ref{softmax first}).
\begin{align}
\text{Similarity}_{{head}_1} &=  \text{Softmax}\frac{Q_i\cdot K_i^T}{\sqrt{D_i} },~i. \in [1,8]. \label{softmax first}
\end{align}
Based on the similarity scores, the rectified tokens $T^a_{h_i}$ in each head can be obtained by multiplying with $V_i$, as in Eq.~(\ref{softmax score}).
\begin{align}
T^a_{h_i} &= \text{Similarity}_{{head}_i} \cdot V_i,~ i \in [1,8] .\label{softmax score}
\end{align}
The final tokens $T^a_h \in \mathbb{R}^{2W \times C_2}$ are obtained by concatenation, as in Eq.~(\ref{final score}).
\begin{align}
T^a_{h} &= \text{Concat}\{T^a_{h_1}, T^a_{h_2} ..., T^a_{h_8}\}.  \label{final score}
\end{align}
Given the final tokens $T_h^a$, an MLP layer is adapted, and a linear layer is to project tokens into demotion $C_1$. The final feature ${\overline{F^a_{h}}} \in \mathbb{R}^{H \times 2W \times C_1}$ after Global Rectification can be formulated as in the Eq.~(\ref{MLP}),
\begin{align}
    \overline{F^a_{h}} = \text{LN}(C_2,C_1)(\text{MLP}(F^a_{h}) + F^a_{h}), \label{MLP}
\end{align}
where $LN(C_2,C_1)(\cdot)$ denotes linear projection.

By decoupling the feature ${\overline{F^a_{h}}}$, the rectified feature stack $F_h^l \in \mathbb{R}^{2 \times H \times W \times C_1}$ is obtained. Furthermore, to further rectify the features in surrounding areas, the \textbf{Local Rectification} stage, which contains three convolutional layers, is implemented as in Eq.~(\ref{3d conv}), where $C$ denotes the input channel and output channel, $\{\cdot\}\otimes  2$ denotes repeat those layers two times. 
\begin{align}
    \Hat{F_h^l} &= \{{LReLU}({Conv3D}(C,C)(F^l_h))\}\otimes 2,\\
    {F}^H &= {Conv3D}(C,C)(\Hat{F_h^l}).\label{3d conv}
\end{align}
By disentangling of ${F}^H\in \mathbb{R}^{2 \times H \times W \times C_1}$, $F^H_{agl} \in \mathbb{R}^{ H \times W \times C_1} $ and $F^H_{spl} \in \mathbb{R}^{H \times W \times C_1}$ is obtained and further fed into Vertical Operation after concatenation.

%


%%%%%%%%%%%%%%%%%%%%%%%%%%%%%%%%%%%%%%%%%%%%%%%%%%%%%%%%%%%%%%%%%%%%%%%%%%%%%%%%%%%%%%%%%%%%%%%%%%%%%%%%%%%%%%%%%%%%%% Experiment Result %%%%%%%%%%%%%%%%%%%%%%%%%%%%%%%%
\section{Experiment Results}
\subsection{Datasets}\label{sec:4_1_da}
\section{Information about datasets}\label{appsec:data}
In this section, we give a few more details about the data used for the experiments.

\subsection{Natural image patches}
In this experiment, we consider 10 flower images from the ImageNet database~\cite{deng_imagenet_2009}. Those were downloaded from Kaggle (\url{https://www.kaggle.com/datasets/prasunroy/natural-images}) and extracted from the folder \texttt{natural\_images/flower/} from \texttt{flower\_0000.jpg} up to \texttt{flower\_0009.jpg}.

\subsection{Eigenfaces}
In this experiment, we consider 31 digital images from the CMU Face Images database~\cite{mitchell_cmu_1997}. Those were downloaded from Kaggle (\url{https://www.kaggle.com/datasets/raviprakash22/cmu-face-images}) and extracted from the folder \texttt{faces/faces/choon}. We only extracted the $(60, 64)$ images, corresponding to all the files ending with \texttt{\_2.pgm}.

\subsection{Structured data}
For the structured data experiment (cf. Fig.~\ref{fig:exp_rotation}) and the relative eigengap tables (cf. Fig.~\ref{fig:releigengap_UCI} and Fig.~\ref{appfig:releigengaps_real}), we consider data from the UCI Machine Learning Repository (\url{https://archive.ics.uci.edu/}): Ionosphere~\cite{sigillito_ionosphere_1989}, Wine~\cite{aeberhard_wine_1991}, Wisconsin~\cite{wolberg_breast_1995}, Glass~\cite{german_glass_1987}, Iris~\cite{fisher_iris_1936}, Spambase~\cite{hopkins_spambase_1999}, Digits~\cite{alpaydin_optical_1998}, Covertype~\cite{blackard_covertype_1998}.

\subsection{Implementation Details}
\begin{table}[t!]
\centering
\renewcommand{\arraystretch}{1.4}
\setlength{\tabcolsep}{2pt}
\begin{adjustbox}{width=0.48\textwidth}
\centering
\begin{tabular}{l|c| cc| c}
\toprule[1mm]
\textbf{Method} & \textbf{Type} & \textbf{Acc (\%)} & \textbf{mAcc (\%)} & \textbf{mIoU (\%)} \\ \midrule[1.5pt]\hline

PSPNet~\cite{zhao2017pyramid} & RGB & 91.21 & 83.87 & 76.34 \\
$\text{DeepLabv3}^+$~\cite{chen2018encoder} & RGB & 91.02 & 83.53 & 76.27  \\
{SETR}~\cite{zheng2021rethinking} & RGB & 92.16 & 84.27 & 77.74\\ 
{OCR}~\cite{yuan2020object} & RGB & 92.02 & 85.17 & 78.60 \\ 
{Accel}~\cite{jain2019accel} & Video & 89.15 & 80.69 & 71.64 \\ 
{TDNet}~\cite{hu2020temporally} & Video & 91.05 & 83.38 & 76.48 \\ 
{DAVSS}~\cite{zhuang2020video} & Video & 91.04 & 83.54 & 75.91 \\ 
{TMANet}~\cite{wang2021temporal} & Video & 91.67 & 84.13 & 77.14 \\ 
{PSPNet-LF}~\cite{sheng2022urbanlf} & LF & 92.14 & 84.86 & 78.10 \\
{OCR-LF}~\cite{sheng2022urbanlf} & LF & 92.51 & 86.31 & 79.32 \\ 
{LF-IENet$^4$}~\cite{cong2023combining} & LF & 92.01 & 85.10 & 78.09 \\ 
{LF-IENet$^3$}~\cite{cong2023combining} & LF & 92.09 & 86.03 & 79.19 \\ \hdashline[1pt/1pt]
\textbf{OAFuser9} & LF & {\color[HTML]{FF0000}\textbf{94.45 (+1.94)}} & {\color[HTML]{FF0000}\textbf{88.21 (+1.90)}} & {\color[HTML]{FF0000}\textbf{82.69 (+3.37)}} \\ 
\textbf{OAFuser17} & LF & 94.08 (+1.52) & 87.74 (+1.43) & 82.21 (+2.89) \\ 
\hline %\bottomrule[1mm]
\end{tabular}
\end{adjustbox}
\caption{Quantitative results on the UrbanLF-Real dataset. Acc (\%), mAcc (\%), and mIoU (\%) are reported. The best results are highlighted in red. The variation term indicates the performance difference from the previous best result.}
\label{result:real}
\end{table}

The image size for UrbanLF-Syn is $640{\times}480$ while applying zero padding converts samples in UrbanLF-Real into a size of $640{\times}480$.
Data augmentation is applied with random flipping with a probability of $0.5$, random scaling factors $\{0.5, 0.75, 1, 1.25, 1.5, 1.75\}$, normalization with mean factors $\{0.485, 0.456, 0.406\}$, and standard deviation factors $\{0.229, 0.224, 0.225\}$.
We use the AdamW optimizer with momentum parameters $\{0.9, 0.999\}$ and a weight decay of $0.01$.
The original learning rate is set to ${6e}^{-5}$ and scheduled using the polynomial strategy with a power of $0.9$.
The first $10$ epochs are used to warm up the models.

For experiments on the three different datasets, the training process adapts on three A40 GPUs with a batch size of $3$ on each GPU and the number of training epochs is limited to a maximum of $500$, and the model is based on the MiT-B4 per-train weight~\cite{xie2021segformer}.
For the ablation study of architecture, we train our model with MiT-B2~\cite{xie2021segformer} on one A5000 GPU with a batch size of $2$ and epoch $200$.
With the MiT-B0~\cite{xie2021segformer} on one A5000 GPU with a batch size of $2$, we conduct an ablation study on the CARM and the investigation on the selection of sub-aperture images.
For the selection of sub-aperture images, we choose images that are rich in angular information, mostly in the diagonal, as shown in Fig.~\ref{fig:array}. The experiments on the selection strategy of sub-aperture light-field images are discussed in Sec.~\ref{sec:5_d_select}.



\subsection{Quantitative Results}
To verify our methods, we compare OAFuser with other methods, which include RGB-based methods~\cite{chen2018encoder,zhao2017pyramid,zheng2021rethinking,yuan2020object}, video-based light field semantic segmentation methods~\cite{zhuang2020video,wang2021temporal,jain2019accel,hu2020temporally}, and several specific designs for light field semantic segmentation~\cite{sheng2022urbanlf,cong2023combining} on three datasets, \ie, UrbanLF-Real, UrbanLF-Syn, UrbanLF-RealE. 

\subsubsection{Results on UrbanLF-Real} \
\begin{table}[t!]
  \centering
  \renewcommand{\arraystretch}{1.3}
  \setlength{\tabcolsep}{2pt}
  \begin{adjustbox}{width=0.48\textwidth}
\begin{tabular}{l|c|ccc}
\toprule[1mm]
\textbf{Method} & \textbf{Type} & \textbf{Acc (\%)} & \textbf{mAcc (\%)} & \textbf{mIoU (\%)} \\ \midrule[1.5pt]\hline
{PSPNet~\cite{zhao2017pyramid}} & RGB & 89.39 & 84.48 & 75.78 \\ 
SETR~\cite{zheng2021rethinking} & RGB & 90.97 & 85.26 & 77.69 \\ 
$\text{DeepLabv3}^+$~\cite{chen2018encoder} & RGB & 89.60 & 83.55 & 75.39 \\ 
{OCR~\cite{yuan2020object}} & RGB & 91.50 & 86.96 & 79.36 \\ 
{ACNet~\cite{hu2019acnet}} & RGB-D & 92.53 & 86.62 & 78.56 \\
{MTINet~\cite{vandenhende2020mti}} & RGB-D & 91.24 & 86.94 & 79.10 \\ 
{ESANet~\cite{seichter2021efficient}} & RGB-D & 91.81 & 86.26 & 79.43 \\ 
{SA-Gate~\cite{chen2020bi}} & RGB-D & 92.10 & 87.04 & 79.53 \\ 
{Accel~\cite{jain2019accel}} & Video & 87.56 & 80.52 & 70.48 \\ 
{TDNet~\cite{hu2020temporally}} & Video & 89.06 & 83.43 & 74.71 \\ 
{DAVSS~\cite{zhuang2020video}} & Video & 89.47 & 82.94 & 74.27 \\ 
{TMANet~\cite{wang2021temporal}} & Video & 89.76 & 84.44 & 76.41 \\
{PSPNet-LF~\cite{sheng2022urbanlf}} & LF & 90.55 & 85.91 & 77.88 \\ 
{OCR-LF~\cite{sheng2022urbanlf}} & LF & 92.01 & 87.71 & 80.43 \\ 
{LF-IENet$^4$~\cite{cong2023combining}} & LF & 90.42 & 86.17 & 78.27 \\ 
{LF-IENet$^3$~\cite{cong2023combining}} & LF & {92.41} & {\color[HTML]{FF0000} {\textbf{88.31}}} & {81.78} \\ \hdashline[1pt/1pt]
\textbf{OAFuser9} & LF & 93.23 (+0.82) & 88.26 (-0.05) & 81.64 (-0.14) \\ 
\textbf{OAFuser17} & LF & {\color[HTML]{FF0000} \textbf{93.42 (+1.01)}} & 88.22 (-0.09) & {\color[HTML]{FF0000} \textbf{81.93 (+0.15)}} \\ \hline
\end{tabular}
\end{adjustbox}
\caption{Quantitative results on the UrbanLF-Syn dataset. Acc (\%), mAcc (\%), and mIoU (\%) are reported. The best results are highlighted in red. The variation term indicates the performance difference from the previous best result.}
\label{result:syn}
\end{table}

Table~\ref{result:real} presents the quantitative results on the UrbanLF-Real dataset, which is challenging due to issues like out-of-focus from the plenoptic camera and remaining consistency with light field camera implementation without further data pre-processing in real-world scenarios.
Our OAFuser9 model achieves a state-of-the-art mIoU score of $82.69\%$, showing an improvement of $3.37\%$ compared to previous methods.
Similarly, our OAFuser17 model achieves a mIoU of $82.21\%$, with an increase of $2.89\%$. Regarding Acc and mAcc, both our OAFuser9 and OAFuser17 have improved by over $1\%$ compared to previous works.
The small performance gap between our OAFuser9 and OAFuser17 is attributed to the image quality, which will be discussed in Sec.~\ref{sec:5_DCPC}. 

%
As we increase the number of sub-aperture images, we also observe a corresponding increase in the presence of irrelevant features.
Furthermore, the abundance of out-of-focus and blurry images has led to inaccurate guidance, which surprisingly benefits our network.
From another perspective, it also demonstrates the necessity of our correction module in handling semantic segmentation with light field cameras.
%
The results clearly demonstrate the remarkable effectiveness of the suggested module structure and support our proposed approach, which leverages the rich angular information present in sub-aperture images and combines it with the spatial information from the central view. This fusion of data proves to be beneficial for accurately segmenting the central view image.



\subsubsection{Results on UrbanLF-Syn} \

The quantitative results on the synthetic dataset are shown in Table~\ref{result:syn}.  
%
Among all the models assessed, OAFuser17 attains the highest mIoU. Comparatively, OAFuser9 exhibits slightly inferior performance to LF-IENet$^3$. This discrepancy in performance can be primarily attributed to the restricted size of the synthetic dataset, encompassing merely $173$ samples for training purposes.
% Figure environment removed
%
In concrete terms, OAFuser makes use of the small differences present among different sub-aperture images to capture intricate angular details. These details are then combined with the appropriate spatial information obtained from the main viewpoint. This approach differs from previous methods that mainly focus on spatial information limited to the central view and overlook the angular information available from the light field camera.
%
Additionally, a key feature of OAFuser is its ability to eliminate the need for pre-processing of initial light field images, showcasing its capability to handle raw light field data by effectively utilizing various angular information. Importantly, when using fully focused images from synthetic datasets, the network's processing capabilities are not adequately demonstrated. 
Furthermore, it can be clearly seen from Fig.~\ref{fig:syn} in areas of complex structures, such as the leaves and bicycle pillion seats. Compared with LF-IENet, which leads to loss of global context for objects, such as the gaps between objects and their connection with branches being disrupted, our approach demonstrates better segmentation results even with $17$ sub-aperture images. This means the excellent performance of our network to utilize sub-aperture images and, by rectification, OAFuser remains consistent with the information captured from different points of view. This claim has been also supported by thorough experiments conducted on both the UrbanLF-Real dataset and the UrbanLF-RealE dataset.
%




\subsubsection{Results on UrbanLF-RealE} \
\input{Contents/Experiment_Results/Result_on_UrbanLF_Real_E/Result_on_UrbanLF-Real_E}

%%%%%%%%%%%%%%%%%%%%%%%%%%%%%%%%%%%%%%%%%%%%%%%%%%%%%%%%%%%%%%%%%%%%%%%%%%%%%%%%%%%%%%%%%%%%%%%%%%%%%%%%%%%%%%%%%%%%%%%% Ablation Study %%%%%%%%%%%%%%%%%%%%%%%%%%%%%%%%%%
\section{Ablation Studies}
%
{In this section, we conduct several ablation studies to validate the influence of various modules within our proposed method. The experiments are carried out in Sec.~\ref{sec:5_a_model} to thoroughly investigate the effects of diverse components incorporated in our method. Additionally, we conduct several experiments in Sec.~\ref{sec:5_a_ma} to determine the optimal combination in the CARM. Furthermore, we compare the impact of different datasets in Sec.~\ref{sec:5_DCPC} and analyze the per-class performance in Sec.~\ref{sec:5_c_per}. Moreover, the exploration of the contributions of sub-aperture images is conducted in Sec.~\ref{sec:5_d_select}.}

\subsection{Ablation Study for the Overall Model}\label{sec:5_a_model}
\begin{table}[t]
\centering
\renewcommand{\arraystretch}{1.3}
\begin{adjustbox}{width=0.48\textwidth}
\begin{tabular}{l|cc}
\toprule[1mm]
\multicolumn{1}{c|}{\textbf{Model}} & \textbf{\#Params(M)} & \textbf{mIoU(\%)} \\ \midrule[1.5pt] \hline
\textbf{OAFuser (ours)} & 79.2 & 77.18 \\ \hdashline[1pt/1pt]
\textbf{-~Without CARM} & 65.0~(-14.2) &75.01~(-2.17) \\ \hdashline[1pt/1pt]
%
\textbf{-~SAIs only at Stage One} & 65.0~(-14.2) & 73.50~(-3.68) \\
\hdashline[1pt/1pt]
\textbf{-~SAIs only at Stage Four} & 65.0~(-14.2)& 73.46~(-3.72) \\
\hdashline[1pt/1pt]
\textbf{-~Baseline Method~\cite{zhang2022cmx}} & 65.0~(-14.2)& 73.25~(-3.93) \\ \hdashline[1pt/1pt]
\textbf{-~Without FFM} & 58.4~(-20.8) &70.21~(-6.93) \\ \hline
\end{tabular}
\end{adjustbox}
\caption{{Ablation study of the OAFuser framework: ``SAIs'' indicates the sub-aperture features. ``SAIs only at Stage One'' denotes that the sub-aperture features are fused in the first stage only. ``SAIs only at Stage Four'' means the SAI features are calculated but not fused and fed into the Transformer block for stages one, two, and three. The fusion (the second step \revised{in the} SAFM) occurs only in the fourth stage.}}
\label{tab:ablationstudy for model}
\end{table}
% Figure environment removed

As shown in Table~\ref{tab:ablationstudy for model}, we gradually ablate the proposed OAFuser structure.
When the CARM is replaced with a feature rectification module~\cite{zhang2022cmx}, accuracy dramatically decreases by $2.17\%$.
Although the number of parameters is also reduced, the CARM module is essential to overcome challenges such as image mismatching and out-of-focus issues.~{\revised{Furthermore, we progressively ablate the second component of SAFM,} \ie, {using SAIs in the first or last stage, and without SAIs. \revised{The performance} of these three variants decreases significantly ($2{\sim}3\%$ drops \revised{compared with OAFuser}). }
%
Especially in these variants, the number of parameters remains unchanged, \revised{because angle information} is obtained through pixel-level feature aggregation with shared weights and \revised{addition operations.}
This also validates \revised{the claim} that the introduction of the SAFM efficiently achieves the selection and fusion of rich information from the {LF} camera, enabling the proposed network to handle arbitrary SAIs. Note that the baseline method, CMX~\cite{zhang2022cmx}, adopts a dual-pipeline SegFormer structure. Therefore, we maintain a dual-branch structure and eliminate the addition of angular features (the second step \revised{in the} SAFM) at certain stages.}
{In addition, the FFM~\cite{zhang2022cmx} is also essential because combining complementary features is crucial in the LF semantic segmentation task.}~
%
In addition, we conduct visualization comparisons in this ablation study. Fig.~\ref{fig:ARlation} illustrates the difference maps of the cropped region. The visual result indicates the effectiveness of using the CARM and SAFM for LF semantic segmentation. % for the cropped area.

  

\subsection{Ablation Study for CARM} \label{sec:5_a_ma}

{{To rigorously assess the efficacy of \revised{the proposed} method for rectifying misaligned features, we embarked on a comprehensive suite of ablation studies centered around the innovative CARM framework.}}
\revised{As shown in Table} \ref{subtab:tableB}, we remove the local rectification module and increase the embedding dimension of the global module from $C$ to $4C$ to assess the optimal \revised{configuration}. Subsequently, upon projecting the dimensions to $2C$, our network achieves the highest score of $72.87\%$. \revised{Moreover}, excessively large dimensions (${>}2C$) might hinder the capacity to parse the features. We further explore the number of layers of 3D convolution in \revised{the Local Rectification} to evaluate \revised{its} impact.
\revised{As shown in} Table~\ref{subtab:tableA}, the \revised{use} of 3D convolutions is advantageous for feature rectification when the number of convolutional layers is ${<}4$, and the best mIoU peaks at $73.39\%$. However, when utilizing four layers of 3D convolutions, the interconnections between different features are disrupted, leading to a \revised{reduced} mIoU score of $71.49\%$.

{Subsequently, after exploring the combination of the Local Rectification Module and the Global Rectification Module through the parallel addition of features from both, \revised{we} observed a degradation in the network’s discriminative capability. This unexpected outcome also \revised{demonstrates} the effectiveness of \revised{the proposed} network.}
\begin{table}[t]
\centering

\begin{subtable}{0.48\textwidth}
\centering
\renewcommand{\arraystretch}{1.5}
\begin{tabular}{l|ccccc}
\toprule[1mm]
Dimension & C & 2C & 3C & 4C & 5C \\ \hdashline[1pt/1pt]
mIoU & 70.55 & \color[HTML]{FF0000} \textbf{72.87} & 71.52 & 71.93 & 70.57 \\ \hline
\end{tabular}
\caption{Exploration of the embedding dimension in \revised{the Global Rectification}. The \textit{Dimension} denotes \revised{the} embedding dimension.}
\label{subtab:tableB}
\end{subtable}

\hfill

\begin{subtable}{0.48\textwidth}
\centering
\renewcommand{\arraystretch}{1.5}
\begin{tabular}{l|cccccc}
\toprule[1mm] %\midrule[1.5pt]
Layers & 0 & 1 & 2 & 3 & 4 & P \\ \hdashline[1pt/1pt]
mIoU & 72.87 & 73.13 & 73.06 & \color[HTML]{FF0000} \textbf{73.39} & 71.49 & 61.49\\ 
\hline
%
% \hline
\end{tabular}
\caption{Exploration of \revised{the Local Rectification}.~\textbf{P} denotes the parallel addition of features from both groups.}
%
\label{subtab:tableA}
\end{subtable}

\caption{Ablation Study of the Central Angular Rectification Module~(CARM). \textit{Dimension} denotes the embedding dimension in Global Operation, and \textit{Layers} presents the number of 3D convolutions within Local Operation. mIoU (\%) is reported. The best results are highlighted in red.}
\label{tab:twosubtables}
\vspace{-0.5em}
\end{table}

\subsection{Dataset Comparison} \label{sec:5_DCPC}

%



\begin{table}[b]
  \centering
    \begin{adjustbox}{width=0.48\textwidth}
    \begin{tabular}{l|ccc|ccc}
    \toprule[1mm]
    Datasets & \multicolumn{3}{c|}{\#{(Real - SS)}} & \multicolumn{3}{c}{ \#(BS - SS)} \\ \midrule[1.5pt] \midrule
    \textbf{Model} & \textbf{OCR-LF} & \textbf{LF-IENet$^3$} & \textbf{OAFuser} & \textbf{OCR-LF} & \textbf{LF-IENet$^3$} & \textbf{OAFuser} \\ \hline
    
    \textbf{Acc} & -0.35  & 0.43 & \color[HTML]{FF0000} \textbf{1.22} & -1.74   & -1.64      & \color[HTML]{FF0000} \textbf{0.87}\\ 
    
    {mAcc} & -2.61  & -1.08 &  \color[HTML]{FF0000} \textbf{-0.05} & -3.07   & -3.50      & \color[HTML]{FF0000} \textbf{0.28}\\
    
    {mIoU} & -1.38 & -0.02 &  \color[HTML]{FF0000} \textbf{1.05}& -2.67   & -2.93      & \color[HTML]{FF0000} \textbf{-0.23} \\ 
    
    \bottomrule
    \end{tabular}
    \end{adjustbox}
\caption{\revised{Performance comparison} between different networks. The top three networks that perform best on the synthetic dataset are selected. \#{(Real - SS)} represents the result from the real-world dataset minus \revised{those} from the small disparity dataset, and \#{(BS - SS)} represents the result from the large disparity dataset minus \revised{those} from the small disparity dataset.~ Acc (\%), mAcc (\%), and mIoU (\%) are reported. The best results are highlighted in red.}
\label{fig:dataset comparison}
\end{table}


{To further prove the performance of \revised{the proposed network}, we analyze the impact of image quality and disparity range between distinct datasets. \revised{Although the number of classes and their distribution exhibit similarity across the datasets,} the most prominent difference lies in the image quality and disparity range, as expounded in UrbanLF~\cite{sheng2022urbanlf}.}

{We select the top three methods from the synthetic dataset for comparison (LF-IENet++ is not chosen, as it does not conduct experiments on the synthetic dataset). \textit{Image Quality:} As shown in Fig.~\ref{fig:dataset comparison}, the presence of noise in out-of-focus images \revised{results in} performance degradation for LF-IENet$^3$, and OCR-LF.
\textit{Disparity Range:} \revised{Although a significant disparity provides more information from different angles, it exacerbates misalignment between pixels, leading to poor handling of smaller categories or boundaries in images.} OCR-LF and LF-IENet \revised{exhibit} a performance decline of over $2.50\%$.
However, OAFuser \revised{exhibits} a 0.23\% decrease in mIoU, with improvements in Acc and mAcc. \revised{These improvements contribute to the high performance of the proposed method, which leverages abundant angular information and rectifies features from various viewpoints, thereby reducing the impact of image quality and disparity range on the proposed model. Regardless of noisy conditions or cases of big disparities, the proposed network exhibits superior performance.}}
\begin{table*}[ht!]
  \centering
  \LARGE
  \renewcommand{\arraystretch}{1.8}
  \begin{adjustbox}{width=1\textwidth}
\begin{tabular}{c|c|cccccccccccccccccc}
    \toprule[2mm]

\multirow{2}{*}{\textbf{Dataset}} & \multirow{2}{*}{\textbf{Method}} & \multicolumn{14}{c|}{\textbf{IoU}} & \multicolumn{1}{c|}{\multirow{2}{*}{\textbf{Acc}}} & \multicolumn{1}{c|}{\multirow{2}{*}{\textbf{mAcc}}} & \multirow{2}{*}{\textbf{mIoU}} \\ \cline{3-16}
 &  & \multicolumn{1}{c|}{\textbf{Bike}} & \multicolumn{1}{c|}{\textbf{Building}} & \multicolumn{1}{l|}{\textbf{Fence}} & \multicolumn{1}{c|}{\textbf{Others}} & \multicolumn{1}{c|}{\textbf{Person}} & \multicolumn{1}{l|}{\textbf{Pole}} & \multicolumn{1}{c|}{\textbf{Road}} & \multicolumn{1}{c|}{\textbf{Sidewalk}} & \multicolumn{1}{l|}{\textbf{Traffic Sign}} & \multicolumn{1}{c|}{\textbf{Vegetation}} & \multicolumn{1}{c|}{\textbf{Vehicle}} & \multicolumn{1}{c|}{\textbf{Bridge}} & \multicolumn{1}{c|}{\textbf{Rider}} & \multicolumn{1}{c|}{\textbf{Sky}} & \multicolumn{1}{c|}{} & \multicolumn{1}{c|}{} &  \\ \midrule[3pt]\hline
\multicolumn{1}{l|}{\multirow{3}{*}{RealE}} & \multicolumn{1}{l|}{Proportion} & 2.27 & 33.48 & 3.86 & 1.59 & 3.08 & 1.42 & 21.36 & 8.00 & 0.65 & 3.05 & 16.46 & 2.34 & 0.24 & 2.19 & n.a. & n.a. & n.a. \\ \cline{2-19}
\multicolumn{1}{l|}{} & \multicolumn{1}{l|}{CMX MiT-B4} & 86.88 & 90.52 & 76.27 & 43.47 & 91.31 & 73.47 & 90.74 & 68.70 & 86.22 & 85.21 & 96.45 & 74.33 & 69.15 & 96.61 & 93.00 & 87.00 & 80.66 \\ \cdashline{2-19}
\multicolumn{1}{l|}{} & \multicolumn{1}{l|}{OAFuser9} & 87.92 & 91.95 & 87.64 & 48.56 & 93.63 & 77.55 & 91.74 & 71.92 & 87.43 & 89.04 & 97.05 & 88.89 & 79.03 & 96.61 & 94.61 & 89.84 & \color[HTML]{FF0000} \textbf{84.93~(+4.55)} \\ \hline
\end{tabular}
\end{adjustbox}
\caption{Per-class statistics of CMX and OAFuser on the UrbanLF-RealE dataset are presented. \textit{Proportion} represents the class percentage, and the values are given in percentage (\%). The best result is highlighted in red.}
\label{tab:propotion}
\end{table*}


\subsection{Per-Class Accuracy Analysis} \label{sec:5_c_per}\
\begin{table*}[t]
  \centering
  \LARGE
  \renewcommand{\arraystretch}{1.8}
  \begin{adjustbox}{width=1\textwidth}
\begin{tabular}{c|c|cccccccccccccccccc}
    \toprule[2mm]

\multirow{2}{*}{\textbf{Dataset}} & \multirow{2}{*}{\textbf{Method}} & \multicolumn{14}{c|}{\textbf{IoU}} & \multicolumn{1}{c|}{\multirow{2}{*}{\textbf{Acc}}} & \multicolumn{1}{c|}{\multirow{2}{*}{\textbf{mAcc}}} & \multirow{2}{*}{\textbf{mIoU}} \\ \cline{3-16}
 &  & \multicolumn{1}{c|}{\textbf{Bike}} & \multicolumn{1}{c|}{\textbf{Building}} & \multicolumn{1}{l|}{\textbf{Fence}} & \multicolumn{1}{c|}{\textbf{Others}} & \multicolumn{1}{c|}{\textbf{Person}} & \multicolumn{1}{l|}{\textbf{Pole}} & \multicolumn{1}{c|}{\textbf{Road}} & \multicolumn{1}{c|}{\textbf{Sidewalk}} & \multicolumn{1}{l|}{\textbf{Traffic Sign}} & \multicolumn{1}{c|}{\textbf{Vegetation}} & \multicolumn{1}{c|}{\textbf{Vehicle}} & \multicolumn{1}{c|}{\textbf{Bridge}} & \multicolumn{1}{c|}{\textbf{Rider}} & \multicolumn{1}{c|}{\textbf{Sky}} & \multicolumn{1}{c|}{} & \multicolumn{1}{c|}{} &  \\ \midrule[3pt]\hline

%
\multicolumn{1}{l|}{\multirow{3}{*}{RealE}} & \multicolumn{1}{l|}{Proportion} & 2.27 & 33.48 & 3.86 & 1.59 & 3.08 & 1.42 & 21.36 & 8.00 & 0.65 & 3.05 & 16.46 & 2.34 & 0.24 & 2.19 & n.a. & n.a. & n.a. \\ \cline{2-19}
\multicolumn{1}{l|}{} & \multicolumn{1}{l|}{CMX MiT-B4} & 86.88 & 90.52 & 76.27 & 43.47 & 91.31 & 73.47 & 90.74 & 68.70 & 86.22 & 85.21 & 96.45 & 74.33 & 69.15 & 96.61 & 93.00 & 87.00 & 80.66 \\ \cdashline{2-19}
\multicolumn{1}{l|}{} & \multicolumn{1}{l|}{OAFuser9} & 87.92 & 91.95 & 87.64 & 48.56 & 93.63 & 77.55 & 91.74 & 71.92 & 87.43 & 89.04 & 97.05 & 88.89 & 79.03 & 96.61 & 94.61 & 89.84 & \color[HTML]{FF0000} \textbf{84.93~(+4.55)} \\ \hline
\end{tabular}
\end{adjustbox}
\caption{Per-class statistics of CMX and OAFuser on the UrbanLF-RealE dataset are presented. \textit{Proportion} represents the class percentage, and the values are given in percentage (\%). The best result is highlighted in red.}
\label{tab:propotion}
\end{table*}

%

To comprehensively assess the model's performance per class and delve deeper into the improvement achieved by our model in comparison with the baseline method, we present a summary of statistical information from different methods in Table~\ref{tab:propotion}. Moreover, the class proportion assisting in data analysis is also introduced.

Given that our baseline method, CMX~\cite{zhang2022cmx}, is designed for two modalities, we first aggregate the sub-aperture images before feeding them into the network. It can be seen that OAFuser significantly outperforms the baseline method across various categories, such as \textit{fence, pole, sidewalk, vehicle, bridge} and \textit{rider}, which are crucial for the road system in urban scenarios. {Significantly, the recognition capability for \textit{vehicle} even exceeds $97\%$ in IoU.}

Those results prove the achievement of our proposed model for segmentation tasks in urban scenes. By incorporating SAFM and CARM, the network unleashes the potential of the angle information from different sub-aperture images and leverages the consistency of variations among sub-aperture images. This enables OAFuser to deliver promising performance.


 

\subsection{Investigation on the Selection of Sub-Aperture Images} \label{sec:5_d_select}
%
\begin{table}[t!]
\renewcommand{\arraystretch}{1.2}
\begin{adjustbox}{width=0.48\textwidth}
\centering
\setlength{\tabcolsep}{10pt}
\begin{tabular}{c|c|c}
\toprule[1mm]
\textbf{Method}           & \textbf{mIoU} & \textbf{Improvement}
\\ \midrule[1.5pt] \hline
CMX MiT-B0~(RGB V11)~\cite{zhang2022cmx}& 68.40    & \\  \hdashline[1pt/1pt]
OAFuser2 MiT-B0  & 71.17    & +2.77\\ 
OAFuser5 MiT-B0& 71.92   & +3.52 \\
OAFuser9 MiT-B0& 72.57  & +4.17\\
OAFuser13 MiT-B0& 72.60  & +4.20 \\
OAFuser17 MiT-B0& \color[HTML]{FF0000}\textbf{72.87}  & +4.47  \\
{OAFuser21} MiT-B0& 72.22  & +3.82 \\ \hdashline[1pt/1pt]
OAFuser17~MiT-B4& \color[HTML]{FF0000} \textbf{81.93}   & +13.53 \\\hline
\end{tabular}
\end{adjustbox}
\caption{
%
Exploration of the contribution of different numbers of sub-aperture images. mIoU~(\%) is reported. V11 denotes the sub-aperture image from the top-left viewpoint. The best results are highlighted in red.}
\label{tab:Sub-aperture images}
\end{table}

{{To demonstrate that a more significant number of sub-aperture images, as compared to a single image, can provide more effective information for scene understanding, several experiments are conducted.}} As shown in Table~\ref{tab:Sub-aperture images}, compared with our baseline method CMX~\cite{zhang2022cmx}, which is designed for similar modalities fusion, our OAFuser achieves an increase of more than $2\%$ in mIoU.
This result further confirms the effectiveness of our design in handling asymmetric data from the {LF} camera for segmentation tasks.
Furthermore, our network perceives the scene from multiple perspectives, and with increased viewpoints, it achieves progressively higher accuracy. {This also showcases our network's capacity to harness angular information and underscores the importance of leveraging a higher count of sub-aperture images for the segmentation task. Moreover, under the same model, the performance continually improves with the incremental addition of sub-aperture images, further validating the efficacy of sub-aperture images' extraction and utilization in enhancing road scene understanding.} OAFuser reaches its peak at OAFuser17. 


\section{Conclusion}
\section{Conclusion}\label{sec:conclusion}

This paper presents our empirical domain knowledge distillation framework using ChatGPT and discusses our observations from the framework application experiments in the autonomous driving domain. The key finding is that: 1) with proper design of prompt engineering and execution flow, fully automated domain knowledge (in the ontology format) distillation is possible. However, due to the randomness in the response and the butterfly effect, the quality of fully automated distillation results is not guaranteed. To address this, we develop a web-based assistant to enable manual supervision and early intervention at runtime. We hope our findings and tools inspire future research toward revolutionizing the engineering processes of knowledge-based systems across domains.

\bibliographystyle{IEEEtran}
\bibliography{bib}

\end{document}
