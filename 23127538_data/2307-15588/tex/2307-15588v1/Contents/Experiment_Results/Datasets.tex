\begin{table}[t!]
\centering
\setlength{\tabcolsep}{10pt}
\renewcommand{\arraystretch}{1.5}
\begin{adjustbox}{width=0.48\textwidth}
\begin{tabular}{c|ccc|c}
\toprule[1mm]
\textbf{Dataset}           & \textbf{Train} & \textbf{Val} & \textbf{Test} & \textbf{Total} \\ \midrule[1.5pt] \hline

UrbanLF-Real      & 580   & 80  & 164  & 824   \\ 
UrbanLF-Syn       & 172   & 28  & 50   & 250   \\ 
UrbanLF-RealE & 780   & 80  & 164  & 1024   \\ \hline
\end{tabular}
\end{adjustbox}
\caption{
%
\textbf{Statistic information of light field semantic segmentation datasets.} UrbanLF-RealE denotes UrbanLF-Real with an extension of synthetic samples for training.}
\label{tab:dataset}
\end{table}

% Figure environment removed

Our experiments are based on the UrbanLF dataset~\cite{sheng2022urbanlf}.
The dataset comprises $14$ categories for urban semantic scene understanding.
Each sample in the dataset consists of $81$ sub-aperture images with an angular resolution of $9{\times}9$.
In our experiments, we follow the protocols proposed in UrbanLF and conducted three sets of experiments: Urban-Syn, Urban-Real, and UrbanLF-RealE. The UrbanLF-RealE is an extension of all real and synthetic data.
Table~\ref{tab:dataset} presents the number of images used for training, evaluation, and testing.




