\begin{table}[t!]
\centering

\begin{subtable}{0.48\textwidth}
\centering
\renewcommand{\arraystretch}{1.5}
\begin{tabular}{l|ccccc}
\midrule[1.5pt]
Dimension & C & 2C & 3C & 4C & 5C \\ \hdashline[1pt/1pt]
mIoU & 70.55 & \color[HTML]{FF0000} \textbf{72.87} & 71.52 & 71.93 & 70.57 \\ \hline
\end{tabular}
\caption{Exploration of the embedding dimension in Global Rectification. The \textit{Dimension} denotes embedding dimension.}
\label{subtab:tableB}
\end{subtable}

\hfill

\begin{subtable}{0.48\textwidth}
\centering
\renewcommand{\arraystretch}{1.5}
\begin{tabular}{l|ccccc}
\midrule[1.5pt]
Layers & 0 & 1 & 2 & 3 & 4 \\ \hdashline[1pt/1pt]
mIoU & 72.87 & 73.13 & 73.06 & \color[HTML]{FF0000} \textbf{73.39} & 71.49\\ 
\hline
%
% \hline
\end{tabular}
\caption{Exploration of Local Rectification.}
%
\label{subtab:tableA}
\end{subtable}

\caption{Ablation Study of the Central Angular Rectification Module~(CARM). \textit{Dimension} denotes the embedding dimension in Global Operation and \textit{Layers} presents the number of 3D convolution within Local Operation. mIoU (\%) is reported. The best results are highlighted in red.}
\label{tab:twosubtables}
\end{table}
To explore the effectiveness of rectifying asymmetric features, we conduct a series of ablation experiments regarding the proposed CARM. 
As depicted in Table \ref{subtab:tableB}, we remove the local rectification module and increase the embedding dimension of the global module from $C$ to $4C$ to assess the optimal manner. Subsequently, upon projecting the dimensions to $2C$, our network achieves the highest score of $72.87\%$. Besides, excessively large dimensions (${>}2C$) might hinder the capacity to parse the features. To address the issue, we further explore the number of layers of 3D convolution in Local Rectification, to evaluate the impact.
%
As in Table~\ref{subtab:tableA}, the employment of 3D convolutions proves advantageous for feature rectification when the number of convolutional layers is ${<}4$, and the best mIoU peaks at $73.39\%$. However, when utilizing four layers of 3D convolutions, the interconnections between different features are disrupted, leading to a diminished mIoU score of $71.49\%$.
%

Subsequently, we explore another combination of the Local Rectification Module and Global Rectification Module by parallel addition of features from two groups, which results in a degradation of the discriminative capability of the network.
