% ****** Start of file apssamp.tex ******
%
%   This file is part of the APS files in the REVTeX 4.2 distribution.
%   Version 4.2a of REVTeX, December 2014
%
%   Copyright (c) 2014 The American Physical Society.
%
%   See the REVTeX 4 README file for restrictions and more information.
%
% TeX'ing this file requires that you have AMS-LaTeX 2.0 installed
% as well as the rest of the prerequisites for REVTeX 4.2
%
% See the REVTeX 4 README file
% It also requires running BibTeX. The commands are as follows:
%
%  1)  latex apssamp.tex
%  2)  bibtex apssamp
%  3)  latex apssamp.tex
%  4)  latex apssamp.texx`
%
\documentclass[%
 reprint,
%superscriptaddress,
% groupedaddress,
%unsortedaddress,
%runinaddress,
%frontmatterverbose, 
% preprint,
%preprintnumbers,
%nofootinbib,
%nobibnotes,
%bibnotes,
% amsmath,amssymb,
  aps,
 prl,
%pra,
%prb,
%rmp,
%prstab,
%prstper,
%floatfix,
]{revtex4-2}

\usepackage{graphicx}% Include figure files
\usepackage{xcolor}%
\usepackage{dcolumn}% Align table columns on decimal point
\usepackage{bm}% bold math
%\usepackage{hyperref}% add hypertext capabilities
%\usepackage[mathlines]{lineno}% Enable numbering of text and display math
% \linenumbers\relax % Commence numbering lines

%\usepackage[showframe,%Uncomment any one of the following lines to test 
%%scale=0.7, marginratio={1:1, 2:3}, ignoreall,% default settings
%%text={7in,10in},centering,
%%margin=1.5in,
%%total={6.5in,8.75in}, top=1.2in, left=0.9in, includefoot,
%%height=10in,a5paper,hmargin={3cm,0.8in},
%]{geometry}

\def\ZZ{\mathrm{Z}}
\def\AC{\mathrm{A}}
\def\Step{\mathrm{Edge}}
\def\thetaZZ{\theta_\ZZ}
\def\thetaAC{\theta_\AC}
\def\thetaStep{\theta_\Step}
\def\GAC{G_\AC}
\def\GZZ{G_\ZZ}
\def\gammaAC{\gamma_\AC}
\def\gammaZZ{\gamma_\ZZ}
\def\Kone{K_\mathrm{\uppercase\expandafter{\romannumeral1}}}
\def\Ktwo{K_\mathrm{\uppercase\expandafter{\romannumeral2}}}
\def\Gc{G_\mathrm{c}}
\newcommand\zp[1]{{\color{blue}#1}}
\newcommand\pj[1]{{\color{magenta}#1}}

\begin{document}

% \preprint{APS/123-QED}

\title{Non-Equilibrium Nature of Fracture Determines the Crack Paths}% Force line breaks with \\
% \thanks{A footnote to the article title}%

\author{Pengjie Shi}
\author{Shizhe Feng}%
\author{Zhiping Xu}%
\email{xuzp@tsinghua.edu.cn}
\affiliation{%
 Applied Mechanics Laboratory and Department of Engineering Mechanics, Tsinghua University, Beijing, 100084, China.
}%


\date{\today}% It is always \today, today,
             %  but any date may be explicitly specified

\begin{abstract}
A high-fidelity neural network-based force field, NN-F$^{3}$, is developed to cover the strain states up to material failure and the non-equilibrium, intermediate nature of fracture.
Simulations of fracture in 2D crystals using NN-F$^{3}$ reveal spatial complexities from lattice-scale kinks to sample-scale patterns.
We find that the fracture resistance cannot be quantified by the energy densities of relaxed edges as in the literature.
Instead, the fracture patterns, critical stress intensity factors at the kinks, and energy densities of edges in the intermediate, unrelaxed states offer reasonable measures for the fracture toughness and its anisotropy.
\end{abstract}

%\keywords{Suggested keywords}%Use showkeys class option if keyword
%display desired
\maketitle

%\tableofcontents


%\textit{Introduction.}-
Fracture is a catastrophic process in nature and engineering, which leaves facets and kinks along the crack paths. %\cite{ritchie_2011}.
In 2D crystals such as graphene and h-BN, different types of edges can be cleaved by fracture, including zigzag (Z), armchair (A), and mixed zigzag-armchair or chiral (C) edges (Fig.\,\ref{Fig_1}a).
%The atomic-level structures of the edges modulate their electronic and magnetic properties \cite{magda2014room, slota2018magnetic} as well as the carrier transport process in the bulk \cite{karakachian2020one}, which hold great promises for next-generation electronics \cite{wang2021graphene}. 
The relative stabilities of graphene edge structures were explored experimentally through the abundance of edges created by various techniques such as fracture \cite{feng_2022}, mechanical exfoliation \cite{qu_2022}, and irradiation \cite{fujihara_2015}. 
Most observations show almost the same probabilities of zigzag and armchair edges \cite{kim_2013,neubeck_2010,jia_2009}, while a few of them report either zigzag or armchair direction is preferred over the other \cite{fujihara_2015,shi_2020,girit_2009,qu_2022,kim_2012}.
These facts suggest that the stabilities of zigzag and armchair edges could be quite close, which contradicts theoretical predictions from first-principles calculations.
Ground-state calculations based on the density functional theory (DFT) show that electronic and structural relaxation of the armchair edge of graphene significantly reduces its energy density, $\gammaAC$, which becomes lower than that of the zigzag edge, $\gammaZZ$ \cite{jun_2008,koskinen_2008,huang_2009,gan_2010,Boris_2010,gao_2011,yin_2015}.
The disagreement between theory and experiment remains unsolved for more than a decade.

% Figure environment removed

The selection of crack paths during fracture is closely related to the relative stability of edges. 
Under the framework of fracture mechanics, the crack driving force can be measured by the energy release rate (ERR), $G$, while the energy cost to activate the fracture is defined as the fracture toughness, $\Gc$ \cite{lawn_2004}.
The value of $\Gc$ is difficult to determine by theory due to the non-equilibrium nature of fracture and thus commonly measured by experiments \cite{lawn_2004}.
In theoretical studies, the fracture resistance is usually approximated by the surface or edge energy densities as $\Gc = 2\gamma$ (Fig.\,\ref{Fig_1}b) \cite{griffith_1921,lawn_2004,zhigong_2021,zhang_2014}.
The directional dependence of $\Gc\left(\theta\right)$, which defines the relative stability of different edges, is expected to align with that of $2\gamma\left(\theta\right)$ \cite{takei_2013,feng_2022}.

By analyzing the crack path under specific loading conditions, the relative stability or the anisotropy of fracture can be deduced.
The anisotropy in $\Gc\left(\theta\right)$ and $\gamma\left(\theta\right)$ of crystals with a honeycomb lattice such as graphene can be quantified by the ratios between their values at post-fracture zigzag (Z) and armchair (A) edges, $A_G=\GZZ/\GAC$, and $A_\gamma=\gammaZZ/\gammaAC$ \cite{Boris_2010}, respectively \cite{kim_2013, neubeck_2010, jia_2009}.
Recently, direct tensile tests of monolayer graphene and peeling tests of highly-oriented pyrolytic graphite (HOPG), 
%were reported \cite{feng_2022,qu_2022}.
%The experimental results, 
although unable to resolve the atomic-scale edge structures,
%at the lattice scale, 
conclude weak anisotropies ($A_G = 1.06$ \cite{feng_2022} and $0.971$ \cite{qu_2022}) from the overall orientation of cracks.

Energy densities of edges cleaved by fracture cannot be directly measured in experiments, and the use of theoretically calculated $2\gamma(\theta)$ as the fracture toughness remains questionable.
In fact, experimentally measured fracture toughness is usually much higher than the value of $2\gamma$ even for brittle crystals where plastic dissipation is absent \cite{zhigong_2021,zhang_2014,delrio_2015, delrio_2022} (Fig.\,\ref{Fig_1}b).
Large-scale molecular dynamics (MD) simulations may help address the issue if provided with force fields of high accuracy and low cost.
Empirical force fields (FFs) reported in the literature cannot capture the non-equilibrium nature and high, non-uniform lattice distortion at the crack front \cite{jung_2019,Boris_2010, sen_2010,kim_2012, zhang_2014, yin_2015, song_2017, zhang2022atomistic}. 
The values of $A_\gamma$ calculated using Stillinger-Weber or Tersoff potentials are the same as the bond-cutting estimation, $\sqrt{3}/2$, and do not correctly capture the bonding characteristics of materials \cite{hossain_2018,zhang2022atomistic}.
By including the chemistry of interatomic bonding, the adaptive intermolecular reactive empirical bond order (AIREBO) predicts $A_\gamma < 1$ \cite{Boris_2010, zhang_2014, yin_2015}, while the reactive force field (ReaxFF) yields opposite results, $A_\gamma>1$  \cite{sen_2010,kim_2012,song_2017} (Fig.\,\ref{Fig_1}c).
Compared with the experimental measurements \cite{feng_2022,qu_2022}, DFT calculations predict a relatively strong anisotropy with $A_\gamma>1.1$, where electronic and structural relaxation of the edges are taken into account \cite{jun_2008,koskinen_2008,huang_2009,gan_2010,Boris_2010,gao_2011,yin_2015}\,\cite{supp_info}.
%(Fig.\,S1\,\cite{supp_info}).
The DFT predictions are also quantitatively different from the AIREBO and ReaxFF results.
Recently, the implementation of neural network-based force fields \cite{friederich_2021} shows the capability to resolve the accuracy-cost dilemma and led to significant progress in several fields \cite{galib2021reactive, font2022predicting, li2022origin, hedman2023dynamics}.
However, the lack of a reasonable description of the non-equilibrium nature and exploration of the full space of strain states leaves the atomistic approach to fracture still immature.

Here we develop a neural network-based force field for fracture (NN-F$^{3}$) for 2D crystals including graphene and h-BN based on first-principles calculations and an active-learning framework \cite{friederich_2021}.
%The NN-F$^{3}$ incorporates the finite-strain effects, bond-breaking, and (re)formation events at the crack tip, as well as relaxation processes at the cleaved edges, which have been overlooked in previous machine-learning models \cite{Boris_2010,mortazavi_2021,achar_2021}.
%MD simulations using NN-F$^{3}$ identify the relationship between the fracture toughness and various definitions of the edge energy densities (Fig.\,\ref{Fig_1}b).
%We find that, as an indicator of the success of NN-F$^{3}$, the anisotropy of fracture resistance can be reasonably measured by the energy densities of unrelaxed edges in the intermediate states of fracture, which matches the simulation and experimental data.
%In practice, the high-fidelity NN-F$^{3}$ allows us to explore the selection of crack paths by the stress intensity factors (SIFs), which define the roughness of fractured edges and the fracture patterns.
%\textit{Development and Performance of NN-F$^3$.}—In developing our neural network-based force fields, 
The tensorial nature of strain states and the undercoordination nature of cleavaged edges~\cite{Boris_2010} demand a large training set of DFT data and are addressed by an active-learning workflow~\cite{dpgen}.
Our training sets for NN-F$^{3}$ include structures with strained lattices (the uniaxial strain in the range of $0-0.25$ along different lattice directions), cleaved edges (zigzag and armchair segments as well as the kinks between them), and cracks ($203,554$ datasets in total).
The Deep Potential Smooth Edition (DeepPot-SE) model~\cite{NEURIPS2018_e2ad76f2,deepmd} is used to train NN-F$^{3}$, 
%, the performance of which is validated by the energies, forces, and stress-strain relations~\cite{supp_info}.
%(Figs.\,\ref{Fig_1}b, c, and S3).
the performance of which is validated by reporting mean absolute errors (MAEs) of the energies per atom, the edge energy densities, and the interatomic forces below $2\,\mathrm{meV}$, $2.2\,\mathrm{meV/\AA}$ and $43\,\mathrm{meV/\AA}$, respectively.
The relative error (RE) in the stress-strain relations is under $2\%$ (Fig.\,\ref{Fig_1}d).
The workflow thus assures an accurate description of the crack-tip processes\,\cite{supp_info}.
%(See details in Supplemental Material\,\cite{supp_info}, Note\,S1)

% Figure environment removed

\textit{Fracture Anisotropy of Graphene}-
The fracture of graphene is explored by quasi-static uniaxial tension using the Large-scale Atomic/Molecular Massively Parallel Simulator (LAMMPS) \cite{lammps,supp_info}.
%(See details in Supplemental Material\,\cite{supp_info}, Note\,S2).
In order to host relatively long cracks, wide samples ($W\approx50\,\mathrm{nm}$) are constructed (Fig.\,\ref{Fig_2}).
One atom at the left edge is removed to initialize the crack.
Periodic boundary conditions (PBCs) are enforced along the tensile direction.
The span $L$ is in the range of $4 - 8~\mathrm{nm}$ (and $15 - 20~\mathrm{nm}$ to see the size dependence) to accommodate different lattice orientations ($\thetaZZ\in\left[0^\circ,30^\circ\right]$, measured from the zigzag motif, Fig.\,\ref{Fig_2}a, b).
%The size effect is assessed by comparative MD simulations using larger samples, which suggest that the interaction between the crack and its periodic images is negligible (Fig.\,\ref{Fig_2}b).

% Figure environment removed

MD simulation results show that the cracks are straight at the sample scale (Figs.\,\ref{Fig_3})~\cite{supp_info}.
%, \ref{Fig_3}d and S5b).
Their overall orientations are denoted by $\thetaStep$ (measured from the zigzag motif, see Fig.\,\ref{Fig_2}a).
However, the cracks may deflect at the lattice scale, leaving kinks between the zigzag and armchair segments behind (Fig.\,\ref{Fig_2}c).
The relations between $\thetaStep$ and $\thetaZZ$ are summarized in Fig.\,\ref{Fig_2}b, which can be classified into three regimes.
For $\thetaZZ\in[0^\circ,10^\circ]$ or $[27^\circ,30^\circ]$, the crack advances along the zigzag ($\thetaStep=0^\circ$) or armchair ($\thetaStep=30^\circ$) direction, respectively.
For $\thetaZZ\in[10^\circ,27^\circ]$, the crack advances between them ($\thetaStep\in\left(0^\circ,30^\circ\right)$).
Cleavage of zigzag edges dominates if the loading direction is uniformly sampled, which is attributed to the fact of $\GZZ<\GAC$\,\cite{supp_info}.
%(Fig.\,S5a\,\cite{supp_info}).
This finding conforms with the observations in the peeling experiments of HOPG where the polycrystalline texture is randomly oriented \cite{qu_2022}.
%or transmission electron microscopy (TEM) experiments where the loading conditions can be arbitrary \cite{kim_2012}.
However, the energy densities of relaxed edge $\gamma(\theta)$ obtained from DFT calculations display an opposite trend of $\gammaZZ > \gammaAC$ \cite{jun_2008, koskinen_2008, huang_2009, gan_2010, Boris_2010, gao_2011, yin_2015, supp_info}.
%(Figs.\,S3d and \ref{Fig_4}).
This inconsistency indicates that $\gamma(\theta)$ fails to correctly characterize the anisotropy in fracture resistance.

The crack driving force under uniaxial tensile stress $\sigma_y$ is $G\left(\thetaStep\right)\sim\cos^2\left(\thetaZZ-\thetaStep\right)\sigma_y^2$\,\cite{supp_info}.
%(See Supplemental Material\,\cite{supp_info}, Note\,S3).
Following the criterion of maximum ERR (MERR), the crack will advance in the direction with $G\left(\thetaStep\right) \geq \Gc\left(\thetaStep\right)$. 
In the honeycomb lattice of graphene, the cleaved edges consist of zigzag and armchair segments, and the value of $\Gc(\thetaStep)$ can be estimated as the average value of $\GAC$ and $\GZZ$ weighted by their lengths \cite{qu_2022,Boris_2010}, that is
\begin{equation}
\Gc\left(\thetaStep\right)=2\GAC\left[\sin\left(\thetaStep\right)+A_G\sin\left(30^\circ-\thetaStep\right)\right].\label{Eq_1}
\end{equation}
This result presumes that the formation and interaction energies of lattice kinks are negligible in comparison with the edge energies \cite{Boris_2010, lee2023importance}.
The direction of crack propagation, $\thetaStep$, can thus be obtained from the lattice orientation, $\thetaZZ$, by finding the minimum of $\sigma_y^2$ that satisfies $G = \Gc$, that is
\begin{equation}
	\thetaStep = \arg\min\sigma_y^2 = \arg\min\frac{\Gc\left(\thetaStep\right)}{\cos^2\left(\thetaZZ-\thetaStep\right)}.\label{Eq_2}
\end{equation}
%The relation between $\thetaZZ$ and $\thetaStep$ is shown in Fig.\,\ref{Fig_2}b.
The predictions using $A_G=0.96$ and $0.93$ fit the simulation results for $\thetaStep$ smaller and larger than the critical value of $\theta_{\rm c} = 19.11^\circ$ (Fig.\,\ref{Fig_2}b), respectively.
At $\thetaStep=19.11^\circ$, the numbers of zigzag and armchair segments are the same along the edge (Fig.\,\ref{Fig_2}c).
The smaller values of $A_G$ at $\thetaStep>19.11^\circ$ may be attributed to the asymmetry between the A and B sites at the armchair edges (Fig.\,\ref{Fig_2}d), which can elevate the fracture toughness and promote deflection.
Similar effects of the edge asymmetry on crack deflection and toughening were also observed in h-BN \cite{zhigong_2021,supp_info} and $\mathrm{WS_2}$ \cite{jung_2019}.
% (Fig.\,S9\,\cite{supp_info})

\textit{Origin of Edge Kinks}-
The high-fidelity NN-F$^{3}$ allows us to explore the edge structures at the atomic level.
%by performing atomistic simulations.
Zigzag and armchair segments as well as lattice kinks connecting them can be resolved at length scales where fracture mechanics can be applied for analysis.
Large-scale MD simulations excluding the size effects show periodic crack patterns\,\cite{supp_info}
%(Fig.\,S9\,\cite{supp_info})
, and highlight the advantages of NN-F$^{3}$ in offering high accuracy at the first-principles level and low computation cost that allows direct simulations up to the experimental scale\,\cite{supp_info,feng_2022} 
%(Fig.\,S9\,\cite{supp_info})
% and Ref. \cite{feng_2022}.
%In the continuum framework, 
The criterion of MERR \cite{nuismer1975,takei_2013,feng_2022} suggests that the direction of a propagating crack defined in the local coordinate system (Fig.\,\ref{Fig_3}c and \ref{Fig_3}d) is

\begin{equation}
	\alpha = \arg\max\frac{G\left(\alpha\right)}{\Gc\left(\alpha\right)},\label{Eq_3}
\end{equation}
where $G(\alpha)$ is evaluated by the SIFs in the tensile and shear modes ($\Kone$ and $\Ktwo$, respectively) as the out-of-plane displacement is ignored \cite{song_2017}.
The values of $\Kone$ and $\Ktwo$ can be determined by fitting the crack-tip displacement field with the Williams power expansion \cite{williams1957,supp_info}.
%(See details in Fig.\,\ref{Fig_3}c and Supplemental Material\,\cite{supp_info}, Note\,S4).

For loading conditions with crack directions not aligning with the zigzag or armchair motifs, we find that the presence of a mode-{\uppercase\expandafter{\romannumeral2}} feature could deflect the mode-{\uppercase\expandafter{\romannumeral1}} crack \cite{lawn_2004, feng_2022}.
The effect can be measured by the ratio $\Ktwo/\Kone$ extracted from MD simulations.
Two representative examples are shown in Figs.\,\ref{Fig_3}c and \ref{Fig_3}d, where the cleaved edges are dominated by the zigzag and armchair segments, respectively.
The value of $\Ktwo/\Kone$ oscillates as the crack propagates (Figs.\,\ref{Fig_3}a and \ref{Fig_3}b), indicating that the deflection is activated as the ratio approaches the threshold values $\left(\Ktwo/\Kone\right)_\mathrm{c}$.
The threshold depends on the loading conditions and lattice orientations and is higher for cracks advancing in the zigzag direction than that along the armchair ones.
The asymmetry between the A and B sites of the armchair edge further breaks the symmetry (Fig.\,\ref{Fig_2}d) and yields two thresholds (Fig.\,\ref{Fig_3}b), which confirms the effect of edge asymmetry on $A_{G}$ (Fig.\,\ref{Fig_2}b). 

To estimate the values of $\left(\Ktwo/\Kone\right)_\mathrm{c}$ and their relations with $A_G$, a dimensionless quantity $\Delta$ is introduced based on Eq.\,\ref{Eq_3} as\,\cite{supp_info}
%(see Supplemental Material\,\cite{supp_info}, Note\,S5)
\begin{equation}
	\alpha = \arg\max\left[\frac{G\left(\alpha\right)}{\Gc\left(\alpha\right)}\frac{2\GAC E}{\Kone^2}\right]= \arg\max\left[\Delta\left(\alpha,A_G,\frac{\Ktwo}{\Kone}\right)\right],\label{Eq_4}
\end{equation}
%\noindent where $\GAC$ is the fracture toughness along the armchair motif and $E$ is Young's modulus.
The direction of cracks determined from $A_G$ and $\Ktwo/\Kone$ follows the armchair or zigzag motifs ($\alpha = 0^\circ$ or $30^\circ$) due to the discrete nature of lattices. 
The relations between $\Ktwo/\Kone$ and $\Delta$ in the armchair- and zigzag-dominated regimes with $A_G=0.96$ are summarized in Figs.\,\ref{Fig_3}e and \ref{Fig_3}f, where the thresholds $\left(\Ktwo/\Kone\right)_\mathrm{c}$ are identified as $0.092$ and $0.18$, respectively.
Alternatively, the values of $A_G$ can be obtained from $\left(\Ktwo/\Kone\right)_\mathrm{c}$ that is directly determined by experiments or simulations (Fig.\,\ref{Fig_3}g and Table\,\ref{Tab_1}).
The results show that the value of $A_G$ does not match the anisotropy measured from the energies of relaxed edges in direct NN-F$^{3}$ or DFT calculations, $A_\gamma=\gammaZZ/\gammaAC = 1.113$ (Figs.\,\ref{Fig_1}c)~\cite{supp_info}.
%and S3d).

% Figure environment removed

\begin{table}[b]
\caption{The number of kinks in the sample with width $W \approx 50$ nm, and the relation between $\left(\Ktwo/\Kone\right)_{\rm c}$ and $A_G$ under loading conditions defined by $\thetaZZ$.} \label{Tab_1}%
\begin{ruledtabular}
\begin{tabular}{cccc}
\textrm{$\thetaZZ$}&
\textrm{\# kinks }&
\textrm{$\left(\Ktwo/\Kone\right)_{\rm c}$}&
\textrm{$A_G$}\\
\colrule
$25.87^\circ$ & 8  & $\left[0.080,0.100\right]$  & $\left[0.940,0.966\right]$  \\
$10.89^\circ$ & 13 &$\left[0.205,0.210\right]$   & $\left[0.935,0.940\right]$   \\
$9.82^\circ$  &  2 & $\left[0.195,0.200\right]$ & $\left[0.945,0.950\right]$  \\  
\end{tabular}
\end{ruledtabular}
\end{table}

\textit{Energy Densities of Unrelaxed Edges}-
The mismatch between $A_{G}$ and $A_{\gamma}$ implies that the anisotropy in the edge energies density fails to capture the atomistic kinetics of fracture, which selects the crack path.
%, and a better measure for the fracture anisotropy is needed.
Since the work of fracture should not depend on posterior edge relaxation processes after the event of cleavage, the energy densities of unrelaxed edges, $2\Gamma(\theta)$, are calculated using NN-F$^{3}$ or DFT calculations and compared to $2\gamma(\theta)$ for relaxed edges.
The results, $A_{\Gamma}=0.959<1$, suggest a weak anisotropy in the fracture toughness, agreeing well with the experimental evidence \cite{qu_2022,feng_2022} and the simulation results (Fig.\,\ref{Fig_4}a).
The values of $2\Gamma_{\rm Z}$ and $2\Gamma_{\rm A}$ also conform qualitatively well with $G_{\rm Z}$ and $G_{\rm A}$, respectively, by ignoring the lattice-trapping effects (Fig.\,\ref{Fig_1}b).
We investigate several measures of fracture energies\,\cite{supp_info}, 
%(See details in Supplemental Material\,\cite{supp_info}, Note\,S6)
and conclude that the energy density of unrelaxed edges, $A_\Gamma$, can be a good indicator of fracture resistance.
The consistency between $A_G$ and $A_\Gamma$ indicates that $2\Gamma$ characterizes the non-equilibrium nature of the fracture.
Specifically, for $\thetaStep<19.11^\circ$, the value $A_G=0.959$ fitted from $\thetaStep-\thetaZZ$ relation matches well with $A_\Gamma$ (Fig.\,\ref{Fig_2}b).
For $\thetaStep>19.11$, the fitting result of $A_G=0.93$ is slightly smaller than $A_\Gamma=0.959$, which is attributed to the nature of asymmetric fracture where cracks advancing along the armchair motif prefer to deflect into the zigzag directions (Fig.\,\ref{Fig_3}b).
The relations between $\thetaStep$ and $\thetaZZ$ summarized in Fig.\,\ref{Fig_5} show that predictions by assuming $A_G=A_{\gamma}$ do not prefer zigzag edges, while the results using $A_G=A_{\Gamma}$ confirm the experimental results \citep{qu_2022}.
We also find that the energy densities of unrelaxed edges are very close to the measured fracture toughness ($\Gc\left(\theta\right) \approx 2\Gamma\left(\theta\right)$) (Figs.\,\ref{Fig_1}b, \ref{Fig_2}b and \ref{Fig_3}), although the strain states at the crack tip are different from that in lattice decohesion \cite{van2004thermodynamics}.
%It should be noted here that although not involved in the processes of bond breakage, post-fracture events such as phonon excitation \cite{lawn_2004}, defects evolution, and phase transformation \cite{apte_2018, ly_2017}, as well as the inertial effects, can regulate the fracture toughness by modifying the crack-tip stress field, which was reported in the literature \cite{lawn_2004,apte_2018, ly_2017} but not observed in our simulations.

% Figure environment removed

%\textit{Discussion.}-
%Our results show that the $\thetaStep-\thetaZZ$ relation, $\left(\Ktwo/\Kone\right)_{\rm c}$, and $\Gamma\left(\theta\right)$ offer reasonable measures of the fracture anisotropy, that are, $A_G=0.96$, $A_G = 0.935-0.966$, and $A_\Gamma = 0.959$, respectively.
%The first two measures can be obtained from experiments or simulations with the atomic-level resolution, while the third one can be considered as material parameters and determined by first-principles calculations.
%The results of $A_G$ extracted from $\left(\Ktwo/\Kone\right)_{\rm c}$ in the MD simulations show data dispersion (Fig.\,\ref{Fig_3}g and Table\,\ref{Tab_1}), which can be attributed to the method used to extract the SIFs (see Supplemental Material\,\cite{supp_info}, Note\,S4).
%The domain size chosen in the procedure affects the value of $\Ktwo/\Kone$ and thus $A_G$ because of the nonlinearity of deformation and discrete nature of lattices (Fig.\,S8\,\cite{supp_info}). 
%Considering this effect, we use the same size of domains for analysis.
%Additionally, the Williams power expansion used for fitting is the solution to straight cracks and may not be accurate for cracks with kinks. 
%As shown in Table\,\ref{Tab_1} (Fig.\,S7\,\cite{supp_info}), the value of $\left(\Ktwo/\Kone\right)_{\rm c}$ increases with the density of kinks (Fig.\,\ref{Fig_3}b), from $0.195-0.200$ for $\thetaZZ=9.82^\circ$ ($2$ kinks for the sample with width $W \approx 50$ nm) to $0.205-0.210$ for $\thetaZZ=10.89^\circ$ ($13$ kinks).
%However, determination of $\left(\Ktwo/\Kone\right)_{\rm c}$ requires atomistic simulations of the fracture process that may suffer from size and geometrical effects in practice \cite{buehler_2010}, as well as the methods used to extract the SIFs, while the $\thetaStep-\thetaZZ$ relation can be computed directly from the fracture patterns using Eq.\,\ref{Eq_2}.
%The energies of unrelaxed edges ($\Gamma\left(\theta\right)$) are directly determined from the decohesion calculations.
%The intermediate state of the crack tip is between unrelaxed and fully relaxed, and thus only a range of the fracture anisotropy can be estimated from $\Gamma\left(\theta\right)$.

\textit{Conclusion}-
Using a high-fidelity neural network-based force field developed in this work, we find that the kinetics of fracture is much determined by the intermediate, unrelaxed states of the crack tip.
The energy density of relaxed edges widely used in the literature fails to offer a reasonable measure of fracture toughness.
Instead, the $\thetaStep-\thetaZZ$ relation, $\left(\Ktwo/\Kone\right)_{\rm c}$, and $\Gamma\left(\theta\right)$ offer reasonable measures of the fracture anisotropy, that are, $A_G=0.96$, $A_G = 0.935-0.966$, and $A_\Gamma = 0.959$ (for graphene), respectively.
The first two measures can be obtained from experiments or simulations with the atomic-level resolution, while the third one can be considered as material parameters and determined by first-principles calculations.
%The multiscale kinetics of fracture in 2D crystals is investigated here by developing a neural network-based force field for fracture using graphene as an example.
%The nature of the highly-strained lattice and undercoordinated edges at the crack tip is included to model the fracture patterns at an accuracy of first-principles, electronic structure-level calculations.
%The deflection of cracks is shown to be controlled by the alternation of stress intensity factors, resulting in armchair and zigzag segments as well as the kinks between them.
%The loading conditions and material anisotropy thus determine the crack paths and edge roughness.
%We find that in contrast to the energy density of relaxed edges that is widely used in the literature as a measure of fracture toughness, the kinetics is much determined by the intermediate, unrelaxed states of the crack tip.
%The fracture toughness and its anisotropy can then be measured from the fracture patterns, the crack-tip stress intensity factors, or the energy densities of unrelaxed edges.
%The values extracted from our simulation results agree well with the weak fracture anisotropy reported in recent experimental measurements \cite{qu_2022,feng_2022}.
This work highlights the multiscale and non-equilibrium nature of the fracture and the theory and methodology developed for graphene are extended to other 2D crystals such as h-BN and MoS$_{2}$ (Figs.\,\ref{Fig_4}b-\ref{Fig_4}e)~\cite{supp_info}.
%and S9).

This study was supported by the National Natural Science Foundation of China
through grants 11825203, 11832010, 11921002, and 52090032.
The
computation was performed on the Explorer 100 cluster system of the Tsinghua National Laboratory for Information Science and Technology.

\bibliography{apssamp}% Produces the bibliography via BibTeX.

\end{document}
%
% ****** End of file apssamp.tex ******