\documentclass[12pt,a4paper]{amsart}
\usepackage{amssymb,amsmath,amsfonts,amsthm,mathrsfs}
\usepackage[dvipsnames]{xcolor}
\usepackage{longtable,booktabs,colortbl,multicol,tikz,enumitem}
%\usepackage{color}

\usepackage[
    backend=biber,
    style=alphabetic,
 %   maxcitenames=5,
 %   maxbibnames=9,
 %   sorting=nyt,
 %   sortlocale=de_DE,
 %   natbib=true,
 %   url=false, 
 %   doi=true,
 %   eprint=true
]{biblatex}
\addbibresource{lotbib.bib}
\renewbibmacro{in:}{}
\usepackage{graphicx}

\usepackage{float}

\graphicspath{ {./images/} }

\usepackage{fancyvrb}
\usepackage{comment}
\usepackage{appendix}
%\usepackage{geometry}

%\usepackage{cite}

\input{prologlst.tex}
\definecolor{OliveGreen}{rgb}{0.8,0.83,0.6}
\definecolor{burntumber}{rgb}{0.54, 0.2, 0.14}
\definecolor{coolblack}{rgb}{0.0, 0.18, 0.39}
\definecolor{darkterracotta}{rgb}{0.8, 0.31, 0.36}
\definecolor{frenchbeige}{rgb}{0.65, 0.48, 0.36}
\usepackage{hyperref}
\hypersetup{colorlinks=true,
linkcolor=NavyBlue,
    filecolor=OliveGreen,      
    urlcolor=burntumber,
    citecolor=darkterracotta,
    anchorcolor=frenchbeige}

    \usepackage{letltxmacro}
    \newcommand*{\SavedLstInline}{}
    \LetLtxMacro\SavedLstInline\lstinline
    \DeclareRobustCommand*{\lstinline}{%
      \ifmmode
        \let\SavedBGroup\bgroup
        \def\bgroup{%
          \let\bgroup\SavedBGroup
          \hbox\bgroup
        }%
      \fi
      \SavedLstInline
    }

\usetikzlibrary{calc}



\newcommand{\red}{\textcolor{red}}
\newcommand{\green}{\textcolor{OliveGreen}}
\newcommand{\blue}{\textcolor{blue}}
\newcommand{\eps}{\varepsilon}
\newcommand{\R}{\mathbb{R}}
\newcommand{\del}{\partial}
\renewcommand{\sl}{\mathfrak{sl}}
\newcommand{\psl}{\mathfrak{psl}}
\newcommand{\OO}{\mathcal{O}}
\newcommand{\CC}{\mathbb{C}}
\newcommand{\Z}{\mathbb{Z}}
\newcommand{\N}{\mathbb{N}}
\newcommand{\F}{\mathbb{F}}
\newcommand{\ud}{\mathrm{d}}
\newcommand{\DD}{\mathscr{D}}
\newcommand{\TW}{\mathrm{TW}}
\newcommand{\WT}{\mathrm{WT}}
\newcommand{\W}{\mathcal{W}}
\newcommand{\E}{\mathfrak{E}}
\newcommand{\BB}{\mathscr{B}}
\renewcommand{\O}{\mathcal{O}}
\newcommand{\lst}{\lstinline}

\DeclareMathOperator{\GF}{GF}
\DeclareMathOperator{\GL}{GL}
\DeclareMathOperator{\im}{im}
\DeclareMathOperator{\ad}{ad}
\DeclareMathOperator{\soc}{soc}
\DeclareMathOperator{\Der}{Der}
\DeclareMathOperator{\Hom}{Hom}
\DeclareMathOperator{\End}{End}
\DeclareMathOperator{\efc}{efc}
\DeclareMathOperator{\Char}{char}
\DeclareMathOperator{\cost}{cost}


%\usepackage[backend=biber, style=alphabetic, maxcitenames=5, maxbibnames=9,sorting=nyt]{biblatex}
%\addbibresource{References-new.bib}




\newcommand{\FF} {\mathbb{F} }

%\lstdefinestyle{mystyle}{
%	backgroundcolor=\color{backcolour},   
%	commentstyle=\color{codegreen},
%	keywordstyle=\color{magenta},
%	numberstyle=\tiny\color{codegray},
%	stringstyle=\color{codepurple},
%	basicstyle=\ttfamily\footnotesize,
%	breakatwhitespace=false,         
%	breaklines=true,                 
%	captionpos=b,                    
%	keepspaces=true,                 
%	numbers=left,                    
%	numbersep=5pt,                  
%	showspaces=false,                
%	showstringspaces=false,
%	showtabs=false,                  
%	tabsize=2
%}
%
%\lstset{style=mystyle} 

\lstset{%
	tabsize=4
}

\usepackage[capitalise]{cleveref}
%\thm`style{plain}
\newtheorem{thm}{Theorem}[section]
\newtheorem{prop}[thm]{Proposition}
\newtheorem{cor}[thm]{Corollary}
\newtheorem{lemma}[thm]{Lemma}
\crefname{mainthm}{thm}{Main Theorem}
\newtheorem*{mainthm}{Main Theorem}
\theoremstyle{definition}
\newtheorem{defin}[thm]{Definition}
\newtheorem{ex}[thm]{Example}
\newtheorem{conj}[thm]{Conjecture}
\newtheorem{question}[thm]{Question}
\theoremstyle{remark}
\newtheorem{rmk}[thm]{Remark}
\newtheorem{rmks}[thm]{Remarks}



\setlength\parindent{0pt}
\setlength\parskip{5pt}

\setlength{\textwidth}{\paperwidth}
\addtolength{\textwidth}{-2in}
\setlength{\textheight}{\paperheight}
\addtolength{\textheight}{-2in}
\calclayout 

%\DeclareTextFontCommand{\emph}{\bfseries\itshape}
\newcommand{\degg}{{\rm deg}}
\newcommand{\reg}{{\rm reg}}
\newcommand{\gl}{{\mathfrak{gl}}}
\newcommand{\fsl}{{\mathfrak{fsl}}}
\newcommand{\s}{{\mathfrak{s}}}
\newcommand{\diag}{\operatorname{diag}}
\newcommand{\gws}[1]{\textcolor{red}{#1}}
\newcommand{\B}{\mathcal{B}}
\newcommand{\C}{\mathcal{C}}

%\title[Minimal Lottery designs]{ Calculation of minimal lottery designs through constraint programming}
\title[Minimal Lottery designs]{You need 27 tickets to guarantee a win on the UK National lottery}
%\date{June 2023}

    \author[Cushing]{David Cushing}
    \address{Department of Mathematics, The University of Manchester, Manchester, UK}
    \email{david.cushing@manchester.ac.uk}
    
    \author[Stewart]{David I. Stewart}
    \address{Department of Mathematics, The University of Manchester, Manchester, UK}
    \email{david.i.stewart@manchester.ac.uk}

\begin{document}

\begin{abstract}
    This paper develops and deploys a set of constraints for the purpose of calculating minimal sizes of lottery designs, with an application to the UK National Lottery: in case there are $n=59$ balls, the minimum number of tickets of size six which are needed to match at least two balls on any draw of size six is precisely $27$. In fact, we calculate this minimum value for all $n\leq 70$.
    \end{abstract}
    
    
    \maketitle
    
    \section{Introduction}
    An $(n,k,p,t;j)$-lottery design is a hypergraph $H=(X,\B)$ where $X$ denotes a set of vertices of order $n$, and $\B$ a set of subsets of $X$ with size $|\B|=j$; futhermore, $H$ should be $k$-uniform---that is, $|B|=k$ for any $B\in \B$---and for any subset $D$ of $X$ of order $p$, we have $|D\cap B|\geq t$ for some $B\in \B$. More colloquially, $X$ is a set of balls labelled $1$ to $n$, and $\B$ is a set of $j$ tickets each containing $k$ of these numbers, such that for any draw $D$ of $p$ distinct balls from $X$, we can find at least one ticket $B\in \B$ which matches $D$ in at least $t$ places. Mostly because of a consequence of its application to the UK national lottery, this paper is concerned with finding the minimum value $L(n,k,p,t)$ of $j$ when $k=p=6$ and $t=2$. 
    
    In \cite{BateRees}, the numbers $L(n,6,6,2)$ are calculated for $n\leq 54$. We bolster the techniques of \textit{op.~cit.~}both theoretically and computationally to get:
    
    \begin{mainthm}\label{mainthm}
        The values of $L(n,6,6,2)$ for $n\leq 70$ are listed in  \cref{theconfigs}.
    \end{mainthm}
    A consequence of our theorem is the easy generation of an explicit $(n,6,6,2;j)$-lottery design with $j$ minimal as a disjoint union of covering designs---see \cref{sec:uppbound} for the definition. Assuming the sizes of  appropriate covering designs are known, one easily gets from this upper bounds for the minimal sizes of lottery designs; but ruling out the possibility of smaller designs is where the real difficulty lies.
    \begin{cor}For $32\leq n\leq 70$, there always exists a $(n,6,6,2;j)$-lottery design with $j$ minimal which is a disjoint union of five $(a_i,6,2)$-covering designs. Hence there exist $\{a_1,\dots,a_5\}$ with $\sum a_i=n$  and
        \[L(n,6,6,2)=\sum_i C(a_i,6,2).\]
    \end{cor}
    Viable choices of covering designs are listed in \cref{sec:configs}.
        
    Perhaps of most interest here is that we make substantial use of the constraint programming library \cite{COC97} in SICStus Prolog\footnote{A free evaluation copy of SICStus can be downloaded from \url{https://sicstus.sics.se/}} \cite{sicstus}. In that respect, this paper continues a programme of work the authors began in \cite{CSS} to apply Prolog and constraint programming to questions arising in pure mathematics. We hope the pure mathematics community might also see the potential of Prolog; while there is a steep learning curve, Prolog ultimately offers one of the cleanest interfaces for constraint programming. (See \textit{op.~cit.}~for further discussion of Prolog.) 
    
    Let us say something about the proof (during which we also put \cite{BateRees} on a more rigorous footing). 
    The Handbook of Combindatorial Designs \cite{Hand} gives the following theorem as a highlight in its section on lottery designs, which furnishes us with a highly useful lower bound.
    
    \begin{thm}[Furedi--Sz\'ekely--Zubor]\label{FurediBound} We have 
        \[L(n, k, p, 2) \geq \frac{1}{k}\cdot\min_{\sum_{i=1}^{p-1} a_{i} = n }\left(  \sum_{i=1}^{p-1} a_{i}\left\lceil \frac{a_{i} - 1}{k-1} \right\rceil\right).\]
    \end{thm}

    Often, this lower bound for $L(n,6,6,2)$ can be achieved through covering designs and $L(n,6,6,2)$ is known. Where it cannot, we need to show that the upper bound $j$ we get from covering designs cannot be improved. So we assume that there exists a lottery design with $j-1$ blocks, with a view towards a contradiction. The non-existence of a draw of $6$ numbers intersecting all blocks in at most $1$ place is equivalent to the assertion about a lottery design that all  independent sets of vertices have order at most $5$. We can control the degrees of elements in $I$, which allows us to focus on the local data $\B_I$ where $\B_I$ denotes the set of all blocks meeting $I$. The consequent deliberations generate a fairly complex system of constraints which we implement in Prolog. It turns out that these constraints all lead to contradictions for $n\leq 70$.
    
    \begin{rmk}Let us also describe the more mundane concern that inspired the paper. In October 2015, the UK national lottery (where $k=p=6$) underwent a change in which the number of balls  $n$ was increased from $49$ to $59$. By way of compensation, a `lucky dip' prize was introduced for matching two balls on a ticket. Consequently, the value of $L(59,6,6,2)$---namely $27$ by our main theorem---is somewhat significant. For the interested reader, \cref{27tix} also lists a minimal set of tickets explicitly.
    
    Having observed that the set of tickets we describe below would have netted the authors \pounds 1810 in the lottery draw of 21 June 2023, the authors were motivated to road-test the tickets in the lottery draw of 1 July 2023; they matched just two balls on three of the tickets, the reward being three lucky dip tries on a subsequent lottery, each of which came to nothing. Since a ticket costs \pounds 2, the experiment represented a loss to the authors of \pounds 54. This unfortunate incident therefore serves both as a verification of our result and of the principle that one should expect to lose money when gambling. 
    
    For a more philosophical discussion of the National Lottery and its implementation for supporting charitable causes, we recommend David Runciman's article in the London Review of Books \cite{runc}.\end{rmk}
    



\section{Definitions and notation} 
A \emph{hypergraph} $H$ is a pair $(X,\B)$ with $X$ a set and $\B$ a set of subsets of $X$. We will refer to the elements $x\in X$ as \emph{vertices} and the elements $B\in\B$ as \emph{blocks}. The \emph{order} of $H$ is the cardinality of the set $X$. The \emph{size} of $H$ is the number of blocks;~i.e. the cardinality of the set $\B$. A hypergraph is said to be \emph{$k$-uniform} if each $B\in\B$ has order $k$. Note that a $2$-uniform hypergraph is a graph: a block $B=\{x,y\}$ identifies with an edge $x-y$.

\begin{defin}An \emph{$(n,k,p,t;j)$-lottery design} is a $k$-uniform hypergraph $H=(X,\B)$ of order $n$ and size $j$ such that  for any subset $D\subseteq X$ with $|D|=p$, there is at least one $B\in \B$ with $|B\cap D|\geq t$.

An \emph{$(n,k,p,t)$-lottery design} is an $(n,k,p,t;j)$-lottery design in which $j$ is minimal; we denote this minimal integer by $L(n,k,p,t)$.\end{defin}

\begin{rmk}Thinking of $X$ as a set of balls, $\B$ as a set of tickets, and $D$ as a draw, the definition says that in any draw of $p$ balls, at least one ticket matches at least $t$ of the balls in $D$.\end{rmk}

In order to avoid vacuous or incorrect statements we shall always insist $n\geq k\geq t\geq 2$.

We say a block $B$ is an \emph{$x$-block}---or that $B$ is \emph{incident} with a vertex $x$---if $x\in B$. The set $\B_x$ of all $x$-blocks is the \emph{star} of $x$. We define the function \[d:X\to\Z_{\geq 0};\ x\mapsto|\B_x|\] so that $d(x)$ is the \emph{degree} of $x$; i.e.~the number $|\B_x|$ of blocks incident with $x$. More generally, if $I$ is any subset of the vertices, then $B$ is an $I$-block if it is an $x$-block for some $x\in I$. We let $\B_I=\bigcup_{x\in I} \B_x$, and $d(I)=\bigcup_{x\in I} d(x)$ its degree. We will denote by $d_i:=|d^{-1}(i)|$ the number of vertices of degree $i$ in $H$. We also need to analyse the multiset of degrees of vertices in a subset $I\subseteq X$, so we define the function from the power set $\mathcal{P}(X)$ of $X$ to non-decreasing sequences of non-negative integers: \[\delta : \mathcal{P}(X)\to \bigcup_{r=0}^n(\Z_{\geq 0})^r;\ I\mapsto (d(x_1),\dots,d(x_n)),\text{ such that } d(x_1)\leq d(x_2)\leq \dots\leq d(x_n).\]
It will be convenient to assume that the elements of $I$ are always listed in order of their degree, so for example we may have $I=\{x_1,\dots,x_5\}$ such that $\delta(I)=(2,2,3,3,4)$ indicating $d(x_1)=2,\dots,d(x_5)=4$. 

Two vertices $x,y\in X$ are \emph{adjacent} if they are contained in a common block. The set of all vertices adjacent to $x$ is its \emph{neighbourhood} $N(x)$. A subset $I$ of $X$ is an \emph{independent set} or \emph{coclique} if no pair of elements in $I$ are adjacent. An independent set is \emph{maximal} if there is no independent set $J\subseteq X$ with $I\subsetneq J$.

\begin{rmks}\label{indsetrems}(i) If $I$ is an independent set in an $(n, k, p, t; x)$-lottery design, then $|I|<p$ or else there would be a draw $D=I$ matching at most $1$ vertex in any block (noting $t>1$). 
    
(ii) If $I$ is a maximal independent set then we must have $\B_I=X$ or else there is another element $y\in X$ which is not adjacent to any $x\in I$; which would imply there were a larger maximal independent set $X\cup \{y\}$.

(iii) It is easy to see that maximal independent sets need not have the same order. For example, in the graph $x-y-z$, $\{y\}$ is maximal independent, but $\{x,z\}$ is a  maximal independent set of higher cardinality. \end{rmks}

A set $\B'\subseteq\B$ of blocks is \emph{disjoint} if $B_1\cap B_2=\emptyset$ for any $B_1,B_2\in\B'$. In the literature, such a $\B'$ is also referred to as a \emph{matching}. A block $B\in \B$ is \emph{isolated} if $B$ is disjoint from all other blocks in $\B$. (Of course, each vertex of an isolated block has degree $1$.) If $d_0=0$ and each vertex of degree $1$ is contained in an isolated block, then we say $H$ is \emph{segregated}.

We will show later that $H$ may be assumed segregated under the hypotheses of our main theorem. In that case, the interesting analysis reduces to the full subhypergraph induced by the vertices of degree at least $2$. With that in mind, we give a further couple of definitions which will prove central to our argument. Let $I$ be an independent set. Then a \emph{toe}, or more explicitly \emph{$I$-toe} is a vertex $z$ of degree at least $2$ appearing in just one block of $\B_I$. For $x\in I$, we say $z$ is an $x$-toe if it is adjacent to $x$; the set of all $x$-toes is denoted $F_x$, and the union $F_I:=\bigcup_{x\in I} F_x$ is the \emph{foot} or \emph{$I$-foot}. (Note that an $\{x\}$-toe may fail to be an $I$-toe for $\{x\}\subsetneq I$.) Suppose $I=\{x_1,\dots,x_\ell\}$, with $\delta(I)=(d(x_1),\dots,d(x_\ell))$, where $d(x_1)>1$. Then set 
\[\tau(I)=\left(\left|F_I\cap \bigcup\B_{x_1}\right|,\dots,\left|F_I\cap\bigcup\B_{x_\ell}\right|\right),\]
which denotes the distribution of toes among the $x$-blocks as $x$ ranges over the vertices in $I$. 

\section{Preliminaries}
We use this section to collect a miscellany of results on hypergraphs and lottery designs of varying generality, which we use in the sequel. Our strategy is heavily influenced by the paper \cite{BateRees} and we use most of its results in one form or another. However, we noticed a number of infelicities among the statements and proofs in \emph{op.~cit.}~and so we take the opportunity here to correct and simplify  the statements and proofs therein. Happily, it follows this paper is largely self-contained. 

\subsection{Upper bounds}\label{sec:uppbound} An \emph{$(n,k,t)$-covering design} is a $k$-uniform hypergraph $H=(X,\B)$ such that every subset of $X$ of size $t$ appears as a subset in at least one block of $\B$; we assume $n\geq k\geq t$. Define $C(n,k,t)$ as the minimal size of an $(n,k,t)$-covering design. Obviously, an $(n,k,t)$-covering design is equivalently an $(n,k,t,t)$-lottery design, so $L(n,k,p,t)\leq C(n,k,t)$. 
\begin{lemma}\label{upperbound} We have 
    \[L(n, k, p, 2) \leq \min_{\sum_{i=1}^{p-1} a_{i} = n}\left(  \sum_{i=1}^{p-1} C( a_{i}, k, 2) \right).\]
\end{lemma}
\begin{proof} Let $H_i=(X_i,\B_i)$ be $(a_i,k,2)$-covering designs of size $C(a_i,k,2)$ such that $X=\bigsqcup_{i=1}^{p-1}X_i$, and let $D$ be a draw of $p$ vertices. Then at least two vertices of $D$ lie in at least one $X_i$, and there is a block $B$ of $\B_i$ containing those two vertices.\end{proof} 

\cref{upperbound} can be deployed using the entries from \cref{coveringnumbers}. The table lists upper bounds for $C(n,6,2)$ for $n\geq 6$ which were harvested from \url{https://ljcr.dmgordon.org/cover.php}; these upper bounds are all known to be sharp except when $n=23$ or $24$.
\begin{table}
\begin{tabular}{|c|c||c|c|}\hline
    $n$ & $C(n,6,2)\leq$ & $n$ & $C(n,6,2)\leq$\\\hline
    $6$ & $1$ & $17$ & $12$\\
    $7$ & $3$ & $18$ & $12$\\
    $8$ & $3$ & $19$ & $15$\\
    $9$ & $3$ & $20$ & $16$\\
    $10$ & $4$ & $21$ & $17$\\
    $11$ & $6$ & $22$ & $19$\\
    $12$ & $6$ & $23$ & $21$\\
    $13$ & $7$ & $24$ & $22$\\
    $14$ & $7$ & $25$ & $23$\\
    $15$ & $10$ & $26$ & $24$\\
    $16$ & $10$ & $27$ & $27$\\
    \hline
\end{tabular}\caption{Upper bounds for $C(n,6,2)$}\label{coveringnumbers}
\end{table}

Secondly, it is useful to know that lottery numbers increase mononotonically with $n$.

\begin{lemma}\label{doesntgodown}
    We have $L(n,k,p,t)\leq L(n+1,k,p,t)$.
\end{lemma}
\begin{proof}
    Take an $(n+1,k,p,t;j)$-lottery design $H=(X,\B)$ with $|\B|=j$. Pick $x\in X$ and construct $j$ subsets $\C$ of $X\setminus\{x\}$ from those of $\B$ by replacing $x\in B$ with any other vertex of $X\setminus B$, where necessary. Then it is clear that the hypergraph $H_0=(X\setminus \{x\},\C)$ is an $(n,k,p,t;j)$-lottery design.
\end{proof}


    
\subsection{Reductions and constraints}

\begin{lemma}\label{BR1}
    Let $H=(X,\B)$ be an $(n, k, p, t; j)$-lottery design  with $j \geq n/k$. Then there exists an $(n, k, p, t; j)$-lottery design $H_0=(X,\B_0)$ with $\bigcup\B_0=X$; i.e.~there are no elements of degree $0$.
        \end{lemma}
        \begin{proof}Let $H$ be a counterexample with $d_0>0$ minimal. Suppose $x\in X$ has degree $0$. Since $jk\geq n$, there must be $y\in X$ with $d(y)\geq 2$. Suppose $y\in B$ for some block $B$ of $\B$ and set $B_0=(B\setminus\{y\})\cup \{x\}$. Now set $H_0=(X,\B_0)$, where $\B_0$ is $\B$ with the block $B$ replaced by $B_0$. Then $H_0$ is a $k$-uniform hypergraph of order $n$ and size $j$. If $D$ is any subset of $X$ of order $p$, then either there is $C\in\B_0$ with $|C\cap D|\geq t$ or we may assume $|B\cap D|=t$, $y\in D$ and $|B_0\cap D|=t-1$. This implies also $x\not \in D$; but then the alternative draw $(D\setminus\{y\})\cup\{x\}$ cannot intersect a block of $\B$ in $t$ elements.\end{proof}
\begin{lemma}\label{segregated}
If $H=(X,\B)$ is an $(n, k, p, t; j)$-lottery design with $n \geq k(p-1)$ and $d_0=0$, then there exists a segregated $(n, k, p, t; j)$-lottery design.
\end{lemma}
\begin{proof}
%We need to show exhibit a lottery design with $d_1=rk$. 
Suppose there are $r$ isolated blocks in $H$. Then taking one vertex from each yields an independent set $I$, so $r\leq p-1$ by \cref{indsetrems}(i). If $r=p-1$ then the isolated blocks supply all $kr=k(p-1)$ vertices of $X$ and the statement holds. 

Suppose $r=p-2$; then there are $n-k(p-2)$ elements not in isolated blocks. A draw of order $p$ containing one vertex from each isolated block together with two non-isolated vertices can only intersect a non-isolated block in at least $t$ places. Thus the non-isolated blocks must between them contain every pair of the non-isolated vertices. This means that each appears at least twice, or $n-k(p-2)=2$ and they both appear exactly once. But the latter says they are themselves in an isolated block, a contradiction.

Hence we may assume $r\leq p-3$, leaving at least $2k$ elements not in isolated blocks. Let $B$ be a non-isolated block and assume that there are $x,y\in B$ with $d(x)=1$ and $d(y)>1$. We modify $H$ to give a lottery design $H_0$ with $d(y)=1$. By an evident induction, this implies the existence of the required design.

Let $B=B_1,\dots,B_\ell$ be the blocks containing $y$. For $2\leq i\leq \ell$, find a non-isolated element $z_i$ such that $z_i\not\in B_i$; this is possible since there are at least $2k$ non-isolated elements. Form $C_i$ by replacing $y$ with $z_i$ and let $C_1=B_1$. Then we claim we get a new $(n,k,p,t;j)$-lottery design $H_0=(X,\C)$ with $d_0=0$ by letting $\C=(\B\setminus\{B_1,\dots,B_\ell\})\cup\{C_1,\dots,C_\ell\}$.

To prove the claim, take a draw $D$ and assume $D$ does not intersect any block $C\in\C$ in at least $t$ elements. Then we may assume $D$ contains $y$, $|D\cap B_i|=t$ for some $2\leq i \leq l$ and $|D\cap C_i|=t-1$.  Furthermore, since $d(x)=1$, if $x\in D$, then $D\cap B_1=D\cap C_1$ has at least $t$ elements, so we may assume $x\not\in D$. Then replacing $y$ with $x$ in $D$ gives a draw $D_0$ which intersects no block of $\B$ in at least $t$ elements.
\end{proof}


The above results are essentially the same as \cite[Lem.~3.2, Thm.~3.5]{BateRees}. In between is \cite[Lem.~3.4]{BateRees} which shows (correctly) that given an $(n, k, p, 2)$-lottery design with $n>k(p-2)$, then there is another with a maximal independent set of size $p-1$. It is combined with the above two results to claim the existence of a lottery design satisfying the conclusions of all three results. Unfortunately the methods of proof go by altering the degrees of vertices in the design, and it is unclear whether this can be done compatibly. There is a further issue in the proof of \cite[Prop.~4.6]{BateRees} where it incorrectly assumed that an independent set $I$ with $d_2(I)$ maximal can be extended to an independent set of maximal cardinality.

We resolve these issue in \cref{1s2s3s} below, with a stronger result. As in \cite{BateRees}, we make use of Shannon's bound \cite{Sha49} on the chromatic number of a graph, though we get better results by using a sharpness result due to Vizing. (Both Shannon's and Vizing's theorems are given a good exposition in \cite{scheide}.)

Let $G=(V,E)$ be a finite undirected graph possibly with multiple edges between distinct vertices but with no loops. A \emph{$c$-edge colouring} of $G$ is an assignment of a colour to each edge in $G$ such that no two adjacent edges have the same colour and at most $c$ different colours are used. The \emph{chromatic index} $\chi'(G)$ of $G$ is the smallest integer $c$ such that $G$ admits a $c$-edge colouring. From the definition it follows immediately that the maximum degree $\Delta(G)$ is a lower bound for the chromatic index. In fact,
\begin{thm}[Shannon's bound]\label{shannonbound} We have $\chi'(G)\leq \left\lfloor \frac{3}{2}\Delta(G)\right\rfloor$.\end{thm}

For $x,y\in V$, let $\mu_{G}(x,y)$ denote the number of edges between $x$ and $y$. Then a \emph{Shannon graph} of degree $d\geq 2$ is a graph $G$ consisting of three vertices $x,y,z$ such that $\mu_{G}(x,y)=\mu_{G}(y,z)=\left\lfloor\frac{d}{2}\right\rfloor$ and $\mu_{G}(x,z)= \left\lfloor\frac{d+1}{2}\right\rfloor$. For a Shannon graph of degree $d$ we have $\Delta(G)=d$ and $\chi'(H)=\left\lfloor\frac{3}{2}d\right\rfloor$, so 
that the bound in \cref{shannonbound} is sharp. More generally:

\begin{thm}[Vizing]\label{Vizing} Suppose $\chi'(G)=\left\lfloor \frac{3}{2}\Delta(G)\right\rfloor$ where $\Delta=\Delta(G)\geq 4$. Then $G$ contains a Shannon graph of degree $\Delta$ as a subgraph.\end{thm}

For our situation we need a dual version for hypergraphs. Let $H=(X,\B)$ be a hypergraph with blocks of order at most $k$ and let $H_0=(X_0,\B_0)$ of $H$ be a subhypergraph. We say $H_0$ is \emph{$k$-Shannon} if $\B_0=\{B_1,B_2,B_3\}$, $X_0=B_1\cup B_2\cup B_3$, there is at most $1$ vertex in $X_0$ of degree not $2$, and where $|B_1\cap B_2| = |B_2\cap B_3|=\left\lfloor\frac{k}{2}\right\rfloor$ and $|B_1\cap B_3|=\left\lfloor\frac{k+1}{2}\right\rfloor$. If $k=2m$ then this means there are $3m$ vertices in $X_0$ and $H=H_0\sqcup H_1$ is a disjoint union of $H_0$ with another subhypergraph $H_1$. If $k=2m+1$ then there are $3m+1$ vertices of degree $2$ and at most one further vertex $v$, where only $v$ may appear in blocks outside of $\B_0$.

\begin{prop} \label{ninthdeg2}
    Let $H=(X,\B)$ be a hypergraph whose blocks $B\in \B$ are of order at most $k$ and suppose $H$ contains precisely $s$ distinct $k$-Shannon subhypergraphs. Then there exists an independent set $I$ containing 
    \[s+\left\lceil\frac{d_2-\left\lfloor\frac{3k+1}{2}\right\rfloor s}{\left\lfloor\frac{3k}{2}\right\rfloor-1} \right\rceil\]
    vertices of degree $2$.
\end{prop}
\begin{proof}
    Let $H_0=(X_0,\B_0)$ be the subhypergraph of $H$ induced by the vertices not contained in $k$-Shannon subhypergraphs. The sizes of the blocks of $H_0$ are still at most of size $k$. Furthermore $H_0$ contains no $k$-Shannon subhypergraphs. 
    
    Form a graph $G$ whose vertex set is $\B_0$ and there is an edge between $B_1$ and $B_2$ for each element $x\in B_1\cap B_2$ of degree $2$. Note that $G$ has at most $d_2-\left\lfloor\frac{3k+1}{2}\right\rfloor s$ edges, and contains no Shannon subgraphs of degree $k$. Suppose $\Delta(G_0)\leq 3$. Then \cref{shannonbound} implies the existence of a $c$-colouring with $c\leq 4\leq \left\lfloor\frac{3k}{2}\right\rfloor-1$. Otherwise we may apply \cref{Vizing} to find a $c$-colouring with $c\leq \left\lfloor \frac{3}{2}k\right\rfloor-1$. 
    
    So in any case, there must exist a set of edges with size at least $$\left\lceil\frac{d_2-\left\lfloor\frac{3k+1}{2}\right\rfloor s}{c}\right\rceil \geq \left\lceil\frac{d_2-\left\lfloor\frac{3k+1}{2}\right\rfloor s}{\left\lfloor\frac{3k}{2}\right\rfloor-1} \right\rceil$$ having the same colour. Any monochromatic set of $m$ edges of $G$ represent $m$ independent vertices of $H_0$ of degree $2$.
    
    Observe that every $k$-Shannon subhypergraph $S$ of $H$ has at least one vertex of degree $2$ which is not adjacent to any vertex outside of $S$; adding these $s$ vertices gives the lower bound in the theorem.
\end{proof}

An elementary rearrangement of the bound above yields the following.

\begin{cor}\label{shanrearrangement}With the hypotheses of the proposition, let $\delta_2$ be the maximum number of independent vertices in $H$ of degree $2$ and let $k=2m+\rho$ with $\rho\in \{0,1\}$. Then 
    \[d_2\leq \delta_2\left(\left\lfloor\frac{3k}{2}\right\rfloor-1\right) + (1 + \rho)s\]
\end{cor}
    


\begin{prop}\label{1s2s3s} Let $H=(X,\B)$ be a segregated $k$-uniform hypergraph with $|X|=n$, $|\B|=j$, $d_1=kr$, $k=2m+\rho$ and $\rho\in\{0,1\}$. Suppose $H$ contains $s$ distinct $k$-Shannon subhypergraphs. Then there is an independent set $I_0\subseteq X$ containing $r$ vertices of degree $1$ and \[\kappa:=s+\left\lceil\frac{d_2-\left\lfloor\frac{3k+1}{2}\right\rfloor s}{\left\lfloor\frac{3k}{2}\right\rfloor-1} \right\rceil\] vertices of degree $2$. Furthermore, $I_0$ can be extended to an independent set $I\subseteq X$ of order $\pi\geq\kappa+r$ whose vertices have degree at most $3$, provided 
    \[\tag{*} 4n-jk-\left(2 -m\right) s -(3k+m-2)(\pi-1)+(m-2)r>0.\]

% Furthermore if the left-hand side of the inequality (*) is instead equal to zero then there is an independent set $I$ of order $\pi-1$ with $d(I)\in\{1,2\}$, all vertices have degree at most $4$, and 

% \[d_2=(\pi-1-r)\left(\left\lfloor\frac{3k}{2}\right\rfloor-1\right) + (1 + \rho)s.\]
\end{prop}
Before giving the proof of the proposition, let us make some elementary observations that are used several times in the sequel. 

Let $H=(X,\B)$ be a $k$-uniform hypergraph with $|X|=n$ and $|\B|=j$. Recall $d_i$ is the number of vertices having degree $i$. Clearly

\begin{equation}\label{setsum}
    \sum_{i\geq 0}d_i=n.
\end{equation}

Let $\BB$ denote the multiset $\bigsqcup B_i$, which contains the vertices of the hypergraph counted with their multiplicities in the blocks $\B$. As there are $j$ blocks we have $|\BB|=jk$. Of course a vertex $x$ will occur in exactly $d(x)$ of the blocks, and so: 

\begin{equation}\label{multisetsum}
    \sum_{i\geq 1}id_i=jk.
\end{equation}


\begin{proof}[Proof of \cref{1s2s3s}] The existence of $I_0$ is immediate from \cref{ninthdeg2} and segregation. Moreover, we may assume $s$ of vertices of $I_0$ each live in distinct $k$-Shannon subhypergraphs, whose union accounts for at most $\lfloor\frac{3k+1}{2}\rfloor\cdot s$ vertices.
    Now take $I\supseteq I_0$ maximal subject to $d(I)\subseteq \{1,2,3\}$ and assume for a contradiction that $|I|\leq\pi-1$. let $\delta_i$ denote the number of vertices of $I$ of degree $i$; note that $\delta_1=r$, again by segregation. 
    
    We now bound from above and below the number $d_2+d_3$ of vertices in $X$ having degrees $2$ or $3$. 
    
     Let $J\subseteq I$ denote set of vertices not in isolated blocks. Then $\B_J$ contains at most $$(2k-1)(\delta_2-s)+\left\lfloor\frac{3k}{2}\right\rfloor s+(3k-2)\delta_3$$ distinct elements. Now if there were a vertex of degree $2$ or $3$ not in $\bigcup\B_J$, then $I$ was not maximal. Therefore,
    \begin{align}d_2+d_3&\leq (2k-1)(\delta_2-s)+\left\lfloor\frac{3k}{2}\right\rfloor s+(3k-2)\delta_3,\nonumber\\
        &\leq (2k-1)(\delta_2-s)+\left\lfloor\frac{3k}{2}\right\rfloor s+(3k-2)(\pi-1-r-\delta_2),\nonumber\\
        \notag&\hspace{7cm} \text{since }\delta_1+\delta_2+\delta_3\leq \pi-1,\nonumber\\
        &\leq (1-k)\delta_2+\left(\left\lfloor\frac{3k}{2}\right\rfloor-2k+1 \right)s +(3k-2)(\pi-1-r),\nonumber\\
        &\leq (1-k)\delta_2+\left\lfloor\frac{2-k}{2}\right\rfloor  s +(3k-2)(\pi-1-r)
               %&\leq (1-k)\delta_2+(1-\left\lfloor\frac{k}{2}\right\rfloor)≈ s +(3k-2)(\pi-1-r).
               \label{combine}
    \end{align}
    
    
    On the other hand, consider the multiset $\BB=\bigsqcup_{B\in\B} B$, of order $jk$. The $I$-blocks contribute exactly $\delta_1k+2\delta_2k+3\delta_3k$ of these elements. Since there are $2d_2+3d_3$ elements of degrees $2$ or $3$, there are $2\delta_2k+3\delta_3k-2d_2-3d_3$ elements of $\BB$ of order at least $4$ coming from the $I$-blocks. The remaining $(j-\delta_1-2\delta_2-3\delta_3)$ blocks all consist of elements of order at least $4$. Thus the multiset $|\BB_{\geq 4}|$ of all elements of degrees $4$ and above has order
    \[|\BB_{\geq 4}|=(j-\delta_1-2\delta_2-3\delta_3)k + 2\delta_2k+3\delta_3k-2d_2-3d_3 =jk-rk-2d_2-3d_3.\] 
    Since $|\BB_{\geq 4}|=\sum_{i\geq 4} id_i$, we get 
    \[\sum_{i\geq 4} d_i \leq \frac{jk-rk-2d_2-3d_3}{4}.\]
    So \[n=\sum d_i\leq rk+d_2+d_3+\frac{jk-rk-2d_2-3d_3}{4}.\] 
    
    Rearranging gives $d_2+d_3\geq 4n-jk-d_2-3rk$,   
    which combines with (\ref{combine}) to give 
    \begin{eqnarray*} (1-k)\delta_2+\left\lfloor\frac{2-k}{2}\right\rfloor s +(3k-2)(\pi-1-r) \geq 4n-jk-d_2-3rk;\\
        \text{i.e.}\quad  d_2+(1-k)\delta_2 \geq 4n-jk-\left\lfloor\frac{2-k}{2}\right\rfloor s -(3k-2)(\pi-1)-2r.\end{eqnarray*}
    
    Now using \cref{shanrearrangement} we get
    \begin{align*}
        \delta_2\left(\left\lfloor\frac{3k}{2}\right\rfloor-1\right)& + (1 + \rho)s +(1-k)\delta_2\geq 4n-jk-\left\lfloor\frac{2-k}{2}\right\rfloor s -(3k-2)(\pi-1)-2r\\
        \left\lfloor\frac{k}{2}\right\rfloor\delta_2  &\geq 4n-jk-\left(\left\lfloor\frac{2-\rho}{2}\right\rfloor+ 1 + \rho-m\right) s -(3k-2)(\pi-1)-2r\\
        \left\lfloor\frac{k}{2}\right\rfloor\delta_2  &\geq 4n-jk-\left(2 -m\right) s -(3k-2)(\pi-1)-2r.
    \end{align*}
    Using $\delta_1+\delta_2\leq \pi-1$, with equality if and only if $\delta_3=0$, then
    \[\left\lfloor\frac{k}{2}\right\rfloor(\pi-1-r)  \geq 4n-jk-\left(2 -m\right) s -(3k-2)(\pi-1)-2r;\]
    so
    \[4n-jk-\left(2 -m\right) s -(3k+m-2)(\pi-1)+(m-2)r\leq 0,\]
    which is a contradiction, proving the 
    proposition.
    \end{proof}

%\vspace{10pt}\hrule\vspace{10pt}

% \begin{rmk} Can replace elts of frequency $2$ by elts of frequency $r$ in the above. Get an independent set containing at least $\frac{d_r}{lfloor \frac{3(r-1)k}{2}\rfloor}$ elements. If $r=3$ get $\frac{d_3}{3k}$ elements in the set. 

%     Doing this with elts of degree 2 and 3 we seem to get $\frac{2d_2+3d_3}{18}$. So that gives a lower bound for $\delta_2+\delta_3$. We can also maybe use $\sum i d_i=jk$, or $2d_2+3d_3\leq 2\delta_2+3\delta_3$. \dots\end{rmk}

% \begin{prop}\label{1s2s3s} Suppose $H=(X,\B)$ is a segregated $k$-uniform hypergraph with $|X|=n$, $|\B|=j$, $d_1=kr$ and $\pi$ is the maximal order of an independent set in $H$. Then if 
%     \[\tag{*} 4n-jk-(3k-2)(\pi-1)+r(\lfloor k/2\rfloor -1)-(\pi-1)\cdot(\lfloor k/2\rfloor+1)>0,\]
% there is a (maximal) independent set $I\subseteq X$ of order $\pi$ consisting entirely of vertices whose degrees are $1$, $2$ or $3$ and containing at least $\left\lceil\frac{d_2}{\lfloor 3k/2\rfloor}\right\rceil$ vertices of degree $2$. 

% Furthermore if the left-hand side of the inequality (*) is instead equal to zero then there is an independent set $I$ of order $\pi-1$ with $d(I)\in\{1,2\}$, all vertices have degree at most $4$, and $d_2=(\pi-1-r)\lfloor \frac{3k}{2}\rfloor$.\end{prop}

% Before giving the proof of the proposition, let us make some elementary observations that are used several times in the sequel. 

% Let $H=(X,\B)$ be a $k$-uniform hypergraph with $|X|=n$ and $|\B|=j$. Recall $d_i$ is the number of vertices having degree $i$. Clearly

% \begin{equation}\label{setsum}
%     \sum_{i\geq 0}d_i=n.
% \end{equation}

% Let $\BB$ denote the multiset $\bigsqcup B_i$, which contains the vertices of the hypergraph counted with their multiplicities in the blocks $\B$. As there are $j$ blocks we have $|\BB|=jk$. Of course a vertex $x$ will occur in exactly $d(x)$ of the blocks, and so: 

% \begin{equation}\label{multisetsum}
%     \sum_{i\geq 1}id_i=jk.
% \end{equation}




% \begin{proof}[Proof of \cref{1s2s3s}] Suppose for a contradiction that there is no independent set $I$ of order $\pi$ with $d(I)\subset\{1,2,3\}$.  We will show this leads to a contradiction. 
    
% For a given independent set $I$, we denote by $\delta_i=d^{-1}(i)\cap I$ the number of elements of degree $i$ in $I$. By \cref{ninthdeg2} there exists an independent set with 
% \begin{equation}\delta_2\geq \frac{d_2}{\lfloor\frac{3k}{2}\rfloor}.\end{equation} Let $J$ be the union of this set together with one element chosen from each of the $r$ isolated blocks. Now take $I\supseteq J$ maximal subject to $d(I)\leq 3$; therefore  $\delta_1+\delta_2+\delta_3\leq \pi-1$. 
% Note that by segregation, the $\delta_1=r$ isolated $I$-blocks account for all elements in $X$ of degree $1$. 

% We now bound from above and below the number $d_2+d_3$ of vertices in $X$ having degrees $2$ or $3$. 

% Since $I$ is maximal, we have $X=\bigcup\B_I$. By segregation, no vertex of degree $2$ or $3$ is adjacent to a vertex of degree $1$ and so it must live in $\bigcup\B_x$, with $x$ running over the elements of $I$ of degree $2$ or $3$. The latter contains at most $$(2k-1)(\delta_2-\sigma)+\left\lfloor\frac{k+1}{2}\right\rfloor\sigma+(3k-2)\delta_3$$ distinct elements. Using $\delta_1+\delta_2+\delta_3\leq \pi-1$, we get 
% \begin{align}d_2+d_3&\leq (2k-1)\delta_2+(3k-2)\delta_3\\
%     &\leq (2k-1)\delta_2+(3k-2)(\pi-1-r-\delta_2)\\
%     &\leq (1-k)\delta_2+(3k-2)(\pi-1-r)\label{combine}
% \end{align}


% On the other hand, consider the multiset $\BB=\bigsqcup_{B\in\B} B$, of order $jk$. The $I$-blocks contribute exactly $\delta_1k+2\delta_2k+3\delta_3k$ of these elements. Since there are $2d_2+3d_3$ elements of degrees $2$ or $3$, there are $2\delta_2k+3\delta_3k-2d_2-3d_3$ elements of $\BB$ of order at least $4$ coming from the $I$-blocks. The remaining $(j-\delta_1-2\delta_2-3\delta_3)$-blocks all contain elements of order at least $4$. Thus the multiset $|\BB_{\geq 4}|$ of all elements of degrees $4$ and above has order
% \[|\BB_{\geq 4}|=(j-\delta_1-2\delta_2-3\delta_3)k + 2\delta_2k+3\delta_3k-2d_2-3d_3 =jk-rk-2d_2-3d_3.\] 
% Since $|\BB_{\geq 4}|=\sum_{i\geq 4} id_i$, we get 
% \begin{equation}\sum_{i\geq 4} d_i \leq \frac{jk-rk-2d_2-3d_3}{4}.\end{equation}
% Now $n=\sum d_i\leq rk+d_2+d_3+\frac{jk-rk-2d_2-3d_3}{4}$. 

% Rearranging, we get:
% \begin{equation}
%     d_2+d_3\geq 4n-jk-d_2-3rk,
% \end{equation}

% which combines with (\ref{combine}) to give 
% \[(1-k)\delta_2+(3k-2)(\pi-1-r) \geq 4n-jk-d_2-3rk.\]
% Now using \cref{shannonbound} we get 
% \[\delta_2\cdot\lfloor 3k/2\rfloor-(k-1)\delta_2\geq 4n-jk-(3k-2)(\pi-1)-2r.\]
% Using $\delta_1+\delta_2\leq \pi-1$, with equality if and only if $\delta_3=0$, then after some manipulation, we get
% % (r+\delta_2)\cdot(\lfloor k/2\rfloor+1) &\geq 4n-jk-(3k-2)(\pi-1)-2r+r(\lfloor k/2\rfloor+1) \\
% % (r+\delta_2)\cdot(\lfloor k/2\rfloor+1) &\geq 4n-jk-(3k-2)(\pi-1)+r(\lfloor k/2\rfloor -1) \\
% % (\pi-1)\cdot(\lfloor k/2\rfloor+1) &\geq 4n-jk-(3k-2)(\pi-1)+r(\lfloor k/2\rfloor -1) \end{align*}
% % Hence,
% \[4n-jk-(3k-2)(\pi-1)+r(\lfloor k/2\rfloor -1)-(\pi-1)\cdot(\lfloor k/2\rfloor+1)\leq 0,\]
% which is a contradiction, proving the first statement of the proposition.

% If $4n-jk-(3k-2)(\pi-1)+r(\lfloor k/2\rfloor -1)-(\pi-1)\cdot(\lfloor k/2\rfloor+1)=0$ then we must have had equality in all the previous inequalities that were used, including \cref{shannonbound}. Thus, $\delta_2\lfloor 3k/2\rfloor=d_2$, $\delta_3=0$, $\pi-1=\delta_1+\delta_2$ and all vertices are degree at most $4$.
% \end{proof}

The following results are all used to give constraints to Prolog. The first two respectively bound above and below the number of isolated blocks.

\begin{lemma}\label{rbound}
    Let $(X, \mathcal{B} )$ be a segregated $(n, k, p, 2; j)$-lottery design. Suppose that $\B$ has $r$ isolated blocks. Then,
    $$ r \geq \left\lceil\frac{2n-jk}{k} \right\rceil. $$
 \end{lemma}
\begin{proof}We have $jk=d_1+\sum_{i\geq 2} id_i \geq d_1+2(n-d_1)$. Write $d_1=rk$ and rearrange to get the formula above.
\end{proof}

\begin{lemma}\label{rboundforfuredi}Let $H=(X,\B)$ be a segregated $(n, k, p, t; j)$-lottery design with $k=2m$, at least $r$ isolated blocks and at least $s$ disjoint $k$-Shannon subhypergraphs. Then 
\[L(n-2mr-3ms,k,p-s-r,t)\leq j-r-3s\]    
    %$$r\leq j-L(n-rk,k,p-r,t).$$
\end{lemma}
\begin{proof}Suppose $B_1,\dots,B_r$ are isolated and $H_1\dots H_s$ are the disjoint $k$-Shannon subhypergraphs of $H$, with $H_i=(X_i,\{C_{i1},C_{i2},C_{i3}\})$. Let $Y$ be the remaining vertices inducing a subhypergraph $H_0=(Y,\B_0)$ of $H$ where $|\B_0|=j-r-3s$. Fix one vertex $v_i\in B_i$ for $1\leq i\leq r$ and $v_{r+1}\dots v_{r+s}$ in each of the $C_{i1}$ with $1\leq i\leq s$. For any choice $\{v_{r+s+1},\dots,v_p\}$ of vertices from $Y$, the draw $D=\{v_1,\dots,v_p\}$ intersects with some $B\in \B$ in at least $t$ vertices. By construction of $D$, $B$ is neither isolated nor one of the $C_{ij}$. But this means $B$ must match $t$ of the remaining $p-r-s$ elements of $D$. Hence $H_0$ is an $(n-2mr-3ms,k,p-s-r,t;j-r-3s)$-lottery design, which implies the inequality as shown.\end{proof}


\begin{lemma}\label{lowerboundd2}Let $H$ be a segregated $k$-uniform hypergraph. Then 
    \[d_2\geq 3n-2rk-jk\]
\end{lemma}
\begin{proof}We recall $\sum d_i=n$ and $\sum id_i=jk$, which implies 
    \[jk-n=d_2+\sum_{i\geq 3}(i-1)d_i\geq d_2+2\sum d_i=-2d_1-d_2+2n.\qedhere\]\end{proof}

\begin{lemma}\label{minnumiblocks}
    Let $(X,\B)$ be a $k$-uniform hypergraph with maximal independent set $I$. Then 
    %segregated $(n,k,p,2;j)$-lottery design with maximal independent set $I$. Then 
    $$|\B_I|\geq \left\lceil\frac{n-p+1}{k-1}\right\rceil.$$
\end{lemma}
\begin{proof}
    We must have $\bigcup \B_I=X$ or $I$ is not maximal. For $x\in I$, two blocks $B,C\in\B_x$ intersect in at least $x$ so $|\bigcup \B_x|\leq (k-1)|\B_x|+1$. Since $|I|\leq p-1$, then summing over $x\in I$ yields $n\leq (k-1)|\B_I|+p-1$.
\end{proof}






\subsection{Excess, toes and webbings}For the values of $(n,k,p)$ under consideration, one tends to find $L(n,k,p,2)\sim n/2$. Thus, on average, the degree of a vertex in an $(n,k,p,2)$-lottery design is about $2$. The next definition follows \cite{BateRees} with a view to bounding the extent of departure from this average value.
\begin{defin}For a set of vertices $Y\subseteq X$,  the \emph{excess} of $Y$ is the sum \[\E(Y)=\sum_{i>2}(i-2)\cdot |d^{-1}(i)\cap Y |.\]\end{defin}

Note that if $Y=X$, we get $\E(X)=\sum_{i>2}(i-2)\cdot d_i$. The following gives an easy characterisation of $\E(X)$.

\begin{lemma}\label{excessfromisolated} Let $H=(X, \mathcal{B} )$ be a segregated hypergraph with $r$ isolated blocks. Then
    \[\E(X)=jk+rk-2n.\]\end{lemma}
   \begin{proof}We have $jk=\sum id_i$. Since $n=\sum d_i$, we get \[jk-2n=-d_1+\sum_{i\geq 3}(i-2) d_i=-d_1+\E(X).\qedhere\]
\end{proof}

The possible number of toes is constrained by the value of the excess $\E(X)$, in a manner we now describe. First, the following is \cite[Lem.~4.7]{BateRees} after the removal of a significant typo. 

\begin{lemma}\label{mintoes}
    Let $(X, \mathcal{B} )$ be a segregated $(n, k, p, 2; j)$-lottery design with maximal independent set $I$ and $F_I$ its foot. Suppose further that $\B$ has $r$ isolated blocks. Then,
    $$ |F_I| \geq 2n - 2p +2 -(k-1)(|\B_I|+r).$$
\end{lemma}
\begin{proof}As $I$ is maximal, we have $\bigcup\B_I=X$. The multiset \[\mathscr{R}:=\bigsqcup_{x\in I,d(x)>1,B\in \B_x} B\setminus\{x\}\] contains the $kj-rk-|I|$ vertices which are neither isolated nor contained in $I$, and by definition, the toes are the elements of multiplicity $1$ in $\mathscr{R}$. Therefore each of the remaining $kj-rk-|I|-|F_I|$ vertices appears at least twice in $\mathscr{R}$. Since $|\mathscr{R}|=(k-1)(|\B_I|-r)$, we have 
    \[|F_I|+2(n-rk-(|I|-r)-|F_I|)\leq (k-1)(|\B_I|-r).\]
Rearranging and using $|I|\leq p-1$, we get the inequality as claimed.
\end{proof}

Now suppose $x\in I$ for $I$ a maximal independent set, with $F_x$ the set of $x$-toes. Let us assume 
\begin{equation*}|F_x|\geq k.\tag{*}\end{equation*}
Then it follows there are at least two blocks $B,C\in \B_x$, say, containing (necessarily distinct) toes; say $y\in B$ and $z\in C$. If $y$ and $z$ were not adjacent, then replacing $x$ with $y,z$ in $I$ would yield a larger independent set, a contradiction. Thus there is a block $W$ with $y,z\in W$. We refer to such blocks as \emph{webbings}. Formally, $W$ is an \emph{$x$-webbing} if $W\not\in\B_I$ and $W$ contains distinct $x$-toes; the set of $x$-webbings is later denoted $\W_x$. Note that under the assumption (*), each toe appears at least once in a webbing, so that it must have degree at least $2$. 

More precisely, suppose there are $\tau_i$ toes in distinct $x$-blocks $B_i$ with $\tau_1\geq\tau_2\dots\geq\tau_s\geq 1$. Then each of the $\sum_{i<j}\tau_i\tau_j$ pairs of toes must appear in some webbing. In the case $k=6$ and $|F_x|\geq 7$, one can see that some toes must have degree at least $3$, for example. This implies non-trivial lower bounds on the excess $\E(F_x)\leq \E(X)$.

\begin{lemma}\label{toetable}
    Let $x\in X$ with $x$ of degree $2$ or $3$ and $F_x$ the set of $x$-toes. The table below gives minimum values of $\E(F_x)$ in terms of $|F_x|$.
    \begin{center}
    \begin{table}[ht]\label{table:exs}
    \begin{tabular}{ c|c|c|c|c|c|c|c|c|c|c|c| } 
    $|F_x|$ & $\leq 5$ & $6$ & $7$ & $8$ & $9$ & $10$ & $11$ & $12$ & $13$ & $14$ & $15$\\\hline
    $\min(\E(F_x))$ & $0$ & $0$ & $2$ & $3$ & $7$ & $10$ & $11$ & $12$ & $20$ & $25$ & $27$
    \end{tabular}
    \end{table}
    \end{center}
If moreover $F_x$ is known to contain no elements of degree $2$, then $\min(\E(F_x))\geq |F_x|$.\end{lemma}
\begin{proof}For the second statement of the lemma, just observe that any $x$-toe of degree at least $3$ contributes at least $1$ to $\E(F_x)$.

For the table itself, suppose $|F_x|=\tau_1+\tau_2+\tau_3$ is a partition of $|F_x|$ into summands of size at most $6$. If $|F_x|\leq 6$, then one webbing $W$ suffices to cover all toes, and so the minimum possible excess of $0$ is a achieved by a configuration of $x$-blocks and one webbings in which each toe appears just twice. Otherwise suppose there are $w>1$ webbings, containing each of the $\tau_1\tau_2+\tau_1\tau_3+\tau_2\tau_3$ pairs of toes, so that 
\[\left\lceil\frac{\tau_1\tau_2+\tau_1\tau_3+\tau_2\tau_3}{3}\right\rceil.\] is an upper limit for $w$. 

We used the powerful linear programming solver Gurobi to solve the following problem. let $M$ be a $w\times |F_x|$-matrix of variables taking values in $0$ and $1$, with the rows representing webbings and a $1$ appearing in the $(i,j)$-th entry if a toe labelled $j$ is in the $i$th webbing. Since $|F_x|>6$, we know that toe $j$ appears once in the $x$-blocks and at least once in the webbings, so $\E(F_x)+|F_x|$ is the sum $S$ of all entries of the $M$. Furthermore, the rows of $M$ must all sum up to integers less than or equal to $k=6$, and columns $j_1$ and $j_2$ must have scalar product at least $1$ whenever $j_1$ and $j_2$ come from different parts of the partition \[\{1,\dots,n\}=\{1,\dots,\tau_1\}\cup\{\tau_1+1,\dots,\tau_1+\tau_2\}\cup\{\tau_1+\tau_2+1,\dots,\tau_1+\tau_2+\tau_3\}.\] We ask Gurobi to minimise the sum $S$ subject to these constraints, and it results in the table in the lemma. 

Since Gurobi only gives answers up to a percentage accuracy, its output does not amount to a proof of optimality, and so we wrote some additional Prolog code to check that the values of $\min(F_x)$ one below those in the table are infeasible.\footnote{Code is available at \url{github.com/cushydom88/lottery-problem}} %See \cref{appendb}.
\end{proof}

\begin{rmk}This improves the bound in \cite[Table 1]{BateRees} for $\min(|F_x|)$ when $|F_x|=13$ from $16$ to $20$. Since Gurobi outputs a feasible configuration of webbings for each value of $|F_x|$, we know that the minima can be achieved.\end{rmk}

The following rather specific two results lead to some surprisingly effective constraints. The first is an easy check left to the reader.

\begin{lemma}\label{reduce3to2}
    Let $H=(X,\B)$ be a hypergraph and $I\subseteq X$ an independent set. Let $x,y\in I$ with $d(x)=2$, $d(y)=3$ and $z$ another vertex of degree $2$ adjacent to $x$ and $y$. Suppose there exists a toe $w$ of degree $2$ adjacent to $x$ but not adjacent to $z$. Then $I'=I\setminus \{x,y\}\cup\{w,z\}$ is an independent set. 
\end{lemma}

\begin{lemma}\label{adjacency}
    Let $H=(X,\B)$ be a hypergraph and $I\subseteq X$ an independent set of maximal order. Let $x,y\in I$ with $d(x)=2$, $d(y)=3$ and $z$ another vertex of degree $2$ adjacent to $x$ and $y$. Suppose there exists a toe $w$ adjacent to $x$ but not adjacent to $z$. Let $v$ be a toe adjacent to $y$ but not adjacent to $z$. Then $w$ and $v$ are adjacent.
\end{lemma}
\begin{proof} Suppose $w$ and $v$ are not adjacent. Then $I'=I\setminus\{x,y\}\cup\{v,w,z\}$ is independent of larger order than $I$. 
\end{proof}

In the following let $\tau_r=|\tau^{-1}(r)|$ be the number of vertices $x$ in $I$ such that $|F_x|=r$. Say an independent set $I$ is \emph{$2$-max} if it is of maximal order and contains a maximal order subset of independent vertices of degree $2$.

% , its vertices have degree at most $3$ and there is no other such independent set with strictly more vertices of degree $2$. If $I$ is $2$-max, then note that all vertices of degree $2$ are either in $I$ or adjacent to the $\delta_2$vertices in $I$ of degree $2$.

\begin{lemma}\label{d3excess}
    Let $H=(X,\B)$ be a $(n,6,6,2;j)$-lottery design containing $r$ isolated blocks and $s$ Shannon subhypergraphs. Let $I$ be a $2$-max independent set whose vertices have degree at most $3$, containing $\delta_2$ vertices of degree $2$. Then 
    \[6j-3n+s+8\delta_2+12r\geq 7\tau_{13} +11\tau_{14}+12\tau_{15}.\]
\end{lemma}
\begin{proof} For $Y\subseteq X$, let $\mathfrak{F}(Y)=\sum_{i\geq 4}(i-3)|d^{-1}(i)\cap Y|$. In particular,
    \[\mathfrak{F}(X)=\sum_{i\geq 4}(i-3)d_i=\E(X)-n+rk+d_2=6j+12r-3n+d_2,\] 
    where we use \cref{excessfromisolated}.  Let $x\in I$ with $\tau(x)=13$. Then all $x$-toes have degree at least $3$: one observes that each toe $t\in F_x$ is opposite at least $8\geq 6$ others; thus there must be at least two $x$-webbings containing. From \cref{toetable} we have $\E(F_x)\geq 20$. Thus \[\mathfrak{F}(F_x)\geq \E(F_x)-13\geq 20-13=7.\] Similar arguments for $\tau(x)=14$ or $15$ yield \[\mathfrak F(X)\geq 7\tau_{13} +11\tau_{14}+12\tau_{15}.\]
    Since $I$ is $2$-max, \cref{shanrearrangement} gives $d_2\leq 8\delta_2+s$ and we are done.
\end{proof}

\begin{lemma}\label{changingsocks}
    Let $H=(X,\B)$ be a $(n,6,p,2;j)$-lottery design containing $r$ isolated blocks and $s$ Shannon subhypergraphs. Let $I$ be an independent set of order $p-1$ and take $x\in I$ with $d(x)=3$, $\tau(x)=12$ and $\E(F_x)\leq 14$. Then at least one of the following holds:
    \begin{enumerate}
        \item there exists $y\in X$ with $d(y)=3$ such that $I'=I\setminus\{x\}\cup\{y\}$ is independent with $\tau(y)\leq 11$;
        \item there exists an $(n-14,6,p-2,2;j-7)$-lottery design.
    \end{enumerate}
\end{lemma}
\begin{proof}
    Assume (1) does not hold.

    It can be shown that the hypotheses imply, up to isomorphism, that 
    \[\B_{x}=\{\{x,1,2,3,4,\_\},\{x,5,6,7,8,\_\},\{x,9,10,11,12,\_\}\},\text{ and}\]
    \[\W_x\supseteq \{\{1,2,5,6,9,10\},\{1,2,7,8,11,12\},\{3,4,5,6,11,12\},\{3,4,7,8,9,10\}\},\]
where $F_x=\{1,\dots,12\}$. Since $\E(F_x)\leq 14$, at least one of the toes in $F_x$ must have degree $3$; say, with label $1$. 
% ; set this to be $y$. 
Now
$\B_1=\{\{1,x,2,3,4,z\},\{1,2,5,6,9,10\},\{1,2,7,8,11,12\}\}.$ Since (1) does not hold, it follows that $z$ is a toe. Moreover,
\[\mathcal{W}_1\supseteq \mathcal{V}_1:=\{\{x,z,5,6,7,8\},\{x,z,9,10,11,12\},\{3,4,5,6,11,12\},\{3,4,7,8,9,10\}\}.\]
Hence the vertices $Y:=\{x,z,1,2\dots,12\}$ induce a subhypergraph of $H$ containing a $(14,6,2,2;7)$-lottery design with blocks $\C=\B_1\cup\mathcal{V}_1$. For each block $B\in\B\setminus\C$, replace any occurrence of a vertex in $Y$ with a vertex in $X\setminus (Y\cup B)$. For any draw of size $p-1$ from $X\setminus Y$, create one of size $p$ by appending the vertex $1$ to it. This must intersect a block of $\B\setminus\C$ in at least $2$ vertices and hence gives the lottery design as specified in (2).
\end{proof}

\section{Proof of the theorem}
\subsection{SICStus Prolog code}
The code is available to download from \url{github.com/cushydom88/lottery-problem}, which implements the constraints described in the previous section. We give a description of the strategy and its functionality. 

The user first loads SICStus Prolog and consults the file \lstinline{lottery.pl} through the command 
\begin{lstlisting}
    ?- ['$PATH_TO_DIRECTORY/lottery.pl'].
\end{lstlisting} 
Then one queries Prolog at the command line by asking it to solve (for any unbound variables) in a conjunction of predicates. The following is an example of the output produced from the main predicate in our code:
% Figure environment removed 

The principal predicates are:

\lstinline{lottery_numbers_in_range( NMin, NMax)}. This predicate writes output to the terminal of the form given in \cref{outputexample}; i.e.~it works sequentially with $n$ from the value of \lstinline{NMin} to the value of \lstinline{NMax} either outputting a line stating the value of $L(n,6,6,2)$ or a conjectured value of it, which is correct modulo a list of exceptional cases which could perhaps be checked by hand. The conjectured value for \lstinline{NMin} is first computed from scratch, using a lower bound of $1$; for efficiency, from that point onwards, it then passes the (in some cases, conjectured) value of $L(n,6,6,2)$ as a lower bound for $L(n+1,6,6,2)$, invoking \cref{doesntgodown}. 

\lstinline{upper_bound( N, Guess, UB )} is called by the previous predicate, and is checked recursively. It holds when \lstinline{N}~$=n$ and the integer \lstinline{Guess} can be achieved as a sum of at most $p-1=5$ values $C(a_i,6,2)$ in \cref{coveringnumbers}. Then \lstinline{UB} is bound to \lstinline{Guess}. Otherwise it is declared that \lstinline{upper_bound( N, Guess+1, UB )} should hold.

This predicate is first asserted with \lstinline{Guess} as the (conjectured) value $j_{n-1}$ of $L(n-1,6,6,2)$. In many cases, it turns out that $j_{n-1}$ can be achieved as a sum of at most $5$ values $C(a_i,6,2)$ and so \lstinline{UB} is bound to $j_{n-1}$. In that case, by \cref{doesntgodown}, we conclude immediately $L(n,6,6,2)=L(n-1,6,6,2)$---modulo any previous cases that remain to be checked by hand.

Otherwise, \lstinline{UB} $>j_{n-1}$ and we seek to rule out $L(n,6,6,2)=$ \lstinline{UB}$-1$. Assume therefore, in search of a contradiction, that there is a lottery design $H=(X,\B)$ with $|\B|=$~\lstinline{UB}$-1$. Then further predicates are engaged which either generate the sought contradiction, or progressively collect a list of information about possible cases that cannot be ruled out.

\lst{bound_isolated_blocks_and_num_shans( N, UB, Rs, RSPairs )} takes the value of \lstinline{UB} from the above predicate and binds \lstinline{RSPairs} to a list of plausible pairs $(r,s)$ where $r=d_1/6$ is the number of isolated blocks in $H$, and $s$ is the number of Shannon subhypergraphs in $H$---a pair will be determined as plausible by the following: Assume $H$ is a $(n,6,6,2;\lstinline{UB}-1)$-lottery design with $(r,s)$ its number of isolated blocks and Shannon subhypergraphs. Then $r+s\leq 5$ (or else there exists $I$ of order $6$); $r$ must satisfy the inequality in \cref{rbound}; and $(r,s)$ must satisfy the inequality in \cref{rboundforfuredi}, implying there exists a $(n-6r-6s,6,6-s-r,2;\lstinline{UB}-1-r-3s)$-lottery design which in turn constrains $r$ and $s$ by appeal to \cref{FurediBound}.

\lstinline{get_deltaI_exceptions( N, UB, MinNumIBlocks, [R,S], DeltaExceptions ) }. Here we assume that the bound labelled (*) in \cref{1s2s3s} holds with $\pi=5$ and consider independent sets $I$ satisfying its conclusion. We find all possibilities for $\delta(I)$ satisfying a long list of constraints. More specifically, for any given pair \lstinline{[R,S]} in \lstinline{RSPairs}, this predicate binds \lstinline{DeltaExceptions} to a list whose entries are tuples \[\lstinline{DeltaException=[R,S,D2L,D2U,Delta]}\] where \lstinline{Delta} represents a tuple $\delta(I)$ satisfying our system of  constraints. It is accompanied by the extra data \lstinline{D2L} and \lstinline{D2U}, which are lower and upper bounds for the value of $d_2$---these arise as by-products of the calculations we describe below. \lstinline{Delta} always takes the following form:
\begin{enumerate}\item the first $r=\lstinline{R}$ elements are $1$;
    \item the next $s=\lstinline{S}$ elements are $2$;
\item the next $q$ elements are each $2$, where $q$ is the minimum value implied by \cref{ninthdeg2};
\item the remaining $5-r-s-q$ elements of \lstinline{Delta} are either $2$ or $3$;
\item the sum of \lstinline{Delta}, equal to $|\B_I|$, satisfies the bound in \cref{minnumiblocks} with $\pi=5$;
\item letting $J$ be the last $5-r-s$ elements of $I$, and \lstinline{DeltaTail} the corresponding sublist $\delta(J)\subseteq\delta(I)$, then \lstinline{DeltaTail} satisfies the predicate \lstinline{can_populate_toes_in_Iblocks} as we describe next.
% \[\lstinline{can_populate_toes_in_Iblocks(N, UB, R, S, DeltaTail, [DeltaTail, D2L, D2U])}.\]
\end{enumerate}

\lstinline{can_populate_toes_in_Iblocks(N, UB, R, S, DeltaTail, [DeltaTail, D2L, D2U])}. This  binds a variable \lst{Excess} to the value $\E(X)=6(\lstinline{UB}-1+\lstinline{R})-2\lstinline{N}$ (see \cref{excessfromisolated}), and a variable \lst{MinToes} to the lower bound on $|F_{J}|$ supplied by \cref{mintoes}. Suppose there are $\delta_3$ vertices in $I$ of degree $3$. Each contributes $1$ to $\E(X)$, and since these vertices are not toes by definition, the contribution to the excess $\E(X)$ from toes must be at most $\E(X)-\delta_3$. Hence a variable \lst{FootExcess} is created and bound to \lst{Excess-NumThrees}. 

\lstinline{populate_toes_in_Iblocks( DeltaTail, MinToes, Excess, Vs )} binds \lstinline{Vs} to a solution of a  constraint problem to determine $\tau(J)$; in other words, to determine viable distributions of toes among the $J$-blocks. (If it fails, the case (\lstinline{Delta,R,S}) will not feature in the list \lstinline{DeltaExceptions}.) The constraint problem is as follows: \begin{enumerate}%\item $J=\{x_{r+s+1},\dots,x_{p-1}\}\subseteq I$ is the subset of $I$ of vertices of degree at least $2$, such that $\delta(J)$ is bound to \lst{DeltaNoOnes};
\item \lstinline{Vs} is a sequence of variables taking values $(\tau(x_{1}),\dots,\tau(x_{5-r-s}))$ where $J=\{x_1,\dots,x_{5-r-s}\}$ and the $i$th entry in \lstinline{DeltaTail} is $d(x_i)$; 
\item the total number of toes $\sum\tau(J)$ represented by \lstinline{sum(Vs)} is at least the calculated lower bound \lstinline{MinToes};
\item the sum over $1 \leq i\leq 5-r-s$ of the minimum values for  $\E(F_{x_i})$ implied by \cref{toetable} is at most \lst{FootExcess}.
\end{enumerate}

The solutions \lstinline{Vs} are then compelled to satisfy \lstinline{d3_excess_check}, which implements the constraint \cref{d3excess}; and \lstinline{twos_lie_with_twos}, which implements the constraints from \cref{reduce3to2} and \cref{adjacency}. Finally \lstinline{changing_socks} implements \cref{changingsocks} repeatedly.



\lstinline{get_bad_RS_tuples(N, UB, RSPairs, BadRSTuples1 )}. Here we suppose the inequality (*) in \cref{1s2s3s} does not hold with $\pi=5$. 
% Then we cannot conclude from that proposition that $I_0$ can be extended to an independent set of size $5$ whose elements are all of order $1,2,3$. %Say a maximal independent set is \emph{fat} if it is of order at most $4$ or contains at least one vertex of degree at least $4$. 

This predicate binds \lstinline{BadRSTuples1} to a list of tuples \lstinline{[R,S,A,B,C]} where:

\begin{enumerate}\item \lstinline{[R,S]} is a member of \lstinline{RSPairs} representing the pair $(r,s)$ such that the bound (*) in \cref{1s2s3s} fails, with $j=\lstinline{UB}-1$;

%\item \lstinline{A} is the minimum number of vertices of degree $2$ in an independent set $I$ as guaranteed by \cref{ninthdeg2};

\item \lstinline{A} is a lower bound for $d_2$ implied by  \cref{lowerboundd2}; and 

\item \lstinline{B} is an upper bound for $d_2$ achieved by setting $\delta_2=4-r$ in \cref{shanrearrangement}.\end{enumerate}

Take a tuple \lstinline{[R,S,A,B]} in  \lstinline{BadRSTuples1}, and assume it satisfies the bound in \cref{1s2s3s} with $\pi=4$. Then it is possible to find an independent set $I$ of size $4$ with \lstinline{R} elements of degree $1$ and at least \lstinline{S} of degree $2$ and the rest of degree $2$ or $3$. For every value of $d_2$ lying between $B$ and $C$ a list of possible values of $d_3$ are calculated. For each such pair $(d_2,d_3)$, we then solve a constraint problem to find solutions for $\delta(I)$---or $\delta_2$, which amounts to the same thing. More specifically, we let \lstinline{DeltaTail} represent the last $4-r-s$ entries of $\delta(I)$ with $\delta_2-s$ of degree $2$ and $\delta_3$ of degree $3$, corresponding to the subset $J\subseteq I$.  We have $|\BB_{J}\setminus J|=2(\delta_2-s)+3\delta_3$. These must contain at least one each of the $d_2-9s$ vertices of degree $2$ not in Shannon subhypergraphs and each of the $d_3$ vertices of degree $3$. Hence we may bound below the size of the $I$-foot by
\begin{align*}|F_{I}| \geq 2(d_2 -9s &+ d_3 - 4 + r + s) - 10(\delta_2-s) - 15\delta_3\\ &= 2d_2 - 6s + 2d_3 - 8 + 2r - 10\delta_2 - 15\delta_3.\end{align*}
Now $\E(F_{I})\geq d_3-\delta_3$. Finally the predicate \lstinline{populate_toes_in_Iblocks} is invoked. If there are no solutions for any pair $(d_2,d_3)$, then the tuple is excluded from  \lstinline{BadRSTuples1} to form  \lstinline{BadRSTuples}.

The list \lstinline{BadRSTuples} is displayed on the user output stream. (In case $n\leq 70$ this list is empty.)



\section{Configurations for minimal lottery designs}\label{sec:configs}
Tickets of size $6$ are represented in the diagrams of \cref{tickpics} by concatenating the labels over the vertices in each line. Choosing a set of diagrams whose vertices form a disjoint union of $n$ give minimal $(n, 6, 6, 2; j)$-lottery designs for all $32\leq n\leq 70$. For example our $(59, 6, 6, 2; 27)$-lottery design in \cref{27tix} has configuration $(B,C,E,E,E).$ 

% Figure environment removed

\begin{thm}\cref{theconfigs} lists $j=L(n,6,6,2)$ together with configurations described in \cref{tickpics} which afford an $(n,6,6,2;j)$-lottery design.\label{alltogetherthm}\end{thm}

\begin{table}[ht]
    \begin{center}
\begin{tabular}{ |c|c|c| } 
  \hline
 $n$ & $L(n,6,6,2)$ & Configuration \\ 
 \hline
 32 & 7 & $(A,A,A,A,B)$ \\ 
 33 & 7 & $(A,A,A,A,C)$  \\ 
34 & 7 & $(A,A,A,A,D)$  \\
35 & 9 & $(A,A,A,B,C)$  \\
36 & 9 & $(A,A,A,C,C)$  \\
37 & 10 & $(A,A,A,C,D)$  \\
38 & 11 & $(A,A,A,A,E)$  \\
39 & 11 & $(A,A,C,C,C)$  \\
40 & 12 & $(A,A,C,C,D)$  \\
41 & 13 & $(A,A,A,C,E)$  \\
42 & 13 & $(A,C,C,C,C)$  \\
43 & 14 & $(A,C,C,C,D)$  \\
44 & 15 & $(A,A,C,C,E)$  \\
45 & 15 & $(C,C,C,C,C)$  \\
46 & 16 & $(C,C,C,C,D)$  \\
47 & 17 & $(A,C,C,C,E)$  \\
48 & 18 & $(A,C,C,D,E)$  \\
49 & 19 & $(A,A,C,E,E)$  \\
50 & 19 & $(A,A,A,A,E)$  \\
51 & 20 & $(A,A,A,D,E)$  \\
  \hline
\end{tabular}
\begin{tabular}{ |c|c|c| } 
  \hline
 $n$ & $L(n,6,6,2)$ & Configuration \\ 
 \hline
52 & 21 & $(A,C,C,E,E)$  \\
53 & 22 & $(A,C,D,E,E)$  \\
54 & 23 & $(A,A,E,E,E)$  \\
55 & 23 & $(C,C,C,E,E)$  \\
56 & 24 & $(C,C,D,E,E)$  \\
57 & 25 & $(A,C,E,E,E)$  \\
58 & 26 & $(A,D,E,E,E)$  \\
59 & 27 & $(B,C,E,E,E)$  \\
60 & 27 & $(C,C,E,E,E)$ \\
61 & 28 & $(C,D,E,E,E)$ \\
62 & 29 & $(D,D,E,E,E)$ \\
63 & 30 & $(C,E,E,E,F)$ \\
64 & 31 & $(D,E,E,E,F)$ \\
65 & 31 & $(C,E,E,E,E)$ \\
66 & 32 & $(D,E,E,E,E)$ \\
67 & 34 & $(E,E,E,F,G)$ \\
68 & 34 & $(E,E,E,E,F)$ \\
69 & 35 & $(E,E,E,E,G)$ \\
70 & 35 & $(E,E,E,E,E)$ \\
&&\\
  \hline
\end{tabular}
\caption{Lottery numbers and their configurations}
\label{theconfigs}
\end{center}
\end{table}

\begin{center}
    \begin{table}
\scalebox{0.9}{\small
\begin{tabular}{ c c c c c } 
\cellcolor{lightgray} $1,2,3,4,5,6$ & $9,10,11,12,13,14$ & \cellcolor{lightgray} $18,19,20,21,26,27$ & $32,33,34,35,40,41$ & \cellcolor{lightgray} $46,47,48,49,54,55$\\ 
\cellcolor{lightgray} $1,2,3,4,7,8$ & $9,10,11,15,16,17$ & \cellcolor{lightgray} $18,19,22,23,30,31$ & $32,33,36,37,44,45$ & \cellcolor{lightgray} $46,47,50,51,58,59$\\
\cellcolor{lightgray} $1,2,5,6,7,8$ & $12,13,14,15,16,17$ & \cellcolor{lightgray} $18,19,24,25,28,29$ & $32,33,38,39,42,43$ & \cellcolor{lightgray}  $46,47,52,53,56,57$ \\
\cellcolor{lightgray}  &  & \cellcolor{lightgray} $20,21,22,23,28,29$ & $34,35,36,37,42,43$ & \cellcolor{lightgray} $48,49,50,51,56,57$\\
\cellcolor{lightgray}  &  & \cellcolor{lightgray} $20,21,24,25,30,31$ & $34,35,38,39,44,45$ & \cellcolor{lightgray} $48,49,52,53,58,59$\\
\cellcolor{lightgray}  &  & \cellcolor{lightgray} $22,23,24,25,26,27$ & $36,37,38,39,40,41$  & \cellcolor{lightgray} $50,51,52,53,54,55$\\
\cellcolor{lightgray}  &  & \cellcolor{lightgray} $26,27,28,29,30,31$ & $40,41,42,43,44,45$ & \cellcolor{lightgray} $54,55,56,57,58,59$
\end{tabular}
}\caption{Working set of $27$ tickets for  $n=59$ balls using the configuration $(B,C,E,E,E)$.\label{27tix}}\end{table}
\end{center}

The reader may check for themselves that any draw of $6$ numbers from $1$ to $59$ will match at least two numbers with at least one of the above tickets.


{\bf{Acknowledgement:}} The authors are supported by the Leverhulme Trust Research Project Grant number RPG-2021-080. 
 
\AtNextBibliography{\footnotesize} 
\printbibliography

\end{document}



% \newpage\ \newpage\section*{\Large\sc Appendices}
% \begin{appendices}
% \section{Main Prolog code}\label{appendixcode}
%     \begin{lstlisting}
%     :-use_module(library(clpfd) ).
%     :-use_module(library(lists)).
%     :-use_module(library(ordsets)).
%     :-use_module(library(ugraphs)).
%     :-use_module(library(between)).
    
%     % ['lottery.pl'].
    
%     % compute_lottery_numbers_in_range(30,61).
    
%     %%% COMPUTE LOTTERY NUMBERS IN A RANGE %%%
%     % Given Nmin and Nmax try to compute L(n,6,6,2) for Nmin <= n <= Nmax.
%     % The value of L(n,6,6,2) is passed in to the computation of L(n+1,6,6,2)
%     compute_lottery_numbers_in_range( Nmin, Nmax ) :-
%         PrevLottoNum #= 1,
%         compute_lottery_numbers_in_range2( PrevLottoNum, Nmin, Nmax ).
%     compute_lottery_numbers_in_range2( PrevLottoNum, N, Nmax ) :-
%         N #= Nmax,
%         possible_lottery_number( N, PrevLottoNum, _ ).
%     compute_lottery_numbers_in_range2( PrevLottoNum, N, Nmax ) :-
%         N #\= Nmax,
%         possible_lottery_number( N, PrevLottoNum, NewLottoNum ),
%         NewN #= N + 1,
%         compute_lottery_numbers_in_range2( NewLottoNum, NewN, Nmax ).
%     %%%%%%
    
%     %%% POSSIBLE LOTTERY NUMBER %%%
%     possible_lottery_number( N, PrevLottoNum, UB ) :-
%         % first calculate the upper bound
%         upper_bound( N, PrevLottoNum, UB ),
%         possible_lottery_number2( N, PrevLottoNum, UB ).
%     possible_lottery_number2( N, PrevLottoNum, UB ) :-
%         PrevLottoNum #= UB,
%         print_message(informational, format(' L(~w,6,6,2) = ~w ',[N,UB]) ).
%     possible_lottery_number2( N, PrevLottoNum, UB ) :-
%         PrevLottoNum #\= UB,
%         % Calculate an upper bound on d1
%         bound_isolated_blocks( N, UB, Rs ),
%         get_min_num_Iblocks( N, MinNumIBlocks ),
%         % get any exceptions for constructing a slim I
%         maplist( get_slim_I_exceptions(N, UB), Rs, PossSlimIExceptions ),
%         include( interval_check, PossSlimIExceptions, SlimIExceptions1 ),
%         % get Delta(I) exceptions
%         maplist( get_deltaI_exceptions(N, UB, MinNumIBlocks), Rs, DeltaExceptionsList ),
%         append( DeltaExceptionsList, DeltaExceptions),
%         write_lotto_result( SlimIExceptions1, DeltaExceptions, N, UB ).
%     %%%%%%
    
%     %%% D1 UPPER BOUND %%%
%     bound_isolated_blocks( N, UB, PossD1s ) :-
%         get_r_min( N, UB, D1Min ),
%         include( bound_isolated_blocks_helper(N, UB, D1Min), [0,1,2,3,4], PossD1s ).
%     bound_isolated_blocks_helper(_, _, D1Min, 0) :-
%         D1Min #< 1.
%     bound_isolated_blocks_helper(N, UB, D1Min, R) :-
%         R #> 0,
%         D1Min #=< 6*R,
%         P #= 6 - R,
%         LB #= UB - R,
%         M #= N - 6 * R,
%         furedi_lower_bound( Vs, M, 6, P, LB ),
%         labeling([], Vs).
%     get_r_min( N, UB, D1Min ) :-
%         T #= 2*N - 6*(UB - 1),
%         T #=< 0,
%         D1Min #= 0.
%     get_r_min( N, UB, D1Min ) :-
%         T #= 2*N - 6*(UB - 1),
%         T #> 0,
%         D1Min #= T.
%     %%%%%%
    
%     %%% COVERING DESIGN UPPER BOUND %%%
%     % Start with a guess for the upper bound and increase the guess until a 
%     % solution is found
%     upper_bound( N, Guess, UB ) :-
%         findall( Vs, ( test_upper_bound( Vs, N, Guess ), labeling([], Vs)),  Sols ),
%         upper_bound2( N, Guess, Sols, UB ).
%     upper_bound2( N, Guess, [], UB ) :-
%         Guess2 #= Guess + 1,
%         upper_bound( N, Guess2, UB ).
%     upper_bound2( _, Guess, Sols, UB ) :-
%         length(Sols, KK),
%         KK #>= 1, 
%         UB #= Guess.
    
%     test_upper_bound( Vs, N, UB ) :- 
%         length( Vs, 5 ),
%         domain( Vs, 1, 65 ),
%         M is N - 25,
%         sum( Vs, #=, M ),
%         get_covering_design_nums( Vs, Scores ),
%         sum( Scores, #=<, UB ),
%         sorting(Vs,[1,2,3,4,5],Vs).
    
%     get_covering_design_nums( Vs, Scores ) :-
%         Vals = [1,3,3,3,4,
%             6,6,7,7,10,10,12,12,15,16,
%             17,19,21,22,23,24,27,28,30,31,
%             31,38,39,40,42,47,50,51,54,55,
%             59,63,65,67,70,73,79,80,82,87,
%             90,96,98,99,105,110,114,117,119,128,
%             132,135,140,142,143,143,157,160,163,172],
%         maplist( get_covering_design_num(Vals), Vs, Scores ).
%     get_covering_design_num(Vals, I, Score ) :-
%         nth1( I, Vals, Score ).
%     %%%%%%
    
%     %%% FUREDI LOWER BOUND %%%
%     furedi_lower_bound(Vs, N, K, P, LB ) :-
%         LLB is K * LB,
%         NP is P - 1,
%         length( Vs, NP ),
%         domain( Vs, 1, N ),
%         sum( Vs, #=, N ),
%         maplist( furedi_lower_bound_helper(K), Vs, Ws ),
%         sum( Ws, #=<, LLB  ).
%     furedi_lower_bound_helper( K,  X, Y ) :- 
%         ((X -1 ) mod (K - 1 )  ) #= 0,
%         Y #= X * ( (X-1) / (K - 1 ) ).
%     furedi_lower_bound_helper( K, X, Y ) :- 
%         ((X -1 ) mod (K - 1 )  ) #\= 0,
%         Y #= X * ( 1 + ( (X-1) / ( K - 1 ) ) ).
%     %%%%%%
    
%     %%% SLIM I EXCEPTIONS
%     get_slim_I_exceptions( N, UB, R, [R, [D2Lower, D2Upper] ] ) :-
%         D2Upper #= 9 * (4 - R),
%         D2Lower0 #= ( 9*( 4*N - 6*(UB-1) - 18*R - 16*(4-R) )) / 4,
%         make_non_negative(D2Lower0,D2Lower).
%     interval_check([_, [A,B] ]) :- A #=< B.
%     make_non_negative(X, Y) :-
%         X #< 0,
%         Y #= 0.
%     make_non_negative(X, Y) :-
%         X #>= 0,
%         Y #= X.
%     %%%%%%
    
%     %%% DELTA(I) EXCEPTIONS %%%
%     % Lower bound the number of blocks in B_I
%     get_min_num_Iblocks( N, Min) :-
%         ( ( N - 5 ) mod 5 ) #= 0,
%         Min #= (N - 5)/5.
%     get_min_num_Iblocks( N, Min ) :-
%         ( ( N - 5 ) mod 5 ) #\= 0,
%         Min #= 1 + ((N - 5)/5).
    
%     % Find the values of Delta(I) we cannot immediately rule out
%     get_deltaI_exceptions( N, UB, MinNumIBlocks, R, Deltas ) :-
%         % Lower bound how many 2s are assumed to be in Delta(I)
%         get_delta_two( N, UB, R, Delta2 ),
%         % Fill Delta(I) with as many 1s and 2s that can be assumed
%         % Then generate all ways to complete Delta(I) by adding 2s and 3s
%         row_of_n_ms( R, 1, L1 ),
%         row_of_n_ms( Delta2, 2, L2 ),
%         append( L1, L2, L3 ),
%         length( L3, K1),
%         K2 #= 5 - K1,
%         findall( Xs, ( length(Xs, 2), domain(Xs, 0, K2), sum(Xs, #=, K2), 
%             labeling([], Xs) ), VSizes ),
%         maplist( make_delta_end, VSizes, DeltaEnds ),
%         maplist( append(L3), DeltaEnds, Deltas1 ),
%         % Remove out Delta(I) where |B_I| is too small 
%         include( at_least_m_blocks(MinNumIBlocks), Deltas1, Deltas2 ),
%         % Upper bound the excess of the whole design
%         get_base_excess( N, R, UB, BaseExcess ),
%         % For each possible Delta(I) consider the constraint problem of populating toes in the IBlocks 
%         include( can_populate_toes_in_Iblocks(N, R, BaseExcess), Deltas2, Deltas ).
    
%     get_delta_two( N, UB, R, Delta2 ) :-
%         get_delta_two2( N, UB, R, PossDelta2 ),
%         make_non_negative(PossDelta2, Delta2).
%     get_delta_two2( N, UB, R, Delta2 ) :-
%         X #= (3*N - 12 * R - 6 * (UB - 1) ),
%         (X mod 9) #= 0,
%         Delta2 #= X / 9.
%     get_delta_two2( N, UB, R, Delta2 ) :-
%         X #= (3*N - 12 * R - 6 * (UB - 1) ),
%         (X mod 9) #\= 0,
%         Delta2 #= 1 + (X / 9).
%     row_of_n_ms( N, M, Row ) :-
%         length( Row, N ),
%         maplist( #=(M), Row ).
%     make_delta_end( [A,B], DeltaEnd ) :-
%         row_of_n_ms( A, 2, X1 ),
%         row_of_n_ms( B, 3, X2 ),
%         append(X1, X2, DeltaEnd).
    
%     at_least_m_blocks( MinNumIBlocks, PossDelta ) :-
%         sum(PossDelta, #>=, MinNumIBlocks).
    
%     get_base_excess( N, R, UB, BaseExcess ) :-
%         BaseExcess #= 6*(UB-1) + 6*R - 2*N. 
    
%     can_populate_toes_in_Iblocks(N, R, BaseExcess, Delta) :-
%         sum(Delta, #=, B),
%         MinToes #= 2*N - 10 - 5*(B+R),
%         delete( Delta, 3, Residue),
%         length( Residue, NumOnesAndTwos ),
%         NumThrees #= 5 - NumOnesAndTwos,
%         Excess #= BaseExcess - NumThrees,
%         delete( Delta, 1, DeltaNoOnes),
%         populate_toes_in_Iblocks( DeltaNoOnes, MinToes, Excess, Vs ), 
%         labeling([], Vs).
    
%     populate_toes_in_Iblocks( DeltaNoOnes, MinToes, Excess, Vs ) :-
%         same_length( Vs, DeltaNoOnes, N ),
%         domain( Vs, 1, 15 ),
%         maplist( adjust_domain, DeltaNoOnes, Vs ), 
%         maplist( get_ex, Vs, DeltaNoOnes, Excesses ),
%         sum(Excesses, #=<, Excess ),
%         sum(Vs, #>=, MinToes ),
%         numlist(N,L),
%         sorting(Vs,L,Vs).
%     adjust_domain( 2, V ) :- V #=< 10.
%     adjust_domain( 3, _).
%     get_ex( V, 2, Ex ) :-
%         element( V, [0,0,0,0,0,0,2,3,7,10], Ex ).
%     get_ex( V, 3, Ex ) :-
%         element( V, [1,2,3,4,5,6,7,8,9,10,11,12,20,25,27], Ex ).
%     %%%%%%
    
%     %%% RESULT OUTPUT %%%
%     write_lotto_result( [], [], N, UB ) :-
%         print_message(informational, format(' L(~w,6,6,2) = ~w ',[N,UB]) ).
%     write_lotto_result( SlimIExceptions, DeltaExceptions, N, UB ) :-
%         append( SlimIExceptions, DeltaExceptions, TotalExceptions ),
%         length(TotalExceptions, Ex),
%         Ex #> 0,
%         print_message(informational, format('We conjecture that  L(~w,6,6,2) = ~w and must rule out the following cases',[N,UB]) ),
%         write_slimI_exceptions( SlimIExceptions ),
%         write_delta_exceptions( DeltaExceptions ).
    
%     write_slimI_exceptions( Exceptions ) :-
%         maplist( write_slimI_exception, Exceptions ).
%     write_slimI_exception( [R, [D2, D2] ] ) :-
%         D1 #= 6*R,
%         print_message(informational, format(' d_1 = ~w and d_2 = ~w ',[D1,D2]) ).
%     write_slimI_exception([R, [D2L, D2U] ] ) :-
%         D1 #= 6*R,
%         D2L #\= D2U,
%         print_message(informational, format(' d_1 = ~w and d2 in the range [~w,~w] ',[D1,D2L, D2U]) ).
    
%     write_delta_exceptions( Exceptions ) :-
%         maplist( write_delta_exception, Exceptions ).
%     write_delta_exception( Delta ) :-
%         delete( Delta, 2, Residue),
%         length( Residue, NumNoneTwos ),
%         NumTwos #= 5 - NumNoneTwos,
%         D2 #= NumTwos * 9,
%         print_message(informational, format(' d_2 <= ~w and \\delta(I) = ~w ',[D2, Delta]) ).
    
%     writeln( Stream ) :-
%             write( Stream ),
%             write('\n').
%     %%%%%%
%     \end{lstlisting}

% \end{appendices}


% \subsection{test}
% Suppose $I=I_1\cup I_2\cup I_3$ with $d(I_i)=i$ is an independent set. Let $y$ be a non-toe of degree $2$ appearing only once in $\B_{I_2}$; say $y\in A\in\B_{x_1}=\{A,B\}$. Then $y$ must appear in $C\in\B_{x_2}$ for some $x_2\in I_3$. Suppose $B$ contains a toe of degree $2$, $z$ say. Then claim $I\setminus\{x_1,x_2\}\cup\{y,z\}$ is an independent set. Certainly $y$ cannot lie in $B$ which contains $z$, and by virtue of being a toe, $z\not\in C$ so $y$ and $z$ are independent. The blocks $A$ and $C$ contain $x_1$ and $x_2$ respectively, so cannot contain any other element of $I$. As these are the only blocks containing $y$, we see that $y$ is independent from $I\setminus\{x_1,x_2\}$. Lastly since $z$ is an $x_1$-toe it is in just one block of $\B_I$, namely $B$. But that means it is independent from $I\setminus\{x_1,x_2\}$ too. This establishes the claim.

% In particular, there exists an independent set $I'$ with $|I'_2|>I_2$.

% Let's take the [2,2,2,3,3] case below. We have $d_{2}\geq 21$. On the other hand, there are at most $15$ toes of degree $2$, for $B_{x_i}$ for $1\leq i\leq 3$ has two blocks with toes split either $4,4$ or $5,3$. The former yields only $4$ of degree $2$ after webbing, and the latter $5$. Both cases leave just $2$ spaces in $B_{x_i}$ remaining. 


% Let $d_2=t_2+s_2+\delta_2$ where $t_2,s_2$ are the number of toes and untoes respectively. If $t_2+2s_2\geq 11\cdot\delta_2$ then there must be at least one untoe which appears in $B_{x_4,x_5}$.

% \subsection{The remaining cases}\label{remainingcases} The output from \lst{compute_lottery_numbers_in_range(30,62)} tells us that we must consider separately the following cases:
% \begin{enumerate}[label=(\roman*)] 
%     \item $n=58$, $j=25$, $s=0$, $d_2=24$, $\delta(I)=(2,2,2,3,3)$;
%     \item $n=58$, $j=25$, $s=1$, $24\leq d_2\leq 25$, $\delta(I)=(2,2,2,3,3)$;
%     \item $n=59$, $j=26$, $s=0$, $21\leq d_2\leq 24$, $\delta(I)=(2,2,2,3,3)$;
% \end{enumerate}

% We treat the case where $n=59$; the other two cases are similar and easier. So assume $j=26$, $s=0$, $21\leq d_2\leq 24$, and $I=\{x_1,\dots,x_5\}$ with $\delta(I)=(2,2,2,3,3)$. Since this is the only situation in which our constraints are satisfied, we have that there is no independent set $J$ of order $5$ with $4$ vertices of degree $2$. In particular, every vertex of degree $2$ appears in $\bigcup \B_{\{x_1,x_2,x_3\}}$: supposing $y$ had $d(y)=2$ and $y\not\in\bigcup \B_{\{x_1,x_2,x_3\}}$, then there is an independent set $\{x_1,x_2,x_3,y\}$ which can be extended to an independent set $I'$ with $\delta(I')=(2,2,2,2,3)$ or $(2,2,2,2,2)$ by \cref{1s2s3s}---but these cases have been ruled out. 

% By \cref{lowerboundd2}, we have $d_2\geq 21$, and \cref{mintoes} implies $|F_I|\geq 48$. Prolog informs us of the following possibilies for $\tau(I)$:

% \begin{lstlisting}
%     ?- populate_toes_in_Iblocks([2,2,2,3,3], 48, 36, Vs ), labeling([], Vs).
%     Vs = [7,8,9,12,12] ? ;
%     Vs = [8,8,8,12,12] ? ;
%     Vs = [8,8,9,11,12] ? ;
%     no
%     \end{lstlisting}

% These solutions lead to $\E(F_I)\geq 36$ except for $\tau(I)=(8,8,8,12,12)$ which implies only $\E(F_I)\geq 33$. Let us rule out the cases where $\E(F_I)\geq 36$.

% Any toe $y\in F_I$ occurs once in the blocks of $\B_I$ and all other occurrences are in blocks not in $\B_I$. Let $d_{i,\tau}$ denote the number of toes having degree $i$. Then  counted with multiplicy in $\BB$, the toes account for $T=\sum_{i\geq 2} id_{i,\tau}$ elements of $\BB$. We have $\E(F_I)=\sum (i-2)d_{i,\tau}$, and $\sum d_{i,\tau}=:n_\tau$ is the number of toes. So $T=\E(F_I)+2\cdot n_\tau$ and exactly $n_\tau$ of these entries come from $F_I$, so there are $\E(F_I)+n_\tau\geq 48+36=84$ elements lying in blocks not in $\B_I$. But $84/6=14$ so there are at least $14$ blocks not lying in $\B_I$. Since $|\B_I|=12$ and $|\B|=26$, this exhausts $\B$, implies $n_\tau=48$, that \begin{equation*}\E(F_I)=36=\sum_{1\leq i\leq 5}\E(F_{x_i}),\tag{$\dagger$}\end{equation*} and that the blocks $\B\setminus\B_I$ contain only toes. 


% Consider $F_{x_2}$, which, by ($\dagger$) contributes exactly $3$ to $\E(F_I)$. It is easy to see that this means there must be $5$ toes in $B$ and $3$ toes in $C$, say, where $\{B,C\}=\B_{x_2}$; also that there are exactly two $x_2$-webbings $B'$ and $C'$. 
% Let us label the vertices of $B$ and $C$ as $B=\{x_2,1,2,3,4,5\}$, $C=\{x_2,6,7,8,\_,\_\}$. Then up to an action of $S_5\times S_3$ permuting the toes, we may write $B'=\{1,2,3,6,7,8\}$ and $C'=\{4,5,6,7,8,z\}$, where $z$ is as yet undetermined. 
% But since $C'\not\in \B_I$ it follows that $z$ is a toe. We cannot have $z\in \{1,\dots,8\}$ since that would imply $E(F_{x_2})>3$. 
% Therefore $z\in F_{x_i}$ for $i\neq 2$. But as $|F_{x_i}|\geq k=6$, each $x_i$-toe must already appear once in $\B_{x_i}$ and again in some $D\in\B\setminus\B_i$ with some other $x_i$-toe. We conclude $D\neq C'$ and so $C'$ is not a webbing of $x_i$-toes. Since the webbings of $x_i$-toes account entirely for the occurrences of the $x_i$-toes outside of $\B_I$, we reach a contradiction.

% Therefore we must have $\E(X)<36$ and so $\tau(I)=(8,8,8,12,12)$. We use a similar counting argument as above: for $x=x_{4}$ or $x_5$, we have $|F_x|=12$ and these incur an excess $\E(F_x)\geq 12$. Since these $x$-toes must have degree at least $3$, it follows that these $12$ $x$-toes must appear at least $24$ times in the webbings $\W_x$. This implies that $|\W_x|\geq 4$ and if equality holds, then $\bigcup \W_x$ consists entirely of $x$-toes; moreover a closer inspection of the solutions found by Gurobi implies that the toes are distributed in such a way as to have $4$ in each $x$-block. Similarly, if $x\in\{x_1,x_2,x_3\}$ then $|\W_x|\geq 2$ and in the case of equality, there is at most $1$-non-toe in $\bigcup(\W_x)$. Since $\sum_{x\in I}|\W_x|\geq 14$ and there are $12$ $I$-blocks, we do indeed have equality everywhere. In particular, $|\mathcal{W}(x_4)|=4$. Up to isomorphism, there is just one configuration:
% \[\B_{x_4}=\{\{x_4,1,2,3,4,\_\},\{x_4,5,6,7,8,\_\},\{x_4,9,10,11,12,\_\}\}\]
% \[\mathcal{W}(x_4)=\{\{1,2,5,6,9,10\},\{1,2,7,8,11,12\},\{3,4,5,6,11,12\},\{3,4,7,8,9,10\}\}\]

% Now the vertices $1,\dots,12$ have degree at least $3$. If they all had degree at least $4$, we would have $\E(\F_{x_4})\geq 24$ which is impossible. So at least one, $1$ say, has degree $3$. Since $1$ represents a toe, we may swap $1$ with $x_4$ to create a new independent set $I'=\{x_1,x_2,x_3,1,x_5\}$. We may assume $\tau(I')=(8,8,8,12,12)$ since we have ruled out all other possibilities. Now, the set of $1$-blocks is $\B_1=\{\{x_4,1,2,3,4,\_\},\{1,2,5,6,9,10\},\{1,2,7,8,11,12\}\}$. There are $59-5-48=6$ elements which are neither toes nor elements of $I'$. But there are $12$ elements in the multiset $\BB_{I'}$ not toes or in $I'$, so each of these $6$ elements must appear exactly twice. However, $2$ is not a toe and appears $3$ times in $\B_1$, a contradiction.

% This completes the proof of the main theorem.
