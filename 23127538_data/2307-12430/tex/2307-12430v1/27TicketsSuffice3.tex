\documentclass[12pt,a4paper]{amsart}
\usepackage{amssymb,amsmath,amsfonts,amsthm,mathrsfs}
\usepackage[dvipsnames]{xcolor}
\usepackage{longtable,booktabs,colortbl,multicol,tikz,enumitem}
%\usepackage{color}

\usepackage[
    backend=biber,
    style=alphabetic,
    maxcitenames=5,
    maxbibnames=9,
    sorting=nyt,
    sortlocale=de_DE,
    natbib=true,
    url=false, 
    doi=true,
    eprint=true
]{biblatex}
\addbibresource{lotbib.bib}

\usepackage{graphicx}

\usepackage{float}

\graphicspath{ {./images/} }

\usepackage{fancyvrb}
\usepackage{comment}
\usepackage{appendix}
%\usepackage{geometry}

%\usepackage{cite}

\input{prologlst.tex}
\definecolor{OliveGreen}{rgb}{0.8,0.83,0.6}
\definecolor{burntumber}{rgb}{0.54, 0.2, 0.14}
\definecolor{coolblack}{rgb}{0.0, 0.18, 0.39}
\definecolor{darkterracotta}{rgb}{0.8, 0.31, 0.36}
\definecolor{frenchbeige}{rgb}{0.65, 0.48, 0.36}
\usepackage{hyperref}
\hypersetup{colorlinks=true,
linkcolor=NavyBlue,
    filecolor=OliveGreen,      
    urlcolor=burntumber,
    citecolor=darkterracotta,
    anchorcolor=frenchbeige}



\usetikzlibrary{calc}



\newcommand{\red}{\textcolor{red}}
\newcommand{\green}{\textcolor{OliveGreen}}
\newcommand{\blue}{\textcolor{blue}}
\newcommand{\eps}{\varepsilon}
\newcommand{\R}{\mathbb{R}}
\newcommand{\del}{\partial}
\renewcommand{\sl}{\mathfrak{sl}}
\newcommand{\psl}{\mathfrak{psl}}
\newcommand{\OO}{\mathcal{O}}
\newcommand{\CC}{\mathbb{C}}
\newcommand{\Z}{\mathbb{Z}}
\newcommand{\N}{\mathbb{N}}
\newcommand{\F}{\mathbb{F}}
\newcommand{\ud}{\mathrm{d}}
\newcommand{\DD}{\mathscr{D}}
\newcommand{\TW}{\mathrm{TW}}
\newcommand{\WT}{\mathrm{WT}}
\newcommand{\W}{\mathcal{W}}
\newcommand{\E}{\mathfrak{E}}
\newcommand{\BB}{\mathscr{B}}
\renewcommand{\O}{\mathcal{O}}
\newcommand{\lst}{\lstinline}

\DeclareMathOperator{\GF}{GF}
\DeclareMathOperator{\GL}{GL}
\DeclareMathOperator{\im}{im}
\DeclareMathOperator{\ad}{ad}
\DeclareMathOperator{\soc}{soc}
\DeclareMathOperator{\Der}{Der}
\DeclareMathOperator{\Hom}{Hom}
\DeclareMathOperator{\End}{End}
\DeclareMathOperator{\efc}{efc}
\DeclareMathOperator{\Char}{char}
\DeclareMathOperator{\cost}{cost}


%\usepackage[backend=biber, style=alphabetic, maxcitenames=5, maxbibnames=9,sorting=nyt]{biblatex}
%\addbibresource{References-new.bib}




\newcommand{\FF} {\mathbb{F} }

%\lstdefinestyle{mystyle}{
%	backgroundcolor=\color{backcolour},   
%	commentstyle=\color{codegreen},
%	keywordstyle=\color{magenta},
%	numberstyle=\tiny\color{codegray},
%	stringstyle=\color{codepurple},
%	basicstyle=\ttfamily\footnotesize,
%	breakatwhitespace=false,         
%	breaklines=true,                 
%	captionpos=b,                    
%	keepspaces=true,                 
%	numbers=left,                    
%	numbersep=5pt,                  
%	showspaces=false,                
%	showstringspaces=false,
%	showtabs=false,                  
%	tabsize=2
%}
%
%\lstset{style=mystyle} 

\lstset{%
	tabsize=4
}

\usepackage[capitalise]{cleveref}
%\thm`style{plain}
\newtheorem{thm}{Theorem}[section]
\newtheorem{prop}[thm]{Proposition}
\newtheorem{cor}[thm]{Corollary}
\newtheorem{lemma}[thm]{Lemma}
\crefname{mainthm}{thm}{Main Theorem}
\newtheorem*{mainthm}{Main Theorem}
%\theoremstyle{definition}
\newtheorem{defin}[thm]{Definition}
\newtheorem{ex}[thm]{Example}
\newtheorem{conj}[thm]{Conjecture}
\newtheorem{question}[thm]{Question}
\theoremstyle{remark}
\newtheorem{rmk}[thm]{Remark}
\newtheorem{rmks}[thm]{Remarks}



\setlength\parindent{0pt}
\setlength\parskip{5pt}

\setlength{\textwidth}{\paperwidth}
\addtolength{\textwidth}{-2in}
\setlength{\textheight}{\paperheight}
\addtolength{\textheight}{-2in}
\calclayout 

%\DeclareTextFontCommand{\emph}{\bfseries\itshape}
\newcommand{\degg}{{\rm deg}}
\newcommand{\reg}{{\rm reg}}
\newcommand{\gl}{{\mathfrak{gl}}}
\newcommand{\fsl}{{\mathfrak{fsl}}}
\newcommand{\s}{{\mathfrak{s}}}
\newcommand{\diag}{\operatorname{diag}}
\newcommand{\gws}[1]{\textcolor{red}{#1}}
\newcommand{\B}{\mathcal{B}}
\newcommand{\C}{\mathcal{C}}

\title{ You need 27 tickets to guarantee a win \\on the UK National Lottery }
%\date{June 2023}

    \author[Cushing]{David Cushing}
    \address{Department of Mathematics, The University of Manchester, Manchester, UK}
    \email{david.cushing@manchester.ac.uk}

    %     \author[Meg]{Mystic Thorin Meg}
    % \address{Department of Mathematics, The University of Manchester, Manchester, Great Britain}
    % \email{mystic.meg@manchester.ac.uk}
    
    \author[Stewart]{David I. Stewart}
    \address{Department of Mathematics, The University of Manchester, Manchester, UK}
    \email{david.i.stewart@manchester.ac.uk}



\begin{document}


\begin{abstract}
In the UK National Lottery, players purchase tickets comprising their choices of six different numbers between 1 and 59. During the draw, six balls are randomly selected without replacement from a set numbered from 1 to 59. A prize is awarded to any player who matches at least two of the six drawn numbers. We identify 27 tickets that guarantee a prize, regardless of which of the 45,057,474 possible draws occurs. Moreover, we determine that 27 is the optimal number of tickets required, as achieving the same guarantee with 26 tickets is not possible.
\end{abstract}


\maketitle

\section{Introduction}
A lottery is a popular method of gambling in which players try to guess random numbers in advance of their generation. For our purposes, a lottery revolves around sampling random draws of size $p$ from the set of numbers $\{1,\dots,n\}$; participants purchase tickets containing their choice of a subset of $k$ of these numbers and will win a prize if one of their tickets has at least $t$ numbers in common with those drawn. A well-studied question is to determine the minimum number of tickets, denoted $j=L(n,k,p,t)$, that must be purchased to guarantee that no matter which numbers appear in the random draw, at least one of the $j$ tickets will have at least $t$ numbers in common with the draw.

Calculating these numbers has proven to be challenging, particularly for lotteries that are actively played worldwide. One common set of parameters is $(n,k,p)=(49,6,6)$, an example of which was the UK national lottery upon its introduction in 1994. %In a paper by Bate and van Rees \cite{BateRees}, the authors computed the values of $L(n,6,6,2)$ for $n\leq 54$. However, it is unfortunate that the UK national lottery in this form only offers a prize when $t\geq 3$. The current best-known bounds for winning a prize in this lottery are $89 \leq L(49,6,6,3) \leq 163$. Therefore, the minimum number of tickets required to guarantee a prize in this lottery remains unknown.
In October 2015, the UK national lottery underwent a change in which $n$ was increased to $59$. By way of compensation, a prize was introduced for matching $2$ balls on a ticket. Consequently, the value of $L(59,6,6,2)$ is somewhat significant.

\begin{mainthm}\label{mainthm}
    We have $L(59,6,6,2) = 27.$
\end{mainthm}

%Or in words ``You need $27$ tickets to guarantee a win on the UK National Lottery".
In \cref{sec:ticks} we give an explicit list of $27$ tickets which are always guaranteed to win some prize; we are pleased that we could describe this rather elegantly using some basic structures coming from finite geometry. The hard part in proving the theorem is to show that there does not also exist a set of $26$ tickets that work. To this end, the best available lowest bound in the literature is to be found in \cite{Furedi}:

\begin{thm}[Furedi--Sz\'ekely--Zubor]\label{FurediBound} We have 
    \[L(n, k, p, 2) \geq \frac{1}{k}\cdot\min_{\sum_{i=1}^{p-1} a_{i} = n }\left(  \sum_{i=1}^{p-1} a_{i}\left\lceil \frac{a_{i} - 1}{k-1} \right\rceil\right).\]
\end{thm}

Applied to our situation this yields only $L(59,6,6,2)\geq 23$, which is some distance from the true value. In the foundational paper, \cite{BateRees}, the authors compute $L(n,6,6,2)$ for $n \leq 54$. We bolster the techniques of \textit{op.~cit.~}theoretically and computationally, recovering and extending the results therein to calculate all values of $L(n,6,6,2)$ for $n\leq 61$; see \cref{alltogetherthm}. Notably, we make substantial use of the constraint programming library \cite{COC97} in SICStus Prolog\footnote{A free evaluation copy of SICStus can be downloaded from \url{https://sicstus.sics.se/}} \cite{sicstus}. In that respect, this paper continues a programme of work the authors began in \cite{CSS} to apply Prolog and constraint programming to questions arising in pure mathematics. See \textit{op.~cit.}~for a discussion of Prolog.

\begin{rmk}Having observed that the set of tickets we describe below would have netted the authors \pounds 1810 in the lottery draw of 21 June 2023, the authors were motivated to road-test the tickets in the lottery draw of 1 July 2023; they matched just two balls on three of the tickets, the reward being three lucky dip tries on a subsequent lottery, each of which came to nothing. Since a ticket costs \pounds 2, the experiment represented a loss to the authors of \pounds 54.

This unfortunate incident therefore serves both as a verification of our result and of the principle that one should expect to lose money when gambling. 

For a more philosophical discussion of the National Lottery and its implementation for supporting charitable causes, we recommend David Runciman's article in the London Review of Books \cite{runc}.\end{rmk}

\section{The 27 Tickets}\label{sec:ticks}
An appealing way to construct $27$ tickets that will match two numbers no matter which balls are drawn is through labelling the vertices of some finite geometrical structures as in \cref{finitegeom}.

Perhaps the most well-known finite geometry is the projective plane of order $2$---also known as a Fano plane---which is depicted by the three diagrams in \cref{finitegeom} which have $7$ vertices. There are $7$ `lines' (one being represented by a circle) that satisfy the property that any two points lie in exactly one line, and two lines intersect in exactly one point. Tickets may be read off the Fano plane diagrams by concatenating the labels on the points in each of the $7$ lines. (Together this accounts for $21$ of the tickets.)

The triangle can be viewed as the projective plane of order $1$; it supplies $3$ tickets by writing down each of the two labels on its three lines.

Lastly, the triangle with a point in the middle is a sort of `decompactified' projective plane of order $1$: every line should be considered also to go through the point $\varnothing$; this gives us the final $3$ tickets.

% Figure environment removed

Explicity, we get:
\begin{center}
\scalebox{0.9}{
\begin{tabular}{ c c c c c } 
\cellcolor{lightgray} $1,2,3,4,5,6$ & $9,10,11,12,13,14$ & \cellcolor{lightgray} $18,19,20,21,26,27$ & $32,33,34,35,40,41$ & \cellcolor{lightgray} $46,47,48,49,54,55$\\ 
\cellcolor{lightgray} $1,2,3,4,7,8$ & $9,10,11,15,16,17$ & \cellcolor{lightgray} $18,19,22,23,30,31$ & $32,33,36,37,44,45$ & \cellcolor{lightgray} $46,47,50,51,58,59$\\
\cellcolor{lightgray} $1,2,5,6,7,8$ & $12,13,14,15,16,17$ & \cellcolor{lightgray} $18,19,24,25,28,29$ & $32,33,38,39,42,43$ & \cellcolor{lightgray}  $46,47,52,53,56,57$ \\
\cellcolor{lightgray}  &  & \cellcolor{lightgray} $20,21,22,23,28,29$ & $34,35,36,37,42,43$ & \cellcolor{lightgray} $48,49,50,51,56,57$\\
\cellcolor{lightgray}  &  & \cellcolor{lightgray} $20,21,24,25,30,31$ & $34,35,38,39,44,45$ & \cellcolor{lightgray} $48,49,52,53,58,59$\\
\cellcolor{lightgray}  &  & \cellcolor{lightgray} $22,23,24,25,26,27$ & $36,37,38,39,40,41$  & \cellcolor{lightgray} $50,51,52,53,54,55$\\
\cellcolor{lightgray}  &  & \cellcolor{lightgray} $26,27,28,29,30,31$ & $40,41,42,43,44,45$ & \cellcolor{lightgray} $54,55,56,57,58,59$
\end{tabular}
}
\end{center}

The reader may check for themselves that any draw of $6$ numbers from $1$ to $59$ will match at least two numbers with at least one of the above tickets.

\section{Definitions and notation} 
A \emph{hypergraph} $H$ is a pair $(X,\B)$ with $X$ a set and $\B$ a set of subsets of $X$. We will refer to the elements $x\in X$ as \emph{vertices} and the elements $B\in\B$ as \emph{blocks}. The \emph{order} of $H$ is the cardinality of the set $X$. The \emph{size} of $H$ is the number of blocks in $\B$;~i.e. the cardinality of the set $\B$. A hypergraph is said to be \emph{$k$-uniform} if each $B\in\B$ has size $k$. Note that a $2$-uniform hypergraph is a graph: a block $B=\{x,y\}$ identifies with an edge $x-y$.

An \emph{$(n,k,p,t;j)$-lottery design} is a hypergraph $H=(X,\B)$, such that:
\begin{enumerate}[label=(\roman*)]\item there are $n$ vertices in $X$---i.e.~$H$ has order $n$;
\item there are $j$ elements $B\in \B$---i.e.~$H$ has size $j$;
\item each $B\in\B$ contains $k$ vertices---i.e.~$H$ is $k$-uniform;
\item for any subset $D$ of vertices with $|D|=p$, there is at least one $B\in \B$ such that $|B\cap D|\geq t$.\label{condiv}
\end{enumerate}
An \emph{$(n,k,p,t)$-lottery design} is an $(n,k,p,t;j)$-lottery design in which $j$ is minimal; we denote this minimal integer by $L(n,k,p,t)$.

\begin{rmk}Thinking of $X$ as a set of balls, $\B$ as a set of tickets, and $D$ as a draw, condition \ref{condiv} says that at least one ticket in $\B$ matches at least $t$ of the balls in $D$.\end{rmk}

In order to avoid vacuous or incorrect statements we shall always insist $n\geq k\geq t\geq 2$.

We say a block $B$ is an \emph{$x$-block}---or that $B$ is \emph{incident} with a vertex $x$---if $x\in B$. The set $\B_x$ of all $x$-blocks is the \emph{star} of $x$. We define the function \[d:X\to\Z_{\geq 0};\ x\mapsto|\B_x|\] so that $d(x)$ is the \emph{degree} of $x$; i.e.~the number $|\B_x|$ of blocks incident with $x$. More generally, if $I$ is any subset of the vertices, then $B$ is an $I$-block if it is an $x$-block for some $x\in I$. We let $\B_I=\bigcup_{x\in I} \B_x$, and $d(I)=\bigcup_{x\in I} d(x)$ its degree. We will denote by $d_i:=|d^{-1}(i)|$ the number of vertices of degree $i$ in $H$. We also need to analyse the multiset of degrees of vertices in a subset $I\subseteq X$, so we define the function from the power set $\mathcal{P}(X)$ of $X$ to non-decreasing sequences of non-negative integers: \[\delta : \mathcal{P}(X)\to \bigcup_{r=0}^n(\Z_{\geq 0})^r;\ I\mapsto (d(x_1),\dots,d(x_n)),\text{ such that } d(x_1)\leq d(x_2)\leq \dots\leq d(x_n).\]
It will be convenient to label the elements of  $I$ consistent with the output of $\delta$, so for example we may have $I=\{x_1,\dots,x_5\}$ such that $\delta(I)=(2,2,3,3,4)$ indicating $d(x_1)=2,\dots,d(x_5)=4$. 

Two vertices $x,y\in X$ are \emph{adjacent} if they are contained in a common block. The set of all vertices adjacent to $x$ is its \emph{neighbourhood} $N(x)$. A subset $I$ of $X$ is an \emph{independent set} or \emph{coclique} if no pair of elements in $I$ are adjacent. An independent set is \emph{maximal} if there is no independent set $J\subset X$ with $I\subsetneq J$.

\begin{rmks}\label{indsetrems}(i) If $I$ is an independent set in an $(n, k, p, t; x)$-lottery design, then $|I|<p$ or else there would be a draw $D=I$ matching at most $1<t$ vertex in any block. 
    
(ii) If $I$ is a maximal independent set then we must have $\B_I=X$ or else there is another element $y\in X$ which is not adjacent to any $x\in I$; which would imply there were a larger maximal independent set $X\cup \{y\}$.

(iii) It is easy to see that maximal independent sets need not have the same order. For example, in the graph $x-y-z$, $\{y\}$ is maximal independent, but $\{x,z\}$ is a  maximal independent set of higher cardinality. \end{rmks}

A set $\B'\subset\B$ of blocks is \emph{disjoint} if $B_1\cap B_2=\emptyset$ for any $B_1,B_2\in\B'$. In the literature, such a $\B'$ is also referred to as a \emph{matching}. A block $B\in \B$ is \emph{isolated} if $B$ is disjoint from any other block in $\B$. (Of course, each vertex of an isolated block has degree $1$.) If $d_0=0$ and each vertex of degree $1$ is contained in an isolated block, then we say $H$ is \emph{segregated}.

We will show later that $H$ may be assumed segregated under the hypotheses of our main theorem. In that case, the interesting analysis reduces to the full subhypergraph induced by the vertices of degree at least $2$. With that in mind, we give a further couple of definitions which will prove central to our argument. Let $I$ be an independent set. Then an \emph{$I$-toe} is a vertex of degree at least $2$ appearing in just one block of $\B_I$. The \emph{$I$-foot} $F_I$ is the set of $I$-toes. If $I=\{x\}$ we refer more succinctly to $x$-toes. (Note that an $x$-toe may fail to be an $J$-toe for $\{x\}\subsetneq J$.) If the independent set $I$ is clear from the context, then we will simply refer to toes and feet. So suppose $I=\{x_1,\dots,x_\ell\}$, with $\delta(I)=(d(x_1),\dots,d(x_\ell))$, where $d(x_1)>1$. Then set 
\[\tau(I)=\left(\left|F_I\cap \bigcup\B_{x_1}\right|,\dots,\left|F_I\cap\bigcup\B_{x_\ell}\right|\right),\]
which gives the distribution of the toes among the $x$-blocks as $x$ ranges over the elements of $I$. 

\section{Preliminaries}
We use this section to collect a miscellany of results on hypergraphs and lottery designs of varying generality, which we use in the sequel. Our strategy is heavily influenced by the paper \cite{BateRees} and we use most of its results in one form or another. However, we noticed a number of infelicities among the statements and proofs in \emph{op.~cit.}~and so we take the opportunity here to correct and simplify  the statements and proofs therein. Happily, it follows this paper is largely self-contained. 

\subsection{Upper bounds} An \emph{$(n,k,t)$-covering design} is a $k$-uniform hypergraph $H=(X,\B)$ such that every subset of $X$ of size $t$ appears as a subset in at least one block of $\B$; we assume $n\geq k\geq t$. Define $C(n,k,t)$ as the minimal size of an $(n,k,t)$-covering design. Obviously, an $(n,k,t)$-covering design is an $(n,k,t,t)$-lottery design, so $L(n,k,p,t)\leq C(n,k,t)$. 
\begin{lemma}\label{upperbound} We have 
    \[L(n, k, p, 2) \leq \min_{\sum_{i=1}^{p-1} a_{i} = n}\left(  \sum_{i=1}^{p-1} C( a_{i}, k, 2) \right).\]
\end{lemma}
\begin{proof} Let $H_i=(X_i,\B_i)$ be $(a_i,k,2)$-covering designs of size $C(a_i,k,2)$ such that $X=\bigsqcup(X_i)$, and let $D$ be a draw of $p$ elements. Then at least two elements of $D$ lie in at least one $X_j$, so one block $B$ of $\B_i$ contains those two elements.\end{proof} 

\cref{upperbound} can be deployed using the entries from \cref{coveringnumbers}. The table lists upper bounds for $C(n,6,2)$ for $n\geq 6$ which were harvested from \url{https://ljcr.dmgordon.org/cover.php}; these upper bounds are all known to be sharp except when $n=23$ or $24$.
\begin{table}
\begin{tabular}{|c|c||c|c|}\hline
    $n$ & $C(n,6,2)\leq$ & $n$ & $C(n,6,2)\leq$\\\hline
    $6$ & $1$ & $17$ & $12$\\
    $7$ & $3$ & $18$ & $12$\\
    $8$ & $3$ & $19$ & $15$\\
    $9$ & $3$ & $20$ & $16$\\
    $10$ & $4$ & $21$ & $17$\\
    $11$ & $6$ & $22$ & $19$\\
    $12$ & $6$ & $23$ & $21$\\
    $13$ & $7$ & $24$ & $22$\\
    $14$ & $7$ & $25$ & $23$\\
    $15$ & $10$ & $26$ & $24$\\
    $16$ & $10$ & $27$ & $27$\\
    \hline
\end{tabular}\caption{Upper bounds for $C(n,6,2)$}\label{coveringnumbers}
\end{table}

Secondly, it is useful to know that lottery numbers increase mononotonically with $n$.

\begin{lemma}\label{doesntgodown}
    We have $L(n,k,p,t)\leq L(n+1,k,p,t)$.
\end{lemma}
\begin{proof}
    Take an $(n+1,k,p,t;j)$-lottery design $H=(X,\B)$ with $|\B|=j$. Pick $x\in X$ and construct $j$ subsets $\C$ of $X\setminus\{x\}$ from those of $\B$ by replacing $x\in B$ with any other vertex of $X\setminus B$, where necessary. Then it is clear that the hypergraph $H_0=(X\setminus \{x\},\C)$ is an $(n,k,p,t;j)$-lottery design.
\end{proof}


    
\subsection{Reductions and constraints}

\begin{lemma}\label{BR1}
    Let $H=(X,\B)$ be an $(n, k, p, t; j)$-lottery design  with $j \geq n/k$. Then there exists an $(n, k, p, t; j)$-lottery design $H_0=(X,\B_0)$ with $\bigcup\B_0=X$; i.e.~there are no elements of degree $0$.
        \end{lemma}
        \begin{proof}Let $H$ be a counterexample with $d_0>0$ minimal. Suppose $x\in X$ has degree $0$. Since $jk\geq n$, there must be $y\in X$ with $d(y)\geq 2$. Suppose $y\in B$ for some block $B$ of $\B$ and set $B_0=(B\setminus\{y\})\cup \{x\}$. Now set $H_0=(X,\B_0)$, where $\B_0$ is $\B$ with the block $B$ replaced by $B_0$. Then $H_0$ is a $k$-uniform hypergraph of order $n$ and size $j$. If $D$ is any subset of $X$ of order $p$, then either there is $C\in\B_0$ with $|C\cap D|\geq t$ or we may assume $|B\cap D|=t$, $y\in D$ and $|B_0\cap D|=t-1$. This implies also $x\not \in D$; but then the alternative draw $(D\setminus\{y\})\cup\{x\}$ cannot intersect a block of $\B$ in $t$ elements.\end{proof}
\begin{lemma}\label{segregated}
If $H=(X,\B)$ is an $(n, k, p, t; j)$-lottery design with $n \geq k(p-1)$ and $d_0=0$, then there exists a segregated $(n, k, p, t; j)$-lottery design.
\end{lemma}
\begin{proof}
%We need to show exhibit a lottery design with $d_1=rk$. 
Suppose there are $r$ isolated blocks in $H$. Then taking one vertex from each yields an independent set $I$, so $r\leq p-1$ by \cref{indsetrems}(i). If $r=p-1$ then the isolated blocks supply all $kr=k(p-1)$ vertices of $X$ and the statement holds. 

Suppose $r=p-2$; then there are $n-k(p-2)$ elements not in isolated blocks. A draw of order $p$ containing one vertex from each isolated block together with two non-isolated vertices can only intersect a non-isolated block in at least $t$ places. Thus the non-isolated blocks must between them contain every pair of the non-isolated vertices. This means that each appears at least twice, or $n-k(p-2)=2$ and they both appear exactly once. But the latter says they are themselves in an isolated block, a contradiction.

Hence we may assume $r\leq p-3$, leaving at least $2k$ elements not in isolated blocks. Let $B$ be a non-isolated block and assume that there are $x,y\in B$ with $d(x)=1$ and $d(y)>1$. We modify $H$ to give a lottery design $H_0$ with $d(y)=1$. By an evident induction, this implies the existence of the required design.

Let $B=B_1,\dots,B_\ell$ be the blocks containing $y$. For $2\leq i\leq \ell$, find a non-isolated element $z_i$ such that $z_i\not\in B_i$; this is possible since there are at least $2k$ non-isolated elements. Form $C_i$ by replacing $y$ with $z_i$ and let $C_1=B_1$. Then we claim we get a new $(n,k,p,t;j)$-lottery design $H_0=(X,\C)$ with $d_0=0$ by letting $\C=(\B\setminus\{B_1,\dots,B_\ell\})\cup\{C_1,\dots,C_\ell\}$.

To prove the claim, take a draw $D$ and assume $D$ does not intersect any block $C\in\C$ in at least $t$ elements. Then we may assume $D$ contains $y$, $|D\cap B_i|=t$ for some $2\leq i \leq l$ and $|D\cap C_i|=t-1$.  Furthermore, since $d(x)=1$, if $x\in D$, then $D\cap B_1=D\cap C_1$ has at least $t$ elements, so we may assume $x\not\in D$. Then replacing $y$ with $x$ in $D$ gives a draw $D_0$ which intersects no block of $\B$ in at least $t$ elements.
\end{proof}


The above results are essentially the same as \cite[Lem.~3.2, Thm.~3.5]{BateRees}. In between these is \cite[Lem.~3.4]{BateRees} which shows (correctly) that given an $(n, k, p, 2)$-lottery design with $n>k(p-2)$, then there is another with a maximal independent set of size $p-1$. It is combined with the above two results to claim the existence of a lottery design satisfying the conclusions of all three results. Unfortunately the methods of proof go by altering the degrees of vertices in the design, and it is unclear whether this can be done compatibly.

We resolve this issue in \cref{1s2s3s} below, with a stronger result. As in \emph{op.~cit.}, we make use of the following lemma, which is an application of the main theorem of \cite{Sha49}. One should also note that \cite{BateRees} implicity assumes, without proof, that the independent set guaranteed by \cref{ninthdeg2} can be enlarged to one of maximal cardinality; in fact, this can actually fail, and results in a separate case to check---\textit{viz.}~Case \ref{casei} in \cref{remainingcases} below.

\begin{lemma} \label{ninthdeg2}
    Let $(X,\B)$ be a $k$-uniform hypergraph. Then there exists an independent set $I$ containing at least $\frac{d_2}{\lfloor 3k/2\rfloor}$ vertices of degree $2$.
\end{lemma}

% \begin{rmk} Can replace elts of frequency $2$ by elts of frequency $r$ in the above. Get an independent set containing at least $\frac{d_r}{lfloor \frac{3(r-1)k}{2}\rfloor}$ elements. If $r=3$ get $\frac{d_3}{3k}$ elements in the set. 

%     Doing this with elts of degree 2 and 3 we seem to get $\frac{2d_2+3d_3}{18}$. So that gives a lower bound for $\delta_2+\delta_3$. We can also maybe use $\sum i d_i=jk$, or $2d_2+3d_3\leq 2\delta_2+3\delta_3$. \dots\end{rmk}

\begin{prop}\label{1s2s3s} Suppose $H=(X,\B)$ is a segregated $k$-uniform hypergraph with $|X|=n$, $|\B|=j$, $d_1=kr$ and $\pi$ is the maximal order of an independent set in $H$. Then if 
    \[\tag{*} 4n-jk-(3k-2)(\pi-1)+r(\lfloor k/2\rfloor -1)-(\pi-1)\cdot(\lfloor k/2\rfloor+1)>0,\]
there is a (maximal) independent set $I\subseteq X$ of order $\pi$ consisting entirely of vertices whose degrees are $1$, $2$ or $3$ and containing at least $\lceil\frac{d_2}{\lfloor\frac{3k}{2}\rfloor}\rceil$ vertices of degree $2$. 

Furthermore if the left-hand side of the inequality (*) is instead equal to zero then there is an independent set $I$ of order $\pi-1$ with $d(I)\in\{1,2\}$, all vertices have degree at most $4$, and $d_2=(\pi-1-r)\lfloor \frac{3k}{2}\rfloor$.\end{prop}

Before giving the proof of the proposition, let us make some elementary observations that are used several times in the sequel. 

Let $H=(X,\B)$ be a $k$-uniform hypergraph with $|X|=n$ and $|\B|=j$. Recall $d_i$ is the number of vertices having degree $i$. Clearly

\begin{equation}\label{setsum}
    \sum_{i\geq 0}d_i=n.
\end{equation}

Let $\BB$ denote the multiset $\bigsqcup B_i$, which contains the vertices of the hypergraph counted with their multiplicities in the blocks $\B$. As there are $j$ blocks we have $|\BB|=jk$. Of course a vertex $x$ will occur in exactly $d(x)$ of the blocks, and so: 

\begin{equation}\label{multisetsum}
    \sum_{i\geq 1}id_i=jk.
\end{equation}




\begin{proof}[Proof of \cref{1s2s3s}] Suppose for a contradiction that there is no independent set $I$ of order $\pi$ with $d(I)\subset\{1,2,3\}$.  We will show this leads to a contradiction. 
    
For a given independent set $I$, we denote by $\delta_i=d^{-1}(i)\cap I$ the number of elements of degree $i$ in $I$. By \cref{ninthdeg2} there exists an independent set with 
\begin{equation}\delta_2\geq \frac{d_2}{\lfloor\frac{3k}{2}\rfloor}.\label{shannonbound}\end{equation} Let $J$ be the union of this set together with one element chosen from each of the $r$ isolated blocks. Now take $I\supseteq J$ maximal subject to $d(I)\leq 3$; therefore  $\delta_1+\delta_2+\delta_3\leq \pi-1$. 
Note that by segregation, the $\delta_1=r$ isolated $I$-blocks account for all elements in $X$ of degree $1$. 

We now bound from above and below the number $d_2+d_3$ of vertices in $X$ having degrees $2$ or $3$. 

Since $I$ is maximal, we have $X=\bigcup\B_I$. By segregation, no vertex of degree $2$ or $3$ is adjacent to a vertex of degree $1$ and so it must live in $\bigcup\B_x$, with $x$ running over the elements of $I$ of degree $2$ or $3$. The latter contains at most $(2k-1)\delta_2+(3k-2)\delta_3$ distinct elements. Using $\delta_1+\delta_2+\delta_3\leq \pi-1$, we get 
\begin{align}d_2+d_3&\leq (2k-1)\delta_2+(3k-2)\delta_3\\
    &\leq (2k-1)\delta_2+(3k-2)(\pi-1-r-\delta_2)\\
    &\leq (1-k)\delta_2+(3k-2)(\pi-1-r)\label{combine}
\end{align}


On the other hand, consider the multiset $\BB=\bigsqcup_{B\in\B} B$, of order $jk$. The $I$-blocks contribute exactly $\delta_1k+2\delta_2k+3\delta_3k$ of these elements. Since there are $2d_2+3d_3$ elements of degrees $2$ or $3$, there are $2\delta_2k+3\delta_3k-2d_2-3d_3$ elements of $\BB$ of order at least $4$ coming from the $I$-blocks. The remaining $(j-\delta_1-2\delta_2-3\delta_3)$-blocks all contain elements of order at least $4$. Thus the multiset $|\BB_{\geq 4}|$ of all elements of degrees $4$ and above has order
\[|\BB_{\geq 4}|=(j-\delta_1-2\delta_2-3\delta_3)k + 2\delta_2k+3\delta_3k-2d_2-3d_3 =jk-rk-2d_2-3d_3.\] 
Since $|\BB_{\geq 4}|=\sum_{i\geq 4} id_i$, we get 
\begin{equation}\sum_{i\geq 4} d_i \leq \frac{jk-rk-2d_2-3d_3}{4}.\end{equation}
Now $n=\sum d_i\leq rk+d_2+d_3+\frac{jk-rk-2d_2-3d_3}{4}$. 

Rearranging, we get:
\begin{equation}
    d_2+d_3\geq 4n-jk-d_2-3rk,
\end{equation}

which combines with (\ref{combine}) to give 
\[(1-k)\delta_2+(3k-2)(\pi-1-r) \geq 4n-jk-d_2-3rk.\]
Now using \cref{shannonbound} we get 
\[\delta_2\cdot\lfloor 3k/2\rfloor-(k-1)\delta_2\geq 4n-jk-(3k-2)(\pi-1)-2r.\]
Using $\delta_1+\delta_2\leq \pi-1$, with equality if and only if $\delta_3=0$, then after some manipulation, we get
% (r+\delta_2)\cdot(\lfloor k/2\rfloor+1) &\geq 4n-jk-(3k-2)(\pi-1)-2r+r(\lfloor k/2\rfloor+1) \\
% (r+\delta_2)\cdot(\lfloor k/2\rfloor+1) &\geq 4n-jk-(3k-2)(\pi-1)+r(\lfloor k/2\rfloor -1) \\
% (\pi-1)\cdot(\lfloor k/2\rfloor+1) &\geq 4n-jk-(3k-2)(\pi-1)+r(\lfloor k/2\rfloor -1) \end{align*}
% Hence,
\[4n-jk-(3k-2)(\pi-1)+r(\lfloor k/2\rfloor -1)-(\pi-1)\cdot(\lfloor k/2\rfloor+1)\leq 0,\]
which is a contradiction, proving the first statement of the proposition.

If $4n-jk-(3k-2)(\pi-1)+r(\lfloor k/2\rfloor -1)-(\pi-1)\cdot(\lfloor k/2\rfloor+1)=0$ then we must have had equality in all the previous inequalities that were used, including \cref{shannonbound}. Thus, $\delta_2\lfloor 3k/2\rfloor=d_2$, $\delta_3=0$, $\pi-1=\delta_1+\delta_2$ and all vertices are degree at most $4$.
\end{proof}

The following results are all used to give constraints to Prolog. The first two respectively bound above and below the number of isolated blocks.

\begin{lemma}\label{rbound}
    Let $(X, \mathcal{B} )$ be a segregated $(n, k, p, 2; j)$-lottery design. Suppose that $\B$ has $r$ isolated blocks. Then,
    $$ r \geq \left\lceil\frac{2n-jk}{k} \right\rceil. $$
 \end{lemma}
\begin{proof}We have $jk=d_1+\sum_{i\geq 2} id_i \geq d_1+2(n-d_1)$. Write $d_1=rk$ and rearrange to get the formula above.
\end{proof}

\begin{lemma}\label{rboundforfuredi}Let $H=(X,\B)$ be a segregated $(n, k, p, t; j)$-lottery design with at least $r$ isolated blocks. Then $$r\leq j-L(n-rk,k,p-r,t).$$\end{lemma}
\begin{proof}Suppose $B_1,\dots,B_r$ are isolated and let $Y=X\setminus \bigcup_{1\leq i\leq r}B_i$ be the set of non-isolated vertices and $\C=\B\setminus\{B_1,\dots,B_r\}$; so $(Y,\C)$ is a hypergraph with order $n-rk$ and size $j-r$. Fix one vertex $v_i\in B_i$ for each $i$. For any choice $\{v_{r+1},\dots,v_p\}$ of vertices from $Y$, the draw $D=\{v_1,\dots,v_p\}$ intersects with some $B\in \B$ in at least $t$ vertices. Clearly we cannot have $B=B_i$ for $1\leq i\leq r$ and so $B\in\C$. But this means no $v_1,\dots, v_r$ is in $B$, so $B$ must match $t$ of the remaining $p-r$ elements of $D$. Hence $(Y,\C)$ is an $(n-rk,k,p-r,t;j-r)$-lottery design, which implies the inequality as shown.\end{proof}



\begin{prop}Let $H$ be a segregated $k$-uniform hypergraph. Then 
    \[d_2\geq 3n-2rk-jk\]
\label{lowerboundd2}\end{prop}
\begin{proof}We recall $\sum d_i=n$ and $\sum id_i=jk$, which implies 
    \[jk-n=d_2+\sum_{i\geq 3}(i-1)d_i\geq d_2+2\sum d_i=-2d_1-d_2+2n.\qedhere\]\end{proof}

\begin{lemma}\label{minnumiblocks}
    Let $(X,\B)$ be a $k$-uniform hypergraph with maximal independent set $I$. Then 
    %segregated $(n,k,p,2;j)$-lottery design with maximal independent set $I$. Then 
    $$|\B_I|\geq \left\lceil\frac{n-p+1}{k-1}\right\rceil.$$
\end{lemma}
\begin{proof}
    We must have $\bigcup \B_I=X$ or $I$ is not maximal. For $x\in I$, two blocks $B,C\in\B_x$ intersect in at least $x$ so $|\bigcup \B_x|\leq (k-1)|\B_x|-1$. Since $|I|\leq p-1$, then summing over $x\in I$ yields the statement of the lemma.
\end{proof}






\subsection{Excess, toes and webbings}For the values of $(n,k,p)$ under consideration, one tends to find $L(n,k,p,2)\sim n/2$. Thus, on average, the degree of a vertex in an $(n,k,p,2)$-lottery design is about $2$. The next definition follows \cite{BateRees} with a view to bounding the extent of departure from this average value.
\begin{defin}For a set of vertices $Y\subset X$,  the \emph{excess} of $Y$ is the sum \[\E(Y)=\sum_{i>2}(i-2)\cdot |d^{-1}(i)\cap Y |.\]\end{defin}

Note that if $Y=X$, we get $\E(X)=\sum_{i>2}(i-2)\cdot d_i$. The following gives an easy characterisation of $\E(X)$.

\begin{lemma}Let $H=(X, \mathcal{B} )$ be a segregated hypergraph with $r$ isolated blocks. Then
    \[\E(X)=jk+rk-2n.\]\label{excessfromisolated} \end{lemma}
   \begin{proof}We have $jk=\sum id_i$. Since $n=\sum d_i$, we get \[jk-2n=-d_1+\sum_{i\geq 3}(i-2) d_i=-d_1+\E(X).\qedhere\]
\end{proof}


The possible number of toes are constrained by the value of the excess $\E(X)$, in a manner we now describe. First, the following is \cite[Lem.~4.7]{BateRees} after the removal of a significant typo. 

\begin{lemma}\label{mintoes}
    Let $(X, \mathcal{B} )$ be a segregated $(n, k, p, 2; j)$-lottery design with maximal independent set $I$ and $F_I$ its $I$-foot. Suppose further that $\B$ has $r$ isolated blocks. Then,
    $$ |F_I| \geq 2n - 2p +2 -(k-1)(|\B_I|+r).$$
\end{lemma}
\begin{proof}As $I$ is maximal, we have $\bigcup\B_I=X$. The multiset \[\mathscr{R}:=\bigsqcup_{x\in I,d(x)>1,B\in \B_x} B\setminus\{x\}\] contains the $kj-rk-|I|$ vertices which are neither isolated nor contained in $I$, and by definition, the toes are the elements of multiplicity $1$ in $\mathscr{R}$. Therefore each of the remaining $kj-rk-|I|-|F_I|$ vertices appears at least twice in $\mathscr{R}$. Since $|\mathscr{R}|=(k-1)(|\B_I|-r)$, we have 
    \[|F_I|+2(n-rk-(|I|-r)-|F_I|)\leq (k-1)(|\B_I|-r).\]
Rearranging and using $|I|\leq p-1$, we get the inequality as claimed.
\end{proof}

Now suppose $x\in I$ for $I$ a maximal independent set, with $F_x$ a set of $x$-toes. Let us assume 
\begin{equation*}|F_x|\geq k.\tag{*}\end{equation*}
Then it follows there are at least two blocks $B,C\in \B_x$, say, containing (necessarily distinct) $x$-toes; say $y\in B$ and $z\in C$. If $y$ and $z$ were not adjacent, then replacing $x$ with $y,z$ in $I$ would yield a larger independent set, a contradiction. Thus there is a block $W$ with $y,z\in W$. We refer to such blocks as \emph{webbings}. Formally, $W$ is an \emph{$x$-webbing} if $W\not\in\B_I$ and $W$ contains distinct $x$-toes; the set of $x$-webbings is later denoted $\mathcal{W}(x)$. Note that under the assumption (*), each toe appears at least once in a webbing, so that it must have degree at least $2$. 

More precisely, suppose there are $\tau_i$ toes in distinct $x$-blocks $B_i$ with $\tau_1\geq\tau_2\dots\geq\tau_s\geq 1$. Then each of the $\sum_{i<j}\tau_i\tau_j$ pairs of toes must appear in some webbing. In the case $k=6$ and $|F_x|\geq 7$, one can see that some toes must have degree at least $3$, for example. This implies non-trivial lower bounds on the excess $\E(F_x)\leq \E(X)$.

\begin{lemma}\label{toetable}
    Let $x\in X$ with $x$ of degree $2$ or $3$ and suppose $F_x$ is its $x$-foot; that is, $F_x$ the set of all vertices occuring just once in the star of $x$. The table below gives minimum values of $\E(F_x)$ in terms of $|F_x|$.
    \begin{center}
    \begin{table}[h!]
    \begin{tabular}{ c|c|c|c|c|c|c|c|c|c|c|c| } 
    $|F_x|$ & $\leq 5$ & $6$ & $7$ & $8$ & $9$ & $10$ & $11$ & $12$ & $13$ & $14$ & $15$\\\hline
    $\min(\E(F_x))$ & $0$ & $0$ & $2$ & $3$ & $7$ & $10$ & $11$ & $12$ & $20$ & $25$ & $27$
    \end{tabular}.
    \label{table:exs}
    \end{table}
    \end{center}

If, moreover, $F_x$ is known to contain no elements of degree $2$, then $\min(\E(F_x))\geq |F_x|$.\end{lemma}
\begin{proof}For the second statement of the lemma, just observe that any $x$-toe of degree at least $3$ contributes at least $1$ to $\E(F_x)$.

For the table itself, suppose $|F_x|=\tau_1+\tau_2+\tau_3$ is a partition of $|F_x|$ into summands of size at most $6$. If $|F_x|\leq 6$, then one webbing $W$ suffices to cover all toes, and so the minimum possible excess of $0$ is a achieved by a configuration of $x$-blocks and one webbings in which each toe appears just twice. Otherwise suppose there are $w>1$ webbings, containing each of the $\tau_1\tau_2+\tau_1\tau_3+\tau_2\tau_3$ pairs of toes, so that 
\[\left\lceil\frac{\tau_1\tau_2+\tau_1\tau_3+\tau_2\tau_3}{3}\right\rceil.\] is an upper limit for $w$. 

We used the immensely powerful linear programming solver Gurobi to solve the following problem. let $M$ be a $w\times |F_x|$-matrix of variables taking values in $0$ and $1$, with the rows representing webbings and a $1$ appearing in the $(i,j)$-th entry if a toe labelled $j$ is in the $i$th webbing. Since $|F_x|>6$, we know that toe $j$ appears once in the $x$-blocks and at least once in the webbings, so $\E(F_x)+|F_x|$ is the sum $S$ of all entries of the $M$. Furthermore, the rows of $M$ must all sum up to integers less than or equal to $k=6$, and columns $j_1$ and $j_2$ must have scalar product at least $1$ whenever $j_1$ and $j_2$ come from different parts of the partition \[\{1,\dots,n\}=\{1,\dots,\tau_1\}\cup\{\tau_1+1,\dots,\tau_1+\tau_2\}\cup\{\tau_1+\tau_2+1,\dots,\tau_1+\tau_2+\tau_3\}.\] We ask Gurobi to minimise the sum $S$ subject to these constraints, and it results in the table in the lemma. 

Since Gurobi only gives answers up to a percentage accuracy, its output does not amount to a proof of optimality, and so we wrote some Prolog code to check that the values of $\min(F_x)$ one below those in the table are infeasible. %See \cref{appendb}.
\end{proof}

\begin{rmk}This improves the bound in \cite[Table 1]{BateRees} for $\min(|F_x|)$ when $|F_x|=13$ from $16$ to $20$. Since Gurobi outputs a feasible configuration of webbings for each value of $|F_x|$, we know that the minima can be achieved.\end{rmk}




\section{Proof of the theorem}
\subsection{SICStus Prolog code}
The code in Appendix \cref{appendixcode}---also available at \url{github.com/cushydom88/`lottery-problem}---implements the constraints described in the previous section. We give a description of the strategy and its functionality. 

One first loads SICStus Prolog and consults the file \lstinline{lottery.pl} through the command 
\begin{lstlisting}
    ?- ['$PATH_TO_DIRECTORY/lottery.pl'].
\end{lstlisting} 
Then one queries Prolog at the command line by asking it to solve for variables in a (conjunction) of predicates. The following is an example of the output produced from the main predicate:
% Figure environment removed 

The principal predicates are:

\lstinline{lottery_numbers_in_range( NMin, NMax)}. This predicate writes output to the terminal of the form given in \cref{outputexample}; i.e.~it works sequentially with $n$ from the value of \lstinline{NMin} to the value of \lstinline{NMax} either outputting a line stating the value of $L(n,6,6,2)$ or a conjectured value of it, which is correct modulo a list of cases to be checked by hand. The first case is computed from scratch, using a lower bound of $1$ but then in order to reduce unnessary computation, it passes the (conjectured) value of $L(n,6,6,2)$ as a lower bound for $L(n+1,6,6,2)$, using \cref{doesntgodown}. 

\lstinline{upper_bound( N, Guess, UB )} is called by the previous predicate, and it holds when \lstinline{Guess} equals $L(N-1,6,6,2)$ and \lstinline{UB} is an upper bound for $L(N,6,6,2)$. The predicate considers the list of upper bounds for sizes of covering designs in \cref{coveringnumbers} searching for, initially whether \lstinline{Guess} can be achieved as a sum of such values according to \cref{upperbound}, or increases \lstinline{Guess} until it can.

In many cases, one finds that the (conjectured) value of $L(n-1,6,6,2)$ coincides with the value \lstinline{UB}. In that case, by \cref{doesntgodown}, we conclude immediately $L(n,6,6,2)=L(n-1,6,6,2)$ modulo any previous cases that require checking by hand. 

Otherwise, further predicates are engaged to establish when the hypotheses of \cref{1s2s3s} can fail and otherwise compute possible vectors $\delta(I)$ satisfying the conclusions of \cref{1s2s3s}. We start by using the predicate \lstinline{r_bound} to give lower and upper bounds for  $r=d_1/k$ using, respectively, \cref{rbound} and \cref{FurediBound} together with \cref{rboundforfuredi}. 

\lstinline{get_slim_I_exceptions(N, UB, R, PossSlimIException )}. For brevity, say an independent set is \emph{slim} if it is of maximal cardinality and its component vertices have degrees at most $3$. Then \lstinline{get_slim_I_exceptions} is true when \lstinline{R} is a possible value of $r$ from \lstinline{rbound} and \cref{1s2s3s} does not imply the existence of a slim independent set. In that case, the variable \lstinline{PossSlimIException} is bound to a pair $(r,(a,b))$, indicating the value of $r$ and an interval $a\leq d_2\leq b$ where $a$ is taken from \cref{1s2s3s} and $b$ comes from \cref{shannonbound}. Any feasible values for \lstinline{PossSlimIException} are sent to user output. Otherwise \lstinline{get_min_num_Iblocks( N, MinNumIBlocks )} binds MinNumIBlocks to the value implied by \cref{minnumiblocks} and it is passed to the next predicate along with $n$, the upper bound from above, and queried for each feasible value of $r$.

\lstinline{get_deltaI_exceptions( N, UB, MinNumIBlocks, R, Deltas ) } binds \lstinline{Deltas} to a list of possible tuples $\delta(I)$ which are constrained by the following:
\begin{enumerate}\item the first $r$ elements of $\delta(I)$ are $1$;
\item the next $s$ elements of $\delta(I)$ are each $2$, where $s$ is determined by the lower bound \cref{lowerboundd2} combined with \cref{shannonbound}.
\item the remaining $p-1-r-s$ elements are either $2$ or $3$;
\item the number of $I$-blocks is at least the minimum already described.\end{enumerate}

Finally, for each feasible value of $r$ and feasible sequence $\delta(I)$ in \lstinline{Deltas}, we solve a constraint problem to determine the number and distribution of toes among $I$-blocks subject to their contribution to the excess $\E(X)$ as implied by \cref{toetable}. Since $r$ is known, the value $\E(X)$ can be computated from \cref{excessfromisolated} and \lstinline{get_base_excess} binds \lstinline{BaseExcess} to this value.

\lstinline{can_populate_toes_in_Iblocks(N, R, BaseExcess, Delta)}. This creates and a binds a variable \lst{MinToes} to the lower bound supplied by \cref{mintoes}. Since any vertex in $I$ of degree at least $3$ contributes $1$ to $\E(X)$, the contribution to the excess $\E(X)$ from toes must be at most $\E(X)-\delta_3$, where $\delta_3=|d^{-1}(3)\cap I|$. Hence a variable \lst{Excess} is created and bound to \lst{BaseExcess-NumThrees}. This information is passed to the final predicate with the $1$s stripped from $\Delta$ to give the sequence \lst{DeltaNoOnes}.

\lstinline{populate_toes_in_Iblocks( DeltaNoOnes, MinToes, Excess, Vs )} solves the following constraint problem:\begin{enumerate}\item $J=\{x_{r+1},\dots,x_{p-1}\}\subseteq I$ is the subset of $I$ of vertices of degree at least $2$, such that $\delta(J)$ is bound to \lst{DeltaNoOnes};
\item \lst{Vs} is a sequence of variables representing $\tau(J)$; so the total number of toes $\sum\tau(J)$ represented by \lst{sum(Vs)} is at least the calculated bound \lst{MinToes};
\item given the values of $d(x_i)$ and $\tau(x_i)$, the sum over $r+1 \leq i\leq p-1$ of the minimum excesses implied by \cref{toetable} is at most \lst{Excess}.
\end{enumerate}

Any solutions for the variables \lst{Vs} together with the choice of \lst{Delta} are outputted for manual checking. 

\subsection{The remaining cases}\label{remainingcases} The output from \lst{compute_lottery_numbers_in_range(30,61)} tells us that we must consider separately the following cases:
\begin{enumerate}
    \item $n=54$, $j=22$, $d_2\leq 36$, $\delta(I)=(2,2,2,2,3)$;
    \item $n=54$, $j=22$, $d_2\leq 18$, $\delta(I)=(1,2,2,3,3)$;
    \item $n=57$, $j=24$, $d_2\leq 27$, $\delta(I)=(2,2,2,3,3)$;
    \item $n=58$, $j=25$, $d_2\leq 27$, $\delta(I)=(2,2,2,3,3)$;
\end{enumerate}
And for $n=59$, we must rule out the following cases for $j=26$:
\begin{enumerate}[label=(\roman*)] 
    \item $d_1=0$, $d_2=36$;\label{casei}
    \item $d_2\leq 9$, $\delta(I)=(1,2,3,3,3)$;\label{caseii}
    \item $d_2\leq 27$, $\delta(I)=(2,2,2,3,3)$;\label{caseiii}
\end{enumerate}

In the next three subsections, we treat the cases for $n=59$; the cases where $n<59$ are similar and easier. 

\subsection*{Case \ref{casei}}
In this case, we have $d_2=36$ and so we cannot apply \cref{1s2s3s} in order to get a maximal independent set of order $5$. However, by \cref{shannonbound}, we do get an independent set $I_0=\{x_1,\dots,x_4\}$ with $\delta(I_0)=(2,2,2,2)$. But then the $8$ blocks of $B_{I_0}$ supply at most $44$ elements, so we can enlarge $I_0$ to a maximal independent set $I=I_0\cup\{x_5\}$. Since the Prolog query did not flag up the cases where $\delta(I)=(2,2,2,2,2)$ or $(2,2,2,2,3)$, we may assume $d(x_5)\geq 4$. Furthermore, all vertices of degree $2$ or $3$ must appear in a block of $\B_{I_0}$, or we could have chosen $x_5$ of degree $2$ or $3$. Since there are at most $44$ distinct vertices in these blocks, we have $d_3\leq 8$. 

Now, the number of elements $d_{\geq 4}$ of degree at least $4$ is $d_{\geq 4}=59-36-d_3=23-d_3$. Hence,
\begin{align*}156=\sum id_i&\geq 2\cdot 36 + 3\cdot 8+4\cdot(23-d_3),\\
\text{whence }\qquad 84 &\geq 3\cdot d_3+4\cdot (23-d_3).\end{align*}
Therefore $d_3\geq 8$, which gives $d_3=8$, $d_4=15$ and $d(x_5)=4$. Consequently, the elements of order $4$ cannot appear in any blocks of $B_{I_0}$ so must each appear at least once in the blocks $B_{x_5}$, which can contain a maximum of $21$ distinct elements, so that there are at most $6$ vertices in $B_{x_5}$ with degree $2$ or $3$. This shows that there must be at least $34$ $I$-toes coming from vertices in blocks of $B_{I_0}$. Checking \cref{toetable}, it is easy to see that $\E(\bigcup B_{I_0})>8$ which implies there are at least $9$ vertices of degree $3$, a contradiction.


\subsection*{Case \ref{caseii}}
Here we assume \[\delta(I)=(1,2,3,3,3),\] with $I=\{x_1,\dots,x_5\}$ and that there is no independent set $J$ with $\delta_2(J)>1$. In particular, every vertex of degree $2$ appears in a block with every other.

By \cref{lowerboundd2}, we have $d_2\geq 9$, hence $d_2=9$. Furthermore we have $d_1=6$ so 
\[156\geq 6+2\cdot 9 +3\cdot d_{\geq 3}\] 
with $d_{\geq 3}=44$. Hence we have equality above and $d_{\geq 3}=d_3=44$.  If $y\in B\in \B_{x_i}$ with $i>2$, $d(y)=2$, then there is another independent set $J\subset X$ with $\delta(J)=(1,2,2,3,3)$, a contradiction. So all elements of degree $2$ are contained in the blocks $\{B_1,B_2\}=\B_{x_2}$. The multiset $\BB_{x_2}=B_1\sqcup B_2$ has $10$ elements not equal to $x_2$. Now, $8$ of these must be the remaining vertices of degree $2$, so there are at least $6$ occurring just once. In other words $F_{x_2}$ has at least $6$ toes of degree $2$. Suppose $\tau_1$ of these appear in $B_1$ and $\tau_2$ appear in $B_2$. Since $\tau_1+\tau_2\geq 6$ we see $\tau_1,\tau_2>0$. This tells us that all these toes  must lie in a single webbing. It follows that $\tau_1+\tau_2=6$ and so there are exactly $6$ toes in $F_{x_2}$, meaning that the remaining $2$ vertices of degree $2$ both appear twice. 

Returning to the output of the Prolog code
\begin{lstlisting}
?- populate_toes_in_Iblocks([2,3,3,3], 43, 41, Vs ), labeling([], Vs).
Vs = [7,12,12,12] ? ;
Vs = [8,11,12,12] ? ;
Vs = [8,12,12,12] ? ;
Vs = [9,10,12,12] ? ;
Vs = [9,11,11,12] ? ;
no
\end{lstlisting}
we conclude only the first possibility can hold. Hence the $43$ toes are divided between the $x_i$-blocks in proportion $6$, $12$, $12$ and $13$ for $2\leq i\leq 5$. Now from \cref{toetable} we see that if $F_x=13$ then $\E(F_x)\geq 16$. But if every $y\in F_x$ had $d(y)\leq 3$ then $\E(F_x)\leq 13$. Thus some  $d(y)\geq 4$, a contradiction.

\subsection*{Case \ref{caseiii}} 
Here we assume $d_2\leq 27$ and \[\delta(I)=(2,2,2,3,3),\] with $I=\{x_1,\dots,x_5\}$ and that any independent set $J$ with degrees in $\{2,3\}$ has $\delta(J)=\delta(I)$. In particular, every vertex of degree $2$ appears in $\bigcup \B_{\{x_1,x_2,x_3\}}$ and appears in no block of $\B_{\{x_4,x_5\}}$. 

By \cref{lowerboundd2}, we have $d_2\geq 21$, and \cref{mintoes} implies $|F_I|\geq 48$. Prolog informs us of the following possibilies for $\tau(I)$:

\begin{lstlisting}
    ?- populate_toes_in_Iblocks([2,2,2,3,3], 48, 36, Vs ), labeling([], Vs).
    Vs = [7,8,9,12,12] ? ;
    Vs = [8,8,8,12,12] ? ;
    Vs = [8,8,9,11,12] ? ;
    no
    \end{lstlisting}

These solutions lead to $\E(F_I)\geq 36$ except for $\tau(I)=(8,8,8,12,12)$ which implies only $\E(F_I)\geq 33$. Let us rule out the cases where $\E(F_I)\geq 36$.

Any toe $y\in F_I$ occurs once in the blocks of $\B_I$ and all other occurrences are in blocks not in $\B_I$. Let $d_{i,\tau}$ denote the number of toes having degree $i$. Then  counted with multiplicy in $\BB$, the toes account for $T=\sum_{i\geq 2} id_{i,\tau}$ elements of $\BB$. We have $\E(F_I)=\sum (i-2)d_{i,\tau}$, and $\sum d_{i,\tau}=:n_\tau$ is the number of toes. So $T=\E(F_I)+2\cdot n_\tau$ and exactly $n_\tau$ of these entries come from $F_I$, so there are $\E(F_I)+n_\tau\geq 48+36=84$ elements lying in blocks not in $\B_I$. But $84/6=14$ so there are at least $14$ blocks not lying in $\B_I$. Since $|\B_I|=12$ and $|\B|=26$, this exhausts $\B$, implies $n_\tau=48$, that \begin{equation*}\E(F_I)=36=\sum_{1\leq i\leq 5}\E(F_{x_i}),\tag{$\dagger$}\end{equation*} and that the blocks $\B\setminus\B_I$ contain only toes. 


Consider $F_{x_2}$, which, by ($\dagger$) contributes exactly $3$ to $\E(F_I)$. It is easy to see that this means there must be $5$ toes in $B$ and $3$ toes in $C$, say, where $\{B,C\}=\B_{x_2}$; also that there are exactly two $x_2$-webbings $B'$ and $C'$. 
Let us label the vertices of $B$ and $C$ as $B=\{x_2,1,2,3,4,5\}$, $C=\{x_2,6,7,8,\_,\_\}$. Then up to an action of $S_5\times S_3$ permuting the toes, we may write $B'=\{1,2,3,6,7,8\}$ and $C'=\{4,5,6,7,8,z\}$, where $z$ is as yet undetermined. 
But since $C'\not\in \B_I$ it follows that $z$ is a toe. We cannot have $z\in \{1,\dots,8\}$ since that would imply $E(F_{x_2})>3$. 
Therefore $z\in F_{x_i}$ for $i\neq 2$. But as $|F_{x_i}|\geq k=6$, each $x_i$-toe must already appear once in $\B_{x_i}$ and again in some $D\in\B\setminus\B_i$ with some other $x_i$-toe. We conclude $D\neq C'$ and so $C'$ is not a webbing of $x_i$-toes. Since the webbings of $x_i$-toes account entirely for the occurrences of the $x_i$-toes outside of $\B_I$, we reach a contradiction.

Therefore we must have $\E(X)<36$ and so $\tau(I)=(8,8,8,12,12)$. We use a similar counting argument as above: for $x=x_{4}$ or $x_5$, we have $|F_x|=12$ and these incur an excess $\E(F_x)\geq 12$. It follows that these $12$ $x$-toes must appear at least $24$ times in the webbings $\mathcal{W}(x)$. This implies that $|\mathcal{W}(x)|\geq 4$ and if equality holds, then $\bigcup \mathcal{W}(x)$ consists entirely of $x$-toes; moreover a closer inspection of the solutions found by Gurobi implies that the toes are distributed in such a way as to have $4$ in each $x$-block. Similarly, if $x\in\{x_1,x_2,x_3\}$ then $|\mathcal{W}(x)|\geq 2$ and in the case of equality, there is at most $1$-non-toe in $\bigcup(\mathcal{W}(x))$. Since $\sum_{x\in I}|W(x)|\geq 14$ and there are $12$ $I$-blocks, we do indeed have equality everywhere.

Consequently, there are $59-5-48=6$ vertices which are neither in $I$ nor $I$-toes. Consider the draw $D$ comprising these $6$ elements. 
We have by assumption that $D\cap B$ has at least $2$ elements for some $B\in \B$. 
But by the above remarks, we cannot have $B\in\W(x)$ for any $x\in I$, since such a $B$ contains at most $1$ non-$x$-toe; equally we cannot have $B\in B_x$ for $x\in\{x_4,x_5\}$ as $B\setminus\{x\}$ has at most $1$ non-$x$-toe. 
Therefore $B\in\B_{x_1}=\{E,F\}$, say and that there are $3$ toes and $2$-non-toes in $E$ and $5$ toes in $F$. Let us therefore label $E=\{x_1,1,2,3,-1,-2\}$, $F=\{x_1,4,5,6,7,8\}$, $D=\{-1,-2,-3,-4,-5,-6\}$, accordingly. Up to $S_3\times S_5$ symmetry, we have $\mathcal{W}(x)=\{B',C'\}$ with $B'=\{1,2,3,4,5,6\}$, $C'=\{1,2,3,7,8,\_\}$. Therefore all the pairs $\{-1,-2\}\times \{4,5,6\}$ are non-adjacent.  Being toes, none of $4,5$ or $6$ can appear in a block of $\B_{x_2}$ or $\B_{x_3}$ and furthermore, there is just one space in $\W(x_2)$ and $\W(x_3)$ where any of $\{4,5,6\}$ could appear. Without loss of generality, then, $4$ is no block in $\B_{x_2},\B_{x_3},\W(x_2),\W(x_3)$. 

Now consider the draw $D_1=\{4,-2,-3,-4,-5,-6\}$, which must intersects a block $B_1$ in $\B_{x_2}=\{B_1,C_1\}$ say, in at least $2$ places. Without loss of generality, we have either $B_1=\{x_2,9,10,11,-2,-3\}$ or $B_1=\{x_2,9,10,11,-3,-4\}$ and $C_1=\{x_2,12,13,14,15,16\}$. In a similar to manner to before, we may assume $12$ does appear in any of $\B_{x_i},\W(x_i)$ for $i\in\{1,3\}$. 

Now let $D_2=\{4,-2,12,-4,-5,-6\}$. Again, $D_2$ intersects a block which must be in $\B_{x_3}=\{B_3,C_3\}$, say, in a similar configuration of containing $3$ and $5$ toes, respectively. Label $B_3=\{x_3,17,18,19,-5,-6\}$: the other cases are similar or easier. Writing $C_3=\{x_3,20,21,22,23,24\}$, and assuming $20$ does not appear in $\B_{x_i}$ or $\mathcal{W}(x_i)$ for $i\in\{1,2\}$, we finally set $D_3=\{4,-2,12,-4,20,-6\}$. By construction this does not intersect any other block in at least $2$ places, and leads to a final contradiction.

This completes the proof of the main theorem.


\section{Configurations for other lottery designs}
Tickets of size $6$ are represented in the diagrams of \cref{tickpics} by concatenating the labels over the vertices in each line. These sets of blocks can be combined to give $(n, 6, 6, 2; x)$-lottery designs. For example our $(59, 6, 6, 2; 27)$-lottery design from \cref{sec:ticks} has configuration $(B,C,E,E,E).$ 

% Figure environment removed

\begin{thm}\cref{theconfigs} lists $j=L(n,6,6,2)$ together with configurations described in \cref{tickpics} which afford an $(n,6,6,2;j)$-lottery design.\label{alltogetherthm}\end{thm}

\begin{table}[!ht]
    \begin{center}
\begin{tabular}{ |c|c|c| } 
  \hline
 $n$ & $L(n,6,6,2)$ & Configuration \\ 
 \hline
 32 & 7 & $(A,A,A,A,B)$ \\ 
 33 & 7 & $(A,A,A,A,C)$  \\ 
34 & 7 & $(A,A,A,A,D)$  \\
35 & 9 & $(A,A,A,B,C)$  \\
36 & 9 & $(A,A,A,C,C)$  \\
37 & 10 & $(A,A,A,C,D)$  \\
38 & 11 & $(A,A,A,A,E)$  \\
39 & 11 & $(A,A,C,C,C)$  \\
40 & 12 & $(A,A,C,C,D)$  \\
41 & 13 & $(A,A,A,C,E)$  \\
42 & 13 & $(A,C,C,C,C)$  \\
43 & 14 & $(A,C,C,C,D)$  \\
44 & 15 & $(A,A,C,C,E)$  \\
45 & 15 & $(C,C,C,C,C)$  \\
46 & 16 & $(C,C,C,C,D)$  \\
  \hline
\end{tabular}
\begin{tabular}{ |c|c|c| } 
  \hline
 $n$ & $L(n,6,6,2)$ & Configuration \\ 
 \hline
47 & 17 & $(A,C,C,C,E)$  \\
48 & 18 & $(A,C,C,D,E)$  \\
49 & 19 & $(A,A,C,E,E)$  \\
50 & 19 & $(A,A,A,A,E)$  \\
51 & 20 & $(A,A,A,D,E)$  \\
52 & 21 & $(A,C,C,E,E)$  \\
53 & 22 & $(A,C,D,E,E)$  \\
54 & 23 & $(A,A,E,E,E)$  \\
55 & 23 & $(C,C,C,E,E)$  \\
56 & 24 & $(C,C,D,E,E)$  \\
57 & 25 & $(A,C,E,E,E)$  \\
58 & 26 & $(A,D,E,E,E)$  \\
59 & 27 & $(B,C,E,E,E)$  \\
60 & 27 & $(C,C,E,E,E)$ \\
61 & 28 & $(C,D,E,E,E)$ \\
  \hline
\end{tabular}
\caption{Lottery numbers and their configurations}
\label{theconfigs}
\end{center}
\end{table}


\newpage\ \newpage\section*{\Large\sc Appendices}
\begin{appendices}
\section{Main Prolog code}\label{appendixcode}
    \begin{lstlisting}
    :-use_module(library(clpfd) ).
    :-use_module(library(lists)).
    :-use_module(library(ordsets)).
    :-use_module(library(ugraphs)).
    :-use_module(library(between)).
    
    % ['lottery.pl'].
    
    % compute_lottery_numbers_in_range(30,61).
    
    %%% COMPUTE LOTTERY NUMBERS IN A RANGE %%%
    % Given Nmin and Nmax try to compute L(n,6,6,2) for Nmin <= n <= Nmax.
    % The value of L(n,6,6,2) is passed in to the computation of L(n+1,6,6,2)
    compute_lottery_numbers_in_range( Nmin, Nmax ) :-
        PrevLottoNum #= 1,
        compute_lottery_numbers_in_range2( PrevLottoNum, Nmin, Nmax ).
    compute_lottery_numbers_in_range2( PrevLottoNum, N, Nmax ) :-
        N #= Nmax,
        possible_lottery_number( N, PrevLottoNum, _ ).
    compute_lottery_numbers_in_range2( PrevLottoNum, N, Nmax ) :-
        N #\= Nmax,
        possible_lottery_number( N, PrevLottoNum, NewLottoNum ),
        NewN #= N + 1,
        compute_lottery_numbers_in_range2( NewLottoNum, NewN, Nmax ).
    %%%%%%
    
    %%% POSSIBLE LOTTERY NUMBER %%%
    possible_lottery_number( N, PrevLottoNum, UB ) :-
        % first calculate the upper bound
        upper_bound( N, PrevLottoNum, UB ),
        possible_lottery_number2( N, PrevLottoNum, UB ).
    possible_lottery_number2( N, PrevLottoNum, UB ) :-
        PrevLottoNum #= UB,
        print_message(informational, format(' L(~w,6,6,2) = ~w ',[N,UB]) ).
    possible_lottery_number2( N, PrevLottoNum, UB ) :-
        PrevLottoNum #\= UB,
        % Calculate an upper bound on d1
        bound_isolated_blocks( N, UB, Rs ),
        get_min_num_Iblocks( N, MinNumIBlocks ),
        % get any exceptions for constructing a slim I
        maplist( get_slim_I_exceptions(N, UB), Rs, PossSlimIExceptions ),
        include( interval_check, PossSlimIExceptions, SlimIExceptions1 ),
        % get Delta(I) exceptions
        maplist( get_deltaI_exceptions(N, UB, MinNumIBlocks), Rs, DeltaExceptionsList ),
        append( DeltaExceptionsList, DeltaExceptions),
        write_lotto_result( SlimIExceptions1, DeltaExceptions, N, UB ).
    %%%%%%
    
    %%% D1 UPPER BOUND %%%
    bound_isolated_blocks( N, UB, PossD1s ) :-
        get_r_min( N, UB, D1Min ),
        include( bound_isolated_blocks_helper(N, UB, D1Min), [0,1,2,3,4], PossD1s ).
    bound_isolated_blocks_helper(_, _, D1Min, 0) :-
        D1Min #< 1.
    bound_isolated_blocks_helper(N, UB, D1Min, R) :-
        R #> 0,
        D1Min #=< 6*R,
        P #= 6 - R,
        LB #= UB - R,
        M #= N - 6 * R,
        furedi_lower_bound( Vs, M, 6, P, LB ),
        labeling([], Vs).
    get_r_min( N, UB, D1Min ) :-
        T #= 2*N - 6*(UB - 1),
        T #=< 0,
        D1Min #= 0.
    get_r_min( N, UB, D1Min ) :-
        T #= 2*N - 6*(UB - 1),
        T #> 0,
        D1Min #= T.
    %%%%%%
    
    %%% COVERING DESIGN UPPER BOUND %%%
    % Start with a guess for the upper bound and increase the guess until a solution is found
    upper_bound( N, Guess, UB ) :-
        findall( Vs, ( test_upper_bound( Vs, N, Guess ), labeling([], Vs)),  Sols ),
        upper_bound2( N, Guess, Sols, UB ).
    upper_bound2( N, Guess, [], UB ) :-
        Guess2 #= Guess + 1,
        upper_bound( N, Guess2, UB ).
    upper_bound2( _, Guess, Sols, UB ) :-
        length(Sols, KK),
        KK #>= 1, 
        UB #= Guess.
    
    test_upper_bound( Vs, N, UB ) :- 
        length( Vs, 5 ),
        domain( Vs, 1, 65 ),
        M is N - 25,
        sum( Vs, #=, M ),
        get_covering_design_nums( Vs, Scores ),
        sum( Scores, #=<, UB ),
        sorting(Vs,[1,2,3,4,5],Vs).
    
    get_covering_design_nums( Vs, Scores ) :-
        Vals = [1,3,3,3,4,
            6,6,7,7,10,10,12,12,15,16,
            17,19,21,22,23,24,27,28,30,31,
            31,38,39,40,42,47,50,51,54,55,
            59,63,65,67,70,73,79,80,82,87,
            90,96,98,99,105,110,114,117,119,128,
            132,135,140,142,143,143,157,160,163,172],
        maplist( get_covering_design_num(Vals), Vs, Scores ).
    get_covering_design_num(Vals, I, Score ) :-
        nth1( I, Vals, Score ).
    %%%%%%
    
    %%% FUREDI LOWER BOUND %%%
    furedi_lower_bound(Vs, N, K, P, LB ) :-
        LLB is K * LB,
        NP is P - 1,
        length( Vs, NP ),
        domain( Vs, 1, N ),
        sum( Vs, #=, N ),
        maplist( furedi_lower_bound_helper(K), Vs, Ws ),
        sum( Ws, #=<, LLB  ).
    furedi_lower_bound_helper( K,  X, Y ) :- 
        ((X -1 ) mod (K - 1 )  ) #= 0,
        Y #= X * ( (X-1) / (K - 1 ) ).
    furedi_lower_bound_helper( K, X, Y ) :- 
        ((X -1 ) mod (K - 1 )  ) #\= 0,
        Y #= X * ( 1 + ( (X-1) / ( K - 1 ) ) ).
    %%%%%%
    
    %%% SLIM I EXCEPTIONS
    get_slim_I_exceptions( N, UB, R, [R, [D2Lower, D2Upper] ] ) :-
        D2Upper #= 9 * (4 - R),
        D2Lower0 #= ( 9*( 4*N - 6*(UB-1) - 18*R - 16*(4-R) )) / 4,
        make_non_negative(D2Lower0,D2Lower).
    interval_check([_, [A,B] ]) :- A #=< B.
    make_non_negative(X, Y) :-
        X #< 0,
        Y #= 0.
    make_non_negative(X, Y) :-
        X #>= 0,
        Y #= X.
    %%%%%%
    
    %%% DELTA(I) EXCEPTIONS %%%
    % Lower bound the number of blocks in B_I
    get_min_num_Iblocks( N, Min) :-
        ( ( N - 5 ) mod 5 ) #= 0,
        Min #= (N - 5)/5.
    get_min_num_Iblocks( N, Min ) :-
        ( ( N - 5 ) mod 5 ) #\= 0,
        Min #= 1 + ((N - 5)/5).
    
    % Find the values of Delta(I) we cannot immediately rule out
    get_deltaI_exceptions( N, UB, MinNumIBlocks, R, Deltas ) :-
        % Lower bound how many 2s are assumed to be in Delta(I)
        get_delta_two( N, UB, R, Delta2 ),
        % Fill Delta(I) with as many 1s and 2s that can be assumed
        % Then generate all ways to complete Delta(I) by adding 2s and 3s
        row_of_n_ms( R, 1, L1 ),
        row_of_n_ms( Delta2, 2, L2 ),
        append( L1, L2, L3 ),
        length( L3, K1),
        K2 #= 5 - K1,
        findall( Xs, ( length(Xs, 2), domain(Xs, 0, K2), sum(Xs, #=, K2), labeling([], Xs) ), VSizes ),
        maplist( make_delta_end, VSizes, DeltaEnds ),
        maplist( append(L3), DeltaEnds, Deltas1 ),
        % Remove out Delta(I) where |B_I| is too small 
        include( at_least_m_blocks(MinNumIBlocks), Deltas1, Deltas2 ),
        % Upper bound the excess of the whole design
        get_base_excess( N, R, UB, BaseExcess ),
        % For each possible Delta(I) consider the constraint problem of populating toes in the IBlocks 
        include( can_populate_toes_in_Iblocks(N, R, BaseExcess), Deltas2, Deltas ).
    
    get_delta_two( N, UB, R, Delta2 ) :-
        get_delta_two2( N, UB, R, PossDelta2 ),
        make_non_negative(PossDelta2, Delta2).
    get_delta_two2( N, UB, R, Delta2 ) :-
        X #= (3*N - 12 * R - 6 * (UB - 1) ),
        (X mod 9) #= 0,
        Delta2 #= X / 9.
    get_delta_two2( N, UB, R, Delta2 ) :-
        X #= (3*N - 12 * R - 6 * (UB - 1) ),
        (X mod 9) #\= 0,
        Delta2 #= 1 + (X / 9).
    row_of_n_ms( N, M, Row ) :-
        length( Row, N ),
        maplist( #=(M), Row ).
    make_delta_end( [A,B], DeltaEnd ) :-
        row_of_n_ms( A, 2, X1 ),
        row_of_n_ms( B, 3, X2 ),
        append(X1, X2, DeltaEnd).
    
    at_least_m_blocks( MinNumIBlocks, PossDelta ) :-
        sum(PossDelta, #>=, MinNumIBlocks).
    
    get_base_excess( N, R, UB, BaseExcess ) :-
        BaseExcess #= 6*(UB-1) + 6*R - 2*N. 
    
    can_populate_toes_in_Iblocks(N, R, BaseExcess, Delta) :-
        sum(Delta, #=, B),
        MinToes #= 2*N - 10 - 5*(B+R),
        delete( Delta, 3, Residue),
        length( Residue, NumOnesAndTwos ),
        NumThrees #= 5 - NumOnesAndTwos,
        Excess #= BaseExcess - NumThrees,
        delete( Delta, 1, DeltaNoOnes),
        populate_toes_in_Iblocks( DeltaNoOnes, MinToes, Excess, Vs ), 
        labeling([], Vs).
    
    populate_toes_in_Iblocks( DeltaNoOnes, MinToes, Excess, Vs ) :-
        same_length( Vs, DeltaNoOnes, N ),
        domain( Vs, 1, 15 ),
        maplist( adjust_domain, DeltaNoOnes, Vs ), 
        maplist( get_ex, Vs, DeltaNoOnes, Excesses ),
        sum(Excesses, #=<, Excess ),
        sum(Vs, #>=, MinToes ),
        numlist(N,L),
        sorting(Vs,L,Vs).
    adjust_domain( 2, V ) :- V #=< 10.
    adjust_domain( 3, _).
    get_ex( V, 2, Ex ) :-
        element( V, [0,0,0,0,0,0,2,3,7,10], Ex ).
    get_ex( V, 3, Ex ) :-
        element( V, [1,2,3,4,5,6,7,8,9,10,11,12,20,25,27], Ex ).
    %%%%%%
    
    %%% RESULT OUTPUT %%%
    write_lotto_result( [], [], N, UB ) :-
        print_message(informational, format(' L(~w,6,6,2) = ~w ',[N,UB]) ).
    write_lotto_result( SlimIExceptions, DeltaExceptions, N, UB ) :-
        append( SlimIExceptions, DeltaExceptions, TotalExceptions ),
        length(TotalExceptions, Ex),
        Ex #> 0,
        print_message(informational, format('We conjecture that  L(~w,6,6,2) = ~w and must rule out the following cases',[N,UB]) ),
        write_slimI_exceptions( SlimIExceptions ),
        write_delta_exceptions( DeltaExceptions ).
    
    write_slimI_exceptions( Exceptions ) :-
        maplist( write_slimI_exception, Exceptions ).
    write_slimI_exception( [R, [D2, D2] ] ) :-
        D1 #= 6*R,
        print_message(informational, format(' d_1 = ~w and d_2 = ~w ',[D1,D2]) ).
    write_slimI_exception([R, [D2L, D2U] ] ) :-
        D1 #= 6*R,
        D2L #\= D2U,
        print_message(informational, format(' d_1 = ~w and d2 in the range [~w,~w] ',[D1,D2L, D2U]) ).
    
    write_delta_exceptions( Exceptions ) :-
        maplist( write_delta_exception, Exceptions ).
    write_delta_exception( Delta ) :-
        delete( Delta, 2, Residue),
        length( Residue, NumNoneTwos ),
        NumTwos #= 5 - NumNoneTwos,
        D2 #= NumTwos * 9,
        print_message(informational, format(' d_2 <= ~w and \\delta(I) = ~w ',[D2, Delta]) ).
    
    writeln( Stream ) :-
            write( Stream ),
            write('\n').
    %%%%%%
    \end{lstlisting}
\end{appendices}

{\bf{Acknowledgement:}} The authors are supported by the Leverhulme Trust Research Project Grant number RPG-2021-080. 
 
\AtNextBibliography{\footnotesize} 
\printbibliography

\end{document}