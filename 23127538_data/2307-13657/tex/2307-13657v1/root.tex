%%%%%%%%%%%%%%%%%%%%%%%%%%%%%%%%%%%%%%%%%%%%%%%%%%%%%%%%%%%%%%%%%%%%%%%%%%%%%%%%
%2345678901234567890123456789012345678901234567890123456789012345678901234567890
%        1         2         3         4         5         6         7         8

\documentclass[letterpaper, 10 pt, conference]{ieeeconf}  % Comment this line out if you need a4paper

%\documentclass[a4paper, 10pt, conference]{ieeeconf}      % Use this line for a4 paper
\usepackage[utf8]{inputenc}
\usepackage[T1]{fontenc}
\usepackage{verbatim}
\usepackage{graphicx}
\usepackage{caption}
\usepackage{subcaption}
\usepackage{wrapfig}
\usepackage{amsmath}
%\usepackage{cite}
\usepackage{biblatex}
\addbibresource{myBib.bib}
\usepackage[dvipsnames]{xcolor}
\newcommand{\customfig}[1]{\textit{Figure \ref{#1}}}
\newcommand{\customfigBracket}[1]{\textit{(Figure \ref{#1})}}



\newcommand\ka{\textcolor{violet}}
\newcommand\kO[1]{\textbf{\textcolor{violet}{KA: #1}}}    %this is useful!
\newcommand\tm[1]{\textbf{\textcolor{blue}{TM: #1}}}
\newcommand\mt[1]{\textbf{\textcolor{red}{MT: #1}}}
\newcommand\kz[1]{\textbf{\textcolor{green}{KZ: #1}}}
\IEEEoverridecommandlockouts                              % This command is only needed if 
                                                          % you want to use the \thanks command

\overrideIEEEmargins                                      % Needed to meet printer requirements.

%In case you encounter the following error:
%Error 1010 The PDF file may be corrupt (unable to open PDF file) OR
%Error 1000 An error occurred while parsing a contents stream. Unable to analyze the PDF file.
%This is a known problem with pdfLaTeX conversion filter. The file cannot be opened with acrobat reader
%Please use one of the alternatives below to circumvent this error by uncommenting one or the other
%\pdfobjcompresslevel=0
%\pdfminorversion=4

% See the \addtolength command later in the file to balance the column lengths
% on the last page of the document

% The following packages can be found on http:\\www.ctan.org
%\usepackage{graphics} % for pdf, bitmapped graphics files
%\usepackage{epsfig} % for postscript graphics files
%\usepackage{mathptmx} % assumes new font selection scheme installed
%\usepackage{times} % assumes new font selection scheme installed
%\usepackage{amsmath} % assumes amsmath package installed
%\usepackage{amssymb}  % assumes amsmath package installed

\begin{document}
\title{\LARGE \bf
	A Soft Robotic Gripper with Active Palm for In-Hand Object Reorientation
}


%$^{1}$ and Bernard D. Researcher$^{2}$
\author{Thomas Mack, Ketao Zhang, Kaspar Althoefer, \textit{Senior Member, IEEE}% <-this % stops a space
    \thanks{This work was supported in part by an Alan Turing Institute funded project on Intuitive human-robot interaction in work environments.}
    \thanks{The first author is funded by an iCASE EPSRC PhD studentship.}
    \thanks{For the purpose of open access, the authors have applied a Creative Commons Attribution (CC BY) license to any Accepted Manuscript version arising.}
    \thanks{Authors are with the Centre for Advanced Robotics @ Queen Mary, School of Engineering and Materials Science, Queen Mary University of London, United Kingdom.
        %\tt\small{t.c.mack@qmul.ac.uk}
        }%
    }
\begin{comment}
I'll sort it, don't worry
should begin document be above?
It was fine before
I know. I think it's supposed to be part of the preamble

In Cem's paper begin_document was above 
still broken
yes...
cry
I leave my fingers off and let you work on this.
What about your PhD studentship? Shall we leave it in? - May as well. Enrollment will probably be finished by the time it would be seen by others (*if* it gets published)
I will look now for the details of the Turing project.
Dawood is available if you need help submitting the paper.
Thanks! I probably will
{thumbsup}
why does it care about whitespace there?
\end{comment}
\maketitle
\thispagestyle{empty}
\pagestyle{empty}


\begin{abstract}

The human hand has an inherent ability to manipulate and re-orientate objects without external assistance. As a consequence, we are able to operate tools and perform an array of actions using just one hand, without having to continuously re-grasp objects. Emulating this functionality in robotic end-effectors remains a key area of study with efforts being made to create advanced control systems that could be used to operate complex manipulators. In this paper, a three fingered soft gripper with an active rotary palm is presented as a simpler, alternative method of performing in-hand rotations. The gripper, complete with its pneumatic suction cup to prevent object slippage,  was tested and found to be able to effectively grasp and rotate a variety of objects both quickly and precisely.



%\tm{Yes of course - Thanks}
%\kO{I am having a break now. Will think about the extra para/sentence later... Paper looks very good! It might be an idea to add Ketao to the list of authors.}
%\tm{Will do. Thank you very much for your help!}

    %Keywords: Soft robot grippers, In-hand manipulation, Pneumatic actuation, Active palm.

\end{abstract}

%%%%%%%%%%%%%%%%%%%%%%%%%%%%%%%%%%%%%%%%%%%%%%%%%%%%%%%%%%%%%%%%%%%%%%%%%%%%%%%%
\section{Introduction}

	The act of repositioning objects within our grasp comes so naturally to humans that we often barely notice ourselves doing it. When lifting a pen we usually have to reorient it before we start writing - a precise action that most humans are able to execute in half a second. Some are even able to deftly twirl pens and similar objects between their fingers. Removing the pen lid is more challenging, but most people are able to accomplish this task single-handedly using three fingers to grip the pen and the thumb and index finger to lift the lid off. While writing, the fingers accurately move the nib in a plane to trace out letters legible enough to be read by others.

 % Figure environment removed
%%%%%%%%%%%%%%%%%%%%%%%%%%%%%%%%%%%%%%%%%%%%%%%%%%%%%%%%%%%%%%%%%%%%%%%%%%%%%%%%

	Replicating this level of dexterity, either fully or partially, in a robotic manipulator is clearly a huge challenge for the industry, on account of the myriad of applications it would proffer. Assembly lines and packing robots would benefit enormously from this technology as items with arbitrary shapes and weights could be easily reoriented by the gripper. Potentially it could even remove the need for specialised manipulators for different stages of production.
	
	Most environments and implements encountered in everyday life are designed for humans use. Underlying any design is an inherent assumption that the humans working in those environments or using those implements will have the level of dexterity that comes so naturally. A general purpose collaborative robot would, of course, benefit from human-like dexterity in most basic tasks such as opening doors, or reorienting objects to make them easier for a human to receive. Advanced generalised grippers would be able to handle tools, enabling them perform the same tasks as humans without the need of specialised end effectors.
	
	%HRI should probably go here - if included: 
	    %presenting to others (want to be in a convenient position)
	        %just generally useful for performing human-like tasks
	    %may wish to be more personal? closer to interacting with a human than a robot?
	    %extra dexterity to avoid injuring human?
	    %objects likely to be used by the human will probably need a human level of skill to manipulate?
	        %assisting human in a task may require the use of the same tools?
        %
	    %could benefit cognitive robotics field? - let the robots/AIs freely interact with the world
	    %extra dexterity/freedoms of movement in manipulators help with communication (bit of a stretch)
	    
    %Original: A gripper capable of reorientating objects in its palm would also be useful for the handover of objects, such as tools, to humans. A tool when picked up by a robot gripper with the intention to pass this on to the human is rarely in the right orientation for the human to be received in a way that they can use the tool for an intended task.
    
    During interactions with a robot, a human would find it most useful if objects, such as tools, were presented to them in a convenient manner. A tool, when picked up by a robot gripper with the intention to pass this on to a human, would rarely be in an orientation that can be received easily and safely.
    
    Here, a robotic system capable of rearranging the grasped tool (e.g., a screwdriver) such that when the handover to the human occurs, the human can receive it handle first, to conduct the task (e.g., fastening screws). It would immediately, make a difference in many industrial settings such as manufacturing, assembly, maintenance and inspection.
    
	%assistive/prosthetics
	The development of robotic in-hand manipulation would also enhance prosthetics and other assistive technologies. Current commercial robotic prosthetic hands are very expensive and rely mostly on an external interface setting different poses that can be proportionally controlled or toggled \cite{prosthetics} using electromyography (EMG) \cite{emg}. However, the muscles required for EMG are not always intact. Even if finger control via EMG is possible, it may not offer the precision required to manipulate objects within the hand. A method of automatically grasping everyday objects would be a great enhancement in terms of quality of life, eliminating the need to changing the grasp pose for each different object. Indeed the ability to reorient an object within the grasp using a simple control scheme would be the next step and an ideal solution if only a limited number of EMG inputs were available.
	
	%anthro
	%\kO{Do you mention the SHADOW Hand?} \tm{No, it was mostly as an example. I can remove it or write something related to it if need be}. \kO{maybe a weblink to the Shadow hand?}. Wait - I have linked an article to the shadow hand. I'll look for the proper website once I've submitted the first time.
	Many different approaches to creating grippers capable of in-hand manipulation have been adopted in the past. As the most prominent example of a high precision, fast manipulator is the human hand, many anthropomorphic designs have been created that try to closely mimic it \cite{shadowhand}\cite{RBO3}\cite{floppy} or that are at least inspired by it. However, their inherent complexity makes them difficult to build. A human hand has over 23 degrees of freedom \cite{anatomy} and their positions and torque can be sensed with inbuilt proprioception \cite{proprioception}. Tendons are often used for actuation in robotic hands as fitting so many actuators and force and position sensors in to the fingers is almost impossible when working at a scale similar to a human hand at least with today's technology; using tendons, sensors can be situated remotely, e.g., in the arm that the gripper is mounted on.
	
	%skin
	Human skin is a soft, thin, waterproof covering filled with biological sensors that provide us a wealth of information when grasping. Robotic skin is a field of study in its own right \cite{skin} and has many applications beyond in-hand manipulation. Our skin has directional force sensors that let us gauge not just the weight of an object but the torque that it exerts on our hands, which we can use to infer its centre of mass - a function which has been implemented in some grippers \cite{torqueBased}. These force sensors can also detect texture when moved over a material. Combined with the ability to sense temperature, a human can make a reasonable estimate as to the material properties through touch alone. This would all be useful information to a robotic gripper, potentially allowing it to adjust its grasp for individual objects without visual information.
	
	%human hand needs human brain?
	Controlling such a complex system to perform in-hand manipulation presents its own set of challenges. While the human hand can be used as an example for manipulation strategies, it is computationally expensive to model an object with five distinct points of contact. Most finger based in-hand manipulation is performed with the fingertips grasping objects from the side, incurring a risk of slipping which would need to be mitigated by the robot. To do this it would need to have knowledge of the object's shape and properties to successfully plan movements. This information may need to be gathered using computer vision or through contact sensors on the robot which adds another layer of complexity.

    %non-anthro
	Non-anthropomorphic systems have also been explored as simpler solutions to in-hand manipulation \cite{mccann}\cite{twoFinger}\cite{paralell}. They will often use fewer fingers or non-standard methods to manipulate objects. Instead of trying to achieve a wide variety of manipulation tasks, these grippers may be designed to effectively perform just one or a few tasks. Using fewer fingers makes the robot much easier and cheaper to build and reduces the computational load when modelling and planning movements. However, using the minimum number of fingers for manipulation means there are no extra redundant fingers that can be used to adapt the grasping style mid reorientation. As few as two fingers can achieve limited human-like in-hand rotation and translation \cite{twoFinger}.
	
	%soft
	The recent advance in soft robotics has prompted its incorporation into in-hand manipulation. A soft fingertip will deform during a grasp, distributing the force over a wider area which can reduce the risk of damaging fragile objects. Materials like silicone which are often used in soft robotics also increase friction, reducing the risk of slipping. Whole fingers can be made of soft materials with chambers that can be pressurised to cause continuous bending throughout the structure \cite{abondance}\cite{continuous}\cite{pagoli}, similar to a continuum robot \cite{kasparGift_2017}. These compliant fingers can be designed to grasp a wide variety of shapes as they deform and adapt if too much air pressure is applied. This means soft grippers can often grasp objects without needing their exact dimensions - a key advantage. As the applied force is distributed, fragile objects will also tolerate more excessive force than they would with a rigid gripper. However, deformable materials and the use of pneumatics can introduce non-linearities and make soft actuators more difficult to control, sometimes requiring the use of machine learning \cite{softML}. They can also be prone to leaks and can be easily damaged.
	
	%rigid
	Rigid solutions \cite{mccann}\cite{paralell} are still being explored as the actuators and construction methods are well established. The kinematics are comparatively simple and can allow for far greater precision than an entirely soft robot. However, the rigidity requires precise knowledge of the manipulated object and the system must control the trajectories of all the fingers to ensure stable movement without excessive force.

    %\tm{Would it be an idea to take bits of my literature review and scatter them in relevant places through the introduction?}
    %\kO{The above reads very well. I think that the Intro can stay as it is.}
    %\tm{I'll save it on my PC as a .txt file in case I need it later then.}

\section{Conceptual Design of the 3-Fingered Soft Gripper}
    Some of the more successful grippers utilise a palm built into them for stability. \textcite{abondance} employed a palm which acted as a flat surface on which objects were rotated and translated. \textcite{pagoli} used an in-built suction cup to support objects while the fingers release and re-position them. In both cases, as with many others, these palms are in an effective place to enable in-hand rotation as they lie in the axis of rotation. They can also have adhesive materials or devices attached to stabilise the objects they handle.
    
    Here, a gripper is proposed that utilises a rotating palm with an integrated suction cup for in-hand rotation. As with other solutions, the palm is located centrally, and is surrounded by fingers used for grasping.
    
    The normal use case for this gripper would not involve the fingers during rotation. They would be used to initially grasp the object from above. The entire gripper would then rotate 180$^\circ$ to place the palm below the object. The object would then be released and dropped on to the palm so that it could then be rotated to the desired position. The suction cup would help secure the object to the palm during rotation. The object would then be re-grasped for placing back down onto a surface \customfigBracket{fig:process}.
    
    % Figure environment removed
    
    Using a motor for rotation would allow an object to be turned quickly and precisely. It would also be largely unaffected by interactions between fingers and the object's geometry during rotation in the way that finger based manipulators are. Another advantage over finger based rotation is that no re-grasping needs to take place to achieve large changes in angle, allowing the target orientation to be reached in a single smooth twist.
    
    Because the fingers are not the source of rotation they only need to be able to reliably grasp objects. Soft pneumatic fingers are ideal as they can achieve effective grasping using a minimal control scheme \cite{continuous}. They could simply be designed to open and close, utilising their compliance rather than any specific knowledge of the objects' dimensions. This would however limit the freedom of movement of the fingers, preventing the gripper from translating objects.
    
    The proposed in-hand rotation could still be used in many applications such as in human-robot interaction. Objects could be re-positioned in the manipulator to render them easier to receive by a human. The fingers would also be soft, making any contact with a human safer. This kind of in-hand rotation could also be very useful for delicate tasks like fruit picking or packing. The only force being applied to the fruit would be from suction cup, and that would only be a marginal force, just enough to prevent the fruit from falling off the palm.

\section{Design and Prototyping}
    
    % Figure environment removed
    
    The gripper uses a motorised platform for the palm with an integrated pneumatic suction cup \customfigBracket{fig:palm}, attached directly at the base to an XL-320 servo motor by ROBOTIS. The suction cup is moulded from the softest variety of Ecoflex silicone from SmoothOn in the shape of a half torus shell. It is hollow to make it more deformable so as to form a better seal with non flat objects. A channel that passes from the inside of the suction cup to below the outside of the palm is designed house a silicone tube that flexes as the palm twists \customfigBracket{fig:palm:cut}. This tube is fed through the chassis and used to propagate the vacuum that is toggled with a valve. 
    
    % Figure environment removed
    
    The palm is then surrounded by three soft pneumatic fingers for grasping that all flex simultaneously. They are mounted on the chassis and spread out at 25$^\circ$ to increase the size of objects that they are able to handle.
    
    Different types of continuous pneumatic silicone fingers were designed and evaluated to find an optimal version that would allow most objects to be grasped. The soft nature of the fingers meant that they would naturally adapt to different sizes and shapes, which then left maximum liftable weight as the principal challenge.
    
    %Because pneu-net style actuators often flex to extreme angles and curl to angles over 180$^\circ$, the fingertips were made cylindrical. This was to encourage the fingertips to roll over objects as the fingers curled, bringing it closer to the palm.
    Each finger was moulded from Dragonskin 30. A pneu-net type design inspired by \textcite{abondance} and softroboticstoolkit.com \textcite{toolkit} was first implemented, with a single set of chambers that inflate to continuously flex the finger \customfigBracket{fig:finger:fabric}. The actuators were fabricated in a single step using a 3D printed, Polylactic Acid (PLA) mould and a Polyvynil Acetate sacrificial insert. Because pneu-net style actuators often flex to extreme angles and curl over 180$^\circ$, the fingertips were made cylindrical. This was to encourage the fingertips to roll over objects as the fingers curled, bringing it closer to the palm. A layer of fabric was glued to the inside edge of the finger to prevent extension. However, when grasping was attempted, they would often deform laterally and rotate past objects, dropping them.
    
    An attempt was then made to stiffen the finger laterally. The design was modified to have a thicker base in which a polyethylene strip could be embedded \customfigBracket{fig:finger:plastic}. Both of these increased the stiffness of the finger, and the polyethylene strip resisted lateral deformation.

    % Figure environment removed

    % Figure environment removed
    
    Another pneumatic finger was designed with an oval cross section to encourage bending over its shortest length. A semicircular cavity was placed in the outer side of the finger. It was noticed that due to variations in the curing conditions each finger had a slightly different response to pressure which was rectified by having a separate SMC ITV2050 pressure regulator with its own control curve.
    
    All of the silicone fingers were attached to 3D printed PLA inlets using Sil-Poxy with 4mm pipes glued in to connect them to the pressure regulators. The inlets were also used to hold the fingers on to the chassis.
    
    %Initially, they were moulded from Dragonskin 30 picture.fig. (They were oval shape to encourage the fingers to bend towards the centre of the gripper with a semicircular channel through the outside half crosssection.fig.  || They were inspired by \textcite{abondance} and \textbf{softroboticstoolkit.com} and were designed with a single set of cavities lining the outer edge and a thick layer on the centre facing edge. A 1mm layer of polyethylene was also embedded in the thick layer to provide extra stiffness and to make it resist lateral deformation. ) The fingertips were made to be cylindrical to avoid them camming around and moving past grasped objects. It was noticed that due to variations in fabrication \textbf{curing conditions?} each finger had a slightly different response to pressure which was rectified by having a seperate SMC ITV2050 pressure regulator with its own control curve.
    
    They were later replaced by a set of three 3D printed soft fingers similar manufactured by Inkbit. These had a uniform response to pressure so could be actuated using a single regulator. They were also much lighter and would not deform under their own weight, though they were shorter, which limited the size of graspable objects.
    
    %\kO{These were made by a company, specialising in 3D printing. Can't remember the name. Wasn't it written on the fingers? PERFECT}. 
    
    The chassis was 3D printed from PLA in two parts. The base plate had bolt holes to attach the servo motor \customfigBracket{fig:chassis:base}. The cover had mount-points for each finger recessed into slanted edges that were designed for easy replacement \customfigBracket{fig:chassis:top}. The palm attached directly to the servo motor through a hole in the cover.
    
    % Figure environment removed
    
    The control system consisted of an Arduino, a Raspberry Pi and a PC linked together using a ROS framework. The Raspberry Pi and Arduino provided a 0-5V pulse-width-modulated output to each pressure controller, a serial output to the motor, and toggled the 24V supply to the valve for the vacuum pump, while the PC was used as the controller.
    
    When the moulded fingers were used, the Arduino was programmed to map a single linear input from the PC to the three separate voltage ranges for the pressure controllers. The ranges were manually set to align the responses, allowing the fingers to bend simultaneously. When using the 3D printed fingers, a single actuation single going to all three fingers in parallel was sufficient.

\section{Experimental Study}
    
    Grasping and rotating were tested with a variety of different weights and shapes to ascertain how well the gripper would cope with manipulating real-world objects. The sizes of objects used were restricted to those within the grasping capabilities of the fingers. They had to be small enough to be grasped and below 80g for the fingers to be able to lift them.
    
    The tests were carried out by performing the full grasping process on each object five times. All of the objects were tested with both the oval moulded fingers and the 3D printed fingers \textit{(Table \ref{tbl:weights})}. Each one was grasped from above, rotated 180$^\circ$ to above the gripper, dropped on the palm, re-oriented, re-grasped and placed back down. Because the whole sequence had to be completed without error for an object to be deemed as having been successfully manipulated, the individual stages were carefully observed to ascertain how a particular object's properties affected the process. If, for example, the process repeatedly failed at a particular stage, the test would be continued at the start of the subsequent stage.
    
    %\tm{Table of items? I made it so let me know what you think} \kO{good}
    
    %\begin{comment}
    \begin{table}
        \begin{center}
            \begin{tabular}{|| l | r ||}
                Object & Mass (g) \\
                \hline
                Styrofoam Egg & 1 \\
                Cylindrical container & 33 \\
                Glove & 40 \\
                Tape & 50 \\
                Tennis ball & 62
            \end{tabular}
            \caption{A list of the objects used to test the gripper and their associated weights}
            \label{tbl:weights}
        \end{center}
    \end{table}
    %\end{comment}

    A styrofoam egg weighing 1g was tested first as it was an ideal shape to be grasped and rotated \customfigBracket{fig:experiment}. The moulded fingers had some trouble grasping it as the non-uniform bending would push it out of the centre of the grasping area. They also visibly sagged under their own weight while the gripper was turning face up. The 3D printed fingers grasped and rotated the egg with no issues. Both sets of fingers were able to drop the egg on to the palm easily. The 3D printed fingers held it very close to the palm, so there was little room for deviation. The conical shape the oval fingers that formed when relaxed guided it towards the palm. Once on the palm, the egg could be freely rotated within the servo motor's range of movement. Re-grasping the egg was only possible with the 3D printed fingers as the moulded fingers converged above the egg when flexing.
    
    % Figure environment removed
    
    %\mt{The 3D printed fingers had less friction than the Dragonskin fingers and could partially release the container, letting it slide through them and removing the risk of it falling sideways if it was not already touching the palm. NOT CLEAR.}\tm{Could just remove this. It's not that important.} \kO{Okay, let's remove this.} 
    A cylindrical container was then tested as although it was still an easy shape to grasp, it would have a different weight distribution. Both sets of fingers reliably grasped the container but while rotating it above the palm, the moulded fingers would deform and lose grip. The 3D printed fingers were shorter and could not grasp the cylinder around its center of mass, so the cylinder sometimes twisted out of their grasp. When it was dropped on to the palm it rotated easily. Both sets of fingers were also able to re-grasp the cylinder as it was long enough to pass through the aperture created when the moulded fingers converged. When rotating the palm face down, the moulded fingers were able to support it as they pressed it against the palm. It acted as a fourth point of contact, preventing the cylinder from twisting.
    
    A roll of tape weighing 58g was tested as a heavy object. It was possible to lift with both sets of fingers, but only when held by the gripper with the 3D printed fingers, the tape could be consistently rotated without it dropping. Because it was wide, when it was dropped by the silicone fingers, the differential responses would cause one finger to stay in contact, tipping it as it fell. The roll of tape would then hit the palm at an angle, and either bounce off, falling between the fingers or rest partially on the palm with one edge on a finger.
    
    A rubber glove was also tested to simulate cloth-like items. These were grasped successfully by the 3D printed fingers. The moulded fingers only successfully grasped the glove after pinching it, raising part of the glove off the table to a height at which the cylindrical fingertips could then grasp it. Dropping on to the palm was difficult as the glove would end up draped over the fingers. When it was successfully dropped on to the palm, the glove would fall into the gaps between the fingers prevent it from being turned, even with the suction cup active.
    
    The whole manipulation process was consistently performed successfully on a tennis ball weighing 62g. The fingers interacted well with it as they could reach under it, cage it, and pull it onto the palm before the gripper was rotated to the face up position. This was advantageous as having the center of mass of the object closer to the palm exerted a smaller force on the fingers. As a consequence the ball was never dropped, despite being the heaviest object tested.

\begin{comment}
\section{Discussion}
    Rotating objects using the motorised palm was successful in all cases except for the glove and tape. Cloth like items like the glove get caught in the fingers and are unable to be rotated even with the vacuum. The tape would land with an edge still resting on the inside of the fingers which would prevent rotation. It can be seen that the fingers can obstruct and limit rotation in their current configuration.
    
    To avoid this problem, the fingers could be made retractable by sliding down the gripper. Another solution would be to make the fingers rotate away from the palm until they face backwards after an object was dropped on to the palm. The second solution would also increase the range of sizes that the gripper could grasp and allow the fingers to reach through narrower openings.
    
    Once rotated, non-circular objects could be displaced by the fingers to an undesired angle when re-grasping as seen with tape. If the positions of the fingers around the palm could be partially readjusted in real time similar to Lu Q (2021), it would be possible to avoid this problem. However, the complexity would be increased significantly as the gripper would no longer just be relying on the soft fingers adapting to the object to grasp it. A computer-vision approach or similar may be required to properly orient each finger to not disrupt the object.
    
    A different solution might be to increase the number of soft fingers as there would be more points of contact between the fingers and object and a higher chance of some of them being stable.
    
    If the fingers were made independent, it may be possible with the use of contact sensors to close each one until they applied pressure. During this, the suction cup would be able to hold most objects in place, preventing any extra displacement until a secure, even grasp had been established.
    
    It was seen during testing that the palm can also be used for lifting objects. While this might make the fingers redundant in some cases, the fingers will be able to grasp most of the items that the palm can't lift which can be used as a form of redundancy. If used in tandem, they will also increase the maximum liftable and manipulable weight.
\end{comment}

\section{Conclusion}
	A soft gripper with rotary palm was designed and implemented. It was tested with a variety of objects and found to be effective at grasping and reorienting them. After successfully grasping, objects weighing up to 62g were easily rotated above 600$^\circ$ per second without slipping. Once on the palm, the objects' dimensions and shape did not effect accuracy - a common occurrence in other fingered grippers. The exceptions were cloth-like objects and large objects which were both obstructed by the surrounding fingers.

	The suction cup on the palm was demonstrably powerful enough to lift objects without using the fingers, which adds extra redundancy to the design. It could also be used in tandem alongside the fingers to increase the maximum lifting capacity.
	
    In future work, the fingers should be made reorientable and retractable to allow for larger objects to be rotated, as the base of the fingers can obstruct larger objects during rotation. The ability to change the angle between the fingers and the centre of the gripper would enable us to address issues such as flexible, cloth-like objects partially falling down in-between the fingers, and wide objects resting on only the palm and not the fingers. A wider spreading of pressurised fingers would also allow wider objects to be grasped. 
    %\mt{PLEASE CHECK PARA ABOVE - HAVE TRIED TO CLARIFY.} \tm{reworded slightly for clarity} \kO{Sounds very good to me now.}
    
    It may also be necessary to find a method of preventing the fingers from displacing non-circular objects when re-grasping them. One possible method would be to add contact sensors and individually inflating the fingers until they applied pressure to the object while it is held in place by the vacuum palm. An alternative would be to make the fingers reorient themselves to re-grasp the object in a way that does not disturb it, but that would significantly increase the complexity of the gripper and would require knowledge of the object's shape.
	
	The fingers of the gripper are currently too weak for practical application as they can only reliably lift about 80g. This could be improved by using fibre reinforced silicone or fabric based fingers. Adding electroadhesion would also be a space efficient addition for improving grasping quality.
	%-- PERFECT} \tm{nice}
	
	%The maximum lift-able weight of the fingers should be brought closer to that of the palm so that they can better assist grasping. \mt{Currently when gripping a smooth object from the side, they can only hold a maximum of 80g which is completely outclassed by the vacuum. YOU MEAN THAT THE VACUUM FORCE IS GREATER?} \tm{yes. Maybe insert something like: This may be possible to achieve with fibre reinforced silicone or fabric based fingers.} \kO{I suggest to rewrite this paragraph. Say something like the current fingers are weak and could be improved by fibre reinforcement or fabric based materials. The suction capabilities of the palm were more than adequate in our prototypes.}

\addtolength{\textheight}{-12cm}   % This command serves to balance the column lengths
                                  % on th}e last page of the document manually. It shortens
                                  % the textheight of the last page by a suitable amount.
                                  % This command does not take effect until the next page
                                  % so it should come on the page before the last. Make
                                  % sure that you do not shorten the textheight too much.



%\section*{APPENDIX}



\section*{Acknowledgment}

%\tm{What do I need to put here?} \kO{Thank Mish for his help.}
Thank you very much to Mish Toszeghi for proof reading.
%\tm{Anyone else? - NO. It's enough}

\printbibliography
%%%%%%%%%%%%%%%%%%%%%%%%%%%%%%%%%%%%%%%%%%%%%%%%%%%%%%%%%%%%%%%%%%%%%%%%%%%%%%%%

\end{document}
