\section{Ex-ante envy-free and Pareto-optimal lotteries of high social welfare}\label{sec:np-hardness}
As our last technical contribution, we study the problem of optimizing social welfare over ex-ante envy-free and Pareto-optimal allocation lotteries and prove the following statement for its decision version. 

\begin{theorem} \label{thm:NPhard}
    The problem of, given a fair division instance with partition-based utilities and $K>0$, deciding whether there exists an ex-ante envy-free and Pareto-optimal allocation lottery of social welfare at least $K$ is \NP-complete.
\end{theorem}
%%NEED TO SPECIFY WHAT K IS
% an ex-ante envy-free and Pareto-optimal lottery with a social welfare of at least $K=18+\frac{r+1}{t}-\varepsilon(r+6t)$, where $n=t+\frac{t}{\eps}+3r+3$ and $m=3t$, for some small $\eps>0$.

It is easy to see that the above problem belongs to the complexity class \NP. First, notice that it is trivial to check whether a given lottery $\q$ is ex-ante envy-free and has social welfare at least $K$. To verify Pareto optimality, it suffices to search for another lottery $\widetilde{\q}$ which gives to any agent expected utility at least as high as her expected utility in $\q$, maximizing the total excessive utility, through the following linear program:
\begin{equation*}
  \begin{array}{ll@{}ll}
    \text{maximize}  & \sum\limits_{i=1}^n{t_i} &\\
    \text{subject to}& \displaystyle u_i(\widetilde{\q};i)\geq u_i(\q;i)+t_i\quad\quad &\text{for all} \ i \in [n]\\
    \displaystyle & \sum\limits_{i=1}^n \widetilde{q}^k_{ij} =\widetilde{p}_k &\text{for all} \ j \in [n], k \in [m]  \\
    \displaystyle  & \sum\limits_{j=1}^n \widetilde{q}^k_{ij} =\widetilde{p}_k & \text{for all} \ i \in [n], k \in [m] \\
    \displaystyle    & \sum\limits_{k=1}^m{\widetilde{p}_k}=1 & \\
    \displaystyle     & \widetilde{q}_{ij}^k\geq 0 & \text{for all} \ i,j \in [n], k \in [m] \\
    \displaystyle    & \widetilde{p}_k\geq 0 & \text{for all} \ k\in [m]\\
    \displaystyle   & t_i\geq 0 & \text{for all} \ i\in [n]
  \end{array}
\end{equation*}
Clearly, the lottery $\widetilde{\q}$ Pareto-dominates $\q$ if and only if the objective value of the above linear program is strictly positive.

For proving \NP-hardness, we will develop a polynomial-time reduction from the classic \NP-complete problem \emph{Exact Cover by 3-Sets} (X3C) \cite{Karp1972} to our problem. X3C is defined as follows:
\begin{quote}
\textit{Instance}: A universe $\mathcal{E}= \{e_1, e_2, \dots, e_r\}$ of $r$ elements, a family $\mathcal{S}=\{S_1, S_2, \dots, S_t\}$ of triplets from $\mathcal{E}$, i.e., $S_j \subseteq \mathcal{E}$ with $\abs{S_j}=3$ for all $j \in [t]$. 

\textit{Question:} Does there exist an exact cover, i.e., a set of $r/3$ triplets from $\mathcal{S}$ that includes all elements of the universe $\mathcal{E}$? 
\end{quote}

For $i\in [r]$, we let $f_i$ denote the frequency of occurrence of element $e_i$, i.e., $f_i := \abs{\{j:e_i \in S_j\}}$. 

\subsection{The reduction} \label{sec:construction}
Starting with an instance $\phi$ of X3C, our reduction constructs a fair division instance $\In$ as follows. Instance $\In$ has the following set of $n=t+1+2t^2+3r$ agents.
\begin{itemize}
    \item $t+1$ \emph{base agents} $b_0, b_1, b_2, \dots, b_t$, 
    \item $2t$ \emph{set agents} $h_{j,1}, h_{j,2}, \dots, h_{j,2t}$, for every $j\in [t]$,
    \item three \emph{element agents} $v_i,w_i,$ and $z_i$ for every $i\in [r]$ 
\end{itemize}
 The set $\mathcal{P}$ of admissible partitions of an underlying set of items consists of $m=3t$ partitions $P_{j,c}$ for $j \in [t]$ and $c \in [3]$. We identify the $n$ bundles of partitions in accordance to the type of agents. So, each partition has $t+1$ bundles $B_0, B_1, \dots, B_t$, $2t$ bundles $H_{j,1}, \dots, H_{j,2t}$ for every $j \in [t]$, and three bundles $V_i, W_i$, and $Z_i$ for every $i \in [r]$.

%\[P_{j,c} = \{ A_1, A_2, B_0, B_1, \dots B_t, H_{x,y} : x \in [t], y \in [p], V_i, W_i, Z_i : i \in [r]\}\]


The utilities of the agents for the bundles of partition $P_{j,c}$ for $j \in [t]$ and $c \in [3]$ are given in the following table. The table includes only non-zero utilities; any utility that is not specified in the table is equal to zero. In our reduction, we use parameters $\eps = \frac{1}{12t^2}, R=\frac{6t^3}{\eps}$, and $Q=\frac{6t}{\eps}$.


%and hence they can have different utilities for the same bundle occurring in two distinct partitions. We now define the utility functions of agents for the partitions in $\mathcal{P}$ as follows. The non-zero utilities of agents in a partition $P_{j,c}$ for $j \in [t]$ and $c \in [3]$ are specified in the following table.
    
  \begin{center}
     \begin{tabular}{llll}
\toprule
  $c$ & agent & bundle & utility\\\midrule
 any   &  $b_0$ & $B_0$ & $R/t$ \\
 & $b_0$ & $B_j$ & $R$ \\
  & $b_j$ & $B_j$ & $R$ \\ 
& $z_i$ for $i \in [r]:e_i \in S_j$ & $Z_i$ & $1/f_i$ \\ \midrule
  $1$ & $h_{j,1}$ & $H_{j,1}$ & $Q$ \\
 & $v_i$ for $i \in [r]:e_i \in S_j$ & $V_i$ & $2$ \\
 & $z_i$ for $i \in [r]:e_i \in S_j$ & $V_i$ & $2/3$ \\ \midrule
 $2$ & $h_{j,2}$ & $H_{j,2}$ & $Q$ \\
  & $w_i$ for $i \in [r]:e_i \in S_j$ & $W_i$ & $2$ \\ 
   & $z_i$ for $i \in [r]:e_i \in S_j$ & $W_i$ & $(1+1/f_i)/f_i$  \\ \midrule
  $3$ & $h_{j,1}$ & $H_{j,1}$ & $Q(1-\eps)$ \\
 & $h_{j,2}$ & $H_{j,2}$ & $Q(1-\eps)$ \\
 & $h_{j,\ell}$ for $\ell = 3, ..., t+1$ & $H_{j,1}$ & $\eps$ \\
  & $h_{j,\ell}$ for $\ell = t+2, ..., 2t $ & $H_{j,2}$ & $\eps$ \\
  & $v_i$ for $i \in [r]:e_i \in S_j$ & $V_i$ & $2$ \\
   & $w_i$ for $i \in [r]:e_i \in S_j$ & $W_i$ & $2$ \\ 
    &  $z_i$ for $i \in [r]:e_i \in S_j$ & $V_i$ & $2/3$\\
    &  $z_i$ for $i \in [r]:e_i \in S_j$ & $W_i$ & $(1+1/f_i)/f_i$\\
 \bottomrule 
\end{tabular}
\end{center}

%The instance $\In$ has $n=t+tp+3r+1$ agents and $m=3t$ partitions. 
The reduction is clearly computable in polynomial time. We shall, without loss of generality, assume in the following that $t\geq 9$ and $r\leq3t$; otherwise, it is trivial to decide~$\phi$.

\begin{definition}[Canonical allocation]
For any partition $P_{j,c}$ with $j \in [t]$ and $c \in [3]$, we define the \emph{canonical} allocation as follows: bundle $B_k$ is assigned to base agent $b_k$ for $k \in \{0,1,\dots,t\}$, bundle $H_{j,\ell}$ is assigned to set agent $h_{j,\ell}$ for $\ell \in [2t]$, and, finally, bundle $V_i$ is assigned to element agent $v_i$, bundle $W_i$ is assigned to element agent $w_i$, and $Z_i$ is assigned to element agent $z_i$ for $i\in [r]$.
%
%We shall also say that an agent is assigned the canonical bundle in a partition when the agent is assigned the bundle specified by the canonical allocation.
\end{definition}


\subsection{Proof of Theorem~\ref{thm:NPhard}}
We now prove the correctness of our reduction. We remark that when we refer to the expected social welfare achieved by a set $F$ of agents in a lottery $\q$, we refer to the sum of the expected utilities of agents in $F$ in $\q$. We begin by presenting two simple technical lemmas.


%Corollary~\ref{cor:binary_switch}, and Lemma~\ref{claim:element_agents}) about the structure of ex-ante envy-free and Pareto-optimal lotteries in the fair division instance $\In$. 

%For the ease of our understanding, we will denote the three partitions $P_{j,1}, P_{j,2}$, and $P_{j,3}$ collectively as (partition) $P_j$ for every $j \in [t]$. 

\begin{lemma} \label{lem:SW_set_element}
    Consider an ex-ante envy-free lottery of instance $\In$. For $j \in [t]$, the expected utility the set and element agents can get from each of the partitions $P_{j,1}$ or $P_{j,2}$, conditioned on the partition being the outcome of the lottery, is at most $Q+9$. Similarly, the expected utility the set and element agents can get from the partition $P_{j,3}$, conditioned on the partition being the outcome of the lottery, is at most $Q/2+9$.
\end{lemma}

\begin{proof}
    Consider a lottery $\q$ of instance $\In$ and let $j \in [t]$. In partitions $P_{j,1}$ and $P_{j,2}$, the set agents $h_{j,1}$ and $h_{j,2}$ can get a utility of at most $Q$, while the maximum utility from the element agents is $2$ from each of the six bundles $V_i$ or $W_i$ for each $e_i \in S_j$ and $1$ from each of the three bundles $Z_i$ for $e_i \in S_j$. Overall, the expected utility set and element agents get from each of the partitions $P_{j,1}$ and $P_{j,2}$, conditioned on the partition being the outcome of the lottery, is at most $Q+9$.

Now, assume that the lottery $\q$ is ex-ante envy-free. Consider partition $P_{j,3}$ and observe that the set agents $h_{j,1}, h_{j,3}, \dots, h_{j,t+1}$ have utility only for the bundle $H_{j,1}$ of partition $P_{j,3}$. Due to ex-ante envy-freeness, all these agents receive bundle $H_{j,1}$ with conditional probability $1/t$. Similarly, all set agents $h_{j,2}, h_{j,t+2}, \dots, h_{j,2t}$ receive bundle $H_{j,2}$ with conditional probability $1/t$. Also, the maximum utility that can be obtained in partition $P_{j,3}$ from the element agents is $2$ from each of the six bundles $V_i$ and $W_i$ for $e_i \in S_j$, and $1$ from each of the three bundles $Z_i$ for $e_i \in S_j$. Overall, using our assumption $t \geq 9$, which clearly also gives $Q \geq 36$ (recall that $Q=\frac{6t}{\eps}=72t^3$), we have that the expected utility set and element agents get from partition $P_{j,3}$, conditioned on the partition being the outcome of the lottery, is $\frac{2Q}{t} \cdot (1-\eps) + \frac{2t-2}{t} \cdot \eps +15 \leq \frac{Q}{3}+15 \leq \frac{Q}{2}+9$.
\end{proof}

\begin{lemma} \label{lem:base}
    In instance $\In$, for any partition $P_{j,c}$ with $j \in [t]$ and $c \in [3]$, any allocation in the support of a Pareto-optimal lottery, either assigns bundle $B_j$ to agent $b_0$ or assigns bundle $B_0$ to agent $b_0$ and bundle $B_j$ to agent $b_j$.
 \end{lemma}

\begin{proof}
    Consider a Pareto-optimal lottery $\q$ and assume, for the sake of contradiction, that it has in its support an allocation in partition $P_{j,c}$ for $j\in [t]$ and $c\in [3]$ which assigns to the base agent $b_0$ neither bundle $B_0$ nor bundle $B_j$. Then, since the base agent $b_0$ is the only one who can get positive utility from bundle $B_0$, the lottery $\widetilde{\q}$, which moves probability mass from the above allocation to the one in which the agent who gets bundle $B_0$ and the base agent $b_0$ have their bundles swapped, Pareto-dominates $\q$, contradicting its Pareto-optimality.

Now, assume that $\q$ has in its support an allocation in partition $P_{j,c}$ for $j\in [t]$ and $c\in [3]$, in which the base agent $b_0$ is assigned to bundle $B_0$ but bundle $B_j$ is not assigned to the base agent $b_j$. Then, since the base agent $b_j$ is the only agent besides $b_0$ who has positive utility for bundle $B_j$ at partition $P_{j,c}$, the lottery $\widetilde{\q}$, which moves probability mass from this allocation to the one in which the agent who gets bundle $B_j$ and the base agent $b_j$ have their bundles swapped, Pareto-dominates $\q$, again contradicting its Pareto-optimality. The lemma follows. 
\end{proof}

In the statements and proofs below, for a given lottery, we denote by $p_{j,c}$ the probability of partition $P_{j,c}$ being the outcome of the lottery, for $j\in [t]$ and $c\in [3]$.

The next lemma shows that ex-ante envy-free lotteries of high social welfare must place close to total probability~$\frac{1}{t}$ on the partitions $P_{j,1}$, $P_{j,2}$ and $P_{j,3}$, for each~$j\in [t]$.

\begin{lemma} \label{lem:uniform}
    In any ex-ante envy-free and Pareto-optimal lottery of instance $\In$ in which the base agents have social welfare at least $R+R/t+r/t-3$, it holds that $p_{j,1}+p_{j,2}+p_{j,3} \in [\frac{1-\eps}{t},\frac{1+\eps}{t}]$, for each $j \in [t]$.
\end{lemma}


\begin{proof}
    Consider an ex-ante envy-free and Pareto-optimal lottery $\q$ of social welfare at least $R+R/t+r/t-3$ for the base agents. For $j \in [t]$, denote by $\theta_j$ the total probability that bundle $B_0$ is assigned to agent $b_0$ in partitions $P_{j,1}, P_{j,2}$, and $P_{j,3}$. Also, let $\theta=\sum_{j \in [t]}\theta_j$ and $p_j=p_{j,1}+p_{j,2}+p_{j,3}$ for $j \in [t]$; clearly, $\sum_{j\in [t]}{p_j}=1$. By Pareto-optimality and Lemma~\ref{lem:base}, agent $b_0$ gets bundle $B_j$ in partitions $P_{j,1}, P_{j,2}$, and $P_{j,3}$ with total probability $p_j - \theta_j$ for each $j\in [t]$. Then, the expected utility of agent $b_0$ is 
    \[
    u_{b_0}(\q;b_0)=\theta \cdot R/t+ \sum_{j \in [t]}(p_j-\theta_j) \cdot R = \theta \cdot R/t + (1-\theta)\cdot R \enspace .
    \]

    Now, consider the base agent $b_j$ for $j \in [t]$. By Pareto-optimality and Lemma~\ref{lem:base}, this agent gets bundle $B_j$ with total probability $\theta_j$ in partitions $P_{j,1}, P_{j,2}$, and $P_{j,3}$ (i.e., whenever agent $b_0$ gets bundle $B_0$). Hence, we have $u_{b_0}(\q;b_j)=\theta_j \cdot R$. By ex-ante envy-freeness of the lottery $\q$, we have 
    \[
    \theta \cdot (R/t) + (1- \theta) \cdot R=u_{b_0}(\q;b_0) \geq u_{b_0}(\q;b_j) = \theta_j \cdot R \enspace ,
    \] i.e., 
    \begin{align} \label{eq:1}
        \theta_j \leq 1-\theta\cdot (1-1/t) 
    \end{align}
  for all $j \in [t]$.
   The expected utility of the base agent $b_j$ in partitions $P_{j,1}, P_{j,2}$, and $P_{j,3}$  is $\theta_j \cdot R$. In total, the social welfare of the base agents is
   \[
   u_{b_0}(\q;b_0)+ \sum_{j \in [t]}u_{b_j}(\q;b_j) = \theta \cdot R/t+  (1-\theta) \cdot R + \sum_{j \in [t]}\theta_j \cdot R = R+ \theta \cdot R/t \enspace .
   \]
   Since the social welfare of the base agents is at least $R+R/t+r/t-3$, we have 
   \begin{align} \label{eq:2}
       \theta \geq 1 - \frac{3t-r}{R}.
   \end{align}
   We now claim that 
   \begin{align} \label{eq:3}
       p_j \leq \theta_j+ \frac{3t-r}{R}
   \end{align}
    for every $j \in [t]$. Indeed, assume that this is not the case and, instead, $\theta_{j^*} < p_{j^*} - \frac{3t-r}{R}$ for some $j^* \in [t]$. Using the inequality $\theta_j \leq p_j$ for $j \in [t] \setminus \{j^*\}$, and summing these inequalities up, we get $\theta = \sum_{j \in [t]} \theta_j< \sum_{j \in [t]}p_j - \frac{3t-r}{R}= 1- \frac{3t-r}{R}$, contradicting inequality~(\ref{eq:2}). Using equations~(\ref{eq:3}), (\ref{eq:1}), and (\ref{eq:2}) (in this order), we have 
    \begin{align*}
        p_j &\leq \theta_j+\frac{3t-r}{R} \leq 1-\theta \cdot (1-1/t)+ \frac{3t-r}{R}  \leq 1- \left(1-\frac{3t-r}{R}\right) \cdot \left(1-1/t\right) + \frac{3t-r}{R} \\
        &= \frac{1}{t} + \frac{3t-r}{R} \cdot \left(2-\frac{1}{t}\right)  \leq \frac{1}{t} + \frac{\eps}{t^2},
        \end{align*}
        which clearly implies the desired upper bound on $p_j$. The last inequality follows by the definition of $R$ (recall that $R=\frac{6t^3}{\eps}$). Then, 
        \begin{align*}
            p_j = 1-\sum_{j'\in [t]\setminus\{j\}}p_{j'} \geq  1-(t-1)\cdot \left(\frac{1}{t}+\frac{\eps}{t^2}\right) \geq \frac{1-\eps}{t},
        \end{align*}
        which completes the proof.
 \end{proof}



Our next technical lemma shows that in Pareto-optimal lotteries, almost all of the total probability given to the two partitions~$P_{j,1}$ and $P_{j,2}$ is given to one of them.

\begin{lemma} \label{lem:PO}
    Any Pareto-optimal lottery in instance $\In$ satisfies $\max\{p_{j,1},p_{j,2}\} \geq (1-~\eps)(p_{j,1}+p_{j,2})$, for all $j \in [t]$.
\end{lemma}

\begin{proof}
The claim is clear for $j \in [t]$ such that $p_{j,1}=0$ or $p_{j,2}=0$. So, consider a Pareto-optimal lottery $\q$ with $p_{j^*,1}>0$ and $ p_{j^*,2}>0$ for some $j^* \in [t]$. For the sake of contradiction, let $\max\{p_{j^*,1},p_{j^*,2}\} < (1-\eps)(p_{j^*,1}+p_{j^*,2})$.

We construct the lottery $\widetilde{\q}$ which has the same probability as $\q$ for every allocation in partition $P_{j,c}$ with $j \in [t] \setminus \{j^*\}$ and $c \in [3]$, probability $0$ for every allocation in partition $P_{j^*,1}$ and $P_{j^*,2}$, and a probability for the canonical allocation of partition $P_{j^*,3}$ that is $p_{j^*,1}+p_{j^*,2}$ higher than the corresponding probability in $\q$. In other words, compared to $\q$, $\widetilde{\q}$ has moved the probability mass of allocations of partitions $P_{j^*,1}$ and $P_{j^*,2}$ to the canonical allocation of partition $P_{j^*,3}$. Note that, the base agents, the set agents different than $h_{j^*,1}$ and $h_{j^*,2}$, and the element agents have at least as high expected utility in $\widetilde{\q}$ as in $\q$. Agent $h_{j^*,1}$ (respectively, $h_{j^*,2}$) gets utility (at most) $Q$ from partition $P_{j^*,1}$ (respectively, $P_{j^*,2}$), and utility $Q(1-\eps)$ from (the canonical allocation of) partition $P_{j^*,3}$. Hence, the increase of expected utility from $\q$ to $\widetilde{\q}$ for agents $h_{j^*,1}$ and $h_{j^*,2}$ is $Q(1-\eps)(p_{j^*,1}+p_{j^*,2}) - Q \cdot p_{j^*,1}$, and $Q(1-\eps)(p_{j^*,1}+p_{j^*,2}) - Q \cdot p_{j^*,2}$, respectively. By our assumption on $p_{j^*,1}$ and $p_{j^*,2}$, both quantities are strictly positive, contradicting the Pareto-optimality of $\q$.    
\end{proof}

Together, the lemmas above allow us to show that any ex-ante envy-free and Pareto-optimal lottery of high social welfare has an {\em almost combinatorial structure}. We remark that this is the crucial property of our reduction that essentially allows us to embed the combinatorial search space of X3C into the continuous space of allocation lotteries. Namely, for each $j \in [t]$, the lottery must give a probability mass of almost $1/t$ to one of the partitions $P_{j,1}$ or $P_{j,2}$, and a probability mass of almost $0$ to the other. This is stated more precisely in the following lemma.

\begin{lemma} \label{lem:binary_switch}
In instance $\In$, any ex-ante envy-free and Pareto-optimal lottery, in which the expected social welfare of the set and element agents is at least $Q+6+r/t$ and the expected social welfare of the base agents is at least $R+R/t+r/t-3$, satisfies $\max\{p_{j,1},p_{j,2}\} \geq \frac{1-3\eps}{t}$, $\min\{p_{j,1},p_{j,2}\} \leq \frac{2\eps}{t}$, and $p_{j,3}\leq \frac{\eps}{t}$ for each $j \in [t]$.
\end{lemma}

\begin{proof}
Let $\q$ be an  ex-ante envy-free and Pareto-optimal lottery in instance $\In$ with the stated social welfare guarantees. Let $j \in [t]$. By Pareto optimality and the required bound on the social welfare of the base agents, Lemmas~\ref{lem:uniform} and  \ref{lem:PO} yield $\min\{p_{j,1},p_{j,2}\} \leq \eps \cdot (p_{j,1}+p_{j,2}) \leq \eps \cdot \frac{1+\eps}{t} \leq \frac{2 \eps}{t}$. Next we prove the bounds on $p_{j,3}$ and $\max \{p_{j,1}, p_{j,2}\}$.

By the bounds on the expected utility of the set and element agents in partitions $P_{j,1}$, $P_{j,2}$, and $P_{j,3}$ from Lemma~\ref{lem:SW_set_element}, the probabilities of allocations in these partitions in the support of lottery $\q$, and 
    the bound on the social welfare of these agents in the statement of the lemma, we have 
   \[
   Q+6+\frac{r}{t} \leq (Q+9) \cdot \left(1- \sum_{j \in [t]}p_{j,3}\right)+\left(\frac{Q}{2}+9\right) \cdot \sum_{j \in [t]} p_{j,3}
   \]
   and, thus,
   \[
   \sum_{j \in [t]} p_{j,3} \leq \frac{2}{Q} \cdot \left(3-\frac{r}{t}\right) \leq \frac{\eps}{t} \enspace .
   \]
   Trivially, this implies the desired bound $p_{j,3} \leq \frac{\eps}{t}$ for $j \in [t]$. The last inequality follows by the definition of $Q$ (recall that $Q=\frac{6t}{\eps}$). Now, by Lemma~\ref{lem:uniform}, we get $p_{j,1}+p_{j,2} \geq \frac{1-2\eps}{t}$ and, by Lemma~\ref{lem:PO}, we have $\max\{p_{j,1},p_{j,2}\} \geq (1-\eps) \cdot \frac{1-2 \eps}{t} \geq \frac{1-3\eps}{t}$ for $j \in [t]$.
\end{proof}

Our arguments in the next two lemmas use a particular type of non-canonical allocations.

\begin{definition}
An allocation in partition $P_{j,c}$ for $j\in [t]$ and $c\in [3]$ is called {\em defective} if there is $i\in [r]$ such that $e_i\in S_j$ and agent $z_i$ is not assigned bundle $Z_i$.
\end{definition}
In the proofs of the next two lemmas, for a given lottery, we will denote by $\gamma_{j,c}$ the probability mass put on defective allocations in partition $P_{j,c}$ for $j\in [t]$ and $c\in [3]$. We denote by $\gamma$ the total probability mass put on defective allocations, i.e., $\gamma=\sum_{j\in [t]}{\left(\gamma_{j,1}+\gamma_{j,2}+\gamma_{j,3}\right)}$.

Our next technical lemma proves an upper bound on the probability mass put by any Pareto-optimal lottery with high enough social welfare on defective allocations.

\begin{lemma} \label{lem:noncanonical}
    In instance $\In$, any ex-ante envy-free and Pareto-optimal lottery with social welfare at least $R+R/t+r/t-3$ for the base agents and at least $Q+6+r/t$ for the set and element agents, must put a probability mass of at most $5 \eps$ on defective allocations.
\end{lemma}
%% we can even have the bound as at most $5 \eps$
\begin{proof}
Let $\q$ be an ex-ante envy-free and Pareto-optimal lottery in instance $\In$ with the stated social welfare guarantees. Let $j\in [t]$ and consider an allocation in partition $P_{j,1}$ in the support of lottery $\q$. We claim that for every $i\in [r]$ such that $e_i\in S_j$, this allocation either (1) assigns bundle $Z_i$ to the element agent $z_i$ and the bundle $V_i$ to the element agent $v_i$ or (2) assigns bundle $V_i$ to the element agent $z_i$. Indeed, for the sake of contradiction, assume that for some $i^*\in [r]$ such that $e_{i^*}\in S_j$, the element agent $z_{i^*}$ is assigned neither bundle $Z_{i^*}$ nor bundle $V_{i^*}$. Then, the lottery $\widetilde\q$, which moves probability mass from this allocation to the allocation in which the agent who gets bundle $Z_{i^*}$ and agent $z_{i^*}$ have their bundles swapped, Pareto-dominates lottery $\q$ (notice that bundle $Z_{i^*}$ gives zero utility to any other agent besides $z_{i^*}$, and agent $z_{i^*}$ gets zero utility from any bundle different than $Z_{i^*}$ and $V_{i^*}$), contradicting its Pareto-optimality. Now, again for the sake of contradiction, assume that agent $z_{i^*}$ is assigned bundle $Z_{i^*}$ but agent $v_{i^*}$ does not get bundle $V_{i^*}$. Then, the lottery $\widetilde\q$, which moves probability mass from this allocation to the allocation in which the agent who gets bundle $V_{i^*}$ and agent $v_{i^*}$ have their bundles swapped, Pareto-dominates lottery $\q$ (notice that bundle $V_{i^*}$ gives zero utility to any other agent besides agents $v_{i^*}$ and $z_{i^*}$, and agent $v_{i^*}$ gets zero utility from any bundle different than $V_{i^*}$), contradicting its Pareto-optimality. 

Thus, a non-defective allocation in partition $P_{j,1}$ in the support of lottery $\q$ gives utility $\sum_{i\in [r]:e_i\in S_j}{(2+1/f_i)}$ to the element agents. In a defective allocation in partition $P_{j,1}$ in the support of lottery $q$, the element agent $z_{i^*}$ has utility at most $(1+1/f_{i^*})/f_{i^*}$ instead of $1/f_{i^*}$ and the element agent $v_{i^*}$ has utility $0$ instead of $2$, for some $i^*\in [r]$ such that $e_{i^*}\in S_j$. Thus, a defective allocation in partition $P_{j,1}$ in the support of lottery $\q$ gives utility at most $\sum_{i\in [r]:e_i\in S_j}{(2+1/f_i)}-1$ to the element agents. 

Following analogous reasoning to the two paragraphs above, we can show that a non-defective allocation in partition $P_{j,2}$ (respectively, $P_{j,3}$) in the support of lottery $\q$ gives utility $\sum_{i\in [r]:e_i\in S_j}{(2+1/f_i)}$ (respectively, $\sum_{i\in [r]:e_i\in S_j}{(4+1/f_i)}$) to the element agents, and a defective allocation in partition $P_{j,2}$ (respectively, $P_{j,3}$) in the support of lottery $\q$ gives utility $\sum_{i\in [r]:e_i\in S_j}{(2+1/f_i)}-1$ (respectively, $\sum_{i\in [r]:e_i\in S_j}{(4+1/f_i)}-1$) to the element agents. 

We are now ready to upper-bound the social welfare of the element agents in lottery $\q$ by
\begin{align}\nonumber
    &\sum_{j \in [t]}{\left((p_{j,1}-\gamma_{j,1})\cdot \sum_{i\in [r]: e_i \in S_j}{\left(2+\frac{1}{f_i}\right)}+(p_{j,2}-\gamma_{j,2})\cdot \sum_{i\in [r]: e_i \in S_j}{\left(2+\frac{1}{f_i}\right)}\right.}\\\nonumber
    &\quad\quad {+(p_{j,3}-\gamma_{j,3})\cdot \sum_{i\in [r]: e_i \in S_j}{\left(4+\frac{1}{f_i}\right)}+\gamma_{j,1}\cdot \left(\sum_{i\in [r]: e_i \in S_j}{\left(2+\frac{1}{f_i}\right)}-1\right)}\\\nonumber
    &\quad\quad {\left.+\gamma_{j,2}\cdot \left(\sum_{i\in [r]: e_i \in S_j}{\left(2+\frac{1}{f_i}\right)}-1\right)+\gamma_{j,3}\cdot \left(\sum_{i\in [r]: e_i \in S_j}{\left(4+\frac{1}{f_i}\right)}-1\right)\right)}\\\nonumber
    & = \sum_{j \in [t]}{(6p_{j,1}+6p_{j,2}+12p_{j,3})}+ \sum_{j\in [t]}{\sum_{i\in [r]: e_i \in S_j}{ (p_{j,1}+p_{j,2}+p_{j,3})\cdot \frac{1}{f_i}}}-3\cdot \sum_{j \in [t]}{(\gamma_{j,1}+ \gamma_{j,2}+\gamma_{j,3})}\\\label{eq:sw-from-element-agents}
    & \leq 6+12\eps + \frac{1+\eps}{t}\cdot \sum_{i \in [r]}{\sum_{j\in [t]: e_i \in S_j}{\frac{1}{f_i}}} -3\gamma  = 6 + 12 \eps + (1+\eps) \cdot \frac{r}{t} - 3\gamma\leq 6+\frac{r}{t}+15\eps-3\gamma.
\end{align}
The first inequality follows using the inequality $p_{j,1}+p_{j,2}+p_{j,3}\leq \frac{1+\eps}{t}$ from Lemma~\ref{lem:uniform} and $p_{j,e}\leq \frac{\eps}{t}$ from Lemma~\ref{lem:binary_switch}. The second equality follows by the definition of $f_i$, and the last inequality follows by our assumption that $r\leq 3t$.

Now, recall (by our construction) that the social welfare of the set agents is no more than $Q$ and, hence, the lottery $\q$ has a social welfare of at least $6+r/t$ from the element agents. Using this observation and the upper bound on the social welfare of the element agents in  (\ref{eq:sw-from-element-agents}), we obtain that $\gamma \leq  5 \eps$, as desired.
\end{proof}


We are now ready to prove the soundness and completeness of our reduction. This is done in Lemmas~\ref{lem:soundness} and \ref{lem:completeness}, respectively, which complete the proof of Theorem~\ref{thm:NPhard}.
\begin{lemma} \label{lem:soundness}
    If instance $\In$ admits an ex-ante envy-free and Pareto-optimal lottery of social welfare at least $R+R/t+Q+6+r/t$, then instance $\phi$ has an exact cover.
\end{lemma}

\begin{proof}
Let $\q$ be an ex-ante envy-free and Pareto-optimal lottery in instance $\In$ with the stated social welfare guarantee. By Lemma~\ref{lem:SW_set_element}, the expected utility set and element agents have is at most $Q+9$. Hence, the social welfare of the base agents is at least $R+R/t+r/t-3$ and the conditions of Lemma~\ref{lem:uniform} are satisfied. Now, observe that for $i\in [r]$, the utility of agent $z_i$ is at most $1/f_i$ in any non-defective allocation in partitions $P_{j,c}$ for $j\in [t]$ such that $e_i\in S_j$ and $c\in [3]$, while it is at most $\max\left\{1/f_i,2/3,(1+1/f_i)/f_i\right\}\leq 1+1/f_i$ in any defective allocation in partition $P_{j,c}$ for $j\in [t]$ such that $e_i\in S_j$ and $c\in [3]$. Clearly, the utility of agent $z_i$ is zero in any allocation in partition $P_{j,c}$ for $j\in [t]$ such that $e_i\not\in S_j$. Thus, the expected utility of agent $z_i$ for $i\in [r]$ is 
   \begin{align}\nonumber
       u_{z_i}(\q;z_i) & \leq  \sum_{j \in [t]:e_i \in S_j}\left((p_{j,1}+p_{j,2}+p_{j,3}-\gamma_{j,1}-\gamma_{j,2}-\gamma_{j,3}) \cdot \frac{1}{f_i} + (\gamma_{j,1}+\gamma_{j,2}+\gamma_{j,3}) \cdot \left(1+\frac{1}{f_i}\right)\right) \\\nonumber
       &= \sum_{j\in [t]:e_i\in S_j}{(p_{j,1}+p_{j,2}+p_{j,3})\cdot \frac{1}{f_i}}+\sum_{j\in [t]:e_i\in S_j}{(\gamma_{j,1}+\gamma_{j,2}+\gamma_{j,3})}\\\label{eq:z_i-exp-utility-ub}
       &\leq \frac{1+\eps}{t}  \sum_{j \in [t]:e_{i} \in S_j}{\frac{1}{f_i}} +  \gamma \leq \frac{1}{t} + \frac{\eps}{t}+5\eps < \frac{1}{t}+\frac{1}{2t^2}.
   \end{align}
The second inequality follows by Lemma~\ref{lem:uniform} which asserts that $p_{j,1}+p_{j,2}+p_{j,3}\leq \frac{1+\eps}{t}$, the third one by the definition of $f_i$ and Lemma~\ref{lem:noncanonical}, and the last one by the definition of $\eps$ (recall that $\eps=\frac{1}{12t^2}$) and since $t\geq 9$.

On the other hand, notice that the bundle $B_0$ gives utility $R/t$ only to the base agent $b_0$, while for $j\in [t]$ and $c\in [3]$, the only bundle among $B_1, B_2, ..., B_t$ that gives non-zero utility (equal to $R$) to some base agent is bundle $B_j$. Thus, the social welfare of the base agents is at most $R+R/t$ and, hence, the social welfare of the set and element agents in lottery $\q$ is at least $Q+6+r/t$. Together with the properties of ex-ante envy-freeness and Pareto-optimality and the lower bound on the social welfare of the base agents claimed above, the conditions of Lemma~\ref{lem:binary_switch} are satisfied, meaning that the lottery $\q$ has the combinatorial structure indicated by it. 

Define $C=\{j\in [t]: p_{j,1} \geq \frac{1-3\eps}{t}\}$. We will show that $C$ forms an exact cover of $\phi$. For the sake of contradiction, assume otherwise that there exists an element $e_{i^*}$ for some $i^* \in [r]$ that is included in either none or in at least two sets $S_j$ such that $j \in C$. We distinguish between two cases:
    
\paragraph{Case 1.} If $e_{i^*}$ is not included in any set $S_j$ such that $j \in C$, then $p_{j,1} < \frac{1-3\eps}{t}$ and, by Lemma~\ref{lem:binary_switch}, $p_{j,2} \geq \frac{1-3\eps}{t}$ for all $j \in [t]$ such that $e_{i^*} \in S_j$. Now, notice that agent $w_{i^*}$ is assigned bundle $W_{i^*}$ (for which agent $z_{i^*}$ has utility $(1+1/f_{i^*})/f_{i^*}$) in every non-defective allocation in partition $P_{j,2}$ for $j\in [t]$ such that $e_{i^*}\in S_j$. Thus, the expected utility agent $z_{i^*}$ has for the bundle assigned to agent $w_{i^*}$ is 
%\pagebreak    
\begin{align}\nonumber
         u_{z_{i^*}}(\q;w_{i^*}) & \geq \sum_{j \in [t]:e_{i^*} \in S_j}{(p_{j,2} - \gamma_{j,2})\cdot \left(1+\frac{1}{f_{i^*}}\right) \cdot \frac{1}{f_{i^*}}} \geq \left(1+\frac{1}{t}\right)  \cdot \sum_{j \in [t]:e_{i^*} \in S_j}{(p_{j,2} - \gamma_{j,2}) \cdot \frac{1}{f_{i^*}}} \\\nonumber  
         &\geq  \left(1+\frac{1}{t}\right)  \cdot \frac{1-3\eps}{t} \cdot\sum_{j \in [t]:e_{i^*} \in S_j}{ \frac{1}{f_{i^*}}} - \left(1+\frac{1}{t}\right) \cdot \sum_{j \in [t]:e_{i^*}\in S_j} \frac{\gamma_{j,2}}{f_{i^*}}\\\label{eq:z_i-exp-utility-lb}
         & \geq \left(1+\frac{1}{t}\right) \cdot \frac{1-3\eps}{t}-\left(1+\frac{1}{t}\right) \cdot \gamma \geq \frac{1}{t}+\frac{1}{t^2} -\frac{3\eps}{t^2}-\frac{8\eps}{t}-5\eps > \frac{1}{t}+\frac{1}{2t^2}.
    \end{align}
The second inequality follows since $f_{i^*}\leq t$ by definition, the third one since $p_{j,2}\geq \frac{1-3\eps}
{t}$, the fourth one by the definitions of $\gamma$ and $f_{i^*}$, the fifth one by Lemma~\ref{lem:noncanonical}, and the last one by the definition of $\eps$ (recall that $\eps=\frac{1}{12t^2}$ and since $t\geq 9$. By inequalities (\ref{eq:z_i-exp-utility-ub}) and (\ref{eq:z_i-exp-utility-lb}), we obtain that $u_{z_{i^*}}(\q;z_{i^*})<u_{z_{i^*}}(\q;w_{i^*})$, meaning that agent $z_{i^*}$ is envious of agent $w_{i^*}$, a contradiction.

\paragraph{Case 2.} Let $D=\{j\in C: e_{i^*}\in S_j\}$ and assume that $|D|\geq 2$. Since $D\subseteq C$, we have $p_{j,1}\geq \frac{1-3\eps}{t}$ for every $j \in D$. Notice that agent $v_{i^*}$ is assigned bundle $V_{i^*}$ (for which agent $z_{i^*}$ has utility $2/3$) in every non-defective allocation in partition $P_{j,1}$ for $j\in [t]$ such that $e_{i^*}\in S_j$. Thus, the expected utility agent $z_{i^*}$ has for the bundle assigned to agent $v_{i^*}$ is
   \begin{align}\label{eq:z_i-exp-utility-lb-2}
         u_{z_{i^*}}(\q;v_{i^*}) \geq  \sum_{j\in D}  \frac{2}{3}  \cdot (p_{j,1} - \gamma_{j,1}) \geq \frac{4}{3} \cdot \frac{1-3\eps}{t} - \frac{2}{3} \cdot  \gamma 
         \geq \frac{4}{3t}-\frac{4\eps}{t}-\frac{10}{3}\eps> \frac{1}{t} +\frac{1}{2t^2}.
    \end{align}
The third inequality follows by Lemma~\ref{lem:noncanonical} and the last one by the definition of $\eps$ (recall that $\eps=\frac{1}{12t^2}$) and since $t\geq 9$. By inequalities (\ref{eq:z_i-exp-utility-ub}) and (\ref{eq:z_i-exp-utility-lb-2}), we obtain that $u_{z_{i^*}}(\q;z_{i^*})<u_{z_{i^*}}(\q;w_{i^*})$, again meaning that agent $z_{i^*}$ is envious of agent $w_{i^*}$, a contradiction.
\end{proof}

\begin{lemma} \label{lem:completeness}
    If instance $\phi$ has an exact cover, then instance $\In$ admits an ex-ante envy-free and Pareto-optimal lottery of social welfare at least $R+R/t+Q+6+r/t$.
\end{lemma}
\begin{proof}
Let $C$ be an exact cover of instance $\phi$. Construct the lottery for instance $\In$ which has $p_{j,1}=1/t$ and $p_{j,2}=0$ for $j \in C$, $p_{j,1}=0$ and $p_{j,2}=1/t$ for $j \notin C$, and $p_{j,3}=0$ for $j \in [t]$, and uses the canonical allocations only. 
    
We first justify the claimed social welfare of the lottery. Clearly, the base agent $b_0$ has expected utility $R/t$. For $j\in [t]$, the base agent $b_j$ gets utility $R$ from partitions $P_{j,1}$ and $P_{j,2}$ only. Each of them has probability $1/t$. So, the overall expected utility of the base agents $b_1, ..., b_j$ is $R$. For $j\in C$, agent $h_{j,1}$ is the only set agent who gets utility $Q$ from partition $P_{j,1}$. For $j\not\in C$, agent $h_{j,2}$ is the only set agent who gets utility $Q$ from partition $P_{j,2}$. So, the overall expected utility of set agents is $Q$. For $i\in [r]$, the element agent $z_i$ gets utility $1/f_i$ with probability $1/t$ from either partition $P_{j,1}$ or partition $P_{j,2}$ for every $j\in [t]$ such that $e_i\in S_j$. Thus, agent $z_i$ has expected utility $1/t$; so, the $r$ element agents $z_i$ for $i\in [r]$ have overall expected utility $r/t$. For $j\in C$, partition $P_{j,1}$ gives a utility of $6$ to the three element agents $v_i$ corresponding to each element $e_i\in S_j$. Similarly, for $j\not\in C$, partition $P_{j,2}$ gives a utility of $6$ to the three agents $w_i$ corresponding to each element $e_i\in S_j$. So, the total expected utility of element agents $v_i$ and $w_i$ is $6$.

Let us now examine ex-ante envy-freeness. Notice that in any allocation in the support of the lottery, all base agents besides agent $b_0$, all set agents, and all element agents besides agent $z_i$ for $i\in [r]$ are assigned the bundle that gives them maximum utility. So, to justify ex-ante envy-freeness, we just need to examine whether agent $b_0$ envies agent $b_j$ for $j\in [t]$ and whether agent $z_i$ envies agents $v_i$ and $w_i$ for $i\in [r]$. Agent $b_0$ gets utility $R/t$ from bundle $B_0$ in any allocation of the lottery. Her utility for the bundle $B_j$ assigned to agent $b_j$ in partitions $P_{j,1}$ and $P_{j,2}$ is $R$ and is thus non-envious as one of them is part of the lottery with probability $1/t$. For $i\in [r]$, agent $z_i$ gets utility 
$1/f_i$ in each allocation of the lottery corresponding to a partition $P_{j,1}$ or $P_{j,2}$ with $e_i\in S_j$. Notice that there are exactly $f_i$ such partitions appearing in the lottery with probability $1/t$, for an expected utility of $1/t$ for agent $z_i$. Now, notice that agent $v_i$ gets bundle $V_i$ of partition $P_{j,1}$ for the single $j\in C$ such that $e_i\in S_j$. This happens with probability $1/t$, and agent $z_i$ has a value of only $2/3$ for this bundle. For $i\in [r]$, agent $w_i$ gets bundle $W_i$ of partition $P_{j,2}$ for $f_i-1$ different values of $j\not\in C$ so that $e_i\in S_j$, i.e., with probability $\frac{f_i-1}{t}$. Agent $z_i$ has total utility $\frac{f_i-1}{t}\cdot (1+1/f_i)/f_i < 1/t$ for the bundles assigned to agent $w_i$. Thus, indeed, for each $i\in [r]$, agent $z_i$ does not envy agents $v_i$ and $w_i$.

To prove Pareto optimality, notice that the base agents get their maximum expected utility of $R+R/t$, the set agents get their maximum expected utility of $Q$, and the element agents $v_i$ and $w_i$ for $i\in [r]$ get their maximum expected utility of $6$. Then, any allocation in which the utility of some agent $z_i$ increases should harm some other element agent. This concludes the proof of the lemma.
\end{proof}

