\section{An efficient algorithm for constant number of agents}
In this section, we develop a very simple polynomial time algorithm for computing an ex-ante envy-free and Pareto-optimal lottery when the number of agents is constant. Consider a fair division instance $\mathcal{I}$ consisting of $n$ agents, a set of $m$ partitions $\mathcal{P}=\{P^1,P^2,..., P^m\}$, and agent utilities $u_{ij}^k$ for $i,j\in [n]$ and $k\in [m]$. The algorithm begins with evaluating the agents' valuations in the $n!$ possible allocations for each partition $P^k$ for $k\in [m]$. That is, we obtain $n!$ utility profiles in $\mathbb{R}^n$ for each partition and $m\cdot n!$ utility profiles overall. The Pareto-optimal lotteries are formed by faces of the convex hull of these utility profiles. 

Since the dimension $n$ is constant, the convex hull can be computed in polynomial time~\cite{Chazelle1993OptimalConvexHull}. We may then enumerate over the faces forming the Pareto-frontier. For each of these faces, we compute a hyperplane $H$ that contains the face. For such a hyperplane $H = \{x \in \mathbb{R}^n \mid w_1 x_1 + w_2 x_2 + \dots + w_n x_n = w_0\}$, we can determine whether it contains an envy-free lottery by linear programming.
\begin{equation*}
  \begin{array}{ll@{}ll}
    \text{Find}  & (\q,\p) &\\
    \text{subject to}& 
    \displaystyle   u_i(\q;i)\geq u_i(\q;i')\quad\quad &\text{for all} \ i,i' \in [n]\\
    \displaystyle & \sum\limits_{i=1}^n w_i u_i(\q;i) = w_0\\
    \displaystyle & \sum\limits_{i=1}^n q^k_{ij} =p_k &\text{for all} \ j \in [n], k \in [m]  \\
    \displaystyle  & \sum\limits_{j=1}^n q^k_{ij} =p_k & \text{for all} \ i \in [n], k \in [m] \\
    \displaystyle    & \sum\limits_{k=1}^m{p_k}=1 & \\
    \displaystyle     & p_k\geq 0 & \text{for all}\ k \in [m] & \\
    \displaystyle     & q_{ij}^k\geq 0 & \text{for all} \ i,j \in [n], k \in [m] \\
  \end{array}
\end{equation*}

Since we know that there does exist an ex-ante envy-free and Pareto-optimal lottery, at least one of these linear programs must be feasible. The next statement summarizes the discussion above.
%Note that, for a constant number of agents, the total number of faces\footnote{We have, the number of faces %$\binom{m\cdot n!}{n} \leq (m \cdot n!)^n \leq m^{O(n^2 \log n)}$.} is polynomial in $m$, and hence our algorithm %runs in polynomial time.



\begin{theorem}
    For fair division instances with a constant number of agents, an ex-ante envy-free and Pareto-optimal allocation can be computed in polynomial time. 
\end{theorem}