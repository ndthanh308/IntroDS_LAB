\section{The model}

Consider a set $[n]=\{1,2, \dots, n\}$ of $n$ agents and a collection $\mathcal{P}=\{P^1,P^2,\dots, P^m\}$ of admissible partitions of a set $M$ of items. Every partition $P^k$ for $k \in [m]$ consists of $n$ bundles, i.e.,  $\abs{P^k}=n$  and the union of those bundles is $\bigcup_{A \in P^k}A \subseteq M$. Agents are endowed with utility functions $u_i$'s that specify their cardinal preferences for all bundles in every different partition. In particular, the function $u^k_{ij}$ specifies \emph{partition-based cardinal utilities} of agent $i \in [n]$ for the $j$th  bundle (for $j \in [n]$) in partition $P^k \in \mathcal{P}$. It is important to note that an agent with partition-based utilities can have different utilities for the exact same bundle, occurring in two distinct partitions.
We will denote a fair division instance by the tuple $\I=\langle[n], \mathcal{P}, \{u^k_{ij}\}_{i,j \in [n],k \in [m] }\rangle$.

For a given fair division instance, we define an \emph{allocation} to be an assignment of the $n$ bundles of a partition in $\mathcal{P}$ to the agents, such that every agent receives exactly one bundle. We assume that for any partition, the set of (admissible) allocations is specified by the $n!$ permutations of $n$ bundles among $n$ agents, and that the utility of an agent depends only on the partition and the bundle received and not to whom the remaining bundles are given. We refer to this property of a  fair division instance as the \emph{anonymity property}.
%, i.e., any bundle of a partition can be assigned to any agent.
Therefore, we have a total of  $m \cdot n!$ many distinct admissible \emph{allocations} in a given fair division instance. Furthermore, in a given fair division instance, we define a \emph{lottery} to be a probability distribution over these allocations.


The overarching goal is to find a \emph{fair} and \emph{efficient} lottery among agents from the given set of admissible partitions. As mentioned, Cole and Tao \cite{cole2021existence} established the existence of fair and efficient lotteries for fair division instances with the anonymity property using Kakutani's fixed-point theorem~\cite{Kakutani1941}.
Since there are a total of $m \cdot n!$ many allocations, one can specify probabilities with which every allocation occurs in a lottery. This leads to a very convenient but also very inefficient way of representing a lottery via an exponential-dimensional  vector $(p_1, p_2, \dots, p_{m \cdot n!})$, where $p_i$ represents the probability with which the $i$-th allocation is chosen. This representation was used by Cole and Tao~\cite{cole2021existence} for their proof of existence, but it is clearly not suitable for studying the computational aspects of finding lotteries.
Instead, we will represent a lottery in the following  manner: Let $\p=\{p_k\}_{k \in [m]}$, where $p_k \in [0,1]$ denotes the probability with which  partition $P^k \in \mathcal{P}$ is selected in a lottery. The vector $\q=\{q^k_{ij}\}_{i,j \in [n], k \in [m]}$ of length $m \cdot n^2$ then specifies the full lottery, where $q^k_{ij}$ is the probability with which the lottery $\q$ assigns the $j$th bundle in partition $P^k$ to agent $i \in [n]$. The vectors $\p$ and $\q$ characterized by the following constraints.
\begin{align*}
    \sum_{i=1}^n q^k_{ij} &=p_k \ \text{for all} \ j \in [n], k \in [m] \tag{Every bundle is assigned to one agent}\\
     \sum_{j=1}^n q^k_{ij} &=p_k \ \text{for all} \ i \in [n], k \in [m] \tag{Every agent receives one bundle}\\
     \text{and,} \ \sum_{k=1}^m p_k &=1 
\end{align*}
%Here, for $k \in [m]$, the term $p_k$ represents the probability mass on partition $P^k \in \mathcal{P}$, while for $i, j \in [n]$, the term $q^k_{ij}$ is the probability with which the lottery $\q$ assigns the $j$th bundle in partition $P^k$ to agent $i \in [n]$.
We can now express the expected utility, $\e [u_i(\q)]$, for agent $i \in [n]$ in a lottery $\q$ as % =\{q^k_{ij}\}_{i,j \in [n], k \in [m]}$ as 
\[ \e [u_i(\q)] \coloneqq \sum_{k=1}^m \sum_{j=1}^n u^k_{ij} \cdot q^k_{ij} \enspace .\]
More generally, let \[u_i(\q;i') \coloneqq \sum_{k=1}^m \sum_{j=1}^n u^k_{ij} \cdot q^k_{i' j} \] denote the expected utility of agent $i$ for the bundle of agent $i'$  in the lottery $\q$. Observe that, we have $u_i(\q;i)=\e [u_i(\q)]$ for any agent $i \in [n]$.

Let us now define the standard notions of fairness and optimality in resource allocation settings. A lottery $\q$ is said to be \emph{ex-ante envy-free} if $u_i(\q;i) \geq u_i(\q;i')$ holds for all $i,i' \in [n]$. Furthermore, we say that $\q$ is  \emph{ex-ante Pareto-optimal} if there does not exist any other lottery $\widetilde{\q}$ such that $u_i(\widetilde{\q};i) \geq u_i(\q;i)$ holds for all $i \in [n]$, with a strict inequality for at least one agent $i \in [n]$. \emph{Social welfare} is a standard notion of measuring the collective welfare of an allocation. We define \emph{social welfare} of a lottery $\q$ as the sum of the expected utilities of all agents, i.e., $\mathrm{SW}(\q) = \sum_{i \in [n]} u_i(\q;i)$.

