\section{Conclusion}

In this work, we considered the general setting of the problem of dividing indivisible items in a fair and efficient manner to agents having partition-based utilities. We have shown membership of the total problem of finding ex-ante envy-free and Pareto-optimal allocation lotteries in the class \PPAD. We consider settling the precise computational complexity of the problem an important question. From an algorithmic perspective it would also be very interesting to see if Lemke's algorithm~\cite{Lemke1964-Bimatrix} could be adapted to solve the problem, as this would likely lead to a practical algorithm.

%In case the problem has a polynomial-time algorithm, developing an algorithm that approximates the social welfare over the space of ex-ante envy-free and Pareto-optimal allocation lotteries becomes a relevant line of direction. 


%that the problem of finding an ex-ante envy-free and Pareto-optimal allocation lottery in a fair division instance with partition-based utilities belongs to the complexity class \PPAD. As a corollary, we obtain a useful implication that proves the existence of an ex-ante envy-free and Pareto-optimal rational-valued lottery for instances with only rational-valued utilities. For a constant number of agents, we develop a polynomial-time algorithm to find an ex-ante envy-free and Pareto-optimal lottery. An important direction for future research is to decide the computational complexity of finding an ex-ante envy-free and Pareto-optimal lottery for instances with a single partition.Finally, we prove that maximizing social welfare over ex-ante envy-free and Pareto optimal allocatios lotteries is \NP-hard.

