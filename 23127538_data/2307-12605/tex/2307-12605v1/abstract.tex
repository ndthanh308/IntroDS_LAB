\begin{abstract}
We study the classic problem of dividing a collection of indivisible resources in a \emph{fair} and \emph{efficient} manner among a set of agents having varied preferences. \emph{Pareto optimality} is a standard notion of economic efficiency, which states that it should be impossible to find an allocation that improves some agent’s utility without reducing any other’s. On the other hand, a fundamental notion of fairness in resource allocation settings is that of \emph{envy-freeness}, which renders an allocation to be fair if every agent (weakly) prefers her own bundle over that of any other agent's bundle. Unfortunately, an envy-free allocation may not exist if we wish to divide a collection of indivisible items. Introducing randomness is a typical way of circumventing the non-existence of solutions, and therefore, {\em allocation lotteries}, i.e., distributions over allocations have been explored while relaxing the notion of fairness to \emph{ex-ante} envy freeness.

We consider a general fair division setting with $n$ agents and a family of admissible $n$-partitions of an underlying set of items. Every agent is endowed with \emph{partition-based utilities}, which specify her cardinal utility for each bundle of items in every admissible partition. In such fair division instances, Cole and Tao~(2021) have proved that an ex-ante envy-free and Pareto-optimal allocation lottery is always guaranteed to exist. We strengthen their result while examining the computational complexity of the above total problem and establish its membership in the complexity class \PPAD. Furthermore, for instances with a constant number of agents, we develop a polynomial-time algorithm to find an ex-ante envy-free and Pareto-optimal allocation lottery. On the negative side, we prove that maximizing social welfare over ex-ante envy-free and Pareto-optimal allocation lotteries is \NP-hard.
\end{abstract}