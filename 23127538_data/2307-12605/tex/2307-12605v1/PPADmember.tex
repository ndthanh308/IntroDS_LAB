 \section{\PPAD-membership}

In this section, we show that the problem of finding an exact ex-ante envy-free and Pareto-optimal lottery in a given fair division instance belongs to the class \PPAD.
Our proof is based on (i) a significant simplification of the existence proof of Cole and Tao \cite{cole2021existence}, (ii) a characterization of \PPAD\ in terms of computing fixed points of piecewise linear arithmetic circuits due to Etessami and Yannakakis~\cite{EtessamiY2010-FIXP} (i.e.\ $\PPAD=\LinearFIXP$), and (iii) a framework for proving \FIXP\ and \PPAD-membership via convex optimization recently developed by Filos-Ratsikas~et~al.~\cite{Filos-RatsikasH2021-FIXP,Filos-RatsikasH2023-PPAD}. Formally, we obtain the following theorem.

\begin{theorem}
  The problem of finding an ex-ante envy-free and Pareto-optimal
  lottery in a fair division instance belongs to \PPAD.
  \label{THM:PPAD}
\end{theorem}

It is possible to adapt the existence proof of Cole and Tao (by changing to our succinct representation of lotteries) to obtain a proof of $\FIXP$ membership using the framework of Filos-Ratsikas~et~al~\cite{Filos-RatsikasH2021-FIXP}. The proof of Cole and Tao employs Kakutani's fixed point theorem to a correspondence defined on pairs consisting a lottery $\q$ and a vector of positive weights $\w \in W_\eps$ for the agents, from a closed set $W_\eps$. This correspondence maps $(\q,\w)$ to pairs $(\q',\w')$ such that $\q'$ is a lottery maximizing the weighted sum of utilities of the agents and where $\w'$ is obtained from $\w$ by translating each coordinate by a nonlinear function of the lottery $\q$ followed by a projection to the set $W_\eps$.

The maximization of the weighted sum of utilities may be phrased as a linear program and the projection may be phrased as a convex quadratic program. While both of these fall in the scope of the framework of Filos-Ratsikas~et~al.~\cite{Filos-RatsikasH2023-PPAD} for proving \PPAD-membership, the nonlinear transformation involved cannot be computed be a piecewise linear artihmetic circuit.

Our simplified proof involves only optimization of a linear program and the solution of a feasibility program with conditional linear constraints, together with operations computable by linear arithmetic circuits. In this case the framework Filos-Ratsikas~et~al.\ applies to give \PPAD-membership~\cite{Filos-RatsikasH2023-PPAD}.

Another framework for proving \PPAD-membership was also recently introduced by Papadimitriou, Vlatakis-Gkaragkounis and Zampetakis~\cite{PapadimitriouVZ2023-Kakutani}. With this framework, however, it would only be possible to directly prove \PPAD-membership for an \emph{approximate} version of the problem, rather than the exact problem.

%An interesting and immediate implication of \PPAD-membership is the existence of a
%rational-valued ex-ante envy-free and Pareto-optimal lottery.

%\begin{corollary}
%  Any fair division instance with rational-valued utilities has an ex-ante envy-free and Pareto-optimal %rational-valued lottery.
%\end{corollary}

In the remainder of this section, we let $\I=\langle[n], \mathcal{P}, \{u^k_{ij}\}_{i,j,k}\rangle$ denote a fair division instance with $n$~agents and $m=\abs{\mathcal{P}}$ partitions, where the utilities $u_{ij}^k$ are given as rational numbers.
%that, without loss of generality, satisfy $0\leq u_{ij}^k \leq 1$, for all $i,j,k$.

\subsection{Fixed point formulation}
We first present our fixed point formulation for ex-ante envy-free and Pareto-optimal lotteries; afterwards we consider the implications for the computational complexity of the problem.

A standard technique for expressing the Pareto frontier of an
optimization problem, also employed by Cole and Tao, is the \emph{weighted sum
method}~\cite{Zadeh1963-Pareto}. Let $w_1,\dots,w_n>0$ be strictly positive weights. Then, any lottery $\q$ maximizing the weighted sum of utilities $\sum_{i=1}^n w_i u_i(\q;i)$ must be Pareto-optimal. Conversely, if $\q$ is a Pareto-optimal lottery, there are strictly positive weights such that $\q$ maximizes the weighted sum of utilities.

The task of maximizing the weighted sum of utilities can be expressed by the following linear program with decision variables $q_{ij}^k$ and $p_k$, and parameterized by the variables $w_i$.

\begin{equation}
  \begin{array}{ll@{}ll}
    \text{maximize}   & \sum\limits_{i=1}^n w_i \sum\limits_{k=1}^m \sum\limits_{j=1}^n u_{ij}^k q_{ij}^k&\\
    \text{subject to} & \sum\limits_{i=1}^n q^k_{ij} =p_k & \text{for all} \ j \in [n], k \in [m]  \\
    \displaystyle     & \sum\limits_{j=1}^n q^k_{ij} =p_k & \text{for all} \ i \in [n], k \in [m] \\
    \displaystyle     & \sum\limits_{k=1}^m p_k =1 &  \\   
    \displaystyle     & p_k\geq 0 & \text{for all}\ k \in [m] & \\
    \displaystyle     & q_{ij}^k\geq 0 & \text{for all} \ i,j \in [n], k \in [m] \\
  \end{array}
  \label{EQ:MaxWeightedSumLP}
\end{equation}

The proof of Cole and Tao~\cite{cole2021existence} shows the existence of positive weights such that any lottery $\q$ maximizing the corresponding weighted sum of utilities is also ex-ante envy-free.

We next define the following key quantity $0<\rho\leq \frac{1}{2}$ and state Lemma~\ref{lem:cole-tao} (proved in \cite{cole2021existence}) that will be used to place
restrictions of weights. For completeness we give the proof of the lemma in Appendix~\ref{app:413}, adapting the 
proof of \cite[Claim~4.13]{cole2021existence} to suit our representation of
lotteries.

\begin{definition}
  Let
  $J=\{(k,l,h,a,b) \in [m] \times [n]^4 \mid u_{la}^k < u_{lb}^k \text{ and } u_{ha}^k < u_{hb}^k\}$. We define $\rho$ as follows,
 \[   
\rho = 
     \begin{cases}
       \frac{1}{2} \min_{(k,l,h,a,b)\in J}  (u_{lb}^k-u_{la}^k)/(u_{hb}^k-u_{ha}^k) & \text{ if } J \neq \emptyset\\
         \frac{1}{2} & \text{ otherwise }
     \end{cases}
\]
\label{DEF:rho}
\end{definition}
\begin{lemma}[cf.\ {\cite[Claim~4.13]{cole2021existence}}] \label{lem:cole-tao}
  Suppose that $(\q,\p)$ is an optimal solution of
  LP~(\ref{EQ:MaxWeightedSumLP}). If $0<w_h \leq \rho w_l$ it follows
  that $u_l(\q;l) \geq u_l(\q;h)$ (i.e.\ that agent~$l$ does not envy
  agent~$h$).
\label{LEM:WeightInequalityImpliesEnvyFree}
\end{lemma}

Define $\eps=\rho^n/n$ and let
$W_\eps = \{w \in \RR^n \colon \sum_{i=1}^n w_i=1 \text{ and } w_i
\geq \eps\ \forall \ i \in [n]\}$.
We shall restrict the weights to belong to $W_\eps$, which in
particular, ensure that they are strictly positive. We consider the following feasibility problem with conditional linear constraints having decision variables
$w_i$, and parameterized by the variables $q_{ij}^k$. 
\begin{equation}
  \begin{array}{ll}
    \left[u_l(\q;h) - u_l(\q;l)>0\right]  \Rightarrow \left[w_h - \rho w_l \leq 0\right]  & \text{ for all } l,h \in [n]\\
    \sum\limits_{i=1}^n w_i = 1\\
    w_i \geq \eps & \text{ for all } i \in [n]
  \end{array}
  \label{EQ:WeightConditionalLP}
\end{equation}
Here, the conditional constraint $\left[u_l(\q;h) - u_l(\q;l)>0\right]  \Rightarrow \left[w_h - \rho w_l \leq 0\right]$ is satisfied if either $u_l(\q;h) - u_l(\q;l) \leq 0$ or $w_h - \rho w_l \leq 0$. In words, whenever agent~$l$ envies agent $h$ in the lottery $q$, a $\w$ solution of the system must satisfy $0< w_h \leq \rho w_l$, which is precisely the antecedent stated in Lemma~\ref{LEM:WeightInequalityImpliesEnvyFree}. We can think of the feasibility problem as a system of inequalities in variables $\w$, some of which may be ``disabled'' by inequalities expressed in the variables  $\q$.

In order to characterize the solvability of this feasibility program, it is convenient
to introduce the \emph{envy graph} of the lottery $\q$. Filos-Ratsikas~et~al.~\cite{Filos-RatsikasH2023-PPAD} consider feasibility programs as above in a general form and characterizes their solvability in terms of a \emph{feasibility graph}. In our case, this feasibility graph is exactly the same as the \emph{envy graph} defined next.
\begin{definition}[Envy graph]
  For a given lottery $\q$, denote by $\G(\q)$ the \emph{envy graph} with nodes
  $[n]$ and an arc $(l,h)$ whenever $u_l(\q;l) < u_l(\q;h)$, for all
  $l,h \in [n]$. We let $A(\G(\q))$ denote the set of arcs of $\G(\q)$.
\end{definition}
We can then precisely characterize the solvability of the feasibility problem~(\ref{EQ:WeightConditionalLP}) by the graph structure of~$\G(\q)$.
\begin{lemma}
  Suppose that $\q$ is a lottery such that $\G(\q)$ is acyclic. Then the feasibility program~(\ref{EQ:WeightConditionalLP}) is solvable.
\label{LEM:AcyclicImpliesSolvable}
\end{lemma}
\begin{proof}
  First note that the condition $u_l(\q;h) - u_l(\q;l)>0$ is satisfied
  precisely when $(l,h) \in A(\G(\q))$. Thus we are to find weights
  $w_i$ such that $w_h \leq \rho w_l$, whenever $(l,h) \in A(\G(\q))$.

  For $i \in [n]$, let $d_i$ denote the length of a longest path in
  $\G(\q)$ from node $i$ to a sink node, and define
  \[
    w_i = \frac{\rho^{d_i}}{\sum_{j=1}^n \rho^{d_j}} \text{ for all } i \in [n] \enspace .
  \]
  Clearly $\sum_{i=1}^n w_i = 1$, and since $d_i \leq n$ and
  $\rho \leq 1$ we also have $w_i \geq \eps$.  Suppose now that
  $(l,h) \in A(\G(\q))$. This means that $d_l \geq d_h+1$ and thus
  also $w_l \leq \rho w_h$. In conclusion, we have that the weights
   $w_i$ are a solution to the feasibility program~(\ref{EQ:WeightConditionalLP}).
\end{proof}
We can note that acyclicity of $\G(\q)$ is also necessary for the solvability of the feasibility program~(\ref{EQ:WeightConditionalLP}), since the
inequalities $w_l \leq \rho w_h$ given by the arcs $(l,h)$ of a cycle
in $\G(\q)$ are contradictory. But note also that if $\G(\q)$ contains a cycle, all agents in the cycle
will increase their utility if the lottery is shifted along the
cycle. We thus have the following simple but crucial observation.
\begin{observation}[cf.\ {\cite[Claim~4.8]{cole2021existence}}]
  If $\q$ is Pareto-optimal, the envy graph $\G(\q)$ is acyclic.
  \label{OBS:ParetoImpliesAcyclic}
\end{observation}
We can now conclude with the following fixed-point formulation, showing that a pair $(\q,\w)$ that is  simultaneously solving the linear program~(\ref{EQ:MaxWeightedSumLP}) and the feasibility problem~(\ref{EQ:WeightConditionalLP}) give an ex-ante envy-free and Pareto-optimal lottery.
\begin{proposition}
  Suppose that $\q$ is a lottery and $\w \in W_\eps$ are weights such
  that $\q$ is an optimal solution of the linear program
  LP~(\ref{EQ:MaxWeightedSumLP}) with respect to the weights $\w$, and
  $\w$ is a solution of the feasibility program of conditional linear
  constraints~(\ref{EQ:WeightConditionalLP}) with conditions given by
  $\q$ (note that the system is in fact solvable by the optimality of $\q$, Observation~\ref{OBS:ParetoImpliesAcyclic} and
  Lemma~\ref{LEM:AcyclicImpliesSolvable}). Then $\q$ is an ex-ante
envy-free and Pareto-optimal lottery.
\label{PROP:FixedPointFormulation}
\end{proposition}
\begin{proof}
  Since the weights $\w$ are strictly positive and $\q$ is an optimal
  solution of LP~(\ref{EQ:MaxWeightedSumLP}) it follows that $\q$ is
  Pareto-optimal. Suppose now for contradiction that there exists
  agents $l$ and $h$ such that agent $l$ envies agent $h$, that is,
  $u_l(\q;h) > u_l(\q;l)$. Since $\w$ is a solution to the
  system~(\ref{EQ:WeightConditionalLP}) with conditions given by $\q$ given it
  follows that $w_h \leq \rho w_l$. But then
  Lemma~\ref{LEM:WeightInequalityImpliesEnvyFree} gives
  $u_l(\q;h) \leq u_l(\q;l)$, contradicting the assumption. It thus follows that $\q$ must also be
  ex-ante envy-free.
\end{proof}

\subsection{\PPAD, \FIXP, and \LinearFIXP}
The complexity class \PPAD\ was originally defined in seminal work of Papadimitriou~\cite{Papadimitriou1994-TFNP} as the class of total \NP\ search problems reducible to a concrete problem called \textsc{End-Of-Line}. As mentioned above, to obtain result, we shall instead make use of a characterization of \PPAD\ in terms of computation of fixed points of functions computed by piecewise linear arithmetic circuits. Below we briefly introduce this characterization and refer to~\cite{EtessamiY2010-FIXP} for further details.

An arithmetic circuit is a circuit $C$ with gates computing binary operations belonging to the set $\{+,-,\ast,\div,\max,\min\}$ together with rational constants. The size of $C$ refers to the size of an encoding of $C$. A \emph{piecewise linear} arithmetic circuit $C$ restricts the allowable binary operations to the set $\{+,-,\max,\min\}$, but allows also for multiplication by rational constants.

The class \FIXP\ consists of (real-valued) search problems that reduce to finding a fixed point of a function $F \colon D \to D$, where $D$ is an explicitly given convex polytope and $F$ is a function computable by an algebraic circuit. By Brouwer's fixed point theorem such a fixed point is guaranteed to exist, thus making the search problem a total search problem. \LinearFIXP\ is the subclass obtained by restricting the arithmetic circuits to be piecewise linear. 

As defined above, the classes \FIXP\ and \LinearFIXP\ consist of real-valued search problems, which means that reductions must specify a real-valued function mapping fixed points of the function $F$ to solutions of the search problem. In the case when $F$ is computed by a piecewise linear arithmetic circuit $C$, there exists rational-valued fixed points of polynomial bitsize in the size of $C$~\cite[Theorem~5.2]{EtessamiY2010-FIXP}, which allows the use of ordinary polynomial-time reductions. With this convention, Etessami and Yannakakis~\cite{EtessamiY2010-FIXP} showed that $\PPAD = \LinearFIXP~\cite[Theorem~5.4]{EtessamiY2010-FIXP}$.

\subsection{\PPAD-membership via convex optimization}
From the characterization $\PPAD = \LinearFIXP$, in order to prove Theorem~\ref{THM:PPAD}, it is sufficient to reduce the task of computing an ex-ante envy-free and Pareto-optimal lottery to that of computing a fixed point of a piecewise linear arithmetic circuit defined on an explicitly given convex polytope. 

Constructing such a suitable circuit from scratch can potentially be a very challenging task, as many existing proofs of \PPAD-membership in the literature give evidence of. Recently however, Filos-Ratsikas~et~al.~\cite{Filos-RatsikasH2021-FIXP,Filos-RatsikasH2023-PPAD} introduced a general technique for proving \FIXP\ and \PPAD-membership, by which the arithmetic circuit defining the fixed point search problem can be augmented with \emph{pseudo-gates} that solve very general convex optimization problems. By a pseudo-gate is meant a (multi-input and multi-output) gate that is only required to compute the correct output at a fixed point of the full circuit. More precisely, the pseudo-gate is implemented by an arithmetic circuit using auxiliary variables, and when these auxiliary variables are in a fixed point, the pseudo-gate computes the correct output.
\begin{definition}[Pseudo-circuit]
    A pseudo-circuit with $n$ inputs and $m$ outputs is an arithmetic circuit $C$ computing a function $F \colon \RR^n \times [0,1]^\ell \to \RR^m \times [0,1]^\ell$. The output of $C$ on input $x \in \RR^n$ is any $y \in \RR^m$ such that there exists $z \in [0,1]^\ell$ such that $F(x,z)=(y,z)$. The variables $z \in [0,1]^\ell$ are called auxiliary variables.
\end{definition}
By a pseudo-gate is simply meant the use of a pseudo-circuit as a sub-circuit of larger pseudo-circuit, and where the auxiliary variables of the pseudo-gate is augmented to the auxiliary variables of the larger pseudo-circuit. The simple but crucial observation about pseudo-circuits is that, for the purpose of proving \FIXP\ and \PPAD-membership they are just as good as normal arithmetic circuits.

In the setting of proving \PPAD-membership, Filos-Ratsikas~et~al.~\cite{Filos-RatsikasH2023-PPAD} developed a pseudo-gate, coined the \emph{linear-OPT-gate}, implemented as a piecewise linear arithmetic circuit, that in particular can be used to solve both the linear program~(\ref{EQ:MaxWeightedSumLP}) and the feasibility problem~(\ref{EQ:WeightConditionalLP}). For the linear program~(\ref{EQ:MaxWeightedSumLP}) this is possible since the coefficients of all linear constraints are constants and that the coefficients of the objective function are linear functions of the parameter variables $\w$. For the feasibility program
 (\ref{EQ:WeightConditionalLP}) this is possible since the coefficients of all linear
constraints are constants and the that the antecedents of the conditional linear
constraints are given by a strict linear inequalities for functions computable by piecewise linear circuits applied to the parameter variables $\q$. We provide precise statements of the capabilities of the linear-OPT-gate in Appendix~\ref{app:linearOPT}.

\subsection{Proof of Theorem~\ref{THM:PPAD}}
We finally show how our fixed point formulation for ex-ante envy-free and Pareto-optimal lotteries in conjunction with the framework of Filos-Ratsikas~et~al.~\cite{Filos-RatsikasH2023-PPAD} allows for a simple proof of \PPAD\ membership for the problem of computing such lotteries.

The fixed point formulation of Proposition~\ref{PROP:FixedPointFormulation} amounts to finding
$(\q,\p,\w)$ such that $(\q,\p)$ is an optimal solution of the linear program~(\ref{EQ:MaxWeightedSumLP}), parametrized by $\w$, and such that $w$ is a solution to the feasibility program of conditional linear
constraints~(\ref{EQ:WeightConditionalLP}), parametrized by $\q$.

We thus build a piecewise linear arithmetic pseudo-circuit $C$ accomplishing both tasks. The circuit $C$ takes as input the variables $(\q,\p,\w)$. Using the linear-OPT-gate of~\cite{Filos-RatsikasH2023-PPAD} we let $C$ output $(\q',\p',\w')$ such that:
\begin{enumerate}
    \item $(\q',\p')$ is an optimal solution of LP~(\ref{EQ:MaxWeightedSumLP}) parametrized by $w$.
    \item If the feasibility program~(\ref{EQ:WeightConditionalLP}) parametrized by $q$ is feasible, then $w'$ is a solution.
\end{enumerate} 
Suppose now that $(\q,\p,\w)$ is a fixed point of the circuit $C$  (where also the auxiliary inputs of $C$ are assumed to be in a fixed point). Since $(\q,\p)$ is then an optimal solution of LP~\ref{EQ:MaxWeightedSumLP}) parametrized by $\w$, this means that $\q$ is Pareto-optimal by the weighted sum method. From Observation~\ref{OBS:ParetoImpliesAcyclic} and Lemma~\ref{LEM:AcyclicImpliesSolvable} we then have that the feasibility program~(\ref{EQ:WeightConditionalLP}) parametrized by $\q$ is in fact feasible, and this then means that $\w$ is a solution. By Proposition~\ref{PROP:FixedPointFormulation} we can then conclude that $\q$ is an ex-ante envy-free and Pareto-optimal lottery.

We have thus reduced the task of computing an ex-ante envy-free and Pareto-optimal lottery to the task of computing a fixed point of a piecewise linear arithmetic pseudo-circuit defined on a explicitly given convex polytope, thereby completing the proof.
