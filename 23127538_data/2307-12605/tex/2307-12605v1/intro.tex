\section{Introduction}
Fairly dividing a collection of resources among individuals (often dubbed as agents) with varied preferences forms a key concern in the design of many social institutions. Such problems arise naturally in many real-world scenarios such as assigning computational resources in a cloud com\-pu\-ting environment,  air traffic management, dividing business assets, allocation of radio and television spectrum, course assignments, and so on \cite{AdjustedWinner,etkin2007spectrum,moulin2004fair,vossen2002fair}. The fundamental problem of fair division lies at the interface of economics, social science, mathematics, and computer science, and its formal study dates back about seven decades~\cite{dubins1961cut,steihaus1948problem}. In the last few decades, the area of fair division has witnessed a flourishing flow of research; see \cite{survey2022,brams1996fair,brandt2016handbook} for excellent expositions. 
%%examples that do not include money

Traditionally, in early literature, fair division has been studied for a single \emph{divisible} resource, classically known as \emph{fair cake cutting}. Here, each agent specifies her valuations over a unit interval cake via a probability distribution over $[0,1]$ and the problem is to divide the cake among agents in a fair manner. The quintessential notion of fairness in this line of work is that of \emph{envy-freeness}, introduced by Foley \cite{foley1966resource} and Varian \cite{Varian1974Equity}. A cake division is said to be \emph{envy-free} if every agent prefers her own share of the cake over any other agent's share. Stromquist \cite{stromquist1980cut} famously proved that an envy-free cake division (where every agent receives a connected interval of the cake) is always guaranteed to exist, under mild conditions. Later, Su \cite{su1999rental}  developed another existential proof using Sperner's Lemma and established a connection between the notion of envy-freeness and topology. Such strong existential results have arguably placed the notion of envy-freeness as the flagship bearer of fairness in resource allocation settings. 


On the other hand, \emph{Pareto optimality} is a standard notion of economic efficiency, which states that it should be impossible to find an allocation that improves some agent’s utility without reducing any other's. Another important notion of (collective) efficiency measure of an allocation is that of \emph{social welfare} \cite{moulin2004fair}  which is the sum of all the utilities derived by agents from their assigned bundle.

The goal of being fair towards the participating agents and achieving collective (economic) efficiency form the two important paradigms of resource allocation problems.   Unfortunately, for an indivisible set of items, an envy-free allocation may not exist. For example, an instance with two agents having positive value for a single item admits no envy-free allocation.
%This kind of unfairness naturally motivates the use of randomness while allocating the goods. The power of randomness establishes the existence of ex-ante envy-free lotteries. 

A fair division instance consists of a set $[n] = \{1,2 \dots, n\}$ of $n$ agents and a set $M$ of items. In the most basic setting, every agent $i$ has an additive utility function $u_i \colon 2^M \to \mathbb{R}$ that specifies her cardinal preferences for the items of a given bundle; in particular, $u_i(j) \coloneqq u_i(\{j\})$ denotes agent $i$'s utility for item $j \in M$. We say an \emph{allocation} is a partition of items into $n$ bundles, where every agent is assigned one bundle. The goal of simultaneously achieving fairness and efficiency is challenging for the problem of allocating indivisible items.
%To begin with, an envy-free allocation may not exist, for example, an instance with two agents having positive %value for a single item admits no envy-free allocation. And, for instances where envy-free allocations do exist, 
Besides the mentioned fact that an envy-free allocation is not guaranteed to exists, in cases where envy-free allocations do exist, envy-freeness may not be compatible with Pareto optimality~\cite{BL16}.

The above discussion suggests that one should consider distributions over allocations (to be referred as \emph{allocation lotteries}) in order to simultaneously achieve fairness and efficiency guarantees. In the random assignment literature in economics, the idea of constructing a fractional allocation and implementing it as a lottery over deterministic allocations was introduced by Hylland and Zeckhauser \cite{HyllandZ1979-HZ}. Introducing randomness is a typical way of circumventing the non-existence of various solution concepts, especially in computational social choice theory \cite{abdulkadirouglu1998random,aziz2019probabilistic,caragiannis2019stable,dougan2016efficiency}. In the process of exploring allocation lotteries, we appropriately relax the notion of fairness to \emph{ex-ante envy-freeness}, which values the random bundles allocated agents in terms of expected utility. Recent works of \cite{aziz2020simultaneously,budish2013designing,caragiannis2021interim,freeman2020best} deals with various computational aspects of allocation lotteries that are fair and efficient for fair division instances with additive utilities.
Observe that, allocation lotteries that are just ex-ante envy-free or just ex-ante Pareto-optimal can be trivially computed in polynomial time. For the former, one can solve a linear program, while for the latter, one can assign each bundle to the agent that has the highest utility for it. That is, these notions of fairness and efficiency are tractable if dealt with individually. Therefore, the important question is to understand the computational complexity of computing allocation lotteries that are simultaneously ex-ante envy-free and Pareto-optimal. In this work, we consider this question for the most general setting of fair division, as detailed in the following section.



%agent $a_1$ has to receive item $g_1$ in any envy-free allocation. Similarly, agent $a_2$ must be assigned at least one item from $g_2$ and $g_4$, while agent $a_3$ must be assigned at least one item from $g_3$ and $g_4$ in any envy-free allocation. 


%%%Define vanilla setting and the counter example. The goal of achieving fairness and efficiency simultaneously is challenging. EF may not exists, and even when they do, they may not efficient. EXAMPLE. Therefore, we consider lotteries, relaxing fairness to ex-ante envy fairness: doable via LP. ex-ante EF lotteries are polynomial-time computable. PO is an easy goal on its own, for example, give an item to agent who values it the most. 
%
%Important question that we consider here is achieving them together. And, in this work, we consider the most general setting.

%In the random assignment literature in economics, the idea of constructing a fractional assignment and implementing it as a lottery over pure assignments was introduced by Hylland and Zeckhauser \cite{HyllandZ1979-HZ}

%We strengthen Cole and Tao existential result and establish the PPAD membership. PPAD is a subset of FIXP, and hence FIXP membership follows as a corollary.

%% Justify why the set of admissible partitions be a part of input. One can let go of bad partitions, and say that these are the admissible partitions. Work with them and give me a fair and efficient lottery. 
%Informally define our model
%%Significance and techniques of the results

\subsection{Context and overview of our results}
In this work, we consider a very general fair division setting with $n$ agents and a family of admissible $n$-partitions of an underlying set of items. Every agent is endowed with \emph{partition-based utilities} that specify her cardinal utility for different bundles in every partition. For such a broad class of fair division instances with partition-based utilities, including negative-valued utilities, the recent work of Cole and Tao \cite{cole2021existence} proves that an ex-ante envy-free and Pareto-optimal allocation lottery is always guaranteed to exist. 

Note that, partition-based utilities provide a much broader way of expressing agents' utilities. In particular, it is possible that an agent may value the exact same bundle of items in two distinct partitions at two different values. Or, there may be a certain partition of items that is not favourable or suitable (depending on the context of application), and this generalization allows us to remove unsuitable partitions from the family of admissible partitions, and still, the existence of ex-ante envy-free and Pareto optimal allocation lotteries is guaranteed. 

In this work, we examine the computational complexity of the above total search problem and strengthen the work of Cole and Tao~\cite{cole2021existence}. In particular, we establish that the problem of finding an ex-ante envy-free and Pareto optimal allocation lottery for fair division instances with partition-based utilities belongs to the complexity class \PPAD. This containment result is even interesting for the special case of a single admissible partition. Namely, our \PPAD\ membership result is for the \emph{exact} search problem, of computing a rational valued lottery. This can be contrasted with the lottery provided by the Hylland-Zeckhauser (HZ) pseudo-market. Vazirani and Yannakakis \cite{VaziraniY2021-HZ} gave a simple example with four agents and four goods where the \emph{unique} HZ equilibrium gives an irrational-valued lottery. This fact means that any algorithm for computing a HZ equilibrium exactly must overcome numerical challenges.
Our result on the other hand gives hope for the possibility of developing a \emph{practical} algorithm for computing an exact ex-ante envy-free and Pareto optimal allocation lottery, for instance by an adaptation of Lemke's algorithm~\cite{Lemke1964-Bimatrix}.

For instances with a constant number of agents, we develop a polynomial-time algorithm to compute an exact ex-ante envy-free and Pareto-optimal lottery. On the negative side, we prove that maximizing social welfare over ex-ante envy-free and Pareto optimal allocation lotteries is \NP-hard.

\subsection{Further related work}
Fairness in resource-allocation settings is extensively studied in the economics, mathematics, and computer science literature (see \cite{brams1996fair,brandt2016handbook,moulin2004fair}). As mentioned above, envy-free allocations may not exist for the case of indivisible items. Since envy-freeness is arguably a fundamental notion of fairness, as evident from its importance in fair cake cutting, there has been a significant body of research aimed towards finding ex-ante envy-free allocation lotteries in the indivisible setting. 
%In the probabilistic assignment literature in economics, the idea of constructing a fractional assignment and implementing it as a lottery over deterministic assignments was introduced by Hylland and Zeckhauser \cite{HyllandZ1979-HZ}.
The work of Freeman~et~al.~\cite{freeman2020best} addresses the key question of whether ex-ante envy-freeness
can be achieved in combination with \emph{ex-post envy-freeness up to one item}. They settle it positively by designing an efficient algorithm that achieves both properties simultaneously. 
Caragiannis~et~al.~\cite{caragiannis2021interim} explore the \emph{interim allocation lotteries (iEF)} which provide fairness guarantees that lie between ex-post and ex-ante envy-freeness. They develop
polynomial-time algorithms for computing iEF lotteries that maximize various efficiency notions.
%In this work, we explore the computational complexity of ex-ante envy-freeness along with Pareto optimality for a broad class of fair division instances, captured by partition-based utilities. 
 
Budish~et~al.~\cite{budish2013designing} employ a general class of random allocation mechanisms to achieve ex-ante fairness and efficiency in the presence of real-world constraints. Several other works explore fairness and efficiency guarantees of allocation lotteries as well, but for ordinal utilities~\cite{abdulkadirouglu1998random,bogomolnaia2001new,chen2002improving}.

Another line of research has explored various relaxations of envy-freeness. The notion of \emph{envy-freeness up to one item (EF1)} was introduced by Budish~et~al.~\cite{budish2011combinatorial} as one of the first `good' relaxations of envy-freeness in the indivisible setting. We say an
allocation is EF1 when every agent (weakly) prefers her own bundle over any other agent~$j$'s bundle after removing some item from~$j$’s bundle. EF1 allocations are guaranteed to always exist for general monotone valuations and can be computed efficiently \cite{lipton2004approximately}. Moreover, this fairness notion is compatible with the economic efficiency objective of Pareto-optimality~\cite{CKMPSW19}. Later, \emph{envy-freeness up to any item (EFX)} was introduced by Caragiannis~et~al.~\cite{CKMPSW19} as a refinement of EF1 and is now considered as the most compelling fairness criterion while dividing indivisible items. We say an allocation is EFX when every agent
(weakly) prefers her own bundle than any other agent~$j$’s bundle after removing her least positively-valued item from~$j$’s bundle. Recent works \cite{amanatidis2021maximum,amanatidis2020multiple,chaudhury2020efx,chaudhury2021little} have shown existential guarantees for EFX in various special cases.
%EF1,EFX,lotteries


% We assume instances to have an \emph{anonymity property} which allows any bundle in any partition to be assigned to any agent, i.e., there are $n!$ possible allocations corresponding to each admissible partition. 

%We consider the fundamental problem in resource allocation settings that entails dividing a collection of discrete goods among a set of participating agents in a \emph{fair} and \emph{efficient} manner. The formal roots of the problem of fair division dates back about seven decades ago \cite{steihaus1948problem,dubins1961cut} and it simultaneously enjoy real-world applications in today's world \cite{AdjustedWinner,moulin2004fair,vossen2002fair,etkin2007spectrum}. Over the last few decades, the area of fair division has witnessed a flourishing flow of research; see \cite{brams1996fair,brandt2016handbook} for excellent expositions. 

