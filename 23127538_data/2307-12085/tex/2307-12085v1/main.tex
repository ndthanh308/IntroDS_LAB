\documentclass[11pt]{article}
\usepackage{xcolor}

\usepackage[title]{appendix}

\usepackage{pst-all}
\usepackage{pgfplots}
\usepackage{pst-plot}
\usetikzlibrary{calc}

\usepackage{xcolor}
\usepackage{tikz}
\usepackage{tkz-euclide}

\usepackage{bbm}
\usepackage{hyperref}
\hypersetup{
%backref=true,       
 %   pagebackref=true,             
  %  hyperindex=true,              
    colorlinks=true,              
    breaklinks=true,              
    urlcolor= black,              
    linkcolor= blue,              
   % bookmarks=true,               
    bookmarksopen=false,
    filecolor=black,
    citecolor=blue,
    linkbordercolor=blue}

\usepackage{chngcntr}

\usepackage{mathrsfs}
\usepackage{nicefrac}

\usepackage[margin=1in]{geometry} 


\usepackage{amsmath,amsthm,amssymb}

\usepackage{bbm}

\usepackage{graphicx}
\usepackage{faktor}

\newcommand{\U}{\mathcal{U}}

\newcommand{\R}{\mathbb{R}}

\newcommand{\C}{\mathbb{C}}


\newcommand{\N}{\mathbb{N}}

\newcommand{\Z}{\mathbb{Z}}

\newcommand{\cov}{\normalfont \text{Cov}}
\newcommand{\vol}{\normalfont \text{Vol}}

\newcommand{\End}{\normalfont \text{End}}

\newcommand{\pr}{\normalfont \text{pr}}

\newcommand{\GL}{\normalfont \text{GL}}

\newcommand{\ka}
{\kappa}

\newcommand{\al}
{\alpha}

\newcommand{\be}
{\beta}

\newcommand{\SL}{\normalfont \text{SL}}

\newcommand{\SO}{\normalfont \text{SO}}

\newcommand{\Ad}{\normalfont \text{Ad}}

\newcommand{\Supp}{\normalfont \text{Supp}}

\newcommand{\p}{\textbf{p}}

\newcommand{\q}{\textbf{q}}

\newcommand{\h}{\normalfont \textbf{h}}

\newcommand{\f}{\normalfont \textbf{f}}

\newcommand{\g}{\normalfont \textbf{g}}

\newcommand{\s}{\normalfont \textbf{s}}

\newcommand{\aaa}{\normalfont \textbf{a}}
\newcommand{\bb}{\normalfont \textbf{b}}
\newcommand{\cc}{\normalfont \textbf{c}}

\newcommand{\st}{\sqrt{t}}

\newcommand{\e}{\epsilon}

\newcommand{\de}{\delta}
\newcommand{\De}{\Delta}
\newcommand{\la}{\lambda}

\newcommand{\La}{\Lambda}

\newcommand{\Ga}{\Gamma}



\newcommand{\Ha}{\mathscr{H}}
\newcommand{\Pa}{\mathscr{P}}

\newcommand{\Sph}{\mathbb{S}}

\newcommand{\D}{\mathcal{D}}
\newcommand{\V}{\mathcal{V}}
\newcommand{\Sing}{\normalfont \textbf{Sing}}
\newcommand{\VSing}{\normalfont \textbf{VSing}}
\newcommand{\BA}{\normalfont \textbf{BA}}
\newcommand{\Span}{\normalfont \text{Span}}
\newcommand{\diag}{\normalfont \text{diag}}


\newtheorem{theorem}{Theorem}[section]

\newtheorem{lemma}[theorem]{Lemma}

\newtheorem{proposition}[theorem]{Proposition}

\newtheorem{corollary}[theorem]{Corollary}



\newtheorem{claim}{Claim}


\newtheorem{tad}[theorem]{Theorem and Definition}

\theoremstyle{definition}

\newtheorem{definition}[theorem]{Definition}

\newtheorem{example}[theorem]{Example}

\newtheorem{question}[theorem]{Question}

\newtheorem{xca}[theorem]{Exercise}

\newtheorem{conjecture}[theorem]{Conjecture}

\theoremstyle{remark}

\newtheorem{remark}[theorem]{Remark}



\numberwithin{equation}{section}



\newenvironment{solution}{\begin{proof}[Solution]}{\end{proof}}
 
\usepackage{setspace}

\usepackage[
%backend=biber,
style=alphabetic,
sorting=nty,
maxnames=99,
maxalphanames=99
%sorting=ynt
]{biblatex}
\renewbibmacro{in:}{}
\addbibresource{ref.bib}

\DeclareFieldFormat[article]{citetitle}{\mkbibemph{#1}}
\DeclareFieldFormat[article]{title}{\mkbibemph{#1}}
\DeclareFieldFormat[article]{journaltitle}{#1}

\usepackage{enumerate}

\AtEndDocument{%
  \par
  \medskip
    \begin{tabular}{@{}l@{}}%
    \textsc{Department of mathematics, the Ohio State University}\\
    \vspace{0.5cm}
    \textit{E-mail address}: \texttt{bersudsky.1@osu.edu}\\
    
    \textsc{Department of mathematics, the Ohio State University}\\
    \textit{E-mail address}: \texttt{xing.211@osu.edu}
  \end{tabular}}
  
\title{Limiting distribution of dense orbits in a moduli space of rank $m$ discrete subgroups in $(m+1)$-space}
\author{Michael Bersudsky and Hao Xing}

\begin{document}
\date{}
\maketitle
\begin{abstract}
Consider the moduli space $X_{m,m+1}$  rank-$m$ discrete subgroups of covolume equal to one in $\R^{m+1}$. There is a natural action of $\SL(m+1,\R)$ on $X_{m,m+1}$, and it turns out that for every lattice subgroup $\Ga\leq\SL(m+1,\R)$, each orbit $x_0.\Ga$ is dense in $X_{m,m+1}$. In this paper we compute the limiting distribution of these orbits with respect to a filtration of growing norm balls, where the norm is given by the sum of squares. The main motivation for this result comes from the work of Sargent and  Shapira where they studied random walks in $X_{2,3}$.

Another motivation for our work is to extend the scope of applications of the duality principle in homogeneous dynamics. The moduli space $X_{m,m+1}$ identifies as $H\backslash \SL(m+1,\R)$, and the duality principle recasts the above problem into the problem of establishing certain volume growth properties of growing skewed balls in $H$ and proving ergodic theorems of the left action of $H$ on $\SL(m+1,\R)/\Gamma$ along the skewed balls.   Specifically, we use the duality principle as developed in the work of Gorodnik and Weiss. Our ergodic theorems are proven by applying theorems of Shah building on the linearisation technique. Previously, the duality principle wasn't applied in the setting where $H$ has infinitely many non-compact connected components. Our general result is in the setting where $H$ is a certain subgroup of a minimal parabolic group of $\SL(m+1,\R)$ such that in the Levi-component there is a lattice.
%Specifically, we consider  $\Gamma_T=\{\gamma\in\Gamma:\|\gamma\|\leq T\}$ and show that, for any fixed $x_0\in X_{m,m+1}$ and $\varphi\in C_c(X_{m,m+1})$, 
%$$\lim_{T\to\infty}\frac{1}{\#\Gamma_T}\sum_{\gamma\in\Gamma_T}\varphi(x_0\cdot\gamma)=\int_{X_{m,m+1}}\varphi(x)d \tilde\nu_{x_0}(x),$$
%where $\tilde\nu_{x_0}$ is an explicit probability measure on $X_{m,m+1}$ depending on $x_0$. 

\iffalse
Consider the moduli space $X_{m,m+1}$ of homothety classes of rank-two discrete subgroups with orientation in $\R^{3}$. The group $\SL(m+1,\R)$ acts transitively on $X_{m,m+1}$, so that $X_{m,m+1}\cong H\backslash \SL(m+1,\R)$, where $H\leq\SL(m+1,\R)$ is a closed subgroup. It turns out that for each $x_{0}\in X_{m,m+1}$ and each lattice subgroup $\Gamma \leq \SL(m+1,\R)$, the orbit $ x_{0}\cdot \Gamma $ is dense in $X_{m,m+1}$. We compute the limiting distribution of those orbits with respect to norm balls, where the norm is given by $\|g\|^2=\sum g_{i,j}^2$, $g\in\SL(m+1,\R)$, the sum of squares of the entries of $g$. Namely, we consider $\Gamma_{T}=\{\gamma\in\Gamma : \|\gamma\| \leq T \}$, and we show that for all $\varphi\in C_c(X_{m,m+1})$             
$$\lim_{T
\to\infty} \frac{1}{\#\Gamma_{T}}\sum_{\gamma\in \Ga_T}\varphi(x_{0}\cdot \gamma)=c_{\Ga}\int_{X_{m,m+1}}\varphi(x)d\nu_{x_{0}}(x)$$ 
for some explicit measure $\nu_{x_0}$ on $X_{m,m+1}$ which depends on $x_{0}$ and constant $c_{\Ga}$ which depends on $\Gamma$. To prove our result, we use the general recipe for applying the duality principle developed by Gorodnik and Weiss which reduces the above problem to computation of volumes of growing skewed balls in H and proving ergodic theorems of the left H-action on $\SL(m+1,\R)/\Gamma$. The main novelty of our work is that we apply the duality principle in the case that H has infinitely many connected components. 
\fi

%   We define a closed subgroup closed subgroup $H$ in $\SL(m+1,\R)$ such that the homogeneous space $H\backslash \SL(m+1,\R)$ is identified with the space of rank two lattices in $\R^{m}$. By estimating the volume of skewed balls and using the technique of linearization, we prove an equidistribution result for the normalized measures on $\SL(m+1,\R)/\SL(3,\Z)$. 
\end{abstract}

\section{Introduction}

%The interest in the shape of rank $k$ sublattices in $\R^d$ dates back to the work of Schmidt \cite{Schmidt1998TheDO}. A refining problem is studied by \cite{emss, AES_lat}. As we described below, the space $X_{m,m+1}$ is a homogeneous space of $\SL(m+1,\R)$ and it is natural to study the dynamics of the subgroups of $\SL(m+1,\R)$. 

%\cite{Sargent2017DynamicsOT},\cite{gorodnik2022stationary}, \cite{Oh05}.

%The case when $H$ is discrete done by \cite{Oh05} and generalized by \cite{Gorodnik2004DistributionOL}.

%The case when $H$ is algebraic were studied in generality by \cite{GorodnikNevo2012} \cite{Gorodnik2003LatticeAO} 

%Recall that the space of unimodular lattices in $\R^{m}$ can be identified with $G/\Ga:=\SL(m+1,\R)/\SL(3,\Z)$, for the action of matrix on column vectors on the left (or $\Ga\backslash G$ for the action of matrix on row vectors on the right).
In this paper we study the asymptotic distributional properties of the action of a lattice $\Ga\leq\SL(m+1,\R)$ in the space $X_{m,m+1}$ of normalized $m$-dimensional discrete subgroups of $\R^{m+1}$ with respect to a filtration given by growing norm balls (see precise definitions below). Such a research direction is a natural continuation\footnote{This work started during the first-named author's Ph.D. studies under the guidance of Uri Shapira who suggested the problem about the limiting distribution in $X_{m,m+1}$ of the action of a lattice $\Ga\leq\SL(n,\R)$ with respect to growing norm balls, which was inspired by \cite{Sargent2017DynamicsOT}.} of the study initiated in \cite{Sargent2017DynamicsOT} which considers random walks on $X_{2,3}$. See also the more recent work \cite{gorodnik2022stationary} which generalizes \cite{Sargent2017DynamicsOT}. 
Another motivation for our work is to extend the scope of applications of the duality principle in homogeneous dynamics to the ergodic theory of lattice subgroups, see Section \ref{sec:intro the duality princ} below for more details. We start with our results in $X_{m,m+1}$. In what follows, $m$ is a natural number strictly larger than 1.
%We note that the space $X_{m,m+1}$ is a "hybrid" of the more familiar space $X_m:=\SL(m,\R)/\Delta$ of unimodular lattices in $\R^m$  with the grassmanian of hyperplanes in $\R^{m+1}$. The space of lattices $X_m$ plays a fundamental role in many of the applications of homogeneous dynamics, and therefore we find it interesting to consider the moduli space $X_{m,m+1}$. 

We say that $\La\subset\R^{m+1}$ is a $m$-lattice if $\La$ is the $\Z$-Span of a tuple of linearly independent vectors  ${v_1,v_2,...,v_m}\in\R^{m}$, that is, $$\Lambda:=\text{Span}_{\Z}\{v_1,v_2,...,v_m\}.$$ For $\La$ we let $$\cov(\Lambda):=\sqrt{\det(\langle v_i,v_j\rangle)},$$
which is the area of a fundamental parallelogram of $\Lambda$. 
An $m$-lattices $\La$ is called \textit{unimodular} if $\cov(\La)=1$. 
 Next, we recall the definition of the shape of lattices, a notion that was extensively studied in e.g \cite{emss, AES_lat}, which refined the classical work of Schmidt \cite{Schmidt1998TheDO}.
We view $\R^{m+1}$ as row vectors, and for a unimodular $m$-lattice $\La\subset\R^{m+1}$, let $w\in\mathbb{S}^{m}$ be such that $w\perp \La$. We choose a $\rho \in \SO(m+1,\R)$ such that $w \rho=e_{m+1}:=(0,\dots,0,1)$, and we define the shape of the pair $(\La,w)$ as
\begin{equation}\label{definition of shape}
 \s(\La,w):=\La \rho\begin{bmatrix}
\SO(m,\R) & 0 \\
0 &  1 
\end{bmatrix}, 
\end{equation}
which is independent of the choice of $\rho$. By identifying the unimodular $m$-lattices $\La\subset e_{m+1}^\perp$ with $X_m:=\SL(m,\Z)\backslash \SL(m,\R)$, we view $\s(\Lambda,w)$ as a point in $X_m\slash \SO(m,\R)$. Note that in general $$\s(\La,w)\neq\s(\La,-w).$$
\begin{remark}
 A more intrinsic definition of a shape of a discrete subgroup, see e.g. \cite{Sargent2017DynamicsOT, Schmidt1998TheDO}, is defined by the equivalence class under the equivalence relation of scaling and rotations. When defining shape in this way, one gets a point in $X_m/\text{O}(m,\R)$, and it captures slightly less information. Our definition \eqref{definition of shape} is mainly motivated by the definition in \cite{emss, AES_lat}, which produces a point in the more familiar space $X_m/\SO(m,\R)$.
\end{remark}
We consider $$X_{m,m+1}:=\{(\Lambda,w):\cov(\La)=1,w\in\mathbb{S}^m,w\perp\La\},$$ and note that $\s$ defined in \eqref{definition of shape} yields a map $$\s:X_{m,m+1}\to X_m/\SO(m,\R).$$


We define a right $\SL(m+1,\R)$ action on $X_{m,m+1}$ using the usual right matrix multiplication by \begin{equation} \label{action of SL on X_m,3}
   (\La,w). g:=\left(\frac{\La g}{\sqrt{\cov(\La g)}}, \frac{w (^tg^{-1})}{\|w (^tg^{-1})\|} \right),~g\in \SL(m+1,\R),~w\in \mathbb{S}^m, 
\end{equation}  where $\|\cdot\|$ is the usual Euclidean norm. We note that this action is transitive.
\iffalse
To motivate our main result, we note the following statement.
\begin{proposition}
   Let $\Ga\leq\SL(m+1,\R)$ be a lattice and fix $x_0\in X_{m,m+1}$. Then $x_0.\Gamma$ is dense.
\end{proposition}

\textbf{TODO: write how it can be proven and give reference to Dani} 
\fi

It turns out that for each lattice subgroup $\Ga\leq \SL(m+1,\R)$ and each $x_0\in X_{m,m+1}$, the orbit $x_0\Ga$  is dense in $X_{m,m+1}$. This follows for example by \cite{Sargent2017DynamicsOT}, and a more direct proof of this fact is obtained by applying a duality argument as follows --- $X_{m,m+1}$ is a homogeneous space identified with $H\backslash\SL(m+1,\R)$, with $H$ as in \eqref{eq:H as in Xm m+1}. Now a $\Ga$-orbit $Hg \Ga$ is dense in $H\backslash\SL(m+1,\R)$ if and only if the "dual" $H$-orbit $Hg\Ga$ is dense in $\SL(m+1,\R)/\Ga$. By Proposition 1.5 of \cite{Dani1980OrbitsOE}, it follows that all $H$-orbits in $G/\Ga$ are dense. 

Our goal in this paper will be to compute the limiting distribution of $\Ga$ orbits in $X_{m,m+1}$ with respect to growing Hilbert-Schmidt norm balls. Namely, let $\|g\|=\sqrt{\text{Trace}(^tgg)}=\sqrt{\sum_{ij}g_{ij}^2}$ be the Hilbert-Schmidt norm of $g\in\SL(m+1,\R)$, and let \begin{equation}\label{eq:def of Ga T}
 \Ga_T:=\{\gamma \in \Ga:\|\gamma\|\leq T\}.   
\end{equation}For $$x_0:=(\La_0,w_0)\in X_{m,m+1},$$ consider the probability measures $$\mu_{T,x_0}:=\frac{1}{\#\Ga_T}\sum_{\gamma\in \Ga_T}\delta_{x_0.\gamma},~T>0.$$
\begin{remark}
    When $m=1$ the space $X_{m,m+1}$ is naturally identifies with $\Sph^1$ the unit circle in $\R^2$. The limiting distribution of $\mu_{T,x_0}$ in the case of $m=1$ was obtained in \cite{Gorodnik2003LatticeAO}. 
\end{remark}
Our result below states  that the probability measures $\mu_{T,x_0}$ converge as $T\to\infty$ to a probability measure $\tilde\nu_{x_0}$ depending on ${x_0}\in X_{m,m+1}$ which we describe now. We observe that $X_{m,m+1}$ has a natural projection to $\mathbb{S}^m$ defined by $$\pi_{\perp}(\La,w):=w,$$ which endows $X_{m,m+1}$ with a fiber-bundle structure, where the fibers are isomorphic to $$X_m:=\SL(m,\Z)\backslash\SL(m,\R).$$ 
To define the measure $\tilde\nu_{x_0}$, we define a measure $\mu_{x_0,w}$ on each fiber $\pi_{\perp}^{-1}(w)$, and we integrate those measures by the unique rotation invariant probability measure $\mu_{\mathbb S^m}$ on $\mathbb S^m$. We note that the measures $\mu_{x_0,w}$ have a slightly surprising form; they are a combination of a $\SL(m,\R)$-invariant measure with a density involving the Hilbert-Schmidt norm of operators which we define now.

For an operator $T$ from a hyperplane $U\subset \R^{m}$ to another hyperplane $V\subset \R^{m}$, we define
\begin{equation}\label{defining hilbert-schmits for op. from hyp. to hyp.}
    \|T\|^2_{\text{HS}}:=\sum_{i=1}^m\|Tu_i\|^2,
\end{equation}
where $\{u_1,u_2,...,u_m\}$ is an orthonormal basis of $U$, and where the norm on the right hand side is the usual Euclidean norm on $\R^{m}$. We note that this norm is independent of the choice of an orthonormal basis $\{u_1,u_2,...,u_m\}$. Moreover, this norm is bi-$\SO(m+1,\R)$ invariant in the following sense. If $\rho_1,\rho_2\in\SO(m+1,\R)$, then $\rho_2 \circ T \circ \rho_1:\rho_1^{-1}U\to \rho_2V$ satisfies $$\|\rho_2 \circ T \circ \rho_1\|_{\text{HS}}=\|T\|_{\text{HS}}.$$
For an ordered tuple of linearly independent vectors $B=(u_1,u_2,...,u_m)\in\R^{m\times m}$ we define the linear map $T_B:\text{Span}_{\R}  \{e_1,e_2,...,e_m\} \to\text{Span}_{\R}\{u_1,u_2,...,u_m\}$, by sending $e_1\mapsto u_1,...,e_m\mapsto u_m$. Now fix unimodular $m$-lattice $\Lambda_0\subset\R^{m}$, and let $\mathscr{B}_0$ be an ordered tuple of linearly independent vectors forming a $\Z$-basis for $\La_0 $. We define for an arbitrary unimodular $m$-lattice $\La\subset\R^{m}$, \begin{equation}\label{defining psi_La_0}
    \Psi_{\La_0} (\La):=\sum_{\Span_\Z{\mathscr{B}}=\La}\frac{1}{\|T_{\mathscr{B}}\circ T_{\mathscr{B}_0}^{-1}\|_{\text{HS}}^{m^2}}.
\end{equation}
We note that $\Psi_{\La_0}$ is independent of the choice of basis $\mathscr{B}_0$, and we observe that by bi-$\SO(m+1,\R)$ invariance of the Hilbert-Schmidt norm that the values of the function $\Psi_{\La_0}(\La)$ only depends on the shapes of $\La$ and $\La_0$.

By identifying $\pi_\perp^{-1}(e_{m+1})$ with $\SL(m,\Z)\backslash \SL(m,\R)$, we obtain the $\SL(m,\Z)\backslash \SL(m,\R)$ invariant measure $\mu_{e_{m+1}}$ supported on $\pi_\perp^{-1}(e_{m+1})$ scaled such that the measure $\nu_{x_0,e_{m+1}}$ defined by $$\nu_{x_0,e_{m+1}}(f):=\int_{\pi^{-1}_\perp(e_{m+1})}f(\La,e_{m+1})\Psi_{\La_0}(\La)d\mu_{e_{m+1}}(\La),~f\in C_c(X_{m,m+1}),$$ is a probability measure. Then, the measure supported on $\pi_\perp^{-1}(w), \text{ for } w\in \mathbb{S}^m$, is defined by choosing $\rho_w\in \SO(m+1,\R)$ such that $w=e_{m+1} \rho_w$, and by letting,$$\nu_{x_0,w}:=(\rho_w)_*\nu_{x_0,e_{m+1}},$$ which is the push-forward of the right translation by $\rho_w$ via the right action of $\SO(m+1,\R)$ on $X_{m,m+1}$ defined in \eqref{action of SL on X_m,3}. Note that $\nu_{x_0,w}$ is independent of the choice of $\rho_w$. Finally, we define $\tilde\nu_{x_0}$ by $$\tilde\nu_{x_0}(f)=\int_{\mathbb{S}^m}\nu_{x_0,w}(f)d\mu_{\mathbb{S}^m}(w).$$   
%We note that the above action is transitive.



\begin{theorem}\label{equidistribution result on  G mod H in Xm,m+1}
 Let $\Ga\leq\SL(m+1,\R)$ be a lattice and fix ${x_0}\in X_{m,m+1}$. Then, $\mu_{T,{x_0}}$ converges in the weak-* topology to $\tilde\nu_{x_0}$ as $T\to\infty$. In other words,
 for all $f\in C_c(X_{m,m+1})$, we have 
\begin{equation}
\lim_{T
\to\infty} \frac{1}{\#\Gamma_{T}}\sum_{\gamma\in \Ga_T}f(x_{0}\cdot \gamma)=\int_{X_{m,m+1}}f(x)d \tilde\nu_{x_{0}}(x).
\end{equation}

\end{theorem}
 We observe that that the push-forward of $\tilde{\nu}_{x_0}$ by $\s$ is given by $$\s_*\tilde{\nu}_{x_0}=\s_*\nu_{x_0,e_{m+1}}.$$
\begin{corollary}
     Let $\Ga\leq\SL(m+1,\R)$ be a lattice and fix ${x_0}\in X_{m,m+1}$. Then, the probability measures on $X_m\slash \SO(m,\R)$ given by $$\s_*\mu_{T,{x_0}}=\frac{1}{\#\Ga_T}\sum_{\gamma\in\Ga_T}\delta_{\s({x_0}.\gamma)},~T>0,$$ 
    converge in the weak-* topology to the probability measure $\s_*\nu_{x_0,e_{m+1}},$ as $T\to\infty$.
\end{corollary}




\subsection{Connection to homogeneous dynamics - the duality principle}\label{sec:intro the duality princ}

To simplify notation, we denote $G:=\SL(m+1,\R)$. We note that $G$-action on $X_{m,m+1}$ given in \eqref{action of SL on X_m,3} is transitive, and we observe that the stabilizer subgroup of the base point $(\Span_\Z\{e_1,\dots,e_{m}\},e_{m+1})$ is  
\begin{equation}\label{eq:H as in Xm m+1}
H=\left \{\begin{bmatrix}
t^{-\frac{1}{m}}q & 0 \\
v &  t 
\end{bmatrix}:t> 0, q \in \SL(m,\Z), v \in \R^m \right\}.
\end{equation} 
Then we obtain the identification $H \backslash G \cong X_{m,m+1}$.

This connects our problem to the study of distribution of orbits of closed subgroups in homogeneous spaces.  The duality principle allows us to connect the equidistributional properties of the $\Ga$-orbits on $H\backslash G$ to the equidistributional 
properties of the $H$-orbits in the dual action of $H$ on $G/\Ga$, see Section 1.7 of \cite{GN14duality} for an extensive exposition of the existing literature on this principle.  A general recipe for applying the duality principle was developed in \cite{GN14duality} and \cite{Gorodnik2004DistributionOL}.  Our approach in this paper uses a theorem of Gorodnik and Weiss (\cite{Gorodnik2004DistributionOL}), since it allows to prove equidistribution for every starting point. 

\vspace{5mm}
Prior to this work, the duality principle wasn't applied to the setting in which $H$ has infinitely many non-trivial connected components. We note that the case when $H$ is connected algebraic was studied in great generality in \cite{GorodnikNevo2012,Gorodnik2004DistributionOL}, and the case where $H$ is a lattice was studied in \cite{Oh05} and in greater generality in \cite{Gorodnik2004DistributionOL}. 

In the section below we formulate our main result which we view as a first step towards a more general theorem in the setting which $G$ is a semi-simple group and $H$ is a subgroup of a parabolic group $P$ such that in at least one of the levi-components of $P$ appears a lattice.
\subsection{Our general results}
From now on, $G:=\SL(m+1,\R)$, $\Ga\leq G$ is a lattice, \begin{equation}
H:=\left \{\begin{bmatrix}
t^{-\frac{1}{m}}q & 0 \\
v &  t 
\end{bmatrix}:t> 0, q \in \De, v \in \R^m \right\},
\end{equation}where $\De\leq\SL(m,\R)$ is a lattice, and \begin{equation}
    P:=\left \{\begin{bmatrix}
t^{-\frac{1}{m}}\eta & 0 \\
v &  t 
\end{bmatrix}:t> 0, \eta \in \SL(m,\R), v \in \R^m \right\}.
\end{equation}  
Let $x_0:=Hg_0\in H\backslash G$, and consider the one-parameter set of probability measures on $H\backslash G$
$$\mu_{T,x_0}:=\frac{1}{\#\Ga_T}\sum_{\gamma\in\Ga_T}\de_{x_0.\gamma},$$
for $T>0$. Our main result below will show that $\mu_{T,x_0}$ equidistributes as $T
\to\infty$ with respect to the  $\tilde{\nu}_{x_0}$ the probability measure we define below. 

The space $H\backslash G$ is naturally a fiber-bundle over $P\backslash G$ with respect to the natural map sending $$\pi(Hg):=Pg,$$ (note that $P\backslash G$ is identified with $\Sph^{m}$ the unit sphere in $\R^{m+1}$ via the right action $w.g:=\frac{w (^tg^{-1})}{\|w (^tg^{-1})\|},~g\in G,~w\in\Sph^{m}$). The fibers are isomorphic to $\Delta\backslash\SL(m,\R)$. Notice that $\SO(m+1,\R)$ acts transitively on $P\backslash G$ (with respect to the natural action), and we let $\mu_{P.\SO(m+1,\R)}$ be the right $\SO(m+1,\R)$-invariant probability on $P\backslash G=P.\SO(m+1,\R)$. Now we define measures on each fiber $$\pi^{-1}(P\rho)=HP\rho,~\rho\in\SO(m+1,\R).$$ We write (as we may, using Iwasawa decomposition) $$x_0=H \begin{bmatrix} G_0  & 0 \\ v_0 & \nicefrac{1}{\det(G_0)} \end{bmatrix}\rho_0,$$where $G_0\in\SL(m,\R)$, and $\rho_0\in\SO(m+1,\R)$, and consider the function $$\Phi_{x_0}\left(H\begin{bmatrix} \eta  & 0 \\ v & \nicefrac{1}{\det(\eta)} \end{bmatrix}\rho\right)=\sum_{q\in \Delta}\frac{1}{\|G_0^{-1}q \eta\|^{m^2}}.$$
Here $\rho\in\SO(m+1,\R)$ and $\eta\in \SL(m,\R)$. The experession on the right is well-defined as the Hilbert-Schmidt norm is bi-$\SO(m+1,\R)$ invariant. The infinite sum is convergent due to Lemma \ref{lemma on the integral and summation in a ball} below. The ``standard" fiber $\pi^{-1}(P)=H\backslash P$ is naturally identified with $\Delta\backslash \SL(m,\R)$, and we let $\mu_{H.\SL(m,\R))}$ be the $\SL(m,\R)$ invariant measure on $H\backslash P$ such that the measure $$\nu_{x_0,\pi^{-1}(P)}(f):=\int_{\pi^{-1}(P)}f(y)\Phi_{x_0}(y)d\mu_{H.\SL(m,\R)}(y),~f\in C_c(H\backslash G)$$ is a probability measure. On any other fiber $\pi^{-1}(P\rho),~\rho\in\SO(m+1,R)$, we define the pushed measure $$\nu_{x_0,\pi^{-1}(P\rho)}:=\rho_*\nu_{x_0,\pi^{-1}(P)},$$where the pushforward is via the right translation by $\rho.$ Having defined the above measures on the base space and the fibers of the fiber bundle $H\backslash G$, we can now define our probability measure on $H\backslash G$ to be $$\tilde\nu_{x_0}(f)=\int_{P.\SO(m+1,\R)}\nu_{x_0,\pi^{-1}(b)}(f)d\mu_{P.\SO(m+1,\R)}(b),~f\in C_c(H\backslash G).$$   
\begin{theorem}\label{equidistribution result on  G mod H}
 Let $\Ga\leq\SL(m+1,\R)$ be a lattice and fix ${x_0}\in H\backslash G$. Then, $\mu_{T,{x_0}}$ converges in the weak-* topology to $\tilde\nu_{x_0}$ as $T\to\infty$. 
\end{theorem}
 Theorem \ref{equidistribution result on  G mod H in Xm,m+1} is a particular case of Theorem \ref{equidistribution result on  G mod H}. We leave the details to the reader.
 
 As mentioned above, to prove our main result we will follow the method developed in \cite{Gorodnik2004DistributionOL}. The key ingredients  are certain volume estimates and certain ergodic theorems which we present in the following section. 

\subsection{Volume estimates of expanding skew balls in $H$ and an equidistribution theorem on $G/\Ga$}

Let us first describe the left invariant measure on $H$. Put
\begin{equation}
    \De_{m,1}:=\begin{bmatrix}
\De & 0 \\
0 &  1 
\end{bmatrix}, A=\left \{\begin{bmatrix}
t^{-\frac{1}{m}}I_2 & 0 \\
0 & t 
\end{bmatrix}:t>0 \right\}, U:=\begin{bmatrix}
I_2 & 0 \\
\mathbb{\R}^m & 1 
\end{bmatrix}
\end{equation}

Then the group $H$ has the decomposition:
$$H=U\rtimes(\De_{m,1}\times A)=U\rtimes(A \times \Delta_{m,1}).$$
Notice that any element in $H$ can be uniquely represented as $uq a$, where $u\in U$, $q \in \De_{m,1}(\Z)$, $a \in A$ (note that $a$ commutes with $q$),

where the semidirect product is given by:
\begin{equation}
    (u_1,q_1,a_1)\cdot (u_2,q_2, a_2):= (u_1 q_1 a_1 u_2 (q_1 a_1)^{-1}, q_1 q_2, a_1 a_2)
\end{equation}

%Note that clearly (as a product of closed sets), $H$ is a closed subgroup of $G$. Hence it follows from the Theorem 3.58 in \cite{Wa83} that we have the following locally smooth section property:for any $x\in H\backslash G$, there exists a neighborhood $x \in W \subset H\backslash G$ and a smooth map $\sigma:W \to G$ such that 
%\begin{equation*}
%    \pi \circ \sigma = \text{id}.
%\end{equation*}

Now we give a formula for the left Haar measure $\mu$ on $H$. 
For $x\in H$, write $x=u_v a_t  q $, where

\begin{equation}
    q:=\begin{bmatrix}
q & 0 \\
0 &  1 
\end{bmatrix} \footnote{Here and henceforth we shall abuse the notation $q$, allowing it to represent both the $m\times m$ and $(m+1) \times (m+1)$ matrices whenever its meaning is evident from the context.}, 
a_t:=\begin{bmatrix}
t^{-\frac{1}{m}}I_m & 0 \\
0 & t 
\end{bmatrix}, 
u_v:=\begin{bmatrix}
I_m & 0 \\
v & 1 
\end{bmatrix},   
\end{equation}
where $q\in \Delta,t>0, v \in \R^m$. Then a left Haar measure on $H$ is given by
\begin{align}\label{eq:Haar measure on $H$}
    \int_H f(x)d\mu(x)
    =&\sum_{q\in \Delta}\int_0^{\infty}\int_{\R^m} f(u_v a_tq) dv \frac{1}{t^{m+2}} dt, 
\end{align}
where the measure $dv$  and $dt$ denote the Lebesgue measures on $\R^m$ and $\R$ correspondingly. 

\iffalse
\begin{proof}
 To check the left invariance, fix $q_0 \in \SL_{m,1}(\Z)$, $a_{t_0}:=\begin{bmatrix}
t_0^{-\frac{1}{2}}I_2 & 0 \\
0 & t_0 
\end{bmatrix}, t_0\ne 0$ and $u_{v_0}:=\begin{bmatrix}
I_2 & 0 \\
v_0 & 1 
\end{bmatrix}$, $v_0:=(x_0,y_0)\in \R^m$, then 
\begin{align*}
      u_{v_0} a_{t_0} q_0 \cdot u_v a_t q
      =& u_{v_0} [a_{t_0} q_0 u_v (a_{t_0} q_0)^{-1}] a_{t_0} q_0 a_t q \\
      =& u_{v_0} u_{t_0^{\frac{3}{2}} vq_0^{-1}}a_{t_0} a_t q_0 q \\
      =& u_{v_0+t_0^{\frac{3}{2}} vq_0^{-1}}a_{t_0 t} q_0 q
\end{align*}
It follows that
\begin{align*}
 & \int_{\SL_{m,1}(\Z)}\int_0^{\infty}\int_{\R^m} L_{u_{v_0} a_{t_0} q_0}[f(uaq) ] dx dy \frac{1}{t^{m+2}}dt dq \\
 =&  \int_{\SL_{m,1}(\Z)}\int_0^{\infty}\int_{\R^m} f(u_{v_0} a_{t_0} q_0uaq)  dx dy \frac{1}{t^{m+2}}dt dq \\
 =&  \int_{\SL_{m,1}(\Z)}\int_0^{\infty}\int_{\R^m} f(u_{v_0+t_0^{\frac{3}{2}} vq_0^{-1}}a_{t_0 t} q_0 q) dx dy \frac{1}{t^{m+2}}dt  dq
\end{align*}
By the change of variable
\begin{equation*}
    v:=v\mapsto v':=(x',y')=v_0+t_0^{\frac{3}{2}}vq_0^{-1}, t\mapsto t':=t_0t, q \mapsto q':=q_0 q,
\end{equation*}
and noticing that $q_0$ (with determinant $1$) does not contribute to the Jacobian, the last line of the equation above becomes
\begin{equation*}
    \int_{\SL_{m,1}(\Z)}\int_0^{\infty}\int_{\R^m} f(u_{v'}a_{t'}q') (t_0^{-\frac{3}{2}})^2 dx' dy' \frac{t_0^4}{t'^4}d(\frac{t'}{t_0})  dq'=    \int_{\SL_{m,1}(\Z)}\int_0^{\infty}\int_{\R^m} f(u_{v'}a_{t'}q') dx' dy' t'^4 dt'  dq'
\end{equation*}
This verifies the invariance of measure. Finally, by the uniqueness of Haar measure, the measure $dq$ on $\SL_{m,1}(\Z)$ is the counting measure and therefore the out-most integral reduces to the sum. 
\end{proof}
\fi

Let $\|\cdot\|$ denote the Hilbert-Schmidt norm on matrices. Namely $\|A\|:=\sqrt{\text{Trace}(A^TA)}$, or the square root of the sum of squares of all entries of the matrix $A$.

For any subgroup $L$ of $G=\SL(m+1,\R)$, let
\begin{equation}\label{meaning of the subscript T}
    L_T:=\{g\in L: \|g\|<T\}.
\end{equation}

Following \cite{Gorodnik2004DistributionOL} (cf. \cite{GorodnikNevo2012}), for $g_1,g_2 \in G$ and $T>0$, we define the so-called ``skewed balls" as follows:
\begin{equation}
    H_T[g_1,g_2]:=\{h\in H: \|g_1^{-1}hg_2\|<T \}.
\end{equation}
Let 
\begin{equation}
    V_{q,T}[g_1,g_2]:=\{h\in U A q: \|g_1^{-1}hg_2\|<T \},
\end{equation}
then it follows that 
\begin{equation}\label{decomposition of H_T into V_T}
    H_T[g_1,g_2]=\bigsqcup_{q \in \Delta}V_{q,T}[g_1,g_2].
\end{equation}

For $g_1, g_2 \in G$, by using Iwasawa decomposition (for block-lower-triangular matrix), we may assume 
\begin{equation}\label{matrix representation of g1 and g2 old}
    g_1:=\begin{bmatrix} \text{g}_1 & 0 \\\textbf{v}_1 & \det(\text{g}_1)^{-1} \end{bmatrix}k_1, ~g_2:=\begin{bmatrix} \text{g}_2 & 0 \\\textbf{v}_2 & \det(\text{g}_2)^{-1} \end{bmatrix} k_2
\end{equation}
where $k_1,k_2 \in \SO(m+1,\R)$, $\text{g}_1,\text{g}_2\in \R^{m\times m}$, and $\textbf{v}_1, \textbf{v}_2\in \R^{1\times m}$.

The key volume estimates we will prove are given in the following proposition.
\begin{proposition}\label{computation for H and V} Let $\Gamma(z)$ be the classical Gamma function, and let $$C(m):=\frac{\pi^\frac{m}{2}m\Ga(\frac{m^2}{2})}{2\Ga(\frac{m^2}{2}+\frac{m}{2}+1)}.$$There exists $\kappa>0$ such that for any bounded subset $B\subseteq \SL(m+1,\R)$ it holds for all $g_1,g_2\in B$ that for $q\in\Delta, $\begin{equation}\label{eq:main estimate of V skew balls}
    \mu(V_{q,T}[g_1,g_2])=
    C(m)\frac{\det(\emph{\text{g}}_1)^m}{ \det(\emph{\text{g}}_2)}\frac{T^{m(m+1)}}{\|\emph{\text{g}}_1^{-1}q\emph{\text{g}}_2\|^{m^2}}+C_q O(T^{m(m+1)-\kappa}),
   \end{equation}
where $C_q>0,$ and
   \begin{equation}\label{eq:the estimate for H skew balls}
    \mu(H_T[g_1,g_2])=C(m)\frac{\det(\emph{\text{g}}_1)^m}{ \det(\emph{\text{g}}_2)}T^{m(m+1)}\sum_{q\in \Delta}\frac{1}{\|\emph{\text{g}}_1^{-1}q\emph{\text{g}}_2\|^{m^2}}+O(T^{m(m+1)-\kappa}).  
   \end{equation}

\end{proposition}
As an immediate corollary, we  obtain the following statements which are key requirements for the method of \cite{Gorodnik2004DistributionOL}.
\begin{corollary}
    [Uniform volume growth for skewed balls in $H$, property D1 in \cite{Gorodnik2004DistributionOL}]
For any bounded subset $B\subset G$ and any $\e>0$, there are $T_0$ and $\de>0$ such that for all $T>T_0$ and all $g_1,g_2 \in B$ we have:
\begin{equation}
    \mu \left(H_{(1+\de)T}[g_1,g_2] \right)\le (1+\e)\mu \left(H_T[g_1,g_2] \right).
\end{equation}
\end{corollary}


\begin{corollary}[Limit volume ratios, property D2 in \cite{Gorodnik200a4DistributionOL}]
For any $g_1,g_2 \in G$. the limit 
\begin{equation}\label{eq:alpha}
    \alpha(g_1,g_2):=\lim_{T\to \infty}\frac{\mu \left(H_T[g_1,g_2] \right)}{\mu \left(H_T \right)}=\frac{\det(\emph{\text{g}}_1)^m}{ \det(\emph{\text{g}}_2)}\frac{\sum_{q\in \Delta}\frac{1}{\|\emph{\text{g}}_1^{-1}q\emph{\text{g}}_2\|^{m^2}}}{\sum_{q\in \Delta}\frac{1}{\|q\|^{m^2}}},
\end{equation}
exists and is positive and finite. 
\end{corollary}
Next, we give our ergodic theorem.
For $F\in C_c(G/\Ga)$ and $g_1,g_2\in G$, we consider the measure defined by the integral 
\begin{equation}\label{definition of measure funtional}
    \mu_{T,g_1,g_2}(F):=\frac{1}{\mu \left(H_T[g_1,g_2] \right)}\int_{H_T[g_1,g_2]}F(h^{-1}g_1\Gamma)d\mu(h).
\end{equation}
%The following properties follow immediately from \eqref{measure of H_T}:
%\begin{proof}
%It follows from Proposition \ref{computation for H and V} that 
%\begin{equation}\label{definition and computation of alpha}
 %   \alpha(g_1^{-1},g_2):=\lim_{T\to \infty}\frac{\mu \left(H_T[g_1,g_2] \right)}{\mu \left(H_T \right)}=\frac{1}{G_4^m |\det(H_1)|}\frac{\sum_{q\in \Delta}\frac{1}{A_m(q)^{m^2}}}{\sum_{q\in \Delta}\frac{1}{\|q \|^{m^2}}}.
%\end{equation}
%\end{proof}

\begin{theorem}\label{The $G$-invariance of limiting measure}
Let $\mu_X$ be the normalized $G$-invariant probability measure on $X=G/\Ga$. For all $g_1, g_2\in G$  and all $F\in C_c(X)$
\begin{equation}
    \lim_{T\to \infty} \mu_{T,g_1,g_2}(F)= \int_{X}F(x)d\mu_X(x).
\end{equation}
\end{theorem}

\iffalse
\begin{theorem}\label{equidistribution result on  G mod H}
 For each point $Hg_0\in H\backslash G$, there exists a measure $\nu_{Hg_0}$ (not $G$-invariant) on $H\backslash G$ such that for any compactly supported function $\varphi \in C_c(H\backslash G)$, we have
\begin{equation}
    \lim_{T\to \infty}\frac{1}{\mu(H_T)}\int_{G_T}\varphi(Hg_0g)dg=\int_{H\backslash G} \varphi d\nu_{Hg_0}.
\end{equation}
\end{theorem}

\subsection{Structure of the paper and further work}
\textcolor{red}{TODO:Rephrase}
\begin{itemize}
    \item The outline of our proof of Theorem \ref{The $G$-invariance of limiting measure} is as follows. We will first give an estimate for the volume of $H_T[g_1,g_2]$ (Section 2). This estimate will help us to show the unipotent invariance of limiting measure (Section 3), which in turn opens the gate to the application of Ratner theory. In particular, a dichotomy theorem (Theorem \ref{Shah dichotomy theorem}) due to Shah plays the central role here. To exclude the first outcome of the dichotomy theorem, we use a lemma (Lemma \ref{very important inequaity lemma of nimish-gorodnik}) on the expansion of the norms of vectors under the action of unipotent subgroups. This allows us to prove the non-escape of mass on the space of one-point compactification of $G/\Ga$ and the $G$-invariance will follow from Ratner's measure classification theorem (\cite{Ratner91a}, \cite{Mozes1995OnTS}).
    \item In Section 5 we prove Theorem \ref{equidistribution result on  G mod H} as a consequence of the duality principle stated in Theorem 2.2 to Corollary 2.4 in \cite{Gorodnik2004DistributionOL}. 
    %Moreover, we construct a lift $Y$ of $H\backslash G$ to $G$ and a probability measure on $Y$.
\end{itemize}


\vspace{5mm}
\textcolor{blue}{TODO:write here that our method of proof gives a more general result... sl Z can be replaced with other lattices.}We remark that despite that the theorem is stated from the case of rank-$m$ lattices in $\R^{m+1}$. Our work in progress is to generalize the results to more general classes of moduli spaces of discrete subgroup of the Euclidean space. The main difficulty lies in the estimtate of the volumes of balls as well as the expansion of norms under unipotent actions. 
\textcolor{blue}{TODO:add small section with notations for the asymptotic symbols etc}
\fi

\section*{Notations and conventions}
Throughout this paper, for function $f$ and $g$ on $\R$, by $f(x)=O(g(x))$ or $f(x)\ll g(x)$ we mean there is some $C>0$ such that $|f(x)|\le C |g(x)|$ for sufficiently large $x$; by $f(x)\asymp g(x)$ we mean $f(x)\ll g(x)$ and $g(x) \ll f(x)$; by $f(x)\sim g(x)$ we mean $\lim_{x\to \infty}\left|\frac{f(x)}{g(x)} \right|=1$ or $\lim_{x\to 0}\left|\frac{f(x)}{g(x)}\right|=1$, depending on the context.
 


\section{Volume estimates of skewed balls in $H$}\label{sec:Estimate of the Haar measure growth of skewed balls in H}

The main goal of this section is to prove Proposition \ref{computation for H and V}. In our arguments below we will make use of the following Lemma.
\begin{lemma}\label{lemma on the integral and summation in a ball}
   For an integer $n\ge 1$ and $\sigma\in \R$, when $T\to\infty$,
   \begin{align}
\int_{\SL(n,\R)_T}\|g\|^{\sigma}dg
\sim & \vol(\SL(n,\R)/\De)\int_{\De_T}\|q\|^{\sigma}dq\\
\sim &  \begin{cases}
        C_n\frac{n(n-1)}{n(n-1)+\sigma} T^{n(n-1)+\sigma} & \text{if } \sigma > -n(n-1)\\
        %n(n-1)C_n\log T & \text{if } \sigma=-n(n-1)\\
        I_{n,\sigma}+O(T^{n(n-1)+\sigma}) & \text{if } \sigma < -n(n-1)
    \end{cases},
   \end{align}
where
$I_{n,\sigma}$ and $C_n$ are constants, and $d\gamma$ is the counting measure.
\end{lemma}
\begin{remark}
We note that our method of proof doesn't give explicitly the constants $I_{n,\sigma}$. The constants $C_n$ come from formula A.1.15 in \cite{DRS93}. We also note that our method of proof below also gives the asymptotics for $\sigma=-n(n-1)$, but they will not be used in our paper.   
\end{remark}
\begin{proof}
The case $\sigma=0$ is the estimate that as $T\to\infty$, $\vol(\SL(n,\R)/\De)\cdot \# \De_T\sim\vol(\SL(n,\R)_T)$ and the estimate $\vol(\SL(n,\R)_T)\sim C_n T^{n(n-1)}$, which were obtained in \cite{Gorodnik2009CountingLP} (see formula A.1.15 in \cite{DRS93} for the expression of the constant $C_n$). For $\sigma>0$, by Fubini argument and using A1.15 in \cite{DRS93}, we have
    \begin{align*}
\int_{\SL(n,\R)_T}\|g\|^{\sigma}dg 
&= \int_{\SL(n,\R)_T} \int_0^{\|g\|^{\sigma}}1dtdg\\
&= \int_{\SL(n,\R)_T} \int_0^{\infty}\mathbf{1}_{[t<\|g\|^{\sigma}]}dtdg\\
&=\int_0^{\infty}\int_{\SL(n,\R)_T} \mathbf{1}_{[t<\|g\|^{\sigma}]}dgdt\\
&=\int_0^{T^{\sigma}}\text{Vol}([t^{\frac{1}{\sigma}}<\|g\|<T]) dt\\
&=\int_0^{T^{\sigma}}\text{Vol}([\|g\|<T])dt-\int_0^{T^{\sigma}}\text{Vol}([\|g\|\le t^{\frac{1}{\sigma}}]) dt\\
&=C_n T^{n(n-1)+\sigma}- \int_0^{T^{\sigma}}C_n t^{n(n-1)/\sigma}dt +o(T^{n(n-1)+\sigma})\\
&=C_n\frac{n(n-1)}{n(n-1)+\sigma}T^{n(n-1)+\sigma}+o(T^{n(n-1)+\sigma}).
    \end{align*}
    
%Note that in \cite{DRS93}, $s_n$ is normalized to be $1$, but here we use the volume of $\SO(n,\R)$ under the standard spherical parametrization. 

For $-n(n-1)\ne \sigma<0$,
    \begin{align*}
\int_{\SL(n,\R)_T}\|g\|^{\sigma}dg 
&= \int_{\SL(n,\R)_T} \int_0^{\|g\|^{\sigma}}1dtdg\\
&= \int_{\SL(n,\R)_T} \int_0^{\infty}\mathbf{1}_{[t<\|g\|^{\sigma}]}dtdg\\
&=\int_0^{\infty}\int_{\SL(n,\R)_T} \mathbf{1}_{[t<\|g\|^{\sigma}]}dgdt\\
&=\int_{0}^{\infty}\text{Vol}([\|g\|<\min(t^{\frac{1}{\sigma}},T)]) dt\\
&=\int_{T^{\sigma}}^{\infty}\text{Vol}([\|g\|<t^{\frac{1}{\sigma}}])dt+ \int_0^{T^{\sigma}}\text{Vol}([\|g\|<T])dt\\
&=\int_{T^{\sigma}}^{1}\text{Vol}([\|g\|<t^{\frac{1}{\sigma}}])dt+ \int_0^{T^{\sigma}}\text{Vol}([\|g\|<T])dt \tag{For $g\in \SL(n,\R)$, $\|g\|\ge 1$}\\
&=\int_{T^{\sigma}}^{1}\vol([\|g\|<t^{\frac{1}{\sigma}}])dt +C_n T^{n(n-1)+\sigma}+o(T^{n(n-1)+\sigma})
    \end{align*}
%\\&=I_{n,\sigma}+C_n\frac{n(n-1)}{n(n-1)+\sigma}T^{n(n-1)+\sigma}+o(T^{n(n-1)+\sigma}).
Since $\vol([\|g\|<t^{\frac{1}{\sigma}}])\asymp t^\frac{n(n-1)}{\sigma}$ as $t\to 0$ (recall that here $\sigma<0$), and since $\int_0^1t^\frac{n(n-1)}{\sigma}dt$ converges, we have by dominated convergence that $$\int_0^1\vol([\|g\|<t^{\frac{1}{\sigma}}])dt=I_{n,\sigma}$$ for some (implicit) constant $=I_{n,\sigma}$. Now
\begin{align*}
  \int_{T^{\sigma}}^{1}\vol([\|g\|<t^{\frac{1}{\sigma}}])dt=&I_{n,\sigma}-\int^{T^{\sigma}}_{0}\vol([\|g\|<t^{\frac{1}{\sigma}}])dt\\
  = & I_{n,\sigma}+O\left(\int^{T^{\sigma}}_{0}t^\frac{n(n-1)}{\sigma}dt\right)=I_{n,\sigma}+O(T^{n(n+1)+\sigma}),
\end{align*}
which concludes the proof of the estimate for $\sigma<-n(n-1)$.

The proof for the statement with $\De_T$ is similar --- use the Fubini argument with the counting measure $dq$ instead of $dg$, and then apply the estimate that $\De_\tau\sim C_n\tau^{n(n-1)}$ as $\tau\to\infty$.
\iffalse
In particular, if $\sigma<-n(n-1)$, the integral is finite as $T\to \infty$.
For $\sigma=-n(n-1)$, the final estimate is 
\begin{equation}
    n(n-1)C_n\log T+O(1).
\end{equation}
In particular for $\sigma<-n(n-1)$, we have the tail bound
    \begin{align*}
\int_{\SL(n,\R)-\SL(n,\R)_T}\|g\|^{\sigma}dg 
&= \int_{\SL(n,\R)-\SL(n,\R)_T} \int_0^{\|g\|^{\sigma}}1dtdg\\
&= \int_{\SL(n,\R)-\SL(n,\R)_T} \int_0^{\infty}\mathbf{1}_{[t<\|g\|^{\sigma}]}dtdg\\
&=\int_0^{\infty}\int_{\SL(n,\R)-\SL(n,\R)_T} \mathbf{1}_{[t<\|g\|^{\sigma}]}dgdt\\
&=\int_{0}^{\infty}\text{Vol}([T \le \|g\|<t^{\frac{1}{\sigma}}]) dt\\
&=\int_{0}^{T^{\sigma}}\text{Vol}([T \le \|g\|<t^{\frac{1}{\sigma}}])dt\\
&=\int_0^{T^{\sigma}}[\text{Vol}([\|g\|\le t^{\frac{1}{\sigma}}])-\text{Vol}([\|g\|\le T])]dt\\
&=\int_0^{T^{\sigma}}\text{Vol}([\|g\|\le t^{\frac{1}{\sigma}}])dt -\int_0^{T^{\sigma}}\text{Vol}([\|g\|\le T])dt\\
&=\int_0^{T^{\sigma}}c_ns_nt^{\frac{n(n-1)}{\sigma}}dt-c_ns_nT^{n(n-1)+\sigma}+o(T^{n(n-1)+\sigma})\\
&=\frac{-n(n-1)}{n(n-1)+\sigma}c_ns_n T^{n(n-1)+\sigma}+o(T^{n(n-1)+\sigma})
    \end{align*}
\fi
\end{proof}
For $g_1\in\SL(m+1,\R)$ we write 
\begin{equation}\label{matrix representation of g1 and g2}
    g_1^{-1}:=k_1\begin{bmatrix} G_1 & 0 \\G_3 & G_4 \end{bmatrix}, ~g_2:=\begin{bmatrix} H_1 & 0 \\H_3 & H_4 \end{bmatrix} k_2,
\end{equation}where $k_1,k_2 \in \SO(m+1,\R)$, $G_1,H_1\in \R^{m\times m}$, $G_3, H_3\in \R^{1\times m}$ and $G_4, H_4\in \R$.
We proceed now to study $\mu \left(H_T[g_1,g_2] \right)$, where $ H_T[g_1,g_2]:=\{h\in H: \|g_1^{-1}hg_2\|<T \}.$  Recall that our $H_T[g_1,g_2]$ is the disjoint union \eqref{decomposition of H_T into V_T}, so that
\begin{align}\label{eq:haar measure of skewed H ball as a sum of skewed balls on connected component}
    \mu \left(H_T[g_1,g_2] \right)
    =& \sum_{\substack{q\in \Delta}} \mu \left(V_{q,T}[g_1,g_2]\right),
\end{align}
where we  recall   $V_{q,T}[g_1,g_2]:=\{h\in U A q: \|g_1^{-1}hg_2\|<T \}$. We will now inspect closely $V_{q,T}[g_1,g_2]$. Since the Hilbert-Schmidt norm is bi-$\SO(m+1,\R)$ invariant, we may assume that $k_1$ and $k_2$ in \eqref{matrix representation of g1 and g2} are equal to the identity matrix.

A generic term  $h\in UA\gamma$ is of the form
\begin{equation*}
h:=\begin{bmatrix}
I_2 & 0 \\
v & 1 
\end{bmatrix}
\begin{bmatrix}
t^{-\frac{1}{m}}I_2 & 0 \\
0 & t 
\end{bmatrix}
\begin{bmatrix}
q & 0 \\
0 &  1 
\end{bmatrix}    
=\begin{bmatrix}
t^{-\frac{1}{m}}q & 0 \\
t^{-\frac{1}{m}}vq &  t 
\end{bmatrix}, 
\end{equation*}
where $t\ne 0, q \in \Delta, v\in \R^m$.
\iffalse
By definition $h=\begin{bmatrix}
t^{-\frac{1}{m}}q & 0 \\
t^{-\frac{1}{m}}vq & t 
\end{bmatrix} \in H_T[g_1,g_2]$ if and only if $\|g_1hg_2\| \le T$. 
\fi
Note
\begin{align*}
    g_1hg_2=&\begin{bmatrix} G_1 & 0 \\G_3 & G_4 \end{bmatrix}
    \begin{bmatrix}
t^{-\frac{1}{m}}q & 0 \\
t^{-\frac{1}{m}}vq &  t 
\end{bmatrix}
    \begin{bmatrix} H_1 & 0 \\H_3 & H_4 \end{bmatrix}\\
    =&\begin{bmatrix}
t^{-\frac{1}{m}}G_1q & 0 \\
t^{-\frac{1}{m}}G_3q+t^{-\frac{1}{m}}G_4vq &  G_4t 
\end{bmatrix}\begin{bmatrix} H_1 & 0 \\H_3 & H_4 \end{bmatrix}\\
=&\begin{bmatrix}
t^{-\frac{1}{m}}G_1q H_1 & 0 \\
t^{-\frac{1}{m}}G_3q H_1+t^{-\frac{1}{m}}G_4vq H_1 + G_4 H_3t &  G_4H_4t 
\end{bmatrix}.
\end{align*}
Upon taking the sum of squares and rearranging terms, we conclude that $\|g_1hg_2\| \le T$ is equivalent to
\begin{equation}\label{equation defining the ball of integration}
    \|G_3q H_1+G_4 vq H_1+G_4 H_3 t^{\frac{1}{m}+1}\|^2 
    \le -|G_4 H_4|^2 t^{\frac{2}{m}+2}+t^{\frac{2}{m}}T^2-\|G_1 q H_1\|^2
\end{equation}

In view of \eqref{equation defining the ball of integration}, let 
\begin{equation}\label{the notation D for the ball}
    D_{q, T,t}:=\{v\in \R^{m}:     \|G_3q H_1+G_4 vq H_1+G_4 H_3 t^{\frac{1}{m}+1}\|^2 
    \le -B_1^2 t^{\frac{2}{m}+2}+t^{\frac{2}{m}}T^2-A_m(q)^2\},
\end{equation}
where
\begin{equation} \label{eq:def of Am}
   A_m(q)=\|G_1 q H_1\| 
\end{equation}
and
\begin{equation}\label{eq:definition of B1}
  B_1=|G_4 H_4|.  
\end{equation}
Recall the expression \eqref{eq:Haar measure on $H$} for the left Haar measure on $H$, which gives
\begin{equation}
    \mu \left(V_{q,T}[g_1,g_2]\right):=\int_0^{\infty}\int_{\R^m} \mathbf{1}_{V_{q,T}[g_1,g_2]}(u_va_tq) dv \frac{1}{t^{m+2}}dt=\int_0^{\infty}\vol(D_{q,T,t})\frac{1}{t^{m+2}}dt,
\end{equation}
where $\vol(D_{q,T,t})$ is the Lebesgue measure of $D_{q,T,t}$. 
We  observe that $D_{q,T,t}$ is the interior of an ellipse in $\R^m$, whose area is 
\begin{equation}\label{eq:volume of D_gamma,t}
    \vol(D_{q,T,t})= \begin{cases}
        v_{m}\frac{\left( -B_1^2 t^{\frac{2}{m}+2}+t^{\frac{2}{m}}T^2-A_m(q)^2 \right)^{\frac{m}{2}} }{|G_4|^m |\det(H_1)|} & \text{if } -B_1^2 t^{\frac{2}{m}+2}+t^{\frac{2}{m}}T^2-A_m(q)^2>0 \\
        0 & \text{otherwise.}
    \end{cases}  
\end{equation}
Here $v_m=\frac{\pi^{\frac{m}{2}}}{\Ga(\frac{m}{2}+1)}$ denotes the volume of Euclidean ball of radius one in $\R^m$. We now inspect more closely \eqref{eq:volume of D_gamma,t} with the goal to shed light on how it changes as we vary $\gamma$ and $T$. Roughly, we will show that when $A_m(q)$ is ``large" with respect to $T$, then the volume of the ellipse is ``small".
 %where $C_{\alpha}(g_1,g_2)\in [1,2]$ (depending on how much overlap two balls may have) is a constant related to $g_1,g_2$. Observe that when $H_3=0$, we have two ellipses merge into one, and $C_{\alpha}(g_1,g_2)=0$.

%Note if $g_1=g_2=e$, then $G_1=H_1=I_2$, $G_3=H_3=0$, $G_4=H_4=1$ and the equation above reduces to
%\begin{equation}
%    \|vq\|^2\le -t^{\frac{2}{m}+2}+t^{\frac{2}{m}}T^2-\|q\|^2.
%\end{equation}
%In this case, $C_{\alpha}(e,e)=1$.


 First, observe that the maximal value of $-B_1^2 t^{\frac{2}{m}+2}+t^{\frac{2}{m}}T^2$ for $t>0$ is \begin{equation}\label{eq:maximal value}
     M_{T}:=\frac{m}{m+1}\left(\frac{1}{m+1}\right)^{\frac{1}{m}} \frac{T^{2+\frac{2}{m}}}{B_1^{\frac{2}{m}}},
 \end{equation} which is attained at
 \begin{equation}\label{eq:critical value where M_T is attained}
     \theta(T):=\frac{1}{\sqrt{1+m}} \frac{T}{B_1}.
 \end{equation}Then, according to \eqref{eq:volume of D_gamma,t}, we get that $m(D_{\gamma,T,t})\neq0$ only if 
\begin{equation}\label{relationship between Am and T}
    A_m(q)<\sqrt{M_T}\asymp \frac{T^{1+\frac{1}{m}}}{B_1}.
\end{equation} Next, observe that  $f(t)=-B_1^2 t^{\frac{2}{m}+2}+t^{\frac{2}{m}}T^2-A_m(q)^2$ is monotonic increasing in $(0,\theta(T))$ and monotonic decreasing in $(\theta(T),\infty)$. Also, note that $f(0)<0$ and that $f(\sqrt{m+1}\theta(T))<0$. Then, whenever $A_m(q)^2< M_T$, it is easy to see that $-B_1^2 t^{\frac{2}{m}+2}+t^{\frac{2}{m}}T^2-A_m(q)^2$ has two positive roots, which we denote by $\al_{q,T}$ and $\be_{q,T}$, and the following bounds hold 
 \begin{equation}\label{bounds on roots}
   0<\al_{q,T}<\theta(T)<\be_{q,T}<\sqrt{m+1} \theta(T), 
 \end{equation}whenever $A_m(q)<\sqrt{M_T}$.
In particular, \begin{equation}\label{eq:volume of D_ga,T using the roots}\vol(D_{q,T,t})= \begin{cases}
        v_{m}\frac{\left( -B_1^2 t^{\frac{2}{m}+2}+t^{\frac{2}{m}}T^2-A_m(q)^2 \right)^{\frac{m}{2}} }{|G_4|^m |\det(H_1)|} & \text{if } t\in (\al_{q,t},\be_{q,t}) \\
        0 & \text{otherwise,}
    \end{cases} 
\end{equation}
and by recalling \eqref{eq:haar measure of skewed H ball as a sum of skewed balls on connected component}, we conclude that 
\begin{equation}\label{eq:measure of V skewed balls as integral}
  \mu \left(V_{q,T}[g_1,g_2] \right)
   = \frac{v_m}{|G_4|^m |\det(H_1)|}\int_{\al_{\gamma,T}}^{\be_{\gamma,T}} \frac{(-B_1^2 t^{\frac{2}{m}+2}+t^{\frac{2}{m}}T^2-A_m(q)^2)^{\frac{m}{2}}}{t^{m+2}}dt. 
\end{equation}
The following gives more precise bounds for the root $\al_{\gamma,T}$ as $q$ and $T$ vary. By rearranging terms in $$-B_1^2 \al_{q,T}^{\frac{2}{m}+2}+\al_{q,T}^{\frac{2}{m}}T^2-A_m(q)^2=0,$$  we get that
\begin{equation}\label{eq:recursive form of a}
   \al_{q,T}^{\frac{2}{m}}=\frac{A_m(q)^2}{T^2-B_1^2 \al_{q,T}^2}, 
\end{equation}
and by using that $\al_{q,T}<\theta(T)$, see \eqref{bounds on roots}, we conclude that
\begin{equation}\label{range for a}
    \al_{q,T}^{\frac{2}{m}}=\frac{A_m(q)^2}{T^2-B_1^2 \al_{q,T}^2}\in
    \left(\frac{A_m(q)^2}{T^2}, \frac{(m+1)A_m(q)^2}{mT^2}\right).
\end{equation} 
In particular, it follows that
\begin{equation}
        \al_{q,T}\asymp \frac{A_m(q)^m}{T^m}, \label{estimate for a in all range}
\end{equation}
and moreover, by \eqref{eq:recursive form of a} and \eqref{range for a}, \begin{equation}\label{eq:asymp for a when A small then sqrt T}
    \al_{q,T}^{\frac{2}{m}}=\frac{A_m(q)^2}{T^2}+O\left(\frac{1}{T^{m+1.5}}\right),~\text{when }A_m(q)\leq\sqrt{T}
\end{equation}
\begin{lemma}\label{lem:estimate of V skewed balls as power of T and fract of A_m to power m^2}It holds that
\begin{align}
    \mu(V_{q,T}[g_1,g_2])
    \le & \frac{v_m}{|G_4|^m |\det(H_1)|}\frac{T^{m(m+1)}}{\|A_m(q)\|^{m^2}}\label{eq:bound on measure of Vga T} \\
    \leq& \frac{v_m\|G_1^{-1}\|^{m^2}\|H_1^{-1}\|^{m^2}}{|G_4|^m |\det(H_1)|}\frac{T^{m(m+1)}}{\|q\|^{m^2}}.\label{eq:bound on volume of connected component}
\end{align}
     In particular, when $g_1,g_2$ vary in a compact subset of $\SL(m+1,\R)$, it holds that
    $$\mu(V_{q,T}[g_1,g_2])\leq T^{m(m+1)}O\left(\frac{1}{\|q\|^{m^2}}\right).$$
\end{lemma}
\begin{proof}
   We have
   \begin{align}
      &\int_{\al_{q,T}}^{\be_{q,T}} \frac{(-B_1^2t^{\frac{2}{m}+2}+t^{\frac{2}{m}}T^2-A_m(q)^2)^{\frac{m}{2}}}{t^{m+2}}dt\\
    \le &\int_{\al_{q,T}}^{\be_{q,T}} \frac{(t^{\frac{2}{m}}T^2)^{\frac{m}{2}}}{t^{m+2}}dt\\
    =& T^m\int_{\al_{q,T}}^{\be_{q,T}}\frac{1}{t^{m+1}}\\ 
    \leq& T^m\frac{1}{m\al_{q,T}^m}.
   \end{align}
Using \eqref{range for a}, we get 
$$\frac{1}{\al_{q,T}^m}\leq\frac{T^{m^2}}{A_m(q)^{m^2}}\leq T^{m^2}\frac{\|G_1^{-1}\|^{m^2}\|H_1^{-1}\|^{m^2}}{\|\gamma\|^{m^2}},$$
which proves the claim upon recalling \eqref{eq:measure of V skewed balls as integral}. 
\end{proof}
We now return to $\mu \left(H_T[g_1,g_2] \right)$. We conclude by \eqref{eq:measure of H skewed balls as sums of integrals} that 
\begin{equation}\label{eq:measure of H skewed balls as sums of integrals}
  \mu \left(H_T[g_1,g_2] \right)
   =\sum_{\substack{q\in \Delta,\\ A_m(q)< \sqrt{M_T}}} \frac{v_m}{|G_4|^m |\det(H_1)|}\int_{\al_{q,T}}^{\be_{q,T}} \frac{(-B_1^2 t^{\frac{2}{m}+2}+t^{\frac{2}{m}}T^2-A_m(q)^2)^{\frac{m}{2}}}{t^{m+2}}dt, 
\end{equation}
where $M_T$ is given by \eqref{eq:maximal value}. We split the sum into two parts as 
\begin{equation}\label{eq:truncation of A_m<sqrt T}
  \left(\sum_{\substack{q\in \Delta,\\ A_m(q)< \sqrt{T}}}+\sum_{\substack{q\in \Delta,\\ \sqrt{T}<A_m(q)< \sqrt{M_T}}} \right) \frac{v_m}{|G_4|^m |\det(H_1)|}\int_{\al_{q,T}}^{\be_{q,T}} \frac{(-B_1^2 t^{\frac{2}{m}+2}+t^{\frac{2}{m}}T^2-A_m(q)^2)^{\frac{m}{2}}}{t^{m+2}}dt. 
\end{equation}

We will now estimate the second sum. The second sum is relatively easier to estimate, and it will eventually follow that it is of lower order in $T$ compared to the first sum.
\begin{lemma}\label{lem:bound on skewed V vol}
    If $g_1,g_2$ vary in a bounded set, then
    \begin{equation}
        \sum_{\substack{q\in \Delta,\\ \sqrt{T}<A_m(q)< \sqrt{M_T}}}  \frac{v_m}{|G_4|^m |\det(H_1)|}\int_{\al_{q,T}}^{\be_{q,T}} \frac{(-B_1^2 t^{\frac{2}{m}+2}+t^{\frac{2}{m}}T^2-A_m(q)^2)^{\frac{m}{2}}}{t^{m+2}}dt=O(T^{m(m+1)-m/2})
    \end{equation}
\end{lemma}
\begin{proof}
In the following  $g_1,g_2$ vary in a bounded set. By Lemma \ref{lem:estimate of V skewed balls as power of T and fract of A_m to power m^2}, 
 \begin{align}
    &\sum_{\substack{q\in \Delta,\\ \sqrt{T}<A_m(q)< \sqrt{M_T}}}\int_{\al_{q,T}}^{\be_{q,T}} \frac{(-B_1^2t^{\frac{2}{m}+2}+t^{\frac{2}{m}}T^2-A_m(q)^2)^{\frac{m}{2}}}{t^{m+2}}dt\\
    \le &  T^{m(m+1)}O\left(\sum_{\substack{q\in \Delta,\\ \sqrt{T}<A_m(q)}} \frac{1}{\|q\|^{m^2}}\right).
\end{align} 
We have $$A_m(q)=\|G_1\gamma H_1\|\leq\|G_1\|\|\gamma\|\|H_1\|,$$ where $\|G_1\|,\|H_1\|$ are bounded above since they depend continuously on $g_1,g_2$. Then, for some $C>0$, we conclude that $$O\left(\sum_{\substack{q\in \Delta,\\ \sqrt{T}<A_m(q)}} \frac{1}{\|q\|^{m^2}}\right)=O\left(\sum_{\substack{q\in \Delta,\\ C\sqrt{T}<\|q\|}} \frac{1}{\|q\|^{m^2}}\right).$$
By Lemma \ref{lemma on the integral and summation in a ball} with $n=m$ and $\sigma=-m^2$, it follows that 
$\sum_{\substack{q\in \Delta}} \frac{1}{\|q\|^{m^2}}$ converges
with the tail estimate 
$$\sum_{\substack{q\in \Delta,\\ \sqrt{T}<\|q\|}} \frac{1}{\|q\|^{m^2}}=O(\sqrt{T}^{-m}),$$
which proves our claim.
\end{proof}


\vspace{5mm}
We now proceed to treat the first part of the sum \eqref{eq:truncation of A_m<sqrt T}. Namely, in the following, assume that $A_m(q)\leq\sqrt{T}.$ 

\iffalse
Meanwhile by Taylor expansion,
\begin{align}
    a=&\frac{A_m(q)^m}{T^m}\left(\frac{1}{1-\frac{G_4^2H_4^2 a^2}{T^2}}\right)^{\frac{m}{2}} \nonumber\\ 
    =&\frac{A_m(q)^m}{T^m}\left(1+\frac{B_1^2 a^2}{T^2}+\cdots \right)^{\frac{m}{2}} \label{taylor expansion for a}
\end{align}

Moreover, 
%for the $q$ with smaller norms, 
when $A_m(q)\leq T$ we have that it follows from \eqref{taylor expansion for a} that
\begin{equation}\label{estimate for a in smaller range taylor expansion}
    a=\frac{A_m(q)^m}{T^m}\left(1+B_1^2 O_{g_1,g_2}\left(\frac{1}{T^2} \right)\right)^{\frac{m}{2}}, \text{ whenever }A_m(q)\le T.
\end{equation}
\fi
For $\e_1$ we define $\de=\de(T,q,\e_1)$ by
\begin{equation}\label{eq:definition of delta}
  \de:=\frac{A_m(q)}{T^{1+\e_1}},\end{equation} we define $\al_\de=\al_\de(q,T,\e_1)$ by\begin{equation}\label{eq:defintion of a_delta}
    \al_\de^{\frac{1}{m}}:= \al_{q,T}^{\frac{1}{m}}+\de,
\end{equation}and for $\e_2\in(0,1)$ we let $\la=\la(q,T,\e_2)$ 
\begin{equation}\label{eq:definition of c}
    \la:=\left(\frac{A_m(q)}{T^{\e_2}}\right)^{\frac{m}{m+1}}.
\end{equation}We consider the following partition of the integral appearing in the terms of \eqref{eq:truncation of A_m<sqrt T},
\begin{align}\label{eq:partition of integral into four part}
&\int_{\al_{q,T}}^{\be_{q,T}} \frac{(-B_1^2 t^{\frac{2}{m}+2}+t^{\frac{2}{m}}T^2-A_m(q)^2)^{\frac{m}{2}}}{t^{m+2}}dt\\
=&\left(\int_{\al_{q,T}}^{\al_{\de}(q,T,\e_1)}+\int_{\al_{\de}(q,T,\e_1)}^{\la(q,T,\e_2)}+\int_{\la(q,T,\e_2)}^{\be_{q,T}}\right) \frac{(-B_1^2 t^{\frac{2}{m}+2}+t^{\frac{2}{m}}T^2-A_m(q)^2)^{\frac{m}{2}}}{t^{m+2}}dt
\end{align}
Our key point in the computation below will be that the main term among the three integrals above is the integral in the range $\int_{\al_{\de}(q,T,\e_1)}^{\la(q,T,\e_2)}$. For the following, we note that 
\begin{equation}\label{eq:lower bound on A_m}
\|G_1^{-1}\|^{-1}\|H_1^{-1}\|^{-1}\|q\|\leq A_m(q),
\end{equation}
and we note that $\|G_1^{-1}\|^{-1},~\|H_1^{-1}\|^{-1}$ are bounded from below when $g_1,g_2$ vary in a bounded set.
\begin{lemma} \label{lemma for integral from a to a delta}
    Suppose that $g_1,g_2$ vary in a bounded subset of $\SL(m+1,\R)$ and fix $\e_1\in(0,1)$. Then
    \begin{align}
        \int_{\al_{q,T}}^{\al_{\de}(q,T,\e_1)} \frac{(-B_1^2 t^{\frac{2}{m}+2}+t^{\frac{2}{m}}T^2-A_m(q)^2)^{\frac{m}{2}}}{t^{m+2}}dt =O\left(\frac{T^{m(m+1)-\e_1\frac{m+2}{2}}}{\|q\|^{m^2}}\right)   
    \end{align}
\end{lemma}    
\begin{proof}
By substituting the variable $s=t^{\frac{1}{m}}$ in $-B_1^2 t^{\frac{2}{m}+2}+t^{\frac{2}{m}}T^2-A_m(q)^2$, the expression becomes
\begin{align}
    f(s):=-B_1^2 s^{2m+2}+T^2s^2-A_m(q)^2.
\end{align}
Recall $$\al_\de(q,T,\e_1)^\frac{1}{m}:=\al_{q,T}^{\frac{1}{m}}+\delta(q,T,\e_1)=\al_{q,T}^{\frac{1}{m}}+\frac{A_m(q)}{T^{1+\e_1}},$$ and note that  $f(\al_{q,T}^{\frac{1}{m}})=0.$ In the following, to ease the reading, we will omit from the notations the dependencies on $q$, $T$ and $\e_1$. We apply Taylor expansion to $f(s)$ in the range $s\in[\al^{\frac{1}{m}},\al_\de^{\frac{1}{m}}]$
at $s=\al^{\frac{1}{m}}$ which gives
\begin{align}
    f(\al_\de^\frac{1}{m})=& [-(2m+2)B_1^2 \al^\frac{2m+1}{m}+2T^2\al^\frac{1}{m}]\de+\frac{f''(\xi)}{2}\de^2\nonumber\\
    \ll&T^2\al^\frac{1}{m}\de+f''(\xi)\de^2\label{eq:bound by taylor}
\end{align}
where $\xi\in [\al^\frac{1}{m},\al_\de^{\frac{1}{m}}]$. When $g_1,g_2$ are bounded, we get that $B_1$ is bounded above and away from zero. Then the second derivative $f''(\xi)$ over $[\al^\frac{1}{m},\al_\de^{\frac{1}{m}}]$ will be bounded as
\begin{align}
    |f''(\xi)|
    =&|-(2m+2)(2m+1)B_1^2 \xi^{2m}+2T^2|\nonumber\\
    \le & (2m+2)(2m+1)B_1^2 \al_\de^\frac{2m}{m}+2T^2\nonumber \\
    \ll & T^2.
\label{eq:bound on sectond der}\end{align} We note that for all large $T$ (independently of $q$ and $\e_1$) $f$ is monotonically increasing on the interval $(\al^\frac{1}{m},\al_\de^{\frac{1}{m}})$. To see this, recall that $\theta(T)^\frac{1}{m}\sim T^\frac{1}{m}$ is a critical point for $f(s)$,  $f(s)$ is monotonically increasing in $(0,\theta(T)^\frac{1}{m})$, and $\al_\de=o(1)$. Then

\begin{align*}
    &\int_{\al}^{\al_{\de}} \frac{(-B_1^2 t^{\frac{2}{m}+2}+t^{\frac{2}{m}}T^2-A_m(q)^2)^{\frac{m}{2}}}{t^{m+2}}dt\\
    \le&\frac{1}{\al^{m+2}}\int_{\al}^{\al_{\de}} (-B_1^2 t^{\frac{2}{m}+2}+t^{\frac{2}{m}}T^2-A_m(q)^2)^{\frac{m}{2}}dt\\
    =&\frac{1}{\al^{m+2}}\int_{\al^\frac{1}{m}}^{\al_{\de}^\frac{1}{m}} f(s)^{\frac{m}{2}}ms^{m-1}ds\\
    \ll&\frac{1}{\al^{m+2}}f(\al_\de^{\frac{1}{m}})^\frac{m}{2}\al_\de^{\frac{m-1}{m}}\de\tag{monotonicity of $f(s)$}\\
    \ll&\frac{1}{\al^{m+1+\frac{1}{m}}}f(\al_\de^{\frac{1}{m}})^\frac{m}{2}\de\tag{$\al_\de\ll\al$}
    \\
    \ll&\frac{\left\{T^2\al^{\frac{1}{m}}\de+T^2\de^2\right\}^{\frac{m}{2}}}{\al^{m+1+\frac{1}{m}}}\cdot\delta\tag{by \eqref{eq:bound by taylor} and \eqref{eq:bound on sectond der}}\\
    =&\frac{\left\{T^2\al_\de^{\frac{1}{m}}\right\}^{\frac{m}{2}}}{\al^{m+1+\frac{1}{m}}}\cdot\delta^{\frac{m+2}{2}}\\
     \ll& \frac{\left\{T^2\al^{\frac{1}{m}}\right\}^{\frac{m}{2}}}{\al^{m+1+\frac{1}{m}}}\cdot\delta^{\frac{m+2}{2}} 
    \end{align*}
Using the bounds of $\al$ in \eqref{range for a} and by applying the definition of $\delta$ in \eqref{eq:definition of delta}, we get\begin{equation}
\frac{\left\{T^2\al^{\frac{1}{m}}\right\}^{\frac{m}{2}}}{\al^{m+1+\frac{1}{m}}}\cdot\delta^{\frac{m+2}{2}}\ll\frac{T^{m(m+1)-\e_1\frac{m+2}{2}}}{A_m(q)^{m^2}},  
\end{equation}
which concludes our proof.    
\end{proof}


We proceed to treat the middle integral, which will be the main term in \eqref{eq:partition of integral into four part}. 
\begin{lemma}\label{lem:integral from a_delta to c}
Suppose that $A_m(q)\leq\sqrt{T}$, and $g_1,g_2$ vary in a bounded set of $\SL(m+1,\R)$. Suppose that $0<\e_1<2\e_2<1$, and let \begin{equation}
\e:=\min\{\e_1,2\e_2-\e_1,1+\frac{1-2\e_2}{m+1}\}.
\end{equation}Then \begin{align}
   &\int_{\al_{\de}(q,T,\e_1)}^{\la(q,T,\e_2)}\frac{(-B_1^2 t^{\frac{2}{m}+2}+t^{\frac{2}{m}}T^2-A_m(q)^2)^{\frac{m}{2}}}{t^{m+2}}dt\nonumber\\
=&\frac{m\Ga(\frac{m}{2}+1)\Ga(\frac{m^2}{2})}{2\Ga(\frac{m^2}{2}+\frac{m}{2}+1)}
\cdot \frac{T^{m(m+1)}}{A_m(q)^{m^2}}+O\left(\frac{T^{m(m+1)-\e}}{\|q\|^{m^2}}\right). \label{eq:main term estimate for integral in a delta to c}
\end{align}
\end{lemma}
\begin{proof}
Suppose that $A_m(q)\leq\sqrt{T}$, and assume that $g_1,g_2$ vary in a bounded set. In particular, recall that $B_1$ will be bounded above. To ease the reading, we will omit from some notations the dependencies on $q$, $T$ and $\e_1,\e_2$. The dependencies should be clear from the context. We first rewrite our integral as 
$$ \int_{\al_{\de}}^{\la}\frac{(-B_1^2 t^{\frac{2}{m}+2}+t^{\frac{2}{m}}T^2-A_m(q)^2)^{\frac{m}{2}}}{t^{m+2}}dt=\int_{\al_\de}^\la \frac{(t^{\frac{2}{m}}T^2-A_m(q)^2)^{\frac{m}{2}}}{t^{m+2}}
\left(1-\frac{B_1^2t^{\frac{2}{m}+2}}{t^{\frac{2}{m}}T^2-A_m(q)^2} \right)^{\frac{m}{2}}dt.$$We now show that in the range $t\in(\al_\de,\la)$ it holds that 
\begin{equation}\label{much less than 1}
   \frac{B_1^2t^{\frac{2}{m}+2}}{t^{\frac{2}{m}}T^2-A_m(q)^2}= O\left(\frac{1}{T^{2\e_2-\e_1}}\right),
\end{equation}
Indeed, 
\begin{align*}
    \frac{B_1^2t^{\frac{2}{m}+2}}{t^{\frac{2}{m}}T^2-A_m(q)^2}
    \ll & \frac{\la^{2+\frac{2}{m}}}{\al_\de^\frac{2}{m}T^2}\\
    = &  \frac{\la^{2\frac{m+1}{m}}}{(\al^\frac{1}{m}+\de)^2T^2} \tag{definition of $\al_\de$ \eqref{eq:defintion of a_delta}} \\
    \ll &\frac{\la^{2\frac{m+1}{m}}}{\al^\frac{1}{m}\de T^2} \\
    \leq & \frac{(\nicefrac{A_m(q)}{T^{\e_2}})^2}{ \frac{A_m(q)}{T}\frac{A_m(q)}{T^{1+\e_1}}T^2}=O\left(\frac{1}{T^{2\e_2-\e_1}}\right)\tag{see \eqref{eq:definition of c}, \eqref{range for a} and \eqref{eq:definition of delta}}.
\end{align*} 

Now in view of the estimate $(1+x)^{\alpha}\sim 1+\alpha x$ as $x\to0$, we conclude that 
\begin{align}
 \int_{\al_\de}^\la \frac{(t^{\frac{2}{m}}T^2-A_m(q)^2)^{\frac{m}{2}}}{t^{m+2}}
\left(1-\frac{B_1^2t^{\frac{2}{m}+2}}{t^{\frac{2}{m}}T^2-A_m(q)^2} \right)^{\frac{m}{2}}dt\nonumber\\=\left(\int_{\al_\de}^\la \frac{(t^{\frac{2}{m}}T^2-A_m(q)^2)^{\frac{m}{2}}}{t^{m+2}}
dt\right)& \left(1+O\left(\frac{1}{T^{2\e_2-\e_1}}\right)\right). \label{eq:estimate for the fraction in the integral a_de to c}  
\end{align}
We shall now estimate $\int_{\al_\de}^\la \frac{(t^{\frac{2}{m}}T^2-A_m(q)^2)^{\frac{m}{2}}}{t^{m+2}}
dt$. To this end, we use the substitute $u=\frac{A_m(q)^2}{T^2t^{\frac{2}{m}}}$ (or conversely $t=\frac{A_m(q)^m}{T^m u^{\frac{m}{2}}}$), 
so that
\begin{align}
&\int_{\al_\de}^\la \frac{(t^{\frac{2}{m}}T^2-A_m(q)^2)^{\frac{m}{2}}}{t^{m+2}}dt\\
=&\int_{\al_\de}^\la\frac{T^m}{t^{m+1}}\left(1-\frac{A_m(q)^2}{T^2 t^{\frac{2}{m}}}\right)^{\frac{m}{2}}dt \\
=&\frac{m}{2} \frac{T^{m^2+m}}{A_m(q)^{m^2}}\int^{\nicefrac{A_m(q)^2}{T^2 \al_{\de}^{\frac{2}{m}}}}_{\nicefrac{A_m(q)^2}{T^2 \la^{\frac{2}{m}}}}
u^{\frac{m^2}{2}-1}(1-u)^{\frac{m}{2}}du. \label{upper and lower limits of the integral}
\end{align} We will now prove that 
\begin{align}
 \int^{\nicefrac{A_m(q)^2}{T^2 \al_{\de}^{\frac{2}{m}}}}_{\nicefrac{A_m(q)^2}{T^2 \la^{\frac{2}{m}}}}
u^{\frac{m^2}{2}-1}(1-u)^{\frac{m}{2}}du=&\int^{1}_{0}
u^{\frac{m^2}{2}-1}(1-u)^{\frac{m}{2}}du+O\left(\frac{1}{T^{\e_1}}\right)+O\left(\frac{1}{T^{1+\frac{1-2\e_2}{m+1}}}\right)\nonumber\\
=&\frac{m\Ga(\frac{m}{2}+1)\Ga(\frac{m^2}{2})}{2\Ga(\frac{m^2}{2}+\frac{m}{2}+1)}+O\left(\frac{1}{T^{\e}}\right)
.\label{eq:estimate of the diffence for the beta integral}
\end{align} where in the last equality we used the classical formula for the beta function. Notice that by \eqref{eq:estimate for the fraction in the integral a_de to c}, by \eqref{upper and lower limits of the integral} and by \eqref{eq:estimate of the diffence for the beta integral} the proof is complete.

To prove \eqref{eq:estimate of the diffence for the beta integral} it suffices to estimate the differences between the endpoints of the corresponding integral, namely it suffices to estimate $\left|{1}-{\frac{A_m(q)^2}{T^2 \al_{\de}^{\frac{2}{m}}}}\right|$ and $\left|\frac{A_m(q)^2}{T^2\la^{\frac{2}{m}}}-0\right|=\frac{A_m(q)^2}{T^2\la^{\frac{2}{m}}}.$ 
We have
\begin{align*}
\left|{1}-{\frac{A_m(q)^2}{T^2 \al_{\de}^{\frac{2}{m}}}}\right|=&\left|\frac{T^2\al_\de^\frac{2}{m}-A_m(q)^2}{T^2\al_\de^\frac{2}{m}}\right|\\
= &\left|\frac{(T^2\alpha^\frac{2}{m}-A_m(q)^2)+(2\al^\frac{1}{m}\de+\de^2)T^2}{T^2\al_\de^\frac{2}{m}}\right|\tag{using definition of $\al_\de$, see \eqref{eq:defintion of a_delta}}\\
= &\frac{\left|\nicefrac{1}{T^{m-0.5}}+(2\al^\frac{1}{m}\de+\de^2)T^2\right|}{T^2\al_\de^\frac{2}{m}}\tag{see  \eqref{eq:asymp for a when A small then sqrt T}}\\
\leq & \frac{\left|\nicefrac{1}{T^{m-0.5}}+(2\al^\frac{1}{m}\de+\de^2)T^2\right|}{\left|T^2\al^\frac{2}{m}\right|}\\
= & O\left(\frac{1}{T^{\e_1}}\right)\tag{we have $T^2\al^\frac{2}{m}=O(1)$ and $(2\al^\frac{1}{m}\de+\de^2)T^2=O(\frac{1}{T^{\e_1}})$},   
\end{align*}
\iffalse
\begin{align}
\left|{1}-{\frac{A_m(q)^2}{T^2 \al_{\de}^{\frac{2}{m}}}}\right|=&\left|{1}-\frac{A_m(q)^2}{T^2a^{\frac{2}{m}}(1+a^{-\frac{1}{m}}\de)^2}\right|\\
= &\left|{1}-\frac{A_m(q)^2}{T^2\left(\frac{A_m(q)^2}{T^2}+O(\frac{1}{T^{m+1.5}})\right)(1+a^{-\frac{1}{m}}\de)^2}\right|\\
= & \left|{1}-\frac{A_m(q)^2}{T^2\left(\frac{A_m(q)^2}{T^2}+O(\frac{1}{T^{m+1.5}})\right)}(1+O(a^{-\frac{1}{m}}\de))\right|\\
= & \left|{1}-\left(1+O(\frac{1}{A_m(q)^2T^{m-0.5}})\right)\left(1+O(\frac{T}{A_m(q)}\frac{A_m(q)}{T^{1+\e_1}})\right)\right|\\
= & O\left(\frac{1}{T^{\e_1}}\right).   
\end{align}
\fi
and finally
\begin{equation}
  \frac{A_m(q)^2}{T^2\la^{\frac{2}{m}}}=\frac{A_m(q)^2}{T^2\left(\nicefrac{A_m(q)}{T^{\e_2}}\right)^{\frac {2}{m+1}}}=\frac{A_m(q)^{2-\frac{2}{m+1}}}{T^{2-\frac{2\e_2}{m+1}}}\underbrace{=}_{A_m(q)\leq\sqrt{T}}O\left(\frac{1}{T^{1+\frac{1-2\e_2}{m+1}}}\right).  
\end{equation}
\end{proof}


For the last integral, we have the following.
\begin{lemma}\label{lem:integral from c to b}
 Suppose that $g_1,g_2$ vary in a bounded set and fix $\e_2\in(0,1)$. Then \begin{equation}
\int_{\la(q,T,\e_2)}^{\be_{q,T}} \frac{(-B_1^2 t^{\frac{2}{m}+2}+t^{\frac{2}{m}}T^2-A_m(q)^2)^{\frac{m}{2}}}{t^{m+2}}dt=T^{m(m+1)-m^2(1-\frac{\e_2 }{m+1})}O\left(\frac{1}{\|q\|^{\frac{m^2}{m+1}}}\right).      
 \end{equation}
\end{lemma}
\begin{proof}
Again, to ease the reading, we will omit from the notations the dependencies on $q$,$T$ and $\e_2$. 
We have
    \begin{align}
&\int_\la^\be \frac{(-B_1^2 t^{\frac{2}{m}+2}+t^{\frac{2}{m}}T^2-A_m(q)^2)^{\frac{m}{2}}}{t^{m+2}}dt\\
\le &\int_\la^\be \frac{(t^{\frac{2}{m}}T^2)^{\frac{m}{2}}}{t^{m+2}} dt =\int_\la^\be \frac{T^m}{t^{m+1}}dt \\
\le & \frac{T^m}{m\la^m}=\frac{T^m}{m(A_m(q)/T^{\e_2})^{\frac{m^2}{m+1}}}=\frac{T^{m+\frac{\e_2 m^2}{m+1}}}{mA_m(q)^{\frac{m^2}{m+1}}}
\end{align} 
\end{proof}

\begin{proof}[Proof of Proposition \ref{computation for H and V}]
 We first conclude the estimate for the volume of $V_{q,T}[g_1,g_2]$. By combining  Lemmata \ref{lemma for integral from a to a delta} - \ref{lem:integral from c to b}, we conclude that there exists $\kappa>0$ and $C_q>0$ (which can be explicitly determined by optimizing $\e_1$ and $\e_2$) such that whenever $g_1,g_2$ vary in a bounded set, we have
 \begin{align}
    \mu(V_{q,T}[g_1,g_2]) \underbrace{=}_{\eqref{eq:measure of V skewed balls as integral}} &\frac{v_m}{|G_4|^m |\det(H_1)|}\int_{\al_{q,T}}^{\be_{q,T}} \frac{(-B_1^2 t^{\frac{2}{m}+2}+t^{\frac{2}{m}}T^2-A_m(q)^2)^{\frac{m}{2}}}{t^{m+2}}dt\nonumber\\
=\left(\int_{\al_{q,T}}^{\al_{\de}(q,T,\e_1)}+\int_{\al_{\de}(q,T,\e_1)}^{\la(q,T,\e_2)}+\int_{\la(q,T,\e_2)}^{\be_{q,T}}\right)& \frac{v_m}{|G_4|^m |\det(H_1)|}\int_{\al_{q,T}}^{\be_{q,T}} \frac{(-B_1^2 t^{\frac{2}{m}+2}+t^{\frac{2}{m}}T^2-A_m(q)^2)^{\frac{m}{2}}}{t^{m+2}}dt\nonumber\\
\underbrace{=}_{\text{Lemmata }\ref{lemma for integral from a to a delta}-\ref{lem:integral from c to b}}\frac{v_m}{|G_4|^m |\det(H_1)|}&\frac{m\Ga(\frac{m}{2}+1)\Ga(\frac{m^2}{2})}{2\Ga(\frac{m^2}{2}+\frac{m}{2}+1)}\frac{T^{m(m+1)}}{A_m(\gamma)^{m^2}}+C_\gamma O(T^{m(m+1)-\kappa})\nonumber.
 \end{align}   
Now we compute the total volume $\mu \left(H_T[g_1,g_2] \right)$. 
\begin{align}
    \mu \left(H_T[g_1,g_2] \right)
    =& \sum_{\substack{q\in \Delta,\\ A_m(q)^2< M_T}} \mu \left(V_{q,T}[g_1,g_2]\right)\\
    \underbrace{=}_{\text{Lemma \ref{lem:bound on skewed V vol} and \eqref{eq:lower bound on A_m}}}&~\sum_{\substack{q\in \Delta,\\ \|q\|< C\sqrt{T}}}\mu \left(V_{q,T}[g_1,g_2]\right)+O(T^{m(m+1)-m/2}).
\end{align} 
As above, we split the computation of the volume of $V_{q,T}[g_1,g_2]$ into the sum of the three integrals $$\left(\int_{\al_{q,T}}^{\al_{\de}(q,T,\e_1)}+\int_{\al_{\de}(q,T,\e_1)}^{\la(q,T,\e_2)}+\int_{\la(q,T,\e_2)}^{\be_{q,T}}\right) \frac{v_m}{|G_4|^m |\det(H_1)|}\int_{a}^{b} \frac{(-B_1^2 t^{\frac{2}{m}+2}+t^{\frac{2}{m}}T^2-A_m(q)^2)^{\frac{m}{2}}}{t^{m+2}}dt.$$ But now, in order to conclude the volume estimate \eqref{eq:the estimate for H skew balls}, we need to use Lemma \ref{lemma on the integral and summation in a ball} when summing over $\gamma\in\Delta$   the estimates depending on $\|\gamma\|$ that appear in lemmata \ref{lemma for integral from a to a delta} - \ref{lem:integral from c to b}. For the summation  over $\gamma$ of the estimates in lemmata \ref{lemma for integral from a to a delta} - \ref{lem:integral from a_delta to c}, note that by Lemma \ref{lemma on the integral and summation in a ball}, the infinite sum  $\sum_{\substack{q\in \Delta}}\frac{1}{\|q\|^{m^2}}$ converges. For the estimate appearing in Lemma \ref{lem:integral from c to b}, observe that
 
\begin{align}
    & \sum_{\substack{q\in \Delta,\\ \|q\|<C\sqrt T}} \frac{T^{m+\frac{\e_2 m^2}{m+1}}}{A_m(q)^{\frac{m^2}{m+1}}}\\
    = & T^{m+\frac{\e_2 m^2}{m+1}}O(\sqrt T^{m(m-1)-\frac{m^2}{m+1}})\\
    =&O(T^{0.5(m^2+m)-\frac{(0.5-\e_2)m^2}{m+1}}).
\end{align} 
As we may choose any $\e_1,\e_2$ with $0<\e_1<2\e_2<2$ (this restriction appears only in Lemma \ref{lem:integral from a_delta to c}), it follows that there exists a $\kappa>0$ such that $$H_T[g_1,g_2]=\frac{v_m}{G_4^m |\det(H_1)|}\frac{m\Ga(\frac{m}{2}+1)\Ga(\frac{m^2}{2})}{2\Ga(\frac{m^2}{2}+\frac{m}{2}+1)}T^{m(m+1)}\sum_{q\in \Delta}\frac{1}{A_m(q)^{m^2}}+O(T^{m(m+1)-\kappa}).$$ To obtain the exact expressions appearing in Proposition \ref{computation for H and V}, recall that $A_m(q)=\|G_1q H_1\|$ and note that $\nu_m$ the volume of the unit ball in $m$-space is $\frac{\pi^\frac{m}{2}}{\Ga(\frac{m}{2}+1)}$ 
\end{proof}
\section{Proof of equidistribution along skewed $H$-balls}
We start by reducing Theorem \ref{The $G$-invariance of limiting measure} to the following equidistribution statement along skewed balls of the connected component of $H$. Consider for $g_1,g_2\in\SL(m+1,\R)$ and $T>0$ the measures
\begin{equation}
   \label{eq:measures along the connected component} \mu^\circ_{T,g_1,g_2}(F):=\frac{1}{\mu(V_T[g_1,g_2])}\int_{V_T[g_1,g_2]}F(a_t^{-1}u_{v}^{-1}g_1\Ga)dv\frac{1}{t^{m+2}}dt.
\end{equation}
\begin{theorem}\label{thm:equidisitribution along skewed balls of the connected component}
   For all $g_1,g_2\in\SL(m+1,\R)$ and all $F\in C_c(X)$ it holds that $$\lim_{T\to\infty}\mu_{T,g_1,g_2}^\circ(F)=\mu_X(F).$$
\end{theorem}
\begin{proof}[Proof of Theorem \ref{The $G$-invariance of limiting measure} assuming Theorem \ref{thm:equidisitribution along skewed balls of the connected component}]
In view of \eqref{decomposition of H_T into V_T}, we can decompose $\mu_{T,g_1,g_2}(F)$ as the following convex linear combination:\begin{align}
    \mu_{T,g_1,g_2}(F)
    =\sum_{\substack{q\in \Delta}}\frac{\mu \left(V_{q,T}[g_1,g_2] \right)}{\mu \left(H_T[g_1,g_2] \right)}\frac{1}{\mu \left(V_{q,T}[g_1,g_2] \right)}\int_{V_{q,T}[g_1,g_2]} F(q^{-1}a_t^{-1}u_v^{-1}g_1\Gamma) \frac{1}{t^{m+2}}dv dt
\end{align} 
Notice that \begin{equation}
    \mu(V_{q,T}[g_1,g_2])=\mu(V_T[g_1,q g_2]),
\end{equation}and observe that \begin{equation}
    \frac{1}{\mu \left(V_{q,T}[g_1,g_2] \right)}\int_{V_{q,T}[g_1,g_2]} F(q^{-1}a_t^{-1}u_v^{-1}g_1\Gamma) \frac{1}{t^{m+2}}dv dt=\mu_{T,g_1,q g_2}^\circ(L_{q^{-1}}(F)).
\end{equation}By assuming Theorem \ref{thm:equidisitribution along skewed balls of the connected component}, we have for all $q\in\Delta$,$$\lim_{T\to\infty}\mu_{T,g_1,q g_2}^\circ(L_{q^{-1}}(F))=\mu_X(L_{q^{-1}}(F))\underbrace{=}_{\text{invariance}}\mu_X(F).$$We denote \begin{equation}
    c_{q,T}:=\frac{\mu \left(V_{q,T}[g_1,g_2] \right)}{\mu \left(H_T[g_1,g_2] \right)}.
\end{equation}
Then clearly, $$\sum_{q\in\Delta}c_{q,T}=1,~\forall T,\text{ and }c_{q,T}\leq 1,~\forall q,T.$$
Importantly, by Lemma \ref{lem:bound on skewed V vol} and by \eqref{eq:the estimate for H skew balls} there is $C>0$ such that for all $T>0$ $$c_{q,T}\leq\frac{C}{\|q\|^{m^2}}.$$
By Lemma \ref{lemma on the integral and summation in a ball}, $\sum_{q\in\Delta}\frac{1}{\|q\|^{m^2}}$ converges. Then for arbitrary small $\e>0$, we may find $N_\e$ such that $$\sum_{\substack{q\in\Delta,\\\|q\|>N_\e}}c_{q,T}\leq \e,~ \forall T>0.$$
Thus, \begin{align}
    |\mu_{T,g_1,g_2}(F)-\mu_X(F)|=&\left|\sum_{q\in\Delta}c_{q,T}(\mu_{T,g_1,q g_2}^\circ(F)-\mu_X(F))\right|\\
    \leq &\sum_{q\in\Delta}c_{q,T}\left|\mu_{T,g_1,q g_2}^\circ(F)-\mu_X(F)\right|\\
    \leq&\sum_{\substack{q\in\Delta,\\\|q\|\leq N_\e}}\left|\mu_{T,g_1,q g_2}^\circ(F)-\mu_X(F)\right|+2\e\|F\|_\infty.
\end{align}
Then, by assuming Theorem \ref{thm:equidisitribution along skewed balls of the connected component}, we get that $$\limsup_{T\to\infty}|\mu_{T,g_1,g_2}(F)-\mu_X(F)|\leq2\e\|F\|_\infty,$$which concludes our proof.
\end{proof}

In the rest of the section we will be proving Theorem \ref{thm:equidisitribution along skewed balls of the connected component}. Let $\overline{X} = \overline{G/\Ga}$ denote the one-point compactification of $X=G/\Ga$. Our plan is to show that if a finite measure $\eta$ on $\overline{X}$  is a weak-* limit of the measures $\mu^\circ_{T,g_1,g_2},~ T>0$ (recall that by the Banach-Alaoglu theorem, there's always a weak-* limit)\footnote{recall if any convergent subsequence of a bounded sequence converges to the same limit, then so does the original bounded sequence.},  then 
\begin{enumerate}
    \item There's no escape of mass, namely $\eta(\infty)=0$, and
    \item $\eta$ is $G$-invariant.
\end{enumerate}
As a result we must have $\eta|_X=\mu_X$ the unique $G$-invariant probability on $X$.

\vspace{3mm}

There are essentially two key facts that stand behind our proof of the above statements. The first is that the measures $\mu^\circ_{T,g_1,g_2}$ involve integration along families of polynomial trajectories, which allows to utilize deep results due to Shah on the behavior of polynomial trajectories in $G/\Ga$. We note that those results are generalization of the celebrated results of Dani-Margulis \cite{Dani1993LimitDO} building on the linearisation technique. The second key fact is that $\eta$ is invariant by a unipotent group, see Theorem \ref{unipotent invariance of haar measure}. This opens the way for the application of Ratner's theorems on measure rigidity \cite{Ratner91a} in combination with the results of Shah. 
We note that our apporach is similar to the one taken by Gorodnik in  \cite{Gorodnik2003LatticeAO}. The essential difference between our approach compared to \cite{Gorodnik2003LatticeAO} is in our treatment of the divergence of the polynomial trajectories in the representation space, see Section \ref{our lemma expansion inequality for applying shah dichotomy}.
\subsection{The $U$-invariance of the measure along skewed balls of $V$}
The main goal here is to prove that any weak-* limit of $\mu_{T,g_1,g_2}^\circ$  is $U:=\begin{bmatrix}
I_m & 0 \\
\mathbb{\R}^m & 1 
\end{bmatrix}$-invariant. For any $u_0\in U$, consider $L_{u_0}(F)(x):=F(u_0^{-1}x)$.
\begin{theorem}\label{unipotent invariance of haar measure}
For all $g_1,g_2\in \SL(m+1,\R)$ and all $F\in C_c(X)$ it holds that 
\begin{equation}
    \lim_{T\to \infty}\left(\mu^\circ_{T,g_1,g_2}(F)-\mu^\circ_{T,g_1,g_2}(L_{u_0} (F)\right)=0.
\end{equation}
\end{theorem}

The following lemma will be needed: 

\begin{lemma}\label{lemma on symmetric difference}
Let $f:\R^{d}\to \R$ be a bounded continuous function. Let $E\subset \R^{d}$ be an ellipsoid with surface area $S$, then for $y\in \R^{d}$ we have the estimate:
\begin{equation}
    \left|\int_{E}[f(v)-f(y+v)]dv\right| \le \|f\|_\infty \|y\|S.
\end{equation}

\end{lemma}
\begin{proof}
Indeed, 
\begin{align*}
    \left|\int_{E}[f(v)-f(y+v)]dv\right|
  = &\left|\int_{E}f(v)dv - \int_{E}f(v)dv\right|\\
  \le & \int_{E\triangle
 E-y}|f(v)|dv\\
  \le & \|f\|_\infty\|y\|S, \tag{$f$ is bounded}
\end{align*}
where the last line follows from the Theorem 1 of \cite{symmetricdifferencesch2010}.
\end{proof}
\begin{proof}[Proof of Theorem \ref{unipotent invariance of haar measure}]Recall that $$V_T[g_1,g_2]=\{u_va_t:t>0,~v\in D_{T,t}\},$$where $D_{T,t}:=D_{I_m,T,t}$ which is given by \eqref{the notation D for the ball}.  We denote $\al_T:=\al_{I_m,T}$ and $\be_T:=\be_{I_m,T}$ the roots of $-B_1^2 t^{\frac{2}{m}+2}+t^{\frac{2}{m}}T^2-A_m(q)^2$ which determine $D_{T,t}$, see Section \ref{sec:Estimate of the Haar measure growth of skewed balls in H}. Let $F\in C_c(X)$ and fix $u_{v_0}=\begin{bmatrix}
I_m & 0 \\
v_0 & 1 
\end{bmatrix}\in U$. We have
\begin{align*}
    &\left| \int_{V_{T}[g_1,g_2]} F(h^{-1}g_1\Gamma) d\mu(h)-\int_{V_{T}[g_1,g_2]} F(u_{v_0}^{-1}h^{-1}g_1\Gamma) d\mu(h) \right |\\
    =&\left|\int_{V_{T}[g_1,g_2]} \left(F(h^{-1}g_1\Gamma) - F(u_{v_0}^{-1}h^{-1}g_1\Gamma) \right) d\mu(h)\right |\\
    =&\left|\int_{\al_{T}}^{\be_{T}}\int_{D_{T,t}} \left(F( a_t^{-1} u_v^{-1} g_1\Gamma) - F(u_{v_0}^{-1} a_t^{-1} u_v^{-1}g_1\Gamma) \right) dv \frac{1}{t^{m+2}}dt\right |\\
    =&\left|\int_{\al_{T}}^{\be_{T}} \int_{D_{T,t}} \left(F( a_t^{-1} u_v^{-1} g_1\Gamma) - F(a_t^{-1} [a_t u_{v_0}^{-1}a_t^{-1}]  u_v^{-1}g_1\Gamma) \right) dv \frac{1}{t^{m+2}}dt\right |\\
    =&\left|\int_{\al_{T}}^{\be_{T}} \int_{D_{T,t}} \left(F( a_t^{-1} u_{-v} g_1\Gamma) - F(a_t^{-1} u_{-t^{\frac{1}{m}+1}v_0-v}  g_1\Gamma) \right) dv \frac{1}{t^{m+2}}dt\right |\\
    =&\left| \left( \int_{\al_T}^{c_T} + \int_{c_T}^{\be_T}  \right) \int_{D_{T,t}} \left(F( a_t^{-1} u_v^{-1} g_1\Gamma) - F(a_t^{-1} u_{-t^{\frac{1}{m}+1}v_0-v}  g_1\Gamma) \right) dv \frac{1}{t^{m+2}}dt\right |,
\end{align*}
where the auxiliary parameter $c_T$ is chosen by \begin{equation}\label{the truncation c}
    c_T:=\frac{1}{T^{\frac{m}{2}}}.
\end{equation}
To estimate the integral $\int_{c_T}^{\be_T}$ we use the trivial bound
\begin{align*}
    &\left|  \int_{D_{T,t}} \left(F( a_t^{-1} u_{-v} g_1\Gamma) - F(a_t^{-1} u_{-t^{\frac{1}{m}+1}v_0q^{-1}-v}  g_1\Gamma) \right) dv  \right|\\
    \le &   \int_{D_{T,t}} 2\|F\|_\infty dv  
    =  2\|F\|_\infty v_m \frac{(-B_1^2 t^{\frac{2}{m}+2}+t^{\frac{2}{m}}T^2-A_m^2)^{\frac{m}{2}}}{G_4^m |\det(H_1)|}\tag{by \eqref{eq:volume of D_gamma,t}}, 
\end{align*}
so that the integral $\int_{c_T}^{\be_T}$ will be bounded by 
\begin{align*}
    &\int_{c_T}^{\be_T} 2\|F\|_\infty v_m \frac{(-B_1^2 t^{\frac{2}{m}+2}+t^{\frac{2}{m}}T^2-A_m^2)^{\frac{m}{2}}}{G_4^m |\det(H_1)|}\frac{1}{t^{m+2}}dt\\
\ll & \int_{c_T}^{\be_T}\frac{T^{m}}{t^{m+1}}dt\\
\ll & {T^{\frac{m^2}{2}+m}}\tag{using \eqref{the truncation c}}.
\end{align*}

After normalization by $V_{T}[g_1,g_2]\asymp T^{m(m+1)}$, this integral vanishes as $T\to \infty$.

Now it remains to estimate the integral $\int_{\al_T}^{c_T}$. For that we use Lemma \ref{lemma on symmetric difference}.
%Note that in the range $(a_{\de},c)$, since $c=\frac{1}{T^{\e'}} \left( \frac{A_m(q)}{T}\right)^{\frac{m}{m+1}}\ll t_0:=\frac{T}{\sqrt{m+1}B_1}$ as $T\to \infty$ and thus $-B_1^2 t^{\frac{2}{m}+2}+t^{\frac{2}{m}}T^2$ is in the range of monotonic increase and thus reaches its maximum at $c$. 

\begin{align*}
    &\left| \int_{\al_T}^{c_T}  \int_{D_{T,t}} \left(F( a_t^{-1} u_{-v} g_1\Gamma) - F(a_t^{-1} u_{-t^{\frac{1}{m}+1}v_0-v}  g_1\Gamma) \right) dv \frac{1}{t^{m+2}}dt \right|\\
    \le &  \|F\|_\infty \int_{\al_T}^{c_T}  \left( \|-t^{\frac{1}{m}+1}v_0\| \sqrt{\frac{-B_1^2 t^{\frac{2}{m}+2}+t^{\frac{2}{m}}T^2-A_m^2}{G_4^m |\det(H_1)|}}^{m-1}\right) \frac{1}{t^{m+2}}dt  \tag{by Lemma \ref{lemma on symmetric difference}}\\
    = &  \|F\|_\infty \int_{\al_T}^{c_T}   \|t^{\frac{1}{m}-m-1}v_0\| \sqrt{\frac{-B_1^2 t^{\frac{2}{m}+2}+t^{\frac{2}{m}}T^2-A_m^2}{G_4^m |\det(H_1)|}}^{m-1} dt \\
    \le &  \|F\|_\infty \int_{\al_T}^{c_T}   \|\al_T^{\frac{1}{m}-m-1}v_0\| \sqrt{\frac{c_T^{\frac{2}{m}}T^2}{G_4^m |\det(H_1)|}}^{m-1} dt\\
    \ll & ~c_T\al_T^{\frac{1}{m}-m-1}(c_T^{\frac{1}{m}}T)^{m-1} \\
    \ll & T^{m(m+1)-\frac{5}{2}} \tag{using \eqref{range for a} the range for the root $a$}.
\end{align*}
 Again, this term goes to zero as $T\to \infty$, after normalization by $V_{T}[g_1,g_2]\asymp T^{m(m+1)}$. 
\end{proof}

\subsection{The non-escape of mass}\label{sec:non escape of mass}
Recall $G=\SL(m+1,\R)$ and $\Ga\leq G$ is a lattice. Let $\frak g$ be the Lie algebra of $G$. For positive integers $d$ and $n$, denote by
$\mathcal {P}_{d,n}(G)$ the set of functions $\varphi: \R^n \to G$ such that for any $a, b \in \R^n$, the map
\begin{equation}
    \tau\in \R \mapsto \Ad(\varphi(\tau a+b))\in \frak{g}
\end{equation}
is a polynomial of degree at most $d$ with respect to some basis of $\frak g$.

Let $\V_G = \sum_{i=1}^{\dim \frak g} \bigwedge^i \frak{g}$. There is a natural action of $G$ on $\V_G$ induced from the adjoint
representation (in other words, we are considering a representation $\pi:G \to \text{GL}(\V_G)$ but sometimes omit the symbol $\pi$). Fix a norm $\|\cdot\|$ on $\V_G$. For a Lie subgroup H of $G$ with Lie algebra $\frak{h}$, take a unit vector $p_\text{H} \in \bigwedge^{\dim \frak h} \frak{h}$.

\begin{theorem}[Special case of the theorems 2.1 and 2.2 in \cite{Shah1996LimitDO}, combined]\label{Shah dichotomy theorem}
With notations above, there exist closed subgroups $U_i (i=1,2,...,l)$ such that each $U_i$ is the unipotent radical of a parabolic subgroup, $U_i\Ga$ is compact in $X=G/\Ga$ and for any $d,n\in \N$, $\e,\de>0$, there exists a compact set $C \subset G/\Ga$ such that for any $\varphi \in \mathcal{P}_{d,n}(G)$ and a bounded open convex set $D\subset \R^n$, one of the following holds:
\begin{enumerate}
    \item there exist $\gamma\in \Ga$ and $i=1,...,l$ such that $\sup_{v\in D}\|\varphi(v)\gamma.p_{U_i}\| \le \de$; 
    \item $\vol(v\in D:\varphi(v)\Ga \notin C)< \e \vol(D)$, where $\vol$ is the Lebesgue measure on $\R^n$.
\end{enumerate}
\end{theorem}
%For $q \in \Ga=\SL(3,\Z)$, we shall consider the following polynomial map: Let $n=2, t>0$, $v\in \R^m$ and define $q(t,v)=q(t,v,q):=q a_t^{-1} u_v^{-1}$. 

Fix  $g_1,g_2\in\SL(m+1,\R)$. We shall consider the family of polynomial maps  which appear in the integration along the measures $\mu^\circ_{T,g_1,g_2}$ which are given by \begin{equation}\label{eq:definition of q t v}
   \varphi_t(v):= a_t^{-1} u_v^{-1} g_1,
\end{equation}
where  $t>0$ and $v\in\R^m$.
For each fixed $t$, the map $\varphi_t(\cdot)$ is in $\mathcal{P}_{2,m}(G)$. 
Our strategy is to investigate how $\varphi_t(v)$ fails the condition 1 of the Theorem \ref{Shah dichotomy theorem} by studying the expanding phenomenon of the map $\varphi_t(v)$. This will be the key fact which will allow to prove the non-escape of mass. 

\vspace{5mm}

%Extend representation $\pi:G \to \text{GL}(\V_G)$ as in the theorem above to a complex representation by replacing $\V_G$ with $\V_G \otimes \C$, and 


Denote by $\mathcal{H}_{\Ga}$ the family of all proper closed connected subgroups H of $G$ such that $\Ga \cap \text{H}$ is a lattice in H, and $\Ad(\text{H} \cap \Ga)$ is Zariski-dense in $\Ad(\text{H})$. We have the following important theorem:

\begin{theorem}[Theorem 1.1 \cite{Ratner91a}, Theorems 2.1 and 3.4 \cite{Dani1993LimitDO}. see also Section 3 and Proposition 4.1 in \cite{Shah1996LimitDO}]\label{ratner's theorem on discreteness}
The set $\mathcal{H}_{\Ga}$ is countable. For any $\emph{\text{H}}\in \mathcal{H}_{\Ga}, \Ga.p_\emph{\text{H}}$ is discrete.    
\end{theorem}

Consider the finite set of unipotent radicals of parabolic subgroups, denoted $U_1,...,U_l$ appearing in Theorem \ref{Shah dichotomy theorem}.  By Theorem \ref{ratner's theorem on discreteness}, we have that  $\Ga.p_{U_i}$ is discrete in $\V_G$ for all $i$.

Write $\V_G=\V_0 \bigoplus \V_1$, where $\V_0$ is the space of vectors fixed by $G$ and $\V_1$ is its $G$-invariant complement (exists because every finite-dimensional representation of a semisimple Lie group is completely reducible). Denote by $\Pi$ the projection of $\V_G$ onto $\V_1$ with kernel $\V_0$, and note that by Lemma 17 in \cite{Gorodnik2003LatticeAO} it holds that $\Pi(\Ga.p_{U_i})$ is discrete for all $i$. Since $p_{U_i}$ is not fixed by $G$ (this is because the action is through conjugation and $U_i$'s are not normal subgroups in the simple group $G$), $\Pi(\Ga.p_{U_i})$ does not contain $0$. So it follows that 
\begin{equation}\label{inf of the projection is at least r}
    \inf_{x\in \cup_{i=1}^l \Ga.p_{U_i}} \|\Pi(x)\|:=r>0.
\end{equation}
The following lemma will play a crucial role in proving $\eta(\infty)=0$:


\begin{lemma}\label{our lemma expansion inequality for applying shah dichotomy} Let $B\subseteq \R^m$ be a bounded set, and let $\chi\in(0,1)$. Then for any $\nu>0$, there exist $t_0>0$ such that for any $0<t<t_0$, any $\xi\in B$ and any $x\in \V_G$ such that $\|\Pi(x)\|\geq r$, it holds
\begin{equation}\label{expansion inequality for applying shah dichotomy}
    \sup_{v\in D\left(t^{\frac{1-\chi}{m}}\right)+\xi} \|a_t^{-1}u_v^{-1}g_1.x\| >\nu,
\end{equation}
where $D(\beta)$ is the ball of radius $\beta$ in $\R^m$ centered at the origin.
\end{lemma}

The idea of the proof of the lemma to decompose the elements  $a_t^{-1}u_v^{-1}$ into  $m$ elements, each belongs to a subgroup isomorphic to $\SL(2,\R)$ and then use the representation theory of $\SL(2,\R)$ for each component.  We note that a similar approach was used in \cite{Kleinbock2006DIRICHLETSTO}.

Let  $E_{i,j}$ be the $(m+1)\times(m+1)$ matrix where the $(i,j)$-th entry is $1$ while all other entries are zero. Consider the following copy of $\SL(2,\R)$ in $\SL(m+1,\R)$  
\begin{equation}\label{eq:j-th copy of SL2 in SLm+1}
\SL^{(j)}(2,\R):=\{I_{m+1}+(a-1)E_{jj}+bE_{m+1,j}+cE_{j,m+1}+(d-1)E_{m+1,m+1}:ad-bc=1\},
\end{equation}
for $j=1,2,...,m$. We will denote \begin{equation}
    \pi^{(j)}\begin{bmatrix}
        a&b\\c&d
    \end{bmatrix}:=I_{m+1}+(a-1)E_{jj}+bE_{m+1,j}+cE_{j,m+1}+(d-1)E_{m+1,m+1}.
\end{equation}
We have the following observation which we leave the reader to verify.
\begin{lemma}\label{change of order lemma}
    Let $\sigma:\{1,2,\dots m\}\to \{1,2,\dots,m\}$ be a permutation. Then it holds that \begin{equation}\label{eq:decompostion of AU to SL2 mats}
   a_t^{-1}u_v^{-1}=\prod_{j=1}^m\pi^{(\sigma(j))}\left(
   \begin{bmatrix}
   t^{\frac{1}{m}}&0\\
   0&t^{-\frac{1}{m}}
 \end{bmatrix}
 \begin{bmatrix}
   1&0\\
   -t^{-\frac{m-j}{m}}v_{\sigma(j)}&1
 \end{bmatrix}\right),   
    \end{equation}
    where $v=(v_1,...,v_m)$ and $t>0$.
\end{lemma}
 
The decomposition \eqref{eq:decompostion of AU to SL2 mats} reduces the proof of  Lemma \ref{our lemma expansion inequality for applying shah dichotomy} to the study of the expansion of elements of the form $$\begin{bmatrix}
t^{\frac{1}{m}} & 0 \\
0 & t^{-\frac{1}{m}} 
\end{bmatrix}\begin{bmatrix}
1 & 0 \\
y & 1 
\end{bmatrix}$$ in representations of $\SL(2,\R)$. In the following we review some of the basic facts on the $\SL(2,\R)$-irreducible representations, and then we will proceed with the proof of Lemma \ref{our lemma expansion inequality for applying shah dichotomy}.

Recall that if $\pi$ is an $(n+1)$-dimensional irreducible representation of $\SL(2,\R)$ and $\pi'$ is the induced Lie algebra representation, then there exists a basis $v_0,v_1,...,v_n$ such that
\begin{align}
    &\pi'(H)(v_i)=(n-2i)v_i, i=0,1,...,n;\\
    &\pi'(X)(v_i)=i(n-i+1)v_{i-1}, i=0,1,...,n;\\
    &\pi'(Y)(v_i)=v_{i+1}, i=0,1,...,n~(v_{n+1}=0).
\end{align}
where
$H = \begin{bmatrix}
    1 & ~~0\\
    0 & -1
  \end{bmatrix},
  X = \begin{bmatrix}
    0 & 1\\
    0 & 0
  \end{bmatrix}$, and
  $Y = \begin{bmatrix}
    0 & 0\\
    1 & 0
  \end{bmatrix}$ form a generating set of $\frak{sl}(2,\R)$.

\vspace{5mm}
 Under the matrix Lie group-Lie algebra correspondence, 
\[\pi \left(\begin{bmatrix}
    t^{\frac{1}{m}} & 0\\
    0 & t^{-\frac{1}{m}}
\end{bmatrix} \right)=\exp(\pi'(\log(t^{\frac{1}{m}})H))\] has the matrix representation 
\begin{equation}\label{matrix rep of a_t under the basis}
    \begin{bmatrix}
    t^{\frac{n}{m}} & & &\\
    & t^{\frac{n-2}{m}} & & \\
    & & \ddots &\\
    & & & {t}^{-\frac{n}{m}}
  \end{bmatrix}
\end{equation}
and 
\[\pi \left(\begin{bmatrix}
    1 & 0\\
    -y & 1
\end{bmatrix} \right)=\exp(\pi'(-yY))\]
has the matrix representation
\begin{equation}\label{matrix rep of u_y under the basis}
    \begin{bmatrix}
    1 & & &\\
    p_{21}(y) & 1 & & \\
    \vdots   & \ddots & \ddots &\\
    p_{nn}(y) & \cdots &  p_{n1}(y)  &  1
  \end{bmatrix}
\end{equation}
where $p_{kl}(y)$ is a monomial in $y$ of degree $l$.
Both matrices are under the basis $v_0,v_1,...,v_n$. Observe that the line $$W:=\R v_n$$ is the fixed subspace of $\pi \left(\begin{bmatrix}
    1 & 0\\
    -y & 1
\end{bmatrix}\right)$. This space is important for us since it's the  eigenspace of the matrices $\pi \left(\begin{bmatrix}
    t^{\frac{1}{m}} & 0\\
    0 & t^{-\frac{1}{m}}
\end{bmatrix}\right)$ which corresponds to the largest eigenvalue as $t\to 0$. 
We denote by $\pr_W$ the orthogonal projection on $W$ with respect to some scalar product on $\V_n$. 

The following lemma is a special case of \cite[Lemma 13]{Gorodnik2003LatticeAO}, which is essentially \cite[Lemma 5.1]{Shah1996LimitDO}.
\begin{lemma}\label{very important inequaity lemma of nimish-gorodnik}
Let $\pi:\SL(2,\R)\to \V_n$ be the $n+1$-dimensional irreducible representation of $\SL(2,\R)$, and let $$\Theta(y):=\pi \left(\begin{bmatrix}
    1 & 0\\
    -y & 1
\end{bmatrix}\right).$$ 
 Fix a bounded interval \emph{I}. Then there exists a constant $c_0>0$ such that for any $\beta \in (0,1)$, $\tau\in \emph{\text{I}}$ and \emph{$\textbf{x}\in \V_n$},
\begin{equation}
    \sup_{n\in \Theta([0,\beta]+\tau)}\|\pr_W(n\textbf{x})\| \ge c_0 \beta^n\|\textbf{x}\|.
\end{equation}
\end{lemma}
We have the following corollary.
\begin{corollary}\label{cor:main statement for expansion of sl2 reps}
 Suppose that $\pi:\SL(2,\R)\to\V$ is a representation, fix an interval \emph{I} and let $\chi\in (0,1)$. Then for all $t\in(0,1)$ the following holds.
 \begin{enumerate}
     \item\label{enu:when pi is irreduc} If $\pi$ is irreducible, then for all $\tau\in\emph{\text{I}}$ it holds that\emph{\begin{equation}\label{eq:expansion of sl(2,r) reps of y in t interval}
    \sup_{y\in[0,t^{\frac{1-\chi}{m}}]+\tau}\left \|\pi\left(\begin{bmatrix}
t^{\frac{1}{m}} & 0 \\
0 & t^{-\frac{1}{m}} 
\end{bmatrix}
\begin{bmatrix}
1 & 0 \\
-y & 1 
\end{bmatrix}\right)\textbf{x} \right \|\gg t^{-\frac{\chi}{m}}\|\textbf{x}\|,
\end{equation}}for all $\emph{\textbf{x}}\in\V$, where the implied constant depends on $\pi$ only.
\item\label{enu:when pi not necessearily irred} If $\pi$ is any representation, then for all $\tau\in \emph{\text{I}}$ \emph{\begin{equation}\label{eq:expansion of sl(2,r) in not necessarily irreducible}
    \sup_{y\in[0,t^{\frac{1-\chi}{m}}]+\tau}\left \|\pi\left(\begin{bmatrix}
t^{\frac{1}{m}} & 0 \\
0 & t^{-\frac{1}{m}} 
\end{bmatrix}
\begin{bmatrix}
1 & 0 \\
-y & 1 
\end{bmatrix}\right)\textbf{x} \right \|\gg \|\textbf{x}\|,
\end{equation}}
for all $\emph{\textbf{x}}\in\V$, where the implied constant depends on $\pi$ only.
 \end{enumerate} 
\end{corollary}
\begin{proof}
    In the following, for convenience, we will omit the representation symbol $\pi$. Suppose that $\V$ is the $n+1$'th irreducible representation. Then for all $y\in \R$ and all $\textbf{x}\in \V$
\begin{align}
& \left \|\begin{bmatrix}
t^{\frac{1}{m}} & 0 \\
0 & t^{-\frac{1}{m}} 
\end{bmatrix}
\begin{bmatrix}
1 & 0 \\
-y & 1 
\end{bmatrix}.\textbf{x} \right \| \nonumber \\
\ge & C \left \| \pr_W\left(\begin{bmatrix}
t^{\frac{1}{m}} & 0 \\
0 & t^{-\frac{1}{m}} 
\end{bmatrix}
\begin{bmatrix}
1 & 0 \\
-y & 1 
\end{bmatrix}.\textbf{x} \right) \right \| \tag{by the boundedness of the linear operator $\pr_W$}\\
= & C \left \| \begin{bmatrix}
t^{\frac{1}{m}} & 0 \\
0 & t^{-\frac{1}{m}} 
\end{bmatrix}.
 \pr_W\left(
\begin{bmatrix}
1 & 0 \\
-y & 1 
\end{bmatrix}.\textbf{x} \right) \right \| \tag{since the matrix \eqref{matrix rep of a_t under the basis} commutes with $\pr_W=\pr_{\R e_n}$}\\
=& C t^{-\frac{n}{m}} \left \|
 \pr_W\left(\Theta(y)\textbf{x} \right) \right \| \tag{see the last component of \eqref{matrix rep of a_t under the basis}}
\end{align}
Then, part 1 of our corollary follows by Lemma \ref{very important inequaity lemma of nimish-gorodnik}. To conclude part 2, note that we may always decompose $\V$ as $\V_0\oplus\V_1$ where $\V_0$ is the space of fixed vectors, and $\V_1$ is an invariant complement. By further decomposing $\V_1$ into irreducible representations, and applying part 1 of the corollary in each component, we get the result.
\end{proof}
\subsubsection{The proof for Lemma \ref{our lemma expansion inequality for applying shah dichotomy}}
Recall $\V_G=\V_0\oplus \V_1$ where $\V_0$ is the space of $\SL(m+1,\R)$-fixed vectors, and $\V_1$ is it's invariant complement. Note that in order to prove Lemma \ref{our lemma expansion inequality for applying shah dichotomy}, it's enough to verify that for a fixed $r>0$, a bounded set $B\subset \R^m$ and $\chi\in(0,1)$ it holds that \begin{equation}\label{eq:what we need to show for lem 3.6}
    \min_{\textbf{x}\in \V_1,~\|\textbf{x}\|\geq r}\sup_{v\in D\left(t^{\frac{1-\chi}{m}}\right)+\xi} \|a_t^{-1}u_v^{-1}\textbf{x}\|\gg t^{-\chi/m},~\forall\xi\in B
\end{equation}
where $t\in(0,1)$. As $\V_1$ is invariant by each $\SL^{(j)}(2,\R)$ action, we may further decompose $\V_1$ by
\begin{align}
    \V_1 &= \V_{0}^{(j)} \oplus \V_{1}^{(j)} 
\label{decompositive of vs w.r.t. SL2}
\end{align}where $\V_0^{(j)}$ is the subspace of fixed vectors of the action of $\SL^{(j)}(2,\R)$ for $j=1,2,...,m$ and $\V_1^{(j)}$ is its invariant complement. 
For any $\textbf{x}\in \V_1$ we write 
\begin{equation}    \textbf{x}=\textbf{x}_{0}^{(j)}+\textbf{x}_{1}^{(j)}, j=1,2,...,m.
\end{equation}
where $\textbf{x}_{i}^{(j)} \in \V_{i}^{(j)}, i=1,2,...,m$ from above. 

\begin{lemma} It holds that
\emph{$\inf_{\textbf{x}\in \V_1, \|\textbf{x}_1\|\ge r} \left( \|\textbf{x}_{1}^{(1)}\|+\cdots+\|\textbf{x}_{1}^{(m)}\| \right) = Lr$} for some $L>0$
\end{lemma}
\begin{proof}
We recall that $UA$ is an  epimorphic group, cf. \cite{Shah2000OnAO}. This simply means that if $\textbf{x}\in \V_1$ is fixed by $UA$, then it's fixed by all of $\SL(m+1,\R)$. But in such case, $\textbf{x}=0$, since $\V_1$ is an invariant complement of the fixed vectors. In particular, this implies that $\bigcap_{i=1}^m\V_{0}^{(i)}=\{0\}$. Since the sphere of radius $1$ is compact and as $\bigcap_{i=1}^m\V_{0}^{(i)}=\{0\}$, we have $$\inf_{\textbf{x}_1\in \V_1, \|\textbf{x}_1\|= 1} \left( \|\textbf{x}_{1}^{(1)}\|+\cdots+\|\textbf{x}_{1}^{(m)}\| \right) = L > 0.$$ 
By scaling, it is easy to see for $r \ge 1$, 
\begin{align*}
    \inf_{\textbf{x}_1\in \V_1, \|\textbf{x}_1\|= r} \left( \|\textbf{x}_{1}^{(1)}\|+\cdots+\|\textbf{x}_{1}^{(m)}\| \right) 
%    =& \frac{r}{1}\inf_{w_1\in V_1, \|w_1\|= r} \left( \|w_{1}^{(1)}\|+\|w_{1}^{(2)}\| \right) \\
    = & r \inf_{\textbf{x}_1\in \V_1, \|\textbf{x}_1\|= 1} \left( \|\textbf{x}_{1}^{(1)}\|+\cdots+\|\textbf{x}_{1}^{(m)}\| \right) \tag{by linearity and scaling} \\
    =& Lr>0.
\end{align*}
\end{proof}
Now take $\textbf{x}\in \V_1$ and fix $j$ such that $\|\textbf{x}^{(j)}_1\|\geq Lr$. We pick a permutation $\sigma:\{1,...,m\}\to\{1,...,m\}$ such that $\sigma(m)=j$. Using Lemma \ref{change of order lemma} we write \begin{align}\label{eq:chose permutation for j with large component}
a_t^{-1}u_v^{-1}\textbf{x}=\prod_{l=1}^{m-1}&\pi^{(\sigma(l))}\left(
   \begin{bmatrix}
   t^{\frac{1}{m}}&0\\
   0&t^{-\frac{1}{m}}
 \end{bmatrix}
 \begin{bmatrix}
   1&0\\
   -t^{-\frac{m-l}{m}}v_{\sigma(l)}&1
 \end{bmatrix}\right)
 \pi^{(j)}
 \left(
   \begin{bmatrix}
   t^{\frac{1}{m}}&0\\
   0&t^{-\frac{1}{m}}
 \end{bmatrix}
 \begin{bmatrix}
   1&0\\
   -v_{j}&1
 \end{bmatrix}\textbf{x}\right).
\end{align}
Note that $$\left(\left[0,\frac{1}{\sqrt{m}}t^{\frac{1-\chi}{m}}\right]+\text{I}_1 \right)\times...\times \left(\left[0,\frac{1}{\sqrt{m}}t^{\frac{1-\chi}{m}}\right]+\text{I}_m \right)\subseteq D\left(t^{\frac{1-\chi}{m}}\right)+B,$$ for some intervals $\text{I}_l$, $1\leq l\leq m$. Then \eqref{eq:what we need to show for lem 3.6} is obtained by applying Corollary \ref{cor:main statement for expansion of sl2 reps} as follows: We decompose $\V_1^{(j)}$ into $\SL^{(j)}(2,\R)$-irreducible representations, and we assume (without loss of generality, as all norms are equivalent) that our norm $\|\cdot\|$ on $\V_1$ is obtained by taking the sup norm with respect to a basis of $\V_1$ composed out of a basis for $\V_0^{(j)}$ and bases for each of the $\SL^{(j)}(2,\R)$-irreducible spaces.  Then, using Corollary \ref{cor:main statement for expansion of sl2 reps} \eqref{enu:when pi is irreduc}, we have for all $\tau_j\in \text{I}_j$  that
\begin{equation}
     \sup_{v_j\in[0,t^{\frac{1-\chi}{m}}]+\tau_j}\left \|\pi^{(j)}\left(\begin{bmatrix}
t^{\frac{1}{m}} & 0 \\
0 & t^{-\frac{1}{m}} 
\end{bmatrix}
\begin{bmatrix}
1 & 0 \\
-y & 1 
\end{bmatrix}\right)\textbf{x} \right \|\gg t^{-\frac{\chi}{m}}r,
\end{equation}
and by further applying Corollary \ref{cor:main statement for expansion of sl2 reps},\eqref{enu:when pi not necessearily irred} when taking the supremum of \eqref{eq:chose permutation for j with large component} over the parameters $\left(\left[0,\frac{1}{\sqrt{m}}t^{\frac{1-\chi}{m}}\right]+\text{I}_1 \right)\times...\times \left(\left[0,\frac{1}{\sqrt{m}}t^{\frac{1-\chi}{m}}\right]+\text{I}_m \right)$, we obtain \eqref{eq:what we need to show for lem 3.6}. 

\iffalse
As before we make the substitution $s=t^{\frac{1}{m}}$. So the equation above becomes 
\begin{equation}
    \sqrt{-B_1^2s^{2+2m}+s^2T^2-A_m(q)^2} \gtrsim
c_1s^{1+m}+2s^{1-\chi}.
\end{equation}

Put $s_a=a^{\frac{1}{m}}$ and $s=s_a+z$ with $z\ge \de:=\frac{A_m(q)}{T^{1+\e}}$. $z$ is treated as a variable. When $t\in (a_{\de}, c)$, we have $z\in (\de,c^{\frac{1}{m}}-s_a)$. Recall from \eqref{the truncation c} we used the truncation 
\begin{equation}
    c=\left(\frac{A_m(q)}{T^{3\e}}\right)^{\frac{m}{m+1}}.
\end{equation}

Since $s_a\asymp \frac{A_m(q)}{T}$ is a root of $-B_1^2s^{2+2m}+s^2T^2-A_m(q)^2$, we have 
\begin{align}
    &\left(t^{-\frac{1}{m}}\sqrt{-B_1^2t^{\frac{2}{m}+2}+t^{\frac{2}{m}}T^2-A_m(q)^2}\right)^2\\
    =&\left(s^{-1}\sqrt{-B_1^2s^{2+2m}+s^2T^2-A_m(q)^2}\right)^2\\
    =
 &(s_a+z)^{-2}\left[(2s_az+z^2)T^2-B_1^2[(s_a+z)^{2m+2}-s_a^{2m+2}]\right]
\end{align}

Note on the right hand side of the last line, the dominating term is 
$$f_T(z):=\frac{(2s_az+z^2)T^2}{(s_a+z)^{2}}=\left[1-\frac{s_a^2}{(s_a+z)^2}\right]T^2$$
which is a monotonically increasing function on $(\de, c^{\frac{1}{m}}-s_a)$ and the minimal value is when $z$ approaches $\de$. But 
\begin{equation}
    \lim_{T\to \infty}f_T(\de)=\infty.
\end{equation}
This justifies that \eqref{expansion inequality} is indeed expanding on $(a_{\de},c)$ as $T\to \infty$.
\fi
 

\subsubsection{Proof of $\eta{(\infty)}=0$}\label{section 4.2}

\iffalse
Therefore, for any $x\in \cup_{i=1}^r \Ga.p_{U_i}$,
\begin{equation}
    \lim_{t\to 0} \sup_{v\in D(\frac{1}{2}\st^{\frac{1-\chi}{m}})} \|a_t^{-1}u_v^{-1}g_1.x\| =\infty.
\end{equation}
\fi

Recall that $\eta$ is a weak-* limit of the measures
\begin{align}
    \mu^\circ_{T,g_1,g_2}(F):=& \frac{1}{\mu \left(V_{T}[g_1,g_2] \right)} \int_{\al_T}^{\be_T} \int_{D_{T,t}} F \left(a_t^{-1} u_v^{-1} g_1\Gamma \right) dv \frac{1}{t^{m+2}}dt,~F\in C_c(G/\Ga),
\end{align}
where we recall that $D_{T,t}$ is the ellipsoid centered at $-G_4^{-1}G_3 -H_3H_1^{-1}t^{\frac{1}{m}+1} \in \R^m$ given by
\begin{equation}\label{definition of D}
    D_{T, t}:=\{v\in \R^m: \|G_3 H_1+G_4 v H_1+G_4 H_3 t^{\frac{1}{m}+1}\|^2 \le -B_1^2 t^{\frac{2}{m}+2}+t^{\frac{2}{m}}T^2-A_m^2\},
\end{equation}
$A_m:=A_m(I_m)$ which was defined in \eqref{eq:def of Am}, and $0<\al_T:=\al_{I_m,T}<\be_T:=\be_{I_m,T}$ are the two positive roots of $-B_1^2 t^{\frac{2}{m}+2}+t^{\frac{2}{m}}T^2-A_m^2.$  We note that $D_{T,t}$ contains the ball with the same center whose radius is equal to the \textit{shortest radius} of the ellipse, which is $$R_{T,t}:=C_{g_1,g_2}\sqrt{ -B_1^2 t^{\frac{2}{m}+2}+t^{\frac{2}{m}}T^2-A_m^2},$$
 for some constant $C_{g_1,g_2}>0$. The  displacement of the center from the origin has two parts: one part ($-G_4^{-1}G_3$) is constant and the other part ($-H_3H_1^{-1}t^{\frac{1}{m}+1}$) involves $t$ but is  bounded when $t\in(0,1)$. To use Lemma \ref{our lemma expansion inequality for applying shah dichotomy} in our proof for $\eta(\infty)=0$, we would like to have $R_{T,t}>t^{\frac{1-\chi}{m}}$. But, when $t$ approaches to either $\al_T$ or to $\be_T$, $R_{T,t}$ becomes too small. Thus we need to truncate the range for $t$ as following.
Using Lemmata \ref{lemma for integral from a to a delta} and \ref{lem:integral from c to b} together with the estimate of $\mu(V_T[g_1,g_2])$ in \eqref{eq:main estimate of V skew balls}, we have that for $\e_1,\e_2\in(0,1)$ there exists some $\ka=\ka(\e_1,\e_2)>0$ such that for any fixed $F\in C_c(G/\Ga)$ it holds \begin{equation}\label{eq:truncated integral}
\mu^\circ_{T,g_1,g_2}(F)= \frac{1}{\mu \left(V_{T}[g_1,g_2] \right)} \int_{\al_{\de}(T,\e_1)}^{\la(T,\e_2)} \int_{D_{T,t}} F \left(a_t^{-1} u_v^{-1} g_1\Gamma \right) dv \frac{1}{t^{m+2}}dt+O(T^{-\ka}).
\end{equation} 
\begin{lemma}\label{lem:the domain Dt,T has a ball of sufficiently large radius in it}
  Fix $\chi\in(0,1)$. Then, there exist $0<\e_1<\e_2<1$ such that for all $T$ large enough, we have that for all $t\in(\al_{\de}(T,\e_1),\la(T,\e_2))$ it holds that
   \begin{equation}
   R_{T,t}>t^{\frac{1-\chi}{m}}  
   \end{equation} 
  \iffalse
  For all $T$ large enough its holds that for all $t\in(\al_{\de,T},\la_T)$ 
   \begin{equation}
     \sqrt{ -B_1^2 t^{\frac{2}{m}+2}+t^{\frac{2}{m}}T^2-A_m^2}\geq t^{\frac{1-\chi}{m}}  
   \end{equation}
   \fi
\end{lemma}
\begin{proof}
To prove the inequality, we estimate from below the left hand side, and estimate from above the right hand side. For the right hand side, we have for $t\in(\al_{\de}(T,\e_1),\la(T,\e_2))$ that
\begin{equation}
    \la(T,\e_2)^{\frac{1-\chi}{m}}\geq t^\frac{1-\chi}{m}.
\end{equation} By definition of $\la$, see \eqref{eq:definition of c}, we have 
\begin{equation}\label{eq:asymp for righthand side}
    \la(T,\e_2)^{\frac{1-\chi}{m}}\asymp \frac{1}{T^\frac{\e_2(1-\chi)}{m+1}}
\end{equation}We now estimate $R_{T,t}$. As we already noted, the function $f(t):=-B_1^2 t^{\frac{2}{m}+2}+t^{\frac{2}{m}}T^2-A_m^2$ has one critical point $\theta_T$ in the range $t>0$, and $f(t)$ is monotonically increasing in $(0,\theta_T)$, where we recall that $\theta_T\asymp T$. Since $\la(T,\e_2)=o(1)$ as $T\to\infty$, we conclude that for all $t\in(\al_{\de}(T,\e_1),\la(T,\e_2))$ it holds \begin{equation}
\sqrt{-B_1^2 t^{\frac{2}{m}+2}+t^{\frac{2}{m}}T^2-A_m^2}\geq\sqrt{-B_1^2 \al_{\de}(T,\e_1)^{\frac{2}{m}+2}+\al_{\de}(T,\e_1)^{\frac{2}{m}}T^2-A_m^2}. 
\end{equation}
%Now
%\begin{align}
 %   \sqrt{-B_1^2 \al_{T,\de}^{\frac{2}{m}+2}+\al_{T,\de}^{\frac{2}{m}}T^2-A_m^2}&=\al_{T,\de}^{\frac{1}{m}}T\sqrt{1-B_1^2 \al_{T,\de}^{2}/T^2-A_m^2/\al_{T,\de}^{\frac{2}{m}}T^2},\nonumber
%\end{align}
Note that as $T\to\infty$ (by  \eqref{eq:defintion of a_delta} and by \eqref{eq:asymp for a when A small then sqrt T})  $$\al_{\de}(T,\e_1)^\frac{2}{m}=\left(\al_T^{\frac{1}{m}}+\frac{A_m}{T^{1+\e_1}}\right)^2\asymp A_m^2\left(\frac{1}{T^2}+\frac{2}{T^{2+\e_1}}\right).$$
Then, as $T\to\infty$, we have
\begin{equation}\label{eq:asymp for lefthand side}
    \sqrt{-B_1^2 \al_{\de}(T,\e_1)^{\frac{2}{m}+2}+\al_{\de}(T,\e_1)^{\frac{2}{m}}T^2-A_m^2}\asymp \frac{1}{T^{\e_1/2}}.
\end{equation}
We are free to choose $\e_1$ and $\e_2$ in the interval $(0,1)$ as we wish. In particular, we as well may assume that $$\e_1/2<\frac{\e_2(1-\chi)}{m+1}.$$With the latter choice,  we obtain our claim by the estimates \eqref{eq:asymp for lefthand side} and \eqref{eq:asymp for righthand side}.
\end{proof}
We are now ready to prove $\eta(\infty)=0.$ 
%Let $\alpha=t^{0.3}$ and consider the ball $D_{q,t}(\alpha)$ which has the same center with $D$ but with radius $\alpha$. For sufficiently small 
By the above Lemma \ref{lem:the domain Dt,T has a ball of sufficiently large radius in it} and Lemma \ref{our lemma expansion inequality for applying shah dichotomy}, we conclude that the first outcome in Theorem \ref{Shah dichotomy theorem} for the translated ellipsoids $D=D_{T,t}$ fails in the range $t\in(\al_\de(T,\e_1),\la(T,\e_2))$ for all $T$ large enough. Here $\e_1,\e_2$ are fixed such that Lemma \ref{lem:the domain Dt,T has a ball of sufficiently large radius in it} holds.
Thus, by the second outcome of Theorem \ref{Shah dichotomy theorem}, for arbitrarily small $\e>0$  there's a compact $C\subset G/\Ga$  such that
\begin{equation}\label{eq:small proporotion of visits outside a compact C}
    \vol(v\in D_{T,t}:\varphi_t(v)\Ga \notin C)< \e \vol(D_{T,t}),~\forall t\in(\al_\de(T,\e_1),\la(T,\e_2)).
\end{equation}
Now take $F\in C_c(G/\Ga)$ with $\mathbf{1}_C \le F \le 1$. As $T\to \infty$,
\begin{align*}
\mu_{T,g_1,g_2}^\circ(F)=&\frac{1}{\mu \left(V_{T}[g_1,g_2] \right)}\int_{V_{T}[g_1,g_2]} F( h^{-1}g_1\Gamma) d\mu(h)\\
=& \frac{1}{\mu \left(V_{T}[g_1,g_2] \right)} \int_{\al_{\de}(T,\e_1)}^{\la(T,\e_2)} \int_{D_{T,t}} F \left(\varphi_t(v)\Ga \right) dv \frac{1}{t^{m+2}}dt+O(T^{-\ka})\tag{by \eqref{eq:truncated integral}}\\
\ge& \frac{1}{\mu \left(V_{T}[g_1,g_2] \right)} \int_{a_{\de}(T,\e_1)}^{\la(T,\e_2)} \int_{D_{T,t}} \mathbf{1}_C \left(\varphi_t(v)\Ga \right) dv \frac{1}{t^{m+2}}dt +O(T^{-\ka})\\
\ge& \frac{1}{\mu \left(V_{T}[g_1,g_2] \right)} \int_{a_{\de}(T,\e_1)}^{\la(T,\e_2)}  (1-\e)\vol(D_{T,t}) \frac{1}{t^{m+2}}dt+O(T^{-\ka}) \tag{by \eqref{eq:small proporotion of visits outside a compact C}}\\
=&(1-\e)\frac{1}{\mu \left(V_{T}[g_1,g_2] \right)} \int_{\al_T}^{\be_T} \vol(D_{T,t}) \frac{1}{t^{m+2}}dt +O(T^{-\ka})\tag {by Lemmata \ref{lemma for integral from a to a delta} and \ref{lem:integral from c to b}}\\
=&1-\e+O(T^{-\ka}).
\end{align*}
Therefore,
\begin{equation}
    \liminf_{T\to \infty}\mu_{T,g_1,g_2}^\circ(F)\ge 1-\e
\end{equation}
so that
\begin{equation}
    \eta(\infty)\le\limsup_{T\to \infty}\mu_{T,g_1,g_2}^\circ(\text{support}(F)^c) \le \e
\end{equation}

Since $\e$ is arbitrary $\eta(\infty)=0$.

\subsection{Proof of $G$-invariance}

Recall $U=\begin{bmatrix}
    I_m & 0\\
    \R^m & 1
\end{bmatrix}$. For a closed subgroup
H of $G$, denote
\begin{align}
    &N(\text{H},U):=\{g\in G:Ug\subset g\text{H}\},\\
    &S(\text{H},U):=\cup_{\text{H}'\subsetneq \text{H}, \text{H}'\in \mathcal{H}_{\Ga}}N(\text{H}',U).
\end{align}
Consider,
\begin{equation}
    Y:=\bigcup_{\text{H}\in \mathcal{H}_{\Ga}}N(\text{H},U)\Ga=\bigcup_{\text{H}\in \mathcal{H}_{\Ga}}[N(\text{H},U)-S(\text{H},U)]\Ga \subset G/\Ga,
\end{equation}
where $\mathcal{H}_
{\Ga}$ was defined above Theorem \ref{ratner's theorem on discreteness}.
The equality holds since for any $g\in N(\text{H},U)$, if $g$ is also in $S(\text{H},U)$, then $g$ must belong to $N(\text{H}',U)$ for some $\text{H}'\subsetneq H$ (note for Lie subgroups, this condition means $\dim \text{H}' < \dim \text{H}$) and $\text{H}'\in \mathcal{H}_{\Ga}$. Since $\text{H}'$ has strictly lower dimension, by repeating this argument we see eventually, $g$ will fall into some $N(\tilde{\text{H}}, U)-S(\tilde{\text{H}}, U)$ (with $S(\tilde{\text{H}}, U)$ possibly empty when $\tilde{\text{H}}$ has minimal dimension).

Now we perform the ergodic decomposition of the $U$-invariant measure $\eta$. By Theorem 2.2 of \cite{Mozes1995OnTS}, each ergodic component of $\eta$ is either $G$-invariant or supported on $Y\cup \{\infty\}$. Thus, in order to show that $\eta$ is $G$-invariant, it is sufficient to prove the following lemma.
\begin{lemma}\label{lem:eta of the singular set is zero}
    $\eta(Y)=0$.
\end{lemma}
    By the discreteness of $\mathcal{H}_{\Ga}$ (Theorem \ref{ratner's theorem on discreteness}), it suffices to show that for each fixed $\text{H}\in\mathcal{H}_{\Ga}$ it holds
    \begin{equation}
        \eta([N(\text{H},U)-S(\text{H},U)]\Ga)=0.
    \end{equation} Since $[N(\text{H},U)-S(\text{H},U)]\Ga$ is a countable union of compact subsets in $G/\Ga$ (See \cite{Mozes1995OnTS} Proposition 3.1), it suffices to show $\eta(C)=0$ for any compact subset $C$ of $[N(\text{H},U)-S(\text{H},U)]\Ga$.

The main tool in the proof of the latter statement is the following consequence of Proposition 5.4 in \cite{Shah1994LimitDO}. We use in the following the same notations as in Section \ref{sec:non escape of mass}.
\begin{theorem}\label{second dichotomy theorem of Shah}
    Let $d,n \in \N, \e>0, \emph{\text{H}}\in \mathcal{H}_{\Ga}$. For any compact set $C \subset [N(\emph{\text{H}}, U)-S(\emph{\text{H}},U)]\Ga$, there exists a compact set $F \subset \V_G$ such that for any neighborhood $\Phi$ of $F$ in $\V_G$, there exists a
neighborhood $\Psi$ of $C$ in $G/\Ga$ such that for any $\varphi \in \mathcal{P}_{d,n}(G)$ and a bounded open convex set
$D \subset \R^n$, one of the following holds:
\begin{enumerate}
    \item There exist $\gamma\in \Ga$ such that $\varphi(D)\gamma.p_\emph{\text{H}} \subset \Phi$.
    \item $\vol(t\in D: \varphi(t)\Ga \in \Psi)<\e \vol(D)$, where $\vol$ is the Lebesgue measure on $\R^n$.
\end{enumerate}
\end{theorem}
Fix $\e>0$ and $\text{H}\in\mathcal{H}_\Ga$. Recall that $\varphi_t(\cdot)$ which was defined in \eqref{eq:definition of q t v} is in $\mathcal{P}_{2,m}$. Let $C \subset [N({\text{H}}, U)-S({\text{H}},U)]\Ga$ be a compact set and  take a compact set $F\subset \V_G$ satisfying the outcome of Theorem \ref{second dichotomy theorem of Shah}. By the same argument as in the Section 4.3 with Lemma \ref{our lemma expansion inequality for applying shah dichotomy} applied, the first outcome of Theorem \ref{second dichotomy theorem of Shah} fails for all $T$ large enough. Then, for $t\in (\al_{\de}(T,\e_1), \la(T,\e_2))$ for all $T$ large enough,
\begin{equation}\label{eq:visits to Psi is negligble}
    \vol(v\in D_{T,t}:\varphi_t(v)\Ga \in \Psi)<\e \vol(D_{T,t}),
\end{equation}
where $C\subseteq\Psi$ is the compact neighborhood from Theorem \ref{second dichotomy theorem of Shah}. 
Now let $f\in C_c(G/\Ga)$ be such that $\mathbf{1}_C \le f \le 1$ and $\text{support}(f)\subset \Psi$, it follows that
\begin{align*}
\eta(C)\leq& \limsup_{T\to \infty}\mu_{T,g_1,g_2}^\circ(f)\\
=& \limsup_{T\to \infty} \frac{1}{\mu \left(V_{T}[g_1,g_2] \right)} \int_{\al_T}^{\be_T} \int_{D_{T,t}} f \left(\varphi_t(v)\Ga \right) dv\frac{1}{t^{m+2}}dt \\
=& \limsup_{T\to \infty} \frac{1}{\mu \left(V_{T}[g_1,g_2] \right)} \int_{\al_{\de}(T,\e_1)}^{\la(T,\e_2)} \int_{D_{T,t}} f \left(\varphi_t(v)\Ga \right) dv \frac{1}{t^{m+2}}dt \tag{by Lemmata \ref{lemma for integral from a to a delta} and \ref{lem:integral from c to b}}\\
\le& \limsup_{T\to \infty} \frac{1}{\mu \left(V_{T}[g_1,g_2] \right)} \int_{\al_{\de}(T,\e_1)}^{\la(T,\e_2)} \int_{D_{T,t}} \mathbf{1}_{\Psi} \left(\varphi_t(v)\Ga \right) dv \frac{1}{t^{m+2}}dt \\
\le& \limsup_{T\to \infty} \frac{1}{\mu \left(V_{T}[g_1,g_2] \right)} \int_{\al_{\de}(T,\e_1)}^{\la(T,\e_2)} \e \vol(D_{T,t}) \frac{1}{t^{m+2}}dt \\
\le & \e.
\end{align*}
Since $\e>0$ is arbitrary, $\eta(C)=0$ and thus $\eta(Y)=0$.


\section{Proof of Theorem \ref{equidistribution result on  G mod H}: The limiting measure on $H\backslash G$.}
\iffalse
We need to prove for all $\varphi\in C_c(X_{m,m+1})$, we have 
\begin{equation}
\lim_{T
\to\infty} \frac{1}{\#\Gamma_{T}}\sum_{\gamma\in \Ga_T}\varphi(x_{0}\cdot \gamma)=\int_{X_{m,m+1}}\varphi(x)d\tilde\nu_{x_{0}}(x)
\end{equation}
for some  measure $\tilde\nu_{x_0}$ on $X_{m,m+1}$ which depends on $x_{0}$.


The first thing we need to do is to construct a measure $\nu_{x_0}$ on $H\backslash G$. Since $G$ is unimodular while $H$ is not (the measure $\mu$ we defined is not right invariant), there is no $G$-invariant Haar measure on $H\backslash G$. 

Nevertheless, it is still possible to find a lift $Y\subset G$ of $H\backslash G$ such that $H \times Y \to G$ is a Borel isomorphism and decompose the Haar measure on $G$ to obtain a measure on $Y$. Here 
\fi

We follow Section 2.5 of \cite{Gorodnik2004DistributionOL}.
First, we provide an explicit measurable section $$\sigma:H\backslash G \to Y\subset G,$$and then we define a measure $\nu_Y$ on $Y$ such that \begin{equation} \label{unfolding haar measure on G using section}
    dg=d\mu d\nu_Y,
\end{equation} where $d\mu$ is the left Haar measure on $H$ given by \eqref{eq:Haar measure on $H$} and $dg$ is the Haar measure on $G$ normalized such that $\vol(G/\Ga)=1$

To define the section,  let $\mathcal{F}_m \subset \begin{bmatrix}
\SL(m,\R) & 0\\
0 & 1
\end{bmatrix}$ denote a measurable fundamental domain of 
\begin{equation}
   \begin{bmatrix}
    \Delta & 0\\
    0 & 1
    \end{bmatrix} \bigg\backslash 
   \begin{bmatrix}
    \SL(m,\R) & 0\\
    0 & 1
    \end{bmatrix} \bigg\slash  \begin{bmatrix}
    \SO(m,\R) & 0\\
    0 & 1
    \end{bmatrix} 
    \cong
    \Delta\backslash \SL(m,\R) /\SO(m,\R),
\end{equation}
and consider the product $$Y:=\mathcal{F}_m\cdot \SO(m+1,\R).$$

\iffalse
\footnote{In fact, since $\begin{bmatrix}
    \SO(m,\R) & 0\\
    0 & 1
    \end{bmatrix}$ is contained in $\SO(m+1,\R)$, $Y$ remains the same even if we take $\mathcal{F}_m$ to be the fundamental domain of $\begin{bmatrix}
    \Delta & 0\\
    0 & 1
    \end{bmatrix} \bigg\backslash  \begin{bmatrix}
    \SL(m,\R) & 0\\
    0 & 1
    \end{bmatrix}$.}
\fi
Then, we claim that the product map $H \times Y \to G$ is a Borel isomorphism, which defines a section $\sigma$ identifying $H\backslash G$ with $Y$. The surjectivity is clear from the block-wise Iwasawa decomposition. We only verify the injectivity here:

Suppose $h_1g_1s_1=h_2g_2s_2$ where $h_i\in H, g_i\in \mathcal{F}_m$ and $s_i\in \SO(m+1,\R)$. Then 
$$\begin{bmatrix}
* & 0 \\
* & *
\end{bmatrix}\ni g_2^{-1} h_2^{-1}h_1 g_1=s_2 s_1^{-1} \in \SO(m+1,\R),$$
and it follows from the definition of $\SO(m+1,\R)$ that both side must lie in $\begin{bmatrix}\SO(m,\R) & 0\\0 & 1 \end{bmatrix}$. So $h_2^{-1}h_1=g_2s_2s_1^{-1}g_1^{-1}\in \begin{bmatrix}\SL(m,\R) & 0\\0 & 1 \end{bmatrix}$. But $\begin{bmatrix}\SL(m,\R) & 0\\0 & 1 \end{bmatrix} \cap H= \begin{bmatrix}\Delta & 0\\0 & 1 \end{bmatrix}$. Hence $q g_1s_1=g_2s_2$ for some $q \in \begin{bmatrix}
\Delta & 0 \\0 & 1
\end{bmatrix}$. It follows that $g_2^{-1}q g_1=s_2 s_1^{-1}\in \begin{bmatrix}\SL(m,\R) & 0 \\0 & 1 \end{bmatrix} \cap \SO(m+1,\R)=\begin{bmatrix}\SO(m,\R) & 0 \\0 & 1 \end{bmatrix}$. Hence $g_1=g_2, s_1=s_2$ by the definition of fundamental domain, and $h_1=h_2$.

%Let $K=\{(k,k):k\in \mathcal{F}_m\cap \SO(m+1,\R) \}$. Then $\frac{\mathcal{F}_m\times \SO(m+1,\R)}{K} \cong \mathcal{F}_m \SO(m+1,\R)$

Next, let $T:=
\left\{ \begin{bmatrix}
t^{-\frac{1}{m}}\eta & 0\\
v & t
\end{bmatrix}:\eta\in\SL(m,\R),~v\in\R^m,~t>0 \right\}$ and $S:=\SO(m+1,\R)$. On $T$ we define the Haar measure  $d\tau=dv \frac{dt}{t^{m+2}} d\eta$, where $d\eta$ is the Haar measure on $\SL(m,\R)$ defined through standard Iwasawa decomposition, under which $\SO(m,\R)$ has volume $1$. We note the formula \begin{equation}
    \int_{\SL(m,\R)}\varphi(\eta)d\eta=\int_{\mathcal{F}_m}\sum_{q\in\Delta}\int_{\SO(m,\R)}\varphi(q\eta\rho')d\rho' d\eta.\label{eq:unfolding formula on the double coset space}
\end{equation}
On $S$ we take $d\rho$ to be the  Haar measure probability measure. Then, by using Theorem 8.32 of \cite{KN02}, we unfold a Haar measure on $G$ (with some implicit normalization)  as follows
\begin{align*}
    &\int_G f(g)dg\\
   =&\int_S \int_T f(\tau \rho) d\tau d\rho \\
   =&\int_S\int_{\SL(m,\R)} \int_0^{\infty}\int_{\R^m} f(u_v a_t \eta \rho)    dv \frac{dt}{t^{m+2}}d\eta d\rho \tag{by definition of $d\tau$}\\
   =&\int_S\sum_{q \in \Delta}\int_{\mathcal{F}_m}\int_{\SO(m,\R)}  \int_0^{\infty}\int_{\R^m} f(u_v a_t q \eta\rho'  \rho)    dv \frac{dt}{t^{m+2}}d\eta  d\rho'd\rho\tag{by formula \eqref{eq:unfolding formula on the double coset space}}\\
   =& \int_S\sum_{q \in \Delta}\int_{\mathcal{F}_m} \int_0^{\infty}\int_{\R^m}  f(u_v a_t q \eta\rho)    dv \frac{dt}{t^{m+2}}d\eta d\rho\tag{invariance of $d\rho$}\\
   =& \int_S\sum_{q \in \Delta} \int_0^{\infty}\int_{\R^m} \int_{\mathcal{F}_m}  f(u_v a_t q \eta\rho)  d\eta  dv \frac{dt}{t^{m+2}}d\rho\\
   =&\int_H \int_S\int_{\mathcal{F}_m}  f(h \eta\rho)  d\eta d\rho  d\mu(h)\\
%   =&\text{Vol}(\SO(m,\R))^2 \int_H \int_{\frac{\mathcal{F}_m \times S}{K}}  f(h gK) d(gK)  dh
%   =&\text{Vol}(\SO(m,\R))^2\int_H \int_{\mathcal{F}_m S}  f(h g) dg  dh\\
   =& \int_H \int_{\mathcal{F}_m}\int_S  f(h \eta\rho)  d\rho d\eta  d\mu(h) 
%   =& \int_H \int_{\mathcal{F}_m S}  f(h g) (\vol(\SO(m,\R)) dg) dh
\end{align*}

\vspace{3mm}
Then it follows that the measure $\nu_Y$ on $Y$ defined by  $$d\nu_Y:=d\rho d\tilde{\eta},$$
where $d\tilde{\eta}:=\frac{1}{\vol(G/\Ga)}d\eta$, satisfies \eqref{unfolding haar measure on G using section}.

%For g_1\in G with decomposition %$g_1^{-1}:=k\begin{bmatrix}
%G_1 & 0 \\ G_3 & G_4
%\end{bmatrix}, k\in \SO(m+1,\R)$. 

Fix $Hg_0\in H\backslash G$. By identifying $H \backslash G$ with $Y$, we define a measure on $H \backslash G$ via
\begin{equation}\label{relation between measures on H G and Y}
    d\nu_{Hg_0}(Hg):=\alpha(\sigma(Hg_0),\sigma(Hg))d\nu_{Y}(\sigma(Hg)).
\end{equation}
where $\alpha(\cdot,\cdot)$ is given in \eqref{eq:alpha}. 

\iffalse
Then it follows from Theorem 2.3 of \cite{Gorodnik2004DistributionOL} that for any compactly supported function $\varphi \in C_c(H\backslash G)$, we have
\begin{equation}
    \lim_{T\to \infty}\frac{1}{\mu(H_T)}\int_{G_T}\varphi(Hg_0g)dg=\int_{H\backslash G} \varphi d\nu_{Hg_0}.
\end{equation}
\fi
\vspace{3mm}
In view of Theorem 2.2 and Corollary 2.4 in \cite{Gorodnik2004DistributionOL} (duality principle), an immediate consequence of Theorem \ref{The $G$-invariance of limiting measure} is the following

\begin{corollary}
Fix $g_0\in G$. For any compactly supported $\varphi \in C_c(H\backslash G)$,
\begin{equation}
    \lim_{T\to \infty}\frac{1}{\mu(H_T)}\sum_{\gamma \in \Ga_T}\varphi(Hg_0.\gamma)= \int_{H\backslash G}\varphi(Hg)d\nu_{Hg_0}(Hg).
\end{equation}
\end{corollary}

%This proves Theorem \ref{equidistribution result on  G mod H} with $\tilde \nu_{x_0}=c_{\Gamma}\nu_{x_0}$, where $x_0\in X_{m,m+1}$ is identified with $Hg_0\in H\backslash G$ and 

%\frac{1}{m(G/\Ga)}

Now we would like to replace the normalization factor $\mu(H_T)$ by $\# \Ga_T$.
By Theorem 1.7 of \cite{GorodnikNevo2012},
\begin{equation}
    \lim_{T\to \infty}\frac{\vol(G_T)}{\# \Ga_T}=1.
\end{equation}

\iffalse
On the other-hand, by Lemma 6 of \cite{Gorodnik2003LatticeAO} (we choose the standard spherical coordinate parameterization under which the volume of $\SO(m+1,\R) \slash\SO(m,\R)$ is identified with the surface area of $\mathbb S^m$, which is $\frac{2\pi^{\frac{m+1}{2}}}{\Ga((m+1)/2)}$ and thus $\vol(\SO(m+1,\R))=\frac{2^m\pi^{\frac{m(m+3)}{4}}}{\prod_{k=2}^{m+1}\Ga(k/2)}$), we have
\begin{equation}
    \vol(G_T)\sim \vol(\SO(m+1,\R))\cdot \frac{\pi^{\frac{m(m+1)}{4}}}{2^m\Ga\left(\frac{m(m+1)}{2}+1 \right)}\prod_{k=1}^m \Ga(k/2)\cdot T^{m(m+1)}=\frac{2\pi^{\frac{m(m+2)}{2}}}{m(m+1)}.
\end{equation}
\fi
We notice that by formula A.1.15 in \cite{DRS93} and by Proposition \ref{computation for H and V}, the limit $$L:=\lim_{T\to \infty}\frac{\mu(H_{T})}{\vol(G_T)}$$ exists. Thus we conclude
\begin{equation}
    \lim_{T\to \infty}\frac{\mu(H_{T})}{\#\Ga_T}=\lim_{T\to \infty}\frac{\mu(H_{T})}{\vol(G_T)}\frac{\vol(G_T)}{\#\Ga_T}=\lim_{T\to \infty}\frac{\mu(H_{T})}{\vol(G_T)}=L,
\end{equation}
which shows that $$\lim_{T\to \infty}\frac{1}{\#\Ga_T}\sum_{\gamma \in \Ga_T}\varphi(Hg_0.\gamma)= L\int_{H\backslash G}\varphi(Hg)d\nu_{Hg_0}(Hg).$$

Our goal now will be to show that $L\nu_{Hg_0}(H\backslash G)=1$. We first show that the total measure $\nu_{Hg_0}(H\backslash G)$ is independent of the choice of the base point $Hg_0$. Namely $\nu_{Hg_0}(H\backslash G)=\nu_{H}(H\backslash G)$ for all $Hg_0\in H \backslash G$.  For $g_0,y \in Y$, we consider the decompositions 
\begin{equation} \label{decompositions of g_0 and y in Y}
    g_0:=\begin{bmatrix} G_0 & 0 \\0 & 1 \end{bmatrix}\rho_0,~ y:=\begin{bmatrix} \eta & 0 \\0 & 1 \end{bmatrix} \rho,
\end{equation}
where $G_0,\eta\in\mathcal{F}_m$ and $\rho_0,\rho\in\SO(m+1,\R).$ Hence $\alpha$ in \eqref{eq:alpha} takes a more simpler form (note $g_0^{-1}:=\rho_0^{-1}\begin{bmatrix} G_0^{-1} & 0 \\0 & 1 \end{bmatrix}$)
\begin{align*}
    \alpha(g_0,y)
    =& \frac{\sum_{q\in \Delta}\frac{1}{\|G_0^{-1}q \eta\|^{m^2}}}{\sum_{q\in \Delta}\frac{1}{\|q \|^{m^2}}}.
%    =& C_{\alpha}(g_0^{-1},y)\frac{\sum_{q\in \Delta}\frac{1}{\|G_1^{-1}q H_1\|^{m^2}}+\sum_{q\in \SL_{- 1}(2,\Z)}\frac{1}{\|G_1^{-1}q H_1\|^{m^2}}}{\sum_{q\in \Delta}\frac{1}{\|q \|^{m^2}}}.
  \end{align*}


Using the invariance of the $\SO(m+1,\R)$ invariance of the Hilbert-Schmidt norm, we compute 
\begin{align*}\label{folding trick}
    &\nu_{Hg_0}(H\backslash G)\\
    =&\int_{Y}\alpha(g_0,y)d\nu_{Y}\\
 %   =& \vol(\SO(m+1,\R)) \int_{\mathcal{F}_m}2\pi dg\\
    =&\frac{1}{\sum_{q\in \Delta}\frac{1}{\|q \|^{m^2}}} \int_{\SO(m+1,\R)}\int_{\mathcal{F}_m}\sum_{q\in \Delta}\frac{1}{\|G_0^{-1}q \eta\|^{m^2}} d\tilde{\eta}d\rho \tag{definition of $d\nu_Y$}\\
    =&\frac{1}{\sum_{q\in \Delta}\frac{1}{\|q \|^{m^2}}} \int_{\SL(m,\R)}\frac{1}{\|G_0^{-1}\eta\|^{m^2}}  d\tilde{\eta}\tag{by formula \eqref{unfolding haar measure on G using section}}\\
    =&\frac{1}{\sum_{q\in \Delta}\frac{1}{\|q \|^{m^2}}} \int_{\SL(m,\R)}\frac{1}{\|\eta\|^{m^2}}  d\tilde{\eta}\tag{invariance of $d\tilde{\eta}$}\\
    =&\nu_{H}(H\backslash G).
\end{align*}
This also confirms that $\nu_{Hg_0}(H\backslash G)$ is finite. Finally, we prove

\begin{proposition}
    $\nu_{Hg_0}(H\backslash G)=\nu_{H}(H\backslash G)=\lim_{T\to \infty}\frac{\vol(G_T)}{\mu(H_T)}=\frac{1}{L}$.
\end{proposition}

\begin{proof}
%[Proof of the claim]\renewcommand{\qedsymbol}{\ensuremath{\#}}
    %The proof here follows from Theorem 2.3 in \cite{Gorodnik2004DistributionOL} and its proof.  
$\nu_H$ is a Radon measure since is finite and Borel. So for any $\e>0$, we can choose $f_{\e}\in C_c(H\backslash G)$ with support $B_{\e}$ (note $B_{\e}$ is bounded) such that (recall that $Y$ is a lift of $H\backslash G$ to G such that $H\times Y\to G$ is a Borel isomorphism)
    \begin{equation}
        \int_{Y}|f_{\e}(Hy)-1|\alpha(e,y)d\nu_Y(y)=\int_{H\backslash G}|f_{\e}(Hg)-1|d\nu_H(Hg)\le \e.
\end{equation}
As in \cite{Gorodnik2004DistributionOL}, we observe that    \begin{align}
        \frac{1}{\mu(H_T)}\int_{G_T}|f_{\e}(Hg)-1|dg
        =&\frac{1}{\mu(H_T)}\int_Y\int_{\{h:\|hy\|<T\}}|f_{\e}(Hy)-1|d\mu(h)d\nu_Y(y)\\
        =&\frac{1}{\mu(H_T)}\int_Y |f_{\e}(Hy)-1|\mu(\{h:\|hy\|<T\})d\nu_{Y}(y)\\
        =&\int_Y |f_{\e}(Hy)-1| \frac{\mu(H_T[e,y])}{\mu(H_T)}d\nu_{Y}(y)
    \end{align}
Recall that $\lim_{T\to\infty}\frac{\mu(H_T[e,y])}{\mu(H_T)}=\alpha(e,y).$
We will use below the dominated convergence theorem, and for that we will now show that the integrand $\frac{\mu(H_T[e,y])}{\mu(H_T)}$ is bounded by a function in $L^1(Y)$ for large $T$. Recall that $\mu(H_T)\sim T^{m(m+1)}$ and it only depends on the variable $T$ (constant over $Y$).

Here since $g_1=e$ and $g_2=y$, $A_m(q)=\|G_0^{-1}q \eta\|=\|q \eta\|$ in view of \eqref{decompositions of g_0 and y in Y}. Recall $Y=\mathcal{F}_m\cdot \SO(m+1,\R)$ and the factor $\SO(m+1,\R)$ does not affect the finiteness of the integral over $Y$, therefore in this proof we shall ignore the $\SO(m+1,\R)$ part and only integrate against the variable $H_1\in \mathcal{F}_m$. 

By \eqref{eq:bound on measure of Vga T} and by \eqref{eq:haar measure of skewed H ball as a sum of skewed balls on connected component}, we see  
\begin{equation}
\frac{\mu(H_T[e,y])}{\mu(H_T)} \ll \sum_{q\in \Delta}\frac{1}{\|q \eta\|^{m^2}}.
\end{equation}
By Lemma \ref{lemma on the integral and summation in a ball}, we get that $\Psi(y):=\sum_{q\in \Delta}\frac{1}{\|q \eta\|^{m^2}}\in L^{1}(Y)$ (more precisely, to see that this is a $L^1$ function, use \eqref{eq:unfolding formula on the double coset space} and then Lemma \ref{lemma on the integral and summation in a ball}).

By the dominant convergence theorem, the second term satisfies
\begin{equation}
     \lim_{T\to \infty}\int_{Y} |f_{\e}(Hy)-1| \frac{\mu(H_T[e,y])}{\mu(H_T)}d\nu_{Y}(y)= \int_{Y} |f_{\e}(Hy)-1| d\nu_{H}(y)\le \e
\end{equation}
    
Therefore, by triangular inequality
\begin{align*}
    &\limsup_{T\to \infty}\left|\frac{\vol(G_T)}{\mu(H_T)}-\nu_H(H\backslash G)\right|\\
    \le & \limsup_{T\to \infty}\left|\frac{\vol(G_T)}{\mu(H_T)}-\frac{1}{\mu(H_T)}\int_{G_T}f_{\e}(Hg)dg\right|\\
    &+\limsup_{T\to \infty}\left|\frac{1}{\mu(H_T)}\int_{G_T}f_{\e}(Hg)dg-\int_{H\backslash G}f_{\e}(Hg)d\nu_H(g) \right|\\
    &+\limsup_{T\to \infty}\left|\int_{H\backslash G}f_{\e}(Hg)dg-\nu_H(H\backslash G) \right|\\
    \le& \e+0+\e \tag{the middle term vanishes because of Theorem 2.3, \cite{Gorodnik2004DistributionOL}}
\end{align*}
Now let $\e\to 0$, and this finishes the proof.
\end{proof}
Therefore, $\tilde\nu_{Hg_0}:=L\nu_{Hg_0}$ is a probability measure, and we conclude 
\begin{equation}
    \lim_{T\to \infty}\frac{1}{\# \Ga_T}\sum_{\gamma \in \Ga_T}\varphi(Hg_0.\gamma)= \int_{H\backslash G}\varphi(Hg)d\tilde \nu_{Hg_0}(g).
\end{equation}
\iffalse
\vspace{5mm}
\emph{Concluding the proof of Theorem \ref{equidistribution result on  G mod H in Xm,m+1}.}
We express $\tilde\nu_{Hg_0}$ as $$d\tilde\nu_{Hg_0}=\Phi_{Hg_0}(y)dg d\rho,$$ where $$\Phi_{Hg_0}(y):=\sum_{q\in \GL(m,\Z)}\frac{1}{\|G_1^{-1}q H_1\|^{m^2}},$$  $\Phi_{Hg_0}(y)dg$ is a probability measure on $\mathcal{F}_m$, and $d\rho$ is a probability measure on $\SO(m+1,\R)$. Now we interpret the coefficient $\Phi_{Hg_0}(y)$
in terms of rank-two discrete subgroups in $\R^{m}$. Let $(\La_0,w_0)$ and $(\La,w)$ denote the oriented rank-two discrete subgroups of $\R^{m}$ corresponding to $g_0$ and $y$, respectively. Let $e_1,e_2,...,e_m$ be the canonical basis of $\R^m$, written in terms of row vectors:

Then by \eqref{decompositions of g_0 and y in Y} it follows that  $$\mathscr{B}_0:=\{(e_1G_1,0)\rho_0,...,(e_mG_1,0)\rho_0 \},$$ and $$\mathscr{B}:=\{(e_1q H_1,0)\rho,...,(e_mq H_1,0)\rho\}$$ form  $\Z$-bases of $\La_0$ and $\La$, respectively.

Consider the following linear maps between two-dimensional subspaces of $\R^{m}$:
\begin{equation}
    T_{\mathscr{B}_0}: \Span_{\R}\{(e_1,0),...,(e_m,0)\} \to \Span_{\R}\{\mathscr{B}_0\}, (e_i,0)\mapsto (e_iG_1,0)\rho_0, i=1,2,...,m
\end{equation}
and
\begin{equation}
    T_{\mathscr{B}}: \Span_{\R}\{(e_1,0),...,(e_m,0)\} \to \Span_{\R}\{\mathscr{B}\}, (e_i,0)\mapsto (e_iq H_1,0)\rho, i=1,2,...,m.
\end{equation}


Recall from the introduction (see \eqref{defining hilbert-schmits for op. from hyp. to hyp.}) the Hilbert-Schmidt norm   $\|T\|_{\text{HS}}$ of an operator $T$ from the two-dimensional subspace $U\subset \R^{m}$ to the two-dimensional subspace $V\subset \R^{m}$, which is computed by choosing an orthonormal basis $\{u_1,u_2,...,u_m\}$ of $U$ and evaluating 
\begin{equation}
    \|T\|^2_{\text{HS}}:=\sum_{i=1}^m\|Tu_i\|^2,
\end{equation}
where the norm on the right hand side is the usual Euclidean norm on $\R^{m}$. Using the orthonormal basis $\{(e_1,0)\rho_0 , (e_2,0)\rho_0,...,(e_m,0)\rho_0\}$ of $\Span_\R\{\mathcal{B}_0\}$, we get $$\|T_{\mathscr{B}}\circ T_{\mathscr{B}_0}^{-1}\|_{\text{HS}}^2=\|G_1^{-1}q H_1\|^2,$$
which shows
\begin{equation}\label{intrinsic interpretation of the sum}
    \Phi_{g_0}(y)=\sum_{q\in \GL(m,\Z)}\frac{1}{\|G_1^{-1}q H_1\|^{m^2}}=\sum_{\Span_{\Z}\mathscr{B}=\La}\frac{1}{\|T_{\mathscr{B}}\circ T_{\mathscr{B}_0}^{-1}\|_{\text{HS}}^{m^2}}=\Psi_{\La_0}(\La),
\end{equation}
 

where $\Psi_{\La_0}(\La)$ defined in \eqref{defining psi_La_0}.
It's now straight forward to verify that the measure $\tilde\nu_{x_0}$ for $x_0=(\Span_\Z\{e_1,e_2,...,e_m\},e_{m+1})\cdot g_0$ defined before Theorem \ref{equidistribution result on  G mod H} is identified with $\tilde\nu_{H_{g_0}}$, which we leave to the reader to verify.
\fi
\iffalse

\tilde\nu_{x_0}=\tilde\nu_{Hg_0}$ under the identification of $X_{m,m+1}$ with $H\backslash G$ and $x_0$ with $Hg_0$.

Let $X_m=\Delta\backslash\SL(m,\R)$. For any $f\in C_c(X_{m,m+1})$, we have
\begin{align}
    \tilde\nu_{x_0}(f)
    :=&\int_{\mathbb S^2}\int_{\pi_{\perp}^{-1}(w)}f((\La,w),\rho_w) \Phi_{\La_0}(\La)d((\rho_w)_*\mu_{e_{m+1}})(\La)d\mu_{\mathbb S^2}(w)\\
    =&\int_{\mathbb S^2}\int_{\pi_{\perp}^{-1}(e_{m+1})}f((\La,e_{m+1}),\rho_w) \Psi_{\La_0}(\La)d\mu_{e_{m+1}}(\La)d\mu_{\mathbb S^2}(w)\\
    =&\int_{\mathbb S^2}\int_{X_m}f(\Z^2\times 0,\eta\rho_w) \Psi_{\La_0}(\Z^2\times 0)d\eta d\mu_{\mathbb S^2}(w)\\
    =&\frac{1}{\vol(\SO(m,\R))}\int_{\mathbb S^2}\int_{\SO(m,\R)}\int_{X_m}f(\Z^2\times 0,\eta k \rho_w) \Psi_{\La_0}(\Z^2\times 0)d\eta dk d\mu_{\mathbb S^2}(w)\\
    =&\frac{1}{\vol(\SO(m,\R))}\int_{ \SO(m+1,\R)}\int_{X_m}f(\Z^2\times 0,\eta\rho) \Psi_{\La_0}(\Z^2\times 0)d\eta d\rho\\ 
    =&\frac{1}{\vol(\SO(m,\R))}\int_{ \SO(m+1,\R)}\int_{\mathcal F_2}\int_{\SO(m,\R)}f(\Z^2\times 0,\eta\rho k) \Psi_{\La_0}(\Z^2\times 0)d\eta d\rho dk\\
    =&\int_{ \SO(m+1,\R)}\int_{\mathcal F_2}f(\Z^2\times 0,\eta\rho) \Psi_{\La_0}(\Z^2\times 0)d\eta d\rho
\end{align}

The last line is the same as $\int_{H\backslash G}\varphi(Hg)d\tilde\nu_{Hg_0}(g)=\int_{H\backslash G}\varphi(Hg)d\tilde c_{\Ga}\nu_{Hg_0}(g)$ (note that both $\tilde \nu_{x_0}$ and $\tilde \nu_{Hg_0}$ are probability measures).
This proves Theorem \ref{equidistribution result on  G mod H}. 
\fi
%Since we are integrating with respect to the variable $y$. It is an interesting question how $\alpha(g_1,y)$ depends $g_1$.

\section*{Acknowledgements}
The authors would like to thank Nimish Shah for helpful discussions and generous sharing of his ideas leading to the proof of Lemma \ref{our lemma expansion inequality for applying shah dichotomy}. We thank Uri Shapira for many helpful discussions and encouragement. We  thank Alex Gorodnik and Barak Weiss for their comments on a previous version of our manuscript. We would also like to thank Osama Khalil for discussions related to this work. This work has received funding from the European Research Council (ERC) under the European Union’s Horizon 2020 Research and Innovation Program, Grant agreement No. 754475.

\printbibliography[
heading=bibintoc,
title={Bibliography}
]
\end{document}