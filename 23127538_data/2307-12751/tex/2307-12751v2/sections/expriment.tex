

\section{Experiments}\label{sec:experiment}
%
We first introduce training and evaluation configurations of the proposed ICF-SRSR framework.
%
Then we conduct comprehensive experiments, extensive quantitative and qualitative comparisons with the other methods, and an in-depth analysis of our proposed method.

\subsection{Training details}
%
\Paragraph{Dataset.}
%
We train and evaluate our method on two scenarios. 1) Synthetic datasets, where the training and testing LR images are synthesized by a uniform degradation process~(\eg, bicubic down-sampling) from HR images. 2) Real-world datasets, which provide paired LR-HR images from the real-world captured by adjusting the focal length of a camera.

To train our ICF-SRSR, we use $800$ bicubic LR images from the DIV2K~\cite{agustsson2017ntire} dataset.
%
For evaluation, we adopt five standard benchmarks: Set5~\cite{bevilacqua2012low}, Set14~\cite{zeyde2010single}, BSD100~\cite{martin2001database}, Urban100~\cite{huang2015single}, and Manga109~\cite{Manga109}.
%
We also use the high-quality DIV2K validation set for evaluation.
%

To train and evaluate our ICF-SRSR under real-world scenarios, we utilize real-world datasets~\cite{cai2019toward, wei2020component} for the SISR task.
%
RealSR-V3~\cite{cai2019toward} includes paired LR-HR images captured by two different cameras, Canon and Nikon.
%
For each camera, about $200$ training images are captured from different scenes for each scaling factor $\times2$, $\times3$, and $\times4$.
%
We use only the LR images with scaling factors $\times2$ and $\times4$ for training and evaluate our method on the $50$ test pairs for each scale. 
%
DRealSR~\cite{wei2020component} also contains images captured by five DSLR cameras.
%
We conduct our experiments using images for $\times 2$ and $\times 4$ SR, containing 884 and 840 LR images, respectively.
%
For evaluation, we use $83$ and $93$ test pairs in DRealSR for $\times 2$ and $\times 4$, respectively.
\begin{table*}[t]
    \small
    \centering
    %\setlength\tabcolsep{1pt}
    \begin{tabularx}{\linewidth}{c c l c >{\centering\arraybackslash}X
    >{\centering\arraybackslash}X >{\centering\arraybackslash}X >{\centering\arraybackslash}X >{\centering\arraybackslash}X}
    \toprule
    \multirow{2}{*}{\textbf{Training Set}}& 
    \multirow{2}{*}{\textbf{Supervision}}&
    \multirow{2}{*}{\bf Method} & 
    %\multicolumn{3}{c}{\textbf{City100}}  & 
    %\multicolumn{3}{c}{\textbf{RealSR}} &
    \multicolumn{2}{c}{\textbf{RealSR~(Canon)}} & \multicolumn{2}{c}{\textbf{RealSR~(Nikon)}} &  \multicolumn{2}{c}{\textbf{DRealSR}} \\
    %\cline{4-7}
    & & &  \bf $\times2$ & \bf $\times4$ & \bf $\times2$ & \bf $\times4$  & \bf $\times2$ & \bf $\times4$\\ 
    %\cline{4-7}
    \midrule
    & & Bicubic & {30.35}  &  {25.80} & {29.66} & {25.50} & {32.67} & {30.56}\\
    \midrule
    \multirow{5}{*}{Synthetic} & \multirow{5}{*}{Supervised} 
    %& Bicubic & {}  & {-}& {-} & {-} & {-}\\
    & EDSR~\cite{lim2017enhanced}& \textbf{30.58}  & {26.05} &  \textbf{30.00}  & {25.89} & \textbf{32.82} & \textbf{30.64}\\
    & & RRDB~\cite{wang2018esrgan}& {-} & {26.05} &  {-}  & {25.91} & {-}& {-}\\
    &  & IKC~\cite{gu2019blind} & {-} & {25.71} &  {-}  & {25.27}& {-}& {-}\\
    & & BilndSR~\cite{sr_blindsr}& {25.80}   & {-}&  {24.17} & {-}& {-}& {-}\\
    & & DRN-S~\cite{guo2020closed} & \textbf{30.58} & \textbf{26.07} &  {29.99}  & \textbf{25.92} & {32.81}& {-}\\
    %&  & \textbf{IMF-SRSR$^{\dagger}$} & {30.62}  &  {-} & {30.00} & {-}& {32.84}& {-}\\
    %&  & \textbf{IMF-SRSR$^{\ddagger}$} & {}  &  {-} & {} & {-}& {-}& {-}\\
    %& \textbf{IMF-SRSR~(EDSR~(HR))} & {32.56}  &  {} & {} & {}\\    
    \midrule
    %BM3D~\cite{sparse} & {25.65} & {0.685}& {34.51} & {0.850} & {25.65} & {0.685}& {34.51} & {0.850} \\
    %WNNM~\cite{6909762}& {25.78} & {0.809} & {34.67} & {0.864} & {25.65} & {0.685}& {34.51} & {0.850} \\
    %K-SVD~\cite{DBLP:journals/corr/abs-1909-13164} & {26.88} & {0.842}& \textbf{36.49} & \textbf{0.897}& {25.65} & {0.685}& {34.51} & {0.850} \\
    %EPLL~\cite{Hurault_2018} & \textbf{27.11} & \textbf{0.870}  & {33.51} & {0.824}& {25.65} & {0.685}& {34.51} & {0.850} \\
    %\hline
    \multirow{8}{*}{Real-world} & \multirow{5}{*}{Supervised}
    & EDSR~\cite{lim2017enhanced}& {32.45}   & {27.59} &  \textbf{31.59} & {27.14} & {34.24}& \textbf{32.03}\\
    & & RRDB~\cite{wang2018esrgan}& {-} & \textbf{27.90} & {-} & \textbf{27.39} & {-}& {-}\\
    & & RCAN~\cite{zhang2018image}& {-} & {-} & {-} & {-} & \textbf{34.34} & {31.85}\\
    & & LP-KPN~\cite{cai2019toward}& {-} & {27.40} & {-} & {26.69} & {33.88} & {31.58}\\
    & & DRN-S~\cite{guo2020closed} & \textbf{32.50} & {-} & {31.43} & {-} & {33.91} & {-} \\
    %& & \textbf{IMF-SRSR}$^{\dagger}$ & {\textbf{32.58}} & {-} & {-} & {-} & {-} & {-} \\
    %& & IMF-SRSR$^{\ddagger}$ & {\textbf{32.68}} & {-} & {-} & {-} & {-} & {-} \\
    \cmidrule{2-9}
    & Unsupervised
    & ZSSR~\cite{shocher2018zero}+~\cite{bell2019blind} & {28.79} & {23.68} &  {27.54}  & {22.46}& {-}& {-}\\
    \cmidrule{2-9}
    & \multirow{2}{*}{Self-supervised} 
    %\textbf{IMF-SRSR~(Test)} & {30.67}  &  {26.08} & {29.99} & {25.76}& {32.83}& {30.62}\\
    & \textbf{ICF-SRSR} & {30.98}  &  {26.26} & {30.31} & {25.89}& {32.87}& {30.65}\\
    & & \textbf{EDSR}~(LLR,LR) & \textbf{31.13}  &  \textbf{26.32} & \textbf{30.33} & \textbf{25.92} & \textbf{32.91}& \textbf{30.68}\\
    \bottomrule
    \end{tabularx}
    %\vspace{1mm}
    \caption{
        \textbf{Quantitative comparison on real-world datasets.} We compare our self-supervised ICF-SRSR and EDSR~(LLR,LR), \ie, the model EDSR~\cite{lim2017enhanced} trained on our generated paired dataset (LLR,LR), to several supervised/unsupervised methods trained on synthetic DIV2K~\cite{agustsson2017ntire}, real-world RealSR-V3~\cite{cai2019toward} and DRealSR~\cite{wei2020component} datasets for scales $\times2$ and $\times4$.
        %
    }
    \label{tab:real_data}
    \vspace{-4mm}
\end{table*}


%
\Paragraph{Hyperparameters.}
%
During the training, we extract random patches of size $48\times48$ from LR images of both synthetic and real-world datasets.
%
For all our experiments, we set the batch size to $16$, and  $\lambda_{\text{Color}}=0.2$.
%
Random flip and rotation augmentations are applied to the input images to increase the number of effective training samples. 
%
We train our model using ADAM~\cite{kingma2017adam} optimizer with the initial learning rate $1 \times 10^{-4}$, which decays by a factor $0.5$ after every $200$ epochs.
%
For quantitative comparisons, we adopt structural similarity~(SSIM)~\cite{measure_ssim} and peak signal-to-noise ratio~(PSNR) on the luminance channel for the experiments on synthetic datasets and real-world dataset DRealSR~\cite{wei2020component} and also on RGB channels for dataset RealSR-V3~\cite{cai2019toward}.
%
All experiments are done using PyTorch 1.8.1 and Quadro RTX 8000 GPUs.
%%%%%%%%%%%%%%%%%%%%%%%%%%%%%%%%%%%%%%%%%%%%%%%%%%%%%%%%%%%%%%%%%%%%%%%%%%%%%%%%%%%%%%%%
\subsection{Evaluation on synthetic datasets}\label{sec:ex_syn}
%
We train our ICF-SRSR on the DIV2K~\cite{agustsson2017ntire} dataset with EDSR-baseline~\cite{lim2017enhanced} and test it on the public benchmark datasets~\cite{bevilacqua2012low, zeyde2010single, martin2001database, huang2015single, Manga109} and also the validation set of DIV2K.
%
We note that the proposed method is trained in a self-supervised manner by targeting a certain scale $s$.
%
Specifically, we train $\left( \times 2, \times \nicefrac{1}{2} \right)$ ICF and $\left( \times 4, \times \nicefrac{1}{4} \right)$ ICF independently.
%
\cref{tab:benchmark} shows extensive comparisons between the proposed self-supervised approach and the other representative supervised/unsupervised SR methods with PSNR metric.
%
We demonstrate that our ICF-SRSR approach achieves superior performance compared to the SelfExSR~\cite{huang2015single} model and comparable performance to the other unsupervised and supervised methods.
%
We note that ground-truth HR images in Set5 and Set14 are relatively noisier than the other datasets, preventing our self-supervised framework from learning accurate scaling functions.
%
We will discuss more details about the noisy cases in our supplementary material.
%
Notably, ICF-SRSR outperforms the unsupervised method ZSSR~\cite{shocher2018zero} by $1.05$dB on scale $\times2$ of Urban100 dataset and the supervised methods~\cite{kim2016accurate, chen2021learning} on both scales of DIV2K validation set.

Moreover, we apply the trained ICF-SRSR to LR images from the DIV2K training dataset and get LLR-LR paired images.
%
Then, we train off-the-shelf EDSR on the synthesized paired data from scratch and evaluate it on the test datasets as shown in \cref{tab:benchmark}.
%
The results demonstrate that EDSR~(LLR, LR) trained on our generated pairs~(LLR, LR) achieves superior performance than ICF-SRSR, which illustrates the merit of our method to generate useful training image pairs.
%

\cref{fig:benchmark} further visualizes the qualitative results of ICF-SRSR on two validation images from the DIV2K~\cite{agustsson2017ntire} dataset.
%
Our method achieves comparable results to the supervised methods~\cite{lim2017enhanced,chen2021learning} while restoring more details compared to the unsupervised methods~\cite{shocher2018zero,soh2020meta}. 
%
We note that the results on ZSSR~\cite{shocher2018zero} show lost information and scratched texts, and on MZSR~\cite{soh2020meta} include severe artifacts and color shifting.
%
For an in-depth comparison, we also provide quantitative results with SSIM metric in our supplementary material.
%

\subsection{Evaluation on real-world datasets}\label{sec:ex_real}
%
We train and evaluate ICF-SRSR for each scale $\times 2$ and $\times 4$ independently on the LR images of each Canon and Nikon camera from the real-world dataset RealSR-V3~\cite{cai2019toward} separately and also on the LR images of the real-world dataset DRealSR~\cite{wei2020component} in a self-supervised manner.
%
We further train the model EDSR~\cite{lim2017enhanced} on our generated~(LLR, LR) image pairs. 
%
We compare our method with the supervised methods~\cite{lim2017enhanced, wang2018esrgan, zhang2018image, cai2019toward, guo2020closed} trained on real paired images, which serve as the upper bounds for the SR problem.
%
\input{sections/figures/realsr2}

On the other hand, we employ the pre-trained supervised models EDSR~\cite{lim2017enhanced}, RRDB~\cite{wang2018esrgan}, IKC~\cite{gu2019blind}, BlindSR~\cite{sr_blindsr} and DRN-S~\cite{guo2020closed} on the synthetic DIV2K~\cite{agustsson2017ntire} dataset to super-resolve the LR images in the testing sets of RealSR-V3~\cite{cai2019toward} and DRealSR~\cite{wei2020component}.
%
Moreover, we utilize Kernel-GAN~\cite{bell2019blind} to approximate the down-sampling kernel from a single LR image and use ZSSR~\cite{shocher2018zero} as a zero-shot SR to apply to real LR images.
%
Our extensive comparisons with the various methods trained on real and synthetic datasets are summarized in \cref{tab:real_data}.
%
We illustrate that our self-supervised method can achieve superior performance compared to the methods pre-trained on the synthetic datasets and unsupervised method ZSSR~\cite{shocher2018zero}+Kernel-GAN~\cite{bell2019blind} in terms of both PSNR and SSIM metrics, which emphasizes the fact that the trained models on synthetic datasets with known degradations cannot perform well on real-world scenarios.
%
We qualitatively compare our method with the various existing methods on the RealSR-V3 dataset and visualize the SR results and their corresponding error maps with respect to the GT~(HR) in \cref{fig:realsr}. 
%
We demonstrate that our self-supervised method can achieve comparable and sometimes better performance to the supervised method LP-KPN~\cite{cai2019toward} trained on real paired images. 
%
We note that our method is generally more suitable for restoring the texture and preserving color compared to supervised method IKC~\cite{gu2019blind} and unsupervised method ZSSR~\cite{shocher2018zero}+Kernel-GAN~\cite{bell2019blind} as evident in appearance and PSNR, SSIM, and mean absolute error~(MAE) metrics.
%
We show more qualitative results in the supplementary material.
%
\subsection{Ablation study}\label{sec:ablation}
%
We conduct various ablation studies on the model design, down-sampling operators, few-shot learning, augmentation, and the effect of loss functions to better analyze our method.

\Paragraph{Model design.}
%
We conduct an experiment to show the superiority of a developed baseline as a single conditional model compared to two independent models and also the effect of training our two-stage framework compared to training each Up-Down and Down-Up stage separately.
%
Our results on synthetic dataset DIV2K~\cite{agustsson2017ntire} and Canon and Nikon images from real-world dataset RealSR-V3~\cite{cai2019toward} for scale $\times2$ show that training with two independent models or using only one stage~(half) results in unsatisfactory performance, demonstrating the uniqueness of our method in using a single invertible scale-conditional model as shown in \cref{tab:reb-two-models}.

\begin{table}[h]
    %\vspace{-3mm}
    \small
    \centering
    %\setlength\tabcolsep{0.5pt}
    %\setlength\tabcolsep{0.0001pt}
    %\resizebox{0.65\columnwidth}{!}{
    \begin{tabularx}{\linewidth}{l 
    >{\centering\arraybackslash}X 
    >{\centering\arraybackslash}X 
    >{\centering\arraybackslash}X  
    }
    \toprule
    \textbf{Method} 
    & \textbf{DIV2K~($\times 2$)}&
    \textbf{Canon~($\times 2$)} & \textbf{Nikon~($\times 2$)}  \\
    %\cline{4-7}
    \midrule
     Two Models & {34.81} & {30.61} & {30.01} \\
     Up-Down & {29.92}& {28.56} & {27.52}  \\
     Down-Up & {34.59}& {30.58} & {30.00} \\
     \midrule
     \textbf{ICF-SRSR} & \textbf{35.19} & \textbf{30.98} & \textbf{30.31} \\
    \bottomrule
    \end{tabularx}
    \vspace{-2mm}
    \caption{
    \textbf{Ablation on model design.}
        %
    }
    \label{tab:reb-two-models}
    \vspace{-2mm}
\end{table}



%

\Paragraph{Evaluation of down-sampling.} 
%
Due to the invertibility attribute of ICF, our method can be interpreted as a learnable down-sampler.
%
Therefore, we analyze our model $f_{\theta}$ as a down-sampling operator in three aspects.
%

\Paragraph{First.} We train ICF-SRSR on HR images from RealSR-V3~\cite{cai2019toward} and evaluate the model on HR images of the test dataset to gather the generated down-sampled images. 
%
Then, we compare ground-truth LR images with our generated LR images, as well as LR images obtained by down-sampling functions \eg, Nearest, Bicubic, Gaussian+Nearest, and Gaussian+bicubic~($\sigma=0.4$).
%
\cref{tab:downsampling} provides a comparison of LR images for different down-sampling models based on PSNR.
%
The values show the superiority of our learnable down-sampling method in generating more realistic LR images compared to ones with other down-sampling operators. 

\begin{table}[h]
    \small
    \centering
    %\setlength\tabcolsep{0.5pt}
    \begin{tabularx}{\linewidth}{l >{\centering\arraybackslash}X >{\centering\arraybackslash}X >{\centering\arraybackslash}X  >{\centering\arraybackslash}X
    }
    \toprule
    \multirow{2}{*}{\textbf{Down-sampling}} & 
    \multicolumn{2}{c}{\textbf{Canon}} & 
    \multicolumn{2}{c}{\textbf{Nikon}} \\
    %\cline{4-7}
    & $\times2$ & $\times4$ & $\times2$ & $\times4$\\
    %\cline{4-7}
    
    \midrule
     Nearest & {29.35} & {24.51} & {28.54} & {23.91} \\
     Bicubic & {30.27} & {25.76} & {29.71} & {25.56} \\
     Gaussian+Nearest & {29.62} & {24.65} & {28.87} & {24.09} \\
     Gaussian+Bicubic & {30.61} & {25.95} & {30.12} & {25.81} \\
     
     \midrule
     \textbf{ICF-SRSR} &  \textbf{32.46} & \textbf{28.93} & \textbf{32.12} & \textbf{29.15}\\
    \bottomrule
    \end{tabularx}
    
    \vspace{-2mm}
    \caption{
        \textbf{Ablation on down-sampling performance.} %Comparisons of the generated LR images from the given HR images using various known down-sampling operators with our learnable down-sampling method ICF-SRSR.
        %
    }
    \label{tab:downsampling}
    \vspace{-4mm}
\end{table}

\Paragraph{Second.} 
%
We further analyze our learnable down-sampling operator $f_{\theta}$ compared to non-learnable down-sampling approaches.
%
We use our learnable down-sampling operator $f_{\theta}$, bicubic down-sampling, and Gaussian~($\sigma=0.4$) filtering followed by different nearest and bicubic down-sampling operators to generate the LLR images from given input LR images on the training sets.
%
Then, we train the model EDSR on the generated paired images~(LLR, LR) to learn generating SR images given LR counterparts.
%
We summarize the results for scale $\times2$ of the benchmarks Set5~\cite{bevilacqua2012low} and Set14~\cite{zeyde2010single}, and Canon and Nikon sets of RealSR-V3~\cite{cai2019toward} dataset for both non-learnable and our learnable down-sampling operators in \cref{tab:downsampling2}.
%
The results indicate the effect of our learnable down-sampling operator to generate appropriate image pairs for training, which results in a significant improvement compared to known down-sampling operators.

\begin{table}[h]
    \small
    \centering
    %\setlength\tabcolsep{0.5pt}
    \begin{tabularx}{\linewidth}{l >{\centering\arraybackslash}X >{\centering\arraybackslash}X  >{\centering\arraybackslash}X 
    >{\centering\arraybackslash}X
    }
    \toprule
    \textbf{Down-sampling}  & 
    \textbf{Set5} & \textbf{Set14} & 
    \textbf{Canon} & 
    \textbf{Nikon}  \\
    \midrule
     Bicubic  & {35.30} & {31.53} & {30.41} & {29.80}\\
     Gaussian+Nearest  & {30.79} & {28.39} & {29.41} & {28.60}  \\
     Gaussian+Bicubic  & {35.43} & {31.84} & {30.47} & {29.86} \\
     \midrule
     \textbf{ICF-SRSR} & \textbf{37.09} & \textbf{32.91} &  \textbf{31.13} & \textbf{30.33}\\
    \bottomrule
    \end{tabularx}
    
    \vspace{-2mm}
    \caption{\textbf{Comparison with non-learnable down-sampling operators to generate paired training data for SR task.}
        %\textbf{SISR results using different down-sampling operators.}
        %Effects of different down-sampling operators to generate appropriate paired images for training.
        %
    }
    \label{tab:downsampling2}
    \vspace{-4mm}
\end{table}


\Paragraph{Third.}
%
By using different down-sampling methods, we first generate LR samples from the real training HR images and then train a vanilla EDSR model using the generated pairs, \ie, (LR, HR).
%
As shown in \cref{tab:reb-down}, our synthesized pairs can provide more suitable training data compared to ones by previous learnable down-sampling methods ADL~\cite{son2021toward} and DRN-S~\cite{guo2020closed} as the EDSR performs much better for the $\times 2$ SR tasks on real dataset RealSR-V3~\cite{cai2019toward}.

\begin{table}[h]
    %\vspace{-3mm}
    \small
    \centering
    %\setlength\tabcolsep{0.5pt}
    %\setlength\tabcolsep{0.0001pt}
    %\resizebox{0.65\columnwidth}{!}{
    \begin{tabularx}{\linewidth}{l 
    >{\centering\arraybackslash}X 
    >{\centering\arraybackslash}X 
    }
    \toprule
    \textbf{Downsampling} 
    & \textbf{Canon~$(\times2)$} & \textbf{Nikon~$(\times2)$} \\
    %\cline{4-7}
    \midrule
     ADL~[\textcolor{blue}{49}] & {30.76} & {30.44}  \\
     DRN-S~[\textcolor{blue}{20}] & {30.82} & {30.24}  \\
     \midrule
    \textbf{ICF-SRSR} & \textbf{31.94} & \textbf{31.24}\\
    \bottomrule
    \end{tabularx}
    \vspace{-2mm}
    \caption{\textbf{Comparison with learnable down-sampling operators to generate paired training data for SR task.}
    %    \textbf{}
        %
    }
    \label{tab:reb-down}
    \vspace{-2mm}
\end{table}





\paragraph{Few-shot learning.}
%
We train and evaluate our method on small datasets to show the advantage of our method to learning from only a few images without requiring a large-scale training dataset.
%
Therefore, we train the model ICF-SRSR~(Small) on the test sets of synthetic datasets Set14~\cite{zeyde2010single}, BSD100~\cite{martin2001database} and Urban100~\cite{huang2015single} and also real-world datasets RealSR-V3~\cite{cai2019toward} and DRealSR~\cite{wei2020component} and show their results on the corresponding test datasets in \cref{tab:ab-few-shot}. 
%
We demonstrate that our method can achieve slightly lower performance even when trained on very small datasets compared to our model ICF-SRSR~(Large) trained on large-scale training datasets.
\begin{table}[h]
    %\vspace{-3mm}
    \small
    \centering
    %\setlength\tabcolsep{0.5pt}
    \begin{tabularx}{\linewidth}{l >{\centering\arraybackslash}X
    >{\centering\arraybackslash}X
    >{\centering\arraybackslash}X >{\centering\arraybackslash}X  >{\centering\arraybackslash}X >{\centering\arraybackslash}X     }
    \toprule
    \multirow{2}{*}{\textbf{Training set}} &\multicolumn{2}{c}{\textbf{Set14}} & 
    \multicolumn{2}{c}{\textbf{BSD100}} & \multicolumn{2}{c}{\textbf{Urban100}} \\
    %\cline{4-7}
    & $\times2$ & $\times4$ & $\times2$ & $\times4$ & $\times2$ & $\times4$\\
    \midrule
     Large & {32.86} & {27.76} & {31.54} & {26.99} & {30.39} & {24.72} \\
     Small & {32.44} & {27.19} & {31.34} & {26.82} & {30.26} & {24.66}  \\
     \midrule
    \multirow{2}{*}{\textbf{Training set}} &
    \multicolumn{2}{c}{\textbf{Canon}} & 
    \multicolumn{2}{c}{\textbf{Nikon}} & \multicolumn{2}{c}{\textbf{DRealSR}} \\
    %\cline{4-7}
    & $\times2$ & $\times4$ & $\times2$ & $\times4$ & $\times2$ & $\times4$\\
    %\cline{4-7}
    \midrule
     Large & {30.98} & {26.26} & {30.31} & {25.89} & {32.87} & {30.65} \\
     Small & {30.67} & {26.08} & {29.99} & {25.76} & {32.83} & {30.62}  \\
    \bottomrule
    \end{tabularx}
    \vspace{-2mm}
    \caption{
        \textbf{Few-shot learning.}
        %
        %For Set14, BSD100, and Urban100, Large denotes the DIV2K training set, while Small corresponds to each test dataset.
        %Similarly, for Canon, Nikon, and DRealSR, Large denotes training splits of the corresponding dataset and Small is their test splits.
    }
    \label{tab:ab-few-shot}
    \vspace{-2mm}
\end{table}




\Paragraph{Multi-scale augmentation.}
%
As we mention in \cref{sec:net_arch}, augmented data with different scales can lead to performance improvement.
%
Therefore, when we train ICF-SRSR directly on the test samples, we adopt diverse scaling factors as well as their reciprocals to compensate for the limited number of training data.
%
In \cref{tab:ab-scale}, we show that increasing the number of inputs induced by various scaling factors, \eg, $\times2$, $\times4$, and $\times8$, and their inverses can lead to obtaining superior performance on the RealSR-V3~\cite{cai2019toward} dataset.
%
More details about our multi-scale augmentation strategy are described in our supplementary material.

\begin{table}[t]
    %\vspace{-3mm}
    \small
    \centering
    %\setlength\tabcolsep{0.5pt}
    %\setlength\tabcolsep{0.0001pt}
    %\resizebox{0.65\columnwidth}{!}{
    \begin{tabularx}{\linewidth}{l 
    >{\centering\arraybackslash}X 
    >{\centering\arraybackslash}X 
    >{\centering\arraybackslash}X 
    >{\centering\arraybackslash}X 
    }
    \toprule
    \textbf{Scale} 
    & \textbf{Canon~$(\times2)$} & \textbf{Nikon~$(\times2)$} \\
    %\cline{4-7}
    \midrule
     \text{2} & {30.67} & {29.99}  \\
     \text{2,4} & {30.75} & {30.09} \\
      \text{2,4,8} & \textbf{30.78} & \textbf{30.11} \\
    \bottomrule
    \end{tabularx}
    \vspace{-2mm}
    \caption{
        \textbf{Multi-scale augmentation.}
        %
    }
    \label{tab:ab-scale}
    \vspace{-2mm}
\end{table}





\Paragraph{Effects of loss functions.}
%
\textcolor{blue}
We also analyze the effect of each loss function discussed in \cref{sec:loss}.
%
As shown in \cref{tab:ab-loss}, our novel self-supervised consistency loss $\mathcal{L}^{\text{Cons}}$ can drastically improve the model performance when it is added to color preserving loss $\mathcal{L}^{\text{Color}}$ on both synthetic and real-world datasets.
%
In our supplementary material, we further discuss the effect of the weight $\lambda_{\text{Color}}$.

\begin{table}[h]
    %\vspace{-3mm}
    \small
    \centering
    %\setlength\tabcolsep{0.5pt}
    %\setlength\tabcolsep{0.0001pt}
    %\resizebox{0.65\columnwidth}{!}{
    \begin{tabularx}{\linewidth}{l 
    >{\centering\arraybackslash}X 
    >{\centering\arraybackslash}X 
    >{\centering\arraybackslash}X  
    }
    \toprule
    \textbf{Loss} 
    & \textbf{DIV2K~$(\times2)$} & \textbf{Canon~$(\times2)$} & \textbf{Nikon~$(\times2)$}\\
    %\cline{4-7}
    \midrule
     $\mathcal{L}^{\text{Color}}$ only & {30.31} & {29.12} & {28.38} \\
     $\mathcal{L}^{\text{Color}}$,$\mathcal{L}^{\text{Cons}}$ & \textbf{35.19} & \textbf{30.98} & \textbf{30.31}\\
    \bottomrule
    \end{tabularx}
    \vspace{-2mm}
    \caption{
        \textbf{Effect of loss functions.}
        %
    }
    \label{tab:ab-loss}
    \vspace{-5mm}
\end{table}





