\section{Method}\label{sec:method}
%
We first introduce an Invertible scale-Conditional Function~(ICF) to design our self-supervised real-world single image super-resolution framework~(ICF-SRSR); then, we discuss our defined loss functions and the network architecture.
%
For convenience, we denote $X\in\mathbb{R}^{H\times W\times 3}$ as the input LR image with the arbitrary size of $H$ and $W$.
%%%%%%%%%%%%%%%%%%%%%%%%%%%%%%%%%%%%%%%%%%%%%%%%%%%%%
\subsection{Invertible scale-Conditional Function}
%
For a given input $X$, a conditional function $f(X|s)$ returns different outputs for different conditions $s$.
% 
In this paper, we design an Invertible scale-Conditional Function~(ICF) as a specific conditional function, which can act as an operation and the inverse operation for different scale conditions. 
%
Without losing generality, we consider $f$ as an image-to-image mapping and $s$ as an arbitrary scaling factor, respectively.
%
Then, we can resize an arbitrary image $X$ as follows:
%%%%%%%%%%%%%%%%%%%%%%%%%%%%%%%%%%
\begin{equation}
    X_s = f \left( X| s \right),
\end{equation}
%%%%%%%%%%%%%%%%%%%%%%%%%%%%%%%%%%%
where $X_s \in \mathbb{R}^{sH \times sW \times 3}$ is a resized image.
%
Furthermore, for the same function $f$, we can get the original input $X$ again by the inverse scaling factor $\nicefrac{1}{s}$ as follows:
%%%%%%%%%%%%%%%%%%%%%%%%%%%%%%%%%%%
\begin{equation}
    X = f \left( X_s|\nicefrac{1}{s} \right).
\end{equation}
%%%%%%%%%%%%%%%%%%%%%%%%%%%%%%%%%%%
Therefore, $f$ as an ICF can project an image to its arbitrary-scale representation and back-project it to the original input for the scale conditions $s$ and $\nicefrac{1}{s}$, respectively.
\cref{fig:framework_a} illustrates the concept of our ICF.
%
We note that if $s=\nicefrac{1}{s}=1$ the function is identity which implies $f(X|1)=X$.


%%%%%%%%%%%%%%%%%%%%%%%%%%%%%%%%%%%%%%%%%%%%%%%%%%%%%

\subsection{Self-supervised SISR using ICF}
%
One of the challenges in real-world SR is that we cannot acquire the ground-truth HR image for an arbitrary LR image.
%
To overcome this limitation, we develop a novel self-supervised SR framework, ICF-SRSR, based on the concept of ICF.
%
As shown in \cref{fig:framework_b}, our method can simultaneously super-resolve and down-sample the given LR image $X$ with different scale conditions $s$ and $\nicefrac{1}{s}$, without requiring any paired/unpaired LR-HR training samples.
%
Specifically, we first parameterize an ICF $f_\theta$ with CNNs and utilize its property to optimize the model.
%
Then, we repeatedly apply $f_\theta$ to an LR image $X$ with different scale conditions to acquire two outputs $\check{X}, \hat{X} \in \mathbb{R}^{H \times W \times 3}$ as follows:
%
%%%%%%%%%%%%%%%%%%%%%%%%%%%%%%%%%%%%%%%%%%%%%%%%%%%%
%
\begin{equation}
    \begin{split}
        f_{\theta}(f_{\theta}(X|s)|\nicefrac{1}{s}) &= f_{\theta}(X_s|\nicefrac{1}{s})=\check{X}, \\
        f_{\theta}(f_{\theta}(X|\nicefrac{1}{s})|s) &= f_{\theta}(X_{\nicefrac{1}{s}}|s)=\hat{X},
    \end{split}
    \label{eq:icf}
\end{equation}
%%%%%%%%%%%%%%%%%%%%%%%%%%%%%%%%%%%%%%%%%%%%%%%%%%%%
%
where for $s>1$, $X_{s}\in \mathbb{R}^{sH \times sW \times 3}$ and $X_{\nicefrac{1}{s}} \in \mathbb{R}^{\nicefrac{H}{s} \times \nicefrac{W}{s} \times 3}$ are generated super-resolution~(SR) and low-low-resolution~(LLR) images, respectively.
%
For simplicity, we assume that both $\nicefrac{H}{s}$ and $\nicefrac{W}{s}$ are integers.
%

For an ideal ICF $f_\theta$, both $\check{X}$ and $\hat{X}$ in \cref{eq:icf} should be the same as the original LR image $X$. 
%
Therefore, we train $f_\theta$ in a self-supervised manner by reducing the distance between $X$ and the generated images $\check{X}$ and $\hat{X}$ in two stages simultaneously, as shown in \cref{fig:framework_b}.
%
In the up-down stage, we minimize the distance between $\check{X}$ and $X$.
%
By doing so, the network can learn to down-sample the generated SR image $X_s$ by restoring the output $\check{X}$ as the approximation of the original input $X$.
%
On the other hand, in the down-up stage, we aim to approximate the original input $X$ by reducing the distance between $\hat{X}$ and $X$.
%
Then, the network can learn to up-sample the generated LLR image $X_{\nicefrac{1}{s}}$.
%
Therefore, by leveraging the learned up-sampler and down-sampler applied on the generated images $X_{\nicefrac{1}{s}}$ and $X_s$, respectively, we can generate favorable SR and LLR images $X_s$ and $X_{\nicefrac{1}{s}}$ by employing the learned model $f_\theta$ on the input $X$ with the scale conditions $s$ and $\nicefrac{1}{s}$, respectively.
%

\begin{table*}[t]
    \small
    \centering
    %\setlength\tabcolsep{0.5pt}
    \begin{tabularx}{\linewidth}{c l >{\centering\arraybackslash}X >{\centering\arraybackslash}X >{\centering\arraybackslash}X >{\centering\arraybackslash}X
    >{\centering\arraybackslash}X >{\centering\arraybackslash}X
    }
    \toprule
    \multirow{2}{*}{\textbf{Supervision}} & 
    \multirow{2}{*}{\bf Method} & 
    \textbf{Set5} & 
    \textbf{Set14} &
    \textbf{BSD100} &
    \textbf{Urban100} &
    \textbf{Manga109} &
    \textbf{DIV2K}\\
    %\cline{4-7}
    & & $\times2$/$\times4$ & $\times2$/$\times4$ & $\times2$/$\times4$ & $\times2$/$\times4$ & $\times2$/$\times4$  & $\times2$/$\times4$ \\
    %\cline{4-7}
    
    \midrule
    %BM3D~\cite{sparse} & {25.65} & {0.685}& {34.51} & {0.850} & {25.65} & {0.685}& {34.51} & {0.850} \\
    %WNNM~\cite{6909762}& {25.78} & {0.809} & {34.67} & {0.864} & {25.65} & {0.685}& {34.51} & {0.850} \\
    %K-SVD~\cite{DBLP:journals/corr/abs-1909-13164} & {26.88} & {0.842}& \textbf{36.49} & \textbf{0.897}& {25.65} & {0.685}& {34.51} & {0.850} \\
    %EPLL~\cite{Hurault_2018} & \textbf{27.11} & \textbf{0.870}  & {33.51} & {0.824}& {25.65} & {0.685}& {34.51} & {0.850} \\
    %\hline
    & {\footnotesize Bicubic} & {33.66/28.42} & {30.24/26.00} & {29.56/25.96} & {26.88/23.14} & {30.80/24.89} & {31.01/26.66} \\
    \midrule
    \multirow{8}{*}{{\footnotesize Supervised}}
    & {\footnotesize VDSR~\cite{kim2016accurate}} & {37.53/31.35} & {33.03/28.01} & {31.90/27.29} & {30.76/25.18} & {37.22/28.83} & {33.66/28.17} \\
    & {\footnotesize EDSR~\cite{lim2017enhanced}} & {38.11/32.46} & {33.92/28.80} & {32.32/27.71} & {32.93/26.64} & {39.10/31.02} & \textbf{36.22}/{30.52} \\
    & {\footnotesize CARN~\cite{ahn2018fast}} & {37.76/32.13} & {33.52/28.60} & {32.09/27.58} & {31.92/26.07} & {38.36/30.47} & \hspace{9pt}{-\hspace{8pt}/30.10}  \\
    & {\footnotesize RCAN~\cite{zhang2018image}} & {38.27/32.63} & {34.12/28.87} & {32.41/27.77} & {33.34/26.82} & {39.44/31.19} & {36.13}/{30.52} \\
    & {\footnotesize RDN~\cite{zhang2018residual}} & {38.24/32.47} & {34.01/28.81} & {32.34/27.72} & {32.89/26.61} & {39.18/31.00} & {-\hspace{8pt}/\hspace{8pt}-} \\
    & {\footnotesize DRN-S~\cite{guo2020closed}} & {37.80/32.68} & {33.30/28.93} & {31.97/27.78} & {31.40/26.84} & {38.11/31.52} & {35.77}/\textbf{30.79} \\
    & {\footnotesize LIIF~\cite{chen2021learning}} & {38.17/32.50} & {33.97/28.80} & {32.32/27.74} & {32.87/26.68} & {-\hspace{8pt}/\hspace{8pt}-} & {34.99/29.27} \\
    & {\footnotesize ELAN~\cite{ELAN-light}} & \textbf{38.36}/\textbf{32.75} & \textbf{34.20}/\textbf{28.96} & \textbf{32.45}/\textbf{27.83} & \textbf{33.44}/\textbf{27.13} & \textbf{39.62}/\textbf{31.68} & {-\hspace{8pt}/\hspace{8pt}-} \\
    %&  \textbf{IMF-SRSR$^{\dagger}$} & {} & {-} & {-} & {-} & {-} & {-} \\
    %&  \textbf{IMF-SRSR$^{\ddagger}$} & {-} & {-} & {-} & {-} & {-} &  {-} \\
    \midrule
    \multirow{4}{*}{\footnotesize Unsupervised} 
    & {\footnotesize SelfExSR~\cite{huang2015single}}  & {36.49/30.31} & {32.22/27.40} & {31.18/26.84} & {29.54/24.82} & {35.78/27.82} & {-\hspace{8pt}/\hspace{8pt}-} \\
    & {\footnotesize ZSSR~\cite{shocher2018zero}}  & {37.37/31.13} & {33.00/28.01} & {31.65/27.12} & {29.34/24.12} & {35.57/27.04} & \textbf{34.45}/\textbf{29.08} \\
    &  {\footnotesize MZSR~\cite{soh2020meta}} &  {37.25/31.59} & {33.16/27.90} & \hspace{-9pt}{31.64/\hspace{8pt}-} & {30.41/25.52} & \textbf{36.70}/\textbf{29.58} & {-\hspace{8pt}/\hspace{8pt}-} \\
    &  {\footnotesize DASR~\cite{wang2021unsupervised}} &  \textbf{37.87}/\textbf{31.99} & \textbf{33.34}/\textbf{28.50} & {\textbf{32.03/27.52}} & \textbf{31.49}/\textbf{25.82} & {-\hspace{8pt}/\hspace{8pt}-} & {-\hspace{8pt}/\hspace{8pt}-} \\
    \midrule
    \multirow{2}{*}{\footnotesize Self-supervised}
    %&  \textbf{IMF-SRSR~(Test)} & {36.41/29.49} & {32.44/27.19} & {31.34/26.82} & {30.26/24.66} & {36.29/27.82} & {35.02/29.45} \\
    &  {\footnotesize \textbf{ICF-SRSR}~(Ours)} & {37.01/30.81} & {32.86/27.76} & {31.54/26.99} & {30.39/24.72} & {36.45/28.01} & {35.19/29.48} \\
    &  {\footnotesize \textbf{EDSR~(LLR,LR)}~(Ours)} & {\textbf{37.09/31.06}} & {\textbf{32.91/27.97}} & {\textbf{31.63/27.10}} & {\textbf{30.51/24.92}} & {\textbf{36.68/28.29}} & {\textbf{35.26/29.64}} \\ 
    %\multirow{3}{*}{(fully self-supervised)}
    \bottomrule
    \end{tabularx}
    \vspace{-2mm}
    \caption{
        \textbf{Quantitative comparisons on synthetic datasets.} 
        %
        We compare ICF-SRSR with several supervised/unsupervised methods on the benchmarks~\cite{bevilacqua2012low, zeyde2010single, martin2001database, huang2015single, Manga109} and DIV2K~\cite{agustsson2017ntire} validation set for scales $\times 2$ and $\times 4$ with PSNR metric. 
        %
        ICF-SRSR refers to our self-supervised method, while EDSR~(LLR,LR) is the model EDSR trained on our generated pairs (LLR,LR) of the DIV2K.
        %
    }
    \label{tab:benchmark}
    \vspace{-2mm}
\end{table*}


%
We also note that our method is different from CycleGAN~\cite{zhu2017unpaired}, which utilizes unpaired LR-HR images and performs two independent cycles, one on the LR and the other on the HR images.
%
Rather, our model is trained in a self-supervised manner by optimizing the $f_{\theta}$ jointly with two stages on LR images only, without requiring the adversarial loss.
%
In other words, $f_{\theta}$ can perform simultaneous up-sampling and down-sampling without requiring prior information or paired/unpaired data.
%
\input{sections/figures/b100}
%%%%%%%%%%%%%%%%%%%%%%%%%%%%%%%%%%%%%%%%%%%%%%%%%%%%%%%
\subsection{Training loss functions}
%
\label{sec:loss}
%
To train the proposed ICF $f_{\theta}$, we design a set of self-supervised loss functions.
%
First, we formulate the consistency loss $\mathcal{L}^{\text{Cons}}$, which preserves information during the simultaneous up-down and down-up stages.
%
The proposed consistency loss $\mathcal{L}^\text{Cons}$ on the approximated LR images $\hat{X}$ and $\check{X}$, and the original input $X$ is defined as follows:
%
%%%%%%%%%%%%%%%%%%%%%%%%%%%%%%%%%%%%%%%%%%%%%%%%%%%%
\begin{equation}\label{eq:loss_cons}
\begin{split}
\mathcal{L}^{\text{Cons}} &=\lVert \hat{X}-X \rVert + \lVert \check{X}-X \rVert.
\end{split}
\end{equation}
%%%%%%%%%%%%%%%%%%%%%%%%%%%%%%%%%%%%%%%%%%%%%%%%%%%%
%
For simplicity, we use $\lVert\cdot\rVert$ to represent the L1 norm.
%
The proposed consistency term $\mathcal{L}^{\text{Cons}}$ guarantees to generate reliable up-sampled and down-sampled images simultaneously.
%
%%%%%%%%%%%%%%%%%%%%%%%%%%%%%%%%%%%%%%%%%%%%%%%%%%%%
Furthermore, to stabilize the training and preserve colors between the input and intermediate images $X_s$ and $X_{\nicefrac{1}{s}}$, we utilize the low-frequency loss~\cite{son2021toward}.
%
We implement the low-pass filter with a spatial pooling operator $\mathbf{P} \left( \cdot, w, s \right)$, where $w$ and $s$ are window size and stride, respectively.
%
Our color-preserving loss $\mathcal{L}^\text{Color}$ is defined as follows:
%
%%%%%%%%%%%%%%%%%%%%%%%%%%%%%%%%%%%%%%%%%%%%%%%%%%%%
\begin{equation}\label{eq:loss_pool}
%\begin{aligned}
\begin{split}
\mathcal{L}^{\text{Color}} &= 
\lVert \mathbf{P} \left( X_s, 4s, 4s \right) - \mathbf{P} \left( X, 4, 4 \right) \rVert
\\ &+ \lVert \mathbf{P} \left( X_{\nicefrac{1}{s}}, 4, 4 \right) - \mathbf{P} \left( X, 4s, 4s \right) \rVert,
\end{split}
\end{equation}
%%%%%%%%%%%%%%%%%%%%%%%%%%%%%%%%%%%%%%%%%%%%%%%%%%%%
where the window size and stride are adjusted to match dimensions between each of $\left( X_s, X \right)$ and $\left( X_{\nicefrac{1}{s}}, X \right)$.
%
The total training objective $\mathcal{L}^{\text{Total}}$ is the combination of the aforementioned two loss terms, which is defined as follows: 
%%%%%%%%%%%%%%%%%%%%%%%%%%%%%%%%%%%%%%%%%%%%%%%%%%%%
\begin{equation}\label{eq:loss_total}
\mathcal{L}^{\text{Total}} = \mathcal{L}^{\text{Cons}}+\lambda_{\text{Color}}\mathcal{L}^{\text{Color}}.
\end{equation}
%%%%%%%%%%%%%%%%%%%%%%%%%%%%%%%%%%%%%%%%%%%%%%%%%%%%
 
\subsection{Network architecture}\label{sec:net_arch}
%
Our ICF-SRSR architecture leverages a single model to handle different scale conditions.
%
To implement the proposed method, we modify the existing SISR model, \eg, EDSR~\cite{lim2017enhanced} as our baseline backbone architecture.
%
Since the body part is invariant to the scale image (\ie, the input and output have the same resolution), we introduce multiple tail parts for different scale conditions.
%
Employing a single network with the shared body part is more efficient and can improve performance by observing more augmented data, \ie, images with different scales, during the training.
%
In the supplementary material, we provide the details of the network architecture and illustrate that our method is model-agnostic and can leverage different SOTA baseline models. 
%
We will also publish our ICF-SRSR implementation.
