\documentclass[10pt,twocolumn,letterpaper]{article}

%%%%%%%%% PAPER TYPE  - PLEASE UPDATE FOR FINAL VERSION
\usepackage{iccv}  
\usepackage{times}
\usepackage{epsfig}
\usepackage{graphicx}
\usepackage{amsmath}
\usepackage{amssymb}
\usepackage{booktabs}

% Include other packages here, before hyperref.

% If you comment hyperref and then uncomment it, you should delete
% egpaper.aux before re-running latex.  (Or just hit 'q' on the first latex
% run, let it finish, and you should be clear).


\usepackage{xcolor}         % colors
\definecolor{ForestGreen}{rgb}{0.0, 0.27, 0.13}
\newcommand{\paren}[1]{\left( #1 \right)}
\newcommand{\parens}[1]{( #1 )}
\newcommand{\bracket}[1]{\left[ #1 \right]}
%\newcommand{\normtwo}[1]{\left\lVert #1 \right\rVert_2^2}
\newcommand{\normtwo}[1]{\left\lVert #1 \right\rVert}

\newcommand{\Paragraph}[1]{\noindent \textbf{#1}}
% Include other packages here, before hyperref.
\usepackage{appendix}
\usepackage{titling}
    \pretitle{\vspace*{-3pt}\begin{center}\Large \bf}
    \posttitle{\vspace*{10pt}\par\end{center}}
    \preauthor{\large\begin{center}\par}
    \postauthor{\vspace*{6pt}\par\end{center}}
    \predate{}
    \date{}
    \postdate{}
% Support for easy cross-referencing
%\crefname{section}{Sec.}{Secs.}
%\Crefname{section}{Section}{Sections}
%\Crefname{table}{Table}{Tables}
%\crefname{table}{Tab.}{Tabs.}

\usepackage[utf8]{inputenc} % allow utf-8 input
\usepackage[T1]{fontenc}    % use 8-bit T1 fonts
%\usepackage{hyperref}       % hyperlinks
\usepackage{url}            % simple URL typesetting
\usepackage{booktabs}       % professional-quality tables
\usepackage{amsfonts}       % blackboard math symbols
\usepackage{nicefrac}       % compact symbols for 1/2, etc.
\usepackage{microtype}      % microtypography
%\usepackage{xcolor}  % colors
\usepackage{graphicx}
\usepackage{amsmath}
\usepackage{amssymb}
\usepackage{cancel}
\usepackage{enumerate}
\usepackage{comment}
\usepackage{stackengine}
%\usepackage{gensymb}
\usepackage{pifont}
\usepackage[accsupp]{axessibility}
\usepackage{tabularx}
\usepackage{multirow}

\newcommand*\samethanks[1][\value{footnote}]{\footnotemark[#1]}
%\usepackage[pagebackref=true,breaklinks=true,letterpaper=true,colorlinks,citecolor=blue,bookmarks=false]{hyperref}
%\usepackage[pagebackref,breaklinks,colorlinks,]{hyperref}
\usepackage{subcaption}
\captionsetup{compatibility=false}

\usepackage[pagebackref=true,breaklinks=true,letterpaper=true,colorlinks,citecolor=blue,bookmarks=false]{hyperref}
\usepackage[capitalize]{cleveref}
\iccvfinalcopy % *** Uncomment this line for the final submission




%%%%%%%%% PAPER ID  - PLEASE UPDATE
\def\iccvPaperID{5616} % *** Enter the ICCV Paper ID here
\def\httilde{\mbox{\tt\raisebox{-.5ex}{\symbol{126}}}}
% Pages are numbered in submission mode, and unnumbered in camera-ready
\ificcvfinal\fi

\begin{document}

\title{ICF-SRSR: Invertible scale-Conditional Function \textit{for}\\ Self-Supervised Real-world Single Image Super-Resolution}

\author{Reyhaneh Neshatavar$^{1}$\thanks{equal contribution} \qquad Mohsen Yavartanoo$^{1}$\samethanks \qquad Sanghyun Son$^{1}$ \qquad Kyoung Mu Lee$^{1,2}$ \\$^{1}$Dept. of ECE \& ASRI, $^{2}$IPAI, Seoul National University, Seoul, Korea\\
{\tt\small \{reyhanehneshat,myavartanoo,thstkdgus35,kyoungmu\}@snu.ac.kr}}

% The \author macro works with any number of authors. There are two commands
% used to separate the names and addresses of multiple authors: \And and \AND.
%





%\usepackage{hyperref}
\usepackage{url}
\usepackage{xurl}
\usepackage{graphicx}
\usepackage{booktabs}
\usepackage{amssymb}
\usepackage{amsmath}
\usepackage{subcaption}
\usepackage{xcolor}
\usepackage{adjustbox} 
\usepackage{multirow}
\usepackage{wrapfig}
\usepackage{tikz}
\usepackage{xspace}
\usepackage{soul}
\newcommand{\ie}{\textit{i}.\textit{e}., }
\newcommand{\eg}{\textit{e}.\textit{g}., }
\usepackage{titletoc}
% \usepackage{flushend}

\definecolor{bg_blue}{RGB}{213,227,251}
\definecolor{bg_yellow}{RGB}{250,243,187}
\definecolor{bg_purple}{RGB}{177,167,207}
\definecolor{bg_red}{RGB}{200,169,188}
\definecolor{bg_green}{RGB}{192,213,175}
\definecolor{bg_skin}{RGB}{245,232,210}

\definecolor{red_color}{RGB}{255,0,0}
\newcommand{\redtext}[1]{\textcolor{red_color}{#1}}
\definecolor{yellow_color}{RGB}{255,202,47}
\newcommand{\yellowtext}[1]{\textcolor{yellow_color}{#1}}

\definecolor{purple_color}{RGB}{64,103,139}
\newcommand{\purpletext}[1]{\textcolor{purple_color}{#1}}

\definecolor{dark_red}{RGB}{153, 31, 41}
\newcommand{\drtext}[1]{\textcolor{dark_red}{#1}}
\definecolor{green_color}{RGB}{130,139,78}
\newcommand{\greentext}[1]{\textcolor{green_color}{#1}}
\definecolor{brown_color}{RGB}{205,90,161}
\newcommand{\browntext}[1]{\textcolor{brown_color}{#1}}
\definecolor{lg_color}{RGB}{63,147,139}
\newcommand{\lgtext}[1]{\textcolor{lg_color}{#1}}

\definecolor{com_color}{RGB}{0,0,139}
\newcommand{\com}[1]{\textcolor{com_color}{#1}}


\definecolor{orange_color}{RGB}{255,148,63}
\newcommand{\orangetext}[1]{\textcolor{orange_color}{#1}}

\definecolor{gray_color}{RGB}{169,169,169}
\newcommand{\graytext}[1]{\textcolor{gray_color}{#1}}


\definecolor{lightgray}{RGB}{220,220,220}
\newcommand{\codehighlight}[1]{\colorbox{lightgray}{{#1}}}

\definecolor{lightgreen}{RGB}{179,207,176}
\newcommand{\codehighlightgreen}[1]{\colorbox{lightgreen}{{#1}}}

\definecolor{lightblue}{RGB}{181,209,230}
\newcommand{\codehighlightblue}[1]{\colorbox{lightblue}{{#1}}}


\newcommand{\model}{\mbox{\sc LLM-Rec}\xspace}


\newcommand{\ctext}[3][RGB]{%
  \begingroup
  \definecolor{hlcolor}{#1}{#2}\sethlcolor{hlcolor}%
  \hl{#3}%
  \endgroup
}


\newcommand\DoToC{%
  \startcontents
  \printcontents{}{1}{\textbf{Table of Contents (Appendix)}\vskip9pt\hrule\vskip5pt}
  \vskip3pt\hrule\vskip5pt
}



\maketitle
% Remove page # from the first page of camera-ready.
%\ificcvfinal\thispagestyle{empty}\fi


%%%%%%%%% ABSTRACT
\begin{abstract}

The Fast Reciprocal Square Root Algorithm is a well-established approximation technique consisting of two stages: first, a coarse approximation is obtained by manipulating the bit pattern of the floating point argument using integer instructions, and second, the coarse result is refined through one or more steps, traditionally using Newtonian iteration but alternatively using improved expressions with carefully chosen numerical constants found by other authors. The algorithm was widely used before microprocessors carried built-in hardware support for computing reciprocal square roots. At the time of writing, however, there is in general no hardware acceleration for computing other fixed fractional powers. This paper generalises the algorithm to cater to all rational powers, and to support any polynomial degree(s) in the refinement step(s), and under the assumption of unlimited floating point precision provides a procedure which automatically constructs provably optimal constants in all of these cases. It is also shown that, under certain assumptions, the use of monic refinement polynomials yields results which are much better placed with respect to the cost/accuracy tradeoff than those obtained using general polynomials. Further extensions are also analysed, and several new best approximations are given.

\end{abstract}

%%%%%%%%% BODY TEXT
\section{Introduction}
Current quantum hardware is unable to carry out universal quantum computations due to the buildup of errors that occur during the computation. 
The magnitude of the individual error is currently above the value that the Threshold Theorem requires in order to kick-start quantum error correction and fault-tolerant quantum computation~\cite[Section 10.6]{nielsen_chuang_2010}. 
Although the experimentally achieved fidelity rates are promising and the error bounds are inching closer to the required threshold, we will have to work for the foreseeable future with quantum hardware with errors that build-up during the computation.  This implies that we can only do a limited number of steps before the output of the computation has become completely uncorrelated with the intended one.

For fault-tolerant quantum computing, we repeat four steps: 
1) We apply a number of single and two-qubit quantum gates, in parallel whenever possible; 
2) We perform a syndrome measurement on a subset of the qubits; 
3) We perform fast classical computations to determine which errors have occurred and how to correct them; 
and, 4) We apply correction terms based on the classical computations.
We then repeat these four steps with a next sequence of gates. 
These four steps are essential to fault-tolerant quantum computing. 


The starting point of this work is to use the four steps outlined above, not to carry out error correction and fault-tolerant computation, but to enhance short, constant-depth, {\em uncorrected} quantum circuits that perform single qubit gates and {\em nearest-neighbor} two qubit gates. 
Since in the long run we will have to implement error-correction and fault-tolerant computation anyhow, and this is done by such a four-step process, why not make other use of this architecture? Moreover, on some of the quantum hardware platforms, these operations are already in place.
Embracing this idea we naturally arrive at the question: what is the computational power of \textit{low-depth} quantum-classical circuits organized as in the four steps outlined above? 
We thus investigate circuits that execute a small, ideally constant, number of stages, where at each stage we may apply, in parallel, single qubit gates and {\em nearest-neighbor} two qubit gates, followed by measurements, followed by low-depth classical computations of which the outcome can control quantum gates in later stages. 
It is not clear, at first, whether such circuits, especially with constant depth, can do anything remotely useful. 
But we will see that this is indeed the case: many quantum computations can be done by such circuits in constant depth. 
By parallelizing quantum computations in this way, we improve the overall computational capabilities of these circuits, as we do not incur errors on qubits that are idle, simply because qubits are not idle for a very long time. 
Furthermore, reducing the depth of quantum circuits, at the cost of increasing width, allows the circuit to be run faster even if errors occur.

The first usage of such a four-step layout, not to do error correction, but to perform computations, can be found in the paradigm of measurement-based quantum computing~\cite{gottesman1999demonstrating,raussendorf2001one,jozsa2006introduction,clark2007generalised}: 
A universal form of quantum computing where a quantum state is prepared and operations are performed by measuring qubits in different bases, depending on previous measurements and intermediate measurements.

\citeauthor{PhamSvore2013} were the first to formalize the four-step protocol for performing computations~\cite{PhamSvore2013}. They included specific hardware topologies by considering two-dimensional graphs for imposing constraints on qubit interactions. In their model, they develop circuits for particularly useful multi-qubit gates, including specifying costs in the width, number of qubits, depth, number of concurrent time steps, size, and total number of non-Identity operations.
As a result, they find an algorithm that factors integers in polylogarithmic depth.
\citeauthor{Browne:2011} showed that the main tool in the work by \citeauthor{PhamSvore2013}, the fan-out gate, can also be replaced by additional log-depth classical computations in the measurement-based quantum computing setting~\cite{Browne:2011}.

More recently, \citeauthor{Cirac:2021} introduced a scheme to implement unitary operations involving quantum circuits combined with Local Operations and Classical Communication ($\mathsf{LOCC}$) channels: $\mathsf{LOCC}$-assisted quantum circuits~\cite{Cirac:2021}. Similarly to the four-step scheme we just described, they allow for a short depth circuit to be run on the qubits, followed by one round of $\mathsf{LOCC}$, in which ancilla qubits are measured and local unitaries are applied based on the measurement outcomes. They show that in this model any 1D transitionally invariant matrix-product state (MPS) with fixed bond dimension is in the same phase of matter as the trivial state. Similar ideas can be found in~\cite{TVV_NonAbelianTopologicalOrder_2022, tantivasadakarn2021long}.

In this work, we introduce a new model, called \textit{Local Alternating Quantum-Classical Computations} ($\LAQCC$). In this model we alternate between running quantum circuits (constrained by locality), ending in the measurement of a subset of qubits, and fast classical computations based on the measurement results. The outcome of the classical computations are then used to control future quantum circuits. We allow for flexibility in this model, by giving different constraints to the power of both the quantum circuits and the classical circuits as well as the number of alternations between them. 
Most attention will be given to $\LAQCC$ containing quantum circuits of constant depth, classical circuits of logarithmic depth and at most a constant number of alternations between them. 
Any circuit constructed in this model is considered to be of constant depth. 
We restrict ourselves to logarithmic depth classical computations, as this is the first natural and non-trivial extension beyond constant-depth classical computations. 
Constant-depth classical computations do however also have an equivalent constant-depth quantum implementation.

The definition of $\LAQCC$ sharpens the original definition of \citeauthor{PhamSvore2013} by adding constraints to the intermediate classical computations. This allows us to bound the power of $\LAQCC$ from above. 

The main result of \citeauthor{Cirac:2021}, that 1D translational invariant MPS with fixed bond dimension can be prepared by $\mathsf{LOCC}$-assisted circuits, relies on local symmetries of the MPS. These symmetries allow them to prepare local states (on a constant number of qubits) and glue them together by doing one round of the appropriate entangling measurement and corrections, after which they run a round of local unitaries to get the desired result. This general scheme for preparing states that exhibit an MPS description with the appropriate local symmetries requires only geometrically local unitaries and one round of measurement and corrections an therefore is accessible in $\LAQCC$. Studying different local symmetries, known as Symmetry Protected Topological (SPT) phases of matter, to find measurement-based constant depth circuits for states is a broad ongoing field of research~\cite{TVV_NonAbelianTopologicalOrder_2022, tantivasadakarn2021long, smith2023deterministic}. 
All these schemes have a $\LAQCC$ implementation.

%$\LAQCC$-circuits also exist for general schemes of preparing local states, based on the local tensors, and gluing them together using one round of entangled measurement and corrections, based on the local symmetry. 
%The main result of \citeauthor{Cirac:2021}, that 1D translational invariant MPS with fixed bond dimension can be prepared by $\mathsf{LOCC}$-assisted circuits, relies heavily on local symmetries of the MPS and as a result also has an equivalent $\LAQCC$ implementation. 
%The corrections applied after the measurement round are local unitaries depending on the local symmetries of the MPS. 

 

%This general scheme of preparing local states, based on the local tensors, and gluing it together by doing one round of entangled measurement and corrections, based on the local symmetry, is accessible in $\LAQCC$.
Note however that \citeauthor{Cirac:2021} also suggest a circuit for the $W$-state.
This circuit uses sequentially and dependent measurement-based corrections of the ancilla qubits. 
These dependent measurements translate to sequential alternations between the quantum and classical circuits and therefore increase the total depth to linear depth, exceeding the constant-depth constraints imposed by $\LAQCC$-circuits. 

We study the power of the $\LAQCC$ model with respect to state preparation, showing that even with only constant quantum-depth and logarithmic classical depth it remains possible to prepare states with long-range entanglement.
Another surprising result is that it is unlikely that $\LAQCC$ circuits are classically simulatable. We show that any instantaneous quantum polynomial-time (IQP) circuit~\cite{Bremner2010,Shepherd2009} has an $\LAQCC$ implementation.
Classical simulation of IQP circuits implies the collapse of the polynomial hierarchy to the third level, which is not believed to be true~\cite{Bremner2017}. Therefore, we expect that $\LAQCC$ circuits are unlikely to be classically simulatable. We bound the power of $\LAQCC$ by showing that it is contained in $\QNC^1$, the class of polynomial-size, log-depth circuits.

Next, we also study the power that intermediate classical calculations can add to quantum computations, by considering a new model that alternates between polynomially many polynomial-depth quantum circuits and unbounded classical computations
We study this model by doing a complexity theoretical analysis, where we draw inspiration from the notions of complexity given by \citeauthor{RosenthalYuen:2022}, \citeauthor{MetgerYuen:2023}, and \citeauthor{Aaronson:2004}.
All three complexity notions are based on the notion of state preparation, instead of more traditional definition of complexity such as the decidability of a computational problem. 
The first two consider classes based on sequences of quantum states preparable by a polynomial-sized quantum circuit, where the circuits are uniformly generated by a computational class, for instance, the class $\mathsf{PSPACE}$, which results in the complexity class $\mathsf{StatePSPACE}$~\cite{RosenthalYuen:2022,MetgerYuen:2023}.
The third notion considers a relative complexity, where the complexity is measured between two given states, and is measured by the number of gates, from a given gate-set, required to transform one state in another state~\cite{Aaronson:2004}. 
For our definition of state preparation complexity, we drop the uniformity constraint from~\cite{RosenthalYuen:2022,MetgerYuen:2023} and define a class as $\mathsf{StateX}$, which refers to states preparable by circuits of type $\mathsf{X}$. 
As an example, if $\mathsf{X} = \QNC^0$, this results in the class $\mathsf{StateQNC^0}$, which is the set of states preparable from the $\ket{0}^n$ state by poly-size constant-depth circuits. 
This notion is similar to the relative complexity from~\cite{Aaronson:2004}, where one state is the  $\ket{0}^n$ state and instead of counting the number of gates we consider the set of states preparable by a fixed number of gates. Using this notion of complexity we show that any state preparable by an $\LAQCC^*$ circuit is also preparable by a $\mathsf{PostQPoly}$ circuit, the class of circuits of polynomial depth with an additional post-selection gate. 

All Clifford circuits have a constant-depth $\LAQCC$ implementation, implying that any stabilizer state can be implemented by a constant-depth $\LAQCC$ circuit, see Section~\ref{sec:clifford_circuits} for a proof of this statement. 
Efficient circuits for stabilizer states have been known already through measurement-based quantum computing. Therefore this paper focuses on the preparation of non-stabilizer states, and as a surprising result we find novel constant-depth protocols for four very natural classes of non-stabilizer states.
Despite the extensive research into these four classes of non-stabilizer states and the many applications of them, no efficient constant- or low-depth state preparation protocols are known yet. We specifically consider these four classes as they are all often used as initial states in other algorithms.

The first state is a uniform superposition over an arbitrary number of states. 
This state finds applications in many quantum algorithms, as they often start with a uniform superposition over multiple states. 
This superposition is often achieved by applying Hadamard gates to every qubit due to its simplicity to prepare. 
Yet, the analysis of many algorithms, such as Shor's algorithm~\cite{Shor:1997}, would benefit from a different initial superposition. 
The circuit to prepare the uniform superposition over an arbitrary number of states uses an exact version of Grover search as a subroutine, that turns a probabilistic circuit, with a known constant probability of success, into a deterministic circuit. 
We use the circuit for preparing a uniform superposition over an arbitrary number of states as a subroutine in the next two quantum state preparation protocols. 

The second state is the $W$-state, the uniform superposition over all computational basis states of Hamming-weight~$1$, a natural long-ranged entangled state that displays a fundamentally nonequivalent type of entanglement from the Greenberger–Horne–Zeilinger state~\cite{WState:2000}, for which $\LAQCC$-type constant-depth circuits were previously known~\cite{PhamSvore2013, Cirac:2021}. 
The $W$-state is often used as benchmark for new quantum hardware~\cite{Haffner2005,Neeley2010,GarciaPerez:2021}. 
A novel way to prepare the $W$-state therefore gives a new way to benchmark different quantum devices with each other. 
A circuit for preparing the $W$-state was given in~\cite{Cirac:2021}, but this implementation requires sequentially alternating measurements followed by local unitaries, which in the $\LAQCC$ model is not considered to be of constant depth. 
We improve this protocol by giving an $\LAQCC$ implementation of the $W$-state, based on a compress-uncompress method that links the one-hot and binary encoding of integers.

The third state considered is the Dicke state, a generalization of the $W$-state, a superposition over all computational basis states with Hamming-weight $k$~\cite{Dicke:1954}. 
Dicke states have relevance in various practical settings.
For instance, for quantum game theory~\cite{zdemir2007}, quantum storage~\cite{Bacon_Compress:2006,Plesch:2010}, quantum error correction~\cite{ouyang2014permutation}, quantum metrology~\cite{toth2012multipartite}, and quantum networking~\cite{prevedel2009experimental}. 
Dicke states have been used as a starting state for variational optimization algorithms, most notably Quantum Alternating Operator Ansatz (QAOA)~\cite{Hadfield2019}, to find solutions to problems such as Maximum k-vertex Cover~\cite{Brandhofer2022,cook2020quantum}.
The ground states of physical Hamiltonians describing one-dimensional chains tend to show a resemblance to Dicke states such as states resulting from the Bethe ansatz, making them an ideal starting state when investigating the ground state behavior of these Hamiltonians~\cite{TDL_BetheAnsatzDerivation:2010,B_ExcitedStateQuantumPhaseTransitions:2013,DickeTransitions:2021}. 
For instance, the algorithm by \citeauthor{van2021preparing}, who give an algorithm to prepare the Bethe ansatz eigenstates of the spin-1/2 XXZ spin chain, starts by first preparing a Dicke state~\cite{van2021preparing}. 
A Dicke-state preparation protocol based on the compress-uncompress methodology used in the $W$-state furthermore finds applications in entanglement distillation, where the entanglement of a large state is concentrated on only a few qubits. 
Efficient deterministic circuits for preparing Dicke states have been proposed by \citeauthor{bartschi2019deterministic}~\cite{bartschi2019deterministic, bartschi2022deterministic_short_depth}. 
They provide a quantum circuit of depth $\mathO(k \log(\frac{n}{k}))$, allowing arbitrary connectivity, to prepare a Dicke state, which they conjecture to be optimal when $k$ is constant. 
In this work, we provide a constant-depth $\LAQCC$ circuit below their conjectured bound already for constant $k$. 
However, this does not directly disprove their conjecture, as we allow for intermediate measurements and classical computations. 
More significantly, we even construct constant-depth $\LAQCC$ circuits for $k = \mathO(\sqrt{n})$ greatly improving their bound.
This construction extends the compress-uncompress method for the $W$-state combined with additional subroutines. 

We continue with a log-depth state preparation protocol for the Dicke-state for arbitrary $k$. 
This protocol implements an efficient transformation between the factoradic number representation and the combinatorial number representation of a positive integer. 
The combinatorial number representation relates directly to the Dicke state. 
The provided efficient transformation between number representation systems might be of independent interest. 

We conclude by modifying our protocol for preparing a Dicke-state to a protocol that prepares quantum many-body scar states in constant-depth. 
These states have low entanglement and longer coherence times than states with similar energy density.
These characteristics make many-body scar states interesting to analyze and relevant within physics.
Many-body scar states appear for instance in the AKLT model~\cite{AKLT:1987,MRBAR:2018,MRB:2018} and different spin models~\cite{SI:2019,MOBFR:2020}.
Known methods for preparing these states have polynomial-depth~\cite{Gustafson:2023}, whereas our circuit has constant depth. 

% We conclude by studying the power that intermediate classical calculations can add to quantum computations. 
% In this study, we define a new model that relaxes constant-depth quantum circuits to polynomial depth quantum circuits, log-depth classical calculations to unbounded classical computations and a constant number of alternations to a polynomial number of alternations. 
% We call this model $\LAQCC^*$. 
% We study this model by doing a complexity theoretical analysis, where we draw inspiration from the notions of complexity given by \citeauthor{RosenthalYuen:2022}, \citeauthor{MetgerYuen:2023}, and \citeauthor{Aaronson:2004}.
% All three complexity notions are based on the notion of state preparation, instead of more traditional definition of complexity such as the decidability of a computational problem. 
% The first two consider classes based on sequences of quantum states preparable by a polynomial-sized quantum circuit, where the circuits are uniformly generated by a computational class, for instance, the class $\mathsf{PSPACE}$, which results in the complexity class $\mathsf{StatePSPACE}$~\cite{RosenthalYuen:2022,MetgerYuen:2023}.
% The third notion considers a relative complexity, where the complexity is measured between two given states, and is measured by the number of gates, from a given gate-set, required to transform one state in another state~\cite{Aaronson:2004}. 
% For our definition of state preparation complexity, we drop the uniformity constraint from~\cite{RosenthalYuen:2022,MetgerYuen:2023} and define a class as $\mathsf{StateX}$, which refers to states preparable by circuits of type $\mathsf{X}$. 
% As an example, if $\mathsf{X} = \QNC^0$, this results in the class $\mathsf{StateQNC^0}$, which is the set of states preparable from the $\ket{0}^n$ state by poly-size constant-depth circuits. 
% This notion is similar to the relative complexity from~\cite{Aaronson:2004}, where one state is the  $\ket{0}^n$ state and instead of counting the number of gates we consider the set of states preparable by a fixed number of gates. Using this notion of complexity we show that any state preparable by an $\LAQCC^*$ circuit is also preparable by a $\mathsf{PostQPoly}$ circuit, the class of circuits of polynomial depth with an additional post-selection gate. 

\paragraph{Summary of results}
\begin{itemize}
    \item We give a new definition of a computational model that captures the power of the four step process: applying a constant number of layers of one- and two-qubit gates; performing a syndrome measurement; perform a fast classical computation determining corrections; apply corrections. We call this model \emph{Local Alternating Quantum Classical Computations}, or $\LAQCC$ for short. In this model we bound the allowed quantum operations, intermediate classical calculations, and number of rounds separately. In Section~\ref{sec:LAQCC_model} we define this model and give a list of operations based on results from literature contained in this computational model. In some of these operations we explicitly use that we allow for multiple, but at most constant, rounds  of corrections.
    \item  We show show that there exist $\LAQCC$ circuits that can not be weakly simulated in Section~\ref{sec:IQP_in_LAQCC}. We further show that for every $\LAQCC$ circuit there exists a $\QNC^1$ circuit simulating it perfectly, in Section~\ref{sec:LAQCC_in_QNC1}.
    \item We introduce a new type computational complexity for preparing states and show that the extension of $\LAQCC$ where we allow a polynomial number of rounds and unbounded classical computation, is contained in $\mathsf{PostQPoly}$, the class of polynomial circuits with post-selection, in Section~\ref{sec:Complexity results}.
    \item We show a protocol to prepare the uniform superposition state of size $q$ in $\LAQCC$ using $\mathO(\ceil{\log_2(q)}^2)$ qubits in Section~\ref{sec:superposition_modulo_q}. 
    \item We show a protocol to prepare the $W_n$ state in $\LAQCC$ using $\mathO(n\log(n))$ qubits in Section~\ref{sec:W_state_in_LAQCC}.
    \item We show two ways of preparing the Dicke-$(n,k)$ state. The first method is in $\LAQCC$, works up to $k = \mathO(\sqrt{n})$, uses $\mathO(n^2\log(n))$ qubits, and is found in Section~\ref{sec:dicke:small_k}. The second method is in $\LAQCC\text{-}\mathsf{LOG}$ (an extension of $\LAQCC$ allowing for logarithmic number of alterations instead of constant), works for any $k$, uses $\mathO(\text{poly}(n))$ qubits, and is found in Section~\ref{sec:Dicke_in_LAQCC_LOG}. 
    \item We extend on our $\LAQCC$ method of generating Dicke-$(n,k)$ states for $k = \mathO(\sqrt{n})$ and show a protocol to generate many-body scar states for a particular Hamiltonian in $\LAQCC$ (Section~\ref{sec:many_body_scar}). 
\end{itemize}
Summarized in a table, we provide the following state generation protocols:
\begin{table}[htb]
\centering
\begin{tabular}{l|l|l|l}
\textbf{State description} & \textbf{Width} & \textbf{Depth} & \textbf{Implementation}\\
\hline 
Uniform superposition mod $q$: $\frac{1}{\sqrt{q}} \sum_{i = 0}^{q-1}\ket{i}$ & $\mathO(\ceil{\log^2 q})$ & $\mathO(1)$ & Section~\ref{sec:superposition_modulo_q}\\

$W$-state: $\frac{1}{\sqrt{n}}\sum_{i = 0}^{n-1}\ket{e_i}$ & $\mathO(n \log n)$ & $\mathO(1)$ & Section~\ref{sec:W_state_in_LAQCC}\\

Dicke-$(n,k)$, $k = \mathO(\sqrt{n})$: $\binom{n}{k}^{-1/2}\sum_{x \in \{0,1\}^n: |x| = k} \ket{x}$ &  $\mathO(n^2\log n)$ & $\mathO(1)$ 
&Section~\ref{sec:dicke:small_k}\\

Dicke-$(n,k)$: $\binom{n}{k}^{-1/2}\sum_{x \in \{0,1\}^n: |x| = k} \ket{x}$ & $\mathO(\text{poly}(n))$ & $\mathO(\log n)$ &Section~\ref{sec:Dicke_in_LAQCC_LOG}\\

QMBS: $\ket{S_k} = \frac{1}{k! \sqrt{\mathcal N(n,k)}}(Q^\dagger)^k \ket{\Omega}$ &  $\mathO(n^2\log n)$ & $\mathO(1)$  &  Section~\ref{sec:many_body_scar}
\end{tabular}
\caption{Summary of state preparation protocols given in this paper.}
\label{tab:sate_prep}
\end{table}
In the entry for the quantum many-body scar state $Q$ denotes the raising operator and $\mathcal N(n,k)=\binom{n-k-1}{k}$. 
Section~\ref{sec:many_body_scar} will provide more details on the variables and the implementation. 

\paragraph{Organization of the paper}
\noindent We first introduce relevant preliminaries in Section~\ref{sec:preliminaries}. 
In Section~\ref{sec:LAQCC_model} we formally define the class of Local Alternating Quantum-Classical Computations ($\LAQCC$). We also show that any Clifford circuit can be implemented in constant depth $\LAQCC$ (a result based on a result from measurement-based quantum computing~\cite{jozsa2006introduction}). 
This result allows us to give many useful multi-qubit gates and routines in Section~\ref{sec:gates_created_in_LAQCC}. 
Beyond that we show that constant depth $\LAQCC$ circuits are contained in $\QNC^1$ and that any $\mathsf{IQP}$ circuit has an $\LAQCC$ implementation.
We conclude this section with an analysis of a more powerful instantiation of $\LAQCC$ and show an inclusion with respect to the class $\mathsf{PostQPoly}$, which is the class of circuits of polynomial depth with one additional post-selection gate. 
In Section~\ref{sec:state_prep_in_LAQCC} we give $\LAQCC$ circuit implementations for preparing the uniform superposition over an arbitrary number of states, the $W$-state and the Dicke state up to $k = \mathO(\sqrt{n})$. We furthermore give a log-depth circuit implementation for preparing the Dicke state for any $k$. We conclude by showing a $\LAQCC$ circuit for generating many body scar states of a particular type of Hamiltonian.


\section{Related Work}
%\subsection{Cost Volume based Deep Stereo Matching}
%Stereo matching is a typical problem that has been studied for decades and a well-known four-step pipeline \cite{scharstein2002taxonomy} has been established, where cost volume construction is an indispensable step. Current state-of-the-art stereo matching methods are all cost volume based methods and they can be categorized into two types. Typically, a cost volume is a 4D tensor of height, width, disparity, and features. The first category just uses a full correlation to generate a single-feature cost volume. Such methods are usually efficient but lose much information because of the decimation of feature channels. Many previous work, including Dispnet \cite{dispnet}, MADNet \cite{madnet}, IResNet \cite{iresnet} and AANet \cite{aanet}, belong to this category. The second category usually uses concatenation \cite{gcnet} or group-wise correlation \cite{gwcnet} to generate a multi-feature 4D cost volume. Such a method can achieve better performance while requiring higher computational complexity and memory consumption. Actually, a majority of the top-performing networks in public leaderboards belong to this category, such as GANet \cite{ganet}, CSPN \cite{cspn} and ACFNet \cite{acfnet}. These methods generally employ multiple 3D convolution layers to constantly regularize the 4D cost volume and then apply softmax over the disparity dimension to produce a discrete disparity probability distribution. The final predicted disparity is obtained by softly weighting indices according to their probability, which is also called soft argmin in GCNet \cite{gcnet}. However, soft argmin leaves the output susceptible to multi-modal disparity probability distributions. ACFNet \cite{acfnet} observes this problem and proposes to directly supervise the cost volume with unimodal ground truth distributions. In contrast, we define an uncertainty estimation to quantify the degree to which the cost volume tends to be multi-modal distribution, higher implies the higher possibility of estimation error.

\subsection{Multi-scale Cost Volume based Stereo Matching}
Cost volume construction is an indispensable step in the well-known four-step pipeline for stereo matching \cite{scharstein2002taxonomy, pamisurvey1, pamisurvey2}. Typically, current state-of-the-art stereo matching methods can be categorized into two types of cost volume-based methods, where the cost volume is a 4D tensor of height, width, disparity, and features. The first category usually uses the single-feature 3D cost volume generated by full correlation, which is efficient while losing much information due to the decimation of feature channels. Many real-time methods, such as Dispnet \cite{dispnet}, MADNet \cite{madnet, madnet_pami} and AANet \cite{aanet}, belongs to the category. Moreover, two-stage refinement \cite{mcvmfc} and pyramidal towers \cite{madnet} are commonly applied in the single-feature cost volume based network to construct multi-scale cost volume. The second category usually uses the multi-feature 4D cost volume generated by concatenation \cite{gcnet} or group-wise correlation \cite{gwcnet}, which can achieve better performance with higher computational complexity and memory consumption. Most top-performing networks, including GANet \cite{ganet}, CSPN \cite{cspn} and ACFNet \cite{acfnet} belong to this category. 
% In these methods, the 4D cost volume is constantly regularized by multiple 3D convolution layers and then a discrete disparity probability distribution can be produced by softmax. Next, the final predicted disparity can be obtained by softly weighting indices according to their probability \cite{gcnet}. However, such output is susceptible to multimodal disparity probability distributions and ACFNet \cite{acfnet} gives a solution by directly supervising the cost volume with unimodal ground truth distributions to alleviate this problem. 
Recently, to alleviate the high computational complexity and memory consumption when employing multi-feature 4D cost volumes, \cite{cvpmvsnet, cascade, uscnet} propose to use cascade cost volume representation in multi-view stereo. These methods usually first predict an initial disparity at the coarsest resolution of the image and then gradually refine the disparity by narrowing down the disparity search space. More closely related to our approach is Casstereo \cite{cascade}, which first extended such representation to stereo matching. It selected to uniform sample a pre-defined range to generate the next stage’s disparity search range. Instead, we employ pixel-level uncertainty estimation to adaptively adjust the next stage disparity searching range and generate pseudo-labels for subsequent domain adaptation. Our method also shares similarities with UCSNet \cite{uscnet}, which constructs uncertainty-aware cost volume in multi-view stereo while it doesn’t employ uncertainty estimation to generate pseudo-labels.

%\subsection{Multi-scale Cost Volume based Deep Stereo Matching} 
% \subsection{Multi-scale Cost Volume based Stereo Matching} 
%Multi-scale cost volume firstly was applied in the single-feature cost volume based network with the form of two-stage refinement \cite{mcvmfc} and pyramidal towers \cite{madnet}. Recently, cascade cost volume representation \cite{cvpmvsnet, cascade, uscnet} was proposed in multi-view stereo to alleviate the high computational complexity and memory consumption when employing multi-feature 4D cost volumes. These methods generally predict an initial disparity at the coarsest resolution of the image. Then, they will narrow down the disparity search space and gradually refine the disparity. More closely related to our approach is Casstereo \cite{cascade}, which first extended such representation to stereo matching. It selected to uniform sample a pre-defined range to generate the next stage’s disparity search range. Instead, we employ uncertainty estimation to adaptively adjust the next stage pixel-level disparity searching range and push the next stage's cost volume to be predominantly unimodal.

% The single-feature cost volume based network with the form of two-stage refinement \cite{mcvmfc} and pyramidal towers \cite{madnet} first employ multi-scale cost volume for stereo matching. Recently, to alleviate the high computational complexity and memory consumption when employing multi-feature 4D cost volumes, \cite{cvpmvsnet, cascade, uscnet} propose to use cascade cost volume representation in multi-view stereo, which generally predict an initial disparity at the coarsest resolution of the image. Then, the disparity search space is narrowed down and the disparity is gradually refined. More closely related to our approach is Casstereo \cite{cascade}, which first extended such representation to stereo matching. It selected to uniform sample a pre-defined range to generate the next stage’s disparity search range. Instead, we employ uncertainty estimation to adaptively adjust the next stage pixel-level disparity searching range and push the next stage's cost volume to be predominantly unimodal.

% Figure environment removed

\subsection{Robust Stereo Matching} 
There exist three categories of generalization definitions for robust stereo matching. 1) Cross-domain Generalization: the network’s ability to perform well on unseen scenes (cannot see the image pairs of the target domain in advance). Towards this end, Jia et al \cite{sungeneralizaiton} propose to incorporate scene geometry priors into an end-to-end network. Zhang et al \cite{dsmnet} introduce a domain normalization and a trainable non-local graph-based filter to construct a domain-invariant stereo matching network. 2) Adapt Generalization: the network’s ability to adapt pre-trained models to the new domain with unlabeled target data. Previous work usually pre-trains the models on synthetic data and then adapts it to new target domains with Graph Laplacian regularization \cite{zoom}, non-adversarial progressive color transfer \cite{adastereo}, and Knowledge Reverse Distillation \cite{aohnet}. More closely related to our approach are \cite{aohnet, unsuperviseddomainadaptation} in stereo matching and Monoresmatch \cite{monoresmatch} in monocular depth estimation, which also proposes to generate a pseudo-label for domain adaptation. However, these methods all select to employ classical stereo matching methods \cite{sgm} alongside with confidence estimators, e.g., left-right consistency check to generate pseudo-labels. That is all these methods need an independent method to generate corresponding pseudo-labels. Instead, the proposed method is an end-to-end network that can generate the predicted disparity map, corresponding uncertainty map and pseudo-labels jointly, which is a more simple, yet efficient way. 
% Instead, our proposed method can employ pixel-level and area-level uncertainty estimation to self-distill the predicted disparity maps of our pre-training model and generate sparse while reliable pseudo-labels to align the domain gap, which is a more simple, yet efficient way. 
3) Joint Generalization: the network’s ability to perform well on a variety of datasets with the same model parameters. MCV-MFC \cite{mcvmfc} introduces a two-stage finetuning scheme to achieve a good trade-off between generalization and fitting capability on multiple datasets. However, it doesn’t touch the inner difference between diverse datasets, e.g, the unbalanced disparity distribution. To further address this problem, we propose a cascade cost volume to adaptively the next stage disparity searching space, where the pixel-level uncertainty estimation is at the core.

% \subsection{Monocular Depth Estimation}
% Monocular depth estimation aims to estimate depth values from a single image, instead of stereo images or multiple frames in a video. This problem is ill-posed because of the ambiguity of object sizes. However, humans could estimate the depth from a single image with prior knowledge of the scenes. Recently, learning based methods were explored to learn depth values by supervised or unsupervised learning. Eigen et al. first employed Convolutional Neural Networks (CNN) to predict depth in a coarse-to-fine manner and further improved its performance by multi-task learning. Liu et al. presented deep convolutional neural fields model by combining deep model with continuous CRF. Li et al. [22] refined deep CNN outputs with a hierarchical CRF. Multi-scale continuous CRF was formulated into a deep sequential network by Xu et al. [45] to refine depth estimation. Unsupervised methods tried to train monocular depth estimation with stereo
% image pairs or image sequences and test on single images. Garg et al. [9] used novel image view synthesis loss to train a depth estimation network in an unsupervised way. Godard et al. [11] introduced left-right consistency regularization to improve the performance of view synthesis loss. Recently, some work also propose to use the stereo matching network as a proxy to learn depth from synthetic data or directly employ traditional stereo matching methods to distill proxies labels from the target domain, which proves the feasibility of distilling stereo matching networks to learn monocular depth estimation.



\section{Method} \label{method_hybridaugment}
In this section, we formally define the problem, motivate our work and then present our proposed techniques.


\subsection{Preliminaries}
Let $\mathcal{F}(x;W)$ be an image classification CNN trained on the training set $\mathcal{T}_\text{train} = (x_{i}, y_{i})^{N}_{i=1}$  with $N$ samples, where $x$ and $y$ correspond to images and labels. The clean accuracy (CA) of $\mathcal{F}(x;W)$ is formally defined as its accuracy over a clean test set $\mathcal{T}_\text{test} = (x_{j}, y_{j})^{M}_{j=1}$. Assume two operators ${A}(\cdot)$ and ${C}(c, s)$ that adversarially attacks or corrupts a given set of images with the corruption category $c$ and severity $s$, respectively.  Let $A\mathcal{T}_\text{test}$ and $C\mathcal{T}_\text{test}$ be the adversarially attacked and corrupted versions of $\mathcal{T}_\text{test}$, and let $\mathcal{F}(x;W)$ have a robust accuracy (RA) on $A\mathcal{T}_\text{test}$ and a corruption accuracy (CRA) on $C\mathcal{T}_\text{test}$. 
The aim is to fit $\mathcal{F}(x;W)$ such that the model gains robustness (\ie. increased RA and CRA compared its the baseline version), while retaining (or improving) the clean accuracy of its baseline version trained without robustness concerns.


\noindent \textbf{What we know.} Our work builds on the following crucial observations: i) CNNs favour high-frequency content \cite{wang2020high}, ii) adversaries and corruptions often reside in high-frequency \cite{wang2020towards}, iii) images are dominated by low-frequency \cite{Saikia_2021_ICCV} and iv) models relying on low-frequency components are more robust \cite{li2022robust,wang2020towards}. The robustness-accuracy trade-off is visible; low-frequency reliant models are more robust, but tend to miss out on clean accuracy brought by the high-frequency components. 

\subsection{HybridAugment}
We hypothesize that a \textit{sweet spot} in the robustness-accuracy trade-off can be found. Unlike the \textit{hard} approaches that completely rule out the reliance on high-frequency components (i.e. low-pass filters), we propose to \textit{reduce} the reliance on them. To this end, we adopt a data augmentation approach that aims to diversify $\mathcal{T}_\text{train}$ by an operation $\mathcal{HA(\cdot)}$. Keeping the strong relation intact between labels and low-frequency content (i.e. labels come from low-frequency-component image), we propose to swap high and low-frequency components of images in a batch on-the-fly. Unlike \cite{mukai2022improving}, we \textit{do not} restrict the images to belong to the same class; this diversifies the training distribution even further while preserving the image semantics. We call this basic version of our approach \textit{HybridAugment}, which corresponds to: 
%
\begin{equation} \label{hybrid_augment_paired}
    \mathcal{HA_{P}}(x_{i}, x_{j}) = \mathcal{LF}(x_{i}) + \mathcal{HF}(x_{j})
\end{equation}
%
where $x_{i}$ is the input image and $x_{j}$ is a randomly sampled image from the whole training set, which we simply sample from the mini batch at each training iteration in practice. $\mathcal{HF}$ and $\mathcal{LF}$ operators select the high and low-frequency components of an input image, for which we use:
%
\begin{equation} \label{eq:cutoff}
\begin{split}
    \mathcal{LF}(x) = GaussBlur(x) \\
    \mathcal{HF}(x) = x - \mathcal{LF}(x)
    \end{split}
\end{equation}
%
where $GaussBlur$ is used as a low-pass filter. Note that a similar outcome is possible by using Discrete Fourier Transforms (DFT), swapping the frequency bands and then applying Inverse DFT (IDFT). We find the gaussian blur operation to be faster and better in practice. 


Inspired from \cite{chen2021amplitude}, in addition to the image-pair scheme in Eq.~\ref{hybrid_augment_paired}, we propose a single image variant of \textit{HybridAugment}. In the single image variant, instead of combining two images, $x_i$ and $x_{j}$ are obtained by applying randomly sampled augmentations to a single image. The single image variant $\mathcal{HA_{S}}$ can therefore be defined as 
%
\begin{equation} \label{hybrid_augment_single}
    \mathcal{HA_{S}}(x_{i}) = \mathcal{LF}(Aug(x_{i})) + \mathcal{HF}(\hat{Aug}(x_{i}))
\end{equation}
%
where $Aug$ and $\hat{Aug}$ correspond to two sets of randomly sampled augmentation operations. Note that paired and single versions can work in tandem ($\mathcal{HA_{PS}}$), and actually outperform single or paired image versions. 


\subsection{HybridAugment++}


The frequency analysis is a vast literature, however, two core aspects often stand out; frequency-band analysis (i.e. low, high) and the decomposition of signals into amplitude and phase. \textit{HybridAugment} covers the former and shows competitive results in various benchmarks (see Section \ref{sec:exp_hybridaugment}). The latter is investigated in $\mathcal{APR}$ \cite{chen2021amplitude}, where phase is shown to be the more relevant component for correct classification, and training models based on their phase labels and swapping amplitude components of images randomly lead to more robust models. Note that frequency-band and phase/amplitude discussions are arguably orthogonal, since frequency, phase and amplitude provide distinct characterizations of a signal: intuitively speaking, frequency, phase and amplitude can be seen as the separation of visual patterns in terms of scale, location and significance. 


We hypothesize these two approaches can be complementary; a model reliant on low-frequency and spatial information (i.e. phase) can further improve robustness. Inspired by the successes of cascaded augmentation methods \cite{hendrycks2019augmix,wang2021augmax,calian2022defending}, we unify these two core aspects into a single, hierarchical augmentation method. We refer to this method as \textit{HybridAugment++} and define its paired version as:
%
\begin{equation}
  \mathcal{HA_{P}}^{++}(x_{i}, x_{j}, x_{z}) = \mathcal{APR_{P}}(\mathcal{LF}(x_{i}), x_{z}) + \mathcal{HF}(x_{j})
\end{equation}
%
where $x_{i}$, $x_{j}$ and $x_{z}$ are images sampled from the same batch. Here, $\mathcal{APR_{P}}$~\cite{chen2021amplitude} is defined as
\begin{equation}
    \mathcal{APR_{P}}(x_{i}, x_{z}) = \mathcal{IDFT}(A_{x_{z}} \otimes e^{i. P_{x_{i}}}) \\
\end{equation}
%
where $\otimes$ is element-wise multiplication, $A$ is the amplitude and $P$ is the phase component. Similar to $\mathcal{HA}$ and $\mathcal{APR}$, we also define a single-image version of \textit{HybridAugment++} as
%
\begin{equation}
 \mathcal{HA_{S}}^{++}(x_{i}) = \mathcal{APR_{S}}(\mathcal{LF}(Aug(x_{i}))) + \mathcal{HF}(\hat{Aug}(x_{i}))
\end{equation}
%
where $\mathcal{APR_{S}}$~\cite{chen2021amplitude} is defined as
%
\begin{equation}
\mathcal{APR_{S}}(x_{i}) = \mathcal{IDFT}\left(A_{\bar{Aug}(x_{i})} \otimes e^{i. P_{\overline{Aug}\left(x_{i}\right)}}\right)    
\end{equation}
%
where $Aug$, $\hat{Aug}$, $\bar{Aug}$ and $\overline{Aug}$ are different sets of randomly sampled augmentation operations. Note that we essentially propose a framework; one can use different single and paired image augmentations, either individually or together, and can still achieve competitive results (see ablations in Section \ref{sec:exp_hybridaugment}). There are also other alternatives, such as swapping phase/amplitude first and then performing $\mathcal{HA}$, but we observe poor performance in practice; dividing the phase component into frequency-bands is not interpretable as frequencies of the phase component are not well defined. The pseudo-code of our methods can be found in the supplementary material.






\section{Experiments}\label{sec:experiment}
%
We first introduce training and evaluation configurations of the proposed ICF-SRSR framework.
%
Then we conduct comprehensive experiments, extensive quantitative and qualitative comparisons with the other methods, and an in-depth analysis of our proposed method.

\subsection{Training details}
%
\Paragraph{Dataset.}
%
We train and evaluate our method on two scenarios. 1) Synthetic datasets, where the training and testing LR images are synthesized by a uniform degradation process~(\eg, bicubic down-sampling) from HR images. 2) Real-world datasets, which provide paired LR-HR images from the real-world captured by adjusting the focal length of a camera.

To train our ICF-SRSR, we use $800$ bicubic LR images from the DIV2K~\cite{agustsson2017ntire} dataset.
%
For evaluation, we adopt five standard benchmarks: Set5~\cite{bevilacqua2012low}, Set14~\cite{zeyde2010single}, BSD100~\cite{martin2001database}, Urban100~\cite{huang2015single}, and Manga109~\cite{Manga109}.
%
We also use the high-quality DIV2K validation set for evaluation.
%

To train and evaluate our ICF-SRSR under real-world scenarios, we utilize real-world datasets~\cite{cai2019toward, wei2020component} for the SISR task.
%
RealSR-V3~\cite{cai2019toward} includes paired LR-HR images captured by two different cameras, Canon and Nikon.
%
For each camera, about $200$ training images are captured from different scenes for each scaling factor $\times2$, $\times3$, and $\times4$.
%
We use only the LR images with scaling factors $\times2$ and $\times4$ for training and evaluate our method on the $50$ test pairs for each scale. 
%
DRealSR~\cite{wei2020component} also contains images captured by five DSLR cameras.
%
We conduct our experiments using images for $\times 2$ and $\times 4$ SR, containing 884 and 840 LR images, respectively.
%
For evaluation, we use $83$ and $93$ test pairs in DRealSR for $\times 2$ and $\times 4$, respectively.
\begin{table*}[t]
    \small
    \centering
    %\setlength\tabcolsep{1pt}
    \begin{tabularx}{\linewidth}{c c l c >{\centering\arraybackslash}X
    >{\centering\arraybackslash}X >{\centering\arraybackslash}X >{\centering\arraybackslash}X >{\centering\arraybackslash}X}
    \toprule
    \multirow{2}{*}{\textbf{Training Set}}& 
    \multirow{2}{*}{\textbf{Supervision}}&
    \multirow{2}{*}{\bf Method} & 
    %\multicolumn{3}{c}{\textbf{City100}}  & 
    %\multicolumn{3}{c}{\textbf{RealSR}} &
    \multicolumn{2}{c}{\textbf{RealSR~(Canon)}} & \multicolumn{2}{c}{\textbf{RealSR~(Nikon)}} &  \multicolumn{2}{c}{\textbf{DRealSR}} \\
    %\cline{4-7}
    & & &  \bf $\times2$ & \bf $\times4$ & \bf $\times2$ & \bf $\times4$  & \bf $\times2$ & \bf $\times4$\\ 
    %\cline{4-7}
    \midrule
    & & Bicubic & {30.35}  &  {25.80} & {29.66} & {25.50} & {32.67} & {30.56}\\
    \midrule
    \multirow{5}{*}{Synthetic} & \multirow{5}{*}{Supervised} 
    %& Bicubic & {}  & {-}& {-} & {-} & {-}\\
    & EDSR~\cite{lim2017enhanced}& \textbf{30.58}  & {26.05} &  \textbf{30.00}  & {25.89} & \textbf{32.82} & \textbf{30.64}\\
    & & RRDB~\cite{wang2018esrgan}& {-} & {26.05} &  {-}  & {25.91} & {-}& {-}\\
    &  & IKC~\cite{gu2019blind} & {-} & {25.71} &  {-}  & {25.27}& {-}& {-}\\
    & & BilndSR~\cite{sr_blindsr}& {25.80}   & {-}&  {24.17} & {-}& {-}& {-}\\
    & & DRN-S~\cite{guo2020closed} & \textbf{30.58} & \textbf{26.07} &  {29.99}  & \textbf{25.92} & {32.81}& {-}\\
    %&  & \textbf{IMF-SRSR$^{\dagger}$} & {30.62}  &  {-} & {30.00} & {-}& {32.84}& {-}\\
    %&  & \textbf{IMF-SRSR$^{\ddagger}$} & {}  &  {-} & {} & {-}& {-}& {-}\\
    %& \textbf{IMF-SRSR~(EDSR~(HR))} & {32.56}  &  {} & {} & {}\\    
    \midrule
    %BM3D~\cite{sparse} & {25.65} & {0.685}& {34.51} & {0.850} & {25.65} & {0.685}& {34.51} & {0.850} \\
    %WNNM~\cite{6909762}& {25.78} & {0.809} & {34.67} & {0.864} & {25.65} & {0.685}& {34.51} & {0.850} \\
    %K-SVD~\cite{DBLP:journals/corr/abs-1909-13164} & {26.88} & {0.842}& \textbf{36.49} & \textbf{0.897}& {25.65} & {0.685}& {34.51} & {0.850} \\
    %EPLL~\cite{Hurault_2018} & \textbf{27.11} & \textbf{0.870}  & {33.51} & {0.824}& {25.65} & {0.685}& {34.51} & {0.850} \\
    %\hline
    \multirow{8}{*}{Real-world} & \multirow{5}{*}{Supervised}
    & EDSR~\cite{lim2017enhanced}& {32.45}   & {27.59} &  \textbf{31.59} & {27.14} & {34.24}& \textbf{32.03}\\
    & & RRDB~\cite{wang2018esrgan}& {-} & \textbf{27.90} & {-} & \textbf{27.39} & {-}& {-}\\
    & & RCAN~\cite{zhang2018image}& {-} & {-} & {-} & {-} & \textbf{34.34} & {31.85}\\
    & & LP-KPN~\cite{cai2019toward}& {-} & {27.40} & {-} & {26.69} & {33.88} & {31.58}\\
    & & DRN-S~\cite{guo2020closed} & \textbf{32.50} & {-} & {31.43} & {-} & {33.91} & {-} \\
    %& & \textbf{IMF-SRSR}$^{\dagger}$ & {\textbf{32.58}} & {-} & {-} & {-} & {-} & {-} \\
    %& & IMF-SRSR$^{\ddagger}$ & {\textbf{32.68}} & {-} & {-} & {-} & {-} & {-} \\
    \cmidrule{2-9}
    & Unsupervised
    & ZSSR~\cite{shocher2018zero}+~\cite{bell2019blind} & {28.79} & {23.68} &  {27.54}  & {22.46}& {-}& {-}\\
    \cmidrule{2-9}
    & \multirow{2}{*}{Self-supervised} 
    %\textbf{IMF-SRSR~(Test)} & {30.67}  &  {26.08} & {29.99} & {25.76}& {32.83}& {30.62}\\
    & \textbf{ICF-SRSR} & {30.98}  &  {26.26} & {30.31} & {25.89}& {32.87}& {30.65}\\
    & & \textbf{EDSR}~(LLR,LR) & \textbf{31.13}  &  \textbf{26.32} & \textbf{30.33} & \textbf{25.92} & \textbf{32.91}& \textbf{30.68}\\
    \bottomrule
    \end{tabularx}
    %\vspace{1mm}
    \caption{
        \textbf{Quantitative comparison on real-world datasets.} We compare our self-supervised ICF-SRSR and EDSR~(LLR,LR), \ie, the model EDSR~\cite{lim2017enhanced} trained on our generated paired dataset (LLR,LR), to several supervised/unsupervised methods trained on synthetic DIV2K~\cite{agustsson2017ntire}, real-world RealSR-V3~\cite{cai2019toward} and DRealSR~\cite{wei2020component} datasets for scales $\times2$ and $\times4$.
        %
    }
    \label{tab:real_data}
    \vspace{-4mm}
\end{table*}


%
\Paragraph{Hyperparameters.}
%
During the training, we extract random patches of size $48\times48$ from LR images of both synthetic and real-world datasets.
%
For all our experiments, we set the batch size to $16$, and  $\lambda_{\text{Color}}=0.2$.
%
Random flip and rotation augmentations are applied to the input images to increase the number of effective training samples. 
%
We train our model using ADAM~\cite{kingma2017adam} optimizer with the initial learning rate $1 \times 10^{-4}$, which decays by a factor $0.5$ after every $200$ epochs.
%
For quantitative comparisons, we adopt structural similarity~(SSIM)~\cite{measure_ssim} and peak signal-to-noise ratio~(PSNR) on the luminance channel for the experiments on synthetic datasets and real-world dataset DRealSR~\cite{wei2020component} and also on RGB channels for dataset RealSR-V3~\cite{cai2019toward}.
%
All experiments are done using PyTorch 1.8.1 and Quadro RTX 8000 GPUs.
%%%%%%%%%%%%%%%%%%%%%%%%%%%%%%%%%%%%%%%%%%%%%%%%%%%%%%%%%%%%%%%%%%%%%%%%%%%%%%%%%%%%%%%%
\subsection{Evaluation on synthetic datasets}\label{sec:ex_syn}
%
We train our ICF-SRSR on the DIV2K~\cite{agustsson2017ntire} dataset with EDSR-baseline~\cite{lim2017enhanced} and test it on the public benchmark datasets~\cite{bevilacqua2012low, zeyde2010single, martin2001database, huang2015single, Manga109} and also the validation set of DIV2K.
%
We note that the proposed method is trained in a self-supervised manner by targeting a certain scale $s$.
%
Specifically, we train $\left( \times 2, \times \nicefrac{1}{2} \right)$ ICF and $\left( \times 4, \times \nicefrac{1}{4} \right)$ ICF independently.
%
\cref{tab:benchmark} shows extensive comparisons between the proposed self-supervised approach and the other representative supervised/unsupervised SR methods with PSNR metric.
%
We demonstrate that our ICF-SRSR approach achieves superior performance compared to the SelfExSR~\cite{huang2015single} model and comparable performance to the other unsupervised and supervised methods.
%
We note that ground-truth HR images in Set5 and Set14 are relatively noisier than the other datasets, preventing our self-supervised framework from learning accurate scaling functions.
%
We will discuss more details about the noisy cases in our supplementary material.
%
Notably, ICF-SRSR outperforms the unsupervised method ZSSR~\cite{shocher2018zero} by $1.05$dB on scale $\times2$ of Urban100 dataset and the supervised methods~\cite{kim2016accurate, chen2021learning} on both scales of DIV2K validation set.

Moreover, we apply the trained ICF-SRSR to LR images from the DIV2K training dataset and get LLR-LR paired images.
%
Then, we train off-the-shelf EDSR on the synthesized paired data from scratch and evaluate it on the test datasets as shown in \cref{tab:benchmark}.
%
The results demonstrate that EDSR~(LLR, LR) trained on our generated pairs~(LLR, LR) achieves superior performance than ICF-SRSR, which illustrates the merit of our method to generate useful training image pairs.
%

\cref{fig:benchmark} further visualizes the qualitative results of ICF-SRSR on two validation images from the DIV2K~\cite{agustsson2017ntire} dataset.
%
Our method achieves comparable results to the supervised methods~\cite{lim2017enhanced,chen2021learning} while restoring more details compared to the unsupervised methods~\cite{shocher2018zero,soh2020meta}. 
%
We note that the results on ZSSR~\cite{shocher2018zero} show lost information and scratched texts, and on MZSR~\cite{soh2020meta} include severe artifacts and color shifting.
%
For an in-depth comparison, we also provide quantitative results with SSIM metric in our supplementary material.
%

\subsection{Evaluation on real-world datasets}\label{sec:ex_real}
%
We train and evaluate ICF-SRSR for each scale $\times 2$ and $\times 4$ independently on the LR images of each Canon and Nikon camera from the real-world dataset RealSR-V3~\cite{cai2019toward} separately and also on the LR images of the real-world dataset DRealSR~\cite{wei2020component} in a self-supervised manner.
%
We further train the model EDSR~\cite{lim2017enhanced} on our generated~(LLR, LR) image pairs. 
%
We compare our method with the supervised methods~\cite{lim2017enhanced, wang2018esrgan, zhang2018image, cai2019toward, guo2020closed} trained on real paired images, which serve as the upper bounds for the SR problem.
%
\input{sections/figures/realsr2}

On the other hand, we employ the pre-trained supervised models EDSR~\cite{lim2017enhanced}, RRDB~\cite{wang2018esrgan}, IKC~\cite{gu2019blind}, BlindSR~\cite{sr_blindsr} and DRN-S~\cite{guo2020closed} on the synthetic DIV2K~\cite{agustsson2017ntire} dataset to super-resolve the LR images in the testing sets of RealSR-V3~\cite{cai2019toward} and DRealSR~\cite{wei2020component}.
%
Moreover, we utilize Kernel-GAN~\cite{bell2019blind} to approximate the down-sampling kernel from a single LR image and use ZSSR~\cite{shocher2018zero} as a zero-shot SR to apply to real LR images.
%
Our extensive comparisons with the various methods trained on real and synthetic datasets are summarized in \cref{tab:real_data}.
%
We illustrate that our self-supervised method can achieve superior performance compared to the methods pre-trained on the synthetic datasets and unsupervised method ZSSR~\cite{shocher2018zero}+Kernel-GAN~\cite{bell2019blind} in terms of both PSNR and SSIM metrics, which emphasizes the fact that the trained models on synthetic datasets with known degradations cannot perform well on real-world scenarios.
%
We qualitatively compare our method with the various existing methods on the RealSR-V3 dataset and visualize the SR results and their corresponding error maps with respect to the GT~(HR) in \cref{fig:realsr}. 
%
We demonstrate that our self-supervised method can achieve comparable and sometimes better performance to the supervised method LP-KPN~\cite{cai2019toward} trained on real paired images. 
%
We note that our method is generally more suitable for restoring the texture and preserving color compared to supervised method IKC~\cite{gu2019blind} and unsupervised method ZSSR~\cite{shocher2018zero}+Kernel-GAN~\cite{bell2019blind} as evident in appearance and PSNR, SSIM, and mean absolute error~(MAE) metrics.
%
We show more qualitative results in the supplementary material.
%
\subsection{Ablation study}\label{sec:ablation}
%
We conduct various ablation studies on the model design, down-sampling operators, few-shot learning, augmentation, and the effect of loss functions to better analyze our method.

\Paragraph{Model design.}
%
We conduct an experiment to show the superiority of a developed baseline as a single conditional model compared to two independent models and also the effect of training our two-stage framework compared to training each Up-Down and Down-Up stage separately.
%
Our results on synthetic dataset DIV2K~\cite{agustsson2017ntire} and Canon and Nikon images from real-world dataset RealSR-V3~\cite{cai2019toward} for scale $\times2$ show that training with two independent models or using only one stage~(half) results in unsatisfactory performance, demonstrating the uniqueness of our method in using a single invertible scale-conditional model as shown in \cref{tab:reb-two-models}.

\begin{table}[h]
    %\vspace{-3mm}
    \small
    \centering
    %\setlength\tabcolsep{0.5pt}
    %\setlength\tabcolsep{0.0001pt}
    %\resizebox{0.65\columnwidth}{!}{
    \begin{tabularx}{\linewidth}{l 
    >{\centering\arraybackslash}X 
    >{\centering\arraybackslash}X 
    >{\centering\arraybackslash}X  
    }
    \toprule
    \textbf{Method} 
    & \textbf{DIV2K~($\times 2$)}&
    \textbf{Canon~($\times 2$)} & \textbf{Nikon~($\times 2$)}  \\
    %\cline{4-7}
    \midrule
     Two Models & {34.81} & {30.61} & {30.01} \\
     Up-Down & {29.92}& {28.56} & {27.52}  \\
     Down-Up & {34.59}& {30.58} & {30.00} \\
     \midrule
     \textbf{ICF-SRSR} & \textbf{35.19} & \textbf{30.98} & \textbf{30.31} \\
    \bottomrule
    \end{tabularx}
    \vspace{-2mm}
    \caption{
    \textbf{Ablation on model design.}
        %
    }
    \label{tab:reb-two-models}
    \vspace{-2mm}
\end{table}



%

\Paragraph{Evaluation of down-sampling.} 
%
Due to the invertibility attribute of ICF, our method can be interpreted as a learnable down-sampler.
%
Therefore, we analyze our model $f_{\theta}$ as a down-sampling operator in three aspects.
%

\Paragraph{First.} We train ICF-SRSR on HR images from RealSR-V3~\cite{cai2019toward} and evaluate the model on HR images of the test dataset to gather the generated down-sampled images. 
%
Then, we compare ground-truth LR images with our generated LR images, as well as LR images obtained by down-sampling functions \eg, Nearest, Bicubic, Gaussian+Nearest, and Gaussian+bicubic~($\sigma=0.4$).
%
\cref{tab:downsampling} provides a comparison of LR images for different down-sampling models based on PSNR.
%
The values show the superiority of our learnable down-sampling method in generating more realistic LR images compared to ones with other down-sampling operators. 

\begin{table}[h]
    \small
    \centering
    %\setlength\tabcolsep{0.5pt}
    \begin{tabularx}{\linewidth}{l >{\centering\arraybackslash}X >{\centering\arraybackslash}X >{\centering\arraybackslash}X  >{\centering\arraybackslash}X
    }
    \toprule
    \multirow{2}{*}{\textbf{Down-sampling}} & 
    \multicolumn{2}{c}{\textbf{Canon}} & 
    \multicolumn{2}{c}{\textbf{Nikon}} \\
    %\cline{4-7}
    & $\times2$ & $\times4$ & $\times2$ & $\times4$\\
    %\cline{4-7}
    
    \midrule
     Nearest & {29.35} & {24.51} & {28.54} & {23.91} \\
     Bicubic & {30.27} & {25.76} & {29.71} & {25.56} \\
     Gaussian+Nearest & {29.62} & {24.65} & {28.87} & {24.09} \\
     Gaussian+Bicubic & {30.61} & {25.95} & {30.12} & {25.81} \\
     
     \midrule
     \textbf{ICF-SRSR} &  \textbf{32.46} & \textbf{28.93} & \textbf{32.12} & \textbf{29.15}\\
    \bottomrule
    \end{tabularx}
    
    \vspace{-2mm}
    \caption{
        \textbf{Ablation on down-sampling performance.} %Comparisons of the generated LR images from the given HR images using various known down-sampling operators with our learnable down-sampling method ICF-SRSR.
        %
    }
    \label{tab:downsampling}
    \vspace{-4mm}
\end{table}

\Paragraph{Second.} 
%
We further analyze our learnable down-sampling operator $f_{\theta}$ compared to non-learnable down-sampling approaches.
%
We use our learnable down-sampling operator $f_{\theta}$, bicubic down-sampling, and Gaussian~($\sigma=0.4$) filtering followed by different nearest and bicubic down-sampling operators to generate the LLR images from given input LR images on the training sets.
%
Then, we train the model EDSR on the generated paired images~(LLR, LR) to learn generating SR images given LR counterparts.
%
We summarize the results for scale $\times2$ of the benchmarks Set5~\cite{bevilacqua2012low} and Set14~\cite{zeyde2010single}, and Canon and Nikon sets of RealSR-V3~\cite{cai2019toward} dataset for both non-learnable and our learnable down-sampling operators in \cref{tab:downsampling2}.
%
The results indicate the effect of our learnable down-sampling operator to generate appropriate image pairs for training, which results in a significant improvement compared to known down-sampling operators.

\begin{table}[h]
    \small
    \centering
    %\setlength\tabcolsep{0.5pt}
    \begin{tabularx}{\linewidth}{l >{\centering\arraybackslash}X >{\centering\arraybackslash}X  >{\centering\arraybackslash}X 
    >{\centering\arraybackslash}X
    }
    \toprule
    \textbf{Down-sampling}  & 
    \textbf{Set5} & \textbf{Set14} & 
    \textbf{Canon} & 
    \textbf{Nikon}  \\
    \midrule
     Bicubic  & {35.30} & {31.53} & {30.41} & {29.80}\\
     Gaussian+Nearest  & {30.79} & {28.39} & {29.41} & {28.60}  \\
     Gaussian+Bicubic  & {35.43} & {31.84} & {30.47} & {29.86} \\
     \midrule
     \textbf{ICF-SRSR} & \textbf{37.09} & \textbf{32.91} &  \textbf{31.13} & \textbf{30.33}\\
    \bottomrule
    \end{tabularx}
    
    \vspace{-2mm}
    \caption{\textbf{Comparison with non-learnable down-sampling operators to generate paired training data for SR task.}
        %\textbf{SISR results using different down-sampling operators.}
        %Effects of different down-sampling operators to generate appropriate paired images for training.
        %
    }
    \label{tab:downsampling2}
    \vspace{-4mm}
\end{table}


\Paragraph{Third.}
%
By using different down-sampling methods, we first generate LR samples from the real training HR images and then train a vanilla EDSR model using the generated pairs, \ie, (LR, HR).
%
As shown in \cref{tab:reb-down}, our synthesized pairs can provide more suitable training data compared to ones by previous learnable down-sampling methods ADL~\cite{son2021toward} and DRN-S~\cite{guo2020closed} as the EDSR performs much better for the $\times 2$ SR tasks on real dataset RealSR-V3~\cite{cai2019toward}.

\begin{table}[h]
    %\vspace{-3mm}
    \small
    \centering
    %\setlength\tabcolsep{0.5pt}
    %\setlength\tabcolsep{0.0001pt}
    %\resizebox{0.65\columnwidth}{!}{
    \begin{tabularx}{\linewidth}{l 
    >{\centering\arraybackslash}X 
    >{\centering\arraybackslash}X 
    }
    \toprule
    \textbf{Downsampling} 
    & \textbf{Canon~$(\times2)$} & \textbf{Nikon~$(\times2)$} \\
    %\cline{4-7}
    \midrule
     ADL~[\textcolor{blue}{49}] & {30.76} & {30.44}  \\
     DRN-S~[\textcolor{blue}{20}] & {30.82} & {30.24}  \\
     \midrule
    \textbf{ICF-SRSR} & \textbf{31.94} & \textbf{31.24}\\
    \bottomrule
    \end{tabularx}
    \vspace{-2mm}
    \caption{\textbf{Comparison with learnable down-sampling operators to generate paired training data for SR task.}
    %    \textbf{}
        %
    }
    \label{tab:reb-down}
    \vspace{-2mm}
\end{table}





\paragraph{Few-shot learning.}
%
We train and evaluate our method on small datasets to show the advantage of our method to learning from only a few images without requiring a large-scale training dataset.
%
Therefore, we train the model ICF-SRSR~(Small) on the test sets of synthetic datasets Set14~\cite{zeyde2010single}, BSD100~\cite{martin2001database} and Urban100~\cite{huang2015single} and also real-world datasets RealSR-V3~\cite{cai2019toward} and DRealSR~\cite{wei2020component} and show their results on the corresponding test datasets in \cref{tab:ab-few-shot}. 
%
We demonstrate that our method can achieve slightly lower performance even when trained on very small datasets compared to our model ICF-SRSR~(Large) trained on large-scale training datasets.
\begin{table}[h]
    %\vspace{-3mm}
    \small
    \centering
    %\setlength\tabcolsep{0.5pt}
    \begin{tabularx}{\linewidth}{l >{\centering\arraybackslash}X
    >{\centering\arraybackslash}X
    >{\centering\arraybackslash}X >{\centering\arraybackslash}X  >{\centering\arraybackslash}X >{\centering\arraybackslash}X     }
    \toprule
    \multirow{2}{*}{\textbf{Training set}} &\multicolumn{2}{c}{\textbf{Set14}} & 
    \multicolumn{2}{c}{\textbf{BSD100}} & \multicolumn{2}{c}{\textbf{Urban100}} \\
    %\cline{4-7}
    & $\times2$ & $\times4$ & $\times2$ & $\times4$ & $\times2$ & $\times4$\\
    \midrule
     Large & {32.86} & {27.76} & {31.54} & {26.99} & {30.39} & {24.72} \\
     Small & {32.44} & {27.19} & {31.34} & {26.82} & {30.26} & {24.66}  \\
     \midrule
    \multirow{2}{*}{\textbf{Training set}} &
    \multicolumn{2}{c}{\textbf{Canon}} & 
    \multicolumn{2}{c}{\textbf{Nikon}} & \multicolumn{2}{c}{\textbf{DRealSR}} \\
    %\cline{4-7}
    & $\times2$ & $\times4$ & $\times2$ & $\times4$ & $\times2$ & $\times4$\\
    %\cline{4-7}
    \midrule
     Large & {30.98} & {26.26} & {30.31} & {25.89} & {32.87} & {30.65} \\
     Small & {30.67} & {26.08} & {29.99} & {25.76} & {32.83} & {30.62}  \\
    \bottomrule
    \end{tabularx}
    \vspace{-2mm}
    \caption{
        \textbf{Few-shot learning.}
        %
        %For Set14, BSD100, and Urban100, Large denotes the DIV2K training set, while Small corresponds to each test dataset.
        %Similarly, for Canon, Nikon, and DRealSR, Large denotes training splits of the corresponding dataset and Small is their test splits.
    }
    \label{tab:ab-few-shot}
    \vspace{-2mm}
\end{table}




\Paragraph{Multi-scale augmentation.}
%
As we mention in \cref{sec:net_arch}, augmented data with different scales can lead to performance improvement.
%
Therefore, when we train ICF-SRSR directly on the test samples, we adopt diverse scaling factors as well as their reciprocals to compensate for the limited number of training data.
%
In \cref{tab:ab-scale}, we show that increasing the number of inputs induced by various scaling factors, \eg, $\times2$, $\times4$, and $\times8$, and their inverses can lead to obtaining superior performance on the RealSR-V3~\cite{cai2019toward} dataset.
%
More details about our multi-scale augmentation strategy are described in our supplementary material.

\begin{table}[t]
    %\vspace{-3mm}
    \small
    \centering
    %\setlength\tabcolsep{0.5pt}
    %\setlength\tabcolsep{0.0001pt}
    %\resizebox{0.65\columnwidth}{!}{
    \begin{tabularx}{\linewidth}{l 
    >{\centering\arraybackslash}X 
    >{\centering\arraybackslash}X 
    >{\centering\arraybackslash}X 
    >{\centering\arraybackslash}X 
    }
    \toprule
    \textbf{Scale} 
    & \textbf{Canon~$(\times2)$} & \textbf{Nikon~$(\times2)$} \\
    %\cline{4-7}
    \midrule
     \text{2} & {30.67} & {29.99}  \\
     \text{2,4} & {30.75} & {30.09} \\
      \text{2,4,8} & \textbf{30.78} & \textbf{30.11} \\
    \bottomrule
    \end{tabularx}
    \vspace{-2mm}
    \caption{
        \textbf{Multi-scale augmentation.}
        %
    }
    \label{tab:ab-scale}
    \vspace{-2mm}
\end{table}





\Paragraph{Effects of loss functions.}
%
\textcolor{blue}
We also analyze the effect of each loss function discussed in \cref{sec:loss}.
%
As shown in \cref{tab:ab-loss}, our novel self-supervised consistency loss $\mathcal{L}^{\text{Cons}}$ can drastically improve the model performance when it is added to color preserving loss $\mathcal{L}^{\text{Color}}$ on both synthetic and real-world datasets.
%
In our supplementary material, we further discuss the effect of the weight $\lambda_{\text{Color}}$.

\begin{table}[h]
    %\vspace{-3mm}
    \small
    \centering
    %\setlength\tabcolsep{0.5pt}
    %\setlength\tabcolsep{0.0001pt}
    %\resizebox{0.65\columnwidth}{!}{
    \begin{tabularx}{\linewidth}{l 
    >{\centering\arraybackslash}X 
    >{\centering\arraybackslash}X 
    >{\centering\arraybackslash}X  
    }
    \toprule
    \textbf{Loss} 
    & \textbf{DIV2K~$(\times2)$} & \textbf{Canon~$(\times2)$} & \textbf{Nikon~$(\times2)$}\\
    %\cline{4-7}
    \midrule
     $\mathcal{L}^{\text{Color}}$ only & {30.31} & {29.12} & {28.38} \\
     $\mathcal{L}^{\text{Color}}$,$\mathcal{L}^{\text{Cons}}$ & \textbf{35.19} & \textbf{30.98} & \textbf{30.31}\\
    \bottomrule
    \end{tabularx}
    \vspace{-2mm}
    \caption{
        \textbf{Effect of loss functions.}
        %
    }
    \label{tab:ab-loss}
    \vspace{-5mm}
\end{table}






\section{Conclusion and Future Work}
In this work, I design corruption-robust algorithms for the Lipschitz contextual search problem. I present the \emph{agnostic checking} technique and demonstrate its effectiveness in designing corruption-robust algorithms. There are several open problems for future research. First, in the algorithm I propose for pricing loss, the schedule for agnostic checks is fixed upfront. Can the learner design an adaptive checking schedule for the pricing loss? Second, this work assumes the learner has knowledge of the Lipschitz constant $L$. Can the learner design efficient no-regret algorithms without knowledge of $L$? 



%%%%%%%%% REFERENCES
{\small
\bibliographystyle{ieee_fullname}
\bibliography{ICF-SRSR}
}

%\begin{comment}
\section{System Architecture}
\label{appendix:architecture}
\system has a novel modularized system architecture with three key components: 
\emph{StreamManager}, 
\emph{TxnManager} and \emph{TxnScheduler}. 
These components are instantiated in each thread locally.
The execution outline of \system is presented in Algorithm~\ref{alg:algo}.
Transactional stream processing is continuous and potentially never ends (Line 1$\sim$8).
The dependency resolution and execution of state transactions are separated into two non-overlapping phases by punctuations~\cite{Tucker:2003:EPS:776752.776780} (Line 2 and 5), which guarantees that no subsequent input event will have a smaller timestamp. 
Effectively, a batch of state transactions is collected during the first phase, and processed during the second phase.

In the first phase (i.e., stream processing phase), 
the \emph{StreamManager} conducts preprocessing for every input event ($e$). Similar to some prior works~\cite{tstream}, state transactions may be issued but not immediately processed during preprocessing (Line 3).
The \emph{pre\_processing} and \emph{post\_processing} functions are exposed as APIs to users.
The \emph{TxnManager} handles dependency resolution (Line 4) among state transactions and insert decomposed operations to construct a \tpg. We discuss the detailed two-phase \tpg construction process in Section~\ref{subsec:construction}.

In the second phase  (i.e., transaction processing phase), 
the \emph{TxnManager} is first involved again to refine (Line 6) the constructed \tpg with further dependency resolution.
The \emph{TxnScheduler} 
schedules operations for concurrent execution based on the constructed \tpg according to the three dimensions of scheduling decisions (Line 7). 
In particular, a scheduling decision model $M$ is instantiated based on the constructed \tpg (Line 14).
\textbf{\circled{1}} Guided by $M$, execution threads adopt an exploration strategy (Section~\ref{subsec:explore}) to explore the constructed \tpg for operations available to be scheduled constrained by dependencies. 
\textbf{\circled{2}} 
During exploration, one or multiple operations may be treated as the 
% basic 
unit of scheduling (Section~\ref{subsec:granularity}). 
Subsequently, \textbf{\circled{3}} every thread executes operation(s) in the unit of scheduling with various abort handling mechanisms (Section~\ref{subsec:abort_handling}).
Only when state transactions are processed (i.e., committed or aborted) can the associated input events be postprocessed (Line 8) by the \emph{StreamManager} based on transaction processing results.
\end{comment}

\begin{comment}
\begin{algorithm}
\footnotesize
    \KwData{$e$ \tcp{Input event}}
    \KwData{$txn_{ts}$ \tcp{State transaction}}
    \KwData{$G$ \tcp{The currently constructed TPG}}
    \While{!finish processing of input streams}{
        \eIf(\tcp*[h]{Phase 1}){\text{$e$ is not a $punctuation$}}{
                $txn_{ts}$ $\gets$ PRE\_Processing($e$)\;
                \textbf{TPG\_Construction}($G$, $txn_{ts}$)\; 
          }(\tcp*[h]{Phase 2}){
                \textbf{TPG\_Refinement}($G$)\; 
                \textbf{TXN\_Scheduling}($G$)\; 
                POST\_Processing()\;
          }
    }
    
    \SetKwFunction{FMain}{TPG\_Construction}
    \SetKwProg{Fn}{Function}{:}{}
    \Fn{\FMain{$G$, $txn_{ts}$}}{
        $O_{1..k}$ $\gets$ \textbf{Partition} $txn_{ts}$\;
        \ForEach{\text{operation $O_{i}$ $\in$ $O_{1..k}$}}{
            \textbf{Identify} its \ld\;
            $G$ $\gets$ $G$ + $O_{i}$ \;
        }
    }
    \SetKwFunction{FMain}{TPG\_Refinement}
    \SetKwProg{Fn}{Function}{:}{}
    \Fn{\FMain{$G$}}{
        \ForEach{\text{vertex $e_{i}$ $\in$ $G$}}{
            \textbf{Identify} its \td, \pd\;
        }
    }
    
    \SetKwFunction{FMain}{TXN\_Scheduling}
    \SetKwProg{Fn}{Function}{:}{}
    \Fn{\FMain{$G$}}{
        $M$ $\gets$ Instantiated with $G$;\tcp{A decision model}
        \While{!finish scheduling of $G$
        }{
          \textbf{\circled{2}} $Scheduling Unit$ $\gets$ \textbf{\circled{1}} \emph{Explore}($G$, $M$)\; 
            \textbf{\circled{3}} \emph{Execute with Abort Handling} ($Scheduling Unit$)\; 
        }
    }
  \caption{Execution Outline of \system}
  \label{alg:algo}
\end{algorithm}
\end{comment}
%%%%%%%%%%%%%%%%%%%%%%%%%%%%%%%%%%%%%%%%%%%%%%%%%%%%%%%%%%%%
\appendixpageoff
\appendixtitleoff

\begin{appendices}

%%%%%%%%% TITLE - PLEASE UPDATE
\title{Supplementary Material \textit{for} \\
ICF-SRSR: Invertible scale-Conditional Function \textit{for}\\ Self-Supervised Real-world Single Image Super-Resolution}

\author{Reyhaneh Neshatavar$^{1\ast}$ \qquad Mohsen Yavartanoo$^{1\ast}$ \qquad Sanghyun Son$^{1}$ \qquad Kyoung Mu Lee$^{1,2}$ 
\\$^{1}$Dept. of ECE \& ASRI, $^{2}$IPAI, Seoul National University, Seoul, Korea\\
{\tt\small \{reyhanehneshat,myavartanoo,thstkdgus35,kyoungmu\}@snu.ac.kr}}

\maketitle
%\renewcommand{\thetable}{S\arabic{table}}
%\renewcommand{\thefigure}{S\arabic{figure}}
%\renewcommand{\theequation}{S\arabic{equation}}
\renewcommand{\thesection}{S\arabic{section}}
\renewcommand{\thefigure}{S1}
% Figure environment removed     



\renewcommand{\thefigure}{S2}
% Figure environment removed     



%%%%%%%%%%%%%%%%%%%%%%%%%%%%%%%%%%%%%%%%%%%%%%%%%%%%%%%%%
\section{Details of network architecture}
%
As described in Section~\textcolor{red}{3.4} of our main manuscript, our ICF-SRSR adopts EDSR~\cite{lim2017enhanced} as a baseline.
However, to handle both up-sampling and down-sampling operations with the same network, we slightly modify the tail part of the original EDSR architecture for each scaling factor, \eg, $\times2$ and $\times4$, and their inverses.
%
Figure~\ref{fig:supp_edsr} shows the original EDSR~(Figure~\ref{fig:supp_edsr}\subref{fig:supp_edsr_original}) and our modified EDSR~(Figure~\ref{fig:supp_edsr}\subref{fig:supp_edsr_developed}). 
%
We use the pixel-unshuffle operator to down-sample an input image and generate the corresponding LLR image.
%
For more stable optimization, we use the detach operator of PyTorch before passing the first outputs to the network again.


\section{Details of multi-scale augmentation strategy}
%
As we mention in Section~\textcolor{red}{4.4} of our main manuscript, we can generate images with various scaling factors, \eg, $\times2$, $\times4$, and $\times8$ and their corresponding inverses from a single LR input.
%
Figure~\ref{fig:supp_multi1} shows our multi-tail architecture, which introduces a tail for each of the scale conditions.
%by introducing a tail for each scale condition as shown in Figure~\ref{fig:supp_multi1}.
%
Then, we pass the generated output images of different scales to the model $f_\theta$ with their inverse scaling factors.
%
By doing so, we reconstruct the input LR image as shown in Figure~\ref{fig:supp_multi2}.
%
Accordingly, to train our model $f_\theta$ under such a configuration, we minimize the loss functions $\mathcal{L}^{\text{Cons}}$ and $\mathcal{L}^{\text{Color}}$ defined in Section~\textcolor{red}{3.3} of our main manuscript between the generated images and the input LR image.



\renewcommand{\thetable}{S1}
\begin{table*}[t]
    \small
    \centering
    %\setlength\tabcolsep{0.5pt}
    \begin{tabularx}{\linewidth}{c l >{\centering\arraybackslash}X >{\centering\arraybackslash}X >{\centering\arraybackslash}X >{\centering\arraybackslash}X
    >{\centering\arraybackslash}X 
    }
    \toprule
    \multirow{2}{*}{\textbf{Supervision}} & 
    \multirow{2}{*}{\bf Method} & 
    \textbf{Set5} & 
    \textbf{Set14} &
    \textbf{BSD100} &
    \textbf{Urban100} &
    \textbf{Manga109}\\
    %\cline{4-7}
    & & $\times2$/$\times4$ & $\times2$/$\times4$ & $\times2$/$\times4$ & $\times2$/$\times4$ & $\times2$/$\times4$   \\
    %\cline{4-7}
    
    \midrule
    %BM3D~\cite{sparse} & {25.65} & {0.685}& {34.51} & {0.850} & {25.65} & {0.685}& {34.51} & {0.850} \\
    %WNNM~\cite{6909762}& {25.78} & {0.809} & {34.67} & {0.864} & {25.65} & {0.685}& {34.51} & {0.850} \\
    %K-SVD~\cite{DBLP:journals/corr/abs-1909-13164} & {26.88} & {0.842}& \textbf{36.49} & \textbf{0.897}& {25.65} & {0.685}& {34.51} & {0.850} \\
    %EPLL~\cite{Hurault_2018} & \textbf{27.11} & \textbf{0.870}  & {33.51} & {0.824}& {25.65} & {0.685}& {34.51} & {0.850} \\
    %\hline
    & Bicubic & {0.929/0.810} & {0.868/0.702} & {0.843/0.667} & {0.840/0.657} & {0.933/0.789}  \\
    \midrule
    \multirow{8}{*}{Supervised}
    & VDSR~\cite{kim2016accurate}& {0.959/0.884} & {0.912/0.768} & {0.896/0.725} & {0.914/0.752} & {0.975/0.887}  \\
    & EDSR~\cite{lim2017enhanced}& {0.960/0.898} & {0.919/0.787} & {0.901/0.742} & {0.935/0.803} & {0.977/0.915}  \\
    & CARN~\cite{ahn2018fast}& {0.959/0.894} & {0.916/0.781} & {0.897/0.735} & {0.925/0.784} & {0.976/0.908}  \\
    & RCAN~\cite{zhang2018image} & {0.961/0.900} & {0.921/0.788} & {0.902/0.743} & {0.938/0.806} & {0.978/0.917} \\
    & RDN~\cite{zhang2018residual} & {0.961/0.899} & {0.921/0.787} & {0.901/0.741} & {0.935/0.802} & {0.978/0.915}  \\
    & DRN-S~\cite{guo2020closed} & {0.960/0.901} & {0.910/0.790} & {0.900/0.744} & {0.920/0.807} & {\textbf{0.980}/0.919}  \\
    & LIIF~\cite{chen2021learning} & {0.933/0.898} & {0.882/0.788} & {0.871/0.742} & {0.905/0.805} & {-}  \\
    &ELAN~\cite{ELAN-light} & {\textbf{0.962/0.902}} & {\textbf{0.922}/0.791} & {\textbf{0.903/0.745}} & {\textbf{0.939/0.816}} & {0.979/\textbf{0.922}} \\
    %&  \textbf{IMF-SRSR$^{\dagger}$} & {} & {-} & {-} & {-} & {-} & {-} \\
    %&  \textbf{IMF-SRSR$^{\ddagger}$} & {-} & {-} & {-} & {-} & {-} &  {-} \\
    \midrule
    \multirow{3}{*}{Unsupervised} 
    & SelfExSR~\cite{huang2015single} & {0.953/0.861} & {0.903/0.751} & {0.885/0.710} & {-} & {-} \\
    & ZSSR~\cite{shocher2018zero}  & {\textbf{0.957/0.879}} & {\textbf{0.910/0.765}} & {\textbf{0.892/0.721}} & {-} & {-} \\
    &  MZSR~\cite{soh2020meta} &  {0.956/~~~~-~~~~} & {-} & {\textbf{0.892}/~~~~-~~~~} & {\textbf{0.909}/~~~~-~~~~} & {-}  \\
    %&  DASR~[\textcolor{blue}{53}] &  \textbf{-/-} & {\textbf{-/-}} & {\textbf{-/-}} & {\textbf{-/-}} & {-}  \\
    \midrule
    \multirow{2}{*}{Self-supervised}
    %&  \textbf{IMF-SRSR~(Test)} & {36.41/29.49} & {32.44/27.19} & {31.34/26.82} & {30.26/24.66} & {36.29/27.82} & {35.02/29.45} \\
    &  \textbf{ICF-SRSR} & {0.956/0.874} & {0.908/0.760} & {0.888/0.715} & {0.910/0.740} & {0.970/0.872}  \\
    &  \textbf{EDSR}~(LLR,LR) & {\textbf{0.957/0.876}} & {\textbf{0.909/0.763}} & {\textbf{0.889/0.717}} & {\textbf{0.911/0.745}} & {\textbf{0.971/0.876}} \\ 
    %\multirow{3}{*}{(fully self-supervised)}
    \bottomrule
    \end{tabularx}
    
    %\vspace{-2mm}
    \caption{
        \textbf{Quantitative comparisons of different methods on synthetic datasets by SSIM.} 
        %
        We compare our ICF-SRSR with several supervised and unsupervised methods on the five standard benchmark datasets~\cite{bevilacqua2012low, zeyde2010single, martin2001database, huang2015single, Manga109} on scales $\times 2$ and $\times 4$. 
        %
        ICF-SRSR refers to our self-supervised method, while EDSR~(LLR,LR) is the model EDSR trained on our generated pairs (LLR,LR) of the DIV2K dataset.
        %
        We also note that MZSR does not report SSIM for $\times 4$ SR in the original paper.
    }
    \label{tab:supp-benchmark}
    %\vspace{-2mm}
\end{table*}


\renewcommand{\thetable}{S2}
\begin{table*}%[t]
    \small
    \centering
    %\setlength\tabcolsep{0.5pt}
    \begin{tabularx}{\linewidth}{l >{\centering\arraybackslash}X >{\centering\arraybackslash}X >{\centering\arraybackslash}X  >{\centering\arraybackslash}X
    >{\centering\arraybackslash}X
    >{\centering\arraybackslash}X
    }
    \toprule
    {\textbf{Baseline}} & 
    {\textbf{Set5}} &
    {\textbf{Set14}} & 
    {\textbf{BSD100}} & 
    {\textbf{Urban100}} &
    {\textbf{DIV2K}}
    \\
    %& PSNR  & PSNR  & PSNR & SSIM & PSNR & PSNR
    \midrule
    \textbf{ICF-SRSR~(LIIF)} & {36.46} & {32.39} & {31.18} & {29.74} & {34.52}  \\
    \textbf{ICF-SRSR~(EDSR)} & {37.01} & {32.86} & {31.54} & {30.39} & {35.19} \\
    \textbf{ICF-SRSR~(RDN)} & {37.03} & {32.87} & {31.56} & {30.42} & {35.18} \\
     \textbf{ICF-SRSR~(RCAN)} & \textbf{37.12} &  \textbf{32.92} &  \textbf{31.59} &  \textbf{30.50} &  \textbf{35.21}  \\
     
     
         
    \bottomrule
    \end{tabularx}
    
    \vspace{1mm}
    \caption{
        \textbf{Evaluation of our ICF-SRSR with different baselines by PSNR metric on scale $\times 2$.} 
        %
    }
    \label{tab:supp-baseline}
    \vspace{1mm}
\end{table*}

%%%%%%%%%%%%%%%%%%%%%%%%%%%%%%%%%%%%%%%%%%%%%%%%%%%%%% 
\section{Evaluation by SSIM}
We quantitatively show the results of our ICF-SRSR and EDSR~(LLR,LR) methods compared to other supervised and unsupervised methods trained on DIV2K~\cite{agustsson2017ntire} dataset and tested on the five standard benchmarks Set5~\cite{bevilacqua2012low}, Set14~\cite{zeyde2010single}, BSD100~\cite{martin2001database}, Urban100~\cite{huang2015single}, and Manga109~\cite{Manga109} by SSIM metric in Table~\ref{tab:supp-benchmark}.
%
According to the results, our method outperforms unsupervised method~\cite{huang2015single} on both scaling factors $\times2$ and $\times4$ and supervised method~\cite{chen2021learning} on scaling factor $\times2$ and is comparable with other methods.
%%%%%%%%%%%%%%%%%%%%%%%%%%%%%%%%%%%%%%%%%%%%%%%%%%%%%%%%%
\section{Ablation on baseline model}
We employ different models LIIF~\cite{chen2021learning}, EDSR~\cite{lim2017enhanced}, RDN~\cite{zhang2018residual}, and RCAN~\cite{zhang2018image} as the baseline of our ICF-SRSR framework.
%
In the case of EDSR, RDN, and RCAN, we develop the original network architecture to generate multi-scale images by applying a tail for each scaling factor $s$ and its inverse $\nicefrac{1}{s}$, individually.
%
In the case of LIIF, we leverage its continuous attribute to generate any scale of images by sub-sampling from the reconstructed continuous image.
%
Table~\ref{tab:supp-baseline} shows the results of our ICF-SRSR with different baselines.
%
We illustrate that our method is model-agnostic and can leverage different state-of-the-art~(SOTA) baseline models.
%
We note that our method can achieve better performance using advanced baselines except LIIF, which is not trained with continuous scales due to the limitation of the color loss $\mathcal{L}^{\text{Color}}$.
%
We select the model EDSR as our baseline due to its training time efficiency. 






%%%%%%%%%%%%%%%%%%%%%%%%%%%%%%%%%%%%%
\section{Ablation on the hyperparameter $\lambda_{\text{Color}}$.} 


We conduct an ablation study to investigate the importance of our color loss $\mathcal{L}^\text{Color}$ defined in Section~\textcolor{red}{3.3} by changing its weight $\lambda_{\text{Color}}$.
%
Specifically, We increase the weight from 0.1 to 10 and report the performance of our ICF-SRSR trained on the scale $\times2$ of test sets of both real-world dataset RealSR~\cite{cai2019toward} and synthetic datasets Set5~\cite{bevilacqua2012low} and DIV2K~\cite{agustsson2017ntire} validation in Table~\ref{tab:ab-loss}.
%
The results indicate that $\lambda_{\text{Color}}=0.2$ achieves the best performance on different datasets.
\section{Ablation study on YCBV}
\label{sec:ablation_ycbv}

In Tab.~\ref{tab:ablation_ycbv} we report the results of our ablation study on YCBV~\cite{ycbv}.
We choose the Large Marker object and train a single model on it for each modification we applied.
Each model is trained for 20 epochs on the standard training set.
For the computation of the Feature Matching Recall (FMR), we set the distance threshold $\tau_1=10$ voxels and the inlier ratio threshold $\tau_2=5$\%, to account for the different density of the scene point cloud in YCBV.
All the other settings and parameters are the same as those in our ablation study on LMO~\cite{lmo} in the main paper.

We can observe that some changes do not increment performance, but instead cause a slight drop, in particular when adapting the safety threshold to the object dimension (third row, $-0.4$) and when colour augmentation is applied (sixth row, $-$0.3).
These additions do not benefit this particular object, but are instead advantageous when averaging all the object in the dataset.

We can note that, as in the ablation study on the LMO dataset in the main paper, the most significant improvements in ADD-S AUC result from applying the safety threshold ($+$1.5), adding RGB information ($+$5.5), and using the Adam optimiser ($+$12.3).
\renewcommand{\arraystretch}{0.9}
\begin{table}%[t!]
\centering
\tabcolsep 3pt
\caption{
Ablation study on the Large Marker object of YCBV.
Performance are compared in terms of RRE [radiants] and RTE [cm] errors (the lower the better), and FMR and ADD-S AUC (shortened to ADD) scores (the higher the better).
$\Delta$ shows the improvement of each contribution in terms of ADD-S AUC with respect to the previous row.
}
\vspace{-3mm}
\resizebox{\columnwidth}{!}{%
\begin{tabular}{clrrrrr}
\toprule
& Improvements &
RRE{\color{black!50}{$\,\downarrow$}} &
RTE{\color{black!50}{$\,\downarrow$}} & 
FMR{\color{black!50}{$\,\uparrow$}} & 
ADD{\color{black!50}{$\,\uparrow$}} & 
$\Delta$ \\ 
\toprule
& Baseline & 2.0 & 4.6 & 0.00 & 77.2 & -- \\
\midrule
\multirow{2}{*}{\rotatebox{90}{Loss}} & $+$ $\tau_{NS} = 0.1 D_S$ & 2.0 & 4.2 & 0.00 & 78.7 & $+$1.5 \\
& $+$ $\tau_{NS} = 0.1 D_O$ & 2.0 & 4.3 & 0.00 & 78.3 & $-$0.4 \\
\midrule
\multirow{2}{*}{\rotatebox{90}{Arch.}} & $+$ Independent weights & 2.0 & 4.1 & 0.00 & 79.4 & $+$1.1 \\
& $+$ Add RGB information & 1.2 & 3.2 & 49.1 & 84.9 & $+$5.5 \\
\midrule
\multirow{2}{*}{\rotatebox{90}{Aug.}} & $+$ Color augmentation & 1.2 & 3.3 & 50.0 & 84.6 & $-$0.3 \\
& $+$ Random erasing & 1.2 & 3.1 & 53.4 & 85.2 & $+$0.6 \\
\midrule
\multirow{2}{*}{\rotatebox{90}{Optim.}} & $+$ SGD $\to$ Adam & 0.0 & 0.4 & 100 & 97.5 & $+$12.3 \\
& $+$ Adam $\to$ AdamW  & 0.0 & 0.4 & 100 & 97.5 & 0 \\
& $+$ Exp $\to$ Cosine & 0.0 & 0.4 & 100 & 97.4 & $-$0.1 \\\bottomrule
\end{tabular}}
\label{tab:ablation_ycbv}
\end{table}
\renewcommand{\arraystretch}{1}

\section{Additional ablation study on LMO}

We include an ablation study on the $t_\text{scale}$ hyperparameter, which is used to set the radius of the ball volume in which negative mining around a certain point is not allowed. We train on the Can object of LMO using the standard setting, and varying only $t_\text{scale}$. The results are shown in Tab.~\ref{tab:ablation_ycbv}.
We can observe that our choice of $t_\text{scale} = 0.1$ leads to the best result. When $t_\text{scale}$ is increased, many candidate points are forbidden to be used as negatives, therefore decreasing the final performance. On the other hand, a lower $t_\text{scale}$ implies negative pairs composed by points which are near in the 3D space. This reduces the performance, as similar points are forced to have different descriptors. Notably, the worst results is obtained when $t_\text{scale} = 0.1$, i.e. when no negative candidates are excluded.

\begin{table}
\tabcolsep 3pt
\caption{
Ablation study on the Can object of LMO. Performance is shown in terms of ADD-0.1 (the higher the better) in function of the hyperparameter $t_\text{scale}$.}
\centering
\resizebox{.9\columnwidth}{!}{
\begin{tabular}{c|ccccc}
    \toprule
    $t_\text{scale}$ & 0.0 & 0.01 & 0.05 & \textbf{0.1} & 0.5 \\
    ADD-0.1d & 66.55 & 91.80 & 93.79 & \textbf{93.95} & 81.28 \\
    \bottomrule
\end{tabular}
\label{tab:tscale}
}
\end{table}

%
%%%%%%%%%%%%%%%%%%%%%%%%%%%%%%%%%%%%%%%%%%%%%%%%%%%%%
\section{Noise-free results}
%
\renewcommand{\thefigure}{S3}
% Figure environment removed

% Figure environment removed
 
% Figure environment removed

\setcounter{figure}{2}
% Figure environment removed



In Section~\textcolor{red}{4.2} of our main manuscript, we note that the ground-truth images of Set5~\cite{bevilacqua2012low} and Set14~\cite{zeyde2010single} datasets are noisy while our SR images are noise-free.
%
We show the difference between our SR images and the noisy ground-truth images in Figure~\ref{fig:supp-noise}.
%
The results prove our claim and show that we can restore SR images without any noise.
%%%%%%%%%%%%%%%%%%%%%%%%%%%%%%%%%%%%%%%%%%%%%%%%%%%%%%%%

\section{Complicated down-sampling degradations}
%
As we show in Section~\textcolor{red}{4.3} of our main manuscript, the proposed method can learn from real-world datasets with unknown degradations~(real LR usually includes complicated degradations).
%
For example, we can train our model $f_\theta$ on images from RealSR-V3~\cite{cai2019toward} and DRealSR~\cite{wei2020component} datasets directly and achieve promising results. 
%
Furthermore, we train and test our method ICF-SRSR on a dataset with more complicated degradations generated by the Real-ESRGAN~\cite{wang2021realesrgan} down-sampling strategy.
%
We note that the generated LR images by the Real-ESRGAN~\cite{wang2021realesrgan} down-sampling model are synthesized by a sequence of classical degradations such as blur, resize, noise, JPEG compression, and artifacts to simulate more practical degradations.
%
Figure~\ref{fig:reb_degrade} demonstrates that our method ICF-SRSR can perform $\times 2$ SR faithfully even on images with mild noise and artifacts.
%
\renewcommand{\thefigure}{S4}
% Figure environment removed     

%
%\vspace{-9.2mm}
\section{Visualization of the generated images}
%\vspace{1mm}
In Figure~\ref{fig:supp_scales2} and \ref{fig:supp_scales4}, we visualize the generated down-sampled~(LLR) and up-sampled~(SR) images by our ICF-SRSR framework for different scaling factors $\times2$ and $\times4$, respectively on various benchmark datasets Set14~\cite{zeyde2010single}, BSD100~\cite{martin2001database}, and Urban100~\cite{huang2015single} and also real-world dataset RealSR-V3~\cite{cai2019toward}.
%
We further restore the down-sampled LR images given HR images for scaling factor $\times2$ of Canon and Nikon sets from the RealSR-V3~\cite{cai2019toward} dataset as shown in Figure~\ref{fig:reb_LRHR}.
%
%We further compare the generated and real LR images in Figure~\ref{fig:reb_LRHR}.
%
The comparison demonstrates that the generated down-sampled LR images by our self-supervised method ICF-SRSR look similar to the real LR images, validating the ability of our method to synthesize realistic LR-HR image pairs.
%
Such generated paired images LR-HR are useful to train other off-the-shelf supervised methods, as evident in Table~\textcolor{red}{6} of our main manuscript.
\renewcommand{\thefigure}{S5}
% Figure environment removed     

\renewcommand{\thefigure}{S6}
% Figure environment removed     

\renewcommand{\thefigure}{S7}
% Figure environment removed     

%%%%%%%%%%%%%%%%%%%%%%%%%%%%%%%%%%%%%%%%%%%%%%%%%%%%%%



%%%%%%%%%%%%%%%%%%%%%%%%%%%%%%%%%%%%%%%%%%%%%%%%%%%%%%
\section{Training on a single image}
\renewcommand{\thefigure}{S8}
% Figure environment removed



\renewcommand{\thefigure}{S9}
% Figure environment removed



In Section~\textcolor{red}{4.4} of our main manuscript, we show that our method ICF-SRSR can learn to restore SR images by training on a small dataset and even a single image as shown in Figure~\textcolor{red}{1}. 
%
We show more samples to illustrate the ability of our method to learn from only a single image.
%
Therefore, we train and evaluate our ICF-SRSR model on a single LR image from the test set of the RealSR-V3~\cite{cai2019toward} dataset captured by the Nikon camera for scaling factor $\times2$.
%
Our results in Figure~\ref{fig:supp_single} demonstrate that our method can restore an SR image by training the model on only the same image.
%
Furthermore, our result for the single-image case is not only on par with the multi-image case but also shows better performance for some samples in terms of PSNR metric and visual appearance.
%
This attribute makes our method more practical in real-world scenarios where there are not many sample images for training.
%
Moreover, we train and evaluate our self-supervised method ICF-SRSR on a single real-world smartphone photo and show the results in Figure~\ref{fig:supp_real_single}.

\end{appendices}

\end{document}