%!TEX root = ../../main.tex

\begin{table}[t]
	\centering
	\begin{tabular}{llcc}
		\toprule

		 & & \multicolumn{2}{c}{\textsc{Accuracy}} \\

		 & \textsc{Fine-tuning} & \textsc{ImageNet} & \textsc{iNaturalist} \\
		\midrule
		\multirow{2}{*}{\textsc{No Parts}}
			& \emph{only FC layer}
			& \pcent{63.28} \scriptsize{$(\pm \pcent{0.45})$}
			& \pcent{86.60} \scriptsize{$(\pm \pcent{0.42})$} \\

			& \emph{entire CNN}
			& \pcent{89.46} \scriptsize{$(\pm \pcent{0.88})$}
			& \pcent{90.54} \scriptsize{$(\pm \pcent{1.10})$} \\
		\midrule
		\multirow{2}{*}{\textsc{With Parts}}
		% \textsc{CS-parts}~\scriptsize{\cite{Korsch19_CSPARTS}}
			& \emph{only FC layer}
			& \pcent{71.82} \scriptsize{$(\pm \pcent{0.35})$}
			& \pcent{87.96} \scriptsize{$(\pm \pcent{0.38})$} \\
			& \emph{entire CNN}
			& \pcent{91.50} \scriptsize{$(\pm \pcent{0.61})$}
			& \pcent{93.13} \scriptsize{$(\pm \pcent{0.76})$} \\
		\bottomrule
	\end{tabular}
	\caption{Classification results on cropped images of the EU-Moths dataset. The results show the effects of the different fine-tuning strategies, the two pre-training datasets, and the usage of additional information in the form of parts.}
	\label{tab:classification_results:eumoths}
\end{table}
