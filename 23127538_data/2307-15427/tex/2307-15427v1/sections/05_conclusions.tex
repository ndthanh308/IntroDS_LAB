%!TEX root = ../main.tex

\section{Conclusions}
\label{sec:conclusions}

In this paper, we presented an automated detection and classification pipeline for \num{200} moth species.
We plan to deploy this pipeline in the visual monitoring system of the AMMOD project, the so-called moth scanner, which will help ecologists observe the population trends of the insects.
Since light sources easily attract moths, the moth scanner consists of an illuminated white surface and a camera that automatically captures images of insects resting on this surface.
We first localized the moths with a single-shot MultiBox detector (SSD) in the recorded images and then classified the resulting detections using a CNN classifier.
We also showed the effect of different training configurations on the final classification accuracy: the choice of fine-tuning strategy, the selection of the pre-training dataset, and the extension of the classification with an unsupervised part estimator.
% \todo{training configurations klingt komisch und passt nicht so richtig zu der Aufzaehlung am Ende, z.B. ist die fine-tuning-Strategie eine Trainingskonfiguration? evtl. die Wahl der fine-tuning-Strategie, also the choice of..., ebenso dann fuer die anderen: the selection of the pre-training dataset, the extension of the classifier with an unsupervised part estimator}
% \todo{wie ich deinen Kommentar verstanden habe, fandest du, dass die Formulierungen nicht gepasst haben (bzw fehlende Verben). Ich hab es so angepasst, wie du es vorgeschlagen hast. Oder fehlt deiner Meinung noch was?}
In our experiments, each part of the pipeline achieved promising results: a detection rate of up to \pcent{99.01} (mAP@50) and classification accuracy on images depicting a single insect of up to \pcent{93.13}.
Finally, we evaluated both parts of the pipeline together and improved the classification accuracy on original images captured by the moth scanner from~\pcent{79.62} to~\pcent{88.05} compared to a setup without a preceding moth detector.
