%!TEX root = ../main.tex

\begin{abstract}
Biodiversity monitoring is crucial for tracking and counteracting adverse trends in population fluctuations.
However, automatic recognition systems are rarely applied so far, and experts evaluate the generated data masses manually.
Especially the support of deep learning methods for visual monitoring is not yet established in biodiversity research, compared to other areas like advertising or entertainment.
In this paper, we present a deep learning pipeline for analyzing images captured by a moth scanner, an automated visual monitoring system of moth species developed within the AMMOD project.
We first localize individuals with a moth detector and afterward determine the species of detected insects with a classifier.
Our detector achieves up to \pcent{99.01} mean average precision and our classifier distinguishes \num{200} moth species with an accuracy of \pcent{93.13} on image cutouts depicting single insects.
Combining both in our pipeline improves the accuracy for species identification in images of the moth scanner from~\pcent{79.62} to~\pcent{88.05}.
% \redmark{150/150 max allowed words}
% \todo[inline]{``images captured by the moth scanner'' kommt zwei mal vor, ist eig eine Dopplung, aber ich finde, es schafft auch eine Verbindung oder einen Rahmen, und im letzten Satz schlägt man damit den Bogen zurück zu der Pipeline. Was sagst du?}
\end{abstract}

\begin{keywords}
Biodiversity Monitoring \and
Deep Learning \and
Convolutional Neural Networks \and
Insect Detection \and
Species Classification \and
Unsupervised Part Estimation %\and
\end{keywords}

\endinput
From the author's template:
This is a brief overview of the paper, which should be 70 to 150 words long and
include the most relevant points. This has to be a single paragraph.
