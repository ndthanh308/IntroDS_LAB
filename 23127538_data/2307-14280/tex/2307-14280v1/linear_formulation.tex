
\section{Non-algebraic \ac{NC} alternative}
\label{sec:optlp}

\ac{DiffNC} is not the first attempt at combining optimization techniques with \ac{NC}.
An \ac{LP} formulation of an \ac{NC} model was proposed in \cite{Bouillard2010}.
It converts the equations introduced in \cref{sec:dncintro} to linear constraints, under the assumption that curves are piecewise linear functions, either concave or convex. %
Flows are backtracked to derive constraints capturing, among others, their mutual impact.
Complexity grows exponentially when computing a flow's tight delay bound and a heuristic called \ac{ULP} was proposed.
The \ac{ULP} was shown to have only limited scalability~\cite{Bondorf2017a}, yet it is our only hope for a non-algebraic \ac{NC} competitor.
We complement the \ac{ULP} to include our $p_{f_i,j}$ variables and to optimize for multiple flows.
The resulting formulation is able to find configurations, yet preliminary evaluation already shows that it scales poorly.

The \ac{ULP} is based around two classes of variables.
Time variables $t_h \in \R^+$ represent departure or arrival time of bits of data of the flows at the different servers of the network.
Function variables $A_{f_i}^{s_k}(t_h) \in \R^+$ represent the departure and arrival processes of the data of flows at the different servers of the network, i.e., the arrival and departure functions introduced in \cref{sec:dncintro} as $A(t)$ and $A'(t)$.

Based on these variables, the \ac{ULP} translates arrival curves from \cref{eq:arrival_curve} as linear constraints:
\begin{equation} \label{eq:arrival_curve:constraint}
	A_{f_i}^{s_k}(t_{h+1}) - A_{f_i}^{s_k}(t_h) \leq \alpha_i(t_{h+1} - t_h), \forall s_k, f_i
\end{equation}
and similarly service curves from \cref{eq:service_curve} as:
\begin{equation}
	\sum_{f_i} \left( A_{f_i}^{s_k}(t_{h+1}) - A_{f_i}^{s_k}(t_h) \right) \geq \beta_k(t_{h+1} - t_h), \forall s_k
\end{equation}
Additional constraints representing, e.g., causality are also added.
We refer to \cite{Bouillard2010} for a full formulation.%

We extend here this formulation to take into account different paths for the flows.
For each flow $i$ and each potential path $j$, we define the variables $A_{f_{i,j}}^{s_k}(t_h)$ as the departure and arrival processes of the data of the virtual flows along the path $j$.
The variables $A_{f_{i,j}}^{s_k}(t_h)$ are constrained as in the original formulation from \cite{Bouillard2010} as if they were normal flows.
Following \cref{eq:vitual_arrival_curve}, the following constraints are added:
\begin{equation} \label{eq:lp_As_constr}
	A_{f_{i,j}}^{s_k}(t_h) \leq M \cdot p_{f_{i,j}}, \forall s_k, f_{i,j}, t_h
\end{equation}
with $M$ a large constant chosen such that $\alpha_i(t_{h+1} - t_h) \leq M, \forall t_{h+1}, t_h$ in the \ac{LP} formulation.
Using the big-M method, \cref{eq:lp_As_constr} achieves the same effect as \cref{eq:vitual_arrival_curve}: the $A_{f_{i,j}}^{s_k}(t_h)$ are constrained to 0 on the paths where $p_{f_{i,j}} = 0$ -- i.e., removing their impact on the delay calculation of the other flows -- and leaving them unconstrained when $p_{f_{i,j}} = 1$.

While this \ac{ULP} formulation of the optimal routing problem is attractive, it suffers from two important drawbacks: difficulty for expressing delay constraints and optimization goals, and poor scalability.
The first drawback of this approach is that some requirements regarding the optimization problem are not straightforward to translate into the \ac{ULP}.
This drawback mainly stems from the fact that the delay bound itself is calculated by maximizing an expression in the \ac{ULP}.
This leads to difficulty at implementing an objective function which would minimize average delay bounds.
Similarly, adding a constraint regarding a maximum delay requirement as in \cref{eq:delay_req} is not straightforward: the objective function maximizes the delay bound, but such a constraint would result in an underestimation of the delay bound in some cases.


Secondly, as noted in \cite{Bouillard2010,Bouillard2014HDR} and subsequently numerically illustrated in \cite{Bondorf2017a}, the \ac{ULP} is only tractable on relatively small networks due to its exponentially growing number of constraints.
To illustrate this point, we evaluated our modified \ac{ULP} including the $p_{f_i,j}$ variables and the constraints from \cref{eq:lp_As_constr} on a set of randomly generated networks.
Details about the networks are explained later in \cref{sec:dataset}.
For the objective function, we maximize the sum of delay bounds.
We extended the \ac{ULP} implementation from NCorg DNC v2.6.2~\cite{Bondorf2014} for our evaluation.
\Cref{fig:nclp_networksize_vs_solvetime} illustrates the time to find a solution with a time limit of 1 hour using IBM's CPLEX 20.1.0 on an Intel Xeon Gold 5120 at \SI{2.2}{\GHz}.

% Figure environment removed

As expected, the solve time grows exponentially, exceeding the one hour time limit even on small networks.
This result highlights why an alternative solution for optimizing networks is necessary for larger networks.
As a further motivating comparison, \cref{fig:nclp_networksize_vs_solvetime} also illustrates the optimization time on the same networks with our contributed approach: \ac{DiffNC} with Frank-Wolfe.%

