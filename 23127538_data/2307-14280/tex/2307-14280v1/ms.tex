
\documentclass[10pt,journal,letterpaper]{IEEEtran}
\IEEEoverridecommandlockouts

\usepackage[english]{babel}
\usepackage[utf8]{inputenc}
\usepackage[T1]{fontenc}

\usepackage{amssymb}
\usepackage{amsmath}
\usepackage{graphicx}
\graphicspath{{imgs/}{figures/}{plots/}}
\usepackage[scaled]{beramono}
\usepackage[numbers]{natbib}
\usepackage{paralist}
\usepackage{textcomp}

\usepackage[inline]{enumitem}
\usepackage{algorithm}
\usepackage[noend]{algpseudocode}


\usepackage{array}
\usepackage{subfig}
\usepackage{booktabs}
\usepackage{datetime}
\usepackage{caption}
\usepackage{etoolbox}
\usepackage[detect-all]{siunitx}
\usepackage{blkarray}

\usepackage{acro}
\DeclareAcronym{NC}{short=NC, long=Network Calculus}
\DeclareAcronym{DNC}{short=DNC, long=Deterministic Network Calculus}
\DeclareAcronym{TFA}{short=TFA, long=Total Flow Analysis}
\DeclareAcronym{SFA}{short=SFA, long=Separate Flow Analysis}
\DeclareAcronym{PMOO}{short=PMOO, long=Pay Multiplexing Only Once}
\DeclareAcronym{TSN}{short=TSN, long=Time Sensitive Networking}
\DeclareAcronym{DiffNC}{short=DiffNC, long=Differential Network Calculus}
\DeclareAcronym{AD}{short=AD, long=automatic differentiation}
\DeclareAcronym{SQP}{short=SQP, long=sequential quadratic programming}
\DeclareAcronym{SLSQP}{short=SLSQP, long=sequential least squares quadratic programming}
\DeclareAcronym{IPOPT}{short=IPOPT, long=Interior Point OPTimizer}
\DeclareAcronym{MMA}{short=MMA, long=method of moving asymptotes}
\DeclareAcronym{CCSA}{short=CCSA, long=conservative convex separable approximation}
\DeclareAcronym{BONMIN}{short=BONMIN, long=Basic Open-source Nonlinear Mixed INteger programming}
\DeclareAcronym{GNN}{short=GNN, long=graph neural network}
\DeclareAcronym{NN}{short=NN, long=neural network}
\DeclareAcronym{LP}{short=LP, long=linear program}
\DeclareAcronym{NLP}{short=NLP, long=nonlinear programming}
\DeclareAcronym{ILP}{short=ILP, long=integer linear programming}
\DeclareAcronym{MINLP}{short=MINLP, long=mixed-integer nonlinear programming}
\DeclareAcronym{ML}{short=ML, long=machine learning}
\DeclareAcronym{CAS}{short=CAS, long=computer algebra system}
\DeclareAcronym{TDMA}{short=TDMA, long=time-division multiple access}
\DeclareAcronym{FIFO}{short=FIFO, long=first-in first-out}
\DeclareAcronym{ULP}{short=ULP, long=unique linear program}
\DeclareAcronym{AFDX}{short=AFDX, long=Avionics Full-Duplex Switched Ethernet}
\DeclareAcronym{VM}{short=VM, long=virtual machine}

\usepackage{hyphenat}
\hyphenation{sto-chas-tic re-cur-sion tight-est}

\newcommand\R{\mathbb{R}} %
\newcommand{\B}[1]{\mathbf{#1}} %
\newcommand{\TP}[1]{10\textsuperscript{#1}}

\usepackage{mathtools}
\DeclarePairedDelimiter\ceil{\lceil}{\rceil}
\DeclarePairedDelimiter\floor{\lfloor}{\rfloor}

\newtheorem{theorem}{Theorem}
\newtheorem{definition}{Definition}
\newtheorem{corollary}{Corollary}
\newtheorem{lemma}{Lemma}

\author{Fabien Geyer and Steffen Bondorf%
\IEEEcompsocitemizethanks{%
	\IEEEcompsocthanksitem The preliminary version of this article is published in \cite{GeyerBondorf_INFOCOM2022} [DOI: 10.1109/INFOCOM48880.2022.9796777]. \textit{(Corresponding author: Fabien Geyer.)}
	\IEEEcompsocthanksitem F. Geyer is with Airbus Central Research \& Technology, Munich, Germany (email: fabien.geyer@airbus.com).
	\IEEEcompsocthanksitem S. Bondorf is with the Faculty of Computer Science, Ruhr University Bochum, Germany (email: steffen.bondorf@rub.de).
}}

\usepackage{nameref}
\usepackage[capitalize,noabbrev]{cleveref}



\begin{document}

\title{Differentiable Programming \& \acl{NC}:\\
Configuration Synthesis under Delay Constraints}

\maketitle
\acresetall

\begin{abstract}

The Fast Reciprocal Square Root Algorithm is a well-established approximation technique consisting of two stages: first, a coarse approximation is obtained by manipulating the bit pattern of the floating point argument using integer instructions, and second, the coarse result is refined through one or more steps, traditionally using Newtonian iteration but alternatively using improved expressions with carefully chosen numerical constants found by other authors. The algorithm was widely used before microprocessors carried built-in hardware support for computing reciprocal square roots. At the time of writing, however, there is in general no hardware acceleration for computing other fixed fractional powers. This paper generalises the algorithm to cater to all rational powers, and to support any polynomial degree(s) in the refinement step(s), and under the assumption of unlimited floating point precision provides a procedure which automatically constructs provably optimal constants in all of these cases. It is also shown that, under certain assumptions, the use of monic refinement polynomials yields results which are much better placed with respect to the cost/accuracy tradeoff than those obtained using general polynomials. Further extensions are also analysed, and several new best approximations are given.

\end{abstract}


\IEEEdisplaynontitleabstractindextext
\IEEEpeerreviewmaketitle

\acresetall
\section{Introduction}
Current quantum hardware is unable to carry out universal quantum computations due to the buildup of errors that occur during the computation. 
The magnitude of the individual error is currently above the value that the Threshold Theorem requires in order to kick-start quantum error correction and fault-tolerant quantum computation~\cite[Section 10.6]{nielsen_chuang_2010}. 
Although the experimentally achieved fidelity rates are promising and the error bounds are inching closer to the required threshold, we will have to work for the foreseeable future with quantum hardware with errors that build-up during the computation.  This implies that we can only do a limited number of steps before the output of the computation has become completely uncorrelated with the intended one.

For fault-tolerant quantum computing, we repeat four steps: 
1) We apply a number of single and two-qubit quantum gates, in parallel whenever possible; 
2) We perform a syndrome measurement on a subset of the qubits; 
3) We perform fast classical computations to determine which errors have occurred and how to correct them; 
and, 4) We apply correction terms based on the classical computations.
We then repeat these four steps with a next sequence of gates. 
These four steps are essential to fault-tolerant quantum computing. 


The starting point of this work is to use the four steps outlined above, not to carry out error correction and fault-tolerant computation, but to enhance short, constant-depth, {\em uncorrected} quantum circuits that perform single qubit gates and {\em nearest-neighbor} two qubit gates. 
Since in the long run we will have to implement error-correction and fault-tolerant computation anyhow, and this is done by such a four-step process, why not make other use of this architecture? Moreover, on some of the quantum hardware platforms, these operations are already in place.
Embracing this idea we naturally arrive at the question: what is the computational power of \textit{low-depth} quantum-classical circuits organized as in the four steps outlined above? 
We thus investigate circuits that execute a small, ideally constant, number of stages, where at each stage we may apply, in parallel, single qubit gates and {\em nearest-neighbor} two qubit gates, followed by measurements, followed by low-depth classical computations of which the outcome can control quantum gates in later stages. 
It is not clear, at first, whether such circuits, especially with constant depth, can do anything remotely useful. 
But we will see that this is indeed the case: many quantum computations can be done by such circuits in constant depth. 
By parallelizing quantum computations in this way, we improve the overall computational capabilities of these circuits, as we do not incur errors on qubits that are idle, simply because qubits are not idle for a very long time. 
Furthermore, reducing the depth of quantum circuits, at the cost of increasing width, allows the circuit to be run faster even if errors occur.

The first usage of such a four-step layout, not to do error correction, but to perform computations, can be found in the paradigm of measurement-based quantum computing~\cite{gottesman1999demonstrating,raussendorf2001one,jozsa2006introduction,clark2007generalised}: 
A universal form of quantum computing where a quantum state is prepared and operations are performed by measuring qubits in different bases, depending on previous measurements and intermediate measurements.

\citeauthor{PhamSvore2013} were the first to formalize the four-step protocol for performing computations~\cite{PhamSvore2013}. They included specific hardware topologies by considering two-dimensional graphs for imposing constraints on qubit interactions. In their model, they develop circuits for particularly useful multi-qubit gates, including specifying costs in the width, number of qubits, depth, number of concurrent time steps, size, and total number of non-Identity operations.
As a result, they find an algorithm that factors integers in polylogarithmic depth.
\citeauthor{Browne:2011} showed that the main tool in the work by \citeauthor{PhamSvore2013}, the fan-out gate, can also be replaced by additional log-depth classical computations in the measurement-based quantum computing setting~\cite{Browne:2011}.

More recently, \citeauthor{Cirac:2021} introduced a scheme to implement unitary operations involving quantum circuits combined with Local Operations and Classical Communication ($\mathsf{LOCC}$) channels: $\mathsf{LOCC}$-assisted quantum circuits~\cite{Cirac:2021}. Similarly to the four-step scheme we just described, they allow for a short depth circuit to be run on the qubits, followed by one round of $\mathsf{LOCC}$, in which ancilla qubits are measured and local unitaries are applied based on the measurement outcomes. They show that in this model any 1D transitionally invariant matrix-product state (MPS) with fixed bond dimension is in the same phase of matter as the trivial state. Similar ideas can be found in~\cite{TVV_NonAbelianTopologicalOrder_2022, tantivasadakarn2021long}.

In this work, we introduce a new model, called \textit{Local Alternating Quantum-Classical Computations} ($\LAQCC$). In this model we alternate between running quantum circuits (constrained by locality), ending in the measurement of a subset of qubits, and fast classical computations based on the measurement results. The outcome of the classical computations are then used to control future quantum circuits. We allow for flexibility in this model, by giving different constraints to the power of both the quantum circuits and the classical circuits as well as the number of alternations between them. 
Most attention will be given to $\LAQCC$ containing quantum circuits of constant depth, classical circuits of logarithmic depth and at most a constant number of alternations between them. 
Any circuit constructed in this model is considered to be of constant depth. 
We restrict ourselves to logarithmic depth classical computations, as this is the first natural and non-trivial extension beyond constant-depth classical computations. 
Constant-depth classical computations do however also have an equivalent constant-depth quantum implementation.

The definition of $\LAQCC$ sharpens the original definition of \citeauthor{PhamSvore2013} by adding constraints to the intermediate classical computations. This allows us to bound the power of $\LAQCC$ from above. 

The main result of \citeauthor{Cirac:2021}, that 1D translational invariant MPS with fixed bond dimension can be prepared by $\mathsf{LOCC}$-assisted circuits, relies on local symmetries of the MPS. These symmetries allow them to prepare local states (on a constant number of qubits) and glue them together by doing one round of the appropriate entangling measurement and corrections, after which they run a round of local unitaries to get the desired result. This general scheme for preparing states that exhibit an MPS description with the appropriate local symmetries requires only geometrically local unitaries and one round of measurement and corrections an therefore is accessible in $\LAQCC$. Studying different local symmetries, known as Symmetry Protected Topological (SPT) phases of matter, to find measurement-based constant depth circuits for states is a broad ongoing field of research~\cite{TVV_NonAbelianTopologicalOrder_2022, tantivasadakarn2021long, smith2023deterministic}. 
All these schemes have a $\LAQCC$ implementation.

%$\LAQCC$-circuits also exist for general schemes of preparing local states, based on the local tensors, and gluing them together using one round of entangled measurement and corrections, based on the local symmetry. 
%The main result of \citeauthor{Cirac:2021}, that 1D translational invariant MPS with fixed bond dimension can be prepared by $\mathsf{LOCC}$-assisted circuits, relies heavily on local symmetries of the MPS and as a result also has an equivalent $\LAQCC$ implementation. 
%The corrections applied after the measurement round are local unitaries depending on the local symmetries of the MPS. 

 

%This general scheme of preparing local states, based on the local tensors, and gluing it together by doing one round of entangled measurement and corrections, based on the local symmetry, is accessible in $\LAQCC$.
Note however that \citeauthor{Cirac:2021} also suggest a circuit for the $W$-state.
This circuit uses sequentially and dependent measurement-based corrections of the ancilla qubits. 
These dependent measurements translate to sequential alternations between the quantum and classical circuits and therefore increase the total depth to linear depth, exceeding the constant-depth constraints imposed by $\LAQCC$-circuits. 

We study the power of the $\LAQCC$ model with respect to state preparation, showing that even with only constant quantum-depth and logarithmic classical depth it remains possible to prepare states with long-range entanglement.
Another surprising result is that it is unlikely that $\LAQCC$ circuits are classically simulatable. We show that any instantaneous quantum polynomial-time (IQP) circuit~\cite{Bremner2010,Shepherd2009} has an $\LAQCC$ implementation.
Classical simulation of IQP circuits implies the collapse of the polynomial hierarchy to the third level, which is not believed to be true~\cite{Bremner2017}. Therefore, we expect that $\LAQCC$ circuits are unlikely to be classically simulatable. We bound the power of $\LAQCC$ by showing that it is contained in $\QNC^1$, the class of polynomial-size, log-depth circuits.

Next, we also study the power that intermediate classical calculations can add to quantum computations, by considering a new model that alternates between polynomially many polynomial-depth quantum circuits and unbounded classical computations
We study this model by doing a complexity theoretical analysis, where we draw inspiration from the notions of complexity given by \citeauthor{RosenthalYuen:2022}, \citeauthor{MetgerYuen:2023}, and \citeauthor{Aaronson:2004}.
All three complexity notions are based on the notion of state preparation, instead of more traditional definition of complexity such as the decidability of a computational problem. 
The first two consider classes based on sequences of quantum states preparable by a polynomial-sized quantum circuit, where the circuits are uniformly generated by a computational class, for instance, the class $\mathsf{PSPACE}$, which results in the complexity class $\mathsf{StatePSPACE}$~\cite{RosenthalYuen:2022,MetgerYuen:2023}.
The third notion considers a relative complexity, where the complexity is measured between two given states, and is measured by the number of gates, from a given gate-set, required to transform one state in another state~\cite{Aaronson:2004}. 
For our definition of state preparation complexity, we drop the uniformity constraint from~\cite{RosenthalYuen:2022,MetgerYuen:2023} and define a class as $\mathsf{StateX}$, which refers to states preparable by circuits of type $\mathsf{X}$. 
As an example, if $\mathsf{X} = \QNC^0$, this results in the class $\mathsf{StateQNC^0}$, which is the set of states preparable from the $\ket{0}^n$ state by poly-size constant-depth circuits. 
This notion is similar to the relative complexity from~\cite{Aaronson:2004}, where one state is the  $\ket{0}^n$ state and instead of counting the number of gates we consider the set of states preparable by a fixed number of gates. Using this notion of complexity we show that any state preparable by an $\LAQCC^*$ circuit is also preparable by a $\mathsf{PostQPoly}$ circuit, the class of circuits of polynomial depth with an additional post-selection gate. 

All Clifford circuits have a constant-depth $\LAQCC$ implementation, implying that any stabilizer state can be implemented by a constant-depth $\LAQCC$ circuit, see Section~\ref{sec:clifford_circuits} for a proof of this statement. 
Efficient circuits for stabilizer states have been known already through measurement-based quantum computing. Therefore this paper focuses on the preparation of non-stabilizer states, and as a surprising result we find novel constant-depth protocols for four very natural classes of non-stabilizer states.
Despite the extensive research into these four classes of non-stabilizer states and the many applications of them, no efficient constant- or low-depth state preparation protocols are known yet. We specifically consider these four classes as they are all often used as initial states in other algorithms.

The first state is a uniform superposition over an arbitrary number of states. 
This state finds applications in many quantum algorithms, as they often start with a uniform superposition over multiple states. 
This superposition is often achieved by applying Hadamard gates to every qubit due to its simplicity to prepare. 
Yet, the analysis of many algorithms, such as Shor's algorithm~\cite{Shor:1997}, would benefit from a different initial superposition. 
The circuit to prepare the uniform superposition over an arbitrary number of states uses an exact version of Grover search as a subroutine, that turns a probabilistic circuit, with a known constant probability of success, into a deterministic circuit. 
We use the circuit for preparing a uniform superposition over an arbitrary number of states as a subroutine in the next two quantum state preparation protocols. 

The second state is the $W$-state, the uniform superposition over all computational basis states of Hamming-weight~$1$, a natural long-ranged entangled state that displays a fundamentally nonequivalent type of entanglement from the Greenberger–Horne–Zeilinger state~\cite{WState:2000}, for which $\LAQCC$-type constant-depth circuits were previously known~\cite{PhamSvore2013, Cirac:2021}. 
The $W$-state is often used as benchmark for new quantum hardware~\cite{Haffner2005,Neeley2010,GarciaPerez:2021}. 
A novel way to prepare the $W$-state therefore gives a new way to benchmark different quantum devices with each other. 
A circuit for preparing the $W$-state was given in~\cite{Cirac:2021}, but this implementation requires sequentially alternating measurements followed by local unitaries, which in the $\LAQCC$ model is not considered to be of constant depth. 
We improve this protocol by giving an $\LAQCC$ implementation of the $W$-state, based on a compress-uncompress method that links the one-hot and binary encoding of integers.

The third state considered is the Dicke state, a generalization of the $W$-state, a superposition over all computational basis states with Hamming-weight $k$~\cite{Dicke:1954}. 
Dicke states have relevance in various practical settings.
For instance, for quantum game theory~\cite{zdemir2007}, quantum storage~\cite{Bacon_Compress:2006,Plesch:2010}, quantum error correction~\cite{ouyang2014permutation}, quantum metrology~\cite{toth2012multipartite}, and quantum networking~\cite{prevedel2009experimental}. 
Dicke states have been used as a starting state for variational optimization algorithms, most notably Quantum Alternating Operator Ansatz (QAOA)~\cite{Hadfield2019}, to find solutions to problems such as Maximum k-vertex Cover~\cite{Brandhofer2022,cook2020quantum}.
The ground states of physical Hamiltonians describing one-dimensional chains tend to show a resemblance to Dicke states such as states resulting from the Bethe ansatz, making them an ideal starting state when investigating the ground state behavior of these Hamiltonians~\cite{TDL_BetheAnsatzDerivation:2010,B_ExcitedStateQuantumPhaseTransitions:2013,DickeTransitions:2021}. 
For instance, the algorithm by \citeauthor{van2021preparing}, who give an algorithm to prepare the Bethe ansatz eigenstates of the spin-1/2 XXZ spin chain, starts by first preparing a Dicke state~\cite{van2021preparing}. 
A Dicke-state preparation protocol based on the compress-uncompress methodology used in the $W$-state furthermore finds applications in entanglement distillation, where the entanglement of a large state is concentrated on only a few qubits. 
Efficient deterministic circuits for preparing Dicke states have been proposed by \citeauthor{bartschi2019deterministic}~\cite{bartschi2019deterministic, bartschi2022deterministic_short_depth}. 
They provide a quantum circuit of depth $\mathO(k \log(\frac{n}{k}))$, allowing arbitrary connectivity, to prepare a Dicke state, which they conjecture to be optimal when $k$ is constant. 
In this work, we provide a constant-depth $\LAQCC$ circuit below their conjectured bound already for constant $k$. 
However, this does not directly disprove their conjecture, as we allow for intermediate measurements and classical computations. 
More significantly, we even construct constant-depth $\LAQCC$ circuits for $k = \mathO(\sqrt{n})$ greatly improving their bound.
This construction extends the compress-uncompress method for the $W$-state combined with additional subroutines. 

We continue with a log-depth state preparation protocol for the Dicke-state for arbitrary $k$. 
This protocol implements an efficient transformation between the factoradic number representation and the combinatorial number representation of a positive integer. 
The combinatorial number representation relates directly to the Dicke state. 
The provided efficient transformation between number representation systems might be of independent interest. 

We conclude by modifying our protocol for preparing a Dicke-state to a protocol that prepares quantum many-body scar states in constant-depth. 
These states have low entanglement and longer coherence times than states with similar energy density.
These characteristics make many-body scar states interesting to analyze and relevant within physics.
Many-body scar states appear for instance in the AKLT model~\cite{AKLT:1987,MRBAR:2018,MRB:2018} and different spin models~\cite{SI:2019,MOBFR:2020}.
Known methods for preparing these states have polynomial-depth~\cite{Gustafson:2023}, whereas our circuit has constant depth. 

% We conclude by studying the power that intermediate classical calculations can add to quantum computations. 
% In this study, we define a new model that relaxes constant-depth quantum circuits to polynomial depth quantum circuits, log-depth classical calculations to unbounded classical computations and a constant number of alternations to a polynomial number of alternations. 
% We call this model $\LAQCC^*$. 
% We study this model by doing a complexity theoretical analysis, where we draw inspiration from the notions of complexity given by \citeauthor{RosenthalYuen:2022}, \citeauthor{MetgerYuen:2023}, and \citeauthor{Aaronson:2004}.
% All three complexity notions are based on the notion of state preparation, instead of more traditional definition of complexity such as the decidability of a computational problem. 
% The first two consider classes based on sequences of quantum states preparable by a polynomial-sized quantum circuit, where the circuits are uniformly generated by a computational class, for instance, the class $\mathsf{PSPACE}$, which results in the complexity class $\mathsf{StatePSPACE}$~\cite{RosenthalYuen:2022,MetgerYuen:2023}.
% The third notion considers a relative complexity, where the complexity is measured between two given states, and is measured by the number of gates, from a given gate-set, required to transform one state in another state~\cite{Aaronson:2004}. 
% For our definition of state preparation complexity, we drop the uniformity constraint from~\cite{RosenthalYuen:2022,MetgerYuen:2023} and define a class as $\mathsf{StateX}$, which refers to states preparable by circuits of type $\mathsf{X}$. 
% As an example, if $\mathsf{X} = \QNC^0$, this results in the class $\mathsf{StateQNC^0}$, which is the set of states preparable from the $\ket{0}^n$ state by poly-size constant-depth circuits. 
% This notion is similar to the relative complexity from~\cite{Aaronson:2004}, where one state is the  $\ket{0}^n$ state and instead of counting the number of gates we consider the set of states preparable by a fixed number of gates. Using this notion of complexity we show that any state preparable by an $\LAQCC^*$ circuit is also preparable by a $\mathsf{PostQPoly}$ circuit, the class of circuits of polynomial depth with an additional post-selection gate. 

\paragraph{Summary of results}
\begin{itemize}
    \item We give a new definition of a computational model that captures the power of the four step process: applying a constant number of layers of one- and two-qubit gates; performing a syndrome measurement; perform a fast classical computation determining corrections; apply corrections. We call this model \emph{Local Alternating Quantum Classical Computations}, or $\LAQCC$ for short. In this model we bound the allowed quantum operations, intermediate classical calculations, and number of rounds separately. In Section~\ref{sec:LAQCC_model} we define this model and give a list of operations based on results from literature contained in this computational model. In some of these operations we explicitly use that we allow for multiple, but at most constant, rounds  of corrections.
    \item  We show show that there exist $\LAQCC$ circuits that can not be weakly simulated in Section~\ref{sec:IQP_in_LAQCC}. We further show that for every $\LAQCC$ circuit there exists a $\QNC^1$ circuit simulating it perfectly, in Section~\ref{sec:LAQCC_in_QNC1}.
    \item We introduce a new type computational complexity for preparing states and show that the extension of $\LAQCC$ where we allow a polynomial number of rounds and unbounded classical computation, is contained in $\mathsf{PostQPoly}$, the class of polynomial circuits with post-selection, in Section~\ref{sec:Complexity results}.
    \item We show a protocol to prepare the uniform superposition state of size $q$ in $\LAQCC$ using $\mathO(\ceil{\log_2(q)}^2)$ qubits in Section~\ref{sec:superposition_modulo_q}. 
    \item We show a protocol to prepare the $W_n$ state in $\LAQCC$ using $\mathO(n\log(n))$ qubits in Section~\ref{sec:W_state_in_LAQCC}.
    \item We show two ways of preparing the Dicke-$(n,k)$ state. The first method is in $\LAQCC$, works up to $k = \mathO(\sqrt{n})$, uses $\mathO(n^2\log(n))$ qubits, and is found in Section~\ref{sec:dicke:small_k}. The second method is in $\LAQCC\text{-}\mathsf{LOG}$ (an extension of $\LAQCC$ allowing for logarithmic number of alterations instead of constant), works for any $k$, uses $\mathO(\text{poly}(n))$ qubits, and is found in Section~\ref{sec:Dicke_in_LAQCC_LOG}. 
    \item We extend on our $\LAQCC$ method of generating Dicke-$(n,k)$ states for $k = \mathO(\sqrt{n})$ and show a protocol to generate many-body scar states for a particular Hamiltonian in $\LAQCC$ (Section~\ref{sec:many_body_scar}). 
\end{itemize}
Summarized in a table, we provide the following state generation protocols:
\begin{table}[htb]
\centering
\begin{tabular}{l|l|l|l}
\textbf{State description} & \textbf{Width} & \textbf{Depth} & \textbf{Implementation}\\
\hline 
Uniform superposition mod $q$: $\frac{1}{\sqrt{q}} \sum_{i = 0}^{q-1}\ket{i}$ & $\mathO(\ceil{\log^2 q})$ & $\mathO(1)$ & Section~\ref{sec:superposition_modulo_q}\\

$W$-state: $\frac{1}{\sqrt{n}}\sum_{i = 0}^{n-1}\ket{e_i}$ & $\mathO(n \log n)$ & $\mathO(1)$ & Section~\ref{sec:W_state_in_LAQCC}\\

Dicke-$(n,k)$, $k = \mathO(\sqrt{n})$: $\binom{n}{k}^{-1/2}\sum_{x \in \{0,1\}^n: |x| = k} \ket{x}$ &  $\mathO(n^2\log n)$ & $\mathO(1)$ 
&Section~\ref{sec:dicke:small_k}\\

Dicke-$(n,k)$: $\binom{n}{k}^{-1/2}\sum_{x \in \{0,1\}^n: |x| = k} \ket{x}$ & $\mathO(\text{poly}(n))$ & $\mathO(\log n)$ &Section~\ref{sec:Dicke_in_LAQCC_LOG}\\

QMBS: $\ket{S_k} = \frac{1}{k! \sqrt{\mathcal N(n,k)}}(Q^\dagger)^k \ket{\Omega}$ &  $\mathO(n^2\log n)$ & $\mathO(1)$  &  Section~\ref{sec:many_body_scar}
\end{tabular}
\caption{Summary of state preparation protocols given in this paper.}
\label{tab:sate_prep}
\end{table}
In the entry for the quantum many-body scar state $Q$ denotes the raising operator and $\mathcal N(n,k)=\binom{n-k-1}{k}$. 
Section~\ref{sec:many_body_scar} will provide more details on the variables and the implementation. 

\paragraph{Organization of the paper}
\noindent We first introduce relevant preliminaries in Section~\ref{sec:preliminaries}. 
In Section~\ref{sec:LAQCC_model} we formally define the class of Local Alternating Quantum-Classical Computations ($\LAQCC$). We also show that any Clifford circuit can be implemented in constant depth $\LAQCC$ (a result based on a result from measurement-based quantum computing~\cite{jozsa2006introduction}). 
This result allows us to give many useful multi-qubit gates and routines in Section~\ref{sec:gates_created_in_LAQCC}. 
Beyond that we show that constant depth $\LAQCC$ circuits are contained in $\QNC^1$ and that any $\mathsf{IQP}$ circuit has an $\LAQCC$ implementation.
We conclude this section with an analysis of a more powerful instantiation of $\LAQCC$ and show an inclusion with respect to the class $\mathsf{PostQPoly}$, which is the class of circuits of polynomial depth with one additional post-selection gate. 
In Section~\ref{sec:state_prep_in_LAQCC} we give $\LAQCC$ circuit implementations for preparing the uniform superposition over an arbitrary number of states, the $W$-state and the Dicke state up to $k = \mathO(\sqrt{n})$. We furthermore give a log-depth circuit implementation for preparing the Dicke state for any $k$. We conclude by showing a $\LAQCC$ circuit for generating many body scar states of a particular type of Hamiltonian.


\section{Related Work}
%\subsection{Cost Volume based Deep Stereo Matching}
%Stereo matching is a typical problem that has been studied for decades and a well-known four-step pipeline \cite{scharstein2002taxonomy} has been established, where cost volume construction is an indispensable step. Current state-of-the-art stereo matching methods are all cost volume based methods and they can be categorized into two types. Typically, a cost volume is a 4D tensor of height, width, disparity, and features. The first category just uses a full correlation to generate a single-feature cost volume. Such methods are usually efficient but lose much information because of the decimation of feature channels. Many previous work, including Dispnet \cite{dispnet}, MADNet \cite{madnet}, IResNet \cite{iresnet} and AANet \cite{aanet}, belong to this category. The second category usually uses concatenation \cite{gcnet} or group-wise correlation \cite{gwcnet} to generate a multi-feature 4D cost volume. Such a method can achieve better performance while requiring higher computational complexity and memory consumption. Actually, a majority of the top-performing networks in public leaderboards belong to this category, such as GANet \cite{ganet}, CSPN \cite{cspn} and ACFNet \cite{acfnet}. These methods generally employ multiple 3D convolution layers to constantly regularize the 4D cost volume and then apply softmax over the disparity dimension to produce a discrete disparity probability distribution. The final predicted disparity is obtained by softly weighting indices according to their probability, which is also called soft argmin in GCNet \cite{gcnet}. However, soft argmin leaves the output susceptible to multi-modal disparity probability distributions. ACFNet \cite{acfnet} observes this problem and proposes to directly supervise the cost volume with unimodal ground truth distributions. In contrast, we define an uncertainty estimation to quantify the degree to which the cost volume tends to be multi-modal distribution, higher implies the higher possibility of estimation error.

\subsection{Multi-scale Cost Volume based Stereo Matching}
Cost volume construction is an indispensable step in the well-known four-step pipeline for stereo matching \cite{scharstein2002taxonomy, pamisurvey1, pamisurvey2}. Typically, current state-of-the-art stereo matching methods can be categorized into two types of cost volume-based methods, where the cost volume is a 4D tensor of height, width, disparity, and features. The first category usually uses the single-feature 3D cost volume generated by full correlation, which is efficient while losing much information due to the decimation of feature channels. Many real-time methods, such as Dispnet \cite{dispnet}, MADNet \cite{madnet, madnet_pami} and AANet \cite{aanet}, belongs to the category. Moreover, two-stage refinement \cite{mcvmfc} and pyramidal towers \cite{madnet} are commonly applied in the single-feature cost volume based network to construct multi-scale cost volume. The second category usually uses the multi-feature 4D cost volume generated by concatenation \cite{gcnet} or group-wise correlation \cite{gwcnet}, which can achieve better performance with higher computational complexity and memory consumption. Most top-performing networks, including GANet \cite{ganet}, CSPN \cite{cspn} and ACFNet \cite{acfnet} belong to this category. 
% In these methods, the 4D cost volume is constantly regularized by multiple 3D convolution layers and then a discrete disparity probability distribution can be produced by softmax. Next, the final predicted disparity can be obtained by softly weighting indices according to their probability \cite{gcnet}. However, such output is susceptible to multimodal disparity probability distributions and ACFNet \cite{acfnet} gives a solution by directly supervising the cost volume with unimodal ground truth distributions to alleviate this problem. 
Recently, to alleviate the high computational complexity and memory consumption when employing multi-feature 4D cost volumes, \cite{cvpmvsnet, cascade, uscnet} propose to use cascade cost volume representation in multi-view stereo. These methods usually first predict an initial disparity at the coarsest resolution of the image and then gradually refine the disparity by narrowing down the disparity search space. More closely related to our approach is Casstereo \cite{cascade}, which first extended such representation to stereo matching. It selected to uniform sample a pre-defined range to generate the next stage’s disparity search range. Instead, we employ pixel-level uncertainty estimation to adaptively adjust the next stage disparity searching range and generate pseudo-labels for subsequent domain adaptation. Our method also shares similarities with UCSNet \cite{uscnet}, which constructs uncertainty-aware cost volume in multi-view stereo while it doesn’t employ uncertainty estimation to generate pseudo-labels.

%\subsection{Multi-scale Cost Volume based Deep Stereo Matching} 
% \subsection{Multi-scale Cost Volume based Stereo Matching} 
%Multi-scale cost volume firstly was applied in the single-feature cost volume based network with the form of two-stage refinement \cite{mcvmfc} and pyramidal towers \cite{madnet}. Recently, cascade cost volume representation \cite{cvpmvsnet, cascade, uscnet} was proposed in multi-view stereo to alleviate the high computational complexity and memory consumption when employing multi-feature 4D cost volumes. These methods generally predict an initial disparity at the coarsest resolution of the image. Then, they will narrow down the disparity search space and gradually refine the disparity. More closely related to our approach is Casstereo \cite{cascade}, which first extended such representation to stereo matching. It selected to uniform sample a pre-defined range to generate the next stage’s disparity search range. Instead, we employ uncertainty estimation to adaptively adjust the next stage pixel-level disparity searching range and push the next stage's cost volume to be predominantly unimodal.

% The single-feature cost volume based network with the form of two-stage refinement \cite{mcvmfc} and pyramidal towers \cite{madnet} first employ multi-scale cost volume for stereo matching. Recently, to alleviate the high computational complexity and memory consumption when employing multi-feature 4D cost volumes, \cite{cvpmvsnet, cascade, uscnet} propose to use cascade cost volume representation in multi-view stereo, which generally predict an initial disparity at the coarsest resolution of the image. Then, the disparity search space is narrowed down and the disparity is gradually refined. More closely related to our approach is Casstereo \cite{cascade}, which first extended such representation to stereo matching. It selected to uniform sample a pre-defined range to generate the next stage’s disparity search range. Instead, we employ uncertainty estimation to adaptively adjust the next stage pixel-level disparity searching range and push the next stage's cost volume to be predominantly unimodal.

% Figure environment removed

\subsection{Robust Stereo Matching} 
There exist three categories of generalization definitions for robust stereo matching. 1) Cross-domain Generalization: the network’s ability to perform well on unseen scenes (cannot see the image pairs of the target domain in advance). Towards this end, Jia et al \cite{sungeneralizaiton} propose to incorporate scene geometry priors into an end-to-end network. Zhang et al \cite{dsmnet} introduce a domain normalization and a trainable non-local graph-based filter to construct a domain-invariant stereo matching network. 2) Adapt Generalization: the network’s ability to adapt pre-trained models to the new domain with unlabeled target data. Previous work usually pre-trains the models on synthetic data and then adapts it to new target domains with Graph Laplacian regularization \cite{zoom}, non-adversarial progressive color transfer \cite{adastereo}, and Knowledge Reverse Distillation \cite{aohnet}. More closely related to our approach are \cite{aohnet, unsuperviseddomainadaptation} in stereo matching and Monoresmatch \cite{monoresmatch} in monocular depth estimation, which also proposes to generate a pseudo-label for domain adaptation. However, these methods all select to employ classical stereo matching methods \cite{sgm} alongside with confidence estimators, e.g., left-right consistency check to generate pseudo-labels. That is all these methods need an independent method to generate corresponding pseudo-labels. Instead, the proposed method is an end-to-end network that can generate the predicted disparity map, corresponding uncertainty map and pseudo-labels jointly, which is a more simple, yet efficient way. 
% Instead, our proposed method can employ pixel-level and area-level uncertainty estimation to self-distill the predicted disparity maps of our pre-training model and generate sparse while reliable pseudo-labels to align the domain gap, which is a more simple, yet efficient way. 
3) Joint Generalization: the network’s ability to perform well on a variety of datasets with the same model parameters. MCV-MFC \cite{mcvmfc} introduces a two-stage finetuning scheme to achieve a good trade-off between generalization and fitting capability on multiple datasets. However, it doesn’t touch the inner difference between diverse datasets, e.g, the unbalanced disparity distribution. To further address this problem, we propose a cascade cost volume to adaptively the next stage disparity searching space, where the pixel-level uncertainty estimation is at the core.

% \subsection{Monocular Depth Estimation}
% Monocular depth estimation aims to estimate depth values from a single image, instead of stereo images or multiple frames in a video. This problem is ill-posed because of the ambiguity of object sizes. However, humans could estimate the depth from a single image with prior knowledge of the scenes. Recently, learning based methods were explored to learn depth values by supervised or unsupervised learning. Eigen et al. first employed Convolutional Neural Networks (CNN) to predict depth in a coarse-to-fine manner and further improved its performance by multi-task learning. Liu et al. presented deep convolutional neural fields model by combining deep model with continuous CRF. Li et al. [22] refined deep CNN outputs with a hierarchical CRF. Multi-scale continuous CRF was formulated into a deep sequential network by Xu et al. [45] to refine depth estimation. Unsupervised methods tried to train monocular depth estimation with stereo
% image pairs or image sequences and test on single images. Garg et al. [9] used novel image view synthesis loss to train a depth estimation network in an unsupervised way. Godard et al. [11] introduced left-right consistency regularization to improve the performance of view synthesis loss. Recently, some work also propose to use the stereo matching network as a proxy to learn depth from synthetic data or directly employ traditional stereo matching methods to distill proxies labels from the target domain, which proves the feasibility of distilling stereo matching networks to learn monocular depth estimation.




\section{Deterministic Network Calculus}
\label{sec:dncintro}

\acf{DNC} is built around two concepts \cite{LeBoudec2001}:
\begin{enumerate*}[label=\textit{\alph*)}] 
	\item the network of deterministic queueing locations crossed by constrained data flows, and
	\item resource modeling with cumulative functions in interval time.
\end{enumerate*}
Based on a fully specified model, a traditional \ac{DNC} analysis derives an upper bound on an analyzed flow's end-to-end delay.

We are interested in networks with point-to-point connections between devices, such as IEEE Ethernet and its extensions \ac{TSN} and \ac{AFDX}.
A model of such a network usually consists of the devices, e.g., switches, that are connected by undirected links.
I.e., an undirected graph $\mathcal{G}_{\text{device}} = (\mathcal{D}, \mathcal{L})$ is formed from devices $d\in\mathcal{D}$ and links $l\in\mathcal{L}$.
Most relevant for the \ac{NC} queueing analysis is the behavior at the devices' output ports.
Given the point-to-point connection, we therefore split the undirected links into directed ones such that there is one queueing location per directed link.
For convenience, we create the edge-to-vertex dual of this graph such that these queueing locations are represented by vertices.
The behavior at queueing locations can be defined by a wide range of scheduling policies, ranging from single \ac{FIFO} queues, over round robin schedulers (e.g., WRR, DRR) to a combination priority queues, shaping, time-triggered scheduling etc. as, for example, defined by \ac{TSN}.
For schedulers with static configurations to seperate service provision, the respective vertex is split into the defined number of priority levels or round robin class etc.
We call these vertices the servers and the resulting graph the server graph $\mathcal{G} = (\mathcal{S}, \mathcal{E})$ where each server $s\in\mathcal{S}$ forwards queued data.
The guaranteed service is expressed by non-negative, non-decreasing functions
$\mathcal{F}^{+}_{0}\!=\!\left\{ f:\mathbb{R}^{+}\rightarrow\mathbb{R}^{+}\,|\,f(0)\!=\!0,\;\forall s\le t\,:\,f(t)\!\geq\!f(s)\right\}$
called service curves.
\begin{definition}[Service Curve]
	If a server receives a data input $A\in\mathcal{F}^{+}_{0}$ and produces an output $A'\in\mathcal{F}^{+}_{0}$,
	then it is said to offer service curve $\beta \in \mathcal{F}_0$ iff
	\begin{equation}
	\label{eq:service_curve}
	\forall t : A'(t) \geq \inf_{0 \leq d \leq t} \{ A(t - d) + \beta(d) \}.
	\end{equation}
\end{definition}
The \ac{NC} literature provides a rich set of results to derive the service curves for different schedulers' service classes, e.g., \ac{TSN} \cite{Zhao2018,Zhao2020}.

Data flows cross the graph~$\mathcal{G}$ from a source server to a given number of destination servers, both in~$\mathcal{S}$.
The unicast or multicast flow path/s are subgraphs of $\mathcal{G}$ and a restriction on the data sent by a flow is assumed to be known at its source.
\begin{definition}[Arrival Curve]
	\label{def:Arrival-Curve}
	Given a flow $f$ described by $A\in\mathcal{F}^{+}_{0}$, 
	a function $\alpha\in\mathcal{F}_{0}$ is an arrival curve for $f$ iff
	\begin{equation}
	\label{eq:arrival_curve}
	\forall\,0\leq d\le t\,:\,A(t)-A(t-d)\leq\alpha(d).
	\end{equation}
\end{definition}

A traditional \ac{NC} analysis takes as input a fully specified queueing network model, i.e., a server graph with its service curves as well as flows, their paths and arrival curves.
It aims at deriving the end-to-end delay bound of a specific flow of interest (foi).
To do so, it needs to derive worst-case bounds on the mutual impact of flows at shared servers -- the queueing aspects not quantified by the model itself.
Our work is based on traditional (min,plus)-algebraic \ac{NC} analyses that assume the absence of cyclic dependencies between flows as well as the arbitrary multiplexing assumption.
These analyses derive a (min,plus)-algebraic term that bounds the end-to-end delay, consisting of the following operations:
\begin{definition}[Operations]%
	\label{def:MinPlusOperations}
	Given $\beta_1,\beta_2,\alpha_1,\alpha_2$, we can
	\begin{eqnarray}
	\text{aggregate:} &  & \hspace{-6.5mm}\left(\alpha_1+\alpha_2\right)\left(d\right) = \alpha_1\left(d\right)+\alpha_2\left(d\right)\!,\\
	\text{concatenate:} &  & \hspace{-6.5mm}\left(\beta_1\otimes \beta_2\right)(d) = {\displaystyle \inf_{0\leq u\leq d}}\left\{ \beta_1(d-u)+\beta_2(u)\right\}\!,\\
	\hspace{-5.5mm}\text{output bound:} &  & \hspace{-6.5mm}\left(\alpha_1\oslash \beta_1\right)(d) = \sup_{u\geq0}\left\{ \alpha_1(d+u)-\beta_1(u)\right\}\!,\\
	\text{left-over:} &  & \hspace{-6.5mm}\left(\beta_1\ominus \alpha_1\right)(d) = \sup_{0\leq u \leq d} \left\{\beta_1(u) - \alpha_1(u)\right\}\!, \\
	\text{delay bound:} &  & \hspace{-6.5mm} h(\alpha_1 ,\beta_1) = \inf\left\{ d\geq0|\!\left(\alpha_1\oslash\beta_1\right)(-d)\leq0\right\}\!\hspace{4mm}
	\end{eqnarray}
	flows and servers, respectively, in a worst-case manner.
\end{definition}

In this article, we will apply the \ac{SFA} \cite{LeBoudec2001} as the set of rules to derive the (min,plus)-algebraic, delay-bounding term from the server graph. %


\section{Differential Network Calculus}
\label{sec:diffnc}

The steps towards computing a delay bound as described in \Cref{sec:dncintro} cater to an analysis that requires a fully specified model.
I.e., in case there are design alternatives, each of these needs to be fully specified and analyzed independently.
A related topic are \ac{NC} analyses that explore alternative orders of (min,plus)-operations internally.
There, exhaustive enumeration may be feasible \cite{Bondorf2017a} but is often prohibitive \cite{Bondorf2017c}.
For either case, it was shown to be possible to design heuristics that vastly increase efficiency at little cost in terms of loss of delay bound accuracy \cite{Geyer2019b,2023-GSB-1}.
We propose \acl{DiffNC} (\acs{DiffNC}) that allows applying gradient-based \ac{NLP} to efficiently find a design alternative in a design space, subject to \ac{NC} delay bounds and at very low cost loss of accuracy.

\subsection{Generalized, parameterized \ac{NC} model}
\label{sec:formulation}

We generalize the server graph model of \ac{NC} to a more comprehensive, parameterized one where newly added parameters define design space alternatives.
We simply allow each variable in the model (e.g., those defining an arrival or service curve) to remain open and each part of the \ac{NC} model (e.g., entire curves) to be accompanied by an open parameter.
The added parameters can, e.g., be binary to decide whether the parameterized part of the model is to be considered;
continuous parameters can be added for weighing parameterized parts.
Combinations of parameters can also be restricted to express certain mutually exclusive design alternatives.
Without further restrictions on the parameters or their combinations, finding a parameter setting may become a \ac{MINLP}.
In this article, we exemplarily consider two instances of the generalized, parameterized \ac{NC} model.

\subsubsection{Alternative flow paths}
\label{sec:path_assignment}

Let $\mathcal{G} = (\mathcal{S}, \mathcal{E})$ be the server graph of the analyzed communication network and let $\mathcal{F}$ be the set of flows crossing $\mathcal{G}$.
We adopt here a path flow model, where each flow $f_i\in\mathcal{F}$ is associated with multiple alternative paths $\mathcal{P}_{f_i}$.
We amend each potential path $j \in \mathcal{P}_{f_i}$, or rather the data sent over $j$ by $f_i$, with a binary variable $p_{f_{i,j}}$ such that \cref{eq:arrival_curve} becomes:
\begin{equation} \label{eq:vitual_arrival_curve}
	\forall\,0\leq d\le t\,:\, A_{f_{i,j}}(t) - A_{f_{i,j}}(t-d) \leq \alpha_{f_i}(d) \cdot p_{f_{i,j}}
\end{equation}

To express that only a single path can be taken by a flow, we add the following constraint to our model:
\begin{equation} \label{eq:p_ij_sum}
	\sum_{j \in \mathcal{P}_{f_i}} p_{f_{i,j}} = 1, \forall f_i \in \mathcal{F}
\end{equation}

We call the alternatives along the different potential paths \emph{virtual flows}.
Virtual flow arrival curves would then be set to $0$ on the non-optimal paths, and remain constrained by $\alpha_{f_i}$ on the optimal paths.
See \cref{fig:illustration_virtual_flow} for an illustration of the model.

% Figure environment removed



Due to our formulation, servers may be overloaded, leading to invalid network configurations.
Hence the following constraint is required for each server $s$ in the server graph:
\begin{equation}
	\sum_{f \in \mathcal{F}_s} \mathit{rate}_f \leq \mathit{rate}_s
\end{equation}
with $\mathcal{F}_s$ the set of flows traversing server $s$.


As expected, we essentially defined a \ac{MINLP}, a combinatorial optimization problem with a number of potential solutions growing in $\mathcal{O}(|\mathcal{F}|^{|\mathcal{P}|})$.


\subsubsection{Flow priority assignment}

We can use the virtual flow concept to define the search space for (network-wide) priority assignment.
Namely, a flow $f_i$ is extended to a set of virtual flows, one for each priority class.
For each virtual flow of $f_i$ with its potential path $j$ and potential priority class $k$, we define $p_{f_{i,j,k}}$ as a binary variable representing the choice of path $j$ and priority $k$ for flow $f_i$.
The arrival curve of each virtual flow is then amended by $p_{f_{i,j,k}}$, analog to \cref{eq:vitual_arrival_curve}. 
Again, the sum of $p_{f_{i,j,k}}$ variable must be equal to $1$ to enforce that only one virtual flow is selected, i.e., only one path and priority class are globally defined per flow.
\Cref{fig:illustration_virtual_priority} illustrates this model.

% Figure environment removed


\subsection{Generalized \ac{NC} analysis by constrained \ac{NLP}}
\label{sec:constr_nlp}

Traditional \ac{NC} analyses such as \ac{SFA} would not be able to capture the virtual flow model.
Their backtracking of flow dependencies can succeed by ignoring the newly introduced binary variables, 
i.e, by including all alternatives of a virtual flow in the resulting (min,plus)-algebraic \ac{NC} term.
Analysis results are then valid, yet, overly pessimistic.
For an example of this practice, see \cite{2016-BG-1} where multicast flows are converted into a set of partially overlapping unicast flows.
If we extended the resulting (min,plus)-algebraic \ac{NC} term with the binary variables, then any unchanged traditional analyses would not be able to analyze this term anymore.
To solve this problem, we propose to use nonlinear optimization as analysis method.

\subsubsection{Closed-form expressions}

First, we need to derive closed-form expressions of the \ac{NC} operations presented in \cref{def:MinPlusOperations}.
We restrict our models to the curves shapes most commonly used in practice for industrial networks:
the rate-latency service curve $\beta_{R,L}$ and the token-bucket arrival curve $\gamma_{r,B}$:
\begin{eqnarray}
	\beta_{R,L}(t) & = & R [t - L]^+, \forall t \geq 0 \\
	\gamma_{r,B}(t) & = & B + r \cdot t, \forall t \geq 0
\end{eqnarray}
with $R$, $L$, $r$, and $B$ in $\R^+$, and $[x]^+ = x$ if $x \geq 0$ and 0 otherwise.

Applying \ac{NC}'s (min,plus)-algebraic operations to the above curve shapes, the following closed-form expression can be derived:
\begin{lemma}[Closed-form expression of \ac{NC} operations]
	\label{thm:nc_ops}
	With the assumption of using rate-latency service curves and token-bucket arrival curves, the \ac{NC} operations listed in \cref{def:MinPlusOperations} have the following closed-form solutions:
	\begin{eqnarray}
		\text{aggregation:} & \gamma_{r_1,B_1} + \gamma_{r_2,B_2} = \gamma_{r_1+r_2,B_1+B_2} \\
		\text{concatenation:} & \beta_{R_1,L_1} \otimes \beta_{R_2,L_2} = \beta_{\min(R_1, R_2),L_1 + L_2} \\
		\text{output bounding:} & \gamma_{r,B} \oslash \beta_{R,L} = \gamma_{r,B+r \cdot L} \\
		\text{left-over:} & \beta_{R,L} \ominus \gamma_{r,B} = \beta_{R - r, (B + R \cdot L) / (R - r)} \\
		\text{delay bound:} & h(\gamma_{r,B}, \beta_{R,L}) = B / R + L
	\end{eqnarray}
	under the previously mentioned stability condition that $r < R$.
\end{lemma}

Note that we restrict the curve shapes for brevity, our approach is not limited to these curve shapes.
Using different shapes requires to provide the closed-form expressions as in \cref{thm:nc_ops}.
This can be relatively simple, e.g., for concave arrival curves and convex service curves.

\subsubsection{Constrained \ac{NLP} Modeling}

Next, we show here how to model the network design problem as a differentiable \ac{NLP} of the following form:
\begin{align}
\min_{x \in \R^n}      & \quad f(x) \label{eq:general_nonlin_obj} \\
\text{s.t.} & \quad  g_l \leq g(x) \leq g_u \label{eq:general_nonlin_constr}
\end{align}
with $f()$ and $g()$ differentiable functions w.r.t. $x$, and $g_l$ and $g_u$ the upper and lower bounds for $g()$.

To make this problem solvable in polynomial time, we apply a commonly used technique known as relaxation, namely: the $p_{f_{i,j}}$ binary variables are relaxed as continuous variables on the interval $[0, 1]$.
Following \cref{thm:nc_diff}, the end-to-end delay bound expression of a virtual flow is then differentiable w.r.t. the $p_{f_{i,j}}$ variables.
This relaxation technique transforms the \ac{MINLP} into a continuous \ac{NLP}, enabling the use of \ac{NLP} methods based on gradient information.

Using on the previous formulations, we define the following constrained nonlinear optimization problem that minimizes the average delay bound:
\begin{align} \label{eq:nonlin_obj}
	\min_{p_{f_{i,j}}, \forall f_i \in \mathcal{F}, j \in \mathcal{P}_{f_i}} & \quad \frac{1}{|\mathcal{F}|} \sum_{i, j} \textit{delay bound}(f_{i,j}) \cdot p_{f_{i,j}} \\
	\text{s.t.} & \quad 0 \leq p_{f_{i,j}} \leq 1, \forall f_i \in \mathcal{F}, j \in \mathcal{P}_{f_i} \\
	& \quad \sum_{j \in \mathcal{P}_{f_i}} p_{f_{i,j}} = 1, \forall f_i \in \mathcal{F} \\
	& \quad \sum_{i \in T(k)} r_i \cdot p_{f_{i,j}} \leq R_k, \forall k \in \mathcal{S} \label{eq:cstr_server_overload}
\end{align}
with $T(k)$ the set of virtual flows traversing server $k$ with service curve $\beta_{R_k,L_k}$, and $\textit{delay bound}(f_{i,j})$ the end-to-end delay bound of the virtual flow $f_{i,j}$ computed with any of the classical algebraic \ac{NC} analyses (e.g., \ac{SFA}).
Note, that we thus also generalized the analysis to consider the delay bounds of multiple flows $f_{i,j}$ simultaneously
whereas the classical analyses can only analyze a single flow of interest in isolation.
Due to the operations in \cref{thm:nc_ops}, \cref{eq:nonlin_obj} is a non-convex objective function.


% Figure environment removed


\subsubsection{Extended Analysis Capabilities}

We already showed the ability to consider multiple flows' delay bounds simultaneously.
With our formulation, we also enable a wider range of constraints and objective functions.
Constraints can be added such that a maximum delay requirement is satisfied for a given flow, saving us a subsequent check against the requirement:
\begin{equation} \label{eq:delay_req}
	\sum_{j \in \mathcal{P}_{f_i}} \textit{delay bound}(f_{i,j}) \cdot p_{f_{i,j}} \leq \textit{requirement}
\end{equation}

This formulation is able to express complex objectives, enabling finer control over the type of solution which is required.
A popular mathematical framework for describing hard or soft requirements on network performance (such as delay or bandwidth) is the concept of utility-based network optimization introduced in \cite{Kelly1998}.
The objective function can be formulated with nonlinear utility functions $U_i$ for the delay bounds:
\begin{equation}
	\min_{p_{f_{i,j}}, \forall i, j} \sum_{i} U_i \left(\sum_j \textit{delay bound}(f_{i,j}) \cdot p_{f_{i,j}} \right)
\end{equation}
with $U_i$ a differentiable utility function mapping the delay bonds to a utility value in the interval $[0, 1]$.

Additionally, aspects such as the tail of the delay bound distribution can be minimized by defining the objective function:
\begin{equation}
	\min_{p_{f_{i,j}}, \forall i, j} \max_{i} \left(\sum_j \textit{delay bound}(f_{i,j}) \cdot p_{f_{i,j}} \right)
\end{equation}

The optimization formulation can be applied to any curve parameter, including the service curve parameters.
This means that the optimization formulation can also be defined w.r.t. scheduler characteristics.


\subsection{Differentiation of expressions for gradient-based \ac{NLP}}
\label{sec:diff_delay_term}

\ac{NLP} techniques based on gradient information -- such as Newton's method -- are known to usually outperform other \ac{NLP} optimization techniques.
Therefore, we show here that the delay bounding terms as well as the constraints are differentiable and confirm later in \cref{sec:numerical_evaluation} that this is key to a high performance analysis.

From \cref{thm:nc_ops}, the following theorem is derived:
\begin{theorem}[Differentiability of delay expression]
	\label{thm:nc_diff}
	With the assumption of using rate-latency service curves and token-bucket arrival curves, a \ac{NC} end-to-end delay bound is differentiable w.r.t. the curves parameters.
\end{theorem}

\textit{Proof:} Using the closed-form (min,plus) operations from \cref{thm:nc_ops}, all \ac{NC} operations use the following basic operators: addition, multiplication, division and min.
For the $\min$ operator, we use the following partial derivates for $x \neq y$:
\begin{align*}
	\frac{\partial \min(x, y)}{\partial x} = \begin{cases}
		1 \text{, if } x < y \\
		0 \text{, if } x > y
	\end{cases} & & \frac{\partial \min(x, y)}{\partial y} = \begin{cases}
	0 \text{, if } x < y \\
	1 \text{, if } x > y
	\end{cases}
\end{align*}
All of the applied operators are then differentiable, proving \cref{thm:nc_diff}.
\hfill $\square$

Partial derivates for the min operator used in the above proof are easily implemented using the Heaviside step function.

Following the previous theorems, \crefrange{eq:nonlin_obj}{eq:cstr_server_overload} are differentiable w.r.t the relaxed $p_{f_{i,j}}$ variables.


\section{Automatic differentiation and optimization} %
\label{sec:diffnc_ad}

The previous theorems build the mathematical foundations of \ac{DiffNC}.
Connecting them to the conceptual \ac{DiffNC} proceeding is straight-forward.
\Cref{fig:system_overview} illustrates the steps.
We detail here how to put \ac{DiffNC} into practice in order to efficiently compute partial derivatives of the end-to-end delay bounds w.r.t. the curves parameters.
We also present a preliminary numerical evaluation of the performances of our toolchain.


\subsection{Software architecture}

While computer-assisted symbolic differentiation could be used for deriving closed-form expressions of the gradient, our initial numerical evaluations with SymPy \cite{Meurer2017} showed that this method had difficulties scaling to networks with 100+ flows.

To overcome this scalability issues, we selected \ac{AD}.
It is a family of techniques based on the calculus' chain rule for efficiently and accurately evaluating derivatives of numeric functions expressed as computer programs.
This technique has gained a lot of popularity recently due its wide use in computing packages used for machine learning \cite{Baydin2018}.

While the first version of \ac{DiffNC} from \cite{GeyerBondorf_INFOCOM2022} was based on CasADi \cite{Andersson2019}, we implemented our own \acf{AD} tool in Go for this work, specialized for (min,plus) operations and delay bound calculations.
As shown later in \cref{sec:eval:toolchain,sec:eval:execution_time}, our implementation enables us better scalability on large networks and to parallelize various parts of the code.
This stems from the fact that most available \ac{AD} tools are targeting relatively small computation graphs (i.e. series of mathematical operations described as a directed graph) with arithmetically intense operations such as large matrix multiplications.
This specialization makes these tools poorly scalable for \ac{NC} operations, as millions of (min,plus) operations are required to compute delay bounds in our use-cases.
Our tool also directly uses the (min,plus) operations, enabling a better scalability compared to a conversion to basic mathematical operations.

As network analysis, we choose \ac{SFA} and execute its backtracking to derive a (min,plus)-algebraic \ac{NC} term.
Then we extend the term with the $p_{f_{i,j}}$ binary variables (see \cref{sec:formulation}) and convert it to closed form expressions in (plus,times) algebra according to \cref{thm:nc_ops}.
Combined with nonlinear optimization methods using gradients -- as detailed later in \cref{sec:diffnc_implementation} -- our generalized network models can efficiently be optimized.

Additionally, since most optimization methods require multiple evaluations of the objective function, our tool also allows us to run the \ac{NC} network analysis a single time and generate a so-called computation graph.
This graph translates the delay expressions (i.e., the objective function) to a combination of basic mathematical operations (addition, multiplication, etc.) and (min,plus) operations.
These saved operations can then be evaluated multiple times in our so-called \emph{\ac{DiffNC} \ac{VM}}, without requiring to run the \ac{NC}-specific parts again.

A network to-be-optimized with alternative flow paths is used as input of our framework.
Based on the end-to-end delay bounds calculations, the computation graph of the objective function and its gradient, and the constraint functions and its gradients, are then generated and compiled for our \ac{DiffNC} \ac{VM}.
The compiled formulas are then used as input to a nonlinear optimizer supporting the generalized \ac{NLP} presented in \cref{eq:general_nonlin_obj,eq:general_nonlin_constr}.


\subsection{The Frank-Wolfe algorithm}
\label{sec:fw}

We detail here the nonlinear optimization part using gradient-based constrained optimization methods with open-source implementations.

We selected the Frank-Wolfe algorithm \cite{Frank1956,Braun2022} -- also known as conditional gradient method -- as our main solution for solving the \ac{NLP} described in \cref{sec:diffnc}.
To optimize a \ac{NLP} with objective function $f$, the Frank-Wolfe algorithm can be summarized as the following loop.
Given a solution $x_k$ at iteration $k$:
\begin{itemize}
	\item \emph{Step 1:} Find the solution $s_k$ to the linearized version of the \ac{NLP} using the gradient information (i.e., $s_k^T \triangledown f(x_k)$),
	\item \emph{Step 2:} Update $x_{k+1} \gets x_k + \delta (s_k - x_k)$.
\end{itemize}
The choice of $\delta$ determines the so-called step size of the Frank-Wolfe algorithm.

While this algorithm was designed to solve convex \ac{NLP}, it was shown to also perform well on non-convex problems in practice by choosing $\delta = 1 / \sqrt{k+1}$ \cite{Braun2022}.
As we show later, it achieves good optimality at a low computational cost.

Several extensions of this algorithm have been proposed in the literature.
We explored some of them and found that its variant with momentum \cite{Braun2022} achieves good results in our problem setting.
Momentum is a technique to build inertia in a direction in the search space in order to overcome the oscillations of noisy gradients.

We implemented the standard Frank-Wolfe algorithm to be our main optimization algorithm in the following evaluations.
\Cref{sec:eval:frankwolfe} provides a peak into the performance of the Frank-Wolfe algorithm with momentum.


\subsection{Implementation details of our toolchain}
\label{sec:eval:toolchain}

We describe and numerically evaluate in this section key elements of our toolchain to enable fast evaluations.
The numerical evaluations were performed on the datasets described later in \cref{sec:dataset}.

\subsubsection{Parallelization}
\label{sec:eval:parallelization}

We evaluate in this section the impact of parallelization of the operations on the \ac{DiffNC} \ac{VM}.
For our implementation, we use different strategies for parallelization.
First, we parallelize the different delay bound computations both during the preparation of the (min,plus) terms and during the execution in the \ac{DiffNC} \ac{VM}.
These computations are independent of each other and our objective function from \cref{eq:nonlin_obj} requires the computation of the delay bounds of all flows in the network.
Secondly, we also parallelized part of the computations during the preparation of the (min,plus) terms of the \ac{SFA} network analysis.

The impact of the parallelization is highlighted in \cref{fig:execution_time_parallelization_afdx}, where the delay bound calculations of all flows from the \ac{AFDX} network were performed.
We notice that, as we increase the number of cores used for the computations, the total execution time of the analysis is reduced.
Overall, a gain of more than one order of magnitude was possible thanks to parallelization.

The benefit of precomputing the (min,plus) computations is also illustrated in \cref{fig:execution_time_parallelization_afdx}.
Running the compiled analysis can be performed 50 times faster than running it without precomputation.

% Figure environment removed




\subsubsection{Impact of VM vs. native code}

Finally, we also benchmark our \ac{DiffNC} \ac{VM} against compiled code, i.e., against the same \ac{NC} operations compiled to Intel assembly.
Results on the \ac{AFDX} topology are presented in \cref{fig:execution_time_comparison_netcalvm_cjit_afdx}.

When comparing the execution time of the (min,plus) operations, the \ac{DiffNC} \ac{VM} is slower than compiled code.
Yet, the cost of compiling the (min,plus) operations to assembly makes it overall slower than using the \ac{DiffNC} \ac{VM}, as shown by the total execution time.

% Figure environment removed


\section{Non-algebraic \ac{NC} alternative}
\label{sec:optlp}

\ac{DiffNC} is not the first attempt at combining optimization techniques with \ac{NC}.
An \ac{LP} formulation of an \ac{NC} model was proposed in \cite{Bouillard2010}.
It converts the equations introduced in \cref{sec:dncintro} to linear constraints, under the assumption that curves are piecewise linear functions, either concave or convex. %
Flows are backtracked to derive constraints capturing, among others, their mutual impact.
Complexity grows exponentially when computing a flow's tight delay bound and a heuristic called \ac{ULP} was proposed.
The \ac{ULP} was shown to have only limited scalability~\cite{Bondorf2017a}, yet it is our only hope for a non-algebraic \ac{NC} competitor.
We complement the \ac{ULP} to include our $p_{f_i,j}$ variables and to optimize for multiple flows.
The resulting formulation is able to find configurations, yet preliminary evaluation already shows that it scales poorly.

The \ac{ULP} is based around two classes of variables.
Time variables $t_h \in \R^+$ represent departure or arrival time of bits of data of the flows at the different servers of the network.
Function variables $A_{f_i}^{s_k}(t_h) \in \R^+$ represent the departure and arrival processes of the data of flows at the different servers of the network, i.e., the arrival and departure functions introduced in \cref{sec:dncintro} as $A(t)$ and $A'(t)$.

Based on these variables, the \ac{ULP} translates arrival curves from \cref{eq:arrival_curve} as linear constraints:
\begin{equation} \label{eq:arrival_curve:constraint}
	A_{f_i}^{s_k}(t_{h+1}) - A_{f_i}^{s_k}(t_h) \leq \alpha_i(t_{h+1} - t_h), \forall s_k, f_i
\end{equation}
and similarly service curves from \cref{eq:service_curve} as:
\begin{equation}
	\sum_{f_i} \left( A_{f_i}^{s_k}(t_{h+1}) - A_{f_i}^{s_k}(t_h) \right) \geq \beta_k(t_{h+1} - t_h), \forall s_k
\end{equation}
Additional constraints representing, e.g., causality are also added.
We refer to \cite{Bouillard2010} for a full formulation.%

We extend here this formulation to take into account different paths for the flows.
For each flow $i$ and each potential path $j$, we define the variables $A_{f_{i,j}}^{s_k}(t_h)$ as the departure and arrival processes of the data of the virtual flows along the path $j$.
The variables $A_{f_{i,j}}^{s_k}(t_h)$ are constrained as in the original formulation from \cite{Bouillard2010} as if they were normal flows.
Following \cref{eq:vitual_arrival_curve}, the following constraints are added:
\begin{equation} \label{eq:lp_As_constr}
	A_{f_{i,j}}^{s_k}(t_h) \leq M \cdot p_{f_{i,j}}, \forall s_k, f_{i,j}, t_h
\end{equation}
with $M$ a large constant chosen such that $\alpha_i(t_{h+1} - t_h) \leq M, \forall t_{h+1}, t_h$ in the \ac{LP} formulation.
Using the big-M method, \cref{eq:lp_As_constr} achieves the same effect as \cref{eq:vitual_arrival_curve}: the $A_{f_{i,j}}^{s_k}(t_h)$ are constrained to 0 on the paths where $p_{f_{i,j}} = 0$ -- i.e., removing their impact on the delay calculation of the other flows -- and leaving them unconstrained when $p_{f_{i,j}} = 1$.

While this \ac{ULP} formulation of the optimal routing problem is attractive, it suffers from two important drawbacks: difficulty for expressing delay constraints and optimization goals, and poor scalability.
The first drawback of this approach is that some requirements regarding the optimization problem are not straightforward to translate into the \ac{ULP}.
This drawback mainly stems from the fact that the delay bound itself is calculated by maximizing an expression in the \ac{ULP}.
This leads to difficulty at implementing an objective function which would minimize average delay bounds.
Similarly, adding a constraint regarding a maximum delay requirement as in \cref{eq:delay_req} is not straightforward: the objective function maximizes the delay bound, but such a constraint would result in an underestimation of the delay bound in some cases.


Secondly, as noted in \cite{Bouillard2010,Bouillard2014HDR} and subsequently numerically illustrated in \cite{Bondorf2017a}, the \ac{ULP} is only tractable on relatively small networks due to its exponentially growing number of constraints.
To illustrate this point, we evaluated our modified \ac{ULP} including the $p_{f_i,j}$ variables and the constraints from \cref{eq:lp_As_constr} on a set of randomly generated networks.
Details about the networks are explained later in \cref{sec:dataset}.
For the objective function, we maximize the sum of delay bounds.
We extended the \ac{ULP} implementation from NCorg DNC v2.6.2~\cite{Bondorf2014} for our evaluation.
\Cref{fig:nclp_networksize_vs_solvetime} illustrates the time to find a solution with a time limit of 1 hour using IBM's CPLEX 20.1.0 on an Intel Xeon Gold 5120 at \SI{2.2}{\GHz}.

% Figure environment removed

As expected, the solve time grows exponentially, exceeding the one hour time limit even on small networks.
This result highlights why an alternative solution for optimizing networks is necessary for larger networks.
As a further motivating comparison, \cref{fig:nclp_networksize_vs_solvetime} also illustrates the optimization time on the same networks with our contributed approach: \ac{DiffNC} with Frank-Wolfe.%


\section{Evaluation} \label{sec:evaluation}

\begin{table*}[tbp]
\centering
\small
\begin{tabular}{cccccccccc}
\toprule
& \multicolumn{3}{c}{\msr} & \multicolumn{3}{c}{\negc} & \multicolumn{3}{c}{\wsj} \\
& Acc. & F1 & wF1 & Acc. & F1 & wF1 & Acc. & F1 & wF1 \\ \cmidrule(lr){2-4} \cmidrule(lr){5-7} \cmidrule(lr){8-10} 
\udel & 66.86 & 56.76 & 64.3 & \textbf{80.80} & 55.45 & 77.9 & 63.74 & 64.23 & 63.2 \\
\icsi & \underline{71.19} & 64.73 & 70.4 & 80.36 & 64.53 & \underline{78.6} & 64.62 & 64.15 & 63.4 \\
\cnts & 68.59 & 61.39 & 67.2 & 78.68 & 61.62 & 76.8 & 64.31 & 64.59 & 64.4 \\
\osu & 68.02 & 60.28 & 66.6 & 79.24 & 57.04 & 76.5 & 69.20 & 69.63 & 68.9 \\
\isg & 67.05 & 58.83 & 65.3 & 77.34 & 59.52 & 75.6 & 69.15 & 69.35 & 69.2 \\ \midrule
\bert & \textbf{71.68} & \underline{66.70} & \textbf{71.4} & 77.79 & \underline{72.87} & 77.7 & \underline{80.95} & \underline{80.93} & \underline{80.9} \\
\roberta & 70.91 & \textbf{67.53} & \underline{70.7} & \textbf{80.80} & \textbf{77.29} & \textbf{80.7} & \textbf{82.61} & \textbf{82.70} & \textbf{82.6} \\ \midrule
Average & 69.19 & 62.32 & 67.99 & 79.29 & 64.05 & 77.69 & 70.65 & 70.80 & 70.37 \\
\bottomrule
\end{tabular}
\caption{\label{tab:performance} Overall accuracy (Acc.), macro-averaged F1 (F1), and weighted-macro F1 (wF1) scores of the algorithms depicted in Section~\ref{sec:algorithm}. For instance, \msr-\udel refers to a C5.0 classifier trained on the \msr~corpus, using the feature set mentioned in \citet{greenbacker-mccoy-2009-udel}.}
%Its Acc., F1 and wF1 of this model are 66.86, 56.76, and 64.3, respectively.}
\end{table*}


In this section, we introduce the evaluation protocol and report the performance of the models.

\subsection{Implementation Details} \label{sec:implementation}

For \bert and \roberta, we used \textit{bert-base-cased} and \textit{roberta-base}, both from Hugging Face. For fine-tuning, we set the batch size to 16, the learning rate to 1e-3, the dropout rate to 0.5, and the size of the output layer to 256. We ran each model for 20 epochs and used the one that achieved the highest F1 score on the development set. The implementation details of the classic ML-based models can be found in Appendix~\ref{sec:appendixML}.

\subsection{Evaluation Protocol} \label{sec:protocol}

The main evaluation metric in the GREC-MSR shared tasks was accuracy. 
In addition to accuracy, we also report macro-F1 and weighted-macro F1. We argue that different metrics evaluate algorithms from different perspectives and provide us with different meaningful insights. 
For pragmatic tasks like REG, it makes sense to ask how well an algorithm performs on naturally distributed data which is often imbalanced. For these cases, reporting accuracy and weighted F1 are logical. 
Furthermore, analogous to other classification tasks, minority categories should not be overlooked. Take as an example the class \emph{description} in the \negc corpus, which occurs only 4\%. If a model fails to produce this class, the produced document might sound unnatural. Therefore, it is important to ensure that an algorithm is not over- or under-generating certain classes. Looking into accuracy and macro-F1 together provides insights into such cases.

\subsection{Performance of the Models}\label{subsec:overallacc}

The overall accuracy of the models, their macro F1, and their weighted-macro F1 are presented in Table \ref{tab:performance}. 
We also present the ranking of the models based on these scores in Appendix~\ref{sec:app_rank}. 


\paragraph{PLM-based Models.} The best-performing models across all corpora and metrics are PLM-based models.  In six out of nine rankings, \bert and \roberta are ranked as the top two models. The sole exception is \negc, where \bert is the second worst model. The benefit of using PLMs is the largest on the \wsj corpus. For example, \roberta improves the macro F1 score from 69.63 (i.e., the performance of the best ML-based model) to 82.70.


\paragraph{ML-based Models.} In contrast to the robust performance of the PLM models, the performance of the classic ML models is more corpus-dependent. In the case of \msr and \negc, \icsi is the best-performing model, while in the case of \wsj, it is at the bottom section of the rankings. Another interesting observation is the performance of the \udel models. In terms of accuracy, \udel has the highest performance in \negc, while it has the lowest performance in both \msr and \wsj. In terms of macro-F1 rankings, the \negc \udel model dropped from first to last place, whereas \bert improved from penultimate place to second place. In general, our ML models yielded lower scores than the original models used in the GREC study \citep{belz2009generating}. This could be attributed to a variety of factors, including differences in feature engineering and model parameters.

\paragraph{Comparing Different Metrics.} 

Upon comparing average scores across the three metrics, we observe that for \msr and \negc, PLMs are clear winners only when macro-F1 is the metric in question. However, for \wsj, PLMs are winners on all three metrics. This may be because the distribution of categories in \wsj is much more balanced than in the other two corpora.

\acresetall
\section{Conclusion and Future Work}
In this work, I design corruption-robust algorithms for the Lipschitz contextual search problem. I present the \emph{agnostic checking} technique and demonstrate its effectiveness in designing corruption-robust algorithms. There are several open problems for future research. First, in the algorithm I propose for pricing loss, the schedule for agnostic checks is fixed upfront. Can the learner design an adaptive checking schedule for the pricing loss? Second, this work assumes the learner has knowledge of the Lipschitz constant $L$. Can the learner design efficient no-regret algorithms without knowledge of $L$? 

\footnotesize
\yyyymmdddate
\bibliographystyle{IEEEtran}
\bibliography{IEEEabrv,biblio}


\begin{IEEEbiographynophoto}{Fabien Geyer}
is currently with Airbus Central Research \& Technologies and Technical University of Munich (TUM) working on methods for network analytics, network performances and architectures. He received the master of engineering in telecommunications from Telecom Bretagne, France in 2011 and the Ph.D. degree in computer science from TUM in 2015. His research interests include novel methods for data-driven networking, formal methods for performance evaluation and modeling of networks.
\end{IEEEbiographynophoto}

\begin{IEEEbiographynophoto}{Steffen Bondorf}
is the Professor of Distributed and Networked Systems in the Faculty of Computer Science
at Ruhr University Bochum, Germany. 
Steffen received his Dr.-Ing. in Computer Science from TU Kaiserslautern, Germany, in 2016.
After graduation, he was a research fellow at National University of Singapore and an ERCIM Fellow at NTNU Trondheim, Norway.
Steffen's research interests are in performance analysis of networked systems.
\end{IEEEbiographynophoto}

\end{document}
