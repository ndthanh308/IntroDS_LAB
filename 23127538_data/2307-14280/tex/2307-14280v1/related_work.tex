
\section{Related work}
\label{sec:related_work}

\subsubsection*{Optimization with delay bounds}

Various works already investigated delay bound minimization.
\cite{Bouillard2008b} proposed one of the early works on route optimization based on \ac{NC} by using shortest path on graphs with weights set according to delay bounds.
While this approach was shown to be efficient, it is limited to the optimization of a single flow's route, restricting its use when multiple routes need to be optimized.

In \cite{Cattelan2017}, various iterative algorithms for rerouting flows to minimize tail delays were detailed.
These \ac{NC}-based algorithms appeared to scale to realistic network sizes. %
Various works modeled delay bounds as an \ac{ILP} for optimizing routes.
\cite{DeDinechin2014} used a linear formulation of the \ac{NC} end-to-end delay bound for optimizing the network-on-chip of a many core processor.
Their approach showed promising results compared to a nonlinear formulation on a small network. %
Similarly, \cite{Schweissguth2020} recently proposed another \ac{ILP} formulation tailored to \ac{TSN} and multicast flows, optimizing paths and schedules.
For both \cite{DeDinechin2014} and \cite{Schweissguth2020}, the scalability of both approaches to larger networks remains unclear.

In the scope of \ac{TSN}, \cite{Laursen2016} applied worst-case delay calculations in combination with a greedy optimization approach.
While the results show improvements over a shortest path approach, the formulation is tailored to the \ac{TSN} schedulers and the optimality of the solution is difficult to assess.
More recently, \cite{Bulbul2022} applied reinforcement learning to routing for routing for \ac{TSN} to meet flow deadlines.
Their method is based on packet-level simulation of the network, effectively giving no guarantees about the network delays computed.

\subsubsection*{Derivation of service requirements}

An \ac{NC}-based approach that can derive a lower bound on the system service was proposed in~\cite{Vastag11,Buchholz2017}.
It extends the (min,plus)-algebraic \ac{NC} with novel theory to take as input a function upper bounding the delay to be guaranteed under a certain load level.
However, the approach is currently restricted to \ac{FIFO} systems.
\ac{DiffNC} can perform the same task, yet without any restricting assumptions. 
It can fully use existing algebraic \ac{NC} theory and analyses.

\subsubsection*{NC combined with other methods}

(min,plus) algebra can be replaced with (max,plus) to better fit discrete event systems \cite{Liebeherr2017}.
\ac{NC} was paired with event stream theory \cite{Boyer2016} and with timed automata \cite{Lampka2009} for state-based system modeling.
\ac{NC} has been applied to the component-based models of real-time systems \cite{Thiele2000}, giving rise to the so-called real-time calculus.

Various formulations of the (min,plus) algebra as \ac{LP} were proposed, either addressing networks without assumptions on the multiplexing of flows \cite{Bouillard2010}, or with \ac{FIFO} scheduling \cite{Bouillard2015}.
These formulations provide tight delay bounds but scale poorly, as shown by \cite{Bondorf2017a} and later also in \cref{sec:optlp}.
These concerns were partially addressed recently in \cite{Bouillard2022}.
Additionally, \cite{Dang2014} also proposed an \ac{ILP} for optimizing \ac{TDMA} schedules in combination with \ac{NC}.

Finally, \ac{ML} was recently brought to \ac{NC} to speed-up costly network analyses originally requiring a search mimicking optimization in the algebraic approach.
DeepTMA was proposed in \cite{Geyer2019b,Geyer2020b} as a framework for predicting the best contention model.
Similarly, DeepFP \cite{Geyer2021a} targeted the prediction of best flow prolongation.
\cite{Mai2021} applied similar deep learning techniques for checking feasibility of network configurations, yet not for their synthesis.

