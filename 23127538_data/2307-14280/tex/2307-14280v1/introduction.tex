
\section{Introduction}

\IEEEPARstart{W}{ith} the recent development of networking solutions with stringent reliability and safety requirements (such as IEEE \ac{TSN}), the formal verification and optimization of safety-critical networks has become an important step in the design process in various industries \cite{Geyer2016}.
While the use of mathematical models and the formalization of end-to-end delay bounds has now become common practice, the optimization and fine-tuning of networks under such formulations remains a challenging task.
The main difficulty arises from the inherent combinatorial property of the formal models which are oftentimes nonlinear as well.
This makes them hard problems to solve in polynomial time.
Previous attempts have often been limited to small networks.

In this article, we propose a novel approach for modeling, optimizing, and synthesizing networks under hard end-to-end delay constraints, able to scale to networks of realistic sizes.
We present an approach able to efficiently synthesize parameters; illustrated on flow paths and flow priorities, yet extendable to. e.g., schedulers' parameters, too.
We bound end-to-end delays using \ac{NC} based on the (min,plus) algebra~\cite{LeBoudec2001}.
While this method is commonly used in some industries to formally validate delay requirements, 
it is rarely used for synthesis or as a design tool.
Existing \ac{NC} analyses have been designed to analyze already completed network designs, making them only suitable for design space exploration that enumerates and ranks different designs.

We present an extension of \ac{NC} called \ac{DiffNC}.
We formally show that under the assumptions traditionally used for validating industrial networks (i.e., token-bucket and rate-latency curves), the delay bound of a flow computed using the (min,plus) algebra is differentiable according to the parameters of the different curves in the network.
This enables a wide range of applications, including gradient-based \ac{NLP}.
Via variable relaxation, we demonstrate that traditional \ac{NLP} methods based on Newton's method can efficiently solve the aforementioned network optimization problems -- and synthesize configurations.
We show that these optimization methods are highly efficient, scale well and provide the best solutions, making them applicable to networks of sizes found in the industry.

In the realm of \ac{NC}, previous works already formalized \ac{NC} as an optimization problem, by proposing a formulation of the end-to-end delay bounds as an \ac{LP}~\cite{Bouillard2010}.
This approach is able to achieve tight delay bounds.
We illustrate that this \ac{LP} formulation can also be extended to  %
multiple flows in its objective function. 
It can be used to optimize paths of flows, but the objective function suffers from poor expressiveness for some important types of constraints. 
In addition, we show that it suffers from poor scalability, requiring more than an hour of computation even on relatively small networks.
These limitations make the approach unsuitable for realistic problems.

Our proposed approach has the following advantages.
First, we use an existing \ac{NC} analysis to derive a (min,plus)-algebraic term bounding the delay, yet with integer variables encoding alternatives such as potential flow paths or flow priority. 
Then, for finding the best alternative with \ac{NLP}, we can include nonlinear constraints and nonlinear objective functions, allowing for concepts like utility functions \cite{Kelly1998} on the delay bounds.

To solve the \ac{NLP}, we chose the Frank-Wolfe algorithm, an efficient \ac{NLP} based on the well-known gradient descent algorithm.
We show that it can be used to find a good solution in terms of optimality in a short amount of time.
In our numerical evaluation, we demonstrate that we are able to optimize the \ac{AFDX} network used in the Airbus A350 in a matter of seconds, with better optimality than previous works.
We thus illustrate that our approach is scalable to real networks with more than \num{1000} flows.
Our implementation is based on efficient \ac{CAS} and \ac{AD}, which allows us to efficiently compute the end-to-end delay bounds and their gradient without paying dearly in terms of computation time.

This article is organized as follows:
\cref{sec:related_work} presents the related work, followed by \ac{NC} in \cref{sec:dncintro}.
\Cref{sec:diffnc} presents the mathematical foundations of \ac{DiffNC} and suitable network optimization problems,
followed by our \ac{DiffNC} implementation in \cref{sec:diffnc_ad}.
\cref{sec:optlp} extends existing \ac{LP}-based \ac{NC} to compete with \ac{DiffNC}.
We numerically evaluate and compare \ac{DiffNC} against other optimization methods in \cref{sec:numerical_evaluation}.
Finally, \cref{sec:conclusion} concludes the article.
