\documentclass[10pt]{amsart}
\usepackage{amsmath,amssymb,latexsym,cancel,rotating}
\usepackage{graphicx,amssymb,mathrsfs,amsmath,color,fancyhdr,amsthm}
\usepackage[all]{xy}

\textwidth15.1cm \textheight21cm \headheight12pt
\oddsidemargin.4cm \evensidemargin.4cm \topmargin0.5cm

\addtolength{\marginparwidth}{-13mm}
\newcommand{\margin}[1]{\marginpar{\tiny #1}}

\newtheorem{theorem}{Theorem}[section]
\newtheorem{lemma}[theorem]{Lemma}
\newtheorem{corollary}[theorem]{Corollary}
\newtheorem{definition}[theorem]{Definition}
\newtheorem{proposition}[theorem]{Proposition}
\newtheorem{conjecture}[theorem]{Conjecture}
\newtheorem{remark}[theorem]{Remark}
\newtheorem{problem}[theorem]{Problem}
\newtheorem{example}[theorem]{Example}
\newtheorem{question}[theorem]{Question}
\newtheorem{thm}{Theorem}



\begin{document}

\title[Lusztig sheaves and  integrable highest weight modules]
{Lusztig sheaves and  integrable highest weight modules}
\author[Fang,Lan,Xiao]{Jiepeng Fang,Yixin Lan,Jie Xiao}
\address{School of mathematical secience, Peking University, Beijing 100871, P. R. China}
\email{fangjp@math.pku.edu.cn (J.Fang)}

\address{Academy of Mathematics and Systems Science, Chinese Academy of Sciences, Beijing 100190, P.R.China}
 \email{lanyixin@amss.ac.cn (Y.Lan)}

\address{School of mathematical seciences, Beijing Normal University, Beijing 100875, P. R. China}
\email{jxiao@bnu.edu.cn (J.Xiao)}




\begin{abstract}
	Using Lusztig's category $\mathcal{Q}_{\mathbf{V}}$ for quivers, we consider the split semisimple subcategory $\mathcal{Q}^{0}_{\mathbf{V}}$ and its localization $\mathcal{Q}^{0}_{\mathbf{V}}/\mathcal{N}_{\mathbf{V}}$, and define functors  $E^{(n)}_{i},F^{(n)}_{i},K^{\pm}_{i},n\in \mathbb{N},i \in I$ between localizations. With these functors, the Grothendieck group of localizations realizes the irreducible integrable highest weight modules $L(\Lambda)$ of quantum groups. Moreover, the nonzero simple perverse sheaves in localizations form the canonical bases of $L(\Lambda)$. We also compare our realization (at $v \rightarrow 1$) with Nakajima's realization via quiver varieties and prove that  the transition matrix between canonical bases and fundamental classes is upper triangular with diagonal entries all equal to $1$ (up to sign).
\end{abstract}


\keywords{perverse sheaves, quantum groups, integrable highest weight modules,Nakajima quiver varieties}

\subjclass[2000]{16G20, 17B37}

\date{\today}

\bibliographystyle{abbrv}

\maketitle


\setcounter{tocdepth}{1}\tableofcontents

\section{Introduction}
Given a symmetric Cartan datum $(I,(-,-))$, we can associate an acyclic quiver $Q=(I,H,\Omega)$ and define its Kac-Moody Lie algbera $\mathfrak{g}$ and the quantized enveloping algbera (or quantum group) $\mathbf{U}=\mathbf{U}_{v}(\mathfrak{g})$. In \cite{MR1088333},\cite{MR1227098} and \cite{MR1653038}, G.Lusztig  has considered the moduli space $\mathbf{E}_{\mathbf{V},\Omega}$ of quiver representations and introduced his category $\mathcal{Q}_{\mathbf{V}}$  of semisimple perverse sheaves (complexes).  Perverse sheaves in $\mathcal{Q}_{\mathbf{V}}$ are called Lusztig sheaves. By using Grothendieck's six operators, the induction functor  $\mathbf{Ind}^{\mathbf{V}}_{\mathbf{V}',\mathbf{V}''}$ and the restriction functor $\mathbf{Res}^{\mathbf{V}}_{\mathbf{T},\mathbf{W}}$ have been defined. Together with the induction and restriction functors, the Grothendieck group $\mathcal{K}$ of the category $\bigoplus \limits_{\mathbf{V}}\mathcal{Q}_{\mathbf{V}}$ becomes a bialgebra, which is canonically isomorphic to the integral form ${_{\mathcal{A}}\mathbf{U}^{+}}$ (or ${_{\mathcal{A}}\mathbf{U}^{-}}$) of the positive (or negative) part of the quantum group. (Here $\mathcal{A}=\mathbb{Z}[v,v^{-1}]$.) Moreover, the set $\mathcal{P}$ of simple Lusztig sheaves forms a basis of ${_{\mathcal{A}}\mathbf{U}^{+}}$,  which is called the canonical basis by Lusztig. The canonical basis has many remarkable properties, such as integral property and positive property.

Given a dominant weight $\Lambda$, one can define the irreducible highest weight module $L(\Lambda)$. Even though Lusztig hasn't provided a categorification of $L(\Lambda)$, he has constructed the canonical basis of $L(\Lambda)$. Indeed, if we identify $\mathcal{K}$ with ${_{\mathcal{A}}\mathbf{U}}^{-}$ and consider the canonical map $$\pi: \mathbf{U}^{-} \rightarrow \mathbf{U}^{-}/ \sum\limits_{i \in I } \mathbf{U}^{-} f_{i}^{\langle \Lambda, \alpha_{i}^{\vee} \rangle +1} \cong L(\Lambda) $$  
then $\{ \pi([L])\neq 0| L\in \mathcal{P} \}$ forms a basis of $L(\Lambda)$. However, a categorical realization  of $L(\Lambda)$ and its canonical basis is still expected.

H.Zheng took a breakthrough in his work \cite{MR3200442}. He categorified the irreducible integrable highest weight modules and their tensor products by using classes of micro-local perverse sheaves $\mathfrak{D}_{\overrightarrow{\omega}}$ on moduli stacks of framed quivers. Later, Y.Li in \cite{MR3177922}  provided a crystal structure for a class of simple perverse sheaves on framed quivers to realize the crystal structure $B(\Lambda)$ of $L(\Lambda)$ ( and their tensor products).   \nocite{MR4379282}

Compared with Lusztig's theory, another approach to categorify the quantum group is to use the projective representations of quiver Hecke algebras \cite{MR2525917} ,\cite{MR2763732}, \cite{MR2837011}. S-J.Kang and M.Kashiwara considered the cyclotomic quiver Hecke algebras $R^{\Lambda}$ in \cite{MR2995184}, which is a quotient of quiver Hecke algebras. They defined functors $F^{\Lambda}_{i},E^{\Lambda}_{i},i \in I$ for the category of modules of cyclotomic quiver Hecke algebras, which satisfy the following relation
\begin{equation*}
	q_{i}^{-2}F^{\Lambda}_{i}E^{\Lambda}_{i} \oplus \bigoplus \limits_{k \geq 0}^{ \langle h_{i},\Lambda \rangle -1} q_{i}^{2k} Id \cong E^{\Lambda}_{i} F^{\Lambda}_{i},  \langle h_{i},\Lambda \rangle \geq 0,
\end{equation*}
\begin{equation*}
	q_{i}^{-2}F^{\Lambda}_{i}E^{\Lambda}_{i}  \cong E^{\Lambda}_{i} F^{\Lambda}_{i} \oplus \bigoplus \limits_{k \geq 0}^{ -\langle h_{i},\Lambda \rangle -1} q_{i}^{2k} Id,  \langle h_{i},\Lambda \rangle \leq 0.
\end{equation*}
Then the Grothendieck group of projective modules becomes a $\mathbf{U}$-module and $[Proj(R^{\Lambda})] \cong {_{\mathcal{A}}L(\Lambda)}.$ The key ingredient of their construction is the following exact sequence 
\begin{equation*}
	0 \rightarrow \bar{F}_{i}M \rightarrow F_{i}M \rightarrow F^{\Lambda}_{i}M \rightarrow 0,
\end{equation*}
see \cite[Theorem 4.7]{MR2995184}, which categorifies the following equation 
\begin{equation*}
	[e_{i}, P]= \frac{K^{-1}_{i}e'_{i}(P)-K_{i}e''_{i}(P) }{q_{i}^{-1}-q_{i}}.
\end{equation*}
This provides a successful model to categorify $L(\Lambda)$ \nocite{MR3084241}. 

Inspired by H. Zheng's work \nocite{zheng2007geometric} \cite{MR3200442} and with our puzzles about his proof, we expect to go back to Lusztig's theory and give a categorical realization of $L(\Lambda)$. In the present paper, we introduce the split semisimple category $\mathcal{Q}^{0}_{\mathbf{V}}$, which is a subcategory of  $\mathcal{Q}_{\mathbf{V}}$ consisting of the same objects, and define its localization $\mathcal{L}_{\mathbf{V}}(\Lambda)=\mathcal{Q}^{0}_{\mathbf{V}}/\mathcal{N}_{\mathbf{V}}$. After defining functors $E^{(n)}_{i},F^{(n)}_{i},K^{\pm}_{i},n\in \mathbb{N},i \in I$ of localizations, we have the following main result:
\begin{theorem}
	With the action of linear operators induced by functors $E^{(n)}_{i},F^{(n)}_{i},K^{\pm}_{i},n\in \mathbb{N},i \in I$, the Grothendieck group $\mathcal{K}_{0}(\Lambda)$ of $\coprod\limits_{\mathbf{V}}\mathcal{Q}^{0}_{\mathbf{V}}/\mathcal{N}_{\mathbf{V}}$  becomes a $_{\mathcal{A}}\mathbf{U}$-module, and there exists a canonical isomorphism of $_{\mathcal{A}}\mathbf{U}$-modules
	\begin{equation*}
		\varsigma^{\Lambda}:\mathcal{K}_{0}(\Lambda) \rightarrow {_{\mathcal{A}}L(\Lambda)}.
	\end{equation*}
	The morphism $\varsigma^{\Lambda}$ sends the constant sheaf $[\overline{\mathbb{Q}}_{l}]=[L_{0}]$ on
	$\mathbf{E}_{0,\Omega}$
	to the highest weight vector $v_{\Lambda}$ in  ${_{\mathcal{A}}L(\Lambda)}$.
	
	Moreover, the set
	$\{\varsigma^{\Lambda}([L])|L$ is a simple perverse sheaf in $\mathcal{L}_{\mathbf{V}}(\Lambda)\}$ form a bar-invariant $\mathcal{A}$-basis of ${_{\mathcal{A}}L_{\mathbf{V}}(\Lambda)}$, which is exactly the canonical basis of $L(\Lambda)$.
\end{theorem}

We also define category $\hat{\mathcal{Q}}^{0}$ and functors $\hat{\mathcal{R}}^{\Lambda}_{i}$ and $ {_{i}\hat{\mathcal{R}}^{\Lambda}}$, which categorify linear operators $\frac{ v^{ \langle \Lambda,\alpha^{\vee}_{i} \rangle } }{v^{-1}-v}\bar{r}_{i}$ and $\frac{ v^{(i,|\mathbf{V}'|)- \langle \Lambda ,\alpha^{\vee}_{i} \rangle } }{v^{-1}-v} {_{i}\bar{r}}: {_{\mathcal{A}}\mathbf{U}^{-}  } \rightarrow \mathbf{U}^{-}$ respectively, then we obtain a split exact sequence as follows.
\begin{theorem}
	For any object $L$ of $\mathcal{Q}^{0}_{\mathbf{V}}$, we have a split exact sequence in $\hat{\mathcal{Q}}^{0}_{\mathbf{V}'}/ \mathcal{N}_{\mathbf{V}'}$ 
	\begin{equation*}
		0 \rightarrow {\hat{\mathcal{R}}^{\Lambda}_{i}}(L)  \rightarrow {_{i}\hat{\mathcal{R}}^{\Lambda}}(L)  \rightarrow E_{i}(L) \rightarrow 0.
	\end{equation*}
\end{theorem}
The theorem above is parrallel to the exact sequence in \cite[Theorem 4.7]{MR2995184} and categorifies the equation $$E_{i}(x \cdot  v_{\Lambda} )=( v^{(i,|x|-i)- \langle \Lambda,\alpha^{\vee}_{i} \rangle}{_{i} \bar{r}}(x)\cdot  v_{\Lambda} - v^{\langle \Lambda,\alpha^{\vee}_{i} \rangle} \bar{r}_{i}(x)\cdot  v_{\Lambda} ) /(v^{-1}-v ).  $$




Recall that H.Nakajima provided a construction of irreducible highest weight $\mathfrak{g}$-modules (denoted by $L_{0}(\Lambda)$) via Borel-Moore cohomology groups of quiver varieties $\mathfrak{m}(\nu,\omega)$ and $\mathfrak{L}(\nu,\omega)$ in \cite{MR1302318} and \cite{MR1604167}. He defined operators $E_{i},F_{i},i \in I$ by using Hecke correspondences and proved that with operators $E_{i},F_{i}$, $\bigoplus \limits_{\nu}H_{top}( \mathfrak{L}(\nu,\omega))$ becomes a $\mathbf{U}(\mathfrak{g})$-module, which is isomorphic to the irreducible integrable highest weight $\mathfrak{g}$-module $L(\Lambda)$ ( We denote this isomorphism by $\varkappa^{\Lambda}$). Moreover, the fundamental classes of irreducible components of  $\mathfrak{L}(\nu,\omega)$ form a basis of $L_{0}(\Lambda)$. We compare our sheaf realization (taking $v \rightarrow 1$) with Nakajima's construction via  quiver varieties and provide a correspondence $\Phi^{\Lambda}$ between  the set of simple objects in localizations and the set of  irreducible components of  $\mathfrak{L}(\nu,\omega)$. After defining  orders $\preceq$ and $\preceq'$ on these sets respectively, we can state our next main result:
\begin{theorem}
	The transition matrix from canonical basis to  fundamental classes is upper triangular and with diagonal entries all equal to $\pm 1$.
	More precisely, if $X$ is an irreducible component of $\mathfrak{L}(\nu,\omega)$  and $[L]=\Phi^{\Lambda}(X)$, then we have 
	\begin{equation*}
	\varsigma^{\Lambda}([L])=\varkappa^{\Lambda}(sgn(X)[X]) +\sum\limits_{X \preceq X' } c_{X'}\varkappa^{\Lambda} (sgn(X')[X'])
\end{equation*}
	where $c_{X'} \in \mathbb{Q}$ are constants.
\end{theorem} 

Obviously, as what has been done in  \cite{MR1865400},  \cite{MR3077693} and \cite{MR3200442},  a realizaition of tensor products of integrable highest weight modules via Lusztig sheaves is expected. We have done this work in the preprint \cite{fang2023tensor}. 


\section{Lusztig sheaves and  quantized enveloping algebras}
 In this section, we recall Lusztig's theory of semisimple perverse sheaves and refer \cite{MR1227098} for details..
 \subsection{Induction functor and restriction functor}
Given a symmetric Cartan datum $(I,(-,-))$, let $\mathbf{\Gamma}$ be the finite graph without loops associated to $(I,(-,-))$, where $I$ is the set of vertices and $H$ is the set of pairs consisting of edges with an orientation. More precisely, to give an edge with an orientation is equivalent to give $h',h'' \in I$ and we adapt the notation $h' \xrightarrow{h} h''$. Let $-:h \mapsto \bar{h}$ be the involution of $H$ such that $\bar{h}'=h'',\bar{h}''=h'$ and $\bar{h} \neq h$. An orientation of the graph $\Gamma$ is a subset $\Omega \subset H$ such that $\Omega \cap \bar{\Omega} =\emptyset$ and $\Omega \cup \bar{\Omega} = H$.

Let $k=\overline{\mathbb{F}}_q$ be the algebraic closure of the finite field $\mathbb{F}_q$. Given $\nu \in \mathbb{N}[I]$, a subset $\tilde{H} \subseteq H$ and an $I$-graded $k$-vector space $\mathbf{V}$ of dimension vector $|\mathbf{V}|=\nu$,  we take
\begin{center}
	$\mathbf{E}_{\mathbf{V}}= \bigoplus\limits_{h \in H} \mathbf{Hom}(\mathbf{V}_{h'},\mathbf{V}_{h''})$, \\
	$\mathbf{E}_{\mathbf{V}, \tilde{H}}= \bigoplus\limits_{h \in \tilde{H}} \mathbf{Hom}(\mathbf{V}_{h'},\mathbf{V}_{h''}).$
\end{center} 
In particular, for an orientation $\Omega$, we have 
\begin{center}
	$\mathbf{E}_{\mathbf{V}, \Omega}= \bigoplus\limits_{h \in \Omega} \mathbf{Hom}(\mathbf{V}_{h'},\mathbf{V}_{h''}).$
\end{center}

The algebraic group $G_{\mathbf{V}}= \prod\limits_{i \in I} \mathbf{GL}(\mathbf{V}_{i})$ acts on $\mathbf{E}_{\mathbf{V}},\mathbf{E}_{\mathbf{V},\tilde{H}}$ and $\mathbf{E}_{\mathbf{V}, \Omega}$ by $(g \cdot x)_{h} =g_{h''} x_{h} g_{h'}^{-1} $. Let $\mathcal{D}^{b}_{G_{\mathbf{V}}}(\mathbf{E}_{\mathbf{V},\Omega})$ be the  $G_{\mathbf{V}}$-equivariant derived category of mixed sheaves on $\mathbf{E}_{\mathbf{V},\Omega}$. For any $n \in \mathbb{Z}$, we denote by $\mathbf{\Sigma}^{n}$ the shift functor, $(\frac{n}{2}
)$ the Tate twist if $n$ is even or the
square root of the Tate twist if $n$ is odd, and $[n]$ the composition $(\mathbf{\Sigma})^{n} (\frac{n}{2})$. Excpet in Section 3.5, the shift functors  $\mathbf{\Sigma}^{n}$ always appear together with $(\frac{n}{2})$. We say complexes $A$ and $B$ are isomorphic up to shifts, if  $A$ and $B[n]$ are isomorphic for some $n \in \mathbb{Z}$. 

Given $\nu'+\nu''=\nu \in \mathbb{N}[I] $ and graded vector spaces $\mathbf{V},\mathbf{V}',\mathbf{V}''$ of dimension vectors $\nu, \nu', \nu''$ respectively, let $\mathbf{E}'_{\Omega}$ be the variety consisting of $(x,\tilde{\mathbf{W}}, \rho_{1}, \rho_{2})$, where $x \in \mathbf{E}_{\mathbf{V},\Omega}$,$\tilde{\mathbf{W}}$ is an $I$-graded $x$-stable  subspace of $\mathbf{V}$ of dimension vector $\nu''$ and $ \rho_{1}: \mathbf{V}/\tilde{\mathbf{W}} \simeq \mathbf{V}',\rho_{2}:\tilde{\mathbf{W}} \simeq \mathbf{V}''$ are linear isomorphisms. Here we say $\tilde{\mathbf{W}}$ is $x$-stable if and only if $x_{h}(\tilde{\mathbf{W}}_{h'}) \subset \tilde{\mathbf{W}}_{h''}$ for any $h \in \Omega$.  Let $\mathbf{E}''_{\Omega}$ be the variety consisting of $(x,\tilde{\mathbf{W}})$ as above.
Consider the following diagram
\begin{center}
	$\mathbf{E}_{\mathbf{V}',\Omega} \times \mathbf{E}_{\mathbf{V}'',\Omega} \xleftarrow{p_{1}} \mathbf{E}'_{\Omega} \xrightarrow{p_{2}} \mathbf{E}''_{\Omega} \xrightarrow{p_{3}} \mathbf{E}_{\mathbf{V},\Omega}$
\end{center}
where $p_{1}(x,\tilde{\mathbf{W}},\rho_{1},\rho_{2})=(\rho_{1,\ast}(\bar{x}|_{\mathbf{V}/\tilde{\mathbf{W}}}),\rho_{2,\ast}(x|_{\tilde{\mathbf{W}}})  )$, $p_{2}(x,\tilde{\mathbf{W}},\rho_{1},\rho_{2}) =(x, \tilde{\mathbf{W}}) $ and $p_{3}(x,\tilde{\mathbf{W}})=x$,  where $\bar{x}|_{\mathbf{V}/\tilde{\mathbf{W}}}$ is the natural linear map induced by $x$ on the quotient space $\mathbf{V}/\tilde{\mathbf{W}}$ and $x|_{\tilde{\mathbf{W}}}$ is the restriction of $x$ on the subspace $\tilde{\mathbf{W}}$, then $ \rho_{1,\ast}(\bar{x}|_{\mathbf{V}/\tilde{\mathbf{W}}})= \rho_{1} (\bar{x}|_{\mathbf{V}/\tilde{\mathbf{W}}}) \rho_{1}^{-1}\in \mathbf{E}_{\mathbf{V}'}$ and $\rho_{2,\ast}(x|_{\tilde{\mathbf{W}}})=\rho_{2}(x|_{\tilde{\mathbf{W}}}) \rho_{2}^{-1}\in \mathbf{E}_{\mathbf{V}''}$. Notice that  $p_{1}$ is smooth with connected fibers, $p_{2}$ is a principle $G_{\mathbf{V}'} \times G_{\mathbf{V}''}$-bundle and $p_{3}$ is proper.  Let $d_{1}$ be the dimension of the fibers of $p_{1}$ and $d_{2}$ be the dimension of the fibers of $p_{2}$. Lusztig's induction functor is defined by
\begin{align*}
&\mathbf{Ind}^{\mathbf{V}}_{\mathbf{V'},\mathbf{V''}}: \mathcal{D}^{b}_{G_{\mathbf{V}'}}(\mathbf{E}_{\mathbf{V}',\Omega}) \times \mathcal{D}^{b}_{G_{\mathbf{V}''}}(\mathbf{E}_{\mathbf{V}'',\Omega}) \rightarrow \mathcal{D}^{b}_{G_{\mathbf{V}}}(\mathbf{E}_{\mathbf{V},\Omega})\\
&\mathbf{Ind}^{\mathbf{V}}_{\mathbf{V'},\mathbf{V''}}(A\boxtimes B)= (p_{3})_{!}(p_{2})_{\flat}(p_{1})^{\ast}(A\boxtimes B)[d_{1}-d_{2}].
\end{align*}

We fix a decomposition $\mathbf{T} \oplus \mathbf{W} =\mathbf{V}$ of graded vector space such that $|\mathbf{T}|=\nu'$ and $|\mathbf{W}|=\nu''$, let $F_{\Omega}$ be the closed subvariety of $\mathbf{E}_{\mathbf{V},\Omega}$ consisting of $x$ such that $\mathbf{W}$ is $x$-stable. 
 Consider the following diagram
  \begin{center}
 	$\mathbf{E}_{\mathbf{T},\Omega} \times \mathbf{E}_{\mathbf{W},\Omega} \xleftarrow{\kappa_{\Omega} } F_{\Omega} \xrightarrow{\iota_{\Omega}} \mathbf{E}_{\mathbf{V},\Omega}$
 \end{center} 
where $\iota_{\Omega}$ is the natural embedding and $\kappa_{\Omega}(x)=(\overline{x}|_{\mathbf{T}},x|_{\mathbf{W}}) \in \mathbf{E}_{\mathbf{T},\Omega} \times \mathbf{E}_{\mathbf{W},\Omega} $ for any $x \in F_{\Omega}$. Notice that $\kappa_{\Omega}$ is a vector bundle. Lusztig's restriction functor is defined by
\begin{align*}
	&\mathbf{Res}^{\mathbf{V}}_{\mathbf{T},\mathbf{W}}: \mathcal{D}^{b}_{G_{\mathbf{V}}}(\mathbf{E}_{\mathbf{V},\Omega}) \rightarrow \mathcal{D}^{b}_{G_{\mathbf{V}'} \times G_{\mathbf{V}''}}(\mathbf{E}_{\mathbf{T},\Omega}\times \mathbf{E}_{\mathbf{W},\Omega})\\
	&\mathbf{Res}^{\mathbf{V}}_{\mathbf{T},\mathbf{W}}(C)=(\kappa_{\Omega})_{!} (\iota_{\Omega})^{\ast}(C)[-\langle\nu',\nu''\rangle],
\end{align*}
where $\langle \nu',\nu''\rangle=\sum_{i\in I}\nu'_i\nu''_i-\sum_{h\in \Omega}\nu'_{h'}\nu''_{h''}$ is the Euler form.

We denote by $\mathcal{S}_{|\mathbf{V}|}$ the set of sequences $\underline{\nu}=(\nu^{1},\nu^{2},\cdots, \nu^{m})$ such that each $\nu^{l}$ is of the form $(i_{l})^{a_{l}} \in \mathbb{N}[I]$ for some $i_l\in I,a_l\in \mathbb{N}$ and $\sum_{l=1}^m\nu^l=|\mathbf{V}|$. For any $\underline{\nu}=(\nu^{1},\nu^{2},\cdots, \nu^{m}) \in \mathcal{S}_{\mathbf{V}}$, the flag variety $\tilde{\mathcal{F}}_{\underline{v},\Omega}$ is the variety consisting of $(x,f)$, where $x \in \mathbf{E}_{\mathbf{V},\Omega}$ and $f=(0=\mathbf{V}^{m} \subseteq \mathbf{V}^{m-1} \subseteq \cdots \subseteq \mathbf{V}^{0}=\mathbf{V})$ such that  $x(\mathbf{V}^{k}) \subseteq \mathbf{V}^{k}$ and $|\mathbf{V}^{k-1}/\mathbf{V}^{k}|=\nu^{k}$ for every $1\leqslant k \leqslant m$.

The flag variety $\tilde{\mathcal{F}}_{\underline{v},\Omega}$ is smooth and the natural projection map $\pi_{\underline{v},\Omega}:\tilde{\mathcal{F}}_{\underline{v},\Omega} \rightarrow \mathbf{E}_{\mathbf{V},\Omega}$ is proper. Then by the decomposition theorem in \cite{MR751966}, the complex $L_{\underline{v}}= (\pi_{\underline{v},\Omega})_{!} \bar{\mathbb{Q}}_{l}[\dim \tilde{\mathcal{F}}_{\underline{v},\Omega}]$ is a semisimple complex on $\mathbf{E}_{\mathbf{V},\Omega}$, where $\bar{\mathbb{Q}}_{l}$ is the constant sheaf on $\tilde{\mathcal{F}}_{\underline{v},\Omega}$. 

Let $\mathcal{P}_{\mathbf{V},\Omega}$ be the full subcategory of $\mathcal{D}^{b}_{G_{\mathbf{V}}}(\mathbf{E}_{\mathbf{V},\Omega})$ consisting of simple perverse sheaves appearing as direct summands of some $L_{\underline{\nu}}, \underline{\nu}\in \mathcal{S}_{|\mathbf{V}|}$ up to $[n]$ shifts, and let $\mathcal{Q}_{\mathbf{V},\Omega}$ be the full subcategory of $\mathcal{D}^{b}_{G_{\mathbf{V}}}(\mathbf{E}_{\mathbf{V},\Omega})$ consisting of direct sums of  shifts of objects in $\mathcal{P}_{\mathbf{V},\Omega}$. Objects in $\mathcal{Q}_{\mathbf{V},\Omega}$ are called Lusztig sheaves in \cite{MR3202708}.

\begin{proposition}\cite[Lemma 3.2, Proposition 4.2]{MR1088333} \label{indres formula}
For any $\nu'+\nu''=\nu$ and $\underline{\nu}'\in \mathcal{S}_{|\mathbf{V}'|},\underline{\nu}''\in \mathcal{S}_{|\mathbf{V}''|},\underline{\nu}\in \mathcal{S}_{|\mathbf{V}|}$, we have
	\begin{equation*}
		\mathbf{Ind}^{\mathbf{V}}_{\mathbf{V}',\mathbf{V}''}(L_{\underline{\nu}'} \boxtimes L_{\underline{\nu}''})= L_{\underline{\nu}' \underline{\nu}''},
	\end{equation*}
	\begin{equation*}
		\mathbf{Res}^{\mathbf{V}}_{\mathbf{T},\mathbf{W}}( L_{\underline{\nu}}) =\bigoplus \limits L_{\underline{\tau}} \boxtimes L_{\underline{\omega}}[M(\underline{\tau},\underline{\omega})],
	\end{equation*}
	where the direct sum is taken over $\underline{\tau}\in \mathcal{S}_{|\mathbf{T}|}, \underline{\omega}\in \mathcal{S}_{|\mathbf{W}|}$ satisfying $\underline{\tau}+\underline{\omega}=\underline{\nu}$ and 
	\begin{equation*}
		\begin{split}
			M(\underline{\tau},\underline{\omega})=&-\sum\limits_{h \in H, l' < l}(\tau^{l'}_{h'}\omega^{l}_{h''}+\tau^{l'}_{h''}\omega^{l}_{h'})
			+\sum\limits_{h \in H}({\rm{dim} \mathbf{T}_{h'}}{\rm{dim} \mathbf{W}_{h''}}+{\rm{dim} \mathbf{T}_{h''}}{\rm{dim} \mathbf{W}_{h'}}) \\
			&-\sum\limits_{i \in I,l < l'} \tau_{i}^{l'}\omega_{i}^{l}+\sum\limits_{i \in I,l > l'} \tau_{i}^{l'}\omega_{i}^{l}-\sum\limits_{i \in I}{\rm{dim} \mathbf{T}_{i}}{\rm{dim} \mathbf{W}_{i}} .   
		\end{split}
	\end{equation*}
\end{proposition}

As a corollary, the induction and restriction functors can be restricted to
	\begin{align*}
		&\mathbf{Ind}^{\mathbf{V}}_{\mathbf{V}',\mathbf{V}''}:\mathcal{Q}_{\mathbf{V}'}\times \mathcal{Q}_{\mathbf{V}''}\rightarrow \mathcal{Q}_{\mathbf{V}},\\
		&\mathbf{Res}^{\mathbf{V}}_{\mathbf{T},\mathbf{W}}:\mathcal{Q}_{\mathbf{V}}\rightarrow \mathcal{Q}_{\mathbf{T}}\boxtimes \mathcal{Q}_{\mathbf{W}}.
\end{align*}

Let $\mathcal{K}_{\mathbf{V},\Omega}$ be the Grothendieck group of
$\mathcal{Q}_{\mathbf{V},\Omega}$ and $\mathcal{K}_{\Omega}=\bigoplus\limits_{\mathbf{V}} \mathcal{K}_{\mathbf{V},\Omega}$ which have a $\mathcal{A}$-module structure given by
\begin{center}
	$v[L]=[L[1]].$ 
\end{center} 

\begin{theorem}\cite[Theorem 10.17]{MR1088333}\label{Lussztig1}
	With the induction and restriction functors, the Grothendieck group $\mathcal{K}$ becomes a bialgebra, and is canonically isomorphic to the (integral form of) positive part of the quantized enveloping algebra ${_{\mathcal{A}}}\mathbf{U}^{+}$:
	\begin{equation*}
		\varsigma:[L_{\underline{i^{(p)}}}] \mapsto E_{i}^{(p)}
	\end{equation*} 
	where $L_{\underline{i^{(p)}}}$ is the constant sheaf on $\mathbf{E}_{\mathbf{V},\Omega}$ with $|\mathbf{V}|=pi$.
	Moreover, the images of simple perverse sheaves in $\mathcal{P}_{\mathbf{V}}$ form a $\mathcal{A}$-basis of ${_{\mathcal{A}}}\mathbf{U}_{\mathbf{V}}^{+}$, which is called the canonical basis. The canonical basis is bar-invariant and has positivity.
\end{theorem}


 
\subsection{Fourier-Deligne transformation}
We fix a nontrivial character $\mathbb{F}_{q} \rightarrow \bar{\mathbb{Q}}_{l}^{\ast}$. This character defines an Artin-Schreier local system of rank $1$ on $k$. 

For two orientations $\Omega,\Omega'$, we define $T: \mathbf{E}_{\mathbf{V},\Omega \cup \Omega'} \rightarrow k$ by $T(x)=\sum \limits_{h \in \Omega \backslash \Omega'}tr(x_{h}x_{\bar{h}})$. Then the inverse image of the Artin-Schreier local system under $T$ is a well-defined $G_{\mathbf{V}}$-equivariant local system of rank $1$ on $\mathbf{E}_{\mathbf{V},\Omega \cup \Omega'}$, denote by $\mathcal{L}_{T}$.

Consider the following diagram
\begin{center}
	$\mathbf{E}_{\mathbf{V},\Omega} \xleftarrow{\delta} \mathbf{E}_{\mathbf{V},\Omega \cup \Omega'} \xrightarrow{\delta'} \mathbf{E}_{\mathbf{V},\Omega'}$
\end{center}
where $\delta,\delta'$ are the forgetting maps defined by
\begin{center}
	$\delta((x_{h})_{h \in \Omega \cup \Omega'} )= ((x_{h})_{h \in \Omega}),$\\
	$\delta'((x_{h})_{h \in \Omega \cup \Omega'} )= ((x_{h})_{h \in \Omega'}).$
\end{center}

Lusztig defined the Fourier-Deligne transformation for quivers to be the functor
\begin{align*}
	&\mathcal{F}_{\Omega,\Omega'}:\mathcal{D}^{b}_{G_{\mathbf{V}}}(E_{\mathbf{V},\Omega}) \rightarrow \mathcal{D}^{b}_{G_{\mathbf{V}}}(E_{\mathbf{V},\Omega'})\\ &\mathcal{F}_{\Omega,\Omega'}(L)=\delta'_{!}(\delta^{\ast}(L)\otimes \mathcal{L}_{T})[D],
\end{align*}  
here $D=\sum\limits_{h \in \Omega \backslash \Omega'}\dim \mathbf{V}_{h'}\dim\mathbf{V}_{h''}$.

\begin{proposition} \cite[Theorem 5.4]{MR1088333} \label{FD0}
	With the notations above, we have
\begin{center}
	$\mathcal{F}_{\Omega,\Omega'}(\mathbf{Ind}^{\mathbf{V}} _{\mathbf{V}',\mathbf{V}''}(L_{1} \boxtimes L_{2})) \cong  \mathbf{Ind}^{\mathbf{V}} _{\mathbf{V}',\mathbf{V}''}(\mathcal{F}_{\Omega,\Omega'}(L_{1}) \boxtimes \mathcal{F}_{\Omega,\Omega'}(L_{2})).$
\end{center}
\end{proposition}


\begin{corollary}\cite[Proposition 10.14]{MR1088333} \label{FD1}
	The functor $\mathcal{F}_{\Omega,\Omega'}$ induces an algebra isomorphism between $\mathcal{K}_{\Omega}$ and $\mathcal{K}_{\Omega'}$. 
\end{corollary}

\begin{corollary}\cite[Corollary 5.6]{MR1088333} \label{FD2}
	The functor $\mathcal{F}_{\Omega,\Omega'}$ induces an equivalence of categories $\mathcal{Q}_{\mathbf{V},\Omega} \cong \mathcal{Q}_{\mathbf{V},\Omega'}$ and a bijection $\eta_{\Omega,\Omega'}:\mathcal{P}_{\Omega}\rightarrow \mathcal{P}_{\Omega'}$. Moreover, for orientations $\Omega,\Omega',\Omega''$, we have $\eta_{\Omega',\Omega''}\eta_{\Omega,\Omega'}=\eta_{\Omega,\Omega''}$. 
\end{corollary}

With two corollaries above, we denote $\mathcal{K}_{\Omega}$ by $\mathcal{K}$ and $\mathcal{P}_{\mathbf{V},\Omega}$ by $\mathcal{P}_{\mathbf{V}}$ respectively, if there is no ambiguity.



\subsection{Analysis at sink}
Fix $i \in I$ and an orientation $\Omega$ such that $i$ is a sink, for any $p\in \mathbb{N}$, we define $\mathbf{E}_{\mathbf{V},i,p}$ to be the locally closed subset of $\mathbf{E}_{\mathbf{V},\Omega}$ consisting of $x$ such that ${\rm{codim}}_{\mathbf{V}_{i}} ( {\rm{Im}} \bigoplus\limits_{h \in \Omega, h''=i} x_{h}) =p$. Then $\mathbf{E}_{\mathbf{V}}$ has a partition $\mathbf{E}_{\mathbf{V},\Omega}= \bigcup \limits_{p} \mathbf{E}_{\mathbf{V},i,p}$, and the union $\mathbf{E}_{\mathbf{V},i, \geq p}= \bigcup\limits_{p' \geq p} \mathbf{E}_{\mathbf{V},i, p'}$ is a closed subset of $\mathbf{E}_{\mathbf{V},\Omega}$.


Given $L \in \mathcal{P}_{\mathbf{V},\Omega}$, there exists a unique integer $t$ such that $\textrm{supp}(L) \subseteq \mathbf{E}_{\mathbf{V},i, \geq t}$ but $\textrm{supp}(L) \nsubseteq \mathbf{E}_{\mathbf{V},i, \geq t+1}$ and we set $t_{i}(L)=t$. Notice that $t_{i}(L) \leq \nu_{i}$.
 
 The following lemma is the key lemma in Lusztig's categorification theory.
 \begin{lemma} \cite[Lemma 6.4]{MR1088333} \label{lkey}
 	With the notation above, fix $0 \leq t \leq \nu_{i}$ and assume $|\mathbf{T}|=|\mathbf{V}'|=ti$.
 	
 	(1) Let $L\in\mathcal{P}_{\mathbf{V},\Omega}$ be such that $t_{i}(L)=t$, then $\mathbf{Res}^{\mathbf{V}}_{\mathbf{T},\mathbf{W}}(L) \in \mathcal{Q}_{\mathbf{W},\Omega}$ is a direct sum of finitely many summands of the form $K'[f']$ for various $K' \in \mathcal{P}_{\mathbf{W},\Omega}$ and $f' \in \mathbb{Z}$. Moreover, exactly one of these summands, denoted by $K[f]$, satisfies $t_{i}(K)=0$ and $f=0$ and the others satisfy $t_{i}(K')> 0$.
 	
 	(2) Let $K \in \mathcal{P}_{\mathbf{V}'',\Omega}$ be such that $t_{i}(K)=0$, then $\mathbf{Ind}^{\mathbf{V}}_{\mathbf{V}',\mathbf{V}''}(\bar{\mathbb{Q}}_{l} \boxtimes K)$ is a direct sum of finitely many summands of the form $L'[g']$ for various $L' \in \mathcal{P}_{\mathbf{V},\Omega}$ and $g' \in \mathbb{Z}$. Moreover, exactly one of these summands, denoted by $L[g]$, satisfies $t_{i}(L)=t$ and $g=0$ and the others satisfy $t_{i}(L')> t$.
 	
 	(3) There is a bijection $$\pi_{i,t}:\{K \in \mathcal{P}_{\mathbf{V}'',\Omega}|t_{i}(K)=0 \} \rightarrow \{L \in \mathcal{P}_{\mathbf{V},\Omega}|t_{i}(L)=t \}$$   induced by the decompositions of the direct sums above.
 \end{lemma}



If $|\mathbf{V}'|=ri$, we denote $\mathbf{V'}$ by $\mathbf{V}'_{r}$ and $\mathbf{V}''$ by $\mathbf{V}''_{r}$. For an orientation $\Omega'$ and $L \in \mathcal{P}_{\mathbf{V},\Omega'}$, we define $s_{i}(L)$ to be the largest integer $r$ satisfying that there exists $L' \in \mathcal{P}_{\mathbf{V}''_{r},\Omega'}$ such that $L$ is isomorphic to a shift of a direct summand of $\mathbf{Ind}^{\mathbf{V}}_{\mathbf{V}'_{r},\mathbf{V}''_{r}}(\bar{\mathbb{Q}}_{l} \boxtimes L')$. Notice that the definition of $s_{i}(L)$ does not depend on the choice of $\Omega'$ by Proposition \ref{FD0}.

\begin{proposition} \cite[Proposition 6.6]{MR1088333} \label{lt}
	With the notations above, we have:
	
	(1)  There exist $L'_{r'} \in \mathcal{P}_{\mathbf{V}''_{r'},\Omega'}$ for $r' > s_{i}(L)$ and $L''_{r'} \in  \mathcal{P}_{\mathbf{V}''_{r'},\Omega'}$ for $r' \geq s_{i}(L)$ such that 
	\begin{center}
		$L \oplus \bigoplus \limits_{r' >s_{i}(L)}\mathbf{Ind}^{\mathbf{V}}_{\mathbf{V}'_{r'},\mathbf{V}''_{r'}}(\bar{\mathbb{Q}}_{l} \boxtimes L'_{r'}) \cong \bigoplus \limits_{r' \geq s_{i}(L)}\mathbf{Ind}^{\mathbf{V}}_{\mathbf{V}'_{r'},\mathbf{V}''_{r'}}(\bar{\mathbb{Q}}_{l} \boxtimes L''_{r'}) .$
	\end{center}

   (2) $s_{i}(L)=t_{i}(L)$ if $i$ is a sink in $\Omega'$.

\end{proposition}


\subsection{Analysis at source}
 
 Fix $i \in I$ and an orientation $\Omega$ such that $i$ is a source, we define $\mathbf{E}_{\mathbf{V},i}^{p}$ to be the subset of $\mathbf{E}_{\mathbf{V},\Omega}$ consisting of $x$ such that ${\rm{dim}}( {\rm{Ker}} \bigoplus\limits_{h \in \Omega, h'=i} x_{h}) =p$. Then $\mathbf{E}_{\mathbf{V}}$ has a partition $\mathbf{E}_{\mathbf{V},\Omega}= \bigcup \limits_{p} \mathbf{E}_{\mathbf{V},i}^{p}$. and the union $\mathbf{E}_{\mathbf{V},i}^{\geq p}= \bigcup\limits_{p' \geq p} \mathbf{E}_{\mathbf{V},i}^{p'}$ is a closed subset.
 
 
 Given $L \in \mathcal{P}_{\mathbf{V},\Omega}$, there exists a unique integer $t$ such that $\textrm{supp}(L) \subseteq \mathbf{E}_{\mathbf{V},i}^{ \geq t}$ but $\textrm{supp}(L) \nsubseteq \mathbf{E}_{\mathbf{V},i}^{\geq t+1}$ and we write $t_{i}^{\ast}(L)=t$. Notice that $t_{i}^{\ast}(L) \leq \nu_{i}$.
 
 The following lemma is dual to Lemma \ref{lkey}.
 
 \begin{lemma}\label{rkey}
 	With the notation above, fix $0 \leq t \leq \nu_{i}$ and assume $|\mathbf{W}|=|\mathbf{V}''|=ti$.
 	
 	(1) Let $L\in\mathcal{P}_{\mathbf{V},\Omega}$ be such that $t_{i}^{\ast}(L)=t$, then $\mathbf{Res}^{\mathbf{V}}_{\mathbf{T},\mathbf{W}}(L) \in \mathcal{Q}_{\mathbf{T},\Omega}$ is a direct sum of finitely many summands of the form $K'[f']$ for various $K' \in \mathcal{P}_{\mathbf{T},\Omega}$ and $f' \in \mathbb{Z}$. Moreover, exactly one of these summands, denoted by $K[f]$, satisfies $t_{i}^{\ast}(K)=0$ and $f=0$ and the others satisfy $t_{i}^{\ast}(K')> 0$.
 	
 	(2) Let $K \in \mathcal{P}_{\mathbf{V}',\Omega}$ be such that $t_{i}^{\ast}(K)=0$, then $\mathbf{Ind}^{\mathbf{V}}_{\mathbf{V}',\mathbf{V}''}(K \boxtimes \bar{\mathbb{Q}}_{l} )$ is a direct sum of finitely many summands of the form $L'[g']$ for various $L' \in \mathcal{P}_{\mathbf{V},\Omega}$ and $g' \in \mathbb{Z}$. Moreover, exactly one of these summands, denoted by $L[g]$, satisfies $t_{i}^{\ast}(L)=t$ and $g=0$ and the others satisfy $t_{i}^{\ast}(L')> t$.
 	
 	(3) There is a bijection $$\pi^{\ast}_{i,t}:\{K \in \mathcal{P}_{\mathbf{V}',\Omega}|t_{i}^{\ast}(K)=0 \} \rightarrow  \{L \in \mathcal{P}_{\mathbf{V},\Omega}|t_{i}^{\ast}(L)=t \} $$  induced by the decompositions of the direct sums above.
 \end{lemma}

 
  In this section, if $|\mathbf{V}''|=ri$, we denote $\mathbf{V}'$ by $\mathbf{V}'_{r}$ and $\mathbf{V}''$ by $\mathbf{V}''_{r}$. For an orientation $\Omega'$ and $L \in \mathcal{P}_{\mathbf{V},\Omega' }$, we define $s_{i}^{\ast}(L)$ to be the largest integer $r$ satisfying that there exists $L' \in \mathcal{P}_{\mathbf{V}'_{r},\Omega'}$ such that $L$ is isomorphic to a direct summand of $\mathbf{Ind}^{\mathbf{V}}_{\mathbf{V}'_{r},\mathbf{V}''_{r}}(L' \boxtimes \bar{\mathbb{Q}}_{l})$. 
 
The following Proposition is dual to Proposition \ref{lt}.

 \begin{proposition}\label{rt}
 	(1)  There exist $L'_{r'} \in \mathcal{P}_{\mathbf{V''}_{r'},\Omega'}$ for $r' > s_{i}^{\ast}(L)$ and $L''_{r'} \in  \mathcal{P}_{\mathbf{V''}_{r'},\Omega'}$ for $r' \geq s_{i}^{\ast}(L)$ such that 
 	\begin{center}
 		$L \oplus \bigoplus \limits_{r' >s_{i}^{\ast}(L)}\mathbf{Ind}^{\mathbf{V}}_{\mathbf{V}'_{r'},\mathbf{V}''_{r'}}(L'_{r'} \boxtimes \bar{\mathbb{Q}}_{l}) \cong \bigoplus \limits_{r' \geq s_{i}^{\ast}(L)}\mathbf{Ind}^{\mathbf{V}}_{\mathbf{V}'_{r'},\mathbf{V}''_{r'}}(L''_{r'} \boxtimes \bar{\mathbb{Q}}_{l}). $
 	\end{center}
 	
 	(2) $s_{i}^{\ast}(L)=t_{i}^{\ast}(L)$ if $i$ is a source in $\Omega'$.
 	
 \end{proposition}

\section{Realization of the integrable highest weight modules}

Given a symmetric Cartan datum $(I,(-,-))$, we denote by $\alpha_{i}^{\vee}$ the simple coroot for $i\in I$. In this section, we fix a dominant weight $\Lambda$ and set $d_{i}=\langle \Lambda,\alpha_{i}^{\vee} \rangle\in \mathbb{N}$ for $i \in I$.

\subsection{Split semisimple category and its localization}

\begin{definition}
	Define the split semisimple category $\mathcal{Q}^{0}_{\mathbf{V}}$ to be the subcategory of $\mathcal{D}^{b}_{G_{\mathbf{V}}}(\mathbf{E}_{\mathbf{V},\Omega})$ whose objects are the same as $\mathcal{Q}_{\mathbf{V}}$ and 
	\begin{equation*}
		{\rm{Hom}}_{\mathcal{Q}^{0}_{\mathbf{V}}}(A[n],B[m])=  \left\{
		\begin{aligned}
			&	\overline{\mathbb{Q}}_{l},  & &n=m,A\cong B,\\
			&	0,  & &otherwise,
		\end{aligned}
		\right. 
	\end{equation*} 
	for $A,B \in \mathcal{P}_{\mathbf{V}}$ and $m,n \in \mathbb{Z}$.
	
	Similarly, for any variety $X$ with $G$-action, let $\mathcal{D}^{b,ss}_{G}(X)$ be the subcategory of $\mathcal{D}^{b}_{G}(X)$ consisting of semisimple complexes. Define the split semisimple category $\mathcal{D}^{b,0}_{G}(X)$ to be the subcategory of $\mathcal{D}^{b,ss}_{G}(X)$ whose objects are the same as $\mathcal{D}^{b,ss}_{G}(X)$ and 
	\begin{equation*}
		{\rm{Hom}}_{\mathcal{D}^{b,0}_{G}(X)}(A[n],B[m])=  \left\{
		\begin{aligned}
			&	\overline{\mathbb{Q}}_{l},  & &n=m,A\cong B, \\
			&	0,  & &otherwise,
		\end{aligned}
		\right. 
	\end{equation*} 
	for simple objects $A,B$ and $m,n \in \mathbb{Z}$.
	
	
\end{definition}

For any $i \in I$, we choose an orientation $\Omega^{i}$ such that $i$ is a source in $\Omega^{i}$. By section 2.4, there is a partition, $\mathbf{E}_{\mathbf{V},i}^{\geq d_{i}+1} \cup \mathbf{E}_{\mathbf{V},i}^{\leq d_{i}}= \mathbf{E}_{\mathbf{V},\Omega^{i}}$ where $\mathbf{E}_{\mathbf{V},i}^{\geq d_{i}+1}$ is a closed subset and $\mathbf{E}_{\mathbf{V},i}^{\leq d_{i}}$ is an open subset. Let $\mathcal{N}_{\mathbf{V},i}$ be the full subcategory of $\mathcal{D}^{b}_{G_{\mathbf{V}}}(\mathbf{E}_{\mathbf{V},\Omega^{i}})$ consisting of objects whose supports are contained in $\mathbf{E}_{\mathbf{V},i}^{\geq d_{i}+1}$, then $\mathcal{N}_{\mathbf{V},i}$ is a thick subcategory. We can see that the Verdier quotient $\mathcal{D}^{b}_{G_{\mathbf{V}}}(\mathbf{E}_{\mathbf{V},\Omega^{i}})/\mathcal{N}_{\mathbf{V},i}$ is a triangulated category,  with a natural $t$-structure induced from the $t$-structure of $\mathcal{D}^{b}_{G_{\mathbf{V}}}(\mathbf{E}_{\mathbf{V},\Omega^{i}})$.

\begin{definition}
	(a) For any $i\in I$ and orientation $\Omega^{i}$, define the localization (at $i$) of the split semisimple category $\mathcal{Q}^{0}_{\mathbf{V}}/\mathcal{N}_{\mathbf{V},i}$  to be the additive quotient of $\mathcal{Q}^{0}_{\mathbf{V}}$ by the subcategory consisting of objects in $\mathcal{N}_{\mathbf{V},i}$ .\\
	(b)For an orientation $\Omega$, define the global localization of the split semisimple category $\mathcal{Q}^{0}_{\mathbf{V}}/\mathcal{N}_{\mathbf{V}}$ to be the additive quotient of $\mathcal{Q}^{0}_{\mathbf{V}}$ by the subcategory consisting of objects in $ \mathcal{F}_{\Omega^{i},\Omega}( \mathcal{N}_{\mathbf{V},i}),i \in I$.
\end{definition}

\begin{remark}
Let $\varphi_i:\mathcal{D}^{b}_{G_{\mathbf{V}}}(\mathbf{E}_{\mathbf{V},\Omega^{i}}) \rightarrow \mathcal{D}^{b}_{G_{\mathbf{V}}}(\mathbf{E}_{\mathbf{V},\Omega^{i}})/\mathcal{N}_{\mathbf{V},i}$ be the natural functor, then it is easy to see that $\mathcal{Q}^{0}_{\mathbf{V}}/\mathcal{N}_{\mathbf{V},i}$ is equivalent to the image $\varphi_i(\mathcal{Q}^{0}_{\mathbf{V}})$.
\end{remark}


For any open embedding $j: U\rightarrow X$, the middle extension functor
\begin{equation*}
	j_{!\ast}:Perv(U) \rightarrow Perv(X)
\end{equation*}
 can be naturally extended to the split semisimple category. More precisely, for any direct sum of simple perverse sheaves up to shifts $L=\bigoplus\limits K[n]$, we set
\begin{equation*}
	j_{!\ast}(L)=\bigoplus\limits j_{!\ast}(K)[n].
\end{equation*}
In particular, for the open embedding $j_{\mathbf{V},i}: \mathbf{E}_{\mathbf{V},i}^{\leq d_{i}} \rightarrow \mathbf{E}_{\mathbf{V},\Omega^{i}}$, we have an addictive functor 
$$(j_{\mathbf{V},i})_{!\ast}: \mathcal{D}^{b,0}_{G_{\mathbf{V}},}(\mathbf{E}^{\leq d_{i} }_{\mathbf{V},i}) \rightarrow \mathcal{D}^{b,0}_{G_{\mathbf{V}},}(\mathbf{E}_{\mathbf{V},\Omega^{i}})$$ between the split semisimple categories.

\begin{lemma}\label{local}
	For any semisimple complex $L$ on $\mathbf{E}_{\mathbf{V},\Omega^{i}}$, we have $(j_{\mathbf{V},i})_{!\ast} (L) \cong (j_{\mathbf{V},i})_{!} (L)$  in $\mathcal{D}^{b}_{G_{\mathbf{V}}}(\mathbf{E}_{\mathbf{V},\Omega^{i}})/\mathcal{N}_{\mathbf{V},i}$. 
	In particular, the restriction of the functor $(j_{\mathbf{V},i})_{!\ast}$ to $(j_{\mathbf{V},i})^{\ast} (\mathcal{Q}^{0}_{\mathbf{V}})$ defines an equivalence of categories 
		\[
		\xymatrix{
			(j_{\mathbf{V},i})^{\ast} (\mathcal{Q}^{0}_{\mathbf{V}}) \ar@<0.5ex>[r]^{(j_{\mathbf{V},i})_{!\ast}} & \mathcal{Q}^{0}_{\mathbf{V}}/\mathcal{N}_{\mathbf{V},i} \ar@<0.5ex>[l]^{(j_{\mathbf{V},i})^{\ast}}
		}
		\]
		with the quasi-inverse $(j_{\mathbf{V},i})^{\ast}$.
\end{lemma}
\begin{proof}
	
	We only need to consider simple perverse sheaf $L$. Let $K=(j_{\mathbf{V},i})_{!\ast} (L) $, then there is a canonical triangle 
	\begin{equation*}
		(j_{\mathbf{V},i})_{!} (j_{\mathbf{V},i})^{\ast} (K)  \rightarrow K \rightarrow  i_{\ast}i^{\ast}(K) \rightarrow (j_{\mathbf{V},i})_{!} (j_{\mathbf{V},i})^{\ast} \mathbf{\Sigma}(K)
	\end{equation*}
	where $i:\mathbf{E}^{\geq d_{i}+1}_{\mathbf{V},i}\rightarrow \mathbf{E}_{\mathbf{V},\Omega^{i}}$ is the closed embedding. Notice that $i_{\ast}i^{\ast}(K)$ has support contained in $\mathbf{E}^{\geq d_{i}+1}_{\mathbf{V},i}$. Hence  $(j_{\mathbf{V},i})_{!} (j_{\mathbf{V},i})^{\ast} (K)$ is isomorphic to $K$ in the localization $\mathcal{D}^{b}_{G_{\mathbf{V}}}(\mathbf{E}_{\mathbf{V},\Omega^{i}})/\mathcal{N}_{\mathbf{V},i}$, and the first statement follows from $(j_{\mathbf{V},i})^{\ast} (K) \cong L$.
	
	Notice that the morphisms of split semisimple category are linear combinations of isomorphisms between simple perverse sheaves (up to shifts), we only need to deal with objects. It is obvious that $(j_{\mathbf{V},i})^{\ast}(j_{\mathbf{V},i})_{!\ast}\cong \textrm{Id}$. However, for simple object $ L'$ in $\mathcal{Q}^{0}_{\mathbf{V}}$, if the support of $L'$ is contained in $\mathbf{E}^{\geq d_{i}+1}_{\mathbf{V},i}$, then $L'$ is isomorphic to the zero object of $\mathcal{Q}^{0}_{\mathbf{V}}/\mathcal{N}_{\mathbf{V},i}$ and we also have  $(j_{\mathbf{V},i})^{\ast} (L') \cong 0$; otherwise, $(j_{\mathbf{V},i})^{\ast}(L')$ is a nonzero simple perverse sheaf and  $(j_{\mathbf{V},i})_{!\ast} (j_{\mathbf{V},i})^{\ast}(L') $ is isomorphic to $L'$. Hence $(j_{\mathbf{V},i})_{!\ast}$ and  $(j_{\mathbf{V},i})^{\ast}$ are equivalences.
\end{proof}

\subsection{The functor $\mathcal{E}_{i}^{(n)},\mathcal{F}_{j}^{(n)}$}
In this subsection, we fix an orientation $\Omega=\Omega^{i}$ such that $i$ is a source and set $d=d_{i}$.

For any $n\in \mathbb{N}$, take graded spaces $\mathbf{V}, \mathbf{V}'$ of dimension vector $\nu',\nu$ respectively, such that $\nu'+ni=\nu$, we will define varieties and morphisms appearing in following diagram, and then define a functor $\mathcal{E}_{i}^{(n)}$.
\[
\xymatrix{
	\mathbf{E}_{\mathbf{V},\Omega} 
	&
	& \mathbf{E}_{\mathbf{V}',\Omega} \\	
	\mathbf{E}_{\mathbf{V},i,d} \ar[u]^{\pi_{\mathbf{V},i}}
	&
	& \mathbf{E}_{\mathbf{V}',i,d} \ar[u]^{\pi_{\mathbf{V}',i}} \\
	\mathbf{E}^{0}_{\mathbf{V},i,d} \ar[d]_{\phi_{\mathbf{V},i}} \ar[u]^{j^{0}_{\mathbf{V},i}}
	&
	& \mathbf{E}^{0}_{\mathbf{V}',i,d} \ar[d]^{\phi_{\mathbf{V}',i}} \ar[u]_{j^{0}_{\mathbf{V}',i}} \\
	\dot{\mathbf{E}}_{\mathbf{V},i} \times \mathbf{Grass}(\nu_i, \tilde{\nu}_{i})
	& \dot{\mathbf{E}}_{\mathbf{V},i} \times \mathbf{Flag}(\nu_{i}-n,\nu_{i},\tilde{\nu}_{i}) \ar[r]^{q_{2}} \ar[l]_{q_{1}}
	& \dot{\mathbf{E}}_{\mathbf{V},i} \times \mathbf{Grass}(\nu_{i}-n, \tilde{\nu}_{i})
}
\]

Define the affine space 
\begin{equation*}
	\mathbf{E}_{\mathbf{V},i,d}=\{(x,f)| x \in \mathbf{E}_{\mathbf{V},\Omega}, f: \mathbf{V}_{i} \rightarrow \mathbb{C}^{d} \textrm{ is a linear map} \},
\end{equation*}
then the natural first projection $\pi_{\mathbf{V},i}:\mathbf{E}_{\mathbf{V},i,d} \rightarrow  \mathbf{E}_{\mathbf{V},\Omega}$ is a vector bundle of rank $d\cdot\dim \mathbf{V_i}$. Let
\begin{equation*}
	\mathbf{E}^{0}_{\mathbf{V},i,d}=\{x \in \mathbf{E}_{\mathbf{V},i,d}| {\rm{dim}}\ {\rm{Ker}}( \bigoplus\limits_{h \in \Omega, h'=i} x_{h}\oplus f: \mathbf{V}_{i} \rightarrow \bigoplus\limits_{h'=i} \mathbf{V}_{h''} \oplus \mathbb{C}^{d})=0 \}\subset \mathbf{E}_{\mathbf{V},i,d}
\end{equation*}
be an open subset and let $j^{0}_{\mathbf{V},i}:\mathbf{E}^{0}_{\mathbf{V},i,d} \rightarrow \mathbf{E}_{\mathbf{V},i,d}$ be the open embedding. Define an affine subspace
\begin{equation*}
	\dot{\mathbf{E}}_{\mathbf{V},i} =\bigoplus\limits_{h \in \Omega, h'\neq i} \mathbf{Hom}(\mathbf{V}_{h'},\mathbf{V}_{h''}) .
\end{equation*}
For any $x \in \mathbf{E}_{\mathbf{V},\Omega}$, we denote by $\dot{x}=(x_{h})_{h\in \Omega,h' \neq i}$, then there is a morphism 
	\begin{align*}
		\phi_{\mathbf{V},i}:\mathbf{E}^{0}_{\mathbf{V},i,d} &\rightarrow  \dot{\mathbf{E}}_{\mathbf{V},i} \times \mathbf{Grass}(\nu_i, \tilde{\nu}_{i})\\
		(x,f)&\mapsto (\dot{x}, {\rm{Im}}  (\bigoplus \limits_{h \in \Omega, h'=i} x_{h} \oplus f ) ),
\end{align*}
where $\nu_{i}={\rm{dim}}\mathbf{V}_{i}$, $\tilde{\nu}_{i}=\sum \limits_{h'=i}{\rm{dim}}\mathbf{V}_{h''}+d_{i}$, and $\mathbf{Grass}(\nu_i, \tilde{\nu}_{i})$ is the Grassmannian consisting of $\nu_{i}$-dimensional subspaces of $\tilde{\nu}_{i}$-dimensional space $(\bigoplus\limits_{h'=i}\mathbf{V}_{h''})\oplus \mathbb{C}^{d}$. We can check by definition that $\phi_{\mathbf{V},i}$ is a principal $\mathbf{GL}(\mathbf{V}_{i})$-bundle. Let  
	\begin{equation*}
		\mathbf{Flag}(\nu_{i}-n,\nu_{i},\tilde{\nu}_{i})=\{ \mathbf{U}_{1}\subset \mathbf{U}_{2}  \subset (\bigoplus\limits_{h'=i}\mathbf{V}_{h''})\oplus \mathbb{C}^{d} )|{\rm{dim}} \mathbf{U}_{1} = \nu_{i}-n, {\rm{dim}}\mathbf{U}_{2}=\nu_{i}  \}.
	\end{equation*}
	be the flag variety and $q_{1},q_{2}$ are natural projections
\begin{equation*}
	q_{1}(\dot{x}, \mathbf{U}_{1},\mathbf{U}_{2})=(\dot{x},\mathbf{U}_{2});
\end{equation*}
\begin{equation*}
	q_{2}(\dot{x},\mathbf{U}_{1},\mathbf{U}_{2})=(\dot{x},\mathbf{U}_{1}).
\end{equation*}

\begin{definition}
Define the functor $\mathcal{E}^{(n)}_{i}:\mathcal{D}^{b}_{G_{\mathbf{V}}}(\mathbf{E}_{\mathbf{V},\Omega}) \rightarrow \mathcal{D}^{b}_{G_{\mathbf{V}'}}(\mathbf{E}_{\mathbf{V}',\Omega})$ via
	\begin{equation*}
		\mathcal{E}^{(n)}_{i}=(\pi_{\mathbf{V}',i})_{!} (j^{0}_{\mathbf{V}',i})_{!} (\phi_{\mathbf{V}',i})^{\ast} (q_{2})_{!}(q_{1})^{\ast} (\phi_{\mathbf{V},i})_{\flat}(j^{0}_{\mathbf{V},i})^{\ast}(\pi_{\mathbf{V},i})^{\ast}[2\nu_{i}d_{i}-n(\nu_{i}+d_{i})]
	\end{equation*}
Notice  that $\mathcal{E}^{(n)}_{i}(\mathcal{N}_{\mathbf{V},i} )=0$, thus $\mathcal{E}^{(n)}_{i}$ descends to a functor between localizations, still denoted by 
	\begin{equation*}
		\mathcal{E}^{(n)}_{i}:\mathcal{D}^{b}_{G_{\mathbf{V}}}(\mathbf{E}_{\mathbf{V},\Omega})/\mathcal{N}_{\mathbf{V},i} \rightarrow \mathcal{D}^{b}_{G_{\mathbf{V}'}}(\mathbf{E}_{\mathbf{V}',\Omega})/\mathcal{N}_{\mathbf{V'},i}.
	\end{equation*}
	In particular, we denote by $\mathcal{E}_{i}=\mathcal{E}_{i}^{(1)}. $
\end{definition} 
We will prove that the functor $\mathcal{E}_{i}$ induces a functor between localizations in Corollary \ref{commute3}.

	For $j \in I$, we take graded spaces $\mathbf{V},\mathbf{V}'$  and $\mathbf{V}''$ such that  $|\mathbf{V}''|-nj=|\mathbf{V}|,|\mathbf{V}'|=nj$ and define the functor $\mathcal{F}_{j}^{(n)}:\mathcal{D}^{b}_{G_{\mathbf{V}}}(\mathbf{E}_{\mathbf{V},\Omega}) \rightarrow \mathcal{D}^{b}_{G_{\mathbf{V}''}}(\mathbf{E}_{\mathbf{V}'',\Omega}) $ (or $\mathcal{F}_{j}^{(n)}:\mathcal{Q}^{0}_{\mathbf{V}} \rightarrow \mathcal{Q}^{0}_{\mathbf{V}''} $ ) via
		\begin{equation*}
		\mathcal{F}_{j}^{(n)}= \mathbf{Ind}^{\mathbf{V}''}_{\mathbf{V}',\mathbf{V}}(\overline{\mathbb{Q}}_{l} \boxtimes -)
	\end{equation*}
where $\overline{\mathbb{Q}}_{l}$ is the constant sheaf of $\mathbf{E}_{\mathbf{V}',\Omega}$.


\begin{lemma}
	The functor $\mathcal{F}^{(n)}_{j}$ sends objects of $\mathcal{Q}^{0}_{\mathbf{V}} \cap \mathcal{N}_{\mathbf{V},k}$ to objects of $\mathcal{Q}^{0}_{\mathbf{V}''} \cap \mathcal{N}_{\mathbf{V}'',k}$.
\end{lemma}
\begin{proof}
	By Proposition \ref{rt}, we can see that objects of $\mathcal{Q}^{0}_{\mathbf{V}} \cap \mathcal{N}_{\mathbf{V},k}$ appear as direct summands of $L_{(\underline{v},k^{m})}$ for some $\underline{v}$ and $m > d_{k}$. Notice that $\mathcal{F}_{j}^{(n)}(L_{(\underline{v},k^{m})} ) =L_{(j^{n},\underline{v},k^{m})}  $ belongs to $\mathcal{Q}^{0}_{\mathbf{V}''} \cap \mathcal{N}_{\mathbf{V}'',k}$, we get a proof.
\end{proof}

By this lemma, we can see that the functor $\mathcal{F}_{j}$ induces a functor between localizations.

\begin{definition}
	For $j \in I$, we take graded spaces $\mathbf{V},\mathbf{V}'$  and $\mathbf{V}''$ such that  $|\mathbf{V}''|-nj=|\mathbf{V}|,|\mathbf{V}'|=nj$ and define the functor $\mathcal{F}_{j}^{(n)}:\mathcal{Q}^{0}_{\mathbf{V}}/\mathcal{N}_{\mathbf{V},i} \rightarrow \mathcal{Q}^{0}_{\mathbf{V}''}/\mathcal{N}_{\mathbf{V}'',i}$ (or $\mathcal{F}_{j}^{(n)}:\mathcal{Q}^{0}_{\mathbf{V}}/\mathcal{N}_{\mathbf{V}} \rightarrow \mathcal{Q}^{0}_{\mathbf{V}''}/\mathcal{N}_{\mathbf{V}''} $ ) via
	\begin{equation*}
		\mathcal{F}_{j}^{(n)}= \mathbf{Ind}^{\mathbf{V}''}_{\mathbf{V}',\mathbf{V}}(\overline{\mathbb{Q}}_{l} \boxtimes -)
	\end{equation*}
	where $\overline{\mathbb{Q}}_{l}$ is the constant sheaf of $\mathbf{E}_{\mathbf{V}',\Omega}$. In particular, we set $\mathcal{F}_{j}=\mathcal{F}_{j}^{(1)}. $
\end{definition}

The functor $\mathcal{F}_{j}$ is defined by Lusztig's induction functor. It can be described by functors induced by morphisms appearing in the definition of $\mathcal{E}_i$ as follows.

Take graded spaces $\mathbf{V}$ and $\mathbf{V}''$ such that  $|\mathbf{V}|+j=|\mathbf{V}''|$. Let $ \mathbf{E}^{''}_{\mathbf{V}'',i,d} $ be the variety consisting of triples $(x,f,\tilde{\mathbf{W}})$ such that $(x,f) \in \mathbf{E}_{\mathbf{V}'',\Omega} $  and $(x,\tilde{\mathbf{W}}) \in \mathbf{E}^{''}_{\Omega}$. And let $ \mathbf{E}^{'}_{\mathbf{V}'',i,d} $ be the variety consisting of triples $(x,f,\tilde{\mathbf{W}},\rho)$ such that $(x,f) \in \mathbf{E}_{\mathbf{V}'',\Omega},  (x,\tilde{\mathbf{W}}) \in \mathbf{E}^{''}_{\Omega}$ and $\rho: \tilde{\mathbf{W}} \cong \mathbf{V}$ is a linear isomorphism. Then we have the following commutative diagram
\[
\xymatrix{
	\mathbf{E}_{\mathbf{V},\Omega} & \mathbf{E}^{'}_{\Omega}\ar[l]_{p_{1}} \ar[r]^{p_{2}}& \mathbf{E}^{''}_{\Omega} \ar[r]^{p_{3}}& \mathbf{E}_{\mathbf{V}'',\Omega} \\
	\mathbf{E}_{\mathbf{V},i,d} \ar[u]_{\pi_{1}} & \mathbf{E}^{'}_{\mathbf{V}'',i,d} \ar[l]_{p'_{1}} \ar[r]^{p'_{2}} \ar[u]_{\pi_{2}} & \mathbf{E}^{''}_{\mathbf{V}'',i,d}  \ar[r]^{p'_{3}} \ar[u]_{\pi_{3}} & \mathbf{E}_{\mathbf{V}'',i,d} \ar[u]_{\pi_{4}} 
}
\]
where $\pi_{i},i=1,2,3,4$ are vector bundles forgetting the component $f$ and $p'_{1},p'_{2},p'_{3}$ are obvious morphisms induced from $p_{1},p_{2},p_{3}$ respectively. Notice that the right square is Cartesian, we have 
\begin{equation}\label{Eq1}
(\pi_{4})^{\ast}(p_{3})_{!}(p_{2})_{\flat}(p_{1})^{\ast} =(p'_{3})_{!}(p'_{2})_{\flat}(p'_{1})^{\ast} (\pi_{1})^{\ast}.
\end{equation}
We set $\mathbf{E}^{'',0}_{\mathbf{V}'',i,d}=(p'_{3})^{-1}(\mathbf{E}^{0}_{\mathbf{V}'',i,d})$ and $\mathbf{E}^{',0}_{\mathbf{V}'',i,d}=(p'_{2})^{-1}(\mathbf{E}^{'',0}_{\mathbf{V}'',i,d})$, then we have the following commutative diagram
\[
\xymatrix{
	\mathbf{E}_{\mathbf{V},i,d} & \mathbf{E}^{'}_{\mathbf{V}'',i,d}\ar[l]_{p'_{1}} \ar[r]^{p'_{2}}& \mathbf{E}^{''}_{\mathbf{V}'',i,d} \ar[r]^{p'_{3}}& \mathbf{E}_{\mathbf{V}'',i,d} \\
	\mathbf{E}^{0}_{\mathbf{V},i,d} \ar[u]_{j_{1}} & \mathbf{E}^{',0}_{\mathbf{V}'',i,d} \ar[l]_{\tilde{p}_{1}} \ar[r]^{\tilde{p}_{2}} \ar[u]_{j_{2}} & \mathbf{E}^{'',0}_{\mathbf{V}'',i,d}  \ar[r]^{\tilde{p}_{3}} \ar[u]_{j_{3}} & \mathbf{E}^{0}_{\mathbf{V}'',i,d} \ar[u]_{j_{4}} 
}
\]
where $j_{i},i=1,2,3,4$ are open embeddings  $\tilde{p}_{1},\tilde{p}_{2},\tilde{p}_{3}$ are obvious morphisms induced from $p'_{1},p'_{2},p'_{3}$ respectively. Notice that the right square is Cartesian, we have 
\begin{equation}
(j_{4})^{\ast}(p'_{3})_{!}(p'_{2})_{\flat}(p'_{1})^{\ast} =(\tilde{p}_{3})_{!}(\tilde{p}_{2})_{\flat}(\tilde{p}_{1})^{\ast} (j_{1})^{\ast}.
\end{equation}

\textbf{Case (I)} $i=j$. Consider the following commutative diagram
\[
\xymatrix{
	\mathbf{E}^{0}_{\mathbf{V},i,d} \ar[d]_{\phi_{1}} & \mathbf{E}^{',0}_{\mathbf{V}'',i,d} \ar[l]_{\tilde{p}_{1}} \ar[r]^{\tilde{p}_{2}} \ar[d]_{\phi_{2}}& \mathbf{E}^{'',0}_{\mathbf{V}'',i,d}  \ar[r]^{\tilde{p}_{3}}  \ar[d]_{\phi_{3}}& \mathbf{E}^{0}_{\mathbf{V}'',i,d} \ar[d]_{\phi_{4}} \\
	\txt{$\dot{\mathbf{E}}_{\mathbf{V},i}$\\ $\times$\\ $\mathbf{Grass}(\nu_{i}, \tilde{\nu}_{i})$}  & \txt{$\dot{\mathbf{E}}_{\mathbf{V},i}$\\ $\times$\\ $\mathbf{Flag}(\nu_{i},\nu''_{i},\tilde{\nu}_{i})$} \ar[l]_{q_{1}'} \ar@2{-}[r]  & \txt{$\dot{\mathbf{E}}_{\mathbf{V},i}$\\  $\times$\\  $\mathbf{Flag}(\nu_{i},\nu''_{i},\tilde{\nu}_{i})$}  \ar[r]^{q_{2}'}  & \txt{$\dot{\mathbf{E}}_{\mathbf{V}'',i}$\\ $\times$\\$\mathbf{Grass}(\nu''_{i}, \tilde{\nu}_{i}$)}
}
\]
where
\begin{equation*}
	q_{1}'((\dot{x},\mathbf{U}_{1}\subset \mathbf{U}_{2} ))=(\dot{x},\mathbf{U}_{1}), 
\end{equation*}
\begin{equation*}
	q_{2}'((\dot{x},\mathbf{U}_{1}\subset \mathbf{U}_{2} ))=(\dot{x},\mathbf{U}_{2}),
\end{equation*}
\begin{equation*}
	\phi_{1}=\phi_{\mathbf{V},i},\  \phi_{2}=\tilde{\phi}\circ \tilde{p}_{2},\ \phi_{3}=\tilde{\phi},\ \phi_{4}=\phi_{\mathbf{V}'',i}
\end{equation*}
and $\tilde{\phi}: \mathbf{E}^{'',0}_{\mathbf{V}'',i,d} \rightarrow \dot{\mathbf{E}}_{\mathbf{V}'',i} \times \mathbf{Flag}(\nu_{i},\nu''_{i},\tilde{\nu}_{i})$ is defined by
\begin{equation*}
	\tilde{\phi}((x,f,\tilde{\mathbf{W}}))=(\dot{x},{\rm{Im}}  (f \oplus \bigoplus \limits_{h \in \Omega, h'=i}  x_{h})|_{\tilde{\mathbf{W}}}) \subset {\rm{Im}}  (f \oplus \bigoplus \limits_{h \in \Omega, h'=i} x_{h} )  \subset (\bigoplus\limits_{h'=i}\mathbf{V}_{h''})\oplus \mathbb{C}^{d} ).
\end{equation*}
Notice that the right square is Cartesian, we have 
\begin{equation}
(q'_{2})_{!}(q'_{1})^{\ast}(\phi_{1})_{\flat} =(\phi_{4})_{\flat}(\tilde{p}_{3})_{!}(\tilde{p}_{2})_{\flat}(\tilde{p}_{1})^{\ast}.
\end{equation}

\textbf{Case (II)} $i \neq j$ and $i,j$ are connected by some edge. Let
 $Z_{1}$ be the variety consisting of quadruples $(\dot{x},\dot{\mathbf{U}} \subset \dot{\mathbf{V}}'',\dot{\rho},\tilde{\mathbf{V}}''\subset  \bigoplus\limits_{h'=i} \mathbf{U}_{h''} \oplus \mathbb{C}^{d})$ such that $\dot{\mathbf{U}}$ is a $\dot{x}$-stable subspace of $\dot{\mathbf{V}}''$, $\dot{\rho}: \dot{\mathbf{U}} \rightarrow \dot{\mathbf{V}}$ is a linear isomorphism and $\tilde{\mathbf{V}}''$ is a subspace of dimension $\nu_{i}$. And let $Z_{2}$ be the variety consisting of triples $(\dot{x},\dot{\mathbf{U}} \subset \dot{\mathbf{V}},\tilde{\mathbf{V}}''\subset  \bigoplus\limits_{h'=i} \mathbf{U}_{h''} \oplus \mathbb{C}^{d} )$ such that $\dot{\mathbf{U}}$ is a $\dot{x}$-invariant subspace of $\dot{\mathbf{V}}''$ and $\tilde{\mathbf{V}}''$ is subspace of dimension $\nu_{i}$. Consider the following commutative diagram
 \[
 \xymatrix{
 	\mathbf{E}^{0}_{\mathbf{V},i,d}\ar[d]_{\phi_{1}} & \mathbf{E}^{',0}_{\mathbf{V}'',i,d} \ar[l]_{\tilde{p}_{1}} \ar[r]^{\tilde{p}_{2}} \ar[d]_{\phi_{2}}& \mathbf{E}^{'',0}_{\mathbf{V}'',i,d}  \ar[r]^{\tilde{p}_{3}} \ar[d]_{\phi_{3}} & \mathbf{E}^{0}_{\mathbf{V}'',i,d}  \ar[d]_{\phi_{4}} \\
 	\dot{\mathbf{E}}_{\mathbf{V},i} \times \mathbf{Grass}(\nu_{i}, \tilde{\nu}_{i})  & Z_{1} \ar[l]_-{q'_{1}} \ar[r]^{q'_{2}}  & Z_{2} \ar[r]^-{q_{3}'}  & \dot{\mathbf{E}}_{\mathbf{V}'',i} \times \mathbf{Grass}(\nu_{i}, \tilde{\nu}_{i}+1)
 }
 \]
 where 
 \begin{align*}
 	&q'_{1}(\dot{x},\dot{\mathbf{U}},\dot{\rho},\tilde{\mathbf{V}}'')=((\dot{\rho})_{\ast}(\dot{x}), \dot{\rho}(\tilde{\mathbf{V}}'')),\\
	 &q'_{2}(\dot{x},\dot{\mathbf{U}},\dot{\rho},\tilde{\mathbf{V}}'')=(\dot{x},\dot{\mathbf{U}},\tilde{\mathbf{V}}''),\\
	 &q'_{3}((\dot{x},\dot{\mathbf{U}},\tilde{\mathbf{V}}''))=(\dot{x},\tilde{\mathbf{V}}'')\\
	 &\phi_{1}=\phi_{\mathbf{V},i},\\
 &\phi_{2}((x,\mathbf{U},\rho))=(\dot{x},\dot{\mathbf{U}},\dot{\rho},{\rm{Im}}  (f \oplus \bigoplus \limits_{h \in \Omega, h'=i} x_{h}) ),\\
&\phi_{3}((x,\mathbf{U}))=(\dot{x},\dot{\mathbf{U}},{\rm{Im}}  (f\oplus \bigoplus \limits_{h \in \Omega, h'=i} x_{h})),\\
&\phi_{4}=\phi_{\mathbf{V}'',i}.
 \end{align*}
Notice that the right square is Cartesian, we have 
\begin{equation}
(q'_{3})_{!}(q'_{2})_{\flat}(q'_{1})^{\ast}(\phi_{1})_{\flat} =(\phi_{4})_{\flat}(\tilde{p}_{3})_{!}(\tilde{p}_{2})_{\flat}(\tilde{p}_{1})^{\ast}.
\end{equation}

\textbf{Case (III)} $i \neq j$ and there is no edge connecting $i,j$. Let
$Z_{1}$ be the variety consisting of quadruples $(\dot{x},\dot{\mathbf{U}} \subset \dot{\mathbf{V}}'',\dot{\rho},\tilde{\mathbf{V}}''\subset  \bigoplus\limits_{h'=i} \mathbf{U}_{h''} \oplus \mathbb{C}^{d})$ such that $\dot{\mathbf{U}}$ is a $\dot{x}$-stable subspace of $\dot{\mathbf{V}}''$, $\dot{\rho}: \dot{\mathbf{U}} \rightarrow \dot{\mathbf{V}}$ is a linear isomorphism and $\tilde{\mathbf{V}}''$ is a subspace of dimension $\nu_{i}$. And let $Z_{2}$ be the variety which consists of $(\dot{x},\dot{\mathbf{U}} \subset \dot{\mathbf{V}},\tilde{\mathbf{V}}''\subset  \bigoplus\limits_{h'=i} \mathbf{U}_{h''} \oplus \mathbb{C}^{d} )$ such that $\dot{\mathbf{U}}$ is a $\dot{x}$-invariant subspace of $\dot{\mathbf{V}}''$ and $\tilde{\mathbf{V}}''$ is a subspace of dimension $\nu_{i}$. Consider the following commutative diagram
\[
\xymatrix{
\mathbf{E}^{0}_{\mathbf{V},i,d}\ar[d]_{\phi_{1}} & \mathbf{E}^{',0}_{\mathbf{V}'',i,d} \ar[l]_{\tilde{p}_{1}} \ar[r]^{\tilde{p}_{2}} \ar[d]_{\phi_{2}}& \mathbf{E}^{'',0}_{\mathbf{V}'',i,d}  \ar[r]^{\tilde{p}_{3}} \ar[d]_{\phi_{3}} & \mathbf{E}^{0}_{\mathbf{V}'',i,d}  \ar[d]_{\phi_{4}} \\
	\dot{\mathbf{E}}_{\mathbf{V},i} \times \mathbf{Grass}(\nu_{i}, \tilde{\nu}_{i})  & Z_{1} \ar[l]_-{q'_{1}} \ar[r]^{q'_{2}}  & Z_{2} \ar[r]^-{q_{3}'}  & \dot{\mathbf{E}}_{\mathbf{V}'',i} \times \mathbf{Grass}(\nu_{i}, \tilde{\nu}_{i})
}
\]
where $q'_1,q'_2,q'_3,\phi_1,\phi_2,\phi_3,\phi_4$ are given by the same formulas in \textbf{Case (II)} above.
Notice that the right square is Cartesian, we have 
\begin{equation}\label{Eq5}
(q'_{3})_{!}(q'_{2})_{\flat}(q'_{1})^{\ast}(\phi_{1})_{\flat} =(\phi_{4})_{\flat}(\tilde{p}_{3})_{!}(\tilde{p}_{2})_{\flat}(\tilde{p}_{1})^{\ast}.
\end{equation}
 
 \subsection{Commutative relations in localizations at $i$}
At the beginning of this section, we recall two lemmas in algebraic geometry.

\begin{lemma}\label{projection formula}
For any morphism $f:X\rightarrow Y$ and any complex $A$ on $Y$, we have $f_{!}f^{\ast}A \cong f_{!}\overline{\mathbb{Q}}_{l} \otimes A$.
\end{lemma}	
\begin{proof}
By the projection formula, see \cite[Theorem 1.4.9]{MR4337423}, we have 
$$f_{!}\overline{\mathbb{Q}}_{l} \otimes A\cong f_{!}(\overline{\mathbb{Q}}_{l}\otimes f^*A)\cong f_{!}f^{\ast}(A),$$ 
as desired.
\end{proof}


\begin{lemma}[\cite{MR1227098}, 8.1.6]\label{Lusztig BBD}
	If $X =\coprod\limits_{n} X_{n}$ is a partition such that for any $n$, $X^{\leq n}=\coprod\limits_{m \leq n }X_{m} $ is closed and the restriction $f_{n}$ of $f$ on $X_{n}$ can be decomposed as
	\begin{equation*}
		X_{n} \xrightarrow{g_{n}} Z_{n} \xrightarrow{h_{n}} Y
	\end{equation*}	
	where $Z_{n}$ is smooth, $g_{n}$ is a vector bundle of rank $d_{n}$ and $h_{n}$ is proper, then
	\begin{equation*}
		f_{!}(\overline{\mathbb{Q}}_{l}|_{X}) \cong \bigoplus\limits_{n} (f_{n})_{!} (\overline{\mathbb{Q}}_{l}|_{X_n})\cong \bigoplus\limits_{n} (h_{n})_{!} (\overline{\mathbb{Q}}_{l}|_{Z_n})[-2d_{n}].
	\end{equation*}	
\end{lemma}

Let $\tilde{\mathcal{N}}_{\mathbf{V},i}$ be the subcategory of $\mathcal{D}^{b}_{G_{\mathbf{V}}}(\mathbf{E}_{\mathbf{V},i,d})$ consisting of objects whose supports are contained in the complement of $\mathbf{E}_{\mathbf{V},i,d}^{0}$, then $\tilde{\mathcal{N}}_{\mathbf{V},i}$ is a thick subcategory, and we can define the localization $\mathcal{D}^{b}_{G_{\mathbf{V}}}(\mathbf{E}_{\mathbf{V},i,d})/\tilde{\mathcal{N}}_{\mathbf{V},i} $ and the additive quotient $ (\pi_{\mathbf{V},i})^{\ast}(\mathcal{Q}^{0}_{\mathbf{V}}) /\tilde{\mathcal{N}}_{\mathbf{V},i}$.

For graded spaces $\mathbf{V}$ and $\mathbf{V}'$ such that  $|\mathbf{V}'|+ni=|\mathbf{V}|$, consider the functor $\tilde{\mathcal{E}}^{(n)}_{i}:\mathcal{D}^{b}_{G_{\mathbf{V}}}(\mathbf{E}_{\mathbf{V},i,d}) \rightarrow \mathcal{D}^{b}_{G_{\mathbf{V}'}}(\mathbf{E}_{\mathbf{V}',i,d})$ defined by
\begin{equation*}
	\tilde{\mathcal{E}}^{(n)}_{i}=  (j^{0}_{\mathbf{V}',i})_{!} (\phi_{\mathbf{V}',i})^{\ast} (q_{2})_{!}(q_{1})^{\ast} (\phi_{\mathbf{V},i})_{\flat}(j^{0}_{\mathbf{V},i})^{\ast}[-n\nu_{i}].
\end{equation*}
In particular, we denote $\tilde{\mathcal{E}}^{(1)}_{i}$ by $\tilde{\mathcal{E}}_{i} $.

For graded spaces $\mathbf{V}$ and $\mathbf{V''}$ such that $|
\mathbf{V}|=|\mathbf{V}''|+j$, consider the functor $\tilde{\mathcal{F}}_{j}:\mathcal{D}^{b}_{G_{\mathbf{V}''}}(\mathbf{E}_{\mathbf{V}'',i,d})\rightarrow \mathcal{D}^{b}_{G_{\mathbf{V}}}(\mathbf{E}_{\mathbf{V},i,d})$ defined by
\begin{equation*}
	\tilde{\mathcal{F}}_{j}=\begin{cases}(j^{0}_{\mathbf{V},i})_{!\ast} (\phi_{\mathbf{V},i})^{\ast} (q'_{2})_{!}(q'_{1})^{\ast} (\phi_{\mathbf{V}'',i})_{\flat}(j^{0}_{\mathbf{V}'',i})^{\ast}[\tilde{\nu}_{i}-\nu_{i}+1], i=j,\\(j^{0}_{\mathbf{V},i})_{!\ast} (\phi_{\mathbf{V},i})^{\ast} (q'_{3})_{!} (q'_{2})_{\flat}(q'_{1})^{\ast} (\phi_{\mathbf{V}'',i})_{\flat}(j^{0}_{\mathbf{V}'',i})^{\ast}[\tilde{\nu}_{i}+\nu_{i}], i\neq j.\end{cases}
\end{equation*}

By definitions and formulas (\ref{Eq1})-(\ref{Eq5}), we have
\begin{align}\label{tilde E and E}
\tilde{\mathcal{E}}_{i}^{(n)} ((\pi_{\mathbf{V},i})^{\ast}[\nu_{i}d_{i}] )= ((\pi_{\mathbf{V}',i})^{\ast}[\nu'_{i}d_{i}] )\mathcal{E}_{i}^{(n)},
\end{align}
\begin{align}\label{tilde F and F}
\tilde{\mathcal{F}}_{j} ((\pi_{\mathbf{V}'',i})^{\ast}[\nu_{i}''d_{i}] )= ((\pi_{\mathbf{V},i})^{\ast}[\nu_{i}d_{i}] )\mathcal{F}_{j}. 
\end{align}

Note that $\tilde{\mathcal{E}}^{(n)}_{i}(\tilde{\mathcal{N}}_{\mathbf{V},i})=0, \tilde{\mathcal{F}}_{j}(\tilde{\mathcal{N}}_{\mathbf{V},i})=0$, thus they induce functors
\begin{align*}
&\tilde{\mathcal{E}}^{(n)}_{i}:\mathcal{D}^{b}_{G_{\mathbf{V}}}(\mathbf{E}_{\mathbf{V},i,d})/\tilde{\mathcal{N}}_{\mathbf{V},i} \rightarrow \mathcal{D}^{b}_{G_{\mathbf{V}'}}(\mathbf{E}_{\mathbf{V}',i,d})/\tilde{\mathcal{N}}_{\mathbf{V}',i},\\
&\tilde{\mathcal{F}}_{j}:\mathcal{D}^{b}_{G_{\mathbf{V}''}}(\mathbf{E}_{\mathbf{V}'',i,d})/\tilde{\mathcal{N}}_{\mathbf{V}'',i} \rightarrow \mathcal{D}^{b}_{G_{\mathbf{V}}}(\mathbf{E}_{\mathbf{V},i,d})/\tilde{\mathcal{N}}_{\mathbf{V},i}.
\end{align*}



\begin{lemma} \label{lemma c1}
	For any graded space $\mathbf{V},\mathbf{V}'$ such that $|\mathbf{V}'|+(n-1)i=|\mathbf{V}| $ and any simple perverse sheaf $L$ on $\mathbf{E}_{\mathbf{V},i,d}$ such that $(j^{0}_{\mathbf{V},i})^{\ast}(L) \neq 0$, then there is an isomorphism in the localization $\mathcal{D}^{b}_{G_{\mathbf{V}'}}(\mathbf{E}_{\mathbf{V}',i,d})/\tilde{\mathcal{N}}_{\mathbf{V}',i}$,
	\begin{equation*}
		\tilde{\mathcal{E}}^{(n)}_{i}\tilde{\mathcal{F}}_{i}(L) \oplus \bigoplus\limits_{0\leqslant m \leqslant N-1} \tilde{\mathcal{E}}^{(n-1)}_{i}(L)[N-1-2m] \cong \tilde{\mathcal{F}}_{i}\tilde{\mathcal{E}}^{(n-1)}_{i}(L) \oplus \bigoplus\limits_{0\leqslant m \leqslant -N-1} \tilde{\mathcal{E}}^{(n)}_{i}(L)[-2m-N-1],
	\end{equation*}
	where $N=2\nu_{i}-\tilde{\nu}_{i}-n+1$. More precisely,
	\begin{align*}
		&\tilde{\mathcal{E}}^{(n)}_{i}\tilde{\mathcal{F}}_{i}(L)\cong \tilde{\mathcal{F}}_{i}\tilde{\mathcal{E}}^{(n)}_{i}(L), &\textrm{if}\ N=0;\\
		&\tilde{\mathcal{E}}^{(n)}_{i}\tilde{\mathcal{F}}_{i}(L)\oplus\bigoplus_{0\leqslant m\leqslant N-1}\tilde{\mathcal{E}}^{(n-1)}_{i}(L)[N-1-2m]\cong \tilde{\mathcal{F}}_{i}\tilde{\mathcal{E}}_{i}(L),&\textrm{if}\  N\geqslant 1;\\
		&\tilde{\mathcal{E}}^{(n)}_{i}\tilde{\mathcal{F}}_{i}(L)\cong \tilde{\mathcal{F}}_{i}\tilde{\mathcal{E}}^{(n)}_{i}(L) \oplus \bigoplus\limits_{0\leqslant m \leqslant -N-1} \tilde{\mathcal{E}}^{(n-1)}_{i}(L)[-2m-N-1], &\textrm{if}\ N\leqslant -1.
	\end{align*}
\end{lemma}
\begin{proof}
	On the one hand, take graded space $\mathbf{V}''$  such that $|\mathbf{V}|=|\mathbf{V}''|-i $ and consider the diagrams
	\[
	\xymatrix{
		\mathbf{E}_{\mathbf{V},i,d} & \mathbf{E}^{'}_{\mathbf{V}'',i,d}\ar[l]_{p'_{1}} \ar[r]^{p'_{2}}& \mathbf{E}^{''}_{\mathbf{V}'',i,d} \ar[r]^{p'_{3}}& \mathbf{E}_{\mathbf{V}'',i,d}\\
		\mathbf{E}^{0}_{\mathbf{V},i,d} \ar[d]_{\phi_{1}}  \ar[u]^{j_{1}}& \mathbf{E}^{',0}_{\mathbf{V}'',i,d} \ar[l]_{\tilde{p}_{1}} \ar[r]^{\tilde{p}_{2}} \ar[d]_{\phi_{2}}\ar[u]^{j_{2}}& \mathbf{E}^{'',0}_{\mathbf{V}'',i,d}  \ar[r]^{\tilde{p}_{3}}  \ar[d]_{\phi_{3}}\ar[u]^{j_{3}}& \mathbf{E}^{0}_{\mathbf{V}'',i,d} \ar[d]_{\phi_{4}} \ar[u]^{j_{4}}\\
		\txt{$\dot{\mathbf{E}}_{\mathbf{V},i}$\\ $\times$\\ $\mathbf{Grass}(\nu_{i}, \tilde{\nu}_{i})$}  & \txt{$\dot{\mathbf{E}}_{\mathbf{V},i}$\\ $\times$\\ $\mathbf{Flag}(\nu_{i},\nu''_{i},\tilde{\nu}_{i})$} \ar[l]_{q_{1}'} \ar@2{-}[r]  & \txt{$\dot{\mathbf{E}}_{\mathbf{V},i}$\\ $\times$\\ $\mathbf{Flag}(\nu_{i},\nu''_{i},\tilde{\nu}_{i})$}  \ar[r]^{q_{2}'}  & \txt{$\dot{\mathbf{E}}_{\mathbf{V}'',i}$\\ $\times$\\ $\mathbf{Grass}(\nu''_{i}, \tilde{\nu}_{i}),$}
	}
	\]
	
	\[
	\xymatrix{
		\mathbf{E}_{\mathbf{V}'',i,d}
		&
		& \mathbf{E}_{\mathbf{V}',i,d} \\	
		\mathbf{E}^{0}_{\mathbf{V}'',i,d}  \ar[d]_{\phi_{\mathbf{V}'',i}} \ar[u]^{j^{0}_{\mathbf{V}'',i}}
		&
		& \mathbf{E}^{0}_{\mathbf{V}',i,d}\ar[d]^{\phi_{\mathbf{V}',i}} \ar[u]_{j^{0}_{\mathbf{V},i}} \\
		\dot{\mathbf{E}}_{\mathbf{V}'',i} \times \mathbf{Grass}(\nu''_{i}, \tilde{\nu}_{i})
		& \dot{\mathbf{E}}_{\mathbf{V}'',i} \times \mathbf{Flag}(\nu''_{i},\nu_{i},\tilde{\nu}'_{i}) \ar[r]^{q_{2}} \ar[l]_{q_{1}}
		& \dot{\mathbf{E}}_{\mathbf{V}',i} \times \mathbf{Grass}(\nu'_{i}, \tilde{\nu}_{i}).
	}
	\]
	By base change, we have
	\begin{equation*}
		\begin{split}
			\tilde{\mathcal{E}}^{(n)}_{i}\tilde{\mathcal{F}}_{i} \cong &(j^{0}_{\mathbf{V}',i})_{!\ast} (\phi_{\mathbf{V}',i})^{\ast} (q_{2})_{!}(q_{1})^{\ast} (\phi_{\mathbf{V}'',i})_{\flat}(j^{0}_{\mathbf{V}'',i})^{\ast}(j_{4})_{!\ast}(\phi_{4})^{\ast}(q'_{2})_{!}(q'_{1})^{\ast}(\phi_{1})_{\flat}(j_{1})^{\ast}[\tilde{\nu}_{i}-(n-1)\nu_i-n]\\
			\cong &(j^{0}_{\mathbf{V}',i})_{!\ast} (\phi_{\mathbf{V}',i})^{\ast} (q_{2})_{!}(q_{1})^{\ast} (q'_{2})_{!}(q'_{1})^{\ast}(\phi_{1})_{\flat}(j_{1})^{\ast}[\tilde{\nu}_{i}-(n-1)\nu_i-n]\\
			\cong &(j^{0}_{\mathbf{V}',i})_{!\ast} (\phi_{\mathbf{V}',i})^{\ast} (q_{2})_{!}(q_{1})^{\ast} (q'_{2})_{!}(q'_{1})^{\ast}(\phi_{\mathbf{V},i})_{\flat}(j^{0}_{\mathbf{V},i})^{\ast}[\tilde{\nu}_{i}-(n-1)\nu_i-n],
		\end{split}
	\end{equation*}
	where the second isomorphism follows from $j_{4}= j^{0}_{\mathbf{V}'',i}, \phi_{4}=\phi_{\mathbf{V}'',i}$.
	
	On the other hand, take a graded space $\mathbf{V}'''$ such that $|\mathbf{V}|=|\mathbf{V}'''|+ni$ and consider the diagrams
	
	
	\[
	\xymatrix{
		\mathbf{E}_{\mathbf{V},i,d} 
		&
		& \mathbf{E}_{\mathbf{V}''',i,d} \\	
		\mathbf{E}^{0}_{\mathbf{V},i,d} \ar[d]_{\phi_{\mathbf{V},i}} \ar[u]^{j^{0}_{\mathbf{V},i}}
		&
		& \mathbf{E}^{0}_{\mathbf{V}''',i,d} \ar[d]^{\phi_{\mathbf{V}''',i}} \ar[u]_{j^{0}_{\mathbf{V}''',i}} \\
		\dot{\mathbf{E}}_{\mathbf{V},i} \times \mathbf{Grass}(\nu_{i}, \tilde{\nu}_{i})
		& \dot{\mathbf{E}}_{\mathbf{V},i} \times \mathbf{Flag}(\nu'''_{i},\nu_{i},\tilde{\nu}_{i}) \ar[r]^{q_{2}} \ar[l]_{q_{1}}
		& \dot{\mathbf{E}}_{\mathbf{V}''',i}\times \mathbf{Grass}(\nu'''_{i}, \tilde{\nu}_{i}),
	}
	\]
	
	\[
	\xymatrix{
		\mathbf{E}_{\mathbf{V}''',i,d} & \mathbf{E}^{'}_{\mathbf{V}',i,d}\ar[l]_{p'_{1}} \ar[r]^{p'_{2}}& \mathbf{E}^{''}_{\mathbf{V}',i,d} \ar[r]^{p'_{3}}& \mathbf{E}_{\mathbf{V}',i,d} \\
		\mathbf{E}^{0}_{\mathbf{V}''',i,d} \ar[d]_{\phi_{1}}  \ar[u]^{j_{1}}& \mathbf{E}^{',0}_{\mathbf{V}',i,d} \ar[l]_{\tilde{p}_{1}} \ar[r]^{\tilde{p}_{2}} \ar[d]_{\phi_{2}}\ar[u]^{j_{2}}& \mathbf{E}^{'',0}_{\mathbf{V}',i,d}  \ar[r]^{\tilde{p}_{3}}  \ar[d]_{\phi_{3}}\ar[u]^{j_{3}}& \mathbf{E}^{0}_{\mathbf{V}',i,d}\ar[d]_{\phi_{4}} \ar[u]^{j_{4}}\\
		\txt{$\dot{\mathbf{E}}_{\mathbf{V}''',i}$\\$\times$\\ $\mathbf{Grass}(\nu'''_{i}, \tilde{\nu}_{i})$}  & \txt{$\dot{\mathbf{E}}_{\mathbf{V}',i}$\\ $\times$\\ $\mathbf{Flag}(\nu'''_{i},\nu'_{i},\tilde{\nu}_{i})$} \ar[l]_{q_{1}'} \ar@2{-}[r]  & \txt{$\dot{\mathbf{E}}_{\mathbf{V}''',i}$\\ $\times$\\  $\mathbf{Flag}(\nu'''_{i},\nu'_{i},\tilde{\nu}_{i})$}  \ar[r]^{q_{2}'}  & \txt{$\dot{\mathbf{E}}_{\mathbf{V}',i}$\\ $\times$\\ $\mathbf{Grass}(\nu'_{i}, \tilde{\nu}_{i})$.}
	}
	\]
	
	
	Similarly, we have
	\begin{equation*}
		\begin{split}
			\tilde{\mathcal{F}}_{i}\tilde{\mathcal{E}}^{(n)}_{i}=& (j_{4})_{!\ast}(\phi_{4})^{\ast}(\tilde{q}'_{2})_{!}(\tilde{q}'_{1})^{\ast}(\phi_{1})_{\flat}(j_{1})^{\ast} (j^{0}_{\mathbf{V}',i})_{!\ast} (\phi_{\mathbf{V}''',i})^{\ast} (\tilde{q}_{2})_{!}(\tilde{q}_{1})^{\ast} (\phi_{\mathbf{V},i})_{\flat}(j^{0}_{\mathbf{V},i})^{\ast}[\tilde{\nu}_{i}-(n-1)\nu_i-n]\\
			\cong&(j_{4})_{!\ast}(\phi_{4})^{\ast}(\tilde{q}'_{2})_{!}(\tilde{q}'_{1})^{\ast} (\tilde{q}_{2})_{!}(\tilde{q}_{1})^{\ast} (\phi_{\mathbf{V},i})_{\flat}(j^{0}_{\mathbf{V},i})^{\ast}[\tilde{\nu}_{i}-(n-1)\nu_i-n]\\
			\cong&(j^{0}_{\mathbf{V}',i})_{!\ast}(\phi_{\mathbf{V}',i})^{\ast}(\tilde{q}'_{2})_{!}(\tilde{q}'_{1})^{\ast} (\tilde{q}_{2})_{!}(\tilde{q}_{1})^{\ast} (\phi_{\mathbf{V},i})_{\flat}(j^{0}_{\mathbf{V},i})^{\ast}[\tilde{\nu}_{i}-(n-1)\nu_i-n].
		\end{split}
	\end{equation*}
	
	Since $\phi_{\flat}$ and $\phi^{\ast}$ are quasi inverse to each other, we only need to calculate the difference between $(q_{2})_{!}(q_{1})^{\ast} (q'_{2})_{!}(q'_{1})^{\ast}$ and $(\tilde{q}'_{2})_{!}(\tilde{q}'_{1})^{\ast} (\tilde{q}_{2})_{!}(\tilde{q}_{1})^{\ast}$. Notice that we have $\nu_{i}=\nu'_{i}+n-1=\nu_{i}''-1=\nu_{i}'''+n$,$\tilde{\nu}_{i}=\tilde{\nu}'_{i}=\tilde{\nu}''_{i}=\tilde{\nu}'''_{i}$ and $\dot{\mathbf{E}}_{\mathbf{V},i}=\dot{\mathbf{E}}_{\mathbf{V}',i}=\dot{\mathbf{E}}_{\mathbf{V}'',i}=\dot{\mathbf{E}}_{\mathbf{V}''',i}$. Consider the following commutative diagram
	\[
	\xymatrix{
		Y_{0} \ar[rr]^{\tilde{\pi}} \ar[dd]_{\tilde{\pi}'} &  & \dot{\mathbf{E}}_{\mathbf{V},i} \times \mathbf{Grass}(\nu_{i}+1-n, \tilde{\nu}_{i})  \\
		&  Y_{1} \ar[ul]^{r_{1}} \ar[r]^-{\pi_1} \ar[d]^{\pi'_1} & \dot{\mathbf{E}}_{\mathbf{V},i} \times \mathbf{Flag}(\nu_{i}+1-n,\nu_{i}+1,\tilde{\nu}_{i}), \ar[u]_{q_{2}} \ar[d]^{q_{1}}\\
		\dot{\mathbf{E}}_{\mathbf{V},i} \times \mathbf{Grass}(\nu_{i},\tilde{\nu}_{i}) & \ \dot{\mathbf{E}}_{\mathbf{V},i} \times \mathbf{Flag}(\nu_{i},\nu_{i}+1,\tilde{\nu}_{i})\ar[l]_{q'_{1}} \ar[r]^-{q'_{2}} &  \dot{\mathbf{E}}_{\mathbf{V}'',i} \times \mathbf{Grass}((\nu_{i}+1), \tilde{\nu}_{i}), 
	}
	\]
	where the varieties
	\begin{equation*}
		Y_{0}=\dot{\mathbf{E}}_{\mathbf{V},i}\times \mathbf{Grass}(\nu_{i}, \tilde{\nu}_{i}) \times \mathbf{Grass}(\nu_{i}+1-n, \tilde{\nu}_{i}), 
	\end{equation*}
	and \begin{align*}
		Y_{1}=&\{(\dot{x}, \mathbf{V}^{1} \subseteq \mathbf{V}^{3} \subseteq  (\bigoplus\limits_{h'=i}\mathbf{V}_{h''})\oplus \mathbb{C}^{d}), \mathbf{V}^{2} \subseteq \mathbf{V}^{3} \subseteq  (\bigoplus\limits_{h'=i}\mathbf{V}_{h''})\oplus \mathbb{C}^{d}) \\
		&|~ \dot{x} \in   \dot{\mathbf{E}}_{\mathbf{V},i},~{\rm{dim}} \mathbf{V}^{1}=\nu_{i},~{\rm{dim}} \mathbf{V}^{2}=\nu_{i}-n+1,~{\rm{dim}} \mathbf{V}^{3}=\nu_{i}+1.\}
	\end{align*}
	is the fiber product of $q'_{2},q_{1}$, the morphisms $\pi_1,\pi'_1,\tilde{\pi},\tilde{\pi}'$ are natural projections and
	\begin{equation*}
		r_{1}((\dot{x},\mathbf{V}^{1},\mathbf{V}^{2},\mathbf{V}^{3}))=(\dot{x},\mathbf{V}^{1},\mathbf{V}^{2}). 
	\end{equation*}
	By base change, we have
	\begin{equation*}
		\begin{split}
			(q_{2})_{!}(q_{1})^{\ast} (q'_{2})_{!}(q'_{1})^{\ast}\cong& (q_{2})_{!}(\pi_1)_{!} (\pi'_1)^{\ast}(q'_{1})^{\ast}\\
			=&  (\tilde{\pi})_{!} (r_{1})_{!}(r_{1})^{\ast}(\tilde{\pi}')^{\ast}.
		\end{split}
	\end{equation*}
	
	Similarly, consider the commutative diagram
	\[
	\xymatrix{
		Y_{0} \ar[rr]^{\tilde{\pi}} \ar[dd]_{\tilde{\pi}'} &  & \dot{\mathbf{E}}_{\mathbf{V},i} \times \mathbf{Grass}(\nu_{i}-n+1, \tilde{\nu}_{i})  \\
		&  Y_{2} \ar[ul]^{r_{2}} \ar[r]^-{\pi_2} \ar[d]^{\pi'_2} & \dot{\mathbf{E}}_{\mathbf{V},i} \times \mathbf{Flag}(\nu_{i}-n,\nu_{i}-n+1,\tilde{\nu}_{i}) \ar[u]_{\tilde{q}_{2}'} \ar[d]^{\tilde{q}_{1}'}\\
		\dot{\mathbf{E}}_{\mathbf{V},i} \times \mathbf{Grass}(\nu_{i},\tilde{\nu}_{i}) & \ \dot{\mathbf{E}}_{\mathbf{V},i} \times \mathbf{Flag}(\nu_{i}-n,\nu_{i},\tilde{\nu}_{i})\ar[l]_{\tilde{q}_{1}} \ar[r]^-{\tilde{q}_{2}} &  \dot{\mathbf{E}}_{\mathbf{V}',i}\times \mathbf{Grass}(\nu_{i}-n, \tilde{\nu}_{i}),
	}
	\]
	where the variety
	\begin{align*}
		Y_{2}=&\{(\dot{x},\mathbf{V}^{0}\subseteq \mathbf{V}^{1} \subseteq  (\bigoplus\limits_{h'=i}\mathbf{V}_{h''})\oplus \mathbb{C}^{d},\mathbf{V}^{0}\subseteq \mathbf{V}^{2} \subseteq  (\bigoplus\limits_{h'=i}\mathbf{V}_{h''})\oplus \mathbb{C}^{d})\\
		&|~ \dot{x} \in   \dot{\mathbf{E}}_{\mathbf{V},i}, {\rm{dim}} \mathbf{V}^{1}= \nu_{i}, {\rm{dim}} \mathbf{V}^{2} =\nu_{i}-n+1 ,{\rm{dim}} \mathbf{V}^{0}=\nu_{i}-n.\}
	\end{align*}
	is the fiber product of $\tilde{q}'_{1},\tilde{q}_{2}$, the morphisms $\pi_2,\pi'_2,\tilde{\pi},\tilde{\pi}'$ are natural projections and 
	\begin{equation*}
		r_{2}((\dot{x},\mathbf{V}^{0},\mathbf{V}^{1},\mathbf{V}^{2}))=(\dot{x},\mathbf{V}^{1},\mathbf{V}^{2}). 
	\end{equation*}
	By base change, we have
	\begin{equation*}
		(\tilde{q}'_{2})_{!}(\tilde{q}'_{1})^{\ast} (\tilde{q}_{2})_{!}(\tilde{q}_{1})^{\ast}\cong(\tilde{\pi})_{!} (r_{2})_{!}(r_{2})^{\ast}(\tilde{\pi}')^{\ast}.
	\end{equation*}
	
	It remains to calculate the difference between $(\tilde{\pi})_{!} (r_{1})_{!}(r_{1})^{\ast}(\tilde{\pi}')^{\ast}$ and $(\tilde{\pi})_{!} (r_{2})_{!}(r_{2})^{\ast}(\tilde{\pi}')^{\ast}$. We divide $Y_0$ into the disjoint union $Y_{0}=Y_{0}^{0} \cup Y_{0}^{1}$, where
	\begin{equation*}
		Y^{0}_{0}=\{(\dot{x},\mathbf{V}^{1},\mathbf{V}^{2})\in Y_{0}|\mathbf{V}^{2} \subseteq \mathbf{V}^{1} \},
	\end{equation*}
	\begin{equation*}
		Y^{1}_{0}=\{(\dot{x},\mathbf{V}^{1},\mathbf{V}^{2})\in Y_{0}|\mathbf{V}^{2} \nsubseteq \mathbf{V}^{1} \}.
	\end{equation*}
	For $a,b=0,1$, let $Y_{a}^{0}=(r_{a})^{-1}(Y_{0}^{0}), Y_{a}^{1}=(r_{a})^{-1}(Y_{0}^{1})$ and $\iota^{b}:Y_{0}^{b} \rightarrow Y_{0}$ be the natural embedding. Let $\tilde{r}_{a}^{b}:Y_a^b\rightarrow Y_0^b$ be the restriction of $r_{a}$ on $Y_{a}^{b}$, and $r_a^b=\iota^b\tilde{r}_{a}^{b}:Y_a^b\rightarrow Y_0$.
	
	Notice that when $\mathbf{V}^{2} \nsubseteq \mathbf{V}^{1}$, we have $\mathbf{V}^{0} = \mathbf{V}^{1}\cap\mathbf{V}^{2} , \mathbf{V}^{3}= \mathbf{V}^{1}\cup\mathbf{V}^{2}$, hence $\tilde{r}^{1}_{a}: Y^1_a\rightarrow Y_0^1$ is an isomorphism. For any complex $L'$, by Lemma \ref{projection formula} and \ref{Lusztig BBD}, we have 
	\begin{equation*}
		\begin{split}
			(\tilde{\pi})_{!} (r_{1})_{!}(r_{1})^{\ast}(\tilde{\pi}')^{\ast}L' \cong & (\tilde{\pi})_{!} ((r_{1})_{!}\overline{\mathbb{Q}}_{l}\otimes (\tilde{\pi}')^{\ast} L') \\
			\cong &  (\tilde{\pi})_{!} ((r^{0}_{1})_{!}\overline{\mathbb{Q}}_{l}\otimes (\tilde{\pi}')^{\ast} L') \oplus (\tilde{\pi})_{!} ((r^{1}_{1})_{!}\overline{\mathbb{Q}}_{l}\otimes (\tilde{\pi}')^{\ast} L')  \\ 
			\cong &  (\tilde{\pi})_{!} ((r^{0}_{1})_{!}\overline{\mathbb{Q}}_{l}\otimes (\tilde{\pi}')^{\ast} L') \oplus (\tilde{\pi})_{!} ((\iota^{1})_{!}\overline{\mathbb{Q}}_{l}\otimes (\tilde{\pi}')^{\ast} L') \\
			\cong &  (\tilde{\pi})_{!} ( (\iota^{0})_{!}(\tilde{r}^{0}_{1})_{!}\overline{\mathbb{Q}}_{l}\otimes (\tilde{\pi}')^{\ast} L')  \oplus (\tilde{\pi})_{!} ((\iota^{1})_{!}\overline{\mathbb{Q}}_{l}\otimes (\tilde{\pi}')^{\ast} L').
		\end{split}
	\end{equation*}
	Similarly, we have 	
	\begin{equation*}
		(\tilde{\pi})_{!} (r_{2})_{!}(r_{2})^{\ast}(\tilde{\pi}')^{\ast}L'\cong (\tilde{\pi})_{!} ( (\iota^{0})_{!}(\tilde{r}^{0}_{2})_{!}\overline{\mathbb{Q}}_{l}\otimes (\tilde{\pi}')^{\ast} L')  \oplus (\tilde{\pi})_{!} ((\iota^{1})_{!}\overline{\mathbb{Q}}_{l}\otimes (\tilde{\pi}')^{\ast} L'). 
	\end{equation*}
	Since $\tilde{r}^{0}_{1}$ is a fiber bundle with each fiber isomorphic to $\mathbf{Grass}(1,\tilde{\nu}_{i}-\nu_{i})$ and $\tilde{r}^{0}_{2}$ is a fiber bundle with each fiber isomorphic to $\mathbf{Grass}(\nu_{i}-n,\nu_{i}-n+1)$, we have
	\begin{equation*}
		(\tilde{r}^{0}_{1})_{!}(\overline{\mathbb{Q}}_{l}) \cong \bigoplus\limits_{0\leqslant m \leqslant \tilde{\nu}_{i}-\nu_i-1} \overline{\mathbb{Q}}_{l}[-2m],
	\end{equation*}
	\begin{equation*}
		(\tilde{r}^{0}_{2})_{!}(\overline{\mathbb{Q}}_{l}) \cong \bigoplus\limits_{0\leqslant m \leqslant \nu_{i}-n} \overline{\mathbb{Q}}_{l}[-2m].
	\end{equation*}
	
	Note that $Y^{0}_{0}$ is isomorphic to $\dot{\mathbf{E}}_{\mathbf{V},i} \times \mathbf{Flag}(\nu_{i}-n+1,\nu_{i},\tilde{\nu}_{i})$ by definition, and morphisms $\tilde{\pi}'\iota^{0}:Y^{0}_{0} \rightarrow \dot{\mathbf{E}}_{\mathbf{V},i} \times \mathbf{Grass}(\nu_{i}, \tilde{\nu}_{i})$ and $\tilde{\pi}\iota^{0}:Y^{0}_{0} \rightarrow \dot{\mathbf{E}}_{\mathbf{V},i} \times \mathbf{Grass}(\nu_{i}-n+1, \tilde{\nu}_{i})$ can be respectively identified with the morphisms $\check{q}_{1}$ and $\check{q}_{2}$ in the definition of  $\mathcal{E}^{(n-1)}_{i}$ as follows
		\[
	\xymatrix{
		\mathbf{E}_{\mathbf{V},i,d} 
		&
		& \mathbf{E}_{\mathbf{V}',i,d} \\	
		\mathbf{E}^{0}_{\mathbf{V},i,d} \ar[d]_{\phi_{\mathbf{V},i}} \ar[u]^{j^{0}_{\mathbf{V},i}}
		&
		& \mathbf{E}^{0}_{\mathbf{V}',i,d} \ar[d]^{\phi_{\mathbf{V}'',i}} \ar[u]_{j^{0}_{\mathbf{V}',i}} \\
		\dot{\mathbf{E}}_{\mathbf{V},i} \times \mathbf{Grass}(\nu_{i}, \tilde{\nu}_{i})
		& \dot{\mathbf{E}}_{\mathbf{V},i} \times \mathbf{Flag}(\nu'_{i},\nu_{i},\tilde{\nu}_{i}) \ar[r]^{\check{q}_{2}} \ar[l]_{\check{q}_{1}}
		& \dot{\mathbf{E}}_{\mathbf{V}',i}\times \mathbf{Grass}(\nu'_{i}, \tilde{\nu}_{i}).
	}
	\]
    Applying Lemma \ref{projection formula} for morphisms $\iota^0$ and $\tilde{\pi}'\iota^{0},\tilde{\pi}\iota^{0}$ respectively, we obtain
	\begin{equation*}
		(\tilde{\pi})_{!} ( (\iota^{0})_{!}\overline{\mathbb{Q}}_{l}\otimes (\tilde{\pi}')^{\ast} L') 
		\cong (\tilde{\pi})_{!} (\iota^{0})_{!} (\iota^{0})^{\ast} (\tilde{\pi}')^{\ast}L'\cong (\check{q}_{2})_{!}(\check{q}_{1})^{\ast}L'.
	\end{equation*}
	Therefore, the difference between $(\tilde{\pi})_{!} (r_{1})_{!}(r_{1})^{\ast}(\tilde{\pi}')^{\ast}L'$ and $(\tilde{\pi})_{!} (r_{2})_{!}(r_{2})^{\ast}(\tilde{\pi}')^{\ast}L'$ only involves direct sums of shifts of $(\check{q}_{2})_{!}(\check{q}_{1})^{\ast}L'$, and so the difference between $\tilde{\mathcal{E}}^{(n)}_{i}\tilde{\mathcal{F}}_{i}(L)$ and $\tilde{\mathcal{F}}_{i}\tilde{\mathcal{E}}^{(n)}_{i}(L)$ only involves direct sums of shifts of $\tilde{\mathcal{E}}^{(n-1)}_{i}(L)$. By direct calculation, we have
	\begin{align*}
		&\tilde{\mathcal{E}}^{(n)}_{i}\tilde{\mathcal{F}}_{i}(L)\cong \tilde{\mathcal{F}}_{i}\tilde{\mathcal{E}}^{(n)}_{i}(L), &\textrm{if}\ N=0;\\
		&\tilde{\mathcal{E}}^{(n)}_{i}\tilde{\mathcal{F}}_{i}(L)\oplus\bigoplus_{0\leqslant m\leqslant N-1}\tilde{\mathcal{E}}^{(n-1)}_{i}(L)[N-1-2m]\cong \tilde{\mathcal{F}}_{i}\tilde{\mathcal{E}}_{i}(L),&\textrm{if}\  N\geqslant 1;\\
		&\tilde{\mathcal{E}}^{(n)}_{i}\tilde{\mathcal{F}}_{i}(L)\cong \tilde{\mathcal{F}}_{i}\tilde{\mathcal{E}}^{(n)}_{i}(L) \oplus \bigoplus\limits_{0\leqslant m \leqslant -N-1} \tilde{\mathcal{E}}^{(n-1)}_{i}(L)[-2m-N-1], &\textrm{if}\ N\leqslant -1.
	\end{align*}
	as desired.	
\end{proof}

\begin{lemma} \label{lemma c2}
	For any graded space $\mathbf{V},\mathbf{V}'$ such that $|\mathbf{V}'|+ni=|\mathbf{V}|+j $, and any simple perverse sheaf $L$ on $\mathbf{E}_{\mathbf{V},i,d}$ such that
	$(j^{0}_{\mathbf{V},i})^{\ast}(L) \neq 0$, then for $i \neq j$, there is an isomorphism in the localization $\mathcal{D}^{b}_{G_{\mathbf{V}'}}(\mathbf{E}_{\mathbf{V}',i,d})/\tilde{\mathcal{N}}_{\mathbf{V}',i}$,
	\begin{equation*}
		\tilde{\mathcal{E}}^{(n)}_{i}\tilde{\mathcal{F}}_{j}(L) \cong \tilde{\mathcal{F}}_{j}\tilde{\mathcal{E}}^{(n)}_{i}(L). 
	\end{equation*}
\end{lemma}
\begin{proof}
	We assume that $i,j$ are connected by some edges, the other case can be proved by a similar argument. On the one hand, take graded spaces $\mathbf{V}',\mathbf{V}''$ such that $|\mathbf{V}|+j=|\mathbf{V}''|=|\mathbf{V}'|+ni$ and consider the diagrams
	\[
	\xymatrix{
		\mathbf{E}_{\mathbf{V},i,d} & \mathbf{E}^{'}_{\mathbf{V}'',i,d}\ar[l]_{p'_{1}} \ar[r]^{p'_{2}}& \mathbf{E}^{''}_{\mathbf{V}'',i,d} \ar[r]^{p'_{3}}& \mathbf{E}_{\mathbf{V}'',i,d}\\
		\mathbf{E}^{0}_{\mathbf{V},i,d} \ar[d]_{\phi_{1}}  \ar[u]^{j_{1}}& \mathbf{E}^{',0}_{\mathbf{V}'',i,d} \ar[l]_{\tilde{p}_{1}} \ar[r]^{\tilde{p}_{2}} \ar[d]_{\phi_{2}}\ar[u]^{j_{2}}& \mathbf{E}^{'',0}_{\mathbf{V}'',i,d}  \ar[r]^{\tilde{p}_{3}}  \ar[d]_{\phi_{3}}\ar[u]^{j_{3}}& \mathbf{E}^{0}_{\mathbf{V}'',i,d} \ar[d]_{\phi_{4}} \ar[u]^{j_{4}}\\
		\dot{\mathbf{E}}_{\mathbf{V},i} \times \mathbf{Grass}(\nu_{i}, \tilde{\nu}_{i})  & Z_{1} \ar[l]_-{q_{1}'} \ar[r]^{q'_{2}}  & Z_{2} \ar[r]^-{q'_{3}}& \dot{\mathbf{E}}_{\mathbf{V}'',i} \times \mathbf{Grass}(\nu''_{i}, \tilde{\nu}_{i}+1),
	}
	\]
	\[
	\xymatrix{
		\mathbf{E}_{\mathbf{V}'',i,d}
		&
		& \mathbf{E}_{\mathbf{V}',i,d} \\	
		\mathbf{E}^{0}_{\mathbf{V}'',i,d} \ar[d]_{\phi_{\mathbf{V}'',i}} \ar[u]^{j^{0}_{\mathbf{V}'',i}}
		&
		& \mathbf{E}^{0}_{\mathbf{V}',i,d} \ar[d]^{\phi_{\mathbf{V},i}} \ar[u]_{j^{0}_{\mathbf{V},i}} \\
		\dot{\mathbf{E}}_{\mathbf{V}'',i} \times \mathbf{Grass}(\nu''_{i}, \tilde{\nu}_{i}+1)
		& \dot{\mathbf{E}}_{\mathbf{V}'',i} \times \mathbf{Flag}(\nu''_{i},\nu'_{i},\tilde{\nu}_{i}+1) \ar[r]^{q_{2}} \ar[l]_{q_{1}}
		& \dot{\mathbf{E}}_{\mathbf{V}',i} \times \mathbf{Grass}(\nu'_{i}, \tilde{\nu}_{i}+1).
	}
	\]
	By base change, up to shifts we have
	\begin{equation*}
		\begin{split}
			\tilde{\mathcal{E}}_{i}\tilde{\mathcal{F}}_{j }\cong&(j^{0}_{\mathbf{V}',i})_{!\ast} (\phi_{\mathbf{V}',i})^{\ast} (q_{2})_{!}(q_{1})^{\ast} (\phi_{\mathbf{V}'',i})_{\flat}(j^{0}_{\mathbf{V}'',i})^{\ast}(j_{4})_{!\ast}(\phi_{4})^{\ast}(q'_{3})_{!}(q'_{2})_{\flat}(q'_{1})^{\ast}(\phi_{1})_{\flat}(j_{1})^{\ast}\\
			\cong&(j^{0}_{\mathbf{V}',i})_{!\ast} (\phi_{\mathbf{V}',i})^{\ast} (q_{2})_{!}(q_{1})^{\ast} (q'_{3})_{!}(q'_{2})_{\flat}(q'_{1})^{\ast}(\phi_{1})_{\flat}(j_{1})^{\ast}\\
			\cong&(j^{0}_{\mathbf{V}',i})_{!\ast} (\phi_{\mathbf{V}',i})^{\ast} (q_{2})_{!}(q_{1})^{\ast} (q'_{3})_{!}(q'_{2})_{\flat}(q'_{1})^{\ast}(\phi_{\mathbf{V},i})_{\flat}(j^{0}_{\mathbf{V},i})^{\ast}.
		\end{split}
	\end{equation*}
	
	
	On the other hand, take a graded space $\mathbf{V}'''$ such that $|\mathbf{V}|=|\mathbf{V}'''|+ni$ and consider the diagrams
	\[
	\xymatrix{
		\mathbf{E}_{\mathbf{V},i,d} 
		&
		& \mathbf{E}_{\mathbf{V}''',i,d} \\	
		\mathbf{E}^{0}_{\mathbf{V},i,d} \ar[d]_{\phi_{\mathbf{V},i}} \ar[u]^{j^{0}_{\mathbf{V},i}}
		&
		& \mathbf{E}^{0}_{\mathbf{V}''',i,d} \ar[d]^{\phi_{\mathbf{V}''',i}} \ar[u]_{j^{0}_{\mathbf{V}''',i}} \\
		\dot{\mathbf{E}}_{\mathbf{V},i} \times \mathbf{Grass}(\nu_{i}, \tilde{\nu}_{i})
		& \dot{\mathbf{E}}_{\mathbf{V},i} \times \mathbf{Flag}(\nu_{i}-n,\nu_{i},\tilde{\nu}_{i}) \ar[r]^{\tilde{q}_{2}} \ar[l]_{\tilde{q}_{1}}
		& \dot{\mathbf{E}}_{\mathbf{V}''',i} \times \mathbf{Grass}(\nu_{i}-n, \tilde{\nu}_{i}),
	}
	\]
	\[
	\xymatrix{
		\mathbf{E}_{\mathbf{V}''',i,d} & \mathbf{E}^{'}_{\mathbf{V}',i,d}\ar[l]_{p'_{1}} \ar[r]^{p'_{2}}& \mathbf{E}^{''}_{\mathbf{V}',i,d} \ar[r]^{p'_{3}}& \mathbf{E}_{\mathbf{V}',i,d} \\
		\mathbf{E}^{0}_{\mathbf{V}''',i,d} \ar[d]_{\phi_{1}}  \ar[u]^{j_{1}}& \mathbf{E}^{',0}_{\mathbf{V}',i,d} \ar[l]_{\tilde{p}_{1}} \ar[r]^{\tilde{p}_{2}} \ar[d]_{\phi_{2}}\ar[u]^{j_{2}}& \mathbf{E}^{'',0}_{\mathbf{V}',i,d}  \ar[r]^{\tilde{p}_{3}}  \ar[d]_{\phi_{3}}\ar[u]^{j_{3}}& \mathbf{E}^{0}_{\mathbf{V}',i,d} \ar[d]_{\phi_{4}} \ar[u]^{j_{4}}\\
		\dot{\mathbf{E}}_{\mathbf{V}''',i} \times \mathbf{Grass}(\nu_{i}-n, \tilde{\nu}_{i})  & \tilde{Z}_{1} \ar[l]_-{\tilde{q}_{1}'} \ar[r]^{\tilde{q}'_{2}}  & \tilde{Z}_{2} \ar[r]^-{\tilde{q}'_{3}}& \dot{\mathbf{E}}_{\mathbf{V}',i} \times \mathbf{Grass}(\nu_{i}-n, \tilde{\nu}_{i}+1).
	}
	\]
	Similarly, up to shifts we have
	\begin{equation*}
		\begin{split}
			\tilde{\mathcal{F}}_{j}\tilde{\mathcal{E}}_{i}\cong& (j_{4})_{!\ast}(\phi_{4})^{\ast}(\tilde{q}_{3}')_{!}(\tilde{q}'_{2})_{\flat}(\tilde{q}'_{1})^{\ast}(\phi_{1})_{\flat}(j_{1})^{\ast} (j^{0}_{\mathbf{V}''',i})_{!\ast} (\phi_{\mathbf{V}''',i})^{\ast} (\tilde{q}_{2})_{!}(\tilde{q}_{1})^{\ast} (\phi_{\mathbf{V},i})_{\flat}(j^{0}_{\mathbf{V},i})^{\ast}\\
			\cong&(j_{4})_{!\ast}(\phi_{4})^{\ast}(\tilde{q}_{3}')_{!}(\tilde{q}'_{2})_{\flat}(\tilde{q}'_{1})^{\ast} (\tilde{q}_{2})_{!}(\tilde{q}_{1})^{\ast} (\phi_{\mathbf{V},i})_{\flat}(j^{0}_{\mathbf{V},i})^{\ast}\\
			\cong&(j^{0}_{\mathbf{V}',i})_{!\ast}(\phi_{\mathbf{V}',i})^{\ast}(\tilde{q}_{3}')_{!}(\tilde{q}'_{2})_{\flat}(\tilde{q}'_{1})^{\ast} (\tilde{q}_{2})_{!}(\tilde{q}_{1})^{\ast} (\phi_{\mathbf{V},i})_{\flat}(j^{0}_{\mathbf{V},i})^{\ast}.
		\end{split}
	\end{equation*}
	
	We only need to calculate the difference between $(q_{2})_{!}(q_{1})^{\ast} (q'_{3})_{!}(q'_{2})_{\flat}(q'_{1})^{\ast}$ and $(\tilde{q}_{3}')_{!}(\tilde{q}'_{2})_{\flat}(\tilde{q}'_{1})^{\ast} (\tilde{q}_{2})_{!}(\tilde{q}_{1})^{\ast}$. Notice that we have $\nu_{i}'+n=\nu_{i}=\nu_{i}''=\nu'''_{i}+n$,$\tilde{\nu}''_{i}=\tilde{\nu}_{i}+1=\tilde{\nu}'_{i}$ and $\dot{\mathbf{E}}_{\mathbf{V}',i}=\dot{\mathbf{E}}_{\mathbf{V}'',i},\dot{\mathbf{E}}_{\mathbf{V},i}=\dot{\mathbf{E}}_{\mathbf{V}''',\mathbf{W},i}$. Consider the following commutative diagram
	\[
	\xymatrix{
		&  & & \dot{\mathbf{E}}_{\mathbf{V}',i} \times \mathbf{Grass}(\nu_{i}-n, \tilde{\nu}_{i}+1)  \\
		&  Y_{1}' \ar[dl]_{\pi_{1}} \ar[r]^-{r_{2}} \ar[d]^{\pi'} & Y_{1}'' \ar[ur]^{\pi_{2}} \ar[d]_{r_{1}} \ar[r]_-{r_{3}} & \dot{\mathbf{E}}_{\mathbf{V}',i} \times \mathbf{Flag}(\nu_{i}-n,\nu_{i},\tilde{\nu}_{i}+1), \ar[u]_{q_{2}} \ar[d]^{q_{1}}\\
		\dot{\mathbf{E}}_{\mathbf{V},i} \times \mathbf{Grass}(\nu_{i}, \tilde{\nu}_{i})  & Z_{1} \ar[l]_-{q'_{1}} \ar[r]^{q'_{2}}  & Z_{2} \ar[r]^-{q_{3}'}  & \dot{\mathbf{E}}_{\mathbf{V}'',i} \times \mathbf{Grass}(\nu_{i}, \tilde{\nu}_{i}+1),
	}
	\]
	where $Y_{1}''$ is the fiber product of $q'_{3}$ and $q_{1}$, and $Y_{1}'$ is the fiber product of $q'_{2}$ and $r_{1}$. 
	
	More precisely, the variety $Y_{1}''$ consists of quadruples $(\dot{x}, \dot{\mathbf{U}},\mathbf{V}^{1},\mathbf{V}^{2})$, where $\dot{x} \in   \dot{\mathbf{E}}_{\mathbf{V}'',i}$, $\dot{\mathbf{U}}\subseteq \dot{\mathbf{V}}''$ is a $\dot{x}$-stable graded subspace and $\mathbf{V}^{1}\subseteq \mathbf{V}^{2} \subseteq  (\bigoplus\limits_{h'=i}\mathbf{U}_{h''})\oplus \mathbb{C}^{d})$ is a flag such that $|\dot{\mathbf{U}}|=|\dot{\mathbf{V}}|$ and $ {\rm{dim}} \mathbf{V}^{1} =\nu_{i}-n,{\rm{dim}} \mathbf{V}^{2}=\nu_{i}$. 
	
	The variety $Y_{1}'$ consists of quadruples $(\dot{x}, \dot{\mathbf{U}},\mathbf{V}^{1},\mathbf{V}^{2},\dot{\rho})$ such that $\dot{x} \in   \dot{\mathbf{E}}_{\mathbf{V}'',i}$, $\dot{\mathbf{U}}\subseteq \dot{\mathbf{V}}''$ is $\dot{x}$-stable graded subspace, $\mathbf{V}^{1}\subseteq \mathbf{V}^{2} \subseteq  (\bigoplus\limits_{h'=i}\mathbf{U}_{h''})\oplus \mathbb{C}^{d})$ is a flag and $\dot{\rho}: \dot{\mathbf{U}} \rightarrow \dot{\mathbf{V}}$ is a linear isomorphism such that $|\dot{\mathbf{U}}|=|\dot{\mathbf{V}}|$ and $ {\rm{dim}} \mathbf{V}^{1} =\nu_{i}-n,{\rm{dim}} \mathbf{V}^{2}=\nu_{i}$. 
	
	The morphisms $\pi',r_{1},r_{2}.r_{3}$ are natural projections and 
	\begin{equation*}
		\pi_{1}( (\dot{x}, \dot{\mathbf{U}},\mathbf{V}^{1},\mathbf{V}^{2},\dot{\rho}) )=(\dot{\rho}_{\ast}(\dot{x}|_{\dot{\mathbf{U}}}), \dot{\rho}(\mathbf{V}^{2})  ),
	\end{equation*}
	\begin{equation*}
		\pi_{2}( (\dot{x}, \dot{\mathbf{U}},\mathbf{V}^{1},\mathbf{V}^{2}) )=(\dot{x}, \mathbf{V}^{1}),
	\end{equation*}
	where we still denote $\dot{\rho}\oplus \textrm{Id}_{\dot{\mathbf{W}}}$ by $\dot{\rho}$. By base change, we have
	\begin{equation*}
		\begin{split}
			(q_{2})_{!}(q_{1})^{\ast} (q'_{3})_{!}(q'_{2})_{\flat}(q'_{1})^{\ast}=& (q_{2})_{!}(r_{3})_{!}(r_{1})^{\ast}(q'_{2})_{\flat}(q'_{1})^{\ast}\\
			=&  (q_{2})_{!}(r_{3})_{!}(r_{2})_{\flat}(\pi')^{\ast}(q'_{1})^{\ast} \\
			=&   (\pi_{2})_{!} (r_{2})_{\flat}(\pi_{1}^{\ast}).
		\end{split}
	\end{equation*}
	
	Similarly, consider the commutative diagram
	\[
	\xymatrix{
		&  & \dot{\mathbf{E}}_{\mathbf{V}',i} \times \mathbf{Grass}(\nu_{i}-n, \tilde{\nu}_{i}+1)  \\
		&  Y_{2}'' \ar[ur]^{\tilde{\pi}_{2}} \ar[r]^{r_{4}} & \tilde{Z}_{2} \ar[u]_{\tilde{q}_{3}'}   \\
		&  Y_{2}' \ar[dl]_{\tilde{\pi}_{1}} \ar[u]^{r_{2}} \ar[r]^-{r_{3}} \ar[d]^{r_{1}} & 
		\tilde{Z}_{1}
		\ar[u]_{\tilde{q}_{2}'} \ar[d]^{\tilde{q}_{1}'}\\
		\dot{\mathbf{E}}_{\mathbf{V},i} \times \mathbf{Grass}(\nu_{i},\tilde{\nu}_{i}) & \ \dot{\mathbf{E}}_{\mathbf{V},i} \times \mathbf{Flag}(\nu_{i}-n,\nu_{i},\tilde{\nu}_{i})\ar[l]_{\tilde{q}_{1}} \ar[r]^-{\tilde{q}_{2}} &  \dot{\mathbf{E}}_{\mathbf{V}''',i} \times \mathbf{Grass}(\nu_{i}-n, \tilde{\nu}_{i}),
	}
	\]
	where the variety $Y_{2}'$ is the fiber product of $\tilde{q}'_{1}$ and $\tilde{q}_{2}$. 
	
	More precisely, recall that the variety $\tilde{Z}_{1}$ consists of quadruples $(\dot{x}',\dot{\mathbf{U}} \subseteq \dot{\mathbf{V}}',\dot{\rho},\tilde{\mathbf{V}}'''\subseteq  \bigoplus\limits_{h'=i} \mathbf{U}_{h''} \oplus \mathbb{C}^{d} )$, where $\dot{x}' \in   \dot{\mathbf{E}}_{\mathbf{V}',i}$, $\dot{\mathbf{U}}$ is a $\dot{x}'$-stable subspace of  $\dot{\mathbf{V}'}$, $\dot{\rho}: \dot{\mathbf{U}} \rightarrow \dot{\mathbf{V}}'''$ is a linear isomorphism and $\tilde{\mathbf{V}}'''$ is a  subspace of dimension $\nu_{i}-n$, then $Y_{2}'$ consists of quadruples $(\dot{x}', \dot{\mathbf{U}},\mathbf{V}^{1},\mathbf{V}^{2},\dot{\rho})$ such that $\dot{x}' \in   \dot{\mathbf{E}}_{\mathbf{V}',i}$, $\dot{\mathbf{U}}$ is a $\dot{x}'$-stable subspace of  $\dot{\mathbf{V}'}$, $\dot{\rho}: \dot{\mathbf{U}} \rightarrow \dot{\mathbf{V}}'''$ is a linear isomorphism and $\tilde{\mathbf{V}}'''$ is a  subspace of dimension $\nu_{i}-n$ and $\mathbf{V}^{1}\subseteq \mathbf{V}^{2} \subseteq  (\bigoplus\limits_{h'=i}\mathbf{U}_{h''})\oplus \mathbb{C}^{d}$ is a flag such that $ {\rm{dim}} \mathbf{V}^{1} =\nu_{i}-1,{\rm{dim}} \mathbf{V}^{2}=\nu_{i}\}$. 
	
	Let $Y_{2}''$ be the variety consists of quadruples $(\dot{x}', \dot{\mathbf{U}},\mathbf{V}^{1},\mathbf{V}^{2})$ such that $(\dot{x}', \dot{\mathbf{U}},\mathbf{V}^{1},\mathbf{V}^{2})$ satisfies the same conditions as in $Y_{2}'$. 
	
	The morphisms $r_{1},r_{2},r_{3}$ are projections and 
	\begin{equation*}
		r_{4}((\dot{x}', \dot{\mathbf{U}},\mathbf{V}^{1},\mathbf{V}^{2},\dot{\rho}))=((\dot{\rho})_{\ast}(\dot{x}'|_{\dot{\mathbf{U}}}),\dot{\rho}(\mathbf{V}^{1}),\dot{\rho}(\mathbf{V}^{2})),
	\end{equation*}
	\begin{equation*}
		\tilde{\pi}_{1} ((\dot{x}', \dot{\mathbf{U}},\mathbf{V}^{1},\mathbf{V}^{2},\dot{\rho}))= ((\dot{\rho})_{\ast}(\dot{x}'|_{\dot{\mathbf{U}}}),\dot{\rho}(\mathbf{V}^{2})),
	\end{equation*}
	\begin{equation*}
		\tilde{\pi}_{2}((\dot{x}', \dot{\mathbf{U}},\mathbf{V}^{1},\mathbf{V}^{2}))= (\dot{x}',\mathbf{V}^{1}),
	\end{equation*}
	then the middle square is Cartesian. By base change, we have
	\begin{equation*}
		(\tilde{q}_{3}')_{!}(\tilde{q}'_{2})_{\flat}(\tilde{q}'_{1})^{\ast} (\tilde{q}_{2})_{!}(\tilde{q}_{1})^{\ast} \cong (\tilde{\pi}_{2})_{!} (r_{2})_{\flat}(\tilde{\pi}_{1}^{\ast}).
	\end{equation*}
	
	Notice that $\dot{\mathbf{E}}_{\mathbf{V}',i}=\dot{\mathbf{E}}_{\mathbf{V}'',i},\dot{\mathbf{E}}_{\mathbf{V},i}=\dot{\mathbf{E}}_{\mathbf{V}''',i}$, there are  natural isomorphisms $Y_{1}' \cong Y_{2}'$,$Y_{1}''\cong Y_{2}''$. Under the isomorphisms, we have $\tilde{\pi}_{1}=\tilde{\pi}_{2}$ and $\pi_{1}=\pi_{2}$, and so
	\begin{equation*}
		(q_{2})_{!}(q_{1})^{\ast} (q'_{3})_{!}(q'_{2})_{\flat}(q'_{1})^{\ast}\cong(\tilde{q}_{3}')_{!}(\tilde{q}'_{2})_{\flat}(\tilde{q}'_{1})^{\ast} (\tilde{q}_{2})_{!}(\tilde{q}_{1})^{\ast},
	\end{equation*}
	as desired.
\end{proof}

\begin{corollary}\label{corollary tilde E}
	The functor  $\tilde{\mathcal{E}}^{(n)}_{i}:\mathcal{D}^{b}_{G_{\mathbf{V}}}(\mathbf{E}_{\mathbf{V},i,d})/\tilde{\mathcal{N}}_{\mathbf{V},i} \rightarrow \mathcal{D}^{b}_{G_{\mathbf{V}'}}(\mathbf{E}_{\mathbf{V}',i,d})/\tilde{\mathcal{N}}_{\mathbf{V}',i}$ sends  an object of $(\pi_{\mathbf{V},i})^{\ast}(\mathcal{Q}^{0}_{\mathbf{V}})$ to an object of $(\pi_{\mathbf{V}',i})^{\ast}(\mathcal{Q}^{0}_{\mathbf{V}'})$.
\end{corollary}
\begin{proof}
	It suffices to prove that $\tilde{\mathcal{E}}^{(n)}_{i}( (\pi_{\mathbf{V},i})^{\ast} L_{\underline{\nu}} ) $ is isomorphic to a direct sum of some $(\pi_{\mathbf{V}',i})^{\ast} L_{\underline{\nu}'}$. We argue by induction on the length of $\underline{\nu}$ and $n$. 
	
	Without loss of generality, we can replace the flag type $\underline{\nu}=( i_{1}^{a_{1}},i_{2}^{a_{2}},\cdots ,i_{k}^{a_{k}})$ by $$(i_{1},\cdots,i_{1},i_{2},\cdots,i_{2},\cdots,i_{k},\cdots,i_{k})$$ such that each $i_{l}$ appears repeatedly for $a_{l}$ times for $1\leqslant l\leqslant k$, then $L_{\underline{\nu}}=\mathcal{F}_{i_{1}} L_{\underline{\nu}'}$ for $\underline{\nu}'= (i_{1},i_{1},\cdots,i_{k})$ such that $i_{1}$ appears for $a_{1}-1$ times and the other $i_{l}$ appear repeatedly for $a_{l}$ times for $1<l\leqslant k$. 
	Assume $|\mathbf{V}'|=|\mathbf{V}|-ni,|\mathbf{V}''|=|\mathbf{V}|-i_{1}$ and $|\mathbf{V}'''|=|\mathbf{V}|-ni-i_{1}$, then up to shifts, we have
	\begin{equation*}
		\begin{split}
			\tilde{\mathcal{E}}^{(n)}_{i}	(\pi_{\mathbf{V},i})^{\ast} L_{\underline{\nu}}  = &\tilde{\mathcal{E}}^{(n)}_{i}	(\pi_{\mathbf{V},i})^{\ast} \mathcal{F}_{i_{1}} L_{\underline{\nu}} \\
			\cong & \tilde{\mathcal{E}}^{(n)}_{i}\tilde{\mathcal{F}}_{i_{1}}(\pi_{\mathbf{V}'',i})^{\ast} L_{\underline{\nu}'},
		\end{split}
	\end{equation*}
	see formula (\ref{tilde F and F}). 
	
	If $i_{1} \neq i$, by Lemma \ref{lemma c2},  $\tilde{\mathcal{E}}^{(n)}_{i}	(\pi_{\mathbf{V},i})^{\ast} L_{\underline{\nu}}  \cong   \tilde{\mathcal{F}}_{i_{1}} \tilde{\mathcal{E}}^{(n)}_{i}(\pi_{\mathbf{V}'',i})^{\ast} L_{\underline{\nu}'}$. 
	
	Otherwise, by Lemma \ref{lemma c1}, $\tilde{\mathcal{E}}^{(n)}_{i}	(\pi_{\mathbf{V},i})^{\ast} L_{\underline{\nu}}$ is either isomorphic to the direct sum of  $\tilde{\mathcal{F}}_{i_{1}} \tilde{\mathcal{E}}^{(n)}_{i}(\pi_{\mathbf{V}'',i})^{\ast} L_{\underline{\nu}'}$ and some shifts of $\tilde{\mathcal{E}}^{(n-1)}_{i}(\pi_{\mathbf{V}'',i})^{\ast} L_{\underline{\nu}'}$ or isomorphic to a direct summand of $\tilde{\mathcal{F}}_{i_{1}} \tilde{\mathcal{E}}^{(n)}_{i}(\pi_{\mathbf{V}'',i})^{\ast} L_{\underline{\nu}'}$. By inductive hypothesis and the fact that $\tilde{\mathcal{F}}_{i_{1}}$ sends an object of $(\pi_{\mathbf{V}''',i})^{\ast}(\mathcal{Q}^{0}_{\mathbf{V}'''})$ to an object of $(\pi_{\mathbf{V}',i})^{\ast}(\mathcal{Q}^{0}_{\mathbf{V}'})$, we finish the proof.
\end{proof}

\begin{corollary}\label{commute3}
	The functor $\mathcal{E}^{(n)}_{i}:\mathcal{D}^{b}_{G_{\mathbf{V}}}(\mathbf{E}_{\mathbf{V},\Omega})/\mathcal{N}_{\mathbf{V},i} \rightarrow \mathcal{D}^{b}_{G_{\mathbf{V}'}}(\mathbf{E}_{\mathbf{V}',\Omega})/\mathcal{N}_{\mathbf{V'},i}$ sends an object of $\mathcal{Q}^{0}_{\mathbf{V}}$ to an object of $\mathcal{Q}^{0}_{\mathbf{V}'}$. 
    Moreover, for $i \neq j$ and graded spaces such that $|\mathbf{V}''|+ni=|\mathbf{V}|+j$, there is an isomorphism of functors $\mathcal{Q}^{0}_{\mathbf{V}}/\mathcal{N}_{\mathbf{V},i} \rightarrow \mathcal{Q}^{0}_{\mathbf{V}''}/\mathcal{N}_{\mathbf{V}'',i}$ as follows
	\begin{equation*}
		\mathcal{E}^{(n)}_{i}\mathcal{F}_{j}=\mathcal{F}_{j}\mathcal{E}^{(n)}_{i}.
	\end{equation*}
   For $i=j$ and graded spaces such that $|\mathbf{V}''|+(n-1)i=|\mathbf{V}|$, let $N=2\nu_{i}-\tilde{\nu}_{i}-n+1$, then there is an isomorphism of functors $\mathcal{Q}^{0}_{\mathbf{V}}/\mathcal{N}_{\mathbf{V},i} \rightarrow \mathcal{Q}^{0}_{\mathbf{V}''}/\mathcal{N}_{\mathbf{V}'',i}$ as follows
	\begin{equation*}
		\mathcal{E}^{(n)}_{i}\mathcal{F}_{i} \oplus \bigoplus\limits_{0\leqslant m \leqslant N-1} \mathcal{E}^{(n-1)}_{i} [N-1-2m] \cong \mathcal{F}_{i}\mathcal{E}^{(n)}_{i} \oplus \bigoplus\limits_{0\leqslant m \leqslant -N-1} \mathcal{E}^{(n-1)}_{i} [-2m-N-1].
	\end{equation*}
	 More precisely, in this case we have
	\begin{align*}
		&{\mathcal{E}}^{(n)}_{i}{\mathcal{F}}_{i}\cong {\mathcal{F}}_{i}{\mathcal{E}}^{(n)}_{i}, &\textrm{if}\ N=0;\\
		&{\mathcal{E}}^{(n)}_{i}{\mathcal{F}}_{i}\oplus\bigoplus_{0\leqslant m\leqslant N-1}\mathcal{E}^{(n-1)}_{i} [N-1-2m]\cong {\mathcal{F}}_{i}{\mathcal{E}}^{(n)}_{i},&\textrm{if}\  N\geqslant 1;\\
		&{\mathcal{E}}^{(n)}_{i}\tilde{\mathcal{F}}_{i}\cong {\mathcal{F}}_{i}{\mathcal{E}}^{(n)}_{i} \oplus \bigoplus\limits_{0\leqslant m \leqslant -N-1} \mathcal{E}^{(n-1)}_{i} [-2m-N-1], &\textrm{if}\ N\leqslant -1.
	\end{align*}
\end{corollary}
\begin{proof}
	Notice that $\pi_{\mathbf{V},i}$ is a trivial vector bundle, the functors $(\pi_{\mathbf{V},i})^{\ast}[d_{i}\nu_{i}] $ and $(\pi_{\mathbf{V},i})_{!}[d_{i}\nu_{i}]$ form quasi-inverse equivalences between $\mathcal{Q}^{0}_{\mathbf{V}} $ and $ (\pi_{\mathbf{V},i})^{\ast}(\mathcal{Q}^{0}_{\mathbf{V}})$.
	By definition, we have $$\mathcal{E}^{(n)}_{i}( \mathcal{Q}^{0}_{\mathbf{V}})= (\pi_{\mathbf{V}',i})_{!}[d_{i}\nu'_{i}]\tilde{\mathcal{E}}^{(n)}_{i} (\pi_{\mathbf{V},i})^{\ast}[d_{i}\nu_{i}]( \mathcal{Q}^{0}_{\mathbf{V}}).$$ By Corollary \ref{corollary tilde E}, objects in $\tilde{\mathcal{E}}^{(n)}_{i} (\pi_{\mathbf{V},i})^{\ast}[d_{i}\nu_{i}]( \mathcal{Q}^{0}_{\mathbf{V}})$ belong to $(\pi_{\mathbf{V}',i})^{\ast}( \mathcal{Q}^{0}_{\mathbf{V}'})$, and the first statement follows. The commutative relations follow from Lemma \ref{lemma c1} and \ref{lemma c2}.
\end{proof}

\begin{remark}
	Obviously, the commutative diagrams and techniques used in our proof are similar to those which have appeared in H.Zheng's work \cite{MR3200442}.
\end{remark}

\subsection{The integrable highest weight modules}

In this subsection, we fix an orientation $\Omega$.

\begin{lemma}\label{Lemma 15}
	For a fixed $i \in I$ and any $j\neq i$ in $I$, the functor $\mathcal{E}^{(n)}_{i}:\mathcal{Q}^{0}_{\mathbf{V}}/\mathcal{N}_{\mathbf{V},i} \rightarrow \mathcal{Q}^{0}_{\mathbf{V}'}/\mathcal{N}_{\mathbf{V}',i}$ sends objects of $\mathcal{F}_{\Omega^{j},\Omega^{i}}(\mathcal{N}_{\mathbf{V},j}) $ to objects of $\mathcal{F}_{\Omega^{j},\Omega^{i}}(\mathcal{N}_{\mathbf{V}',j}) $.
\end{lemma}
\begin{proof}
	By Proposition \ref{rt}, it suffices to show $\mathcal{E}^{(n)}_{i}$ sends complexes of the form $L_{(\underline{\nu},j^{d})},d>d_{j}$ to  direct sums of some complexes $L_{(\underline{\nu}',j^{d'})},d'>d_{j}$. Using Corollary \ref{commute3}, we can argue by induction on the length of $\underline{\nu}$ and $n$ to get a proof.
\end{proof}

\begin{lemma}\label{Lemma 16}
	The functors $\mathcal{E}^{(n)}_{i}$ for $n\geq 1$ satisfy the following relation
	\begin{equation*}
		\bigoplus \limits_{0 \leqslant m < n } \mathcal{E}^{(n)}_{i}[n-1-2m] \cong \mathcal{E}^{(n-1)}_{i}\mathcal{E}_{i},\  n \geq 2,
	\end{equation*}
	as endofunctors of the localization $\bigoplus\limits_{\mathbf{V}}\mathcal{Q}^{0}_{\mathbf{V}}/\mathcal{N}_{\mathbf{V},i}$.
\end{lemma}

\begin{proof}
	Take graded spaces $\mathbf{V},\mathbf{V}'$ and $\mathbf{V}''$ such that $|\mathbf{V}|-|\mathbf{V}'|=i$ and $|\mathbf{V}|-|\mathbf{V}''|=ni$, consider the following commutative diagram
	\[
	\xymatrix@C=0.5em@R=3ex{
		\dot{\mathbf{E}}_{\mathbf{V},i} \times \mathbf{Flag}(\nu_{i}-n,\nu_{i},\tilde{\nu}_{i}) \ar[rr]^{q''_{2}} \ar[dd]_{q_{1}''} &  & \dot{\mathbf{E}}_{\mathbf{V},i} \times \mathbf{Grass}(\nu_{i}-n, \tilde{\nu}_{i})  \\
		&  \dot{\mathbf{E}}_{\mathbf{V},i} \times \mathbf{Flag}(\nu_{i}-n,\nu_{i}-1,\nu_{i},\tilde{\nu}_{i}) \ar[ul]^{r} \ar[r]^-{\pi} \ar[d]^{\pi'} & \dot{\mathbf{E}}_{\mathbf{V},i} \times \mathbf{Flag}(\nu_{i}-n,\nu_{i}-1,\tilde{\nu}_{i}), \ar[u]_{q'_{2}} \ar[d]^{q'_{1}}\\
		\dot{\mathbf{E}}_{\mathbf{V},i} \times \mathbf{Grass}(\nu_{i},\tilde{\nu}_{i}) & \ \dot{\mathbf{E}}_{\mathbf{V},i} \times \mathbf{Flag}(\nu_{i}-1,\nu_{i},\tilde{\nu}_{i})\ar[l]_{q_{1}} \ar[r]^-{q_{2}} &  \dot{\mathbf{E}}_{\mathbf{V},i} \times \mathbf{Grass}(\nu_{i}-1, \tilde{\nu}_{i}) 
	}
	\]
	where the morphisms are obvious forgetting maps. By base change, we have 
	\begin{equation*}
		\begin{split}
			(q'_{2})_{!}(q'_{1})^{\ast}(q_{2})_{!}(q_{1})^{\ast} \cong & (q'_{2} \pi)_{!}( q_{1} \pi' )^{\ast} \\
			\cong &  (q''_{2})_{!} r_{!}r^{\ast} (q''_{1})^{\ast} \\
			\cong & \bigoplus \limits_{0 \leqslant m < n } (q''_{2})_{!}(q''_{1})^{\ast}[n-1-2m ],
		\end{split}
	\end{equation*}
	where the last isomorphism holds by the projection formula, since $r$ is a trivial fiber bundle with fiber isomorphic to $\mathbb{P}^{(n-1)}$. Compose them with quasi-inverse equivalences $(\phi_{\mathbf{V},i})^{\ast},(\phi_{\mathbf{V},i})_{\flat}$ and $(\pi_{\mathbf{V},i})^{\ast},(\pi_{\mathbf{V},i})_{!}$, we get a proof.
\end{proof}


\begin{definition}	
For $n\in \mathbb{N}$ and $i\in I$, we define functors between $\mathcal{L}_{\mathbf{V}}(\Lambda)=\mathcal{Q}^{0}_{\mathbf{V}}/\mathcal{N}_{\mathbf{V}}$ as follows.

(1) For graded spaces $\mathbf{V}$ and $\mathbf{V}'$ such that $|\mathbf{V}|=|\mathbf{V}'|+ni$, define $E^{(n)}_{i}:\mathcal{L}_{\mathbf{V}}(\Lambda) \rightarrow \mathcal{L}_{\mathbf{V}'}(\Lambda)$ via
	\begin{equation*}
		E^{(n)}_{i}=\mathcal{F}_{\Omega^{i},\Omega}\mathcal{E}^{(n)}_{i}\mathcal{F}_{\Omega,\Omega^{i}}.
	\end{equation*}

(2) For graded spaces $\mathbf{V}$ and $\mathbf{V}''$ such that $|\mathbf{V}|+ni=|\mathbf{V}''|$, define $F^{(n)}_{i}:\mathcal{L}_{\mathbf{V}}(\Lambda) \rightarrow \mathcal{L}_{\mathbf{V}''}(\Lambda)$ via
	\begin{equation*}
		F^{(n)}_{i}=\mathcal{F}_i^{(n)}.
	\end{equation*}

(3) Define $K_{i}: \mathcal{L}_{\mathbf{V}}(\Lambda) \rightarrow \mathcal{L}_{\mathbf{V}}(\Lambda)$ via
	\begin{equation*}
		K_{i}=\textrm{Id}\ [\bar{\nu}_{i}-\nu_{i}].
	\end{equation*}
In particular, we denote by $E_{i}=E^{(1)}_{i}$ and $F_{i}=F^{(1)}_{i}$. Note that $K_{i}$ is invertible, and we denote $K^{-}_{i}=\textrm{Id}\ [\nu_{i}-\bar{\nu}_{i}] $ be its inverse.
\end{definition}

By Corollary \ref{commute3} and Lemma \ref{Lemma 15}, the functors $E_i^{(n)}$ are well-defined. By definitions and Corollary \ref{commute3}, we obtain the following proposition.
\begin{proposition}\label{relation1}
	The functors $E_{i}$, $F_{i}$ and $K_{i},i\in I$ satisfy the following relations
	\begin{equation*} 
		K_{i}K_{j}=K_{j}K_{i},
	\end{equation*}
	\begin{equation*}
		E_{i}K_{j}=K_{j}E_{i}[-a_{j,i}],
	\end{equation*}
	\begin{equation*}
	F_{i}K_{j}=K_{j}F_{i}[a_{i,j}],
	\end{equation*}
	\begin{equation*}
		E_{i}F_{j}=F_{j}E_{i}\ \textrm{for}\ i \neq j,
	\end{equation*}
	\begin{equation*}
		E_{i}F_{i} \oplus \bigoplus\limits_{0\leq m \leq N-1} Id[N-1-2m] \cong F_{i}E_{i} \oplus \bigoplus\limits_{0\leq m \leq -N-1} Id[-2m-N-1]: \mathcal{L}_{\mathbf{V}}(\Lambda) \rightarrow \mathcal{L}_{\mathbf{V}}(\Lambda),
	\end{equation*}
as endofunctors of $\mathcal{L}(\Lambda)= \coprod \limits_{\mathbf{V}} \mathcal{L}_{\mathbf{V}}(\Lambda)$, where $N=\nu_{i}-\bar{\nu}_{i}$.
\end{proposition}


By Theorem \ref{Lussztig1}, we also have the following proposition.
\begin{proposition}\label{relation2}
	 The functors $F_{i}$ for $i\in I$ satisfy the following relations
	\begin{equation*}
		\bigoplus\limits_{0\leq m \leq 1- a_{i,j},m\ is\ odd}F^{(m)}_{i}F_{j}F^{(1-a_{i,j}-m)}_{i}\cong 	\bigoplus\limits_{0\leq m \leq 1- a_{i,j},m\ is \ even}F^{(m)}_{i}F_{j}F^{(1-a_{i,j}-m)}_{i},
	\end{equation*}
 \begin{equation*}
	\bigoplus \limits_{0 \leq m < n } F^{(n)}_{i}[n-1-2m] \cong F^{(n-1)}_{i}F_{i}\ \textrm{for}\ n \geq 2,
\end{equation*}
as endofunctors of $\mathcal{L}(\Lambda)= \coprod \limits_{\mathbf{V}} \mathcal{L}_{\mathbf{V}}(\Lambda)$.
\end{proposition}
\begin{proof}
	Notice that for $\underline{v}=(i^{a_{1}}_{1}, \cdots i^{a_{k}}_{k})$, by definition we have $ F^{a_{1}}_{i_{1}} F^{a_{2}}_{i_{2}}  \cdots  F^{a_{k}}_{i_{k}} =\mathbf{Ind}^{\mathbf{V}}_{\mathbf{V}',\mathbf{V}''}(L_{\underline{v}}\boxtimes -)$. Then the proposition follows from Theorem \ref{Lussztig1}.
\end{proof}

\begin{proposition}\label{relation3}
	The functors  $E_{i}$ for $i\in I$ satisfy the following relations
	\begin{equation*}
		\bigoplus\limits_{0\leq m \leq 1- a_{i,j},m~odd}E^{(m)}_{i}E_{j}E^{(1-a_{i,j}-m)}_{i}\cong 	\bigoplus\limits_{0\leq m \leq 1- a_{i,j},m~even}E^{(m)}_{i}E_{j}E^{(1-a_{i,j}-m)}_{i} ;
	\end{equation*}
	\begin{equation*}
		\bigoplus \limits_{0 \leq m < n } E^{(n)}_{i}[n-1-2m] \cong E^{(n-1)}_{i}E_{i}, n \geq 2,
	\end{equation*}
as endofunctors of $\mathcal{L}(\Lambda)= \coprod \limits_{\mathbf{V}} \mathcal{L}_{\mathbf{V}}(\Lambda)$.
\end{proposition}
\begin{proof}
	The second isomorphism follows from Lemma 3.15. We only need to prove the first one. For any $L \in \mathcal{Q}^{0}_{\mathbf{V}}/\mathcal{N}_{\mathbf{V}}$, let $N= 2\nu_{i}-\tilde{\nu}_{i}$. By Corollary \ref{commute3}, if $m \geqslant 1+N$, we have
	 	\begin{equation}
	 		\begin{split}
	 				 	E^{(1-a_{i,j}-m)}_{i}E_{j}E^{(m)}_{i}F_{i}(L)&\cong F_{i}E^{(1-a_{i,j}-m)}_{i}E_{j}E^{(m)}_{i}(L) \\ & \oplus\bigoplus_{0\leqslant l \leqslant -N-2+m}E^{(1-a_{i,j}-m)}_{i}E_{j}E^{(m-1)}_{i}(L) [-N-2+m-2l] \\
	 			&\oplus\bigoplus_{0\leqslant l \leqslant -N+m-1}E^{(1-a_{i,j}-m-1)}_{i}E_{j}E^{(m)}_{i}(L) [-N+m-1-2l].
	 		\end{split} 
	   \end{equation}
   Otherwise, if $ m < 1+N$, or equivalently $m \leqslant N$, we have
   	\begin{equation}
   	\begin{split}
	F_{i}E^{(1-a_{i,j}-m)}_{i}E_{j}E^{(m)}_{i}(L)&\cong E^{(1-a_{i,j}-m)}_{i}E_{j}E^{(m)}_{i}F_{i}(L) \\ &  \oplus\bigoplus_{0\leqslant l \leqslant N-2-m}E^{(1-a_{i,j}-m)}_{i}E_{j}E^{(m-1)}_{i}(L) [N-2-m-2l] \\
    &\oplus\bigoplus_{0\leqslant l \leqslant N-m-1}E^{(1-a_{i,j}-m-1)}_{i}E_{j}E^{(m)}_{i}(L) [N-m-1-2l].
   	\end{split} 
   \end{equation}
   Let $M= 2\nu_{j}-\tilde{\nu}_{j}-m a_{i,j}$. By Corollary \ref{commute3}, if $M \leq 0$, we have
   \begin{equation}
   	\begin{split}
   		E^{(1-a_{i,j}-m)}_{i}E_{j}E^{(m)}_{i}F_{j}(L)&\cong F_{j}E^{(1-a_{i,j}-m)}_{i}E_{j}E^{(m)}_{i}(L) \\ 
   		&\oplus\bigoplus_{0\leqslant l \leqslant -M-1}E^{(1-a_{i,j}-m)}_{i}E^{(m)}_{i}(L) [-M-1-2l].
   	\end{split} 
   \end{equation}
   If $M > 0$, we have
    \begin{equation}
   	\begin{split}
   		F_{j}E^{(1-a_{i,j}-m)}_{i}E_{j}E^{(m)}_{i}(L)&\cong E^{(1-a_{i,j}-m)}_{i}E_{j}E^{(m)}_{i}F_{j}(L) \\
   		&\oplus\bigoplus_{0\leqslant l \leqslant M-1}E^{(1-a_{i,j}-m)}_{i}E^{(m)}_{i}(L) [M-1-2l].
   	\end{split} 
   \end{equation}

 For $k \neq i,j$, we have 
     \begin{equation}
 	\begin{split}
 		F_{k}E^{(1-a_{i,j}-m)}_{i}E_{j}E^{(m)}_{i}(L)&\cong E^{(1-a_{i,j}-m)}_{i}E_{j}E^{(m)}_{i}F_{k}(L).
 	\end{split} 
 \end{equation}
    
	 We claim that  $$\bigoplus\limits_{0\leqslant m \leqslant 1- a_{i,j},m~odd}E^{(m)}_{i}E_{j}E^{(1-a_{i,j}-m)}_{i}(L_{\underline{\nu}})\cong 	\bigoplus\limits_{0\leqslant m \leqslant 1- a_{i,j},m~even}E^{(m)}_{i}E_{j}E^{(1-a_{i,j}-m)}_{i}(L_{\underline{\nu}})$$ for any flag type $\underline{\nu}$. We replace $\underline{\nu}=( i_{1}^{a_{1}},i_{2}^{a_{2}},\cdots ,i_{k}^{a_{k}})$ by $\underline{\nu}'=(i_{1},\cdots,i_{1},i_{2},\cdots,i_{2},\cdots,i_{k},\cdots,i_{k}).$ 
	 Then using the commutation relations above, we can argue by induction on length of $\underline{\nu}'$ to show that $$\bigoplus\limits_{0\leqslant m \leqslant 1- a_{i,j},m~odd}E^{(m)}_{i}E_{j}E^{(1-a_{i,j}-m)}_{i}(L_{\underline{\nu}'})\cong 	\bigoplus\limits_{0\leqslant m \leqslant 1- a_{i,j},m~even}E^{(m)}_{i}E_{j}E^{(1-a_{i,j}-m)}_{i}(L_{\underline{\nu}'}).$$
	 Since $L_{\underline{\nu}}$ is isomorphic to a direct sum of shifts of $L_{\underline{\nu}'}$, the claim is proved.
	 
	 By Proposition \ref{lt}, for any simple perverse sheaf $L$, we can find  families of flag types $\underline{\tau}$ and $\underline{\omega}$ such that  $$L \oplus \bigoplus \limits_{\tau,n }L_{\underline{\tau}}^{\oplus N(\tau,n) }[n] \cong \bigoplus \limits_{\omega,n }L_{\underline{\omega}}^{\oplus N(\omega,n) }[n], $$
	 where $N(\tau,n)$ and $N(\omega,n)$ are  multiplicities. Using the claim, we can see that $$\bigoplus\limits_{0\leqslant m \leqslant 1- a_{i,j},m~odd}E^{(m)}_{i}E_{j}E^{(1-a_{i,j}-m)}_{i}(L)\cong 	\bigoplus\limits_{0\leqslant m \leqslant 1- a_{i,j},m~even}E^{(m)}_{i}E_{j}E^{(1-a_{i,j}-m)}_{i}(L)$$
	 for any perverse sheaf $L$.
\end{proof}


\begin{proposition}
	The functors $F^{(n)}_{i},E^{(n)}_{i},K_{i}$ for $n\in \mathbb{N}, i \in I$ and Verdier duality functor $\mathbf{D}$ satisfy the following relations
	\begin{equation*}
		F^{(n)}_{i}\mathbf{D}\cong\mathbf{D}F^{(n)}_{i},
	\end{equation*}
	\begin{equation*}
		E^{(n)}_{i}\mathbf{D}\cong\mathbf{D}E^{(n)}_{i},
	\end{equation*}
	\begin{equation*}
		K_{i}\mathbf{D}\cong\mathbf{D}(K_{i})^{-1}.
	\end{equation*}
\end{proposition}
\begin{proof}
	The first relation holds, since the induction functors commute with the  Verdier duality functor. The last relation can be easily checked by definition. We only prove the second relation for $n=1$, the other case can be proved by a similar argument. Notice that $E_{i}$ can be written as 
	\begin{equation*}
	E_{i}=((\pi_{\mathbf{V}',i})_{!}[d_{i}\nu'_{i}]) (j^{0}_{\mathbf{V}',i})_{!} ((\phi_{\mathbf{V}',i})^{\ast}[(\nu')^{2}_{i}]) (q_{2})_{!}((q_{1})^{\ast}[\nu_{i}-1]) ((\phi_{\mathbf{V},i})_{\flat}[-\nu^{2}_{i}])(j^{0}_{\mathbf{V},i})^{\ast}((\pi_{\mathbf{V},i})^{\ast}[d_{i}\nu_{i}]).
	\end{equation*}
Since each functor in the expression commutes with the Verdier duality functor, so does $E_{i}$. 
\end{proof}
\begin{definition}
	Define $\mathcal{K}_{0}(\Lambda)=\mathcal{K}_{0}(\mathcal{L}(\Lambda))$ to be the Grothendieck group of $\mathcal{L}(\Lambda)$, which can be endowed with a $\mathcal{A}$-module structure. More precisely, $\mathcal{K}_{0}(\Lambda)$ is the $\mathcal{A}$-module spanned by objects $[L]$ in $\mathcal{L}(\Lambda)$ subject to relations
	\begin{equation*}
		[X \oplus Y]=[X]+[Y],
	\end{equation*}
	\begin{equation*}
		[X[1]]=v[X].
	\end{equation*}
    Similarly, we denote the Grothendieck group of $\mathcal{L}_{\mathbf{V}}(\Lambda)$ by $\mathcal{K}_{0,|\mathbf{V}|}(\Lambda)$.
    
	The functors $E^{(n)}_{i},F^{(n)}_{i},K^{\pm}_{i}$ for $n\in \mathbb{N},i \in I$ induces $\mathcal{A}$-linear operators on $\mathcal{K}_{0}(\Lambda)$, and we still denote these operators by $E^{(n)}_{i},F^{(n)}_{i},K^{\pm}_{i}$ for $n\in \mathbb{N},i \in I$, respectively.
\end{definition}


\begin{theorem}\label{thm1}
The linear operators induced by functors $E^{(n)}_{i},F^{(n)}_{i},K^{\pm}_{i}$ for $n\in \mathbb{N},i \in I$ defines a $_{\mathcal{A}}\mathbf{U}$-module structure on $\mathcal{K}_{0}(\Lambda)$ which is isomorphic to the integrable highest weight $_{\mathcal{A}}\mathbf{U}$-module ${_{\mathcal{A}}L(\Lambda)}$ via the canonical isomorphism
	\begin{equation*}
		\varsigma^{\Lambda}:\mathcal{K}_{0}(\Lambda) \rightarrow {_{\mathcal{A}}L(\Lambda)}.
	\end{equation*}
	such that $\varsigma^{\Lambda}$ sends the constant sheaf $[L_0]=[\overline{\mathbb{Q}}_{l}]$ on
	$\mathbf{E}_{0,\Omega}$
	to the highest weight vector $v_{\Lambda}\in {_{\mathcal{A}}L(\Lambda)}$.
Moreover, the set
	$\bigcup\limits_{\mathbf{V}}\{\varsigma^{\Lambda}([L])|L$ is a nonzero simple perverse sheaf in $\mathcal{L}_{\mathbf{V}}(\Lambda)\}$ form a bar-invariant $\mathcal{A}$-basis of ${_{\mathcal{A}}L(\Lambda)}$.
\end{theorem}

\begin{proof}
	 By Proposition \ref{relation1}, \ref{relation2} and \ref{relation3}, $\mathcal{K}_{0}(\Lambda)$ is a $\mathbf{U}$-module. By Lemma \ref{lkey} and Theorem \ref{Lussztig1}, for any simple perverse sheaf $L$, its image $[L]\in  \mathcal{K}_{0}(\Lambda)$ can be written as a $\mathcal{A}$-linear combination of some $[L_{\underline{v}}]= [F^{(a_{1})}_{i_{1}} F^{(a_{2})}_{i_{2}}  \cdots  F^{(a_{k})}_{i_{k}}L_{0}]$, hence $\mathcal{K}_{0}(\Lambda)$ is a highest weight module, where $[L_0]$ is the highest weight vector.
	 
	 It remains to prove that $\mathcal{K}_{0}(\Lambda)$ is integrable. Let $L$ be a simple perverse sheaf in $\mathcal{Q}^{0}_{\mathbf{V}}$. On the one hand, for $N> \nu_{i}$, we have $(E_{i})^{N}([L])=0$. On the other hand, note that if $\nu_{i}-\sum\limits_{h'=i}\nu_{h''}>{\rm{dim}}\mathbb{C}^{d_{i}}$, then $\mathbf{E}^{\geq d_{i}+1 }_{\mathbf{V},\Omega^{i}}=\mathbf{E}_{\mathbf{V},\Omega^{i}}$, so any objects of $\mathcal{Q}^{0}_{\mathbf{V}}$ belong to $\mathcal{N}_{\mathbf{V},i}$ and $\mathcal{L}_{\mathbf{V}}(\Lambda)=0$. For large enough $N$,  we have $(F_{i})^{N}([L])\in \mathcal{L}_{\mathbf{V}'}(\Lambda)$, where $\mathbf{V}'$ satisfies the $\nu'_{i}-\sum\limits_{h'=i}\nu'_{h''}>{\rm{dim}}\mathbb{C}^{d_{i}}$, and so $(F_{i})^{N}([L])=0$.
	 
	It is clear that nonzero simple perverse sheaves form a bar-invariant $\mathcal{A}$-basis of ${_{\mathcal{A}}L(\Lambda)}$.
\end{proof}

\begin{remark}
	(1) Notice that the nonzero simple objects in $\mathcal{Q}^{0}_{\mathbf{V}}/\mathcal{N}_{\mathbf{V}}$ are exactly the simple perverse sheaves in $\mathcal{Q}^{0}_{\mathbf{V}}$ but not in  $\mathcal{N}_{\mathbf{V}}$. We denote the set of these simple perverse sheaves by $\mathcal{P}_{\mathbf{V}} \backslash \mathcal{N}_{\mathbf{V}}$, then one can see that these simple objects form a basis of  ${_{\mathcal{A}}L(\Lambda)}$ and 
	\begin{equation*}
		|\mathcal{P}_{\mathbf{V}}\backslash \mathcal{N}_{\mathbf{V}}|={\rm{dim}}_{\mathbb{Q}(v)} L(\Lambda).
	\end{equation*}
	
	(2) Under the identification of $\mathcal{K}=\bigoplus\limits_{\mathbf{V}}K_0(\mathcal{Q}_{\mathbf{V}})$ with ${_{\mathcal{A}}\mathbf{U}}^{-}$, a Lusztig's simple perverse sheaf $L \in \mathcal{P}_{\mathbf{V}}$ is a zero object in $\mathcal{Q}^{0}_{\mathbf{V}}/\mathcal{N}_{\mathbf{V}}$ if and only if it is contained in some $\mathcal{N}_{\mathbf{V},i}$, if and only if its image $[L]$ belongs to the $\mathcal{A}$-submodule $\sum\limits_{i \in I} {_{\mathcal{A}}\mathbf{U}}^{-} f_{i}^{(\langle \Lambda,\alpha_{i}^{\vee} \rangle+1 )} $ by Proposition \ref{rt}. Let $\mathcal{I}$ be the $\mathcal{A}$-submodule of $\mathcal{K}$ spanned by simple objects in $\mathcal{N}_{\mathbf{V},i}, i\in I$ and $\tilde{\pi}$ be the canonical projection $\mathcal{K} \rightarrow \mathcal{K}/\mathcal{I} \cong \mathcal{K}_{0}(\Lambda)$, then we have the following commutative diagram and recover Lusztig's construction of canonical basis for $L(\Lambda)$.
	\[
	\xymatrix{
		\mathcal{K} \ar[d]^{\varsigma} \ar[r]^{\pi}
 		&  \mathcal{K}_{0}(\Lambda) \ar[d]^{\varsigma^{\Lambda}}
		 \\	
		{_{\mathcal{A}}\mathbf{U}}^{-} \ar[r]^{\tilde{\pi}}
		& {_{\mathcal{A}}L(\Lambda)},
	}
	\]
where $\pi:\mathcal{K}\rightarrow \mathcal{K}_{0}(\Lambda)$ is the natural projection induced by $\mathcal{Q}^0_{\mathbf{V}}\rightarrow \mathcal{Q}^0_{\mathbf{V}}/\mathcal{N}_{\mathbf{V}}$ for various $\mathbf{V}$. In particular, $\bigcup\limits_{\mathbf{V}}\{\varsigma^{\Lambda}([L])|L$ is a nonzero simple perverse sheaf in $\mathcal{L}_{\mathbf{V}}(\Lambda)\}$ is exactly the canonical basis of ${_{\mathcal{A}}L(\Lambda)}$.
\end{remark}

\subsection{Compare the functor $E_{i}$ with derivation functors}

In this section, we construct a split exact sequence which is analogy to the exact sequence in \cite[Theorem 4.7]{MR2995184}. We fix a vertex $i\in I$ and an orientation $\Omega$ such that $i$ is a source. 

\begin{definition}
	(1) Define $\hat{\mathcal{Q}}^{0}_{\mathbf{V}}$ to be the subcatgeory of $\mathcal{D}(\mathbf{E}_{\mathbf{V},\Omega})$ whose objects are direct sums of Lusztig's sheaves such that multiplicity of each simple summand $A[n]$ is finite for any $A \in \mathcal{P}_{\mathbf{V}}, n \in \mathbb{Z}$, and whose morphisms are linear combinations of isomorphisms.
	
	(2) Define the localization $\hat{\mathcal{Q}}^{0}_{\mathbf{V}}/\mathcal{N}_{\mathbf{V},i}$ to be the additive quotient of $\hat{\mathcal{Q}}^{0}_{\mathbf{V}}$ by the subcategory consisting of objects in $\mathcal{N}_{\mathbf{V},i}$, and define the global localization $\hat{\mathcal{Q}}^{0}_{\mathbf{V}}/\mathcal{N}_{\mathbf{V}}$ to be the additive quotient of $\hat{\mathcal{Q}}^{0}_{\mathbf{V}}$ by the subcategory consisting of objects in some $\mathcal{F}_{\Omega^{i},\Omega}(\mathcal{N}_{\mathbf{V},i}), i\in I$.
\end{definition}
It's easy to see that $\mathcal{Q}^{0}_{\mathbf{V}}$, $\mathcal{Q}^{0}_{\mathbf{V}}/\mathcal{N}_{\mathbf{V},i}$ and $\mathcal{Q}^{0}_{\mathbf{V}}/\mathcal{N}_{\mathbf{V}}$  are  full subcategories of $\hat{\mathcal{Q}}^{0}_{\mathbf{V}}$, $\hat{\mathcal{Q}}^{0}_{\mathbf{V}}/\mathcal{N}_{\mathbf{V},i}$ and $\hat{\mathcal{Q}}^{0}_{\mathbf{V}}/\mathcal{N}_{\mathbf{V}}$ respectively.

Let $\hat{\mathcal{K}}_{\mathbf{V}}$ be the Grothendieck group of
$\hat{
\mathcal{Q}}^{0}_{\mathbf{V},\Omega}$ and $\hat{\mathcal{K}}=\bigoplus\limits_{\mathbf{V}} \hat{\mathcal{K}}_{\mathbf{V}}$. They have  $\mathbb{Z}[[v,v^{-1}]]$-module structures and the set $\bigcup\limits_{\mathbf{V}}\{[L]|L\in \mathcal{P}_{\mathbf{V},\Omega}\}$ forms a bar-invariant $\mathbb{Z}[[v,v^{-1}]]$-basis of $\hat{\mathcal{K}}$. 

Let  ${\rm{H}}^{\ast}(\mathbb{P}^{\infty})$ be the cohomology of the infinite projective variety, then following the proof of \cite[Lemma 12.3.6]{MR1227098},  $${\rm{H}}^{\ast}(\mathbb{P}^{\infty})\cong \bigoplus\limits_{m \geqslant 0  } \overline{\mathbb{Q}}_{l} [-2m],$$
$$\mathbf{D}{\rm{H}}^{\ast}(\mathbb{P}^{\infty})\cong \bigoplus\limits_{m \geqslant 0  } \overline{\mathbb{Q}}_{l} [2m].$$

\begin{definition}
	For graded spaces $\mathbf{V},\mathbf{V}',\mathbf{V}''$ such that $|\mathbf{V}|=|\mathbf{V}'|+i,|\mathbf{V}''|=i$ and $\mathbf{V}=\mathbf{V}'\oplus \mathbf{V}''$. 
	
	(1) define the functor $\hat{\mathcal{R}}^{\Lambda}_{i}: \mathcal{Q}^{0}_{\mathbf{V}} \rightarrow \hat{\mathcal{Q}}^{0}_{\mathbf{V}'}$ by
	\begin{equation*}
		\hat{\mathcal{R}}^{\Lambda}_{i}(L)=\mathbf{Res}^{\mathbf{V}}_{\mathbf{V}',\mathbf{V}''}(L)\otimes (\mathbf{D}{\rm{H}}^{\ast}(\mathbb{P}^{\infty})) [\langle \Lambda,\alpha^{\vee}_{i} \rangle+1].
	\end{equation*}

	
	(2) Define the functor ${_{i}\hat{\mathcal{R}}^{\Lambda}}: \mathcal{Q}^{0}_{\mathbf{V}} \rightarrow \hat{\mathcal{Q}}^{0}_{\mathbf{V}'}$ by
	\begin{equation*}
		{_{i}\hat{\mathcal{R}}^{\Lambda}}(L)=\mathbf{Res}^{\mathbf{V}}_{\mathbf{V}'',\mathbf{V}'}(L)\otimes (\mathbf{D}{\rm{H}}^{\ast}(\mathbb{P}^{\infty})) [(i,|\mathbf{V}'|)-\langle \Lambda,\alpha^{\vee}_{i} \rangle+1].
	\end{equation*}

   (3) Define the functor $\mathcal{E}^{\Lambda}_{i}:\mathcal{Q}^{0}_{\mathbf{V}} \rightarrow \mathcal{D}(\mathbf{E}_{\mathbf{V},\Omega})$ by 
    \begin{equation*}
      \mathcal{E}^{\Lambda}_{i}(L)=\mathbf{\Sigma}\hat{\mathcal{R}}^{\Lambda}_{i}(L)\oplus	 ({_{i}\hat{\mathcal{R}}^{\Lambda}}(L)).
      \end{equation*}
\end{definition}

The following lemmas show that even though $\mathcal{E}^{\Lambda}_{i}(L)$ is not an object in $\mathcal{Q}^{0}_{\mathbf{V}'}$ for $L \in \mathcal{Q}^{0}_{\mathbf{V}}$, the functor $\mathcal{E}^{\Lambda}_{i}$ induces a linear operator $\mathcal{K} \rightarrow \mathcal{K}$.

\begin{lemma}\label{Verdier and restriction}
	For graded spaces $\mathbf{V},\mathbf{V}',\mathbf{V}''$ such that $|\mathbf{V}|=|\mathbf{V}'|+i,|\mathbf{V}''|=i$ and $\mathbf{V}=\mathbf{V}'\oplus \mathbf{V}''$ and any $L\in \mathcal{Q}^0_{\mathbf{V}}$, we have 
	\begin{equation*}
		{_{i}\hat{\mathcal{R}}^{\Lambda}}(L)=\mathbf{D}\mathbf{Res}^{\mathbf{V}}_{\mathbf{V}',\mathbf{V}''}\mathbf{D}(L)\otimes (\mathbf{D}{\rm{H}}^{\ast}(\mathbb{P}^{\infty}))[-\langle \Lambda,\alpha^{\vee}_{i} \rangle+1].
	\end{equation*}
\end{lemma}
\begin{proof}
	It is well-known that the restriction functor $\mathbf{Res}^{\mathbf{V}}_{\mathbf{V}',\mathbf{V}''}$ is a hyperbolic localization functor, that is, there is a $k^*$-action on $\mathbf{E}_{\mathbf{V}}$ such that the following diagrams commute
	\[
	\xymatrix{
		{(\mathbf{E}_\mathbf{V}})^{k^*} \ar[d]_{\cong} 
		&  {\mathbf{E}^+_\mathbf{V}} \ar[l]_{\pi^+} \ar[d]^{\cong} \ar[r]^{g^+} &\mathbf{E}_{\mathbf{V}} \ar@{=}[d] &{(\mathbf{E}_\mathbf{V}})^{k^*} \ar[d]_{\cong} \ar[d]_{\cong} 
		&  {\mathbf{E}^-_\mathbf{V}} \ar[l]_{\pi^-} \ar[d]^{\cong} \ar[r]^{g^-} &\mathbf{E}_{\mathbf{V}} \ar@{=}[d]
		\\	
		{\mathbf{E}_{\mathbf{V}'}} 
		& F \ar[l]_{\kappa_{\Omega}} \ar[r]^{\iota_{\Omega}} &\mathbf{E}_{\mathbf{V}} &{\mathbf{E}_{\mathbf{V}'}} 
		& F' \ar[l]_{\kappa'_{\Omega}} \ar[r]^{\iota'_{\Omega}} &\mathbf{E}_{\mathbf{V}},
	}
	\]
	where morphisms $\kappa_{\Omega},\iota_{\Omega}$ appear in the definition of $\mathbf{Res}^{\mathbf{V}}_{\mathbf{V}',\mathbf{V}''}$, and $\kappa'_{\Omega},\iota'_{\Omega}$ appear in the definition of $\mathbf{Res}^{\mathbf{V}}_{\mathbf{V}'',\mathbf{V}'}$, see \cite{MR1996415} and \cite[Proposition 2.10]{MR4524567} for details. By formula (1) and Theorem 1 in \cite{MR1996415}, we have 
	\begin{align*}
		&\mathbf{D}\mathbf{Res}^{\mathbf{V}}_{\mathbf{V}',\mathbf{V}''}\mathbf{D}(L)=\mathbf{D}((\kappa_{\Omega})_!(\iota_{\Omega})^*\mathbf{D}L[-\langle|\mathbf{V}'|,i\rangle])\\
		\cong &(\kappa_{\Omega})_*(\iota_{\Omega})^!L[\langle|\mathbf{V}'|,i\rangle]
		\cong (\pi^+)_*(g^+)^!L[\langle|\mathbf{V}'|,i\rangle]\\
		\cong &(\pi^-)_!(g^-)^*L[\langle|\mathbf{V}'|,i\rangle]\cong (\kappa'_{\Omega})_!(\iota'_{\Omega})^*L[\langle|\mathbf{V}'|,i\rangle]\\
		=&\mathbf{Res}^{\mathbf{V}}_{\mathbf{V}'',\mathbf{V}'}L[(i,|\mathbf{V}'|)],
	\end{align*}
	and so 
	\begin{equation*}
		{_{i}\hat{\mathcal{R}}^{\Lambda}}(L)=\mathbf{D}\mathbf{Res}^{\mathbf{V}}_{\mathbf{V}',\mathbf{V}''}\mathbf{D}(L)\otimes (\mathbf{D}{\rm{H}}^{\ast}(\mathbb{P}^{\infty})) [-\langle \Lambda,\alpha^{\vee}_{i} \rangle+1],
	\end{equation*}
	as desired.
\end{proof}


Let $\tilde{\mathcal{R}}^{\Lambda}_{i}: \mathcal{Q}^{0}_{\mathbf{V}} \rightarrow \hat{\mathcal{Q}}^{0}_{\mathbf{V}'}$ be the functor defined by $$\tilde{\mathcal{R}}^{\Lambda}_{i}= {_{i}\hat{\mathcal{R}}^{\Lambda}}[-2\langle |\mathbf{V}'|, i \rangle +2 \langle \Lambda,\alpha^{\vee}_{i} \rangle  ]. $$

\begin{lemma} \label{lemma28}
	If $\langle \Lambda,\alpha^{\vee}_{i} \rangle < \langle |\mathbf{V}'|, i \rangle$, ${_{i}\hat{\mathcal{R}}^{\Lambda}}(L)$ is a direct summand of $\tilde{\mathcal{R}}^{\Lambda}_{i}(L)$ and its complement $C_{1}^{-}(L)$ is indeed an object in $\mathcal{Q}^{0}_{\mathbf{V}'}$.  Otherwise, if $\langle \Lambda,\alpha^{\vee}_{i} \rangle \geqslant \langle |\mathbf{V}'|, i \rangle$,  $\tilde{\mathcal{R}}^{\Lambda}_{i}(L)$ is a direct summand of ${_{i}\hat{\mathcal{R}}^{\Lambda}}(L)$  and its complement $C_{1}^{+}(L)$ is also an object in $\mathcal{Q}^{0}_{\mathbf{V}'}$
\end{lemma}
\begin{proof}
	Notice that for each non-negative integer $n$, $(\mathbf{D}{\rm{H}}^{\ast}(\mathbb{P}^{\infty}))= \bigoplus \limits_{m \geqslant 0 } \overline{\mathbb{Q}}_{l}[2m]$ is a direct summand of $(\mathbf{D}{\rm{H}}^{\ast}(\mathbb{P}^{\infty}))[-2n]=\bigoplus \limits_{m \geqslant -n } \overline{\mathbb{Q}}_{l}[2m]$ and its complement is exactly $\bigoplus \limits_{-n \leqslant m \leqslant -1 } \overline{\mathbb{Q}}_{l}[2m]$. We can easily see that if $\langle \Lambda,\alpha^{\vee}_{i} \rangle < \langle |\mathbf{V}'|, i \rangle$, $$C_{1}^{-}(L)= \bigoplus_{0 \leqslant m \leqslant \langle |\mathbf{V}'|, i \rangle-\langle \Lambda,\alpha^{\vee}_{i} \rangle-1 }\mathbf{D}\mathbf{Res}^{\mathbf{V}}_{\mathbf{V}',\mathbf{V}''}\mathbf{D}(L)[ (1+\langle \Lambda,\alpha^{\vee}_{i} \rangle- \langle |\mathbf{V}'|, i \rangle +2m)  ] $$
	belongs to $\mathcal{Q}^{0}_{\mathbf{V}'}$. Similarly, if $\langle \Lambda,\alpha^{\vee}_{i} \rangle \geqslant \langle |\mathbf{V}'|, i \rangle$, 
	$$C_{1}^{+}(L)= \bigoplus_{0 \leqslant m \leqslant -\langle |\mathbf{V}'|, i \rangle+\langle \Lambda,\alpha^{\vee}_{i} \rangle-1 }\mathbf{D}\mathbf{Res}^{\mathbf{V}}_{\mathbf{V}',\mathbf{V}''}\mathbf{D}(L)[ (1-\langle \Lambda,\alpha^{\vee}_{i} \rangle-+\langle |\mathbf{V}'|, i \rangle +2m)  ] $$
	belongs to $\mathcal{Q}^{0}_{\mathbf{V}'}$.
\end{proof}


\begin{lemma}\label{lemma29}
	For any $\underline{\nu} \in \mathcal{S}_{|\mathbf{V}|}$, $\tilde{\mathcal{R}}^{\Lambda}_{i}(L_{\underline{\nu}})$ is a direct summand of $\hat{\mathcal{R}}^{\Lambda}_{i}(L_{\underline{\nu}})$ and its complement $C_{2}(L_{\underline{\nu}})$ is a finite direct sum of some $[L_{\underline{\omega}}]$, where $\underline{\omega} \in \mathcal{S}_{\mathbf{V}'}$ runs over  flag types which can be obtained from $\underline{\nu}$ by reducing $i$ from some $v^{k}$.
\end{lemma}

\begin{proof}	
	Consider the following morphisms in the definition of $\mathbf{Res}^{\mathbf{V}}_{\mathbf{V}',\mathbf{V}''}$,
	\begin{equation*}
		\mathbf{E}_{\mathbf{V}',\Omega} \xleftarrow{\kappa_{\Omega}} F \xrightarrow{ \iota_{\Omega} }  \mathbf{E}_{\mathbf{V},\Omega}.
	\end{equation*}
	Since $i$ is a source, we have $F=\mathbf{E}_{\mathbf{V}}$ and $\iota_{\Omega}$ is the identity,  and so  
	\begin{align*}
		&\mathbf{D}\mathbf{Res}^{\mathbf{V}}_{\mathbf{V}',\mathbf{V}''}\mathbf{D} (L_{\underline{\nu}}) =\mathbf{D}(\kappa_{\Omega})_!\mathbf{D}L_{\underline{\nu}}[\langle|\mathbf{V}'|,i\rangle]=(\kappa_{\Omega})_*L_{\underline{\nu}}[\langle|\mathbf{V}'|,i\rangle] ,\\
		&\mathbf{Res}^{\mathbf{V}}_{\mathbf{V}',\mathbf{V}''} (L_{\underline{\nu}}) =(\kappa_{\Omega})_!L_{\underline{\nu}}[-\langle|\mathbf{V}'|,i\rangle].
	\end{align*}
Hence 
\begin{equation*}
	\begin{split}
			\tilde{\mathcal{R}}^{\Lambda}_{i}(L_{\underline{\nu}})= &
		{_{i}\hat{\mathcal{R}}^{\Lambda}}(L_{\underline{\nu}})[-2\langle |\mathbf{V}'|, i \rangle +2 \langle \Lambda,\alpha^{\vee}_{i} \rangle  ] \\
		=& \mathbf{D}\mathbf{Res}^{\mathbf{V}}_{\mathbf{V}',\mathbf{V}''}\mathbf{D}(L_{\underline{\nu}})\otimes (\mathbf{D}{\rm{H}}^{\ast}(\mathbb{P}^{\infty})) [-\langle \Lambda,\alpha^{\vee}_{i} \rangle+1-2\langle |\mathbf{V}'|, i \rangle +2 \langle \Lambda,\alpha^{\vee}_{i} \rangle  ]\\
		=& (\kappa_{\Omega})_{\ast}(L_{\underline{\nu}})\otimes (\mathbf{D}{\rm{H}}^{\ast}(\mathbb{P}^{\infty})) [1-\langle |\mathbf{V}'|, i \rangle + \langle \Lambda,\alpha^{\vee}_{i} \rangle ],
	\end{split}
\end{equation*}
and 
\begin{equation*}
	\begin{split}
		\hat{\mathcal{R}}^{\Lambda}_{i}(L_{\underline{\nu}})=&\mathbf{Res}^{\mathbf{V}}_{\mathbf{V}',\mathbf{V}''}(L_{\underline{\nu}})\otimes (\mathbf{D}{\rm{H}}^{\ast}(\mathbb{P}^{\infty})) [\langle \Lambda,\alpha^{\vee}_{i} \rangle+1]\\
		=& (\kappa_{\Omega})_{!}(L_{\underline{\nu}})\otimes (\mathbf{D}{\rm{H}}^{\ast}(\mathbb{P}^{\infty})) [1-\langle |\mathbf{V}'|, i \rangle + \langle \Lambda,\alpha^{\vee}_{i} \rangle ].
	\end{split}
\end{equation*}
	
	By Proposition \ref{indres formula}, $\mathbf{Res}^{\mathbf{V}}_{\mathbf{V}',\mathbf{V}''}(L_{\underline{\nu}})$ is a direct sum of some $L_{\underline{\omega}}$ such that $\underline{\omega} \in \mathcal{S}_{\mathbf{V}'}$ is a flag type which can be obtained from $\underline{\nu}$ by reducing $i$ from $\nu^{k}$ for some $k$. More precisely, $\underline{\omega}=(\nu^{1},\nu^{2},\cdots, \nu^{k-1},i^{a_{k}-1},\nu^{k+1},\cdots, \nu^{m} )$ for some $k$ such that $\nu^{k}=i^{a_{k}}, a_{k}>0$. Consider the following commutative diagram in the proof of \cite[Proposition 4.2]{MR1088333},
	\[
	\xymatrix{
		\tilde{\mathcal{F}}_{\underline{\nu},\Omega} (\underline{\omega}) \ar[d]_{ \alpha_{\underline{\omega}} } \ar[r]
		& \tilde{\mathcal{F}}_{\underline{\nu},\Omega} \ar[d]^{\kappa_{\Omega} \pi_{\underline{\nu},\Omega}}
		\\	
		\tilde{\mathcal{F}}_{\underline{\omega},\Omega} \ar[r]^{\tilde{\pi}}
		& \mathbf{E}_{\mathbf{V}',\Omega}
	}
	\]
	where $\tilde{\mathcal{F}}_{\underline{\nu},\Omega} (\underline{\omega})$ is the locally closed subset of $\tilde{\mathcal{F}}_{\underline{\nu},\Omega}$ consisting of $(x,f)$ such that $f$ induces a flag of $\mathbf{V}'$ such that its type is $\underline{\omega}$. By \cite[Lemma 4.4]{MR1088333},  $\alpha_{\underline{\omega}} $ is a vector bundle, we denote its rank by $f_{\underline{\omega}}$. Then we have 
$$(\kappa_{\Omega})_{!} (L_{\underline{\nu}})=\bigoplus\limits_{\underline{\omega}} L_{\underline{\omega}}[-\dim \tilde{\mathcal{F}}_{\underline{\nu},\Omega}+\dim \tilde{\mathcal{F}}_{\underline{\omega},\Omega} -2 f_{\underline{\omega}} ],$$  
	$$(\kappa_{\Omega})_{\ast} (L_{\underline{\nu}})=\bigoplus\limits_{\underline{\omega}} L_{\underline{\omega}}[-\dim \tilde{\mathcal{F}}_{\underline{\nu},\Omega}+\dim \tilde{\mathcal{F}}_{\underline{\omega},\Omega}],$$
where the direct sum follows from similar arguments as in the proof of \cite[Lemma 4.7]{MR1088333} or \cite[Corollary 3.7]{MR4524567}. Hence by a similar argument as in Lemma \ref{lemma28}, we can see that $\tilde{\mathcal{R}}^{\Lambda}_{i}(L_{\underline{\nu}})$ is a direct summand of $\hat{\mathcal{R}}^{\Lambda}_{i}(L_{\underline{\nu}})$ and  $$C_{2}(L_{\underline{\nu}})=\bigoplus\limits_{\underline{\omega}}\bigoplus\limits_{ 0  < m \leqslant f_{\underline{\omega}} } L_{\underline{\omega}}[1-\langle |\mathbf{V}'|, i \rangle + \langle \Lambda,\alpha^{\vee}_{i} \rangle -\dim \tilde{\mathcal{F}}_{\underline{\nu},\Omega}+\dim \tilde{\mathcal{F}}_{\underline{\omega},\Omega}-2m]  $$ belongs to $\mathcal{Q}^{0}_{\mathbf{V}'}$.
\end{proof}

\begin{proposition}
	The functor $\mathcal{E}^{\Lambda}_{i}$ induces an $\mathcal{A}$-linear operator  $\mathcal{E}^{\Lambda}_{i}: \mathcal{K} \rightarrow \mathcal{K}$.
\end{proposition}
\begin{proof}
	By definition, 
	\begin{align*}
		\mathcal{E}^{\Lambda}_{i}([L])=&-[\hat{\mathcal{R}}^{\Lambda}_{i}(L)]+[{_{i}\hat{\mathcal{R}}^{\Lambda}}(L)] \\
		=& (-[\hat{\mathcal{R}}^{\Lambda}_{i}(L)]+ [\tilde{\mathcal{R}}^{\Lambda}_{i}(L)])+ (-[\tilde{\mathcal{R}}^{\Lambda}_{i}(L)]+[{_{i}\hat{\mathcal{R}}^{\Lambda}}(L)])
	\end{align*}
is $\mathcal{A}$-linear, since it is induced by functors. 
If $\langle \Lambda,\alpha^{\vee}_{i} \rangle < \langle |\mathbf{V}'|, i \rangle$,  take $\underline{\nu} \in \mathcal{S}_{|\mathbf{V}|}$, then
\begin{align*}
	\mathcal{E}^{\Lambda}_{i}([L_{\underline{\nu}}])
	=& -[C^{-}_{1}(L_{\underline{\nu}})]- [C_{2}(L_{\underline{\nu}})] \in \mathcal{K}.
\end{align*} 
Otherwise, when $\langle \Lambda,\alpha^{\vee}_{i} \rangle \geqslant \langle |\mathbf{V}'|, i \rangle$,
\begin{align*}
	\mathcal{E}^{\Lambda}_{i}([L_{\underline{\nu}}])
	=& [C^{+}_{1}(L_{\underline{\nu}})]- [C_{2}(L_{\underline{\nu}})] \in \mathcal{K}.
\end{align*} 

Notice that $\mathcal{K}$ is $\mathcal{A}$-spanned by $[L_{\underline{\nu}}], \underline{\nu} \in \mathcal{S}_{|\mathbf{V}|}$, $\mathcal{E}^{\Lambda}_{i}([L]) \in \mathcal{K}$ for any simple perverse sheaf $L$, therefore $\mathcal{E}^{\Lambda}_{i}: \mathcal{K} \rightarrow \mathcal{K}$. 
\end{proof}


Denote the natural functor $\hat{\mathcal{Q}}^{0}_{\mathbf{V}} \rightarrow \hat{\mathcal{Q}}^{0}_{\mathbf{V}}/\mathcal{N}_{\mathbf{V}}$ by $\varphi$, 
consider compositions 
$$ \mathcal{Q}^{0}_{\mathbf{V}} \xrightarrow{ {_{i}\hat{\mathcal{R}}^{\Lambda}} } \hat{\mathcal{Q}}^{0}_{\mathbf{V}'} \xrightarrow{ \varphi}  \hat{\mathcal{Q}}^{0}_{\mathbf{V}'}/\mathcal{N}_{\mathbf{V}'}$$
$$ \mathcal{Q}^{0}_{\mathbf{V}} \xrightarrow{{\hat{\mathcal{R}}^{\Lambda}_{i}} } \hat{\mathcal{Q}}^{0}_{\mathbf{V}'} \xrightarrow{\varphi}  \hat{\mathcal{Q}}^{0}_{\mathbf{V}'}/\mathcal{N}_{\mathbf{V}'}$$
and denote them still by ${_{i}\hat{\mathcal{R}}^{\Lambda}}$  and ${\hat{\mathcal{R}}^{\Lambda}_{i}}$ respectively. 
\begin{theorem}\label{exact}
	For any object $L$ of $\mathcal{Q}^{0}_{\mathbf{V}}$, we have a split exact sequence in $\hat{\mathcal{Q}}^{0}_{\mathbf{V}'}/ \mathcal{N}_{\mathbf{V}'}$ 
	\begin{equation*}
		0 \rightarrow {\hat{\mathcal{R}}^{\Lambda}_{i}}(L)  \rightarrow {_{i}\hat{\mathcal{R}}^{\Lambda}}(L)  \rightarrow E_{i}(L) \rightarrow 0.
	\end{equation*}
\end{theorem}

\begin{proof}
	We claim that in $\mathcal{K}_{0}(\Lambda)$, $[\mathcal{E}^{\Lambda}_{i}(L)]=[E_{i}(L)]$ for any $L \in \mathcal{Q}^{0}$. Indeed, by \cite{MR1227098} and Corollary 3.1 and 3.3 in \cite{zhao2022derivation}, teake $|\mathbf{V}'|+j=|\mathbf{V}|+i$, then there are isomorphisms of  functors $\mathcal{Q}^{0}_{\mathbf{V}'} \rightarrow \hat{\mathcal{Q}}^{0}_{\mathbf{V}}$ as follows,
	$${\hat{\mathcal{R}}^{\Lambda}_{i}}F_{j} \cong (F_{j}{\hat{\mathcal{R}}^{\Lambda}_{i}}) \oplus (\delta_{i,j} \rm{Id} \otimes \mathbf{D}H^{\ast}(\mathbb{P}^{\infty})[\langle \Lambda,\alpha^{\vee}_{i} \rangle-(i,|\mathbf{V}'|)+1] ),$$
	$${_{i}\hat{\mathcal{R}}^{\Lambda}}F_{j} \cong ({F_{j}} {_{i}\hat{\mathcal{R}}^{\Lambda}}) \oplus (\delta_{i,j} \rm{Id} \otimes \mathbf{D}H^{\ast}(\mathbb{P}^{\infty})[ (i,|\mathbf{V}|')-\langle \Lambda,\alpha^{\vee}_{i} \rangle +1] ).$$
	Hence, we can check that the linear operator $\mathcal{E}^{\Lambda}_{i}={\hat{\mathcal{R}}^{\Lambda}_{i}}-{_{i}\hat{\mathcal{R}}^{\Lambda}} :\mathcal{K} \rightarrow \mathcal{K}$ satisfies the relations 
	$$ \mathcal{E}^{\Lambda}_{i}F_{j}= F_{j}\mathcal{E}^{\Lambda}_{i}, i\neq j; $$
	$$	\mathcal{E}^{\Lambda}_{i}F_{i} + \sum \limits_{0\leqslant m \leqslant N-1} v^{N-1-2m} {\rm{Id}} =F_{i}\mathcal{E}^{\Lambda}_{i}\oplus \sum \limits_{0\leqslant m \leqslant -N-1} v^{-2m-N-1}{\rm{Id}}: \mathcal{K}_{\mathbf{V}}\rightarrow \mathcal{K}_{\mathbf{V}}.$$
	Taking composition with $\varphi$, we get the same equation in $\mathcal{K}_{0}(\Lambda)$. Since the opeartor $E_{i}$ also satisfies the same commutative relation with $F_{j}$, we can argue by induction to show that $$[\mathcal{E}^{\Lambda}_{i}(L_{\underline{\nu}'})]=[E_{i}(L_{\underline{\nu}'})]$$ for any flag type $\underline{\nu}'=(\nu'^{1},\nu'^{2}, \cdots, \nu'^{k})$ with each $\nu'^{l} \in I$. Note that each $L_{\underline{\nu}}$ can be write as a finite  direct sum of some $L_{\underline{\nu}'}$,  $[\mathcal{E}^{\Lambda}_{i}(L_{\underline{\nu}})]=[E_{i}(L_{\underline{\nu}})]$  for any flag type $\underline{\nu}$. Since $\mathcal{K}$ is $\mathcal{A}$-spanned by those $[L_{\underline{\nu}}]$,   for any $L \in \mathcal{Q}^{0}$, we have the equation  $[\mathcal{E}^{\Lambda}_{i}(L)]=[E_{i}(L)]$ in $\mathcal{K}_{0}(\Lambda)$.
	
	Hence we have $$ [{_{i}\hat{\mathcal{R}}^{\Lambda}}(L)] = [E_{i}(L)]+ [{\hat{\mathcal{R}}^{\Lambda}_{i}}(L)].$$ Since all of the terms are semisimple objects in $\hat{\mathcal{Q}}^{0}_{\mathbf{V}'}/ \mathcal{N}_{\mathbf{V}'}$ , we must have 
	$$ {_{i}\hat{\mathcal{R}}^{\Lambda}}(L)\cong E_{i}(L)\oplus {\hat{\mathcal{R}}^{\Lambda}_{i}}(L)$$
	as desired. 
\end{proof}

\begin{remark}
	Recall that the derivation operators $\bar{r}_{i}$ and $_{i}\bar{r}$ can be realized by the restriction functor, see \cite{MR1227098} or \cite{zhao2022derivation}. If we formally write $\frac{1}{v^{-1}-v}= \sum \limits_{n\geqslant 0 } v^{1+2n}$, then $\hat{\mathcal{R}}^{\Lambda}_{i}$ and $ {_{i}\hat{\mathcal{R}}^{\Lambda}}$ indeed realize the  linear operators  $\frac{ v^{ \langle \Lambda,\alpha^{\vee}_{i} \rangle } }{v^{-1}-v}\bar{r}_{i}, \frac{ v^{(i,|\mathbf{V}'|)- \langle \Lambda ,\alpha^{\vee}_{i} \rangle } }{v^{-1}-v} {_{i}\bar{r}}: {_{\mathcal{A}}\mathbf{U}^{-}  } \rightarrow \mathbf{U}^{-}  $ respectively. Therefore, Theorem \ref{exact} indeed categorify the equation
	$$E_{i}(x \cdot  v_{\Lambda} )=( v^{(i,|x|-i)- \langle \Lambda,\alpha^{\vee}_{i} \rangle}{_{i} \bar{r}}(x)\cdot  v_{\Lambda} - v^{\langle \Lambda,\alpha^{\vee}_{i} \rangle} \bar{r}_{i}(x)\cdot  v_{\Lambda} ) /(v^{-1}-v ).  $$
\end{remark}




\section{Compare with Nakajima's realization}
In this section, we recall Nakajima's realization of integrable highest weight modules via his quiver varieties in \cite{MR1302318} and \cite{MR1604167} and compare our construction (at $v \rightarrow 1$) with the realization of Nakajima.
\subsection{Nakajima quiver variety}
For given quiver $Q=(I,H,\Omega)$, Nakajima considered the double framed quiver $\hat{Q}$, its set of vertices is $I\cup \hat{I}$ and its set of arrows is $\hat{H}=H \cup \{ i \rightarrow \hat{i},\hat{i} \rightarrow i|i\in I\}$.

For dimension vectors $\nu,\omega \in \mathbb{N}[I]$ and $I$-graded spaces $\mathbf{V},\mathbf{W}$ such that $|\mathbf{V}|=\nu$ and $|\mathbf{W}|=\omega$, we consider the moduli space
\begin{equation*}
	\mathbf{E}_{\mathbf{V},\mathbf{W}}= \bigoplus\limits_{h \in \hat{H}} \mathbf{Hom}(\mathbf{V}_{h'},\mathbf{V}_{h''})=   \bigoplus\limits_{h \in H} \mathbf{Hom}(\mathbf{V}_{h'},\mathbf{V}_{h''}) \oplus \bigoplus\limits_{i \in I} \mathbf{Hom}(\mathbf{V}_{i},\mathbf{W}_{i}) \oplus \bigoplus\limits_{i \in I} \mathbf{Hom}(\mathbf{W}_{i},\mathbf{V}_{i})
\end{equation*}

We denote an element of $\mathbf{E}_{\mathbf{V},\mathbf{W}}$ by $(B,i,j)$, where $B$ is an element of $\mathbf{E}_{\mathbf{V}}$,$ i,j$ are elements of $ \bigoplus\limits_{i \in I} \mathbf{Hom}(\mathbf{V}_{i},\mathbf{W}_{i})$ and $\bigoplus\limits_{i \in I} \mathbf{Hom}(\mathbf{W}_{i},\mathbf{V}_{i})$ respectively. The moment map of $\mathbf{E}_{\mathbf{V},\mathbf{W}}$ is defined by
\begin{equation*}
	(\mu( (B,i,j) ))_{k \in I}= \sum\limits_{h \in H, h''=k} \epsilon(h)B_{h}B_{\bar{h}} +i_{k}j_{k} .
\end{equation*} 
The algebraic group
\begin{equation*}
	G=G_{\mathbf{V}}= \prod\limits_{i \in I} \mathbf{GL}(\mathbf{V}_{i})
\end{equation*} 
acts on $\mathbf{E}_{\mathbf{V},\mathbf{W}}$ by 
\begin{equation*}
	g \cdot (B,i,j) =(gBg^{-1},gi,jg^{-1}).
\end{equation*}

\begin{definition}
	Nakajima quiver variety $\mathfrak{m}_{0}(\nu,\omega)$ and $\mathfrak{m}(\nu,\omega)$ are defined as the affine and projective geometric quotients of $\mu^{-1}(0)$ respectively. More precisely, let $A(\mu^{-1}(0))$ be the coordinate ring of the affine variety $\mu^{-1}(0)$, the affine quiver variety $\mathfrak{m}_{0}(\nu,\omega)=\mathfrak{m}_{0}$ is defined by
	\begin{equation*}
		\mathfrak{m}_{0}(\nu,\omega)=\mu^{-1}(0)//G={\rm{Spec}} \ A(\mu^{-1}(0))^{G}.
	\end{equation*}
	Take character $\chi_{G}:G \rightarrow \mathbb{C}; g \mapsto \prod\limits_{k\in I} {\rm{det}} g_{k}^{-1}$ and set
	\begin{equation*}
		A(\mu^{-1}(0))^{G,\chi_{G}^{n}}=\{f\in A(\mu^{-1}(0))|f(g(B,i,j))=\chi_{G}(g)^{n}f((B,i,j))\}
	\end{equation*}
	then the projective quiver variety $\mathfrak{m}(\nu,\omega)=\mathfrak{m}$ is defined by 
	\begin{equation*}
		\mathfrak{m}(\nu,\omega)={\rm{Proj}}( \bigoplus \limits_{n \geq 0} 	A(\mu^{-1}(0))^{G,\chi_{G}^{n}} ).
	\end{equation*}
\end{definition}	

In order to describe $\mathfrak{m}(\nu,\omega)$, Nakajima introduced the following stable conditions.
\begin{lemma}
	Assume that $(B,i,j)\in \mu^{-1}(0)$, then $(B,i,j)$ is stable if and only if the following statement holds,
	
	(S)\ For any $B$-stable subspace $S \subset \mathbf{V}$ such that $S \subset {\rm{Ker}}i$, we have $S=0$.
\end{lemma}	

Let $\mu^{-1}(0)^{s}$ be the subset of stable points, then Nakajima proved the following lemma and corollary in \cite{MR1604167}.

\begin{lemma}
	If $(B,i,j) \in \mu^{-1}(0)^{s}$,then:\\
	(1) ${\rm{Stab}}_{G}((B,i,j))=\{1\}$;\\
	(2) The tangent map $d\mu$ is surjective at $(B,i,j)$. In particular $\mu^{-1}(0)^{s}$ is nonsigular and of dimension $\sum\limits_{k\in I} (2\nu_{k}\omega_{k}-\nu_{k}^{2}) +(\nu,\nu)$.
\end{lemma}

\begin{corollary}
	Nakajima quiver variety $\mathfrak{m}(\nu,\omega)$ is the geometric quotient of  $\mu^{-1}(0)^{s}$. In particular, $\mathfrak{m}(\nu,\omega)$ is nonsigular and of dimension $\sum\limits_{k\in I} (2\nu_{k}\omega_{k}-2\nu_{k}^{2}) +(\nu,\nu)$. Moreover, the closed points in $\mathfrak{m}(\nu,\omega)$ are bijective to orbits of $\mu^{-1}(0)^{s}$.
\end{corollary}

We denote the geometric point corresponding to the orbit of $(B,i,j)\in \mu^{-1}(0)^{s}$ by $[B,i,j]$. If the orbit of $(B,i,j)\in \mu^{-1}(0)$ is close, then $[B,i,j]$ is also a geometric point of $\mathfrak{m}_{0}(\nu,\omega)$. The geometric invariant theory provides a morphism $\pi:\mathfrak{m}(\nu,\omega)\rightarrow \mathfrak{m}_{0}(\nu,\omega)$:
\begin{equation*}
	\pi([B,i,j])=[B^{0},i^{0},j^{0}]
\end{equation*}
where $G(B^{0},i^{0},j^{0})$ is the unique close orbit contained in the closure of $G(B,i,j)$.

\begin{proposition}
	The algebraic variety $\pi^{-1}(0) \subset \mathfrak{m}(\nu,\omega)$ is Lagrangian and homotopic to $\mathfrak{m}(\nu,\omega)$. In particular, we denote $\pi^{-1}(0)$ by $\mathfrak{L}(\nu,\omega)$.
\end{proposition}

For dimension vectors $\nu^{1},\nu^{2}$ and graded spaces $\mathbf{V}^{1},\mathbf{V}^{2}$ such that $|\mathbf{V}^{i}|=\nu^{i},i=1,2$, we define the morphisms $\pi_{i}: \mathfrak{m}(\nu^{i},\omega) \rightarrow \mathfrak{m}_{0}(\nu^{1}+\nu^{2},\omega)$ to be the composition of projections $\pi: \mathfrak{m}(\nu^{i},\omega) \rightarrow \mathfrak{m}_{0}(\nu^{i},\omega)$ and embeddings $\mathfrak{m}_{0}(\nu^{i},\omega) \rightarrow \mathfrak{m}_{0}(\nu^{1}+\nu^{2},\omega)$.

\begin{definition}
	Let $Z(\nu^{1},\nu^{2},\omega)$  be the subvariety $\mathfrak{m}(\nu^{1},\omega)\times \mathfrak{m}(\nu^{2},\omega)$ defined by:
	\begin{equation*}
		Z(\nu^{1},\nu^{2},\omega)=\{(x_{1},x_{2})\in \mathfrak{m}(\nu^{1},\omega)\times \mathfrak{m}(\nu^{2},\omega) |\pi_{1}(x_{1})=\pi_{2}(x_{2})\}.
	\end{equation*}
\end{definition}

Let $\mathfrak{m}_{0}(\nu,\omega)^{reg}$ be the subset of $\mathfrak{m}_{0}(\nu,\omega)$ consisting of $[B,i,j]$ with trivial stabilizer ${\rm{Stab}(B,i,j)}$. Then we have the following result by Nakajima:
\begin{lemma}
	If $[B,i,j] \in \mathfrak{m}_{0}(\nu,\omega)^{reg}$, then $(B,i,j) \in \mu^{-1}(0)^{s}$. Moreover, $\pi$ induces an isomorphism between $\mathfrak{m}_{0}(\nu,\omega)^{reg}$ and $\pi^{-1}(\mathfrak{m}_{0}(\nu,\omega)^{reg})$.
\end{lemma}

\begin{definition}
	Let $Z^{reg}(\nu^{1},\nu^{2},\omega)$ be the complement of the closure of the set
	\begin{equation*}
		\{(x_{1},x_{2})| \pi_{1}(x_{1})=\pi_{2}(x_{2}) \notin \mathfrak{m}_{0}(\nu,\omega)^{reg}, \subset \mathfrak{m}(\nu^{1}+\nu^{2},\omega)^{reg} \}.
	\end{equation*}
\end{definition}
Nakajima proved that $Z^{reg}(\nu^{1},\nu^{2},\omega)$ is Lagrangian and its irreducible components are of dimension $\frac{1}{2}( {\rm{dim}} \mathfrak{m}(\nu^{1},\omega)+{\rm{dim}} \mathfrak{m}(\nu^{2},\omega) )$, $Z^{reg}(\nu^{1},\nu^{2},\omega)$ is open in  $Z(\nu^{1},\nu^{2},\omega)$.

For dimension vectors $\nu^{i},i=1,2,3$, we define the projections
\begin{equation*}
	p_{i,j}: \mathfrak{m}(\nu^{1},\omega)\times \mathfrak{m}(\nu^{2},\omega) \times \mathfrak{m}(\nu^{3},\omega) \rightarrow \mathfrak{m}(\nu^{i},\omega)\times \mathfrak{m}(\nu^{j},\omega),
\end{equation*}	
then we can define the convolution product of the the Borel-Moore homology groups
\begin{equation*}
	H_{\ast}( Z(\nu^{1},\nu^{2},\omega) ) \otimes 	H_{\ast}( Z(\nu^{2},\nu^{3},\omega) ) \rightarrow 	H_{\ast}( Z(\nu^{1},\nu^{3},\omega) )	
\end{equation*}	
\begin{equation*}
	(c_{1},c_{2}) \mapsto (p_{1,3})_{\ast}( p^{\ast}_{1,2}c_{1} \cap p^{\ast}_{2,3}c_{2})
\end{equation*}
where $H_{\ast}( Z(\nu^{1},\nu^{2},\omega) )$ is the Borel-Moore homology group of $ Z(\nu^{1},\nu^{2},\omega)  $ and $H_{top}( Z(\nu^{1},\nu^{2},\omega) )$ is the homology group of top degree. The fundamental classes of irreducible components  of $Z(\nu^{1},\nu^{2},\omega)$ form a basis of $H_{top}( Z(\nu^{1},\nu^{2},\omega) )$. Under the convolution product, $ \bigoplus \limits_{\nu^{1},\nu^{2}}H_{\ast}( Z(\nu^{1},\nu^{2},\omega) ) $ becomes a $\mathbb{Q}$-algebra and $ \bigoplus \limits_{\nu^{1},\nu^{2}}H_{top}( Z(\nu^{1},\nu^{2},\omega) ) $ is a subalgebra.

For given $\nu^{0}$ and $x\in \mathfrak{m}_{0}(\nu^{0},\omega)^{reg} \subset \mathfrak{m}(\nu,\omega) $, let $\mathfrak{m}(\nu,\omega)_{x}$ be the fiber at $x$ of the morphism $\pi:\mathfrak{m}(\nu,\omega) \rightarrow \mathfrak{m}_{0}(\nu,\omega)$. In particular, for $x=0$, we have 
\begin{equation*}
	\mathfrak{m}(\nu,\omega)_{0}=\mathfrak{L}(\nu,\omega)
\end{equation*}
Under the action defined by convolution, $\bigoplus \limits_{\nu}H_{\ast}( \mathfrak{m}(\nu,\omega)_{x}) $ becomes a $ \bigoplus \limits_{\nu^{1},\nu^{2}}H_{\ast}( Z(\nu^{1},\nu^{2},\omega) ) $-module. Moreover, $\bigoplus \limits_{\nu}H_{top}( \mathfrak{m}(\nu,\omega)_{x}) $ becomes a $ \bigoplus \limits_{\nu^{1},\nu^{2}}H_{top}( Z(\nu^{1},\nu^{2},\omega) ) $-module.

Fix $k \in I$ and consider graded spaces $\mathbf{V}^{2}$ such that $|\mathbf{V}^{2}|=\nu$ and $\mathbf{V}^{1}$ such that $|\mathbf{V}^{1}|=\nu-k$. We set $\nu^{2}=\nu,\nu^{1}=\nu-k$.
Nakajima introduced the Hecke correspondence $\beta_{k}(\nu,\omega)$, which is a nonsigular Lagrangian subvariety of $\mathfrak{m}(\nu^{1},\omega)\times\mathfrak{m}(\nu^{2},\omega)$(See details in \cite{MR1604167} Section 5). He also defined $E_{k}\in \prod \limits_{\nu^{1}}H_{top}( Z(\nu^{1},\nu^{1}+k,\omega) )$ to be the following formal sum
\begin{equation*}
	E_{k}=\sum\limits_{\nu} [\beta_{k}(\nu,\omega)]
\end{equation*}
and defined $F_{k}\in \prod \limits_{\nu^{1}}H_{top}( Z(\nu^{1},\nu^{1}-k,\omega) )$ to be the following formal sum
\begin{equation*}
	F_{k}=\sum\limits_{\nu} (-1)^{r(\nu,\omega)} [\tilde{\omega}(\beta_{k}(\nu,\omega))]
\end{equation*}
where $\tilde{\omega}: \mathfrak{m}(\nu^{1},\omega) \times \mathfrak{m}(\nu^{2},\omega) \rightarrow \mathfrak{m}(\nu^{2},\omega)\times \mathfrak{m}(\nu^{1},\omega)$ is the morphism exchanging two factors and
\begin{equation*}
	r(\nu,\omega)=\frac{1}{2}({\rm{dim}} \mathfrak{m}(\nu-k,\omega)-{\rm{dim}} \mathfrak{m}(\nu,\omega) ).
\end{equation*}


Nakajima proved the following main theorem.
\begin{theorem}[\cite{MR1604167}] \label{Nmain}
With the action by $E_{k},F_{k},k\in I$, the group
 $\bigoplus \limits_{\nu}H_{top}( \mathfrak{L}(\nu,\omega))$ becomes a (left) $\mathbf{U}(\mathfrak{g})$-module. Moreover, it is isomorphic to the highest weight module $L_{0}(\Lambda_{\omega})$ with highest weight $\Lambda_{\omega}$ and the highest weight vector $[\mathfrak{L}(0,\omega)]$, and we have an $\mathbf{U}(\mathfrak{g})$-linear isomorphism satisfying
	\begin{equation*}
		\varkappa^{\Lambda_{\omega}}:\bigoplus \limits_{\nu}H_{top}( \mathfrak{L}(\nu,\omega)) \rightarrow L_{0}(\Lambda_{\omega})
	\end{equation*}
	\begin{equation*}
		[\mathfrak{L}(0,\omega)] \mapsto  v_{\Lambda_{\omega}}
	\end{equation*}
where $\Lambda_{\omega} $ is the dominant weight such that $\langle \Lambda_{\omega}, \alpha_{i}^{\vee} \rangle=\omega_{i},i \in I$.
\end{theorem}


The quiver variety $\mathfrak{m}(\nu,\omega)$ admits a partition
$\bigcup\limits_{r\geq 0}\mathfrak{m}_{k,r}(\nu,\omega)=\mathfrak{m}(\nu,\omega)$, where
\begin{equation*}
	\mathfrak{m}_{k,r}(\nu,\omega)=\{ [B,i,j]\in \mathfrak{m}(\nu,\omega)|{\rm{codim}} (\sum\limits_{h \in H, h''=k} {\rm{Im}}B_{h} + {\rm{Im}}j_{k})=r \}.
\end{equation*}
We also set $\mathfrak{m}_{k,\leq r}(\nu,\omega)= \bigcup\limits_{s\leq r} \mathfrak{m}_{k,s}(\nu,\omega)$. There is a natural map defined by 
\begin{equation*}
	p:\mathfrak{m}_{k,\leq r}(\nu,\omega) \rightarrow \mathfrak{m}_{k,\leq 0}(\nu-rk,\omega);
\end{equation*}
\begin{equation*}
	[B,i,j]\mapsto [B,i,j]|_{{\rm{Im}}B_{h} + {\rm{Im}}j_{k} \oplus \bigoplus \limits_{i \neq k}\mathbf{V}_{i}  }.
\end{equation*}
By definition, one can check that each fiber of $p$ is isomorphic to a Grassmannian, hence $p$ is smooth with connected fiber.

For any irreducible component $X \subset \mathfrak{m}(\nu,\omega)_{x} $  and $k \in I$, there exists a unique integer $r$ such that $X \cap \mathfrak{m}_{k,r}(\nu,\omega)_{x}$ is dense  in $X$ and we define $r=t_{k}(X)$.

Since $p:\mathfrak{m}_{k, r}(\nu,\omega) \rightarrow \mathfrak{m}_{k, 0}(\nu-rk,\omega)$ is smooth and with connected fibers, $p$ induces a bijection between the set of irreducible components of $\mathfrak{m}_{k, r}(\nu,\omega)_{x}$ and  the set of irreducible components  of $\mathfrak{m}_{k, 0}(\nu-rk,\omega)_{x}$. It also induces a bijection $	\varrho_{k,r}$ between sets $Irr \mathfrak{m}(\nu,\omega)_{x}$ of irreducible components:
\begin{equation*}
	\varrho_{k,r}:\{X \subset  \mathfrak{m}(\nu,\omega)_{x} |t_{k}(X)=r \} \rightarrow \{X' \subset  \mathfrak{m}(\nu-rk,\omega)_{x} |t_{k}(X')=0 \}
\end{equation*}
\begin{equation*}
	X \mapsto \overline{p(X \cap \mathfrak{m}_{k,r}(\nu,\omega) )}
\end{equation*}
where $\overline{X}$ is the closure of the set $X$. 

\begin{lemma}[\cite{MR1604167}, Lemma 10.1]
	If $t_{k}(X)=r >0$ and $\varrho_{k,r}(X)= \varrho_{k,r-1}(X'') $,then we have
	\begin{equation*}
		F_{k}[X'']=\pm r [X] + \sum\limits_{t_{k}(X')>r } c_{X'}[X'].
	\end{equation*}
\end{lemma}
As a corollary, we have the following lemma, which is parallel to  Lemma \ref{lkey}.
\begin{lemma}\label{Nkey}
	For irreducible components $X \subset \mathfrak{L}(\nu,\omega)$ such that $t_{k}(X)=r>0$ and $X''=\varrho_{k,r}(X) $, we have the following equation in $\bigoplus \limits_{\nu}H_{top}( \mathfrak{L}(\nu,\omega))$:
	\begin{equation*}
		F_{k}^{(r)} ([X''])=\pm [X] + \sum\limits_{t_{k}(X')>r} c_{X'}[X']
	\end{equation*}
	where $F_{k}^{(r)}= \frac{F_{k}^{r}}{r!}$ and $c_{X'}$ are constants.
\end{lemma}
\begin{proof}
	We prove by induction on $r$. If $r=1$, the lemma  trivially holds. For general $r$, we have
	\begin{equation*}
		\begin{split}
			F_{k}^{(r)} ([X''])=&\frac{1}{r}F_{k}  F_{k}^{(r-1)}([X''])\\
			=&\frac{1}{r} F_{k}  (\pm [\tilde{X}] + \sum\limits_{t_{k}(X')>r-1} c_{X'}[X'])\\
			=&\pm [X] + \sum\limits_{t_{k}(X')>r} c_{X'}[X']
		\end{split}
	\end{equation*}
	where $\tilde{X}$ is the irreducible component such that $X''=\varrho_{k,r-1}(\tilde{X})$. The second equation holds by the inductive assumption and  the third equation holds by applying the above lemma to $\tilde{X}$ and $X'$ respectively.
\end{proof}


\subsection{The left graphs and the isomorphism $\Phi^{\Lambda}$}


We recall the left graph of the canonical basis defined by Lusztig. In this and the next section, we usually omit the notation of Fourier-Deligne transformations if there is no ambiguity. (For example, we denote $t_{i}(\mathcal{F}_{\Omega,\Omega_{i}}(L) )$ by $t_{i}(L)$ and denote  $\mathcal{F}_{\Omega_{i},\Omega}\pi_{i,p} \mathcal{F}_{\Omega,\Omega_{i}}$ by $\pi_{i,p}$.)
\begin{definition}
	We define the left graph $\mathcal{G}_{1}=(\mathcal{V}_{1},\mathcal{E}_{1}) $ to be the $I \times \mathbb{N}_{>0}$-colored graph consisting of the set of vertices  $\mathcal{V}_{1}$ and the set of arrows $\mathcal{E}_{1}$ as follows,
	\begin{equation*}
		\mathcal{V}_{1}=\{ [L]  \in \mathcal{K}| L \in \mathcal{P}_{\mathbf{V}} , |\mathbf{V}|  \in \mathbb{N}[I] \},
	\end{equation*}
	\begin{equation*}
		\mathcal{E}_{1}=\{ [L] \xrightarrow{(k,r)} [L']|k\in I, 0 < r \in \mathbb{N}, \pi_{k,r}(L')=L \text{ for some } |\mathbf{V}| \in \mathbb{N}[I] \}.
	\end{equation*}
\end{definition}

Now we can introduce the left graph of $\mathcal{L}(\Lambda)$, which is isomorphic to a subgraph of $\mathcal{G}_{1}$.
\begin{definition}
	We define the left graph  of $\mathcal{L}(\Lambda)$ to be the $I \times \mathbb{N}_{>0}$-colored graph $\mathcal{G}_{1}(\Lambda)=(\mathcal{V}_{1}(\Lambda),\mathcal{E}_{1}(\Lambda)) $ , which consists of the following set of vertices  $\mathcal{V}_{1}(\Lambda)$ and the set of arrows $\mathcal{E}_{1}(\Lambda)$
	\begin{equation*}
		\mathcal{V}_{1}(\Lambda)=\{ [L]  \in \mathcal{K}_{0}(\Lambda)| L \in \mathcal{P}_{\mathbf{V}}, L \ncong 0 \textbf{ in } \mathcal{L}_{\mathbf{V}}(\Lambda), |\mathbf{V}|  \in \mathbb{N}[I] \},
	\end{equation*}
	\begin{equation*}
		\mathcal{E}_{1}(\Lambda)=\{ [L] \xrightarrow{(k,r)} [L']|[L],[L']\in \mathcal{V}_{1}(\Lambda)  ,k\in I, 0 < r \in \mathbb{N}, \pi_{k,r}(L')=L \text{ for some } |\mathbf{V}| \in \mathbb{N}[I] \}.
	\end{equation*}
\end{definition}

We say a sequence $\underline{s}=((i_{1},n_{1}),(i_{2},n_{2}),\cdots ,(i_{l},n_{l}) )$ of $I \times \mathbb{N}_{>0}$ is a left admissible path of $L \in \mathcal{P}_{\mathbf{V}}$, if $\pi_{i_{1},n_{1}} \pi_{i_{2},n_{2}}\cdots \pi_{i_{l},n_{l}}(L_{0})=L$, where $L_{0}$ is the unique simple perverse sheaf in $\mathcal{P}_{0}$. By Lemma 7.2 in \cite{MR1088333}, for any $L \in \mathcal{P}_{\mathbf{V}}$, there exists a left admisible path of $L$.



Consider the affine variety $\Lambda_{\mathbf{V},\mathbf{W}} \subset \mathbf{E}_{\mathbf{V},\mathbf{W}}$
\begin{equation*}
	\Lambda_{\mathbf{V},\mathbf{W}} =\{ (B,i,j) \in \mu^{-1}(0)|j=0,B \text{ is nilpotent } \}
\end{equation*}
In particular, for $\mathbf{W}=0$, we denote $\Lambda_{\mathbf{V},0}$ by $\Lambda_{\mathbf{V}}$.




Let $\Lambda_{\mathbf{V},i,p}$ be the close subset of $\Lambda_{\mathbf{V}}$ defined by
\begin{equation*} 
	\Lambda_{\mathbf{V},i,p}=\{x\in \Lambda_{\mathbf{V}}| {\rm{codim}}_{\mathbf{V}_{i}} ( {\rm{Im}} \sum\limits_{h \in H, h''=i} x_{h}) =p\},
\end{equation*} 
then for any irreducible component $Z$, there exists a unique $p$ such that $Z \cap \Lambda_{\mathbf{V},i, p}$ is dense in $Z$ and we define $t_{i}(Z)=p$. Similarly, we can consider the close subsets 
\begin{equation*} 
	\Lambda_{\mathbf{V},i}^{p}=\{x\in \Lambda_{\mathbf{V}}| {\rm{dim}}_{\mathbf{V}_{i}} ( {\rm{Ker}} \bigoplus\limits_{h \in H, h'=i} x_{h}) =p\}
\end{equation*} 
and define $t_{i}^{\ast}(Z)=p$ if $p$ is the unique integer such that $Z \cap \Lambda_{\mathbf{V},i}^{p}$ is dense in $Z$. 

 Given $\nu'+\nu''=\nu \in \mathbb{N}[I] $ and graded vector spaces $\mathbf{V},\mathbf{V}',\mathbf{V}''$ with dimension vectors $\nu, \nu', \nu''$ respectively, let  $\Lambda'$ be the set which consists of $(x,\tilde{\mathbf{W}},\rho_{1},\rho_{2})$ where $x \in \Lambda_{\mathbf{V}}$, $\tilde{\mathbf{W}}$ is a $x$-stable subspace of $\mathbf{V}$ with dimension $\nu''$ and $\rho_{1}:\mathbf{V}/\tilde{\mathbf{W}} \simeq \mathbf{V}', \rho_{2}: \tilde{\mathbf{W}} \simeq \mathbf{V}''$ are linear isomorphisms, and let $\Lambda''$ be the set which consists of $(x,\tilde{\mathbf{W}})$ as above. Consider the morphisms
\begin{center}
	$\Lambda_{\mathbf{V}'} \times \Lambda_{\mathbf{V}''} \xleftarrow{p} \Lambda' \xrightarrow{r} \Lambda'' \xrightarrow{q} \Lambda_{\mathbf{V}}$
\end{center}
where $p(x,\tilde{\mathbf{W}},\rho_{1},\rho_{2})=(\rho_{1,\ast}(\bar{x}|_{\mathbf{V}/\tilde{\mathbf{W}}}),\rho_{2,\ast}(x|_{\tilde{\mathbf{W}}})  )$, $r(x,\tilde{\mathbf{W}},\rho_{1},\rho_{2}) =(x, \tilde{\mathbf{W}}) $ and $q(x,\tilde{\mathbf{W}})=x$. Note that $r$ is a principle $G_{\mathbf{V}'} \times G_{\mathbf{V}''}$-bundle and $q$ is proper. 

We assume that $|\mathbf{V}''|+pi=|\mathbf{V}|$ and denote $qr: \Lambda' \rightarrow \Lambda_{\mathbf{V}}$ by $q'$. Then $p^{-1} (\Lambda_{\mathbf{V''},i,0})= (q'^{-1})(\Lambda_{\mathbf{V},i, p})$. We denote $p^{-1} (\Lambda_{\mathbf{V''},i,0})$ by $\Lambda'_{i, p}$ and denote $(q^{-1})(\Lambda_{\mathbf{V},i, p})$ by $\Lambda''_{i,p}$. Then we have the following commutative diagram

\[
\xymatrix{
	\Lambda_{\mathbf{V}'',i,0}\ar[d]^{i_{1}} & \Lambda'_{i,p}\ar[d]^{i_{2}}\ar[l]_{p} \ar[r]^{r}& \Lambda''_{i,p} \ar[d]^{i_{3}}\ar[r]^{q}& \Lambda_{\mathbf{V},i, p} \ar[d]^{i_{4}}\\
	\Lambda_{\mathbf{V}''} & \Lambda' \ar[l]_{p} \ar[r]^{r} & \Lambda''  \ar[r]^{q} & \Lambda_{\mathbf{V}}
}
\]
here $i_{1},i_{2},i_{3}$ and $i_{4}$ are the natural embeddings.

\begin{lemma}[\cite{MR3202708}]
	(1) $q': \Lambda'_{i,p} \rightarrow \Lambda_{\mathbf{V},i, p}$ is a principle $G_{\mathbf{V''}} \times G_{\mathbf{V}'}$ -bundle.\\
	(2) $p: \Lambda'_{i, p} \rightarrow \Lambda_{\mathbf{V}'',i,0}$ is a smooth map whose fibers are connected of dimension $(\sum \limits_{i \in I} \nu_{i}^{2} )- p(\nu'',i )$.
\end{lemma}
 \begin{corollary}
	$q'p^{-1}$ induces a bijection $\eta_{i,p}:Irr \Lambda_{\mathbf{V}'',i,0} \rightarrow  Irr \Lambda_{\mathbf{V},i, p} $ from the set of irreducible components of $\Lambda_{\mathbf{V}'',i,0}$ to the set of irreducible components $\Lambda_{\mathbf{V},i, p}$. The map $\eta_{i,p}$ also induces a bijection ( still denoted by $\eta_{i,p}$) from $\{Z \in Irr\Lambda_{\mathbf{V}''}| t_{i}(Z)=0 \}$ to $\{ Z \in Irr \Lambda_{\mathbf{V}}|t_{i}(Z)=p \} $ defined by: $$\eta_{i,p}( \bar{Z})= \overline{\eta_{i,p}(Z)}$$ for $Z \in Irr \Lambda_{\mathbf{V}'',i,0}$. Here $\bar{Z}$ and $\overline{\eta_{i,p}(Z)}$ are the closure of $Z$ and $\eta_{i,p}(Z)$ respectively.
\end{corollary}

Now we can define the left graph $\mathcal{G}_{2}$ for $Irr \Lambda_{\mathbf{V}}$.
\begin{definition}
	We define the left graph $\mathcal{G}_{2}=(\mathcal{V}_{2},\mathcal{E}_{2}) $ to be the $I \times \mathbb{N}_{>0}$-colored graph consisting of the following set of vertices  $\mathcal{V}_{2}$ and the set of arrows $\mathcal{E}_{2}$ as follows,
	\begin{equation*}
		\mathcal{V}_{2}=\{ Z | Z \in Irr \Lambda_{\mathbf{V}} , |\mathbf{V}|  \in \mathbb{N}[I] \},
	\end{equation*}
	\begin{equation*}
		\mathcal{E}_{2}=\{ Z \xrightarrow{(k,r)} Z'|k\in I, 0 < r \in \mathbb{N}, \eta_{k,r}(Z')=Z \text{ for some } |\mathbf{V}| \in \mathbb{N}[I] \}.
	\end{equation*}
\end{definition}

Similarly, we say a sequence $\underline{s}=((i_{1},n_{1}),(i_{2},n_{2}),\cdots ,(i_{l},n_{l}) )$ of $I \times \mathbb{N}_{>0}$ is a left addmissible path of $Z \in Irr \Lambda_{\mathbf{V}}$, if $\eta_{i_{1},n_{1}} \eta_{i_{2},n_{2}}\cdots \eta_{i_{l},n_{l}}(Z_{0})=Z$, where $Z_{0}$ is the unique irreducible component of $\Lambda_{0}$. By Corollary 1.6 in \cite{MR1758244}, for any  $Z \in Irr \Lambda_{\mathbf{V}}$, there exists a left addmissible path of $Z$.
Then by \cite{MR1458969} or Theorem 7.1 in \cite{fang2022correspondence}, we have the following result:
\begin{proposition}
	There is a bijection $\Phi: \mathcal{V}_{1} \rightarrow \mathcal{V}_{2}$ such that $\Phi$ commutes with the arrows in $\mathcal{E}_{1}$ and $\mathcal{E}_{2}$ and $\Phi$ preserves the value $t_{i}$ and $t^{\ast}_{i}$.
	\begin{equation*}
		\Phi([L]) \xrightarrow{(k,r)} \Phi([L']) \iff L \xrightarrow{(k,r)} L',k \in I, 0< r\in \mathbb{N}; 
	\end{equation*}
    \begin{equation*}
    	t_{i}(L)=t_{i}(\Phi([L])),t^{\ast}_{i}(L)=t^{\ast}_{i}(\Phi([L])), i\in I.
    \end{equation*}
Moreover, if $\underline{s}=((i_{1},n_{1}),(i_{2},n_{2}),\cdots ,(i_{l},n_{l}) )$ is a left addmissible path of $L$, then  $\Phi([L])=\eta_{i_{1},n_{1}} \eta_{i_{2},n_{2}}\cdots \eta_{i_{l},n_{l}}(Z_{0})$.
\end{proposition}



We can also define the left graph $\mathcal{G}_{0}(\Lambda_{\omega})$ for Nakajima's quiver variety, which is an analogy of  $\mathcal{G}_{1}(\Lambda)$.
\begin{definition}
	We define the left graph $\mathcal{G}_{0}(\Lambda_{\omega})=(\mathcal{V}_{0},\mathcal{E}_{0}) $ of Nakajima's quiver variety (associated with $\omega$) to be the $I \times \mathbb{N}$-colored graph, which consists of the following set of vertices  $\mathcal{V}_{0}$ and the set of arrows $\mathcal{E}_{0}$
	\begin{equation*}
		\mathcal{V}_{0}=\{ X \in Irr \mathfrak{L}(\nu,\omega)| \nu \in \mathbb{N}[I] \},
	\end{equation*}
	\begin{equation*}
		\mathcal{E}_{0}=\{ X' \xrightarrow{(k,r)} X|k\in I, 0 < r \in \mathbb{N}, \varrho_{k,r}(X)=X' \text{ for some } \nu \in \mathbb{N}[I] \}.
	\end{equation*}
\end{definition}

\begin{lemma}[\cite{MR1302318}, Lemma 5.8]
	(1) $\mathfrak{L}(\nu,\omega)$ is isomorphic to the geometric quotient of $\Lambda_{\mathbf{V},\mathbf{W}} \cap \mu^{-1}(0)^{s}$. In particular, the set $Irr \mathfrak{L}(\nu,\omega)$ is bijective to the set of $G_{\mathbf{V}}$-invariant Lagrangian irreducible components of $\Lambda_{\mathbf{V},\mathbf{W}}\cap \mu^{-1}(0)^{s}$.\\
	(2) The projection $\pi_{\mathbf{W}}:\Lambda_{\mathbf{V},\mathbf{W}} \rightarrow \Lambda_{\mathbf{V}};(B,i,0) \mapsto B$
	is a vector bundle. In particular, the set $Irr \Lambda_{\mathbf{V},\mathbf{W}}$ is bijective to the set $ Irr \Lambda_{\mathbf{V}}$.
\end{lemma}
Notice that the set of $G_{\mathbf{V}}$-invariant Lagrangian irreducible components of $\Lambda_{\mathbf{V},\mathbf{W}}\cap \mu^{-1}(0)^{s}$ can be naturally regarded as a subset of $Irr \Lambda_{\mathbf{V},\mathbf{W}}$. With the lemma above, consider an injective map $\Psi:Irr \mathfrak{L}(\nu,\omega) \rightarrow  Irr \Lambda_{\mathbf{V}}$ defined by composing the two bijections above. If $\Lambda=\Lambda_{\omega}$, then for some $\nu \in \mathbb{N}[I]$ and $X \in Irr \mathfrak{L}(\nu,\omega)$, we define
$\Phi^{\Lambda}(X)= (\Phi)^{-1} (\Psi(X)) $. By definition, $\Phi^{\Lambda}:\mathcal{V}_{0} \rightarrow V_{1} $ is injective.

\begin{theorem}\label{thm2}
	  The image of $\Phi^{\Lambda}$ is contained in $\mathcal{V}_{1}(\Lambda)$. Moreover, $\Phi^{\Lambda}: \mathcal{V}_{0} \rightarrow \mathcal{V}_{1}(\Lambda)$ is exactly an isomorphism of ($I \times \mathbb{N}_{>0}$-colored) left graphs,
	  	\begin{equation*}
	  	X' \xrightarrow{(k,r)} X \iff 	\Phi^{\Lambda}(X') \xrightarrow{(k,r)} \Phi^{\Lambda}(X) ,k \in I, 0< r\in \mathbb{N}.
	  	\end{equation*}
\end{theorem}


\begin{proof}
	We firstly prove that ${\rm{Im}}(\Phi^{\Lambda}) \subset \mathcal{V}_{1}(\Lambda)$. Assume that  $\Phi^{\Lambda}(X)=[L]$ and $L \in \mathcal{N}_{\mathbf{V}}$ for some $X$, then there exists some $i\in I$ such that $t_{i}^{\ast}(L)>d_{i}$. Then the irreducible component $Z=\Phi(L) \subset \Lambda_{\mathbf{V}}$ satisfies $t_{i}^{\ast}(Z)=r> d_{i}$. Let $Z'=\pi_{\mathbf{W}}^{-1}(Z)$ be the irreducible component of $\Lambda_{\mathbf{V},\mathbf{W}}$ corresponding to $Z$. By definition, $\pi_{\mathbf{W}}^{-1}(\Lambda_{\mathbf{V},i}^{r} \cap Z)$ is dense in $Z'$.
	We claim that $\pi_{\mathbf{W}}^{-1}(\Lambda_{\mathbf{V},i}^{r} \cap Z) \cap \mu^{-1}(0)^{s}$ is empty. Indeed, for any $(B,i,0)\in  \pi_{\mathbf{W}}^{-1}(\Lambda_{\mathbf{V},i}^{r} \cap Z)$, we consider $S={\rm{Ker}} \bigoplus \limits_{h \in H, h'=i} B_{h} $. $S$ is a $B$-stable subspace, and $S \cap {\rm{Ker}} i \neq 0$. (Otherwise, $i|_{S}$ is an injective linear map from the $r$-dimensional space $S$ to the $d_{i}$-dimensional space $W$). Hence $S \cap {\rm{Ker}} i \subset {\rm{Ker}} i $ fails the stablility condition (S). However, $Z' \cap \mu^{-1}(0)^{s}$ is dense in $Z$. We get a contradiction and have proved that ${\rm{Im}}(\Phi^{\Lambda}) \subset \mathcal{V}_{1}(\Lambda)$.
	
	Notice that by Theorem \ref{thm1}, we have
	\begin{equation*}
		|\mathcal{P}_{\mathbf{V}}\backslash \mathcal{N}_{\mathbf{V}}|={\rm{dim}}_{\mathbb{Q}(v)} L_{|\mathbf{V}|}(\Lambda)
	\end{equation*}
 where $\mathcal{P}_{\mathbf{V}}\backslash \mathcal{N}_{\mathbf{V}}$ is the set of nonzero simple perverse sheaves in $\mathcal{L}_{\mathbf{V}}(\Lambda)$. And by Theorem \ref{Nmain}, we have
	\begin{equation*}
		{\rm{dim}}_{\mathbb{Q}(v)} L_{|\mathbf{V}|}(\Lambda)={\rm{dim}}_{\mathbb{Q}} L_{0,|\mathbf{V}|}(\Lambda) = |Irr\mathfrak{L}(\nu,\omega)|
	\end{equation*}
  hence $\Phi^{\Lambda}: \mathcal{V}_{0} \rightarrow \mathcal{V}_{1}(\Lambda)$ is a bijection. Notice that $\Psi$ commutes with $\eta^{-1}_{k,r}$ and $\varrho_{k,r}$ by definition. Indeed, both  $\eta^{-1}_{k,r}$ and $\varrho_{k,r}$ can be obtained by restricting $B$ to ${\rm{Im}}B_{h} \oplus \bigoplus \limits_{i \neq k}\mathbf{V}_{i} $ for generic $B$. Hence  $\Phi^{\Lambda}$ commutes with the arrows. More precisely, we have
\begin{equation*}
	X' \xrightarrow{(k,r)} X \iff 	\Phi^{\Lambda}(X') \xrightarrow{(k,r)} \Phi^{\Lambda}(X) ,k \in I, 0< r\in \mathbb{N},
\end{equation*}
as desired.
\end{proof}
\nocite{MR1942245}

\subsection{Monomial bases from left graphs}
Let $\mathcal{S}$ be the set of sequences of $I \times \mathbb{N}_{>0}$.
\begin{definition}
For a given order $(i_{1}\prec i_{2} \prec \cdots \prec
i_{n})$ of $I$, we inductively define a map $\underline{s}^{\prec}=\underline{s}: \mathcal{V}_{1}(\Lambda) \rightarrow \mathcal{S}$ as follows, \\
(1) If $L$ is the unique simple perverse sheaf on $\mathbf{E}_{\mathbf{V},\Omega},|\mathbf{V}|=pi,p \leq d_{i} $, we define $\underline{s}([L])=((i,p))$. 
In particular $\underline{s}([L_{0}])=\emptyset $.\\
(2) For the other $L \in \mathcal{V}_{1}(\Lambda)$, there exists a unique $r \in \mathbb{N}$ such that $t_{i_{r}}(L)=n>0$ but $t_{i_{s}}(L)=0$ holds for any $r<s$, take $K$ such that $L \xrightarrow{(i_{r},n)} K$ and we define $\underline{s}([L])=((i_{r},n),\underline{s}([K]))$.
\end{definition}

\begin{lemma}
	The map $\underline{s}^{\prec}=\underline{s}: \mathcal{V}_{1}(\Lambda) \rightarrow \mathcal{S}$ is well-defined and injective.
\end{lemma}
\begin{proof}
    It suffices to show that the induction can be continued.
    If $t_{i_{r}}(L)=n>0$ in $\mathcal{V}_{1}(\Lambda)$, but $K$ such that $L \xrightarrow{(i_{r},n)} K$ belongs to $\mathcal{N}_{\mathbf{V}'}$. Then there exits $j$ such that $s_{j}^{\ast}(K)> 0$ and $K$ is a direct summand of $L_{(\underline{v},j^{d})},d>d_{j}$. By definition of $\pi_{i_{r},n}$, we know that $L$ is a a direct summand of $L_{(i_{r}^{n},\underline{v},j^{d})}$ and belongs to $\mathcal{N}_{\mathbf{V},j}$, a contradiction.
\end{proof}


Similarly, we can inductively define $\underline{s}':\mathcal{V}_{0} \rightarrow \mathcal{S}$. By definition, $\underline{s}([L])$ is a left admissible path of $L$ for any $L \in \mathcal{P}_{\mathbf{V}}\backslash \mathcal{N}_{\mathbf{V}}$ and $\underline{s}(X')$ is a left admissible path of $X$ for any $X\in Irr \mathfrak{L}(\nu,\omega)$.

\begin{definition}
	Given two sequences $$\underline{s}_{1}=((i_{n_{1}},m_{1} ), (i_{n_{2}},m_{2}),\cdots,(i_{n_{k}},m_{k}) ),\ \underline{s}_{2}=((i_{n'_{1}},m'_{1} ), (i_{n'_{2}},m_{2}),\cdots,(i_{n'_{l}},m'_{l}) )  \in \mathcal{S}$$ such that $\sum \limits_{1 \leq s \leq k}m_{s} i_{n_{s}}=\sum\limits_{1 \leq s \leq l}m'_{s}i_{n'_{s}},$ we say $\underline{s}_{2}\prec \underline{s}_{1}$ if there exists $r\in \mathbb{N}$ such that $(i_{n_{t}},m_{t})=(i_{n'_{t}},m'_{t})$ for $1 \leq t< r$ and $(i_{n'_{r}},m'_{r})\prec (i_{n_{r}},m_{r})$.  
\end{definition}
Let $\mathcal{S}_{\mathbf{V}}$ be the subset of $\mathcal{S}$ which consists of the sequences $\underline{s}=((i_{n_{1}},m_{1} ), (i_{n_{2}},m_{2}),\cdots,(i_{n_{k}},m_{k}) )$ such that $\sum \limits_{1 \leq s \leq k}m_{s} i_{n_{s}}=|\mathbf{V}|$, then $(\mathcal{S}_{\mathbf{V}},\prec)$ becomes a partially ordered set. Regraded as  subsets of $\mathcal{S}$ via $\underline{s}$ and $\underline{s}'$, $\mathcal{V}_{1}(\Lambda)$ and  $\mathcal{V}_{0}$ also become  partially ordered sets. We still denote their partial order by $\prec$.


For any sequence
\begin{equation*}
	\underline{s}=((i_{n_{1}},m_{1} ), (i_{n_{2}},m_{2}),\cdots,(i_{n_{k}},m_{k}))  \in \mathcal{S},
\end{equation*}
we define
\begin{equation*}
	m^{\Lambda}_{\underline{s}}=F_{i_{n_{1}}}^{(m_{1})} \ast F_{i_{n_{2}}}^{(m_{2})} \ast \cdots \ast F_{i_{n_{k}}}^{(m_{k})} [\mathfrak{L}(0,\omega)] \in \bigoplus \limits_{\nu}H_{top}( \mathfrak{L}(\nu,\omega)) 
\end{equation*} 
and 
\begin{equation*}
	[M^{\Lambda}_{\underline{s}}]=F_{i_{n_{1}}}^{(m_{1})} F_{i_{n_{2}}}^{(m_{2})} \cdots F_{i_{n_{k}}}^{(m_{k})}[L_{0}] \in \mathcal{K}_{0}(\Lambda).
\end{equation*}

\begin{proposition}
	For fixed $|\mathbf{V}|=\nu$, we set
	\begin{equation*}
		\mathbf{M}^{\Lambda}_{\mathbf{V}}= \{[M^{\Lambda}_{\underline{s}([L])}]|[L] \in \mathcal{V}_{1}(\Lambda) , L \in \mathcal{P}_{\mathbf{V}} \},
	\end{equation*}
	\begin{equation*}
		\mathbf{M'}^{\Lambda}_{\mathbf{V}}= \{m^{\Lambda}_{\underline{s}'(X)}|X \in Irr \mathfrak{L}(\nu,\omega)  \}
	\end{equation*}
	then we have \\
	(1) The set $\mathbf{M}^{\Lambda}_{\mathbf{V}}$ is a $\mathcal{A}$-basis of $\mathcal{K}_{0,|\mathbf{V}|}(\Lambda)$.\\
	(2) The set $\mathbf{M'}^{\Lambda}_{\mathbf{V}}$ is a  $\mathbb{Q}$-basis of $H_{top}( \mathfrak{L}(\nu,\omega))$.\\
	(3) The transition matrix from $\mathbf{B}^{\Lambda}_{1,\mathbf{V}}=\{[L]
	|L \in \mathcal{V}_{1}(\Lambda) ,L \in \mathcal{P}_{\mathbf{V}} \}$ to $\mathbf{M}^{\Lambda}_{\mathbf{V}}$ is upper triangular  (with respect to $\prec$) and with diagonal entries all equal to 1.\\
	(4) The transition matrix from $\mathbf{B}^{\Lambda}_{2,\mathbf{V}}=\{[X]|X \in Irr \mathfrak{L}(\nu,\omega)   \}$ to  $\mathbf{M'}^{\Lambda}_{\mathbf{V}}$ is upper triangular  (with respect to $\prec$) and with diagonal entries all equal to $\pm 1$.
\end{proposition}


\begin{proof}
	We claim that for any $[L] \in \mathcal{V}_{1}(\Lambda)$,
	\begin{equation*}
		M^{\Lambda}_{\underline{s}([L])}=[L]+\sum \limits_{\underline{s}([L'])\succ \underline{s}([L]) } c_{L,L'}[L']
	\end{equation*}
	where $c_{L,L'}$ are constants in $\mathcal{A}$. We argue by induction on the length $k$ of $$\underline{s}([L])=((i_{n_{1}},m_{1} ), (i_{n_{2}},m_{2}),\cdots,(i_{n_{k}},m_{k})).$$ If $k=1$ and $\underline{s}([L])=((i_{n_{1}},m_{1} )$, then $\nu= m_{1}i_{n_{1}}$ and 
	\begin{equation*}
		[L_{i_{n_{1}}^{m_{1}} }] =F_{i_{n_{1}}}^{(m_{1})}[L_{0}]=M^{\Lambda}_{\underline{s}([L])}
	\end{equation*}
	holds trivially. If $k>1$, by Lemma \ref{lkey}, we can take $L' \in \mathcal{P}_{\mathbf{V}'}$ ,$|\mathbf{V}|= |\mathbf{V'}|+m_{1}i_{n_{1}}$  such that $t_{i_{1}}(L')=0, \pi_{i_{1},m_{1}}(L')=L$, and then
	\begin{equation*}
		F_{i_{n_{1}}}^{(m_{1})}[L']=[L]+\sum\limits_{t_{i_{n_{1}}}(L'')>m_{1} } c_{L',L''}[L'']
	\end{equation*}
By the inductive assumption, 
	\begin{equation*}
		M^{\Lambda}_{\underline{s}([L'])}=[L']+\sum \limits_{\underline{s}([L''])\succ \underline{s}([L']) }c_{L',L''}[L''].
	\end{equation*}
	Notice that $\underline{s}([L])=((i_{n_{1}},m_{1}), \underline{s}([L'])) $, we have  
	\begin{equation*}
		\begin{split}	
			M^{\Lambda}_{\underline{s}([L])}=& F_{i_{n_{1}}}^{(m_{1})} M^{\Lambda}_{\underline{s}([L'])} \\
			=& F_{i_{n_{1}}}^{(m_{1})} ([L']+\sum \limits_{\underline{s}([L''])\succ \underline{s}([L']) }c_{L',L''}[L''] ) \\
			=&[L]+\sum\limits_{t_{i_{n_{1}}}(L'')>m_{1} } c_{L',L''}[L'']+\sum \limits_{\underline{s}([L''])\succ \underline{s}([L']) }c_{L',L''}(\mathcal{F}_{i_{n_{1}}})^{(m_{1})}[L''].
		\end{split}
	\end{equation*}
Notice that if $t_{i_{n_{1}}}([L''])=m>m_{1}$, then $\underline{s}([L''])$ either starts with $(i_{n_{1}},m)$ or starts  with $(i_{r},m')$ for some $r<n_{1}$, hence  
	\begin{center}
		$(\sharp)\ \ \underline{s}([L'']))\succ \underline{s}([L])$ holds for $L''$ which satisfies $t_{i_{n_{1}}}(L'')=m>m_{1}$.
	\end{center}
	
	
	We only need to show
	\begin{equation*}
		F_{i_{n_{1}}}^{(m_{1})}[L'']= \sum \limits_{\underline{s}([L'''])\succ \underline{s}([L]) } d_{L'''}[L''']
	\end{equation*}
for those $L''$ which satisfy $\underline{s}([L''])\succ \underline{s}([L'])$. For those $L''$ such that $t_{i_{n_{1}}}([L''])>0$, by \ref{lt}  we have $[L'']$ belongs to the $\mathcal{A}$- submodule  
	$F_{i_{n_{1}}} \mathcal{K}_{0}(\Lambda)$,
	hence
	$F_{i_{n_{1}}}^{(m_{1})}  [L'']$ belongs to $F_{i_{n_{1}}}^{(m_{1}+1)} \mathcal{K}_{0}(\Lambda).$ By Proposition \ref{lt}, $\{[L]|L\in \mathcal{P}, t_{i_{n_{1}}}(L) \geq m_{1}+1  \}$ form a basis of $[L_{i_{n_{1}}^{(m_{1}+1)}}]\ast\mathcal{K}$ in in $\mathcal{K}$. Project this basis to $\mathcal{K}_{0}(\Lambda)$, we know that $\{[L]|L\in \mathcal{P}, L \notin \mathcal{N}, t_{i_{n_{1}}} \geq m_{1}+1  \}$ form a basis of $F_{i_{n_{1}}}^{(m_{1}+1)}\ast\mathcal{K}_{0}(\Lambda)$.
	\begin{equation*}
		F_{i_{n_{1}}}^{(m_{1})}  [L'']=\sum \limits_{t_{i_{n_{1}}}([L'''])>m_{1} } e_{L'''}[L'''].
	\end{equation*}
	By $(\sharp)$, those $L'''$ satisfy $t_{i_{n_{1}}}([L'''])=m>m_{1}$, hence  $\underline{s}([L'''])\succ \underline{s}([L])$. For those $L''$ such that $t_{i_{n_{1}}}([L''])=0$, we have
	\begin{equation*}
		F_{i_{n_{1}}}^{(m_{1})}  [L'']=[\tilde{L}] + \sum \limits_{t_{i_{n_{1}}}([L'''])>m_{1} } e_{L'''}[L''']
	\end{equation*} 
	where $\tilde{L}$ is  a simple perverse sheaf such that $t_{i_{n_{1}}}([\tilde{L}])=m_{1}$. By $(\sharp)$, we have $\underline{s}([L'''])\succ \underline{s}([L])$. So we only need to show $\underline{s}([\tilde{L}])\succ \underline{s}([L])$. Indeed, if $t_{i_{r}}([\tilde{L}]) >0$ holds for some $r< n_{1}$, then $\underline{s}([\tilde{L}])$ starts with $(i_{r},m'),r<n_{1}$ and $\underline{s}([\tilde{L}])\succ \underline{s}([L])$ holds by definition. Otherwise, we have $\underline{s}([\tilde{L}])=((i_{n_{1}},m_{1}),\underline{s}([L''])$. Since $\underline{s}([L''])\succ \underline{s}([L'])$, we have
	\begin{equation*}
		\underline{s}([\tilde{L}])=((i_{n_{1}},m_{1}),\underline{s}([L'']) \succ ((i_{n_{1}},m_{1}),\underline{s}([L']))=\underline{s}([L]).
	\end{equation*}
In a conclusion, we have proved our claim. Then (1) and (3) follow by basic linear algebra and our claim. Similarly, we can prove
	\begin{equation*}
		m^{\Lambda}_{\underline{s}'(X)}=\pm[X]+\sum \limits_{\underline{s}'(X')\succ \underline{s}'(X) } c_{X,X'}[X'] 
	\end{equation*}
	where $c_{X,X'}$ are constants in $\mathbb{Q}$. Then (2) and (4) hold. 
\end{proof}


Recall that we have isomorphisms $\varsigma^{\Lambda}: \mathcal{K}_{0}^{\Lambda} \rightarrow {_{\mathcal{A}}L(\Lambda)}$ and $\varkappa^{\Lambda}:\bigoplus \limits_{\nu}H_{top}( \mathfrak{L}(\nu,\omega)) \rightarrow L_{0}(\Lambda)$. We still denote the composition of $\varsigma^{\Lambda}: \mathcal{K}_{0}^{\Lambda} \rightarrow {_{\mathcal{A}}L(\Lambda)}$ and the classical limits by $\varsigma^{\Lambda}: \mathbb{Z} \otimes _{\mathcal{A}}\mathcal{K}_{0}^{\Lambda} \rightarrow  {_{\mathbb{Z}}L(\Lambda)}$.
\begin{equation*}
\mathbb{Z} \otimes	\mathcal{K}_{0}(\Lambda) \xrightarrow{\varsigma^{\Lambda}} {_{\mathbb{Z}}L_{0}(\Lambda)} 
\end{equation*}
\begin{equation*}
	\bigoplus \limits_{\nu}H_{top}( \mathfrak{L}(\nu,\omega))\xrightarrow{\varkappa^{\Lambda}}  L_{0}(\Lambda) 
\end{equation*}

\begin{definition}
	We define $sgn=sgn^{\succ}: \bigcup \limits_{\nu} Irr \mathfrak{L}(\nu,\omega) \rightarrow \{\pm 1\} $ to be the function which satisfies
	\begin{equation*}
		m^{\Lambda}_{\underline{s}'([X])}=sgn(X)[X]+\sum \limits_{\underline{s}'([X'])\succ \underline{s}'([X]) } c_{X,X'}[X'].
	\end{equation*}
\end{definition}	

For the integrable highest weight module $L_{0}(\Lambda)$ of the universal enveloping algbra, we define 
\begin{equation*}
	\tilde{\mathbf{B}}^{\Lambda}_{1,\mathbf{V}}=\{\varsigma^{\Lambda}([L]
	)|L \in \mathcal{V}_{1}(\Lambda) ,L \in \mathcal{P}_{\mathbf{V}} \};\tilde{\mathbf{B}}_{1}=\tilde{\mathbf{B}}^{\Lambda}_{1}=\bigcup \limits_{\mathbf{V}}\tilde{\mathbf{B}}^{\Lambda}_{1,\mathbf{V}}
\end{equation*}
\begin{equation*}
	\tilde{\mathbf{B}}^{\Lambda}_{2,\mathbf{V}}=\{ \varkappa^{\Lambda}(sgn(X)[X])|X \in Irr \mathfrak{L}(\nu,\omega) \};  \tilde{\mathbf{B}}_{2}=\tilde{\mathbf{B}}^{\Lambda}_{2}=\bigcup \limits_{\mathbf{V}}\tilde{\mathbf{B}}^{\Lambda}_{2,\mathbf{V}}
\end{equation*}

\begin{theorem}
	The transition matrix from $\tilde{\mathbf{B}}_{1}$ to $\tilde{\mathbf{B}}_{2}$ is upper triangular  (with respect to $\prec$) and with diagonal entries all equal to 1.
More precisely, if $X \in \mathcal{V}_{0}$ and $[L]=\Phi^{\Lambda}(X)$, then we have 
	\begin{equation*}
		\varsigma^{\Lambda}([L])=\varkappa^{\Lambda}(sgn(X)[X]) +\sum\limits_{\underline{s}'([X'])\succ \underline{s}'([X]) } c_{X'}\varkappa^{\Lambda} (sgn(X')[X'])
	\end{equation*}
	where $c_{X'} \in \mathbb{Q}$ are constants.
\end{theorem} 
\begin{proof}
	Notice that $\underline{s}(\Phi^{\Lambda}(X))=\underline{s}'(X)$, we have 
	\begin{equation*}
		\begin{split}
			\varkappa^{\Lambda}(m^{\Lambda}_{\underline{s}'(X)})=&F_{i_{n_{1}}}^{(m_{1})}F_{i_{n_{2}}}^{(m_{2})}\cdots F_{i_{n_{k}}}^{(m_{k})} v_{\Lambda} \\
			= &	\varsigma^{\Lambda} ( [M^{\Lambda}_{\underline{s}([L])}])
		\end{split}
	\end{equation*}
	hence $\varkappa^{\Lambda}(\mathbf{M'}^{\Lambda})=\varsigma^{\Lambda}(\mathbf{M}^{\Lambda})$  is a basis of $L_{0}(\Lambda)$, denoted by $\tilde{\mathbf{M}}^{\Lambda}$. Let $P_{i},i=1,2$ be the transition matrices from $\tilde{\mathbf{M}}^{\Lambda}$ to $\tilde{\mathbf{B}}_{i},i=1,2$ respectively, then each  $P_{i}$ is upper triangular  (with respect to $\prec$) and with diagonal entries all equal to 1. Since $\Phi^{\Lambda}$ preserves the order $\prec$, the transition matrix $P_{2}P_{1}^{-1}$ from $\tilde{\mathbf{B}}_{1}$ to $\tilde{\mathbf{B}}_{2}$ is also upper triangular and with diagonal entries all equal to 1.
\end{proof}

\begin{remark}
	We can see that the theorem does not depend on the choice of the order of $I$. More precisely, if we choose another order $\tilde{\prec}$ of $I$, then $\tilde{\prec}$ induces an order of $\mathcal{S}$. We can also define $\underline{s}^{\tilde{\prec}}:\mathcal{V}_{1} \rightarrow \mathcal{S}$ and $\underline{s}'^{\tilde{\prec}}:\mathcal{V}_{2} \rightarrow \mathcal{S}$ in a similar way. Then with the notation above, we still have $\varsigma([L])=\varkappa(f_{Z}) +\sum\limits_{\underline{s}'^{\tilde{\prec}}(f_{Z'})\tilde{\succ} \underline{s}'^{\tilde{\prec}}(f_{Z}) } c_{Z'}\varkappa (f_{Z'})$. Moreover, the transition matrix $P_{2}P_{1}^{-1}$ does not depend on the choice of $\prec$ (up to a permutation). In particular, the function $sgn$ does not depend on the choice of $\prec$, either.
\end{remark}


\begin{definition}
	For $X,X' \in \mathcal{V}_{0}$, we define $[X] \preceq' [X']$ if and only if for any order $\prec$ of $I$ and the induced map $\underline{s}'^{\prec}: \mathcal{V}_{0} \rightarrow \mathcal{S}$, we always have $\underline{s}'^{\prec}([X]) \prec \underline{s}'^{\prec}([X'])$. 
	
	Similarly, given $L,K \in \mathcal{V}_{1}(\Lambda)$, we say $L \preceq K$ if and only if for any order $\prec$ of $I$ and the induced map $\underline{s}^{\prec}: \mathcal{V}_{1} \rightarrow \mathcal{S}$, we have $\underline{s}^{\prec}([L]) \prec \underline{s}^{\prec}([K])$.  
\end{definition}

One can check that $X \preceq' X'$  if and only if $\Phi^{\Lambda}([X]) \preceq \Phi^{\Lambda}([X'])$.

\begin{corollary}
		The transition matrix from $\tilde{\mathbf{B}}_{1}$ to $\tilde{\mathbf{B}}_{2}$ is upper triangular  (with respect to $\preceq$ and $\preceq'$) and with diagonal entries all equal to 1.
	More precisely, if $X \in \mathcal{V}_{0}$ and $[L]=\Phi^{\Lambda}(X)$, then we have 
	\begin{equation*}
		\varsigma^{\Lambda}([L])=\varkappa^{\Lambda}(sgn(X)[X]) +\sum\limits_{X \preceq X' } c_{X'}\varkappa^{\Lambda} (sgn(X')[X'])
	\end{equation*}
	where $c_{X'} \in \mathbb{Q}$ are constants.
\end{corollary}



\section*{}
\subsection*{Acknowledgements.}
J. Fang is partially supported by National Key R$\&$D Program of China [Grant No.
2020YFE0204200] and Tsinghua University Initiative Scientific Research Program [Grant No. 2019Z07L01006]. Y. Lan is partially supported by the National Natural Science Foundation of China [Grant No. 1288201] and Tsinghua University Initiative Scientific Research Program [Grant No. 2019Z07L01006]. J. Xiao is
partially supported by Natural Science Foundation of China [Grant No. 12031007].


\bibliography{mybibfile}

\end{document}
