\begin{abstract}
We propose a weakly supervised approach for creating maps using free-form textual descriptions. We refer to this work of creating textual maps as zero-shot mapping. Prior works have approached mapping tasks by developing models that predict a fixed set of attributes using overhead imagery. However, these models are very restrictive as they can only solve highly specific tasks for which they were trained. Mapping text, on the other hand, allows us to solve a large variety of mapping problems with minimal restrictions. To achieve this, we train a contrastive learning framework called Sat2Cap on a new large-scale dataset with 6.1M pairs of overhead and ground-level images. For a given location and overhead image, our model predicts the expected CLIP embeddings of the ground-level scenery. The predicted CLIP embeddings are then used to learn about the textual space associated with that location. Sat2Cap is also conditioned on date-time information, allowing it to model temporally varying concepts over a location. Our experimental results demonstrate that our models successfully capture ground-level concepts and allow large-scale mapping of fine-grained textual queries. Our approach does not require any text-labeled data, making the training easily scalable. The code, dataset, and models will be made publicly available.
\end{abstract}