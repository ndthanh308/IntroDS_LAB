\documentclass[
10pt, % Main document font size
a4paper, % Paper type, use 'letterpaper' for US Letter paper
oneside, % One page layout (no page indentation)
%twoside, % Two page layout (page indentation for binding and different headers)
headinclude,footinclude, % Extra spacing for the header and footer
BCOR=5mm, % Binding correction
]{scrartcl}

\usepackage[top=2.5cm, bottom=2.5cm, left=2.5cm, right=2.5cm]{geometry}
\hyphenation{Fortran hy-phen-ation} % Specify custom hyphenation points in words with dashes where you would like hyphenation to occur, or alternatively, don't put any dashes in a word to stop hyphenation altogether
\bibliographystyle{naturemag}

\usepackage{amssymb,amsmath,gensymb}
\usepackage{subfigure}
\usepackage{graphicx}
\usepackage[dvipsnames]{xcolor}
\usepackage{hyperref}
\usepackage{listings}
\graphicspath{{figures/}}
\usepackage{caption}
\usepackage{array}
\newcolumntype{L}{>{\centering\arraybackslash}m{4cm}}
\usepackage{braket}
\makeatletter
\DeclareOldFontCommand{\it}{\normalfont\itshape}{\mathit}
\makeatother

% \usepackage[backend=bibtex,style=nature]{biblatex}

% \addbibresource{bib.bib} 

\usepackage{authblk}

\title{Supporting Information. The \textit{rule of four}: anomalous stoichiometries of inorganic compounds}
\author[1]{Elena Gazzarrini}
\author[2]{Rose K. Cersonsky}
\author[1]{Marnik Bercx}
\author[1]{Carl S. Adorf}
\author[1]{Nicola Marzari}
\affil[1]{Theory and Simulation of Materials (THEOS) and National Center for Computational Design and Discovery of Novel Materials (MARVEL), École Polytechnique Fédérale de Lausanne, CH-1015 Lausanne, Switzerland}
\affil[2]{Department of Chemical and Biological Engineering, University of Wisconsin - Madison, Madison, Wisconsin, USA}
\date{}                     %% if you don't need date to appear
\setcounter{Maxaffil}{0}
\renewcommand\Affilfont{\itshape\small}

\begin{document}

\maketitle

\newpage

\section*{I. Materials Databases}

In this section we introduce the two datasets employed in the study and explain how the raw data is obtained.

\subsection*{The Materials Cloud 3-dimensional crystals Database - MC3D}

The MC3D~\cite{mc3d} is a database of structures optimized with the Quantum ESPRESSO code~\cite{Giannozzi2009, Giannozzi2017} using fully-automated workflows developed in AiiDA~\cite{aiida, huber_aiida_2020}.
The starting set of structures for the geometry optimization is obtained from the COD~\cite{COD-ref}, the ICSD~\cite{icsd_n} and the MPDS~\cite{Pauling} databases.
Each CIF file is parsed via an AiiDA workflow that removes unnecessary tags, performs minor corrections to the syntax, and parses the contents to extract the corresponding structure.
The parsed structures are subsequently normalized and primitivized using SeeK-path~\cite{seekpath}, and a uniqueness analysis is performed to remove duplicate structures.
Finally, hydrogen-containing structures from the COD are removed due to the prevalence of molecular crystals in this database, and any structure containing an actinide is also excluded from the database.
The resulting 79\,854 structures \textit{before geometry optimization} are labelled as MC3D-source and used for the analysis in this paper.
% Most (63\,093) of the structures in the MC3D-source come from the MPDS, 13\,798 were obtained from the ICSD and 2\,963 from the COD.
In this early version of MC3D-source, most (63 093) of the structures came from the MPDS, 13\,798 were obtained from the ICSD and 2\,963 from the COD.
Although the vast majority of the structures in the MC3D-source are experimental, some of the structures extracted from the ICSD and COD were found to be flagged as theoretical, i.e. hypothesized in a theoretical study instead of being observed experimentally.
Screening the metadata for these flags, we find 3\,071 theoretical structures, so approximately 3.85\% of the full structure set.

Due to licensing constraints, we are not allowed to publish the full MC3D-source structure set.
Instead, we provide a YAML file on the Materials Cloud archive~\cite{materialscloudarchive} called \texttt{MC3D\_ids.yaml} that contains the list of versions and IDs for each structure extracted from the three databases.

\subsection*{Materials Project - MP}

The Materials Project\,(MP) \cite{jain_commentary_2013} dataset used contains a total of 83\,989 bulk, crystalline, inorganic compounds that have been relaxed with first-principles calculations starting from experimental databases or from structure-prediction methods. It is retrieved through the Matminer ~\cite{matminer} Python library. The version of the database employed in the study dates back to 10/18/2018, corresponding to the \texttt{p\_all\_20181018} dataset retrieved with the \texttt{matminer.datasets} module~\cite{data_access}.

\newpage

\section*{II. SOAP and FPS} 
In this study, it is necessary to have an ML representation that is invariant to symmetry operations and changes smoothly with the Cartesian coordinates.
We choose Smooth Overlap of Atomic Positions (SOAP) vectors, a representation based on smoothed atomic densities: these are abstract feature vectors based on an expansion of atom-centered Gaussians in radial basis functions and spherical harmonics.
This representation discretizes a three-body correlation function including information on each atom, its relationships with neighbouring atoms, and the relationships between sets of neighbours, quantifying similarities between atomic neighborhoods. 

% Using a kernel, the originally linear operations of PCA are performed in a reproducing kernel Hilbert space (RKHS). 
The 3-body SOAP vector is built as 
%
\begin{equation}
\braket{\alpha n \alpha^\prime n^\prime l |\mathcal{X}} \propto \frac{1}{\sqrt{2l+1}}\sum_m \braket{\alpha n lm |\mathcal{X}}^*\braket{\alpha^\prime n^\prime lm |\mathcal{X}}
    % \rho_{i}^{\alpha}(r)=\sum_{j\in \alpha}N_{\sigma}(r-r_{ij})=\sum_{nlm}\langle \alpha n l m | X_{i} \rangle B_{nlm}(r) ~~~, 
\end{equation}
%
where $\alpha$ refers to the species of the considered atoms, and $\braket{\alpha n lm |\mathcal{X}}$ is the expansion of a density field over spherical harmonics and radial bases with $n$ radial bases and $l$ angular channels
%
\begin{equation}
    \braket{\alpha n lm |\mathcal{X}} = \int d\mathbf{r} R_n(r) Y_m^l(\hat{\mathbf{r}})\braket{\alpha r |\mathcal{X}}
\end{equation}
%
Here $\braket{\alpha r |\mathcal{X}}$ encodes the species-tagged density field as a function of $r$ with radius $\sigma$ and accumulated until an interaction cutoff of $r_{cut}$.
For the examples contained in the text, we have used $n_{max} = 4$, $l_{max} = 4$, $\sigma = 0.5$, and $r_{cut} = 3.5$. When using species-invariant SOAP vectors, we use the same hyperparameters but combine all species channels together.

When working with species-tagged SOAP vectors, in order to increase computational efficiency we select the most diverse features using a Furthest Point Sampling (FPS) algorithm, an unsupervised selection method which maximizes diversity (variance) of the selected vectors as measured by the mutual Euclidean distance. 


\newpage

% \section*{IV. Structural symmetry: reduced and original configurations, system types}

% The unit cell is reduced by only considering the biggest atoms which compose it. The percentage of \textit{magic} structures which are also closely packed does not increase considerably. 

% \begin{table}[htbp] 
% \centering
% % \begin{tabular}{|p{0.9cm}|p{1.7cm}|p{1.7cm}|p{1.7cm}|p{1.7cm}|}
% \begin{tabular}{|c|c|c|c|c|}
% \hline
%       symmetry type& \multicolumn{2}{c |}{FCC} & \multicolumn{2}{c |}{HCP}  \\
% \hline
%       structure type& original & reduced & original & reduced \\
% \hline
% \hline
% 3DCD & 38.42\% & 37.87\%  & 49.05\% & 45.38\%\\
% \hline
% MP & 37.69\% &  55.24\% & 40.87\% &  44.67\% \\
% % \hline
% % OQMD & x & x \\
% \hline
% \end{tabular}
% \caption{
%     Percentages of compounds within both the original and the reduced configurations of each data set which are both \textit{magic} and densely-packed (FCC and HCP).
% }
% \label{table:fcc}
% \end{table}

% The combination between unit cell type (P - primitive, I - body-centred, F - face-centred, C - side-centred) and the 7 crystal types (on the right panel of Figure~\ref{fig:br_latt}) gives rise to the Bravais Lattice (BL) types, which we also use as descriptors to show how \textit{magic} compounds mostly belong to simple primitive BL (mP and oP), as shown in the left panel of Figure~\ref{fig:br_latt}.


% % Figure environment removed
% \newpage


\section*{III. PCovR: tuning the mixing parameter} 

The Principate Covariates Regression (PCovR) \cite{helfrecht_structure-property_2020} combines the losses of Linear Ridge Regression (LRR) and Principal Component Analysis (PCA) through the mixing parameter $\beta$. 
The feature matrix which embeds the reduced SOAP representation is projected into latent space with an orthogonal projection. 
Finding the optimal projection to the latent space amounts to minimizing the loss, which happens when the projection is built out of the principal eigenvectors of the covariance matrix of the initial feature matrix.

% Figure environment removed


\newpage


\section*{IV. Energetic analysis with PCovR} 

The following section explores the PCovR energetic analysis performed on the MP dataset with the aim of classifying the structures into the two subgroups by performing a linear regression on local energetic descriptors only. Different covariates are plotted against the first principal covariate (on the $x$ axis each time) to explore the full database variance. Each image is reported in two different views: on the left, the compounds are coloured according to their energetic property, i.e. formation energy per atom (Figure \ref{fig:e_form}), energy above the convex hull (Figure \ref{fig:convex_hull}) and band gap energy (Figure \ref{fig:band_e}), while on the right the same data is coloured according to the subset it belongs to using a kernel density probability estimation (KDE) normalised to the whole set of data.  
The isolated areas containing only \textit{magic} structures mostly contain structures with Mg-O square bonds, ionic bonds with high bond energy and therefore lower formation energy per atom. They validate the SOAP representation's usefulness in separating between compounds' subgroups, but are not enough to draw insightful conclusions on the RoF.
% The top panel on the right of the figure is in agreement with the consideration relating Mg-O square bonds with low formation energy, being them characterized by a high $PF$.

% % Figure environment removed

% Figure environment removed


% Figure environment removed


% Figure environment removed

% % Figure environment removed

% % Figure environment removed
\newpage

\section*{V. Classification algorithms}

The statistics of the accuracy on the test set achieved by the different classification methods are reported in Table \ref{table:classif}. The Random Forest classifier \cite{breiman2001random} is not only the best-performing classifier, but has the smallest discrepancy between using species-invariant and species-tagged representations, which implies that the classification is primarily embedded in the local symmetries, despite the species information.

\begin{table}[htbp!]
\centering
\begin{tabular}{|m{1.5cm}|m{3.0cm}|m{3cm}|m{7.5cm}|}
\hline
Classifier & Test Set R$^2$ & Test Set R$^2$ &Classifier parameters\\
 & (Species-Invariant) & (Species-Tagged) &\\
\hline
\hline
Random Forest & 0.871 & 0.88 &\{\texttt{bootstrap}: True, \texttt{ccp alpha}: 0.0, \texttt{class weight}: None, \texttt{criterion}: gini, \texttt{max depth}: None, \texttt{max features}: sqrt, \texttt{max leaf nodes}: None, \texttt{max samples}: None, \texttt{min impurity decrease}: 0.0, \texttt{min samples leaf}: 1, \texttt{min samples split}: 2, \texttt{min weight fraction leaf}: 0.0, \texttt{n estimators}: 100, \texttt{n jobs}: 4, \texttt{oob score}: False, \texttt{random state}: 2, \texttt{verbose}: 2, \texttt{warm start}: False\}\\
\hline
MLP Classifier & 0.728 & 0.801 &\{\texttt{activation}: relu, \texttt{alpha}: 0.0001, \texttt{batch size}: auto, \texttt{beta 1}: 0.9, \texttt{beta 2}: 0.999, \texttt{early stopping}: False, \texttt{epsilon}: 1e-08, \texttt{hidden layer sizes}: (100,), \texttt{learning rate}: constant, \texttt{learning rate init}: 0.001, \texttt{max fun}: 15000, \texttt{max iter}: 200, \texttt{momentum}: 0.9, \texttt{n iter no change}: 10, \texttt{nesterovs momentum}: True, \texttt{power t}: 0.5, \texttt{random state}: 2, \texttt{shuffle}: True, \texttt{solver}: adam, \texttt{tol}: 0.0001, \texttt{validation fraction}: 0.1, \texttt{verbose}: 2, \texttt{warm start}: False\}\\
\hline
Decision Tree & 0.587 & 0.805 &\{\texttt{ccp alpha}: 0.0, \texttt{class weight}: None, \texttt{criterion}: gini, \texttt{max depth}: None, \texttt{max features}: 80, \texttt{max leaf nodes}: None, \texttt{min impurity decrease}: 0.0, \texttt{min samples leaf}: 1, \texttt{min samples split}: 2, \texttt{min weight fraction leaf}: 0.0, \texttt{random state}: 2, \texttt{splitter}: best\}\\
\hline
Linear SVM & 0.594 & 0.67 &\{\texttt{C}: 1.0, \texttt{class weight}: None, \texttt{dual}: True, \texttt{fit intercept}: True, \texttt{intercept scaling}: 1, \texttt{loss}: squared hinge, \texttt{max iter}: 1000, \texttt{multi class}: ovr, \texttt{penalty}: l2, \texttt{random state}: 2, \texttt{tol}: 0.0001, \texttt{verbose}: 2\}\\
\hline
Cross-Validated Logistic Regression & 0.628 & 0.677 &\{\texttt{Cs}: 10, \texttt{class weight}: None, \texttt{cv}: 2, \texttt{dual}: False, \texttt{fit intercept}: True, \texttt{intercept scaling}: 1.0, \texttt{l1 ratios}: None, \texttt{max iter}: 100, \texttt{multi class}: auto, \texttt{n jobs}: 4, \texttt{penalty}: l2, \texttt{random state}: 2, \texttt{refit}: True, \texttt{scoring}: None, \texttt{solver}: lbfgs, \texttt{tol}: 0.0001, \texttt{verbose}: 2\}\\
\hline
Stochastic Gradient Descent Classifier & 0.594 & 0.61 &\{\texttt{alpha}: 0.0001, \texttt{average}: False, \texttt{class weight}: None, \texttt{early stopping}: False, \texttt{epsilon}: 0.1, \texttt{eta0}: 0.0, \texttt{fit intercept}: True, \texttt{l1 ratio}: 0.15, \texttt{learning rate}: optimal, \texttt{loss}: hinge, \texttt{max iter}: 100, \texttt{n iter no change}: 5, \texttt{n jobs}: 4, \texttt{penalty}: l2, \texttt{power t}: 0.5, \texttt{random state}: 2, \texttt{shuffle}: True, \texttt{tol}: 0.001, \texttt{validation fraction}: 0.1, \texttt{verbose}: 2, \texttt{warm start}: False\}\\
% \hline
% % QDA & 0.593 & 0.606 &\{\texttt{priors}: None, \texttt{reg param}: 0.0, \texttt{store covariance}: False, \texttt{tol}: 0.0001\}\\
% % \hline
% Gaussian Naive Bayes & 0.59 & 0.63 &\{\texttt{priors}: None, \texttt{var smoothing}: 1e-09\}\\
\hline\end{tabular}
\caption{Accuracy on test set achieved by different classifiers.}  
\label{table:classif}
\end{table}

\newpage



% \section*{VI. PCovR: group numbers}
% % The topological descriptors of each compound are merged into one $n$-dimensional array, which is treated as the target vector for the LRR.
% % Figures~\ref{fig:3dcd_geo}~and~\ref{fig:mp_geo} display the PCovR plot generated with Chemiscope, an interactive visualisation tool \cite{chemiscope}, of the 3DCD and MP data sets respectively.
% % The scatter plot of the first and second principal covariates  shows some clustering of \textit{magic} compounds along its extremities, albeit the phenomenon is not classified into the desired sub-classes.
% % The \textit{magic} aggregates in the MP database mostly contain right-angle bonds, square shaped bonds of O and Mg.
% % If the same plot is coloured according to PF and $\alpha$, the vertical cluster in the MP data set is characterized by mid-values for both parameters, showing that the discrete geometric descriptors do not highly correlate to the MRF.
% % A similar behaviour is observed for the experimental data set, where topological descriptors do not convey the diversity of the data set and are not able to provide any sort of classification of \textit{magic} structures.

% % % Figure environment removed

% % % Figure environment removed


% Once the RF classifier succeeds in predicting the probability of a compounds obeying the RoF, an attempt is made to understand which local descriptor correlates to the first principal covariate of the plot in Figure 6 of the main manuscript. In this direction, the SOAP vectors are generated by considering elements belonging to the  same group number as identical, therefore simplifying the materials' space. These peculiar SOAP vectors are used to train the classification algorithm, the classification probabilities are used as target properties, and the PCovR plots are coloured according to some specific group numbers, as shown on the right panel of Figure \ref{fig:mp_group}. The group numbers are orthogonal to the first principal covariate, and are therefore not the determining factors which originate the correlation of the RoF with local symmetries. 

% % Figure environment removed




\newpage

% \bibliographystyle{naturemag}
% \bibliography{bib.bib}

% \printbibliography
\bibliography{bib.bib}

% \bibliography{si}

\end{document}