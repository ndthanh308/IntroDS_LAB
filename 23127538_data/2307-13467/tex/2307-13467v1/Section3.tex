 %!TEX root = jsac_v6.tex

 \vspace{-0.3cm}
\section{Holographic MIMO Communications} \label{SectionIV}
We consider a communication system where the BS is equipped with $M_{\rm BS}$ antennas and serves $K$ single-antenna UEs. The uplink and downlink signal models are derived on the basis of the multiport communication model provided in~\eqref{eq:ioTotal1}, by taking into account that in the uplink $N=1$ and $M=M_{\rm BS}$ while $N=M_{\rm BS}$ and $M = 1$ in the downlink. We assume that lossless matching networks are used at each UE in uplink (i.e., for \emph{power matching}) and downlink (i.e., for \emph{noise matching}). This is reasonable since a single antenna is used at each UE.

%We assume that the same impedances are used for uplink and downlink transmissions, i.e., $Z_G=Z_{G}^{\rm ul}=Z_{G}^{\rm ul}$ and $Z_L=Z_{L}^{\rm ul}=Z_{L}^{\rm dl}$, since differences can be computed by using calibration methods.



\subsection{Uplink data transmission}
%Assume that UE $k$ is transmitting in the uplink. In this case, the matrix ${\bf D}$ and the vector ${\bf v}_{\rm G}$ in~\eqref{eq:ioTotal1} must be replaced with a vector  ${\bf d}_k \in \mathbb{C}^{M}$ and a scalar ${v}_{{\rm G},k} \in \mathbb{C}$, respectively. The vector 

In the uplink, the vector ${\bf v}_{\rm L} \in \mathbb{C}^{M_{\rm BS}}$ of voltages measured at the BS is generated by the superposition of the generator's voltages $\{{v}_{{\rm G},i}; i=1,\ldots,K\}$ of the $K$ single-antenna (i.e., $N=1$) transmitting UEs. The \emph{dimensionless} input-output relation can be obtained from~\eqref{eq:ioTotal1} as
\begin{equation}
\label{eq:ioTotal1_ul}
\frac{{\bf v}_{\rm L}^{\rm ul}}{\sqrt{c}}=\sum_{i=1}^{K}  {\bf d}_i^{\rm ul} \frac{{v}_{{\rm G},i}^{\rm ul}}{\sqrt{c}} +\frac{{\boldsymbol \eta}^{\rm ul}}{\sqrt{c}}
\end{equation}
where $c$ is an arbitrary constant, measured in ${\rm V}^2$, needed to obtain a dimensionless relationship. The vector ${\bf d}_i^{\rm ul}\in \mathbb{C}^{M_{\rm BS}}$ associated with the single-antenna UE $i$ is obtained from~\eqref{eq:D_OC_relation} and reads 
\begin{align}\label{d_i_ul}
{\bf d}_{i}^{\rm ul} = {\bf Q}^{\rm ul}{\bf F}_{{\rm R}}^{\rm ul}{\bf d}_{{\rm OC},i}^{\rm ul} \mathop{=}^{(a)} \frac{{ F}_{{\rm T}}^{\rm ul}} {Z_{\rm G}^{\rm ul}+{ Z}_{{\rm T}}^{\rm ul} }{\bf Q}^{\rm ul}{\bf F}_{{\rm R}}^{\rm ul}  {\bf z}_{{\rm ART},i}^{\rm ul} \mathop{=}^{(b)} \alpha_{\rm ul}(Z_{\rm L}^{\rm ul}{\bf I}_{M_{\rm BS}}+{\bf Z}_{{\rm R}}^{\rm ul})^{-1}{\bf F}_{{\rm R}}^{\rm ul}  {\bf z}_{{\rm ART},i}^{\rm ul}
\end{align}
where $(a)$ follows from~\eqref{voc} whereas $(b)$ is because a matching network for maximum power transfer is used by UE $i$. From~\eqref{Z_T_opt} and~\eqref{F_T_matched}, this implies ${Z}_{\rm T}^{\rm ul} = {(Z^{{\rm ul}}_{\rm G})}^{\ast}$ and ${F}_{{\rm T}}^{\rm ul} = -\mathrm{j} \sqrt{R_{\rm G}}\re\{{Z}_{{\rm AT}}^{\rm ul} \}^{-1/2}$, where ${Z}_{{\rm AT}}^{\rm ul} $ is the transmitting antenna impedance. In~\eqref{d_i_ul}, we have defined
\begin{equation}
\label{alfa_ul}
\alpha_{\rm ul} = -\frac{\mathrm{j}Z_{\rm L}^{\rm ul}}{2\sqrt{{R_{\rm G}^{\rm ul}}{\re\{{Z}_{{\rm AT}}^{\rm ul}\}}}}.
\end{equation}
From~\eqref{available_power} and~\eqref{eq:P_T_new}, the transmit power of a single-antenna UE $i$ can be computed as $P_{{\rm T},i}= b^{\rm ul} P_{{\rm a},i}$
%\begin{equation}
%\label{transmit_power}
%P_{{\rm T},i}= b P_{{\rm A},i}
%\end{equation}
where $P_{{\rm a},i} = \frac{1}{4R_{\rm G}}\Exp\{|v_{{\rm G},i}|^{2} \}$ is its available power and
\begin{equation}
\label{b}
b^{\rm ul}=4R_{\rm G}\dfrac{\re(Z_{\rm T}^{\rm ul})}{|Z_{\rm G}^{\rm ul}+Z_{\rm T}^{\rm ul}|^{2}}\mathop{=}^{(a)} 1
%=\dfrac{4R_{\rm G}\re(Z_{\rm AT})|F_{\rm T}|^{2}}{|Z_{\rm G}+Z_{\rm T}|^{2}}
\end{equation}
is the fraction of available power delivered to the transmitting antenna. Notice that $(a)$ follows because ${Z}_{\rm T}^{\rm ul} = {(Z^{{\rm ul}}_{\rm G})}^{\ast}$ when a matching network for maximum power transfer is used by UE $i$.
% \footnote{Notice that the power matching design does not guarantee the maximum efficiency, which is achieved when $b^{\rm ul}=1/2$.} \textcolor{red}{La nota non è molto chiara} %and requires to use $F_{\rm T}=-\mathrm{j} \sqrt {{R_{\rm G}}/{\re(Z_{\rm AT})}}$.

% % Figure environment removed

By setting $\vect{y}^{\rm ul}= {\bf v}_{\rm L}/\sqrt{c}$, $\vect{h}_{i}^{\rm ul} = {\bf d}^{\rm ul}_i$, $x_i^{\rm ul} = {v}_{{\rm G},i} / \sqrt{c}$ and $\vect{n}^{\rm ul}={\boldsymbol \eta}^{\rm ul}/{\sqrt{c}}$, the input-output relation of the multi-user MIMO system in the form~\eqref{eq:MIMO_channel} follows:
\begin{align} \label{eq:uplink-signal-model_ul}
\vect{y}^{\rm ul}=  \sum_{i=1}^{K} \vect{h}_{i}^{\rm ul} x_i^{\rm ul} + \vect{n}^{\rm ul}.
\end{align}
 The data signal $x_i^{\rm ul}$ from UE~$i$ is modelled as $x_i^{\rm ul}\sim \CN({0}, p_i)$ with 
\begin{equation}
\label{TxPow2}
P_{{\rm T},i}=P_{{\rm a},i} = \dfrac{c}{4R_{\rm G}}p_i .
\end{equation}
%where ${(a)}$ follows from~\eqref{transmit_power} since $b^{\rm ul}=1$ and ${(b)}$ comes from~\eqref{available_power}. 
%The vector $\vect{n}^{\rm ul}\sim \CN(\vect{0}_{M}, \vect{R}_{n}^{\rm ul})$ is independent noise with covariance matrix $\vect{R}_{n}^{\rm ul}$ given by~\eqref{Reta}.
The vector $\vect{n}^{\rm ul}\sim \CN(\vect{0}_{M_{\rm BS}}, \vect{R}_{n}^{\rm ul})$ is independent noise with covariance matrix $\vect{R}_{n}^{\rm ul}= c^{-1}{\bf R}_{\eta}$, where ${\bf R}_{\eta}$ is given by~\eqref{Reta}.
Since $c$ is an arbitrary constant, we assume $c=1\,{\rm V}^2$ without loss of generality.

To decode $x_k^{\rm ul}$, the vector $\mathbf{y}^{\rm ul}$ is processed with the combining vector
$\mathbf{u}_k\in\mathbb{C}^{M_{\rm BS}}$. By treating the interference as noise, the spectral efficiency (SE) for UE $k$ is
$\log_2\left(1+\gamma_k^{\rm ul}\right)$ where
\begin{align}\label{eq:sinr_ul}
  \gamma_k^{\rm ul} = \frac{p_k\left|\mathbf{u}_k^{\Htran} \mathbf{h}_k^{\rm ul}\right|^2}
  { \sum_{i\neq k}{p_i\left|\mathbf{u}_k^{\Htran} \mathbf{h}_i^{\rm ul}\right|^2} + \mathbf{u}_k^{\Htran}\mathbf{R}_n^{\rm ul}\mathbf{u}_k}
\end{align}
is the SINR. We consider both MR and MMSE combining \cite{massivemimobook}. MR has low computational complexity and maximizes the power of the desired signal, but neglects interference. MMSE has higher complexity but it maximizes the SINR in~\eqref{eq:sinr_ul}. In the first
case, $\mathbf{u}_k=\mathbf{h}_k^{\rm ul}/\left\|\mathbf{h}_k^{\rm ul}\right\|$, while in the second case $  \mathbf{u}_k = ( \sum_{i=1}^{K}{{p_i\mathbf{h}_i^{\rm ul} {(\mathbf{h}_i^{\rm ul})}^{\Htran} }} + \mathbf{R}_n^{\rm ul})^{-1} \mathbf{h}_k^{\rm ul}$.
% \begin{align}\label{eq:mmse_ul}
%   \mathbf{u}_k = ( \sum_{i=1}^{K}{{p_i\mathbf{h}_i^{\rm ul} {(\mathbf{h}_i^{\rm ul})}^{\Htran} }} + \mathbf{R}_n^{\rm ul})^{-1} \mathbf{h}_k^{\rm ul}.
% \end{align}


\subsection{Downlink data transmission} 
In the downlink, the voltage ${v}_{{\rm L},k}^{\rm dl} \in \mathbb{C}$ measured at the single antenna of UE $k$ is generated by the voltage vector ${\bf v}_{\rm G} \in \mathbb{C}^{M_{\rm BS}}$ at the BS array. From~\eqref{eq:ioTotal1}, the \emph{dimensionless} input-output relation is
\begin{equation}
\label{eq:ioTotal1_dl}
\frac{{v}_{{\rm L},k}^{\rm dl}}{\sqrt{c}}=  ({\bf d}_k^{\rm dl})^{\Ttran} \frac{{\bf v}_{\rm G}}{\sqrt{c}} +\frac{ \eta_k}{\sqrt{c}}
\end{equation}
where $c$ is an arbitrary constant measured in ${\rm V}^2$. The vector ${\bf d}_k^{\rm dl}\in \mathbb{C}^{M_{\rm BS}}$ is obtained from~\eqref{eq:D_OC_relation}:
\begin{align}\label{d_i_dl}
{\bf d}_{i}^{\rm dl} = {Q}^{\rm dl}{F}_{{\rm R}}^{\rm dl}{\bf d}_{{\rm OC},i}^{\rm dl} \mathop{=}^{(a)} {Q}^{\rm dl}{F}_{{\rm R}}^{\rm dl}  (Z_{\rm G}^{\rm dl}{\bf I}_{M_{\rm BS}}+{\bf Z}_{{\rm T}}^{\rm dl})^{-1} {\bf F}_{{\rm T}}^{\rm dl} {\bf z}_{{\rm ART},i}^{\rm dl} \mathop{=}^{(b)} \alpha_{\rm dl} (Z_{\rm G}^{\rm dl}{\bf I}_{M_{\rm BS}}+{\bf Z}_{{\rm T}}^{\rm dl})^{-1} {\bf F}_{{\rm T}}^{\rm dl} {\bf z}_{{\rm ART},i}^{\rm dl}
\end{align}
since the receiving UE has a single antenna. In particular, $(a)$ derives from~\eqref{doc} as ${\bf D}_{\rm OC}$ is a $1 \times {M_{\rm BS}}$ matrix (i.e., a row vector) whose transpose is exactly $(Z_{\rm G}^{\rm dl}{\bf I}_{M_{\rm BS}}+{\bf Z}_{{\rm T}}^{\rm dl})^{-1} {\bf F}_{{\rm T}}^{\rm dl} {\bf z}_{{\rm ART},i}^{\rm dl}$, whereas $(b)$ follows from~\eqref{ZR_matched} and~\eqref{FR_matched}. %\textcolor{red}{Come si vede, nella precedente equazione compare erroneamente ${\bf I}_N$ perchè siamo in trasmissione e abbiamo applicato le formule generali in cui le antenne in trasmissione sono in numero pari a $N$. Se vogliamo evitare confusione, dovremmo indicare il numero delle antenne della BS con un simbolo diverso, ad esempio $A_{\rm BS}$ e dire che in uplink $N=1$ e $M=A_{\rm BS}$, mentre in downlink $N=A_{\rm BS}$ and $M=1$.} 
Also, we have defined
\begin{equation}
\label{alfa_dl}
\alpha_{\rm dl} = \frac{ \mathrm{j} Z_{\rm L}^{\rm dl} \sqrt{{\re\{Z_{\rm opt}^{\rm dl}\}}}} {(Z_{\rm L}^{\rm dl}+Z_{\rm opt}^{\rm dl}) \sqrt{{\re\{{Z}_{{\rm AR}}^{\rm dl}\}}}}.
\end{equation}
By setting ${y}_k^{\rm dl}= {v}_{{\rm L},k}^{\rm dl}/\sqrt{c}$, ${n}_k^{\rm dl}={\eta_k^{\rm dl}}/{\sqrt{c}}$ and
\begin{align}\label{eq:channel-model_dl}
\vect{h}_{k}^{\rm dl} &= {\bf B}_{\rm dl}^{-\Ttran/2}{\bf d}_k^{\rm dl} \\ {\bf x}^{\rm dl} &= \frac{1}{\sqrt{c}}{\bf B}_{\rm dl}^{1/2}{\bf v}_{\rm G} \label{eq:data-model_dl}
\end{align}
the input-output relation follows in the form
\begin{align} \label{eq:uplink-signal-model_dl}
{y}_k^{\rm dl}=  {(\vect{h}_k^{\rm dl})}^{\Ttran} {\bf x}^{\rm dl} + {n}_k^{\rm dl}
\end{align}
with $P_{\rm T}=\frac{1}{4R_{\rm G}} \Exp\{\re({\bf v}_{\rm G}^{\Htran}{\bf B}_{\rm dl}{\bf v}_{\rm G})\}=\frac{c}{4R_{\rm G}}\Exp\{ ||{\bf x}^{\rm dl} ||^2\} $.
% \begin{equation}
% \label{eq:P_T_new_dl}
% P_{\rm T}=\frac{1}{4R_{G}} \Exp\{\re({\bf v}_{\rm G}^{\Htran}{\bf B}_{\rm dl}{\bf v}_{\rm G})=\frac{c}{4R_{G}}\Exp\{ ||{\bf x}^{\rm dl} ||^2\} \}.
% \end{equation}
Since a noise matching network is used at each UE, we have that {${n}_k^{\rm dl}\sim \CN(0, c^{-1}\sigma_{\rm dl}^2)$} with $\sigma_{\rm dl}^2$ given by~\eqref{sigma_2}.
The vector ${\bf x}^{\rm dl}$ is obtained as 
\begin{align}
{\bf x}^{\rm dl} = \sum_{i=1}^{K}{\bf w}_i{\tilde x}_{i}^{\rm{dl}}
\end{align}
where $\tilde x_i^{\rm dl}\sim \CN({0}, p_i)$ is the information-bearing signal and ${\bf w}_i$ is the precoding vector associated with UE $i$ that satisfies $\Exp\{ ||{\bf w}_i||^2\} = 1$ so that $\Exp\{ ||{\bf x}^{\rm dl}||^2\} = \sum_{i=1}^K p_i$ and $P_{\rm T} = \frac{c}{4R_{\rm G}}\sum_{i=1}^K p_i$. By treating the interference as noise, the downlink SE for UE $k$ is
$\log_2\left(1+\gamma_k^{\rm dl}\right)$, where
\begin{align}\label{eq:sinr_dl}
  \gamma_k^{\rm dl} = \frac{p_k\left|{\mathbf{w}^{\Htran}_k \mathbf{h}_k^{\rm dl}}\right|^2}
  { \sum_{i\neq k}{p_i\left|{\mathbf{w}_i^{\Htran} \mathbf{h}_k^{\rm dl}}\right|^2} + \sigma_{\rm dl}^2}
\end{align}
is the SINR for $c=1\,{\rm V}^2$. We assume that ${\bf w}_k = {\overline{\bf w}_k}/{||\overline{\bf w}_k||}$
% \begin{align}
% {\bf w}_k = \frac{\overline{\bf w}_k}{||\overline{\bf w}_k||}
% \end{align}
and consider both MR precoding with $\overline{\bf w}_k=\mathbf{h}_k^{\rm dl}$ and MMSE precoding with $  \overline{\bf w}_k = ( \sum_{i=1}^{K}{{p_i\mathbf{h}_i^{\rm dl} {(\mathbf{h}_i^{\rm dl})}^{\Htran} }} + \sigma_{\rm dl}^2 {\bf I}_{M_{\rm BS}})^{-1} \mathbf{h}_k^{\rm dl}$.

% \begin{align}\label{eq:mmse_dl}
%   \overline{\bf w}_k = \left( \sum_{i=1}^{K}{{p_i\mathbf{h}_i^{\rm dl} {(\mathbf{h}_i^{\rm dl})}^{\Htran} }} + \sigma_{\rm dl}^2 {\bf I}_M\right)^{-1} \!\!\!\mathbf{h}_k^{\rm dl}.
% \end{align}

\begin{table*}[t!]\vspace{-0.9cm}
\renewcommand{\arraystretch}{1.3}
\centering
\caption{Relationship between uplink and downlink channels with different matching designs.}\vspace{-0.5cm}
\begin{tabular}{c|c|c|c}
{ \bf Channel} & {\bf Arbitrary matching networks} & {\bf Without matching networks } & {\bf With full matching networks } \\
\hline
% Uplink physical channel ${\bf d}_k^{\rm ul}$ & $ \alpha_{\rm ul}(Z_{L}^{\rm ul}{\bf I}_{M}+{\bf Z}_{{\rm R}}^{\rm ul})^{-1}{\bf F}_{{\rm R}}^{\rm ul}  {\bf z}_{{\rm ART},i}^{\rm ul} $ & $\alpha_{\rm ul} (Z_{L}^{\rm ul}{\bf I}_{M}+{\bf Z}_{{\rm AR}}^{\rm ul})^{-1} {\bf z}_{{\rm ART},k}^{\rm ul}$ &  $\xi_{\rm ul} \re\{{\bf Z}^{\rm ul}_{{\rm AR}}\}^{-1/2} {\bf z}_{{\rm ART},k}^{\rm ul}$\\
% %\hline
% %Antennas at the BS & Hertzian dipoles of size $h = \lambda/2$ \\
% \hline
%Dissipation resistance   & $R_{\mathrm r} = \frac{2}{3} \pi Z_{FO}\left(\frac{h}{\lambda}\right)^2$ \\
${\bf d}_k^{\rm dl}$& $ \dfrac {\alpha_{\rm dl}} {\alpha_{\rm ul}} {\bf A}_{\rm dl,ul} {\bf d}_k^{\rm ul} $  & $ \dfrac {\alpha_{\rm dl}} {\alpha_{\rm ul}}  {\bf d}_k^{\rm ul}
$ & $\dfrac {\xi_{\rm dl}} {\xi_{\rm ul}} {\bf d}_k^{\rm ul}$\\
\hline
% Uplink channel ${\bf h}_k^{\rm ul}$& $ {\bf d}_k^{\rm ul}$ & ${\bf d}_k^{\rm ul}$ & ${\bf d}_k^{\rm ul}$\\
% %\hline
% %Antennas at the BS & Hertzian dipoles of size $h = \lambda/2$ \\
% \hline
%Dissipation resistance   & $R_{\mathrm r} = \frac{2}{3} \pi Z_{FO}\left(\frac{h}{\lambda}\right)^2$ \\
${\bf h}_k^{\rm dl}$ & $ \dfrac {\alpha_{\rm dl}} {\alpha_{\rm ul}} {\bf B}_{\rm dl}^{-\Ttran/2}{\bf A}_{\rm dl,ul}{\bf h}_k^{\rm ul}$ & $ \dfrac {\alpha_{\rm dl}} {\alpha_{\rm ul}} {\bf B}_{\rm dl}^{-\Ttran/2} {\bf h}_k^{\rm ul}$ & $\dfrac {\xi_{\rm dl}} {\xi_{\rm ul}} {\bf h}_k^{\rm ul}$\\
\hline
\end{tabular}\vspace{-0.75cm}
\label{tab:duality}
\end{table*}


\subsection{Uplink and downlink duality}
% The reciprocity between uplink and downlink channels is an important property of multiuser MIMO systems operating in TDD that can be exploited in different ways. Particularly, this implies that the channels can be estimated at the BS only using uplink pilots, and used for both uplink and downlink operation. 


% In the context of multiuser MIMO systems, uplink and downlink duality implies that the channel characteristics observed in the uplink are equivalent to those in the downlink. Specifically, if the system satisfies time-reversal symmetry and the channel conditions remain unchanged, the uplink channel vector is mathematically equivalent to the transpose of the downlink channel vector. 

The concept of uplink and downlink duality in wireless communications refers to the relationship between the uplink and downlink channels in a communication system, with the exception of a scaling factor. The duality principle states that the uplink channel vector is proportional to the transpose of the downlink channel vector, with a scaling factor that depends on various factors (e.g., antenna gains). The significance of uplink and downlink duality lies in its practical implications for system design and optimization \cite{massivemimobook}. By exploiting this duality, system parameters and algorithms can be jointly designed for both uplink and downlink transmissions, simplifying system complexity and improving overall performance \cite[Sec. 4]{massivemimobook}. For example, channel estimation and combining techniques developed for uplink can be applied to the downlink without modification, leading to significant savings in complexity. Next, we will discuss this duality in three different cases at the BS: $1$) when arbitrary matching networks (e.g., self-impedance matching networks) are used, $2$) when no matching network is employed, and $3$) when full power and noise matching networks are employed. It should be noted that in all these cases, ${\bf z}_{{\rm ART},k}^{\rm ul}={\bf z}_{{\rm ART},k}^{\rm dl}$ holds, which is a result of the reciprocity principle in electromagnetic propagation.    

\subsubsection{Arbitrary matching networks} In general, when arbitrary matching networks are used (e.g., self-impedance matching networks), we have that ${\bf F}_{{\rm T}}^{\rm dl} \ne {\bf F}_{{\rm R}}^{\rm ul}$ and ${\bf Z}_{{\rm T}}^{\rm dl} \ne {\bf Z}_{{\rm R}}^{\rm ul}$. From~\eqref{d_i_ul} and~\eqref{d_i_dl}, the physical channels ${\bf d}_{k}^{\rm ul}$ and ${\bf d}_{k}^{\rm dl}$ exhibit reciprocity up to a linear transformation. Particularly, ${\bf d}_k^{\rm dl}$ can be obtained from ${\bf d}_k^{\rm dl}$ as
\begin{equation}
\label{eq:d_uld_dl_arbitrary_matching}
{\bf d}_k^{\rm dl} = \dfrac {\alpha_{\rm dl}} {\alpha_{\rm ul}} {\bf A}_{\rm dl,ul} {\bf d}_k^{\rm ul}
\end{equation}
where we have defined ${\bf A}_{\rm dl,ul} = (Z_{\rm G}^{\rm dl}{\bf I}_{M_{\rm BS}}+{\bf Z}_{{\rm T}}^{\rm dl})^{-1} {\bf F}_{{\rm T}}^{\rm dl} ({\bf F}_{{\rm R}}^{\rm ul})^{-1}(Z_{\rm L}^{\rm ul}{\bf I}_{M_{\rm BS}}+{\bf Z}_{{\rm R}}^{\rm ul})$. 
% \begin{equation}
% \label{ }
% {\bf A}_{\rm dl,ul} = (Z_{G}^{\rm dl}{\bf I}_N+{\bf Z}_{{\rm T}}^{\rm dl})^{-1} {\bf F}_{{\rm T}}^{\rm dl} ({\bf F}_{{\rm R}}^{\rm ul})^{-1}(Z_{L}^{\rm ul}{\bf I}_{M}+{\bf Z}_{{\rm R}}^{\rm ul}).
% \end{equation}
A similar reciprocity condition is evident for ${\bf h}_k^{\rm ul}$ and ${\bf h}_k^{\rm dl}$ since $\vect{h}_{k}^{\rm ul} = {\bf d}^{\rm ul}_k$ and $\vect{h}_{k}^{\rm dl} = {\bf B}_{\rm dl}^{-\Ttran/2}{\bf d}_k^{\rm dl}$. In particular, from~\eqref{eq:d_uld_dl_arbitrary_matching}, ${\bf h}_k^{\rm dl}$ can be obtained from $\vect{h}_{k}^{\rm ul}$ only if premultiplied by the matrix ${\bf B}_{\rm dl}^{-\Ttran/2}{\bf A}_{\rm dl,ul}$. 
%This aspect is crucial in the design of the precoding vectors to ensure that the system operates in a properly matched mode. 
The above results for ${\bf d}_k^{\rm dl}$ and ${\bf h}_k^{\rm dl}$ are summarized in the first column of Table \ref{tab:duality}.




\subsubsection{No matching networks} 
In the absence of matching networks at the BS, we have that ${\bf F}_{{\rm T}}^{\rm dl} = {\bf F}_{{\rm R}}^{\rm ul} = {\bf I}_M$, ${\bf Z}_{{\rm T}}^{\rm dl} = {\bf Z}_{{\rm AT}}^{\rm dl}$, ${\bf Z}_{{\rm R}}^{\rm ul} = {\bf Z}_{{\rm AR}}^{\rm ul}$. Accordingly,~\eqref{d_i_ul} and~\eqref{d_i_dl} become
\begin{align}\label{d_i_ul_woMR}
{\bf d}_{k}^{\rm ul} = \alpha_{\rm ul} (Z_{\rm L}^{\rm ul}{\bf I}_{M_{\rm BS}}+{\bf Z}_{{\rm AR}}^{\rm ul})^{-1} {\bf z}_{{\rm ART},k}^{\rm ul} \\\label{d_i_dl_woMR}
{\bf d}_{k}^{\rm dl} = \alpha_{\rm dl} (Z_{\rm G}^{\rm dl}{\bf I}_{M_{\rm BS}}+{\bf Z}_{{\rm AT}}^{\rm dl})^{-1} {\bf z}_{{\rm ART},k}^{\rm dl}.\end{align}
% with
% \begin{equation}
% \label{alfa_ul}
% \alpha_{\rm ul} = -\frac{\mathrm{j}Z_{L}^{\rm ul}}{2\sqrt{{R_{\rm G}^{\rm ul}}{\re\{{Z}_{{\rm AT}}^{\rm ul}\}}}}
% \end{equation}
% and
% \begin{equation}
% \label{alfa_dl}
% \alpha_{\rm dl} = \frac{ \mathrm{j} Z_{L}^{\rm dl} \sqrt{{\re\{Z_{\rm opt}^{\rm dl}\}}}} {(Z_{L}^{\rm dl}+Z_{\rm opt}^{\rm dl}) \sqrt{{\re\{{Z}_{{\rm AR}}^{\rm dl}\}}}}.
% \end{equation}
We observe that ${\bf z}_{{\rm ART},k}^{\rm ul}={\bf z}_{{\rm ART},k}^{\rm dl}$ and ${\bf Z}^{\rm dl}_{\rm AT} = {\bf Z}^{\rm ul}_{{\rm AR}}$ in~\eqref{d_i_ul_woMR} and~\eqref{d_i_dl_woMR}. Therefore, ${\bf d}_{k}^{\rm ul}={\bf d}_{k}^{\rm dl}$ when $Z_{\rm L}^{\rm ul} = Z_{\rm G}^{\rm dl}$. This condition is easily satisfied as it involves the load and generator impedances at the BS. 
% Therefore, it is reasonable to assume $Z_{\rm L}^{\rm ul} = Z_{\rm G}^{\rm dl}$.


As for ${\bf h}_k^{\rm ul}$ and ${\bf h}_k^{\rm dl}$, from~\eqref{matB2} we observe that, in the absence of a power matching network, ${\bf B}_{\rm dl}$ is no longer equal to the identity matrix  ${\bf I}_{M_{\rm BS}}$ but is given by
\begin{equation}
{\bf B}_{\rm dl} =  4R^{\rm dl}_{\rm G}(Z^{\rm dl}_{\rm G}{\bf I}_{M_{\rm BS}}+{\bf Z}^{\rm dl}_{{\rm AT}})^{-\Htran} \re\{{\bf Z}^{\rm dl}_{{\rm AT}}\}  (Z^{\rm dl}_{\rm G}{\bf I}_{M_{\rm BS}}+{\bf Z}^{\rm dl}_{{\rm AT}})^{-1}. 
\end{equation}
Hence, from~\eqref{d_i_ul_woMR} and~\eqref{d_i_dl_woMR} it follows that 
\begin{equation}
\label{ }
{\bf h}_k^{\rm dl} = \dfrac {\alpha_{\rm dl}} {\alpha_{\rm ul}} {\bf B}_{\rm dl}^{-\Ttran/2} {\bf h}_k^{\rm ul}
\end{equation}
where we have used $\vect{h}_{k}^{\rm ul} = {\bf d}^{\rm ul}_k$, $\vect{h}_{k}^{\rm dl} = {\bf B}_{\rm dl}^{-\Ttran/2}{\bf d}_k^{\rm dl}$ and $Z_{\rm L}^{\rm ul} = Z_{\rm G}^{\rm dl}$. The equation above demonstrates that ${\bf h}_k^{\rm dl}$ can be derived from ${\bf h}_k^{\rm ul}$ by multiplying it with the matrix $\dfrac {\alpha_{\rm dl}} {\alpha_{\rm ul}}{\bf B}_{\rm dl}^{-\Ttran/2}$. The results for the downlink channel are summarized in the second column of Table \ref{tab:duality}.



%We first consider the \textit{physical} models for uplink and downlink transmissions given by 
%\eqref{eq:ioTotal1_ul} and~\eqref{eq:ioTotal1_dl}, respectively, in the two cases of \textit{noise matching} In the uplink, the channel vector $\mathbf{h}_k^{\rm ul}\in\mathbb{C}^M$ of UE $k$ in~\eqref{eq:uplink-signal-model_ul} is equal to ${\bf d}_k$, and thus can be obtained as 
%\begin{align}\label{eq:channel_mMIMO_ul}
%{\bf h}_k^{\rm ul} = {\bf Q}_{\rm ul} {\bf F}_{{\rm R}}^{\rm ul}{\bf d}_{{\rm OC},k}^{\rm ul}
%\end{align}
%as it follows from~\eqref{eq:D_OC_relation}. In the downlink, the channel vector $\vect{h}_k^{\rm dl}\in\mathbb{C}^M$ of UE $k$ is defined as~\eqref{eq:channel-model_dl}, and thus can be computed as 
%\begin{align}\label{eq:channel_mMIMO_dl}
%{\bf h}_k^{\rm dl} = {q}_{\rm dl}{\bf B}_{\rm dl}^{-\Htran/2} {\bf F}_{{\rm T}}^{\rm dl}{\bf d}_{{\rm OC},k}^{\rm dl}
%\end{align}
%with ${q}_{\rm dl} = \frac{Z_{L}}{Z_{L}+Z_{\rm opt}}$.
\smallskip
\subsubsection{With power and noise matching networks}
When a noise matching network is used at the BS, the impedance matrix ${\bf Z}_{{\rm MR}}$ is equal to ${\bf Z}_{{\rm MR}}^\star$ in~\eqref{Z_MR}. From~\eqref{FR_matched} and~\eqref{Q_matched}, ${\bf d}_{k}^{\rm ul}$ in~\eqref{d_i_ul} becomes
\begin{equation}
\label{d_i_ul_MR}
{\bf d}_{k}^{\rm ul} = \xi_{\rm ul} \re\{{\bf Z}^{\rm ul}_{{\rm AR}}\}^{-1/2} {\bf z}_{{\rm ART},k}^{\rm ul}
\end{equation}
with $\xi_{\rm ul} = \dfrac{1}{2 \sqrt{R_{\rm G}^{\rm ul} \re\{ Z_{{\rm AT}}^{\rm ul} \}}} \dfrac{Z_{\rm L}^{\rm ul} \sqrt{\re\{Z_{\rm opt}^{\rm ul}\}} }{Z_{\rm L}^{\rm ul} + Z_{\rm opt}^{\rm ul}}$.
% \begin{align}
% \xi_{\rm ul} = \dfrac{1}{2 \sqrt{R_{G}^{\rm ul} \re\{ Z_{{\rm AT}}^{\rm ul} \}}} \dfrac{Z_{L}^{\rm ul} \sqrt{\re\{Z_{\rm opt}^{\rm ul}\}} }{Z_{L}^{\rm ul} + Z_{\rm opt}^{\rm ul}}.
% \end{align}
In the downlink, if a power matching network is used by the BS, then ${\bf Z}_{{\rm MT}}={\bf Z}_{{\rm MT}}^\star$ so that ${\bf Z}_{{\rm T}}^{\rm dl}$ and ${\bf F}_{{\rm T}}^{\rm dl}$ reduce to~\eqref{Z_T_opt}  and~\eqref{F_T_matched}. Hence,~\eqref{d_i_dl} becomes
\begin{align}\label{d_i_dl_MR}
{\bf d}_k^{\rm dl} = \xi_{\rm dl}  \re\{{\bf Z}^{\rm dl}_{{\rm AT}}\}^{-1/2} {\bf z}_{{\rm ART},k}^{\rm dl}
\end{align}
with $\xi_{\rm dl} = \dfrac{1}{2 \sqrt{R_{\rm G}^{\rm dl}\re\{ Z_{{\rm AR}}^{\rm dl} \}}} \dfrac{Z_{\rm L}^{\rm dl} \sqrt{\re\{Z_{\rm opt}^{\rm dl}\}} }{Z_{\rm L}^{\rm dl} + Z_{\rm opt}^{\rm dl}}$.
% \begin{align}
% \xi_{\rm dl} = \dfrac{1}{2 \sqrt{R_{G}^{\rm dl}\re\{ Z_{{\rm AR}}^{\rm dl} \}}} \dfrac{Z_{L}^{\rm dl} \sqrt{\re\{Z_{\rm opt}^{\rm dl}\}} }{Z_{L}^{\rm dl} + Z_{\rm opt}^{\rm dl}}.
% \end{align}
Notice that ${\bf z}_{{\rm ART},k}^{\rm ul}={\bf z}_{{\rm ART},k}^{\rm dl}$. If the BS uses the same array for transmission and reception, then ${\bf Z}^{\rm dl}_{\rm AT} = {\bf Z}^{\rm ul}_{{\rm AR}}$. Putting together the above results yields
\begin{equation}
\begin{split}
\label{ }
\xi_{\rm dl}^{-1}{\bf d}_k^{\rm dl} &=\re\{{\bf Z}^{\rm dl}_{{\rm AT}}\}^{-1/2} {\bf z}_{{\rm ART},k}^{\rm dl} = \re\{{\bf Z}^{\rm ul}_{{\rm AR}}\}^{-1/2} {\bf z}_{{\rm ART},k}^{\rm ul} = \xi_{\rm ul}^{-1} {\bf d}_k^{\rm ul}
\end{split}
\end{equation}
which shows that ${\bf d}_k^{\rm ul}$ and ${\bf d}_k^{\rm dl}$ differ only for a scaling factor. This holds also for $\vect{h}_{i}^{\rm ul}$ and $\vect{h}_{i}^{\rm dl}$ since, in the presence of a power matching network, ${\bf B}_{\rm dl}$ reduces to ${\bf I}_{M_{\rm BS}}$ as it follows from~\eqref{matB2}. The results are summarized in the third column of Table \ref{tab:duality}.

% \smallskip
% \subsubsection{Without matching networks at the BS}
 
% In the absence of matching networks at the BS, we have that ${\bf F}_{{\rm T}}^{\rm dl} = {\bf F}_{{\rm R}}^{\rm ul} = {\bf I}_M$, ${\bf Z}_{{\rm T}}^{\rm dl} = {\bf Z}_{{\rm AT}}^{\rm dl}$, ${\bf Z}_{{\rm R}}^{\rm ul} = {\bf Z}_{{\rm AR}}^{\rm ul}$. Accordingly,~\eqref{d_i_ul} and~\eqref{d_i_dl} become
% \begin{align}\label{d_i_ul_woMR}
% {\bf d}_{k}^{\rm ul} = \alpha_{\rm ul} (Z_{L}^{\rm ul}{\bf I}_{M}+{\bf Z}_{{\rm AR}}^{\rm ul})^{-1} {\bf z}_{{\rm ART},k}^{\rm ul}
% \end{align}
% \begin{align}\label{d_i_dl_woMR}
% {\bf d}_{k}^{\rm dl} = \alpha_{\rm dl} (Z_{G}^{\rm dl}{\bf I}_N+{\bf Z}_{{\rm AT}}^{\rm dl})^{-1} {\bf z}_{{\rm ART},k}^{\rm dl}.\end{align}
% % with
% % \begin{equation}
% % \label{alfa_ul}
% % \alpha_{\rm ul} = -\frac{\mathrm{j}Z_{L}^{\rm ul}}{2\sqrt{{R_{\rm G}^{\rm ul}}{\re\{{Z}_{{\rm AT}}^{\rm ul}\}}}}
% % \end{equation}
% % and
% % \begin{equation}
% % \label{alfa_dl}
% % \alpha_{\rm dl} = \frac{ \mathrm{j} Z_{L}^{\rm dl} \sqrt{{\re\{Z_{\rm opt}^{\rm dl}\}}}} {(Z_{L}^{\rm dl}+Z_{\rm opt}^{\rm dl}) \sqrt{{\re\{{Z}_{{\rm AR}}^{\rm dl}\}}}}.
% % \end{equation}
% We notice that ${\bf z}_{{\rm ART},k}^{\rm ul}={\bf z}_{{\rm ART},k}^{\rm dl}$ and ${\bf Z}^{\rm dl}_{\rm AT} = {\bf Z}^{\rm ul}_{{\rm AR}}$ in~\eqref{d_i_ul_woMR} and~\eqref{d_i_dl_woMR}. Despite this, ${\bf d}_{i}^{\rm ul}$ and ${\bf d}_{i}^{\rm dl}$ are reciprocal only when $Z_{L}^{\rm ul} = Z_{G}^{\rm dl}$. Such a condition can easily be met since it involves the load and generator impedances at the BS. %Accordingly, it can reasonably be assumed $Z_{L}^{\rm ul} = Z_{G}^{\rm dl}$.

% As for the reciprocity between ${\bf h}_k^{\rm ul}$ and ${\bf h}_k^{\rm dl}$, from~\eqref{matB2} we observe that, in the absence of a power matching network, ${\bf B}_{\rm dl}$ is no longer equal to the identity matrix  ${\bf I}_M$ but is given by
% \begin{equation}
% {\bf B}_{\rm dl} =  4R^{\rm dl}_{\rm G}(Z^{\rm dl}_{\rm G}{\bf I}_{N}+{\bf Z}^{\rm dl}_{{\rm AT}})^{-\Htran} \re\{{\bf Z}^{\rm dl}_{{\rm AT}}\}  (Z^{\rm dl}_{\rm G}{\bf I}_{N}+{\bf Z}^{\rm dl}_{{\rm AT}})^{-1}. 
% \end{equation}
% Hence, from~\eqref{d_i_ul_woMR} and~\eqref{d_i_dl_woMR} it follows that 
% \begin{equation}
% \label{ }
% {\bf h}_k^{\rm dl} = \dfrac {\alpha_{\rm dl}} {\alpha_{\rm ul}} {\bf B}_{\rm dl}^{-\Ttran/2} {\bf h}_k^{\rm ul}
% \end{equation}
% where we have used $\vect{h}_{k}^{\rm ul} = {\bf d}^{\rm ul}_k$, $\vect{h}_{k}^{\rm dl} = {\bf B}_{\rm dl}^{-\Ttran/2}{\bf d}_k^{\rm dl}$ and $Z_{L}^{\rm ul} = Z_{G}^{\rm dl}$. The above equation shows that ${\bf h}_k^{\rm dl}$ and ${\bf h}_k^{\rm dl}$ are not reciprocal since they do not simply differ for a scaling factor. To obtain ${\bf h}_k^{\rm dl}$ from ${\bf h}_k^{\rm ul}$, it is necessary to premultiply it with the matrix ${\bf B}_{\rm dl}^{-\Ttran/2}$. This is particularly important in the design of the precoding vectors to avoid that the system operates in a mismatch mode. 
%Notice that in ~\cite{Laas2020_Reciprocity} the preprocessing of ${\bf h}_k^{\rm ul}$ involves both ${\bf B}_{\rm dl}$ and $\vect{R}_{n}^{\rm ul}$. This is because~\cite{Laas2020_Reciprocity} define the uplink channel as $\vect{h}_{i}^{\rm ul} ={ (\vect{R}_{n}^{\rm ul})}^{-1/2} {\bf d}^{\rm ul}_i$.

 %However, it is worth noting that the lack of reciprocity between ${\bf h}_k^{\rm ul}$ and ${\bf h}_k^{\rm dl}$ does not represent a big issue~\cite{Laas2020_Reciprocity}. Indeed, assume that the base station wants to exploit 

%TO BE CONTINUED -- FARE ESEMPIO DELLA PROCEDURA DI PRE-CODING IN DOWNLINK.


%\subsection{Uplink and downlink duality}
%In the uplink, the channel vector $\mathbf{h}_k^{\rm ul}\in\mathbb{C}^M$ of UE $k$ in~\eqref{eq:uplink-signal-model_ul} is equal to ${\bf d}_k$, and thus can be obtained as 
%\begin{align}\label{eq:channel_mMIMO_ul}
%{\bf h}_k^{\rm ul} = {\bf Q}_{\rm ul} {\bf F}_{{\rm R}}^{\rm ul}{\bf d}_{{\rm OC},k}^{\rm ul}
%\end{align}
%as it follows from~\eqref{eq:D_OC_relation}. In the downlink, the channel vector $\vect{h}_k^{\rm dl}\in\mathbb{C}^M$ of UE $k$ is defined as~\eqref{eq:channel-model_dl}, and thus can be computed as 
%\begin{align}\label{eq:channel_mMIMO_dl}
%{\bf h}_k^{\rm dl} = {q}_{\rm dl}{\bf B}_{\rm dl}^{-\Htran/2} {\bf F}_{{\rm T}}^{\rm dl}{\bf d}_{{\rm OC},k}^{\rm dl}
%\end{align}
%with ${q}_{\rm dl} = \frac{Z_{L}}{Z_{L}+Z_{\rm opt}}$.
%\subsubsection{With matching networks at the BS}
%If ${\bf Z}_{{\rm MR}}^\star$ in~\eqref{Z_MR} is otherwise used, then from~\eqref{FR_matched} and~\eqref{Q_matched} we have that~\eqref{eq:channel_mMIMO_ul} reduces to
%\begin{align}\label{eq:channel_mMIMO_Matching_ul}
%{\bf h}_k^{\rm ul}  = \xi_{\rm ul}\re\{{\bf Z}_{{\rm AR}}^{\rm ul}\}^{-1/2}{\bf d}_{{\rm OC},k}^{\rm ul}
%\end{align}
%with
%\begin{align}
%\xi_{\rm ul} =  \mathrm{j}\frac{ Z_{L}\sqrt{\re\{Z_{\rm opt}\}}}{Z_{L}+Z_{\rm opt}}.
%\end{align}
%If ${\bf Z}_{{\rm MT}}^\star$ is otherwise used, then ${\bf B}^{\rm dl} = {\bf I}_M$ and ${\bf F}_{{\rm T}}^{\rm dl}$ reduces to~\eqref{F_T_matched} so that~\eqref{eq:channel_mMIMO_ul} becomes
%\begin{align}\label{eq:channel_mMIMO_Matching_dl}
%{\bf h}_k^{\rm dl} = \xi_{\rm dl}\re\{{\bf Z}_{{\rm AT}}^{\rm dl}\}^{-1/2}{\bf d}_{{\rm OC},k}^{\rm dl}
%\end{align}
%with
%\begin{align}
%\xi_{\rm dl}  = -\mathrm{j} \frac{Z_{L} \sqrt{R_{\rm G}}}{Z_{L}+Z_{\rm opt}}.
%\end{align}
%If the matching networks are used otherwise, from~\eqref{eq:channel_mMIMO_Matching_ul} and~\eqref{eq:channel_mMIMO_Matching_dl} we obtain
%In fact, we have that  
%\begin{equation}
%\vect{h}_k^{\rm dl} =  \frac{\xi_{\rm dl}}{\xi_{\rm ul}}\re\{{\bf Z}_{{\rm AT}}^{\rm dl}\}^{-1/2}\re\{{\bf Z}_{{\rm AR}}^{\rm ul}\}^{1/2}\vect{h}_k^{\rm ul}.
%\end{equation} 
%
%\begin{table}[t]
%\renewcommand{\arraystretch}{1.8}
%\centering
%\caption{Channel models in UL and DL with and without MN.}
%\begin{tabular}{c|c|c}
%{ \bf Channel model} & {\bf w/o MN } & {\bf w/ MN } \\
%\hline
%Uplink ${\bf h}_k^{\rm ul}$ & ${\bf Q}^{\rm ul} {\bf d}_{{\rm OC},k}^{\rm ul}$ & $\alpha_{\rm ul}\re\{{\bf Z}_{{\rm AR}}^{\rm ul}\}^{-1/2}{\bf d}_{{\rm OC},k}^{\rm ul}$\\
%%\hline
%%Antennas at the BS & Hertzian dipoles of size $h = \lambda/2$ \\
%\hline
%%Dissipation resistance   & $R_{\mathrm r} = \frac{2}{3} \pi Z_{FO}\left(\frac{h}{\lambda}\right)^2$ \\
%Downlink ${\bf h}_k^{\rm dl}$  & $ {q}_{\rm dl}{\bf B}_{\rm dl}^{-\Htran/2} {\bf d}_{{\rm OC},k}^{\rm dl}
%$ & $\alpha_{\rm dl}\re\{{\bf Z}_{{\rm AT}}^{\rm dl}\}^{-1/2}{\bf d}_{{\rm OC},k}^{\rm dl}$\\
%\end{tabular}
%\label{tab:scheme}
%\end{table}
%
%\subsubsection{Without matching networks at the BS}
% 
%If no matching network is used at the receiver, i.e., ${\bf F}_{{\rm R}}^{\rm ul} = {\bf I}_M$, then ~\eqref{eq:channel_mMIMO_ul} reduces to
%\begin{align}\label{eq:channel_mMIMO_noMatching_ul}
%{\bf h}_k^{\rm ul} = {\bf Q}_{\rm ul} {\bf d}_{{\rm OC},k}^{\rm ul}
%\end{align}
%with 
%\begin{align}
%{\bf Q}_{\rm ul}=Z_{L}(Z_{L}{\bf I}_{M}+{\bf Z}_{{\rm AR}}^{\rm ul})^{-1}.
%\end{align}
%If no matching network is used at the transmitter, i.e., ${\bf F}_{{\rm T}}^{\rm dl} = {\bf I}_M$, then~\eqref{eq:channel_mMIMO_dl} reduces to
%\begin{align}\label{eq:channel_mMIMO_noMatching_dl}
%{\bf h}_k^{\rm dl} = {q}_{\rm dl}{\bf B}_{\rm dl}^{-\Htran/2} {\bf d}_{{\rm OC},k}^{\rm dl}
%\end{align}
%with 
%\begin{equation}
%\label{matB2_dl}
%{\bf B}_{\rm dl}=4R_{G} (Z_{G}{\bf I}_{N}+{\bf Z}_{{\rm AT}}^{\rm dl})^{-\Htran} \re\{{\bf Z}_{{\rm AT}}^{\rm dl}\}  (Z_{G}{\bf I}_{N}+{\bf Z}_{{\rm AT}}^{\rm dl})^{-1}.
%\end{equation}
%From~\eqref{eq:channel_mMIMO_noMatching_ul} and~\eqref{eq:channel_mMIMO_noMatching_dl} it follows that 
%\begin{equation}
%\vect{h}_k^{\rm dl} =  {q}_{\rm dl}{\bf B}_{\rm dl}^{-\Htran/2}{\bf Q}_{\rm ul}^{-1}\vect{h}_k^{\rm ul}
%\end{equation} 
%from which it follows that a different reciprocity condition holds true because of the whitening of noise coupling through ${\bf Q}_{\rm ul}^{-1}$ and transmit power constraint through ${\bf B}_{\rm dl}^{-\Htran/2}$. 



% Figure environment removed



\vspace{-0.3cm}
\section{The Effect of Coupling: A case study with two antennas in a single path LoS scenario}
To showcase what is the impact of mutual coupling in multi-user MIMO, next we consider a simple scenario in uplink with $K=2$ UEs and ${M_{\rm BS}} =2 $ half-wavelength dipoles in side-by-side
configuration. Transmission takes place over a single LoS propagation path with $\{(\theta_{k},\phi_{k});k=1,2\}$ being the directions of UEs in the far-field of the BS array. For convenience, we let 
\begin{align}\label{Z_AR_2_2}
\re\{{\bf Z}_{{\rm AR}}^{\rm ul}\}= \left(R_{\rm r}+ R_{\rm d}\right)\left[\begin{matrix}
1 & \mu \\
\mu & 1 
\end{matrix}\right]
\end{align}
where $|\mu| < 1$ accounts for the normalized mutual coupling between the two receiving antennas at the BS. The shape of $\mu$ as a function of the normalized antenna spacing $d_H/\lambda$ is reported in Fig.~\ref{fig:mu} for $R_{\rm r} = 73$\si{\ohm} and $R_{\rm d} = 10^{-3}R_{\rm r}$.  Also, we call 
\begin{align}\label{eq:varphi}
\psi_k = 2\pi \frac{d_H}{\lambda}\cos(\theta_k)\sin(\phi_k).
\end{align}
\subsection{Array gain}
The following result is found for the array gain, which is valid %with both MR and MMSE, 
assuming a full matching network.
\begin{lemma} Consider the uplink with $M_{\rm BS}=2$. If a full matching network is used at the BS, then in single path LoS propagation the array gain (compared to a single antenna BS) for UE $k$ is
\begin{align}\label{eq:chanGain_simplified_1}
{\rm Array \, Gain} = 2 \frac{1 - \mu\cos \left(\psi_k\right)}{1 - \mu^2}.
\end{align}
\end{lemma}
\begin{proof}
In the case of full matching networks, the SNR $\gamma_k^{\rm ul} = \frac{p_k\left|\mathbf{u}_k^{\Htran} \mathbf{h}_k^{\rm ul}\right|^2}
  {  \mathbf{u}_k^{\Htran}\mathbf{R}_n^{\rm ul}\mathbf{u}_k}$ in \eqref{eq:sinr_ul} reduces to
  \begin{equation}
      \gamma_k^{\rm ul} = \frac{p_k}{\sigma^2} {{\bf z}_{{\rm ART},k}^{{\rm ul},\Htran}}\re\{{\bf Z}^{\rm ul}_{{\rm AR}}\}^{-1} {\bf z}_{{\rm ART},k}^{\rm ul}
  \end{equation}
as it follows from \eqref{d_i_ul_MR}. By using ${\bf z}_{{\rm ART},k}^{\rm ul} = \alpha(\theta_{k},\phi_{k}) {\bf a}(\theta_{k},\phi_{k})$ and computing the inverse of \eqref{Z_AR_2_2} yields
  \begin{equation}\label{snr_M_2_K_1}
      \gamma_k^{\rm ul} =  \frac{2}{R_{\rm r}+ R_{\rm d}}\frac{1 - \mu\cos \left(\psi_k\right)}{1 - \mu^2} \frac{p_k}{\sigma^2}|\alpha(\theta_{k},\phi_{k})|^2.
  \end{equation}
The array gain is obtained after normalization with $\frac{1}{R_{\rm r}+ R_{\rm d}}\frac{p_k}{\sigma^2}|\alpha(\theta_{k},\phi_{k})|^2$. 
\end{proof}
%We notice that the array gain~\eqref{eq:chanGain_simplified_1} jointly depends on the parameters $\{d_H, (\theta_k, \phi_k)\}$ through $\psi_k$. Particularly,~\eqref{eq:chanGain_simplified_1} takes its maximum and minimum values for $\psi_k = \pi$ and $\psi_k = 0$, respectively. 
%From~\eqref{eq:varphi}, this implies
%\begin{align}
%\left\{\begin{array}{cc}
%\frac{d_H}{\lambda}\cos(\theta_k)\sin(\phi_k) = \frac{1}{2} & \psi_k = \pi  \\ \\
% \frac{d_H}{\lambda}\cos(\theta_k)\sin(\phi_k)= 0 & \psi_k = 0.  \\ \end{array}   \right.\end{align}
% In these two cases,~\eqref{eq:chanGain_simplified_1} reduces to
%\begin{align}
%{\rm Array \, Gain}  =\left\{\begin{array}{cc}
% \frac{2}{{1 - \mu}}& \psi_k = \pi  \\ \\
%\frac{2}{{1 + \mu}} & \psi_k = 0.  \\ \end{array}   \right.\end{align}
%Since $0 \le \mu < 1$, it follows that the coupling between the two antennas may have a positive or negative effective on the array gain depending on $\{d_H, (\theta_k, \phi_k)\}$. Assume for example that the antenna spacing $d_H$ and the elevation angle $\theta_k$ are fixed.\footnote{In a LoS scenario, the elevation angle depends on the distance of UE $k$ and the location of BS array}. In these conditions, the array gain is only a function of the azimuth angle $\phi_k$ and takes the maximum and minimum values respectively for $\phi_k = \sin^{-1}\left(\frac{\lambda}{2\cos(\theta_k)d_H}\right)$ and $\phi_k = n\pi$ with $n$ being an integer.

% Figure environment removed


Lemma 1 shows that an array gain greater than $2$ can only be obtained if $\mu \ne 0$. From Fig.~\ref{fig:mu}, we see that $\mu$ can be positive or negative depending on $d_H/\lambda$, and the first null is at $d_H/\lambda \approx 0.43$. Particularly, $\mu > 0$ for $d_H/\lambda < 0.43$ while negative values are observed for $0.43 < d_H/\lambda < 1$. This has an important impact on the direction of arrival $(\theta_k,\phi_k)$ corresponding to the maximum value of the array gain. 
From~\eqref{eq:chanGain_simplified_1} it can be observed that for a fixed value of $d_H/\lambda$, the maximum value of the array gain is achieved when $\mu > 0$ and corresponds to the minimum value of $\cos \psi_k$ for $(\theta_k,\phi_k)$. On the other hand, if $\mu < 0$ the maximum is achieved for $(\theta_k,\phi_k)$ corresponding to the maximum value of $\cos \psi_k$. 
Assume for example $d_H/\lambda < 0.43$, which means $0 \le 2 \pi d_H/\lambda < 0.86 \pi < \pi$. Since $\mu > 0$, the maximum array gain is attained when $\cos \psi_k$ is minimum, i.e., when $\cos(\theta_k)\sin(\phi_k)=\pm 1$. This condition requires $\theta_k=0$ and $\phi_k=\pm \pi/2$, which represents the \textit{end-fire} direction of arrival. The corresponding maximum array gain is given by 
\begin{equation}
{\rm Maximum \, Array \, Gain} = 2\dfrac{ 1 - \mu \cos(2 \pi d_H / \lambda)}{1 - \mu^2}.    
\end{equation}
If $0.43 < d_H/\lambda < 1$, then $\mu <0$ and the maximum array gain is achieved when $\cos \psi_k$ is maximum, i.e., when $\cos(\theta_k)\sin(\phi_k)=0$. This requires $\phi_k = 0$ or $\theta_k = \pm \pi/2$. In particular, $\theta_k = 0$ and $\phi_k = 0$ corresponds to the \textit{front-fire} direction of arrival. In this case, we obtain 
\begin{equation}
{\rm Maximum \, Array \, Gain} = \dfrac{ 2 }{1 + \mu}.
\end{equation}

% Figure environment removed

Fig.~\ref{fig:sec5_fig4a} reports the SNR in dB for UE $1$ as a function of $\phi_1$ for different values of $d_H$ and with a \textit{full} matching network, i.e., ${\bf Z}_{\rm MR} = {\bf Z}^{\star}_{\rm MR}$. We assume that UE $1$ is located at a distance of $50$ meters and that the BS array is at an height of $10$ meters, which means $\theta_1 \approx -11^{\circ}$. The key parameters of the BS antenna array are reported in Table~\ref{tab:array_parameters}. For comparison, the SNR for the single-antenna case (i.e., $M_{\rm BS}=1$) is shown together with the line corresponding to an array gain of $3$ dB, i.e., the maximum array gain achievable with two uncoupled antennas. In agreement with the discussion above, the results of Fig.~\ref{fig:sec5_fig4a} show that, in the presence of a noise matching network, the array gain is maximum for $\phi_1 = \pm \pi/2$ (end-fire), when the antenna spacing $d_H$ is below $\lambda/4$ since $\mu>0$. On the contrary, it takes the maximum value for $\phi_1 = 0$ (front-fire) when $d_H=\lambda/2$ since $\mu<0$. For all the considered values of $d_H$, there exist ranges of $\phi_1$ for which the array gain is above $3$ dB. This proves that moving the antennas close to each other may have a positive effect that becomes negligible when $d_H$ is further reduced below $\lambda/10$. Interestingly, an array gain greater than $3$ dB can also be obtained for $d_H = \lambda/2$, when the transmitter is in front-fire. This is possible simply because $\mu \ne 0$ for $d_H = \lambda/2$, as shown in Fig.~\ref{fig:mu}. 


The impact of the choice of the matching network on the performance when moving the antennas close to each other is illustrated in Fig.~\ref{fig:sec5_fig4b} and Fig.~\ref{fig:sec5_fig4c}, where we plot the SNR obtained with the self-impedance matching design (see Sect. II.K) and without a matching network. It can be observed that, for a fixed antenna spacing, the maxima and minima occur at the same values of $\phi_1$, regardless of the matching network design. However, the specific values of these maxima and minima are strongly influenced by the choice of the matching network. For instance, Fig.~\ref{fig:sec5_fig4b} demonstrates that reducing $d_H$ below $\lambda/4$ has a negative impact on both SNR and array gain. Furthermore, it is evident that the best performance, whether with a self-impedance matching network or without any matching network, is achieved when $d_H = \lambda/2$ and $\phi_1 = 0$.



% The situation is quite different in the absence of a noise matching network, as shown in Fig.~\ref{fig:sec5_fig4b}. Although the maxima and minima are achieved for the same values of $\phi_1$, the array gain reduces largely as $d_H$ decreases. Particularly, an array gain lower than $3$ dB is achieved for $d_H \le \lambda/4$.    


% To quantify the array gain of UE $1$, Fig.~\ref{fig:array_gain_vs_phi} reports~\eqref{eq:chanGain_simplified_1} in dB as a function of $\phi_1$ for different values of $d_H$. We assume that UE $1$ is located at a distance of $50$ meters and that the BS array is at an height of $10$ meters. Both cases with and without noise matching networks are considered. The key parameters of the BS antenna array are reported in Table~\ref{tab:array_parameters}. In agreement with the above discussion, the results of Fig.~\ref{fig:sec5_fig4a} show that, in the presence of a noise matching network, the array gain is maximum for $\phi_1 = -\pi/2$ (end-fire), when the antenna spacing $d_H$ is below $\lambda/4$ since $\mu>0$. On the contrary, it takes the maximum value for $\phi_1 = 0$ (front-fire) when $d_H=\lambda/2$ since $\mu<0$. For all the considered values of $d_H$, there exist ranges of $\phi_1$ for which the array gain is above $3$ dB, i.e., the maximum array gain achievable with two uncoupled antennas. This proves that the coupling may have a positive effect that vanishes as $d_H$ reaches $\lambda/10$. Interestingly, this positive effect also manifests for $d_H = \lambda/2$, which is typically considered the antenna distance for having uncoupled antennas. The situation is different if no noise matching network is used as Fig.~\ref{fig:sec5_fig4b} shows. Although the maxima and minima are achieved for the same values of $\phi_1$, the array gain reduces largely as $d_H$ reduces. Particularly, it is always below $-6$ dB when $d_H = \lambda/8$ and $d_H = \lambda/10$. 


% \textcolor{red}{Nella figura 5 riporterei una linea orizzontale che rappresenta un guadagno di 3 dB}

% Figure environment removed

To gain further insights into the effect of coupling as $\phi_1$ varies, Fig.~\ref{fig:SNR_spacing} plots the SNR of UE $1$ with a full noise matching network for $0.01 \le d_H/\lambda \le 1$. In particular, the black dashed curve has been obtained with $\phi_1$ uniformly distributed between $-\pi/2$ and $\pi/2$. The other parameters are the same as in Fig.~\ref{fig:sec5_fig4a}. The results are in agreement with those from Fig.~\ref{fig:sec5_fig4a}. Specifically, Fig.~\ref{fig:SNR_spacing} shows that in the presence of a noise matching network gains are achieved depending on the values of $\phi_1$. If $\phi_1$ is uniformly distributed between $-\pi/2$ and $\pi/2$, a minimal gain is achieved for $0.1 \le d_H/\lambda \le 1$ compared to uncoupled antennas. A loss is observed for small values of $d_H/\lambda$. This is a direct consequence of the dissipation resistance. To better understand this effect, the following corollary is given with $\mu_0 = \frac{R_{\rm r}}{R_{\rm r} + R_{\rm d}}$, $\mu_2 = \frac{\pi}{2} \frac{Z_0}{R_{\rm r} + R_{\rm d}}$ and $Z_0 = 377 \si{\ohm}$.
% \begin{equation}
% \mu_0 = \frac{R_{\rm r}}{R_{\rm r} + R_{\rm d}} \quad \quad .
% \end{equation}
\begin{corollary} If ${d_{H}/\lambda \approx 0}$, then~\eqref{eq:chanGain_simplified_1} reduces to  
\begin{equation}\label{eq:asymp_array_gain}
2\frac{1 - \mu_0 + \left[2\mu_0\pi^{2} (\cos \theta_k \sin \phi_k)^{2} + \mu_2\right](d_{H}/\lambda)^{2}}{(1 + \mu_0)\left[1 - \mu_0 + \mu_2 (d_{H}/\lambda)^{2}\right]}.
\end{equation}
If ${d_{H}}/{\lambda} \to 0$, then~\eqref{eq:chanGain_simplified_1} tends to $\frac{2\mu_0\pi^{2} (\cos \theta_k \sin \phi_k)^{2} + \mu_2}{\mu_0\mu_2}$.
% \begin{equation}\label{eq:asymp_array_gain_limit}
% \frac{2\mu_0\pi^{2} (\cos \theta_k \sin \phi_k)^{2} + \mu_2}{\mu_0\mu_2}.
% \end{equation}
\end{corollary} 
\begin{IEEEproof}
With half-wavelength dipoles in side-by-side configuration, $\mu$ can be approximated as $\mu \approx \mu_{0}-\mu_{2} (d_{H}/\lambda)^{2}$ as $ d_{H}/\lambda \approx 0$. 
The Taylor expansion of $\cos \left(\psi_k\right)$ for $ d_{H}/\lambda \approx 0$ is $1-2 \pi^{2} (\cos\theta_k \sin\phi_k)^{2} (d_{H}/\lambda)^{2}$. Plugging these expressions into~\eqref{eq:chanGain_simplified_1} yields $\eqref{eq:asymp_array_gain}$ from which the asymptotic value for ${d_{H}}/{\lambda} \to 0$ follows.
\end{IEEEproof}


Fig.~\ref{fig:ArrayGain_vs_Rd} depicts the variations of~\eqref{eq:chanGain_simplified_1} and~\eqref{eq:asymp_array_gain} with respect to $d_H/\lambda$ for different values of the ratio $R_{\rm d}/R_{\rm r}$. The parameter $\phi_1$ is fixed at $-\pi/2$. The figure demonstrates the significant impact of $R_{\rm d}$ on the array gain. Specifically, as $R_{\rm d}/R_{\rm r}$ increases, the maximum array gain occurs at larger values of $d_H/\lambda$, while poor performance is observed when $d_H/\lambda$ approaches $0$.

% If $R_d = 0$ then~\eqref{eq:asymp_array_gain} becomes
% \begin{equation}
% 4 \pi \left(  \frac{R_{r}}{Z_0} \left(\cos\theta_k \sin\phi_k\right)^{2} +1\right).
% \end{equation}

% \begin{remark}
% \textcolor{red}{INSERIAMO QUALCHE COMMENTO? Anche in merito al fatto che il modello decade quando $\frac{d_{H}}{\lambda} \to 0$}
% \textcolor{red}{IL VALORE ASINTOTICO AUMENTA ALL'AUMENTARE DI $R_d$?}
% \end{remark}


\subsection{Interference gain}
The mutual coupling between antennas has also an impact on the interference term $\left|\mathbf{u}_k^{\Htran} \mathbf{h}_i^{\rm ul}\right|^2$ in~\eqref{eq:sinr_ul}. To show this, the following result is given for MR, i.e., $\mathbf{u}_k = \mathbf{h}_k^{\rm ul}$.

% \textcolor{red}{Continuo a pensare che chiamarlo \textit{interference gain} sia misleading perchè fa pensare che più è grande meglio si comporta il sistema nei confronti dell'interferenza.}

% Figure environment removed

\begin{lemma} \label{lemma_3}Consider the uplink with MR and assume that $M_{\rm BS}=2$. If a full matching network is used at the BS, then in a single path LoS propagation scenario the interference gain (compared to a single antenna BS) between UEs $k$ and $i$ is
\begin{align}
{\rm Interference \, Gain} = \dfrac{1+\mu^{2}-2\mu (\cos \psi_{k} + \cos \psi_{i})}{(1-\mu \cos \psi_{k} )(1-\mu^{2})}  + \dfrac{\cos(\psi_{k}-\psi_{i})
+ \mu^{2} \cos(\psi_{k}+\psi_{i})}{(1-\mu \cos \psi_{k} )(1-\mu^{2})}\label{Gamma}
\end{align}
\end{lemma}
\begin{IEEEproof}
With MR and full matching networks, the normalized interference term is 
\begin{equation}
\label{Interf1}
\frac{p_i\left|\mathbf{u}_k^{\Htran} \mathbf{h}_i^{\rm ul}\right|^2}
  {  \mathbf{u}_k^{\Htran}\mathbf{R}_n^{\rm ul}\mathbf{u}_k} = \dfrac{p_i |\alpha(\theta_{i},\phi_{i})|^2}{\sigma^2}  \dfrac{|{\bf a}(\theta_{k},\phi_{k})^{H} \re\{{\bf Z}_{{\rm AR}}^{\rm ul}\}^{-1} {\bf a}(\theta_{i},\phi_{i})|^{2}}{{\bf a}(\theta_{k},\phi_{k})^{H} \re\{{\bf Z}_{{\rm AR}}^{\rm ul}\}^{-1} {\bf a}(\theta_{k},\phi_{k})}
\end{equation}
as it follows from \eqref{d_i_ul_MR}. From \eqref{Z_AR_2_2}, ${\bf a}(\theta_{k},\phi_{k})^{H} \re\{{\bf Z}_{{\rm AR}}^{\rm ul}\}^{-1} {\bf a}(\theta_{i},\phi_{i})$ is obtained as
\begin{equation}
2 \dfrac{1+\mu^{2}-2\mu (\cos \psi_{k} + \cos \psi_{i}) + \cos(\psi_{k}-\psi_{i})
+ \mu^{2} \cos(\psi_{k}+\psi_{i})}{(R_{r}+R_{d})^{2}(1-\mu^{2})^{2}}
\end{equation}
and ${\bf a}(\theta_{k},\phi_{k})^{H} \re\{{\bf Z}_{{\rm AR}}^{\rm ul}\}^{-1} {\bf a}(\theta_{k},\phi_{k})$ can be obtained from \eqref{snr_M_2_K_1}. The result in \eqref{Gamma} follows. 
\end{IEEEproof}
% Similarly to the array gain, the interference gain also depends on the parameters $\{d_H, (\theta_k, \phi_k)\}$ and $\{d_H, (\theta_i, \phi_i)\}$ through $\psi_k$ and $\psi_i$. The expression is slightly more involved to gain insights into the interplay of different parameters. 
% Numerical results (not shown for space limitations) show that the coupling arising with densely spaced antennas may have or not positive effects on the  interference rejection capabilities of MR and MMSE depending on the directions of arrival of the interfering signals. The use of matching networks is mandatory.

% Both the array gain and the interference gain contribute to the SINR, as given in~\eqref{eq:sinr_ul}, and hence, in the last analysis, to the spectral efficiency for the different UEs.



Similarly to the array gain, the interference gain is also influenced by the parameters ${d_H, (\theta_k, \phi_k)}$ and ${d_H, (\theta_i, \phi_i)}$ through $\psi_k$ and $\psi_i$. The expression for the interference gain is more complex, making it challenging to gain direct insights into the interplay of these parameters. However, the numerical results shown in Fig.~\ref{fig:Interference_vs_phi_M2K2} reveal that the coupling effects observed with densely spaced antennas can either enhance or hinder the interference rejection capabilities of MR (but the same considerations apply to MMSE), depending on the directions of arrival of the interfering signals. For example, assuming that UE 1 is in end-fire (as in Fig.~\ref{fig:sec5_fig6a}) the best performance is observed with $\phi_2=0^{\circ}$ when $d_H = \lambda/2$ and for $\phi_2 \approx 24^{\circ}$ when $d_H = \lambda/10$. It is worth observing that, when $\phi_1 = 0^{\circ}$, moving the antennas close to each other has minimal effects on interference rejection, as shown in Fig.~\ref{fig:sec5_fig6c}. In this case, the best performance is obtained with $d_H = \lambda/2$ and $\phi_2 \pm 90^{\circ}$. 
% Other results (not shown for space limitations) indicate that the use of matching networks is crucial in multi-user scenarios.



% Figure environment removed

\subsection{Spectral efficiency}
Both array and interference gains contribute to the overall SINR and ultimately impact the spectral efficiency of the different UEs. To quantify this, Fig.~\ref{fig:SE_vs_phi_M2K2} plots 
the spectral efficiency  of UE 1 as a function of $\phi_2$ in the same simulation scenario of Fig.~\ref{fig:Interference_vs_phi_M2K2}. The single antenna case is also reported as a benchmark. It is assumed the same transmit power for the two users. The results in Figs.~\ref{fig:sec5_fig6b}-\ref{fig:sec5_fig6d} can easily be explained with those in Fig.~\ref{fig:SNR_spacing} and Figs.~\ref{fig:sec5_fig6a}-\ref{fig:sec5_fig6c}. In particular, as expected, the points of minimum/maximum in Figs.~\ref{fig:sec5_fig6b} and \ref{fig:sec5_fig6d} correspond to the points of maximum/minimum in Figs.~\ref{fig:sec5_fig6a} and \ref{fig:sec5_fig6c}. When $\phi_1 = -90^{\circ}$ (as in Fig.~\ref{fig:sec5_fig6b}), the maximum SE (more than four times larger compared to the single antenna case) is achieved with $d_H = \lambda/10$ (when $\phi_2 \approx 24^{\circ}$) but significant gains are also observed for $d_H = \lambda/2$ when $\phi_2 = 0^{\circ}$. On the other hand, when $\phi_1 = 0^{\circ}$ poor performance is obtained by moving the antennas close to each other, and $d_H=\lambda/2$ is the best option. 



Fig.~\ref{fig:Uplni_SE_with_K_2} plots the average SE per UE with different matching networks: `Full', `SI (Self-impedance)' and `No' matching network (MN). 
% The key parameters are reported in Table~\ref{tab:array_parameters}.
We assume that the angles of arrival $\phi_1$ and $\phi_2$ for the two UEs are within the range $[-\pi/2, \pi/2]$ and that both UEs are located at a distance of $50$ meters.
% % In a multi-user scenario, the main parameter for UE $k$ is represented by its SINR $\gamma_{k}^{\rm ul}$, given in~\eqref{eq:sinr_ul}. Similarly to the SNR, $\gamma_{k}^{\rm ul}$ depends on $\{d_H, (\theta_k, \phi_k)\}$ but also on the power and directions of arrival of the signals of the other users.  In the two-users case, expressions of the SINRs can be obtained as functions of $\{d_H, (\theta_1, \phi_1), (\theta_2, \phi_2)\}$ and the relative powers but they are slightly more involved (compared to~\eqref{eq:chanGain_simplified_1}) to gain insights into the interplay of different parameters. Numerical results (not reported for space limitations) show that the coupling arising with densely spaced antennas may have or not positive effects on the interference rejection capabilities of MR and MMSE depending on the directions of arrival of the interfering signals. The use of matching networks is mandatory.
% % The SINR determines the spectral efficiency which is given by
% \textcolor{red}{Bisogna definire l'efficienza spettrale. BISOGNA DIRE COME SONO STATE OTTENUTE LE CURVE IN FIG. 7. SE HO BEN CAPITO, SI RIPORTA L'EFFICIENZA SPETTRALE MEDIA PER UTENTE, CIO\`E L'EFFICIENZA SPETTRALE MEDIA TOTALE DIVISO 2. CHE VALORI ASSUMONO $(\theta_1, \phi_1)$ e $(\theta_2, \phi_2)$? Come si distribuiscono gli utenti rispetto alla stazione base? Sono alla stessa distanza, che è fissa per tutta la simulazione? }
% In Fig.~\ref{fig:Uplni_SE_with_K_2}, we assume $K=2$ and plot the uplink SE of UE $1$ with and without the noise matching network when the azimuh angle of UE $2$ is uniformly distributed within the interval $[-\pi,0]$. Two different setups for UE 1 are considered with $\theta_1 = -\pi/2$ (end-fire) or $\theta_1\in [-\pi,0]$. 
The results indicate that decreasing $d_H/\lambda$ has a detrimental impact on the spectral efficiency, regardless of the matching network employed. Additionally, it is evident that the best performance is achieved when $d_H/\lambda \geq 0.5$, meaning that there is no significant advantage in using a full matching network compared to the self-impedance matching design. As anticipated, a considerable reduction in spectral efficiency is observed when no noise matching network is utilized.

% Figure environment removed



%\begin{lemma} If a matching network is used in the UL at the BS, then in LoS propagation with $M_{\rm BS}=2$ the interference term $\frac{p_2}{\sigma^2}{{\left|\mathbf{h}_1^{\Htran} \mathbf{h}_2\right|^2}}/{ \mathbf{h}_1^{\Htran} \mathbf{h}_1}$ reduces to~\eqref{eq:interf_simplified}.
%\end{lemma}
%\begin{proof}
%The proof is provided in Appendix B.
%\end{proof}



%We proceed considering the case in which a matching network is not used at the BS. We let 
%\begin{align}
%{\bf Z}_{{\rm AR}}^{\rm ul}= \left(Z_r+ Z_d\right)\left[\begin{matrix}
%1 & \mu \\
%\mu & 1 
%\end{matrix}\right]
%\end{align}
%and define $ \underline \gamma_{i, M_{\rm BS} = 1}$ the SNR of UE $i=1,2$ with a single antenna at the BS. Its epxression is given in~\eqref{eq_snr_M_1_no_matching}.
%\begin{lemma} If no matching network is used in the UL at the BS, then in LoS propagation we have that
%\begin{align}\label{eq:chanGain_simplified_1_noMatching}
%\mathbf{h}_1^{\Htran} \mathbf{h}_1 = 2 \frac{1 - \mu\cos \left(\psi_1\right)}{1 - \mu^2}   \underline \gamma_{1, M_{\rm BS} = 1}
%\end{align}
%with $\psi_1 = 2\pi \frac{d_H}{\lambda}\cos(\theta_1)\sin(\phi_1)$.
%%\begin{align}\label{eq:varphi}
%%\psi_1 = 2\pi \frac{d_H}{\lambda}\cos(\theta_1)\sin(\phi_1).
%%\end{align}
%\end{lemma}

