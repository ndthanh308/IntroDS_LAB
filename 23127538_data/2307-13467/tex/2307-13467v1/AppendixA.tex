%!TEX root = jsac_v6.tex

\section*{Appendix A}\label{Appendix}
Let ${\bf E}_{\rm inc}$ denote an electric field incident on the receive array, produced by the voltage sources. We model ${\bf E}_{\rm inc}$ as a plane wave that reaches the receiver from a particular azimuth angle $\phi_{\rm inc} \in [-\pi/2, \pi/2)$ and elevation angle $\theta_{\rm inc} \in [-\pi/2, \pi/2)$. In this case, assuming that the receive array consists of canonical minimum scattering (CMS) antennas and the incident field is linearly polarized, the $m$th element of ${\bf v}_{{\rm OC}}$ can be written as
\begin{equation}
\label{voc_einc}
v_{{\rm OC},m} = {\bf E}_{\rm inc} \cdot {\bf l}_{{\rm eff},m}(\theta_{\rm inc},\phi_{\rm inc})
\end{equation}
where ${\bf l}_{{\rm eff},m}(\theta_{\rm inc},\phi_{\rm inc})$ is the \textit{effective length} (or \textit{effective height}) \cite[Sect. 2.15]{balanis} of the \textit{isolated} $m$th element of the array, in the $(\theta_{\rm inc},\phi_{\rm inc})$ direction. The effective length is a far-field parameter given by ${\bf l}_{{\rm eff}}(\theta,\phi) =  {l}_{\theta}(\theta,\phi)\widehat{\boldsymbol \theta} +{l}_{\phi}(\theta,\phi)\widehat{\boldsymbol \phi}$~\cite[Eq. (2-91)]{balanis}
% \begin{equation}
% \label{leff}
% {\bf l}_{{\rm eff}}(\theta,\phi) =  {l}_{\theta}(\theta,\phi)\widehat{\boldsymbol \theta} +{l}_{\phi}(\theta,\phi)\widehat{\boldsymbol \phi}
% \end{equation}
where $\widehat{\boldsymbol \theta}$ and $\widehat{\boldsymbol \phi}$ are the \textit{local} orthogonal unit vectors in the directions of $\theta$ and $\phi$. For $\lambda/2$-dipoles, we have ${\bf l}_{{\rm eff}}(\theta,\phi) =  \lambda g(\theta,\phi)\widehat{\boldsymbol \theta}$
% \begin{equation}
% \label{leff}
% {\bf l}_{{\rm eff}}(\theta,\phi) =  \lambda g(\theta,\phi)\widehat{\boldsymbol \theta}
% \end{equation}
where $g(\theta,\phi)=\frac{\cos\left(\frac{\pi}{2} \sin \theta \right)}{\pi \cos \theta}$ is the \textit{radiation pattern}. The effective length is also useful for expressing the field ${\bf E}$ radiated by an antenna at a distance $d$, in the far-zone region. Denoting with $i_{\rm AT}$ the current feeding the antenna, we can write~\cite[Eq. (2-92)]{balanis}
\begin{equation}
\label{eq:Einc}
\begin{split}
{\bf E}= -\imagunit k  Z_{0} i_{\rm AT} \dfrac{e^{- \imagunit k d}}{4 \pi d} {\bf l}_{{\rm eff}} (\theta_{t}, \phi_t)  
\end{split}
\end{equation}
where $(\theta_{t},\phi_{t})$ is the angle of departure.
% Now, consider the electric field produced by a $\lambda/2$-dipole  at a distance $d$, in the far-field region. We have
% \begin{equation}
% \label{eq:Einc}
% \begin{split}
% {\bf E}=\imagunit k  Z_{0} i_A \dfrac{e^{- \imagunit k d}}{4 \pi d} {\bf l}_{{\rm eff}} (\theta_{t}, \phi_t)  
% \end{split}
% \end{equation}
% where $i_A$ is the current feeding the antenna and $(\theta_{t},\phi_{t})$ is the angle of departure. 
Based on \eqref{voc_einc} and \eqref{eq:Einc}, we can compute ${\bf v}_{{\rm OC}}$ and ${\bf Z}_{{\rm ART}}$ in a LoS scenario. Assuming a single transmitting antenna (i.e., ${\bf Z}_{{\rm ART}}$ reduces to a vector ${\bf z}_{{\rm ART}}$) and a receive ULA consisting of CMS antennas with the same effective length, we obtain 
\begin{equation}
\label{z_ART}
{\bf z}_{{\rm ART}} =  Z_0 \alpha'(\theta_{\rm inc},\phi_{\rm inc})  {\bf a}(\theta_{\rm inc},\phi_{\rm inc})
\end{equation}
where
\begin{equation}
\label{alfap}
\alpha'(\theta_{\rm inc},\phi_{\rm inc}) =  \imagunit e^{\imagunit \psi_0}\dfrac{{\bf l}_{{\rm eff}} (\theta_{t}, \phi_t) \cdot {\bf l}_{{\rm eff}} (\theta_{\rm inc},\phi_{\rm inc})}{2 \lambda d} 
\end{equation}
$\psi_0$ is the reference phase at the center of receive array, $d$ is the distance between the centers of transmitting dipole and receive array, ${\bf a}(\theta,\phi) = [ e^{\imagunit {\bf k}(\theta,\phi)^{\Ttran}{\bf u}_1},\ldots, e^{\imagunit {\bf k}(\theta,\phi)^{\Ttran}{\bf u}_M}]^{\Ttran}$, ${\bf k}(\theta,\phi) = \frac{2\pi}{\lambda} [\cos(\theta) \cos(\phi), \cos(\theta) \sin(\phi), \sin(\theta)]^{\Ttran}$ is the wave vector that describes the phase variation of the plane wave with respect to the Cartesian coordinates and ${\bf u}_m$ is the center coordinate of dipole $m$. It is worth noting that the term ${\bf l}_{{\rm eff}} (\theta_{t}, \phi_t) \cdot {\bf l}_{{\rm eff}} (\theta_{\rm inc},\phi_{\rm inc})$ in \eqref{alfap} accounts for the polarization loss~\cite[Sect. 2.12.2]{balanis}. Taking \eqref{z_ART} and \eqref{alfap} into account, from ${\bf v}_{{\rm OC}}=i_{\rm AT} {\bf z}_{{\rm ART}} = v_{\rm G} {\bf d}_{{\rm OC}}$ and $i_{\rm AT} = F^T_{\rm T} (Z_{\rm G}+Z_{\rm T})^{-1} v_{\rm G}$  we obtain
%\footnote{If the receive antennas are not CMS antennas, the (isolated) radiation pattern ${g} (\theta_{\rm inc},\phi_{\rm inc})$ must be replaced with a vector of \textit{embedded} radiation patterns ${\bf g} (\theta_{\rm inc},\phi_{\rm inc})$~\cite[Eq. (10)]{Friedlander2018}. This yields ${\bf g} (\theta_{\rm inc},\phi_{\rm inc}) \odot {\bf a}(\theta_{\rm inc},\phi_{\rm inc})$~\cite[Eq. (9)]{Friedlander2018}.} ${\bf d}_{{\rm OC}} =  \alpha(\theta_{\rm inc},\phi_{\rm inc}) {\bf a}(\theta_{\rm inc},\phi_{\rm inc})$
% \begin{equation}
% \label{doc}
% {\bf d}_{{\rm OC}} =  \alpha(\theta_{\rm inc},\phi_{\rm inc}) {\bf a}(\theta_{\rm inc},\phi_{\rm inc})
% \end{equation}
with $\alpha(\theta_{\rm inc},\phi_{\rm inc}) = F^T_{\rm T} (Z_{\rm G}+Z_{\rm T})^{-1} Z_0\alpha'(\theta_{\rm inc},\phi_{\rm inc})$. In general, the open-circuit voltage ${\bf v}_{{\rm OC}}$ is produced by the superposition of  a (possibly large) number $L$ of incoming waves. Assuming that each wave can be modeled as a plane wave with a particular azimuth angle $\phi_i \in [-\pi/2, \pi/2)$ and elevation angle $\theta_i\in [-\pi/2, \pi/2)$, we obtain \eqref{multipath}.
