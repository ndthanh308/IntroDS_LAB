%!TEX root = jsac_v7.tex

\vspace{-0.4cm}
\section*{Appendix B \\ Connection with the scattering representation}
% Instead of working with voltages and currents, incident and reflected \textit{power waves} can be used to describe multiport systems (e.g.,\cite{Wallace2004}). Particularly, at port $n$ of an arbitrary multiport network, we can define the \textit{scattering parameters} $a_{n}$ and $b_{n}$, which represent the complex envelopes of the inward-propagating (incident) and outward-propagating (reflected) power waves,  respectively. They are related to the voltage and current, $v_{n}$ and $i_{n}$, measured at the same port, as
Instead of dealing with voltages and currents, incident and reflected \textit{power waves} can describe multiport systems (e.g., \cite{Wallace2004}). At port $n$ of a multiport network, we define the \textit{scattering parameters} $a_{n}$ and $b_{n}$, representing the complex envelopes of the inward-propagating (incident) and outward-propagating (reflected) power waves, respectively. They relate to the voltage and current, $v_{n}$ and $i_{n}$, measured at the same port, as~\cite{Kurokawa1965},\cite[Ch. 4]{Pozar}:
\begin{equation}
\label{An}
a_{n}=\dfrac{v_{n}+Z_{n}i_{n}}{2 \sqrt{\re(Z_{n})}} \quad \quad
b_{n}=\dfrac{v_{n}-Z_{n}^{\ast}i_{n}}{2 \sqrt{\re(Z_{n})}}
\end{equation}
where $Z_{n}$ is a chosen reference impedance used for computing the scattering parameters. The physical meaning of $a_{n}$ and $b_{n}$ can be appreciated by computing $|a_{n}|^{2}-|b_{n}|^{2}=\re(v_{n}i^{\ast}_{n})$. %$|a_{n}|^{2}-|b_{n}|^{2}$. Taking~\eqref{An} and~\eqref{Bn} into account, yields
% \begin{equation}
% \label{AnBn}
% |a_{n}|^{2}-|b_{n}|^{2}=\re(v_{n}i^{\ast}_{n})
% \end{equation}
which represents the total power flowing into port $n$. This is valid for any reference impedance $Z_{n}$. Hence, the total power flowing into a multiport system is $\re({\bf v}^{\Htran}{\bf i})=\|{\bf a}\|^{2}-\|{\bf b}\|^{2}$
% \begin{equation}
% \label{ }
% \re({\bf v}^{\Htran}{\bf i})=\|{\bf a}\|^{2}-\|{\bf b}\|^{2}
% \end{equation}
where ${\bf v}$ and ${\bf i}$ are the vectors of the voltages and currents at the ports of the network, while $\bf a$ and $\bf b$ are vectors collecting the scattering parameters $a_{n}$ and $b_{n}$, respectively. The amplitudes of the incident and reflected waves are such that ${\bf b} = {\bf S} {\bf a}$ where $\bf S$ is the \textit{scattering} matrix.
%\begin{equation}
%\label{matScattering}
%{\bf b} = {\bf S} {\bf a}.
%\end{equation}
The latter can be obtained from the impedance matrix $\bf Z$ as, e.g.,~\cite[Ch.~4, Eq. (4.68)]{Pozar} 
\begin{equation}
\label{SvsZ}
{\bf S}={\bf F} ({\bf Z-G^{\ast}}) ({\bf Z+G})^{-1}{\bf F}^{-1}
\end{equation}
where $\bf F$ and $\bf G$ are diagonal matrices with the $n$th diagonal elements $1/2\sqrt{\re(Z_{n})}$ and $Z_{n}$, respectively. By substituting each impedance matrix with its corresponding scattering matrix, based on~\eqref{SvsZ}, we describe the system in terms of scattering parameters instead of voltages and currents. Both descriptions are equivalent. For CMS antennas, the impedance description is preferred because it can be obtained directly from the isolated radiation pattern.
% where $\bf F$ and $\bf G$ are diagonal matrices with the $n$th diagonal elements given by $1/2\sqrt{\re(Z_{n})}$ and $Z_{n}$, respectively. Based on~\eqref{SvsZ}, each impedance matrix can be replaced with the corresponding scattering matrix. In doing so, we obtain a description of the communication system in terms of scattering parameters instead of voltages and currents. 
% The representation of a communication system through the scattering or impedance matrices is only a matter of convenience. %Both descriptions are almost equivalent (except for constructed special cases). 
% The authors in \cite{Wallace2004} found the use of S-matrices more convenient for capacity computation.