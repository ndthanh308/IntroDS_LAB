 %!TEX root = jsac_v7.tex
% Figure environment removed


\section{Review of Multiport Communication Theory}  \label{sec:system_model}

Consider a narrowband communication system equipped with $M$ antennas at the receiver and $N$ antennas at the source. This is  described by the following discrete-time input-output relation~\cite{TseBook}:
\begin{equation}\label{eq:MIMO_channel}
{\bf y} = {\bf H}{\bf x} + {\bf n}
\end{equation} 
where ${\bf y}\in \mathbb{C}^{M}$ and ${\bf x}\in \mathbb{C}^{N}$ denote the output and input vectors, respectively. The vector ${\bf x}$ must satisfy $\mathbb{E}\{{{\bf x}^{\Htran}{\bf x}}\}\le P_{\rm T}$ to constrain the total transmit power. Also, ${\bf n}\sim \mathcal{N}_\mathbb{C}({\bf 0}, {\bf R}_n)$ is the additive Gaussian noise and ${\bf H}\in \mathbb{C}^{M \times N}$ is the MIMO channel matrix. {The input-output relation in \eqref{eq:MIMO_channel} can be used to model a great variety of multiple antenna communication systems. In order to successfully model a particular one, there is the need to encode the physical context of the system into it. This is exactly the point where the circuit theoretic concept of linear multiports from~\cite{Nossek2014} comes into play.}
% Particularly, it acts as a medium between the mathematical world of signal and information theories and the physical world of electromagnetic, which governs the particular communication system.
%
The physical model, based on the circuit theoretic approach, is shown in Fig.~\ref{fig: Network Model}. It consists of four basic parts: \emph{signal generation}, \emph{impedance matching}, \emph{antenna mutual coupling}, and \emph{noise}. The meaning of each part is briefly reviewed next. More details can be found in~\cite{Nossek2014}. 

\subsection{Signal generation and power} 
The generation of the $n$th physical signal that is to be transmitted is modeled by a voltage source, with complex envelope $v_{{\rm G},n}$, in series with the impedance \textcolor{blue}{$Z_{{\rm G}}=R_{{\rm G}}+\imagunit X_{{\rm G}}$}. The average available power of the voltage generator is $P_{{\rm a},n}=\frac{\Exp\{ |{v}_{{\rm G},n}|^{2}\}}{4R_{{\rm G}}}$
% \begin{equation}\label{available_power}
% P_{A,n}=\frac{\Exp\{ |{v}_{{\rm G},n}|^{2}\}}{4R_{{\rm G}}}
% \end{equation}
where the expectation accounts for signal randomness. Letting ${\bf v}_{{\rm G}} = [{v}_{{\rm G},1}, {v}_{{\rm G},2}, \ldots,{v}_{{\rm G},N}]^{T}$, the total average available power is thus
\begin{equation}\label{available_power}
P_{{\rm a}}=\sum\limits_{n=1}^NP_{{\rm a},n} = \frac{\Exp\{ {\bf v}^{\Htran}_{{\rm G}}{\bf v}_{{\rm G}}\}}{4R_{{\rm G}}}.
\end{equation}
% with ${\bf v}_{{\rm G}} = [{v}_{{\rm G},1}, {v}_{{\rm G},2}, \ldots,{v}_{{\rm G},N}]^{T}$.

\subsection{Impedance matrices} The transmit/receive matching networks are multiport systems described by the impedance matrices {\ZMT} and {\ZMR}. In particular, ${\bf Z}_{\rm MT}\in \mathbb{C}^{2N \times 2N}$ and ${\bf Z}_{\rm MR} \in \mathbb{C}^{2M \times 2M}$ are given by
\begin{equation}
\label{eq:ZMT}
\begin{split}
{\bf Z}_{\rm MT} = \left[\begin{array}{cc} {\bf Z}_{{\rm MT},11} & {\bf Z}_{{\rm MT},12} \\ {\bf Z}_{{\rm MT},21} & {\bf Z}_{{\rm MT},22} \end{array}\right] \\  {\bf Z}_{\rm MR} = \left[\begin{array}{cc} {\bf Z}_{{\rm MR},11} & {\bf Z}_{{\rm MR},12} \\ {\bf Z}_{{\rm MR},21} & {\bf Z}_{{\rm MR},22} \end{array}\right]
\end{split}
\end{equation}
with $\{{\bf Z}_{{\rm MT},ij}\in \mathbb{C}^{N \times N}; i =1,2,  j =1,2\}$ and $\{{\bf Z}_{{\rm MR},ij}\in \mathbb{C}^{M \times M}; i =1,2,  j =1,2\}$. We assume that the impedance matching networks are \textit{lossless}, {reciprocal}~\cite{Nossek2010}, and noiseless~\cite{Nyquist1928}. 
%This implies that {\ZMT} and {\ZMR} must be symmetric matrices with vanishing real parts. Accordingly, $Z_{{\rm MT},12} = Z_{{\rm MT},21}$, and ${\bf Z}_{{\rm MR},11} ={\bf Z}^{T}_{{\rm MR},11}$, ${\bf Z}_{{\rm MR},22} ={\bf Z}^{T}_{{\rm MR},22}$, ${\bf Z}_{{\rm MR},12} ={\bf Z}^{T}_{{\rm MR},21}$. 

% \subsection{Mutual coupling between antennas}
The impedance matrix ${\bf Z}_{\rm A} \in \mathbb{C}^{(N+M) \times (N+M)}$ accounts for the mutual coupling between antennas and can be partitioned as:
\begin{equation}
\label{eq:ZA}
{\bf Z}_{\rm A}=\left[\begin{array}{cc} {\bf Z}_{{\rm AT}} & {\bf Z}_{{\rm ATR}} \\ {\bf Z}_{{\rm ART}} & {\bf Z}_{{\rm AR}} \end{array}\right].
\end{equation}
Particularly, ${\bf Z}_{{\rm AT}} \in \mathbb C^{N \times N}$ and ${\bf Z}_{{\rm AR}} \in \mathbb{C}^{M \times M}$ quantify the mutual coupling at the transmit and receive sides (\textit{intra-array coupling}), respectively, while ${\bf Z}_{{\rm ATR}} \in  \mathbb{C}^{N \times M}$ and ${\bf Z}_{{\rm ART}} \in  \mathbb{C}^{M \times N}$ model the mutual coupling between the transmit  and the receive arrays (\textit{inter-array coupling}).  Because antennas are reciprocal, e.g.~\cite{balanis}, we have  ${\bf Z}_{{\rm AT}}={\bf Z}^{T}_{{\rm AT}}$, ${\bf Z}_{{\rm AR}}={\bf Z}^{T}_{{\rm AR}}$, and ${\bf Z}_{{\rm ATR}}={\bf Z}^{T}_{{\rm ART}}$.

A common approximation for ${\bf Z}_{\rm A}$ follows from the unilateral assumption, according to which ${\bf Z}_{{\rm ATR}} \approx {\bf 0}_{N \times M}$.
% \begin{equation}
% \label{eq:UnilAppx}
% {\bf Z}_{\rm A} \approx \left[\begin{array}{cc} {\bf Z}_{{\rm AT}} & {\bf 0}_{N \times M} \\ {\bf Z}_{{\rm ART}} & {\bf Z}_{{\rm AR}} \end{array}\right].
% \end{equation}
This basically implies that the currents at the receiver do not produce effects on the transmitter. From a mathematical standpoint, it requires that $||{\bf Z}_{{\rm AT}} {\bf i}_{\rm AT}|| \gg ||{\bf Z}_{{\rm ATR}} {\bf i}_{\rm AR}||$
%\begin{equation}
%\label{}
%|{Z}_{{\rm AT}} {i}_{A}| \gg |{\bf z}_{{\rm ATR}} {\bf i}_{B}|
%\end{equation}
where ${\bf i}_{\rm AT}$ and ${\bf i}_{\rm AR}$ are the vectors of currents at the transmitting and receiving arrays, respectively (see Fig.~\ref{fig: Network Model}). In practice, it implies that the transmitting antennas are not affected by the presence of the receiving antennas. This is true as long as the transmit and receive arrays are sufficiently separated in space as it happens in any practical communication network.\footnote{It may not hold true if different short-range applications are considered, e.g., near-field communications or short-range simultaneous wireless information and power transfer systems.}

\subsection{Losses in the antennas}
Notice that even though antennas may possess minimal loss, this can become significant when substantial electric currents are required to transmit a specific power. Particularly when dealing with close antenna spacing, the internal losses within the antenna can significantly impact the performance. A common way to account for this, it is to include a dissipation resistance that is connected in series. This implies that the impedance matrices ${ \bf Z}_{\rm AT}$ and ${\bf Z}_{\rm AR}$ must be replaced with
\begin{align}
{ \bf Z}_{\rm AT} &\to { \bf Z}_{\rm AT}+ R_{\rm d} {\bf I}_N \quad \quad
{ \bf Z}_{\rm AR} \to { \bf Z}_{\rm AR}+ R_{\rm d} {\bf I}_M.
\end{align}
If different dissipation resistances are used at the different antennas, then the matrices $R_{\rm d}{\bf I}_N$ and $R_{\rm d} {\bf I}_M$ should be replaced with diagonal matrices. For a half-wavelength dipole the expression of the dissipation resistance can be found in~\cite[Example 2.13]{balanis}.



\subsection{Noise sources}
The vector ${\bf v}_{{\rm EN}}$ accounts for the \textit{extrinsic noise} originating from the background radiation, and its entries represent the complex envelopes of the voltages that appear at the antenna ports when no currents flow, i.e., open-circuit noise voltages~\cite{Nossek2010}. The elements of ${\bf v}_{{\rm EN}}$ are zero-mean correlated random variables, with ${\bf R}_{\rm EN}=\Exp{\{{\bf v}_{{\rm EN}} {\bf v}^{\Htran}_{{\rm EN}}\}}$. A commonly adopted model is \cite[Sec. II-E]{Nossek2010}
\begin{equation}
\label{RA}
{\bf R}_{\rm EN}=4 k_{B} T_{\rm A} \Delta f \re({\bf Z}_{{\rm AR}})
\end{equation} 
where $k_{B}$ is the Boltzmann constant, $T_{\rm A}$ is the noise temperature of the antennas, while $\Delta f$ is the equivalent noise bandwidth that depends on the bandwidth of the desired signal. 


The \textit{intrinsic noise} is produced by the subsystems that follow the receive matching network such as \textcolor{blue}{low noise amplifiers (LNAs), mixers, and analog-to-digital converters (ADCs)}. Most of the noise originates from the LNAs, and thus can be modelled by using the voltage and current vectors~\cite{Nossek2010,Rothe1956} given by ${\bf v}_{{\rm LNA}}$ and ${\bf i}_{{\rm LNA}}$, respectively. Both ${\bf v}_{{\rm LNA}}$ and ${\bf i}_{{\rm LNA}}$ are zero-mean random vectors, with the following statistics~\cite[Eq. (10)]{Nossek2010}: $\Exp{\{{\bf i}_{{\rm LNA}} {\bf i}^{\Htran}_{{\rm LNA}}\}}=\sigma^{2}_{i}{\bf I}_{M}$, $\Exp{\{{\bf v}_{{\rm LNA}} {\bf v}^{\Htran}_{{\rm LNA}}\}}=R^{2}_{\rm N} \sigma^{2}_{i} {\bf I}_{M}$ and %\Exp{\{{\bf v}_{{\rm LNA}} {\bf i}^{\Htran}_{{\rm LNA}}\}}=\rho R_{\rm N} \sigma^{2}_{i} {\bf I}_{M}$
\begin{equation}\label{CrossCorLna}
% \Exp{\{{\bf i}_{{\rm LNA}} {\bf i}^{\Htran}_{{\rm LNA}}\}}=\sigma^{2}_{i}{\bf I}_{M} \quad
% \Exp{\{{\bf v}_{{\rm LNA}} {\bf v}^{\Htran}_{{\rm LNA}}\}}=R^{2}_{\rm N} \sigma^{2}_{i} {\bf I}_{M} \quad
\Exp{\{{\bf v}_{{\rm LNA}} {\bf i}^{\Htran}_{{\rm LNA}}\}}=\rho R_{\rm N} \sigma^{2}_{i} {\bf I}_{M} \quad
\end{equation}
where $R_{\rm N}$ is the so-called \textit{noise resistance} of the LNAs, usually indicated in the manufacturer data sheets. The complex parameter 
\begin{equation} \label{ro}
\rho=\dfrac{\Exp\{{v}_{{\rm LNA},m}{i}^{\ast}_{{\rm LNA},m}\}}{\sqrt{\Exp\{|{v}_{{\rm LNA},m}|^{2}\}\Exp\{|{i}_{{\rm LNA},m}|^{2}\}}}
\end{equation}
accounts for the correlation between voltage and current noise generators at each port.
% \begin{equation}
% \label{ro}
% \rho=\dfrac{\Exp\{{v}_{{\rm LNA},m}{i}^{\ast}_{{\rm LNA},m}\}}{\sqrt{\Exp\{|{v}_{{\rm LNA},m}|^{2}\}\Exp\{|{i}_{{\rm LNA},m}|^{2}\}}} \quad\quad m=1,2,\ldots,M.
% \end{equation}


%% Figure environment removed


\subsection{Input-Output Relation}
Under the unilateral approximation, the input-output relation is~\cite[Eq. (16)]{Nossek2010}
% \footnote{Compared to~\cite[Eq. (16)]{Nossek2010}, the normalization factor $\sqrt{R_{\rm L}}$ has been removed.}
\begin{equation}
\label{eq:ioTotal1}
{\bf v}_{\rm L}= {\bf D} {\bf v}_{\rm G} +{\boldsymbol \eta}
\end{equation}
where ${\bf D}$ and ${\boldsymbol \eta}$ are given by~\cite[Eq. (17)]{Nossek2010}
\begin{align}
\label{matD}
{\bf D}&={\bf Q}\,{\bf Z}_{{\rm RT}}(Z_{\rm G}{\bf I}_N+{\bf Z}_{{\rm T}})^{-1}\\
\label{veceta}
{\boldsymbol \eta}&= {\bf Q}({\bf F}_{\rm R}{\bf v}_{\rm EN}-{\bf v}_{\rm LNA}+{\bf Z}_{{\rm R}}{\bf i}_{\rm LNA})
\end{align}
with~\cite[Eq. (19)]{Nossek2010}
\begin{align}
\label{ZR}
{\bf Z}_{{\rm R}}&={\bf Z}_{{\rm MR},11}-{\bf F}_{{\rm R}}{\bf Z}_{{\rm MR},21}\\\label{ZT}
{\bf Z}_{{\rm T}}&={\bf Z}_{{\rm MT},11}-{\bf F}_{{\rm T}} {\bf Z}_{{\rm MT},21}\\
\label{ZRT}
{\bf Z}_{{\rm RT}}&= {\bf F}_{{\rm R}}  {\bf Z}_{{\rm ART}}  {\bf F}_{{\rm T}} ^{^{\Ttran}} \end{align}
and~\cite[Eq. (20)]{Nossek2010}
\begin{align}\label{matF_R}
{\bf F}_{{\rm R}} &= {\bf Z}_{{\rm MR},12}\left({\bf Z}_{{\rm MR},22} + {\bf Z}_{{\rm AR}}\right)^{-1}\\\label{matF_T}
{\bf F}_{{\rm T}} &={\bf Z}_{{\rm MT},12} \left({\bf Z}_{{\rm MT},22} + {\bf Z}_{{\rm AT}}\right)^{-1} \\\label{matQ}
{\bf Q}&=Z_{\rm L}(Z_{\rm L}{\bf I}_{M}+{\bf Z}_{{\rm R}})^{-1}.
\end{align}
The input-output relation~\eqref{eq:ioTotal1} can be written in a slightly different form (which will turn useful later on) as ${\bf v}_{\rm L} = {\bf Q}({\bf F}_{{\rm R}} {\bf v}_{\rm OC} +\tilde{\boldsymbol \eta})$
% \begin{equation}
% \label{eq:ioTotal2}
% {\bf v}_{L} = {\bf Q}({\bf F}_{{\rm R}} {\bf v}_{\rm OC} +\tilde{\boldsymbol \eta})
% \end{equation}
where $\tilde{\boldsymbol \eta}={\bf F}_{\rm R}{\bf v}_{\rm EN}-{\bf v}_{\rm LNA}+{\bf Z}_{{\rm R}}{\bf i}_{\rm LNA}$
% \begin{equation}
% \label{tildeEta}
% \tilde{\boldsymbol \eta}={\bf F}_{\rm R}{\bf v}_{\rm EN}-{\bf v}_{\rm LNA}+{\bf Z}_{{\rm R}}{\bf i}_{\rm LNA}
% \end{equation}
%\begin{equation}
%\label{doc}
%{\bf d}^\prime={F}_{{\rm T}} (Z_{G}+{Z}_{{\rm T}})^{-1} {\bf z}_{{\rm ART}}
%\end{equation} 
and
\begin{equation}
\label{voc}
%{\bf v}_{\rm oc}={v}_{G}{\bf d}^\prime={F}_{{\rm T}} (Z_{G}+{Z}_{{\rm T}})^{-1} {v}_{G}{\bf z}_{{\rm ART}}=i_{A} {\bf z}_{{\rm ART}}
{\bf v}_{\rm OC}= {\bf D}_{\rm OC} {\bf v}_{\rm G}\mathop{=}^{(a)} {\bf Z}_{{\rm ART}} {\bf i}_{{\rm AT}}
\end{equation}
with \begin{equation}\label{doc}
{\bf D}_{\rm OC} =  {\bf Z}_{{\rm ART}}  {\bf F}_{{\rm T}}^{\Ttran} (Z_{\rm G}{\bf I}_N+{\bf Z}_{{\rm T}})^{-1}.
\end{equation}
%is the voltage vector at the terminals of the receiving antennas induced by ${i}_{A}$ and assuming ${\bf i}_{B}=0$. The subscript in ${\bf v}_{\rm OC}$ stands for 
Notice that ${\bf v}_{\rm OC}$
is the \textit{open circuit} voltage vector as induced by ${\bf i}_{{\rm AT}}$ when ${\bf i}_{{\rm AR}}={\bf 0}$, as it follows from $(a)$ in~\eqref{voc}. In general, ${\bf i}_{{\rm AR}}={\bf 0}$ does not imply that the elements of ${\bf v}_{\rm OC}$ are the same as if the receive antennas were isolated. \textcolor{blue}{This holds true only if the receive antennas are \textit{canonical minimum scattering} (CMS) antennas.}\footnote{According to~\cite{Kahn1965}, \textit{a canonical minimum-scattering antenna is “invisible” when the accessible waveguide terminals are open-circuited.} This means that a vanishing electric current in the antenna does not alter the electromagnetic field.} This is the case of half-wavelength dipoles~\cite{Laas2020}. We finally notice that ${\bf D} {\bf v}_{\rm G} = {\bf Q}{\bf F}_{{\rm R}} {\bf v}_{\rm OC}$ so that, using~\eqref{voc}, we get
 \begin{equation}\label{eq:D_OC_relation}
{\bf D} =  {\bf Q}{\bf F}_{{\rm R}}{\bf D}_{\rm OC}.
\end{equation}

\begin{remark}(Input-output relation without matching networks)
In the absence of a transmit matching network, the input-output relation can simply be obtained by setting ${\bf Z}_{\rm T}={\bf Z}_{\rm AT}$ and ${\bf F}_{\rm T}={\bf I}_{N}$ in~\eqref{matD} and~\eqref{ZRT}, respectively. Analogously, with no receive matching network the input-output relation can be obtained by replacing ${\bf Z}_{\rm R}$ with ${\bf Z}_{\rm AR}$ and ${\bf F}_{\rm R}$ with ${\bf I}_{M}$.
\end{remark}



\subsection{Transmit power and noise covariance matrix} The \textit{transmit power} is defined as the average active power at the output of the transmit matching network, or equivalently, at the input of the transmit antenna array, i.e., \textcolor{blue}{$P_{\rm T}= \frac{1}{2}\Exp\{\re({\bf v}^{\Htran}_{\rm AT}{\bf i}_{\rm AT}) \}$}.
% \begin{equation}
% \label{eq:P_T}
% P_{\rm T}=\Exp\{\re({\bf v}^{\Htran}_{\rm T}{\bf i}_{\rm T}) \}.
% \end{equation} 
% Under the unilateral approximation, we have that ${\bf v}_{\rm AT} = {\bf Z}_{\rm AT}{\bf i}_{\rm AT}$. 
Assuming a lossless transmit matching network, we have $\Exp\{\re({\bf v}^{\Htran}_{\rm AT}{\bf i}_{\rm AT}) \} = \Exp\{\re({\bf v}^{\Htran}_{\rm T}{\bf i}_{\rm T}) \}$. Now, observe that, under the unilateral approximation, ${\bf v}_{\rm T} = {\bf Z}_{\rm T} {\bf i}_{\rm T}$. Since ${\bf v}_{\rm T} = {\bf v}_{\rm G} - Z_{\rm G} {\bf i}_{\rm T}$, we obtain ${\bf i}_{\rm T} = \left(Z_{\rm G} {\bf I}_{N} + {\bf Z}_{\rm T} \right)^{-1}{\bf v}_{\rm G}$
% \begin{equation}
% {\bf i}_{\rm T} = \left(Z_{\rm G} {\bf I}_{N} + {\bf Z}_{\rm T} \right)^{-1}{\bf v}_{\rm G}
% \end{equation}
so that $P_{\rm T}$ reduces to
\textcolor{blue}{\begin{equation}
\label{eq:P_T_new}
P_{\rm T}=\frac{1}{2R_{\rm G}}\Exp\{\re({\bf v}_{\rm G}^{\Htran}{\bf B}{\bf v}_{\rm G}) \}
\end{equation}}
with 
\textcolor{blue}{\begin{equation}
\label{matB2}
{\bf B}=R_{\rm G} (Z_{\rm G}{\bf I}_{N}+{\bf Z}_{{\rm T}})^{-\Htran} \re\{{\bf Z}_{{\rm T}}\}  (Z_{\rm G}{\bf I}_{N}+{\bf Z}_{{\rm T}})^{-1}.
\end{equation}}
Notice that $P_{\rm T}$ coincides with the \textit{radiated power} $P_{\rm rad}$ only if the transmit antennas are lossless.

From the statistics of the extrinsic and intrinsic noise, the covariance matrix ${\bf R}_{\eta}$ of ${\boldsymbol \eta}$ in~\eqref{veceta} is:
\begin{equation}
\label{Reta}
{\bf R}_{\eta}={\bf Q}{\bf U}{\bf Q}^{\Htran}
\end{equation}
where ${\bf Q}$ is given in~\eqref{matQ} and ${\bf U}= {\bf U}_{\rm IN} + {\bf U}_{\rm EN}$ is the correlation matrix of $\tilde{\boldsymbol \eta}$ with
\begin{align}\label{Upsilon2_IN}
{\bf U}_{\rm IN}&= \sigma^{2}_{\rm i}\left({\bf Z}_{{\rm R}}{\bf Z}^{\ast}_{{\rm R}}  - 2 R_{\rm N} \re \left( \rho^{\ast} {\bf Z}_{{\rm R}} \right) + R_{\rm N}^2{\bf I}_M\right)
\end{align}
and ${\bf U}_{\rm EN}=  {\bf F}_{\rm R} {\bf R}_{\rm EN} {\bf F}^{\Htran}_{\rm R}$.
% \begin{align}
% \label{Upsilon2_EN}
% {\bf U}_{\rm EN}&=  {\bf F}_{\rm R} {\bf R}_{\rm EN} {\bf F}^{\Htran}_{\rm R}.
% \end{align}

\subsection{Matching network optimization}
The transmit matching network ${\bf Z}_{{\rm MT}}$ can be designed to maximize the power delivered to antennas (\textit{power matching} or \emph{maximum power transfer})~\cite{Nossek2010}. This yields ${\bf B}={\bf I}_N$ in~\eqref{matB2}, and 
\begin{align}\label{Z_T_opt}
{\bf Z}_{\rm T} = Z^{\ast}_{\rm G}{\bf I}_N. 
\end{align}
By taking~\eqref{ZT} and~\eqref{matF_T} into account, this can be obtained by setting~\cite{Nossek2010}
\begin{align}\label{Z_MT}
\!\!\!{\bf Z}_{{\rm MT}}^\star= \begin{bmatrix}
-\mathrm{j}{X_{\rm G}}{\bf I}_{N} & -\mathrm{j} \sqrt{R_{\rm G}}\re\{{\bf Z}_{{\rm AT}}\}^{1/2}  \\
-\mathrm{j} \sqrt{R_{\rm G}}\re\{{\bf Z}_{{\rm AT}}\}^{1/2} & -\mathrm{j}{\imag\{{\bf Z}_{\rm AT}\}}  \\
\end{bmatrix}
\end{align}
which yields
\begin{align}\label{F_T_matched}
{\bf F}_{{\rm T}}= -\mathrm{j} \sqrt{R_{\rm G}}\re\{{\bf Z}_{{\rm AT}}\}^{-1/2}.
\end{align}
The receive matching network ${\bf Z}_{{\rm MR}}$  can be  designed to  ensure that the signal-to-noise ratio (SNR) is as large as it can be (\textit{noise matching} or \emph{SNR maximization})~\cite{Nossek2010}. This is achieved with
\begin{align}\label{Z_MR}
{\bf Z}_{{\rm MR}}^\star&= \begin{bmatrix}
\mathrm{j}{\imag\{Z_{\rm opt}\}}{\bf I}_M & \mathrm{j} \sqrt{\re\{Z_{\rm opt}\}}\re\{{\bf Z}_{{\rm AR}}\}^{1/2}  \\
\mathrm{j} \sqrt{\re\{Z_{\rm opt}\}}\re\{{\bf Z}_{{\rm AR}}\}^{1/2} & -\mathrm{j}{\imag\{{\bf Z}_{\rm AR}\}}  \\
\end{bmatrix}.\end{align}
Plugging~\eqref{Z_MR} into~\eqref{ZR} and~\eqref{matF_R} yields
\begin{align}
\label{ZR_matched}
{\bf Z}_{{\rm R}} = Z_{\rm opt} {\bf I}_M\end{align}
with $Z_{\rm opt} = R_{\rm N} \left(\sqrt{1 - (\imag\{\rho\})^2} + \mathrm{j}\imag\{\rho\}\right)$ and 
\begin{align}\label{FR_matched}
{\bf F}_{{\rm R}} = \mathrm{j} \sqrt{\re\{Z_{\rm opt}\}}\re\{{\bf Z}_{{\rm AR}}\}^{-1/2}.
\end{align}
Also, notice that 
\begin{align}\label{Q_matched}
{\bf Q} = \frac{Z_{\rm L}}{Z_{\rm L}+Z_{\rm opt}} {\bf I}_M 
\end{align}
and the covariance matrix ${\bf R}_{\eta} = |Z_{\rm L}|^2 |Z_{\rm L}+Z_{\rm opt}|^{-2}\sigma^2 {\bf I}_M$
% \begin{align}
% {\bf R}_{\eta} = \dfrac{|Z_{L}|^2}{|Z_{L}+Z_{\rm opt}|^2} \sigma^2 {\bf I}_M
% \end{align}
becomes diagonal with 
\begin{align}\label{sigma_2}
\begin{split}
\sigma^2&=\sigma^{2}_{i}\left(|Z_{\rm opt}|^2 - 2 R_{\rm N} \re \left( \rho^{\ast} Z_{\rm opt}\right)+R_{\rm N}^2\right) \\ &+ 4 k_{B} T_{A} \Delta f \re\{Z_{\rm opt}\}).
\end{split}
\end{align}
% \subsection{Self-impedance matching networks}
The design of coupled matching networks is very challenging for arrays with a large number of antennas \cite{Nossek2010,Nossek2014}. A practical approach is to make use of a \emph{self-impedance matching network}~\cite[Sect. III.B]{Warnick2009}, instead of a full multiport matching network. This approach neglects the mutual coupling among antennas and replaces, in the design of the matching networks, the impedance matrices ${\bf Z}_{{\rm AT}}$ and ${\bf Z}_{{\rm AR}}$ with the diagonal matrices $\diag({\bf Z}_{{\rm AT}})$ and $\diag({\bf Z}_{{\rm AR}})$ that contain only their diagonal elements. It becomes possible to substitute these matrices for the actual ones in~\eqref{Z_MT} and~\eqref{Z_MR} and specify uncoupled matching networks, as described above. 
% However, notice that when evaluating the performance, the non-diagonal form of the matrices should be used~\eqref{matF_R} and~\eqref{matF_T} \textcolor{red}{??????}.

\begin{table*}[t!]
\renewcommand{\arraystretch}{1.2}
\centering
\caption{Parameters of the antenna array at the BS.}
\begin{tabular}{c|c|c|c}
{ \bf Parameter} & {\bf Value}&{ \bf Parameter} & {\bf Value}  \\
\hline
Carrier frequency & $3.5$ GHz &Variance of the current noise source & $\sigma^2_i = 2 k_B B_W T_{\rm A}/R_{\mathrm N}$\\
\hline
Bandwidth  & $B_W = 20$ MHz & Radiation resistance   & $R_{\mathrm r} = 73$ \si{\ohm} \\
\hline
Transmit power & $P_T=-30$ dBW &LNA noise resistance & $R_{\mathrm N} = 5\,\si{\ohm}$  \\
\hline
Amplifier and load impedance  & $Z_{\rm G}= Z_{\rm L}= (186 - {\mathrm j}31.6)$ \si{\ohm} &Complex correlation coefficient& $\rho =0.1$ \\
\hline
Noise temperature of antennas  & $T_{\rm A}= 290$ &Dissipation resistance   & $R_{\mathrm d} = 10^{-3}R_{\mathrm r}$ \si{\ohm} \\
\hline
\end{tabular}
\label{tab:array_parameters}
\end{table*}

\section{Computation of the Mutual Coupling Impedance Matrix}
Next, we show how to compute the mutual coupling impedance matrix ${\bf Z}_{\rm A}$. In particular, we assume that the antennas at both sides are half-wavelength dipoles of length $l_{d}=\lambda/2$ and radius $a_{d} \ll l_{d}$. \textcolor{blue}{Specifically, we set $a_{d} = 10^{-4}l_{d}$.} Moreover, we assume that the receiver is equipped with a uniform linear array.
% The dipoles can be viewed as \textit{thin cylindrical antennas}, which implies that the current densities are directed along the dipoles' axes and confined to zero transverse dimension. 
% Also, we assume that the receiver is equipped with a uniform linear array, arranged as shown in Fig.~\ref{fig:ULA}. 

\vspace{-0.3cm}
\subsection{Impedance matrix ${\bf Z}_{{\rm AR}}$} 
%The impedance matrix ${\bf Z}_{\rm AT}$ (which obviously depends on the type of antennas and on the geometric configuration of the array) can be derived on the basis of electromagnetic and circuit theory models. Here, we assume that the transmit array consists of arbitrarily oriented dipoles of length $l_{d}$ and radius $a_{d}$, with $a_{d} \ll l_{d}$. Accordingly, the dipoles can be viewed as \textit{thin cylindrical antennas}. Essentially, this means that the current densities are directed along the dipoles' axes and confined to zero transverse dimension. 
We consider ${\bf Z}_{\rm AR}$ but the same analysis follows for ${\bf Z}_{\rm AT}$.
The mutual impedance between dipole $p$ and dipole $q$ is computed as \cite[Eq. (25.4.14)]{orfanidis}
\begin{equation}
\label{Zat,mn}
[{\bf Z}_{\rm AR}]_{pq}=-\dfrac{1}{I_{p}I_{q}}\int\limits_{-l_{d}/2}^{l_{d}/2} e_{qp}(s) I_{q}(s) ds
\end{equation}
where $e_{qp}(s)$ is the component (along the direction of dipole $q$) of the electric field produced by a current $I_{p}(s')$ flowing in dipole $p$, $I_{q}(s)$ is the current flowing in dipole $q$, and finally $I_{p}$ and $I_{q}$ are the currents at the input terminals of dipoles $p$ and $q$, respectively. The current distributions $I_{p}(s')$ and $I_{q}(s)$, for $p,q=1,2,\ldots,N$,  can be found by solving a system of Hall\'en integral equations~\cite[Sec.~25.7]{orfanidis}. 
% Once, $I_{p}(s')$ and $I_{q}(s)$ are available, $e_{qp}(s)$ can easily be determined by standard electromagnetic formulas. 
The solutions can be found by numerical methods (e.g. the method of moments discussed in~\cite[Sec.~24.8]{orfanidis}). A very good approximation of the current distributions $I_{n}(s)$ for \textit{center-fed} dipoles is represented by the \textit{sinusoidal current model}, i.e.,
\begin{equation}
\label{Insin}
I_{n}(s)=I_{n} \dfrac{\sin \left[k\left({l_{d}}/{2}-|s|\right)\right]}{ \sin(kl_{d}/2)}.
\end{equation}
where $k=2 \pi/\lambda$ is the wavenumber. Based on~\eqref{Insin}, closed form expressions for the mutual impedance can be found in~\cite[Eqs. (8.69) and (8.71a-b)]{balanis} for dipoles in \textit{side-by-side configuration}. Closed form expressions are also provided for dipoles in \textit{collinear configuration}~\cite[Eqs. (8.72a-b)]{balanis}, and in \textit{parallel-in-echelon configuration}~\cite[Eqs. (8.73a-b)]{balanis}. As for the self impedance, which coincides with the diagonal elements of ${\bf Z}_{{\rm AR}}$, this can be found in~\cite[Eqs. (8.60a-b) and (8.61a-b)]{balanis}. 
% it is given by~\eqref{Self}~\cite[Eqs. (8.60a-b) and (8.61a-b)]{balanis}, where $a_{d}$ is the dipole radius and $C=0.5772$ is Euler’s constant.


% Figure environment removed

% % Figure environment removed
Fig.~\ref{fig:Eigenvalues_ZAR} shows the normalized eigenvalues of ${\bf Z}_{{\rm AR}}$ for an array of $\lambda/2$-dipoles in side-by-side configuration, for three different values of the inter-element spacing, namely $d_H = \lambda/10, \lambda/4$ and $\lambda/2$. The array size is $L_H = 6 \lambda$, and the number of array elements (and hence of eigenvalues) is $L_H / d_H +1$. Matrix ${\bf Z}_{{\rm AR}}$ has been calculated by using the sinusoidal model \eqref{Insin}. %, with $Z_{\rm self}$ and $Z_{\rm mut}$ given by \eqref{Self} and \eqref{Mutual}, respectively. 
We see that the number of significant eigenvalues does not change appreciably when $d_H$ decreases below $\lambda/2$, and is approximately $2L_H /\lambda +1$. This means that for $d_H < \lambda/2$ mutual coupling introduces a significant correlation between the different array elements, as expected. 

Fig.~\ref{fig:Eigenvalues_U} shows the normalized eigenvalues of $\bf U$ in \eqref{Reta}, obtained without a matching network (see Remark 1) and with the parameter values reported in Table~\ref{tab:array_parameters}, \textcolor{blue}{e.g., \cite{Nossek2010} and references therein.} The behavior of these eigenvalues is quite different from that of Fig.~\ref{fig:Eigenvalues_ZAR}, because noise correlation not only depends on ${\bf Z}_{{\rm AR}}$ but also on the LNA parameters, and on the presence (and type) of matching networks, as shown in \eqref{Upsilon2_IN}. In particular, the curve corresponding to $d_H=\lambda/2$ seems to indicate that a significant correlation exists between the elements of $\tilde{\boldsymbol{\eta}}$ even with a half-wavelength spacing between the antennas.    

\begin{remark}
Other models for ${\bf Z}_{{\rm AR}}$ rely on the assumption of isotropic antennas or Hertzian dipoles~\cite{yordanov2009arrays,Ivrlac2009ICC,Nossek2010,MullerISIT2012,Nossek2014,Friedlander2020}. Isotropic antennas are inconsistent with the Maxwell equations \cite[Sec.~3.2]{kraus1988antennas}. \textcolor{blue}{Hence, models based on this assumption have no physical meaning \cite[Sec.~III]{Nossek2010}, though they are commonly used in the literature for analytical tractability.} 
%Also, they are unable to capture the effect of polarization, which is essential in modeling physical antennas~\cite{Friedlander2020}. 
In the case of Hertzian dipoles, a uniform current distribution is typically assumed, which approximates well the current distribution of an \textit{infinitesimal} linear wire $(l_{d} \le \lambda/50)$ with plates at its endpoints~\cite[Sec.~4.2]{balanis}. The sinusoidal model is a more accurate representation of the current distribution of any wire antenna~\cite[Sec.~4.3]{balanis}.\vspace{-0.3cm} 
\end{remark}


% The impact of the current model on ${\bf Z}_{\rm AR}$ is analyzed in Fig.~\ref{fig:ZAR_vs_CurrentModel} with $L_H=6 \lambda$ and $d_H = \lambda/8$. Similar behaviors are observed in the isotropic case and with a sinusoidal current model, while a uniform current leads to a different eigenvalue distribution. In particular, compared to the sinusoidal model, there is a smaller number of significant eigenvalues which corresponds to a stronger correlation between the array elements. This means that the uniform model (which results in simpler expressions of the elements of ${\bf Z}_{{\rm AR}}$) should be used for a theoretical analysis only when the dipoles' length is sufficiently smaller than the wavelength (i.e., with Hertzian dipoles).


% \textcolor{red}{BISOGNA DIRE CHE QUESTI RISULTATI VALGONO SIA PER UN ARRAY CHE RICEVE CHE PER UN ARRAY CHE TRASMETTE (articolo di Jensen e Wallace).} 


% % Figure environment removed

%% Figure environment removed


\vspace{-0.3cm}
\subsection{Impedance matrix ${\bf Z}_{{\rm ART}}$}
% The impedance matrix ${\bf Z}_{{\rm ART}}$ depends not only on the type of antennas and array configuration but also on the transmission medium.

\textcolor{blue}{The impedance matrix ${\bf Z}_{{\rm ART}}$ accounts for the mutual coupling between the transmit and receive antennas. It represents the physical wireless propagation channel and can in principle be computed starting from any (e.g., deterministic or stochastic) channel model. Operationally, ${\bf Z}_{{\rm ART}}$ determines the \textit{open-circuit} voltage array response represented by ${\bf v}_{\rm OC}={\bf Z}_{{\rm ART}} {\bf i}_{\rm AT}$, which is obviously influenced by various factors, including the type of antennas, array configuration, polarization and transmission medium. Notice that ${\bf v}_{\rm OC}={\bf Z}_{{\rm ART}} {\bf i}_{\rm AT}={\bf D}_{\rm OC} {\bf v}_{\rm G}$. Hence, it can also be obtained from ${\bf D}_{\rm OC}$. 
% In Appendix A, we consider a uniform linear array composed of (arbitrarily oriented) half-wavelength dipole antennas, located in the far-field region of a single (arbitrarily oriented) half-wavelength transmitting dipole. We assume that the transmission takes place in a LoS propagation scenario. This means that ${\bf v}_{{\rm OC}}$ is produced by a single plane wave, reaching the receiver from a particular azimuth angle $\phi \in [-\pi/2, \pi/2)$ and elevation angle $\theta\in [-\pi/2, \pi/2)$.
In Appendix A, we consider an arbitrary array of CMS antennas located in the far-field region (Assumption 1) of a single transmit antenna, so that ${\bf Z}_{{\rm ART}}$ and ${\bf D}_{\rm OC}$ reduce to the vectors ${\bf Z}_{{\rm ART}}$ and ${\bf D}_{\rm OC}$, respectively. We also assume that the transmission takes place in a LoS propagation scenario, and that the electromagnetic wave generated by the transmitting antenna and incident on an antenna of the receiving array can be approximated locally (i.e., at each receiving element) by a plane wave (Assumption 2). Let $(\theta_m,\phi_m)$ denote the direction of arrival of the plane wave incident on the $m$th receive antenna, and let $r_m$ denote the distance between its center and that of the transmit antenna. Under the above conditions, in Appendix A we show that
\begin{equation}
\label{eq:d_OC}
{\bf d}_{{\rm OC}} = \boldsymbol{\alpha}(\boldsymbol{\psi},{\bf r})\odot{\bf a}({\bf r})
\end{equation}
where $\boldsymbol{\psi}$ and ${\bf r }$ are vectors collecting the directions of arrival and the distances, $\boldsymbol{\alpha}(\boldsymbol{\psi},{\bf r})$ is the vector of the channel gains, and ${\bf a}(\bf r)$ is the array response vector. Their expressions can be found in Appendix A. Notice that \eqref{eq:d_OC} is a quite general model for LoS propagation that applies to arbitrary array configurations and accounts for: \emph{i}) the distances to the different antennas over the array; \emph{ii}) the effective antenna length; \emph{iii}) the losses from polarization mismatch.}

\textcolor{blue}{If the transmitting antenna is in the far-field of the receiving array, the well-known \textit{planar wave approximation} can be achieved \cite{Friedlander2019}. In this scenario, the direction of arrival $(\theta,\phi)$ is aligned with the direction of the line connecting the centers of the array and the transmit antenna, and $r_m$ is replaced by $r$, representing the distance between the two centers. Consequently, \eqref{eq:d_OC} reduces to (e.g., \cite{Friedlander2019})
\begin{equation}
 \label{doc}
 {\bf d}_{{\rm OC}} =  \alpha(\theta,\phi,r) {\bf a}(\theta,\phi)
\end{equation}
where $\alpha(\theta,\phi,r)$ is defined in Appendix A and ${\bf a}(\theta,\phi) = [ e^{\imagunit {\bf k}(\theta,\phi) \cdot {\boldsymbol \delta}_1},\ldots, e^{\imagunit {\bf k}(\theta,\phi)\cdot {\boldsymbol \delta}_M}]^{\Ttran}$ where ${\boldsymbol \delta}_m$ is the displacement vector from the array center to the center of the $m$th receive antenna and ${\bf k}(\theta,\phi)$ is the wave vector~\cite[eq. (17.1.4)]{orfanidis}.}

% \textcolor{blue}{for a discussion about these approximate models). Notice that the planar wave approximation of \eqref{eq:d_OC} is different from the planar wave approximation of Assumption 2, because the former is relevant to the array while the latter involves the single array element. Under the planar wave approximation, we consider a single direction of arrival $(\theta,\phi)$, coincident with the direction of the line joining the centers of the array and the transmit antenna, and replace $r_m$ with $r$ in the denominator of \eqref{voc_einc_2}, $r$ being the distance between the two centers. Accordingly, \eqref{eq:d_OC} reduces to
% \begin{equation}
%  \label{doc}
%  {\bf d}_{{\rm OC}} =  \alpha(\theta,\phi,r) {\bf a}(\theta,\phi)
% \end{equation}
% where $\alpha(\theta,\phi,r) = F_{\rm T} (Z_{\rm G}+Z_{\rm T})^{-1} Z_0\alpha'(\theta,\phi,r)$, $\alpha'(\theta,\phi) =  - \imagunit e^{\imagunit \psi_0}\dfrac{{\bf l}^{(\rm t)}_{{\rm eff}} (\theta, \phi) \cdot {\bf l}^{(\rm r)}_{{\rm eff}}(\theta,\phi)}{2 \lambda r}$, $\psi_0 = - 2 \pi r/\lambda$ being the reference phase at the center of the array. Finally, the array response vector is ${\bf a}(\theta,\phi) = [ e^{\imagunit {\bf k}(\theta,\phi) \cdot {\boldsymbol \delta}_1},\ldots, e^{\imagunit {\bf k}(\theta,\phi)\cdot {\boldsymbol \delta}_M}]^{\Ttran}$ where ${\boldsymbol \delta}_m$ is the displacement vector from the center of the array to the center of the $m$th receive antenna and ${\bf k}(\theta,\phi)$ is the wave vector~\cite[eq. (17.1.4)]{orfanidis}.}

% Notice that with a single transmitting antenna ${\bf Z}_{{\rm ART}}$ and ${\bf D}_{\rm OC}$ reduce to the vectors ${\bf z}_{{\rm ART}}$ and ${\bf d}_{\rm OC}$, respectively.}
% \begin{equation}
% \label{eq:d_OC}
% {\bf d}_{{\rm OC}} = \sum\limits_{i=1}^{L}\alpha(\theta_{i},\phi_{i}) {\bf a}(\theta_{i},\phi_{i})
% \end{equation}
% where ${\bf a}(\theta,\phi) = [ e^{\imagunit {\bf k}(\theta,\phi)^{\Ttran}{\bf u}_1},\ldots, e^{\imagunit {\bf k}(\theta,\phi)^{\Ttran}{\bf u}_M}]^{\Ttran}$, ${\bf k}(\theta,\phi) = \frac{2\pi}{\lambda} [\cos(\theta) \cos(\phi), \cos(\theta) \sin(\phi), \sin(\theta)]^{\Ttran}$ is the wave vector that describes the phase variation of the plane wave with respect to the Cartesian coordinates and ${\bf u}_m$ is the center coordinate of dipole $m$.
% \textcolor{red}{The model~\eqref{multipath} can be generalized to a continuum of plane waves as follows (e.g.,~\cite{Sayeed2002a})
% \begin{equation}
% \label{multipathCont}
% {\bf d}_{{\rm OC}}  = \! \int\!\!\int_{-\pi/2}^{\pi/2} \!\!\! \alpha(\theta,\phi) {\bf a}(\theta,\phi) {\rm d} \theta {\rm d} \phi
% \end{equation}
% where $\alpha(\theta,\phi)$ accounts for the small-scale fading and must be modeled stochastically. A common way to model $\alpha(\theta,\phi)$ is a zero-mean, complex-Gaussian random process with cross-correlation~\cite{Sayeed2002a}
% \begin{equation} \label{eq:scattering-correlation-model}
% \mathbb{E} \{ \alpha(\theta,\phi) \alpha^*(\theta',\phi') \} = \beta(\theta,\phi) f(\theta,\phi) \delta(\phi-\phi')  \delta(\theta-\theta')
% \end{equation}
% where $\beta(\theta,\phi)$ is the average channel gain, and $f(\theta,\phi)$ is the normalized \emph{spatial scattering function} \cite{Sayeed2002a}, such that $\iint f(\theta,\phi) d\theta d\phi  = 1$. Plugging~\eqref{eq:scattering-correlation-model} into~\eqref{multipathCont} yields ${\bsy \Sigma} = \mathbb{E}\{{\bf d}_{{\rm OC}}   {{\bf d}^{\Htran} _{{\rm OC}}}\}$ with
% $
% \left[{\bsy \Sigma}\right]_{n,m} = \! \int\!\!\int_{-\pi/2}^{\pi/2} \beta(\theta,\phi) f(\theta,\phi) e^{\imagunit {\bf k}(\theta,\phi)^{\Ttran}({\bf u}_n -{\bf u}_m) } {\rm d}\theta {\rm d} \phi.$
% This model is valid for any $f(\theta,\phi)$.}

% The open-circuit voltage ${\bf v}_{{\rm OC},n}= {\bf d}_{{\rm OC},n}  {v}_{G,n}$ induced by antenna $n$ is produced by the superposition of  a (possibly large) number $L$ of incoming waves that are generated by the voltage source. If the  receive array is in the far-field region of the transmitter, each wave can thus be modeled as a plane wave that reaches the receiver from a particular azimuth angle $\phi_i \in [-\pi/2, \pi/2)$ and elevation angle $\theta_i\in [-\pi/2, \pi/2)$ in front of it. In this case, we obtain 
% \begin{equation}
% \label{multipath}
% {\bf d}_{{\rm OC}} = \sum\limits_{i=1}^{L}\alpha_{i} {g} (\theta_{i},\phi_{i}) {\bf a}(\theta_{i},\phi_{i})
% \end{equation}
% where $\alpha_i \in \mathbb C$ accounts for the gain and phase-rotation of plane wave $i$, and ${g} (\theta_{i},\phi_{i}) {\bf a}(\theta_{i},\phi_{i})$ is the (open-circuit) array manifold, e.g.,~\cite{Friedlander2018}. Particularly, $ {g} (\theta_{i},\phi_{i}) = \frac{2 \cos\left(\frac{\pi}{2} \sin \theta_i \right)}{\cos \theta_i}$ is the radiation pattern of the \textit{isolated} $\lambda/2$-dipole 
% % \begin{equation}
% % {g} (\theta_{i},\phi_{i}) = \frac{2 \cos\left(\frac{\pi}{2} \sin \theta_i \right)}{\cos \theta_i}
% % \end{equation}
% and ${\bf a}(\theta_{i},\phi_{i}) = [ e^{\imagunit {\bf k}(\theta_{i},\phi_{i})^{\Ttran}{\bf u}_1},\ldots, e^{\imagunit {\bf k}(\theta_{i},\phi_{i})^{\Ttran}{\bf u}_M}]^{\Ttran}$
% % \begin{equation}
% % \label{aisotropic}
% % {\bf a}(\theta_{i},\phi_{i}) = \left[ e^{\imagunit {\bf k}(\theta_{i},\phi_{i})^{\Ttran}{\bf u}_1},\ldots, e^{\imagunit {\bf k}(\theta_{i},\phi_{i})^{\Ttran}{\bf u}_M}\right]^{\Ttran}
% % %{\bf a}_{\rm isotropic}(\theta,\phi) = \left[\begin{array}{c} e^{\imagunit \frac{2 \pi}{\lambda}[\cos \theta(x_{1} \cos \phi+y_{1} \sin \phi)+z_{1} \sin \theta}] \\ e^{\imagunit \frac{2 \pi}{\lambda}[\cos \theta(x_{2} \cos \phi+y_{2} \sin \phi)+z_{2} \sin \theta}] \\ \vdots \\ e^{\imagunit \frac{2 \pi}{\lambda}[\cos \theta(x_{M} \cos \phi+y_{M} \sin \phi)+z_{M} \sin \theta}] \end{array}\right]
% % \end{equation}
% where ${\bf k}(\theta,\phi) = \frac{2\pi}{\lambda} [\cos(\theta) \cos(\phi), \cos(\theta) \sin(\phi), \sin(\theta)]^{\Ttran}$ is the wave vector that describes the phase variation of the plane wave with respect to the Cartesian coordinates and ${\bf u}_n$ is the center coordinate of dipole $n$. 

% The model~\eqref{multipath} can be generalized to a continuum of plane waves as follows (e.g.,~\cite{Sayeed2002a})
% \begin{equation}
% \label{multipathCont}
% {\bf d}_{{\rm OC}}  = \! \int\!\!\int_{-\pi/2}^{\pi/2} \!\!\! \alpha(\theta,\phi) {g} (\theta,\phi) {\bf a}(\theta,\phi) {\rm d} \theta {\rm d} \phi
% \end{equation}
% where $\alpha(\theta,\phi)$ accounts for the small-scale fading and must be modeled stochastically. A common way to model $\alpha(\theta,\phi)$ is a zero-mean, complex-Gaussian random process with cross-correlation~\cite{Sayeed2002a}
% \begin{equation} \label{eq:scattering-correlation-model}
% \mathbb{E} \{ \alpha(\theta,\phi) \alpha^*(\theta',\phi') \} = \beta f(\theta,\phi) \delta(\phi-\phi')  \delta(\theta-\theta')
% \end{equation}
% where $\beta$ is the average channel gain, and $f(\theta,\phi)$ is the normalized \emph{spatial scattering function} \cite{Sayeed2002a} that describes the angular plane wave distribution and is such that $\iint f(\theta,\phi) d\theta d\phi  = 1$. Plugging~\eqref{eq:scattering-correlation-model} into~\eqref{multipathCont}, we obtain ${\bsy \Sigma} = \mathbb{E}\{{\bf d}_{{\rm OC}}   {{\bf d}^{\Htran} _{{\rm OC}}}\}$ with
% \begin{equation} \label{eq:spatial-correlation}
% \begin{split}
% \!\!\left[{\bsy \Sigma}\right]_{n,m} = \beta \! \int\!\!\int_{-\pi/2}^{\pi/2} \!\!\!{g} (\theta,\phi) f(\theta,\phi) e^{\imagunit {\bf k}(\theta,\phi)^{\Ttran}({\bf u}_n -{\bf u}_m) } {\rm d}\theta {\rm d} \phi.\!
% \end{split}
% \end{equation}
% The above model is valid for any $f(\theta,\phi)$. 

%Assume that the plane waves arriving at the receiver are uniformly distributed over the half-space in front of it. This corresponds to an isotropic propagation condition, characterized by
%\begin{align}\label{eq:pdf_angular}
%f(\theta,\varphi) = \frac{\cos (\theta)}{2 \pi} \; \phi \in [-\pi/2, \pi/2),\theta\in [-\pi/2, \pi/2).
%\end{align}
%Plugging~\eqref{eq:pdf_angular} into~\eqref{eq:spatial-correlation} yields 
%\begin{align}\notag
%\left[{\bsy \Sigma}\right]_{n,m} &=  \int\!\!\int_{-\pi/2}^{\pi/2}  \frac{\cos (\theta)}{2 \pi}e^{\imagunit {\bf k}(\theta,\phi)^{\Ttran}({\bf u}_n -{\bf u}_m) } {\rm d}\theta {\rm d} \phi  \\&=\sinc \left(\frac{2||{\bf u}_n -{\bf u}_m||}{\lambda}\right).\label{rho_r}
%\end{align}
%
%\Blue{EVITEREI DI METTERE L'ESPRESSIONE DEL CASO ISOTROPICO PERCH\`E NON \`E CORRETTA: NON TIENE CONTO DEL RADIATION PATTERN E NON TIENE CONTO DELL'EVENTUALE POLARIZZAZIONE.}




%
%
%The impedance vector ${\bf z}_{{\rm ART}}$ depends not only on antennas and array configuration but also on the transmission medium. Its computation requires to obtain the \textit{open-circuit} voltage array response represented by ${\bf v}_{\rm OC}={i}_{A}{\bf z}_{{\rm ART}}=v_{G} {\bf d}^\prime$ with
%\begin{equation}
%\label{doc}
%{\bf d}^\prime={F}_{{\rm T}} (Z_{G}+{Z}_{{\rm T}})^{-1} {\bf z}_{{\rm ART}}.
%\end{equation} 
%\Mar{To simplify the discussion, consider first a free-space propagation scenario and assume that the receive array is in the far-field region of the transmit antenna. Under a \textbf{plane-wave approximation} of the received electromagnetic field, we have
%\begin{equation}
%\label{doc2}
%{\bf d}^\prime = \alpha {\bf a}
%\end{equation}
%where $\alpha \in \mathbb C$ accounts for the path-loss and ${\bf a}$ is the (open-circuit) array manifold~\cite{Friedlander2018}. The latter can be written as
%\begin{equation}
%\label{aoc2}
%{\bf a}={g} (\theta,\phi) {\bf a}_{\rm isotropic}(\theta,\phi)
%\end{equation}
%where $\theta$ and $\phi$ are the elevation and azimuthal angles, respectively, associated to the direction of arrival of the signal, ${g} (\theta,\phi)$ is the radiation pattern of an \textit{isolated} $\lambda/2$-dipole, and
%\begin{equation}
%\label{aisotropic}
%{\bf a}_{\rm isotropic}(\theta,\phi) = \left[\begin{array}{c} e^{\imagunit \frac{2 \pi}{\lambda}[\cos \theta(x_{1} \cos \phi+y_{1} \sin \phi)+z_{1} \sin \theta}] \\ e^{\imagunit \frac{2 \pi}{\lambda}[\cos \theta(x_{2} \cos \phi+y_{2} \sin \phi)+z_{2} \sin \theta}] \\ \vdots \\ e^{\imagunit \frac{2 \pi}{\lambda}[\cos \theta(x_{M} \cos \phi+y_{M} \sin \phi)+z_{M} \sin \theta}] \end{array}\right]
%\end{equation}
%$(x_{m}, y_{m}, z_{m})$ being the coordinates of the dipoles' centers\footnote{If the receive antennas are not CMS antennas, the (isolated) radiation pattern ${g} (\theta,\phi)$ must be replaced with a vector of \textit{embedded} radiation patterns ${\bf g} (\theta,\phi)$~\cite[Eq. (10)]{Friedlander2018} and~\eqref{aoc2} becomes ${\bf a}={\bf g} (\theta,\phi) \odot {\bf a}_{\rm isotropic}(\theta,\phi)$~\cite[Eq. (9)]{Friedlander2018}, where $\odot$ denotes the Hadamard product.}.
%%\begin{equation}
%%\label{aoc}
%%{\bf a}_{\rm oc}={\bf g}_{\rm oc} (\theta,\phi) \odot {\bf a}_{\rm isotropic}(\theta,\phi)
%%\end{equation}
%%where $\theta$ and $\phi$ are the elevation and azimuthal angles, respectively, associated to the direction of arrival of the signal,
%%\begin{equation}
%%\label{goc}
%%{\bf g}_{\rm oc} (\theta,\phi) = \left[\begin{array}{c} {g}_{\rm oc,1} (\theta,\phi) \\ {g}_{\rm oc,2} (\theta,\phi) \\ \vdots \\ {g}_{{\rm oc},M} (\theta,\phi) \end{array}\right]
%%\end{equation}
%%is the radiation pattern of the \textit{embedded} antennas,
%%\begin{equation}
%%\label{aisotropic}
%%{\bf a}_{\rm isotropic}(\theta,\phi) = \left[\begin{array}{c} e^{\imagunit \frac{2 \pi}{\lambda}[\cos \theta(x_{1} \cos \phi+y_{1} \sin \phi)+z_{1} \sin \theta}] \\ e^{\imagunit \frac{2 \pi}{\lambda}[\cos \theta(x_{2} \cos \phi+y_{2} \sin \phi)+z_{2} \sin \theta}] \\ \vdots \\ e^{\imagunit \frac{2 \pi}{\lambda}[\cos \theta(x_{M} \cos \phi+y_{M} \sin \phi)+z_{M} \sin \theta}] \end{array}\right]
%%\end{equation}
%%$(x_{m}, y_{m}, z_{m})$ being the coordinates of the dipoles' centers. In general, the elements of ${\bf g}_{\rm oc} (\theta,\phi)$ are different, even if the antennas are identical and with the same orientation, due to coupling effects. They become equal in the presence of canonical minimum scattering antennas, which is the case with half-wavelength dipoles [7]. Accordingly, for the ULA shown in Fig.~\ref{fig:ULA} it can safely be assumed that
%%\begin{equation}
%%\label{aoc2}
%%{\bf a}_{\rm oc}={g}_{\rm oc} (\theta,\phi) {\bf a}_{\rm isotropic}(\theta,\phi)
%%\end{equation}
%%where ${g}_{\rm oc} (\theta,\phi)$ is the common value of the entries of ${\bf g}_{\rm oc} (\theta,\phi)$. 
%It is worth noting that the values of $\alpha$ and ${g}(\theta,\phi)$ are easily derived from standard electromagnetic formulas. Now, consider a propagation scenario in which multiple plane waves arrive at the receive array from $N_{p}$ different directions. In such a case we have
%\begin{equation}
%\label{multipath}
%{\bf d}^\prime = \sum\limits_{p=1}^{N_{p}}\alpha_{p} {g} (\theta_{p},\phi_{p}) {\bf a}_{\rm isotropic}(\theta_{p},\phi_{p})
%\end{equation}
%where $\alpha_{p}$ and $(\theta_{p},\phi_{p})$ are the complex gain and the direction-of-arrival for the $p$-th path. The channel model~\eqref{multipath} can straightforwardly be generalized to a continuum of plane waves~\cite{Sayeed2002a} as follows
%\begin{equation}
%\label{multipathCont}
%{\bf d}^\prime = \mathop{\iint}\limits_{-\pi/2}^{\pi/2} \alpha(\theta,\phi) {g}_{\rm oc} (\theta,\phi) {\bf a}_{\rm isotropic}(\theta,\phi) {\rm d} \theta {\rm d} \phi
%\end{equation}
%The microscopic fading created by small-scale mobility is captured by $\tilde{g}(\theta,\phi)=\alpha(\theta,\phi) {g} (\theta,\phi)$ which is a time-varying variable that can be modeled stochastically. In accordance to \cite{Sayeed2002a}, we take $g(\theta,\phi)$ as a spatially uncorrelated circularly symmetric Gaussian stochastic process with cross-correlation
%\begin{equation} \label{eq:scattering-correlation-model}
%\mathbb{E} \{ \tilde{g}(\theta,\phi) \tilde{g}^*(\theta',\phi') \} = \beta f(\theta,\phi) \delta(\phi-\phi')  \delta(\theta-\theta')
%\end{equation}
%where $\delta(\cdot)$ denotes the Dirac delta function, $\beta$ denotes the average channel gain (i.e., capturing path-loss and shadowing), and $f(\theta,\phi)$ is the normalized \emph{spatial scattering function} \cite{Sayeed2002a}.  The latter function describes the angular multipath distribution and the directivity gain of the antennas, and is normalized so that $\iint f(\theta,\phi) d\theta d\phi  = 1$. It thus follows that
%\begin{equation} \label{eq:spatial-correlation}
%\begin{split}
%{\bsy \Sigma} &= \mathbb{E}\{{\bf d}^\prime  {\bf d}^\prime^{\Htran} \} \\ &= \beta  \iint\limits_{-\pi/2}^{\pi/2} f(\theta,\phi) {\bf a}_{\rm isotropic}(\theta,\phi) {\bf a}_{\rm isotropic}^{\Htran}(\theta,\phi) {\rm d}\theta {\rm d} \phi
%\end{split}
%\end{equation}
%where the last equality follows from~\eqref{eq:scattering-correlation-model}.  
%}
%
