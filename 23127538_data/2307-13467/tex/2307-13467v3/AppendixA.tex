%!TEX root = jsac_v7.tex

{\color{blue}\section*{Appendix A}\label{Appendix}
In this Appendix, we derive the expression of ${\bf d}_{{\rm OC}}$ given in \eqref{eq:d_OC}. The computation will be made for a LoS propagation scenario on the basis of the two following assumptions:

\begin{assumption}
Each antenna of the BS array is in the far field of the UE transmit antenna. In the far-field region, the transmit antenna behaves like a \textit{source point}, so the radiated field can be approximately characterized by spherical wavefronts. If the transmitting antenna has a maximum dimension of $L_t$, the far-field region is at distances greater than $2L_t^2/\lambda$.

\end{assumption}
\begin{assumption}
The electromagnetic wave produced by the UE transmitting antenna and impinging on a receive antenna of the BS array can be \textit{locally} approximated by a plane wave. 
This approximation can be made provided that the distance between the transmit and receive antennas is greater than $2 L_r^2 / \lambda$, where $L_r$ represents the maximum dimension of the \textit{receive} antenna. 
\end{assumption}

Let ${\bf E}_{{\rm inc},m}$ denote the electric field incident on the $m$th antenna of the BS array, produced by the UE's voltage source. Based on Assumption 2, we model ${\bf E}_{{\rm inc},m}$ as a plane wave that reaches the receive antenna from a particular azimuth angle $\phi_{{\rm inc},m} \in [-\pi/2, \pi/2)$ and elevation angle $\theta_{{\rm inc},m} \in [-\pi/2, \pi/2)$. Assuming that the BS array consists of canonical minimum scattering (CMS) antennas and the incident field is linearly polarized, the $m$th element of ${\bf v}_{{\rm OC}}$ reads~\cite[Eq. (2-93)]{balanis}
\begin{equation}
\label{voc_einc}
v_{{\rm OC},m} = {\bf E}_{{\rm inc},m} \cdot {\bf l}^{(\rm r)}_{{\rm eff},m}(\theta_{{\rm inc},m},\phi_{{\rm inc},m})
\end{equation}
where ${\bf l}^{(\rm r)}_{{\rm eff},m}(\theta_{{\rm inc},m},\phi_{{\rm inc},m})$ is the \textit{effective length} ~\cite[Eq. (2-91)]{balanis} of the \textit{isolated} $m$th element of the array towards the $(\theta_{{\rm inc},m},\phi_{{\rm inc},m})$ direction. 
% The effective length is also useful for expressing the field radiated by an antenna in the far-field region. 
From Assumption 1, we have that~\cite[Eq. (2-92)]{balanis}
\begin{equation}
\label{eq:Einc}
\begin{split}
{\bf E}_{{\rm inc},m} = -\imagunit k  Z_{0} i_{\rm AT} \dfrac{e^{- \imagunit \frac{2\pi}{\lambda} r_m}}{4 \pi r_m} {\bf l}^{(\rm t)}_{{\rm eff}} (\theta_{{\rm inc},m},\phi_{{\rm inc},m})  
\end{split}
\end{equation}
where $i_{\rm AT}$ is the current feeding the antenna, $r_m$ is the distance between the centers of the transmit and receive antennas, ${\bf l}^{(\rm t)}_{{\rm eff}} (\theta_{{\rm inc},m},\phi_{{\rm inc},m})$ is the effective length of the transmit antenna in the direction of departure which, in a LoS scenario, coincides with the direction of arrival $(\theta_{{\rm inc},m},\phi_{{\rm inc},m})$. Plugging \eqref{eq:Einc} into \eqref{voc_einc} yields
\begin{equation}
\label{voc_einc_2}
\begin{split}
v_{{\rm OC},m} =  \alpha'(\theta_m,\phi_m,r_m)  Z_{0} i_{\rm AT} e^{- \imagunit \frac{2\pi}{\lambda} r_m}
\end{split}
\end{equation}
with
\begin{equation}
\label{alfap}
\alpha'(\theta_m,\phi_m,r_m) =  - \imagunit \dfrac{{\bf l}^{(\rm t)}_{{\rm eff}} (\theta_m, \phi_m) \cdot {\bf l}^{(\rm r)}_{{\rm eff}}(\theta_m,\phi_m)}{2 \lambda r_m} 
\end{equation}
where the term ${\bf l}^{(\rm t)}_{{\rm eff}} (\theta_m, \phi_m)\cdot{\bf l}^{(\rm r)}_{{\rm eff}}(\theta_m,\phi_m)$ accounts for the polarization loss~\cite[Sect. 2.12.2]{balanis}. For the sake of notation, we have dropped the subscript $_{\rm inc}$ so that $\theta_{{\rm inc},m}$ and $\phi_{{\rm inc},m}$ become $\theta_{m}$ and $\phi_{m}$, respectively. According to \eqref{voc_einc_2}, we can write ${\bf v}_{\rm OC} = Z_{0} i_{\rm AT} \boldsymbol{\alpha}'(\boldsymbol{\psi},\bf{r})\odot{\bf a}({\bf r})$. Finally, from ${\bf v}_{{\rm OC}} = v_{\rm G} {\bf d}_{{\rm OC}}$ and $i_{\rm AT} = F_{\rm T} (Z_{\rm G}+Z_{\rm T})^{-1} v_{\rm G}$, we obtain \eqref{eq:d_OC} where $\boldsymbol{\alpha}(\boldsymbol{\psi},{\bf r}) = F_{\rm T} (Z_{\rm G}+Z_{\rm T})^{-1} Z_0\boldsymbol{\alpha}'(\boldsymbol{\psi},{\bf r})$.
% where $\boldsymbol{\psi} = ((\theta_1,\phi_1),(\theta_2,\phi_2),\cdots,(\theta_M,\phi_M))$ and ${\bf r} = (r_1,r_2,\cdots,r_M)$ are vectors collecting the directions of arrival and the distances, $\boldsymbol{\alpha}'(\boldsymbol{\psi},{\bf r}) = [\alpha'(\theta_1,\phi_1,r_1),\alpha'(\theta_2,\phi_2,r_2), \cdots,\alpha'(\theta_M,\phi_M,r_M)]^\Ttran$, ${\bf a}({\bf r}) = [e^{- \imagunit \frac{2\pi}{\lambda} r_1}, e^{- \imagunit \frac{2\pi}{\lambda} r_2}, \cdots, e^{- \imagunit \frac{2\pi}{\lambda} r_M}]^\Ttran$.
% $\boldsymbol{\alpha}'(\boldsymbol{\theta},\boldsymbol{\phi})=[\alpha'(\theta_1,\phi_1),\alpha'(\theta_2,\phi_2), \cdots,\alpha'(\theta_M,\phi_M)]^\Ttran$, $\boldsymbol{\theta}=(\theta_1,\theta_2,\cdots,\theta_M)$, $\boldsymbol{\phi}=(\phi_1,\phi_2,\cdots,\phi_M)$, ${\bf a}({\bf r}) = [e^{- \imagunit \frac{2\pi}{\lambda} r_1}, e^{- \imagunit \frac{2\pi}{\lambda} r_2}, \cdots, e^{- \imagunit \frac{2\pi}{\lambda} r_M}]^\Ttran$, ${\bf r} = (r_1,r_2,\cdots,r_M)$, and $\odot$ denotes the Hadamard product.
% \begin{equation}
% \label{voc_einc_2}
% \begin{split}
% v_{{\rm OC},m} =  - \imagunit k  Z_{0} i_{\rm AT} \dfrac{e^{- \imagunit \frac{2\pi}{\lambda} r_m}}{4 \pi r_m} {\bf l}^{(\rm t)}_{{\rm eff}} (\theta_{{\rm inc},m}, \phi_{{\rm inc},m}) \cdot {\bf l}^{(\rm r)}_{{\rm eff},m}(\theta_{{\rm inc},m},\phi_{{\rm inc},m}).
% \end{split}

Depending on the relationship between the size of the array and its distance from the transmitting antenna, the expression of $v_{{\rm OC},m}$ can be simplified according to, for example, the \textit{Fresnel approximation} or the well-known \textit{planar wave approximation} \cite{Friedlander2019}. The latter differs from the planar wave approximation of Assumption 2 because it is relevant to the array while the one in Assumption 2 is relevant to the single array element. Under the planar wave approximation, 
% we consider a single direction of arrival $(\theta,\phi)$, coincident with the direction of the line connecting the centers of the array and the transmit antenna, and replace $r_m$ with $r$ in the denominator of \eqref{voc_einc_2}, where $r$ is the distance between the two centers. Correspondingly, 
\eqref{eq:d_OC} is reduced to the well-known expression in \eqref{doc}
where $\alpha(\theta,\phi,r) = F_{\rm T} (Z_{\rm G}+Z_{\rm T})^{-1} Z_0\alpha'(\theta,\phi,r)$, 
\begin{equation}
\alpha'(\theta,\phi, r) =  - \imagunit e^{\imagunit \psi_0}\dfrac{{\bf l}^{(\rm t)}_{{\rm eff}} (\theta, \phi) \cdot {\bf l}^{(\rm r)}_{{\rm eff}}(\theta,\phi)}{2 \lambda r}
\end{equation}
and $\psi_0 = - 2 \pi r/\lambda$ being the reference phase at array center. %Finally, the array response vector is ${\bf a}(\theta,\phi) = [ e^{\imagunit {\bf k}(\theta,\phi) \cdot {\boldsymbol \delta}_1},\ldots, e^{\imagunit {\bf k}(\theta,\phi)\cdot {\boldsymbol \delta}_M}]^{\Ttran}$ where ${\boldsymbol \delta}_m$ is the displacement vector from the center of the array to the center of the $m$th receive antenna and ${\bf k}(\theta,\phi)$ is the wave vector~\cite[eq. (17.1.4)]{orfanidis}.
}

% The expression of $v_{{\rm OC},m}$ is simplified if the size of the receiving array is much smaller than its distance from the transmitter. More precisely, let $d$ be the distance between the center of the array and the center of the transmitting antenna, and ${\boldsymbol \delta}_m$ be the displacement vector from the center of the array to the center of the $m$th receiving antenna. If $d \gg \pi \delta_m^2/\lambda$, with $\delta_m = \|{\boldsymbol \delta}_m\|$, we have that $r_m \approx d + {\bf k}(\theta_{,m}, \phi_{,m}) \cdot {\boldsymbol \delta}_m$ \cite{Friedlander2019} where ${\bf k}(\theta,\phi)$ is the wave vector~\cite[eq. (17.1.4)]{orfanidis}. 
% % The condition $d \gg \pi \delta_m^2/\lambda$ is commonly replaced by $d \gg 2L^2_{\rm A} /\lambda$, where $L_{\rm A}$ is the maximum dimension of the receive array or, equivalently, the diameter of the smallest sphere that completely contains it. 
% In such a case, two further approximations can be made. The first is to replace $r_m$ by $d$ in the denominator of \eqref{alfap}. The second is to assume that the arrival directions $(\theta_{m},\phi_{m})$ are independent of $m$ and coincide with the direction $(\theta,\phi)$ of the line connecting the center of the array and the center of the transmit antenna. Thus, \eqref{voc_einc_2} reduces to

% with
% \begin{equation}
% \label{alfap}
% \alpha'(\theta,\phi) =  - \imagunit e^{\imagunit \psi_0}\dfrac{{\bf l}^{(\rm t)}_{{\rm eff}} (\theta, \phi) \cdot {\bf l}^{(\rm r)}_{{\rm eff}}(\theta,\phi)}{2 \lambda d} 
% \end{equation}
% where $\psi_0 = - 2 \pi d/\lambda$ is the reference phase at the center of the array and the term ${\bf l}^{(\rm t)}_{{\rm eff}} (\theta, \phi)\cdot{\bf l}^{(\rm r)}_{{\rm eff}}(\theta,\phi)$ accounts for the polarization loss~\cite[Sect. 2.12.2]{balanis}. 
% According to \eqref{voc_einc_3}, we can write ${\bf v}_{\rm OC} = \alpha'(\theta,\phi) Z_{0} i_{\rm AT} {\bf a}(\theta,\phi)$
% \begin{equation}
% \label{voc_vec}
% {\bf v}_{\rm OC} = \alpha'(\theta_{\rm inc},\phi_{\rm inc}) Z_{0} i_{\rm AT} {\bf a}(\theta_{\rm inc},\phi_{\rm inc})
% \end{equation}
% with ${\bf a}(\theta,\phi) = [ e^{\imagunit {\bf k}(\theta,\phi) \cdot {\boldsymbol \delta}_1},\ldots, e^{\imagunit {\bf k}(\theta,\phi)\cdot {\boldsymbol \delta}_M}]^{\Ttran}$. 
% If the transmitter has a single-antenna,  ${\bf Z}_{{\rm ART}}$ reduces to a vector 
% \begin{equation}
% {\bf z}_{{\rm ART}} = \frac{{\bf v}_{{\rm OC}}}{i_{\rm AT}}=  Z_{0}  \boldsymbol{\alpha}'(\boldsymbol{\psi},{\bf r})\odot{\bf a}({\bf r}). 
% \end{equation}

% \begin{equation}
%  \label{doc}
%  {\bf d}_{{\rm OC}} =  \alpha(\theta_{\rm inc},\phi_{\rm inc}) {\bf a}(\theta_{\rm inc},\phi_{\rm inc})
%  \end{equation}
% $\alpha(\theta_{\rm inc},\phi_{\rm inc}) = F^T_{\rm T} (Z_{\rm G}+Z_{\rm T})^{-1} Z_0\alpha'(\theta_{\rm inc},\phi_{\rm inc})$.



% The effective length is a far-field parameter~\cite[Eq. (2-91)]{balanis} that is also useful for expressing the field ${\bf E}_{\rm rad}$ radiated by an antenna at a distance $r$. 
% Denoting with $i_{\rm AT}$ the current feeding the antenna, we can write~\cite[Eq. (2-92)]{balanis}
% \begin{equation}
% \label{eq:Einc}
% \begin{split}
% {\bf E}_{\rm rad} = -\imagunit k  Z_{0} i_{\rm AT} \dfrac{e^{- \imagunit \frac{2\pi}{\lambda} r}}{4 \pi r} {\bf l}^{(\rm t)}_{{\rm eff}} (\theta, \phi)  
% \end{split}
% \end{equation}
% where ${\bf l}^{(\rm t)}_{{\rm eff}} (\theta, \phi)$ is the effective length of the transmitting antenna towards the direction of departure $(\theta,\phi)$. Based on the above results, in a LoS propagation scenario the expression of $v_{{\rm OC},m}$ can be derived for an arbitrary BS array configuration.  Taking A1 into account, from \eqref{eq:Einc} we can write the incident field on the $m$th receive antenna as ${\bf E}_{{\rm inc},m} = -\imagunit k  Z_{0} i_{\rm AT} \dfrac{e^{- \imagunit \frac{2\pi}{\lambda} r_m}}{4 \pi r_m} {\bf l}^{(\rm t)}_{{\rm eff}} (\theta_{{\rm t},m}, \phi_{{\rm t},m})$
%