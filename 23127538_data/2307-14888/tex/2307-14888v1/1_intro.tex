\section{Introduction}
\label{sec:intro}

The Standard Model (SM) was successfully confirmed when the Higgs boson was discovered by the Large Hadron Collider (LHC)\cite{Aad_2012,Chatrchyan_2012} experiments. Since then, the measurement of its properties with the highest accuracy has become a primary goal.
For that reason, in parallel with the LHC data exploitation, the high-energy accelerator-based particle physics community works towards constructing the next large accelerator after the LHC. Such a machine will be an $e^{+}e^{-}$ collider at relatively high energy to achieve the highest precision on the Higgs physics measurements. 
Different projects have been proposed and are under discussion.
These are the so-called \textit{Higgs factories}.

The International Linear Collider (ILC)\cite{Behnke:2013xla,Baer:2013cma,Adolphsen:2013jya,Adolphsen:2013kya,Behnke:2013lya} is one of the \textit{Higgs factories} proposed for the future. 
The ILC operation foresees $e^{+}e^{-}$ collisions at different centres of mass energies ranging from the Z-pole (Giga-Z) to up to 1 TeV after energy upgrades. 
The nominal program defines an initial stage at 250GeV (ILC250), with a luminosity upgrade, and a following one at 500GeV (ILC500), after an energy upgrade. 
The ILC also features polarised beams: $80\%$ for electrons and $30\%$ for positrons.
The ILC250 physics program foresees a total integrated luminosity of 2000 \fb distributed between four different beam polarisation schemes: $45\%$ in $P_{\mathrm{e^{-}e^{+}}}=(-0.8,+0.3)$, $45\%$ in $P_{\mathrm{e^{-}e^{+}}}=(+0.8,+0.3)$, $5\%$ in $P_{\mathrm{e^{-}e^{+}}}=(-0.8,-0.3)$ and $5\%$ in $P_{\mathrm{e^{-}e^{+}}}=(+0.8,-0.3)$. 
The ILC500 program foresees a total integrated luminosity of 4000 \fb distributed among the different polarisation configurations in the same proportions as the ILC250 case.

The ILD~\cite{Behnke:2013lya,ILD:2020qve} is one of the two proposed detector concepts for the ILC. 
It is a highly-hermetic multi-purpose detector designed for the maximal exploitation of particle flow techniques in event reconstruction. 

This document updates the studies presented in \cite{Irles:2023ojs}.
In that work, the experimental prospects for \eeqq measurements at ILC250 are studied. 
The study of such topologies is critical to obtain a complete picture of the electroweak sector interactions between SM bosons and fermions, particularly quarks.
For instance, models of new physics featuring extra-dimensions \cite{Yoon:2018xud,Funatsu:2017nfm,Funatsu:2020haj} have been proposed to explain the striking mass hierarchy in the fermion sector. These models predict deviations in the electroweak sector with  modification of the SM couplings and new heavy resonances $Z^{\prime}$ that interact with the fermions.
Furthermore, the LEP/SLC anomaly in \eebb is still unexplained \cite{Djouadi:2006rk}. The effects of new physics may differ for different fermion chiralities and additional terms associated with various mediators (SM $Z$ and $\gamma$ or beyond SM $Z^{\prime}$ or mixing of these). This motivates the study of quark pair production in high energy $e^{-}e^{+}$ collisions at past lepton colliders \cite{ALEPH:2005ab} and at future ones \cite{Irles:2023ojs}.  

In \cite{Irles:2023ojs}, two experimental observables have been studied in detail for the ILC250 using the ILD detector in full simulation.
The observables are the Hadronic Fraction (\Rq) and Forward-Backward Asymmetry (\AFB). The observable \Rq is defined as:
\begin{equation}
R_{q}=\frac{\sigma_{e{-}e^{+}\rightarrow q\bar{q}}}{\sigma_{had.}}
\label{formula:Rq}
\end{equation} 
where $\sigma_{had.}$ is the integration of $\sigma_{q\bar{q}}$ for all quark flavours except the top quark. In the case of \AFB the definition reads:
\begin{equation}
A^{q\bar{q}}_{FB}=\frac{\sigma^{F}_{e{-}e^{+}\rightarrow q\bar{q}}-\sigma^{B}_{e{-}e^{+}\rightarrow q\bar{q}}}{\sigma^{F}_{e{-}e^{+}\rightarrow q\bar{q}}+\sigma^{B}_{e{-}e^{+}\rightarrow q\bar{q}}}
\label{formula:AFB}
\end{equation}
where $\sigma^{F/B}_{e{-}e^{+}\rightarrow q\bar{q}}$ is the cross-section in the forward/backward hemisphere as defined by the polar angle $\theta_q$.

In this contribution, we report on incorporating an improved flavour tagger that uses the hadron identification (protons, kaons, pions) as input. In addition, the study has been extended to include the prospects of the same study at ILC500. Finally, we include a prospective study of the precisions that could be reached if a pixel TPC is used for the ILD.