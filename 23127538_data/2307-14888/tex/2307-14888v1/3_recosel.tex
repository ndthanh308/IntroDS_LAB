\section{\Rq and \AFB reconstruction at ILD}
\label{sec:recosel}
The reconstruction of the events begins with the track reconstruction performed by the \texttt{MarlinTrk} framework, part of \texttt{ILCSoft}. 
Later, \texttt{Pandora}\cite{Marshall:2015rfa} runs the particle flow algorithm (PFA) that matches the tracking information with the high-granular calorimetry information following pattern recognition techniques. The resulting reconstructed objects are denominated particle flow objects (PFO) and are treated as single particles.

Once the PFOs are reconstructed, the vertex reconstruction, jet reconstruction and flavour tagging are performed with \texttt{LCFI+} software tool.

Once all high-level reconstruction objects are obtained, the event preselection is performed. It is based on a series of kinematical cuts applied to enrich the data sample with signal events while removing the backgrounds. The preselection for 250 GeV is the same as described in \cite{Irles:2023ojs}, where the variables are described, with minor adaptations made for the study of the 500 GeV samples (noted in parenthesis): 
\begin{enumerate}
    \item photon veto cuts, rejecting events if:
    \begin{enumerate}
        \item at least one of the jets contains only one PFO;
        \item at least one of the jets contains a reconstructed $\gamma_{cluster}$ with E > 115 GeV (220 GeV);
    \end{enumerate}
    \item events with $\sin \Psi_{acol}$ > 0.3 are rejected;
    \item events with $m_{jj}$ < 140 GeV (200 GeV) are rejected;
    \item events with $y_{23}$ > 0.02 (0.007) are rejected.
\end{enumerate}
The most noticeable change for the ILC500 case is performed in the last cut, which is tightened to reduce the much larger $WW$ background contamination in the case of left-handed polarisation. This last cut decreases the final preselection efficiency to $\sim50\%$ instead of $\sim75\%$ for the ILC250 case.

The next step is applying the Double Tag (DT) method stated in \cite{Irles:2023ojs} by which a cut is performed in the \btag (or \ctag) likelihood in both jets of the event. The working points are defined as in \cite{Irles:2023ojs} such that the mistagging of the other quarks when performing the tagging of \bquark (or \cquark) is smaller than the $\sim1.5\%$ $(\sim 3\%)$, assuming the same size for the $b/c/uds$ samples. The DT yields a high-purity selection of the events for each flavour and is the last necessary step to reconstruct $R_q$ measurements.

To study \AFB, the charge of the jets needs to be reconstructed to fully determine the direction of the $q\bar{q}$ system. As described in \cite{Irles:2023ojs}, two independent methods are used for the charge measurement: \Bc or \Kc. The \Bc counts the charge of all tracks in secondary vertices in the jet, while the \Kc only uses the information of tracks identified as Kaons from the TPC dE/dx PID. The Double Charge (DC) method performs the final charge measurement, which combines the possible measurements of both methods and accepts only jets with opposite charges. 
The differential, in \costheta, cross section is reconstructed, and the \AFB extracted from a fit restricted to ($\|\costheta\|>0.9$) to avoid the region where the reconstruction efficiency drops due to insufficient acceptance of the tracking detectors. %The last part of the reconstruction of \AFB is an extrapolation to such regions by fitting the data in the -0.9 < \costheta < 0.9 region.

All the methods (Preselection, DT, DC, Fit) have been developed for the analysis at 250 GeV and are explained in detail in \cite{Irles:2023ojs} together with a comprehensive study of the most dominant systematic uncertainties.
The same methods are used for the study at 500 GeV shown in this document.
In Appendix \ref{sec:appendixPlots}, we collect the most relevant performance plots for the ILC500 case to be compared with the ones in \cite{Irles:2023ojs}.

%\begin{itemize}
%    \item Describe, very briefly, the methods described in the ILD note.
%    \item Focus on the Flavour tagging and jet-charge measurement methods.
%    \item Adaptations required to analyze 500GeV
%\end{itemize}

