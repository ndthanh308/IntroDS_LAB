\section{The ILD Concept and MC Samples}
\label{sec:ILD}

%0.5 pages.
%Describe here the ILD detector (tracking system in particular, adding the description of a potential ILD with a pixel TPC), the samples, the reconstruction tools etc.
%silicon tracking systems (silicon internal tracking, SIT, and forward tracking detector, FTD)
%Mention here the tracking software and the LCFIplus.\\
The ILD subdetector layout consists of a high-precision vertex detector (VTX), silicon tracking systems, a time projection chamber (TPC), a highly granular calorimeter system (ECAL and HCAL) and a muon catcher. All the aforementioned subdetectors are placed inside a solenoid providing a magnetic field of $3.5$\,T, surrounded by an iron yoke instrumented for muon detection. The ILD TPC \cite{thelctpccollaboration2016time,LCTPC:2022pvp} is a large volume time projection chamber that allows continuous 3D tracking and particle identification. %, usually based on energy loss (\dEdx). 
Its baseline design consists of a barrel-shaped structure with an inner radius of 329 mm and an outer radius larger than 1808 mm. 
It provides a single-point resolution of 100 \textmu m over about 200 readout points and a \dEdx resolution of $\sim4.5\%$. It is based on $1\times 6$ mm$^{2}$ pads readout with GEM or MicroMegas technologies.
This document will also explore the potential of an alternative approach: a pixel TPC. Several technological solutions are under study, and all envision a four times greater readout density with a pixel size of 300 \textmu m. Simulations extrapolating beam test results show that an improved relative resolution of $\sim3$-$4\%$ will be feasible using cluster counting techniques (\dNdx) instead of the traditional approach \cite{LCTPC:2022pvp}. 

This study has been conducted running full simulation via \texttt{ILCSOFT} \textit{v02-02-03}\footnote{Link: \url{https://github.com/iLCSoft}}, which merge different software packages and algorithms that operate in a modular way from MC events to final reconstruction. All simulations use the ILD-L\cite{ILD:2020qve} model, whose geometry, material and readout systems are implemented in the \texttt{DD4HEP} framework\cite{Frank:2014zya}, interfaced with \texttt{Geant4} toolkit. Both signal and background events are generated with the \texttt{WHIZARD}\cite{Kilian:2007gr} event generator at LO. The beam energy spectrum, beam-beam interaction and QED ISR is generated via \texttt{Guinea-Pig}\cite{Schulte:1999tx}. Non-perturbative effects, such as the FSR and hadronization, are provided by \texttt{Pythia} event generator\cite{Sj_strand_2006}.

To reproduce different stages of the ILC operation, two different samples have been provided by the ILD concept group; one for 250 GeV and one for 500 GeV. Each set of samples features full polarised beams in different configurations. The main samples for this study are the $\mathrm{e^-_{L}}\mathrm{e^+_{R}}$ and $\mathrm{e^-_{R}}\mathrm{e^+_{L}}$. Background processes from electroweak bosons, Higgs and top-quark production (at 500 GeV) are studied using additional MC samples. Backgrounds from the production of lepton pairs have been ignored since they are expected to be easily identified.
All distributions and results have been reweighted to match the baseline polarization and full luminosity scenarios for ILC250 and ILC500, so-called the H20 scenario \cite{Bambade:2019fyw}.

The generation of the \eeqqbar signal events and the \eeZgammaqqbar with the Z-boson generated on-shell are done simultaneously. Following the same recipe as in \cite{Irles:2023ojs}, we define the signal as the events with acolinearity of the \qqbar system (at parton level) smaller than 0.3 and a large invariant mass of the \qqbar pair larger than 140 GeV - for the ILC250 case - or larger than 200 GeV for the ILC500 case  (i.e. large enough to be away of the Z-mass peak).

The signal and background cross sections are listed in Tables 1 and 2 from \cite{Irles:2023ojs} for ILC250 and in Tables \ref{tab:crosssection500} and \ref{tab:crosssection_bkg500} for ILC500. Only backgrounds leading to fully hadronic final states are considered. Backgrounds involving leptons in the final states are ignored since those are expected to be easily identified.

\begin{table}[!ht]
  \centering
  \begin{tabular}{c|ccc|ccc}
    \hline
     & \multicolumn{3}{|c|}{ $\sigma_{\eeqqbar}$[fb]} & \multicolumn{3}{|c}{ Radiative Return BKG [fb]} \\
    \hline
    Polarisation & \bbbar & \ccbar & \qqbar ($q=uds$) & \bbbar & \ccbar & \qqbar ($q=uds$) \\
    \hline
    \eLpR & 611.4 & 1545.9 & 2770.2 & 5506.4 & 5118.0 & 16244.2\\
    \eRpL & 416.9 & 893.8 & 1728.3 & 3002.5 & 2766.9 & 8773.5 \\
    \hline
  \end{tabular}
  \caption{Production cross section of signal and background originated by di-boson production. \label{tab:crosssection500}}
\end{table}

\begin{table}[!ht]
 %\renewcommand{\arraystretch}{1.6}
 \centering
  \begin{tabular}{c|c|c|c|c}
    \hline
    & \multicolumn{4}{|c}{  $\sigma_{e^{-}e^{+}\rightarrow\,X}$ [fb]} \\
    \hline
	Polarisation &  $X= WW \rightarrow q_{1} \bar{q_{2}} q_{3} \bar{q_{4}}$ & $X= ZZ \rightarrow q_{1} \bar{q_{1}}q_{2} \bar{q_{2}}$ & $X= HZ \rightarrow q\bar{q}H$ & $X=t\bar{t} \rightarrow b\bar{b}q_{1}\bar{q}_{2}q_{3}\bar{q}_{4}$\\
    \hline
    \eLpR & 7680.0 & 680.2 & 114.7 & 660 \\
    \eRpL & 33.5 & 271.9 & 73.4 & 254.8 \\
      \hline
  \end{tabular}
  \caption{\label{tab:crosssection_bkg500} Cross sections at 250 GeV for processes producing at least one pair of $q$-quarks using fully polarised beams. }
\end{table}