\documentclass[11pt,a4paper]{scrartcl}

% Defines default style and includes several useful packages
\usepackage{ILD}

% Useful macros for writing ILDdp notes
\usepackage[symbol]{footmisc}
\usepackage{feynmf}
\usepackage{multirow}
\usepackage{textpos}
\usepackage{lineno}
\usepackage{afterpage}

\usepackage{tabularx}
    \newcolumntype{L}{>{\raggedright\arraybackslash}X}
%\linenumbers

%\setlength\dashlinedash{0.2pt}
%\setlength\dashlinegap{1.5pt}
\setlength\arrayrulewidth{0.3pt}

%============================================%
% Set up the title page
%============================================%

% Set the title of the note
\newcommand{\old}[1]{\textcolor{red}{{#1}}}
\newcommand{\todo}[1]{\textcolor{blue}{{#1}}}
\newcommand{\todored}[1]{\textcolor{red}{{#1}}}
\newcommand{\todomagenta}[1]{\textcolor{magenta}{{#1}}}

% Commands & definitions

\def\Gammaqq{\ensuremath{\Gamma_{q\bar{q}}}\xspace}
\def\Gammahad{\ensuremath{\Gamma_{hadrons}}\xspace}

\def\Kgamma{\ensuremath{E_{\gamma}}\xspace}
\def\Kcut{\ensuremath{E_{\gamma}^{cut}}\xspace}
\def\acolcut{\ensuremath{\sin{(\Psi_{acol})}^{cut}}\xspace}

\def\Kreco{\ensuremath{K_{reco}}\xspace}
\def\kaonness{\ensuremath{\Delta_{\dEdx-K}}\xspace}
\def\pionness{\ensuremath{\Delta_{\dEdx-\pi}}\xspace}
\def\protonness{\ensuremath{\Delta_{\dEdx-p}}\xspace}
\def\epsilonhad{\ensuremath{\epsilon_{had}}\xspace}
\def\epsilonb{\ensuremath{\epsilon_{b}}\xspace}
\def\epsilonc{\ensuremath{\epsilon_{c}}\xspace}
\def\epsilonuds{\ensuremath{\epsilon_{uds}}\xspace}
\def\epsilonb2{\ensuremath{\epsilon^{2}_{b}}\xspace}
\def\epsilonc2{\ensuremath{\epsilon^{2}_{c}}\xspace}
\def\epsilonuds2{\ensuremath{\epsilon^{2}_{uds}}\xspace}

\def\costheta{\ensuremath{\cos \theta}\xspace}
\def\costhetab{\ensuremath{\cos \theta_{b}}\xspace}
\def\costhetaq{\ensuremath{\cos \theta_{q}}\xspace}
\def\sinthetaq{\ensuremath{\sin \theta_{q}}\xspace}
\def\costhetasq{\ensuremath{\cos^2 \theta_{b}}\xspace}
\def\sinthetasq{\ensuremath{\sin^2 \theta_{b}}\xspace}
\def\costhetaj{\ensuremath{\cos \theta_{j}}\xspace}
\def\costhetac{\ensuremath{\cos \theta_{c}}\xspace}

\def\fb{fb\ensuremath{^{-1}}\xspace}
\def\mum{\textmu\ensuremath{m}\xspace}
%\def\eL{\ensuremath{e_L^{\mbox{\scriptsize -}}}\xspace}
%\def\eR{\ensuremath{e_R^{\mbox{\scriptsize -}}}\xspace}
%\def\pL{\ensuremath{e_L^{\mbox{\scriptsize +}}}\xspace}
%\def\pR{\ensuremath{e_R^{\mbox{\scriptsize +}}}\xspace}

\def\eL{\ensuremath{e_L^{-}}\xspace}
\def\eR{\ensuremath{e_R^{-}}\xspace}
\def\pL{\ensuremath{e_L^{+}}\xspace}
\def\pR{\ensuremath{e_R^{+}}\xspace}

\def\b{\ensuremath{b}\xspace}
\def\bbar{\ensuremath{\overline{b}}\xspace}
\def\bbbar{\ensuremath{b}\ensuremath{\overline{b}}\xspace}
\def\qqbar{\ensuremath{q}\ensuremath{\overline{q}}\xspace}
\def\ccbar{\ensuremath{c}\ensuremath{\overline{c}}\xspace}
\def\ttbar{\ensuremath{t}\ensuremath{\overline{t}}\xspace}
\def\eebbbar{\ensuremath{e^{-}e^{+}\rightarrow b\bar{b}}\xspace}%\ensuremath{e^{\mbox{\scriptsize +}}}\ensuremath{\rightarrow}\ensuremath{b}\ensuremath{\overline{b}}\xspace}
\def\eeccbar{\ensuremath{e^{-}e^{+}\rightarrow c\bar{c}}\xspace}
\def\eeqqbar{\ensuremath{e^{-}e^{+}\rightarrow q\bar{q}}\xspace}
\def\ee{\ensuremath{e^{-}e^{+}}\xspace}
\def\eeZ{\ensuremath{e^{-}e^{+}\rightarrow Z}\xspace}
\def\eeZqqbar{\ensuremath{e^{-}e^{+}\rightarrow Z\rightarrow q\bar{q}}\xspace}
\def\eeZgammaqqbar{\ensuremath{e^{-}e^{+}\rightarrow Z \gamma \rightarrow q\bar{q} \gamma}\xspace}

\def\eebb{\ensuremath{e^{-}e^{+}\rightarrow b\bar{b}}\xspace}%\ensuremath{e^{\mbox{\scriptsize +}}}\ensuremath{\rightarrow}\ensuremath{b}\ensuremath{\overline{b}}\xspace}
\def\eecc{\ensuremath{e^{-}e^{+}\rightarrow c\bar{c}}\xspace}
\def\eeqq{\ensuremath{e^{-}e^{+}\rightarrow q\bar{q}}\xspace}
\def\eeZqq{\ensuremath{e^{-}e^{+}\rightarrow Z\rightarrow q\bar{q}}\xspace}
\def\eeZgammaqq{\ensuremath{e^{-}e^{+}\rightarrow Z \gamma \rightarrow q\bar{q} \gamma}\xspace}

%\def\eecc{\ensuremath{e^{\mbox{\scriptsize -}}}\ensuremath{e^{\mbox{\scriptsize +}}}\ensuremath{\rightarrow}\ensuremath{c}\ensuremath{\overline{c}}\xspace}
%\def\eeqq{\ensuremath{e^{\mbox{\scriptsize -}}}\ensuremath{e^{\mbox{\scriptsize +}}}\ensuremath{\rightarrow}\ensuremath{q}\ensuremath{\overline{q}}\xspace}
%\def\ee{\ensuremath{e^{\mbox{\scriptsize -}}}\ensuremath{e^{\mbox{\scriptsize +}}}\xspace}
%\def\eeZ{\ensuremath{e^{\mbox{\scriptsize -}}}\ensuremath{e^{\mbox{\scriptsize +}}}\ensuremath{\rightarrow}\ensuremath{Z}\xspace}
%\def\eeZqq{\ensuremath{e^{\mbox{\scriptsize -}}}\ensuremath{e^{\mbox{\scriptsize +}}}\ensuremath{\rightarrow}\ensuremath{Z}\ensuremath{\rightarrow}\ensuremath{q}\ensuremath{\overline{q}}\xspace}
%\def\eLpRbb{\ensuremath{e_L^{\mbox{\scriptsize -}}}\ensuremath{e_R^{\mbox{\scriptsize +}}}\ensuremath{\rightarrow}\ensuremath{b}\ensuremath{\overline{b}}\xspace}
%\def\eRpLbb{\ensuremath{e_R^{\mbox{\scriptsize -}}}\ensuremath{e_L^{\mbox{\scriptsize +}}}\ensuremath{\rightarrow}\ensuremath{b}\ensuremath{\overline{b}}\xspace}
%\def\eLpRcc{\ensuremath{e_L^{\mbox{\scriptsize -}}}\ensuremath{e_R^{\mbox{\scriptsize +}}}\ensuremath{\rightarrow}\ensuremath{c}\ensuremath{\overline{c}}\xspace}
%\def\eRpLcc{\ensuremath{e_R^{\mbox{\scriptsize -}}}\ensuremath{e_L^{\mbox{\scriptsize +}}}\ensuremath{\rightarrow}\ensuremath{c}\ensuremath{\overline{c}}\xspace}
%\def\eLpRqq{\ensuremath{e_L^{\mbox{\scriptsize -}}}\ensuremath{e_R^{\mbox{\scriptsize +}}}\ensuremath{\rightarrow}\ensuremath{q}\ensuremath{\overline{q}}\xspace}
%\def\eRpLqq{\ensuremath{e_R^{\mbox{\scriptsize -}}}\ensuremath{e_L^{\mbox{\scriptsize +}}}\ensuremath{\rightarrow}\ensuremath{q}\ensuremath{\overline{q}}\xspace}
\def\cme{\ensuremath{c.m.e.}\xspace}
\def\eLpR{\ensuremath{e_L^{-}e_R^{+}}\xspace}
\def\eRpL{\ensuremath{e_R^{-}e_L^{+}}\xspace}
\def\eLpRqq{\ensuremath{e_L^{-}e_R^{+}\rightarrow q\bar{q}}\xspace}
\def\eRpLqq{\ensuremath{e_R^{-}e_L^{+}}\rightarrow q\bar{q}\xspace}


%\def\eLpR{\ensuremath{e_L}\ensuremath{p_R}\xspace}
%\def\eRpL{\ensuremath{e_R}\ensuremath{p_L}\xspace}
\def\dEdx{\ensuremath{dE/\/dx}\xspace}
\def\dNdx{\ensuremath{dN/\/dx}\xspace}
\def\Afbb{\ensuremath{A^{b\bar{b}}_{FB}}\xspace}
\def\AFBb{\ensuremath{A^{b\bar{b}}_{FB}}\xspace}
\def\Afb{\ensuremath{A_{FB}}\xspace}
\def\AFB{\ensuremath{A_{FB}}\xspace}

\def\ALR{\ensuremath{A_{LR}}\xspace}
\def\Rb{\ensuremath{R_{b}}\xspace}
\def\Rc{\ensuremath{R_{c}}\xspace}
\def\Rq{\ensuremath{R_{q}}\xspace}
\def\Rqp{\ensuremath{R_{q\prime}}\xspace}
\def\Ruds{\ensuremath{R_{uds}}\xspace}
\def\Rbcostheta{\ensuremath{R_{b}(|cos\theta_{b}|)}\xspace}
\def\Rccostheta{\ensuremath{R_{c}(|cos\theta_{c}|)}\xspace}
\def\Rqcostheta{\ensuremath{R_{q}(|cos\theta_{q}|)}\xspace}
\def\Rudscostheta{\ensuremath{R_{uds}(|cos\theta_{uds}|)}\xspace}
\def\Afbc{\ensuremath{A^{c\bar{c}}_{FB}}\xspace}
\def\Afbq{\ensuremath{A^{q\bar{q}}_{FB}}\xspace}
\def\AFBc{\ensuremath{A^{c\bar{c}}_{FB}}\xspace}
\def\AFBq{\ensuremath{A^{q\bar{q}}_{FB}}\xspace}
\def\eett{\ensuremath{e^{\mbox{\scriptsize +}}}\ensuremath{e^{\mbox{\scriptsize -}}}\ensuremath{\rightarrow}\ensuremath{t}\ensuremath{\overline{t}}\xspace}
\def\bquark{\ensuremath{b}-quark\xspace}
\def\bjet{\ensuremath{b}-jet\xspace}
\def\bjets{\ensuremath{b}-jets\xspace}
\def\btagging{\ensuremath{b}-tagging\xspace}
\def\btag{\ensuremath{b_{tag}}\xspace}
\def\cquark{\ensuremath{c}-quark\xspace}
\def\cjet{\ensuremath{c}-jet\xspace}
\def\cjets{\ensuremath{c}-jets\xspace}
\def\ctagging{\ensuremath{c}-tagging\xspace}
\def\ctag{\ensuremath{c_{tag}}\xspace}
\def\udsjet{\ensuremath{uds}-jet\xspace}
\def\udsjets{\ensuremath{uds}-jets\xspace}
\def\tquark{\ensuremath{t}-quark\xspace}
\def\Zboson{\ensuremath{Z}-boson\xspace}
\def\Zpole{\ensuremath{Z}-pole\xspace}
\def\Zprime{\ensuremath{Z^{\prime}}\xspace}
\def\Zbb{\ensuremath{Z_{b\bar{b}}}\xspace}
\def\ZbRbR{\ensuremath{Z_{b_{R}\bar{b}_R}}\xspace}
\def\ZbLbL{\ensuremath{Z_{b_{L}\bar{b}_L}}\xspace}
\def\Bc{\ensuremath{Vtx}-method\xspace}
\def\Kc{\ensuremath{K}-method\xspace}
\def\BcKc{\ensuremath{Vtx/K}-method\xspace}
\def\BcKcsame{\ensuremath{Vtx/K_{same\,jet}}-method\xspace}
\def\BcBc{\ensuremath{Vtx/Vtx}-method\xspace}
\def\KcKc{\ensuremath{K/K}-method\xspace}

\def\Pb{\ensuremath{P_{chg.}}\xspace}
\def\Qb{\ensuremath{Q_{chg.}}\xspace}
%\def\PbB{\ensuremath{P_{chg.,Vtx}}\xspace}
%\def\PbK{\ensuremath{P_{chg.,K}}\xspace}
%\def\QbB{\ensuremath{q_{chg.,Vtx}}\xspace}
%\def\QbK{\ensuremath{q_{chg.,K}}\xspace}

\def\PbB{\ensuremath{P_{chg.,M_{1}}}\xspace}
\def\PbK{\ensuremath{P_{chg.,M_{2}}}\xspace}
\def\QbB{\ensuremath{Q_{chg.,M_{1}}}\xspace}
\def\QbK{\ensuremath{Q_{chg.,M_{2}}}\xspace}

\def\Pbi{\ensuremath{P_{chg.,i}}\xspace}
\def\Pbj{\ensuremath{P_{chg.,j}}\xspace}
\def\Qbi{\ensuremath{Q_{chg.,i}}\xspace}
\def\Qbj{\ensuremath{Q_{chg.,j}}\xspace}


\def\pmp{\ensuremath{+-}\xspace}
\def\mpp{\ensuremath{-+}\xspace}
\def\pp{\ensuremath{++}\xspace}
\def\mm{\ensuremath{--}\xspace}

\newcommand\blankpage{%
    \null
    \thispagestyle{empty}%
    \addtocounter{page}{-1}%
    \newpage}


\title{Experimental prospects for precision observables in \eeqq with $q=b,c$ processes at the ILC operating at 250 and 500 GeV of center of mass.}
% Uncomment this line to remove the stamp with the ILD note number from the top right corner
% of the title page
\titlecomment{This work was carried out in the framework of the ILD concept group}
\titlecomment{Talk presented at the International Workshop on Future Linear Colliders (LCWS 2023), 15-19 May 2023. C23-05-15.3.}

% Set the ILD note number
% Numbering convention:
% TTTT = topic (phys, soft or tech)
% YYYY = year
% NNN = number
% Papers and topical papers:
%\ildphys{YYYY}{NNN} % Physics 
%\ildsoft{YYYY}{NNN} % Software
%\ildtech{YYYY}{NNN} % Technical
% ILD notes and conference proceedings:
%\ildpublic{PHYS}{2023}{NNN}% Public note 
%\ildinternal{TTTT}{YYYY}{NNN} % Internal note 
\ildproc{PHYS}{2023}{004, IFIC/23-37} % Proceedings% Example:
%\ildphys{2023}{NNN}
%\ildpublic{phys}{2021}{001}
%\ildproc{phys}{2021}{001}
% ILD numbering convention:
% https://confluence.desy.de/display/ILD/ILD+Publication+and+Speakers+Bureau

% Set the publication date
\date{\today}
%\date{\formatdate{18}{6}{2014}}


% Define the authors and their institutes, they will appear exactly in the order as they are added
% Footnotes can be added using the \thanks command

\addauthor{A. Irles}{\institute{1} \footnote{\href{mailto:adrian.irles@ific.uv.es}{adrian.irles@ific.uv.es}}}
\addauthor{J.P. M\'arquez}{\institute{1} 
\footnote{Speaker \href{mailto:jesus.marquez@ific.uv.es}{jesus.marquez@ific.uv.es}}}


\addinstitute{1}{IFIC, Universitat de Val\`encia and CSIC, C./ Catedr\'atico Jos\'e Beltr\'an 2, E-46980 Paterna, Spain}

% Add "On behalf of ... (optional)"
%\onbehalfof{the ILD detector concept group}

% Define an abstract for the note 
\abstract{
Future Higgs Factories will allow the precise study of \eeqq with $q=s,c,b,t$ interactions at different energies, from the Z-pole up to high energies never reached before.
In this contribution, we will discuss the experimental prospects for the measurement of differential observables in \eebb and \eecc processes at high energies, 250 and 500 GeV, using full simulation samples and the full reconstruction chain from the ILD concept group.
These processes call for superb primary and secondary vertex measurements, a high tracking efficiency to correctly measure the vertex charge and excellent hadron identification capabilities using \dEdx. This latter aspect will be discussed in detail together with its implementation within the standard flavour tagging tools developed for ILD (LCFI+). In addition, prospects associated with potential improvements using cluster counting techniques instead of traditional \dEdx will be discussed. %Finally, we will briefly discuss the potential of the discovery of BSM models such as Randall-Sundrum models with warped extra dimensions, profiting from measurements of $b/c$-quark related observables at different beam energies and polarisations.
}


%\notitlestamp

%============================================%
% Bibliography
%============================================%
% define the list of bibliography data files
\addbibresource{./references.bib}
%============================================%
% Search path for images
%============================================%
\graphicspath{ {./logos/}{./figures/} }

%============================================%
% Start of the actual document
%============================================%
\begin{document}

% generates the title page
\titlepage

\tableofcontents


%\section{Introduction}
%\label{sec:intro}
\section{Introduction}

% Figure environment removed

Reinforcement Learning from Human Feedback (RLHF) has recently been used to great effect to align pretrained large language models (LLMs) to human preferences, optimizing for desirable qualities like harmlessness and helpfulness~\citep{bai2022training} and achieving state-of-the-art results across a variety of natural language tasks~\citep{openai2023gpt4}. %RLHF approaches fundamentally rely on collecting pairs of LLM outputs $(o_1, o_2)$ from a shared prompt $p$, with a human indicating which output in each pair is better on a specified attribute.
% A fundamental component of RLHF is a preference model derived from human labels, typically formatted as pairs of LLM outputs $(o_1, o_2)$ generated from a shared prompt $p$.

A standard RLHF procedure fine-tunes an initial unaligned LLM using an RL algorithm such as PPO~\citep{schulman2017proximal}, optimizing the LLM to align with human preferences. %\violet{not sure whether we need to provide this detail in the intro, especially this has nothing to do with our contribution.} % i feel like this context is useful later when e.g. explaining that context distillation is SFT
RLHF is thus critically dependent on a reward model derived from human-labeled preferences, typically \textit{pairwise preferences} on LLM outputs $(o_1, o_2)$ generated from a shared prompt $p$. % and labeled by humans. 

However, collecting human pairwise preference data, especially high-quality data, may be expensive and time consuming at scale. To address this problem, approaches have been proposed to obtain labels without human annotation, such as Reinforcement Learning from AI Feedback (RLAIF) and context distillation. 

\iffalse
raising the question of whether we can generate high-quality data for RLHF without using human labeling. %accurately-labeled preference pairs $(o_1, o_2)$
%, motivating model alignment approaches that aim to generate accurately-labeled preference pairs $(o_1, o_2)$ without human involvement. 
Two major categories of such approaches are . 
\fi

RLAIF approaches (e.g.,~\citet{bai2022constitutional}) simulate human pairwise preferences by scoring $o_1$ and $o_2$ with an LLM (Figure \ref{fig:rlcd_differences} center); the scoring LLM is often the same as the one used to generate the original pairs $(o_1, o_2)$. Of course, the resulting LLM pairwise preferences will be somewhat noisier compared to human labels. However, this problem is exacerbated by using the same prompt $p$ to generate both $o_1$ and $o_2$, causing $o_1$ and $o_2$ to often be of very similar quality and thus hard to differentiate (e.g., Table~\ref{tab:rlaif_bad_example}). Consequently, training signal can be overwhelmed by label noise, yielding lower-quality preference data. 

% While it avoids human labeling efforts, it has weakness. First, LLM preference labels will naturally be somewhat noisier compared to human labels. Furthermore, since the same prompt $p$ is used to generate both $o_1$ and $o_2$, their quality is often very similar and hard to differentiate (See Table~\ref{tab:rlaif_bad_example}). As a result, training signals can be overwhelmed by label noise, yielding lower-quality preference data. 

Meanwhile, context distillation methods (e.g., \citet{sun2023principle}) create more training signal by modifying the initial prompt $p$. 
%to create more significant training signal. 
The modified prompt $p_+$ typically contains additional context encouraging a \textit{directional attribute change} in the output $o_+$ (Figure \ref{fig:rlcd_differences} right). However, context distillation methods only generate a single output $o_+$ per prompt $p_+$, which is then used for supervised fine-tuning, losing the pairwise preferences which help RLHF-style approaches to 
%rather than using a RLHF-style preference model to 
derive signal from the contrast between outputs. 
Multiple works have observed that RL approaches using preference models for pairwise preferences can substantially improve over supervised fine-tuning by itself when aligning LLMs~\citep{ouyang2022training,dubois2023alpacafarm}. 

% conduct alignment by running supervised fine-tuning on model outputs $o_+$ generated from a modified prompt $p_+$. $p_+$ typically contains additional context encouraging desirable attributes (Figure \ref{fig:rlcd_differences} right), such as in \citet{sun2023principle}. However, multiple works have observed that RLHF-style approaches can substantially improve over supervised fine-tuning by itself when aligning LLMs~\citep{ouyang2022training,dubois2023alpacafarm}. 

Therefore, while both RLAIF and context distillation approaches have already been successfully applied in practice to align language models, we posit that it may be even more effective to combine the key advantages of both. That is, we will use RL with \textit{pairwise preferences}, while also using modified prompts to encourage \textit{directional attribute change} in outputs. %In particular, we will adapt the RLAIF data generation process with two different prompts rather than a single $p$, modifying both prompts similarly to context distillation. %\violet{this motivation is a little unexciting. I think we can more specifically discuss the potential benefits of our approach, like the benefits from RL: exploration/data generation; benefits from contrast. I don't think we get too much benefits from context distillation since we switched to the RL framework.} 

Concretely, we propose \oursfull{} (\ours{}). 
\ours{} generates preference data as follows. Rather than producing two i.i.d.\ model outputs $(o_1, o_2)$ from the same prompt $p$ as in RLAIF, \ours{} creates two variations of $p$: a \textit{positive prompt} $p_+$ similar to context distillation which encourages directional change toward a desired attribute, and a \textit{negative prompt} $p_-$ which encourages directional change \textit{against} it (Figure \ref{fig:rlcd_differences} left). We then generate model outputs $(o_+, o_-)$ respectively, and automatically label $o_+$ as preferred---that is, \ours{} automatically ``generates'' pairwise preference labels by construction. %, without further post hoc labeling.\violet{should make it clearer that our approach `generates' labels by construction} 
We then follow the standard RL pipeline of training a preference model followed by PPO. 

Compared to RLAIF-generated preference pairs $(o_1, o_2)$ from the same input prompt $p$, there is typically a clearer difference in the quality of $o_+$ and $o_-$ generated using \ours{}'s directional prompts $p_+$ and $p_-$, which may result in less label noise. %which may result in better training signal for the preference model. 
That is, intuitively, \ours{} exchanges having examples be \textit{closer to the classification boundary} for much more \textit{accurate labels} on average. Compared to standard context distillation methods, on top of leveraging pairwise preferences for RL training, \ours{} can derive signal not only from the positive prompt $p_+$ which improves output quality, but also from the negative prompt $p_-$ which degrades it. %\ours{} is not learning to imitate $o_+$, but to distill the \textit{contrast} between $o_+$ and $o_-$. 
Positive outputs $o_+$ don't need to be perfect; they only need to contrast with $o_-$ on the desired attribute while otherwise following a similar style.

% \todo{discuss our method and why intuitively it may be better.}

We evaluate the practical effectiveness of \ours{} through both human and automatic evaluations on three tasks, aiming to improve the ability of LLaMA-7B~\citep{touvron2023llama} to generate harmless outputs, helpful outputs, and high-quality story outlines. %\ours{} outperforms both RLAIF and context distillation baselines in pairwise comparisons on 
As shown in Sec. \ref{sec:experiments}, \ours{} substantially outperforms both RLAIF and context distillation baselines in pairwise comparisons when simulating preference data with LLaMA-7B, while still performing equal or better when simulating with LLaMA-30B. 
%On all three tasks, \ours{} substantially outperforms both RLAIF and context distillation baselines in pairwise comparisons---by a margin of at least 9\% and often more than 30\%---validating our method's efficacy. 
We will release all code at a later date, although in any case \ours{} is fairly easy to implement by modifying any reference RLAIF codebase. %We release all code at \todo{github link}.

%\section{The ILD Concept and MC Samples}
%\label{sec:ILDMC}
\section{The ILD Concept and MC Samples}
\label{sec:ILD}

%0.5 pages.
%Describe here the ILD detector (tracking system in particular, adding the description of a potential ILD with a pixel TPC), the samples, the reconstruction tools etc.
%silicon tracking systems (silicon internal tracking, SIT, and forward tracking detector, FTD)
%Mention here the tracking software and the LCFIplus.\\
The ILD subdetector layout consists of a high-precision vertex detector (VTX), silicon tracking systems, a time projection chamber (TPC), a highly granular calorimeter system (ECAL and HCAL) and a muon catcher. All the aforementioned subdetectors are placed inside a solenoid providing a magnetic field of $3.5$\,T, surrounded by an iron yoke instrumented for muon detection. The ILD TPC \cite{thelctpccollaboration2016time,LCTPC:2022pvp} is a large volume time projection chamber that allows continuous 3D tracking and particle identification. %, usually based on energy loss (\dEdx). 
Its baseline design consists of a barrel-shaped structure with an inner radius of 329 mm and an outer radius larger than 1808 mm. 
It provides a single-point resolution of 100 \textmu m over about 200 readout points and a \dEdx resolution of $\sim4.5\%$. It is based on $1\times 6$ mm$^{2}$ pads readout with GEM or MicroMegas technologies.
This document will also explore the potential of an alternative approach: a pixel TPC. Several technological solutions are under study, and all envision a four times greater readout density with a pixel size of 300 \textmu m. Simulations extrapolating beam test results show that an improved relative resolution of $\sim3$-$4\%$ will be feasible using cluster counting techniques (\dNdx) instead of the traditional approach \cite{LCTPC:2022pvp}. 

This study has been conducted running full simulation via \texttt{ILCSOFT} \textit{v02-02-03}\footnote{Link: \url{https://github.com/iLCSoft}}, which merge different software packages and algorithms that operate in a modular way from MC events to final reconstruction. All simulations use the ILD-L\cite{ILD:2020qve} model, whose geometry, material and readout systems are implemented in the \texttt{DD4HEP} framework\cite{Frank:2014zya}, interfaced with \texttt{Geant4} toolkit. Both signal and background events are generated with the \texttt{WHIZARD}\cite{Kilian:2007gr} event generator at LO. The beam energy spectrum, beam-beam interaction and QED ISR is generated via \texttt{Guinea-Pig}\cite{Schulte:1999tx}. Non-perturbative effects, such as the FSR and hadronization, are provided by \texttt{Pythia} event generator\cite{Sj_strand_2006}.

To reproduce different stages of the ILC operation, two different samples have been provided by the ILD concept group; one for 250 GeV and one for 500 GeV. Each set of samples features full polarised beams in different configurations. The main samples for this study are the $\mathrm{e^-_{L}}\mathrm{e^+_{R}}$ and $\mathrm{e^-_{R}}\mathrm{e^+_{L}}$. Background processes from electroweak bosons, Higgs and top-quark production (at 500 GeV) are studied using additional MC samples. Backgrounds from the production of lepton pairs have been ignored since they are expected to be easily identified.
All distributions and results have been reweighted to match the baseline polarization and full luminosity scenarios for ILC250 and ILC500, so-called the H20 scenario \cite{Bambade:2019fyw}.

The generation of the \eeqqbar signal events and the \eeZgammaqqbar with the Z-boson generated on-shell are done simultaneously. Following the same recipe as in \cite{Irles:2023ojs}, we define the signal as the events with acolinearity of the \qqbar system (at parton level) smaller than 0.3 and a large invariant mass of the \qqbar pair larger than 140 GeV - for the ILC250 case - or larger than 200 GeV for the ILC500 case  (i.e. large enough to be away of the Z-mass peak).

The signal and background cross sections are listed in Tables 1 and 2 from \cite{Irles:2023ojs} for ILC250 and in Tables \ref{tab:crosssection500} and \ref{tab:crosssection_bkg500} for ILC500. Only backgrounds leading to fully hadronic final states are considered. Backgrounds involving leptons in the final states are ignored since those are expected to be easily identified.

\begin{table}[!ht]
  \centering
  \begin{tabular}{c|ccc|ccc}
    \hline
     & \multicolumn{3}{|c|}{ $\sigma_{\eeqqbar}$[fb]} & \multicolumn{3}{|c}{ Radiative Return BKG [fb]} \\
    \hline
    Polarisation & \bbbar & \ccbar & \qqbar ($q=uds$) & \bbbar & \ccbar & \qqbar ($q=uds$) \\
    \hline
    \eLpR & 611.4 & 1545.9 & 2770.2 & 5506.4 & 5118.0 & 16244.2\\
    \eRpL & 416.9 & 893.8 & 1728.3 & 3002.5 & 2766.9 & 8773.5 \\
    \hline
  \end{tabular}
  \caption{Production cross section of signal and background originated by di-boson production. \label{tab:crosssection500}}
\end{table}

\begin{table}[!ht]
 %\renewcommand{\arraystretch}{1.6}
 \centering
  \begin{tabular}{c|c|c|c|c}
    \hline
    & \multicolumn{4}{|c}{  $\sigma_{e^{-}e^{+}\rightarrow\,X}$ [fb]} \\
    \hline
	Polarisation &  $X= WW \rightarrow q_{1} \bar{q_{2}} q_{3} \bar{q_{4}}$ & $X= ZZ \rightarrow q_{1} \bar{q_{1}}q_{2} \bar{q_{2}}$ & $X= HZ \rightarrow q\bar{q}H$ & $X=t\bar{t} \rightarrow b\bar{b}q_{1}\bar{q}_{2}q_{3}\bar{q}_{4}$\\
    \hline
    \eLpR & 7680.0 & 680.2 & 114.7 & 660 \\
    \eRpL & 33.5 & 271.9 & 73.4 & 254.8 \\
      \hline
  \end{tabular}
  \caption{\label{tab:crosssection_bkg500} Cross sections at 250 GeV for processes producing at least one pair of $q$-quarks using fully polarised beams. }
\end{table}

%\section{High-level Reconstruction and Event Selection}
%\label{sec:recosel}
\section{\Rq and \AFB reconstruction at ILD}
\label{sec:recosel}
The reconstruction of the events begins with the track reconstruction performed by the \texttt{MarlinTrk} framework, part of \texttt{ILCSoft}. 
Later, \texttt{Pandora}\cite{Marshall:2015rfa} runs the particle flow algorithm (PFA) that matches the tracking information with the high-granular calorimetry information following pattern recognition techniques. The resulting reconstructed objects are denominated particle flow objects (PFO) and are treated as single particles.

Once the PFOs are reconstructed, the vertex reconstruction, jet reconstruction and flavour tagging are performed with \texttt{LCFI+} software tool.

Once all high-level reconstruction objects are obtained, the event preselection is performed. It is based on a series of kinematical cuts applied to enrich the data sample with signal events while removing the backgrounds. The preselection for 250 GeV is the same as described in \cite{Irles:2023ojs}, where the variables are described, with minor adaptations made for the study of the 500 GeV samples (noted in parenthesis): 
\begin{enumerate}
    \item photon veto cuts, rejecting events if:
    \begin{enumerate}
        \item at least one of the jets contains only one PFO;
        \item at least one of the jets contains a reconstructed $\gamma_{cluster}$ with E > 115 GeV (220 GeV);
    \end{enumerate}
    \item events with $\sin \Psi_{acol}$ > 0.3 are rejected;
    \item events with $m_{jj}$ < 140 GeV (200 GeV) are rejected;
    \item events with $y_{23}$ > 0.02 (0.007) are rejected.
\end{enumerate}
The most noticeable change for the ILC500 case is performed in the last cut, which is tightened to reduce the much larger $WW$ background contamination in the case of left-handed polarisation. This last cut decreases the final preselection efficiency to $\sim50\%$ instead of $\sim75\%$ for the ILC250 case.

The next step is applying the Double Tag (DT) method stated in \cite{Irles:2023ojs} by which a cut is performed in the \btag (or \ctag) likelihood in both jets of the event. The working points are defined as in \cite{Irles:2023ojs} such that the mistagging of the other quarks when performing the tagging of \bquark (or \cquark) is smaller than the $\sim1.5\%$ $(\sim 3\%)$, assuming the same size for the $b/c/uds$ samples. The DT yields a high-purity selection of the events for each flavour and is the last necessary step to reconstruct $R_q$ measurements.

To study \AFB, the charge of the jets needs to be reconstructed to fully determine the direction of the $q\bar{q}$ system. As described in \cite{Irles:2023ojs}, two independent methods are used for the charge measurement: \Bc or \Kc. The \Bc counts the charge of all tracks in secondary vertices in the jet, while the \Kc only uses the information of tracks identified as Kaons from the TPC dE/dx PID. The Double Charge (DC) method performs the final charge measurement, which combines the possible measurements of both methods and accepts only jets with opposite charges. 
The differential, in \costheta, cross section is reconstructed, and the \AFB extracted from a fit restricted to ($\|\costheta\|>0.9$) to avoid the region where the reconstruction efficiency drops due to insufficient acceptance of the tracking detectors. %The last part of the reconstruction of \AFB is an extrapolation to such regions by fitting the data in the -0.9 < \costheta < 0.9 region.

All the methods (Preselection, DT, DC, Fit) have been developed for the analysis at 250 GeV and are explained in detail in \cite{Irles:2023ojs} together with a comprehensive study of the most dominant systematic uncertainties.
The same methods are used for the study at 500 GeV shown in this document.
In Appendix \ref{sec:appendixPlots}, we collect the most relevant performance plots for the ILC500 case to be compared with the ones in \cite{Irles:2023ojs}.

%\begin{itemize}
%    \item Describe, very briefly, the methods described in the ILD note.
%    \item Focus on the Flavour tagging and jet-charge measurement methods.
%    \item Adaptations required to analyze 500GeV
%\end{itemize}



\section{Optimisation of the $\Afb$ reconstruction at ILC}
\label{sec:optimization}
%2 pages. (3 as super max)
\subsection{Optimisation of the default flavour tagging of ILD: Adding hadron PID as input.} \label{PIDsection}

The flavour tagging is performed by the \texttt{LCFI+} package \cite{Suehara:2015ura}.
The vertex reconstruction method distinguishes between vertices and pseudo-vertices, which are \textit{single-track vertices} candidates to be merged in a fully reconstructed vertex in a second iteration of the algorithm. At the end of the process, if these pseudo-vertices are not merged in a full vertex, they are kept in the list of objects and treated as \textit{single-track vertices}.
%Promoting a track to a single-track pseudo-vertex is decided by the algorithm with statistical significance criteria based on impact parameters ($d_0,z_0$) and the vertex displacement values expected for \bquark and \cquark subproducts (as described in \cite{Suehara:2015ura}, page 6).
%Generally, the single-track vertex is meant to represent when \bjets containing only one reconstructed secondary vertex have a possible track that could be interpreted as the result of additional secondary decay. 

The total number of vertices and pseudo-vertices is always two or fewer. Four different categories are defined (A, B, C, D): A are jets without vertices and up to two pseudo-vertices, B are jets with one vertex and no pseudo-vertices, C are jets with one vertex, and one pseudo-vertex and D are jets with two vertices. The flavour content in each category is different as expected due to the different hadronisation products of heavy quarks: Above 95\% of light-quark jets are in category A, more than 90\% of \cquark jets are spread between categories A and B while more than 80\% of the \bquark jets are split between categories B, C and D.

To prepare weights for flavour tagging via \texttt{LCFI+}, two main algorithms have to be run: \textit{MakeNTuple} and \textit{TrainMVA}. MakeNTuple prepares the \texttt{ROOT} files with NTuples with all the different variables that could be used for the flavour tagging. TrainMVA uses 3 NTuples as input (one for \bjets, one for \cjets and one for \udsjets) and then runs a classificator based on Boosted Decision Trees (BDT implemented in \texttt{ROOT}'s TMVA). The input is the signal data separated into the different categories described above. A \texttt{Marlin} processor\footnote{Link: \url{https://github.com/marherje/LCIO_Extraction.git}} was designed \textit{ad hoc} to do this selection. 

New variables were built using the TPC's PID capabilities via \dEdx to improve the flavour tagging. The new variables were a count of kaons, protons and pion candidates. The charged kaon, proton and pion IDs are defined by constructing \textit{likenesses} variables using a statistical distance measurement, comparing theoretical and experimental values. for instance, the definition of \textit{kaonness}, \kaonness, is defined as:
\begin{equation}
\Delta_{\dEdx-K}=\left(\frac{dE/dx_{exp}-dE/dx_{K,BB}}{\Delta dE/dx_{exp}}\right),
\label{formula:kaonness}
\end{equation}
where $dE/dx_{K,BB}$ is the theoretical value given by the Bethe-Bloch formula, $dE/dx_{exp}$ is the experimental value from the TPC's measurements and $\Delta dE/dx_{K,exp}$ is the error associated to that measurement as implemented in the official ILD reconstruction software. This definition is extended to other hadron types (\textit{pionness}, \textit{protonness}). This distance can be modelled as normally distributed. It can be constructed concerning each particle's theoretical values, hence having the Gaussian centred in 0 for each case (\textit{kaonness}, \textit{pionness}, \textit{protonness}). To have a proper distribution, all tracks with momenta below 3 GeV and $\|\cos\theta\| > 0.95$ were discarded before selecting particle candidates. Low-momenta tracks are removed because the Bethe-Bloch curves at low energies for pions and kaons overlap. Removal of very forward/backward tracks is for a loss of geometrical acceptance of the TPC. We count as candidate particles of each type those whose absolute \textit{likeness} value is below 1.5, which minimises overlapping with other particles. All these new variables and selections were implemented in a beta version of \texttt{LCFI+} that is already available and waiting for an official \textit{pull} to the main \texttt{GitHub} repository of \texttt{LCFI+}. This process counts only tracks associated with a secondary vertex (or pseudo-vertex).%, counting tracks in the primary vertex was also explored, but it is not useful. 
The final three new variables added to \textit{MakeNTuple} are named: dEdxNKaonSec, dEdxNProtonSec and dEdxNPionSec, as they represent the number of each particle in secondary vertices.

%Both the \bquark and \cquark hadronisation products can decay into charged kaons. 
There is a direct correlation between quark flavour and kaon production via B-mesons (\bquark) and D-mesons (\bquark and \cquark) decays which motivate using kaon PID in flavour tagging. Identification of kaons is also useful for the charge reconstruction as stated in Sec. \ref{sec:recosel}, which is especially important in the case of \cquark. Using \dEdx is strongly motivated given the expected separation power of high momenta tracks (above 3 GeV)(see \cite{ILD:2020qve}, Figure 8.6). 

The standard training process runs once through the four vertexing categories using the same configuration of the BDTs. Still, in the case of this optimisation, independent training has been performed for each category. To ensure that the BDT configuration is optimal for each different scenario (energy, category and variable selections), a Particle Swarm Optimisation (PSO) was run. 
The PSO\cite{488968} is a parameter-free, stochastic, bio-inspired algorithm that searches for the optimal configuration of a problem by doing an iterative scanning of configurations for the problem.
The \textit{particles} are positions in the space of configurations for the given problem, which move to a new position after each iteration.
It requires the definition of a Function Of Merit (FOM) as scoring to determine the movement of the \textit{particles} after each iteration. 
In this case, a PSO software\cite{CMS:2018hnq} has been adapted and extended to a 3-class classifier with filters suitable to work with \texttt{LCFI+}\footnote{Link: \url{https://github.com/marherje/PSOforLCFIPlus.git}}. The space of configurations for flavour tagging are the hyper-parameters of the BDTs, and the \textit{particles} of the PSO are different selections of such hyper-parameters. The different hyper-parameters for the BDTs are the number of trees, the maximum number of leaves for each tree, the shrinkage, the bagging fraction and the number of bins (for the physical input variables' histograms). The function of merit of choice for this study was the average value of the integral of the Receiving Operating Characteristic (ROC) curve for the three different flavour categories ($b$, $c$, $uds$) from the test sample. To avoid overfitting to the test sample, two different statistical tests (Kolmogorov-Smirnov\cite{kolmogorov1933sulla,smirnov1948table,hodges1958significance} and Anderson-Darling\cite{cf37c5fc-d933-3771-9586-9d6b4b285d8b,doi:10.1080/01621459.1954.10501232,engmann2011comparing}) were run each time a particle finished its training, comparing the BDTs output between the test and training samples. In both statistical tests, a threshold value was set upon observation, always keeping the p-value over the standard $p>0.05$ criteria but also checking that the fluctuations in the FOM values were below 1.5 \%. The complete list of variables used in each category is in Table \ref{tab:bdtvariables} of the Appendix \ref{sec:annex}, where the bold text are the new variables and the rest are original \textit{LCFI+} variables as introduced in \cite{Suehara:2015ura}.

\subsection{Prospects of using a pixel TPC device: From \dEdx to \dNdx } \label{dNdxsection}
A better PID resolution is expected when using \dNdx instead of \dEdx. This will be possible with a pixel-based TPC. 
According to \cite{LCTPC:2022pvp}, an improvement of $\sim 30-40\%$ in the kaon/pion separation power is achievable for tracks with momentum between 3-50 GeV, with the separation power as defined in  \cite{ILD:2020qve} and in the following equation: 
\begin{equation}
    \eta_{A,B}(p)=\frac{\lvert\mu_A(p)-\mu_B(p)\rvert}{\sqrt{\frac{1}{2}(\sigma_A^2(p)+\sigma_B^2(p)}},
    \label{eq:separationpower}
\end{equation}
where $\mu_(p)$ and $\sigma$ are the mean and standard deviations of the gaussian fits to the experimental value of \dEdx, defined as a function of the momenta $p$ of each bin. 
However, the \dNdx reconstruction is not yet implemented in the ILD software, so it is not possible to introduce a \dNdx PID variable for the flavour tagging and charge measurement directly, as for \dEdx. 
Instead, we apply a correction to the \textit{likenesses} distributions by the expected value.
This is done by a dedicated Marlin processor that emulates a $25\%$ reduction in standard deviation in the 
\textit{likenesses} distributions.
This $25\%$ reduction in the  standard deviation in the 
\textit{likenesses} distributions is equivalent to a to a $\sim33\%$ improvement in the separation power,
In addition, in the flavour tagging case, an Ideal PID resolution has been considered by reducing the standard deviation of the simulated \textit{likeness} distributions by a $99\%$, compared with the full simulation.

% Figure environment removed

Once the expected improvement in resolution for the \dNdx case is accounted for, the full analysis is repeated. Given that now the \textit{likeness} distributions offer a better separation between types of hadrons (see Fig.\ref{fig:dEdxdist_cquark}), the criteria to accept a track as kaon candidate requires an absolute value of the \textit{kaonness} smaller than 1.0 instead of 1.5. Besides the separation criteria, the optimisation process is the same, including the PSO for each possible scenario. In Fig.\ref{fig:flavourtagging}, the effects on flavour tagging when adding the three new variables from kaon, protons and pions are shown for the case of \dEdx,\dNdx and ideal PID with 100\% efficiency for both 250 and 500 GeV. In Fig.\ref{fig:kaonID}, the impact of moving from \dEdx to \dNdx, envisioning a pixel PID, is shown for 250 and 500 GeV.


% Figure environment removed

% Figure environment removed

%This has to be 1 page only too. Maybe 1.5
%\begin{itemize}
%    \item The pixel TPC should have been described before.
%    \item Explain how we estimate the better dEdx resolution.
%    \item Plot of the efficiency vs purity (b/c and 250GeV and 500GeV --> Adrian)
%\end{itemize}





\section{Experimental prospects for \Afb  and \Rq at ILC250 and ILC500}
\label{sec:results}

Three different scenarios have been studied: reconstruction without TPC kaon ID, a reconstruction using TPC Kaon ID (via \dEdx) for charge measurement as well as adding \dEdx in the flavour tagging and a reconstruction using TPC Kaon ID (via \dNdx) for charge measurement as well as adding \dNdx in the flavour tagging, being the later an estimation as described in section \ref{dNdxsection}. These three scenarios are covered for both 250 and 500 GeV, for the cases of $P_{\mathrm{e^{-}e^{+}}}=(-0.8,+0.3)$ and $P_{\mathrm{e^{-}e^{+}}}=(+0.8,-0.3)$. A comparison for each case using only statistical uncertainties has also been plotted. The results are summarised in Fig. \ref{fig:results}.


% Figure environment removed





%\section{Conclusion}
%\label{sec:concl}
\section{Summary and prospects}
\label{sec:conclusions}
%New text
In this document, an update of the previous work at ILC250 \cite{Irles:2023ojs} has been done, extending it to ILC500. 
Furthermore, a detailed study of the impact of the TPC PID capabilities at ILD in both scenarios has been covered, including prospects of using cluster counting (\dNdx) with a pixel TPC.
Adding PID from TPC measurements into the flavour tagging software for $b$ and \cquark only provides a residual improvement since the main flavour-tagging variables are related to the vertex reconstruction. However, the TPC PID plays a key role in the reconstruction of \AFB. Prospects of a pixel TPC using \dNdx for PID envision a noticeable impact in the charge reconstruction of the \qqbar systems, especially for the \cquark case, where a factor $\sim2$ improvement in the experimental precision is obtained.

%beta text
Ongoing work by the authors of this note is the evaluation of experimental sensitivity of BSM models predicting new heavy resonances $Z^{\prime}$\cite{Yoon:2018xud,Funatsu:2017nfm,Funatsu:2020haj}, which could affect the predictions of \Rq and \AFB at ILC250/500. 

The ILC program also envisions a run at the Z-pole, the GigaZ mode\cite{irles2019complementarity}. 
With that run. one would expect at least one order of magnitude of improvements for the Z-couplings measurements of \bquark and \cquark compared to those of SLC and LEP and give a complete picture of the quark EW couplings when combining measurements at Giga-Z, ILC250 and ILC500. However, since a study at the Giga-Z would involve the development of new simulations and methodologies, it is considered out of this work's scope. 

%\afterpage{\blankpage}


\section*{Acknowledgements}
We would like to thank the LCC generator working group and the ILD software working group for providing the simulation and reconstruction tools and producing the Monte Carlo samples used in this study.
This work has benefited from computing services provided by the ILC Virtual Organization, supported by the national resource providers of the EGI Federation and the Open Science GRID.

The authors are funded by the Generalitat Valenciana (Spain) under the grant number CIDEGENT/2020/21. AI also acknowledges the financial support from the MCIN with funding from the European Union NextGenerationEU and Generalitat Valenciana in the call Programa de Planes Complementarios de I+D+i (PRTR 2022) Project \textit{Si4HiggsFactories}, reference ASFAE$/2022/015$.

\section{Appendix}

\subsection{Flavour tagging variables}
\label{sec:annex}
\begin{table}[ht]
\centering
\begin{tabularx}{\linewidth}{|c|L|}
\hline
Cat. & Variables                                                                                  \\ \hline
A    & trk1d0sig trk2d0sig trk1z0sig trk2z0sig trk1pt\_jete trk2pt\_jete jprobr2 jprobr25sigma jprobz2 jprobz25sigma d0bprob2 d0cprob2 d0qprob2 z0bprob2 z0cprob2 z0qprob2 \textbf{dEdxNKaonSec} \textbf{dEdxNPionSec} \textbf{dEdxNProtonSec} \\ \hline
B    & trk1d0sig trk2d0sig trk1z0sig trk2z0sig trk1pt\_jete trk2pt\_jete jprobr2 jprobz2
      vtxlen1\_jete vtxsig1\_jete vtxdirang1\_jete vtxmom1\_jete vtxmass1 vtxmult1 vtxmasspc vtxprob
      d0bprob2 d0cprob2 d0qprob2 z0bprob2 z0cprob2 z0qprob2
      trkmass nelectron nmuon
      \textbf{dEdxNKaonSec} \textbf{dEdxNPionSec} \textbf{dEdxNProtonSec}  \\ \hline
C    &   trk1d0sig trk2d0sig trk1z0sig trk2z0sig trk1pt\_jete trk2pt\_jete jprobr2 jprobz2
      vtxlen1\_jete vtxsig1\_jete vtxdirang1\_jete vtxmom1\_jete vtxmass1 vtxmult1 vtxmasspc vtxprob
      1vtxprob vtxlen12all\_jete vtxmassall
      \textbf{dEdxNKaonSec} \textbf{dEdxNPionSec} \textbf{dEdxNProtonSec} \\ \hline
D    & trk1d0sig trk2d0sig trk1z0sig trk2z0sig trk1pt\_jete trk2pt\_jete jprobr2 jprobz2
      vtxlen1\_jete vtxsig1\_jete vtxdirang1\_jete vtxmom1\_jete vtxmass1 vtxmult1 vtxmasspc vtxprob
      vtxlen2\_jete vtxsig2\_jete vtxdirang2\_jete vtxmom2\_jete vtxmass2 vtxmult2
      vtxlen12\_jete vtxsig12\_jete vtxdirang12\_jete vtxmom\_jete vtxmass vtxmult
      1vtxprob 
      \textbf{dEdxNKaonSec} \textbf{dEdxNPionSec} \textbf{dEdxNProtonSec} \\ \hline
\end{tabularx}
\caption{Variables used in the TMVA's BDT in each of the categories in \texttt{LCFI+}. This selection of variables is the same for the case with \dEdx, \dNdx or ideal PID scenarios. The bold text represent the new variables while the rest are original \texttt{LCFI+} variables introduced in \cite{Suehara:2015ura}.
\label{tab:bdtvariables}}
\end{table}

\newpage

\subsection{Reconstruction performance plots for the  ILC500 case}
\label{sec:appendixPlots}

% Figure environment removed

% Figure environment removed

% Figure environment removed

% Figure environment removed


\newpage%
\clearpage




% add references
\printbibliography[title=References]
\end{document}