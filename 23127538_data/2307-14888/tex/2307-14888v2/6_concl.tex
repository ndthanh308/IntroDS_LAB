\section{Summary and prospects}
\label{sec:conclusions}
%New text
In this document, an update of the previous work at ILC250 \cite{Irles:2023ojs} has been done, extending it to ILC500. 
Furthermore, a detailed study of the impact of the TPC PID capabilities at ILD in both scenarios has been covered, including prospects of using cluster counting (\dNdx) with a pixel TPC.
Adding PID from TPC measurements into the flavour tagging software for $b$ and \cquark only provides a residual improvement since the main flavour-tagging variables are related to the vertex reconstruction. However, the TPC PID plays a key role in the reconstruction of \AFB. Prospects of a pixel TPC using \dNdx for PID envision a noticeable impact in the charge reconstruction of the \qqbar systems, especially for the \cquark case, where a factor $\sim2$ improvement in the experimental precision is obtained.

%beta text
Ongoing work by the authors of this note is the evaluation of experimental sensitivity of BSM models predicting new heavy resonances $Z^{\prime}$\cite{Yoon:2018xud,Funatsu:2017nfm,Funatsu:2020haj}, which could affect the predictions of \Rq and \AFB at ILC250/500. 

The ILC program also envisions a run at the Z-pole, the GigaZ mode\cite{irles2019complementarity}. 
With that run. one would expect at least one order of magnitude of improvements for the Z-couplings measurements of \bquark and \cquark compared to those of SLC and LEP and give a complete picture of the quark EW couplings when combining measurements at Giga-Z, ILC250 and ILC500. However, since a study at the Giga-Z would involve the development of new simulations and methodologies, it is considered out of this work's scope. 

%\afterpage{\blankpage}