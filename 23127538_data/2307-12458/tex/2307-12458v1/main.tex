 \documentclass{amsart}
\usepackage[utf8]{inputenc}
\usepackage{amsfonts}
\usepackage{amsmath}
\usepackage{amssymb}
\usepackage{amsthm}
\usepackage{mathrsfs} 
\usepackage{enumerate}
\usepackage{xcolor}
\usepackage{rotating}
\usepackage{hyperref}

\usepackage[justification=centering]{caption}
%\usepackage{scalerel}
%\usepackage{lineno}
%\linenumbers
%\usepackage{showlabels}
\usepackage{tikz}
\usetikzlibrary{decorations.shapes}
%\usepackage[round]{natbib}

\usepackage{lipsum}
\usetikzlibrary{arrows.meta,automata,quotes}
%\usepackage[authoryear]{natbib}
\theoremstyle{plain}
\newtheorem{thm}{Theorem}%[section]
\newtheorem{cor}[thm]{Corollary}
\newtheorem{lem}[thm]{Lemma}
\newtheorem*{lem*}{Lemma}
\newtheorem{prop}[thm]{Proposition}
\newtheorem{conj}[thm]{Conjecture}

\theoremstyle{definition}
\newtheorem{defi}[thm]{Definition}
%\theoremstyle{remark}
\newtheorem{rem}[thm]{Remark}
\newtheorem{example}[thm]{Example}
\newtheorem{problem}[thm]{Problem}
\newtheorem{obs}[thm]{Observation}
%\newcommand{\eop}{{\hfill $\blacksquare$} }
\newcommand{\ignore}[1]{}
%\renewcommand{\L}{\mathcal L}
%\newcommand{\R}{\mathcal R}
\newcommand{\N}{\ensuremath{\mathcal N}}
\renewcommand{\P}{\ensuremath{\mathcal P}}
\newcommand{\B}{\mathcal B}
\newcommand{\A}{\mathcal A}
\newcommand{\X}{\mathcal X}
\newcommand{\n}{\mathbb N}
\newcommand{\nz}{\mathbb N_0}
\newcommand{\interval}[1]{\ensuremath{\{1,\ldots , #1\}}}
\newcommand{\x}{\boldsymbol x}
\newcommand{\y}{\boldsymbol y}
\newcommand{\s}{\boldsymbol s}
\renewcommand{\ge}{\geqslant}
\renewcommand{\le}{\leqslant}
\renewcommand{\geq}{\geqslant}
\renewcommand{\leq}{\leqslant}
\renewcommand{\ngeq}{\ngeqslant}
\renewcommand{\nleq}{\nleqslant}
%\usepackage{ulem} 
% For  the authors to provide comments
\newcommand{\ur}[1]{\textcolor{red}{#1}}
\newcommand{\In}[1]{\textcolor{blue}{#1}}
\newcommand{\Ma}[1]{\textcolor{green}{#1}}
%%%%%%%%%%%%%%%%%%%%%%%%%%%%%
\usepackage{algorithm,algpseudocode}
\newcounter{algsubstate}
\renewcommand{\thealgsubstate}{\alph{algsubstate}}
\newenvironment{algsubstates}
  {\setcounter{algsubstate}{0}%
   \renewcommand{\State}{%
     \stepcounter{algsubstate}%
     \Statex {\footnotesize\thealgsubstate:}\space}}
  {}
 \usepackage{bbm,dsfont}

\title{Finite subtraction games in more than one dimension} 

\author{Urban Larsson}
\address{Urban Larsson, IIT Bombay, India}
\email{larsson@iitb.ac.in}
\author{Indrajit Saha}
\address{Indrajit Saha, Kyushu University, Japan}
\email{indrajit@inf.kyushu-u.ac.jp}
\author{Makoto  Yokoo}
\address{Makoto  Yokoo, Kyushu University, Japan}
\email{yokoo@inf.kyushu-u.ac.jp}
\date{\today}
\begin{document}
%%%%%%%%%%%%%%%%%%%%%%%%%%%%%%%%%%%%%%%%%%%%%%%%%%%%%%%%%%%
\begin{abstract}
We study two-player impartial vector subtraction games (on tuples of nonnegative integers) with finite rulesets, and solve all two-move games. Through multiple computer visualizations of outcomes of two-dimensional rulesets, we observe that they tend to partition the game board into periodic mosaics on very few regions/segments, which can depend on the number of moves in a ruleset. For example, we have found a five-move ruleset with an outcome segmentation into six semi-infinite slices. We prove that games in two dimensions are row/column eventually periodic. Several regularity conjectures are provided. Through visualizations of some rulesets, we pose open problems on the generic hardness of games in two dimensions. 

\noindent\textbf{Keywords:} Combinatorial Game, Subtraction Game, Impartial Game, Outcome Geometry, Outcome Pattern.  
\end{abstract}


%%%%%%%%%%%%%%%%%%%%%%%%%%%%%%%%%%%%%%%%%%%%%%%%%%%%%%%%%%%%%%%%%%%%%%%%%%%%%%%%%%%%

\maketitle
%%%%%%%%%%%%%%%%%%%%%%%%%%%%%%%%%%%%%%%%%%%%%%%%%%%%%%%%%%%%%%%%%%%%%%%%%%
\section{Introduction}
\label{sec:intro}
Figure~\ref{fig:dilbox} depicts a typical win-loss tessellation of games studied in this paper. 
% Figure environment removed
%%%%%%%%%%%%%%%%%%%%%%%%%%%%%%%%%%%%%%%%%%%%%%%%%%%%%%%%%%%%%%%%%%%%%%%%%%

Squirrel plays a game with his best friend the Hedgehog. On the table, there are two piles consisting of  $5$ peanuts and $6$ walnuts respectively. The players take turns to collect either $1$ peanut and $3$ walnuts or $2$ peanuts and $1$ walnut.
Suppose Hedgehog is very hungry and that walnut is her favorite food. By her first move, she collects $1$ peanuts and $3$ walnuts. After this, the two piles consist of $4$ peanuts and $3$ walnuts. Now when Squirrel copies the move of Hedgehog then there is no further option for Hedgehog. She is very disappointed. The next time she starts this game, she decides instead to grab $2$ peanuts and $1$ walnut. This leaves $3$ peanuts and $5$ walnuts on the table. Squirrel has a moment of realization: whatever she does, Hedgehog will get the last move. He suggests to his friend that the next time they play, he gets to start. 

We study a generalization of such games under the convention ``a player who cannot move loses'', which is called {\em normal-play}. Combinatorial games are two-player alternating-turn games with perfect and complete information, i.e., there is no hidden information and no chance move. 
  Popular recreational versions of combinatorial games are {\sc tic-tac-toe}, {\sc chess}, {\sc go}, {\sc checkers} and so on. Such games would admit infinite play if extra conditions were not imposed (e.g. the ko-rule in {\sc go} and the fifty-move-draw convention in {\sc chess}). We will restrict attention to games where every sequence of play (not necessarily alternating) is finite. Moreover, we will restrict attention to {\em impartial} games, where, as in the ruleset {\sc squirrel\&hedgehog}, the players have the same set of options, and each option is impartial. The prime example of such games is the ruleset  {\sc nim} \cite{bouton1901nim}. 
  
  A note on terminology used: the style {\sc ruleset} is used for a specific ruleset, but it is also used for a {\sc family of rulesets}. For example, we would write {\sc chess} but also {\sc board games}.
    
The perfect play {\em  outcome} of an impartial game is a loss or win depending on who starts \cite{berlekamp2004winning}. The starting player wins a game $G$ if $G\in\N$ (\N~ stands for the next player winning set), and otherwise $G\in \P$  (\P~ stands for the previous player winning set). Because of the normal play convention, any terminal position is in \P. As a variation of Zermelo's folk theorem \cite{zermelo1913anwendung}, the sets $\P$ and $\N$ partition the set of all game positions. 
  
  Let $\n=\{1,2,\ldots\}$ and $\nz=\n\cup\{0\}$ denote the positive and nonnegative integers respectively. The family of rulesets 
  {\sc finite subtraction} \cite{golomb1966mathematical, berlekamp2004winning} is another popular example of impartial games.\footnote{This class of rulesets is usually called ``subtraction games''. The word ``game'' is used widely without specific connotation, and sometimes it refers to a full ruleset. If we intend a specific game that people can play, then we may use the word position, starting position or ``game position''. Heap games are typically specified as a ruleset, and there are infinitely many possible starting positions.} Those are defined by a finite subtraction set $S\subset \n$, together with a position $x\in \nz$. By moving, a player subtracts their choice from the current position; the next position is $x-s$, for some $s\in S$, provided $s\le x$. If there is no such $s\in S$, then $x$ is a {\em terminal position}, and the current player loses.

 Let us define $d$-dimensional subtraction games, a.k.a the family of rulesets {\sc vector subtraction}. Let $d \in \n$.  The {\em game board} is   $\B=\B_d:=\nz^d $. 
The standard partial order of integer vectors applies, $\boldsymbol x=(x_1,  \ldots,  x_d) \ge \boldsymbol y=(y_1, \ldots,  y_d)$
 if for all $i$, $x_i\ge y_i$. And $\boldsymbol x > \boldsymbol y$
 if $\x \ge \y$, and for some $i$, $x_i > y_i$. Addition, subtraction and scalar multiplication is component-wise, e.g. $\x +\y = (x_1+y_1,  \ldots,  x_d+y_d)$.

\begin{defi}[{\sc vector subtraction}~\cite{golomb1966mathematical,larsson2012operator}]
An instance of {\sc vector subtraction} is an impartial normal-play ruleset played on $d$-tuples of non-negative integers $\x\in \B$. A set of  $d$-tuples of nonnegative integers, $S\subseteq \B\setminus \{\boldsymbol 0\}$ defines the move options. A move consists in subtracting some $ \s \in  S$ from $\x$, for which  $\x-\s \ge \boldsymbol 0$. 
\end{defi}
We study {\sc finite vector subtraction}, i.e. the instances for which $|S|<\infty$. 
Most parts of this paper concern  {\sc two-dimensional finite subtraction}. As an example of computations of the outcomes of such games, see  Figure~\ref{fig:dilbox} 
 and Figure~\ref{fig:hedgehog} (with ruleset $S=\{(2,16),(13,1)\}$). The black cells correspond to the \P-positions. 


% Figure environment removed



Similar games on infinite subtraction sets are well-studied. Popular examples are {\sc nim} ($S=\{a\s\mid \sum_{i} s_i=1,a\in \n\}$) and {\sc wythoff nim} \cite{wythoff1907modification} ($S=\{(a,0),(0,a),(a,a)\mid a\in \n\}$) and their many variations; see for example \cite{duchene2019wythoff} for a survey on a multitude of variations of {\sc wythoff nim}. {\sc nim} has become famous because of its generic capacity in encoding any impartial normal play game, via the famous Sprague-Grundy theory.  
In contrast, there are not many papers that study finite two-dimensional subtraction games. 

A notable exception is \cite{abuku2019combination}, where the family of rulesets  {\sc cyclic nimhoff} \cite{fraenkel1991nimhoff} is generalized to the setting of {\sc finite subtraction}. The authors provide a  closed form for the nim-values whenever the component  has a  so-called $h$-stair  structure (Theorem~2.1 and Definition~2.1 of their paper).\footnote{The $h$-stair definition originates in Siegel's Master's Thesis in 2005 \cite{siegel2005finite}.} From many examples, we observe that the rulesets considered in \cite{abuku2019combination} have exactly one {\em outcome segment}. 

Here we consider more general two-dimensional rulesets for which one segment is very rare. An outcome segment is a  special type of geometry on the outcomes, with a distinct regular outcome pattern (a distinguishing tesselation/mosaic), and an outcome segmentation is a finite union of all such outcome segments, for a given ruleset. Intuitively, the segments are separated by imaginary lines. See Figure~\ref{fig:symasym} for some examples of this phenomenon. If a ruleset admits an outcome segmentation then we can sometimes describe the outcomes by a finite set of {\em update rules} similar to the situation for eventual periodicity on one heap subtraction game,  where the update rule of each \P-position would be the period length, discounted for any preperiod. We discuss this development in Section~\ref{sec:outcomesegment}. 

\begin{rem}\label{rem: notation}
Regarding ruleset terminology, we often omit words such as``finite'' and ``vector'', and write, for example, {\sc subtraction} instead of {\sc finite vector subtraction}.  Similarly, we may omit ``subtraction'' in {\sc three-move subtraction} and simply write {\sc three-move}, and so on. The surrounding context would decide if we study one-dimensional games, or some class of {\sc vector subtraction}. 
\end{rem}

\begin{rem}An intuition: An {\em outcome geometry} describes a larger picture of the outcomes, such as lines, or line segments that separate regions with distinguished {\em outcome patterns} (mosaics). When we describe an outcome geometry there is no requirement to describe the various outcome patterns within the separated regions. The outcome patterns provide the usual information of how to win a game. 
See also subsection~\ref{sec:geom} for some background literature. 
\end{rem}

 % Figure environment removed

In this paper, we provide a complete characterization of the outcomes of {\sc two-move vector subtraction} (i.e. whenever $|S|=2$ in any dimension). The outcome patterns are much more varied than in the case of one dimension, but still tractable, and the solution is surprisingly simple, given the number of possible relations between the move parameters. It turns out that a case-by-case analysis is not necessary;  instead, we will invoke a certain `\P-to-\P\ principle'  and a `translation rule', where the outcome patterns are inherited from trivial {\sc one-move}. Let us begin by giving here a proof of the aforementioned principle. 

\begin{lem}[{\sc two-move} \P-to-\P]
\label{lem:PtoPd}
Consider  {\sc two-move vector subtraction}  $S= \{ \s_1, \s_2\}$. Then  $\x \in \P$  if and only if $\x+\s_1 +\s_2\in \P$. 
\end{lem}
\begin{proof}
 Suppose that $\x \in \P$. Then $\{\x+\s_1, \x+\s_2\}\subset \N$. The set of options of the position $\x+\s_1 +\s_2$ is the same set. Hence $\x+\s_1 +\s_2\in \P$.
 
 Suppose $\x+\s_1 +\s_2\in \P$. The set of options is $\{\x+\s_1, \x+\s_2\}\subset \N$. Each of these positions has a \P-position as an option. Hence, if $(\x+\s_1)-\s_2\in\N$ or $(\x+\s_2)-\s_1\in\N$, then this forces $\x\in\P$.   Otherwise $\{\x+\s_1-\s_2,\x+\s_2-\s_1\}\subset \P$. This implies that $\{\x-\s_2,\x-\s_1\}\subset \N$. Hence, $\x\in\P$.
 \end{proof}
 

In {\sc three-move two-dimension} we consider additive games,  where the third move is the sum of the first two moves. And we demonstrate a few instances when \P-to-\P~ update rules continue to hold.

The outcomes of any instance of {\sc one-dimensional finite subtraction} is ultimately periodic \cite{golomb1966mathematical}. We prove that such results still hold in two dimensions when we restrict the view to rows or columns: {\sc two-dimensional finite subtraction} is eventually row and column periodic. In contrast, the problem of generic periodicity for {\sc vector subtraction} is much harder, and indeed, a recent result \cite{larsson2013impartial, larsson2013heaps} proves Turing completeness of a slight extension of such games. 
Gurvich et al. show that {\sc vector subtraction} is EXP-complete \cite{Gurvich}.


%\subsection{Organization}
%\label{subsec:Organization}
The rest of the paper is organized as follows. Section~\ref{sec:literature} provides an overview of some relevant literature. All other sections, except Section~\ref{sec:anydimension} concern  play in two dimensions. 

In Section~\ref{sec:2dresemble1d}, we study two-dimensional rulesets that inherit the outcome structures from one-dimensional ditto. In particular, we find some instances where adjoining a move preserves those inherited \P-position structures. 

We solve {\sc two-move vector subtraction}:  Section~\ref{sec:twomove} focuses on two dimensions, and takes a geometric approach,  and Section~\ref{sec:anydimension} provides a \P/\N\ ~decision problem approach,  valid  in any dimension. 

Section~\ref{Sec:rowperiodicity} provides a proof of row/column eventual periodicity for generic {\sc two-dimensional subtraction}.

Section~\ref{sec:three_move_game}  generalizes ideas developed for {\sc two-move} and investigates their validity for {\sc three-move}. We find  generalized \P-to-\P~ update rules, and develop  theory about additive rulesets. 

This leads us to  Section~\ref{sec:outcomesegment}, where we introduce the concept of outcome segmentation via a defined coloring-scheme/automaton. %that finds the \P-positions of a given game, via defined update rules, in linear time. % given that defined \P-to-\P~ updates inherited from {\sc two-move subtraction} can be validated
 By using these techniques, we solve some instances of {\sc three-move}. %where such updates are valid, and thus obtain a linear solution to the decision problem, whether a given position is in $\P$. 

Finally, in Section~\ref{sec:picopen}, we display some experimental results and pose open problems and conjectures regarding outcome segmentation, and the complexity of generic {\sc two-dimensional subtraction}.  




\section{Literature review}\label{sec:literature}
Let us review some of the {\sc subtraction} literature.
%\ur{Where should we put this?}

\subsection{{\sc subtraction} mostly in one dimension}\label{sec:1dsubtr}

%%%%%%%%%%% Golomb paper%%%%%%%%%%%%%%%%

In the 1960s, Golomb  \cite{golomb1966mathematical} considers a variant of {\sc vector subtraction}, where a player wins by moving to the 0-vector, and establishes a draw-free condition in this setting.\footnote{We feel that it is more natural to adapt the convention ``who cannot move loses''. In this way, there cannot be draw games, and the theory develops more easily, as in \cite{larsson2012operator,larsson2013heaps}.} 
He explores the one-dimensional ruleset {\sc take-a-square}, i.e. $S=\{1,4,9,\ldots \}$, but does not provide any theoretical result on this ruleset: we quote, ``A detailed study of the properties of the losing set may well be as difficult as the study of the distribution of the prime
numbers". He proves that if a ruleset $S$ contains arbitrarily large gaps then its set of \P-positions is infinite. Lastly, he defines a nonlinear shift register model which determines the winning and losing positions. This construction proves the ultimately periodicity of any finite one-dimensional subtraction game. %Surprisingly, researchers paid less attention to the shift register model constructions.\ur{Actually, we all use it but we do not call it that. This is just our ``window'' tool.} 
Eppstein  \cite{eppstein2018faster}  continues the study of {\sc take-a-square} and provides an experimental justification that the set of \P-positions  has a ``density comparable to that of the densest known square-difference-free sets". He also studies complexity problems for {\sc subtraction}, and provides a fast algorithm for computing  nim-values, provided a slow growth (as in {\sc take-a-square}). Both Golomb and Eppstein contemplate on why {\sc take-a-square} is so much more complex than the subtraction game derived from the Moser–de Bruijn sequence (all numbers with 0,1 in base 4 expansion). These rulesets have the same asymptotic density, but the \P-positions of  {\sc take-a-square} are more dense: $\sim n^{0.7}$ compared with $\sqrt{n}$ \P-positions for {\sc moser–de bruijn}, up to position $n$.



As mentioned, every {\sc finite subtraction} on a single heap has ultimately periodic outcomes/nim-values. The method of proof gives an exponential bound on preperiod and period length in the largest element of the subtraction set. However, to our best knowledge,  exponential period lengths have not yet been rigorously established (see Flammencamp's thesis \cite{Flammenkamp_1997} for experimental results). Let us review some more recent development in the literature on the quest of preperiod/period length of {\sc finite subtraction}. 
Even if we restrict our attention to {\sc three-move}, preperiod and period length are not yet fully understood. 






In the early 1980s, Berlekamp et al. \cite[vol. 1, chap. 4, p. 83]{berlekamp2004winning} explore {\sc finite subtraction} and compute the nim-sequences for all rulesets with maximum subtraction $7$.\footnote{The nim-value of an impartial game can be computed recursively via a minimal exclusive algorithm \cite{siegel2013combinatorial} on the nim-values of its options. Any \P-position, such as the terminal positions, has nim-value $0$. Nim-sequences are a refinement of outcome sequences because they reveal the winning strategies in disjunctive sum play of impartial normal-play games. We will not require any of this material in this work, since we do not study game addition.} They note that the (ultimate) period length is the sum of two elements from the subtraction set, for all but one ruleset, namely $S=\{2,5,7\}$ (which has period length  $22$). In  \cite[vol. 3, chap. 5, p. 529]{berlekamp2004winning}, they examine {\em additive games} of the form $S = \{a, b, a+b\}$, and provide a complete statement for the nim-sequence whenever $b=2ha-r$, for some $0<r<a$, $h\in\n$. Specifically, the period length is $2b+r$. They highlight Ferguson's pairing property to derive the nim-value 1 from the 0s in the harder case when instead $b=2ha+r$ for some $0<r<a$, $h\in\n$. In this case, the period length is $a(2b+r)$. (They omit proofs of these results.) Ferguson's pairing property states that $\mathcal{G}(n)=1$ if and only if $\mathcal{G}(n-a)=0$, where $a$ represents the smallest number in the subtraction set.  
Austin \cite{austin1976impartial} analyses  subtraction games in his Master's thesis. He demonstrates that a subtraction game has no preperiod if there exists a $p$ such that $p-s\in S$ whenever $s\in S$. In several instances, his results are similar to those in Winning Ways; he studies additive rulesets and also rulesets of the form $S=\{a,b, 2b-a\}$. 

Alth{\"o}fer  and B{\"u}ltermann  \cite{althofer1995superlinear} prove some results on the nim-values of {\sc three-move}  $S= \{a,2a+1, 3a+1\}$, $a \in \n$. This ruleset has a quadratic polynomial period length in the parameter $a$ and no preperiod.  They consider {\sc four-move} of the form $S=\{a,4a,12a+1,16a+1\}$ with $1\le a \le 26$, and prove pure periodicity; the period length is a cubic polynomial in the parameter $a$. They wonder if the {\sc five-move} ruleset   
$S=\{a,8a,30a+1,37a+1,38a+1\}$, $a \in \n$, has super-polynomial period length. 

Flammenkamp \cite{Flammenkamp_1997} expands on \cite{althofer1995superlinear}. He searches for long periods in {\sc finite subtraction} via extensive computations.  He shows that {\sc three-move} and other rulsets, given $\max S=s_3$ follow various fractal-looking behaviors on the classification of preperiods and periods, whether they are of the form $s_1+s_2$, $s_1+s_3$, $s_2+s_3$, or something else. He shows that very few have other period lengths, and that about half of the rulesets have period length $s_1+s_3$. Flammenkamp is not convinced by the mentioned {\sc five-move} problem from  \cite{althofer1995superlinear}. He suggests instead various {\sc four-move} and {\sc five-move} rulesets, where he, through extensive computations, provides heuristics that supports ``exponential'' rather than ``polynomial'' eventual period lengths, in some cases via a ruleset parameter or in other cases via a ``record holder'' ruleset in terms of $\max S$. When it comes to {\sc four-move} he uses the second method to point at a tendency towards exponential vs. monomial behavior. He does this by testing the period lengths of record holders in the range $80\le \max S\le 235$, whether they can be lower bounded by an exponential expression of the form $2^{\alpha\max{S}}$ for some $ \alpha > 0$, or if they can be upper bounded by a monomial of the form $\max S^\beta$, for some $\beta>1$. As far as his computation goes, it seems that perhaps both are wrong.  Namely $\alpha\sim 0.2$, but with a decreasing tendency, whereas an upper bound $\beta$ seems even more unlikely, because, by indexing with $\max S$,  $\beta_{80}\approx 2.8$ while $\beta_{235}$ has increased to $\approx 5.2$. Motivated by the former, Flammenkamp makes more experiments. He completes a table with record holders for any finite ruleset with $\max S\le 30$, and finds a distinguishing property of members of record holder sets. They contain many elements of the form: both $\max S-s$ and $s$ belong to $S$.  Let us define this important property: $S$ is {\em max-symmetric} if, for all $1\le s\le \max S$, if $\max S-s \in S$, then $s\in S$. By restricting the experiments to max-symmetric sets, he finds that most record holders have size 5. By computing record holders among such sets, for all $61 \le \max S\le 117$, he finds that the period lengths tend to be $2^{\alpha\max S}$, for $\alpha\approx 0.3$. This certainly points towards the existence of subtraction sets with exponential period lengths.


Cairns and Ho \cite{cairns2010ultimately} introduce {\sc ultimately bipartite subtraction}. A ruleset is called bipartite if the ultimate period length is two. They first prove a necessary and sufficient condition for a ruleset to be bipartite. Secondly, they prove that if a ruleset is ultimately bipartite, then, for sufficiently large heap sizes, it is an \N-position if and only if the heap size is odd.  Ho  \cite{ho2015expansion} studies the periodicity of {\sc three-move subtraction} and discusses how to adjoin moves without changing the nim-value sequences.
He poses a conjecture that ``The subtraction set of an ultimately bipartite game is non-expandable".


In  \cite{ward2016conjecture}, Ward considers {\sc three-move subtraction} $S=\{a,b,c\}$, where $a<b<c$, and conjecture a precise characterization of the nim-value periods. The conjecture is stated in two different regimes. When $S$ is additive, i.e. $c= a+b$, the period is at most a quadratic polynomial in  $a$ and $b$. This part is stated as a theorem in \cite{Flammenkamp_1997}, but concerning only outcomes, both \cite{austin1976impartial} and \cite{berlekamp2004winning} have at least partial solutions to this part. Both \cite{Flammenkamp_1997} and \cite{ward2016conjecture} conjecture that when $c \neq a+b$, the period length has seven possibilities; in essence the period length is a divisor of the sum of two elements of the subtraction set.
Coleman et al.  \cite{coleman2020department} consider {\sc two-move subtraction}. They describe necessary and sufficient conditions for obtaining a given nim-value at a position. %Shenxing  
Zhang  \cite{zhang2021linearity} studies how the period of the nim-sequence changes when one adjoins a move to a given ruleset. The subtraction game literature and periodicity-related questions expand to partizan subtraction games (where the players have different subtraction sets) \cite{fraenkel1987partizan}. Duchene et al. \cite{duchene2022partizan} show that computing the preperiod length of the outcome sequence is NP-hard.  They compute the period length when each player's subtraction set is of size at most two.  






As mentioned, finite subtraction games have eventually periodic nim-sequences. 
What can one say about infinite subtraction games? Of course, classical {\sc nim} is a simple example with an aperiodic and unbounded nim-sequence. In \cite{siegel2005finite} Siegel studies subtraction games where every removal but a finite set is allowed and proves the arithmetic periodicity of nim-values.\footnote{A sequence $(a_n)$ is arithmetic-periodic if there is a period $p>0$ and a saltus $s > 0$ such that for all $n > 0$ , $a_{n + p} = a_n + s$.}  Researchers have been interested in the question of whether there is any subtraction game with an aperiodic but bounded nim-sequence. Fox \cite{fox2014aperiodic} found a positive answer,  using a technique from combinatorics on words,  via a ruleset with an aperiodic nim-sequence bounded by 3, but the  description of the subtraction set and the proofs are complicated. Inspired by this, Fox and Larsson \cite{FoxLar} provides a simpler game and analysis with the same properties. They consider the subtraction set $S= \{ F_{2n+1} -1 \}_{n\in\n} =\{1,4,12, \cdots \}$ where $F_k$
is the $k$th Fibonacci number with $F_0=0$, $F_1 =1$.

\subsection{Other settings} 
Cohensius et al. \cite{cohensius2019cumulative}  solve instances of scoring-play {\sc subtraction}, referred to as {\sc cumulative subtraction}. Two players alternatively remove stones from a pile that adds or removes points to a common score (depending on who played), and where any removal is restricted to a standard finite subtraction set $S$. They prove that the zero-sum outcomes in optimal play of such games are eventually periodic and the period is $2\max S$, and they find an exact characterization of the perfect play outcomes whenever $|S|=2$. Larsson  \cite{larsson2012operator} considers   {\em invariant subtraction games}, which is another name for vector subtraction games. %The game is defined by the set of  invariant moves $d$-tuples of  non-negative integers and a starting position. 
He defines the $\star$-operator of invariant subtraction games, and proves that any sequence of invariant subtraction games, generated by the  $\star\star$-operator, converges point-wise. The $\star$-operator defines the new subtraction game from the set of non-zero \P-positions of the old subtraction game. Thus, the $\star\star$-operator defines a game, by iterating $\star$ twice. He also provides a necessary and sufficient condition for duality, i.e. for $S= (S^{\star})^{\star}$ to hold. Note that the $\star$-operator can be applied to both finite and infinite subtraction sets. 

Carvalho et al. \cite{carvalho2020combinatorics} consider {\sc jenga}, a  popular recreational game. They compute nim-values of {\sc jenga} by viewing them as two-dimensional addition games, similar to the ones studied in \cite{larsson2013heaps}. For a study of the game values (a refinement of outcomes) of a finite two-dimensional partizan subtraction game, see  \cite{carvalho2012recursive}. We know of only one one-dimensional infinite partizan subtraction ruleset in the literature, namely  {\sc wythoff partizan subtraction} \cite{LarssonUrbanNeil}. Here, the players have complementary subtraction sets (via the complementary sequences of {\sc wythoff nim}'s set of \P-positions $S_L=\{1,3,4,6,\ldots\}$ and $S_R=\{2,5,7,10,\ldots \}$). The authors study the so-called `reduced canonical form' of a game value \cite{siegel2013combinatorial}, and relate this concept to various Fibonacci-type properties including methods from combinatorics on words; the authors achieve a complete solution of that game in terms of numbers and reduced canonical form switches. %\In{Is it the correct reference?}
%\ur{This new in blue is good.}

\subsection{Outcome geometry}\label{sec:geom}
Recent research on {\sc two-dimensional subtraction} has exposed a fair amount of geometry on the outcomes. This is certainly true for a {\sc generalized diagonal wythoff nim} \cite{Larsson2012GDWN,Larsson2014split} and in more generality (not necessarily symmetric rulesets) for the family {\sc linear nimhoff} \cite{friedman2019geometric}. The idea of those papers is to adjoin not single moves, but entire `lines' of moves to the original move-lines of {\sc nim} or {\sc wythoff nim}. The surprise is that the \P-positions appear to reorganize along various combinations of lines, new and/or old, with with distinct slopes and asymptotic densities as projected on the x-axis (e.g. {\sc nim} has one \P-line of slope 1 and density 1; {\sc wythoff nim} has two \P-lines of slopes $\phi=\frac{\sqrt{5}+1}{2}$ and $\phi^{-1}$ with densities $\phi^{-1}$ and $\phi^{-2}$ respectively). The regions (segments) between those \P-lines are visually, i.e. experimentally, void of \P-positions.  By making such assumptions, and by using a method from physics, called {\em renormalization}, a system of equations is formulated on the respective slopes and densities of the conjectured lines. The renormalization approach gives very good estimates of the conjectured geometry of games, as discussed in \cite{friedman2019geometric}. They also discuss when new move-lines do not affect the slopes and densities of existing \P-lines. That work has been an inspiration to this work, although the method used is different. 

Larsson et al. \cite{larsson2014maharaja}  introduces {\sc  maharaja nim}, which combines the moves of the Queen and Knight of Chess.  
They prove that for  {\sc  maharaja nim} the \P-positions remain close to those of {\sc wythoff nim}; they are within a bounded distance to the line with slopes  $\phi$ or  $\phi^{-1}$. More generally they wonder if this global outcome geometry remains the same when any finite number of subtractions is adjoined to those of {\sc wythoff nim}.

Larsson \cite{larsson2011blocking} introduces {\sc blocking wythoff nim}; for a given parameter $k$, at each stage of play, at most $k-1$ options may be blocked by the opponent; after the move any blocking is forgotten. He finds the winning strategies when $k=2,3$. Cook et al. \cite{Cook2017} continue by exploring the geometry of this game, and they find a cellular automaton that emulates a certain blocking-refinement of the outcomes. They find a surprising amount of self-organization in this system; the global geometry of the refined outcomes remains the same for all large blocking parameters. The system ``self-organizes itself into 11 visually distinguishable 
regions with 14 borders (half-lines or line-segments) and 6 junctions''.   






\section{ Two-dimensional rulesets that resemble one-dimensional  ones}
\label{sec:2dresemble1d}

For a one-dimensional subtraction game with two moves, say $S=\{a,b\}$, the period length of outcomes is at most $a+b$, and there is no pre-period. We recall the well-known result. We state it in a way that we will generalize later. Note that the outcomes of a one move game $S=\{a\}$ satisfy  $x\in\P$ if and only if 
    \begin{align}\label{eq:onemoveonedim}
    x\in \{0,\ldots , a-1\} \pmod{2a}.
    \end{align}
%\In{We should explain the accordion principle or suggest some appropriate name for the Theorem.}
The name of this result is due to the method of proof. We are not aware of the first appearance of proof. For a modern approach, see \cite{coleman2020department}. 
We use $\x\in\P(S)$ or $\x\in\N(S)$  for the outcome of position $\x$, with ruleset $S$. For $i\in\{1,2\}$, we write $o_i(\x)\in \{\P,\N\}$ for the outcome of $\x$ with rules as in $S_i$. 
    \begin{thm}\label{thm:twomoveonedim}
        Suppose $a<b$, with $S_1=\{a\}$ and $S_2=\{a,b\}$. Then the outcomes of the ruleset $S_2$ are purely periodic, with period length $a+b$, or $2a$ in case $2a\mid a+b$  Namely,  $o_1(x)=o_2(x)$ if $0\le x<b$, and $o_2(x)=\N$ if $b\le x<a+b$. 
    \end{thm}
    \begin{proof}
Base case: Obviously $o_1(x)=o_2(x)$ if $0\le x<b$. And $o_2(x)=\N$ if $b\le x<a+b$, since in this case $o_2(x-b)=\P$. Now apply Lemma~\ref{lem:PtoPd} with $d=1$.
\end{proof}

{\sc two-move} that resemble {\sc one-move} satisfy the condition $2a\ge b$, with notation as in Theorem~\ref{thm:twomoveonedim}. The following consequence will be useful for analysing some two-dimensional games.

    \begin{cor}\label{cor:onedim}
    Consider $S = \{a,b\}$. The outcomes are purely periodic of the form ``$x\in\P$ if and only if $x\in \{0,\ldots , a-1\} \pmod{a+b}$'', if and only if $a < b\le 2a$. The position $2a\in \P$ if and only if $2a<b$.
     %The outcomes of $S = \{a,b\}$ are purely periodic of the form ``$x\in\P$ if and only if $x\in \{0,\ldots , a-1\} \pmod{a+b}$'' if and only if  $2a\ge b$. The position $2a\in \P$ if and only if $2a<b$. 
    \end{cor}
    \begin{proof}
        Combine equation~ \eqref{eq:onemoveonedim} with Theorem~\ref{thm:twomoveonedim}.
    \end{proof}
    

\begin{defi}[Symmetric Ruleset]
\label{def:sym}
Consider $\B_2$. A ruleset $S$ is symmetric if, for all $(a,b)\in S$, $(b,a)\in S$. 

\end{defi}

\begin{defi}[Twin Ruleset]
\label{def:twin}
A twin ruleset $S$ satisfies, for all $(a,b)\in S$, $a=b$. 
A sibling ruleset $S$ satisfies, for given $k,\ell\in \n$ for all $(a,b)\in S$, $ka=\ell b$.
\end{defi}

For example $S=\{(1,2),(2,1)\}$ is a symmetric ruleset and $S=\{(1,1),(2,2)\}$ is a twin ruleset. So, twin rulesets are trivially symmetric. And twin rulesets are derivations of one-dimensional games in the following sense. 
\begin{defi}[Ruleset Derivation]
\label{def: Ruleset Derivation}
Consider a pair of game boards $\B_1$ and $\B_2$ with corresponding dimensions $d_1<d_2$. Let $S_1=S_1(\B_1)$ and let $S_2=S_2(\B_2)$ be two rulesets. Then $S_2$ is an $f$-derivation of $S_1$ if there is a linear  surjective function $f:\B_2\rightarrow \B_1$ such that, for all positions $ \boldsymbol x\in \B_2$, $o_{S_2}(f(\boldsymbol x))=o_{S_1}(\boldsymbol x)$. 
\end{defi}

For example let $d_1=1$ and $d_2=2$, with $S_2=\{(s,0)\mid s\in S_1\}$. Then $S_2$ is an $f$-derivation of $S_1$ for $f((x,y))=x$. 

In fact, a ruleset derivation is often much stronger than an outcome correspondence. In many cases, the game trees are multiplied for the larger ruleset, and there is no interaction between them. But for our purpose, it suffices to interpret derivation as `outcome-derivation'. Indeed, this gives an opportunity to explore when adjoining moves to the second ruleset does not interfere with the outcomes of the first ruleset. 

\begin{prop}[Twin Set Derivations]\label{prop:twinsetder}
    Let $S_1$ be an arbitrary one-dimensional subtraction game. Let $S_2=\{(s,s)\mid s\in S_1\}$. For all $(x,y)\in\B_2$, let %$f(x,y)$ as follows
\begin{equation*}
\label{Eqn_twin_Set}
	f(x,y)=
	\begin{cases}
		x, &\text{if }\  x\le y\\
		y , & \text{otherwise}.
	\end{cases}
\end{equation*}
 Then $S_2$ is an $f$-derivation of $S_1$.
\end{prop}
\begin{proof} %Suppose $o_1(x) = \P$. Then, for all $s\in S_1$, $o_1(x-s)= \N$. 
Suppose $x\le y$. For a base case take $x=0$. Observe that, for all $y$, $o_2(0,y)=\P$, because $(0,y)$ is terminal, since $S_2$ is a twin set. Fix an $ x' > 0$ and assume that for all $x< x'$,    $o_1(x) = \P$ if and only if for all $y$, $o_2(x,y)=\P$.\\

\noindent \textbf{Case 1:}  $x' \in \P$. We have to show that, for all $y \ge x'$, $o_2(x',y)=\P$. 
 Since  $x' \in \P$  then for all $s \in S_1$, $o_1(x'-s)=\N$. By the induction hypothesis, for all $y\ge x'$, for all $(s,s) \in S_2$,  $o_2(x'-s,y-s)=\N$  (because $y-s \ge x'-s$). Hence, for all $y\ge x'$, $o_2(x',y)=\P$.\\

\noindent \textbf{Case 2:}  $x' \in \N$. We have to show that, for all $y\ge x' $, $o_2(x',y)=\N$. Since  $x' \in \N$  then there exist an  $s \in S_1$ such that $o_1(x'-s)=\P$.
By the induction hypothesis,  for any $y \ge x'$, $o_2(x'-s,y-s)=\P$  (because $y-s \ge x'-s$).  Hence  for all $y$, $o_2(x',y)=\N$.

The proof is symmetric for $x \ge y$.

\end{proof}
% Figure environment removed

Sometimes (symmetric) expansions of twin-sets remain derivations of one-dimensional games. 
\begin{prop}\label{pro:twinadjoin}
    For fixed $a,b\in \n$, with $a<b$ let $S=\{(a,a),(b,b)\}$. Then $\P(S)=\P(S\cup\{(a,b)\})=\P(S\cup\{(a,b),(b,a)\})$ if and only if $a\ge b/2$. 
\end{prop}
\begin{proof}

Let $S'=\{S\}\cup \{(a,b)\}$. We begin by proving  $\P(S)=\P(S')$ in the case of $b/2 \le a < b $. The base case  is  $(0,0) \in \P(S) \cap \P(S')$. Fix $(x', y')$ and 
assume that, for  all positions $(x ,y)<(x', y')$,  $o_S(x , y)=  o_{S'}(x , y)$.\\

\noindent \textbf{Case 1:} $(x', y') \in \P(S)$. We have to show that $ ( x', y')\in \P(S')$. That holds if $(x'-a, y'-b)\in \N(S')$. By the induction assumption, it suffices to prove that $(x'-a, y'-b)\in \N(S)$, which holds by using the Proposition~\ref{prop:twinsetder} and the first part of the Corollary ~\ref{cor:onedim}, since $b\le 2a$.\\


\noindent \textbf{Case 2:} $(x', y') \in \N(S)$.  We have to show that $ ( x', y')\in \N(S')$. That holds since either $(x'-a, y'-a)\in \P(S')$ or $(x'-b, y'-b)\in \P(S')$, by the induction assumption. \\

Now, the equality $\P(S\cup\{(a,b)\}=\P(S\cup\{(a,b),(b,a)\}$ follows by an analogous argument.

Next, suppose that $b/2 > a$. Consider $\x=(2a,b)$. Then $\x\in\P(S)$, by combining Corollary~\ref{cor:onedim} with Proposition~\ref{prop:twinsetder}, but $\x\in\N(S')$, since $\x-(a,b)\in\P(S')$. 
\end{proof}


\section{Two-move subtraction in two dimensions.}\label{sec:twomove}
This section aims to study the geometry of the outcomes, while Section ~\ref{sec:anydimension} focuses more on the \P/\N~ decision problem and its {\sc two-move} complexity. In this section, we develop geometrical ideas through Figure~\ref{fig:cuttingprinciple}. 
We begin by recalling the trivial she-loves-me-she-loves-me-not outcome structures of {\sc one-move}. %The one-move game is  in the 
This characterization is useful for the analysis of {\sc two-move}.

\begin{lem}[{\sc one-move}]
\label{Lemma_with_onemove}
 Let $a$ and $b$ be non-negative integers such that at least one is positive. Consider the subtraction game $S=\{(a,b)\}$. Then,
\begin{enumerate}
    \item [a)]  the position $(x,y)\in \P$ if and only if $(x+a, y+b)\in \N$;
  
    \item [b)]  the position $(x,y)\in \P$ if and only if $(x+2a, y+2b)\in \P$;
   
    \item [c)]  the position $(x,y)\in \N$ if and only if  $(x+2a, y+2b)\in \N$.
\end{enumerate}
\end{lem}

\begin{proof} 
Omitted. 
 \end{proof}


\tikzset{decorate sep/.style 2 args=
{decorate,decoration={shape backgrounds,shape=circle,shape size=#1,shape sep=#2}}}

% Figure environment removed


% Figure environment removed

For $a, b \in \n $,   $k\in \nz $, let 

The L-shapes $L_k$ for any $k \in \nz$ collapse to parallel strips, horizontal or vertical, whenever $b=0$ or $a=0$ respectively. See Figures~\ref{Fig_withonemoveset_Lshape_pic3} and \ref{Fig_withonemoveset_Lshape_pic1}.

The following result can be used to derive the outcome patterns of  {\sc two-move}.

\begin{lem}[{\sc 2-d one-move} Structure]
\label{Lemma_withonerulset}
  Consider any ruleset  $S= \{(a,b)\}$. 
\begin{enumerate}[a)]
     \item If both $a,b>0$, then $(x,y)\in \P$ if and only if  $(x,y) \in \bigcup_{k\in \nz}  L_k$;
    \item  If  $0=a<b$,  then $(x,y) \in \P$ if and only  if 
              $$
              y \in \bigcup_{k\in \nz}\big\{2kb, \cdots  ,(2k+1)b-1\big \} ;
    $$
\item If $0=b<a$, then $(x,y)\in \P$ if and only  if 
$$
x \in \bigcup_{k\in \nz} \big\{2ka, \cdots  ,(2k+1)a-1\big \}.
$$
\end{enumerate}
 
 \end{lem}


\begin{proof}
a) If $x< a$ or $y<b$, then there is no move and hence $(x,y)\in \P\cap L_0$. The rest of the proof follows by Lemma~\ref{Lemma_with_onemove}.  The proofs of b) and c) are special cases of a) by setting the parameters $a=0$ and $b=0$, respectively.
\end{proof}


The following result is proved in the Section  \ref{sec:intro} in more generality (see Lemma~\ref{lem:PtoPd})
%and so we omit a proof here.  
 \begin{lem*}[$\P$-to-$\P$]
\label{lem:PtoP}
Let $S= \{ (a,b), (c,d)\}$.  Then $(x,y) \in \P$  if and only if   $(x+a+c,  y+b+d)\in  \P$.

\end{lem*}


Due to Lemma~\ref{lem:PtoP}  we define a ruleset dependent translation. 

 \begin{defi}[Translation Set]
\label{def:PtoP} Let $S=\{(a,b),(c,d)\}$ and let $X\subset \B_2$. Then 
 $T(X)=T_S(X):=\{(x+a+c,y+b+d) \mid (x,y)\in X\}$.
 \end{defi}
 We may iterate this set translation.

\begin{obs}
    Note that, by Lemma~\ref{lem:PtoP}, for all $k\in\nz$, $X\subset\P$, if and only if $T^k(X)=\{(x+k(a+c), y+k(b+d)) \mid (x,y)\in X\}\subset \P$.
\end{obs}


The \P-positions of the two move games can be learned from the one move games in a linear fashion. 

% Figure environment removed

\begin{defi}[Slope]
   Consider $S=\{(a,b),(c,d)\}$. Then $\delta=\delta(S)=\frac{b+d}{a+c}$. 
\end{defi}


\begin{obs}\label{obs:cut}
    In Figure~\ref{fig:cuttingprinciple}, region $C$ contains only \N-positions if $d\ge\delta c$. Otherwise $(x,y)\in \N(S)\cap C$ if and only if $(x-a,y-b)\in \P(\{(a,b)\})\cap(A\cup B)$. Namely, in the latter case, the move $(c,d)$ is not feasible.
\end{obs}

This observation together with Lemma~\ref{lem:PtoP} provide the main result of this section.

\begin{thm}[{\sc 2-d two-move}]  
\label{Theorem_2D_outcomepattern}
Consider the 2-move game $S=\{(a,b), (c,d)\}$. % where $a\le \min \{b,c,d\}$. 
Let $S_1=\{(a,b)\}$. Let $\Gamma_c=\P(S_1)\cap \{(x,y)\mid 0\le x\le c-1\}$. % and $\Gamma_2=\P(S_2)\cap \{(x,y)\mid 0\le y\le d-1\}$. 
Consider positions in the upper region, with respect to the parameters of the ruleset $S$, $U=U(S)=\{(x,y)\mid y\ge \delta x\}$.
\begin{itemize}
    \item[(i)]  If $c\delta \ge d$, then $(x,y) \in \P\cap U$ if and only if $(x,y) \in\left (\cup_{k\in \nz} T^k(\Gamma_c)\right)\cap U$.
    \item[(ii)]
 Otherwise, consider $B'=\{(x-a,y-b)\mid \delta x\le y<d, c\le x<c+a\}\cap \N(S_1)$.  Let $C'=B'+(a,b)=\{(x+a,y+b)\mid (x,y)\in B'\}$. Then $(x,y) \in \P\cap U$ if and only if $(x,y) \in\left (\cup_{k\in \nz} T^k(\Gamma_c\cup C')\right)\cap U$.
\end{itemize}
\end{thm}

\begin{proof} 
%Let $S=\{(a,b), (c,d)\}$ be the ruleset.  
Observe that $T^0(\Gamma)=\Gamma\subset \P(S')$. Let $C=\{ (x,y):  c \le x < a+c, y\ge \delta x\}$. 

Consider case (i), and observe that $C\subset \N$, since $(x-c,y-d)\in\P$ is terminal. Thus, the result follows, by Lemma~\ref{lem:PtoP}, and by iterating the translation $T$.

Similarly, case (ii) holds because any new \P-position in the $C$-region, is a consequence of either no possible move (in case $a>c$) or the option is in $\N(S_1)\subset B'$. Then, again apply Lemma~\ref{lem:PtoP}, and iterate the translation $T$.
\end{proof} 

\section{{\sc two-move vector subtraction}}\label{sec:anydimension}%\label{sec:onemoveanydimension}
This section elaborates on the natural generalization of {\sc two-move}, from the previous section to any dimension. This study focuses on the \P/\N-decision problem of a generic game position. Recall the `\P-to-\P\ principle' for {\sc two-move vector subtraction}, Lemma~\ref{lem:PtoPd}:
\begin{lem*}[General \P-to-\P]
%\label{lem:PtoPd}
Consider  {\sc two-move vector subtraction}  $S= \{ \s_1, \s_2\}$. Then  $\x \in \P$  if and only if $\x+\s_1 +\s_2\in \P$. 
\end{lem*}

 \begin{rem}\label{rem:NtoN}
     By Lemma~\ref{lem:PtoPd},   if $S= \{ \s_1, \s_2\}$, then $\x \in \N$  if and only if $\x+\s_1 +\s_2\in \N$. 
 \end{rem}



\begin{defi}[Translation Function]\label{def:trans}
    Consider a $d$-dimensional ruleset $S=\{\s_1 ,\s_2\}$ of size two. 
    Let the translation, of $\X\in \B$, be 
    $t(\x)=t_S(\x)=\x +\s_1+\s_2$.
\end{defi}
We will iterate this function: $t^k(\x)=t(t^{k-1}(\x))$, where $t^0(\x)=\x$. Thus, for all $k$, $t^k(\x) = \x +k(\s_1+\s_2)$. 
\begin{thm}[General {\sc two-move}]\label{thm:2mddim}
    Consider {\sc two-move vector subtraction} $S=\{\s_1,\s_2\}$. 
    Then $\x\in\P$ if and only if $\x-t^k(\x)\in\P$, where $k \in\nz$  is such that $\x-t^{k+1}(\x)\ngeq\boldsymbol 0$.
    
\end{thm}
\begin{proof}
This is immediate by Lemma~\ref{lem:PtoPd}.
\end{proof}
Thus, the complexity of the $\P$/$\N$ decision problem for a generic position $\x\in\B$ reduces to computing the outcome of its smallest  representative, say $\x'$ in $\B$ $\pmod {\s_1+\s_2}$. So, the complexity is in essence the same as determining the remainder after division, but here in $d$ dimensions. 
   The standard na\"ive algorithm for the \N/\P~  membership problem is exponential in succinct input size \cite{fraenkel2004complexity}. 
  
   As a corollary, we get an improvement for the case of {\sc two-move vector subtraction}. 
\begin{cor}
    Algorithm~1 for the \P/\N~   membership problem of {\sc two-move vector subtraction} is linear in the succinct input size.
\end{cor}
\begin{proof}
   The problem reduces to one out of two situations:
\begin{itemize}
\item[(i)] $0\le \x'\ngeq \{\s_i,\s_{3-i}\}$, for both $i\in\{1,2\}$;
\item[(ii)] $0\le \x'-\s_i\ngeq \{\s_i,\s_{3-i}\}$, for some $i\in\{1,2\}$;
\item[(iii)] find a $k\in\nz$ such that $(k+1)\s_i\ngeq\x'\ge k\s_i$, $i\in\{1,2\}$. 
\end{itemize}
In case (i), $\x'$ is a terminal \P-position, and so, by Theorem~\ref{thm:2mddim}, $\x\in\P$. 
In case (ii), there is a move to a terminal \P-position; hence $\x'\in \N$, and thus, by Theorem~\ref{thm:2mddim}, $\x\in \N$. If (i) does not hold, then test if she-loves-me-she-loves-me-not according to (ii) returns a terminal \P-position for some $i\in \{1,2\}$. Denote this procedure by Algorithm~1. Observe that a response to the move $\s_i$ with the move $\s_{3-i}$ is impossible by the choice of $\x$ as the smallest representative in $\B$ $\pmod {\s_1+\s_2}$. 
\end{proof}


\section{Eventual row periodicity}
\label{Sec:rowperiodicity}
Let us define row periodicity and eventual row periodicity for {\sc two-dimensional subtraction}. Analogous definitions will be used for the columns. 
\begin{defi}[Row Periodicity]
Consider {\sc two-dimensional subtraction}. 
\begin{enumerate}[a)]
\item Consider row $j\in \nz$.  The outcome is row $j$ periodic  if  $\exists$ $T > 0$ such that  $\forall i \ge 0$, $ o(i,j)= o(i+T,j)$. 
 \item  The outcome is row periodic if  $\forall j \ge 0$  the outcome is row $j$ periodic.
\end{enumerate}
\end{defi}
In practice, row periodicity is very rare.

\begin{defi}[Row Eventual Periodicity]
\label{def_eventualperiodic}

Consider {\sc Two-dimensional subtraction}.
\begin{enumerate}[a)]
    \item  Consider row $j\in \nz$. The outcome is  row $j$ eventually periodic if there exists a column  $i^{'}\in \nz $  and a $T >0$ such that $\forall i \ge i'$
    $
    o(i,j) = o(i+T, j).
    $  
    \item The smallest such $i'\in \nz $ is the preperiod length of row $j$.
    \item The outcome is row eventually  periodic if  $\forall j \ge 0$ the outcome is row $j$  eventually periodic.
\end{enumerate}
\end{defi}


\begin{thm}[Row-Column Periodicity]
The outcome of any two-dimensional subtraction game on a finite ruleset is row (column) eventually periodic.
\end{thm}
\begin{proof}
Let $S=\{(a_i,b_i)\}_{1\le i \le k}$ be the subtraction set, and let $A=\max_{i} \{a_i\}$ be the maximum of the $x$-coordinates of the subtraction set.

Fix a row $y$, and define the window $W(x,y) = (o(x,y), o(x+1,y), \cdots , o(x+A-1,y))$. The outcome function $o(x,y) \in \{\N, \P\}$. Hence there are $2^A$ possible outcome patterns within the window range. 

We prove the theorem using induction.  

Base case: the outcomes of row $0$  are eventually row periodic. We first prove this statement. 

Since $2^A$ is a finite number, there exists a smallest $x$ and an $\alpha$ such that $W(x+\alpha,0) = W(x,0)$. 
That is, the outcome patterns are repeated in the window. By computing the next outcome, because of the choice of window size, we find that $W(x+i+\alpha,0) = W(x+i,0)$, for all $i\ge 0$. Hence, the repetition occurs before $A 2^A$, and the period length $p_0$ is at most $A2^A$.



Hypothesis: Fix a row $y'$ and assume that the outcomes are eventually row periodic, for all rows $0\le y < y'$.
Let the period lengths for the rows $0, 1, \cdots y'-1$ be  $p_0, p_1, \cdots, p_{y'-1}$, respectively.  We aim to prove that the $y'$-th row is eventually row periodic. 
%\begin{defi}
%\label{def:superperiod}

Let $B=\max_{i} \{b_i\}$ be the maximum of the $y$-coordinates of the subtraction set. Define the superperiod length 
\begin{equation}
\label{eq:superperiod}
\pi_{y'} = \prod_{i=y'-B}^{y'-1}p_i.
\end{equation}
%\end{defi}
For any sufficiently large $x$, consider the windows $W(x+m\pi_{y'},y')$, where $m \in \nz$. Since the number of possible outcome patterns within these windows is finite, there will be a repetition, for say $m'$ and $m''$ respectively. When this repetition occurs, we use that $x$ is sufficiently large, meaning that the lower rows have become periodic. It follows that row $y'$ is eventually periodic, because $o(x+m'\pi_{y'}+1,y')=o(x+m''\pi_{y'}+1,y')$ (the argument is the same as in the base case). 
\end{proof}
  
 
 \section{Three moves in two dimensions}\label{sec:three_move_game}
We consider special types of  {\sc three-move subtraction} and identify some properties of their outcomes. In particular, we are interested in: when does the \P-to-\P\ property from {\sc two-move subtraction}  continue to hold?\footnote{The nature of the results is `lemma', and we predict that they will be useful in future work if not used here.} One attractive family is the family of {\sc three-move additive} rulesets. 

\begin{defi}[{\sc additive subtraction}]\label{def:addsym}
The ruleset $S=\{(a,b),(c,d),(e,f)\}$ is additive if $e=a+c$ and $f=b+d$.
\end{defi}

%\ur{When do both directions hold?}
Figure~\ref{fig:additive} depicts outcomes of the additive rulesets $S=\{(1,2),(2,1),(3,3)\}$ and $S=\{(1,2),(3,4),(4,6)\}$. See also Figures~\ref{fig:122133color} and \ref{fig:coloring123446}.

  % Figure environment removed
\begin{lem}[{\sc additive subtraction}]\label{lem:three_move_withsepecialstructure}
Let $S=\{(a,b),(c,d), (a+c,b+d)\}$. If  $(x+2a+c, y+2b+d)\in \P$ and $(x+2c+a, y+2d+b)\in \P$ then $(x,y) \in \P$.
 \end{lem}
\begin{proof}
The set of options of $(x+2a+c,y+2b+d)$ is 
$$\{(x+a+c,y+b+d),(x+a,y+b),(x+2a,y+2b)\}\subset \N.
$$  
The set of options of $(x+2c+a,y+2d+b)$ is 
$$\{(x+c+a,y+d+b),(x+c,y+d),(x+2c,y+2d)\}\subset \N.
$$ 
For each option, there is a move to a $\P$-position. The option $(x+a+c,y+b+d)\in\N$ occurs in both sets. But both $(x+a,y+b),(x+c,y+d)\in\N$. Hence the third move must lead to a $\P$-position. Hence $(x,y)\in \P$.

\end{proof}

\begin{rem}
 The reverse direction of Lemma~ \ref{lem:three_move_withsepecialstructure} is not true. Consider the ruleset $S =\{(1,1),(2,2),(3,3)\}$. Then $(0,0) \in \P$ but  $(5,5) \notin \P$. 
\end{rem}
The next lemma shows that we can still have \P-to-\P\ update rules for {\sc additive subtraction}.
\begin{lem}[{\sc additive subtraction} \P-to-\P]\label{lem:three_move_PtoP}
Let $S=\{(a,b),(c,d), (a+c,b+d)\}$ and suppose $(x,y)\in\P$. 
\begin{enumerate}[(i)]
    \item If $(x+2a+2c, y+2b+2d)\in \N$ then $(x+2c+a, y+2d+b)\in \P$ or $(x+2a+c, y+2b+d)\in \P$;
    \item if $(x+2a+c, y+2b+d)\in \N$ then $(x+2a,y+2b)\in\P$;
    \item if $(x+2c+a, y+2d+b)\in \N$ then  $(x+2c,y+2d)\in\P$;
    \item if both $(x+2c+a, y+2d+b)\in \N$ and $(x+2a+c, y+2b+d)\in \N$ then $(x+2a+2c, y+2b+2d)\in \P$.%\ur{This is the converse of (i).}
\end{enumerate}
 \end{lem}
\begin{proof}
Let $S=\{(a,b),(c,d), (a+c,b+d)\}$. 
The set of positions with $(x,y)\in \P$ as an option is 
\begin{align}\label{eq:xopt}
\{(x+a+c,y+b+d),(x+a,y+b),(x+c,y+d)\}\subset \N.
\end{align} 
For (i), observe that one of the options must be in \P. But $(x+a+c,y+b+d)\in \N$. Hence $(x+2c+a, y+2d+b)\in \P$ or $(x+2a+c, y+2b+d)\in \P$.
For (ii) and (iii), observe that two of the options lead to \N-positions in \eqref{eq:xopt}. Hence the third must be in \P. 
For (iv), observe that all three options of  $(x+2a+2c, y+2b+2d)$ are \N-positions. 
\end{proof}



The following result is an adaptation from one-dimensional.

\begin{lem}
\label{Lemma_with_twin}
Let $S=\{(a,b),(a+c,b+c), (a+2c,b+2c)\}$. If $(x, y)\in \P$ then  $(x+2a+2c, y+2b+2c)\in \P$.
\end{lem}
\begin{proof}

  Let $S=\{(a,b),(a+c,b+c), (a+2c,b+2c)\}$ be the ruleset. The set of positions with $(x,y)\in \P$ as an option is 
$$
\{(x+a,y+b), (x+a+c,y+b+c), (x+a+2c, y+b+2c)\}\subset \N.
$$
This set coincides with the set of options of $(x+2a+2c, y+2b+2c)$. %is 

Therefore $ (x+2a+2c, y+2b+2c)\in \P$.
\end{proof}



Figure~\ref{fig:233112_100} gives an example of a ruleset with asymmetric additive rules. 

\begin{defi}[{\sc asymmetric additive subtraction}]
    The ruleset $$S=\{(a,b),(c,d),(e,f)\}$$ is asymmetric additive if $e=a+c$ and $d=b+f$.
\end{defi}




\begin{lem}[{\sc asymmetric additive subtraction}]\label{lem:three_move_sum}


Let $S=\{(a,b),(c,b+d), (a+c,d)\}$ and suppose $(x,y)\in\P$. 
\begin{enumerate}[(i)]
\item If  $(x+2a+2c, y+2d) \in \N$  then   $(x+a+2c, y+2d-b) \in \P$ or  $(x+2a+c, y+d-b) \in \P$;
\item if $(x+a+c, y+2b+d) \in \N$ then $(x,y+2b) \in \P$;
\item if  $(x+2a+c, y+b+d) \in \N$ then $(x+2a,y) \in \P$;
\item  if both  $(x+a+2c, y+2d-b) \in \N$ and  $(x+2a+c, y+d-b) \in \N$ then $(x+2a+2c, y+2d) \in \P$.
\end{enumerate}
\end{lem}

\begin{proof}
Let  $S=\{(a,b),(c,b+d),(a+c,d)\}$. The set of positions with $(x,y)\in \P$ as an option is 
\begin{align}\label{eq:xopt_three}
\{(x+a,y+b),(x+c,y+b+d),(x+a+c,y+d)\}\subset \N.
\end{align} 

Consider the assumptions of item (i). The set of options of $(x+2a+2c, y+2d)\in\N$ is 
$\{(x+a+2c,y+2d-b),(x+2a+c,y+d-b),(x+a+c,y+d)\}.
$
By \eqref{eq:xopt_three}, hence $(x+a+2c, y+2d-b) \in \P$ or  $(x+2a+c, y+d-b) \in \P$.

Consider the assumption of item (ii).  The set of options of $(x+a+c , y+2b+d)\in\N$ is 
$\{(x+c,y+b+d),(x+a,y+b),(x,y+2b)\}$.   
By \eqref{eq:xopt_three}, $\{(x+c,y+b+d),(x+a,y+b)\} \subset \N$. Hence $(x,y+2b) \in \P$. 

For the assumption of item (iii), the set of options of $(x+2a+c, y+b+d)\in\N$ is 
$\{(x+a+c,y+d),(x+2a,y),(x+a,y+b)\}$.  
By \eqref{eq:xopt_three}, $(x+a+c,y+d), (x+a,y+b) \subset \N$. Hence  $(x+2a,y) \in \P$.

Consider the assumptions of item (iv). The set of options of $(x+2a+2c, y+2d)$ is $\{(x+a+2c,y+2d-b),(x+2a+c,y+d-b),(x+a+c,y+d)\}\subset \N$. 
By \eqref{eq:xopt_three}, hence $(x+2a+2c, y+2d) \in \P$.
\end{proof}

% Figure environment removed

\section{Outcome segmentations}
\label{sec:outcomesegment}
When it comes to generic {\sc finite two-dimensional subtraction}, we will provide a thrilling open problem in Section~\ref{sec:picopen}, which requires a couple of somewhat technical definitions. The definitions are also used to compute \P-positions in some instances in this section.

An {\em outcome segment} is a special type of geometry on the outcomes. %that we define as follows. 

\begin{defi}[Coloring Scheme/Automaton]
   Subsets of the first quadrant are colored by means of update rules. 
An update rule $u=(u_1,u_2)$ takes as input a colored position $(x,y,c)$, where $c\in C_u\subset C$, a given  set of colors and where $(x,y)\in\B$. The output is a colored position $(x+u_1,y+u_2, c')$, $c'\in C$. Thus, a single color can have several update rules attached to it. A legal coloring scheme $(X_0, U, C)$ gives at most one color to each position; here $X_0$ is a set of initially $C$-colored positions and $U$ is a finite set of update rules.  
\end{defi}

Thus, a coloring scheme could be regarded as a simple cellular automaton, where the color of each cell depends on at most one younger cell. For known Turing complete cellular automatons (such as rule 110), updates depend on several (three) younger cells.

Note that a coloring scheme permits a cell to be `updated' more than once provided it is given the same color.

\begin{problem}
    Prove some regularity property, such as tiling periodicity, on coloring schemes.
\end{problem}

   \begin{defi}[Outcome Segment]
   Consider a subset of the first quadrant $\Gamma\subset \B_2$ and a finite set of update rules $U$. Then $\Gamma$ is a  $U$-segment if there is a coloring scheme of $ \Gamma$ via the rules in $U$. If a $U$-segment $\Gamma$ is a $\P$-subset of a given ruleset, we call it an outcome segment.
   \end{defi}

\begin{example}[Outcome Segment]\label{ex:OS}
    
    See the middle segment in Figure~\ref{fig:tiling_example}. Set the initial coloring as $\mathbf{red}=\{(0,0),(0,1),(0,2)\}$, $\mathbf{green}=\{(1,0),(1,1)\}$ and $\mathbf{blue}=\{(2,1)\}$. The update rule $\vdash$ ``adjoin'' is: 
    \begin{itemize}
        \item $(x,y)\in \mathbf{red}\rightarrow (x+5,y+4)\vdash  \mathbf{green}$,
        \item $(x,y)\in \mathbf{green}\rightarrow (x+3,y+5)\vdash \mathbf{red}$,
        \item $(x,y)\in \mathbf{gre  en}\cup \mathbf{blue}\rightarrow (x+5,y+4)\vdash \mathbf{blue}$.
    \end{itemize}
    In Proposition~\ref{prop:OS}, we prove that this coloring scheme corresponds exactly to the \P-positions of the middle segment of the asymmetric additive ruleset $$S=\{(1,2),(2,3),(3,1)\}.$$
\end{example}

\begin{defi}[Outcome Segmentation]
    Consider a ruleset $S$. A finite union of outcome segments $\cup_{i}\Gamma_i$ is an outcome segmentation of $S$ (or just a segmentation of $S$) if, for all $i,j$, $\Gamma_i\cap \Gamma_j=\varnothing$, and, for all sufficiently large game boards of $n$ positions, they contain at least $n-o(n)$ positions. A segmentation is perfect if it partitions the first quadrant. A $k$-segmentation has $k$ outcome segments. 
\end{defi}

   \begin{example}[Outcome Segmentation]\label{ex:titlingex}
       There are three segments in Figure~\ref{fig:tiling_example}. The middle segment is described in Example~\ref{ex:OS}. The update rule for the lower segment is $(5,4)$ and the initial set consists of all positions $(x,0)$ with $x\ge 2$. The upper segment has two alternating update rules $(4,4)$ and $(4,5)$, with $(4,4)$ applied to the initial set $\{(0,y)\mid y\ge 3\}$. We prove that altogether this is a perfect outcome segmentation of the game in Figure~\ref{fig:tiling_example}.
   \end{example}
   The terminology ``outcome segmentation" is motivated by multiple observations where segments are separated by imagined lines, as in the following example.  
\begin{example}\label{ex:lines}
Consider the two lines $f(x)=9x/8+2$ and $g(x)=4x/5-4/5$. The positions $(x,y)$ in the middle (lower, upper) segment satisfy $\lceil g(x)\rceil < y <  \lceil f(x)\rceil $ ($y\le \lceil g(x)\rceil $, $y\ge \lceil f(x)\rceil $). 
\end{example}

% Figure environment removed

Let us analyze the proposed perfect outcome segmentation from Example~\ref{ex:titlingex}.  
\begin{prop}\label{prop:OS}
    Let  $S =\{(1,2),(2,3),(3,1)\}$. This game has a perfect outcome segmentation given by three sets of update rules together with three initial sets of colored positions.
\end{prop}
\begin{proof}
    The update rules for the two lower segments are obtained by \P-to-\P\ combinations from the ruleset. %The \P-to-\P rule for the upper segment is $(3,1)+(1,2)=(4,3)$. 

    The update rule for the lower segment $\Gamma_0$ is $(2,3)+(3,1) = (5,4)$. The initial coloring corresponds to the terminal \P-positions $\{(2,0),(3,0),\ldots\}$. Since no rule has y-coordinate greater than three, there is no colored position within this segment that can reach another colored position. Hence \P-to-\N property holds. Enumerate the moves by $i:(1,2)$, $ii:(2,3)$ and $iii:(3,1)$. In Figure~\ref{fig:tiling_example}, row-wise generic type $ii, i, iii$ moves have been inserted at the lower segment, for positions with $y$-coordinates $1,2,3\pmod 4$ respectively. Clearly, they reach colored positions. 
    Hence the lower segment is an outcome segment for this ruleset.  

    For the middle segment $\Gamma_1$, we must check that, starting from any colored non-terminal position, the missing rules from the \P-to-\P\ updates reach \N-positions.  Indeed, the initial coloring given in Example~\ref{ex:OS} corresponds to terminal \P-positions, and the caption of Figure~\ref{fig:tiling_example} verifies this: starting from a colored position in the middle segment, the missing rules are smaller than the \P-to-\P\ rules that define the coloring. 
    
    The \N-to-\P~ property of the middle segment is justified by the translations of the colored regions to the two nuances of brown cells: the dark-brown cells are translated by the move $(1,2)$ and the light-brown are translated by the move $(2,3)$. By adapting this brown-coloring scheme at each level, there will be 3 or 2 non-colored cells between each pair of red-green-blue colored regions. Those cells take the same move as described for the lower and upper segments. By decomposing each color update, for $k\ge 2$, in two copies of the shape of level $k-2$. The proof follows by induction, similar to the simpler case in Proposition~\ref{prop:symadd}. We omit further technical details.

    Consider the upper segment $\Gamma_2$. Claim: if we manually color all positions of the form $(8k, 3+9k)$,  $(4+8k,7+9k)$ and $(4+8k,7+9k)$, then the \P-to-\P\ rule $(3,1)+(1,2)=(4,3)$ colors exactly the remaining cells from the rule: ``The upper segment has two alternating update rules $(4,4)$ and $(4,5)$, with $(4,4)$ applied to the initial set $\{(0,y)\mid y\ge 3\}$''. Indeed, if we would apply the \P-to-\P\ rule to those manually colored positions, we would recolor exactly the upper red cells from the coloring of the middle segment. And it is then easy to verify that, starting from a colored position in the upper segment, the missing rule $(2,3)$ reaches \N-positions.  Similar to the lower segment, the upper segment has type $i, ii, iii$ moves from non-colored positions as indicated in the Figure~\ref{fig:tiling_example}, for positions with $x$-coordinates $1,2,3\pmod 4$ respectively. Hence, the upper segment defines correctly the outcomes of the ruleset $S$.

    Finally, by the observation in Example~\ref{ex:lines} it is clear that $\Gamma_0\cap\Gamma_1=\varnothing$, and $\Gamma_1\cap\Gamma_2=\varnothing$.
\end{proof}

  

% Figure environment removed


We generalize the left-most ruleset in Figure~\ref{fig:additive}. 
Let us prove that the update rules in Figure~\ref{fig:122133color} are correct. The result holds in a somewhat more general statement, a symmetric restriction of Definition~\ref{def:addsym}. 
  
\begin{prop}[{\sc symmetric additive}]\label{prop:symadd}
    Consider symmetric additive rulesets of the form $S=\{(a,b),(b,a),(a+b,b+a)\}$, where $a\ge b/2$.  The set of $\P$-positions is given by the following updates. The set of terminal positions is $T_L\cup T_U\cup T_M$, where $T_L=\{(x,y)\mid y<a\}$, $T_U=\{(x,y)\mid x<a\}$ and $T_M=\{(x,y)\mid x< b, y< b\}$. Then, update the middle segment by the rules $r_1=(2a+b,2b+a)$ and $r_2=(2b+a,2a+b)$, by initializing with $T_M$. Update the lower (upper) segment via the rule $r_1$ ($r_2$), by initializing using the lower and upper terminal positions respectively.  
\end{prop}
\begin{proof}
We prove it for the case $S=\{(1,2),(2,1),(3,3)\}$. Similar to Section~\ref{sec:2dresemble1d}, one can check that the method of proof generalizes to $a<b\le 2a$. Clearly the initial coloring $\{(x,y) \mid x=0 \text{ or } y=0  \text{ or } x=y=1\}$ as in Figure~\ref{fig:additive} corresponds to the set of terminal \P-positions. Suppose that the updates are correct (i.e., colored if and only if a \P-position) for all positions $\x<\x'$. We must verify that $\x'\in\P$ if and only if cell $\x'$ has been colored. By symmetry, it suffices to justify {\bf green} and {\bf blue}. Suppose first that $\mathrm{color}(\x')=\mathbf{green}$. It was updated from {\bf green}, which is a combination of the moves $(2,1)$ and $(3,3)$. By induction, the cell $\x'-(5,4)+(4,2)=\x'-(1,2)\in \N$. Therefore, the converse of Lemma~\ref{lem:three_move_PtoP} (iii) implies that $\x'\in\P$. 

We claim that it suffices to prove that every position below the line of slope $4/5$ with offset $(1,0)$, and between the terminal green cells and the first green update, is an $\N$-position. Namely, induction can be used as follows: suppose $\x'$ is non-colored below the line of slope $4/5$ with offset $(1,0)$. Then, by induction $\x'-(5,4)\in \N$. Hence, there is a move  $\x'-(5,4)-\s\in \P$. This same move can be used from $\x'$, and clearly, $\x'-\s\in \mathbf{green}$, which by induction is in \P. 

To prove the claim, notice that the leftmost cells in that region are $(3,1)$, $(4,2)$, and $(5,3)$. And $(3,1)-(2,1)\in \P$, $(4,2)-(1,2)\in \P$ and $(5,3)-(3,3)\in\P$. Similarly, every position in the set $\{(x,y) \mid 0<y<4, x\ge y+2\}$ has a move to a green terminal \P-position.

To complete the proof, we next study the middle region. One can check manually that the first non-colored region is a subset of $\N$. The proof continues by induction. If the outcomes of the middle segment are correct until the $k$th update, then we translate the $\N$-positions between the $(k-1)$th and $k$th updates, by using the updates $(4,5)$ and $(5,4)$. These translated positions will be $\N$-positions using the same moves as on the lower level. We have illustrated a way of inductive proof that does not depend on the symmetry of the move set and hence can be used more generally. The yellow and brown cells in the lower left corner are \N-positions by a base case computation. The yellow and brown cells above the third update level of the coloring automaton are copy-pasted from the level just below and are \N-positions by induction. The remaining cells at that level (between colored cell levels) are all covered by the \N-positions from the base case, namely the terminal blue cells correspond to the `difference' of the inductive case and the level to be proved.  
\end{proof}

The word `difference' in the final paragraph of the proof, allows for some overlap, as is often the case of the coloring automaton. 

The cases where $a<b/2$ have more complexity as is evident by Figure~\ref{fig:1551add2552}, and we delay further analysis.


% Figure environment removed


\subsection{An additive rule built on an arithmetic progression}
In view of Figure~\ref{fig:coloring123446}, we define a family of {\sc two-dimensional subtraction} that depends on just one parameter. 
\begin{defi}[{\sc arithmetic additive}]
Let $a\ge 1$. Then 
{\small $
S=\{(a,2a),(3a,4a),(4a,6a)\}
$ }is an {\sc arithmetic additive} ruleset. 
\end{defi}
We leave it as an open problem to verify that the coloring scheme updates in Figure~\ref{fig:coloring123446} corresponds to the middle segment's \P-positions of the arithmetic additive ruleset $S=\{(1,2),(3,4),(4,6)\}$.  
% Figure environment removed


\section{Some open problems}\label{sec:picopen}
A classification problem emerges from this experimental section. For a given $k\in \n$, which rulesets have exactly $k$ outcome segments? We have run some c-code and found instances for $k=1,\ldots ,6$. See the pictures in this section for $k>3$. Figure~\ref{fig:symasym} illustrates two 4-segmentations (a symmetric and an asymmetric) where the segmentation settles very early. In contrast, we display Figures~\ref{fig:3move4segm_thin} and \ref{fig:3move4segm_fat}, where the eventual segmentation settles a lot slower. In the latter case, a thick part of the middle region turns out not to shape a segment, but a semi-infinite (periodic) strip. We do not currently have any argument or intuition to point at why the latter of those two rulesets has a much wider (lower) middle segment.    Figures~\ref{fig:5segm} and \ref{fig:6segm} display a 5 and 6-segmentation that satisfies the hypothesis of Conjecture~\ref{conj:maxsegm}. Finally, we display a ruleset for which we believe that ``outcome segmentation'' is not the correct concept (Figures~\ref{fig:chaos} and \ref{fig:chaos2}).
\begin{conj}[{\sc three-move} Segmentation]\label{conj:threesegm}
If $|S|=3$, there is an outcome segmentation.
\end{conj}
\begin{problem}\label{prob:OS}
Give necessary and/or sufficient conditions such that a finite ruleset $S$ forms an outcome segmentation.
\end{problem}
Experimental results lead us to the following conjecture. 
\begin{conj}[Maximum Segmentation]\label{conj:maxsegm}
 For each $k\in \n$, $k\ne 2$, there exists a ruleset of size $k$ that has a $(k+1)$-segmentation. For each $k\in \n$, there is no ruleset of size $k$ that has a $(k+2)$-segmentation.
\end{conj}

 
 

% Figure environment removed
% Figure environment removed
 % Figure environment removed

 % Figure environment removed

Some rulesets seem harder than others. We have computed the outcomes of the game $S=\{(0,1), (1,0), (1,1), (1,2), (2,2), (2,51), (4,3), (4,4), (13,1)\}$ up to 20000 by 20000; see Figure~\ref{fig:chaos} for the first few outcomes. And the patterns do not seem to stabilize; see Figure~\ref{fig:chaos2}. Why does every glider appear to have the same slope? Does this contribute to a simpler solution, then if the slopes of outgoing gliders were different (as for example in the rule 110 CA updates)? At least, if true, a linear search would suffice to identify the full two-dimensional pattern. Still, an arbitrary line of rational slope does not appear to define periodic outcomes. This ruleset certainly seem more complicated than those that satisfy outcome segmentation.

The next problem concerns the outcome patterns. 
\begin{problem}\label{prob:percolation}
A question related to percolation:  Which outcome segments for given rulesets satisfy {\em \N-percolation}, defined as an infinite path of connected \N-positions within a segment? For example, in Figure~\ref{fig:additive} the middle segment of the right-most picture satisfies \N-percolation whereas the left-most does not. In Figure~\ref{fig:3move4segm_fat} both middle segments percolate, whereas in Figure~\ref{fig:3move4segm_thin} only the upper middle segment percolates.
\end{problem}
In all our examples where we define a coloring scheme, we were able to set the initial colors at the terminal \P-positions of the ruleset. We have not yet studied a more advanced case, such as in the 4-segmentations in Figures~\ref{fig:3move4segm_fat} and~\ref{fig:3move4segm_thin}, where the patterns that (visibly) define the outcome segments start later.

\begin{problem}
    Study a coloring scheme for a ruleset where outcome segmentation does not start at the origin, e.g. Figures~\ref{fig:3move4segm_thin} and~\ref{fig:3move4segm_fat}.
\end{problem}

The following generalizes the notion of ``symmetric ruleset'' as defined in \cite{Flammenkamp_1997} to several dimensions. A ruleset $S$ is {\em max-symmetric} if there is a unique maximum $\s_{\max}$ such that, if $\s\in S\setminus\{\s_{\max}\}$, then $\s_{\max}-\s\in S$. Such rulesets have fewer \P-to-\P~ update rules, which seems to weaken  tendencies to pattern formation.
See Figures~\ref{fig:maxsym1},~\ref{fig:maxsym2},~ \ref{fig:maxsym3} and \ref{fig:maxsym4} for some examples. Note that the same-slope-glider formation in Figure~\ref{fig:chaos2} does not seem to happen for {\sc five-move max-symmetric}. There appear to be two outer outcome segments. However, the middle segments do not stabilize to any visible regularity. 
We write $S_{\mathrm{ms}}$ for a max-symmetric ruleset, where max-symmetric moves have been omitted. For example $S=\{(1,7),(7,1),(9,3),(3,9),(10,10)\}=\{(1,7),(7,1),(10,10)\}_{\mathrm{ms}}$
\begin{problem}
    Study max-symmetric rulesets.
\end{problem}

% Figure environment removed
 
 % Figure environment removed

% Figure environment removed
  
% Figure environment removed

% Figure environment removed


% Figure environment removed
  
  % Figure environment removed
  
  \clearpage
\noindent {\bf Acknowledgements.} We thank Raj Aryan Agrawal, Dr. Koki Suetsugu, Dr. Tomoaki Abuku, Dr. Hironori Kiya and Prof. Hideki Tsuiki for inspiring discussions regarding patterns and games. 

  %\bibliographystyle{plain} % Other options: unsrtnat, unsrt, abbrvnat, plainnat, rusnat
  %\bibliography{my_ref.bib}
%\printbibliography



%% For Subtraction Game in More than one dimension Bib for arxiv submission
 \begin{thebibliography}{1}
\bibitem{abuku2019combination} Tomoaki Abuku, Masanori Fukui, Ko Sakai, and Koki Suetsugu. On a combination of the cyclic Nimhoff and subtraction games. Tsukuba Journal of Mathematics 43, no. 2 (2019): 241-249.

\bibitem{althofer1995superlinear}   Ingo Alth{\"o}fer, and   B{\"u}ltermann J{\"o}rg ,. Superlinear period lengths in some subtraction games. Theoretical computer science 148, no. 1 (1995): 111-119.

\bibitem{austin1976impartial} Richard Bruce Austin. Impartial and partisan games (M.sc. thesis). Department of Mathematics
and Statistics, Calgary, Alberta, 1976.
\bibitem{berlekamp2004winning} Elwyn R Berlekamp, John H Conway, and Richard K Guy. Winning Ways for Your Mathematical Plays. AK Peters/CRC Press, 2004 (1st edition Academic Press, 1982).

\bibitem{bouton1901nim} Charles L Bouton. Nim, a game with a complete mathematical theory. The Annals of Mathematics, 3(1/4):35–39, 1901.

\bibitem{cairns2010ultimately} Ultimately bipartite subtraction games. Australasian Journal of Combinatorics, 48:213–220, 2010.

\bibitem{carvalho2020combinatorics} Alda Carvalho, Jo{\~a}o  Neto, and Carlos Santos. Combinatorics of jenga. The Australasian Journal of Combinatorics, 76(1):87–104, 2020.

\bibitem{carvalho2012recursive} Alda Carvalho, Carlos P Santos, C{\'a}tia Lente Dias, Francisco Coelho,  Jo{\~a}o Pedro Neto, and Sandra Vinagre. A recursive process related to a partizan variation of wythoff. Integers, 12(5):1029–1045, 2012.

\bibitem{cohensius2019cumulative} Gal Cohensius, Urban Larsson, Reshef Meir, and David Wahlstedt. Cumulative subtraction games. The Electronic Journal of Combinatorics, 26(P4.52), 2019.

\bibitem{coleman2020department} Deidra Coleman, Jack Good, Michael Smith, Jennifer Travis, and Mark Daniel Ward. The periodicity of nim-sequences in two-element subtraction games. Integers, 20:G5, 2020.

\bibitem{Cook2017}  Matthew Cook, Urban Larsson, and Turlough Neary. A cellular automaton for blocking queen games. Natural Computing, 16:397–410, 2017.

\bibitem{duchene2019wythoff} Eric Duch{\^e}ne, Aviezri S. Fraenkel, Vladimir Gurvich, Nhan Ho, Clark Kimberling, and Urban Larsson.  Wythoff visions. In Games of No Chance 5, volume 70, pages 35–87. Mathematical Sciences Research Institute Publica-
tions, Cambridge University Press, 2019.

\bibitem{duchene2022partizan} Eric Duch{\^e}ne, Marc Heinrich, Richard Nowakowski, and Aline Parreau. Partizan subtraction games. Integers, 21B(A8), 2022.


\bibitem{eppstein2018faster} David Eppstein. Faster evaluation of subtraction games. In 9th International Conference on Fun with Algorithms (FUN 2018), 2018.


 \bibitem{fox2014aperiodic} Nathan Fox. On aperiodic subtraction games with bounded nim sequence. Preprint arXiv:1407.2823, 2014.

\bibitem{Flammenkamp_1997}Achim Flammenkamp. Lange Perioden in Subtraktions-Spielen. 
der Fakult\"at f"ur Mathematik der Universit\"at Bielefeld, 1997.

\bibitem{FoxLar} Nathan Fox and Urban Larsson. An aperiodic subtraction game of nim-dimension two. Journal of Integer Sequences., 18(15.7.4.), 2015.


\bibitem{fraenkel2004complexity} Aviezri S Fraenkel. Complexity, appeal and challenges of combinatorial games. Theoretical Computer Science, 313(3):393–415, 2004.

\bibitem{fraenkel1987partizan} Aviezri S Fraenkel and Anton Kotzig. Partizan octal games: partizan subtraction games. International Journal of Game Theory, 16(2):145–154, 1987.

\bibitem{fraenkel1991nimhoff} Nimhoff games. Journal of Combinatorial Theory, Series A, 58(1):1–25, 1991.

\bibitem{friedman2019geometric} Eric Friedman, Scott M. Garrabrant, Ilona K. Phipps Morgan, Adam Landsberg, and Urban Larsson.  Geometric analysis of a generalized Wythoff game. In Games of No Chance 5, volume 70, pages 125-168. Mathematical Sciences Research Institute Publications, Cambridge University Press, 2019.

\bibitem{golomb1966mathematical} Golomb, Solomon W. A mathematical investigation of games of “take-away”. Journal of Combinatorial Theory, 1(4):443-458, 1966.


\bibitem{Gurvich} Vladimir Gurvich and Michail Vyalyi. Computational hardness of multidimensional subtraction games. International Computer Science Symposium in Russia, pages 237-249, 2020.

\bibitem{ho2015expansion} Nhan Bao  Ho. On the expansion of three-element subtraction sets. Theoretical
Computer Science, 582:35-47, 2015.

\bibitem{larsson2011blocking} Urban Larsson. Blocking Wythoff nim. The Electronic Journal of Combinatorics, 18(1):P120, 2011.

\bibitem{larsson2012operator} Urban Larsson. The $\star$-operator and invariant subtraction games. Theoretical Computer Science 422 (2012): 52-58.


\bibitem{Larsson2012GDWN} Urban Larsson. A generalized diagonal wythoff nim. Integers, 12(5):1003–1027, 2012.

\bibitem{larsson2013impartial}  Urban Larsson. Impartial games and recursive functions. Chalmers Tekniska H\"ogskola (Sweden), 2013.

\bibitem{Larsson2014split} Urban Larsson. Wythoff nim extensions and splitting sequences. Journal of Integer Sequences., 17(5):14.5.7, 2014.

\bibitem{LarssonUrbanNeil} Urban Larsson, Neil A. McKay, Richard J. Nowakowski, and Siegel. Angela
A. Wythoff partizan subtraction. International Journal of Game Theory,
47(2):613–652, 2018.

\bibitem{larsson2013heaps} Urban Larsson and Johan W{\"a}stlund. From heaps of matches to the limits of computability. The Electronic Journal of Combinatorics, 20(P41), 2013.

\bibitem{larsson2014maharaja}  Urban Larsson and Johan W{\"a}stlund. Maharaja nim, Wythoff’s queen meets
the knight. Integers, 14:G05, 2014.

\bibitem{siegel2013combinatorial} Aaron N Siegel. Combinatorial game theory, volume 146. American Mathematical Society, 2013.

\bibitem{siegel2005finite} Finite Excluded Subtraction Sets and Infinite Modular
Nim. Dalhousie University, 2005.

\bibitem{ward2016conjecture} Mark Daniel Ward. A conjecture about periods in subtraction games. Preprint
arXiv:1606.04029, 2016.

\bibitem{wythoff1907modification} A modification of the game of nim. Nieuw Archief voor
Wiskunde, 7(2):199–202, 1907.

\bibitem{zermelo1913anwendung} {\"U}ber eine Anwendung der Mengenlehre auf die Theorie des Schachspiels. Zermelo, Ernst. In Proceedings of the fifth international congress of mathematicians, vol. 2, pp. 501-504. Cambridge: Cambridge University Press, 1913.

\bibitem{zhang2021linearity} Shenxing Zhang. On linearity of the periods of subtraction games. Preprint \url{https://zhangshenxing.gitee.io}. 
\end{thebibliography}
\end{document}

