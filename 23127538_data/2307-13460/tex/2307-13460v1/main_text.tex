% ****** Start of file apssamp.tex ******
%
%   This file is part of the APS files in the REVTeX 4.2 distribution.
%   Version 4.2a of REVTeX, December 2014
%
%   Copyright (c) 2014 The American Physical Society.
%
%   See the REVTeX 4 README file for restrictions and more information.
%
% TeX'ing this file requires that you have AMS-LaTeX 2.0 installed
% as well as the rest of the prerequisites for REVTeX 4.2
%
% See the REVTeX 4 README file
% It also requires running BibTeX. The commands are as follows:
%
%  1)  latex apssamp.tex
%  2)  bibtex apssamp
%  3)  latex apssamp.tex
%  4)  latex apssamp.tex
%
\documentclass[%
reprint,
superscriptaddress,
%groupedaddress,
%unsortedaddress,
%runinaddress,
%frontmatterverbose, 
%preprint,
%preprintnumbers,
%nofootinbib,
%nobibnotes,
%bibnotes,
amsmath,amssymb,
aps,
%pra,
%prb,
%rmp,
%prstab,
%prstper,
%floatfix,
]{revtex4-1}

\usepackage{mathtools}
\usepackage{dcolumn}% Align table columns on decimal point
\usepackage{bm}% bold math
%\usepackage{hyperref}% add hypertext capabilities
%\usepackage[mathlines]{lineno}% Enable numbering of text and display math
%\linenumbers\relax % Commence numbering lines

%\usepackage[showframe,%Uncomment any one of the following lines to test 
%%scale=0.7, marginratio={1:1, 2:3}, ignoreall,% default settings
%%text={7in,10in},centering,
%%margin=1.5in,
%%total={6.5in,8.75in}, top=1.2in, left=0.9in, includefoot,
%%height=10in,a5paper,hmargin={3cm,0.8in},
%]{geometry}

\usepackage[dvips]{graphicx}
\usepackage{amsmath,amssymb,amsthm,mathrsfs,amsfonts,dsfont,amscd,keyval}
\usepackage{mathtools,mathrsfs}
\usepackage{yfonts} %  for \vcentcolon
\usepackage{ytableau}
\usepackage{subfigure, epsfig}
\usepackage{braket}
\usepackage{bm}
\usepackage{fancyhdr}
\usepackage{enumerate}
\usepackage{color}
\usepackage{hyperref}
\usepackage{tabularx}
\usepackage{times}
\hypersetup{colorlinks=true, linkcolor=blue, citecolor=blue, urlcolor=black}
\usepackage{multirow}
\usepackage{float}
\usepackage{pdfpages}
\usepackage{pgffor}
%\usepackage{fancybox, graphicx}
%\usepackage[skins]{tcolorbox}f
%
%\newtcolorbox[auto counter]{tbox}[2][]{%
	%    enhanced, float=hbt, drop fuzzy shadow southeast,
	%    colback=white!5!white, colframe=white!50!black,
	%    width= .97\columnwidth,sharp corners,boxrule=0.8pt,
	%    title={Algorithm \thetcbcounter: #2}, #1
	%}
%
\newcommand{\overbar}[1]{\mkern 1.5mu\overline{\mkern-1.5mu#1\mkern-1.5mu}\mkern 1.5mu}
\DeclareMathOperator{\Ima}{Im}
\renewcommand\vec{\mathbf}
\newcommand{\wt}{\widetilde}

\theoremstyle{definition}
\newtheorem{definition}{Definition}[section]
\newtheorem*{example}{Example}
\newtheorem{proposition}[definition]{Proposition}

\theoremstyle{plain}
\newtheorem{theorem}[definition]{Theorem}
\newtheorem{lemma}[definition]{Lemma}
\newtheorem{corollary}[definition]{Corollary}

\theoremstyle{remark}
\newtheorem*{remark}{Remark}
\newtheorem*{observation}{Observation}
%
\newcommand{\comments}[1]{}
\newcommand{\han}{\textcolor{olive}}
\newcommand{\zimu}{Zimu: \textcolor{brown}} 
\newcommand{\ketbra}[2]{\vert #1 \rangle \langle #2 \vert}
\newcommand{\dbar}{d\hspace*{-0.08em}\bar{}\hspace*{0.1em}}

\newcommand{\Hamilt}{{H}}

\makeatletter
\AtBeginDocument{\let\LS@rot\@undefined}
\makeatother

%----------------------------------------------------------------------------------------------------------------------------------------

\begin{document}

\title{Fundamental causal bounds of quantum random access memories}
	
\author{Yunfei Wang}

\affiliation{Martin A. Fisher School of Physics, Brandeis University, Waltham, MA 02453, USA}

\author{Yuri Alexeev}

\affiliation{Computational Science Division, Argonne National Laboratory, Lemont, IL 60439, USA}

\affiliation{Chicago Quantum Exchange, Chicago, IL 60637, USA}

\affiliation{Department of Computer Science, The University of Chicago, Chicago, IL 60637, USA}

\author{Liang Jiang}

\affiliation{Chicago Quantum Exchange, Chicago, IL 60637, USA}

\affiliation{Pritzker School of Molecular Engineering, The University of Chicago, Chicago, IL 60637, USA}

\author{Frederic T. Chong}

\affiliation{Department of Computer Science, The University of Chicago, Chicago, IL 60637, USA}

\author{Junyu Liu}

\affiliation{Chicago Quantum Exchange, Chicago, IL 60637, USA}

\affiliation{Department of Computer Science, The University of Chicago, Chicago, IL 60637, USA}

\affiliation{Pritzker School of Molecular Engineering, The University of Chicago, Chicago, IL 60637, USA}

\affiliation{Kadanoff Center for Theoretical Physics, The University of Chicago, Chicago, IL 60637, USA}

\affiliation{qBraid Co., Chicago, IL 60615, USA}

\affiliation{SeQure, Chicago, IL 60615, USA}
	
\date{\today}% It is always \today, today,
%  but any date may be explicitly specified

\begin{abstract}
    Quantum devices should operate in adherence to quantum physics principles. Quantum random access memory (QRAM), a fundamental component of many essential quantum algorithms for tasks such as linear algebra, data search, and machine learning, is often proposed to offer $\mathcal{O}(\log N)$ circuit depth for $\mathcal{O}(N)$ data size, given $N$ qubits. However, this claim appears to breach the principle of relativity when dealing with a large number of qubits in quantum materials interacting locally. In our study we critically explore the intrinsic bounds of rapid quantum memories based on causality, employing the relativistic quantum field theory and Lieb--Robinson bounds in quantum many-body systems. In this paper, we consider a hardware-efficient QRAM design in hybrid quantum acoustic systems. Assuming clock cycle times of approximately $10^{-3}$ seconds and a lattice spacing of about 1 micrometer, we show that QRAM can accommodate up to $\mathcal{O}(10^7)$ logical qubits in 1 dimension, $\mathcal{O}(10^{15})$ to $\mathcal{O}(10^{20})$ in various 2D architectures, and $\mathcal{O}(10^{24})$ in 3 dimensions. We contend that this causality bound broadly applies to other quantum hardware systems. Our findings highlight the impact of fundamental quantum physics constraints on the long-term performance of quantum computing applications in data science and suggest potential quantum memory designs for performance enhancement.
\end{abstract}

\maketitle

%------------------------------------------------------------------------------------------------------------------------------------------------

\section{Introduction}
Quantum algorithms executed on quantum computers have the potential to provide significant benefits over their classical equivalents \cite{shor1999polynomial,grover1997quantum,harrow2009quantum}. However, a large portion of these algorithms necessitate data inputs, which are predominantly classical in today's computing sector. Consequently, the creation of interfaces that bridge classical and quantum processors may be a crucial factor in achieving broad and applicable quantum advantages. To illustrate, for an algorithm that classically handles $N$ data in polynomial time, we would need to transfer the classical data at an exponentially fast rate in $N$ to ensure that the overall quantum speedup is exponential. The development of such a fast quantum memory is a critical component for the future of the quantum industry, akin to the quantum version of data centers that merge quantum networks and quantum memories \cite{Liu:2022ubu}.

Quantum random access memory (QRAM) is a promising option to establish this efficient interface. For classical data of size $N$, QRAM circuits can achieve the upload of classical data within $\mathcal{O}(\log N)$ circuit depth and $\mathcal{O}(N)$ qubits. We primarily focus on the Bucket Brigade QRAM design due to its favorable circuit depth scaling. This design allows QRAM to serve as a fast quantum memory, as the coherence time of qubits does not scale rapidly with the qubit number $N$ and does not heavily rely on quantum information propagation within the lattice. Hence, our main concern lies in the bounds of from speed propagations. Numerous quantum algorithms, such as the Harrow--Hassidim--Lloyd algorithm used for linear systems of equations \cite{harrow2009quantum}, may depend on QRAM to ensure significant accelerations when dealing with dense classical data. Indeed, QRAM is also a fundamental component in many quantum machine learning applications \cite{Liu:2023coc,Liu:2022rxb,Jiang:2022oie}. However, the implementation of large-scale, fault-tolerant QRAM presents a substantial challenge in quantum hardware \cite{di2019methods,connorthesis,hann2019hardware,di2020fault,hann2021resilience}. While encouraging endeavors have been made in the development of QRAM, such as spin photon-based \cite{chen2021scalable} and hybrid quantum acoustic systems \cite{hann2019hardware}, there still exists ambiguity surrounding the timeline and methodology for large-scale QRAM production. We highlight that one potential hindrance for QRAM at an asymptotic scale is intricately tied to core principles in quantum physics: relativity and causality.

Quantum hardware is realized by quantum many-body systems, whose low-energy descriptions are Lorentz-invariant, local, relativistic quantum field theories. There is no up-to-date evidence about superluminal information transfer and significant Lorentz violation \cite{Adams:2006sv,Mattingly:2005re}. Assuming that interactions are local, it seems that fundamentally one cannot ensure an $\mathcal{O} (\log N)$ circuit depth for $\mathcal{O}(N)$ qubits asymptotically. Here we make a simple estimation with the following two assumptions. Suppose we place qubits as particles split by the lattice spacing $a =10^{-6} \text{m}$ in one dimension. The total length of the qubit chain is thus given by $L = a N$, where $N$ is the number of qubits. We further assume that running the total QRAM circuit takes time $T=\Delta T \log N$, where $\Delta T = 10^{-3} \text{s}$ per clock cycle. From causality, for 1 dimension chain QRAM system, we simply have
\begin{align} \label{Naive}
\frac{L}{T} = \frac{{Na}}{{\Delta T\log N}} \le c~,
\end{align}
where $c$ is the speed of light. Taking $c = 3 \times 10^8 \text{ m/s}$, we get an upper bound on the number of qubits $N \le 8.9\times 10^{12}$. That is, in the above setup, we can maximally have around 1000 billion qubits to deal with 1000 billion data in total.
The number itself is comparable to the numbers of training parameters and training date size in a commercial large language model in machine learning. For instance, \texttt{GPT-3} from OpenAI \cite{brown2020language} uses 175 billion training parameters to train 45,000 billion bytes of text data. If a quantum algorithm is somehow serving as part of the service with QRAM, it is approaching the bounds where causality is allowed to have exponential speedup \cite{Liu:2023coc}. The causality bound we describe is, as far as we know, original for generic quantum hardware. 

The objective of this paper is to substantiate and provide a rigorous framework for the statement above. Causality constraints in quantum many-body physics can be derived from the Lieb--Robinson bound for a local Hamiltonian \cite{lieb2004finite,Roberts:2016wdl,cramer2008locality}. In the low-energy description, basic gates in QRAM circuits can be represented as local interactions within relativistic quantum field theory. In our study we utilize  both approaches as examples to depict maximum information transmission using commutators in hybrid quantum acoustic systems that actualize QRAM \cite{hann2019hardware}. Not only will our work establish robust foundations for causal restrictions in QRAM, but it will also offer practical advice on quantum hardware design to enhance performance against causality bounds (refer to FIG.~\ref{FundamentalLimit3} and  FIG.~\ref{FundamentalLimit4}). Furthermore, our exploration of quantum gates as Feynman diagrams in relativistic quantum field theories (see FIG.~\ref{FDQG}) will contribute to our understanding of quantum hardware under extreme conditions. 

We will make two extra comments before we start detailed discussions. First, in the derivation of the bound Eqn.~(\ref{Naive}), the fundamental time scale $\Delta T$ is considered to be the clock cycle time for implementing QRAM circuits. In the actual hardware designs, this clock cycle time might be related or bounded by the gate operation time or the decoherence time of quantum hardware, representing operational time costs of each clock cycle. One can note that a smaller $\Delta T$ might lead to stronger bounds in $N$, since a smaller time separation might lead to smaller space separation for the same velocity. However, this does not indicate that we should not improve the gate time. For instance, a larger choice of $\Delta T$ will lead to higher infidelities for QRAM \cite{hann2021resilience}. Thus, a comprehensive design of QRAM hardware will lead to hybrid considerations including both causality bounds and fidelity. Second, we will primarily use the Hamiltonian of transmon and phonon in Eqn.~(\ref{QRAMEffHamiltonian}), where the speed limit is from the speed of sound determined from the hardware. However, one could also use classical communication, or alternative Hamiltonian constructions (and their corresponding quantum field theories ), and the speed limit could be given by the speed of light. To introduce classical communication, corresponding classical fields will be considered to enable the QRAM coupling, and could be treated as classical field backgrounds in the quantum systems (Thus, classical communication is a way to improve the causality bound from sound speed to the light speed, consistent with \cite{xu2023systems}). Thus, our results will be generic for known quantum materials following causality, considering the speed of light as the upper bound. Both of the issues will be discussed in more details in the Supplemental Materials. 



Our paper is structured as follows. In Section \ref{sec:hardware} we will discuss our approach to QRAM hardware models within the context of hybrid quantum acoustic systems. Section~\ref{sec:qft} will provide a comprehensive discussion of causality bounds utilizing quantum field theory and Lieb--Robinson bounds, along with their implications for quantum hardware design. In Section~\ref{sec:conc} we will conclude and offer future perspectives. Some detailed proofs and technical discussions are presented in the Supplemental Materials. 

\section{QRAM hardware}\label{sec:hardware}

Our study employs QRAM hardware, the technical details of which have been meticulously outlined in references \cite{connorthesis,hann2019hardware}. We direct interested readers to these sources for a comprehensive analysis. In this section we offer a succinct overview of the procedure used for implementing quantum gates to extract data from a QRAM system comprising $N$ qubits, and we compute the total time of operation. The Hamiltonian that controls this QRAM system is as follows:

\begin{equation}\label{QRAMEffHamiltonian}
    \begin{split}
        & H = \omega_q q^{\dagger} q - \frac{\alpha}{2} q^{\dagger} q^{\dagger} q q\\
        & + \sum_k \bigg( \omega_k m_k^{\dagger} m_k + g_k q^{\dagger} m_k + g_k^{*} q m_k^{\dagger} \bigg)\\
        & + \sum_j \bigg( \Omega_j q^{\dagger} e^{i \omega_j t} + \text{H.C.} \bigg)~.
    \end{split}
\end{equation}

Here $q$ represents the annihilation operator for the transmon qubit, while $m_k$ represents the annihilation operator for the $k^{th}$ phonon mode. The transmon is described as an anharmonic oscillator with a Kerr nonlinearity parameter $\alpha$, and it is coupled to the $k^{th}$ phonon mode with a coupling strength of $g_k$. The external drives applied to the transmon with frequencies $\omega_j$, which are represented as $H_d = \sum_j \Omega_j q^{\dagger} e^{i \omega_j t} + \text{H.C.}$. In this content, $\Omega_j$ denotes the amplitude of the drive for the $j^{th}$ frequency, and H.C. stands for the Hermitian conjugate.

The operation of the QRAM system is underpinned by two pivotal stages: initialization and routing. The initialization phase engages a SWAP gate, whereas the routing phase necessitates the integration of a controlled-SWAP gate and a SWAP gate. Throughout the initialization phase, the address qubit is systematically integrated into the system for individual initialization. For a QRAM comprising $N$ qubits, $\log{N}$ steps are required to accomplish comprehensive initialization. Each step, denoted as $k$ in FIG.~\ref{kthStep}, encompasses routing the $k$th address qubit for ($k-1$) times, followed by a SWAP operation to facilitate initialization.

% Figure environment removed

Upon comprehensive initialization of the QRAM system, information extraction from the database becomes feasible. The precise extraction protocol is contingent on the type of information being accessed, as elaborated in \cite{hann2019hardware}. At the leading order, the overall operation times for extracting both classical and quantum information are essentially equivalent, with only insignificant discrepancies in numerical factors, where we use the time calculated in \cite{hann2019hardware}:

\begin{equation} \label{OperationTimeScale}
    T_{c \text{ or } q} \sim \tau_0 \times \log^2{N}~.
\end{equation}

Here we define $\frac{\pi}{g_1} + \frac{\pi}{g_2} \equiv \tau_0$. Unlike the linear dependence on $\log{N}$ in the total operation time of the simple example (Eqn.~(\ref{Naive})), the specific example we discuss in this paper exhibits a quadratic dependence on the circuit depth $\log{N}$.

\section{Lieb--Robinson bound, relativistic quantum field theory, and causality bounds}\label{sec:qft}

\subsection{Lieb--Robinson bound}

Considering the locality constraint inherent in QRAM,  employing a local Hamiltonian is essential. However, the one referenced in \cite{hann2019hardware} may not be ideally suited in scenarios around the speed of light (or sound) because of the unfettered coupling between the transmon qubit and phonon modes. As a result, we commence by formulating a local Hamiltonian that mirrors the system's properties:

\begin{equation} \label{FullDiscreteHamiltonian3D}
\begin{split}
    & H = H_{\mathcal{U}} + H_{\phi} + H_I\\
    = & \sum_{\vec{r}} \bigg( \frac{\mathcal{P}_{\vec{r}}^2}{2m} + \sum_{j}^{\nu} \sum_{\alpha,\beta} \frac{\lambda_j}{2} (\mathcal{U}_{\alpha}(\vec{r}) - \mathcal{U}_{\alpha}(\vec{r} + j \vec{e_{\beta}}))^2 \bigg)\\
    & + \int d^d \mathbf{x} \frac{1}{2} \bigg( \dot{\phi}(t,\mathbf{x})^2 + |\nabla \phi(t,\mathbf{x})|^2 \bigg)\\
    & + \int d^d \mathbf{x} \bigg[ \sum_{\vec{r}} \sum_{\alpha} \bigg( \mathcal{C} \phi(t,\mathbf{x}) \mathcal{U}_{\alpha}(\vec{r}) \bigg) + \frac{h}{4!} \phi^4(t,\mathbf{x}) \bigg]~.\\
\end{split}
\end{equation}

We consider the general case for $d = 1, 2, 3$ dimensions, where the indices $\alpha$ and $\beta$ refer to the directions in these dimensions. The lattice sites in the solid are denoted by the vector $\vec{r}$, and distortions in the $\alpha$ direction at site $\vec{r}$ are represented as $\mathcal{U}_{\alpha}(\vec{r})$. Additionally, $\vec{e}_{\beta}$ denotes the unit vector in the $\beta$ direction, and $\lambda_j$ represents the coupling constant between sites $\vec{r}$ and $\vec{r} + j \vec{e_{\beta}}$. The construction of $H_{\mathcal{U}}$ is detailed in the Supplemental Materials. $\phi$ is a massless scalar field, and $H_{\phi}$ corresponds to the standard Hamiltonian in quantum field theory textbooks \cite{Coleman:2018mew}. The interaction Hamiltonian is analogous to the interactions in Eqn.~(\ref{QRAMEffHamiltonian}), as explained also in the Supplemental Materials. 

For mathematical simplicity, throughout most of the paper we assume the isotropy of the solid. Otherwise, we would just have different frequency for modes traveling in different directions. As demonstrated in Eqn.~(\ref{FullDiscreteHamiltonian3D}), the coupling constants $\lambda_j$ remain the same for different values of $\alpha$. The construction of this Hamiltonian is fundamentally anchored in reintroducing locality into the system. We represent the phonon modes utilizing the position distortion operator, denoted as $\mathcal{U}_{\alpha}$. To confine couplings between distantly located sites, we impose a ceiling limit symbolized by $\nu$. This approach intuitively aligns with the concept that atoms distanced further apart are likely to be weakly coupled. Unlike the approach of modeling the transmon qubit as a single harmonic oscillator with a drive, as depicted in Eqn.~(\ref{QRAMEffHamiltonian}), we now portray it as a massless scalar field. Through transforming the drive frequency into the frequency of the $\phi$ operator, we further endorse locality, guided by the principles of special relativity and the architectural groundwork of quantum field theory. It is also important to note that we are working with a two-level system, hence qubits.


Thus, we present the Lieb--Robinson bound for the Hamiltonian given by Eqn.~(\ref{FullDiscreteHamiltonian3D}), which provides us with a bound on the velocity of information propagation. Define Weyl operator: $$W(f) \equiv \exp \bigg\{i \sum_{n \in \Lambda_L} \operatorname{Re}[f(n)]q_n + \operatorname{Im}[f(n)]p_n \bigg\}.$$ Here $f(n)$ is a complex function and $n$ denote sites in the whole system represented as $\Lambda_L$ here. 


\begin{proposition}
    For Hamiltonian Eqn.~(\ref{FullDiscreteHamiltonian3D}), we have the following Lieb--Robinson bound for the operator commutator norm:

    \begin{equation}
        \begin{split}
            ||[\tau_t^{\Lambda} (W & (f)),  W(g)]|| \\
            \leq C \sum_{x,y \in \Lambda_L} & |f(x)| |g(y)|\\
            & e^{-\mu m \big[ d(X,Y) - c_{\omega,\lambda} \operatorname{max} \big( \frac{2}{\mu},e^{(\mu/2)+1} \big) |t| \big]}~,
        \end{split}
    \end{equation}

    where 

    \begin{equation}
        c_{\omega,\lambda} = \bigg( d \sum_{j}^{\nu} \frac{\lambda_{j}}{m} \bigg)^{1/2}~.
    \end{equation}
\end{proposition}

A meticulous and rigorous elaboration of the aforementioned causality statement is provided in the Supplemental Materials. Upon performing optimization with respect to $\mu$, we derive the subsequent Lieb--Robinson velocity:

\begin{equation} \label{dDSpeedLimit}
    v_h \leq 4 \bigg(d \sum_{j=1}^{\nu} \frac{\lambda_j}{m} \bigg)^{1/2}~.
\end{equation}

This bound acts as a maximum speed for transmitting information in a system described by Eqn.~(\ref{FullDiscreteHamiltonian3D}). Interestingly, the couplings to the transmon qubit do not affect this bound because they commute with the chosen Weyl operator. Additionally, Eqn.~(\ref{dDSpeedLimit}) is directly influenced by the parameter $\nu$, indicating that the presence of couplings between distant sites increases the speed limit. This is intuitive as stronger interactions between distant sites pose a challenge to locality. We note that the stronger the coupling between each site of the solid, the faster information can travel. Also noteworthy is the relationship between the speed limits in different dimensions (for $d = 1,2,3$):

\begin{equation} \label{2D3DSpeedLimitTo1D}
    \begin{split}
        v_h^{(d)} = \sqrt{d} v_h^{(1d)} = 4 \sqrt{d} \bigg(\sum_{j=1}^{\nu} \frac{\lambda_j}{m} \bigg)^{1/2}~.
    \end{split}
\end{equation}

Hence, a direct relationship exists between the material's dimensionality and the speed limit. Materials with higher dimensionality possess increased speed limits due to enhanced coupling opportunities and degrees of freedom. Furthermore, more pronounced dependencies are observable when considering the ratio between the number of qubits and the clock cycle time, a topic we will delve into in the subsequent sections.

\subsection{Causality bounds in relativistic quantum field theory} \label{CausalLim}

Alternatively, we propose a fresh perspective to comprehend the system by examining it in a low-temperature context. In this regime, only the long-wavelength modes with small $k$ are consequential. We implement a coarse-graining of the system by setting $k \rightarrow 0$ and subsequently perform a Legendre transformation to derive the subsequent Lagrangian. A comprehensive discussion on the coarse-graining procedure as well as the reason for the selection of this specific Lagrangian density can be found in the Supplemental Materials.

\begin{equation} \label{FullLagrangianDensitySimplified}
    \begin{split}
        \mathscr{L} = & \frac{\rho}{2} |\mathcal{\Dot{U}}|^2 - \sum_{\alpha,\beta}^{d} \frac{\lambda^{(d)}}{2}\bigg( \frac{\partial \mathcal{U}_{\alpha}}{\partial x^{\beta}} \bigg)^2 + \frac{1}{2} \partial_{\mu} \phi \partial^{\mu} \phi\\
        & - \sum_{\alpha} \mathcal{C} \phi \mathcal{U}_{\alpha} - \frac{h}{4!} \phi^4~.
    \end{split}
\end{equation}

As a result of the coarse-graining procedure, the site distortions $\mathcal{U}_{\alpha}(\vec{r})$ metamorphose into a continuous function $\mathcal{U}_{\alpha}(t, \vec{x})$. The coupling constant $\lambda^{(d)}$ in $d$ dimensions bears a direct correlation to the original couplings $\lambda_j$ in $d$ dimensions. By performing an expansion up to the first order in $k$, a relationship is established between the two coupling constants, as follows:

\begin{equation}\label{dDEquivalence}
    \lambda^{(d)} = d \sum_{j = 1}^{\nu} \lambda_j a j^2~.
\end{equation}

Our attention is devoted primarily to the causal bound of the system, a state attained via the development of the Lagrangian. By hypothesizing that the Lagrangian can be expressed as an integral of a density dependent on the derivatives of $\mathcal{U}_{\alpha}$, we can curtail interactions between distantly located sites. This is accomplished by analyzing the commutator $\big[ \pi_{\alpha}(x), \mathcal{U}_{\beta}(y) \big]$, an approach akin to the computation involved in the Lieb--Robinson bound. We initially underscore the fact that this quantity invariably amounts to zero when $x$ and $y$ are spacelike separated, a phenomenon attributed to Lorentz symmetry:

\begin{equation}
    \big[ \pi_{\alpha}(x), \mathcal{U}_{\beta}(y) \big] = 0 \hspace{3mm} \text{for $(x - y)^2 < 0$}~.
\end{equation}

This ensures that special relativity is respected and information cannot travel faster than the speed of light. To get more information, we focus on the analytical structure of $\big[ \pi_{\alpha}(x), \mathcal{U}_{\beta}(y) \big]$ when $x$ and $y$ are timelike separated, specifically for the case $d = 1$ as given in Eqn.~(\ref{LocalityConstraintSimplified}) up to leading order (for explicit calculation, see in the Supplemental Materials):

\begin{equation} \label{LocalityConstraintSimplified}
     \big[\mathcal{U}_{\beta}(y), \pi_{\alpha}(x) \big] < \frac{1}{\pi} \frac{e^{- (r - \sqrt{\lambda^{(1d)}/\rho} t) \epsilon}}{\big( r - \sqrt{\lambda^{(1d)}/\rho} t \big)}~.
\end{equation}

In 1D, the commutator is exponentially suppressed if $(r - \sqrt{\lambda^{(1d)}/\rho} t) > 0$. This sets the velocity limit for the system in 1D:

\begin{equation}
    v^{(1d)} = \sqrt{\lambda^{(1d)}/\rho}~.
\end{equation}

The velocity limits in different dimensions ($d = 1, 2, 3$) follow a  relation similar to that in the Lieb--Robinson bound:

\begin{equation} \label{dDSpeedLimit2}
    v^{(d)} = \bigg(\frac{\lambda^{(d)}}{\rho}\bigg)^{1/2} = \sqrt{d} v^{(1d)} = \sqrt{d} \bigg(\frac{\lambda^{(1d)}}{\rho}\bigg)^{1/2}~,
\end{equation}

It is not surprising that the bound is independent of $\phi$ and interaction terms, as the interaction parts commute with $\mathcal{U}_{\alpha}$. By relating the coupling constants in the discrete and continuous cases through coarse graining, we observe that the two speed limits are effectively the same. The numerical factor $j$ differs because of the imprecision of coarse graining when considering farther site couplings and ignoring higher derivative terms in Eqn.~(\ref{FullLagrangianDensitySimplified}). However, if we consider only couplings between nearest neighbors, the two speed limits are equal. This can be understood intuitively, as $\partial \mathcal{U}_{\alpha} / \partial x^{\beta}$ is the continuous counterpart of the coupling between nearest neighbors. Couplings between more distant sites can  be accurately described only by including higher derivatives of $\mathcal{U}_{\alpha}$ in the Lagrangian, which challenges our assumptions of locality.

\subsection{Scattering process as quantum gates}

The Feynman diagrams corresponding to Eqn.~(\ref{FullLagrangianDensitySimplified}) can reproduce the wave mixing process described in \cite{hann2019hardware}. By utilizing the Kerr nonlinearity in Eqn.~(\ref{QRAMEffHamiltonian}), the authors achieved wave mixing between external drives and phonon modes, resulting in the realization of quantum gates for QRAM.

We observe that the interaction term $\phi \mathcal{U}_{\alpha}$ allows us to convert any incoming $\phi$ field into a phonon mode in the Feynman diagrams generated by Eqn.~(\ref{FullLagrangianDensitySimplified}). By attaching $\mathcal{U}_{\alpha}$ to the tree-level diagram via $\phi^4$ interactions, we can mix the transmon waves and phonon modes, reproducing the wave mixing process described in \cite{hann2019hardware}.

Since $\phi$ is a scalar field, its propagator at the tree-level process, for dimensions $d = 1,2,3$, is a classical number. We can incorporate the propagators into the vertices and introduce effective coupling constants $g_1$ and $g_2$. Consequently, we construct the following effective Lagrangian density. We put $l_0$ in just to keep $g_1$ and $g_2$ dimension fixed for different $d$. (Details of this wave-mixing processes are presented in the Supplemental Materials.) 

\begin{equation} \label{EffectiveLagrangianDensity}
    \begin{split}
        \mathscr{L} = & \frac{\rho}{2} |\mathcal{\Dot{U}}|^2 - \sum_{\alpha,\beta}^{d} \frac{\lambda^{(d)}}{2}\bigg( \frac{\partial \mathcal{U}_{\alpha}}{\partial x^{\beta}} \bigg)^2\\
        & + \frac{1}{2} \partial_{\mu} \phi \partial^{\mu} \phi - \frac{g_1}{4 l_0^{2-d}} \sum_{\alpha} \phi^2 \mathcal{U}_{\alpha}^2\\
        & - \frac{g_1}{2 l_0^{2-d}} \sum_{\alpha \neq \beta} \phi^2 \mathcal{U}_{\alpha} \mathcal{U}_{\beta}  - \frac{g_2}{3! l_0^{2-d}} \sum_{\alpha} \phi \mathcal{U}_{\alpha}^3\\
        &  - \frac{g_2}{l_0^{2-d}} \sum_{\alpha \neq \beta \neq \gamma} \phi \mathcal{U}_{\alpha} \mathcal{U}_{\beta} \mathcal{U}_{\gamma} - \frac{g_2}{2 l_0^{2-d}} \sum_{\alpha \neq \beta} \phi \mathcal{U}_{\alpha}^2 \mathcal{U}_{\beta}~.
    \end{split}
\end{equation}

Hence, there exists a correspondence between the Feynman diagrams generated by Eqn.~(\ref{EffectiveLagrangianDensity}) and the quantum gates employed in QRAM.

% Figure environment removed

Through direct calculations, as presented in the Supplemental Materials, we determine the time scale for the three gates that will be utilized. The time scale for the SWAP gates, beam-splitter, and CZ gates are determined respectively as follows: $t_{\text{sw}} = \frac{\pi}{2 g_1}$, $t_{\text{bs}} = \frac{\pi}{4 g_1}$, and $t_{\text{cz}} = \frac{\pi}{g_2}$. Importantly, these time scales are in the identical form as the times calculated in \cite{hann2019hardware}.

To construct QRAM, an essential building block is the controlled-SWAP gate, as depicted in FIG.~\ref{kthStep}. In our specific example, the controlled-SWAP gate is realized using a combination of basic gates: beam-splitter + CZ + beam-splitter, illustrated in FIG.~\ref{CSWAP}. 

% Figure environment removed

The detailed demonstration of this construction can be found in the Supplemental Materials and the reference \cite{hann2019hardware}. It's important to note that the decomposition of controlled-SWAP gates shown in FIG.~\ref{CSWAP} is a specific case, assuming channel 1 is a qubit and channels 2 and 3 can be bosonic modes. Under the assumption that all modes are two-level systems, there might be even simpler decompositions of the controlled-SWAP gates, such as using 3 Toffoli gates.

Reference \cite{gao2018entangling} provides experimental demonstrations for the controlled-SWAP gates in the specific design shown in FIG.~\ref{CSWAP}. However, that paper did not demonstrate the coherence of the controlled-SWAP due to significant dephasing in channel 1. On the other hand, reference \cite{chapman2022high} uses a slightly different method and improved device coherence properties to demonstrate deterministic controlled-SWAP operation.

For classical communication systems, electromagnetic fields can be treated classically with interactions based on specific designs. The key is to identify the medium through which the information is carried. In the case of classical light speed communication, Maxwell's equations can be manipulated to derive the wave equation for electromagnetic waves, which reveals the speed limit of $c = 3 \times 10^8 \text{ m/s}$ for these waves. This result is well-documented in classical electromagnetic textbooks \cite{feynman2011feynman,jackson2021classical,griffiths2017introduction}. The upper bound of the memory system can be estimated by analyzing the specific design of the communication system, similar to previous sections.

\subsection{Limit of QRAM by locality}

The speed limits imposed by Eqn.~(\ref{dDSpeedLimit}) or Eqn.~(\ref{dDSpeedLimit2}) place constraints on the size of the QRAM. In a system governed by Hamiltonian Eqn.~(\ref{FullLagrangianDensitySimplified}), all information propagation is subject to these bounds, which are equivalent to the Lieb--Robinson bound of Hamiltonian Eqn.~(\ref{FullDiscreteHamiltonian3D}). The information stored in the database can travel at most a distance of $N \times a$ (where $N$ is the total number of qubits and $a$ is the separation distance) by applying a series of quantum gates generated by Eqn.~(\ref{FullLagrangianDensitySimplified}). Therefore, the total time required for the information to travel is given by the operation time calculated in Eqn.~(\ref{OperationTimeScale}). It is essential to ensure that the speed at which information travels is slower than the bound described by Eqn.~(\ref{dDSpeedLimit}) or Eqn.~(\ref{dDSpeedLimit2}).

\begin{equation}
    \begin{split}
        \frac{N}{\log^2{N}} \leq \frac{4 \sqrt{d} \big(\frac{\pi}{g_{1}} + \frac{\pi}{g_{2}} \big)}{a} \bigg(\sum_{j=1}^{\nu} \frac{\lambda_j}{m} \bigg)^{1/2} \hspace{2mm} \text{or} \hspace{2mm} \bigg(\frac{\lambda^{(1d)}}{\rho}\bigg)^{1/2}~.
    \end{split}
\end{equation}

We illustrate the relationship between the material constants used in constructing QRAM and the size bound in FIG.~\ref{FundamentalLimit3}.

% Figure environment removed

We assume that the clock cycle time used in FIG.~\ref{FundamentalLimit3} is $10^{-3}$ second. The identification of this is not important, since what we care about is the total operation time. In terms of dissecting the total operation time into either individual gates or clock cycles will not make a big difference. The rough $\log{N}$ scaling for time will work for both identifications. We also consider the speed limit to be approximately the sound speed in the resonator, which is around $6 \times 10^3 \text{ m/s}$. However, it is possible to exceed this bound with alternative QRAM designs, in which case the ultimate speed limit would be the speed of light, $c = 3 \times 10^8 \text{ m/s}$. 

The upper bounds for QRAM systems in higher dimensions are larger mainly because of the increased number of qubits they can accommodate (speed limits are also higher, as discussed before). In 2D systems, the total number of qubits is proportional to $N^2$ and $N^3$ for 3D. This difference in scaling allows for a significantly higher bounds in higher-dimensional hardware designs, as illustrated in FIG.~\ref{FundamentalLimit3}. The quadratic dependence of the total operation time on the circuit depth in this model does not significantly impact the overall results (at most up to one order of magnitude). It is specific to the model considered in this paper. For other models, one can calculate the corresponding dependence and extrapolate the results presented here.

Regarding differences between dimensions, we specifically mention an architecture design described in \cite{xu2023systems}, where quantum teleportation is used for qubit routing instead of SWAP actions. While most quantum gates in this design are confined by the sound speed of the 2D solid, the routing steps employing quantum teleportation are constrained by the speed of light. As the system size increases, a larger proportion of qubit distances are covered by quantum teleportation, resulting in only a few steps being confined by sound speed. Based on this observation, we estimate the maximum bounds for the QRAM size $N$ in this 2D design to be on the order of $\mathcal{O}(10^{20}) \sim \mathcal{O}(10^{22})$ qubits, whose performance is somewhere between 2D and 3D in FIG.~\ref{FundamentalLimit3}. For explicit calculations, please see the Supplemental Materials. 

In FIG.~\ref{FundamentalLimit4}, the horizontal axis represents the coupling constants, while the vertical axis represents the square of the speed limit. We consider $g_1$ and $g_2$ to have approximately the same value, both smaller than 1. This assumption is fundamental in perturbation theory, as used in \cite{hann2019hardware}, and in quantum field theory in our analysis.

% Figure environment removed

With more realistic clock cycle times, the restriction on the number of qubits, $N$, becomes even larger, extending to up to $10^{14}$ qubits. However, it is crucial not to excessively lean toward the left in FIG.~\ref{FundamentalLimit4}, as doing so would result in exceptionally prolonged operation times for QRAM. To elevate the upper bound on $N$, one would need to identify a solid where atoms are strongly coupled or, alternatively, decrease the coupling constants between the transmon qubit and the resonator to extend the clock cycle times. Nevertheless, the upper bounds depicted in FIG.~\ref{FundamentalLimit3} and FIG.~\ref{FundamentalLimit4} already suffice for the requirements of most quantum algorithms involving large quantities of qubits.

\section{Conclusion}\label{sec:conc}
In this paper we establish a universal bound on the maximum number of qubits permissible in quantum random access memory (QRAM), derived from causality and relativity principles. We substantiate the causality bounds through rigorous applications of relativistic quantum field theories and Lieb--Robinson bounds in quantum apparatus, using hybrid quantum acoustic systems as illustrative examples. While the bounds are elucidated by using specific examples, we propose that similar bounds would likely be present in general quantum devices constructed by local quantum many-body physics systems that implement QRAM. Our research not only offers essential insights into the constraints of quantum devices under extreme circumstances but also proposes strategies to optimize hardware designs, ensuring rapid quantum memory in the asymptotic context.

Our work initiates a new avenue of research that delves into the fundamental aspects of quantum memories and their computational constraints. Future research trajectories may encompass the development of strategies to optimize and circumvent causality bounds, comprehension of the ultimate capabilities of quantum memories for specific quantum algorithm applications, the conception of innovative alternative quantum hardware models with augmented connectivity and nonlocal interactions, and the exploration of implications on the causality bound from hybrid quantum-classical co-designs that take into account data contents. We present these potential research themes as fertile ground for future investigation.



%------------------------------------------------------------------------------------------------------------------------------------------------

\begin{acknowledgments}
We thank Scott Aaronson for helpful discussions about QRAM in NSF Workshop on Quantum Advantage and Next Steps in 2022. We thank Yongshan Ding, Jens Eisert and Shifan Xu for many helpful conversations. Y.A. acknowledges support from DOE Q-NEXT and DE-AC02-06CH11357. L.J. acknowledges support from the ARO(W911NF-23-1-0077), ARO MURI (W911NF-21-1-0325), AFOSR MURI (FA9550-19-1-0399, FA9550-21-1-0209), NSF (OMA-1936118, ERC-1941583, OMA-2137642), NTT Research, Packard Foundation (2020-71479), and the Marshall and Arlene Bennett Family Research Program. This material is based upon work supported by the U.S. Department of Energy, Office of Science, National Quantum Information Science Research Centers. This work is funded in part by EPiQC, an NSF Expedition in Computing, under award CCF-1730449; in part by STAQ under award NSF Phy-1818914; in part by NSF award 2110860; in part by the US Department of Energy Office of Advanced Scientific Computing Research, Accelerated Research for Quantum Computing Program; and in part by the NSF Quantum Leap Challenge Institute for Hybrid Quantum Architectures and Networks (NSF Award 2016136) and in part based upon work supported by the U.S. Department of Energy, Office of Science, National Quantum Information Science Research Centers. F.T.C. is Chief Scientist for Quantum Software at ColdQuanta and an advisor to Quantum Circuits, Inc. J. L. is supported in part by International Business Machines (IBM) Quantum through the Chicago Quantum Exchange, and the Pritzker School of Molecular Engineering at the University of Chicago through AFOSR MURI (FA9550-21-1-0209).
\end{acknowledgments}



\bibliographystyle{apsrev4-1}
\bibliography{QRAM.bib}

\newpage
\pagebreak
\clearpage
\foreach \x in {1,...,\the\pdflastximagepages}
{
	\clearpage
	\includepdf[pages={\x,{}}]{Supplemental_Material.pdf}
}

	
	
\end{document}
%
% ****** End of file apssamp.tex ******
