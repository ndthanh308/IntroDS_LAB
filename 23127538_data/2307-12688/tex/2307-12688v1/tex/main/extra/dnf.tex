Let $\Pi$ be a fixed, finite set of non-strict inequalities over the set of variables $X$.  We suppose that $\Pi$ is closed under binding variables to zero, that is, 
if $\pi \in \Pi$ then $\pi[x/0] \in \Pi$. (This is needed to support the $d[\lambda/0]$ calculation in the snd and rcv rules.) It follows that $true, false \in \Pi$.

Let $\delta = \pi | \neg \delta | \delta_1 \wedge \delta_2$.

To demonstrate that $\delta_1 \rightarrow \delta_2$ is computable we need to find a symbolic representation of $\delta_i$ that allows $\delta_1 \rightarrow \delta_2$ to be expressed in the same representation.   
We elect to represent $\delta_i$ as a disjunct of prime implicants where each prime implicant is a (non-redundant) conjunct of non-strict inequalities (positive literals) and negated inequalities (negative literals). This gives a compact and intuitive representation which is amenable to derivation using an SMT solver supporting difference logic or linear arithmetic.

To see this, first recall that $\delta_1$ and $\delta_2$ can be transformed to equi-satisfiable SMT formulae $f_1$ and $f_2$, with only a polynomial increase in space, by introducing fresh propositional (Tseytin) variables such that if $f_i$ holds then $t_i$ is true iff $\delta_i$ holds. Then the formula $f = (t_1 \rightarrow t_2) \wedge f_1 \wedge f_2$ is equi-satisfiable with $\delta_1 \rightarrow \delta_2$.

Now solve the SMT instance $f$. Suppose that $f$ is satisfiable with an assignment 
$\theta = \{ x_1 \mapsto c_1, \ldots, x_n \mapsto c_n \}$ where $X = \{ x_1, \ldots, x_n \}$ and $c_i \in \mathbb{Q}$.

Given the assignment $\theta$, construct the cube $c = q_1 \wedge \ldots q_n$ where
$q_i = \pi_i$ if $\pi_i$ holds for $\theta$ and $q_i = \neg \pi_i$ otherwise.
Observe that $c \models f$ since the truth or falsity of $f$ is defined only by the truth or falsity of inequalities of $\Pi$.  Now find a minimal subcube $c_1$ of $c$ such that $c_1 \models f$. To do so, consider eliminating $q_i$ and put $c’ = c \setminus q_i$.
If $c’ \wedge \neg f$ is unsatisfiable then $c’ \models f$ hence $q_i$ is redundant, else $q_i$ must be retained. Consider each $q_i$ in turn to derive a minimal subcube $c_1$.  Such a minimal subcube is called a prime implicant.

Now repeat the process with $f’ = f \wedge (\neg c_1)$ to give another minimal cube $c_2$, continuing until $f \wedge (\neg c_1) \ldots (\neg c_m)$ is unsatisfiable.  Then $\delta_1 \rightarrow \delta_2 \equiv f \equiv \vee_{j = 1}^{m} c_j$.

\begin{example}
Suppose $\Pi = \{  p, q, r \}$ and let $p = (x \leq y)$, $q = (y – z \leq -2)$
and $r = (z – y \leq 1)$. Now consider computing a representation for $f_0 = \delta_1 \rightarrow \delta_2$ where $\delta_1 = p$ and $\delta_2 = (q \vee r)$.
\begin{itemize}

\item
Consider the assignment $\theta_0 = \{  x \mapsto 1, y \mapsto 0, z \mapsto 0 \}$ of $f_0$.
Then $c = (\neg p) \wedge (\neg q) \wedge r$ and $c_0 = (\neg p)$.
Put $f_1 = f_0 \wedge (\neg c_0) = f_0 \wedge p$.

\item
Now consider the assignment $\theta_1 = \{  x \mapsto 0, y \mapsto 1, z \mapsto 0 \}$ of $f_1$.
Then $c = p \wedge (\neg q) \wedge r$ and $c_1 = (p \wedge r)$.

Put $f_2 = f_1 \wedge (\neg c_1) = f_0 \wedge p \wedge (\neg p \vee \neg r) \equiv f_0 \wedge p \wedge (\neg r)$.

\item
Now consider the assignment $\theta_2 = \{  x \mapsto 0, y \mapsto 1, z \mapsto 4 \}$ of $f_2$.
Then $c = p \wedge q \wedge (\neg r)$ and $c_2 = (p \wedge q)$.

Put $f_3 = f_1 \wedge (\neg c_1) = f_0 \wedge p \wedge (\neg p \vee \neg r) \wedge (\neg p \vee \neg q) \equiv f_0 \wedge p \wedge (\neg r) \wedge (\neg q)$.

\item
But $f_3$ is unsatiable hence $f_0 = c_1 \vee c_2 \vee c_3$.

\end{itemize}
\end{example}

Note that we could work in disjunctive normal form (DNF) and calculate $\delta_1 \rightarrow \delta_2 = (\neg \delta_1) \vee \delta_2$ but applying de Morgan's and a distribution law to place $\neg \delta_1$ is DNF. However, we would still need to check that each disjunct is satisfiable (which follows immediately in the above since the cube $c$ is derived from an assignment) and there is no guarantee that the disjuncts will be short (prime) or few in number.