
Isolated configurations capture the behaviour of types without any influences beyond itself. All transitions reflect local actions, hidden to anything but itself.
An isolated configuration \CfgIso*\ is a tuple comprised of a valuation of clocks \ValClocks*\ and type \TypeS*, and may be written as \IsoCfg*.
\begin{note}
    Any configuration with a transition $\IsoCfg\Act\IsoCfg'$ may be written as $\IsoCfg\Act$, to mean that there exists some possible transition \ProgAction*\ such that a new configuration \IsoCfg*'\ can be reached.
\end{note}
The semantics of isolated configurations is presented in~\cref{fig:configs_iso_rule}.


% ~ interact
\NewDocumentCommand{\CfgIsoRuleInteract}{s}{%
\IfBooleanT{#1}{$}% mm wrap
\infer{%[{\RName[act]}]{% * judgement
\CfgIso2{\TypInteract}
\Act[\TypComm_j\IMsgType_j]
\CfgIso1{\Resets1{\ValClocks}2{\RSet_j}}2{\TypeS_j}
}{% * premise
\ValClocks\models\Const_j
&
j\in\SetI
}
\IfBooleanT{#1}{$}% mm wrap
}

% ~ recursion
\NewDocumentCommand{\CfgIsoRuleRecursion}{s}{%
    \IfBooleanT{#1}{$}% mm wrap
    \infer{%[{\RName[rec]}]{% * judgement
        \CfgIso2{\CfgRecDef.\TypeS}
        \Act
        \CfgIso''
    }{% * premise
        \CfgIso2{\TypeS\Subst[\CfgRecDef.\TypeS][\CfgRecLabel]}
        \Act
        \CfgIso''
    }
    \IfBooleanT{#1}{$}% mm wrap
}

% ~ iso-time
\NewDocumentCommand{\CfgIsoRuleIsoTime}{s}{%
    \IfBooleanT{#1}{$}% mm wrap
    %
    \CfgIso
    \Act[\ValTime]
    \CfgIso1{\ValClocks+\ValTime}
    %
    % \quad
    % %
    % {\RName[tick]}
    \IfBooleanT{#1}{$}% mm wrap
}

\endinput

% ~ template
\NewDocumentCommand{\CfgIsoRule}{s}{%
    \IfBooleanT{#1}{$}% mm wrap
    \infer{%[{\RName[temp]}]{% * judgement
        \CfgIso
    }{% * premise
        \CfgIso''
    }
    \IfBooleanT{#1}{$}% mm wrap
}

Rule \LblCfgIsoUnfold*\ handles recursive types, unfolding them by substituting any occurrence of the corresponding call \CfgRecDef*, in the proceeding type \TypeS*, with the sequence of types between \TypeS*\ and any recursive call \CfgRecDef*. The resulting type \TypeS*'\ may be some intermediate type between the recursive definition and a corresponding call.
Rule \LblCfgIsoTick*\ allows time to pass over the clocks.

The \LblCfgIsoInteract*\ rule allows a configuration to perform any interaction with constraints satisfied by the current valuation of clocks. The resulting configuration has the type \TypeS*_j\ corresponding to the interaction performed, along with any conditional resets being applied to \ValClocks*.
