
The behaviour of communication between types can be captured using social configuration. A social configuration \CfgSoc*\ is a triple with a similar composition as an isolated configuration, but with the addition of a queue \Queue*. A social configuration \CfgSoc*\ may be written as \SocCfg*\ just as \CfgIso*\ can be written as \IsoCfg*.
The transitions for social configurations, shown in~\cref{fig:configs_soc_rule}, often depend on a corresponding isolated configuration transition, comprised of the same \ValClocks*\ and \TypeS*.

% ~
\NewDocumentCommand{\CfgSocRuleSend}{s}{%
    \IfBooleanT{#1}{$}% mm wrap
    \infer{%[{\RName[snd]}]{% * judgement
        \CfgSoc%
        \Act[\TypSend\Msg]%
        \CfgSoc''%
    }{% * premise
        \CfgIso%
        \Act[\TypSend\Msg]%
        \CfgIso''%
    }%
    \IfBooleanT{#1}{$}% mm wrap
}

% ~
\NewDocumentCommand{\CfgSocRuleEnqu}{s}{%
    \IfBooleanT{#1}{$}% mm wrap
    %
    \CfgSoc%
    \Act[\TypRecv\Msg]%
    \CfgSoc3{\Queue;\Msg}%
    %
    % \quad
    % %
    % {\RName[que]}
    %
    \IfBooleanT{#1}{$}% mm wrap
}

% ~
\NewDocumentCommand{\CfgSocRuleRecv}{s}{%
    \IfBooleanT{#1}{$}% mm wrap
    \infer{%[{\RName[rcv]}]{% * judgement
        \CfgSoc3{\Msg;\Queue}%
        \Act[\SiltAction]%
        \CfgSoc''%
    }{% * premise
        \CfgIso%
        \Act[\TypRecv\Msg]%
        \CfgIso''%
    }%
    \IfBooleanT{#1}{$}% mm wrap
}

% ~
\NewDocumentCommand{\CfgSocRuleTime}{s}{%
\IfBooleanT{#1}{$}% mm wrap
\infer{%[{\RName[time]}]{% * judgement
\CfgSoc%
\Act[\ValTime]%
\CfgSoc'%
}{% * premise
\CfgIso%
\Act[\ValTime]%
\CfgIso'%
%
&%
%
\FutureEn1{\CfgIso}[\TypComm]%
\implies%
\FutureEn1{\CfgIso'}[\TypComm']%
%
&%
%
\forall \ValTime'<\ValTime:%
\CfgSoc1{\ValClocks+\ValTime'}%
\Act|[\SiltAction]%
}%
\IfBooleanT{#1}{$}% mm wrap
}

\endinput

% ~ template
\NewDocumentCommand{\CfgSocRule}{s}{%
    \IfBooleanT{#1}{$}% mm wrap
    \infer{%[{\RName[temp]}]{% * judgement
        \CfgSoc%
    }{% * premise
        \CfgSoc'''%
    }%
    \IfBooleanT{#1}{$}% mm wrap
}

Rule \LblCfgSocSend*\ depends on \LblCfgIsoInteract*, requiring that the message \Msg*\ being sent corresponds to an interaction within \TypeS*, that has its constraints satisfied by \ValClocks*. The resulting configuration, as defined in \LblCfgIsoInteract*, has any resets applied to \ValClocks*'\ and continues as the proceeding type \TypeS*'. The queue is unaffected by this behaviour.

Message reception is handled by two rules, \LblCfgSocEnqu*\ and \LblCfgSocRecv*. The former allows messages to be enqueued into \Queue*, and describes the abstraction of how messages are received over a network.
The latter, \LblCfgSocRecv*, requires a message \Msg*\ to have already been enqueued into \Queue*, so that the system can receive it. Just as with \LblCfgSocSend*, this rule is dependant on \LblCfgIsoInteract*, and yields the same results; with one exception: the message \Msg*\ must be removed from the queue \Queue*. This is described by the silent \SiltAction*, which in this instance describes both of the actions taking place.
Only the message at the head of the queue is received as, following \emph{communication safety}, all messages are expected and able to be received.

The rule \LblCfgSocTime*\ allows an amount of time \ValTime*\ to pass, requiring the following premises hold
\begin{inline}*+
    \item the value of \ValTime*\ is added to all valuations of clocks, by rule \LblCfgIsoTick*
    \item no sending opportunities are missed by the time step
    \item that there exists no smaller time step that would allow a message to be received sooner
\end{inline}
Together, these ensure that as few and as small time steps are taken so that interactions are performed as soon as possible; this is explored further in~\cref{sssec:configs_sys_compatibility}.
\begin{note}
    The notion of \emph{liveness} (see~\cref{def:configs_iso_live}) can be applied to social configurations.
\end{note}
