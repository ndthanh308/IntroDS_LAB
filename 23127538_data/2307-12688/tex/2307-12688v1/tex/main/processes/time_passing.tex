
The time passing function \PFuncTime*\ is defined in~\cref{fig:prc_func_time} for each valid case of \Prc*\ and is undefined for any other.
%
%
%
% Figure environment removed
% \footnotetext{(Used when typing processes.)}
%

The auxiliary functions that \PFuncTime*\ depend on are largely unchanged from~\cite{Bocchi2019} and can be found in~\cref{fig:prc_func_wait,fig:prc_func_neq}.
Together they are used to ensure that time does not pass while a process is waiting to receive a message already in their queue.
% The function \PFuncWait*\ recursively unfolds processes until returning either
% \begin{inline}+
%     %
%     \item a message \PrcMsg*\ that is waiting to be received within the given \ValTime*\ specified by the outer \PFuncTime*\ call
%     %
%     \item the emptyset, signifying no messages are waiting to be received within the given time frame
%     %
% \end{inline}
% In short, \PFuncWait*\ returning something other than the empty set indicates that there exists a process currently trying to receive a message from the returned channel.
% The function \PFuncNEQ*\ is similar returning either the empty set or a non-empty channel buffer.
%
% Together, as used in~\cref{fig:prc_func_time} for the passing time over parallel processes, the intersection of two distinct processes yielding nothing indicates that there does not exist a message in a buffer that is able to be received by a process.
