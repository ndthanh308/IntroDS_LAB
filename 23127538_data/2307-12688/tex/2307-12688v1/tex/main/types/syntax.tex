
The syntax of TOAST (or just \emph{types}) is given in~\cref{eq:types_syntax}.
%
A type \TypeS*\ is a choice $\simplechoice$, recursive definition $\TypRecDef.\TypeS$, call $\TypRecCall$, or termination type $\TypeEnd$.
%
\section{Machine State, Instructions, Naming Registers and Memory Translation}
\label{sec:syntax}
To develop our core logical ideas and explain their intepretation, we need only a handful of instructions, and a machine model corresponding to execution of x86-64 assembly instructions with virtual memory enabled on the CPU.

\subsection{Registers and Memory}
Programs we demonstrate in this paper requires accessing two types of computer resource: registers and memory. A register identifier, $\reg$, is chosen from a fixed finite set of register identifiers,$\regset$. We use these identifiers to access the register values, $\regval \in \regvaltype$. Unlike registers, we do not abstract the memory indices as a special type but instead, for the sake of clarity and ease of representation, we show differently masked machine words, $\loc \in \Loc$, with the subscripts showing the length of a 64-bit machine word after masking, e.g. $\kw{w}_{12}$ is a 12-bit resource which can be obtained after masking 52-bit of a 64-bit word.
This simple syntax in \ref{fig:syntax} includes a simple syntax for values because our language is purely \textit{imperative}, i.e. an instruction does not return a value. Intentionally, we restrict any stream of instruction, $\instrs$, to eventually reduce to $\iskip$ instruction which is injected into unit. Our syntax capture the no-op ``$\iskip$'' and the sequencing construct ``$\iseq{\instr}{\instrs}$'' in their standard forms.
%\begin{wrapfigure}{l}{0.1\textwidth}
  \centering
\newcommand{\commentary}[1]{ & \text{\small\it #1} \\}
\[
  \begin{array}{r@{\;}c@{\;}l}
    \loc & \in & \Loc \\
    \reg & \in & \regset \\
    \regval & \in & \regvaltype \\
    \val & ::= & \vunit

\\
    \instrs & ::= &
    \begin{array}[t]{@{}l@{\hspace{10mm}}l@{}}
    \begin{array}[t]{@{}ll@{}}
      \iskip
                   \commentary{no-op}
      \iseq\instr\instrs
                   \commentary{sequencing}
      \instr
                   \commentary{executing}             
    \end{array}
    \end{array}
    \\
  \end{array}
\]
\caption{Syntax}
\Description{Syntax}
\label{fig:syntax}
\end{wrapfigure}

\todo[inline]{Ismail: this paper is not about the binding of the machine model to Iris. Low-level details like ``injected into unit'' are not really important unless someone asks.}


\noindent Type $\simplechoice$ models a choice among options $i$ ranging over a non-empty set $I$. Each option $i$ is a selection/send action if $\TypComm=\TypSend$, or alternatively a branching/receive action if $\TypComm=\TypRecv$. 
An option sends (resp. receives) a label $l$ and a message of a specified data type \DataType*\ is delineated by $\langle\cdot\rangle$.
The send or receive action of an option is guarded by a time constraint \Const*. After the action, the clocks within \RSet*\ are reset to 0.
%
Data types, ranged over by $T$, $T_i$, $\ldots$ can be sorts (e.g., natural, boolean), or higher order types $(\delta,S)$ to model session delegation.
Only the message label is exchanged when the data type is \BTypeNone*.
%
Labels of the options in a choice are pairwise distinct.
%
Recursion and terminated types are standard.

\paragraph{Remarks on the notation}
One convention is to model the exchange of payloads as a separated action with respect to the communication of branching labels. 
%
In this paper we follow~\cite{Bocchi2015,Yoshida2021}, and model them as unique actions. 
When irrelevant, we omit the payload, yielding a notation closer to that of timed automata.
% blur the distinction between labels and data, abbreviating the pair $\IMsgType$ with \texttt{a} or \texttt{b}.
%
% This yields a notation closer to that of timed automata, simplifying the encoding of examples.
%
% The notation $\IMsgType$ is still when we do type checking of payloads and handle higher order types.
%
% In the paper we often abstract from the distinction between labels and data, and identify the pair $\IMsgType$ with \texttt{a} or \texttt{b}, as in the time automata notation. 
% We omit the payload when not relevant; e.g.: an action $!~{\mathtt{hello\text{<}None\text{>}}}$ may be written as $!~\mathtt{hello}$.
%% $\mathtt{\text{<}None\text{>}}$
%
% We write $\TypInteract|$, omitting the label, in choices with one option, and omit the payload whenever not relevant.

\endinput
The syntax of Timeout Session Types is as in~\cref{eq:types_syntax}; where
\begin{inline}+
  %
  \item \TypeS*\ is a session type
  \item \DataType*\ is a data type
  \item \TypComm*\ is a communication direction
  \item \MsgType*\ is a message type
  %
\end{inline}
Hereafter, a \emph{type} refers to a session type \TypeS*.
% Session types are defined recursively, and specify how a participant may behave and communicate as part of a session.
A type of \TypeEnd*\ describes a termination point of the session; no more types may follow.
\section{Machine State, Instructions, Naming Registers and Memory Translation}
\label{sec:syntax}
To develop our core logical ideas and explain their intepretation, we need only a handful of instructions, and a machine model corresponding to execution of x86-64 assembly instructions with virtual memory enabled on the CPU.

\subsection{Registers and Memory}
Programs we demonstrate in this paper requires accessing two types of computer resource: registers and memory. A register identifier, $\reg$, is chosen from a fixed finite set of register identifiers,$\regset$. We use these identifiers to access the register values, $\regval \in \regvaltype$. Unlike registers, we do not abstract the memory indices as a special type but instead, for the sake of clarity and ease of representation, we show differently masked machine words, $\loc \in \Loc$, with the subscripts showing the length of a 64-bit machine word after masking, e.g. $\kw{w}_{12}$ is a 12-bit resource which can be obtained after masking 52-bit of a 64-bit word.
This simple syntax in \ref{fig:syntax} includes a simple syntax for values because our language is purely \textit{imperative}, i.e. an instruction does not return a value. Intentionally, we restrict any stream of instruction, $\instrs$, to eventually reduce to $\iskip$ instruction which is injected into unit. Our syntax capture the no-op ``$\iskip$'' and the sequencing construct ``$\iseq{\instr}{\instrs}$'' in their standard forms.
%\begin{wrapfigure}{l}{0.1\textwidth}
  \centering
\newcommand{\commentary}[1]{ & \text{\small\it #1} \\}
\[
  \begin{array}{r@{\;}c@{\;}l}
    \loc & \in & \Loc \\
    \reg & \in & \regset \\
    \regval & \in & \regvaltype \\
    \val & ::= & \vunit

\\
    \instrs & ::= &
    \begin{array}[t]{@{}l@{\hspace{10mm}}l@{}}
    \begin{array}[t]{@{}ll@{}}
      \iskip
                   \commentary{no-op}
      \iseq\instr\instrs
                   \commentary{sequencing}
      \instr
                   \commentary{executing}             
    \end{array}
    \end{array}
    \\
  \end{array}
\]
\caption{Syntax}
\Description{Syntax}
\label{fig:syntax}
\end{wrapfigure}

\todo[inline]{Ismail: this paper is not about the binding of the machine model to Iris. Low-level details like ``injected into unit'' are not really important unless someone asks.}


\newcommand{\choice}{\mathtt c}

\newcommand{\simplechoice}{ \{ \choice_i.S_i \}_{i\in I} }

\[        \begin{array}[c]{lcl p{2ex} lcl p{4ex} lcl}
    %
    \TypeS %
    & ::= &                                                                % * 
    \simplechoice    \mid ~
    %
    \TypRecDef.\TypeS   ~\mid ~
    %
    \TypRecCall    ~ \mid ~
    %
    \TypeEnd
    \\[0.3cm]
    \choice & ::= & \TypComm \IMsgType (\delta,\gamma)\qquad \TypComm \in \{
    \TypSend, \TypRecv \
    \}
    \\[0.3cm]
    \DataType            % 
    & ::= & \BTypeNat \mid  \BTypeBool \mid   \ldots\mid \TmpTypBTDelegate
  \end{array}
\]

A type \TypeS\ can be a choice type $\simplechoice$, a recursive definition $\TypRecDef.\TypeS$, a recursive call $\TypRecCall$, and the terminated type $\TypeEnd$.

Type $\simplechoice$ models a choice among a number of options $i$ over a non-empty set $I$. Each option $i$ can be either a selection/send action, if $\square = \, !$, or a branching/receive action, if $\square = \, ?$. An option sends (resp. receives) a label $l$ and a message of a specified data type $T$. The send or receive action of an option is guarded by a time constraint $\delta$. After the action, the clocks within $\lambda$ are reset to 0.
Data types, ranged over by $T$, $T_i$, $\ldots$ can be  sorts (e.g., natural, boolean) or higher order types $(\delta,S)$ to model session delegation.
Labels of the options in a choice are pairwise distinct.
%
Recursion and terminated types are standard.

\paragraph{Remarks on the notation} It is often the case that the exchange of payload is modelled as a separated action with respect to the communication of branching labels. In this paper we model them as a unique action as in ~\cite{}. This yields a notation that is closer to time automata, hence making the encoding of examples more straightforward.
%
In the paper we often abstract from the distinction between labels and data, and identify the pair $\IMsgType$ with $\MsgType$, as in the time automata notation. The notation $\IMsgType$ is still useful in this paper when we do type checking of payloads and handle higher order types.
%
We write $\TypInteract|$, omitting the label, in choices with one option, and omit the payload whenever not relevant.

