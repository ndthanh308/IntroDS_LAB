
\ThmSubjectReduction*

% ~ wf config actions

% * 23
\begin{lemma}
    %
    If \CIso*\ is \emph{well-formed}
    and $\exists\TypComm,\Msg,\Const$ such that $\ValClocks\models\Const$
    and \Trans*{\CIso}:{\CommMsg},
    then $\forall\TypComm',\Const':
        \TypComm\neq\TypComm'
        ~\implies~
        \ValClocks\not\models\Const'$.
    %
\end{lemma}
\begin{proof}
    %
    By~\cref{lem:cfg_wf_then_live} \CIso*\ is \emph{live}.
    We proceed by induction on the transition \Trans*{\CIso}:{\CommMsg}, analysing the last rule applied:
    \begin{inductivecase}
        %
        %
        % ~ act
        \item\NewCase[\LblCfgIsoInteract*]
        then $\TypeS=\simplechoice$. By the premise of \LblTypChoice*\ the hypothesis holds.
        %
        %
        % ~ unfold
        \item\NewCase[\LblCfgIsoUnfold*]
        then $\TypeS=\mu\alpha.\TypeS'$ and \Trans*{\CIso;[\TypeS']}:{\CommMsg}. The hypothesis holds depending on \TypeS*'; similar to~\cref{itm:cfg_wf_then_live_choice} of~\cref{lem:cfg_wf_then_live} and in~\cref{lem:cfgs_trans_wf_pres,lem:configs_trans_compat_pres}.
        %
    \end{inductivecase}
    %
\end{proof}

% * 31
\begin{lemma}\label{lem:cfg_fe_timepass_wf_fe}
    %
    If \VIso*\ is \isFE*\
    and \Trans*{\VSoc}:{\ValTime}[\VSoc']\
    and \VIso*'\ is \emph{well-formed}
    then \VIso*'\ is \isFE*.
    %
\end{lemma}
\begin{proof}
    %
    By~\cref{lem:cfg_wf_then_live} \VIso*'\ is \emph{live} and either $\TypeS=\TypeEnd$ or \VIso*'\ is \isFE*.
    By induction hypothesis $\exists\ValTime,\TypComm,\Msg$ such that \Trans*{\VSoc}:{\ValTime,\CommMsg}[\VSoc']\ and therefore by the premise of \LblCfgSocTime*\ (persistency) $\exists\ValTime',\TypComm',\Msg'$ such that \Trans*{\VSoc'}:{\ValTime',\TypComm',\Msg'}[\VSoc''].
    %
\end{proof}

% ~ delayed processes
% * 35
\begin{lemma}
    %
    If \PFuncTime*\ is defined
    then $\PFuncWait~\cap~\PFuncNEQ=\emptyset$.
    %
\end{lemma}
\begin{proof}
    %
    Suppose by contradiction that $\PFuncWait~\cap~\PFuncNEQ\neq\emptyset$, then \Prc*\ must be structured such that $\Prc\equiv{\PCalcRecv~\mid~\ISes:\QHead~\mid~\Prc'}$, and by~\cref{fig:prc_func_time} \PFuncTime*\ cannot be defined which contradicts with they hypothesis.
    Therefore $\PFuncWait~\cap~\PFuncNEQ=\emptyset$.
    %
\end{proof}

% * 53
\begin{lemma}\label{lem:timepass_func_defined_t}
    %
    For all \Prc*:
    \begin{enumerate}
        \item\label{itm:timepass_func_defined_t_forall} either \PCalc*{\TimePass_{\ValTime}{\Prc}}\ is defined for all \ValTime*,
        \item\label{itm:timepass_func_defined_t_exists} or, $\exists\ValTime'<\ValTime$ such that \PCalc*{\TimePass_{\ValTime'}{\Prc}}\ is defined for.
    \end{enumerate}
    %
\end{lemma}
\begin{proof}
    %
    We proceed by induction on the structure of \Prc*\ such that \PCalc*{\TimePass_{\ValTime'}{\Prc}}\ being defined is determined by the value of \ValTime*.
    \begin{inductivecase}
        %
        \item\NewCase[\PCalc*{\TimePass_{\ValTime}{\mathtt{delay}({\ValTime}''').\Prc'}}] then $\ValTime\leq{\ValTime}'''$ and~\cref{itm:timepass_func_defined_t_exists} holds.
        %
        \item\NewCase[\PCalc*{\TimePass_{\ValTime}{\PCalcRecv}}] if $\PrcAfter=\infty$ then~\cref{itm:timepass_func_defined_t_forall} holds;
        else $\ValTime\leq\PrcAfter$~\cref{itm:timepass_func_defined_t_exists} holds.
        %
    \end{inductivecase}
    %
\end{proof}

% * 52
\begin{lemma}
    %
    If \PCalc*{\TimePass_{\ValTime}{\Prc}}\ is defined then \PCalc*{\TimePass_{\ValTime+\ValTime'}{\Prc}=\TimePass_{\ValTime}{\TimePass_{\ValTime'}{\Prc}}}.
    %
\end{lemma}
\begin{proof}
    %
    \TODO[check?]
    %
\end{proof}

% ~ typing processes
% * 36
\begin{lemma}\label{lem:typ_prc_role_wf}
    %
    If $\VarEnv~\Entails~\Prc~\PrcTyped~\SesEnv,~\ISes:\CIso$
    then \CIso*\ is \emph{well-formed}.
    %
\end{lemma}
\begin{proof}
    %
    We proceed by induction on the typing derivation analysing the last rule applied
    \begin{inline}|
        \item \LblPrcTypDQue*\
        \item \LblPrcTypBranch*\
        \item \LblPrcTypVRecv*\
        \item \LblPrcTypDRecv*\
        \item \LblPrcTypVSend*\
        \item \LblPrcTypDSend*\
    \end{inline}
    Each requires as a premise that $\ValClocks\models\Const$.
    By~\cref{def:types_wf} it follows that \CIso*\ is \emph{well-formed}.
    %
\end{proof}

% * 48
\begin{lemma}
    %
    If $\VarEnv~\Entails~\Prc~\PrcTyped~\SesEnv$
    and $\Prc\equiv\Qrc$
    then $\VarEnv~\Entails~\Qrc~\PrcTyped~\SesEnv$.
    %
\end{lemma}
\begin{proof}
    %
    It follows as standard, with the addition that $\mathtt{delay}(0).P\equiv\Prc$.
    %
\end{proof}

% * 58
\begin{lemma}
    %
    If $\VarEnv~\Entails~\Prc~\PrcTyped~\SesEnv,\ISes:\Queue$
    then $\exists\QHead,\Prc'$ such that $\Prc\equiv\ISes:\QHead~\mid~\Prc'$.
    %
\end{lemma}
\begin{proof}
    %
    We proceed by induction on the typing derivation of \Prc*, analysing the last rule applied
    \begin{inline}|
        \item \LblPrcTypEmptQ*\
        \item \LblPrcTypVQue*\
        \item \LblPrcTypDQue*\
    \end{inline}
    The hypothesis holds for each, and by the \emph{well-formedness} of processes \Prc*'\ does not contain any other \emph{free-queues}.
    \TODO[check]
    %
\end{proof}

% ~ well-formed processes
%
\begin{definition}[Free Queues of Processes]\label{def:prc_free_queues}
    %
    The set of \emph{free queues} \PFuncFQ*[\Prc]\ of a process \Prc*\ is defined inductively as follows:
    \begin{minieq}*%
        \begin{array}[t]{l}%
            \begin{array}[c]{l p{4ex} l p{4ex} l}
                \begin{array}[c]{l c p{0ex} l}
                    \begin{array}[t]{l}%
                        \PFuncFQ[\RecVar], %
                        \mbox{}~\PFuncFQ[\PrcEnd]%
                    \end{array}%
                     & = &  & %
                    \emptyset %
                \end{array}
                %
                 &  &
                \begin{array}[c]{l c p{0ex} l}
                    \PFuncFQ[\PCfgQueue]%
                     & = &  & %
                    \PCalcEndPoints %
                \end{array}
                %
                 &  &
                \begin{array}[c]{l c p{0ex} l}
                    \PFuncFQ[\PCalcScope]%
                     & = &  & %
                    \PFuncFQ[\Prc]\backslash\mkSet[\PrcP\PrcQ,\PrcQ\PrcP] %
                \end{array}
            \end{array}
            %
            \\[-1ex]\\
            \begin{array}[c]{l c p{0ex} l}
                \PFuncFQ[\PCalcRecv] %
                 & = &  & %
                \SetUnion_{i\in\SetI} \PFuncFQ[\Prc_i]
            \end{array}
            %
            \\[-1ex]\\
            \begin{array}[c]{l c p{0ex} l}
                \begin{array}[t]{l}%
                    \PFuncFQ[\PCalcIf], %
                    \mbox{}~\PFuncFQ[\Par[\Prc][\Qrc]]%
                \end{array}%
                 & = &  & %
                \PFuncFQ[\Prc]~\cup~\PFuncFQ[\Qrc]
            \end{array}
            %
            \\[-1ex]\\
            \begin{array}[c]{l c p{0ex} l}
                \begin{array}[t]{l}%
                    \PFuncFQ[\PCalcSend],      %
                    \mbox{}~\PFuncFQ[\PCalcSetTimer],  %
                    \mbox{}~\PFuncFQ[\PFuncDelay[\Const]], %
                    \mbox{}~\PFuncFQ[\PFuncDelay[\ValTime]], \\
                    \mbox{}~\PFuncFQ[\PrcRecDef]             %
                \end{array}%
                 & = &  & %
                \PFuncFQ[\Prc] %
            \end{array}
            %
        \end{array}
    \end{minieq}
    %
\end{definition}
% 

Building upon the notion of well-formed types (\cref{def:types_well_formed}), a well-formed configuration exhibits behaviour without deadlocks, where there is always some action able to be performed in the future, until the termination type is reached.
This behaviour is referred to as \emph{liveness}, and is given in~\cref{def:configs_iso_live}, and depends on \emph{future enabled} actions.
\begin{definition}[Future-enabled Configurations (\isFE*)]\label{def:configs_fe}
   A configuration \VIso*\ is \emph{future-enabled (\isFE*)} if\ %
   $\exists\ValTime$ such that \Trans*{\VIso}:{\ValTime,\CommMsg}[\VIso'].
   The same applies for \VSoc*.
 \end{definition}
\begin{definition}[Live Configurations]\label{def:configs_live}
    \CIso*\ is \emph{live} if
    $\TypeS=\TypeEnd$ or if \VIso*\ is \isFE*.
\end{definition}
The liveness of a configuration indicates whether a deadlock has been reached. There must either be an action immediately viable, some future enabled actions, or the type must have reached termination.

Building upon the notion of well-formed types (\cref{def:types_well_formed}), a well-formed configuration exhibits behaviour without deadlocks, where there is always some action able to be performed in the future, until the termination type is reached.
This behaviour is referred to as \emph{liveness}, and is given in~\cref{def:configs_iso_live}, and depends on \emph{future enabled} actions.
\begin{definition}[Future-enabled Configurations (\isFE*)]\label{def:configs_fe}
   A configuration \VIso*\ is \emph{future-enabled (\isFE*)} if\ %
   $\exists\ValTime$ such that \Trans*{\VIso}:{\ValTime,\CommMsg}[\VIso'].
   The same applies for \VSoc*.
 \end{definition}
\begin{definition}[Live Configurations]\label{def:configs_live}
    \CIso*\ is \emph{live} if
    $\TypeS=\TypeEnd$ or if \VIso*\ is \isFE*.
\end{definition}
The liveness of a configuration indicates whether a deadlock has been reached. There must either be an action immediately viable, some future enabled actions, or the type must have reached termination.

Building upon the notion of well-formed types (\cref{def:types_well_formed}), a well-formed configuration exhibits behaviour without deadlocks, where there is always some action able to be performed in the future, until the termination type is reached.
This behaviour is referred to as \emph{liveness}, and is given in~\cref{def:configs_iso_live}, and depends on \emph{future enabled} actions.
\input{tex/defs/configs/iso/future_en.tex}
\input{tex/defs/configs/iso/live.tex}
The liveness of a configuration indicates whether a deadlock has been reached. There must either be an action immediately viable, some future enabled actions, or the type must have reached termination.
\input{tex/defs/configs/iso/well_formed.tex}
Given a type \TypeS*\ well-formed against \ValClocks*, it holds that a configuration of \IsoCfg*\ will have future enabled actions, and therefore be live.
\begin{note}
    For a \emph{well-formed} \TypeS*, \CfgIso*1{\ValClocks_{0}} is \emph{live}.
\end{note}

Given a type \TypeS*\ well-formed against \ValClocks*, it holds that a configuration of \IsoCfg*\ will have future enabled actions, and therefore be live.
\begin{note}
    For a \emph{well-formed} \TypeS*, \CfgIso*1{\ValClocks_{0}} is \emph{live}.
\end{note}

Given a type \TypeS*\ well-formed against \ValClocks*, it holds that a configuration of \IsoCfg*\ will have future enabled actions, and therefore be live.
\begin{note}
    For a \emph{well-formed} \TypeS*, \CfgIso*1{\ValClocks_{0}} is \emph{live}.
\end{note}


% ~ inversion lemma
% * 30
\begin{lemma}[Inversion Lemma]\label{lem:typ_prc_inversion}
    %
    The following claims hold:
    %
    % ~ send
    \begin{claim}
        %
        If $\VarEnv~\Entails~\PCalcSend~\PrcTyped~\SesEnv$
        then $\exists p, \SesEnv'$
        such that $\SesEnv=\SesEnv',~{p:\CIso;[\TypInteract[i][i]]}$
        and $\exists i\in I$
        such that $\ValClocks\models\Const_i$
        and $\TypComm_i=\TypSend$
        and $\PrcLabel=\MsgLabel_i$
        and $\PFuncNEQ[\SesEnv]=\emptyset$
        and either:
        \begin{enumerate}
            %
            % ~ value
            \item $\DataType_i\neq\CIso[\Const]'$
                  and $\VarEnv~\Entails~\Prc~\PrcTyped~\SesEnv',~{p:\CIso+{\ReSet[]_i};[\TypeS_i]}$.
                  %
                  % ~ delegation
            \item $\DataType_i=\CIso[\Const]'$
                  and $\ValClocks'\models\Const'$
                  and $\exists{\SesEnv}''$ \begin{tabular}[t]{rl}
                      such that & $\SesEnv'={\SesEnv}'',\PrcVal:\CIso'$                                             \\
                      and       & $\VarEnv~\Entails~\Prc~\PrcTyped~{\SesEnv}'',~{p:\CIso+{\ReSet[]_i};[\TypeS_i]}$.
                  \end{tabular}
                  %
        \end{enumerate}
        %
    \end{claim}
    %
    % ~ branch
    \begin{claim}
        %
        If $\VarEnv~\Entails~\PCalcRecv~\PrcTyped~\SesEnv$
        then $\PFuncNEQ[\SesEnv]=\emptyset$
        and $\exists p, \SesEnv'$
        such that $\SesEnv=\SesEnv',~{p:\CIso;[\TypInteract[j][j]]}$
        and $\forall i \in I, \exists j\in j$
        such that $\TypComm_j=\TypRecv$
        and $\PrcLabel_i=\MsgLabel_j$
        and $\forall\ValTime\leq\PrcAfter$
        such that $\ValClocks+\ValTime\models\Const_j$
        and \Trans*{\SesEnv'+\ValTime}|:{\RecvMsg}\
        and:
        \begin{itemize}
            %
            % ~ typing
            \item $\VarEnv+\ValTime~\Entails~\PCalcRecvSingle2{\PrcAfter-\ValTime}_{i}~\PrcTyped~\SesEnv'+\ValTime,~{p:\CIso+{+\ValTime};[\TypInteract|_[j][j]]}$.
                  %
                  % ~ after
            \item $\PrcAfter\neq\infty~\Implies~\VarEnv+\PrcAfter~\Entails~\Qrc~\PrcTyped~\SesEnv'+\PrcAfter,~{p:\CIso+{+\PrcAfter};[\TypInteract[j][j]]}$.
                  %
                  % ~ error
            \item $\Qrc=\PrcErr$ if $\CIso+{+\PrcAfter};[\TypInteract[i][i]]$ is not \emph{well-formed}.
                  %
        \end{itemize}
        %
    \end{claim}
    %
    % ~ recv
    \begin{claim}
        %
        If $\VarEnv~\Entails~\PCalcRecvSingle~\PrcTyped~\SesEnv$
        then $\exists p, \SesEnv'$
        such that $\SesEnv=\SesEnv',~{p:\CIso;[\TypInteract|]}$
        and $\TypComm=\TypRecv$
        and $\PrcLabel=\MsgLabel$
        and $\forall\ValTime\leq\PrcAfter$
        such that $\ValClocks\models\Const$
        and \Trans*{\SesEnv'+\ValTime}|:{\RecvMsg}\
        and $\PFuncNEQ[\SesEnv]=\emptyset$
        and either:
        \begin{enumerate}
            %
            % ~ value
            \item $\DataType\neq\CIso[\Const]'$
                  and $\VarEnv+\ValTime~\Entails~\Prc~\PrcTyped~\SesEnv'+\ValTime,~\PrcMsg:\DataType,~{p:\CIso+{+\ValTime\ReSet[]};[\TypeS]}$.
                  %
                  % ~ delegation
            \item $\DataType=\CIso[\Const]'$
                  and $\ValClocks'\models\Const'$
                  and $\VarEnv+\ValTime~\Entails~\Prc~\PrcTyped~\SesEnv'+\ValTime,~{p:\CIso+{+\ValTime\ReSet[]};[\TypeS]},~\PrcMsg:\CIso'$.
                  %
        \end{enumerate}
        %
    \end{claim}
    %
    % ~ parl
    \begin{claim}
        %
        If $\VarEnv~\Entails~\Parl{\Prc,\Qrc}~\PrcTyped~\SesEnv$
        then $\SesEnv=\SesEnv_1,\SesEnv_2$
        and $\VarEnv~\Entails~\Prc~\PrcTyped~\SesEnv_1$
        and $\VarEnv~\Entails~\Qrc~\PrcTyped~\SesEnv_2$.
        %
    \end{claim}
    %
    % ~ scope
    \begin{claim}
        %
        If $\VarEnv~\Entails~{(\nu q p)\Prc}~\PrcTyped~\SesEnv$
        then $\VarEnv~\Entails~\Prc~\PrcTyped~\SesEnv,
            ~{p:\VIso_1},
            ~{q:\VIso_2},
            ~{qp:\Queue_1},
            ~{pq:\Queue_2}$
        and \Compat*.
        %
    \end{claim}
    %
    % ~ rec var
    \begin{claim}
        %
        If $\VarEnv~\Entails~\RecPrcCall~\PrcTyped~\SesEnv$
        then $\exists\VarEnv'$
        such that $\VarEnv=\VarEnv',~\EnvPrcVar$
        and $\SesEnv\in\RecSetRoles$
        \linebreak
        and $\forall \DataType\in\RecPrcMsg,\exists \PrcMsg\in\RecSetMsg$
        such that $\VarEnv~\Entails~\PrcMsg:\DataType$
        and $\forall \Const\in\RecPrcTimers,\exists x\in\RecSetTimers$
        such that $x\models\Const$.
        %
    \end{claim}
    %
    % ~ rec def
    \begin{claim}
        %
        If $\VarEnv~\Entails~\PrcRecDef~\PrcTyped~\SesEnv$
        then $\forall \PSesSetCfgs\in\SesSet$
        such that
        \linebreak
        $\VarEnv,~\RecSetMsg:\RecPrcMsg,~\RecSetTimers:\RecPrcTimers,~\RecVar:\EnvPrcVar%
            ~\Entails~%
            \Prc%
            ~\PrcTyped~%
            \RecSetRoles:\PSesSetCfgs$
        and $\VarEnv,~\RecVar:\EnvPrcVar%
            ~\Entails~%
            \Qrc%
            ~\PrcTyped~%
            \SesEnv$.
        %
    \end{claim}
    %
    % ~ par rec
    \begin{claim}
        %
        If $\VarEnv~\Entails~\PrcRecDef>{\Parl{\Qrc,\Qrc'}}~\PrcTyped~\SesEnv$
        then $\SesEnv=\SesEnv_1,\SesEnv_2$
        and $\VarEnv,~\EnvPrcVar~\Entails~\Qrc~\PrcTyped~\SesEnv_1$
        and $\VarEnv,~\EnvPrcVar~\Entails~\Qrc'~\PrcTyped~\SesEnv_2$.
        %
    \end{claim}
    %
    % ~ empty buffer
    \begin{claim}
        %
        If $\VarEnv~\Entails~{qp:\emptyset}~\PrcTyped~\SesEnv$
        then $\exists \SesEnv'$
        such that $\SesEnv=\SesEnv',~{qp:\emptyset}$
        %
    \end{claim}
    %
    % ~ buffer
    \begin{claim}
        %
        If $\VarEnv~\Entails~{qp:\PrcLabel\PrcMsg\cdot\QHead}~\PrcTyped~\SesEnv$
        then $\exists \SesEnv'$
        such that $\SesEnv=\SesEnv',~{qp:\IMsgType[\MsgLabel'];\Queue}$
        and $\PrcLabel=\MsgLabel'$
        then $\PFuncNEQ[\SesEnv]=\emptyset$
        and either:
        \begin{enumerate}
            %
            % ~ value
            \item $\DataType\neq\CIso[\Const]'$
                  and $\VarEnv~\Entails~{\PrcMsg:\DataType}$
                  and $\VarEnv~\Entails~{qp:\QHead}~\PrcTyped~\SesEnv',~{qp:\Queue}$.
                  %
                  % ~ delegation
            \item $\DataType=\CIso[\Const]'$
                  and $\ValClocks'\models\Const'$
                  and $\PrcMsg=q$
                  and $\exists {\SesEnv}''$
                  such that $\SesEnv'={\SesEnv}'',~{q:\CIso'}$
                  and $\VarEnv~\Entails~{qp:\QHead}~\PrcTyped~\SesEnv',~{qp:\Queue}$.
                  %
        \end{enumerate}
        %
    \end{claim}
    %
    % ~ delay
    \begin{claim}
        %
        If $\VarEnv~\Entails~{\mathtt{delay}(\ValTime).\Prc}~\PrcTyped~\SesEnv$
        then $\PFuncNEQ[\SesEnv]=\emptyset$
        and $\forall \ValTime'<\ValTime$
        such that \Trans*{\SesEnv+\ValTime}|:{\RecvMsg}\
        and $\VarEnv+\ValTime~\Entails~\Prc~\PrcTyped~\SesEnv+\ValTime$.
        %
    \end{claim}
    %
\end{lemma}
\begin{proof}
    %
    By inspection of the typing rules for processes in~\cref{fig:prc_typ_rule}, given the structure of \Prc*\ each claim holds.
    %
\end{proof}

% ~ fail free processes
%
\begin{definition}[Fail-Free Process]\label{def:prc_fail_free}
    %
    When $\Prc=\PrcFail$, \Prc*\ is \emph{not} fail-free.%
    For all other valuations of \Prc*\ below, \Prc*\ is \emph{fail-free} if its corresponding condition holds:
    \par\noindent
    \begin{minipage}{\textwidth}\centering
        \begin{tabular}{l p{2ex} l}
            %
            \PCalcSend*            &  & %
            is \emph{fail-free} if \Prc*\ is \emph{fail-free},%
            %
            \\
            \PCalcRecv*            &  & %
            is \emph{fail-free} if \Prc*_{i}\ is \emph{fail-free}, for all $i\in\SetI$,%
            %
            \\
            \PCalcScope*           &  & %
            is \emph{fail-free} if \Prc*\ is \emph{fail-free},%
            %
            \\
            \Par*[\Prc][\Qrc]      &  & %
            is \emph{fail-free} if both \Prc*\ and \Qrc*\ are \emph{fail-free},%
            %
            \\
            \PCalcBuffer*          &  & %
            is \emph{fail-free},%
            %
            \\
            \PrcRecDef*          &  & %
            is \emph{fail-free} if \Qrc*\ is \emph{fail-free},%
            %
            \\
            \PCalcSetTimer*        &  & %
            is \emph{fail-free} if \Prc*\ is \emph{fail-free},%
            %
            % \\
            % \PCalcRecVar*|         &  & %
            % is \emph{fail-free} if \Prc*\ is \emph{fail-free},%
            %
            \\
            \PFuncDelay*           &  & %
            is \emph{fail-free} if \Prc*\ is \emph{fail-free},%
            %
            \\
            \PCalcIf*              &  & %
            is \emph{fail-free} if both \Prc*\ and \Qrc*\ are \emph{fail-free},%
            %
            \\
            \PFuncDelay*[\ValTime] &  & %
            is \emph{fail-free} if \Prc*\ is \emph{fail-free},%
            %
            \\
            \PrcEnd*               &  & %
            is \emph{fail-free}.%
            %
        \end{tabular}
    \end{minipage}
    %
\end{definition}
% %
% \begin{restatable}[Fail-Free Process]{definition}{DefPrcFailFree}\label{def:prc_fail_free}
%     %
%     A process \Prc*\ is \emph{fail-free} if:
%     \begin{tabular}{l p{2ex} l}
%         %
%         \PCalcSend*.\Qrc* & & %
%         is \emph{fail-free} if \Qrc*\ is \emph{fail-free}%
%         %
%         \\
%         \PCalcRecv*4{\Qrc} & & %
%         is \emph{fail-free} if \Qrc*_{i}\ is \emph{fail-free}, for all $i\in\SetI$%
%         %
%         \\
%         \PCalcSend*.\Qrc* & & %
%         is \emph{fail-free} if \Qrc*\ is \emph{fail-free}%
%         %
%         \\
%         \PCalcSend*.\Qrc* & & %
%         is \emph{fail-free} if \Qrc*\ is \emph{fail-free}%
%         %
%         \\
%         \PCalcSend*.\Qrc* & & %
%         is \emph{fail-free} if \Qrc*\ is \emph{fail-free}%
%         %
%         \\
%         \PCalcSend*.\Qrc* & & %
%         is \emph{fail-free} if \Qrc*\ is \emph{fail-free}%
%         %
%         \\
%         \PCalcSend*.\Qrc* & & %
%         is \emph{fail-free} if \Qrc*\ is \emph{fail-free}%
%         %
%         \\
%         \PCalcSend*.\Qrc* & & %
%         is \emph{fail-free} if \Qrc*\ is \emph{fail-free}%
%         %
%         \\
%         \PCalcSend*.\Qrc* & & %
%         is \emph{fail-free} if \Qrc*\ is \emph{fail-free}%
%         %
%     \end{tabular}
%     %
%  \end{restatable}
%  %

% ~ balanced sessions
In~\cref{def:session_balanced} the notion of compatibility (\cref{def:configs_compat}) is extended from individual system configurations, across all \SesEnv*\ in the balanced set \BalSes*.
%
%
\begin{definition}[Balanced \SesEnv*]\label{def:session_balanced}
   %
   Let \BalSes*\ be the set containing all session environments \SesEnv*\ that adhere to the following:
   \begin{enumerate}
      \item $\SesEnv=\SesEnv',%
               ~{\PrcP:\CIso},%
               ~{\PrcQ\PrcP:\Msg;\Queue}%
               ~\implies~%
               \exists\ValClocks',\TypeS':%
               \Trans{\CIso}:{\RecvMsg}[\CIso']$
            and $\SesEnv',
               ~{\PrcP:\mkTup[\ValClocks'][\TypeS']},%
               ~{\PrcP\PrcQ:\Queue}$
            and $\SesEnv'\in\BalSes$.
            %
      \item $\SesEnv=\SesEnv',%
               ~{\PrcP:\CIso_1},
               ~{\PrcQ\PrcP:\Queue_1},%
               ~{\PrcQ:\CIso_2},%
               ~{\PrcP\PrcQ:\Queue_2}%
               ~\implies~%
               \Compat$.
   \end{enumerate}
   A session environment \SesEnv*\ is said to be \emph{balanced} if $\SesEnv\in\BalSes$.
   %
\end{definition}
%
%
\begin{definition}[Fully Balanced \SesEnv*]\label{def:session_fully_balanced}
   %
   A session environment \SesEnv*\ is said to be \emph{fully balanced} if \SesEnv*\ is \emph{balanced} and:
   \begin{enumerate}
      \item $\SesEnv=\SesEnv',\PrcP:\mkTup[\ValClocks_1][\TypeS_1]%
               ~\implies~\begin{array}[t]{rl}%
               \exists\PrcQ,\Queue_1,\ValClocks_2,\TypeS_2,\Queue_2,{\SesEnv}'' & \\%
               \text{such that} & \SesEnv'={\SesEnv}'',%
               ~{\PrcQ\PrcP:\Queue_1},%
               ~{\PrcQ:\CIso_2},%
               ~{\PrcP\PrcQ:\Queue_2}\end{array}$
            %
      \item $\SesEnv=\SesEnv',\PrcQ\PrcP:\Queue_1%
               ~\implies~\begin{array}[t]{rl}%
               \exists\ValClocks_1,\TypeS_1,\ValClocks_2,\TypeS_2,\Queue_2,{\SesEnv}'' & \\%
               \text{such that} & \SesEnv'={\SesEnv}'',%
               ~{\PrcP:\CIso_1},%
               ~{\PrcQ:\CIso_2},%
               ~{\PrcP\PrcQ:\Queue_2}\end{array}$
   \end{enumerate}
   %
\end{definition}
% 

% ~ free queues of session
%
\begin{definition}[Free Queues of Processes]\label{def:prc_free_queues}
    %
    The set of \emph{free queues} \PFuncFQ*[\Prc]\ of a process \Prc*\ is defined inductively as follows:
    \begin{minieq}*%
        \begin{array}[t]{l}%
            \begin{array}[c]{l p{4ex} l p{4ex} l}
                \begin{array}[c]{l c p{0ex} l}
                    \begin{array}[t]{l}%
                        \PFuncFQ[\RecVar], %
                        \mbox{}~\PFuncFQ[\PrcEnd]%
                    \end{array}%
                     & = &  & %
                    \emptyset %
                \end{array}
                %
                 &  &
                \begin{array}[c]{l c p{0ex} l}
                    \PFuncFQ[\PCfgQueue]%
                     & = &  & %
                    \PCalcEndPoints %
                \end{array}
                %
                 &  &
                \begin{array}[c]{l c p{0ex} l}
                    \PFuncFQ[\PCalcScope]%
                     & = &  & %
                    \PFuncFQ[\Prc]\backslash\mkSet[\PrcP\PrcQ,\PrcQ\PrcP] %
                \end{array}
            \end{array}
            %
            \\[-1ex]\\
            \begin{array}[c]{l c p{0ex} l}
                \PFuncFQ[\PCalcRecv] %
                 & = &  & %
                \SetUnion_{i\in\SetI} \PFuncFQ[\Prc_i]
            \end{array}
            %
            \\[-1ex]\\
            \begin{array}[c]{l c p{0ex} l}
                \begin{array}[t]{l}%
                    \PFuncFQ[\PCalcIf], %
                    \mbox{}~\PFuncFQ[\Par[\Prc][\Qrc]]%
                \end{array}%
                 & = &  & %
                \PFuncFQ[\Prc]~\cup~\PFuncFQ[\Qrc]
            \end{array}
            %
            \\[-1ex]\\
            \begin{array}[c]{l c p{0ex} l}
                \begin{array}[t]{l}%
                    \PFuncFQ[\PCalcSend],      %
                    \mbox{}~\PFuncFQ[\PCalcSetTimer],  %
                    \mbox{}~\PFuncFQ[\PFuncDelay[\Const]], %
                    \mbox{}~\PFuncFQ[\PFuncDelay[\ValTime]], \\
                    \mbox{}~\PFuncFQ[\PrcRecDef]             %
                \end{array}%
                 & = &  & %
                \PFuncFQ[\Prc] %
            \end{array}
            %
        \end{array}
    \end{minieq}
    %
\end{definition}
% 

% ~ typing environmental variables
% * 45
\begin{lemma}
    %
    The following hold:
    \begin{enumerate}
        %
        % ~ DRecv
        \item If $\VarEnv~\Entails~\Prc~\PrcTyped~\SesEnv,~{p:\CIso}$
              and $\PrcMsg\not\in\SesDomain$
              then $\VarEnv~\Entails~\Prc\Subst[\PrcMsg][p]~\PrcTyped~\SesEnv,~\PrcMsg:\CIso$.
              %
              % ~ VRecv
        \item If $\VarEnv_1,~{\PrcMsg:\DataType}~\Entails~\Prc~\PrcTyped~\SesEnv$
              and $\VarEnv_2~\Entails~{\PrcVal:\DataType}$
              and $\SesDomain[\VarEnv_1]\cap\SesDomain[\VarEnv_2]=\emptyset$
              then \hspace*{\fill}\begin{tabular}[t]{l}\\\llap{$\VarEnv_1,\VarEnv_2~\Entails~\Prc\Subst[\PrcVal][\PrcMsg]~\PrcTyped~\SesEnv$}.\end{tabular}
              %
              % ~ Var
        \item If $\VarEnv_1,~{x:n}~\Entails~\Prc~\PrcTyped~\SesEnv$
              and $\VarEnv_2~\Entails~{y\models\Const}$
              and $\SesDomain[\VarEnv_1]\cap\SesDomain[\VarEnv_2]=\emptyset$
              then \hspace*{\fill}\begin{tabular}[t]{l}\\\llap{$\VarEnv_1,\VarEnv_2~\Entails~\Prc\Subst[y][x]~\PrcTyped~\SesEnv$}.\end{tabular}
              %
    \end{enumerate}
    %
\end{lemma}
\begin{proof}
    %
    By the premise of the relevant typing rule in~\cref{fig:prc_typ_rule}
    \begin{inline}|
        \item \LblPrcTypDRecv*\
        \item \LblPrcTypVRecv*\
        \item \LblPrcTypVar*\
    \end{inline}
    %
\end{proof}

% * 46
\begin{lemma}
    %
    If $\VarEnv~\Entails~{\PrcVal:\DataType}$
    then \DataType*\ is a \emph{base type}.
    %
\end{lemma}
\begin{proof}
    %
    Follows~\cref{fig:prc_typ_rule} that \DataType*\ is not a \emph{delegation type} by inspecting the premise of the relevant typing rule
    \begin{inline}|
        \item \LblPrcTypVSend*\
        \item \LblPrcTypVQue*\
    \end{inline}
    %
\end{proof}

% * 47
\begin{lemma}
    %
    Let $\VarEnv~\Entails~{qp:\QHead}~\PrcTyped~\SesEnv,~{qp:\Queue}$.
    The following hold:
    \begin{itemize}
        %
        % ~ val
        \item If $\VarEnv~\Entails~{\PrcVal:\DataType}$
              then $\VarEnv~\Entails~{qp:\QHead\cdot\PrcVal}~\PrcTyped~\SesEnv,~{qp:\Queue;\DataType}$.
              %
              % ~ delegation
        \item If $\DataType=\CIso[\Const]$
              and $\ValClocks\models\Const$
              and $p\not\in\SesDomain$
              then $\VarEnv~\Entails~{qp:\QHead\cdot p}~\PrcTyped~\SesEnv,~{qp:\Queue;\DataType},~{p:\CIso}$.
              %
    \end{itemize}

    %
\end{lemma}
\begin{proof}
    %
    Follows~\cref{fig:prc_typ_rule} by inspecting the premise of the relevant typing rule
    \begin{inline}|
        \item \LblPrcTypDQue*\
        \item \LblPrcTypVQue*\
    \end{inline}
    %
\end{proof}

% ~ typing receptions
% * 32
\begin{lemma}
    %
    If $\VarEnv~\Entails~\Prc~\PrcTyped~\SesEnv,~{p:\CIso}$
    and \PCalc*{\TimePass{\Prc}}\ is defined and \emph{error-free}
    and $\exists \Msg$
    such that \Trans*{\CIso}:{\RecvMsg}\
    then $p\in\PFuncWait$.
    %
\end{lemma}
\begin{proof}
    %
    We proceed by induction on the transition \Trans*{\CIso}:{\RecvMsg}\ via either
    \begin{inline}|
        \item \LblCfgIsoInteract*\
        \item \LblCfgIsoUnfold*\
    \end{inline}
    \begin{inductivecase}
        %
        % ~ choice
        \item\NewCase[\LblCfgIsoInteract*] then $\TypeS=\simplechoice$.
        By~\cref{lem:typ_prc_role_wf} \CIso*\ is \emph{well-formed} and by~\cref{lem:cfg_wf_then_live} is also \isFE*.
        By inner induction on the typing derivation of the hypothesis.
        \begin{inductivecase}
            %
            % ~ branch
            \item\NewCase[\LblPrcTypBranch*] then $\Prc=\PCalcRecv$ and the hypothesis holds.
            %
            % ~ d recv
            \item\NewCase[\LblPrcTypDRecv*] then $\Prc=\PCalcRecvSingle$ and the hypothesis holds. \LblPrcTypVRecv*\ is similar.
            %
            % ~ rec
            \item\NewCase[\LblPrcTypRec*] then $\Prc=\PrcRecDef$ and the hypothesis depends on the structure of \Qrc*\ (see other cases).
            %
        \end{inductivecase}
        % Cases \LblPrcTypRes*\ and \LblPrcTypPar*\ are straightforward.
        %
    \end{inductivecase}
    We omit case \LblCfgIsoUnfold*\ as $\TypeS=\mu\alpha.\TypeS'$ as it follows~\cref{lem:cfg_wf_then_live,lem:cfgs_trans_wf_pres,lem:configs_trans_compat_pres} that there exist some continuation of \TypeS*\ such that case \LblCfgIsoInteract*\ applies.
    %
\end{proof}

% * 33
\begin{lemma}\label{typ_prc_ses_que_then_neq}
    %
    If $\VarEnv~\Entails~\Prc~\PrcTyped~\SesEnv,~{qp:\Msg;\Queue}$
    then $p\in\PFuncNEQ$.
    %
\end{lemma}
\begin{proof}
    %
    By induction on the typing derivation the last rule applied is either
    \begin{inline}|
        \item \LblPrcTypVQue*\
        \item \LblPrcTypDQue*\
    \end{inline}
    \Prc*\ must be structured such that $\VarEnv~\Entails~{qp:\PrcMsg\cdot\QHead}~\PrcTyped~\SesEnv,~{qp:\Msg;\Queue}$ and $\PFuncNEQ[qp:\PrcMsg\cdot\QHead]=\{p\}$.
    %
\end{proof}

% * 34
\begin{lemma}\label{typ_prc_neq_then_ses_que}
    %
    If $\VarEnv~\Entails~\Prc~\PrcTyped~\SesEnv$
    and $p\in\PFuncNEQ$
    then $\exists \SesEnv'$
    such that $\SesEnv=\SesEnv',~{qp:\Msg;\Queue}$.
    %
\end{lemma}
\begin{proof}
    %
    \Prc*\ must be structured such that $\Prc\equiv\Parl{qp:\PrcMsg\cdot\QHead,\Qrc}$.
    The hypothesis follows~\cref{typ_prc_ses_que_then_neq}.
    %
\end{proof}

% ~ delaying processes
% * 37
\begin{lemma}
    %
    If $\VarEnv~\Entails~\Prc~\PrcTyped~\SesEnv,
        ~{p:\VIso_1},
        ~{q:\VIso_2},
        ~{qp:\Queue_1},
        ~{pq:\Queue_2}$
    and \Compat*\
    and \SesEnv*\ is \emph{balanced}
    and \PCalc*{\TimePass{\Prc}}\ is defined and \emph{error-free}
    then $p\not\in\PFuncNEQ$
    and $q\not\in\PFuncNEQ$.
    %
\end{lemma}
\begin{proof}
    %
    We only show the case of $p:\CIso_1$.
    By~\cref{lem:typ_prc_inversion} and \LblPrcTypRes*\ the structure of \Prc*\ must be such that $\Prc\equiv\Parl{\Prc',\Qrc,{qp:\QHead},{pq:\QHead'}}$.
    By the induction hypothesis \PCalc*{\TimePass{\Parl{\Prc',\Qrc,{qp:\QHead},{pq:\QHead'}}}}\ is defined and therefore $p\not\in\PFuncNEQ$.
    %
\end{proof}

% * 39
\begin{lemma}
    %
    If $\VarEnv~\Entails~\Prc~\PrcTyped~\SesEnv,~{p:\CIso}$
    and \CIso*\ is \isFE*\
    and $\PFuncNEQ=\emptyset$
    and \PCalc*{\TimePass{\Prc}}\ is defined and \emph{error-free}
    then \CIso*+{+\ValTime}\ is \isFE*.
    %
\end{lemma}
\begin{proof}
    %
    By~\cref{lem:cfg_wf_then_live} $\TypeS=\simplechoice$
    and $\exists \Const_i$
    such that $\ValClocks\models\Past[\Const_i]$
    and $\emptyset;\Past[\Const_i]~\Entails~\TypInteract[i][i]$.
    We proceed by induction on the structure of \Prc*\ such that \PCalc*{\TimePass{\Prc}}\ is defined and the value of \ValTime*\ is relevant.
    \begin{inductivecase}
        %
        %
        % ~ delay
        \item\NewCase[$\Prc=\mathtt{delay}(\ValTime').\Prc'$] then $\ValTime\leq\ValTime'$.
        By~\cref{lem:typ_prc_inversion} $\VarEnv+\ValTime'~\Entails~\Prc'~\PrcTyped~\SesEnv+\ValTime',~{p:\CIso+{+\ValTime'}}$ and by~\cref{lem:typ_prc_role_wf} \CIso*+{+\ValTime'}\ is \emph{well-formed}. It follows~\cref{lem:cfg_fe_timepass_wf_fe} that the hypothesis holds.
        %
        %
        % ~ receive
        \item\NewCase[$\Prc=\PCalcRecv$] By~\cref{lem:typ_prc_inversion} and \LblPrcTypBranch*\ $\forall\ValTime'\leq\PrcAfter$ it holds that $\VarEnv+\ValTime'~\Entails~\Prc~\PrcTyped~\SesEnv+\ValTime',~{p:\CIso+{+\ValTime'}}$.
        If $\ValTime>\PrcAfter$ then \PCalc*{\TimePass{\Prc}=\TimePass{\Qrc}}\ and by the hypothesis $\Qrc\neq\PrcErr$.
        By~\cref{lem:typ_prc_role_wf,lem:timepass_func_defined_t} the hypothesis holds.
        Case $\Prc=\PCalcRecvSingle$ is similar.
        %
    \end{inductivecase}
    %
\end{proof}

% ~ delayable sessions
%
\begin{definition}[Delayable Session Environment (\SesEnv*)]\label{def:prc_typ_sesenv_delayable}
    %
    \SesEnv*\ is \emph{delayable} if $\forall\ISes\in\SesDomain$ such that $\SesEnv(\ISes)\neq\emptyset%
    \Implies%
    \PrcQ\not\in\SesDomain$
    %
\end{definition}
% 
\endinput
a session environment is delayable if for non-empty channel there does not exist a role able to receive from it
% * 40
\begin{lemma}
    %
    If \SesEnv*\ is \emph{balanced} and \emph{delayable}
    and $\SesEnv+\ValTime$ is \emph{well-formed}
    then $\SesEnv+\ValTime$ is \emph{balanced}.
    %
\end{lemma}
\begin{proof}
    %
    By~\cref{def:session_balanced} $\SesEnv=\SesEnv',%
        ~{\PrcP:\VIso_1},
        ~{\PrcQ\PrcP:\Queue_1},%
        ~{\PrcQ:\VIso_2},%
        ~{\PrcP\PrcQ:\Queue_2}$
    and \Compat*\ and by~\cref{def:prc_typ_sesenv_delayable} $\Queue_1=\emptyset=\Queue_2$.
    By~\cref{lem:cfgs_trans_wf_pres,lem:configs_trans_compat_pres} if \Compat*\ and \Trans*{\VSoc_1}:{\ValTime}[\VSoc'_1]\ and \Trans*{\VSoc_2}:{\ValTime}[\VSoc'_2]\ then \Compat*'.
    %
\end{proof}

% * 41
\begin{lemma}
    %
    If $\VarEnv~\Entails~\Prc~\PrcTyped~\SesEnv$
    and \PCalc*{\TimePass{\Prc}}\ is defined
    and \SesEnv*\ is \emph{balanced}
    then \SesEnv*\ is \emph{delayable}.
    %
\end{lemma}
\begin{proof}
    %
    We proceed by induction on the structure of \Prc*\ such that \PCalc*{\TimePass{\Prc}}\ is defined and \emph{well-typed} against a \emph{balanced} \SesEnv*.
    % ~ parl
    If $\Prc=\Parl{\Prc',\Qrc}$ then \PCalc*{\TimePass{\Prc'}}\ and \PCalc*{\TimePass{\Qrc}}\ are defined.
    Suppose by contradiction that \PCalc*{\TimePass{\Prc}}\ is \emph{not} defined. \Prc*'\ and \Qrc*\ must be structured such that $\PFuncWait[\Prc']\cup\PFuncNEQ[\Qrc]\neq\emptyset$ and it must be that $\Prc'=\PCalcRecv[i][i][\Qrc']$ and $\Qrc={qp:\PrcMsg\cdot\QHead}$.
    By~\cref{def:session_balanced} $\exists \SesEnv'$ such that $\SesEnv=\SesEnv',~{p:\CIso_1},~{qp:\Msg;\Queue_1}$ and $\exists \Msg:\Trans{\CIso_1}:{\RecvMsg}$. However as \PCalc*{\TimePass{\Prc}}\ is defined, there are no messages able to be received by \Prc*\ which is \emph{well-typed} against \SesEnv*. Therefore follows that \SesEnv*\ is \emph{delayable} by~\cref{def:prc_typ_sesenv_delayable}.
    %
\end{proof}

% * 42
\begin{lemma}\label{lem:well_typed_delayed_prc}
    %
    If $\VarEnv~\Entails~\Prc~\PrcTyped~\SesEnv$
    and \PCalc*{\TimePass{\Prc}}\ is defined and \emph{error-free}
    and $\PFuncNEQ=\emptyset$
    and $\SesEnv+\ValTime$ is \emph{well-formed}
    then $\VarEnv+\ValTime~\Entails~\PCalc{\TimePass{\Prc}}~\PrcTyped~\SesEnv+\ValTime$.
    %
\end{lemma}
\begin{proof}
    %
    We proceed by induction on the structure of \Prc*\ such that \PCalc*{\TimePass{\Prc}}\ is defined and \emph{error-free}.
    \begin{inductivecase}
        %
        %
        % ~ recv
        \item\NewCase[$\Prc=\PCalcRecv$] By~\cref{lem:typ_prc_inversion} $\SesEnv=\SesEnv',~{p:\CIso;[\simplechoice_{j}]}$.
        By inner induction on the typing derivation analysing the last rule applied: \LblPrcTypBranch*\ %(and \LblPrcTypDRecv*, \LblPrcTypVRecv*)
        \begin{inductivecase}
            %
            % ~ t <= n
            \item\NewCase[If $\ValTime\leq\PrcAfter$ then] $\forall \ValTime'\leq\PrcAfter-\ValTime$~\cref{eq:well_typed_delayed_prc_branch} holds and $\forall {\ValTime}''\leq\ValTime:\Trans{\SesEnv+\ValTime'+{\ValTime}''}|:{\RecvMsg}$ and $\forall j\in J$ such that $\ValClocks+\ValTime+\ValTime'\models\Const_j~\implies~\TypComm_j=\TypRecv$ and $\exists i\in I$ such that $\PrcLabel_i=\MsgLabel_j$ and:
            \begin{itemize}
                %
                % ~ val
                \item$\DataType_j\neq\CIso[\Const]~\implies~\VarEnv+\ValTime+\ValTime',\PrcMsg_i:\DataType_j~\Entails~\Prc_i~\PrcTyped~\SesEnv+\ValTime+\ValTime',~{p:\CIso+{+\ValTime+\ValTime'\ReSet[]_j};[\TypeS_j]}$.
                %
                % ~ delegation
                \item$\DataType_j=\CIso[\Const]~\implies~\VarEnv+\ValTime+\ValTime'~\Entails~\Prc_i~\PrcTyped~\SesEnv+\ValTime+\ValTime',~{p:\CIso+{+\ValTime+\ValTime'\ReSet[]_j};[\TypeS_j]},~{\PrcMsg_i:\CIso'}$ %\linebreak \hspace*{\fill} 
                and $\ValClocks'\models\Const'$.
                %
            \end{itemize}
            \begin{minieq}\label{eq:well_typed_delayed_prc_branch}
                \VarEnv+\ValTime+\ValTime'~\Entails~\PCalcRecv2{\PrcAfter-(\ValTime+\ValTime')}~\PrcTyped~\SesEnv+\ValTime+\ValTime',~{p:\CIso+{+\ValTime+\ValTime'};[\simplechoice_{j}]}
            \end{minieq}\vspace{-4ex}%
            %
            % ~ t > n
            \item\NewCase[If $\ValTime>\PrcAfter$ then] \PCalc*{\TimePass{\Prc}=\TimePass_{\ValTime-\PrcAfter}{\Qrc}}\ and $\Qrc\neq\PrcErr$ and $\VarEnv+\PrcAfter~\Entails~{\Qrc}~\PrcTyped~\SesEnv+\PrcAfter$. It follows by induction hypothesis $\VarEnv+\ValTime~\Entails~\PCalc{\TimePass_{\ValTime-\PrcAfter}{\Qrc}}~\PrcTyped~\SesEnv+\ValTime$.
            %
        \end{inductivecase}
        %
        %
        % ~ delay
        \item\NewCase[$\Prc=\mathtt{delay}(\ValTime').\Prc'$] then $\ValTime\leq\ValTime'$ and $\forall {\ValTime}''\leq\ValTime':\Trans{\SesEnv+{\ValTime}''}|:{\RecvMsg}$. By induction hypothesis $\VarEnv+\ValTime~\Entails~{\mathtt{delay}(\ValTime'-\ValTime).\Prc'}~\PrcTyped~\SesEnv+\ValTime$.
        %
        %
        % ~ parl
        \item\NewCase[$\Prc=\Parl{\Prc',\Qrc}$] By~\cref{lem:typ_prc_inversion} $\SesEnv=\SesEnv_1,\SesEnv_2$.
        %
        %
        % ~ scope
        \item\NewCase[$\Prc=(\nu qp)\Prc'$] By~\cref{lem:typ_prc_inversion} $\SesEnv=\SesEnv',~{p:\CIso;[\simplechoice_{j}]}$.
        %
    \end{inductivecase}
    %
\end{proof}

% ~ typing delayed processes
% * 43
\begin{lemma}
    %
    %
\end{lemma}
\begin{proof}
    %
    If $\VarEnv~\Entails~\Prc~\PrcTyped~\SesEnv$
    then
    %
\end{proof}

% * 44
\begin{lemma}
    %
    If $\VarEnv~\Entails~\Prc~\PrcTyped~\SesEnv$
    then
    %
\end{lemma}
\begin{proof}
    %
    %
\end{proof}

% ~ typing reduction steps
% * 49
\begin{lemma}
    %
    If $\VarEnv~\Entails~\Prc~\PrcTyped~\SesEnv$
    then
    %
\end{lemma}
\begin{proof}
    %
    %
\end{proof}

% * 50
\begin{lemma}
    %
    If $\VarEnv~\Entails~\Prc~\PrcTyped~\SesEnv$
    then
    %
\end{lemma}
\begin{proof}
    %
    %
\end{proof}

% ~ fail-free processes
% * 51
\begin{lemma}
    %
    %
\end{lemma}
\begin{proof}
    %
    %
\end{proof}

% * 52
\begin{lemma}
    %
    %
\end{lemma}
\begin{proof}
    %
    %
\end{proof}

% * 54
\begin{lemma}
    %
    %
\end{lemma}
\begin{proof}
    %
    %
\end{proof}

% ~ fail free reduction
% * 55
\begin{lemma}
    %
    %
\end{lemma}
\begin{proof}
    %
    %
\end{proof}

% * 56
\begin{lemma}
    %
    %
\end{lemma}
\begin{proof}
    %
    %
\end{proof}

% * 57
\begin{lemma}
    %
    %
\end{lemma}
\begin{proof}
    %
    %
\end{proof}

% * 62
\begin{lemma}
    %
    %
\end{lemma}
\begin{proof}
    %
    %
\end{proof}

% * 61
\begin{lemma}
    %
    %
\end{lemma}
\begin{proof}
    %
    %
\end{proof}

% ~ untimed processes
% * 60
\begin{lemma}
    %
    %
\end{lemma}
\begin{proof}
    %
    %
\end{proof}

% * 63
\begin{lemma}
    %
    %
\end{lemma}
\begin{proof}
    %
    %
\end{proof}

% ~ subject reduction
% * 64
\begin{lemma}
    %
    %
\end{lemma}
\begin{proof}
    %
    %
\end{proof}

\endinput

% * 
\begin{lemma}
    %
    %
\end{lemma}
\begin{proof}
    %
    %
\end{proof}

\begin{definition}[Future-enabled Actions (\isFE*$\TypSend/\TypRecv$)]\label{def:fe_actions}
    %
    A configuration \VIso*\
    is \isFE*\
    if $\exists\ValTime,\TypComm,\Msg$
    such that \Trans*{\VIso}:{\ValTime,\CommMsg}[\VIso'];
    \VIso*\ is \isFE*!\ if $\TypComm=\TypSend$,
    and is \isFE*?\ if $\TypComm=\TypRecv$.
    %
\end{definition}

%
\begin{definition}[Fail-Free Process]\label{def:prc_fail_free}
    %
    When $\Prc=\PrcFail$, \Prc*\ is \emph{not} fail-free.%
    For all other valuations of \Prc*\ below, \Prc*\ is \emph{fail-free} if its corresponding condition holds:
    \par\noindent
    \begin{minipage}{\textwidth}\centering
        \begin{tabular}{l p{2ex} l}
            %
            \PCalcSend*            &  & %
            is \emph{fail-free} if \Prc*\ is \emph{fail-free},%
            %
            \\
            \PCalcRecv*            &  & %
            is \emph{fail-free} if \Prc*_{i}\ is \emph{fail-free}, for all $i\in\SetI$,%
            %
            \\
            \PCalcScope*           &  & %
            is \emph{fail-free} if \Prc*\ is \emph{fail-free},%
            %
            \\
            \Par*[\Prc][\Qrc]      &  & %
            is \emph{fail-free} if both \Prc*\ and \Qrc*\ are \emph{fail-free},%
            %
            \\
            \PCalcBuffer*          &  & %
            is \emph{fail-free},%
            %
            \\
            \PrcRecDef*          &  & %
            is \emph{fail-free} if \Qrc*\ is \emph{fail-free},%
            %
            \\
            \PCalcSetTimer*        &  & %
            is \emph{fail-free} if \Prc*\ is \emph{fail-free},%
            %
            % \\
            % \PCalcRecVar*|         &  & %
            % is \emph{fail-free} if \Prc*\ is \emph{fail-free},%
            %
            \\
            \PFuncDelay*           &  & %
            is \emph{fail-free} if \Prc*\ is \emph{fail-free},%
            %
            \\
            \PCalcIf*              &  & %
            is \emph{fail-free} if both \Prc*\ and \Qrc*\ are \emph{fail-free},%
            %
            \\
            \PFuncDelay*[\ValTime] &  & %
            is \emph{fail-free} if \Prc*\ is \emph{fail-free},%
            %
            \\
            \PrcEnd*               &  & %
            is \emph{fail-free}.%
            %
        \end{tabular}
    \end{minipage}
    %
\end{definition}
% %
% \begin{restatable}[Fail-Free Process]{definition}{DefPrcFailFree}\label{def:prc_fail_free}
%     %
%     A process \Prc*\ is \emph{fail-free} if:
%     \begin{tabular}{l p{2ex} l}
%         %
%         \PCalcSend*.\Qrc* & & %
%         is \emph{fail-free} if \Qrc*\ is \emph{fail-free}%
%         %
%         \\
%         \PCalcRecv*4{\Qrc} & & %
%         is \emph{fail-free} if \Qrc*_{i}\ is \emph{fail-free}, for all $i\in\SetI$%
%         %
%         \\
%         \PCalcSend*.\Qrc* & & %
%         is \emph{fail-free} if \Qrc*\ is \emph{fail-free}%
%         %
%         \\
%         \PCalcSend*.\Qrc* & & %
%         is \emph{fail-free} if \Qrc*\ is \emph{fail-free}%
%         %
%         \\
%         \PCalcSend*.\Qrc* & & %
%         is \emph{fail-free} if \Qrc*\ is \emph{fail-free}%
%         %
%         \\
%         \PCalcSend*.\Qrc* & & %
%         is \emph{fail-free} if \Qrc*\ is \emph{fail-free}%
%         %
%         \\
%         \PCalcSend*.\Qrc* & & %
%         is \emph{fail-free} if \Qrc*\ is \emph{fail-free}%
%         %
%         \\
%         \PCalcSend*.\Qrc* & & %
%         is \emph{fail-free} if \Qrc*\ is \emph{fail-free}%
%         %
%         \\
%         \PCalcSend*.\Qrc* & & %
%         is \emph{fail-free} if \Qrc*\ is \emph{fail-free}%
%         %
%     \end{tabular}
%     %
%  \end{restatable}
%  %

In~\cref{def:session_balanced} the notion of compatibility (\cref{def:configs_compat}) is extended from individual system configurations, across all \SesEnv*\ in the balanced set \BalSes*.
%
%
\begin{definition}[Balanced \SesEnv*]\label{def:session_balanced}
   %
   Let \BalSes*\ be the set containing all session environments \SesEnv*\ that adhere to the following:
   \begin{enumerate}
      \item $\SesEnv=\SesEnv',%
               ~{\PrcP:\CIso},%
               ~{\PrcQ\PrcP:\Msg;\Queue}%
               ~\implies~%
               \exists\ValClocks',\TypeS':%
               \Trans{\CIso}:{\RecvMsg}[\CIso']$
            and $\SesEnv',
               ~{\PrcP:\mkTup[\ValClocks'][\TypeS']},%
               ~{\PrcP\PrcQ:\Queue}$
            and $\SesEnv'\in\BalSes$.
            %
      \item $\SesEnv=\SesEnv',%
               ~{\PrcP:\CIso_1},
               ~{\PrcQ\PrcP:\Queue_1},%
               ~{\PrcQ:\CIso_2},%
               ~{\PrcP\PrcQ:\Queue_2}%
               ~\implies~%
               \Compat$.
   \end{enumerate}
   A session environment \SesEnv*\ is said to be \emph{balanced} if $\SesEnv\in\BalSes$.
   %
\end{definition}
%
%
\begin{definition}[Fully Balanced \SesEnv*]\label{def:session_fully_balanced}
   %
   A session environment \SesEnv*\ is said to be \emph{fully balanced} if \SesEnv*\ is \emph{balanced} and:
   \begin{enumerate}
      \item $\SesEnv=\SesEnv',\PrcP:\mkTup[\ValClocks_1][\TypeS_1]%
               ~\implies~\begin{array}[t]{rl}%
               \exists\PrcQ,\Queue_1,\ValClocks_2,\TypeS_2,\Queue_2,{\SesEnv}'' & \\%
               \text{such that} & \SesEnv'={\SesEnv}'',%
               ~{\PrcQ\PrcP:\Queue_1},%
               ~{\PrcQ:\CIso_2},%
               ~{\PrcP\PrcQ:\Queue_2}\end{array}$
            %
      \item $\SesEnv=\SesEnv',\PrcQ\PrcP:\Queue_1%
               ~\implies~\begin{array}[t]{rl}%
               \exists\ValClocks_1,\TypeS_1,\ValClocks_2,\TypeS_2,\Queue_2,{\SesEnv}'' & \\%
               \text{such that} & \SesEnv'={\SesEnv}'',%
               ~{\PrcP:\CIso_1},%
               ~{\PrcQ:\CIso_2},%
               ~{\PrcP\PrcQ:\Queue_2}\end{array}$
   \end{enumerate}
   %
\end{definition}
% 

%
\begin{definition}[Free Queues of Processes]\label{def:prc_free_queues}
    %
    The set of \emph{free queues} \PFuncFQ*[\Prc]\ of a process \Prc*\ is defined inductively as follows:
    \begin{minieq}*%
        \begin{array}[t]{l}%
            \begin{array}[c]{l p{4ex} l p{4ex} l}
                \begin{array}[c]{l c p{0ex} l}
                    \begin{array}[t]{l}%
                        \PFuncFQ[\RecVar], %
                        \mbox{}~\PFuncFQ[\PrcEnd]%
                    \end{array}%
                     & = &  & %
                    \emptyset %
                \end{array}
                %
                 &  &
                \begin{array}[c]{l c p{0ex} l}
                    \PFuncFQ[\PCfgQueue]%
                     & = &  & %
                    \PCalcEndPoints %
                \end{array}
                %
                 &  &
                \begin{array}[c]{l c p{0ex} l}
                    \PFuncFQ[\PCalcScope]%
                     & = &  & %
                    \PFuncFQ[\Prc]\backslash\mkSet[\PrcP\PrcQ,\PrcQ\PrcP] %
                \end{array}
            \end{array}
            %
            \\[-1ex]\\
            \begin{array}[c]{l c p{0ex} l}
                \PFuncFQ[\PCalcRecv] %
                 & = &  & %
                \SetUnion_{i\in\SetI} \PFuncFQ[\Prc_i]
            \end{array}
            %
            \\[-1ex]\\
            \begin{array}[c]{l c p{0ex} l}
                \begin{array}[t]{l}%
                    \PFuncFQ[\PCalcIf], %
                    \mbox{}~\PFuncFQ[\Par[\Prc][\Qrc]]%
                \end{array}%
                 & = &  & %
                \PFuncFQ[\Prc]~\cup~\PFuncFQ[\Qrc]
            \end{array}
            %
            \\[-1ex]\\
            \begin{array}[c]{l c p{0ex} l}
                \begin{array}[t]{l}%
                    \PFuncFQ[\PCalcSend],      %
                    \mbox{}~\PFuncFQ[\PCalcSetTimer],  %
                    \mbox{}~\PFuncFQ[\PFuncDelay[\Const]], %
                    \mbox{}~\PFuncFQ[\PFuncDelay[\ValTime]], \\
                    \mbox{}~\PFuncFQ[\PrcRecDef]             %
                \end{array}%
                 & = &  & %
                \PFuncFQ[\Prc] %
            \end{array}
            %
        \end{array}
    \end{minieq}
    %
\end{definition}
% 

% ~ future enabled
%
% ! (lemma 23) : only send or recv viable (changed from fe)
\begin{lemma}\label{lem:typing_configs_viable_actions}
    %
    Given a \VIso*\ that is both \emph{well-formed} and \isFE*, both of the following hold:
    \begin{enumerate}
        \item\label{itm:typing_configs_viable_actions_lem_i} If $\exists\Msg$ such that \Trans*{\VIso}:{\SendMsg}\ then $\nexists\Msg'$ such that \Trans*{\VIso}:{\RecvMsg}.
        %
        \item\label{itm:typing_configs_viable_actions_lem_ii} Similarly, if $\exists\Msg$ such that \Trans*{\VIso}:{\RecvMsg}\ then $\nexists\Msg'$ such that \Trans*{\VIso}:{\SendMsg}.
    \end{enumerate}
    %
\end{lemma}
\begin{proof}
    %
    By the hypothesis, the judgement \ITJudgement*;{\Const}[\TypeS]\ holds against the formation rules given in~\cref{fig:types_rule}.
    %
    The proof proceeds considering~\cref{itm:typing_configs_viable_actions_lem_i} of the hypothesis, as~\cref{itm:typing_configs_viable_actions_lem_ii} is similar.
    % $\Trans{\VIso}:{\CommAction}\Implies\Trans{\VIso}:{\CommAction'}$ where $\CommAction\neq\CommAction'$, which are 
    The transition \Trans*{\VIso}:{\SendMsg}\ only possible via either
    \begin{inline}|.
        \item \LblCfgIsoInteract*
        \item \LblCfgIsoUnfold*
    \end{inline} of~\cref{fig:typesemantics}.
    %
    By induction on the depth of the derivation tree, analysing the last rule applied:
    \begin{inductivecase}
        %
        % ~ interact
        \item\NewCase[\LblCfgIsoInteract*]\label{case:typing_configs_viable_actions_interact} Then $\TypeS=\TypInteract$ and $\exists\AsSet[j][i]$ such that \Sat*:[\Const_j]\ and $\TypComm_j=\TypSend$.
        %
        By the hypothesis \TypeS*\ must adhere to the formation rules in~\cref{fig:types_rule}; in particular with \LblTypChoice*.

        Suppose $\exists\AsSet[k][i]$ such that \Sat*:[\Const_k]\ %
        and $\TypComm_k=\TypRecv$,
        and both \Trans*{\CIso}:{\SendMsg_j}\
        and \Trans*{\CIso}:{\RecvMsg_k}\ hold;
        by the premise of \LblTypChoice*, as $\TypComm_j\neq\TypComm_k$ then $\Const_j\cap\Const_k=\emptyset$ must hold.
        and both \Sat*:[\Const_j]\
        and \Sat*:[\Const_k] cannot hold.
        %
        Therefore the hypothesis holds for~\cref{itm:typing_configs_viable_actions_lem_i}, and similarly for~\cref{itm:typing_configs_viable_actions_lem_ii}.
        %
        % ~ recursion
        \item\NewCase[\LblCfgIsoUnfold*]\label{case:typing_configs_viable_actions_recursion} \TODO[recusive case].
        %
    \end{inductivecase}
    %
\end{proof}
%  % 23
%
% ! (lemma 31) : if iso cfg is fe, and time step wf, then time step fe
\begin{lemma}\label{lem:configs_wf_timestep_fe}
    %
    If \CIso*\ is \isFE*\ %
    and \Trans*{\CSoc}:{\ValTime}[\CSoc+{+\ValTime}]\ %
    and \CIso*+{+\ValTime}\ is \emph{well-formed}, %
    \hfill\ \linebreak%
    then \CIso*+{+\ValTime}\ is \isFE*.
    %
\end{lemma}
\begin{proof}
    %
    By~\cref{def:types_wf,def:configs_wf}, a \emph{well-formed} \TypeS*\ is

    has a \emph{future-enabled} action (see~\cref{def:configs_fe}).
    %
\end{proof}
%  % 31

% ~ typing judgements
% depends on:: lem:refl lem:trans lem:narrow
\begin{proof}
    \def\currentprefix{inv}
    By induction on the derivations.
    \begin{enumit}
        \item Straightforward with Lemma \ref{lem:refl} and \ref{lem:trans}.
        \item Straightforward with Lemma \ref{lem:refl} and \ref{lem:trans}.
        \item Straightforward with Lemma \ref{lem:refl} and \ref{lem:trans}.
        \item 
        \begin{description}
            \item[Case \rulename{T-Op}:] Obvious with Lemma \ref{lem:refl}.
            \item[Case \rulename{T-CSub}:] We have
                \begin{enumrm}
                    \item\llabel{ty-op} $\jdty{\Gamma}{\expop{v}{y}{c}}{\tycomp{\Sigma'}{T'}{S'}}$,
                    \item\llabel{sub-C} $\jdsub{\Gamma}{\tycomp{\Sigma'}{T'}{S'}}{\tycomp{\Sigma}{T}{S}}$, and
                    \item $\jdwf{\Gamma}{\tycomp{\Sigma}{T}{S}}$
                \end{enumrm}
                for some $\Sigma', T'$, and $C'$.
                W.l.o.g., we can assume that $y \notin \fv(\Sigma) \cup \fv(T)$.
                By inversion of \lref{sub-C}, we have
                \begin{enumrm}[resume]
                    \item\llabel{sub-sig} $\jdsub{\Gamma}{\Sigma}{\Sigma'}$,
                    \item\llabel{sub-T} $\jdsub{\Gamma}{T'}{T}$, and
                    \item\llabel{sub-S} $\jdsub{\Gamma \mid T'}{S'}{S}$~.
                \end{enumrm}
                By the IH on \lref{ty-op}, we have
                \begin{enumrm}[resume]
                    \item\llabel{in-sig} $\Sigma' \ni \op \forall \rep{X: \rep{B}}. \op: (x: T_1) \rarr ((y: T_2) \rarr C_1) \rarr C_2$,
                    \item\llabel{wf-A} $\rep{\jdty{\Gamma}{A}{\rep{B}}}$,
                    \item\llabel{ty-v} $\jdty{\Gamma}{v}{T_1[\rep{A/X}]}$,
                    \item\llabel{sub-T'} $\jdsub{\Gamma}{T''}{T'}$,
                    \item\llabel{ty-c} $\jdty{\Gamma, y: T_2[\rep{A/X}][v/x]}{c}{\tycomp{\Sigma}{T''}{\tyctl{z}{C_0}{C_1[\rep{A/X}][v/x]}}}$,
                    \item\llabel{sub-S'} $\jdsub{\Gamma \mid T''}{\tyctl{z}{C_0}{C_2[\rep{A/X}][v/x]}}{S'}$, and
                    \item\llabel{in-y} $y \notin \fv(\Sigma') \cup \fv(T') \cup \fv(T'') \cup (\fv(C_0) \setminus \{ z \})$
                \end{enumrm}
                for some $\rep{X}, \rep{\rep{B}}, \rep{A}, x, z, T'', T_1, T_2, C_0, C_1$, and $C_2$.
                From the assumption on $y$ and \lref{in-y}, we have
                \begin{enumrm}[resume]
                    \item \llabel{in-y'} $y \notin \fv(\Sigma) \cup \fv(T) \cup \fv(T'') \cup (\fv(C_0) \setminus \{ z \})$~.
                \end{enumrm}
                Inversion on \lref{sub-sig} implies that all field in $\Sigma'$ is also in $\Sigma$,
                and therefore we have
                \begin{enumrm}[resume]
                    \item\llabel{in-sig'} $\Sigma \ni \op: \forall \rep{X: \rep{B}}. (x: T_1) \rarr ((y: T_2) \rarr C_1) \rarr C_2$
                \end{enumrm}
                from \lref{in-sig}.
                Lemma \ref{lem:narrow} with \lref{sub-T'} and \lref{sub-S} implies
                \begin{enumrm}[resume]
                    \item\llabel{sub-S-str} $\jdsub{\Gamma \mid T''}{S'}{S}$~.
                \end{enumrm}
                Lemma \ref{lem:trans} with \lref{sub-T}, \lref{sub-T'}, \lref{sub-S-str} and \lref{sub-S'} implies
                $\jdsub{\Gamma}{T''}{T}$ and $\jdsub{\Gamma \mid T''}{\tyctl{z}{C_0}{C_2[\rep{A/X}][v/x]}}{S}$~.
                From these two with \lref{wf-A}, \lref{ty-v}, \lref{ty-c}, \lref{in-y'} and \lref{in-sig'},
                we have the conclusion.
        \end{description}
    \end{enumit}
\end{proof} % 30
%
% ! (lemma 36) : roles in sesenv are wf
\begin{lemma}\label{lem:typing_iso_wf}
    %
    If \IPJudgement*>:{\PrcP:\CIso}\ then \CIso*\ is \emph{well-formed}.%
\end{lemma}
\begin{proof}
    %
    By the hypothesis and inspection the rules for typing processes in~\cref{fig:prc_typ_rule}, the last rule applied must be one of the following
    \begin{inline}|
        \item \LblPrcTypBranch*
        \item \LblPrcTypDRecv*
        \item \LblPrcTypVRecv*
        \item \LblPrcTypDSend*
        \item \LblPrcTypVSend*
    \end{inline}
    %
    Each of these rules require \Sat*\ holds as part of their premise.
    %
    By~\cref{def:configs_wf} the hypothesis holds.
    %
\end{proof}
% % 36
% 48

% ~ substitution lemma
% 45
% 46

% ~ delaying processes
%
% ! (lemma 32) : delay prc and fe recv sesenv role => role is waiting
\begin{lemma}\label{lem:typing_prc_role_waiting}
    %
    If \IPJudgement*>:{\PrcP:\CIso}\ %
    and \PFuncTime*\ is defined 
    and \emph{fail-free} 
    and \Trans*{\CIso}:{\RecvMsg}, 
    then $\PrcP\in\PFuncWait$.
    %
\end{lemma}
\begin{proof}
    %
    By the hypothesis it must be that $\Prc=\PCalcRecv$ (where $\PrcAfter=\infty\Implies\Qrc=\PrcFail$). 
    %
    \Prc*\ cannot be anything else as it is typed against \CIso*, which is able to make a transition via \LblCfgIsoInteract* (where $\TypComm=\TypRecv$).
    %
    Additionally, \PFuncTime*\ being defined ensures that \Prc*\ is not a parallel composition of processes.
    Combined with~\cref{fig:prc_func_wait} the hypothesis holds.
    %
\end{proof}
%
\endinput
if a process typed against a configuration that is currently able to receive and able to let time pass, then the role is currently being waited upon. % 32
%
% ! (lemma 33) : queue in sesenv has corresponding prc buffer
\begin{lemma}\label{lem:typing_prc_buff_nonempty}
    %
    If \IPJudgement*>:{\ISes:\Msg;\Queue}\ then $\PrcP\in\PFuncNEQ$.
    %
\end{lemma}
\begin{proof}
    %
    By the hypothesis it must be that $\Prc=\ISes:\Msg\cdot\QHead$.
    By~\cref{fig:prc_func_neq} the hypothesis holds.
    %
\end{proof}
%  % 33
%
% ! (lemma 35) : delay process defined => no msg waiting ot be received
\begin{lemma}\label{lem:delay_defined}
    %
    If \PFuncTime*\ is defined then $\PFuncWait\cap\PFuncNEQ=\emptyset$.
    %
\end{lemma}
\begin{proof}
    %
    By inspection of~\cref{fig:prc_func_time} the hypothesis holds for any \Prc*\ where \PFuncTime*\ is defined.
    %
\end{proof}
%  % 35

% ~ typing delayed processes
%
% ! (lemma 35) : delay process defined => no msg waiting ot be received
\begin{lemma}\label{lem:delay_defined}
    %
    If \PFuncTime*\ is defined then $\PFuncWait\cap\PFuncNEQ=\emptyset$.
    %
\end{lemma}
\begin{proof}
    %
    By inspection of~\cref{fig:prc_func_time} the hypothesis holds for any \Prc*\ where \PFuncTime*\ is defined.
    %
\end{proof}
%  % 37
%
% ! (lemma 39) : balanced and delayable => fe
\begin{lemma}\label{lem:typing_delayed_fe}
    %
    If \IPJudgement*>:{\PrcP:\CIso}\ %
    and \ActFE*!{\CIso}\ %
    and $\PFuncNEQ=\emptyset$
    and \PFuncTime*\ is defined and \emph{fail-free},
    then \ActFE*!{\CIso+{+\ValTime}}.
    %
\end{lemma}
\begin{proof}
    %
    \TODO
    %
\end{proof}
%  % 39

% * time passing sesenv
% A \SesEnv*\ is delayable given that for every non-empty queue, there does not exist a role capable of receiving from it.
%
\begin{definition}[Delayable Session Environment (\SesEnv*)]\label{def:prc_typ_sesenv_delayable}
    %
    \SesEnv*\ is \emph{delayable} if $\forall\ISes\in\SesDomain$ such that $\SesEnv(\ISes)\neq\emptyset%
    \Implies%
    \PrcQ\not\in\SesDomain$
    %
\end{definition}
% 
\endinput
a session environment is delayable if for non-empty channel there does not exist a role able to receive from it

%
% ! (lemma 40) : balanced and delayable => timestep is balanced
\begin{lemma}\label{lem:delayed_balanced}
    %
    If \SesEnv*\ is \emph{balanced} 
    and \emph{delayable}
    and $\SesEnv+\ValTime$ is \emph{well-formed},
    then $\SesEnv+\ValTime$ is \emph{balanced}.
    %
\end{lemma}
\begin{proof}
    %
    \TODO
    %
\end{proof}
%  % 40
%
% ! (lemma 41) : delay defined, fail free, balanced => delayable
\begin{lemma}\label{lem:defined_delayable}
    %
    If \IPJudgement*\ %
    and \PFuncTime*\ is defined and \emph{fail-free}
    and \SesEnv*\ is \emph{balanced},
    then \SesEnv*\ is \emph{delayable}.
    %
\end{lemma}
\begin{proof}
    %
    \TODO
    %
\end{proof}
%  % 41
%
% ? (lemma 42) : neq prc, delay defined prc and sesenv => sesenv types delayed prc
\begin{lemma}\label{lem:typing_delayed_prc}
    %
    If \IPJudgement*\ %
    and $\PFuncTime=\Prc'$ is defined and \emph{fail-free}
    and $\PFuncNEQ=\emptyset$
    and $\SesEnv+\ValTime$ is \emph{well-formed},
    then \IPJudgement*<[\Prc']>[\SesEnv+\ValTime].
    %
\end{lemma}
\begin{proof}
    %
    \TODO
    %
\end{proof}
%  % 42

% 43
% 44

% 47

% ~ typing preserved across session reduction
\input{tex/figs/processes/session_reduction.tex}
% 49
% 50

% ~ fail free delay
% 51
% 52
% 53
% 54

% ~ fail free session
% 55

% ! TODO: untimed ? ? ? ? ?

% ! (lemma 43) : timestep preserve balanced sesenv
\begin{lemma}%\label{lem:}
    %
    \TODO
    %
\end{lemma}
\begin{proof}
    %
    \TODO
    %
\end{proof}

% ? (lemma 44) : 
\begin{lemma}%\label{lem:}
    %
    \TODO
    %
\end{lemma}
\begin{proof}
    %
    \TODO
    %
\end{proof}

% ! (lemma 45) : 
\begin{lemma}%\label{lem:}
    %
    \TODO
    %
\end{lemma}
\begin{proof}
    %
    \TODO
    %
\end{proof}

% ! (lemma 46) : 
\begin{lemma}%\label{lem:}
    %
    \TODO
    %
\end{lemma}
\begin{proof}
    %
    \TODO
    %
\end{proof}

% ! (lemma 47) : 
\begin{lemma}%\label{lem:}
    %
    \TODO
    %
\end{lemma}
\begin{proof}
    %
    \TODO
    %
\end{proof}

% ! (lemma 49) : 
\begin{lemma}%\label{lem:}
    %
    \TODO
    %
\end{lemma}
\begin{proof}
    %
    \TODO
    %
\end{proof}

% ! (lemma 50) : 
\begin{lemma}%\label{lem:}
    %
    \TODO
    %
\end{lemma}
\begin{proof}
    %
    \TODO
    %
\end{proof}

% ! (lemma 52) : 
\begin{lemma}%\label{lem:}
    %
    \TODO
    %
\end{lemma}
\begin{proof}
    %
    \TODO
    %
\end{proof}

% ! (lemma 54) : 
\begin{lemma}%\label{lem:}
    %
    \TODO
    %
\end{lemma}
\begin{proof}
    %
    \TODO
    %
\end{proof}

% ! (lemma 55) : 
\begin{lemma}%\label{lem:}
    %
    \TODO
    %
\end{lemma}
\begin{proof}
    %
    \TODO
    %
\end{proof}

% ! (lemma 56) : 
\begin{lemma}%\label{lem:}
    %
    \TODO
    %
\end{lemma}
\begin{proof}
    %
    \TODO
    %
\end{proof}

% ! (lemma 57) : 
\begin{lemma}%\label{lem:}
    %
    \TODO
    %
\end{lemma}
\begin{proof}
    %
    \TODO
    %
\end{proof}

% ! (lemma 58) : 
\begin{lemma}%\label{lem:}
    %
    \TODO
    %
\end{lemma}
\begin{proof}
    %
    \TODO
    %
\end{proof}

% ! (lemma 59) : 
\begin{lemma}%\label{lem:}
    %
    \TODO
    %
\end{lemma}
\begin{proof}
    %
    \TODO
    %
\end{proof}

% ! (lemma 60) : 
\begin{lemma}%\label{lem:}
    %
    \TODO
    %
\end{lemma}
\begin{proof}
    %
    \TODO
    %
\end{proof}

% ! (lemma 61) : 
\begin{lemma}%\label{lem:}
    %
    \TODO
    %
\end{lemma}
\begin{proof}
    %
    \TODO
    %
\end{proof}

% ! (lemma 62) : 
\begin{lemma}%\label{lem:}
    %
    \TODO
    %
\end{lemma}
\begin{proof}
    %
    \TODO
    %
\end{proof}

% ! (lemma 63) : 
\begin{lemma}%\label{lem:}
    %
    \TODO
    %
\end{lemma}
\begin{proof}
    %
    \TODO
    %
\end{proof}

% ! (lemma 64) : 
\begin{lemma}%\label{lem:}
    %
    \TODO
    %
\end{lemma}
\begin{proof}
    %
    \TODO
    %
\end{proof}

\endinput

% ! (lemma ) : 
\begin{lemma}%\label{lem:}
    %
    \TODO
    %
\end{lemma}
\begin{proof}
    %
    \TODO
    %
\end{proof}

% ~ () 
\begin{claim}
    %
    \TODO
    %
\end{claim}
