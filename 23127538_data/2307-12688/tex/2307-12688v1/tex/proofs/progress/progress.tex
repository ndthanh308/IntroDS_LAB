
% 
Building upon the notion of well-formed types (\cref{def:types_well_formed}), a well-formed configuration exhibits behaviour without deadlocks, where there is always some action able to be performed in the future, until the termination type is reached.
This behaviour is referred to as \emph{liveness}, and is given in~\cref{def:configs_iso_live}, and depends on \emph{future enabled} actions.
\begin{definition}[Future-enabled Configurations (\isFE*)]\label{def:configs_fe}
   A configuration \VIso*\ is \emph{future-enabled (\isFE*)} if\ %
   $\exists\ValTime$ such that \Trans*{\VIso}:{\ValTime,\CommMsg}[\VIso'].
   The same applies for \VSoc*.
 \end{definition}
\begin{definition}[Live Configurations]\label{def:configs_live}
    \CIso*\ is \emph{live} if
    $\TypeS=\TypeEnd$ or if \VIso*\ is \isFE*.
\end{definition}
The liveness of a configuration indicates whether a deadlock has been reached. There must either be an action immediately viable, some future enabled actions, or the type must have reached termination.

Building upon the notion of well-formed types (\cref{def:types_well_formed}), a well-formed configuration exhibits behaviour without deadlocks, where there is always some action able to be performed in the future, until the termination type is reached.
This behaviour is referred to as \emph{liveness}, and is given in~\cref{def:configs_iso_live}, and depends on \emph{future enabled} actions.
\begin{definition}[Future-enabled Configurations (\isFE*)]\label{def:configs_fe}
   A configuration \VIso*\ is \emph{future-enabled (\isFE*)} if\ %
   $\exists\ValTime$ such that \Trans*{\VIso}:{\ValTime,\CommMsg}[\VIso'].
   The same applies for \VSoc*.
 \end{definition}
\begin{definition}[Live Configurations]\label{def:configs_live}
    \CIso*\ is \emph{live} if
    $\TypeS=\TypeEnd$ or if \VIso*\ is \isFE*.
\end{definition}
The liveness of a configuration indicates whether a deadlock has been reached. There must either be an action immediately viable, some future enabled actions, or the type must have reached termination.

Building upon the notion of well-formed types (\cref{def:types_well_formed}), a well-formed configuration exhibits behaviour without deadlocks, where there is always some action able to be performed in the future, until the termination type is reached.
This behaviour is referred to as \emph{liveness}, and is given in~\cref{def:configs_iso_live}, and depends on \emph{future enabled} actions.
\input{tex/defs/configs/iso/future_en.tex}
\input{tex/defs/configs/iso/live.tex}
The liveness of a configuration indicates whether a deadlock has been reached. There must either be an action immediately viable, some future enabled actions, or the type must have reached termination.
\input{tex/defs/configs/iso/well_formed.tex}
Given a type \TypeS*\ well-formed against \ValClocks*, it holds that a configuration of \IsoCfg*\ will have future enabled actions, and therefore be live.
\begin{note}
    For a \emph{well-formed} \TypeS*, \CfgIso*1{\ValClocks_{0}} is \emph{live}.
\end{note}

Given a type \TypeS*\ well-formed against \ValClocks*, it holds that a configuration of \IsoCfg*\ will have future enabled actions, and therefore be live.
\begin{note}
    For a \emph{well-formed} \TypeS*, \CfgIso*1{\ValClocks_{0}} is \emph{live}.
\end{note}

Given a type \TypeS*\ well-formed against \ValClocks*, it holds that a configuration of \IsoCfg*\ will have future enabled actions, and therefore be live.
\begin{note}
    For a \emph{well-formed} \TypeS*, \CfgIso*1{\ValClocks_{0}} is \emph{live}.
\end{note}


\newcommand{\FullChoice}[0]{\left\{\TypComm_i \,\MsgLabel_i\left\langle T_i\right\rangle \left(\delta_i,\lambda_i\right).\TypeS_i\right\}_{i\in I}}

\begin{lemma}\label{lem:cfg_wf_neq_alpha}
   %
   If \CIso*\ is \emph{well-formed} then $\TypeS\neq\alpha$.
   %
\end{lemma}
\begin{proof}
   %
   By the hypothesis \CIso*\ is \emph{well-formed} meaning $\exists\Const$ such that $\ValClocks\models\Const$ and  $\emptyset;\Const~\Entails\TypeS$.

   Consider by contradiction of the hypothesis that $\TypeS=\alpha$.
   If \CIso*;[\alpha]\ is \emph{well-formed} then, by the judgement of rule \LblTypVar*\ $\alpha:\Const;\Const~\Entails\alpha$ and the evaluation shown in~\Cref{eq:cfg_wf_neq_alpha_eval} must hold, where there must be some recursive definition earlier in the derivation tree $\mu\alpha.\TypeS'$ corresponding to $\alpha$.
   %
   \begin{minieq}\label{eq:cfg_wf_neq_alpha_eval}
   \begin{array}[c]{l}
      \infer[\LblTypRec]{%
         \emptyset;\Const~\Entails\mu\alpha.\TypeS'%
      }{%   
         \infer{\alpha:\Const;\Const~\Entails\TypeS'}{%
            \infer[\vdots]{}{%
               \infer[\LblTypVar]{%
                  \alpha:\Const;\Const~\Entails\alpha%
               }{}%
            }%
         }%
      }
    \end{array}
   \end{minieq}
   
   \noindent If $\exists\ValClocks',\mu\alpha.\TypeS'$ such that \CIso*[\ValClocks'];[\mu\alpha.\TypeS']\ is \emph{well-formed} and \Trans*{\CIso[\ValClocks'];[\mu\alpha.\TypeS']}*[\CIso;[\alpha]], then the immediate transition must be via rule \LblCfgIsoUnfold*\ as shown in~\Cref{eq:cfg_wf_neq_alpha_unfold}; where either $\CIso[{\ValClocks}''];[{\TypeS}'']=\CIso;[\alpha]$ or \Trans*{\CIso[{\ValClocks}''];[{\TypeS}'']}*[\CIso;[\alpha]].
   %
   By the premise of rule \LblCfgIsoUnfold*\ any recursive calls following their definition are replaced by the definition of their next unfolding denoted by $\TypeS'\Subst[\mu\alpha.\TypeS'][\alpha]$.
   %
   \begin{minieq}\label{eq:cfg_wf_neq_alpha_unfold}
   \begin{array}[c]{l}
      \infer[\LblCfgIsoUnfold]{%
      \Trans{\CIso[\ValClocks'];[\mu\alpha.\TypeS']}:{\ProgAction}[\CIso[{\ValClocks}''];[{\TypeS}'']]
      }{%
      \Trans{\CIso[\ValClocks'];[{\TypeS}'\Subst[\mu\alpha.\TypeS'][\alpha]]}:{\ProgAction}[\CIso[{\ValClocks}''];[{\TypeS}'']]
      }
    \end{array}
   \end{minieq}
   
   \noindent Therefore, if \CIso*\ is \emph{well-formed} then $\TypeS\neq\alpha$.
   %
\end{proof}

% \begin{definition}[Live Configurations]\label{def:configs_live}
    \CIso*\ is \emph{live} if
    $\TypeS=\TypeEnd$ or if \VIso*\ is \isFE*.
\end{definition}

\begin{lemma}\label{lem:cfg_wf_end}
   %
   \CIso*;[\TypeEnd]\ is always \emph{well-formed}.
   %
\end{lemma}
\begin{proof}
   %
   By the hypothesis $\exists\Const$ such that $\ValClocks\models\Const$ and $\emptyset;\Const~\Entails\TypeS$, and by rule \LblTypEnd*\ $\emptyset;\TypeTrue~\Entails\TypeEnd$.
%   
   Therefore the hypothesis holds as $\ValClocks\models\TypeTrue$ always holds.
   %
\end{proof}

%
\begin{lemma}\label{lem:cfg_wf_then_live}
    %
    If \CIso*\ is \emph{well-formed}, then \CIso*\ is \emph{live}.
    %
\end{lemma}
\begin{proof}
    %
    By the hypothesis, there must $\exists\Const$ such that $\emptyset;\Const~\Entails\TypeS$ and $\ValClocks\models\Const$.
    %
    We proceed by induction on each case of \TypeS*:
    \begin{inductivecase}
        %
        %
        %
        % ~ choice
        \item\NewCase[$\TypeS=\simplechoice$]\label{itm:cfg_wf_then_live_choice} 
        By the hypothesis and the judgement of rule \LblTypChoice*\ $\exists\Const_i$ such that $\ValClocks\models\Past[\Const_i]$ and $\emptyset;\Past[\Const_i]~\Entails\FullChoice$.
        %
        By~\Cref{def:configs_wf,def:configs_fe} \CIso*\ is \isFE*, and therefore, by~\Cref{def:configs_live} \CIso*\ is \emph{live}.
        %
        %
        %
        % ~ rec def
        \item\NewCase[$\TypeS=\mu\alpha.{\TypeS}'$]\label{itm:wf_then_live_recdef} 
        By the hypothesis $\exists\Const$ such that $\ValClocks\models\Const$ and $\emptyset;\Const~\Entails\TypRecDef$, and by the premise of rule \LblTypRec*\ $\alpha:\Const;\Const~\Entails{\TypeS}'$ and \CIso*;[\TypeS']\ is \emph{well-formed}.
        
        We proceed by inner induction on each case of \TypeS*':
        \begin{inductivecase}
            %
            %
            % ~ rec def -> choice
            \item\NewCase[$\TypeS'=\simplechoice$] 
            By the judgement of rule \LblTypChoice*\ and the premise of rule \LblTypRec*, $\exists\Const_i$ such that $\ValClocks\models\Const_i$ and $\emptyset;\Const_i~\Entails\TypRecDef$ and $\alpha:\Const_i;\Past[\Const_i]~\Entails\FullChoice$.
            %
            It follows~\Cref{itm:cfg_wf_then_live_choice} of~\Cref{lem:cfg_wf_then_live} that, by~\Cref{def:configs_wf,def:configs_fe,def:configs_live}, \CIso*;[\simplechoice]\ is \isFE*, \emph{well-formed} and \emph{live}.
            Therefore, it holds that \CIso*\ is \emph{live}.
            % Then by the judgement of \LblTypChoice*\ and the premise of \LblTypRec*\ $\exists\Const_i$ such that $\ValClocks\models\Const_i$ and $\emptyset;\Const_i~\Entails\TypRecDef$ and $\alpha:\Const_i;\Past[\Const_i]~\Entails\simplechoice$ and \CIso*;[\simplechoice]\ is \emph{well-formed} and \emph{future-enabled} by~\cref{def:configs_fe}.
            %
            % Therefore \CIso*;[\simplechoice]\ is \emph{live} by definition.
            %
            %
            % ~ rec def -> rec def
            \item\NewCase[$\TypeS'=\mu\alpha'.{\TypeS}''$] 
            Then $\alpha:\Const;\Const~\Entails\mu\alpha'.{\TypeS}''$ and $\alpha:\Const,\alpha':\Const;\Const~\Entails{\TypeS}''$ and \CIso*;[{\TypeS}'']\ is \emph{well-formed}.
            %
            Therefore, \CIso*\ is \emph{live} if \CIso*;[{\TypeS}'']\ is \emph{live} (see other cases).
            %
            %
            % ~ rec def -> end
            \item $\TypeS'=\TypeEnd$ is \emph{live} by~\Cref{def:configs_live}.
            %
            %
            % ~ rec def -> alpha
            \item ${\TypeS'}$ cannot equal $\alpha$ by~\Cref{lem:cfg_wf_neq_alpha}.
            %
        \end{inductivecase}
        %
        %
        %
        % ~ end
        \item\NewCase[$\TypeS=\TypeEnd$]\label{itm:wf_then_live_end} 
        By~\Cref{def:configs_live} it holds that \CIso*;[\TypeEnd]\ is live.
        %
        %
        %
        % ~ alpha
        \item\NewCase[$\TypeS\neq\alpha$] By~\Cref{lem:cfg_wf_neq_alpha}.
        %
    \end{inductivecase}

    \noindent Therefore, it holds that for any $S$ that \CIso*\ is \emph{well-formed}, \CIso*\ is \emph{live}.
    %
\end{proof}
%
%

%

\begin{lemma}\label{lem:init_wf_then_live}
   %
   Given a \emph{well-formed} \TypeS*, \CIso*[\ValClocks_0]\ is \emph{live}.
   %
\end{lemma}
\begin{proof}
   %
   By~\Cref{def:types_wf}, \Sat*[\ValClocks_0]:[\Past]\ holds for any valid \Const*.
   %
\end{proof}

% ~ configuration transitions
% ! 
%
% ! (lemma 12) : iso cfg trans
\begin{lemma}\label{lem:configs_iso_trans}
   %
   The following holds for transitions of configurations:
   \begin{minieq}*
      \begin{array}[c]{c c l}
         %
         \Trans{\CIso}:{\ValTime}[\CIso']%
         & \implies & %
         \begin{array}[c]{lcl}
            %
            \ValClocks'=\ValClocks+\ValTime & %
            \land & %
            \TypeS'=\TypeS%
            %
         \end{array}
         %
         \\[-1ex]\\%
         \Trans{\CIso}:{\CommAction}[\CIso']%
         & \implies & %
         \begin{array}[t]{lcl c lcl c lcl}
            %
            \ValClocks & \models & \Const & %
            \land & %
            \ValClocks' & = & \ReSet[\ValClocks] & %
            \land & %
            \Const' & = & \ReSet[\Const]%
            %
            \\[-1ex]\\%
            \mathllap{\land\;}\ValClocks' & \models & \Const' & %
            \land & %
            \Const' & \subseteq & \TypEnvCond[\TypeS'] %
            %
         \end{array}%
         %
      \end{array}%
   \end{minieq}
   %
\end{lemma}
\begin{proof}
   %
   By inspection of the formation and transition rules in~\Cref{fig:types_rule,fig:typesemantics_tuple}.
   %
\end{proof}
%
%
% ! (lemma 11) : soc cfg trans
\begin{lemma}\label{lem:configs_soc_trans}
   %
   The following holds for transitions of configurations with queues:
   \begin{minieq}*
      \begin{array}[c]{c c lcl c lcl c l}
         %
         \Trans{\CSoc}:{\ValTime}[\CSoc']%
         & \implies & %
            %
            \TypeS' & = & \TypeS & \land & %
            \Queue' & = & \Queue & \land & %
            \ValClocks'=\ValClocks+\ValTime%
            %
         \\[1ex]%
         \Trans{\CSoc}:{\RecvMsg}[\CSoc']%
         & \implies & %
            %
            \TypeS' & = & \TypeS & \land & %
            \Queue' & = & \Queue;\Msg & \land & %
            \ValClocks'=\ValClocks
            %
         \\[1ex]%
         \Trans{\CSoc}:{\SendMsg}[\CSoc']%
         & \implies & %
            %
            \MsgType & = & \Msg & \land & %
            \Queue' & = & \Queue & \land & %
            \Trans{\CIso}:{\SendAction}[\CIso'] %
            %
         \\[1ex]%
         \Trans{\CSoc}:{\SiltAction}[\CSoc']%
         & \implies & %
            % 
            \MsgType & = & \Msg & \land & %
            \Queue' & = & \Msg;\Queue & \land & %
            \Trans{\CIso}:{\RecvAction}[\CIso']%
            %
      \end{array}
   \end{minieq}
   %
\end{lemma}
\begin{proof}
   %
   By inspection of the transitions in~\Cref{fig:typesemantics_triple}, supported by~\Cref{lem:configs_iso_trans}.
   %
\end{proof}
%

% ~ preservation of wf and compat
% \begin{definition}[Compatible Systems]\label{def:configs_compat}
    %
    Let $\VSoc_1 = \CSoc_1$ and $\VSoc_2 = \CSoc_2$. 
    System \VSys*\ is \emph{compatible} (written $\VSoc_1\bot\, \VSoc_2$) if:
    \begin{enumerate}
    \item\label{itm:configs_compat_non_empty_queues} $\Queue_1=\emptyset%
            ~\lor~%
            \Queue_2=\emptyset$%
        %
        \\
       \item\label{itm:configs_compat_dual_types} $\Queue_1=\Queue_2=\emptyset%
            ~\implies~%
            \ValClocks_1=\ValClocks_2%
            ~\land~%
            \TypeS_1=\Dual_2$
        %
        \\
  \item
    \label{itm:configs_compat_expected_receive} 
    $\Queue_1=\Msg;\Queue'_1
            ~\Implies~%
            \exists\ValClocks'_1,\TypeS'_1:
            \Trans{\CIso_1}:{\RecvMsg}[\CIso'_1]%
            ~\land~%
            \CSoc'_1 \bot\, \VSoc_2$
         %   \Compat[\VSoc_1'][\VSoc_2]$%
            \\
\item
    %\label{itm:configs_compat_expected_receive} 
    $\Queue_2=\Msg;\Queue'_2
            ~\Implies~%
            \exists\ValClocks'_2,\TypeS'_2:
            \Trans{\CIso_2}:{\RecvMsg}[\CIso'_2]%
            ~\land~%
            \VSoc_1 \bot\, \CSoc'_2$
         %   \Compat[\VSoc_1'][\VSoc_2]$%
            \\
        % $\forall i, j\in\mkSet[1,2]:%
        % i\neq j%
        % \quad%
        % \Queue_i=\Msg;\Queue_i'%
        % ~\Implies~%
        % \exists\ValClocks'_i,\TypeS'_i:%
        % \Trans{\CIso_i}:{\RecvMsg}[\CIso'_i]%
        % ~\land~%
        % \Compat[\VSoc'_i][\VSoc_j]$%
        %
    \end{enumerate}
    %
    % \noindent We write \Compat*\ if system \VSys*\ is compatible.
    % \Cref{itm:configs_compat_expected_receive} is symmetric.
\end{definition}
% \begin{definition}[Latest-enabled Action (\isLE*)]\label{def:configs_le}
    %
    A configuration \CIso*, that is future-enabled, has a \emph{latest-enabled send} (resp. \emph{latest-enabled receive}), or \isLE*!\ for short (resp. \isLE*?), 
    if $\forall t$ such that $\CIso+{+t}\isFE~\Implies~\exists t'\geq t: \Trans{\VIso}:{\ValTime',\SendMsg}$.
\end{definition}

% ! 
%
% ! (lemma 17) : compat wf, single transition -> wf
\begin{lemma}\label{lem:cfgs_trans_wf_pres}
    %
    If \VIso*_1\ and \VIso*_2\ are both \emph{well-formed} 
    and \Compat*[\VSoc_1][\VSoc_2]\ 
    and \Trans*{\Parl{\VSoc_1,\VSoc_2}}[\Parl{\VSoc'_1,\VSoc'_2}], 
    then both \VIso*'_1\ and \VIso*'_2\ are \emph{well-formed}.
    %
\end{lemma}
\begin{proof}
    %
    We proceed by induction on the depth of the derivation tree, analysing each case of the last rule applied for the transition \Trans*{\Parl{\VSoc_1,\VSoc_2}}[\Parl{\VSoc'_1,\VSoc'_2}].
    % \begin{inline}+
    %     \item \LblCfgSysWait*
    %     \item \LblCfgSysLComm*
    %     \item \LblCfgSysLPar*
    % \end{inline}
    %
    \begin{inductivecase}
        %
        %
        %
        %
        %
        % ~ wait
        \item\NewCase[\LblCfgSysWait*]\label{case:cfgs_trans_wf_pres_wait} 
        As shown in~\Cref{eq:cfgs_trans_wf_pres_wait_trans}, both \VSoc*_1\ and \VSoc*_2\ make a rule \LblCfgSocTime*\ transition with the same valuation of \ValTime*.
        If $\ValTime=0$ then by~\Cref{lem:configs_iso_trans,lem:configs_soc_trans} $\VIso_1=\VIso'_1$ and $\VIso_2=\VIso'_2$ and both \VIso*'_1\ and \VIso*'_2\ are \emph{well-formed}.
        %
        \begin{minieq}\label{eq:cfgs_trans_wf_pres_wait_trans}
        \begin{array}[c]{l}
            \infer[\LblCfgSysWait]{%
                \Trans{\Parl{\VSoc_1,\VSoc_2}}:{\ValTime}[\Parl{\VSoc'_1,\VSoc'_2}]
            }{%
                \infer[\LblCfgSocTime]{%
                    \Trans{\VSoc_1}:{\ValTime}[\VSoc'_1]
                }{\dots}
                & %
                \infer[\LblCfgSocTime]{%
                    \Trans{\VSoc_2}:{\ValTime}[\VSoc'_2]
                }{\dots}
            }
            \end{array}
        \end{minieq}
            
        \noindent We proceed with only \VSoc*_1\ as the analysis covers \VSoc*_2.
        If $\ValTime>0$ then by~\Cref{lem:configs_iso_trans,lem:configs_soc_trans} $\ValClocks'_1=\ValClocks_1+\ValTime$ and $\TypeS'_1=\TypeS_1$.
        %
        By induction on the depth of the derivation tree, analysing the last rule applied for the transition \Trans*{\VIso_1}:{\ValTime}[\VIso'_1]:
        \begin{inductivecase}
            %
            %
            % ~ wait -> tick
            \item\NewCase[\LblCfgIsoTick*]\label{case:cfgs_trans_wf_pres_wait_tgtz_tick} 
            Then \Trans*{\CIso_1}:{\ValTime}[\CIso+{+\ValTime}_1].
            We proceed by inner induction on the different cases of \TypeS*_1:
            \begin{inductivecase}
                %
                % ~ wait -> tick -> choice
                \item\NewCase[$\TypeS_1=\simplechoice$]\label{case:cfgs_trans_wf_pres_wait_tgtz_tick_choice} 
                By rule \LblCfgIsoInteract*\ $\exists\Const_i$ such that $\ValClocks_1\models\Past[\Const_i]$ and $\emptyset;\Past[\Const_i]~\Entails\FullChoice$, as in~\Cref{itm:cfg_wf_then_live_choice} of~\Cref{lem:cfg_wf_then_live}.
                %
                By induction hypothesis $\exists\Const'_i$ such that $\ValClocks_1+\ValTime\models\Past[\Const'_i]$ and $\emptyset;\Past[\Const'_i]~\Entails\FullChoice$, which is assured by the (persistency) premise of rule \LblCfgSocTime*.
                %
                Therefore, it holds that \CIso*[\ValClocks_1]+{+\ValTime};[\simplechoice]\ is \emph{well-formed}.
                % By~\Cref{def:configs_fe} \CIso*_1\ is \emph{future-enabled} and by rule \LblCfgSocTime*\ $\text{(persistency)}$ it holds that \CIso*+{+\ValTime}_1\ is also \emph{future-enabled}.
                % %
                % Therefore the hypothesis holds, \CIso*[\ValClocks_1]+{+\ValTime};[\simplechoice]\ is \emph{well-formed}; by~\Cref{def:types_wf} and rule \LblTypChoice*\ $\exists\Const'_i$ such that $\ValClocks_1+\ValTime\models\Past[\Const'_i]$ and $\emptyset;\Past[\Const'_i]~\Entails\simplechoice$.
                %
                %
                % ~ wait -> tick -> recursion
                \item\NewCase[$\TypeS_1=\mu\alpha.{\TypeS}''_1$]\label{case:cfgs_trans_wf_pres_wait_tgtz_tick_recursion} 
                By~\Cref{def:types_wf} $\exists\Const$ such that $\ValClocks_1\models\Const$ and $\emptyset;\Const~\Entails\mu\alpha.{\TypeS}''_1$, and (by rule \LblTypRec*) $\alpha:\Const;\Const~\Entails{\TypeS}''_1$ and \CIso*[\ValClocks_1];[{\TypeS}''_1]\ is \emph{well-formed}.
                %
                Therefore, the well-formedness of \CIso*[\ValClocks_1]+{+\ValTime};[\mu\alpha.{\TypeS}''_1]\ is dependant on the well-formedness of \CIso*[\ValClocks_1]+{+\ValTime};[{\TypeS}''_1]. (See other cases, as in~\Cref{itm:wf_then_live_recdef} of~\Cref{lem:cfg_wf_then_live}.)
                %
                %
                % ~ wait -> tick -> end
                \item\NewCase[$\TypeS_1=\TypeEnd$]\label{case:cfgs_trans_wf_pres_wait_tgtz_tick_end} 
                By~\Cref{lem:cfg_wf_end} \CIso*+{+\ValTime};[\TypeEnd]\ is \emph{well-formed}.
                %
                %
                % ~ wait -> tick -> rec call
                \item\label{case:cfgs_trans_wf_pres_wait_tgtz_tick_reccal} ${\TypeS}_1$ cannot equal $\alpha$ by~\Cref{lem:cfg_wf_neq_alpha}.
                %
            \end{inductivecase}
            %
            %
            % ~ wait -> unfold
            \item\NewCase[\LblCfgIsoUnfold*]\label{case:cfgs_trans_wf_pres_wait_tgtz_unfold} 
            Then $\TypeS_1=\mu\alpha.{\TypeS}''_1$ and by the hypothesis $\exists\Const$ such that $\ValClocks_1\models\Const$ and $\emptyset;\Const~\Entails\TypRecDef$, and by rule \LblTypRec*\ $\alpha:\Const;\Const~\Entails{\TypeS}''_1$.
            The transition is as shown below:
            %
            \begin{minieq}*%\label{eq:cfgs_trans_wf_pres_wait_tgtz_unfold_trans}
        \begin{array}[c]{l}
                \infer[\LblCfgIsoUnfold]{%
                    \Trans{\CIso[\ValClocks_1];[\mu\alpha.{\TypeS}''_1]}:{\ProgAction}[\CIso'_1]
                }{%
                    \Trans{\CIso[\ValClocks_1];[{\TypeS}''_1\Subst[\mu\alpha.{\TypeS}''_1][\alpha]]}:{\ValTime}[\CIso'_1]
                }
                \end{array}
            \end{minieq}
            
            \noindent By inner induction on the different cases of ${\TypeS}''_1$:
            \begin{inductivecase}
                %
                % ~ wait -> unfold -> choice
                \item\NewCase[${\TypeS}''_1=\simplechoice$]\label{case:cfgs_trans_wf_pres_wait_tgtz_unfold_choice} 
                Then, by rule \LblCfgIsoTick*:
                \[\Trans{\CIso[\ValClocks_1];[{\simplechoice}\Subst[\mu\alpha.{\simplechoice}][\alpha]]}:{\ValTime}[\CIso[\ValClocks_1]+{+\ValTime};[\simplechoice]]\]
                
                \noindent By the rules \LblTypRec*\ and \LblTypChoice*\ the following holds:
                \[
                \infer[\LblTypRec]{%
                \emptyset;\Const_i~\Entails\mu\alpha.{\FullChoice}
                }{%
                \infer[\LblTypChoice]{%
                    \alpha:\Const_i;\Past_i~\Entails\FullChoice
                }{%
                \dots
                }
                }
                \]
                
                % \noindent It holds that \CIso*[\ValClocks_1];[{\simplechoice}\Subst[\mu\alpha.{\simplechoice}][\alpha]]\ is \emph{well-formed} and \emph{future-enabled}.
                %
                \noindent By induction hypothesis: \[\exists\Const'_i:\ValClocks_1+\ValTime\models\Past[\Const'_i] ~\land~ \emptyset;\Past[\Const'_i]~\Entails\FullChoice\] which is assured by the (persistency) premise of rule \LblCfgSocTime*, as in~\Cref{case:cfgs_trans_wf_pres_wait_tgtz_tick_choice} of~\Cref{lem:cfgs_trans_wf_pres}.
% 
                Therefore, \CIso*[\ValClocks_1]+{+\ValTime};[\mu\alpha.{\simplechoice}]\ is \emph{well-formed} as \CIso*[\ValClocks_1]+{+\ValTime};[{\simplechoice}\Subst[\mu\alpha.{\simplechoice}][\alpha]]\ the following is \emph{well-formed}.
                %
                %
                % ~ wait -> unfold -> recursion
                \item\NewCase[${\TypeS}''_1=\mu\alpha'.{\TypeS}'''_1$]\label{case:cfgs_trans_wf_pres_wait_tgtz_unfold_recursion} 
                % By the hypothesis $\exists\Const$ such that $\ValClocks_1\models\Const$ and $\emptyset;\Const~\Entails\mu\alpha.\mu\alpha'.{\TypeS}'''_1$, by the premise of rule \LblTypRec*\ $\alpha:\Const;\Const~\Entails\mu\alpha'.{\TypeS}'''_1$ and $\alpha:\Const,\alpha':\Const;\Const~\Entails{\TypeS}'''_1$, and \CIso*[\ValClocks_1];[{\TypeS}'''_1]\ is \emph{well-formed}.
                % 
                The well-formedness of \CIso*[\ValClocks_1]+{+\ValTime};[\mu\alpha'.{\TypeS}'''_1]\ depends on the well-formedness of \CIso*[\ValClocks_1]+{+\ValTime};[{\TypeS}'''_1]. (See other cases of $S$.) %, as in~\Cref{itm:wf_then_live_recdef} of~\Cref{lem:cfg_wf_then_live}.)
                %
                %
                % ~ wait -> unfold -> end
                \item\NewCase[${\TypeS}''_1=\TypeEnd$]\label{case:cfgs_trans_wf_pres_wait_tgtz_unfold_end} 
                By~\Cref{lem:cfg_wf_end} \CIso*+{+\ValTime};[\TypeEnd]\ is \emph{well-formed}.
                %
                %
                % ~ wait -> unfold -> rec call
                \item\label{case:cfgs_trans_wf_pres_wait_tgtz_unfold_reccall} ${\TypeS}''_1$ cannot equal $\alpha$ by~\Cref{lem:cfg_wf_neq_alpha}.
                %
            \end{inductivecase}
            %
        \end{inductivecase}

        \noindent Therefore, it holds that well-formedness is preserved by transition made by \emph{well-formed} configurations via rule \LblCfgSysWait*.
        %
        %
        %
        %
        %
        % ~ comm
        \item\NewCase[\LblCfgSysLComm*]\label{case:cfgs_trans_wf_pres_comm} 
        % Then by~\cref{case:configs_trans_compat_pres_comm} of~\cref{lem:configs_trans_compat_pres} $\Queue_1=\Queue'_1=\emptyset$.
        The transition is as shown below:
        %
        \begin{minieq}*%\label{eq:cfgs_trans_wf_pres_comm_trans}
        \begin{array}[c]{l}
            \infer[\LblCfgSysLComm]{%
                \Trans{\Parl{\VSoc_1,\VSoc_2}}:{\SiltAction}[\Parl{\VSoc'_1,\VSoc'_2}]
            }{%
                \infer[\LblCfgSocSend]{%
                    \Trans{\VSoc_1}:{\SendMsg}[\VSoc'_1]
                }{\dots}
                & %
                % \infer[\LblCfgSocEnqu]{%
                    \Trans{\VSoc_2}:{\RecvMsg}[\VSoc'_2]
					\quad \LblCfgSocEnqu
                % }{\dots}
            }
            \end{array}
        \end{minieq}
        
        % ~ comm-l s1
        \noindent Focusing first on \VSoc*_1, we proceed by induction on the depth of the derivation tree, analysing the last rule applied for the transition \Trans*{\VIso_1}:{\SendMsg}[\VIso'_1]:
        \begin{inductivecase}
            %
            %
            % ~ comm-l s1 -> act
            \item\NewCase[\LblCfgIsoInteract*]\label{case:cfgs_trans_wf_pres_comm_act} 
            Then $\TypeS_1=\simplechoice$, and the evaluation is shown below:
            %
            \begin{minieq}*\label{eq:cfgs_trans_wf_pres_comm_act_trans}
        \begin{array}[c]{l}
                \infer[\LblCfgSocSend]{%
                    \Trans{\CSoc[\ValClocks_1];[\simplechoice]:{\Queue_1}}:{\SendMsg}[\CSoc[\ValClocks_1]+{\ReSet[]_j};[\TypeS_j]:{\Queue_1}]
                }{%
                    \infer[\LblCfgIsoInteract]{%
                        \Trans{\CIso[\ValClocks_1];[\TypInteract]}:{\SendMsg}[\CIso[\ValClocks_1]+{\ReSet[]_j};[\TypeS_j]]
                    }{%
                        \ValClocks_1\models\Const_j
                        & %
                        {m}={l_j\left\langle T_j \right\rangle}
                        & % 
                        {\TypSend=\TypComm_j}
                        & % 
                        j\in I
                    }
                }
                \end{array}
            \end{minieq}
            
            \noindent By rule \LblTypChoice*\ $\exists\Const_i:\ValClocks_1\models\Past[\Const_i]$ and $\emptyset;\Past[\Const_i]~\Entails\FullChoice$, and by the premise of rule \LblTypChoice*\ it holds that $\Const_i\ReSet[]_i\subseteq\gamma$ and $\emptyset;\gamma~\Entails\TypeS_i$.
            %
            Combined with~\Cref{lem:configs_iso_trans} it holds that $\ValClocks_1\models\Const_j$ and $\emptyset;\Const_j\ReSet[]_j~\Entails\TypeS_j$ and $\ValClocks_1\ReSet[]_j\models\Const_j\ReSet[]_j$.
            
            Therefore, \CIso*[\ValClocks_1]+{\ReSet[]_j};[\TypeS_j]\ is \emph{well-formed}.
            %
            %
            % ~ comm-l s1 -> unfold
            \item\NewCase[\LblCfgIsoUnfold*]\label{case:cfgs_trans_wf_pres_comm_unfold} 
            Then $\TypeS_1=\mu\alpha.{\TypeS}''_1$.
            The transition is as shown below:
            %
            \begin{minieq}*\label{eq:cfgs_trans_wf_pres_comm_unfold_trans}
        \begin{array}[c]{l}
                \infer[\LblCfgIsoUnfold]{%
                    \Trans{\CIso[\ValClocks_1];[\mu\alpha.{\TypeS}''_1]}:{\ProgAction}[\CIso'_1]
                }{%
                    \Trans{\CIso[\ValClocks_1];[{\TypeS}''_1\Subst[\mu\alpha.{\TypeS}''_1][\alpha]]}:{\SendMsg}[\CIso'_1]
                }
                \end{array}
            \end{minieq}
            
            \noindent By the hypothesis $\exists\Const$ such that $\ValClocks_1\models\Const$ and $\emptyset;\Const~\Entails\TypRecDef$, and by the premise of rule \LblTypRec*\ $\alpha:\Const;\Const~\Entails{\TypeS}''_1$, and \CIso*[\ValClocks_1];[{\TypeS}''_1\Subst[\mu\alpha.{\TypeS}''_1][\alpha]]\ is \emph{well-formed}.
            %
            The well-formendess of \CIso*'_1\ is dependant on the state of ${\TypeS}''_1$, which for the transition \Trans*{\CIso[\ValClocks_1];[{\TypeS}''_1\Subst[\mu\alpha.{\TypeS}''_1][\alpha]]}:{\SendMsg}[\CIso'_1]\ must be either $\simplechoice$ or $\mu\alpha'.{\TypeS}'''_1$ (see other cases, as in~\Cref{lem:cfg_wf_then_live}).
            %
        \end{inductivecase}
        %
        % ~ comm-l s1
        Now, focusing on \VSoc*_2, the transition \Trans*{\CSoc_2}:{\RecvMsg}[\CSoc'_2]\ via rule \LblCfgSocEnqu*\ yields $\ValClocks_2'=\ValClocks_2$ and $\TypeS_2'=\TypeS_2$ and $\Queue_2'=\Queue_2;\Msg$ by~\Cref{lem:configs_soc_trans}.
        %
        Therefore \CIso*'_2\ is \emph{well-formed} as $\VIso_2=\VIso'_2$.
        %
        Transitions via rule \LblCfgSysRComm*\ are symmetric and omitted.
        %
        %
        %
        %
        %
        % ~ par-l
        \item\NewCase[\LblCfgSysLPar*]\label{case:cfgs_trans_wf_pres_par} 
        By~\Cref{lem:configs_soc_trans} $\Queue'_1=\Msg;\Queue_1$.
        The transition is as shown below: %in~\Cref{eq:cfgs_trans_wf_pres_par_trans}.
        %
        \begin{minieq}*\label{eq:cfgs_trans_wf_pres_par_trans}
        \begin{array}[c]{l}
            \infer[\LblCfgSysLPar]{%
                \Trans{\Parl{\VSoc_1,\VSoc_2}}:{\SiltAction}[\Parl{\VSoc'_1,\VSoc'_2}]
            }{%
                \infer[\LblCfgSocRecv]{%
                    \Trans{\VSoc_1}:{\SiltAction}[\VSoc'_1]
                }{\dots}
            }
            \end{array}
        \end{minieq}
        
        \noindent We proceed by induction on the depth of the derivation tree, analysing the last rule applied for the transition \Trans*{\VIso_1}:{\RecvMsg}[\VIso'_1]\ via the premise of rule \LblCfgSocRecv*:
        \begin{inductivecase}
            %
            %
            % ~ par-l -> act
            \item\NewCase[\LblCfgIsoInteract*]\label{case:cfgs_trans_wf_pres_par_act} 
            Then $\TypeS_1=\simplechoice$.
            The transition is as shown below: %in~\Cref{eq:cfgs_trans_wf_pres_par_act_trans}.
            %
            \begin{minieq}*\label{eq:cfgs_trans_wf_pres_par_act_trans}
        \begin{array}[c]{l}
                \infer[\LblCfgSocRecv]{%
                    \Trans{\CSoc[\ValClocks_1];[\simplechoice]:{\Msg;\Queue_1}}:{\SiltAction}[\CSoc[\ValClocks_1]+{\ReSet[]_j};[\TypeS_j]:{\Queue_1}]
                }{%
                    \infer[\LblCfgIsoInteract]{%
                        \Trans{\CIso[\ValClocks_1];[\TypInteract]}:{\RecvMsg}[\CIso[\ValClocks_1]+{\ReSet[]_j};[\TypeS_j]]
                    }{%
                        \ValClocks_1\models\Const_j
                        & %
                        {m}={l_j\left\langle T_j \right\rangle}
                        & % 
                        {\TypRecv=\TypComm_j}
                        & % 
                        j\in I
                    }
                }
                \end{array}
            \end{minieq}
            
            \noindent By the hypothesis and the judgement of rule \LblTypChoice*\ $\exists\Const_i$ such that $\emptyset;\Past[\Const_i]~\Entails\simplechoice$ and $\ValClocks_1\models\Past[\Const_i]$, and by the premise of rule \LblTypChoice*\ $\Const_i\ReSet[]_i\subseteq\gamma$ and $\emptyset;\gamma~\Entails\TypeS_i$.
            
            It follows~\Cref{case:cfgs_trans_wf_pres_comm_act} of~\Cref{lem:cfgs_trans_wf_pres} that \CIso*[\ValClocks_1]+{\ReSet[]_j};[\TypeS_j]\ is \emph{well-formed}.
            % Therefore \CIso*[\ValClocks_1]+{\ReSet[]_j};[\TypeS_j]\ is \emph{well-formed} as $\ValClocks_1\models\Const_j$ and $\emptyset;\Const_j\ReSet[]_j~\Entails\TypeS_j$ and $\ValClocks_1\ReSet[]_j\models\Const_j\ReSet[]_j$ (as in~\Cref{case:cfgs_trans_wf_pres_comm_act} of~\Cref{lem:cfgs_trans_wf_pres}).
            %
            %
            % ~ par-l -> unfold
            \item\NewCase[\LblCfgIsoUnfold*]\label{case:cfgs_trans_wf_pres_par_unfold} 
            Then $\TypeS_1=\mu\alpha.{\TypeS}''_1$.
            The transition shown below, and is analogous to the one in~\Cref{eq:cfgs_trans_wf_pres_comm_unfold_trans} of~\Cref{lem:cfgs_trans_wf_pres}:
            % in~\Cref{eq:cfgs_trans_wf_pres_par_unfold_trans} and is analogous to the one in~\Cref{eq:cfgs_trans_wf_pres_comm_unfold_trans}.
            %
            \begin{minieq}\label{eq:cfgs_trans_wf_pres_par_unfold_trans}
        \begin{array}[c]{l}
                \infer[\LblCfgIsoUnfold]{%
                    \Trans{\CIso[\ValClocks_1];[\mu\alpha.{\TypeS}''_1]}:{\ProgAction}[\CIso'_1]
                }{%
                    \Trans{\CIso[\ValClocks_1];[{\TypeS}''_1\Subst[\mu\alpha.{\TypeS}''_1][\alpha]]}:{\RecvMsg}[\CIso'_1]
                }
                \end{array}
            \end{minieq}
            
            \noindent By the hypothesis $\exists\Const$ such that $\ValClocks_1\models\Const$ and $\emptyset;\Const~\Entails\TypRecDef$, and by the premise of rule \LblTypRec*\ $\alpha:\Const;\Const~\Entails{\TypeS}''_1$, and \CIso*[\ValClocks_1];[{\TypeS}''_1\Subst[\mu\alpha.{\TypeS}''_1][\alpha]]\ is \emph{well-formed}.
            %
            The well-formendess of \CIso*'_1\ is dependant on ${\TypeS}''_1$, which for the transition by the premise of rule \LblCfgIsoUnfold*\ must be either $\simplechoice$ or $\mu\alpha'.{\TypeS}'''_1$ (see other cases, as in~\Cref{lem:cfg_wf_then_live}).
            %
        \end{inductivecase}
        %
    \end{inductivecase}

    \noindent Therefore, it holds that any transition made by a system composed of compatible and \emph{well-formed} configurations will result in configurations that are \emph{well-formed}.
    %
\end{proof}
% 

% ~ time passing
% ! 
% \newpage
%
% ! (lemma 13) : time passing implies empty queues
\begin{lemma}\label{lem:sys_compat_time_trans}
   %
   If \Compat*[\VSoc_1][\VSoc_2]\ and \Trans*{\Parl{\VSoc_1,\VSoc_2}}:{\ValTime}\ and $\ValTime>0$ then $\Queue_1=\emptyset=\Queue_2$.
   %
\end{lemma}
\begin{proof}
   %
   Such a transition is only specified by \LblCfgSysWait*, which by its premise requires a \LblCfgSocTime*\ transition of \ValTime*\ for each \VSoc*_1\ and \VSoc*_2.
   %
   By contradiction, if one queue were \emph{non-empty}, say $\Queue_1=\Msg;\Queue_1$, then by~\Cref{itm:configs_compat_expected_receive} of~\Cref{def:configs_compat} message \Msg*\ must be able to be received immediately.
   %
   The premise of \LblCfgSocTime*\ (urgency) ensures that \ValTime*\ must be valued such that no time passes while a message is able to be received.
   %
   % It holds that $t$ must equal 0 when there is a non-empty queue.
   
   Therefore the hypothesis holds.
   %if a system makes a $t$ transition where $t>0$ then all queues in the system must be empty.
   % Therefore, \ValTime*\ must equal $0$ when there is a message in any queue in a system composed of compatible configurations.
   %
 \end{proof}
 % 

%
% ! (lemma 15) : compat, single transition -> compat
\begin{lemma}\label{lem:configs_trans_compat_pres}
	%
	If \VIso*_1\ and \VIso*_2\ are both \emph{well-formed} 
    and \Compat*[\VSoc_1][\VSoc_2]\ 
    and \Trans*{\Parl{\VSoc_1,\VSoc_2}}[\Parl{\VSoc'_1,\VSoc'_2}], 
	then \Compat*[\VSoc'_1][\VSoc'_2].
	%
\end{lemma}
\begin{proof}
	%
	We proceed by induction on the depth of the derivation tree, analysing each case of the last rule applied for the transition \Trans*{\Parl{\VSoc_1,\VSoc_2}}[\Parl{\VSoc'_1,\VSoc'_2}]:
	% \begin{inline}+
	% 	\item \LblCfgSysWait*
	% 	\item \LblCfgSysLComm*
	% 	\item \LblCfgSysLPar*
	% \end{inline}
	%
	\begin{inductivecase}
		%
		%
		%
		%
		%
		% ~ wait
		\item\NewCase[\LblCfgSysWait*]\label{case:configs_trans_compat_pres_wait}
		Then both \VSoc*_1\ and \VSoc*_2\ make a \ValTime*\ transition via \LblCfgSocTime*\ as shown in~\Cref{eq:configs_trans_compat_pres_wait_trans}.
		%
        If $\ValTime=0$ then by~\Cref{lem:configs_iso_trans,lem:configs_soc_trans} $\VSoc_1=\VSoc'_1$ and $\VSoc_2=\VSoc'_2$ and the hypothesis holds; \Compat*[\VSoc'_1][\VSoc'_2].
        %
        \begin{minieq}\label{eq:configs_trans_compat_pres_wait_trans}
            \infer[\LblCfgSysWait]{%
                \Trans{\Parl{\VSoc_1,\VSoc_2}}:{\ValTime}[\Parl{\VSoc'_1,\VSoc'_2}]
            }{%
                \infer[\LblCfgSocTime]{%
                    \Trans{\VSoc_1}:{\ValTime}[\VSoc'_1]
                }{\dots}
                & %
                \infer[\LblCfgSocTime]{%
                    \Trans{\VSoc_2}:{\ValTime}[\VSoc'_2]
                }{\dots}
            }
        \end{minieq}
        
        \noindent If $\ValTime>0$ then by~\Cref{lem:configs_iso_trans,lem:configs_soc_trans} $\ValClocks'_1=\ValClocks_1+\ValTime$ and $\TypeS'_1=\TypeS_1$ and $\Queue'_1=\Queue_1$ (and the same for \ValClocks*'_2, \TypeS*'_2\ and \Queue*'_2).
		%
        By~\Cref{lem:cfgs_trans_wf_pres} \VIso*'_1\ and \VIso*'_2\ are both \emph{well-formed}.
		%
		By~\Cref{lem:sys_compat_time_trans} $\Queue_1=\emptyset=\Queue_2$ and by~\Cref{itm:configs_compat_dual_types} of~\Cref{def:configs_compat} $\ValClocks_1=\ValClocks_2$ and $\TypeS_1=\Dual[\TypeS_2]$.
		%
		Therefore \Compat*[\CSoc[\ValClocks_1]+{+\ValTime};[\TypeS_1]:{\emptyset}][\CSoc[\ValClocks_2]+{+\ValTime};[\Dual[\TypeS_2]]:{\emptyset}].
		%
		%
		%
		%
		%
		% ~ comm
		\item\NewCase[\LblCfgSysLComm*]\label{case:configs_trans_compat_pres_comm}
		By~\cref{lem:cfgs_trans_wf_pres} both \VIso*'_1\ and \VIso*'_2\ are \emph{well-formed}.
        The transition is as shown below: %in~\cref{eq:configs_trans_compat_pres_comm_trans}.
		%
		\begin{minieq}*\label{eq:configs_trans_compat_pres_comm_trans}
			% \begin{array}{c}%\mathllap{%
			\resizebox{\linewidth}{!}{$%
				\infer[\LblCfgSysLComm]{%
					\Trans{\Parl{\CSoc[\ValClocks_1];[\TypeS_1]:{\emptyset},\CSoc[\ValClocks_2];[\TypeS_2]:{\Queue_2}}}:{\SiltAction}[\Parl{\CSoc[\ValClocks'_1];[\TypeS'_1]:{\emptyset},\CSoc[\ValClocks_2];[\TypeS_2]:{\Queue_2;\Msg}}]
				}{%
					\infer[\LblCfgSocSend]{%
						\Trans{\CSoc[\ValClocks_1];[\TypeS_1]:{\emptyset}}:{\SendMsg}[\CSoc[\ValClocks_1]+{\ReSet[]_j};[\TypeS_j]:{\emptyset}]
					}{%
						\infer[\LblCfgIsoInteract]{%
							\Trans{\CIso[\ValClocks_1];[\TypInteract]}:{\SendMsg}[\CIso[\ValClocks_1]+{\ReSet[]_j};[\TypeS_j]]
						}{%
                        \ValClocks_1\models\Const_j
                        & %
                        {m}={l_j\left\langle T_j \right\rangle}
                        & % 
                        {\TypSend=\TypComm_j}
                        & % 
                        j\in I
						}
					}
					& %
					% \infer[\LblCfgSocEnqu]{%
						\Trans{\VSoc_2}:{\RecvMsg}[\CSoc[\ValClocks_2];[\TypeS_2]:{\Queue_2;\Msg}]
						\quad \LblCfgSocEnqu
					% }{\dots}
				}
			$}%
	%	}\end{array}
		\end{minieq}
		
        \noindent We proceed by inner induction on each combination of the contents of queues:
		\begin{inductivecase}
			%
			%
			% ~ comm -> e e
			\item\NewCase[$\Queue_1=\emptyset$, $\Queue_2=\emptyset$]\label{case:configs_trans_compat_pres_comm_ee}
			By~\Cref{itm:configs_compat_dual_types} of~\Cref{def:configs_compat} $\ValClocks_1=\ValClocks_2$ and $\TypeS_1=\Dual[\TypeS_2]$.
			%
			The resulting system is no longer \emph{dual}.
			%
			By~\cref{lem:cfgs_trans_wf_pres,lem:sys_compat_time_trans} time cannot pass if $\Queue_2\neq\emptyset$.
			%
			By~\Cref{def:types_dual} the message \Msg*\ sent by \VSoc*_1\ must have a corresponding receiving action in \VSoc*_2\ as in~\Cref{itm:configs_compat_expected_receive} of~\Cref{def:configs_compat}.
			%
			Therefore \Compat*[\CSoc[\ValClocks'_1];[{\TypeS}'_1]:{\emptyset}][\CSoc;+{\Msg}_2].
			%
			%
			%
			% ~ comm -> e m
			\item\NewCase[$\Queue_1=\emptyset$, $\Queue_2\neq\emptyset$]\label{case:configs_trans_compat_pres_comm_en}
			By~\Cref{itm:configs_compat_expected_receive} of~\Cref{def:configs_compat} $\exists\Msg',{\ValClocks}'',{\TypeS}''$ such that $\Queue_2=\Msg';\Queue_2$ and \Trans*{\CIso_2}:{\TypRecv,\Msg'}[\CIso[{\ValClocks}''_2];[{\TypeS}''_2]]\ and \Compat*[\CSoc[\ValClocks'_1];[{\TypeS}'_1]:{\emptyset}][\CSoc[{\ValClocks}''_2];[{\TypeS}''_2]:{{\Queue}_2}]\ and by \LblCfgSocTime*\ (urgency) time cannot pass.
			%
			If a system has a configuration with sequence of outgoing sending actions and each has constraints that are satisfiable immediately after the other, then the system can both receive the messages as they arrive, or accumulate the messages and instantly receive each in succession and become \emph{dual} again (by inspection of~\Cref{def:types_dual,def:configs_compat} and~\Cref{fig:types_rule,fig:typesemantics_tuple,fig:typesemantics_triple}).
			%
			Therefore \Compat*[\CSoc[\ValClocks'_1];[{\TypeS}'_1]:{\emptyset}][\CSoc[{\ValClocks}''_2];[{\TypeS}''_2]:{\Msg';{\Queue}_2;\Msg}].
			%
			%
			% ~ comm -> m e
			\item\NewCase[$\Queue_1\neq\emptyset$, $\Queue_2=\emptyset$]\label{case:configs_trans_compat_pres_comm_ne}
			Contradicts the hypothesis by~\Cref{itm:configs_compat_expected_receive} of~\Cref{def:configs_compat} as by \LblCfgSocTime*\ (urgency) messages must be removed from a queue immediately, and by \LblTypChoice*\ of~\Cref{fig:types_rule} sending and receiving actions cannot be performed at the same time.
			%
			%
			% ~ comm -> m m
			\item\NewCase[$\Queue_1\neq\emptyset$, $\Queue_2\neq\emptyset$]\label{case:configs_trans_compat_pres_comm_nn}
			Contradicts the hypothesis by~\Cref{itm:configs_compat_non_empty_queues} of~\Cref{def:configs_compat}.
			%
		\end{inductivecase}

		\noindent Therefore, compatibility is preserved across \LblCfgSysLComm*\ transitions.
		%
		%
		%
		%
		%
		% ~ dequ
		\item\NewCase[\LblCfgSysLPar*]\label{case:configs_trans_compat_pres_dequ}
		By~\Cref{lem:configs_soc_trans} $\Queue_2=\Msg;\Queue_2$ and by~\Cref{itm:configs_compat_expected_receive} of~\Cref{def:configs_compat} \Compat*[\VSoc'_1][\VSoc'_2], the hypothesis holds.
		%
	\end{inductivecase}

	\noindent Therefore, it holds that any transition made by a compatible system composed of well-formed types will result in configurations that are \emph{compatible}.
	%
\end{proof}
% 

% ~ preservation is preserved 
% ! 
%
% ! (lemma 16) : compat wf, end or future enabled
\begin{lemma}\label{lem:configs_compat_wf_fe}
	%
	If both \VIso*_1\ and \VIso*_2\ are \emph{well-formed} 
    and \Compat*[\VSoc_1][\VSoc_2],
	then both \VSoc*_1\ and \VSoc*_2\ are \emph{final} 
	or $\exists\ValTime$ such that \Trans*{\VSys}:{\ValTime,\SiltAction}[\Parl{\VSoc'_1,\VSoc'_2}].%
	%
\end{lemma}
\begin{proof}
	%
	By~\Cref{lem:cfgs_trans_wf_pres} \VIso*'_1\ and \VIso*'_2\ are \emph{well-formed} and by~\Cref{lem:cfg_wf_then_live} are \emph{live}.
	%
	By~\Cref{def:types_progress} if \VSoc*_2\ is \emph{final} then $\VSoc_2=\CSoc[\ValClocks_2];[\TypeEnd]:{\emptyset}$.
    %
    We proceed with the assumption that at least one participant is \emph{not final}, and hereafter only consider \VSoc*'_1.
	%
	The transition is given below:
 %in~\Cref{eq:configs_compat_wf_fe_trans}.
	%
	\begin{minieq}*\label{eq:configs_compat_wf_fe_trans}
		\Trans{\VSys}:{\ValTime}[\Trans{\Parl{\CSoc+{+\ValTime}_1,\CSoc+{+\ValTime}_2}}:{\SiltAction}[\Parl{\CSoc'_1,\CSoc'_2}]]
	\end{minieq}
	%
	We proceed only considering each case of \VSoc*_1\ not being \emph{final}.
	By induction on the cases of \TypeS*_1:
	%
	\begin{inductivecase}
		%
		%
		% ~ choice
		\item\NewCase[$\TypeS=\simplechoice$] 
		% Then by judgement of \LblTypChoice*\ $\exists\Const_i$ such that $\ValClocks\models\Past[\Const_i]$ and $\emptyset;\Past[\Const_i]~\Entails\simplechoice$.
		%
		As described in~\Cref{sec:types} we write $\Past$ if $\exists\ValTime$ such that $\ValClocks+\ValTime\models\Const$.
		Therefore, if $\ValClocks\not\models\Const$ and $\emptyset;\Const~\Entails\TypeS$ for a \emph{well-formed} \CIso*\ then $\ValClocks\models\Past[\Const]$.
		%
		By~\Cref{lem:cfgs_trans_wf_pres} the only possible values of \ValTime*\ ensure that the latest system-wide sending action is never missed and messages are received as soon as they arrive in a queue by rule \LblCfgSocTime*.
		
		Therefore, the hypothesis holds for systems composed where one participant is known to be a \emph{non-final} choice type.
		%
		%
		% ~ rec def
		\item\NewCase[$\TypeS=\mu\alpha.\TypeS'$]
		It follows~\Cref{lem:cfgs_trans_wf_pres} that \TypeS*'\ is \emph{well-formed} against $\ValClocks+\ValTime$.
		%
	\end{inductivecase}
	
	\noindent Therefore, if a \emph{well-formed} and compatible system \Parl*{\VSoc_1,\VSoc_2}\ that is not \emph{final}, then there is some value of time $\ValTime\geq 0$ that will enable a future action, which will result in a \emph{well-formed} and compatible system \Parl*{\VSoc'_1,\VSoc'_2}, which may or may not be \emph{final}, and to which this behaviour still applies.
	%
\end{proof}
% 
%
% ! (lemma 18) : compat wf, any amount of transitions -> compat wf
\begin{lemma}\label{lem:configs_trans_compat_wf_pres}
    %
    If both \VIso*_1 and \VIso*_2 are \emph{well-formed}
    and \Compat*[\VSoc_1][\VSoc_2]\ 
    and $\Trans{\Parl{\VSoc_1\!,\VSoc_2}}*[\Parl{\VSoc'_1\!,\VSoc'_2}]$, 
    then \Compat*[\VSoc'_1][\VSoc'_2]\ 
    and both \VIso*'_1\ and \VIso*'_2\ are \emph{well-formed}.%
    %
\end{lemma}
\begin{proof}
    %
    By~\Cref{lem:cfgs_trans_wf_pres,lem:configs_trans_compat_pres} the hypothesis holds for single transitions and that the resulting configurations are \emph{live}, and either \emph{final} or \emph{satisfies progress} by~\Cref{lem:cfg_wf_then_live,lem:configs_compat_wf_fe}.
    %
    Therefore it holds that \Compat*[\VSoc'_1][\VSoc'_2]\ 
    and both \VIso*'_1\ and \VIso*'_2\ are \emph{well-formed} across an arbitrary number of transitions, as each single transition preserves compatibility and well-formedness.
    %
\end{proof}
% 

% ~ system progress
% ! 
%
% ! (lemma 20) : sys cfg progress
\begin{lemma}\label{lem:configs_sys_progress}
    %
    For all \TypeS*, \ValClocks* such that \ITJudgement*[\emptyset];{\Const}[\TypeS]\ and \Sat*\ :\par\noindent
    \hfill\ \Parl*{\CSoc:{\emptyset},\CSoc;[\Dual]:{\emptyset}}\ satisfies progress.\hfill\ \ %
    %
\end{lemma}
\begin{proof}
    %
    By the hypothesis the system is compatible and composed of well-formed dual types. 
    %
    By~\Cref{lem:configs_trans_compat_wf_pres} any configurations reachable by such a system will be compatible and \emph{well-formed}.
    %
    By~\Cref{lem:configs_compat_wf_fe} such a system adheres to~\Cref{def:types_progress} and \emph{satisfies progress}.
    %
\end{proof}
% 

% ! thm proof
\ThmProgress*
\begin{proof}
   %
   By~\Cref{def:types_wf}, types \TypeS*\ and \Dual*\ are always \emph{well-formed} against \ValClocks*_0.
   %
   By \Cref{lem:init_wf_then_live} and~\Cref{itm:configs_compat_dual_types} of~\Cref{def:configs_compat}, both \CSoc*[\ValClocks_0]:{\emptyset}\ and \CSoc*[\ValClocks_0];[\Dual]:{\emptyset} are \emph{live} and \emph{compatible}.
   %
   It follows~\Cref{lem:configs_compat_wf_fe,lem:configs_trans_compat_wf_pres} that such as system will always perform actions when possible, waiting if necessary (never missing the \emph{latest-enabled} action), until reaching a \emph{final} configuration, and any \emph{non-final} configuration is guaranteed to be \emph{well-formed}, \emph{compatible} and \emph{live}.

   Therefore, it holds that an initial system composed of dual types that are well-formed is compatible, and guaranteed to \emph{satisfy progress}.
   %
\end{proof}
