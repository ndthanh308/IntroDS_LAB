%
% ! (lemma 17) : compat wf, single transition -> wf
\begin{lemma}\label{lem:cfgs_trans_wf_pres}
    %
    If \VIso*_1\ and \VIso*_2\ are both \emph{well-formed} 
    and \Compat*[\VSoc_1][\VSoc_2]\ 
    and \Trans*{\Parl{\VSoc_1,\VSoc_2}}[\Parl{\VSoc'_1,\VSoc'_2}], 
    then both \VIso*'_1\ and \VIso*'_2\ are \emph{well-formed}.
    %
\end{lemma}
\begin{proof}
    %
    We proceed by induction on the depth of the derivation tree, analysing each case of the last rule applied for the transition \Trans*{\Parl{\VSoc_1,\VSoc_2}}[\Parl{\VSoc'_1,\VSoc'_2}].
    % \begin{inline}+
    %     \item \LblCfgSysWait*
    %     \item \LblCfgSysLComm*
    %     \item \LblCfgSysLPar*
    % \end{inline}
    %
    \begin{inductivecase}
        %
        %
        %
        %
        %
        % ~ wait
        \item\NewCase[\LblCfgSysWait*]\label{case:cfgs_trans_wf_pres_wait} 
        As shown in~\Cref{eq:cfgs_trans_wf_pres_wait_trans}, both \VSoc*_1\ and \VSoc*_2\ make a rule \LblCfgSocTime*\ transition with the same valuation of \ValTime*.
        If $\ValTime=0$ then by~\Cref{lem:configs_iso_trans,lem:configs_soc_trans} $\VIso_1=\VIso'_1$ and $\VIso_2=\VIso'_2$ and both \VIso*'_1\ and \VIso*'_2\ are \emph{well-formed}.
        %
        \begin{minieq}\label{eq:cfgs_trans_wf_pres_wait_trans}
        \begin{array}[c]{l}
            \infer[\LblCfgSysWait]{%
                \Trans{\Parl{\VSoc_1,\VSoc_2}}:{\ValTime}[\Parl{\VSoc'_1,\VSoc'_2}]
            }{%
                \infer[\LblCfgSocTime]{%
                    \Trans{\VSoc_1}:{\ValTime}[\VSoc'_1]
                }{\dots}
                & %
                \infer[\LblCfgSocTime]{%
                    \Trans{\VSoc_2}:{\ValTime}[\VSoc'_2]
                }{\dots}
            }
            \end{array}
        \end{minieq}
            
        \noindent We proceed with only \VSoc*_1\ as the analysis covers \VSoc*_2.
        If $\ValTime>0$ then by~\Cref{lem:configs_iso_trans,lem:configs_soc_trans} $\ValClocks'_1=\ValClocks_1+\ValTime$ and $\TypeS'_1=\TypeS_1$.
        %
        By induction on the depth of the derivation tree, analysing the last rule applied for the transition \Trans*{\VIso_1}:{\ValTime}[\VIso'_1]:
        \begin{inductivecase}
            %
            %
            % ~ wait -> tick
            \item\NewCase[\LblCfgIsoTick*]\label{case:cfgs_trans_wf_pres_wait_tgtz_tick} 
            Then \Trans*{\CIso_1}:{\ValTime}[\CIso+{+\ValTime}_1].
            We proceed by inner induction on the different cases of \TypeS*_1:
            \begin{inductivecase}
                %
                % ~ wait -> tick -> choice
                \item\NewCase[$\TypeS_1=\simplechoice$]\label{case:cfgs_trans_wf_pres_wait_tgtz_tick_choice} 
                By rule \LblCfgIsoInteract*\ $\exists\Const_i$ such that $\ValClocks_1\models\Past[\Const_i]$ and $\emptyset;\Past[\Const_i]~\Entails\FullChoice$, as in~\Cref{itm:cfg_wf_then_live_choice} of~\Cref{lem:cfg_wf_then_live}.
                %
                By induction hypothesis $\exists\Const'_i$ such that $\ValClocks_1+\ValTime\models\Past[\Const'_i]$ and $\emptyset;\Past[\Const'_i]~\Entails\FullChoice$, which is assured by the (persistency) premise of rule \LblCfgSocTime*.
                %
                Therefore, it holds that \CIso*[\ValClocks_1]+{+\ValTime};[\simplechoice]\ is \emph{well-formed}.
                % By~\Cref{def:configs_fe} \CIso*_1\ is \emph{future-enabled} and by rule \LblCfgSocTime*\ $\text{(persistency)}$ it holds that \CIso*+{+\ValTime}_1\ is also \emph{future-enabled}.
                % %
                % Therefore the hypothesis holds, \CIso*[\ValClocks_1]+{+\ValTime};[\simplechoice]\ is \emph{well-formed}; by~\Cref{def:types_wf} and rule \LblTypChoice*\ $\exists\Const'_i$ such that $\ValClocks_1+\ValTime\models\Past[\Const'_i]$ and $\emptyset;\Past[\Const'_i]~\Entails\simplechoice$.
                %
                %
                % ~ wait -> tick -> recursion
                \item\NewCase[$\TypeS_1=\mu\alpha.{\TypeS}''_1$]\label{case:cfgs_trans_wf_pres_wait_tgtz_tick_recursion} 
                By~\Cref{def:types_wf} $\exists\Const$ such that $\ValClocks_1\models\Const$ and $\emptyset;\Const~\Entails\mu\alpha.{\TypeS}''_1$, and (by rule \LblTypRec*) $\alpha:\Const;\Const~\Entails{\TypeS}''_1$ and \CIso*[\ValClocks_1];[{\TypeS}''_1]\ is \emph{well-formed}.
                %
                Therefore, the well-formedness of \CIso*[\ValClocks_1]+{+\ValTime};[\mu\alpha.{\TypeS}''_1]\ is dependant on the well-formedness of \CIso*[\ValClocks_1]+{+\ValTime};[{\TypeS}''_1]. (See other cases, as in~\Cref{itm:wf_then_live_recdef} of~\Cref{lem:cfg_wf_then_live}.)
                %
                %
                % ~ wait -> tick -> end
                \item\NewCase[$\TypeS_1=\TypeEnd$]\label{case:cfgs_trans_wf_pres_wait_tgtz_tick_end} 
                By~\Cref{lem:cfg_wf_end} \CIso*+{+\ValTime};[\TypeEnd]\ is \emph{well-formed}.
                %
                %
                % ~ wait -> tick -> rec call
                \item\label{case:cfgs_trans_wf_pres_wait_tgtz_tick_reccal} ${\TypeS}_1$ cannot equal $\alpha$ by~\Cref{lem:cfg_wf_neq_alpha}.
                %
            \end{inductivecase}
            %
            %
            % ~ wait -> unfold
            \item\NewCase[\LblCfgIsoUnfold*]\label{case:cfgs_trans_wf_pres_wait_tgtz_unfold} 
            Then $\TypeS_1=\mu\alpha.{\TypeS}''_1$ and by the hypothesis $\exists\Const$ such that $\ValClocks_1\models\Const$ and $\emptyset;\Const~\Entails\TypRecDef$, and by rule \LblTypRec*\ $\alpha:\Const;\Const~\Entails{\TypeS}''_1$.
            The transition is as shown below:
            %
            \begin{minieq}*%\label{eq:cfgs_trans_wf_pres_wait_tgtz_unfold_trans}
        \begin{array}[c]{l}
                \infer[\LblCfgIsoUnfold]{%
                    \Trans{\CIso[\ValClocks_1];[\mu\alpha.{\TypeS}''_1]}:{\ProgAction}[\CIso'_1]
                }{%
                    \Trans{\CIso[\ValClocks_1];[{\TypeS}''_1\Subst[\mu\alpha.{\TypeS}''_1][\alpha]]}:{\ValTime}[\CIso'_1]
                }
                \end{array}
            \end{minieq}
            
            \noindent By inner induction on the different cases of ${\TypeS}''_1$:
            \begin{inductivecase}
                %
                % ~ wait -> unfold -> choice
                \item\NewCase[${\TypeS}''_1=\simplechoice$]\label{case:cfgs_trans_wf_pres_wait_tgtz_unfold_choice} 
                Then, by rule \LblCfgIsoTick*:
                \[\Trans{\CIso[\ValClocks_1];[{\simplechoice}\Subst[\mu\alpha.{\simplechoice}][\alpha]]}:{\ValTime}[\CIso[\ValClocks_1]+{+\ValTime};[\simplechoice]]\]
                
                \noindent By the rules \LblTypRec*\ and \LblTypChoice*\ the following holds:
                \[
                \infer[\LblTypRec]{%
                \emptyset;\Const_i~\Entails\mu\alpha.{\FullChoice}
                }{%
                \infer[\LblTypChoice]{%
                    \alpha:\Const_i;\Past_i~\Entails\FullChoice
                }{%
                \dots
                }
                }
                \]
                
                % \noindent It holds that \CIso*[\ValClocks_1];[{\simplechoice}\Subst[\mu\alpha.{\simplechoice}][\alpha]]\ is \emph{well-formed} and \emph{future-enabled}.
                %
                \noindent By induction hypothesis: \[\exists\Const'_i:\ValClocks_1+\ValTime\models\Past[\Const'_i] ~\land~ \emptyset;\Past[\Const'_i]~\Entails\FullChoice\] which is assured by the (persistency) premise of rule \LblCfgSocTime*, as in~\Cref{case:cfgs_trans_wf_pres_wait_tgtz_tick_choice} of~\Cref{lem:cfgs_trans_wf_pres}.
% 
                Therefore, \CIso*[\ValClocks_1]+{+\ValTime};[\mu\alpha.{\simplechoice}]\ is \emph{well-formed} as \CIso*[\ValClocks_1]+{+\ValTime};[{\simplechoice}\Subst[\mu\alpha.{\simplechoice}][\alpha]]\ the following is \emph{well-formed}.
                %
                %
                % ~ wait -> unfold -> recursion
                \item\NewCase[${\TypeS}''_1=\mu\alpha'.{\TypeS}'''_1$]\label{case:cfgs_trans_wf_pres_wait_tgtz_unfold_recursion} 
                % By the hypothesis $\exists\Const$ such that $\ValClocks_1\models\Const$ and $\emptyset;\Const~\Entails\mu\alpha.\mu\alpha'.{\TypeS}'''_1$, by the premise of rule \LblTypRec*\ $\alpha:\Const;\Const~\Entails\mu\alpha'.{\TypeS}'''_1$ and $\alpha:\Const,\alpha':\Const;\Const~\Entails{\TypeS}'''_1$, and \CIso*[\ValClocks_1];[{\TypeS}'''_1]\ is \emph{well-formed}.
                % 
                The well-formedness of \CIso*[\ValClocks_1]+{+\ValTime};[\mu\alpha'.{\TypeS}'''_1]\ depends on the well-formedness of \CIso*[\ValClocks_1]+{+\ValTime};[{\TypeS}'''_1]. (See other cases of $S$.) %, as in~\Cref{itm:wf_then_live_recdef} of~\Cref{lem:cfg_wf_then_live}.)
                %
                %
                % ~ wait -> unfold -> end
                \item\NewCase[${\TypeS}''_1=\TypeEnd$]\label{case:cfgs_trans_wf_pres_wait_tgtz_unfold_end} 
                By~\Cref{lem:cfg_wf_end} \CIso*+{+\ValTime};[\TypeEnd]\ is \emph{well-formed}.
                %
                %
                % ~ wait -> unfold -> rec call
                \item\label{case:cfgs_trans_wf_pres_wait_tgtz_unfold_reccall} ${\TypeS}''_1$ cannot equal $\alpha$ by~\Cref{lem:cfg_wf_neq_alpha}.
                %
            \end{inductivecase}
            %
        \end{inductivecase}

        \noindent Therefore, it holds that well-formedness is preserved by transition made by \emph{well-formed} configurations via rule \LblCfgSysWait*.
        %
        %
        %
        %
        %
        % ~ comm
        \item\NewCase[\LblCfgSysLComm*]\label{case:cfgs_trans_wf_pres_comm} 
        % Then by~\cref{case:configs_trans_compat_pres_comm} of~\cref{lem:configs_trans_compat_pres} $\Queue_1=\Queue'_1=\emptyset$.
        The transition is as shown below:
        %
        \begin{minieq}*%\label{eq:cfgs_trans_wf_pres_comm_trans}
        \begin{array}[c]{l}
            \infer[\LblCfgSysLComm]{%
                \Trans{\Parl{\VSoc_1,\VSoc_2}}:{\SiltAction}[\Parl{\VSoc'_1,\VSoc'_2}]
            }{%
                \infer[\LblCfgSocSend]{%
                    \Trans{\VSoc_1}:{\SendMsg}[\VSoc'_1]
                }{\dots}
                & %
                % \infer[\LblCfgSocEnqu]{%
                    \Trans{\VSoc_2}:{\RecvMsg}[\VSoc'_2]
					\quad \LblCfgSocEnqu
                % }{\dots}
            }
            \end{array}
        \end{minieq}
        
        % ~ comm-l s1
        \noindent Focusing first on \VSoc*_1, we proceed by induction on the depth of the derivation tree, analysing the last rule applied for the transition \Trans*{\VIso_1}:{\SendMsg}[\VIso'_1]:
        \begin{inductivecase}
            %
            %
            % ~ comm-l s1 -> act
            \item\NewCase[\LblCfgIsoInteract*]\label{case:cfgs_trans_wf_pres_comm_act} 
            Then $\TypeS_1=\simplechoice$, and the evaluation is shown below:
            %
            \begin{minieq}*\label{eq:cfgs_trans_wf_pres_comm_act_trans}
        \begin{array}[c]{l}
                \infer[\LblCfgSocSend]{%
                    \Trans{\CSoc[\ValClocks_1];[\simplechoice]:{\Queue_1}}:{\SendMsg}[\CSoc[\ValClocks_1]+{\ReSet[]_j};[\TypeS_j]:{\Queue_1}]
                }{%
                    \infer[\LblCfgIsoInteract]{%
                        \Trans{\CIso[\ValClocks_1];[\TypInteract]}:{\SendMsg}[\CIso[\ValClocks_1]+{\ReSet[]_j};[\TypeS_j]]
                    }{%
                        \ValClocks_1\models\Const_j
                        & %
                        {m}={l_j\left\langle T_j \right\rangle}
                        & % 
                        {\TypSend=\TypComm_j}
                        & % 
                        j\in I
                    }
                }
                \end{array}
            \end{minieq}
            
            \noindent By rule \LblTypChoice*\ $\exists\Const_i:\ValClocks_1\models\Past[\Const_i]$ and $\emptyset;\Past[\Const_i]~\Entails\FullChoice$, and by the premise of rule \LblTypChoice*\ it holds that $\Const_i\ReSet[]_i\subseteq\gamma$ and $\emptyset;\gamma~\Entails\TypeS_i$.
            %
            Combined with~\Cref{lem:configs_iso_trans} it holds that $\ValClocks_1\models\Const_j$ and $\emptyset;\Const_j\ReSet[]_j~\Entails\TypeS_j$ and $\ValClocks_1\ReSet[]_j\models\Const_j\ReSet[]_j$.
            
            Therefore, \CIso*[\ValClocks_1]+{\ReSet[]_j};[\TypeS_j]\ is \emph{well-formed}.
            %
            %
            % ~ comm-l s1 -> unfold
            \item\NewCase[\LblCfgIsoUnfold*]\label{case:cfgs_trans_wf_pres_comm_unfold} 
            Then $\TypeS_1=\mu\alpha.{\TypeS}''_1$.
            The transition is as shown below:
            %
            \begin{minieq}*\label{eq:cfgs_trans_wf_pres_comm_unfold_trans}
        \begin{array}[c]{l}
                \infer[\LblCfgIsoUnfold]{%
                    \Trans{\CIso[\ValClocks_1];[\mu\alpha.{\TypeS}''_1]}:{\ProgAction}[\CIso'_1]
                }{%
                    \Trans{\CIso[\ValClocks_1];[{\TypeS}''_1\Subst[\mu\alpha.{\TypeS}''_1][\alpha]]}:{\SendMsg}[\CIso'_1]
                }
                \end{array}
            \end{minieq}
            
            \noindent By the hypothesis $\exists\Const$ such that $\ValClocks_1\models\Const$ and $\emptyset;\Const~\Entails\TypRecDef$, and by the premise of rule \LblTypRec*\ $\alpha:\Const;\Const~\Entails{\TypeS}''_1$, and \CIso*[\ValClocks_1];[{\TypeS}''_1\Subst[\mu\alpha.{\TypeS}''_1][\alpha]]\ is \emph{well-formed}.
            %
            The well-formendess of \CIso*'_1\ is dependant on the state of ${\TypeS}''_1$, which for the transition \Trans*{\CIso[\ValClocks_1];[{\TypeS}''_1\Subst[\mu\alpha.{\TypeS}''_1][\alpha]]}:{\SendMsg}[\CIso'_1]\ must be either $\simplechoice$ or $\mu\alpha'.{\TypeS}'''_1$ (see other cases, as in~\Cref{lem:cfg_wf_then_live}).
            %
        \end{inductivecase}
        %
        % ~ comm-l s1
        Now, focusing on \VSoc*_2, the transition \Trans*{\CSoc_2}:{\RecvMsg}[\CSoc'_2]\ via rule \LblCfgSocEnqu*\ yields $\ValClocks_2'=\ValClocks_2$ and $\TypeS_2'=\TypeS_2$ and $\Queue_2'=\Queue_2;\Msg$ by~\Cref{lem:configs_soc_trans}.
        %
        Therefore \CIso*'_2\ is \emph{well-formed} as $\VIso_2=\VIso'_2$.
        %
        Transitions via rule \LblCfgSysRComm*\ are symmetric and omitted.
        %
        %
        %
        %
        %
        % ~ par-l
        \item\NewCase[\LblCfgSysLPar*]\label{case:cfgs_trans_wf_pres_par} 
        By~\Cref{lem:configs_soc_trans} $\Queue'_1=\Msg;\Queue_1$.
        The transition is as shown below: %in~\Cref{eq:cfgs_trans_wf_pres_par_trans}.
        %
        \begin{minieq}*\label{eq:cfgs_trans_wf_pres_par_trans}
        \begin{array}[c]{l}
            \infer[\LblCfgSysLPar]{%
                \Trans{\Parl{\VSoc_1,\VSoc_2}}:{\SiltAction}[\Parl{\VSoc'_1,\VSoc'_2}]
            }{%
                \infer[\LblCfgSocRecv]{%
                    \Trans{\VSoc_1}:{\SiltAction}[\VSoc'_1]
                }{\dots}
            }
            \end{array}
        \end{minieq}
        
        \noindent We proceed by induction on the depth of the derivation tree, analysing the last rule applied for the transition \Trans*{\VIso_1}:{\RecvMsg}[\VIso'_1]\ via the premise of rule \LblCfgSocRecv*:
        \begin{inductivecase}
            %
            %
            % ~ par-l -> act
            \item\NewCase[\LblCfgIsoInteract*]\label{case:cfgs_trans_wf_pres_par_act} 
            Then $\TypeS_1=\simplechoice$.
            The transition is as shown below: %in~\Cref{eq:cfgs_trans_wf_pres_par_act_trans}.
            %
            \begin{minieq}*\label{eq:cfgs_trans_wf_pres_par_act_trans}
        \begin{array}[c]{l}
                \infer[\LblCfgSocRecv]{%
                    \Trans{\CSoc[\ValClocks_1];[\simplechoice]:{\Msg;\Queue_1}}:{\SiltAction}[\CSoc[\ValClocks_1]+{\ReSet[]_j};[\TypeS_j]:{\Queue_1}]
                }{%
                    \infer[\LblCfgIsoInteract]{%
                        \Trans{\CIso[\ValClocks_1];[\TypInteract]}:{\RecvMsg}[\CIso[\ValClocks_1]+{\ReSet[]_j};[\TypeS_j]]
                    }{%
                        \ValClocks_1\models\Const_j
                        & %
                        {m}={l_j\left\langle T_j \right\rangle}
                        & % 
                        {\TypRecv=\TypComm_j}
                        & % 
                        j\in I
                    }
                }
                \end{array}
            \end{minieq}
            
            \noindent By the hypothesis and the judgement of rule \LblTypChoice*\ $\exists\Const_i$ such that $\emptyset;\Past[\Const_i]~\Entails\simplechoice$ and $\ValClocks_1\models\Past[\Const_i]$, and by the premise of rule \LblTypChoice*\ $\Const_i\ReSet[]_i\subseteq\gamma$ and $\emptyset;\gamma~\Entails\TypeS_i$.
            
            It follows~\Cref{case:cfgs_trans_wf_pres_comm_act} of~\Cref{lem:cfgs_trans_wf_pres} that \CIso*[\ValClocks_1]+{\ReSet[]_j};[\TypeS_j]\ is \emph{well-formed}.
            % Therefore \CIso*[\ValClocks_1]+{\ReSet[]_j};[\TypeS_j]\ is \emph{well-formed} as $\ValClocks_1\models\Const_j$ and $\emptyset;\Const_j\ReSet[]_j~\Entails\TypeS_j$ and $\ValClocks_1\ReSet[]_j\models\Const_j\ReSet[]_j$ (as in~\Cref{case:cfgs_trans_wf_pres_comm_act} of~\Cref{lem:cfgs_trans_wf_pres}).
            %
            %
            % ~ par-l -> unfold
            \item\NewCase[\LblCfgIsoUnfold*]\label{case:cfgs_trans_wf_pres_par_unfold} 
            Then $\TypeS_1=\mu\alpha.{\TypeS}''_1$.
            The transition shown below, and is analogous to the one in~\Cref{eq:cfgs_trans_wf_pres_comm_unfold_trans} of~\Cref{lem:cfgs_trans_wf_pres}:
            % in~\Cref{eq:cfgs_trans_wf_pres_par_unfold_trans} and is analogous to the one in~\Cref{eq:cfgs_trans_wf_pres_comm_unfold_trans}.
            %
            \begin{minieq}\label{eq:cfgs_trans_wf_pres_par_unfold_trans}
        \begin{array}[c]{l}
                \infer[\LblCfgIsoUnfold]{%
                    \Trans{\CIso[\ValClocks_1];[\mu\alpha.{\TypeS}''_1]}:{\ProgAction}[\CIso'_1]
                }{%
                    \Trans{\CIso[\ValClocks_1];[{\TypeS}''_1\Subst[\mu\alpha.{\TypeS}''_1][\alpha]]}:{\RecvMsg}[\CIso'_1]
                }
                \end{array}
            \end{minieq}
            
            \noindent By the hypothesis $\exists\Const$ such that $\ValClocks_1\models\Const$ and $\emptyset;\Const~\Entails\TypRecDef$, and by the premise of rule \LblTypRec*\ $\alpha:\Const;\Const~\Entails{\TypeS}''_1$, and \CIso*[\ValClocks_1];[{\TypeS}''_1\Subst[\mu\alpha.{\TypeS}''_1][\alpha]]\ is \emph{well-formed}.
            %
            The well-formendess of \CIso*'_1\ is dependant on ${\TypeS}''_1$, which for the transition by the premise of rule \LblCfgIsoUnfold*\ must be either $\simplechoice$ or $\mu\alpha'.{\TypeS}'''_1$ (see other cases, as in~\Cref{lem:cfg_wf_then_live}).
            %
        \end{inductivecase}
        %
    \end{inductivecase}

    \noindent Therefore, it holds that any transition made by a system composed of compatible and \emph{well-formed} configurations will result in configurations that are \emph{well-formed}.
    %
\end{proof}
% 