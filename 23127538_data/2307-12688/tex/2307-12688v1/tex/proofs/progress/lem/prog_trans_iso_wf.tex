\begin{lemma}\label{lem:prog_trans_iso_wf}
   Two compatible participants \compconfig{}{1}{}{2}, where \isoconfig{1}\ and \isoconfig{2} are \emph{well-formed}, that take a transition \action:
   \par\noindent\hspace{5ex}\rule{0pt}{2.5ex}\configtrans[\maction]{\mparconfig{}{1}{}{2}}{\mparconfig{'}{1}{'}{2}},
   \par\noindent\hspace{11ex}\rule{0pt}{2.5ex}result with \isoconfig[']{1}\ and \isoconfig[']{2} that are well-formed.
   %
\end{lemma}
\begin{proof}
   This proof proceeds by induction on the depth of the derivation tree, proceeding by analysing the last rule applied
   \begin{inline}
      \item \sysrWait
      \item \sysrComm
      \item \sysrDequ
   \end{inline}
   \begin{description}[itemsep=1.75ex,parsep=1.25ex]
      % ! Wait
      \item[{Rule \sysrWait}]
         Then $\maction=\mtval$, and by the premise of \sysrWait, the transition of \tval\ is a time stepping transition only possible via \socrTime\ for social configurations:
         \begin{equation*}
            \msocconfig{1}%
            \transition{\mtval}%
            \msocconfig[']{1}%
            %
            \qquad\text{and}\qquad
            %
            \msocconfig{2}%
            \transition{\mtval}%
            \msocconfig[']{2}%
         \end{equation*}
         and via \isorTime\ for isolated configurations:
         \begin{equation*}
            \misoconfig{1}%
            \transition{\mtval}%
            \misoconfig[']{1}%
            %
            \qquad\text{and}\qquad
            %
            \misoconfig{2}%
            \transition{\mtval}%
            \misoconfig[']{2}%
         \end{equation*}
         By~\cref{lem:configs_iso_trans,lem:configs_soc_trans} it is known that
         \begin{inline}
            \item $\of{\mst}{1}'=\of{\mst}{1}$
            \item $\of{\mval}{1}'=\of{\mval}{1}+\mtval$
            \item $\of{\queue}{1}'=\of{\queue}{1}$
         \end{inline}
         The same applies for
         \begin{inline}
            \item $\of{\mst}{2}'$
            \item $\of{\mval}{2}'$
            \item $\of{\queue}{2}'$
         \end{inline}
         Therefore, for the hypothesis to hold, $(\of{\mval}{1}+\mtval,\,\of{\mst}{1})\quad\text{and}\quad(\of{\mval}{2}+\mtval,\,\of{\mst}{2})$ must be well-formed.

         When $\mtval=0$, there are no changes, indicating that this is either the last opportunity for a viable sending action, or, that there is a message ready to be received in the queue. By~\cref{lem:configs_iso_trans,lem:configs_soc_trans}, $\miso[']{}=\miso{}$ and $\msoc[']{}=\msoc{}$.
         The premise of \socrTime\ has the following additional items:
         \begin{enumerate}
            \item $\mactFE{(\mval,\,\mst)}{\msend}\implies\mactFE{(\mval',\,\mst)}{\msend}$
                  ensuring that as sending actions become viable, time is allowed to pass across the clocks up until the soonest constraint would be come unsatisfied, at which point one action must be taken. It is important to note that a \emph{future-enabled} sending action is not necessarily viable, but has the potential to be viable in the future.
                  %
            \item $\forall\mtval'<\mtval:(\mval+\mtval',\,\mst,\,\queue)\ntransition{\tau}$
                  ensuring that messages are received as soon as they become available in the queue.
                  %
         \end{enumerate}
         Whenever a transition of \tval\ is performed, it must be valued so that both of the conditions above are met by reducing its valuation.

         By inner induction of the derivation of
         $\miso{1}:\menvempty\,\of{\mconstr}{1}\imp\of{\mst}{1}$ and $\of{\mval}{1}\models\of{\mconstr}{1}$,
         by case analysis of the last rule applied is as follows (with \iso{2} being the same)
         \begin{inline}
            \item \formEnd
            \item \formChoice
            \item \formRec
         \end{inline}
         In the scenarios that follow, it is assumed that \iso{1}\ is next to perform an action, and therefore the focus as it is responsible for reducing the value of \tval.
         \begin{description}[parsep=0.75ex,listparindent=4ex]
            \item[{Rule \formEnd}]
               Then $\of{\mst}{1}=\mend$. \iso{1}\ is always well-formed as $\mtrue\imp\mend$ and $\of{\mval}{1}\models\mtrue$ always holds.
               Therefore, $(\of{\mval}{1}+\mtval,\,\mend)$ is well-formed as $\menvempty\,\mtrue\imp\of{\mst}{1}$ and $\of{\mval}{1}+\mtval\models\mtrue$.
               %
            \item[{Rule \formChoice}]
               Then $\of{\mst}{1}=\msetinteract{i}$.
               By~\cref{def:types_well_formed,def:configs_iso_well_formed} and the hypothesis, the following holds for \iso{1}:
               \begin{equation*}
                  \menvempty\,\mpast[\of{\bigvee}{\elemofset{i}{i}}\,\of{\mconstr}{i}]\imp\msetinteract[i]{i}
                  %
                  \qquad\text{and}\qquad
                  %
                  \of{\mval}{1}\models\of{\mconstr}{i}
                  %
               \end{equation*}
               The premise of \formChoice\ enforces that constraints on actions of one kind (sending or receiving) are only satisfiable when the constraints on actions of the other kind (receiving or sending) are not satisfied:
               \begin{equation*}
                  \forall\elemofset{i \neq j}{i}\quad%
                  \of{\mconstr}{i}\cap\of{\mconstr}{j}=\emptyset%
                  \quad\lor\quad%
                  \of{\mcomm}{i}=\of{\mcomm}{j}%
               \end{equation*}
               Therefore interactions with different actions (sending or receiving) \emph{cannot} compete to be taken at the same time.
               Combined with the premises of \socrTime\ discussed earlier in this Lemma, \tval\ is valued so that $\of{\mval}{1}+t\models\of{\mconstr}{i}$ and $\menvempty\,\of{\mconstr}{i}\imp\of{\mst}{1}$ where $\elemofset{i}{i}=\lvert\of{\mst}{1}\rvert$.
               %
            \item[{Rule \formRec}]
               Then $\of{\mst}{1}=\mrec\mathtt{t}.\of{\mst}{1}'$. By the hypothesis and~\cref{def:types_well_formed,def:configs_iso_well_formed}, the following is known of \iso{1}:
               \begin{equation*}
                  \menvempty\,\mconstr\imp\mrec\mathtt{t}.\of{\mst}{1}'
                  %
                  \qquad\text{and}\qquad
                  %
                  \of{\mval}{1}\models\mconstr
                  %
               \end{equation*}
               By induction on the depth of the derivation tree, the analysis of \isorRec\ premised by \isorTime, is as follows:
               \begin{equation*}
                  \infer[\begin{array}{l}\raisebox{1ex}{$\rulem{iso-time}$}\\\raisebox{2.5ex}{$\rulem{recursion}$}\end{array}]%
                  {%
                  (\of{\mval}{1},\,\mrec\mathtt{t}.\of{\mst}{1}')
                  \transition{\mtval}
                  (\of{\mval}{1}+\mtval,\,\of{\mst}{1}')
                  }{%
                  (\of{\mval}{1},\,\of{\mst}{1}' [\mrec\mathtt{t}.\of{\mst}{1}'/\mathtt{t}])
                  \transition{\mtval}
                  (\of{\mval}{1}+\mtval,\,\of{\mst}{1}')
                  }%
               \end{equation*}
               The rule \isorRec\ only unfolds recursive types, and by the premise of the rule \formRec, it is shown that the constraints on clocks is unchanged as it progresses to the proceeding type $\of{\mst}{1}'$:
               \begin{equation*}
                  \menvempty[,\,\mathtt{t}:\mconstr]\,\mconstr\imp\of{\mst}{1}'
                  %
                  \qquad\text{and}\qquad
                  %
                  \of{\mval}{1}+\mtval\models\mconstr
                  %
               \end{equation*}
               Therefore, whether $(\of{\mval}{1}+t,\,\of{\mst}{1}')$ is well-formed depends on $\of{\mst}{1}'$. Each possibility is discussed in each of the cases of this inner induction of \sysrWait.
               %
         \end{description}

         % ! Comm
      \item[{Rule \sysrComm}]
         The following only considers the scenario where \soc{1}\ sends the message $\mmsg$, and \soc{2}\ enqueues it. The evaluation is as follows:
         \begin{equation*}\label{eqn:prog_trans_iso_wf_comm}
            \infer[\begin{array}{l}\raisebox{1ex}{$\rulem{enqu}$}\\\raisebox{2.5ex}{$\rulem{comm}$}\end{array}]%
            {%
               \msocconfig{1}\mid\msocconfig{2}%
               \transition{\tau}%
               \msocconfig[']{1}\mid\msocconfig[']{2}%
            }%
            {%
               \infer[\rulem{send}]%
               {%
                  \msocconfig{1}%
                  \transition{\msend\,\mmsg}%
                  (\of{\mval}{1}',\,\of{\mst}{1}',\,\of{\queue}{1})%
               }%
               {%
                  \misoconfig{1}%
                  \transition{\msend\,\mlabel}%
                  \misoconfig[']{1}
                  %
                  &%
                  \mlabel=\mmsg%
                  &%
                  \mcomm=\msend
               }%
               %
               &%
               %
               (\of{\mval}{2},\,\of{\mst}{2},\,\of{\queue}{2})%
               \transition{\mrecv\,\mmsg}%
               (\of{\mval}{2},\,\of{\mst}{2},\,\of{\queue}{2};\mmsg)%
               %
            }%
         \end{equation*}
         By~\cref{lem:configs_soc_trans}, $\miso{2}=\miso[']{2}$, therefore \iso[']{2}\ is well-formed.
         By induction of the derivation of
         $\miso{1}:\menvempty\,\of{\mconstr}{1}\imp\of{\mst}{1}$ and $\of{\mval}{1}\models\of{\mconstr}{1}$,
         the case analysis of the last rule applied is as follows:
         \begin{inline}
            \item \formEnd
            \item \formChoice
            \item \formRec
         \end{inline}
         \begin{description}[parsep=0.75ex,listparindent=4ex]
            \item[{Rule \formEnd}]
               \REDO[As with the previous case \sysrWait.] Then $\of{\mst}{1}=\mend$. \iso{1} is always well-formed as $\mtrue\imp\mend$ and $\of{\mval}{1}\models\mtrue$ always holds.
               Therefore, $(\of{\mval}{1}+\mtval,\,\mend)$ is well-formed as $\menvempty\,\mtrue\imp\of{\mst}{1}$ and $\of{\mval}{1}+\mtval\models\mtrue$.
               %
            \item[{Rule \formChoice}]
               Then $\of{\mst}{1}=\msetinteract{i}$.
               By the induction hypothesis, and the premise of \formChoice, \iso{1}\ being well-formed requires that the following must hold, following~\cref{def:types_well_formed,def:configs_iso_well_formed}:
               \begin{equation}
                  \menvempty\,\mpast[\of{\mconstr}{i}]\imp\of{\mst}{1}
                  \quad\text{and}\quad
                  \of{\mval}{1}\models\mpast[\of{\mconstr}{i}]
               \end{equation}
               For the above, $\mpast[\of{\mconstr}{i}]$ is used by following the judgement of \formChoice, to indicate that there is still a viable action.

               Proceeding by induction on the depth of the derivation tree, the analysis of \socrSend\ premised by \isorInteract, is as follows, beginning with its evaluation:
               \begin{equation*}
                  \infer[\rulem{send}]%
                  {%
                     (\of{\mval}{1},\,\msetinteract{i},\,\of{\queue}{1})
                     \transition{\msend\,\mmsg}
                     (\of{\mval}{1}',\,\of{\mst}{j},\,\of{\queue}{1})
                  }{%
                     \infer[\rulem{interact}]%
                     {%
                        (\of{\mval}{1},\msetinteract[i]{i})%
                        \transition{\of{\mcomm}{j}\,\of{\mlabel}{j}}%
                        (\of{\mval}{1}\actReset[\of{\mresets}{j}],\of{\mst}{j})%
                        %
                        \quad%
                        %
                        \of{\mlabel}{j}=\mmsg%
                        \quad%
                        \of{\mcomm}{j}=\msend
                     }%
                     {%
                        \of{\mval}{1}\models\of{\mconstr}{j}&%
                        (\of{\mconstr}{j}\cap\of{\mconstr}{k}=\emptyset%
                        \,\vee\,%
                        \of{\mcomm}{j}=\of{\mcomm}{k})&%
                        \elemofset{j\neq k}{i}%
                     }%
                  }%
               \end{equation*}
               The premise of \isorInteract\ coincides with the induction hypothesis, indicating that \iso{1}\ is well-formed (note that the premise of \isorInteract\ does not use $\of{\mval}{1}\models\mpast[\of{\mconstr}{j}]$, indicating that the constraints are currently satisfied with the valuation of clocks). In order fo $\miso[']{1}=(\of{\mval}{1}',\,\of{\mst}{j},\,\of{\queue}{1})$ to be well-formed, the following must hold:
               \begin{equation}
                  \menvempty\,\mpast[\mconstr']\imp\of{\mst}{1}'
                  \quad\text{and}\quad
                  \of{\mval}{1}'\models\mpast[\mconstr']
               \end{equation}
               Where $\mconstr'=\of{\mconstr}{k}$, and, $\elemofset{k}{k}=\lvert\of{\mst}{1}'\rvert$, and
               \begin{inline}
                  \item the new type corresponds to the action taken $\of{\mst}{1}'=\of{\mst}{j}$
                  %
                  \item the relevant clocks are reset in the new valuation $\of{\mval}{1}'=\of{\mval}{1}\actReset[\of{\mresets}{j}]$
                  %
                  \item the new type is a set of interactions $\of{\mst}{j}=\msetinteract[k]{k}$
                  %
               \end{inline}

               It must be that the current evaluation of clocks satisfies an actions constraints $\of{\mval}{1}\models\of{\mconstr}{j}$ by the premise of \isorInteract.
               It follows that the same clock resets imposed on both the valuation and constraints of clocks yields the similarly equivalent: $\of{\mval}{1}\actReset[\of{\mresets}{j}]\models\of{\mconstr}{j}\actReset[\of{\mresets}{j}]$.

               By the induction hypothesis, the initial configuration being well-formed requires that the types adhere to the formation rules, and by \formChoice, subsequent types must have at least one interaction with constraints that are satisfiable given the last constraints imposed on the clocks: $\of{\mconstr}{j}\actReset[\of{\mresets}{j}]\subseteq\mconstr'$; and, a satisfiable constraint allows types to be reached: $\of{\mconstr}{j}\imp\of{\mst}{j}$.

               Therefore, it holds that the new valuation of clocks must be able to satisfy the constraints of the proceeding type, either immediately or in the future $\of{\mval}{1}'\models\mpast[\mconstr']$, and that when this constraint is satisfied an interaction can be taken $\menvempty\,\mconstr'\imp\of{\mst}{1}'$. Therefore, \isoconfig[']{1}\ is well-formed.
               \TODO[need to do case of $\of{\mst}{1}'$ being a recursive type?]
               % 
            \item[{Rule \formRec}]
               Then $\of{\mst}{1}=\mrec\mathtt{t}.\of{\mst}{1}'$.
               \TODO[same as with case \rulet{wait}?]
               %
         \end{description}

         % ! Dequ
      \item[{Rule \sysrDequ}]
         The following only considers the scenario where \soc{2}\ is receiving a message $\mmsg$ from its queue. The evaluation is as follows:
         \begin{equation*}
            \infer[\rulem{dequ}]%
            {%
               \msocconfig{1}\mid\msocconfig{2}%
               \transition{\tau}%
               \msocconfig{1}\mid\msocconfig[']{2}%
            }%
            {%
               \infer[\rulem{recv}]%
               {%
                  (\of{\mval}{2},\,\of{\mst}{2},\,\mmsg;\of{\queue}{2})%
                  \transition{\mrecv\,\mmsg}%
                  (\of{\mval}{2}',\,\of{\mst}{2}',\,\of{\queue}{2})%
               }%
               {%
                  \infer[\rulem{interact}]%
                  {%
                     (\of{\mval}{2},\msetinteract[i]{i})%
                     \transition{\of{\mcomm}{j}\,\of{\mlabel}{j}}%
                     (\of{\mval}{2}\actReset[\of{\mresets}{j}],\of{\mst}{j})%
                     %
                     \quad%
                     %
                     \of{\mlabel}{j}=\mmsg%
                     \quad%
                     \of{\mcomm}{j}=\mrecv
                  }%
                  {%
                     \of{\mval}{2}\models\of{\mconstr}{j}\quad%
                     (\of{\mconstr}{j}\cap\of{\mconstr}{k}=\emptyset%
                     \,\vee\,%
                     \of{\mcomm}{j}=\of{\mcomm}{k})\quad%
                     \elemofset{j\neq k}{i}%
                  }%
               }%
            }%
         \end{equation*}
         By~\cref{lem:configs_soc_trans}, $\miso{1}=\miso[']{1}$, therefore \iso[']{1}\ is well-formed.
         By induction of the derivation of
         $\miso{2}:\menvempty\,\of{\mconstr}{2}\imp\of{\mst}{2}$ and $\of{\mval}{2}\models\of{\mconstr}{2}$,
         the case analysis of the last rule applied is as follows:
         \begin{inline}
            \item \formEnd
            \item \formChoice
            \item \formRec
         \end{inline}
         \begin{description}[parsep=0.75ex,listparindent=4ex]
            \item[{Rule \formEnd}]
               \REDO[As with the previous case \sysrWait.] Then $\of{\mst}{2}=\mend$. \iso{2} is always well-formed as $\mtrue\imp\mend$ and $\of{\mval}{2}\models\mtrue$ always holds.
               Therefore, $(\of{\mval}{2}+\mtval,\,\mend)$ is well-formed as $\menvempty\,\mtrue\imp\of{\mst}{2}$ and $\of{\mval}{2}+\mtval\models\mtrue$.
               %
            \item[{Rule \formChoice}]
               Then $\of{\mst}{2}=\msetinteract{i}$.
               By the induction hypothesis, and following the premise of \formChoice, \iso{2}\ being well-formed requires that the following must hold, following~\cref{def:types_well_formed,def:configs_iso_well_formed}:
               \begin{equation}
                  \menvempty\,\mpast[\of{\mconstr}{i}]\imp\of{\mst}{2}
                  \quad\text{and}\quad
                  \of{\mval}{2}\models\mpast[\of{\mconstr}{i}]
               \end{equation}
               \REDO[The rest of the proof is analogous to the one found in case \rulet{comm} of this Lemma.]
               %
            \item[{Rule \formRec}]
               \REDO[The rest of the proof is analogous to the one found in case \rulet{comm} of this Lemma.]
               %
         \end{description}
   \end{description}
\end{proof}

\endinput