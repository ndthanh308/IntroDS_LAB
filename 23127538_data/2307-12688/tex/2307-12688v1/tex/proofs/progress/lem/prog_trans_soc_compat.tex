\begin{lemma}\label{lem:prog_trans_soc_compat}
   Given two participants that are compatible \compconfig{}{1}{}{2}, and, both \isoconfig{1}\ and \isoconfig{2} are \emph{well-formed}:\par
   %
   If \;\configtrans[\maction]{\mparconfig{}{1}{}{2}}{\mparconfig{'}{1}{'}{2}}, \par\noindent
   \hspace{7ex}\rule{0pt}{2.5ex} then \compconfig{'}{1}{'}{2}.
   %
\end{lemma}
\begin{proof}
   This proof proceeds by induction on the depth of the derivation tree, proceeding by analysing the last rule applied
   \begin{inline}
      \item \sysrWait
      \item \sysrComm
      \item \sysrDequ
   \end{inline}
   \begin{description}[itemsep=1.25ex,parsep=1.25ex]
      % ! Wait
      \item[{Rule \sysrWait}]
         then $\maction=\mtval$. The evaluation is the same as in~\cref{lem:prog_trans_iso_wf}, {Rule \rulet{wait}}:
         \begin{equation*}
            \infer[\rulem{wait}]%
            {%
               (\of{\mval}{1},\,\of{\mst}{1},\,\of{\queue}{1})
               \,\mid\,
               (\of{\mval}{2},\,\of{\mst}{2},\,\of{\queue}{2})
               %
               \transition{\mtval}
               %
               (\of{\mval}{1}',\,\of{\mst}{1},\,\of{\queue}{1})
               \,\mid\,
               (\of{\mval}{2}',\,\of{\mst}{2},\,\of{\queue}{2})
            }{%
               \infer[\rulem{time}]%
               {%
                  (\of{\mval}{1},\,\of{\mst}{1},\,\of{\queue}{1})
                  \transition{\mtval}
                  (\of{\mval}{1}',\,\of{\mst}{1},\,\of{\queue}{1})
               }{%
                  \dots
               }%
               &
               \infer[\rulem{time}]%
               {%
                  (\of{\mval}{2},\,\of{\mst}{2},\,\of{\queue}{2})
                  \transition{\mtval}
                  (\of{\mval}{2}',\,\of{\mst}{2},\,\of{\queue}{2})
               }{%
                  \dots
               }%
            }%
         \end{equation*}
         By~\cref{lem:configs_iso_trans,lem:configs_soc_trans} it is known that
         \begin{inline}
            \item $\of{\mst}{1}'=\of{\mst}{1}$
            \item $\of{\mval}{1}'=\of{\mval}{1}+\mtval$
            \item $\of{\queue}{1}'=\of{\queue}{1}$
         \end{inline}
         The same applies for
         \begin{inline}
            \item $\of{\mst}{2}'$
            \item $\of{\mval}{2}'$
            \item $\of{\queue}{2}'$
         \end{inline}

         Regardless of the value of \mtval, the compatibility of the configurations will remain the same. Below the evaluation of \soc{1}\ is continued (with the evaluation of \soc{2}\ being analogous):
         \begin{equation*}
            \infer[\rulem{time}]%
            {%
               (\of{\mval}{1},\,\of{\mst}{1},\,\of{\queue}{1})%
               \transition{\mtval}%
               (\of{\mval}{1}',\,\of{\mst}{1},\,\of{\queue}{1})%
            }{%
               (\of{\mval}{1},\,\of{\mst}{1})%
               \transition{\mtval}%
               (\of{\mval}{1}+\mtval,\,\of{\mst}{1})%
               &
               \mactFE{(\of{\mval}{1},\,\of{\mst}{1})}{\msend}%
               \implies%
               \mactFE{(\of{\mval}{1}',\,\of{\mst}{1})}{\msend}%
               &
               \forall\mtval'<\mtval:(\of{\mval}{1}+\mtval',\,\of{\mst}{1},\,\of{\queue}{1})%
               \ntransition{\tau}%
            }%
         \end{equation*}
         By the premise of \isorTime, as in~\cref{lem:prog_trans_iso_wf}, $\mtval=0$ indicates that there is either a message in the queue or a sending action must be performed. Again, by~\cref{lem:configs_iso_trans,lem:configs_soc_trans} it is known that, for $\mtval=0$ it must be that $\msoc[']{1}\,\bot\,\msoc[']{2}$.

         $\mtval>0$ indicates that the queues are empty $\of{\queue}{1}=\emptyset=\of{\queue}{2}$ until at least $\mval+\mtval$; by~\cref{itm:configs_soc_compat_duality} of~\cref{def:configs_soc_compat}, the types are dual $\of{\mst}{1}=\mdual[\of{\mst}{2}]$ and have matching clock valuations $\of{\mval}{1}=\of{\mval}{2}$. Hereafter, the analysis of \sysrWait\ continues with only focusing on \soc{1}\ as by duality, \soc{2}\ \REDO[is analogous.]

         The inner induction of the derivation of $\of{\mst}{1}$ and $\of{\mst}{2}$ for a time stepping transition \tval, is the same as shown in~\cref{lem:prog_trans_iso_wf}. As the initial configurations queues are empty, they are dual, and time will progress equally. Therefore the resulting configurations \socconfig[']{1}\ and \socconfig[']{2}\ are also compatible by adhereing to the conditions specified in~\cref{def:configs_soc_compat}.

         % ! Comm
      \item[{Rule \sysrComm}]
         The following only considers the scenario where \soc{1}\ sends the message $\mmsg$, and \soc{2}\ enqueues it. The evaluation is as follows:
         \begin{equation*}
            \infer[\begin{array}{l}\raisebox{1ex}{$\rulem{enqu}$}\\\raisebox{2.5ex}{$\rulem{comm}$}\end{array}]%
            {%
               \msocconfig{1}\,\mid\,\msocconfig{2}%
               \transition{\tau}%
               \msocconfig[']{1}\,\mid\,\msocconfig[']{2}%
            }%
            {%
               \infer[\rulem{send}]%
               {%
                  \msocconfig{1}%
                  \transition{\msend\,\mmsg}%
                  (\of{\mval}{1}',\,\of{\mst}{1}',\,\of{\queue}{1})%
               }%
               {%
                  \misoconfig{1}%
                  \transition{\msend\,\mlabel}%
                  \misoconfig[']{1}
                  %
                  &%
                  \mlabel=\mmsg%
                  &%
                  \mcomm=\msend
               }%
               %
               &%
               %
               (\of{\mval}{2},\,\of{\mst}{2},\,\of{\queue}{2})%
               \transition{\mrecv\,\mmsg}%
               (\of{\mval}{2},\,\of{\mst}{2},\,\of{\queue}{2};\mmsg)%
               %
            }%
         \end{equation*}
         Proceeding by inner induction on the queue contents $\of{\queue}{1}$ and $\of{\queue}{2}$, analysing the possible configurations by the induction hypothesis
         \begin{inlineEnum}
            \item $\of{\queue}{1}=\emptyset\;\mid\;\of{\queue}{2}=\emptyset$\;
            %
            \item $\of{\queue}{1}=\emptyset\;\mid\;\of{\queue}{2}=\mmsg';\of{\queue}{2}$\;
            %
            \item $\of{\queue}{1}=\mmsg';\of{\queue}{1}\;\mid\;\of{\queue}{2}'=\emptyset$\;
            %
         \end{inlineEnum}
         \begin{description}[parsep=0.75ex,listparindent=4ex]
            \item[{Case $\of{\queue}{1}=\emptyset\;\mid\;\of{\queue}{2}=\emptyset$}]
               Both queues are empty, by~\cref{itm:configs_soc_compat_duality} of~\cref{def:configs_soc_compat}, the types are dual $\of{\mst}{1}=\mdual[\of{\mst}{2}]$.
               The resulting configurations $\msoc[']{1}\,\mid\,\msoc[']{2}$ are as follows:
               \begin{equation*}
                  (\of{\mval}{1}',\,\of{\mst}{1}',\,\emptyset)%
                  \,\mid\,%
                  (\of{\mval}{2},\,\of{\mst}{2},\,\mmsg;\emptyset)%
               \end{equation*}
               As~\cref{def:types_duality} specifies, dual types have a corresponding receiving action for every sending action; both types sharing the same constraints and resets over their respective local and equivalent clocks.

               Therefore, it is guaranteed that the message will be received by \soc[']{2}, which by adhering to~\cref{itm:configs_soc_compat_expected_receive} of~\cref{def:configs_soc_compat}, means that $\msoc[']{1}\,\mid\,\msoc[']{2}$ are compatible.
               %
            \item[Case $\of{\queue}{1}=\emptyset\;\mid\;\of{\queue}{2}=\mmsg';\of{\queue}{2}$]
               There is a message $\mmsg'$ already in the recipients queue $\of{\queue}{2}$, waiting to be received.
               By the hypothesis, and~\cref{itm:configs_soc_compat_expected_receive} of~\cref{def:configs_soc_compat}, the message $\mmsg'$ is expected, and will lead to a new configuration \soc['']{2}\ that is compatible with the sender \soc{1}.
               The resulting configurations $\msoc[']{1}\,\mid\,\msoc[']{2}$ are as follows:
               \begin{equation*}
                  (\of{\mval}{1}',\,\of{\mst}{1}',\,\emptyset)%
                  \,\mid\,%
                  (\of{\mval}{2},\,\of{\mst}{2},\,\mmsg';\mmsg;\emptyset)%
               \end{equation*}
               If, instead of \soc{1}\ sending the second message $\mmsg$, \soc{2}\ were to have received the message $\mmsg'$ that is already in its queue, the resulting configurations would have empty queues queues. By~\cref{itm:configs_soc_compat_duality} of~\cref{def:configs_soc_compat}, types would be dual $\of{\mst}{1}=\mdual[\of{\mst}{2}'']$, and clock valuations would match $\of{\mval}{1}=\of{\mval}{2}''$. Afterwards, if \soc{1}\ were to continue sending the message $\mmsg$, by the notion of duality in~\cref{def:types_duality}, any sending action must have a corresponding receiing action, as shown in the previous case, {Rule $\of{\queue}{1}=\emptyset,\;\of{\queue}{2}=\emptyset$}.

               Therefore, it holds that, for compatible and well-formed configurations, enqueuing subsequent messages into a non-empty queue will result in compatible configurations.
               %
            \item[Case $\of{\queue}{1}=\mmsg;\of{\queue}{1}\;\mid\;\of{\queue}{2}=\emptyset$]
               The resulting configurations $\msoc[']{1}\,\mid\,\msoc[']{2}$ are \emph{not compatible} by~\cref{itm:configs_soc_compat_empty_queues} of~\cref{def:configs_soc_compat}, as shown below:
               \begin{equation*}
                  (\of{\mval}{1}',\,\of{\mst}{1}',\,\mmsg';\emptyset)%
                  \,\mid\,%
                  (\of{\mval}{2},\,\of{\mst}{2},\,\mmsg;\emptyset)%
               \end{equation*}
               However, the initial configurations \soc{1}\ and \soc{2}\ do not coincide with the hypothesis, as well-formed configurations should not have a viable sending action while there is a message waiting to be received.

               As shown in~\cref{lem:prog_trans_iso_wf}, case {Rule \rulet{wait}}, time cannot pass over the clocks while there is a message waiting to be received.
               By induction hypothesis, the message $\mmsg'$ in the queue of \soc{1}\ is known to be an expected and viable receiving action, by~\cref{itm:configs_soc_compat_expected_receive} of~\cref{def:configs_soc_compat}.

               By induction hypothesis, \iso{1}\ and \iso{2}\ are well-formed, and must adhere to the formation rules. By inner induction on the derivation of $\of{\mst}{1}:(\of{\mval}{1},\,\of{\mst}{1},\,\mmsg';\emptyset)\transition{\msend\,\mmsg}$, analysing the last rule applied: \REDO[add recursive?]
               \begin{description}[parsep=0.75ex,listparindent=4ex]
                  \item[Rule \formChoice]
                     Then $\of{\mst}{1}=\msetinteract{i}$.
                     The premise of \formChoice\ ensures that sending and receiving actions cannot happen at the same time.

                     Therefore, it holds that it is not possible for a message to be sent while the senders queue is not empty, for the following reasons
                     \begin{inlineEnum}
                        \item By induction hypothesis, the message $\mmsg'$ in the queue of \soc{1}\ is able to be received
                        %
                        \item By~\cref{lem:prog_trans_iso_wf}, case {Rule \rulet{wait}}, time cannot pass while any message can be received
                        %
                        \item By the premise of \formChoice, it is not possible for a sending action to be viable while a receiving action waits for the message
                        %
                     \end{inlineEnum}
                     %
               \end{description}
         \end{description}

         % ! Dequ
      \item[{Rule \sysrDequ}]
         The following only considers the scenario where \soc{2}\ is receiving a message $\mmsg$ from its queue. The evaluation is as follows:
         \begin{equation*}
            \infer[\rulem{dequ}]%
            {%
               \msocconfig{1}\,\mid\,\msocconfig{2}%
               \transition{\tau}%
               \msocconfig[']{1}\,\mid\,\msocconfig[']{2}%
            }{%
               \infer[\rulem{recv}]%
               {%
                  \msocconfig{2}%
                  \transition{\tau}%
                  \msocconfig[']{2}%
               }{%
                  (\of{\mval}{2},\,\of{\mst}{2})%
                  \transition{\mrecv\,\mlabel}%
                  (\of{\mval}{2}',\,\of{\mst}{2}')%
                  &
                  \mlabel=\mmsg%
               }%
            }%
         \end{equation*}
         Proceeding by inner induction on the derivation of $\of{\mst}{2}:(\of{\mval}{2},\,\of{\mst}{2},\,\emptyset)\transition{\mrecv\,\mmsg}$, analysing the last rule applied:
         \begin{description}[parsep=0.75ex,listparindent=4ex]
            \item[Rule \formChoice]
               Then $\of{\mst}{2}=\msetinteract{i}$.
               The evaluation continues as follows:
               \begin{equation*}
                  \infer[\rulem{dequ}]%
                  {%
                     \msocconfig{1}\,\mid\,\msocconfig{2}%
                     \transition{\tau}%
                     \msocconfig[']{1}\,\mid\,\msocconfig[']{2}%
                  }{%
                     \infer[\rulem{recv}]%
                     {%
                        \msocconfig{2}%
                        \transition{\tau}%
                        \msocconfig[']{2}%
                     }{%
                        \infer[\rulem{interact}]%
                        {%
                           (\of{\mval}{2},\msetinteract[i]{i})%
                           \transition{\of{\mcomm}{j}\,\of{\mlabel}{j}}%
                           (\of{\mval}{2}\actReset[\of{\mresets}{j}],\of{\mst}{j})%
                           %
                           \quad%
                           %
                           \of{\mlabel}{j}=\mmsg%
                           \quad%
                           \of{\mcomm}{j}=\mrecv
                        }{%
                           \of{\mval}{2}\models\of{\mconstr}{j}&%
                           (\of{\mconstr}{j}\cap\of{\mconstr}{k}=\emptyset%
                           \,\vee\,%
                           \of{\mcomm}{j}=\of{\mcomm}{k})&%
                           \elemofset{j\neq k}{i}%
                        }%
                     }%
                  }%
               \end{equation*}
               By the induction hypothesis, \soc{1}\ and \soc{2}\ being compatible requires that the message $\mmsg$ in the queue of \soc{2}\ is guaranteed to be a viable receiving action, which will lead to another compatible configuration. Therefore, \soc[']{1}\ and \soc[']{2}\ are compatible.
               %
            \item[Rule \formRec]
               Then $\of{\mst}{2}=\mrec\mathtt{t}.\of{\mst}{2}''$.
               The evaluation continues as follows:
               \begin{equation*}
                  \infer[\rulem{dequ}]%
                  {%
                  \msocconfig{1}\,\mid\,\msocconfig{2}%
                  \transition{\tau}%
                  \msocconfig[']{1}\,\mid\,\msocconfig[']{2}%
                  }{%
                  \infer[\rulem{recv}]%
                  {%
                  \msocconfig{2}%
                  \transition{\tau}%
                  \msocconfig[']{2}%
                  }{%
                  \infer[\rulem{recursion}]%
                  {%
                  (\of{\mval}{2},\mrec\mathtt{t}.\of{\mst}{2}'')%
                  \transition{\mrecv\,\mlabel}%
                  (\of{\mval}{2}',\of{\mst}{2}')%
                  }{%
                  (\of{\mval}{2},\of{\mst}{2}''[\mrec\mathtt{t}.\of{\mst}{2}''/\mathtt{t}])%
                  \transition{\mrecv\,\mlabel}%
                  (\of{\mval}{2}',\of{\mst}{2}')%
                  }%
                  }%
                  }%
               \end{equation*}
               \REDO[continue, redirecting to other cases depending on the value of ]$\of{\mst}{2}''$?
               %
         \end{description}
   \end{description}
\end{proof}

\endinput
