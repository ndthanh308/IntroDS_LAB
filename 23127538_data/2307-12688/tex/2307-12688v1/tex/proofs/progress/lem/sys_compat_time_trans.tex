%
% ! (lemma 13) : time passing implies empty queues
\begin{lemma}\label{lem:sys_compat_time_trans}
   %
   If \Compat*[\VSoc_1][\VSoc_2]\ and \Trans*{\Parl{\VSoc_1,\VSoc_2}}:{\ValTime}\ and $\ValTime>0$ then $\Queue_1=\emptyset=\Queue_2$.
   %
\end{lemma}
\begin{proof}
   %
   Such a transition is only specified by \LblCfgSysWait*, which by its premise requires a \LblCfgSocTime*\ transition of \ValTime*\ for each \VSoc*_1\ and \VSoc*_2.
   %
   By contradiction, if one queue were \emph{non-empty}, say $\Queue_1=\Msg;\Queue_1$, then by~\Cref{itm:configs_compat_expected_receive} of~\Cref{def:configs_compat} message \Msg*\ must be able to be received immediately.
   %
   The premise of \LblCfgSocTime*\ (urgency) ensures that \ValTime*\ must be valued such that no time passes while a message is able to be received.
   %
   % It holds that $t$ must equal 0 when there is a non-empty queue.
   
   Therefore the hypothesis holds.
   %if a system makes a $t$ transition where $t>0$ then all queues in the system must be empty.
   % Therefore, \ValTime*\ must equal $0$ when there is a message in any queue in a system composed of compatible configurations.
   %
 \end{proof}
 % 