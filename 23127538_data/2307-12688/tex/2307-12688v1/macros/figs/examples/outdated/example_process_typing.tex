
\newcommand{\extyp}[1][0]{\of{\mathtt{\mst}}{#1}}
\newcommand{\extypA}[0]
{\msend\mathtt{start}\,\left(\mtrue\right)}

\newcommand{\extypB}[0]
{\mrecv\mathtt{ready}\,\left({x\leq5},x\right)}

\newcommand{\extypC}[0]
{\mrec\,{{\mreclabel}^{A}}}

\newcommand{\extypD}[0]
{\mrecv\mathtt{stop}\,\left({x\leq1}\right)}

\newcommand{\extypE}[0]
{\mend}

\newcommand{\extypF}[0]
{\msend\mathtt{data}\,\left({1<x\leq3}\right)}

\newcommand{\extypG}[0]
{{\mreclabel}^{A}}

\newcommand{\extypH}[0]
{\msend\mathtt{redo}\,\left({3<x}\right)}

\newcommand{\extypI}[0]
{{\mreclabel}^{A}}

\newcommand{\extypJ}[0]
{\msend\mathtt{close}\,\left({5<x}\right)}

\newcommand{\extypK}[0]
{\mend}

\begin{equation*}
   %
   \extypA % ! start
   \,.\,
   \left\{
   \begin{array}{l}
      \extypB % ? ready
      \,.\,
      \extypC % # def rec
      \,.\,
      \left\{
      \begin{array}{l}
         \extypD % ? stop
         \,.\,
         \extypE % end
         %
         \\[2ex]
         \extypF % ! data
         \,.\,
         \extypG % rec call
         %
         \\[2ex]
         \extypH % ! redo
         \,.\,
         \extypI % rec call
      \end{array}
      \right\}
      %
      \\[2ex]
      \extypJ % ! close
      \,.\,
      \extypK % end
      %
   \end{array}
   \right\}
   %
\end{equation*}


\newcommand{\exprc}[1][0]{\of{\mathtt{\mprc}}{#1}}
\newcommand{\exprcA}[0]
{\mpchan\mpsend\mathtt{start}}

\newcommand{\exprcAb}[0]
{\mprctimerset[y]}

\newcommand{\exprcAcA}[0]
{{\mpchan}^{5}}

\newcommand{\exprcAcAb}[0]
{\mprecv}

\newcommand{\exprcB}[0]
{\mathtt{ready}}

\newcommand{\exprcBb}[0]
{\mprctimerset[y]}

\newcommand{\exprcC}[0]
{\mathtt{def}\;\mprcvar[\mathtt{ready},\mpchan,y]=\exprc[2]\;\mathtt{in}}

\newcommand{\exprcDaA}[0]
{{\mpchan}^{1}}

\newcommand{\exprcDaAb}[0]
{\mprecv}

\newcommand{\exprcDb}[0]
{\mathtt{stop}}

\newcommand{\exprcE}[0]
{0}

\newcommand{\exprcDaB}[0]
{\mathtt{after}^{1}}

\newcommand{\exprcFa}[0]
{y\models\left(1<x\leq3\right)\;\mathtt{then}}

\newcommand{\exprcFb}[0]
{\mpchan\mpsend\mathtt{data}}

\newcommand{\exprcG}[0]
{\exprc[2]}

\newcommand{\exprcHa}[0]
{y\models\left(3<x\right)\;\mathtt{then}}

\newcommand{\exprcHb}[0]
{\mpchan\mpsend\mathtt{redo}}

\newcommand{\exprcI}[0]
{\exprc[2]}

\newcommand{\exprcAcB}[0]
{\mathtt{after}^{5}}

\newcommand{\exprcJ}[0]
{\mpchan\mpsend\mathtt{close}}

\newcommand{\exprcK}[0]
{0}

\begin{equation*}
   %
   \begin{array}[t]{r l}
      %
      \exprcAb % set x
      \,.\,
      \exprcA % ! start
      \,.\, &
      \begin{array}[t]{r c l} % recv, after
         \exprcAcA            & \exprcAcAb & % recv
         \begin{array}[t]\{{l c l}\}
            \exprcB  &     % ? ready
            .        &
            \exprcBb % set x
            %
            \\[1ex]
                     &
            .        &
            \exprcC % def rec
            %
            \\[1.2ex]
            \mbox{ } & . &
            \begin{array}[t]{r c l} % recv, after
               \exprcDaA            & \exprcDaAb & % recv
               \begin{array}[t]\{{r c l}\}
                  \exprcDb & % ? stop
                  .        &
                  \exprcE % end
               \end{array}
               % 
               \\[1ex]
               \mathllap{\exprcDaB} & \mbox{ }   & % after
               \begin{array}[t]{r c l}
                  \mathtt{if}              &
                  \mbox{ }                 &
                  \exprcFa % if data
                  \\[1ex]
                  \mbox{ }                 &
                  \mbox{ }                 &
                  \exprcFb % ! data
                  \,.\,
                  \exprcG % rec call
                  %
                  \\[1ex]
                  \mathllap{\mathtt{else}} &
                  \mbox{ }                 &
                  \exprcHb % ! redo
                  \,.\,
                  \exprcI % rec call
               \end{array}
               %
            \end{array}
         \end{array}
         % 
         \\[20ex]
         \mathllap{\exprcAcB} & \mbox{ }   & % after
         \exprcJ % ! close
         \,.\,
         \exprcK % end
         %
      \end{array}
      %
   \end{array}
   %
\end{equation*}

\paragraph*{Evaluation}
\begin{equation*}
   %
   \infer[\rulem{timer}]{%
      %
      \mpenv
      \imp
      %
      \exprcAb.\,\exprcA\dots
      %
      \quad\mptyped\quad
      %
      \msenv,\mprcrole:(\nu,\extypA\dots)
      %
   }{%
      %
      \infer[\rulem{send}]{%
         %
         \mpenv
         \imp
         %
         \exprcA.\,\exprcAcA\exprcAcAb\left\{\dots\right\}\mathtt{after}\dots
         %
         \quad\mptyped\quad
         %
         \msenv,\mprcrole:(\nu,\extypA\dots)
         %
      }{%
         %
         \infer[\rulem{recv}]{%
            %
            \mpenv
            \imp
            %
            \exprcAcA\exprcAcAb\left\{\exprcB\dots\right\}\mathtt{after}\;\exprcJ\dots
            %
            \quad\mptyped\quad
            %
            \msenv,\mprcrole:(\nu,\extypA\dots)
            %
         }{%
            %
            \mtrue
            %
         }
      }
   }
   %
\end{equation*}

\endinput
\paragraph*{Structural Layout}
\begin{equation*}
   %
   \begin{array}[t]{l c l}
      %
      \text{\textbf{Types}}     &
      \mbox{ }                  &
      \text{\textbf{Processes}}
      %
      \\[2ex] % * init clocks
      \begin{array}[t]{l c l}
      \end{array}   &   &
      \begin{array}[t]{l c l}
         \exprc[0] & = & \exprcAb \\[1.2ex]
      \end{array}
      %
      \\[2ex] % ! start
      \begin{array}[t]{l c l}
         \extyp[A] & = & \extypA
      \end{array}  &   &
      \begin{array}[t]{l c l}
         \exprc[A] & = & \exprcA
      \end{array}
      %
      \\[2ex] % ? ready
      \begin{array}[t]{l c l}
         \extyp[B1] & = & \extypB
      \end{array} &   &
      \begin{array}[t]{l c l}
         \exprc[B1a] & = & \exprcAcA\exprcAcAb\exprcB \\[1.2ex]
         \exprc[B1b] & = & \exprcBb
      \end{array}
      %
      \\[0ex]\\[0ex] % * def rec
      \begin{array}[t]{l c l}
         \extyp[C] & = & \extypC
      \end{array}  &   &
      \begin{array}[t]{l c l}
         \exprc[C] & = & \exprcC
      \end{array}
      %
      \\[2ex] % ? stop
      \begin{array}[t]{l c l}
         \extyp[D1] & = & \extypD
      \end{array} &   &
      \begin{array}[t]{l c l}
         \exprc[D1a] & = & \exprcDaA  \\[1.2ex]
         \exprc[D1b] & = & \exprcDaAb \\[1.2ex]
         \exprc[D1c] & = & \exprcDb
      \end{array}
      %
      \\[2ex] % * end
      \begin{array}[t]{l c l}
         \extyp[E] & = & \extypE
      \end{array}  &   &
      \begin{array}[t]{l c l}
         \exprc[E] & = & \exprcE
      \end{array}
      %
      \\[2ex] % ! data
      \begin{array}[t]{l c l}
         \extyp[D2] & = & \extypF
      \end{array} &   &
      \begin{array}[t]{l c l}
         \exprc[D2a] & = & \exprcDaB \\[1.2ex]
         \exprc[D2b] & = & \exprcFa  \\[1.2ex]
         \exprc[D2c] & = & \exprcFb
      \end{array}
      %
      \\[2ex] % * rec call
      \begin{array}[t]{l c l}
         \extyp[F] & = & \extypG
      \end{array}  &   &
      \begin{array}[t]{l c l}
         \exprc[F] & = & \exprcG
      \end{array}
      %
      \\[2ex] % ! redo
      \begin{array}[t]{l c l}
         \extyp[D3] & = & \extypH
      \end{array} &   &
      \begin{array}[t]{l c l}
         \exprc[D3a] & = & \exprcHa \\[1.2ex]
         \exprc[D3b] & = & \exprcHb
      \end{array}
      %
      \\[2ex] % * rec call
      \begin{array}[t]{l c l}
         \extyp[G] & = & \extypI
      \end{array}  &   &
      \begin{array}[t]{l c l}
         \exprc[G] & = & \exprcI
      \end{array}
      %
      \\[2ex] % ! close
      \begin{array}[t]{l c l}
         \extyp[B2] & = & \extypJ
      \end{array} &   &
      \begin{array}[t]{l c l}
         \exprc[B2a] & = & \exprcAcB \\[1.2ex]
         \exprc[B2b] & = & \exprcJ
      \end{array}
      %
      \\[2ex] % * end
      \begin{array}[t]{l c l}
         \extyp[H] & = & \extypK
      \end{array}  &   &
      \begin{array}[t]{l c l}
         \exprc[H] & = & \exprcK
      \end{array}
      %
   \end{array}
   %
\end{equation*}

\endinput




\newcommand{\exampleprocesstypestructural}[0]{%
   \extyp[0] % ! start
   \,.\,
   \left\{
   \begin{array}{l}
      \extyp[1] % ? ready
      \,.\,
      \extyp[2] % # def rec
      \,.\,
      \left\{
      \begin{array}{l}
         \extyp[3] % ? stop
         \,.\,
         \extyp[4] % end
         %
         \\[2ex]
         \extyp[5] % ! data
         \,.\,
         \extyp[6] % rec call
         %
         \\[2ex]
         \extyp[7] % ! redo
         \,.\,
         \extyp[8] % rec call
      \end{array}
      \right\}
      %
      \\[2ex]
      \extyp[9] % ! close
      \,.\,
      \extyp[10] % end
      %
   \end{array}
   \right\}
   %
}

\newcommand{\exampleprocessprocessstructural}[0]{%
   %
   %
}


\paragraph*{Structural Layout}

\begin{tabular}[t]{c c}
   %
   \textbf{Types} & \textbf{Process} \\
   %
   \begin{equation*}
      \exampleprocesstypestructural
   \end{equation*}
   %
                  &
   %
   \begin{equation*}
      \exampleprocessprocessstructural
   \end{equation*}
   %
\end{tabular}


