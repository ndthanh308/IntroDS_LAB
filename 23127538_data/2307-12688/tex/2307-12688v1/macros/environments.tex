
% % for subfigures with no outer figure caption
% % https://tex.stackexchange.com/questions/111302/1-a-numbering-of-subfigures-within-a-figure-without-a-caption
% \makeatletter
% \newcommand\NoOuterCaption{%
%    \renewcommand\p@subfigure{}
%    \renewcommand\thesubfigure{\thefigure\alph{subfigure}}
% }
% \makeatother

\crefname{task}{condition}{conditions}
\crefformat{task}{condition~#2#1#3}
\crefmultiformat{task}{conditions~#2#1#3}{ and~#2#1#3}{, #2#1#3}{ and~#2#1#3}
\crefrangeformat{task}{conditions~#3#1#4 to~#5#2#6}

\Crefname{task}{Condition}{Conditions}
\Crefformat{task}{Condition~#2#1#3}
\Crefmultiformat{task}{Conditions~#2#1#3}{ and~#2#1#3}{, #2#1#3}{ and~#2#1#3}
\Crefrangeformat{task}{Conditions~#3#1#4 to~#5#2#6}

\newlist{inductivecase}{enumerate}{4}

\setlist[inductivecase,1]{label={\itshape{Case~\arabic*.}},ref={\arabic*},topsep=1ex,itemsep=1ex,parsep=0.5ex,leftmargin=5ex,itemindent=4ex,left=0ex}

\setlist[inductivecase,2]{label={\itshape{\roman*}.},ref={\arabic{inductivecasei}.\roman*},topsep=0.5ex,itemsep=0.75ex,parsep=0.5ex,leftmargin=4ex,itemindent=0ex,left=-2ex}

\setlist[inductivecase,3]{label={\itshape{\alph*}.},ref={\arabic{inductivecasei}.\roman{inductivecaseii}.\alph*},topsep=0.5ex,itemsep=0.75ex,parsep=0.5ex,leftmargin=4ex,itemindent=0ex,left=-2ex}

\crefname{inductivecasei}{case}{cases}
\crefformat{inductivecasei}{case~#2#1#3}
\crefmultiformat{inductivecasei}{cases~#2#1#3}{ and~#2#1#3}{, #2#1#3}{ and~#2#1#3}
\crefrangeformat{inductivecasei}{cases~#3#1#4 to~#5#2#6}

\crefname{inductivecaseii}{subcase}{subcases}
\crefformat{inductivecaseii}{subcase~#2#1#3}
\crefmultiformat{inductivecaseii}{subcases~#2#1#3}{ and~#2#1#3}{, #2#1#3}{ and~#2#1#3}
\crefrangeformat{inductivecaseii}{subcases~#3#1#4 to~#5#2#6}

\crefname{inductivecaseiii}{subcase}{subcases}
\crefformat{inductivecaseiii}{subcase~#2#1#3}
\crefmultiformat{inductivecaseiii}{subcases~#2#1#3}{ and~#2#1#3}{, #2#1#3}{ and~#2#1#3}
\crefrangeformat{inductivecaseiii}{subcases~#3#1#4 to~#5#2#6}

\Crefname{inductivecasei}{Case}{Cases}
\Crefformat{inductivecasei}{Case~#2#1#3}
\Crefmultiformat{inductivecasei}{Cases~#2#1#3}{ and~#2#1#3}{, #2#1#3}{ and~#2#1#3}
\Crefrangeformat{inductivecasei}{Cases~#3#1#4 to~#5#2#6}

\Crefname{inductivecaseii}{Subcase}{Subcases}
\Crefformat{inductivecaseii}{Subcase~#2#1#3}
\Crefmultiformat{inductivecaseii}{Subcases~#2#1#3}{ and~#2#1#3}{, #2#1#3}{ and~#2#1#3}
\Crefrangeformat{inductivecaseii}{Subcases~#3#1#4 to~#5#2#6}

\Crefname{inductivecaseiii}{Subcase}{Subcases}
\Crefformat{inductivecaseiii}{Subcase~#2#1#3}
\Crefmultiformat{inductivecaseiii}{Subcases~#2#1#3}{ and~#2#1#3}{, #2#1#3}{ and~#2#1#3}
\Crefrangeformat{inductivecaseiii}{Subcases~#3#1#4 to~#5#2#6}

\NewDocumentCommand{\NewCase}{e| o}{\IfValueT{#2}{\IfValueT{#1}{#1~}{#2}:}\hspace{1.5ex}}

% ~ case for rule induction
% \newcounter{inductivecase}[lemma]
% \NewDocumentEnvironment{inductivecase}{o}{% * begdef
% 	%
% 	\stepcounter{inductivecase}%
% 	{Case~\arabic{inductivecase}}%
% 	\IfValueT{#1}{:~{#1}}%
% 	%
% 	\par\noindent%
% 	\hfill%
% 	\begin{minipage}{\subcaseindent}\Buff[0]%
% 	%
% }{% * enddef
% 	%
% 	\end{minipage}%
% 	%
% }

% ~ inner case environment
% \declaretheoremstyle[
% 	headfont=\normalfont\itshape
% ]{romanstyle}
% \declaretheorem[name=Subcase,style=romanstyle,refname={subcase,subcases},Refname={Subcase,Subcases},]{scase}

% ~ make scase counter roman
% \renewcommand\thescase{\roman{scase}}

% ~ make cases sub lemmas
% \AtBeginEnvironment{lemma}{\setcounter{inductivecase}{0}}
% \AtBeginEnvironment{inductivecase}{\setcounter{scase}{0}}
% \AtBeginEnvironment{lemma}{\setcounter{case}{0}}
% \AtBeginEnvironment{case}{\setcounter{scase}{0}}

% ~ make proof correspond to lemma counter
\newcounter{proof}
\AtBeginEnvironment{lemma}{\setcounter{proof}{\thelemma}}

\crefalias{proof}{lemma}

% \crefname{proof}{proof}{proofs}
% \crefformat{proof}{proof of lemma~#2#1#3}
% \crefmultiformat{proof}{proof of lemma's~#2#1#3}{ and~#2#1#3}{, #2#1#3}{ and~#2#1#3}
% \crefrangeformat{proof}{proof of lemma's~#3#1#4 to~#5#2#6}

% ~ highlight
% for starting and ending color zone
\makeatletter
\def\colorstart{\edef\colorend{\pdfliteral{\current@color}}}
\makeatother
\newcommand{\highlight}[2][magenta]{\color{#1}\colorstart {#2} \colorend}

% ~ note & annotations
\newcommand{\lnote}[1]{\color{magenta}#1\color{black}}

% ~ old todo/redo/done
% \newcommand{\todo}[1][...]{{\ \highlight[teal]{ToDo: {#1}}}\ }
% \newcommand{\redo}[1][...]{{\ \highlight[orange]{ReDo: {#1}}}\ }
% \newcommand{\done}[1][...]{{\ \highlight[green]{Done: {#1}}}\ }

% ~ new todo/redo/done
\NewDocumentCommand{\TODO}{s o}{%
   \IfBooleanT{#1}{$} % mm wrap
   {\highlight[red]{ToDo\IfNoValueF{#2}{\footnote{\highlight[red]{ToDo: #2}}}}}\!
   \IfBooleanT{#1}{$} % mm wrap
}
\NewDocumentCommand{\REDO}{s o}{%
   \IfBooleanT{#1}{$} % mm wrap
   {\highlight[orange]{Redo\IfNoValueF{#2}{\footnote{\highlight[orange]{Redo: #2}}}}}\!
   \IfBooleanT{#1}{$} % mm wrap
}
\NewDocumentCommand{\NOTE}{s o}{%
   \IfBooleanT{#1}{$} % mm wrap
   {\highlight[teal]{Note\IfNoValueF{#2}{\footnote{\highlight[teal]{Note: #2}}}}}\!
   \IfBooleanT{#1}{$} % mm wrap
}
\NewDocumentCommand{\DONE}{s o}{%
   \IfBooleanT{#1}{$} % mm wrap
   {\highlight[green]{Done\IfNoValueF{#2}{\footnote{\highlight[green]{Done: #2}}}}}\!
   \IfBooleanT{#1}{$} % mm wrap
}

% ~ defaults for itemize/enumerate
\setlist[itemize]{topsep=0.5pt,itemsep=0.5pt}
\setlist[enumerate]{topsep=0.5pt,itemsep=0.5pt}

% ~ inline list
% #1 : s  : numbered
% #2 : O  : number format
% #3 : t/ : no leading : ?
% #4 : t+ : final and
% #5 : t| : final or
% #6 : t; : 1) x; or, 2) y
% #7 : t. : no trailing ?
% #8 : t> : flow diagram
% #9 : O> : flow symbol
\NewDocumentEnvironment{inline}{s O{(\arabic{*})} t/ t+ t| t; t. t> O{\;$\rightarrow$\,}}{%
   \begin{enumerate*}[%
         label={\IfBooleanT{#1}{#2}},%
         before=\unskip{\IfBooleanF{#3}{:~}},%
         itemjoin=\discretionary{\IfBooleanTF{#8}{#9}{;}}{}{\hbox{\IfBooleanTF{#8}{#9}{;}}},%
         %
         itemjoin*=\discretionary{% * pre-break
            \hbox{\IfBooleanTF{#8}{#9}{\IfBooleanTF{#6}{;}{,}}}%
         }{% * post-break
            \hbox{\IfBooleanTF{#4}{~and}{\IfBooleanT{#5}{~or}}\IfBooleanTF{#6}{,}{}}%
         }{% * no break
            \hbox{\IfBooleanTF{#8}{#9}{\IfBooleanTF{#6}{;}{,}\IfBooleanTF{#4}{~and}{\IfBooleanT{#5}{~or}}\IfBooleanTF{#6}{,}{}}}%
         },%
         %
         after=\unskip{\IfBooleanTF{#7}{\!}{.}}%
      ]%
      }{% bottom
   \end{enumerate*}\!
}

% ~ indented environment
% https://tex.stackexchange.com/questions/235925/how-to-change-margins-in-enunciation-theorem-like-environment
\newenvironment{indentedenvironment}{%
   \begin{list}{}{%
         \setlength{\topsep}{0pt}%
         \setlength{\leftmargin}{1cm}%    
         \setlength{\rightmargin}{1cm}%  
         \setlength{\listparindent}{\parindent}%
         \setlength{\itemindent}{\parindent}%
         \setlength{\parsep}{\parskip}%
      }%
      \item[]%
      }{\end{list}}%

% ~ patch claim environment
\newcounter{claim}
% \numberwithin{claim}{lemma}
\let\oldclaim\claim
\let\oldendclaim\endclaim
\renewenvironment{claim}{%
   \setlength{\topsep}{0pt}%
   \setlength{\leftmargin}{1cm}%    
   \setlength{\rightmargin}{1cm}%  
   \setlength{\listparindent}{\parindent}%
   \setlength{\itemindent}{\parindent}%
   \setlength{\parsep}{\parskip}%
   \stepcounter{claim}%
   \oldclaim[\theclaim]%
   % \begin{indentedenvironment}%\vskip-\baselineskip%
}{%
   % \end{indentedenvironment}%
   \oldendclaim%
}%
\AtBeginEnvironment{lemma}{\setcounter{claim}{0}}

% ~ note environment
% \let\oldnote\note
% \let\oldendnote\endnote
% \renewenvironment{note}{%
% 	\let\noteoldabovedisplayskip\abovedisplayskip%
% 	\let\noteoldbelowdisplayskip\belowdisplayskip%
% 	\setlength{\abovedisplayskip}{0pt}%
% 	\setlength{\belowdisplayskip}{2ex plus1ex minus1ex}%
% 	\par\hfill\begin{minipage}{0.95\textwidth}%
% 		\oldnote%
% 		\Buff[3]%
% 		}{%
% 		\oldendnote\end{minipage}%
% 	\setlength{\abovedisplayskip}{\noteoldabovedisplayskip}%
% 	\setlength{\belowdisplayskip}{\noteoldbelowdisplayskip}\!%
% }

% ~ mini equation environment
% #1 : s  : numbered
% #2 : o  : r-aligned caption
\NewDocumentEnvironment{minieq}{s o}{%
\let\minieqoldabovedisplayskip\abovedisplayskip%
\let\minieqoldbelowdisplayskip\belowdisplayskip%
\let\minieqoldpartopsep\partopsep%
\setlength{\abovedisplayskip}{7pt plus2pt minus4pt}%
\setlength{\belowdisplayskip}{7pt plus2pt minus4pt}%
\setlength{\partopsep}{0pt}
\IfBooleanT{#1}{\begin{equation*}}\IfBooleanF{#1}{\begin{equation}}%
            }{%
            \IfBooleanF{#1}{\end{equation}}\IfBooleanT{#1}{\end{equation*}}%
\IfNoValueF{#2}{\hfill\ \emph{Note:}\;{\footnotesize #2}}%
\setlength{\abovedisplayskip}{\minieqoldabovedisplayskip}%
% \setlength{\belowdisplayskip}{\minieqoldbelowdisplayskip}%
\setlength{\partopsep}{\minieqoldpartopsep}%
\!%
}

% ~ description environment presets
% #1 : s  : outer of nest
% #2 : e1 : preset: 
% #3 : e2 : preset: 
% #4 : e3 : preset: 
% #5 : e4 : preset: 
\NewDocumentEnvironment{desc}{s e1 e2 e3 e4}{%
\IfBooleanTF{#1}
{ % * outer of nest
   \begin{description}[itemsep=1.25ex,labelsep=2ex,parsep=0.5ex,listparindent=4ex]
}{ % * not nested
   \begin{description}[itemsep=0.5ex,labelsep=2ex,parsep=0ex,listparindent=0ex]
      }
      }{%
   \end{description}
}

% ~ tcolorbox text mode
\NewTotalTCBox{\ColBoxText}{O{\tcolorOld} m}{%
   size=title,tcbox raise base,nobeforeafter,top=2pt,bottom=0pt,left=2pt,right=2pt,boxsep=0mm,colback=#1,colframe=black!0}{{\rule{0pt}{1ex}#2}}

% ~ tcolorbox new
\NewDocumentCommand{\ColBoxNew}{s m}{%
   \IfBooleanTF{#1}{% * text mode
      \ColBoxText[\tcolorNew]{#2}%
   }{% * math mode
      \tcboxmath[size=title,boxsep=0mm,colback=\tcolorNew,colframe=black!0]{\begin{array}[c]{c}#2\end{array}}
   }%
}

% ~ tcolorbox new
\NewDocumentCommand{\ColBoxMod}{s m}{%
   \IfBooleanTF{#1}{% * text mode
      \ColBoxText[\tcolorMod]{#2}%
   }{% * math mode
      \tcboxmath[size=title,boxsep=0mm,colback=\tcolorMod,colframe=black!0]{\begin{array}[c]{c}#2\end{array}}
   }%
}

% ~ tcolorbox old
\NewDocumentCommand{\ColBoxOld}{s m}{%
   \IfBooleanTF{#1}{% * text mode
      \ColBoxText[\tcolorOld]{#2}%
   }{% * math mode
      \tcboxmath[size=title,boxsep=0mm,colback=\tcolorOld,colframe=black!0]{\begin{array}[c]{c}#2\end{array}}
   }%
}

\endinput

% ~ case for rule induction
\newcounter{inductivecase}[lemma]
\NewDocumentEnvironment{inductivecase}{o}{% * begdef
%
\stepcounter{inductivecase}%
{Case~\arabic{lemma}.\arabic{inductivecase}}%
\IfValueT{#1}{:~{#1}}%
%
\par\noindent\hfill%
\begin{minipage}{\subcaseindent}\Buff[0]%
   %
   }{% * enddef
   %
\end{minipage}%
%
}

% ~ subcase environment
\newcounter{scase}[inductivecase]
\NewDocumentEnvironment{scase}{o}{% * begdef
%
\stepcounter{scase}%
{Case~\arabic{inductivecase}.\roman{scase}}%
\IfValueT{#1}{:~{#1}}%
%
\par\noindent\hfill%
\begin{minipage}{\subcaseindent}\Buff[0]%
   %
   }{% * enddef
   %
\end{minipage}%
%
}

% ~ make cases sub lemmas
\AtBeginEnvironment{lemma}{\setcounter{case}{0}}
\AtBeginEnvironment{lemma}{\setcounter{inductivecase}{0}}
\AtBeginEnvironment{inductivecase}{\setcounter{scase}{0}}
