\section{Introduction}
Histology is critical for accurately diagnosing all cancers in modern medical imaging analysis. 
% 
However, the complex scanning procedure for histological whole-slide images (WSIs) digitization may result in the alteration of tissue structures, due to improper removal, fixation, tissue processing, embedding, and storage~\cite{artifacts}. 
% 
Typically, these changes in tissue details can be caused by various extraneous factors such as bubbles, tissue folds, uneven illumination, pen marks, altered staining, and \textit{etc}~\cite{artifacttype}.
% 
Formally, the changes in tissue structures are known as artifacts.
% 
The presence of artifacts not only makes the analysis more challenging for pathologists but also increases the risk of misdiagnosis for Computer-Aided Diagnosis (CAD) systems~\cite{artifactresult}. 
% 
Particularly, deep learning models, which have become increasingly prevalent in histology analysis, have shown vulnerability to the artifact, resulting in a two-times increase in diagnosis errors~\cite{zhang2022benchmarking}. 



% Figure environment removed



In real clinical practice, rescanning the WSIs that contain artifacts can partially address this issue.
% 
However, it may require multiple attempts before obtaining a satisfactory WSI, which can lead to a waste of time, medical resources, and deplete tissue samples.
% 
Discarding the local region with artifacts for deep learning models is another solution, but it may result in the loss of critical contextual information.
% 
Therefore, learning-based artifact restoration approaches have gained increasing attention.
% 
For example, CycleGAN~\cite{cycleGAN} formulates the artifact restoration as an image-to-image transfer problem by learning the transfer between the artifact and artifact-free image domains from unpaired images, as depicted in Fig.~\ref{fig:intro}(a).
% 
However, existing artifact restoration solutions are confined to Generative Adversarial Networks (GANs)~\cite{GANs}, which are difficult to train due to the mode collapse and are prone to suffer from unexpected stain style mistransfer.
% 
To address these issues, we make the first attempt at a diffusion probabilistic model for artifact restoration approach~\cite{DDPM}, as shown in Fig.~\ref{fig:intro}(b).  
% 
Innovatively, our framework formulates the artifact restoration as a regional denoising process, which thus can to the most extent preserve the stain style and avoid the loss of contextual information in the non-artifact region.
% 
Furthermore, our approach is trained solely with artifact-free images, which reduces the difficulty in data collection.



The major contributions are two-fold.
% 
(1) 
% 
We make the first attempt at a denoising diffusion probabilistic model for artifact removal, called \texttt{ArtiFusion}. 
% 
This approach differs from GAN-based methods that require either paired or unpaired artifacts and artifact-free images, as our \texttt{ArtiFusion} relies solely on artifact-free images, resulting in a simplified training process.
% 
(2) To capture the local-global correlations in the gradual regional artifact restoration process, we innovatively propose a Swin-Transformer denoising architecture to replace the commonly-used U-Net and a time token scheme for optimal Swin-Transformer denoising. 
% 
Extensive evaluations on real-world histology datasets and downstream tasks demonstrate the superiority of our framework in artifact removal performance, which can generate reliable restored images while preserving the stain style.  
