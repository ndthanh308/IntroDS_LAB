\begin{abstract}
Histological whole slide images (WSIs) can be usually compromised by artifacts, such as tissue folding and bubbles, which will increase the examination difficulty for both pathologists and Computer-Aided Diagnosis (CAD) systems. 
% 
Existing approaches to restoring artifact images are confined to Generative Adversarial Networks (GANs), where the restoration process is formulated as an image-to-image transfer.
% 
Those methods are prone to suffer from mode collapse and unexpected mistransfer in the stain style, leading to unsatisfied and unrealistic restored images.
% 
Innovatively, we make the first attempt at a denoising diffusion probabilistic model for histological artifact restoration, namely \texttt{ArtiFusion}.
% 
Specifically, \texttt{ArtiFusion} formulates the artifact region restoration as a gradual denoising process, and its training relies solely on artifact-free images to simplify the training complexity.
% 
Furthermore, to capture local-global correlations in the regional artifact restoration, a novel Swin-Transformer denoising architecture is designed, along with a time token scheme. 
% 
Our extensive evaluations demonstrate the effectiveness of \texttt{ArtiFusion} as a pre-processing method for histology analysis, which can successfully preserve the tissue structures and stain style in artifact-free regions during the restoration.
% 
Code is available at \url{https://github.com/zhenqi-he/ArtiFusion}.
\keywords{Histological Artifact Restoration \and Diffusion Probabilistic Model \and Swin-Transformer Denoising Network.}
\end{abstract}

