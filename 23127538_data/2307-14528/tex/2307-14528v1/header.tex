 % style
\usepackage{fullpage}
\usepackage{layout}
\usepackage{multirow}
% \usepackage{cases}

% ams
\usepackage{amsfonts}
\usepackage{amsmath}
\usepackage{amsthm}
\usepackage{amssymb} % NOT COMATIBLE WITH svjour3
\usepackage{dsfont}

% nice fractions
\usepackage{nicefrac}
\usepackage{mathtools}

\newcommand{\colvec}[1]{\begin{bmatrix}#1\end{bmatrix}}

%%%%%%%%%%%%%%%%%%%%%%%%
%%%%% Normal THEOREMS 
%%%%%%%%%%%%%%%%%%%%%%%%
%\newtheorem{assumption}{Assumption}
%\newtheorem{lemma}{Lemma}
%\newtheorem{algorithms}{Algorithm}
%\newtheorem{theorem}{Theorem}
%\newtheorem{proposition}{Proposition}
%\newtheorem{example}{Example}
%\newtheorem{remark}{Remark}
%\newtheorem{claim}[theorem]{Claim}
%\newtheorem{corollary}{Corollary}
%\newtheorem{exercise}[theorem]{Exercise}
%
%\theoremstyle{definition}
%\newtheorem{definition}[theorem]{Definition}


%%%%%%%%%%%%%%%%%%%%%%%%
%%%%% shaded THEOREMS 
%%%%%%%%%%%%%%%%%%%%%%%%
\usepackage{mdframed} 
\usepackage{thmtools,thm-restate}

\definecolor{shadecolor}{gray}{0.90}
\declaretheoremstyle[
headfont=\normalfont\bfseries,
notefont=\mdseries, notebraces={(}{)},
bodyfont=\normalfont,
postheadspace=0.5em,
spaceabove=1pt,
mdframed={
  skipabove=8pt,
  skipbelow=8pt,
  hidealllines=true,
  backgroundcolor={shadecolor},
  innerleftmargin=4pt,
  innerrightmargin=4pt}
]{shaded}





\declaretheorem[style=shaded,within=section]{definition}
\declaretheorem[style=shaded,sibling=definition]{theorem}
\declaretheorem[style=shaded,sibling=definition]{proposition}
\declaretheorem[style=shaded,sibling=definition]{assumption}
\declaretheorem[style=shaded,sibling=definition]{corollary}
\declaretheorem[style=shaded,sibling=definition]{conjecture}
\declaretheorem[style=shaded,sibling=definition]{lemma}
\declaretheorem[style=shaded,sibling=definition]{remark}
\declaretheorem[style=shaded,sibling=definition]{example}

% algorithms
\usepackage{algorithm}
\usepackage[noend]{algpseudocode}



% captions
% \usepackage{caption}
% \usepackage{subcaption}


% graphics
 \usepackage{xcolor}
\usepackage{color}
\usepackage{graphicx}
\graphicspath{{./figures/}}  % Path to figures folder


% various
\usepackage{nameref,hyperref,cleveref}
%\usepackage[cp1250]{inputenc}
%\usepackage[T1]{fontenc}


\newcommand\tagthis{\addtocounter{equation}{1}\tag{\theequation}}

%%%%%%%%%%%%%%%%%%%%%%%%%
%%%%%% BIBLIOGRAPHY
%%%%%%%%%%%%%%%%%%%%%%%%%

%\usepackage[maxbibnames=99, maxcitenames=10,doi=false,isbn=false,url=false,backend=bibtex]{biblatex}
%%\bibliography{SDA.bib}
%\newcommand{\Ref}[1]{../ref/#1}
%%\input{\Ref{biblatex_journal_def}}

%%% Basic sets
\newcommand{\R}{\mathbb{R}} % Reals
\newcommand{\N}{\mathbb{N}} % Naturals

% caligraphic
\newcommand{\cA}{{\cal A}}
\newcommand{\cB}{{\cal B}}
\newcommand{\cC}{{\cal C}}
\newcommand{\cD}{{\cal D}}
\newcommand{\cE}{{\cal E}}
\newcommand{\cF}{{\cal F}}
\newcommand{\cG}{{\cal G}}
\newcommand{\cH}{{\cal H}}
\newcommand{\cJ}{{\cal J}}
\newcommand{\cK}{{\cal K}}
\newcommand{\cL}{{\cal L}}
\newcommand{\cM}{{\cal M}}
\newcommand{\cN}{{\cal N}}
\newcommand{\cO}{{\cal O}}
\newcommand{\cP}{{\cal P}}
\newcommand{\cQ}{{\cal Q}}
\newcommand{\cR}{{\cal R}}
\newcommand{\cS}{{\cal S}}
\newcommand{\cT}{{\cal T}}
\newcommand{\cU}{{\cal U}}
\newcommand{\cV}{{\cal V}}
\newcommand{\cX}{{\cal X}}
\newcommand{\cY}{{\cal Y}}
\newcommand{\cW}{{\cal W}}
\newcommand{\cZ}{{\cal Z}}


% bold
\newcommand{\bA}{{\bf A}}
\newcommand{\bB}{{\bf B}}
\newcommand{\bC}{{\bf C}}
\newcommand{\bE}{{\bf E}}
\newcommand{\bI}{{\bf I}}
\newcommand{\bS}{{\bf S}}
\newcommand{\bZ}{{\bf Z}}

% red matrices
%\newcommand{\mA}{{\color{red}\bf A}}
%\newcommand{\mB}{{\color{red}\bf B}}
%\newcommand{\mC}{{\color{red}\bf C}}
%\newcommand{\mE}{{\color{red}\bf E}}
%\newcommand{\mF}{{\color{red}\bf F}}
%\newcommand{\mG}{{\color{red}\bf G}}
%\newcommand{\mH}{{\color{red}\bf H}}
%\newcommand{\mI}{{\color{red}\bf I}}
%\newcommand{\mJ}{{\color{red}\bf J}}
%\newcommand{\mK}{{\color{red}\bf K}}
%\newcommand{\mL}{{\color{red}\bf L}}
%\newcommand{\mM}{{\color{red}\bf M}}
%\newcommand{\mN}{{\color{red}\bf N}}
%\newcommand{\mO}{{\color{red}\bf O}}
%\newcommand{\mP}{{\color{red}\bf P}}
%\newcommand{\mQ}{{\color{red}\bf Q}}
%\newcommand{\mR}{{\color{red}\bf R}}
%\newcommand{\mS}{{\color{red}\bf S}}
%\newcommand{\mT}{{\color{red}\bf T}}
%\newcommand{\mU}{{\color{red}\bf U}}
%\newcommand{\mV}{{\color{red}\bf V}}
%\newcommand{\mW}{{\color{red}\bf W}}
%\newcommand{\mX}{{\color{red}\bf X}}
%\newcommand{\mY}{{\color{red}\bf Y}}
%\newcommand{\mZ}{{\color{red}\bf Z}}

% matrices
\newcommand{\mA}{{\bf A}}
\newcommand{\mB}{{\bf B}}
\newcommand{\mC}{{\bf C}}
\newcommand{\mD}{{\bf D}}

\newcommand{\mE}{{\bf E}}
\newcommand{\mF}{{\bf F}}
\newcommand{\mG}{{\bf G}}
\newcommand{\mH}{{\bf H}}
\newcommand{\mI}{{\bf I}}
\newcommand{\mJ}{{\bf J}}
\newcommand{\mK}{{\bf K}}
\newcommand{\mL}{{\bf L}}
\newcommand{\mM}{{\bf M}}
\newcommand{\mN}{{\bf N}}
\newcommand{\mO}{{\bf O}}
\newcommand{\mP}{{\bf P}}
\newcommand{\mQ}{{\bf Q}}
\newcommand{\mR}{{\bf R}}
\newcommand{\mS}{{\bf S}}
\newcommand{\mT}{{\bf T}}
\newcommand{\mU}{{\bf U}}
\newcommand{\mV}{{\bf V}}
\newcommand{\mW}{{\bf W}}
\newcommand{\mX}{{\bf X}}
\newcommand{\mY}{{\bf Y}}
\newcommand{\mZ}{{\bf Z}}
\newcommand{\mLambda}{{\bf \Lambda}}

\newcommand{\zeros}{{\bf 0}}
\newcommand{\ones}{{\bf 1}}

% Commenting
\usepackage[colorinlistoftodos,bordercolor=orange,backgroundcolor=orange!20,linecolor=orange,textsize=scriptsize]{todonotes}
\newcommand{\rob}[1]{\todo[inline]{\textbf{Robert: }#1}}
\newcommand{\guillaume}[1]{\todo[inline]{\textbf{Guillaume: }#1}}
\newcommand{\aaron}[1]{\todo[inline]{\textbf{Felix: }#1}}
\newcommand{\fabian}[1]{\todo[inline]{\textbf{Fabian: }#1}}


\newcommand{\YY}{\gamma}
\newcommand{\XX}{\omega}


% basic
% \newcommand{\eqdef}{\overset{\text{def}}{=}} 
\newcommand{\eqdef}{\coloneqq} 

%\newcommand{\eqdef}{\stackrel{\text{def}}{=}}
\newcommand{\st}{\;:\;} % such that
\newcommand{\ve}[2]{\langle #1 ,  #2 \rangle} % inner
\newcommand{\dotprod}[1]{\left< #1\right>} % product
\newcommand{\norm}[1]{ \left\| #1 \right\|}      % norm 


% statistical
%\DeclareMathOperator{\Exp}{\mathbf{E}} % expectation
\DeclareMathOperator{\Cov}{Cov}         % covariance
\DeclareMathOperator{\Var}{Var}         % variance
\DeclareMathOperator{\Corr}{Corr}       % correlation
\DeclareMathOperator{\nnz}{nnz}       % number of non zeros.
%\DeclareMathOperator{\Prob}{Prob}
\newcommand{\Prob}[1]{\mathbb{P}[#1]}
\DeclareMathOperator{\Null}{Null}  % nullpsace
\DeclareMathOperator{\Range}{Range}     % range
\DeclareMathOperator{\Image}{Im}        % image

% functions and operators
\DeclareMathOperator{\sign}{sign}     % signum/sign of a scalar
\DeclareMathOperator{\dom}{dom}         % domain
\DeclareMathOperator{\epi}{epi}         % epigraph
\DeclareMathOperator{\argmin}{argmin}        % argmin
\DeclareMathOperator{\prox}{prox}       % proximal operator      

% topology
\DeclareMathOperator{\interior}{int}    % interior
\DeclareMathOperator{\ri}{rint}         % relative interior
\DeclareMathOperator{\rint}{rint}       % relative interior
\DeclareMathOperator{\bdry}{bdry}       % boundary
\DeclareMathOperator{\cl}{cl}           % closure

% vectors, matrices
\DeclareMathOperator{\linspan}{span}
\DeclareMathOperator{\linspace}{linspace}
\DeclareMathOperator{\cone}{cone}
\DeclareMathOperator{\traceOp}{tr}           % trace
\DeclareMathOperator{\rank}{rank}       % rank
\DeclareMathOperator{\conv}{conv}       % convex hull
%\DeclareMathOperator{\Diag}{Diag}       % Diag(v) = diagonal matrix with v_i on the diagonal
    % diag(D) = the diagonal vector of matrix D
\DeclareMathOperator{\Arg}{Arg}         % Argument

% operators with parentheses
\newcommand{\diag}[1]{\mathbf{diag}\left( #1\right)}
\providecommand{\null}[1]{{\rm Null}\left( #1\right)}
\providecommand{\range}[1]{{\rm Range}\left( #1\right)}
\providecommand{\span}[1]{{\rm Span}\left\{ #1\right\}}
\providecommand{\trace}[1]{{\rm Trace}\left( #1\right)}
\newcommand{\Tr}[1]{\mathbf{Tr}\left( #1\right)}

\newcommand{\Exp}[1]{{{\rm E}}[#1] }    % expectation
%\newcommand{\inner}[1]{\langle#1\rangle}
\newcommand{\E}[1]{\mathbb{E}\left[#1\right] } 
\newcommand{\EE}[2]{\mathbb{E}_{#1}\left[#2\right] } 
\newcommand{\ED}[1]{\mathbb{E}_{\mS\sim \cD}\left[#1\right] } 


%\renewcommand{\qedsymbol}{\ding{114}}
