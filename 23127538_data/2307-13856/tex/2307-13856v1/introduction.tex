The goal of image restoration is to recover high-quality images from degraded observations. The degradation could be due to a variety of factors such as noise, blur, artifacts due to jpeg compression, raindrops, haze, and other factors.  Earlier methods for image restoration \cite{richardson1972bayesian,nonlocalmean,Combettes2004,bm3d,Babacan2012} employed carefully chosen priors and degradation models to derive degradation-specific restoration algorithms. Yet, such methods are limited by the strength of the image prior and the accuracy in modeling or estimating the degradation operator. 
 The past decade saw a  large-scale adoption of deep learning methods to image restoration \cite{su2022survey}, which outperformed the classical approaches. 
 Recent approaches \cite{zamir2022restormer,wang2022uformer,tu2022maxim} successfully adopt novel architectures such as Transformers \cite{vaswani2017attention,dosovitskiy2021an} and MLP-mixers \cite{tolstikhin2021mlp} for restoration.

Yet, CNNs, MLP-mixers as well as Transformer have been shown to be vulnerable to 
carefully crafted adversarial examples \cite{pgd,fgsm}. 
Recent work \cite{agnihotri2023cospgd, estimators_robustness,ijcai2022p211,ga2022deblurring} also confirms the existence of such vulnerabilities in deep learning-based image restoration. Yet, existing works mainly analyze the robustness of CNN-based restoration methods.
Conversely, with the introduction of novel network architectures such as vision Transformers \cite{liu2021swin,dosovitskiy2021an}, MLP mixers \cite{tolstikhin2021mlp}, and improved convolutional architectures \cite{liu2022convnet,bit2020} which outperform the earlier networks such as ResNets \cite{he2016deep}, there have been several studies on the robustness of these new architectures \cite{bhojanapalli2021understanding,shao2022on,tang2021robustart,croce2022interplay, agnihotri2023cospgd}. To the best of our knowledge, very limited works~\cite{croce2022interplay,NEURIPS2021_e19347e1} investigate the effect of architectural components and training recipes.
Existing works focus on image classification and do not study restoration. 
Thus to bridge this gap we investigate the adversarial robustness of recent Transformers specialized to image restoration. 

In this work we study the adversarial robustness of Transformer based restoration networks, Restormer \cite{zamir2022restormer}, and two architectures introduced in \cite{chen2022simple} the \emph{Baseline network} and the \emph{Non-linear Activation free Network~(NAFNet)}, both obtained by simplifying the original Restormer, with modifications to the channel attention and activation functions.
Further, to better understand the architectural design choices made by \cite{chen2022simple}, we include an \emph{Intermediate network} also considered by \cite{chen2022simple} which serves as a step between the Baseline network and NAFNet.
This study is particularly interesting as recent works \cite{xie2020smooth,NEURIPS2021_e19347e1} indicate that the choice of activation function significantly impacts adversarial robustness. 
We study the network robustness under standard and adversarial attacks, by considering $\ell_\infty$ perturbations crafted using PGD attack \cite{pgd} and  CosPGD attack proposed in \cite{agnihotri2023cospgd} for dense prediction tasks. 
We conduct our experiments on dynamic deblurring using the Go-Pro dataset \cite{gopro}.

Our experiments reveal that under standard training settings, Transformer based restoration networks are not robust to adversarial attacks in general.
As shown in Figure~\ref{fig:teaser}, the networks also exhibit distinct artifacts in the reconstructions under attack. 
The images from the Baseline network and the Restormer exhibit severe ringing artifacts~\cite{ringing_artifacts}, whereas the NAFNet reconstructs images with very strong grid and color artifacts under adversarial attack. 
We find that adversarial training can largely reduce the artifacts and significantly improve the robustness of all three networks.
However, the recently proposed NAFNet and Baseline network fail to rival the performance of Restormer, which leads us to contemplate the importance of the architectural components necessary to achieve  robust generalization.

The main contributions of this work can be summarized as follows:
\begin{itemize}
    \item We investigate the robustness of recently proposed Transformer based architectures for image restoration, namely image deblurring.
    \item We analyze the quality of the restored images and the spectral artifacts introduced by models under the aforementioned adversarial attacks.
    \item We understand the effects of defense strategy against adversarial attacks that consequently reduce the spectral artifacts in reconstructed images.
    \item Lastly, we study the effect of certain architectural design choices in the recently proposed \emph{state-of-the-art} image restoration model, NAFNet, to improve their robustness.
\end{itemize}