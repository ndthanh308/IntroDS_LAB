% This is samplepaper.tex, a sample chapter demonstrating the
% LLNCS macro package for Springer Computer Science proceedings;
% Version 2.20 of 2017/10/04
%
\documentclass[runningheads]{llncs}
%
\usepackage{graphicx}
\usepackage[table]{xcolor}
\usepackage{url}
\usepackage{siunitx}

% Used for displaying a sample figure. If possible, figure files should
% be included in EPS format.
%
% If you use the hyperref package, please uncomment the following line
% to display URLs in blue roman font according to Springer's eBook style:
% \renewcommand\UrlFont{\color{blue}\rmfamily}

\usepackage{color}
\newcommand{\bharath}[1]{{\color{red}~{\em Comment by Bharath: #1}}}
\newcommand{\biplav}[1]{{\color{blue}~{\em Comment by Biplav: #1}}}
\usepackage{subcaption}
\newcommand{\vishal}[1]{{\color{magenta}~{\em Comment by Vishal: #1}}}

\newcommand{\mike}[1]{{\color{orange}~{\em Comment by Mike: #1}}}

% -------------------

\begin{document}
%
% \title{{\textbf{\Large PROPEL}}: Building and Using a \underline{P}lanning \underline{R}esource \underline{O}ntology for Planner Selection, \underline{P}erformance Enhancement, and Generating Understandable \underline{E}xp\underline{l}anations}

% \titlerunning{PROPEL}

\title{A Planning Ontology to Represent and Exploit Planning Knowledge for Performance Efficiency}

\titlerunning{Planning Ontology}

% title Options
% Leveraging a Planning Ontology to Boost Planner Performance
% A Planning Ontology to Boost Planner Performance and Simplify Planning Tasks

%

% \author[1]{Bharath Muppasani}
% \author[1]{Vishal Pallagani}
% \author[1]{Biplav Srivastava}
% \address[1]{University of South Carolina, USA}
% %\titlerunning{Abbreviated paper title}
% % If the paper title is too long for the running head, you can set
% % an abbreviated paper title here
% %
% \author[2]{Raghava Mutharaju}

\author{Bharath Muppasani\inst{1} \and
Vishal Pallagani\inst{1} \and
Biplav Srivastava\inst{1} \and
Raghava Mutharaju\inst{2} \and
Michael N. Huhns\inst{1} \and
Vignesh Narayanan\inst{1}
}
%
\authorrunning{B. Muppasani et al.}

% First names are abbreviated in the running head.
% If there are more than two authors, 'et al.' is used.
%
% \institute{Princeton University, Princeton NJ 08544, USA \and
% Springer Heidelberg, Tiergartenstr. 17, 69121 Heidelberg, Germany
% \email{lncs@springer.com}\\
% \url{http://www.springer.com/gp/computer-science/lncs} \and
% ABC Institute, Rupert-Karls-University Heidelberg, Heidelberg, Germany\\
% \email{\{abc,lncs\}@uni-heidelberg.de}}
\institute{University of South Carolina, USA \and IIIT-Delhi, India \\
\email{bharath@email.sc.edu},
\email{vishalp@mailbox.sc.edu},
\email{biplav.s@sc.edu},
\email{raghava.mutharaju@iiitd.ac.in},
\email{huhns@sc.edu},
\email{vignar@sc.edu}
}
%
\maketitle              % typeset the header of the contribution
%
\begin{abstract}


Ontologies are known for their ability to organize rich metadata, support the identification of novel insights via semantic queries, and promote reuse. In this paper, we consider the problem of automated planning, where the objective is to find a sequence of actions that will move an agent from an initial state of the world to a desired goal state. We hypothesize that given a large number of available planners and diverse planning domains; they carry essential information that can be leveraged to identify suitable planners and improve their performance for a domain. We use data on planning domains and planners from the International Planning Competition (IPC) to construct a planning ontology and demonstrate via experiments in two use cases that the ontology can lead to the selection of promising planners and improving their performance using macros - a form of action ordering constraints extracted from planning ontology. We also make the planning ontology and associated resources available to the community to promote further research. 

    % Automating planning and decision-making tasks is a fundamental goal of artificial intelligence (AI) research. The vast number of available planners and diverse planning domains carry essential information that can be leveraged to improve planner performance. For instance, by analyzing the performance of different planners on various problem configurations, we can identify which planners excel in particular domains and improve their efficiency.  To address this issue, we propose a novel approach that employs an ontology to represent the characteristics of planning domains and the capabilities of planners. We gather data on planning domains and planners from the International Planning Competition (IPC) to construct the ontology. Our ontology for AI planning captures critical parameters relevant to planning, including the best planners for a given problem configuration, the ranking of relevant heuristics, and macros to aid in solving complex problems. The ontology represents planning resources, enabling their reuse and improving planner performance.
    
    % Automating planning and decision-making tasks has been a long-standing goal in the field of artificial intelligence (AI). However, selecting the best-performing planner for a given planning domain remains a challenge due to the vast number of available planners and the diversity of planning domains. To address this challenge, we propose a novel approach that uses an ontology to represent the features of the planning domain and the capabilities of planners. We use the International Planning Competition (IPC) and planner information to collect data on the domain and planners and use this data to construct our ontological.
    \keywords{Ontology  \and Automated Planning \and Planner Improvement.} \\
    
    \textbf{Resource Type:} Ontology, Knowledge Graph \\
    \textbf{Licence:} Creative Commons Attribution 4.0 License \\
    \textbf{URL:} \url{https://github.com/BharathMuppasani/AI-Planning-Ontology}
    
    
\end{abstract}
%
%
%
\section{Introduction}
Current quantum hardware is unable to carry out universal quantum computations due to the buildup of errors that occur during the computation. 
The magnitude of the individual error is currently above the value that the Threshold Theorem requires in order to kick-start quantum error correction and fault-tolerant quantum computation~\cite[Section 10.6]{nielsen_chuang_2010}. 
Although the experimentally achieved fidelity rates are promising and the error bounds are inching closer to the required threshold, we will have to work for the foreseeable future with quantum hardware with errors that build-up during the computation.  This implies that we can only do a limited number of steps before the output of the computation has become completely uncorrelated with the intended one.

For fault-tolerant quantum computing, we repeat four steps: 
1) We apply a number of single and two-qubit quantum gates, in parallel whenever possible; 
2) We perform a syndrome measurement on a subset of the qubits; 
3) We perform fast classical computations to determine which errors have occurred and how to correct them; 
and, 4) We apply correction terms based on the classical computations.
We then repeat these four steps with a next sequence of gates. 
These four steps are essential to fault-tolerant quantum computing. 


The starting point of this work is to use the four steps outlined above, not to carry out error correction and fault-tolerant computation, but to enhance short, constant-depth, {\em uncorrected} quantum circuits that perform single qubit gates and {\em nearest-neighbor} two qubit gates. 
Since in the long run we will have to implement error-correction and fault-tolerant computation anyhow, and this is done by such a four-step process, why not make other use of this architecture? Moreover, on some of the quantum hardware platforms, these operations are already in place.
Embracing this idea we naturally arrive at the question: what is the computational power of \textit{low-depth} quantum-classical circuits organized as in the four steps outlined above? 
We thus investigate circuits that execute a small, ideally constant, number of stages, where at each stage we may apply, in parallel, single qubit gates and {\em nearest-neighbor} two qubit gates, followed by measurements, followed by low-depth classical computations of which the outcome can control quantum gates in later stages. 
It is not clear, at first, whether such circuits, especially with constant depth, can do anything remotely useful. 
But we will see that this is indeed the case: many quantum computations can be done by such circuits in constant depth. 
By parallelizing quantum computations in this way, we improve the overall computational capabilities of these circuits, as we do not incur errors on qubits that are idle, simply because qubits are not idle for a very long time. 
Furthermore, reducing the depth of quantum circuits, at the cost of increasing width, allows the circuit to be run faster even if errors occur.

The first usage of such a four-step layout, not to do error correction, but to perform computations, can be found in the paradigm of measurement-based quantum computing~\cite{gottesman1999demonstrating,raussendorf2001one,jozsa2006introduction,clark2007generalised}: 
A universal form of quantum computing where a quantum state is prepared and operations are performed by measuring qubits in different bases, depending on previous measurements and intermediate measurements.

\citeauthor{PhamSvore2013} were the first to formalize the four-step protocol for performing computations~\cite{PhamSvore2013}. They included specific hardware topologies by considering two-dimensional graphs for imposing constraints on qubit interactions. In their model, they develop circuits for particularly useful multi-qubit gates, including specifying costs in the width, number of qubits, depth, number of concurrent time steps, size, and total number of non-Identity operations.
As a result, they find an algorithm that factors integers in polylogarithmic depth.
\citeauthor{Browne:2011} showed that the main tool in the work by \citeauthor{PhamSvore2013}, the fan-out gate, can also be replaced by additional log-depth classical computations in the measurement-based quantum computing setting~\cite{Browne:2011}.

More recently, \citeauthor{Cirac:2021} introduced a scheme to implement unitary operations involving quantum circuits combined with Local Operations and Classical Communication ($\mathsf{LOCC}$) channels: $\mathsf{LOCC}$-assisted quantum circuits~\cite{Cirac:2021}. Similarly to the four-step scheme we just described, they allow for a short depth circuit to be run on the qubits, followed by one round of $\mathsf{LOCC}$, in which ancilla qubits are measured and local unitaries are applied based on the measurement outcomes. They show that in this model any 1D transitionally invariant matrix-product state (MPS) with fixed bond dimension is in the same phase of matter as the trivial state. Similar ideas can be found in~\cite{TVV_NonAbelianTopologicalOrder_2022, tantivasadakarn2021long}.

In this work, we introduce a new model, called \textit{Local Alternating Quantum-Classical Computations} ($\LAQCC$). In this model we alternate between running quantum circuits (constrained by locality), ending in the measurement of a subset of qubits, and fast classical computations based on the measurement results. The outcome of the classical computations are then used to control future quantum circuits. We allow for flexibility in this model, by giving different constraints to the power of both the quantum circuits and the classical circuits as well as the number of alternations between them. 
Most attention will be given to $\LAQCC$ containing quantum circuits of constant depth, classical circuits of logarithmic depth and at most a constant number of alternations between them. 
Any circuit constructed in this model is considered to be of constant depth. 
We restrict ourselves to logarithmic depth classical computations, as this is the first natural and non-trivial extension beyond constant-depth classical computations. 
Constant-depth classical computations do however also have an equivalent constant-depth quantum implementation.

The definition of $\LAQCC$ sharpens the original definition of \citeauthor{PhamSvore2013} by adding constraints to the intermediate classical computations. This allows us to bound the power of $\LAQCC$ from above. 

The main result of \citeauthor{Cirac:2021}, that 1D translational invariant MPS with fixed bond dimension can be prepared by $\mathsf{LOCC}$-assisted circuits, relies on local symmetries of the MPS. These symmetries allow them to prepare local states (on a constant number of qubits) and glue them together by doing one round of the appropriate entangling measurement and corrections, after which they run a round of local unitaries to get the desired result. This general scheme for preparing states that exhibit an MPS description with the appropriate local symmetries requires only geometrically local unitaries and one round of measurement and corrections an therefore is accessible in $\LAQCC$. Studying different local symmetries, known as Symmetry Protected Topological (SPT) phases of matter, to find measurement-based constant depth circuits for states is a broad ongoing field of research~\cite{TVV_NonAbelianTopologicalOrder_2022, tantivasadakarn2021long, smith2023deterministic}. 
All these schemes have a $\LAQCC$ implementation.

%$\LAQCC$-circuits also exist for general schemes of preparing local states, based on the local tensors, and gluing them together using one round of entangled measurement and corrections, based on the local symmetry. 
%The main result of \citeauthor{Cirac:2021}, that 1D translational invariant MPS with fixed bond dimension can be prepared by $\mathsf{LOCC}$-assisted circuits, relies heavily on local symmetries of the MPS and as a result also has an equivalent $\LAQCC$ implementation. 
%The corrections applied after the measurement round are local unitaries depending on the local symmetries of the MPS. 

 

%This general scheme of preparing local states, based on the local tensors, and gluing it together by doing one round of entangled measurement and corrections, based on the local symmetry, is accessible in $\LAQCC$.
Note however that \citeauthor{Cirac:2021} also suggest a circuit for the $W$-state.
This circuit uses sequentially and dependent measurement-based corrections of the ancilla qubits. 
These dependent measurements translate to sequential alternations between the quantum and classical circuits and therefore increase the total depth to linear depth, exceeding the constant-depth constraints imposed by $\LAQCC$-circuits. 

We study the power of the $\LAQCC$ model with respect to state preparation, showing that even with only constant quantum-depth and logarithmic classical depth it remains possible to prepare states with long-range entanglement.
Another surprising result is that it is unlikely that $\LAQCC$ circuits are classically simulatable. We show that any instantaneous quantum polynomial-time (IQP) circuit~\cite{Bremner2010,Shepherd2009} has an $\LAQCC$ implementation.
Classical simulation of IQP circuits implies the collapse of the polynomial hierarchy to the third level, which is not believed to be true~\cite{Bremner2017}. Therefore, we expect that $\LAQCC$ circuits are unlikely to be classically simulatable. We bound the power of $\LAQCC$ by showing that it is contained in $\QNC^1$, the class of polynomial-size, log-depth circuits.

Next, we also study the power that intermediate classical calculations can add to quantum computations, by considering a new model that alternates between polynomially many polynomial-depth quantum circuits and unbounded classical computations
We study this model by doing a complexity theoretical analysis, where we draw inspiration from the notions of complexity given by \citeauthor{RosenthalYuen:2022}, \citeauthor{MetgerYuen:2023}, and \citeauthor{Aaronson:2004}.
All three complexity notions are based on the notion of state preparation, instead of more traditional definition of complexity such as the decidability of a computational problem. 
The first two consider classes based on sequences of quantum states preparable by a polynomial-sized quantum circuit, where the circuits are uniformly generated by a computational class, for instance, the class $\mathsf{PSPACE}$, which results in the complexity class $\mathsf{StatePSPACE}$~\cite{RosenthalYuen:2022,MetgerYuen:2023}.
The third notion considers a relative complexity, where the complexity is measured between two given states, and is measured by the number of gates, from a given gate-set, required to transform one state in another state~\cite{Aaronson:2004}. 
For our definition of state preparation complexity, we drop the uniformity constraint from~\cite{RosenthalYuen:2022,MetgerYuen:2023} and define a class as $\mathsf{StateX}$, which refers to states preparable by circuits of type $\mathsf{X}$. 
As an example, if $\mathsf{X} = \QNC^0$, this results in the class $\mathsf{StateQNC^0}$, which is the set of states preparable from the $\ket{0}^n$ state by poly-size constant-depth circuits. 
This notion is similar to the relative complexity from~\cite{Aaronson:2004}, where one state is the  $\ket{0}^n$ state and instead of counting the number of gates we consider the set of states preparable by a fixed number of gates. Using this notion of complexity we show that any state preparable by an $\LAQCC^*$ circuit is also preparable by a $\mathsf{PostQPoly}$ circuit, the class of circuits of polynomial depth with an additional post-selection gate. 

All Clifford circuits have a constant-depth $\LAQCC$ implementation, implying that any stabilizer state can be implemented by a constant-depth $\LAQCC$ circuit, see Section~\ref{sec:clifford_circuits} for a proof of this statement. 
Efficient circuits for stabilizer states have been known already through measurement-based quantum computing. Therefore this paper focuses on the preparation of non-stabilizer states, and as a surprising result we find novel constant-depth protocols for four very natural classes of non-stabilizer states.
Despite the extensive research into these four classes of non-stabilizer states and the many applications of them, no efficient constant- or low-depth state preparation protocols are known yet. We specifically consider these four classes as they are all often used as initial states in other algorithms.

The first state is a uniform superposition over an arbitrary number of states. 
This state finds applications in many quantum algorithms, as they often start with a uniform superposition over multiple states. 
This superposition is often achieved by applying Hadamard gates to every qubit due to its simplicity to prepare. 
Yet, the analysis of many algorithms, such as Shor's algorithm~\cite{Shor:1997}, would benefit from a different initial superposition. 
The circuit to prepare the uniform superposition over an arbitrary number of states uses an exact version of Grover search as a subroutine, that turns a probabilistic circuit, with a known constant probability of success, into a deterministic circuit. 
We use the circuit for preparing a uniform superposition over an arbitrary number of states as a subroutine in the next two quantum state preparation protocols. 

The second state is the $W$-state, the uniform superposition over all computational basis states of Hamming-weight~$1$, a natural long-ranged entangled state that displays a fundamentally nonequivalent type of entanglement from the Greenberger–Horne–Zeilinger state~\cite{WState:2000}, for which $\LAQCC$-type constant-depth circuits were previously known~\cite{PhamSvore2013, Cirac:2021}. 
The $W$-state is often used as benchmark for new quantum hardware~\cite{Haffner2005,Neeley2010,GarciaPerez:2021}. 
A novel way to prepare the $W$-state therefore gives a new way to benchmark different quantum devices with each other. 
A circuit for preparing the $W$-state was given in~\cite{Cirac:2021}, but this implementation requires sequentially alternating measurements followed by local unitaries, which in the $\LAQCC$ model is not considered to be of constant depth. 
We improve this protocol by giving an $\LAQCC$ implementation of the $W$-state, based on a compress-uncompress method that links the one-hot and binary encoding of integers.

The third state considered is the Dicke state, a generalization of the $W$-state, a superposition over all computational basis states with Hamming-weight $k$~\cite{Dicke:1954}. 
Dicke states have relevance in various practical settings.
For instance, for quantum game theory~\cite{zdemir2007}, quantum storage~\cite{Bacon_Compress:2006,Plesch:2010}, quantum error correction~\cite{ouyang2014permutation}, quantum metrology~\cite{toth2012multipartite}, and quantum networking~\cite{prevedel2009experimental}. 
Dicke states have been used as a starting state for variational optimization algorithms, most notably Quantum Alternating Operator Ansatz (QAOA)~\cite{Hadfield2019}, to find solutions to problems such as Maximum k-vertex Cover~\cite{Brandhofer2022,cook2020quantum}.
The ground states of physical Hamiltonians describing one-dimensional chains tend to show a resemblance to Dicke states such as states resulting from the Bethe ansatz, making them an ideal starting state when investigating the ground state behavior of these Hamiltonians~\cite{TDL_BetheAnsatzDerivation:2010,B_ExcitedStateQuantumPhaseTransitions:2013,DickeTransitions:2021}. 
For instance, the algorithm by \citeauthor{van2021preparing}, who give an algorithm to prepare the Bethe ansatz eigenstates of the spin-1/2 XXZ spin chain, starts by first preparing a Dicke state~\cite{van2021preparing}. 
A Dicke-state preparation protocol based on the compress-uncompress methodology used in the $W$-state furthermore finds applications in entanglement distillation, where the entanglement of a large state is concentrated on only a few qubits. 
Efficient deterministic circuits for preparing Dicke states have been proposed by \citeauthor{bartschi2019deterministic}~\cite{bartschi2019deterministic, bartschi2022deterministic_short_depth}. 
They provide a quantum circuit of depth $\mathO(k \log(\frac{n}{k}))$, allowing arbitrary connectivity, to prepare a Dicke state, which they conjecture to be optimal when $k$ is constant. 
In this work, we provide a constant-depth $\LAQCC$ circuit below their conjectured bound already for constant $k$. 
However, this does not directly disprove their conjecture, as we allow for intermediate measurements and classical computations. 
More significantly, we even construct constant-depth $\LAQCC$ circuits for $k = \mathO(\sqrt{n})$ greatly improving their bound.
This construction extends the compress-uncompress method for the $W$-state combined with additional subroutines. 

We continue with a log-depth state preparation protocol for the Dicke-state for arbitrary $k$. 
This protocol implements an efficient transformation between the factoradic number representation and the combinatorial number representation of a positive integer. 
The combinatorial number representation relates directly to the Dicke state. 
The provided efficient transformation between number representation systems might be of independent interest. 

We conclude by modifying our protocol for preparing a Dicke-state to a protocol that prepares quantum many-body scar states in constant-depth. 
These states have low entanglement and longer coherence times than states with similar energy density.
These characteristics make many-body scar states interesting to analyze and relevant within physics.
Many-body scar states appear for instance in the AKLT model~\cite{AKLT:1987,MRBAR:2018,MRB:2018} and different spin models~\cite{SI:2019,MOBFR:2020}.
Known methods for preparing these states have polynomial-depth~\cite{Gustafson:2023}, whereas our circuit has constant depth. 

% We conclude by studying the power that intermediate classical calculations can add to quantum computations. 
% In this study, we define a new model that relaxes constant-depth quantum circuits to polynomial depth quantum circuits, log-depth classical calculations to unbounded classical computations and a constant number of alternations to a polynomial number of alternations. 
% We call this model $\LAQCC^*$. 
% We study this model by doing a complexity theoretical analysis, where we draw inspiration from the notions of complexity given by \citeauthor{RosenthalYuen:2022}, \citeauthor{MetgerYuen:2023}, and \citeauthor{Aaronson:2004}.
% All three complexity notions are based on the notion of state preparation, instead of more traditional definition of complexity such as the decidability of a computational problem. 
% The first two consider classes based on sequences of quantum states preparable by a polynomial-sized quantum circuit, where the circuits are uniformly generated by a computational class, for instance, the class $\mathsf{PSPACE}$, which results in the complexity class $\mathsf{StatePSPACE}$~\cite{RosenthalYuen:2022,MetgerYuen:2023}.
% The third notion considers a relative complexity, where the complexity is measured between two given states, and is measured by the number of gates, from a given gate-set, required to transform one state in another state~\cite{Aaronson:2004}. 
% For our definition of state preparation complexity, we drop the uniformity constraint from~\cite{RosenthalYuen:2022,MetgerYuen:2023} and define a class as $\mathsf{StateX}$, which refers to states preparable by circuits of type $\mathsf{X}$. 
% As an example, if $\mathsf{X} = \QNC^0$, this results in the class $\mathsf{StateQNC^0}$, which is the set of states preparable from the $\ket{0}^n$ state by poly-size constant-depth circuits. 
% This notion is similar to the relative complexity from~\cite{Aaronson:2004}, where one state is the  $\ket{0}^n$ state and instead of counting the number of gates we consider the set of states preparable by a fixed number of gates. Using this notion of complexity we show that any state preparable by an $\LAQCC^*$ circuit is also preparable by a $\mathsf{PostQPoly}$ circuit, the class of circuits of polynomial depth with an additional post-selection gate. 

\paragraph{Summary of results}
\begin{itemize}
    \item We give a new definition of a computational model that captures the power of the four step process: applying a constant number of layers of one- and two-qubit gates; performing a syndrome measurement; perform a fast classical computation determining corrections; apply corrections. We call this model \emph{Local Alternating Quantum Classical Computations}, or $\LAQCC$ for short. In this model we bound the allowed quantum operations, intermediate classical calculations, and number of rounds separately. In Section~\ref{sec:LAQCC_model} we define this model and give a list of operations based on results from literature contained in this computational model. In some of these operations we explicitly use that we allow for multiple, but at most constant, rounds  of corrections.
    \item  We show show that there exist $\LAQCC$ circuits that can not be weakly simulated in Section~\ref{sec:IQP_in_LAQCC}. We further show that for every $\LAQCC$ circuit there exists a $\QNC^1$ circuit simulating it perfectly, in Section~\ref{sec:LAQCC_in_QNC1}.
    \item We introduce a new type computational complexity for preparing states and show that the extension of $\LAQCC$ where we allow a polynomial number of rounds and unbounded classical computation, is contained in $\mathsf{PostQPoly}$, the class of polynomial circuits with post-selection, in Section~\ref{sec:Complexity results}.
    \item We show a protocol to prepare the uniform superposition state of size $q$ in $\LAQCC$ using $\mathO(\ceil{\log_2(q)}^2)$ qubits in Section~\ref{sec:superposition_modulo_q}. 
    \item We show a protocol to prepare the $W_n$ state in $\LAQCC$ using $\mathO(n\log(n))$ qubits in Section~\ref{sec:W_state_in_LAQCC}.
    \item We show two ways of preparing the Dicke-$(n,k)$ state. The first method is in $\LAQCC$, works up to $k = \mathO(\sqrt{n})$, uses $\mathO(n^2\log(n))$ qubits, and is found in Section~\ref{sec:dicke:small_k}. The second method is in $\LAQCC\text{-}\mathsf{LOG}$ (an extension of $\LAQCC$ allowing for logarithmic number of alterations instead of constant), works for any $k$, uses $\mathO(\text{poly}(n))$ qubits, and is found in Section~\ref{sec:Dicke_in_LAQCC_LOG}. 
    \item We extend on our $\LAQCC$ method of generating Dicke-$(n,k)$ states for $k = \mathO(\sqrt{n})$ and show a protocol to generate many-body scar states for a particular Hamiltonian in $\LAQCC$ (Section~\ref{sec:many_body_scar}). 
\end{itemize}
Summarized in a table, we provide the following state generation protocols:
\begin{table}[htb]
\centering
\begin{tabular}{l|l|l|l}
\textbf{State description} & \textbf{Width} & \textbf{Depth} & \textbf{Implementation}\\
\hline 
Uniform superposition mod $q$: $\frac{1}{\sqrt{q}} \sum_{i = 0}^{q-1}\ket{i}$ & $\mathO(\ceil{\log^2 q})$ & $\mathO(1)$ & Section~\ref{sec:superposition_modulo_q}\\

$W$-state: $\frac{1}{\sqrt{n}}\sum_{i = 0}^{n-1}\ket{e_i}$ & $\mathO(n \log n)$ & $\mathO(1)$ & Section~\ref{sec:W_state_in_LAQCC}\\

Dicke-$(n,k)$, $k = \mathO(\sqrt{n})$: $\binom{n}{k}^{-1/2}\sum_{x \in \{0,1\}^n: |x| = k} \ket{x}$ &  $\mathO(n^2\log n)$ & $\mathO(1)$ 
&Section~\ref{sec:dicke:small_k}\\

Dicke-$(n,k)$: $\binom{n}{k}^{-1/2}\sum_{x \in \{0,1\}^n: |x| = k} \ket{x}$ & $\mathO(\text{poly}(n))$ & $\mathO(\log n)$ &Section~\ref{sec:Dicke_in_LAQCC_LOG}\\

QMBS: $\ket{S_k} = \frac{1}{k! \sqrt{\mathcal N(n,k)}}(Q^\dagger)^k \ket{\Omega}$ &  $\mathO(n^2\log n)$ & $\mathO(1)$  &  Section~\ref{sec:many_body_scar}
\end{tabular}
\caption{Summary of state preparation protocols given in this paper.}
\label{tab:sate_prep}
\end{table}
In the entry for the quantum many-body scar state $Q$ denotes the raising operator and $\mathcal N(n,k)=\binom{n-k-1}{k}$. 
Section~\ref{sec:many_body_scar} will provide more details on the variables and the implementation. 

\paragraph{Organization of the paper}
\noindent We first introduce relevant preliminaries in Section~\ref{sec:preliminaries}. 
In Section~\ref{sec:LAQCC_model} we formally define the class of Local Alternating Quantum-Classical Computations ($\LAQCC$). We also show that any Clifford circuit can be implemented in constant depth $\LAQCC$ (a result based on a result from measurement-based quantum computing~\cite{jozsa2006introduction}). 
This result allows us to give many useful multi-qubit gates and routines in Section~\ref{sec:gates_created_in_LAQCC}. 
Beyond that we show that constant depth $\LAQCC$ circuits are contained in $\QNC^1$ and that any $\mathsf{IQP}$ circuit has an $\LAQCC$ implementation.
We conclude this section with an analysis of a more powerful instantiation of $\LAQCC$ and show an inclusion with respect to the class $\mathsf{PostQPoly}$, which is the class of circuits of polynomial depth with one additional post-selection gate. 
In Section~\ref{sec:state_prep_in_LAQCC} we give $\LAQCC$ circuit implementations for preparing the uniform superposition over an arbitrary number of states, the $W$-state and the Dicke state up to $k = \mathO(\sqrt{n})$. We furthermore give a log-depth circuit implementation for preparing the Dicke state for any $k$. We conclude by showing a $\LAQCC$ circuit for generating many body scar states of a particular type of Hamiltonian.


%%%%%%%%%%%%%%%%%%%%%%%%%%%%%%%%%%%%%%%%%%%%%%%%%%%%%%
\section{Preliminaries}
In this section, we describe the necessary background for automated planning and the significance of the International Planning Competition. 

% \subsection{Ontology}
% A formal ontology is typically represented as a set of concepts, relations, and axioms. A concept represents a set of objects or entities that share common properties, while a relation represents a connection or association between two or more concepts. Axioms are statements that define the relationships between concepts and relations. It is a formal representation of knowledge that is designed to facilitate automated reasoning and information processing. It acts as a structured vocabulary that describes a domain and promotes interoperability, data integration, and communication between humans and machines. Formally, an ontology $O$ can be represented as a tuple $(C, R, A)$, where $C$ is the set of concepts, $R$ is the set of relations, and $A$ is the set of axioms. Each concept \textit{c} $\in$ $C$ can be represented as a set of attributes, denoted as $Att(c)$. Similarly, each relation \textit{r} $\in$ $R$ can be represented as a set of attributes, denoted as $Att(r)$.

% Ontology is a branch of philosophy that deals with the nature of existence and being. In the field of computer science, however, ontology refers to a formal representation of knowledge that is designed to facilitate automated reasoning and information processing. It is a structured vocabulary that describes a domain and promotes interoperability, data integration, and communication between humans and machines. Various tools and methodologies, including Protege and ontology editors, are available for ontology creation. Ontologies are increasingly important in artificial intelligence, knowledge engineering, and the semantic web, and researchers are exploring their potential in diverse domains and applications.

% Figure environment removed

\subsection{Automated Planning}

Automated planning, also known as AI planning, is the process of finding a sequence of actions that will transform an initial state of the world into a desired goal state \cite{ghallab2004automated}. It involves constructing a plan or a sequence of actions that will achieve a specified objective while respecting any constraints or limitations that may be present. Formally, automated planning can be defined as a tuple $(S, A, T, I, G)$, where:
\begin{itemize}
    \item $S$ is the set of possible states of the world
    \item $A$ is the set of possible actions that can be taken
    \item $T$ is the transition function that describes the effects of taking an action on the current state of the world
    \item $I$ is the initial state of the world
    \item $G$ is the desired goal state
\end{itemize}
Using this notation, the problem of automated planning can be framed as finding a sequence of actions $\prec a_1, a_2, ..., a_k\succ$ that will transform the initial state $I$ into the goal state $G$, while respecting any constraints or limitations on the actions. 
 % In automated planning, 
 A problem is defined in terms of a domain and a problem instance. The domain defines the possible actions that can be taken and the effects of each action, while the problem instance specifies the initial state of the world and the desired goal state. 
Various techniques can be used to solve the planning problem, such as search algorithms, constraint-based reasoning, and optimization methods. These techniques involve exploring the space of possible plans and selecting the one that satisfies the objective and any constraints. Figure \ref{fig:planning_bw} illustrates an automated planning scenario for the blocksworld domain, where an initial state can be transformed into a goal state by executing a sequence of actions.

% \noindent \textbf{Attributes modeled about a domain.}
%   %\noindent \textbf{Attributes modeled in a domain file}
%  \begin{enumerate}
%      \item \textbf{Requirements:} A list of requirements that the planner must satisfy in order to solve the domain. Requirements include durative actions, conditional effects, or negative preconditions. For example, in blocksworld domain with types involved, one of the requirements is \emph{typing}.
%     \item \textbf{Predicates:} Predicates are fundamental elements in the planning domain that define the properties of the world. They are used to describe the initial and goal states, as well as the preconditions and effects of actions. Predicates are usually defined as logical expressions over a set of variables, where each variable can take on a finite number of values. In the context of planning, predicates are typically used to represent facts about the world that can be true or false, such as the location of an object or the status of a machine. For example, in blocksworld domain, the predicate \verb|(on b1 b2)| could indicate that block 'b2' is on top of block 'b1'.
%      \item \textbf{Actions:} Actions are the basic units of change in the planning domain. They represent atomic operations that can be performed to transform the world from one state to another. Each action has a name, a set of parameters, preconditions that must be satisfied before the action can be executed, and effects that describe the changes that the action makes to the world. Actions can be used to model a wide variety of operations, ranging from simple movements or transformations to complex processes such as planning or decision-making. For example, in blocksworld domain, the action \verb|unstack b2 b1| can be used to unstack block 'b2' from block 'b1'. 
     
%      \item \textbf{Preconditions:} Preconditions are the conditions that must be true before an action can be executed. They are usually defined using predicates and can involve multiple variables. Preconditions can also be negative, which means that a certain condition must not be true for an action to be executed. In planning, preconditions ensure that actions are only executed when the necessary conditions have been met, such as ensuring that a machine is turned off before it is serviced. For example, in blocksworld domain, the action \verb|unstack b2 b1| has a precondition of \verb|(on b1 b2)|, meaning that for the action to be valid, the block 'b2' should be on top of block 'b1'.
     
%      \item \textbf{Effects:} Effects describe the changes that an action makes to the world. They are usually defined using predicates and can involve multiple variables. Effects can be positive, which means that a certain condition becomes true after the action is executed, or negative, which means that a certain condition becomes false after the action is executed. In the context of planning, effects are used to model the changes that result from executing an action, such as moving an object from one location to another or turning a machine on. For example, in blocksworld domain, when the action \verb|unstack b2 b1| is executed, one of its effect is \verb|(not (on b1 b2))|, indicating that block 'b2' is no longer on top of block 'b1'.
     
%      \item \textbf{Constants:} Constants are values that are fixed and do not change during the execution of the planning problem. They are used to represent objects or entities in the world that have a fixed value, such as the speed limit on a road. Constants can be used to simplify the planning problem by reducing the number of variables that need to be considered and by providing a fixed set of values that can be used in predicates and actions. For example, in blocksworld domain, the constant \emph{table} could represent the surface on which the blocks are initially placed.
     
%      \item \textbf{Types:} Types are used to classify objects or entities in the world based on their attributes or properties. They are used to define the domain of values that a variable can take on and can be used to constrain the values that are assigned to variables. In the context of planning, types are typically used to group related objects or entities together, such as cars or bicycles, and to specify the properties that are common to all members of a type, such as their color or size. For example, in blocksworld domain with types involved, one can represent the predicate as \verb|(on ?x - block ?y - block)| stating that the parameters in the predicate are of type \emph{block}.

%  \end{enumerate}


% ######### Shorter version for AI Planning preliminaries
% \subsection{Automated Planning}

% Automated planning, also known as AI planning, finds actions transforming an initial world state into a goal state \cite{ghallab2004automated}. It involves creating a plan, respecting constraints, defined as $(S, A, T, I, G)$ where $S$ is the world states set, $A$ is the actions set, $T$ is the state transition function, $I$ is the initial state, and $G$ is the goal state. The challenge is to find actions $\prec a_1, a_2, ..., a_k\succ$ converting $I$ to $G$ under constraints. 

% A problem has a domain (defining actions and effects) and an instance (specifying initial and goal states). Various techniques can be used to solve the planning problem, such as search algorithms, constraint-based reasoning, and optimization methods. These techniques involve exploring the space of possible plans and selecting the one that satisfies the objective and any constraints. Figure \ref{fig:planning_bw} illustrates an automated planning scenario for the blocksworld domain, where an initial state can be transformed into a goal state by executing a sequence of actions.

\noindent \textbf{Attributes modeled about a domain.}
 \begin{enumerate}
     \item \textbf{Requirements:} A list of requirements that the planner must satisfy to solve the given domain, e.g., \emph{typing} in blocksworld with types.
     \item \textbf{Predicates:} Define world properties, e.g., \verb|(on b1 b2)| in blocksworld.
     \item \textbf{Actions:} Units of change with preconditions and effects, e.g., \verb|unstack b2 b1| in blocksworld.
     \item \textbf{Preconditions:} Conditions for action execution, e.g., \verb|(on b1 b2)| for \\ \verb|unstack b2 b1|.
     \item \textbf{Effects:} Post-action world changes, e.g., \verb|(not (on b1 b2))| after \\ \verb|unstack b2 b1|.
     \item \textbf{Constants:} Fixed values, e.g., \emph{table} in blocksworld.
     \item \textbf{Types:} Classifications based on attributes, e.g., \\ \verb|(on ?x - block ?y - block)| in typed blocksworld.
 \end{enumerate}

\noindent \textbf{Attributes modeled about a problem instance from a domain.}
\begin{enumerate}
    \item \textbf{Name:} The name of the planning problem.
    \item \textbf{Domain:} The name of the planning domain that the problem belongs to.
    \item \textbf{Objects:} A list of objects that are present in the planning problem. Objects are typically defined in terms of their type and name. In the example shown in Figure \ref{fig:planning_bw}, objects are b1, b2, and b3.
    \item \textbf{Initial State:} A description of the initial state of the world, including the values of all relevant predicates. Figure \ref{fig:planning_bw} represents an example initial state.
    \item \textbf{Goal State:} A description of the desired goal state of the world, including the values of all relevant predicates. Figure \ref{fig:planning_bw} represents an example goal state.
\end{enumerate}

% \vspace{2cm}
\subsection{International Planning Competition (IPC)}

% IPC serves as a significant means of assessing and comparing various planning systems. By presenting new planners and benchmark problems each year, the competitions aim to stimulate the advancement of new planning methodologies and reflect current trends and challenges in the field. The competition comprises multiple tracks, each covering various planning problems such as classical, temporal, and probabilistic planning. These tracks include benchmark problems that evaluate the performance of planners concerning parameters such as plan quality, plan length, and run time. The results of these competitions provide insights into the current state-of-the-art in planning and help identify the strengths and weaknesses of different planning systems. IPC can serve as an excellent starting point for building a planning-related ontology as the benchmark problems used in these competitions can provide a comprehensive overview of the domain and the types of problems that planners need to solve. 

IPC is pivotal for evaluating and contrasting planning systems. Introducing new planners and benchmarks, it promotes innovative planning methodologies and reflects the field's evolving challenges. The competition has multiple tracks, such as classical and probabilistic planning, with benchmarks assessing plan quality, length, and run time. IPC results offer a glimpse into the latest in planning, highlighting system pros and cons. The benchmarks from IPC are ideal for crafting a planning-related ontology, encapsulating the domain's breadth and planners' challenges.


\section{Related Work}
%\subsection{Cost Volume based Deep Stereo Matching}
%Stereo matching is a typical problem that has been studied for decades and a well-known four-step pipeline \cite{scharstein2002taxonomy} has been established, where cost volume construction is an indispensable step. Current state-of-the-art stereo matching methods are all cost volume based methods and they can be categorized into two types. Typically, a cost volume is a 4D tensor of height, width, disparity, and features. The first category just uses a full correlation to generate a single-feature cost volume. Such methods are usually efficient but lose much information because of the decimation of feature channels. Many previous work, including Dispnet \cite{dispnet}, MADNet \cite{madnet}, IResNet \cite{iresnet} and AANet \cite{aanet}, belong to this category. The second category usually uses concatenation \cite{gcnet} or group-wise correlation \cite{gwcnet} to generate a multi-feature 4D cost volume. Such a method can achieve better performance while requiring higher computational complexity and memory consumption. Actually, a majority of the top-performing networks in public leaderboards belong to this category, such as GANet \cite{ganet}, CSPN \cite{cspn} and ACFNet \cite{acfnet}. These methods generally employ multiple 3D convolution layers to constantly regularize the 4D cost volume and then apply softmax over the disparity dimension to produce a discrete disparity probability distribution. The final predicted disparity is obtained by softly weighting indices according to their probability, which is also called soft argmin in GCNet \cite{gcnet}. However, soft argmin leaves the output susceptible to multi-modal disparity probability distributions. ACFNet \cite{acfnet} observes this problem and proposes to directly supervise the cost volume with unimodal ground truth distributions. In contrast, we define an uncertainty estimation to quantify the degree to which the cost volume tends to be multi-modal distribution, higher implies the higher possibility of estimation error.

\subsection{Multi-scale Cost Volume based Stereo Matching}
Cost volume construction is an indispensable step in the well-known four-step pipeline for stereo matching \cite{scharstein2002taxonomy, pamisurvey1, pamisurvey2}. Typically, current state-of-the-art stereo matching methods can be categorized into two types of cost volume-based methods, where the cost volume is a 4D tensor of height, width, disparity, and features. The first category usually uses the single-feature 3D cost volume generated by full correlation, which is efficient while losing much information due to the decimation of feature channels. Many real-time methods, such as Dispnet \cite{dispnet}, MADNet \cite{madnet, madnet_pami} and AANet \cite{aanet}, belongs to the category. Moreover, two-stage refinement \cite{mcvmfc} and pyramidal towers \cite{madnet} are commonly applied in the single-feature cost volume based network to construct multi-scale cost volume. The second category usually uses the multi-feature 4D cost volume generated by concatenation \cite{gcnet} or group-wise correlation \cite{gwcnet}, which can achieve better performance with higher computational complexity and memory consumption. Most top-performing networks, including GANet \cite{ganet}, CSPN \cite{cspn} and ACFNet \cite{acfnet} belong to this category. 
% In these methods, the 4D cost volume is constantly regularized by multiple 3D convolution layers and then a discrete disparity probability distribution can be produced by softmax. Next, the final predicted disparity can be obtained by softly weighting indices according to their probability \cite{gcnet}. However, such output is susceptible to multimodal disparity probability distributions and ACFNet \cite{acfnet} gives a solution by directly supervising the cost volume with unimodal ground truth distributions to alleviate this problem. 
Recently, to alleviate the high computational complexity and memory consumption when employing multi-feature 4D cost volumes, \cite{cvpmvsnet, cascade, uscnet} propose to use cascade cost volume representation in multi-view stereo. These methods usually first predict an initial disparity at the coarsest resolution of the image and then gradually refine the disparity by narrowing down the disparity search space. More closely related to our approach is Casstereo \cite{cascade}, which first extended such representation to stereo matching. It selected to uniform sample a pre-defined range to generate the next stage’s disparity search range. Instead, we employ pixel-level uncertainty estimation to adaptively adjust the next stage disparity searching range and generate pseudo-labels for subsequent domain adaptation. Our method also shares similarities with UCSNet \cite{uscnet}, which constructs uncertainty-aware cost volume in multi-view stereo while it doesn’t employ uncertainty estimation to generate pseudo-labels.

%\subsection{Multi-scale Cost Volume based Deep Stereo Matching} 
% \subsection{Multi-scale Cost Volume based Stereo Matching} 
%Multi-scale cost volume firstly was applied in the single-feature cost volume based network with the form of two-stage refinement \cite{mcvmfc} and pyramidal towers \cite{madnet}. Recently, cascade cost volume representation \cite{cvpmvsnet, cascade, uscnet} was proposed in multi-view stereo to alleviate the high computational complexity and memory consumption when employing multi-feature 4D cost volumes. These methods generally predict an initial disparity at the coarsest resolution of the image. Then, they will narrow down the disparity search space and gradually refine the disparity. More closely related to our approach is Casstereo \cite{cascade}, which first extended such representation to stereo matching. It selected to uniform sample a pre-defined range to generate the next stage’s disparity search range. Instead, we employ uncertainty estimation to adaptively adjust the next stage pixel-level disparity searching range and push the next stage's cost volume to be predominantly unimodal.

% The single-feature cost volume based network with the form of two-stage refinement \cite{mcvmfc} and pyramidal towers \cite{madnet} first employ multi-scale cost volume for stereo matching. Recently, to alleviate the high computational complexity and memory consumption when employing multi-feature 4D cost volumes, \cite{cvpmvsnet, cascade, uscnet} propose to use cascade cost volume representation in multi-view stereo, which generally predict an initial disparity at the coarsest resolution of the image. Then, the disparity search space is narrowed down and the disparity is gradually refined. More closely related to our approach is Casstereo \cite{cascade}, which first extended such representation to stereo matching. It selected to uniform sample a pre-defined range to generate the next stage’s disparity search range. Instead, we employ uncertainty estimation to adaptively adjust the next stage pixel-level disparity searching range and push the next stage's cost volume to be predominantly unimodal.

% Figure environment removed

\subsection{Robust Stereo Matching} 
There exist three categories of generalization definitions for robust stereo matching. 1) Cross-domain Generalization: the network’s ability to perform well on unseen scenes (cannot see the image pairs of the target domain in advance). Towards this end, Jia et al \cite{sungeneralizaiton} propose to incorporate scene geometry priors into an end-to-end network. Zhang et al \cite{dsmnet} introduce a domain normalization and a trainable non-local graph-based filter to construct a domain-invariant stereo matching network. 2) Adapt Generalization: the network’s ability to adapt pre-trained models to the new domain with unlabeled target data. Previous work usually pre-trains the models on synthetic data and then adapts it to new target domains with Graph Laplacian regularization \cite{zoom}, non-adversarial progressive color transfer \cite{adastereo}, and Knowledge Reverse Distillation \cite{aohnet}. More closely related to our approach are \cite{aohnet, unsuperviseddomainadaptation} in stereo matching and Monoresmatch \cite{monoresmatch} in monocular depth estimation, which also proposes to generate a pseudo-label for domain adaptation. However, these methods all select to employ classical stereo matching methods \cite{sgm} alongside with confidence estimators, e.g., left-right consistency check to generate pseudo-labels. That is all these methods need an independent method to generate corresponding pseudo-labels. Instead, the proposed method is an end-to-end network that can generate the predicted disparity map, corresponding uncertainty map and pseudo-labels jointly, which is a more simple, yet efficient way. 
% Instead, our proposed method can employ pixel-level and area-level uncertainty estimation to self-distill the predicted disparity maps of our pre-training model and generate sparse while reliable pseudo-labels to align the domain gap, which is a more simple, yet efficient way. 
3) Joint Generalization: the network’s ability to perform well on a variety of datasets with the same model parameters. MCV-MFC \cite{mcvmfc} introduces a two-stage finetuning scheme to achieve a good trade-off between generalization and fitting capability on multiple datasets. However, it doesn’t touch the inner difference between diverse datasets, e.g, the unbalanced disparity distribution. To further address this problem, we propose a cascade cost volume to adaptively the next stage disparity searching space, where the pixel-level uncertainty estimation is at the core.

% \subsection{Monocular Depth Estimation}
% Monocular depth estimation aims to estimate depth values from a single image, instead of stereo images or multiple frames in a video. This problem is ill-posed because of the ambiguity of object sizes. However, humans could estimate the depth from a single image with prior knowledge of the scenes. Recently, learning based methods were explored to learn depth values by supervised or unsupervised learning. Eigen et al. first employed Convolutional Neural Networks (CNN) to predict depth in a coarse-to-fine manner and further improved its performance by multi-task learning. Liu et al. presented deep convolutional neural fields model by combining deep model with continuous CRF. Li et al. [22] refined deep CNN outputs with a hierarchical CRF. Multi-scale continuous CRF was formulated into a deep sequential network by Xu et al. [45] to refine depth estimation. Unsupervised methods tried to train monocular depth estimation with stereo
% image pairs or image sequences and test on single images. Garg et al. [9] used novel image view synthesis loss to train a depth estimation network in an unsupervised way. Godard et al. [11] introduced left-right consistency regularization to improve the performance of view synthesis loss. Recently, some work also propose to use the stereo matching network as a proxy to learn depth from synthetic data or directly employ traditional stereo matching methods to distill proxies labels from the target domain, which proves the feasibility of distilling stereo matching networks to learn monocular depth estimation.




%%%%%%%%%%%%%%%%%%%%%%%%%%%%%%%%%%%%%%%%%%%%%%%%%%%%%

\section{Planning Ontology}

% Figure environment removed

% % Figure environment removed

This section covers the construction of planning ontology to capture the essential details of automated planning. We will discuss the considerations, challenges, benefits, and limitations of using ontologies for automated planning, to provide a better understanding of how they can improve the efficiency and effectiveness of automated planning systems.
% \bharath{Scope and Methodology}
% \rmv {Give a name to the ontology. I suggest adding the following subsections. 1) Methodology. An iterative approach suggested in {\url{https://protege.stanford.edu/publications/ontology_development/ontology101.pdf}} can be followed. Did you sit with a domain expert, used literature on planning (add refs) and/or used the data to come up with the ontology? This should justify the terminology that was used in the ontology. 2) Competency Questions. These are the minimum set of questions that the ontology should answer. These are used to restrict the scope of the ontology. 3) Ontology Description. Along with the schema diagram (Fig. 1), list the important classes, properties and axioms. In each case, mention how they can be used. 4) Evaluation. Answer the competency questions using ontology. You can either have a SPARQL query for each CQ or just answer it manually using the classes/properties from the ontology. Discuss a use case/scenario of putting this ontology or part of the ontology to use.}

\subsection{Competency Questions}

Competency questions for an ontology are focused on the needs of the users who will be querying the ontology. These questions are designed to help users explore and understand the concepts and relationships within the ontology, and to find the information they need within the associated knowledge base. By answering these questions, the ontology can be better scoped and tailored to meet the needs of its users. 

We designed the following competency questions to model an Ontology to represent the general aspects of Automated Planning.

\begin{itemize}
    \item C1: What are the different types of planners used in automated planning?
    \item C2: What is the relevance of planners in a given problem domain?
    \item C3: What are the available actions for a given domain?
    \item C4: What problems in a domain satisfy a given condition?
    \item C5: What are all the requirements a given domain has?
    \item C6: What is the cost associated with generating a plan for a given problem?
    \item C7: How many parameters does a specific action have?
    \item C8: What planning type does a specific planner belong to?
    \item C9: What requirements does a given planner support?
    \item C10: What are the different parameter \verb|types| present in a domain?
\end{itemize}

\subsection{Design}
An ontology is a formal and explicit representation of concepts, entities, and their relationships in a particular domain. In this case, ontology is concerned with the domain of automated planning, which refers to the process of generating a sequence of actions to achieve a particular goal within a given set of constraints. The ontology aims to provide a structured framework for organizing and integrating knowledge about this domain, which can be useful in various applications, such as designing planning algorithms, extracting best-performing planners given a domain, or learning domain-specific macros.

Figure \ref{fig:ontology} shows an ontology that aims to encompass the various concepts of automated planning separated into categories of \verb|Domain|, \verb|Problem|, \verb|Plan|, and \verb|Planner|. The ontology for automated planning is composed of 19 distinct classes and 25 object properties. These classes and properties are designed to represent the various elements of the automated planning domain and its associated problems. In the design of our ontology, all axioms are formulated using Description Logic \cite{KSH14:DLintro}, providing a formal and expressive framework for representing and reasoning about the concepts and relationships within our domain.
% \vspace{-0.1cm}

\subsubsection{Domain}
The Domain category in our ontology comprises the characteristics of the AI planning domain through several classes. These include \texttt{PlanningDomain} - \texttt{DomainRequirement}, detailing domain modeling; \texttt{ParameterType}, defining parameter varieties in a typed domain; \texttt{DomainPredicate}, encompassing applicable predicates; \texttt{DomainConstant}, representing invariant constants; and \texttt{Action}, for domain operations. \texttt{Action} class is further linked with \texttt{ActionPrecondition}, \texttt{ActionEffect}, and \texttt{Parameter}. This structured approach aids applications like algorithm design, planner optimization, and macro learning in domain-specific contexts.

The \texttt{PlanningDomain} conceptualization is articulated through axioms to represent fundamental elements of planning scenarios. Axiom~\ref{ax: domain1} signifies that every planning domain entails certain actions. Actions are fundamental to planning as they represent the steps or decisions that can be taken to transform a state within the domain. Predicates are essential for defining the states within a planning domain. Axiom~\ref{ax: domain2} ensures that each domain includes predicates to represent these states, facilitating the definition of preconditions and effects of actions. Axiom~\ref{ax: domain3} states that every planning domain possesses certain defined requirements. Requirements in AI Planning are necessary to define various types of domain modeling, such as conditional effects and numeric fluents. Such specifications are not only essential for characterizing the domain but also serve as a criterion to assess whether a planner is compatible with and can support these specific domain modeling features.

\begin{equation}
\texttt{PlanningDomain} \sqsubseteq \exists\texttt{hasAction}.\texttt{Action}
\label{ax: domain1}
\end{equation}
\begin{equation}
\texttt{PlanningDomain} \sqsubseteq \exists\texttt{hasPredicate}.\texttt{DomainPredicate}
\label{ax: domain2}
\end{equation}
\begin{equation}
\texttt{PlanningDomain} \sqsubseteq \exists\texttt{hasRequirement}.\texttt{DomainRequirement}
\label{ax: domain3}
\end{equation}

The \texttt{Action} class is characterized by its effects, a fundamental aspect of planning. Axiom~\ref{ax: action1} addresses the transformative nature of actions in a planning domain. Understanding the effects of actions is essential for planning algorithms to predict and evaluate the outcomes of different action sequences.
\begin{equation}
\texttt{Action} \sqsubseteq \exists\texttt{hasEffect}.\texttt{ActionEffect}
\label{ax: action1}
\end{equation}

Axioms~\ref{ax: action2} and \ref{ax: action3} capture the dynamics of how actions can add or delete predicates in a state, emphasizing the mutable nature of states within the planning domain. This depiction is essential for accurately modeling the consequences and feasibility of actions in AI Planning.
\begin{equation}
\texttt{ActionEffect} \sqsubseteq \exists\texttt{addsPredicate}.\texttt{State}
\label{ax: action2}
\end{equation}
\begin{equation}
\texttt{ActionEffect} \sqsubseteq \exists\texttt{deletesPredicate}.\texttt{State}
\label{ax: action3}
\end{equation}

\subsubsection{Problem}
The Problem category of the ontology includes classes that represent specific problems within a given domain. These classes are designed to capture the details of a particular problem, such as the \verb|Objects| defined in the problem, which is an instance of different \emph{types} defined in the planning domain, the \verb|Initial State| of the problem, and the \verb|Goal State| which are a subclass of the parent class \verb|State| which is a state description of the given domain. 

The axioms defined for \texttt{PlanningProblem} conceptualized the key aspects of a planning problem. Axiom~\ref{ax: problem1} indicates that each planning problem is defined with a specific \texttt{GoalState}, which is the desired outcome or objective of the problem. Axiom~\ref{ax: problem2} asserts that each planning problem also has a defined \texttt{InitialState}, which provides the starting conditions and context for the planning process. Lastly, Axiom~\ref{ax: problem3} identifies the \texttt{Objects} present within a planning problem, denoting the various entities that are subject to manipulation or consideration during the course of planning. Finally, the axiom~\ref{ax: problem4} underscores that every planning problem includes a potential plan or series of actions that lead to the goal state.

\begin{equation}
\texttt{PlanningProblem} \sqsubseteq =1 \texttt{hasGoalState}.\texttt{GoalState}
\label{ax: problem1}
\end{equation}
\begin{equation}
\texttt{PlanningProblem} \sqsubseteq =1 \texttt{hasInitialState}.\texttt{InitialState}
\label{ax: problem2}
\end{equation}
\begin{equation}
\texttt{PlanningProblem} \sqsubseteq \exists\texttt{hasObject}.\texttt{ProblemObject}
\label{ax: problem3}
\end{equation}
\begin{equation}
\texttt{PlanningProblem} \sqsubseteq \exists\texttt{hasPlan}.\texttt{Plan}
\label{ax: problem4}
\end{equation}

\subsubsection{Plan}
The Plan category of the ontology includes classes that represent the sequence of actions that must be taken to solve a given problem. The \verb|Plan| class captures the knowledge about the plans that planners generate for specific problems. The plan cost for each plan is a data property (non-negative integer) of the \verb|Plan| class. This enables planners to be compared based on the quality of the plans they generate and the cost of those plans.

The axioms defined for the \texttt{Plan} category outline the essential features of plans in the AI planning process. Axiom~\ref{ax: plan1} mandates that each plan must have an associated plan cost, precisely quantified as a non-negative integer. This is crucial for evaluating and comparing the efficiency of different plans. Axiom~\ref{ax: plan2} establishes that every plan is generated by some planner, connecting each plan to its generator and allowing for an understanding of the planning process and the assessment of various planners.


\begin{equation}
\texttt{Plan} \sqsubseteq =1 \texttt{hasPlanCost}. \texttt{xsd:nonNegativeInteger}
\label{ax: plan1}
\end{equation}
\begin{equation}
\texttt{Plan} \sqsubseteq \exists\texttt{isGeneratedBy}.\texttt{Planner}
\label{ax: plan2}
\end{equation}
% \vspace{-0.1cm}

\subsubsection{Planner}
The Planner category of the ontology includes classes that capture the details of the planner, planner type, and the planner performance from previous IPCs. Specifically, \verb|Planning Domain| relevance to a \verb|Planner| is classified based on the percentage of problems they have successfully solved, which is then categorized into three levels of relevance to the planner: \textit{low}, \textit{medium}, and \textit{high}. By incorporating this information into the ontology, planners can be evaluated based on their performance in different planning domains, and more informed decisions can be made. In addition, this information can be used to guide the development of new planners and to evaluate their performance against established benchmarks.

The axioms defined for the \texttt{Planner} category provide a foundation for understanding and assessing the capabilities of planners in the AI planning domain. Axiom~\ref{ax: planner1} classifies planners into different types based on their characteristics or strategies, enabling a nuanced understanding of various planning approaches. Axiom~\ref{ax: planner2} links planners with the specific domain requirements they can solve, highlighting their applicability in different planning scenarios.

\begin{equation}
\texttt{Planner} \sqsubseteq \exists\texttt{ofPlannerType}.\texttt{PlannerType}
\label{ax: planner1}
\end{equation}
\begin{equation}
\texttt{Planner} \sqsubseteq \exists\texttt{solvesRequirement}.\texttt{DomainRequirement}
\label{ax: planner2}
\end{equation}
% To incorporate the details of planner performance into the ontology, we have used information from previous IPCs. Specifically, we have analyzed the number of problems that a given planner has successfully solved and categorized this information into three distinct levels of relevance to the planner. Planners that have solved a relatively small number of problems are classified as of low relevance, whereas those who have solved a moderate number of problems are considered to have medium relevance. Finally, planners that have solved a large number of problems, including many challenging ones, are classified as having high relevance for a given domain.

% By incorporating this information into the ontology, we can better assess the performance of planners in different problem domains and make more informed decisions about which planners to use for a given problem. In addition, this information can be used to guide the development of new planners and to evaluate their performance against established benchmarks.

\subsection{Accessing Planning Ontology}
We have taken various measures to ensure that our planning ontology follows the FAIR principles \cite{wilkinson2016fair} of being Findable, Accessible, Interoperable, and Reusable. To assist users in exploring and utilizing our ontology, we have made it accessible through a persistent URL\footnote[1]{\label{purl}PURL - \url{https://purl.org/ai4s/ontology/planning}} and our GitHub repository\footnote[2]{\label{footnote: repo}\url{https://github.com/BharathMuppasani/AI-Planning-Ontology}}. Our repository contains ontology model files, mapping scripts, and utility scripts that extract information from PDDL domains and problems into intermediary JSON format and add the extracted data as triples using our model ontology, creating a knowledge graph. We provide sample SPARQL queries that address the ontology's competency questions mentioned earlier. Moreover, our ontology documentation, which is accessible through the GitHub repository, provides a comprehensive overview of the ontology's structure, concepts, and relations, including ontology visualization. This documentation serves as a detailed guide for users to comprehend the ontology's applications in the automated planning domain. We also provide the scripts and results from the ontology evaluation, which are presented as use cases of our ontology in later sections, in our repository, along with accompanying documentation.
Furthermore, our commitment includes a proactive approach to constantly updating and refining the ontology. This involves periodic updates and community-driven modifications, ensuring its continuous alignment with evolving standards and practices in the field of automated planning.


% Figure environment removed

% In the process of creating an ontology for automated planning, several tools were used for different purposes. The ontology was created using \verb|Protege|\footnote[1]{https://protege.stanford.edu/}, which is a widely used open-source ontology editor and knowledge management system. \verb|Protege| provides an intuitive user interface that allows users to easily create and edit ontologies. It also supports a wide range of ontology languages, including OWL, RDF, and RDFS.

% After creating the ontology, we utilized the \verb|rdflib|\footnote[2]{https://github.com/RDFLib/rdflib}, a Python library, which provides various functionalities for parsing and manipulating RDF data, to access the RDF-based ontology and extract the relevant information. To begin populating the ontology, we captured the domain and problem data in JSON format. Subsequently, we incorporated the data triples from different domains into the ontology using the \verb|rdflib| library. Additionally, we included information about the performance of various planners from previous IPCs in the ontology. Our GitHub repository\footnote[3]{\label{footnote: repo}https://github.com/BharathMuppasani/AI-Planning-Ontology} provides the RDF model file for the ontology, as well as Python scripts to extract domain and problem data in JSON format and add the extracted data as triples to the model ontology, creating a knowledge graph. To query the resulting knowledge graph, we utilized the SPARQL query language, which is the standard query language for RDF data. SPARQL allows users to query data stored in RDF format and is supported by many RDF tools, including Protege. With SPARQL, we were able to query the knowledge graph to extract information on specific domains and planners.

%%%%%%%%%%%%%%%%%%%%%%%%%%%%%%%%%%%%%%%%%%%%%%%%%%%%%

% \begin{table}[!t]
% \centering
% \begin{tabular}{|l|l|l|}
% \hline
% \textbf{Domain} & \textbf{Relevance} & \textbf{Planner} \\ \hline
% caldera & hasHighRelevancePlanner & Delfi1 \\ \hline
% caldera & hasHighRelevancePlanner & Complementary2 \\ \hline
% caldera & hasHighRelevancePlanner & Planning\_PDBs  \\ \hline
% caldera & hasHighRelevancePlanner & Scorpion \\ \hline
% caldera & hasHighRelevancePlanner & FDMS2 \\ \hline
% caldera & hasHighRelevancePlanner & FDMS1 \\ \hline
% caldera & hasHighRelevancePlanner & Metis1 \\ \hline
% caldera & hasHighRelevancePlanner & Metis2 \\ \hline
% caldera & hasMediumRelevancePlanner & Complementary1 \\ \hline
% caldera & hasMediumRelevancePlanner & symb\_Bi\_dir  \\ \hline
% caldera & hasMediumRelevancePlanner & Delfi2 \\ \hline
% caldera & hasMediumRelevancePlanner & DecStar \\ \hline
% caldera & hasMediumRelevancePlanner & MSP \\ \hline
% caldera & hasMediumRelevancePlanner & Blind \\ \hline
% caldera & hasMediumRelevancePlanner & Symple\_2 \\ \hline
% caldera & hasMediumRelevancePlanner & Symple\_1 \\ \hline
% caldera & hasLowRelevancePlanner & maplan\_2 \\ \hline
% caldera & hasLowRelevancePlanner & maplan\_1 \\ \hline
% \end{tabular}
% \caption{Domain-Relevance-Planner triples extracted for caldera domain from the knowledge graph created using IPC-2018 data}
% \label{tab:domain-relevance-planner}
% \end{table}

\section{Usage of Planning Ontology}
In the following section, we show the evaluation of a few competency questions and discuss two use cases of our planning ontology.
% \vspace{-0.3cm}

% Figure environment removed

\subsubsection{Evaluation of Competency questions:}
For the evaluation of the competency questions, we have considered a sample knowledge graph, shown in Figure \ref{fig:bw_kg}, for \verb|blocksworld| from IPC-2000 domain created using planning ontology shown in Figure \ref{fig:ontology}. SPARQL queries for each of these questions can be found at our GitHub Repository\textsuperscript{\ref{footnote: repo}}.
% \begin{enumerate}
%     \item C1: What are the different types of planners used in automated planning?\\
%     \textbf{Question Type:} Extracting planner information.\\
%     \textbf{Sufficiency Condition:} There should exist at least one individual for \verb|Planner| class. \\
%     \textbf{Result:} Shown in Table \ref{tab:c1}.  
%     \begin{table}[!h]
%         \centering
%         % \vspace{-0.5cm}
%         \begin{tabular}{ll}
%             \hline
%             \textbf{S.No} & \textbf{Planner} \\ \hline
%             1 & FF \\
%             2 & FastDownward \\
%             3 & LPG \\ \hline
%         \end{tabular}
%         % \vspace{0.2cm}
%         \caption{Results for C1 with knowledge graph in Figure \ref{fig:bw_kg}}
%         \label{tab:c1}
%         \vspace{-0.2cm}
%     \end{table}

%     \item C2: What is the relevance of planners in 'blocksworld' domain? \\
%     \textbf{Question Type:} Extracting best planner for a domain. \\
%     \textbf{Sufficiency Condition:} There should exist at least one \verb|Planner| individual having either of the relevance properties with 'blocksworld' individual of \verb|PlanningDomain| class. \\
%     \textbf{Result:} Shown in Table \ref{tab:c2}. 

%     \begin{table}[!h]
%         \centering
%         % \vspace{-0.5cm}
%         \begin{tabular}{llll}
%             \hline
%             \textbf{S.No} & \textbf{Domain} & \textbf{Relation} & \textbf{Planner} \\ \hline
%             1 & blocksworld & hasLowRelevancePlanner & CPT4 \\
%             2 & blocksworld & hasHighRelevancePlanner & FastDownward \\
%             3 & blocksworld & hasMediumRelevancePlanner & LPG \\ \hline
%         \end{tabular}
%         % \vspace{0.2cm}
%         \caption{Results for C2 with knowledge graph in Figure \ref{fig:bw_kg}}
%         \label{tab:c2}
%         % \vspace{-1cm}
%     \end{table}

%     \item C3: What are the available actions for 'blocksworld' domain? \\
%     \textbf{Question Type:} Extracting domain information. \\
%     \textbf{Sufficiency Condition:} For the 'blocksworld' individual of \verb|PlanningDomain|, there must be at least one \verb|DomainAction| individual with the relation \verb|hasAction|. \\
%     \textbf{Result:} Shown in Table \ref{tab:c3}. 
%     \begin{table}[!h]
%         \centering
%         % \vspace{-0.3cm}
%         \begin{tabular}{llll}
%             \hline
%             \textbf{S.No} & \textbf{Domain} & \textbf{Relation} & \textbf{Action} \\ \hline
%             1 & blocksworld & hasAction & put-down \\
%             2 & blocksworld & hasAction & pick-up \\
%             3 & blocksworld & hasAction & stack \\
%             4 & blocksworld & hasAction & unstack \\ \hline
%         \end{tabular}
%         % \vspace{0.2cm}
%         \caption{Results for C3 with knowledge graph in Figure \ref{fig:bw_kg}}
%         \label{tab:c3}
%         % \vspace{-0.7cm}
%     \end{table}
    
%     \item C4: Which problems in 'blocksworld' have problems with the goal state of 'b1' being on the table? \\
%     \textbf{Question Type:} Extracting problem information \\
%     \textbf{Sufficiency Condition:} For the 'blocksworld' individual of \verb|PlanningDomain|, there must be at least one \verb|PlanningProblem| individual with the relation \verb|hasProblem| and the problem should have '(ontable b1)' \verb|GoalState|.\\
%     \textbf{Result:} Shown in Table \ref{tab:c4}. 
%     \begin{table}[!h]
%         \centering
%         % \vspace{-0.5cm}
%         \begin{tabular}{llll}
%             \hline
%             \textbf{S.No} & \textbf{Domain} & \textbf{Relation} & \textbf{Problem} \\ \hline
%             1 & blocksworld & hasProblem & problem\_3\_1 \\ \hline
%         \end{tabular}
%         % \vspace{0.2cm}
%         \caption{Results for C4 with knowledge graph in Figure \ref{fig:bw_kg}}
%         \label{tab:c4}
%         % \vspace{-0.7cm}
%     \end{table}
    
%     \item C5: What are all requirements a given domain has? \\
%     \textbf{Question Type:} Extracting domain information \\
%     \textbf{Sufficiency Condition:} For the 'blocksworld' individual of \verb|PlanningDomain|, there must exist at least one \verb|DomainRequirement| individual with the relation \verb|hasRequirement|.\\
%     \textbf{Result:} Shown in Table \ref{tab:c5}. 

%     \begin{table}[!h]
%         \centering
%         % \vspace{-0.5cm}
%         \begin{tabular}{llll}
%             \hline
%             \textbf{S.No} & \textbf{Domain} & \textbf{Relation} & \textbf{Requirement} \\ \hline
%             1 & blocksworld & hasRequirement & :strips \\ \hline
%         \end{tabular}
%         % \vspace{0.2cm}
%         \caption{Results for C5 with knowledge graph in Figure \ref{fig:bw_kg}}
%         \label{tab:c5}
%         % \vspace{-1cm}
%     \end{table}

%     \item C6: What is the cost associated with generating a plan for a given problem? \\
%     \textbf{Question Type:} Extracting plan cost information. \\
%     \textbf{Sufficiency Condition:} There must exist at least one \verb|Plan| individual with with data property \verb|hasPlanCost|. \\
%     \textbf{Result:} Shown in Table \ref{tab:c6}. 
%     \begin{table}[!h]
%         \centering
%         \begin{tabular}{llll}
%             \hline
%             \textbf{S.No} & \textbf{Plan} & \textbf{Relation} & \textbf{Cost} \\ \hline
%             1 & plan\_3\_1 & hasPlanCost & 6 \\
%             % Add additional rows as necessary
%             \hline
%         \end{tabular}
%         \caption{Results for C6 with knowledge graph in Figure \ref{fig:bw_kg}}
%         \label{tab:c6}
%     \end{table}
    
%     \item C7: How many parameters does a specific action have? \\
%     \textbf{Question Type:} Extracting action parameter information. \\
%     \textbf{Sufficiency Condition:} There must exist at least one \verb|DomainAction| individual with an associated \verb|ActionParameter|. \\
%     \textbf{Result:} Shown in Table \ref{tab:c7}. 
%     \begin{table}[!h]
%         \centering
%         \begin{tabular}{llll}
%             \hline
%             \textbf{S.No} & \textbf{Action} & \textbf{Parameter-Count} \\ \hline
%             1 & pickup & 2 \\
%             % Add additional rows as necessary
%             \hline
%         \end{tabular}
%         \caption{Results for C7 with knowledge graph in Figure \ref{fig:bw_kg}}
%         \label{tab:c7}
%     \end{table}
    
%     \item C8: What planning type a specific planner belongs to? \\
%     \textbf{Question Type:} Extracting planner type information. \\
%     \textbf{Sufficiency Condition:} There should exist at least one \verb|Planner| individual with an associated \verb|PlanningType|. \\
%     \textbf{Result:} Shown in Table \ref{tab:c8}. 
%     \begin{table}[!h]
%         \centering
%         \begin{tabular}{llll}
%             \hline
%             \textbf{S.No} & \textbf{Planner} & \textbf{Relation} & \textbf{PlannerType} \\ \hline
%             1 & FastDownward & ofPlannerType & Classical Planner \\
%             % Add additional rows as necessary
%             \hline
%         \end{tabular}
%         \caption{Results for C8 with knowledge graph in Figure \ref{fig:bw_kg}}
%         \label{tab:c8}
%     \end{table}
    
%     \item C9: What requirements does a given planner support? \\
%     \textbf{Question Type:} Extracting planner requirement information. \\
%     \textbf{Sufficiency Condition:} There must exist at least one \verb|Planner| individual with associated \verb|DomainRequirements|. \\
%     \textbf{Result:} Shown in Table \ref{tab:c9}. 
%     \begin{table}[!h]
%         \centering
%         \begin{tabular}{llll}
%             \hline
%             \textbf{S.No} & \textbf{Planner} & \textbf{Relation} & \textbf{DomainRequirement} \\ \hline
%             1 & FastDownward & solvesRequirement & :strips \\
%             % Add additional rows as necessary
%             \hline
%         \end{tabular}
%         \caption{Results for C9 with knowledge graph in Figure \ref{fig:bw_kg}}
%         \label{tab:c9}
%     \end{table}

%     \item C10: What are the different \verb|ParameterType| present in a domain? \\
%     \textbf{Question Type:} Extracting domain type information. \\
%     \textbf{Sufficiency Condition:} There must exist at least one \verb|PlanningDomain| individual with associated \verb|ParameterType|. \\
%     \textbf{Result:} Shown in Table \ref{tab:c10}. 
%     \begin{table}[!h]
%         \centering
%         \begin{tabular}{llll}
%             \hline
%             \textbf{S.No} & \textbf{Domain} & \textbf{Relation} & \textbf{ParameterType} \\ \hline
%             1 & blocksworld & hasParameterType & block \\
%             2 & blocksworld & hasParameterType & table \\
%             % Add additional rows as necessary
%             \hline
%         \end{tabular}
%         \caption{Results for C10 with knowledge graph in Figure \ref{fig:bw_kg}}
%         \label{tab:c10}
%     \end{table}

    
% \end{enumerate}


\begin{table}[!t]
\centering
\caption{Demonstrating the effectiveness of two different policies employed to choose a planner for problem-solving.}
\begin{tabular}{lcccc}
\hline
\multicolumn{1}{c}{\textbf{Domain}} & \multicolumn{2}{c}{\textbf{Ontology Policy}} & \multicolumn{2}{c}{\textbf{Random Policy}}  \\ \cline{2-5} 
\multicolumn{1}{c}{}                                 & \multicolumn{1}{c}{Avg. Exp.}  & Avg. Plan Cost & \multicolumn{1}{c}{Avg. Exp} & Avg. Plan Cost \\ \hline
scanalyzer                                             & \multicolumn{1}{c}{\textbf{8,588}}    & 20                                 & \multicolumn{1}{c}{8,706}    & 20                                 \\ 
elevators                                              & \multicolumn{1}{c}{\textbf{1,471}}    & 52                                 & \multicolumn{1}{c}{64,541}   & 52                                 \\ 
transport                                              & \multicolumn{1}{c}{165,263}           & 491                                & \multicolumn{1}{c}{\textbf{132,367}}  & 491                                \\ 
parking*                                               & \multicolumn{1}{c}{\textbf{367,910}}  & 18                                 & \multicolumn{1}{c}{488,830}  & 17                                 \\ 
woodworking                                            & \multicolumn{1}{c}{\textbf{1,988}}    & 211                                & \multicolumn{1}{c}{19,844}   & 211                                \\ 
floortile**                                            & \multicolumn{1}{c}{283,724}           & 54                                 & \multicolumn{1}{c}{\textbf{2,101}}    & 49                                 \\ 
barman                                                 & \multicolumn{1}{c}{\textbf{1,275,078}} & 90                                 & \multicolumn{1}{c}{5,816,476} & 90                                 \\ 
openstacks                                             & \multicolumn{1}{c}{\textbf{132,956}}  & 4                                  & \multicolumn{1}{c}{139,857}  & 4                                  \\ 
nomystery                                              & \multicolumn{1}{c}{1,690}             & 13                                 & \multicolumn{1}{c}{1,690}    & 13                                 \\ 
pegsol                                                 & \multicolumn{1}{c}{\textbf{89,246}}   & 6                                  & \multicolumn{1}{c}{101,491}  & 6                                  \\ 
visitall                                               & \multicolumn{1}{c}{5}                & 4                                  & \multicolumn{1}{c}{5}       & 4                                  \\ 
tidybot**                                              & \multicolumn{1}{c}{\textbf{1,173}}    & 17                                 & \multicolumn{1}{c}{3,371}    & 33                                 \\ 
parcprinter                                            & \multicolumn{1}{c}{541}              & 441,374                             & \multicolumn{1}{c}{\textbf{417}}     & 441,374                             \\ 
sokoban                                                & \multicolumn{1}{c}{\textbf{9,653}}    & 25                                 & \multicolumn{1}{c}{156,600}  & 25                                 \\ \hline
\end{tabular}
% \vspace{0.2cm}
\label{tab:best_planner_eval}
% \vspace{-0.5cm}
\end{table}

\subsubsection{Usecase 1: Identifying Most Promising Planner} - 
One of the major challenges in the field of artificial intelligence (AI) is the automated selection of the best-performing planner for a given planning domain. This challenge arises due to the vast number of available planners and the diversity of planning domains. The traditional way to select a planner is to experiment with various search algorithms and heuristics and settle on an appropriate combination as seen in IPC competitions. To address this challenge, we now present a new approach by using our planning ontology to represent the features of the planning domain and the capabilities of planners.

The ontology for planning aims to capture the connection between the Planning Domain and the Planner by indicating the relevance of a planner to a specific domain. We made use of data acquired from International Planning Competitions (IPCs) to furnish specific details regarding the relevance of planners. The IPC results provide us with relevant details on the planners that took part in the competition and the domains that were evaluated during that particular year. This information includes specifics on how each planner performed against all the domains that participated.

To show the usage of extracting the most promising planners for a given domain, we have used IPC-2011 data\footnote[3]{\label{ipc-2011}http://www.plg.inf.uc3m.es/ipc2011-deterministic/} (optimal track). The ontology was populated with data acquired from the IPC-2011, which provided relevant details on the planners that took part in the competition and the domains that were evaluated during IPC-2011. A relevance relation of either \textit{low}, \textit{medium}, or \textit{high} was assigned to each planner based on the percentage, \textit{low-}below 35\%, \textit{medium-}35\% to 70\%, \textit{high-}70\% and above, of problems they solved in a given domain. In this experiment, we consider that the experimental environment has four planners available: Fast Downward Stone Soup 1\footnote[4]{\label{fd_ipcPlanners}https://www.fast-downward.org/IpcPlanners}, LM-Cut\textsuperscript{\ref{fd_ipcPlanners}}, Merge and Shrink\textsuperscript{\ref{fd_ipcPlanners}}, and BJOLP\textsuperscript{\ref{fd_ipcPlanners}}. We evaluate 3 problem instances of each domain from IPC-2011 with 2 policies for selecting planners to generate plans for each of these problem instances - 
\begin{enumerate}
    \item \textbf{Random Policy:} To solve each problem instance, this policy selects a random planner from the available planners.
    \item \textbf{Ontology Policy:} To solve each problem instance, this policy extracts the information on the best planner for the problem domain from the ontology populated with IPC-2011 data.
\end{enumerate}

Table \ref{tab:best_planner_eval} presents the results of our evaluation, indicating the average number of nodes expanded and plan cost for each policy in a given domain.
%The evaluation results are presented in Table \ref{tab:best_planner_eval}, which provides details on the average number of nodes expanded and the average plan cost for each policy in a given domain. 
The table provides a comprehensive summary of the performance of different planners in terms of their efficiency and effectiveness. An ideal planner is expected to generate a solution with low values for both these metrics. 
%By comparing the performance of the planners selected using the two policies, we demonstrate the effectiveness of the Ontology Policy in selecting the best-performing planner for a given planning domain. 
The {\em Ontology Policy}, designed to select the best-performing planner for a given domain, outperformed the {\em Random Policy} in terms of the average number of nodes expanded to find a solution. Moreover, the {\em Random Policy} failed to solve problems in the parking (1 out of 3), floortile (2 out of 3), and tidybot (2 out of 3) domains, which highlights the limitations of choosing a planner randomly. But if a domain is easily solvable by relevant planners that can tackle them, {\em Random Policy} may still do well. 

What we demonstrate is a rather simple usage of the Ontology for Planner Selection policy. Creating more advanced strategies is a promising area for further research.

% Our ontology is constructed using data collected from the International Planning Competition (IPC) and planner information. This enables us to represent the knowledge about the planning domain and the planners in a structured format. This structured representation of knowledge can be used to automate the planner selection process by making use of reasoning techniques. We have created a knowledge graph from the data on planning domains and planner performance from IPC-2018. As an example, Table \ref{tab:domain-relevance-planner} shows the triples extracted from the knowledge graph for the domain \verb|caldera|, indicating the relevance of the planners presented in IPC-2018.

% Please add the following required packages to your document preamble:
% \usepackage{multirow}
% \begin{table}[htbp]
%     \centering
%     \begin{tabular}{>{\columncolor{blue!20}}c|>{\centering\arraybackslash}c}
%         Cell 1, Row 1 & Cell 2, Row 1 \\
%         \hline
%         \rowcolor{green!20} Cell 1, Row 2 & Cell 2, Row 2 \\
%         Cell 1, Row 3 & \cellcolor{red!20}Cell 2, Row 3 \\
%     \end{tabular}
% \end{table}

% % Figure environment removed

\subsubsection{Usecase 2: Extracting Macro Operators} -
While automated planning has been successful in many domains, it can be computationally expensive, especially for complex problems. One approach to improve efficiency is by using macro-operators, which are sequences of primitive actions that can be executed as a single step. However, identifying useful macro-operators manually can be time-consuming and challenging. Authors in \cite{chrpa2010generation} introduce a novel method for improving the efficiency of planners by generating macro-operators. The proposed approach involves analyzing the inter-dependencies between actions in plans and extracting macro-operators that can replace primitive actions without losing the completeness of the problem domain. The soundness and complexity of the method are assessed and compared to other existing techniques. The paper asserts that the generated macro-operators are valuable and can be seamlessly integrated into planning domains without losing the completeness of the problem. In \cite{botea2005learning}, the authors detail a three-step method for learning and utilizing macro-operators to enhance planning efficiency in new problems. Initially, a comprehensive set of macros is generated from the solution graphs of various training problems. This set is then narrowed down through a filtering process. The selected macros are subsequently applied to expedite problem-solving. The generation phase involves extracting and selecting specific subgraphs from solution graphs to create individual macros.

Based on the ontology depicted in Figure \ref{fig:ontology}, we extract macro-operators that can enhance the efficiency of planners. To demonstrate this, we have considered three different domains: \verb|blocksworld|(bw), \verb|driverlog|(dl), and \verb|grippers|(gr), presented in IPC-2000, 2002, and 1998 respectively. We initially developed a knowledge graph using the ontology represented in Figure \ref{fig:ontology} for the three domains of interest. Subsequently, we employed a SPARQL query to retrieve the stored plans for these domains. We then examined these plans to identify the sequences of action pairs and ranked them based on their frequency of occurrence. To improve the effectiveness of this technique, it is essential to consider both the frequency of occurrence of action pairs and the properties of the domain. Specifically, the precondition and effect of actions should be analyzed to ensure that the first action leads to the precondition of the second action in the pair. We employed another SPARQL query to extract the preconditions and effects associated with each of these actions. We analyzed the resulting action pairs to verify their validity of occurrence, thereby filtering out pairs that did not have a combined effect. The results of this extraction process are shown in Table \ref{tab: macros}. These action relations are stored back into the knowledge graph in the \verb|MacroAction| class and can be utilized by planners to enhance their efficiency.

% Macros - branching Factor - Action Selection process is not better informed (heuristic) --> degrade performance 

\begin{table}[t]
\centering
\caption{Extracted action relations, ordered based on their frequency, for domains \texttt{blocksworld}, \texttt{driverlog}, and \texttt{grippers}.}
\begin{tabular}{p{2.5cm}p{9cm}}
\hline
\textbf{Domains} & \textbf{Extracted Action Relations} \\
\hline
\cellcolor{blue!25}\texttt{blocksworld} & \texttt{unstack} * \texttt{put-down}; \texttt{pick-up} * \texttt{stack}; \texttt{put-down} * \texttt{unstack}; \texttt{stack} * \texttt{pick-up}; \texttt{unstack} * \texttt{stack}; \texttt{put-down} * \texttt{pick-up}; \texttt{stack} * \texttt{unstack}\\
\hline
\cellcolor{green!25}\texttt{driverlog} & \texttt{drive-truck} * \texttt{unload-truck}; \texttt{drive-truck} * \texttt{load-truck}; \texttt{board-truck} * \texttt{drive-truck}; \texttt{walk} * \texttt{board-truck}\\
\hline
\cellcolor{red!25}\texttt{grippers} & \texttt{pick} * \texttt{move}; \texttt{move} * \texttt{drop}\\
\hline
\end{tabular}
% \vspace{0.2cm}

\label{tab: macros}
% \vspace{-0.5cm}
\end{table}

\begin{table}[b]
\centering
\begin{tabular}{lcccccc}
\hline
 \multicolumn{1}{c}{\textbf{Domain}} & \multicolumn{3}{c}{\textbf{Original Domain}} & \multicolumn{3}{c}{\textbf{Domain With Macros}} \\ \cline{2-7}
\multicolumn{1}{c}{} & \multicolumn{1}{c}{Avg. Exp.} & \multicolumn{1}{c}{Avg. Eval.} & Avg. Gen. & \multicolumn{1}{c}{Avg. Exp.} & \multicolumn{1}{c}{Avg. Eval.} & Avg. Gen. \\ \hline
\textbf{blocksworld} & \multicolumn{1}{c}{20219}      & \multicolumn{1}{c}{59090}      & 106321     & \multicolumn{1}{c}{18}        & \multicolumn{1}{c}{310}        & 359       \\
\textbf{gripper} & \multicolumn{1}{c}{2672}      & \multicolumn{1}{c}{10660}      & 30871     & \multicolumn{1}{c}{510}        & \multicolumn{1}{c}{3974}        & 11468       \\
\textbf{driverlog} & \multicolumn{1}{c}{3753}      & \multicolumn{1}{c}{17849}      & 45753     & \multicolumn{1}{c}{14888}        & \multicolumn{1}{c}{720008}        & 209760       \\ \hline

\end{tabular}

% \vspace{0.2cm}
\caption{Comparison of planner performance between original and macro-enabled versions of three planning domains, showing the average number of nodes expanded, evaluated, and generated.}

\label{tab: macros_results}
% \vspace{-0.7cm}
\end{table}

Table \ref{tab: macros_results} shows the comparison of a planner performance given the original domain and macros-enabled version of the domain. For this evaluation, we have considered the FastDownward planner \cite{helmert2006fast} with LM-Cut Heuristic \cite{helmert2011lm} to generate plans for 20 problems of varying complexities for each domain. We evaluate the performance of each domain based on the average number of nodes expanded, evaluated, and generated to find a solution. This study demonstrates that macro operators can enhance the planner performance in most of the domains tested, with the exception of the \verb|driverlog| domain. In this domain, the planner performs worse when macro operators are included, as they increase the average number of nodes expanded, evaluated, and generated. This is due to the fact that the macro operators introduce more actions to the domain, which increases the branching factor and challenges the heuristic to select the optimal action at each step. Hence, the applicability of macro operators depends on the features of the domain and the planner. Macro operators can facilitate the planning process by decreasing the search depth, but they can also hinder it by increasing the search width. A potential improvement is to use a more informative heuristic that guides the planner to choose the best action at each step.

% This technique involves analyzing the statistics of multiple plans stored in an ontology and extracting common patterns of actions, which are then combined to form higher-level plans, also known as macro-operators. To improve the effectiveness of this technique, it is essential to consider both the frequency of occurrence of action pairs and the properties of the domain. Specifically, the precondition and effect of actions should be analyzed to ensure that the first action leads to the precondition of the second action in the pair. By doing so, we can filter out irrelevant or invalid action pairs and obtain a set of meaningful and applicable macro-operators. These macro operators can replace primitive actions in a domain. 
% \begin{table}[t]
% \centering
% \begin{tabular}{|l|c|c|c|c|c|c|}
% \hline
%            & bw                                                           & \textbf{bw*}                                                         & \textbf{dl}                                                          & dl*                                                         & \textbf{gr}                                                          & gr*                                                         \\ \hline
% Successful & \begin{tabular}[c]{@{}c@{}}90.04\% \\ (88.44\%)\end{tabular} & \textbf{\begin{tabular}[c]{@{}c@{}}94.08\%\\ (92.36\%)\end{tabular}} & \textbf{\begin{tabular}[c]{@{}c@{}}76.56\%\\ (52.61\%)\end{tabular}} & \begin{tabular}[c]{@{}c@{}}40.08\%\\ (35.69\%)\end{tabular} & \textbf{\begin{tabular}[c]{@{}c@{}}82.97\%\\ (69.47\%)\end{tabular}} & \begin{tabular}[c]{@{}c@{}}72.42\%\\ (44.94\%)\end{tabular} \\ \hline
% Failed     & 9.94\%                                                       & \textbf{5.92\%}                                                      & \textbf{23.44\%}                                                     & 42.86\%                                                     & \textbf{16.61\%}                                                     & 21.97\%                                                     \\ \hline
% Incomplete & 0.02\%                                                       & \textbf{0\%}                                                         & \textbf{0\%}                                                         & 17.06\%                                                     & \textbf{0.42\%}                                                      & 5.61\%                                                      \\ \hline
% \end{tabular}
% \label{plansformer-results}
% \caption{Table showing the results of plan validation for Plansformer with the percentage of optimal plans shown in parentheses., with an asterisk denoting domains that had extracted action relations added to the prompt.}
% \end{table}


% We have used the extracted action relations to test Plansformer \cite{pallagani2022plansformer}, a generative model for AI planning. Plansformer is obtained by fine-tuning CodeT5, a Large Language Model that is pre-trained on code. We use Plansformer for our experimentation as it is easy to infuse macros by appending to the input and obtain the generated plan for validation. The results are presented in Table \ref{plansformer-results}. By directly including the extracted action relations in the prompt, we can observe an increase in percentage for valid plans in \verb|bw| domain, whereas the percentage declined for other domains. 

%%%%%%%%%%%%%%%%%%%%%%%%%%%%%%%%%%%%%%%%%%%%%%%%%%%%%
\section{Conclusion}

In this work, we build and share a planning ontology that provides a structured representation of concepts and relations for planning, allowing for efficient extraction of domain, problem, and planner properties. The ontology's practical utility is demonstrated in identifying the best-performing planner for a given domain and extracting macro operators using plan statistics and domain properties. Standardized benchmarks from IPC domains and planners offer an objective and consistent approach to evaluating planner performance, enabling rigorous comparisons in different domains to identify the most suitable planner. The planning ontology can aid researchers and practitioners in automated planning, and its use can simplify planning tasks and boost efficiency. As the field of AI planning continues to evolve, planning ontology can play a crucial role in advancing the state-of-the-art while leveraging the past.

% Future work could explore the use of mixed reasoning strategy with both ontologies (top-down) and Large Language Models (LLMs) (bottom-up) knowledge \cite{Mittal2017ThinkingFA-ontology}. For instance, one could leverage the planning ontology in the context of LLMs, which have recently shown promise for automated planning \cite{plansformer-paper-pallagani2022}. Moreover, the application of this mixed reasoning approach could be extended to complex domains, such as multi-agent systems, where coordinating actions between multiple agents is crucial.

Future work could explore the use of a mixed reasoning strategy that combines the structured, top-down approach of ontologies with the dynamic, bottom-up capabilities of Large Language Models (LLMs) \cite{Mittal2017ThinkingFA-ontology}. This approach can be particularly effective in contexts like LLMs, which have shown promise for automated planning \cite{plansformer-paper-pallagani2022}. Furthermore, our ontology, with its specific data properties for storing Action explanations, can be leveraged to enhance this hybrid model. It can provide comprehensive explanations for planning decisions as shown in the workflow Figure \ref{fig:use}, adding an interpretive layer that is crucial for complex domains such as multi-agent systems, where understanding the rationale behind each agent's actions is key. This blend of ontology-based clarity and LLM-driven adaptability could offer nuanced insights into coordinating actions and explaining them in a way that is both transparent and informative.

% In this work, we have presented a planning ontology that enables efficient extraction of domain, problem, and planner properties, and demonstrated its practical utility in identifying the best-performing planner for a given domain and extracting macro operators. The ontology can facilitate identifying and comparing planners' performance in specific domains, aiding researchers and practitioners in automated planning. As the field of AI planning continues to evolve, the further development of ontologies will play a crucial role in advancing the state-of-the-art and supporting the creation of intelligent, knowledge-based systems. Future work could explore the use of ontology in the context of LLMs and the development of hybrid systems combining ontology with other AI techniques. Moreover, complex domains like multi-agent systems could also be explored, where the coordination of actions between multiple agents is necessary.

% Automated planning is an essential component of intelligent systems, enabling machines to make decisions and perform tasks autonomously. 
% Over the years, significant progress has been made in the field leading to the development of a variety of planners that try to tackle an expanding range of complex applications in diverse domains. However, there was little work to organize this information systematically and draw insights to select and improve promising planners for a given domain. 

% In this work, we build and share a planning ontology that provides a structured representation of concepts and relations for planning, allowing for efficient extraction of domain, problem, and planner properties. The practical utility of the built ontology is demonstrated in two areas: identifying the best-performing planner for a given domain and extracting macro operators using plan statistics and domain properties. The use of standardized benchmarks from IPC domains and planners offers an objective and consistent approach to evaluating planner performance, enabling rigorous comparisons in different domains to identify the most suitable planner. The ontology can be used to more easily identify and compare planners' performance in specific domains, aiding researchers and practitioners in automated planning. Additionally, the extraction of macro operators from the ontology can simplify planning tasks and boost efficiency.
% As the field of AI planning continues to evolve with the emergence of new solution approaches (e.g., combining reinforcement learning and search, using large language models (LLMs)), the planning ontology can play a crucial role in advancing the state-of-the-art leveraging the past. 

% One can extend the work in many directions. First, following the trend of a mixed reasoning strategy with both  ontologies (top-down) and LLMs (bottom-up) knowledge \cite{Mittal2017ThinkingFA-ontology}, 
% one can support reasoning with the planning ontology in the context of LLMs that have started to show promise for automated planning \cite{plansformer-paper-pallagani2022}. Additionally, the development of hybrid systems that combine ontology with other AI techniques, such as machine learning, can lead to more robust and adaptable planning systems.

% Second, one can explore  more complex domains, such as multi-agent systems, where planning involves coordinating actions between multiple agents, comparing the problems, and finding relevant planners from past experiences. Additionally, the development of hybrid systems that combine ontology with other AI techniques, such as machine learning, can lead to more robust and adaptable planning systems. 
% As the field of AI planning continues to evolve, the further development of ontologies will play a crucial role in advancing the state-of-the-art and supporting the creation of intelligent, knowledge-based systems.

% In conclusion, the development of an ontology for automated planning represents a significant step toward the creation of more intelligent and efficient planning systems. The use of standardized benchmarks and the extraction of macro operators from the ontology can aid in evaluating planner performance and simplifying planning tasks. As the field of AI planning continues to evolve, the further development of ontologies will play a crucial role in advancing the state-of-the-art and supporting the creation of intelligent, knowledge-based systems.

%Over the years, significant progress has been made in the development of planners, which are algorithms that generate sequences of actions to achieve specific goals. However, the complexity of planning domains has presented significant challenges for researchers and practitioners in the field.

% There  the development of an ontology for automated planning has proven to be a valuable tool to capture the intricacies of the AI planning domain. 
% In recent years, the development of an ontology for automated planning has proven to be a valuable tool to capture the intricacies of the AI planning domain. 
 
 % The ontology provides a structured representation of concepts and relations for AI planning domains, allowing for efficient extraction of domain properties. 


% As the field of AI planning continues to evolve, the development of ontologies will undoubtedly play a crucial role in advancing the state-of-the-art. Further research and development in the ontology can lead to the creation of more intelligent and efficient planning systems. Ontology can be used to model knowledge from multiple domains and integrate it into a unified planning framework, thereby supporting the creation of intelligent, knowledge-based systems. 
% Future prospects for the development of ontology in automated planning 
% include the exploration of more complex domains, such as multi-agent systems, where planning involves coordinating actions between multiple agents, comparing the problems, and finding relevant planners from past experiences. Additionally, the development of hybrid systems that combine ontology with other AI techniques, such as machine learning, can lead to more robust and adaptable planning systems.

% In conclusion, the development of an ontology for automated planning has proven to be a valuable tool to capture the intricacies of the AI planning domain. By providing a structured representation of concepts and relations for AI planning domains, the ontology allows efficient extraction of domain properties. Moreover, ontology has been shown to have practical applications in two areas: identifying the best-performing planner for a given domain and extracting macro operators using plan statistics and domain properties.

% Standardized benchmarks from IPC domains and planners offer an objective and consistent approach to evaluating planner performance, enabling rigorous comparisons in different domains to identify the most suitable planner. Standardized benchmarks also aid in experiment replication and comparison of results, which is essential to advance automated planning. Researchers and practitioners in automated planning can use ontology to more easily identify and compare planners' performance in specific domains. Macro operators extracted from the ontology can simplify planning tasks and boost efficiency. These tools help automate planning for real-world problems.

% As the field of AI planning continues to evolve, the development of ontologies will undoubtedly play a crucial role in advancing the state-of-the-art. With further research and development, ontologies could potentially be used to support a wide range of applications, from automated decision-making to intelligent systems in various industries. Ultimately, the development of ontologies for automated planning represents a significant step toward the creation of more intelligent and efficient planning systems.

% The use of benchmarks from IPC domains and planners provides a standardized and objective means of evaluating the performance of planners. This allows for a more rigorous comparison of planners based on their performance in different domains, which is crucial to identifying the most suitable planner for a specific planning task. Moreover, the use of standardized benchmarks enables researchers to replicate experiments and compare results, which is essential for advancing the field of automated planning. Through the use of ontology, researchers, and practitioners in the field of automated planning can more easily identify and compare planners based on their performance in specific domains. Additionally, the ability to extract macro operators from the ontology provides a means to simplify planning tasks and increase efficiency.



%%%%%%%%%%%%%%%%%%%%%%%%%%%%%%%%%%%%%%%%%%%%%%%%%%%%%
% \section{Introduction}
Current quantum hardware is unable to carry out universal quantum computations due to the buildup of errors that occur during the computation. 
The magnitude of the individual error is currently above the value that the Threshold Theorem requires in order to kick-start quantum error correction and fault-tolerant quantum computation~\cite[Section 10.6]{nielsen_chuang_2010}. 
Although the experimentally achieved fidelity rates are promising and the error bounds are inching closer to the required threshold, we will have to work for the foreseeable future with quantum hardware with errors that build-up during the computation.  This implies that we can only do a limited number of steps before the output of the computation has become completely uncorrelated with the intended one.

For fault-tolerant quantum computing, we repeat four steps: 
1) We apply a number of single and two-qubit quantum gates, in parallel whenever possible; 
2) We perform a syndrome measurement on a subset of the qubits; 
3) We perform fast classical computations to determine which errors have occurred and how to correct them; 
and, 4) We apply correction terms based on the classical computations.
We then repeat these four steps with a next sequence of gates. 
These four steps are essential to fault-tolerant quantum computing. 


The starting point of this work is to use the four steps outlined above, not to carry out error correction and fault-tolerant computation, but to enhance short, constant-depth, {\em uncorrected} quantum circuits that perform single qubit gates and {\em nearest-neighbor} two qubit gates. 
Since in the long run we will have to implement error-correction and fault-tolerant computation anyhow, and this is done by such a four-step process, why not make other use of this architecture? Moreover, on some of the quantum hardware platforms, these operations are already in place.
Embracing this idea we naturally arrive at the question: what is the computational power of \textit{low-depth} quantum-classical circuits organized as in the four steps outlined above? 
We thus investigate circuits that execute a small, ideally constant, number of stages, where at each stage we may apply, in parallel, single qubit gates and {\em nearest-neighbor} two qubit gates, followed by measurements, followed by low-depth classical computations of which the outcome can control quantum gates in later stages. 
It is not clear, at first, whether such circuits, especially with constant depth, can do anything remotely useful. 
But we will see that this is indeed the case: many quantum computations can be done by such circuits in constant depth. 
By parallelizing quantum computations in this way, we improve the overall computational capabilities of these circuits, as we do not incur errors on qubits that are idle, simply because qubits are not idle for a very long time. 
Furthermore, reducing the depth of quantum circuits, at the cost of increasing width, allows the circuit to be run faster even if errors occur.

The first usage of such a four-step layout, not to do error correction, but to perform computations, can be found in the paradigm of measurement-based quantum computing~\cite{gottesman1999demonstrating,raussendorf2001one,jozsa2006introduction,clark2007generalised}: 
A universal form of quantum computing where a quantum state is prepared and operations are performed by measuring qubits in different bases, depending on previous measurements and intermediate measurements.

\citeauthor{PhamSvore2013} were the first to formalize the four-step protocol for performing computations~\cite{PhamSvore2013}. They included specific hardware topologies by considering two-dimensional graphs for imposing constraints on qubit interactions. In their model, they develop circuits for particularly useful multi-qubit gates, including specifying costs in the width, number of qubits, depth, number of concurrent time steps, size, and total number of non-Identity operations.
As a result, they find an algorithm that factors integers in polylogarithmic depth.
\citeauthor{Browne:2011} showed that the main tool in the work by \citeauthor{PhamSvore2013}, the fan-out gate, can also be replaced by additional log-depth classical computations in the measurement-based quantum computing setting~\cite{Browne:2011}.

More recently, \citeauthor{Cirac:2021} introduced a scheme to implement unitary operations involving quantum circuits combined with Local Operations and Classical Communication ($\mathsf{LOCC}$) channels: $\mathsf{LOCC}$-assisted quantum circuits~\cite{Cirac:2021}. Similarly to the four-step scheme we just described, they allow for a short depth circuit to be run on the qubits, followed by one round of $\mathsf{LOCC}$, in which ancilla qubits are measured and local unitaries are applied based on the measurement outcomes. They show that in this model any 1D transitionally invariant matrix-product state (MPS) with fixed bond dimension is in the same phase of matter as the trivial state. Similar ideas can be found in~\cite{TVV_NonAbelianTopologicalOrder_2022, tantivasadakarn2021long}.

In this work, we introduce a new model, called \textit{Local Alternating Quantum-Classical Computations} ($\LAQCC$). In this model we alternate between running quantum circuits (constrained by locality), ending in the measurement of a subset of qubits, and fast classical computations based on the measurement results. The outcome of the classical computations are then used to control future quantum circuits. We allow for flexibility in this model, by giving different constraints to the power of both the quantum circuits and the classical circuits as well as the number of alternations between them. 
Most attention will be given to $\LAQCC$ containing quantum circuits of constant depth, classical circuits of logarithmic depth and at most a constant number of alternations between them. 
Any circuit constructed in this model is considered to be of constant depth. 
We restrict ourselves to logarithmic depth classical computations, as this is the first natural and non-trivial extension beyond constant-depth classical computations. 
Constant-depth classical computations do however also have an equivalent constant-depth quantum implementation.

The definition of $\LAQCC$ sharpens the original definition of \citeauthor{PhamSvore2013} by adding constraints to the intermediate classical computations. This allows us to bound the power of $\LAQCC$ from above. 

The main result of \citeauthor{Cirac:2021}, that 1D translational invariant MPS with fixed bond dimension can be prepared by $\mathsf{LOCC}$-assisted circuits, relies on local symmetries of the MPS. These symmetries allow them to prepare local states (on a constant number of qubits) and glue them together by doing one round of the appropriate entangling measurement and corrections, after which they run a round of local unitaries to get the desired result. This general scheme for preparing states that exhibit an MPS description with the appropriate local symmetries requires only geometrically local unitaries and one round of measurement and corrections an therefore is accessible in $\LAQCC$. Studying different local symmetries, known as Symmetry Protected Topological (SPT) phases of matter, to find measurement-based constant depth circuits for states is a broad ongoing field of research~\cite{TVV_NonAbelianTopologicalOrder_2022, tantivasadakarn2021long, smith2023deterministic}. 
All these schemes have a $\LAQCC$ implementation.

%$\LAQCC$-circuits also exist for general schemes of preparing local states, based on the local tensors, and gluing them together using one round of entangled measurement and corrections, based on the local symmetry. 
%The main result of \citeauthor{Cirac:2021}, that 1D translational invariant MPS with fixed bond dimension can be prepared by $\mathsf{LOCC}$-assisted circuits, relies heavily on local symmetries of the MPS and as a result also has an equivalent $\LAQCC$ implementation. 
%The corrections applied after the measurement round are local unitaries depending on the local symmetries of the MPS. 

 

%This general scheme of preparing local states, based on the local tensors, and gluing it together by doing one round of entangled measurement and corrections, based on the local symmetry, is accessible in $\LAQCC$.
Note however that \citeauthor{Cirac:2021} also suggest a circuit for the $W$-state.
This circuit uses sequentially and dependent measurement-based corrections of the ancilla qubits. 
These dependent measurements translate to sequential alternations between the quantum and classical circuits and therefore increase the total depth to linear depth, exceeding the constant-depth constraints imposed by $\LAQCC$-circuits. 

We study the power of the $\LAQCC$ model with respect to state preparation, showing that even with only constant quantum-depth and logarithmic classical depth it remains possible to prepare states with long-range entanglement.
Another surprising result is that it is unlikely that $\LAQCC$ circuits are classically simulatable. We show that any instantaneous quantum polynomial-time (IQP) circuit~\cite{Bremner2010,Shepherd2009} has an $\LAQCC$ implementation.
Classical simulation of IQP circuits implies the collapse of the polynomial hierarchy to the third level, which is not believed to be true~\cite{Bremner2017}. Therefore, we expect that $\LAQCC$ circuits are unlikely to be classically simulatable. We bound the power of $\LAQCC$ by showing that it is contained in $\QNC^1$, the class of polynomial-size, log-depth circuits.

Next, we also study the power that intermediate classical calculations can add to quantum computations, by considering a new model that alternates between polynomially many polynomial-depth quantum circuits and unbounded classical computations
We study this model by doing a complexity theoretical analysis, where we draw inspiration from the notions of complexity given by \citeauthor{RosenthalYuen:2022}, \citeauthor{MetgerYuen:2023}, and \citeauthor{Aaronson:2004}.
All three complexity notions are based on the notion of state preparation, instead of more traditional definition of complexity such as the decidability of a computational problem. 
The first two consider classes based on sequences of quantum states preparable by a polynomial-sized quantum circuit, where the circuits are uniformly generated by a computational class, for instance, the class $\mathsf{PSPACE}$, which results in the complexity class $\mathsf{StatePSPACE}$~\cite{RosenthalYuen:2022,MetgerYuen:2023}.
The third notion considers a relative complexity, where the complexity is measured between two given states, and is measured by the number of gates, from a given gate-set, required to transform one state in another state~\cite{Aaronson:2004}. 
For our definition of state preparation complexity, we drop the uniformity constraint from~\cite{RosenthalYuen:2022,MetgerYuen:2023} and define a class as $\mathsf{StateX}$, which refers to states preparable by circuits of type $\mathsf{X}$. 
As an example, if $\mathsf{X} = \QNC^0$, this results in the class $\mathsf{StateQNC^0}$, which is the set of states preparable from the $\ket{0}^n$ state by poly-size constant-depth circuits. 
This notion is similar to the relative complexity from~\cite{Aaronson:2004}, where one state is the  $\ket{0}^n$ state and instead of counting the number of gates we consider the set of states preparable by a fixed number of gates. Using this notion of complexity we show that any state preparable by an $\LAQCC^*$ circuit is also preparable by a $\mathsf{PostQPoly}$ circuit, the class of circuits of polynomial depth with an additional post-selection gate. 

All Clifford circuits have a constant-depth $\LAQCC$ implementation, implying that any stabilizer state can be implemented by a constant-depth $\LAQCC$ circuit, see Section~\ref{sec:clifford_circuits} for a proof of this statement. 
Efficient circuits for stabilizer states have been known already through measurement-based quantum computing. Therefore this paper focuses on the preparation of non-stabilizer states, and as a surprising result we find novel constant-depth protocols for four very natural classes of non-stabilizer states.
Despite the extensive research into these four classes of non-stabilizer states and the many applications of them, no efficient constant- or low-depth state preparation protocols are known yet. We specifically consider these four classes as they are all often used as initial states in other algorithms.

The first state is a uniform superposition over an arbitrary number of states. 
This state finds applications in many quantum algorithms, as they often start with a uniform superposition over multiple states. 
This superposition is often achieved by applying Hadamard gates to every qubit due to its simplicity to prepare. 
Yet, the analysis of many algorithms, such as Shor's algorithm~\cite{Shor:1997}, would benefit from a different initial superposition. 
The circuit to prepare the uniform superposition over an arbitrary number of states uses an exact version of Grover search as a subroutine, that turns a probabilistic circuit, with a known constant probability of success, into a deterministic circuit. 
We use the circuit for preparing a uniform superposition over an arbitrary number of states as a subroutine in the next two quantum state preparation protocols. 

The second state is the $W$-state, the uniform superposition over all computational basis states of Hamming-weight~$1$, a natural long-ranged entangled state that displays a fundamentally nonequivalent type of entanglement from the Greenberger–Horne–Zeilinger state~\cite{WState:2000}, for which $\LAQCC$-type constant-depth circuits were previously known~\cite{PhamSvore2013, Cirac:2021}. 
The $W$-state is often used as benchmark for new quantum hardware~\cite{Haffner2005,Neeley2010,GarciaPerez:2021}. 
A novel way to prepare the $W$-state therefore gives a new way to benchmark different quantum devices with each other. 
A circuit for preparing the $W$-state was given in~\cite{Cirac:2021}, but this implementation requires sequentially alternating measurements followed by local unitaries, which in the $\LAQCC$ model is not considered to be of constant depth. 
We improve this protocol by giving an $\LAQCC$ implementation of the $W$-state, based on a compress-uncompress method that links the one-hot and binary encoding of integers.

The third state considered is the Dicke state, a generalization of the $W$-state, a superposition over all computational basis states with Hamming-weight $k$~\cite{Dicke:1954}. 
Dicke states have relevance in various practical settings.
For instance, for quantum game theory~\cite{zdemir2007}, quantum storage~\cite{Bacon_Compress:2006,Plesch:2010}, quantum error correction~\cite{ouyang2014permutation}, quantum metrology~\cite{toth2012multipartite}, and quantum networking~\cite{prevedel2009experimental}. 
Dicke states have been used as a starting state for variational optimization algorithms, most notably Quantum Alternating Operator Ansatz (QAOA)~\cite{Hadfield2019}, to find solutions to problems such as Maximum k-vertex Cover~\cite{Brandhofer2022,cook2020quantum}.
The ground states of physical Hamiltonians describing one-dimensional chains tend to show a resemblance to Dicke states such as states resulting from the Bethe ansatz, making them an ideal starting state when investigating the ground state behavior of these Hamiltonians~\cite{TDL_BetheAnsatzDerivation:2010,B_ExcitedStateQuantumPhaseTransitions:2013,DickeTransitions:2021}. 
For instance, the algorithm by \citeauthor{van2021preparing}, who give an algorithm to prepare the Bethe ansatz eigenstates of the spin-1/2 XXZ spin chain, starts by first preparing a Dicke state~\cite{van2021preparing}. 
A Dicke-state preparation protocol based on the compress-uncompress methodology used in the $W$-state furthermore finds applications in entanglement distillation, where the entanglement of a large state is concentrated on only a few qubits. 
Efficient deterministic circuits for preparing Dicke states have been proposed by \citeauthor{bartschi2019deterministic}~\cite{bartschi2019deterministic, bartschi2022deterministic_short_depth}. 
They provide a quantum circuit of depth $\mathO(k \log(\frac{n}{k}))$, allowing arbitrary connectivity, to prepare a Dicke state, which they conjecture to be optimal when $k$ is constant. 
In this work, we provide a constant-depth $\LAQCC$ circuit below their conjectured bound already for constant $k$. 
However, this does not directly disprove their conjecture, as we allow for intermediate measurements and classical computations. 
More significantly, we even construct constant-depth $\LAQCC$ circuits for $k = \mathO(\sqrt{n})$ greatly improving their bound.
This construction extends the compress-uncompress method for the $W$-state combined with additional subroutines. 

We continue with a log-depth state preparation protocol for the Dicke-state for arbitrary $k$. 
This protocol implements an efficient transformation between the factoradic number representation and the combinatorial number representation of a positive integer. 
The combinatorial number representation relates directly to the Dicke state. 
The provided efficient transformation between number representation systems might be of independent interest. 

We conclude by modifying our protocol for preparing a Dicke-state to a protocol that prepares quantum many-body scar states in constant-depth. 
These states have low entanglement and longer coherence times than states with similar energy density.
These characteristics make many-body scar states interesting to analyze and relevant within physics.
Many-body scar states appear for instance in the AKLT model~\cite{AKLT:1987,MRBAR:2018,MRB:2018} and different spin models~\cite{SI:2019,MOBFR:2020}.
Known methods for preparing these states have polynomial-depth~\cite{Gustafson:2023}, whereas our circuit has constant depth. 

% We conclude by studying the power that intermediate classical calculations can add to quantum computations. 
% In this study, we define a new model that relaxes constant-depth quantum circuits to polynomial depth quantum circuits, log-depth classical calculations to unbounded classical computations and a constant number of alternations to a polynomial number of alternations. 
% We call this model $\LAQCC^*$. 
% We study this model by doing a complexity theoretical analysis, where we draw inspiration from the notions of complexity given by \citeauthor{RosenthalYuen:2022}, \citeauthor{MetgerYuen:2023}, and \citeauthor{Aaronson:2004}.
% All three complexity notions are based on the notion of state preparation, instead of more traditional definition of complexity such as the decidability of a computational problem. 
% The first two consider classes based on sequences of quantum states preparable by a polynomial-sized quantum circuit, where the circuits are uniformly generated by a computational class, for instance, the class $\mathsf{PSPACE}$, which results in the complexity class $\mathsf{StatePSPACE}$~\cite{RosenthalYuen:2022,MetgerYuen:2023}.
% The third notion considers a relative complexity, where the complexity is measured between two given states, and is measured by the number of gates, from a given gate-set, required to transform one state in another state~\cite{Aaronson:2004}. 
% For our definition of state preparation complexity, we drop the uniformity constraint from~\cite{RosenthalYuen:2022,MetgerYuen:2023} and define a class as $\mathsf{StateX}$, which refers to states preparable by circuits of type $\mathsf{X}$. 
% As an example, if $\mathsf{X} = \QNC^0$, this results in the class $\mathsf{StateQNC^0}$, which is the set of states preparable from the $\ket{0}^n$ state by poly-size constant-depth circuits. 
% This notion is similar to the relative complexity from~\cite{Aaronson:2004}, where one state is the  $\ket{0}^n$ state and instead of counting the number of gates we consider the set of states preparable by a fixed number of gates. Using this notion of complexity we show that any state preparable by an $\LAQCC^*$ circuit is also preparable by a $\mathsf{PostQPoly}$ circuit, the class of circuits of polynomial depth with an additional post-selection gate. 

\paragraph{Summary of results}
\begin{itemize}
    \item We give a new definition of a computational model that captures the power of the four step process: applying a constant number of layers of one- and two-qubit gates; performing a syndrome measurement; perform a fast classical computation determining corrections; apply corrections. We call this model \emph{Local Alternating Quantum Classical Computations}, or $\LAQCC$ for short. In this model we bound the allowed quantum operations, intermediate classical calculations, and number of rounds separately. In Section~\ref{sec:LAQCC_model} we define this model and give a list of operations based on results from literature contained in this computational model. In some of these operations we explicitly use that we allow for multiple, but at most constant, rounds  of corrections.
    \item  We show show that there exist $\LAQCC$ circuits that can not be weakly simulated in Section~\ref{sec:IQP_in_LAQCC}. We further show that for every $\LAQCC$ circuit there exists a $\QNC^1$ circuit simulating it perfectly, in Section~\ref{sec:LAQCC_in_QNC1}.
    \item We introduce a new type computational complexity for preparing states and show that the extension of $\LAQCC$ where we allow a polynomial number of rounds and unbounded classical computation, is contained in $\mathsf{PostQPoly}$, the class of polynomial circuits with post-selection, in Section~\ref{sec:Complexity results}.
    \item We show a protocol to prepare the uniform superposition state of size $q$ in $\LAQCC$ using $\mathO(\ceil{\log_2(q)}^2)$ qubits in Section~\ref{sec:superposition_modulo_q}. 
    \item We show a protocol to prepare the $W_n$ state in $\LAQCC$ using $\mathO(n\log(n))$ qubits in Section~\ref{sec:W_state_in_LAQCC}.
    \item We show two ways of preparing the Dicke-$(n,k)$ state. The first method is in $\LAQCC$, works up to $k = \mathO(\sqrt{n})$, uses $\mathO(n^2\log(n))$ qubits, and is found in Section~\ref{sec:dicke:small_k}. The second method is in $\LAQCC\text{-}\mathsf{LOG}$ (an extension of $\LAQCC$ allowing for logarithmic number of alterations instead of constant), works for any $k$, uses $\mathO(\text{poly}(n))$ qubits, and is found in Section~\ref{sec:Dicke_in_LAQCC_LOG}. 
    \item We extend on our $\LAQCC$ method of generating Dicke-$(n,k)$ states for $k = \mathO(\sqrt{n})$ and show a protocol to generate many-body scar states for a particular Hamiltonian in $\LAQCC$ (Section~\ref{sec:many_body_scar}). 
\end{itemize}
Summarized in a table, we provide the following state generation protocols:
\begin{table}[htb]
\centering
\begin{tabular}{l|l|l|l}
\textbf{State description} & \textbf{Width} & \textbf{Depth} & \textbf{Implementation}\\
\hline 
Uniform superposition mod $q$: $\frac{1}{\sqrt{q}} \sum_{i = 0}^{q-1}\ket{i}$ & $\mathO(\ceil{\log^2 q})$ & $\mathO(1)$ & Section~\ref{sec:superposition_modulo_q}\\

$W$-state: $\frac{1}{\sqrt{n}}\sum_{i = 0}^{n-1}\ket{e_i}$ & $\mathO(n \log n)$ & $\mathO(1)$ & Section~\ref{sec:W_state_in_LAQCC}\\

Dicke-$(n,k)$, $k = \mathO(\sqrt{n})$: $\binom{n}{k}^{-1/2}\sum_{x \in \{0,1\}^n: |x| = k} \ket{x}$ &  $\mathO(n^2\log n)$ & $\mathO(1)$ 
&Section~\ref{sec:dicke:small_k}\\

Dicke-$(n,k)$: $\binom{n}{k}^{-1/2}\sum_{x \in \{0,1\}^n: |x| = k} \ket{x}$ & $\mathO(\text{poly}(n))$ & $\mathO(\log n)$ &Section~\ref{sec:Dicke_in_LAQCC_LOG}\\

QMBS: $\ket{S_k} = \frac{1}{k! \sqrt{\mathcal N(n,k)}}(Q^\dagger)^k \ket{\Omega}$ &  $\mathO(n^2\log n)$ & $\mathO(1)$  &  Section~\ref{sec:many_body_scar}
\end{tabular}
\caption{Summary of state preparation protocols given in this paper.}
\label{tab:sate_prep}
\end{table}
In the entry for the quantum many-body scar state $Q$ denotes the raising operator and $\mathcal N(n,k)=\binom{n-k-1}{k}$. 
Section~\ref{sec:many_body_scar} will provide more details on the variables and the implementation. 

\paragraph{Organization of the paper}
\noindent We first introduce relevant preliminaries in Section~\ref{sec:preliminaries}. 
In Section~\ref{sec:LAQCC_model} we formally define the class of Local Alternating Quantum-Classical Computations ($\LAQCC$). We also show that any Clifford circuit can be implemented in constant depth $\LAQCC$ (a result based on a result from measurement-based quantum computing~\cite{jozsa2006introduction}). 
This result allows us to give many useful multi-qubit gates and routines in Section~\ref{sec:gates_created_in_LAQCC}. 
Beyond that we show that constant depth $\LAQCC$ circuits are contained in $\QNC^1$ and that any $\mathsf{IQP}$ circuit has an $\LAQCC$ implementation.
We conclude this section with an analysis of a more powerful instantiation of $\LAQCC$ and show an inclusion with respect to the class $\mathsf{PostQPoly}$, which is the class of circuits of polynomial depth with one additional post-selection gate. 
In Section~\ref{sec:state_prep_in_LAQCC} we give $\LAQCC$ circuit implementations for preparing the uniform superposition over an arbitrary number of states, the $W$-state and the Dicke state up to $k = \mathO(\sqrt{n})$. We furthermore give a log-depth circuit implementation for preparing the Dicke state for any $k$. We conclude by showing a $\LAQCC$ circuit for generating many body scar states of a particular type of Hamiltonian.


% \vspacebeforesection
\section{Background}
\label{sec:background}

In this section, we provide the necessary background information to ensure a comprehensive understanding of the attack described in this paper. We start with a description of the Distributed Hash Table (DHT) used by IPFS, followed by its content resolution mechanisms. We also detail techniques for network size estimation, necessary for our attack detection and mitigation mechanisms.

\vspacebeforesection
\subsection{IPFS DHT}
\label{sec:kad_dht}

We review the features of the Kademlia DHT~\cite{maymounkov2002kademlia} and its \texttt{libp2p} implementation~\cite{libp2p_github} that are the most relevant to our attack.
To participate in the DHT, each peer generates a public/private key pair and derives an identity $\peerid \in \{0,1\}^{256}$ as the hash of its public key.
Ideally, each peer generates a random key pair and, therefore, peer IDs are distributed uniformly and independently over the space $\{0,1\}^{256}$.
While honest nodes follow this rule, malicious nodes may generate and choose from an arbitrary number of key pairs.
Each peer maintains a routing table consisting of $m=256$ buckets.
The $i$-th bucket contains the addresses of up to $k=20$ peers whose peer IDs share a common prefix of exactly $i$ bits with the peer's own peer ID. 

%
A new participant node joins the IPFS network by contacting one of the hardcoded bootstrap nodes. This bootstrap node provides the new node with some initial peers allowing it to join the DHT. The new node uses this information to perform a walk through the DHT towards its own peer ID.
The walk allows to: \textit{(i)}~make sure that there is no other node in the network with the same ID; \textit{(ii)}~discover new peers and fill the newcomer's DHT routing table. At the same time, the newcomer establishes \bitswap~\cite{de2021accelerating} connections to a subset of encountered peers (usually around 300 of them). The core role of the \bitswap protocol is to enable bilateral content transfer and to play the role of a cache for recently-accessed content.

The main DHT operation $\Call{GetClosestPeers}{\key}$ returns the $k=20$ closest peers to $\key$. 
%
In Kademlia, the distance between two keys $x$ and $y$ in the key space is given by $x \oplus y \in \{0,...,2^{256}-1\}$, where $\oplus$ denotes the bitwise XOR operation on the keys; the resulting binary string is interpreted as an integer.
%
When a client wants to find the peers with IDs closest to $\key$, it sends a request to the $\alpha=3$ peers in its routing table whose peer IDs are closest to $\key$. Each of these peers returns the $k$ closest peers to $\key$ in its own routing table and the addresses of these peers. 
%
The client again sends a request to the $\alpha$ peers closest to $\key$, among peers in its routing table and those whose addresses it just received. This process repeats until the client does not find any more peers closer to $\key$.
Due to network churn and imperfect routing tables, we observed in our experiments that successive calls to $\Call{GetClosestPeers}{\key}$ do not always return the same set of $k=20$ peers (we provide more details in \Cref{sec:evaluation}, \Cref{fig:20closest}). This is an important limitation affecting our attack.

\vspacebeforesection
\subsection{Content Resolution in IPFS}
\label{sec:ipfs}

IPFS is a content-centric network.
It allows its participant to request files without specifying their location. 
%
Content is indexed by content IDs $\cid \in \{0,1\}^{256}$ that are derived from a hash of that content.
Both peer IDs and CIDs are used as keys in the DHT.
Each node can play the role of a \provider, \downloader, or \resolver. 
The process of content advertisement and resolution is illustrated in \Cref{fig:add_get_provider}.

%
When a \provider wishes to publish content with a given $\cid$ on IPFS, it creates a \emph{provider record} that contains $cid$ and the \provider's address.
During a $\Call{Provide}{\cid}$ operation, the \provider first uses $\Call{GetClosestPeers}{\cid}$ to locate the $k=20$ peers with their peer IDs closest to $\cid$, 
%
and then sends them a $\mathsf{PutProvider}$ message including the provider record (\Cref{fig:add_get_provider}(a)).
We call the peers that hold provider records for $\cid$ the \emph{resolvers} for $\cid$.

Each CID can have several \providers. In fact, by default, each IPFS client becomes a provider for each piece of content it downloads for a fixed amount of time (12h, 24h, or 48h depending on the client version or custom configuration). As a result, the system provides an auto-scaling feature with supply automatically rising with demand.

%
When a \downloader wishes to fetch a piece of content, it first sends a request to all its \bitswap peers. If none of them has the content, the \downloader uses the DHT-based resolution system. We stress that the \bitswap protocol plays the supporting role of a cache in the dissemination of popular files. However, the mechanism does not provide reliable content resolution, in particular for new or less popular content. %

When \bitswap unstructured search fails, the \downloader resolves $\cid$ using $\Call{FindProviders}{\cid}$. This operation uses a DHT walk identical to that of $\Call{GetClosestPeers}{\cid}$ to find $k$ \resolvers but also queries encountered nodes for a provider record for $\cid$ (\Cref{fig:add_get_provider}(b)). The process terminates when either 20 \providers have been found, or all \resolvers have been asked. Querying all encountered nodes (\ie, not only the designated \resolvers) is useful because some of the encountered nodes may have a provider record in their cache.
%

Upon receiving a provider record, the client connects to the address specified in the provider record to retrieve the actual content (\Cref{fig:add_get_provider}(c)).
Provider records are not authenticated, and therefore malicious \providers may respond with incorrect provider records (or may not respond at all). However, the integrity of the content is preserved because the hash of the retrieved content can be verified against its $\cid$.
%


%

\input{img/add_get_provider.tex}

\vspacebeforesection
\subsection{Network Size Estimator}
\label{sec:netsize}

The number of nodes in a decentralized system is generally unknown due to the avoidance of centralized membership management.
This number is nonetheless useful for optimizations, deciding on individual node configurations, or security mechanisms.
Various methods were proposed for the decentralized estimation of unstructured and structured networks~\cite{eli-sohl-dht-size-estimation,kostoulas2005decentralized, manku2003symphony}.
We use in this work a mechanism developed initially by Protocol Labs as part of a mechanism for decreasing the latency of publishing content in IPFS~\cite{network-size-estimation-notion,network-size-estimation-github-pr}.

%
%
%
%
%
%
%
%
%
%

Each node in the DHT refreshes its routing table periodically (every $10$ minutes in \texttt{libp2p}). 
For this, the node samples $m$ random keys (one for each bucket of its routing table)
%
and queries the DHT to obtain the $k=20$ closest peer IDs to each key.
Using these, the node then computes the average distance between each one of these keys $\key_j$ for $j=1,\dots,m$ and their $i$-th closest peer ID for $i=1,...,k$ (with $m=256$ and $k=20$).
\begin{equation}
    \label{equ:avg-dist}
    \overline{D}_i = \frac{1}{m} \sum_{j=1}^m \operatorname{dist}(\key_j, \peerid_{j}^{(i)})
\end{equation}
where $\peerid_{j}^{(i)}$ is the $i$-th closest peer ID to $\key_j$.
With $N$ peers in the DHT and peer IDs uniformly distributed in the hash space, the expected distance between a $\key$ and its $i$-th closest peer ID is $\frac{2^{256}i}{N+1}$. The node then runs a least square regression to compute the value of $N$ for which the expected distances best fit the empirical average distances, \ie,
\begin{equation}
    \label{equ:netsize-least-squares}
    \hat{N} = \arg\min_{N} \sum_{i=1}^k \left(\overline{D}_i - \frac{2^{256}i}{N+1}\right)^2.
\end{equation}
The resulting estimate $\hat{N}$ can be computed in closed form.
%

When a node starts running, it must perform DHT queries for a few random keys to initialize its network size estimate. 
Since a larger number of queries will result in higher accuracy, making more queries than what is needed to initialize one's routing table is recommended.
Thereafter, keeping the estimate up-to-date does not require any excess DHT queries beyond what is already used for refreshing the routing table as this is done frequently (every 10 minutes).

While the network size estimate has a stochastic variance resulting from the probability distribution of the honest peer IDs, it is hard for an attacker to bias the estimate significantly. Since the estimator uses the density of peer IDs around keys chosen uniformly at random, the adversary would require numerous Sybil nodes (on the order of the whole network size) to significantly affect the peer ID density around those keys.

%
% ---- Bibliography ----
%
% BibTeX users should specify bibliography style 'splncs04'.
% References will then be sorted and formatted in the correct style.
%
\bibliographystyle{splncs04}
% \bibliography{mybibliography}
%
\bibliography{references}
% \begin{thebibliography}{8}
% \bibitem{ref_article1}
% Author, F.: Article title. Journal \textbf{2}(5), 99--110 (2016)

% \bibitem{ref_lncs1}
% Author, F., Author, S.: Title of a proceedings paper. In: Editor,
% F., Editor, S. (eds.) CONFERENCE 2016, LNCS, vol. 9999, pp. 1--13.
% Springer, Heidelberg (2016). \doi{10.10007/1234567890}

% \bibitem{ref_book1}
% Author, F., Author, S., Author, T.: Book title. 2nd edn. Publisher,
% Location (1999)

% \bibitem{ref_proc1}
% Author, A.-B.: Contribution title. In: 9th International Proceedings
% on Proceedings, pp. 1--2. Publisher, Location (2010)

% \bibitem{ref_url1}
% LNCS Homepage, \url{http://www.springer.com/lncs}. Last accessed 4
% Oct 2017
% \end{thebibliography}
\end{document}
