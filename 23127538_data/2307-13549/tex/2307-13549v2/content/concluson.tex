\section{Conclusion}

In this work, we build and share a planning ontology that provides a structured representation of concepts and relations for planning, allowing for efficient extraction of domain, problem, and planner properties. The ontology's practical utility is demonstrated in identifying the best-performing planner for a given domain and extracting macro operators using plan statistics and domain properties. Standardized benchmarks from IPC domains and planners offer an objective and consistent approach to evaluating planner performance, enabling rigorous comparisons in different domains to identify the most suitable planner. The planning ontology can aid researchers and practitioners in automated planning, and its use can simplify planning tasks and boost efficiency. As the field of AI planning continues to evolve, planning ontology can play a crucial role in advancing the state-of-the-art while leveraging the past.

% Future work could explore the use of mixed reasoning strategy with both ontologies (top-down) and Large Language Models (LLMs) (bottom-up) knowledge \cite{Mittal2017ThinkingFA-ontology}. For instance, one could leverage the planning ontology in the context of LLMs, which have recently shown promise for automated planning \cite{plansformer-paper-pallagani2022}. Moreover, the application of this mixed reasoning approach could be extended to complex domains, such as multi-agent systems, where coordinating actions between multiple agents is crucial.

Future work could explore the use of a mixed reasoning strategy that combines the structured, top-down approach of ontologies with the dynamic, bottom-up capabilities of Large Language Models (LLMs) \cite{Mittal2017ThinkingFA-ontology}. This approach can be particularly effective in contexts like LLMs, which have shown promise for automated planning \cite{plansformer-paper-pallagani2022}. Furthermore, our ontology, with its specific data properties for storing Action explanations, can be leveraged to enhance this hybrid model. It can provide comprehensive explanations for planning decisions as shown in the workflow Figure \ref{fig:use}, adding an interpretive layer that is crucial for complex domains such as multi-agent systems, where understanding the rationale behind each agent's actions is key. This blend of ontology-based clarity and LLM-driven adaptability could offer nuanced insights into coordinating actions and explaining them in a way that is both transparent and informative.

% In this work, we have presented a planning ontology that enables efficient extraction of domain, problem, and planner properties, and demonstrated its practical utility in identifying the best-performing planner for a given domain and extracting macro operators. The ontology can facilitate identifying and comparing planners' performance in specific domains, aiding researchers and practitioners in automated planning. As the field of AI planning continues to evolve, the further development of ontologies will play a crucial role in advancing the state-of-the-art and supporting the creation of intelligent, knowledge-based systems. Future work could explore the use of ontology in the context of LLMs and the development of hybrid systems combining ontology with other AI techniques. Moreover, complex domains like multi-agent systems could also be explored, where the coordination of actions between multiple agents is necessary.

% Automated planning is an essential component of intelligent systems, enabling machines to make decisions and perform tasks autonomously. 
% Over the years, significant progress has been made in the field leading to the development of a variety of planners that try to tackle an expanding range of complex applications in diverse domains. However, there was little work to organize this information systematically and draw insights to select and improve promising planners for a given domain. 

% In this work, we build and share a planning ontology that provides a structured representation of concepts and relations for planning, allowing for efficient extraction of domain, problem, and planner properties. The practical utility of the built ontology is demonstrated in two areas: identifying the best-performing planner for a given domain and extracting macro operators using plan statistics and domain properties. The use of standardized benchmarks from IPC domains and planners offers an objective and consistent approach to evaluating planner performance, enabling rigorous comparisons in different domains to identify the most suitable planner. The ontology can be used to more easily identify and compare planners' performance in specific domains, aiding researchers and practitioners in automated planning. Additionally, the extraction of macro operators from the ontology can simplify planning tasks and boost efficiency.
% As the field of AI planning continues to evolve with the emergence of new solution approaches (e.g., combining reinforcement learning and search, using large language models (LLMs)), the planning ontology can play a crucial role in advancing the state-of-the-art leveraging the past. 

% One can extend the work in many directions. First, following the trend of a mixed reasoning strategy with both  ontologies (top-down) and LLMs (bottom-up) knowledge \cite{Mittal2017ThinkingFA-ontology}, 
% one can support reasoning with the planning ontology in the context of LLMs that have started to show promise for automated planning \cite{plansformer-paper-pallagani2022}. Additionally, the development of hybrid systems that combine ontology with other AI techniques, such as machine learning, can lead to more robust and adaptable planning systems.

% Second, one can explore  more complex domains, such as multi-agent systems, where planning involves coordinating actions between multiple agents, comparing the problems, and finding relevant planners from past experiences. Additionally, the development of hybrid systems that combine ontology with other AI techniques, such as machine learning, can lead to more robust and adaptable planning systems. 
% As the field of AI planning continues to evolve, the further development of ontologies will play a crucial role in advancing the state-of-the-art and supporting the creation of intelligent, knowledge-based systems.

% In conclusion, the development of an ontology for automated planning represents a significant step toward the creation of more intelligent and efficient planning systems. The use of standardized benchmarks and the extraction of macro operators from the ontology can aid in evaluating planner performance and simplifying planning tasks. As the field of AI planning continues to evolve, the further development of ontologies will play a crucial role in advancing the state-of-the-art and supporting the creation of intelligent, knowledge-based systems.

%Over the years, significant progress has been made in the development of planners, which are algorithms that generate sequences of actions to achieve specific goals. However, the complexity of planning domains has presented significant challenges for researchers and practitioners in the field.

% There  the development of an ontology for automated planning has proven to be a valuable tool to capture the intricacies of the AI planning domain. 
% In recent years, the development of an ontology for automated planning has proven to be a valuable tool to capture the intricacies of the AI planning domain. 
 
 % The ontology provides a structured representation of concepts and relations for AI planning domains, allowing for efficient extraction of domain properties. 


% As the field of AI planning continues to evolve, the development of ontologies will undoubtedly play a crucial role in advancing the state-of-the-art. Further research and development in the ontology can lead to the creation of more intelligent and efficient planning systems. Ontology can be used to model knowledge from multiple domains and integrate it into a unified planning framework, thereby supporting the creation of intelligent, knowledge-based systems. 
% Future prospects for the development of ontology in automated planning 
% include the exploration of more complex domains, such as multi-agent systems, where planning involves coordinating actions between multiple agents, comparing the problems, and finding relevant planners from past experiences. Additionally, the development of hybrid systems that combine ontology with other AI techniques, such as machine learning, can lead to more robust and adaptable planning systems.

% In conclusion, the development of an ontology for automated planning has proven to be a valuable tool to capture the intricacies of the AI planning domain. By providing a structured representation of concepts and relations for AI planning domains, the ontology allows efficient extraction of domain properties. Moreover, ontology has been shown to have practical applications in two areas: identifying the best-performing planner for a given domain and extracting macro operators using plan statistics and domain properties.

% Standardized benchmarks from IPC domains and planners offer an objective and consistent approach to evaluating planner performance, enabling rigorous comparisons in different domains to identify the most suitable planner. Standardized benchmarks also aid in experiment replication and comparison of results, which is essential to advance automated planning. Researchers and practitioners in automated planning can use ontology to more easily identify and compare planners' performance in specific domains. Macro operators extracted from the ontology can simplify planning tasks and boost efficiency. These tools help automate planning for real-world problems.

% As the field of AI planning continues to evolve, the development of ontologies will undoubtedly play a crucial role in advancing the state-of-the-art. With further research and development, ontologies could potentially be used to support a wide range of applications, from automated decision-making to intelligent systems in various industries. Ultimately, the development of ontologies for automated planning represents a significant step toward the creation of more intelligent and efficient planning systems.

% The use of benchmarks from IPC domains and planners provides a standardized and objective means of evaluating the performance of planners. This allows for a more rigorous comparison of planners based on their performance in different domains, which is crucial to identifying the most suitable planner for a specific planning task. Moreover, the use of standardized benchmarks enables researchers to replicate experiments and compare results, which is essential for advancing the field of automated planning. Through the use of ontology, researchers, and practitioners in the field of automated planning can more easily identify and compare planners based on their performance in specific domains. Additionally, the ability to extract macro operators from the ontology provides a means to simplify planning tasks and increase efficiency.



%%%%%%%%%%%%%%%%%%%%%%%%%%%%%%%%%%%%%%%%%%%%%%%%%%%%%