\section{Introduction}

Automated planning, where the objective is to find a sequence of actions that will transition an agent from the initial state of the world to a desired goal state, is an active sub-field of Artificial Intelligence (AI). The ability to generate plans and make decisions in complex domains, such as robotics, logistics, and manufacturing, has led to significant progress in the automation of planning. Currently, there are numerous planning domains, planners, search algorithms, and associated heuristics in the field of automated planning. Each planner, in conjunction with a search algorithm and heuristic, generates plans with varying degrees of quality, cost, and optimality. The empirical results available for various planning problems, ranked by planner performance and the heuristics used as available in International Planning Competition (IPC), can provide valuable information to identify various tunable parameters to improve planner performance. Traditionally, improving planner performance involves manually curating potential combinations to identify the optimal planner configuration. However, there has been limited effort to model the available information in a structured knowledge representation, such as an ontology, to facilitate efficient reasoning and enhance planner performance.




% Automating planning, where the objective is to find a
% sequence of actions that will transform an initial state of the world into a desired goal state, 
% has been a long-standing objective in the field of Artificial Intelligence (AI). The ability to generate plans and make decisions in complex domains, such as robotics, logistics, and manufacturing, has led to significant progress in the automation of planning. Currently, there are numerous planning domains, planners, search algorithms, and associated heuristics in the field of automated planning. Each planner, in conjunction with a search algorithm and heuristic, generates plans with varying degrees of quality, cost, and optimality. The empirical results available for various planning problems, ranked by planner performance and the heuristics used as available in IPC, can provide valuable information to identify various tunable parameters to improve planner performance. Traditionally, improving planner performance involves manually curating potential combinations to identify the optimal planner configuration. However, there has been limited effort to model the available information in a structured knowledge representation, such as an ontology, to facilitate efficient reasoning and further enhance planner performance.

% However, selecting the best-performing planner for a given planning domain remains a challenge due to the vast number of available planners and the diversity of planning domains. Traditional approaches to this challenge involve comparing the performance of different planners on a set of benchmark problems. However, this approach may not always be effective as the benchmarks may not accurately reflect the characteristics of the target domain.

To address the challenge of representing planning problems and associated information in a structured manner, we propose an ontology for AI planning. An ontology formally represents concepts and their relationships \cite{guarino2009ontology}, which enables systematic analysis of planning domains and planners. The proposed ontology captures the features of a domain and the capabilities of planners, facilitating reasoning with existing planning problems, identifying similarities, and suggesting different planner configurations. Planning ontology can also be a useful resource for the creation of new planners as it captures essential information about planning domains and planners, which can be leveraged to design more efficient planning algorithms. Furthermore, ontology can promote knowledge sharing and collaboration within the planning community.
  
%\biplav{Previous work on plan ontologies.}
In the field of planning, several attempts have been made to create ontologies to enhance the understanding of planners' capabilities. For instance, Plan-Taxonomy \cite{plan-taxonomy} introduced a taxonomy that aimed to explain the functionality of planners. Additionally, authors in \cite{gil2000planet} present a comprehensive ontology called PLANET, which represents plans in real-world domains and can be leveraged to construct new applications. Nonetheless, the reusability of PLANET is limited as it is not open-sourced. Consequently, researchers face difficulty in extending or replicating the ontology.

% The ontology can also help in the development of new planners by providing a systematic representation of the planning domain and the capabilities of existing planners.
% To address this challenge of representing planning problems and associated information in a structured representation, we present an ontology for AI Planning to represent the features of the planning domains and planner capabilities. An ontology is a formal representation of a set of concepts and the relationships between them. By representing planning domains and planners in a structured way, we can systematically analyze their characteristics, making it easier to identify patterns and dependencies. Furthermore, the ontology can be used to share knowledge and best practices among the planning community, improving collaboration and knowledge sharing.

% By constructing an ontology for AI planning domains, we can systematically capture the features of a domain and the capabilities of planners. Given a new planning problem, the proposed ontology for AI planning can be used to reason with existing planning problems, find similarities and suggest different planner configurations. The ontology-based approach can also facilitate the development of new planners by providing a systematic representation of the planning domain and the capabilities of existing planners.

This paper outlines our methodology for constructing an ontology to represent AI planning domains, leveraging information obtained from the IPC. In our current work, we extended our initial research \cite{planning-ontology}. Specifically, we have enhanced the ontology to more accurately depict the various concepts within the planning domain. Furthermore, we include additional use cases of our ontology and provide experimental evaluations to support our findings further. Building a planning ontology using data from IPC offers several benefits, such as comprehensive coverage of planning domains, a rich source for various benchmark evaluation metrics, and documentation for planners. However, the ontology is not limited to the PDDL representation or  domains in IPC and can easily  be extended to any. % planning domain. 
%Our paper makes  important 
Our contributions are at the intersection of ontologies and AI planning and can be summarized as follows.
\begin{itemize}
    \item \textbf{Planning Ontology}: We developed an ontology for AI planning that can be used to represent and organize knowledge related to planning problems. We designed the competency questions to ensure that our ontology provides a structured way to capture the relationships between different planning concepts, enabling more efficient and effective knowledge sharing and reuse.

    \item \textbf{Usecase 1: Identifying Most Promising Planner for Performance}: We demonstrate the ontology's usage for identifying the best-performing planner for a specific planning domain using data from IPC-2011.

        % \item \textbf{Performance Evaluation}: In our paper, we utilized the ontology we developed to extract the best-performing planner for a specific planning domain. We demonstrate this by considering the data of IPC-2018.

    \item \textbf{Usecase 2: Macro Selection for Improving Planner Performance}: We demonstrate the usage of ontology to extract domain-specific macros - which are action orderings and show that they can improve planner performance drastically.

        % \item \textbf{Macro Extraction}: We proposed a method to extract domain-specific macros, which are action rules that simplify the planning process. Our approach involves using the ontology to identify relevant domain properties and plan statistics, which guide the macros extraction process. By reducing the search space, our approach improves planning efficiency.
\end{itemize}

In the remainder of the paper, we start with preliminaries about on automated planning and IPC. We then give an overview of the existing literature on ontologies for planning. Following this, we present a detailed description of the ontology construction process and its usage. We then discuss the proposed planning ontology and conclude with future research directions.