\section{Related Work}


The use of ontology-based knowledge representation and reasoning has been extensively studied in various domains, including automated planning. This section focuses on the applications of ontology-based knowledge representation and reasoning in the context of planning and related domains.
In \cite{valente1999building}, an ontology is constructed for the Joint Forces Air Component Commander (JFACC) to represent knowledge from the air campaign domain. The ontology is modularized to facilitate data organization and maintenance, but its applicability is domain-specific, unlike our approach. In \cite{vzakova2010automating}, the authors automate the knowledge discovery workflow using ontology and AI planning, creating a Knowledge Discovery (KD) ontology to represent the KD domain and converting its variables to a Planning Domain Definition Language (PDDL) format to obtain the PDDL domain. The ontology's objects represent initial and goal states, forming the KD task, which represents a specific problem. The authors use the Fast-Forward (FF) planning system to generate the required plans.

In a survey of ontology-based knowledge representation and reasoning in the planning domain, \cite{gayathri2018ontology} suggests that knowledge reasoning approaches can draw new conclusions in non-deterministic contexts and assist with dynamic planning. In \cite{gil2000planet}, a reusable ontology, PLANET, is proposed for representing plans. PLANET includes representations for planning problem context, goal specification, plan, plan task, and plan task description. However, PLANET does not include representations for some entities commonly associated with planning domains, such as resources and time. Our planning ontology draws inspiration from PLANET and appends more metadata for planner improvement.
In \cite{babli2019extending}, a domain-independent approach is presented that advances the state of the art by augmenting the knowledge of a planning task with pertinent goal opportunities. The authors demonstrate that incorporating knowledge obtained from an ontology can aid in producing better-valued plans, highlighting the potential for planner enhancement using more tuning parameters, which are captured in our planning ontology. The CARESSES ontology \cite{khaliq2018culturally} is another significant development in planning-oriented ontologies, focusing on cultural competence in socially assistive robots for elderly care. Our work incorporates aspects from this ontology, specifically the concepts of \texttt{Action} and \texttt{Parameter}. In general, these studies demonstrate the potential of ontology-based knowledge representation and reasoning in the planning domain, including applications such as representing plans, aiding in air campaign planning, automating knowledge discovery workflows, and developing context-aware planning services.


\begin{table}[b]
\centering
\caption{Concepts reused from various ontologies}
\begin{tabular}{ll}
\hline
\textbf{Concept} & \textbf{Ontology} \\ \hline
Action           & http://caressesrobot.org/ontology \cite{khaliq2018culturally} \\ 
Parameter        & http://caressesrobot.org/ontology \cite{khaliq2018culturally} \\ 
Plan             & http://www.ontologydesignpatterns.org/ont/dul/DUL.owl \cite{mascardi2008comparison} \\
State            & http://purl.org/vocab/lifecycle/schema \\

\hline
% Add more rows as needed
\end{tabular}
\label{tab:ontology-concepts}
\end{table}
