\section{Usage of Planning Ontology}
In the following section, we show the evaluation of a few competency questions and discuss two use cases of our planning ontology.
% \vspace{-0.3cm}

% Figure environment removed

\subsubsection{Evaluation of Competency questions:}
For the evaluation of the competency questions, we have considered a sample knowledge graph, shown in Figure \ref{fig:bw_kg}, for \verb|blocksworld| from IPC-2000 domain created using planning ontology shown in Figure \ref{fig:ontology}. SPARQL queries for each of these questions can be found at our GitHub Repository\textsuperscript{\ref{footnote: repo}}.
% \begin{enumerate}
%     \item C1: What are the different types of planners used in automated planning?\\
%     \textbf{Question Type:} Extracting planner information.\\
%     \textbf{Sufficiency Condition:} There should exist at least one individual for \verb|Planner| class. \\
%     \textbf{Result:} Shown in Table \ref{tab:c1}.  
%     \begin{table}[!h]
%         \centering
%         % \vspace{-0.5cm}
%         \begin{tabular}{ll}
%             \hline
%             \textbf{S.No} & \textbf{Planner} \\ \hline
%             1 & FF \\
%             2 & FastDownward \\
%             3 & LPG \\ \hline
%         \end{tabular}
%         % \vspace{0.2cm}
%         \caption{Results for C1 with knowledge graph in Figure \ref{fig:bw_kg}}
%         \label{tab:c1}
%         \vspace{-0.2cm}
%     \end{table}

%     \item C2: What is the relevance of planners in 'blocksworld' domain? \\
%     \textbf{Question Type:} Extracting best planner for a domain. \\
%     \textbf{Sufficiency Condition:} There should exist at least one \verb|Planner| individual having either of the relevance properties with 'blocksworld' individual of \verb|PlanningDomain| class. \\
%     \textbf{Result:} Shown in Table \ref{tab:c2}. 

%     \begin{table}[!h]
%         \centering
%         % \vspace{-0.5cm}
%         \begin{tabular}{llll}
%             \hline
%             \textbf{S.No} & \textbf{Domain} & \textbf{Relation} & \textbf{Planner} \\ \hline
%             1 & blocksworld & hasLowRelevancePlanner & CPT4 \\
%             2 & blocksworld & hasHighRelevancePlanner & FastDownward \\
%             3 & blocksworld & hasMediumRelevancePlanner & LPG \\ \hline
%         \end{tabular}
%         % \vspace{0.2cm}
%         \caption{Results for C2 with knowledge graph in Figure \ref{fig:bw_kg}}
%         \label{tab:c2}
%         % \vspace{-1cm}
%     \end{table}

%     \item C3: What are the available actions for 'blocksworld' domain? \\
%     \textbf{Question Type:} Extracting domain information. \\
%     \textbf{Sufficiency Condition:} For the 'blocksworld' individual of \verb|PlanningDomain|, there must be at least one \verb|DomainAction| individual with the relation \verb|hasAction|. \\
%     \textbf{Result:} Shown in Table \ref{tab:c3}. 
%     \begin{table}[!h]
%         \centering
%         % \vspace{-0.3cm}
%         \begin{tabular}{llll}
%             \hline
%             \textbf{S.No} & \textbf{Domain} & \textbf{Relation} & \textbf{Action} \\ \hline
%             1 & blocksworld & hasAction & put-down \\
%             2 & blocksworld & hasAction & pick-up \\
%             3 & blocksworld & hasAction & stack \\
%             4 & blocksworld & hasAction & unstack \\ \hline
%         \end{tabular}
%         % \vspace{0.2cm}
%         \caption{Results for C3 with knowledge graph in Figure \ref{fig:bw_kg}}
%         \label{tab:c3}
%         % \vspace{-0.7cm}
%     \end{table}
    
%     \item C4: Which problems in 'blocksworld' have problems with the goal state of 'b1' being on the table? \\
%     \textbf{Question Type:} Extracting problem information \\
%     \textbf{Sufficiency Condition:} For the 'blocksworld' individual of \verb|PlanningDomain|, there must be at least one \verb|PlanningProblem| individual with the relation \verb|hasProblem| and the problem should have '(ontable b1)' \verb|GoalState|.\\
%     \textbf{Result:} Shown in Table \ref{tab:c4}. 
%     \begin{table}[!h]
%         \centering
%         % \vspace{-0.5cm}
%         \begin{tabular}{llll}
%             \hline
%             \textbf{S.No} & \textbf{Domain} & \textbf{Relation} & \textbf{Problem} \\ \hline
%             1 & blocksworld & hasProblem & problem\_3\_1 \\ \hline
%         \end{tabular}
%         % \vspace{0.2cm}
%         \caption{Results for C4 with knowledge graph in Figure \ref{fig:bw_kg}}
%         \label{tab:c4}
%         % \vspace{-0.7cm}
%     \end{table}
    
%     \item C5: What are all requirements a given domain has? \\
%     \textbf{Question Type:} Extracting domain information \\
%     \textbf{Sufficiency Condition:} For the 'blocksworld' individual of \verb|PlanningDomain|, there must exist at least one \verb|DomainRequirement| individual with the relation \verb|hasRequirement|.\\
%     \textbf{Result:} Shown in Table \ref{tab:c5}. 

%     \begin{table}[!h]
%         \centering
%         % \vspace{-0.5cm}
%         \begin{tabular}{llll}
%             \hline
%             \textbf{S.No} & \textbf{Domain} & \textbf{Relation} & \textbf{Requirement} \\ \hline
%             1 & blocksworld & hasRequirement & :strips \\ \hline
%         \end{tabular}
%         % \vspace{0.2cm}
%         \caption{Results for C5 with knowledge graph in Figure \ref{fig:bw_kg}}
%         \label{tab:c5}
%         % \vspace{-1cm}
%     \end{table}

%     \item C6: What is the cost associated with generating a plan for a given problem? \\
%     \textbf{Question Type:} Extracting plan cost information. \\
%     \textbf{Sufficiency Condition:} There must exist at least one \verb|Plan| individual with with data property \verb|hasPlanCost|. \\
%     \textbf{Result:} Shown in Table \ref{tab:c6}. 
%     \begin{table}[!h]
%         \centering
%         \begin{tabular}{llll}
%             \hline
%             \textbf{S.No} & \textbf{Plan} & \textbf{Relation} & \textbf{Cost} \\ \hline
%             1 & plan\_3\_1 & hasPlanCost & 6 \\
%             % Add additional rows as necessary
%             \hline
%         \end{tabular}
%         \caption{Results for C6 with knowledge graph in Figure \ref{fig:bw_kg}}
%         \label{tab:c6}
%     \end{table}
    
%     \item C7: How many parameters does a specific action have? \\
%     \textbf{Question Type:} Extracting action parameter information. \\
%     \textbf{Sufficiency Condition:} There must exist at least one \verb|DomainAction| individual with an associated \verb|ActionParameter|. \\
%     \textbf{Result:} Shown in Table \ref{tab:c7}. 
%     \begin{table}[!h]
%         \centering
%         \begin{tabular}{llll}
%             \hline
%             \textbf{S.No} & \textbf{Action} & \textbf{Parameter-Count} \\ \hline
%             1 & pickup & 2 \\
%             % Add additional rows as necessary
%             \hline
%         \end{tabular}
%         \caption{Results for C7 with knowledge graph in Figure \ref{fig:bw_kg}}
%         \label{tab:c7}
%     \end{table}
    
%     \item C8: What planning type a specific planner belongs to? \\
%     \textbf{Question Type:} Extracting planner type information. \\
%     \textbf{Sufficiency Condition:} There should exist at least one \verb|Planner| individual with an associated \verb|PlanningType|. \\
%     \textbf{Result:} Shown in Table \ref{tab:c8}. 
%     \begin{table}[!h]
%         \centering
%         \begin{tabular}{llll}
%             \hline
%             \textbf{S.No} & \textbf{Planner} & \textbf{Relation} & \textbf{PlannerType} \\ \hline
%             1 & FastDownward & ofPlannerType & Classical Planner \\
%             % Add additional rows as necessary
%             \hline
%         \end{tabular}
%         \caption{Results for C8 with knowledge graph in Figure \ref{fig:bw_kg}}
%         \label{tab:c8}
%     \end{table}
    
%     \item C9: What requirements does a given planner support? \\
%     \textbf{Question Type:} Extracting planner requirement information. \\
%     \textbf{Sufficiency Condition:} There must exist at least one \verb|Planner| individual with associated \verb|DomainRequirements|. \\
%     \textbf{Result:} Shown in Table \ref{tab:c9}. 
%     \begin{table}[!h]
%         \centering
%         \begin{tabular}{llll}
%             \hline
%             \textbf{S.No} & \textbf{Planner} & \textbf{Relation} & \textbf{DomainRequirement} \\ \hline
%             1 & FastDownward & solvesRequirement & :strips \\
%             % Add additional rows as necessary
%             \hline
%         \end{tabular}
%         \caption{Results for C9 with knowledge graph in Figure \ref{fig:bw_kg}}
%         \label{tab:c9}
%     \end{table}

%     \item C10: What are the different \verb|ParameterType| present in a domain? \\
%     \textbf{Question Type:} Extracting domain type information. \\
%     \textbf{Sufficiency Condition:} There must exist at least one \verb|PlanningDomain| individual with associated \verb|ParameterType|. \\
%     \textbf{Result:} Shown in Table \ref{tab:c10}. 
%     \begin{table}[!h]
%         \centering
%         \begin{tabular}{llll}
%             \hline
%             \textbf{S.No} & \textbf{Domain} & \textbf{Relation} & \textbf{ParameterType} \\ \hline
%             1 & blocksworld & hasParameterType & block \\
%             2 & blocksworld & hasParameterType & table \\
%             % Add additional rows as necessary
%             \hline
%         \end{tabular}
%         \caption{Results for C10 with knowledge graph in Figure \ref{fig:bw_kg}}
%         \label{tab:c10}
%     \end{table}

    
% \end{enumerate}


\begin{table}[!t]
\centering
\caption{Demonstrating the effectiveness of two different policies employed to choose a planner for problem-solving.}
\begin{tabular}{lcccc}
\hline
\multicolumn{1}{c}{\textbf{Domain}} & \multicolumn{2}{c}{\textbf{Ontology Policy}} & \multicolumn{2}{c}{\textbf{Random Policy}}  \\ \cline{2-5} 
\multicolumn{1}{c}{}                                 & \multicolumn{1}{c}{Avg. Exp.}  & Avg. Plan Cost & \multicolumn{1}{c}{Avg. Exp} & Avg. Plan Cost \\ \hline
scanalyzer                                             & \multicolumn{1}{c}{\textbf{8,588}}    & 20                                 & \multicolumn{1}{c}{8,706}    & 20                                 \\ 
elevators                                              & \multicolumn{1}{c}{\textbf{1,471}}    & 52                                 & \multicolumn{1}{c}{64,541}   & 52                                 \\ 
transport                                              & \multicolumn{1}{c}{165,263}           & 491                                & \multicolumn{1}{c}{\textbf{132,367}}  & 491                                \\ 
parking*                                               & \multicolumn{1}{c}{\textbf{367,910}}  & 18                                 & \multicolumn{1}{c}{488,830}  & 17                                 \\ 
woodworking                                            & \multicolumn{1}{c}{\textbf{1,988}}    & 211                                & \multicolumn{1}{c}{19,844}   & 211                                \\ 
floortile**                                            & \multicolumn{1}{c}{283,724}           & 54                                 & \multicolumn{1}{c}{\textbf{2,101}}    & 49                                 \\ 
barman                                                 & \multicolumn{1}{c}{\textbf{1,275,078}} & 90                                 & \multicolumn{1}{c}{5,816,476} & 90                                 \\ 
openstacks                                             & \multicolumn{1}{c}{\textbf{132,956}}  & 4                                  & \multicolumn{1}{c}{139,857}  & 4                                  \\ 
nomystery                                              & \multicolumn{1}{c}{1,690}             & 13                                 & \multicolumn{1}{c}{1,690}    & 13                                 \\ 
pegsol                                                 & \multicolumn{1}{c}{\textbf{89,246}}   & 6                                  & \multicolumn{1}{c}{101,491}  & 6                                  \\ 
visitall                                               & \multicolumn{1}{c}{5}                & 4                                  & \multicolumn{1}{c}{5}       & 4                                  \\ 
tidybot**                                              & \multicolumn{1}{c}{\textbf{1,173}}    & 17                                 & \multicolumn{1}{c}{3,371}    & 33                                 \\ 
parcprinter                                            & \multicolumn{1}{c}{541}              & 441,374                             & \multicolumn{1}{c}{\textbf{417}}     & 441,374                             \\ 
sokoban                                                & \multicolumn{1}{c}{\textbf{9,653}}    & 25                                 & \multicolumn{1}{c}{156,600}  & 25                                 \\ \hline
\end{tabular}
% \vspace{0.2cm}
\label{tab:best_planner_eval}
% \vspace{-0.5cm}
\end{table}

\subsubsection{Usecase 1: Identifying Most Promising Planner} - 
One of the major challenges in the field of artificial intelligence (AI) is the automated selection of the best-performing planner for a given planning domain. This challenge arises due to the vast number of available planners and the diversity of planning domains. The traditional way to select a planner is to experiment with various search algorithms and heuristics and settle on an appropriate combination as seen in IPC competitions. To address this challenge, we now present a new approach by using our planning ontology to represent the features of the planning domain and the capabilities of planners.

The ontology for planning aims to capture the connection between the Planning Domain and the Planner by indicating the relevance of a planner to a specific domain. We made use of data acquired from International Planning Competitions (IPCs) to furnish specific details regarding the relevance of planners. The IPC results provide us with relevant details on the planners that took part in the competition and the domains that were evaluated during that particular year. This information includes specifics on how each planner performed against all the domains that participated.

To show the usage of extracting the most promising planners for a given domain, we have used IPC-2011 data\footnote[3]{\label{ipc-2011}http://www.plg.inf.uc3m.es/ipc2011-deterministic/} (optimal track). The ontology was populated with data acquired from the IPC-2011, which provided relevant details on the planners that took part in the competition and the domains that were evaluated during IPC-2011. A relevance relation of either \textit{low}, \textit{medium}, or \textit{high} was assigned to each planner based on the percentage, \textit{low-}below 35\%, \textit{medium-}35\% to 70\%, \textit{high-}70\% and above, of problems they solved in a given domain. In this experiment, we consider that the experimental environment has four planners available: Fast Downward Stone Soup 1\footnote[4]{\label{fd_ipcPlanners}https://www.fast-downward.org/IpcPlanners}, LM-Cut\textsuperscript{\ref{fd_ipcPlanners}}, Merge and Shrink\textsuperscript{\ref{fd_ipcPlanners}}, and BJOLP\textsuperscript{\ref{fd_ipcPlanners}}. We evaluate 3 problem instances of each domain from IPC-2011 with 2 policies for selecting planners to generate plans for each of these problem instances - 
\begin{enumerate}
    \item \textbf{Random Policy:} To solve each problem instance, this policy selects a random planner from the available planners.
    \item \textbf{Ontology Policy:} To solve each problem instance, this policy extracts the information on the best planner for the problem domain from the ontology populated with IPC-2011 data.
\end{enumerate}

Table \ref{tab:best_planner_eval} presents the results of our evaluation, indicating the average number of nodes expanded and plan cost for each policy in a given domain.
%The evaluation results are presented in Table \ref{tab:best_planner_eval}, which provides details on the average number of nodes expanded and the average plan cost for each policy in a given domain. 
The table provides a comprehensive summary of the performance of different planners in terms of their efficiency and effectiveness. An ideal planner is expected to generate a solution with low values for both these metrics. 
%By comparing the performance of the planners selected using the two policies, we demonstrate the effectiveness of the Ontology Policy in selecting the best-performing planner for a given planning domain. 
The {\em Ontology Policy}, designed to select the best-performing planner for a given domain, outperformed the {\em Random Policy} in terms of the average number of nodes expanded to find a solution. Moreover, the {\em Random Policy} failed to solve problems in the parking (1 out of 3), floortile (2 out of 3), and tidybot (2 out of 3) domains, which highlights the limitations of choosing a planner randomly. But if a domain is easily solvable by relevant planners that can tackle them, {\em Random Policy} may still do well. 

What we demonstrate is a rather simple usage of the Ontology for Planner Selection policy. Creating more advanced strategies is a promising area for further research.

% Our ontology is constructed using data collected from the International Planning Competition (IPC) and planner information. This enables us to represent the knowledge about the planning domain and the planners in a structured format. This structured representation of knowledge can be used to automate the planner selection process by making use of reasoning techniques. We have created a knowledge graph from the data on planning domains and planner performance from IPC-2018. As an example, Table \ref{tab:domain-relevance-planner} shows the triples extracted from the knowledge graph for the domain \verb|caldera|, indicating the relevance of the planners presented in IPC-2018.

% Please add the following required packages to your document preamble:
% \usepackage{multirow}
% \begin{table}[htbp]
%     \centering
%     \begin{tabular}{>{\columncolor{blue!20}}c|>{\centering\arraybackslash}c}
%         Cell 1, Row 1 & Cell 2, Row 1 \\
%         \hline
%         \rowcolor{green!20} Cell 1, Row 2 & Cell 2, Row 2 \\
%         Cell 1, Row 3 & \cellcolor{red!20}Cell 2, Row 3 \\
%     \end{tabular}
% \end{table}

% % Figure environment removed

\subsubsection{Usecase 2: Extracting Macro Operators} -
While automated planning has been successful in many domains, it can be computationally expensive, especially for complex problems. One approach to improve efficiency is by using macro-operators, which are sequences of primitive actions that can be executed as a single step. However, identifying useful macro-operators manually can be time-consuming and challenging. Authors in \cite{chrpa2010generation} introduce a novel method for improving the efficiency of planners by generating macro-operators. The proposed approach involves analyzing the inter-dependencies between actions in plans and extracting macro-operators that can replace primitive actions without losing the completeness of the problem domain. The soundness and complexity of the method are assessed and compared to other existing techniques. The paper asserts that the generated macro-operators are valuable and can be seamlessly integrated into planning domains without losing the completeness of the problem. In \cite{botea2005learning}, the authors detail a three-step method for learning and utilizing macro-operators to enhance planning efficiency in new problems. Initially, a comprehensive set of macros is generated from the solution graphs of various training problems. This set is then narrowed down through a filtering process. The selected macros are subsequently applied to expedite problem-solving. The generation phase involves extracting and selecting specific subgraphs from solution graphs to create individual macros.

Based on the ontology depicted in Figure \ref{fig:ontology}, we extract macro-operators that can enhance the efficiency of planners. To demonstrate this, we have considered three different domains: \verb|blocksworld|(bw), \verb|driverlog|(dl), and \verb|grippers|(gr), presented in IPC-2000, 2002, and 1998 respectively. We initially developed a knowledge graph using the ontology represented in Figure \ref{fig:ontology} for the three domains of interest. Subsequently, we employed a SPARQL query to retrieve the stored plans for these domains. We then examined these plans to identify the sequences of action pairs and ranked them based on their frequency of occurrence. To improve the effectiveness of this technique, it is essential to consider both the frequency of occurrence of action pairs and the properties of the domain. Specifically, the precondition and effect of actions should be analyzed to ensure that the first action leads to the precondition of the second action in the pair. We employed another SPARQL query to extract the preconditions and effects associated with each of these actions. We analyzed the resulting action pairs to verify their validity of occurrence, thereby filtering out pairs that did not have a combined effect. The results of this extraction process are shown in Table \ref{tab: macros}. These action relations are stored back into the knowledge graph in the \verb|MacroAction| class and can be utilized by planners to enhance their efficiency.

% Macros - branching Factor - Action Selection process is not better informed (heuristic) --> degrade performance 

\begin{table}[t]
\centering
\caption{Extracted action relations, ordered based on their frequency, for domains \texttt{blocksworld}, \texttt{driverlog}, and \texttt{grippers}.}
\begin{tabular}{p{2.5cm}p{9cm}}
\hline
\textbf{Domains} & \textbf{Extracted Action Relations} \\
\hline
\cellcolor{blue!25}\texttt{blocksworld} & \texttt{unstack} * \texttt{put-down}; \texttt{pick-up} * \texttt{stack}; \texttt{put-down} * \texttt{unstack}; \texttt{stack} * \texttt{pick-up}; \texttt{unstack} * \texttt{stack}; \texttt{put-down} * \texttt{pick-up}; \texttt{stack} * \texttt{unstack}\\
\hline
\cellcolor{green!25}\texttt{driverlog} & \texttt{drive-truck} * \texttt{unload-truck}; \texttt{drive-truck} * \texttt{load-truck}; \texttt{board-truck} * \texttt{drive-truck}; \texttt{walk} * \texttt{board-truck}\\
\hline
\cellcolor{red!25}\texttt{grippers} & \texttt{pick} * \texttt{move}; \texttt{move} * \texttt{drop}\\
\hline
\end{tabular}
% \vspace{0.2cm}

\label{tab: macros}
% \vspace{-0.5cm}
\end{table}

\begin{table}[b]
\centering
\begin{tabular}{lcccccc}
\hline
 \multicolumn{1}{c}{\textbf{Domain}} & \multicolumn{3}{c}{\textbf{Original Domain}} & \multicolumn{3}{c}{\textbf{Domain With Macros}} \\ \cline{2-7}
\multicolumn{1}{c}{} & \multicolumn{1}{c}{Avg. Exp.} & \multicolumn{1}{c}{Avg. Eval.} & Avg. Gen. & \multicolumn{1}{c}{Avg. Exp.} & \multicolumn{1}{c}{Avg. Eval.} & Avg. Gen. \\ \hline
\textbf{blocksworld} & \multicolumn{1}{c}{20219}      & \multicolumn{1}{c}{59090}      & 106321     & \multicolumn{1}{c}{18}        & \multicolumn{1}{c}{310}        & 359       \\
\textbf{gripper} & \multicolumn{1}{c}{2672}      & \multicolumn{1}{c}{10660}      & 30871     & \multicolumn{1}{c}{510}        & \multicolumn{1}{c}{3974}        & 11468       \\
\textbf{driverlog} & \multicolumn{1}{c}{3753}      & \multicolumn{1}{c}{17849}      & 45753     & \multicolumn{1}{c}{14888}        & \multicolumn{1}{c}{720008}        & 209760       \\ \hline

\end{tabular}

% \vspace{0.2cm}
\caption{Comparison of planner performance between original and macro-enabled versions of three planning domains, showing the average number of nodes expanded, evaluated, and generated.}

\label{tab: macros_results}
% \vspace{-0.7cm}
\end{table}

Table \ref{tab: macros_results} shows the comparison of a planner performance given the original domain and macros-enabled version of the domain. For this evaluation, we have considered the FastDownward planner \cite{helmert2006fast} with LM-Cut Heuristic \cite{helmert2011lm} to generate plans for 20 problems of varying complexities for each domain. We evaluate the performance of each domain based on the average number of nodes expanded, evaluated, and generated to find a solution. This study demonstrates that macro operators can enhance the planner performance in most of the domains tested, with the exception of the \verb|driverlog| domain. In this domain, the planner performs worse when macro operators are included, as they increase the average number of nodes expanded, evaluated, and generated. This is due to the fact that the macro operators introduce more actions to the domain, which increases the branching factor and challenges the heuristic to select the optimal action at each step. Hence, the applicability of macro operators depends on the features of the domain and the planner. Macro operators can facilitate the planning process by decreasing the search depth, but they can also hinder it by increasing the search width. A potential improvement is to use a more informative heuristic that guides the planner to choose the best action at each step.

% This technique involves analyzing the statistics of multiple plans stored in an ontology and extracting common patterns of actions, which are then combined to form higher-level plans, also known as macro-operators. To improve the effectiveness of this technique, it is essential to consider both the frequency of occurrence of action pairs and the properties of the domain. Specifically, the precondition and effect of actions should be analyzed to ensure that the first action leads to the precondition of the second action in the pair. By doing so, we can filter out irrelevant or invalid action pairs and obtain a set of meaningful and applicable macro-operators. These macro operators can replace primitive actions in a domain. 
% \begin{table}[t]
% \centering
% \begin{tabular}{|l|c|c|c|c|c|c|}
% \hline
%            & bw                                                           & \textbf{bw*}                                                         & \textbf{dl}                                                          & dl*                                                         & \textbf{gr}                                                          & gr*                                                         \\ \hline
% Successful & \begin{tabular}[c]{@{}c@{}}90.04\% \\ (88.44\%)\end{tabular} & \textbf{\begin{tabular}[c]{@{}c@{}}94.08\%\\ (92.36\%)\end{tabular}} & \textbf{\begin{tabular}[c]{@{}c@{}}76.56\%\\ (52.61\%)\end{tabular}} & \begin{tabular}[c]{@{}c@{}}40.08\%\\ (35.69\%)\end{tabular} & \textbf{\begin{tabular}[c]{@{}c@{}}82.97\%\\ (69.47\%)\end{tabular}} & \begin{tabular}[c]{@{}c@{}}72.42\%\\ (44.94\%)\end{tabular} \\ \hline
% Failed     & 9.94\%                                                       & \textbf{5.92\%}                                                      & \textbf{23.44\%}                                                     & 42.86\%                                                     & \textbf{16.61\%}                                                     & 21.97\%                                                     \\ \hline
% Incomplete & 0.02\%                                                       & \textbf{0\%}                                                         & \textbf{0\%}                                                         & 17.06\%                                                     & \textbf{0.42\%}                                                      & 5.61\%                                                      \\ \hline
% \end{tabular}
% \label{plansformer-results}
% \caption{Table showing the results of plan validation for Plansformer with the percentage of optimal plans shown in parentheses., with an asterisk denoting domains that had extracted action relations added to the prompt.}
% \end{table}


% We have used the extracted action relations to test Plansformer \cite{pallagani2022plansformer}, a generative model for AI planning. Plansformer is obtained by fine-tuning CodeT5, a Large Language Model that is pre-trained on code. We use Plansformer for our experimentation as it is easy to infuse macros by appending to the input and obtain the generated plan for validation. The results are presented in Table \ref{plansformer-results}. By directly including the extracted action relations in the prompt, we can observe an increase in percentage for valid plans in \verb|bw| domain, whereas the percentage declined for other domains. 