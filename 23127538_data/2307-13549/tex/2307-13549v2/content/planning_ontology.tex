\section{Planning Ontology}

% Figure environment removed

% % Figure environment removed

This section covers the construction of planning ontology to capture the essential details of automated planning. We will discuss the considerations, challenges, benefits, and limitations of using ontologies for automated planning, to provide a better understanding of how they can improve the efficiency and effectiveness of automated planning systems.
% \bharath{Scope and Methodology}
% \rmv {Give a name to the ontology. I suggest adding the following subsections. 1) Methodology. An iterative approach suggested in {\url{https://protege.stanford.edu/publications/ontology_development/ontology101.pdf}} can be followed. Did you sit with a domain expert, used literature on planning (add refs) and/or used the data to come up with the ontology? This should justify the terminology that was used in the ontology. 2) Competency Questions. These are the minimum set of questions that the ontology should answer. These are used to restrict the scope of the ontology. 3) Ontology Description. Along with the schema diagram (Fig. 1), list the important classes, properties and axioms. In each case, mention how they can be used. 4) Evaluation. Answer the competency questions using ontology. You can either have a SPARQL query for each CQ or just answer it manually using the classes/properties from the ontology. Discuss a use case/scenario of putting this ontology or part of the ontology to use.}

\subsection{Competency Questions}

Competency questions for an ontology are focused on the needs of the users who will be querying the ontology. These questions are designed to help users explore and understand the concepts and relationships within the ontology, and to find the information they need within the associated knowledge base. By answering these questions, the ontology can be better scoped and tailored to meet the needs of its users. 

We designed the following competency questions to model an Ontology to represent the general aspects of Automated Planning.

\begin{itemize}
    \item C1: What are the different types of planners used in automated planning?
    \item C2: What is the relevance of planners in a given problem domain?
    \item C3: What are the available actions for a given domain?
    \item C4: What problems in a domain satisfy a given condition?
    \item C5: What are all the requirements a given domain has?
    \item C6: What is the cost associated with generating a plan for a given problem?
    \item C7: How many parameters does a specific action have?
    \item C8: What planning type does a specific planner belong to?
    \item C9: What requirements does a given planner support?
    \item C10: What are the different parameter \verb|types| present in a domain?
\end{itemize}

\subsection{Design}
An ontology is a formal and explicit representation of concepts, entities, and their relationships in a particular domain. In this case, ontology is concerned with the domain of automated planning, which refers to the process of generating a sequence of actions to achieve a particular goal within a given set of constraints. The ontology aims to provide a structured framework for organizing and integrating knowledge about this domain, which can be useful in various applications, such as designing planning algorithms, extracting best-performing planners given a domain, or learning domain-specific macros.

Figure \ref{fig:ontology} shows an ontology that aims to encompass the various concepts of automated planning separated into categories of \verb|Domain|, \verb|Problem|, \verb|Plan|, and \verb|Planner|. The ontology for automated planning is composed of 19 distinct classes and 25 object properties. These classes and properties are designed to represent the various elements of the automated planning domain and its associated problems. In the design of our ontology, all axioms are formulated using Description Logic \cite{KSH14:DLintro}, providing a formal and expressive framework for representing and reasoning about the concepts and relationships within our domain.
% \vspace{-0.1cm}

\subsubsection{Domain}
The Domain category in our ontology comprises the characteristics of the AI planning domain through several classes. These include \texttt{PlanningDomain} - \texttt{DomainRequirement}, detailing domain modeling; \texttt{ParameterType}, defining parameter varieties in a typed domain; \texttt{DomainPredicate}, encompassing applicable predicates; \texttt{DomainConstant}, representing invariant constants; and \texttt{Action}, for domain operations. \texttt{Action} class is further linked with \texttt{ActionPrecondition}, \texttt{ActionEffect}, and \texttt{Parameter}. This structured approach aids applications like algorithm design, planner optimization, and macro learning in domain-specific contexts.

The \texttt{PlanningDomain} conceptualization is articulated through axioms to represent fundamental elements of planning scenarios. Axiom~\ref{ax: domain1} signifies that every planning domain entails certain actions. Actions are fundamental to planning as they represent the steps or decisions that can be taken to transform a state within the domain. Predicates are essential for defining the states within a planning domain. Axiom~\ref{ax: domain2} ensures that each domain includes predicates to represent these states, facilitating the definition of preconditions and effects of actions. Axiom~\ref{ax: domain3} states that every planning domain possesses certain defined requirements. Requirements in AI Planning are necessary to define various types of domain modeling, such as conditional effects and numeric fluents. Such specifications are not only essential for characterizing the domain but also serve as a criterion to assess whether a planner is compatible with and can support these specific domain modeling features.

\begin{equation}
\texttt{PlanningDomain} \sqsubseteq \exists\texttt{hasAction}.\texttt{Action}
\label{ax: domain1}
\end{equation}
\begin{equation}
\texttt{PlanningDomain} \sqsubseteq \exists\texttt{hasPredicate}.\texttt{DomainPredicate}
\label{ax: domain2}
\end{equation}
\begin{equation}
\texttt{PlanningDomain} \sqsubseteq \exists\texttt{hasRequirement}.\texttt{DomainRequirement}
\label{ax: domain3}
\end{equation}

The \texttt{Action} class is characterized by its effects, a fundamental aspect of planning. Axiom~\ref{ax: action1} addresses the transformative nature of actions in a planning domain. Understanding the effects of actions is essential for planning algorithms to predict and evaluate the outcomes of different action sequences.
\begin{equation}
\texttt{Action} \sqsubseteq \exists\texttt{hasEffect}.\texttt{ActionEffect}
\label{ax: action1}
\end{equation}

Axioms~\ref{ax: action2} and \ref{ax: action3} capture the dynamics of how actions can add or delete predicates in a state, emphasizing the mutable nature of states within the planning domain. This depiction is essential for accurately modeling the consequences and feasibility of actions in AI Planning.
\begin{equation}
\texttt{ActionEffect} \sqsubseteq \exists\texttt{addsPredicate}.\texttt{State}
\label{ax: action2}
\end{equation}
\begin{equation}
\texttt{ActionEffect} \sqsubseteq \exists\texttt{deletesPredicate}.\texttt{State}
\label{ax: action3}
\end{equation}

\subsubsection{Problem}
The Problem category of the ontology includes classes that represent specific problems within a given domain. These classes are designed to capture the details of a particular problem, such as the \verb|Objects| defined in the problem, which is an instance of different \emph{types} defined in the planning domain, the \verb|Initial State| of the problem, and the \verb|Goal State| which are a subclass of the parent class \verb|State| which is a state description of the given domain. 

The axioms defined for \texttt{PlanningProblem} conceptualized the key aspects of a planning problem. Axiom~\ref{ax: problem1} indicates that each planning problem is defined with a specific \texttt{GoalState}, which is the desired outcome or objective of the problem. Axiom~\ref{ax: problem2} asserts that each planning problem also has a defined \texttt{InitialState}, which provides the starting conditions and context for the planning process. Lastly, Axiom~\ref{ax: problem3} identifies the \texttt{Objects} present within a planning problem, denoting the various entities that are subject to manipulation or consideration during the course of planning. Finally, the axiom~\ref{ax: problem4} underscores that every planning problem includes a potential plan or series of actions that lead to the goal state.

\begin{equation}
\texttt{PlanningProblem} \sqsubseteq =1 \texttt{hasGoalState}.\texttt{GoalState}
\label{ax: problem1}
\end{equation}
\begin{equation}
\texttt{PlanningProblem} \sqsubseteq =1 \texttt{hasInitialState}.\texttt{InitialState}
\label{ax: problem2}
\end{equation}
\begin{equation}
\texttt{PlanningProblem} \sqsubseteq \exists\texttt{hasObject}.\texttt{ProblemObject}
\label{ax: problem3}
\end{equation}
\begin{equation}
\texttt{PlanningProblem} \sqsubseteq \exists\texttt{hasPlan}.\texttt{Plan}
\label{ax: problem4}
\end{equation}

\subsubsection{Plan}
The Plan category of the ontology includes classes that represent the sequence of actions that must be taken to solve a given problem. The \verb|Plan| class captures the knowledge about the plans that planners generate for specific problems. The plan cost for each plan is a data property (non-negative integer) of the \verb|Plan| class. This enables planners to be compared based on the quality of the plans they generate and the cost of those plans.

The axioms defined for the \texttt{Plan} category outline the essential features of plans in the AI planning process. Axiom~\ref{ax: plan1} mandates that each plan must have an associated plan cost, precisely quantified as a non-negative integer. This is crucial for evaluating and comparing the efficiency of different plans. Axiom~\ref{ax: plan2} establishes that every plan is generated by some planner, connecting each plan to its generator and allowing for an understanding of the planning process and the assessment of various planners.


\begin{equation}
\texttt{Plan} \sqsubseteq =1 \texttt{hasPlanCost}. \texttt{xsd:nonNegativeInteger}
\label{ax: plan1}
\end{equation}
\begin{equation}
\texttt{Plan} \sqsubseteq \exists\texttt{isGeneratedBy}.\texttt{Planner}
\label{ax: plan2}
\end{equation}
% \vspace{-0.1cm}

\subsubsection{Planner}
The Planner category of the ontology includes classes that capture the details of the planner, planner type, and the planner performance from previous IPCs. Specifically, \verb|Planning Domain| relevance to a \verb|Planner| is classified based on the percentage of problems they have successfully solved, which is then categorized into three levels of relevance to the planner: \textit{low}, \textit{medium}, and \textit{high}. By incorporating this information into the ontology, planners can be evaluated based on their performance in different planning domains, and more informed decisions can be made. In addition, this information can be used to guide the development of new planners and to evaluate their performance against established benchmarks.

The axioms defined for the \texttt{Planner} category provide a foundation for understanding and assessing the capabilities of planners in the AI planning domain. Axiom~\ref{ax: planner1} classifies planners into different types based on their characteristics or strategies, enabling a nuanced understanding of various planning approaches. Axiom~\ref{ax: planner2} links planners with the specific domain requirements they can solve, highlighting their applicability in different planning scenarios.

\begin{equation}
\texttt{Planner} \sqsubseteq \exists\texttt{ofPlannerType}.\texttt{PlannerType}
\label{ax: planner1}
\end{equation}
\begin{equation}
\texttt{Planner} \sqsubseteq \exists\texttt{solvesRequirement}.\texttt{DomainRequirement}
\label{ax: planner2}
\end{equation}
% To incorporate the details of planner performance into the ontology, we have used information from previous IPCs. Specifically, we have analyzed the number of problems that a given planner has successfully solved and categorized this information into three distinct levels of relevance to the planner. Planners that have solved a relatively small number of problems are classified as of low relevance, whereas those who have solved a moderate number of problems are considered to have medium relevance. Finally, planners that have solved a large number of problems, including many challenging ones, are classified as having high relevance for a given domain.

% By incorporating this information into the ontology, we can better assess the performance of planners in different problem domains and make more informed decisions about which planners to use for a given problem. In addition, this information can be used to guide the development of new planners and to evaluate their performance against established benchmarks.

\subsection{Accessing Planning Ontology}
We have taken various measures to ensure that our planning ontology follows the FAIR principles \cite{wilkinson2016fair} of being Findable, Accessible, Interoperable, and Reusable. To assist users in exploring and utilizing our ontology, we have made it accessible through a persistent URL\footnote[1]{\label{purl}PURL - \url{https://purl.org/ai4s/ontology/planning}} and our GitHub repository\footnote[2]{\label{footnote: repo}\url{https://github.com/BharathMuppasani/AI-Planning-Ontology}}. Our repository contains ontology model files, mapping scripts, and utility scripts that extract information from PDDL domains and problems into intermediary JSON format and add the extracted data as triples using our model ontology, creating a knowledge graph. We provide sample SPARQL queries that address the ontology's competency questions mentioned earlier. Moreover, our ontology documentation, which is accessible through the GitHub repository, provides a comprehensive overview of the ontology's structure, concepts, and relations, including ontology visualization. This documentation serves as a detailed guide for users to comprehend the ontology's applications in the automated planning domain. We also provide the scripts and results from the ontology evaluation, which are presented as use cases of our ontology in later sections, in our repository, along with accompanying documentation.
Furthermore, our commitment includes a proactive approach to constantly updating and refining the ontology. This involves periodic updates and community-driven modifications, ensuring its continuous alignment with evolving standards and practices in the field of automated planning.


% Figure environment removed

% In the process of creating an ontology for automated planning, several tools were used for different purposes. The ontology was created using \verb|Protege|\footnote[1]{https://protege.stanford.edu/}, which is a widely used open-source ontology editor and knowledge management system. \verb|Protege| provides an intuitive user interface that allows users to easily create and edit ontologies. It also supports a wide range of ontology languages, including OWL, RDF, and RDFS.

% After creating the ontology, we utilized the \verb|rdflib|\footnote[2]{https://github.com/RDFLib/rdflib}, a Python library, which provides various functionalities for parsing and manipulating RDF data, to access the RDF-based ontology and extract the relevant information. To begin populating the ontology, we captured the domain and problem data in JSON format. Subsequently, we incorporated the data triples from different domains into the ontology using the \verb|rdflib| library. Additionally, we included information about the performance of various planners from previous IPCs in the ontology. Our GitHub repository\footnote[3]{\label{footnote: repo}https://github.com/BharathMuppasani/AI-Planning-Ontology} provides the RDF model file for the ontology, as well as Python scripts to extract domain and problem data in JSON format and add the extracted data as triples to the model ontology, creating a knowledge graph. To query the resulting knowledge graph, we utilized the SPARQL query language, which is the standard query language for RDF data. SPARQL allows users to query data stored in RDF format and is supported by many RDF tools, including Protege. With SPARQL, we were able to query the knowledge graph to extract information on specific domains and planners.

%%%%%%%%%%%%%%%%%%%%%%%%%%%%%%%%%%%%%%%%%%%%%%%%%%%%%

% \begin{table}[!t]
% \centering
% \begin{tabular}{|l|l|l|}
% \hline
% \textbf{Domain} & \textbf{Relevance} & \textbf{Planner} \\ \hline
% caldera & hasHighRelevancePlanner & Delfi1 \\ \hline
% caldera & hasHighRelevancePlanner & Complementary2 \\ \hline
% caldera & hasHighRelevancePlanner & Planning\_PDBs  \\ \hline
% caldera & hasHighRelevancePlanner & Scorpion \\ \hline
% caldera & hasHighRelevancePlanner & FDMS2 \\ \hline
% caldera & hasHighRelevancePlanner & FDMS1 \\ \hline
% caldera & hasHighRelevancePlanner & Metis1 \\ \hline
% caldera & hasHighRelevancePlanner & Metis2 \\ \hline
% caldera & hasMediumRelevancePlanner & Complementary1 \\ \hline
% caldera & hasMediumRelevancePlanner & symb\_Bi\_dir  \\ \hline
% caldera & hasMediumRelevancePlanner & Delfi2 \\ \hline
% caldera & hasMediumRelevancePlanner & DecStar \\ \hline
% caldera & hasMediumRelevancePlanner & MSP \\ \hline
% caldera & hasMediumRelevancePlanner & Blind \\ \hline
% caldera & hasMediumRelevancePlanner & Symple\_2 \\ \hline
% caldera & hasMediumRelevancePlanner & Symple\_1 \\ \hline
% caldera & hasLowRelevancePlanner & maplan\_2 \\ \hline
% caldera & hasLowRelevancePlanner & maplan\_1 \\ \hline
% \end{tabular}
% \caption{Domain-Relevance-Planner triples extracted for caldera domain from the knowledge graph created using IPC-2018 data}
% \label{tab:domain-relevance-planner}
% \end{table}