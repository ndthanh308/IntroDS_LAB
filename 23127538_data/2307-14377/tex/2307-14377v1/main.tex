%%
%% This is file `sample-manuscript.tex',
%% generated with the docstrip utility.
%%
%% The original source files were:
%%
%% samples.dtx  (with options: `manuscript')
%% 
%% IMPORTANT NOTICE:
%% 
%% For the copyright see the source file.
%% 
%% Any modified versions of this file must be renamed
%% with new filenames distinct from sample-manuscript.tex.
%% 
%% For distribution of the original source see the terms
%% for copying and modification in the file samples.dtx.
%% 
%% This generated file may be distributed as long as the
%% original source files, as listed above, are part of the
%% same distribution. (The sources need not necessarily be
%% in the same archive or directory.)
%%
%% Commands for TeXCount
%TC:macro \cite [option:text,text]
%TC:macro \citep [option:text,text]
%TC:macro \citet [option:text,text]
%TC:envir table 0 1
%TC:envir table* 0 1
%TC:envir tabular [ignore] word
%TC:envir displaymath 0 word
%TC:envir math 0 word
%TC:envir comment 0 0
%%
%%
%% The first command in your LaTeX source must be the \documentclass command.
%%%% Small single column format, used for CIE, CSUR, DTRAP, JACM, JDIQ, JEA, JERIC, JETC, PACMCGIT, TAAS, TACCESS, TACO, TALG, TALLIP (formerly TALIP), TCPS, TDSCI, TEAC, TECS, TELO, THRI, TIIS, TIOT, TISSEC, TIST, TKDD, TMIS, TOCE, TOCHI, TOCL, TOCS, TOCT, TODAES, TODS, TOIS, TOIT, TOMACS, TOMM (formerly TOMCCAP), TOMPECS, TOMS, TOPC, TOPLAS, TOPS, TOS, TOSEM, TOSN, TQC, TRETS, TSAS, TSC, TSLP, TWEB.
% \documentclass[acmsmall]{acmart}

%%%% Large single column format, used for IMWUT, JOCCH, PACMPL, POMACS, TAP, PACMHCI
% \documentclass[acmlarge,screen]{acmart}

%%%% Large double column format, used for TOG
% \documentclass[acmtog, authorversion]{acmart}

%%%% Generic manuscript mode, required for submission
%%%% and peer review
%\documentclass[manuscript,screen,review]{acmart}
\documentclass[acmlarge,screen,nonacm]{acmart}
%% Fonts used in the template cannot be substituted; margin 
%% adjustments are not allowed.
%%
%% \BibTeX command to typeset BibTeX logo in the docs
\AtBeginDocument{%
  \providecommand\BibTeX{{%
    \normalfont B\kern-0.5em{\scshape i\kern-0.25em b}\kern-0.8em\TeX}}}

%% Rights management information.  This information is sent to you
%% when you complete the rights form.  These commands have SAMPLE
%% values in them; it is your responsibility as an author to replace
%% the commands and values with those provided to you when you
%% complete the rights form.
% \setcopyright{acmcopyright}
% \copyrightyear{2018}
% \acmYear{2018}
% \acmDOI{XXXXXXX.XXXXXXX}

% %% These commands are for a PROCEEDINGS abstract or paper.
% \acmConference[Conference acronym 'XX]{Make sure to enter the correct
%   conference title from your rights confirmation emai}{June 03--05,
%   2018}{Woodstock, NY}
% %
% %  Uncomment \acmBooktitle if th title of the proceedings is different
% %  from ``Proceedings of ...''!
% %
% \acmBooktitle{Woodstock '18: ACM Symposium on Neural Gaze Detection,
%  June 03--05, 2018, Woodstock, NY} 
% \acmPrice{15.00}
% \acmISBN{978-1-4503-XXXX-X/18/06}


%%
%% Submission ID.
%% Use this when submitting an article to a sponsored event. You'll
%% receive a unique submission ID from the organizers
%% of the event, and this ID should be used as the parameter to this command.
%%\acmSubmissionID{123-A56-BU3}

%%
%% For managing citations, it is recommended to use bibliography
%% files in BibTeX format.
%%
%% You can then either use BibTeX with the ACM-Reference-Format style,
%% or BibLaTeX with the acmnumeric or acmauthoryear sytles, that include
%% support for advanced citation of software artefact from the
%% biblatex-software package, also separately available on CTAN.
%%
%% Look at the sample-*-biblatex.tex files for templates showcasing
%% the biblatex styles.
%%

%%
%% The majority of ACM publications use numbered citations and
%% references.  The command \citestyle{authoryear} switches to the
%% "author year" style.
%%
%% If you are preparing content for an event
%% sponsored by ACM SIGGRAPH, you must use the "author year" style of
%% citations and references.
%% Uncommenting
%% the next command will enable that style.
%%\citestyle{acmauthoryear}


% ====== author-added packages
\usepackage{pdfpages}
\usepackage{listings} % settings in macros.tex
\usepackage{xcolor} % colors defined in macros.tex
\usepackage{xspace} % used for latin abbrevation macros
\usepackage{graphicx} %% used by snippet-styling
\usepackage{caption} %% used by snippet-styling
\usepackage{mdframed} %% used by snippet-styling
\usepackage{lipsum} %% used for testing
\usepackage{subcaption}
\usepackage{ifthen}
\usepackage{wrapfig}
\usepackage{enumitem}
\usepackage{placeins}
\usepackage{afterpage}
\usepackage{subfiles}
\usepackage{adjustbox}
\usepackage{multirow}
% ====== end author-added packages

%%
%% end of the preamble, start of the body of the document source.

\begin{document}

% ====== author-added macros, commands, settings
\newcommand\calF{\mathcal{F}}
\newcommand\calG{\mathcal{G}}
\newcommand\calM{\mathcal{M}}
\newcommand\calV{\mathcal{V}}
\newcommand\calU{\mathcal{U}}
\newcommand\calW{\mathcal{W}}
\newcommand\calP{\mathcal{P}}
\newcommand\calD{\mathbb{D}}
%%%%%%%%%%%%%%%%%
%% macros introduced by Luke 
\newcommand\mydef[1]{{\bf\em #1}}
%%%%%%%%%%%%%%%%%

\newcommand{\numviparams}{{| \lambda |}}
\newcommand{\scoreaccvars}[1]{s_1^{#1}, \ldots, s_{\numviparams}^{#1}}
\newcommand{\scoreaccvar}[2]{s_{#1}^{#2}}
\newcommand{\isdeterm}[1]{\text{Deterministic}({#1})}


\newcommand{\expect}[1]{\mathbb{E}\left[{#1}\right]}
\newcommand{\var}[1]{\mathbb{V}\left[ {#1} \right]}
\newcommand{\expectdist}[2]{\mathbb{E}_{#1}\left[ {#2} \right]}
\newcommand{\vardist}[2]{\mathbb{V}_{#1}\left[ {#2} \right]}
\newcommand{\cov}[2]{\mathbb{C}\text{ov}[{#1}][{#2}]}
\newcommand{\covv}[1]{\mathbb{C}\text{ov}[{#1}]}
\newcommand{\corr}[1]{\mathbb{C}\text{orr}[{#1}]}

\newcommand{\fix}[1]{\mathit{fix}\left({#1}\right)}
\newcommand{\sbr}[1]{\left\llbracket {#1} \right\rrbracket}
\newcommand{\ctxtype}[3]{{#1} \cong_\text{ctx} {#2} : {#3}}
\newcommand{\bigstep}[3]{{#1} \Downarrow_{#2} {#3}}


% PCF types
\newcommand{\bool}{\mathit{bool}}
\newcommand{\nat}{\mathit{nat}}

\newcommand{\ctx}[1]{\mathcal{C}\left[ {#1}\right] }
\newcommand{\pcft}[1]{\text{PCF}_{#1}}

\newcommand{\nfl}{\mathbb{N}_\bot}
\newcommand{\bfl}{\mathbb{B}_\bot}

% PCF constructs
\newcommand{\succc}[1]{\mathbf{succ}({#1})}
\newcommand{\succcn}[2]{\mathbf{succ}^{#1}({#2})}
\newcommand{\zero}{\mathbf{0}}
\newcommand{\zerotest}[1]{\mathbf{zero}\left({#1}\right)}
\newcommand{\pred}[1]{\mathbf{pred}\left( {#1} \right)}
\newcommand{\predn}[2]{\mathbf{pred}^{#1}\left( {#2} \right)}
\def\solvable{\#}

\newcommand{\true}{\mathbf{true}}
\newcommand{\false}{\mathbf{false}}
\newcommand{\pcffix}[1]{\mathbf{fix}\left({#1}\right)}
\newcommand{\pcffn}[3]{\mathbf{fn}~{#1}:{#2}\mathpunct{.}{#3}}
\newcommand{\pairtype}[2]{{#1} * {#2}}
\newcommand{\pairexp}[2]{\mathbf{pair}({#1}, {#2})}
\newcommand{\leftexp}[1]{\mathbf{left}({#1})}
\newcommand{\rightexp}[1]{\mathbf{right}({#1})}

\newcommand{\RationalPos}{\mathbb{Q}^{+}}

\newcommand{\meas}[1]{\mathbb{M}\left( {#1} \right) }
\newcommand{\integ}[1]{\sbr{#1}_I}

\newcommand{\notbigstep}[2]{{#1}~\cancel{\Downarrow}_{#2}}
\newcommand{\subtrace}[3]{{#1}^{{#2} \ldots {#3}}}
\newcommand{\supp}[1]{\textsf{supp}\left({#1}\right)}
\newcommand{\dom}[1]{\textsf{Dom}\left({#1}\right)}
\newcommand{\suppk}[2]{\textsf{Supp}^{#1}\left({#2}\right)}
\newcommand{\tracespace}{\bigcup_{n \in \mathbb{N}}[0, 1]^n}
\newcommand{\generictracespace}{\mathbb{T}}
\newcommand{\nnreals}{\mathbb{R}_{\geq 0}}
\newcommand{\posreals}{\mathbb{R}_{> 0}}
\newcommand{\reals}{\mathbb{R}}

\newcommand{\unrollkM}[2]{\textsf{unroll}_{#1}\left({#2}\right)}
\newcommand{\nphmcint}[5]{\Psi_\textsf{NP}\left({#1}, {#2}, {#3}, {#4}, {#5}\right)}

%SPCF constructs
\newcommand{\spcfvalues}{\Lambda^0_v}

\newcommand{\prevalueM}[1]{\textsf{value}^{-1}_{#1}(\spcfvalues{})}
\newcommand{\num}[1]{\underline{#1}}

% \theoremstyle{definition}
% \newtheorem{thm}{Theorem}
% \newtheorem{lem}{Lemma}
% \newtheorem{defn}{Definition}
% \newtheorem{conj}{Conjecture}
% \newtheorem{prop}{Proposition}

%\theoremstyle{definition}
%\newtheorem{defn}{Definition}[section]
%\newtheorem{example}[defn]{Example}
%
%
%\theoremstyle{plain}
%\newtheorem{thm}{Theorem}[section]
%\newtheorem{lem}[thm]{Lemma}
%\newtheorem{cor}[thm]{Corollary}
%\newtheorem{conj}[thm]{Conjecture}
%\newtheorem{prop}[thm]{Proposition}
%\newtheorem{remark}[thm]{Remark}

%% Proofs
%\let\oldproof\proof
%\renewcommand{\proof}{\color{blue}\oldproof}


\definecolor{codegreen}{rgb}{0,0.6,0}
\definecolor{codegray}{rgb}{0.5,0.5,0.5}
\definecolor{codepurple}{rgb}{0.58,0,0.82}
\definecolor{backcolour}{rgb}{0.95,0.95,0.92}

\lstdefinestyle{myStyle}{
    belowcaptionskip=1\baselineskip,
    breaklines=true,
    frame=none,
    basicstyle=\footnotesize\ttfamily,
    keywordstyle=\bfseries\color{green!40!black},
    commentstyle=\itshape\color{purple!40!black},
    identifierstyle=\color{blue},
    backgroundcolor=\color{gray!10!white},
    %backgroundcolor=\color{backcolour}, 
    numberstyle=\tiny\color{codegray},
    stringstyle=\color{codepurple},
    breakatwhitespace=false,                          
    keepspaces=true,                 
    numbers=left,       
    numbersep=5pt,                  
    showspaces=false,                
    showstringspaces=false,
    showtabs=false,                  
    tabsize=2,
}

% argmin/argmax
\DeclareMathOperator*{\argmax}{arg\,max}
\DeclareMathOperator*{\argmin}{arg\,min}

% Concatenation of lists
\newcommand\doubleplus{+\kern-1.3ex+\kern0.8ex}

% Program configurations
\newcommand{\tuple}[1]{\ensuremath{\langle #1 \rangle}}
% Rule based definitions
\newcommand{\Rule}[4][]{\ensuremath{\inferrule*[lab={\hypertarget{#2}{(\TirName{#2})}},#1]{#3}{#4}}}

% Calligraphic symbols
\newcommand{\calI}{{\mathcal I}} 
\newcommand{\calT}{{\mathcal T}}

%  Macro for new Y operator.
\newcommand{\yBounded}[3]{\mu^{#1}_{#2}\rvert_{#3}}

%%%%%%%%%%%%%%%%%
 
%%%%%%%%%%%%%%%%%

\newcommand{\expv}{\mathbb{E}}

\newcommand{\combTr}[2]{\left[\begin{matrix}
		#1\\
		#2
	\end{matrix} \right]}

\newcommand{\exType}[2]{\left\{\begin{matrix}
		#1\\
		#2
	\end{matrix} \right\}}
\newcommand{\myint}[1]{ [#1]}
\newcommand{\Uniform}{\ensuremath{\mathrm{Uniform}}}
\newcommand{\Normal}{\ensuremath{\mathrm{normal}}}
\DeclareMathOperator{\abs}{abs}
\DeclareMathOperator{\pdf}{pdf}

\newcommand{\intConf}[1]{\lceil#1\rceil}
\newcommand{\tr}{\boldsymbol{t}}

\newcommand{\sample}{\tt{sample}}
%\newcommand{\fix}{\texttt{fix}}
%\newcommand{\num}[1]{\underline{#1}}
\newcommand{\myif}{\texttt{if}}
\newcommand{\mylet}{\texttt{let} \, }
\newcommand{\myin}{\, \texttt{in} \,}
\newcommand{\mythen}{\, \texttt{then} \,}
\newcommand{\myelse}{\, \texttt{else} \,}
\newcommand{\score}{\tt{score}}
\newcommand{\tick}{\tt{tick}}

\newcommand{\term}{\tt{term}}
\newcommand{\pv}{\mathbf{v}}
\newcommand{\rv}{\mathbf{r}}

\newcommand{\interval}{\mathfrak{I}}

\newcommand{\typeReal}{\textbf{\textsf{R}}}

\newcommand{\symbolInt}{\myint{\cdot}}

\newcommand{\LambdaInterval}{\Lambda_{\interval}}
\newcommand{\LambdaSymbolic}{\Lambda_{\text{sym}}}

\newcommand{\toIntervalTerm}[1]{#1^{2\interval}}

%Others
\newcommand{\Sset}{\mathbb{S}}
\newcommand{\Iset}{\mathbb{I}}
\newcommand{\Rset}{\mathbb{R}}
\newcommand{\Nset}{\mathbb{N}}
\newcommand{\Zset}{\mathbb{Z}}

\newcommand{\Term}{\mathbb{T}}
\newcommand{\prob}{\mathbb{P}}
\newcommand{\expt}{\mathbb{E}}


\newcommand{\Leb}{\tt{Leb}}
\newcommand{\Red}{\tt{Red}}
\newcommand{\cost}{\text{cost}}

%\newcommand{\intervalab}[2]{\underline{[#1,#2]}}
\newcommand{\intervalab}{\underline{[a,b]}}
\newcommand{\interI}{\mathcal{I}}
\newcommand{\trans}{\mathcal{T}}

\newcommand{\iv}{\mathbb{I}}

% Programming language constructs
\newcommand{\lit}[1]{\underline{#1}}
\newcommand{\letIn}[1]{\mathsf{let}\,{#1}\,\mathsf{in}\,}
\newcommand{\fixLam}[2]{\mu {#1} {#2}.}
\newcommand{\ifElse}[3]{\mathsf{if} (#1 \le \num{0}) \, {#2} \,\mathsf{else}\, {#3}}

%%Basic notions
\newcommand{\pspace}{(\Omega,\mathcal{F},\probm)}
\newcommand{\probm}{\mathbb{P}}
\newcommand{\condexpv}[2]{{\expt}{\left[{#1} \mid {#2}\right]}}

\newcommand{\stdConf}[1]{(#1)}
%\newcommand{\intConf}[1]{\lceil#1\rceil}
%\newcommand{\intConf}[1]{(#1)}
%\newcommand{\symConf}[1]{\langle\!\langle  #1 \rangle\!\rangle}
%\newcommand\symPath[1]{(#1)}
\newcommand{\symPath}[1]{\langle\!\langle  #1 \rangle\!\rangle}
\newcommand\symConf[1]{(#1)}

\newcommand{\ifSimple}[3]{\mathsf{if}(#1, #2, #3)}
%\newcommand{\ifElse}[3]{\mathsf{if} (#1 \le 0) \, \allowbreak {#2} \, \allowbreak \mathsf{else}\, {#3}}
%\newcommand{\ifElse}[3]{\ifSimple{#1}{#2}{#3}}

%\newcommand{\trace}{\mathsf{s}}
%
%\newcommand\defn[1]{{\bf \em #1}}
\newcommand{\traces}{\mathbb{T}}
%
%\newcommand{\stdConf}[1]{(#1)}
%%\newcommand{\intConf}[1]{\lceil#1\rceil}
%\newcommand{\intConf}[1]{(#1)}
%%\newcommand{\symConf}[1]{\langle\!\langle  #1 \rangle\!\rangle}
%%\newcommand\symPath[1]{(#1)}
%\newcommand{\symPath}[1]{\langle\!\langle  #1 \rangle\!\rangle}
%\newcommand\symConf[1]{(#1)}

\newcommand{\valueSem}[1]{\mathsf{val}_{#1}} % value (semantics)
\newcommand{\weightSem}[1]{\mathsf{wt}_{#1}} % weight (semantics)
\newcommand{\measureSem}[1]{\llbracket #1 \rrbracket}
\newcommand{\posterior}{\mathsf{posterior}}


%%%%%%%%%
% 
%%%%%%%%
\newcommand{\loc}{\ell}
\newcommand{\locs}{\mathit{L}}
\newcommand{\blocs}{\mathit{L}_{\mathrm{b}}}

\newcommand{\iflocs}{\mathit{L}_{\mathrm{if}}}
\newcommand{\looplocs}{\mathit{L}_{\mathrm{while}}}

\newcommand{\alocs}{\mathit{L}_{\mathrm{a}}}
\newcommand{\wlocs}{\mathit{L}_{\mathrm{w}}}
\newcommand{\rlocs}{\mathit{L}_{\mathrm{r}}}
\newcommand{\Alocs}[1]{\mathit{L}_{\mathrm{A}}^{\mathsf{#1}}}
\newcommand{\Dlocs}{\mathit{L}_{\mathrm{nd}}}
\newcommand{\transitions}{{\rightarrow}}

%%% 
\newcommand{\plocs}{\mathit{L}_{\mathrm{p}}}
\newcommand{\tlocs}{\mathit{L}_{\mathrm{t}}}

\newcommand{\lin}{\loc_\mathrm{init}}
\newcommand{\lout}{\loc_\mathrm{out}}
\newcommand{\val}[1]{\mbox{\sl Val}_{#1}}

\newcommand{\pvars}{V_\mathrm{p}}
\newcommand{\rvars}{V_{\mathrm{r}}}
\newcommand{\pre}{\mathrm{pre}}

\newcommand{\sle}{\sqsubseteq}
\newcommand{\sge}{\sqsupseteq}

\newcommand{\lfp}{\mathrm{lfp}}
\newcommand{\gfp}{\mathrm{gfp}}

\newcommand{\rdvarjdis}{\mathcal D}
\newcommand{\sampset}{\textit{supp}}

\newcommand{\upd}{\mbox{\sl upd}}
\newcommand{\wet}{\mbox{\sl wt}}
\newcommand{\transset}{\mathfrak T}
\newcommand{\valin}{\pv_{\mathrm{init}}}
\newcommand{\ret}{\mbox{\sl ret}}

\newcommand{\win}{w_{\mathrm{init}}}

\newcommand{\sampdpd}{\overline{\Upsilon}}

\newcommand{\outmap}{\text{O}}
\newcommand{\sat}[1]{\langle #1 \rangle}
\newcommand{\monoid}{\mbox{\sl Monoid}}
\newcommand{\handelmanformat}{(\dagger)}

\newcommand{\trunc}{\mathcal{B}}

\newcommand{\ewt}{\mbox{\sl ewt}}
\newcommand{\statemap}{\text{St}}

\newcommand{\valrd}{{\mathbf{r}}}
\newcommand{\frmloc}{\ell^{\mathrm{src}}}
\newcommand{\toloc}{\ell^{\mathrm{dst}}}

\newcommand{\monomials}{\mathbf{M}}
%% Requires \usepackage{xcolor}
%% Requires \usepackage{lstlistings}
\definecolor{codegreen}{rgb}{0,0.6,0}
\definecolor{codeblue}{rgb}{.11,.56,1}
\definecolor{codegray}{rgb}{0.5,0.5,0.5}
\definecolor{codepurple}{rgb}{0.58,0,0.82}

\definecolor{codeKeyword}{RGB}{211	54	130}
% \definecolor{codeKeyword}{RGB}{133, 153, 0}
\definecolor{codeComment}{RGB}{42	161	152}
\definecolor{codeOmitted}{RGB}{108	113	196}
\definecolor{codeNumbers}{rgb}{0.5,0.5,0.5}
\definecolor{codeString}{RGB}{128, 161, 16}

\definecolor{textusercolor}{RGB}{40 20 10}
\definecolor{textgptcolor}{RGB}{62, 65, 115}

\definecolor{codebackcolour}{RGB}{	253	246	227}
\definecolor{backuserprompt}{RGB}{
253, 250, 250}
%\definecolor{backgptresponse}{rgb}{0.93,0.93,0.98}

\definecolor{backgptresponse}{RGB}{226 228 255}

\newcommand{\gpticon}{figures/formatting/chatgpt-logo.png}
\newcommand{\usericon}{figures/formatting/person-raising-hand.png}



%% Defining custom languages -- try not to put styling here, only keywords/comment signifiers/etc. this way all languages will look the same. 
\lstdefinelanguage{JavaScript}{
  keywords={typeof, new, true, false, catch, function, return, null, catch, switch, var, if, in, while, do, else, case, break, const},
  ndkeywords={class, export, boolean, throw, implements, import, this, require},
  sensitive=false,
  comment=[l]{//},
  morecomment=[s]{/*}{*/},
  morestring=[b]',
  morestring=[b]"
}

%% consistent styling used for all languages
\lstdefinestyle{codestyle}{
    commentstyle=\color{codeComment},
    keywordstyle=\color{codeKeyword},
    numberstyle=\tiny\color{codeNumbers},
    stringstyle=\color{codeString},
    basicstyle=\linespread{0.85}\footnotesize,
    columns=flexible,
    breakatwhitespace=false,         
    breaklines=true,                 
    captionpos=b,                    
    % numbers=left,                    
    % numbersep=5pt,     
    showspaces=false,
    showstringspaces=false,
    showtabs=false,
    tabsize=2,
    escapeinside={\$}{\$},
}

\surroundwithmdframed[
  hidealllines=true,
  backgroundcolor=codebackcolour,
  innerleftmargin=0pt,
  innertopmargin=0pt,
  innerbottommargin=0pt]{gptcodeblock}


\newcommand\colboxcolor{codeComment} % temporary
%% colored box -- takes background color as parameter
\newsavebox{\savedcolorbox}
\newenvironment{colbox}[2]
  {\renewcommand\colboxcolor{#1}%
   \begin{lrbox}{\savedcolorbox}%
    \begin{minipage}{\dimexpr\columnwidth-2\fboxsep\relax}

   \footnotesize
   \bgroup\color{#2}
   }
  {\egroup\end{minipage}\end{lrbox}%
   \begin{center}
   \colorbox{\colboxcolor}{\usebox{\savedcolorbox}}
   \end{center}
}



%% text responses
\newsavebox{\savedfigurebox}
\newenvironment{blurbwithfig}[5]
{
    \newcommand{\figurewidth}{#1}
    \newcommand{\iconwidth}{0.025\textwidth}
    \newcommand{\blurbwidth}{0.982\textwidth - \figurewidth - \iconwidth}
    \newcommand{\imagetoshow}{#2}
    \newcommand{\backgroundcolor}{#3}
    \newcommand{\boxtextcolor}{#4}
    \newcommand{\icontoshow}{#5}

    % collect the to-be-right-aligned figure block
    \begin{lrbox}{\savedfigurebox}%
    \begin{minipage}[t]{\figurewidth}
        \vspace{3pt}
        \ifthenelse{\equal{\imagetoshow}{}}{}{% Figure removed}
        % \captionof{figure}{note}
        % \label{fig:figure2}
    \end{minipage}\end{lrbox}%

    % start the user icon
    \noindent
    \begin{minipage}[t]{\iconwidth}
    \vspace{2pt}
    \centering
    % Figure removed
    \end{minipage}
    %
    % start the left-aligned chat block
    \noindent
    \begin{minipage}[t]{\blurbwidth}
    \vspace{0pt}
    \begin{colbox}{\backgroundcolor}{\boxtextcolor}
}
{
    \end{colbox}
    \end{minipage}
    % end the primary prompt area
    \hfill
    % begin the image portion
    \usebox{\savedfigurebox}
}


\newenvironment{userprompt}[2]
{
    \begin{blurbwithfig}{#1}{#2}{backuserprompt}{textusercolor}{\usericon}
}
{
    \end{blurbwithfig}
}


\newenvironment{gptresponse}[2]
{
    \begin{blurbwithfig}{#1}{#2}{backgptresponse}{textgptcolor}{\gpticon}
}
{

    \end{blurbwithfig}
}


\lstnewenvironment{gptcodeblock}[1]
{
    \lstset{style=codestyle} %% change this to be a local setting, so it doesn't affect others
    \lstset{language=#1}
}
{}



\newenvironment{chat}[1]
{
    \newcommand{\preventbreaks}{#1}
    \begin{center}
    \mdfsetup{nobreak=\preventbreaks}
    \begin{mdframed}[
        linecolor=black,
        innerleftmargin=0.04cm,
        innerrightmargin=0cm
        innertopmargin=0cm
        innerbottommargin=0cm
    ]{}

}
{ 
    \end{mdframed}
    \end{center}
}


%% ============================
%% Macros to be used inside the chat environment, to help shorten/clarify the included chats
%% ============================

\newcommand{\authorremark}[1]{\footnotesize\textit{(Author remark: #1)}}

% parameters: #1 is the summary of the omitted content (will show up in doc if provided; can be left blank), #2 can be placed around the content to omit, so you can leave it in the tex, it just doesn't get rendered anywhere
\newcommand{\omitted}[2]{
    % \textcolor{codeblue}{
    \ifthenelse{\equal{#1}{}}{\textit{(... content omitted by authors ...)}} 
                            {\textit{(... omitted by authors: #1 ...)}}
}

% parameters: #1 is the summary of the omitted code (will show up in doc if provided; can be left blank), #2 can be placed around the content to omit, so you can leave it in the tex, it just doesn't get rendered anywhere
%% IMPORTANT: to use this inside a \gptcodeblock, you must wrap it in $$ to escape into latex mode. For example: 

% for x in range(0,3):
% $\omittedCode{update and output i}{
%.     i = x
%      print("the value of i is " + str(i))
% }$

\newcommand{\omittedCode}[2]{
    \textcolor{codeOmitted}{
    \ifthenelse{\equal{#1}{}}{\textit{(... code omitted by authors ...)}} 
                            {\textit{(... omitted by authors: #1 ...)}}
    }
}




%%
%% The "title" command has an optional parameter,
%% allowing the author to define a "short title" to be used in page headers.
\title{How Can Large Language Models Help Humans in Design And Manufacturing?}
% \title[An Analysis of \llm Integration for \cdam]{An Analysis of Opportunities for \llm Integration in Computational Design And Manufacturing}


%%
%% The "author" command and its associated commands are used to define
%% the authors and their affiliations.
%% Of note is the shared affiliation of the first two authors, and the
%% "authornote" and "authornotemark" commands
%% used to denote shared contribution to the research.

%% keeping for reference: shared author affiliation, notes
% \author{Ben Trovato}
% \authornote{Both authors contributed equally to this research.}
% \email{trovato@corporation.com}
% \orcid{1234-5678-9012}
% \author{G.K.M. Tobin}
% \authornotemark[1]
% \email{webmaster@marysville-ohio.com}
% \affiliation{%
%   \institution{Institute for Clarity in Documentation}
%   \streetaddress{P.O. Box 1212}
%   \city{Dublin}
%   \state{Ohio}
%   \country{USA}
%   \postcode{43017-6221}
% }

\newcommand{\mitAffil}{
 \institution{MIT}
 \streetaddress{77 Massachusetts Ave}
 \city{Cambridge}
 \state{MA}
 \postcode{02139}
 \country{USA}
}

\newcommand{\harvardAffil}{
 \institution{Harvard University}
 \streetaddress{Massachusetts Hall}
 \city{Cambridge}
 \state{MA}
 \postcode{02138}
 \country{USA}
}

\newcommand{\uwAffil}{
 \institution{University of Washington}
 \streetaddress{1410 NE Campus Parkway}
 \city{Seattle}
 \state{WA}
 \postcode{98195}
 \country{USA}
}



\author{Liane Makatura}
\orcid{0000-0003-4804-2173}
\email{makatura@mit.edu}
% \affiliation{\mitAffil}
%
\author{Michael Foshey}
\orcid{0000-0002-9047-8387}
\email{mfoshey@mit.edu}
% \affiliation{\mitAffil}
%
\author{Bohan Wang}
\orcid{0000-0003-1439-1455}
\email{bohanw@mit.edu}
\affiliation{\mitAffil}

\author{Felix H\"{a}hnlein}
\orcid{0000-0002-3484-4004}
\email{fhahnlei@cs.washington.edu}
\affiliation{\uwAffil}

\author{Pingchuan Ma}
\orcid{0000-0002-5698-9503}
\email{pcma@csail.mit.edu}
% \affiliation{\mitAffil}
%
\author{Bolei Deng}
\orcid{0000-0003-2589-2837}
\email{boleiden@mit.edu}
% \affiliation{\mitAffil}
%
\author{Megan Tjandrasuwita}
\orcid{0000-0002-4950-8679}
\email{megantj@mit.edu}
\affiliation{\mitAffil}

\author{Andrew Spielberg}
\orcid{0000-0002-6937-6204}
\email{aespielberg@seas.harvard.edu}
\affiliation{\harvardAffil}

\author{Crystal Elaine Owens}
\orcid{0000-0002-2433-7025}
\email{crystalo@mit.edu}
% \affiliation{\mitAffil}
% 
\author{Peter Yichen Chen}
\orcid{0000-0003-1919-5437}
\email{pyc@csail.mit.edu}
% \affiliation{\mitAffil}
% 
\author{Allan Zhao}
\orcid{0000-0002-9162-6716}
\email{azhao@csail.mit.edu}
\affiliation{\mitAffil}
% 

\author{Amy Zhu} %%% Amy -- please update. Thanks!
\orcid{0000-0001-5766-7090}
\email{amyzhu@cs.washington.edu}
\affiliation{\uwAffil}


\author{Wil J Norton}
\orcid{0000-0001-9465-9751}
\email{wn1024@mit.edu}
% \affiliation{\mitAffil}
% 
\author{Edward Gu}
\orcid{0009-0006-0641-7757}
\email{egu@mit.edu}
% \affiliation{\mitAffil}
% 
\author{Joshua Jacob}
\orcid{0009-0001-3165-1012}
\email{jmjacob@csail.mit.edu}
% \affiliation{\mitAffil}
% 
\author{Yifei Li}
\orcid{0000-0002-3770-0575}
\email{liyifei@csail.mit.edu}
\affiliation{\mitAffil}


\author{Adriana Schulz}
\orcid{0000-0002-2464-0876}
\email{adriana@cs.washington.edu}
\affiliation{\uwAffil}

\author{Wojciech Matusik}
\orcid{0000-0003-0212-5643}
\email{wojciech@csail.mit.edu}
\affiliation{\mitAffil}



%%
%% By default, the full list of authors will be used in the page
%% headers. Often, this list is too long, and will overlap
%% other information printed in the page headers. This command allows
%% the author to define a more concise list
%% of authors' names for this purpose.
\renewcommand{\shortauthors}{Makatura et al.}

%%
%% The abstract is a short summary of the work to be presented in the
%% article.
\begin{abstract}
  The advancement of Large Language Models (\llms), including \gpt, provides exciting new opportunities for generative design. We investigate the application of this tool across the entire design and manufacturing workflow. Specifically, we scrutinize the utility of \llms in tasks such as: converting a text-based prompt into a design specification, transforming a design into manufacturing instructions, producing a design space and design variations, computing the performance of a design, and searching for designs predicated on performance. Through a series of examples, we highlight both the benefits and the limitations of the current \llms. By exposing these limitations, we aspire to catalyze the continued improvement and progression of these models.
\end{abstract}

%%
%% The code below is generated by the tool at http://dl.acm.org/ccs.cfm.
%% Please copy and paste the code instead of the example below.
%%
\begin{CCSXML}
<ccs2012>
   <concept>
       <concept_id>10010147.10010341</concept_id>
       <concept_desc>Computing methodologies~Modeling and simulation</concept_desc>
       <concept_significance>500</concept_significance>
       </concept>
   <concept>
       <concept_id>10010147.10010178.10010187.10010197</concept_id>
       <concept_desc>Computing methodologies~Spatial and physical reasoning</concept_desc>
       <concept_significance>300</concept_significance>
       </concept>
   <concept>
       <concept_id>10003120.10003121.10003124.10010870</concept_id>
       <concept_desc>Human-centered computing~Natural language interfaces</concept_desc>
       <concept_significance>300</concept_significance>
       </concept>
    <concept>
       <concept_id>10003120.10003121.10003128.10011753</concept_id>
       <concept_desc>Human-centered computing~Text input</concept_desc>
       <concept_significance>300</concept_significance>
       </concept>
 </ccs2012>
\end{CCSXML}

\ccsdesc[500]{Computing methodologies~Modeling and simulation}
\ccsdesc[300]{Computing methodologies~Spatial and physical reasoning}
\ccsdesc[300]{Human-centered computing~Natural language interfaces}
\ccsdesc[300]{Human-centered computing~Text input}

%%
%% Keywords. The author(s) should pick words that accurately describe
%% the work being presented. Separate the keywords with commas.
\keywords{Large Language Models, GPT-4, computational design, computational fabrication, CAD, CAM, design for manufacturing, simulation, inverse design}

%% A "teaser" image appears between the author and affiliation
%% information and the body of the document, and typically spans the
%% page.
%\begin{teaserfigure}
%  % Figure removed
%  \caption{Seattle Mariners at Spring Training, 2010.  They still have not won a world series.  In fact, they're the only Major League Baseball team that has never even made it to a World Series.  But their stadium is nice.  Photo is clearly photoshopped since it is not raining in Seattle.}
%  \label{fig:teaser}
%\end{teaserfigure}

% \received{20 February 2007}
% \received[revised]{12 March 2009}
% \received[accepted]{5 June 2009}



%%
%% This command processes the author and affiliation and title
%% information and builds the first part of the formatted document.
\maketitle

\section{Introduction}
Current quantum hardware is unable to carry out universal quantum computations due to the buildup of errors that occur during the computation. 
The magnitude of the individual error is currently above the value that the Threshold Theorem requires in order to kick-start quantum error correction and fault-tolerant quantum computation~\cite[Section 10.6]{nielsen_chuang_2010}. 
Although the experimentally achieved fidelity rates are promising and the error bounds are inching closer to the required threshold, we will have to work for the foreseeable future with quantum hardware with errors that build-up during the computation.  This implies that we can only do a limited number of steps before the output of the computation has become completely uncorrelated with the intended one.

For fault-tolerant quantum computing, we repeat four steps: 
1) We apply a number of single and two-qubit quantum gates, in parallel whenever possible; 
2) We perform a syndrome measurement on a subset of the qubits; 
3) We perform fast classical computations to determine which errors have occurred and how to correct them; 
and, 4) We apply correction terms based on the classical computations.
We then repeat these four steps with a next sequence of gates. 
These four steps are essential to fault-tolerant quantum computing. 


The starting point of this work is to use the four steps outlined above, not to carry out error correction and fault-tolerant computation, but to enhance short, constant-depth, {\em uncorrected} quantum circuits that perform single qubit gates and {\em nearest-neighbor} two qubit gates. 
Since in the long run we will have to implement error-correction and fault-tolerant computation anyhow, and this is done by such a four-step process, why not make other use of this architecture? Moreover, on some of the quantum hardware platforms, these operations are already in place.
Embracing this idea we naturally arrive at the question: what is the computational power of \textit{low-depth} quantum-classical circuits organized as in the four steps outlined above? 
We thus investigate circuits that execute a small, ideally constant, number of stages, where at each stage we may apply, in parallel, single qubit gates and {\em nearest-neighbor} two qubit gates, followed by measurements, followed by low-depth classical computations of which the outcome can control quantum gates in later stages. 
It is not clear, at first, whether such circuits, especially with constant depth, can do anything remotely useful. 
But we will see that this is indeed the case: many quantum computations can be done by such circuits in constant depth. 
By parallelizing quantum computations in this way, we improve the overall computational capabilities of these circuits, as we do not incur errors on qubits that are idle, simply because qubits are not idle for a very long time. 
Furthermore, reducing the depth of quantum circuits, at the cost of increasing width, allows the circuit to be run faster even if errors occur.

The first usage of such a four-step layout, not to do error correction, but to perform computations, can be found in the paradigm of measurement-based quantum computing~\cite{gottesman1999demonstrating,raussendorf2001one,jozsa2006introduction,clark2007generalised}: 
A universal form of quantum computing where a quantum state is prepared and operations are performed by measuring qubits in different bases, depending on previous measurements and intermediate measurements.

\citeauthor{PhamSvore2013} were the first to formalize the four-step protocol for performing computations~\cite{PhamSvore2013}. They included specific hardware topologies by considering two-dimensional graphs for imposing constraints on qubit interactions. In their model, they develop circuits for particularly useful multi-qubit gates, including specifying costs in the width, number of qubits, depth, number of concurrent time steps, size, and total number of non-Identity operations.
As a result, they find an algorithm that factors integers in polylogarithmic depth.
\citeauthor{Browne:2011} showed that the main tool in the work by \citeauthor{PhamSvore2013}, the fan-out gate, can also be replaced by additional log-depth classical computations in the measurement-based quantum computing setting~\cite{Browne:2011}.

More recently, \citeauthor{Cirac:2021} introduced a scheme to implement unitary operations involving quantum circuits combined with Local Operations and Classical Communication ($\mathsf{LOCC}$) channels: $\mathsf{LOCC}$-assisted quantum circuits~\cite{Cirac:2021}. Similarly to the four-step scheme we just described, they allow for a short depth circuit to be run on the qubits, followed by one round of $\mathsf{LOCC}$, in which ancilla qubits are measured and local unitaries are applied based on the measurement outcomes. They show that in this model any 1D transitionally invariant matrix-product state (MPS) with fixed bond dimension is in the same phase of matter as the trivial state. Similar ideas can be found in~\cite{TVV_NonAbelianTopologicalOrder_2022, tantivasadakarn2021long}.

In this work, we introduce a new model, called \textit{Local Alternating Quantum-Classical Computations} ($\LAQCC$). In this model we alternate between running quantum circuits (constrained by locality), ending in the measurement of a subset of qubits, and fast classical computations based on the measurement results. The outcome of the classical computations are then used to control future quantum circuits. We allow for flexibility in this model, by giving different constraints to the power of both the quantum circuits and the classical circuits as well as the number of alternations between them. 
Most attention will be given to $\LAQCC$ containing quantum circuits of constant depth, classical circuits of logarithmic depth and at most a constant number of alternations between them. 
Any circuit constructed in this model is considered to be of constant depth. 
We restrict ourselves to logarithmic depth classical computations, as this is the first natural and non-trivial extension beyond constant-depth classical computations. 
Constant-depth classical computations do however also have an equivalent constant-depth quantum implementation.

The definition of $\LAQCC$ sharpens the original definition of \citeauthor{PhamSvore2013} by adding constraints to the intermediate classical computations. This allows us to bound the power of $\LAQCC$ from above. 

The main result of \citeauthor{Cirac:2021}, that 1D translational invariant MPS with fixed bond dimension can be prepared by $\mathsf{LOCC}$-assisted circuits, relies on local symmetries of the MPS. These symmetries allow them to prepare local states (on a constant number of qubits) and glue them together by doing one round of the appropriate entangling measurement and corrections, after which they run a round of local unitaries to get the desired result. This general scheme for preparing states that exhibit an MPS description with the appropriate local symmetries requires only geometrically local unitaries and one round of measurement and corrections an therefore is accessible in $\LAQCC$. Studying different local symmetries, known as Symmetry Protected Topological (SPT) phases of matter, to find measurement-based constant depth circuits for states is a broad ongoing field of research~\cite{TVV_NonAbelianTopologicalOrder_2022, tantivasadakarn2021long, smith2023deterministic}. 
All these schemes have a $\LAQCC$ implementation.

%$\LAQCC$-circuits also exist for general schemes of preparing local states, based on the local tensors, and gluing them together using one round of entangled measurement and corrections, based on the local symmetry. 
%The main result of \citeauthor{Cirac:2021}, that 1D translational invariant MPS with fixed bond dimension can be prepared by $\mathsf{LOCC}$-assisted circuits, relies heavily on local symmetries of the MPS and as a result also has an equivalent $\LAQCC$ implementation. 
%The corrections applied after the measurement round are local unitaries depending on the local symmetries of the MPS. 

 

%This general scheme of preparing local states, based on the local tensors, and gluing it together by doing one round of entangled measurement and corrections, based on the local symmetry, is accessible in $\LAQCC$.
Note however that \citeauthor{Cirac:2021} also suggest a circuit for the $W$-state.
This circuit uses sequentially and dependent measurement-based corrections of the ancilla qubits. 
These dependent measurements translate to sequential alternations between the quantum and classical circuits and therefore increase the total depth to linear depth, exceeding the constant-depth constraints imposed by $\LAQCC$-circuits. 

We study the power of the $\LAQCC$ model with respect to state preparation, showing that even with only constant quantum-depth and logarithmic classical depth it remains possible to prepare states with long-range entanglement.
Another surprising result is that it is unlikely that $\LAQCC$ circuits are classically simulatable. We show that any instantaneous quantum polynomial-time (IQP) circuit~\cite{Bremner2010,Shepherd2009} has an $\LAQCC$ implementation.
Classical simulation of IQP circuits implies the collapse of the polynomial hierarchy to the third level, which is not believed to be true~\cite{Bremner2017}. Therefore, we expect that $\LAQCC$ circuits are unlikely to be classically simulatable. We bound the power of $\LAQCC$ by showing that it is contained in $\QNC^1$, the class of polynomial-size, log-depth circuits.

Next, we also study the power that intermediate classical calculations can add to quantum computations, by considering a new model that alternates between polynomially many polynomial-depth quantum circuits and unbounded classical computations
We study this model by doing a complexity theoretical analysis, where we draw inspiration from the notions of complexity given by \citeauthor{RosenthalYuen:2022}, \citeauthor{MetgerYuen:2023}, and \citeauthor{Aaronson:2004}.
All three complexity notions are based on the notion of state preparation, instead of more traditional definition of complexity such as the decidability of a computational problem. 
The first two consider classes based on sequences of quantum states preparable by a polynomial-sized quantum circuit, where the circuits are uniformly generated by a computational class, for instance, the class $\mathsf{PSPACE}$, which results in the complexity class $\mathsf{StatePSPACE}$~\cite{RosenthalYuen:2022,MetgerYuen:2023}.
The third notion considers a relative complexity, where the complexity is measured between two given states, and is measured by the number of gates, from a given gate-set, required to transform one state in another state~\cite{Aaronson:2004}. 
For our definition of state preparation complexity, we drop the uniformity constraint from~\cite{RosenthalYuen:2022,MetgerYuen:2023} and define a class as $\mathsf{StateX}$, which refers to states preparable by circuits of type $\mathsf{X}$. 
As an example, if $\mathsf{X} = \QNC^0$, this results in the class $\mathsf{StateQNC^0}$, which is the set of states preparable from the $\ket{0}^n$ state by poly-size constant-depth circuits. 
This notion is similar to the relative complexity from~\cite{Aaronson:2004}, where one state is the  $\ket{0}^n$ state and instead of counting the number of gates we consider the set of states preparable by a fixed number of gates. Using this notion of complexity we show that any state preparable by an $\LAQCC^*$ circuit is also preparable by a $\mathsf{PostQPoly}$ circuit, the class of circuits of polynomial depth with an additional post-selection gate. 

All Clifford circuits have a constant-depth $\LAQCC$ implementation, implying that any stabilizer state can be implemented by a constant-depth $\LAQCC$ circuit, see Section~\ref{sec:clifford_circuits} for a proof of this statement. 
Efficient circuits for stabilizer states have been known already through measurement-based quantum computing. Therefore this paper focuses on the preparation of non-stabilizer states, and as a surprising result we find novel constant-depth protocols for four very natural classes of non-stabilizer states.
Despite the extensive research into these four classes of non-stabilizer states and the many applications of them, no efficient constant- or low-depth state preparation protocols are known yet. We specifically consider these four classes as they are all often used as initial states in other algorithms.

The first state is a uniform superposition over an arbitrary number of states. 
This state finds applications in many quantum algorithms, as they often start with a uniform superposition over multiple states. 
This superposition is often achieved by applying Hadamard gates to every qubit due to its simplicity to prepare. 
Yet, the analysis of many algorithms, such as Shor's algorithm~\cite{Shor:1997}, would benefit from a different initial superposition. 
The circuit to prepare the uniform superposition over an arbitrary number of states uses an exact version of Grover search as a subroutine, that turns a probabilistic circuit, with a known constant probability of success, into a deterministic circuit. 
We use the circuit for preparing a uniform superposition over an arbitrary number of states as a subroutine in the next two quantum state preparation protocols. 

The second state is the $W$-state, the uniform superposition over all computational basis states of Hamming-weight~$1$, a natural long-ranged entangled state that displays a fundamentally nonequivalent type of entanglement from the Greenberger–Horne–Zeilinger state~\cite{WState:2000}, for which $\LAQCC$-type constant-depth circuits were previously known~\cite{PhamSvore2013, Cirac:2021}. 
The $W$-state is often used as benchmark for new quantum hardware~\cite{Haffner2005,Neeley2010,GarciaPerez:2021}. 
A novel way to prepare the $W$-state therefore gives a new way to benchmark different quantum devices with each other. 
A circuit for preparing the $W$-state was given in~\cite{Cirac:2021}, but this implementation requires sequentially alternating measurements followed by local unitaries, which in the $\LAQCC$ model is not considered to be of constant depth. 
We improve this protocol by giving an $\LAQCC$ implementation of the $W$-state, based on a compress-uncompress method that links the one-hot and binary encoding of integers.

The third state considered is the Dicke state, a generalization of the $W$-state, a superposition over all computational basis states with Hamming-weight $k$~\cite{Dicke:1954}. 
Dicke states have relevance in various practical settings.
For instance, for quantum game theory~\cite{zdemir2007}, quantum storage~\cite{Bacon_Compress:2006,Plesch:2010}, quantum error correction~\cite{ouyang2014permutation}, quantum metrology~\cite{toth2012multipartite}, and quantum networking~\cite{prevedel2009experimental}. 
Dicke states have been used as a starting state for variational optimization algorithms, most notably Quantum Alternating Operator Ansatz (QAOA)~\cite{Hadfield2019}, to find solutions to problems such as Maximum k-vertex Cover~\cite{Brandhofer2022,cook2020quantum}.
The ground states of physical Hamiltonians describing one-dimensional chains tend to show a resemblance to Dicke states such as states resulting from the Bethe ansatz, making them an ideal starting state when investigating the ground state behavior of these Hamiltonians~\cite{TDL_BetheAnsatzDerivation:2010,B_ExcitedStateQuantumPhaseTransitions:2013,DickeTransitions:2021}. 
For instance, the algorithm by \citeauthor{van2021preparing}, who give an algorithm to prepare the Bethe ansatz eigenstates of the spin-1/2 XXZ spin chain, starts by first preparing a Dicke state~\cite{van2021preparing}. 
A Dicke-state preparation protocol based on the compress-uncompress methodology used in the $W$-state furthermore finds applications in entanglement distillation, where the entanglement of a large state is concentrated on only a few qubits. 
Efficient deterministic circuits for preparing Dicke states have been proposed by \citeauthor{bartschi2019deterministic}~\cite{bartschi2019deterministic, bartschi2022deterministic_short_depth}. 
They provide a quantum circuit of depth $\mathO(k \log(\frac{n}{k}))$, allowing arbitrary connectivity, to prepare a Dicke state, which they conjecture to be optimal when $k$ is constant. 
In this work, we provide a constant-depth $\LAQCC$ circuit below their conjectured bound already for constant $k$. 
However, this does not directly disprove their conjecture, as we allow for intermediate measurements and classical computations. 
More significantly, we even construct constant-depth $\LAQCC$ circuits for $k = \mathO(\sqrt{n})$ greatly improving their bound.
This construction extends the compress-uncompress method for the $W$-state combined with additional subroutines. 

We continue with a log-depth state preparation protocol for the Dicke-state for arbitrary $k$. 
This protocol implements an efficient transformation between the factoradic number representation and the combinatorial number representation of a positive integer. 
The combinatorial number representation relates directly to the Dicke state. 
The provided efficient transformation between number representation systems might be of independent interest. 

We conclude by modifying our protocol for preparing a Dicke-state to a protocol that prepares quantum many-body scar states in constant-depth. 
These states have low entanglement and longer coherence times than states with similar energy density.
These characteristics make many-body scar states interesting to analyze and relevant within physics.
Many-body scar states appear for instance in the AKLT model~\cite{AKLT:1987,MRBAR:2018,MRB:2018} and different spin models~\cite{SI:2019,MOBFR:2020}.
Known methods for preparing these states have polynomial-depth~\cite{Gustafson:2023}, whereas our circuit has constant depth. 

% We conclude by studying the power that intermediate classical calculations can add to quantum computations. 
% In this study, we define a new model that relaxes constant-depth quantum circuits to polynomial depth quantum circuits, log-depth classical calculations to unbounded classical computations and a constant number of alternations to a polynomial number of alternations. 
% We call this model $\LAQCC^*$. 
% We study this model by doing a complexity theoretical analysis, where we draw inspiration from the notions of complexity given by \citeauthor{RosenthalYuen:2022}, \citeauthor{MetgerYuen:2023}, and \citeauthor{Aaronson:2004}.
% All three complexity notions are based on the notion of state preparation, instead of more traditional definition of complexity such as the decidability of a computational problem. 
% The first two consider classes based on sequences of quantum states preparable by a polynomial-sized quantum circuit, where the circuits are uniformly generated by a computational class, for instance, the class $\mathsf{PSPACE}$, which results in the complexity class $\mathsf{StatePSPACE}$~\cite{RosenthalYuen:2022,MetgerYuen:2023}.
% The third notion considers a relative complexity, where the complexity is measured between two given states, and is measured by the number of gates, from a given gate-set, required to transform one state in another state~\cite{Aaronson:2004}. 
% For our definition of state preparation complexity, we drop the uniformity constraint from~\cite{RosenthalYuen:2022,MetgerYuen:2023} and define a class as $\mathsf{StateX}$, which refers to states preparable by circuits of type $\mathsf{X}$. 
% As an example, if $\mathsf{X} = \QNC^0$, this results in the class $\mathsf{StateQNC^0}$, which is the set of states preparable from the $\ket{0}^n$ state by poly-size constant-depth circuits. 
% This notion is similar to the relative complexity from~\cite{Aaronson:2004}, where one state is the  $\ket{0}^n$ state and instead of counting the number of gates we consider the set of states preparable by a fixed number of gates. Using this notion of complexity we show that any state preparable by an $\LAQCC^*$ circuit is also preparable by a $\mathsf{PostQPoly}$ circuit, the class of circuits of polynomial depth with an additional post-selection gate. 

\paragraph{Summary of results}
\begin{itemize}
    \item We give a new definition of a computational model that captures the power of the four step process: applying a constant number of layers of one- and two-qubit gates; performing a syndrome measurement; perform a fast classical computation determining corrections; apply corrections. We call this model \emph{Local Alternating Quantum Classical Computations}, or $\LAQCC$ for short. In this model we bound the allowed quantum operations, intermediate classical calculations, and number of rounds separately. In Section~\ref{sec:LAQCC_model} we define this model and give a list of operations based on results from literature contained in this computational model. In some of these operations we explicitly use that we allow for multiple, but at most constant, rounds  of corrections.
    \item  We show show that there exist $\LAQCC$ circuits that can not be weakly simulated in Section~\ref{sec:IQP_in_LAQCC}. We further show that for every $\LAQCC$ circuit there exists a $\QNC^1$ circuit simulating it perfectly, in Section~\ref{sec:LAQCC_in_QNC1}.
    \item We introduce a new type computational complexity for preparing states and show that the extension of $\LAQCC$ where we allow a polynomial number of rounds and unbounded classical computation, is contained in $\mathsf{PostQPoly}$, the class of polynomial circuits with post-selection, in Section~\ref{sec:Complexity results}.
    \item We show a protocol to prepare the uniform superposition state of size $q$ in $\LAQCC$ using $\mathO(\ceil{\log_2(q)}^2)$ qubits in Section~\ref{sec:superposition_modulo_q}. 
    \item We show a protocol to prepare the $W_n$ state in $\LAQCC$ using $\mathO(n\log(n))$ qubits in Section~\ref{sec:W_state_in_LAQCC}.
    \item We show two ways of preparing the Dicke-$(n,k)$ state. The first method is in $\LAQCC$, works up to $k = \mathO(\sqrt{n})$, uses $\mathO(n^2\log(n))$ qubits, and is found in Section~\ref{sec:dicke:small_k}. The second method is in $\LAQCC\text{-}\mathsf{LOG}$ (an extension of $\LAQCC$ allowing for logarithmic number of alterations instead of constant), works for any $k$, uses $\mathO(\text{poly}(n))$ qubits, and is found in Section~\ref{sec:Dicke_in_LAQCC_LOG}. 
    \item We extend on our $\LAQCC$ method of generating Dicke-$(n,k)$ states for $k = \mathO(\sqrt{n})$ and show a protocol to generate many-body scar states for a particular Hamiltonian in $\LAQCC$ (Section~\ref{sec:many_body_scar}). 
\end{itemize}
Summarized in a table, we provide the following state generation protocols:
\begin{table}[htb]
\centering
\begin{tabular}{l|l|l|l}
\textbf{State description} & \textbf{Width} & \textbf{Depth} & \textbf{Implementation}\\
\hline 
Uniform superposition mod $q$: $\frac{1}{\sqrt{q}} \sum_{i = 0}^{q-1}\ket{i}$ & $\mathO(\ceil{\log^2 q})$ & $\mathO(1)$ & Section~\ref{sec:superposition_modulo_q}\\

$W$-state: $\frac{1}{\sqrt{n}}\sum_{i = 0}^{n-1}\ket{e_i}$ & $\mathO(n \log n)$ & $\mathO(1)$ & Section~\ref{sec:W_state_in_LAQCC}\\

Dicke-$(n,k)$, $k = \mathO(\sqrt{n})$: $\binom{n}{k}^{-1/2}\sum_{x \in \{0,1\}^n: |x| = k} \ket{x}$ &  $\mathO(n^2\log n)$ & $\mathO(1)$ 
&Section~\ref{sec:dicke:small_k}\\

Dicke-$(n,k)$: $\binom{n}{k}^{-1/2}\sum_{x \in \{0,1\}^n: |x| = k} \ket{x}$ & $\mathO(\text{poly}(n))$ & $\mathO(\log n)$ &Section~\ref{sec:Dicke_in_LAQCC_LOG}\\

QMBS: $\ket{S_k} = \frac{1}{k! \sqrt{\mathcal N(n,k)}}(Q^\dagger)^k \ket{\Omega}$ &  $\mathO(n^2\log n)$ & $\mathO(1)$  &  Section~\ref{sec:many_body_scar}
\end{tabular}
\caption{Summary of state preparation protocols given in this paper.}
\label{tab:sate_prep}
\end{table}
In the entry for the quantum many-body scar state $Q$ denotes the raising operator and $\mathcal N(n,k)=\binom{n-k-1}{k}$. 
Section~\ref{sec:many_body_scar} will provide more details on the variables and the implementation. 

\paragraph{Organization of the paper}
\noindent We first introduce relevant preliminaries in Section~\ref{sec:preliminaries}. 
In Section~\ref{sec:LAQCC_model} we formally define the class of Local Alternating Quantum-Classical Computations ($\LAQCC$). We also show that any Clifford circuit can be implemented in constant depth $\LAQCC$ (a result based on a result from measurement-based quantum computing~\cite{jozsa2006introduction}). 
This result allows us to give many useful multi-qubit gates and routines in Section~\ref{sec:gates_created_in_LAQCC}. 
Beyond that we show that constant depth $\LAQCC$ circuits are contained in $\QNC^1$ and that any $\mathsf{IQP}$ circuit has an $\LAQCC$ implementation.
We conclude this section with an analysis of a more powerful instantiation of $\LAQCC$ and show an inclusion with respect to the class $\mathsf{PostQPoly}$, which is the class of circuits of polynomial depth with one additional post-selection gate. 
In Section~\ref{sec:state_prep_in_LAQCC} we give $\LAQCC$ circuit implementations for preparing the uniform superposition over an arbitrary number of states, the $W$-state and the Dicke state up to $k = \mathO(\sqrt{n})$. We furthermore give a log-depth circuit implementation for preparing the Dicke state for any $k$. We conclude by showing a $\LAQCC$ circuit for generating many body scar states of a particular type of Hamiltonian.


\section{Related Work}
\label{appsec: related work}
Bayesian causal discovery literature has primarily focused on inference in linear models with closed-form posteriors or marginalized parameters. Early works considered sampling directed acyclic graphs (DAGs) for discrete~\cite{cooper1992bayesian, madigan1995bayesian, heckerman2006bayesian} and Gaussian random variables~\cite{friedman2003being, tong2001active} using Markov chain Monte Carlo (MCMC) in the DAG space. However, these approaches exhibit slow mixing and convergence~\cite{eaton2012bayesian,grzegorczyk2008improving}, often requiring restrictions on number of parents~\cite{kuipers2017partition}. %Alternative exact dynamic programming methods are limited to small settings~\cite{koivisto2012advances}. 

Recent advances in variational inference~\cite{zhang2018advances} have facilitated graph inference in DAG space, with gradient-based methods employing the NOTEARS DAG penalty \cite{zheng2018dags}.\cite{annadani2021variational} samples DAGs from autoregressive adjacency matrix distributions, while \cite{lorch2021dibs} utilizes Stein variational approach \cite{liu2016stein} for DAGs and causal model parameters. \cite{cundy2021bcd} proposed a variational inference framework on node orderings using the gumbel-sinkhorn gradient estimator \cite{mena2018learning}. \cite{deleu2022bayesian,nishikawa2022bayesian} employ the GFlowNet framework \cite{bengio2021gflownet} for inferring the DAG posterior. Most methods, except\cite{lorch2021dibs} are restricted to linear models, while \cite{lorch2021dibs} has high computational costs and lacks DAG generation guarantees compared to our method.
% at least quadratic scaling complexity, both with respect to the number of nodes (due to the DAG penalty) as well as number of posterior samples. Our proposed approach instead has linear complexity with respect to number of posterior samples and does not require any additional DAG penalty.     

In contrast, \emph{quasi-Bayesian} methods, such as DAG bootstrap \cite{friedman2013data}, demonstrate competitive performance. DAG bootstrap resamples data and estimates a single DAG using PC \cite{spirtes2000causation}, GES \cite{chickering2002optimal}, or similar algorithms, weighting the obtained DAGs by their unnormalized posterior probabilities. Recent neural network-based works employ variational inference to learn DAG distributions and point estimates for nonlinear model parameters \cite{charpentier2022differentiable,geffner2022deep}.
\section{Secure Design of \puma}\label{sec:design}
In this section, we first present an overview of \puma, and present the protocols for secure $\gelu$ , $\softmax$, embedding, and $\layernorm$ used by \puma. Note that the linear layers such as matrix multiplication are straightforward in replicated secret sharing, so we mainly describe our protocols for non-linear layers in this manuscript.

\subsection{Overview of \puma}\label{sec:overview}
To achieve secure inference of Transformer models, \puma\ defines three kinds of roles: one model owner, one client, and three computing parties. The model owner and the client  provide their models or inputs to the computing parties (i.e., $P_0$, $P_1$, and $P_2$) in a secret-shared form, then the computing parties execute the MPC protocols and send the results back to the client. Note that the model owner and client can also act as one of the computing party, we describe them separately for generality. \eg, when the model owner acts as $P_0$, the client acts as  $P_1$, a third-party dealer acts as $P_2$, the system model becomes the same with \mpcformer~\citep{li2023mpcformer}.

During the secure inference process, a key invariant is maintained: For any layer, the computing parties always start with 2-out-of-3 replicated secret shares of the previous layer's output and the model weights, and end with 2-out-of-3 replicated secret shares of this layer's output. As the shares do not leak any information to each party, this ensures that the layers can be sequentially combined for arbitrary depths to obtain a secure computation scheme for any Transformer-based model.
%The main focus of \puma\ is to reduce the computation and communication costs between the computing parties while maintaining the desired level of security. 



\iffalse
\textbf{Threat Model.}
Following previous works~\citep{aby3,li2023mpcformer},
\puma\ resists a semi-honest (a.k.a., honest-but-curious) adversary in honest-majority~\citep{lindell2009proof}, where the adversary passively corrupts no more than one computing party. Such an adversary follows the protocol specification exactly, but may try to learn more information than permitted. Please note that \puma\ cannot protect against the extraction of information from the inference results, and the examination of mitigating solutions (\eg, differential privacy~\citep{abadi2016deep}) falls outside the scope of this study.
\fi 

\subsection{Protocol for Secure GeLU}\label{sec:gelu}
Most of the current approaches view the $\gelu$ function as a composition of smaller functions and try to optimize each piece of them, making them to miss the
chance of optimizing the private $\gelu$ as a whole. Given the $\gelu$ function:
\begin{equation}\label{eq:gelu}
\begin{split}
    \gelu(x) &= \frac{x}{2} \cdot \left(1 + \tanh \left( \sqrt{\frac{2}{\pi}} \cdot \left(x + 0.044715 \cdot x^3 \right) \right) \right)\\
    &\approx x\cdot \mathsf{sigmoid}(0.071355\cdot x^3 + 1.595769\cdot x) 
\end{split},
\end{equation}
these approaches~\citep{hao2022iron,characmpctranformer} focus either on designing efficient protocols for function $\tanh$
or using the existing MPC protocols of exponentiation and reciprocal for $\mathsf{sigmoid}$. 

However, none of current approaches have utilized the fact that $\gelu$ function is almost linear on the two sides (\ie, $\gelu(x)\approx 0$ for $x<-4$ and $\gelu(x)\approx x$ for $x>3$). 
Within the short interval $[-4,3]$ of $\gelu$,
we suggest a piece-wise approximation of low-degree polynomials is a more efficient and easy-to-implement choice for its secure protocol. Concretely, our piece-wise low-degree polynomials are shown as equation~(\ref{eq:geluapprox}):
\begin{equation}\label{eq:geluapprox}
\gelu(x)=
\begin{cases}
0, & x<-4 \\
F_0(x), & -4 \le x < -1.95 \\
F_1(x), & -1.95 \le x \le 3 \\
x, & x >3
\end{cases},
\end{equation}
where polynomials $F_0()$ and $F_1()$ are computed by library $\mathsf{numpy.ployfit}$\footnote{\url{https://numpy.org/doc/stable/reference/generated/numpy.polyfit.html}} as equation~(\ref{eq:f0f1}). Surprsingly, the above simple poly fit works very well and our $\mathsf{max\ error}< 0.01403$, $\mathsf{median\ error}< 4.41e-05$, and $\mathsf{mean\ error}< 0.00168$.
\begin{equation}\label{eq:f0f1}
\begin{cases}
F_0(x) &= -0.011034134030615728 x^3 -0.11807612951181953 x^2 \\
&- 0.42226581151983866 x -0.5054031199708174\\
F_1(x) &= 0.0018067462606141187x^6 -0.037688200365904236 x^4 \\
&+ 0.3603292692789629x^2 + 0.5x + 0.008526321541038084
\end{cases}
\end{equation}

Formally, given secret input $\share{x}$, our secure $\gelu$ protocol $\Pi_{\gelu}$ is constructed as algorithm~\ref{protocol:gelu}. 
\iffalse
\begin{itemize}
    \item The parties jointly compute
$\share{b_0}^2 = \Pi_{\mathsf{LT}}(\share{x}, 4)$,
$\share{b_1}^2 = \Pi_{\mathsf{LT}}(\share{x}, -1.95)$, and
$\share{b_2}^2 = \Pi_{\mathsf{LT}}(3, \share{x})$.

\item  Then, each $P_i$ locally compute
$\share{b_3}^2 = \share{b_1}^2 \oplus \share{b_2}^ \oplus 1$ and
$\share{b_4}^2 = \share{b_0}^2 \oplus \share{b_1}^2$

\item Finally, the parties compute and return 
$\share{b_2}^2 \cdot \share{x} + \share{b_4}^2 \cdot F_0(\share{x}) + \share{b_3}^2 \cdot F_1(\share{x})$, where polynomials $(F_0, F_1)$ can be computed easily using secure addition and multiplication (and its variants, \eg, secure square)~\citep{spu}. 
\end{itemize}
\fi 

\begin{algorithm}[tp]
\caption{Secure $\gelu$ Protocol $\Pi_{\mathsf{GeLU}}$}\label{protocol:gelu}
\begin{algorithmic}[1]
\REQUIRE
$P_i$ holds the 2-out-of-3 replicate secret share $\share{x}_i$ for $i\in \{0,1,2\}$ 
\ENSURE
$P_i$ gets the 2-out-of-3 replicate secret share $\share{y}_i$ for $i\in \{0,1,2\}$, where $y=\gelu(x)$.

\STATE $P_0$, $P_1$, and $P_2$ jointly compute
\begin{equation*}
\begin{split}
&\shareb{b_0} = \Pi_{\mathsf{LT}}(\share{x}, -4),~~~\vartriangleright b_0 = 1\{x<-4\}\\
&\shareb{b_1} = \Pi_{\mathsf{LT}}(\share{x}, -1.95),~~~\vartriangleright b_1 = 1\{x<-1.95\} \\
&\shareb{b_2} = \Pi_{\mathsf{LT}}(3, \share{x}),~~~~~~\vartriangleright b_2 = 1\{3<x\}
\end{split}
\end{equation*}
and compute 
$\shareb{z_0} = \shareb{b_0} \oplus \shareb{b_1}$,
$\shareb{z_1} = \shareb{b_1} \oplus \shareb{b_2} \oplus 1$, and $\shareb{z_2}=\shareb{b_2}$. Note that $z_0 = 1\{-4\le x < -1.95\}$, $z_1 = 1\{-1.95\le x\le 3\}$, and $z_2 = 1\{x>3\}$.

\STATE Jointly compute $\share{x^2} = \Pi_{\mathsf{Square}}(\share{x})$, $\share{x^3} = \Pi_{\mathsf{Mul}}(\share{x}, \share{x^2})$, $\share{x^4} = \Pi_{\mathsf{Square}}(\share{x^2})$, and $\share{x^6} = \Pi_{\mathsf{Square}}(\share{x^3})$.

\STATE Computing polynomials $\share{F_0(x)}$ and $\share{F_1(x)}$ based on $\{\share{x}, \share{x^2}, \share{x^3}, \share{x^4}, \share{x^6}\}$ as equation~(\ref{eq:geluapprox}) securely.


\RETURN$\share{y} = \Pi_{\mathsf{Mul_{BA}}}(\shareb{z_0}, \share{F_0(x)}) + \Pi_{\mathsf{Mul_{BA}}}(\shareb{z_1}, \share{F_1(x)})+\Pi_{\mathsf{Mul_{BA}}}(\shareb{z_2}, \share{x})$.

\end{algorithmic}
\end{algorithm}



\subsection{Protocol for Secure Softmax}\label{sec:secureatten}

In the function $\attention(\Q,\K,\V)=
\softmax(\Q \cdot \K^\mathsf{T} + \M) \cdot \V$, where $\M$ can be viewed as a bias matrix, the key challenge is computing function $\softmax$. For the sake of numerical stability, the $\softmax$ function is computed as
\begin{equation}\label{eq:softmax}
    \softmax(\x)[i]=\frac{\exp(\x[i] - \bar{x} - \epsilon)}{\sum_i \exp(\x[i] - \bar{x} - \epsilon)},
\end{equation}
where $\bar{x}$ is the maximum element of the input vector $\x$. 
For the normal plaintext softmax, $\epsilon=0$. For a two-dimension matrix, we apply equation~(\ref{eq:softmax}) to each of its row vector.

Formally, our detailed secure protocol  $\Pi_{\softmax}$ is illustrated in algorithm~\ref{protocol:softmax}, where we propose two optimizations:
\begin{itemize}
\item 
For the first optimization, we set $\epsilon$ in equation~\ref{eq:softmax} to a tiny and positive
value, e.g., $\epsilon =
10^{-6}$, so that the inputs to exponentiation
in equation~\ref{eq:softmax} are all negative. We exploit the negative operands
for acceleration. Particularly, we compute the exponentiation using the Taylor series~\citep{tan2021cryptgpu} with a simple clipping
\begin{equation}\label{eq:negexp}
\mathsf{negExp}(x) = \begin{cases}
    0, &x < T_{\exp} \\
    (1+\frac{x}{2^t})^{2^t}, &x\in [T_{\exp},0].
\end{cases}
\end{equation}
Indeed, we apply the less-than for the branch $x < T_{\exp}$
The division by $2^t$ can be achieved using
$\Pi_{\mathsf{Trunc}}^t$ since the input is already negative. Also, we can
compute the power-of-$2^t$ using $t$-step sequences of square function $\Pi_{\mathsf{square}}$ and $\Pi_{\mathsf{Trunc}}^f$. Suppose our MPC program uses
$18$-bit fixed-point precision. Then we set $T_{\exp}=-14$ given $\exp(-14) < 2^{-18}$, and empirically set $t = 5$.


\item 
Our second optimization is to reduce the number of divisions, which ultimately saves computation and communication costs.
To achieve this, for a vector $\x$ of size $n$, we have replaced the operation $\mathsf{Div}(\x, \mathsf{Broadcast}(y))$ with $\x \cdot  \mathsf{Broadcast}(\frac{1}{y})$, where $y=\sum_{i=1}^n\x[i]$. By making this replacement, we effectively reduce $n$ divisions to just one reciprocal operation and $n$ multiplications.
This optimization is particularly beneficial in the case of the $\softmax$ operation. The $\frac{1}{y}$ in the $\softmax$ operation is still large enough to maintain sufficient accuracy under fixed-point values. As a result, this optimization can significantly reduce the computational and communication costs while still providing accurate results.
\end{itemize}

\begin{algorithm}[tp]
\caption{Secure $\softmax$ Protocol $\Pi_{\softmax}$}\label{protocol:softmax}
\begin{algorithmic}[1]
\REQUIRE
$P_i$ holds the 2-out-of-3 replicate secret share $\share{\x}_i$ for $i\in \{0,1,2\}$, and $\x$ is a vector of size $n$. 
\ENSURE
$P_i$ gets the 2-out-of-3 replicate secret share $\share{\y}_i$ for $i\in \{0,1,2\}$, where $\y=\softmax(\x)$.

\STATE $P_0$, $P_1$, and $P_2$ jointly compute
$\shareb{\mathbf{b}} = \Pi_{\mathsf{LT}}(T_{\exp}, \share{\x})$ and the maximum $\share{\bar{x}} = \Pi_{\mathsf{Max}}(\share{\x})$.

\STATE Parties locally computes $\share{\hat{\x}} = \share{\x} - \share{\bar{x}} - \epsilon$, and jointly compute $\share{\z_0} = 1+  \Pi_{\mathsf{Trunc}}^t(\share{\hat{\x}})$.

\FOR{$j=1,2,\dots, t$}
\STATE $\share{\z_j} = \Pi_{\mathsf{Square}}(\share{\z_{j-1}})$.
\ENDFOR

\STATE Parties locally compute $\share{z} = \sum_{i=1}^n \share{\z[i]}$ and jointly compute $\share{1/z} = \Pi_{\mathsf{Recip}}(\share{z})$.

\STATE Parties jointly compute $\share{\z / z} = \Pi_{\mathsf{Mul}}(\share{\z}, \share{1/z})$

\RETURN $\share{\y} = \Pi_{\mathsf{Mul}_{\mathsf{BA}}}( \shareb{\mathbf{b}}, \share{\z / z})$.

\end{algorithmic}
\end{algorithm}

\subsection{Protocol for Secure Embedding}\label{sec:embed}


The current secure embedding procedure described in~\citep{li2023mpcformer} necessitates the client to  generate a one-hot vector using the token $\tokenid$ locally. This deviates from a plaintext Transformer workflow where the one-hot vector is generated inside the model. As a result, they have to carefully strip off the one-hot step from the pre-trained models, and add the step to the client side, which could be an obstacle for deployment. 



To address this issue, we propose a secure embedding design as follows. Assuming that the token $\tokenid\in [n]$ and all embedding vectors are denoted by $\E= (\e_1^T, \e_2^T, \dots, \e_n^T)$, the embedding can be formulated as $\e_{\tokenid} = \mathbf{E}[\tokenid]$. Given $(\tokenid, \E)$ are in secret-shared fashion, our secure embedding protocol $\Pi_{\mathsf{Embed}}$ works as follows:
\begin{itemize}
    \item The computing parties securely compute the one-hot vector $\shareb{\mathbf{o}}$ after receiving $\share{\tokenid}$ from the client. Specifically, $\shareb{\mathbf{o}[i]}=\Pi_{\mathsf{Eq}}(i,\share{\tokenid})$ for $i\in [n]$.
    \item The parties can compute the embedded vector via $\share{\e_{\tokenid}} = \Pi_{\mathsf{Mul_{BA}}}(\share{\E}, \shareb{\mathbf{o}})$, where  does not require secure truncation.
\end{itemize}
In this way, our $\Pi_{\mathsf{Embed}}$ does not require explicit modification of the workflow of plaintext Transformer models, at the cost of more $\Pi_{\mathsf{Eq}}$ and $\Pi_{\mathsf{Mul_{BA}}}$ operations. 



\subsection{Protocol for Secure LayerNorm}\label{sec:seclayernorm}
Recall that given a vector $\x$ of size $n$, $\layernorm(\x)[i] =  \gamma \cdot \frac{\x[i]-\mu}{\sqrt{\sigma}} + \beta$, where $(\gamma, \beta)$ are trained parameters, $\mu = \frac{\sum_{i=1}^n \x[i]}{n}$, and $\sigma = \sum_{i=1}^n (\x[i] - \mu)^2$. In MPC, the key challenge is the evaluation of the divide-square-root $\frac{\x[i]-\mu}{\sqrt{\sigma}}$ formula. To securely evaluate this formula, CrypTen sequentially executes the MPC protocols of square-root, reciprocal, and multiplication. However, we observe that $\frac{\x[i]-\mu}{\sqrt{\sigma}}$ is equal to $(\x[i]-\mu)\cdot \sigma^{-1/2}$. And in the MPC side, the costs of computing the inverse-square-root $\sigma^{-1/2}$ is similar to that of the square-root operation~\citep{rSqrt}. Besides, inspired by the second optimization of \S~\ref{sec:secureatten}, we can first compute $\sigma^{-1/2}$ and then $\mathsf{Broadcast}(\sigma^{-1/2})$ to support fast and secure $\layernorm(\x)$. And our formal protocol $\Pi_{\layernorm}$ is shown in algorithm~\ref{protocol:layernorm}.

\begin{algorithm}[tp]
\caption{Secure $\mathsf{LayerNorm}$ Protocol $\Pi_{\mathsf{LayerNorm}}$}\label{protocol:layernorm}
\begin{algorithmic}[1]
\REQUIRE
$P_i$ holds the 2-out-of-3 replicate secret share $\share{\x}_i$ for $i\in \{0,1,2\}$, and $\x$ is a vector of size $n$. 
\ENSURE
$P_i$ gets the 2-out-of-3 replicate secret share $\share{\y}_i$ for $i\in \{0,1,2\}$, where $\y=\mathsf{LayerNorm}(\x)$.

\STATE $P_0$, $P_1$, and $P_2$ compute $\share{\mu} = \frac{1}{n}\cdot \sum_{i=1}^n\share{\x[i]}$ and $\share{\sigma} = \sum_{i=1}^n \Pi_{\mathsf{Square}}(\share{\x} - \share{\mu})[i]$.

\STATE Parties jointly compute $\share{\sigma^{-1/2}} = \Pi_{\mathsf{rSqrt}}(\share{\sigma})$.

\STATE Parties jointly compute $\share{\mathbf{c}} = \Pi_{\mathsf{Mul}}((\share{\x} - \share{\mu}), \share{\sigma^{-1/2}})$

\RETURN $\share{\y} = \Pi_{\mathsf{Mul}}(\share{\gamma}, \share{\mathbf{c}}) + \share{\beta}$.

\end{algorithmic}
\end{algorithm}


\section{Text-To-Design} 
\label{sec:text_to_design}

% A pivotal step in getting a large language model (LLM) to generate a design involves crafting a prompt that clearly delineates the design language to be employed. This encompasses the specification of both the language's syntax and the semantics associated with each operator, presented in a manner that the LLM can comprehend. 

% We usually accomplish this via a three-step process. We first develop a succinct language specification that defines the operators and outlines their respective inputs and outputs.
% Following this, we demonstrate an example employing this language to devise a simple model.
% Finally, we incorporate additional semantic information about the model or design guidelines that can assist the LLM.

% Next, we will delve into how to establish design prompts across four domains: 2D vector drawings, 3D parametric geometry, actuated systems, and electronics. Each of these domains adheres to similar specifications. In each of these domains, we discuss both prompt generation and algorithms for converting outputs from the LLM to standard formats 


% \wojciech{Too much focus on spatial}
For our first line of inquiry, we explore the extent to which \gpt is able to generate designs across a variety of domains. 
Even within the specific context of manufacturable design, the concept of a ``design'' is quite broad, and exists at many scales.
For example, we may want to specify a single self-contained part, or a sizable hierarchical assembly containing several levels of sub-assemblies and/or other individual component modules. 
Such assemblies may be completely customized/self-contained, with all parts designed simultaneously, or they may be hybrid designs that integrate existing, pre-manufactured elements such as brackets or motors.
In many cases, our target design tasks also include dynamic considerations such as assembly mating or articulated joints.

Although these complex tasks may initially seem out-of-scope for lexical models such as {\llm}s, there are many modeling and design paradigms that can be expressed in terms of potentially-\llm-compatible language.
To guide our exploration of \gpt's ability to interface with each of these models, we pose the following questions:

\begin{itemize}
\item \textbf{Q1} Can \gpt generate a meaningful design when provided with a high-level description of the goal and a given modeling language?
\item \textbf{Q2} To what extent is the user able to control the designs created by \gpt? Is \gpt able to interpret and respect user-defined constraints, such as spatial relationships between objects or integration of standard pre-fabricated parts?
\item \textbf{Q3} Is \gpt able to incorporate high-level abstractions used by human designers, such as modular (de)composition?
\end{itemize}

% \wojciech{For the copter example, you first need get the appropriate components (e.g., What are the main components of the copter?). There should be a question on this.}\liane{part sourcing is addressed in the manufacturing section. All examples in this section explicitly assume that we already know the parts we want (we also make an explicit reference to the fact that we'll explore gpt sourcing in section 6.}


\subsection{Simple, self-contained designs from high-level input (Q1)}

% Can \gpt successfully instantiate and coordinate (a set of)  primitives from a given modeling language in order to build/approximate a meaningful design? Does \gpt seem particularly well- (or poorly-) suited to any particular modeling paradigm(s)?
%Can \gpt generate a meaningful design when provided with a high-level description of the goal and a set of primitives from a given modeling language?


To explore \gpt's capacity for design, we first test its ability to do one- (or few-) shot generation of an object from a minimal high-level text description as input. 
Ideally, we would like to understand \gpt's ability to complete design tasks independent of any particular modeling paradigm.
However, it is not immediately clear how much dependence there may be on the specific representation that is chosen, because the variation in possible language-based modeling paradigms is significant. 
Some languages are very general and versatile, with a wide variety of features and capabilites, while others may be highly-specialized for a specific set of tasks or outcomes. 
Similarly, some languages are well-established with plentiful online documentation or examples, while others may be custom-defined, poorly documented, or otherwise underrepresented in \gpt's training repository. 
Finally, some languages are fairly streamlined, while others may be syntactically complex and/or require the use/coordination of many modules.
Each possibility offers unique capabilities and challenges. 
Thus, we set out to test a wide variety of them, in an effort to determine 
\llms' ability to use each representation;
whether there are any conclusions that seem to span across different representations;
and
whether any particular representations seem uniquely well- or poorly-suited for \llm integration.


% \liane{for an initial draft of each section, it'd be great to list out e.g. what kind of representation you're using (if \gpt needs to use an API/function calls, generate a mesh directly, come up with a mix of discrete components/continuous values, etc.); how much you had to tell/teach \gpt vs. whether it could just use the API directly; the examples/experiments you tried; what \gpt was good at/struggled with; and include [not necessarily polished] figures of your results. Things might shuffle around, but this will give us a good starting point}


\subsubsection{Vector Graphics with SVG/DXF} 
%\team{Mike}


% Our first design domain is 2D vector graphics.
% Vector formats like SVGs or DXFs are commonly used to describe manufacturing files, 
% for example for laser cutting.
% We want to investigate whether \gpt could empower designers to convert text directly into vector files, 
% readily usable for manufacturing.
% To test this, we have experimented with generating an SVG file that could be used to create a manufacturing file. 
% We also experimented with converting a DXF format of the same design.  

% In the experiment, the goal was to design and construct an SVG file of a cabinet with specific dimensions using 1/2 inch plywood. The cabinet was intended to have three shelves, with the cabinet having overall dimensions measuring 6 feet in height, 1 foot in depth, and 4 feet in width. The experiment aimed to accurately account for the thickness of the plywood when designing the cabinet, ensuring that the dimensions of the various parts were adjusted accordingly. To achieve this, a Python script was developed to generate an SVG file representing the cabinet layout using \gpt. The script calculated the required clearances for the wood thickness and appropriately positioned the side panels, top and bottom panels, shelves, and back panel, while considering the specified spacing between the parts and utilized svgwrite to generate the SVG file. The resulting SVG file provided a visual representation of the cabinet's design that could be used for cutting out the parts. Similarly, we repeated this experiment to create a DXF file where \gpt utilized ezdxf to generate the file. Results are shown in Figure \ref{fig:SVG_DXF_Gen}.  \gpt was able to use the APIs to generate the file in the correct format without any simplification, however, multiple iterations were needed to ensure \gpt did not overlap the parts of the cabinet.  

% \liane{is there something specific we want to highlight from the chat excerpt? It's very long right now and I'm not sure what is the interesting part to look at / message to take away. also, would be nice to have a visual of the svgs} \adriana{yes, this figure is strange}\bolei{probably it is better to show the patterns of svg files instead of the code?}
% \wojciech{This is written as a manufacturing section not design section.}

Our initial focus in the design domain is on 2D vector graphics.
Vector formats such as SVGs or DXFs are prevalently utilized in manufacturing processes,
like those for laser or waterjet cutting.
The goal of our investigation was to ascertain whether \gpt could empower designers to 
transform their text directly into these vector formats. 
To evaluate this, we conducted experiments to determine 
if \gpt is capable of generating a valid SVG file and converting the design into DXF format.

The primary aim of our experiment was to design an SVG file for a cabinet, 
with predetermined dimensions, to be constructed from 1/2 inch plywood. 
This implies that the thickness of each wall, a preset parameter, is 0.5 inches.
The experimental setup involved the design of a cabinet comprising three shelves, 
with overall dimensions measuring 6 feet in height, 1 foot in depth, and 4 feet in width.
A crucial aspect of the investigation was to see if \gpt could accurately account for this wall thickness 
during the design of the cabinet, appropriately adjusting the dimensions of its various components.
\gpt was able to design the specified cabinet and 
subsequently generated a Python script to create an SVG file reflecting the cabinet's layout.
The script considered the necessary clearances for the thickness and 
accurately positioned the side panels, top and bottom panels, shelves, and back panel.
Moreover, it factored in the prescribed spacing between parts and 
leveraged `svgwrite' to generate the SVG file. 
The resulting SVG file provided a visual depiction of the cabinet's design.
We also replicated the experiment to create a DXF file, 
where \gpt utilized `ezdxf' to generate the file.
The results of these experiments are depicted in Figure \ref{fig:SVG_DXF_Gen}.

In conclusion, \gpt demonstrated its capability to employ the APIs for 
generating the vector file in the correct format without any simplifications.
Nevertheless, it was necessary to perform several iterations to 
ensure \gpt did not cause any overlap among the cabinet parts.

% Figure environment removed
\afterpage{\FloatBarrier}



\subsubsection{CSG with self-defined primitives}
\label{sec:textToDesign_CSG_boxes}
%% Text to Design: CSG with self defined primitives
% \team{Bohan}

The next design domain we are investigating is CSG.
As outlined in Sec.\ref{sec:overview_domains_design}, CSG languages generally operate by building up a collection of primitives that have been altered or combined via linear transformations and Boolean operations. 
Because the associated design logic can be quite complex, it was not immediately clear that \gpt should be able to generate designs using these languages.
Thus, to progressively test \gpt's modeling capabilities, we begin by exploring a very simple, custom CSG language based on a single primitive: a box.

Boxes are one of the most common primitives seen in manufacturing. 
Moreover, many shapes can be considered as a combination of boxes with different sizes.
Because of the simplicity of a box or any shape formed by the boxes, 
we would like to see if \gpt is able to generate designs of such kind of simple shapes, 
such as tables and chairs.

Our initial approach to this task is performed in 2D.
We provide a function, foo(x, y, w, h), 
which forms a box of dimensions $w \times h$ centred at the point $(x, y)$. 
We subsequently employ this function to generate letters composed of axis-aligned bars, 
such as `F' and `E'.
During the testing phase, we observed that 
while the system understands the requirement of 2D boxes, 
it struggles with their accurate placement. 
A particularly prominent issue is the collision problem. 
More specifically, the \gpt system fails to determine 
whether two boxes are overlapping or whether there is a vacant space between them. 
This issue is observable when creating letters like `T' and `E'.
Using three to five targeted prompts enabled \gpt to ascertain the correct positions.
However, these prompts had to be granular and often involved providing the direct solution. 
The outcomes of these attempts are demonstrated in Figure~\ref{fig:letter}.
Interestingly, after addressing this issue, \gpt appears to retain the corrections. 
This is evidenced by its successful generation of the new letters `F' and `L' in a single attempt. 
These letters share a similar structure to `T' and `E', 
and the results can be seen in Figure~\ref{fig:letter}.

Our next step involved venturing into 3D, 
which holds more practical values. 
Analogous to the 2D scenarios, 
we inform \gpt of a pre-established function, box(x, y, z, w, h, d), 
which generates a 3D box of dimensions $w \times h \times d$ centred at the 3D coordinates $(x, y, z)$.
We then tested if \gpt could write a program to produce a simple box of specified dimensions, 
for instance, $100 \times 100 \times 40$, utilizing function `box'. 
\gpt successfully accomplished this task, and the resulting text explanation 
illustrates its understanding of the box concept and the usage of our predefined function.
Next, we presented a more complex challenge: having \gpt design a simple table, 
typically consisting of four legs and a tabletop in the real world. 
We posed the question of whether \gpt could craft a program to
generate such a table with a provided size using solely our box function.
The output text explanation revealed that 
\gpt accurately comprehends the structure of a basic table.
Given that we only provide the overall table size, 
\gpt lacks information about individual leg lengths or tabletop thickness. 
Yet, it was able to identify these missing parameters and make reasonable assumptions. 
Consequently, \gpt succeeded in writing a program to represent 
the table by creating five boxes using our predefined function.
Upon visualizing the 3D table, however, 
the relative positioning of each pair of boxes was not always accurate.
We noticed that the tabletop appeared to be suspended in the air, 
not in contact with the legs, as shown in Figure~\ref{fig:tabletable}. 
This difficulty, also observed in our 2D tests (Figure~\ref{fig:letter}), 
pertains to \gpt's understanding of mathematical concepts.
In this instance, we expedited the process by directly providing \gpt with the solution.
We indicated the necessary translations for the misplaced boxes, 
acknowledging that it would take several prompts to rectify the issue otherwise. 
After correcting the floating tabletop, the table appeared as intended, as demonstrated in Figure~\ref{fig:tabletable}.
Therefore, to create a table, it only required two prompts, 
significantly streamlining the procedure for generating a basic table.

% Figure environment removed


Once we successfully generate the table, 
our next more challenging goal is to design a few accompanying chairs.
We tasked \gpt with creating a chair compatible with the table, using only our predefined function. Similar to its approach with the table, 
\gpt successfully deduced the basic structure of a simple chair, 
comprising the seat, four legs, and a backrest.
Unlike the table instance, we didn't observe any `floating' issues in this scenario.
It appears that \gpt might have indeed gleaned some insights from previous experiences,
as we also observed when creating 2D letters. 
After we rectified the letters `T' and `E', 
there were no issues with the remaining letters.
Additionally, \gpt demonstrated comprehension of the concept of 
compatibility by outputting a chair of an appropriate size.
However, it was not successful in all aspects, as depicted in Figure~\ref{fig:chairchair}. 
We attempted to correct the backrest but were unable to do so.
As a result, we had to manually adjust the position,
directing \gpt to the specific lines that needed modification to correct the structure.
The final result can be seen in Figure~\ref{fig:chairchair}.
We believe the root of these issues lies in \gpt's struggles to comprehend geometric concepts,
a difficulty also observed in previous examples.
Despite these hurdles, the process for creating a basic table and chairs
has been considerably simplified.

Our final objective was to position four identical chairs around the table. 
Although theoretically feasible without invoking rotation, 
\gpt failed to generate the chairs with the correct orientations.
We believe this failure stems from the same root cause we've encountered previously, 
namely, \gpt's difficulty in handling mathematical and geometric concepts.
Creating four chairs with correct orientations without the support of rotation 
entails complex geometric transformations. 
\gpt must comprehend that a box rotated 90 degrees 
around its center is equivalent to a swap of its width and depth dimensions.
To alleviate this issue, we expanded our `box` function to include an additional input argument, `angle`, corresponding to a rotation angle around the vertical axis.
With this extension, \gpt was able to create a program using solely the `box` function 
that successfully positioned four chairs around the table with correct orientations, 
as displayed in Figure~\ref{fig:chairchair}. 
We surmise that the introduction of `angle` considerably simplifies the logic behind chair placement, 
enabling \gpt to create such a program.

In conclusion, \gpt exhibits strong understanding of posed questions 
and excels at analyzing requested objects to determine their composition.
However, it demonstrates a weakness in handling geometric and mathematical concepts.
While it can provide nearly accurate solutions when mathematics is involved,
it struggles to comprehend the underlying mathematical principles and, 
as a result, cannot independently correct math-related issues when they arise.


% Figure environment removed

% Figure environment removed



\subsubsection{CSG with PyVista}
%% Text to Design: CSG with PyVista
% \team{Yifei}

Building on \gpt's success generating CSG-like models with boxes, we set out to explore \gpt's capacity to use a larger suite of primitives.
For this, we used an existing 3D visualization library, PyVista, which allows us to create and place a variety of 3D primitives such as spheres and cones.
Thanks to the library's documentation, \gpt is able to automatically assemble a functional python program using PyVista's primitive functions.

We asked \gpt to use PyVista's primitives to model several variations of a fish, including specific bio-inspirations such as goldfish, a manta ray, and a loach (\fref{fig:fishes}). 
\gpt successfully selected and scaled an appropriate set of primitives for each example, and provided sound bio-inspired rationale for its decisions.
In particular, although most of the fish are composed using a sphere for the body, \gpt intuits that a loach would be most effectively approximated by using \lstinline{two cones for the body to give it an elongated shape}.

One area in which \gpt struggled was the determination of the primitives' orientations. 
It often produced results that indicated an internal confusion of some of the axes, or an otherwise flawed approximation of the orientation that would be required to achieve a desired effect.
After engaging in a dialogue with \gpt, it was able to rectify the orientations of the primitives to more closely resemble the target creatures.
While promising, these tests reiterate \gpt's seemingly limited capacity to account for local coordinate frames.

% GPT's rationale:
% \ignore{
% Goldfish: "This code creates a fish that has a larger, more rounded body and smaller, more forward-positioned head, which are more characteristic of a goldfish. The fins are also larger and positioned higher and lower on the body to mimic the large, flowing fins of some types of goldfish."
% Manta ray: "A manta ray has a flatter, wider body and long "wings", or pectoral fins. The tail is thin and long. We can use a large, flat sphere for the body, a thin, wide cone for the pectoral fins, and a long, thin cone for the tail."
% Loach: "A loach is typically elongated and has several fins along the body. We can create a simple representation using multiple cones for the body and fins. For simplicity, let's use two cones for the body to give it an elongated shape, and additional cones for the fins."
% }




 % \adriana{I think the images in this section are taking too much space, I propose a single image with all 4 results as it's not adding much to the discussion.}

% Figure environment removed


% % Figure environment removed

























% =====================
% OLD TEXT
% =====================

\ignore{
\FH{start of old text}

In this conversation, we explored using 3D mesh library to generate and visualize various designs based on constructive solid geometry (CSG) principles  with the aid of \gpt. 
The integration of PyVista, a 3D visualization library, with large language models facilitates an interactive, real-time computational design process. 
This approach simplifies 3D operations and democratizes design by allowing natural language inputs, fostering broader accessibility. 
The dynamic interaction enables efficient exploration of design parameters and serves as an instructive tool for understanding 3D modeling and computational design. 

We started by creating simple 3D primitives such as spheres. 
\gpt is able to assemble a python program automatically using PyVista's primitive functions that meets our design goals.  
Then, we expanded upon this concept to model fish (Fig.~\ref{fig:fish_and_variation}) and its variations with different bio-inspirations such as goldfish (Fig.~\ref{fig:fish_goldfish_variation}). 

We initially attempted to use boolean operations like union to combine different primitives. 
However, this process encountered some issues as PyVista's boolean operations require all the meshes to be composed of triangles, which wasn't the case with some of the primitives used by \gpt. 
This could result from the model being trained on outdated library version.

As a workaround, instead of using boolean operations, we explored a different approach where we directly added all the part meshes to the PyVista plotter for visualization. 
This method effectively bypassed the requirement for the meshes to be triangulated, enabling us to create more complex shapes from primitive elements.

Using this method, we generated a series of parametric designs, demonstrating the flexibility of this approach. We experimented with a variety of aquatic creatures, including a generic fish, a goldfish, a manta ray~\ref{fig:fish-mentaray}, and a loach~\ref{fig:fish-loach}. During the process, we discovered that \gpt has confusion in terms of how to correctly orient individual primitives to achieve desired looks, and we have to manually tweak orientations of the primitives to closely resemble the target creatures.

In conclusion, our exploration highlights the potential of large language models in aiding computational design. 
With an intuitive and interactive dialogue, we can generate a wide range of designs from basic primitives using libraries like PyVista. 
While certain limitations exist, such as \gpt's failure to meet the requirement for triangulated meshes in boolean operations (which could result from being trained from old codebases) as well as confusion in coordinate systems, creative solutions can be found to bypass these and generate complex and versatile designs. 
This shows that with the right tools and approach, large language models can significantly aid in the computational design process, making it more accessible and intuitive.
}

\ignore{
% Figure environment removed
\afterpage{\FloatBarrier}
}


\subsubsection{CSG with \jscad}
\label{sec:textToDesign_JSCAD_basic}
%% Text to Design: CSG with OpenJSCAD
% \team{Liane}

To explore a full-fledged approach for \llm-aided CSG, we test \gpt's ability to generate meaningful designs using the open source javascript-based CSG library, \jscad \citep{jscad}.
\jscad has extensive documentation available online, and we found that \gpt natively possesses a good grasp of the API, its components, and the required code structure.
In particular, it understood that it needed to import each function from the corresponding modules, and that it needed to define and export a function named \lstinline{main}.
For our experiments, we provided \gpt with access to the full API, and generally allowed it to select the appropriate primitives and operations without user interference. 

To test \gpt's design abilities, we ask it to design a simple cabinet with one shelf, as shown in \fref{fig:textToDesign_simpleCabinet}.
\gpt reliably selects and instantiates the required primitives, along with intuitive naming conventions and structure within the \jscad code. 
\gpt's initial orientation of the parts was also generally reasonable, but the specific positioning of each part was often incorrect. 
Despite multiple attempts, \gpt was unable to generate any fully-correct cabinet in a single shot, with no subsequent user intervention.


Moreover, \gpt frequently produced highly disparate results from one run to the next. Even when using an identical prompt on fresh chat environments, \gpt's responses varied widely in terms of their overall code structure, design accuracy, and the specific errors or oversights made. 
\fref{fig:cabinet_vertical_explosion} shows one example of a drastically different design process, even when seeded with the same initial prompt as \fref{fig:textToDesign_simpleCabinet}.

Throughout our experiments, we found that \gpt encountered a few common pitfalls when generating designs in \jscad.
Occasionally, \gpt made small syntatic errors such as generating incorrect boilerplate, importing functions from incorrect modules, or making ``typos'' in API calls -- \eg, trying to import from the \lstinline{boolean} module rather than the correct \lstinline{booleans} module, or calling the \lstinline{cube()} function with parameters that were intended to generate a \lstinline{cuboid()}. 
In an attempt to avoid these pitfalls, we created a small list of ``hints''/``reminders'' for best practices when working with \jscad; this short list was always passed in alongside our initial prompt. See \appref{sec:appx-jscad-hints} for a full listing of these reminders.
Although these reminders seemed to help mitigate these issues, we were unable to eradicate them entirely. 
However, \gpt can easily correct the majority of these issues when they were pointed out by the user.
Often, the process of correcting the issue through prompts and responses was faster than actually adjusting the code manually, making {\llm}s a useful design partner.

One pervasive issue that seemed more difficult to correct was the fact that \gpt had issues positioning the primitives in 3D space.
In particular, \gpt frequently seemed to forget that \jscad positions elements relative to the \textit{center} of a given primitive, rather than an external point on the primitive (\eg, the lower left corner). 
\gpt's arrangements were frequently incorrect due to this issue.
When \gpt is reminded of this convention, it does generally alter the design, but it is not always able to correct the issue. 
If sufficiently many rounds of local edits prove unable to address the alignment issues, we found that it was generally more effective to direct \gpt to disregard all existing measurements, and re-derive the elements' positions from scratch (see \fref{fig:cabinet_vertical_explosion}).

Overall, we find that \gpt is able to generate reasonable \jscad models from high-level input.
However, the design specifications that emerge on the first attempt are rarely fully correct, so users should expect to engage in some amount of corrective feedback or iteration in order to attain the desired result.

% Figure environment removed
%\afterpage{\FloatBarrier}

\newcommand{\cabVertExpImHeight}{2.2cm}
% Figure environment removed

\subsubsection{Sketch-based CAD with OnShape}
\label{sec:textToDesign_OnShape_basic}
%% Text to Design: CAD with OnShape
% \team{Felix}

Another popular method for 3D shape modeling comes from contemporary computer-aided design (CAD) software.
Rather than directly constructing and modifying solid primitives (as in the CSG approaches discussed above), modern parametric CAD systems generally work by lifting planar sketches into 3D and subsequently modifying the 3D geometry.
These sketches are placed on planes, which can be offsetted construction planes, or planar faces of the current 3D model.
The selected sketching plane serves as a local coordinate system in which the sketch primitives are defined.
In graphical user interfaces, this change of coordinate systems is accounted for by letting the user easily align their camera view to a top down view onto the sketch plane.
This change of view effectively comes back to drawing sketches in 2D, removing the cognitive burden of having to think about sketches in 3D.
Despite the lack of graphical assistance, we want to investigate whether \gpt is able to design objects using a sketch-based modeling language.

However, since the graphical assistance is very prevalent in this modeling paradigm, CAD models are mostly constructed via a GUI and not via textual programming, even though textual APIs exist, e.g. Onshape's Featurescript \cite{featurescript}.
Therefore, documentation and examples are less available than for the modeling paradigms from the previous sections.
And indeed, \gpt performs poorly when trying to generate Featurescript code directly, which is why we decided to provide a simplified DSL.

For our experiments, we constructed a single prompt containing the following DSL description:
Our DSL exposes two operators, \lstinline{createSketch} and \lstinline{extrude}, and two sketch primitives, \lstinline{circle} and \lstinline{rectangle}.
Additionally, we provide a construction example using this language of a single leg round table.
Lastly, we also add some hints about how to write the program, e.g. to explicitly use design variables and to write in \lstinline{syntactically correct python}.
All of the output designs generated by \gpt in this section are automatically translated into Onshape PartStudios.
The full prompt can be found in the supplemental material. 
% Figure environment removed

Our first task is the design of a \lstinline{chair with 4 legs, a rectangular seat and a rectangular back}, see Fig.~\ref{fig:cad_chair_design}.
We asked \gpt to perform this task several times and observed the following.
\begin{itemize}
    \item The design sometimes includes cylindrical legs, sometimes rectangular legs.
    \item The design is always constructed in a single direction, the $Z$ direction.
    Our input example of the round table only used the $Z$ direction to select sketching planes, but the description of our language documented the use of other plane directions.
    \item We observe mainly two types of designs: (i) designs which are constructed in both the negative and positive $Z$ direction starting from the seat, see first answer in Fig.~\ref{fig:cad_chair_design}, and (ii) designs which start from a leg, see the second response in Fig.~\ref{fig:cad_chair_design}.
    We observe that the first type of designs has a higher chance of being correct, whereas the second type fails more often.
    The failures are due to changes in the coordinate system.
    For example, when selecting the top plane of the first leg as a sketch plane for the seat, the sketch plane's origin will be in the center of the leg.
    \gpt will often ignore this or won't be able to account for it when pointed out.
    Conversely, when starting with the seat and choosing the lower seat plane as a sketch plane for the legs, it can specify the leg sketch coordinates in global coordinates, since the global origin coincides with the seat's origin.
    The same is true for the backrest.
    
\end{itemize}

From this test, we can observe that \gpt seems to have difficulties translating the coordinate system's origin on the XY plane.

%% Figure environment removed

% Figure environment removed

Next, we want to see if \gpt can account for rotating sketch planes.
To test this, we ask it to design a car.
\gpt always suggests a simple car shape, composed out of 4 cylindrical wheels and a rectangular car body, see Fig.\ref{fig:cad_car_design}.
The difficulty with this shape is that the cylinder sketches of the wheels have to be extruded on the side planes of the car body.
There are a couple different modeling strategies to achieve this, but we observe that \gpt has difficulties coming up with these designs without any further indication.
Instead, it often extrudes the car body along its \lstinline{height}, starting from the ground plane, and then places the wheel circles on the bottom plane of the car, which is also the ground plane.
This has the effect that the car wheels will be extruded vertically.
Although we were able to correct this design in an iterative prompt-based fashion, we had little success engineering the initial prompt in such a way that we could effectively prevent this behavior.

Note that intuitively placing wheels at the bottom of a car body makes sense and that without any graphical feedback, humans could also easily make this mistake.
From this test, we can observe that \gpt is struggling to rotationally change coordinate systems.

To address this, we changed our design language description to allow \gpt to specify sketch primitive coordinates directly in a single global coordinate system. 
Now, a sketch primitive center takes as input three coordinates, which we project in post-processing directly on the selected sketch plane.
The extrude direction is still defined by the sketch plane's normal vector.
This means that \gpt does not have to take coordinate translations into account anymore.
We observe that this change in the DSL led to a higher success rate in generated designs, see second answer in Fig.~\ref{fig:cad_car_design}.

In conclusion, \gpt is able to design models in a sketch-based parametric CAD framework. 
However it is not successful at changing coordinate systems.
In this case, our backup strategy is to use a single global coordinate system.
One possible future direction is to let \gpt communicate with a geometric solver and create a feedback loop.

\afterpage{\FloatBarrier}


\subsubsection{URDF}
\label{sec:textToDesign_urdf}
%% Text to Design: Robots from URDF description

% \team{Wil, maybe with input from Megan/Andy/Allan)} 

% Figure environment removed

The Universal Robot Description Format (URDF) is a common XML-based language for describing articulated structures in robotics. 
URDF files specify a robot's structure (including both visual and collision geometry), joint locations, and dynamics information. 
The URDF format appears well-suited for potential \llm design because it is human-readable and heavily documented online.

\paragraph{Open Chain Robot Arms}
Initially, we asked \gpt to generate simple open chain robots (commonly called ``arms'') with a particular number of links. 
However, when we used the word ''arm'' to prompt \gpt to generate a robot, \gpt was unable to determine that the links should connect at the end.
Most often, \gpt placed the joints such that each link revolved about its center, and the links were not connected to each other (\fref{fig:multi-link-urdf}, initial prompt).
As shown in the subsequent prompts of \fref{fig:multi-link-urdf}, to achieve an arm with two connected links, it was necessary to describe both the joint position relative to the link 
(\lstinline{``the joint origin must be half the link's length past the link origin''}, rather than \lstinline{``the joint origin should be at the end of the link''} ) 
as well as the joint axis (\lstinline{``a revolute joint about the x axis''}).
Given this prompt pattern, \gpt was easily able to generate proper N-link robots. 

%\adriana{I don't really  understand what this section is trying to say. is the point that it can't connect things directly if you don't have that as a constraint of the DSL? if so this would be better expressed if you group fig 10 and 11 in one single discussion like Felix did in Figure 9 }



\paragraph{Wheeled Robots}
Next, we asked \gpt to generate wheeled robots composed of N wheels attached to a central rectangular platform.
A proper design of this type must have wheels that 
(1) are aligned to share an axis of rotation normal to and through the center of their circular faces;
(2) have circular faces displaced along said axis of rotation, and 
(3) contact, but do not intersect, either side of the center platform.
The combination of non-intersection and geometry relation constraints prove challenging for \gpt, which seems to exhibit limited geometric reasoning. 
Initially, we tried to specify these using language-based constraints (i.e. ``the wheels should touch, but not intersect, either side of the platform'').
These proved ineffective, as shown in \fref{fig:urdf-wheeled-constraints} (middle).
To overcome these challenges, we crafted prompts with very explicit numeric constraints (i.e. ``wheels should be offset on the global y axis by half the width of the platform plus half the height of the wheel cylinder''). 
This style of prompt successfully generated a viable result, as shown in \fref{fig:urdf-wheeled-constraints} (right).

As in the case of robot arms, we find that \gpt is immediately able to generalize a successful two-wheeled design into a four-wheeled robot. 
We achieve this by asking for a duplicate, shifted version of the existing wheel configuration, as shown in \fref{fig:urdf-four-wheel}.
However, we were unable to directly generate a successful four-wheel robot; in general, we found that as the number of constraints in a prompt increases, it becomes increasingly likely that \gpt will ignore any individual constraint. 
Thus, rather than directly requesting a four-wheeled robot in a single prompt, we found greater success by first generating a two-wheeled robot and then prompting \gpt to modify the URDF by adding additional wheels than placing the text in a single prompt.

% Figure environment removed


% Figure environment removed



\paragraph{Robot Grippers}
To test the effectiveness of our iterative, multi-prompt approach for building robots of increasing complexity, we seeded \gpt with a successful two-link open chain URDF, then asked it to modify this design into a collection of multi-finger robot grippers. As shown in \fref{fig:urdf-hands}, we were able to build two-, four-, and five-finger grippers using a sequence of prompts to add features and change proportions. 
To create a two-finger gripper, we asked \gpt to use two of the previously generated two-link open chain robots as fingers, separated by a distance equal to half the height of the finger, and connected by a rectangular platform on the base.
The four-finger gripper was similarly derived from the two-link arm by specifying that the hand should consist of four two-link robots right next to each other on a rectangular platform. To specify a five finger hand, we requested a rectangular link that hinges as a base for the thumb, then prompted \gpt to add another finger on that link and to adjust the hand proportions.

% Figure environment removed



























\ignore{

% Figure environment removed

% Figure environment removed

}



\subsubsection{Graph-based DSL}
%% Text to Design: robots from graph description
% \team{Allan}

% \todo{describe the tie to graph-based design, and also how this representation is different from URDF discussed before.}
% \liane{We also want to reduce the amount of direct copy/paste chat history. Can you summarize the chats' key processes/challenges/successes in text, and reduce the chat history to illustrate only a few specific points with visualizations? The full chat history should be in the github repo, and you can include some longer snippets in the appendix too, if you like.}

While designing an entire robot end-to-end using LLMs may not be feasible, we find that \gpt has the ability to reason about the spatial layout of robot components. These spatial layouts are naturally represented as graphs where the nodes are components and edges are connections between them. Unlike URDF, this representation is more general and is applicable in domains outside of simulation.

To generate robot design graphs using \gpt, we first need a text-based graph representation. Our first approach involved asking \gpt to output the popular GraphViz format. While convenient, this format makes it difficult for \gpt to provide metadata for each part (such as motor torque, size) in a format usable by downstream applications. Instead, we take advantage of \gpt's ability to generate Python code that conforms to a provided domain-specific language (DSL). The full DSL is detailed in \appref{sec:appx_graph_robots}. 

When prompted with a small DSL embedded in Python, \gpt is able to write code that selects and places robot components at a high level of abstraction. By supplying a function that translates components in three-dimensional space, we can extract \gpt's concept of each component's position relative to the others. 
% \adriana{This is way too big, please crop also please discuss the limitaitons}

% Figure environment removed

In this example, we ask \gpt to generate a humanoid robot using the provided functions. \gpt makes appropriate calls to \texttt{add\_link} to create nodes in the design graph, \texttt{add\_joint} to create edges between them, and \texttt{translate} to establish their relative positions.

We manually implement the functions described in the prompt in order to visualize the resulting robot topology. The arms are positioned beside the torso, the legs are positioned below, and the head rests on top as expected for a humanoid robot.

We saw similar success when asking \gpt to construct a snake robot, car robot, and scorpion robot. When requesting a robot dog, however, \gpt only adds two legs initially. Specifying a ``robot dog with four legs'' was necessary to obtain the expected behavior. We also encountered difficulties when attempting to obtain a more detailed design for the robot dog. Asking for a ``robot dog with four legs, two links per leg'' produced a graph with two nodes per leg, but \gpt did not position them relative to each other.

% \subsubsection{Gerber file?}
% %% Text to Design: PCB specification from Gerber file
% \team{Young, maybe with input from Alyssa/Mike}


Designs with pre-defined schematics (imported components), use chatgpt to generate command for pcb layout variation.
electronics example of 8 LEDs as an array, 8:1 multiplexer, ADC, and microcontroller (arduino). Use KiCAD and python-based code



\subsubsection{Summary Discussion}
In light of these experiments, we conclude that \gpt is capable of generating designs based on high-level text input, even across a wide variety of representations and problem domains. 
We note that several of \gpt's capabilites and limitations remain consistent independent of the representation.
For example, in all cases, \gpt is able to generate sensible, well-structured code with semantically meaningful variables and comments.
Moreover, independent of the representation or the problem domain, \gpt consistently shows superior performance with respect to the high-level, \textit{discrete} elements of a problem (\eg, identifying the correct type and quantity of each primitive/operation) as opposed to the lower-level continuous parameter assignments (\eg, correctly positioning the primitives relative to one another). 
A more detailed discussion of capabilities, limitations and opportunities will follow in \sref{sec:textToDesign_discussion}.
For now, we rely on the similarities between various representations to justify a reduced scope for our future experiments.
In particular, moving forward, we study each question with respect to only a subset of the design representations and domains introduced above. 









\subsection{Interpreting and Respecting User Control (Q2)}
% Can \gpt generate a design matching some \textit{specific} user intent, when provided with a more detailed description?
% Generating Specific Designs from Lower-Level Guidance/Feedback
% \item \textbf{Q3} To what extent is the user able to control the designs created by \gpt? Is \gpt able to interpret and respect user-defined constraints, such as spatial relationships between objects, mating constraints for multi-part assemblies, or integration of standard pre-fabricated parts?

The above examples demonstrate \gpt's ability to generate a design based on very high-level semantic input.
However, we also wanted to test its ability to generate designs that adhere to a specific user-given intent.
This section also tests whether \gpt is able to overcome its own potential biases induced by the training data, in order to generate something that truly adheres to a user's specified constraints -- whether or not those constraints match the ``common'' form of a given design target. 
In particular, we choose to study whether \gpt is able to 
(1) understand and respect semantically meaningful spatial constraints, and
(2) incorporate specific pre-fabricated elements into a design.



\subsubsection{Spatial Constraints}
\label{sec:textTODesign_spatial_constraints}
% For example, can we specify the constraints such as "non-overlapping", "symmetric", "above/below", "at a specific relative/global position"? What types of constraints? Which ones does it understand natively, and if there are multiple ways to phrase a constraint, do some approaches seem to work better than others?
Through the general experiments above, \gpt has already shown some capacity to respect high-level spatial constraints, such as a design element's absolute size or its position relative to another element of the design. 
\gpt's compliance with such requests was frequently flawed at the outset, but the results were generally workable after some amount of interactive feedback. 
This section aims to explore the types of constraints \gpt is able to natively understand, and how we might best interact with \gpt in order to improve the chance of successful compliance with such constraints.

%% non-overlapping, position, relative location of 
As an initial experiment, we explored whether \gpt is able to construct a version of the previous cabinet design that includes a door and a handle (see \fref{fig:cabinet_with_handle}). 
We started from a fresh chat, and provided \gpt with a prompt similar to the one described in \sref{sec:textToDesign_JSCAD_basic}, asking for a cabinet to be built from scratch.
However, this time, we also request a door at the front of the cabinet, with a handle on the right hand side of its outward-facing face.
As shown in \fref{fig:cabinet_with_handle_start}, \gpt initially struggled to position several of the cabinet elements -- particularly the side panels and the door.
Although \gpt corrected the position of the side boards immediately, \gpt continued to have trouble placing the door, as it was oriented incorrectly relative to the rest of the design. 
When reminded that the door should be oriented vertically, \gpt was able to comply with the request, but the corrected position was still not fully suitable, as the door coincided with the cabinet's side panel.
After another reminder that the door should reside at the front of the cabinet, with the handle on the right so it could be attached with hinges on the left, \gpt was able to place the door correctly. 
However, the handle remained ill-positioned as it was located on the left-hand side, and was protruding into the door panel. 
After 2 additional prompts, \gpt was able to correct the position to the left hand side. To correct the protrusion issues, \gpt needed 3 more prompts. During these iterations, \gpt moved the handle fully to the \textit{inside} of the door; it needed explicit reminder that the handle should be placed on the \textit{outside} of the door.

With a fresh \gpt session, we also tried providing the previous \jscad specification of the cabinet as part of our input prompt, then asking \gpt to modify the existing design such that it contained a door and a handle, as before. 
Despite the different starting points, \gpt followed a similar trajectory, as shown in \fref{fig:cabinet_with_door_add}: the door was initially aligned incorrectly, as it coincided with one of the side panels; after 1 prompt, \gpt was able to correct the door placement. However, despite \gpt's explicit assertion that \lstinline{the handle is also placed on the right side of the door's exterior face}, the handle remained on the left. Finally, after another prompt, \gpt was able to correct the handle position such that it was on the right rather than the left. 

The way in which \gpt dealt with the under-specified handle request also proved interesting. 
In \fref{fig:cabinet_with_handle_start}, \gpt opted for an additional cuboid that would be unioned into the final design. 
By contrast, in \fref{fig:cabinet_with_door_add}, \gpt opted to create the handle by subtracting a small cuboid from the door panel. 
In still other examples, \gpt refused to add the handle, and instead offered the following disclaimer: \lstinline{Note that the handle for the door is not included in this script, as its size, shape, and position would depend on additional details not provided. This would likely require additional modules, such as cylinder from @jscad/primitives, and might be added as an eighth component in the main function.}

These interactions provide a promising basis for interactive user control of the design, but the process is somewhat tedious at the moment, as \gpt requires very explicit instructions about the design or correction intent. 
The addition of highly-detailed user constraints also seems to confuse \gpt to an extent, as it seems to ``forget'' the larger context of the design in the process, so it must be frequently reminded. 

% Figure environment removed
% \wojciech{Fix colors in the figure}

% Figure environment removed


%% symmetry 





%\subsubsection{Generating Mated, Multi-Part Assemblies}
%% Can \gpt generate multi-part assemblies with valid mating characteristics? 
%
%%% Text to Design: CAD with mating assemblies
% \team{Felix}

CSG and parametric CAD modeling paradigms are mainly used to design 3D parts which can be later on manufactured.
Putting these parts together into one functional object is usually done in an \textit{assembly} step.
In an assembly, specifying how two parts are connected, or \textit{mated}, comes down to defining (i) \emph{where} two parts are mated and (ii) \emph{what} the movement constraints are between two parts.
For example, to mate two lego bricks, we would define (i) the mate connector of the first brick on the top disk of one of its knobs, the mate connector of the second brick on the inner disk of its bottom tube and (ii) define the mate type as a sliding mate, allowing for a vertical translational degree of freedom (DOF). \todo{Lego mating figure}

In commercial CAD software, assemblies are defined via graphical interactions, allowing users to easily reason spatially and to correct their mate definitions thanks to real-time feedback.
GPT and other {\llm}s are purely text based, so they seem like a suboptimal choice for assembly. 
However, prior research has shown that \gpt has a good understanding over contextual and part relationships \cite{}, which could be useful for facilitating the mating process.

As for the parametric CAD modeling paradigms, we try to overcome the domain gap between the textual and the graphical domain by providing \gpt with a simplified DSL to express a minimal set of mating operations.
We define two operators: (i) an operator \lstinline{choose_planar_face(solid_name, plane_side)} which returns a mate connector based on the selected solid and its planar face and (ii) three mating operators \lstinline{mate}, \lstinline{mate_slider} and \lstinline{mate_revolute} which take as input two mate connectors and mate with either no DOF, one translational DOF and one rotational DOF, respectively.
\lstinline{mate_slider} and \lstinline{mate_revolute} additionally take the translation axis and the revolution axis as input argument.

For example, the program in Fig.\ref{fig:mate_stack_cuboids} (a) shows how to stack two cuboids on top of each other with this language.
Given our DSL definition, and this example program, \gpt can successfully stack three cuboids on top of each other, as is shown in Fig.\ref{fig:mate_stack_cuboids}(b) and (c). 
This simple example shows an important strength of \gpt, namely that it can learn from examples and is able to translate this to similar problems.
When assembling an object with a lot of parts and connectors, time-intensive manual labor is required to perform repeating patterns of mating operations \cite{automate}.
Leveraging {\llm}s which can reason about part relationships could prove to be a real time saver for designers.
\todo{}
In Fig.\ref{fig:mate_screws}, we see an example of a screw mating pattern which is performed once and which has to be repeated three more times in the remaining corners of the plate.
When asked to complete the assembly process, \gpt finishes the mating job successfully. 
It should be noted that the variables used in this example are explicitly descriptive.
\gpt performs less well when presented with no semantic context.
\\
\todo{}
\begin{itemize}
    \item explain slider mate example: place a drawer in a box
    \item failure case/limitation: 
    \item explain revolute mate example: door and hinge
    \item failure case/limitation:
\end{itemize}
Our mating DSL for \gpt shows potential benefits for real-world design scenarios, but also some limitations. 
It should be noted that the proposed DSL stays close to current graphical workflow.
Other DSLs could be better suited to interface with \gpt and maybe allow it to textually describe matings more naturally.
\\




\subsubsection{Incorporating pre-fabricated elements}
\label{sec:prefabbed-ele}

It's also common to design an object around specific pre-manufactured elements, such as hinges, brackets, or motors. 
We explore the possibility of using \gpt to source the parts in \sref{sec:part_sourcing} -- at that time, we explore whether \gpt can identify the required part categories, provide options, and/or select a set of options that are compatible with one another and the intended overall design. 

For now, we assume that the user has a specific (set of) part(s) in mind that they would like to incorporate into their design.
Then we investigate whether, given these components, \gpt is able to (1) build a reasonable proxy of this design, then (2) effectively use it as a module within a larger assembly.

\paragraph{Cabinet with Standard Hardware} 
%% Text to Design: CSG with OpenJSCAD, integrating pre-fabbed parts
% \team{Liane}
\newcommand{\bracketImHeight}{1cm}

% Figure environment removed

\newcommand{\bracketPlacingImWidth}{0.162\textwidth}
% Figure environment removed
\afterpage{\FloatBarrier}

To make the cabinet design more stable, a designer may wish to include extra support brackets to work with. 
Many pre-fabricated variations of these brackets exist, and they are inexpensive and readily available.
Given this, it does not make sense to design or manufacture these parts via \gpt.
Rather, we'd like to incorporate instances of a pre-fabricated version.
To do this, \gpt must first build a proxy of the part, place the proxies throughout the design appropriately, and adjust the remaining elements of the design to accommodate these components.

For our first experiment, we chose to incorporate the Prime-Line 1/4 in. Nickel-Plated Shelf Support Pegs from Home Depot into our design.
We provided \gpt with a URL to this part's listing on the Home Depot website, which contained a text description of the item and the schematic diagram pictured in \fref{fig:bracket-buildup}(left).
We then asked \gpt to build a simple geometric proxy that we could incorporate into our design as a placeholder.
As shown in \fref{fig:bracket-buildup}(right, top), \gpt was able to infer and generate the appropriate primitives (one cylinder for the peg and two cuboids for the L bracket).
However, it was not able to correctly scale, orient, or position the elements. 
In an effort to test \gpt's understanding of the structure, we asked it to describe the structure in its own words. 
Although it gave a reasonable description of the bracket, there was little improvement in the result when it was asked to improve the script accordingly.
Thus, even with several iterations of user feedback, \gpt was unable to construct this shape from high-level third-party (URL) or user input.

Ultimately, we had to provide \gpt with an explicit description of the structure that we wanted.
Moreover, we found that even with an explicit description, \gpt was unable to generate the correct shape when provided with all directions at once.
Instead, we had to create the shape in an iterative fashion, beginning with the L bracket and then adding in the peg, as shown in \fref{fig:bracket-buildup}(right, bottom).
Eventually, it was able to generate the structure and consolidate the instructions into a high-level module called \lstinline{createBracketWithPeg}, as desired.

We then provided the module \lstinline{createBracketWithPeg} as an input to \gpt, and asked it to incorporate these structures into the design, as detailed in \fref{fig:cabinet_with_handle}. In particular, we asked for four brackets under each shelf, with the pegs protruding into the cabinet's side walls, the back face of the bracket's vertical leg in contact with (but not protruding into) the side wall, and the top face of the bracket's horizontal leg in contact with (but not protruding into) the bottom face of the shelf. 
We initially tried to complete this experiment in a single continuous chat that (1) designed the cabinet, (2) designed the L-bracket, and then (3) incorporated the brackets into the cabinet. 
However, we found that after the extended discussion regarding the L-bracket design, \gpt seemed to have completely forgotten its cabinet specification.
Despite multiple prompts, it was unable to recover the previous design. 
Instead, we directly provided \gpt with the L-bracket module and its prior cabinet design, and then asked for a modification.
This approach was far more successful.
Overall, we found that \gpt was able to instantiate the correct number of brackets, but it struggled to rotate and position them appropriately. 
After several user prompts, \gpt was able to successfully place the brackets in their locations.
Finally, we asked \gpt to adjust the shelf in order to (1) not protrude into the brackets, and (2) incorporate some additional allowance so the shelf could easily fit between the supporting brackets in a physical assembly.
\gpt was able to complete these requests without issue.

Overall, although \gpt initially struggled to build a proxy of the pre-fabricated part we had in mind, \gpt seemed quite capable of incorporating the completed proxy into a given design, as desired.


\paragraph{Quadcopter}
%% Text to Design: quadcopter with integrated prefabbed parts
% \team{Allan/Bohan, maybe with help from Edward/Pingchuan/Megan/Andy}

% example: quadcopter frame designed around a collection of components.
% include eg:
% - how you chose the components; 
% - selecting components that are compatible; 
% - size constraints (from eg the "car sized copter" you had at some point to the final design).
% - coming up with the geometric primitive proxies for each element
% - design process for the frame using those components

% wip

% Figure environment removed

% block commented
\iffalse
 \adriana{strange way to start - where do you determine the components?}
Once the components are determined, 
we would like to design a quadcopter that uses them with the help of \gpt
and study how \gpt can help with such task.
\adriana{replace previous sentence with: The design of a quadcopter requires the incorporation of pre-fabricated elements such as X, Y, Z. Assuming these have been identified and sourced, the design of the frame should be dictated by their dimensions. Next, we'll examine how GPT can assist with this design task.}
Nonetheless, it is not simple to let \gpt represent those components in details.
To ease the design task for \gpt, we represent them 
using either a box with size $w\times h\times d$
or a cylinder with radius $r$ and height $h$,
which we found that it understands them very well 
as shown in the previous sections. \adriana{which sections?}
The first thing we tell \gpt is 
the dimensions of these existing parts represented 
by boxes or cylinders.
To make it more understandable, 
we also give three functions to describe the box, the cylinder and their placement.
The first function createBox(w, h, d) returns a box with size w * h * d centering at (0, 0, 0).
The second function createCylinder(r, h) returns a cylinder with radius r and height h center at (0, 0, 0).
The last function place(item, x, y, z, a) first rotates the item around z axis by angle a
and then translate the item to position (x, y, z).
Next, we ask \gpt to create a design that incorporates those given parts using only those functions. 
The main part of the quadcopter that needs to be designed 
is the frame which can hold those selected components.
At the first try, \gpt is able to give a roughly correct design description,
but it fails to produce a correct geometry.
It understands that the quadcopter needs four arms 
to connect four motors with four propellers
and that the remaining parts are at the center.
However, it could not make all parts correctly positioned and oriented, as shown in Figure~\ref{fig:quadcopter-stage}(a).
After closely looking into the generated program, 
it is not hard to figure out several design issues. 
First, the frame is not oriented correctly.
This is fixed by telling \gpt 
which dimensions represent the cross section of the frame.
After the adjustment, we can get a near correct quadcopter (Figure~\ref{fig:quadcopter-stage}(b)).
Second, the parts are intersected with other boxes, as highlighted in red rectangles in Figure~\ref{fig:quadcopter-stage}(b).
This problem has been happening constantly 
when we designed other items earlier.
We found that it is difficult to let \gpt understand 
what it means when we say box A and box B are intersected
and how it can be fixed.
Therefore, our solution to fix this problem is 
to manually tell \gpt how much translation is needed 
for each problematic part.
Finally, the frame is not practical because
1) it directly attached to the motor cylinder and
2) it is not sufficient to hold components including
the battery, the controller and the signal receiver.
To solve the problem, we did not ask the \gpt about
how you can solve the problem.
Instead, we provide a concrete solution to the issues
and then ask \gpt step by step.
We first asked \gpt to add a cylinder base
at the bottom of each motor.
The cylinder base directly connects to the frame bar
and each motor sits on each cylinder base without touching the frame bar.
In addition, we ask \gpt to add a box body to
strengthen the frame bars and serve as a plate to hold to remaining parts.
After a few iterations of minor adjustments,
we were able to obtain a valid design which will be further verified
in the simulator or in the real world.
The final result is shown in Figure~\ref{fig:quadcopter-stage}(c).
Throughout the design process, we found that \gpt is good at analyzing the design problem and give a valid design in text.
However, it cannot nicely incorporate the mathematical and physical concepts. 
For example, it does not understand colliding and 
whether the frame is strong to hold any parts.
Therefore, human has to help it in this regard.
 \adriana{I think this section could use some polish to shorten and create understandable paragraphs and opposed to one long text. I also think its best to reference 4.1.2 and say you use the same primitives instead of defining a language all over again.  }

 \fi

 
Designing a quadcopter involves integrating pre-built elements like the motor, propeller, and battery. Detailed sourcing of these parts will be addressed in the later section (Section~\ref{sec:part_sourcing}). Once these components are sourced, the frame must be designed to accommodate their dimensions. We'll explore how \gpt can assist with this task.

However, enabling \gpt to accurately represent these parts isn't straightforward. To simplify the task, parts are represented as either a box of dimensions $w \times h \times d$ or a cylinder with radius $r$ and height $h$. \gpt can handle these representations well as demonstrated in Section~\ref{sec:textToDesign_CSG_boxes}. Rather than having a single function which creates a primitive and translates it as in Section~\ref{sec:textToDesign_CSG_boxes}, we introduce three functions for ease of design: createBox(w, h, d), createCylinder(r, h), and place(item, x, y, z, a). The first two functions generate a box or a cylinder at origin (0,0,0), while the third rotates and moves the item to desired coordinates.

Subsequently, we task \gpt with creating a design that integrates these parts using only the above functions. The primary element \gpt must design is the frame, which should hold the selected components. Initially, \gpt produced a correct textual design, but struggled with the geometric representation, similar to Section~\ref{sec:textToDesign_CSG_boxes}. It understood the quadcopter structure, but had issues with part positioning and orientation (Figure~\ref{fig:quadcopter-stage}(a)). Problems included incorrect frame orientation and part intersections. By guiding \gpt in correcting these issues, we achieved a near-correct quadcopter design (Figure~\ref{fig:quadcopter-stage}(b)).

The initial frame design wasn't practical because it was directly attached to the motor cylinder and insufficient to hold components like the battery, controller, and signal receiver. To address this, we asked \gpt to incrementally implement specific solutions, such as adding a cylinder base under each motor and a box body to reinforce the frame bars and house remaining parts. After minor adjustments, we arrived at a valid design, which will undergo further testing in a simulator or real world conditions (Figure~\ref{fig:quadcopter-stage}(c)).

Throughout the design process, \gpt demonstrated proficiency in textual design analysis but struggled with mathematical and physical concepts such as collision and structural integrity. Thus, human guidance remains crucial in these areas.






\subsection{Incorporating Abstractions such as Modular/Hierarchical Designs (Q3)}
\label{sec:textTODesign_abstractions}
% \item \textbf{Q2} Is \gpt able to incorporate abstractions used by human designers, such as modular (de)composition?

As we have seen from previous examples, \gpt is inclined to use some abstractions like variables by default.
It is also clear that \gpt is well suited to the use of modular or hierarchical design, as in the case of the pre-fabricated L-brackets that it was able to instantiate several copies of, and distribute throughout a design. 
However, there are often instances where a user might want to impose their own specific modules -- for example, a certain hierarchical grouping may facilitate easier debugging or cleaner code.

To test \gpt's abilities in this area, we revisit the cabinet example, and try to modify it such that it contains multiple shelves. Because we have already incorporated pre-fabricated brackets, this modification is non-trivial, as \gpt must instantiate and position the appropriate number of shelves \textit{and} all associated support brackets. 
We began by directly asking \gpt to make this modification on top of the existing code, by generating two evenly spaced shelves within the cabinet instead of one. 
\gpt correctly identifies the elements which must be duplicated, and it instantiates the correct number of them. 
However, it is unable to correctly adjust the position of each module; after the initial request, neither the shelves nor the brackets were in reasonable locations. 
It took 4 additional user prompts to correct the relative positions of these components. 
After this correction, \gpt did seem able to generalize its logic directly to generate cabinets with a varying number of shelves. 
However, the code itself is fairly convoluted.

To avoid these issues, it may be more natural to consider a shelf with its appropriate supporting brackets as a single module.
This way, the entire ``subassembly'' could be instantiated and positioned as a unit on future calls. 
We asked \gpt to implement this plan, by requesting the creation of a module named \lstinline{supportedShelves()}, which instantiates and appropriately positions a shelf and its associated support brackets within the design.
Then, we asked \gpt to refactor the original script such that it used the new module to generate a cabinet with two evenly-spaced shelves.
The initial response had a minor compilation error, a shelf tolerance issue, and a bracket alignment issue, as before, but each of these issues were immediately corrected after a single user prompt. 

Overall, the approaches resulting from both experiments seem equally effective and flexible once they have been fine-tuned. 
Thus, we conclude that \gpt is able to effectively create and use modules, whether they are explicit (\eg, in the form of a function, as in the second experiment) or implicit (\eg, in the form of a for-loop, as in the first experiment).
However, it seems as if the explicit module made it slightly easier for \gpt to reason about a challenging alignment problem.
Moreover, it is useful to know that users can effectively request this kind of hierarchical refactoring, as most human programmers/designers would generally find it easier to reason over a function in this scenario. 
% \todo{(liane) generate figure with the incorrect/corrected placements}









\subsection{Discussion}
\label{sec:textToDesign_discussion}
% \team{Liane}

In this section, we elaborate on the key capabilities (C), limitations (L), and dualisms (D) previously outlined, particularly as they relate to the domain of text-to-design.

\noindent \textbf{C.1 Extensive Knowledge Base in Design and Manufacturing:} Within the text-to-design space, \gpt exhibited proficiency in supporting high-level structure and discrete composition. For instance, \gpt consistently generated the correct primitives (type and quantity) for a given task, regardless of the specific design language it was using. 
\gpt also demonstrated a capacity for interpreting and auto-completing under-specified prompts, as in the case of the CSG table example, where \gpt inferred and provided reasonable values for a set of missing parameters (see \sref{sec:textToDesign_CSG_boxes}).
Finally, \gpt generated readable, explainable, and maintainable code that contained descriptive variable names and comments, along with appropriate modularity and other high-level structural elements.  


\noindent \textbf{C.2 Iteration Support:} Even when \gpt did not immediately arrive at a suitable design solution, it often succeeded in rectifying errors after a reasonably small number of user interactions. For example, it was able to successfully adjust the placement of the cabinet handle after a handful of additional prompts. The ability to engage in iterative design is also very helpful when building up complex structures such as the wheeled robot from \sref{sec:textToDesign_urdf} or the L-bracket proxy discussed in \sref{sec:prefabbed-ele}, because users can start with a simple prompt, then iteratively increase the complexity to arrive at a suitable result.

\noindent \textbf{C.3 Modularity Support:} \gpt effectively incorporates modules and hierarchical structures, using natural language as a powerful tool for conceptualization and orientation.

\noindent \textbf{L.1 Reasoning Challenges:} Spatial reasoning posed a significant challenge for \gpt. Well-crafted domain-specific languages (DSLs) may be able to mitigate this issue. We noted specific difficulties with constructive solid geometry (CSG) due to the computational requirements for object placement. Sketch and extrude languages that utilize reference points can minimize this challenge to an extent, as they offload the computation to reference resolution. This approach is effective for simpler designs but falters when managing complex sequences of transformations. As discussed in the sketch-based car example from \sref{sec:textToDesign_OnShape_basic}, we found that DSLs that balance the benefits of reference-based language with global positioning information may be more effective. 

\gpt's lack of spatial awareness also created difficulties with constraint handling, such as when \gpt was asked to ensure that elements were non-overlapping. We found that iterative refinements and careful prompting often provided a workaround for these issues. For example, \gpt typically failed to respect ``non-overlapping'' constraints, but it generally responded well to the instruction that some element should be ``in contact with (but not protruding into)'' another element.

\noindent \textbf{L.2 Correctness and Verification:} \gpt is not able to reliably verify its own output, and it frequently makes contradictory claims. For example, when asked to place a handle on the right side of the cabinet structure, \gpt frequently placed the handle on the left-hand side of the cabinet, then immediately declared its design a success, because the handle was on the right, as requested. This seems to suggest that external verification tools may be helpful, particularly in cases where the contradictions are less obvious.

\noindent \textbf{L.3 Scalability:} \gpt's success seems to decline as the number of simultaneous requests increases. For example, it is best to issue 1-2 constraints or correct 1-2 issues at a time, rather than trying to issue several constraints or correct several issues at once. 
Similarly, \gpt encountered challenges when interpreting high-level information to build proxies for more complex designs all at once; instead, the models must be built iteratively, with gradually increasing complexity.
This iterative modeling was most effective when the user provides explicit instructions about both the aspects that should change, as well as the aspects that should remain unaltered (either because they are already correct, or because they will be addressed later). 
Despite \gpt's initial difficulty creating complex models, \gpt is able to effectively use and combine existing modules to create more intricate models. 

\noindent \textbf{L.4 Iterative Editing:}
As discussed in \sref{sec:prefabbed-ele}, \gpt seems to exhibit limited memory and attention span. In particular, it often ``forgets'' things from previous messages. We address this by occasionally reminding \gpt of its previous input/output, either by asking it to summarize a previous interaction/finding, or by explicitly including a prior result as a starting point in our prompt.


% \noindent \textbf{D.1 Context Information:} \liane{todo}
\noindent \textbf{D.2 Unprompted Responses:} \gpt is frequently able to recognize and address under-specified problem statements.
For example, in the CSG table specification (\sref{sec:textToDesign_CSG_boxes}), \gpt correctly inferred the need to assign a tabletop thickness value.
Similarly, when augmenting the cabinet with a door and a handle in \sref{sec:textTODesign_spatial_constraints}, \gpt responded with several distinct approaches for handle design.
This can be powerful, as it may alert the user to parameters or variations which may otherwise have gone overlooked; then, users have an explicit opportunity to consider and refine the specification accordingly.
Moreover, it allows users to undertake a design process and begin receiving feedback without first needing to craft a perfect specification or prompt.
However, if \gpt confidently hallucinates a particular solution to an under-specified aspect of a design problem -- rather than explicitly prompting the user to consider a range of options -- it may limit and/or bias their exploration in unexpected ways.





% finds the discrete set of primitives, can't align them reliably (e.g. Felix and PyVista ) this is true accross different DSLs
% With CSG everythin live in global space and things are either pretruding or separated
% sketch and extrude is better but still struggles
% The DSL can help but it's not enough to solve the problem 
% What are the characteristics of the DSL changes that have helped?
% It has fewer things to do so that be helpful but it will still get things wrong if it's still hve to do multi-step reasoning about the different frames of reference is still hard (order specific) you start with a good reference point from which you can do the rest of the design but if it needs  a lot of reasoning it has a hard time. (.e.g composing transformations or chaining or logic is hard). 
% Expose high level constructs that are explicit of the thigns that it can do (e..g rotate by 90 degrees). Felix may have done somethign similar (added a global positioning -
% ***********This is the conclusion*********:  GPT is bad a two things: computation and staking references  so it  struggles with CSG and it struggles with many references. Solution references with some global information. 

% Good at re-using. It is easier for it to build things iterative/ with modulatiry. Do a lot of iterations to make you thing work and can replicate things well

% Hard to handle constraints or controlled specificatons of where you want things to be (e.g. the handle). This is of course even worse if you're trying to get mulitple thigns at once. 


% Good at:
% 1) high-level structure (discrete part of problems)
% 2) good commenting the code and naming variables in ways that are semantically meaningful 
% 3) good at re-using things once it has a design (maybe partially because the stuff it creates has good semantic information - this will be discussed more later)  

% Bad at
% 1) Continous positioning of elements in a global space and reasoning about local coordinate systems and compositions of transformations. 

% Solution: DSLs that combine referencing and global information. 

% 2) Respecting Constraints 

% Solution: feedback loop and alternative phrasing. 

% 3) Taking a lot of information at once. 

% Solution: leverage modularity 




\ignore{

- Text-To-Design Takeaways
	- good at high level structure, bad at placement
	- can use a wide range of DSLs, but best if they're intrinsically consistent, seems to have trouble reasoning about local coordinate systems (needs to be reminded about e.g. centered positioning, sketch based CAD hard,)
	- better to build it up gradually, has trouble with large requests all at once
	- constraints are difficult -- has trouble reasoning about them, but can usually be accomplished after some back and forth / playing with the word choice or prompt style 

	- has difficulty directly interpreting/building proxy from high level information, but can use it once it exists to create more complex designs.
	-
As part of this exploration, we also examine \gpt's ability to provide semantically-meaningful design specifications -- \eg, scripts including variables rather than hard-coded ``magic'' numbers, and reusable functions or modules where appropriate.

\begin{itemize}
    \item \textbf{Q} Is there a universally effective style of prompt that seems to generate the desired design quickly, with few errors (wrt design or boilerplate code)?
    \item \textbf{Q} Is there a limit to the number of primitives, constraints, modules, levels of hierarchy that \gpt seems to encounter? 
    \item \textbf{Q} (How) can we iteratively evaluate validity and the correctness of the design, and use this evaluation either to improve the model through feedback or to automatically post-process the design so that it is valid and correct.?
\end{itemize}


When specifying geometric constraints, effectiveness of that constraint being reflected in the final design varies based on the adjectives used to describe the constraint. For example, asking \gpt to make two primitives "non-overlapping" may still produce a result where the two bodies intersect, while asking \gpt to make it so that a body "does not protrude" into the other is effective in preventing intersections. So far, a pattern as to which adjectives are interpreted correctly by \gpt has not been observed. 

Specifying constraints via numeric means --- i.e. offset the second cube so that it's top face is aligned with z = 0 --- is almost always effective, whereas asking for geometric relations --- i.e. offset the second cube so its top face is aligned with the bottom face of the first cube --- does not guarantee consistent interpretation.

Introducing multiple constraints in a single prompt often results in some of the constraints being ignored, whereas introducing the same constraints in sequential prompts results in the desired behaviour. 

}













% ============================
% old formulation
% ============================
\ignore{
\paragraph{Problem setup}

There are three aspects to a prompt for specifying a design task: The objective, the primitives, and the additional constraints.  We describe in each turn, and then provide an example of a complete prompt and its output.



\includepdf[pages=1]{figures/BohanExample.pdf}

Function box(x1,x2,x3, dx1, dx2, dx3) creates a box centered at point (x1, x2, x3) with dimensions (dx1, dx2, dx3). Create a capital letter T from non-overlapping boxes.

This prompt has three components. Specifying a primitive, specifying the objective/goal, and constraints.


\paragraph{Design Representation}
Design can be expressed code, i.e. formally (e.g., CAD, OpenSCAD, SVG, STL, etc.). Sequential List of commands. 
Design can be represented using graph. Nodes (components). Edges connections (e.g., connectivity) between components. 


\paragraph{Design specifications}
Describe a text example of a goal.

\paragraph{Design primitive}
How do specify design primitives.
Can you specify different primitives? How many? Can you mix different primitives. 

\paragraph{Constraints}
Describe different types of constraints.
Describe which constraints work and which do not work.
Symmetries. Spatial constraints (on the top, on the side, colinear, at the distance from X, parallel, perpendicular, at an an angle, reflective, minimum, maximum size, setting something to a value)

\paragraph{Hierarchy}
Describe how to introduce hierarchy.
E.g,, Specify in detail a leg for a table and a tabletop and then how to make the whole table. A leg for a table is made of a few components. a tabletop is made from a few components.
Check if you can make multiple levels of hierarchy.
Check if constraints still hold across multiple levels of hierarchy.

}


\section{Text-To-Design-Space}
\label{sec:text_to_design_space}


%\begin{fi}
%\begin{lstlisting}

\hfill

% Figure environment removed
%\todo{how to make a vertical line between these minipages?}

%\paragraph{Overview}
%\begin{itemize}
%    \item Explain that one design is useful but in many cases you want to specify the design space.
%    \item This is in many ways implicit constraints.
%    \item Examples include parameterizations, graphs, grammars.
%    \item These design spaces are generators of individual designs from which an LLM can then intelligently sample.
%    \item Typical methods for design space generation are ML generative models
%\end{itemize}

A design is a sequence of construction operations which take input values and which modify the current state of the design.
These input values can directly be represented as numbers. For example in Fig.~\ref{fig:gear_model} (left), the design of a 3D gear is constructed by directly using 3D coordinates and dimensions.
While this representation has the merit of being direct, without any references to previous code, it does not expose the degrees of freedom of a design.
To modify the thickness of the gear, we have to modify several input values at once to obtain the desired 3D model.
The introduction of \emph{design parameters} in Fig.~\ref{fig:gear_model} (right) makes this change easier by modifying a single variable, namely \lstinline{gear_thickness}.
We call this representation a \emph{parametric design}.
Note that design parameters can be continuous or discrete, e.g. \lstinline{gear_thickness} or \lstinline{tooth_count} respectively.

To explore different design variations, either manually or automatically, having a parametric design is not enough.
We still need to know which specific values we can assign to the design parameters.
For this, we introduce lower and upper \emph{bounds} for each design parameter.
Each design parameter can take any value within its specific bounds.
Together, a parametric design and parameter bounds define a \emph{design space} which is the set of all possible design variations.

Design spaces are an import tool to understand what a design can accommodate for.
This is important for both the manual and automatic optimization of designs.
With this in mind, we want to investigate the following questions:
\begin{itemize}
    \item \textbf{Q1} Can GPT-4 create a design space from text?
    \item \textbf{Q2} Can GPT-4 create a design space from an existing design?
    \item \textbf{Q3} Can GPT-4 create a design space from multiple designs?
    \item \textbf{Q4} Can GPT-4 explore a given design space?
\end{itemize}

For each of these questions, we want to find out what is currently possible and what seems to be beyond its capabilities.

\subsection{Generating a Design Space from Text (Q1)}
In Sec.\ref{sec:text_to_design}, we showed that GPT-4 is capable of generating designs.
The next step towards generating a design space is to test if it can also generate \emph{parametric} designs.
To enforce the generation of parametric designs in our prompts, we ask it to \lstinline{explicitly use high-level design parameters} and to use \lstinline{as few variables as possible}.
It should be noted that GPT-4 often introduces variables to improve readability by itself, without explicitly being asked to do so.
However, we found that including this in our prompts \emph{always} resulted in parametric designs.

We also notice that when asking for a simple design and asking for a parametric design of the same object, there are generally fewer mistakes in the reuse of certain dimensions.
For example, in Fig.~\ref{fig:chair_dimensions}, at first, \gpt positions the backrest on top of the seat using the correct numerical values, but not for the correct dimensions.

% Figure environment removed

%% Figure environment removed
Whereas when asked for a parametric design, the use of \lstinline{width} and \lstinline{length} suffixes in the parameter names seem to be more consistently associated with the corresponding 3D axis.


%Wrong positioning of parts can also be frequently observed when asking for \emph{multi-part} models.
%While the semantic specification of part relationships is usually correct, it does not get translated correctly.
%Commonly found issues are wrongly positioned or oriented parts and intersecting 3D geometries.
%We tested this for OpenSCAD code (global coordinate system) and a sketch-extrude DSL (local coordinate systems).


%\begin{wrapfigure}{r}{0.5\textwidth}
% Figure environment removed
To generate a design space, we need parameter bounds.
When asked for lower and upper bounds for parameters, GPT-4 proposes bounds that are \lstinline{based on typical proportions} of the designed object.
This implies that the scale is often arbitrary but that bounds are semantically reasonable relative to each other.
For example, when asked to design a parametric car with exposed parameter bounds, GPT-4 returns lower and upper bounds and arguments for these bounds in terms of inequalities, see Fig.~\ref{fig:car_bounds}.
According to GPT-4, the width of the car body \lstinline{should be less than the length but larger than the height} and the radius for the cylindrical wheels \lstinline{should be less than the height of the car's body so the wheels don't exceed the height of the body}.
These constraints between design parameters can also be queried in the form of actual inequalities, which is useful for downstream optimization when combined with parameter bounds.


However, these bounds are based on semantic knowledge about the object and not on the geometric design sequence.
For example, for a pen holder, the angle of a rotated cylinder will get a lower bound of $45^{\circ}$ to prevent any pen from falling out, but not to prevent the 3D object from creating unwanted intersections with other parts.
Constraints in real-world design sequences often need to also consider purely geometric aspects of a design.


%\paragraph{Tasks}
%\begin{itemize}
%    \item Task 1.1: can you ask it to create a design where high-level editing parameters are automatically exposed. This implicitly creates aspects of a design space (it tells you degrees of freedom but not their bounds)
%    \item Task 1.2: can you ask it to create a design where high-level parameters are exposed and their ranges are also defined?  
%    \item Comparison: can you compare this task to generating a design without these exposed parameters? Does forcing GPT to expose the parameters help makes things more cohesive (e.g. things will be symmetric without it having to run all the computations to ensure it) ?
%\end{itemize}

%\paragraph{Results: Generating a Design Space from Text}
%\begin{itemize}
%
%    \item Task 1.1: Yes, we can ask it to create designs with exposed high-level editing parameters. This works for continuous and discrete parameters. Discrete parameters will be exposed in loops. (Chair example, temple pillar example). 
%    
%    \item Task 1.2: Yes, we can ask it for bounds which are linked to design parameters.
%    Interestingly, it will give reasons for these bounds by comparing a particular dimension with other dimensions.
%
%
%    \item Comparison: GPT-4 is often introducing variables itself. 50\% of the time no use of variables. Without variables, it can mistake what numbers it used for a previous operation, e.g. it confuses dimensions.
%\end{itemize}

\subsection{Generate a Design Space from an existing design (Q2)}
Given the current limitations of creating designs and design spaces from text prompts alone, it is interesting to understand how \gpt can create design spaces from existing designs, made by human designers.
Just as regular code, input designs for \gpt can vary in quality of semantic annotations and comments about what is being constructed.
%Input designs for GPT-4 come in a variety of quality and annotations, where the following presents a non-exhaustive list, going from weakly annotated to strongly annotated designs:
%\begin{enumerate}
%    \item Non-parametric design without any indication about the design object
%    \item Non-parametric design with an object name but without semantic information about the designed parts of the object
%    \item Non-parametric design with semantic object and part names
%    \item Parametric design with semantic object and part names 
%\end{enumerate}
For all of these inputs, we are interested in how easy it is for \gpt to create a design space, i.e., a parametric design with parameter bounds. 
We investigate how helpful semantic context is to parametrize designs.
%\question{do we consider constraints being part of a design space definition?}
For the prompts of the following experiments, we have found that we get more consistently a good parametrization when we include that it should \lstinline{expose high-level design parameters} while using \lstinline{as few variables as possible} and that it should \lstinline{keep the same program structure and the resulting input values to modeling functions}.
These constraints prevent it from slightly modifying operator input values to extract fewer design parameters.

%\begin{enumerate}
First, when given a design with no semantic context, we observe that \gpt exposes design parameters based on equivalence between numerical values and based on which design operators these values were used in.
For example, in Fig.\ref{fig:chair_parametrization}, it introduces a variable \lstinline{cube_size} which replaces the value \lstinline{19} which was used for both the chair's width and length.
For the mug in Fig.\ref{fig:mug_parametrization}, we can observe that the exposed variables also stay close to their original usage for a given geometric operator.

Second, we repeat the previous experiment with additional semantic context.
Providing \gpt with the name of the object that is being modeled proves useful for generating a parametric design.
We can see that now, design parameters get exposed which are semantically more useful for modifying the design.
For example, the cylinder radii in Fig.~\ref{fig:mug_parametrization} gets replaced for a parameter \lstinline{mug_wall_thickness} which controls the thickness of the mug by considering both radii jointly.
Also, some ambiguity caused by numerical equivalence can be resolved and produce more useful parametrizations.
In Fig.~\ref{fig:chair_parametrization}, the \lstinline{cube_size} from the previous parametrization without any semantic context, gets disentangled into a \lstinline{length} and a \lstinline{width} parameter, allowing to have more control over the shape.
This might prove especially useful in this case, since all the slats are associated to the chair's width and not its length.

% Figure environment removed

% Figure environment removed

Once parametrized, we can complete the design space by asking for parameter bounds, see Fig.~\ref{fig:mug_bounds}.
Again, notice how these bounds are \lstinline{somewhat arbitrary} and not based on the 3D design sequence.

While these results are encouraging, GPT-4 is easily confused by the final effect of a series of geometric transformations.
An example for this is the generated parameter \lstinline{handle_thickness} in Fig.~\ref{fig:mug_parametrization} which actually modifies the $y$ position of the handle.
Once again, it is limited by cases where geometric computation prevails over semantic reasoning.
%\afterpage{\FloatBarrier}



%\begin{itemize}
%     \item Task 2.1: you are given a single program with many parameters to describe a design and you want to abstract away a few high-level parameters and their ranges to  define the design space.  
%    \item Task 2.2: Given a set of examples, can it generate a program with high-level parameters where configurations of these parameters generates each of these examples. (like library learning) 
% 
%\end{itemize}
%\paragraph{Results: Generate a Design Space from Existing Designs}
%\begin{itemize}
%    \item Initial promising result for finding the total car length of a three-part car body. TODO: find ranges and more complex example.
%     \item Task 2.1: Yes, it can do that. OpenScad examples.
%\end{itemize}

\subsection{Can GPT-4 create a design space from multiple designs? (Q3)}
Design spaces based on a single design are useful to explore the family of possible shapes generated by varying the design parameters.

However, sometimes a designer might want to make more structural changes, inspired by another design of the same object class, they want to interpolate them.
Interpolating two designs can be difficult to achieve and there are a number of difficult questions which arise:
Are two designs modeled in a similar way?
Do they have the same dimensions and if not, how do you match the dimensions between two sub-designs?
Do you have to add extra operations to combine two parts?
Can you actually extract a subpart of an object from a design?
If you cannot exactly extract a sub-design, can you design something which is \textit{inspired} by two design sequences?
How do you accurately refer to two sub-designs in a text prompt?

To investigate if \gpt can help with design interpolation, we test three different design scenarios.
All of the designs were presented to \gpt in our sketch-based parametric CAD DSL, explained in Sec.~\ref{sec:text_to_design}.



% Figure environment removed

% Figure environment removed

%% Figure environment removed

First, we present it with two chairs which are modeled similarly, but the first chair has cylindrical legs and the second chair has a backrest with splats, see Fig.~\ref{fig:interpolation_chairs}.
In our prompt, we ask if it can \lstinline{mix these two designs to create a chair with cylindrical legs and splats in the back}.
The result can be seen in Fig.~\ref{fig:interpolation_chairs} (c).
It should be noted that variables in the code are descriptive, e.g. \lstinline{leg4_solid} and \lstinline{splat_3_sketch}, which helps provide semantic cues.
Also, in our designs, the first half of the code describes the construction of the seat and the legs and the second part describes the construction of the backrest.
This means that mixing these two designs comes down to replacing the second half of the first design with the second half of the second design.

% Figure environment removed
Next, we present \gpt with two designs of a temple, involving a different number of pillars, one with 4 pillars and one with 10 pillars on each side, see Fig.~\ref{fig:temple_interpolation}.
In our prompt, we ask it to \lstinline{design a temple with steps, a roof and 6 pillars on the left and right side}.
For this, \gpt has to find how these pillars have been modeled and how to model a varying number of pillars, given the two input examples.
The code for the design of the pillars did not contain any looping structures nor variables and it was more spread out throughout the program than in the chair example, to make it more challenging.
Despite these challenges, \gpt manages to extract the construction logic of the pillars and introduces variables and a looping structure to place them correctly, see Fig.~\ref{fig:temple_interpolation} (c).
Note that we have mentioned the steps and the roof in the prompt. We have noticed that without this reminder, it would solely focus on the construction of the pillars and forget about the rest of the design.

% Figure environment removed
Our last test is structurally more challenging.
We present \gpt with a design of a bicycle and a design of a quad-bike, see Fig.~\ref{fig:bicycle_interpolation}. 
The two designs differ not only by the number of wheels in the front and the back, but also by the construction of the bike forks.
In the case of the bicycle, the fork surrounds the wheel and in the case of the quad-bike, the wheels are connected by a horizontal bar to the vertical bar of the frame.
This makes the mixing of sub-designs more complex.
And indeed, when asked to design a tricycle, \gpt reasons correctly about the number of wheels in the front and the back, and where to find these structures.
It also adjusts the size of the quad-bike's vertical bar such that the two back wheels and the front wheel are on the same plane.
This was not the case for the quad-bike and the bicycle in the input designs.
But it does not succeed at extracting the complete fork from the bicycle design, as can be seen in Fig.~\ref{fig:bicycle_interpolation} (c). 
Note that this experiment was performed via a single prompt and \gpt would likely be able to copy the missing part via further interaction with the user.
%\wojciech{Is it possible to fix this with more prompt engineering?} \FH{At the time of the experiments, I worked in a single prompt fashion and tried multiple times with different prompts to make it work. But it never would. But I think that it would be able to fix it in an iterative discussion.}

We find these examples promising, as they show how \gpt manages to combine its general knowledge about part relationships and its coding abilities.
One of the observed limitations is the ability to extract long sub-sequences and to detect which other parts are still important for plausible interpolation.

%\todo{}
%\paragraph{Generate Design Variations}
%\begin{itemize}
%    \item Given a design and a textual description of a modification, can you find that modification? This is hard because you don't really know how the design can be modified and what are the constraints. -> remove the armrests from the chair?  
%    \item If you have a design space -> design variations happen by simply searching over that space to find a match. gpt can typically do this correctly because it doesn't have to worry about constraints. 
%\end{itemize}
%

\subsection{Exploration of a given Design Space (Q4)}
A design space is conceptually useful to reliably generate variations of a given design.
However, coming up with parameters which represent \textit{meaningful} design variations can be a time-consuming iterative process.

To investigate if \gpt can help with this task, we perform the following experiment.
We present it with a parametric design of a Lego brick, see Fig.~\ref{fig:lego_parametric_design}.
Then, we ask it to generate \lstinline{parameter bounds} and \lstinline{parameter constraints}.
Interestingly, \gpt generated the non-trivial constraint that the length and width of the brick should be multiples of 3.
We ask it to use the design space to \lstinline{come up with 10 different parameter settings which correspond to meaningful lego bricks}.
Finally, it should \lstinline{give each variation a name}, see Fig.~\ref{fig:lego_exploration}.

% Figure environment removed
%\afterpage{\FloatBarrier}

% Figure environment removed
We can observe that the proposed parameter settings respect the previously generated bounds and constraints and that they lead to distinct 3D models, for which it generates plausible semantic labels.


\subsection{Discussion}

In this section, we summarize the key capabilities (C), limitations (L), and dualisms (D) specific to the creation and manipulation of design spaces.

\noindent \textbf{C.1 Extensive Knowledge Base in Design and Manufacturing:} 
We observe that we can leverage \gpt's semantic knowledge base to create parameters, bounds and constraints for text-based designs and already existing designs.
Additionally, \gpt can be useful for finding semantically meaningful design variations in a given design space.

\noindent \textbf{C.3 Modularity Support:} 
We observe that \gpt can interpolate existing designs by extracting and adapting sub-designs based on their program representations.
Interestingly, even when designs are not presented in a modular fashion, it tries to recognize and abstract sub-modules in input designs.

\noindent \textbf{L.1 Reasoning Challenges:} 
The design spaces created by \gpt are based both on semantic knowledge and on code interpretation. 
However, it does not take into account geometric considerations, such as intersecting or non-connecting parts.
As a result, generated parameter bounds can create non-valid geometry and it has proven difficult to make \gpt correct these.
However, in general the generation of valid parameter bounds and constraints is a difficult problem for which mainly approximations have been proposed \cite{mathur2021constraint}.

\noindent \textbf{L.3 Scalability:} 
The interpolation task revealed that \gpt has limited capabilities to infer what parts of a design should be linked to a semantic part specified in a prompt. 
One promising future direction to manage increasingly complex designs is to make them increasingly modular by adding intermediate levels of abstraction.

\noindent \textbf{D.1 Context Information:}
We observe that the generation of correct parametric designs and the reparametrization of already existing designs can be improved by providing semantic cues, such as the name of the modeled object.
As seen in Sec.\ref{sec:text_to_design}, \gpt creates designs which contain a lot of semantic information and it generally performs even better when using meaningful variable names.
Leveraging this aspect in the generation of design spaces and throughout other aspects in the design process should prove extremely useful.

% \afterpage{\FloatBarrier}
%\noindent \textbf{C.1 Generation of parametric designs:} 
%We observe that \gpt is able to generate parametric designs and that explicitly introducing parameters increases the chances of generating correct designs.
%Design parameters can be continuous or discrete.
%In the latter case, they will be programmed via loop structures.
%
%\noindent \textbf{C.2 Generation of bounds and constraints:} 
%We observe that \gpt can generate parameter bounds and constraints which are based on general semantic knowledge about the object.
%
%\noindent \textbf{C.3 Parametrization:} 
%We observe that \gpt can be useful for taking a design created in a modeling software and augmenting it with a design space.
%
%\noindent \textbf{C.4 Interpolation:} 
%We observe that \gpt's capability to parse code and to reason semantically can be used to manipulate multiple existing designs and to create new ones.
%
%\noindent \textbf{C.5 Exploration:}  
%We observe that \gpt can be useful for starting the design process and overcoming the blank canvas by generating multiple, semantically meaningful design variations.
%
%\noindent \textbf{L.1 Geometric reasoning:}
%The design spaces created by \gpt are based both on semantic knowledge and on code analysis.
%However, they do not consider geometric considerations, such as intersecting or non-connecting parts.
%
%\noindent \textbf{L.2 Linking semantic knowledge to parts:}
%The interpolation task revealed that \gpt has limited capabilities to infer what parts of a design should be linked to a semantic part specified in a prompt.
%
%\noindent \textbf{O.1 Linking geometry and semantics:}
%One major opportunity for improving the generation of design spaces and design variations is to enforce geometric reasoning.


%% Figure environment removed

%\begin{itemize}
%    \item Given a design space (e.g. the output of the previous task) can it explore it given high-level prompts? 
%   \item Can GPT interpolate 2 designs? How does the interpolation task compare to Task 2.2? 
%
%\end{itemize}
%\textbf{Old description of the section}
%
%\paragraph{Parametric Designs}
%How to make one design and how add parameters.
%How to create design variations based on these parameters.
%
%Allan's example could go here:
%Given a code example for a quadcopter, give me a design for hexacopter.
%
%\paragraph{Interpolation (and extrapolation) between designs}
%Given two designs (specified using code), can you mix these two designs?
%Given that you have two types of chairs, can you get something that is in between?  Examples include a square-shaped back and a circle-shaped back, interpolating smoothly with rounded corners.
%
%Can you interpolate over discrete parameters; e.g. Given a table with 10 legs, and a table with 4 legs, can you generate tables with 5, 6, 7, 8, and 9 legs?  You'd want them to be constrained such that they would be placed concentrically in a circle.
%
%\paragraph{Specifying design spaces using grammar}
% Show how to specify a design space using a grammar. 
% Show how to create regular context free grammars. Show shape grammars.
%
% Show examples using robots/graphs. Show molecules as well.

\section{Design-For-Manufacturing}
\label{sec:design_for_manufacturing}

The utilization of \llms in the context of Design for Manufacturing (DfM) provides a broad range of applications that have the potential to enhance the design and manufacturing process of different parts and assemblies. One useful application of \llms involves leveraging their pattern identification and language interpretation capabilities to imitate a manufacturing expertise bank that can be tapped into during various parts of the design and manufacturing stages. Furthermore, because \llms such as \gpt have the ability to create programs and find and interpret patterns in text, it can potentially be used to generate and alter design and manufacturing files. Currently, DfM is often accomplished by human expertise with the aid of CAD software. Engineers and designers review design plans and use their industry experience to suggest alterations that would improve manufacturability. The CAD software then allows these alterations to be modeled. The replacement of human manufacturing knowledge with \gpt in this context could streamline the design for manufacturing process, offering more consistent, scalable, and efficient decision-making, which is not limited by individual human capacity.

In this section, we propose multiple ways that this new manufacturing expertise bank could be used in design and manufacturing, as shown in \fref{fig:DfM_Overview_Figure}. \gpt can be used to select optimal manufacturing techniques based on a part's features. Furthermore, it can propose and implement modifications to a design to improve its manufacturability, ultimately leading to more efficient production processes. Additionally, this idea can be extended to part sourcing by leveraging the model's reasoning capabilities to identify potential suppliers based on the part's desired function and performance. Finally, it could be used to develop manufacturing instructions for various processes. To understand \gpt's ability to alter designs based on manufacturing/sourcing constraints, we pose the following questions: 


\begin{itemize}
    \item \textbf{Q1} Given a part geometry, production run and other desired outcomes, can \gpt select optimal manufacturing processes? 
    \item \textbf{Q2} Given a manufacturing process, can \gpt directly suggest and make design alterations to a parts file based on constraints driven by the process capabilities? 
    \item \textbf{Q3} Given a desired functionality and geometric specifications, can an LLM find a source for a part that fits those specifications? 
    \item \textbf{Q4} Given a design can an LLM create a set of manufacturing and assembly instructions? 

\end{itemize}
% Figure environment removed

\subsection{Finding Optimal Manufacturing Process (Q1)}
To test these capabilities, we tasked \gpt with advising on identifying an optimal manufacturing process for a part with the geometry shown in \fref{fig:Optimal_Manufacturing_Process}. We tested it with four different cases where in each case the geometry, material, tolerance requirements, and quantity were varied. We described the part's geometry as an \jscad file. Finally, given a set of priorities, we task \gpt to select an optimal manufacturing process. In Case four, we provided a finite list of manufacturing processes to evaluate the effectiveness of the selection process under the constraint of a limited set of options. The goal was to determine how well the process could choose the appropriate manufacturing processes to meet the specified priorities.

\gpt was successful at selecting an optimal manufacturing process for three out the four cases. For cases one, two, and four, \gpt selected the optimal process that was approved by an expert. However, in case three, shown in \fref{fig:PTFE_Example}, \gpt suggested an injection molding process, which is not suitable for processing a Polytetrafluoroethylene (PTFE) material. In all cases, \gpt initially only provided a range of manufacturing options; it required additional prompts to arrive at the optimal manufacturing process selection.

% Figure environment removed

% Figure environment removed

\subsection{Design Alterations for Manufacturability (Q2)}
In this section, we assessed \gpt's capability to enhance designs for better manufacturing optimization. To accomplish this, we included the text of a \jscad file in the prompt, allowing \gpt to analyze and modify it accordingly. Our focus in this case was on the CNC machining of a 10-inch diameter disk, which involved creating bolt holes along the edge and a central blind square pocket. We included in the geometry, two intentional features that would be difficult to machine. As depicted in \fref{fig:Optimal_Manufacturing_Design}, the process began with \gpt identifying any manufacturing complexities within the design features. Since an \llm interprets text, \gpt interprets the text of the \jscad file, rather than the geometry that is rendered once compiled which humans interpret. After \gpt identifies any complexities, we instructed it to adjust the geometry of the \jscad file to address the challenging aspects by directly changing the text of the \jscad file.

Although \gpt accomplished these tasks with a moderate degree of success, there were a few inaccuracies. Firstly, \gpt correctly identified two potential machining issues: the small radius of the internal pocket and the thin wall at the pocket base. However, it also misunderstood a number of geometric features described in the \jscad file. These include perceiving holes on a curved surface and anticipating an undercut from the pocket. These misinterpretations might be attributed to \gpt's reliance on the text of the \jscad file for feature identification, as some features become more visible once the file is compiled into a geometric representation. After pointing out these interpretation errors to \gpt, it was able to correct its analysis but introduced another mistake. \gpt incorrectly stated that the bolt holes presented machining difficulties and inquired about additional information regarding the machining area. Once provided with the necessary details, \gpt independently rectified its mistake about the bolt holes. \gpt was also aware of potential issues with the size of the part and machining area of the CNC machine. Furthermore, it was able to compute whether there was a potential issue. 

In the final stage, \gpt was asked to modify the \jscad file to address the manufacturing concerns. It improved the wall thickness from 0.02" to 0.04", making it machinable. Given the additional specification of utilizing a 1/4" endmill, \gpt also adeptly adjusted the internal pocket's radius to accommodate this tooling requirement better.

% Figure environment removed

\subsection{Part Sourcing (Q3)}
\label{sec:part_sourcing}
The massive dataset backing \llms contains some specialized knowledge about parts needed for manufacturing. Consequently, we posit that \llms can be useful for reasoning about these parts, from identifying the correct part names to describing necessary properties for their functionality.
% We posit that the massive dataset backing \llms will have specialized knowledge about parts needed for manufacturing, from identifying the correct part names to describing necessary properties for their functionality.

\paragraph{Cabinet part sourcing} As part of generating the design and fabrication instructions for our cabinet, we asked \gpt to find appropriate shelf brackets for the shelf within the cabinet, starting from a concrete design specification in \jscad. In each iteration, \gpt provided several suggestions as links to products on Home Depot, with a short sentence differentiating them. Numbers in the part descriptions were inaccurate: one bracket pair held up to 300 lbs, but \gpt claimed it could hold 1000. Another pair was a `` heavy-duty option that can support up to 500 lbs. when properly installed.'', but could actually hold 1300 lbs. Otherwise, the short descriptions were true, and all described parts could plausibly serve as shelf brackets.
%we believe all were valid suggestions as shelf brackets \liane{maybe get rid of the "we believe" and phrase it more objectively like "all suggestions could plausibly serve as shelf brackets"}. 
Figure~\ref{fig:brackets} shows the presumed brackets suggested. 
 Overall, we found success for this relatively simple use case.

% Figure environment removed%\liane{this figure could probably be more space efficient (though not sure if it's worth the effort. It's a nice figure otherwise :) )}

\paragraph{Copter part sourcing} We also asked for help sourcing parts when designing the quadcopter example. First, we asked for a parts list that would encompass everything needed for the design. \gpt compiled a list including batteries, frames, propellers, transmitters and receivers, electronic speed controllers, etc. We found that the list was comprehensive and accurate. 
Next, we tried narrowing down the response from a list of parts to a list of specific parts with more tailored guidance for each use case. 
Asking for a range of numerical specifications (e.g. specific amperages for batteries) produced correct and sensible numerical estimates for parts.
Specifying that the copter should be able to hold a weight of 10kgs for 10 minutes yielded a list of
very large and powerful parts. Specifying an indoor copter led to smaller and more lightweight part suggestions.
Pushing \gpt beyond specification resulted in errors. %\liane{not sure what you mean; what did you ask for?}
Asking for specific names of part listings or parts and manufacturers, as in~\autoref{fig:copter-parts} tended to result in lists with incompatible parts, or in naming parts that do not exist. 
%\liane{This sentence has a lot going on -- I'd break it into 2-3 sentences that are more clear}
%For example, it listed Crazepony propeller guards on the parts list; Crazepony does not sell propeller guards. Within one list, we noted four major inaccuracies within our best results, out of ten identified parts. Two were that a certain part did not exist from the listed manufacturer, one was a redundancy where listed parts were already included in the purchase of another part (the flight controller kit recommended by \gpt included a power distribution board, so there was no need to purchase one separately), and one was an item that \gpt believed to be a transmitter and receiver, but was only a transmitter. A subsequent attempt to point this out rectified the issue and \gpt recommended a receiver correctly. We ordered a modified version of the parts list where the four errors were fixed.
Iterating on the errors with \gpt, as seen in our follow-up question in~\autoref{fig:copter-parts}, produced correct new parts.

Though asking \gpt to produce the names of real-world parts was unsuccessful, we still found impressive results in its comprehensiveness and ability to form fairly specific and accurate part lists. \gpt was also able to dispense meaningful advice on ensuring parts were compatible, even though it was unable to generate parts lists satisfying compatibility itself. We believe that \gpt can be a useful guide for delivering domain-specific knowledge and providing complete parts lists, but that precise numbers and specs should be cross-referenced before being used.

%\liane{I'd remove the H options from the chats; it breaks up the text flow (though, this is maybe something to decide as group for the whole paper)}
% Figure environment removed


\paragraph{Geometry-based part sourcing} McMaster-Carr is a deep compendium of knowledge for hardware parts, with geometric information and even CAD models available for many items. McMaster-Carr already has a ``search by geometry'' feature, so we wanted to know if we could perform higher-level searches that involve both context and geometry. First, we tried describing specific scenarios and asking \gpt for search terms that would procure us the correct part. Asking for a nut to be used in a tight space without room for a wrench and submerged in saltwater produced two appropriate results, ``316 Stainless Steel Wing Nuts'' and ``316 Stainless Steel Knurled-Head Thumb Nuts'', where the correct form and material was identified. Asking for a tamper-proof nut also produced the correct search, ``316 stainless steel tamper-resistant nut''. 
%\liane{not a suggestion, just noting that this is a cool and very useful experiment! Never would have thought of this use case.} \amy{:) thanks!}
Next, we tried a more open-ended geometric compatibility scenario by asking for parts for an at-home carbonation system (\autoref{fig:carbonation-parts}). We also then asked it for a comprehensive Bill of Materials. It seemed as though all parts were compatible, at least geometrically; we suspect this is because the items in the domain are standardized for compatibility, McMaster-Carr's dataset is quite rich, and there is great availability of each part across varying sizes. 


% Figure environment removed


\paragraph{Parts of mechanisms} Part of what makes \gpt a compelling tool for design is its simple user interface. A user might interact with \gpt by describing a desired functionality and asking what parts would be necessary to achieve it. For example, we described a hypothetical bar cart with two features: a lower shelf with rails, and a tabletop where a portion could be folded down for compact storage. We asked \gpt to tell us what the name of the fold-down tabletop mechanism was, and recommend a part that could be used to build it. It correctly identified the function as a drop-leaf mechanism, explained that since the drop-leaf would be 20x15 inches, the mechanism should be at least 15 inches long, and named steel or brass as appropriate materials. It also was able to generate a specific search term for the part. However, \gpt did not recommend a particular type of mechanism in how it moved or functioned. %\liane{maybe split the sentence here? or is there a dependency that I'm missing between the clause before and the clause after?} \amy{No, good catch!}
We asked it to list the different sub-types and their use cases, which it did successfully, naming and differentiating a swing arm bracket, a slide-out support, a hinged bracket support, support bars, and a rule joint. We were able to find examples of four of the types, but the support bars, which were described as ``lengths of wood or metal that are stored separately and inserted into brackets on the table and leaf to hold it in place'', did not seem to exist under that terminology or perhaps at all. We then asked it to recommend a type for our use case, and it recommended the swing arm or hinged bracket supports.

% Not sure if this is a strong example.
We also tried a loom example, where we asked \gpt to provide a fabrication plan for a 4-shaft table loom, and asked about the name of the mechanism that lifts and lowers the heddle frames and the names of specific parts that make up this mechanism. In general, it was accurate, but \gpt sometimes erroneously named components that only pertained to countermarche looms or floor looms instead of table loom-specific parts. We speculate that this could be due to a dearth of literature on loom construction in \gpt's training dataset. 

Our examples show potential in using LLMs to identify and source parts, with major caveats. We note a recurring theme of \gpt's ability to produce programs that generate valid programs or instructions to verify validity, and its inability to apply those rules to its own output. In general, we found that we could ask for general, pointed, and precise guidance with great success, but asking for product names or specific items often resulted in incompatible or nonexistent parts lists. Furthermore, best results were produced in the simpler and more common domains, or when the domain we were querying had very rich information, as was the case with McMaster-Carr. We believe that \gpt is useful for making comprehensive checklists, and can lend domain expertise and suggestions, so long as all information can be checked or cross-referenced. Since \gpt can interface across many levels of jargon, experts may derive the most value from its use currently, given that they are best able to make common sense checks over the output. For non-domain experts, \gpt delivers very convincing, confident information that can be incorrect. LLMs are poised to become a powerful ``design for everyone'' tool, but more verification steps are needed to guide novice users. % something about hedging and being less confident


\subsection{Create Manufacturing Instructions (Q4)}
Computer-Aided Manufacturing (CAM) is a technology that utilizes software to generate manufacturing instructions from digital design files. It plays a vital role in the efficient and accurate translation of design concepts into tangible products. CAM bridges the digital design and the physical manufacturing stages, enabling seamless communication and translation of design specifications into machine-readable instructions. CAM encompasses a range of techniques and tools that leverage computer systems to automate various manufacturing processes, including planning, toolpath generation, and machine control. By utilizing CAM, manufacturers can streamline production, improve precision, and enhance overall efficiency. In this section, we delve into the creation of machine-readable and human-readable manufacturing instructions with help of \gpt and open-source CAM software. Specifically, we explore additive, subtractive, and assembly manufacturing processes, highlighting the capabilities and challenges associated with each approach.
\subsubsection{{Additive}}
\label{sec:additive}
%\adriana{this section is not formated in the same way as the others - please cut some of the introduction, don't use bullet points and reference the figures as part of the text}
Additive design, often employed in the realm of 3D printing, can be time-consuming and labor-intensive, requiring spatial reasoning, precision, and multiple iterations. We posit that \gpt will improve this process, as it comprehends complex specifications in natural language, generates designs efficiently, simulates outcomes, and explores innovative possibilities from diverse sources, optimizing functionality and aesthetics.

We first try to directly use \gpt to generate the G-code from a natural language description. However, due to the complexity and length of G-code, \gpt fails to output complete code that precisely models the specified shape, as shown in Figure~\ref{fig:gcode-fail}.
To overcome this, we have developed a two-stage approach.

% \wojciech{This is still too general. It does not fit with the rest.}
% Additive design, often employed in the realm of 3D printing, carries inherent challenges when undertaken manually. It requires an extensive understanding of spatial reasoning, detailed precision, and meticulous planning. Manual additive design can be time-consuming and labor-intensive, as it demands a hands-on approach for each aspect of the design process. There's also the need for numerous iterations and adjustments to optimize the design, as small errors in manual execution can lead to significant flaws, negatively affecting the functional and aesthetic aspects of the final product.
% Transitioning to a \gpt based approach to additive design can drastically enhance the process. \gpt, with its advanced language and problem-solving capabilities, can comprehend and execute complex design specifications expressed in natural language, providing an intuitive interface for designers. It can process and generate design models more efficiently, thereby reducing the time and labor needed. \gpt can also help simulate the outcomes and foresee potential issues within the design stage itself, avoiding the wastage of resources that could occur in manual iterations. Moreover, the AI model can help explore innovative design possibilities by learning from vast datasets, incorporating functionality and aesthetics from diverse sources, and hence promoting design innovation.
% In our approach to the design-to-manufacturing problem, we first need to understand its formulation. This problem can be split into two primary components: the input and the output.
% \begin{itemize}
% \item \textbf{Input}: The input side of this formulation involves a shape parameterized by natural language. That means, we should be able to describe the desired shape or design using everyday language. This allows for a more intuitive interaction with the design process, as designers or users can express their design intentions in a familiar, easy-to-understand format. Alongside the shape, the input also includes hardware-related configuration parameters, such as the nozzle diameter in a 3D printing scenario. This enables the system to adapt to different physical constraints imposed by the hardware used in the manufacturing process.
% \item \textbf{Output}: Transitioning to the output side of the formulation, this involves the generation of the G-Code. For those unfamiliar, G-Code is the language used by most automated machine tools like 3D printers. It essentially instructs the machine how to create the specified design. So in this problem, the output is the G-Code that corresponds to the specific shape and hardware configurations defined in the input stage. This integration between the design and manufacturing stages, mediated by G-Code, provides a seamless pipeline from conception to creation, enabling a more efficient and coherent process.
% \end{itemize}
% To make the process more manageable and efficient, our approach to the design-to-manufacturing problem is divided into two distinct but interconnected stages. This structure allows for an organized progression from the initial design concept, specified in natural language, to the final manufacturing instructions in the form of G-Code.
% \begin{itemize}
% \item Convert the shape described in natural language into an intermediate 3D shape representation. To achieve this, we've chosen to use a triangle mesh as our representation. Triangle meshes offer significant advantages due to their ability to generalize complex shapes, capturing detail and structure in a compact form. To manage these meshes, we employ the Python library trimesh. This open-source tool provides a robust and flexible platform for storing and manipulating triangle meshes, allowing us to effectively process the shape data extracted from the natural language input.
% \item Translate this intermediate representation into G-Code, which will be tailored to the specific hardware configurations provided. This step requires deep domain knowledge of fabrication processes, which is why we have elected to use slic3r, a professional G-Code generation software, as the backbone of this phase. Through Python, we can directly interface with slic3r, ensuring we generate high-quality G-Code that can accurately and reliably guide the manufacturing process.
% \end{itemize}
% The entire pipeline is seamlessly realized with the help of \gpt. The underlying conversation between these modular components is maintained using this advanced language model, providing a cohesive flow of data and instructions across the pipeline stages. By leveraging the capabilities of \gpt, we can make this pipeline more efficient and effective, ultimately yielding superior manufacturing outcomes. 

% Figure environment removed



\paragraph{Stage I} We transform the concept expressed in natural language into an intermediate 3D shape representation using triangle meshes. This choice provides compact and comprehensive representations, capturing intricate details accurately. Leveraging the Python library trimesh, we effectively manage and process the shape data extracted from the natural language input (\fref{fig:trimesh}).

% Figure environment removed

\paragraph{Stage II} We translate this intermediate representation into G-Code, customized for the specific hardware configurations at hand. This critical step demands deep domain expertise in fabrication processes, which is why we rely on slic3r \cite{slic3rSlic3rOpen}, a professional G-Code generation software. Through Python integration, we interface directly with slic3r, ensuring the production of high-quality G-Code that precisely guides the manufacturing process. In \fref{fig:gcode}, we visualize the output G-Code using \texttt{Repetier}\cite{repetierRepetierSoftware}, a manufacturing tool, to validate the fabrication pipeline.

% Figure environment removed

Throughout the entire pipeline, the cohesive communication between these modular components is facilitated by the powerful capabilities of \gpt. As an advanced language model, \gpt maintains a seamless conversation, ensuring a smooth flow of data and instructions across the various stages of the pipeline. By harnessing the potential of \gpt, we optimize the pipeline, achieving enhanced efficiency and superior additive manufacturing outcomes.

\subsubsection{Subtractive}
\label{sec:designForManufacturing_Subtractive}
Subtractive manufacturing is a widely used technique that involves removing material from a workpiece to create the desired shape or form. This process is commonly employed in various industries, including woodworking and metal fabrication. Leveraging the power of \gpt, we explore how this approach can be enhanced and streamlined to achieve optimal results.

To demonstrate the design-to-subtractive manufacturing process, we focus on the previously designed cabinet (\fref{fig:textToDesign_simpleCabinet}) and employ a laser cutter and wood pieces for fabrication. Specifically, our goal is to translate the \jscad design into precise manufacturing instructions.
To tackle this task, we simply provide \gpt with the \jscad code and request the generation of laser cutting patterns in DXF files. \gpt showcases an understanding of the cabinet's fundamental geometry relationships and topological structure. It recognizes that the 3D cabinet comprises various 2D boards, including top and bottom boards, a shelf board, side boards, and back boards (\fref{fig:cabinet1}). However, \gpt encounters challenges when accurately determining the dimensions of the 2D cutting patterns based on the given 3D geometry input. Some inaccuracies arise, such as confusion between the cabinet's depth and the board thickness, resulting in overly thin side boards. Additionally, distinguishing between height and width in the 3D context presents difficulties, leading to back boards that are too short. Lastly, \gpt struggles with precise hole positioning (\fref{fig:cabinet1}).

% Figure environment removed

To address these errors, human intervention becomes essential in explicitly identifying the issues and proposing potential solutions (\fref{fig:cabinet2}). After a round of communication, \gpt successfully generates the correct DXF files for laser cutting. To ensure their validity, these files were verified by human experts.

% Figure environment removed

\subsubsection{Assembly}
We conducted an experiment to explore the potential of \gpt in generating assembly instructions that are both machine-readable for robots and human-readable as standard operating procedures. The experiment focused on assembling a wooden box using a specific set of tools and materials. In \fref{fig:Machine_Assembly}, we presented the prompt for generating machine-readable instructions, which involved creating a set of functions to specify different tasks for the robot and generating corresponding sequences to execute those tasks. Since the functions were designed to be system-agnostic, the response from \gpt printed the actions performed by the robot.

Subsequently, we prompted \gpt to generate a standard operating procedure to convert the machine-readable instructions into human-readable text. This procedure provides a detailed description of the assembly process, enabling humans to follow along and understand the steps involved. By generating both machine-readable and human-readable instructions, we sought to assess the versatility and applicability of \gpt in facilitating effective communication and collaboration between robots and human operators in assembly tasks. 


% Figure environment removed

\subsection{Discussion}
In this section, we elaborate on the key capabilities (C), limitations (L), and opportunities (O) previously outlined, particularly as they relate to the domain of text-to-design.

\noindent \textbf{C.1 Extensive Knowledge Base in Design and Manufacturing:} We have discovered that \gpt possesses an understanding of various manufacturing processes and their capabilities including CNC machining, injection molding, additive manufacturing, and laser cutting.  Moreover, it is able to apply this knowledge to various problems in design for manufacturing. Although it is not consistently accurate, it can utilize this knowledge to offer suggestions about what is the best manufacturing practice to use, if certain geometric features will be hard to produce. Moreover, because \gpt has the ability to generate code, it can be utilized to modify geometry directly and generate manufacturing files based on supplied files. 

Additionally, we have discovered that \gpt possesses the capability to search for parts that fulfill a desired functionality as described to it. This allows it to be used to source parts based on a description, geometry, functionality, and performance. 

\noindent \textbf{C.2 Iteration Support:} 
\gpt also possesses the ability to perform iterative debugging when creating and modifying files required for manufacturing. This enables the opportunity to iterate when prompts are not ideal for generating the desired outcome or when \gpt generates something incorrect.

% \noindent \textbf{c.4 Search Support:} % propose moving to C.1

\noindent \textbf{L.1 Reasoning Challenges:} Our observations indicate that \gpt exhibits constraints in quantitative reasoning. For instance, when tasked with generating manufacturing instructions, \gpt struggled to accurately perform basic calculations for tool path placements. However, this limitation can be mitigated by employing symbolic computations within a script. A case in point: we achieved accurate DXF file generation by designing a script to produce the file, instead of having \gpt generate the file directly.

\noindent \textbf{L.2 Correctness and Verification:} We have found that \gpt will provide incorrect information about manufacturing processes in some cases.  For example, when selecting a manufacturing process, it proposed injection molding as an optimal manufacturing process for a PTFE part which is incorrect. We have not found a solution to this in this work to resolve \gpt giving incorrect information. 


% \usepackage{listings}
% \usepackage{xcolor}

% \definecolor{codegreen}{rgb}{0,0.6,0}
% \definecolor{codegray}{rgb}{0.5,0.5,0.5}
% \definecolor{codepurple}{rgb}{0.58,0,0.82}
% \definecolor{codered}{rgb}{0.79,0.15,0.15}
% \definecolor{backcolour}{rgb}{0.95,0.95,0.92}

% \lstdefinestyle{mystyle}{
%     backgroundcolor=\color{backcolour},   
%     commentstyle=\color{codegreen},
%     keywordstyle=\color{codepurple},
%     numberstyle=\tiny\color{codegray},
%     stringstyle=\color{codered},
%     basicstyle=\ttfamily\footnotesize,
%     breakatwhitespace=false,         
%     breaklines=true,                 
%     captionpos=b,                    
%     keepspaces=true,                 
%     numbers=left,                    
%     numbersep=5pt,                  
%     showspaces=false,                
%     showstringspaces=false,
%     showtabs=false,                  
%     tabsize=2
% }

\newcommand{\co}{{\color{blue}\{crystal\}}}
\newcommand{\pc}{{\color{blue}\{peter\}}}
% \newcommand{\egu}{{\color{blue}\{edward\}}} % already defined in macros.tex
\section{Design-To-Performance}
\label{sec:design_to_perf}

To assess the suitability of a particular design, it is common to evaluate \textit{performance metrics} based on features of the design, such as geometry and materials used. Common metrics include mechanical performance, dynamic functionality, or adherence to geometric restrictions. It is common to compute performance with respect to an individual criterion or multiple criteria. The purpose of this evaluation can be to form a set of quantitative metrics to describe the design further, as a foundation for numerical optimization or to verify whether a design meets given specifications. This performance assessment can result in a single quantitative result or an array of results. A more complex design evaluation can further classify or compare between designs in order to enable further optimization or to select a final part for production. 

Within the range of performance evaluation, there are objective, semi-subjective, and subjective criteria that all contribute to the final design performance. Objective criteria include quantitative features that are calculable or measurable, including features such as object weight, size, load capacity, impact resistance, speed, battery life, vibration resistance, and price. Semi-subjective criteria include features that are generally agreed upon but require some insight or estimations to evaluate. Such criteria may be evaluated by proxy measurements, and may vary based on the evaluator, the culture, or the use case; examples include ergonomics, product lifespan, sustainability, portability, safety, and accessibility. Subjective criteria include features that may differ markedly based on the evaluator, such as comfort, aesthetics, customer satisfaction, novelty, and value. With this in mind, we aim to answer the following pair of questions: 

\begin{itemize}
    \item \textbf{Q1} Can \gpt evaluate the performance of an input design that is consistent with classical, objective metrics?
    \item \textbf{Q2} Can \gpt support performance evaluation in ways not possible with classical approaches, such as using semi\nobreakdash-subjective and subjective metrics?
\end{itemize}

This section describes the current abilities of \gpt and identifies best practices, limitations, and full failures in its capabilities to address each of these questions through the use of several examples per question.

Evaluations were tested using different input styles (\eg, method of design description) and requested output forms (\eg, direct classification or function creation). Demonstrative examples are shown in Figure~\ref{fig:performance_prompt_design}. We did not test all combinations of design style and output requests but focused on key comparisons and types. In particular, to address Q1 we focused on comparing text-based designs (DS1) and generic designs (DS2), comparing output requests for direct evaluation in a text response (RF1) and evaluation by the creation of a function (RF2), and comparing code-based designs described with salient semantics (DS3) and no semantics (DS4). To address Q2 with more subjective features we also tested output requests for categorization (RF3) along with ranking and pairwise comparisons between designs, and separately used scoring (RF4) with varying levels of complexity. 

% Figure environment removed 
% \wojciech{This is not clear. Expand what each of these mean.}


\subsection{Objective Evaluation (Q1)} 
\label{designToPerformance_Objective}

Once a design or design space has been created, a typical design process proceeds by evaluating basic geometric features such as size, weight, and strength of the object. In effect, this answers the question: does the item do what it was created to do? Most typically, certain features need to satisfy functional requirements in order to be suitable designs. 

\subsubsection{Mechanical properties}

Here, we focus on analyzing the mechanical integrity of (1) a chair and (2) a cabinet. 
We began with a simple input design in text form (DS1) and a request for direct evaluation in calculated form (RF1) with an additional binary output asking whether a chair of a given design could support a given load. The specific prompt is included in Figure~\ref{fig:chair_abstraction}. \gpt immediately demonstrated the capacity to handle ambiguity well, assuming a type of wood (oak) and producing numerical material properties for that material when both were unspecified. It made and stated further assumptions about load and failure types. 
It evaluated the failure point by comparing the yield stress to compressive stress, computed as one quarter of the applied load over the cross-section of a chair leg. This is included in the chat snippet shown in Figure~\ref{fig:chair_abstraction}. However, in text form it outputted 94,692.2 Pa, while direct evaluation of the equation it listed in the output gives 94,937.7 Pa; thus, \gpt occasionally failed to perform basic correct in-line arithmetic or algebra. Although the number is only off by a small amount in this case, it can sporadically differ by much greater magnitudes. Along with the evaluation, it included discussion of other, more sophisticated considerations for failure, such as the type of connection between the legs and the seat. Also, upon repeating the same prompt, \gpt would vary whether it included self-weight in the load analysis and whether it evaluated uniform weight or only one leg, leading to small variations in results. 

When asking for a function to evaluate chair failure (RF2), \gpt successfully generated Python code to evaluate whether a chair will break due to excessive compressive stress on the legs, using the same formula as described in the text exchange (RF1). \gpt was able to readily add multiple types of failure without error, also incorporating bending failure of the seat, and excessive stress on the back using simple beam bending and structural mechanics equations. This multi-part failure assessment is included in Figure~\ref{fig:chair_abstraction}. It further automatically generated a function that could intake a parametric chair design with sensible feature parameters like \texttt{leg\_cross\_sectional\_area}, \texttt{seat\_thickness} and \texttt{seat\_material\_bending\_strength}, allowing versatile use of this evaluation. 

When generating the function, it continued to handle ambiguity by make assumptions including that the load would be distributed across all four legs, centered and uniform on the seat, and that the load on the back of the chair would be one third of the total weight. In the case of writing the function (RF2) as compared to text evaluation (RF1), it did not explicitly list all of the assumptions; rather, they had to be interpreted based on the equations used. \gpt also incorporated several small errors and oversights in both cases. For instance, when generating a function to evaluate seat bending failure, it treated the seat as a simply supported cantilever beam, and assumed that the chair would break along the width (separating front from back) rather than along the length or at an angle to the base. It also assumed that the bending stress on the back was evaluated as the load over the total area of the back rather than at the connection surface of the back to the seat of the chair. However, as these functions were identified, they could be further refined by iterated discourse with \gpt to produce a more correct function. 

% Figure environment removed

In a comparison of these two output form requests, RF1 and RF2, functional evaluation was easier to read, more accurate, and able to be implemented for a variety of input designs, but directly incorporated more assumptions into equations. During both types of evaluation, \gpt actively reported on potential causes of error in the evaluation, such as how the chair legs were attached to the seat. It consistently overlooked potential causes of failure such as buckling of the legs unless specifically prompted. We found \gpt to adequately assess most basic mechanical properties of interest. 

Some properties relying on an understanding of the spatial arrangement of chair components were not able to be adequately assessed. \gpt had significant trouble generating a suitable evaluation of stability, and failed entirely to calculate a reasonable center of gravity for an input design despite many attempts. The closest attempt using the simple assumption that the center of gravity would be in the center of the chair seat. 

However, other complex  physical properties were readily assessed. \gpt generated first-order code to assess the failure of a chair upon impact with a spherical projectile, with no difference in quality of the computation compared to static mechanical properties.

To evaluate \gpt's performance on code-based input (DS3 and DS4), we provided \gpt with an \jscad chair specification. When the parameters and parts of the chair were clearly-named salient features (DS3) like \texttt{backThickness}, \texttt{leg1}, \texttt{chairSeat}, and \texttt{chairBack}, \gpt was readily able to recognize the item as a chair and analyze desired properties, such as the breaking load of the seat. However, when we used identically-structured code with variable and object names that had been obscured (DS4), it could not recognize parts of the item to assess properties, for example to locate the seat or synonyms of the seat.
This was true whether the names had been slightly obscured (\eg, as \texttt{XZ\_height}, \texttt{stick1}, \texttt{platform}, and \texttt{barrier}, respectively) or entirely obscured (\eg, as \texttt{Q}, \texttt{A1}, \texttt{B1} and so on). 
When asked about the design in the two obscured forms, \gpt guessed that the final item was a table with a narrow bookshelf and exhibited poor interpretation of the design and parts. Even when \gpt was challenged, it claimed that it could not be a chair because the back was not connected appropriately to the chair seat; this was an incorrect interpretation of the code, again indicating poor spatial reasoning. In a second case, when an input design for a cabinet (DS3) had one variable named \texttt{shelfAllowance} (used to slightly reduce the shelf width for easy assembly), \gpt erroneously assumed that this indicated number of shelves. These results reinforce the idea that \llms perform based on semantics, and that a design without clear descriptive words becomes much less manageable, causing DS4 to generally fail. 

The evaluation process was repeated with DS3 and RF2 for the \jscad design of a cabinet as a box with shelves, a door, and a handle. From the inputted design, \gpt was prompted to create functions to evaluate a set of criteria: storage capacity, load capacity, material cost, and, for a more ambiguous feature, accessibility for a person in a wheelchair. Storage capacity was computed as total volume enclosed by the cabinet, excluding shelves, as expected. In assessing load capacity, \gpt used the ``sagulator" formula, a standard estimation found online for carpentry. However, \gpt's implementation gives strange results and \gpt was unable to provide a more correct form of the equation. For price, \gpt computed the volume of the cabinet walls and a cost per volume. Finally, to address accessibility, it estimated height and depth ranges that would be beneficial, assigning a higher accessibility score to shorter and deeper cabinets. However, it did not provide a source for the height and depth ranges that it scored more highly. 

This points to a potential limitation in the use of \gpt and \llms for this kind of analysis: the source material for equations and standards of analysis may be unknown or even intentionally anonymized. Even when the equations are the standard first-order textbook equations per topic, they are almost always unreferenced. When different standards exist, across different countries or for different use cases, much more refinement would be needed to use \gpt to assess the mechanical integrity of a design. In addition, these equations often work well for objects of a typical design, but for edge cases or unusual designs they would miss key failure modes, such as the buckling of a table with very slender legs or the excessive bending of a chair made from rubber. In a particularly apparent example of this type of failure (\ie, creating functions based on pattern-matching rather than judicious observation of likely failures), \gpt was asked over a series of iterations to help write code to render a spoon with sizes within a set of ranges in \jscad, then to assess ergonomics, which it evaluated based on dimensions. Finally, we requested \gpt to create a function to compute the spoon's breaking strength. Since it had been inadvertently primed by the long preceding discussion of spoon geometry, it proposed a strength evaluation using the basic heuristic of whether the spoon is within a standard size range (Figure~\ref{fig:spoon_assessment}). \gpt had to be prompted specifically for a yield analysis before offering a mechanics-based equation. At that point, it continued to handle ambiguity well and chose a most likely breaking point (the point between the handle and spoon scoop). But for a novice design engineer who might have assumed \gpt's initial output was sound, this bold proposition of an unreasonable strength analysis on first pass without further explanation causes some alarm. This serves as a reminder to not rely on \gpt alone without external validation of every step. 

% Figure environment removed

When assessing designs in text form (DS1, RF1) at an abstract level, \gpt was found to readily identify problems and present a sophisticated discussion of problem areas and considerations for the particular design in question and the metrics being considered. As such, we propose the workflow for rigorous performance evaluation using \gpt to begin with a text-based discussion of the design (DS1 or DS2 with RF1) to understand the relevant features, with no other preceding text in that chat, followed by the development of equations with enough sophistication for the use case, presented in the form of functions for rapid assessment of an input design (RF2). This workflow is depicted in Figure~\ref{fig:performance_workflow}, along with additional steps to ideally validate the final result. 

If an input design of a specific type was used, whether \jscad or another DSL, the form of the input was also provided using well-named variables with each iteration of the chat requesting new code to ensure the variable names did not mutate over time as would otherwise happen. 

There was a failure of \gpt to suggest refinement to the performance codes without specific prompting. For example, there are simple differences in von Mises, Tresca, and Mohr-Coulomb yield criteria for evaluating material failure under applied stress; however, \gpt would simply default to the most common, von Mises, without comment. It would regularly object that the analysis function was an oversimplification; additionally, it would assert that for proper evaluation, more features should be evaluated, more sophisticated tools such as FEA should be used, and structural analysis should be validated by a licensed professional engineer, especially for designs in which factor of safety is a concern. These are all valid points: despite \gpt's very large internal knowledge, it pattern-matches and does not reason at a level to generate the most correct or sophisticated analysis, and will tend to generate more simple rather than more complex equation-based analysis unless specifically walked through refining the code. However, it is capable of more sophisticated text-based discussion, which is why we have found that beginning with text and proceeding to functions provides a more effective workflow, as in Figure~\ref{fig:performance_workflow}.

% Figure environment removed

\subsubsection{Quadcopter}
\label{sec:perf_quadcopter}
We next explored the assessment of dynamic electronic device, a quadcopter, as an example of using the workflow of Figure~\ref{fig:performance_workflow}. \gpt was provided with specifications for the quadcopter that included battery voltage, battery capacity, total weight, and the dimensions of the copter (DS1). We prompted it to generate functions that evaluated the maximum amount of time the copter could hover in the air, the maximum distance it could travel, and the maximum vertical or horizontal velocity and acceleration with which it could travel (RF2). From the provided physical parameters, \gpt was able to generate equations to calculate the copter's inertial tensor, voltage-torque relation, and other kinematics and dynamics. We also independently asked \gpt to generate the physical parameters that would be needed to calculate such metrics, and it came up with the following: maximum thrust, total copter weight, battery capacity, aerodynamic characteristics (\eg{} drag coefficient, rotor size, blade design), responsiveness and efficiency of the control system of the copter, additional payload, environmental conditions, and operational constraints. Although these parameters are all highly relevant, \gpt's output lacked many crucial considerations without explicit prompting in text form. 

In this evaluation, \gpt did not initially include the constraint that the voltage of the controller needed to stay constant, even though this would be obvious to someone familiar with the domain of knowledge. This means that seemingly ``obvious'' considerations need to be explicitly included in the prompt in order for a feasible output to be generated. When asked to include this constraint, \gpt was able to understand the underlying reasons for the constraint, stating that a constant voltage is mandatory for the stability and accuracy of the flight controller. Through this exploration, we also determined that \gpt is able to successfully suggest a product and evaluate the copter based on specific batteries from a particular seller, such as HobbyKing LiPo batteries (\eg{} 3S 2200mAh 11.1V). 

\gpt seems to lack basic spatial intuition of what a copter should look like if the prompt only included the dimensions of the entire copter rather than the dimensions of individual parts. It would hence incorrectly assume that the shape of the copter was a uniform convex solid such as a cylinder or rectangular prism, simplifying and limiting the possible analysis significantly. Thus, we would need to incorporate \gpt's geometric design of the copter's frame, where the dimensions of all components are known, to properly assess aerodynamic performance. And, as with our prior trials assessing chair and cabinet designs, \gpt repeatedly failed to calculate center of gravity or stability metrics, even when given sufficient detail about the design and much iterated discussion. 

% Iteration 1: % without dimensions of copter
% \begin{lstlisting}[language=Python]
%     battery\_life = (battery\_capacity * battery\_voltage) / power\_consumption
%     battery\_life\_minutes = battery\_life * 60
%     longest\_flight\_time = round(battery\_life\_minutes, 2)
%     maximum\_speed = (power\_consumption * weight) / (battery\_voltage)
%     longest\_distance = speed * (longest\_flight\_time * 60)
% \end{lstlisting}

% Iteration 2: {\color{blue} TODO fill in when finishing copter demo}

% \begin{lstlisting}[language=Python]
% \end{lstlisting}

For the most part, \gpt was able to perform the correct arithmetic operations using its own performance functions. But because the generated functions lack complete real-world considerations,it is best to compare \gpt's calculated performance results with what is observed in simulation. We find that these performance functions are a reasonable approximator of copter performance in simulation. The \llm recognizes that the reliability of these results are directly dependent on the accuracy of the inputs, and additional inputs or conditions such as motor efficiency and aerodynamics need to be included in the prompt to match the real copter. 

\subsubsection{Finite element analysis}

To investigate the computational performance analysis capabilities of \gpt, and to build on the first-order mechanical calculations already done, we challenged it to develop a comprehensive framework for advanced performance analysis and structural evaluation using the finite element method (FEM). The primary focus was determining the likelihood of a chair breaking when subjected to external forces. Figure \ref{fig:chair_stress} lists the response and final code generated by \gpt. With the application of FEM through the external library FEniCS, \gpt evaluates the von Mises stress, a crucial parameter in material failure prediction. By comparing this stress with the yield strength of the material, one could assess if the chair would fail under the applied load. For the development of the code, substantial back-and-forth iteration was required to create successful code due to its overall complexity. One helpful point for gradually increasing complexity was to create code for a 2D example before asking \gpt to create a 3D version. In spite of these challenges, \gpt was highly efficient and successful in formulating a precise solution using the FEniCS library, an advanced tool for numerical solutions of PDEs. Not only did \gpt integrate the library into the Python code correctly, but it also applied a wide variety of FEniCS features, including defining material properties and boundary conditions and solving the problem using FEM. Caution must be taken, as \gpt occasionally suggests libraries and functions that do not exist. However, with correction it quickly recovers and suggests valid options. 

The stress distribution visualization in Figure \ref{fig:chair_stress} is performed on the chair previously designed by \gpt in Figure~\ref{fig:cad_chair_design} and is the output of \gpt 's code rendered in Paraview (which \gpt also gives assistance to use), as well as on a chair mesh found from other sources. The result reveals a susceptibility to high stress at the back attachment section of the chair design proposed by \gpt, as seen in Figure \ref{fig:cad_chair_design}. This observation underscores the potential for future enhancements in this object's design.

Beyond code generation, \gpt also lends support in the local installation of these external libraries, such as FEniCS, so users can run the generated code. This assistance proves invaluable for practitioners who may have limited familiarity with these libraries, which are initially suggested by \gpt itself. Notably, studies have delved into the potential of \gpt to generate code integrating other external libraries, like OpenFOAM, for the purpose of performing computational performance analysis \cite{kashefi2023chatgpt}. 

It's worth noting that \gpt's capabilities in utilizing these libraries have certain limitations. It can only harness some of the basic features of FEniCS and struggles with more specific, custom usages of the library, such as applying complex loading conditions. Furthermore, \gpt assumes homogeneous material properties for the chair, an oversimplification that doesn't align with the more diverse characteristics found in real-world materials. Moreover, the training date cutoff for \gpt means that sometimes only older functions or libraries may be used, without current updates. 


% Figure environment removed


\subsection{Subjective Evaluation (Q2)} 
\label{designToPerformance_Subjective}

Subjective properties have a higher dependence on lexical input, making their evaluation using \llms an intriguing proposition. We began with an assessment to compare the use of semantics for assessing subjective properties via 3 output forms: categorization or labeling (RF3), pairwise comparison, and overall ranking. We generated a simple parametric 4-legged chair with a back, then input eight versions with different leg lengths, seat widths, and back heights into \gpt (DS1). \gpt was then asked three similar queries: (1) assign to each chair a label of "large," "medium," or "small" (RF3); (2) rank all chairs from largest to smallest; and (3) in a pairwise comparison, indicate if a given chair was larger or smaller than another. Each of these inputs were given independently, to not influence the different tests based on prior answers in the chat dialogue. In each case, \gpt assigned the same overall ranking. Figure~\ref{fig:chair_size} shows the chairs rendered in ranked order including the labels for categorization, using a combined implicit consideration of seat area, back height, and leg height. In a similar query, spoons of different handle length and thickness, and scoop length, width, and curvature were compared, finding similar results. In that case, \gpt elected to compare spoons by the length of the scoop alone, handling the ambiguity of the question by making a decision about what single quantity mattered most. When handling higher levels of ambiguity, \eg assigning comfort levels to shoes described in text input, \gpt sometimes refused to answer. To bypass this, we determined that it was essential to ask \gpt directly to give an output of a certain kind, such as classification into set categories. For instance, the question "Is this shoe comfortable?" would raise objections, a non-answer, and a discussion of what contributes to general shoe comfort. We could circumvent this by asking "Please state whether this shoe is likely very comfortable, comfortable, uncomfortable, or very uncomfortable, and provide one sentence of justification." Despite it's continued objections, \gpt's responses were usually reasonably justified, noting aspects like use of breathable material, adjustability of laces, shock absorption, and traction of the sole. These results indicate that the semantics of the type of assessment (ranking, categorization, or scoring) do not have a large influence on the final result of subjective analysis, as long as some type is chosen. However, certain prompt structures may be required to avoid refusals to answer, and the simplest prompt structure to ensure this was asking for any certain kind of output response. 

% % Figure environment removed


% Figure environment removed


\subsubsection{Semi-subjective evaluation of sustainability}
To challenge \gpt to evaluate subjective criteria dependent on more abstract input parameters, we asked it to create a list of key criteria that go into evaluating sustainability, and to evaluate chair designs based on these criteria, scoring each category from one to ten (RF4). Given \gpt's limited understanding of numerically-specified meshes or spatial arrangements, we used text-based information (DS1) for commercial chairs from Ikea and Home Depot. \gpt was unable to access this information on its own when prompted with product names, so for this test case, the text from product pages was pasted into the \gpt chat dialogue. This information included each chair's name, a text description of its design, material, and appearance, and some numerical information such as dimensions, weight, and cost (Figure~\ref{fig:chair_style}). Upon requesting the evaluated score for sustainability metrics, it outputted seemingly reasonable numbers with justification based on the text description. 


% Figure environment removed

The justification for each property score were generally reasonable but rarely entirely correct. For example, the remark in Figure~\ref{fig:chair_style} for \texttt{modular\_design} about swapping seat shells was a misinterpretation of the product description: chairs with different seat shell colors were available for purchase, but a single chair could not swap shells. In addition, for this example and most other tests, \gpt refrained from assigning high scores (9-10/10) or low scores (1-3/10) within each category, which likely contributed to errors. 
A further function generated by \gpt readily combined the individual property scores into an overall sustainability score for a given input design. 

\subsubsection{Fully subjective aesthetic evaluation}
% \paragraph{cabinet} \co{}

% \paragraph{chair} \co{}
To evaluate the aesthetic design of an item, the physical appearance must be known, so again the listings from product pages were used as the input data. When prompted to create a function to evaluate aesthetics in general, \gpt refused, noting that it is "highly subjective and can vary greatly depending on individual tastes and preferences" and wrote a function in python with a rather simple subfunction for aesthetics: \lstinline{# Here we'll use a provided aesthetic score}. 

% \begin{lstlisting}[language=Python, style=mystyle, caption=Aesthetic evaluation...?]
%     # Here we'll use a provided aesthetic score
% \end{lstlisting}

In a more carefully curated prompting setup, a range of historical periods were identified that influence chair design, including Egyptian, Greek and Roman, Renaissance, Bauhaus (a semi-minimalist German-inspired design including rounded features), and Minimalist. \gpt identified criteria to differentiate between these historical styles based on seven properties: material choice, decorative complexity, evidence of handcrafting, extent of ornamentation, deviation from standard proportions, upholstery use and quality, and material innovation. Based on these categories, \gpt evaluated each historical period and chair, and created a function to use the scores to categorize the style of each chair. A selection of text from one input/output is included in Figure~\ref{fig:chair_style}. In every output \gpt would also give a reminder that scores were approximate or arbitrary and should be adjusted. And as before, scoring on a 1-10 scale was generally limited to intermediate values in the range, for instance for \texttt{Degree of Decorative Complexity}, a score of 3/10 is given even though the justification lists that no decorative elements were indicated. Even so, the results of the categorization (Figure~\ref{fig:chair_style}) seem generally reasonable with most chairs placed into categories that appear subjectively appropriate; a plain metal stool was classified as minimalist, a soft lounge chair with a floral pattern was classified as Renaissance, and a double end chaise lounge was classified as Greek and Roman. A couple of types of mistakes occurred in the classification. First, most chairs were sorted into the Minimalist category, including the faux leather swivel lounge chair and two soft-sided recliners (not shown). Second, several other design styles that may have been a better fit were included in the scoring but were not found to be best fits in the evaluation, indicating that this set of \gpt's scoring for the historical periods was not appropriately distributed to capture the right features for all chairs. Third, upon re-evaluating scores over a few iterations, we found that different categories could be established and chairs could switch categories at times due to subjective scoring. Nevertheless, these general issues persisted, such as occasional mistaken categorizations and having one "catch-all" category that was used more than others. 
 

% % Figure environment removed

In a similar testing setup, \gpt was used to identify criteria to help a user decide the most appropriate room in a house in which to place a chair of a given design. In this second case, it created categories for criteria used to select the room of a house for a chair including size, comfort,  weight, pet-friendliness, and weather resistance. It further created a list of weightings for the importance of each of these criteria based on the room in question, and ideal ranges for the quantitative features size and weight. It was finally used to create a function to distribute a set of chairs to the set of most appropriate rooms in a house. However, upon evaluation, the results were mediocre: for instance, a lounge chair was sent to the kitchen. It otherwise sorted a soft chair to the living room, a weather-resistant chair to the porch, and a sturdy chair with a soft lining to the study room. More careful selection of evaluation criteria could certainly improve on these results, as well as the inclusion of more details about the chairs and their desired purposes in the rooms in question. 

\subsection{Discussion}
In the evaluation of performance, \gpt was generally successful, though it exhibited an array of intriguing behavior patterns. In this section, we elaborate on \gpt's key capabilities (C), limitations (L), dualisms (D), and opportunities in the context of design to performance, as illustrated by our example cases in the present section. 

% \noindent \textbf{Capabilities:} \linebreak
\noindent \textbf{C.1 Extensive Knowledge Base in Performance:}
Through discussing in text form, \gpt could suggest design considerations and metrics at a fairly sophisticated level. Even when asked to evaluate ambiguous requests, when details are left out, or when the performance metric is complex, \gpt is still able to output reasonable first-order approximation functions. The generated output evaluation functions usually worked, having no coding errors in python; errors in javascript or \jscad were more frequent, but they were usually directly resolvable. \gpt was also able to sort items into categories, and to generate rankings among a set of designs without giving explicit intermediate evaluations. 

\noindent \textbf{C.2 Iteration Support:} 
\gpt was able to eventually assess any property we tested, although the quality of assessment varied. When mistakes were made, further questioning could support the refinement of code to a point where it improved. Particularly for the complex example of the FEA, this took many steps to refine but \gpt responded well enough to stay on track, respond to troubleshooting feedback as well as conceptual feedback, and finally create usable code. 

\noindent \textbf{C.3 Modularity Support:} 
Functions could be effectively built up point by point, with modifications made according to changing needs. \gpt could adjust part of a scoring system, such as switching one item for another, or to create the same type of scoring system for another use case using the framework of the first system to create the second one. 

% \noindent \textbf{Limitations:} \linebreak
\noindent \textbf{L.1 Reasoning Challenges:}
\gpt relied on semantic clues, such as variable names, to understand and assess designs. It overall failed to appropriately evaluate performance that required spatial reasoning, like center of gravity or stability, for items having multiple components. In addition, earlier parts of conversation could cause issues for \gpt to poorly choose evaluation metrics, such as a discussion of spoon dimensions leading it to evaluate whether a spoon is ``strong" based on whether its size is within a normal range. When considering subjective metrics that are not typically quantified, \gpt would object. Upon requesting more sophisticated or more abstract evaluation, it would refuse to answer on the first attempt. 

\noindent \textit{Potential Solutions:}
To understand designs, they must be described with enough text-based semantic clues for \gpt to handle. Spatial reasoning issues could be resolved using external methods, such as external FEA analysis or other existing APIs to perform these evaluations. To choose the quality of evaluation equations, more discussion with \gpt could reveal the use-case for the chosen equations and alternatives, allowing a user to decide if another option may be more suitable. To assess subjective metrics, it worked best to develop scoring systems by breaking down a subjective feature into smaller, more quantifiable parts that \gpt could approach. And to bypass refusals to give a concrete answer, prompt engineering on its own could solve this, by requesting a specific enough type of output. 

\noindent \textbf{L.2 Correctness and Verification:}
The source material for equations used by \gpt in evaluation was usually undefined, which can contribute to error, and often embeds assumptions. When calling external libraries, \gpt occasionally invented fake libraries that could not function. Or, when working with \jscad designs it occasionally created designs using nonfunctional methods or nonworking code and simply complained that the language had been updated past its training cutoff. 

\noindent \textit{Potential Solutions:}
An external checker would be needed to verify the source of equations against an objective standard to ensure reliability, and when challenged, \gpt can uncover assumptions in choices of evaluation equations. External options for checking could include using metrics and equations established by published standards for engineering codes and proposed for items such as sustainability, safety, and ergonomics as appropriate to the use case. 
To solve the use of fake libraries or using fake methods, once \gpt was challenged enough times it would eventually offer an existing option. A more efficient solution when it cycled through fake options for \jscad programming was to input a working example of any kind into \gpt along with the request for a working code, using its capacity for modularity to help it structure a working response. 

\noindent \textbf{L.3 Scalability:} 
Other challenges provided obstacles to evaluation. For objective criteria, first order analysis is readily available on all metrics tested, but the scalability in complexity is limited. It was possible but more difficult to get more advanced characterization, for example generating code for FEA for mechanics. As another challenge, the quality of evaluation was found to be best when 1-2 performance metrics were analyzed at once. When too much was requested at once the output quality decreased. 

\noindent \textit{Potential Solutions:}
To handle the limitation of scalability of the complexity of analysis in a given domain, use of existing domain-specific APIs would be suggested. To handle the limitation in amount of metrics to be assessed, the analysis for metrics should be developed one by one into subfunctions that are then stitched together. However, making a longer chat in this format then runs into memory issues of \gpt, for which we found it to forget sets of function inputs and other details within two exchanges. This, in turn, requires giving reminders of the important parts of previous answers (such as the overall function input) when generating each subfunction. When generating the FEA code, a suitable solution was to have \gpt keep repeating the same entire code, and occasionally switch between asking for 2D and 3D versions to create something simple enough before increasing the challenge level, and iterating back again when next parts of the code were found to break, until the entire function worked. 

% \noindent \textbf{Opportunities:} 
\paragraph{Opportunities}
We recommended that a good workflow for analyzing performance would utilize a buildup of complexity, beginning with discussing the design in text form and then generating a function to evaluate a design input in a parametric form. Many issues arising from performance evaluation could be attenuated by relying more on existing methods, libraries, and APIs that have already been created for the use-case in question. 














% % good at 
% \begin{enumerate}
% \item \gpt complains when asked to evaluate when requests are ambiguous, details are left out, or when the performance metric is complex, but is still able to output reasonable first-order approximation functions
% \item It could also generate rankings between a set of designs without giving explicit intermediate scorings. 
% \item To assess subjective metrics, it worked best to develop scoring systems by breaking down a subjective feature into smaller, more quantifiable parts. The use of categories also worked. 
% \item A good workflow for analyzing performance is discussing the design in text form and then generating a function inputting a design in parametric form. It was able to eventually assess any property we tested. Output evaluation functions it generated usually always worked, the code had no errors. 
% % like w/design section, the buildup helped
% \item Through discussing in text form, \gpt could suggest design considerations and metrics at a fairly sophisticated level

% finds challenging & solutions 
% \item For objective criteria, first order analysis is easy, it's possible but harder to get more advanced characterization, for example generating code for FEA for mechanics
% % limited complexitly, scalability in complexity 
% \item The source material for equations it uses in evaluation can be a mystery, can contribute to error, sometimes embed assumptions, but \gpt can uncover those assumptions if asked 
% % reliability / verification. It can try to justify but even be wrong. May need an external checker. 
% \item The quality/accuracy of evaluation is best when 1-2 performance metric are analyzed at once. When too much is requested at once the output quality decreases. 
% % how to solve? do one thing at a time. write subfunctions. 
% % forgets --- keep reminding it. every 2 messages. Go down trees. Give it reminders of the answer it gave you back. 

% % unsolved failures 
% % how to solve???? using an existing API 
% \item Failed to appropriately evaluate performance that required spatial reasoning, like center of gravity or stability, for items having multiple components 


% \end{enumerate} 

% types of failure: things it's bad at reasoning at, verification, scalability
% highlight similarities and differences between sections -- after


% \paragraph{subjective criteria: categorization, ranking, and scoring}

% In order to more carefully address \gpt's evaluation process, three methods of output were considered: 
% \begin{enumerate}
%     \item Categorization
%     \item Scoring
%     \item Ranking
% \end{enumerate}



% What we have tried:
% - try to have it generate reproduce functions from high-level specifications which a user assumes "exists" 
% - 
% examples: given a circuit test for correctness -> does it do what it's supposed to do

% - Making suggestions of discrete changes


% given a design - can it describe what it does, can it undertand a design, evlaaute how it can be used?

% Copter example -


% \begin{itemize}

% \item \textbf{Q} Can GPT assess the performance for a given text input design of an item? 
% \item \textbf{Q} Can GPT generate generalized evaluation functions from high-level specifications for a parametrized design which a user assumes "exists"? (ex. "a chair of a typical design") 
% \item \textbf{Q} What kinds of metrics can it assess? What kinds of objects can it assess? 
% \item \textbf{Q} To what accuracy can it assess objective performance metrics like weight and strength? Can it generate functions with higher level complexity, like FEA for mechanics?

% %------------------

% \item \textbf{Q} Can it assess more subjective metrics? 
% \item \textbf{Q} Can GPT assess performance when objectives not easily quantifiable? 
% \item \textbf{Q} Can GPT assess performance when not everything about the design is known?

% \item \textbf{Q} How to use GPT for multi-objective metrics? Can it help us with weighting? Can it \textit{interpolate} metrics, so manipulating the \textit{metric space}?
% \item \textbf{Q} If we cannot come up with an analytic metric, can it still help us to evaluate the performance by comparing two or even more designs?
% \end{itemize}

% \paragraph{Overview}
% Give a simple example.
% Given this chair made of wood:
% Binary: Can it support 100kg?
% Scalar: what is the maximum weight it can support?
% Vector: What is the maximum weight it can support and how tall is it?

% \paragraph{Input}
% The input is a design specified using a prompt. It can be converted to code description, the input is also the performance metric. 
% How to specify the performance metric? 
% Give examples: weight, cost, amount of material, size, aesthetics, mechanical (or other physical) properties, 

% \paragraph{Output}
% The output is the behavior. Evaluation of the behaviour.
% This can be a binary value (yes/no).
% This can be a continuous score. Or multiple values.


% \paragraph{Quantitative Metrics}
% - show that is struggles with spacial reasoning and more quantitative approaches 
% - use APIs to run metrics/simulation etc.  




% \paragraph{Scoring, Categorizing, and Ranking}
% - how tall is this?
% - 2 version and ask which one is taller?
% - say something can be tall, medium, short and see how good it is





% \paragraph{Performance evaluation examples}
% Static objects
% Dynamic objects
% Stationary objects with on-board (embedded) function (electronics or chemicals, computation (programs))?
% Subjective objects or those that require human feedback (drugs, website design, aesthetics, human interaction)


% Start with boxes, tables, chairs, 
% Cars - 
% Quadcopter - how far does it fly on a battery?  How fast can it go?
% Pharmaceuticals - yes/no is drug effective inhibitor (does it pass a randomized trial)?




%%%\input{sections/GraphArrows/DesignToManufacturing.tex}

\section{Performance And Design Space To Design (Inverse Design)}\label{sec:inverse}
Although generative algorithms can produce candidates for designs, there is no guarantee concerning their quality. Inverse design is focused on producing designs that are, by some metric, as close to optimal as possible, given the constraints.  
Put in the vocabulary of the preceding sections, given a design space and performance metrics (which can define values to be optimized or constraints to be satisfied), inverse design answers the following question: which design in our space provides optimal performance without violating any constraints?

A design generated by an \llm must therefore satisfy several requirements: \textit{1)} it must be valid,  \textit{2)}  it must be performant, \textit{3)} it must satisfy design constraints, and, \textit{4)} in the context of manufacturing, it must be buildable.  With \textit{3)} and {4)}, we note the persistent reality of the sim-to-real gap --- that is, objective and constraint values may differ \textit{in silico} and \textit{in situ}.  Basic challenges involve specifications of the inverse problem to an \llm (much of which was described in previous sections), as well as generation of an effective algorithm for design optimization.
Although {\llm}s cannot natively search for optimal solutions to a novel problem, they can make educated starting guesses and output optimization code that users can execute.  Much of this section is thus focused on prompting {\llm}s to generate meaningful code for problems dependent on aspects such as their parameterization support (\textit{e.g.} continuous versus discrete domains), performance objective landscape, or fabrication constraints.  
% formulate code to be executed by function calling\wojciech{Reword}
Real-world problems introduce nuanced challenges, including exploring over multiple competing objectives, difficult-to-specify objectives (such as aesthetics and objectives that depend on long-term use), and an evolving landscape of emerging methods that an \llm may not know about.  In this context, we could consider whether \gpt can propose strategies (even novel ones) that free designers from some of the typical burdens associated with the optimization pipeline. 
% \wojciech{Replace LLM with GPT}

With these considerations, we aim to investigate the following questions:


\textbf{Q1} When can \gpt solve a problem analytically, and when does it need to resort to using an outside tool (\eg, a programmed algorithm)? 
% table example 
 
 \textbf{Q2} Can \gpt choose reasonable algorithms for different types of supports for constraints, objectives, and decision spaces (\eg, continuous, discrete, binary)?

 \textbf{Q3} Can \gpt assist designers in navigating the landscape of possible trade-offs when multiple conflicting objectives are present?

 \textbf{Q4} Can \gpt support optimization in contexts that require additional knowledge, specifically when a design space is not properly defined or is missing constraints?

In this section, we investigate, generally speaking, modern {\llm}s' abilities to navigate and (semi-)automate design optimization problems.



\subsection{\gpt: Analytical vs. Outside Tools \textbf{(Q1)} }
We know that \gpt has the ability to reason about many mathematical operations, including both algebra and calculus, which is sufficient to solve many real-world engineering problems.  We emphasize ``reasoning'' because, although \gpt clearly proposes reasonable analysis steps, is not obvious that \gpt is correctly executing those steps;  as we'll see, \gpt often makes mathematical errors.  Still, it is reasonable to wonder if \gpt's own internal reasoning is sufficient for inverse design.  Where are the limits of that reasoning?  When must it  resort to code and external libraries, or plugins?  Each of these approaches has its own pitfalls that suggest caution for developers.

Consider an example in which we maximize the stability of a table (Fig.~\ref{fig:table_intuitive}).  \gpt correctly describes that an object is statically stable when its center of mass lies within its support polygon.  One considers stability of an object to be maximized when it remains stable under as large of a perturbation as possible.  In principle, that typically means two things: \textit{1)} moving the center of mass as far away from the boundary of the support polygon as possible, \textit{2)} decreasing the object's experienced motion (typically caused by gravitational torquing) when perturbed. \gpt is able to apply these intuitive principles to reason about the optimal solution within given bounds in this case. 

In a similar example shown in Fig. \ref{fig:table_wolfram}, the Wolfram plugin is enabled, which \gpt can selectively call at its discretion.  While the Wolfram plugin was a natural choice for solving what appears to be a simple analytical optimization problem, \gpt timed out. In practice, this can happen for at least three reasons: \textit{1)} not enough time was allotted for the computation, \textit{2)} the problem is too difficult for Wolfram to handle, or, more generally, \textit{3)} the query may produce a problem that is not tractable or -- in the extreme case -- not computable \cite{turing1936computable}.  Although it might seem like it would be trivial to provide Wolfram with more time to complete the computation, in practice, the user has no feedback on how long the computation would take.  It is unreasonable to ask a user to wait indefinitely without feedback, and most numerical optimization algorithms will be unable to provide a reasonable estimate of progress. In this case, ``anytime algorithms'' (which can return a valid partial or approximate solution even if interrupted early)  may be especially practical \cite{zilberstein1996using}.

\input{sections/GraphArrows/InverseDesignExperiments/TableExperimentWolfram}

% \input{sections/GraphArrows/InverseDesignExperiments/TableExperimentPython}

After failing to optimize over the full space using Wolfram, \gpt continues the conversation by reasoning that the optimum value will occur near the boundary of the constraints (Figure~\ref{fig:table_wolfram}). By exploiting this reasoning, it successfully uses the Wolfram Plugin to compute and evaluate the equations corresponding to a small set of extremal points in the design space.  Despite this, it fails to realize that certain solutions dominate others, and does not prune out bad candidates. Moreover, \gpt neglects to justify or prove its claim that the optimum should occur near the boundaries; though it was correct in this case, this approach may fail in general.

In a follow-up experiment (Fig. \ref{fig:table_python1}), \gpt is asked to perform the same optimization task via Python code, which enables it to use an external library.  It chooses L-BFGS-B, which is a reasonable, standard, and easily accessible (though not state-of-the-art) solver for continuous valued problems.  It does not, however, provide gradients that can expedite the computation unless prompted for them. We explicitly prompt \gpt to provide the gradients (Fig. \ref{fig:table_python2}) and visualize the results in Fig. \ref{fig:table_opt_render}. Generally speaking, the unoptimized approach on \gpt's part is an issue \wrt performance, as not all users will be intimately familiar with all (or perhaps any) optimization libraries, and they may not realize that by providing additional information (\eg. gradients), the computation can be expedited. \gpt also does not elect to make use of Wolfram or autodifferentiation; in practice, lack of direct computation can lead to errors.
Later in this section, we demonstrate how \gpt struggles to solve a (much) more difficult version of this optimization problem.

\input{sections/GraphArrows/InverseDesignExperiments/TableExperimentPython}


% (\todo{}: timings comparison)

\input{sections/GraphArrows/InverseDesignExperiments/Table}



Throughout these experiments, we noticed several common issues in \gpt's approach.  First, if users do not prompt \gpt explicitly to show its work, it may resort to ``common-sense'' reasoning about a problem.  Although this reasoning \textit{could} be correct, \gpt provides no certificate to a user, as seen in the ``intuitive''  physics of Fig. \ref{fig:table_intuitive}, or the boundary-aligned optima assertion in \fref{fig:table_wolfram}.
Another issue occurs if it is difficult to find a library to solve a particular task; in this case, \gpt often gives up or attempts to write its own code. If the code is detail-heavy, it may be too difficult for \gpt to write correctly and the code/solution may be incorrect. If a library does exist but is used uncommonly, \gpt may give incorrect instructions on how to install/use that library; or, in some cases, \gpt may hallucinate a library altogether. 


\subsection{Reasoning about different problem types and selecting appropriate solvers (\textbf{Q2}) }

To test \gpt's understanding of various problem domains and its ability to identify appropriate solutions for each, we conducted several experiments spanning a wide range of search spaces, constraint spaces, performance spaces. In some cases, the problems have additional real-world considerations of which \gpt must be cognizant in order to choose a suitable optimization approach.
Tables \ref{tab:inverse_table1} and \ref{tab:inverse_table2} provide a comparison of different problems that \gpt was asked to solve.  We describe each example in additional detail below, with the exception of the table stability optimization (which was presented in the previous section).

Overall, we found that even over varying problem types, \gpt exhibits extreme robustness when reasoning about and choosing an adequate solver for any given problem.  In cases where a more sophisticated algorithm was needed, it tended to choose at least the correct algorithm class, even if it was not always aware of the best version or implementation.  One notable example was in optimizing robot topologies for ground locomotion: \gpt identified an evolutionary algorithm as an effective optimization method, but did not choose any state-of-the-art specific algorithms or implementations.

 

\begin{table}[bt]
\small
\begin{tabular}{|c|c|c|}
 \hline
 \multicolumn{3}{|c|}{Inverse Design Problems} \\
 \hline
 Problem Name & Search Space & Constraint Space \\
 \hline
 Table Stability & Continuous & Parameter Bounded \\
 \hline
 Robot Arm & Continuous  & Parameter Bounded \& Continuous Function\\
 \hline
 3D Printer Parameters & Continuous  & Parameter Bounded \\
 % \hline
 %Robot Topology Design & Graph  & None & Continuous & None & Evolutionary Algorithm\\
 %\hline
 % Multicopter Optimization & Continuous  & Parameter Bounded & Continuous & LQR Control Synthesis Subroutine & Second-Order Gradient-Based/Grid Search\\
 \hline
 %Topology Optimization & Continuous  & Parameter \& Function Bounded \\
 %\hline
 Cabinet Optimization & Continuous  & Parameter Bounded \\
 \hline
 Robot Arm Planning & Continuous  & Continuous \\
 \hline
 Chair Design & Continuous  & Continuous \& Bounded  \\
 \hline
\end{tabular}
\caption{\label{tab:inverse_table1} \textbf{Descriptions of Search and Constraint Space of Inverse Design Problems.}}
\vspace{-5mm}
\end{table}

\begin{table}[bt]
\resizebox{\textwidth}{!}{
\begin{tabular}{|c|c|c|c|}
 \hline
 \multicolumn{4}{|c|}{Inverse Design Problems} \\
 \hline
  Problem Name & Objective Space & Other Considerations & Chosen Optimization Method\\
 \hline
 Table Stability &   Continuous & None & Analytical/Second-Order Gradient-Based \\
 \hline
 Robot Arm  & Continuous & None & Second-Order Gradient-Based\\
 \hline
 3D Printer Parameters & Continuous & Expensive Real-World Experiments & Bayesian Optimization\\
 % \hline
 %Robot Topology Design & Graph  & None & Continuous & None & Evolutionary Algorithm\\
 %\hline
 % Multicopter Optimization & Continuous  & Parameter Bounded & Continuous & LQR Control Synthesis Subroutine & Second-Order Gradient-Based/Grid Search\\
 \hline
 %Topology Optimization  & Continuous & None & N/A \\
 %\hline
 Cabinet Optimization  & Function Bounded & None & Second-Order Gradient-Based\\
 \hline
 Robot Arm Planning & Continuous & Logical Reasoning with High-level Primitives & Greedy Search, Brute Force\\
 \hline
 Chair Design & Continuous & Multi-Objective & NSGA-II (Evolutionary Algorithm) \\
 \hline
\end{tabular}}
\caption{\label{tab:inverse_table2}\textbf{ Results of the Inverse Design Queries to \gpt.}}
\vspace{-5mm}
\end{table}
% \begin{adjustbox}{angle=90}
% \small
% \begin{tabular}{ |c|c|c|c|c|c|  }
%  \hline
%  \multicolumn{6}{|c|}{Inverse Design Problems} \\
%  \hline
%  Problem Name & Search Space & Constraint Space & Objective Space & Other Considerations & Chosen Optimization Method\\
%  \hline
%  Table Stability & Continuous & Parameter Bounded & Continuous & None & Analytical/Second-Order Gradient-Based \\
%  \hline
%  Robot Arm & Continuous  & Parameter Bounded \& Continuous Function & Continuous & None & Second-Order Gradient-Based\\
%  \hline
%  3D Printer Parameters & Continuous  & Parameter Bounded & Continuous & Expensive Real-World Experiments & Bayesian Optimization\\
%  % \hline
%  %Robot Topology Design & Graph  & None & Continuous & None & Evolutionary Algorithm\\
%  %\hline
%  % Multicopter Optimization & Continuous  & Parameter Bounded & Continuous & LQR Control Synthesis Subroutine & Second-Order Gradient-Based/Grid Search\\
%  \hline
%  Topology Optimization & Continuous  & Parameter \& Function Bounded & Continuous & None & N/A \\
%  \hline
%  Cabinet Optimization & Continuous  & Parameter Bounded & Function Bounded & None & Second-Order Gradient-Based\\
%  \hline
%  Robot Arm Planning & Continuous  & Continuous & Continuous & Logical Reasoning with High-level Primitives & Greedy Search, Brute Force\\
%  \hline
%  Chair Design & Continuous  & Continuous \& Bounded & Continuous & Multi-Objective & NSGA-II (Evolutionary Algorithm) \\
%  \hline
%  \caption{Results of the inverse design queries to \gpt.}
%  \label{tab:inverse}
% \end{tabular}
% \end{adjustbox}



\paragraph{Robot Arm Optimization}
In this example, shown in \fref{fig:robot_arm_exp}, a robot arm is to be optimized such that it reaches a target location in space.  As requested, \gpt generates a two-link robot design parametrized by the link lengths, and then uses inverse kinematics to provide a solution for the link lengths so as to reach a target location in space.  When asked to transform this into a design optimization problem, \gpt sets up an optimization problem, creating an appropriate constraint (end-effector touching goal), an objective (sum of link lengths, as a proxy for material cost), and parameters with reasonable bounds.  All of these were automatically provided by \gpt, without explicit request.  Notably, the optimization code is easily generalizable to arbitrary locations in space (though certain aspects like parameter bounds may need to be modified). As an optimization procedure, \gpt chooses L-BFGS; a reasonable choice given the continuous nature of the problem. A rendering of the optimized robot can be seen in \fref{fig:robot_arm_opt}.

\input{sections/GraphArrows/InverseDesignExperiments/RobotArmExperiment}

\input{sections/GraphArrows/InverseDesignExperiments/RobotArm}


% \paragraph{\textbf{Robot Topology Design}}
% In this example, a user asks for help in generating robots that can efficiently locomote forward.  Notably, it is important to be specific about the dimensions of the problem --- asking for a ``robot'' could lead to a 2D planar robot, which abstracts away complex real-world dynamics.  \gpt initially provides code to generate the simulation environment, as prompted, and then subsequently provides a way to programmatically intantiate and optimize robot designs.  We note that if the entire problem is queried to \gpt, it will tend to say that the problem is too difficult, but if broken up in this way --- first asking for a performance evaluation method, then asking for code for the design space, and then asking for code for the inverse design problem, \gpt can do an adequate job.  The evolutionary algorithm provided by \gpt was one it had to be prompted to fill out in detail; it is worth noting that open-source software options for such optimization exist, but \gpt did not opt to employ them for this problem.  The example can be seen in Listing (TODO), while example outputs can be seen in Fig. (TODO). \wojciech{Not enough details. Anything about the constrains. Which BO library.}
\input{sections/GraphArrows/InverseDesignExperiments/PrintingParameters}

\paragraph{Optimizing 3D Printing Parameters}
In this more abstract example, \gpt is simply asked which algorithm to use in order to optimize the parameters of a slicer used in 3D printing.  It chooses Bayesian optimization, which is a good choice for problems with real-world experiments where it is preferable to minimize the number of required experiments. \gpt also provides skeleton code for the optimization.  As this is a more abstract example, specifics are not supplied. The listing can be found in \fref{fig:printingparams}.

%\paragraph{\textbf{Material Topology Optimization}}
%In this ``failure'' example, \gpt is asked to optimize the material distribution of an input geometry for structural loading.  However, upon many repeated attempts, \gpt continually either \textit{a)} hallucinates software libraries and/or their capabilities, or \textit{b)} does not provide useful instructions for installing \wojciech{We never ask for installing libraries. Just using the libraries. Explain. } them, making it difficult to provide an on-ramp for users.  Without a way to evaluate a finite element system, inverse design becomes hopeless. \wojciech{Maybe move this to the end. Give more details on the problem. Is it 2D/3D? }


\paragraph{Cabinet Optimization}

We investigate whether GPT can output a reasonable cost function and design that optimizes the function when provided an example design, a parameterization of the design space, and a text description of the objective. One instantiation of this problem setting is with furniture: can GPT optimize the design of a cabinet such that the result has a user-specified volume while minimizing the cost to build it? First, we prompt \gpt with an example cabinet design in \jscad (\sref{sec:textToDesign_JSCAD_basic}) and a parameterization of the design (including bounds for the parameters). Then, we ask it to generate functions to compute volume and material cost. Once the user verifies the accuracy of the functions, we have \gpt  output a python script that can minimize the cabinet's material cost with respect to a given volume constraint. The resulting code is shown in \fref{fig:cabinet_opt}, with renders of an optimized cabinet in \fref{fig:cabinet_opt_render}.

% Figure environment removed

% Figure environment removed

\paragraph{Robot Arm Planning}

We now study a planning problem: given a claw attached to an arm and an environment with objects and bins, \gpt must control the arm-claw robot with a sequence of commands that efficiently picks up all objects and places them into bins. Each bin an only hold one object. In the arm-claw interface provided to \gpt, the physical embodiment of the arm-claw robot does not matter; this allows \gpt to simply reason about the movement of the claw and whether the claw should grasp or release an object. Due to the nature of the problem, there is one critical constraint to consider: the claw must visit an object to pick it up before dropping it off at a bin. Formalizing this constraint is non-trivial, but the performance objective is much simpler: minimize the claw's travel distance. To simplify the problem, we also add that the maximum number of objects and bins is 3 each, making brute force a valid solution. The initial prompt and result are shown in \fref{fig:claw1}.

\input{sections/GraphArrows/InverseDesignExperiments/RobotArmPlanning1}

Overall, \gpt understands that it needs to keep track of the claw's position to compute the correct distances and that the claw should move to an object before moving to a bin. Still, it is unable to output an optimal solution, even when the problem statement permits a brute force approach 

\input{sections/GraphArrows/InverseDesignExperiments/RobotArmPlanning2}

To address this, we explicitly emphasize that the output function should guarantee that the minimum distance is traveled. Even in this case, the optimal solution is not necessarily reached. As shown in \fref{fig:claw2}, \gpt's code fails to consider all possible bins that an object could be placed into once it has been picked up. However, we note that the solutions have high variance -- on a different run, \gpt does produce a correct brute force solution. A third run produces code that guarantees an optimal solution but is inefficient, as it computes the translation for paths that do not obey the constraint that a claw must pick up an object before placing it in a bin.






 \subsection{ Navigating trade-offs between multiple conflicting objectives  (\textbf{Q3})}\label{sec:chair_opt}

Although our previous experiments focused on optimizing a single performance objective, we now explore the scenario where a user wishes to navigate a higher-dimensional (\ie. multi-objective) performance space.  The user begins by asking \gpt for reasonable performance metrics for evaluating a chair. After \gpt provides eight such metrics, our user purposefully selects stability and sustainability, since they can be mathematically quantified by tipping angle and volume respectively. The user then asks for parameters over which to search. Since \gpt has not been given a design template, \gpt proposes parameters abstractly; we note that it might have been more useful if \gpt first proposed a skeleton for the chair geometry, especially so that a user could understand the ramifications of these parameters. After iterating with \gpt to generate correct \scad code for the design, the user requests code to evaluate and optimize the chair. \gpt proposes the use of NSGA-II -- a very common evolutionary method for computing the Pareto front of the multi-objective trade-off space -- and provides code for the optimization.  As an oversight, \gpt initially excludes design parameters bounds from the optimization, despite verbally providing ideas earlier in the conversation.  When prompted to add the bounds into the optimization code, because of its limited memory, \gpt suggests reasonable, but notably different parameter bounds. Additionally, \gpt must be prompted again to enforce the bounds consistently throughout the algorithm (specifically, in the crossover and selection operators).  Results can be found in \fref{fig:chair_opt}.



\input{sections/GraphArrows/InverseDesignExperiments/Chair}

Through this example, we conclude that \gpt has the potential to aid users in both \textit{a)} understanding the trade-offs involved in different candidate designs, and \textit{b)} providing pointers to a reasonable algorithm that can help navigate that space.


\iffalse

TODO:
Format as: Name, Search Space, Constraint Space, Objective Space, Other considerations, Chosen optimization method.

-Robot Arm optimization - Continuous Search Space, Reachability (continuous) constraint, Continuous Objective, 2nd order gradient-based optimization.

-Robot topology design - Graph-search space, No constraints, objective continuous, TODO, Evolution optimization method

-Printing Parameters - Continuous search space, bound constraints, objective continuous, very expensive evaluation function, Bayesian Optimization

-Multicopter Optimization - Continuous search space, bound constraints, objective continuous, requires simulation in the loop and knowledge of quadcopter materials, Grid Search

-Mixed Optimization?  Bookshelf. TODO

-Logical Constraints, continuous objective --- Robot motion planning TODO

\fi



\subsection{Supporting optimization in contexts that require additional knowledge (\textbf{Q4})}

In many cases, it can be daunting to fully specify a given inverse design problem in a new domain: for example, it may be difficult to specify appropriate design spaces and objective functions, and it may be unclear how to deal with underspecified/unknown constraints.  In this section, we briefly examine how \llms may reduce the burden of this process to make inverse design more accessible.  %Further, we demonstrate the ability of LLMs to aid in generating design specifications that are difficult to specify or quantify mathematically.

%  has nuances \wojciech{what are these nuances?} that they have not considered \wojciech{improve this sentence}

The chair example discussed in Section \ref{sec:chair_opt} demonstrates \gpt's ability to recommend reasonable parameters for a design without needing explicit, low-level prompts from a user. Indeed, when prompted for ``parameters'', \gpt is able to apply its knowledge of the target domain to offer continuous parameterizations of a typical chair (and provide a 3D model on request), along with reasonable ranges for each parameter.  Although discrete parameters are possible with a chair, they are less likely to have a significant impact relative to its raw dimensions, and most chairs are comprised specifically of four legs, a seat, and a back. 

For completely novel problems, \gpt cannot rely on its existing knowledge to generate an exact design space. However, it can apply knowledge of aspects of a problem to new problems in familiar domains. The conversation in \fref{fig:fworp_example} presents a brief example of \gpt being queried about a novel invention: the Fworp\footnote{Name chosen to be deliberately nonsensical so as not to give context clues to \gpt.}.  The Fworp is a robot car with a body made of silicone rubber.  While the value of such a device is unclear (perhaps shock absorption),  it is synthesized from existing ideas: namely, remote control vehicles and soft robotics.  \gpt uses its knowledge of those preexisting domains and their components to recommend reasonable design parameters, and their ranges, including analyzing size, weight, wheel size, power source, peripherals/sensors, and build material properties.  It also provides guidance on performance metrics without being prompted, but classifies these under ``parameters,'' which may confuse users.  Further, when queried about the advantages and disadvantages of such a device when compared with non-rubbery autonomous vehicles and soft robots without wheels, it provides reasonable comparisons.  In particular, it notes that, compared with non-soft robot vehicles, the fworp could (possibly) be more durable, shock absorbent, safer, and quieter, while also potentially being more expensive to produce and tacky.  Compared with non-vehicular soft robots, it has the potential to be more mobile, stable, energy-efficient, and simple to produce and control, but would lack the versatility and human-interaction potential typically afforded by most soft robots; further, while safer than a typical vehicle, it would be more dangerous than most current soft robots.

This experiment highlights the notion the \gpt can be an effective partner when formulating a novel inverse design problem, as it can make connections between the proposed problems and more established domains. Then, \gpt is able to use its existing knowledge base about those related domains to provide reasonable starting points for the problem at hand. With continued user interaction, \gpt can also help to refine, formalize, evaluate, and ultimately act upon the newly created formulation.


\input{sections/GraphArrows/InverseDesignExperiments/fworpExperiment}

\subsection{Discussion}

This section elaborates on \gpt's key capabilities (C), limitations (L), and dualisms (D) in the realm of inverse design.


\noindent \textbf{C.1 Extensive Knowledge Base in Design and Manufacturing:}  \gpt has knowledge of how to formulate design spaces, objectives, and constraints. It also successfully selects suitable search algorithm for a given problem, suggesting that \llms are useful as a building block when formulating inverse design systems. In its current form, \gpt exhibits a number of abilities that make it highly usable.  For example, it was able to choose an adequate design optimization algorithm for almost every problem it was given; when asked, \gpt was also able to justify its choice of algorithm. 

\gpt is also helpful in automatically providing code for a significant portion of a problem formulation without requiring user input. These aspects include parameter choice, parameter ranges, and objective functions.  In the best case, this feature can relieve a user of much of the ``busywork'' associated with a problem (loose bounds, necessary constraints, \etc.). Even when \gpt falls short of this ideal, \gpt is usually able to recommend a useful starting point.

\gpt's reasoning capabilities can also further provide value in novel domains.  If a user is inexperienced with a particular domain or if they are working on a novel problem, \gpt has the capability to synthesize from the problem's constituent domains to provide suitable advice, as demonstrated with the fworp example.  

\noindent \textbf{L.1 Reasoning Challenges:}   When asking for help in setting up a problem, \gpt's advice can be confusing. For example, it often does not disambiguate between the design parameters (which practitioners have direct control over) and performance metrics (which are emergent from the design).  Less experienced designers may then find themselves confused, believing there must be a way to modify a system's performance directly.
\textit{Potential Solutions:} By following up with \gpt about how a given ``parameter'' is computed, one can attempt to disambiguate parameters from metrics.  In general, however, this verbal confusion is difficult to systematically address.

The addition of function calling in \llms, and specifically plugins using \gpt, can eventually allow for direct execution of arbitrary code, even code that \gpt writes.  However, there are no guarantees on the execution time of that code, and it is unclear how to manage problems that might arise, such as long runtimes (which are common in hard optimization problems), or even infinite loops.  In our experiments, the Wolfram plugin was given a brief time window for computation before it timed out, which largely negated its value in the face of more challenging problems.

\textit{Potential Solutions:} Methods to allow for function calling while providing guarantees or control by a user (say, in the form of anytime algorithms) would be beneficial.  For now, writing one's own plugin may allow greater granularity over the type of algorithm being used.  Thus, the algorithms can at least be catered to \gpt's behavior.





\noindent \textbf{L.3 Scalability:} Inverse design relies on several complex building blocks, including the specification of design spaces and objective functions. However, as discussed in previous sections, \gpt frequently encounters difficulties when faced with these tasks. Such errors prohibit \gpt from scaling to inverse design exploration altogether. This occurred twice during our experiments.  In one failure case, we had difficulty in evaluating the performance of a soft body system using finite elements; although the example is not detailed in the paper, \sref{sec:design_to_perf} has already shown this to be difficult. In effect, this failure currently prevents \gpt-assisted inverse design of a soft robot \wrt FEA-derived metrics. In a different example, we attempted to design a long, multi-link arm, but found that \gpt struggled with properly geometrical alignment of the links and rotation axes (as shown in \sref{sec:textToDesign_urdf}).

\textit{Potential Solutions:}  
Pointing out problems with solutions (such as runtime errors) can allow \gpt to iterate, but requires intervention and potentially fine-grain coding or engineering knowledge by a user.  In practice, it is often effective to blindly ask \gpt to assess its own output and report any errors it finds until \gpt is satisfied with its own work. In our experiments, this frequently converged to a correct solution.  However, this is not foolproof and can be slow and computationally costly.  Further, without access to web search, \gpt may not know how to reconcile out-of-date knowledge about an API.

A second scalability issue is that \gpt does not always choose the best algorithm for solving a problem, and sometimes does not use a given method in the most efficient way (such as not providing gradient information).  Since \gpt tends to be coy about available methods and how to best use them, a more novice user may be unsure how to navigate the intricacies of optimization and diagnose issues.  Furthermore, although \gpt tends to choose adequate algorithm classes, it does not always choose state-of-the-art methods; instead, it tends to default to standard methods that are highly popular.  Because of its knowledge cutoff, without access to web search, any given \llm may not be aware of state-of-the-art methods or how to implement them. Even if an easy-to-use implementation exists on an online repository (\eg. Github), the \llm may not be aware of the code's existence or how to use it.

\textit{Potential Solutions:} Web search, which has been previously available for \gpt, can help to mitigate these issues, as one could ask for the latest, state-of-the-art methods, and \gpt could provide solutions based on current repository code.  However, there is no guarantee that \gpt will be able to understand what makes newer methods optimal for a problem without sizeable crowd knowledge, which may not be available.

The third scalability issue is that, as mentioned in previous sections, \gpt's ``short memory'' can cause it to forget specifics it had generated earlier in a context; this notably occurred in the multi-objective chair example.  While this problem emerged in other aspects of the design-to-manufacturing pipeline, its impacts were most salient when defining inverse problems, whose specification can be especially long.

\textit{Potential Solutions:} Since inverse design problems can be quite lengthy to define and specify, it may be easier to decompose a problem in the following order: \textit{1)} Ask \gpt for a definition of a design space (including its implementation), \textit{2} Ask \gpt for a definition of a performance metric and constraints (including their implementations) while abstracting away the code from \textit{1)} as an API call; \textit{3)} Ask \gpt to write code for the inverse design search, abstracting away the code from \textit{2)} as an API call.  This can keep definitions shorter and easier to manage.


 \noindent \textbf{D.2 Unprompted Responses:}
Throughout our experiments, \gpt always unilaterally selected an optimization algorithm and proceeded to generate code. In particular, \gpt never provided the user with options for possible optimization algorithms, unless it was explicitly asked to provide such options as an intermediate step. Although \gpt's automated selection may satisfy many users, it runs the risk of creating mistakes that would be difficult to diagnose. This is particularly true because \gpt rarely justified its choice to users.  Furthermore, \gpt's assertion may imply that there is a single ``correct'' algorithm for a given problem, and they may not realize that there are better (or even \textit{alternative}) options available in any given circumstance.  

\gpt's tendency to fill in aspects of an inverse design problem before being asked about them may also lead to mathematical problem definitions which are ill-suited or otherwise suboptimal for a user's real-world design problem.  In these cases, \gpt's tendency to autocomplete and plow ahead could lead users to blindly follow the \llm down bad ``rabbit holes,'' only to discover that a fundamental problem existed much earlier.  Furthermore, since \gpt does not have a native way to execute arbitrary code, it will not always realize that a codeblock has errors.
















\iffalse


The good:
1. Able to find an adequate choice for an algorithm for many problems - Able to also justify it - it doesn't always give you options right out of the gate - can sometimes get confused and rely on the domain to choose the algorithm.
2. Autocompletes a lot of content even before you ask (parameter bounds, constraints, objective function)
3. Can transfer some knowleldge to novel domains based on components - it can be fooled.
4. When the problem has an intuitive answer, can sometimes give that directly. (but can a user trust that?)

The Bad:
1. Some domains, ChatGPT will assume that it knows libraries that don't exist.  This also came up in performance evaluation.
2. Sometimes runs into memory issues setting up the problem.
3. Bad at spaital reasoning.

The Ugly:
1. Plugins can help, but sometimes lead to trouble.
2. Not always choosing the absolute best algorithm, especially if code doesn't exist.  So don't expect the most advanced algorithm.

\fi




%\section{Auxiliary Tools}


This is a section for describing how to enforce:

\paragraph{Visualization Tools}
Describe standard tools for:
SVG (e.g., inkscape, Gimp, Adobe Illustrator)
STL: Meshlab, or any CAD sofware
gcode: repetier, cura, IceSL
CAD formats: OpenJscad, Autodesk Fusion 360
Knitting: KnitOut?
Graph visualization tool - e.g., graph converter tool from Allan, Graphviz, Networkx

\paragraph{Output Verification Correctness Checking}

We want to make sure that the output is correct.
We need to have virtual tools to make sure that the output is valid and correct.

Tools to make sure that the output is valid:

Tools that ensure that the output is correct, can be done by human visual inspection. They can be also done by an automated tools that runs the visualizer or simulator. Show an example of such a tool and simulator.

\paragraph{Post-process Fixing Tools}

Results might not be correct. Describe tools to post-process the output. For example, making sure that things are connected. Non-overlapping. Write a simple post-processing script that fixes the incorrect results.


\paragraph{Using Solvers in an Interactive Loop}
Generate design (geometry) and boundary conditions, materials
We run FEA solver, (we report what max stress is)
LLM can draw additional conclusions from these results.


Constraint solvers (CSP) - TODO
Logical solvers - Coq
Numerical Solvers - Optimizers such as SNOPT, NLOpt, or linear algebra packages like Eigen.

\paragraph{Program Synthesizers}
Sketch, etc.

\paragraph{Simulators}
Static simulators - SimScale, tools built into CAD software
Dynamic simulators - Chrono, Mujoco, Bullet, etc.

\paragraph{Surrogate/learned models}
CLEVERER and intuitive physics types models
etc.





\section{Experimental Results}\label{sec:results}
    \subsection{General Results}
        The basic ResSAN model is used to determine reference results which our expanded model can be compared to as it is structurally similar to ResLAN but does not possess the Lidar adaptive components of it. Further, we compare with the full-size PackNet-SAN and the unmodified NLSPN architecture. 
        As it can be seen from Tab.\,\ref{tab:sota-results}, our LiDAR-adaptive ResLAN achieves competitive performance compared to state-of-the-art standard depth completion methods, which are specialized to the unfiltered 64-beam-LiDAR. The performance differences are in the range of a few centimetres in terms of MAE, which is acceptable given the practical advantage that ResLAN can generalize to different beam patterns as will be shown below.

        Furthermore, we compared the architectures for a set of three different input types that contained 64, 32 or 16 LiDAR channels using both filter types on the metrics from the KITTI benchmark. The NLSPN model was trained for the standard depth completion task and then evaluated with different input data. As for the ResSAN models, we trained one model for each input type and tested it for the corresponding one which serve serve as the \emph{Baseline} in Tab.\,\ref{tab:overall-results}. Our ResLAN model was jointly trained for all three settings. As listed in Tab.\,\ref{tab:overall-results}, the ResLAN models outperform the challenging baseline in all metrics for FOV filtering and all but one for sparse filtering. This implies that our LiDAR adaptive model is able to outperform dedicated models in case of very sparse input depth. Fig.\,\ref{fig:comp-plot} shows this is indeed the case for 32 and even more for 16 channels. For FOV-filtered inputs with 16 channels, the ResLAN exhibits approx. $10\%$ smaller MAE than the baseline. As for the NLSPN, it becomes apparent that it is not capable of generalizing to other input types since it shows clearly worse results. The difference is especially pronounced for the FOV filtering where on average more than every fourth predicted pixel is more than $25 \%$ deviating from the ground truth\,($\delta_{1.25}$). Therefore, using a weight-adapting network in combination with differently filtered input depths allows us to train models that outperform their non-adaptive counterparts.

        \begin{table}[]
            \centering
    	    \small
            \vspace{0.4cm}
            \caption{\textbf{Depth estimation result for standard depth completion} when the ResSAN model was only trained for 64 channels and the ResLAN model for multiple tasks. The PackNet-SAN and NLSPN models were trained with the setup that was also used for our model architecture.}
            \footnotesize
            \setlength{\tabcolsep}{5pt}
            \begin{tabular}{@{}lrrrrl@{}}
            \toprule
            \multicolumn{6}{c}{\textbf{Standard LiDAR Depth Completion}}                                                                                                                         \\ \midrule
            \multicolumn{1}{l|}{Method}          & RMSE $\downarrow$            & MAE  $\downarrow$            & iRMSE $\downarrow$             & iMAE $\downarrow$ & $\delta_{1.25}$ $\uparrow$ \\
            \multicolumn{1}{l|}{}                & \multicolumn{1}{l}{{[}mm{]}} & \multicolumn{1}{l}{{[}mm{]}} & \multicolumn{1}{l}{{[}1/km{]}} & {[}1/km{]}        &                            \\ \midrule
            \multicolumn{1}{l|}{PackNet-SAN}     &  914                            &  298                            &  2.78                              &  1.4                 &  99.65 \%                          \\
            \multicolumn{1}{l|}{NLSPN}           &  \textbf{889}                            &   \textbf{263}                           &  \textbf{2.62}                              &   \textbf{1.3}                &   \textbf{99.61} \%                         \\ \midrule
            \multicolumn{1}{l|}{ResSAN (Ours)}   & 948                             &  275                            &  2.75                              &    1.4               &   99.58 \%                         \\
            \multicolumn{1}{l|}{ResLAN (Ours)} &   969                           &  283                            &   2.83                             &   1.4                &  99.56 \%                          \\ \bottomrule
            \end{tabular}
            \vspace{0.2cm}
            \label{tab:sota-results}
        \end{table}

        \begin{table}[]
    	    \centering
    	    \small
    	    \caption{\textbf{Depth estimation results of the two baseline setups and the explicit and implicit ResSAN} when evaluated on a combination of 16, 32 and 64 channel depth inputs. Please note that Specialist Methods need to train three specialized networks, one for each of the three types of inputs while our method only uses one network.}
            \footnotesize
            \setlength{\tabcolsep}{4.8pt}
            \begin{tabular}{@{}lrrrrl@{}}
                \toprule
                \multicolumn{6}{c}{\textbf{Sparse Channel Filter}}                                                                                                                                  \\ \midrule
                \multicolumn{1}{l|}{Method}        & RMSE $\downarrow$            & MAE  $\downarrow$            & iRMSE $\downarrow$             & iMAE $\downarrow$ & $\delta_{1.25}$ $\uparrow$  \\
                \multicolumn{1}{l|}{}              & \multicolumn{1}{l}{{[}mm{]}} & \multicolumn{1}{l}{{[}mm{]}} & \multicolumn{1}{l}{{[}1/km{]}} & {[}1/km{]}        &                             \\ \midrule
                \multicolumn{1}{l|}{NLSPN}         &  1396                            &  437                            & 5.54                               &  2.2                 &  98.82 \%                           \\
                \multicolumn{1}{l|}{Baseline}      & \textbf{1207}                             &  381                            & 4.41                               &  1.8                 &  \textbf{99.37} \%                           \\
                \multicolumn{1}{l|}{ResLAN (Ours)} &  1215                            &  \textbf{378}                            &  \textbf{4.27}                              &  \textbf{1.7}                 &  99.31 \%                           \\ \toprule
                \multicolumn{6}{c}{\textbf{Field-of-View Filter}}                                                                                                                                   \\ \midrule
                \multicolumn{1}{l|}{Method}        & RMSE $\downarrow$            & MAE  $\downarrow$            & iRMSE $\downarrow$             & iMAE $\downarrow$ & $\delta_{1.25}$ $\uparrow$ \\
                \multicolumn{1}{l|}{}              & \multicolumn{1}{l}{{[}mm{]}} & \multicolumn{1}{l}{{[}mm{]}} & \multicolumn{1}{l}{{[}1/km{]}} & {[}1/km{]}        &                             \\ \midrule
                \multicolumn{1}{l|}{NLSPN}         &  2738                            &  1702                            & 12.3                              &  4.3                 &  74.69 \%                           \\
                \multicolumn{1}{l|}{Baseline}      &  1556                            &  525                            &  6.8                              &  3.0                 & 98.14 \%                            \\
                \multicolumn{1}{l|}{ResLAN (Ours)} &  \textbf{1548}                            &  \textbf{519}                            &  \textbf{6.44}                              &  \textbf{2.8}                 & \textbf{98.52 \%}                            \\ \bottomrule
            \end{tabular}
            \label{tab:overall-results}
        \end{table}

        
        
        % Figure environment removed
        
        % Figure environment removed

    \subsection{Filter Effects}
        Comparing the effect of the two different types of depth input filters on the model performance, it becomes apparent that FOV filtering is the more challenging task. In that setting, reducing LiDAR channels is more detrimental to the performance than sparse filtering as it creates regions where no depth information is available. Effectively, the model is forced to perform depth prediction in these regions. These effects are highlighted in the depth images in Fig.\,\ref{fig:dense-maps} where the effect of a 16-channel sparse depth filter and a 16-channel FOV can be compared.

    \subsection{Generalization Capabilities}
        We trained three models for both filter types eaach, so the combinations and number of filtered depth inputs they receive are different. This serves the purpose of testing the generalization capabilities of the ResLAN architecture as well as the robustness to different filter settings. After training, the models were evaluated for the depth input settings they were trained for, as well as for ones they weren't exposed to. Overall, ResLAN shows good generalization capabilities. As one can gather from Fig.\,\ref{fig:explicit-comp} and Fig.\,\ref{fig:implicit-comp}, the consequences of slightly varying sets of input depth settings are limited. The most considerable deviations can be seen when the model is tasked to extrapolate. For instance, the model $\{64, 32, 16\}$ shows a noticeably higher MAE for eight-channel depth inputs than the model that was trained for it. Similar behaviour can be seen for the FOV filtering case as well for the model $\{64, 48, 32\}$ when tasked to generalize for a 16-channel input. There is no such pronounced effect for generalization tasks that lie between two filter settings the model was trained for. At most, it can be observed that models that were trained for a smaller range of filter values perform slightly better than ones that have to cover a wider range. The number of filter settings used in a fixed range does not relevantly influence the model performance, as can be seen, when comparing the two models in Fig.\,\ref{fig:implicit-comp}, which are both trained for a range of 64 to 32 channels but one with three filter settings and the other one with five.
    
    % Figure environment removed
    
    
    % Figure environment removed
%\section{Limitations}

Describe limitation of each of the components of the workflow

\paragraph{Text-To-Design}
What are the limits on design primitives?
What are the limits of constraints?
What are the limits on modularity/hierarchical design?

Evaluate validity of the design.
Evaluate correctness.

\paragraph{Text-To-Design-Space}
Limits of parametric design.
Limits of interpolation/extrapolation.
Limits of grammars.

\paragraph{Design-To-Manufacturing}
Limits of validity of the translation.
Limits of correctness.
Limits on the length of the output. Complexity of the output.
Limits on custom manufacturing processes.

\paragraph{Design-To-Performance}
Which performance metrics are understandable?
Which performance metrics need external simulation?

\paragraph{Performance-and-Design-Space To Design}
Which objective function are understandable?
What are the limits on specifying constraints?
What are the limits of the search process for the inverse?


\section{Discussion}
\label{sec: discussion}
\kmsdelete{In this work} We study \kmsreplace{Fairness-Aware PAC learning}{Fair-ERM} in the malicious noise model, and  in some cases allow 
the learner to maintain optimal overall accuracy despite the signal in Group $B$ being almost entirely washed out.
%when we allow learners to use the
%$\PQ$ randomized expansion of the hypothesis class $\mathcal{H}$
In particular we show that different fairness constraints have fundamentally different behavior in the presence of Malicious Noise, in terms of the amount of accuracy loss that a given level of Malicious Noise could cause a fairness-constrained learner to incur. 
The key to achieving our results, which are more optimistic than those in \cite{lampert}, is allowing for improper learners using the (P,Q)-randomized expansions of the given class $\mathcal{H}$.
%We \kmsreplace{present a picture of the}{prove upper and lower bounds on}
%accuracy loss for a range of fairness notions, given \kmsreplace{this simple randomization step.}{learning over $\PQ$.
%In general our results indicate Fair-ERM (given learning over $\PQ$) is more robust than claimed in \cite{lampert}.
The type of smoothness we create by using $\PQ$ seems to be a natural property that is likely shared by many natural hypothesis classes.

Fairness notions are motivated as a response to learned disparities when there is \kmsdelete{data corruption or} systemic error affecting \kmsdelete{the data for}
one group. 
Fairness notions are supposed to mitigate this by ruling out classifiers that have worse performance on a sub-group. 
This can peg both classifiers at a lower level of performance \kmsdelete{(e.g that the lower subgroup)} in order to \emph{motivate} \cite{hardt16} improving the data collection or labelling process to obtain more reliable performance. 
%So in \kmsreplace{some}{a} sense, sensitivity of the fairness notion to poor sub-group performance caused by malicious noise is the \textit{point} of fairness constraints! 
However, it also desirable that fairness constraints perform gracefully when subject to Malicious Noise because fairness constraints will be used in contexts where the data is unreliable and noisy and this might not be known to the learner.
This tension, exposed by our work, motivates 
%a revisiting of fairness notions from first principles approach and trying to axiomatize the 
%desired properties of a fairness intervention a la cryptography and privacy. \footnote{Work in multi-calibration \cite{multicalib} is a viable direction for this problem but it is unclear how 
%that and related notions behave with unreliable data. }
on going work studying the sensitivity level of fairness constraints. 
%If we we are to take a view, if a classifier is deployed 



%%
%% The acknowledgments section is defined using the "acks" environment
%% (and NOT an unnumbered section). This ensures the proper
%% identification of the section in the article metadata, and the
%% consistent spelling of the heading.
\begin{acks}
This material is based upon work supported in part by Defense Advanced Research Projects Agency (DARPA) Grant No. FA8750-20-C-0075.
\end{acks}

%%
%% The next two lines define the bibliography style to be used, and
%% the bibliography file.
\bibliographystyle{ACM-Reference-Format}
\bibliography{sources}

%%
%% If your work has an appendix, this is the place to put it.
\appendix
\begin{comment}
\section{System Architecture}
\label{appendix:architecture}
\system has a novel modularized system architecture with three key components: 
\emph{StreamManager}, 
\emph{TxnManager} and \emph{TxnScheduler}. 
These components are instantiated in each thread locally.
The execution outline of \system is presented in Algorithm~\ref{alg:algo}.
Transactional stream processing is continuous and potentially never ends (Line 1$\sim$8).
The dependency resolution and execution of state transactions are separated into two non-overlapping phases by punctuations~\cite{Tucker:2003:EPS:776752.776780} (Line 2 and 5), which guarantees that no subsequent input event will have a smaller timestamp. 
Effectively, a batch of state transactions is collected during the first phase, and processed during the second phase.

In the first phase (i.e., stream processing phase), 
the \emph{StreamManager} conducts preprocessing for every input event ($e$). Similar to some prior works~\cite{tstream}, state transactions may be issued but not immediately processed during preprocessing (Line 3).
The \emph{pre\_processing} and \emph{post\_processing} functions are exposed as APIs to users.
The \emph{TxnManager} handles dependency resolution (Line 4) among state transactions and insert decomposed operations to construct a \tpg. We discuss the detailed two-phase \tpg construction process in Section~\ref{subsec:construction}.

In the second phase  (i.e., transaction processing phase), 
the \emph{TxnManager} is first involved again to refine (Line 6) the constructed \tpg with further dependency resolution.
The \emph{TxnScheduler} 
schedules operations for concurrent execution based on the constructed \tpg according to the three dimensions of scheduling decisions (Line 7). 
In particular, a scheduling decision model $M$ is instantiated based on the constructed \tpg (Line 14).
\textbf{\circled{1}} Guided by $M$, execution threads adopt an exploration strategy (Section~\ref{subsec:explore}) to explore the constructed \tpg for operations available to be scheduled constrained by dependencies. 
\textbf{\circled{2}} 
During exploration, one or multiple operations may be treated as the 
% basic 
unit of scheduling (Section~\ref{subsec:granularity}). 
Subsequently, \textbf{\circled{3}} every thread executes operation(s) in the unit of scheduling with various abort handling mechanisms (Section~\ref{subsec:abort_handling}).
Only when state transactions are processed (i.e., committed or aborted) can the associated input events be postprocessed (Line 8) by the \emph{StreamManager} based on transaction processing results.
\end{comment}

\begin{comment}
\begin{algorithm}
\footnotesize
    \KwData{$e$ \tcp{Input event}}
    \KwData{$txn_{ts}$ \tcp{State transaction}}
    \KwData{$G$ \tcp{The currently constructed TPG}}
    \While{!finish processing of input streams}{
        \eIf(\tcp*[h]{Phase 1}){\text{$e$ is not a $punctuation$}}{
                $txn_{ts}$ $\gets$ PRE\_Processing($e$)\;
                \textbf{TPG\_Construction}($G$, $txn_{ts}$)\; 
          }(\tcp*[h]{Phase 2}){
                \textbf{TPG\_Refinement}($G$)\; 
                \textbf{TXN\_Scheduling}($G$)\; 
                POST\_Processing()\;
          }
    }
    
    \SetKwFunction{FMain}{TPG\_Construction}
    \SetKwProg{Fn}{Function}{:}{}
    \Fn{\FMain{$G$, $txn_{ts}$}}{
        $O_{1..k}$ $\gets$ \textbf{Partition} $txn_{ts}$\;
        \ForEach{\text{operation $O_{i}$ $\in$ $O_{1..k}$}}{
            \textbf{Identify} its \ld\;
            $G$ $\gets$ $G$ + $O_{i}$ \;
        }
    }
    \SetKwFunction{FMain}{TPG\_Refinement}
    \SetKwProg{Fn}{Function}{:}{}
    \Fn{\FMain{$G$}}{
        \ForEach{\text{vertex $e_{i}$ $\in$ $G$}}{
            \textbf{Identify} its \td, \pd\;
        }
    }
    
    \SetKwFunction{FMain}{TXN\_Scheduling}
    \SetKwProg{Fn}{Function}{:}{}
    \Fn{\FMain{$G$}}{
        $M$ $\gets$ Instantiated with $G$;\tcp{A decision model}
        \While{!finish scheduling of $G$
        }{
          \textbf{\circled{2}} $Scheduling Unit$ $\gets$ \textbf{\circled{1}} \emph{Explore}($G$, $M$)\; 
            \textbf{\circled{3}} \emph{Execute with Abort Handling} ($Scheduling Unit$)\; 
        }
    }
  \caption{Execution Outline of \system}
  \label{alg:algo}
\end{algorithm}
\end{comment}


\end{document}
\endinput
%%
