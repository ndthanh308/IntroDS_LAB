\subsection{Scope of Evaluation}
\label{sec:domains}

To conduct a holistic survey of \gpt-assisted \cdam, our experiments span a number of different 
design domains (\sref{sec:overview_domains_design}), 
performance metrics (\sref{sec:overview_domains_perf}) and 
manufacturing methods (\sref{sec:overview_domains_manufacturing}).
Here, we briefly describe each domain of interest, along with the specific challenges they pose and the sort of representative, transferable insight we hope to glean by studying each domain in connection with \llms. 

\subsubsection{Target Design Domains} 
%\team{Liane}
\label{sec:overview_domains_design}
Our experiments are concentrated in three main design domains, including 2d vector graphic design, 3D parametric modeling, and articulated robotics problems.  

% \paragraph{Vector Graphics} 
%\team{Liane, Mike}
Vector graphics use a series of text-based commands to represent paths and areas that form a given design.
Vector image formats are an important part of \cdam, as they can be used as both a design specification and a manufacturing specification for \eg laser cutters. 
Despite their simplicity, vector graphics can represent a wide range of 2D and 3D objects, such as artistic engravings or flat-pack furniture.
We examine \llms' capacity to generate two popular vector formats: SVG and DFX.
These formats present several challenges: 
they contain boilerplate formatting that \gpt may struggle to reproduce; 
it may be difficult to layout individual pieces on the canvas; and finally, it may be difficult to decompose higher-dimensional designs into 2d.
Thus, vector graphics will test \gpt's spatial reasoning and ability to respect highly-constrained syntax, either on its own or with the use of external libraries.


% \paragraph{3D Parametric Geometry} 
%\team{Liane, Felix}
Parametric modeling languages generate 3D geometry through a sequence of constructive instructions. The term ``parametric modeling'' reflects how each constructive operator exposes a set of parameters, such as the radius of a circle.
%
We explore two distinct approaches that are powerful, widely-used, and well-documented online.
The first is rooted in classic Constructive Solid Geometry (CSG), which constructs shapes by successively deploying boolean operations (union, intersection, subtraction) over basic shapes or primitives (such as cuboids, spheres, cylinders, and so forth) that can undergo transformations such as translations, rotations, and scaling.
The CSG approach is intended to test the \textit{global} spatial reasoning capacity of \gpt, as every CSG operation/transformation occurs \wrt a shared coordinate space.
The second representation relies on the contemporary B-rep format used by modern CAD systems.
Here, geometry is built through a sequence of operations like sketching, extruding, and filleting. 
Each operation in this context is parametric and uses references to previously created geometry to \eg, select a plane for a sketch design or select a sketch for an extrusion. 
Sketch-based CAD will test \gpt's ability to effectively switch between and reason over multiple relative, local coordinate frames.


% \paragraph{Robotics} 
%\team{Wil, maybe with input from Allan/Andy/Megan/others}

Robotics offers a particularly rich design domain, as \gpt must coordinate a set of \textit{articulated} and \textit{actuated} geometries to form complex objects such as open chain robot arms, wheeled robots, copters/UAVs, and robot grippers.
Robotics representations must describe not only the high-level geometry of each part, but also their properties and relationships -- including the joints between parts, the degrees of freedom that those joints exhibit, and dynamics information such as the inertia of a given part. 
Several existing formats support these tasks, but we primarily use the XML-based language known as the Universal Robot Description Format (URDF). 
We also investigate the use of a more general graph-based robot representation.
These formats test \gpt's ability to simultaneously reason about multiple aspects of design, such as static geometric bodies and dynamic articulation constraints.



\subsubsection{Target Performance Domains} 
\label{sec:overview_domains_perf}
% \team{Crystal, maybe support from Andy + others?}
% brief overview of what kind of things to expect in the performance exploration section(s)
Diverse performance domains within engineering design require evaluation of aspects such as structural and material properties, mechanical integrity, geometry-based functionality, materials use, electromechanical integration, and subjective features. The results of such evaluation allow us to (dis)qualify a design for use and to use the evaluation to further understand and improve the design. Using \gpt, we focus on assessing mechanical and structural properties through generating first-order analysis equations for input designs of standard objects like chairs, cabinets, and a quadcopter, which test the ability of \gpt to sufficiently understand a given input design in text form or through a DSL and to evaluate criteria for functionality and failure. Mechanical properties assessed include weight, size, load capacity, storage capacity, and stability. Analysis of electromechanical functionality include battery life and quadcopter travel distance. Further use of \gpt aims to streamline the computationally intensive process of Finite Element Analysis (FEA), a crucial tool for understanding structural behavior in detail under various conditions, and we apply this to the case of a load on a set of chairs. 

In addition to these technical aspects, our investigation extends into the subjective domains of sustainability and aesthetics, which cannot be strictly quantified. The inherent complexity and qualitative nature of these areas present unique challenges in evaluation. While it is well-known that computational systems can compute quantitative features, machine learning systems are becoming more sophisticated in artistic domains, and so we seek to leverage the capacity of \llms for lexical analysis to aid more holistically in the more ambiguous realms of the design process and to find its limits. For example, could an \llm reasonably address whether a piece of furniture of a given size is ``large'', or if a shoe of a given design is ``comfortable,'' or can it only handle classically quantifiable features? Can it even help us to reason more objectively about what aspects delineate these properties? To this end, we test evaluation of subjective domains and use \gpt to generate a scoring system and functions for quantifying the sustainability of a chair, the classification of chairs based on categories of aesthetic influence, and the appropriate distribution of a set of chairs into a set of rooms in a house, among other examples. %\liane{I think this is the only place where we give an explicit preview of the findings (eg, gpt does well/poorly) rather than just the questions we're asking. Maybe defer the findings? } \amy{what's our motivation for asking if gpt-4 can help with criteria like aesthetics? is that something that we wish was computable? maybe need a few more words here since it's not us simply trying to replicate existing parts of the computational design workflow, (and it seems like our other goals have been oriented towards that). I think most people would accept that the onus is on a human to pass judgement on things like aesthetics.}

We further combine these performance metric evaluations with the principles of inverse design. Inverse design entails setting desired performance attributes and employing computational methodologies to deduce design parameters that satisfy these attributes, both by generating areas for improvement within a design domain and by testing the effects of implementing improvements suggested by \gpt or target design goals of our own interest,
%implementation\amy{I don't understand what either of these mean in this context... what is "their implmentation", how do you "implement conceptual areas of improvement"? and areas of improvement only seem to apply to an existing design -- should it be iteratively generating areas of improvement?}
as well as selecting appropriate methods of optimization. In this case, given a design/decision space for an object, we use \gpt to generate and implement methods to computationally improve or optimize qualifying designs to satisfy designated performance goals. This methodical approach evaluates if \llms can apply constructive logic for design enhancement and innovation.

\subsubsection{Target Manufacturing Domains} 
\label{sec:overview_domains_manufacturing}

 Leveraging language models like \gpt in DfM context can yield more consistent and scalable decision-making, potentially augmenting human expertise and reducing our reliance on CAD software usage. Potential applications of \gpt include the selection of optimal manufacturing techniques, suggestion of design modifications that would enable easier production, identification of potential suppliers, and creation of manufacturing instructions. The approach is aimed to alleviate many of the bottlenecks caused by the designers' lack of knowledge and experience in DfM. 

In a set of experiments, we've explored \gpt's capabilities across various tasks. Firstly, \gpt was used to identify the optimal manufacturing process for a given part, considering factors such as part geometry, material, production volume, and tolerance requirements. Next, \gpt was tasked with optimizing a component design for CNC machining. Given the geometry of the component, \gpt identified potential manufacturing difficulties and modified the design to address these. We also leveraged \gpt's extensive dataset knowledge to identify parts needed for manufacturing.

In addition to these, \gpt was used to create manufacturing instructions for both additive and subtractive design processes. Additive design can be challenging due to the need for spatial reasoning, precision, and meticulous planning, and often requires many iterations. We've explored the generation of fabrication instructions using subtractive manufacturing techniques for a cabinet design. We also investigated \gpt's potential in generating machine-readable instructions for robot assembly tasks and converting those into human-readable standard operating procedures. This allowed for effective communication and collaboration between robots and human operators.


