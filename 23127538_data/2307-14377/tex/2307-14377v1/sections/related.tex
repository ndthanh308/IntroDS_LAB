\section{Background \& Related Work}
To contextualize our work, we briefly describe the state of the art for generative \llms and various aspects of \cdam. 

\subsection{LLMs for Generative Modeling}
\label{sec:related_llm}
% Images, text, sound, motion, code, 3D models
% Different models throughout the years (GPT-2 through 4), (LLaMa), Falcon
% RLHF (ChatGPT, Stability.AI's open-source software)
% Domain-specific models - PaLM, coding, translation
Large Language Models (\llms) have garnered significant interest in the research community and beyond, as a result of both their already-demonstrated generative capabilities and their seemingly unbounded promise. 
Although these models are recognized primarily for their influence on text generation \cite{radford2019language}, their reach has been extended to impact various other domains, including image generation \cite{ramesh2021zero}, music generation \cite{dhariwal2020jukebox}, motion generation \cite{jiang2023motiongpt}, code generation \cite{chen2021evaluating}, 3D model creation \cite{liu2023zero}, and robotic control \cite{mirchandani2023large}. Notable foundational models include OpenAI's GPT series, ranging from GPT-2 to the more recent GPT-4 \cite{openai2023gpt4}. These models have showcased progressive improvements in fluency, coherence, and generalization capabilities. Meta AI's LLaMa model has further extended the reach of LLMs by demonstrating proficiency in both text and image synthesis \cite{touvron2023llama}. The Falcon LLM \cite{penedo2023refinedweb}, trained exclusively on properly filtered and deduplicated web data, has exhibited comparable performance to models trained on meticulously curated datasets. These models have been utilized in conjunction with Reinforcement Learning from Human Feedback (RLHF) to improve the quality of the generated content \cite{ouyang2022training}. This is done by incorporating human feedback into the training process, where humans rate the quality of the generated outputs and provide examples of ideal outputs for a given input \cite{christiano2017deep}. In parallel, domain-specific LLMs have also been trained for performance within a specific subject area. For example, ProtGPT2 specializes in predicting protein folding structures \cite{ferruz2022protgpt2}, while Codex has been specifically tailored to understand and generate code \cite{chen2021evaluating}. 
In this work, we investigate the generative capabilities of generic, pre-trained \llms within \cdam.






\subsection{Computational Design and Manufacturing}
\label{sec:related_cdam}
The \cdam workflow is often decomposed into a series of steps including (1) representing a design, (2) representing and exploring a design space, (3) preparing a design for manufacturing, (4) computing the performance of a design, and (5) finding a design with optimal performance.
For each phase, we provide a brief overview of the relevant work, with a focus on aspects that offer the best opportunities for \llm integration.

\paragraph{Design Representations.} 
The cornerstone of computational design is the capacity to digitally represent and manipulate the salient aspects of a given design -- such as geometry, articulated joints, material composition, \etc. 
There are many ways to represent such aspects, but we focus on focus on design representations that are compact, understandable, and editable. 
For example, modern CAD systems represent a shape as a sequence of operations such as 2D sketches, extrusions and Boolean operations~\cite{willis2021fusion}. 
These can be represented as compact programs written in domain specific languages (DSLs) such as OnShape's FeatureScript~\cite{featurescript}. 
Designs can also be represented compactly as a graph~\cite{prusinkiewicz2012algorithmic, zhang2018graphit}, in which the nodes typically represent individual components, while edges represent multi-component interactions. 
Such graphs have been used to efficiently and hierarchically represent CAD models~\cite{du2018inversecsg}, robots~\cite{zhao2020robogrammar}, metamaterials~\cite{makatura2023procedural}, architecture~\cite{muller2006procedural}, and chemical molecules~\cite{guo2022data}. 
To represent even more complex designs -- such as a quadcopter with a physical design and a software controller -- multiple DSLs may be used simultaneously.
For example, the copter's physical design may be encoded using CAD, while its software is coded using a control-specific DSL.   


\paragraph{Design Space Representations} A design space represents an entire family of designs -- rather than a single instantiation -- which allows for design exploration, customization, and performance-driven design optimization. 
One of the most popular design space representations is parametric design, in which  a few exposed parameters are used to control a design.
This is commonly used in CAD systems, where \eg a bookshelf may be parametrized by its height, width, depth, and number of shelves.
Another popular option is formal languages such as L-systems~\cite{rozenberg1980mathematical} or shape-grammars~\cite{stiny1980introduction,ozkar2009shape}, which generate design variations by manipulating a set of terminal and non-terminal symbols according to given rewrite rules. 
Formal languages have been used in domains such as architecture~\cite{muller2006procedural}, robotics~\cite{zhao2020robogrammar}, and chemistry~\cite{guo2022data}.




\paragraph{Design for Manufacturing} 
Design for Manufacturing (DfM) is a planning process used to generate designs that can be fabricated with maximal efficiency and minimal cost.
One prominent aspect of this is Computer-Aided Manufacturing (CAM), which transforms a digital design into a viable fabrication plan for some manufacturing process, such as 3D printing, 3- or 5-axis CNC milling, or sheet-metal stretching. 
CAM also extends to multi-process representations such as STEP-NC, which abstracts away from machine-specific G-code in favor of tool-type-specific machining operations that are interpretable on different hardware.
Because all of these fabrication plans can also be described as a program in some DSL,
%, such as a sequence of steps describing an assembly process, or G-code instructions for CNC milling, laser cutting, or 3d printing.
CAM can be interpreted as a \textit{translation} from a design DSL to a manufacturing-oriented DSL.
DfM also includes many other aspects, such as selecting an appropriate manufacturing method, optimizing manufacturing process parameters \cite{erps2021processOpt}, sourcing parts and materials, or modifying a design in light of manufacturing constraints \cite{Koo2017zeroWaste}.
% Our work considers all of these aspects.



\paragraph{Performance Prediction} Before manufacturing a design, engineers typically want to understand its predicted performance.
For example, automobile engineers may wish to evaluate and iteratively refine a candidate design's efficiency, safety, and aesthetics.
To do this, engineers frequently make use of numerical simulation methods such as general-purpose finite element analysis (FEA)~\cite{du2021diffpd} or more domain-specific approaches for \eg acoustics~\cite{o2002synthesizing}, robotics~\cite{erez2015simulation}, and electromagnetism~\cite{sullivan2013electromagnetic}. 
Commercial CAD systems (e.g., Autodesk~\cite{AutodeskSimulation} and Dassault Systèmes~\cite{DassaultSimulation}) integrate simulation into their ecosystem. 
Since engineers are primarily interested in the performance of the design's manufactured counterpart, it is crucial to minimize the gap between an object's performance in simulation versus reality.


\paragraph{Performance Optimization:} Given a design space and a way to predict performance, it is natural to seek designs that perform best with respect to a particular metric. 
Although this search could be performed via manual trial and error, it is more efficient and effective to use automated exploration tools.
One process known as \textit{inverse design} can automatically search (or optimize) over a given design space to find a design that exhibits some target performance~\cite{ma2021diffaqua}. 
Inverse design has already been applied to many problem domains. 
For example, a parametric design space can be searched for designs that have the best value of a simulated metric~\cite{xu2021end}. 
Topology optimization has been applied to problems such as minimum compliance. 
In addition, designs can be optimized for metrics such as weight, cost, and manufacturing time.



