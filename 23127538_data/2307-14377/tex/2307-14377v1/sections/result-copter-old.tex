



\liane{this section feels a bit out of order at the moment -- by the end of 9.2.2 we have a fabbed frame; then 9.2.3 talks about sim without making any link back to how/whether the sim is used for the real world example; then 9.2.4 talks about selecting the components, around which the frame from earlier was designed. I think reordering this [e.g., sourcing --> generic frame design --> (sim?) --> manufacturing considerations --> real object] could make it a lot less confusing. }

\subsubsection{Text-to-Design}
\label{sec:copter-text-to-design}

We use \gpt to generate a javascript program using our custom defined box function and cylinder function, as described in Section~\ref{sec:prefabbed-ele}. The implementation of the box function is based on \jscad, so we can use the online \jscad viewer to directly export an STL file for later manufacturing. The details are explained in Section~\ref{sec:prefabbed-ele}.

\adriana{what about component selection here?}

\subsubsection{Design-to-Manufacturing}
\label{sec:copter-manufacturing}

% Figure environment removed

Our approach is streamlined by using an STL file described in Section~\ref{sec:copter-text-to-design}. 
As mentioned earlier, the design uses several existing parts, and it is difficult to describe the geometric details of each part.
Moreover, \gpt does not understand how to make the frame such that the quadcopter is physically balanced.
Thus, we do not directly ask \gpt to create a manufacturable frame that all parts can be mounted onto, but instead we issue low-level instructions to guide \gpt to 
generate a valid frame.
%We follow the steps in Section~\ref{sec:additive} to convert our language-based design to an STL that can quantitatively describe the geometry of the designed copter's frame.
%\FH{what do you mean by quantitatively? Doesn't it just describe the geometry?}

We first adjust the size of the frame bar cross section through \gpt
so it is 3D printable and sufficiently stiff.
Then, we union all the boxes and cylinders that belong to the frame.
The battery is placed such that half of it is sunken into the frame, 
and then we subtract it from the frame.
This gives the final frame.
Next, we consider how the remaining parts will be mounted onto the frame.
The battery does not require any further modification to ensure a secure attachment, because the frame will hold the battery tightly by construction.
For the controller and the signal receiver,
they can be glued to the battery, so no additional handling is needed as well.
Finally, for the motors, they are mounted using screws.
so there should be screw holes on the frame.
We would like to use \gpt to make holes
and see how far it can proceed to solve this problem.
For each motor, we ask \gpt to generate 4 cylinders that represent the needed holes.
We use text to describe the specification of the holes to \gpt.
Although the mounting holes are not trivial,
we were still able to get the correct cylinders after a few iterations with \gpt.
They are shown in dark gray in Figure~\ref{fig:quadcopter-man}(Left).
Once the hole cylinders for one motor is ready,
we let \gpt group them and duplicate them using our \lstinline{place} function.
This again poses difficult mathematical problems to \gpt.
We were able to get the hole cylinders to be placed into the correct position, 
but we couldn't achieve the correct rotation.
This is later fixed by manually adjusting the angle.
As we can see, the hole cylinders are overlapping with the frame bar.
Because we cannot change the thickness of the frame bar, 
we have to adjust the thickness of the tip of the frame bar 
so that it does not block the holes.
This part is also achieved manually 
because we found it difficult to describe the problem and the solution to \gpt.

Another limitation we found during our experiments is that 
it is difficult to tweak the designs we finished earlier.
After a long conversation with \gpt, it becomes difficult to communicate with it
about the topics we have designed earlier.
As a result, if we find any issues in the part we addressed earlier,
it is hard to bring it up again to \gpt and let it make changes.
In other words, it would be ideal if one could finish any design without the need to re-touch it at any future point.
Otherwise, it would be challenging to re-adjust it later.
The final result of the frame is shown in Figure~\ref{fig:quadcopter-man}(Right).
Overall, with the aid of a user, we can design a manufacturable quadcopter through \gpt.  \adriana{good point we don't talk so much about iterations, probably worth bringing it to the broader discussion. }

\adriana{everything up until here sounded like design, I don't understand why it's in the fabrication section and how it realtes to the design section above.}

Once the frame is determined, we use a 3D printer to create the frame of the copter. The printer is a Fortus 400, a model renowned for its high-performance and precision manufacturing capabilities, making it ideally suited to the task at hand. Its substantial build volume, one of its standout features, is especially advantageous for larger projects like the copter. This is crucial given the sizeable dimensions of the copter, which could potentially outsize the build volume of many other printer models.

The Fortus 400 is capable of Fused Deposition Modeling (FDM) technology, a popular form of additive manufacturing where layers of heated and extruded thermoplastic are successively laid down according to the design's geometry. This makes it an excellent choice for creating robust, durable parts like the quadcopter frame we intend to manufacture.
\FH{The way you describe the 3D printer makes it sound like you are paid by Fortus. I don't think you have to justify why you chose a particular 3D printer, just say we print it.}
\liane{seconded; I'd probably summarize these two paragraphs by saying "Once the frame is determined, we fabricate it using a 3D printer. We use the Fortus 400, because it has a sufficiently large build volume and it realizes precise, robust, and durable parts.}

We instruct the printer to use the least amount of infill possible in the print settings. By choosing a lower infill percentage, the printer will create a sparse or hollow internal structure rather than a solid one. This decision helps conserve material and reduce print time without significantly compromising the strength or weight of the copter's frame, given that it is not a load-bearing component.

% By leveraging the advanced capabilities of the Fortus 400 3D printer and the efficiency of the infill pattern, we are able to produce a high-quality copter frame that
The resulting fabricated copter frame
not only meets the required dimensions, but also balances the strength and weight, necessary for optimum flight performance. We visualize the printed frame in Figure~\ref{fig:copter-frame}.

% Figure environment removed


\subsubsection{Text-to-Performance}

% Leader: Megan, Edward, Andy

We use \gpt to measure the quadcopter's performance through simulation. As described in Section~\ref{sec:quadcopter_sim}, this first involves converting a given \jscad design to a format amenable to simulating multibody robots, such as URDF. We then prompt \gpt to provide a LQR controller which takes in both the design in URDF format and relevant physical properties. We also describe in Section~\ref{sec:quadcopter_sim} how the quadcopter's design can be optimized through simulation. \liane{do you use these results for the fabricated example? right now feels a bit whiplash-y to go to from fab back to simulation with no tie-in for how/if the sim is applied to the real world example}


\subsubsection{Real-world example}
\liane{this header seems weird (in the cabinet section too). maybe fabricated example? [coordinate with cabinet folks to pick a consistent title]}

%We follow the operation from the previous sections and ask \gpt for the best out-of-shelf components we can get from the market.

The previous examples evaluate the text-to-design capabilities of \gpt in a CAD-like setting by constructing objects from basic primitives. However, many real-world systems such as drones are constructed from pre-designed components with varying specifications. This next example examines the ability of \gpt to select real-world components for a specific use case: constructing an indoor drone capable of a given speed.
\liane{I don't think you need to summarize/motivate here -- we've already done that in the paper. I'd just say "Now we want to create a real-world drone capable of achieving a given speed"}



\begin{chat}{false}

\begin{userprompt}{0\textwidth}{}
Which components do I need to build an indoor drone that can fly at 30 mph? Prefer easily available components if possible. List the manufacturer, model, quantity, and Amazon search term for each component.

Next, write a Python function that constructs the drone. The following functions are available to use:

\begin{gptcodeblock}{Python}
add_cylindrical_component(manufacturer: str, model: str, mass_grams: float, diameter_mm: float, height_mm: float, parent: int) -> int: Adds an instance of a cylindrical component and returns its ID. The new instance is attached to the instance with ID `parent`.
add_cuboid_component(manufacturer: str, model: str, mass_grams: float, length: float, width_mm: float, height_mm: float, parent: int) -> int: Adds an instance of a cuboid component and returns its ID. The new instance is attached to the instance with ID `parent`.
\end{gptcodeblock}
\end{userprompt}

\begin{gptresponse}{0\textwidth}{}
Building an indoor drone that can fly at 30 mph requires several components. Here's a list of the necessary components along with their specifications and Amazon search terms:

\begin{enumerate}
    \item Frame:
    \begin{itemize}
        \item Manufacturer: Hobbypower
        \item Model: Hobbypower F450
        \item Quantity: 1
        \item Amazon search term: Hobbypower F450 frame kit
    \end{itemize}

    \item Flight Controller:
    \begin{itemize}
        \item Manufacturer: Holybro
        \item Model: Kakute F7 Mini
        \item Quantity: 1
        \item Amazon search term: Holybro Kakute F7 Mini flight controller
    \end{itemize}

    \item Motors:
    \begin{itemize}
        \item Manufacturer: EMAX
        \item Model: EMAX RS2205S
        \item Quantity: 4
        \item Amazon search term: EMAX RS2205S motors
    \end{itemize}

    \item Electronic Speed Controllers (ESCs):
    \begin{itemize}
        \item Manufacturer: Hobbypower
        \item Model: Hobbypower SimonK 30A
        \item Quantity: 4
        \item Amazon search term: Hobbypower SimonK 30A ESC
    \end{itemize}

    \item Propellers:
    \begin{itemize}
        \item Manufacturer: DALPROP
        \item Model: DALPROP T5045C
        \item Quantity: 4 pairs (8 propellers)
        \item Amazon search term: DALPROP T5045C propellers
    \end{itemize}

    \item Battery:
    \begin{itemize}
        \item Manufacturer: Tattu
        \item Model: Tattu 4S 1300mAh
        \item Quantity: 1
        \item Amazon search term: Tattu 4S 1300mAh LiPo battery
    \end{itemize}

    \item Radio Transmitter and Receiver:
    \begin{itemize}
        \item Manufacturer: FrSky
        \item Model: FrSky Taranis X9D Plus
        \item Quantity: 1
        \item Amazon search term: FrSky Taranis X9D Plus transmitter and receiver
    \end{itemize}

    \item Power Distribution Board (PDB):
    \begin{itemize}
        \item Manufacturer: Matek Systems
        \item Model: Matek PDB-XT60
        \item Quantity: 1
        \item Amazon search term: Matek PDB-XT60 power distribution board
    \end{itemize}

    \item Propeller Guards (optional but recommended for indoor use):
    \begin{itemize}
        \item Manufacturer: Crazepony
        \item Model: Crazepony 2 Sets 65mm
        \item Quantity: 1 set
        \item Amazon search term: Crazepony 65mm propeller guards
    \end{itemize}
\end{enumerate}

Now, let's write the Python function to construct the drone using the provided functions:

\begin{gptcodeblock}{Python}
def construct_indoor_drone():
    # Add frame
    frame_id = add_cuboid_component("Hobbypower", "Hobbypower F450", 300, 450, 55, 55, 0)
    
    # Add flight controller
    fc_id = add_cuboid_component("Holybro", "Kakute F7 Mini", 15, 35, 35, 10, frame_id)
    
    # Add motors
    motor_ids = []
    for _ in range(4):
        motor_ids.append(add_cylindrical_component("EMAX", "EMAX RS2205S", 30, 28, 17.5, frame_id))
    
    # Add ESCs
    esc_ids = []
    for _ in range(4):
        esc_ids.append(add_cylindrical_component("Hobbypower", "Hobbypower SimonK 30A", 15, 22, 6, frame_id))
    
    # Add propellers
    propeller_ids = []
    for _ in range(4):
        propeller_ids.append(add_cylindrical_component("DALPROP", "DALPROP T5045C", 5,
\end{gptcodeblock}
\end{gptresponse}

\begin{userprompt}{0\textwidth}{}
I already have a radio transmitter, please suggest a receiver only
\end{userprompt}

\begin{gptresponse}{0\textwidth}{}
Certainly! If you already have a radio transmitter and only need a receiver, you can consider the following option:

\begin{enumerate}
    \item Receiver:
    \begin{itemize}
        \item Manufacturer: FrSky
        \item Model: FrSky XSR
        \item Quantity: 1
        \item Amazon search term: FrSky XSR receiver
    \end{itemize}
\end{enumerate}

Please note that the FrSky XSR receiver is compatible with the FrSky Taranis X9D Plus transmitter, which you mentioned having.
\end{gptresponse}

\end{chat}
\FH{Please fix the page-break problems with this chat. You can break it up into multiple smaller chats and also only show relevant information.}
\liane{isn't a more space-efficient version of this exact conversation already included in the earlier sections? \fref{fig:copter-parts}. Could you just reference that?}
\liane{I see there's they python block which isn't in the original; if you really want this block, you can just use the omitted command to comment out the part list/reference the other fig and then include the python.}


The most common failure cases include suggesting nonexistent components, placeholder components (e.g. ``XYZ Motor''), and unnecessary components (e.g. ESCs \liane{I didn't know what an ESC was -- maybe define it once in the main text too, not just in the chat snippet} and power distribution board for an all-in-one controller). We found that requesting search terms for a specific online retailer \liane{such as?} resulted in fewer instances of placeholder components. The chosen components are generally electrically compatible, with the ESCs supporting the series cell count of the battery (e.g. 2S, 3S, 4S). Mechanical incompatibility is more of an issue, with propellers and motors occasionally having different shaft sizes. This is generally not an issue, however, since propeller manufacturers tend to include adapter rings with their products.\liane{nitpick, but weird phrasing with "mechanical incompatibility is more of an issue, ... This is generally not an issue"}

One significant strength of \gpt in this text-to-design scenario is that it can tailor the component selection to the use case. For example, specifying a ten-kilogram payload requirement will result in \gpt suggesting much larger motors and propellers. As demonstrated in the transcript \liane{which transcript?}, further interactions with \gpt can be useful for finding alternate components.

After verifying that the components are indeed compatible, we purchased them from an online retailer. Note that we omit the propeller guards and frame (which we manufacture using 3D printing) as well as the ESCs and power distribution board, which are included in the controller. We show the purchased components in Figure~\ref{fig:purchased}.

% Figure environment removed




\liane{I would split the section here. Take all the \gpt-assisted part sourcing info and start the section with that; then talk about the frame design  + (maybe sim if relevant?) + frame manufacturing; then talk about assembly + testing }

After the 3D printed frame is prepared following Section~\ref{sec:copter-manufacturing}, the next step is the assembly of the copter with the mechanical components that have been prepared before. These components form the backbone of the copter's functionality and are crucial to its flight performance.

\FH{Do we need the explanation of the assembly? What is our message here? Does this have to do anything with the \gpt exploration?}\liane{I think it makes sense if focused on the interface between the \gpt-designed frame and the pre-fabbed parts. Not sure if we need the explanation of what each part does, but it does help justify the attachment mechanism some. The description is nice and I learned a lot from it, though -- I'd probably at least keep in an appendix if it gets nixed from the main text}
First, we have the flight controller, which serves as the brain of the copter. It integrates all sensor data, interprets inputs from the radio receiver, and makes necessary adjustments to the motors' power for maintaining stability and control during flight. This component is secured onto the copter frame using double-sided tape and wrappers to ensure a firm and secure fit, preventing it from moving or detaching during operation.

Next, we have 4 motors, which provide the power needed for the copter's propulsion and maneuverability. They are attached at each corner of the copter frame, maintaining an equidistant configuration for balanced performance. Attachment is done using screws, which provide a sturdy and reliable fastening method, ensuring that the motors stay firmly in place even under the high-speed rotation and vibrational forces experienced during flight.

Each motor is paired with a propeller, which converts the motor's rotary motion into lift, enabling the copter to ascend and descend. These are also affixed using screws, ensuring a secure fit to handle the high rotational speeds without risk of detachment or imbalance.

The battery, which provides the energy needed for the flight controller, motors, and other electrical components, is also fastened onto the frame using double-sided tape and wrappers. This positioning method allows for secure placement while also enabling easy access for battery replacement or recharging.

Finally, we have the radio receiver, which receives input signals from the remote control transmitter. It communicates these commands to the flight controller, initiating changes in flight direction, speed, and altitude. Like the flight controller and the battery, the radio receiver is attached to the frame using double-sided tape and wrappers, ensuring that it remains secure and stable during flight while still being easily accessible for maintenance or troubleshooting.

Each component plays a pivotal role in the operation of the copter, and the assembly process involves meticulous placement and secure attachment of each part. All elements are firmly and stably attached as shown in Figure~\ref{fig:copter-assemble}, creating a sturdy and reliable copter ready for flight.

% Figure environment removed

The final phase is a crucial real-world evaluation of the newly-assembled copter. This hands-on, operational testing allows us to assess and validate the functionality and performance of our copter under real-world conditions, ensuring it performs as intended.

First, we conduct an ascending test. This involves commanding the copter to lift off the ground and rise to a predetermined altitude. This test serves to evaluate the collective thrust of the motors and the performance of the propellers in translating the motors' rotary motion into lift. It also gives us a chance to observe the responsiveness of the copter to commands from the radio transmitter, the efficacy of the flight controller in interpreting these commands, and the general stability of the copter during ascent. We show the motion in Figure~\ref{fig:copter-flight-test} (left).

Following the ascent, the next stage of evaluation involves a hovering test. In this phase, the copter is commanded to maintain its altitude and position for a set duration. Hovering requires the continuous, simultaneous operation of all four motors to counteract the pull of gravity. This test is particularly revealing of the copter's ability to achieve and sustain stable flight, a critical attribute of any functional copter. We show the motion in Figure~\ref{fig:copter-flight-test} (mid).

Lastly, we conduct a descending test, commanding the copter to safely and smoothly lower itself back to the ground. This tests the copter's controlled reduction of thrust and the resultant downward movement, as well as the flight controller's ability to interpret and execute the descent command. It is also a critical test of the copter's landing capabilities, as a safe and smooth landing is vital to preserving the integrity of the copter and its components. We show the motion in Figure~\ref{fig:copter-flight-test} (right).

% Figure environment removed
