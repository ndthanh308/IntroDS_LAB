% Figure environment removed

In this subsection, our objective is to design a functional indoor quadcopter capable of flight, hovering, and landing, using \gpt with minimal human intervention. The process involves sourcing parts, creating a 3D design incorporating the chosen parts, and ultimately, manufacturing, assembling and testing the quadcopter,
as depicted in Figure~\ref{fig:quadcopter-process}. % \wojciech{Add a figure and short discussion on how the workflow looks like for this exampl}

\subsubsection{Parts Sourcing}
Many real-world systems, like drones, are built from pre-existing components with various kinds of  specifications. Therefore, our first task is to utilize \gpt to select appropriate parts for our specific use case. This was successfully accomplished using \gpt. The detailed process is elaborated in Section~\ref{sec:part_sourcing}, and the selected parts are shown in Figure~\ref{fig:copter-assemble} (left).

\subsubsection{Text-to-Design}
\label{sec:copter-text-to-design}

% Figure environment removed

Upon identifying the components, we employ \gpt once more to generate a viable quadcopter design incorporating those parts. With minimal human intervention, we successfully crafted a geometric design for the quadcopter, as detailed in Section~\ref{sec:prefabbed-ele}. Now, we shift our focus to practical issues: 1) how parts are mounted onto the designed frame, and 2) whether the frame is manufacturable. Given that describing each part's geometric details is challenging and that \gpt doesn't fully comprehend how to design the frame for optimal physical balance, we provide \gpt with low-level instructions to guide adjustments to the current frame design rather than expecting it to independently modify the design.

First, we adjust the frame bar's cross-sectional size using \gpt to ensure it's 3D-printable and adequately robust. Then, we combine all boxes and cylinders that form the frame in the geometric design. To stabilize the battery placement, we semi-integrate it into the frame, and subtract it from the frame so the frame securely holds the battery. For the controller and signal receiver, which are much lighter and smaller than the battery, we simply glue them onto the battery, eliminating the need for additional accommodations.

Lastly, we mount the motors using screws for stability, requiring screw holes in the frame. To minimize human effort, we utilize \gpt to create holes in the frame. For each motor, we instruct \gpt to generate four cylinders representing the required holes, detailing the hole specifications via text according to the motor's specifications. Even though crafting mounting holes isn't trivial, we successfully produced the correct cylinders after a few prompts with \gpt. The cylinders are shown in dark gray in Figure~\ref{fig:quadcopter-man}(Left). Once the hole cylinders for one motor are ready, we let \gpt group and duplicate them using our \lstinline{place} function. \gpt managed to position the hole cylinders correctly but had issues with proper rotation, which we later manually corrected.

As seen, the hole cylinders overlap with the frame bar. Since we cannot change the frame bar's thickness due to the manufacturing concern, we manually adjusted the frame bar's tip thickness to prevent it from obstructing the holes. This was done manually, as it proved challenging to adequately convey the problem and solution to \gpt.

Our experiments also revealed a limitation: adjusting designs we completed earlier proves difficult. After extensive interaction with \gpt, referring back to previously discussed design elements becomes a challenge. Consequently, if any issues arise with earlier addressed parts, it becomes arduous to revisit them with \gpt and prompt modifications. Ideally, we should finalize each design without the need for future revisions, as adjustments later prove difficult. The final frame result is displayed in Figure~\ref{fig:quadcopter-man}(Right).

\subsubsection{Design-to-Manufacturing}
The only part that needs manufacturing is the frame. Once the frame is determined, we fabricate it using a 3D printer. Because the representation of the frame results from Boolean operations of boxes and cylinders, it is simple to directly convert them to .stl format which is widely recognized by the 3D printers. We used OpenJSCAD to do it. Once we have the .stl file, we manufacture the frame using Stratasys Fortus 400 since it has a sufficiently large build volume and it realizes precise, robust, and durable parts. We instruct the printer to use the least amount of infill possible in the print settings. By choosing a lower infill percentage, the printer will create a sparse or hollow internal structure rather than a solid one. This decision helps conserve material and reduce print time without significantly compromising the strength or weight of the copter's frame, given that it is not a load-bearing component. The resulting fabricated copter frame not only meets the required dimensions, but also balances the strength and weight, necessary for optimum flight performance. We visualize the printed frame in Figure~\ref{fig:copter-assemble} (middle).

% Figure environment removed

\subsubsection{Assembling and Real-World Verification}

With the 3D printed frame ready, we proceed to the assembly stage, integrating the pre-prepared components. Given that assembly considerations were incorporated into our \gpt-guided design process, the assembly of the quadcopter is straightforward. The battery is secured in the central frame slot using double-sided tape and wrappers. Similarly, the controller and receiver are placed atop the battery and secured with double-sided tape and wrappers. The four motors are attached using screws. All elements are affixed firmly and stably, resulting in a sturdy copter ready for flight, as shown in Figure~\ref{fig:copter-assemble} (right).

Once assembled, we conduct a series of tests. First, we administer an ascending test, directing the copter to lift off the ground and ascend to a specific altitude. This test gauges the combined thrust of the motors and the propellers' efficacy in converting the motors' rotary motion into lift. It also allows us to evaluate the copter's responsiveness to radio transmitter commands, the flight controller's interpretive capacity, and the copter's overall ascent stability. The motion is depicted in Figure~\ref{fig:copter-flight-test} (left).

Following ascent, we undertake a hovering test. During this phase, the copter is directed to maintain its altitude and position for a set period. Hovering demands continuous, simultaneous operation of all four motors to counter gravity. This test significantly illuminates the copter's capacity to achieve and maintain stable flight, a vital characteristic of any functioning copter. The hovering motion is demonstrated in Figure~\ref{fig:copter-flight-test} (mid).

Finally, we execute a descending test, instructing the copter to safely and gradually descend to the ground. This evaluates the copter's ability to control thrust reduction and the resulting downward motion, as well as the flight controller's capacity to interpret and carry out the descent command. It is also a crucial examination of the copter's landing abilities; a smooth, safe landing is essential to preserve the copter and its components. The descending motion is exhibited in Figure~\ref{fig:copter-flight-test} (right).

% Figure environment removed

\subsubsection{Text-to-Performance}
We also investigate how \gpt can help with measuring the performance of a given quadcopter design.
Given a current design iteration of the copter from Section~\ref{sec:perf_quadcopter}, \gpt is able to identify important trade-offs to optimize and subsequently implement optimization strategies to improve performance. One such trade-off \gpt identified is between weight and size, where smaller copters are generally able to stay afloat longer due to reduced weight and aerodynamic drag, while larger copters have more space to accommodate larger batteries which can provide more energy for longer flight times. Out of all the possible optimization methods to find the best combination of parameters that maximize flight time, speed, and distance while meeting constraints on weight and size, \gpt chose a very suitable numerical method of Particle Swarm Optimization (PSO) from the PySwarm library. Aside from being very efficient and simple to implement, PSO has a strong global search capability, which is beneficial when the optimal solution might be located in a large and complex space, and allows for real-time adjusting of the copter's weight and size based on performance data. \gpt has a strong grasp on the inherent trade-offs of such systems, and is capable of generating tailored ideas and feasible solutions to optimize performance.

We now turn to the details of using simulation to evaluate the quadcopter's performance. In the workflow of fabricating a functional robot, simulation is often used for both control and collecting performance metric statistics that can be used for optimization. Since our fabricated robot includes its own controller, we focus on using the robot's performance in simulation for design optimization. Our design space involves both a parameterized quadcopter whose frame bar lengths can vary but is otherwise constrained by the design created in Section \ref{sec:copter-text-to-design} and the controller design. While it is possible to ask \gpt to provide suggestions on the type of controller to apply, we choose to have \gpt generate a LQR controller, which is widely used for UAVs. We break down multicopter optimization into three steps: 1) Given the \jscad design of the quadcopter, convert the design into a format specific to modeling multibody systems, such as URDF, and the means of computing relevant physical properties that inform the controller design, such as the robot's mass. 2) Given the robot's physical properties, generate a LQR controller for simulation. 3) Given  a robot design in URDF, functions for extracting the design's relevant physical properties, and a controller, synthesize an algorithm to optimize the robot's design.
% \wojciech{When do you create design space. No mention,}

\paragraph{Converting \jscad to URDF} We start with the \jscad quadcopter design developed in Section \ref{sec:copter-text-to-design}. Because there is no straightforward equivalent of subtraction and union from \jscad in URDF, we omit the creation of the holes in the motor base and replace the union of motor base parts with placement of individual links while retaining the essence of the original design. We then take an object-oriented approach to having \gpt synthesize the equivalent URDF code. As seen in \ref{fig:tourdf_component}, we prompt \gpt to create a Component class whose instances store the geometry, mass, position, and orientation attributes of the corresponding \jscad primitives. Component instances also have distinct names to represent URDF links. We additionally prompt \gpt to generate helper functions for placing instances with different geometries in Fig. \ref{fig:tourdf_component}. This framework allows \gpt to generate a function that places components in terms of absolute coordinate positions and orientations and to replicate the Python equivalent of the \jscad, as shown in Fig. \ref{fig:tourdf_place}. 
% Each instance corresponds to a URDF, keeps track of the component’s name, mass, geometry, as well as absolute translation and rotation with respect to the origin.
% Figure environment removed

% Figure environment removed

However, one difficulty is that unlike CAD-like designs, formats for representing robots also require relations between components to be represented as joints in order to accurately simulate the robot dynamics. We prepare for this when synthesizing the component placement script by prompting GPT to store the components as a dictionary, which allows easy access to the components. We tackle the challenge of generating the quadcopter's joints by relying on \gpt's knowledge of the spatial relation between components in a quadcopter. After equipping the Component class with a function that sets the parent link, we use this interface to have \gpt synthesize a sequence of robot joints, as shown in Fig. \ref{fig:tourdf_relation}. We find that although \gpt understands certain substructures, such as the fact that the motor is placed on top of the motor base and the propeller is connected to the top of the motor, its initial definition results in an invalid URDF format, as both of the frame bars are root links. We thus explicitly prompt \gpt to choose one of the frame bars as the root link. 

Finally, \gpt is tasked with creating a full URDF file. Because the robot is represented with modular Component instances that contain all relevant information on individual links' mass and geometry as well as relations to parent links, it is relatively straightforward for \gpt to create helper functions that synthesize URDF links and joints. We note that creating a joint is a more involved task, since the link's absolute position and orientation must be converted to relative position and orientation to the parent link. It is necessary to explicitly prompt \gpt to use the appropriate rotation matrices in its calculation; otherwise, it does not appropriately account for how the parent link's rotation affects the child link's relative translation. Because we choose LQR as the robot's controller in simulation, we ask \gpt to compute the full assembly's mass and moment of inertia given the Python code it has generated thus far. It outputs reasonable Python code that computes the assembly's center of mass and uses the parallel axis theorem to combine the moment of inertia of the individual links.

% Figure environment removed

In summary, we find that explicitly prompting \gpt to provide suitable object-oriented representations, such as the Component class, and modularizing the code generation as much as possible, e.g. asking \gpt to synthesize helper functions, placing components, and defining parent-child relations separately are key techniques in successfully converting from \jscad to URDF.


\paragraph{Deriving a LQR Controller for the Quadcopter.}  Multicopter control is an extensively studied problem, for which various algorithms have been proposed such as PID controllers, LQR controllers, and more complex alternatives. We aim to synthesize a LQR controller, as it is not only a popular choice in the literature but also guarantees optimal control when the multicopter dynamics and stable fixed point are known. In particular, we focus on controlling a quadcopter, which is an underactuated system with 6 degrees of freedom but only 4 independent actuators, resulting in nonlinear dynamics even without aerodynamical effects. However, LQR provides optimal control by linearizing the system around a stable fixed point, which then has a closed form solution. 

As LQR is widely used for control, the algorithm is easily accessible through the \lstinline{control} Python library that performs LQR with a single function call. As such, the main challenge in using LQR for quadcopter control is deriving its state space representation, the $A$ and $B$ matrices, used in the linear model of the system as shown in Equation \ref{eq:linear_model}. $x$ is a 12-dof state vector, including coordinate position, linear velocity, roll, yaw, pitch, and angular position. $u$ is the 4-dof control input, typically consisting of the upward thrust and the external torque applied in the xyz directions. Note that $x$ in \ref{eq:linear_model} is the difference between the current state and a user-specified target state. Similarly, $u$ is with respect to the target control input, where the torques are 0 and the thrust balances out the gravitational force acting on the qaudcopter. \gpt is then asked to provide a suitable fixed point at which the quadcopter is stable, and it does provide a correct response.

\begin{equation}
    \dot{x} = Ax + Bu
    \label{eq:linear_model}
\end{equation}

In order to verify \gpt's results, we ask \gpt to first synthesize the full kinematics model before using a a symbolic manipulation library, sympy, to compute the state space matrices. This forces \gpt to produce an interpretable dynamical model that a domain expert can verify, rather than directly outputting the $A$ and $B$ matrices. This proves to be essential as even though \gpt captures the high-level idea of assuming the quadcopter to be a rigid body and applying Newton-Euler equations to describe its acceleration and angular acceleration, it is unable to zero-shot provide the exact model of the system without user feedback. For instance, as shown in Fig. \ref{fig:quadcopter_lqr}, \gpt formulates the correct rotation matrix but does not apply it correctly to convert from inertial frame (control inputs) to the body frame (linear acceleration). Although \gpt can correct its error when given feedback, this type of error is difficult to catch without rigorously checking the correctness of \gpt's calculations and highlights the limitation that in some inverse design domains, a user cannot generate precise outputs from \gpt without the expert knowledge to perform verification. After this fix, the resulting code for deriving the $A$ and $B$ matrices is still incorrect as the simulated quadcopter sinks downward due to a sign error in the equations for acceleration. Similarly, generating the simulation loop in PyBullet requires several rounds of iteration on the initial code. Some of the mistakes are more obvious, such as incorrectly indexing into the control input vector when applying external force and torques, whereas others are more specific to PyBullet's API. For the former, we directly pointed out \gpt's mistake. In the latter case, we find that simply giving \gpt the error message and asking it to fix its code suffices for our problem. 

% \wojciech{explain how you fix it eventually.}

% Figure environment removed
\paragraph{Co-design of Quadcopter's Shape and Control.} Now that the quadcopter design can be outputted to URDF and the LQR controller can be synthesized from a given design, we ask \gpt to optimize the quadcopter design. Much of the design space has been fixed by the fact that most components have unchangeable dimensions. However, as the framebars are not predefined components but rather constructed from carbon-fiber tubes, we focus on optimizing their lengths. The control then follows directly from a given design, as we only need to compute the design's total mass and moment of inertia  to apply the LQR controller derived by \gpt. With the objective of minimizing the number of simulation steps required to reach a goal height of 1m, we prompt \gpt with the full Python script that converts OpenJSCAD to URDF and performs simulation, and we find that it is able to provide an outline of the optimization, consisting of an objective function that involves creating a quadcopter with specified framebar lengths and performing the simulation loop, as shown in Fig. \ref{fig:quadcopter_opt}. \gpt also provides reasonable bounds on the frame bar lengths of (100mm, 500mm) when prompted further. \gpt then provides an outer optimization script that uses the helper objective function. We also explicitly prompt \gpt to complete its outline by updating component creation code to generate frame bars with the correct masses and to change the placement of components dependent on the length of the frame bars, namely the motors, motor bases, and propellers. 

% \wojciech{Explain how you construct the design space. The title of this section is weird. you are doing inverse design. You are also co-designing shape and control.} 
% Figure environment removed

As seen in Fig. \ref{fig:quadcopter_opt}, \gpt initially proposes SQSLP, an optimization method that requires gradients from the objective function, which is not trivial to apply as we would require gradients computed from simulation. When prompted to provide an alternative, \gpt suggests Differential Evolution, which meets the specification of not requiring a differentiable optimization problem, but can be computationally expensive. We thus explicitly prompt \gpt to provide code to perform grid search over the two frame bar lengths. This terminates in a reasonable amount of time with the result of making the frame bars as short as possible, which is the expected solution as the smaller the quadcopter, the less inertia it faces when taking off and decelerating towards the goal. 



The resulting quadcopter with optimized frame bar lengths is visualized in Fig. \ref{fig:quadcopter_render}.

% % Leader: Megan, Edward, Andy
% Finally, We also investigate how \gpt can help with measuring the performance of the quadcopter. We use \gpt to measure the quadcopter's performance through simulation. As described in Section~\ref{sec:quadcopter_sim}, this first involves converting a given \jscad design to a format amenable to simulating multibody robots, such as URDF. We then prompt \gpt to provide a LQR controller which takes in both the design in URDF format and relevant physical properties. We also describe in Section~\ref{sec:quadcopter_sim} how the quadcopter's design can be optimized through simulation. \liane{do you use these results for the fabricated example? right now feels a bit whiplash-y to go to from fab back to simulation with no tie-in for how/if the sim is applied to the real world example}

% Figure environment removed