\paragraph{Co-design of Quadcopter's Shape and Control.} Now that the quadcopter design can be outputted to URDF and the LQR controller can be synthesized from a given design, we ask \gpt to optimize the quadcopter design. Much of the design space has been fixed by the fact that most components have unchangeable dimensions. However, as the framebars are not predefined components but rather constructed from carbon-fiber tubes, we focus on optimizing their lengths. The control then follows directly from a given design, as we only need to compute the design's total mass and moment of inertia  to apply the LQR controller derived by \gpt. With the objective of minimizing the number of simulation steps required to reach a goal height of 1m, we prompt \gpt with the full Python script that converts OpenJSCAD to URDF and performs simulation, and we find that it is able to provide an outline of the optimization, consisting of an objective function that involves creating a quadcopter with specified framebar lengths and performing the simulation loop, as shown in Fig. \ref{fig:quadcopter_opt}. \gpt also provides reasonable bounds on the frame bar lengths of (100mm, 500mm) when prompted further. \gpt then provides an outer optimization script that uses the helper objective function. We also explicitly prompt \gpt to complete its outline by updating component creation code to generate frame bars with the correct masses and to change the placement of components dependent on the length of the frame bars, namely the motors, motor bases, and propellers. 

% \wojciech{Explain how you construct the design space. The title of this section is weird. you are doing inverse design. You are also co-designing shape and control.} 
% Figure environment removed

As seen in Fig. \ref{fig:quadcopter_opt}, \gpt initially proposes SQSLP, an optimization method that requires gradients from the objective function, which is not trivial to apply as we would require gradients computed from simulation. When prompted to provide an alternative, \gpt suggests Differential Evolution, which meets the specification of not requiring a differentiable optimization problem, but can be computationally expensive. We thus explicitly prompt \gpt to provide code to perform grid search over the two frame bar lengths. This terminates in a reasonable amount of time with the result of making the frame bars as short as possible, which is the expected solution as the smaller the quadcopter, the less inertia it faces when taking off and decelerating towards the goal. 

