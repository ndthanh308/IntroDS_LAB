%% Text to Design: CAD with OnShape
% \team{Felix}

Another popular method for 3D shape modeling comes from contemporary computer-aided design (CAD) software.
Rather than directly constructing and modifying solid primitives (as in the CSG approaches discussed above), modern parametric CAD systems generally work by lifting planar sketches into 3D and subsequently modifying the 3D geometry.
These sketches are placed on planes, which can be offsetted construction planes, or planar faces of the current 3D model.
The selected sketching plane serves as a local coordinate system in which the sketch primitives are defined.
In graphical user interfaces, this change of coordinate systems is accounted for by letting the user easily align their camera view to a top down view onto the sketch plane.
This change of view effectively comes back to drawing sketches in 2D, removing the cognitive burden of having to think about sketches in 3D.
Despite the lack of graphical assistance, we want to investigate whether \gpt is able to design objects using a sketch-based modeling language.

However, since the graphical assistance is very prevalent in this modeling paradigm, CAD models are mostly constructed via a GUI and not via textual programming, even though textual APIs exist, e.g. Onshape's Featurescript \cite{featurescript}.
Therefore, documentation and examples are less available than for the modeling paradigms from the previous sections.
And indeed, \gpt performs poorly when trying to generate Featurescript code directly, which is why we decided to provide a simplified DSL.

For our experiments, we constructed a single prompt containing the following DSL description:
Our DSL exposes two operators, \lstinline{createSketch} and \lstinline{extrude}, and two sketch primitives, \lstinline{circle} and \lstinline{rectangle}.
Additionally, we provide a construction example using this language of a single leg round table.
Lastly, we also add some hints about how to write the program, e.g. to explicitly use design variables and to write in \lstinline{syntactically correct python}.
All of the output designs generated by \gpt in this section are automatically translated into Onshape PartStudios.
The full prompt can be found in the supplemental material. 
% Figure environment removed

Our first task is the design of a \lstinline{chair with 4 legs, a rectangular seat and a rectangular back}, see Fig.~\ref{fig:cad_chair_design}.
We asked \gpt to perform this task several times and observed the following.
\begin{itemize}
    \item The design sometimes includes cylindrical legs, sometimes rectangular legs.
    \item The design is always constructed in a single direction, the $Z$ direction.
    Our input example of the round table only used the $Z$ direction to select sketching planes, but the description of our language documented the use of other plane directions.
    \item We observe mainly two types of designs: (i) designs which are constructed in both the negative and positive $Z$ direction starting from the seat, see first answer in Fig.~\ref{fig:cad_chair_design}, and (ii) designs which start from a leg, see the second response in Fig.~\ref{fig:cad_chair_design}.
    We observe that the first type of designs has a higher chance of being correct, whereas the second type fails more often.
    The failures are due to changes in the coordinate system.
    For example, when selecting the top plane of the first leg as a sketch plane for the seat, the sketch plane's origin will be in the center of the leg.
    \gpt will often ignore this or won't be able to account for it when pointed out.
    Conversely, when starting with the seat and choosing the lower seat plane as a sketch plane for the legs, it can specify the leg sketch coordinates in global coordinates, since the global origin coincides with the seat's origin.
    The same is true for the backrest.
    
\end{itemize}

From this test, we can observe that \gpt seems to have difficulties translating the coordinate system's origin on the XY plane.

%% Figure environment removed

% Figure environment removed

Next, we want to see if \gpt can account for rotating sketch planes.
To test this, we ask it to design a car.
\gpt always suggests a simple car shape, composed out of 4 cylindrical wheels and a rectangular car body, see Fig.\ref{fig:cad_car_design}.
The difficulty with this shape is that the cylinder sketches of the wheels have to be extruded on the side planes of the car body.
There are a couple different modeling strategies to achieve this, but we observe that \gpt has difficulties coming up with these designs without any further indication.
Instead, it often extrudes the car body along its \lstinline{height}, starting from the ground plane, and then places the wheel circles on the bottom plane of the car, which is also the ground plane.
This has the effect that the car wheels will be extruded vertically.
Although we were able to correct this design in an iterative prompt-based fashion, we had little success engineering the initial prompt in such a way that we could effectively prevent this behavior.

Note that intuitively placing wheels at the bottom of a car body makes sense and that without any graphical feedback, humans could also easily make this mistake.
From this test, we can observe that \gpt is struggling to rotationally change coordinate systems.

To address this, we changed our design language description to allow \gpt to specify sketch primitive coordinates directly in a single global coordinate system. 
Now, a sketch primitive center takes as input three coordinates, which we project in post-processing directly on the selected sketch plane.
The extrude direction is still defined by the sketch plane's normal vector.
This means that \gpt does not have to take coordinate translations into account anymore.
We observe that this change in the DSL led to a higher success rate in generated designs, see second answer in Fig.~\ref{fig:cad_car_design}.

In conclusion, \gpt is able to design models in a sketch-based parametric CAD framework. 
However it is not successful at changing coordinate systems.
In this case, our backup strategy is to use a single global coordinate system.
One possible future direction is to let \gpt communicate with a geometric solver and create a feedback loop.

\afterpage{\FloatBarrier}