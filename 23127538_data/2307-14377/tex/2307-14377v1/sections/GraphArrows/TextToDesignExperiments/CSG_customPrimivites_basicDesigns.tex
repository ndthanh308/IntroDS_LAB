%% Text to Design: CSG with self defined primitives
% \team{Bohan}

The next design domain we are investigating is CSG.
As outlined in Sec.\ref{sec:overview_domains_design}, CSG languages generally operate by building up a collection of primitives that have been altered or combined via linear transformations and Boolean operations. 
Because the associated design logic can be quite complex, it was not immediately clear that \gpt should be able to generate designs using these languages.
Thus, to progressively test \gpt's modeling capabilities, we begin by exploring a very simple, custom CSG language based on a single primitive: a box.

Boxes are one of the most common primitives seen in manufacturing. 
Moreover, many shapes can be considered as a combination of boxes with different sizes.
Because of the simplicity of a box or any shape formed by the boxes, 
we would like to see if \gpt is able to generate designs of such kind of simple shapes, 
such as tables and chairs.

Our initial approach to this task is performed in 2D.
We provide a function, foo(x, y, w, h), 
which forms a box of dimensions $w \times h$ centred at the point $(x, y)$. 
We subsequently employ this function to generate letters composed of axis-aligned bars, 
such as `F' and `E'.
During the testing phase, we observed that 
while the system understands the requirement of 2D boxes, 
it struggles with their accurate placement. 
A particularly prominent issue is the collision problem. 
More specifically, the \gpt system fails to determine 
whether two boxes are overlapping or whether there is a vacant space between them. 
This issue is observable when creating letters like `T' and `E'.
Using three to five targeted prompts enabled \gpt to ascertain the correct positions.
However, these prompts had to be granular and often involved providing the direct solution. 
The outcomes of these attempts are demonstrated in Figure~\ref{fig:letter}.
Interestingly, after addressing this issue, \gpt appears to retain the corrections. 
This is evidenced by its successful generation of the new letters `F' and `L' in a single attempt. 
These letters share a similar structure to `T' and `E', 
and the results can be seen in Figure~\ref{fig:letter}.

Our next step involved venturing into 3D, 
which holds more practical values. 
Analogous to the 2D scenarios, 
we inform \gpt of a pre-established function, box(x, y, z, w, h, d), 
which generates a 3D box of dimensions $w \times h \times d$ centred at the 3D coordinates $(x, y, z)$.
We then tested if \gpt could write a program to produce a simple box of specified dimensions, 
for instance, $100 \times 100 \times 40$, utilizing function `box'. 
\gpt successfully accomplished this task, and the resulting text explanation 
illustrates its understanding of the box concept and the usage of our predefined function.
Next, we presented a more complex challenge: having \gpt design a simple table, 
typically consisting of four legs and a tabletop in the real world. 
We posed the question of whether \gpt could craft a program to
generate such a table with a provided size using solely our box function.
The output text explanation revealed that 
\gpt accurately comprehends the structure of a basic table.
Given that we only provide the overall table size, 
\gpt lacks information about individual leg lengths or tabletop thickness. 
Yet, it was able to identify these missing parameters and make reasonable assumptions. 
Consequently, \gpt succeeded in writing a program to represent 
the table by creating five boxes using our predefined function.
Upon visualizing the 3D table, however, 
the relative positioning of each pair of boxes was not always accurate.
We noticed that the tabletop appeared to be suspended in the air, 
not in contact with the legs, as shown in Figure~\ref{fig:tabletable}. 
This difficulty, also observed in our 2D tests (Figure~\ref{fig:letter}), 
pertains to \gpt's understanding of mathematical concepts.
In this instance, we expedited the process by directly providing \gpt with the solution.
We indicated the necessary translations for the misplaced boxes, 
acknowledging that it would take several prompts to rectify the issue otherwise. 
After correcting the floating tabletop, the table appeared as intended, as demonstrated in Figure~\ref{fig:tabletable}.
Therefore, to create a table, it only required two prompts, 
significantly streamlining the procedure for generating a basic table.

% Figure environment removed


Once we successfully generate the table, 
our next more challenging goal is to design a few accompanying chairs.
We tasked \gpt with creating a chair compatible with the table, using only our predefined function. Similar to its approach with the table, 
\gpt successfully deduced the basic structure of a simple chair, 
comprising the seat, four legs, and a backrest.
Unlike the table instance, we didn't observe any `floating' issues in this scenario.
It appears that \gpt might have indeed gleaned some insights from previous experiences,
as we also observed when creating 2D letters. 
After we rectified the letters `T' and `E', 
there were no issues with the remaining letters.
Additionally, \gpt demonstrated comprehension of the concept of 
compatibility by outputting a chair of an appropriate size.
However, it was not successful in all aspects, as depicted in Figure~\ref{fig:chairchair}. 
We attempted to correct the backrest but were unable to do so.
As a result, we had to manually adjust the position,
directing \gpt to the specific lines that needed modification to correct the structure.
The final result can be seen in Figure~\ref{fig:chairchair}.
We believe the root of these issues lies in \gpt's struggles to comprehend geometric concepts,
a difficulty also observed in previous examples.
Despite these hurdles, the process for creating a basic table and chairs
has been considerably simplified.

Our final objective was to position four identical chairs around the table. 
Although theoretically feasible without invoking rotation, 
\gpt failed to generate the chairs with the correct orientations.
We believe this failure stems from the same root cause we've encountered previously, 
namely, \gpt's difficulty in handling mathematical and geometric concepts.
Creating four chairs with correct orientations without the support of rotation 
entails complex geometric transformations. 
\gpt must comprehend that a box rotated 90 degrees 
around its center is equivalent to a swap of its width and depth dimensions.
To alleviate this issue, we expanded our `box` function to include an additional input argument, `angle`, corresponding to a rotation angle around the vertical axis.
With this extension, \gpt was able to create a program using solely the `box` function 
that successfully positioned four chairs around the table with correct orientations, 
as displayed in Figure~\ref{fig:chairchair}. 
We surmise that the introduction of `angle` considerably simplifies the logic behind chair placement, 
enabling \gpt to create such a program.

In conclusion, \gpt exhibits strong understanding of posed questions 
and excels at analyzing requested objects to determine their composition.
However, it demonstrates a weakness in handling geometric and mathematical concepts.
While it can provide nearly accurate solutions when mathematics is involved,
it struggles to comprehend the underlying mathematical principles and, 
as a result, cannot independently correct math-related issues when they arise.


% Figure environment removed

% Figure environment removed

