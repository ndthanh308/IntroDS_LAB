%% Text to Design: CSG with OpenJSCAD
% \team{Liane}

To explore a full-fledged approach for \llm-aided CSG, we test \gpt's ability to generate meaningful designs using the open source javascript-based CSG library, \jscad \citep{jscad}.
\jscad has extensive documentation available online, and we found that \gpt natively possesses a good grasp of the API, its components, and the required code structure.
In particular, it understood that it needed to import each function from the corresponding modules, and that it needed to define and export a function named \lstinline{main}.
For our experiments, we provided \gpt with access to the full API, and generally allowed it to select the appropriate primitives and operations without user interference. 

To test \gpt's design abilities, we ask it to design a simple cabinet with one shelf, as shown in \fref{fig:textToDesign_simpleCabinet}.
\gpt reliably selects and instantiates the required primitives, along with intuitive naming conventions and structure within the \jscad code. 
\gpt's initial orientation of the parts was also generally reasonable, but the specific positioning of each part was often incorrect. 
Despite multiple attempts, \gpt was unable to generate any fully-correct cabinet in a single shot, with no subsequent user intervention.


Moreover, \gpt frequently produced highly disparate results from one run to the next. Even when using an identical prompt on fresh chat environments, \gpt's responses varied widely in terms of their overall code structure, design accuracy, and the specific errors or oversights made. 
\fref{fig:cabinet_vertical_explosion} shows one example of a drastically different design process, even when seeded with the same initial prompt as \fref{fig:textToDesign_simpleCabinet}.

Throughout our experiments, we found that \gpt encountered a few common pitfalls when generating designs in \jscad.
Occasionally, \gpt made small syntatic errors such as generating incorrect boilerplate, importing functions from incorrect modules, or making ``typos'' in API calls -- \eg, trying to import from the \lstinline{boolean} module rather than the correct \lstinline{booleans} module, or calling the \lstinline{cube()} function with parameters that were intended to generate a \lstinline{cuboid()}. 
In an attempt to avoid these pitfalls, we created a small list of ``hints''/``reminders'' for best practices when working with \jscad; this short list was always passed in alongside our initial prompt. See \appref{sec:appx-jscad-hints} for a full listing of these reminders.
Although these reminders seemed to help mitigate these issues, we were unable to eradicate them entirely. 
However, \gpt can easily correct the majority of these issues when they were pointed out by the user.
Often, the process of correcting the issue through prompts and responses was faster than actually adjusting the code manually, making {\llm}s a useful design partner.

One pervasive issue that seemed more difficult to correct was the fact that \gpt had issues positioning the primitives in 3D space.
In particular, \gpt frequently seemed to forget that \jscad positions elements relative to the \textit{center} of a given primitive, rather than an external point on the primitive (\eg, the lower left corner). 
\gpt's arrangements were frequently incorrect due to this issue.
When \gpt is reminded of this convention, it does generally alter the design, but it is not always able to correct the issue. 
If sufficiently many rounds of local edits prove unable to address the alignment issues, we found that it was generally more effective to direct \gpt to disregard all existing measurements, and re-derive the elements' positions from scratch (see \fref{fig:cabinet_vertical_explosion}).

Overall, we find that \gpt is able to generate reasonable \jscad models from high-level input.
However, the design specifications that emerge on the first attempt are rarely fully correct, so users should expect to engage in some amount of corrective feedback or iteration in order to attain the desired result.

% Figure environment removed
%\afterpage{\FloatBarrier}

\newcommand{\cabVertExpImHeight}{2.2cm}
% Figure environment removed