%\team{Mike}


% Our first design domain is 2D vector graphics.
% Vector formats like SVGs or DXFs are commonly used to describe manufacturing files, 
% for example for laser cutting.
% We want to investigate whether \gpt could empower designers to convert text directly into vector files, 
% readily usable for manufacturing.
% To test this, we have experimented with generating an SVG file that could be used to create a manufacturing file. 
% We also experimented with converting a DXF format of the same design.  

% In the experiment, the goal was to design and construct an SVG file of a cabinet with specific dimensions using 1/2 inch plywood. The cabinet was intended to have three shelves, with the cabinet having overall dimensions measuring 6 feet in height, 1 foot in depth, and 4 feet in width. The experiment aimed to accurately account for the thickness of the plywood when designing the cabinet, ensuring that the dimensions of the various parts were adjusted accordingly. To achieve this, a Python script was developed to generate an SVG file representing the cabinet layout using \gpt. The script calculated the required clearances for the wood thickness and appropriately positioned the side panels, top and bottom panels, shelves, and back panel, while considering the specified spacing between the parts and utilized svgwrite to generate the SVG file. The resulting SVG file provided a visual representation of the cabinet's design that could be used for cutting out the parts. Similarly, we repeated this experiment to create a DXF file where \gpt utilized ezdxf to generate the file. Results are shown in Figure \ref{fig:SVG_DXF_Gen}.  \gpt was able to use the APIs to generate the file in the correct format without any simplification, however, multiple iterations were needed to ensure \gpt did not overlap the parts of the cabinet.  

% \liane{is there something specific we want to highlight from the chat excerpt? It's very long right now and I'm not sure what is the interesting part to look at / message to take away. also, would be nice to have a visual of the svgs} \adriana{yes, this figure is strange}\bolei{probably it is better to show the patterns of svg files instead of the code?}
% \wojciech{This is written as a manufacturing section not design section.}

Our initial focus in the design domain is on 2D vector graphics.
Vector formats such as SVGs or DXFs are prevalently utilized in manufacturing processes,
like those for laser or waterjet cutting.
The goal of our investigation was to ascertain whether \gpt could empower designers to 
transform their text directly into these vector formats. 
To evaluate this, we conducted experiments to determine 
if \gpt is capable of generating a valid SVG file and converting the design into DXF format.

The primary aim of our experiment was to design an SVG file for a cabinet, 
with predetermined dimensions, to be constructed from 1/2 inch plywood. 
This implies that the thickness of each wall, a preset parameter, is 0.5 inches.
The experimental setup involved the design of a cabinet comprising three shelves, 
with overall dimensions measuring 6 feet in height, 1 foot in depth, and 4 feet in width.
A crucial aspect of the investigation was to see if \gpt could accurately account for this wall thickness 
during the design of the cabinet, appropriately adjusting the dimensions of its various components.
\gpt was able to design the specified cabinet and 
subsequently generated a Python script to create an SVG file reflecting the cabinet's layout.
The script considered the necessary clearances for the thickness and 
accurately positioned the side panels, top and bottom panels, shelves, and back panel.
Moreover, it factored in the prescribed spacing between parts and 
leveraged `svgwrite' to generate the SVG file. 
The resulting SVG file provided a visual depiction of the cabinet's design.
We also replicated the experiment to create a DXF file, 
where \gpt utilized `ezdxf' to generate the file.
The results of these experiments are depicted in Figure \ref{fig:SVG_DXF_Gen}.

In conclusion, \gpt demonstrated its capability to employ the APIs for 
generating the vector file in the correct format without any simplifications.
Nevertheless, it was necessary to perform several iterations to 
ensure \gpt did not cause any overlap among the cabinet parts.

% Figure environment removed
\afterpage{\FloatBarrier}
