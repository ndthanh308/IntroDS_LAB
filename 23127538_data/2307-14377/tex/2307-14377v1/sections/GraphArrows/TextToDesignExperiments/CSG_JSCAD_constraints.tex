As an initial experiment, we explored whether \gpt is able to construct a version of the previous cabinet design that includes a door and a handle (see \fref{fig:cabinet_with_handle}). 
We started from a fresh chat, and provided \gpt with a prompt similar to the one described in \sref{sec:textToDesign_JSCAD_basic}, asking for a cabinet to be built from scratch.
However, this time, we also request a door at the front of the cabinet, with a handle on the right hand side of its outward-facing face.
As shown in \fref{fig:cabinet_with_handle_start}, \gpt initially struggled to position several of the cabinet elements -- particularly the side panels and the door.
Although \gpt corrected the position of the side boards immediately, \gpt continued to have trouble placing the door, as it was oriented incorrectly relative to the rest of the design. 
When reminded that the door should be oriented vertically, \gpt was able to comply with the request, but the corrected position was still not fully suitable, as the door coincided with the cabinet's side panel.
After another reminder that the door should reside at the front of the cabinet, with the handle on the right so it could be attached with hinges on the left, \gpt was able to place the door correctly. 
However, the handle remained ill-positioned as it was located on the left-hand side, and was protruding into the door panel. 
After 2 additional prompts, \gpt was able to correct the position to the left hand side. To correct the protrusion issues, \gpt needed 3 more prompts. During these iterations, \gpt moved the handle fully to the \textit{inside} of the door; it needed explicit reminder that the handle should be placed on the \textit{outside} of the door.

With a fresh \gpt session, we also tried providing the previous \jscad specification of the cabinet as part of our input prompt, then asking \gpt to modify the existing design such that it contained a door and a handle, as before. 
Despite the different starting points, \gpt followed a similar trajectory, as shown in \fref{fig:cabinet_with_door_add}: the door was initially aligned incorrectly, as it coincided with one of the side panels; after 1 prompt, \gpt was able to correct the door placement. However, despite \gpt's explicit assertion that \lstinline{the handle is also placed on the right side of the door's exterior face}, the handle remained on the left. Finally, after another prompt, \gpt was able to correct the handle position such that it was on the right rather than the left. 

The way in which \gpt dealt with the under-specified handle request also proved interesting. 
In \fref{fig:cabinet_with_handle_start}, \gpt opted for an additional cuboid that would be unioned into the final design. 
By contrast, in \fref{fig:cabinet_with_door_add}, \gpt opted to create the handle by subtracting a small cuboid from the door panel. 
In still other examples, \gpt refused to add the handle, and instead offered the following disclaimer: \lstinline{Note that the handle for the door is not included in this script, as its size, shape, and position would depend on additional details not provided. This would likely require additional modules, such as cylinder from @jscad/primitives, and might be added as an eighth component in the main function.}

These interactions provide a promising basis for interactive user control of the design, but the process is somewhat tedious at the moment, as \gpt requires very explicit instructions about the design or correction intent. 
The addition of highly-detailed user constraints also seems to confuse \gpt to an extent, as it seems to ``forget'' the larger context of the design in the process, so it must be frequently reminded. 

% Figure environment removed
% \wojciech{Fix colors in the figure}

% Figure environment removed