%% Text to Design: quadcopter with integrated prefabbed parts
% \team{Allan/Bohan, maybe with help from Edward/Pingchuan/Megan/Andy}

% example: quadcopter frame designed around a collection of components.
% include eg:
% - how you chose the components; 
% - selecting components that are compatible; 
% - size constraints (from eg the "car sized copter" you had at some point to the final design).
% - coming up with the geometric primitive proxies for each element
% - design process for the frame using those components

% wip

% Figure environment removed

% block commented
\iffalse
 \adriana{strange way to start - where do you determine the components?}
Once the components are determined, 
we would like to design a quadcopter that uses them with the help of \gpt
and study how \gpt can help with such task.
\adriana{replace previous sentence with: The design of a quadcopter requires the incorporation of pre-fabricated elements such as X, Y, Z. Assuming these have been identified and sourced, the design of the frame should be dictated by their dimensions. Next, we'll examine how GPT can assist with this design task.}
Nonetheless, it is not simple to let \gpt represent those components in details.
To ease the design task for \gpt, we represent them 
using either a box with size $w\times h\times d$
or a cylinder with radius $r$ and height $h$,
which we found that it understands them very well 
as shown in the previous sections. \adriana{which sections?}
The first thing we tell \gpt is 
the dimensions of these existing parts represented 
by boxes or cylinders.
To make it more understandable, 
we also give three functions to describe the box, the cylinder and their placement.
The first function createBox(w, h, d) returns a box with size w * h * d centering at (0, 0, 0).
The second function createCylinder(r, h) returns a cylinder with radius r and height h center at (0, 0, 0).
The last function place(item, x, y, z, a) first rotates the item around z axis by angle a
and then translate the item to position (x, y, z).
Next, we ask \gpt to create a design that incorporates those given parts using only those functions. 
The main part of the quadcopter that needs to be designed 
is the frame which can hold those selected components.
At the first try, \gpt is able to give a roughly correct design description,
but it fails to produce a correct geometry.
It understands that the quadcopter needs four arms 
to connect four motors with four propellers
and that the remaining parts are at the center.
However, it could not make all parts correctly positioned and oriented, as shown in Figure~\ref{fig:quadcopter-stage}(a).
After closely looking into the generated program, 
it is not hard to figure out several design issues. 
First, the frame is not oriented correctly.
This is fixed by telling \gpt 
which dimensions represent the cross section of the frame.
After the adjustment, we can get a near correct quadcopter (Figure~\ref{fig:quadcopter-stage}(b)).
Second, the parts are intersected with other boxes, as highlighted in red rectangles in Figure~\ref{fig:quadcopter-stage}(b).
This problem has been happening constantly 
when we designed other items earlier.
We found that it is difficult to let \gpt understand 
what it means when we say box A and box B are intersected
and how it can be fixed.
Therefore, our solution to fix this problem is 
to manually tell \gpt how much translation is needed 
for each problematic part.
Finally, the frame is not practical because
1) it directly attached to the motor cylinder and
2) it is not sufficient to hold components including
the battery, the controller and the signal receiver.
To solve the problem, we did not ask the \gpt about
how you can solve the problem.
Instead, we provide a concrete solution to the issues
and then ask \gpt step by step.
We first asked \gpt to add a cylinder base
at the bottom of each motor.
The cylinder base directly connects to the frame bar
and each motor sits on each cylinder base without touching the frame bar.
In addition, we ask \gpt to add a box body to
strengthen the frame bars and serve as a plate to hold to remaining parts.
After a few iterations of minor adjustments,
we were able to obtain a valid design which will be further verified
in the simulator or in the real world.
The final result is shown in Figure~\ref{fig:quadcopter-stage}(c).
Throughout the design process, we found that \gpt is good at analyzing the design problem and give a valid design in text.
However, it cannot nicely incorporate the mathematical and physical concepts. 
For example, it does not understand colliding and 
whether the frame is strong to hold any parts.
Therefore, human has to help it in this regard.
 \adriana{I think this section could use some polish to shorten and create understandable paragraphs and opposed to one long text. I also think its best to reference 4.1.2 and say you use the same primitives instead of defining a language all over again.  }

 \fi

 
Designing a quadcopter involves integrating pre-built elements like the motor, propeller, and battery. Detailed sourcing of these parts will be addressed in the later section (Section~\ref{sec:part_sourcing}). Once these components are sourced, the frame must be designed to accommodate their dimensions. We'll explore how \gpt can assist with this task.

However, enabling \gpt to accurately represent these parts isn't straightforward. To simplify the task, parts are represented as either a box of dimensions $w \times h \times d$ or a cylinder with radius $r$ and height $h$. \gpt can handle these representations well as demonstrated in Section~\ref{sec:textToDesign_CSG_boxes}. Rather than having a single function which creates a primitive and translates it as in Section~\ref{sec:textToDesign_CSG_boxes}, we introduce three functions for ease of design: createBox(w, h, d), createCylinder(r, h), and place(item, x, y, z, a). The first two functions generate a box or a cylinder at origin (0,0,0), while the third rotates and moves the item to desired coordinates.

Subsequently, we task \gpt with creating a design that integrates these parts using only the above functions. The primary element \gpt must design is the frame, which should hold the selected components. Initially, \gpt produced a correct textual design, but struggled with the geometric representation, similar to Section~\ref{sec:textToDesign_CSG_boxes}. It understood the quadcopter structure, but had issues with part positioning and orientation (Figure~\ref{fig:quadcopter-stage}(a)). Problems included incorrect frame orientation and part intersections. By guiding \gpt in correcting these issues, we achieved a near-correct quadcopter design (Figure~\ref{fig:quadcopter-stage}(b)).

The initial frame design wasn't practical because it was directly attached to the motor cylinder and insufficient to hold components like the battery, controller, and signal receiver. To address this, we asked \gpt to incrementally implement specific solutions, such as adding a cylinder base under each motor and a box body to reinforce the frame bars and house remaining parts. After minor adjustments, we arrived at a valid design, which will undergo further testing in a simulator or real world conditions (Figure~\ref{fig:quadcopter-stage}(c)).

Throughout the design process, \gpt demonstrated proficiency in textual design analysis but struggled with mathematical and physical concepts such as collision and structural integrity. Thus, human guidance remains crucial in these areas.