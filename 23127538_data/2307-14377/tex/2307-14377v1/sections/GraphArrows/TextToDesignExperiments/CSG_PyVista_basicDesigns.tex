%% Text to Design: CSG with PyVista
% \team{Yifei}

Building on \gpt's success generating CSG-like models with boxes, we set out to explore \gpt's capacity to use a larger suite of primitives.
For this, we used an existing 3D visualization library, PyVista, which allows us to create and place a variety of 3D primitives such as spheres and cones.
Thanks to the library's documentation, \gpt is able to automatically assemble a functional python program using PyVista's primitive functions.

We asked \gpt to use PyVista's primitives to model several variations of a fish, including specific bio-inspirations such as goldfish, a manta ray, and a loach (\fref{fig:fishes}). 
\gpt successfully selected and scaled an appropriate set of primitives for each example, and provided sound bio-inspired rationale for its decisions.
In particular, although most of the fish are composed using a sphere for the body, \gpt intuits that a loach would be most effectively approximated by using \lstinline{two cones for the body to give it an elongated shape}.

One area in which \gpt struggled was the determination of the primitives' orientations. 
It often produced results that indicated an internal confusion of some of the axes, or an otherwise flawed approximation of the orientation that would be required to achieve a desired effect.
After engaging in a dialogue with \gpt, it was able to rectify the orientations of the primitives to more closely resemble the target creatures.
While promising, these tests reiterate \gpt's seemingly limited capacity to account for local coordinate frames.

% GPT's rationale:
% \ignore{
% Goldfish: "This code creates a fish that has a larger, more rounded body and smaller, more forward-positioned head, which are more characteristic of a goldfish. The fins are also larger and positioned higher and lower on the body to mimic the large, flowing fins of some types of goldfish."
% Manta ray: "A manta ray has a flatter, wider body and long "wings", or pectoral fins. The tail is thin and long. We can use a large, flat sphere for the body, a thin, wide cone for the pectoral fins, and a long, thin cone for the tail."
% Loach: "A loach is typically elongated and has several fins along the body. We can create a simple representation using multiple cones for the body and fins. For simplicity, let's use two cones for the body to give it an elongated shape, and additional cones for the fins."
% }




 % \adriana{I think the images in this section are taking too much space, I propose a single image with all 4 results as it's not adding much to the discussion.}

% Figure environment removed


% % Figure environment removed

























% =====================
% OLD TEXT
% =====================

\ignore{
\FH{start of old text}

In this conversation, we explored using 3D mesh library to generate and visualize various designs based on constructive solid geometry (CSG) principles  with the aid of \gpt. 
The integration of PyVista, a 3D visualization library, with large language models facilitates an interactive, real-time computational design process. 
This approach simplifies 3D operations and democratizes design by allowing natural language inputs, fostering broader accessibility. 
The dynamic interaction enables efficient exploration of design parameters and serves as an instructive tool for understanding 3D modeling and computational design. 

We started by creating simple 3D primitives such as spheres. 
\gpt is able to assemble a python program automatically using PyVista's primitive functions that meets our design goals.  
Then, we expanded upon this concept to model fish (Fig.~\ref{fig:fish_and_variation}) and its variations with different bio-inspirations such as goldfish (Fig.~\ref{fig:fish_goldfish_variation}). 

We initially attempted to use boolean operations like union to combine different primitives. 
However, this process encountered some issues as PyVista's boolean operations require all the meshes to be composed of triangles, which wasn't the case with some of the primitives used by \gpt. 
This could result from the model being trained on outdated library version.

As a workaround, instead of using boolean operations, we explored a different approach where we directly added all the part meshes to the PyVista plotter for visualization. 
This method effectively bypassed the requirement for the meshes to be triangulated, enabling us to create more complex shapes from primitive elements.

Using this method, we generated a series of parametric designs, demonstrating the flexibility of this approach. We experimented with a variety of aquatic creatures, including a generic fish, a goldfish, a manta ray~\ref{fig:fish-mentaray}, and a loach~\ref{fig:fish-loach}. During the process, we discovered that \gpt has confusion in terms of how to correctly orient individual primitives to achieve desired looks, and we have to manually tweak orientations of the primitives to closely resemble the target creatures.

In conclusion, our exploration highlights the potential of large language models in aiding computational design. 
With an intuitive and interactive dialogue, we can generate a wide range of designs from basic primitives using libraries like PyVista. 
While certain limitations exist, such as \gpt's failure to meet the requirement for triangulated meshes in boolean operations (which could result from being trained from old codebases) as well as confusion in coordinate systems, creative solutions can be found to bypass these and generate complex and versatile designs. 
This shows that with the right tools and approach, large language models can significantly aid in the computational design process, making it more accessible and intuitive.
}

\ignore{
% Figure environment removed
\afterpage{\FloatBarrier}
}
