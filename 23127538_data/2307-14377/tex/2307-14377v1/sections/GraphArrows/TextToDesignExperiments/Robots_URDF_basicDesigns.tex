%% Text to Design: Robots from URDF description

% \team{Wil, maybe with input from Megan/Andy/Allan)} 

% Figure environment removed

The Universal Robot Description Format (URDF) is a common XML-based language for describing articulated structures in robotics. 
URDF files specify a robot's structure (including both visual and collision geometry), joint locations, and dynamics information. 
The URDF format appears well-suited for potential \llm design because it is human-readable and heavily documented online.

\paragraph{Open Chain Robot Arms}
Initially, we asked \gpt to generate simple open chain robots (commonly called ``arms'') with a particular number of links. 
However, when we used the word ''arm'' to prompt \gpt to generate a robot, \gpt was unable to determine that the links should connect at the end.
Most often, \gpt placed the joints such that each link revolved about its center, and the links were not connected to each other (\fref{fig:multi-link-urdf}, initial prompt).
As shown in the subsequent prompts of \fref{fig:multi-link-urdf}, to achieve an arm with two connected links, it was necessary to describe both the joint position relative to the link 
(\lstinline{``the joint origin must be half the link's length past the link origin''}, rather than \lstinline{``the joint origin should be at the end of the link''} ) 
as well as the joint axis (\lstinline{``a revolute joint about the x axis''}).
Given this prompt pattern, \gpt was easily able to generate proper N-link robots. 

%\adriana{I don't really  understand what this section is trying to say. is the point that it can't connect things directly if you don't have that as a constraint of the DSL? if so this would be better expressed if you group fig 10 and 11 in one single discussion like Felix did in Figure 9 }



\paragraph{Wheeled Robots}
Next, we asked \gpt to generate wheeled robots composed of N wheels attached to a central rectangular platform.
A proper design of this type must have wheels that 
(1) are aligned to share an axis of rotation normal to and through the center of their circular faces;
(2) have circular faces displaced along said axis of rotation, and 
(3) contact, but do not intersect, either side of the center platform.
The combination of non-intersection and geometry relation constraints prove challenging for \gpt, which seems to exhibit limited geometric reasoning. 
Initially, we tried to specify these using language-based constraints (i.e. ``the wheels should touch, but not intersect, either side of the platform'').
These proved ineffective, as shown in \fref{fig:urdf-wheeled-constraints} (middle).
To overcome these challenges, we crafted prompts with very explicit numeric constraints (i.e. ``wheels should be offset on the global y axis by half the width of the platform plus half the height of the wheel cylinder''). 
This style of prompt successfully generated a viable result, as shown in \fref{fig:urdf-wheeled-constraints} (right).

As in the case of robot arms, we find that \gpt is immediately able to generalize a successful two-wheeled design into a four-wheeled robot. 
We achieve this by asking for a duplicate, shifted version of the existing wheel configuration, as shown in \fref{fig:urdf-four-wheel}.
However, we were unable to directly generate a successful four-wheel robot; in general, we found that as the number of constraints in a prompt increases, it becomes increasingly likely that \gpt will ignore any individual constraint. 
Thus, rather than directly requesting a four-wheeled robot in a single prompt, we found greater success by first generating a two-wheeled robot and then prompting \gpt to modify the URDF by adding additional wheels than placing the text in a single prompt.

% Figure environment removed


% Figure environment removed



\paragraph{Robot Grippers}
To test the effectiveness of our iterative, multi-prompt approach for building robots of increasing complexity, we seeded \gpt with a successful two-link open chain URDF, then asked it to modify this design into a collection of multi-finger robot grippers. As shown in \fref{fig:urdf-hands}, we were able to build two-, four-, and five-finger grippers using a sequence of prompts to add features and change proportions. 
To create a two-finger gripper, we asked \gpt to use two of the previously generated two-link open chain robots as fingers, separated by a distance equal to half the height of the finger, and connected by a rectangular platform on the base.
The four-finger gripper was similarly derived from the two-link arm by specifying that the hand should consist of four two-link robots right next to each other on a rectangular platform. To specify a five finger hand, we requested a rectangular link that hinges as a base for the thumb, then prompted \gpt to add another finger on that link and to adjust the hand proportions.

% Figure environment removed



























\ignore{

% Figure environment removed

% Figure environment removed

}