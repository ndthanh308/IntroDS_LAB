%% Text to Design: CSG with OpenJSCAD, integrating pre-fabbed parts
% \team{Liane}
\newcommand{\bracketImHeight}{1cm}

% Figure environment removed

\newcommand{\bracketPlacingImWidth}{0.162\textwidth}
% Figure environment removed
\afterpage{\FloatBarrier}

To make the cabinet design more stable, a designer may wish to include extra support brackets to work with. 
Many pre-fabricated variations of these brackets exist, and they are inexpensive and readily available.
Given this, it does not make sense to design or manufacture these parts via \gpt.
Rather, we'd like to incorporate instances of a pre-fabricated version.
To do this, \gpt must first build a proxy of the part, place the proxies throughout the design appropriately, and adjust the remaining elements of the design to accommodate these components.

For our first experiment, we chose to incorporate the Prime-Line 1/4 in. Nickel-Plated Shelf Support Pegs from Home Depot into our design.
We provided \gpt with a URL to this part's listing on the Home Depot website, which contained a text description of the item and the schematic diagram pictured in \fref{fig:bracket-buildup}(left).
We then asked \gpt to build a simple geometric proxy that we could incorporate into our design as a placeholder.
As shown in \fref{fig:bracket-buildup}(right, top), \gpt was able to infer and generate the appropriate primitives (one cylinder for the peg and two cuboids for the L bracket).
However, it was not able to correctly scale, orient, or position the elements. 
In an effort to test \gpt's understanding of the structure, we asked it to describe the structure in its own words. 
Although it gave a reasonable description of the bracket, there was little improvement in the result when it was asked to improve the script accordingly.
Thus, even with several iterations of user feedback, \gpt was unable to construct this shape from high-level third-party (URL) or user input.

Ultimately, we had to provide \gpt with an explicit description of the structure that we wanted.
Moreover, we found that even with an explicit description, \gpt was unable to generate the correct shape when provided with all directions at once.
Instead, we had to create the shape in an iterative fashion, beginning with the L bracket and then adding in the peg, as shown in \fref{fig:bracket-buildup}(right, bottom).
Eventually, it was able to generate the structure and consolidate the instructions into a high-level module called \lstinline{createBracketWithPeg}, as desired.

We then provided the module \lstinline{createBracketWithPeg} as an input to \gpt, and asked it to incorporate these structures into the design, as detailed in \fref{fig:cabinet_with_handle}. In particular, we asked for four brackets under each shelf, with the pegs protruding into the cabinet's side walls, the back face of the bracket's vertical leg in contact with (but not protruding into) the side wall, and the top face of the bracket's horizontal leg in contact with (but not protruding into) the bottom face of the shelf. 
We initially tried to complete this experiment in a single continuous chat that (1) designed the cabinet, (2) designed the L-bracket, and then (3) incorporated the brackets into the cabinet. 
However, we found that after the extended discussion regarding the L-bracket design, \gpt seemed to have completely forgotten its cabinet specification.
Despite multiple prompts, it was unable to recover the previous design. 
Instead, we directly provided \gpt with the L-bracket module and its prior cabinet design, and then asked for a modification.
This approach was far more successful.
Overall, we found that \gpt was able to instantiate the correct number of brackets, but it struggled to rotate and position them appropriately. 
After several user prompts, \gpt was able to successfully place the brackets in their locations.
Finally, we asked \gpt to adjust the shelf in order to (1) not protrude into the brackets, and (2) incorporate some additional allowance so the shelf could easily fit between the supporting brackets in a physical assembly.
\gpt was able to complete these requests without issue.

Overall, although \gpt initially struggled to build a proxy of the pre-fabricated part we had in mind, \gpt seemed quite capable of incorporating the completed proxy into a given design, as desired.
