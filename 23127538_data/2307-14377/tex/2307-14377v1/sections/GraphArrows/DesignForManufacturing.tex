\section{Design-For-Manufacturing}
\label{sec:design_for_manufacturing}

The utilization of \llms in the context of Design for Manufacturing (DfM) provides a broad range of applications that have the potential to enhance the design and manufacturing process of different parts and assemblies. One useful application of \llms involves leveraging their pattern identification and language interpretation capabilities to imitate a manufacturing expertise bank that can be tapped into during various parts of the design and manufacturing stages. Furthermore, because \llms such as \gpt have the ability to create programs and find and interpret patterns in text, it can potentially be used to generate and alter design and manufacturing files. Currently, DfM is often accomplished by human expertise with the aid of CAD software. Engineers and designers review design plans and use their industry experience to suggest alterations that would improve manufacturability. The CAD software then allows these alterations to be modeled. The replacement of human manufacturing knowledge with \gpt in this context could streamline the design for manufacturing process, offering more consistent, scalable, and efficient decision-making, which is not limited by individual human capacity.

In this section, we propose multiple ways that this new manufacturing expertise bank could be used in design and manufacturing, as shown in \fref{fig:DfM_Overview_Figure}. \gpt can be used to select optimal manufacturing techniques based on a part's features. Furthermore, it can propose and implement modifications to a design to improve its manufacturability, ultimately leading to more efficient production processes. Additionally, this idea can be extended to part sourcing by leveraging the model's reasoning capabilities to identify potential suppliers based on the part's desired function and performance. Finally, it could be used to develop manufacturing instructions for various processes. To understand \gpt's ability to alter designs based on manufacturing/sourcing constraints, we pose the following questions: 


\begin{itemize}
    \item \textbf{Q1} Given a part geometry, production run and other desired outcomes, can \gpt select optimal manufacturing processes? 
    \item \textbf{Q2} Given a manufacturing process, can \gpt directly suggest and make design alterations to a parts file based on constraints driven by the process capabilities? 
    \item \textbf{Q3} Given a desired functionality and geometric specifications, can an LLM find a source for a part that fits those specifications? 
    \item \textbf{Q4} Given a design can an LLM create a set of manufacturing and assembly instructions? 

\end{itemize}
% Figure environment removed

\subsection{Finding Optimal Manufacturing Process (Q1)}
To test these capabilities, we tasked \gpt with advising on identifying an optimal manufacturing process for a part with the geometry shown in \fref{fig:Optimal_Manufacturing_Process}. We tested it with four different cases where in each case the geometry, material, tolerance requirements, and quantity were varied. We described the part's geometry as an \jscad file. Finally, given a set of priorities, we task \gpt to select an optimal manufacturing process. In Case four, we provided a finite list of manufacturing processes to evaluate the effectiveness of the selection process under the constraint of a limited set of options. The goal was to determine how well the process could choose the appropriate manufacturing processes to meet the specified priorities.

\gpt was successful at selecting an optimal manufacturing process for three out the four cases. For cases one, two, and four, \gpt selected the optimal process that was approved by an expert. However, in case three, shown in \fref{fig:PTFE_Example}, \gpt suggested an injection molding process, which is not suitable for processing a Polytetrafluoroethylene (PTFE) material. In all cases, \gpt initially only provided a range of manufacturing options; it required additional prompts to arrive at the optimal manufacturing process selection.

% Figure environment removed

% Figure environment removed

\subsection{Design Alterations for Manufacturability (Q2)}
In this section, we assessed \gpt's capability to enhance designs for better manufacturing optimization. To accomplish this, we included the text of a \jscad file in the prompt, allowing \gpt to analyze and modify it accordingly. Our focus in this case was on the CNC machining of a 10-inch diameter disk, which involved creating bolt holes along the edge and a central blind square pocket. We included in the geometry, two intentional features that would be difficult to machine. As depicted in \fref{fig:Optimal_Manufacturing_Design}, the process began with \gpt identifying any manufacturing complexities within the design features. Since an \llm interprets text, \gpt interprets the text of the \jscad file, rather than the geometry that is rendered once compiled which humans interpret. After \gpt identifies any complexities, we instructed it to adjust the geometry of the \jscad file to address the challenging aspects by directly changing the text of the \jscad file.

Although \gpt accomplished these tasks with a moderate degree of success, there were a few inaccuracies. Firstly, \gpt correctly identified two potential machining issues: the small radius of the internal pocket and the thin wall at the pocket base. However, it also misunderstood a number of geometric features described in the \jscad file. These include perceiving holes on a curved surface and anticipating an undercut from the pocket. These misinterpretations might be attributed to \gpt's reliance on the text of the \jscad file for feature identification, as some features become more visible once the file is compiled into a geometric representation. After pointing out these interpretation errors to \gpt, it was able to correct its analysis but introduced another mistake. \gpt incorrectly stated that the bolt holes presented machining difficulties and inquired about additional information regarding the machining area. Once provided with the necessary details, \gpt independently rectified its mistake about the bolt holes. \gpt was also aware of potential issues with the size of the part and machining area of the CNC machine. Furthermore, it was able to compute whether there was a potential issue. 

In the final stage, \gpt was asked to modify the \jscad file to address the manufacturing concerns. It improved the wall thickness from 0.02" to 0.04", making it machinable. Given the additional specification of utilizing a 1/4" endmill, \gpt also adeptly adjusted the internal pocket's radius to accommodate this tooling requirement better.

% Figure environment removed

\subsection{Part Sourcing (Q3)}
\label{sec:part_sourcing}
The massive dataset backing \llms contains some specialized knowledge about parts needed for manufacturing. Consequently, we posit that \llms can be useful for reasoning about these parts, from identifying the correct part names to describing necessary properties for their functionality.
% We posit that the massive dataset backing \llms will have specialized knowledge about parts needed for manufacturing, from identifying the correct part names to describing necessary properties for their functionality.

\paragraph{Cabinet part sourcing} As part of generating the design and fabrication instructions for our cabinet, we asked \gpt to find appropriate shelf brackets for the shelf within the cabinet, starting from a concrete design specification in \jscad. In each iteration, \gpt provided several suggestions as links to products on Home Depot, with a short sentence differentiating them. Numbers in the part descriptions were inaccurate: one bracket pair held up to 300 lbs, but \gpt claimed it could hold 1000. Another pair was a `` heavy-duty option that can support up to 500 lbs. when properly installed.'', but could actually hold 1300 lbs. Otherwise, the short descriptions were true, and all described parts could plausibly serve as shelf brackets.
%we believe all were valid suggestions as shelf brackets \liane{maybe get rid of the "we believe" and phrase it more objectively like "all suggestions could plausibly serve as shelf brackets"}. 
Figure~\ref{fig:brackets} shows the presumed brackets suggested. 
 Overall, we found success for this relatively simple use case.

% Figure environment removed%\liane{this figure could probably be more space efficient (though not sure if it's worth the effort. It's a nice figure otherwise :) )}

\paragraph{Copter part sourcing} We also asked for help sourcing parts when designing the quadcopter example. First, we asked for a parts list that would encompass everything needed for the design. \gpt compiled a list including batteries, frames, propellers, transmitters and receivers, electronic speed controllers, etc. We found that the list was comprehensive and accurate. 
Next, we tried narrowing down the response from a list of parts to a list of specific parts with more tailored guidance for each use case. 
Asking for a range of numerical specifications (e.g. specific amperages for batteries) produced correct and sensible numerical estimates for parts.
Specifying that the copter should be able to hold a weight of 10kgs for 10 minutes yielded a list of
very large and powerful parts. Specifying an indoor copter led to smaller and more lightweight part suggestions.
Pushing \gpt beyond specification resulted in errors. %\liane{not sure what you mean; what did you ask for?}
Asking for specific names of part listings or parts and manufacturers, as in~\autoref{fig:copter-parts} tended to result in lists with incompatible parts, or in naming parts that do not exist. 
%\liane{This sentence has a lot going on -- I'd break it into 2-3 sentences that are more clear}
%For example, it listed Crazepony propeller guards on the parts list; Crazepony does not sell propeller guards. Within one list, we noted four major inaccuracies within our best results, out of ten identified parts. Two were that a certain part did not exist from the listed manufacturer, one was a redundancy where listed parts were already included in the purchase of another part (the flight controller kit recommended by \gpt included a power distribution board, so there was no need to purchase one separately), and one was an item that \gpt believed to be a transmitter and receiver, but was only a transmitter. A subsequent attempt to point this out rectified the issue and \gpt recommended a receiver correctly. We ordered a modified version of the parts list where the four errors were fixed.
Iterating on the errors with \gpt, as seen in our follow-up question in~\autoref{fig:copter-parts}, produced correct new parts.

Though asking \gpt to produce the names of real-world parts was unsuccessful, we still found impressive results in its comprehensiveness and ability to form fairly specific and accurate part lists. \gpt was also able to dispense meaningful advice on ensuring parts were compatible, even though it was unable to generate parts lists satisfying compatibility itself. We believe that \gpt can be a useful guide for delivering domain-specific knowledge and providing complete parts lists, but that precise numbers and specs should be cross-referenced before being used.

%\liane{I'd remove the H options from the chats; it breaks up the text flow (though, this is maybe something to decide as group for the whole paper)}
% Figure environment removed


\paragraph{Geometry-based part sourcing} McMaster-Carr is a deep compendium of knowledge for hardware parts, with geometric information and even CAD models available for many items. McMaster-Carr already has a ``search by geometry'' feature, so we wanted to know if we could perform higher-level searches that involve both context and geometry. First, we tried describing specific scenarios and asking \gpt for search terms that would procure us the correct part. Asking for a nut to be used in a tight space without room for a wrench and submerged in saltwater produced two appropriate results, ``316 Stainless Steel Wing Nuts'' and ``316 Stainless Steel Knurled-Head Thumb Nuts'', where the correct form and material was identified. Asking for a tamper-proof nut also produced the correct search, ``316 stainless steel tamper-resistant nut''. 
%\liane{not a suggestion, just noting that this is a cool and very useful experiment! Never would have thought of this use case.} \amy{:) thanks!}
Next, we tried a more open-ended geometric compatibility scenario by asking for parts for an at-home carbonation system (\autoref{fig:carbonation-parts}). We also then asked it for a comprehensive Bill of Materials. It seemed as though all parts were compatible, at least geometrically; we suspect this is because the items in the domain are standardized for compatibility, McMaster-Carr's dataset is quite rich, and there is great availability of each part across varying sizes. 


% Figure environment removed


\paragraph{Parts of mechanisms} Part of what makes \gpt a compelling tool for design is its simple user interface. A user might interact with \gpt by describing a desired functionality and asking what parts would be necessary to achieve it. For example, we described a hypothetical bar cart with two features: a lower shelf with rails, and a tabletop where a portion could be folded down for compact storage. We asked \gpt to tell us what the name of the fold-down tabletop mechanism was, and recommend a part that could be used to build it. It correctly identified the function as a drop-leaf mechanism, explained that since the drop-leaf would be 20x15 inches, the mechanism should be at least 15 inches long, and named steel or brass as appropriate materials. It also was able to generate a specific search term for the part. However, \gpt did not recommend a particular type of mechanism in how it moved or functioned. %\liane{maybe split the sentence here? or is there a dependency that I'm missing between the clause before and the clause after?} \amy{No, good catch!}
We asked it to list the different sub-types and their use cases, which it did successfully, naming and differentiating a swing arm bracket, a slide-out support, a hinged bracket support, support bars, and a rule joint. We were able to find examples of four of the types, but the support bars, which were described as ``lengths of wood or metal that are stored separately and inserted into brackets on the table and leaf to hold it in place'', did not seem to exist under that terminology or perhaps at all. We then asked it to recommend a type for our use case, and it recommended the swing arm or hinged bracket supports.

% Not sure if this is a strong example.
We also tried a loom example, where we asked \gpt to provide a fabrication plan for a 4-shaft table loom, and asked about the name of the mechanism that lifts and lowers the heddle frames and the names of specific parts that make up this mechanism. In general, it was accurate, but \gpt sometimes erroneously named components that only pertained to countermarche looms or floor looms instead of table loom-specific parts. We speculate that this could be due to a dearth of literature on loom construction in \gpt's training dataset. 

Our examples show potential in using LLMs to identify and source parts, with major caveats. We note a recurring theme of \gpt's ability to produce programs that generate valid programs or instructions to verify validity, and its inability to apply those rules to its own output. In general, we found that we could ask for general, pointed, and precise guidance with great success, but asking for product names or specific items often resulted in incompatible or nonexistent parts lists. Furthermore, best results were produced in the simpler and more common domains, or when the domain we were querying had very rich information, as was the case with McMaster-Carr. We believe that \gpt is useful for making comprehensive checklists, and can lend domain expertise and suggestions, so long as all information can be checked or cross-referenced. Since \gpt can interface across many levels of jargon, experts may derive the most value from its use currently, given that they are best able to make common sense checks over the output. For non-domain experts, \gpt delivers very convincing, confident information that can be incorrect. LLMs are poised to become a powerful ``design for everyone'' tool, but more verification steps are needed to guide novice users. % something about hedging and being less confident


\subsection{Create Manufacturing Instructions (Q4)}
Computer-Aided Manufacturing (CAM) is a technology that utilizes software to generate manufacturing instructions from digital design files. It plays a vital role in the efficient and accurate translation of design concepts into tangible products. CAM bridges the digital design and the physical manufacturing stages, enabling seamless communication and translation of design specifications into machine-readable instructions. CAM encompasses a range of techniques and tools that leverage computer systems to automate various manufacturing processes, including planning, toolpath generation, and machine control. By utilizing CAM, manufacturers can streamline production, improve precision, and enhance overall efficiency. In this section, we delve into the creation of machine-readable and human-readable manufacturing instructions with help of \gpt and open-source CAM software. Specifically, we explore additive, subtractive, and assembly manufacturing processes, highlighting the capabilities and challenges associated with each approach.
\subsubsection{{Additive}}
\label{sec:additive}
%\adriana{this section is not formated in the same way as the others - please cut some of the introduction, don't use bullet points and reference the figures as part of the text}
Additive design, often employed in the realm of 3D printing, can be time-consuming and labor-intensive, requiring spatial reasoning, precision, and multiple iterations. We posit that \gpt will improve this process, as it comprehends complex specifications in natural language, generates designs efficiently, simulates outcomes, and explores innovative possibilities from diverse sources, optimizing functionality and aesthetics.

We first try to directly use \gpt to generate the G-code from a natural language description. However, due to the complexity and length of G-code, \gpt fails to output complete code that precisely models the specified shape, as shown in Figure~\ref{fig:gcode-fail}.
To overcome this, we have developed a two-stage approach.

% \wojciech{This is still too general. It does not fit with the rest.}
% Additive design, often employed in the realm of 3D printing, carries inherent challenges when undertaken manually. It requires an extensive understanding of spatial reasoning, detailed precision, and meticulous planning. Manual additive design can be time-consuming and labor-intensive, as it demands a hands-on approach for each aspect of the design process. There's also the need for numerous iterations and adjustments to optimize the design, as small errors in manual execution can lead to significant flaws, negatively affecting the functional and aesthetic aspects of the final product.
% Transitioning to a \gpt based approach to additive design can drastically enhance the process. \gpt, with its advanced language and problem-solving capabilities, can comprehend and execute complex design specifications expressed in natural language, providing an intuitive interface for designers. It can process and generate design models more efficiently, thereby reducing the time and labor needed. \gpt can also help simulate the outcomes and foresee potential issues within the design stage itself, avoiding the wastage of resources that could occur in manual iterations. Moreover, the AI model can help explore innovative design possibilities by learning from vast datasets, incorporating functionality and aesthetics from diverse sources, and hence promoting design innovation.
% In our approach to the design-to-manufacturing problem, we first need to understand its formulation. This problem can be split into two primary components: the input and the output.
% \begin{itemize}
% \item \textbf{Input}: The input side of this formulation involves a shape parameterized by natural language. That means, we should be able to describe the desired shape or design using everyday language. This allows for a more intuitive interaction with the design process, as designers or users can express their design intentions in a familiar, easy-to-understand format. Alongside the shape, the input also includes hardware-related configuration parameters, such as the nozzle diameter in a 3D printing scenario. This enables the system to adapt to different physical constraints imposed by the hardware used in the manufacturing process.
% \item \textbf{Output}: Transitioning to the output side of the formulation, this involves the generation of the G-Code. For those unfamiliar, G-Code is the language used by most automated machine tools like 3D printers. It essentially instructs the machine how to create the specified design. So in this problem, the output is the G-Code that corresponds to the specific shape and hardware configurations defined in the input stage. This integration between the design and manufacturing stages, mediated by G-Code, provides a seamless pipeline from conception to creation, enabling a more efficient and coherent process.
% \end{itemize}
% To make the process more manageable and efficient, our approach to the design-to-manufacturing problem is divided into two distinct but interconnected stages. This structure allows for an organized progression from the initial design concept, specified in natural language, to the final manufacturing instructions in the form of G-Code.
% \begin{itemize}
% \item Convert the shape described in natural language into an intermediate 3D shape representation. To achieve this, we've chosen to use a triangle mesh as our representation. Triangle meshes offer significant advantages due to their ability to generalize complex shapes, capturing detail and structure in a compact form. To manage these meshes, we employ the Python library trimesh. This open-source tool provides a robust and flexible platform for storing and manipulating triangle meshes, allowing us to effectively process the shape data extracted from the natural language input.
% \item Translate this intermediate representation into G-Code, which will be tailored to the specific hardware configurations provided. This step requires deep domain knowledge of fabrication processes, which is why we have elected to use slic3r, a professional G-Code generation software, as the backbone of this phase. Through Python, we can directly interface with slic3r, ensuring we generate high-quality G-Code that can accurately and reliably guide the manufacturing process.
% \end{itemize}
% The entire pipeline is seamlessly realized with the help of \gpt. The underlying conversation between these modular components is maintained using this advanced language model, providing a cohesive flow of data and instructions across the pipeline stages. By leveraging the capabilities of \gpt, we can make this pipeline more efficient and effective, ultimately yielding superior manufacturing outcomes. 

% Figure environment removed



\paragraph{Stage I} We transform the concept expressed in natural language into an intermediate 3D shape representation using triangle meshes. This choice provides compact and comprehensive representations, capturing intricate details accurately. Leveraging the Python library trimesh, we effectively manage and process the shape data extracted from the natural language input (\fref{fig:trimesh}).

% Figure environment removed

\paragraph{Stage II} We translate this intermediate representation into G-Code, customized for the specific hardware configurations at hand. This critical step demands deep domain expertise in fabrication processes, which is why we rely on slic3r \cite{slic3rSlic3rOpen}, a professional G-Code generation software. Through Python integration, we interface directly with slic3r, ensuring the production of high-quality G-Code that precisely guides the manufacturing process. In \fref{fig:gcode}, we visualize the output G-Code using \texttt{Repetier}\cite{repetierRepetierSoftware}, a manufacturing tool, to validate the fabrication pipeline.

% Figure environment removed

Throughout the entire pipeline, the cohesive communication between these modular components is facilitated by the powerful capabilities of \gpt. As an advanced language model, \gpt maintains a seamless conversation, ensuring a smooth flow of data and instructions across the various stages of the pipeline. By harnessing the potential of \gpt, we optimize the pipeline, achieving enhanced efficiency and superior additive manufacturing outcomes.

\subsubsection{Subtractive}
\label{sec:designForManufacturing_Subtractive}
Subtractive manufacturing is a widely used technique that involves removing material from a workpiece to create the desired shape or form. This process is commonly employed in various industries, including woodworking and metal fabrication. Leveraging the power of \gpt, we explore how this approach can be enhanced and streamlined to achieve optimal results.

To demonstrate the design-to-subtractive manufacturing process, we focus on the previously designed cabinet (\fref{fig:textToDesign_simpleCabinet}) and employ a laser cutter and wood pieces for fabrication. Specifically, our goal is to translate the \jscad design into precise manufacturing instructions.
To tackle this task, we simply provide \gpt with the \jscad code and request the generation of laser cutting patterns in DXF files. \gpt showcases an understanding of the cabinet's fundamental geometry relationships and topological structure. It recognizes that the 3D cabinet comprises various 2D boards, including top and bottom boards, a shelf board, side boards, and back boards (\fref{fig:cabinet1}). However, \gpt encounters challenges when accurately determining the dimensions of the 2D cutting patterns based on the given 3D geometry input. Some inaccuracies arise, such as confusion between the cabinet's depth and the board thickness, resulting in overly thin side boards. Additionally, distinguishing between height and width in the 3D context presents difficulties, leading to back boards that are too short. Lastly, \gpt struggles with precise hole positioning (\fref{fig:cabinet1}).

% Figure environment removed

To address these errors, human intervention becomes essential in explicitly identifying the issues and proposing potential solutions (\fref{fig:cabinet2}). After a round of communication, \gpt successfully generates the correct DXF files for laser cutting. To ensure their validity, these files were verified by human experts.

% Figure environment removed

\subsubsection{Assembly}
We conducted an experiment to explore the potential of \gpt in generating assembly instructions that are both machine-readable for robots and human-readable as standard operating procedures. The experiment focused on assembling a wooden box using a specific set of tools and materials. In \fref{fig:Machine_Assembly}, we presented the prompt for generating machine-readable instructions, which involved creating a set of functions to specify different tasks for the robot and generating corresponding sequences to execute those tasks. Since the functions were designed to be system-agnostic, the response from \gpt printed the actions performed by the robot.

Subsequently, we prompted \gpt to generate a standard operating procedure to convert the machine-readable instructions into human-readable text. This procedure provides a detailed description of the assembly process, enabling humans to follow along and understand the steps involved. By generating both machine-readable and human-readable instructions, we sought to assess the versatility and applicability of \gpt in facilitating effective communication and collaboration between robots and human operators in assembly tasks. 


% Figure environment removed

\subsection{Discussion}
In this section, we elaborate on the key capabilities (C), limitations (L), and opportunities (O) previously outlined, particularly as they relate to the domain of text-to-design.

\noindent \textbf{C.1 Extensive Knowledge Base in Design and Manufacturing:} We have discovered that \gpt possesses an understanding of various manufacturing processes and their capabilities including CNC machining, injection molding, additive manufacturing, and laser cutting.  Moreover, it is able to apply this knowledge to various problems in design for manufacturing. Although it is not consistently accurate, it can utilize this knowledge to offer suggestions about what is the best manufacturing practice to use, if certain geometric features will be hard to produce. Moreover, because \gpt has the ability to generate code, it can be utilized to modify geometry directly and generate manufacturing files based on supplied files. 

Additionally, we have discovered that \gpt possesses the capability to search for parts that fulfill a desired functionality as described to it. This allows it to be used to source parts based on a description, geometry, functionality, and performance. 

\noindent \textbf{C.2 Iteration Support:} 
\gpt also possesses the ability to perform iterative debugging when creating and modifying files required for manufacturing. This enables the opportunity to iterate when prompts are not ideal for generating the desired outcome or when \gpt generates something incorrect.

% \noindent \textbf{c.4 Search Support:} % propose moving to C.1

\noindent \textbf{L.1 Reasoning Challenges:} Our observations indicate that \gpt exhibits constraints in quantitative reasoning. For instance, when tasked with generating manufacturing instructions, \gpt struggled to accurately perform basic calculations for tool path placements. However, this limitation can be mitigated by employing symbolic computations within a script. A case in point: we achieved accurate DXF file generation by designing a script to produce the file, instead of having \gpt generate the file directly.

\noindent \textbf{L.2 Correctness and Verification:} We have found that \gpt will provide incorrect information about manufacturing processes in some cases.  For example, when selecting a manufacturing process, it proposed injection molding as an optimal manufacturing process for a PTFE part which is incorrect. We have not found a solution to this in this work to resolve \gpt giving incorrect information. 
