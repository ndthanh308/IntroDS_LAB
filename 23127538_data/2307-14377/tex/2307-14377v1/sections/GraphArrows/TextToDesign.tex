\section{Text-To-Design} 
\label{sec:text_to_design}

% A pivotal step in getting a large language model (LLM) to generate a design involves crafting a prompt that clearly delineates the design language to be employed. This encompasses the specification of both the language's syntax and the semantics associated with each operator, presented in a manner that the LLM can comprehend. 

% We usually accomplish this via a three-step process. We first develop a succinct language specification that defines the operators and outlines their respective inputs and outputs.
% Following this, we demonstrate an example employing this language to devise a simple model.
% Finally, we incorporate additional semantic information about the model or design guidelines that can assist the LLM.

% Next, we will delve into how to establish design prompts across four domains: 2D vector drawings, 3D parametric geometry, actuated systems, and electronics. Each of these domains adheres to similar specifications. In each of these domains, we discuss both prompt generation and algorithms for converting outputs from the LLM to standard formats 


% \wojciech{Too much focus on spatial}
For our first line of inquiry, we explore the extent to which \gpt is able to generate designs across a variety of domains. 
Even within the specific context of manufacturable design, the concept of a ``design'' is quite broad, and exists at many scales.
For example, we may want to specify a single self-contained part, or a sizable hierarchical assembly containing several levels of sub-assemblies and/or other individual component modules. 
Such assemblies may be completely customized/self-contained, with all parts designed simultaneously, or they may be hybrid designs that integrate existing, pre-manufactured elements such as brackets or motors.
In many cases, our target design tasks also include dynamic considerations such as assembly mating or articulated joints.

Although these complex tasks may initially seem out-of-scope for lexical models such as {\llm}s, there are many modeling and design paradigms that can be expressed in terms of potentially-\llm-compatible language.
To guide our exploration of \gpt's ability to interface with each of these models, we pose the following questions:

\begin{itemize}
\item \textbf{Q1} Can \gpt generate a meaningful design when provided with a high-level description of the goal and a given modeling language?
\item \textbf{Q2} To what extent is the user able to control the designs created by \gpt? Is \gpt able to interpret and respect user-defined constraints, such as spatial relationships between objects or integration of standard pre-fabricated parts?
\item \textbf{Q3} Is \gpt able to incorporate high-level abstractions used by human designers, such as modular (de)composition?
\end{itemize}

% \wojciech{For the copter example, you first need get the appropriate components (e.g., What are the main components of the copter?). There should be a question on this.}\liane{part sourcing is addressed in the manufacturing section. All examples in this section explicitly assume that we already know the parts we want (we also make an explicit reference to the fact that we'll explore gpt sourcing in section 6.}


\subsection{Simple, self-contained designs from high-level input (Q1)}

% Can \gpt successfully instantiate and coordinate (a set of)  primitives from a given modeling language in order to build/approximate a meaningful design? Does \gpt seem particularly well- (or poorly-) suited to any particular modeling paradigm(s)?
%Can \gpt generate a meaningful design when provided with a high-level description of the goal and a set of primitives from a given modeling language?


To explore \gpt's capacity for design, we first test its ability to do one- (or few-) shot generation of an object from a minimal high-level text description as input. 
Ideally, we would like to understand \gpt's ability to complete design tasks independent of any particular modeling paradigm.
However, it is not immediately clear how much dependence there may be on the specific representation that is chosen, because the variation in possible language-based modeling paradigms is significant. 
Some languages are very general and versatile, with a wide variety of features and capabilites, while others may be highly-specialized for a specific set of tasks or outcomes. 
Similarly, some languages are well-established with plentiful online documentation or examples, while others may be custom-defined, poorly documented, or otherwise underrepresented in \gpt's training repository. 
Finally, some languages are fairly streamlined, while others may be syntactically complex and/or require the use/coordination of many modules.
Each possibility offers unique capabilities and challenges. 
Thus, we set out to test a wide variety of them, in an effort to determine 
\llms' ability to use each representation;
whether there are any conclusions that seem to span across different representations;
and
whether any particular representations seem uniquely well- or poorly-suited for \llm integration.


% \liane{for an initial draft of each section, it'd be great to list out e.g. what kind of representation you're using (if \gpt needs to use an API/function calls, generate a mesh directly, come up with a mix of discrete components/continuous values, etc.); how much you had to tell/teach \gpt vs. whether it could just use the API directly; the examples/experiments you tried; what \gpt was good at/struggled with; and include [not necessarily polished] figures of your results. Things might shuffle around, but this will give us a good starting point}


\subsubsection{Vector Graphics with SVG/DXF} 
%\team{Mike}


% Our first design domain is 2D vector graphics.
% Vector formats like SVGs or DXFs are commonly used to describe manufacturing files, 
% for example for laser cutting.
% We want to investigate whether \gpt could empower designers to convert text directly into vector files, 
% readily usable for manufacturing.
% To test this, we have experimented with generating an SVG file that could be used to create a manufacturing file. 
% We also experimented with converting a DXF format of the same design.  

% In the experiment, the goal was to design and construct an SVG file of a cabinet with specific dimensions using 1/2 inch plywood. The cabinet was intended to have three shelves, with the cabinet having overall dimensions measuring 6 feet in height, 1 foot in depth, and 4 feet in width. The experiment aimed to accurately account for the thickness of the plywood when designing the cabinet, ensuring that the dimensions of the various parts were adjusted accordingly. To achieve this, a Python script was developed to generate an SVG file representing the cabinet layout using \gpt. The script calculated the required clearances for the wood thickness and appropriately positioned the side panels, top and bottom panels, shelves, and back panel, while considering the specified spacing between the parts and utilized svgwrite to generate the SVG file. The resulting SVG file provided a visual representation of the cabinet's design that could be used for cutting out the parts. Similarly, we repeated this experiment to create a DXF file where \gpt utilized ezdxf to generate the file. Results are shown in Figure \ref{fig:SVG_DXF_Gen}.  \gpt was able to use the APIs to generate the file in the correct format without any simplification, however, multiple iterations were needed to ensure \gpt did not overlap the parts of the cabinet.  

% \liane{is there something specific we want to highlight from the chat excerpt? It's very long right now and I'm not sure what is the interesting part to look at / message to take away. also, would be nice to have a visual of the svgs} \adriana{yes, this figure is strange}\bolei{probably it is better to show the patterns of svg files instead of the code?}
% \wojciech{This is written as a manufacturing section not design section.}

Our initial focus in the design domain is on 2D vector graphics.
Vector formats such as SVGs or DXFs are prevalently utilized in manufacturing processes,
like those for laser or waterjet cutting.
The goal of our investigation was to ascertain whether \gpt could empower designers to 
transform their text directly into these vector formats. 
To evaluate this, we conducted experiments to determine 
if \gpt is capable of generating a valid SVG file and converting the design into DXF format.

The primary aim of our experiment was to design an SVG file for a cabinet, 
with predetermined dimensions, to be constructed from 1/2 inch plywood. 
This implies that the thickness of each wall, a preset parameter, is 0.5 inches.
The experimental setup involved the design of a cabinet comprising three shelves, 
with overall dimensions measuring 6 feet in height, 1 foot in depth, and 4 feet in width.
A crucial aspect of the investigation was to see if \gpt could accurately account for this wall thickness 
during the design of the cabinet, appropriately adjusting the dimensions of its various components.
\gpt was able to design the specified cabinet and 
subsequently generated a Python script to create an SVG file reflecting the cabinet's layout.
The script considered the necessary clearances for the thickness and 
accurately positioned the side panels, top and bottom panels, shelves, and back panel.
Moreover, it factored in the prescribed spacing between parts and 
leveraged `svgwrite' to generate the SVG file. 
The resulting SVG file provided a visual depiction of the cabinet's design.
We also replicated the experiment to create a DXF file, 
where \gpt utilized `ezdxf' to generate the file.
The results of these experiments are depicted in Figure \ref{fig:SVG_DXF_Gen}.

In conclusion, \gpt demonstrated its capability to employ the APIs for 
generating the vector file in the correct format without any simplifications.
Nevertheless, it was necessary to perform several iterations to 
ensure \gpt did not cause any overlap among the cabinet parts.

% Figure environment removed
\afterpage{\FloatBarrier}



\subsubsection{CSG with self-defined primitives}
\label{sec:textToDesign_CSG_boxes}
%% Text to Design: CSG with self defined primitives
% \team{Bohan}

The next design domain we are investigating is CSG.
As outlined in Sec.\ref{sec:overview_domains_design}, CSG languages generally operate by building up a collection of primitives that have been altered or combined via linear transformations and Boolean operations. 
Because the associated design logic can be quite complex, it was not immediately clear that \gpt should be able to generate designs using these languages.
Thus, to progressively test \gpt's modeling capabilities, we begin by exploring a very simple, custom CSG language based on a single primitive: a box.

Boxes are one of the most common primitives seen in manufacturing. 
Moreover, many shapes can be considered as a combination of boxes with different sizes.
Because of the simplicity of a box or any shape formed by the boxes, 
we would like to see if \gpt is able to generate designs of such kind of simple shapes, 
such as tables and chairs.

Our initial approach to this task is performed in 2D.
We provide a function, foo(x, y, w, h), 
which forms a box of dimensions $w \times h$ centred at the point $(x, y)$. 
We subsequently employ this function to generate letters composed of axis-aligned bars, 
such as `F' and `E'.
During the testing phase, we observed that 
while the system understands the requirement of 2D boxes, 
it struggles with their accurate placement. 
A particularly prominent issue is the collision problem. 
More specifically, the \gpt system fails to determine 
whether two boxes are overlapping or whether there is a vacant space between them. 
This issue is observable when creating letters like `T' and `E'.
Using three to five targeted prompts enabled \gpt to ascertain the correct positions.
However, these prompts had to be granular and often involved providing the direct solution. 
The outcomes of these attempts are demonstrated in Figure~\ref{fig:letter}.
Interestingly, after addressing this issue, \gpt appears to retain the corrections. 
This is evidenced by its successful generation of the new letters `F' and `L' in a single attempt. 
These letters share a similar structure to `T' and `E', 
and the results can be seen in Figure~\ref{fig:letter}.

Our next step involved venturing into 3D, 
which holds more practical values. 
Analogous to the 2D scenarios, 
we inform \gpt of a pre-established function, box(x, y, z, w, h, d), 
which generates a 3D box of dimensions $w \times h \times d$ centred at the 3D coordinates $(x, y, z)$.
We then tested if \gpt could write a program to produce a simple box of specified dimensions, 
for instance, $100 \times 100 \times 40$, utilizing function `box'. 
\gpt successfully accomplished this task, and the resulting text explanation 
illustrates its understanding of the box concept and the usage of our predefined function.
Next, we presented a more complex challenge: having \gpt design a simple table, 
typically consisting of four legs and a tabletop in the real world. 
We posed the question of whether \gpt could craft a program to
generate such a table with a provided size using solely our box function.
The output text explanation revealed that 
\gpt accurately comprehends the structure of a basic table.
Given that we only provide the overall table size, 
\gpt lacks information about individual leg lengths or tabletop thickness. 
Yet, it was able to identify these missing parameters and make reasonable assumptions. 
Consequently, \gpt succeeded in writing a program to represent 
the table by creating five boxes using our predefined function.
Upon visualizing the 3D table, however, 
the relative positioning of each pair of boxes was not always accurate.
We noticed that the tabletop appeared to be suspended in the air, 
not in contact with the legs, as shown in Figure~\ref{fig:tabletable}. 
This difficulty, also observed in our 2D tests (Figure~\ref{fig:letter}), 
pertains to \gpt's understanding of mathematical concepts.
In this instance, we expedited the process by directly providing \gpt with the solution.
We indicated the necessary translations for the misplaced boxes, 
acknowledging that it would take several prompts to rectify the issue otherwise. 
After correcting the floating tabletop, the table appeared as intended, as demonstrated in Figure~\ref{fig:tabletable}.
Therefore, to create a table, it only required two prompts, 
significantly streamlining the procedure for generating a basic table.

% Figure environment removed


Once we successfully generate the table, 
our next more challenging goal is to design a few accompanying chairs.
We tasked \gpt with creating a chair compatible with the table, using only our predefined function. Similar to its approach with the table, 
\gpt successfully deduced the basic structure of a simple chair, 
comprising the seat, four legs, and a backrest.
Unlike the table instance, we didn't observe any `floating' issues in this scenario.
It appears that \gpt might have indeed gleaned some insights from previous experiences,
as we also observed when creating 2D letters. 
After we rectified the letters `T' and `E', 
there were no issues with the remaining letters.
Additionally, \gpt demonstrated comprehension of the concept of 
compatibility by outputting a chair of an appropriate size.
However, it was not successful in all aspects, as depicted in Figure~\ref{fig:chairchair}. 
We attempted to correct the backrest but were unable to do so.
As a result, we had to manually adjust the position,
directing \gpt to the specific lines that needed modification to correct the structure.
The final result can be seen in Figure~\ref{fig:chairchair}.
We believe the root of these issues lies in \gpt's struggles to comprehend geometric concepts,
a difficulty also observed in previous examples.
Despite these hurdles, the process for creating a basic table and chairs
has been considerably simplified.

Our final objective was to position four identical chairs around the table. 
Although theoretically feasible without invoking rotation, 
\gpt failed to generate the chairs with the correct orientations.
We believe this failure stems from the same root cause we've encountered previously, 
namely, \gpt's difficulty in handling mathematical and geometric concepts.
Creating four chairs with correct orientations without the support of rotation 
entails complex geometric transformations. 
\gpt must comprehend that a box rotated 90 degrees 
around its center is equivalent to a swap of its width and depth dimensions.
To alleviate this issue, we expanded our `box` function to include an additional input argument, `angle`, corresponding to a rotation angle around the vertical axis.
With this extension, \gpt was able to create a program using solely the `box` function 
that successfully positioned four chairs around the table with correct orientations, 
as displayed in Figure~\ref{fig:chairchair}. 
We surmise that the introduction of `angle` considerably simplifies the logic behind chair placement, 
enabling \gpt to create such a program.

In conclusion, \gpt exhibits strong understanding of posed questions 
and excels at analyzing requested objects to determine their composition.
However, it demonstrates a weakness in handling geometric and mathematical concepts.
While it can provide nearly accurate solutions when mathematics is involved,
it struggles to comprehend the underlying mathematical principles and, 
as a result, cannot independently correct math-related issues when they arise.


% Figure environment removed

% Figure environment removed



\subsubsection{CSG with PyVista}
%% Text to Design: CSG with PyVista
% \team{Yifei}

Building on \gpt's success generating CSG-like models with boxes, we set out to explore \gpt's capacity to use a larger suite of primitives.
For this, we used an existing 3D visualization library, PyVista, which allows us to create and place a variety of 3D primitives such as spheres and cones.
Thanks to the library's documentation, \gpt is able to automatically assemble a functional python program using PyVista's primitive functions.

We asked \gpt to use PyVista's primitives to model several variations of a fish, including specific bio-inspirations such as goldfish, a manta ray, and a loach (\fref{fig:fishes}). 
\gpt successfully selected and scaled an appropriate set of primitives for each example, and provided sound bio-inspired rationale for its decisions.
In particular, although most of the fish are composed using a sphere for the body, \gpt intuits that a loach would be most effectively approximated by using \lstinline{two cones for the body to give it an elongated shape}.

One area in which \gpt struggled was the determination of the primitives' orientations. 
It often produced results that indicated an internal confusion of some of the axes, or an otherwise flawed approximation of the orientation that would be required to achieve a desired effect.
After engaging in a dialogue with \gpt, it was able to rectify the orientations of the primitives to more closely resemble the target creatures.
While promising, these tests reiterate \gpt's seemingly limited capacity to account for local coordinate frames.

% GPT's rationale:
% \ignore{
% Goldfish: "This code creates a fish that has a larger, more rounded body and smaller, more forward-positioned head, which are more characteristic of a goldfish. The fins are also larger and positioned higher and lower on the body to mimic the large, flowing fins of some types of goldfish."
% Manta ray: "A manta ray has a flatter, wider body and long "wings", or pectoral fins. The tail is thin and long. We can use a large, flat sphere for the body, a thin, wide cone for the pectoral fins, and a long, thin cone for the tail."
% Loach: "A loach is typically elongated and has several fins along the body. We can create a simple representation using multiple cones for the body and fins. For simplicity, let's use two cones for the body to give it an elongated shape, and additional cones for the fins."
% }




 % \adriana{I think the images in this section are taking too much space, I propose a single image with all 4 results as it's not adding much to the discussion.}

% Figure environment removed


% % Figure environment removed

























% =====================
% OLD TEXT
% =====================

\ignore{
\FH{start of old text}

In this conversation, we explored using 3D mesh library to generate and visualize various designs based on constructive solid geometry (CSG) principles  with the aid of \gpt. 
The integration of PyVista, a 3D visualization library, with large language models facilitates an interactive, real-time computational design process. 
This approach simplifies 3D operations and democratizes design by allowing natural language inputs, fostering broader accessibility. 
The dynamic interaction enables efficient exploration of design parameters and serves as an instructive tool for understanding 3D modeling and computational design. 

We started by creating simple 3D primitives such as spheres. 
\gpt is able to assemble a python program automatically using PyVista's primitive functions that meets our design goals.  
Then, we expanded upon this concept to model fish (Fig.~\ref{fig:fish_and_variation}) and its variations with different bio-inspirations such as goldfish (Fig.~\ref{fig:fish_goldfish_variation}). 

We initially attempted to use boolean operations like union to combine different primitives. 
However, this process encountered some issues as PyVista's boolean operations require all the meshes to be composed of triangles, which wasn't the case with some of the primitives used by \gpt. 
This could result from the model being trained on outdated library version.

As a workaround, instead of using boolean operations, we explored a different approach where we directly added all the part meshes to the PyVista plotter for visualization. 
This method effectively bypassed the requirement for the meshes to be triangulated, enabling us to create more complex shapes from primitive elements.

Using this method, we generated a series of parametric designs, demonstrating the flexibility of this approach. We experimented with a variety of aquatic creatures, including a generic fish, a goldfish, a manta ray~\ref{fig:fish-mentaray}, and a loach~\ref{fig:fish-loach}. During the process, we discovered that \gpt has confusion in terms of how to correctly orient individual primitives to achieve desired looks, and we have to manually tweak orientations of the primitives to closely resemble the target creatures.

In conclusion, our exploration highlights the potential of large language models in aiding computational design. 
With an intuitive and interactive dialogue, we can generate a wide range of designs from basic primitives using libraries like PyVista. 
While certain limitations exist, such as \gpt's failure to meet the requirement for triangulated meshes in boolean operations (which could result from being trained from old codebases) as well as confusion in coordinate systems, creative solutions can be found to bypass these and generate complex and versatile designs. 
This shows that with the right tools and approach, large language models can significantly aid in the computational design process, making it more accessible and intuitive.
}

\ignore{
% Figure environment removed
\afterpage{\FloatBarrier}
}


\subsubsection{CSG with \jscad}
\label{sec:textToDesign_JSCAD_basic}
%% Text to Design: CSG with OpenJSCAD
% \team{Liane}

To explore a full-fledged approach for \llm-aided CSG, we test \gpt's ability to generate meaningful designs using the open source javascript-based CSG library, \jscad \citep{jscad}.
\jscad has extensive documentation available online, and we found that \gpt natively possesses a good grasp of the API, its components, and the required code structure.
In particular, it understood that it needed to import each function from the corresponding modules, and that it needed to define and export a function named \lstinline{main}.
For our experiments, we provided \gpt with access to the full API, and generally allowed it to select the appropriate primitives and operations without user interference. 

To test \gpt's design abilities, we ask it to design a simple cabinet with one shelf, as shown in \fref{fig:textToDesign_simpleCabinet}.
\gpt reliably selects and instantiates the required primitives, along with intuitive naming conventions and structure within the \jscad code. 
\gpt's initial orientation of the parts was also generally reasonable, but the specific positioning of each part was often incorrect. 
Despite multiple attempts, \gpt was unable to generate any fully-correct cabinet in a single shot, with no subsequent user intervention.


Moreover, \gpt frequently produced highly disparate results from one run to the next. Even when using an identical prompt on fresh chat environments, \gpt's responses varied widely in terms of their overall code structure, design accuracy, and the specific errors or oversights made. 
\fref{fig:cabinet_vertical_explosion} shows one example of a drastically different design process, even when seeded with the same initial prompt as \fref{fig:textToDesign_simpleCabinet}.

Throughout our experiments, we found that \gpt encountered a few common pitfalls when generating designs in \jscad.
Occasionally, \gpt made small syntatic errors such as generating incorrect boilerplate, importing functions from incorrect modules, or making ``typos'' in API calls -- \eg, trying to import from the \lstinline{boolean} module rather than the correct \lstinline{booleans} module, or calling the \lstinline{cube()} function with parameters that were intended to generate a \lstinline{cuboid()}. 
In an attempt to avoid these pitfalls, we created a small list of ``hints''/``reminders'' for best practices when working with \jscad; this short list was always passed in alongside our initial prompt. See \appref{sec:appx-jscad-hints} for a full listing of these reminders.
Although these reminders seemed to help mitigate these issues, we were unable to eradicate them entirely. 
However, \gpt can easily correct the majority of these issues when they were pointed out by the user.
Often, the process of correcting the issue through prompts and responses was faster than actually adjusting the code manually, making {\llm}s a useful design partner.

One pervasive issue that seemed more difficult to correct was the fact that \gpt had issues positioning the primitives in 3D space.
In particular, \gpt frequently seemed to forget that \jscad positions elements relative to the \textit{center} of a given primitive, rather than an external point on the primitive (\eg, the lower left corner). 
\gpt's arrangements were frequently incorrect due to this issue.
When \gpt is reminded of this convention, it does generally alter the design, but it is not always able to correct the issue. 
If sufficiently many rounds of local edits prove unable to address the alignment issues, we found that it was generally more effective to direct \gpt to disregard all existing measurements, and re-derive the elements' positions from scratch (see \fref{fig:cabinet_vertical_explosion}).

Overall, we find that \gpt is able to generate reasonable \jscad models from high-level input.
However, the design specifications that emerge on the first attempt are rarely fully correct, so users should expect to engage in some amount of corrective feedback or iteration in order to attain the desired result.

% Figure environment removed
%\afterpage{\FloatBarrier}

\newcommand{\cabVertExpImHeight}{2.2cm}
% Figure environment removed

\subsubsection{Sketch-based CAD with OnShape}
\label{sec:textToDesign_OnShape_basic}
%% Text to Design: CAD with OnShape
% \team{Felix}

Another popular method for 3D shape modeling comes from contemporary computer-aided design (CAD) software.
Rather than directly constructing and modifying solid primitives (as in the CSG approaches discussed above), modern parametric CAD systems generally work by lifting planar sketches into 3D and subsequently modifying the 3D geometry.
These sketches are placed on planes, which can be offsetted construction planes, or planar faces of the current 3D model.
The selected sketching plane serves as a local coordinate system in which the sketch primitives are defined.
In graphical user interfaces, this change of coordinate systems is accounted for by letting the user easily align their camera view to a top down view onto the sketch plane.
This change of view effectively comes back to drawing sketches in 2D, removing the cognitive burden of having to think about sketches in 3D.
Despite the lack of graphical assistance, we want to investigate whether \gpt is able to design objects using a sketch-based modeling language.

However, since the graphical assistance is very prevalent in this modeling paradigm, CAD models are mostly constructed via a GUI and not via textual programming, even though textual APIs exist, e.g. Onshape's Featurescript \cite{featurescript}.
Therefore, documentation and examples are less available than for the modeling paradigms from the previous sections.
And indeed, \gpt performs poorly when trying to generate Featurescript code directly, which is why we decided to provide a simplified DSL.

For our experiments, we constructed a single prompt containing the following DSL description:
Our DSL exposes two operators, \lstinline{createSketch} and \lstinline{extrude}, and two sketch primitives, \lstinline{circle} and \lstinline{rectangle}.
Additionally, we provide a construction example using this language of a single leg round table.
Lastly, we also add some hints about how to write the program, e.g. to explicitly use design variables and to write in \lstinline{syntactically correct python}.
All of the output designs generated by \gpt in this section are automatically translated into Onshape PartStudios.
The full prompt can be found in the supplemental material. 
% Figure environment removed

Our first task is the design of a \lstinline{chair with 4 legs, a rectangular seat and a rectangular back}, see Fig.~\ref{fig:cad_chair_design}.
We asked \gpt to perform this task several times and observed the following.
\begin{itemize}
    \item The design sometimes includes cylindrical legs, sometimes rectangular legs.
    \item The design is always constructed in a single direction, the $Z$ direction.
    Our input example of the round table only used the $Z$ direction to select sketching planes, but the description of our language documented the use of other plane directions.
    \item We observe mainly two types of designs: (i) designs which are constructed in both the negative and positive $Z$ direction starting from the seat, see first answer in Fig.~\ref{fig:cad_chair_design}, and (ii) designs which start from a leg, see the second response in Fig.~\ref{fig:cad_chair_design}.
    We observe that the first type of designs has a higher chance of being correct, whereas the second type fails more often.
    The failures are due to changes in the coordinate system.
    For example, when selecting the top plane of the first leg as a sketch plane for the seat, the sketch plane's origin will be in the center of the leg.
    \gpt will often ignore this or won't be able to account for it when pointed out.
    Conversely, when starting with the seat and choosing the lower seat plane as a sketch plane for the legs, it can specify the leg sketch coordinates in global coordinates, since the global origin coincides with the seat's origin.
    The same is true for the backrest.
    
\end{itemize}

From this test, we can observe that \gpt seems to have difficulties translating the coordinate system's origin on the XY plane.

%% Figure environment removed

% Figure environment removed

Next, we want to see if \gpt can account for rotating sketch planes.
To test this, we ask it to design a car.
\gpt always suggests a simple car shape, composed out of 4 cylindrical wheels and a rectangular car body, see Fig.\ref{fig:cad_car_design}.
The difficulty with this shape is that the cylinder sketches of the wheels have to be extruded on the side planes of the car body.
There are a couple different modeling strategies to achieve this, but we observe that \gpt has difficulties coming up with these designs without any further indication.
Instead, it often extrudes the car body along its \lstinline{height}, starting from the ground plane, and then places the wheel circles on the bottom plane of the car, which is also the ground plane.
This has the effect that the car wheels will be extruded vertically.
Although we were able to correct this design in an iterative prompt-based fashion, we had little success engineering the initial prompt in such a way that we could effectively prevent this behavior.

Note that intuitively placing wheels at the bottom of a car body makes sense and that without any graphical feedback, humans could also easily make this mistake.
From this test, we can observe that \gpt is struggling to rotationally change coordinate systems.

To address this, we changed our design language description to allow \gpt to specify sketch primitive coordinates directly in a single global coordinate system. 
Now, a sketch primitive center takes as input three coordinates, which we project in post-processing directly on the selected sketch plane.
The extrude direction is still defined by the sketch plane's normal vector.
This means that \gpt does not have to take coordinate translations into account anymore.
We observe that this change in the DSL led to a higher success rate in generated designs, see second answer in Fig.~\ref{fig:cad_car_design}.

In conclusion, \gpt is able to design models in a sketch-based parametric CAD framework. 
However it is not successful at changing coordinate systems.
In this case, our backup strategy is to use a single global coordinate system.
One possible future direction is to let \gpt communicate with a geometric solver and create a feedback loop.

\afterpage{\FloatBarrier}


\subsubsection{URDF}
\label{sec:textToDesign_urdf}
%% Text to Design: Robots from URDF description

% \team{Wil, maybe with input from Megan/Andy/Allan)} 

% Figure environment removed

The Universal Robot Description Format (URDF) is a common XML-based language for describing articulated structures in robotics. 
URDF files specify a robot's structure (including both visual and collision geometry), joint locations, and dynamics information. 
The URDF format appears well-suited for potential \llm design because it is human-readable and heavily documented online.

\paragraph{Open Chain Robot Arms}
Initially, we asked \gpt to generate simple open chain robots (commonly called ``arms'') with a particular number of links. 
However, when we used the word ''arm'' to prompt \gpt to generate a robot, \gpt was unable to determine that the links should connect at the end.
Most often, \gpt placed the joints such that each link revolved about its center, and the links were not connected to each other (\fref{fig:multi-link-urdf}, initial prompt).
As shown in the subsequent prompts of \fref{fig:multi-link-urdf}, to achieve an arm with two connected links, it was necessary to describe both the joint position relative to the link 
(\lstinline{``the joint origin must be half the link's length past the link origin''}, rather than \lstinline{``the joint origin should be at the end of the link''} ) 
as well as the joint axis (\lstinline{``a revolute joint about the x axis''}).
Given this prompt pattern, \gpt was easily able to generate proper N-link robots. 

%\adriana{I don't really  understand what this section is trying to say. is the point that it can't connect things directly if you don't have that as a constraint of the DSL? if so this would be better expressed if you group fig 10 and 11 in one single discussion like Felix did in Figure 9 }



\paragraph{Wheeled Robots}
Next, we asked \gpt to generate wheeled robots composed of N wheels attached to a central rectangular platform.
A proper design of this type must have wheels that 
(1) are aligned to share an axis of rotation normal to and through the center of their circular faces;
(2) have circular faces displaced along said axis of rotation, and 
(3) contact, but do not intersect, either side of the center platform.
The combination of non-intersection and geometry relation constraints prove challenging for \gpt, which seems to exhibit limited geometric reasoning. 
Initially, we tried to specify these using language-based constraints (i.e. ``the wheels should touch, but not intersect, either side of the platform'').
These proved ineffective, as shown in \fref{fig:urdf-wheeled-constraints} (middle).
To overcome these challenges, we crafted prompts with very explicit numeric constraints (i.e. ``wheels should be offset on the global y axis by half the width of the platform plus half the height of the wheel cylinder''). 
This style of prompt successfully generated a viable result, as shown in \fref{fig:urdf-wheeled-constraints} (right).

As in the case of robot arms, we find that \gpt is immediately able to generalize a successful two-wheeled design into a four-wheeled robot. 
We achieve this by asking for a duplicate, shifted version of the existing wheel configuration, as shown in \fref{fig:urdf-four-wheel}.
However, we were unable to directly generate a successful four-wheel robot; in general, we found that as the number of constraints in a prompt increases, it becomes increasingly likely that \gpt will ignore any individual constraint. 
Thus, rather than directly requesting a four-wheeled robot in a single prompt, we found greater success by first generating a two-wheeled robot and then prompting \gpt to modify the URDF by adding additional wheels than placing the text in a single prompt.

% Figure environment removed


% Figure environment removed



\paragraph{Robot Grippers}
To test the effectiveness of our iterative, multi-prompt approach for building robots of increasing complexity, we seeded \gpt with a successful two-link open chain URDF, then asked it to modify this design into a collection of multi-finger robot grippers. As shown in \fref{fig:urdf-hands}, we were able to build two-, four-, and five-finger grippers using a sequence of prompts to add features and change proportions. 
To create a two-finger gripper, we asked \gpt to use two of the previously generated two-link open chain robots as fingers, separated by a distance equal to half the height of the finger, and connected by a rectangular platform on the base.
The four-finger gripper was similarly derived from the two-link arm by specifying that the hand should consist of four two-link robots right next to each other on a rectangular platform. To specify a five finger hand, we requested a rectangular link that hinges as a base for the thumb, then prompted \gpt to add another finger on that link and to adjust the hand proportions.

% Figure environment removed



























\ignore{

% Figure environment removed

% Figure environment removed

}



\subsubsection{Graph-based DSL}
%% Text to Design: robots from graph description
% \team{Allan}

% \todo{describe the tie to graph-based design, and also how this representation is different from URDF discussed before.}
% \liane{We also want to reduce the amount of direct copy/paste chat history. Can you summarize the chats' key processes/challenges/successes in text, and reduce the chat history to illustrate only a few specific points with visualizations? The full chat history should be in the github repo, and you can include some longer snippets in the appendix too, if you like.}

While designing an entire robot end-to-end using LLMs may not be feasible, we find that \gpt has the ability to reason about the spatial layout of robot components. These spatial layouts are naturally represented as graphs where the nodes are components and edges are connections between them. Unlike URDF, this representation is more general and is applicable in domains outside of simulation.

To generate robot design graphs using \gpt, we first need a text-based graph representation. Our first approach involved asking \gpt to output the popular GraphViz format. While convenient, this format makes it difficult for \gpt to provide metadata for each part (such as motor torque, size) in a format usable by downstream applications. Instead, we take advantage of \gpt's ability to generate Python code that conforms to a provided domain-specific language (DSL). The full DSL is detailed in \appref{sec:appx_graph_robots}. 

When prompted with a small DSL embedded in Python, \gpt is able to write code that selects and places robot components at a high level of abstraction. By supplying a function that translates components in three-dimensional space, we can extract \gpt's concept of each component's position relative to the others. 
% \adriana{This is way too big, please crop also please discuss the limitaitons}

% Figure environment removed

In this example, we ask \gpt to generate a humanoid robot using the provided functions. \gpt makes appropriate calls to \texttt{add\_link} to create nodes in the design graph, \texttt{add\_joint} to create edges between them, and \texttt{translate} to establish their relative positions.

We manually implement the functions described in the prompt in order to visualize the resulting robot topology. The arms are positioned beside the torso, the legs are positioned below, and the head rests on top as expected for a humanoid robot.

We saw similar success when asking \gpt to construct a snake robot, car robot, and scorpion robot. When requesting a robot dog, however, \gpt only adds two legs initially. Specifying a ``robot dog with four legs'' was necessary to obtain the expected behavior. We also encountered difficulties when attempting to obtain a more detailed design for the robot dog. Asking for a ``robot dog with four legs, two links per leg'' produced a graph with two nodes per leg, but \gpt did not position them relative to each other.

% \subsubsection{Gerber file?}
% %% Text to Design: PCB specification from Gerber file
% \team{Young, maybe with input from Alyssa/Mike}


Designs with pre-defined schematics (imported components), use chatgpt to generate command for pcb layout variation.
electronics example of 8 LEDs as an array, 8:1 multiplexer, ADC, and microcontroller (arduino). Use KiCAD and python-based code



\subsubsection{Summary Discussion}
In light of these experiments, we conclude that \gpt is capable of generating designs based on high-level text input, even across a wide variety of representations and problem domains. 
We note that several of \gpt's capabilites and limitations remain consistent independent of the representation.
For example, in all cases, \gpt is able to generate sensible, well-structured code with semantically meaningful variables and comments.
Moreover, independent of the representation or the problem domain, \gpt consistently shows superior performance with respect to the high-level, \textit{discrete} elements of a problem (\eg, identifying the correct type and quantity of each primitive/operation) as opposed to the lower-level continuous parameter assignments (\eg, correctly positioning the primitives relative to one another). 
A more detailed discussion of capabilities, limitations and opportunities will follow in \sref{sec:textToDesign_discussion}.
For now, we rely on the similarities between various representations to justify a reduced scope for our future experiments.
In particular, moving forward, we study each question with respect to only a subset of the design representations and domains introduced above. 









\subsection{Interpreting and Respecting User Control (Q2)}
% Can \gpt generate a design matching some \textit{specific} user intent, when provided with a more detailed description?
% Generating Specific Designs from Lower-Level Guidance/Feedback
% \item \textbf{Q3} To what extent is the user able to control the designs created by \gpt? Is \gpt able to interpret and respect user-defined constraints, such as spatial relationships between objects, mating constraints for multi-part assemblies, or integration of standard pre-fabricated parts?

The above examples demonstrate \gpt's ability to generate a design based on very high-level semantic input.
However, we also wanted to test its ability to generate designs that adhere to a specific user-given intent.
This section also tests whether \gpt is able to overcome its own potential biases induced by the training data, in order to generate something that truly adheres to a user's specified constraints -- whether or not those constraints match the ``common'' form of a given design target. 
In particular, we choose to study whether \gpt is able to 
(1) understand and respect semantically meaningful spatial constraints, and
(2) incorporate specific pre-fabricated elements into a design.



\subsubsection{Spatial Constraints}
\label{sec:textTODesign_spatial_constraints}
% For example, can we specify the constraints such as "non-overlapping", "symmetric", "above/below", "at a specific relative/global position"? What types of constraints? Which ones does it understand natively, and if there are multiple ways to phrase a constraint, do some approaches seem to work better than others?
Through the general experiments above, \gpt has already shown some capacity to respect high-level spatial constraints, such as a design element's absolute size or its position relative to another element of the design. 
\gpt's compliance with such requests was frequently flawed at the outset, but the results were generally workable after some amount of interactive feedback. 
This section aims to explore the types of constraints \gpt is able to natively understand, and how we might best interact with \gpt in order to improve the chance of successful compliance with such constraints.

%% non-overlapping, position, relative location of 
As an initial experiment, we explored whether \gpt is able to construct a version of the previous cabinet design that includes a door and a handle (see \fref{fig:cabinet_with_handle}). 
We started from a fresh chat, and provided \gpt with a prompt similar to the one described in \sref{sec:textToDesign_JSCAD_basic}, asking for a cabinet to be built from scratch.
However, this time, we also request a door at the front of the cabinet, with a handle on the right hand side of its outward-facing face.
As shown in \fref{fig:cabinet_with_handle_start}, \gpt initially struggled to position several of the cabinet elements -- particularly the side panels and the door.
Although \gpt corrected the position of the side boards immediately, \gpt continued to have trouble placing the door, as it was oriented incorrectly relative to the rest of the design. 
When reminded that the door should be oriented vertically, \gpt was able to comply with the request, but the corrected position was still not fully suitable, as the door coincided with the cabinet's side panel.
After another reminder that the door should reside at the front of the cabinet, with the handle on the right so it could be attached with hinges on the left, \gpt was able to place the door correctly. 
However, the handle remained ill-positioned as it was located on the left-hand side, and was protruding into the door panel. 
After 2 additional prompts, \gpt was able to correct the position to the left hand side. To correct the protrusion issues, \gpt needed 3 more prompts. During these iterations, \gpt moved the handle fully to the \textit{inside} of the door; it needed explicit reminder that the handle should be placed on the \textit{outside} of the door.

With a fresh \gpt session, we also tried providing the previous \jscad specification of the cabinet as part of our input prompt, then asking \gpt to modify the existing design such that it contained a door and a handle, as before. 
Despite the different starting points, \gpt followed a similar trajectory, as shown in \fref{fig:cabinet_with_door_add}: the door was initially aligned incorrectly, as it coincided with one of the side panels; after 1 prompt, \gpt was able to correct the door placement. However, despite \gpt's explicit assertion that \lstinline{the handle is also placed on the right side of the door's exterior face}, the handle remained on the left. Finally, after another prompt, \gpt was able to correct the handle position such that it was on the right rather than the left. 

The way in which \gpt dealt with the under-specified handle request also proved interesting. 
In \fref{fig:cabinet_with_handle_start}, \gpt opted for an additional cuboid that would be unioned into the final design. 
By contrast, in \fref{fig:cabinet_with_door_add}, \gpt opted to create the handle by subtracting a small cuboid from the door panel. 
In still other examples, \gpt refused to add the handle, and instead offered the following disclaimer: \lstinline{Note that the handle for the door is not included in this script, as its size, shape, and position would depend on additional details not provided. This would likely require additional modules, such as cylinder from @jscad/primitives, and might be added as an eighth component in the main function.}

These interactions provide a promising basis for interactive user control of the design, but the process is somewhat tedious at the moment, as \gpt requires very explicit instructions about the design or correction intent. 
The addition of highly-detailed user constraints also seems to confuse \gpt to an extent, as it seems to ``forget'' the larger context of the design in the process, so it must be frequently reminded. 

% Figure environment removed
% \wojciech{Fix colors in the figure}

% Figure environment removed


%% symmetry 





%\subsubsection{Generating Mated, Multi-Part Assemblies}
%% Can \gpt generate multi-part assemblies with valid mating characteristics? 
%
%%% Text to Design: CAD with mating assemblies
% \team{Felix}

CSG and parametric CAD modeling paradigms are mainly used to design 3D parts which can be later on manufactured.
Putting these parts together into one functional object is usually done in an \textit{assembly} step.
In an assembly, specifying how two parts are connected, or \textit{mated}, comes down to defining (i) \emph{where} two parts are mated and (ii) \emph{what} the movement constraints are between two parts.
For example, to mate two lego bricks, we would define (i) the mate connector of the first brick on the top disk of one of its knobs, the mate connector of the second brick on the inner disk of its bottom tube and (ii) define the mate type as a sliding mate, allowing for a vertical translational degree of freedom (DOF). \todo{Lego mating figure}

In commercial CAD software, assemblies are defined via graphical interactions, allowing users to easily reason spatially and to correct their mate definitions thanks to real-time feedback.
GPT and other {\llm}s are purely text based, so they seem like a suboptimal choice for assembly. 
However, prior research has shown that \gpt has a good understanding over contextual and part relationships \cite{}, which could be useful for facilitating the mating process.

As for the parametric CAD modeling paradigms, we try to overcome the domain gap between the textual and the graphical domain by providing \gpt with a simplified DSL to express a minimal set of mating operations.
We define two operators: (i) an operator \lstinline{choose_planar_face(solid_name, plane_side)} which returns a mate connector based on the selected solid and its planar face and (ii) three mating operators \lstinline{mate}, \lstinline{mate_slider} and \lstinline{mate_revolute} which take as input two mate connectors and mate with either no DOF, one translational DOF and one rotational DOF, respectively.
\lstinline{mate_slider} and \lstinline{mate_revolute} additionally take the translation axis and the revolution axis as input argument.

For example, the program in Fig.\ref{fig:mate_stack_cuboids} (a) shows how to stack two cuboids on top of each other with this language.
Given our DSL definition, and this example program, \gpt can successfully stack three cuboids on top of each other, as is shown in Fig.\ref{fig:mate_stack_cuboids}(b) and (c). 
This simple example shows an important strength of \gpt, namely that it can learn from examples and is able to translate this to similar problems.
When assembling an object with a lot of parts and connectors, time-intensive manual labor is required to perform repeating patterns of mating operations \cite{automate}.
Leveraging {\llm}s which can reason about part relationships could prove to be a real time saver for designers.
\todo{}
In Fig.\ref{fig:mate_screws}, we see an example of a screw mating pattern which is performed once and which has to be repeated three more times in the remaining corners of the plate.
When asked to complete the assembly process, \gpt finishes the mating job successfully. 
It should be noted that the variables used in this example are explicitly descriptive.
\gpt performs less well when presented with no semantic context.
\\
\todo{}
\begin{itemize}
    \item explain slider mate example: place a drawer in a box
    \item failure case/limitation: 
    \item explain revolute mate example: door and hinge
    \item failure case/limitation:
\end{itemize}
Our mating DSL for \gpt shows potential benefits for real-world design scenarios, but also some limitations. 
It should be noted that the proposed DSL stays close to current graphical workflow.
Other DSLs could be better suited to interface with \gpt and maybe allow it to textually describe matings more naturally.
\\




\subsubsection{Incorporating pre-fabricated elements}
\label{sec:prefabbed-ele}

It's also common to design an object around specific pre-manufactured elements, such as hinges, brackets, or motors. 
We explore the possibility of using \gpt to source the parts in \sref{sec:part_sourcing} -- at that time, we explore whether \gpt can identify the required part categories, provide options, and/or select a set of options that are compatible with one another and the intended overall design. 

For now, we assume that the user has a specific (set of) part(s) in mind that they would like to incorporate into their design.
Then we investigate whether, given these components, \gpt is able to (1) build a reasonable proxy of this design, then (2) effectively use it as a module within a larger assembly.

\paragraph{Cabinet with Standard Hardware} 
%% Text to Design: CSG with OpenJSCAD, integrating pre-fabbed parts
% \team{Liane}
\newcommand{\bracketImHeight}{1cm}

% Figure environment removed

\newcommand{\bracketPlacingImWidth}{0.162\textwidth}
% Figure environment removed
\afterpage{\FloatBarrier}

To make the cabinet design more stable, a designer may wish to include extra support brackets to work with. 
Many pre-fabricated variations of these brackets exist, and they are inexpensive and readily available.
Given this, it does not make sense to design or manufacture these parts via \gpt.
Rather, we'd like to incorporate instances of a pre-fabricated version.
To do this, \gpt must first build a proxy of the part, place the proxies throughout the design appropriately, and adjust the remaining elements of the design to accommodate these components.

For our first experiment, we chose to incorporate the Prime-Line 1/4 in. Nickel-Plated Shelf Support Pegs from Home Depot into our design.
We provided \gpt with a URL to this part's listing on the Home Depot website, which contained a text description of the item and the schematic diagram pictured in \fref{fig:bracket-buildup}(left).
We then asked \gpt to build a simple geometric proxy that we could incorporate into our design as a placeholder.
As shown in \fref{fig:bracket-buildup}(right, top), \gpt was able to infer and generate the appropriate primitives (one cylinder for the peg and two cuboids for the L bracket).
However, it was not able to correctly scale, orient, or position the elements. 
In an effort to test \gpt's understanding of the structure, we asked it to describe the structure in its own words. 
Although it gave a reasonable description of the bracket, there was little improvement in the result when it was asked to improve the script accordingly.
Thus, even with several iterations of user feedback, \gpt was unable to construct this shape from high-level third-party (URL) or user input.

Ultimately, we had to provide \gpt with an explicit description of the structure that we wanted.
Moreover, we found that even with an explicit description, \gpt was unable to generate the correct shape when provided with all directions at once.
Instead, we had to create the shape in an iterative fashion, beginning with the L bracket and then adding in the peg, as shown in \fref{fig:bracket-buildup}(right, bottom).
Eventually, it was able to generate the structure and consolidate the instructions into a high-level module called \lstinline{createBracketWithPeg}, as desired.

We then provided the module \lstinline{createBracketWithPeg} as an input to \gpt, and asked it to incorporate these structures into the design, as detailed in \fref{fig:cabinet_with_handle}. In particular, we asked for four brackets under each shelf, with the pegs protruding into the cabinet's side walls, the back face of the bracket's vertical leg in contact with (but not protruding into) the side wall, and the top face of the bracket's horizontal leg in contact with (but not protruding into) the bottom face of the shelf. 
We initially tried to complete this experiment in a single continuous chat that (1) designed the cabinet, (2) designed the L-bracket, and then (3) incorporated the brackets into the cabinet. 
However, we found that after the extended discussion regarding the L-bracket design, \gpt seemed to have completely forgotten its cabinet specification.
Despite multiple prompts, it was unable to recover the previous design. 
Instead, we directly provided \gpt with the L-bracket module and its prior cabinet design, and then asked for a modification.
This approach was far more successful.
Overall, we found that \gpt was able to instantiate the correct number of brackets, but it struggled to rotate and position them appropriately. 
After several user prompts, \gpt was able to successfully place the brackets in their locations.
Finally, we asked \gpt to adjust the shelf in order to (1) not protrude into the brackets, and (2) incorporate some additional allowance so the shelf could easily fit between the supporting brackets in a physical assembly.
\gpt was able to complete these requests without issue.

Overall, although \gpt initially struggled to build a proxy of the pre-fabricated part we had in mind, \gpt seemed quite capable of incorporating the completed proxy into a given design, as desired.


\paragraph{Quadcopter}
%% Text to Design: quadcopter with integrated prefabbed parts
% \team{Allan/Bohan, maybe with help from Edward/Pingchuan/Megan/Andy}

% example: quadcopter frame designed around a collection of components.
% include eg:
% - how you chose the components; 
% - selecting components that are compatible; 
% - size constraints (from eg the "car sized copter" you had at some point to the final design).
% - coming up with the geometric primitive proxies for each element
% - design process for the frame using those components

% wip

% Figure environment removed

% block commented
\iffalse
 \adriana{strange way to start - where do you determine the components?}
Once the components are determined, 
we would like to design a quadcopter that uses them with the help of \gpt
and study how \gpt can help with such task.
\adriana{replace previous sentence with: The design of a quadcopter requires the incorporation of pre-fabricated elements such as X, Y, Z. Assuming these have been identified and sourced, the design of the frame should be dictated by their dimensions. Next, we'll examine how GPT can assist with this design task.}
Nonetheless, it is not simple to let \gpt represent those components in details.
To ease the design task for \gpt, we represent them 
using either a box with size $w\times h\times d$
or a cylinder with radius $r$ and height $h$,
which we found that it understands them very well 
as shown in the previous sections. \adriana{which sections?}
The first thing we tell \gpt is 
the dimensions of these existing parts represented 
by boxes or cylinders.
To make it more understandable, 
we also give three functions to describe the box, the cylinder and their placement.
The first function createBox(w, h, d) returns a box with size w * h * d centering at (0, 0, 0).
The second function createCylinder(r, h) returns a cylinder with radius r and height h center at (0, 0, 0).
The last function place(item, x, y, z, a) first rotates the item around z axis by angle a
and then translate the item to position (x, y, z).
Next, we ask \gpt to create a design that incorporates those given parts using only those functions. 
The main part of the quadcopter that needs to be designed 
is the frame which can hold those selected components.
At the first try, \gpt is able to give a roughly correct design description,
but it fails to produce a correct geometry.
It understands that the quadcopter needs four arms 
to connect four motors with four propellers
and that the remaining parts are at the center.
However, it could not make all parts correctly positioned and oriented, as shown in Figure~\ref{fig:quadcopter-stage}(a).
After closely looking into the generated program, 
it is not hard to figure out several design issues. 
First, the frame is not oriented correctly.
This is fixed by telling \gpt 
which dimensions represent the cross section of the frame.
After the adjustment, we can get a near correct quadcopter (Figure~\ref{fig:quadcopter-stage}(b)).
Second, the parts are intersected with other boxes, as highlighted in red rectangles in Figure~\ref{fig:quadcopter-stage}(b).
This problem has been happening constantly 
when we designed other items earlier.
We found that it is difficult to let \gpt understand 
what it means when we say box A and box B are intersected
and how it can be fixed.
Therefore, our solution to fix this problem is 
to manually tell \gpt how much translation is needed 
for each problematic part.
Finally, the frame is not practical because
1) it directly attached to the motor cylinder and
2) it is not sufficient to hold components including
the battery, the controller and the signal receiver.
To solve the problem, we did not ask the \gpt about
how you can solve the problem.
Instead, we provide a concrete solution to the issues
and then ask \gpt step by step.
We first asked \gpt to add a cylinder base
at the bottom of each motor.
The cylinder base directly connects to the frame bar
and each motor sits on each cylinder base without touching the frame bar.
In addition, we ask \gpt to add a box body to
strengthen the frame bars and serve as a plate to hold to remaining parts.
After a few iterations of minor adjustments,
we were able to obtain a valid design which will be further verified
in the simulator or in the real world.
The final result is shown in Figure~\ref{fig:quadcopter-stage}(c).
Throughout the design process, we found that \gpt is good at analyzing the design problem and give a valid design in text.
However, it cannot nicely incorporate the mathematical and physical concepts. 
For example, it does not understand colliding and 
whether the frame is strong to hold any parts.
Therefore, human has to help it in this regard.
 \adriana{I think this section could use some polish to shorten and create understandable paragraphs and opposed to one long text. I also think its best to reference 4.1.2 and say you use the same primitives instead of defining a language all over again.  }

 \fi

 
Designing a quadcopter involves integrating pre-built elements like the motor, propeller, and battery. Detailed sourcing of these parts will be addressed in the later section (Section~\ref{sec:part_sourcing}). Once these components are sourced, the frame must be designed to accommodate their dimensions. We'll explore how \gpt can assist with this task.

However, enabling \gpt to accurately represent these parts isn't straightforward. To simplify the task, parts are represented as either a box of dimensions $w \times h \times d$ or a cylinder with radius $r$ and height $h$. \gpt can handle these representations well as demonstrated in Section~\ref{sec:textToDesign_CSG_boxes}. Rather than having a single function which creates a primitive and translates it as in Section~\ref{sec:textToDesign_CSG_boxes}, we introduce three functions for ease of design: createBox(w, h, d), createCylinder(r, h), and place(item, x, y, z, a). The first two functions generate a box or a cylinder at origin (0,0,0), while the third rotates and moves the item to desired coordinates.

Subsequently, we task \gpt with creating a design that integrates these parts using only the above functions. The primary element \gpt must design is the frame, which should hold the selected components. Initially, \gpt produced a correct textual design, but struggled with the geometric representation, similar to Section~\ref{sec:textToDesign_CSG_boxes}. It understood the quadcopter structure, but had issues with part positioning and orientation (Figure~\ref{fig:quadcopter-stage}(a)). Problems included incorrect frame orientation and part intersections. By guiding \gpt in correcting these issues, we achieved a near-correct quadcopter design (Figure~\ref{fig:quadcopter-stage}(b)).

The initial frame design wasn't practical because it was directly attached to the motor cylinder and insufficient to hold components like the battery, controller, and signal receiver. To address this, we asked \gpt to incrementally implement specific solutions, such as adding a cylinder base under each motor and a box body to reinforce the frame bars and house remaining parts. After minor adjustments, we arrived at a valid design, which will undergo further testing in a simulator or real world conditions (Figure~\ref{fig:quadcopter-stage}(c)).

Throughout the design process, \gpt demonstrated proficiency in textual design analysis but struggled with mathematical and physical concepts such as collision and structural integrity. Thus, human guidance remains crucial in these areas.






\subsection{Incorporating Abstractions such as Modular/Hierarchical Designs (Q3)}
\label{sec:textTODesign_abstractions}
% \item \textbf{Q2} Is \gpt able to incorporate abstractions used by human designers, such as modular (de)composition?

As we have seen from previous examples, \gpt is inclined to use some abstractions like variables by default.
It is also clear that \gpt is well suited to the use of modular or hierarchical design, as in the case of the pre-fabricated L-brackets that it was able to instantiate several copies of, and distribute throughout a design. 
However, there are often instances where a user might want to impose their own specific modules -- for example, a certain hierarchical grouping may facilitate easier debugging or cleaner code.

To test \gpt's abilities in this area, we revisit the cabinet example, and try to modify it such that it contains multiple shelves. Because we have already incorporated pre-fabricated brackets, this modification is non-trivial, as \gpt must instantiate and position the appropriate number of shelves \textit{and} all associated support brackets. 
We began by directly asking \gpt to make this modification on top of the existing code, by generating two evenly spaced shelves within the cabinet instead of one. 
\gpt correctly identifies the elements which must be duplicated, and it instantiates the correct number of them. 
However, it is unable to correctly adjust the position of each module; after the initial request, neither the shelves nor the brackets were in reasonable locations. 
It took 4 additional user prompts to correct the relative positions of these components. 
After this correction, \gpt did seem able to generalize its logic directly to generate cabinets with a varying number of shelves. 
However, the code itself is fairly convoluted.

To avoid these issues, it may be more natural to consider a shelf with its appropriate supporting brackets as a single module.
This way, the entire ``subassembly'' could be instantiated and positioned as a unit on future calls. 
We asked \gpt to implement this plan, by requesting the creation of a module named \lstinline{supportedShelves()}, which instantiates and appropriately positions a shelf and its associated support brackets within the design.
Then, we asked \gpt to refactor the original script such that it used the new module to generate a cabinet with two evenly-spaced shelves.
The initial response had a minor compilation error, a shelf tolerance issue, and a bracket alignment issue, as before, but each of these issues were immediately corrected after a single user prompt. 

Overall, the approaches resulting from both experiments seem equally effective and flexible once they have been fine-tuned. 
Thus, we conclude that \gpt is able to effectively create and use modules, whether they are explicit (\eg, in the form of a function, as in the second experiment) or implicit (\eg, in the form of a for-loop, as in the first experiment).
However, it seems as if the explicit module made it slightly easier for \gpt to reason about a challenging alignment problem.
Moreover, it is useful to know that users can effectively request this kind of hierarchical refactoring, as most human programmers/designers would generally find it easier to reason over a function in this scenario. 
% \todo{(liane) generate figure with the incorrect/corrected placements}









\subsection{Discussion}
\label{sec:textToDesign_discussion}
% \team{Liane}

In this section, we elaborate on the key capabilities (C), limitations (L), and dualisms (D) previously outlined, particularly as they relate to the domain of text-to-design.

\noindent \textbf{C.1 Extensive Knowledge Base in Design and Manufacturing:} Within the text-to-design space, \gpt exhibited proficiency in supporting high-level structure and discrete composition. For instance, \gpt consistently generated the correct primitives (type and quantity) for a given task, regardless of the specific design language it was using. 
\gpt also demonstrated a capacity for interpreting and auto-completing under-specified prompts, as in the case of the CSG table example, where \gpt inferred and provided reasonable values for a set of missing parameters (see \sref{sec:textToDesign_CSG_boxes}).
Finally, \gpt generated readable, explainable, and maintainable code that contained descriptive variable names and comments, along with appropriate modularity and other high-level structural elements.  


\noindent \textbf{C.2 Iteration Support:} Even when \gpt did not immediately arrive at a suitable design solution, it often succeeded in rectifying errors after a reasonably small number of user interactions. For example, it was able to successfully adjust the placement of the cabinet handle after a handful of additional prompts. The ability to engage in iterative design is also very helpful when building up complex structures such as the wheeled robot from \sref{sec:textToDesign_urdf} or the L-bracket proxy discussed in \sref{sec:prefabbed-ele}, because users can start with a simple prompt, then iteratively increase the complexity to arrive at a suitable result.

\noindent \textbf{C.3 Modularity Support:} \gpt effectively incorporates modules and hierarchical structures, using natural language as a powerful tool for conceptualization and orientation.

\noindent \textbf{L.1 Reasoning Challenges:} Spatial reasoning posed a significant challenge for \gpt. Well-crafted domain-specific languages (DSLs) may be able to mitigate this issue. We noted specific difficulties with constructive solid geometry (CSG) due to the computational requirements for object placement. Sketch and extrude languages that utilize reference points can minimize this challenge to an extent, as they offload the computation to reference resolution. This approach is effective for simpler designs but falters when managing complex sequences of transformations. As discussed in the sketch-based car example from \sref{sec:textToDesign_OnShape_basic}, we found that DSLs that balance the benefits of reference-based language with global positioning information may be more effective. 

\gpt's lack of spatial awareness also created difficulties with constraint handling, such as when \gpt was asked to ensure that elements were non-overlapping. We found that iterative refinements and careful prompting often provided a workaround for these issues. For example, \gpt typically failed to respect ``non-overlapping'' constraints, but it generally responded well to the instruction that some element should be ``in contact with (but not protruding into)'' another element.

\noindent \textbf{L.2 Correctness and Verification:} \gpt is not able to reliably verify its own output, and it frequently makes contradictory claims. For example, when asked to place a handle on the right side of the cabinet structure, \gpt frequently placed the handle on the left-hand side of the cabinet, then immediately declared its design a success, because the handle was on the right, as requested. This seems to suggest that external verification tools may be helpful, particularly in cases where the contradictions are less obvious.

\noindent \textbf{L.3 Scalability:} \gpt's success seems to decline as the number of simultaneous requests increases. For example, it is best to issue 1-2 constraints or correct 1-2 issues at a time, rather than trying to issue several constraints or correct several issues at once. 
Similarly, \gpt encountered challenges when interpreting high-level information to build proxies for more complex designs all at once; instead, the models must be built iteratively, with gradually increasing complexity.
This iterative modeling was most effective when the user provides explicit instructions about both the aspects that should change, as well as the aspects that should remain unaltered (either because they are already correct, or because they will be addressed later). 
Despite \gpt's initial difficulty creating complex models, \gpt is able to effectively use and combine existing modules to create more intricate models. 

\noindent \textbf{L.4 Iterative Editing:}
As discussed in \sref{sec:prefabbed-ele}, \gpt seems to exhibit limited memory and attention span. In particular, it often ``forgets'' things from previous messages. We address this by occasionally reminding \gpt of its previous input/output, either by asking it to summarize a previous interaction/finding, or by explicitly including a prior result as a starting point in our prompt.


% \noindent \textbf{D.1 Context Information:} \liane{todo}
\noindent \textbf{D.2 Unprompted Responses:} \gpt is frequently able to recognize and address under-specified problem statements.
For example, in the CSG table specification (\sref{sec:textToDesign_CSG_boxes}), \gpt correctly inferred the need to assign a tabletop thickness value.
Similarly, when augmenting the cabinet with a door and a handle in \sref{sec:textTODesign_spatial_constraints}, \gpt responded with several distinct approaches for handle design.
This can be powerful, as it may alert the user to parameters or variations which may otherwise have gone overlooked; then, users have an explicit opportunity to consider and refine the specification accordingly.
Moreover, it allows users to undertake a design process and begin receiving feedback without first needing to craft a perfect specification or prompt.
However, if \gpt confidently hallucinates a particular solution to an under-specified aspect of a design problem -- rather than explicitly prompting the user to consider a range of options -- it may limit and/or bias their exploration in unexpected ways.





% finds the discrete set of primitives, can't align them reliably (e.g. Felix and PyVista ) this is true accross different DSLs
% With CSG everythin live in global space and things are either pretruding or separated
% sketch and extrude is better but still struggles
% The DSL can help but it's not enough to solve the problem 
% What are the characteristics of the DSL changes that have helped?
% It has fewer things to do so that be helpful but it will still get things wrong if it's still hve to do multi-step reasoning about the different frames of reference is still hard (order specific) you start with a good reference point from which you can do the rest of the design but if it needs  a lot of reasoning it has a hard time. (.e.g composing transformations or chaining or logic is hard). 
% Expose high level constructs that are explicit of the thigns that it can do (e..g rotate by 90 degrees). Felix may have done somethign similar (added a global positioning -
% ***********This is the conclusion*********:  GPT is bad a two things: computation and staking references  so it  struggles with CSG and it struggles with many references. Solution references with some global information. 

% Good at re-using. It is easier for it to build things iterative/ with modulatiry. Do a lot of iterations to make you thing work and can replicate things well

% Hard to handle constraints or controlled specificatons of where you want things to be (e.g. the handle). This is of course even worse if you're trying to get mulitple thigns at once. 


% Good at:
% 1) high-level structure (discrete part of problems)
% 2) good commenting the code and naming variables in ways that are semantically meaningful 
% 3) good at re-using things once it has a design (maybe partially because the stuff it creates has good semantic information - this will be discussed more later)  

% Bad at
% 1) Continous positioning of elements in a global space and reasoning about local coordinate systems and compositions of transformations. 

% Solution: DSLs that combine referencing and global information. 

% 2) Respecting Constraints 

% Solution: feedback loop and alternative phrasing. 

% 3) Taking a lot of information at once. 

% Solution: leverage modularity 




\ignore{

- Text-To-Design Takeaways
	- good at high level structure, bad at placement
	- can use a wide range of DSLs, but best if they're intrinsically consistent, seems to have trouble reasoning about local coordinate systems (needs to be reminded about e.g. centered positioning, sketch based CAD hard,)
	- better to build it up gradually, has trouble with large requests all at once
	- constraints are difficult -- has trouble reasoning about them, but can usually be accomplished after some back and forth / playing with the word choice or prompt style 

	- has difficulty directly interpreting/building proxy from high level information, but can use it once it exists to create more complex designs.
	-
As part of this exploration, we also examine \gpt's ability to provide semantically-meaningful design specifications -- \eg, scripts including variables rather than hard-coded ``magic'' numbers, and reusable functions or modules where appropriate.

\begin{itemize}
    \item \textbf{Q} Is there a universally effective style of prompt that seems to generate the desired design quickly, with few errors (wrt design or boilerplate code)?
    \item \textbf{Q} Is there a limit to the number of primitives, constraints, modules, levels of hierarchy that \gpt seems to encounter? 
    \item \textbf{Q} (How) can we iteratively evaluate validity and the correctness of the design, and use this evaluation either to improve the model through feedback or to automatically post-process the design so that it is valid and correct.?
\end{itemize}


When specifying geometric constraints, effectiveness of that constraint being reflected in the final design varies based on the adjectives used to describe the constraint. For example, asking \gpt to make two primitives "non-overlapping" may still produce a result where the two bodies intersect, while asking \gpt to make it so that a body "does not protrude" into the other is effective in preventing intersections. So far, a pattern as to which adjectives are interpreted correctly by \gpt has not been observed. 

Specifying constraints via numeric means --- i.e. offset the second cube so that it's top face is aligned with z = 0 --- is almost always effective, whereas asking for geometric relations --- i.e. offset the second cube so its top face is aligned with the bottom face of the first cube --- does not guarantee consistent interpretation.

Introducing multiple constraints in a single prompt often results in some of the constraints being ignored, whereas introducing the same constraints in sequential prompts results in the desired behaviour. 

}













% ============================
% old formulation
% ============================
\ignore{
\paragraph{Problem setup}

There are three aspects to a prompt for specifying a design task: The objective, the primitives, and the additional constraints.  We describe in each turn, and then provide an example of a complete prompt and its output.



\includepdf[pages=1]{figures/BohanExample.pdf}

Function box(x1,x2,x3, dx1, dx2, dx3) creates a box centered at point (x1, x2, x3) with dimensions (dx1, dx2, dx3). Create a capital letter T from non-overlapping boxes.

This prompt has three components. Specifying a primitive, specifying the objective/goal, and constraints.


\paragraph{Design Representation}
Design can be expressed code, i.e. formally (e.g., CAD, OpenSCAD, SVG, STL, etc.). Sequential List of commands. 
Design can be represented using graph. Nodes (components). Edges connections (e.g., connectivity) between components. 


\paragraph{Design specifications}
Describe a text example of a goal.

\paragraph{Design primitive}
How do specify design primitives.
Can you specify different primitives? How many? Can you mix different primitives. 

\paragraph{Constraints}
Describe different types of constraints.
Describe which constraints work and which do not work.
Symmetries. Spatial constraints (on the top, on the side, colinear, at the distance from X, parallel, perpendicular, at an an angle, reflective, minimum, maximum size, setting something to a value)

\paragraph{Hierarchy}
Describe how to introduce hierarchy.
E.g,, Specify in detail a leg for a table and a tabletop and then how to make the whole table. A leg for a table is made of a few components. a tabletop is made from a few components.
Check if you can make multiple levels of hierarchy.
Check if constraints still hold across multiple levels of hierarchy.

}
