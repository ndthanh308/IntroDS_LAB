

\section{Introduction}

Advances in computational design and manufacturing (\cdam) have already permeated and transformed numerous industries, including aerospace, architecture, electronics, dental, and digital media, among others. 
Nevertheless, the full potential of the \cdam workflow is still limited by a number of barriers, such as the extensive domain-specific knowledge that is often required to use \cdam software packages or integrate \cdam solutions into existing workflows.
Generative AI tools such as Large Language Models (\llms) have the potential to remove these barriers, by expediting the \cdam process and providing an intuitive, unified, and user-friendly interface that connects each stage of the pipeline. 
However, to date, generative AI and \llms have predominantly been applied to non-engineering domains. 
In this study, we show how these tools can also be used to develop new design and manufacturing workflows. 

Our analysis examines the standard \cdam workflow to identify opportunities for \llm-driven automation or acceleration. 
Specifically, we break the \cdam workflow into five phases, and then assess whether and how the efficiency and quality of each phase could be improved by integrating \llms. 
The components under investigation include (1) generating a design, (2) constructing a design space and design variations, (3) preparing designs for manufacturing, (4) evaluating a design's performance, and (5) discovering high-performing designs based on a given performance and design space.

Although it is feasible to create specialized \llms for design and manufacturing, we demonstrate the opportunities offered by generic, pre-trained models. 
To this end, we conduct all of our experiments using \gpt \cite{openai2023gpt4}\footnote{We use the OpenAI ChatGPT interface to interact with the \gpt versions released between May 24, 2023 and July 19, 2023}, a state-of-the-art general-purpose \llm. 
Our \gpt-augmented \cdam workflows demonstrate how \llms could be used to simplify and expedite the design and production of complex objects. 
Our analysis also showcases how \llms can can leverage existing solvers, algorithms, tools, and visualizers to synthesize an integrated workflow. 
Finally, our work demonstrates current limitations of \gpt in the context of design and manufacturing, which naturally suggests a series of potential improvements for future \llms and \llm-augmented workflows.