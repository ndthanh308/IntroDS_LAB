
\section{Overview}
\label{sec:overview}

% Figure environment removed

The fundamental aim of this study is to conduct an in-depth exploration of the opportunities and challenges of applying contemporary \llms  within the landscape of the \cdam workflow described in \sref{sec:related_cdam}. 
Driven by this objective, we propose a thorough and wide-ranging exploration that is independent of any predefined or proposed framework.

To apply \llms coherently across such diverse tasks, we leverage the insight that all building blocks in the \cdam workflow (design, design spaces, manufacturing instructions, and performance metrics) can be represented by compact programs.
Thus, at a high level, every phase of the \cdam workflow can be seen as a translation layer between an input DSL and an output DSL. 
The fact that \llms excel at such symbolic manipulations suggests that \llms have the potential to address these tasks while simultaneously leveraging and improving upon our traditional solutions.

To achieve comprehensive coverage and uncover the different facets of \llm-assisted \cdam, we have undertaken an extensive suite of experiments, incorporating a broad variety of design representations, manufacturing processes, and performance metrics. These are detailed further in \sref{sec:domains}.


\subsection{Methodology}
\label{sec:methodologyAndCLD}

\begin{table}[tb]
    %\footnotesize
    \centering
    \resizebox{\textwidth}{!}{\begin{tabular}{c|c|l|l}
        Category & Code & Title & Summary\\
        \hline
        \multirow{ 2}{*}{Capabilities} 
         & C.1 & Extensive Knowledge Base in Des.\&Mfg. & \gpt has a broad knowledge of design and mfg. considerations\\
        &  C.2 & Iteration Support & \gpt attempts (and often succeeds) to iterate and recitfy errors when prompted\\
        & C.3 & Modularity Support & \gpt can reuse or adapt previous/provided designs or solutions \\
        %
        \hline
        %
        \multirow{ 2}{*}{Limitations} 
        &L.1 & Reasoning Challenges & \gpt struggles with spatial reasoning, analytical reasoning, and computations\\
        & L.2 & Correctness and Verification & \gpt produces inaccurate results or justifications for its solutions\\
        & L.3 & Scalability & \gpt struggles to respect multiple requests concurrently\\
        & L.4 & Iterative Editing & \gpt forgets/introduces errors when modifying previously-generated designs\\
        % 
        \hline
        %
        \multirow{ 2}{*}{Dualisms} 
        & D.1 & Context Information & \gpt's performance depends on the amount of context provided\\
        & D.2 & Unprompted Responses & \gpt makes inferences/suggestions beyond what is specified in the prompt
    \end{tabular}}
    \caption{\textbf{\gpt's key properties for \cdam} To facilitate discussion of \gpt's applicability for design and manufacturing (Des.\&Mfg.), we have identified 9 key observations about \gpt that persist across several aspects of the \cdam workflow. This includes 3 powerful capabilities, 4 limitations, and 2 dualisms (so named because they may manifest either as an opportunity or a drawback, depending on the context). We use these observations to frame our discussions about \gpt's suitability for each stage of the \cdam workflow.}
    \label{tab:key_properties}
%\vspace{-5mm}
\end{table}


Our methodology is crafted to provide a comprehensive inspection of the opportunities for and efficacy of various interfaces between \gpt and the \cdam workflow. 
We investigate each of the five stages of the design and manufacturing pipeline individually.
As illustrated in \fref{fig:llm_graph}, these stages include: 
design generation (\sref{sec:text_to_design}), 
design space generation (\sref{sec:text_to_design_space}), 
design for manufacturing (\sref{sec:design_for_manufacturing}), 
performance prediction (\sref{sec:design_to_perf}), 
and inverse design (\sref{sec:inverse}).

In each of these stages, we pose fundamental questions about ways in which \gpt may offer some benefit, and then conduct a series of experiments to answer these questions.
For each query, we investigate aspects such as 
(1) strategies for engineering effective prompts,
(2) strategies for integrating human feedback, expertise, or preferences into the \llm-assisted design process, and
(3) tasks that \gpt can accomplish natively versus tasks that are better completed by asking \gpt to leverage external tools.

After a detailed examination of each stage, we sought to understand the implications of incorporating \gpt within an end-to-end \cdam process. 
To this end, we designed and fabricated two practical examples (a cabinet and a quadcopter) with \gpt's support.
The end-to-end design process for each example is detailed in \sref{sec:examples}.

Beyond these individual questions, our comprehensive investigation has also exposed several key insights about \gpt's general capabilities and limitations with respect to \cdam. 
We have also observed a group of properties that we term 'dualisms', because they may manifest either as an opportunity or a drawback, depending on the situation. 
Our findings are summarized in \tref{tab:key_properties}, with a full description in \sref{sec:capabilities}. 
To emphasize the pervasive nature of these properties, we also use these labels as a framework for our discussions and takeaways at the end of each section. 
Specifically, we draw on each section's findings and examples in order to illustrate the manifestation and impact of various properties in \tref{tab:key_properties} throughout the \cdam workflow. 




\subsection{Scope of Evaluation}
\label{sec:domains}

To conduct a holistic survey of \gpt-assisted \cdam, our experiments span a number of different 
design domains (\sref{sec:overview_domains_design}), 
performance metrics (\sref{sec:overview_domains_perf}) and 
manufacturing methods (\sref{sec:overview_domains_manufacturing}).
Here, we briefly describe each domain of interest, along with the specific challenges they pose and the sort of representative, transferable insight we hope to glean by studying each domain in connection with \llms. 

\subsubsection{Target Design Domains} 
%\team{Liane}
\label{sec:overview_domains_design}
Our experiments are concentrated in three main design domains, including 2d vector graphic design, 3D parametric modeling, and articulated robotics problems.  

% \paragraph{Vector Graphics} 
%\team{Liane, Mike}
Vector graphics use a series of text-based commands to represent paths and areas that form a given design.
Vector image formats are an important part of \cdam, as they can be used as both a design specification and a manufacturing specification for \eg laser cutters. 
Despite their simplicity, vector graphics can represent a wide range of 2D and 3D objects, such as artistic engravings or flat-pack furniture.
We examine \llms' capacity to generate two popular vector formats: SVG and DFX.
These formats present several challenges: 
they contain boilerplate formatting that \gpt may struggle to reproduce; 
it may be difficult to layout individual pieces on the canvas; and finally, it may be difficult to decompose higher-dimensional designs into 2d.
Thus, vector graphics will test \gpt's spatial reasoning and ability to respect highly-constrained syntax, either on its own or with the use of external libraries.


% \paragraph{3D Parametric Geometry} 
%\team{Liane, Felix}
Parametric modeling languages generate 3D geometry through a sequence of constructive instructions. The term ``parametric modeling'' reflects how each constructive operator exposes a set of parameters, such as the radius of a circle.
%
We explore two distinct approaches that are powerful, widely-used, and well-documented online.
The first is rooted in classic Constructive Solid Geometry (CSG), which constructs shapes by successively deploying boolean operations (union, intersection, subtraction) over basic shapes or primitives (such as cuboids, spheres, cylinders, and so forth) that can undergo transformations such as translations, rotations, and scaling.
The CSG approach is intended to test the \textit{global} spatial reasoning capacity of \gpt, as every CSG operation/transformation occurs \wrt a shared coordinate space.
The second representation relies on the contemporary B-rep format used by modern CAD systems.
Here, geometry is built through a sequence of operations like sketching, extruding, and filleting. 
Each operation in this context is parametric and uses references to previously created geometry to \eg, select a plane for a sketch design or select a sketch for an extrusion. 
Sketch-based CAD will test \gpt's ability to effectively switch between and reason over multiple relative, local coordinate frames.


% \paragraph{Robotics} 
%\team{Wil, maybe with input from Allan/Andy/Megan/others}

Robotics offers a particularly rich design domain, as \gpt must coordinate a set of \textit{articulated} and \textit{actuated} geometries to form complex objects such as open chain robot arms, wheeled robots, copters/UAVs, and robot grippers.
Robotics representations must describe not only the high-level geometry of each part, but also their properties and relationships -- including the joints between parts, the degrees of freedom that those joints exhibit, and dynamics information such as the inertia of a given part. 
Several existing formats support these tasks, but we primarily use the XML-based language known as the Universal Robot Description Format (URDF). 
We also investigate the use of a more general graph-based robot representation.
These formats test \gpt's ability to simultaneously reason about multiple aspects of design, such as static geometric bodies and dynamic articulation constraints.



\subsubsection{Target Performance Domains} 
\label{sec:overview_domains_perf}
% \team{Crystal, maybe support from Andy + others?}
% brief overview of what kind of things to expect in the performance exploration section(s)
Diverse performance domains within engineering design require evaluation of aspects such as structural and material properties, mechanical integrity, geometry-based functionality, materials use, electromechanical integration, and subjective features. The results of such evaluation allow us to (dis)qualify a design for use and to use the evaluation to further understand and improve the design. Using \gpt, we focus on assessing mechanical and structural properties through generating first-order analysis equations for input designs of standard objects like chairs, cabinets, and a quadcopter, which test the ability of \gpt to sufficiently understand a given input design in text form or through a DSL and to evaluate criteria for functionality and failure. Mechanical properties assessed include weight, size, load capacity, storage capacity, and stability. Analysis of electromechanical functionality include battery life and quadcopter travel distance. Further use of \gpt aims to streamline the computationally intensive process of Finite Element Analysis (FEA), a crucial tool for understanding structural behavior in detail under various conditions, and we apply this to the case of a load on a set of chairs. 

In addition to these technical aspects, our investigation extends into the subjective domains of sustainability and aesthetics, which cannot be strictly quantified. The inherent complexity and qualitative nature of these areas present unique challenges in evaluation. While it is well-known that computational systems can compute quantitative features, machine learning systems are becoming more sophisticated in artistic domains, and so we seek to leverage the capacity of \llms for lexical analysis to aid more holistically in the more ambiguous realms of the design process and to find its limits. For example, could an \llm reasonably address whether a piece of furniture of a given size is ``large'', or if a shoe of a given design is ``comfortable,'' or can it only handle classically quantifiable features? Can it even help us to reason more objectively about what aspects delineate these properties? To this end, we test evaluation of subjective domains and use \gpt to generate a scoring system and functions for quantifying the sustainability of a chair, the classification of chairs based on categories of aesthetic influence, and the appropriate distribution of a set of chairs into a set of rooms in a house, among other examples. %\liane{I think this is the only place where we give an explicit preview of the findings (eg, gpt does well/poorly) rather than just the questions we're asking. Maybe defer the findings? } \amy{what's our motivation for asking if gpt-4 can help with criteria like aesthetics? is that something that we wish was computable? maybe need a few more words here since it's not us simply trying to replicate existing parts of the computational design workflow, (and it seems like our other goals have been oriented towards that). I think most people would accept that the onus is on a human to pass judgement on things like aesthetics.}

We further combine these performance metric evaluations with the principles of inverse design. Inverse design entails setting desired performance attributes and employing computational methodologies to deduce design parameters that satisfy these attributes, both by generating areas for improvement within a design domain and by testing the effects of implementing improvements suggested by \gpt or target design goals of our own interest,
%implementation\amy{I don't understand what either of these mean in this context... what is "their implmentation", how do you "implement conceptual areas of improvement"? and areas of improvement only seem to apply to an existing design -- should it be iteratively generating areas of improvement?}
as well as selecting appropriate methods of optimization. In this case, given a design/decision space for an object, we use \gpt to generate and implement methods to computationally improve or optimize qualifying designs to satisfy designated performance goals. This methodical approach evaluates if \llms can apply constructive logic for design enhancement and innovation.

\subsubsection{Target Manufacturing Domains} 
\label{sec:overview_domains_manufacturing}

 Leveraging language models like \gpt in DfM context can yield more consistent and scalable decision-making, potentially augmenting human expertise and reducing our reliance on CAD software usage. Potential applications of \gpt include the selection of optimal manufacturing techniques, suggestion of design modifications that would enable easier production, identification of potential suppliers, and creation of manufacturing instructions. The approach is aimed to alleviate many of the bottlenecks caused by the designers' lack of knowledge and experience in DfM. 

In a set of experiments, we've explored \gpt's capabilities across various tasks. Firstly, \gpt was used to identify the optimal manufacturing process for a given part, considering factors such as part geometry, material, production volume, and tolerance requirements. Next, \gpt was tasked with optimizing a component design for CNC machining. Given the geometry of the component, \gpt identified potential manufacturing difficulties and modified the design to address these. We also leveraged \gpt's extensive dataset knowledge to identify parts needed for manufacturing.

In addition to these, \gpt was used to create manufacturing instructions for both additive and subtractive design processes. Additive design can be challenging due to the need for spatial reasoning, precision, and meticulous planning, and often requires many iterations. We've explored the generation of fabrication instructions using subtractive manufacturing techniques for a cabinet design. We also investigated \gpt's potential in generating machine-readable instructions for robot assembly tasks and converting those into human-readable standard operating procedures. This allowed for effective communication and collaboration between robots and human operators.












































% ==========================
% OLD TEXT
% ==========================

\ignore{




%\paragraph{Roadmap}



% Rather than proposing and evaluating a specific framework, our hope is that the dissemination of our findings to the broader community will stimulate fresh insights and contribute to the development of innovative frameworks. We firmly believe that if these new frameworks are founded on robust empirical data, as provided by this study, it will lead to a more effective integration of LLMs across a variety of tasks within the realm of design and fabrication.



% Each of the upcoming sections will explore different ways to incorporate {\llm}s into any given arrow within the \llmgraph described in \sref{sec:llm_graph}.
% Our early sections explore a variety of domains and examples, while successive
% sections hone in on a narrower subset of the examples that best allow us to deeply explore a given line of inquiry.
% To illustrate the full \llm-augmented \cdam pipeline, we also have two extended examples that are considered in every section: 
% a carpentry example (cabinet), and a simple robotics example (quadcopter).

% disclaimers/scoping the paper to say what we're aiming to do, create realistic expectation
% exploring what's possible, trying to find and express limitations

% \adriana{Would be good to talk about the scope and disc lamers today in the meeting i can write this.}


% We will conclude the overview section with out key insights then 







% \subsection{Key Takeaways}

% In our analaysis of \gpt's abilities in desing and fabrication we have extracted a few key capabiltiies that have been good at supporting design as well as a few key limiations. In what follows we wll list those and summarize how they manifest and differnt methods we have used or considered for addressing the limiations. We will concreize them in the each of the sections alter and we will refer to these points when we do. 

% Capabilities:
% C. 1) Extensive Knowledge Base and Capabilities in Design and Fabrication domain, being able to solve for a bunch of things and auto-complete specifications. Different examples in differente domains
% - Degins: 

% c.2) Allows for iteration - if you point a mistake it will try to correct it, wont always work  but this is a good thing.

% c.3) Allows for modality - if you work with it to create something adn they ask to reuse it (as long as you give it bad cause it can forget) it will do it.


% Limiatations
% L1) Reasoning Challenges 
% It is bad at certain types of reasoning, particularly ones that require compuations. This reveals itself as chalelnges with spacial reasoning.

% Potential solutions: There are two key solutions to this problem. One is novel Domain specific languages (DSLs). DSLs are essecially a way to  encapsulate knowledge—they capture recurring invariances, rules, and abstractions that have repeatedly proven beneficial. By using the right DSLs we can bypass the knowlege gap.Second if using APIs that can do the resoning that it's bad at. GPT is good at high-level abstractions and we can export those and then plug them in to solvers that can compute the final answer from these abstractions.  

% L2) Correctness and Verification
% it will often get things wrong and it can't verify itse'f

% Potential Solutions: In additoinal to having a human verififier, we can automate verification by using APIs that can run checks an validations. For Fixing issues, we can leverage (C2) to then feed this information back in and loop until a propor solutoins is given. 


% L3) Scalability
% it struggles as problems get larger or more complex. This reveals itself with challenges of doing multiple thigns at the same time.  

% Potential Solutions: Solutuions are to ask it for one thig at the time (e.g. in evaluating peformance instead of asking for all the metrics we want we can ask one at a time and this helps). When we want to build more complex models  the idea of incremntality is also helpful, but here we need to compose them in the end. We can lever use c3 to enable constructing more complex models from a sequence of instructions. 






%% subsection: Domains



% \subsection{Paper Organization} \team{Wojciech}
% Each of the upcoming sections will explore different ways to incorporate {\llm}s into any given arrow within the \llmgraph described in \sref{sec:llm_graph}.
% Our early sections explore a variety of domains and examples, while successive
% sections hone in on a narrower subset of the examples that best allow us to deeply explore a given line of inquiry.
% To illustrate the full \llm-augmented \cdam pipeline, we also have two extended examples that are considered in every section: 
% a carpentry example (cabinet), and a simple robotics example (quadcopter).

% disclaimers/scoping the paper to say what we're aiming to do, create realistic expectation
% exploring what's possible, trying to find and express limitations

% \adriana{Would be good to talk about the scope and disc lamers today in the meeting i can write this.}

}
