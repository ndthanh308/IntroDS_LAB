\section{DSLs and Prompting Tips for Text-To-Design with \gpt }


\subsection{CSG with \jscad}
\label{sec:appx-jscad-hints} 
\gpt was able to use the \jscad library out-of-the-box, with no additional explanation or restriction of the API on the part of the user. 
However, as described in \sref{sec:textToDesign_JSCAD_basic}, \gpt did fall into a number of common pitfalls when constructing designs.
To mitigate the most common mistakes that \gpt made, each time we asked \gpt to build a design using \jscad, we provided the set of hints and reminders shown in \fref{fig:appx-jscad-hints}.

% Figure environment removed






\subsection{Sketch-based parametric CAD DSL}
\label{sec:sketch_based_dsl}
We propose a streamlined version of the standard sketch-based CAD language by exposing only the sketch and extrude operations along with basic sketch primitives, which already cover a wide range of geometric variations. To automatically generate CAD models from \gpt's output, we utilize Onshape's API.  
When aiming for single-shot CAD design (\ie, with no iterative feedback), we found that a four-pronged prompt generally resulted in the most reliable output.
One aspect of the prompt described the specific task that \gpt should complete.
The remaining three aspects of the prompt provided generic context for our target CAD DSL, and largely remained constant throughout our experiments.
The specific aspects were: (1) a description of our modified DSL, (2) an example constructed with this DSL, and (3) a set of tips that \gpt should keep in mind when constructing its own result.

The prompt we used to describe these aspects to work with local coordinate systems and a global coordinate system can be seen in chat format in Fig.\ref{fig:dsl_local} and Fig.\ref{fig:dsl_global}, respectively.
% Figure environment removed

% Figure environment removed







\subsection{URDF}

\gpt was able to use URDF without any intermediate libraries. Similar to OpenJSCAD, there were many common pitfalls that needed to be mitigated via prompt choice --- these are discussed in detail in section \ref{sec:textToDesign_urdf}. 

In brief summary, the following notes list some additions that were useful in mitigating specific problems:

\begin{itemize}
    \item \gpt has difficulties in determining where URDF objects place their origin. When wanting objects to touch but not intersect, or be placed at the ``end'' of other objects, it is useful to specify that the ends are half the length of the object away from the origin. 
    \item Specifying an axis for two objects to be aligned along is more effective than instructing that they be aligned.
    \item \gpt will often omit essential parts of the URDF file for brevity, replacing them with a comment to repeat a part of the file. This can be done manually, but to generate URDF files that are complete directly from the response, \gpt must be instructed to produce a complete file.
    \item \gpt will ignore several constraints or instructions if too many are placed in a single prompt. Splitting the generation process into multiple prompts resolves this issue.
\end{itemize}





\subsection{Graph-based DSL for Robotics}
\label{sec:appx_graph_robots}
The full text of the prompt used to generate the humanoid robot graph (omitted earlier for brevity) is shown in \fref{fig:appx-graph-DSL}.

% Figure environment removed








% \section{Using \gpt to Design and Evaluate an RC Circuit}

% \gpt is capable of designing simple electrical systems and suggesting potential performance improvements for them if provided a trade-off between two variables. An example of this is when \gpt was prompted to first design a circuit for reading a resistive sensor with a value of $500-1000\Omega$, and subsequently analyze the trade-offs between signal and noise. \gpt designed a simple voltage divider circuit in python, having two resistors. The voltage across the sensor can be measured and converted to a digital value using an ADC connected to a Raspberry Pi via its GPIO pins - \gpt correctly used the RPi.GPIO library to read the analog value from the ADC pin. The correct equation was chosen to calculate the voltage at the ADC pin and the resistance of the sensor. 

% In order to minimize noise in the signal, \gpt identified four trade-offs: signal quality, sensor accuracy, noise filtering, and grounding and shielding. To verify if \gpt is capable of implementing one of its own suggestions, we prompted it to add another capacitor in parallel with the resistive sensor and to quantify by how much noise would be reduced. It recognized that adding a large capacitance of 10 Farads would substantially reduce noise but would be highly impractical since it would be extremely large and expensive. Therefore, it suggested a capacitor of 0.1 $\mu F$, a commonly used value in filtering to reduce noise, and provided a series of equations that calculates the reduced noise level with the capacitor. \gpt produced the equation to calculate the cutoff frequency and the related noise reduction factor. Through this example, it is evident that \gpt is able to reason the equations governing circuits, the feasibility of components, and ways to improve electrical systems through its understanding of inherent trade-offs, as well as to generate equations to assess the improvement achieved with the redesign.