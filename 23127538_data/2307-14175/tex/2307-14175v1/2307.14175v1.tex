\documentclass[12pt]{article}

\usepackage{amsmath,amssymb,amsfonts,amsthm}
\usepackage{graphicx}
\usepackage{cite}
\usepackage[all]{xy}

\textheight=242mm
\textwidth=176mm

\hoffset=-18mm
\voffset=-28mm

%%%%%%%%%%%%%%%%%%%%%
\allowdisplaybreaks[3]
%%%%%%%%%%%%%%%%%%%%%
\newcommand{\abull}[1]{\noindent
	\parbox[t]{0.04\textwidth}{\hfill $\bullet$\phantom{.}}
	\parbox[t]{0.95\textwidth}{#1 \\[-0.5\baselineskip]} \par}
%%%%%%%%%%%%%%%%%%%%%
%%% Proposition %%%
\newcounter{propositiona}
\newcommand{\propositiona}[1]{\refstepcounter{propositiona}
\noindent
\textbf{Proposition \thepropositiona.}\, {\it #1}}
%%%%%%%%%%%%%%%%
%%% Definition %%%%%%%
\newcounter{definitiona}
\newcommand{\definitiona}[1]{\refstepcounter{definitiona}
\noindent
\textbf{Definition \thedefinitiona.}\, #1}
%%%%%%%%%%%%%%%
%%% Remark %%%%%%%
\newcounter{remarka}
\newcommand{\remarka}[1]{\refstepcounter{remarka}
\noindent
\textbf{Remark \theremarka.}\, #1}
%%%%%%%%%%%%%%%
%%% Example %%%%%%%
\newcounter{examplea}
\newcommand{\examplea}[1]{\refstepcounter{examplea}
\noindent
\textbf{Example \theexamplea.}\, #1}
%%%%%%%%%%%%%%%
%%% Lemma %%%%%%%
\newcounter{lemmaa}
\newcommand{\lemmaa}[1]{\refstepcounter{lemmaa}
\noindent
\textbf{Lemma \thelemmaa.}\, {\it #1}}
%%%%%%%%%%%%%%%
%%% Theorem %%%
\newcounter{theorema}
\newcommand{\theorema}[1]{\refstepcounter{theorema}
\noindent
\textbf{Theorem\, \thetheorema.}\, {\it #1}}
%%%%%%%%%%%%%%%%
%%% Corollary %%%
\newcounter{corollarya}
\newcommand{\corollarya}[1]{\refstepcounter{corollarya}
\noindent
\textbf{Corollary\, \thecorollarya.}\, {\it #1}}
%%%%%%%%%%%%%%%%

%\usepackage[long]{datetime}
%\newcommand{\No}{\textnumero}
%\usepackage{refcheck}
%\usepackage{showkeys}
%%%%%%%%%%%%%%%

\begin{document}
	
%\large

\thispagestyle{empty}

\begin{center}
{\bf \Large Internal Lagrangians of PDEs as variational principles}\\[4ex]
{\large \bf Kostya Druzhkov}\\[2ex]
%Moscow Institute of Physics and Technology (State
%University),\\ 9 Institutskiy per., Dolgoprudny, 141701 Moscow region, Russia\\
\textit{E-mail: konstantin.druzhkov@gmail.com}
\end{center}

\vspace{0.65ex}

\begin{abstract}\normalsize
A notion of stationary points of an internal Lagrangian is introduced. A connection between symmetries, conservation laws and internal Lagrangians is established. Noether's theorem is formulated in terms of internal Lagrangians. A relation between non-degenerate Lagrangians and the corresponding internal Lagrangians is investigated.\\[-1ex]
\end{abstract}

\noindent
%\centerline{\textbf{\large Abstract}}\\[1ex]
%{A notion of stationary points of internal Lagrangians is introduced. A connection between symmetries, conservation laws and internal Lagrangians is established. Noether's theorem in terms of internal Lagrangians is proposed. A relation between non-degenerate Lagrangians and the corresponding internal Lagrangians is investigated.\\[1ex]
%A notion of internal Lagrangian for a system of differential equations is introduced. A spectral sequence related to internal Lagrangians is obtained. A connection between internal Lagrangians and presymplectic structures is investigated. An interpretation of the term $E^{3,\, n-2}_2$ of Vinogradov's $\mathcal{C}$-spectral sequence is given for irreducible gauge theories.\\[1.0ex]
\textit{Keywords:}
Variational principle, Presymplectic structure, Noether's theorem, non-degenerate Lagrangian\\
%Lagrangian, Variational principle, $\mathcal{C}$-spectral sequence, Presymplectic structure

\section{Introduction\\[-3ex]}{\label{Intr}}

\

Many equations of mathematical physics originate from the principle of stationary action. Some equations are indirectly related to action functionals through their differential consequences. One can say that all such equations, in one way or another, have a variational nature. They encode their variational nature through elements of certain cohomologies, which we call internal Lagrangians. The notion of an internal Lagrangian was introduced in a recent article~\cite{Druzhkov}, where it was established that each variational principle associated with a system of differential equations determines a unique internal Lagrangian.

In this paper we show that internal Lagrangians determine variational principles in terms of the intrinsic geometry of differential equations. Internal variational principles are related to integral functionals defined on particular classes of submanifolds of differential equations. Informally speaking, such submanifolds are composed of initial-boundary conditions lifted to infinitely prolonged equations.
Despite the generality of the considered construction, its application to gauge theories proves to be challenging. Perhaps the approach needs some modification in this case.

The paper is organized as follows. In Section~\ref{Def} we recall some basic concepts from the geometry of differential equations. 
Section~\ref{VariL} explains why any variational principle for a system of differential equations is encoded in its intrinsic geometry. A notion of stationary points of an internal Lagrangian is introduced.
Section~\ref{ILcl} focuses on a relation between internal Lagrangians and conservation laws.
In Section~\ref{iLs} we present Noether's theorem in terms of the intrinsic geometry of differential equations and discuss gauge invariance of internal Lagrangians.
Section~\ref{Compl} is devoted to the Euler-Lagrange equations for non-degenerate Lagrangians. We show that, under certain reasonable restrictions, stationary points of the corresponding internal Lagrangians are precisely local solutions.
Section~\ref{Exam} contains several examples.

\textit{All manifolds and functions considered in this paper are assumed to be smooth of the class $C^{\infty}$. All submanifolds are assumed to be embedded.}

\section{Jets and equations\\[-3ex]}\label{Def}

\

Let us introduce some notation and briefly recall basic facts from the geometry of differential equations. More details can be found in~\cite{VinKr}.


\subsection{Jets\\[-3ex]}

\

Let $\pi\colon E\to M$ be a locally trivial smooth vector
bundle over a smooth manifold $M$, $\mathrm{dim}\, M = n$, 
$\mathrm{dim}\, E = n + m$. The bundle $\pi$ allows one to introduce the corresponding jet bundles
\begin{align*}
\xymatrix{
\ldots \ar[r] & J^3(\pi) \ar[r]^-{\pi_{3, 2}} & J^2(\pi) \ar[r]^-{\pi_{2, 1}} & J^1(\pi) \ar[r]^-{\pi_{1, 0}} & J^0(\pi) = E \ar[r]^-\pi & M
}
\end{align*}
and the inverse limit $J^{\infty}(\pi)$ of the chain. The manifold $J^{\infty}(\pi)$ arises together with the natural projections 
$\pi_{\infty}\colon J^{\infty}(\pi)\to M$ and $\pi_{\infty,\, k}\colon J^{\infty}(\pi)\to J^k(\pi)$. Denote by $\mathcal{F}(\pi)$ the algebra of smooth functions on~$J^{\infty}(\pi)$.\\[-2ex]

\noindent
\textbf{Local coordinates.} Suppose $U\subset M$ is a coordinate neighborhood such that the bundle $\pi$
becomes trivial over $U$. Choose local coordinates $x^1$, \ldots, $x^n$ in $U$ and $u^1$, \ldots, $u^m$  
along the fibers of $\pi$ over $U$. It is convenient to introduce
multi-index $\alpha$ as a formal sum of the form $\alpha = \alpha_1 x^1 + \ldots + \alpha_n x^n = \alpha_i x^i$, where all $\alpha_i$ are non-negative integers, $|\alpha| = \alpha_1 + \ldots + \alpha_n$.
We denote by $u^i_{\alpha}$ the corresponding adapted local coordinates on $J^{\infty}(\pi)$.\\[-2ex]

\noindent
\textbf{Cartan distribution.} The main structure on $J^{\infty}(\pi)$ is the Cartan distribution $\mathcal{C}$. 
The Cartan distribution is spanned by the total derivatives (using the summation convention)
$$
D_{x^i} = \partial_{x^i} + u^k_{\alpha + x^i}\partial_{u^k_{\alpha}}\qquad\quad i = 1,\ldots, n.
$$

\noindent
\textbf{Cartan forms.} The Cartan distribution determines the ideal $\mathcal{C}\Lambda^*(\pi)\subset \Lambda^*(\pi)$ 
of the algebra of differential forms on $J^{\infty}(\pi)$.
The ideal $\mathcal{C}\Lambda^*(\pi)$ consists of Cartan forms, i.e., differential forms vanishing on the Cartan distribution.
A Cartan $1$-form $\omega\in\mathcal{C}\Lambda^1(\pi)$ can be written as a finite sum
$$
\omega = \omega_i^{\alpha}\theta^i_{\alpha}\,,\qquad\ \theta^i_{\alpha} = du^i_{\alpha} - u^i_{\alpha + x^k}dx^k
$$
in adapted local coordinates. Here the coefficients $\omega_i^{\alpha}$ are smooth functions defined on a coordinate domain of $J^{\infty}(\pi)$. We denote by $\mathcal{C}^p\Lambda^*(\pi)$ the $p$-th power of the ideal $\mathcal{C}\Lambda^*(\pi)$.\\[-2ex]

\noindent
\textbf{Infinitesimal symmetries.}
Let $\varkappa(\pi) = \Gamma(\pi^*_{\infty}(\pi))$ be the $\mathcal{F}(\pi)$-module
of sections of the pullback $\pi^*_{\infty}(\pi)$. 
If $\varphi\in \varkappa(\pi)$, there is the corresponding evolutionary vector field on $J^{\infty}(\pi)$
$$
E_{\varphi} = D_{\alpha}(\varphi^i)\partial_{u^i_{\alpha}}\,,
$$
where $\varphi^1$, \ldots, $\varphi^m$ are components of $\varphi$ in adapted local coordinates, $D_{\alpha}$ denotes the composition $D_{x^1}^{\ \alpha_1}\circ\ldots\circ D_{x^n}^{\ \alpha_n}$. Evolutionary vector fields are infinitesimal symmetries of $J^{\infty}(\pi)$. In particular, $\mathcal{L}_{E_{\varphi}}(\mathcal{C}\Lambda^*(\pi))\subset \mathcal{C}\Lambda^*(\pi)$. Here $\mathcal{L}_{E_{\varphi}}$ is the corresponding Lie derivative.\\[-2ex]

\noindent
\textbf{Horizontal forms.}
Cartan forms allow one to consider horizontal $k$-forms
$$
\Lambda^k_h(\pi) = \Lambda^k(\pi)/\mathcal{C}\Lambda^k(\pi)\,.
$$
A differential $k$-form $\omega\in\Lambda^k(\pi)$ represents the corresponding horizontal $k$-form $[\omega]_h = \omega + \mathcal{C}\Lambda^k(\pi)$.
The infinite jet bundle $\pi_{\infty}\colon J^{\infty}(\pi) \to M$ admits the decomposition 
$$
\Lambda^1(\pi) = \mathcal{C}\Lambda^1(\pi) \oplus \mathcal{F}(\pi)\!\cdot\!\pi^*_{\infty}(\Lambda^1(M))\,.
$$
The module of horizontal $k$-forms $\Lambda^k_h(\pi)$ can be identified with the module $\mathcal{F}(\pi)\cdot \pi^*_{\infty}(\Lambda^k(M))$.
Due to the inclusion $d(\mathcal{C}\Lambda^*(\pi)) \subset \mathcal{C}\Lambda^*(\pi)$, the de Rham differential $d$ induces the horizontal differential
\begin{equation*}
d_h\colon \Lambda^{*}_h(\pi)\to \Lambda^{*}_h(\pi).
\end{equation*}

\noindent
\textbf{Adjoint modules.}
Let $\eta\colon E_1\to M$ be a locally trivial smooth vector
bundle. Suppose $P(\pi)$ is the module of sections of the pullback $\pi^*_{\infty}(\eta)$:
$$
P(\pi) = \Gamma(\pi^*_{\infty}(\eta))\,.
$$
There exists the adjoint module $\widehat{P}(\pi)$,
\begin{equation*}
\widehat{P}(\pi) = \mathrm{Hom}_{\mathcal{F}(\pi)}(P(\pi), \Lambda^n_h(\pi))\,.
\end{equation*}
\noindent
\textbf{Euler operator.}
By $\mathrm{E}$ we denote the Euler operator (variational derivative), $\mathrm{E}\colon \Lambda^n_h(\pi)\to \widehat{\varkappa}(\pi)$. Suppose $L\in\Lambda^n(\pi)$ is a differential $n$-form. If $L$ is of the form $\lambda \, dx^1\wedge\ldots\wedge dx^n$, then
\begin{align*}
&\mathrm{E}[L]_h = (-1)^{|\alpha|}D_{\alpha}\Big(\dfrac{\partial \lambda}{\partial u^i_{\alpha}}\Big)\, \theta^i_0\wedge dx^1\wedge\ldots\wedge dx^n\,,\\
&\langle \mathrm{E}[L]_h, \varphi\rangle = i_{E_{\varphi}}\,\mathrm{E}[L]_h = (-1)^{|\alpha|}D_{\alpha}\Big(\dfrac{\partial \lambda}{\partial u^i_{\alpha}}\Big)\varphi^i\, dx^1\wedge\ldots\wedge dx^n\,.
\end{align*}
Here $\langle \cdot , \cdot \rangle$ is the natural pairing between a module and its adjoint.%; $|\alpha| = \sum_{i=1}^{n}\alpha_i$.

The Noether formula links the Lie derivative $\mathcal{L}_{E_{\varphi}}[L]_h$ of a horizontal $n$-form $[L]_h$ and the variational derivative $\mathrm{E}[L]_h$. Namely,
there exists a Cartan form $\omega_L\in \mathcal{C}\Lambda^{n}(\pi)$ such that
\begin{align}
\mathcal{L}_{E_{\varphi}}[L]_h = \langle \mathrm{E}[L]_h, \varphi \rangle + d_h [i_{E_{\varphi}}\omega_L]_h\,.
\label{Noeth}
\end{align}
By $\delta/\delta u^i$ we denote the variational derivative (or simply variation) with respect to~$u^i$,
\begin{align*}
\dfrac{\delta}{\delta u^i} = (-1)^{|\alpha|}D_{\alpha}\circ \partial_{u^i_{\alpha}}\,.
\end{align*}




\subsection{Differential equations\\[-3 ex]}

\

Let $F$ be a (smooth) section of some bundle of the form $\pi^*_{r}(\eta)$. We assume that for each $p\in \{F = 0\}\subset J^{r}(\pi)$, the differentials of coordinate functions $dF^i|_{p}$ are linearly independent. So, $\{F = 0\}\subset J^{r}(\pi)$ is an embedded submanifold. By the infinite prolongation of the differential equation $F = 0$ we mean the subset $\mathcal{E}\subset J^{\infty}(\pi)$ defined by the following infinite system of equations
\begin{align*}
\quad D_{\alpha}(F^i) = 0\,,\qquad |\alpha| \geqslant 0\,.
\end{align*}
Henceforth, we assume that $\pi_{\infty}(\mathcal{E}) = M$.\\[-1ex]

\remarka{We do not require that the number of equations of the form $F^i = 0$ coincide with the number of unknowns.}\\[-1 ex]

By $\mathcal{F}(\mathcal{E})$ we denote the algebra of smooth functions on $\mathcal{E}$, 
$$
\mathcal{F}(\mathcal{E}) = \mathcal{F}(\pi)|_{\mathcal{E}} = \mathcal{F}(\pi)/I\,.
$$
Here $I$ denotes the ideal of the system $\mathcal{E}\subset J^{\infty}(\pi)$. The algebra of smooth functions produces the algebra of differential forms $\Lambda^*(\mathcal{E}) = \Lambda^*(\pi)|_{\mathcal{E}}$. The Cartan distribution on $J^{\infty}(\pi)$ can be restricted to $\mathcal{E}$. We denote by $\mathcal{C}_p$ the Cartan plane at a point $p\in\mathcal{E}$. 
Similarly, there is the ideal $\mathcal{C}\Lambda^*(\mathcal{E})\subset \Lambda^*(\mathcal{E})$ consisting of differential forms that vanish on the Cartan distribution (on $\mathcal{E}$). Let us also introduce the following $\mathcal{F}(\mathcal{E})$-modules
$$
\varkappa(\mathcal{E}) = \varkappa(\pi)/I\cdot \varkappa(\pi)\,,\qquad\quad P(\mathcal{E}) = P(\pi)/I\cdot P(\pi).
$$


\noindent
\textbf{$\mathcal{C}$-differential operators.} By a \textit{$\mathcal{C}$-differential operator} we mean an operator in total derivatives. If $Q_1$ and $Q_2$ are modules of sections of some vector bundles over an infinitely prolonged system of differential equations $\mathcal{E}$, then we denote by $\mathcal{C}(Q_1, Q_2)$ the module of $\mathcal{C}$-differential operators from $Q_1$ to $Q_2$.\\[-2ex]


\noindent
\textbf{Regularity assumptions.}
We say that the infinite prolongation $\mathcal{E}$ of a system of differential equations $F = 0$ is \textit{regular} if for each
function $f\in \mathcal{F}(\pi)$ vanishing on $\mathcal{E}$ (i.e., $f\in I$), there exists a $\mathcal{C}$-differential operator $\Delta\colon P(\pi)\to \mathcal{F}(\pi)$ such that $f = \Delta(F)$.


\textit{In what follows, we
assume that the de Rham cohomology groups $H^i_{dR}(\mathcal{E})$ are trivial for $i > 0$ and consider only regular systems.} These conditions are not restrictive.
The case of non-trivial de Rham cohomology does not lead to essential complications. In this case one can consider quotients by
subgroups of topological (locally trivial) elements.\\[-2ex]


\noindent
\textbf{Infinitesimal symmetries.} A \textit{symmetry} (more precisely, infinitesimal symmetry) of an infinitely prolonged system of equations $\mathcal{E}$ is a vector field $X\in D(\mathcal{E})$ preserving the Cartan distribution in the following sense
$$
[X, \mathcal{C}D(\mathcal{E})]\subset \mathcal{C}D(\mathcal{E}).
$$
Here $\mathcal{C}D(\mathcal{E})\subset D(\mathcal{E})$ is the module of Cartan vector fields, i.e., $\mathcal{F}(\mathcal{E})$-linear combinations of total derivatives.
Elements of $\mathcal{C}D(\mathcal{E})$ are trivial symmetries. Two symmetries of $\mathcal{E}$ are equivalent if they differ by a trivial one.

A differential equation $\{F = 0\}\subset J^{r}(\pi)$ allows us to introduce the linearization $l_{\mathcal{E}}\colon \varkappa(\mathcal{E})\to P(\mathcal{E})$. To this end, we define the $\mathcal{C}$-differential operator $l_F\colon \varkappa(\pi)\to P(\pi)$ by $l_F(\varphi) = E_{\varphi}(F)$ and set $l_{\mathcal{E}} = l_F|_{\mathcal{E}}$.

If $\pi_{\infty,\, 0}(\mathcal{E}) = J^0(\pi)$, symmetries of $\mathcal{E}$ can be identified with elements of $\ker l_{\mathcal{E}}$ using their characteristics (see, e.g.,~\cite{Olver},~\cite{VinKr}).
A gauge symmetry of $\mathcal{E}$ is a symmetry of the form $R(\epsilon)\in \varkappa(\mathcal{E})$, where $R$ is a $\mathcal{C}$-differential operator such that $l_{\mathcal{E}}\circ R = 0$.\\[-2ex]

\noindent
\textbf{$\mathcal{C}$-spectral sequence.} Powers of the ideal $\mathcal{C}\Lambda^*(\mathcal{E})$ are stable with respect to the de Rham differential, i.e., $d(\mathcal{C}^p\Lambda^*(\mathcal{E}))\subset \mathcal{C}^p\Lambda^*(\mathcal{E})$, where
$$
\mathcal{C}^p\Lambda^*(\mathcal{E}) = \underbrace{\mathcal{C}\Lambda^1(\mathcal{E})\wedge\ldots\wedge \mathcal{C}\Lambda^1(\mathcal{E})}_{p}\wedge\, \Lambda^*(\mathcal{E})\,.
$$
Thus, the de Rham complex admits the filtration
$$
\Lambda^*(\mathcal{E})\supset \mathcal{C}\Lambda^*(\mathcal{E})\supset \mathcal{C}^2\Lambda^*(\mathcal{E})\supset \mathcal{C}^3\Lambda^*(\mathcal{E})\supset \ldots
$$
The corresponding spectral sequence $(E^{p,\, q}_r(\mathcal{E}), d_r)$ is the Vinogradov $\mathcal{C}$-spectral sequence~\cite{Vin}. 
In particular, the $\mathcal{C}$-spectral sequence allows one to define conservation laws, variational $1$-forms and presymplectic structures of differential equations.

A \textit{conservation law} of an infinitely prolonged system $\mathcal{E}$ is an element of the group (more precisely, of the vector space) $E^{\,0,\,n-1}_1(\mathcal{E})$. A \textit{variational $1$-form} of $\mathcal{E}$ is an element of the group $E^{\,1,\,n-1}_1(\mathcal{E})$. Conservation laws and variational $1$-forms are related by the differential
$$
d_1^{\,0,\,n-1}\colon E^{\,0,\,n-1}_1(\mathcal{E}) \to E^{1,\,n-1}_1(\mathcal{E}).
$$
A \textit{presymplectic structure} of $\mathcal{E}$ is an element of the kernel of the differential
$$
d_1^{\,2,\,n-1}\colon E^{\,2,\,n-1}_1(\mathcal{E})\to E^{\,3,\,n-1}_1(\mathcal{E}).
$$
A presymplectic structure of $\mathcal{E}$ can be (ambiguously) represented by a $\mathcal{C}$-differential operator $\Delta\colon \varkappa(\mathcal{E})\to \widehat{P}(\mathcal{E})$ such that $l^{\,*}_{\mathcal{E}}\circ \Delta = \Delta^* \circ l_{\mathcal{E}}$ (see, e.g.,~\cite{VinKr}).\\[-2ex]

\noindent
\textbf{Cosymmetries (see, e.g.,~\cite{VinKr}).} A \textit{cosymmetry} of an infinitely prolonged system $\mathcal{E}$ is an element of $\ker l^{\,*}_{\mathcal{E}}$. Cosymmetries are related to variational $1$-forms. Namely, if $\psi$ is a cosymmetry of $\mathcal{E}$, there is a $\mathcal{C}$-differential operator $\nabla\colon \varkappa(\mathcal{E})\to \Lambda^{n-1}_h(\mathcal{E})$ such that
\begin{align}
\langle \psi, l_{\mathcal{E}}(\varphi) \rangle = d_h\nabla(\varphi)\qquad \text{for all}\ \ \varphi\in \varkappa(\mathcal{E})\,.
\label{cos1f}
\end{align}
The operator $\nabla$ determines a $d_0$-closed element of the group $E^{1,\, n-1}_0(\mathcal{E})$. The corresponding variational $1$-form is well defined. In fact, this mapping from cosymmetries to variational $1$-forms is surjective.

If an operator $\Delta$ represents a presymplectic structure $\omega\in \ker d_1^{\,2,\,n-1}$ and $\varphi\in \ker l_{\mathcal{E}}$ corresponds to a symmetry $X$, then $\Delta(\varphi)$ is a cosymmetry, and $\Delta(\varphi)$ corresponds to the variational $1$-form $i_X\omega$.\\[-2ex]

\noindent
\textbf{Internal Lagrangians (see~\cite{Druzhkov}).} An \textit{internal Lagrangian} of an infinitely prolonged system $\mathcal{E}$ is an element of the group
\begin{align*}
\widetilde{E}^{\,0,\, n-1}_1(\mathcal{E}) = \dfrac{\{l\in\Lambda^n(\mathcal{E})\colon\ dl\in \mathcal{C}^2\Lambda^{n+1}(\mathcal{E})\}}
{\mathcal{C}^2\Lambda^{n}(\mathcal{E}) + d(\Lambda^{n-1}(\mathcal{E}))}\,.
\end{align*}

If the variational derivative $\mathrm{E}[L]_h$ of a 
horizontal $n$-form $[L]_h\in\Lambda^n_h(\pi)$ vanishes on $\mathcal{E}\subset J^{\infty}(\pi)$, then there exists a Cartan form $\omega_L\in\mathcal{C}\Lambda^{n}(\pi)$ such that $d(L + \omega_L) - \mathrm{E}[L]_h \in\mathcal{C}^2\Lambda^{n+1}(\pi)$. The form $(L + \omega_L)|_{\mathcal{E}}$ represents an element of the group
\begin{align}
\dfrac{\{l\in\Lambda^n(\mathcal{E})\colon\ dl\in \mathcal{C}^2\Lambda^{n+1}(\mathcal{E})\}}
{\mathcal{C}^2\Lambda^{n}(\mathcal{E}) + d(\mathcal{C}\Lambda^{n-1}(\mathcal{E}))}
\label{SIL}
\end{align}
and the corresponding internal Lagrangian. The de Rham differential $d$ induces the differential
$$
\tilde{d}^{\,0,\,n-1}_1\colon \widetilde{E}^{\,0,\, n-1}_1(\mathcal{E})\to E^{\,2,\, n-1}_1(\mathcal{E})\,.
$$
As a matter of fact, $\mathrm{im}\, \tilde{d}^{\,0,\,n-1}_1\subset \ker d_1^{\,2,\,n-1}$, and the differential $\tilde{d}^{\,0,\,n-1}_1$ maps internal Lagrangians to presymplectic structures.



\section{\label{VariL} Stationary points of internal Lagrangians\\[-3 ex]}

\

Let $\mathcal{E}$ be an infinitely prolonged system of differential equations with an $n$-dimensional Cartan distribution, and let $l\in \Lambda^n(\mathcal{E})$ be a differential form such that $dl\in\mathcal{C}^2\Lambda^{n+1}(\mathcal{E})$. Suppose $N$ is a compact $n$-dimensional oriented smooth manifold with boundary $\partial N$.
Our immediate goal is to relate the equivalence class 
$$
l + \mathcal{C}^2\Lambda^{n}(\mathcal{E}) + d(\mathcal{C}\Lambda^{n-1}(\mathcal{E}))
$$
to an integral functional.\\[-1ex]

\definitiona{An embedding $\sigma\colon N\to \mathcal{E}$ is an \textit{almost Cartan embedding} if
\begin{align*}
\dim\, (T_p \,\sigma(N) \cap \mathcal{C}_p) \geqslant n-1 \qquad \text{for all}\ \ p\in \sigma(N)\,.
\end{align*}
}

\examplea{Consider the heat equation and some initial conditions for all $t_0\in\mathbb{R}$
$$
u_t = u_{2x}\,,\qquad u = f(x, t_0)\,.
$$
For each fixed $t_0$, one can determine all the derivatives of $u$ using these data. Namely,
\begin{align}
\begin{aligned}
&u = f(x, t_0)\,,\qquad u_x = \partial_x f(x, t_0)\,,\qquad u_t = \partial_x^2 f(x, t_0)\,,\\
&u_{2x} = \partial_x^2 f(x, t_0)\,,\qquad u_{x+t} = \partial_x^3 f(x, t_0)\,,\qquad u_{2t} = \partial_x^{4} f(x, t_0)\,,\qquad\ldots
\end{aligned}
\label{heat}
\end{align}
Substituting $t$ for $t_0$ in~\eqref{heat}, we obtain an embedding of $\mathbb{R}^2$ to the corresponding infinitely prolonged system. The restriction of this embedding to a compact submanifold of $\mathbb{R}^2$ is an almost Cartan embedding since its differential maps vectors of the form $\partial_x$ to vectors of the form $\,\overline{\!D}_x = D_x|_{\mathcal{E}}$.\\[-1ex]
}

The images of almost Cartan embeddings play a crucial role in the construction of integral functionals determined by the intrinsic geometry of differential equations. Since a Cartan distribution is involutive, one can say that such images are composed of initial-boundary conditions lifted to $\mathcal{E}$. Speaking of stationary points of internal Lagrangians, we prefer the Lagrangian approach to the Eulerian one and consider deformations of embeddings rather than motions of their images.\\[-1ex]

\remarka{If $\mathcal{E}$ is an infinitely prolonged system of ODEs, then all embeddings of $N$ to $\mathcal{E}$ are almost Cartan embeddings.}\\[-1ex]

\remarka{We say that an embedding $h\colon N\to \mathcal{E}$ defines a solution to $\mathcal{E}$ if and only if
\begin{align*}
\dim\, (T_p \,h(N) \cap \mathcal{C}_p) = n\qquad \text{for all}\quad p\in h(N)\,.
\end{align*}
}
We denote by $\mathcal{A}_N(\mathcal{E})$ the set of almost Cartan embeddings of $N$ to $\mathcal{E}$. 
It is easy to see that $\sigma^*(\mathcal{C}^2\Lambda^{n}(\mathcal{E})) = 0$ for any $\sigma\in \mathcal{A}_N(\mathcal{E})$.\\[-1ex]


\definitiona{Let $\sigma\colon N\to \mathcal{E}$ be an almost Cartan embedding. We say that $\sigma$ \textit{defines a boundary value problem} if
$T_p \, \sigma(\partial N) \subset \mathcal{C}_p$
for all $p\in \sigma(\partial N)$.}\\[-1ex]


Denote by $\mathcal{BA}_N(\mathcal{E})$ the set of almost Cartan embeddings (of $N$ to $\mathcal{E}$) that define boundary value problems. 
If $\sigma\in \mathcal{BA}_N(\mathcal{E})$, then Stokes' theorem implies
\begin{align*}
\int_N\sigma^*(\omega) = 0\qquad \text{for all}\quad \omega\in d(\mathcal{C}\Lambda^{n-1}(\mathcal{E}))\,.
\end{align*}
So, the equivalence class $l + \mathcal{C}^2\Lambda^{n}(\mathcal{E}) + d(\mathcal{C}\Lambda^{n-1}(\mathcal{E}))$ does indeed define the integral functional
\begin{align*}
S\colon \mathcal{BA}_N(\mathcal{E}) \to \mathbb{R}\,,\qquad S(\sigma) = \int_N \sigma^*(l)\,.
\end{align*}
This result explains why a variational nature of a system of differential equations is encoded in the intrinsic geometry of such a system: \textit{any variational principle for a system of differential equations generates a variational principle in terms of its intrinsic geometry.}\\[-1ex]

\remarka{It is worth mentioning that if the variational derivative $\mathrm{E}[L]_h$ of a 
horizontal $n$-form $[L]_h\in\Lambda^n_h(\pi)$ vanishes on an infinitely prolonged system of equations $\mathcal{E}\subset J^{\infty}(\pi)$, then $[L]_h$ defines a unique element of group~\eqref{SIL}. In addition, the cohomology class $[L]_h + d_h (\Lambda^{n-1}_h(\pi))$ defines a unique internal Lagrangian in this case.}\\[-1ex]

The form $l$ also represents the internal Lagrangian 
$$
\boldsymbol\ell = l + \mathcal{C}^2\Lambda^{n}(\mathcal{E}) + d(\Lambda^{n-1}(\mathcal{E}))\,.
$$
Due to non-trivial boundary terms, an internal Lagrangian of $\mathcal{E}$ ambiguously defines an integral functional on $\mathcal{A}_N(\mathcal{E})$. However, its derivative along a path is well defined. Such a derivative is completely determined by the corresponding presymplectic structure.
To see this, consider a path in $\mathcal{A}_N({\mathcal{E}})$, i.e., a smooth mapping $\gamma\colon \mathbb{R}\times N\to \mathcal{E}$ such that for all $\tau\in\mathbb{R}$ the mappings
$$
\gamma(\tau)\colon N\to \mathcal{E},\qquad \gamma(\tau)\colon x\mapsto \gamma(\tau, x)
$$ 
are almost Cartan embeddings. Define an embedding $0_N\colon N\to \mathbb{R}\times N$ by $0_N(x) = (0, x)$. Then $\gamma(0) = \gamma\circ 0_N$ and
\begin{align*}
\dfrac{d}{d\tau}\Big|_{\tau = 0}\gamma(\tau)^*(l) = 0_N^*\big(\mathcal{L}_{\partial_{\tau}}\gamma^*(l)\big) = d\,0_N^*\big(i_{\partial_{\tau}}\gamma^*(l)\big) + 0_N^*\big(i_{\partial_{\tau}}\gamma^*(dl)\big).
\end{align*}
Suppose the boundary is fixed: for each $x\in \partial N$, the relation
$$
\gamma(\tau)(x) = \gamma(0)(x)
$$
holds for all $\tau\in\mathbb{R}$.
Then the form $0_N^*\big(i_{\partial_{\tau}}\gamma^*(l)\big)\big)$ vanishes on $\partial N$ and the derivative along $\gamma$
\begin{align*}
\dfrac{d}{d\tau}\Big|_{\tau = 0}\int_N \gamma(\tau)^*(l) = \int_N 0_N^*\big(i_{\partial_{\tau}}\gamma^*(dl)\big)
\end{align*}
is determined by the presymplectic structure represented by $dl$.\\[-1ex]


\definitiona{An almost Cartan embedding $\sigma\colon N \to \mathcal{E}$ is a \textit{stationary point} of an internal Lagrangian $\boldsymbol\ell = l + \mathcal{C}^2\Lambda^{n}(\mathcal{E}) + d(\Lambda^{n-1}(\mathcal{E}))$ if
\begin{align*}
\dfrac{d}{d\tau}\Big|_{\tau = 0}\int_N \gamma(\tau)^*(l) = 0
\end{align*}
for any path $\gamma$ in $\mathcal{A}_N(\mathcal{E})$ such that $\gamma(0) = \sigma$ and the boundary is fixed.}\\[-1ex]

The form $dl$ belongs to $\mathcal{C}^2\Lambda^{n+1}(\mathcal{E})$. Then we get zero by substituting $n$ vectors from a Cartan plane to $dl$. Hence,
if the embedding $\gamma(0)$ defines a solution of $\mathcal{E}$, then $0_N^*\big(i_{\partial_{\tau}}\gamma^*(dl)\big) = 0$. We obtain\\[-1ex]


\propositiona{\label{Sol} Let $\sigma\colon N \to \mathcal{E}$ be an embedding that defines a solution to an infinitely prolonged system of equations $\mathcal{E}$. Then $\sigma$ is a stationary point of any internal Lagrangian of $\mathcal{E}$.}\\[-1 ex]


\noindent
Let us note that a result close in meaning to Proposition~\ref{Sol} is given in~\cite{Grigoriev}, where (another) notion of an intrinsic action is introduced (see also~\cite{GriGri}).\\[-1 ex] %An intrinsic action can be regarded as a single differential form that represents an internal Lagrangian.\\


\remarka{Apparently, one can also define stationary points of conservation laws the same way (and then they are determined by the corresponding variational $1$-forms).}\\[-1 ex]


In the examples below, we use more suitable indices for local coordinates on jets and equations.\\[-1 ex]


\examplea{\label{ex1} Consider the infinite prolongation $\mathcal{E}$ of the Laplace equation 
$$
u_{xx} + u_{yy} = 0\,.
$$
Here $\pi\colon \mathbb{R}\times \mathbb{R}^2\to \mathbb{R}^2$ is the projection onto the second factor. Laplace's equation is the Euler-Lagrange equation for
$$
L = -\frac{u_x^2 + u_y^2}{2}\,dx\wedge dy\,.
$$
Formula~\eqref{Noeth} can be used to find an appropriate Cartan form $\omega_L$. Here we have
$$
\mathcal{L}_{E_{\varphi}}(L) = -\big(u_x D_x(\varphi) + u_y D_y(\varphi)\big)dx\wedge dy\,.
$$
Integrating by parts, we get
$$
\mathcal{L}_{E_{\varphi}}[L]_h = \langle \mathrm{E}[L]_h, \varphi \rangle + d_h[- u_x \varphi\, dy + u_y \varphi\, dx]_h\,.
$$
Then we can take $\omega_L = - u_x\,\theta_0\wedge dy + u_y\,\theta_0\wedge dx$, where $\theta_0 = du - u_x\, dx - u_y\, dy$. 
\noindent
The corresponding internal Lagrangian of the system $\mathcal{E}$ is represented by the restriction of the differential form
$$
L + \omega_L = -\frac{u_x^2 + u_y^2}{2}\,dx\wedge dy - u_x\,\theta_0\wedge dy + u_y\,\theta_0\wedge dx\,.
$$

One can regard the coordinate $u_{yy}$ and its derivatives as external coordinates for the corresponding infinite prolongation. Other coordinates on $J^{\infty}(\pi)$ can be treated as local coordinates on $\mathcal{E}$. Then the restrictions of the total derivatives to the system $\mathcal{E}$ have the form
\begin{align*}
&\,\overline{\!D}_x = \partial_x + u_x\partial_u + u_{xx}\partial_{u_x} + u_{xy}\partial_{u_y} + u_{xxx}\partial_{u_{xx}} + u_{xxy}\partial_{u_{xy}} + u_{xxxx}\partial_{u_{xxx}} + \ldots\\
&\,\overline{\!D}_y = \partial_y + u_y\partial_u + u_{xy}\partial_{u_x} - u_{xx}\partial_{u_y} + u_{xxy}\partial_{u_{xx}} - u_{xxx}\partial_{u_{xy}} + u_{xxxy}\partial_{u_{xxx}} + \ldots
\end{align*}

We can take any compact $2$-dimensional submanifold $N\subset\mathbb{R}^2$. For definiteness, we choose the closed unit disk:
$$
N = \{(x, y)\in \mathbb{R}^2\, |\ x^2 + y^2\leqslant 1\}\,.
$$
Suppose $\sigma\in \mathcal{A}_N(\mathcal{E})$ is a (local) section of the bundle $\pi_{\infty}|_{\mathcal{E}}$ such that
\begin{align}
T_p\, \sigma(N)\, \ni\, \,\overline{\!D}_x|_{p}\qquad \text{for all}\quad p\in\sigma(N).
\label{covec1}
\end{align}
Then $\sigma$ is of the form
\begin{align*}
%\sigma\colon \qquad
u = f\,,\quad u_x = \partial_x f\,,\quad u_y = g\,,\quad u_{xx} = \partial_x^2 f\,,\quad u_{xy} = \partial_x g\,,\quad u_{xxx} = \partial^3_x f\,,\quad \ldots
\end{align*}
Arbitrary smooth functions $f, g\colon N \to \mathbb{R}$ determine an appropriate $\sigma$ and vice versa. Consider the path in local sections satisfying~\eqref{covec1}
\begin{align*}
\gamma\colon \qquad
u = f + \tau \delta f\,,\qquad u_x = \partial_x (f + \tau \delta f)\,,\qquad u_y = g + \tau \delta g\,,\qquad \ldots
\end{align*}
Here $\delta f$ and $\delta g$ denote arbitrary smooth functions on $N$ vanishing together with all their derivatives on $\partial N$. Given this, it suffices to consider the pullback
\begin{align*}
\gamma(0)^*(l) = \Big(-\frac{(\partial_x f)^2 - g^2}{2} - g\, \partial_y f\Big)\,dx\wedge dy\,,
\end{align*}
where $l = (L + \omega_L)|_{\mathcal{E}}$.
%\begin{align*}
%\gamma(\tau)^*(l) = \Big(-\frac{\big(\partial_x (f + \tau \delta f)\big)^2 - (g + \tau\delta g)^2}{2} - (g + \tau\delta g)\,\partial_y (f + \tau \delta f)\Big)\,dx\wedge dy
%\end{align*}
The corresponding Euler-Lagrange equations are
\begin{align*}
\partial_x^2 f + \partial_y \kern 0.06em g = 0\,,\qquad g - \partial_y f = 0\,.
\end{align*}

Therefore, an almost Cartan embedding $\sigma\in \mathcal{A}_N(\mathcal{E})$ that is a local section of the bundle $\pi_{\infty}|_{\mathcal{E}}$ and satisfies~\eqref{covec1} is a stationary point of the internal Lagrangian under consideration iff it defines a solution to the system $\mathcal{E}$. 
Furthermore, in order to obtain this result, it suffices to consider only paths that satisfy condition~\eqref{covec1} for all $\tau\in\mathbb{R}$ and pass through local sections.
As we will see below, this is not a coincidence (see Section~\ref{Compl}).}


\section{\label{ILcl} Internal Lagrangians and conservation laws\\[-3ex]}

\

Conservation laws of an infinitely prolonged system of differential equations $\mathcal{E}$ are related to the trivial internal Lagrangian. More precisely, the differential 
$$
d_1\colon E^{\,0,\, n-1}_1(\mathcal{E}) \to E^{1,\, n-1}_1(\mathcal{E})
$$
maps conservation laws of $\mathcal{E}$ to variational $1$-forms. The group of variational $1$-forms $E^{1,\, n-1}_1(\mathcal{E})$ admits the following description
\begin{align*}
E^{1,\, n-1}_1(\mathcal{E}) = \dfrac{\{\omega\in\mathcal{C}\Lambda^n(\mathcal{E})\colon\ d\omega\in \mathcal{C}^2\Lambda^{n+1}(\mathcal{E})\}}
{\mathcal{C}^2\Lambda^{n}(\mathcal{E}) + d(\mathcal{C}\Lambda^{n-1}(\mathcal{E}))}\,.
\end{align*}
Therefore, each variational $1$-form defines an element of group~\eqref{SIL} and the corresponding internal Lagrangian. Any exact variational $1$-form $w\in \mathrm{im}\, d_1$ can be represented by an exact differential form; accordingly, the $\mathrm{im}\, d_1$ is contained in the kernel of the mapping $E^{1,\, n-1}_1(\mathcal{E}) \to \widetilde{E}^{\,0,\, n-1}_1(\mathcal{E})$.

It can be shown that a more general statement is true.\\[-1ex]


\propositiona{\label{ilclpro} A variational $1$-form $w\in E^{1,\, n-1}_1(\mathcal{E})$ produces the trivial internal Lagrangian if and only if $w\in \mathrm{im}\, d_1$.}\\[1ex]
\textbf{Proof.}{
The inclusions
$\Lambda^{*}(\mathcal{E}) \supset \mathcal{C}\Lambda^{*}(\mathcal{E}) \supset \mathcal{C}^2\Lambda^{*}(\mathcal{E})$
allow us to consider three short exact sequences
\begin{align*}
\xymatrix
{
0\ar[r] &\mathcal{C}\Lambda^{*}(\mathcal{E})\ar[r] &\Lambda^{*}(\mathcal{E}) \ar[r]
& \dfrac{\Lambda^{*}(\mathcal{E})}{\mathcal{C}\Lambda^{*}(\mathcal{E})} \ar[r] &0\,,
}\\
\xymatrix
{
0\ar[r] &\mathcal{C}^2\Lambda^{*}(\mathcal{E})\ar[r] &\Lambda^{*}(\mathcal{E}) \ar[r]
& \dfrac{\Lambda^{*}(\mathcal{E})}{\mathcal{C}^2\Lambda^{*}(\mathcal{E})} \ar[r] &0\,,
}\\
\xymatrix
{
0\ar[r] &\mathcal{C}^2\Lambda^{*}(\mathcal{E})\ar[r] &\mathcal{C}\Lambda^{*}(\mathcal{E}) \ar[r]
& \dfrac{\mathcal{C}\Lambda^{*}(\mathcal{E})}{\mathcal{C}^2\Lambda^{*}(\mathcal{E})} \ar[r] &0\,.
}
\end{align*}
Since the de Rham cohomology groups $H^{i}_{dR}(\mathcal{E})$ are trivial for $i > 0$, the corresponding long exact sequences have the form
\begin{align}
\xymatrix
{
\ldots \ar[r] & 0\ar[r] &E^{\,0,\, n-1}_1(\mathcal{E})\ar[r] & H^{n}(\mathcal{C}\Lambda^{*}(\mathcal{E})) \ar[r] &0 \ar[r]&\ldots\,,
}
\label{LES1}\\[1 ex]
\xymatrix
{
\ldots \ar[r] & 0\ar[r] &\widetilde{E}^{\,0,\, n-1}_1(\mathcal{E})\ar[r] & H^{n+1}(\mathcal{C}^2\Lambda^{*}(\mathcal{E})) \ar[r] &0 \ar[r]&\ldots\,,
}
\label{LES2}\\[1 ex]
\xymatrix
{
\ldots \ar[r] & H^{n}(\mathcal{C}\Lambda^{*}(\mathcal{E})) \ar[r] & E^{1,\, n-1}_1(\mathcal{E}) \ar[r] & H^{n+1}(\mathcal{C}^2\Lambda^{*}(\mathcal{E})) \ar[r] &\ldots
\nonumber
}
\end{align}
Here the mappings
$$
E^{\,0,\, n-1}_1(\mathcal{E})\to H^{n}(\mathcal{C}\Lambda^{*}(\mathcal{E}))\,,\qquad \widetilde{E}^{\,0,\, n-1}_1(\mathcal{E})\to H^{n+1}(\mathcal{C}^2\Lambda^{*}(\mathcal{E}))
$$
are induced by the de Rham differential $d$. Exact sequences~\eqref{LES1} and~\eqref{LES2} show that these mappings
are isomorphisms.
Thus, the assertion of the proposition follows from the long exact sequence (with the desired mappings)
\begin{align*}
\xymatrix
{
\ldots \ar[r] & E^{\,0,\, n-1}_1(\mathcal{E}) \ar[r]^{d_1} & E^{1,\, n-1}_1(\mathcal{E}) \ar[r] & \widetilde{E}^{\,0,\, n-1}_1(\mathcal{E}) \ar[r] &\ldots
}
\end{align*}
}\\[-5ex]

Hereby, a variational $1$-form is related to a conservation law iff it produces the trivial internal Lagrangian.



\section{\label{iLs} Internal Lagrangians and symmetries\\[-3ex]}

\

It is quite natural to expect that some version of Noether's theorem concerns internal Lagrangians. Let us recall that internal Lagrangians of a system of differential equations $\mathcal{E}$ are related to presymplectic structures via the mapping
$$
\tilde{d}^{\,0,\,n-1}_1\colon \widetilde{E}_1^{\,0,\,n-1}(\mathcal{E})\to E_1^{2,\,n-1}(\mathcal{E})
$$
induced by the de Rham differential $d$.

Symmetries of differential equations act on internal Lagrangians by means of the Lie derivative. Suppose $\boldsymbol\ell\in\widetilde{E}^{\,0,\,n-1}_1(\mathcal{E})$ is an internal Lagrangian, $l\in \boldsymbol\ell$ is a differential form representing $\boldsymbol\ell$, $X$ is a symmetry. The Lie derivative $\mathcal{L}_X(l)$ can be written as
$$
\mathcal{L}_X(l) = d(i_X l) + i_X dl\,.
$$
This means that the corresponding internal Lagrangian is represented by the differential form $i_X dl$. In other words, the internal Lagrangian $\mathcal{L}_X\, \boldsymbol\ell$ is produced by the variational $1$-form 
$$
i_X dl + \mathcal{C}^2\Lambda^{n}(\mathcal{E}) + d(\mathcal{C}\Lambda^{n-1}(\mathcal{E})).
$$
Then from Proposition~\ref{ilclpro} it follows that an internal Lagrangian $\boldsymbol\ell$ is invariant under the action of a symmetry $X$ iff $X$ produces conservation laws. We get the following version of Noether's theorem\\[-1ex]


\theorema{(Noether's theorem). Let $\boldsymbol\ell\in \widetilde{E}_1^{\,0,\,n-1}(\mathcal{E})$ be an internal Lagrangian, and let $X$ be a symmetry of an infinitely prolonged system of differential equations $\mathcal{E}$. If $\boldsymbol\ell$ is invariant under the action of $X$, then the variational $1$-form $w = i_{X}\, \tilde{d}^{\, 0,\,n-1}_1 \boldsymbol\ell$ is exact, and $X$ gives rise to the conservation laws $d_1^{\,-1}(w)$. Otherwise, $X$ produces the non-trivial internal Lagrangian $\mathcal{L}_X\, \boldsymbol\ell$.}\\[-1ex]


If an infinitely prolonged system of equations $\mathcal{E}\subset J^{\infty}(\pi)$ admits gauge symmetries and its projection to $J^0(\pi)$ is surjective, then \textit{all internal Lagrangians of $\mathcal{E}$ are gauge invariant}, as are all of its presymplectic structures. This follows from the fact that any gauge symmetry produces the trivial variational $1$-form. Indeed, 
assume that $\Delta$ is an operator representing a presymplectic structure, $R$ is an operator that generates gauge symmetries of $\mathcal{E}$. A gauge symmetry $R(\epsilon)$ gives rise the cosymmetry $\psi = \Delta (R\,(\epsilon))$.
According to Green's formula, there exists a $\mathcal{C}$-differential operator $\square\colon P(\mathcal{E})\to \Lambda^{n-1}_h(\mathcal{E})$ such that
$$
\langle \psi, G \rangle = \langle \epsilon, R^*\circ \Delta^*\, (G) \rangle  + d_h\,\square(G)\qquad \text{for all}\quad G\in P(\mathcal{E})\,.
$$
Since $R^*\circ \Delta^*\circ l_{\mathcal{E}} = R^*\circ l^{\,*}_{\mathcal{E}} \circ \Delta = (l_{\mathcal{E}}\circ R)^* \circ \Delta = 0$,
we can use the operator $\square\circ l_{\mathcal{E}}$ as $\nabla$ in~\eqref{cos1f}. However, operators of the form 
$$
A\circ l_{\mathcal{E}}\,,\qquad A\in \mathcal{C}(P(\mathcal{E}), \Lambda^{n-1}_h(\mathcal{E}))
$$
determine the trivial element of $E^{1,\, n-1}_0(\mathcal{E})$.\\[-1ex]


\remarka{One can also apply the $k$-line theorem (see, e.g.,~\cite{KraVer}) to show that any gauge symmetry produces the trivial variational $1$-form.}\\


\section{\label{Compl} Internal Lagrangians and non-degenerate forms\\[-3 ex]}

\

Let $\pi\colon E\to M$ be a locally trivial smooth vector bundle over a smooth manifold $M$, $\dim M = n$, $\dim E = n + m$. If $\mathcal{E}\subset J^{\infty}(\pi)$ is an infinitely prolonged system of differential equations, then it can be endowed with the bundle structure
$$
\pi_{\mathcal{E}}\colon \mathcal{E}\to M,\qquad \pi_{\mathcal{E}} = \pi_{\infty}|_{\mathcal{E}}.
$$
Suppose $N\subset M$ is a compact oriented $n$-dimensional submanifold.\\[-1ex]


\definitiona{Let $\sigma\colon N\to \mathcal{E}$ be an almost Cartan embedding. We say that $\sigma$ is an \textit{almost Cartan section} of $\pi_{\mathcal{E}}$ if $\pi_{\mathcal{E}}\circ \sigma = \mathrm{id}_N$.}\\[-1ex]


\noindent
%We denote the set of all almost Cartan sections of $\pi_{\mathcal{E}}$ by $\mathcal{S}(\pi_{\mathcal{E}})$.
Almost Cartan sections are also related to initial-boundary conditions. It is reasonable to anticipate that characteristic planes somehow manifest themselves with variations of internal Lagrangians.\\[-1 ex]


\definitiona{Let $k\geqslant 1$ be an integer. A differential $n$-form $L\in \Lambda^n(J^{k}(\pi))$ is \textit{non-degenerate} if\\
1) at every point of the system of equations $\{\mathrm{E}[L]_h = 0\}\subset J^{2k}(\pi)$, there exists a non-zero, non-characteristic covector;\\
2) the projection of the system $\{\mathrm{E}[L]_h = 0\}\subset J^{2k}(\pi)$ to $J^{2k - 1}(\pi)$ is surjective.}\\[-1ex]


We shall say that the Euler-Lagrange equations $\mathrm{E}[L]_h = 0$ for a non-degenerate differential form $L$ are also non-degenerate. If $L\in \Lambda^n(J^{k}(\pi))$ is of the form $\lambda\, dx^1\wedge\ldots \wedge dx^n$, then the non-degeneracy condition is equivalent to the following one. For each $p\in \{\mathrm{E}[L]_h = 0\}$, there exists a covector $\xi = \xi_i\, dx^i\in T^{\,*}_{\pi_{2k}(p)} M$ such that the matrix
\begin{align*}
A_{ij} = \sum_{|\alpha| = |\beta| = k}\, \dfrac{\partial^2 \lambda}{\partial u^i_{\alpha} \partial u^j_{\beta}}\,\xi^{\alpha + \beta}
\end{align*}
has a non-zero determinant at $p$.
Here $\beta$ denotes a multi-index $\beta_1 x^1 + \ldots + \beta_n x^n$,
$$
\xi^{\alpha + \beta} = (\xi_1)^{\alpha_1 + \beta_1}\cdot\ldots\cdot (\xi_n)^{\alpha_n + \beta_n}\,.
$$


\definitiona{Let $\xi\in\Lambda^1(M)$ be a covector field. An almost Cartan section $\sigma\colon N\to \mathcal{E}$ is a $\xi$-\textit{section} of $\pi_{\mathcal{E}}$ if
\begin{align*}
d\sigma(\ker \xi|_x) \subset \mathcal{C}_{\sigma(x)}\qquad \text{for all} \quad x\in N.
\end{align*}
}


\definitiona{Let $\boldsymbol\ell$ be an internal Lagrangian of an infinitely prolonged system of equations $\mathcal{E}$, $l\in \boldsymbol\ell$ be a differential form representing $\boldsymbol\ell$. We say that a $\xi$-section $\sigma$ is a $\xi$-\textit{stationary point} of $\boldsymbol\ell$ if
\begin{align*}
\dfrac{d}{d\tau}\Big|_{\tau = 0}\int_N \gamma(\tau)^*(l) = 0
\end{align*}
for any path $\gamma$ in $\xi$-sections of $\pi_{\mathcal{E}}$ such that $\gamma(0) = \sigma$ and the boundary is fixed.
}\\[-1ex]


Sure, if a $\xi$-section is a stationary point of an internal Lagrangian, it is also a $\xi$-stationary one. We present a theorem establishing a connection between non-degenerate forms and the corresponding internal Lagrangians in terms of $\xi$-stationary points. Its proof is rather cumbersome. For convenience, we have divided it into seven parts.\\


\theorema{\label{Nonchar} Let $L\in \Lambda^n(J^{k}(\pi))$ be a non-degenerate form, and let $\mathcal{E}$ be the infinitely prolonged system of the Euler-Lagrange equations $\mathrm{E}[L]_h = 0$. Suppose $\xi\in\Lambda^1(M)$ is a non-vanishing covector field such that the distribution $\xi = 0$ is integrable, and for each $p\in \{\mathrm{E}[L]_h = 0\}$, the covector $\xi|_{\pi_{2k}(p)}$ is non-characteristic. Then a $\xi$-section $\sigma$ is a $\xi$-stationary point of the corresponding internal Lagrangian if and only if $\sigma$ is a (local) solution to $\pi_{\mathcal{E}}$.}\\[-2ex]


\noindent
\textbf{Proof. }{It suffices to prove the statement locally.\\[0.5ex]
\textit{\textbf{1.}} Let us fix an interior point $x_0\in M$. Since the distribution $\xi = 0$ is integrable, we can choose local coordinates $x^1$, \ldots, $x^n$ in a neighborhood $U$ of $x_0$ such that the distributions $\xi = 0$ and  
$dx^n = 0$ coincide. We shall use the notation $x^n = y$ and the \textit{special multi-index $\nu = \nu_1 x^1 + \ldots + \nu_{n-1} x^{n-1}$}.

Without loss of generality, we can assume that $L$ is of the form
$$
L = \lambda\, dx^1\wedge\ldots \wedge dx^n.
$$
%Here $\lambda$ is a smooth function of independent variables $x^1$, \ldots, $x^n$, dependent variables $u^1$, \ldots, $u^m$ and their derivatives up to order $k$. 
The covector $dy$ is non-characteristic at every point of the system $\{\mathrm{E}[L]_h = 0\}$ over $U$. This implies that the determinant of the matrix
$$
A_{ij} = \dfrac{\partial^2 \lambda}{\partial u^i_{ky} \partial u^j_{ky}}
$$
never vanishes in $\pi_{2k}^{-1}(U)$. Then, in $\pi_{2k}^{-1}(U)$, the quasilinear system $\mathrm{E}[L]_h = 0$ is equivalent to a system of the form
\begin{align}
u^i_{2ky} = h^i.
\label{Simpleform}
\end{align}
The functions $h^i$ do not depend on $u^1_{2ky}$, \ldots, $u^m_{2ky}$. Hence, the coordinates $u^1_{2ky}$, \ldots, $u^m_{2ky}$ and their derivatives can be regarded as external coordinates. All other coordinates on the infinite jets can be treated as coordinates in $\pi_{\infty}^{-1}(U)\cap\mathcal{E}$.
Denote by $\,\overline{\!D}_{x^1}$, $\ldots$\,, $\,\overline{\!D}_{x^n}$ the corresponding total derivatives on $\pi_{\infty}^{-1}(U)\cap\mathcal{E}$.
From now on, we use only these local coordinates.\\[0.5ex]
\textit{\textbf{2.}} A $\xi$-section over $U$ defines smooth functions
\begin{align*}
u^i = f^i_0,\qquad u^i_{y} = f^i_1,\qquad u^i_{2y} = f^i_2,\qquad \ldots, \qquad u^i_{(2k-1)y} = f^i_{2k-1}
\end{align*}
and vice versa. Any $\xi$-section over $U$ has the form
\begin{align}
u^i_{\nu + jy} = \partial_{\nu} f^i_j\qquad\quad j = 0, \ldots, 2k-1\,.
\label{newjets}
\end{align}

More formally, one can interpret the symbols $f^i_j$ (for $j = 0, \ldots, 2k-1$) as notation for coordinates along the fibers of the projection 
$$
\mathrm{pr}_U \colon \mathbb{R}^{2km}\times U\to U.
$$ 
In addition, one can interpret the symbols $\partial_{\alpha} f^i_j$ (again for $j = 0, \ldots, 2k-1$) as notation for adapted coordinates on $J^{\infty}(\mathrm{pr}_U)$.
Then the relations~\eqref{newjets} determine a unique fiber-preserving mapping $\Phi\colon J^{\infty}(\zeta)\to \mathcal{E}$. We denote by $d/dx^1$, $\ldots$\,, $d/dx^n$ the corresponding total derivatives on~$J^{\infty}(\mathrm{pr}_U)$.

So, it suffices to derive the Euler-Lagrange equations for the horizontal form $[\Phi^*(l)]_h$, where $l$ is a differential form representing the corresponding internal Lagrangian.\\[0.5ex]
\textit{\textbf{3.}} We seek a Cartan $n$-form $\omega_L\in\mathcal{C}\Lambda^n(\pi)$ satisfying identity~\eqref{Noeth}. An appropriate form can be found using integration by parts. The component of the Lie derivative $\mathcal{L}_{E_{\varphi}}(L)$ reads
\begin{align*}
\dfrac{\partial \lambda}{\partial u^i_{\alpha}}\, D_{\alpha}(\varphi^i) = &\ D_{x^1}(\ldots) + \ldots + D_{x^{n-1}}(\ldots) + (-1)^{|\nu|}D_{\nu}
\Big(\dfrac{\partial \lambda}{\partial u^i_{\nu + jy}}\Big)D_{jy}(\varphi^i) + \textit{linear in }\varphi.
\end{align*}
Here we have applied integration by parts with respect to the variables $x^1$, \ldots, $x^{n-1}$. All the summands linear in $\varphi$ will end up in the expression for the variational derivative, so they are of no interest to us. Besides, $d\Phi(d/dx^i) = \,\overline{\!D}_{x^i}$ for $i = 1, \ldots, n-1$. Therefore, if a Cartan $n$-form $\chi\in\mathcal{C}\Lambda^n(\mathcal{E})$ is proportional to $dy$, then $[\Phi^*(\chi)]_h = 0$.
That explains why we are interested only in summands of the form~$D_y(\ldots)$. 

One can apply the simple formula for derivatives
\begin{align*}
vw^{(j)} = \Big(\sum_{q = 0}^{j-1} (-1)^{j-1-q}\, v^{(j-1-q)}w^{(q)}\Big)' + \ \textit{linear in}\ w
\end{align*}
to complete the integration by parts.\\[0.5ex]
\textit{\textbf{4.}} Eventually, we obtain:
\begin{align*}
\begin{aligned}
\dfrac{\partial \lambda}{\partial u^i_{\alpha}}\, D_{\alpha}(\varphi^i) =&\ (-1)^{|\alpha|}D_{\alpha}\Big(\dfrac{\partial \lambda}{\partial u^i_{\alpha}}\Big)\, \varphi^i + D_y(B) +{}\\
&+ D_{x^1}(\ldots) + \ldots + D_{x^{n-1}}(\ldots)\,,
\end{aligned}
\qquad\qquad
B = (-1)^{|\alpha|}D_{\alpha}\Big(\dfrac{\partial \lambda}{\partial u^i_{\alpha + (r+1)y}}\Big)D_{ry}(\varphi^i).
\end{align*}
So, the corresponding internal Lagrangian is represented by the form $l = (L + \omega_L)|_{\mathcal{E}}$, where
\begin{align*}
\omega_L =&\ dx^1\wedge\ldots\wedge dx^{n-1}\wedge (-1)^{|\alpha|}D_{\alpha}\Big(\dfrac{\partial \lambda}{\partial u^i_{\alpha + (r+1)y}}\Big)\,\theta^i_{ry} +\ \textit{proportional to}\ dy.
\end{align*}
It is easy to see that the maximum value of the summation index $r$ in the last formula is equal to $k-1$ (while the minimum value is $0$). Then the differential form
\begin{align}
\Big(\Phi^*(\lambda) + (-1)^{|\alpha|}\widetilde{D}_{\alpha}\Big(\dfrac{\partial \lambda}{\partial u^i_{\alpha + (r+1)y}}\Big)(\partial_y f^i_r - f^i_{r+1})\Big)dx^1\wedge\ldots\wedge dx^n\,,\qquad \widetilde{D}_{\alpha} = \Phi^*\circ D_{\alpha}\,.
\label{pullback}
\end{align}
represents the horizontal form $[\Phi^*(l)]_h$.
Now it only remains to derive the Euler-Lagrange equations for the $[\Phi^*(l)]_h$.\\[0.5ex]
\textit{\textbf{5.}} Below we deal with the component of differential form~\eqref{pullback} and the action it determines.

Consider the case $k > 2$. The variables $f^j_{2k-1}$ appear only in the terms corresponding to $\alpha = (k-1)y$, $r = 0$ (their derivatives do not appear in~\eqref{pullback} at all). The variational derivatives $\delta/\delta f^j_{2k-1}$ yield the equations
\begin{align*}
(-1)^{k-1}\,\Phi^*\Big(\dfrac{\partial^2 \lambda}{\partial u^j_{ky} \partial u^i_{ky}}\Big)
(\partial_y f^i_0 - f^i_{1}) = 0.
\end{align*}
The non-degeneracy condition implies $f^i_{1} = \partial_y f^i_0$.

Let us recall that the formula for the variational derivative with respect to a variable is a sum of compositions of derivatives (partial and total ones). From now on, in the terms corresponding to $r = 0$, we apply partial derivatives only to the factors $\partial_y f^i_0 - f^i_{1}$; otherwise we would get trivial summands due to the equations $f^i_1 = \partial_y f^i_0$ and their differential consequences.

Among the terms corresponding to $r > 0$, the variables $f^j_{2k-2}$ appear only in the summands corresponding to $\alpha = (k-2)y$, $r = 1$ (their derivatives do not appear whenever $r > 0$). Thus, varying with respect to $f^j_{2k-2}$, we get
\begin{align*}
(-1)^{k-2}\,\Phi^*\Big(\dfrac{\partial^2 \lambda}{\partial u^j_{ky} \partial u^i_{ky}}\Big)
(\partial_y f^i_1 - f^i_{2}) = 0.
\end{align*}
The non-degeneracy condition implies $f^i_{2} = \partial_y f^i_1$. From now on, in the terms corresponding to $r = 1$, we apply partial derivatives only to the factors $\partial_y f^i_1 - f^i_{2}$.

Following the same line of reasoning for $\delta/\delta f^j_{2k-3}$, $\ldots\,$, $\delta/\delta f^j_{k+1}$, we obtain the equations
\begin{align*}
f^i_{1} = \partial_y f^i_0,\qquad f^i_{2} = \partial_y f^i_1, \qquad\ldots,\qquad f^i_{k-1} = \partial_y f^i_{k-2}.
\end{align*}
\textit{\textbf{6.}} Let us discuss the variational derivative $\delta/\delta f^j_{k}$ (still for~\eqref{pullback}). It suffices to consider the term $\Phi^*(\lambda)$ and the terms corresponding to $r = k-1$, $\alpha = 0$. Then we find
\begin{align*}
\Phi^*\Big(\dfrac{\partial^2 \lambda}{\partial u^j_{ky} \partial u^i_{ky}}\Big)
(\partial_y f^i_{k-1} - f^i_{k}) = 0\,,
\end{align*}
the other terms cancelling. These equations imply that $f^i_{k} = \partial_y f^i_{k-1}$. From now on, in the sum
\begin{align*}
(-1)^{|\alpha|}\widetilde{D}_{\alpha}\Big(\dfrac{\partial \lambda}{\partial u^i_{\alpha + (r+1)y}}\Big)(\partial_y f^i_r - f^i_{r+1})\,,
\end{align*}
we apply partial derivatives only to the factors of the form $\partial_y f^i_{r} - f^i_{r+1}$.

We can consider only the terms corresponding to $r = k-1,\, k-2$ and the $\Phi^*(\lambda)$ when varying with respect to $f^j_{k-1}$.
Note that $\Phi^*(\lambda)$ does not depend on coordinates of the form $\partial_{\alpha + y} f^j_{r}$.
Then we~get
\begin{align*}
0 &= (-1)^{|\nu|}\widetilde{D}_{\nu}\Big(\dfrac{\partial \lambda}{\partial u^j_{\nu + (k-1)y}}\Big) - (-1)^{|\alpha|}\widetilde{D}_{\alpha}\Big(\dfrac{\partial \lambda}{\partial u^j_{\alpha + (k-1)y}}\Big) - (-1)^{|\alpha|} \dfrac{d}{dy}\, \widetilde{D}_{\alpha}\Big(\dfrac{\partial \lambda}{\partial u^j_{\alpha + ky}}\Big) ={}\\
&= (-1)^{|\beta|}\widetilde{D}_{\beta + y}\Big(\dfrac{\partial \lambda}{\partial u^j_{\beta + ky}}\Big) - (-1)^{|\alpha|} \dfrac{d}{dy}\, \widetilde{D}_{\alpha}\Big(\dfrac{\partial \lambda}{\partial u^j_{\alpha + ky}}\Big)
= \widetilde{D}_{y}\Big(\dfrac{\partial \lambda}{\partial u^j_{ky}}\Big) - \dfrac{d}{dy}\, \widetilde{D}_{0}\Big(\dfrac{\partial \lambda}{\partial u^j_{ky}}\Big).
\end{align*}
However, we have already deduced the equations $\partial_{\alpha} f^i_r = \partial_{\alpha + ry}\, f^i_{0}$ for $r\leqslant k$. For this reason, the only non-trivial summands here are
\begin{align*}
\Phi^*\Big(\dfrac{\partial^2 \lambda}{\partial u^i_{ky} \partial u^j_{ky}}\Big)
(f^i_{k+1} - \partial_y f^i_k).
\end{align*}
Hence, the equations $\partial_{\alpha} f^i_r = \partial_{\alpha + ry}\, f^i_{0}$ hold for all $r\leqslant k + 1$.

Again, we can consider only the terms corresponding to $r = k-2,\, k-3$ and the $\Phi^*(\lambda)$ when varying with respect to $f^j_{k-2}$. Similarly we obtain
\begin{align*}
0 &= (-1)^{|\nu|}\widetilde{D}_{\nu}\Big(\dfrac{\partial \lambda}{\partial u^j_{\nu + (k-2)y}}\Big) - (-1)^{|\alpha|}\widetilde{D}_{\alpha}\Big(\dfrac{\partial \lambda}{\partial u^j_{\alpha + (k-2)y}}\Big) - (-1)^{|\alpha|}\dfrac{d}{dy}\,\widetilde{D}_{\alpha}\Big(\dfrac{\partial \lambda}{\partial u^j_{\alpha + (k-1)y}}\Big) =\\
&= (-1)^{|\beta|}\widetilde{D}_{\beta + y}\Big(\dfrac{\partial \lambda}{\partial u^j_{\beta + (k-1)y}}\Big) - (-1)^{|\alpha|}\dfrac{d}{dy}\,\widetilde{D}_{\alpha}\Big(\dfrac{\partial \lambda}{\partial u^j_{\alpha + (k-1)y}}\Big).
\end{align*}
In this formula, one can express in $\partial_{\beta} f^i_0$ everything except $\Phi^*(u^i_{(k+2)y})$ using the equations
\begin{align}
\partial_{\alpha} f^i_r = \partial_{\alpha + ry}\, f^i_{0}\qquad\quad  r\leqslant k + 1.
\label{relat}
\end{align} 
Upon substituting, we get
\begin{align*}
- \Phi^*\Big(\dfrac{\partial^2 \lambda}{\partial u^i_{ky} \partial u^j_{ky}}\Big)\bigg|_{\eqref{relat}}\!
\cdot (f^i_{k+2} - \partial_{(k+2)y}\, f^i_{0}) = 0.
\end{align*}
So, the equations $\partial_{\alpha} f^i_r = \partial_{\alpha + ry}\, f^i_{0}$ hold for all $r\leqslant k + 2$.

Similar reasoning gives the equations $\partial_{\alpha} f^i_{k+3} = \partial_{\alpha + (k+3)y}\, f^i_{0}$\,, $\ldots$\, Finally, we obtain
\begin{align}
\partial_{\alpha} f^i_r = \partial_{\alpha + ry}\, f^i_{0}\qquad \text{for all} \quad r\leqslant 2k-1.
\label{complrelat}
\end{align}
\textit{\textbf{7.}} Now we know that all $\xi$-stationary points of the internal Lagrangian 
$$
\boldsymbol\ell = l + \mathcal{C}^2\Lambda^{n}(\mathcal{E}) + d(\Lambda^{n-1}(\mathcal{E}))
$$
satisfy equations~\eqref{complrelat}. The cases $k = 1$ and $k = 2$ do not differ from those considered above. In these cases, one can derive the same result by direct calculation.

Ultimately, let us discuss the variational derivative $\delta/\delta f^j_0$. Substituting $\partial_{\alpha + ry}\, f^i_{0}$ for $\partial_{\alpha} f^i_r$ in~\eqref{pullback}, we obtain the original Lagrangian written in terms of $f^i_0$. That means: when we consider variations of $\xi$-sections satisfying equations~\eqref{complrelat} within the class of $\xi$-sections that also satisfy~\eqref{complrelat}, they
lead to the original Euler-Lagrange equations $\mathrm{E}[L]_h = 0$ expressed in $f^i_0$. Thus, any $\xi$-stationary point of the internal Lagrangian 
$\boldsymbol\ell$
must be a (local) solution of $\pi_{\mathcal{E}}$. 
This result and Proposition~\ref{Sol} complete the proof.}\\

Informally speaking, the intrinsic geometry of a non-degenerate variational system knows a lot about its variational nature. Unfortunately, all non-degenerate Euler-Lagrange equations are $l$-normal~\cite{VinKr}. Indeed, at least locally, such equations are equivalent to systems of the form~\eqref{Simpleform}. This implies that such equations do not admit gauge symmetries. Prominent examples of degenerate non-gauge Euler-Lagrange equations are the Proca theory and the massive spin-$2$ theory~\cite{FiePau}.




\section{\label{Exam} Examples\\[-3ex]}

\

Example~\ref{ex1} from Section~\ref{VariL} illustrates Theorem~\ref{Nonchar}. We now consider several examples where the conditions of Theorem~\ref{Nonchar} are not satisfied.\\

\examplea{Let us see how characteristics of differential equations can interfere with ``completeness'' of internal Lagrangians.
Consider the wave equation
\begin{align*}
u_{xy} = 0
\end{align*}
and its infinite prolongation $\mathcal{E}$. Here $\pi\colon \mathbb{R}\times \mathbb{R}^2\to \mathbb{R}^2$ is the projection onto the second factor. 

This equation is the Euler-Lagrange equation for the non-degenerate $2$-form
\begin{align*}
L = -\dfrac{u_x u_y}{2}\, dx\wedge dy\,.
\end{align*} 
The variables $x$, $y$, $u$, $u_x$, $u_y$ $u_{xx}$, $u_{yy},\, \ldots$ can be regarded as local coordinates on $\mathcal{E}$. The restrictions of the total derivatives to the system $\mathcal{E}$ have the form
\begin{align*}
&\,\overline{\!D}_x = \partial_x + u_x\partial_u + u_{xx}\partial_{u_x} + u_{xxx}\partial_{u_{xx}} + \ldots\\
&\,\overline{\!D}_y = \partial_y + u_y\partial_u + u_{yy}\partial_{u_y} + u_{yyy}\partial_{u_{yy}} + \ldots
\end{align*}

The covector field $dy$ is characteristic for the wave equation. The total derivative $\,\overline{\!D}_x$ is tangent to the image of any $dy$-section of $\pi_{\mathcal{E}}$. Suppose $N\subset \mathbb{R}^2$ is a connected compact $2$-dimensional submanifold. A $dy$-section $\sigma\colon N\to\mathcal{E}$ has the form
\begin{align*}
u = f,\ \ u_x = \partial_x f,\ \ u_y = h_1, \ \ u_{xx} = \partial_x^{\,2} f, \ \ u_{yy} = h_2, \ \ u_{xxx} = \partial_x^{\,3} f, \ \ u_{yyy} = h_3,\ \ \ldots,
\end{align*}
Here $f$, $h_1$, $h_2,\, \ldots$ are functions on $N$. The function $f$ can be chosen arbitrarily, while the functions $h_1$, $h_2,\, \ldots$ do not depend  on the variable $x$.

The corresponding internal Lagrangian is represented by the form
$$
l = -\frac{u_x u_y}{2}\,dx\wedge dy - \dfrac{u_y}{2}\,\theta_0\wedge dy - \dfrac{u_x}{2}\,dx\wedge \theta_0\,.
$$
Let $\gamma\colon \mathbb{R}\times N\to \mathcal{E}$ be a path in $dy$-sections. Due to the fixed boundary requirement, $\gamma$ has the~form
\begin{align*}
u = f + \tau \delta f,\ \ u_x = \partial_x (f + \tau \delta f),\ \ u_y = h_1, \ \ u_{xx} = \partial_x^{\,2} (f + \tau \delta f), \ \ u_{yy} = h_2,\ \ \ldots
\end{align*}
Then the pullback $\gamma(\tau)^*(l)$ reads
\begin{align*}
& - \dfrac{\partial_x (f + \tau\delta f)\, \partial_y (f + \tau\delta f)}{2}\, dx\wedge dy\,,
\end{align*}
and we obtain only one equation: $\partial_y\,\partial_x f = 0$. 

We didn't get any equations for the functions $h_i$. This means that the set of all $dy$-stationary points of the internal Lagrangian contains more than just local solutions of $\pi_{\mathcal{E}}$.}\\

\examplea{There are examples of other manifestations of characteristics.
Suppose $\pi\colon \mathbb{R}^2\times \mathbb{R}^2 \to \mathbb{R}^2$ is the projection onto the second factor. Consider the non-degenerate differential $2$-form 
\begin{align}
L = \Big(\dfrac{u_t v - u v_t}{2} - \dfrac{u_x^2 + v_x^2}{2} - V(x)\,\dfrac{u^2 + v^2}{2}\Big)dt\wedge dx\,,
\label{Schro}
\end{align}
where $V(x)$ is an arbitrary function. This differential form leads to the $1$-dimensional Schr\"{o}dinger equation for $\hbar = 1$, $m = 1/2$, $\Psi = u + iv$\,:
\begin{align*}
&- v_t + u_{xx} - V(x)u = 0\,,\\
&u_t + v_{xx} - V(x)v = 0\,.
\end{align*}

The Schr\"{o}dinger equation has a unique characteristic distribution determined by the covector field $dt$. Suppose $N\subset \mathbb{R}^2$ is a connected compact $2$-dimensional submanifold. Choosing $t$, $x$, $u$, $v$ and the derivatives with respect to $x$ as local coordinates on the infinite prolongation $\mathcal{E}$, we see that any $dt$-section $\sigma\colon N\to \mathcal{E}$ has the form
\begin{align*}
u = f,\qquad v = g, \qquad u_x = \partial_x f,\qquad v_x = \partial_x \kern 0.05em g,\qquad \ldots
\end{align*}
The functions $f$ and $g$ are arbitrary functions on $N$.

The desired internal Lagrangian is represented by the form
\begin{align*}
l = - \dfrac{u_x^2 + uu_{xx} + v_x^2 + vv_{xx}}{2}\, dt\wedge dx + \dfrac{1}{2}(v\,\theta^u_0 - u\,\theta^v_0)\wedge dx - dt\wedge(u_x\,\theta^u_0 + v_x\,\theta^v_0)\,,
\end{align*}
where $\theta^u_0 = du - u_x\, dx + (v_{xx} - V(x)v)dt$ and $\theta^v_0 = dv - v_x\, dx - (u_{xx} - V(x)u)dt$.

Assume that $\gamma\colon \mathbb{R}\times N\to \mathcal{E}$ is a path in $dt$-sections such that the boundary is fixed. Then $\gamma$ is of the~form
\begin{align*}
u = f + \tau \delta f,\quad v = g + \tau \delta g,\quad u_x = \partial_x (f + \tau \delta f),\quad v_x = \partial_x(g + \tau \delta g),\quad  \ldots
\end{align*}
Here $\delta f$ and $\delta g$ are arbitrary functions on $N$ vanishing together with all their derivatives on $\partial N$.
We~find
\begin{align*}
\gamma(0)^*(l) = \Big(\dfrac{\partial_t f g - f \partial_t g}{2} - \dfrac{(\partial_x f)^2 + (\partial_x g)^2}{2} - V(x)\,\dfrac{f^2 + g^2}{2}\Big) dt\wedge dx.
\end{align*}
Comparing this pullback with~\eqref{Schro}, we can conclude that all $dt$-stationary points of the internal Lagrangian are local solutions to $\pi_{\mathcal{E}}$.
}\\

\examplea{Let us go back to Example~\ref{ex1}. The Laplace equation admits the conservation law represented by the differential form
\begin{align*}
- u_y\, dx + u_x\, dy\,.
\end{align*}
This conservation law determines the differential covering~\cite{VinKr}
\begin{align*}
u_{xx} + u_{yy} = 0\,,\qquad v_x = -u_y\,,\qquad v_y = u_x\,.
\end{align*}
So, we can lift the internal Lagrangian form Example~\ref{ex1} to the Cauchy-Riemann equations
$$
u_y = -v_x\,,\qquad v_y = u_x\,.
$$ 

The variables $x$, $y$, $u$, $v$ and the derivatives with respect to $x$ can be treated as
local coordinates on the infinite prolongation $\mathcal{S}$ of the Cauchy-Riemann equations. The lift of the internal Lagrangian form Example~\ref{ex1} reads
\begin{align*}
\tilde{l} = -\frac{u_x^2 + v_x^2}{2}\,dx\wedge dy - u_x\,\theta^u_0\wedge dy - v_x\,\theta^u_0\wedge dx.
\end{align*}
Here $\theta_0^u = du - u_x\, dx + v_x\, dy$.
The restrictions of the total derivatives to $\mathcal{S}$ are
\begin{align*}
&\,\overline{\!D}_x = \partial_x + u_x\partial_u + v_x\partial_v + u_{xx}\partial_{u_x} + v_{xx}\partial_{v_x} + \ldots\\
&\,\overline{\!D}_y = \partial_y - v_x\partial_u + u_x\partial_v - v_{xx}\partial_{u_x} + u_{xx}\partial_{v_x} + \ldots
\end{align*}

Let $N\subset \mathbb{R}^2$ be a connected compact $2$-dimensional submanifold, and let $\sigma\colon N\to \mathcal{S}$ be an almost Cartan section of the bundle $\pi_{\mathcal{S}}$. Suppose $w\in D(N)$ is a vector field on $N$ such that
\begin{align*}
d\sigma(w|_p) \in \mathcal{C}_{\sigma(p)}\qquad \text{for all}\quad p \in N\,.
\end{align*}
Then for $p = (x, y)\in N$, we have
\begin{align*}
d\sigma(w|_{p}) = X(x, y)\,\overline{\!D}_x|_{\sigma(p)} + Y(x, y)\,\overline{\!D}_y|_{\sigma(p)}.
\end{align*}
The $X$ and $Y$ are some functions on $N$. Assume that $X\neq 0$ in $N$. We can set $X = 1$. In this case $\sigma$ has the form
\begin{align*}
\sigma\colon\qquad\quad
\begin{aligned}
&u = f, \qquad u_x = \dfrac{\partial_x f + Y\partial_y f + Y\partial_x g + Y^2 \partial_y g}{1 + Y^2},\\
&v = g, \qquad v_x = -\dfrac{Y(\partial_x f + Y \partial_y f) - \partial_x g - Y\partial_y g}{1 + Y^2},\qquad
\end{aligned}
\ldots
\end{align*}
Here $f$ and $g$ are arbitrary functions on $N$. The expressions for all other coordinates on $\mathcal{S}$ are unambiguously defined. The pullback reads
\begin{align*}
&\sigma^*(\tilde{l}) = - \dfrac{(\partial_x f + Y\partial_y f)^2 - (\partial_x g + Y\partial_y g)^2 + 2(\partial_x g + Y\partial_y g)(Y\partial_x f - \partial_y f)}{2(1 + Y^2)}\,dx\wedge dy\,.
\end{align*}
Varying the corresponding action with respect to $f$, $g$ and $Y$, we get
\begin{align*}
&\partial_x \Big(\dfrac{\partial_x f + Y\partial_y f + Y\partial_x g + Y^2 \partial_y g}{1 + Y^2}\Big) + \partial_y \Big(\dfrac{Y(\partial_x f + Y \partial_y f) - \partial_x g - Y\partial_y g}{1 + Y^2}\Big) = 0\,,\\
&\partial_x \Big(\dfrac{Y\partial_x f - \partial_y f - \partial_x g  - Y\partial_y g}{1 + Y^2}\Big) + \partial_y \Big(\dfrac{Y(Y\partial_x f - \partial_y f - \partial_x g  - Y\partial_y g)}{1 + Y^2}\Big) = 0\,,\\
&(Y\partial_x f - \partial_y f - \partial_x g - Y\partial_y g)(\partial_x f + Y\partial_y f + Y\partial_x g - \partial_y g) = 0\,.
\end{align*}
Two cases arise due to the last equation.\\
\textbf{Case 1.}\quad $Y\partial_x f - \partial_y f - \partial_x g - Y\partial_y g = 0$.\quad In this case we have 
$$
u = f,\qquad v = g,\qquad u_x = \partial_x f,\qquad v_x = -\partial_y f,\qquad \partial_x^2 f + \partial_y^2 f = 0\,.
$$
\textbf{Case 2.}\quad $\partial_x f + Y\partial_y f + Y\partial_x g - \partial_y g = 0$.\quad In this case we have 
$$
u = f,\qquad v = g,\qquad u_x = \partial_y g,\qquad v_x = \partial_x g,\qquad -\partial_x^{\,2} g - \partial_y^{\,2} g = 0\,.
$$

Thus, even variation within the class of almost Cartan sections does not allow us to exclude any sections from the case $1$, for which $f = 0$, as well as many other local sections. Moreover, one can show that the variation within the class of all almost Cartan embeddings also does not allow excluding them. This means that the set of all stationary points of the internal Lagrangian contains more than just local solutions of $\pi_{\mathcal{S}}$.
}\\

\examplea{\label{potKdV} This example is the last one. It motivates the concept of $\xi$-stationary points.
Consider the potential KdV equation
\begin{align*}
u_t = 3u_x^2 + u_{xxx}
\end{align*}
and its infinite prolongation $\mathcal{E}$. Here $\pi\colon \mathbb{R}\times \mathbb{R}^2\to\mathbb{R}^2$ is the projection onto the second factor. Denote by $\,\overline{\!D}_t$ and $\,\overline{\!D}_x$ the restrictions of the total derivatives to $\mathcal{E}$. 

It is well known that this equation admits the symmetry with the characteristic $1 \in \ker l_{\mathcal{E}}$ and the presymplectic operator $\Delta = \,\overline{\!D}_x$ (see, e.g.,~\cite{Dorf}). As one can see, this characteristic belongs to the kernel of $\Delta$. The corresponding presymplectic structure is degenerate in this sense and can be related only to a variational principle that gives a consequence of the original equation, but not the potential KdV itself. This example is quite surprising because of that.

It is convenient to regard the variable $u_{xxx}$ and all its derivatives as external coordinates. Other coordinates on the jets can be treated as local coordinates on $\mathcal{E}$. There exists a unique internal Lagrangian $\boldsymbol\ell$ such that $\tilde{d}_1^{\,0,\,n-1} \boldsymbol\ell = \omega$. This internal Lagrangian is represented by the form
\begin{align*}
l = \Big(\dfrac{u_x u_t}{2} - u_x^3 + \dfrac{u_{xx}^2}{2}\Big)dt\wedge dx - \dfrac{1}{2}u_t\,dt\wedge \theta_0 + u_{xx}\,dt\wedge \theta_x + \dfrac{1}{2}u_x\, \theta_0\wedge dx\,,
\end{align*}
where $\theta_0 = du - u_x\, dx - u_t\, dt$ and $\theta_x = du_x - u_{xx}\, dx - u_{xt}\, dt$.

Let $N\subset \mathbb{R}^2$ be a connected compact $2$-dimensional submanifold, and let $\sigma\colon N\to \mathcal{E}$ be an almost Cartan section of the bundle $\pi_{\mathcal{E}}$. Suppose $w\in D(N)$ is a vector field on $N$ such that
\begin{align*}
d\sigma(w|_p)\subset \mathcal{C}_{\sigma(p)}\qquad \text{for all}\quad p\in N\,.
\end{align*}
Then for $p = (t, x)\in N$ and some functions $T$, $X$ on $N$, we have
\begin{align*}
d\sigma(w|_{p}) = T(t, x)\,\overline{\!D}_t|_{\sigma(p)} + X(t, x)\,\overline{\!D}_x|_{\sigma(p)}.
\end{align*}

Let us set $T = 1$ and $\xi = dx - X(t, x)dt$. Then $d\sigma(w)$ is non-characteristic at every point of the image $\sigma(N)$. The section $\sigma$ is of the form
\begin{align*}
\sigma\colon\qquad \quad
\begin{aligned}
&u = f,\qquad u_x = g,\qquad u_t = \partial_t f + X(\partial_x f - g), \\
&u_{xx} = h,\qquad u_{xt} = \partial_t g + X(\partial_x g - h),\qquad \ldots
\end{aligned}
\end{align*}
Here $f$, $g$ and $h$ are arbitrary functions on $N$. The expressions for all other coordinates on $\mathcal{E}$ are unambiguously defined. We get
\begin{align*}
\sigma^*(l) = \Big(- \dfrac{\partial_x f \, \partial_t f}{2} + g\,\partial_t f - g^3 + h\,\partial_x g - \dfrac{h^2}{2} - \dfrac{X}{2}(\partial_x f - g)^2\Big)dt\wedge dx\,.
\end{align*}
Varying the corresponding action with respect to $f$, $g$ and $h$, we obtain
\begin{align*}
&\partial_t (\partial_x f - g) + \partial_x \big(X(\partial_x f - g)\big) = 0\,,\\
&\partial_t f - 3g^2 - \partial_x h + X(\partial_x f - g) = 0\,,\\
&\partial_x g - h = 0\,.
\end{align*}
These equations do not imply the relation $\partial_x f = g$. Therefore, the set of $\xi$-stationary points contains more than just local solutions. However, we also can vary with respect to $X$. As a result, we get the missing equation
\begin{align*}
\partial_x f = g\,.
\end{align*}

Thus, if $\xi$ determines a non-characteristic distribution, then a $\xi$-section $\sigma$ is a stationary point of the internal Lagrangian $\boldsymbol\ell$ iff $\sigma$ is a local solution. But this is not the case for $\xi$-stationary points.
}

\vspace{3.0ex}

\centerline{\bf{\large Acknowledgments}}

\vspace{2.0ex}

The author is grateful to I.S.~Krasil'shchik, A.M.~Verbovetsky and M.~Grigoriev for constructive discussions.



\begin{thebibliography}{99}

\bibitem{Druzhkov}
K. Druzhkov, {Lagrangian formalism and the intrinsic geometry of PDEs}, J. Geom. Phys. 189 (2023) 104848.
DOI: 10.1016/j.geomphys.2023.104848.

\bibitem{VinKr}
A.M. Vinogradov, I.S. Krasil'schik (eds.),
Symmetries and Conservation Laws for Differential Equations of Mathematical Physics, Vol.~182, American Mathematical Society, 1999.

\bibitem{Olver}
P.J. Olver, Applications of Lie Groups to Differential Equations, 2nd ed., Springer-Verlag,~1993.

\bibitem{Vin}
A.M.~Vinogradov, The $\mathcal{C}$-spectral sequence, Lagrangian formalism and conservation laws:
I the linear theory; II the non-linear theory, J. Math. Anal. and Appl. 100 (1984) 1--40, 41--129.\\
doi.org/10.1016/0022-247X(84)90071-4, doi.org/10.1016/0022-247X(84)90072-6.

\bibitem{Grigoriev}
M.~Grigoriev, {Presymplectic structures and intrinsic Lagrangians},
arXiv:1606.07532 (2016).

\bibitem{GriGri}
M. Grigoriev, V. Gritzaenko, {Presymplectic structures and intrinsic Lagrangians for massive fields},
Nucl. Phys. B (2022) 975(4):115686. DOI: 10.1016/j.nuclphysb.2022.115686.

\bibitem{KraVer}
J. Krasil’shchik, A.M. Verbovetsky, Homological methods in equations of mathematical physics, in: Advanced Texts in Mathematics, Open Education \& Sciences, Opava, 1998. arXiv:math/9808130.

\bibitem{FiePau}
M. Fierz, W. Pauli, {On relativistic wave equations for particles of arbitrary
spin in an electromagnetic field}, Proc. Roy. Soc. Lond. A173 (1939) 211--232.

\bibitem{Dorf}
I. Dorfman, Dirac Structures and Integrability of Nonlinear Evolution Equations, Vol. 176, JOHN WILEY \& SONS, 1993.





\end{thebibliography}

\end{document}
