\documentclass[12pt]{amsart}
\usepackage{fullpage,url,amssymb,enumerate,colonequals}

% \usepackage{amscd}   % for commutative diagrams
%\usepackage[all]{xy} % for complicated commutative diagrams, use pdf driver
\usepackage{mathrsfs} % for \mathscr (script letters)
% \usepackage{MnSymbol} % for dashed arrow
\usepackage[justification=centering]{caption}
%\usepackage{mathabx}
%\usepackage{MnSymbol}
%\usepackage{extarrows}
%\usepackage{rotating}
%\usepackage{mathrsfs}
%\usepackage{lscape}
%\usepackage[all,cmtip]{xy}
%\usepackage{verbatim}


%%%%% This command allows to modify the part font size without introducing \large in front of it. It looks the same in the pdf but the parts now appear in the file outline on the left panel in Overleaf.
%%%%%

%\usepackage{blindtext}

\makeatletter
%default definition of article.cls
%using \renewcommand instead of \newcommand
\renewcommand\part{%
   \if@noskipsec \leavevmode \fi
   \par
   \addvspace{4ex}%
   \@afterindentfalse
   \secdef\@part\@spart}

\def\@part[#1]#2{%
    \ifnum \c@secnumdepth >\m@ne
      \refstepcounter{part}%
      \addcontentsline{toc}{part}{\thepart\hspace{1em}#1}%
    \else
      \addcontentsline{toc}{part}{#1}%
    \fi
    {\parindent \z@ \raggedright
     \interlinepenalty \@M
     \normalfont
     \ifnum \c@secnumdepth >\m@ne
       \large\bfseries \partname\nobreakspace\thepart.
       %\par\nobreak
     \fi
     \large \bfseries #2%
     %%%\markboth{}{}\par}% removing redefinition of headings
     \par}%
    \nobreak
    \vskip 3ex
    \@afterheading}
\def\@spart#1{%
    {\parindent \z@ \raggedright
     \interlinepenalty \@M
     \normalfont
     \large \bfseries #1\par}%
     \nobreak
     \vskip 3ex
     \@afterheading}
\makeatother

%%%% End the command for the part size %%%%


% This is for resizeable \Sha
%\usepackage{natbib}
%\usepackage[OT2,T1]{fontenc}

%\DeclareSymbolFont{cyrletters}{OT2}{wncyr}{m}{n}
%\DeclareMathSymbol{\Sha}{\mathalpha}{cyrletters}{"58}


\usepackage{color}

\usepackage{xr-hyper}

\usepackage[
%       draft,
        colorlinks, citecolor=darkgreen,
        backref,
        pdfauthor={Filip Najman}, % add other authors
]{hyperref}
%\usepackage[alphabetic,backrefs,lite]{amsrefs} % for bibliography
\usepackage{cleveref}
\usepackage{comment}
%\usepackage{todonotes}



\newcommand{\defi}[1]{\textsf{#1}} % for defined terms

\newcommand{\dashedarrow}{\dashrightarrow}

% Characters
\newcommand{\Aff}{\mathbb{A}}
\newcommand{\C}{\mathbb{C}}
\newcommand{\F}{\mathbb{F}}
\newcommand{\Fbar}{{\overline{\F}}}
\newcommand{\G}{\mathbb{G}}
\newcommand{\Gm}{\mathbb{G}_{\mathrm{m}}}
\newcommand{\bbH}{\mathbb{H}}
\newcommand{\PP}{\mathbb{P}}
\newcommand{\Q}{\mathbb{Q}}
\newcommand{\R}{\mathbb{R}}
\newcommand{\Z}{\mathbb{Z}}
\newcommand{\Qbar}{{\overline{\Q}}}
\newcommand{\Zhat}{{\hat{\Z}}}
\newcommand{\Ebar}{{\overline{E}}}
\newcommand{\Zbar}{{\overline{\Z}}}
\newcommand{\kbar}{{\overline{k}}}
\newcommand{\Kbar}{{\overline{K}}}
\newcommand{\rhobar}{{\overline{\rho}}}
\newcommand{\ksep}{{k^{\operatorname{sep}}}}

\newcommand{\Frp}{{\mathfrak p}}

\newcommand{\Adeles}{\mathbf{A}}
\newcommand{\kk}{\mathbf{k}}

\newcommand{\mm}{\mathfrak{m}}

\newcommand{\eps}{\varepsilon}

\newcommand{\uom}{\underline{\omega}}

% bold characters
\newcommand{\boldf}{\mathbf{f}}
\newcommand{\boldl}{\ensuremath{\boldsymbol\ell}}
\newcommand{\boldL}{\mathbf{L}}
\newcommand{\boldr}{\mathbf{r}}
\newcommand{\boldw}{\mathbf{w}}
\newcommand{\boldzero}{\mathbf{0}}
\newcommand{\boldomega}{\ensuremath{\boldsymbol\omega}}


% mathcal characters
\newcommand{\calA}{\mathcal{A}}
\newcommand{\calB}{\mathcal{B}}
\newcommand{\calC}{\mathcal{C}}
\newcommand{\calD}{\mathcal{D}}
\newcommand{\calE}{\mathcal{E}}
\newcommand{\calF}{\mathcal{F}}
\newcommand{\calG}{\mathcal{G}}
\newcommand{\calH}{\mathcal{H}}
\newcommand{\calI}{\mathcal{I}}
\newcommand{\calJ}{\mathcal{J}}
\newcommand{\calK}{\mathcal{K}}
\newcommand{\calL}{\mathcal{L}}
\newcommand{\calM}{\mathcal{M}}
\newcommand{\calN}{\mathcal{N}}
\newcommand{\calO}{\mathcal{O}}
\newcommand{\calP}{\mathcal{P}}
\newcommand{\calQ}{\mathcal{Q}}
\newcommand{\calR}{\mathcal{R}}
\newcommand{\calS}{\mathcal{S}}
\newcommand{\calT}{\mathcal{T}}
\newcommand{\calU}{\mathcal{U}}
\newcommand{\calV}{\mathcal{V}}
\newcommand{\calW}{\mathcal{W}}
\newcommand{\calX}{\mathcal{X}}
\newcommand{\calY}{\mathcal{Y}}
\newcommand{\calZ}{\mathcal{Z}}

%mathfrak characters

\newcommand{\Fp}{\mathfrak{p}}
\newcommand{\Fq}{\mathfrak{q}}
\newcommand{\Ff}{\mathfrak{f}}
\newcommand{\n}{\mathfrak{n}}


\newcommand{\CC}{\mathscr{C}}
\newcommand{\FF}{\mathscr{F}}
\newcommand{\GG}{\mathscr{G}}
\newcommand{\II}{\mathscr{I}}
\newcommand{\JJ}{\mathscr{J}}
\newcommand{\LL}{\mathscr{L}}
\newcommand{\NN}{\mathscr{N}}
\newcommand{\OO}{\mathscr{O}}
\newcommand{\WW}{\mathscr{W}}
\newcommand{\XX}{\mathscr{X}}
\newcommand{\ZZ}{\mathscr{Z}}

% Math operators
\DeclareMathOperator{\Ann}{Ann}
\DeclareMathOperator{\Aut}{Aut}
\DeclareMathOperator{\Br}{Br}
\DeclareMathOperator{\cd}{cd}
\DeclareMathOperator{\Char}{char}
\DeclareMathOperator{\Cl}{Cl}
\DeclareMathOperator{\codim}{codim}
\DeclareMathOperator{\coker}{coker}
\DeclareMathOperator{\Cor}{Cor}
\DeclareMathOperator{\divv}{div}
\DeclareMathOperator{\Div}{Div}
\DeclareMathOperator{\Det}{Det}
\DeclareMathOperator{\Dic}{Dic}

\DeclareMathOperator{\End}{End}
\newcommand{\END}{{\EE}\!nd}
\DeclareMathOperator{\Eq}{Eq}
\DeclareMathOperator{\Ext}{Ext}
\newcommand{\EXT}{{\E}\!xt}
\DeclareMathOperator{\Fix}{\tt Fix}
\DeclareMathOperator{\Frac}{Frac}
\DeclareMathOperator{\Frob}{Frob}
\DeclareMathOperator{\Gal}{Gal}
\DeclareMathOperator{\Gr}{Gr}
\DeclareMathOperator{\Hom}{Hom}
\newcommand{\HOM}{{\HH}\!om}
\DeclareMathOperator{\im}{im}
\DeclareMathOperator{\Ind}{Ind}
\DeclareMathOperator{\inv}{inv}
\DeclareMathOperator{\Jac}{Jac}
\DeclareMathOperator{\lcm}{lcm}
\DeclareMathOperator{\Lie}{Lie}
\DeclareMathOperator{\Log}{Log}
\DeclareMathOperator{\MakeDecentModel}{\tt MakeDecentModel}
\DeclareMathOperator{\nil}{nil}
\DeclareMathOperator{\Norm}{Norm}
\DeclareMathOperator{\NP}{NP}
\DeclareMathOperator{\Num}{Num}
\DeclareMathOperator{\odd}{odd}
\DeclareMathOperator{\ord}{ord}
\DeclareMathOperator{\Pic}{Pic}
\DeclareMathOperator{\PIC}{\bf Pic}
\DeclareMathOperator{\Prob}{\bf P}
\DeclareMathOperator{\Proj}{Proj}
\DeclareMathOperator{\PROJ}{\bf Proj}
\DeclareMathOperator{\rank}{rank}
\DeclareMathOperator{\re}{Re}
\DeclareMathOperator{\reg}{reg}
\DeclareMathOperator{\res}{res}
\DeclareMathOperator{\Res}{Res}
\DeclareMathOperator{\rk}{rk}
\DeclareMathOperator{\scd}{scd}
\DeclareMathOperator{\Sel}{Sel}
\DeclareMathOperator{\Sp}{Sp}
\DeclareMathOperator{\Spec}{Spec}
\DeclareMathOperator{\SPEC}{\bf Spec}
\DeclareMathOperator{\Spf}{Spf}
\DeclareMathOperator{\supp}{supp}
\DeclareMathOperator{\Sym}{Sym}
\DeclareMathOperator{\tr}{tr}
\DeclareMathOperator{\Tr}{Tr}
\DeclareMathOperator{\trdeg}{tr deg}


% Categories
\newcommand{\Ab}{\operatorname{\bf Ab}}
\newcommand{\Groups}{\operatorname{\bf Groups}}
\newcommand{\Schemes}{\operatorname{\bf Schemes}}
\newcommand{\Sets}{\operatorname{\bf Sets}}

% Text subscripts, superscripts
\newcommand{\ab}{{\operatorname{ab}}}
\newcommand{\an}{{\operatorname{an}}}
\newcommand{\Az}{{\operatorname{Az}}}
\newcommand{\CS}{\operatorname{\bf CS}}
\newcommand{\et}{{\operatorname{et}}}
\newcommand{\ET}{{\operatorname{\bf \acute{E}t}}}
\newcommand{\Fl}{{\operatorname{f\textcompwordmark l}}}
\newcommand{\good}{{\operatorname{good}}}
\newcommand{\op}{{\operatorname{op}}}
\newcommand{\perf}{{\operatorname{perf}}}
\newcommand{\red}{{\operatorname{red}}}
\newcommand{\regular}{{\operatorname{regular}}}
\newcommand{\sing}{{\operatorname{sing}}}
\newcommand{\smooth}{{\operatorname{smooth}}}
\newcommand{\tH}{{\operatorname{th}}}
\newcommand{\tors}{{\operatorname{tors}}}
\newcommand{\nontors}{{\operatorname{non-tors}}}
\newcommand{\unr}{{\operatorname{unr}}}
\newcommand{\nr}{{\operatorname{nr}}}
\newcommand{\Zar}{{\operatorname{Zar}}}
\newcommand{\ns}{{\operatorname{ns}}}
\renewcommand{\sp}{{\operatorname{sp}}}
\newcommand{\vv}{\upsilon}
\newcommand{\Cech}{\v{C}ech}
\newcommand{\E}{{\operatorname{\bf E}}}
\newcommand{\GalQ}{{\Gal}(\Qbar/\Q)}
\newcommand{\GL}{\operatorname{GL}}
\newcommand{\HH}{{\operatorname{H}}}
\newcommand{\HHcech}{{\check{\HH}}}
\newcommand{\HHat}{{\hat{\HH}}}
\newcommand{\M}{\operatorname{M}}
\newcommand{\PGL}{\operatorname{PGL}}
\newcommand{\PSL}{\operatorname{PSL}}
\newcommand{\SL}{\operatorname{SL}}

\newcommand{\del}{\partial}
\newcommand{\directsum}{\oplus} % binary direct sum
\newcommand{\Directsum}{\bigoplus} % direct sum of a collection
\newcommand{\injects}{\hookrightarrow}
\newcommand{\intersect}{\cap} % binary intersection
\newcommand{\Intersection}{\bigcap} % intersection of a collection
\newcommand{\isom}{\simeq}
\newcommand{\notdiv}{\nmid}
\newcommand{\surjects}{\twoheadrightarrow}
\newcommand{\tensor}{\otimes} % binary tensor product
\newcommand{\Tensor}{\bigotimes} % tensor product of a collection
\newcommand{\union}{\cup} % binary union
\newcommand{\Union}{\bigcup} % union of a collection

\newcommand{\Algorithm}{\textbf{Algorithm}\ }
\newcommand{\Subroutine}{\textbf{Subroutine}\ }

\newcommand{\isomto}{\overset{\sim}{\rightarrow}}
\newcommand{\isomfrom}{\overset{\sim}{\leftarrow}}
\newcommand{\leftexp}[2]{{\vphantom{#2}}^{#1}{#2}}
\newcommand{\rholog}{\rho \log}
\newcommand{\sigmaiota}{{\leftexp{\sigma}{\iota}}}
\newcommand{\sigmaphi}{{\leftexp{\sigma}{\phi}}}
\newcommand{\sigmatauphi}{{\leftexp{\sigma\tau}{\phi}}}
\newcommand{\tauphi}{{\leftexp{\tau}{\phi}}}
\newcommand{\To}{\longrightarrow}
\newcommand{\Floor}[1]{\left\lfloor #1 \right\rfloor}

\numberwithin{equation}{section}

\newtheorem{theorem}[equation]{Theorem}
\newtheorem{lemma}[equation]{Lemma}
\newtheorem{corollary}[equation]{Corollary}
\newtheorem{proposition}[equation]{Proposition}
\newtheorem{example}[equation]{Example}
\theoremstyle{definition}
\newtheorem{definition}[equation]{Definition}
\newtheorem{conjecture}[equation]{Conjecture}


\theoremstyle{remark}
\newtheorem{remark}[equation]{Remark}




\definecolor{darkgreen}{rgb}{0,0.5,0}

\setlength{\parindent}{0mm}
\setlength{\parskip}{1ex plus 0.5ex}
\setcounter{tocdepth}{1}


%%%%%% Nicolas' commands %%%%%%
\DeclareRobustCommand\longtwoheadrightarrow
     {\relbar\joinrel\twoheadrightarrow}



% The following is to avoid a bug with the command \contrib
\makeatletter
\let\@wraptoccontribs\wraptoccontribs
\makeatother

%%%%%% end of Nicolas' commands %%%%%%

\begin{document}




\title[On $r$-isogenies over $\mathbb{Q}(\zeta_r)$]{On $r$-isogenies over $\mathbb{Q}(\zeta_r)$ \\ of elliptic curves with rational $j$-invariants}


\author{Filip Najman}
\address{University of Zagreb, Bijeni\v{c}ka Cesta 30, 10000 Zagreb, Croatia}
\email{fnajman@math.hr}



\date{\today}

\keywords{elliptic curves, Galois representations}
\subjclass[]{11G05}


\begin{abstract}
The main goal of this paper is to determine for which odd prime numbers~$r$ can an elliptic curve~$E$ defined over $\Q$ have an $r$-isogeny over $\Q(\zeta_r)$. We study this question under various assumptions on the 2-torsion of $E$. Apart from being a natural question itself, the mod~$r$ representations attached to such $E$ arise in the Darmon program for the generalized Fermat equation of signature~\((r,r,p)\), playing a key role in the proof of modularity of certain Frey varieties in the recent work of Billerey, Chen, Dieulefait and Freitas.
\end{abstract}


\thanks{I gratefully acknowledge support by QuantiXLie Centre of Excellence, a project co-
financed by the Croatian Government and European Union through the European Regional Devel-
opment Fund - the Competitiveness and Cohesion Operational Programme (Grants KK.01.1.1.01.0004
and PK.1.1.02) and by the Croatian Science Foundation under the project no. IP-2022-10-5008.}

\maketitle

{
\hypersetup{linkcolor=black}
%\tableofcontents
}





%%%%%%%%%%%%%%%%%%%%%%%%%%%%%%%%%%%%%%%%%%%%%%%%%%%%%%
\section{Introduction}
%%%%%%%%%%%%%%%%%%%%%%%%%%%%%%%%%%%%%%%%%%%%%%%%%%%%%%

In their recent work~\cite{BCDF3}, Billerey, Chen, Dieulefait and Freitas develop an approach to the generalized Fermat equation of signature~\((r,r,p)\) based on ideas from Darmon's program \cite{DarmonDuke} and a construction of Frey hyperelliptic curves due to Kraus. The results of the present paper play a key role in their proof of the modularity of the Jacobians of these Frey hyperelliptic curves and other abelian varieties considered by Darmon.

Let~\(r\geq 3\) be a prime number. For an elliptic curve~\(E\) defined over a number field~\(K\), we denote by~\(\rhobar_{E,r} : G_K \to \GL_2(\F_r)\) its mod~$r$ Galois representation (after fixing a basis for the $r$-torsion module~$E[r]$). The present paper is concerned with the (ir)reducibility of the representation~\(\rhobar_{E,r}\) where~\(K = \Q(\zeta_r)\) and~\(E\) is a base change to~\(K\) of an elliptic curve defined over~\(\Q\). %We use the letter $r$ to denote the prime degree of an isogeny (as opposed to the more standard $\ell$ of $p$) to better conform to the notation of \cite{BCDF3}.

In some cases our results will also hold for certain elliptic curves $E/\Q(\zeta_r)$ with $j(E)\in \Q$, not just those which are base changes of elliptic curves defined over $\Q$; in those instances we will state our results in this greater generality.

The reducibility of~\(\rhobar_{E,r} : G_{\Q(\zeta_r)}\to \GL_2(\F_r)\) is equivalent to~\(E/\Q(\zeta_r)\) having an isogeny of degree~\(r\) over~\(\Q(\zeta_r)\). We study this question under various assumptions on the \(2\)-torsion of~\(E\).


All code for Magma computations used to verify the claims in the paper can be found at
\begin{center}
{\href{https://github.com/F-Najman/r-isogenies}{\url{https://github.com/F-Najman/r-isogenies}}}.
\end{center}

%

\section*{Acknowledgements}

I thank Nicolas Billerey and Nuno Freitas for asking questions motivating this paper and for many helpful comments, Davide Lombardo for pointing out the proof of \Cref{prop1} for $r<17$, and the anonymous referees for their many helpful suggestions.

%%%%%%%%%%%%%%%%%%%%%
\section{Elliptic curves with $r$-isogenies over $\mathbb{Q}(\zeta_r)$}
%%%%%%%%%%%%%%%%%%%%%

In this subsection we determine when do elliptic curves $E$ defined over $\Q$, or in some instances with $j(E)\in \Q$, have an $r$-isogeny over $\Q(\zeta_r)$. Having an $r$-isogeny over $\Q(\zeta_r)$ is equivalent to $\rhobar_{E,r}(G_{\Q(\zeta_r)})$ being conjugate in $\GL_2(\F_r)$ to a subgroup of a Borel subgroup.




For a subgroup $G\leq \GL_2(\F_r)$, define $S(G):=G\cap \SL_2(\F_r)$. Denote by $C_s(r)$ the subgroup of diagonal matrices in $\GL_2(\F_r)$ and by $C_s^+(r)$ its normalizer in $\GL_2(\F_r)$. Let $\epsilon=-1$ if $r\equiv 3 \pmod 4$ and otherwise let $\epsilon\geq 2$  be the smallest integer which is not a quadratic residue modulo $r$. Denote by $C_{ns}(r)$ the group
$$C_{ns}(r):=\left\{ \begin{pmatrix}
  a & \epsilon b \\
  b & a
\end{pmatrix},(a, b) \in \F_r^2\backslash \{(0,0)\}\right\},$$
whose conjugates in $\GL_2(\F_r)$ are the non-split Cartan subgroups and let

\begin{equation}
\label{cns+def}
C_{ns}^+(r):=\left\{
\begin{pmatrix}
  a & \epsilon b \\
  b & a
\end{pmatrix}, (a, b) \in \F_r^2\backslash \{(0,0)\} \right\}
\cup
\left\{
 \begin{pmatrix}
  c & \epsilon d \\
  -d & -c
\end{pmatrix}, (c, d) \in \F_r^2\backslash \{(0,0)\} \right\}
\end{equation}
be the normalizer of $C_{ns}(r)$ in $\GL_2(\F_r)$.

\begin{lemma}\label{lem:subgroups}
Let $r$ be an odd prime. The group $S(C_s^+(r))$ is conjugate in~\(\GL_2(\F_r)\) to
 $$\left\{ \begin{pmatrix}
  a & 0 \\
  0 & a^{-1}
\end{pmatrix},\begin{pmatrix}
  0 & a \\
  -a^{-1} & 0

\end{pmatrix}, a \in \F_r^\times \right\}.$$

The group $S(C_{ns}^+(r))$ is the subset of matrices in \eqref{cns+def} with determinant $1$. The groups $S(C_s^+(r))$ and $S(C_{ns}^+(r))$ are not conjugate in $\GL_2(\F_r)$ to a subgroup of a Borel subgroup.

\end{lemma}
\begin{proof}
It is straightforward to check that $S(C_s^+(r))$ and $S(C_{ns}^+(r))$ are of the claimed form and that $S(C_s^+(r))$ acts freely on $\F_r^2 \backslash \{(0,0)\}$, from which it follows that the length of each orbit is $2(r-1)$. As a fixed $1$-dimensional subspace would give an orbit of length $\leq r - 1$, it follows that there are no fixed $1$-dimensional subspaces of $\F_r^2$.


The group $S(C_{ns}^+(r))$ has order $2(r+1)$ since $\det :C_{ns}^+(r) \rightarrow\F_r^\times$ is onto. Examine the action of $S(C_{ns}^+(r))$ on $\F_r^2 \backslash \{(0,0)\}$. The characteristic polynomial of a matrix in $S(C_{ns}^+(r))$ is either $x^2-2ax+1$ or $x^2+1$ and hence $1$ is a root of it only in the first case when $a=1$, which happens only for the identity matrix. We conclude that $S(C_{ns}^+(r))$ acts freely on $\F_r^2 \backslash \{(0,0)\}$ and hence the orbit of any $(x,y)$ has cardinality $2(r+1)$ and again we can conclude that there are no fixed $1$-dimensional subspaces of $\F_r^2$.

\begin{comment}
To see that $S(C_{ns}^+(r))$ acts freely on $\F_r^2 \backslash (0,0)$, we have to show that $Ax=x$, where $A\in S(C_{ns}^+(r))$ and $x\in \F_r^2 \backslash (0,0)$ implies $A=I$.
Let first $A=\begin{pmatrix}
    a & \epsilon b \\
    b & a
  \end{pmatrix}.$
$$\begin{pmatrix}
    a & \epsilon b \\
    b & a
  \end{pmatrix}\begin{pmatrix}
                 x \\
                 y
               \end{pmatrix} = \begin{pmatrix}
                 x \\
                 y
               \end{pmatrix}.$$
It follows that
$$x(a-1)=-\epsilon b y,$$
$$y(a-1)=-bx.$$
If $a=1$, then $1-\epsilon b^2=1$, so $b=0$, so we have the identity matrix. If $a\neq 1$, we get
$$y(a-1)=\frac{\epsilon b^2 y}{a-1},$$
so we get $-2ay=0$. Suppose $y\neq 0$; then $a=0$, and we get
$$x= \epsilon b y,$$
$$y=bx,$$
from which we get $y=\epsilon b^2 y$. It follows $\epsilon b^2$, which is a contradiction with $\epsilon$ being a non-square. Thus $y$ has to be $0$; now we get
$$x(a-1)=0,$$
$$bx=0.$$
$x$ cannot be $0$ (since we supposed that $(x,y)\neq (0,0)$, so $a=1$ which is a contradiciton with our assumption.
\end{comment}
\end{proof}
%We will need the following result of Furio and Lombardo \cite{FL}, which builds on work of Zywina \cite[Proposition 1.13.]{zyw} and Le Fourn and Lemos \cite{Le_Fourn_Lemos}.


We will need the following result of Zywina (\cite[Proposition 1.13.]{zyw}); see~\cite[Appendix~B]{Le_Fourn_Lemos} for a published proof.

\begin{proposition}[Zywina]
\label{prop-zyw}
Suppose that $E/\Q$ does not have CM, $r\geq 17$, $(r,j(E))\notin\{ (17,-17\cdot 373^3/2^{17}), (17,-17^2\cdot 101^3/2), (37,-7\cdot 11^3), (37,-7\cdot 137^3\cdot 2083^3)\}$ and $\rhobar_{E,r}$ is not surjective. Then
\begin{enumerate}
\item If $r\equiv 1 \pmod 3$, then $\rhobar_{E,r}(G_\Q)$ is conjugate in $\GL_2(\F_r)$ to $C_{ns}^+(r)$.
\item If $r\equiv 2 \pmod 3$, then $\rhobar_{E,r}(G_\Q)$ is conjugate in $\GL_2(\F_r)$ to either $C_{ns}^+(r)$ or
    $$G_3(r):=\left\{a^3,\ a \in C_{ns}(r)\right\} \cup \left\{\begin{pmatrix}
                        1 &  0\\
                        0 & -1
                      \end{pmatrix}a^3,\ a \in C_{ns}(r)\right\}.$$
\end{enumerate}
\end{proposition}

\begin{remark}
The recent results of Furio and Lombardo \cite{FL} prove that if $r\geq 37$, then the case (2) in \Cref{prop-zyw} is not possible.
\end{remark}

%\begin{proposition}
%\label{prop-zyw}
%Suppose that $E/\Q$ does not have CM, $r\geq 17$, $(r,j(E))\notin\{ (17,-17\cdot 373^3/2^{17}), (17,-17^2\cdot 101^3/2), (37,-7\cdot 11^3), (37,-7\cdot 137^3\cdot 2083^3)\}$ and $\rhobar_{E,r}$ is not surjective. Then $\rhobar_{E,r}(G_\Q)$ is conjugate in $\GL_2(\F_r)$ to $C_{ns}^+(r)$.
%\end{proposition}

%The following propositions give conditions for elliptic curves $E$ satisfying $j(E)\in \Q$, without any assumption on its~$2$-torsion,
%to have $r$-isogenies over $\Q(\zeta_r)$.

Given an elliptic curve $E$ defined over~\(\Q(\zeta_r)\) with $j(E)\neq 0,1728$, having an $r$-isogeny and a certain amount of $2$-torsion is a property depending only on the $j$-invariant, so we can (and will in the proofs of the propositions below) suppose that $E$ is defined over $\Q$. This is justifiable because if $E$ is not defined over $\Q$ and $j(E)\in \Q \backslash \{0,1728\}$, then $E$ has a quadratic twist $E'$ defined over $\Q$ and $E$ and $E'$ will have the same $2$-torsion structure and isogenies of the same degrees over $\Q(\zeta_r)$, and so we can replace~$E$ by~$E'$ in the arguments.

A subgroup $G$ of $\GL_2(\F_r)$ is called \textit{applicable} (cf. \cite[Definition 2.1]{zyw}) if $\det G=\F_r^\times$ and $G$ contains an element of trace $0$ and determinant $-1$ (which is the image of complex conjugation). Note that $\rhobar_{E,r}(G_\Q)$ is always an applicable subgroup \cite[Proposition 2.2]{zyw}.


\begin{proposition}
\label{prop1}
Let $r$ be an odd prime, $E/\Q(\zeta_r)$ with $j(E)\in \Q$ be an elliptic curve without complex multiplication (CM), such that $E$ has an $r$-isogeny over $\Q(\zeta_r)$ and
\begin{multline}\label{eq:execptions}
(r,j(E))\notin\left\{ (7,\frac{3^3\cdot 5 \cdot 7^5}{2^7})\right\}\\
 \cup \left\{\left(5,
\frac{5^4t^3(t^2 + 5t + 10)^3(2t^2 + 5t + 5)^3(4t^4 + 30t^3 + 95t^2 + 150t + 100)^3}{(t^2 + 5t + 5)^5 (t^4 + 5t^3 + 15t^2 + 25t + 25)^5}\right),  t\in \Q \right\}.
\end{multline}
Then $E$ has an $r$-isogeny over $\Q$.
\end{proposition}


\begin{proof}
As explained above we can assume that $E$ is defined over $\Q$. Suppose first $r\geq 17$.  We have shown in Lemma \ref{lem:subgroups} that $S(C_{ns}^+(r))$ does not fix any $1$-dimensional subspace of $\F_r^2$. Assume now $r\equiv 2 \pmod 3$. The group $\{a^3, a\in C_{ns}(r)\}$ is cyclic of order $\frac{r^2-1}{3}$, and the map $t\mapsto t^3$ is an automorphism of $\F_r^\times$, so we conclude that $\#S(G_3(r))=\frac{2(r+1)}{3}$.

Now we study the action of $S(G_3(r))$ on $\F_r^2 \backslash \{(0,0)\}$. Since $S(G_3(r))\leq S(C_{ns}^+(r))$ this action is free. Hence the orbit of any $(x,y)$ has cardinality $2(r+1)/3$. It is again easy to see that there cannot be a fixed $1$-dimensional subspace of $\F_r^2$, as otherwise there would be orbits $S_1, \ldots, S_k$ in $\F_r^2 \backslash \{(0,0)\}$ such that the sum of their lengths is $r-1$. This is clearly impossible.
Now applying Proposition \ref{prop-zyw} completes the proof for $r\geq 17$.

To deal with $r< 17$ we note that $E$ has an $r$-isogeny over $\Q(\zeta_r)$ but no $r$-isogeny over $\Q$ if and only if $\rhobar_{E,r}(G_\Q)$ acts irreducibly on $\F_r^2$, but $\rhobar_{E,r}(G_\Q)\cap \SL_2(\F_r)$ acts reducibly on $\F_r^2$. A brute force search among all proper applicable subgroups of $\GL_2(\F_r)$ yields, up to conjugacy, one possible subgroup for $r=5$, 2 groups for $r=7,11$ and $13$ each.

For $r=13$ the candidate groups are of index 182 and 546, which are not possible by \cite[Theorem 1.8 and Remark 1.9]{zyw} and \cite[Theorem 1.1.]{bdmtv2}. For $r=11$ the candidate groups are of index 132 and 264, which are not possible by \cite[Theorem 1.6]{zyw}.

For $r=7$, the two candidate groups are of index 56 and 112, and both are possible by \cite[Theorem 1.5]{zyw}; they are conjugate in $\GL_2(\F_7)$ to the groups denoted $G_1$ and $H_{1,1}$ in \cite[Section 1.4]{zyw}. As $G_1=\pm H_{1,1}$ and our conditions are quadratic-twist invariant, by \cite[Theorem 1.5]{zyw}, it follows that $E$ has a $7$-isogeny over $\Q(\zeta_7)$, but no $7$-isogeny over $\Q$, if and only if $j(E)=\frac{3^3\cdot 5 \cdot 7^5}{2^7}$.

For $r=5$ the only candidate group we find is conjugate in $\GL_2(\F_5)$ to the group denoted $G_3$ in \cite[Section 1.3]{zyw}. By \cite[Theorem 1.4. (ii)]{zyw}, $\rhobar_{E,3}\subseteq G_3$ if and only if $j(E)$ is of the form given in \eqref{eq:execptions}.
\begin{comment}

Assume now $r\equiv 2 \pmod 3$. The group $\{a^3, a\in C_{ns}(r)\}$ is cyclic of order $\frac{r^2-1}{3}$, and the map $t\mapsto t^3$ is an automorphism of $\F_r^\times$, so we conclude that $\#S(G_3(r))=\frac{2(r+1)}{3}$.

Now we study the action of $S(G_3(r))$ on $\F_r^2 \backslash \{(0,0)\}$. Since $S(G_3(r))\leq S(C_{ns}^+(r))$ this action is free. Hence the orbit of any $(x,y)$ has cardinality $2(r+1)/3$. It is again easy to see that there cannot be a fixed $1$-dimensional subspace of $\F_r^2$, as otherwise there would be orbits $S_1, \ldots, S_k$ in $\F_r^2 \backslash \{(0,0)\}$ such that the sum of their lengths is $r-1$. This is clearly impossible.
\end{comment}

\end{proof}

\begin{remark}
  Recall (see e.g. \cite[Table 4]{lr}) that for $r\geq 17$, $E/\Q$ without CM has an $r$-isogeny if and only if $$(r,j(E))\in\left\{ (17,-17\cdot 373^3/2^{17}), (17,-17^2\cdot 101^3/2), \right.
\left.(37,-7\cdot 11^3), (37,-7\cdot 137^3\cdot 2083^3)\right\},$$ so by \Cref{prop1} these are the only instances when a non-CM elliptic curve defined over $\Q$ can have an $r$-isogeny over $\Q(\zeta_r)$ for $r\geq 17$.
\end{remark}


\begin{proposition}
\label{prop:ired}
Suppose that $r\geq 5$, that $E/\Q(\zeta_r)$ has $CM$ and either $j(E)\in \Q\backslash \{0,1728\}$ or if $j(E)\in\{0,1728\}$ that $E$ is a base change of an elliptic curve defined over $\Q$. The following are equivalent:
\begin{itemize}
  \item[(a)] The curve $E$ has an $r$-isogeny over $\Q(\zeta_r)$.
  \item[(b)] Every elliptic curve $E'/\Q$ with $j(E')=j(E)$ has an $r$-isogeny over $\Q$.
  \item[(c)] $r$ divides $D$, where $-D$ is the discriminant of the imaginary quadratic number field containing the endomorphism ring of $E$.
\end{itemize}
%Then $E$ has an $r$-isogeny over $\Q(\zeta_r)$ if and only if any elliptic curve $E'/\Q$ with $j(E')=j(E)$ has an $r$-isogeny over $\Q$ which occurs if and only if $r$ divides $D$, where $-D$ is the discriminant of the imaginary quadratic number field containing the endomorphism ring of $E$.
\end{proposition}
\begin{proof}

We can assume that $E=E'$, i.e., $E$ is a base change of an elliptic curve defined over $\Q$, using for $j(E)\notin\{0,1728\}$ the same argumentation as before \Cref{prop1}.

For $j(E)\neq 0$, this follows directly from \cite[Proposition 1.14]{zyw} and Lemma \ref{lem:subgroups}.

Suppose $j(E)=0$. By \cite[Proposition 1.16]{zyw} if $r\equiv 2,5, 8\pmod 9$ then $\overline \rho_{E,r}(G_\Q)$ contains the subgroup $G_3(r)$ defined in \Cref{prop-zyw}. Using the same argumentation as in the proof of Proposition \ref{prop1} we conclude that $E$ has no $r$-isogeny over $\Q(\zeta_r)$.

Suppose $r\equiv 1,4, 7\pmod 9$. By \cite[Proposition 1.16]{zyw} it follows that $\overline \rho_{E,r}(G_\Q)$ contains the subgroup $G$ of $C_s^+(r)$ consisting of matrices of the form $\begin{pmatrix}
                        a & 0 \\
                        0 & b
                      \end{pmatrix}$
and $\begin{pmatrix}
                        0 & a \\
                        b & 0
                      \end{pmatrix}$
with $a/b\in (\F_r^\times)^3$. Suppose that $\overline \rho_{E,r}(G_{\Q(\zeta_r)})$ is reducible, so $S(G)$ fixes a 1-dimensional subspace of $\F_r^2$ generated by $(x,y)$. Let $t\in \F_r$ such that $t^3 \neq t^{-3}$; such a $t$ exists for all $r\neq 7$. Then for
$B:=\begin{pmatrix}
                        t^3 & 0 \\
                        0 & t^{-3}
                      \end{pmatrix} \in S(G)$, since we have $B(x,y)=(t^3x,t^{-3}y)=\beta(x,y)$ for some $\beta \in \F_r^\times$ we conclude that either $x=0$ or $y=0$.

Since $A=\begin{pmatrix}
                        0 & 1 \\
                        -1 & 0
                      \end{pmatrix} \in S(G)$ we have $A(x,y)=(y,-x)=\alpha (x,y)$ for some $\alpha \in \F_r^\times$, so $y=\alpha x=-\alpha^2 y$, which implies that if one of $x$ or $y$ is equal to 0, then so is the other. The equality $y=\alpha x=-\alpha^2 y$ also eliminates the case $r=7$ as $-1$ is not a square in $\F_7^\times$, so again we get $y=x=0$. This gives a contradiction and the result.
\end{proof}


%%%%%%%%%%%%%%%%%%%%%
\section{Elliptic curves with a $2$-torsion point and reducible mod $r$ Galois representations}
%%%%%%%%%%%%%%%%%%%%%


In this section we prove results about surjectivity of mod $r$ Galois representations of elliptic curves $E/\Q$ with a point of order $2$ over $\Q$. %or with full $2$-torsion  E have both full $2$-torsion and a mod $r$ representation that is (absolutely) reducible for $r>?$.


\begin{proposition}\label{prop:B1}
Let~\(E/\Q\) be a non-CM elliptic curve such that \(E\) has a \(\Q\)-rational \(2\)-torsion point and let~\(r\ge11\) be a prime number. Then, we have~\(\rhobar_{E,r}(G_\Q) = \GL_2(\F_r)\). In particular, \(E\) has no isogeny of degree~\(r\) over~\(\Q(\zeta_r)\).
\end{proposition}
\begin{proof}
Denote by~\(j(E)\) the \(j\)-invariant of~\(E\) and assume for a contradiction that the representation~\(\rhobar_{E,r}:G_\Q\rightarrow\GL_2(\F_r)\) is not surjective. We first deal with the case~\(r = 13\). Recall that we are in one of the following situations:
\begin{enumerate}
%\item\label{item:SL2} \(\rhobar_{E,13}(G_\Q) = \rhobar_{E,13}(G_\Q)\) contains~\(\SL_2(\F_{13})\);
\item\label{item:Borel} \(\rhobar_{E,13}(G_\Q)\) is included in a Borel subgroup of~\(\GL_2(\F_{13})\);
\item\label{item:NormC} \(\rhobar_{E,13}(G_\Q)\) is included in the normalizer of a Cartan subgroup of~\(\GL_2(\F_{13})\);
\item\label{item:excep} the projective image~\(H_{E,13}\) of~\(\rhobar_{E,13}(G_\Q)\) is isomorphic to~\(A_4\), \(S_4\), or~\(A_5\).
\end{enumerate}
%We discard~(\ref{item:SL2}) as the map~\(\det\rhobar_{E,13}:G_\Q\rightarrow\F_{13}^\times\) is surjective, and hence~\(\rhobar_{E,13}\) would be surjective in that case.

We eliminate~(\ref{item:Borel}) with a result of Mazur-V\'elu \cite{MazurVelu}: If~\(E\) had a \(13\)-isogeny, then it would have a \(26\)-isogeny over~\(\Q\), which is impossible. The main result of~\cite{bdmtv} states that case~(\ref{item:NormC}) does not occur. Finally, \(H_{E,13}\) is not isomorphic to~\(A_5\) as the order of~\(\PGL_2(\F_{13})\) is not divisible by~\(5\). Therefore~\(H_{E,13}\subset S_4\) and according to~\cite[Theorem~1.1 and Section 5.1]{bdmtv2} (see also \cite[Corollary 1.9]{BanwaitCremona} ), we have that
%\begin{multline*}
%j(E) \in\left\{11225615440/1594323, -160855552000/1594323, \right. \\
%\left.90616364985637924505590372621162077487104/197650497353702094308570556640625\right\}.
%\end{multline*}
\begin{multline*}
j(E)\in\left\{2^4\cdot 5\cdot 13^4\cdot 17^3/3^{13}, -2^{12}\cdot 5^3\cdot 11\cdot 13^4/3^{13},\right. \\
\left. 2^{18}\cdot 3^3\cdot 13^4\cdot 127^3\cdot 139^3\cdot 157^3\cdot 283^3\cdot 929/(5^{13}\cdot 61^{13})\right\}.
\end{multline*}
For each of these three values, we check that any elliptic curve with that $j$-invariant has trivial $2$-torsion over $\Q$, which shows that the case (\ref{item:excep}) is not possible. %Since for~\(j\in\Q\) non-CM (i.e., \(j\) not in the list of the thirteen rational CM \(j\)-invariants), elliptic curves~\(F/\Q\) such that~\(j(F) = j\) all have the same amount of \(2\)-torsion, we obtain a contradiction and the desired result for~\(r = 13\).

From now on, assume~\(r\ge11\) and~\(r\neq13\).  We first show that~\(\rhobar_{E,r}(G_\Q)\) is conjugate to a subgroup of~\(C_{ns}^+(r)\). For~\(r = 11\), this follows from~\cite[Theorem~1.6]{zyw} after checking that elliptic curves with $j$-invariant $-11^2$ and $-11\cdot 131^3$ have no non-trivial $2$-torsion over $\Q$. If~\(r\ge17\), this follows from Proposition~\ref{prop-zyw} (see also~\cite[Theorem~1.11]{zyw}), unless we have~\(j(E)\in\{ -17\cdot 373^3/2^{17}, -17^2\cdot 101^3/2, -7\cdot 11^3, -7\cdot 137^3\cdot 2083^3\}\). However, we check as before that these \(j\)-invariants correspond to elliptic curves with trivial rational \(2\)-torsion. Hence we have proved that~\(\rhobar_{E,r}(G_\Q)\) is conjugate to a subgroup of~\(C_{ns}^+(r)\).

According to~\cite[Proposition~2.1]{lemos}, we therefore have~\(j(E)\in\Z\). Since moreover \(E\) has a rational point of order~\(2\), its \(j\)-invariant~\(j(E)\) is of the shape
\[
j(E)=\frac{(t+16)^3}{t}
\]
for some~\(t\in\Q\) (\cite[p.~179]{Birch}) and hence one can see from \cite[p.~142]{lemos} (eliminating the CM \(j\)-invariants) that
\begin{multline*}
j(E) \in \{-2^2\cdot 7^3,-2^4\cdot 3^3,-2^6, 2^7, 2^4\cdot 5^3, 2^{11}, 2^2\cdot 3^6, 2^7\cdot 3^3, 17^3, 2^5\cdot 7^3, 2^5\cdot 3^6, 2^4\cdot 17^3, \\
2^3\cdot 31^3, 2^2\cdot 3^6\cdot 7^3, 2^2\cdot 5^3\cdot 13^3, 2\cdot 127^3, 2\cdot 3^3\cdot 43^3, 257^3\}.
     \end{multline*}
For each element in this list, we find an elliptic curve over~\(\Q\) with the corresponding \(j\)-invariant such that all of its mod $p$ representations are surjective for $p>3$ prime. We check this in LMFDB \cite{lmfdb}. In particular, its mod~\(r\) representation is surjective. Since the mod~\(r\) representation of a single quadratic twist is surjective if and only if it is surjective for all quadratic twists, the same conclusion holds for~\(E\). This gives the desired contradiction.

The last statement follows from the fact that \(\rhobar_{E,r}(G_{\Q(\zeta_r)})=\rhobar_{E,r}(G_{\Q}) \cap \SL_2(\F_r) = \SL_2(\F_r)\) has order greater than the order of a Borel subgroup of~\(\GL_2(\F_r)\).
\end{proof}

\begin{comment}
\todo[inline]{
For $r=7$, we were unable to prove that rational non-CM elliptic curves with a rational point of order $2$ necessarily have surjective mod $7$ representations, as this leads to finding all the rational points on high genus modular curves (see e.g. \cite[Table 14]{DGJ}, entry [2B,7NS]). However we are able to find all such elliptic curves with $7$-isogenies, even when including elliptic curves with CM and those with $j(E)\in \Q$ while requiring that $E$ only have a $\Q(\zeta_7)$-rational $2$-torsion point (instead of a $2$-torsion point over $\Q$).
}
\end{comment}

\begin{proposition} \label{prop7}
The only elliptic curves $E/\Q(\zeta_7)$ with $j(E)\in \Q$, a point of order $2$ and an isogeny of degree $7$ over $\Q(\zeta_7)$ are those with $j(E)\in \{-3^3\cdot5^3, 3^3\cdot5^3\cdot17^3\}$. Furthermore, $E/\Q(\zeta_7)$ with $j(E)\in \Q$ has full $2$-torsion over $\Q(\zeta_7)$ and a $7$-isogeny if and only if we have~$j(E)=-3^3\cdot5^3.$
\end{proposition}
\begin{proof}
  For $r=7$ we compute $X_0(14)(\Q(\zeta_7))\simeq \Z/2\Z\oplus \Z/6\Z$ ($X_0(14)$ is an elliptic curve). Of these $12$ points, 6 are rational; $4$ are cusps and $2$ correspond to elliptic curves over $\Q$ with $j$-invariants $3^{3} \cdot 5^{3} \cdot 17^{3}$ and $-3^{3} \cdot 5^{3}$.  Elliptic curves with either of these $j$-invariants have a single $2$-torsion point over $\Q$, multiplication by an order of $\Q(\sqrt{-7})$, and hence a $7$-isogeny over $\Q$. Recall that an elliptic curve with at least one $2$-torsion point has full $2$-torsion over a number field $k$ if and only if its discriminant is a square in $k$ and that taking a different model of the curve changes the discriminant by a 12th power, so having $2$-torsion is model-invariant.
  Hence, to check whether they acquire full $2$-torsion over $\Q(\zeta_7)$, it remains to check whether their discriminant is a square over $\Q(\zeta_7)$. For $j(E)=-3^3\cdot5^3$, we have $\Delta(E)\in -7 \cdot (\Q^\times)^2$ and for $j(E)=3^3\cdot5^3\cdot17^3$ we have $\Delta(E)\in 7 \cdot (\Q^\times)^2$. As $-7$ is a square in $\Q(\zeta_7)$ and $7$ is not, we conclude that only elliptic curves with $j(E)=-3^3\cdot5^3$ have full $2$-torsion and a $7$-isogeny over $\Q(\zeta_7)$.



  Of the remaining $6$ points which are not rational, $4$ correspond to $j$-invariants which are not $\Q$-rational, and $2$ more points correspond to the $j$-invariant $-3^3\cdot5^3$.  The fact that there are 3 points in $X_0(14)(\Q(\zeta_7))$ corresponding to the $j$-invariant $-3^3\cdot5^3$ is explained by the fact that an elliptic curve with $j$-invariant $-3^3\cdot5^3$ has full $2$-torsion over $\Q(\zeta_7)$ and hence 3 $G_{\Q(\zeta_7)}$-invariant subgroups of order $14$. %We are only considering elliptic curves with $\Q$-rational $j$-invariants and we already know from \Cref{prop:CM} that elliptic curves with $j$-invariants $-3^35^3$ have full $2$-torsion over $\Q(\zeta_7)$ and those with $j$-invariant $j$-invariants $3^{3} \cdot 5^{3} \cdot 17^{3}$ and $-3^{3} \cdot 5^{3}$ do not.

\end{proof}




\begin{comment}
\begin{proposition}
\label{prop:red}
Let $E/ \Q$ be an elliptic curve without CM with a point of order $2$ and $r>13$ be a prime number. Then $\rhobar_{E,r}(G_\Q)=\GL_2(\F_r)$.
\end{proposition}
\begin{proof}
%By \cite[Theorem 1.1]{lemos}, the claim follows for all $r>37$. Suppose now $r\geq 17$.
By Proposition \ref{prop-zyw}, we see that if the mod $r$ representation is not surjective, then it is contained in $C^+_{ns}(r)$, unless $j(E)$ is in the exceptional set listed in Proposition \ref{prop-zyw} (and in which case none of the curve have a $2$-torsion point). By \cite[Proposition 2.1]{lemos}, it follows that the $j(E)$ is integral. Since $E$ has a $2$-torsion point over $\Q$, we have that by \cite[Theorem 1.1.]{zyw}
$$j(E)=\frac{(t+16)^3}{t} \text{ for some }t  \in \Q,$$
and furthermore one can see from \cite{lemos}
that \begin{align*}\label{j}
        j(E) \in & \{- 3^3\cdot 5^3,-2^2\cdot 7^3,-2^4\cdot 3^3,-2^6, 0, 2^7, 2^6\cdot 3^3, 2^4\cdot 5^3, 2^{11}, 2^2\cdot 3^6, 2^7\cdot 3^3, 17^3, 2^6\cdot 5^3, 2^5\cdot 7^3,  \\
        & 2^5\cdot 3^6,2^4\cdot 3^3\cdot 5^3, 2^4\cdot 17^3,2^3\cdot 31^3, 2^3\cdot 3^3\cdot 11^3, 2^2\cdot 3^6\cdot 7^3,
2^2\cdot 5^3\cdot 13^3, 2\cdot 127^3, 2\cdot 3^3\cdot 43^3,\\
& 3^3\cdot 5^3\cdot 17^3, 257^3\}.
     \end{align*}
Since the mod $r$ representation of a single quadratic twist is surjective if and only if it's surjective for all quadratic twists, it is enough to check for a single ($\Q$-isomorphism class of an) elliptic curve with each of the $j$-invariant's. We obtain that for all the non-CM $j$-invarinats in the list, all their mod $r$ representations are surjective for $r>3$. We check this in LMFDB \cite{lmfdb}, elliminating all the CM $j$-invariants. Alternatively, one can use the algorithm in \cite[Section 1.8]{zyw} to find all the non-surjective primes for each elliptic curve.
\end{proof}


\begin{proposition}
\label{13}
Let $E/ \Q$ be an elliptic curve without CM with a point of order $2$ over $\Q$. Then $\rhobar_{E,13}(G_\Q)$ is not contained in a Borel or a normalizer of a Cartan subgroup.
\end{proposition}
\begin{proof}
By \cite{bdmtv}, for a non-CM elliptic curve, $\rhobar_{E,13}(G_\Q)$ is not contained in a normalizer of a Cartan subgroup.

If $E$ had a $13$-isogeny, then it would also have to have a $26$-isogeny over $\Q$, which is impossible.
\end{proof}
\end{comment}


\begin{comment}
\begin{proposition}
\label{prop:11}
Let $E/ \Q$ be an elliptic curve without CM with a point of order $2$ over~$\Q$. Then $\rhobar_{E,11}(G_\Q)=\GL_2(\F_{11})$.
\end{proposition}
\begin{proof}
Suppose $\rhobar_{E,11}(G_\Q)\neq\GL_2(\F_{11})$. Then $\rhobar_{E,11}(G_\Q)$ is contained in either the Borel subgroup or in $C_{ns}^+(11)$. One can show that $\rhobar_{E,11}(G_\Q)\leq C_{ns}^+(11)$ is not possible by the same argument as in \Cref{prop:red}. On the other hand, if $\rhobar_{E,11}(G_\Q)$ was a subgroup of the Borel subgroup, this would imply that $E$ has a $22$-isogeny over $\Q$, which is impossible.
\end{proof}
\end{comment}



\begin{remark}
For $r=3$ and $5$ there exist infinitely many rational elliptic curves with a point of order $2$ and an isogeny of degree $r$, as the modular curves $X_0(6)$ and $X_0(10)$ have genus 0 and have rational cusps. Also, there are infinitely many elliptic curves with full $2$-torsion and reducible mod $3$ Galois representations; any elliptic curve with $\Z/2\Z \times \Z/6\Z$ torsion over $\Q$ is such an example.
\end{remark}


%%%%%%%%%%%%%%%%%%%%%
\section{Elliptic curves with full $2$-torsion and reducible mod $r$ Galois representations}
%%%%%%%%%%%%%%%%%%%%%

In the next two propositions we determine the elliptic curves with $j(E) \in \Q$ with both full $2$-torsion and an $r$-isogeny over $\Q(\zeta_r)$.

\begin{proposition}
\label{prop:CM}

Suppose that $r\geq 5$, that $E/\Q(\zeta_r)$ has CM and either $j(E)\in \Q\backslash \{0,1728\}$ or if $j(E)\in\{0,1728\}$ that $E$ is a base change of an elliptic curve defined over $\Q$. Suppose that $r$ divides the discriminant of the imaginary quadratic field containing the endomorphism ring of $E$ over $\overline \Q$. Then $E$ has full $2$-torsion and an $r$-isogeny over $\Q(\zeta_r)$ if and only if $r=7$ and $j(E)=-3^3\cdot5^3$.
\end{proposition}
\begin{proof}
We can assume that $E$ is defined over $\Q$, using for $j(E)\notin\{0,1728\}$ the same argumentation as before \Cref{prop1}.


By \Cref{prop:ired} it follows that if $E$ has an $r$-isogeny over $\Q(\zeta_r)$, then $E$ has an $r$-isogeny over $\Q$. If $\rhobar_{E,2}(G_\Q)=\GL_2(\F_2)$, then obviously $E$ does not have full $2$-torsion over $\Q(\zeta_r)$ (which is abelian over $\Q$).  Hence by \cite[Proposition 1.15]{zyw} we have
$$j(E)\in \left\{ 2^4\cdot3^3\cdot5^3, 2^3\cdot3^3\cdot 11^3, -3^3\cdot5^3, 3^3\cdot5^3\cdot17^3, 2^6\cdot 5^3\right\}.$$
By \Cref{prop:ired}, $E$ has an $r$-isogeny over $\Q$ if and only if $r$ divides $D$, where $-D$ is the discriminant of the imaginary quadratic number field containing the endomorphism ring of $E$. For $j(E)\in \left\{ 2^4\cdot3^3\cdot5^3, 2^3\cdot3^3\cdot 11^3, 2^6\cdot 5^3\right\},$ $D$ will be divisible by only $2$ or $3$ (see e.g. \cite[Table 1]{zyw}), so we need only consider elliptic curves with $j(E)=-3^3\cdot5^3$ and $3^3\cdot5^3\cdot17^3$. Now the result follows from \Cref{prop7}.

%By \cite[Proposition 1.15]{zyw} and \Cref{prop:ired}, we need only consider elliptic curves with $j(E)=-3^3\cdot5^3$ and $3^3\cdot5^3\cdot17^3$, as these are the only ones with both an $r$-isogeny over $\Q$, for $r\geq 5$, and non-surjective $\rhobar_{E,2}$.
%It turns out we necessarily have $r=7$ and

%For all such $E$, we have $\Delta(E)\in -1 \cdot (\Q^\times)^2$, so $E$ gains full $2$-torsion over $\Q(i)$ which is not a subfield of $\Q(\zeta_7)$.
\end{proof}

\begin{proposition}
Suppose that $r\geq 5$, that $E/\Q(\zeta_r)$ and either $j(E)\in \Q\backslash \{0,1728\}$ or if $j(E)\in\{0,1728\}$ that $E$ is a base change of an elliptic curve defined over $\Q$. Then if $E$ has an $r$-isogeny and full $2$-torsion over $\Q(\zeta_r)$, then $j(E)=-3^3\cdot5^3$ and $r=7$.
\label{prop:full2irred}
\end{proposition}
\begin{proof}

As before we can assume that $E$ is a defined over $\Q$, using for $j(E)\notin\{0,1728\}$ the same argumentation as before \Cref{prop1}.

As we have already dealt with elliptic curves with CM in \Cref{prop:CM}, it remains to deal with elliptic curves without CM.%it can be seen in


For $r\geq 17$, we check that for all elliptic curves with $j(E)\in  \{-17\cdot 373^3/2^{17}, -17^2\cdot 101^3/2, -7\cdot 11^3, -7\cdot 137^3\cdot 2083^3\}$, we have $\Delta(E)\in -10\cdot (\Q^\times)^2$ for the first 2 $j$-invariants, and $\Delta(E)\in -5\cdot (\Q^\times)^2$ for the last two $j$-invariants. Neither $\Q(\sqrt{-5})$ nor $\Q(\sqrt{-10})$ are contained in $\Q(\zeta_{17})$ or $\Q(\zeta_{37})$ and now the claim follows from \Cref{prop-zyw} and Lemma \ref{lem:subgroups}. \


For $r=5$ we compute that $X_0(20)(\Q(\zeta_5))=X_0(20)(\Q)$ ($X_0(20)$ is an elliptic curve). As an elliptic curve with full $2$-torsion and an $r$-isogeny would be $2$-isogenous to an elliptic curve with a $4r$-isogeny (see \cite[Lemma 7]{najman}), we are done with this case as $X_0(20)(\Q)$ consists of cusps \cite{ligozat}.

The case $r=7$ has already been dealt with in \Cref{prop7}.


For $r=11$, we compute $X_0(11)(\Q)=X_0(11)(\Q(\zeta_{11}))$, so the only elliptic curves with $11$-isogenies over $\Q(\zeta_{11})$ are those with $j(E) \in \{ -11^2, -2^{15}, -11\cdot131^3\}$. Elliptic curves with these $j$-invariants have trivial torsion over $\Q$, so cannot gain any $2$-torsion over a number field of degree not divisible by $3$.

Finally, for $r=13$, by \cite[Theorem 1.8.]{zyw}, the discussion after it and \cite[Theorem 1.1 and Theorem 1.3]{bdmtv}, we know that $\rhobar_{E,13}(G_\Q)$ for a non-CM elliptic curve has to be $\GL_2(\F_{13})$, contained in a conjugate of a Borel subgroup or conjugate to the group $G_7$ (using Zywina's notation)
$$G_7:=\left\langle\begin{pmatrix}
               2 & 0 \\
               0 & 2
             \end{pmatrix}, \begin{pmatrix}
               2 & 0 \\
               0 & 3
             \end{pmatrix}, \begin{pmatrix}
               0 & -1 \\
               1 & 0
             \end{pmatrix}, \begin{pmatrix}
               1 & 1 \\
               -1 & 1
             \end{pmatrix}\right\rangle.
             $$
We compute in Magma that the orbits of $S(G_7)$ acting on $\F_{13}^2 \backslash \{(0,0)\}$ are all of length 24, so elliptic curves with such image do not have $13$-isogenies over $\Q(\zeta_{13})$. We conclude that if a non-CM elliptic curve has a $13$-isogeny over $\Q(\zeta_{13})$, it has one already over $\Q$. As there are no $26$-isogenies over $\Q$ \cite{MazurVelu}, an elliptic curve with a $13$-isogeny over $\Q$ cannot have a 2-torsion point over $\Q$. Since $E$ gains full $2$-torsion over $\Q(\zeta_{13})$, it follows that $\Q(E[2])$ is a subfield of $\Q(\zeta_{13})$ and hence $\rhobar_{E,2}(G_\Q)$ is cyclic of order 3. But there are no such curves, which can be seen from \cite[Table A8]{DGJ} in the line [2Cn,13B]; this shows that the modular curve whose rational points parametrize elliptic curves $E/\Q$ having simultaneously $\rhobar_{E,2}(G_\Q)$ cyclic of order 3 (denoted 2Cn) and a $13$-isogeny (denoted 13B) has $2$ rational points, both of which can easily be seen to be cusps.

%and \cite[Table 11]{DGJ}.
\end{proof}

\begin{comment}
\begin{remark}
In all of the above, one can assume that $E$ is not defined over $\Q$, but instead the weaker condition $j(E)\in \Q$.
\end{remark}
\end{comment}

For $r=5$ and $7$, if we assume that $E$ is an elliptic curve over $\Q$ with full $2$-torsion over $\Q$, we can prove a stronger result, similar to the one in \Cref{prop:B1}.

\begin{proposition}
\label{prop:57}
Let $E/ \Q$ be an elliptic curve without CM with full $2$-torsion over $\Q$. Then $\rhobar_{E,r}(G_\Q)=\GL_2(\F_{r})$ for $r=5,7$.
\end{proposition}
\begin{proof}
For $r=7$, by \cite[Theorem C]{Morrow}, if $\rhobar_{E,7}(G_\Q)\neq\GL_2(\F_{7})$, then $\rhobar_{E,7}(G_\Q)$ is contained in a Borel subgroup, which would imply that $E$ has a $14$-isogeny over $\Q$, which is impossible since it does not have CM.

For $r=5$, by \cite[Theorem C]{Morrow}, if $\rhobar_{E,5}(G_\Q)\neq\GL_2(\F_{5})$, then $\rhobar_{E,5}(G_\Q) \leq G_{9,5}$, where (using the notation of \cite{Morrow})
$$G_{9,5}:=\left \langle \begin{pmatrix}
                           2 & 0 \\
                           0 & 1
                         \end{pmatrix} , \begin{pmatrix}
                           1 & 0 \\
                           0 & 2
                         \end{pmatrix}, \begin{pmatrix}
                           0 & -1 \\
                           1 & 0
                         \end{pmatrix}, \begin{pmatrix}
                           1 & 1 \\
                           1 & -1
                         \end{pmatrix}\right \rangle\leq \GL_2(\F_5).$$
The projective image of the group $G_{9,5}$ is isomorphic to $S_4$, and is denoted by $5S4$ in \cite{DGJ}.


 In \cite[Table A13]{DGJ} the line $[2Cs,5S4]$ shows that the modular curve whose rational points parametrize elliptic curves $E/\Q$ having simultaneously $\rhobar_{E,2}(G_\Q)\{I\}$ (denoted $2Cs$) and having $\rhobar_{E,5}(G_\Q)\leq G_{9,5}$ (denoted $S_4$ as mentioned above) has 3 non-cuspidal rational points, all corresponding to $j(E)=2^6\cdot3^3$, and such curves have CM.
\end{proof}

\begin{thebibliography}{10}

\bibitem{bdmtv}
Jennifer Balakrishnan, Netan Dogra, J.~Steffen M\"{u}ller, Jan Tuitman, and Jan
  Vonk.
\newblock Explicit {C}habauty-{K}im for the split {C}artan modular curve of
  level 13.
\newblock {\em Ann. of Math. (2)}, 189(3):885--944, 2019.

\bibitem{bdmtv2}
Jennifer~S. Balakrishnan, Netan Dogra, J.~Steffen M\"{u}ller, Jan Tuitman, and
  Jan Vonk.
\newblock Quadratic {C}habauty for modular curves: algorithms and examples.
\newblock {\em Compos. Math.}, 159(6):1111--1152, 2023.

\bibitem{BanwaitCremona}
Barinder\thinspace{}S. Banwait and John\thinspace{}E. Cremona.
\newblock Tetrahedral elliptic curves and the local-global principle for
  isogenies.
\newblock {\em Algebra \& Number Theory}, 8(5):1201--1229, 2014.

\bibitem{BCDF3}
Nicolas~Billerey, Imin~Chen, Luis~Dieulefait, and Nuno~Freitas.
\newblock On {D}armon's program for the generalized {F}ermat equation, {I}.
preprint.

\bibitem{Birch}
Bryan\thinspace{}J. Birch.
\newblock Some calculations of modular relations.
\newblock In {\em Modular functions of one variable, {I} ({P}roc. {I}nternat.
  {S}ummer {S}chool, {U}niv. {A}ntwerp, 1972)}, pages 175--186. Lecture Notes
  in Mathematics, Vol. 320, 1973.

\bibitem{DGJ}
Harris~B. Daniels and Enrique Gonz\'{a}lez-Jim\'{e}nez.
\newblock Serre's constant of elliptic curves over the rationals.
\newblock {\em Exp. Math.}, 31(2):518--536, 2022.

\bibitem{DarmonDuke}
Henri Darmon.
\newblock Rigid local systems, {H}ilbert modular forms, and {F}ermat's {L}ast
  {T}heorem.
\newblock {\em Duke Math. J.}, 102(3):413--449, 2000.

\bibitem{FL}
Lorenzo Furio and Davide Lombardo.
\newblock Serre's uniformity question and proper subgroups of $C_{ns}^+(p)$
\newblock preprint (2023), \url{https://arxiv.org/abs/2305.17780}

\bibitem{Le_Fourn_Lemos}
Samuel Le~Fourn and Pedro Lemos.
\newblock Residual {G}alois representations of elliptic curves with image
  contained in the normaliser of a nonsplit {C}artan.
\newblock {\em Algebra Number Theory}, 15(3):747--771, 2021.

\bibitem{lemos}
Pedro Lemos.
\newblock Serre's uniformity conjecture for elliptic curves with rational
  cyclic isogenies.
\newblock {\em Trans. Amer. Math. Soc.}, 371(1):137--146, 2019.

\bibitem{ligozat}
Gerard {Ligozat}.
\newblock {Courbes modulaires de genre 1}.
\newblock {\em {Bull. Soc. Math. Fr., Suppl., M\'em.}}, 43:80, 1975.

\bibitem{lr}
Álvaro Lozano-Robledo.
\newblock{On the field of definition of
$p$-torsion points on elliptic curves over the rationals.}
\newblock {\em Math. Ann}. 357:279–-305, 2013.

\bibitem{lmfdb}
The {LMFDB Collaboration}.
\newblock The {L}-functions and modular forms database.
\newblock \url{http://www.lmfdb.org}, 2013.

\bibitem{MazurVelu}
Barry Mazur and Jacques V\'{e}lu.
\newblock Courbes de {W}eil de conducteur {$26$}.
\newblock {\em C. R. Acad. Sci. Paris S\'{e}r. A-B}, 275:A743--A745, 1972.

\bibitem{Morrow}
Jackson\thinspace{}S. Morrow.
\newblock Composite images of {G}alois for elliptic curves over {$\mathbb{Q}$} and
  entanglement fields.
\newblock {\em Math. Comp.}, 88(319):2389--2421, 2019.

\bibitem{najman}
Filip Najman.
\newblock Torsion of rational elliptic curves over cubic fields and sporadic
  points on {$X_1(n)$}.
\newblock {\em Math. Res. Lett.}, 23(1):245--272, 2016.

\bibitem{zyw}
David {Zywina}.
\newblock {On the possible images of the mod $\ell$ representations associated
  to elliptic curves over $\mathbb{Q}$}.
\newblock ArXiv preprint,
  \href{https://arxiv.org/abs/1508.07660}{arXiv:1508.07660}, 2015.

\end{thebibliography}


\end{document}
