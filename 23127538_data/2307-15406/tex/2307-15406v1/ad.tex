
Our methods to compute derivatives of expectation values are based on
the techniques of automatic differentiation (AD). 
By AD we understand a set of techniques to determine the derivative of
a deterministic function specified by a computer program. 
There are various flavors of AD in the market, and our algorithms are
quite agnostic about the particular implementation that is used, but
in order to make the proposal and notation more concrete we are going
to choose a particular method, based on operations of polynomials
truncated at some order. The generalization of our techniques to
other flavors of AD is straightforward. 

In what follows it is useful to use multi-index notation. The
$d$-dimensional multi-index is defined by
\begin{equation}
  n = (n_1,\dots,n_d)\,.
\end{equation}
We can define a partial order for multi-indices by the condition
\begin{equation}
  n \le m \Longleftrightarrow n_i \le m_i\quad \forall i=1,\dots,d\,.
\end{equation}
We also define the absolute value, factorial and power by the relations
\begin{equation}
  |n| = \sum_{i=1}^d n_i\,,;\quad
  n! = \prod_{i=1}^d n_i!\,,\quad
  \epsilon^n = \prod_{i=1}^d \epsilon_i^{n_i}\,.
\end{equation}
Finally the higher-order partial derivative is defined by
\begin{equation}
  \frac{\partial^n}{\partial x^n} = \frac{\partial^{|n|}}{\partial x_1^{n_1}\cdots\partial x_d^{n_d}} \,.
\end{equation}

With this notation, polynomials 
of degree $p = (p_1,\dots,p_d)$ in several variables
$\{\epsilon_i\}_{i=1}^d$ are represented in the compact form
\begin{equation}
  \tilde a(\epsilon) = \sum_{n \le p} c_{n} \epsilon^{n}~.
  \label{eq:truncpol}
\end{equation}
Note that each variable $\epsilon_i$ is raised at most at the power
$p_i$ and that the index of the coefficient $c_n$ is itself a multi-index (i.e. 
$c_n = (c_{n1}, c_{n2},\dots, c_{nd})$). If the coefficients $c_{ni}$
are elements of a field (i.e.  
real numbers), the
addition/multiplication of these polynomials where terms $\mathcal
O(\epsilon_j^{n_j+1})$ are neglected form an algebra over the very same field. 

As an example calculation in this algebra, consider the following two polynomials in
two variables and with degrees $p=(2,3)$
\begin{eqnarray}
  \tilde a(\epsilon) &=& 2 + \epsilon_1 + \epsilon_2^3\,, \\
  \tilde b(\epsilon) &=& 1 + \epsilon_1 + 2\epsilon_1^2 + 3\epsilon_2^2\,,
\end{eqnarray}
we have
\begin{eqnarray}
  (\tilde a+\tilde b)(\epsilon) &=& 3 + 2\epsilon_1 + 2\epsilon_1^2 + 3\epsilon_2^2+ \epsilon_2^3\,, \\
  (\tilde a\cdot \tilde b)(\epsilon) &=& 2 + 3\epsilon_1 + 5\epsilon_1^2 + 2\epsilon_1^3 + 3\epsilon_1\epsilon_2^2 + \epsilon_1\epsilon_2^3 + 2\epsilon_1^2\epsilon_2^3 + \epsilon_2^3\,.
\end{eqnarray}
In particular it is important to note that we have dropped terms
$\propto \epsilon_1^3, \epsilon_2^5$ in the product, since $3 > p_1=2$
and $5>p_2=3$. 
Therefore the ``$=$'' sign in the above equations has to be understood as
``\emph{up to higher order corrections}''.
Elementary operations and functions acting on these polynomials can be
defined in a straightforward way (see~\cite{Haro2022}).

The connection of the algebra of truncated polynomials with AD is a
consequence of Taylor theorem. 
Let $f(x)$ be a deterministic function in the variables $x_1,\dots,x_d$. 
If each variable is promoted to a truncated polynomial
\begin{equation}
  x_i \longrightarrow \tilde x_i(\epsilon) = x_i + \epsilon_i\,,
\end{equation}
and we evaluate the function $f$ with the truncated polynomials as
input
\begin{equation}
  \tilde f(\epsilon) = f(\tilde x(\epsilon)) = \sum_{n\le p} f_n\epsilon^n\,, 
\end{equation}
it is easy to see that the result is a polynomial that is equal to
the Taylor series of $f$ at $x$. 
In particular partial derivatives of the function are obtained by the
relation 
\begin{equation}
  f_n = \frac{1}{n!} \frac{\partial^n f}{\partial x^n}~.
\end{equation}

Note that the analogy with the Taylor expansion is just at the level of the coefficients $f_n$, which are obtained automatically when writing functions of truncated polynomials $\tilde x_i(\epsilon)$, while $\epsilon$ is exclusively a symbolic quantity.


%%% Local Variables:
%%% mode: latex
%%% TeX-master: "paper"
%%% End:
