
In this section we explore in detail some applications in lattice field theory. 
We will use as model the $\lambda-\phi^4$ theory in 4 space-time dimensions. 
In the continuum the Euclidean action of this theory is given by
\begin{equation}
  S(\phi; m, \lambda) = \int {\rm d} x^4 \left\{  \frac{1}{2}(\partial_\mu\phi)^2 + \frac{m^2}{2}\phi^2 + \lambda\phi^4 \right\}\,.
\end{equation}
The discretized version of this action is
\begin{equation}
  \label{eq:Slatt}
  S_{\rm latt}(\hat \phi;\hat m,\lambda) = \sum_x \left\{  \frac{1}{2}\sum_\mu[\hat\phi(x+\hat\mu) - \hat\phi(x)]^2 + \frac{\hat m^2}{2}\hat\phi^2(x) + \lambda\hat\phi^4(x) \right\}  
\end{equation}
where dimension-full quantities have been scaled with appropriate
powers of the lattice spacing $a$ in order to render all quantities
dimensionless
\begin{eqnarray}
  \hat \phi &=& a\phi\,,\\
  \hat m &=& am\,.
\end{eqnarray}

In field theory one is interested in correlation functions, given as
expectation values over the Euclidean partition function
\begin{equation}
  \mathcal Z = \int \prod_x{\rm d} \hat\phi(x) e^{-S_{\rm latt}(\hat\phi;\hat m, \lambda)}\,.
\end{equation}
These expectation values depend on the parameters $\hat m,\lambda$. 
The methods described in sections~\ref{sec:reweighting}
and~\ref{sec:nspt} can be applied to determine the dependence of
correlation functions with these parameters.
In particular we note that for non-negative $\hat m^2$ the potential
given by the action of eq.~(\ref{eq:Slatt}) is convex, guaranteeing
the convergence of the Hamiltonian perturbative expansion (see section~\ref{sec:impl-conv}). 

We have performed simulations on a $L^4$ lattice with $L/a=32,48$ for several
values of the parameter $\lambda$ and $\hat m^2 = 0.05$. 
As example observables, we will consider some simple local quantities:
$\langle \hat\phi^2(x) \rangle, \langle \hat\phi^4(x) \rangle$, as well as the
action density
\begin{equation}
  \label{eq:act}
  \langle \hat s(x) \rangle =  \frac{1}{2}\langle [\hat\phi(x+\hat\mu) - \hat\phi(x)]^2\rangle + \frac{\hat m^2}{2}\langle \hat\phi^2(x)\rangle + \lambda\langle \hat\phi^4(x)  \rangle\,.
\end{equation}
Since we perform our simulations with periodic boundary conditions,
invariance under translations ensures that these local expectation
values are independent on the point measured $x$. 
In order to get better precision we perform volume averages for the
estimations (\textit{e.g.}
$ \langle \tilde s \rangle = \left(\frac{a}{L}\right)^4 \sum_x \langle \tilde s(x) \rangle$).

Table~\ref{tab:lp4} shows the results of both the Hamiltonian and the
reweighting approach applied to the determination of the derivatives
$\partial/\partial m^2$, $\partial/\partial \lambda$,
$\partial^2/\partial m^2\partial\lambda$ of the observables $\langle
\phi^2 \rangle,\langle \phi^4 \rangle, \langle \hat s \rangle$ (see eq.~(\ref{eq:act})).


\begin{landscape}
\begin{table}
  \centering
  \begin{tabular}{lllllllll}
    \toprule
    &&&\multicolumn{6}{c}{$\lambda$} \\
    \cline{4-9}
    &&&0.0&0.1&0.2&0.3&0.4&0.5\\
    \midrule
    \multirow{6}*{$\langle \phi^2 \rangle$}&\multirow{2}*{$\partial_{\hat m^2}$}
     & RW &     -0.0428(20)&     -0.0328(14)&     -0.0270(13)&     -0.0241(12)&     -0.0220(11)&    -0.01974(91) \\
    && HAD &   -0.042526(41)&   -0.030880(14)&   -0.026273(10)&  -0.0233672(82)&  -0.0212721(72)&  -0.0196387(60) \\
    \cline{2-9}
    &\multirow{2}*{$\partial_{\lambda}$}
     & RW &     -0.0779(22)&    -0.05227(94)&    -0.04370(89)&    -0.03534(61)&    -0.03169(50)&    -0.02754(49) \\
    && HAD &   -0.077816(79)&   -0.052499(24)&   -0.042218(19)&   -0.035830(14)&   -0.031323(11)&  -0.0278909(93) \\
    \cline{2-9}
    &\multirow{2}*{$\partial^2_{\hat m^2,\lambda}$}
     & RW &        0.43(43)&        0.03(16)&        0.16(14)&       -0.10(11)&       0.116(77)&      -0.024(69) \\
    && HAD &      0.2733(22)&    0.061593(99)&    0.035082(69)&    0.024240(42)&    0.018263(31)&    0.014553(31) \\
    \midrule
    \multirow{6}*{$\langle \phi^4 \rangle$}&\multirow{2}*{$\partial_{\hat m^2}$}
     & RW &     -0.0391(20)&     -0.0272(12)&     -0.0223(11)&    -0.01809(90)&    -0.01645(86)&    -0.01398(70) \\
    && HAD &   -0.038919(39)&   -0.026247(13)&  -0.0211084(97)&  -0.0179118(73)&  -0.0156615(62)&  -0.0139464(46) \\
    \cline{2-9}
    &\multirow{2}*{$\partial_{\lambda}$}
     & RW &     -0.0844(24)&     -0.0539(11)&    -0.04281(92)&    -0.03340(54)&    -0.02850(43)&    -0.02428(41) \\
    && HAD &   -0.084330(78)&   -0.054229(26)&   -0.041514(19)&   -0.033715(14)&   -0.028357(11)&  -0.0243679(73) \\
    \cline{2-9}
    &\multirow{2}*{$\partial^2_{\hat m^2,\lambda}$}
     & RW &        0.41(44)&        0.01(16)&        0.11(14)&      -0.083(90)&       0.089(61)&      -0.000(56) \\
    && HAD &      0.2848(21)&    0.068858(94)&    0.038917(70)&    0.026391(38)&    0.019393(29)&    0.015080(25) \\
    \midrule
    \multirow{6}*{$\langle s \rangle$}&\multirow{2}*{$\partial_{\hat m^2}$}
     & RW &         -0.0025(42)&          -0.0006(34)&      0.0027(35)&      0.0028(36)&      0.0057(32)&      0.0063(30) \\
    && HAD &   -0.000003(22)&    0.002623(16)&    0.004218(20)&    0.005397(14)&    0.006265(16)&    0.006989(15) \\
    \cline{2-9}
    &\multirow{2}*{$\partial_{\lambda}$}
     & RW &         -0.0686(48)&          -0.0567(26)&     -0.0538(25)&     -0.0447(23)&     -0.0400(17)&     -0.0343(17) \\
    && HAD &   -0.069774(49)&   -0.057738(34)&   -0.050128(40)&   -0.044530(27)&   -0.040250(27)&   -0.036721(26) \\
    \cline{2-9}
    &\multirow{2}*{$\partial^2_{\hat m^2,\lambda}$}
     & RW &             1.1(1.0)&        -0.16(39)&        0.36(43)&       -0.43(32)&        0.16(26)&       -0.15(21) \\
    && HAD &    0.038864(96)&    0.019197(66)&    0.013405(69)&    0.010126(50)&    0.007860(47)&    0.006407(48) \\
    \midrule
    \bottomrule
  \end{tabular}
  \caption{Derivatives with respect to $\hat m^2$ ($\partial_{\hat
      m^2}$), $\lambda$ ($\partial_\lambda$) and the cross derivative 
  ($\partial^2_{\hat m^2,\lambda}$) of different observables. 
Note that the action density $s$ has an explicit dependence on
$\hat m^2,\lambda$ (see Eq.~(\ref{eq:act})). 
All simulations are performed on a $L^4$ lattice with $L/a=32$ and
$\hat m^2 = 0.05$.}
  \label{tab:lp4}
\end{table}
\end{landscape}







%%% Local Variables:
%%% mode: latex
%%% TeX-master: "paper"
%%% End:


It is apparent that results obtained with the
Hamiltonian approach are much more precise (both results use exactly
the same statistics), usually about 100 times more precise for the
first derivatives. This difference in precision is even more clear for
the higher orders: the signal for the cross derivative $\partial^2_{\hat
  m^2,\lambda}$ is completely lost in the reweighting approach,
whereas the Hamiltonian approach determines its value with a precision
better than a 1\%. 
This can
be understood from a general point of 
view by noting that the reweighting approach requires to evaluate the
so called \emph{disconnected} contributions. 
  For example, the leading order derivative $\partial_{\hat m^2}
  \langle \phi^2 \rangle$ as determined by the reweighting approach is
  given by
  \begin{equation}
    \partial_{\hat m^2}
    \langle \hat \phi^2 \rangle = \langle \hat \phi^2 (\partial_{\hat m^2}S_{\rm latt})\rangle
    - \langle \hat \phi^2 \rangle \langle\partial_{\hat m^2}S_{\rm latt}\rangle
  \end{equation}
These disconnected contributions are known to suffer from a large
variance. As explained in section~\ref{sec:hamilt-appr-repar}
  the Hamiltonian approach completely avoids estimating such
  disconnected contributions. 
  The Hamiltonian approach implements an exact version
  of the reparametrization trick, where the field variables $\tilde
  \phi(x)$ (and its dependence with the relevant parameters) are
  determined to any order so that all terms in the Taylor series are
  determined \emph{as connected contributions}.

We conclude that it is the absence of disconnected terms in the
Hamiltonian approach what lies at the heart of the differences in variances
between both approaches.  



%%% Local Variables:
%%% mode: latex
%%% TeX-master: "paper"
%%% End:
