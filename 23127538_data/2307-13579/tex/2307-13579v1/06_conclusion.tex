\section{Conclusion}
In this paper, we accomplished four main objectives: 1. We examined how SA can bridge the well-studied domains of classification and regression. 2. We proposed a novel loss function. 3. We present the first DL model that provably produces survival curves in continuous time and is also a universal approximator. 4. We suggested a more nuanced evaluation of survival models by modifying the integrated Brier score with classifier metrics.

We also retrofitted our new models to the well-established Cox model and its time-dependent version. Further we conducted a comprehensive comparison of recent and traditional methods. Our novel BCE loss outperforms the SuMo loss and the classical setup on classical survival tasks. In contrast, the SuMo loss outperforms on reinterpreted regression tasks where no censoring is present.

Our work presents the potential for more powerful survival models in fields such as clinical data, engineering, economics, and sociology. Therefore we hope our reinterpretation and stronger models -- applicable to modalities like images -- open up SA to a broader audience.

While new DL approaches offer promising results, the quality of uncertainty estimation remains an open area of research. For example, our BCE loss is influenced by $\sigma$ and distributions $\mathcal{T}_\pm$ in~\eqref{eq:T_minus} and \eqref{eq:T_plus}, which can affect the final model. However, this can also be an opportunity to adapt the model's predictions based on the specific application and evaluation criteria.
Finally, we want to point out that any survival model produces biased predictions if the censoring assumption, usually non-informative censoring, is violated.
