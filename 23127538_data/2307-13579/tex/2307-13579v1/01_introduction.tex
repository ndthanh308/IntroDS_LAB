\section{Introduction}
Survival analysis (SA) is a well-established branch of statistics~\cite{cox2018analysis, kalbfleisch2011statistical, kleinbaum2012survival, yang2011oasis, han2016oasis}. Nevertheless, it has only recently been encountered in the world of machine learning and, even more recently, deep learning (DL)~\cite{lee2015bflcrm, kong2018flcrm, lee2018deephit, giunchiglia2018rnn, ching2018cox, yang2018spatio}; see~\cite{wang2019machine, sonabend2021theoretical} for an overview on classical machine learning for SA. As the DL community has only just started focusing on SA, many fundamental questions remain unsettled.

This paper aims to provide the tools to make SA as valuable for the DL community as possible through four main contributions. 1. We examine how SA can bridge the well-studied domains of classification and regression. 2. We propose a novel loss function for SA in the continuous-time setting. 3. We present the first DL model that provably produces survival curves in continuous time without the need for numeric integration methods and is also a universal approximator. 4. We propose a modified integrated Brier score incorporating classifier metrics for a more nuanced evaluation of highly expressive survival models.

We hope our contributions will enable a more flexible and widespread application of SA in DL. As one can apply SA to predict death, but also the occurrence of any event, we see great potential for deeper integration of the two subjects. For example, there is a significant interest in applying SA to a variety of data modalities, e.g., omics data~\cite{zhang2022survbenchmark} and imaging data~\cite{shahin2022survival}. Particularly since 2020, with its utilization during the COVID-19 pandemic~\cite{wiegand2022development,crooks2020predicting,schwab2021real}, there has been increased interest in the adoption of deep learning for SA. 

We will now describe the structure of the paper and its relation to previous work on the subject.

In Section~\ref{sec:reinterpret}, we examine the relationship between SA, classification, and regression. This connection has been partially examined in prior works such as~\cite{zhong2019survival}, but not yet thoroughly. Other classification-inspired works have adapted binary cross-entropy losses for SA~\cite{lee2018deephit, ren2019deep}. Our approach differs by formulating a loss function for continuous time settings, leveraging a probabilistic perspective.

In Section~\ref{sec:universal_and_exclusive}, we first extend the monotonically increasing MONDE networks~\cite{chilinski2020neural} by improving their ability to model complex survival curves.
Secondly, we modify the SuMo networks~\cite{rindt2022survival} to not only enable universal approximation but to guarantee that any output is a survival curve. This modification also brings the SuMo network's formulation closer to one of classical survival models.

In Section~\ref{sec:cox_main_section}, we implement a neural network version of the well-known Cox model and its time-dependent variant using methods from Section~\ref{sec:universal_and_exclusive}. This synthesis enables flexible training setups for Cox models and, especially for the time-dependent model, combines the excellent approximation properties of neural nets with the interpretability of Cox models. The papers~\cite{katzman2018deepsurv, shahin2022survival, kvamme2019time} are closest to this part of our work. But, in contrast, these papers replace the Cox model's linear regression with a neural network, not their baseline function.

In Section~\ref{sec:numerics}, we provide a large-scale comparison of different survival models trained with different loss functions. We use seven datasets, one includes image data, and two are originally regression datasets, demonstrating the applicability of survival models to regression tasks. We choose these datasets for their diversity and to analyze our methodology on real-life datasets.
Here we also introduce the modified integrated Brier scores~\cite{graf1999assessment}. Along with the established concordance index, they open up a rich pool of well-known quality measures, enabling a nuanced exploration of different aspects of a given model.
