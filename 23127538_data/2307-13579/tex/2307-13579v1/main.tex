\documentclass[twoside,11pt]{article}
%\usepackage{blindtext}

% Any additional packages needed should be included after jmlr2e.
% Note that jmlr2e.sty includes epsfig, amssymb, natbib and graphicx,
% and defines many common macros, such as 'proof' and 'example'.
%
% It also sets the bibliographystyle to plainnat; for more information on
% natbib citation styles, see the natbib documentation, a copy of which
% is archived at http://www.jmlr.org/format/natbib.pdf

% Available options for package jmlr2e are:
%
%   - abbrvbib : use abbrvnat for the bibliography style
%   - nohyperref : do not load the hyperref package
%   - preprint : remove JMLR specific information from the template,
%         useful for example for posting to preprint servers.
%
% Example of using the package with custom options:
%
% \usepackage[abbrvbib, preprint]{jmlr2e}

\usepackage{jmlr2e}

\usepackage{amsmath,amsfonts}
\usepackage{algorithmic}
\usepackage{array}
\usepackage[caption=false,font=normalsize,labelfont=sf,textfont=sf]{subfig}
\usepackage{textcomp}
\usepackage{stfloats}
\usepackage{url}
\usepackage{verbatim}
\usepackage{graphicx}

\usepackage{mathrsfs}
% \usepackage{fontspec}
\usepackage{placeins}
\usepackage[norule]{footmisc}


% Definitions of handy macros can go here

\newcommand{\mr}[1]{\textcolor{green}{Mike: #1}}
\newcommand{\ja}[1]{\textcolor{red}{John A: #1}}
\newcommand{\sd}[1]{\textcolor{blue}{S\"oren: #1}}

%\newtheorem{theorem}{Theorem}[section]
%\newtheorem{corollary}{Corollary}[theorem]
%\newtheorem{lemma}[theorem]{Lemma}

\newcommand{\clock}{
 {\mathchoice
  {% Figure removed}
  {% Figure removed}
  {% Figure removed}
  {% Figure removed}
 }
}


\newcommand{\cOx}{\text{C$\clock$x}}
%\newcommand{\cOx}{\text{Cox(t)}}

% Heading arguments are {volume}{year}{pages}{date submitted}{date published}{paper id}{author-full-names}

\usepackage{lastpage}
\jmlrheading{??}{2023}{1-\pageref{LastPage}}{?/??; Revised ?/??}{?/??}{??-0000}{S\"oren Dittmer, Michael Roberts, Jacobus Preller, AIX COVNET, James H.F. Rudd, John A.D. Aston, and Carola-Bibiane Sch\"onlieb}

% Short headings should be running head and authors last names

\ShortHeadings{Reinterpreting survival analysis in the universal approximator age}{Reinterpreting survival analysis in the universal approximator age}
\firstpageno{1}

\begin{document}
\title{Reinterpreting Survival Analysis in the Universal Approximator Age}

\author{
    \name S\"oren Dittmer \email sd870@cam.ac.uk \\
    \addr Cambridge Image Analysis Group,
    University of Cambridge, UK\\
    Center for Industrial Mathematics,
    University of Bremen, Germany
\AND
    \name Michael Roberts \email mr808@cam.ac.uk \\
    \addr Cambridge Image Analysis Group,
    University of Cambridge, UK\\
\AND
    \name Jacobus Preller \email jacobus.preller1@nhs.net \\
    \addr Department of Medicine,
    University of Cambridge, UK\\
\AND
    \name AIX COVNET \\%\email TODO@TODO \\
    \addr Cambridge Image Analysis Group,
    University of Cambridge, UK\\
\AND
    \name James H.F. Rudd \email jhfr2@cam.ac.uk \\
    \addr Department of Medicine,
    University of Cambridge, UK\\
\AND
    \name John A.D. Aston \email jada2@cam.ac.uk \\
    \addr Statistical Laboratory,
    University of Cambridge, UK\\
\AND
    \name Carola-Bibiane Sch\"onlieb \email cbs31@cam.ac.uk \\
    \addr Cambridge Image Analysis Group,
    University of Cambridge, UK\\
}

% \editor{My editor}

\maketitle

\begin{abstract}%   <- trailing '%' for backward compatibility of .sty file
Survival analysis is an integral part of the statistical toolbox. However, while most domains of classical statistics have embraced deep learning, survival analysis only recently gained some minor attention from the deep learning community. This recent development is likely in part motivated by the COVID-19 pandemic. We aim to provide the tools needed to fully harness the potential of survival analysis in deep learning. On the one hand, we discuss how survival analysis connects to classification and regression. On the other hand, we provide technical tools. We provide a new loss function, evaluation metrics, and the first universal approximating network that provably produces survival curves without numeric integration. We show that the loss function and model outperform other approaches using a large numerical study.
\end{abstract}

\begin{keywords}
Deep learning, Survival analysis, Classification, Regression
\end{keywords}
\let\thefootnote\relax\footnotetext{Code available at \url{https://github.com/sdittmer/survival_analysis_sumo_plus_plus}}

% !TEX root = ../AttackGraphBasedRiskAnalysis.tex
% !TEX spellcheck = en_US
% !TEX encoding = UTF-8 Unicode

\section{Introduction}\label{sec: intro}

Traditional cities are becoming smarter. 
One of the core smart city concepts is smart mobility, which has attracted considerable attention from security researchers due to the emergence of smart vehicles and V2X communication that have given rise to novel cybersecurity threats.

Over the last decade, several trends have contributed to the automotive and railway threat landscape. 
First, sophisticated features in smart vehicles come with a higher volume of lines of code, aggravating testability and auditing and increasing the likelihood and severity of vulnerabilities. 
Second, (wireless) communication interfaces in smart vehicles come with a higher volume of external peripheral devices that can connect to smart vehicles, hence increasing the attackers' access point options, and also with a higher volume of connections, hence increasing the risk of malicious interactions. 
Finally, a higher volume of connections between smart vehicles comes with a higher volume of exchanged data, which in most cases is personal and, therefore, immensely valuable. In other words, more data is generated and needs to be considered and protected.

Graphical security modeling is a widely-used and well-established approach for representing and analyzing threat landscapes that examine vulnerabilities of systems and organizations. 
One of the primary strengths of graphical security models is that they allow for the inclusion of user-friendly visual elements with formal semantics and algorithms, enabling both qualitative and quantitative analyses. 
Over the last couple of decades, security researchers have been progressively focusing on graphical security modeling, which has gradually evolved into a valuable tool for the assessment of risks in real-life systems, such as automotive and railway environments.

Threat landscapes include (1) malicious actions of an attacker, whose goal is to harm or damage one or more assets of a system or organization, and (2) countermeasures for either preventing or mitigating such malicious actions. 
The first \emph{tree-based approach} for graphical security modeling was the \emph{threat logic trees}, which was introduced by Weiss in 1991~\cite{weiss1991}, thereby motivating the development of several subsequent frameworks, such as attack trees, which are still considered one of the most important and favored tools for the assessment of risks to date.

In all tree-based approaches, the modeling process begins with identifying a feared event, which is shown as a root node, and continues with the refinement of the attack steps, resulting in a tree model.
However, tree structures are limited to only one path between a pair of nodes. 
In other words, with tree structures, each refined node can only have one parent node. 
This limitation is addressed by the \emph{directed acyclic graph (DAG) structure}, which enables refined nodes to have multiple parent nodes. 
As a result, DAG structures can provide a higher level of detail, but they can also come with a higher level of complexity, which can nevertheless be dealt with modularization, thereby allowing the model to be subdivided into loosely-coupled, independent, and interchangeable parts that can be studied individually and in parallel. 
Finally, while the one-to-many relationship between nodes in tree structures results in a linear analysis of the threat landscape, the many-to-many relationship between nodes in DAG structures can theoretically result in an exponential analysis.
However, the complexity is kept small in practice due to the acyclic structure, and the threat landscape analysis is eventually possible.

Ensuring the security of systems is not a static process that is over after going through once.
The conditions are constantly changing, on the one hand attackers and their capabilities are evolving, and on the other hand, systems themselves are being extended and evolving.
To effectively perform the necessary continuous security management, it is necessary to know not just the threat landscape but to be able to understand the consequences and impacts if attacks are performed successfully.
Hence, it is necessary to continuously perform a risk analysis to identify the potential exposure.
Nowadays, risk management is primarily done using large tables filled with a lot of information and use cases.
Large tables only offer limited visibility, as it is challenging to maintain a comprehensive overview of risks.
With numerous rows and columns, it becomes difficult to identify trends and patterns or prioritize risks effectively.
Furthermore, managing risk can be a tedious and time-consuming process.
Updating and maintaining tables with evolving risks and mitigation measures can require significant effort, especially when dealing with a complex system or multiple risk factors.
This gets even harder when dealing with large tables that often fail to provide the necessary context and connections between different risks.
Additionally, analyzing and interpreting data from large tables can be daunting. 
It may require specialized tools or skills to extract meaningful insights from the extensive amount of information presented in the table format.
Large tables may further lack the flexibility to accommodate changing risk scenarios or evolving requirements. 
Modifying or updating the table structure to incorporate new risks or factors can be cumbersome and may hinder agility in risk management.
With numerous cells and data entries, there is also an increased risk of errors, inaccuracies, or inconsistencies in the large table. 
These issues can undermine the reliability and integrity of the risk management process.

We propose a graphical solution for the risk management process to mitigate these disadvantages of tables.
A visual representation can enhance the understanding and communication of complex risk information and make it easier to identify patterns, trends, and relationships among risks, facilitating effective decision-making.
Complex risk data is further simplified by presenting it in a clear and concise manner.
Understanding  the relationships, dependencies, and interactions between various risk elements is necessary to understand the overall risk landscape.
Visual representations of the entire risk landscape provide this overview, allowing for the identification of interdependencies, hotspots, or areas of high vulnerability.
Graphical solutions can also aid in developing and evaluating risk mitigation strategies. 
By visually representing the potential consequences and effectiveness of different mitigation measures, decision-makers can make more informed choices and allocate resources more efficiently.
Furthermore, it allows for the exploration of different risk scenarios. 
By manipulating variables or parameters within the visual representation, it becomes possible to assess the potential impact of various risk factors and evaluate the effectiveness of different response strategies.
Additionally, as graphical solutions can be more adaptable to changing requirements and evolving risks, they allow for easier updates and modifications, enabling risk management processes to be more responsive and agile.

Consequently, we believe a graphical solution for the risk assessment process improves the maintenance of risk scenarios and facilitates accessibility to different stakeholders, including non-technical audiences.
However, the existing graphical solutions are momentarily used to describe the threat landscape.
Which, of course, is helpful for the risk management process but not sufficient to represent the entire risk management process.
Therefore, motivating us to define a new graphical method for risk assessment by extending existing graphical methods for depicting the threat landscape.
Besides ways to depict attack vectors, their probability, and countermeasures, our method includes a way to depict the consequences of attack vectors and the impact level, enabling us to calculate a risk value.

The remainder of the paper is structured as follows:
After the introduction,~\cref{sec: related work} discusses the related work.
Our definition of attack graphs is given in~\cref{sec: attack graphs}.
The necessary adjustments to use these attack graphs are presented in~\cref{sec: attack graph risk assessment}, including an example of how the risk assessment is performed in our project.
\cref{sec: applicability of attack graphs to risk management standards} validates our defined method by combing it with the risk assessment processes of ISO/SAE 21434~\cite{21434} and CLC/TS 50701~\cite{50701} respectively.
The scalability and practicality are evaluated in~\cref{sec: evaluation}.
Finally,~\cref{sec: conclusion} concludes this paper.
\section{Reinterpreting survival analysis}
\label{sec:reinterpret}
We now discuss how SA can be a portal between classification and regression.

To set the stage, we introduce some key notation. We describe a potentially right-censored survival data sample via
\begin{equation}
    \label{eq:sample}
    (x, e, T) \in \mathbb{R}^n \times \{0, 1\} \times \mathbb{R}_{>0},
\end{equation}
where $x\in\mathbb{R}^n$ denotes input features, $e\in\{0,1\}$ indicates if the event was observed (1) or not (0), and $T$ is the time of the event or the time of right-censoring. We set $T>0$ as one usually assumes that no events occur for $T\le0$.

The goal of SA is to use a set of such samples to create a model that uses the features $x$ to predict the probability of a singular event, such as death, occurring after any given time $t\ge0$. One assumes that the event has not occurred at $t=0$ and is irreversible. Hence, one aims to produce a non-negative, monotonically decreasing curve being $1$ at time $t=0$. Throughout this paper, we will use the terms ``alive'' and ``dead'' to refer to the state before and after the singular event.

\subsection{Survival analysis is classification (with infinitely many classifiers)}
\label{sec:sa_as_classification}
We will pose SA in terms of classification. We begin by recalling the general definition of a survival curve:
\begin{equation}
    \label{eq:S}
    S(t|x): \mathbb{R}_{\ge0}\times\mathbb{R}^n \mapsto \mathbb{P}(\mbox{alive}@t |x) \in [0, 1]
\end{equation}
That is, $S(t|x)$ gives us the probability that the event has not occurred before or at time $t$, given the features $x$.

An alternative way of reading this definition is that $S$ is equivalent to a set of probabilistic binary classifiers
\begin{equation}
    \{S(t|\cdot) = C_t:\mathbb{R}^n\to[0,1]:t\in\mathbb{R}_{\ge0}\},
\end{equation}
each $C_t(x)$ being the probability of surviving the interval $[0, t]$ given the features $x$. Phrased differently, \textit{a survival model is simply an infinite collection of probabilistic classifiers indexed by time}.

Using the classifier perspective, we will derive a new loss function for survival curves. Considering the ideal scenario of having unlimited clean data and a highly capable probabilistic classifier, we can train one classifier for each point in time. This transforms the problem into a prediction of Bernoulli variables, and thus each classifier can be trained using the binary cross entropy (BCE) loss.

As a reminder, the standard BCE loss for a classifier $C:\mathbb{R}^n\to[0,1]$ and a sample $(x, y)\in\mathbb{R}^n\times\{0,1\}$ is defined as
\begin{equation}
    \label{eq:standard_bce}
    -y\log C(x) - (1-y)\log\left(1 - C(x)\right).
\end{equation}

Returning to the SA setting, if we have a censored sample, $(x, 0, T)$, we can use it to train all classifiers before the censoring,
\begin{equation}
    \label{eq:set_early_classifiers}
    \{C_t:\mathbb{R}^n\to[0,1]:t\in[0, T)\},
\end{equation}
via the BCE part
\begin{equation}
    \label{eq:loss_early_classifiers}
    -\log C_t(x),
\end{equation}
i.e., for censored samples, we want to find classifiers that predict $0$ (``alive'') for times before the censoring.

Otherwise, if we have an uncensored sample, $(x, 1, T)$, we can use the loss in~\eqref{eq:loss_early_classifiers} for the early (``premortem'') classifiers given by~\eqref{eq:set_early_classifiers} and train the later (``postmortem'') classifiers given by
\begin{equation}
    \label{eq:set_later_classifiers}
    \{C_t:\mathbb{R}^n\to[0,1]:t\ge T\}
\end{equation}
via the BCE part
\begin{equation}
    \label{eq:loss_later_classifiers}
    -\log(1 - C_t(x)).
\end{equation}

We can formulate a joint loss for all classifiers by choosing two arbitrary random variables, $\mathcal{T}_-(e, T)$ and $\mathcal{T}_+(e, T)$, with an everywhere-supported density over $[0, T]$ and $[T, \infty)$ respectively.
We define this loss by joining~\eqref{eq:loss_early_classifiers} and~\eqref{eq:loss_later_classifiers} to
\begin{align}
\label{eq:bce_loss}
\begin{split}
    L_\text{BCE} = -[
                   & \mathbb{E}_{t_-\sim \mathcal{T}_-}\log S(t_-|x) \\
    + e \, \cdot\, & \mathbb{E}_{t_+\sim \mathcal{T}_+}\log \left(1 - S(t_+|x)\right)
    ].
\end{split}
\end{align}
The loss trains a separate classifier $S(t|\cdot)=C_t(x)$ for every point in time $t\in[0,\infty)$, as both $\mathcal{T}_\pm$ are supported everywhere. We can consider the loss as a weighted integral over the BCE loss. As a result, it inherits many desirable properties of the BCE loss, such as being a strictly proper scoring rule~\cite{du2021beyond}.

In practice, assuming an infinite amount of clean data is unrealistic. However, we have two advantages when dealing with commonly found un- and right-censored data. Firstly, earlier classifiers can use most data points (as little censoring has occurred yet). Secondly, as survival curves decrease over time, later classifiers' predictions are upper bounded by earlier classifiers. To take advantage of this, we need to ensure the monotonicity of our predicted survival curves, which we will discuss in Section~\ref{sec:universal_and_exclusive}.

If the monotonicity holds, we also do not require that the density of $\mathcal{T}_\pm$ is supported everywhere, as later classifiers provide lower bounds for earlier ones. In practice, we therefore do the following: We sample times from a Gaussian centered at $T$ for uncensored samples and only from the left side of that Gaussian for censored samples. We ensure no negative time samples via projection with ReLU. The idea is that while one trains for all $t$, the Gaussian focuses the training effort on values around $T$; here, the variance $\sigma^2$ is a hyperparameter that controls the focus. Other choices are plausible, but we leave this to future work. Formally, we can represent this as
\begin{equation}
    \label{eq:T_minus}
    \mathcal{T}_-(e, T) = 
    \begin{cases}
        \delta(T), & \text{if $e=0$,}\\
        \mbox{ReLU}_\#\mathcal{HN}_-(T, \sigma^2), & \text{otherwise},
  \end{cases}
\end{equation}
and
\begin{equation}
    \label{eq:T_plus}
    \mathcal{T}_+(e, T) = \mathcal{HN}_+(T, \sigma^2),
\end{equation}
where $\delta$ denotes the delta-distribution, $\mathcal{HN}_\pm$ the left- and right-sided half-normal distribution, and $\mbox{ReLU}_\#$ the pushforward operator of the projection into the non-negative real numbers. 

During training, we approximate the expectations in~\eqref{eq:bce_loss} by sampling only one $t$. If $e=0$ we sample $t\sim\mathcal{T}_-$, if $e=1$, we sample with a 50\% probability from $\mathcal{T}_-$ otherwise from $\mathcal{T}_+$.

As a comparison, we use the SuMo loss~\cite{rindt2022survival}
\begin{equation}
    L_\text{SuMo} = -\left[e \log f(t|x) + (1 - e) \log S(t|x)\right],
\end{equation}
where
\begin{equation}
    f(t|x) = - \partial_t S(t|x).
\end{equation}
Both losses can be seen as log-likelihoods and are similar in their $e=1$ part but differ in their $e=0$ part. This is because SuMo's likelihood is conditioned on the event variable $e$, while the BCE's is not.

\subsection{Survival analysis is regression (with uncertainty estimation)}\label{sec:sa_as_regression}
We will now discuss how we can use a survival model to predict the full posterior of $\mathbb{R}^n\to\mathbb{R}_{\ge0}$ regression problems. The idea is similar to the motivation of the paper~\cite{chilinski2020neural}. Using~\eqref{eq:S}, we can express the lifetime distribution function (probability of the event has not occurred yet) as 
\begin{equation}
    F(t|x) = \mathbb{P}(\mbox{dead}@t|x) = 1 - S(t|x).
\end{equation}
Consequently, we can interpret $f(t|x)$ as an event density.

We now want to make a simple but powerful observation. We can treat any dataset with non-negative scalar labels as an uncensored survival dataset. Consequently, the event density $f$ of a survival model that fits the dataset well (see Section~\ref{sec:universal_and_exclusive}) is the full posterior of the original regression problem.

Note that this setup also allows one to train regression models on samples for which one only knows a lower bound on the label.

\subsection{If in doubt: survival analysis}
SA can be viewed from both a classification and regression perspective. Understanding this duality is beneficial when selecting a model or interpreting results. Any one-dimensional regression problem or binary classification involving a threshold can benefit from this perspective by rephrasing the question one asks. For example, the clocks dataset~\cite{clocks} provides images of analog clocks with the displayed times as labels; see Figure~\ref{fig:clock} for two samples. We can interpret the dataset as asking: ``what time does the clock show?''; this is a regression problem. Alternatively, we could ask for any point in time: ``is it earlier than the time the clock shows?''; this subtle shift in question turns the regression into a classification problem. Training a survival model instead of a regression or classification model can answer both questions and many others.

The ability of survival models to answer various questions at inference time is a strong advantage (particularly those not considered during training). Another benefit of framing a problem as a survival problem, where possible, is that the training process receives more information per label. While classification uses binary information about which side of a threshold the value is on, SA uses the value itself.
% Figure environment removed
\section{A universal and exclusive survival curve approximator}
\label{sec:universal_and_exclusive}
We will now introduce MONDE+, a version of MONDE~\cite{chilinski2020neural} more capable of modeling complex survival curves, and SuMo++, an improved version of SuMo~\cite{rindt2022survival} that now guarantees $S(0|x)=1$ and uses MONDE+. This makes the model not only a universal but also an exclusive approximator of survival curves, by which we mean that the model is capable of approximating any survival curve and incapable of producing anything but a survival curve. For a review of MONDE and SuMo, see appendix, Section~\ref{sec:monde_sumo}. We also refer to~\cite{saul2016gaussian} for universal approximators derived via Gaussian processes and~\cite{danks2022derivative} for ones derived via numeric integration of neural networks.

We start by introducing the notation for the cumulative hazard function, $\Lambda(t|x)$, defined via
\begin{equation}
\label{eq:chf_formulation}
S(t|x) = \exp\left[ -\Lambda(t|x)\right].
\end{equation}
One can interpret $\Lambda$ as the total accumulated risk. Since $\Lambda$ determines the model, building a good survival model is equivalent to building a good model for $\Lambda$. The model should be flexible enough to approximate any cumulative hazard. To guarantee survival curves, we have to ensure that $\Lambda$ is monotonically increasing and $\Lambda(0|\cdot)=0$.

\subsection{MONDE+}
As the main building block for such $\Lambda$ we now define the MONDE+ network 
\begin{equation}
    M_+:\mathbb{R} \times \mathbb{R}^n \ni (t, x) \mapsto M_+(t,x) \in \mathbb{R}.
\end{equation}
It is monotonically increasing in its first argument and we define it via layers
\begin{equation}
    \label{eq:monde+}
    z_{k+1}(t, z_k, z_0) = H_kz_k + \sigma_k(\tilde z_k(t, z_k) + B_kz_k + L_kz_0)
\end{equation}
with
\begin{equation}
    \tilde z_k(t, z_k) = A_k\left(\phi_k(a_kt + b_k) \circ \psi_k(G_kz_k)\right)
\end{equation}
where $a_k$ and $b_k$ are vectors, all capitalized letters represent affine maps, and $\circ$ denotes the Hadamard product. $\sigma_k, \phi_k,$ and $\psi_k$ are monotonically increasing functions with non-negative $\phi_k$ and $\psi_k$. In practice, we use $\phi_k=\psi_k=\mbox{softplus}$ and $\sigma_k=\tanh$, in the last layer we set $\sigma_K=\mbox{id}$. Similar to MONDE, we constrain the weight matrices of $A_k, B_k, G_k,$ and $H_k$ and the vector $a_k$ to be pointwise non-negative. Refer to Section~\ref{sec:initalization} of the appendix for details on the initialization.

We recall that a MONDE network  $M:\mathbb{R}^{n+1}\to\mathbb{R}$ consists of layers
\begin{equation}
    \label{eq:monde}
    z_{k+1}(z_k) = \sigma_k(B_kz_k),
\end{equation}
i.e., they cannot differentiate inputs $t, x$, have no residual connections~\cite{he2016deep}, and are a strict subset of the MONDE+ layers.

Unlike MONDE, MONDE+ is monotone in $t$ but has no unnecessary monotonicity constraint in $x$, see the following lemma.
\begin{lemma}
    \label{lemma:monde_plus}
    Let $M_+:\mathbb{R} \times \mathbb{R}^n \to \mathbb{R}$ be a MONDE+ network, i.e., a concatenation of layers defined by~\eqref{eq:monde+}. Then the output of each layer of $M_+$ is pointwise monotonically increasing in $t$.
\end{lemma}

For the proof, see Section~\ref{sec:monde_proof} of the appendix. Each MONDE+ layer contains a MONDE layer as a subset of its operations. As the universal approximation properties of the MONDE network are already established~\cite{lang2005monotonic}, this also makes MONDE+ a universal approximator. As we will show in Section~\ref{sec:numerics}, the MONDE+ component within SuMo++ can successfully model the survival curves for a broad range of applications.

\subsection{SuMo++}
As previously stated, the SuMo++ network aims to improve SuMo's ability to produce accurate survival curves while maintaining its theoretical guarantees of universal approximation.

We define the SuMo++ network as
\begin{equation}
    \mathscr{S}_{++}(t, x) = \exp\left( -\left[M_+(t, q) - M_+(0, q)\right]\right),
\end{equation}
i.e., we set the cumulative hazard function to
\begin{equation}
    \Lambda(t|q) = M_+(t, q) - M_+(0, q)
\end{equation}
with $q = Q(x)\in\mathbb{R}^k$ being some feature extracting network. The exponent in SuMo++ is monotonically decreasing in $t$ and $0$ for $t=0$. Therefore, SuMo++ is not only monotonically decreasing with values in $[0, 1]$, but also guarantees $\mathscr{S}_{++}(0, x)=1$.

In contrast, the SuMo network is defined via a MONDE network as
\begin{equation}
    \mathscr{S}(t, x) = 1 - \mbox{sigmoid}\left(M([t, Q(x)])\right),
\end{equation}
it can, at best, achieve $\mathscr{S}(0, x)=1$ up to some error $\epsilon > 0$ since $\mathscr{S}(t, x) < 1$.

Note that alternatively, one could define SuMo++ via $\mathscr{S}_{++}(t, x) = \frac{1}{1 +\left[M_+(t, q) - M_+(0, q)\right]}.$ However, we did not see any difference in performance, but the former formulation leaves the model and its interpretability closer to classical formulation in~\eqref{eq:chf_formulation}.

We formulated SuMo++ in terms of MONDE+, but we could also do so via MONDE by splitting its input into $z=(t, x)$. For simplicity, we will refer to the SuMo++ approach using MONDE as SuMo+.

Now, as we have established a universal and exclusive approximator of survival curves, we will discuss how SuMo++ is also backward compatible, i.e., how we can use it and MONDE+ to parameterize and treat Cox models as neural networks.
\section{Bringing (time-dependent) Cox models into the deep learning age}
\label{sec:cox_main_section}
The Cox model, introduced in 1972~\cite{cox1972regression}, is arguably the most popular survival model. It uses a baseline cumulative hazard function $\Lambda_0: \mathbb{R}_{\ge0}\ni t \mapsto \Lambda_0(t) \in \mathbb{R}_{\ge0}$, that depends only on time, and a linear regression to modulate the hazard up and down based on the input features $x$.
Formally we write
\begin{equation}
    \mathscr{S}_\text{Cox}(t, x) = \exp\left( -\alpha(x)\Lambda_0(t)\right),
\end{equation}
where
\begin{equation}
    \label{eq:scalar_product_cox}
    \alpha(x) = \exp \left(\langle a, x\rangle\right)
\end{equation}
with the coefficient $a\in\mathbb{R}^n$. This approach seems to strike an outstanding balance between expressiveness and interpretability.

We now describe what we believe is the first implementation of the Cox (and later time-dependent Cox) model that directly parameterizes $ \Lambda_0 $ via a neural network. Note, for the Cox model, \cite{danks2022derivative} implements $\Lambda_0$ indirectly via a neural network but requires the numeric integration of it.

\subsection{A neural network parameterized Cox model}
\label{sec:cox_simple}
We will now introduce a neural network-based implementation of the classical Cox model. While we will use MONDE+ and SuMo++, the following formulations also hold for MONDE and SuMo+ but not SuMo.

The function $\Lambda_0$ is monotonically increasing, with $0$ as a fixed point, and can be parameterized in different ways, e.g., by splines~\cite{efron1988logistic, royston2002flexible} or in a non-parametric fashion~\cite{crowley1984statistical}. We will do so using neural networks.

Using the results in Section~\ref{sec:universal_and_exclusive}, we can parameterize a Cox model by setting
\begin{equation}
    \label{eq:lambda0}
    \Lambda_0(t) = M_+(t, 0) - M_+(0, 0)
\end{equation}
and obtain
\begin{align}
\begin{split}
    \mathscr{S}_\text{Cox}(t, x)
    = \exp\left( -\alpha(x)\Lambda_0(t)\right) 
    = \mathscr{S}_{++}(t, 0)^{\alpha(x)}.
\end{split}
\end{align}
We want to emphasize that in contrast to the DeepSurv model (also known as DeepHit) independently introduced by~\cite{katzman2018deepsurv},~\cite{lee2018deephit} and~\cite{shahin2022survival}, we propose a parameterization for $\Lambda_0$. This makes our approach not a modification, but an implementation, of the Cox model and, therefore, equally interpretable. Still, we enable training and evaluation within DL frameworks using automatic differentiation~\cite{paszke2017automatic}. This enables not only easy training of Cox models using different loss functions, but also straightforward computation of the event density, $f=-\partial_t S$, and hazard function $\lambda(t|x) = \partial_t\Lambda(t|x)$.

\subsection{A neural network parameterized time-dependent Cox model}
\label{sec:cox_time_dependent}
We will now show how we can parameterize Cox models with time-dependent coefficients~\cite{fisher1999time} using neural networks. These models are of the form
\begin{equation}
    \label{eq:cOx}
    \mathscr{S}_\cOx(t, x) = \exp\left( -\alpha(t, x)\Lambda_0(t)\right),
\end{equation}
where one usually sets $\alpha(t, x) = \exp\left[ \langle \omega(t), x\rangle\right]$ with $\omega: \mathbb{R}_{\ge0} \to \mathbb{R}^n$.

Again, as in Section~\ref{sec:cox_simple}, we can parameterize $\Lambda_0(t)$ via~\eqref{eq:lambda0}. For $\alpha$, we propose the parametrization
\begin{align}
\label{eq:cOx_alpha}
    \alpha(t, x) = \exp\left( \langle \omega(t, x), |x - o|\rangle\right)
\end{align}
where $|\cdot|$ is the entrywise absolute value and $o\in\mathbb{R}^n$ a learnable offset parameter. We use two MONDE+ networks $M_+^\pm$ to define
$\omega:\mathbb{R}_{\ge0}\times \mathbb{R}^n\to \mathbb{R}^n$
entrywise as
\begin{equation}
    \omega_i(t, x) =
    \begin{cases}
        \beta^-_i(t), & \text{if $x_i<o_i$,}\\
        \beta^+_i(t), & \text{otherwise},
  \end{cases}
\end{equation}
where we set $\beta^\pm:\mathbb{R}_{\ge0} \to \mathbb{R}^n$ to
\begin{equation}
    \beta^-(t) = M_+^-(t, 0) - M_+^-(0, 0) + M_+^+(0, 0)
\end{equation}
and
\begin{equation}
    \beta^+(t) = M_+^+(t, 0).
\end{equation}

Unlike in Section~\ref{sec:cox_simple}, this is not just a reparametrization but a slight modification guaranteeing $\mathscr{S}_\cOx(t, x)$ to decrease monotonically in $t$, see Lemma~\ref{lemma:cOx}. Still, the coefficients $\omega$ are highly interpretable, as they separate the effects of $t$ from $x$. Essentially, $\omega_i$ provides a time-dependent weight for $x_i$ that is independent of $x$ except for the threshold $o_i$, which marks the ``least dangerous'' value $x_i$ can take.
\begin{lemma}
    \label{lemma:cOx}
    For the model $\mathscr{S}_\cOx(t, x)$, defined by~\eqref{eq:cOx} and \eqref{eq:cOx_alpha}, it holds $\forall x \in\mathbb{R}^n$ that:
    \begin{enumerate}
        \item $\mathscr{S}_\cOx(t, x)\in[0,1]$ for all $t\in\mathbb{R}_{\ge0}.$
        \item We have a maximum $\mathscr{S}_\cOx(0, x)=1$.
        \item $\mathscr{S}_\cOx(t, x)$ is monotonically decreasing in $t$.
    \end{enumerate}
    Further, if the $M_+^\pm$ are continuous in $t$, we have:
    \begin{enumerate}
        \item The $\omega_i$ are continuous in $t$.
        \item For any $x$, the $\omega_i(\cdot, x)$ have their minima at $\omega_i(0, x)=\beta_i^+(0).$
        \item $\alpha$ is continuous in $(t, x)$.
    \end{enumerate}
\end{lemma}
See Section~\ref{sec:proof_cOx} of the appendix for the proof.

\section{Numerical experiments}
\label{sec:numerics}
We will now present the results of our numerical experiments. We will first compare the performance of MONDE and MONDE+ and then perform an extensive comparison of different survival models using several survival and regression datasets.
\subsection{MONDE vs. MONDE+: A toy example}
\label{sec:toy_example}
We now compare the performance of the SuMo, SuMo+, and SuMo++ networks -- i.e., MONDE without and with the initial condition enforced and MONDE+ with the initial condition enforced.

To showcase the limitations of the SuMo and SuMo+ models in recreating survival curves, we designed an experiment using six samples, each containing 32 features, each in the range $[-1,1]$. We assigned survival probabilities for each sample at 2-7 points over the time interval $[0,1]$ in a way that tests the models' ability to capture complex distributions, such as sharp declines followed by plateaus and vice versa.

% Figure environment removed

% Figure environment removed

We trained each network for 512 Adam~\cite{kingma2014adam} iterations with a step size of $10^{-3}$ using the standard binary cross-entropy loss. Figure~\ref{fig:toy_curves} compares the three models over the six samples. To show the variability of the outcome, we also provide histograms of multiple runs of these experiments in Figure~\ref{fig:hists}. The histograms show that it can be difficult for the SuMo+ model to fit the six curves, while the SuMo and SuMo++ can produce better fits.

During these and all following experiments, we used MONDE+ with five hidden layers of width 32. We also use MONDE with the same structure, but after the input layer, we change the width to $98$ to give both networks approximately $32,000$ trainable parameters.

\subsection{Classifier metrics}
\label{sec:classifier_metrics}
We will now discuss how we will evaluate the following experiments. As discussed in~\cite{rindt2022survival}, the classical scoring rules for survival models do not adequately reflect the performance of survival models. Classical scoring rules for survival models, such as time- or hazard-based concordance scores for traditionally trained Cox models, are only effective in their intended context. However, they may not accurately reflect performance in a general setting defined by~\eqref{eq:S}. For example, a model that incorrectly predicts all patients dying within milliseconds of a clinical trial may still have a perfect concordance score, as the score only considers the correct order of events and not their absolute time. This does not make time concordance useless for these models but insufficient. Ideally, one would have multiple scoring rules to choose from to judge the different aspects of a model that are relevant to a given situation.

We propose to use the general integrated Brier score (IBS). However, instead of using the mean squared as the integrated scoring rule, we propose to use general classifier scoring rules, e.g., accuracy. Interestingly this fits squarely into the original general definition of the IBS~\cite{graf1999assessment}, but we have not seen it applied this way.

As a survival model is an infinite number of probabilistic classifiers indexed by time, we can evaluate the survival model's quality from different aspects; simply by choosing from the rich pool of classifier quality measures. E.g., one could use the $F_2$-score if one considers a clinical setting where an overly optimistic prognosis may deny patients timely access to escalation.

Note that we will use threshold-free versions of all classifier scoring rules. For example, if $l\in\{0,1\}^K$ is a binary vector of length $K\in\mathbb{N}$ containing labels and $p\in[0,1]^K$ a vector containing corresponding probabilistic predictions, we define the true-positives to be their scalar product, i.e., $\sum_{k=1}^K p_kl_k$.

The IBS integrates its particular scoring rule over a given time interval. We will use the interval $[0, T_{max}]$ where we choose $T_{max}$ for each dataset based on the $90^{th}$ percentile of times provided by the dataset.

The mean squared error is arguably the most popular scoring rule in regression. Thus the standard IBS neatly connects to the regression point of view we laid out in Section~\ref{sec:sa_as_regression}, while the classifier scoring rules proposed by us connect to Section~\ref{sec:sa_as_classification}.

\subsection{Datasets}
\label{sec:datasets}
We will now provide a brief qualitative overview of the datasets we use in Section~\ref{sec:results_and_discussion}. See Table~\ref{tab:dataset_overview} for a high-level overview. Each dataset consists of samples given by triples as in~\eqref{eq:sample} i.e., $(x, e, T)$. We convert the Clocks and California regression datasets into this format by assigning $T$ the label's value and $e=1$.
\begin{table}
    \centering

    \begin{tabular}{|c||c|c|c|}
        \hline
        Name & \#Samples & \#Features & Test set \\
        \hline
        GBSG2 & 686 & 8 & 25\% \\
        Recur & 1,296 & 4 & 25\% \\
        NKI & 272 & 9 & 25\% \\
        Lymph & 686 & 8 & 25\% \\
        COVID-19 & 1,489 & 16 & 25\% \\
        Clocks & 10,000 & 6,912 & 2.5\% \\
        California & 20,640 & 8 & 2.5\% \\
        \hline
    \end{tabular}

    \caption{An overview of the datasets as we used them. Up to rounding errors, we split the data such that the validation and test have the same size, i.e., if the test set contains 25\% of the sample, the validation contains 25\%, and the training contains 50\% of the samples. For the references see, GBSG2 \cite{sauerbrei1999building}, Recur \cite{lemeshow2011applied}, NKI \cite{nicolau2011topology}, Lymph \cite{schmoor2000role}, Clocks \cite{clocks}, and California \cite{pace1997sparse}.}
    \label{tab:dataset_overview}
\end{table}

The COVID-19 dataset is a private dataset based on COVID-19 patient data from Addenbrooke's Hospital, Cambridge, UK. The seven datasets showcase diverse survival curve dynamics. For example, the dynamics of oncology treatment outcomes and that of critically ill patients differ greatly. The Clocks dataset, with deterministic outcomes, highlights sudden changes in survival curves.

As some of the datasets have relatively few samples, the random seed used to split the data into training, validation, and test sets can significantly impact the evaluation of a model. To mitigate this effect, we ran 1000 splits and selected the seed that minimizes the difference between the three Kaplan-Meier estimators for the training, validation, and test sets. Since Kaplan-Meier estimates the expected unconditioned survival curve for each set, this decreases the chance of having vastly different training, validation, and test sets.

We normalize all input features to have a $0$ mean and standard deviation of $1$. We normalize the times within the data by dividing through by $T_{max}$ as defined in Section~\ref{sec:classifier_metrics}. For more details on the datasets, see Section~\ref{sec:datasets_appendix} in the appendix.

\subsection{Results and discussion}
\label{sec:results_and_discussion}

For the Clocks dataset, containing images, we apply the methods amenable to convolutions to extract the features for the model, namely CoxDeepNN, SuMo, SuMo+, and SuMo++, as well as Kaplan-Meier. For all other datasets, which contain tabular data, we were able to train all models discussed earlier.
See Section~\ref{sec:training_details} in the appendix for training details. We used the standard and classifier metric versions of the IBS discussed in Section~\ref{sec:classifier_metrics} to evaluate the models. We use the classifier metrics accuracy, area under the precision-recall curve (AUPRC), area under the receiver operating characteristic curve (AUROC), balanced accuracy, the $F_\beta$-score for $\beta\in\{0.5, 1, 2\}$, precision, sensitivity, specificity, and the Youden's index~\cite{youden1950index}. In addition to the IBS, we also used the established concordance index based on the restricted mean survival time, restricted by the dataset's $T_{max}$. See Table~\ref{tab:stats_Concordance} and Table~\ref{tab:stats_iibs} in the appendix for the concordance index and the standard IBS. We report the mean of all scores in Table~\ref{tab:stats_Mean}. Note that the CoxDeepNN model approach is well known from other papers~\cite{katzman2018deepsurv, kvamme2019time, shahin2022survival}, commonly referred to as DeepHit or DeepSurv.
\begin{table*}[ht]
\centering
\begin{tabular}{l||rrrrr|rr}
Model Loss-function & COVID-19 & NKI & BGSG2 & Recur & Lymph & Clocks & California \\
\hline \hline
Kaplan-Meier & 0.40 & 0.43 & 0.45 & 0.52 & 0.42 & 0.51 & 0.52 \\
Weibull & 0.67 & 0.52 & 0.51 & 0.75 & 0.47 & $\emptyset$ & 0.68 \\
Log-logistic & 0.67 & 0.53 & 0.51 & 0.76 & 0.47 & $\emptyset$ & 0.71 \\
Log-normal & 0.67 & 0.53 & 0.51 & 0.76 & 0.48 & $\emptyset$ & 0.70 \\
Cox-piecewise & 0.67 & 0.52 & 0.51 & 0.75 & 0.47 & $\emptyset$ & 0.67 \\
Cox-spline & 0.67 & 0.52 & 0.51 & 0.75 & 0.47 & $\emptyset$ & 0.67 \\
\hline
CoxNN SuMo & 0.70 & 0.54 & 0.50 & 0.77 & 0.49 & $\emptyset$ & 0.68 \\
CoxNN BCE & 0.72 & 0.54 & 0.51 & 0.73 & 0.48 & $\emptyset$ & 0.64 \\
\cOx NN SuMo & 0.73 & 0.56 & 0.51 & 0.79 & 0.50 & $\emptyset$ & 0.65 \\
\cOx NN BCE & 0.70 & 0.56 & 0.51 & 0.79 & 0.48 & $\emptyset$ & 0.62 \\
CoxDeepNN SuMo & 0.73 & 0.53 & \textbf{0.55} & 0.79 & 0.50 & 0.69 & 0.69 \\
CoxDeepNN BCE & 0.72 & 0.58 & 0.49 & 0.78 & 0.49 & 0.83 & 0.65 \\
\hline
SuMo SuMo & 0.73 & 0.55 & 0.49 & 0.80 & 0.50 & \textbf{0.97} & 0.80 \\
SuMo BCE & 0.74 & 0.62 & 0.54 & \textbf{0.81} & 0.47 & 0.91 & 0.77 \\
SuMo+ SuMo & 0.72 & 0.50 & 0.50 & 0.79 & 0.49 & \textbf{0.97} & \textbf{0.81} \\
SuMo+ BCE & \textbf{0.76} & 0.59 & 0.54 & 0.79 & 0.48 & 0.91 & 0.78 \\
SuMo++ SuMo & 0.73 & 0.57 & 0.53 & 0.80 & 0.49 & 0.96 & 0.80 \\
SuMo++ BCE & 0.72 & \textbf{0.64} & \textbf{0.55} & 0.79 & \textbf{0.52} & 0.91 & 0.78 \\
\end{tabular}
\caption{
The mean of all over time integrated scores and the concordance index for each model and the relevant test dataset. The higher, the better. We only list our BCE loss and the SuMo loss if applicable. The CoxDeepNN model is also known as DeepHit or DeepSurv. For the individual scores see the appendix, Tables~\ref{tab:stats_A}--~\ref{tab:stats_iibs}.
}
\label{tab:stats_Mean}
\end{table*}

In Section~\ref{sec:separate_scores} of the appendix, we present tables with each separate integrated quality score, along with the non-integrated versions of the balanced accuracy and the $F_1$ score for each dataset in Section~\ref{sec:plots_non_integrated}. We chose these two scores as they are highly correlated with the mean of the scores and complement each other.

Table~\ref{tab:stats_Mean} shows that SuMo, SuMo+, and SuMo++ outperform all other models. For example, plots of survival curves see Section~\ref{sec:predicted_survival_curves} in the appendix. In particular, SuMo+ and SuMo++ slightly outperform SuMo in most cases while guaranteeing the output to be a survival curve. For the losses, the BCE loss tends to outperform the SuMo loss on survival datasets containing right-censoring; for more on the SuMo loss, see the appendix, Section~\ref{sec:training_details}, and~\cite{rindt2022survival}.

In contrast, the SuMo loss outperforms the BCE for the adapted regression datasets. We are not sure why that is, but the SuMo, unlike the BCE loss, is conditioned on whether the event was observed. This might decrease its performance if censoring is not independent of the time of the event. This would not be an issue for an entirely uncensored dataset.
\section{Conclusion}
We introduce a novel event-based approach for background image reconstruction in the presence of dynamic occlusions.
It leverages the complementary nature of event camera and frames to reconstruct true scene information instead of hallucinating occluded areas as done by image inpainting approaches.
Specifically, our proposed data-driven approach reconstructs the background image using only one occluded image and events.
The high temporal resolution of the events provides our method additional information on the relative intensity changes between the foreground and background, making it robust to dense occlusions. %
To evaluate our approach, we present the first large-scale dataset recorded in the real world containing over $230$ challenging scenes with synchronized events, occluded images, and groundtruth images.
Our method achieves an improvement of 3dB in PSNR over state-of-the-art frame-based and event-based methods on both synthetic and real datasets.
We will release our synthetic and recorded dataset representing the first datasets for background image reconstruction using events and images in the presence of dynamic occlusions.
We believe that our proposed method and dataset lay the foundation for future research.


\FloatBarrier
\acks{
There is no direct funding for this study, but the authors are grateful for the following indirect funding: The EU/EFPIA Innovative Medicines Initiative project DRAGON (101005122) (M.R., S.D., AIX-COVNET, C.-B.S.), the Trinity Challenge (M.R., C.-B.S.), the EPSRC Cambridge Mathematics of Information in Healthcare Hub EP/T017961/1 (M.R., S.D., J.H.F.R., J.A.D.A, C.-B.S.), the Cantab Capital Institute for the Mathematics of Information (C.-B.S.), the European Research Council under the European Union’s Horizon 2020 research and innovation programme grant agreement no. 777826 (C.-B.S.), the Alan Turing Institute (C.-B.S.), Wellcome Trust (J.H.F.R.), Cancer Research UK Cambridge Centre (C9685/A25177) (C.-B.S.), British Heart Foundation (J.H.F.R.), the NIHR Cambridge Biomedical Research Centre (J.H.F.R.), HEFCE (J.H.F.R.). In addition, C.-B.S. acknowledges support from the Leverhulme Trust project on ‘Breaking the non-convexity barrier’, the Philip Leverhulme Prize, the EPSRC grants EP/S026045/1 and EP/T003553/1 and the Wellcome Innovator Award RG98755. Finally, the AIX-COVNET collaboration is also grateful to Intel for financial support.\\

We also want to acknowledge and thank the members of the AIX-COVNET collaboration:
Michael Roberts$^{1}$, S{\"{o}}ren Dittmer$^{1,6}$, Ian Selby$^{7}$, Anna Breger$^{1,8}$, Matthew Thorpe$^{9}$, Julian Gilbey$^{1}$, Jonathan R. Weir-McCall$^{7,10}$, Effrossyni Gkrania-Klotsas$^{3}$, Anna Korhonen$^{11}$, Emily Jefferson$^{12}$, Georg Langs$^{13}$, Guang Yang$^{14}$, Helmut Prosch$^{13}$, Jacobus Preller$^{3}$, Jan Stanczuk$^{1}$, Jing Tang$^{15}$, Judith Babar$^{3}$, Lorena Escudero Sánchez$^{7}$, Philip Teare$^{16}$, Mishal Patel$^{16,17}$, Marcel Wassin$^{18}$, Markus Holzer$^{18}$, Nicholas Walton$^{19}$, Pietro Li{\'{o}}$^{20}$, Tolou Shadbahr$^{15}$, James H. F. Rudd$^{4}$, John A.D. Aston$^{5}$, Evis Sala$^{7}$ and Carola-Bibiane Schönlieb$^{1}$.\\


\noindent
${}^{1}$ Department of Applied Mathematics and Theoretical Physics, University of Cambridge, Cambridge, UK
${}^{2}$ A list of authors and their affiliations appears at the end of the paper
${}^{3}$ Addenbrooke’s Hospital, Cambridge University Hospitals NHS Trust, Cambridge, UK.
${}^{4}$ Department of Medicine, University of Cambridge, Cambridge, UK
${}^{5}$ Department of Pure Mathematics and Mathematical Statistics, University of Cambridge, Cambridge, UK
${}^{6}$ ZeTeM, University of Bremen, Bremen, Germany
${}^{7}$ Department of Radiology, University of Cambridge, Cambridge, UK
${}^{8}$ Faculty of Mathematics, University of Vienna, Austria.
${}^{9}$ Department of Mathematics, University of Manchester, Manchester, UK.
${}^{10}$ Royal Papworth Hospital, Cambridge, Royal Papworth Hospital NHS Foundation Trust, Cambridge, UK
${}^{11}$ Language Technology Laboratory, University of Cambridge, Cambridge, UK.
${}^{12}$ Population Health and Genomics, School of Medicine, University of Dundee, Dundee, UK.
${}^{13}$ Department of Biomedical Imaging and Image-guided Therapy, Computational Imaging Research Lab Medical University of Vienna, Vienna, Austria.
${}^{14}$ National Heart and Lung Institute, Imperial College London, London, UK.
${}^{15}$ Research Program in Systems Oncology, Faculty of Medicine, University of Helsinki, Helsinki, Finland.
${}^{16}$ Data Science \& Artificial Intelligence, AstraZeneca, Cambridge, UK.
${}^{17}$ Clinical Pharmacology \& Safety Sciences, AstraZeneca, Cambridge, UK.
${}^{18}$ contextflow GmbH, Vienna, Austria. 
${}^{19}$ Institute of Astronomy, University of Cambridge, Cambridge, UK. 
${}^{20}$ Department of Computer Science and Technology, University of Cambridge, Cambridge, UK.
}

\FloatBarrier
\bibliography{references}

\FloatBarrier
\newpage
\appendix
\begin{comment}
\section{System Architecture}
\label{appendix:architecture}
\system has a novel modularized system architecture with three key components: 
\emph{StreamManager}, 
\emph{TxnManager} and \emph{TxnScheduler}. 
These components are instantiated in each thread locally.
The execution outline of \system is presented in Algorithm~\ref{alg:algo}.
Transactional stream processing is continuous and potentially never ends (Line 1$\sim$8).
The dependency resolution and execution of state transactions are separated into two non-overlapping phases by punctuations~\cite{Tucker:2003:EPS:776752.776780} (Line 2 and 5), which guarantees that no subsequent input event will have a smaller timestamp. 
Effectively, a batch of state transactions is collected during the first phase, and processed during the second phase.

In the first phase (i.e., stream processing phase), 
the \emph{StreamManager} conducts preprocessing for every input event ($e$). Similar to some prior works~\cite{tstream}, state transactions may be issued but not immediately processed during preprocessing (Line 3).
The \emph{pre\_processing} and \emph{post\_processing} functions are exposed as APIs to users.
The \emph{TxnManager} handles dependency resolution (Line 4) among state transactions and insert decomposed operations to construct a \tpg. We discuss the detailed two-phase \tpg construction process in Section~\ref{subsec:construction}.

In the second phase  (i.e., transaction processing phase), 
the \emph{TxnManager} is first involved again to refine (Line 6) the constructed \tpg with further dependency resolution.
The \emph{TxnScheduler} 
schedules operations for concurrent execution based on the constructed \tpg according to the three dimensions of scheduling decisions (Line 7). 
In particular, a scheduling decision model $M$ is instantiated based on the constructed \tpg (Line 14).
\textbf{\circled{1}} Guided by $M$, execution threads adopt an exploration strategy (Section~\ref{subsec:explore}) to explore the constructed \tpg for operations available to be scheduled constrained by dependencies. 
\textbf{\circled{2}} 
During exploration, one or multiple operations may be treated as the 
% basic 
unit of scheduling (Section~\ref{subsec:granularity}). 
Subsequently, \textbf{\circled{3}} every thread executes operation(s) in the unit of scheduling with various abort handling mechanisms (Section~\ref{subsec:abort_handling}).
Only when state transactions are processed (i.e., committed or aborted) can the associated input events be postprocessed (Line 8) by the \emph{StreamManager} based on transaction processing results.
\end{comment}

\begin{comment}
\begin{algorithm}
\footnotesize
    \KwData{$e$ \tcp{Input event}}
    \KwData{$txn_{ts}$ \tcp{State transaction}}
    \KwData{$G$ \tcp{The currently constructed TPG}}
    \While{!finish processing of input streams}{
        \eIf(\tcp*[h]{Phase 1}){\text{$e$ is not a $punctuation$}}{
                $txn_{ts}$ $\gets$ PRE\_Processing($e$)\;
                \textbf{TPG\_Construction}($G$, $txn_{ts}$)\; 
          }(\tcp*[h]{Phase 2}){
                \textbf{TPG\_Refinement}($G$)\; 
                \textbf{TXN\_Scheduling}($G$)\; 
                POST\_Processing()\;
          }
    }
    
    \SetKwFunction{FMain}{TPG\_Construction}
    \SetKwProg{Fn}{Function}{:}{}
    \Fn{\FMain{$G$, $txn_{ts}$}}{
        $O_{1..k}$ $\gets$ \textbf{Partition} $txn_{ts}$\;
        \ForEach{\text{operation $O_{i}$ $\in$ $O_{1..k}$}}{
            \textbf{Identify} its \ld\;
            $G$ $\gets$ $G$ + $O_{i}$ \;
        }
    }
    \SetKwFunction{FMain}{TPG\_Refinement}
    \SetKwProg{Fn}{Function}{:}{}
    \Fn{\FMain{$G$}}{
        \ForEach{\text{vertex $e_{i}$ $\in$ $G$}}{
            \textbf{Identify} its \td, \pd\;
        }
    }
    
    \SetKwFunction{FMain}{TXN\_Scheduling}
    \SetKwProg{Fn}{Function}{:}{}
    \Fn{\FMain{$G$}}{
        $M$ $\gets$ Instantiated with $G$;\tcp{A decision model}
        \While{!finish scheduling of $G$
        }{
          \textbf{\circled{2}} $Scheduling Unit$ $\gets$ \textbf{\circled{1}} \emph{Explore}($G$, $M$)\; 
            \textbf{\circled{3}} \emph{Execute with Abort Handling} ($Scheduling Unit$)\; 
        }
    }
  \caption{Execution Outline of \system}
  \label{alg:algo}
\end{algorithm}
\end{comment}

\end{document}
