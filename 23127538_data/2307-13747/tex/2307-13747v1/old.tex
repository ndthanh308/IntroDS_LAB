
\section{Deamortized argument}


Every vertex $u$ has a true rank $T(u)$ and a fake rank $F(u)$. True ranks are defined by the hierarchical procedure, we will next explain how to define fake ranks. 

We represent the situation by a hierarchical tree where every leaf is on level 0 (=rank 0) corresponds to one point. Inner nodes of the tree correspond to the group structure. Every node can in addition be red and also blue. Red color corresponds to true ranks and blue color corresponds to fake ranks. We keep the following invariants. 

\begin{enumerate}
    \item The leafs are beginnings of blue paths that go up the tree. They are vertex disjoint and maximal. Fake rank of a node equals how high the respective blue path goes.  
    \item The leafs are beginnings of red paths that go up the tree. Red paths are edge disjoint and maximal. True rank equal how high the respective red path goes (including the last edge).     
\end{enumerate}

Whenever a group of nodes goes up from rank r to r+1, we create a new node at level r+1 and connect it with edges to all respective roots at level r. All these edges are made red to keep the invariant that red edges correspond to ranks. Note that the disjointness invariant is satisfied because at most one point from a lower group can go to the larger group. Moreover, an arbitrary of these new edges is colored blue to keep the blue invariant; this corresponds to one fake rank update $r \rightarrow r+1$. 

In total, all group updates imply at most one fake rank update $r \rightarrow r+1$ for every $r$. 

Second type of update is delete. Whenever a node is deleted, we walk up the tree from the respective leaf and delete all the red edges from that tree (not just uncolor, delete). This fixes the red invariants. Then we need to fix the blue invariants. For every blue edge that got deleted, its top vertex connects itself blue with some other child. If not possible, it does not matter because this top vertex is going to be deleted from the tree anyway. Again, we get at most one fake promotion for every rank. 

We keep centers on the $k$ fake points. 

\begin{claim}[Domination]
For every cut on rank $r$ and above, there are more red paths that are cut than blue paths. 
\end{claim}
\begin{proof}
    For the root, every edge going down from it is red, but only one is blue. Then induction on the subtrees. 
\end{proof}

\begin{claim}[Coverage]
    Every point is covered by some other point with center on it with cost roughly the same as optimum. 
\end{claim}
\begin{proof}
We get an arbitrary vertex $u$ and need to find out which vertex with a center covers it. 
First, take vertex $v$ that would cover $u$ in the true solution. By definition, the red path from $v$ cuts the line of OPT. At this cutoff, it meets another blue path so take it and go down until you find $w$. $w$ has center by the Domination claim. Also, it is at most O(OPT) far from $v$ by the fact that the tree edges always correspond to nearby points. 
\end{proof}


% Figure environment removed




\end{document}


\paragraph{Notation}
 Point set $P^{(1)},P^{(2)},\ldots$ where $|P^{(t-1)} \triangle P^{(t)}| = 1$. We define $p^{(t)}$ as the unique point in $P^{(t-1)} \triangle P^{(t)}$.
Output is $C^{(1)},C^{(2)},\ldots$ where $C^{(t)} \subseteq P^{(t)}$ and $|C^{(t)}| \leq k$. For $t \geq 1$, we define $\Delta^{(t)} = |C^{(t-1)} \triangle C^{(t)}|$. We also define $OPT^{(t)} = OPT(P^{(t)},k)$ where

\[OPT(P^{(t)}, k) := \min_{{C \subseteq P^{(t)}, |C| \leq k}} \max_{{p \in P^{(t)}}} d(C, p).\]


\paragraph{Invariants}

The algorithm keeps track of a mapping $r^{(t)} \colon P^{(t)} \mapsto \mathbb{N}$. We also define $P^{(t)}_{r} = \{p \in P^{(t)} \mid r^{(t)} = r\}$ and $P^{(t)}_{\geq r} = \bigcup_{r' \geq r} P^{(t)}_{r'}$. For every $t \in \mathbb{N}$, the following invariants are satisfied:

\begin{enumerate}
\item (Separation) $d(u,v) \geq sep_r$ for every $r \in \mathbb{N}$ and $u,v \in P^{(t)}_{\geq r}$.
\item (Covering) $P^{(t)} \subseteq B \left( P^{(t)}_{\geq r},cov_r \right)$.
\item (Maximality) For every $u \in P^{(t)}$, either $P^{(t)}_{> u} = \emptyset$ or there exists $v \in P^{(t)}_{> u}$ with $d(u,v) < sep_{r + 1}$
\end{enumerate}
Moreover, the different mappings are also monotone in the sense that for every $u \in P^{(t)} \cap P^{(t+1)}$ it holds that $r^{(t+1)}(u) \geq r^{(t)}(u)$.

\paragraph{Algorithm Description}
Insert:
$P^{(t)} \setminus P^{(t-1)} = \{p^{(t)}\}$. We define $r^{(t)}(p^{(t)}) = \max_{r \in \mathbb{N}}$ and set $r^{(t)}(u) = r^{(t-1)}(u)$ for every $u \in P^{(t-1)}$.




\paragraph{Basic Setup and Invariants}

\paragraph{Algorithm Description}


\textbf{Kuba's datastructure:} Every point of the dataset has a rank. We maintain the property that for every rank i, the points of rank at least i have the following property for r = 10000 to i:
1) the are r disjoint
2) they are 100r ruling. 

Solution: Our k centers are always top k centers in terms of rank. Ties are solved by prefering older centers.  

\textbf{INSERT:} 
Inserting point u: We give it as high rank as possible until it first happens that u in B(v,r) for some v and radius r. Then we continue the following. The first time that v is not in MIS, we either try to promote v one level, or if it is not possible, it is because of some v' that rules u. So then we continue with this v'. 

This whole promotion bussiness can change status of log Delta points, but there is additional structure: the promotions look like 1-5, 6-8, 9-11, ... In particular, only one new point can enter top k points. Also, only one new point can have rank r for any r. 

\textbf{DELETE:}
we go from smallest levels to the level equal to the rank of the deleted point u (above it we are fine). For every v which is now not ruled we know that it is dominated one level lower by some w. We will either promote w and if it is not possible then v is in fact ruled. 

We can promote an arbitrary number of points, but there is again some structure. We can identify some points whose promotions look like INSERT, in particular it looks like 1-5 5-9 9-11 ... Additionally, we are creating a bunch of disjoint c-groups 2-3 3-4 4-5 etc. 

\textbf{Definition of a c-group:} 
\begin{itemize}
    \item  a group of points and maybe a marked leader. Every point has rank c, perhaps except of the leader that is rank $>= c$. 
    \item for any c, c-groups are disjoint. 
    \item If the c-group does not have a leader and we are in phase of rank c, it has a token. 
    \item everybody in the c-group is pairwise distant at least c from everybody else (follows from rank c) but also at most 10c-distant
\end{itemize}

\textbf{Phases}
Rank of a step is the rank of the k+1. element in that step. 
A phase is a maximal consecutive sequence of steps with the same rank which is the rank of that phase. 

We have two kinds of phases, chill phases, and nonchill phases where chill phases are those such that the rank of them is smaller than the rank of the previous phase. $T$ stands for the top $k$ points in a given step. 

\textbf{Chill phase}
If the rank of the phase is r, we know by definition of chill that all top k elements are at least r+1 rank. 
The potential at some step of nonchill phase is $$\Phi = \text{number of elements in $T$ of rank $r$}$$ At the beginning, $\Phi = 0$. Let's see what can happen in one step:

INSERT: We need to change $T$ by at most one point. Also, $\Phi$ changes by at most 1 or so. 

DELETE: Unless there is a new $r+1$-group, it is as in INSERT (we use that any new $r$-group will not go to top k points by how we manage ties). If there is a new $r+1$-group, the number of new centers we need to assign is proportional to how much $\Phi$ drops. 

\textbf{NonChill phase}
In nonchill phase of rank $r$, we use a lazy algorithm that does not do any changes to points of rank r+1 or lower, only points with rank r+2 are given a center. At the end of the phase, we do a cleanup where we reassign centers to top k points. 

We define
$$\Phi = \text{number of rank $r$ groups in $X \setminus T$ that do not have a leader and do not have a center on them}$$
(one element of rank is considered as a special case of a r-group to simplify definition of $\Phi$)

At the beginning, $\Phi = 1$. Every created rank $r$ group is given one token at the time of creation. The group does not have a token if and only if it has a leader point with rank $>r$ (somewhere in $T$). 

Whenever a point without center is added to $T$, $\Phi$ correspondingly decreases. Thus, we may pay for the final cleanup in the end. 

INSERT: We need to change $T$ by at most one point. Also, $\Phi$ changes by at most 1 or so. 

DELETE: Unless we create a new rank r+1 group, it is as INSERT. When we create rank r+1 group, we need to change number of centers proportional to the size of the group. However, the potential drops by the same amount, which follows from the fact that from every r-group without a leader, at most one point promotes, and from every r-group with a leader, no point can promote. Finally, whenever we delete a point that happens to be a leader of some r-group, we give that r-group a new token as it does not have a leader anymore. 


\newpage


\section{algorithm and invariants}

Every center is at some point. At top k points, every point has either center on it, or it is in a \textit{group}. In a group, there is exactly one point in it (non necessarily in top k) that has a center on it, the other points do not have a center. When we create a group, we define the buddy distance of every point in it to be its current rank. We will preserve as invariant that the leader of the group has distance to each nonleader at most 2 times the buddy distance of that nonleader. 

INSERT + DELETE: Some new nodes join top k. Unless all of them have a center on them, we make incoming nodes a new group and the point with smallest rank becomes their leader. We put a center on the leader. A case of DELETE and no promotions, hence k+1. point goes to top k should be special case. 
The node moved to the leader is an arbitrary node outside of top k that is not a leader of a group itself. Such a node exists because if there are l noncenters in top k, there are at most l centers outside top k that are leaders but we just created one new group without a leader. 

Moreover, in case of DELETE and the deleted node being a leader of some group, we again reassign a point outside top k that is not a leader to the next smallest rank point of that group. 

Let $U_r$ be all points of rank $r$ or higher. 
\begin{claim}
    invariant: for every $r$, we have that 
    $$ \text{\# centers in $U_r$} + \text{\# noncenters in $T$ with buddy distance at most $r$} \ge k$$
\end{claim}

If we plug in rank $r$ of current step to the invariant, we get that every point in $T$ either has a center or its leader has distance at most $2(r+1)$ from it. We conclude that our solution is constant approximation. 

So, we just need to show that the invariant is preserved. 

INSERT: the only way the invariant goes down is that we push one element out of top k and invariant decreases by one. But then we are adding a new center so it goes up by one. We got this center somewhere outside top k -- for ranks higher than current rank this may mean decreasing invariant by one but for that 

\newpage

Notation: $X$, $k$ $OPT$

\section{A version where we don't care about running time or derandomization or optimizing constant factors}


\newcommand{\OPT}{\text{OPT}}
\begin{algorithm}
\caption{One Phase}
\STATE Let OPT be the optimal solution (and its cost). Define
$OUT \subseteq X$ to be the collection of all (disjoint) subsets $K$ of $X$ such that a) there is $x \in X$ that $rad(X, x) \le OPT$ and $K = B(x, 10OPT)\cap X$. That is, the set of outliers $OUT$ are the well-separated clusters of $X$.  
\STATE Define $k' = k - |OUT|$
\STATE We construct a set of centers $C = C_0 \sqcup C_1$ as follows.
\begin{enumerate}
    \item For every cluster $K \in OUT$, we take the same center point that the optimum takes; this defines $C_0$
    \item Then we define $C_1$ as the minimizer of   \[\min_{C_1 \subseteq X, |C_1| = k'/2} \max_{x \in X} d(x,C_0 \cup C_1).\]
    
\end{enumerate}
\STATE The following $k'/10$ steps is the next phase. For every insertion, we add the point as the new center. 
\end{algorithm}


\begin{claim}
The set of centers $C$ is $O(1)$-approximate at the beginning.
\end{claim}
\begin{proof}[Proof Sketch]
Consider the graph $G$ with $V(G) = C^* \setminus C_0$ and  where any two centers $c_1,c_2 \in V(G)$ are connected by an edge if $d(c_1,c_2) \leq 20 OPT$. Consider any $C' \subseteq V(G)$. If $C^'$ is a dominating set in $G$, then $C_1 \sqcup C'$ has cost $O(OPT)$. Moreover, $G$ has no isolated vertex as otherwise the corresponding vertex would be in OUT. Thus, it is possible to find a dominating set $C'$ of size at most $|C'|/2$. One way to get such a dominating set is as follows. Take an arbitrary node in each connected component. Now, if at most half of the vertices have an odd distance to that node, then take all nodes of odd distance as the dominating set in that component, otherwise take all nodes of even distance as the dominating set in that component.
\end{proof}
\begin{claim}
The algorithm is $O(1)$ approximate. 
\end{claim}
\begin{proof}

\end{proof}

\begin{claim}
For any two consecutive phases, we have $C_{new} \Delta C_{old} = O(k'_{old} + k'_{new})$. 
\end{claim}
\begin{claim}
We will prove that $|C_{new} \cap C_{old} \cap OUT_{new} \cap OUT_{old}| \ge k - O(k'_{old} + k'_{new})$. 
We consider separately two cases, $OPT_{new} \ge OPT_{old}$ and $OPT_{new} \le OPT_{old}$. 

\paragraph{$OPT_{new} \ge OPT_{old}$}
Consider any cluster $K$ which is in $OUT_{old}$. We say that such a cluster $K$ is special if it either contains a point inserted in the old iteration, or if it contains a point from $IN_{old}$. We note that the number of special clusters is at most $O(k'_{old})$ since whenever $K$ contains a point $x \in K' \in IN_{old}$, then $K' \subseteq B(K, 2OPT_{old} \subseteq B(K, 2OPT_{new})$. Since the $2OPT_{new}$ neighborhoods of $OUT_{new}$ clusters are disjoint, we can charge one of at most $k'_{old}$ $IN_{old}$ clusters to every special $K$ containing a point from $IN_{old}$. 

If $K$ is not special, we say that $K \in T_i$ if it intersects $i$ different clusters of $OUT_{old}$. As before, there cannot be $K_{old} \in OUT_{old}$ intersecting two different clusters $K_1, K_2 \in OUT_{new}$. We conclude that
\begin{align}
    |OUT_{new}| = O(k') + |T_1| + |T_2| + \dots
\end{align}
On the other hand, 
\begin{align}
    |OUT_{old}| \ge T_1 + 2T_2 + 3T_3 + \dots \ge T_1 + 2(T_2 + T_3 + \dots )
\end{align}
thus
\begin{align}
    k'_{new} - k'_{old} = |OUT_{old}| - |OUT_{new}| \ge T_2 + T_3 + \dots
\end{align}
Finally, we have
\begin{align}
 &   \Delta(C_{new}, C_{old}) 
    = |\text{optimal extension}| + |\text{special clusters}| +  T_2 + T_3 + \dots\\
&    \le O(k'_{new}) + O(k'_{old}) + T_2 + T_3 + \dots
    = O(k'_{new} + k'_{old})
\end{align}

\paragraph{$OPT_{new} \le OPT_{old}$}
Let $K_{old} \in OUT_{old}$. We say that $K_{old}$ is special if it contains a point deleted during the old phase or if it contains a point from $K \in IN_{new}$ (we can again charge the whole cluster $K$ to just one $K_{old}$). There are $O(k'_{old} + k'_{new})$ special clusters. 
Otherwise, we say $K_{old} \in T_i$ if it interesects $i$ different clusters of $OUT_{new}$. each cluster of $OUT_{new}$ is counted by at most one $K_{old}$. 
Thus, we have
\begin{align}
    k-k'_{old} = |OUT_{old}| \le O(k'_{old} + k'_{new}) + T_1 + T_2 + \dots 
\end{align}
and
\begin{align}
    |OUT_{new}| \ge T_1 + 2T_2 + 3T_3 + \dots 
\end{align}
Thus
\begin{align}
    |OUT_{new}| - |OUT_{old}| \ge T_2 + T_3 + \dots  - O(k'_{old} + k'_{new})
\end{align}
Finally, we have
\begin{align}
|C_{new} \cap  C_{old}|
&\ge T_1 
\ge |OUT_{old}| - O(k'_{old} + k'_{new}) - T_2 - T_3 - \dots\\
&\ge  |OUT_{old}| - O(k'_{old} + k'_{new}) - |OUT_{new}| + |OUT_{old}|\\
&\ge k -  O(k'_{old} + k'_{new})
\end{align}



\end{claim}



