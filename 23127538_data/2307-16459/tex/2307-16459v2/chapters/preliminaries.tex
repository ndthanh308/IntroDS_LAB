\section{Preliminaries}
\label{sec:preliminaries}

\textbf{Lifelong Learning (L3).} L3 consists of a series of $T$ tasks $\mathcal{T}_t \in \{\mathcal{T}_1, \mathcal{T}_2, \cdots, \mathcal{T}_T\} $, where each task $\mathcal{T}_t$ has it's own dataset $\mathcal{D}_t = \{\mathcal{X}_t, \mathcal{Y}_t\}$. 
In our experiments, $\mat{X}_i \in \mathcal{X}_t \subset \mathcal{X}$ denotes a medical image of size $W \times H$ and  $y_i \in \mathcal{Y}_t \subset \mathcal{Y}$ is its associated disease category at task $t$. In class-incremental L3, label space of two tasks is disjoint, hence $\mathcal{Y}_{t} \cap \mathcal{Y}_{t'} = \emptyset; t \neq t'$. The aim of L3 is to  train a model $f: \mathcal{X} \to \mathcal{Y}$ incrementally for each task $t$ to map the input space $\mathcal{X}_t$ to the corresponding target space $\mathcal{Y}_t$ without forgetting all previously learned tasks (\ie, $1, 2, \cdots, t-1$). We assume that a fixed-size memory $\mathcal{M}$ is available to store a subset of previously seen samples to mitigate catastrophic forgetting in L3. 
\\
\noindent
\textbf{Mixed-Curvature Space.} Mixed-Curvature space is formulated as the Cartesian product of fixed-curvature spaces and represented as $M = \bigtimes_{i=1}^C M_i^{d_i}$. Here, $M_i$ can be a Euclidean (zero curvature), hyperbolic (constant negative curvature), or spherical (constant positive curvature) space. %
Furthermore, $\bigtimes$ denotes the Cartesian product, and $d_i$ is the dimensionality of fixed-curvature space $M_i$ with curvature $c_i$. The distance in the mixed-curvature space can be decomposed as $d_M(\vec{x}, \vec{y}) \coloneqq \sum_{i=1}^{C} d_{M_i}(\vec{x}^i, \vec{y}^i)$.

\noindent
\textbf{Hyperbolic Poincar\'e Ball.} Hyperbolic space %
is a Riemannian manifold with negative curvature. The Poincare ball with curvature $-c,~ c>0$, $\mathbb{D}_c^n = \{\vec{x} \in \mathbb{R}^n : c\|\vec{x}\| < 1 \}$ is a model of $n$-dimensional hyperbolic geometry. To perform vector operations on $\mathbb{H}^n$, M$\ddot{\text{o}}$bius Gyrovector space is widely used. M$\ddot{\text{o}}$bius addition between $x \in \mathbb{D}_c^n$ and $y \in \mathbb{D}_c^n$ is defined as follows
\begin{align}
    \vec{x} \oplus_c \vec{y} = \frac{
            (1 + 2 c \langle \vec{x}, \vec{y}\rangle + c \|\vec{y}\|^2_2) \vec{x} + (1 - c \|\vec{x}\|_2^2) \vec{y}
            }{
            1 + 2 c \langle \vec{x}, \vec{y}\rangle + c^2 \|\vec{x}\|^2_2 \|\vec{y}\|^2_2
        }
\end{align}
Using M$\ddot{\text{o}}$bius addition, geodesic distance between two input data points, $\vec{x}$ and $\vec{y}$ in $\mathbb{D}_c^n$ is computed using the following formula.
\begin{align}
\label{eq:geodesic_h}
    d_c(\vec{x}, \vec{y}) = \frac{2}{\sqrt{c}}\tanh^{-1}(\sqrt{c}\|(-\vec{x})\oplus_c \vec{y}\|_2)
\end{align}
Tangent space of data point $\vec{x} \in \mathbb{D}_c^n$ is the inner product space and is defined as $T_x\mathbb{D}_c^n$ which comprises the tangent vector of all directions at $\vec{x}$. Mapping hyperbolic embedding to Euclidean space and vice-versa is crucial for performing operations on $\mathbb{D}^n$. Consequently, a vector $\vec{x} \in T_x\mathbb{D}_c^n$ is embedded onto the Poincar\'e ball $\mathbb{D}^n_c$ with anchor $\vec{x}$ using the exponential mapping function and 
and the inverse process is done using the logarithmic mapping function $\mathrm{log}_{\vec{v}}^c$ that maps $\vec{x} \in \mathbb{D}_c^n$ to the tangent space of $\vec{v}$ as follows 
\begin{align}
\label{eq:log_map}
    \log_{\vec{v}}^c(\vec{x}) = \frac{2}{\sqrt{c}\lambda_{\vec{v}}^c} \tanh^{-1}( 
            \sqrt{c} \|-\vec{v} \oplus_c \vec{x}\|_2)  \frac{-\vec{v} \oplus_c \vec{x}}{\|-\vec{v}\oplus_c \vec{x}\|_2}
\end{align}
where $\lambda_{\vec{v}}^c$ is conformal factor that is defined as $\lambda_{\vec{v}}^c=\frac{2}{1-c\|v\|^2}$. In practice, anchor $\vec{v}$ is set to the origin. Therefore, the exponential mapping is expressed as $\mathrm{exp}_0^c(\vec{x}) = \mathrm{tanh}\big(\sqrt{c}\|\vec{x}\|\big)\frac{\vec{x}}{\sqrt{c}\|\vec{x}\|}$.




\vspace{-2.0ex}