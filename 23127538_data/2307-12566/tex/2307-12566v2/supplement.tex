\documentclass[9pt]{osa-supplemental-document}

\usepackage{graphicx}
\graphicspath{ {./figures/} }
\usepackage{braket}
\usepackage{xfrac}

\setboolean{shortarticle}{false}

\definecolor{cola}{rgb}{0.7,0.1,0.1}
\definecolor{colb}{rgb}{0.9,0.4,0}
\definecolor{colc}{rgb}{0.3,0.7,0}
\definecolor{cold}{rgb}{0,0.35,0.75}
\definecolor{cole}{rgb}{0.63, 0.13, 0.94}
\definecolor{colf}{rgb}{0.5, 0.5, 0.5}

\newcommand{\maria}[1]{{\color{cola} #1}}
\newcommand{\kaimei}[1]{{\color{cole} #1}}
\newcommand{\checkthis}[1]{{\color{colb} #1}}
\newcommand{\notec}[1]{{\color{colc} #1}}
\newcommand{\noted}[1]{{\color{cold} #1}}

\newcommand{\vasilis}[1]{{\color{cole} #1}}
\newcommand{\sam}[1]{{\color{colc} #1}}

\newcommand{\Tone}{$T_1$}
\newcommand{\Ttwos}{$T_2^*$}

\newcommand{\langl}{\begin{picture}(4.5,7)
\put(1.1,2.5){\rotatebox{60}{\line(1,0){5.5}}}
\put(1.1,2.5){\rotatebox{300}{\line(1,0){5.5}}}
\end{picture}}
\newcommand{\rangl}{\begin{picture}(4.5,7)
\put(.9,2.5){\rotatebox{120}{\line(1,0){5.5}}}
\put(.9,2.5){\rotatebox{240}{\line(1,0){5.5}}}
\end{picture}}

\newcommand{\Do}{D\textsuperscript{0}}
\newcommand{\DoX}{D\textsuperscript{0}X}
\newcommand{\DoXs}{D\textsuperscript{0}X\textsuperscript{*}}
\newcommand{\transX}{\Do\,$\leftrightarrow$\,\DoX}
\newcommand{\transXTES}{\Do\,(2s or 2p)\,$\leftrightarrow$\,\DoX}
\newcommand{\transXs}{\Do\,$\leftrightarrow$\,\DoXs}

\newcommand{\up}{$\ket{\uparrow}$}
\newcommand{\down}{$\ket{\downarrow}$}
\newcommand{\uph}{$\ket{\Uparrow}$}
\newcommand{\downh}{$\ket{\Downarrow}$}

\newcommand{\AlZn}{Al\textsubscript{Zn}}
\newcommand{\GaZn}{Ga\textsubscript{Zn}}
\newcommand{\InZn}{In\textsubscript{Zn}}
\newcommand{\AlZns}{Al\textsubscript{Zn}\textsuperscript{*}}
\newcommand{\GaZns}{Ga\textsubscript{Zn}\textsuperscript{*}}
\newcommand{\InZns}{In\textsubscript{Zn}\textsuperscript{*}}
\newcommand{\ZnwNS}{\textsuperscript{67}Zn}


\title{Contributions to the optical linewidth of shallow donor --  bound excitonic transition in ZnO: Supplement}
\author{} %leave this blank
%% DO NOT ADD AUTHOR INFORMATION HERE; IT WILL BE ADDED DURING PRODUCTION

\begin{abstract}

% This template can be used to prepare a supplemental document for inclusion with submission to Optica Publishing Group’s journals.  This document, which may include supplementary information such as expanded descriptions of materials and methods, will be published as a PDF linked to the primary article. The supplemental file should only present information that would be useful and worthwhile for the reader, for example, details that would be necessary to reproduce an experiment. The article, however, must be coherent without the supplemental PDF file.  Please see the \href{https://opg.optica.org/submit/style/supplementary_materials.cfm}{Author Guidelines for Supplementary Materials} for more information. Supplementary documents are not copyedited and so should be prepared carefully with the template provided. Note that a coversheet with final article title, author block, publication date, journal branding, and other details will be added to your supplemental document during production. Do not include such details directly in this document.  Note that this template can be run from your own \TeX\ system or within the cloud-based \href{https://www.overleaf.com}{Overleaf} system. Please note that for Optica Open preprints, Optica Publishing Group cannot host supplementary material, but such material can be hosted elsewhere and be referenced by the preprint.
\end{abstract}

\setboolean{displaycopyright}{false} %copyright statement should not display in the  supplemental document

\begin{document}

\maketitle

\newpage

\section{Experimental setup and equipment}
\label{app:exp_setup}
% Figure environment removed

Fig.~\ref{fig:opt_path} depicts the optical paths of all three experiments described in the main text. 

In this work, we used three different laser systems for aboveband and resonant excitation. 
For near aboveband excitation we used a CNI MSL-F-360-10mW, a continuous wave (CW) laser emitting near 360\,nm with a maximum power of 10\,mW. 
For resonant excitation and precise frequency scanning control we used a SpectraPhysics Matisse-TS Ti:sapphire laser emitting near 739\,nm.
The Matisse is pumped by a SpectraPhysics Millenia EV CW DPSS green laser emitting at 532\,nm with a maximum power of 15\,W. 
The Matisse is then doubled near 368.5\,nm with a SpectraPhysics WaveTrain frequency doubler, which utilizes a unidirectional ring cavity (see Fig.~\ref{fig:opt_path}d) and yields a conversation efficiency between 4\% and 10\%.
For resonant excitation (pump laser) we use a Toptica DL pro HP laser emitting near 368.9\,nm with a maximum power of 30\,mW. 
The laser emission contains a long tail in the longer wavelength regime, which we cut by using our PL filter.

All of our optics and detectors are graded for use in the near-UV regime. Some notable equipment part numbers are listed below.
% Our mirrors are either MaxMirror (ultrabroadband mirrors) or Thorlabs BB1-E01.
% For 50/50 beam splitters, we either use Thorlabs BSW21, Newport UVBS14-1, or Thorlabs PBS051.
% Achromatic lenses are Newport PAC18AR.15, ashperic lens in cryostate is a Edmund Optics LightPath 354330, and fiber-coupler asheric lenses are Thorlabs C610TME-A, C560TME-A, A397TM-A, 
Our polarizers are either a Thoralabs A-coated Glan-Thompson GTH or a Thoralabs LPUV050.
% The waveplates we use are EKSMA Optics  464-4240 ($\lambda$/2), 464-4440 ($\lambda$/4) and YYY.
The PL filters we use are BrightLine single-band bandpass 370/6\,nm filters, while
the sideband filters we use are BrightLine single-band bandpass 380/14\,nm filters.
Our AOMs are Gooch \& Housego 3307-120 powered by a Gooch \& Housego 1300AF-DIFO-2.5.
% The photodiode we use is XXX.
Our single photon detector is a Laser Components Single Photon Counting Module COUNT BLUE COUNT-50B.
Our spectrometer is an Andor Shamrock 750 with a Netwon DU920P-UVB CCD and a turret with NIR gratings (we use the second order, which is not very efficient, but results in higher resolution).



\section{Excitation power correction}
\label{app:osc_cor}
% Figure environment removed

An energy dependent modulation of the transmittance, reflectance, and PLE signals is observed. This modulation is an artifact caused by a beamsplitter (BS) in the excitation path of our experimental setup. Reflection from this BS, $P_R$, is incident on the sample. Either transmission through the BS, $P_T$, or the power before the BS, $P_\mathrm{tot}$  is measured to monitor the power in each experiment.   
We find that the ratio $f$ of reflected power divided by the transmitted power fits the following relation,
\begin{equation}
    \label{eq:osc_corr_f}
    f\left(E, \phi \right) = P_R/P_T = c + A\sin\left(2\pi v E + \phi\right)
\end{equation}
with $A=0.07$, $v \simeq \left(0.18\,\,{\rm meV}\right)^{-1}$, $c=0.58$ and a phase $\phi$ which varies between experiments. By normalizing the data by $f$, we are able to remove the oscillations. Example of PLE data before and after the oscillation correction is shown in Fig.~\ref{fig:ref_corr}. 
This method, while effective, may sometimes not completely eliminate the observed oscillations. 
An example of this correction imperfection is the indium optical depth measurement depicted in Fig.~1d of the main text.


\section{Determination of Optical Depth from Transmission}
\label{app:od}
 
We measure total transmission $T$ as a function of wavelength in the frequency window around each \DoX\ resonance. Raw transmission data is corrected for a wavelength-dependent oscillation in the excitation power. This correction is discussed further in Sec.~\ref{app:osc_cor}. Off resonance, the transmission is 0.15 near the Al and Ga peaks and 0.3 near the In peak. Due to this off-resonant absorption, we neglect the effect of reflection from multiple surfaces throughout the measurement window and approximate the total transmission as
\begin{equation}
    T = T_F^2 e^{-\alpha d},
\end{equation}
where $\alpha$ is the absorption coefficient, $d$ is the sample thickness, and the product $\alpha d = \ln(T_F^2/T)$ is the optical depth (OD) plotted in Fig.~1d. $T_F$ is the single face transmission coefficient. To estimate T$_F$, we measure the reflectance between the Al and Ga peaks to be $R = 0.24 \pm 0.02$ and take $T_F \approxeq 1-R$ to be constant over the entire measurement range. 

% \newpage

\section{PL of Sample A in the transmission setup}

\label{app:PLforOD}
% Figure environment removed

The PL spectrum that corresponds to the transmission experiments in Fig.~1d is shown on Fig.~\ref{fig:pl_for_od}. In addition to being a different sample than that for the spectrum in Fig.~1b, the excitation spot diameter is almost three orders of magnitude larger. In Fig.~\ref{fig:pl_for_od}, the Al emission is $\approx \mathcal{O}\left(10\right)$ stronger than Ga emission which is consistent with the measured OD. 

\section{Donor Density Estimation}
\label{app:density_estimation}

The area under each optical depth (OD) peak in Fig.~1d is proportional to the number of donors in the probed ensemble, and thus can be used to estimate the average donor density.
The donor density of donor $N_x$ ($x$ = Al, Ga, In) can be estimated from~\cite{hilborn1982ecc}
\begin{equation}
\label{eq:donor_density}
    N_x= 8\pi \frac{g_{\Do}}{g_{\DoX}} \left(\frac{n}{\lambda_x}\right)^2\tau_{x, \mathrm{rad}}\int \alpha(\nu) d\nu ,
\end{equation}
where $g_i$ is the degeneracy of state $i$, $n$ is the index of refraction, $\lambda_x$ is the vacuum transition wavelength, $\tau_{x, \mathrm{rad}}$ is the zero-phonon-line radiative lifetime. 
We estimate the ZPL radiative lifetime as the total radiative lifetime determined by the slow decay component of the experimental lifetime, divided by the fraction of emission into the ZPL. This fraction is determined by the Huang-Rhys parameter~\cite{wagner2011bez} and the ratio of the two-electron satellite transitions to the phonon-assisted transitions. 
We obtain $\tau_{x,rad} = 0.95$\,ns, 1.18\,ns, and 1.52\,ns for $x=$ Al, Ga, and In, respectively. Using Eq.~\ref{eq:donor_density}, we find $N_{\rm Al} = 7.5\cdot 10^{15} \rm\, cm^{-3}$, $N_{\rm Ga} = 9.9\cdot 10^{14} \rm\, cm^{-3}$, and $N_{\rm In} = 7.4\cdot 10^{13} \rm\, cm^{-3}$. These values, measured for Sample A, are within an order of magnitude of the SIMS values measured for sample B. The relative concentrations are consistent with the PL spectra for this sample (Sec.~\ref{app:PLforOD}). 


\section{Temperature dependence of implanted In PLE linewidth}
\label{app:implanted_in_temp_dep}

% Figure environment removed

Sample B provided an opportunity to measure the PLE linewidth as a function of temperature for different In implantation fluences (Fig.~\ref{fig:temp_dep_implanted}). At the lowest studied fluences ($< 1.3\cdot 10^{11} $\,cm\textsuperscript{-2}) the dependence of linewidth on temperature closely matches the \textit{in-situ}-doped In in sample A. This lowest implantation density is estimated to be $\sim$50 times higher than the {\it in-situ}-doped density estimated in Sec.~3.1. This similarity supports that the temperature dependence described in Sec.~3.2 is intrinsic to the donor, and independent of environmental damage or defect density. As implantation dose increases further, however, both a broader linewidth and a steeper dependence of the linewidth on temperature is observed. The mechanism for the steeper dependence is unknown, but could be due to relaxation from the \DoX\ state to defect-states created during the implantation process.

\section{\texorpdfstring{\DoXs}~~Magneto-PL for Al, Ga, and In}
\label{app:DoXs_magneto_PL}

% Figure environment removed

The magneto-PL spectra for all three donors is shown in Fig.~\ref{app:DoXs_magneto_PL}. Unlike sample A, sample B has a substantial density of In, which facilitates the study of the weak In \DoXs\ line.
The Al \DoXs\ magnetic field dependence is identical in Sample A and B. The In \DoXs\ has a similar magnetic field dependence to Al, splitting into three lines in Voigt and into two lines in Faraday geometry. 
Unfortunately, the Al \DoX\ emission obscures the Ga \DoXs\ dependence on magnetic field. However, we can observe that the Ga \DoXs\ splits in two lines in the Faraday geometry, while the transition that is independent of the magnetic field is clearly visible in the Voigt geometry.

\section{\texorpdfstring{\Do}~~and \texorpdfstring{\DoX}\ ~State Models}
\label{app:state_models}

The models of the \Do\ and \DoX\ state effective mass envelope functions presented here are used to calculate the effect of the local nuclear spin and isotopic environments on the \transX\ transition. Both the neutral donor and donor-bound exciton effective mass envelope functions are assumed to be isotropic, which is a reasonable approximation as the dielectric constant, electron effective mass, and hole effective mass are all close to isotropic in ZnO~\cite{meyer2004bed}. The same isotropic assumption is made for Si. Si carrier effective masses are determined by conduction measurements~\cite{riffe2002tds}. The effective mass theory parameters used to calculate the carrier envelope functions are listed in Tables~\ref{table:ZnO} and \ref{table:Si}, where $m_0$ is the free electron mass, and $\epsilon_0$ is the permittivity of free space.

\begin{table}[h]
\centering
\begin{tabular}{cc}
 
 \hline
 Constant & Value \\
 % \hline 
 \hline
 electron effective mass & $m_{e}$ = 0.27 $m_0$~\cite{meyer2004bed, rossler1999iic} \\
 % \hline 
 hole effective mass & $m_{h}$ = 0.59 $m_0$ \cite{meyer2004bed, rossler1999iic}  \\
 % \hline 
 dielectric constant & $\epsilon_{\rm ZnO}$ = 8.2 $\epsilon_0$ \cite{meyer2004bed, rossler1999iic}  \\
 % \hline 
 Al \Do\ binding energy & $E_{\rm Al}$ = 51.5 meV \cite{wagner2011bez} \\
 % \hline 
 Ga \Do\ binding energy & $E_{\rm Ga}$ = 54.6 meV \cite{wagner2011bez}  \\
 % \hline 
 In \Do\ binding energy & $E_{\rm In}$ = 63.2 meV \cite{wagner2011bez}  \\
 
 \hline
 
\end{tabular}
\caption{ZnO effective mass theory parameters}
\label{table:ZnO}
\end{table}

\begin{table}[h]
\centering
\begin{tabular}{cc}
 

 \hline
 Constant & Value \\
 \hline 
 % \hline
 electron effective mass & $m_{e}$ = 0.26 $m_0$ \cite{riffe2002tds} \\
 % \hline 
 hole effective mass & $m_{h}$ = 0.33 $m_0$ \cite{riffe2002tds}  \\
 % \hline 
 dielectric constant & $\epsilon_{\rm Si}$ = 11.7 $\epsilon_0$ \cite{dunlap1953dmd}  \\
 % \hline 
 P \Do\ binding energy & $E_{\rm P}$ = 45.59 meV \cite{jagannath1981lee} \\
 
 \hline
 
\end{tabular}
\caption{Si effective mass theory parameters}
\label{table:Si}
\end{table}

\subsection*{\texorpdfstring{\Do}~~Model}

The neutral donor model presented here follows Ref.~\cite{heine1975tis}. The \Do\ electron radial envelope function is given by
\begin{equation}
    \Psi_{D^0,e} = \left(\frac{2^{2n}}{2 n a^3 \Gamma(2n + 1)} \right)^{1/2} r^{n-1} e^{-r/a}
    \label{eq:dowavefunction}
\end{equation}
where
\begin{equation}
    a = \left( \frac{\hbar^2}{2 m_e E_b} \right)^{1/2}, \quad \text{and }\quad n = \left( \frac{E_H}{E_b} \right) ^{1/2}.
\end{equation}
Here, $\Gamma$ is the gamma function, $m_e$ is the electron effective mass, $E_b$ is the donor binding energy, and $E_H$ is the hydrogenic effective mass binding energy. The hydrogenic binding energy is given by 
\begin{equation}
    E_H = \frac{m_e}{m_0} \frac{\epsilon_0^2}{\epsilon_{ZnO}^2} R_y,
\end{equation}
in which $m_0$ is the free electron mass, $\epsilon_0$ is the vacuum permittivity, and $R_y$ is the Rydberg energy. Eq.~\ref{eq:dowavefunction} assumes the hydrogenic approximation is nearly correct, but includes distinct central cell corrections between different impurity elements. 

\subsection*{\texorpdfstring{\DoX}~~Model}

A simple model for the bound exciton is proposed in Ref.~\cite{puls1983esb}, in which the two electrons in the bound exciton reside close to the positively-charged impurity while the positive hole binds to the net negative system consisting of the two electrons and the impurity. The potential binding the hole is given by
\begin{equation}
    V(r_h) = - \frac{e_c^2}{4\pi\epsilon_{ZnO} r_h}[1 - 2e^{-2r_h / a_e} (1 + r_h/a_e)].
    \label{eq:truePotent}
\end{equation}
Here $r_h$ is the radial position of the hole, while $a_e$ is the effective Bohr radius for the two electrons. The two electrons in the bound exciton are assumed to occupy the same 1-s orbital state. The resulting hole potential is not easily solvable, but can be approximated by the Kratzer potential,
\begin{equation}
    V(r_h) \approxeq - 2D \left( \frac{b}{r_h} - \frac{b^2}{2r_h^2}\right),
    \label{eq:kratzer}
\end{equation}
with $D = s e_c^2 / 8\pi\epsilon_{ZnO} a_e$, and $b = t a_e$, where $t$ and $s$ are fitting constants for the best fit of the Kratzer potential Eq.~\ref{eq:kratzer} to Eq.~\ref{eq:truePotent}, with $t = 1.337$, $s = 1.0136$~\cite{puls1983esb}. The solution to the Kratzer potential is calculated in Ref.~\cite{bayrak2007eas}. For modeling the ground state we set the rotational and radial quantum numbers $l_h$ = $n_h$ = 0. With these assumptions, the hole's radial envelope function in the bound exciton ground state is
\begin{equation}
    R(r) = r^{\Lambda_{00}} \text{exp} (-\epsilon r),
\end{equation}
in which, $\epsilon = \sqrt{-2m_h E_{h,00}/\hbar^2}$. The hole energy, $E_{h, n_h l_h}$, and factor $\Lambda_{n_h l_h}$ are defined as follows.
\begin{equation}
    \label{eq:hole_energy}
    E_{h, n_h l_h} = -\frac{(2 m_h b^2/\hbar^2) D^2}{ \left(1 + \Lambda_{00} \right)^{-2}}, \Lambda_{n_hl_h} = -\frac{1}{2} + n_h + \sqrt{\left(l_h + \frac{1}{2}\right)^2 + \frac{2 m_h b^2 D}{\hbar^2}}
\end{equation}
The total energy of the bound exciton, obtained from \cite{puls1983esb}, is Eq.~\ref{eq:aeEnergyPuls}, where the effective Bohr radius of the bound electrons is allowed to vary. The final term of Eq.~\ref{eq:aeEnergyPuls} is the hole energy, Eq.~\ref{eq:hole_energy}. To obtain the bound exciton ground state, Eq.~\ref{eq:aeEnergyPuls} is minimized with respect to the bound electron radius $a_e$, with hole quantum numbers $n_h$ = $l_h$ = 0. 
\begin{equation}
\begin{split}
    E(a_e) = 2E_g + 2R_D\left[\left(\frac{a_D}{a_e}\right)^2 - \frac{11}{8}\frac{a_D}{a_E}\right] \\ - 
    2R_D\left[\frac{s^2 t^2}{2} \frac{m_h}{m_e} \left(n_h + \frac{1}{2} + \sqrt{\left(l_h + \frac{1}{2} \right) + \frac{st^2 a_e}{a_D}\frac{m_h}{m_e}}  \right)^{-2} \right]
    \label{eq:aeEnergyPuls}
\end{split}
\end{equation}

Here $a_D$ is the effective Bohr radius of the neutral donor. To account for the central cell differences between donors in the \Do\ model, this is determined for each donor by $a_D = 2\langl r \rangl/3$, which is exact for perfectly hydrogenic 1s states. The resulting effective Bohr radii of the D$^0$X electrons are $a_{e\textrm{,Al}}$ = 2.08 nm, $a_{e\textrm{,Ga}}$ = 1.98 nm, $a_{e\textrm{,In}}$ = 1.75 nm, which result in values for $b$ in Eq.~\ref{eq:kratzer} of $b_{\textrm{Al}}$ = 2.8 nm, $b_{\textrm{Ga}}$ = 2.6 nm, $b_{\textrm{In}}$ = 2.3 nm. For the phosphorus donor in silicon, we obtain $a_{e\textrm{,P}}$ = 1.95 nm and $b_{\textrm{P}}$ = 2.6 nm.

\section{Discussion of \texorpdfstring{\DoXs}~~States}
\label{app:DoXs_discussion}

A model is proposed by Meyer {\it et al.} in Ref.~\cite{meyer2004bed, meyer2010esp} to explain the excited states of the bound exciton, including those discussed in Sec.~3.2. They introduce two components to this model, one for the lower energy excited states near 2 meV from the ground state, and another for the series of excited states found above 6 meV. Ref.~\cite{meyer2010esp} explains the excited states found around and above 6 meV by electronic excited states of the hole, given by increasing $n_h$ and $l_h$ in Eq.~\ref{eq:aeEnergyPuls} and minimizing with respect to $a_e$. The values for these excited states obtained from our model in Sec.~\ref{app:state_models} agree well with experimental values presented in Table III of Ref.~\cite{meyer2010esp}.

Ref.~\cite{meyer2010esp} attributes the set of energies found at around 2 meV to rotational-vibrational states of the bound exciton. Ref.~\cite{meyer2010esp} evaluates these energies using the Kratzer potential, as given in Eq.~\ref{eq:kratzer}, arriving at the following equation for the rotational-vibrational energies $E(\nu, J)$.

\begin{equation}
    \label{eq:ro-vib}
    E(\nu, J) = -\frac{(2mb^2/\hbar^2)D^2}{\left[\left(\nu + \frac{1}{2}\right) + \sqrt{\left(J+\frac{1}{2}\right)^2 + \left(\frac{2mb^2}{\hbar^2}\right)D}\right]^2}
\end{equation}
In analogy to rotational-vibrational states of diatomic molecules ({\it i.e.} as in Ref.~\cite{pliva1999cre}), Ref.~\cite{meyer2010esp} interprets $D$ as the bound exciton localization energy, $m$ as the effective mass of the hole, and $a$ as the distance between the hole and impurity. While Meyer \textit{et al.}'s interpretation of $a$ and $D$ yield excited states matching those observed around 2\,meV, to determine $b$, Ref.~\cite{meyer2010esp} introduces a \textit{pseudo-acceptor} model approximation for the donor bound exciton, in which the electrons are tightly bound to the impurity and the hole orbits the resulting negatively charged center at a distance equal to the approximate effective Bohr radius of the acceptor in ZnO, 0.8 nm. The bound exciton model in App~\ref{app:state_models} assumes the electrons are nearer to the impurity than the hole similarly to our approach in Sec.~\ref{app:state_models}. However, the energy of the system Eq.~\ref{eq:aeEnergyPuls} is minimized when $b \approxeq$ 2.3-2.8\,nm, far from 0.8\,nm, implying that the \textit{pseudo-acceptor} approximation is inappropriate. Meyer \textit{et al.}'s interpretation also deviates from the original source of this equation \cite{ruhle1978ebn}, where $D$ is the minimum of the Kratzer potential binding the hole and $a$ is the radial position of this minimum. Following the source interpretation of Eq.~\ref{eq:ro-vib}, Eq.~\ref{eq:ro-vib} is equal to the final term of Eq.~\ref{eq:aeEnergyPuls}, and both predict excited states above 6\,meV. Eq.~\ref{eq:ro-vib} and Eq.~\ref{eq:aeEnergyPuls} are only distinguished by the accuracy of their prediction of these states, with Eq.~\ref{eq:aeEnergyPuls}'s improved accuracy due to its inclusion of changes in $a_e$ between excited states of the hole. The interpretation by Meyer \textit{et al.} in analogy to rotational-vibrational states of diatomic molecules also assumes the g-factor of these excited states is equal to that of the main line, which is contradicted by our results in Sec.~3.2 and Sec.~\ref{app:DoXs_magneto_PL}.  

\section{Effect of the Nuclear-Spin Environment }

The nuclear-spin environment primarily affects the \Do\ levels in the \transX\ transition via the electron-nuclear contact hyperfine interaction.
The strength of the contact hyperfine interaction for \DoX\ will be much smaller as the two electrons form a spin-0 singlet state and the hole is predominantly p-type~\cite{preston2008bsz}. The \Do -bound electron interacts with both the lattice nuclear-spins and the donor's nucleus.

The broadening due to the non-zero \ZnwNS\ lattice spins is given by a Gaussian distribution of the hyperfine field, $\exp\left({-B^2/\Delta^2_{B,Zn}}\right)$, and has a dispersion~\cite{merkulov2002esr,linpeng2018cps}
$$
\Delta_{B,Zn} = \frac{\mu_0\mu_{\mathrm{Zn}}}{g_e}\sqrt{\frac{32}{27}}\sqrt{\frac{I_\mathrm{Zn}+1}{I_\mathrm{Zn}}}|u_\mathrm{Zn}
|^2 \sqrt{f\sum_i|\Psi(\vec{R}_i)|^4},$$
in which $g_e$ is the electron g-factor, $\mu_0$ is the vacuum permeability, $I_\mathrm{Zn} = 5/2$ is the \ZnwNS\ nuclear spin, $\mu_\mathrm{Zn} = 0.874\,\mu_N$ is the \ZnwNS\ nuclear magnetic moment in terms of the nuclear magneton $\mu_N$, $f=4.1\%$ is the natural abundance of \ZnwNS, $\Psi{(\vec R_i)}$ is the effective mass wavefunction at the $i$th Zn lattice site, and $|u_\mathrm{Zn}|^2$ is the ratio of the Bloch function density at the Zn site to the average Bloch function density. Ref.~\cite{linpeng2018cps} uses a pure hydrogenic effective mass wavefunction for $\Psi(R)$ with Bohr radius of 1.7\,\AA\ to estimate a 22\,MHz linewidth. Using a slightly modified $\Psi(R)$\  which accounts for the central cell correction due to the three different types of donor (Sec.~\ref{app:state_models}), we obtain \ZnwNS\ nuclear-spin broadened lines of 22, 24 and 29\,MHz for Al, Ga and In, respectively. These values are only slightly larger than the reported optically-detected magnetic resonance linewidths for Ga and In of 19 and 22 MHz respectively~\cite{gonzalez1982mrs}. 

The hyperfine splitting due to the donor nuclear spin of spin $I$ depends on the ratio of the Zeeman splitting $g\mu_B B$ to the hyperfine constant $A$~\cite{mohammady2012aqc}. 
At 0-field, the $2(2I+1)$ states split into two sublevels separated by $A\sqrt{\frac14 +I(I+1)}$. 
At high magnetic field, $g_e\mu_BB \gg A$, each electron Zeeman level splits into (2I + 1) lines with a splitting of $A/2$ between hyperfine levels. 
$A_{\mathrm{Al}}=1.45$\,MHz~\cite{orlinskii208isa}, $A_{\mathrm{Ga}}=11.5$\,MHz~\cite{gonzalez1982mrs}, and $A_{\mathrm{In}}= 100$\,MHz~\cite{gonzalez1982mrs, block1982odm} with $I_{\mathrm{Al}} = 5/2$, $I_{\mathrm{Ga}} = 3/2$ and $I_{\mathrm{In}}=9/2$. 

The nuclear-spin environment contributes to the inhomogeneous broadening of the \transX\ transition linewidth.
However, any experiment conducted over timescales longer than the relaxation time of the nuclear-spin states will be affected by this broadening. Comparing to the several GHz linewidths observed in Fig.~4d, the nuclear-spin contribution is negligible. However, it is the main source of donor ground-state dephasing~\cite{linpeng2018cps}, and significantly affects the two-laser linewidth in coherent population trapping experiments~\cite{wang2023pdq, viitaniemi2022csp}. For In, the hyperfine splitting is comparable to the lifetime-limited linewidth and could theoretically be optically resolved. 

\section{Impurity Isotope Effect}

The shift $\Delta E^{don}_{S,c}$ for state S and carrier c due to substitution of the donor atom isotope is given by Heine and Henry in Ref.~\cite{heine1975tis} as 
\begin{equation}
    \Delta E^{don}_{S,c} =
    \frac{2 \hbar \omega_D}{5}\left(\frac{M_0}{M}\right)^{1/2} \frac{\Delta M}{M} \frac{\gamma_c}{\gamma} \left(-\frac{dE_g}{dkT}\right)_{HT} P_{S,c}
    \label{eq:donorisotope}
\end{equation}
in which $\omega_D$ is the Debye frequency of ZnO, 
$M_0$ is the average substituted atom (Zn) mass, 
$M$ is the mass of the lightest donor isotope, 
$\Delta M$ is the difference between the mass of the heavier and lightest donor isotope, and
$\left(\sfrac{dE_g}{dkT}\right)_{HT}$ is the temperature dependence of the band gap at high temperature.
These values are provided in Table~\ref{table:ZnOdonorAtom} in Sec.~\ref{app:isotope_shift}. 
$\gamma_c$ is the fractional oscillator force constant reduction due the presence of a carrier $c$,
$\gamma$ = $\gamma_e$ + $\gamma_h$, is the sum of the force reduction for holes and electrons, with $\gamma_h$ = 3$\gamma_e$~\cite{heine1975tis}, and $P_{S, c}$ is the average volume per atom multiplied by the average amplitude squared within a sphere encompassing the donor and its nearest neighbors (0.2 nm in ZnO).

\begin{table}[h]
\centering
\begin{tabular}{cc}
 \hline 
 Constant & Value \\
 \hline 
 % \hline
 $\hbar\omega_d$ (meV) & 35.8 ($T_D$ = $416$\,K) \cite{rossler1999iic} \\
 % \hline 
 Ga: $M_0$, $M$, $\Delta M$ (amu) & 69, 71, 2 \cite{cardona2005ieo}  \\
 % \hline 
 In: $M_0$, $M$, $\Delta M$ (amu) & 113, 115, 2 \cite{cardona2005ieo} \\
 % \hline 
 $\left(\sfrac{dE_g}{dkT}\right)_\text{HT}$ (meV/K) & 3.24 at HT 400\,K \cite{rossler1999iic} \\
 
 \hline
 
\end{tabular}

\caption{ZnO donor atom isotopic substitution constants}
\label{table:ZnOdonorAtom}
\end{table}

The shift from impurity atom isotopic substitution is determined by Eq.~\ref{eq:donorisotope} with constants given in Table~\ref{table:ZnOdonorAtom}. Both aluminum and phosphorus have only one commonly occurring isotope and are thus not included in Table~\ref{table:ZnOdonorAtom}. Impurity isotope substitution shifts the \transX\ transition energy by only 16 MHz between Ga$^{69}$ and Ga$^{71}$ and 13 MHz between In$^{113}$ and In$^{115}$, implying that variation in Zn and O isotopes in the defect environment are the dominant isotopic broadening effect. 

\section{Isotopic Environment Simulation}
\label{app:isotope_shift}

\begin{table}[h]
\centering
\begin{tabular}{ccc}

 \hline
 Energy, $W_{i,c}$ & Atom/Isotope & Rel. Abundance \\
 \hline 
 \hline
 \multicolumn{3}{c}{Valence Band, $W_{i,h}$} \\
 \hline
 \hline
 0 meV & $^{64}$Zn & 48.6 $\%$ \\
 % \hline
 0.66 meV & $^{66}$Zn & 27.9 $\%$\\
 % \hline
 0.98 meV & $^{67}$Zn & 4.1 $\%$\\
 % \hline
 1.31 meV & $^{68}$Zn & 18.8 $\%$\\
 % \hline
 1.97 meV & $^{68}$Zn & 0.6 $\%$ \\
 % \hline
 \hline
 0 meV & $^{16}$O & 99.75 $\%$ \\
 % \hline
 2.50 meV & $^{17}$O & 0.05 $\%$\\
 % \hline
 5.00 meV & $^{18}$O & 0.2 $\%$\\
 % \hline
 
 \hline 
 \hline
 \multicolumn{3}{c}{Conduction Band, $W_{i,e}$} \\
 \hline
 \hline
 0 meV & $^{64}$Zn & 48.6 $\%$ \\
 % \hline
 0.16 meV & $^{66}$Zn & 27.9 $\%$\\
 % \hline
 0.25 meV & $^{67}$Zn & 4.1 $\%$\\
 % \hline
 0.33 meV & $^{68}$Zn & 18.8 $\%$\\
 % \hline
 0.49 meV & $^{68}$Zn & 0.6 $\%$ \\
 % \hline
 \hline
 0 meV & $^{16}$O & 99.75 $\%$ \\
 % \hline
 0.65 meV & $^{17}$O & 0.05 $\%$\\
 % \hline
 1.25 meV & $^{18}$O & 0.2 $\%$\\
 \hline
 
\end{tabular}

\caption{Isotopic perturbation energies in ZnO. Isotopic abundances are obtained from~\cite{cardona2005ieo}.}
\label{table:ZnOPerturb}
\end{table}


For Al, Ga, and In defects in ZnO, the isotopic environment is simulated up to 10\,nm from the defect, which is sufficient to include the \Do\ and \DoX\ states. The shifts $W_{i,c}$ from environmental zinc and oxygen are calculated using the dependence of the donor lines on isotopic substitution, which is measured in Ref.~\cite{manjon2003zws} to be $dE_{\DoX}$/$dM_{Zn} = 0.41 \pm 0.05$\,meV/amu for zinc, and $dE_{\DoX}$/$dM_{O} = 3.12 \pm 0.05$\,meV/amu for oxygen. Adopting our 80$\%$ valence and 20$\%$ conduction band shift assumption~\cite{gorkavenko2007ctd}, we obtain 
\begin{equation}
    W_{i,c} = S_{c} \Delta M \left(\frac{dE_{D^0X}}{dM}\right).
    \label{eq:isoShiftDep}
\end{equation}
Here $S_c$ is the fraction of the total band gap shift due to the valence band or conduction band; if the carrier in question is a hole, the energy shift will be due to the shift in the valence band, if it is an electron, the shift will be due to the shift in the conduction band ($S_h$ = 0.8, $S_e$ = 0.2). $\Delta M$ is the difference between the lightest isotope and a given environmental isotope in amu, and $dE_{\DoX}$/$dM_{elem}$ is the environmental mass dependence of the $D^0X$ transition energy for element $elem$. The values for environmental isotopic perturbation in ZnO as obtained from Eq.~\ref{eq:isoShiftDep} are given in Table~\ref{table:ZnOPerturb}. 

\begin{table}[!ht]
\centering
\begin{tabular}{ccc}
 
 \hline
 Energy, $W_{a,b}$ & Atom/Isotope & Rel. Abundance \\
 
 \hline 
 \hline
 \multicolumn{3}{c}{Valence Band, $W_{i,h}$} \\
 \hline
 \hline
 0 meV & $^{28}$Si & 92.2 $\%$ \\
 % \hline
 0.76 meV & $^{29}$Si & 4.7 $\%$\\
 % \hline
 1.52 meV & $^{30}$Si & 3.1 $\%$\\
 % \hline
 
 \hline 
 \hline
 \multicolumn{3}{c}{Conduction Band, $W_{i,e}$} \\
 \hline
 \hline
 0 meV & $^{28}$Si & 92.2 $\%$ \\
 % \hline
 0.26 meV & $^{29}$Si & 4.7 $\%$\\
 % \hline
 0.52 meV & $^{30}$Si & 3.1 $\%$\\
 \hline
 
\end{tabular}

 \caption{Isotopic perturbation energies in Si. Isotopic bundances are obtained from~\cite{cardona2005ieo}.}
 \label{table:SiPerturb}
\end{table}


For the phosphorus defect in Si, the isotopic environment is simulated up to 12 nm (as it is slightly shallower). The breakdown of the band gap shift between the conduction and valence band shifts is noted in~\cite{karaiskaj2003ora} as 75$\%$ valence band and 25$\%$ conduction band. $dE_{\DoX}$/$dM_{Si}$ = 1.02\,meV, which is obtained from~\cite{cardona2005ieo}. From Eq.~\ref{eq:isoShiftDep}, we find the isotopic perturbation energies shown in Table~\ref{table:SiPerturb}.

The isotopic environment's contribution to the linewidth of the phosphorus shallow donors in Si's bound exciton transition has been measured in high quality natural silicon~\cite{yang2009hlp} as 1.1\,GHz.
To verify our model we simulated this transition, yielding a broadening of 0.9\,GHz (see Fig.~\ref{fig:silicon_isotope}), in good agreement with the experimentally observed value.
The primary source of discrepancy is likely found in the simplified \DoX\ state model. Given the difficulty in obtaining an accurate model of the donor bound exciton state, such discrepancies are expected. Despite the uncertainty regarding the \DoX\ state model, the agreement between the model and the experimental data supports the validity of our approach. 

% Figure environment removed

\section{Investigating power broadening of spectral anti-hole burning}
\label{app:rshb_power_dep}
% Figure environment removed
We additionally performed continuous-wave spectral anti-hole burning experiments as a function of pump pulse power. These measurements are performed in the Voigt geometry at 6\,T and 1.8\,K. The pump laser is resonant on the $V_\uparrow$ transition, while the probe laser is scanned near the $V_\downarrow$ and $H_\downarrow$ transitions as shown in Fig.~\ref{fig:rSHB_voigt}a. In this geometry, two anti-holes are observed, split by the hole Zeeman factor (Fig~\ref{fig:rSHB_voigt}b). As shown in Fig~\ref{fig:rSHB_voigt}c, the anti-hole linewidth does not depend on pump power. These results indicate the anti-hole linewidth is already maximally broadened by the lowest probe intensities in these steady-state experiments and/or is dominated by the 290\,nW probe laser. 

\section{Delay dependence of two-laser transient spectroscopy}
\label{app:delay_dep}

% Figure environment removed


We performed additional transient pump-probe experiments as a function of wait time $\tau_w$ between the pump and probe lasers, for two different probe excitation energies, on-resonance with the Al \DoX\ peak, and +5.5\,GHz off-resonance.
Fig.~\ref{fig:delay_dep} depicts the PL intensity integrated over the first two microseconds of the probe pulse. 
We observe the same PL intensity for wait times smaller than 0.1\;ms, while,
After 0.1\;ms, the population of the \down\ state starts to deplete due to the longitudinal spin relaxation ($T_1 = 1.5$\;ms), as described in~\cite{niaouris2022esr}.
The near-constant PL intensity over wait times shorter than \Tone\ indicates that the process governing the homogeneous broadening does not occur while the probe laser is off, but occur under optical excitation. 

\newpage
\bibliography{supplement.bib}


\end{document}