\documentclass[12pt,a4paper,twoside]{article}
\usepackage[utf8]{inputenc}
\usepackage[english]{babel}
\usepackage[margin=3cm]{geometry}

\usepackage{hyperref}
\usepackage{graphicx}
\usepackage{times}
\usepackage{fancyhdr}
\usepackage{epsfig}
\usepackage{multirow}
\usepackage{amssymb,amsfonts,amsmath}
\usepackage{nameref}
\usepackage{csquotes}
\usepackage{xcolor}

% custom commands
\usepackage{graphicx}
\newcommand{\pan}[1]{\left\langle #1\right\rangle}
\newcommand{\de}[1]{\!\operatorname{d}\!{#1}}
\newcommand{\op}[1]{\mathbf{#1}}
% for notes
\usepackage{color}
\newcommand{\note}[1]{{\color{red} \{Note: #1\}}}
\newcommand{\SI}{Supplementary Information}

% Redefine the symbols for real and imaginary parts
\let\Re\relax
\let\Im\relax
\DeclareMathOperator{\Re}{\mathrm{Re}}
\DeclareMathOperator{\Im}{\mathrm{Im}}

\linespread{1.3}


%%% tell texcount to ignore the content of \note.
%TC:macro \note [ignore]

\usepackage[footnotesize,bf]{caption}
\renewcommand{\captionlabelfont}{\sffamily\mdseries\upshape}
\renewcommand{\captionfont}{\sffamily\mdseries\upshape}
\DeclareCaptionLabelFormat{naturelabel}{\textbf{Fig.#2$\mid$}}
\captionsetup[figure]{labelformat=naturelabel}
\captionsetup[figure]{font=footnotesize}
\captionsetup[figure]{labelsep=none}

%%%%%%%%%%%%%%%%
%% SI 

\usepackage{pdfpages} % include pdfs
\usepackage{pgffor} % for loops

% Fix for a pdfpages rotation bug with revtex
\makeatletter
\AtBeginDocument{\let\LS@rot\@undefined}
\makeatother

% the name of the supplement PDF file
\def\supplementfilename{SI.pdf}

% Determine the number of pages 
% in the supplement file and store
\pdfximage{\supplementfilename}
\def\numbersupplementpages{\the\pdflastximagepages}

% Are we submitting to the arXiv? 
\newif\ifarXiv
\arXivtrue 
%\arXivfalse


%%%%%%%%%%%%%%%%
%% BIBLATEX

\usepackage[style=nature,sorting=none,url=false]{biblatex}
\addbibresource[]{biblio.bib}

%fancy
\usepackage{fancyhdr}
\renewcommand{\footrulewidth}{0.4pt}

\begin{document}
\pagestyle{fancy}
\setlength{\headheight}{16pt}
\fancyhead{}
	
%TC:ignore

\renewcommand{\multirowsetup}{\centering}


\thispagestyle{empty} % title page: empty style

\begin{center}
	{\Large\bfseries\sffamily
	Estimates of the reproduction ratio from epidemic surveillance may be biased\\in spatially structured populations
\\
	}
	\vspace{1cm}\large
	{Piero Birello\textsuperscript{1,+},
        Michele Re Fiorentin\textsuperscript{2},
	 Boxuan Wang\textsuperscript{1},\\
	 Vittoria Colizza\textsuperscript{1}, and
	Eugenio Valdano\textsuperscript{1,*}.
 	}\\
	\vspace{1.1cm}
	{\footnotesize \textit{
		\textsuperscript{1}Sorbonne Universit\'{e}, INSERM, Institut Pierre Louis d'Epid\'{e}miologie et de Sant\'{e} Publique,\\ F75012, Paris, France.\\
		\textsuperscript{2}Department of Applied Science And Technology (DISAT), Politecnico di Torino, C.so Duca degli Abruzzi 24, 10129, Torino, Italy.\\
		\textsuperscript{+} Department of Mathematical Sciences ``Giuseppe Luigi Lagrange'' (DISMA), Politecnico di Torino, C.so Duca degli Abruzzi 24, 10129, Torino, Italy.
\\
		\textsuperscript{*} Corresponding author} \texttt{eugenio.valdano@inserm.fr}
		\\
	}
\end{center}

%\vspace{2.2cm}
\vspace{1.1cm}

%TC:endignore

%\newpage


%%%
%%% BODY OF THE PAPER
%%%

\begin{center}\sffamily\large\bfseries
    Abstract
\end{center}

An accurate and timely estimate of the reproduction ratio $R$ of an infectious disease epidemic is crucial to make projections on its evolution and set up the appropriate public health response.
Estimates of $R$ routinely come from statistical inference on timelines of cases or their proxies like symptomatic cases, hospitalizatons, deaths.
Here, however, we prove that these estimates of $R$ may not be accurate if the population is made up of spatially distinct communities, as the interplay between space and mobility may hide the true epidemic evolution from surveillance data. This means that surveillance may  underestimate $R$ over long periods, to the point of mistaking a growing epidemic for a subsiding one, misinforming public health response.
To overcome this, we propose a correction to be applied to surveillance data that removes this bias and ensures an accurate estimate of $R$ across all epidemic phases.
We use COVID-19 as case study; our results, however, apply to any epidemic where mobility is a driver of circulation, including major challenges of the next decades: respiratory infections (influenza, SARS-CoV-2, emerging pathogens), vector-borne diseases (arboviruses).
Our findings will help set up public health response to these threats, by improving epidemic monitoring and surveillance.
%\end{center}

%\vspace{1cm}
\newpage



%TC:ignore

%%%%%%%%%
%% INTRO
%%%%%%%%%
%\section*{Introduction}
\begin{center}\sffamily\large\bfseries
    Main text
\end{center}

\noindent The reproduction ratio $R$ is arguably the most used indicator to monitor the trend in the evolution of an infectious disease epidemic.
$R$ is the average number of secondary cases that each case generates: When it is larger than one, the epidemic wave is growing; when instead it is lower than one, it is subsiding~\cite{keeling_modeling_2007,nishiura_effective_2009}.
The reproduction ratio also measures the effectiveness of public health interventions, whose overarching goal is to bring an unconstrained epidemic ($R>1$) below the epidemic threshold of $R=1$: Accurately estimating the reproduction ratio is thus necessary to ascertain the current epidemic evolution, predict short-term trends, perform scenario analysis and plan public health action~\cite{wallinga_optimizing_2010,ridenhour_unraveling_2018,thompson_control_2018,dhillon_getting_2020,pan_association_2020}.
The standard way to measure $R$ is to infer it from data coming from epidemiological surveillance~\cite{wallinga_how_2007,davoudi_early_2012,obadia_r0_2012,cori_new_2013,thompson_improved_2019}.
These data may be timelines of detected cases or their proxies, like hospitalizations or deaths, and this approach applies to diseases spanning radically different epidemiology, transmission routes and burden, like influenza~\cite{biggerstaff_estimates_2014,thompson_global_2022}, measles~\cite{guerra_basic_2017}, COVID-19~\cite{li_temporal_2021}, Ebola~\cite{maganga_ebola_2014}, cholera~\cite{mukandavire_estimating_2011}, dengue~\cite{codeco_estimating_2018}, malaria~\cite{routledge_estimating_2018}.
The resulting surveillance-based estimates of $R$ are routinely used to design interventions~\cite{noauthor_introducing_2021}: Notwithstanding, we argue in this study that surveillance data may lead to biased estimates of the reproduction ratio in spatially structured populations, where geographically distinct communities (e.g., cities) are connected though human mobility.
We will show that the complex interplay between spatial heterogeneities in transmissibility and the mixing network driven by human mobility hide the true dynamic structure of the epidemic process from population-level surveillance data.
This mirrors the nature of most mathematical models of epidemic spread: they integrate space and spatial data at high resolution~\cite{hufnagel_forecast_2004,balcan_phase_2011,pastor-satorras_epidemic_2015,soriano-panos_spreading_2018,gomez-gardenes_critical_2018,chang_mobility_2020}, but they find it harder to do the reverse, which is extracting high-resolution information from limited and coarse-grained surveillance data in the absence of knowledge of the underlying spatial dynamics is~\cite{coletti_shifting_2018,scarpino_predictability_2019,castro_turning_2020}.
Crucially, this means that inference on surveillance data may either overestimate or underestimate it over long periods.
This is of great public health relevance: measuring for instance a reproduction ratio below one when the true value is above would falsely signal that the epidemic is under control.
Here, we study this bias, identify its origin and compute its magnitude.
Then, we propose a correction to case incidence data that removes this bias and ensures that surveillance-based estimates of the reproduction ratio consistently give the true reproduction ratio of the epidemic. 
Our theoretical findings apply to any epidemic featuring relatively short generation time and for which mobility is a contributing factor in shaping its circulation within and across communities.
This covers some of the global health threats that are being worst affected by climate change and demographic trends:
viruses responsible for respiratory infections --~including SARS-CoV-2 and influenza~--~\cite{li_trends_2022}, vector-borne pathogens --~including the arboviruses dengue, chikungunya, Zika~\cite{messina_current_2019,romanello_2022_2022}, and emergence events of new viruses or new viral strains~\cite{carlson_climate_2022}.
To test and illustrate our findings, we use the French COVID-19 epidemic (see Fig.~\ref{fig:fig1}) before the advent of vaccination as a case study. 

%%%%%%%%%
%% THEORY
%%%%%%%%%

%\section*{Theory}

\section*{Theoretical formalism}

The Galton--Watson branching process is a customary framework to model epidemic spread~\cite{Watson,lloyd-smith_superspreading_2005,hellewell_feasibility_2020}. Let $I(0)$ be the initial number of cases, $I(1)$ the expected number of cases that the initial cases generate, and, generally, let $I(t)$ be the expected number of cases in the $t$-th generation. By definition of the reproduction ratio, we have that $I(t)=R I(t-1)$, which implies that $I(t)=R^{t}I(0)$.
This equation means that the number of cases grows exponentially if $R>1$.
In any real outbreak other factors, like acquired immunity, seasonal effects or public-health interventions, will at some point curb this exponential growth by changing the value of $R$.
Notwithstanding, we may assume $R$ to be fairly constant either in the early phase of an outbreak, when those effects have not yet kicked in, or when the timescale at which immunity and mixing change is much longer than epidemic evolution~\cite{kucharski2016effectiveness,de_meijere_attitudes_2023}.

In the case of a population composed of $N$ spatial communities, we may define the vector $\op{I}(t)\in\mathbb{R}^N$, whose component $I(t)_{i}$ is the number of cases in generation $t$ and community $i$. Likewise, the {\itshape reproduction operator} $\op{R}\in\mathbb{R}^{N,N}$ encodes, in its component $R_{ij}$, the average number of cases generated among the residents of community $i$, by a case belonging to community $j$ \cite{Susswein}.
This definition of $\op{R}$, and the results that we are going to derive from it, applies to any epidemic and disease.
The specific parametrization of $\op{R}$ will instead depend on the specific transmission dynamics and natural history of the disease: for directly-transmitted diseases $\op{R}$ typically depends on mixing patterns among communities~\cite{mazzoli_projecting_2021}; for vector-borne diseases the local abundance of the host vectors, modulating the effective transmissibility, needs to be factored in, too~\cite{messina_current_2019,jourdain_importation_2020}.
The expected epidemic evolution then follows the equation
%
\begin{equation}
    \op{I}(t) = \op{R}^t\op{I}(0).
    \label{eq:Ievol}
\end{equation}
%
$\op{I}(t)$ encodes both the total number of cases in the population in generation $t$ and its spatial distribution. We define the former as the number $I_{tot}(t)=\sum_i I(t)_{i}$ and the latter as the vector $\op{x}(t)\in\mathbb{R}^N$ whose components are $x(t)_{i} = I(t)_{i} / I_{tot}(t)$.

The reproduction ratio $R$ of this process is the spectral radius of $\op{R}$ (i.e., the largest among the absolute values of its eigenvalues)~\cite{diekmann_definition_1990}, which is itself also a (nondegenerate) eigenvalue, because $\op{R}$ is by definition nonnegative and can be assumed irreducible (see Supplementary Methods Section 1.3) so that the Perron-Frobenius theorem holds~\cite{horn_matrix_1990}.
We also define $\op{v}$ as the Perron (right) eigenvector associated with $R$. $\op{v}$ is strictly positive ($v_i>0$) and we normalize it so that $\sum_i v_i=1$.

Measuring the true reproduction ratio of the system thus requires knowledge of the spectral structure of $\op{R}$, i.e., of the spatial structure of the epidemic. Surveillance instead measures the reproduction ratio from the evolution of the incidence of infections or their proxies. 
This may happen globally, at the level of the entire population, or locally in each community.
In our framework, the population-level observed reproduction ratio is $S(t) = I_{tot}(t+1)/I_{tot}(t)$, i.e., the generational growth rate. The local community-level observed reproduction ratio is instead $s_i(t) = I(t+1)_i/I(t)_i$.

A simple observation then underpins our study: in general $S(t)$ and $s_i(t)$ may be different from $R$, the spectral radius of $\mathbf{R}$, and, if that is the case, surveillance will not measure the true reproduction ratio.

To explore this, we will first determine the conditions leading to an unbiased measure of the reproduction ratio: $S(t)=R$.



\section*{When the true and observed reproduction ratios match}
%
By virtue of the Perron-Frobenius theorem, $\op{R}^t\rightarrow R^t \op{v} \op{v}^*$ asymptotically at large $t$, where $\op{v}^*$ is the dual of $\op{v}$ (easily computable as the left Perron eigenvector of $\op{R}$) and normalized so that $\op{v}^*\op{v}=1$.
Asymptotically then equation~\eqref{eq:Ievol} becomes $\op{I}(t)\rightarrow \left[\op{v}^* \op{I}(0)\right] R^t \op{v}$, which implies that $\op{x}(t)\rightarrow \op{v}$, $S(t)\rightarrow R$ and $s_i(t)\rightarrow R$.
The epidemic dynamics thus brings the spatial distribution of cases toward $\op{v}$, which we will refer to as the {\itshape equilibrium spatial distribution of infections}.
Thus, for any epidemic dynamics, if cases are spatially distributed as the equilibrium distribution ($\op{x} = \op{v}$), then the error is zero and the true reproduction ratio is measured both globally $(S=R)$ and locally $s_i=R$.
Fig.\ref{fig:fig1}\textbf{b} shows evidence of the convergence to $\op{v}$ during the COVID-19 epidemic in France in late 2020 and early 2021. We used mobility data from Meta~\cite{Colocation_Maps}, a multinational technology company, to estimate $\op{R}$ for the 94 departments of mainland France, excluding Corsica (see \nameref{sec:Rfromdata}). We reconstructed $\op{x}$ from surveillance data released by the French public health authority (see Supplementary Methods Section 1.1). In a period when $\op{R}$ was fairly constant (required by our formalism) the angle between $\op{x}$ and $\op{v}$ consistently decreased. Such angle, however, never got to zero because then $\op{R}$ changed, and, consistently, the equilibrium distribution $\op{v}$ changed. The description of the whole course of an epidemic wave indeed requires a time-varying $\op{R}$ and that is beyond the scope of this study. Locally in time, however, in periods during which $\op{R}$ is fairly constant, the system will evolve towards the equilibrium distribution determined by the Perron eigenvector of $\op{R}$ at that time.

% Figure environment removed

But there exists a class of operators $\op{R}$ for which the error is globally zero even out of equilibrium ($\op{x}\neq\op{v}$). First, let us rewrite the observed reproduction ratio in matrix form as
%
\begin{equation}
    S(t) = \frac{I_{tot}(t+1)}{I_{tot}(t)} = \op{F}^T \op{R} \op{x}(t),
    \label{eq:S}
\end{equation}
%
where we introduced $\op{F}$ as the unit column vector ($F_i=1\,\forall i$). If we assume that $\op{v}^*=\op{F}^T$ ($v^*_i=1$) then we can apply $\op{R}$ leftwards in equation~\eqref{eq:S} and get $S(t)=R$ at any time and for any spatial distribution $\op{x}$. Now, the requirement $\op{v}^*=\op{F}^T$ imposes that $\op{R}$ is proportional to a left-stochastic matrix: indeed $\op{F}^T \op{R}=R \op{F}^T$ means $\sum_j R_{ji}=R$, so that each column sums to $R$. $r_i\equiv\sum_j R_{ji}$ is by definition the expected number of secondary cases generated by a case resident of $i$ regardless of where they are generated. If $r_i$ is constant, every case, anywhere, has the same overall {\itshape transmission potential}: $r_i=R\; \forall i$. If this is the case, the observed reproduction ratio is unbiased regardless of the spatial epidemic coupling among communities.
This implies that only the combination of spatial epidemic coupling and spatial heterogeneity in transmission potential may cause a global difference between the observed and the true reproduction ratios. Notably, locally-measured reproduction ratios may instead differ from $R$ even in the case $\op{v}^*=\op{F}^T$.

\section*{When the true and observed reproduction ratios do not match}

% Figure environment removed


We now focus on the out-of-equilibrium dynamics ($\op{x}(t)\not=\op{v}$) and measure the bias in the estimate of $R$ as the relative difference between the observed and the true reproduction ratios
%
\begin{equation}
    \Delta(t) = \frac{S(t)-R}{R}. 
    \label{eq:delta1}
\end{equation}
%
We call $\Lambda_\alpha$ ($\alpha=1,\cdots,N-1$) the (possibly degenerate) eigenvalues of $\op{R}$ other than $R$ and, by Perron-Frobenius theorem, $|\Lambda_\alpha|<R$.
With calculations reported in \nameref{sec:delta_calc}, we find that 
%
\begin{equation}
    \Delta(t) = C(t) \sum_\alpha z_\alpha \left(1-\frac{\Lambda_\alpha}{R}\right) \left( \frac{\Lambda_\alpha}{R} \right)^t,
    \label{eq:delta2}
\end{equation}
%
where $C(t)$ is positive and asymptotically constant, and $z_\alpha$ is a (possibly complex) number proportional to the scalar product between $\op{F}$ and the projection of the initial condition $\op{x}(0)$ on the $\alpha$-th mode.
The modes in equation~\eqref{eq:delta2} for which $\Lambda_\alpha\approx R$, or that are almost orthogonal to the initial configuration $\op{x}(0)$, are suppressed from the start and do not bias the estimate of the reproduction ratio.
The other modes, instead, possibly bias the reproduction ratio with an effect that becomes smaller as the epidemic evolves, with a characteristic decay time $\tau_{\alpha}=1/\log\left(R/|\Lambda_\alpha|\right)$.
In addition, those modes for which $\Lambda_\alpha$ is not real and positive have an oscillating term. Specifically, if $\Lambda_\alpha$ has a nonzero imaginary part, then its complex conjugate is also an eigenvalue and their combined contribution oscillates with period $T_{\alpha}=2\pi/|\theta_\alpha|$, where $\theta_\alpha=\arg \Lambda_\alpha$ (with $\theta_\alpha\in(-\pi,\pi]$). This also holds for negative eigenvalues ($\theta_\alpha=\pi$) --~see \nameref{sec:delta_calc} for a detailed calculation.
These modes with $\theta_\alpha\not=0$ will induce visible oscillations in $\Delta(t)$ if they oscillate faster than their characteristic decay time. We can quantify this by requiring the oscillation period to be smaller than the decay time: $T_\alpha\leq \tau_\alpha$. This gives the inequality
%
\begin{equation}
  \frac{|\Lambda_\alpha|}{R} \geq e^{-\frac{|\theta_\alpha|}{2\pi} } \geq e^{-\frac{1}{2}} \approx 0.61,
  \label{eq:oscillation_visible}
\end{equation}
%
where the lower bound in equation~\eqref{eq:oscillation_visible} occurs when $\Lambda_\alpha$ is real and negative ($\theta_\alpha=\pi$).

To test the predictions of our theory in a realistic scenario, we considered again the COVID-19 epidemic in France and built a stochastic metapopulation model using the same mobility data as in Fig.~\ref{fig:fig1}\textbf{b}.
The details of the model are reported in \nameref{sec:metapop}. We measured the true and the observed reproduction ratios, reported in Fig.~\ref{fig:fig2}, which shows that surveillance-based estimates may remain consistently biased for a long period and, depending on where the epidemic wave started (initial conditions), they may either overestimate or underestimate the true reproduction ratio. The case depicted in Fig.~\ref{fig:fig2}b is particularly concerning: during the first month of the simulated epidemic, surveillance records a lower-than-one reproduction ratio which would mistakenly point to a subsiding outbreak. In reality, the true reproduction ratio is fixed to well above one, and only after two months of simulated epidemic does the surveillance based estimate reach the true value.
Alongside the estimate of $S$ given within the framework of the Galton-Watson process (equation~\eqref{eq:S}), in Fig.~\ref{fig:fig2}a,b we also provide an estimate of the observed reproduction ratio by feeding incident cases to the library {\itshape EpiEstim}~\cite{cori_new_2013}, one of the most popular tools to compute the reproduction ratio from surveillance data. The fact that the two measures overlap confirms that the Galton-Watson process correctly reproduces the phenomenology under study even in realistic scenarios.
Notwithstanding, more detailed frameworks ~\cite{diekmann_definition_1990,white_estimating_2013,trevisin_spatially_2023} could be used to study the impact of heterogeneous generation intervals.

Finally, Fig.~\ref{fig:fig2}c and Fig.~\ref{fig:fig2}d show that locally measured reproduction ratios converge to the true value at different times and with different speeds, and that, at the same moment in time, some communities may overestimate $R$ and some underestimate it. This last point can actually be proven to be always the case. The Collatz-Wielandt inequalities tell us that, for any spatial distribution of cases $\op{x}$, $\min_{i|x_i\not=0} (\op{R}\op{x})_i/x_i\leq R$ and $\max_{i|x_i\not=0} (\op{R}\op{x})_i/x_i\geq R$. Given that $s_i=(\op{R}\op{x})_i/x_i$, out of equilibrium there will always be at least one community overestimating the true reproduction ratio ($s_i>R$) and one underestimating it ($s_i<R$).

Fig.~\ref{fig:fig2} shows no oscillations in the sign of $\Delta(t)$, compatible with the fact that the operator $\op{R}$ we built from mobility data has only real and positive eigenvalues. We extended our analysis to 32 European countries: 24 members of the European Union (excluding Cyprus, Ireland and Latvia for lack of data) plus Albania, Bosnia and Hercegovina, Iceland, Montenegro, Norway, Serbia, Sweden, UK - see details in Supplementary Figure 1.
For all of them we built the operator $\op{R}$ using colocation and population data at the admin-2 level, similarly to what we did for France.
We found at least one real, negative eigenvalue in 11 out of 32 countries, but nowhere did they cause visible oscillations, as the oscillation period was always larger than twice the decay time. We did not find nonreal eigenvalues. This begs the question whether oscillations are actually observable in real systems. To rigorously determine the conditions for a specific spectrum in a generic nonnegative matrix is not possible, except for specific or low-dimensional cases~\cite{egleston_nonnegative_2004}.
We can, however, plausibly associate the presence of an oscillating mode with period $T_\alpha$ to the existence of a cycle of approximate length $T_\alpha$ in the (weighted, directed) network which has $\op{R}$ as its adjacency matrix~\cite{kellogg_complex_1978,torre-mayo_nonnegative_2007}.
Slow oscillations (large $T_\alpha$) would then require the presence of long cycles in $\op{R}$, which are unlikely to be generated by the recurrent mobility patterns that drive the spatial spread of epidemic outbreaks following pathogen importation~\cite{schneider_unravelling_2013,gomez-gardenes_critical_2018}. Fast oscillations, and in particular those generated by real, negative eigenvalues, may instead be more common. They would require epidemics that are strongly coupled, i.e., where pairs of communities exist in which infected residents generate, on average, more cases in the other community than in their own, but this is not the case in the countries we examined and for the spatial resolution we considered.

In the absence of oscillations, the observed reproduction ratio consistently either overestimates or underestimates the true reproduction ratio, as $\Delta(t)$ decays to zero without ever changing sign. In this case, we can determine the sign of the bias from the initial condition: $\Delta(0)=\sum_j r_j x(0)_j-R$. By the Perron-Frobenius theorem, $j_{min},j_{max}$ exist so that $r_{j_{min}}\leq R$ and $r_{j_{max}}\geq R$. Thus, the initial location of cases will completely determine the sign of the error that surveillance will make. If the epidemic starts $j_{max}$ --~or in general in communities with high transmission potential~--, surveillance will consistently overestimate the true reproduction ratio until the bias decays to zero. Conversely, if it starts in $j_{min}$ --~or in communities with low transmission potential~--, surveillance will underestimate $R$.


\section*{Correction to surveillance data}

So far we have proven that surveillance-based estimates of the reproduction ratio may be biased.
We will now propose a way to correct for this bias. Equation~\eqref{eq:S} computes, within our simplified model, the reproduction ratio in terms of the overall observed incidence of cases $I_{tot}(t)$. This can also be trivially interpreted as proportional to the unweighted average of the incidence across communities: $I_{tot}(t) = N \left( \sum_i I_i(t) / N \right)$. From this, we define a new modified incidence using an average weighted by the entries of the Perron dual vector:
%
\begin{equation}
 I^{(v)}_{tot}(t) = N \left( \frac{\sum_i v^*_i I_i(t)}{\sum_i v^*_i} \right) = N\sum_i v^*_i I_i(t) = N \op{v}^* \op{I}(t).
 \label{eq:Iv}
\end{equation}

We now define a new modified observed reproduction ratio using the modified incidence ($I^{(v)}_{tot}(t)$) --~compare this with equation~\eqref{eq:S}:
%
\begin{equation}
    S^{(v)}(t) = \frac{ I^{(v)}_{tot}(t+1) }{ I^{(v)}_{tot}(t) } = \frac{\op{v}^* \op{I}(t+1)}{\op{v}^* \op{I}(t)} = \frac{\op{v}^* \op{R}\op{I}(t)}{\op{v}^* \op{I}(t)} = R \frac{\op{v}^* \op{I}(t)}{\op{v}^* \op{I}(t)} = R.
\end{equation}

% Figure environment removed

% Figure environment removed

The practical advantage for epidemic monitoring is clear: our correction gives an unbiased estimate of the reproduction ratio from surveillance data all along the epidemic wave, unlike traditional measures.
It has, however, two potential drawbacks. The former is that if the initial epidemic seeding occurs in communities where $v^*_i$ is small, then $\op{v}^* \op{I}(t)$ will be very small: stochastic fluctuations would then cause large changes in $S^{(v)}$. In that case then $S^{(v)}$ may well be accurate, but not precise. Luckily, however, no initial condition can be orthogonal to $\op{v}^*$ whose entries are strictly positive, so even if $\op{v}^*\op{x}$ is initially small, it is likely to increase quickly and with it the precision of the measurement.
In Fig.~\ref{fig:fig3} we show that $S^{(v)}$ accurately measures the true reproduction ratio from the beginning of the epidemic wave, in the case of the simulated epidemics of Fig.~\ref{fig:fig2}. Notably, Fig.~\ref{fig:fig3} also shows that if you feed $I^{(v)}_{tot}(t)$ to {\itshape EpiEstim} instead of $I_{tot}(t)$ you will also completely remove the bias on the estimate of the reproduction ratio. Our proposed modified incidence can then be readily incorporated to standard tools for public health surveillance, to improve their accuracy.

The latter potential drawback is that our correction requires knowing $\op{v}^*$.
We argue, however, that this does not require knowing or measuring $\op{R}$ in real time (from which $R$ could then be directly measured) and that
a good estimate of $\op{v}^*$ for epidemic monitoring can be computed during {\itshape peace time}, from past population and mobility data (pre-epidemic, or from data collected during earlier epidemic phases). Indeed $\op{v}^*$ is more stable than $\op{R}$ for the fact that any change happening homogeneously across communities (e.g., changes in the rate of immunity, public health interventions) would change the latter, not the former.
Fig.~\ref{fig:fig4} compares the standard observed reproduction ratio of COVID-19 in France between late 2020 and March 2021 to our correction.
The former is computed with EpiEstim on inferred case incidence, the latter is computed with EpiEstim on the corrected incidence $I^{(v)}_{tot}$, with $\op{v}^*$ computed from past mobility data. Notably, we tested different choices of $\op{v}^*$ going back up to August 2020, i.e., five months prior to the period under study, which confirms that our correction is robust to using past mobility data to reconstruct $\op{v}^*$.
Our correction seems to point to the fact that traditional surveillance underestimated the true reproduction ratio of COVID-19 in France during January and February 2021.
This underestimation is even more consequential because surveillance recorded a lower-than-one reproduction ratio during more than two weeks (see also official reports from that time~\cite{sante_publique_france_covid-19_nodate}), indicating a subsiding epidemic wave. This is at odds with what we know happened: a growing epidemic wave --~the French {\itshape third wave}~-- that led to a national lockdown, enforced on April 3 2021, i.e., immediately after the time window depicted in Fig.~\ref{fig:fig4}. Our corrected reproduction ratio would have instead consistently signaled a growing epidemic wave throughout the first three months of 2021.
This discrepancy carries great significance when put into the context of the debate over public health response at that time. In early 2021 a national curfew was in effect but cases were rising due to the introduction and gradual takeover of the Alpha variant of SARS-CoV-2.
Authorities were wary of additional restrictions and were relying on mass vaccination despite models suggesting that it might not be enough~\cite{di_domenico_impact_2021} --~only $3\%$ of the population had received at least one dose by mid February~\cite{sante_publique_france_covid-19_nodate} (week 6 of 2021 in Fig.~\ref{fig:fig4}).
It is conceivable, albeit circumstantial, that the fact that surveillance underestimated the severity of the wave could have contributed to delaying the enforcement of stricter movement restrictions, which became anyway inevitable later in April.

Our study describes a practicable way to improve the accuracy of the information that flows from epidemiological surveillance to public health policymakers. And better information may lead to more effective policies for preventing and controlling epidemic threats.








\section*{Methods}

\subsection*{Calculation of $\Delta(t)$: proof of equation~\eqref{eq:delta2}}\label{sec:delta_calc}

Combining equation~\eqref{eq:Ievol} and equation~\eqref{eq:S} we get the time evolution of the observed reproduction ratio:
%
\begin{equation}
    S(t) = \frac{ \op{F}^T \op{R}^{t+1}\op{x}(0) }{ \op{F}^T \op{R}^{t}\op{x}(0)}.
\end{equation}
%
We insert this into equation~\eqref{eq:delta1} and get
%
\begin{equation}
    \Delta(t) = \frac{1}{R} \frac{ \op{F}^T (\op{R}-R)\op{R}^{t}\op{x}(0) }{ \op{F}^T \op{R}^{t}\op{x}(0) }.
    \label{eq:delta3}
\end{equation}
%
We introduce the eigenvectors of $\op{R}$ (other than $\op{v}$): $\op{w}_\alpha$ eigenvector with corresponding eigenvalue $\Lambda_\alpha$. Analogously we define the corresponding dual vector $\op{w}^*_\alpha$. Then, we decompose $\op{F}^T$ in the dual basis: $\op{F}^T = \op{v}^* + \sum_\alpha \left( \op{F}^T \op{w}_\alpha \right) \op{w}^*_\alpha$. Using this decomposition in equation~\eqref{eq:delta3} and applying $\op{R}$ leftwards on the dual eigenvectors we get
%
\begin{equation}
    \Delta(t) = - \frac{ \op{F}^T\left[ \sum_\alpha \left(\frac{\Lambda_\alpha}{R}\right)^t \left( 
1-\frac{\Lambda_\alpha}{R} \right) \op{w}_\alpha\op{w}_\alpha^* \right]\op{x}(0) }{ \op{F}^T\left[ \op{v}\op{v}^* + \sum_\alpha \left(\frac{\Lambda_\alpha}{R}\right)^t \op{w}_\alpha\op{w}_\alpha^* \right]\op{x}(0) }.
\label{eq:delta4}
\end{equation}
%
The denominator is $C(t)$ in equation~\eqref{eq:delta2}:
%
\begin{equation}
    C(t) = \frac{ 1 }{ \op{F}^T\left[ \op{v}\op{v}^* + \sum_\alpha \left(\frac{\Lambda_\alpha}{R}\right)^t \op{w}_\alpha\op{w}_\alpha^* \right]\op{x}(0) }.
\end{equation}
%
$C(t)$ is always strictly positive because it is proportional to $\op{F}^T \op{R}^{t}\op{x}(0)$ and tends to $\op{v}^* \op{x}(0)$, i.e., the component of the initial condition onto the eigenspace of the Perron eigenvalue. This component is always nonzero because no $\op{x}(0)$ is nonnegative (as it is a spatial distribution of cases) and no nonnegative vector can be orthogonal to a strictly positive vector.
It is thus the numerator which gives the trend and sign of $\Delta(t)$. 
Equation~\eqref{eq:delta4} then gives the value of the factors $z_\alpha$ in equation~\eqref{eq:delta2}:
%
\begin{equation}
    z_\alpha = - \op{F}^T \left( \op{w}_\alpha \op{w}^*_\alpha \right)\op{x}(0)
\label{eq:z_alpha}
\end{equation}
%
In the case of degenerate eigenvalue one should simply replace $\op{w}_\alpha\op{w}^*_\alpha$ with the appropriate projector over the whole eigenspace. Note that, as discussed before, the denominator in equation~\eqref{eq:delta4} is always real and positive so any complex phase of $z_\alpha$ must arise from $\Lambda_\alpha$ and $\op{w}_\alpha\op{w}^*_\alpha$.

\subsection*{Calculation of $\Delta(t)$: $\tau_\alpha, T_\alpha$}

We isolate in equation~\eqref{eq:delta2} the contribution of each mode $M_\alpha(t)$: $\Delta(t) = \sum_\alpha M_\alpha(t)$, where
%
\begin{equation}
    M_\alpha(t) = M_\alpha(0) \left( \frac{\Lambda_\alpha}{R} \right)^t = M_\alpha(0)\left( \frac{|\Lambda_\alpha| }{R} \right)^t e^{i\theta_\alpha t} = M_\alpha(0)e^{-t/\tau_\alpha} e^{i \theta_\alpha t},
\end{equation}
%
where we used the definition of $\tau_\alpha$ given in the main text. The decaying term with characteristic time $\tau_\alpha$ is visible.

If $\Lambda_\alpha$ is real and positive then $\theta_\alpha=0$ and the oscillating term vanishes. If $\Lambda_\alpha$ is real and negative then $\theta_\alpha=\pi$ and the oscillating term becomes an alternating sign: $e^{i \theta_\alpha t}=(-1)^t$. This is an oscillation with period $T_\alpha=2$, which is compatible with the definition of $T_\alpha$ given in the main text. Finally, if $\Lambda_\alpha\not\in\mathbb{R}$, then then $\bar{\Lambda}_\alpha$ is also an eigenvalue, where the bar denotes complex conjugation. We will call $\bar{\alpha}$ the index corresponding to that eigenvalue: $\Lambda_{\bar{\alpha}}=\bar{\Lambda}_\alpha$. Also, the projector over the eigenspace of $\Lambda_{\bar{\alpha}}$ is the elementwise complex conjugate of the projector over the eigenspace of $\Lambda_\alpha$, meaning that $z_{\bar{\alpha}}=\bar{z}_\alpha$, and thus $M_{\bar{\alpha}}=\bar{M}_\alpha(0)$. Then $\alpha, \bar{\alpha}$ contribute in pair, as follows:
%
\begin{align}
    M_\alpha(t)+M_{\bar{\alpha}}(t) &= e^{-t/\tau_\alpha} \left[ M_\alpha(0)e^{i\theta_\alpha t} + M_{\bar{\alpha}}(0)e^{-i\theta_\alpha t} \right] \nonumber \\
    &= 2e^{-t/\tau_\alpha} \left|M_\alpha(0)\right| \Re e^{i\theta_\alpha t + \phi_\alpha} =2e^{-t/\tau_\alpha} \cos\left( \frac{2\pi}{T_\alpha} t + \phi_\alpha \right).
\end{align}
%
Here we used the definition of $T_\alpha$ given in the main text, explicitly showing the emergence of the oscillating term with period $T_\alpha$.

\subsection*{Reconstruction of the reproduction operator from data}\label{sec:Rfromdata}
The main data used for the reconstruction of reproduction operators for mainland France are Meta Colocation Maps\cite{Colocation_Maps}. They give the probability $p_{ij}$ that a randomly chosen person that is resident of community $i$ and a randomly chosen person resident of community $j$ are both located in a same $600m \times 600m$ square, during a randomly chosen five-minutes time window, in a given week. Note that diagonal elements $p_{ii}$ quantify the mixing within each community. From these diagonal probabilities we discounted spurious co-location time due to people staying at home in spatially contiguous dwellings using Movement Range Maps (see \nameref{sec:data} and Supplementary Methods Section 1.2). The data were provided at the resolution of departments (ADM 2). To reconstruct $\op{R}$ from these data, we assumed that the expected number of secondary cases generated among the residents of community $i$, by a case who
is resident of community $j$, is given by $R_{ij} = C p_{ij} n_i$, where $n_i$ is the population of spatial patch $i$, and $C$ is an overall transmissibility parameter. Notably, while the value of the spectral radius of $\op{R}$ clearly depends on $C$, the left and right Perron eigenvectors $\op{v}$ and $\op{v}^*$ do not, and depend solely on the data.

\subsection*{Epidemic simulations}\label{sec:metapop}
The model of epidemic spread used in simulations is a stochastic discrete-time metapopulation model whereby spatially distinct communities are linked through mobility\cite{Colizza_Vesp,balcan_multiscale_2009,gomez-gardenes_critical_2018,chang_mobility_2020}.
We use a synthetic population based on census data from the National Institute of Statistics and Economic Studies (INSEE) in France.
We divide this population in $94$ spatial communities corresponding to the departments of mainland France except Corsica. Meta Colocation Maps \cite{Colocation_Maps} and Movement Range Maps are used to reconstruct the coupling $p_{ij}$ between communities $i$ and $j$ and the within-community $i$ mixing $p_{ii}$. We use a compartmental model of COVID-19 from \cite{Parameters}. We compute the reproduction ratio for our model according to the next generation method \cite{Next_Gen_Matrix}, obtaining:
%
\begin{equation}
    R=\rho(K)\frac{\beta}{  \mu}\left(1-p_{sc}+\beta_I p_{sc}\right),
\end{equation}
%
where $\rho(K)$ is the spectral radius of the matrix $K_{ij}=p_{ij}n_i$, $n_i$ is the population of community $i$, $\mu$ is the recovery rate, $p_{sc}$ is the probability of sub-clinical infections and $\beta_I$ is the factor by which the transmissibility of sub-clinical cases is reduced (see \cite{Parameters}). The other parameters are also taken from \cite{Parameters}, and the overall transmission rate $\beta$ is set so that $R=1.5$.


\section*{Data availability}
\label{sec:data}
Meta Colocation Maps and Meta Movement Range Maps, which were used to reconstruct reproduction operators and to infer between- and within-community mixing for stochastic simulations can be requested at \url{https://dataforgood.facebook.com/dfg/tools/colocation-maps} and \url{https://dataforgood.facebook.com/dfg/tools/movement-range-maps} respectively. Hospital admission data in France are available at \url{https://www.data.gouv.fr}. French census data can be found at \url{https://www.insee.fr}. All websites accessed June 2023.




\section*{Acknowledgements}
Colocation data were available thanks to {\itshape Data For Good at Meta}.



\newpage

%%%BIBLIO_________________________
\printbibliography[]

%\bibliographystyle{unsrt}
%\bibliography{biblio}


\subsection*{Supplementary information}\label{sec:SI}
Supplementary Methods, Supplementary Figure 1 and Supplementary References 1-3.


\ifarXiv
    \foreach \x in {1,...,\numbersupplementpages}
    {
        %\clearpage
        \includepdf[pages={\x}]{\supplementfilename}
        %\includepdf[pages={\x,{}}]{\supplementfilename}
    }
\fi


%TC:endignore

\end{document}