% This is samplepaper.tex, a sample chapter demonstrating the
% LLNCS macro package for Springer Computer Science proceedings;
% Version 2.20 of 2018/03/10
%
\documentclass[runningheads]{llncs}

\pdfoutput=1
\usepackage[T1]{fontenc}
\def\doi#1{\href{https://doi.org/\detokenize{#1}}{\url{https://doi.org/\detokenize{#1}}}}
%
\usepackage{graphicx}
% Used for displaying a sample figure. If possible, figure files should
% be included in EPS format.
%
% If you use the hyperref package, please uncomment the following line
% to display URLs in blue roman font according to Springer's eBook style:
% \renewcommand\UrlFont{\color{blue}\rmfamily}
%
\usepackage{listings}
\lstset{language=Pascal}



%%%%%%%%%%%%% My Packages %%%%%%%%%%%%%
\usepackage[colorlinks=true,linkcolor=green,citecolor=blue]{hyperref}
\usepackage{algorithm}
\usepackage{algorithmic}
\usepackage{adjustbox}
\usepackage{array}
\usepackage{anyfontsize}
\usepackage{amssymb}  % http://ctan.org/pkg/amssymb
\usepackage{booktabs} 
\usepackage{changepage}
\usepackage{float}
\usepackage{gensymb}
\usepackage{graphicx}
\usepackage[all]{hypcap}
\usepackage{ltablex}
\usepackage{mathtools}
\usepackage{multicol}
\usepackage{multirow}
\usepackage{mwe}   % to get dummy images
\usepackage{pdflscape}
\usepackage{pifont}   % http://ctan.org/pkg/pifont
\usepackage{placeins}
\usepackage{rotating}
\usepackage{soul}
\usepackage{scalerel}
\usepackage{stackengine}
\usepackage{tabularx}
\usepackage{tabularray}
\usepackage{textcomp}
\usepackage{xcolor}

%%%%%%%%%%%%% My Macros %%%%%%%%%%%%%

\def \name{SwIPE}

\usepackage[mathscr]{euscript}
\DeclareSymbolFont{rsfs}{U}{rsfs}{m}{n}
\DeclareSymbolFontAlphabet{\mathscrsfs}{rsfs}

% \setlist{nosep, leftmargin=14pt}

\DeclarePairedDelimiter{\ceil}{\lceil}{\rceil}

\newcommand{\cmark}{\ding{51}}%
\newcommand{\xmark}{\ding{55}}%
\newcommand{\Lagr}{\mathcal{L}}
\newcommand{\ra}[1]{\renewcommand{\arraystretch}{#1}}
\newcommand\dunderline[3][-1pt]{{%
  \sbox0{#3}%
  \ooalign{\copy0\cr\rule[\dimexpr#1-#2\relax]{\wd0}{#2}}}}
\def\subsubsec#1{\noindent\scaleto{\underline{\textbf{#1}}.}{10.65pt}}
\def\subsubsec#1{\noindent\scaleto{\dunderline[-.75pt]{0.25pt}{\textbf{#1}}.}{7.6pt}}

\newcommand\tabbig[1][2cm]{\hspace*{#1}}
\newcommand\tabtiny[1][1mm]{\hspace*{#1}}


\newcommand{\algcomment}{\tabbig \color{gray} {\#} \color{black}}
\def\algcomments#1{\tabbig \color{gray!90!black!90} {\small \# #1} \color{black}}

\newlength{\blob}
\settowidth{\blob}{Mi}
\newlength{\nameblob}
\settowidth{\nameblob}{ResUNet2}


\def\ptext#1{\color{green!60!black!90} {\scaleto{(#1)}{6.5pt}} \color{black}}
\def\ntext#1{\color{red} {\scaleto{(#1)}{6.5pt}} \color{black}}
\def\sdtext#1{\color{black} {\scriptsize $\pm$ \hspace{-1.4mm} #1} \color{black}}

\renewcommand{\sectionautorefname}{\S \hspace{-1mm}}
\renewcommand{\subsectionautorefname}{\S \hspace{-1mm}}
\renewcommand{\subsubsectionautorefname}{\S \hspace{-1mm}}
\newcommand{\dsquare}{\hspace*{2pt}\topinset{$\square$}{$\square$}{2pt}{-2pt}}
\newcommand{\ssquare}{\topinset{$\square$}{$\square$}{0.1pt}{-0.1pt}}


%%%%%%%%%%%%% Document Start %%%%%%%%%%%%%


\begin{document}
%
% \title{Vector Prediction: a Surprisingly Robust Self-supervised 3D Pretraining Task for CT Segmentation}
\title{
SwIPE: Efficient and Robust Medical Image Segmentation with Implicit Patch Embeddings}
\titlerunning{\name}

% Non-anon Author List
\author{\vspace*{-0.5mm}
Yejia Zhang \and  
Pengfei Gu \and   
Nishchal Sapkota \and  
Danny Z. Chen     
}
%index{Zhang, Yejia}
%index{Gu, Pengfei}
%index{Sapkota, Nishchal}
%index{Chen, Danny}

\institute{\vspace*{-0.25mm}
University of Notre Dame, Notre Dame IN 46556, USA \\
\email{\{yzhang46,pgu,nsapkota,dchen\}@nd.edu}}
\authorrunning{Zhang et al.}



\maketitle            
\begin{abstract}  % 150 - 250 words
\vspace{-3.4mm}
Modern medical image segmentation methods primarily use \textit{discrete} representations in the form of rasterized masks to learn features and generate predictions. 
Although effective, this paradigm is spatially inflexible, scales poorly to higher-resolution images, and lacks direct understanding of object shapes. 
% To address these limitations, some recent works utilized implicit neural representations (INRs) to learn \textit{continuous} representations for segmentation, but often directly adopted components designed for 3D shape reconstruction. 
% These formulations also constrained themselves to either point-based or global contexts which lack contextual understanding or local fine-grained details, respectively --- both essential for accurate segmentation.
To address these limitations, some recent works utilized implicit neural representations (INRs) to learn \textit{continuous} representations for segmentation.
However, these methods often directly adopted components designed for 3D shape reconstruction. 
More importantly, these formulations were also constrained to either point-based or global contexts, lacking contextual understanding or local fine-grained details, respectively---both critical for accurate segmentation.
To remedy this, we propose a novel approach, \textbf{SwIPE} (\underline{S}egmentation \underline{w}ith \underline{I}mplicit \underline{P}atch \underline{E}mbeddings), that leverages the advantages of INRs and predicts shapes at the patch level---rather than at the point level or image level---to enable both accurate local boundary delineation and global shape coherence.
% by modeling shapes at an image patch level.
% maintaining fine-grained local information and global shape coherence.
Extensive evaluations on two tasks (2D polyp segmentation and 3D abdominal organ segmentation) show that SwIPE significantly improves over recent implicit approaches and outperforms state-of-the-art discrete methods with over 10x fewer parameters.
Our method also demonstrates superior data efficiency and improved robustness to data shifts across image resolutions and datasets.
Code is available on \href{https://github.com/charzharr/miccai23-swipe-implicit-segmentation}{Github}.

\vspace{-2.5mm}
\keywords{Medical Image Segmentation \and Deep Implicit Shape Representations \and Patch Embeddings \and Implicit Shape Regularization}
\end{abstract}



%%%%%%% ------------------------------------------------------------------------- %%%%%%%
%%%%%%% ------------------------------------------------------------------------- %%%%%%%


\section{Introduction}
Deep learning models have been widely used in many applications.
For example, BERT~\citep{devlin_bert_2019}, GPT-3~\citep{brown_language_2020}, and T5~\citep{raffel_exploring_2020} achieved state-of-the-art~(SOTA) results on different natural language processing~(NLP) tasks. 
For computer vision~(CV), Transformer-like models such as ViT~\citep{dosovitskiy_image_2021} and Swin Transformer~\citep{liu_swin_2021} deliver excellent accuracy performance upon multiple tasks. 


At the same time, training deep learning models has been a critical problem troubling the community due to the long training time, especially for those large models with billions of parameters~\citep{brown_language_2020}. 
In order to enhance the training efficiency, researchers propose some manually designed parallel training strategies~\citep{narayanan_efficient_2021,shazeer_mesh-tensorflow_2018,xu_gspmd_2021}. 
However, selecting, tuning, and combining these strategies require extensive domain knowledge in deep learning models and hardware environments. With the increasing diversity of modern hardware architectures~\cite{flynn_very_1966,flynn_computer_1972} and the rapid development of deep learning models, these manually designed approaches are bringing heavier burdens to developers. 
Hence, \emph{automatic parallelism} is introduced to automate the parallel strategy searching for training models.


There are two main categories of parallelism in deep learning models: inter-layer parallelism~\citep{huang_gpipe_2019,narayanan_pipedream_2019,narayanan_memory-efficient_2021,fan_dapple_2021,li_chimera_2021,lepikhin_gshard_2021,du_glam_2022,fedus_switch_2022} and intra-layer parallelism~\citep{li_pytorch_2020,narayanan_efficient_2021,rasley_deepspeed_2020,fairscale_authors_fairscale_2021}. 
Inter-layer parallelism partitions the model into disjoint sets on different devices without slicing tensors. 
Alternatively, intra-layer parallelism partitions tensors in a layer along one or more axes and distributes them across different devices.


Current automatic parallelism techniques focus on optimizing strategies within these two categories. However, they treat these two categories separately. 
Some methods~\citep{zhao_vpipe_2022,jia_exploring_2018,cai_tensoropt_2022,wang_supporting_2019,jia_beyond_2019,schaarschmidt_automap_2021,liu_colossal-auto_2023} overlook potential opportunities for inter- or intra-layer parallelism, the others optimize inter- and intra-layer parallelism hierarchically and sequentially~\citep{narayanan_pipedream_2019,fan_dapple_2021,he_pipetransformer_2021,tarnawski_efficient_2020,tarnawski_piper_2021,zheng_alpa_2022}. 
As a result, current automatic parallelism techniques often fail to achieve the global optima and instead become trapped in local optima. 
Therefore, a unified inter- and intra-layer approach is needed to enhance the effectiveness of automatic parallelism.


This paper aims to find the optimal parallelism strategy while simultaneously considering inter- and intra-layer parallelism. 
It enables us to search in a more extensive strategy space where the globally optimal solution lurk. 
However, unifying inter- and intra-layer parallelism in automatic parallelism brings us two challenges. 
Firstly, to adopt a unified perspective on the inter- and intra-layer automatic parallelism, we should not formalize them with separate formulations as prior works. Therefore, how can we express these parallelism strategies in a unified formulation? 
Secondly, previous methods take a long time to obtain the solution with a limited strategy space. Therefore, how can we ensure that the best solution can be obtained in a reasonable time while expanding the strategy space?


To solve the above challenges, we propose UniAP. For the first challenge, UniAP adopts the mixed integer quadratic programming~(MIQP)~\citep{lazimy_mixed_1982} to search for the globally optimal parallel strategy automatically. 
It unifies the inter- and intra-layer automatic parallelism in a single MIQP formulation. 
For the second challenge, our complexity analysis and experimental results show that UniAP can obtain the globally optimal solution in a significantly shorter time.


The contributions of this paper are summarized as follows: 
\begin{itemize}
    \item We propose UniAP, the first framework to unify inter- and intra-layer automatic parallelism in model training.
    \item The optimal parallel strategies discovered by UniAP exhibit scalability on training throughput and strategy searching time.
    \item The experimental results show that UniAP speeds up model training on four Transformer-like models by up to 1.70$\times$ and reduces the strategy searching time by up to 16$\times$, compared with the SOTA method.
\end{itemize}


\vspace{-5.3mm}
\section{Methodology} \label{sec:2}
\vspace{-2.1mm}

\input{figures/fig1}

The core idea of SwIPE (overviewed in Fig.~\ref{fig:1}) is to use patch-wise 
%implicit neural representations (INRs) 
INRs for semantic segmentation.
To formulate this, we first discuss the shift from discrete to implicit segmentation, then delineate the intermediate representations needed for such segmentation, and overview the major components involved in obtaining these representations.
Note that for the remainder of the paper, we present formulations for 2D data but the descriptions are conceptually congruous in 3D.

In a typical discrete segmentation setting with $C$ classes, an input image $\textbf{X}$ is mapped to class probabilities with the same resolution $f: \textbf{X} \in \mathbb{R}^{H \times W \times 3} \rightarrow \hat{\textbf{Y}} \in \mathbb{R}^{H \times W \times C}$. 
Segmentation with INRs, on the other hand, maps an image $\textbf{X}$ and a continuous image coordinate $\textbf{p}_i = (x, y)$, $x, y \in [-1,1]$, to the coordinate's class-wise occupancy probability $\hat{\textbf{o}}_i \in \mathbb{R}^C$, yielding $f_\theta: (\textbf{p}_i, \textbf{X}) \rightarrow \hat{\textbf{o}}_i$, where $f_\theta$ is parameterized by a neural network with weights $\theta$.
As a result, predictions of arbitrary resolutions can be obtained by modulating the spatial granularity of the input coordinates.
This formulation also enables the direct use of discrete pixel-wise losses like Cross Entropy or Dice with the added benefit of boundary modeling.
Object boundaries are represented as the zero-isosurface in $f_\theta$'s prediction space or, more elegantly, $f_\theta$'s decision boundary.
% As a result, from a fixed size $X$, predictions of arbitrary resolutions can be obtained by modulating the spacial granularity of input coordinates.
% This formulation also allows for seamless integration with existing discrete pipelines with pixel-wise losses like Cross Entropy or Dice, but with the added benefit of robust boundary modeling in the form of the zero-isosurface from $D$'s prediction space.
% Thus, on a high level, our method operates on point and class-occupancy pairs via $f_\theta$ rather than pixel-wise correspondences between discrete images and masks.

SwIPE builds on the INR segmentation setting (e.g., in~\cite{Khan2022IOSNet}), but operates on patches rather than on points or global embeddings (see Tab.~\ref{tab:1} \& left of Tab.~\ref{tab:3} for empirical justifications) to better enable both local boundary details and global shape coherence.
This involves two main steps: (1) encode shape embeddings from an image, and (2) decode occupancies for each point while conditioning on its corresponding embedding(s).
In our case, $f_\theta$ includes an encoder $E_b$ (or backbone) that extracts multi-scale feature maps from an input image, a context aggregation module $E_n$ (or neck) that aggregates the feature maps into vector embeddings for each patch, and MLP decoders $D^\mathbb{P}$ (decoder for local patches where $\mathbb{P}$ is for patch) \& $D^\mathbb{I}$ (decoder for entire images where $\mathbb{I}$ is for image) that output smoothly-varying occupancy predictions given embedding \& coordinate pairs.
To encode patch embeddings in step (1), $E_b$ and $E_n$ map an input image $\textbf{X}$ to a global image embedding $\textbf{z}^\mathbb{I}$ and a matrix $\textbf{Z}^\mathbb{P}$ containing a local patch embedding $\textbf{z}^\mathbb{P}$ at each planar position. 
For occupancy decoding in step (2), $D^\mathbb{P}$ decodes the patch-wise class occupancies $\textbf{o}_i^\mathbb{P}$ using relevant local and global inputs while $D^\mathbb{I}$ predicts occupancies $\textbf{o}_i^\mathbb{I}$ for the entire image using only image coordinates $\textbf{p}_i^\mathbb{I}$ and the image embedding $\textbf{z}^\mathbb{I}$.
% For occupancy decoding in (2), $D^\mathbb{P}$ decodes a patch coordinate $\textbf{p}_i^\mathbb{P}$ and its corresponding patch embedding $\textbf{z}_i^\mathbb{P}$ to patch class occupancies $D^\mathbb{P}: \textbf{p}_i^\mathbb{P}, \textbf{z}_i^\mathbb{P} \rightarrow \hat{\textbf{o}}_i^\mathbb{P}$. 
% Similarly, image occupancy decoding can be written as $D^\mathbb{I}: \textbf{p}_i^\mathbb{I}, \textbf{z}_i^\mathbb{I} \rightarrow \hat{\textbf{o}}_i^\mathbb{I}$ and serves as an auxiliary task to learn global shapes across patches.
Below, we detail the encoding of image \& patch embeddings (Sec.~\autoref{sec:2-1}), the point-wise decoding process (Sec.~\autoref{sec:2-2}), and the training procedure for SwIPE (Sec.~\autoref{sec:2-3}). 



% Since our method operates on patch embeddigns
% $f_\theta$ in SwIPE consists of an encoder $E_b$ (or backbone) that extracts multi-scale feature maps from an input image, a context aggregation module $E_n$ (or neck) that aggregates the feature maps into vector embeddings for each patch, and a decoder $D$ that outputs smoothly-varying occupancy predictions for each patch.

% Below, we detail the components used to extract patch embeddings from image $X$ (see~\autoref{sec:2-1}), the mechanisms for decoding patch shapes from patch embeddings (see~\autoref{sec:2-2}), and the training procedure for SwIPE (see~\autoref{sec:2-3}). 


\vspace{-4mm}
\subsection{Image Encoding and Patch Embeddings} \label{sec:2-1}
\vspace{-1.5mm}

The encoding process utilizes the backbone $E_b$ and neck $E_n$ to obtain a global image embedding $\textbf{z}^\mathbb{I}$ and a matrix $\textbf{Z}^\mathbb{P}$ of patch embeddings.
% $E_n \circ E_b: \textbf{X} \rightarrow \textbf{z}^\mathbb{I}, \textbf{Z}^\mathbb{P}$.
We define an image patch as an isotropic grid cell (i.e., a square in 2D or a cube with identical spacing in 3D) of length $S$ from non-overlapping grid cells over an image.
Thus, an image $\textbf{X} \in \mathbb{R}^{H \times W \times 3}$ with a patch size $S$ will produce $\ceil[\big]{\frac{H}{S}} \cdot \ceil[\big]{\frac{W}{S}}$ patches.
For simplicity, we assume that the image dimensions are evenly divisible by $S$.
% written as $E_n \circ E_b: \textbf{X} \rightarrow \textbf{z}^\mathbb{I}, \textbf{Z}^\mathbb{P}$. where $d$ is the embedding dimension, $\textbf{z}^\mathbb{I} \in \mathbb{R}^d$, and $\textbf{Z}^\mathbb{P} \in \mathbb{R}^{\frac{H}{S} \times \frac{W}{S} \times d}$.
% $\textbf{z}_{rc} \in \textbf{Z}$ indexes the patch embedding at the $r$th row and $c$th column.

A fully convolutional \textbf{encoder backbone} $E_b$ (e.g., Res2Net-50~\cite{gao2019res2net}) is employed to generate multi-scale features from image $\textbf{X}$.
The entire image is processed as opposed to individual crops \cite{jiang2020localimplicitgrid,chabra2020deeplocalshape,Reich2021OSSNetME} to leverage larger receptive fields and integrate intra-patch information.
Transformers \cite{hassani2021CCT} also model cross-patch relations and naturally operate on patch embeddings, but are data-hungry and lack helpful spatial inductive biases (we affirm this in Sec.~\autoref{sec:3-5}).
$E_b$ outputs four multi-scale feature maps from the last four stages, $\{\textbf{F}_n\}_{n=2}^{5}$ ($\textbf{F}_n \in \mathbb{R}^{C_n \times H_n \times W_n}$, $H_n = \frac{H}{2^n}$, $W_n = \frac{W}{2^n}$).
% To obtain $\textbf{Z}^\mathbb{P}$, we prioritize data efficiency and contextual understanding in our encoder design for accurate segmentation with limited medical data.
% For $E_b$ and $E_n$, we elect to use fully convolutional encoder components that operate on the entire input image to encode patch embeddings.
% We avoid directly inputting crops like other patch-based 3D reconstruction approaches~\cite{jiang2020localimplicitgrid,chabra2020deeplocalshape} since they preclude helpful context beyond patches.
% For better, context modeling, Transformers \cite{hassani2021CCT} were also considered since they innately represent image patches as vector embeddings and are effective in modeling long-range relations, but self-attention requires abundant data and lack helpful spacial inductive biases (see backbone studies in Section~\autoref{sec:3-5}).
% Thus, we employ a multi-stage convolutional \textbf{encoder backbone} $E_b$ to generate multiple levels of features $\{\textbf{F}_n\}_{n=2}^{5}$ from image $\textbf{X}$ where $\textbf{F}_n \in \mathbb{R}^{C_n \times H_n \times W_n}$, $H_n = \frac{H_n}{2^n}$.

The \textbf{encoder neck} $E_n$ aggregates $E_b$'s multi-scale outputs $\{\textbf{F}_n\}_{n=2}^{5}$ to produce $\textbf{z}^\mathbb{I}$ (the shape embedding for the entire image) and $\textbf{Z}^\mathbb{P}$ (the grid of shape embeddings for patches).
The feature maps are initially fed into a modified Receptive Field Block \cite{liu2018receptivefieldblockRFB} (dubbed RFB-L or RFB-Lite) that replaces symmetric convolutions with a series of efficient asymmetric convolutions 
(e.g., $(3\times3)$ $\rightarrow$ $(3\times1) + (1\times3)$).
% Like modern object detection schemas \cite{zhu2021tphyolov5}, 
The context-enriched feature maps are then fed through multiple cascaded aggregation and downsampling operations (see $E_n$ in Fig.~\ref{fig:1}) to obtain four multi-stage intermediate embeddings with identical shapes, $\{\textbf{F}_n'\}_{n=2}^{5} \in \mathbb{R} ^ {\frac{H}{32} \times \frac{W}{32} \times d}$.
% Similar to modern object detection schemas \cite{zhu2021tphyolov5}, we adopt an \textbf{encoder neck} $E_n$ to aggregate multi-scale features and improve contextual understanding (see Figure~\ref{fig:1}).
% All multiscale feature maps $\{\textbf{F}_n\}_{n=2}^{5}$ from $E_b$ are initially fed into a modified Receptive Field Block \cite{liu2018receptivefieldblockRFB} (dubbed RFB-L or RFB-Lite) that replaces symmetric convolutions with a series of efficient asymmetric convolutions 
% (e.g., $(3\times3)$ $\rightarrow$ $(3\times1) + (1\times3)$).
% % (e.g., $(3\times3)$ $\rightarrow$ $(3\times1) + (1\times3)$, or $(3\times3\times3)$ $\rightarrow$ $(3\times1\times1) + (1\times3\times1) + (1\times1\times3)$).
% The resulting augmented feature maps are then fed through multiple cascaded aggregation and downsampling operations to obtain four multi-stage feature maps with identical shapes $\{\textbf{F}_n'\}_{n=2}^{5} \in \mathbb{R} ^ {\frac{H}{32} \times \frac{W}{32} \times d}$.
% The vector at each planar position of $\textbf{F}_n'$ represents an intermediate embedding for a shape centered at that feature map position. 

To convert the intermediate embeddings $\{\textbf{F}_n'\}_{n=2}^{5}$ to patch embeddings $\textbf{Z}^\mathbb{P}$, we first resize them to $\textbf{Z}^\mathbb{P}$'s final shape via linear interpolation to produce $\{\textbf{F}_n''\}_{n=2}^{5}$, which contain low-level ($\textbf{F}_2''$) to high-level ($\textbf{F}_5''$) information.
Resizing enables flexibility in designing appropriate patch coverage, which may differ across tasks due to varying structure sizes and shape complexities.
Note that this is different from the interpolative sampling in~\cite{Khan2022IOSNet} and more similar to~\cite{hu2022ifanet}, except the embeddings' spatial coverage in SwIPE are larger and adjustable.
To prevent the polarization of embeddings toward either local or global scopes, we propose a \textbf{multi-stage embedding attention} (MEA) module to enhance representational power and enable dynamic focus on the most relevant abstraction level for each patch. 
Given four intermediate embedding vectors $\{\textbf{e}_n\}_{n=2}^{5}$ from corresponding positions in $\{\textbf{F}_n''\}_{n=2}^{5}$, we compute the attention weights via $\mathcal{W} = Softmax(MLP_1(cat(MLP_0(\textbf{e}_2), MLP_0(\textbf{e}_3), MLP_0(\textbf{e}_4), MLP_0(\textbf{e}_5))))$, where $\mathcal{W} \in \mathbb{R}^{4}$ is a weight vector, $cat$ indicates concatenation, and $MLP_0$ is followed by a ReLU activation. 
The final patch embedding is obtained by $\textbf{z}^\mathbb{P} = MLP_{2}(\sum^{5}_{n=2}  \textbf{e}_n + \sum^{5}_{n=2} w_{n-2} \cdot \textbf{e}_n)$, where $w_i$ is the $i$th weight of $\mathcal{W}$.
Compared to other spatial attention mechanisms like CBAM~\cite{woo2018cbam}, our module separately aggregates features at each position across multiple inputs and predicts a proper probability distribution in $\mathcal{W}$ instead of an unconstrained score. 
The output patch embedding matrix $\textbf{Z}^\mathbb{P}$ is populated with $\textbf{z}^\mathbb{P}$ at each position and models shape information centered at the corresponding patch in the input image (e.g., if $S=32$, $\textbf{Z}^\mathbb{P}[0,0]$ encodes shape information of the top left patch of size $32\times32$ in $\textbf{X}$).
Finally, $\textbf{z}^\mathbb{I}$ is obtained by average-pooling $\textbf{F}'_5$ into a vector.
% $\textbf{e} = MLP_2(\sum_{n=2}^{5} \textbf{e_n} + \mathcal{W}T )$.
% Note that this is similar in spirit to spatial attention in CBAM~\cite{woo2018cbam}, but with probability distribution $\mathcal{W}$ modeling multi-stage relevancy and involving multiple feature sources.



% For $E_d$ and $E_n$, we elect to use fully convolutional encoder components that operate on the entire input image to encode patch embeddings.
% However, there are other approaches to obtaining a descriptive embedding $\textbf{z}$. 
% Recent patch-based methods for 3D reconstruction \cite{jiang2020localimplicitgrid,chabra2020deeplocalshape} directly feed in local patches after cropping to extract embeddings.
% However, this faces similar limitations as the local encodings method in OSSNet \cite{Reich2021OSSNetME} by failing to leverage helpful context beyond the patch. 
% To improve context across patches, another natural approach is to adopt Transformers which already model image patches as vector embeddings and are effective in modeling long-range relations.
% Given limited medical image data, however, the self-attention mechanism lacks helpful inductive biases such as local bias, equivariant processing, and weight-sharing (we empirically affirm this in our backbone studies, see~\autoref{sec:3-5}). 
% Thus, we find a convolutional backbone to be the most suitable given their data, parameter, and computational efficiency.

\vspace{-3mm}
\subsection{Implicit Patch Decoding} \label{sec:2-2}
\vspace{-1.5mm}

% To predict patch-wise occupancies with decoder $D^\mathbb{P}$ ($\mathbb{P}$ for patch), 
Given an image coordinate $\textbf{p}^\mathbb{I}_i$ and its corresponding patch embedding $\textbf{z}_i^\mathbb{P}$, the patch-wise occupancy can be decoded with decoder $D^\mathbb{P}: (\textbf{p}^\mathbb{P}_i, \textbf{z}_i^\mathbb{P}) \rightarrow \hat{\textbf{o}}_i^\mathbb{P}$, where $D^\mathbb{P}$ is a small MLP and 
$\textbf{p}^\mathbb{P}_i$ is the patch coordinate with respect to the patch center $\textbf{c}_i$ associated with $\textbf{z}_i^\mathbb{P}$ and is obtained by $\textbf{p}^\mathbb{P}_i = \textbf{p}^\mathbb{I}_i - \textbf{c}_i$.
But, this design leads to poor global shape predictions and discontinuities around patch borders.
% Despite good local precision, decoding each patch independently in this manner leads to poor global shape coherence and discontinuities around patch borders.

To encourage better \textbf{global shape coherence}, we also incorporate a global image-level decoder $D^\mathbb{I}$.
This image decoder, $D^\mathbb{I}: (\textbf{p}^\mathbb{I}_i, \textbf{z}^\mathbb{I}) \rightarrow \hat{\textbf{o}}_i^\mathbb{I}$, predicts occupancies for the entire input image.
To distill higher-level shape information into patch-based predictions, we also condition $D^\mathbb{P}$'s predictions on $\textbf{p}_i^\mathbb{I}$ and $\textbf{z}^\mathbb{I}$. 
Furthermore, we find that providing the \textbf{source coordinate} gives additional spatial context for making location-coherent predictions.
In a typical segmentation pipeline, the input image $\textbf{X}$ is a resized crop from a source image and we find that giving the coordinate $\textbf{p}_i^\mathbb{S}$ ($\mathbb{S}$ for source) from the original uncropped image improves performance on 3D tasks since the additional positional information may be useful for predicting recurring structures.
Our enhanced formulation for patch decoding can be described as $D^\mathbb{P}: (\textbf{p}^\mathbb{P}_i, \textbf{z}^\mathbb{P}_i, \textbf{p}^\mathbb{I}_i, \textbf{z}^\mathbb{I}, \textbf{p}^\mathbb{S}_i) \rightarrow \hat{\textbf{o}}_i^\mathbb{P}$. 
% To encourage better \textbf{global shape coherence}, we also incorporate a global image-level decoder $D^\mathbb{I}$.
% The image decoder, $D^\mathbb{I}: \textbf{p}^\mathbb{I}_i, \textbf{z}^\mathbb{I} \rightarrow \hat{\textbf{o}}_i$, predicts occupancies for the entire input image given an image coordinate $\textbf{p}^\mathbb{I}_i$ and a global embedding $\textbf{z}^\mathbb{I}$ where $\textbf{z}^I$ is obtained by average-pooling $\textbf{F}'_5$ into a vector.
% To distill higher-level shape information into patch-based prediction, we also condition $D^\mathbb{P}$ predictions on $\textbf{p}_i^\mathbb{I}$ and $\textbf{z}^\mathbb{I}$. 
% Furthermore, we find that providing the \textbf{source coordinate} gives additional spacial context for making location-coherent predictions.
% In usual data pipelines, the input image $\textbf{X}$ is a resized crop from a source image and giving the coordinate $p_i^\mathbb{S}$ ($\mathbb{S}$ for source) from the original image particularly helps for 3D tasks.
% Finally, we have $D^\mathbb{P}: \textbf{p}^\mathbb{P}_i, \textbf{z}^\mathbb{P}, \textbf{p}^\mathbb{I}_i, \textbf{z}^\mathbb{I}, \textbf{p}^\mathbb{S} \rightarrow \hat{\textbf{o}}_i$. 

To address discontinuities at patch boundaries, we propose a training technique called \textbf{Stochastic Patch Overreach} (SPO) which forces patch embeddings to make predictions for coordinates in neighboring patches.
For each patch point and embedding pair ($\textbf{p}^\mathbb{P}_i, \textbf{z}^\mathbb{P}_i$), we create a new pair ($\textbf{p}^\mathbb{P}_i{}', \textbf{z}^\mathbb{P}_i{}'$) by randomly selecting a neighboring patch embedding and updating the local point to be relative to the new patch center.
This regularization is modulated by the set of valid choices to select a neighboring patch (\textit{connectivity}, or \textit{con}) and the number of perturbed points to sample per batch point (\textit{occurrence}, or $N_{\text{SPO}}$).
$con$=$4$ means all adjoining patches are neighbors while $con$=$8$ includes corner patches as well. 
Note that SPO differs from the regularization in \cite{chabra2020deeplocalshape} since no construction of a KD-Tree is required and we introduce a tunable stochastic component which further helps with regularization under limited-data settings.


% \subsubsection{Improving Boundaries}

% The duality of boundary decoding: 1) unsmooth at patch boundaries, 2) too smooth near object boundaries within the patch. 

% To achieve robust patch prediction with adequate global shape regularization, we propose three mechanisms. Firstly, we introduce a stochastic patch overreach scheme to alleviate boundary discontinuity between patches and facilitate contextual understanding beyond a local scope. Secondly, we condition local patches on global embeddings. Lastly, we apply multi-scale self-supervision to improve local and global shape consistency.


\begin{table*}[!t]
\begin{center}
\caption{\label{tab:1}
\textbf{Overall results versus the state-of-the-art}.
Starred* items indicate a state-of-the-art discrete method for each task.
The Dice columns report foreground-averaged scores and standard deviations ($\pm$) across 6 runs (6 different seeds were used while train/val/test splits were kept consistent).
\vspace*{-3mm}
}

\resizebox{0.8\textwidth}{!}{
\begin{tblr}{
    columns={colsep=3pt},
    colspec={l c c c  || l c c c },
    row{1} = {gray!50!black!15},
    row{3,7} = {gray!20!black!5}
    }
\hline 
\SetCell[c=4]{c} 2D Polyp Sessile & & & & \SetCell[c=4]{c} 3D CT BCV \\
\cline{1-4} \cline{5-8}
Method & Params (M) & FLOPs (G) & Dice (\%) & Method & Params (M) & FLOPs (G) & Dice (\%) \\
\hline 
\SetCell[c=8]{l} \textit{Discrete Approaches} \\
\hline 
\makebox[\nameblob][l]{U-Net$_{15}$}
\makebox[\blob][r]{\cite{ronneberger2015unet}} & 7.9 & 83.3 & 63.89$\pm$1.30 & 
\makebox[\nameblob][l]{U-Net$_{15}$}
\makebox[\blob][r]{\cite{ronneberger2015unet}} & 16.3 & 800.9 & 74.47$\pm$1.57  
\\
\makebox[\nameblob][l]{PraNet$_{20}^{*}$} \makebox[\blob][r]{\cite{fan2020pranet}} & 30.5 & 15.7 & 82.56$\pm$1.08 &
\makebox[\nameblob][l]{UNETR$_{21}^{*}$} \makebox[\blob][r]{\cite{hatamizadeh2022unetr}} & 92.6 & 72.6 & 81.14$\pm$0.85
\\
\makebox[\nameblob][l]{Res2UNet$_{21}$} \makebox[\blob][r]{\cite{gao2019res2net}} & 25.4 & 17.8 & 81.62$\pm$0.97 &
\makebox[\nameblob][l]{Res2UNet$_{21}$} \makebox[\blob][r]{\cite{gao2019res2net}} & 38.3 & \textbf{44.2} & 79.23$\pm$0.66 \\
\hline
\SetCell[c=8]{l} \textit{Implicit Approaches} \\
\hline 
\makebox[\nameblob][l]{OSSNet$_{21}$} \makebox[\blob][r]{\cite{Reich2021OSSNetME}} & 5.2 & 6.4 & 76.11$\pm$1.14 & 
\makebox[\nameblob][l]{OSSNet$_{21}$} \makebox[\blob][r]{\cite{Reich2021OSSNetME}} & 7.6 & 55.1 & 73.38$\pm$1.65 \\
\makebox[\nameblob][l]{IOSNet$_{22}$} \makebox[\blob][r]{\cite{Khan2022IOSNet}} & 4.1 & \textbf{5.9} & 78.37$\pm$0.76 & 
\makebox[\nameblob][l]{IOSNet$_{22}$} \makebox[\blob][r]{\cite{Khan2022IOSNet}} & 6.2 & 46.2 & 76.75$\pm$1.37 \\
\hline 
SwIPE (\textit{ours}) & \textbf{2.7} & 10.2 & \textbf{85.05}$\pm$0.82 &
SwIPE (\textit{ours}) & \textbf{4.4} & 71.6 & \textbf{81.21}$\pm$0.94 \\
% \cline{4-5}
\end{tblr}
}
% \vspace*{-6mm}
\end{center}
\vspace*{-10mm}
\end{table*}



 

\vspace{-3mm}
\subsection{Training SwIPE} \label{sec:2-3}
\vspace{-1.5mm}

To optimize the parameters of $f_\theta$, we first sample a set of point and occupancy pairs $\{\textbf{p}^\mathbb{S}_i, \textbf{o}_i\}_{i \in \mathcal{I}}$ for each source image, where $\mathcal{I}$ is the index set for the selected points.
We obtain an equal number of points for each foreground class using Latin Hypercube sampling with 50\% of each class's points sampled within 10 pixels of the class object boundaries. 
The \textbf{point-wise occupancy loss}, written as 
$\mathcal{L}_\text{occ}(\textbf{o}_i, \hat{\textbf{o}}_i) = 0.5 \cdot \mathcal{L}_\text{ce}(\textbf{o}_i, \hat{\textbf{o}}_i) + 0.5 \cdot \mathcal{L}_\text{dc}(\textbf{o}_i, \hat{\textbf{o}}_i)$,
is an equally weighted sum of Cross Entropy loss $\mathcal{L}_\text{ce}(\textbf{o}_i, \hat{\textbf{o}}_i) = -\log \hat{o}_i^c$ and Dice loss $\mathcal{L}_\text{dc}(\textbf{o}_i, \hat{\textbf{o}}_i) = 1 - \frac{1}{C} \sum_{c} \frac{2 \cdot o_i^c \cdot \hat{o}_i^c + 1}{(o_i^c)^2 + (\hat{o}_i^c)^2 + 1}$, where $\hat{o}_i^c$ is the predicted probability for the target occupancy with class label $c$.
Note that in practice, these losses are computed in their vectorized forms; for example, the Dice loss is applied with multiple points per image instead of an individual point (similar to computing the Dice loss between a flattened image and its flattened mask).
% The class occupancy loss for a point consists of an equally weighted sum of Cross Entropy loss $\mathcal{L}_\text{ce}$ and Dice Loss $\mathcal{L}_\text{dc}$ and can be written as  
% where }_i^c$ is the predicted probability for the target occupancy target with class label $c$.
% The class occupancy loss for a point consists of an equally weighted sum of Cross Entropy loss $\mathcal{L}_\text{ce}$ and Dice Loss $\mathcal{L}_\text{dc}$ and can be written as  
% $\mathcal{L}_\text{occ}(\textbf{o}_i, \hat{\textbf{o}}_i) = 0.5 \cdot \mathcal{L}_\text{ce}(\textbf{o}_i, \hat{\textbf{o}}_i) + 0.5 \cdot \mathcal{L}_\text{dc}(\textbf{o}_i, \hat{\textbf{o}}_i)$, 
% where 
% $\mathcal{L}_\text{ce}(\textbf{o}_i, \hat{\textbf{o}}_i) = -log \hat{o}_i^c$,  
% $\mathcal{L}_\text{dc}(\textbf{o}_i, \hat{\textbf{o}}_i) = 1 - \frac{1}{C} \sum_{c} \frac{2 \cdot o_i^c \cdot \hat{o}_i^c + 1}{(o_i^c)^2 + (\hat{o}_i^c)^2 + 1}$, 
% and $\hat{o}_i^c$ is the predicted probability for the target occupancy target with class label $c$.
The \textbf{loss for patch and image decoder} predictions is $\mathcal{L}_{\mathbb{P}\mathbb{I}}(\textbf{o}_i, \hat{\textbf{o}}_i^\mathbb{P}, \hat{\textbf{o}}_i^\mathbb{I}) = \alpha \mathcal{L}_\text{occ}(\textbf{o}_i, \hat{\textbf{o}}_i^\mathbb{P}) + (1 - \alpha) \mathcal{L}_\text{occ}(\textbf{o}_i, \hat{\textbf{o}}_i^\mathbb{I})$, where $\alpha$ is a local-global balancing coefficient.
Similarly, the \textbf{loss for the SPO} occupancy prediction $\hat{\textbf{o}}_i'$ is $\mathcal{L}_{\text{SPO}}(\textbf{o}_i, \hat{\textbf{o}}_i') = \mathcal{L}_\text{occ}(\textbf{o}_i, \hat{\textbf{o}}_i')$. 
Finally, the \textbf{overall loss} for a coordinate is formulated as 
$\mathcal{L} = \mathcal{L}_{\mathbb{P}\mathbb{I}} + \beta \mathcal{L}_{\text{SPO}} + \lambda(||\textbf{z}_i^\mathbb{P}||_2^2 + ||\textbf{z}_i^\mathbb{I}||_2^2)$, where $\beta$ scales SPO and the last term (scaled by $\lambda$) regularizes the patch \& image embeddings, respectively.
% $\mathcal{L} = \mathcal{L}_{\mathbb{P}\mathbb{I}} + \beta \mathcal{L}_{\text{SPO}} + \lambda(||\textbf{z}_i^\mathbb{P}||_2^2 + ||\textbf{z}_i^\mathbb{I}||_2^2 + ||\textbf{z}_i^`||_2^2)$, where $\beta$ scales SPO and the last term regularizes the patch, image, and SPO patch embeddings, respectively.





%******************************************************************************%
%*************************    RESULTS & DISCUSSION   **************************%
%******************************************************************************%
\section{Results and Discussion}
\subsection{Simulated Environment}
When using simulated environments, multiple agents can interact in parallel with their own environment, evolving them independently. In addition, no time is needed between iterations for appreciating the thermal situation of the system and no dissipation time is needed for starting again a new episode. As a consequence, the training time is reduced significantly.

As no real sensors are used in this environments, an initial temperature has to be set to fix a point to start with the simulation. In order to train the agents on different scenarios, the initial temperature was randomized on each scenario instance. This randomization adds some difficulties to the training but also adds variability to the samples and improves the generalization capabilities of the agents.

After training for 12 hours in 8 parallel environments, a mean episode length of 31500 iterations was achieved when starting the simulation 2($\pm$1) degrees Celsius below the temperature limit (Mean episode length is shown in Fig. \ref{fig:simulated training}).

% Figure environment removed

\subsection{Real Hardware}

When the limit temperature is reached, the next training episode cannot be restarted instantly as in the simulated environment. As a result, a cooling period emerges between two neighboring episodes. This is the time required to cool down the device until operational temperature ranges are attained.
In the experimentation, the limit temperature was fixed in 55 degrees Celsius as this threshold was near the operational limit of some subsystems. The cooling period to reach the idle temperature of the payload, which is 50 degrees Celsius on ground laboratory conditions (no vacuum, heat dissipation by airflow and variable room temperature), is at most 2 minutes. 
Different measure/update frequencies have been tried during the experimentation. Finally an update frequency of 5 seconds was found to be acceptable for the temperature evolution speed.

The Fig. \ref{fig:beginning_training} shows the initial episode cycles in temperature and power consumption graphics of the board. In this episodes, the maximum temperature threshold was reached after less than 5 iterations (25 seconds).

% Figure environment removed

After 5hours of training the 73rd episode duration was longer than 12hs, as shown in Fig. \ref{fig:5h_training}, which illustrates the improvement in the agent skills to maintain the temperature below the limit temperature.

% Figure environment removed


The length of the episodes during training can be seen on Fig.\ref{fig:hw_training}. As in the simulated environment, policy performance is low until enough samples are gathered; after this point is achieved, performance improves noticeably.

% Figure environment removed

Both the policy learning and the prediction with it was done without hardware acceleration (CPU only). During the policy learning, the amount of RAM used was stable in 290MB and spikes of 200 millicores (0.2 cores) were required every 5 seconds (prediction frequency).






\vspace*{-2.5mm}
\section{Conclusions} \label{sec:4}
\vspace*{-1.25mm}

% To address this gap, we present SwIPE, a novel approach that utilizes implicit patch embeddings to achieve efficient and robust medical image segmentation. SwIPE directly models object shapes and adopts continuous representations, enabling accurate segmentation while avoiding the limitations of traditional approaches.

SwIPE represents a notable departure from conventional discrete segmentation approaches and directly models object shapes instead of pixels and utilizes continuous rather than discrete representations. 
By adopting both patch and image embeddings, our approach enables accurate local geometric descriptions and improved shape coherence.
% Our approach also adopts both patch and image embeddings to enable the efficient representation of local object geometry and appearance.
% Our approach is based on the use of implicit patch embeddings that enable the representation of local object geometry and appearance in an expressive and memory-efficient manner. 
% SwIPE is designed to work with 
% %a variety of 
% different medical image modalities and has been
% %extensively 
% well evaluated on several benchmark datasets. 
Experimental results show the superiority of SwIPE over state-of-the-art approaches in terms of segmentation accuracy, efficiency, and robustness. 
The use of local INRs represents a new direction for medical image segmentation, and we hope to inspire further research in this direction.

\vspace*{-1mm}

\bibliographystyle{splncs04}
\bibliography{refs}

%%%%%%% ------------------------------------------------------------------------- %%%%%%%
%%%%%%% ------------------------------------------------------------------------- %%%%%%%




\end{document}

