

\section{Experiments and Results} \label{sec:3}

In this section, we present the results of the proposed implicit shape method for constraining medical image segmentation. 
We conducted three studies to assess the performance and robustness of our method.
1) Segmentation efficiency and performance on 3D CT spleen and 2D ultrasound lymph node datasets with varying annotation availability (10\%, 25\%, and 100\% annotations) compared to state-of-the-art approaches. 
2) Robustness of the above approaches when inferring on unseen distributions (across datasets and modalities) while segmenting the same structure. 
3) Ablations on the proposed components and how they contribute to robustness and performance.

\subsection{Data and Implementation} \label{sec:3-1}

[Baselines] \\  
Add in supplementary section the implementation details of the baselines. 
For fully supervised Sessile, mention that the SOTA paper was an ArXiV paper and the code was not made available to us. 
To be safe, put important things in the main paper. 
"Some of the details in the implementation such as the data splits have not been available to us, so we used our splits." \\


\noindent
1) Performance with varying annotation amounts \\
CT BCV (pancreas or liver or stomach)  \\
Endoscopy CVC-ClinicDB (polyp segmentation) \\

\noindent
2a) Robustness across datasets \\
CT BCV (pancreas or liver or stomach) -> CT AMOS (pancreas or liver or stomach) \\
 -> Endoscopy CVC-ClinicDB (polyp segmentation) \\

\noindent
2b) Robustness across modalities \\
CT BCV (pancreas or liver or stomach) -> MR AMOS (pancreas or liver or stomach) \\



\subsection{Baselines} \label{sec:3-2}


\subsection{Study 1: Performance on Varying Annotation Availability} \label{sec:3-3}

We compared our method to state-of-the-art raster-based approaches and recent medical image DISMs.
We also analyze efficiency in terms of model complexity (FLOPs, number of parameters) and computational load (inference time, GPU memory usage).


\subsection{Study 2: Robustness to Data Distribution Shifts} \label{sec:3-4}

In this study, we evaluate the robustness of the proposed method to data distribution shifts across datasets and modalities. 
Specifically, we first trained a model using a CT dataset and applied it directly to a different CT dataset (CT MSD Spleen -> CT BCV Spleen) containing images from other institutions, scanners, and with different spatial properties. 
Additionally, we assessed the ability of methods to adapt to more distinct visual appearances by training using CT images and evaluating on MR images (CT MMWHS Heart -> MR MMWHS Heart). 

% Here, we evaluate the robustness of the proposed method to data distribution shifts across datasets and modalities.
% Specifically, we trained a model using one dataset and applied it directly to a different dataset (CT MSD Spleen -> CT BCV Spleen) to evaluate the generalizability of the method to images from other institutions, scanners, and with different spatial properties.
% We also evaluated the ability of the method to adapt to more distinct visual appearances by training using CT images and evaluating on MR images (CT MMWHS Heart -> MR MMWHS Heart).


Our experiments demonstrate the superior robustness of the proposed method to different shifts in data distribution.


\subsection{Study 3: Ablations for Proposed Components} \label{sec:3-5}

Our experiments provide insight into the individual contributions of the proposed components to the overall performance of the method.

